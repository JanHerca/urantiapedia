% Author of this conversion to LaTeX format: Jan Herca, 2017
\documentclass[twoside, 11pt]{book}
\usepackage[T1]{fontenc} % indica al procesador cómo imprimir los caracteres
\usepackage{fontspec} % permite definir fuentes a partir de las instaladas en el SO
\usepackage{geometry}
\usepackage{graphicx}
\usepackage{float}
\usepackage{tocloft}
\usepackage{titleps}
\usepackage{emptypage}
\usepackage[spanish]{babel}
\usepackage{multicol}
% Text styles
\geometry{paperwidth=16cm, paperheight=24cm, top=2.5cm, bottom=1.7cm, inner=2.5cm, outer=1.2cm}

\makeatletter
\def\@makechapterhead#1{%
	\vspace*{50\p@}%
	{\parindent \z@ \raggedright \normalfont
		\interlinepenalty\@M
		\huge \bfseries #1\par\nobreak
		\vskip 40\p@
}}
\def\@makeschapterhead#1{%
	\vspace*{50\p@}%
	{\parindent \z@ \raggedright
		\normalfont
		\interlinepenalty\@M
		\huge \bfseries  #1\par\nobreak
		\vskip 40\p@
}}
\makeatother

\renewcommand{\cftchapleader}{\cftdotfill{\cftdotsep}}
\renewcommand{\thechapter}{}
\renewcommand{\cftchapfont}{\large}
\cftsetpnumwidth{3em}
\renewcommand{\cftchappagefont}{\large}



\title{La Quinta Revelación \newline Sexto Volumen \newline La Vida y las Enseñanzas de Jesús - I}
\date{}
\begin{document}

\begin{titlepage}
	\centering
	{\Huge\bfseries El Libro de Urantia\par}
	{\huge\bfseries La Quinta Revelación\par}
	\vspace{1cm}
	{\huge\bfseries Sexto Volumen\par}
	\vspace{1cm}
	{\huge\bfseries La Vida y las Enseñanzas de Jesús\par}
	{\huge\bfseries I\par}
	\vfill
	{\scshape\Large URANTIA FOUNDATION\par}
	{\scshape\Large CHICAGO ILLINOIS\par}
	{\Large 2009 Traducción al español Europea\par}
\end{titlepage}
	
	
\par {\textcopyright} 2019 Jan Herca, de la edición
\par {\textcopyright} 2009 Urantia Foundation, de la traducción
\par {\textcopyright} 1993 Urantia Foundation, de otros materiales
\bigbreak
\par Jan Herca
\par Correo electrónico: janherca@gmail.com
\bigbreak
\par Urantia Foundation
\par 533 West Diversey Parkway
\par Chicago, IL 60614 EE.UU.A
\par Oficina: 1+(773) 525-3319
\par Fax: 1 +(773) 525-7739
\par Website: http://www.urantia.org
\par Correo electrónico: urantia@urantia.org
\bigbreak
\par Todos los derechos reservados, incluyendo el de traducción en los Estados Unidos de América, Canadá y en los demás países de la Unión Internacional de copyright. Todos los derechos reservados en los paises firmantes de la Union Panamericana de la Union internacional de copyright.
\par No todo el libro ni parte de él pueden ser copiados, reproducidos o traducidos en forma alguna, ya sea por medio electrónico, mecánico u otra forma, como fotocopia, grabación o archivo computerizado sin autorización por escrito del editor.
\par URANTIA,'' ``URANTIAN,'' ``EL LIBRO DE URANTIA'' y son marcas registradas de Urantia Foundation y su uso está sujeto a licencia.
\bigbreak
\par La Quinta Revelación es una reedición de El Libro de Urantia (Edición Europea). Está dividido en siete volúmenes para hacerlo más manejable y dispone de contenido adicional en forma de ayudas a la lectura integradas en el texto. El Libro de Urantia (Edición Europea) es una traducción de The Urantia Book realizada por la Fundación Urantia en 2009. 
	\newpage
	
\begin{center}
	{\huge\bfseries Las partes del libro\par}
	\vspace{1cm}
	{\scshape\large PRIMER VOLUMEN\par}
	{\scshape\Large DIOS, EL UNIVERSO CENTRAL Y LOS SUPERUNIVERSOS\par}
	\vspace{1cm}
	
	{\scshape\large SEGUNDO VOLUMEN \par}
	{\scshape\Large EL UNIVERSO LOCAL\par}
	\vspace{1cm}
	
	{\scshape\large TERCER VOLUMEN \par}
	{\scshape\Large LA HISTORIA DE NUESTRO PLANETA, URANTIA\par}
	\vspace{1cm}
	
	{\scshape\large CUARTO VOLUMEN \par}
	{\scshape\Large LA EVOLUCIÓN DE LA CIVILIZACIÓN HUMANA\par}
	\vspace{1cm}
	
	{\scshape\large QUINTO VOLUMEN \par}
	{\scshape\Large LA RELIGIÓN, LA SOBREVIVENCIA A LA MUERTE Y LA DEIDAD EXPERIENCIAL\par}
	\vspace{1cm}
	
	{\scshape\large SEXTO VOLUMEN \par}
	{\scshape\Large LA VIDA Y LAS ENSEÑANZAS DE JESÚS - I\par}
	\vspace{1cm}
	
	{\scshape\large SÉPTIMO VOLUMEN \par}
	{\scshape\Large LA VIDA Y LAS ENSEÑANZAS DE JESÚS - II\par}
\end{center}
	
\newpage
\begin{center}
	{\small \textit {Intencionadamente en blanco}\par}
\end{center}
\newpage

\pagestyle{empty}

\tableofcontents


\newpagestyle{main}{
	%\setheadrule{.4pt}% Header rule
	%\setfootrule{.4pt}% Footer rule
	\sethead[\small \thepage]% odd-left
	[]% odd-center
	[\begin{minipage}{0.9\textwidth}\begin{flushright}\scriptsize \MakeUppercase{\chaptertitle}\end{flushright}\end{minipage}]% odd-right
	{\begin{minipage}{0.9\textwidth}\scriptsize \MakeUppercase{\chaptertitle}\end{minipage}}% even-left
	{}% even-center
	{\small \thepage}% even-right
	\setfoot[]% odd-left
	[]% odd-center
	[]% odd-right
	{}% even-left
	{}% even-center
	{}% even-right
}
\pagestyle{main}
\renewcommand{\makeheadrule}{\rule[-.6\baselineskip]{\linewidth}{.4pt}}

\chapter{Documento 119. Las donaciones de Cristo Miguel}
\par
%\textsuperscript{(1308.1)}
\textsuperscript{119:0.1} ME LLAMO Gavalia, soy el Jefe de las Estrellas Vespertinas de Nebadon, y estoy destinado en Urantia por Gabriel con la misión de revelar la historia de las siete donaciones de Miguel de Nebadon, el Soberano de este universo. En el transcurso de esta presentación, me atendré estrictamente a las limitaciones impuestas por mi cometido.

\par
%\textsuperscript{(1308.2)}
\textsuperscript{119:0.2} El atributo de la donación es inherente a los Hijos Paradisiacos del Padre Universal. En su deseo de acercarse a las experiencias de la vida de sus criaturas vivientes subordinadas, las diversas órdenes de Hijos Paradisiacos reflejan la naturaleza divina de sus padres del Paraíso. El Hijo Eterno de la Trinidad del Paraíso mostró el camino en esta práctica donándose siete veces en los siete circuitos de Havona en la época de la ascensión de Grandfanda y de los primeros peregrinos del tiempo y del espacio. Y el Hijo Eterno continúa donándose en los universos locales del espacio en las personas de sus representantes, los Hijos Migueles y los Hijos Avonales.

\par
%\textsuperscript{(1308.3)}
\textsuperscript{119:0.3} Cuando el Hijo Eterno concede un Hijo Creador a un universo local en proyecto, ese Hijo Creador asume la plena responsabilidad de acabar, controlar y componer ese nuevo universo, incluyendo el solemne juramento a la Trinidad eterna de no asumir la plena soberanía de la nueva creación hasta que sus siete donaciones bajo la forma de sus criaturas hayan sido terminadas con éxito y certificadas por los Ancianos de los Días del superuniverso interesado. Cada Hijo Miguel que se ofrece como voluntario para salir del Paraíso y emprender la organización y la creación de un universo, asume esta obligación.

\par
%\textsuperscript{(1308.4)}
\textsuperscript{119:0.4} La finalidad de estas encarnaciones bajo la forma de criaturas consiste en capacitar a estos Creadores para que se conviertan en unos soberanos sabios, compasivos, justos y comprensivos. Estos Hijos divinos son justos de manera innata, pero se vuelven comprensivamente misericordiosos como resultado de estas experiencias sucesivas de donación; son misericordiosos por naturaleza, pero estas experiencias los hacen misericordiosos de una forma nueva y adicional. Estas donaciones son las últimas etapas de su educación y de su formación para la tarea sublime de gobernar los universos locales con rectitud divina y un juicio justo.

\par
%\textsuperscript{(1308.5)}
\textsuperscript{119:0.5} Aunque estas donaciones aportan numerosos beneficios secundarios a los diversos mundos, sistemas y constelaciones, así como a las diferentes órdenes de inteligencias universales a quienes afectan y benefician, sin embargo están destinadas principalmente a completar la formación personal y la educación universal de un Hijo Creador mismo. Estas donaciones no son imprescindibles para dirigir un universo local de manera sabia, justa y eficaz, pero son absolutamente necesarias para administrar de forma equitativa, misericordiosa y comprensiva esa creación, que rebosa de diversas formas de vida y de innumerables criaturas inteligentes pero imperfectas.

\par
%\textsuperscript{(1308.6)}
\textsuperscript{119:0.6} Los Hijos Migueles empiezan su trabajo de organización universal con una comprensión justa y completa de las diversas órdenes de seres que han creado. Tienen unas enormes reservas de misericordia para todas estas diferentes criaturas, e incluso compasión por aquellas que se equivocan y se tambalean en el lodo egoísta que ellas mismas han fabricado. Pero los Ancianos de los Días estiman que estos dones de justicia y de rectitud no son suficientes. Estos gobernantes trinos de los superuniversos nunca certificarán que un Hijo Creador es el Soberano de su universo hasta que no haya adquirido realmente el punto de vista de sus propias criaturas mediante una experiencia efectiva en el entorno donde existen y bajo la forma de esas mismas criaturas. De esta manera, estos Hijos se convierten en unos gobernantes inteligentes y comprensivos; llegan a \textit{conocer} a los diversos grupos a los que gobiernan y sobre los que ejercen su autoridad universal. Adquieren por medio de la experiencia viviente una misericordia práctica, un juicio equitativo, y la paciencia nacida de la existencia experiencial de la criaturas.

\par
%\textsuperscript{(1309.1)}
\textsuperscript{119:0.7} El universo local de Nebadon está ahora gobernado por un Hijo Creador que ha terminado su servicio de donación; reina con una supremacía justa y misericordiosa sobre todos los inmensos dominios de su universo en vías de evolución y de perfeccionamiento. Miguel de Nebadon es la 611.121{\textordfeminine} donación del Hijo Eterno sobre los universos del tiempo y del espacio, y empezó la organización de vuestro universo local hace unos cuatrocientos mil millones de años. Miguel se preparó para su primera aventura de donación hacia la época en que Urantia estaba adquiriendo su forma actual, hace mil millones de años. Sus donaciones se han producido cada ciento cincuenta millones de años aproximadamente, y la última tuvo lugar en Urantia hace mil novecientos años. Ahora procederé a exponer la naturaleza y el carácter de estas donaciones de una manera tan plena como me lo permita mi cometido.

\section*{1. La primera donación}
\par
%\textsuperscript{(1309.2)}
\textsuperscript{119:1.1} Se produjo un acontecimiento solemne en Salvington cuando, hace casi mil millones de años, la asamblea de directores y de jefes del universo de Nebadon escuchó a Miguel anunciar que su hermano mayor, Emmanuel, asumiría pronto la autoridad de Nebadon mientras que él (Miguel) se ausentaría para llevar a cabo una misión no explicada. No se hizo ninguna otra declaración acerca de esta operación, salvo que en la transmisión de despedida a los Padres de la Constelación, se decía entre otras instrucciones: <<Y durante este período os pongo al cargo y cuidado de Emmanuel, mientras voy a hacer lo que me pide mi Padre Paradisiaco>>.

\par
%\textsuperscript{(1309.3)}
\textsuperscript{119:1.2} Después de enviar esta transmisión de despedida, Miguel apareció en el campo de partida de Salvington exactamente igual que en muchas ocasiones anteriores cuando se había preparado para partir hacia Uversa o el Paraíso, excepto que esta vez venía solo. Terminó su declaración de partida con estas palabras: <<Sólo os dejo durante un corto período de tiempo. Sé que muchos de vosotros querrían venir conmigo, pero allá donde voy no podéis venir. Esto que estoy a punto de hacer no podéis hacerlo. Voy a hacer la voluntad de las Deidades del Paraíso, y cuando haya terminado mi misión y haya adquirido esta experiencia, volveré a ocupar mi lugar entre vosotros>>. Después de hablar así, Miguel de Nebadon desapareció de la vista de todos aquellos que estaban reunidos y no volvió a aparecer durante veinte años del tiempo oficial. En todo Salvington, sólo la Ministra Divina y Emmanuel sabían lo que estaba sucediendo, y el Unión de los Días sólo compartió su secreto con Gabriel, el jefe ejecutivo del universo, la Radiante Estrella Matutina.

\par
%\textsuperscript{(1309.4)}
\textsuperscript{119:1.3} Todos los habitantes de Salvington y aquellos que residían en los mundos sede de las constelaciones y de los sistemas se reunieron alrededor de sus respectivas estaciones receptoras de la información universal, esperando recibir alguna noticia sobre la misión y el paradero del Hijo Creador. No se recibió ningún mensaje de posible importancia hasta el tercer día después de la partida de Miguel. Ese día se registró en Salvington, procedente de la esfera Melquisedek, la sede de esta orden en Nebadon, una comunicación que describía simplemente esta operación extraordinaria que nunca se había oído anteriormente: <<Hoy al mediodía ha aparecido en el campo de recepción de este mundo un extraño Hijo Melquisedek, que no es de los nuestros, pero que se parece enteramente a los de nuestra orden. Venía acompañado de un omniafín solitario que traía las credenciales de Uversa y que presentó unas instrucciones dirigidas a nuestro jefe, procedentes de los Ancianos de los Días y con el acuerdo de Emmanuel de Salvington, ordenando que este nuevo Hijo Melquisedek sea recibido en nuestra orden y destinado al servicio de urgencia de los Melquisedeks de Nebadon. Así hemos ordenado que se haga, y así se ha hecho>>.

\par
%\textsuperscript{(1310.1)}
\textsuperscript{119:1.4} Esto es casi todo lo que aparece en los archivos de Salvington con respecto a la primera donación de Miguel. No aparece nada más hasta cien años después, según el tiempo de Urantia, cuando se registró el hecho de que Miguel regresó y volvió a asumir, sin anunciarlo, la dirección de los asuntos del universo. Pero se puede encontrar una extraña inscripción en el mundo Melquisedek, un relato del servicio de este Hijo Melquisedek excepcional del cuerpo de urgencia de aquella época. Este informe se conserva en un sencillo templo que ocupa actualmente el primer término del hogar del Padre Melquisedek, y contiene la narración del servicio de este Hijo Melquisedek transitorio en relación con su tarea en veinticuatro misiones de urgencia en el universo. Este informe, que he vuelto a examinar tan recientemente, termina así:

\par
%\textsuperscript{(1310.2)}
\textsuperscript{119:1.5} <<A mediodía de hoy, sin previo anuncio y observado solamente por tres miembros de nuestra fraternidad, este Hijo visitante de nuestra orden ha desaparecido de nuestro mundo tal como había llegado, acompañado solamente por un omniafín solitario; este informe se cierra ahora con la certificación de que este visitante ha vivido como un Melquisedek, ha trabajado como un Melquisedek en la similitud de un Melquisedek, y ha cumplido fielmente todas sus misiones como Hijo de urgencia de nuestra orden. Por consentimiento universal se ha convertido en el jefe de los Melquisedeks, habiéndose ganado nuestro amor y nuestra adoración con su sabiduría incomparable, su amor supremo y su magnífica devoción al deber. Él nos ha amado, nos ha comprendido y ha servido con nosotros, y seremos para siempre sus fieles y leales compañeros Melquisedeks, pues este desconocido en nuestro mundo se ha vuelto ahora, y para la eternidad, un ministro universal de naturaleza Melquisedek>>\footnote{\textit{De la orden de Melquisedek}: Heb 5:10; 6:20; 7:17.}.

\par
%\textsuperscript{(1310.3)}
\textsuperscript{119:1.6} Esto es todo lo que se me permite deciros sobre la primera donación de Miguel. Comprendemos plenamente, por supuesto, que este extraño Melquisedek que sirvió tan misteriosamente con los Melquisedeks hace mil millones de años, no era otro que Miguel, encarnado durante la misión de su primera donación. Los archivos no afirman específicamente que este Melquisedek excepcional y eficaz fuera Miguel, pero se cree universalmente que lo era. Es probable que la afirmación concreta de este hecho no se pueda encontrar fuera de los archivos de Sonarington, y los registros de este mundo secreto no están abiertos para nosotros. Los misterios de la encarnación y de la donación sólo se conocen plenamente en este mundo sagrado de los Hijos divinos. Todos conocemos los hechos de las donaciones de Miguel, pero no comprendemos cómo se realizan. No sabemos de qué forma el gobernante de un universo, el creador de los Melquisedeks, puede convertirse de manera tan repentina y misteriosa en uno de ellos y, como uno de ellos, vivir en medio de ellos y trabajar como un Hijo Melquisedek durante cien años. Pero esto es lo que ocurrió.

\section*{2. La segunda donación}
\par
%\textsuperscript{(1310.4)}
\textsuperscript{119:2.1} Durante cerca de ciento cincuenta millones de años después de la donación de Miguel como Melquisedek, todo fue bien en el universo de Nebadon hasta que empezaron a surgir dificultades en el sistema 11 de la constelación 37. Este conflicto consistía en un malentendido por parte de un Hijo Lanonandek, un Soberano Sistémico; el conflicto había sido juzgado por los Padres de la Constelación y su fallo había sido aprobado por el Fiel de los Días, el consejero del Paraíso en aquella constelación, pero el Soberano Sistémico que protestaba no estaba plenamente conforme con el veredicto. Después de más de cien años de descontento, condujo a sus asociados a una de las rebeliones más extendidas y desastrosas, en contra de la soberanía del Hijo Creador, que jamás se haya fomentado en el universo de Nebadon, una rebelión que fue juzgada y terminó hace mucho tiempo gracias a la actuación de los Ancianos de los Días de Uversa.

\par
%\textsuperscript{(1311.1)}
\textsuperscript{119:2.2} Lutentia, el Soberano Sistémico rebelde, reinó de manera suprema en el planeta donde tenía su sede durante más de veinte años del tiempo oficial de Nebadon, después de lo cual, los Altísimos, con la aprobación de Uversa, ordenaron su aislamiento y solicitaron a los gobernantes de Salvington que designaran a un nuevo Soberano Sistémico para que asumiera la dirección de este sistema de mundos habitados confuso y desgarrado por los conflictos.

\par
%\textsuperscript{(1311.2)}
\textsuperscript{119:2.3} Al mismo tiempo que se recibía esta petición en Salvington, Miguel anunció la segunda de aquellas extraordinarias proclamaciones de intención de ausentarse de la sede del universo con el fin de <<hacer el mandato de mi Padre Paradisiaco>>, prometiendo <<regresar en el momento adecuado>>, y concentrando toda la autoridad en las manos de Emmanuel, su hermano del Paraíso, el Unión de los Días.

\par
%\textsuperscript{(1311.3)}
\textsuperscript{119:2.4} Luego, empleando la misma técnica que se observó en el momento de su partida para la donación como Melquisedek, Miguel se despidió de nuevo de la esfera de su sede central. Tres días después de esta despedida inexplicada, un nuevo miembro desconocido apareció en el cuerpo de reserva de los Hijos Lanonandeks primarios de Nebadon. Este nuevo Hijo apareció al mediodía, sin anunciarse y acompañado de un terciafín solitario que llevaba las credenciales de los Ancianos de los Días de Uversa, certificadas por Emmanuel de Salvington, ordenando que este nuevo Hijo fuera destinado al sistema 11 de la constelación 37 como sucesor del depuesto Lutentia, y con plena autoridad como Soberano en funciones del Sistema hasta que se nombrara un nuevo soberano.

\par
%\textsuperscript{(1311.4)}
\textsuperscript{119:2.5} Durante más de diecisiete años del tiempo universal, este gobernante provisional extraño y desconocido administró los asuntos y juzgó sabiamente las dificultades de este sistema local confuso y desmoralizado. Ningún Soberano Sistémico fue nunca más ardientemente amado u honrado y respetado de manera más generalizada. Este nuevo gobernante puso en orden con justicia y misericordia el turbulento sistema, mientras servía cuidadosamente a todos sus súbditos, ofreciéndole incluso a su predecesor rebelde el privilegio de compartir el trono de autoridad del sistema con que sólo presentara sus excusas a Emmanuel por sus imprudencias. Pero Lutentia despreció estos ofrecimientos de misericordia, sabiendo muy bien que este nuevo y extraño Soberano del Sistema no era otro que Miguel, el dirigente universal mismo a quien tan recientemente había desafiado. Pero millones de seguidores suyos descaminados y engañados aceptaron el perdón de este nuevo gobernante, conocido en aquella época como el Soberano Salvador del sistema de Palonia.

\par
%\textsuperscript{(1311.5)}
\textsuperscript{119:2.6} Luego llegó el día memorable en que se presentó el Soberano Sistémico recién nombrado, designado por las autoridades universales como sucesor permanente del depuesto Lutentia, y toda Palonia lamentó la partida del gobernante sistémico más noble y más benigno que Nebadon hubiera conocido jamás. Era amado por todo el sistema y adorado por sus compañeros de todos los grupos de Hijos Lanonandeks. Su partida no tuvo lugar sin ceremonias; se organizó una gran celebración cuando dejó la sede central del sistema. Incluso su predecesor equivocado le envió este mensaje: <<Eres justo y recto en todas tus acciones. Aunque continúo rechazando el gobierno del Paraíso, me veo obligado a confesar que eres un administrador justo y misericordioso>>\footnote{\textit{Miguel es justo y misericordioso}: Sal 116:5; Dt 32:4; Ap 15:3.}.

\par
%\textsuperscript{(1312.1)}
\textsuperscript{119:2.7} Entonces, este gobernante provisional del sistema rebelde se despidió del planeta de su breve estancia administrativa, y al tercer día después de esto, Miguel apareció en Salvington y asumió de nuevo la dirección del universo de Nebadon. Poco después se produjo la tercera proclamación de Uversa sobre la extensión jurisdiccional de la soberanía y de la autoridad de Miguel. La primera proclamación tuvo lugar en el momento de su llegada a Nebadon, la segunda se había emitido poco después de concluir su donación como Melquisedek, y ahora seguía la tercera al terminar la segunda misión, o misión Lanonandek.

\section*{3. La tercera donación}
\par
%\textsuperscript{(1312.2)}
\textsuperscript{119:3.1} El consejo supremo de Salvington acababa de estudiar la petición de los Portadores de Vida del planeta 217 del sistema 87 de la constelación 61 para que se enviara en su ayuda a un Hijo Material. Ahora bien, este planeta estaba situado en un sistema de mundos habitados donde otro Soberano Sistémico se había descarriado, la segunda rebelión de este tipo en todo Nebadon hasta aquel momento.

\par
%\textsuperscript{(1312.3)}
\textsuperscript{119:3.2} La respuesta a la solicitud de los Portadores de Vida de este planeta fue aplazada, a petición de Miguel, hasta que Emmanuel la estudiara y presentara su informe. Se trataba de un procedimiento irregular, y recuerdo muy bien que todos nos esperábamos algo fuera de lo normal, y no tuvimos que permanecer mucho tiempo en la incertidumbre. Miguel procedió a poner la dirección del universo en manos de Emmanuel, mientras que confió el mando de las fuerzas celestiales a Gabriel; una vez que traspasó así sus responsabilidades administrativas, se despidió del Espíritu Madre del Universo y desapareció del campo de partida de Salvington exactamente tal como lo había hecho en las dos ocasiones anteriores.

\par
%\textsuperscript{(1312.4)}
\textsuperscript{119:3.3} Como se podía esperar, un Hijo Material desconocido apareció tres días después, sin haberse anunciado, en el mundo central del sistema 87 de la constelación 61, acompañado de un seconafín solitario, acreditado por los Ancianos de los Días de Uversa y certificado por Emmanuel de Salvington. El Soberano en funciones del Sistema nombró inmediatamente a este nuevo y misterioso Hijo Material como Príncipe Planetario en ejercicio del mundo 217, y los Altísimos de la constelación 61 confirmaron enseguida esta designación.

\par
%\textsuperscript{(1312.5)}
\textsuperscript{119:3.4} Este Hijo Material excepcional empezó así su difícil carrera en un mundo en secesión, en rebelión y en cuarentena, situado en un sistema aislado sin ninguna comunicación directa con el universo exterior, y allí trabajó solo durante una generación entera del tiempo planetario. Este Hijo Material de urgencia consiguió el arrepentimiento y la recuperación del Príncipe Planetario rebelde y de todo su estado mayor, y presenció el restablecimiento del planeta al servicio leal del gobierno del Paraíso tal como éste está establecido en los universos locales. Un Hijo y una Hija Materiales llegaron a su debido tiempo a este mundo rejuvenecido y redimido, y después de haber sido debidamente instalados como gobernantes planetarios visibles, el Príncipe Planetario provisional o de urgencia se despidió oficialmente y desapareció un día al mediodía. Tres días después, Miguel apareció en su lugar acostumbrado en Salvington, y las transmisiones del superuniverso difundieron muy pronto la cuarta proclamación de los Ancianos de los Días, anunciando el nuevo avance de la soberanía de Miguel en Nebadon.

\par
%\textsuperscript{(1312.6)}
\textsuperscript{119:3.5} Lamento no tener permiso para narrar la paciencia, la fortaleza y la habilidad con que este Hijo Material hizo frente a las difíciles situaciones de este confuso planeta. La recuperación de este mundo aislado es uno de los capítulos más hermosos y conmovedores en los anales de la salvación de todo Nebadon. Hacia el final de esta misión, para todo Nebadon se había vuelto evidente por qué su amado gobernante escogía embarcarse en estas repetidas donaciones en la similitud de alguna orden subordinada de seres inteligentes.

\par
%\textsuperscript{(1313.1)}
\textsuperscript{119:3.6} Las donaciones de Miguel primero como Hijo Melquisedek, luego como Hijo Lanonandek y después como Hijo Material, son todas igualmente misteriosas y se encuentran más allá de toda explicación. En cada caso apareció \textit{repentinamente} y como un individuo plenamente desarrollado del grupo de la donación. El misterio de estas encarnaciones no será nunca conocido, salvo por aquellos que tienen acceso al círculo interior de los archivos de la esfera sagrada de Sonarington.

\par
%\textsuperscript{(1313.2)}
\textsuperscript{119:3.7} Desde esta maravillosa donación como Príncipe Planetario de un mundo aislado y en rebelión, ninguno de los Hijos o Hijas Materiales de Nebadon ha caído nunca en la tentación de quejarse de sus tareas o de criticar las dificultades de sus misiones planetarias. Los Hijos Materiales saben para siempre que en el Hijo Creador del universo tienen a un soberano comprensivo y a un amigo compasivo, a alguien que ha <<sido probado y comprobado en todos los aspectos>>\footnote{\textit{Ha sido probado y comprobado en todos los aspectos}: Lc 4:2; 22:28; Heb 2:16-18; 4:15.}, tal como ellos han de ser también probados y comprobados.

\par
%\textsuperscript{(1313.3)}
\textsuperscript{119:3.8} A cada una de estas misiones le siguió una era de servicio y de lealtad crecientes por parte de todas las inteligencias celestiales de origen universal, mientras que cada era donadora sucesiva estuvo caracterizada por un progreso y una mejora en todos los métodos de la administración universal y en todas las técnicas de gobierno. Desde esta donación, ningún Hijo o Hija Material se ha unido nunca deliberadamente a una rebelión en contra de Miguel; lo aman y lo honran con demasiada devoción como para rechazarlo nunca conscientemente. Los Adanes de los tiempos recientes sólo se han desviado debido a los engaños y sofismas de personalidades rebeldes de tipo más elevado.

\section*{4. La cuarta donación}
\par
%\textsuperscript{(1313.4)}
\textsuperscript{119:4.1} Al final de uno de los periódicos llamamientos nominales milenarios de Uversa, Miguel procedió a poner el gobierno de Nebadon en las manos de Emmanuel y Gabriel y, por supuesto, al recordar lo que había sucedido en tiempos pasados después de una acción como ésta, todos nos preparamos para presenciar la desaparición de Miguel a fin de emprender su cuarta misión de donación; y no tuvimos que esperar mucho tiempo, ya que pronto se dirigió al campo de partida de Salvington y lo perdimos de vista.

\par
%\textsuperscript{(1313.5)}
\textsuperscript{119:4.2} Al tercer día después de esta desaparición donadora, observamos esta noticia significativa, en las transmisiones universales hacia Uversa, procedente de la sede seráfica de Nebadon: <<Informamos de la llegada no anunciada de un serafín desconocido, acompañado de un supernafín solitario y de Gabriel de Salvington. Este serafín no registrado satisface los requisitos de la orden de Nebadon y trae las credenciales de los Ancianos de los Días de Uversa, certificadas por Emmanuel de Salvington. Este serafín demuestra pertenecer a la orden suprema de ángeles de un universo local, y ya ha sido destinado al cuerpo de consejeros docentes>>.

\par
%\textsuperscript{(1313.6)}
\textsuperscript{119:4.3} Miguel estuvo ausente de Salvington para esta donación seráfica durante un período de más de cuarenta años del tiempo oficial del universo. Durante este tiempo estuvo vinculado como consejero seráfico docente, lo que podríais denominar un secretario particular, a veintiséis instructores superiores diferentes que ejercían su actividad en veintidós mundos distintos. Su tarea última o final fue como consejero y asistente destinado en una misión donadora de un Hijo Instructor Trinitario en el mundo 462 del sistema 84 de la constelación 3 del universo de Nebadon.

\par
%\textsuperscript{(1314.1)}
\textsuperscript{119:4.4} Durante los siete años de esta misión, este Hijo Instructor Trinitario nunca estuvo plenamente persuadido de la identidad de su asociado seráfico. Es verdad que durante aquel período todos los serafines fueron considerados con un interés y una minuciosidad particulares. Todos sabíamos muy bien que nuestro amado Soberano estaba fuera en el universo bajo la apariencia de un serafín, pero nunca pudimos estar seguros de su identidad. Nunca fue identificado totalmente hasta el momento de ser destinado a la misión donadora de este Hijo Instructor Trinitario. Pero a lo largo de este período, los serafines supremos siempre fueron tratados con una solicitud especial, por temor a que cualquiera de nosotros pudiera descubrir que había sido, sin saberlo, el anfitrión del Soberano del universo en misión de donación bajo la forma de una criatura. Así pues, en lo que se refiere a los ángeles, se ha vuelto eternamente cierto que su Creador y Gobernante ha sido <<probado y comprobado, en todos los aspectos, en la similitud de una personalidad seráfica>>\footnote{\textit{Ha sido probado y comprobado en todos los aspectos}: Lc 4:2; 22:28; Heb 2:16-18; 4:15.}.

\par
%\textsuperscript{(1314.2)}
\textsuperscript{119:4.5} A medida que estas donaciones sucesivas compartían de manera creciente la naturaleza de las formas más humildes de la vida universal, Gabriel se convirtió cada vez más en un asociado de estas aventuras de encarnación, actuando como enlace universal entre Miguel, que se estaba donando, y Emmanuel, el gobernante en funciones del universo.

\par
%\textsuperscript{(1314.3)}
\textsuperscript{119:4.6} Miguel ha pasado ahora por la experiencia donadora de tres órdenes de Hijos universales creados por él: los Melquisedeks, los Lanonandeks y los Hijos Materiales. Luego condesciende a personalizarse en la similitud de la vida angélica como un serafín supremo, antes de dirigir su atención hacia las diversas fases de la carrera ascendente de las formas más humildes de criaturas volitivas: los mortales evolutivos del tiempo y del espacio.

\section*{5. La quinta donación}
\par
%\textsuperscript{(1314.4)}
\textsuperscript{119:5.1} Hace poco más de trescientos millones de años, tal como el tiempo se calcula en Urantia, fuimos testigos de otra de aquellas transmisiones de autoridad universal a Emmanuel y observamos que Miguel se preparó para partir. Esta vez fue diferente a las anteriores, en el sentido de que anunció que su destino sería Uversa, la sede central del superuniverso de Orvonton. Nuestro Soberano partió a su debido tiempo, pero las transmisiones del superuniverso no mencionaron nunca la llegada de Miguel a las cortes de los Ancianos de los Días. Poco después de su partida de Salvington, en las transmisiones de Uversa apareció esta declaración significativa: <<Hoy ha llegado un peregrino ascendente de origen mortal, sin anunciarse y sin número, procedente del universo de Nebadon, certificado por Emmanuel de Salvington y acompañado por Gabriel de Nebadon. Este ser no identificado presenta el estado de un verdadero espíritu y ha sido recibido en nuestra comunidad>>.

\par
%\textsuperscript{(1314.5)}
\textsuperscript{119:5.2} Si hoy pudierais visitar Uversa, escucharíais el relato de los tiempos en que Eventod residió allí, pues a este peregrino especial y desconocido del tiempo y del espacio se le conoce en Uversa por este nombre. Este mortal ascendente, o al menos esta magnífica personalidad exactamente semejante a los mortales ascendentes de la fase espiritual, vivió y ejerció su actividad en Uversa durante un período de once años del tiempo oficial de Orvonton. Este ser recibió las misiones y cumplió los deberes de un mortal espiritual de la misma manera que sus compañeros procedentes de los diversos universos locales de Orvonton. <<Fue probado y comprobado en todos los aspectos, al igual que sus compañeros>>\footnote{\textit{Ha sido probado y comprobado en todos los aspectos}: Lc 4:2; 22:28; Heb 2:16-18; 4:15.}, y en todas las ocasiones se mostró digno de la confianza y de la fe de sus superiores, al mismo tiempo que inspiró infaliblemente el respeto y la admiración leal de sus compañeros espirituales.

\par
%\textsuperscript{(1315.1)}
\textsuperscript{119:5.3} En Salvington seguimos la carrera de este peregrino espiritual con un gran interés, sabiendo muy bien, por la presencia de Gabriel, que este espíritu peregrino modesto y sin número no era otro que el gobernante, en misión de donación, de nuestro universo local. Esta primera aparición de Miguel, encarnado en el papel de una fase de la evolución mortal, fue un acontecimiento que emocionó y cautivó a todo Nebadon. Habíamos oído hablar de estas cosas, pero ahora las contemplábamos. Miguel apareció en Uversa como un mortal espiritual plenamente desarrollado y perfectamente entrenado, y continuó su carrera como tal hasta el momento en que un grupo de mortales ascendentes avanzó hacia Havona; después de lo cual, mantuvo una conversación con los Ancianos de los Días y se despidió inmediatamente de Uversa, en compañía de Gabriel, de manera repentina y sin ceremonias, apareciendo poco después en su lugar acostumbrado en Salvington.

\par
%\textsuperscript{(1315.2)}
\textsuperscript{119:5.4} Hasta que no terminó esta donación, no caímos finalmente en la cuenta de que Miguel iba probablemente a encarnarse en la similitud de sus diversas órdenes de personalidades del universo, desde los Melquisedeks más elevados, bajando en la escala hasta los mortales de carne y hueso de los mundos evolutivos del tiempo y del espacio. Hacia esta época, las escuelas de los Melquisedeks empezaron a enseñar la probabilidad de que Miguel se encarnaría algún día como un mortal en la carne, y se hicieron muchas especulaciones sobre la posible técnica de una donación tan inexplicable. El hecho de que Miguel hubiera representado en persona el papel de un mortal ascendente confirió un nuevo interés adicional a todo el programa del progreso de las criaturas a través del universo local y del superuniverso.

\par
%\textsuperscript{(1315.3)}
\textsuperscript{119:5.5} Sin embargo, la técnica de estas donaciones sucesivas continuaba siendo un misterio. Gabriel mismo confiesa que no comprende el método por el cual este Hijo Paradisiaco y Creador del universo puede, a voluntad, asumir la personalidad y vivir la vida de una de sus propias criaturas subordinadas.

\section*{6. La sexta donación}
\par
%\textsuperscript{(1315.4)}
\textsuperscript{119:6.1} Ahora que todo Salvington estaba familiarizado con los preparativos de una donación inminente, Miguel convocó a los residentes de su planeta sede y, por primera vez, reveló el resto de su plan de encarnación, anunciando que pronto iba a dejar Salvington con el fin de asumir la carrera de un mortal morontial en las cortes de los Altísimos Padres en el planeta sede de la quinta constelación. Y entonces escuchamos por primera vez el anuncio de que su séptima y última donación se llevaría a cabo en la similitud de la carne mortal en algún mundo evolutivo.

\par
%\textsuperscript{(1315.5)}
\textsuperscript{119:6.2} Antes de salir de Salvington para su sexta donación, Miguel dirigió la palabra a los habitantes reunidos de la esfera y partió a la vista de todos, acompañado de un serafín solitario y de la Radiante Estrella Matutina de Nebadon. Aunque la dirección del universo se había confiado de nuevo a Emmanuel, las responsabilidades administrativas habían sido distribuidas más ampliamente.

\par
%\textsuperscript{(1315.6)}
\textsuperscript{119:6.3} Miguel apareció en la sede de la quinta constelación como un mortal morontial de estado ascendente, plenamente desarrollado. Lamento que me esté prohibido revelar los detalles de esta carrera de un mortal morontial sin numerar, pues se trata de una de las épocas más extraordinarias y asombrosas de la experiencia donadora de Miguel, sin exceptuar siquiera su estancia dramática y trágica en Urantia. Pero entre las numerosas restricciones que se me impusieron al aceptar esta misión, se encuentra una que me prohíbe intentar revelar los detalles de esta maravillosa carrera de Miguel como mortal morontial de Endantun.

\par
%\textsuperscript{(1316.1)}
\textsuperscript{119:6.4} Cuando Miguel regresó de esta donación morontial, fue evidente para todos nosotros que nuestro Creador se había vuelto uno de nuestros semejantes, que el Soberano del Universo era también el amigo y el ayudante compasivo incluso de las formas de inteligencias creadas más humildes de sus reinos. La adquisición progresiva del punto de vista de las criaturas, el cual se reflejaba en la administración del universo, ya la habíamos notado antes de esto, pues había ido apareciendo gradualmente, pero se hizo más evidente después de terminar su donación como mortal morontial, y mucho más aún después de regresar de su carrera como hijo del carpintero en Urantia.

\par
%\textsuperscript{(1316.2)}
\textsuperscript{119:6.5} Gabriel nos había informado de antemano sobre el momento en que Miguel sería liberado de su donación morontial, y preparamos en consecuencia una recepción adecuada en Salvington. Se reunieron millones y millones de seres procedentes de los mundos sede de las constelaciones de Nebadon, y la mayoría de los residentes de los mundos adyacentes a Salvington se reunieron para darle la bienvenida a su regreso al gobierno del universo. En respuesta a nuestros numerosos discursos de bienvenida y expresiones de apreciación hacia un Soberano tan sumamente interesado en sus criaturas, Miguel se limitó a contestar: <<Simplemente me he ocupado de los asuntos de mi Padre. Sólo hago lo que complace a los Hijos Paradisiacos que aman y desean ardientemente comprender a sus criaturas>>\footnote{\textit{Los asuntos de mi Padre}: Lc 2:49.}.

\par
%\textsuperscript{(1316.3)}
\textsuperscript{119:6.6} Pero desde aquel día hasta el momento en que Miguel emprendió su aventura como Hijo del Hombre en Urantia, todo Nebadon continuó hablando de las numerosas proezas de su Gobernante Soberano cuando éste ejercía su actividad en Endantun, donándose a través de la encarnación de un mortal morontial en proceso de ascensión evolutiva, y siendo probado en todos los aspectos como sus compañeros allí reunidos procedentes de los mundos materiales de toda la constelación donde residía.

\section*{7. La séptima y última donación}
\par
%\textsuperscript{(1316.4)}
\textsuperscript{119:7.1} Durante decenas de miles de años, todos esperamos con ansia la séptima y última donación de Miguel. Gabriel nos había informado que esta donación final se llevaría a cabo en la similitud de la carne mortal, pero ignorábamos por completo el momento, el lugar y la manera de esta aventura culminante.

\par
%\textsuperscript{(1316.5)}
\textsuperscript{119:7.2} El anuncio público de que Miguel había escogido Urantia como teatro para su donación final se efectuó poco después de que nos enteráramos de la falta de Adán y Eva. Y así, durante más de treinta y cinco mil años, vuestro mundo ocupó un lugar muy notable en los consejos de todo el universo. Ninguna etapa de la donación en Urantia (aparte del misterio de la encarnación) se mantuvo en secreto. Desde el principio hasta el fin, incluido el regreso triunfante y final de Miguel a Salvington como Soberano supremo del Universo, todo lo que sucedió en vuestro pequeño, pero sumamente honrado mundo, recibió la más completa publicidad universal.

\par
%\textsuperscript{(1316.6)}
\textsuperscript{119:7.3} Nunca supimos, hasta el momento mismo del acontecimiento, que Miguel aparecería en la Tierra como un niño indefenso del reino, aunque creíamos que éste sería el método. Hasta ese momento siempre había aparecido como un individuo plenamente desarrollado del grupo de personalidades escogido para la donación; por eso la transmisión enviada desde Salvington informando que el bebé de Belén había nacido en Urantia fue una noticia emocionante\footnote{\textit{Nacimiento de Jesús}: Lc 2:6-14.}.

\par
%\textsuperscript{(1316.7)}
\textsuperscript{119:7.4} Entonces no solamente nos dimos cuenta de que nuestro Creador y amigo estaba dando el paso más precario de toda su carrera, arriesgando aparentemente su posición y su autoridad en esta donación como niño indefenso, sino que comprendimos también que su experiencia en esta donación final como mortal lo colocaría eternamente en el trono como soberano indiscutible y supremo del universo de Nebadon. Durante un tercio de siglo del tiempo terrestre, todas las miradas de todas las partes de este universo local estuvieron clavadas en Urantia. Todas las inteligencias se dieron cuenta de que la última donación estaba en curso, y como conocíamos desde hacía mucho tiempo la rebelión de Lucifer en Satania y el descontento de Caligastia en Urantia, comprendimos muy bien la intensidad de la lucha que se originaría cuando nuestro gobernante condescendiera a encarnarse en Urantia en la humilde forma y en la similitud de la carne mortal.

\par
%\textsuperscript{(1317.1)}
\textsuperscript{119:7.5} Josué ben José, el bebé judío, fue concebido y nació en el mundo exactamente igual que todos los demás bebés antes y después que él, \textit{salvo} que este bebé en particular era la encarnación de Miguel de Nebadon\footnote{\textit{Encarnación de Miguel}: Mc 1:1; Lc 1:30-33; 2:4-7; Jn 1:14. \textit{Visita de los magos}: Mt 2:1-12.}, un Hijo divino Paradisiaco y el creador de todo este universo local de cosas y de seres. Este misterio de la encarnación de la Deidad en la forma humana de Jesús, por lo demás de origen natural en el mundo, permanecerá para siempre sin resolverse. Nunca conoceréis, ni siquiera en la eternidad, la técnica y el método de la encarnación del Creador en la forma y la similitud de sus criaturas. Es el secreto de Sonarington, y estos misterios son propiedad exclusiva de los Hijos divinos que han pasado por la experiencia de la donación.

\par
%\textsuperscript{(1317.2)}
\textsuperscript{119:7.6} Algunos hombres sabios de la Tierra conocían la llegada inminente de Miguel. Mediante los contactos entre mundos, estos hombres sabios con perspicacia espiritual se enteraron de la próxima donación de Miguel en Urantia. Y los serafines lo anunciaron, a través de las criaturas intermedias, a un grupo de sacerdotes caldeos cuyo dirigente era Ardnón. Estos hombres de Dios visitaron al niño recién nacido. El único acontecimiento sobrenatural relacionado con el nacimiento de Jesús fue este anuncio a Ardnón y a sus compañeros por parte de los serafines que habían estado vinculados anteriormente a Adán y Eva en el primer jardín.

\par
%\textsuperscript{(1317.3)}
\textsuperscript{119:7.7} Los padres humanos de Jesús eran unas personas de tipo medio de su época y generación, y este Hijo encarnado de Dios nació así de una mujer y fue criado de la misma manera que los niños de aquella raza y de aquel tiempo.

\par
%\textsuperscript{(1317.4)}
\textsuperscript{119:7.8} La historia de la estancia de Miguel en Urantia, el relato de la donación humana del Hijo Creador en vuestro mundo, es un asunto que sobrepasa la incumbencia y la finalidad de esta narración.

\section*{8. El estado de Miguel después de sus donaciones}
\par
%\textsuperscript{(1317.5)}
\textsuperscript{119:8.1} Después de la donación final y con éxito de Miguel en Urantia, no solamente fue aceptado por los Ancianos de los Días como gobernante soberano de Nebadon, sino que también fue reconocido por el Padre Universal como director establecido del universo local creado por él mismo. A su regreso a Salvington, este Miguel, Hijo del Hombre e Hijo de Dios, fue proclamado gobernante establecido de Nebadon. La octava proclamación de la soberanía de Miguel se recibió desde Uversa, mientras que desde el Paraíso llegó la declaración conjunta del Padre Universal y del Hijo Eterno constituyendo a esta unión de Dios y del hombre como jefe exclusivo del universo, y ordenando al Unión de los Días destinado en Salvington que indicara su intención de retirarse al Paraíso. Los Fieles de los Días de las sedes de las constelaciones también recibieron la orden de retirarse de los consejos de los Altísimos. Pero Miguel no quiso consentir la renuncia de los Hijos Trinitarios como consejeros y cooperadores. Los reunió en Salvington y les rogó personalmente que permanecieran de servicio para siempre en Nebadon. Éstos indicaron a sus directores en el Paraíso el deseo de obedecer esta petición, y poco después se promulgaron los mandatos que separaban del Paraíso y destinaban para siempre a estos Hijos del universo central a la corte de Miguel de Nebadon\footnote{\textit{Estado actual de Miguel}: Mc 16:19; Ef 1:20-23; Heb 1:1-4; 8:1-2; 1 P 3:22.}.

\par
%\textsuperscript{(1318.1)}
\textsuperscript{119:8.2} Se necesitaron casi mil millones de años del tiempo de Urantia para terminar la carrera donadora de Miguel y llevar a cabo el establecimiento definitivo de su autoridad suprema en el universo que él mismo había creado. Miguel nació como creador, fue educado como administrador, formado como dirigente, pero se le exigió que ganara su soberanía por experiencia. Vuestro pequeño mundo ha sido así conocido en todo Nebadon como el lugar donde Miguel terminó la experiencia que se le exige a todo Hijo Creador Paradisiaco antes de concedérsele la dirección y el control ilimitados sobre el universo creado por él mismo. A medida que ascendáis en el universo local, aprenderéis más cosas sobre los ideales de las personalidades implicadas en las donaciones anteriores de Miguel.

\par
%\textsuperscript{(1318.2)}
\textsuperscript{119:8.3} Al concluir sus donaciones como criatura, Miguel no sólo establecía su propia soberanía, sino que también acrecentaba la soberanía evolutiva de Dios Supremo. En el transcurso de estas donaciones, el Hijo Creador no solamente se dedicó a una exploración descendente de las diversas naturalezas de la personalidad de las criaturas, sino que también consiguió revelar las voluntades variadamente diversificadas de las Deidades del Paraíso, cuya unidad sintética, tal como la revelan los Creadores Supremos, pone de manifiesto la voluntad del Ser Supremo.

\par
%\textsuperscript{(1318.3)}
\textsuperscript{119:8.4} Estos diversos aspectos volitivos de las Deidades están eternamente personalizados en las diferentes naturalezas de los Siete Espíritus Maestros, y cada una de las donaciones de Miguel reveló de manera particular una de estas manifestaciones de la divinidad. En su donación como Melquisedek manifestó la voluntad unida del Padre, el Hijo y el Espíritu; en su donación como Lanonandek, la voluntad del Padre y del Hijo; en la donación adámica reveló la voluntad del Padre y del Espíritu; en la donación seráfica, la voluntad del Hijo y del Espíritu; en la donación como mortal en Uversa describió la voluntad del Actor Conjunto; en la donación como mortal morontial, la voluntad del Hijo Eterno; y en la donación material en Urantia vivió la voluntad del Padre Universal\footnote{\textit{Jesús vivió la voluntad del Padre}: Mt 26:39,42,44; Mc 14:36,39; Lc 22:42; Jn 4:34; 5:30; 6:38-40; 15:10; 17:4.}, incluso como un mortal de carne y hueso.

\par
%\textsuperscript{(1318.4)}
\textsuperscript{119:8.5} La finalización de estas siete donaciones condujo a la liberación de la soberanía suprema de Miguel y también a crear la posibilidad de la soberanía del Supremo en Nebadon. Miguel no reveló a Dios Supremo en ninguna de sus donaciones, pero la suma total de las siete donaciones es una nueva revelación del Ser Supremo en Nebadon.

\par
%\textsuperscript{(1318.5)}
\textsuperscript{119:8.6} En la experiencia de descender desde Dios hasta el hombre, Miguel experimentó al mismo tiempo la ascensión desde la posibilidad de manifestarse parcialmente hasta la supremacía de la acción finita y la liberación final de su potencial para actuar de manera absonita. Miguel, el Hijo Creador, es un creador espacio-temporal, pero Miguel, el Hijo Maestro séptuple, es un miembro de uno de los cuerpos divinos que componen la Trinidad Última.

\par
%\textsuperscript{(1318.6)}
\textsuperscript{119:8.7} Al pasar por la experiencia de revelar las voluntades de los Siete Espíritus Maestros surgidos de la Trinidad, el Hijo Creador ha pasado por la experiencia de revelar la voluntad del Supremo. Al actuar como revelador de la voluntad de la Supremacía, Miguel, junto con todos los demás Hijos Maestros, se ha identificado eternamente con el Supremo. En esta era del universo, Miguel revela al Supremo y participa en el proceso de hacer que se manifieste la soberanía de la Supremacía. Pero en la próxima era del universo, creemos que colaborará con el Ser Supremo en la primera Trinidad experiencial a favor de los universos del espacio exterior y en ellos.

\par
%\textsuperscript{(1319.1)}
\textsuperscript{119:8.8} Urantia es el santuario sentimental de todo Nebadon, la esfera principal entre diez millones de mundos habitados, el hogar humano de Cristo Miguel, soberano de todo Nebadon, ministro Melquisedek para los reinos, salvador de un sistema, liberador adámico, compañero seráfico, asociado de los espíritus ascendentes, progresor morontial, Hijo del Hombre en la similitud de la carne mortal y Príncipe Planetario de Urantia. Vuestras escrituras dicen la verdad cuando afirman que este mismo Jesús ha prometido regresar\footnote{\textit{El regreso de Jesús}: Jn 14:3,28.} algún día al mundo de su donación final, al Mundo de la Cruz.

\par
%\textsuperscript{(1319.2)}
\textsuperscript{119:8.9} [Este documento, que describe las siete donaciones de Cristo Miguel, es el sexagésimo tercero de una serie de presentaciones, patrocinadas por numerosas personalidades, que narran la historia de Urantia hasta la época de la aparición de Miguel en la Tierra en la similitud de la carne mortal. Estos documentos fueron autorizados por una comisión de doce seres de Nebadon que actuaron bajo la dirección de Mantutia Melquisedek. Redactamos estas narraciones y las tradujimos a la lengua inglesa mediante una técnica autorizada por nuestros superiores, en el año 1935 d. de J.C. del tiempo de Urantia.]



\chapter{Documento 120. La donación de Miguel en Urantia}
\par 
%\textsuperscript{(1323.1)} 
\textsuperscript{120:0.1} DESIGNADO por Gabriel para supervisar la nueva exposición de la vida de Miguel cuando estuvo en Urantia en la similitud de la carne mortal, yo, el Melquisedek director de la comisión reveladora encargada de esta tarea, estoy autorizado a presentar esta narración sobre ciertos acontecimientos que precedieron de inmediato la llegada del Hijo Creador a Urantia para emprender la fase final de su experiencia de donación en el universo. Vivir esas vidas idénticas que él impone a los seres inteligentes de su propia creación, donarse así en la similitud de sus diversas órdenes de seres creados, es una parte del precio que cada Hijo Creador debe pagar para conseguir la soberanía completa y suprema sobre el universo de cosas y de seres creado por él\footnote{\textit{La donación de Miguel en Urantia}: Jn 1:1-18.}.

\par 
%\textsuperscript{(1323.2)}
\textsuperscript{120:0.2} Antes de los acontecimientos que estoy a punto de describir, Miguel de Nebadon se había donado seis veces en la similitud de seis órdenes diferentes de su variada creación de seres inteligentes. Luego se preparó para descender a Urantia en la similitud de los mortales, la orden más humilde de sus criaturas volitivas inteligentes y, como tal humano del reino material, ejecutar el acto final del drama consistente en conseguir la soberanía sobre su universo de acuerdo con los mandatos de los divinos Gobernantes Paradisiacos del universo de universos.

\par 
%\textsuperscript{(1323.3)}
\textsuperscript{120:0.3} En el transcurso de cada una de las donaciones anteriores, Miguel no sólo había adquirido la experiencia finita de un grupo de sus seres creados, sino que también había adquirido una experiencia esencial de cooperación con el Paraíso que, en sí misma y por sí misma, contribuiría además a establecerlo como soberano del universo creado por él. En cualquier momento durante todas las épocas pasadas del universo local, Miguel podía haber afirmado su soberanía personal como Hijo Creador, y, como Hijo Creador, podía haber gobernado su universo de la manera que hubiera escogido. En ese caso, Emmanuel y los Hijos Paradisiacos asociados se habrían marchado del universo. Pero Miguel no deseaba gobernar Nebadon simplemente por su propio derecho aislado, como Hijo Creador. Deseaba ascender, mediante una experiencia efectiva de subordinación cooperativa a la Trinidad del Paraíso, hasta esa elevada posición en el estatus universal en la que estaría cualificado para gobernar su universo y administrar sus asuntos con esa perfección de perspicacia y esa sabiduría de ejecución que algún día caracterizarán al gobierno sublime del Ser Supremo. No aspiraba a la perfección de gobierno como Hijo Creador, sino a la supremacía administrativa como personificación de la sabiduría universal y de la experiencia divina del Ser Supremo.

\par 
%\textsuperscript{(1324.1)}
\textsuperscript{120:0.4} Miguel tenía, por tanto, una doble finalidad al efectuar estas siete donaciones a las diversas órdenes de criaturas de su universo: en primer lugar, completaba la experiencia obligatoria de comprender a las criaturas, que se exige a todos los Hijos Creadores antes de que asuman la soberanía completa. En cualquier momento, un Hijo Creador puede gobernar su universo por su propio derecho, pero sólo puede gobernar como representante supremo de la Trinidad del Paraíso después de pasar por las siete donaciones a las criaturas de su universo. En segundo lugar, aspiraba al privilegio de representar la máxima autoridad de la Trinidad del Paraíso que se puede ejercer en la administración directa y personal de un universo local. En consecuencia, durante la experiencia de cada una de sus donaciones en el universo, Miguel se subordinó voluntariamente, de manera satisfactoria y aceptable, a las voluntades combinadas de las diversas asociaciones de las personas de la Trinidad del Paraíso. Es decir, en la primera donación se sometió a la voluntad combinada del Padre, del Hijo y del Espíritu; en la segunda, a la voluntad del Padre y del Hijo; en la tercera, a la voluntad del Padre y del Espíritu; en la cuarta, a la voluntad del Hijo y del Espíritu; en la quinta, a la voluntad del Espíritu Infinito; en la sexta, a la voluntad del Hijo Eterno; y durante la séptima y última donación en Urantia, a la voluntad del Padre Universal\footnote{\textit{Revelar la voluntad del Padre Universal}: Mt 26:39,42,44; Mc 14:36,39; Lc 22:42; Jn 4:34; 5:30; 6:38-40; 15:10; 17:4.}.

\par 
%\textsuperscript{(1324.2)}
\textsuperscript{120:0.5} Miguel combina pues, en su soberanía personal, la voluntad divina de las fases séptuples de los Creadores universales con la experiencia comprensiva de las criaturas de su universo local. Su administración se ha vuelto así representativa del máximo poder y autoridad, pero desprovista de toda apropiación arbitraria. Su poder es ilimitado, pues procede de una asociación experimentada con las Deidades del Paraíso; su autoridad es incuestionable, ya que fue conseguida mediante una experiencia real en la similitud de las criaturas del universo; su soberanía es suprema, puesto que expresa al mismo tiempo el punto de vista séptuple de la Deidad del Paraíso y el punto de vista de las criaturas del tiempo y del espacio\footnote{\textit{Todas las cosas entregadas en manos de Miguel}: Mt 11:27a; Lc 10:22a.}.

\par 
%\textsuperscript{(1324.3)}
\textsuperscript{120:0.6} Después de determinar el momento de su donación final y después de elegir el planeta donde tendría lugar este acontecimiento extraordinario, Miguel mantuvo con Gabriel la conferencia habitual que precede a una donación, y luego se presentó ante Emmanuel, su hermano mayor y consejero paradisiaco. Miguel entregó entonces a la custodia de Emmanuel todos los poderes de la administración del universo que no habían sido conferidos previamente a Gabriel. Y justo antes de la partida de Miguel para encarnarse en Urantia, Emmanuel aceptó la custodia del universo durante el período de la donación en Urantia, y procedió a dar los consejos para la donación que servirían de guía a Miguel durante su encarnación cuando dentro de poco creciera en Urantia como un mortal del reino.

\par 
%\textsuperscript{(1324.4)}
\textsuperscript{120:0.7} A este respecto se debe tener en cuenta que Miguel había elegido efectuar esta donación en la similitud de la carne mortal sometido a la voluntad del Padre Paradisiaco. El Hijo Creador no necesitaba instrucciones de nadie para llevar a cabo esta encarnación si hubiera tenido el único propósito de conseguir la soberanía sobre su universo, pero había emprendido un programa de revelación del Supremo que implicaba un funcionamiento cooperativo con las diversas voluntades de las Deidades del Paraíso. Y así, cuando consiguiera final y personalmente su soberanía, englobaría realmente la voluntad séptuple de la Deidad tal como ésta culmina en el Supremo. Por ello, anteriormente había sido instruido seis veces por los representantes personales de las diversas Deidades del Paraíso y sus asociaciones; y ahora recibía las instrucciones del Unión de los Días, el embajador de la Trinidad del Paraíso en el universo local de Nebadon, que actuaba en nombre del Padre Universal.

\par 
%\textsuperscript{(1325.1)}
\textsuperscript{120:0.8} La buena disposición con que este poderoso Hijo Creador se subordinaba voluntariamente una vez más a la voluntad de las Deidades del Paraíso, en esta ocasión a la del Padre Universal, había de producir ventajas inmediatas y enormes compensaciones. Mediante esta decisión de efectuar un acto así de subordinación asociativa, Miguel iba a experimentar en esta encarnación no solamente la naturaleza del hombre mortal, sino también la voluntad del Padre Paradisiaco de todos. Además, podía emprender esta donación única con la completa seguridad de que no solamente Emmanuel ejercería la plena autoridad del Padre Paradisiaco en la administración de su universo durante su ausencia debida a la donación en Urantia, sino también con el conocimiento reconfortante de que los Ancianos de los Días del superuniverso habían decretado que su creación estaría segura durante todo el período de la donación.

\par 
%\textsuperscript{(1325.2)}
\textsuperscript{120:0.9} Éste era el escenario en el momento importante en que Emmanuel presentó el cometido de la séptima donación. Tengo permiso para exponer los extractos siguientes de las instrucciones de Emmanuel, antes de la donación, al gobernante del universo que después se convirtió en Jesús de Nazaret (Cristo Miguel) en Urantia:

\section*{1. Misión de la séptima donación}
\par 
%\textsuperscript{(1325.3)}
\textsuperscript{120:1.1} <<Mi hermano Creador, estoy a punto de presenciar tu séptima y última donación universal. Has ejecutado con gran fidelidad y perfección las seis misiones anteriores, y sólo puedo pensar que saldrás igualmente triunfante de ésta, tu donación final camino de la soberanía. Hasta ahora has aparecido en las esferas de tus donaciones como un ser plenamente desarrollado de la orden que habías escogido. Ahora estás a punto de aparecer en Urantia, el planeta desordenado y perturbado que has elegido, no como un mortal plenamente desarrollado, sino como un bebé indefenso. Esto, compañero mío, va a ser para ti una experiencia nueva y no probada. Estás a punto de pagar todo el precio de la donación y de experimentar la iluminación completa de la encarnación de un Creador en la similitud de una criatura>>.

\par 
%\textsuperscript{(1325.4)}
\textsuperscript{120:1.2} <<Durante cada una de tus donaciones anteriores, elegiste someterte voluntariamente a la voluntad de las tres Deidades del Paraíso y de sus interasociaciones divinas. De las siete fases de la voluntad del Supremo, has estado sometido a todas ellas en tus anteriores donaciones, salvo a la voluntad personal de tu Padre Paradisiaco. Ahora que has decidido someterte por completo a la voluntad de tu Padre durante toda tu séptima donación, yo, como representante personal de nuestro Padre, asumo la jurisdicción incondicional sobre tu universo durante el período de tu encarnación>>.

\par 
%\textsuperscript{(1325.5)}
\textsuperscript{120:1.3} <<Al emprender la donación en Urantia, te has despojado voluntariamente de todo apoyo extraplanetario y de toda ayuda especial que hubiera podido prestarte cualquier criatura de tu propia creación. Al igual que tus hijos creados de Nebadon dependen totalmente de ti para conducirse con seguridad durante toda su carrera universal, ahora deberás depender enteramente y sin reservas de tu Padre Paradisiaco para conducirte con seguridad a través de las vicisitudes no reveladas de tu próxima carrera como mortal. Y cuando hayas terminado esta experiencia donadora, conocerás en toda su verdad el pleno sentido y el rico significado de esa confianza por la fe cuyo dominio exiges tan invariablemente a todas tus criaturas como parte de sus relaciones íntimas contigo, como Creador y Padre de su universo local>>.

\par 
%\textsuperscript{(1326.1)}
\textsuperscript{120:1.4} <<Durante toda tu donación en Urantia sólo tienes que preocuparte de una sola cosa, de la comunión ininterrumpida entre tú y tu Padre Paradisiaco; la perfección de esa relación permitirá que el mundo de tu donación, e incluso todo el universo creado por ti, contemplen una revelación nueva y más comprensible de tu Padre y de mi Padre, del Padre Universal de todos. Sólo tienes que preocuparte, pues, de tu vida personal en Urantia. Yo me haré plena y eficazmente responsable de la seguridad y de la administración ininterrumpida de tu universo desde el momento en que renuncies voluntariamente a tu autoridad hasta que regreses a nosotros como Soberano del Universo, confirmado por el Paraíso, y recibas nuevamente de mis manos, no la autoridad de vicegerente que ahora me entregas, sino en lugar de ella, el poder supremo y la jurisdicción sobre tu universo>>.

\par 
%\textsuperscript{(1326.2)}
\textsuperscript{120:1.5} <<Y para que puedas saber con seguridad que tengo la facultad de hacer todo lo que te prometo en este momento (sabiendo muy bien que soy la garantía de todo el Paraíso para el fiel cumplimiento de mi palabra), te anuncio que acaban de comunicarme un mandato de los Ancianos de los Días de Uversa que impedirá todo peligro espiritual en Nebadon durante el período de tu donación voluntaria. Desde el momento en que abandones tu conciencia, al principio de tu encarnación como mortal, hasta que regreses a nosotros como soberano supremo e incondicional de este universo que tú mismo has creado y organizado, nada grave podrá ocurrir en todo Nebadon. Durante el ínterin de tu encarnación, poseo las instrucciones de los Ancianos de los Días que ordenan inequívocamente la destrucción instantánea y automática de cualquier ser culpable de rebelión o que se atreva a instigar una insurrección en el universo de Nebadon mientras estés ausente durante esta donación. Hermano mío, a la vista de la autoridad del Paraíso inherente a mi presencia y acrecentada por el mandato judicial de Uversa, tu universo y todas sus criaturas leales estarán a salvo durante tu donación. Puedes emprender tu misión con un solo pensamiento ---ampliar la revelación de nuestro Padre a los seres inteligentes de tu universo>>.

\par 
%\textsuperscript{(1326.3)}
\textsuperscript{120:1.6} <<Como en cada una de tus donaciones anteriores, quisiera recordarte que recibo la jurisdicción sobre tu universo en calidad de hermano fideicomisario. Ejerzo toda la autoridad y uso todo el poder en tu nombre. Actúo como lo haría nuestro Padre Paradisiaco y de acuerdo con tu petición explícita de que actúe así en tu lugar. Así las cosas, toda esta autoridad delegada podrás ejercerla de nuevo en cualquier momento que estimes oportuno solicitar su restitución. Tu donación es totalmente voluntaria en todas sus fases. Como mortal encarnado en el mundo, estarás desprovisto de facultades celestiales, pero podrás recuperar todo el poder abandonado en cualquier momento que decidas reasumir tu autoridad universal. Si eligieras reinstalarte en tu poder y tu autoridad, recuerda que sería enteramente por razones \textit{personales}, puesto que soy la garantía viviente y suprema cuya presencia y promesa aseguran la administración intacta de tu universo de acuerdo con la voluntad de tu Padre. Una rebelión como ya ha ocurrido tres veces en Nebadon no puede producirse durante tu ausencia de Salvington para esta donación. Para el período de tu donación en Urantia, los Ancianos de los Días han decretado que toda rebelión en Nebadon contendrá la semilla automática de su propia aniquilación>>.

\par 
%\textsuperscript{(1326.4)}
\textsuperscript{120:1.7} <<Mientras estés ausente debido a esta donación final y extraordinaria, me comprometo (con la cooperación de Gabriel) a administrar fielmente tu universo; al encomendarte que emprendas este ministerio de revelación divina y que pases por esta experiencia de comprensión perfeccionada de los humanos, actúo en nombre de mi Padre y tu Padre, y te ofrezco los consejos siguientes que deberían guiarte para vivir tu vida terrestre a medida que tomes conciencia progresivamente de la misión divina de tu estancia continuada en la carne:>>

\section*{2. Las limitaciones de la donación}
\par 
%\textsuperscript{(1327.1)}
\textsuperscript{120:2.1} <<1. De acuerdo con las costumbres y en conformidad con la técnica de Sonarington ---de acuerdo con los mandatos del Hijo Eterno del Paraíso--- lo he previsto todo para que puedas emprender inmediatamente esta donación como mortal en armonía con los planes formulados por ti y que Gabriel me ha entregado para su custodia. Crecerás en Urantia como un hijo del planeta, completarás tu educación humana ---sometido en todo momento a la voluntad de tu Padre Paradisiaco--- vivirás tu vida en Urantia como lo has determinado, terminarás tu estancia planetaria y te prepararás para ascender hasta tu Padre y recibir de él la soberanía suprema sobre tu universo>>\footnote{\textit{Jesús vive sometido a la voluntad del Padre}: Mt 26:39,42,44; Mc 14:36,39; Lc 22:42; Jn 4:34; 5:30; 6:38-40; 15:10; 17:4.}.

\par 
%\textsuperscript{(1327.2)}
\textsuperscript{120:2.2} \guillemotleft2. Aparte de tu misión en la Tierra y de tu revelación al universo, pero inherente a las dos, te aconsejo que, una vez que seas suficientemente consciente de tu identidad divina, asumas la tarea adicional de poner fin técnicamente a la rebelión de Lucifer en el sistema de Satania, y que hagas todo esto como \textit{Hijo del Hombre}. Así pues, como una criatura mortal del mundo que en su debilidad se ha hecho poderosa porque se ha sometido por la fe a la voluntad de su Padre, te sugiero que lleves a cabo con benevolencia todo lo que tantas veces te has negado a realizar arbitrariamente por la fuerza y el poder cuando disponías de estos atributos en la época en que empezó esta rebelión pecaminosa e injustificada. Yo consideraría como una digna culminación de tu donación como mortal que volvieras entre nosotros como Hijo del Hombre, Príncipe Planetario de Urantia, a la vez que como Hijo de Dios, soberano supremo de tu universo. Como hombre mortal, el tipo más inferior de criatura inteligente en Nebadon, haz frente y juzga las pretensiones blasfemas de Caligastia y de Lucifer, y en el humilde estado que habrás asumido, pon fin para siempre a las tergiversaciones vergonzosas de estos hijos de la luz caídos\footnote{\textit{``Hijos de la luz''}: Lc 16:8; Jn 12:36; Ef 5:8; 1 Ts 5:5.}. Ya que has rehusado continuamente desacreditar a estos rebeldes mediante el ejercicio de tus prerrogativas como creador, sería conveniente que ahora, en la similitud de las criaturas más humildes de tu creación, arrebates el poder de las manos de estos Hijos caídos; y así todo tu universo local reconocerá con toda equidad, claramente y para siempre, que has sido justo al hacer, en la forma de la carne mortal, aquellas cosas que la misericordia no te exhortó a hacer con el poder de una autoridad arbitraria. Habiendo establecido así, por medio de tu donación, la posibilidad de la soberanía del Supremo en Nebadon, habrás llevado efectivamente a su término los asuntos pendientes de todas las insurrecciones anteriores, a pesar de la mayor o menor cantidad de tiempo que te lleve realizar esta tarea. Esta acción eliminará lo más esencial de las disensiones pendientes en tu universo. Cuando recibas posteriormente la soberanía suprema sobre tu universo, en ninguna parte de tu gran creación personal podrán producirse desafíos similares a tu autoridad\guillemotright\footnote{\textit{Finalización de la rebelión}: Is 14:12-20; Mt 4:1-11; Mc 1:12-13; Lc 4:1-14; 10:18; 2 P 2:4; Ap 12:7-9.}.

\par 
%\textsuperscript{(1327.3)}
\textsuperscript{120:2.3} <<3. Cuando hayas logrado poner fin a la secesión en Urantia, cosa que harás indudablemente, te aconsejo que aceptes que Gabriel te confiera el título de `Príncipe Planetario de Urantia' como reconocimiento eterno de tu universo por tu experiencia final de donación, y que además hagas todo lo posible, que sea consecuente con el significado de tu donación, por reparar la aflicción y la confusión causadas en Urantia por la traición de Caligastia y la falta adámica posterior>>.

\par 
%\textsuperscript{(1328.1)}
\textsuperscript{120:2.4} \guillemotleft4. De acuerdo con tu petición, Gabriel y todos los interesados cooperarán contigo en el deseo que has expresado de terminar tu donación en Urantia con la declaración de un juicio dispensacional\footnote{\textit{Juicio dispensacional}: Mt 27:52-53; Jn 5:25-29.} del planeta, acompañado por el final de una era, la resurrección de los supervivientes\footnote{\textit{Resurrección de los muertos}: Mt 27:52-53.} mortales dormidos y el establecimiento de la dispensación del Espíritu de la Verdad\footnote{\textit{Espíritu de la Verdad}: Ez 11:19; 18:31; 36:26-27; Jl 2:28-29; Lc 24:49; Jn 7:39; 14:16-18,23,26; 15:4,26; 16:7-8,13-14; 17:21-23; Hch 1:5,8a; 2:1-4,16-18; 2:33; 2 Co 13:5; Gl 2:20; 4:6; Ef 1:13; 4:30; 1 Jn 4:12-15.} otorgado\guillemotright.

\par 
%\textsuperscript{(1328.2)}
\textsuperscript{120:2.5} <<5. En lo que se refiere al planeta de tu donación y a la generación inmediata de hombres que vivirán allí en la época de tu estancia como mortal, te aconsejo que desempeñes principalmente el papel de instructor. Concede tu atención, en primer lugar, a la liberación y a la inspiración de la naturaleza espiritual del hombre. A continuación, ilumina el intelecto ensombrecido de los hombres, cura sus almas y libera sus mentes de los temores seculares. Y luego, de acuerdo con tu sabiduría humana, contribuye al bienestar físico y a la comodidad material de tus hermanos en la carne. Vive la vida religiosa ideal para inspirar y edificar a todo tu universo>>.

\par 
%\textsuperscript{(1328.3)}
\textsuperscript{120:2.6} \guillemotleft6. En el planeta de tu donación, libera espiritualmente al hombre aislado por la rebelión\footnote{\textit{Libertad a los cautivos espirituales}: Is 42:5-7; 49:9; 61:1; Lc 4:18; Gl 5:1,13.}. En Urantia, haz una contribución adicional a la soberanía del Supremo, extendiendo así el establecimiento de esta soberanía por todos los amplios dominios de tu creación personal. En esta donación material en la similitud de la carne, estás a punto de experimentar la iluminación final de un Creador espacio-temporal, la doble experiencia de trabajar dentro de la naturaleza del hombre con la voluntad de tu Padre Paradisiaco. En tu vida temporal, la voluntad de la criatura finita y la voluntad del Creador infinito han de convertirse en una sola, tal como se están uniendo también en la Deidad evolutiva del Ser Supremo. Derrama sobre el planeta de tu donación el Espíritu de la Verdad para que todos los mortales normales de esa esfera aislada tengan así un acceso inmediato y completo al ministerio de la presencia separada de nuestro Padre Paradisiaco, los Ajustadores del Pensamiento de los mundos\guillemotright.

\par 
%\textsuperscript{(1328.4)}
\textsuperscript{120:2.7} <<7. En todo lo que vayas a hacer en el mundo de tu donación, recuerda siempre que estás viviendo una vida para la instrucción y la edificación de todo tu universo. Vas a \textit{donar} esta vida de encarnación mortal en Urantia, pero debes \textit{vivir} dicha vida para inspirar espiritualmente a todas las inteligencias humanas y superhumanas que han vivido, existen ahora o puedan vivir en cada mundo habitado que ha formado, forma ahora o pueda formar parte de la inmensa galaxia de tu dominio administrativo. Tu vida terrestre en la similitud de la carne mortal no debes vivirla para que sirva de \textit{ejemplo} a los mortales de Urantia de la época de tu estancia en la Tierra, ni para ninguna generación posterior de seres humanos de Urantia o de cualquier otro mundo. En lugar de eso, tu vida encarnada en Urantia será una \textit{inspiración} para todas las vidas de todos los mundos de Nebadon de todas las generaciones de los tiempos por venir>>.

\par 
%\textsuperscript{(1328.5)}
\textsuperscript{120:2.8} \guillemotleft8. La gran misión que debes realizar y experimentar en la encarnación mortal está contenida en tu decisión de vivir una vida totalmente dedicada a hacer la voluntad de tu Padre Paradisiaco\footnote{\textit{Jesús dedicado a vivir la voluntad del Padre}: Mt 26:39,42,44; Mc 14:36,39; Lc 22:42; Jn 4:34; 5:30; 6:38-40; 15:10; 17:4.}, y así \textit{revelar a Dios}, tu Padre, en la carne y especialmente a las criaturas de carne. Al mismo tiempo, \textit{interpretarás} también con un nuevo realce a nuestro Padre para los seres supermortales de todo Nebadon. Junto con este ministerio de nueva revelación y de interpretación ampliada del Padre Paradisiaco\footnote{\textit{Jesús dedicado a revelar a Dios a la humanidad}: Mt 5:45,48; 6:1,4,6; 11:25-27; Mc 11:25-26; Lc 6:35-36; 10:22; Jn 1:18; 3:31-34; 4:21-23; 6:45-46; 14:6-11,20; 15:15; 16:25; 17:8,25-26.} para los tipos de mente humana y superhumana, actuarás también de tal manera que efectuarás una nueva revelación del hombre a Dios. Demuestra en tu corta y única vida en la carne, como nunca antes se ha visto en todo Nebadon, las posibilidades trascendentes que puede alcanzar un humano que conoce a Dios durante la breve carrera de la existencia mortal, y efectúa una \textit{interpretación} nueva y reveladora del hombre y de las vicisitudes de su vida planetaria a todas las inteligencias superhumanas de todo Nebadon y para todos los tiempos. Vas a descender a Urantia en la similitud de la carne mortal, y al vivir como un hombre de tu tiempo y de tu generación, actuarás de tal manera que mostrarás a todo tu universo el ideal de una técnica perfeccionada en el compromiso supremo de los asuntos de tu inmensa creación: la hazaña de Dios que busca al hombre y lo encuentra, y el fenómeno del hombre que busca a Dios y lo encuentra; hacer todo esto para su satisfacción mutua, y hacerlo durante una corta vida en la carne\guillemotright.

\par 
%\textsuperscript{(1329.1)}
\textsuperscript{120:2.9} <<9. Te advierto que tengas siempre presente que, aunque de hecho te vas a convertir en un hombre normal del mundo, seguirás siendo en potencia un Hijo Creador del Padre Paradisiaco. Durante toda esta encarnación, aunque vas a vivir y actuar como Hijo del Hombre, los atributos creativos de tu divinidad personal irán contigo desde Salvington a Urantia. Tu voluntad siempre tendrá el poder de dar por terminada la encarnación en cualquier momento posterior a la llegada de tu Ajustador del Pensamiento. Antes de la llegada y de la recepción del Ajustador, yo garantizaré la integridad de tu personalidad. Pero después de la llegada de tu Ajustador, y a medida que reconozcas progresivamente la naturaleza y la importancia de tu misión donadora, deberías abstenerte de formular cualquier deseo superhumano de obtener, de conseguir o de poder, debido al hecho de que tus prerrogativas como creador permanecerán asociadas a tu personalidad mortal, porque estos atributos son inseparables de tu presencia personal. Pero, aparte de la voluntad del Padre Paradisiaco, ninguna repercusión superhumana acompañará tu carrera terrestre, a menos que tú, mediante un acto de voluntad consciente y deliberada, tomes una decisión indivisa que conduzca a la elección de toda tu personalidad>>.

\section*{3. Consejos y advertencias adicionales}
\par 
%\textsuperscript{(1329.2)}
\textsuperscript{120:3.1} <<Y ahora, hermano mío, al despedirme de ti mientras te preparas para partir hacia Urantia, y después de haberte aconsejado sobre la conducta general de tu donación, permíteme presentarte algunas advertencias que son el resultado de una deliberación con Gabriel y que se refieren a aspectos menores de tu vida como mortal. Así pues, te sugerimos además que:>>

\par 
%\textsuperscript{(1329.3)}
\textsuperscript{120:3.2} <<1. En la búsqueda del ideal de tu vida mortal en la Tierra, concedas también alguna atención a la realización y ejemplificación de algunas cosas prácticas e inmediatamente útiles para tus compañeros humanos>>.

\par 
%\textsuperscript{(1329.4)}
\textsuperscript{120:3.3} <<2. En lo que concierne a las relaciones familiares, da prioridad a las costumbres aceptadas de la vida familiar tal como las encuentres establecidas en la época y en la generación de tu donación. Vive tu vida familiar y comunitaria de acuerdo con las prácticas del pueblo en el que has elegido aparecer>>.

\par 
%\textsuperscript{(1329.5)}
\textsuperscript{120:3.4} <<3. En tus relaciones con el orden social, te aconsejamos que limites tus esfuerzos principalmente a la regeneración espiritual y a la emancipación intelectual. Evita todo enredo con la estructura económica y los compromisos políticos de tu época. Conságrate en especial a vivir la vida religiosa ideal en Urantia>>.

\par 
%\textsuperscript{(1329.6)}
\textsuperscript{120:3.5} <<4. En ninguna circunstancia, ni siquiera en el más mínimo detalle, debes interferir en la evolución progresiva, normal y ordenada de las razas de Urantia. Pero no se debe interpretar que esta prohibición limita tus esfuerzos por dejar detrás de ti, en Urantia, un sistema duradero y mejorado de \textit{ética religiosa positiva}. Como Hijo dispensacional, se te han concedido ciertos privilegios relacionados con el avance del estado \textit{espiritual} y \textit{religioso} de los pueblos del mundo>>.

\par 
%\textsuperscript{(1330.1)}
\textsuperscript{120:3.6} <<5. Si lo consideras conveniente, puedes identificarte con los movimientos religiosos y espirituales existentes que puedan encontrarse en Urantia, pero trata de evitar, de todas las maneras posibles, el establecimiento formal de un culto organizado, de una religión cristalizada o de una agrupación ética separada de seres humanos. Tu vida y tus enseñanzas deben convertirse en la herencia común de todas las religiones y de todos los pueblos>>.

\par 
%\textsuperscript{(1330.2)}
\textsuperscript{120:3.7} <<6. Con el fin de que no contribuyas innecesariamente a la creación de sistemas estereotipados posteriores de creencias religiosas en Urantia, o de otros tipos de lealtades religiosas no progresivas, te aconsejamos además: No dejes ningún escrito detrás de ti en el planeta. Abstente de escribir en materiales permanentes; ordena a tus asociados que no hagan imágenes u otros retratos de tu aspecto físico. Asegúrate de que nada potencialmente idólatra se quede en el planeta en el momento de tu partida>>.

\par 
%\textsuperscript{(1330.3)}
\textsuperscript{120:3.8} <<7. Aunque vivirás la vida social normal y corriente del planeta, y serás un individuo normal del sexo masculino, es probable que no entres en la relación del matrimonio, una relación que sería perfectamente honorable y compatible con tu donación; pero debo recordarte que uno de los mandatos de Sonarington, relativos a la encarnación, prohíbe que un Hijo donador originario del Paraíso deje tras de sí una descendencia humana en un planeta cualquiera>>.

\par 
%\textsuperscript{(1330.4)}
\textsuperscript{120:3.9} <<8. Para todos los demás detalles de tu próxima donación, quisiéramos confiarte a la dirección de tu Ajustador interior, a las enseñanzas del espíritu divino siempre presente que guía a los hombres, y al juicio razonable de tu mente humana de origen hereditario y en expansión. Una asociación así de atributos de criatura y de Creador te permitirá vivir para nosotros la vida perfecta del hombre en las esferas planetarias, no necesariamente perfecta tal como la pueda considerar cualquier hombre de cualquier generación en cualquier mundo (y mucho menos en Urantia), pero será evaluada como total y supremamente plena por los mundos más perfeccionados y en vías de perfeccionarse de tu extenso universo>>.

\par 
%\textsuperscript{(1330.5)}
\textsuperscript{120:3.10} <<Y ahora, que tu Padre y mi Padre, que siempre nos ha sostenido en todas las actividades pasadas, te guíe, te sostenga y esté contigo desde el momento en que nos dejes y lleves a cabo el abandono de la conciencia de tu personalidad, durante tu reconocimiento gradual de tu identidad divina encarnada en una forma humana, y luego durante toda tu experiencia de donación en Urantia, hasta tu liberación de la carne y tu ascensión a la derecha de la soberanía de nuestro Padre. Cuando vuelva a verte en Salvington, te acogeremos a tu regreso como soberano supremo e incondicional de este universo que tú mismo has creado, servido y comprendido por completo>>.

\par 
%\textsuperscript{(1330.6)}
\textsuperscript{120:3.11} <<Ahora reino en tu lugar. Asumo la jurisdicción sobre todo Nebadon como soberano en funciones durante el ínterin de tu séptima donación, la de un mortal en Urantia. A ti, Gabriel, te encomiendo la salvaguardia del que está a punto de ser el Hijo del Hombre, hasta que pronto regrese a mí envuelto en poder y gloria como Hijo del Hombre e Hijo de Dios. Gabriel, y yo soy tu soberano hasta que Miguel regrese así>>.

\par 
%\textsuperscript{(1330.7)}
\textsuperscript{120:3.12} Luego, en presencia de todo Salvington reunido, Miguel se retiró inmediatamente de entre nosotros, y ya no volvimos a verlo en su sitio de costumbre hasta que regresó como soberano supremo y personal del universo, después de finalizar su carrera de donación en Urantia.

\section*{4. La encarnación --- la unión de dos en uno}
\par 
%\textsuperscript{(1331.1)}
\textsuperscript{120:4.1} Así pues, ciertos hijos indignos de Miguel, que habían acusado a su padre-Creador de buscar egoístamente la soberanía y que se habían permitido insinuar que el Hijo Creador se mantenía en el poder de manera arbitraria y autocrática debido a la lealtad irracional de las criaturas sumisas de un universo engañado, iban a ser silenciados para siempre y a quedarse confundidos y desilusionados por la vida de servicio altruista que el Hijo de Dios empezaba ahora como Hijo del Hombre ---todo el tiempo sometido a <<la voluntad del Padre Paradisiaco>>\footnote{\textit{Vivir sometido a la voluntad del Padre}: Mt 26:39,42,44; Mc 14:36,39; Lc 22:42; Jn 4:34; 5:30; 6:38-40; 15:10; 17:4.}.

\par 
%\textsuperscript{(1331.2)}
\textsuperscript{120:4.2} Pero no os equivoquéis; aunque Cristo Miguel era verdaderamente un ser de origen dual, no era una personalidad doble. No era Dios en asociación \textit{con} el hombre, sino más bien Dios \textit{encarnado} en el hombre. Y siempre fue exactamente este ser combinado. El único factor progresivo en esta relación incomprensible fue la comprensión y el reconocimiento conscientes y graduales (por parte de su mente humana) de este hecho de ser Dios y hombre.

\par 
%\textsuperscript{(1331.3)}
\textsuperscript{120:4.3} Cristo Miguel no se volvió progresivamente Dios. Dios no se volvió hombre en algún momento vital de la vida terrestre de Jesús. Jesús fue Dios \textit{y} hombre ---siempre e incluso para siempre jamás\footnote{\textit{Jesús, Dios y hombre}: Jn 1:1,14; 10:30; 14:9-11; 17:11b,21-23a.}. Este Dios y este hombre eran, y son ahora, \textit{uno solo}, al igual que la Trinidad del Paraíso compuesta por tres seres es en realidad \textit{una} Deidad.

\par 
%\textsuperscript{(1331.4)}
\textsuperscript{120:4.4} Nunca perdáis de vista el hecho de que el propósito espiritual supremo de la donación de Miguel era realzar la \textit{revelación de Dios}\footnote{\textit{Realzar la revelación de Dios}: Mt 11:27; Lc 10:22; Jn 1:18; 6:45-46; 8:26-27; 12:43-44,49-50; 14:7-11; 17:7-9,25-26.}.

\par 
%\textsuperscript{(1331.5)}
\textsuperscript{120:4.5} Los mortales de Urantia tienen conceptos variables de lo milagroso, pero para nosotros, que vivimos como ciudadanos del universo local, hay pocos milagros, y entre éstos, las donaciones encarnadas de los Hijos Paradisiacos son con mucho los más misteriosos. La aparición de un Hijo divino en vuestro mundo por un proceso aparentemente natural, nosotros la consideramos como un milagro ---el funcionamiento de unas leyes universales que sobrepasan nuestra comprensión. Jesús de Nazaret era una persona milagrosa.

\par 
%\textsuperscript{(1331.6)}
\textsuperscript{120:4.6} A lo largo de toda esta experiencia extraordinaria, Dios Padre escogió manifestarse como siempre lo hace \textit{{}---de la manera habitual---} de la manera normal, natural y fiable de la actuación divina.


\chapter{Documento 121. La época de la donación de Miguel}
\par 
%\textsuperscript{(1332.1)}
\textsuperscript{121:0.1} SOY el intermedio secundario que estuvo en otro tiempo vinculado al apóstol Andrés, y actúo bajo la supervisión de una comisión de doce miembros de la Fraternidad Unida de los Intermedios de Urantia, patrocinada conjuntamente por el director que preside nuestra orden y por el Melquisedek mencionado anteriormente. Estoy autorizado a redactar la narración de los actos de la vida de Jesús de Nazaret tal como fueron observados por mi orden de criaturas terrestres, y tal como fueron después parcialmente registrados por el sujeto humano que estaba bajo mi custodia temporal. Sabiendo cómo su Maestro evitó tan escrupulosamente dejar testimonios escritos detrás de él, Andrés se negó firmemente a multiplicar las copias de su relato escrito. Una actitud similar por parte de los otros apóstoles de Jesús retrasó considerablemente la redacción de los Evangelios.

\section*{1. Occidente en el siglo primero después de Cristo}
\par 
%\textsuperscript{(1332.2)}
\textsuperscript{121:1.1} Jesús no vino a este mundo en una era de decadencia espiritual; en el momento de su nacimiento, Urantia estaba pasando por una reactivación del pensamiento espiritual y de la vida religiosa como no se había conocido en toda su historia anterior desde Adán, ni se ha repetido en ninguna época posterior. Cuando Miguel se encarnó en Urantia, el mundo ofrecía las condiciones más favorables para la donación del Hijo Creador que hubieran prevalecido nunca anteriormente o que hayan existido después. En los siglos inmediatamente anteriores a esta época, la cultura y el idioma griegos se habían extendido hacia Occidente y Oriente próximo, y los judíos, como eran una raza levantina de naturaleza mitad occidental y mitad oriental, estaban sumamente capacitados para utilizar este marco cultural y ling\"uístico a fin de difundir eficazmente una nueva religión tanto en el este como en el oeste. Estas circunstancias tan favorables lo eran aún más gracias al tolerante reinado político de los romanos en el mundo mediterráneo.

\par 
%\textsuperscript{(1332.3)}
\textsuperscript{121:1.2} Toda esta combinación de influencias mundiales se encuentra bien ilustrada en las actividades de Pablo, que siendo un hebreo entre los hebreos por su cultura religiosa\footnote{\textit{Pablo, hebreo entre los hebreos}: 2 Co 11:22; Flp 3:5.}, proclamó el evangelio de un Mesías judío en lengua griega\footnote{\textit{Pablo habla en lengua griega}: Hch 21:37-40; 22:2-3.}, mientras que él mismo era ciudadano romano\footnote{\textit{Pablo era ciudadano romano}: Hch 16:37-39; 22:24-29.}.

\par 
%\textsuperscript{(1332.4)}
\textsuperscript{121:1.3} En Occidente no se ha visto nada comparable a la civilización de los tiempos de Jesús ni antes ni después de aquella época. La civilización europea fue unificada y coordinada bajo una triple influencia extraordinaria:

\par 
%\textsuperscript{(1332.5)}
\textsuperscript{121:1.4} 1. El sistema político y social romano.

\par 
%\textsuperscript{(1332.6)}
\textsuperscript{121:1.5} 2. El idioma y la cultura de Grecia ---y hasta cierto punto, su filosofía.

\par 
%\textsuperscript{(1332.7)}
\textsuperscript{121:1.6} 3. La influencia en rápida expansión de las enseñanzas religiosas y morales de los judíos.

\par 
%\textsuperscript{(1332.8)}
\textsuperscript{121:1.7} Cuando Jesús nació, todo el mundo mediterráneo era un imperio unificado. Por primera vez en la historia del mundo, había buenas calzadas que conectaban entre sí muchos centros principales. Los mares estaban limpios de piratas, y una gran era de comercio y de viajes avanzaba rápidamente. Europa no volvió a disfrutar de un período así de comercio y de viajes hasta el siglo diecinueve después de Cristo.

\par 
%\textsuperscript{(1333.1)}
\textsuperscript{121:1.8} A pesar de la paz interior y de la prosperidad superficial del mundo greco-romano, la mayoría de los habitantes del imperio languidecía en la miseria y la pobreza. La clase alta poco numerosa era rica, pero la mayoría de la humanidad pertenecía a una clase baja miserable y empobrecida. En aquellos tiempos no había una clase media feliz y próspera; esta clase acababa de hacer su aparición en la sociedad romana.

\par 
%\textsuperscript{(1333.2)}
\textsuperscript{121:1.9} Las primeras luchas entre los Estados romano y parto en vías de expansión habían finalizado en el entonces reciente pasado, dejando a Siria en manos de los romanos. En la época de Jesús, Palestina y Siria disfrutaban de un período de prosperidad, de paz relativa y de extensas relaciones comerciales con los países tanto del este como del oeste.

\section*{2. El pueblo judío}
\par 
%\textsuperscript{(1333.3)}
\textsuperscript{121:2.1} Los judíos formaban parte de la raza semita más antigua, que incluía también a los babilonios, los fenicios y a los enemigos más recientes de Roma, los cartagineses. Durante la primera parte del primer siglo después de Cristo, los judíos eran el grupo más influyente de los pueblos semitas, y sucedió que ocupaban una posición geográfica particularmente estratégica en el mundo, tal como en aquel tiempo estaba gobernado y organizado para el comercio.

\par 
%\textsuperscript{(1333.4)}
\textsuperscript{121:2.2} Muchas de las grandes carreteras que unían a las naciones de la antig\"uedad pasaban por Palestina, que se convirtió así en el punto de encuentro, en el cruce de caminos, de tres continentes. Los viajeros, los comerciantes y los ejércitos de Babilonia, Asiria, Egipto, Siria, Grecia, Partia y Roma pasaron sucesivamente por Palestina. Desde tiempos inmemoriales, muchas rutas de caravanas procedentes de Oriente pasaban por alguna parte de esta región hacia los escasos buenos puertos de mar del extremo oriental del Mediterráneo, desde donde los barcos transportaban sus cargamentos a todo el Occidente marítimo. Y más de la mitad del tráfico de estas caravanas pasaba por la pequeña ciudad de Nazaret en Galilea, o cerca de ella.

\par 
%\textsuperscript{(1333.5)}
\textsuperscript{121:2.3} Aunque Palestina era la cuna de la cultura religiosa judía y el lugar de nacimiento del cristianismo, los judíos estaban diseminados por el mundo, residían en muchas naciones y comerciaban en todas las provincias de los Estados romano y parto.

\par 
%\textsuperscript{(1333.6)}
\textsuperscript{121:2.4} Grecia aportó un idioma y una cultura, Roma construyó las carreteras y unificó un imperio, pero la dispersión de los judíos, con sus más de doscientas sinagogas y sus comunidades religiosas bien organizadas repartidas aquí y allá por todo el mundo romano, proporcionó los centros culturales que fueron los primeros en acoger al nuevo evangelio del reino de los cielos, y desde ellos se extendió posteriormente hasta las regiones más remotas del mundo.

\par 
%\textsuperscript{(1333.7)}
\textsuperscript{121:2.5} Cada sinagoga judía toleraba un pequeño número de creyentes gentiles, de hombres <<devotos>>\footnote{\textit{Gentiles ``devotos''}: Hch 10:7; 17:4,17.} o <<temerosos de Dios>>\footnote{\textit{Gentiles ``temerosos de Dios''}: Hch 10:2,22.}, y fue precisamente en este grupo de prosélitos donde Pablo logró la mayor parte de sus primeros conversos al cristianismo. Incluso el templo de Jerusalén tenía un patio ornamentado para los gentiles. Había una relación muy estrecha entre la cultura, el comercio y el culto de Jerusalén y Antioquía. En Antioquía, los discípulos de Pablo fueron llamados por primera vez <<cristianos>>\footnote{\textit{En Antioquía se les llama por primera vez ``cristianos''}: Hch 11:26.}.

\par 
%\textsuperscript{(1333.8)}
\textsuperscript{121:2.6} La centralización del culto judío en el templo de Jerusalén constituyó tanto el secreto de la supervivencia de su monoteísmo como la promesa de que alimentaría y difundiría por el mundo un nuevo concepto ampliado de ese Dios único de todas las naciones y Padre de todos los mortales. El servicio del templo en Jerusalén representaba la supervivencia de un concepto cultural religioso en presencia de la caída de una serie de jefes nacionales y perseguidores raciales gentiles.

\par 
%\textsuperscript{(1334.1)}
\textsuperscript{121:2.7} Aunque el pueblo judío de esta época estuviera bajo la soberanía romana, gozaba de una gran autonomía gubernamental. Y cuando recordaba los actos heroicos de liberación, entonces recientes, de Judas Macabeo y de sus sucesores inmediatos, vibraba con la expectativa de la aparición inminente de un libertador aún más grande, el tan esperado Mesías\footnote{\textit{La mayor parte de los judíos esperaban el Mesías}: Lc 3:15.}.

\par 
%\textsuperscript{(1334.2)}
\textsuperscript{121:2.8} El secreto de la supervivencia de Palestina, el reino de los judíos, como un Estado semi-independiente, radicaba en la política exterior del gobierno romano, que deseaba mantener el control sobre la carretera palestina de tránsito entre Siria y Egipto, así como sobre las estaciones terminales occidentales de las rutas de las caravanas entre Oriente y Occidente. Roma no deseaba que una potencia cualquiera que surgiera en el Levante pudiera refrenar su expansión futura en estas regiones. La política de intrigas que tenía por objeto enfrentar a la Siria seléucida con el Egipto tolemaico, necesitaba conservar a Palestina como un Estado separado e independiente. La política romana, la degeneración de Egipto y el debilitamiento progresivo de los seléucidas ante el poder creciente de los partos, explican por qué, durante varias generaciones, un grupo pequeño y poco poderoso de judíos pudo mantener su independencia contra los seléucidas al norte y los tolomeos al sur. Los judíos atribuían esta libertad fortuita y esta independencia de la autoridad política de los pueblos más poderosos que los rodeaban, al hecho de que eran <<el pueblo elegido>>\footnote{\textit{Pueblo elegido}: 1 Re 3:8; 1 Cr 17:21-22; Sal 33:12; 105:6,43; 135:4; Is 41:8-9; 43:21; 44:1; Dt 7:6; 14:2.}, a la intervención directa de Yahvé. Esta actitud de superioridad racial hizo que les resultara mucho más difícil soportar el dominio romano, cuando éste cayó finalmente sobre su país. Pero incluso en ese triste momento, los judíos no quisieron comprender que su misión mundial era espiritual, y no política.

\par 
%\textsuperscript{(1334.3)}
\textsuperscript{121:2.9} En los tiempos de Jesús, los judíos eran anormalmente recelosos y desconfiados, porque estaban entonces gobernados por un extraño, Herodes el Idumeo\footnote{\textit{El rey Herodes}: Mt 2:1; Lc 1:5a.}, que se había apoderado de la jurisdicción de Judea granjeándose hábilmente el favor de los gobernantes romanos. Aunque Herodes profesara su lealtad a las observancias del ceremonial hebreo, se puso a construir templos para muchos dioses extranjeros.

\par 
%\textsuperscript{(1334.4)}
\textsuperscript{121:2.10} Las relaciones amistosas de Herodes con los gobernantes romanos permitían a los judíos viajar con seguridad por el mundo, lo que abrió así el camino para una mayor penetración de los judíos, con el nuevo evangelio del reino de los cielos, hasta regiones distantes del Imperio Romano y en las naciones aliadas. El reinado de Herodes también contribuyó mucho a la mezcla ulterior de las filosofías hebrea y helenística.

\par 
%\textsuperscript{(1334.5)}
\textsuperscript{121:2.11} Herodes construyó el puerto de Cesarea, cosa que también ayudó a que Palestina fuera el cruce de caminos del mundo civilizado. Murió en el año 4 a. de J.C., y su hijo Herodes Antipas gobernó en Galilea y Perea durante la juventud y el ministerio de Jesús, hasta el año 39 d. de J.C.. Antipas fue, como su padre, un gran constructor. Reconstruyó muchas ciudades de Galilea, incluyendo el importante centro comercial de Séforis.

\par 
%\textsuperscript{(1334.6)}
\textsuperscript{121:2.12} Los dirigentes religiosos y los maestros rabínicos de Jerusalén no tenían una gran simpatía por los galileos. Cuando Jesús nació, Galilea era más gentil que judía.

\section*{3. Entre los gentiles}
\par 
%\textsuperscript{(1334.7)}
\textsuperscript{121:3.1} Aunque las condiciones económicas y sociales del Estado romano no eran del orden más elevado, en todas partes reinaba una paz y una prosperidad internas que eran propicias para la donación de Miguel. En el primer siglo después de Cristo, la sociedad del mundo mediterráneo estaba formada por cinco clases bien definidas:

\par 
%\textsuperscript{(1335.1)}
\textsuperscript{121:3.2} 1. \textit{La aristocracia}. Las clases altas con dinero y con el poder oficial, los grupos dirigentes y privilegiados.

\par 
%\textsuperscript{(1335.2)}
\textsuperscript{121:3.3} 2. \textit{Los grupos comerciales}. Los mercaderes más poderosos y los banqueros, los negociantes ---los grandes importadores y exportadores--- los comerciantes internacionales.

\par 
%\textsuperscript{(1335.3)}
\textsuperscript{121:3.4} 3. \textit{La pequeña clase media}. Aunque este grupo era en efecto pequeño, era muy influyente, y proporcionó la espina dorsal moral de la iglesia cristiana primitiva, que animó a estos grupos a que continuaran ejerciendo sus diversos oficios y comercios. Entre los judíos, muchos fariseos pertenecían a esta clase de mercaderes.

\par 
%\textsuperscript{(1335.4)}
\textsuperscript{121:3.5} 4. \textit{El proletariado libre}. Este grupo tenía poca o ninguna influencia en la sociedad. Aunque estaban orgullosos de su libertad, estaban en gran desventaja, porque se veían obligados a competir con la mano de obra de los esclavos. Las clases altas los miraban con desprecio, considerando que eran inútiles excepto para <<la reproducción>>.

\par 
%\textsuperscript{(1335.5)}
\textsuperscript{121:3.6} 5. \textit{Los esclavos}. La mitad de la población del Estado romano se componía de esclavos; muchos de ellos eran individuos superiores que se abrían camino rápidamente en el proletariado libre e incluso entre los mercaderes. La mayoría era mediocre o muy inferior.

\par 
%\textsuperscript{(1335.6)}
\textsuperscript{121:3.7} La esclavitud, incluso de los pueblos superiores, era una característica de las conquistas militares romanas. El poder del amo sobre su esclavo era ilimitado. La iglesia cristiana primitiva estaba compuesta, en gran parte, por estos esclavos y las clases bajas.

\par 
%\textsuperscript{(1335.7)}
\textsuperscript{121:3.8} Los esclavos superiores a menudo recibían salarios que podían ahorrar para comprar su libertad. Muchos de estos esclavos emancipados llegaron a ocupar altas posiciones en el Estado, en la iglesia y en el mundo de los negocios. Debido precisamente a estas posibilidades, la iglesia cristiana primitiva se mostró muy tolerante con esta forma modificada de esclavitud.

\par 
%\textsuperscript{(1335.8)}
\textsuperscript{121:3.9} No había un problema social generalizado en el Imperio Romano del primer siglo después de Cristo. La mayoría de la población se contentaba con pertenecer al grupo en el que le había tocado en suerte nacer. Siempre había una puerta abierta por la que los individuos con talento y capacidad podían elevarse de las capas inferiores a las capas superiores de la sociedad romana, pero la gente normalmente estaba satisfecha con su categoría social. No tenían una conciencia de clase y tampoco consideraban que estas distinciones de clase fueran malas o injustas. El cristianismo no era en ningún sentido un movimiento económico que tuviera como meta paliar la miseria de las clases oprimidas.

\par 
%\textsuperscript{(1335.9)}
\textsuperscript{121:3.10} La mujer disfrutaba de más libertad en todo el Imperio Romano que en Palestina, con su situación restringida, pero la devoción familiar y la afectividad natural de los judíos sobrepasaban con mucho a las del mundo de los gentiles.

\section*{4. La filosofía de los gentiles}
\par 
%\textsuperscript{(1335.10)}
\textsuperscript{121:4.1} Desde un punto de vista moral, los gentiles eran ligeramente inferiores a los judíos; pero en el corazón de los gentiles más nobles existía un terreno abundante de bondad natural y un potencial de afecto humano donde podía germinar la semilla del cristianismo y producir una abundante cosecha de caracteres morales y de logros espirituales. El mundo de los gentiles estaba entonces dominado por cuatro grandes filosofías, todas más o menos derivadas del platonismo griego más antiguo. Estas escuelas filosóficas eran las siguientes:

\par 
%\textsuperscript{(1335.11)}
\textsuperscript{121:4.2} 1. \textit{Los epicúreos}. Esta escuela de pensamiento se dedicaba a la búsqueda de la felicidad. Los mejores epicúreos no eran dados a los excesos sensuales. Al menos, esta doctrina contribuyó a liberar a los romanos de una forma de fatalismo todavía más nefasta; enseñaba que los hombres podían hacer algo por mejorar su condición en la Tierra. Combatió eficazmente las supersticiones nacidas de la ignorancia.

\par 
%\textsuperscript{(1336.1)}
\textsuperscript{121:4.3} 2. \textit{Los estoicos}. El estoicismo era la filosofía superior de las clases más altas. Los estoicos creían que un Destino-Razón controlador dominaba toda la naturaleza. Enseñaban que el alma del hombre era divina y que estaba apresada en un cuerpo maligno de naturaleza física. El alma del hombre conseguía la libertad viviendo en armonía con la naturaleza, con Dios; así, la virtud se convertía en su propia recompensa. El estoicismo se elevó a una moralidad sublime, a unos ideales que nunca fueron superados después por ningún sistema de filosofía puramente humano. A pesar de que los estoicos se calificaban de <<descendientes de Dios>>\footnote{\textit{La enseñanza estoica de ser los ``descendientes de Dios''}: Hch 17:28-29.}, no consiguieron conocerlo y en consecuencia no lo encontraron. El estoicismo continuó siendo una filosofía y nunca se transformó en una religión. Sus seguidores trataban de adaptar sus mentes a la armonía de la Mente Universal, pero no lograron considerarse como hijos de un Padre amoroso. Pablo tenía una fuerte tendencia hacia el estoicismo cuando escribió: <<He aprendido a sentirme contento, cualquiera que sea mi situación>>\footnote{\textit{Aprender a contentarse}: Flp 4:11.}.

\par 
%\textsuperscript{(1336.2)}
\textsuperscript{121:4.4} 3. \textit{Los cínicos}. Aunque los cínicos remontaban su filosofía hasta Diógenes de Atenas, una gran parte de su doctrina procedía de los restos de las enseñanzas de Maquiventa Melquisedek. Anteriormente, el cinismo había sido más una religión que una filosofía. Al menos, los cínicos hicieron democrática su filosofía religiosa. En los campos y en las plazas de los mercados, predicaban continuamente su doctrina de que <<el hombre podía salvarse si quería>>. Predicaban la sencillez y la virtud, y animaban a los hombres a afrontar la muerte sin temor. Estos predicadores cínicos ambulantes contribuyeron mucho a preparar al pueblo, espiritualmente hambriento, para los misioneros cristianos que llegaron después. El método de sus sermones populares se parecía mucho a las Epístolas de Pablo en cuanto al modelo y al estilo.

\par 
%\textsuperscript{(1336.3)}
\textsuperscript{121:4.5} 4. \textit{Los escépticos}. El escepticismo afirmaba que el conocimiento era engañoso, y que el convencimiento y la seguridad eran imposibles. Se trataba de una actitud puramente negativa y nunca se extendió mucho.

\par 
%\textsuperscript{(1336.4)}
\textsuperscript{121:4.6} Estas filosofías eran semi-religiosas; muchas veces eran fortificantes, éticas y ennoblecedoras, pero normalmente estaban por encima del alcance de la gente común. Con la posible excepción del cinismo, se trataba de filosofías para los fuertes y los sabios, no de religiones de salvación destinadas incluso a los pobres y los débiles.

\section*{5. Las religiones de los gentiles}
\par 
%\textsuperscript{(1336.5)}
\textsuperscript{121:5.1} A lo largo de todas las eras anteriores, la religión había sido principalmente un asunto de la tribu o de la nación; no había sido habitualmente un tema que concerniera al individuo. Los dioses eran tribales o nacionales, pero no personales. Estos sistemas religiosos proporcionaron poca satisfacción a las aspiraciones espirituales individuales de la gente común.

\par 
%\textsuperscript{(1336.6)}
\textsuperscript{121:5.2} En los tiempos de Jesús, las religiones de Occidente comprendían:

\par 
%\textsuperscript{(1336.7)}
\textsuperscript{121:5.3} 1. \textit{Los cultos paganos}. Eran una combinación de mitología, patriotismo y tradición helénica y latina.

\par 
%\textsuperscript{(1336.8)}
\textsuperscript{121:5.4} 2. \textit{La adoración del emperador}. Esta deificación del hombre como símbolo del Estado indignaba profundamente a los judíos y a los primeros cristianos, y condujo directamente a las amargas persecuciones de las dos iglesias por parte del gobierno romano.

\par 
%\textsuperscript{(1337.1)}
\textsuperscript{121:5.5} 3. \textit{La astrología}. Esta seudociencia de Babilonia se transformó en una religión en todo el imperio greco-romano. Incluso en el siglo veinte, los hombres no se han liberado por completo de esta creencia supersticiosa.

\par 
%\textsuperscript{(1337.2)}
\textsuperscript{121:5.6} 4. \textit{Las religiones de misterio}. Una oleada de cultos de misterio, de nuevas y extrañas religiones del Levante, se había abatido sobre este mundo espiritualmente hambriento, habían seducido a la gente común y les había prometido la salvación \textit{individual}. Estas religiones se volvieron rápidamente las creencias aceptadas de las clases inferiores del mundo greco-romano. Y contribuyeron mucho a preparar el camino para la rápida difusión de las enseñanzas cristianas, considerablemente superiores, que presentaban un concepto majestuoso de la Deidad, asociado con una teología fascinante para los inteligentes, y una profunda oferta de salvación para todos, incluido el hombre medio de esta época, ignorante, pero espiritualmente hambriento.

\par 
%\textsuperscript{(1337.3)}
\textsuperscript{121:5.7} Las religiones de misterio marcaron el final de las creencias nacionales y condujeron al nacimiento de numerosos cultos personales. Los misterios eran numerosos pero todos estaban caracterizados por:

\par 
%\textsuperscript{(1337.4)}
\textsuperscript{121:5.8} 1. Una leyenda mítica, un misterio ---de ahí su nombre. Como regla general, el misterio se refería a la historia de la vida, la muerte y el regreso a la vida de algún dios, como lo ilustran las enseñanzas del mitracismo, que durante cierto tiempo fue contemporáneo del culto creciente del cristianismo según Pablo, y le hizo la competencia.

\par 
%\textsuperscript{(1337.5)}
\textsuperscript{121:5.9} 2. Los misterios eran interraciales y no nacionales. Eran personales y fraternales, y dieron origen a fraternidades religiosas y a numerosas sociedades sectarias.

\par 
%\textsuperscript{(1337.6)}
\textsuperscript{121:5.10} 3. Sus servicios religiosos estaban caracterizados por elaboradas ceremonias de iniciación y espectaculares sacramentos de culto. Sus ritos y rituales secretos a veces eran horribles y repugnantes.

\par 
%\textsuperscript{(1337.7)}
\textsuperscript{121:5.11} 4. Cualquiera que fuera la naturaleza de sus ceremonias o el grado de sus excesos, estos misterios prometían invariablemente la \textit{salvación} a sus adeptos, \guillemotleft la liberación del mal\footnote{\textit{Liberación del mal}: Job 5:19; Sal 140:1; Pr 2:12; Mt 6:13; Lc 11:4; Gl 1:4; 2 Ti 4:18.}, la supervivencia después de la muerte\footnote{\textit{Supervivencia a la muerte}: Mt 22:30-33; 27:52-53; Mc 12:24-27; Lc 14:14; 20:35-38; Jn 5:28-29.} y una vida duradera en los reinos de la felicidad\footnote{\textit{Reinos de la felicidad}: Jn 14:2-3; Ap 3:12; 21:2-4.}, más allá de este mundo de tristeza y esclavitud\footnote{\textit{Vida después de la muerte}: Dn 12:2; Mt 19:16,29; 25:46; Mc 10:17,30; Lc 10:25; 18:18,30; Jn 3:15-16,36; 4:14,36; 5:24,39; 6:27,40,47; 6:54,68; 8:51-52; 10:28; 11:25-26; 12:25,50; 17:2-3; Hch 13:46-48; Ro 2:7; 5:21; 6:22-23; Gl 6:8; 1 Ti 1:16; 6:12,19; Tit 1:2; 3:7; 1 Jn 1:2; 2:25; 3:15; 5:11,13,20; Jud 1:21; Ap 22:5.}\guillemotright.

\par 
%\textsuperscript{(1337.8)}
\textsuperscript{121:5.12} Pero no cometáis el error de confundir las enseñanzas de Jesús con los misterios. La popularidad de los misterios revela la búsqueda del hombre por sobrevivir, lo que demuestra un hambre y una sed auténticas de religión personal y de rectitud individual. Aunque los misterios no satisfacieran estas aspiraciones de manera adecuada, prepararon el camino para la aparición posterior de Jesús, que aportó verdaderamente a este mundo el pan y el agua de la vida.

\par 
%\textsuperscript{(1337.9)}
\textsuperscript{121:5.13} En un esfuerzo por aprovechar la aceptación generalizada de los mejores tipos de religiones de misterio, Pablo efectuó ciertas adaptaciones en las enseñanzas de Jesús para hacerlas más aceptables a un mayor número de conversos potenciales. Pero incluso el compromiso de Pablo sobre las enseñanzas de Jesús (el cristianismo) era superior al mejor de los misterios, en el sentido de que:

\par 
%\textsuperscript{(1337.10)}
\textsuperscript{121:5.14} 1. Pablo enseñaba una redención moral, una salvación ética. El cristianismo señalaba hacia una nueva vida y proclamaba un nuevo ideal. Pablo se alejó de los ritos mágicos y de los encantamientos ceremoniales.

\par 
%\textsuperscript{(1337.11)}
\textsuperscript{121:5.15} 2. El cristianismo representaba una religión que trataba las soluciones definitivas del problema humano, porque no sólo ofrecía salvar del dolor e incluso de la muerte, sino que prometía también liberar del pecado y dotarse a continuación de un carácter recto con cualidades de supervivencia eterna.

\par 
%\textsuperscript{(1338.1)}
\textsuperscript{121:5.16} 3. Los misterios estaban basados en mitos. El cristianismo, tal como Pablo lo predicaba, estaba fundamentado en un hecho histórico: la donación de Miguel, el Hijo de Dios, a la humanidad.

\par 
%\textsuperscript{(1338.2)}
\textsuperscript{121:5.17} Entre los gentiles, la moralidad no estaba necesariamente relacionada con la filosofía o la religión. Fuera de Palestina, la gente no siempre tenía la idea de que los sacerdotes de una religión tuvieran que llevar una vida moral. La religión judía, luego las enseñanzas de Jesús, y más tarde el cristianismo evolutivo de Pablo, fueron las primeras religiones europeas que hicieron hincapié tanto en la moral como en la ética, insistiendo en que las personas religiosas prestaran alguna atención a las dos.

\par 
%\textsuperscript{(1338.3)}
\textsuperscript{121:5.18} Jesús nació en Palestina en el seno de esta generación de hombres dominados por estos sistemas filosóficos incompletos, y confundidos por estos cultos religiosos complejos. Y a esta misma generación, ofreció posteriormente su evangelio de religión personal ---la filiación con Dios.

\section*{6. La religión hebrea}
\par 
%\textsuperscript{(1338.4)}
\textsuperscript{121:6.1} Hacia finales del primer siglo antes de Cristo, el pensamiento religioso de Jerusalén había estado enormemente influido, y un tanto modificado, por las enseñanzas culturales griegas, e incluso por la filosofía griega. En la larga disputa entre los puntos de vista de las escuelas oriental y occidental de pensamiento hebreo, Jerusalén y el resto de Occidente, así como el Levante, adoptaron en general el punto de vista de los judíos occidentales, el punto de vista helenista modificado.

\par 
%\textsuperscript{(1338.5)}
\textsuperscript{121:6.2} En los tiempos de Jesús, tres idiomas prevalecían en Palestina: la gente común hablaba un dialecto del arameo, los sacerdotes y los rabinos hablaban el hebreo, las clases instruidas y las capas altas de la población judía hablaban en general el griego. La temprana traducción de las escrituras hebreas al griego, en Alejandría, fue en gran parte responsable del predominio posterior del sector griego de la cultura y de la teología judías. Y los escritos de los educadores cristianos no tardaron en aparecer en el mismo idioma. El renacimiento del judaísmo data de la traducción al griego de las escrituras hebreas. Esta influencia vital fue la que más tarde determinó que el culto cristiano de Pablo derivara hacia el Oeste, en lugar de hacerlo hacia el Este.

\par 
%\textsuperscript{(1338.6)}
\textsuperscript{121:6.3} Aunque las creencias judías helenizadas estaban muy poco influidas por las enseñanzas de los epicúreos, estaban enormemente afectadas por la filosofía de Platón y las doctrinas de la autoabnegación de los estoicos. La gran invasión del estoicismo está ilustrada en el Cuarto Libro de los Macabeos; la penetración tanto de la filosofía platónica como de las doctrinas estoicas se puede observar en la Sabiduría de Salomón. Los judíos helenizados interpretaban las escrituras hebreas de una manera tan alegórica, que no encontraron ninguna dificultad para conformar la teología hebrea con la filosofía de Aristóteles, que ellos veneraban. Pero todo esto condujo a una confusión desastrosa hasta que estos problemas fueron tratados por Filón de Alejandría, que procedió a armonizar y organizar la filosofía griega y la teología hebrea en un sistema compacto y medianamente coherente de creencias y de prácticas religiosas. Esta enseñanza más reciente de filosofía griega y de teología hebrea combinadas es la que prevalecía en Palestina cuando Jesús vivió y enseñó, y la que Pablo utilizó como cimiento para construir su culto cristiano, más avanzado e instructivo que los demás.

\par 
%\textsuperscript{(1338.7)}
\textsuperscript{121:6.4} Filón era un gran maestro; desde Moisés no se había visto a un hombre que ejerciera una influencia tan profunda en el pensamiento ético y religioso del mundo occidental. En la tarea de combinar los mejores elementos de los sistemas contemporáneos de enseñanzas éticas y religiosas, ha habido siete educadores humanos sobresalientes: Sethard, Moisés, Zoroastro, Lao-Tse, Buda, Filón y Pablo.

\par 
%\textsuperscript{(1339.1)}
\textsuperscript{121:6.5} Filón había incurrido en contradicciones en sus esfuerzos por combinar la filosofía mística griega y las doctrinas estoicas de los romanos con la teología legalista de los hebreos. Pablo reconoció muchas de estas contradicciones, aunque no todas, y las eliminó sabiamente de su teología básica precristiana. Filón abrió el camino para que Pablo pudiera restablecer más plenamente el concepto de la Trinidad del Paraíso, que había estado mucho tiempo latente en la teología judía. En una sola cuestión, Pablo no logró mantenerse a la altura de Filón, ni consiguió sobrepasar las enseñanzas de este judío rico e instruido de Alejandría; se trataba de la doctrina de la expiación. Filón enseñaba que había que liberarse de la doctrina de obtener el perdón exclusivamente por el derramamiento de sangre. Es posible también que vislumbrara la realidad y la presencia de los Ajustadores del Pensamiento más claramente que Pablo. Pero la teoría de Pablo sobre el pecado original ---las doctrinas de la culpabilidad hereditaria, del mal innato y de su redención--- era parcialmente de origen mitríaco y tenía pocos puntos en común con la teología hebrea, con la filosofía de Filón, o con las enseñanzas de Jesús. Algunos aspectos de las enseñanzas de Pablo sobre el pecado original y la expiación eran creación suya.

\par 
%\textsuperscript{(1339.2)}
\textsuperscript{121:6.6} El evangelio de Juan, el último de los relatos sobre la vida terrestre de Jesús, se dirigía a los pueblos occidentales y presenta su historia basándose ampliamente en el punto de vista de los cristianos de Alejandría de un período posterior, que también eran discípulos de las enseñanzas de Filón.

\par 
%\textsuperscript{(1339.3)}
\textsuperscript{121:6.7} Aproximadamente en la época de Cristo, un extraño cambio de actitud hacia los judíos se produjo en Alejandría, y desde este antiguo bastión judío partió una virulenta ola de persecuciones que llegó incluso hasta Roma, de donde miles de ellos fueron desterrados. Pero esta campaña de distorsión fue de corta duración; muy pronto el gobierno imperial restableció íntegramente, en todo el imperio, las libertades que se habían restringido a los judíos.

\par 
%\textsuperscript{(1339.4)}
\textsuperscript{121:6.8} A través del vasto mundo, en cualquier parte donde los judíos se hallaran dispersos a causa del comercio o de la opresión, todos estaban de acuerdo en mantener sus corazones centrados en el templo sagrado de Jerusalén. La teología judía que sobrevivió era la que se interpretaba y se practicaba en Jerusalén, a pesar del hecho de que varias veces fue salvada del olvido gracias a la oportuna intervención de ciertos educadores de Babilonia.

\par 
%\textsuperscript{(1339.5)}
\textsuperscript{121:6.9} Hasta dos millones y medio de estos judíos dispersos tenían la costumbre de venir a Jerusalén para celebrar sus fiestas religiosas nacionales. Y cualesquiera que fueran las diferencias teológicas o filosóficas entre los judíos del Este (babilonios) y los del Oeste (helénicos), todos estaban de acuerdo en considerar a Jerusalén como el centro de su culto, y en continuar esperando la llegada del Mesías.

\section*{7. Los judíos y los gentiles}
\par 
%\textsuperscript{(1339.6)}
\textsuperscript{121:7.1} En los tiempos de Jesús, los judíos habían llegado a un concepto estable de su origen, de su historia y de su destino. Habían construido un rígido muro de separación entre ellos y el mundo de los gentiles; todas las costumbres de los gentiles las miraban con un desprecio total. Veneraban la letra de la ley y se complacían en una forma de presunción basada en el falso orgullo del linaje. Se habían formado conceptos preconcebidos del Mesías prometido, y la mayoría de estas expectativas vislumbraban a un Mesías que vendría como parte de su historia racial y nacional. Para los hebreos de aquellos tiempos, la teología judía estaba irrevocablemente establecida, fijada para siempre.

\par 
%\textsuperscript{(1339.7)}
\textsuperscript{121:7.2} Las enseñanzas y las prácticas de Jesús relacionadas con la tolerancia y la benevolencia, iban en contra de la actitud inmemorial de los judíos hacia los otros pueblos, a quienes consideraban paganos. Durante generaciones, los judíos habían cultivado una actitud hacia el mundo exterior que les hacía imposible aceptar las enseñanzas del Maestro sobre la fraternidad espiritual de los hombres. Eran reacios a compartir a Yahvé en términos de igualdad con los gentiles, e igualmente reacios a aceptar como Hijo de Dios a alguien que enseñaba unas doctrinas tan nuevas y extrañas.

\par 
%\textsuperscript{(1340.1)}
\textsuperscript{121:7.3} Los escribas, los fariseos y los sacerdotes mantenían a los judíos en una terrible esclavitud de ritualismo y legalismo, una esclavitud mucho más real que la de la autoridad política romana. Los judíos de la época de Jesús no sólo estaban subyugados a la \textit{ley}, sino que también estaban atados a las exigencias esclavizantes de las \textit{tradiciones}, que envolvían e invadían todos los terrenos de la vida personal y social. Estas minuciosas reglas de conducta perseguían y dominaban a todos los judíos leales, y no es extraño que rechazaran rápidamente a uno de los suyos que se atrevía a ignorar sus sagradas tradiciones, y osaba burlarse de sus reglas de conducta social tanto tiempo veneradas. Difícilmente podían considerar de manera favorable las enseñanzas de alguien que no vacilaba en contradecir los dogmas que ellos estimaban que habían sido establecidos por el mismo Padre Abraham. Moisés les había dado la ley, y no estaban dispuestos a hacer compromisos.

\par 
%\textsuperscript{(1340.2)}
\textsuperscript{121:7.4} Durante el primer siglo después de Cristo, la interpretación oral de la ley por los educadores reconocidos, los escribas, tenía más autoridad que la misma ley escrita. Todo esto facilitó las cosas a ciertos jefes religiosos de los judíos, para predisponer al pueblo contra la aceptación de un nuevo evangelio.

\par 
%\textsuperscript{(1340.3)}
\textsuperscript{121:7.5} Estas circunstancias hicieron imposible que los judíos cumplieran su destino divino como mensajeros del nuevo evangelio de independencia religiosa y de libertad espiritual. No fueron capaces de romper las cadenas de la tradición. Jeremías había anunciado la <<ley que deberá escribirse en el corazón de los hombres>>\footnote{\textit{Ley escrita en el corazón de los hombres}: Jer 31:33; 32:40.}. Ezequiel había hablado de un <<nuevo espíritu que morará en el alma del hombre>>\footnote{\textit{Nuevo espiritu que vive en el corazón de los hombres}: Ez 11:19; 18:31; 36:26-27; Jl 2:28-29; Lc 24:49; Jn 7:39; 14:16-18,23,26; 15:4,26; 16:7-8,13-14; 17:21-23; Hch 1:5,8a; 2:1-4,16-18; 2:33; 2 Co 13:5; Gl 2:20; 4:6; Ef 1:13; 4:30; 1 Jn 4:12-15.}, y el salmista había rogado para que Dios <<creara por dentro un corazón limpio y renovara un espíritu recto>>\footnote{\textit{Crear un corazón limpio y un nuevo espíritu}: Sal 51:10.}. Pero cuando la religión judía de las buenas obras y de la esclavitud a la ley cayó víctima del estancamiento de la inercia tradicionalista, el movimiento de la evolución religiosa se desplazó hacia el oeste, hacia los pueblos europeos.

\par 
%\textsuperscript{(1340.4)}
\textsuperscript{121:7.6} Así es como un pueblo diferente fue llamado para aportar al mundo una teología en progreso, un sistema de enseñanza que comprendía la filosofía de los griegos, la ley de los romanos, la moralidad de los hebreos y el evangelio de la naturaleza sagrada de la personalidad y de la libertad espiritual, formulado por Pablo y basado en las enseñanzas de Jesús.

\par 
%\textsuperscript{(1340.5)}
\textsuperscript{121:7.7} El culto cristiano de Pablo mostraba su moralidad como una marca de nacimiento judía. Los judíos consideraban que la historia era la providencia de Dios ---Yahvé trabajando. Los griegos aportaron a las nuevas enseñanzas unos conceptos más claros de la vida eterna. Las doctrinas de Pablo fueron influidas en su contenido teológico y filosófico, no sólo por las enseñanzas de Jesús, sino también por Platón y Filón. En la ética, estaba inspirado no solamente en Cristo, sino también en los estoicos.

\par 
%\textsuperscript{(1340.6)}
\textsuperscript{121:7.8} El evangelio de Jesús, tal como fue incorporado en el culto paulino del cristianismo de Antioquía, se mezcló con las enseñanzas siguientes:

\par 
%\textsuperscript{(1340.7)}
\textsuperscript{121:7.9} 1. El razonamiento filosófico de los prosélitos griegos del judaísmo, incluyendo algunos de sus conceptos sobre la vida eterna.

\par 
%\textsuperscript{(1340.8)}
\textsuperscript{121:7.10} 2. Las atractivas enseñanzas de los cultos de misterio predominantes, en particular las doctrinas mitríacas de la redención, la expiación y la salvación gracias al sacrificio realizado por algún dios.

\par 
%\textsuperscript{(1340.9)}
\textsuperscript{121:7.11} 3. La sólida moralidad de la religión judía establecida.

\par 
%\textsuperscript{(1341.1)}
\textsuperscript{121:7.12} En la época de Jesús, el imperio romano del Mediterráneo, el reino de los partos y los pueblos vecinos, todos tenían ideas imperfectas y primitivas sobre la geografía del mundo, la astronomía, la salud y la enfermedad; y naturalmente se quedaron asombrados con las declaraciones nuevas y sorprendentes del carpintero de Nazaret. Las ideas de estar poseído por un espíritu, bueno o malo, no solamente se aplicaban a los seres humanos, sino que mucha gente consideraba que las rocas y los árboles también estaban poseídos por los espíritus. Era una época de encantamientos, y todo el mundo creía en los milagros como si fueran incidentes ordinarios.

\section*{8. Los escritos anteriores}
\par 
%\textsuperscript{(1341.2)}
\textsuperscript{121:8.1} Siempre que ha sido posible y compatible con nuestra misión, hemos intentado utilizar y hasta cierto punto coordinar las narraciones existentes relacionados con la vida de Jesús en Urantia. Aunque hemos tenido la suerte de acceder a los escritos perdidos del apóstol Andrés, y nos hemos beneficiado de la colaboración de una multitud de seres celestiales que se encontraban en la Tierra en los tiempos de la donación de Miguel (en particular su Ajustador ahora personalizado), también hemos querido utilizar los evangelios llamados de Mateo, de Marcos, de Lucas y de Juan.

\par 
%\textsuperscript{(1341.3)}
\textsuperscript{121:8.2} Estos escritos del Nuevo Testamento tuvieron su origen en las circunstancias siguientes:

\par 
%\textsuperscript{(1341.4)}
\textsuperscript{121:8.3} 1. \textit{El evangelio según Marcos}. Juan Marcos escribió la primera (a excepción de las notas de Andrés), más breve y más simple historia de la vida de Jesús. Presentó al Maestro como un ministro, como un hombre entre los hombres. Aunque Marcos era un muchacho que presenció muchos de los hechos que describe, su relato es en realidad el evangelio según Simón Pedro. Marcos estuvo asociado primero con Pedro, y más tarde con Pablo. Escribió esta historia a instancias de Pedro y ante la demanda ferviente de la iglesia de Roma. Sabiendo con qué persistencia el Maestro se había negado a escribir sus enseñanzas mientras estuvo como mortal en la Tierra, Marcos, como los apóstoles y otros discípulos importantes, no se decidía a ponerlos por escrito. Pero Pedro tenía el sentimiento de que la iglesia de Roma necesitaba la ayuda de esta narración escrita, y Marcos accedió a emprender su preparación. Tomó muchas notas antes de que Pedro muriera en el año 67. De acuerdo con el esquema aprobado por Pedro, empezó la narración para la iglesia de Roma poco después de la muerte de Pedro. El evangelio fue terminado hacia finales del año 68. Marcos lo escribió íntegramente basándose en su propia memoria y en la de Pedro. Este documento ha sido modificado considerablemente desde entonces; muchos pasajes han sido eliminados y se han efectuado adiciones posteriores para reemplazar la última quinta parte del evangelio original, que se perdió del primer manuscrito antes de que fuera copiada. El documento de Marcos, junto con las notas de Andrés y de Mateo, fue la base escrita para todos los relatos evangélicos posteriores que trataron de describir la vida y las enseñanzas de Jesús.

\par 
%\textsuperscript{(1341.5)}
\textsuperscript{121:8.4} 2. \textit{El evangelio según Mateo}. El llamado evangelio según Mateo es el relato de la vida del Maestro, escrito para la edificación de los cristianos judíos. El autor de este documento trata de mostrar constantemente en la vida de Jesús que muchas de las cosas que hizo fueron <<para que se cumplieran las palabras del profeta>>\footnote{\textit{Profecías} \textit{1) Una virgen concebirá a un niño llamado Emanuel} Is 7:14: Mt 1:22-23. \textit{2) De Belén saldrá el Mesías} Mi 5:2: Mt 2:5-6. \textit{3) De Egipto llamé a mi hijo} Ho 11:1: Mt 2:15. \textit{4) Se oye un grito en Ramá} Je 31:15: Mt 2:17-18. \textit{5) Lo llamarán nazareno} Jg 13:5: Mt 2:23. \textit{6) El pueblo de Zebulón y Neftalí ha visto una gran luz} Is 9:1-2: Mt 4:14-16. \textit{7) Él cargó con nuestras enfermedades} Is 53:4: Mt 8:17. \textit{8) Este es mi siervo, a quien he escogido} Is 42:1-4: Mt 12:17-21. \textit{9) Por mucho que oigan no entenderán} Is 6:9-10: Mt 13:14. \textit{10) Hablaré por medio de parábolas} Ps 78:2: Mt 13:35. \textit{11) Rey montado en un burro} Zc 9:9: Mt 21:4-5. \textit{12) Compra del campo del alfarero con treinta monedas} Zc 11:12: Mt 27:9-10. \textit{13) Se repartieron a suertes su ropa} Ps 22:18: Mt 27:35.}. El evangelio de Mateo presenta a Jesús como un hijo de David\footnote{\textit{Linaje de David}: Mt 1:1,6-17; 9:27; 12:23; 15:22; 20:30-31; 21:9,15; 22:42.}, y lo describe como mostrando un gran respeto por la ley y los profetas.

\par 
%\textsuperscript{(1341.6)}
\textsuperscript{121:8.5} El apóstol Mateo no escribió este evangelio. Fue escrito por Isador, uno de sus discípulos, que para facilitar su trabajo disponía no solamente de los recuerdos personales de Mateo sobre aquellos acontecimientos, sino también de ciertas notas sobre las aserciones de Jesús, que Mateo había redactado inmediatamente después de la crucifixión. Las notas de Mateo estaban escritas en arameo; Isador escribió en griego. No había intención de engaño al atribuir el trabajo a Mateo. En aquellos tiempos, los discípulos tenían la costumbre de honrar así a sus maestros.

\par 
%\textsuperscript{(1342.1)}
\textsuperscript{121:8.6} El escrito original de Mateo fue editado y ampliado en el año
40, poco antes de que Mateo dejara Jerusalén para emprender la predicación del evangelio. Se trataba de un documento privado, y la última copia fue destruida en el incendio de un monasterio sirio en el año 416.

\par 
%\textsuperscript{(1342.2)}
\textsuperscript{121:8.7} Isador huyó de Jerusalén en el año 70, después del bloqueo de la ciudad por los ejércitos de Tito, y se llevó a Pella una copia de las notas de Mateo. En el año 71, mientras vivía en Pella, Isador escribió el evangelio según Mateo. También poseía las cuatro primeras quintas partes del relato de Marcos.

\par 
%\textsuperscript{(1342.3)}
\textsuperscript{121:8.8} 3. \textit{El evangelio según Lucas}. Lucas, el médico de Antioquía en Pisidia, era un gentil convertido por Pablo, y escribió una historia muy distinta de la vida del Maestro. En el año 47 empezó a seguir a Pablo y a instruirse sobre la vida y las enseñanzas de Jesús. Lucas conserva en su relato mucho de la <<gracia del Señor Jesucristo>>\footnote{\textit{La gracia del Señor Jesucristo}: Lc 2:40; Hch 11:23; 13:43; 14:3,26; 15:11,40; 18:27; 20:24,32; 1 Co 13:14.}, ya que recogió estos hechos de Pablo y de otras personas. Lucas presenta al Maestro como el <<amigo de los publicanos y de los pecadores>>\footnote{\textit{Amigo de publicanos y pecadores}: Lc 5:29-30; 7:29,34; 15:1-2; 19:2-6.}. Sólo después de la muerte de Pablo reunió sus numerosas notas en forma de evangelio. Lucas escribió en el año 82 en Acaya. Tenía en proyecto tres libros sobre la historia de Cristo y del cristianismo, pero murió en el año 90, cuando estaba a punto de terminar la segunda de estas obras, los <<Hechos de los Apóstoles>>.

\par 
%\textsuperscript{(1342.4)}
\textsuperscript{121:8.9} Como material para compilar su evangelio, Lucas se basó principalmente en la historia de la vida de Jesús que Pablo le había contado. Por lo tanto, el evangelio de Lucas es, en algunos aspectos, el evangelio según Pablo. Pero Lucas tenía otras fuentes de información. No solamente entrevistó a decenas de testigos oculares de los numerosos episodios de la vida de Jesús que relata, sino que poseía también una copia del evangelio de Marcos (es decir las cuatro primeras quintas partes del libro), la narración de Isador y un breve texto escrito en el año 78 en Antioquía por un creyente llamado Cedes. Lucas poseía también una copia mutilada y muy modificada de unas notas que se atribuían al apóstol Andrés.

\par 
%\textsuperscript{(1342.5)}
\textsuperscript{121:8.10} 4. \textit{El evangelio según Juan}. El evangelio según Juan relata una gran parte de la obra que Jesús realizó en Judea y alrededor de Jerusalén, que no se menciona en los otros relatos. Éste es el llamado evangelio según Juan el hijo de Zebedeo, y aunque Juan no lo escribió, sí lo inspiró. Desde que se escribió por primera vez, ha sido corregido muchas veces para dar la impresión de que fue escrito por el mismo Juan. En el momento de componer esa narración, Juan tenía los otros evangelios y observó que muchas cosas se habían omitido; por este motivo, en el año 101 animó a su asociado Natán, un judío griego de Cesarea, para que emprendiera su redacción. Juan proporcionó el material de memoria y basándose en los tres escritos ya existentes. Él mismo no tenía nada escrito sobre el tema. La epístola que se conoce como <<La primera de Juan>>, fue escrita por el mismo Juan como carta de presentación del trabajo que Natán había realizado bajo su dirección.

\par 
%\textsuperscript{(1342.6)}
\textsuperscript{121:8.11} Todos estos autores presentaron honestas descripciones de Jesús tal como ellos lo habían visto, lo recordaban o se habían informado sobre él, y en la medida en que sus conceptos de aquellos acontecimientos lejanos fueron influidos por su adhesión posterior a la teología cristiana de Pablo. Por muy imperfectos que sean estos documentos, han sido suficientes para cambiar el curso de la historia de Urantia durante cerca de dos mil años.

\par 
%\textsuperscript{(1343.1)}
\textsuperscript{121:8.12} [\textit{Agradecimientos:} Para llevar a cabo mi misión de reexponer las enseñanzas de Jesús de Nazaret y contar de nuevo sus acciones, he utilizado ampliamente todas las fuentes de archivos y de informaciones planetarias. Mi motivo principal ha sido preparar un documento que no solamente ilumine a la generación de hombres que viven en la actualidad, sino que sea igualmente útil para todas las generaciones futuras. En la enorme reserva de información puesta a mi disposición, he seleccionado aquellas que convenían mejor para llevar a cabo este objetivo. En la medida de lo posible, he obtenido mis informaciones de fuentes puramente humanas. Únicamente cuando estas fuentes han resultado insuficientes, he recurrido a los archivos superhumanos. Cuando las ideas y los conceptos de la vida y de las enseñanzas de Jesús han sido expresados aceptablemente por una mente humana, he dado preferencia invariablemente a estos modelos de pensamiento aparentemente humanos. Aunque me he esforzado en adaptar la expresión verbal para adecuarla lo mejor posible a la manera en que nosotros concebimos el sentido real y la verdadera importancia de la vida y de las enseñanzas del Maestro, en todas mis exposiciones me he ajustado, tanto como ha sido posible, a los verdaderos conceptos y modelos de pensamiento de los hombres. Sé muy bien que estos conceptos que se han originado en la mente humana resultarán más aceptables y útiles para la mente de todos los demás hombres. Cuando he sido incapaz de encontrar los conceptos necesarios en los escritos o en las expresiones humanas, he recurrido en segundo lugar a la memoria de mi propia orden de criaturas terrestres, los intermedios. Finalmente, cuando esta fuente secundaria de información ha sido insuficiente, he recurrido sin dudarlo a las fuentes de información superplanetarias.

\par 
%\textsuperscript{(1343.2)}
\textsuperscript{121:8.13} Los memorando que he reunido, a partir de los cuales he preparado este relato de la vida y de las enseñanzas de Jesús ---además de las memorias que el apóstol Andrés había registrado--- contienen joyas del pensamiento y conceptos muy elevados de las enseñanzas de Jesús, procedentes de más de dos mil seres humanos que han vivido en la Tierra desde la época de Jesús hasta el día en que fueron redactadas las presentes revelaciones, o más exactamente estas reexposiciones. El permiso de revelar solamente ha sido utilizado cuando el escrito humano o los conceptos humanos no conseguían proporcionar un modelo de pensamiento adecuado. Mi misión de revelación me prohibía recurrir a fuentes extrahumanas de información o de expresión, hasta que pudiera atestiguar que había agotado todas las posibilidades para encontrar la expresión conceptual necesaria en las fuentes puramente humanas.

\par 
%\textsuperscript{(1343.3)}
\textsuperscript{121:8.14} Aunque he descrito, con la colaboración de mis once compañeros intermedios y bajo la supervisión del Melquisedek ya mencionado, los acontecimientos de este relato según mi concepto sobre el orden en que se produjeron y en respuesta a mi elección de los términos adecuados para describirlos, sin embargo, la mayoría de las ideas e incluso algunas de las expresiones efectivas que he utilizado así tuvieron su origen en la mente de los hombres de numerosas razas que han vivido en la Tierra durante las generaciones intermedias, incluídos aquellos que viven todavía en el momento de efectuar esta tarea. En muchos aspectos, he actuado más como recopilador y adaptador que como narrador original. Me he apropiado sin titubeos de las ideas y de los conceptos, preferentemente humanos, que me permitían crear la descripción más eficaz de la vida de Jesús, y que me cualificaran para reexponer sus enseñanzas incomparables con la fraseología más notablemente provechosa y universalmente enriquecedora. En nombre de la Fraternidad de los Intermedios Unidos de Urantia, reconozco con la mayor gratitud nuestra deuda hacia todas las fuentes de información y de conceptos que se han utilizado para elaborar nuestra nueva exposición de la vida de Jesús en la Tierra].


\chapter{Documento 122. El nacimiento y la infancia de Jesús}
\par 
%\textsuperscript{(1344.1)}
\textsuperscript{122:0.1} SERÍA casi imposible explicar plenamente las numerosas razones que llevaron a elegir Palestina como país para la donación de Miguel, y en especial por qué exactamente se escogió a la familia de José y María como marco inmediato para la aparición de este Hijo de Dios en Urantia.

\par 
%\textsuperscript{(1344.2)}
\textsuperscript{122:0.2} Después de estudiar un informe especial sobre el estado de los mundos aislados, preparado por los Melquisedeks con el asesoramiento de Gabriel, Miguel escogió finalmente Urantia como planeta para efectuar su última donación. Después de esta decisión, Gabriel visitó personalmente Urantia, y como resultado de su estudio de los grupos humanos y de su examen de las características espirituales, intelectuales, raciales y geográficas del mundo y de sus pueblos, decidió que los hebreos poseían aquellas ventajas relativas que justificaban su elección como raza para la donación. Cuando Miguel aprobó esta decisión, Gabriel nombró y envió a Urantia la Comisión Familiar de los Doce ---escogida entre las órdenes más elevadas de personalidades del universo--- con el encargo específico de investigar la vida familiar judía. Cuando esta comisión finalizó su tarea, Gabriel se encontraba en Urantia y recibió el informe que designaba a tres posibles parejas que, en opinión de la comisión, eran igualmente favorables como familias de donación para la encarnación que Miguel tenía en proyecto.

\par 
%\textsuperscript{(1344.3)}
\textsuperscript{122:0.3} De las tres parejas designadas, Gabriel escogió personalmente a José y María; posteriormente se apareció en persona a María y le dio la grata noticia de que había sido elegida para ser la madre terrestre del niño de la donación.

\section*{1. José y María}
\par 
%\textsuperscript{(1344.4)}
\textsuperscript{122:1.1} José, el padre humano de Jesús (Josué ben José) era un hebreo entre los hebreos, aunque poseía muchos rasgos raciales no judíos que, de vez en cuando, se habían añadido a su árbol genealógico a través de las líneas femeninas de sus progenitores\footnote{\textit{Linaje de José}: Mt 1:1-16; Lc 3:23-38.}. Los antepasados del padre de Jesús se remontaban a los tiempos de Abraham, y por medio de este venerable patriarca, a linajes más antiguos que llegaban hasta los sumerios y los noditas y, a través de las tribus meridionales del antiguo hombre azul, hasta Andón y Fonta. David y Salomón no eran antecesores en línea directa de José, cuyo linaje tampoco se remontaba directamente hasta Adán. Los ascendientes próximos de José eran artesanos: constructores, carpinteros, albañiles y herreros. El mismo José era carpintero, y más tarde fue contratista. Su familia pertenecía a una larga e ilustre línea de notables del pueblo, realzada de vez en cuando por la aparición de personalidades excepcionales que se habían distinguido en el ámbito de la evolución de la religión en Urantia.

\par 
%\textsuperscript{(1345.1)}
\textsuperscript{122:1.2} María, la madre terrestre de Jesús, descendía de una larga estirpe de antepasados extraordinarios que comprendía muchas mujeres entre las más notables de la historia racial de Urantia. Aunque María era una mujer típica de su tiempo y de su generación, con un temperamento bastante normal, contaba entre sus antecesores a mujeres tan ilustres como Annon, Tamar, Rut, Betsabé, Ansie, Cloa, Eva, Enta y Ratta. Ninguna mujer judía de la época poseía un linaje que tuviera en común a unos progenitores más ilustres, o que se remontara a unos orígenes más prometedores. Los antepasados de María, como los de José, estaban caracterizados por el predominio de individuos fuertes pero corrientes, resaltando de vez en cuando numerosas personalidades sobresalientes en la marcha de la civilización y en la evolución progresiva de la religión. Desde un punto de vista racial, no es muy apropiado considerar a María como una judía. Por su cultura y sus creencias era judía, pero por sus dones hereditarios era más bien una combinación de estirpes siria, hitita, fenicia, griega y egipcia; su herencia racial era más heterogénea que la de José.

\par 
%\textsuperscript{(1345.2)}
\textsuperscript{122:1.3} De todas las parejas que vivían en Palestina en la época para la que se había proyectado la donación de Miguel, José y María\footnote{\textit{Los padres de Jesús}: Mt 1:16-21,24-25; 2:11; 13:55; 27:56; Mc 6:3; 15:40; Lc 1:26-56; 2:4-5,16,19; 2:33-34,43; 3:23; Jn 1:45; 6:42; 19:25-27; Hch 1:14.} poseían la combinación más ideal de vastos vínculos raciales y de dotaciones de personalidad superiores a la media. El plan de Miguel era aparecer en la Tierra como un hombre \textit{ordinario}, para que la gente común pudiera comprenderlo y recibirlo; por eso Gabriel eligió a unas personas como José y María para ser los padres de la donación.

\section*{2. Gabriel se aparece a Isabel}
\par 
%\textsuperscript{(1345.3)}
\textsuperscript{122:2.1} El trabajo que Jesús realizó durante su vida en Urantia fue empezado, de hecho, por Juan Bautista. Zacarías, el padre de Juan, pertenecía al clero judío, mientras que su madre, Isabel, era miembro de la rama más próspera del mismo gran grupo familiar al que también pertenecía María, la madre de Jesús. Zacarías e Isabel, aunque estaban casados desde hacía muchos años, no tenían hijos\footnote{\textit{Los padres de Juan}: Lc 1:5-7.}.

\par 
%\textsuperscript{(1345.4)}
\textsuperscript{122:2.2} A finales del mes de junio del año 8 a. de J.C., unos tres meses después de que se casaran José y María, Gabriel se apareció a Isabel\footnote{\textit{Aparición de Gabriel}: Lc 1:11-12.}, un día al mediodía, de la misma forma que más tarde hizo conocer su presencia a María. Gabriel dijo:

\par 
%\textsuperscript{(1345.5)}
\textsuperscript{122:2.3} <<Mientras tu marido Zacarías oficia ante el altar en Jerusalén, y mientras el pueblo reunido ruega por la llegada de un libertador, yo, Gabriel, he venido para anunciarte que pronto darás a luz un hijo que será el precursor de este maestro divino; llamarás a tu hijo Juan. Crecerá consagrado al Señor tu Dios, y cuando llegue a la madurez, alegrará tu corazón porque llevará muchas almas hacia Dios, y proclamará también la venida del sanador de almas de tu pueblo y libertador espiritual de toda la humanidad. Tu pariente María será la madre de este hijo de la promesa, y también me apareceré a ella>>\footnote{\textit{Mensaje de Gabriel a Zacarías}: Lc 1:13-17. \textit{Ministerio de Zacarías}: Lc 1:8-10.}.

\par 
%\textsuperscript{(1345.6)}
\textsuperscript{122:2.4} Esta visión asustó mucho a Isabel. Después de la partida de Gabriel, le dio muchas vueltas a esta experiencia en su cabeza, reflexionando largamente las palabras del majestuoso visitante, pero no habló de esta revelación a nadie salvo a su marido, hasta que conversó posteriormente con María a principios de febrero del año siguiente.

\par 
%\textsuperscript{(1345.7)}
\textsuperscript{122:2.5} Sin embargo, Isabel guardó durante cinco meses su secreto incluso a su marido\footnote{\textit{El secreto de Isabel}: Lc 1:24-25.}. Cuando le contó la historia de la visita de Gabriel, Zacarías permaneció muy escéptico y dudó de toda la experiencia durante semanas\footnote{\textit{Dudas de Zacarías}: Lc 1:18.}, consintiendo solamente en creer a medias en la visita de Gabriel a su esposa, hasta que ya no pudo dudar de que estaba esperando un hijo. Zacarías estaba extraordinariamente perplejo ante la próxima maternidad de Isabel, pero no puso en duda la integridad de su mujer, a pesar de su propia edad avanzada. No fue hasta unas seis semanas antes del nacimiento de Juan cuando Zacarías, a consecuencia de un sueño impresionante, se convenció por completo de que Isabel iba a ser la madre de un hijo del destino, el encargado de preparar el camino para la venida del Mesías.

\par 
%\textsuperscript{(1346.1)}
\textsuperscript{122:2.6} Gabriel se apareció a María\footnote{\textit{Aparición de Gabriel ante María}: Lc 1:26-27.} hacia mediados de noviembre del año 8 a. de J.C., mientras ella estaba trabajando en su casa de Nazaret. Más adelante, cuando María supo sin lugar a dudas que iba a ser madre, persuadió a José para que la dejara ir a la Ciudad de Judá, a siete kilómetros en las colinas al oeste de Jerusalén, para visitar a Isabel\footnote{\textit{María visita a Isabel}: Lc 1:39-40.}. Gabriel había informado a cada una de estas futuras madres de su aparición a la otra. Naturalmente estaban impacientes por encontrarse, comparar sus experiencias y hablar del futuro probable de sus hijos. María permaneció tres semanas con su prima lejana\footnote{\textit{Longitud de la visita}: Lc 1:56.}. Isabel contribuyó mucho a fortalecer la fe de María en la visión de Gabriel, de manera que ésta regresó a su hogar más plenamente dedicada a la misión de ser la madre del hijo del destino, a quien muy pronto debería presentar al mundo como un bebé indefenso, como un niño normal y común del planeta.

\par 
%\textsuperscript{(1346.2)}
\textsuperscript{122:2.7} Juan nació en la Ciudad de Judá\footnote{\textit{Nacimiento e infancia de Juan}: Lc 1:57-63.}, el 25 de marzo del año
7 a. de J.C. Zacarías e Isabel sintieron una gran alegría con la llegada de su hijo, como Gabriel había prometido. Al octavo día, cuando presentaron al niño para la circuncisión, lo llamaron oficialmente Juan como se les había ordenado anteriormente. Un sobrino de Zacarías ya había partido para Nazaret llevando el mensaje de Isabel a María de que su hijo había nacido y que se llamaría Juan.

\par 
%\textsuperscript{(1346.3)}
\textsuperscript{122:2.8} Desde la más tierna infancia de Juan, sus padres le inculcaron juiciosamente la idea de que cuando creciera se convertiría en un dirigente espiritual y en un instructor religioso. Y el corazón de Juan siempre fue un terreno favorable donde sembrar estas semillas sugerentes. Incluso siendo niño, se le encontraba con frecuencia en el templo durante los períodos de servicio de su padre, y estaba profundamente impresionado con el significado de todo lo que veía\footnote{\textit{Juventud de Juan}: Lc 1:80.}.

\section*{3. La anunciación de Gabriel a María}
\par 
%\textsuperscript{(1346.4)}
\textsuperscript{122:3.1} Cierta tarde al ponerse el Sol, antes de que José hubiera regresado al hogar, Gabriel se apareció a María al lado de una mesa baja de piedra\footnote{\textit{Aparición de Gabriel ante María}: Lc 1:26-27.}; después de que ella recobrara la serenidad, le dijo: \guillemotleft Vengo por orden de aquel que es mi Maestro, a quien tú amarás y alimentarás. A ti, María, te traigo gratas noticias al anunciarte\footnote{\textit{La anunciación}: Lc 1:28-37.} que tu concepción está ordenada por el cielo, y que cuando llegue el momento serás la madre de un hijo; lo llamarás Josué\footnote{\textit{El nombre de Jesús}: Mt 1:21,25; Lc 1:31.}, y él inaugurará el reino de los cielos en la Tierra y entre los hombres. No menciones esto a nadie salvo a José y a Isabel, tu pariente, a quien también me he aparecido, y que pronto dará igualmente a luz un hijo cuyo nombre será Juan. Éste preparará el camino para el mensaje de liberación que tu hijo proclamará con gran fuerza y profunda convicción a los hombres. No dudes de mi palabra, María, pues este hogar ha sido elegido como morada humana del hijo del destino. Mi bendición te acompaña, el poder de los Altísimos te fortalecerá y el Señor de toda la Tierra te protegerá\guillemotright.

\par 
%\textsuperscript{(1346.5)}
\textsuperscript{122:3.2} Durante varias semanas, María reflexionó sobre esta visita de manera secreta en su corazón. Cuando estuvo segura de que esperaba un hijo, se atrevió por fin a revelar a su marido estos acontecimientos inusitados. Cuando José escuchó toda la historia, y aunque confiaba plenamente en María, se quedó muy preocupado y perdió el sueño durante varias noches. Primero José tuvo dudas sobre la visita de Gabriel. Luego, cuando se persuadió casi por completo de que María había oído realmente la voz y había contemplado la forma del mensajero divino, se torturó la mente preguntándose cómo podían suceder tales cosas. ¿Cómo era posible que un descendiente de seres humanos pudiera ser un hijo del destino divino? José no podía conciliar estas ideas contradictorias hasta que, después de varias semanas de reflexión, tanto él como María llegaron a la conclusión de que habían sido elegidos como padres del Mesías, aunque los judíos casi no tenían el concepto de que el liberador esperado tuviera que ser de naturaleza divina. Una vez que llegaron a esta conclusión trascendental, María se apresuró a partir para charlar con Isabel.

\par 
%\textsuperscript{(1347.1)}
\textsuperscript{122:3.3} A su regreso, María fue a visitar a sus padres, Joaquín y Ana. Sus dos hermanos, sus dos hermanas, así como sus padres, fueron siempre muy escépticos respecto a la misión divina de Jesús, aunque por aquel entonces no sabían nada, por supuesto, de la visita de Gabriel. Pero María sí le confió a su hermana Salomé que creía que su hijo estaba destinado a ser un gran maestro.

\par 
%\textsuperscript{(1347.2)}
\textsuperscript{122:3.4} La anunciación de Gabriel a María tuvo lugar al día siguiente de la concepción de Jesús, y fue el único acontecimiento de naturaleza sobrenatural\footnote{\textit{Único acto sobrenatural}: Lc 1:26-38.} que se produjo en toda su experiencia de gestar y dar a luz al hijo de la promesa.

\section*{4. El sueño de José}
\par 
%\textsuperscript{(1347.3)}
\textsuperscript{122:4.1} José no aceptó la idea de que María iba a ser la madre de un hijo extraordinario hasta después de haber experimentado un sueño bastante impresionante\footnote{\textit{El sueño de José}: Mt 1:20-21,24.}. En este sueño, se le apareció un brillante mensajero celestial que le dijo, entre otras cosas: <<José, aparezco ante ti por orden de Aquel que ahora reina en las alturas; he recibido el mandato de informarte acerca del hijo que María va a tener, y que llegará a ser una gran luz en el mundo. En él estará la vida, y su vida se convertirá en la luz de la humanidad. Vendrá primero hacia su propio pueblo, pero ellos casi no lo recibirán; pero a todos los que lo reciban, les revelará que son hijos de Dios>>. Después de esta experiencia, José no volvió a dudar nunca más de la historia de María sobre la visita de Gabriel, ni de la promesa de que el niño por nacer sería un mensajero divino para el mundo.

\par 
%\textsuperscript{(1347.4)}
\textsuperscript{122:4.2} En todas estas visitas no se había dicho nada sobre la casa de David. Nunca se había insinuado nada de que Jesús fuera a convertirse en el <<liberador de los judíos>>, ni tampoco que debiera ser el tan esperado Mesías. Jesús no era el tipo de Mesías que los judíos esperaban, pero sí era el \textit{libertador del mundo}. Su misión era para todas las razas y para todos los pueblos, no para un grupo en particular.

\par 
%\textsuperscript{(1347.5)}
\textsuperscript{122:4.3} José no descendía del linaje del rey David\footnote{\textit{José no descendía de la línea davídica}: Mt 1:1-16; Lc 1:27; 2:4.}. María tenía más antepasados que José en la rama de David. Es verdad que José fue a Belén\footnote{\textit{José a Belén}: Lc 2:4.}, la ciudad de David, para registrarse en el censo romano, pero esto se debió al hecho de que, seis generaciones antes, el antepasado paterno de José de aquella generación, como era huérfano, había sido adoptado por un tal Zadoc\footnote{\textit{Antepasado adoptado}: Mt 1:14-16; Lc 3:23-24.}, que era descendiente directo de David; por eso José también contaba como perteneciente a la <<casa de David>>.

\par 
%\textsuperscript{(1347.6)}
\textsuperscript{122:4.4} La mayoría de las llamadas profecías mesiánicas del Antiguo Testamento fueron redactadas para acomodarlas a Jesús mucho tiempo después de su vida en la Tierra. Durante siglos, los profetas hebreos habían proclamado la venida de un libertador, y estas promesas habían sido interpretadas por las generaciones sucesivas como que se referían a un nuevo gobernante judío que se sentaría en el trono de David, y que mediante los célebres métodos milagrosos de Moisés, establecería a los judíos en Palestina como una nación poderosa, libre de toda dominación extranjera. Además, muchos pasajes metafóricos que se encontraban por todas partes en las escrituras hebreas fueron, con posterioridad, aplicados erróneamente a la misión de la vida de Jesús. Muchos textos del Antiguo Testamento fueron tergiversados para que parecieran cuadrar con algunos episodios de la vida terrestre del Maestro. Jesús mismo negó una vez, públicamente, toda conexión con la casa real de David\footnote{\textit{Jesús no era del linaje de David}: Mt 22:41-46; Mc 12:35-37; Lc 20:41-44.}. Incluso el pasaje <<una joven dará a luz a un hijo>>, se cambió en <<una virgen dará a luz a un hijo>>\footnote{\textit{Tergiversación de la profecía sobre el nacimiento de una virgen} Is 7:14: Mt 1:22-23.}. Lo mismo sucedió con las numerosas genealogías de José y María que se compusieron después de la carrera de Miguel en la Tierra\footnote{\textit{Equivocación de las genealogías}: Mt 1:1-16; Lc 3:23-37.}. Muchos de estos linajes contienen bastantes antepasados del Maestro, pero en general no son auténticos y no se puede confiar en su exactitud. Con demasiada frecuencia, los primeros seguidores de Jesús sucumbieron a la tentación de hacer que todas las antiguas declaraciones proféticas parecieran encontrar su cumplimiento en la vida de su Señor y Maestro.

\section*{5. Los padres terrestres de Jesús}
\par 
%\textsuperscript{(1348.1)}
\textsuperscript{122:5.1} José era un hombre de modales dulces, extremadamente escrupuloso, y fiel en todos los aspectos a las convenciones y prácticas religiosas de su pueblo. Hablaba poco, pero pensaba mucho. La penosa condición del pueblo judío entristecía mucho a José. En su juventud, conviviendo con sus ocho hermanos y hermanas, había sido más alegre, pero durante los primeros años de su vida matrimonial
(durante la infancia de Jesús) sufrió períodos de ligero desaliento espiritual. Estas manifestaciones temperamentales se atenuaron considerablemente poco antes de su muerte prematura y después de que la situación económica de su familia hubiera mejorado gracias a su ascenso desde la categoría de carpintero a la función de próspero contratista.

\par 
%\textsuperscript{(1348.2)}
\textsuperscript{122:5.2} El temperamento de María era totalmente opuesto al de su marido. Habitualmente alegre, rara vez se encontraba abatida, y poseía un carácter siempre risueño. María se permitía expresar libre y frecuentemente sus sentimientos emocionales, y nunca se la vio afligida hasta después de la muerte súbita de José. Apenas se había recuperado de este golpe cuando tuvo que enfrentarse con las ansiedades y las dudas que despertaron en ella la extraordinaria carrera de su hijo mayor, que se desarrollaba tan rápidamente ante sus ojos asombrados. Pero durante toda esta experiencia insólita, María se mantuvo serena, animosa y bastante juiciosa en sus relaciones con su extraño y poco comprensible hijo mayor, y con sus hermanos y hermanas sobrevivientes.

\par 
%\textsuperscript{(1348.3)}
\textsuperscript{122:5.3} Jesús poseía de su padre gran parte de su dulzura excepcional y de su maravillosa comprensión benevolente de la naturaleza humana; había heredado de su madre su don de gran educador y su formidable capacidad de justa indignación. En sus reacciones emocionales hacia su entorno durante su vida adulta, Jesús era en ciertos momentos como su padre, meditativo y piadoso, a veces caracterizado por una tristeza aparente; pero en la mayoría de los casos continuaba hacia adelante a la manera optimista y decidida del carácter de su madre. En conjunto, el temperamento de María tendía a dominar la carrera del Hijo divino a medida que crecía y avanzaba a grandes pasos hacia su vida de adulto. En algunos detalles, Jesús era una mezcla de los rasgos de sus padres; en otros aspectos, los rasgos de uno predominaban sobre los del otro.

\par 
%\textsuperscript{(1348.4)}
\textsuperscript{122:5.4} Jesús poseía de José su estricta educación en los usos de las ceremonias judías y su conocimiento excepcional de las escrituras hebreas; de María obtuvo un punto de vista más amplio de la vida religiosa y un concepto más liberal de la libertad espiritual personal.

\par 
%\textsuperscript{(1349.1)}
\textsuperscript{122:5.5} Las familias de José y de María eran muy instruidas para su tiempo. José y María poseían una educación que estaba muy por encima del promedio de su época y de su posición social. Él era un pensador; ella sabía planificar, era experta en adaptarse y práctica en la ejecución de las tareas inmediatas. José era moreno con los ojos negros; María era casi rubia con los ojos castaños.

\par 
%\textsuperscript{(1349.2)}
\textsuperscript{122:5.6} Si José hubiera vivido, se habría convertido sin duda alguna en un firme creyente en la misión divina de su hijo mayor. María alternaba entre la creencia y la duda, enormemente influida por la postura que tomaron sus otros hijos y sus amigos y parientes, pero su actitud final siempre estuvo fortalecida por el recuerdo de la aparición de Gabriel inmediatamente después de la concepción del niño.

\par 
%\textsuperscript{(1349.3)}
\textsuperscript{122:5.7} María era una tejedora experta, con una habilidad por encima de la media en la mayoría de las artes hogareñas de la época; era una buena ama de casa, con capacidad sobrada para crear un hogar. Tanto José como María eran buenos educadores, y se preocuparon por que sus hijos estuvieran bien instruídos en los conocimientos de su tiempo.

\par 
%\textsuperscript{(1349.4)}
\textsuperscript{122:5.8} Cuando José era joven, fue contratado por el padre de María para construir un anexo a su casa; en el transcurso de una comida al mediodía, María llevó a José un vaso de agua, y fue en ese momento cuando empezó realmente el cortejo de los dos jóvenes que estaban destinados a ser los padres de Jesús.

\par 
%\textsuperscript{(1349.5)}
\textsuperscript{122:5.9} José y María se casaron\footnote{\textit{La boda de los padres de Jesús}: Mt 1:18-21,24-25.}, de acuerdo con la costumbre judía, en la casa de María, en las afueras de Nazaret, cuando José contaba veintiún años de edad. Esta boda fue la culminación de un noviazgo normal de casi dos años. Poco después se trasladaron a su nueva casa de Nazaret\footnote{\textit{Vivieron en Nazaret}: Lc 1:26; 2:4-5.}, que había sido construida por José con la ayuda de dos de sus hermanos. La casa estaba situada al pie de una elevación que dominaba de manera muy agradable la comarca circundante. En esta casa especialmente preparada, los jóvenes esposos en espera de niño pensaban acoger al hijo de la promesa, sin saber que este importante acontecimiento del universo iba a suceder en Belén de Judea, mientras estaban ausentes de su domicilio.

\par 
%\textsuperscript{(1349.6)}
\textsuperscript{122:5.10} La mayor parte de la familia de José se hizo creyente en las enseñanzas de Jesús, pero muy pocos miembros de la familia de María creyeron en él hasta después de su partida de este mundo. José se inclinaba más hacia el concepto espiritual del Mesías esperado, pero María y su familia, y sobre todo su padre, mantenían la idea de un Mesías como liberador temporal y gobernante político. Los antepasados de María se habían identificado de manera destacada con las actividades de los Macabeos, en tiempos por aquel entonces muy recientes.

\par 
%\textsuperscript{(1349.7)}
\textsuperscript{122:5.11} José sostenía vigorosamente el punto de vista oriental, o babilonio, de la religión judía; María tendía fuertemente hacia la interpretación occidental, o helenística, de la ley y de los profetas, que era más amplia y liberal.

\section*{6. El hogar de Nazaret}
\par 
%\textsuperscript{(1349.8)}
\textsuperscript{122:6.1} La casa de Jesús no estaba lejos de la elevada colina situada en la parte norte de Nazaret, a cierta distancia de la fuente del pueblo, que se encontraba en la sección oriental de la población. La familia de Jesús vivía en las afueras de la ciudad, lo que le facilitó posteriormente a Jesús disfrutar de frecuentes paseos por el campo y subir a la cumbre de esta montaña cercana, la más alta de todas las colinas del sur de Galilea, a excepción de la cadena del Monte Tabor al este, y de la colina de Naín, que tenía aproximadamente la misma altura. Su casa estaba situada un poco hacia el sur y el este del promontorio sur de esta colina, y aproximadamente a mitad de camino entre la base de esta elevación y la carretera que conducía de Nazaret a Caná. Además de subir a la colina, el paseo favorito de Jesús era un estrecho sendero que rodeaba la base de la colina en dirección nordeste, hasta el lugar donde se unía con la carretera de Séforis.

\par 
%\textsuperscript{(1350.1)}
\textsuperscript{122:6.2} La casa de José y María era una construcción de piedra compuesta por una habitación con un techo plano, más un edificio adyacente para alojar a los animales. Los muebles consistían en una mesa baja de piedra, platos y ollas de barro y de piedra, un telar, una lámpara, varios taburetes pequeños y alfombras para dormir sobre el piso de piedra. En el patio trasero, cerca del anexo para los animales, había un cobertizo que protegía el horno y el molino para moler el grano. Se necesitaban dos personas para utilizar este tipo de molino, una para moler y otra para echar el grano. Cuando Jesús era pequeño, echaba grano con frecuencia en este molino mientras que su madre hacía girar la muela.

\par 
%\textsuperscript{(1350.2)}
\textsuperscript{122:6.3} Años más tarde, cuando la familia creció, todos se sentaban en cuclillas alrededor de la mesa de piedra agrandada para disfrutar de sus comidas, y se servían el alimento de un plato o de una olla común. En invierno, la mesa estaba iluminada durante la cena por una pequeña lámpara plana de arcilla que llenaban con aceite de oliva. Después del nacimiento de Marta, José construyó un agregado a esta casa, una amplia habitación que se utilizaba como taller de carpintería durante el día y como dormitorio por la noche.

\section*{7. El viaje a Belén}
\par 
%\textsuperscript{(1350.3)}
\textsuperscript{122:7.1} En el mes de marzo del año 8 a. de J.C. (el mes en que José y María se casaron) César Augusto decretó que todos los habitantes del Imperio Romano tenían que ser contados, que había que hacer un censo para mejorar el sistema de los impuestos. Los judíos siempre habían estado enormemente predispuestos contra cualquier intento por <<contar al pueblo>>\footnote{\textit{``Contar al pueblo'' para los tributos}: Lc 2:1,3. \textit{Reacciones contra los censos}: 1 Cr 21:1,5; 2 Sam 24:1-4,10.}; este hecho, sumado a las graves dificultades internas de Herodes, rey de Judea, había contribuido a retrasar un año este empadronamiento en el reino judío. En todo el Imperio Romano, este censo se llevó a cabo en el año 8 a. de J.C., excepto en el reino de Herodes en Palestina, donde tuvo lugar un año más tarde, en el año 7 a. de J.C.

\par 
%\textsuperscript{(1350.4)}
\textsuperscript{122:7.2} No era necesario que María fuera a Belén para empadronarse ---José estaba autorizado para registrar a su familia--- pero María, que era una persona intrépida y decidida, insistió en acompañarle. Temía quedarse sola por si el niño nacía durante la ausencia de José, y puesto que Belén no estaba lejos de la Ciudad de Judá, María preveía la posibilidad de una agradable charla con su pariente Isabel.

\par 
%\textsuperscript{(1350.5)}
\textsuperscript{122:7.3} José prácticamente prohibió a María que lo acompañara, pero no sirvió de nada; en el momento de empaquetar la comida para el viaje de tres o cuatro días, preparó raciones para dos personas y se aprestó para partir. Pero antes de ponerse efectivamente en camino, José ya había consentido en que María lo acompañara, y dejaron alegremente Nazaret al despuntar el día\footnote{\textit{María va a Belén}: Lc 2:4-5.}.

\par 
%\textsuperscript{(1350.6)}
\textsuperscript{122:7.4} José y María eran pobres, y como sólo tenían una bestia de carga, María, que estaba encinta, montó sobre el animal con las provisiones mientras que José caminaba conduciendo a la bestia. Construir y amueblar la casa había sido un gran gasto para José, que también tenía que contribuir al mantenimiento de sus padres, ya que su padre se había quedado incapacitado hacía poco tiempo. Así es como esta pareja judía partió de su humilde hogar, por la mañana temprano, el 18 de agosto del año 7 a. de J.C., en dirección a Belén.

\par 
%\textsuperscript{(1351.1)}
\textsuperscript{122:7.5} Su primer día de viaje les llevó cerca de los cerros al pie del Monte Gilboa, donde acamparon durante la noche junto al río Jordán, e hicieron muchas especulaciones sobre la naturaleza del hijo que iba a nacer; José se adhería al concepto de un maestro espiritual y María sostenía la idea de un Mesías judío, un liberador de la nación hebrea.

\par 
%\textsuperscript{(1351.2)}
\textsuperscript{122:7.6} A primeras horas de la radiante mañana del 19 de agosto, José y María se pusieron de nuevo en camino. Tomaron su comida del mediodía al pie del Monte Sartaba, que domina el valle del Jordán, y continuaron su viaje, llegando por la noche a Jericó, donde se alojaron en una posada del camino, en las afueras de la ciudad. Después de la cena y de mucho discutir sobre la opresión del gobierno romano, Herodes, la inscripción en el censo y la influencia comparativa de Jerusalén y Alejandría como centros del saber y de la cultura judíos, los viajeros de Nazaret se retiraron a dormir. El 20 de agosto por la mañana temprano reanudaron su viaje, llegando a Jerusalén antes del mediodía; visitaron el templo y continuaron hacia su destino, llegando a Belén a media tarde.

\par 
%\textsuperscript{(1351.3)}
\textsuperscript{122:7.7} La posada estaba atestada, y en consecuencia José buscó alojamiento en casa de unos parientes lejanos, pero todas las habitaciones de Belén estaban llenas a rebosar. Al regresar al patio de la posada, le informaron que los establos para las caravanas, labrados en los lados de la roca y situados justo por debajo de la posada, habían sido desalojados de sus animales y limpiados para recibir huéspedes\footnote{\textit{Alojamiento en el establo}: Lc 2:7b.}. Dejando el asno en el patio, José se echó al hombro las bolsas de ropa y de provisiones, y descendió con María los escalones de piedra hasta su alojamiento en la parte inferior. Se instalaron en lo que había sido un almacén de grano, enfrente de los establos y de los pesebres. Habían colgado cortinas de lona, y se consideraron afortunados por haber conseguido un alojamiento tan cómodo.

\par 
%\textsuperscript{(1351.4)}
\textsuperscript{122:7.8} José había pensado ir a inscribirse enseguida, pero María estaba cansada; se sentía bastante mal y le rogó que permaneciera con ella, lo cual hizo.

\section*{8. El nacimiento de Jesús}
\par 
%\textsuperscript{(1351.5)}
\textsuperscript{122:8.1} María estuvo inquieta toda aquella noche, de manera que ninguno de los dos durmió mucho. Al amanecer, los dolores del parto empezaron claramente, y a mediodía, el 21 de agosto del año 7 a. de J.C., con la ayuda y la asistencia generosa de unas viajeras como ella, María dio a luz a un niño varón. Jesús de Nazaret había nacido en el mundo. Se le envolvió en las ropas que María había traído por precaución, y se le acostó en un pesebre cercano\footnote{\textit{Nacimiento de Jesús}: Mt 1:25b; Lc 2:6-7.}.

\par 
%\textsuperscript{(1351.6)}
\textsuperscript{122:8.2} El niño de la promesa había nacido exactamente de la misma manera que todos los niños que antes y después de ese día han llegado al mundo. Al octavo día, según la costumbre judía, fue circuncidado\footnote{\textit{Circuncisión y dedicación}: Lc 2:21.} y se le llamó oficialmente Josué (Jesús).

\par 
%\textsuperscript{(1351.7)}
\textsuperscript{122:8.3} Al día siguiente del nacimiento de Jesús, José fue a empadronarse. Se encontró con un hombre con quien habían conversado dos noches antes en Jericó, y éste lo llevó a ver a un amigo rico que ocupaba una habitación en la posada, el cual dijo que con mucho gusto intercambiaría su alojamiento con el de la pareja de Nazaret. Aquella misma tarde se trasladaron a la posada, donde permanecieron cerca de tres semanas, hasta que encontraron alojamiento en la casa de un pariente lejano de José.

\par 
%\textsuperscript{(1351.8)}
\textsuperscript{122:8.4} Al segundo día del nacimiento de Jesús, María envió un mensaje a Isabel indicándole que su hijo había nacido, y ésta le respondió invitando a José a que subiera a Jerusalén para hablar con Zacarías de todos sus asuntos. A la semana siguiente, José fue a Jerusalén para conversar con Zacarías. Tanto Zacarías como Isabel habían llegado al sincero convencimiento de que Jesús estaba destinado a ser en verdad el libertador de los judíos, el Mesías, y que su hijo Juan sería el jefe de sus ayudantes, el brazo derecho de su destino. Como María compartía las mismas ideas, no fue difícil convencer a José para que se quedaran en Belén, la Ciudad de David, con objeto de que cuando Jesús creciera, pudiera ocupar el trono de todo Israel como sucesor de David. Por consiguiente, permanecieron más de un año en Belén, y José efectuó mientras tanto algunos trabajos en su oficio de carpintero.

\par 
%\textsuperscript{(1352.1)}
\textsuperscript{122:8.5} Aquel mediodía en que nació Jesús, los serafines de Urantia, reunidos bajo las órdenes de sus directores, cantaron efectivamente himnos de gloria por encima del pesebre de Belén, pero estas expresiones de alabanza no fueron oídas por los oídos humanos. Ningún pastor u otra criatura mortal vino a rendir homenaje al niño de Belén, hasta el día en que llegaron ciertos sacerdotes de Ur, que habían sido enviados por Zacarías desde Jerusalén\footnote{\textit{Ángeles, no pastores}: Lc 2:8-18, 20.}.

\par 
%\textsuperscript{(1352.2)}
\textsuperscript{122:8.6} Hacía algún tiempo, un extraño educador religioso de su país les había dicho a estos sacerdotes de Mesopotamia que había tenido un sueño en el cual se le informaba que la <<luz de la vida>> estaba a punto de aparecer en la Tierra como un niño y entre los judíos. Y hacia allí se dirigieron estos tres sacerdotes en busca de esta <<luz de la vida>>. Después de muchas semanas de búsqueda infructuosa en Jerusalén, estaban a punto de regresar a Ur cuando Zacarías se encontró con ellos, y les reveló su creencia de que Jesús era el objeto de su búsqueda; los envió a Belén, donde encontraron al niño y dejaron sus regalos a María, su madre terrestre. El niño tenía casi tres semanas en el momento de su visita\footnote{\textit{Visita de los hombres sabios}: Mt 2:1-12.}.

\par 
%\textsuperscript{(1352.3)}
\textsuperscript{122:8.7} Estos hombres sabios no vieron ninguna estrella que los guiara hasta Belén. La hermosa leyenda de la estrella de Belén se originó de la manera siguiente: Jesús había nacido el 21 de agosto, a mediodía, del año 7 a. de J.C. El 29 de mayo del mismo año 7 tuvo lugar una extraordinaria conjunción de Júpiter y de Saturno en la constelación de Piscis. Es un hecho astronómico notable que se produjeran conjunciones similares el 29 de septiembre y el 5 de diciembre del mismo año. Basándose en estos acontecimientos extraordinarios, pero totalmente naturales, los seguidores bien intencionados de las generaciones siguientes construyeron la atractiva leyenda de la estrella de Belén, que conducía a los Magos adoradores hasta el pesebre, donde contemplaron y adoraron al niño recién nacido. Las mentes de Oriente y del próximo Oriente se deleitan con los cuentos de hadas y tejen continuamente hermosos mitos como éste alrededor de la vida de sus dirigentes religiosos y de sus héroes políticos. En ausencia de imprenta, cuando la mayoría del conocimiento humano se trasmitía oralmente de una generación a la siguiente, era muy fácil que los mitos se transformaran en tradiciones, y que las tradiciones fueran aceptadas finalmente como hechos\footnote{\textit{No hubo estrella en Belén}: Mt 2:2,7,9-10.}.

\section*{9. La presentación en el templo}
\par 
%\textsuperscript{(1352.4)}
\textsuperscript{122:9.1} Moisés había enseñado a los judíos que cada hijo primogénito pertenecía al Señor\footnote{\textit{Rendención de los primogénitos}: Ex 13:1-2; 22:29-30; Nm 3:13; 8:16-17.}, pero que en lugar de sacrificarlo, como era costumbre entre las naciones paganas, ese hijo podría vivir siempre que sus padres lo redimieran\footnote{\textit{Redención}: Ex 13:15; 34:20; Nm 18:15-16; Lc 2:22-24.} mediante el pago de cinco siclos a cualquier sacerdote autorizado. También existía un mandato mosaico que ordenaba que después de haber pasado cierto tiempo, una madre tenía que presentarse en el templo para purificarse\footnote{\textit{Purificación de las puérperas}: Lv 2:2-3,6,8; Lc 2:22-24.} (o que alguien hiciera en su lugar el sacrificio apropiado). Era costumbre realizar ambas ceremonias al mismo tiempo\footnote{\textit{Dos ritos}: Lc 2:21-24.}. En consecuencia, José y María subieron personalmente al templo, en Jerusalén, para presentar a Jesús ante los sacerdotes, efectuar su redención y hacer al mismo tiempo el sacrificio apropiado para asegurar la purificación ceremonial de María de la supuesta impureza del alumbramiento.

\par 
%\textsuperscript{(1353.1)}
\textsuperscript{122:9.2} Dos personajes notables se paseaban constantemente por los patios del templo: Simeón, un cantor\footnote{\textit{Simeón}: Lc 2:25.}, y Ana, una poetisa\footnote{\textit{Ana}: Lc 2:36-38.}. Simeón era judeo, pero Ana era galilea. Los dos estaban juntos con frecuencia y ambos eran íntimos amigos del sacerdote Zacarías, que les había confiado el secreto de Juan y de Jesús. Tanto Simeón como Ana deseaban ardientemente la venida del Mesías, y su confianza en Zacarías les condujo a creer que Jesús era el libertador esperado por el pueblo judío\footnote{\textit{Ambos veían a Jesús como el Mesías}: Lc 2:26.}.

\par 
%\textsuperscript{(1353.2)}
\textsuperscript{122:9.3} Zacarías sabía el día que José y María tenían que venir al templo con Jesús y había convenido con Simeón y Ana que, en la procesión de los niños primogénitos, haría un saludo con la mano levantada para indicarles cuál era Jesús\footnote{\textit{Los dos estaban en el Templo}: Lc 2:27a.}.

\par 
%\textsuperscript{(1353.3)}
\textsuperscript{122:9.4} Para esta ocasión, Ana había escrito un poema\footnote{\textit{Poema de Ana}: Lc 1:67; 2:28.} que Simeón se puso a cantar, ante el gran asombro de José, de María y de todos los que se encontraban reunidos en los patios del templo. He aquí su himno de redención del hijo primogénito\footnote{\textit{Himno de rendención}: Lc 1:68-79; 2:29-32.}:

\par 
%\textsuperscript{(1353.4)}
\textsuperscript{122:9.5} Bendito sea el Señor, Dios de Israel,

\par 
%\textsuperscript{(1353.5)}
\textsuperscript{122:9.6} Porque nos ha visitado y ha traído la redención a su pueblo;

\par 
%\textsuperscript{(1353.6)}
\textsuperscript{122:9.7} Ha suscitado un poder salvador para todos nosotros

\par 
%\textsuperscript{(1353.7)}
\textsuperscript{122:9.8} En la casa de su siervo David.

\par 
%\textsuperscript{(1353.8)}
\textsuperscript{122:9.9} Según ha dicho por boca de sus santos profetas ---

\par 
%\textsuperscript{(1353.9)}
\textsuperscript{122:9.10} Nos salva de nuestros enemigos y de la mano de todos los que nos odian;

\par 
%\textsuperscript{(1353.10)}
\textsuperscript{122:9.11} Muestra misericordia a nuestros padres y recuerda su santa alianza ---

\par 
%\textsuperscript{(1353.11)}
\textsuperscript{122:9.12} El juramento por el que prometió a Abraham nuestro padre,

\par 
%\textsuperscript{(1353.12)}
\textsuperscript{122:9.13} Que nos concedería, después de librarnos de la mano de nuestros enemigos,

\par 
%\textsuperscript{(1353.13)}
\textsuperscript{122:9.14} Servirle sin temor,

\par 
%\textsuperscript{(1353.14)}
\textsuperscript{122:9.15} En santidad y rectitud delante suya, todos los días de nuestra vida.

\par 
%\textsuperscript{(1353.15)}
\textsuperscript{122:9.16} Sí, y tú, niño de la promesa, serás llamado el profeta del Altísimo;

\par 
%\textsuperscript{(1353.16)}
\textsuperscript{122:9.17} Porque irás delante de la faz del Señor para establecer su reino,

\par 
%\textsuperscript{(1353.17)}
\textsuperscript{122:9.18} Para dar conocimiento de la salvación a su pueblo

\par 
%\textsuperscript{(1353.18)}
\textsuperscript{122:9.19} En la remisión de sus pecados.

\par 
%\textsuperscript{(1353.19)}
\textsuperscript{122:9.20} Regocijáos en la tierna misericordia de nuestro Dios, porque desde lo alto el alba nos ha visitado ahora

\par 
%\textsuperscript{(1353.20)}
\textsuperscript{122:9.21} Para iluminar a los que habitan en las tinieblas y en la sombra de la muerte,

\par 
%\textsuperscript{(1353.21)}
\textsuperscript{122:9.22} Para guiar nuestros pasos por los caminos de la paz.

\par 
%\textsuperscript{(1353.22)}
\textsuperscript{122:9.23} Y ahora deja a tu siervo partir en paz, Oh, Señor, según tu palabra,

\par 
%\textsuperscript{(1353.23)}
\textsuperscript{122:9.24} Porque mis ojos han contemplado tu salvación,

\par 
%\textsuperscript{(1353.24)}
\textsuperscript{122:9.25} Que has preparado delante de la faz de todos los pueblos;

\par 
%\textsuperscript{(1353.25)}
\textsuperscript{122:9.26} Una luz para iluminar incluso a los gentiles\footnote{\textit{Iluminar a los gentiles}: Is 9:2; Jn 1:4-9; 8:12; 9:5; 1 Jn 2:8.}

\par 
%\textsuperscript{(1353.26)}
\textsuperscript{122:9.27} Y para la gloria de tu pueblo Israel.

\par 
%\textsuperscript{(1353.27)}
\textsuperscript{122:9.28} En el camino de vuelta a Belén, José y María permanecieron silenciosos ---confundidos y sobrecogidos\footnote{\textit{Asombro de los padres}: Lc 2:33.}. María estaba muy turbada por el saludo de despedida de Ana, la anciana poetisa, y José no estaba de acuerdo con este esfuerzo prematuro por hacer de Jesús el Mesías esperado del pueblo judío.

\section*{10. Herodes actúa}
\par 
%\textsuperscript{(1353.28)}
\textsuperscript{122:10.1} Pero los espías de Herodes no estaban inactivos. Cuando le informaron de la visita de los sacerdotes de Ur a Belén, Herodes ordenó que estos caldeos se presentaran ante él\footnote{\textit{Herodes ordena que se presenten los sabios}: Mt 2:1-3,.}. Interrogó cuidadosamente a estos sabios sobre el nuevo <<rey de los judíos>>, pero le proporcionaron poca satisfacción, explicando que el niño había nacido de una mujer que había venido a Belén con su marido para registrarse en el censo. Herodes no estaba satisfecho con esta respuesta y los despidió con una bolsa de dinero, ordenándoles que encontraran al niño para que él también pudiera ir a adorarlo, puesto que habían declarado que su reino sería espiritual, y no temporal\footnote{\textit{Herodes envía a los sabios a buscar al niño}: Mt 2:7-9a.}. Como los sabios no regresaban, Herodes empezó a sospechar. Mientras le daba vueltas a estas cosas en su cabeza, sus espías regresaron y le dieron un informe completo sobre los recientes incidentes acaecidos en el templo; le trajeron una copia de algunas partes de la canción de Simeón que se había cantado en las ceremonias de la redención de Jesús. Pero no se les había ocurrido seguir a José y María, y Herodes se encolerizó\footnote{\textit{Enfado de Herodes}: Mt 2:16a.} mucho con ellos cuando no pudieron decirle a dónde se había dirigido la pareja con el niño. Envió entonces a unos indagadores para que localizaran a José y María. Al enterarse que Herodes perseguía a la familia de Nazaret, Zacarías e Isabel permanecieron alejados de Belén. El niño fue ocultado en casa de unos parientes de José.

\par 
%\textsuperscript{(1354.1)}
\textsuperscript{122:10.2} José tenía miedo de buscar trabajo, y sus pocos ahorros estaban desapareciendo rápidamente. Incluso en el momento de las ceremonias de purificación en el templo, José se consideró lo bastante pobre\footnote{\textit{``Pobreza'' de José}: Lv 12:2,6-8; Lc 2:22-24.} como para limitar a dos palomas jóvenes la ofrenda de María, tal como Moisés había ordenado para la purificación de las madres pobres.

\par 
%\textsuperscript{(1354.2)}
\textsuperscript{122:10.3} Después de más de un año de búsqueda, los espías de Herodes aún no habían localizado a Jesús; y como se sospechaba que el niño estaba todavía oculto en Belén, Herodes preparó un decreto ordenando que se hiciera una búsqueda sistemática en todas las casas de Belén, y que mataran a todos los niños varones con menos de dos años de edad. De esta manera, Herodes pretendía asegurarse de que el niño que estaba destinado a ser el <<rey de los judíos>> sería destruido. Y así fue como en un día perecieron dieciséis niños varones en Belén de Judea. La intriga y el asesinato, incluso dentro de su propia familia cercana, eran cosa corriente en la corte de Herodes\footnote{\textit{Asesinato de los bebés varones}: Mt 2:16.}.

\par 
%\textsuperscript{(1354.3)}
\textsuperscript{122:10.4} La masacre de estos niños tuvo lugar a mediados de octubre del año 6 a. de J.C., cuando Jesús tenía poco más de un año. Pero incluso entre los miembros de la corte de Herodes había creyentes en el Mesías venidero, y uno de ellos, al enterarse de la orden de matar a los niños de Belén, se puso en contacto con Zacarías, quien a su vez envió un mensajero a José\footnote{\textit{El aviso}: Mt 2:13.}; la noche antes de la masacre, José y María salieron de Belén con el niño, camino de Alejandría en Egipto. Para evitar atraer la atención, viajaron solos con Jesús hasta Egipto\footnote{\textit{Huida a Egipto}: Mt 2:14.}. Fueron a Alejandría con los fondos que les proporcionó Zacarías, y allí José trabajó en su oficio, mientras que María y Jesús se alojaron con unos parientes acomodados de la familia de José. Vivieron en Alejandría dos años completos, y no regresaron a Belén hasta después de la muerte de Herodes\footnote{\textit{Longitud de la estancia}: Mt 2:15a,19-21.}.


\chapter{Documento 123. Los primeros años de la infancia de Jesús}
\par 
%\textsuperscript{(1355.1)}
\textsuperscript{123:0.1} DEBIDO a las incertidumbres y ansiedades de su estancia en Belén, María no destetó al niño hasta que llegaron sanos y salvos a Alejandría, donde la familia pudo llevar una vida normal. Vivieron con unos parientes, y José pudo mantener fácilmente a su familia porque consiguió trabajo poco después de su llegada. Estuvo empleado como carpintero durante varios meses y luego lo promovieron al puesto de capataz de un gran grupo de obreros que estaban ocupados en la construcción de un edificio público, entonces en obras. Esta nueva experiencia le dio la idea de hacerse contratista y constructor después de que regresaran a Nazaret.

\par 
%\textsuperscript{(1355.2)}
\textsuperscript{123:0.2} Durante todos estos primeros años de infancia en que Jesús estaba indefenso, María mantuvo una larga y constante vigilancia para que no le ocurriera nada a su hijo que pudiera amenazar su bienestar, o que pudiera obstaculizar, de alguna manera, su futura misión en la Tierra; ninguna madre estuvo nunca más consagrada a su hijo. En el hogar donde se encontraba Jesús, había otros dos niños aproximadamente de su misma edad, y entre los vecinos cercanos, seis más cuyas edades se acercaban lo suficiente a la suya como para ser unos compañeros de juego aceptables. Al principio, María estuvo tentada de mantener a Jesús muy cerca de ella. Temía que le ocurriera algo si se le permitía jugar en el jardín con los otros niños, pero José, con la ayuda de sus parientes, consiguió convencerla de que esta actitud privaría a Jesús de la útil experiencia de aprender a adaptarse a los niños de su edad. Comprendiendo que un programa así de protección exagerada e inhabitual podría hacer que el niño se volviera cohibido y un tanto egocéntrico, María dio finalmente su consentimiento al plan que permitía al niño de la promesa crecer exactamente como todos los demás niños. Aunque cumplió con esta decisión, efectuó su papel de estar siempre vigilante mientras que los pequeños jugaban alrededor de la casa o en el jardín. Sólo una madre amorosa puede comprender la carga que María tuvo que soportar en su corazón por la seguridad de su hijo durante estos años de su niñez y de su primera infancia.

\par 
%\textsuperscript{(1355.3)}
\textsuperscript{123:0.3} Durante los dos años de su estancia en Alejandría, Jesús gozó de buena salud y siguió creciendo normalmente. Aparte de unos pocos amigos y parientes, no se dijo a nadie que Jesús era un <<niño de la promesa>>. Uno de los parientes de José lo reveló a unos amigos de Menfis, descendientes del lejano Akenatón. Éstos se reunieron, con un pequeño grupo de creyentes de Alejandría, en la suntuosa casa del pariente y benefactor de José, poco antes de regresar a Palestina, para presentar sus mejores deseos a la familia de Nazaret y sus respetos al niño. En esta ocasión, los amigos reunidos regalaron a Jesús un ejemplar completo de la traducción al griego de las escrituras hebreas. Pero este ejemplar de los textos sagrados judíos no se lo entregaron a José hasta que él y María declinaron definitivamente la invitación de sus amigos de Menfis y Alejandría de permanecer en Egipto. Estos creyentes afirmaban que el hijo del destino podría ejercer una influencia mundial mucho mayor si residía en Alejandría que en cualquier lugar determinado de Palestina. Estos argumentos retrasaron algún tiempo\footnote{\textit{Retraso en el regreso}: Mt 2:21.} su regreso a Palestina, después de recibir la noticia de la muerte de Herodes.

\par 
%\textsuperscript{(1356.1)}
\textsuperscript{123:0.4} Finalmente, José y María se despidieron de Alejandría en un barco propiedad de su amigo Esraeon, con destino a Jope, puerto al que llegaron a finales de agosto del año 4 a. de J.C. Se dirigieron directamente a Belén, donde pasaron todo el mes de septiembre en deliberaciones con sus amigos y parientes para decidir si debían quedarse allí o regresar a Nazaret.

\par 
%\textsuperscript{(1356.2)}
\textsuperscript{123:0.5} María nunca había abandonado por completo la idea de que Jesús debería crecer en Belén, la Ciudad de David. José no creía en realidad que su hijo estuviera destinado a ser un rey liberador de Israel. Además, sabía que él mismo no era un verdadero descendiente de David; el hecho de contar entre el linaje de David se debía a que uno de sus antepasados había sido adoptado por la línea de descendientes davídicos. María consideraba naturalmente que la Ciudad de David era el lugar más apropiado para criar al nuevo candidato al trono de David, pero José prefería tentar la suerte con Herodes Antipas antes que con su hermano Arquelao. Albergaba muchos temores por la seguridad del niño en Belén o en cualquier otra ciudad de Judea; suponía que era más probable que Arquelao continuara con la política amenazadora de su padre Herodes, a que lo hiciera Antipas en Galilea. Aparte de todas estas razones, José expresó abiertamente su preferencia por Galilea, porque lo consideraba un lugar más adecuado para criar y educar al niño, pero necesitó tres semanas para vencer las objeciones de María\footnote{\textit{Elección de Nazaret como hogar}: Mt 2:22-23.}.

\par 
%\textsuperscript{(1356.3)}
\textsuperscript{123:0.6} El primero de octubre, José había convencido a María y a todos sus amigos de que era mejor para ellos regresar a Nazaret\footnote{\textit{Viaje a Nazaret}: Lc 2:39.}. En consecuencia, a principios de octubre del año 4 a. de J.C., partieron de Belén rumbo a Nazaret por el camino de Lida y Escitópolis. Salieron un domingo por la mañana temprano; María y el niño iban montados en la bestia de carga que acababan de adquirir, mientras que José y cinco parientes los acompañaban a pie; los parientes de José no consintieron que viajaran solos hasta Nazaret. Temían ir a Galilea pasando por Jerusalén y el valle del Jordán, y las rutas occidentales no eran del todo seguras para dos viajeros solitarios con un niño de poca edad.

\section*{1. De regreso a Nazaret}
\par 
%\textsuperscript{(1356.4)}
\textsuperscript{123:1.1} Al cuarto día de viaje, el grupo llegó sano y salvo a su destino. Llegaron sin anunciarse a su casa de Nazaret\footnote{\textit{Llegada a Nazaret}: Lc 2:39.}, ocupada desde hacía más de tres años por uno de los hermanos casados de José, que en verdad se quedó sorprendido al verlos; lo habían hecho todo tan calladamente, que ni la familia de José ni la de María sabían siquiera que habían dejado Alejandría. Al día siguiente, el hermano de José se mudó con su familia, y María, por primera vez desde el nacimiento de Jesús, se instaló con su pequeña familia para disfrutar de la vida en su propio hogar. En menos de una semana, José consiguió trabajo como carpintero, y fueron extremadamente felices.

\par 
%\textsuperscript{(1356.5)}
\textsuperscript{123:1.2} Jesús tenía unos tres años y dos meses cuando volvieron a Nazaret. Había soportado muy bien todos estos viajes y gozaba de excelente salud\footnote{\textit{Salud de Jesús}: Lc 2:40.}; estaba lleno de alegría y entusiasmo infantil al tener una casa propia donde poder correr y disfrutar. Pero echaba mucho de menos la relación con sus compañeros de juego de Alejandría.

\par 
%\textsuperscript{(1356.6)}
\textsuperscript{123:1.3} Camino de Nazaret, José había persuadido a María de que sería imprudente divulgar, entre sus amigos y parientes galileos, la noticia de que Jesús era un niño de la promesa. Acordaron no mencionar a nadie este asunto, y ambos cumplieron fielmente esta promesa.

\par 
%\textsuperscript{(1357.1)}
\textsuperscript{123:1.4} Todo el cuarto año de Jesús fue un período de desarrollo físico normal y de actividad mental poco común. Mientras tanto, se había hecho muy amigo de un niño vecino, aproximadamente de su edad, llamado Jacobo. Jesús y Jacobo siempre eran felices jugando juntos, y crecieron siendo grandes amigos y leales compañeros.

\par 
%\textsuperscript{(1357.2)}
\textsuperscript{123:1.5} El siguiente acontecimiento importante en la vida de esta familia de Nazaret fue el nacimiento del segundo hijo, Santiago\footnote{\textit{Santiago, el hermano de Jesús}: Mt 13:55; 27:56; Mc 6:3; 15:40; Gl 1:19.}, al amanecer del 2 de abril del año 3 a. de J.C. Jesús estaba muy emocionado con la idea de tener un hermanito, y permanecía cerca de él durante horas simplemente para observar los primeros gestos del bebé.

\par 
%\textsuperscript{(1357.3)}
\textsuperscript{123:1.6} Fue a mediados del verano de este mismo año cuando José construyó un pequeño taller cerca de la fuente del pueblo y del solar donde se detenían las caravanas. A partir de entonces hizo muy pocos trabajos de carpintería al día. Tenía como socios a dos de sus hermanos y a varios obreros más, a quienes enviaba a trabajar fuera mientras él permanecía en el taller fabricando arados, yugos y otros objetos de madera. También hizo algunos trabajos con el cuero, la soga y la lona. A medida que Jesús crecía, y cuando no estaba en la escuela, repartía su tiempo casi por igual entre ayudar a su madre en los quehaceres del hogar y observar a su padre en el trabajo del taller, escuchando al mismo tiempo las conversaciones y las noticias de los conductores y viajeros de las caravanas procedentes de todos los rincones de la Tierra.

\par 
%\textsuperscript{(1357.4)}
\textsuperscript{123:1.7} En julio de este año, un mes antes de cumplir Jesús los cuatro años, una epidemia maligna de trastornos intestinales, contagiada por los viajeros de las caravanas, se extendió por todo Nazaret. María se alarmó tanto por el peligro al que Jesús estaba expuesto con esta enfermedad epidémica, que preparó a sus dos hijos y huyó a la casa de campo de su hermano, a varios kilómetros al sur de Nazaret, en la carretera de Meguido, cerca de Sarid. Estuvieron fuera de Nazaret durante más de dos meses; Jesús disfrutó mucho con su primera experiencia en una granja.

\section*{2. El quinto año (año 2 a. de J.C.)}
\par 
%\textsuperscript{(1357.5)}
\textsuperscript{123:2.1} Poco más de un año después del regreso a Nazaret, el niño Jesús llegó a la edad de su primera decisión moral personal y sincera; fue entonces cuando vino a residir en él un Ajustador del Pensamiento, un don divino del Padre Paradisiaco, que había servido anteriormente con Maquiventa Melquisedek, adquiriendo así la experiencia de las operaciones relacionadas con la encarnación de un ser supermortal que vive en la similitud de la carne mortal. Este acontecimiento sucedió el 11 de febrero del año 2 a. de J.C. Jesús no tuvo más conciencia de la llegada del Monitor divino que los millones y millones de otros niños que, antes y después de ese día, han recibido igualmente estos Ajustadores del Pensamiento para residir en su mente, trabajar para la espiritualización última de dicha mente y la supervivencia eterna de su alma inmortal evolutiva.

\par 
%\textsuperscript{(1357.6)}
\textsuperscript{123:2.2} En este día de febrero terminó la supervisión directa y personal de los Gobernantes del Universo en lo referente a la integridad de Miguel encarnado como niño. A partir de este momento y durante todo el desarrollo humano de su encarnación, la custodia de Jesús fue encomendada a este Ajustador interior y a los guardianes seráficos asociados, auxiliados de vez en cuando por el ministerio de las criaturas intermedias, designadas para efectuar ciertas tareas específicas, de acuerdo con las instrucciones de sus superiores planetarios.

\par 
%\textsuperscript{(1357.7)}
\textsuperscript{123:2.3} Jesús cumplió cinco años en agosto de este año, y por ello nos referiremos a él como el quinto año (civil) de su vida. En este año 2 a. de J.C., poco más de un mes antes de su quinto cumpleaños, Jesús se sintió muy feliz con la llegada al mundo de su hermana Miriam\footnote{\textit{Miriam, la hermana de Jesús}: Mt 13:56; Mc 6:3.}, que nació en la noche del 11 de julio. Durante el atardecer del día siguiente, Jesús tuvo una larga conversación con su padre sobre la manera en que los diversos grupos de seres vivos nacen en el mundo como individuos diferentes. La parte más valiosa de la primera educación de Jesús la proporcionaron sus padres, respondiendo a sus preguntas reflexivas y penetrantes. José no dejó nunca de cumplir plenamente con su deber, tomándose el trabajo y encontrando el tiempo para contestar a las numerosas preguntas del niño. Desde los cinco hasta los diez años, Jesús fue una interrogación permanente. Aunque José y María no siempre podían contestar a sus preguntas, nunca dejaron de discutirlas a fondo, y lo ayudaban de todas las maneras posibles en sus esfuerzos por encontrar una solución satisfactoria al problema que su mente despierta le había sugerido.

\par 
%\textsuperscript{(1358.1)}
\textsuperscript{123:2.4} Desde su regreso a Nazaret, habían tenido una intensa vida familiar, y José había estado extraordinariamente ocupado con la construcción de su nuevo taller y la reanudación de sus negocios. Tenía tanto trabajo que no había encontrado tiempo para hacer una cuna para Santiago, pero esto pudo remediarlo mucho antes de que naciera Miriam, de manera que ella contó con una cuna muy cómoda en la cual se acurrucaba mientras que la familia la admiraba. El niño Jesús participaba de todo corazón en todas estas experiencias naturales y normales del hogar. Disfrutaba mucho con su hermanito y su hermanita, y ayudaba mucho a María cuidando de ellos.

\par 
%\textsuperscript{(1358.2)}
\textsuperscript{123:2.5} En el mundo de los gentiles de aquellos tiempos, había pocos hogares que pudieran proporcionar a un niño una educación intelectual, moral y religiosa mejor que la de los hogares judíos de Galilea. Estos judíos tenían un programa sistemático para criar y educar a sus hijos. Dividían la vida de los niños en siete etapas:

\par 
%\textsuperscript{(1358.3)}
\textsuperscript{123:2.6} 1. El niño recién nacido hasta el octavo día.

\par 
%\textsuperscript{(1358.4)}
\textsuperscript{123:2.7} 2. El niño de pecho.

\par 
%\textsuperscript{(1358.5)}
\textsuperscript{123:2.8} 3. El destete del niño.

\par 
%\textsuperscript{(1358.6)}
\textsuperscript{123:2.9} 4. El período de dependencia de la madre, hasta el final del quinto año.

\par 
%\textsuperscript{(1358.7)}
\textsuperscript{123:2.10} 5. El comienzo de la independencia del niño, y en el caso de los hijos varones, el padre asumía la responsabilidad de su educación.

\par 
%\textsuperscript{(1358.8)}
\textsuperscript{123:2.11} 6. Los chicos y las chicas adolescentes.

\par 
%\textsuperscript{(1358.9)}
\textsuperscript{123:2.12} 7. Los hombres y las mujeres jóvenes.

\par 
%\textsuperscript{(1358.10)}
\textsuperscript{123:2.13} Los judíos de Galilea tenían la costumbre de que la madre se responsabilizara de la educación del niño hasta que éste cumplía los cinco años, y si el niño era varón, entonces el padre se encargaba en adelante de su educación. Así pues, aquel año Jesús entró en la quinta etapa de la carrera de un niño judío de Galilea; en consecuencia, el 21 de agosto del año 2 a. de J.C., María transfirió formalmente a José la educación futura de su hijo.

\par 
%\textsuperscript{(1358.11)}
\textsuperscript{123:2.14} Aunque José tenía que asumir ahora directamente la responsabilidad de la educación intelectual y religiosa de Jesús, su madre seguía ocupándose de su educación hogareña. Le enseñó a conocer y a cuidar las parras y las flores que crecían en las tapias del jardín que rodeaban por completo la parcela de su hogar. María también se ocupó de poner en el tejado de la casa (el dormitorio de verano) unos cajones de arena poco profundos, en los que Jesús dibujaba mapas y efectuó la mayoría de sus primeras prácticas de escritura en arameo, en griego y más tarde en hebreo, porque aprendió en su momento a leer, escribir y hablar con fluidez estos tres idiomas.

\par 
%\textsuperscript{(1358.12)}
\textsuperscript{123:2.15} Jesús tenía la apariencia física de un niño casi perfecto y continuaba progresando de manera normal en el aspecto mental y emocional. Tuvo un ligero problema digestivo, su primera enfermedad leve, a finales de este año, su quinto año (civil).

\par 
%\textsuperscript{(1359.1)}
\textsuperscript{123:2.16} Aunque José y María hablaban con frecuencia del futuro de su hijo mayor, si hubierais estado allí, únicamente habríais observado el crecimiento de un niño normal de aquel tiempo y lugar, sano, sin preocupaciones, pero extremadamente ávido de saber.

\section*{3. Los acontecimientos del sexto año (año 1 a. de J.C.)}
\par 
%\textsuperscript{(1359.2)}
\textsuperscript{123:3.1} Con la ayuda de su madre, Jesús ya había dominado el dialecto galileo de la lengua aramea; ahora, su padre empezó a enseñarle el griego. María lo hablaba poco, pero José hablaba bien el griego y el arameo. El libro de texto para estudiar la lengua griega era el ejemplar de las escrituras hebreas ---una versión completa de la ley y de los profetas, incluídos los salmos--- que les habían regalado a su partida de Egipto. En todo Nazaret sólo había dos ejemplares completos de las escrituras en griego, y la posesión de uno de ellos por parte de la familia del carpintero hacía de la casa de José un lugar muy solicitado, lo que permitió a Jesús conocer, a medida que crecía, una procesión casi interminable de personas estudiosas serias y de sinceros buscadores de la verdad. Antes de terminar este año, Jesús había asumido la custodia de este manuscrito inestimable, habiéndose enterado el día de su sexto cumpleaños que el libro sagrado se lo habían regalado los amigos y parientes de Alejandría. Muy poco tiempo después podía leerlo con toda facilidad.

\par 
%\textsuperscript{(1359.3)}
\textsuperscript{123:3.2} La primera gran conmoción en la joven vida de Jesús tuvo lugar cuando aún no tenía seis años. Al chico le parecía que su padre ---o al menos su padre y su madre juntos--- lo sabían todo. Imaginad pues la sorpresa que se llevó este niño indagador cuando preguntó a su padre la causa de un leve terremoto que acababa de producirse, y oyó que José le respondía: <<Hijo mío, en verdad no lo sé>>. Así empezó una larga y desconcertante cadena de desilusiones, durante la cual Jesús descubrió que sus padres terrestres no eran infinitamente sabios ni omniscientes.

\par 
%\textsuperscript{(1359.4)}
\textsuperscript{123:3.3} El primer pensamiento de José fue decirle a Jesús que el terremoto había sido causado por Dios, pero un instante de reflexión le advirtió que una respuesta semejante provocaría inmediatamente preguntas posteriores aún más embarazosas. Incluso a una edad muy temprana, era muy difícil contestar a las preguntas de Jesús sobre los fenómenos físicos o sociales diciéndole a la ligera que el responsable era Dios o el diablo. De acuerdo con la creencia predominante del pueblo judío, hacía tiempo que Jesús estaba dispuesto a aceptar la doctrina de los buenos y de los malos espíritus como una posible explicación de los fenómenos mentales y espirituales; pero empezó a dudar muy pronto de que estas influencias invisibles fueran responsables de los acontecimientos físicos del mundo natural.

\par 
%\textsuperscript{(1359.5)}
\textsuperscript{123:3.4} Antes de que Jesús cumpliera los seis años de edad, a principios del verano del año 1 a. de J.C., Zacarías, Isabel y su hijo Juan vinieron a visitar a la familia de Nazaret. Jesús y Juan disfrutaron mucho durante esta visita, la primera que podían recordar. Aunque los visitantes sólo pudieron quedarse unos días, los padres hablaron de muchas cosas, incluyendo los planes para el futuro de sus hijos. Mientras que estaban ocupados en esto, los chicos jugaban en la azotea de la casa con trozos de madera en la arena, y se divertían juntos de otras muchas maneras, como hacen los niños.

\par 
%\textsuperscript{(1359.6)}
\textsuperscript{123:3.5} Después de conocer a Juan, que venía de los alrededores de Jerusalén, Jesús empezó a manifestar un interés extraordinario por la historia de Israel y comenzó a preguntar con mucho detalle por el significado de los ritos del sábado, los sermones de la sinagoga y las fiestas conmemorativas periódicas. Su padre le explicó el significado de todas estas celebraciones. La primera era la fiesta de la iluminación, a mediados del invierno, que duraba ocho días; la primera noche encendían una vela, y cada noche siguiente añadían una nueva. Con esto se conmemoraba la consagración del templo, después de que Judas Macabeo restaurara los oficios mosaicos. A continuación venía la celebración de Purim, a principios de la primavera, la fiesta de Esther y de la liberación de Israel gracias a ella. Luego seguía la solemne Pascua, que los adultos celebraban en Jerusalén siempre que era posible, mientras que en el hogar los niños debían recordar que no se podía comer pan con levadura en toda la semana. Más tarde venía la fiesta de los primeros frutos, la recogida de la cosecha; y por último la más solemne de todas, la fiesta del año nuevo, el día de la expiación. Algunas de estas celebraciones y ceremonias eran difíciles de comprender para la joven mente de Jesús, pero las examinó con seriedad, y luego participó con gran alegría en la fiesta de los tabernáculos, el período de las vacaciones anuales de todo el pueblo judío, la época en que acampaban en cabañas hechas con ramajes y se entregaban al júbilo y a los placeres.

\par 
%\textsuperscript{(1360.1)}
\textsuperscript{123:3.6} Durante este año, José y María tuvieron dificultades con Jesús a propósito de sus oraciones. Insistía en dirigirse a su Padre celestial como si estuviera hablando con José, su padre terrenal. Este abandono de las formas más solemnes y reverentes de comunicación con la Deidad era un poco desconcertante para sus padres, especialmente para su madre, pero no podían persuadirlo para que cambiara; recitaba sus oraciones tal como le habían enseñado, después de lo cual insistía en tener <<una pequeña charla con mi Padre que está en los cielos>>.

\par 
%\textsuperscript{(1360.2)}
\textsuperscript{123:3.7} En junio de este año, José cedió el taller de Nazaret a sus hermanos y empezó formalmente a trabajar como constructor. Antes de terminar el año, los ingresos de la familia se habían más que triplicado. La familia de Nazaret nunca más conoció el apuro de la pobreza hasta después de la muerte de José. La familia creció cada vez más y gastaron mucho dinero en estudios complementarios y en viajes, pero los ingresos crecientes de José siempre se mantuvieron a la altura de los gastos en aumento.

\par 
%\textsuperscript{(1360.3)}
\textsuperscript{123:3.8} Durante los pocos años que siguieron, José hizo trabajos considerables en Caná, Belén (de Galilea), Magdala, Naín, Séforis, Cafarnaúm y Endor, así como muchas construcciones en Nazaret y sus alrededores. Como Santiago había crecido lo suficiente como para ayudar a su madre en los quehaceres domésticos y en el cuidado de los niños más pequeños, Jesús se desplazó frecuentemente con su padre a estas ciudades y pueblos vecinos. Jesús era un observador penetrante y adquirió muchos conocimientos prácticos en estos viajes lejos de su hogar; guardaba asiduamente los conocimientos relacionados con el hombre y su manera de vivir en la Tierra.

\par 
%\textsuperscript{(1360.4)}
\textsuperscript{123:3.9} Este año Jesús hizo grandes progresos para adaptar sus sentimientos enérgicos y sus impulsos vigorosos a las exigencias de la cooperación familiar y de la disciplina del hogar. María era una madre amorosa pero bastante estricta en la disciplina. Sin embargo, en muchos aspectos, José era el que ejercía el mayor control sobre Jesús, porque solía sentarse con el muchacho y le explicaba íntegramente las razones reales y subyacentes por las cuales era necesario disciplinar los deseos personales para contribuir al bienestar y la tranquilidad de toda la familia. Cuando se le explicaba la situación, Jesús siempre cooperaba inteligente y voluntariamente con los deseos paternos y las reglas familiares.

\par 
%\textsuperscript{(1360.5)}
\textsuperscript{123:3.10} Cuando su madre no necesitaba su ayuda en la casa, Jesús dedicaba una gran parte de su tiempo libre a estudiar las flores y las plantas durante el día, y las estrellas por la noche. Mostraba una tendencia molesta a permanecer acostado de espaldas contemplando con admiración el cielo estrellado, mucho después de la hora habitual de acostarse en esta casa bien organizada de Nazaret.

\section*{4. El séptimo año (año 1 d. de J.C.)}
\par 
%\textsuperscript{(1361.1)}
\textsuperscript{123:4.1} Éste fue en verdad un año lleno de acontecimientos en la vida de Jesús. A principios de enero, una gran tormenta de nieve cayó sobre Galilea. La nieve se acumuló hasta sesenta centímetros de espesor; fue la nevada más grande que Jesús conoció en toda su vida y una de las más importantes en Nazaret en los últimos cien años.

\par 
%\textsuperscript{(1361.2)}
\textsuperscript{123:4.2} Las distracciones de los niños judíos en los tiempos de Jesús eran más bien limitadas; con demasiada frecuencia, los niños imitaban en sus juegos las actividades más serias que observaban en los adultos. Jugaban mucho a las bodas y a los funerales, ceremonias que veían con tanta frecuencia y que resultaban tan espectaculares. Bailaban y cantaban, pero tenían pocos juegos organizados como los que gustan tanto a los niños de hoy.

\par 
%\textsuperscript{(1361.3)}
\textsuperscript{123:4.3} En compañía de un niño vecino, y más tarde de su hermano Santiago, a Jesús le encantaba jugar en el rincón más alejado del taller de carpintería de la familia, donde se divertían con el serrín y los trozos de madera. A Jesús siempre le resultaba difícil comprender el daño de ciertos tipos de juegos que estaban prohibidos el sábado, pero nunca dejó de conformarse a los deseos de sus padres. Tenía una capacidad para el humor y los juegos que pocas veces se podía expresar en el entorno de su época y de su generación; pero hasta la edad de catorce años, la mayor parte del tiempo estaba alegre y de buen humor.

\par 
%\textsuperscript{(1361.4)}
\textsuperscript{123:4.4} María tenía un palomar en el tejado del establo contiguo a la casa, y los beneficios de la venta de las palomas los utilizaban como fondo especial de caridad que Jesús administraba, después de deducir el diezmo y haberlo entregado al empleado de la sinagoga.

\par 
%\textsuperscript{(1361.5)}
\textsuperscript{123:4.5} El único accidente verdadero que Jesús sufrió hasta ese momento fue una caída por las escaleras de piedra del patio trasero que conducían al dormitorio con techo de lona. Sucedió en julio, durante una tormenta de arena inesperada procedente del este. Los vientos cálidos con ráfagas de arena fina soplaban por lo general durante la estación de las lluvias, particularmente en marzo y abril. Una tormenta de este tipo era totalmente inesperada en el mes de julio. Cuando se desencadenó la tormenta, Jesús estaba jugando como tenía costumbre en el tejado de la casa, porque durante una gran parte de la temporada seca, éste era su lugar de juego habitual. La arena lo cegó mientras bajaba las escaleras, y cayó. Después de este accidente, José construyó una balaustrada a ambos lados de la escalera.

\par 
%\textsuperscript{(1361.6)}
\textsuperscript{123:4.6} No había manera de prevenir este accidente. No se trató de una negligencia imputable a los guardianes temporales intermedios, uno primario y otro secundario, asignados a la custodia del muchacho; tampoco se podía culpar al serafín guardián. Sencillamente no se pudo evitar. Pero este ligero accidente, ocurrido mientras que José estaba en Endor, ocasionó una ansiedad tan grande en la mente de María, que trató de manera poco razonable de mantener a Jesús pegado a ella durante varios meses.

\par 
%\textsuperscript{(1361.7)}
\textsuperscript{123:4.7} Las personalidades celestiales no intervienen arbitrariamente en los accidentes materiales, que son acontecimientos comunes de naturaleza física. En las circunstancias ordinarias, sólo las criaturas intermedias pueden intervenir sobre las condiciones materiales para salvaguardar a las personas, hombres o mujeres, con un destino determinado; incluso en las situaciones especiales, estos seres sólo pueden actuar así de conformidad con las órdenes específicas de sus superiores.

\par 
%\textsuperscript{(1361.8)}
\textsuperscript{123:4.8} Éste no fue más que uno de los numerosos accidentes menores que le ocurrieron posteriormente a este joven intrépido e investigador. Si pensáis en la niñez y en la juventud normales de un muchacho dinámico, tendréis una idea bastante buena de la carrera juvenil de Jesús, y casi podréis imaginar la cantidad de ansiedad que causó a sus padres, en particular a su madre.

\par 
%\textsuperscript{(1362.1)}
\textsuperscript{123:4.9} José, el cuarto hijo de la familia de Nazaret, nació la mañana del miércoles 16 de marzo del año 1 d. de J.C.\footnote{\textit{José, el hermano de Jesús}: Mt 13:55; 27:56; Mc 6:3; 15:40.}

\section*{5. Los años de escuela en Nazaret}
\par 
%\textsuperscript{(1362.2)}
\textsuperscript{123:5.1} Jesús tenía ahora siete años, la edad en que se suponía que los niños judíos empezaban su educación formal en las escuelas de la sinagoga. Por consiguiente, en agosto de este año comenzó su memorable vida escolar en Nazaret. El muchacho ya leía, escribía y hablaba con soltura dos idiomas, el arameo y el griego. Ahora tenía que imponerse la tarea de aprender a leer, escribir y hablar la lengua hebrea. Estaba realmente impaciente por empezar la nueva vida escolar que se abría ante él.

\par 
%\textsuperscript{(1362.3)}
\textsuperscript{123:5.2} Durante tres años ---hasta que tuvo diez años--- asistió a la escuela primaria de la sinagoga de Nazaret. Durante estos tres años estudió los rudimentos del Libro de la Ley, tal como estaba redactado en lengua hebrea. Durante los tres años siguientes estudió en la escuela superior y memorizó, por el método de repetición en voz alta, las enseñanzas más profundas de la ley sagrada. Se graduó en esta escuela de la sinagoga cuando tenía trece años, y los dirigentes de la sinagoga lo entregaron a sus padres como un <<hijo del mandamiento>> ya educado ---en adelante, un ciudadano responsable de la comunidad de Israel, con derecho a asistir a la Pascua en Jerusalén; en consecuencia, ese año participó en su primera Pascua, en compañía de su padre y su madre.

\par 
%\textsuperscript{(1362.4)}
\textsuperscript{123:5.3} En Nazaret, los alumnos se sentaban en semicírculo en el suelo mientras que su profesor, el chazan, un empleado de la sinagoga, se sentaba enfrente de ellos. Empezaban por el Libro del Levítico, y luego pasaban al estudio de los demás libros de la ley, seguido del estudio de los Profetas y de los Salmos. La sinagoga de Nazaret poseía un ejemplar completo de las escrituras en hebreo. Hasta los doce años, lo único que estudiaban eran las escrituras. En los meses de verano, las horas escolares se reducían considerablemente.

\par 
%\textsuperscript{(1362.5)}
\textsuperscript{123:5.4} Jesús se convirtió muy pronto en un experto en hebreo. Siendo un hombre joven, cuando ningún visitante eminente se encontraba ocasionalmente en Nazaret, se le pedía a menudo que leyera las escrituras hebreas a los fieles reunidos en la sinagoga para los oficios regulares del sábado.

\par 
%\textsuperscript{(1362.6)}
\textsuperscript{123:5.5} Por supuesto, las escuelas de la sinagoga no tenían libros de texto. Para enseñar, el chazan efectuaba una exposición que los alumnos repetían al unísono después de él. Cuando tenían acceso a los libros escritos de la ley, los estudiantes aprendían su lección leyendo en voz alta y repitiendo constantemente.

\par 
%\textsuperscript{(1362.7)}
\textsuperscript{123:5.6} Además de su educación oficial, Jesús empezó a tomar contacto con la naturaleza humana de todos los rincones del mundo, ya que por el taller de reparaciones de su padre pasaban hombres de muy diversos países. Cuando tuvo más edad, se mezclaba libremente con las caravanas que se detenían cerca de la fuente para descansar y comer. Como hablaba muy bien el griego, tenía pocos problemas para conversar con la mayoría de los viajeros y conductores de las caravanas.

\par 
%\textsuperscript{(1362.8)}
\textsuperscript{123:5.7} Nazaret era una etapa en el camino de las caravanas y una travesía para los viajes; una gran parte de la población era gentil. Al mismo tiempo, Nazaret era bien conocida como centro de interpretación liberal de la ley tradicional judía. En Galilea, los judíos se mezclaban más libremente con los gentiles que en Judea. De todas las ciudades de Galilea, los judíos de Nazaret eran los más liberales en interpretar las restricciones sociales basadas en el miedo a contaminarse por estar en contacto con los gentiles. Esta situación dio origen a un dicho corriente en Jerusalén: <<¿Puede salir algo bueno de Nazaret?>>\footnote{\textit{¿Puede salir algo bueno de Nazaret?}: Jn 1:46.}

\par 
%\textsuperscript{(1363.1)}
\textsuperscript{123:5.8} Jesús recibió su enseñanza moral y su cultura espiritual principalmente en su propio hogar. La mayor parte de su educación intelectual y teológica la adquirió del chazan. Pero su verdadera educación ---el equipamiento de mente y corazón para la prueba real de afrontar los difíciles problemas de la vida--- la obtuvo mezclándose con sus semejantes. Esta asociación estrecha con sus semejantes, jóvenes y viejos, judíos y gentiles, le proporcionó la oportunidad de conocer a la raza humana. Jesús era muy instruido, en el sentido de que comprendía a fondo a los hombres y los amaba con devoción.

\par 
%\textsuperscript{(1363.2)}
\textsuperscript{123:5.9} Durante todos sus años en la sinagoga fue un estudiante brillante, con una gran ventaja puesto que conocía bien tres idiomas. Con motivo de la finalización de los cursos de Jesús en la escuela, el chazan de Nazaret comentó a José que temía <<haber aprendido más de las preguntas penetrantes de Jesús>> que lo que había <<sido capaz de enseñar al muchacho>>.

\par 
%\textsuperscript{(1363.3)}
\textsuperscript{123:5.10} En el transcurso de sus estudios, Jesús aprendió mucho y obtuvo una gran inspiración de los sermones regulares del sábado en la sinagoga. Era costumbre solicitar a los visitantes distinguidos que se detenían el sábado en Nazaret que hablaran en la sinagoga. A medida que crecía, Jesús escuchó los puntos de vista de muchos grandes pensadores de todo el mundo judío, y también a muchos judíos poco ortodoxos, puesto que la sinagoga de Nazaret era un centro avanzado y liberal del pensamiento y de la cultura hebreos.

\par 
%\textsuperscript{(1363.4)}
\textsuperscript{123:5.11} Al ingresar en la escuela a los siete años (por aquella época los judíos acababan de sacar una ley sobre la educación obligatoria), era costumbre que los alumnos escogieran su <<texto de cumpleaños>>, una especie de regla de oro que los guiaría a lo largo de sus estudios, y sobre la cual muchas veces tenían que disertar en el momento de graduarse a la edad de trece años. El texto que Jesús escogió estaba sacado del profeta Isaías: <<El espíritu del Señor Dios está sobre mí, porque el Señor me ha ungido; me ha enviado para traer la buena nueva a los mansos, para consolar a los afligidos, para proclamar la libertad a los cautivos y para liberar a los presos espirituales>>\footnote{\textit{El Espíritu del Señor está sobre mí}: Is 61:1.}.

\par 
%\textsuperscript{(1363.5)}
\textsuperscript{123:5.12} Nazaret era uno de los veinticuatro centros sacerdotales de la nación hebrea. Pero el clero de Galilea era más liberal que los escribas y rabinos de Judea en su interpretación de las leyes tradicionales. En Nazaret también eran más liberales en cuanto a la observancia del sábado. Por este motivo, José tenía la costumbre de llevarse de paseo a Jesús los sábados por la tarde; una de sus caminatas favoritas consistía en subir a la alta colina cercana a su casa, de donde podían contemplar una vista panorámica de toda Galilea. Al noroeste, en los días despejados, podían ver la larga cima del Monte Carmelo deslizándose hacia el mar; Jesús escuchó muchas veces a su padre contar la historia de Elías\footnote{\textit{Historia de Elías}: 1 Re 17:1-19:21; 1 Re 21:17-29; 2 Re 1:3-2:11.}, uno de los primeros de la larga lista de profetas hebreos, que criticó a Acab y desenmascaró a los sacerdotes de Baal. Al norte, el Monte Hermón levantaba su pico nevado con un esplendor majestuoso y dominaba el horizonte, con casi 1.000 metros de laderas superiores que resplandecían con la blancura de las nieves perpetuas. A lo lejos, por el este, podían discernir el valle del Jordán, y mucho más allá, las colinas rocosas de Moab. También hacia el sur y el este, cuando el Sol iluminaba los muros de mármol, podían ver las ciudades greco-romanas de la Decápolis, con sus anfiteatros y sus templos presuntuosos. Y cuando se demoraban hasta la puesta del Sol, podían distinguir al oeste los barcos de vela en el lejano Mediterráneo.

\par 
%\textsuperscript{(1364.1)}
\textsuperscript{123:5.13} Jesús podía observar las filas de caravanas que entraban y salían de Nazaret en cuatro direcciones, y hacia el sur podía ver la amplia y fértil llanura de Esdraelón, que se extendía hacia el Monte Gilboa y Samaria.

\par 
%\textsuperscript{(1364.2)}
\textsuperscript{123:5.14} Cuando no subían a las alturas para contemplar el paisaje lejano, se paseaban por el campo y estudiaban la naturaleza en sus distintas manifestaciones, según las estaciones. La educación más precoz de Jesús, exceptuando la del hogar familiar, había consistido en tomar un contacto respetuoso y comprensivo con la naturaleza.

\par 
%\textsuperscript{(1364.3)}
\textsuperscript{123:5.15} Antes de cumplir los ocho años de edad, era conocido por todas las madres y mujeres jóvenes de Nazaret que se habían encontrado y hablado con él en la fuente cercana a su casa, que era uno de los centros sociales de encuentro y de habladurías de toda la ciudad. Este año, Jesús aprendió a ordeñar la vaca de la familia y a cuidar de los demás animales. Durante este año y el siguiente, también aprendió a hacer queso y a tejer. Cuando llegó a los diez años era un experto tejedor. Aproximadamente por esta época, Jesús y Jacobo, el muchacho vecino, se hicieron grandes amigos del alfarero que trabajaba cerca del manantial; mientras observaban los hábiles dedos de Natán moldeando la arcilla en el torno, los dos decidieron muchas veces hacerse alfareros cuando fueran mayores. Natán quería mucho a los muchachos y a menudo les daba arcilla para que jugaran, tratando de estimular su imaginación creativa sugiriéndoles que compitieran en la modelación de objetos y animales diversos.

\section*{6. Su octavo año (año 2 d. de J.C.)}
\par 
%\textsuperscript{(1364.4)}
\textsuperscript{123:6.1} Éste fue un año interesante en la escuela. Aunque Jesús no era un estudiante excepcional, sí era un alumno aplicado y formaba parte del tercio más avanzado de la clase; hacía sus tareas tan bien que durante una semana al mes estaba exento de asistir a la escuela. Dicha semana la pasaba generalmente con su tío el pescador en las orillas del mar de Galilea, cerca de Magdala, o en la granja de otro tío suyo (hermano de su madre) a ocho kilómetros al sur de Nazaret.

\par 
%\textsuperscript{(1364.5)}
\textsuperscript{123:6.2} Aunque su madre se preocupaba exageradamente por su salud y su seguridad, poco a poco se iba habituando a estas ausencias fuera del hogar. Los tíos y las tías de Jesús lo querían mucho; entre ellos se produjo una viva rivalidad, durante todo este año y algunos de los siguientes, para asegurarse su compañía durante estas visitas mensuales. Su primera estancia de una semana (desde la infancia) en la granja de su tío fue en enero de este año; la primera semana de experiencia como pescador en el mar de Galilea tuvo lugar en el mes de mayo.

\par 
%\textsuperscript{(1364.6)}
\textsuperscript{123:6.3} Por esta época, Jesús conoció a un profesor de matemáticas de Damasco, y después de aprender algunas nuevas técnicas aritméticas, dedicó mucho tiempo a las matemáticas durante varios años. Desarrolló un agudo sentido de los números, de las distancias y de las proporciones.

\par 
%\textsuperscript{(1364.7)}
\textsuperscript{123:6.4} Jesús empezó a disfrutar mucho con su hermano Santiago, y al final de este año había empezado a enseñarle el alfabeto.

\par 
%\textsuperscript{(1364.8)}
\textsuperscript{123:6.5} Jesús hizo planes este año para intercambiar productos lácteos por clases de arpa. Tenía una inclinación especial por todo lo musical. Más adelante contribuyó mucho a promover el interés por la música vocal entre sus jóvenes compañeros. A la edad de once años ya era un arpista hábil, y disfrutaba mucho entreteniendo a la familia y a los amigos con sus extraordinarias interpretaciones y con sus hábiles improvisaciones.

\par 
%\textsuperscript{(1365.1)}
\textsuperscript{123:6.6} Aunque Jesús continuaba haciendo progresos considerables en la escuela, no todo se desarrollaba fácilmente para sus padres o sus maestros. Persistía en hacer muchas preguntas embarazosas acerca de la ciencia y de la religión, particularmente en geografía y astronomía. Insistía especialmente en averiguar por qué había una temporada seca y una temporada de lluvias en Palestina. Una y otra vez buscó la explicación de la gran diferencia entre las temperaturas de Nazaret y las del valle del Jordán. Simplemente no paraba nunca de hacer preguntas de este tipo, inteligentes pero inquietantes.

\par 
%\textsuperscript{(1365.2)}
\textsuperscript{123:6.7} Su tercer hermano, Simón, nació la tarde del viernes 14 de abril de este año, el 2 d. de J.C.\footnote{\textit{Simón, hermano de Jesús}: Mt 13:55; Mc 6:3.}

\par 
%\textsuperscript{(1365.3)}
\textsuperscript{123:6.8} Nacor, un profesor de una academia rabínica de Jerusalén, vino en febrero a Nazaret para observar a Jesús, después de haber realizado una misión similar en casa de Zacarías, cerca de Jerusalén. Vino a Nazaret por insistencia del padre de Juan. Aunque al principio le disgustó un poco la franqueza de Jesús y su manera nada convencional de relacionarse con las cosas religiosas, lo atribuyó a que Galilea estaba lejos de los centros de instrucción y de cultura hebreos, y aconsejó a José y María que le permitieran llevarse a Jesús a Jerusalén, donde tendría las ventajas de la educación y de la enseñanza en el centro de la cultura judía. María estaba casi decidida a dar su consentimiento; estaba convencida de que su hijo mayor iba a ser el Mesías, el libertador de los judíos. José dudaba; él también estaba persuadido de que cuando Jesús creciera sería un hombre del destino, pero estaba profundamente inseguro en cuanto a cuál sería ese destino. Pero nunca dudó realmente de que su hijo tuviera que realizar alguna gran misión en la Tierra. Cuanto más pensaba en el consejo de Nacor, más dudaba de la sabiduría de este proyecto de estancia en Jerusalén.

\par 
%\textsuperscript{(1365.4)}
\textsuperscript{123:6.9} Debido a esta diferencia de opinión entre José y María, Nacor solicitó permiso para someter todo el asunto a Jesús. Jesús escuchó con atención y habló con José, con María y con un vecino, Jacobo el albañil, cuyo hijo era su compañero de juego favorito. Dos días más tarde, les manifestó que había diferencias de opinión entre sus padres y sus consejeros, y que no se consideraba cualificado para asumir la responsabilidad de tal decisión, porque no se sentía fuertemente inclinado ni en un sentido ni en otro. En estas circunstancias, había decidido finalmente <<hablar con mi Padre que está en los cielos>>; y aunque no estaba totalmente seguro de la respuesta, sentía que debía más bien quedarse en casa <<con mi padre y mi madre>>, añadiendo: <<Ellos que me quieren tanto, serán capaces de hacer más por mí y de guiarme con más seguridad que unos extraños que sólo pueden ver mi cuerpo y observar mi mente, pero que difícilmente pueden conocerme de verdad>>. Todos se quedaron maravillados, y Nacor emprendió su camino de regreso a Jerusalén. Pasaron muchos años antes de que se volviera a considerar la posibilidad de que Jesús se fuera de su hogar.


\chapter{Documento 124. Los últimos años de la infancia de Jesús}
\par 
%\textsuperscript{(1366.1)}
\textsuperscript{124:0.1} AUNQUE Jesús podría haberse beneficiado en Alejandría de mejores oportunidades para instruirse que en Galilea, no hubiera tenido un entorno tan espléndido para resolver los problemas de su propia vida con un mínimo de guía educativa, disfrutando al mismo tiempo de la gran ventaja de un contacto permanente con una cantidad tan grande de hombres y mujeres de todas clases, procedentes de todos los lugares del mundo civilizado. Si hubiera permanecido en Alejandría, su educación hubiera sido dirigida por judíos y según principios exclusivamente judíos. En Nazaret consiguió una educación y recibió una instrucción que lo prepararon mucho mejor para comprender a los gentiles, y le proporcionaron una idea mejor y más equilibrada de los méritos respectivos de los puntos de vista de la teología hebrea oriental, o babilónica, y de la occidental, o helénica.

\section*{1. El noveno año de Jesús (año 3 d. de J.C.)}
\par 
%\textsuperscript{(1366.2)}
\textsuperscript{124:1.1} Aunque no se puede decir que Jesús estuviera nunca gravemente enfermo, este año sufrió algunas enfermedades menores de la infancia junto con sus hermanos y su hermanita.

\par 
%\textsuperscript{(1366.3)}
\textsuperscript{124:1.2} En la escuela continuaban las clases, y seguía siendo un estudiante favorecido, con una semana libre cada mes; continuaba dividiendo su tiempo en partes más o menos iguales entre los viajes con su padre a las ciudades vecinas, las estancias en la granja de su tío al sur de Nazaret y las excursiones de pesca fuera de Magdala.

\par 
%\textsuperscript{(1366.4)}
\textsuperscript{124:1.3} El incidente más grave ocurrido hasta entonces en la escuela se produjo a finales del invierno, cuando Jesús se atrevió a desafiar la enseñanza del chazan de que todas las imágenes, pinturas y dibujos eran de naturaleza idólatra. A Jesús le encantaba dibujar paisajes y modelar una gran variedad de objetos con arcilla de alfarero. Todo este tipo de cosas estaba estrictamente prohibido por la ley judía, pero hasta ese momento se las había arreglado para calmar las objeciones de sus padres, hasta tal punto que le habían permitido continuar con estas actividades.

\par 
%\textsuperscript{(1366.5)}
\textsuperscript{124:1.4} Pero un nuevo alboroto se produjo en la escuela cuando uno de los alumnos más retrasados descubrió a Jesús haciendo, al carbón, un retrato del profesor en el suelo de la clase. El retrato estaba allí, tan claro como la luz del día, y muchos de los ancianos lo pudieron contemplar antes de que el comité se presentara ante José para exigirle que hiciera algo para reprimir la desobediencia a la ley de su hijo mayor. Aunque no era la primera vez que José y María recibían quejas sobre las actividades de su polifacético y dinámico hijo, ésta era la acusación más seria de todas las que hasta el momento habían presentado contra él. Sentado en una gran piedra junto a la puerta trasera, Jesús escuchó durante un rato cómo condenaban sus esfuerzos artísticos. Le irritó que culparan a su padre de sus pretendidas fechorías; entonces entró en la casa, enfrentándose sin temor a sus acusadores. Los ancianos se quedaron desconcertados. Algunos tendieron a considerar el incidente con humor, mientras que uno o dos parecían pensar que el chico era sacrílego, si no blasfemo. José estaba perplejo y María indignada, pero Jesús insistió en ser escuchado. Lo dejaron hablar, defendió valientemente su punto de vista y anunció con un completo dominio de sí mismo que acataría la decisión de su padre, tanto en este asunto como en cualquier otra controversia. Y el comité de ancianos partió en silencio.

\par 
%\textsuperscript{(1367.1)}
\textsuperscript{124:1.5} María intentó convencer a José para que permitiera a Jesús modelar la arcilla en casa, siempre que prometiera no realizar en la escuela ninguna de estas actividades problemáticas, pero José se vio obligado a ordenar que la interpretación rabínica del segundo mandamiento tenía que prevalecer. Así pues, desde ese día, Jesús no volvió a dibujar ni a modelar una forma cualquiera mientras vivió en la casa de su padre. Sin embargo, no estaba convencido de que lo que había hecho estuviera mal, y abandonar su pasatiempo favorito constituyó una de las grandes pruebas de su joven vida.

\par 
%\textsuperscript{(1367.2)}
\textsuperscript{124:1.6} A finales de junio, Jesús subió por primera vez a la cima del Monte Tabor en compañía de su padre. Era un día claro y la vista era magnífica. Este chico de nueve años tuvo la impresión de que había contemplado realmente el mundo entero, a excepción de la India, África y Roma.

\par 
%\textsuperscript{(1367.3)}
\textsuperscript{124:1.7} Marta, la segunda hermana de Jesús, nació el jueves 13 de septiembre por la noche\footnote{\textit{Marta, la hermana de Jesús}: Mt 13:56; Mc 6:3.}. Tres semanas después del nacimiento de Marta, José, que se encontraba en casa por algún tiempo, empezó la construcción de una ampliación de su casa, una habitación que serviría como taller y dormitorio. Se construyó un pequeño banco de trabajo para Jesús, y por primera vez pudo disponer de sus propias herramientas. Durante muchos años trabajó en este banco en sus ratos libres y se volvió muy experto en la fabricación de yugos.

\par 
%\textsuperscript{(1367.4)}
\textsuperscript{124:1.8} Este invierno y el siguiente fueron los más fríos en Nazaret desde hacía varias décadas. Jesús había visto la nieve en las montañas y varias veces había nevado en Nazaret, aunque sin permanecer mucho tiempo en el suelo; pero hasta este invierno no había visto el hielo. El hecho de que el agua pudiera ser sólida, líquida y gaseosa ---había meditado largamente sobre el vapor que se escapaba del agua hirviendo--- dio al joven mucho que pensar sobre el mundo físico y su constitución; y sin embargo, la personalidad encarnada en este niño en pleno crecimiento era al mismo tiempo la verdadera creadora y organizadora de todas estas cosas en todo un extenso universo.

\par 
%\textsuperscript{(1367.5)}
\textsuperscript{124:1.9} El clima de Nazaret no era riguroso. Enero era el mes más frío, con una temperatura media alrededor de los 10{\textdegree} C. En julio y agosto, los meses más calurosos, la temperatura variaba entre 24{\textdegree} y 32{\textdegree} C. Desde las montañas hasta el Jordán y el valle del Mar Muerto, el clima de Palestina variaba entre el frío y el tórrido. Así pues, en cierto sentido, los judíos estaban preparados para vivir prácticamente en cualquiera de los climas variables del mundo.

\par 
%\textsuperscript{(1367.6)}
\textsuperscript{124:1.10} Incluso durante los meses más calurosos del verano, una brisa fresca del mar soplaba generalmente del oeste desde las 10 de la mañana hasta las 10 de la noche. Pero de vez en cuando, los temibles vientos cálidos procedentes del desierto oriental soplaban en toda Palestina. Estas ráfagas calientes aparecían por lo general en febrero y marzo, hacia el final de la temporada de las lluvias. En esos momentos, la lluvia caía en chaparrones refrescantes desde noviembre hasta abril, pero no llovía de manera continuada. En Palestina sólo había dos estaciones: el verano y el invierno, la temporada seca y la temporada lluviosa. Las flores empezaban a abrir en enero, y a finales de abril todo el país era un vergel florido.

\par 
%\textsuperscript{(1367.7)}
\textsuperscript{124:1.11} En mayo de este año, Jesús ayudó por primera vez a cosechar los cereales en la granja de su tío. Antes de cumplir los trece años, se las había arreglado para saber algo de casi todos los trabajos que realizaban los hombres y las mujeres alrededor de Nazaret, a excepción del trabajo de los metales; cuando fue mayor, después de la muerte de su padre, pasó varios meses en el taller de un herrero.

\par 
%\textsuperscript{(1368.1)}
\textsuperscript{124:1.12} Cuando disminuía el trabajo y el tránsito de las caravanas, Jesús hacía con su padre muchos viajes de placer o de negocios a las ciudades cercanas de Caná, Endor y Naín. Incluso siendo joven había visitado con frecuencia Séforis, situada sólo a cinco kilómetros al noroeste de Nazaret; desde el año 4 a. de J.C. hasta cerca del año 25 d. de J.C., esta ciudad fue la capital de Galilea y una de las residencias de Herodes Antipas.

\par 
%\textsuperscript{(1368.2)}
\textsuperscript{124:1.13} Jesús continuaba su crecimiento físico, intelectual, social y espiritual. Sus viajes fuera del hogar contribuyeron mucho a proporcionarle una comprensión mejor y más generosa de su propia familia; en esta época, sus mismos padres empezaron a aprender de él al mismo tiempo que le enseñaban. Incluso en su juventud, Jesús era un pensador original y un hábil educador. Se encontraba en un conflicto permanente con la llamada <<ley oral>>, pero siempre trataba de adaptarse a las prácticas de su familia. Se llevaba muy bien con los niños de su edad, pero a menudo se desalentaba por su lentitud mental. Antes de cumplir los diez años, se había convertido en el jefe de un grupo de siete muchachos que formaron una sociedad para adquirir los conocimientos de la edad adulta ---físicos, intelectuales y religiosos. Jesús logró introducir entre estos chicos muchos juegos nuevos y diversos métodos mejorados de entretenimiento físico.

\section*{2. El décimo año (año 4 d. de J.C.)}
\par 
%\textsuperscript{(1368.3)}
\textsuperscript{124:2.1} El cinco de julio, el primer sábado del mes, mientras Jesús se paseaba por el campo con su padre, expresó por primera vez unos sentimientos y unas ideas que indicaban que estaba empezando a tomar conciencia de la naturaleza excepcional de su misión en la vida. José escuchó atentamente las importantes palabras de su hijo, pero hizo pocos comentarios y no dio ninguna información. Al día siguiente, Jesús tuvo una conversación similar con su madre, pero más larga. María escuchó igualmente las declaraciones del muchacho, pero ella tampoco proporcionó ninguna información. Pasaron casi dos años antes de que Jesús hablara nuevamente a sus padres de esta revelación creciente, dentro de su propia conciencia, sobre la naturaleza de su personalidad y el carácter de su misión en la Tierra.

\par 
%\textsuperscript{(1368.4)}
\textsuperscript{124:2.2} En agosto ingresó en la escuela superior de la sinagoga. En la escuela, causaba continuas perturbaciones con las preguntas que persistía en hacer. Cada vez tenía más a todo Nazaret en un alboroto más o menos continuo. A sus padres les disgustaba prohibirle que hiciera esas preguntas inquietantes, y su profesor principal estaba muy intrigado por la curiosidad del muchacho, su perspicacia y su sed de conocimientos.

\par 
%\textsuperscript{(1368.5)}
\textsuperscript{124:2.3} Los compañeros de juego de Jesús no veían nada sobrenatural en su conducta; en la mayoría de los aspectos era totalmente como ellos. Su interés por el estudio era un poco superior a la media, pero no tan excepcional. Es verdad que en la escuela hacía más preguntas que los demás niños de su clase.

\par 
%\textsuperscript{(1368.6)}
\textsuperscript{124:2.4} Quizás su característica más excepcional y sobresaliente era su repugnancia a luchar por sus derechos. Aunque era un muchacho bien desarrollado para su edad, a sus compañeros de juego les resultaba extraño que tuviera aversión por defenderse incluso de las injusticias o cuando era sometido a abusos personales. A pesar de todo, no sufrió mucho por culpa de esta tendencia gracias a la amistad de Jacobo, el muchacho vecino, que era un año mayor. Se trataba del hijo del albañil asociado con José en los negocios. Jacobo admiraba mucho a Jesús y se encargaba de estar pendiente para que nadie se le impusiera, aprovechándose de su aversión por las peleas físicas. Varias veces atacaron a Jesús unos jóvenes mayores y violentos, contando con su notoria docilidad, pero siempre recibieron un castigo rápido y seguro de manos de Jacobo, el hijo del albañil, su campeón voluntario y defensor siempre dispuesto.

\par 
%\textsuperscript{(1369.1)}
\textsuperscript{124:2.5} Jesús era el jefe comúnmente aceptado por los muchachos de Nazaret que tenían los ideales más elevados de su tiempo y de su generación. Sus jóvenes amigos lo amaban realmente, no sólo porque era justo, sino también porque poseía una simpatía rara y comprensiva que revelaba el amor y se acercaba a la compasión discreta.

\par 
%\textsuperscript{(1369.2)}
\textsuperscript{124:2.6} Este año empezó a mostrar una marcada preferencia por la compañía de las personas mayores. Le encantaba hablar de temas culturales, educativos, sociales, económicos, políticos y religiosos con pensadores de más edad; la profundidad de sus razonamientos y la fineza de sus observaciones gustaban tanto a sus amigos adultos que siempre estaban más que dispuestos para conversar con él. Hasta que tuvo que hacerse cargo de mantener a la familia, sus padres trataron constantemente de inducirlo a que se asociara con los chicos de su misma edad, o más cercanos a ella, en lugar de personas mayores mejor informadas, por quienes mostraba tanta preferencia.

\par 
%\textsuperscript{(1369.3)}
\textsuperscript{124:2.7} A finales de este año tuvo con su tío una experiencia de dos meses de pesca en el Mar de Galilea, y se le dio muy bien. Antes de llegar a la edad adulta, se había convertido en un experto pescador.

\par 
%\textsuperscript{(1369.4)}
\textsuperscript{124:2.8} Su desarrollo físico continuaba; en la escuela era un alumno avanzado y privilegiado; en el hogar se llevaba francamente bien con sus hermanos y hermanas más jóvenes, contando con la ventaja de tener más de tres años y medio que el mayor de los otros niños. En Nazaret tenían una buena opinión de él, a excepción de los padres de algunos de los niños más torpes, que a menudo decían que Jesús era demasiado engreído, que carecía de la humildad y de la reserva propias de la juventud. Manifestaba una tendencia creciente a orientar las actividades recreativas de sus jóvenes amigos hacia terrenos más serios y reflexivos. Era un instructor nato y sencillamente no podía dejar de actuar como tal, incluso cuando se suponía que estaba jugando.

\par 
%\textsuperscript{(1369.5)}
\textsuperscript{124:2.9} José empezó muy pronto a enseñar a Jesús las diversas maneras de ganarse la vida, explicándole las ventajas de la agricultura sobre la industria y el comercio. Galilea era una comarca más hermosa y próspera que Judea, y vivir allí apenas costaba la cuarta parte de lo que costaba en Jerusalén y Judea. Era una provincia de pueblos agrícolas y de ciudades industriales florecientes, con más de doscientas ciudades por encima de los cinco mil habitantes y treinta con más de quince mil.

\par 
%\textsuperscript{(1369.6)}
\textsuperscript{124:2.10} Durante su primer viaje con su padre para observar la industria pesquera en el lago de Galilea, Jesús casi había decidido hacerse pescador; pero la estrecha relación con el oficio de su padre le impulsó más adelante a hacerse carpintero, mientras que más tarde aún, una combinación de influencias le llevó a escoger definitivamente la carrera de educador religioso de un orden nuevo.

\section*{3. El undécimo año (año 5 d. de J.C.)}
\par 
%\textsuperscript{(1369.7)}
\textsuperscript{124:3.1} Durante todo este año, el muchacho continuó haciendo viajes con su padre fuera del hogar, pero también visitaba con frecuencia la granja de su tío, y en ocasiones iba a Magdala para pescar con el tío que se había instalado cerca de aquella ciudad.

\par 
%\textsuperscript{(1369.8)}
\textsuperscript{124:3.2} José y María a veces estuvieron tentados de mostrar algún tipo de favoritismo especial por Jesús, o de revelar de alguna otra manera su conocimiento de que era un niño de la promesa, un hijo del destino. Pero sus padres eran, los dos, extraordinariamente sabios y sagaces en todos estos asuntos. Las pocas veces que mostraron de alguna manera una preferencia cualquiera por él, incluso en el más ínfimo grado, el muchacho rechazó de inmediato toda consideración especial.

\par 
%\textsuperscript{(1370.1)}
\textsuperscript{124:3.3} Jesús pasaba bastante tiempo en la tienda de abastecimiento de las caravanas; como conversaba con los viajeros de todas las partes del mundo, adquirió una cantidad de información sobre los asuntos internacionales sorprendente para su edad. Éste fue el último año que pudo disfrutar mucho de los juegos y de la alegría juvenil; a partir de este momento, las dificultades y las responsabilidades se multiplicaron rápidamente en la vida de este joven.

\par 
%\textsuperscript{(1370.2)}
\textsuperscript{124:3.4} Judá nació al anochecer del miércoles 24 de junio del año 5 d. de J.C.\footnote{\textit{Judá, el hermano de Jesús}: Mt 13:55; Mc 6:3.} El alumbramiento de este séptimo hijo estuvo acompañado de complicaciones. María estuvo tan enferma durante varias semanas que José se quedó en la casa. Jesús estuvo muy ocupado haciendo recados para su padre y realizando múltiples tareas ocasionadas por la grave enfermedad de su madre. A este joven no le fue posible nunca más volver al comportamiento infantil de sus primeros años. A partir de la enfermedad de su madre ---poco antes de cumplir los once años--- se vio obligado a asumir las responsabilidades de hijo mayor, y a hacer todo esto uno o dos años antes de la fecha en que esta carga hubiera recaído normalmente sobre sus hombros.

\par 
%\textsuperscript{(1370.3)}
\textsuperscript{124:3.5} El chazan pasaba una tarde por semana con Jesús ayudándole a estudiar en profundidad las escrituras hebreas. Le interesaba mucho el progreso de su prometedor alumno, y por eso estaba dispuesto a ayudarlo de muchas maneras. Este pedagogo judío ejerció una gran influencia sobre esta mente en crecimiento, pero nunca pudo comprender por qué Jesús era tan indiferente a todas sus sugerencias sobre la perspectiva de ir a Jerusalén para continuar su educación con los rabinos eruditos.

\par 
%\textsuperscript{(1370.4)}
\textsuperscript{124:3.6} Hacia mediados de mayo, el joven acompañó a su padre en un viaje de negocios a Escitópolis, la principal ciudad griega de la Decápolis, la antigua ciudad hebrea de Bet-seán. Por el camino, José le contó muchas cosas de la antigua historia del rey Saúl, los filisteos y los acontecimientos posteriores de la turbulenta historia de Israel. Jesús se quedó enormemente impresionado por la limpieza y el orden que reinaban en esta ciudad llamada pagana. Se maravilló del teatro al aire libre y admiró el hermoso templo de mármol consagrado a la adoración de los dioses <<paganos>>. A José le inquietó mucho el entusiasmo del joven y trató de contrarrestar estas impresiones favorables alabando la belleza y la grandeza del templo judío de Jerusalén. Desde la colina de Nazaret, Jesús había contemplado a menudo con curiosidad esta magnífica ciudad griega, y había preguntado muchas veces por sus amplias obras públicas y sus edificios adornados, pero su padre siempre había tratado de eludir estas preguntas. Ahora se encontraban cara a cara con las bellezas de esta ciudad gentil, y José ya no podía fingir que ignoraba las preguntas de Jesús.

\par 
%\textsuperscript{(1370.5)}
\textsuperscript{124:3.7} Se dio la circunstancia de que precisamente en aquel momento se estaban celebrando, en el anfiteatro de Escitópolis, los juegos competitivos anuales y las demostraciones públicas de proezas físicas entre las ciudades griegas de la Decápolis. Jesús insistió para que su padre lo llevara a ver los juegos, e insistió tanto que José no se atrevió a negárselo. El joven estaba entusiasmado con los juegos y entró de todo corazón en el espíritu de aquellas demostraciones de desarrollo físico y de habilidad atlética. José se escandalizó indeciblemente al observar el entusiasmo de su hijo mientras contemplaba aquellas exhibiciones de vanagloria <<pagana>>. Después de terminar los juegos, José recibió la mayor sorpresa de su vida cuando oyó a Jesús expresar su aprobación y sugerir que sería bueno que los jóvenes de Nazaret pudieran beneficiarse así de unas sanas actividades físicas al aire libre. José tuvo una larga y seria conversación con Jesús respecto a la naturaleza perversa de tales prácticas, pero supo muy bien que el joven no estaba convencido.

\par 
%\textsuperscript{(1371.1)}
\textsuperscript{124:3.8} La única vez que Jesús vio a su padre enfadado con él fue aquella noche en su habitación de la posada cuando, en el transcurso de su discusión, el chico olvidó los principios del pensamiento judío hasta el punto de sugerir que volvieran a casa y trabajaran a favor de la construcción de un anfiteatro en Nazaret. Cuando José escuchó a su primogénito expresar unos sentimientos tan poco judíos, perdió su calma habitual y, cogiéndolo por los hombros, exclamó encolerizado: <<Hijo mío, que no te oiga nunca más expresar un pensamiento tan perverso en toda tu vida>>. Jesús se quedó sobrecogido ante la manifestación emocional de su padre; nunca había sentido anteriormente el impacto personal de la indignación de su padre, y se quedó pasmado y conmocionado de manera indecible. Se limitó a contestar: <<Muy bien, padre, así lo haré>>. Y mientras vivió su padre, el muchacho no hizo nunca más la más pequeña alusión a los juegos ni a las otras actividades atléticas de los griegos.

\par 
%\textsuperscript{(1371.2)}
\textsuperscript{124:3.9} Más tarde, Jesús vio el anfiteatro griego en Jerusalén y comprendió cuán odiosas eran estas cosas desde el punto de vista judío. Sin embargo, durante toda su vida se esforzó por introducir la idea de un esparcimiento sano en sus planes personales y, en la medida en que lo permitían las costumbres judías, también en el programa posterior de las actividades regulares de sus doce apóstoles.

\par 
%\textsuperscript{(1371.3)}
\textsuperscript{124:3.10} Al final de este undécimo año, Jesús era un joven vigoroso, bien desarrollado, con un moderado sentido del humor, y bastante alegre, pero a partir de este año empezó a pasar cada vez con más frecuencia por períodos peculiares de profunda meditación y de seria contemplación. Se dedicaba mucho a meditar sobre la manera en que iba a cumplir con sus obligaciones familiares y obedecer al mismo tiempo la llamada de su misión para con el mundo; ya había comprendido que su ministerio no debía limitarse a mejorar al pueblo judío.

\section*{4. El duodécimo año (año 6 d. de J.C.)}
\par 
%\textsuperscript{(1371.4)}
\textsuperscript{124:4.1} Éste fue un año memorable en la vida de Jesús. Continuó haciendo progresos en la escuela y nunca se cansaba de estudiar la naturaleza; al mismo tiempo, se dedicaba cada vez más a estudiar los métodos que la gente utilizaba para ganarse la vida. Empezó a trabajar regularmente en el taller familiar de carpintería y se le autorizó para que gestionara su propio salario, un arreglo bastante excepcional en una familia judía. Este año aprendió también la conveniencia de guardar en familia el secreto de estas cosas. Se iba haciendo consciente de la manera en que había causado perturbación en el pueblo, y en adelante se volvió cada vez más discreto, ocultando todo lo que contribuyera a ser considerado como diferente de sus compañeros.

\par 
%\textsuperscript{(1371.5)}
\textsuperscript{124:4.2} Durante todo este año experimentó numerosos períodos de incertidumbre, si no de verdadera duda, en cuanto a la naturaleza de su misión. Su mente humana, que se desarrollaba de manera natural, aún no captaba por completo la realidad de su doble naturaleza. El hecho de tener una sola personalidad hacía difícil que su conciencia reconociera el origen doble de los factores que componían la naturaleza asociada con esta misma personalidad.

\par 
%\textsuperscript{(1371.6)}
\textsuperscript{124:4.3} A partir de este momento logró entenderse mejor con sus hermanos y hermanas. Tenía cada vez más tacto, se mostraba siempre compasivo y considerado por su bienestar y felicidad, y mantuvo buenas relaciones con ellos hasta el principio de su ministerio público. Para ser más explícito, se llevó muy bien con Santiago, Miriam y los dos niños más pequeños, Amós y Rut (que aún no habían nacido). Siempre se llevó bastante bien con Marta. Los disgustos que tuvo en el hogar surgieron principalmente de las fricciones con José y Judá, en particular con éste último.

\par 
%\textsuperscript{(1372.1)}
\textsuperscript{124:4.4} Para José y María fue una experiencia difícil encargarse de criar a un ser que reunía esta combinación sin precedentes de divinidad y de humanidad; merecen que se les reconozca un gran mérito por haber cumplido con tanta fidelidad y con tanto éxito sus responsabilidades parentales. Los padres de Jesús comprendieron cada vez más que había algo sobrehumano en su hijo mayor, pero jamás pudieron soñar ni siquiera un instante que este hijo de la promesa fuera en verdad el creador efectivo de este universo local de cosas y de seres. José y María vivieron y murieron sin enterarse nunca de que su hijo Jesús era realmente el Creador del Universo encarnado en la carne mortal.

\par 
%\textsuperscript{(1372.2)}
\textsuperscript{124:4.5} Este año, Jesús se interesó más que nunca por la música, y continuó enseñando a sus hermanos y hermanas en el hogar. Aproximadamente por esta época, el muchacho se volvió profundamente consciente de la diferencia de puntos de vista entre José y María respecto a la naturaleza de su misión. Meditó mucho sobre la diferencia de opinión de sus padres, y a menudo escuchó sus discusiones cuando ellos creían que estaba profundamente dormido. Se inclinaba cada vez más por el punto de vista de su padre, de manera que su madre estaba destinada a sentirse herida al darse cuenta de que su hijo rechazaba poco a poco sus directrices en las cuestiones relacionadas con la carrera de su vida. A medida que pasaban los años, esta brecha de incomprensión fue incrementándose. María comprendía cada vez menos el significado de la misión de Jesús, y esta madre buena se sintió cada vez más herida porque su hijo favorito no llevaba a cabo sus esperanzas más acariciadas.

\par 
%\textsuperscript{(1372.3)}
\textsuperscript{124:4.6} José creía cada vez más en la naturaleza espiritual de la misión de Jesús; y si no fuera por otras razones más importantes, de hecho es una pena que no viviera lo suficiente como para ver realizarse su concepto de la donación de Jesús en la Tierra.

\par 
%\textsuperscript{(1372.4)}
\textsuperscript{124:4.7} Durante su último año en la escuela, cuando tenía doce años, Jesús manifestó a su padre su protesta por la costumbre hebrea de tocar el trozo de pergamino clavado en el marco de la puerta, cada vez que entraban o salían de la casa, y besar después el dedo que lo había tocado\footnote{\textit{Jesús cuestiona las costumbres}: Dt 6:6-9.}. Como parte de este rito, era costumbre decir: <<El Señor protegerá nuestra entrada y nuestra salida, de ahora en adelante y para siempre>>\footnote{\textit{El Señor protegerá nuestra entrada}: Sal 121:8.}. José y María habían enseñado repetidas veces a Jesús las razones por las cuales estaba prohibido hacer retratos o dibujar cuadros, explicando que estas creaciones se podían utilizar con fines idólatras. Aunque Jesús no llegaba a comprender por completo la prohibición de hacer retratos y dibujos, poseía un elevado concepto de la coherencia, y por eso señaló a su padre la naturaleza esencialmente idólatra de esta reverencia habitual al pergamino de la puerta. Después de estas objeciones de Jesús, José retiró el pergamino.

\par 
%\textsuperscript{(1372.5)}
\textsuperscript{124:4.8} Con el paso del tiempo, Jesús contribuyó mucho a modificar las prácticas religiosas de los suyos, tales como las oraciones familiares y otras costumbres. Muchas de estas cosas se podían hacer en Nazaret porque su sinagoga estaba bajo la influencia de una escuela liberal de rabinos, representada por José, el famoso maestro de Nazaret.

\par 
%\textsuperscript{(1372.6)}
\textsuperscript{124:4.9} Durante este año y los dos siguientes, Jesús sufrió una gran aflicción mental como resultado de sus constantes esfuerzos por conciliar sus opiniones personales sobre las prácticas religiosas y las diversiones sociales, con las creencias enraizadas de sus padres. Estaba angustiado por el conflicto entre la necesidad de ser fiel a sus propias convicciones, y la exhortación de su conciencia a someterse obedientemente a sus padres; su conflicto supremo se encontraba entre dos grandes mandamientos que predominaban en su mente juvenil. El primero era: <<Sé fiel a los dictámenes de tus convicciones más elevadas sobre la verdad y la rectitud>>\footnote{\textit{Sé fiel a tus convicciones}: Sal 15:1-5.}. El otro era: <<Honra a tu padre y a tu madre, porque ellos te han dado la vida y la educación>>\footnote{\textit{Honra a tu padre y a tu madre}: Ex 20:12.}. Sin embargo, nunca eludió la responsabilidad de hacer cada día los ajustes necesarios entre la lealtad a sus convicciones personales y el deber hacia su familia. Consiguió la satisfacción de fundir cada vez más armoniosamente sus convicciones personales con las obligaciones familiares, en un concepto magistral de solidaridad colectiva basada en la lealtad, la justicia, la tolerancia y el amor.

\section*{5. Su decimotercer año (año 7 d. de J.C.)}
\par 
%\textsuperscript{(1373.1)}
\textsuperscript{124:5.1} En este año, el muchacho de Nazaret pasó de la infancia a la adolescencia; su voz empezó a cambiar, y otros rasgos de la mente y del cuerpo revelaron la llegada de la virilidad.

\par 
%\textsuperscript{(1373.2)}
\textsuperscript{124:5.2} Su hermanito Amós nació la noche del domingo 9 de enero del año 7 d. de J.C. Judá no tenía todavía dos años, y su hermanita Rut aún no había nacido. Se puede ver pues que Jesús tenía una numerosa familia de niños pequeños que se quedó a su cuidado cuando su padre encontró la muerte al año siguiente en un accidente.

\par 
%\textsuperscript{(1373.3)}
\textsuperscript{124:5.3} Hacia mediados de febrero, Jesús adquirió humanamente la seguridad de que estaba destinado a efectuar una misión en la Tierra para iluminar al hombre y revelar a Dios. En la mente de este joven se estaban formando importantes decisiones, junto con planes de gran envergadura, mientras que su apariencia exterior era la de un muchacho judío corriente de Nazaret. La vida inteligente de todo Nebadon observaba con fascinación y asombro cómo todo esto empezaba a desarrollarse en el pensamiento y en los actos del hijo, ahora adolescente, del carpintero.

\par 
%\textsuperscript{(1373.4)}
\textsuperscript{124:5.4} El primer día de la semana, el 20 de marzo del año 7, Jesús se graduó en los cursos de enseñanza de la escuela local asociada con la sinagoga de Nazaret. Era un gran día en la vida de cualquier familia judía ambiciosa, el día en que el hijo primogénito era nombrado <<hijo del mandamiento>> y el primogénito rescatado del Señor Dios de Israel, un <<hijo del Altísimo>>\footnote{\textit{Hijo del Altísimo}: Sal 82:6.} y servidor del Señor de toda la Tierra.

\par 
%\textsuperscript{(1373.5)}
\textsuperscript{124:5.5} El viernes de la semana anterior, José había regresado de Séforis, donde estaba encargado de construir un nuevo edificio público, para estar presente en esta feliz ocasión. El profesor de Jesús creía firmemente que su alumno despierto y aplicado estaba destinado a alguna carrera eminente, a alguna misión importante. Los ancianos, a pesar de todos sus disgustos con las tendencias no conformistas de Jesús, estaban muy orgullosos del muchacho y ya habían empezado a hacer planes para que pudiera ir a Jerusalén a continuar su educación en las famosas academias hebreas.

\par 
%\textsuperscript{(1373.6)}
\textsuperscript{124:5.6} A medida que Jesús oía de vez en cuando discutir estos planes, estaba cada vez más seguro de que nunca iría a Jerusalén para estudiar con los rabinos. Sin embargo, poco podía imaginar la tragedia tan próxima que aseguraría el abandono de todos estos proyectos, obligándole a asumir la responsabilidad de mantener y dirigir una familia numerosa que pronto iba a estar compuesta por cinco hermanos y tres hermanas, además de su madre y él mismo. Al tener que criar esta familia, Jesús pasó por una experiencia más extensa y prolongada que la que tuvo José, su padre; y se mantuvo a la altura del modelo que más tarde estableció para sí mismo: ser un educador y hermano mayor sabio, paciente, comprensivo y eficaz para esta familia ---su familia---, tan repentinamente afligida por el dolor y tan inesperadamente acongojada.

\section*{6. El viaje a Jerusalén}
\par 
%\textsuperscript{(1374.1)}
\textsuperscript{124:6.1} Como Jesús había llegado ahora al umbral de la vida adulta y se había graduado oficialmente en las escuelas de la sinagoga, reunía las condiciones necesarias para ir a Jerusalén\footnote{\textit{Viaje a Jerusalén}: Lc 2:42.} con sus padres y participar con ellos en la celebración de su primera Pascua. La fiesta de la Pascua de este año caía el sábado 9 de abril del año 7. Un grupo numeroso (103 personas) se preparó para salir de Nazaret hacia Jerusalén el lunes 4 de abril por la mañana temprano. Viajaron hacia el sur en dirección a Samaria, pero al llegar a Jezreel se desviaron hacia el este, rodeando el Monte Gilboa por el valle del Jordán para evitar tener que cruzar Samaria. A José y a su familia les hubiera gustado atravesar Samaria por la ruta del pozo de Jacob y de Betel, pero como los judíos no querían mezclarse con los samaritanos, decidieron continuar con sus vecinos por el valle del Jordán.

\par 
%\textsuperscript{(1374.2)}
\textsuperscript{124:6.2} El temible Arquelao había sido depuesto, y existía poco peligro en llevar a Jesús a Jerusalén. Habían pasado doce años desde que el primer Herodes había tratado de destruir al niño de Belén, y nadie pensaría ahora en asociar aquel asunto con este muchacho desconocido de Nazaret.

\par 
%\textsuperscript{(1374.3)}
\textsuperscript{124:6.3} Antes de llegar al cruce de Jezreel, prosiguiendo su viaje, muy pronto dejaron a la izquierda el antiguo pueblo de Sunem, y Jesús escuchó de nuevo la historia de la doncella más hermosa de todo Israel que vivió allí en otro tiempo, y también las obras maravillosas que Eliseo había realizado en aquel lugar. Al pasar por Jezreel, los padres de Jesús contaron las acciones de Acab y Jezabel y las hazañas de Jehú. Al pasar cerca del Monte Gilboa, hablaron mucho de Saúl que se suicidó en las vertientes de esta montaña, del rey David, y de los acontecimientos asociados con este lugar histórico.

\par 
%\textsuperscript{(1374.4)}
\textsuperscript{124:6.4} Al rodear la base del Gilboa, los peregrinos podían ver a la derecha la ciudad griega de Escitópolis. Admiraron desde lejos los edificios de mármol, pero no se acercaron a la ciudad gentil por temor a profanarse, lo que les impediría participar en las ceremonias solemnes y sagradas de la Pascua en Jerusalén. María no comprendía por qué ni José ni Jesús querían hablar de Escitópolis. No sabía nada de su controversia del año anterior, porque nunca le habían contado el incidente.

\par 
%\textsuperscript{(1374.5)}
\textsuperscript{124:6.5} Ahora la carretera descendía rápidamente hacia el valle tropical del Jordán, y Jesús pudo pronto contemplar admirado el serpenteante y tortuoso río Jordán, con sus aguas resplandecientes y ondulantes fluyendo hacia el Mar Muerto. Se quitaron los abrigos mientras viajaban hacia el sur por este valle tropical, disfrutando de los fértiles campos de cereales y de las hermosas adelfas cargadas de flores rosadas, mientras que hacia el norte el macizo del Monte Hermón cubierto de nieve se perfilaba a lo lejos, dominando majestuosamente el histórico valle. Poco más de tres horas después de haber pasado Escitópolis, llegaron a una fuente burbujeante y acamparon allí durante la noche bajo el cielo estrellado.

\par 
%\textsuperscript{(1374.6)}
\textsuperscript{124:6.6} En su segundo día de viaje pasaron por el lugar donde el Jaboc, procedente del este, desemboca en el Jordán; al contemplar este valle hacia el este, recordaron los tiempos de Gedeón, cuando los medianitas se extendieron por esta región para invadir el país. Hacia el final del segundo día de viaje, acamparon cerca de la base de la montaña más alta que domina el valle del Jordán, el Monte Sartaba, cuya cima estaba ocupada por la fortaleza alejandrina donde Herodes había encarcelado a una de sus esposas y enterrado a sus dos hijos estrangulados.

\par 
%\textsuperscript{(1375.1)}
\textsuperscript{124:6.7} Al tercer día pasaron por dos pueblos que habían sido construidos recientemente por Herodes y observaron su magnífica arquitectura y sus hermosos jardines de palmeras. Al anochecer llegaron a Jericó, donde permanecieron hasta el día siguiente. Aquella noche, José, María y Jesús caminaron unos dos kilómetros y medio hasta el emplazamiento del antiguo Jericó, donde según la tradición judía, Josué, de quien Jesús había tomado el nombre, había realizado sus famosas hazañas.

\par 
%\textsuperscript{(1375.2)}
\textsuperscript{124:6.8} Durante el cuarto y último día de viaje, la carretera era una procesión continua de peregrinos. Ahora empezaron a subir las colinas que conducían a Jerusalén. Al acercarse a la cumbre, pudieron ver las montañas al otro lado del Jordán, y hacia el sur, las aguas perezosas del Mar Muerto. Aproximadamente a mitad de camino de Jerusalén, Jesús vio por primera vez el Monte de los Olivos (la región que jugaría un papel tan importante en su vida futura). José le indicó que la Ciudad Santa estaba situada justo detrás de aquellas lomas, y el corazón del muchacho se aceleró ante la feliz expectativa de contemplar pronto la ciudad y la casa de su Padre celestial.

\par 
%\textsuperscript{(1375.3)}
\textsuperscript{124:6.9} Se detuvieron para descansar en las pendientes orientales del Olivete, junto a un pueblecito llamado Betania. Los lugareños hospitalarios salieron enseguida para atender a los peregrinos, y dio la casualidad de que José y su familia se habían detenido cerca de la casa de un tal Simón, que tenía tres hijos casi de la misma edad que Jesús ---María, Marta y Lázaro. Éstos invitaron a la familia de Nazaret a que entraran a descansar, y entre las dos familias nació una amistad que duró toda la vida. Más adelante, en el transcurso de su vida llena de acontecimientos, Jesús se detuvo muchas veces en esta casa.

\par 
%\textsuperscript{(1375.4)}
\textsuperscript{124:6.10} Se apresuraron en continuar su camino, y pronto llegaron al borde del Olivete; Jesús vio por primera vez (en su memoria) la Ciudad Santa, los palacios pretenciosos y el templo inspirador de su Padre. Jesús no experimentó nunca más en su vida un estremecimiento puramente humano comparable al que le embargó por completo esta tarde de abril, en el Monte de los Olivos, mientras estaba allí de pie bebiendo con su primera mirada a Jerusalén. Unos años más tarde estuvo en este mismo lugar, y lloró por la ciudad que estaba a punto de rechazar a otro profeta, al último y al más grande de sus educadores celestiales.

\par 
%\textsuperscript{(1375.5)}
\textsuperscript{124:6.11} Se dieron prisa por llegar a Jerusalén. Ahora era jueves por la tarde. Al llegar a la ciudad pasaron por delante del templo, y Jesús no había visto nunca una multitud así de seres humanos. Meditó profundamente sobre cómo estos judíos se habían reunido aquí desde los lugares más distantes del mundo conocido.

\par 
%\textsuperscript{(1375.6)}
\textsuperscript{124:6.12} Poco después llegaron al lugar previsto donde se alojarían durante la semana pascual, la amplia casa de un pariente rico de María, que sabía por Zacarías algo de la historia anterior de Juan y de Jesús. Al día siguiente, el día de la preparación, se dispusieron a celebrar convenientemente el sábado de la Pascua.

\par 
%\textsuperscript{(1375.7)}
\textsuperscript{124:6.13} Aunque todo Jerusalén estaba ocupado con las preparaciones de la Pascua, José encontró tiempo para llevar a su hijo a visitar la academia donde se había convenido que proseguiría su educación dos años más tarde, en cuanto cumpliera la edad requerida de quince años. José estaba realmente perplejo al observar el poco interés de Jesús por todos estos planes cuidadosamente elaborados.

\par 
%\textsuperscript{(1375.8)}
\textsuperscript{124:6.14} Jesús estaba profundamente impresionado por el templo y todos sus servicios y demás actividades asociadas. Por primera vez desde la edad de cuatro años, estaba demasiado preocupado por sus propias meditaciones como para hacer muchas preguntas. Sin embargo, hizo varias preguntas embarazosas a su padre (como ya había hecho en otras ocasiones) sobre por qué razón el Padre celestial exigía la carnicería de tantos animales inocentes e indefensos. Por la expresión del rostro del muchacho, su padre sabía bien que sus respuestas y sus tentativas de explicación no eran satisfactorias para la profundidad de pensamiento y la agudeza de razonamiento de su hijo.

\par 
%\textsuperscript{(1376.1)}
\textsuperscript{124:6.15} El día anterior al sábado de la Pascua, una oleada de iluminación espiritual atravesó la mente mortal de Jesús e inundó su corazón humano de piedad afectuosa por las multitudes espiritualmente ciegas y moralmente ignorantes, reunidas para celebrar la antigua conmemoración de la Pascua. Éste fue uno de los días más extraordinarios que el Hijo de Dios vivió en la carne; y durante la noche, por primera vez en su carrera terrestre, un mensajero especial de Salvington, enviado por Emmanuel, apareció ante él y le dijo: <<Ha llegado la hora. Ya es tiempo de que empieces a ocuparte de los asuntos de tu Padre>>\footnote{\textit{Ha llegado la hora}: Lc 2:49.}.

\par 
%\textsuperscript{(1376.2)}
\textsuperscript{124:6.16} Y así, incluso antes de que las pesadas responsabilidades de la familia de Nazaret recayeran sobre sus hombros juveniles, llegaba el mensajero celestial para recordar a este muchacho menor de trece años que había llegado la hora de reasumir las responsabilidades de un universo. Éste fue el primer acto de una larga serie de acontecimientos que culminaron finalmente en la terminación de la donación del Hijo en Urantia y en la restitución del <<gobierno de un universo sobre sus hombros humano-divinos>>\footnote{\textit{Gobierno sobre sus hombros}: Is 9:6.}.

\par 
%\textsuperscript{(1376.3)}
\textsuperscript{124:6.17} A medida que pasaba el tiempo, el misterio de la encarnación se volvía cada vez más insondable para todos nosotros. Apenas podíamos comprender que este muchacho de Nazaret fuera el creador de todo Nebadon. Y tampoco entendemos en la actualidad cómo están asociados el espíritu de este mismo Hijo Creador y el espíritu de su Padre Paradisiaco con las almas de la humanidad. Con el paso del tiempo, podíamos observar que su mente humana discernía cada vez mejor que, mientras estaba viviendo su vida en la carne, la responsabilidad de un universo reposaba en espíritu sobre sus hombros.

\par 
%\textsuperscript{(1376.4)}
\textsuperscript{124:6.18} Así termina la carrera del muchacho de Nazaret y comienza el relato del joven adolescente ---el hombre divino cada vez más consciente de sí mismo--- que empieza ahora a considerar su carrera en el mundo, mientras se esfuerza por integrar su proyecto de vida en desarrollo con los deseos de sus padres y las obligaciones hacia su familia y la sociedad de su tiempo.


\chapter{Documento 125. Jesús en Jerusalén}
\par 
%\textsuperscript{(1377.1)}
\textsuperscript{125:0.1} DE toda la extraordinaria carrera terrestre de Jesús, ningún acontecimiento fue más atractivo, más humanamente conmovedor, que esta visita a Jerusalén, la primera que recordaba. La experiencia de asistir solo a las discusiones del templo le resultó particularmente estimulante, y se grabó durante mucho tiempo en su memoria como el acontecimiento más importante del final de su infancia y del principio de su juventud. Ésta fue la primera oportunidad que tuvo de disfrutar de unos pocos días de vida independiente, de la alegría de ir y venir sin sujeción ni restricciones. Este breve período viviendo a su aire, durante la semana siguiente a la Pascua, fue el primero totalmente libre de obligaciones que había disfrutado nunca. Pasaron muchos años antes de que volviera a disponer, aunque fuera por poco tiempo, de un período semejante libre de todo sentido de la responsabilidad.

\par 
%\textsuperscript{(1377.2)}
\textsuperscript{125:0.2} Las mujeres asistían rara vez a la fiesta de la Pascua en Jerusalén, porque no se requería su presencia. Sin embargo, Jesús se negó prácticamente a partir a menos que su madre los acompañara. Cuando ella se decidió a ir, muchas mujeres de Nazaret se sintieron motivadas para hacer el viaje, de manera que la expedición pascual contenía, en proporción con los hombres, el mayor número de mujeres que había salido nunca de Nazaret para la Pascua. En el camino de Jerusalén, los viajeros cantaron de vez en cuando el Salmo ciento treinta.

\par 
%\textsuperscript{(1377.3)}
\textsuperscript{125:0.3} Desde el momento en que salieron de Nazaret hasta que llegaron a la cima del Monte de los Olivos, Jesús experimentó todo el tiempo la tensión de la expectativa. Durante toda su alegre infancia, había oído hablar con respeto de Jerusalén y de su templo; ahora iba pronto a contemplarlos en la realidad. Visto desde el Monte de los Olivos, y al observarlo más de cerca desde el exterior, el templo había colmado con creces lo que Jesús esperaba; pero una vez que traspasó las puertas sagradas, la gran desilusión empezó.

\par 
%\textsuperscript{(1377.4)}
\textsuperscript{125:0.4} En compañía de sus padres, Jesús atravesó los recintos del templo para reunirse con el grupo de los nuevos hijos de la ley que estaban a punto de ser consagrados como ciudadanos de Israel. Se sintió un poco decepcionado por el comportamiento general de la gente en el templo, pero la primera gran conmoción del día se produjo cuando su madre los dejó para dirigirse a la galería de las mujeres. A Jesús nunca se le había ocurrido que su madre no lo acompañaría a las ceremonias de la consagración, y estaba completamente indignado porque ella tuviera que soportar una discriminación tan injusta. Estaba enormemente enfadado por esto, pero aparte de unas palabras de protesta a su padre, no dijo nada. Sin embargo reflexionó, y reflexionó profundamente, como lo demostraron sus preguntas a los escribas y educadores una semana después.

\par 
%\textsuperscript{(1377.5)}
\textsuperscript{125:0.5} Participó en los rituales de la consagración, pero le decepcionó su naturaleza superficial y rutinaria. Echaba de menos aquel interés personal que caracterizaba a las ceremonias de la sinagoga de Nazaret. A continuación regresó para saludar a su madre, y se preparó para acompañar a su padre en su primer recorrido por el templo y sus patios, galerías y corredores diversos. Los recintos del templo podían contener más de doscientos mil creyentes a la vez, y aunque la enormidad de estos edificios ---en comparación con otros que hubiera visto antes--- le causó una gran impresión, estaba más interesado en meditar sobre el significado espiritual de las ceremonias del templo y del culto asociado a las mismas.

\par 
%\textsuperscript{(1378.1)}
\textsuperscript{125:0.6} Aunque muchos rituales del templo impresionaron vivamente su sentido de la belleza y de lo simbólico, continuaban decepcionándole las explicaciones que sus padres le ofrecían sobre el significado real de estas ceremonias, en respuesta a sus múltiples preguntas penetrantes. Jesús simplemente no podía aceptar unas explicaciones sobre el culto y la devoción religiosa, basadas en la creencia en la ira de Dios o en la cólera del Todopoderoso. Después de terminar la visita del templo, continuaron discutiendo estas cuestiones y su padre le insistía suavemente para que aceptara las creencias ortodoxas judías; Jesús se volvió repentinamente hacia sus padres y, mirando a los ojos de su padre de manera suplicante, le dijo: <<Padre, no puede ser verdad ---el Padre que está en los cielos no puede mirar de ese modo a sus hijos desviados de la Tierra. El Padre celestial no puede amar a sus hijos menos de lo que tú me amas. Por muy imprudentes que sean mis actos, sé muy bien que nunca derramarías tu ira sobre mi, ni descargarías tu cólera contra mi. Si tú, mi padre terrenal, posees esos reflejos humanos de lo Divino, cuánto más el Padre celestial deberá estar lleno de bondad y rebosante de misericordia. Me niego a creer que mi Padre celestial me ame menos que mi padre terrenal>>.

\par 
%\textsuperscript{(1378.2)}
\textsuperscript{125:0.7} Cuando José y María oyeron estas palabras de su hijo primogénito, se quedaron en silencio. Nunca más trataron de cambiar sus ideas sobre el amor de Dios y la misericordia del Padre que está en los cielos.

\section*{1. Jesús visita el templo}
\par 
%\textsuperscript{(1378.3)}
\textsuperscript{125:1.1} A Jesús le disgustó y le repugnó el espíritu de irreverencia que observó en todos los patios del templo que recorrió. Estimaba que la conducta de las multitudes en el templo no era consecuente con el hecho de estar presentes en <<la casa de su Padre>>\footnote{\textit{La casa de su Padre}: Is 56:7; Mt 21:12-13; Mc 11:16-17; Lc 19:45-46; Jn 2:14-16; 14:2-13.}. Pero recibió el mayor golpe de su joven vida cuando su padre lo acompañó al patio de los gentiles, donde la jerga ruidosa, las voces y las maldiciones se mezclaban indiscriminadamente con el balido de las ovejas y la cháchara ruidosa que revelaba la presencia de los cambistas y de los vendedores de animales para los sacrificios y otras mercancías diversas.

\par 
%\textsuperscript{(1378.4)}
\textsuperscript{125:1.2} Pero por encima de todo, su sentido de lo adecuado se vio ultrajado al observar a las frívolas cortesanas que se pavoneaban por este recinto del templo, iguales a las mujeres repintadas que había visto tan recientemente en una visita a Séforis. Esta profanación del templo suscitó toda su indignación juvenil y no titubeó en expresárselo claramente a José.

\par 
%\textsuperscript{(1378.5)}
\textsuperscript{125:1.3} Jesús admiraba la atmósfera y el servicio del templo, pero le disgustaba la fealdad espiritual que observaba en el rostro de tantos adoradores irreflexivos.

\par 
%\textsuperscript{(1378.6)}
\textsuperscript{125:1.4} A continuación descendieron al patio de los sacerdotes, bajo el borde rocoso delante del templo, donde estaba el altar, para observar la matanza de los rebaños de animales y las abluciones en la fuente de bronce para lavar la sangre de las manos de los sacerdotes que oficiaban la masacre. El pavimento manchado de sangre, las manos ensangrentadas de los sacerdotes y el gemido de los animales agonizantes sobrepasaron lo que podía soportar este muchacho amante de la naturaleza. El terrible espectáculo descompuso a este joven de Nazaret; se agarró al brazo de su padre y le rogó que lo sacara de allí. Regresaron atravesando el patio de los gentiles; incluso las risas groseras y las bromas profanas que escuchó allí fueron un alivio después de lo que acababa de presenciar.

\par 
%\textsuperscript{(1379.1)}
\textsuperscript{125:1.5} José vio cuánto habían afectado a su hijo los ritos del templo y lo llevó sabiamente a ver <<la hermosa puerta>>\footnote{\textit{La Puerta Hermosa}: Hch 3:2,10.}, la puerta artística hecha con bronce corintio. Pero Jesús ya había visto bastante para esta primera visita al templo. Regresaron al patio superior en busca de María y caminaron durante una hora al aire libre, lejos del gentío, mirando el palacio Asmoneo, la residencia imponente de Herodes y la torre de los guardias romanos. Durante este paseo, José explicó a Jesús que sólo los vecinos de Jerusalén tenían permiso para asistir a los sacrificios diarios del templo, y que los habitantes de Galilea sólo venían al templo tres veces al año para participar en el culto: en la Pascua, en la fiesta de Pentecostés (siete semanas después de la Pascua) y en la fiesta de los tabernáculos en octubre. Estas fiestas habían sido establecidas por Moisés. Analizaron a continuación las dos últimas fiestas establecidas, la de la dedicación\footnote{\textit{Fiesta de la dedicación}: 1 Mac 4:52-59.} y la de Purim\footnote{\textit{Fiesta de Purim}: Est 9:17-32.}. Después regresaron a su alojamiento y se prepararon para celebrar la Pascua.

\section*{2. Jesús y la Pascua}
\par 
%\textsuperscript{(1379.2)}
\textsuperscript{125:2.1} Cinco familias de Nazaret habían sido invitadas por la familia de Simón de Betania, o se unieron a ella, para celebrar la Pascua. Simón había comprado el cordero pascual para todo el grupo. La masacre de un número tan enorme de estos corderos es lo que había afectado tanto a Jesús en su visita al templo. Habían planeado comer la Pascua con los parientes de María, pero Jesús persuadió a sus padres para que aceptaran la invitación de ir a Betania.

\par 
%\textsuperscript{(1379.3)}
\textsuperscript{125:2.2} Aquella noche se reunieron para los ritos de la Pascua, comiendo la carne asada con el pan ázimo y las hierbas amargas. Como Jesús era un nuevo hijo de la alianza, se le pidió que contara el origen de la Pascua, y lo hizo muy bien, pero desconcertó un poco a sus padres con la inclusión de numerosos comentarios que reflejaban moderadamente las impresiones que habían hecho en su mente joven, pero reflexiva, las cosas que había visto y oído tan recientemente. Éste fue el comienzo de los siete días de ceremonias de la fiesta pascual.

\par 
%\textsuperscript{(1379.4)}
\textsuperscript{125:2.3} Incluso en esta fecha temprana, y aunque no dijo nada a sus padres sobre este asunto, Jesús había empezado a darle vueltas en la cabeza a la idea de si sería adecuado celebrar la Pascua sin sacrificar el cordero. Estaba mentalmente seguro de que este espectáculo de la ofrenda de los sacrificios no complacía al Padre celestial y, con el paso de los años, estuvo cada vez más resuelto a establecer algún día la celebración de una Pascua sin derramamiento de sangre.

\par 
%\textsuperscript{(1379.5)}
\textsuperscript{125:2.4} Jesús durmió muy poco aquella noche. Su descanso estuvo enormemente alterado con pesadillas de matanzas y sufrimientos. Tenía la mente aturdida y el corazón desgarrado por las inconsistencias y el carácter absurdo de la teología de todo el sistema ceremonial judío. Sus padres durmieron poco también. Estaban muy desconcertados por los acontecimientos del día que acababa de terminar. Tenían el corazón completamente trastornado por la actitud del muchacho, que les parecía extraña y decidida. María experimentó una agitación nerviosa durante la primera parte de la noche, pero José permaneció tranquilo, aunque también estaba perplejo. Los dos temían hablar francamente con el joven de estos problemas, aunque Jesús hubiera conversado gustosamente con sus padres si se hubieran atrevido a estimularlo.

\par 
%\textsuperscript{(1379.6)}
\textsuperscript{125:2.5} Los oficios del día siguiente en el templo fueron más aceptables para Jesús y contribuyeron mucho a mitigar los recuerdos desagradables del día anterior. A la mañana siguiente, el joven Lázaro se hizo cargo de Jesús y empezaron a explorar sistemáticamente Jerusalén y sus alrededores. Antes de terminar el día, Jesús había descubierto los diversos lugares alrededor del templo donde se daban conferencias de enseñanza y respondían a las preguntas de los asistentes; aparte de algunas visitas al santo de los santos, donde se preguntaba maravillado qué había realmente detrás del velo de separación, la mayor parte del tiempo la pasó alrededor del templo en las conferencias de enseñanza.

\par 
%\textsuperscript{(1380.1)}
\textsuperscript{125:2.6} Durante toda la semana de la Pascua, Jesús ocupó su lugar entre los nuevos hijos del mandamiento; esto significaba que tenía que sentarse fuera de la barrera que separaba a todas las personas que no tenían la plena ciudadanía de Israel. Como se le recordaba de esta manera lo joven que era, se contuvo y no hizo todas las preguntas que se amontonaron en su mente; al menos se contuvo hasta que terminó la celebración de la Pascua y se levantaron las restricciones que se habían impuesto a los jóvenes recién consagrados.

\par 
%\textsuperscript{(1380.2)}
\textsuperscript{125:2.7} El miércoles de la semana de la Pascua, Jesús fue autorizado a ir a casa de Lázaro para pasar la noche en Betania. Aquella noche, Lázaro, Marta y María escucharon a Jesús disertar sobre las cosas temporales y eternas, humanas y divinas, y desde aquella noche los tres lo amaron como si hubiera sido su propio hermano.

\par 
%\textsuperscript{(1380.3)}
\textsuperscript{125:2.8} Al final de la semana, Jesús vio menos a Lázaro porque éste ni siquiera podía entrar en el círculo exterior de las discusiones del templo, aunque asistió a algunos discursos públicos que se pronunciaron en los patios exteriores. Lázaro tenía la misma edad que Jesús, pero en Jerusalén, los jóvenes eran admitidos raramente a la consagración de los hijos de la ley antes de que cumplieran los trece años de edad.

\par 
%\textsuperscript{(1380.4)}
\textsuperscript{125:2.9} Durante la semana de la Pascua, los padres de Jesús encontraron repetidas veces a su hijo sentado a solas y profundamente pensativo, con su joven cabeza entre las manos. Nunca lo habían visto comportarse de esta manera y estaban dolorosamente perplejos, sin saber hasta qué punto la confusión reinaba en su mente y la perturbación en su espíritu, a causa de la experiencia que estaba atravesando; no sabían qué hacer. Se alegraban de que terminara la semana de la Pascua y deseaban ver a su hijo, que actuaba de manera extraña, felizmente de regreso en Nazaret.

\par 
%\textsuperscript{(1380.5)}
\textsuperscript{125:2.10} Día tras día, Jesús volvía a pensar en todos sus problemas. Al final de la semana ya había efectuado muchos ajustes; pero cuando llegó la hora de regresar a Nazaret, su joven mente aún hervía de perplejidad y estaba acosada por un montón de preguntas sin respuestas y de problemas sin resolver.

\par 
%\textsuperscript{(1380.6)}
\textsuperscript{125:2.11} Antes de que José y María partieran de Jerusalén, tomaron las medidas oportunas, en compañía del maestro de Jesús en Nazaret, para que Jesús regresara a Jerusalén cuando cumpliera los quince años, a fin de empezar un largo ciclo de estudios en una de las academias rabínicas más famosas. Jesús acompañó a sus padres y a su profesor en sus visitas a la escuela, pero los tres se entristecieron al observar la indiferencia que aparentaba ante todo lo que hacían y decían. María estaba profundamente apenada por sus reacciones a la visita a Jerusalén, y José enormemente perplejo por los extraños comentarios y la conducta insólita del muchacho.

\par 
%\textsuperscript{(1380.7)}
\textsuperscript{125:2.12} Después de todo, la semana de la Pascua había sido un gran acontecimiento en la vida de Jesús. Había disfrutado de la oportunidad de conocer a decenas de muchachos de su misma edad, candidatos como él a la consagración, y utilizó estos contactos como medio para enterarse de cómo vivía la gente en Mesopotamia, Turquestán y Partia, así como en las provincias más occidentales de Roma. Ya conocía bastante bien cómo se desarrollaba la vida de los jóvenes de Egipto y de otras regiones cercanas a Palestina. En aquel momento había miles de jóvenes en Jerusalén, y el muchacho de Nazaret conoció personalmente y entrevistó de manera más o menos extensa a más de ciento cincuenta. Estaba particularmente interesado por los que venían de Extremo Oriente y de los países lejanos de Occidente. Como resultado de estos intercambios, el joven empezó a sentir el deseo de viajar por el mundo con objeto de aprender cómo trabajaban los diversos grupos de sus contemporáneos para ganarse la vida.

\section*{3. La partida de José y María}
\par 
%\textsuperscript{(1381.1)}
\textsuperscript{125:3.1} El grupo de Nazaret había acordado reunirse cerca del templo, a media mañana del primer día de la semana después de terminar la fiesta pascual. Así lo hicieron y emprendieron su viaje de regreso a Nazaret. Jesús había entrado en el templo para escuchar los debates, mientras sus padres aguardaban la llegada de sus compañeros de viaje. La compañía se dispuso a partir enseguida, con los hombres formando un grupo y las mujeres otro, como tenían la costumbre de hacer en sus viajes de ida y vuelta a las fiestas de Jerusalén. Jesús había venido a Jerusalén en compañía de su madre y de las mujeres. Pero ahora, como era un joven consagrado, se suponía que haría el viaje de vuelta a Nazaret con su padre y los hombres. Mientras el grupo de Nazaret partía hacia Betania, Jesús se había quedado en el templo completamente absorto en una discusión sobre los ángeles, totalmente inconsciente de que había pasado la hora de la partida de sus padres. No se dio cuenta de que se había quedado atrás hasta el mediodía, hora en que se suspendían las conferencias del templo\footnote{\textit{Los padres marchan sin Jesús}: Lc 2:43.}.

\par 
%\textsuperscript{(1381.2)}
\textsuperscript{125:3.2} Los viajeros de Nazaret no se dieron cuenta de la ausencia de Jesús porque María suponía que viajaba con los hombres, mientras que José pensaba que iba con las mujeres, puesto que había ido a Jerusalén con las mujeres, conduciendo el asno de María. No descubrieron su ausencia hasta que llegaron a Jericó y se prepararon para pasar la noche\footnote{\textit{Los padres descubren que Jesús no está}: Lc 2:44.}. Después de preguntar a los rezagados del grupo que iban llegando a Jericó, y de haberse enterado que ninguno de ellos había visto a su hijo, pasaron la noche en blanco, haciendo conjeturas sobre qué podría haberle ocurrido, mencionando muchas de sus reacciones insólitas ante los acontecimientos de la semana pascual, y regañándose suavemente el uno al otro por no haberse asegurado de que estaba en el grupo antes de salir de Jerusalén.

\section*{4. El primer y segundo día en el templo}
\par 
%\textsuperscript{(1381.3)}
\textsuperscript{125:4.1} Mientras tanto, Jesús había permanecido en el templo durante toda la tarde, escuchando las discusiones y disfrutando de un ambiente más tranquilo y decoroso, puesto que las grandes multitudes de la semana pascual casi habían desaparecido. Al concluir las discusiones de la tarde, en las cuales no participó, Jesús se dirigió a Betania, donde llegó en el preciso momento en que la familia de Simón se disponía a cenar. A los tres jóvenes les encantó acoger a Jesús, que pasó la noche en casa de Simón. Los vio muy poco durante la velada, pasando la mayor parte del tiempo meditando a solas en el jardín.

\par 
%\textsuperscript{(1381.4)}
\textsuperscript{125:4.2} Al día siguiente, Jesús se levantó temprano y se encaminó hacia el templo. Se detuvo en la cima del Olivete y lloró por el espectáculo que contemplaban sus ojos ---el de un pueblo espiritualmente empobrecido, encadenado por las tradiciones y viviendo vigilado por las legiones romanas. Por la mañana temprano ya se encontraba en el templo, decidido a participar en los debates. Mientras tanto, José y María también se habían levantado al amanecer con la intención de desandar el camino hasta Jerusalén. Primero se dirigieron apresuradamente a la casa de sus parientes donde se habían alojado en familia durante la semana pascual, pero sus indagaciones revelaron que nadie había visto a Jesús. Después de buscarlo todo el día sin encontrar su rastro, regresaron a casa de sus parientes para pasar la noche\footnote{\textit{Los padres regresan y le buscan}: Lc 2:45.}.

\par 
%\textsuperscript{(1382.1)}
\textsuperscript{125:4.3} En la segunda conferencia, Jesús se había atrevido a hacer preguntas y participó en las discusiones del templo de una manera sorprendente, aunque siempre compatible con su juventud. A veces, sus preguntas incisivas ponían un poco en aprietos a los maestros eruditos de la ley judía, pero mostraba tal espíritu de cándida honradez, unido a una sed evidente de aprender, que la mayoría de los maestros del templo estaban dispuestos a tratarle con consideración. Pero cuando se atrevió a poner en duda que fuera justo condenar a muerte a un gentil embriagado que se había extraviado fuera del patio de los gentiles, penetrando inadvertidamente en los recintos prohibidos supuestamente sagrados del templo, uno de los maestros más intolerantes se impacientó por las críticas implícitas del muchacho, lo miró con el ceño fruncido y le preguntó cuántos años tenía. Jesús replicó: <<Me faltan poco más de cuatro meses para cumplir los trece años>>. <<Entonces>>, añadió el maestro ahora encolerizado, <<¿por qué estás aquí, si no tienes edad para ser un hijo de la ley?>> Cuando Jesús explicó que había sido consagrado durante la Pascua y que era un estudiante graduado de las escuelas de Nazaret, los maestros replicaron al unísono, con aire burlón: <<Deberíamos haberlo sabido; es de Nazaret>>. Pero el presidente afirmó que Jesús no tenía la culpa de que los dirigentes de la sinagoga de Nazaret lo hubieran graduado formalmente a los doce años, en lugar de a los trece; aunque algunos de sus detractores se levantaron y se fueron, se decidió que el muchacho podía continuar tranquilamente como alumno en las discusiones del templo.

\par 
%\textsuperscript{(1382.2)}
\textsuperscript{125:4.4} Cuando terminó esta segunda jornada en el templo, Jesús fue otra vez a Betania para pasar la noche. Y salió de nuevo al jardín para meditar y orar. Era evidente que su mente estaba ocupada en la meditación de problemas importantes.

\section*{5. El tercer día en el templo}
\par 
%\textsuperscript{(1382.3)}
\textsuperscript{125:5.1} Durante el tercer día de Jesús en el templo\footnote{\textit{El tercer día en el templo}: Lc 2:46.} con los escribas y maestros, se congregaron numerosos espectadores que habían oído hablar de este joven de Galilea, para disfrutar de la experiencia de ver a un muchacho confundir a los sabios de la ley. Simón también vino desde Betania para observar lo que hacía el muchacho. Durante toda la jornada, José y María continuaron buscando ansiosamente a Jesús e incluso entraron varias veces en el templo, pero nunca se les ocurrió escudriñar los diversos grupos de discusión, aunque en una ocasión se encontraron casi al alcance de su voz fascinante.

\par 
%\textsuperscript{(1382.4)}
\textsuperscript{125:5.2} Antes de terminar el día, toda la atención del principal grupo de debate del templo se había concentrado en las preguntas de Jesús\footnote{\textit{Provocando preguntas}: Lc 2:47.}. Entre sus muchas preguntas se encontraban las siguientes:

\par 
%\textsuperscript{(1382.5)}
\textsuperscript{125:5.3} 1. ¿Qué hay realmente en el santo de los santos, detrás del velo?

\par 
%\textsuperscript{(1382.6)}
\textsuperscript{125:5.4} 2. ¿Por qué las madres de Israel deben estar separadas de los creyentes varones en el templo?

\par 
%\textsuperscript{(1382.7)}
\textsuperscript{125:5.5} 3. Si Dios es un padre que ama a sus hijos, ¿por qué toda esta carnicería de animales para obtener el favor divino? ¿Se ha interpretado erróneamente la enseñanza de Moisés?

\par 
%\textsuperscript{(1382.8)}
\textsuperscript{125:5.6} 4. Puesto que el templo está consagrado al culto del Padre celestial, ¿no es incongruente tolerar la presencia de aquellos que se dedican al trueque y al comercio mundanos?

\par 
%\textsuperscript{(1382.9)}
\textsuperscript{125:5.7} 5. ¿Será el Mesías esperado un príncipe temporal que ocupará el trono de David, o actuará como la luz de la vida en el establecimiento de un reino espiritual?

\par 
%\textsuperscript{(1383.1)}
\textsuperscript{125:5.8} A lo largo de todo el día, los espectadores se maravillaron con estas preguntas\footnote{\textit{Maravillados con su inteligencia}: Lc 2:47.}, pero ninguno estaba más asombrado que Simón. Durante más de cuatro horas, este joven de Nazaret acosó a aquellos maestros judíos con preguntas que daban que pensar y sondeaban el corazón. Hizo pocos comentarios a las observaciones de sus mayores. Trasmitía sus enseñanzas con las preguntas que hacía. Por medio del planteamiento hábil y sutil de sus preguntas, conseguía simultáneamente desafiar sus enseñanzas y sugerir las suyas propias. En su manera de preguntar combinaba con tal encanto la sagacidad y el humor, que se hacía amar incluso por aquellos que se indignaban más o menos por su juventud. Siempre era totalmente honrado y considerado cuando efectuaba estas preguntas penetrantes. Durante esta tarde memorable en el templo, mostró su reticencia característica, confirmada en todo su ministerio público posterior, a sacar ventaja desleal de un adversario. Como adolescente, y más tarde como hombre, parecía estar completamente libre de todo deseo egoísta de ganar una discusión simplemente por el placer de triunfar sobre sus compañeros por medio de la lógica. Una sola cosa le interesaba de manera suprema: proclamar la verdad eterna y efectuar así una revelación más completa del Dios eterno.

\par 
%\textsuperscript{(1383.2)}
\textsuperscript{125:5.9} Cuando terminó el día, Simón y Jesús regresaron a Betania. Durante la mayor parte del camino, el hombre y el niño guardaron silencio. Jesús se detuvo de nuevo en la cima del Olivete, pero al contemplar la ciudad y su templo no lloró; solamente inclinó la cabeza en un gesto de devoción silenciosa.

\par 
%\textsuperscript{(1383.3)}
\textsuperscript{125:5.10} Después de la cena en Betania, rehusó una vez más unirse a la alegre reunión; en lugar de eso, salió al jardín, donde permaneció hasta altas horas de la noche. Se esforzó inútilmente en elaborar un plan definido para abordar el problema de su misión en la vida, y para escoger la mejor manera de trabajar para revelar, a sus compatriotas espiritualmente ciegos, un concepto más hermoso del Padre celestial, y liberarlos así de su terrible esclavitud a la ley, a los ritos, a las ceremonias y a las tradiciones arcaicas. Pero la luz esclarecedora no se le presentó a este joven que buscaba la verdad.

\section*{6. El cuarto día en el templo}
\par 
%\textsuperscript{(1383.4)}
\textsuperscript{125:6.1} Jesús se había olvidado, extrañamente, de sus padres terrenales. Incluso en el desayuno, cuando la madre de Lázaro comentó que sus padres debían estar llegando ahora al hogar, Jesús no pareció darse cuenta de que estarían un poco preocupados porque él se había quedado atrás.

\par 
%\textsuperscript{(1383.5)}
\textsuperscript{125:6.2} De nuevo se dirigió hacia el templo, pero no se detuvo en la cima del Olivete para meditar. Durante las discusiones de la mañana, dedicaron mucho tiempo a la ley y a los profetas, y los maestros se asombraron de que Jesús conociera tan bien las escrituras, tanto en hebreo como en griego. Pero estaban más perplejos por su juventud que por su conocimiento de la verdad.

\par 
%\textsuperscript{(1383.6)}
\textsuperscript{125:6.3} En la conferencia de la tarde, apenas habían empezado a responder a su pregunta sobre la finalidad de la oración cuando el presidente invitó al muchacho a que se acercara, y una vez sentado a su lado, le pidió que expusiera su propio punto de vista respecto a la oración y la adoración.

\par 
%\textsuperscript{(1383.7)}
\textsuperscript{125:6.4} La noche anterior, los padres de Jesús habían oído hablar de un extraño joven que se batía muy hábilmente con los intérpretes de la ley, pero no se les había ocurrido que este muchacho pudiera ser su hijo. Casi habían decidido dirigirse a la casa de Zacarías, pues imaginaban que Jesús podría haber ido allí para ver a Isabel y a Juan. Pensando que Zacarías quizás estuviera en el templo, se detuvieron allí camino de la Ciudad de Judá. Mientras deambulaban por los patios del templo, imaginad su sorpresa y asombro cuando reconocieron la voz del muchacho extraviado, y lo vieron sentado entre los maestros del templo\footnote{\textit{Encuentran a Jesús}: Lc 2:46.}.

\par 
%\textsuperscript{(1384.1)}
\textsuperscript{125:6.5} José se quedó mudo, pero María dio rienda suelta a su temor y ansiedad largo tiempo reprimidos; se abalanzó hacia el joven, que ahora se había levantado para saludar a sus sorprendidos padres, y le dijo: <<Hijo mío, ¿por qué nos has tratado así? Hace ya más de tres días que tu padre y yo te buscamos angustiados. ¿Qué te ha llevado a abandonarnos?>>\footnote{\textit{La regañina de María}: Lc 2:48.} Fue un momento de tensión. Todas las miradas se volvieron hacia Jesús para ver qué iba a contestar. Su padre lo miraba con desaprobación, pero no dijo nada.

\par 
%\textsuperscript{(1384.2)}
\textsuperscript{125:6.6} Hay que recordar que se suponía que Jesús era un hombre joven. Había terminado la escolaridad normal de un niño, había sido reconocido como hijo de la ley y había recibido la consagración como ciudadano de Israel. Sin embargo, su madre le regañaba duramente delante de todo el público reunido, precisamente en mitad del esfuerzo más serio y sublime de su joven vida, poniendo fin de manera poco gloriosa a una de las mayores oportunidades que jamás se le habían presentado de enseñar la verdad, predicar la rectitud y revelar el carácter amoroso de su Padre celestial.

\par 
%\textsuperscript{(1384.3)}
\textsuperscript{125:6.7} Pero el joven se mostró a la altura de las circunstancias. Si tenéis en cuenta con imparcialidad todos los factores que se combinaron para dar lugar a esta situación, estaréis mejor preparados para examinar la sabiduría de la respuesta del chico a la reprimenda inintencionada de su madre. Después de reflexionar un momento, Jesús le dijo: <<¿Por qué me habéis buscado durante tanto tiempo? ¿Acaso no esperabais encontrarme en la casa de mi Padre, puesto que ha llegado la hora de que me ocupe de los asuntos de mi Padre?>>\footnote{\textit{Respuesta de Jesús}: Lc 2:49.}

\par 
%\textsuperscript{(1384.4)}
\textsuperscript{125:6.8} Todo el mundo se asombró de la manera de hablar del muchacho. Todos se alejaron en silencio y lo dejaron a solas con sus padres. El joven suavizó enseguida la embarazosa situación de los tres, diciendo tranquilamente: <<Vamos, padres míos, cada cual ha hecho lo que consideraba mejor. Nuestro Padre que está en los cielos ha ordenado estas cosas; regresemos a casa>>.

\par 
%\textsuperscript{(1384.5)}
\textsuperscript{125:6.9} Partieron en silencio y por la noche llegaron a Jericó. Sólo se detuvieron una vez, en la cima del Olivete, donde el joven levantó su cayado hacia el cielo y, temblando de los pies a la cabeza con la agitación de una intensa emoción, dijo: <<Oh Jerusalén, Jerusalén y sus habitantes, ¡cuán esclavizados estáis ---sometidos al yugo romano y víctimas de vuestras propias tradiciones--- pero volveré para purificar el templo y liberar a mi pueblo de esta esclavitud!>>

\par 
%\textsuperscript{(1384.6)}
\textsuperscript{125:6.10} Durante los tres días de viaje hasta Nazaret, Jesús no dijo casi nada; sus padres tampoco hablaron mucho en su presencia. Estaban realmente desorientados por la conducta de su hijo primogénito, pero atesoraron sus palabras en su corazón, aunque no pudieran comprender plenamente su significado\footnote{\textit{Regreso a Galilea}: Lc 2:50-51.}.

\par 
%\textsuperscript{(1384.7)}
\textsuperscript{125:6.11} Al llegar al hogar, Jesús hizo una breve declaración a sus padres, reiterándoles su afecto y dándoles a entender que no tenían que temer pues no volvería a ocasionarles nuevas ansiedades con su conducta. Concluyó esta importante declaración diciendo: <<Aunque debo hacer la voluntad de mi Padre celestial, también obedeceré a mi padre terrenal. Esperaré a que llegue mi hora>>.

\par 
%\textsuperscript{(1384.8)}
\textsuperscript{125:6.12} Aunque mentalmente Jesús rehusaba muchas veces \textit{aprobar} los esfuerzos bien intencionados, pero descaminados, de sus padres por dictarle el rumbo de sus reflexiones o establecer el plan de su obra en la Tierra, sin embargo, de todas las maneras compatibles con su consagración a hacer la voluntad de su Padre Paradisiaco, se \textit{conformaba} con mucho agrado a los deseos de su padre terrenal y a las costumbres de su familia carnal. Incluso cuando no podía aprobarlos, hacía todo lo posible por conformarse a ellos. Era un artista en el asunto de conciliar su consagración al deber con sus obligaciones de lealtad familiar y de servicio social.

\par 
%\textsuperscript{(1385.1)}
\textsuperscript{125:6.13} José estaba perplejo, pero María, después de reflexionar sobre estas experiencias, se sintió fortificada, acabando por considerar las palabras de Jesús en el Olivete como proféticas de la misión mesiánica de su hijo como liberador de Israel. Se dedicó con renovada energía a moldear los pensamientos de Jesús dentro de canales nacionalistas y patrióticos, y recurrió a la ayuda de su hermano, el tío favorito de Jesús. De todas las maneras posibles, la madre de Jesús se dedicó a la tarea de preparar a su hijo primogénito para que asumiera el mando de los que querían restaurar el trono de David y rechazar para siempre el yugo de la esclavitud política de los gentiles.


\chapter{Documento 126. Los dos años cruciales}
\par 
%\textsuperscript{(1386.1)}
\textsuperscript{126:0.1} DE TODAS las experiencias de la vida terrestre de Jesús, su decimocuarto y decimoquinto años fueron los más cruciales. Los dos años comprendidos entre el momento en que empezó a tomar conciencia de su divinidad y de su destino, y el momento en que logró un alto grado de comunicación con su Ajustador interior, fueron los más penosos de su extraordinaria vida en Urantia. Este período de dos años es el que debería llamarse la gran prueba, la verdadera tentación. Ningún joven humano que haya experimentado las primeras confusiones y los problemas de adaptación de la adolescencia, ha tenido que someterse nunca a una prueba más crucial que la que Jesús atravesó durante su paso de la infancia a la juventud.

\par 
%\textsuperscript{(1386.2)}
\textsuperscript{126:0.2} Este importante período en el desarrollo juvenil de Jesús empezó con el final de la visita a Jerusalén y su regreso a Nazaret. Al principio, María estaba feliz con la idea de haber recobrado a su hijo, de que Jesús había vuelto al hogar para ser un hijo obediente ---aunque nunca hubiera sido otra cosa--- y que en adelante sería más receptivo a los planes que ella forjaba para su vida futura. Pero no se iba a calentar durante mucho tiempo al sol de las ilusiones maternas y del orgullo familiar no reconocido; muy pronto se iba a desilusionar mucho más. El muchacho vivía cada vez más en compañía de su padre; cada vez acudía menos a ella con sus problemas. Al mismo tiempo, sus padres comprendían cada vez menos sus frecuentes alternancias entre los asuntos de este mundo y las meditaciones sobre su relación con los asuntos de su Padre. Francamente, no lo comprendían, pero lo amaban sinceramente.

\par 
%\textsuperscript{(1386.3)}
\textsuperscript{126:0.3} A medida que Jesús crecía, su compasión y su amor por el pueblo judío se hicieron más profundos, pero con el paso de los años, se fue acentuando en su mente un justo resentimiento contra la presencia, en el templo del Padre, de los sacerdotes nombrados por razones políticas. Jesús tenía un gran respeto por los fariseos sinceros y los escribas honestos, pero sentía un gran menosprecio por los fariseos hipócritas y los teólogos deshonestos; miraba con desdén a todos los jefes religiosos que no eran sinceros. Cuando examinaba a fondo la conducta de los dirigentes de Israel, a veces se sentía tentado a ver con buenos ojos la posibilidad de convertirse en el Mesías que esperaban los judíos, pero nunca cedió a esta tentación.

\par 
%\textsuperscript{(1386.4)}
\textsuperscript{126:0.4} El relato de sus hazañas entre los sabios del templo en Jerusalén era gratificante para todo Nazaret, en especial para sus antiguos maestros de la escuela de la sinagoga. Durante algún tiempo, los elogios hacia Jesús estuvieron en boca de todos. Todo el pueblo contaba su sabiduría infantil y su conducta ejemplar, y predecía que estaba destinado a convertirse en un gran jefe de Israel; por fin saldría de Nazaret de Galilea un maestro realmente superior. Todos esperaban el momento en que cumpliera los quince años para que se le permitiera leer regularmente las escrituras en la sinagoga el día del sábado.

\section*{1. Su decimocuarto año (año 8 d. de J.C.)}
\par 
%\textsuperscript{(1387.1)}
\textsuperscript{126:1.1} Éste es el año civil de su decimocuarto cumpleaños. Se había vuelto un buen fabricante de yugos y trabajaba bien tanto la lona como el cuero. También se estaba convirtiendo rápidamente en un experto carpintero y ebanista. Este verano subía con frecuencia a la cima de la colina, situada al noroeste de Nazaret, para orar y meditar. Gradualmente, se iba haciendo más consciente de la naturaleza de su donación en la Tierra.

\par 
%\textsuperscript{(1387.2)}
\textsuperscript{126:1.2} Hacía poco más de cien años que esta colina había sido el <<alto lugar de Baal>>\footnote{\textit{Alto lugar de Baal}: Nm 22:41.}, y ahora se encontraba allí la tumba de Simeón, un santo varón famoso en Israel. Desde la cumbre de la colina de Simeón, Jesús dominaba con la vista todo Nazaret y la región circundante. Divisaba Meguido y recordaba la historia del ejército egipcio que ganó allí su primera gran victoria en Asia; y cómo posteriormente un ejército semejante derrotó a Josías\footnote{\textit{Josías derrotado}: 2 Re 23:29-30; 2 Cr 35:20-24.}, el rey de Judea. No lejos de allí podía divisar Taanac, donde Débora y Barac derrotaron a Sísara\footnote{\textit{Sísara derrotado}: Jue 4:10-16.}. En la distancia podía ver las colinas de Dotán donde, según le habían enseñado, los hermanos de José lo vendieron como esclavo a los egipcios\footnote{\textit{José vendido como esclavo}: Gn 37:23-28.}. Luego, al volver la vista hacia Ebal y Gerizim, rememoraba las tradiciones de Abraham, Jacob y Abimelec. Así es como recordaba y repasaba en su mente los acontecimientos históricos y tradicionales del pueblo de su padre José.

\par 
%\textsuperscript{(1387.3)}
\textsuperscript{126:1.3} Continuó adelante con sus cursos superiores de lectura bajo la dirección de los profesores de la sinagoga, y también continuó con la educación familiar de sus hermanos y hermanas a medida que éstos alcanzaban la edad apropiada.

\par 
%\textsuperscript{(1387.4)}
\textsuperscript{126:1.4} A primeros de este año, José empezó a ahorrar los ingresos procedentes de sus propiedades de Nazaret y Cafarnaúm, para pagar el largo ciclo de estudios de Jesús en Jerusalén; se había planeado que Jesús iría a Jerusalén en agosto del año siguiente, cuando cumpliera los quince años.

\par 
%\textsuperscript{(1387.5)}
\textsuperscript{126:1.5} Desde los comienzos de este año, José y María tuvieron dudas frecuentes sobre el destino de su hijo primogénito. Era ciertamente un muchacho brillante y amable, pero muy difícil de comprender y muy arduo de sondear; además, nunca había sucedido nada de extraordinario o de milagroso. Su madre, orgullosa, había permanecido decenas de veces en una expectativa sin aliento, esperando ver a su hijo realizar alguna acción milagrosa o sobrehumana; pero sus esperanzas siempre terminaban en una cruel decepción. Todo esto era desalentador e incluso descorazonador. La gente piadosa de aquellos tiempos creía sinceramente que los profetas y los hombres de la promesa demostraban siempre su vocación, y establecían su autoridad divina, realizando milagros y haciendo prodigios. Pero Jesús no hacía nada de esto; por ello, la confusión de sus padres aumentaba sin cesar a medida que consideraban su futuro.

\par 
%\textsuperscript{(1387.6)}
\textsuperscript{126:1.6} El mejoramiento de la situación económica de la familia de Nazaret se reflejaba de muchas maneras en el hogar, especialmente en el aumento del número de tablillas blancas y lisas que se utilizaban como pizarras para escribir; la escritura la efectuaban con un carboncillo. A Jesús también se le permitió reanudar sus clases de música, pues le encantaba tocar el arpa.

\par 
%\textsuperscript{(1387.7)}
\textsuperscript{126:1.7} Se puede decir en verdad que, a lo largo de este año, Jesús <<creció en el favor de los hombres y de Dios>>\footnote{\textit{Jesús creció en el favor de los hombres y de Dios}: Lc 2:52.}. Las perspectivas de la familia parecían buenas y el futuro se presentaba resplandeciente.

\section*{2. La muerte de José}
\par 
%\textsuperscript{(1388.1)}
\textsuperscript{126:2.1} Todo fue bien hasta aquel martes fatal 25 de septiembre, cuando un mensajero de Séforis trajo a esta casa de Nazaret la trágica noticia de que José había sido herido de gravedad por la caída de una grúa mientras trabajaba en la residencia del gobernador. El mensajero de Séforis se había detenido en el taller antes de llegar al domicilio de José. Informó a Jesús del accidente de su padre, y los dos juntos fueron a la casa para comunicar la triste noticia a María. Jesús deseaba ir inmediatamente al lado de su padre, pero María no quería oír nada que no fuera salir corriendo para estar junto a su marido. Decidió que iría a Séforis en compañía de Santiago, que por entonces tenía diez años, mientras que Jesús permanecería en la casa con los niños más pequeños hasta su regreso, pues no conocía la gravedad de las heridas de José. Pero José había muerto a consecuencia de sus lesiones antes de que llegara María. Lo trajeron a Nazaret y al día siguiente fue enterrado con sus padres.

\par 
%\textsuperscript{(1388.2)}
\textsuperscript{126:2.2} Justo en el momento en que las perspectivas eran buenas y el futuro parecía sonreírles, una mano aparentemente cruel golpeaba al cabeza de familia de Nazaret. Los asuntos de este hogar saltaron en pedazos y todos los planes con respecto a Jesús y su futura educación quedaron destruidos. Este joven carpintero, que acababa de cumplir catorce años, tomó conciencia de que no sólo tenía que cumplir la misión recibida de su Padre celestial de revelar la naturaleza divina en la Tierra y en la carne, sino que su joven naturaleza humana tenía que asumir también la responsabilidad de cuidar de su madre viuda y de sus siete hermanos y hermanas ---sin contar la que aún no había nacido. Este joven de Nazaret se convertía ahora en el único sostén y consuelo de esta familia tan súbitamente afligida. Así se permitió que sucedieran en Urantia unos acontecimientos de tipo natural que forzaron a este joven del destino a asumir bien pronto unas responsabilidades considerables, pero altamente pedagógicas y disciplinarias. Se convirtió en el jefe de una familia humana, en el padre de sus propios hermanos y hermanas; tenía que sostener y proteger a su madre y actuar como guardián del hogar de su padre, el único hogar que llegaría a conocer mientras estuvo en este mundo.

\par 
%\textsuperscript{(1388.3)}
\textsuperscript{126:2.3} Jesús aceptó de buena gana las responsabilidades que cayeron tan repentinamente sobre él y las asumió fielmente hasta el final. Al menos un gran problema y una dificultad prevista en su vida se habían resuelto trágicamente ---ya no se esperaba que fuera a Jerusalén para estudiar con los rabinos. Siempre fue verdad que Jesús <<no era el discípulo de nadie>>. Siempre estaba dispuesto a aprender incluso del niño más humilde, pero su autoridad para enseñar la verdad nunca la obtuvo de fuentes humanas\footnote{\textit{No estuvo sentado a los pies de nadie}: Hch 22:3.}.

\par 
%\textsuperscript{(1388.4)}
\textsuperscript{126:2.4} Aún no sabía nada de la visita de Gabriel a su madre antes de su nacimiento; sólo lo supo por Juan el día de su bautismo, al comienzo de su ministerio público.

\par 
%\textsuperscript{(1388.5)}
\textsuperscript{126:2.5} A medida que pasaban los años, este joven carpintero de Nazaret medía cada vez más cada institución de la sociedad y cada costumbre de la religión con un criterio invariable: ¿Qué hace por el alma humana? ¿Trae a Dios más cerca del hombre? ¿Lleva al hombre hacia Dios? Aunque este joven no descuidaba por completo los aspectos recreativos y sociales de la vida, cada vez consagraba más su tiempo y sus energías a dos únicas metas: cuidar a su familia y prepararse para hacer en la Tierra la voluntad celestial de su Padre.

\par 
%\textsuperscript{(1389.1)}
\textsuperscript{126:2.6} Este año, los vecinos cogieron la costumbre de dejarse caer por la casa durante las noches de invierno, para escuchar a Jesús tocar el arpa, oír sus historias (pues el muchacho era un excelente narrador) y escuchar cómo leía las escrituras en griego.

\par 
%\textsuperscript{(1389.2)}
\textsuperscript{126:2.7} Los asuntos económicos de la familia continuaron rodando bastante bien, porque disponían de una suma considerable de dinero en el momento de la muerte de José. Jesús no tardó en demostrar que poseía un juicio penetrante para los negocios y sagacidad financiera. Era desprendido, pero moderado, y ahorrativo, pero generoso. Demostró ser un administrador prudente y eficaz de los bienes de su padre.

\par 
%\textsuperscript{(1389.3)}
\textsuperscript{126:2.8} Pero a pesar de todo lo que hacían Jesús y los vecinos de Nazaret para traer alegría a la casa, María, e incluso los niños, estaban llenos de tristeza. José ya no estaba. Había sido un marido y un padre excepcional, y todos lo echaban de menos. Su muerte les parecía aun más trágica cuando pensaban que no habían podido hablar con él o recibir su última bendición.

\section*{3. El decimoquinto año (año 9 d. de J.C.)}
\par 
%\textsuperscript{(1389.4)}
\textsuperscript{126:3.1} A mediados de este decimoquinto año ---contamos el tiempo de acuerdo con el calendario del siglo veinte, y no según el año judío--- Jesús había tomado firmemente el control de la dirección de su familia. Antes de finalizar este año, sus ahorros casi habían desaparecido, y se encontraron en la necesidad de vender una de las casas de Nazaret que José poseía en común con su vecino Jacobo.

\par 
%\textsuperscript{(1389.5)}
\textsuperscript{126:3.2} Rut, la más pequeña de la familia, nació la noche del miércoles 17 de abril del año 9\footnote{\textit{Nacimiento de la hermana, Ruth}: Mt 13:56; Mc 6:3.}. En la medida de sus posibilidades, Jesús se esforzó por ocupar el lugar de su padre, consolando y cuidando a su madre durante esta prueba penosa y particularmente triste. Durante cerca de veinte años (hasta que empezó su ministerio público) ningún padre podría haber amado y educado a su hija con más afecto y fidelidad que Jesús cuidó a la pequeña Rut. Fue igualmente un buen padre para todos los demás miembros de su familia.

\par 
%\textsuperscript{(1389.6)}
\textsuperscript{126:3.3} Durante este año, Jesús formuló por primera vez la oración que enseñó posteriormente a sus apóstoles, y que muchos conocen con el nombre de <<Padre Nuestro>>\footnote{\textit{Oración del Padrenuestro}: Mt 6:9-13; Lc 11:2-4.}. En cierto modo, fue una evolución del culto familiar; tenían muchas fórmulas de alabanza y diversas oraciones formales. Después de la muerte de su padre, Jesús trató de enseñar a los niños mayores a que se expresaran de manera individual en sus oraciones ---como a él tanto le gustaba hacer--- pero no podían comprender su pensamiento y retrocedían invariablemente a sus formas de rezar aprendidas de memoria. En este esfuerzo por estimular a sus hermanos y hermanas mayores para que dijeran oraciones individuales, Jesús trató de mostrarles el camino con frases sugerentes; y pronto se descubrió que, sin intención alguna por su parte, todos utilizaban una forma de rezar ampliamente basada en las ideas directrices que Jesús les había enseñado.

\par 
%\textsuperscript{(1389.7)}
\textsuperscript{126:3.4} Al final, Jesús renunció a la idea de que cada miembro de la familia formulara oraciones espontáneas. Una noche de octubre, se sentó cerca de la pequeña lámpara rechoncha, junto a la mesa baja de piedra; cogió una tablilla de cedro pulido de unos cincuenta centímetros de lado, y con un trozo de carboncillo escribió la oración que sería en adelante la súplica modelo de toda la familia.

\par 
%\textsuperscript{(1389.8)}
\textsuperscript{126:3.5} Este año Jesús estuvo muy inquieto debido a reflexiones desconcertantes. Sus responsabilidades familiares habían alejado, de manera bastante eficaz, toda idea de desarrollar enseguida un plan que se adecuara al mandato recibido en la visita de Jerusalén para que <<se ocupara de los asuntos de su Padre>>\footnote{\textit{Los asuntos de su Padre}: Lc 2:49.}. Jesús razonaba, con acierto, que velar por la familia de su padre terrenal debía tener prioridad sobre cualquier otro deber, que mantener a su familia debía ser su primera obligación.

\par 
%\textsuperscript{(1390.1)}
\textsuperscript{126:3.6} En el transcurso de este año, Jesús encontró en el llamado Libro de Enoc un pasaje que le incitó más tarde a adoptar la expresión <<Hijo del Hombre>>\footnote{\textit{Hijo del Hombre}: Ez 2:1,3,6,8; 3:1-4,10,17; Dn 7:13-14; Mt 8:20; Mc 2:10; Lc 5:24; Jn 1:51; Ap 1:13; 14:14; 1 Hen 46:1-6; 48:1-7; 60:10; 62:1,14; 63:11; 69:26-29; 70:1-2; 71:14-16.} para designarse durante su misión donadora en Urantia. Había estudiado cuidadosamente la idea del Mesías judío y estaba firmemente convencido de que él no estaba destinado a ser ese Mesías. Deseaba intensamente ayudar al pueblo de su padre, pero nunca pensó en ponerse al frente de los ejércitos judíos para liberar Palestina de la dominación extranjera. Sabía que nunca se sentaría en el trono de David en Jerusalén. Tampoco creía que su misión como liberador espiritual o educador moral se limitaría exclusivamente al pueblo judío. Así pues, la misión de su vida no podía ser de ninguna manera el cumplimiento de los deseos intensos y de las supuestas profecías mesiánicas de las escrituras hebreas, al menos no de la manera en que los judíos comprendían estas predicciones de los profetas. Asimismo, estaba seguro de que nunca aparecería como el Hijo del Hombre descrito por el profeta Daniel\footnote{\textit{No como el Hijo del Hombre de Daniel}: Dn 7:13-14.}.

\par 
%\textsuperscript{(1390.2)}
\textsuperscript{126:3.7} Pero cuando le llegara la hora de presentarse públicamente como educador del mundo, ¿cómo se llamaría a sí mismo? ¿De qué manera definiría su misión? ¿Con qué nombre lo llamarían las gentes que se convertirían en creyentes de sus enseñanzas?

\par 
%\textsuperscript{(1390.3)}
\textsuperscript{126:3.8} Mientras le daba vueltas a estos problemas en su cabeza, encontró en la biblioteca de la sinagoga de Nazaret, entre los libros apocalípticos que había estado estudiando, el manuscrito llamado <<El Libro de Enoc>>. Aunque estaba seguro de que no había sido escrito por el Enoc de los tiempos pasados, le resultó muy interesante, y lo leyó y releyó muchas veces. Había un pasaje que le impresionó particularmente, aquel en el que aparecía la expresión <<Hijo del Hombre>>. El autor del pretendido Libro de Enoc continuaba hablando de este Hijo del Hombre, describiendo la obra que debería hacer en la Tierra y explicando que este Hijo del Hombre, antes de descender a esta Tierra para aportar la salvación a la humanidad, había cruzado los atrios de la gloria celestial con su Padre, el Padre de todos; y había renunciado a toda esta grandeza y a toda esta gloria para descender a la Tierra y proclamar la salvación a los mortales necesitados. A medida que Jesús leía estos pasajes (sabiendo muy bien que gran parte del misticismo oriental incorporado en estas enseñanzas era falso), sentía en su corazón y reconocía en su mente que, de todas las predicciones mesiánicas de las escrituras hebreas y de todas las teorías sobre el libertador judío, ninguna estaba tan cerca de la verdad como esta historia incluida en el Libro de Enoc, el cual sólo estaba parcialmente acreditado; allí mismo y en ese momento decidió adoptar como título inaugural <<el Hijo del Hombre>>. Y esto fue lo que hizo cuando empezó posteriormente su obra pública. Jesús tenía una habilidad infalible para reconocer la verdad, y nunca dudaba en abrazarla, sin importarle la fuente de la que parecía emanar.

\par 
%\textsuperscript{(1390.4)}
\textsuperscript{126:3.9} Por esta época ya tenía decididas muchas cosas relacionadas con su futuro trabajo en el mundo, pero no dijo nada de estas cuestiones a su madre, que seguía aferrada a la idea de que él era el Mesías judío.

\par 
%\textsuperscript{(1390.5)}
\textsuperscript{126:3.10} Jesús pasó ahora por la gran confusión de su época juvenil. Después de haber resuelto un poco la naturaleza de su misión en la Tierra, <<ocuparse de los asuntos de su Padre>>\footnote{\textit{Los asuntos de su Padre}: Lc 2:49.} ---mostrar la naturaleza amorosa de su Padre hacia toda la humanidad--- empezó a examinar de nuevo las numerosas declaraciones de las escrituras referentes a la venida de un libertador nacional, de un rey o educador judío. ¿A qué acontecimiento se referían estas profecías? Él mismo, ¿era o no era judío? ¿Pertenecía o no a la casa de David? Su madre afirmaba que sí; su padre había indicado que no. Él decidió que no. Pero, ¿habían confundido los profetas la naturaleza y la misión del Mesías?

\par 
%\textsuperscript{(1391.1)}
\textsuperscript{126:3.11} Después de todo, ¿sería posible que su madre tuviera razón? En la mayoría de los casos, cuando en el pasado habían surgido diferencias de opinión, era ella quien había tenido razón. Si él era un nuevo educador y \textit{no} el Mesías, ¿cómo podría reconocer al Mesías judío si éste aparecía en Jerusalén durante el tiempo de su misión terrestre, y cuál sería entonces su relación con este Mesías judío? Después de que hubiera emprendido la misión de su vida, ¿cuáles serían sus relaciones con su familia, con la religión y la comunidad judías, con el Imperio Romano, con los gentiles y sus religiones? El joven galileo le daba vueltas en su mente a cada uno de estos importantes problemas y los examinaba seriamente mientras continuaba trabajando en el banco de carpintero, ganándose laboriosamente su propia vida, la de su madre y la de otras ocho bocas hambrientas.

\par 
%\textsuperscript{(1391.2)}
\textsuperscript{126:3.12} Antes de finalizar este año, María vio que los fondos de la familia disminuían. Transfirió la venta de las palomas a Santiago. Poco después compraron una segunda vaca y, con la ayuda de Miriam, empezaron a vender leche a sus vecinos de Nazaret.

\par 
%\textsuperscript{(1391.3)}
\textsuperscript{126:3.13} Los profundos períodos de meditación de Jesús, sus frecuentes desplazamientos a lo alto de la colina para orar y todas las ideas extrañas que insinuaba de vez en cuando, alarmaron considerablemente a su madre. A veces pensaba que el joven estaba fuera de sí, pero luego dominaba sus temores al recordar que, después de todo, era un hijo de la promesa y, de alguna manera, diferente a los demás jóvenes.

\par 
%\textsuperscript{(1391.4)}
\textsuperscript{126:3.14} Pero Jesús estaba aprendiendo a no expresar todos sus pensamientos, a no exponer todas sus ideas al mundo, ni siquiera a su propia madre. A partir de este año, sus informaciones sobre lo que pasaba por su mente fueron reduciéndose cada vez más; es decir, hablaba menos sobre cosas que las personas corrientes no podían comprender, y que podían conducirle a ser considerado como un tipo raro o diferente de la gente común. Según las apariencias, se volvió vulgar y convencional, aunque anhelaba encontrar a alguien que pudiera comprender sus problemas. Deseaba vivamente tener un amigo fiel y de confianza, pero sus problemas eran demasiado complejos para que pudieran ser comprendidos por sus compañeros humanos. La singularidad de esta situación excepcional le obligó a soportar solo el peso de su carga.

\section*{4. El primer sermón en la sinagoga}
\par 
%\textsuperscript{(1391.5)}
\textsuperscript{126:4.1} A partir de los quince años, Jesús podía ocupar oficialmente el púlpito de la sinagoga el día del sábado. En muchas ocasiones anteriores, cuando faltaban oradores, habían pedido a Jesús que leyera las escrituras, pero ahora había llegado el día en que la ley le permitía oficiar el servicio. Por consiguiente, el primer sábado después de su decimoquinto cumpleaños, el chazan arregló las cosas para que Jesús dirigiera los oficios matutinos de la sinagoga\footnote{\textit{Primer servicio de Jesús en la sinagoga}: Lc 4:16-20.}. Cuando todos los fieles de Nazaret estuvieron congregados, el joven, que ya había seleccionado un texto de las escrituras, se levantó y comenzó a leer:

\par 
%\textsuperscript{(1391.6)}
\textsuperscript{126:4.2} <<El espíritu del Señor Dios está sobre mí, porque el Señor me ha ungido; me ha enviado para traer buenas nuevas a los mansos, para vendar a los doloridos, para proclamar la libertad a los cautivos y liberar a los presos espirituales; para proclamar el año de la gracia de Dios y el día del ajuste de cuentas de nuestro Dios; para consolar a todos los afligidos y darles belleza en lugar de ceniza, el óleo de la alegría en lugar de luto, un canto de alabanza en vez de un espíritu angustiado, para que puedan ser llamados árboles de rectitud, la plantación del Señor, destinada a glorificarlo>>\footnote{\textit{Lectura de la escritura: El espíritu de Señor está sobre mí}: Is 61:1-3.}.

\par 
%\textsuperscript{(1392.1)}
\textsuperscript{126:4.3} <<Buscad el bien y no el mal para que podáis vivir, y así el Señor, el Dios de los ejércitos, estará con vosotros. Aborreced el mal y amad el bien; estableced el juicio en la puerta. Quizá el Señor Dios será benévolo con el remanente de José>>\footnote{\textit{Lectura de la escritura: Buscad el bien y no el mal}: Am 5:14-15.}.

\par 
%\textsuperscript{(1392.2)}
\textsuperscript{126:4.4} <<Lavaos, purificaos; la maldad de vuestras obras quitadla de delante de mis ojos; dejad de hacer el mal y aprended a hacer el bien; buscad la justicia, socorred al oprimido. Defended al huérfano y amparad a la viuda>>\footnote{\textit{Lectura de la escritura: Lavaos, purificaos}: Is 1:16-17.}.

\par 
%\textsuperscript{(1392.3)}
\textsuperscript{126:4.5} <<¿Con qué me presentaré ante el Señor, para inclinarme ante el Señor de toda la Tierra? ¿Vendré ante él con holocaustos, con becerros de un año? ¿Le agradarán al Señor millares de carneros, decenas de millares de ovejas, o ríos de aceite? ¿Daré mi primogénito por mi transgresión, el fruto de mi cuerpo por el pecado de mi alma? ¡No!, porque el Señor nos ha mostrado, oh hombres, lo que es bueno. ¿Y qué os pide el Señor si no que seáis justos, que améis la misericordia y que caminéis humildemente con vuestro Dios?>>\footnote{\textit{Lectura de la escritura: ¿Con qué me presentaré ante el Señor?}: Miq 6:6-8.}

\par 
%\textsuperscript{(1392.4)}
\textsuperscript{126:4.6} <<¿Con quién, entonces, compararéis a Dios que está sentado en el círculo de la Tierra? Levantad los ojos y mirad quién ha creado todos estos mundos, quién produce sus huestes por multitudes y las llama a todas por su nombre. Él hace todas estas cosas por la grandeza de su poder, y debido a la fuerza de su poder, ninguna fallará. Él da vigor al débil, y multiplica las fuerzas de los que están fatigados. No temáis, porque estoy con vosotros; no desmayéis, porque soy vuestro Dios. Os fortificaré y os ayudaré; sí, os sustentaré con la diestra de mi justicia, porque yo soy el Señor vuestro Dios. Y sostendré vuestra mano derecha, diciéndoos: no temáis, porque yo os ayudaré>>\footnote{\textit{Lectura de la escritura: ¿Con quién, entonces, compararéis a Dios?}: Is 40:18,22,26,29; 41:10,13.}.

\par 
%\textsuperscript{(1392.5)}
\textsuperscript{126:4.7} <<Y tú eres mi testigo, dice el Señor, y mi siervo a quien he elegido para que todos puedan conocerme, creerme y entender que yo soy el Eterno. Yo, sólo yo, soy el Señor, y aparte de mí no hay salvador>>\footnote{\textit{Lectura de la escritura: Y tú eres mi testigo, dice el Señor}: Is 43:10-11.}.

\par 
%\textsuperscript{(1392.6)}
\textsuperscript{126:4.8} Cuando terminó de leer así, se sentó, y la gente se fue a sus casas meditando las palabras que les había leído con tanto agrado. Sus paisanos nunca lo habían visto tan magníficamente solemne; nunca lo habían oído con una voz tan seria y tan sincera; nunca lo habían visto tan varonil y decidido, con tanta autoridad.

\par 
%\textsuperscript{(1392.7)}
\textsuperscript{126:4.9} Ese sábado por la tarde Jesús subió con Santiago a la colina de Nazaret, y cuando regresaron a casa, con un carboncillo escribió los Diez Mandamientos en griego sobre dos tablillas. Más tarde, Marta coloreó y adornó estas tablillas y estuvieron colgadas mucho tiempo en la pared, encima del pequeño banco de trabajo de Santiago.

\section*{5. La lucha financiera}
\par 
%\textsuperscript{(1392.8)}
\textsuperscript{126:5.1} Jesús y su familia volvieron gradualmente a la vida simple de sus primeros años. Sus ropas e incluso sus alimentos se simplificaron. Tenían leche, mantequilla y queso en abundancia. Según la estación, disfrutaban de los productos de su huerto, pero cada mes que pasaba les obligaba a practicar una mayor frugalidad. Su desayuno era muy simple; los mejores alimentos los reservaban para la cena. Sin embargo, la falta de riqueza entre estos judíos no implicaba inferioridad social.

\par 
%\textsuperscript{(1392.9)}
\textsuperscript{126:5.2} Este joven ya poseía una comprensión casi completa de cómo vivían los hombres de su tiempo. Sus enseñanzas posteriores muestran hasta qué punto comprendía bien la vida en el hogar, en el campo y en el taller; revelan plenamente su contacto íntimo con todas las fases de la experiencia humana.

\par 
%\textsuperscript{(1392.10)}
\textsuperscript{126:5.3} El chazán de Nazaret continuaba aferrado a la creencia de que Jesús estaba destinado a convertirse en un gran educador, probablemente en el sucesor del famoso Gamaliel de Jerusalén.

\par 
%\textsuperscript{(1393.1)}
\textsuperscript{126:5.4} Aparentemente, todos los planes de Jesús para su carrera se habían desbaratado. Tal como se desarrollaban las cosas, el futuro no parecía muy brillante. Sin embargo, no vaciló ni se desanimó. Continuó viviendo día tras día, desempeñando bien su deber cotidiano y cumpliendo fielmente con las responsabilidades \textit{inmediatas} de su posición social en la vida. La vida de Jesús es el consuelo eterno de todos los idealistas decepcionados.

\par 
%\textsuperscript{(1393.2)}
\textsuperscript{126:5.5} El salario diario de un carpintero corriente disminuía poco a poco. A finales de este año, y trabajando de sol a sol, Jesús sólo podía ganar el equivalente de un cuarto de dólar al día. Al año siguiente les resultó difícil pagar los impuestos civiles, sin mencionar las contribuciones a la sinagoga y el impuesto de medio siclo para el templo. Durante este año, el recaudador de impuestos intentó arrancarle a Jesús una renta suplementaria, e incluso le amenazó con llevarse su arpa.

\par 
%\textsuperscript{(1393.3)}
\textsuperscript{126:5.6} Temiendo que el ejemplar de las escrituras en griego pudiera ser descubierto y confiscado por los recaudadores de impuestos, Jesús lo donó a la biblioteca de la sinagoga de Nazaret el día de su decimoquinto cumpleaños, como su ofrenda de madurez al Señor.

\par 
%\textsuperscript{(1393.4)}
\textsuperscript{126:5.7} El gran disgusto de su decimoquinto año se produjo cuando Jesús fue a Séforis para recibir el veredicto de Herodes, relacionado con la apelación que habían interpuesto ante él por la controversia sobre la cantidad de dinero que le debían a José en el momento de su muerte accidental. Jesús y María habían esperado recibir una considerable suma de dinero, pero el tesorero de Séforis les había ofrecido una cantidad irrisoria. Los hermanos de José apelaron ante el mismo Herodes, y ahora Jesús se encontraba en el palacio y oyó a Herodes decretar que a su padre no se le debía nada en el momento de su muerte. A causa de esta decisión tan injusta, Jesús nunca más confió en Herodes Antipas. No es extraño que en una ocasión se refiriera a Herodes como <<ese zorro>>\footnote{\textit{Herodes, ``ese zorro''}: Lc 13:32.}.

\par 
%\textsuperscript{(1393.5)}
\textsuperscript{126:5.8} Durante este año y los siguientes, el duro trabajo en el banco de carpintero privó a Jesús de la posibilidad de relacionarse con los viajeros de las caravanas. Un tío suyo ya se había hecho cargo de la tienda de provisiones de la familia y Jesús trabajaba todo el tiempo en el taller de la casa, donde estaba cerca para ayudar a María con la familia. Por esta época empezó a enviar a Santiago a la parada de las caravanas para obtener información sobre los acontecimientos mundiales, intentando así mantenerse al corriente de las noticias del día.

\par 
%\textsuperscript{(1393.6)}
\textsuperscript{126:5.9} A medida que crecía hacia la madurez, pasó por los mismos conflictos y confusiones que todos los jóvenes normales de todos los tiempos anteriores y posteriores. La rigurosa experiencia de tener que mantener a su familia era una salvaguardia segura contra el exceso de tiempo libre para dedicarlo a la meditación ociosa o abandonarse a las tendencias místicas.

\par 
%\textsuperscript{(1393.7)}
\textsuperscript{126:5.10} Éste fue el año en que Jesús arrendó una gran parcela de terreno justo al norte de su casa, que dividieron en huertos familiares. Cada uno de los hermanos mayores tenía un huerto individual, y se hicieron una viva competencia en sus esfuerzos agrícolas. Durante la temporada de cultivo de las legumbres, su hermano mayor pasó cada día algún tiempo con ellos en el huerto. Mientras Jesús trabajaba en el huerto con sus hermanos y hermanas menores, acarició muchas veces la idea de que todos podían vivir en una granja en el campo, donde podrían disfrutar de la libertad y la independencia de una vida sin trabas. Pero no estaban creciendo en el campo, y Jesús, que era un joven totalmente práctico a la vez que idealista, atacó su problema de manera vigorosa e inteligente según se presentaba. Hizo todo lo que estuvo en su mano para adaptarse con su familia a las realidades de su situación, y acomodar su condición para la mayor satisfacción posible de sus deseos individuales y colectivos.

\par 
%\textsuperscript{(1393.8)}
\textsuperscript{126:5.11} En un momento determinado, Jesús tuvo la débil esperanza de que podría reunir los recursos suficientes para justificar la tentativa de comprar una pequeña granja, con tal que pudieran recaudar la considerable suma de dinero que le debían a su padre por sus trabajos en el palacio de Herodes. Había pensado muy seriamente en este proyecto de establecer a su familia en el campo. Pero cuando Herodes se negó a pagarles el dinero que le debían a José, abandonaron el deseo de poseer una casa en el campo. Tal como estaban las cosas, se las ingeniaron para disfrutar de muchas de las experiencias de la vida campesina, pues ahora tenían tres vacas, cuatro ovejas, un montón de polluelos, un asno y un perro, además de las palomas. Incluso los más pequeños tenían sus tareas regulares que hacer dentro del plan de administración bien organizado que caracterizaba la vida hogareña de esta familia de Nazaret.

\par 
%\textsuperscript{(1394.1)}
\textsuperscript{126:5.12} Al finalizar su decimoquinto año, Jesús concluyó la travesía de este período peligroso y difícil de la existencia humana, de esta época de transición entre los años más placenteros de la infancia y la conciencia de la edad adulta que se aproxima, con sus mayores responsabilidades y oportunidades para adquirir una experiencia avanzada en el desarrollo de un carácter noble. El período de crecimiento mental y físico había terminado, y ahora empezaba la verdadera carrera de este joven de Nazaret.


\chapter{Documento 127. Los años de adolescencia}
\par 
%\textsuperscript{(1395.1)}
\textsuperscript{127:0.1} AL EMPEZAR los años de su adolescencia, Jesús se encontró como jefe y único sostén de una familia numerosa. Pocos años después de la muerte de su padre, habían perdido todas sus propiedades. A medida que pasaba el tiempo, se volvió cada vez más consciente de su preexistencia; al mismo tiempo empezó a comprender más plenamente que estaba presente en la Tierra y en la carne con la finalidad expresa de revelar su Padre Paradisiaco a los hijos de los hombres.

\par 
%\textsuperscript{(1395.2)}
\textsuperscript{127:0.2} Ningún adolescente que haya vivido o que pueda vivir alguna vez en este mundo o en cualquier otro mundo ha tenido ni tendrá nunca que resolver problemas más graves o desenredar dificultades más complicadas. Ningún joven de Urantia tendrá nunca que pasar por unos conflictos más probatorios o por unas situaciones más penosas que las que Jesús mismo tuvo que soportar durante el arduo período comprendido entre sus quince y sus veinte años de edad.

\par 
%\textsuperscript{(1395.3)}
\textsuperscript{127:0.3} Tras haber saboreado así la experiencia efectiva de vivir estos años de adolescencia en un mundo acosado por el mal y perturbado por el pecado, el Hijo del Hombre llegó a poseer un conocimiento pleno de la experiencia que vive la juventud de todos los dominios de Nebadon. Así se convirtió para siempre en el refugio comprensivo de los adolescentes angustiados y perplejos de todos los tiempos, en todos los mundos del universo local.

\par 
%\textsuperscript{(1395.4)}
\textsuperscript{127:0.4} De manera lenta pero segura, y por medio de la experiencia efectiva, este Hijo divino va \textit{ganando} el derecho de convertirse en el soberano de su universo, en el gobernante supremo e incontestable de todas las inteligencias creadas en todos los mundos del universo local, en el refugio comprensivo de los seres de todos los tiempos, cualquiera que sea el grado de sus dones y experiencias personales.

\section*{1. El decimosexto año (año 10 d. de J.C.)}
\par 
%\textsuperscript{(1395.5)}
\textsuperscript{127:1.1} El Hijo encarnado pasó por la infancia y experimentó una niñez exentas de acontecimientos notables. Luego emergió de la penosa y probatoria etapa de transición entre la infancia y la juventud ---se convirtió en el Jesús adolescente.

\par 
%\textsuperscript{(1395.6)}
\textsuperscript{127:1.2} Este año alcanzó su máxima estatura física. Era un joven viril y bien parecido. Se volvió cada vez más formal y serio, pero era amable y compasivo. Tenía una mirada bondadosa pero inquisitiva; su sonrisa era siempre simpática y alentadora. Su voz era musical pero con autoridad; su saludo, cordial pero sin afectación. En todas las ocasiones, incluso en los contactos más comunes, parecía ponerse de manifiesto la esencia de una doble naturaleza, la humana y la divina. Siempre mostraba esta combinación de amigo compasivo y de maestro con autoridad. Y estos rasgos de su personalidad comenzaron a manifestarse muy pronto, incluso desde los años de su adolescencia.

\par 
%\textsuperscript{(1395.7)}
\textsuperscript{127:1.3} Este joven físicamente fuerte y robusto también había adquirido el crecimiento completo de su intelecto humano, no la experiencia total del pensamiento humano, sino la plena capacidad para ese desarrollo intelectual. Poseía un cuerpo sano y bien proporcionado, una mente aguda y analítica, una disposición de ánimo generosa y compasiva, un temperamento un poco fluctuante pero dinámico; todas estas cualidades se estaban organizando en una personalidad fuerte, sorprendente y atractiva.

\par 
%\textsuperscript{(1396.1)}
\textsuperscript{127:1.4} A medida que pasaba el tiempo, su madre y sus hermanos y hermanas tenían más dificultades para comprenderlo; tropezaban con lo que decía e interpretaban mal sus acciones. Todos eran incapaces de comprender la vida de su hermano mayor, porque su madre les había dado a entender que estaba destinado a ser el libertador del pueblo judío. Después de haber recibido estas insinuaciones de María como secretos de familia, imaginad su confusión cuando Jesús desmentía francamente todas estas ideas e intenciones.

\par 
%\textsuperscript{(1396.2)}
\textsuperscript{127:1.5} Este año Simón empezó a ir a la escuela, y la familia se vio obligada a vender otra casa. Santiago se encargó ahora de la enseñanza de sus tres hermanas, dos de las cuales eran lo bastante mayores como para empezar a estudiar en serio. En cuanto Rut creció, la pusieron en manos de Miriam y Marta. Habitualmente, las muchachas de las familias judías recibían poca educación, pero Jesús sostenía (y su madre estaba de acuerdo) que las chicas tenían que ir a la escuela lo mismo que los varones, y puesto que la escuela de la sinagoga no las admitiría, lo único que se podía hacer era habilitar una escuela en casa especialmente para ellas.

\par 
%\textsuperscript{(1396.3)}
\textsuperscript{127:1.6} Durante todo este año, Jesús no pudo separarse de su banco de carpintero. Afortunadamente tenía mucho trabajo; lo realizaba de una manera tan superior que nunca se encontraba en paro, aunque la faena escaseara por aquella región. A veces tenía tanto que hacer que Santiago lo ayudaba.

\par 
%\textsuperscript{(1396.4)}
\textsuperscript{127:1.7} A finales de este año tenía casi decidido que, después de haber criado a los suyos y de verlos casados, emprendería su trabajo público como maestro de la verdad y revelador del Padre celestial para el mundo. Sabía que no se convertiría en el Mesías judío esperado, y llegó a la conclusión de que era prácticamente inútil discutir estos asuntos con su madre. Decidió permitirle que siguiera manteniendo todas las ilusiones que quisiera, puesto que todo lo que él había dicho en el pasado había hecho poca o ninguna mella en ella; recordaba que su padre nunca había podido decir algo que la hiciera cambiar de opinión. A partir de este año habló cada vez menos con su madre, o con otras personas, sobre estos problemas. Su misión era tan especial que nadie en el mundo podía darle consejos para realizarla.

\par 
%\textsuperscript{(1396.5)}
\textsuperscript{127:1.8} A pesar de su juventud, era un verdadero padre para su familia. Pasaba todas las horas que podía con los pequeños, y éstos lo amaban sinceramente. Su madre sufría al verlo trabajar tan duramente; le apenaba que estuviera día tras día atado al banco de carpintero para ganar la vida de la familia, en lugar de estar en Jerusalén estudiando con los rabinos, tal como habían planeado con tanto cariño. Aunque María no podía comprender muchas cosas de su hijo, lo amaba de verdad; lo que más apreciaba era la buena voluntad con que asumía la responsabilidad del hogar.

\section*{2. El decimoséptimo año (año 11 d. de J.C.)}
\par 
%\textsuperscript{(1396.6)}
\textsuperscript{127:2.1} Por esta época se produjo una agitación considerable, especialmente en Jerusalén y Judea, a favor de una rebelión contra el pago de los impuestos a Roma. Estaba creándose un fuerte partido nacionalista, que poco después se conocería como los celotes. Los celotes, al contrario que los fariseos, no estaban dispuestos a esperar la llegada del Mesías. Proponían resolver la situación mediante una revuelta política.

\par 
%\textsuperscript{(1396.7)}
\textsuperscript{127:2.2} Un grupo de organizadores de Jerusalén llegó a Galilea y fueron teniendo mucho éxito hasta que se presentaron en Nazaret. Cuando fueron a ver a Jesús, éste los escuchó atentamente y les hizo muchas preguntas, pero rehusó incorporarse al partido. No quiso explicar en detalle todas las razones que le impedían adherirse, y su negativa tuvo por efecto que muchos de sus jóvenes amigos de Nazaret tampoco se afiliaran.

\par 
%\textsuperscript{(1397.1)}
\textsuperscript{127:2.3} María hizo lo que pudo para inducirlo a que se afiliara, pero no logró hacerle cambiar de parecer. Llegó incluso a insinuarle que su negativa a abrazar la causa nacionalista, como ella se lo ordenaba, equivalía a una insubordinación, a una violación de la promesa que había hecho, cuando regresaron de Jerusalén, de que obedecería a sus padres; pero en respuesta a esta insinuación, Jesús se limitó a poner una mano cariñosa en su hombro y mirándola a la cara le dijo: <<Madre, ¿cómo puedes?>> Y María se retractó.

\par 
%\textsuperscript{(1397.2)}
\textsuperscript{127:2.4} Uno de los tíos de Jesús (Simón, el hermano de María) ya se había unido a este grupo, y posteriormente llegó a convertirse en oficial de la sección galilea. Durante varios años, se produjo cierto distanciamiento entre Jesús y su tío.

\par 
%\textsuperscript{(1397.3)}
\textsuperscript{127:2.5} Pero el alboroto se estaba fraguando en Nazaret. La actitud de Jesús en este asunto había dado como resultado la creación de una división entre los jóvenes judíos de la ciudad. Aproximadamente la mitad se había unido a la organización nacionalista, y la otra mitad empezó a formar un grupo opuesto de patriotas más moderados, esperando que Jesús asumiera la dirección. Se quedaron asombrados cuando rehusó el honor que le ofrecían, alegando como excusa sus pesadas responsabilidades familiares, cosa que todos reconocían. Pero la situación se complicó aún más cuando poco después se presentó Isaac, un judío rico prestamista de los gentiles, que propuso mantener a la familia de Jesús si éste abandonaba sus herramientas de trabajo y asumía la dirección de estos patriotas de Nazaret.

\par 
%\textsuperscript{(1397.4)}
\textsuperscript{127:2.6} Jesús, que apenas tenía entonces diecisiete años, tuvo que enfrentarse con una de las situaciones más delicadas y difíciles de su joven vida. Siempre es difícil para los dirigentes espirituales relacionarse con las cuestiones patrióticas, especialmente cuando éstas se complican con unos opresores extranjeros que recaudan impuestos; en este caso era doblemente cierto, puesto que la religión judía estaba implicada en toda esta agitación contra Roma.

\par 
%\textsuperscript{(1397.5)}
\textsuperscript{127:2.7} La posición de Jesús era aún más delicada porque su madre, su tío e incluso su hermano menor Santiago, lo instaban a abrazar la causa nacionalista. Los mejores judíos de Nazaret ya se habían afiliado, y los jóvenes que aún no se habían incorporado al movimiento lo harían en cuanto Jesús cambiara de opinión. Sólo tenía un consejero sabio en todo Nazaret, su viejo maestro el chazan, que le aconsejó sobre cómo responder al comité de ciudadanos de Nazaret cuando vinieran a pedirle su respuesta a la petición pública que se le había hecho. En toda la juventud de Jesús, ésta fue la primera vez que tuvo que recurrir conscientemente a una estratagema pública. Hasta entonces, siempre había contado con una exposición sincera de la verdad para esclarecer la situación, pero ahora no podía proclamar toda la verdad. No podía insinuar que era más que un hombre; no podía revelar su idea de la misión que le aguardaba cuando fuera más maduro. A pesar de estas limitaciones, su fidelidad religiosa y su lealtad nacional estaban puestas en entredicho directamente. Su familia se encontraba agitada, sus jóvenes amigos divididos y todo el contingente judío de la ciudad alborotado. ¡Y pensar que él era el culpable de todo esto! Qué lejos estaba de su intención causar cualquier alboroto y mucho menos una perturbación de este tipo.

\par 
%\textsuperscript{(1397.6)}
\textsuperscript{127:2.8} Había que hacer algo. Tenía que aclarar su postura, y lo hizo de manera valiente y diplomática, para satisfacción de muchos aunque no de todos. Se atuvo a los términos de su argumento original, sosteniendo que su primer deber era hacia su familia, que una madre viuda y ocho hermanos y hermanas necesitaban algo más que lo que simplemente se podía comprar con el dinero ---lo necesario para la vida material---, que tenían derecho a los cuidados y a la dirección de un padre, y que en conciencia no podía eximirse de la obligación que un cruel accidente había arrojado sobre él. Elogió a su madre y al mayor de sus hermanos por estar dispuestos a exonerarlo de esta responsabilidad, pero reiteró que la fidelidad a la memoria de su padre le impedía dejar a la familia, independientemente de la cantidad de dinero que se recibiera para su sostén material, expresando entonces su inolvidable afirmación de que <<el dinero no puede amar>>. En el transcurso de esta declaración, Jesús hizo varias alusiones veladas a la <<misión de su vida>>, pero explicó que, con independencia de que fuera o no compatible con la acción militar, había renunciado a ella así como a todo lo demás para poder cumplir fielmente sus obligaciones hacia su familia. En Nazaret todos sabían muy bien que era un buen padre para su familia, y como esto era algo que tocaba la sensibilidad de todo judío bien nacido, la alegación de Jesús encontró una respuesta favorable en el corazón de muchos de sus oyentes. Algunos otros que no tenían las mismas disposiciones fueron desarmados por un discurso que Santiago pronunció en ese momento, aunque no figurara en el programa. Aquel mismo día, el chazan había hecho que Santiago ensayara su alocución, pero esto era un secreto entre ellos.

\par 
%\textsuperscript{(1398.1)}
\textsuperscript{127:2.9} Santiago declaró que estaba seguro de que Jesús ayudaría a liberar a su pueblo si él (Santiago) tuviera suficiente edad como para asumir la responsabilidad de la familia; si consentían en permitir a Jesús que permaneciera <<con nosotros para ser nuestro padre y educador, la familia de José no sólo os dará un dirigente, sino en poco tiempo cinco nacionalistas leales, porque ¿no somos cinco varones que estamos creciendo y que saldremos de la tutela de nuestro hermano-padre para servir a nuestra nación?>> De esta manera el muchacho llevó a un final bastante feliz una situación muy tensa y amenazadora.

\par 
%\textsuperscript{(1398.2)}
\textsuperscript{127:2.10} La crisis se había superado por el momento, pero este incidente nunca se olvidó en Nazaret. La agitación persistió; Jesús ya no volvió a contar con el favor universal; las diferencias de sentimiento nunca llegaron a superarse del todo. Este hecho, complicado con otros acontecimientos posteriores, fue uno de los motivos principales por los que Jesús se trasladó años más tarde a Cafarnaúm. En adelante, los sentimientos respecto al Hijo del Hombre permanecieron divididos en Nazaret.

\par 
%\textsuperscript{(1398.3)}
\textsuperscript{127:2.11} Santiago terminó este año sus estudios en la escuela y empezó a trabajar a jornada completa en el taller de carpintería de la casa. Se había convertido en un obrero diestro con las herramientas y se hizo cargo de la fabricación de los yugos y arados, mientras que Jesús empezó a hacer más trabajos delicados de ebanistería y de terminación de interiores.

\par 
%\textsuperscript{(1398.4)}
\textsuperscript{127:2.12} Durante este año Jesús progresó mucho en la organización de su mente. Gradualmente había conciliado su naturaleza divina con su naturaleza humana, y efectuó toda esta organización intelectual con la fuerza de sus propias \textit{decisiones} y con la única ayuda de su Monitor interior, un Monitor semejante al que llevan dentro de su mente todos los mortales normales en todos los mundos donde se ha donado un Hijo. Hasta ahora no había sucedido nada sobrenatural en la carrera de este joven, excepto la visita de un mensajero enviado por su hermano mayor Emmanuel, que se le apareció una vez durante la noche en Jerusalén.

\section*{3. El decimoctavo año (año 12 d. de J.C.)}
\par 
%\textsuperscript{(1398.5)}
\textsuperscript{127:3.1} En el transcurso de este año, todas las propiedades de la familia, excepto la casa y el huerto, fueron liquidadas. Se vendió la última parcela de una propiedad en Cafarnaúm (excepto una parte de otra propiedad) que ya estaba hipotecada. Las ganancias se emplearon para pagar los impuestos, comprar algunas herramientas nuevas para Santiago, y pagar una parte de la antigua tienda de reparaciones y abastecimientos de la familia, cercana a la parada de las caravanas. Jesús se proponía ahora comprar de nuevo esta tienda, pues Santiago ya tenía edad para trabajar en el taller de la casa y ayudar a María en el hogar. Liberado por el momento de la presión financiera, Jesús decidió llevar a Santiago a la Pascua. Partieron para Jerusalén un día antes para estar solos, y fueron por el camino de Samaria. Iban a pie y Jesús informó a Santiago sobre los lugares históricos que iban atravesando, como su padre lo había hecho con él cinco años antes en un viaje similar.

\par 
%\textsuperscript{(1399.1)}
\textsuperscript{127:3.2} Al pasar por Samaria observaron muchos espectáculos extraños. Durante este viaje conversaron sobre muchos de sus problemas personales, familiares y nacionales. Santiago era un muchacho con fuertes tendencias religiosas, y aunque no estaba plenamente de acuerdo con su madre sobre lo poco que conocía de los planes relacionados con la obra de la vida de Jesús, esperaba impaciente el momento en que sería capaz de asumir la responsabilidad de la familia, para que Jesús pudiera empezar su misión. Apreciaba mucho que Jesús lo llevara a la Pascua, y hablaron sobre el futuro con más profundidad de lo que nunca lo habían hecho antes.

\par 
%\textsuperscript{(1399.2)}
\textsuperscript{127:3.3} Jesús reflexionó mucho mientras atravesaban Samaria, especialmente en Betel y cuando estuvieron bebiendo en el pozo de Jacob. Examinó con su hermano las tradiciones de Abraham, Isaac y Jacob. Preparó bien a Santiago para lo que iba a presenciar en Jerusalén, tratando así de atenuar una conmoción semejante a la que él mismo había experimentado en su primera visita al templo. Pero Santiago no era tan sensible a algunos de estos espectáculos. Hizo comentarios sobre la manera superficial e indiferente con que algunos de los sacerdotes efectuaban sus deberes, pero en conjunto disfrutó enormemente de su estancia en Jerusalén.

\par 
%\textsuperscript{(1399.3)}
\textsuperscript{127:3.4} Jesús llevó a Santiago a Betania para la cena pascual. Simón había fallecido y descansaba con sus antepasados, y Jesús ocupó el lugar del cabeza de familia para la Pascua, pues había traído del templo el cordero pascual.

\par 
%\textsuperscript{(1399.4)}
\textsuperscript{127:3.5} Después de la cena pascual, María se sentó a charlar con Santiago mientras que Marta, Lázaro y Jesús estuvieron hablando hasta muy entrada la noche. Al día siguiente asistieron a los oficios del templo, y Santiago fue recibido en la comunidad de Israel. Aquella mañana, al detenerse en la cima del Olivete para mirar el templo, Santiago expresó su admiración mientras que Jesús contemplaba Jerusalén en silencio. Santiago no podía comprender el comportamiento de su hermano. Aquella noche regresaron de nuevo a Betania, y al día siguiente habrían partido para su casa, pero Santiago insistía en volver a visitar el templo, explicando que quería escuchar a los maestros. Y aunque esto era cierto, deseaba en secreto oír a Jesús participar en los debates, tal como se lo había oído contar a su madre. Así pues fueron al templo y escucharon los debates, pero Jesús no hizo ninguna pregunta. Todo aquello parecía pueril e insignificante para esta mente de hombre y Dios en vías de despertarse ---sólo podía apiadarse de ellos. A Santiago le decepcionó que Jesús no dijera nada. A sus preguntas, Jesús se limitó a responder: <<Mi hora aún no ha llegado>>.

\par 
%\textsuperscript{(1399.5)}
\textsuperscript{127:3.6} Al día siguiente emprendieron el viaje de vuelta por Jericó y el valle del Jordán. Jesús contó muchas cosas por el camino, entre ellas su primer viaje por esta carretera cuando tenía trece años.

\par 
%\textsuperscript{(1399.6)}
\textsuperscript{127:3.7} A su regreso a Nazaret, Jesús empezó a trabajar en el viejo taller de reparaciones de la familia, y se sintió muy contento de poder encontrarse a diario con tanta gente de todas partes del país y de las comarcas circundantes. Jesús amaba realmente a la gente ---a la gente común y corriente. Cada mes pagaba la mensualidad de la compra del taller, y con la ayuda de Santiago, continuó manteniendo a la familia.

\par 
%\textsuperscript{(1399.7)}
\textsuperscript{127:3.8} Varias veces al año, cuando no había visitantes que lo hicieran, Jesús continuaba leyendo las escrituras del sábado en la sinagoga y muchas veces comentaba la lección; pero habitualmente seleccionaba los pasajes de tal manera que no necesitaban comentarios. Era tan hábil ordenando la lectura de los distintos pasajes, que éstos se iluminaban entre sí. Siempre que hacía buen tiempo, nunca dejaba de llevar a sus hermanos y hermanas a pasear por la naturaleza las tardes del sábado.

\par 
%\textsuperscript{(1400.1)}
\textsuperscript{127:3.9} Por esta época, el chazan inauguró una tertulia de discusiones filosóficas para jóvenes; éstos se reunían en la casa de los diversos miembros y a menudo en la del chazan. Jesús llegó a ser un miembro eminente de este grupo. De este manera pudo recobrar una parte del prestigio local que había perdido al producirse las recientes controversias nacionalistas.

\par 
%\textsuperscript{(1400.2)}
\textsuperscript{127:3.10} Su vida social, aunque restringida, no estaba descuidada por completo. Contaba con muy buenos amigos y fieles admiradores entre los jóvenes y las muchachas de Nazaret.

\par 
%\textsuperscript{(1400.3)}
\textsuperscript{127:3.11} En septiembre, Isabel y Juan vinieron a visitar a la familia de Nazaret. Juan, que había perdido a su padre, se proponía regresar a las colinas de Judea para dedicarse a la agricultura y a la cría de ovejas, a menos que Jesús le aconsejara quedarse en Nazaret para dedicarse a la carpintería o a cualquier otro oficio. Juan y su madre no sabían que la familia de Nazaret estaba prácticamente sin dinero. Cuanto más hablaban María e Isabel de sus hijos, más estaban convencidas de que sería bueno que los dos jóvenes trabajaran juntos y se vieran con más frecuencia.

\par 
%\textsuperscript{(1400.4)}
\textsuperscript{127:3.12} Jesús y Juan tuvieron varias conversaciones a solas y hablaron de algunos asuntos muy íntimos y personales. Al concluir esta visita, los dos decidieron no volver a verse hasta que se encontraran en su ministerio público, después de que <<el Padre celestial los hubiera llamado>> para cumplir con su misión. Juan se quedó enormemente impresionado por lo que vio en Nazaret, y comprendió que debía regresar a su casa y trabajar para mantener a su madre. Se convenció de que participaría en la misión de la vida de Jesús, pero vio que Jesús iba a estar ocupado muchos años cuidando a su familia. Por eso estaba mucho más contento de regresar a su hogar, dedicarse a cuidar su pequeña granja y atender las necesidades de su madre. Juan y Jesús no volvieron a verse nunca más hasta el día en que el Hijo del Hombre se presentó para ser bautizado en el Jordán.

\par 
%\textsuperscript{(1400.5)}
\textsuperscript{127:3.13} La tarde del sábado 3 de diciembre de este año, la muerte golpeó por segunda vez a esta familia de Nazaret. El pequeño Amós, su hermanito, murió después de una semana de enfermedad con fiebre alta. Después de atravesar este período doloroso con su hijo primogénito como único sostén, María reconoció finalmente y en todos los sentidos que Jesús era el verdadero jefe de la familia; y era en verdad un jefe valioso.

\par 
%\textsuperscript{(1400.6)}
\textsuperscript{127:3.14} Durante cuatro años, su nivel de vida había declinado constantemente; año tras año se sentían cada vez más atenazados por la pobreza. Hacia el final de este año se enfrentaron con una de las experiencias más difíciles de todas sus arduas luchas. Santiago todavía no había empezado a ganar mucho, y los gastos de un entierro sumados a todo lo demás les hizo tambalearse. Pero Jesús se limitó a decir a su madre ansiosa y afligida: <<Madre María, la tristeza no nos ayudará; todos hacemos lo mejor que podemos, y la sonrisa de mamá quizás podría inspirarnos para hacerlo aún mejor. Día tras día nos sentimos fortalecidos para estas tareas por nuestra esperanza de disfrutar de tiempos mejores en el futuro>>. Su optimismo práctico y sólido era realmente contagioso; todos los niños vivían en un ambiente donde se esperaban tiempos y cosas mejores. Esta valentía llena de esperanza contribuyó poderosamente a desarrollar en ellos unos caracteres fuertes y nobles, a pesar de su pobreza deprimente.

\par 
%\textsuperscript{(1400.7)}
\textsuperscript{127:3.15} Jesús poseía la facultad de movilizar eficazmente todos los poderes de su mente, de su alma y de su cuerpo para efectuar la tarea que tenía entre manos. Podía concentrar su mente profunda en el problema concreto que deseaba resolver, y esto, unido a su \textit{paciencia} incansable, le permitió soportar con serenidad las pruebas de una existencia mortal difícil ---vivir como si estuviera <<viendo a Aquel que es invisible>>\footnote{\textit{Viendo a Aquel que es invisible}: Heb 11:27.}.

\section*{4. El decimonoveno año (año 13 d. de J.C.)}
\par 
%\textsuperscript{(1401.1)}
\textsuperscript{127:4.1} Por esta época, Jesús y María se entendieron mucho mejor. Ella lo consideraba menos como un hijo; se había vuelto para ella como un padre para sus hijos. La vida cotidiana rebosaba de dificultades prácticas e inmediatas. Hablaban con menos frecuencia de la obra de su vida, porque a medida que pasaba el tiempo, todos sus pensamientos estaban mutuamente consagrados al mantenimiento y a la educación de su familia de cuatro niños y tres niñas.

\par 
%\textsuperscript{(1401.2)}
\textsuperscript{127:4.2} A principios de este año, Jesús había conseguido que su madre aceptara plenamente sus métodos para educar a los niños ---la orden positiva de hacer el bien en lugar del antiguo método judío de prohibir hacer el mal. En su casa y durante toda su carrera de enseñanza pública, Jesús utilizó invariablemente la fórmula de exhortación \textit{positiva}. Siempre y en todas partes decía: <<Haréis esto, deberíais hacer aquello>>. Nunca empleaba la manera negativa de enseñar, derivada de los antiguos tabúes. Evitaba resaltar el mal prohibiéndolo, mientras que realzaba el bien ordenando su ejecución. En esta casa, la hora de la oración era el momento de debatir todos los asuntos relacionados con el bienestar de la familia.

\par 
%\textsuperscript{(1401.3)}
\textsuperscript{127:4.3} Jesús empezó a disciplinar sabiamente a sus hermanos y hermanas a una edad tan temprana que nunca tuvo necesidad de castigarlos mucho para conseguir su pronta y sincera obediencia. La única excepción era Judá, a quien en diversas ocasiones Jesús estimó necesario imponer un castigo por sus infracciones a las reglas del hogar. En tres ocasiones en que se juzgó oportuno castigar a Judá por haber violado deliberadamente las reglas de conducta de la familia, y haberlo confesado, su castigo fue dictado por la decisión unánime de los niños mayores y aprobado por el mismo Judá antes de serle infligido.

\par 
%\textsuperscript{(1401.4)}
\textsuperscript{127:4.4} Aunque Jesús era muy metódico y sistemático en todo lo que hacía, había también, en todas sus decisiones administrativas, una elasticidad de interpretación refrescante y una adaptación individual que impresionaba enormemente a todos los niños por el espíritu de justicia con que actuaba su hermano-padre. Nunca castigó arbitrariamente a sus hermanos y hermanas; esta justicia constante y esta consideración personal hicieron que Jesús fuese muy querido por toda su familia.

\par 
%\textsuperscript{(1401.5)}
\textsuperscript{127:4.5} Santiago y Simón crecieron tratando de seguir el método de Jesús, consistente en aplacar a sus compañeros de juego belicosos y a veces enfurecidos mediante la persuasión y la no resistencia, y muchas veces lo consiguieron; por el contrario, aunque José y Judá aceptaban estas enseñanzas en el hogar, se apresuraban a defenderse cuando eran agredidos por sus compañeros; Judá en particular era culpable de violar el espíritu de estas enseñanzas. Pero la no resistencia no era una \textit{regla} de la familia. No se imponía ningún castigo por violar las enseñanzas personales.

\par 
%\textsuperscript{(1401.6)}
\textsuperscript{127:4.6} Todos los niños en general, pero sobre todo las niñas, consultaban a Jesús acerca de sus aflicciones infantiles y confiaban en él como lo harían en un padre cariñoso.

\par 
%\textsuperscript{(1401.7)}
\textsuperscript{127:4.7} A medida que crecía, Santiago se iba convirtiendo en un joven bien equilibrado y de buen carácter, pero no tenía tantas tendencias espirituales como Jesús. Era mucho mejor estudiante que José, y éste, aunque era un buen trabajador, tenía aún menos tendencias espirituales. José era constante y no llegaba al nivel intelectual de los otros niños. Simón era un muchacho bien intencionado, pero demasiado soñador. Fue lento en establecerse en la vida y causó considerables inquietudes a Jesús y María, pero siempre fue un chico bueno y bien intencionado. Judá era un agitador. Tenía los ideales más elevados, pero poseía un temperamento inestable. Era tan decidido y dinámico como su madre o más aún, pero carecía mucho del sentido que ella tenía de la medida y de la discreción.

\par 
%\textsuperscript{(1402.1)}
\textsuperscript{127:4.8} Miriam era una hija bien equilibrada y sensata, con una aguda apreciación de las cosas nobles y espirituales. Marta pensaba y actuaba lentamente, pero era una chica muy eficiente y digna de confianza. La pequeña Rut era la alegría de la casa; aunque hablaba sin reflexionar, tenía un corazón de lo más sincero. Casi adoraba a su hermano mayor y padre, pero ellos no la mimaban. Era una niña hermosa, pero no tan bien parecida como Miriam, que era la belleza de la familia, si no de la ciudad.

\par 
%\textsuperscript{(1402.2)}
\textsuperscript{127:4.9} A medida que pasaba el tiempo, Jesús contribuyó mucho a liberalizar y modificar las enseñanzas y las prácticas de la familia relativas a la observancia del sábado y a otros muchos aspectos de la religión; María dio su sincera aprobación a todos estos cambios. Por esta época Jesús se había convertido en el jefe incontestable de la casa.

\par 
%\textsuperscript{(1402.3)}
\textsuperscript{127:4.10} Judá empezó a ir a la escuela este año, y Jesús se vio obligado a vender su arpa para costear los gastos. Así desapareció el último de sus placeres recreativos. Le gustaba mucho tocar el arpa cuando tenía la mente cansada y el cuerpo fatigado, pero se consoló con la idea de que al menos el arpa no caería en manos del cobrador de impuestos.

\section*{5. Rebeca, la hija de Esdras}
\par 
%\textsuperscript{(1402.4)}
\textsuperscript{127:5.1} Aunque Jesús era pobre, su posición social en Nazaret no había disminuido en absoluto. Era uno de los jóvenes más destacados de la ciudad y muy considerado por la mayoría de las muchachas. Puesto que Jesús era un espléndido ejemplar de madurez física e intelectual, y dada su reputación como guía espiritual, no es de extrañar que Rebeca, la hija mayor de Esdras, un rico mercader y negociante de Nazaret, descubriera que se estaba enamorando poco a poco de este hijo de José. Primero confió sus sentimientos a Miriam, la hermana de Jesús, y Miriam a su vez se lo comentó a su madre. María se alarmó mucho. ¿Estaba a punto de perder a su hijo, que ahora era el cabeza indispensable de la familia? ¿Nunca se terminarían las dificultades? ¿Qué podría ocurrir después? Entonces se detuvo a meditar sobre el efecto que tendría el matrimonio sobre la futura carrera de Jesús. No muy a menudo, pero al menos de vez en cuando, recordaba el hecho de que Jesús era un <<hijo de la promesa>>. Después de discutir este asunto, María y Miriam decidieron hacer un esfuerzo para ponerle fin antes de que Jesús se enterara; fueron a ver directamente a Rebeca, le expusieron toda la historia y le contaron francamente su creencia de que Jesús era un hijo del destino, que iba a convertirse en un gran guía religioso, tal vez en el Mesías.

\par 
%\textsuperscript{(1402.5)}
\textsuperscript{127:5.2} Rebeca escuchó atentamente; se quedó pasmada con el relato y estuvo más decidida que nunca a unir su destino con el de este hombre de su elección y compartir su carrera de dirigente. Discurría (en su interior) que un hombre así tendría aún más necesidad de una esposa fiel y eficiente. Interpretó los esfuerzos de María por disuadirla como una reacción natural ante el temor de perder al jefe y único sostén de su familia; pero sabiendo que su padre aprobaba la atracción que sentía por el hijo del carpintero, suponía acertadamente que aquel proporcionaría con mucho gusto a la familia la renta suficiente con la que compensar ampliamente la pérdida de los ingresos de Jesús. Cuando su padre aceptó este proyecto, Rebeca mantuvo otras conversaciones con María y Miriam, pero al no conseguir su apoyo, tuvo el atrevimiento de acudir directamente a Jesús. Lo hizo con la cooperación de su padre, que invitó a Jesús a su casa para la celebración del decimoséptimo cumpleaños de Rebeca.

\par 
%\textsuperscript{(1403.1)}
\textsuperscript{127:5.3} Jesús escuchó con atención y simpatía la narración de todo lo sucedido, primero por parte del padre de Rebeca, y luego por ella misma. Contestó con amabilidad que ninguna cantidad de dinero podría reemplazar su obligación personal de criar a la familia de su padre, <<de cumplir con el deber humano más sagrado ---la lealtad a la propia carne y a la propia sangre>>. El padre de Rebeca se sintió profundamente conmovido por las palabras de devoción familiar de Jesús y se retiró de la entrevista. Su único comentario a su esposa María fue: <<No podemos tenerlo como hijo; es demasiado noble para nosotros>>.

\par 
%\textsuperscript{(1403.2)}
\textsuperscript{127:5.4} Entonces empezó la memorable conversación con Rebeca. Hasta ese momento de su vida, Jesús había hecho poca distinción en sus relaciones con los niños y las niñas, con los jóvenes y las muchachas. Su mente había estado demasiado ocupada con los problemas urgentes de los asuntos prácticos de este mundo y con la contemplación misteriosa de su posible carrera <<relacionada con los asuntos de su Padre>>\footnote{\textit{Relacionada con los asuntos de su Padre}: Lc 2:49.}, como para haber considerado nunca seriamente la consumación del amor personal en el matrimonio humano. Pero ahora se encontraba frente a otro de los problemas que cualquier ser humano corriente tiene que afrontar y resolver. En verdad fue <<probado en todas las cosas igual que vosotros>>\footnote{\textit{Probado en todas las cosas igual que vosotros}: Heb 4:15.}.

\par 
%\textsuperscript{(1403.3)}
\textsuperscript{127:5.5} Después de escuchar con atención, agradeció sinceramente a Rebeca la admiración que le expresaba, y añadió: <<Esto me alentará y me confortará todos los días de mi vida>>. Le explicó que no era libre de tener, con una mujer, otras relaciones que las de simple consideración fraternal y la de pura amistad. Precisó que su deber primero y supremo era criar a la familia de su padre, que no podía pensar en el matrimonio hasta que completara esta tarea; y entonces añadió: <<Si soy un hijo del destino, no debo asumir obligaciones para toda la vida hasta el momento en que mi destino se haga manifiesto>>.

\par 
%\textsuperscript{(1403.4)}
\textsuperscript{127:5.6} A Rebeca se le rompió el corazón. No quiso ser consolada, y pidió insistentemente a su padre que se fueran de Nazaret, hasta que éste consintió finalmente en mudarse a Séforis. En los años que siguieron, Rebeca sólo tuvo una respuesta para los numerosos hombres que la pidieron en matrimonio. Vivía con una sola finalidad ---esperar la hora en que aquel que era para ella el hombre más grande que hubiera vivido nunca, empezara su carrera como maestro de la verdad viviente. Lo siguió con devoción durante los años extraordinarios de su ministerio público. Estuvo presente (sin que Jesús lo advirtiera) el día que entró triunfalmente en Jerusalén\footnote{\textit{Entrada triunfal}: Mt 21:8-9; Mc 11:8-11a; Lc 19:36-38; Jn 12:12-13.}; y se hallaba <<entre las otras mujeres>> al lado de María\footnote{\textit{María y las otras mujeres testigos de la crucifixión}: Mt 27:55-56; Mc 15:40; Lc 23:49; Jn 19:25.}, aquella tarde fatídica y trágica en que el Hijo del Hombre fue suspendido en la cruz. Porque para ella, como para innumerables mundos de arriba, él era <<el único enteramente digno de ser amado y el más grande entre diez mil>>\footnote{\textit{El único enteramente digno de ser amado}: Cnt 5:16. \textit{El más grande entre diez mil}: Cnt 5:10.}.

\section*{6. Su vigésimo año (año 14 d. de J.C)}
\par 
%\textsuperscript{(1403.5)}
\textsuperscript{127:6.1} La historia del amor de Rebeca por Jesús se murmuraba en Nazaret y posteriormente en Cafarnaúm, de manera que, aunque en los años siguientes muchas mujeres amaron a Jesús como los hombres lo amaban, nunca más tuvo que rechazar la propuesta personal de la devoción de otra mujer de bien. A partir de este momento, el amor humano por Jesús tuvo más bien la naturaleza de una consideración respetuosa y adoradora. Hombres y mujeres lo amaban con devoción por lo que él era, sin el menor matiz de satisfacción personal y sin el deseo de posesión afectiva. Pero durante muchos años, cada vez que se contaba la historia de la personalidad humana de Jesús, se mencionaba la devoción de Rebeca.

\par 
%\textsuperscript{(1404.1)}
\textsuperscript{127:6.2} Miriam, que conocía bien la historia de Rebeca y sabía cómo su hermano había renunciado incluso al amor de una hermosa doncella (sin percibir el factor de la carrera futura que sería su destino), llegó a idealizar a Jesús y a amarlo con un afecto tierno y profundo, como padre y como hermano.

\par 
%\textsuperscript{(1404.2)}
\textsuperscript{127:6.3} Aunque difícilmente podían permitírselo, Jesús tenía un extraño deseo de ir a Jerusalén para la Pascua. Conociendo su reciente experiencia con Rebeca, su madre lo animó sabiamente a que hiciera el viaje. Sin ser muy consciente de ello, lo que Jesús más deseaba era tener la oportunidad de hablar con Lázaro y visitar a Marta y María. Después de su propia familia, estas tres personas eran las que más amaba.

\par 
%\textsuperscript{(1404.3)}
\textsuperscript{127:6.4} En este viaje a Jerusalén fue por el camino de Meguido, Antípatris y Lida, recorriendo en parte la misma ruta que atravesó cuando fue traído a Nazaret a su regreso de Egipto. Empleó cuatro días para llegar a la Pascua y reflexionó mucho sobre los acontecimientos del pasado que se habían producido en Meguido y sus alrededores, el campo de batalla internacional de Palestina.

\par 
%\textsuperscript{(1404.4)}
\textsuperscript{127:6.5} Jesús atravesó Jerusalén, deteniéndose solamente para contemplar el templo y las multitudes de visitantes. Sentía una extraña y creciente aversión por este templo construido por Herodes, con sus sacerdotes elegidos por razones políticas. Lo que deseaba por encima de todo era ver a Lázaro, Marta y María. Lázaro tenía la misma edad que Jesús y ahora era el cabeza de familia; en el momento de esta visita, la madre de Lázaro había fallecido también. Marta era poco más de un año mayor que Jesús, mientras que María era dos años más joven. Y Jesús era el ideal que los tres idolatraban.

\par 
%\textsuperscript{(1404.5)}
\textsuperscript{127:6.6} Durante esta visita se produjo una de sus manifestaciones periódicas de rebelión contra la tradición ---la expresión de un resentimiento contra aquellas prácticas ceremoniales que Jesús consideraba que representaban falsamente a su Padre celestial. Al no saber que Jesús iba a venir, Lázaro se había preparado para celebrar la Pascua con unos amigos en un pueblo vecino, más abajo en el camino de Jericó. Jesús proponía ahora que celebraran la fiesta allí donde estaban, en la casa de Lázaro. <<Pero>>, dijo Lázaro, <<no tenemos cordero pascual>>. Entonces Jesús emprendió una disertación prolongada y convincente para mostrar que el Padre celestial no se interesaba realmente por aquellos rituales infantiles y desprovistos de sentido. Después de una oración ferviente y solemne, se levantaron y Jesús dijo: <<Dejad que las mentes infantiles e ignorantes de mi pueblo sirvan a su Dios como Moisés ordenó; es mejor que lo hagan. Pero nosotros, que hemos visto la luz de la vida, dejemos de acercarnos a nuestro Padre a través de las tinieblas de la muerte. Seamos libres al conocer la verdad del amor eterno de nuestro Padre>>.

\par 
%\textsuperscript{(1404.6)}
\textsuperscript{127:6.7} Aquella tarde, a la hora del crepúsculo, los cuatro se sentaron y participaron en la primera fiesta de la Pascua que unos judíos piadosos hubieran celebrado nunca sin cordero pascual. El pan ácimo y el vino habían sido preparados para esta Pascua, y Jesús sirvió a sus compañeros estos símbolos, llamándolos <<el pan de la vida>> y el <<agua de la vida>>. Comieron en solemne conformidad con las enseñanzas que acababan de impartirse. Jesús adquirió la costumbre de practicar este rito sacramental en cada una de sus visitas posteriores a Betania. Cuando volvió a su casa, se lo contó todo a su madre. Ésta se escandalizó al principio, pero gradualmente fue comprendiendo su punto de vista; sin embargo, se sintió muy aliviada cuando Jesús le aseguró que no tenía la intención de introducir en su familia esta nueva idea de la Pascua. Año tras año continuó comiendo la Pascua con los niños en el hogar <<según la ley de Moisés>>\footnote{\textit{Comer la Pascua según la ``ley de Moisés''}: Ex 12:1-28.}.

\par 
%\textsuperscript{(1404.7)}
\textsuperscript{127:6.8} Fue durante este año cuando María tuvo una larga conversación con Jesús acerca del matrimonio. Le preguntó francamente si se casaría en el caso de que estuviera libre de sus responsabilidades familiares. Jesús le explicó que, puesto que el deber inmediato le impedía el matrimonio, había pensado poco en este tema. Se expresó como dudando de que llegara a casarse nunca; dijo que todas estas cosas tenían que esperar <<mi hora>>, el momento en que <<el trabajo de mi Padre tendrá que empezar>>. Habiendo decidido ya mentalmente que no iba a ser padre de hijos carnales, dedicó muy poco tiempo a pensar en el tema del matrimonio humano.

\par 
%\textsuperscript{(1405.1)}
\textsuperscript{127:6.9} Este año reemprendió la tarea de unir más su naturaleza humana y su naturaleza divina en una \textit{individualidad humana} sencilla y eficaz. Su estado moral y su comprensión espiritual continuaron creciendo.

\par 
%\textsuperscript{(1405.2)}
\textsuperscript{127:6.10} Aunque todas sus propiedades de Nazaret (a excepción de su casa) se habían vendido, este año recibieron una pequeña ayuda financiera por la venta de una participación en una propiedad de Cafarnaúm. Esto era lo último que quedaba de todos los bienes de José. Este trato inmobiliario en Cafarnaúm se efectuó con un constructor de barcas llamado Zebedeo.

\par 
%\textsuperscript{(1405.3)}
\textsuperscript{127:6.11} José terminó sus estudios este año en la escuela de la sinagoga y se preparó para empezar a trabajar en el pequeño banco del taller de carpintería de su domicilio. Aunque la herencia de su padre se había agotado, las perspectivas de salir de la pobreza habían mejorado, porque ahora eran tres los que trabajaban con regularidad.

\par 
%\textsuperscript{(1405.4)}
\textsuperscript{127:6.12} Jesús se hace hombre rápidamente, no simplemente un hombre joven sino un adulto. Ha aprendido bien a llevar sus responsabilidades. Sabe cómo seguir adelante ante los contratiempos. Resiste con valentía cuando sus planes se contrarían y sus proyectos se frustran temporalmente. Ha aprendido a ser equitativo y justo incluso en presencia de la injusticia. Está aprendiendo a ajustar sus ideales de vida espiritual con las exigencias prácticas de la existencia terrestre. Está aprendiendo a hacer planes para alcanzar una meta idealista superior y distante, mientras trabaja duramente con el fin de satisfacer las necesidades más cercanas e inmediatas. Está adquiriendo con firmeza el arte de ajustar sus aspiraciones a las exigencias convencionales de las circunstancias humanas. Casi ha dominado la técnica de utilizar la energía del impulso espiritual para mover el mecanismo de las realizaciones materiales. Aprende lentamente a vivir la vida celestial mientras continúa con su existencia terrenal. Depende cada vez más de las directrices finales de su Padre celestial, mientras que asume el papel paternal de orientar y dirigir a los niños de su familia terrestre. Se está volviendo experto en el arte de arrancar la victoria de las mismas garras de la derrota; está aprendiendo a transformar las dificultades del tiempo en triunfos de la eternidad.

\par 
%\textsuperscript{(1405.5)}
\textsuperscript{127:6.13} Así, a medida que pasan los años, este joven de Nazaret continúa experimentando la vida tal como se vive en la carne mortal en los mundos del tiempo y del espacio. Vive en Urantia una vida completa, representativa y plena. Dejó este mundo conociendo bien la experiencia que sus criaturas atraviesan durante los cortos y arduos años de su primera vida, la vida en la carne. Y toda esta experiencia humana es propiedad eterna del Soberano del Universo. Él es nuestro hermano comprensivo, nuestro amigo compasivo, nuestro soberano experimentado y nuestro padre misericordioso.

\par 
%\textsuperscript{(1405.6)}
\textsuperscript{127:6.14} Siendo niño acumuló un enorme conjunto de conocimientos; cuando joven ordenó, clasificó y correlacionó esta información. Ahora como hombre del mundo, empieza a organizar estas posesiones mentales con vistas a utilizarlas en su futura enseñanza, ministerio y servicio para sus compañeros mortales de este mundo y de todas las demás esferas habitadas de todo el universo de Nebadon.

\par 
%\textsuperscript{(1405.7)}
\textsuperscript{127:6.15} Nacido en el mundo como un niño del planeta, ha vivido su vida infantil y ha pasado por las etapas sucesivas de la adolescencia y de la juventud. Ahora se encuentra en el umbral de la plena edad adulta, con la rica experiencia de la vida humana, con la comprensión completa de la naturaleza humana y lleno de compasión por las flaquezas de la naturaleza humana. Se está volviendo experto en el arte divino de revelar su Padre Paradisiaco a las criaturas mortales de todas las edades y de todas las etapas.

\par 
%\textsuperscript{(1406.1)}
\textsuperscript{127:6.16} Ahora, como un hombre en posesión de todas sus facultades ---como un adulto del mundo--- se prepara para continuar su misión suprema de revelar Dios a los hombres y de conducir los hombres a Dios.


\chapter{Documento 128. Los primeros años de la vida adulta de Jesús}
\par 
%\textsuperscript{(1407.1)}
\textsuperscript{128:0.1} CUANDO Jesús de Nazaret comenzó los primeros años de su vida adulta, había vivido, y continuaba viviendo, una vida humana normal y corriente en la Tierra\footnote{\textit{Jesús vivió una vida humana}: Heb 2:14-18; 4:15.}. Jesús vino a este mundo exactamente como los demás niños; no tuvo nada que ver en la elección de sus padres. Había escogido este mundo concreto como planeta para llevar a cabo su séptima y última donación, su encarnación en la similitud de la carne mortal; pero aparte de esto, vino al mundo de una manera natural, creció como un niño del planeta y luchó contra las vicisitudes de su entorno de la misma manera que lo hacen los demás mortales en este mundo y en los mundos similares.

\par 
%\textsuperscript{(1407.2)}
\textsuperscript{128:0.2} Tened siempre presente que la donación de Miguel en Urantia tenía una doble finalidad:

\par 
%\textsuperscript{(1407.3)}
\textsuperscript{128:0.3} 1. Comprender en todos sus detalles la experiencia de vivir\footnote{\textit{Jesús ganó la experiencia de vivir como hombre}: Heb 2:9-10.} la vida completa de una criatura humana en la carne mortal, para consumar su soberanía en Nebadon.

\par 
%\textsuperscript{(1407.4)}
\textsuperscript{128:0.4} 2. Revelar el Padre Universal a los habitantes mortales de los mundos del tiempo y del espacio, y conducir con más eficacia a estos mismos mortales a comprender mejor al Padre Universal.

\par 
%\textsuperscript{(1407.5)}
\textsuperscript{128:0.5} Todos los demás beneficios para las criaturas y ventajas para el universo eran adicionales y secundarios ante estas metas principales de la donación como mortal.

\section*{1. El vigésimo primer año (año 15 d. de J.C.)}
\par 
%\textsuperscript{(1407.6)}
\textsuperscript{128:1.1} Al llegar a la edad adulta, Jesús emprendió seriamente y con plena conciencia de sí mismo la tarea de completar la experiencia de conocer a fondo la vida de las formas más humildes de sus criaturas inteligentes; así adquiriría el derecho definitivo y completo a gobernar de manera incondicional el universo que él mismo había creado. Emprendió esta inmensa tarea con una conciencia total de su doble naturaleza. Pero ya había combinado eficazmente estas dos naturalezas en una sola ---la de Jesús de Nazaret.

\par 
%\textsuperscript{(1407.7)}
\textsuperscript{128:1.2} Josué ben José sabía muy bien que era un hombre, un hombre mortal, nacido de una mujer. Esto queda demostrado en la elección de su primera denominación, el \textit{Hijo del Hombre}\footnote{\textit{Hijo del Hombre}: Mt 8:20.}. Compartió realmente la naturaleza de carne y hueso, e incluso ahora que preside con autoridad soberana los destinos de un universo, conserva todavía entre sus numerosos títulos bien ganados el de Hijo del Hombre\footnote{\textit{Hijo del Hombre}: Ez 2:1,3,6,8; 3:1-4,10,17; Dn 7:13-14; Mc 2:10; Lc 5:24; Jn 1:51; Ap 1:13; 14:14; 1 Hen 46:1-6; 48:2; 60:10; 62:1-14; 63:11; 69:26-29; 70:1-2; 71:14-16.}. Es literalmente cierto que el Verbo creador ---el Hijo Creador--- del Padre Universal <<se hizo carne y habitó en Urantia como un hombre del mundo>>\footnote{\textit{El Verbo se hizo carne}: Jn 1:14.}. Trabajaba, se cansaba, descansaba y dormía. Tuvo hambre y sació su apetito con alimentos; tuvo sed y apagó su sed con agua. Experimentó toda la gama de sentimientos y emociones humanas; fue <<probado en todas las cosas de la misma manera que vosotros>>\footnote{\textit{Fue probado en todas las cosas como vosotros}: Heb 4:15.}, sufrió y murió.

\par 
%\textsuperscript{(1407.8)}
\textsuperscript{128:1.3} Obtuvo conocimientos, adquirió experiencia y combinó ambas cosas en sabiduría, como lo hacen otros mortales del mundo. Hasta después de su bautismo no utilizó ningún poder sobrenatural. No empleó ninguna influencia que no formara parte de su dotación humana como hijo de José y de María.

\par 
%\textsuperscript{(1408.1)}
\textsuperscript{128:1.4} En cuanto a los atributos de su existencia prehumana, se despojó de ellos. Antes de empezar su trabajo público, se impuso a sí mismo conocer a los hombres y los acontecimientos exclusivamente por medios humanos. Era un verdadero hombre entre los hombres.

\par 
%\textsuperscript{(1408.2)}
\textsuperscript{128:1.5} Es una verdad eterna y gloriosa que: <<Tenemos un alto gobernante que puede conmoverse con el sentimiento de nuestras debilidades. Tenemos un Soberano que fue, en todos los aspectos, probado y tentado como nosotros, pero sin pecar>>\footnote{\textit{Conmoverse con el sentimiento de nuestras debilidades}: Heb 4:15.}. Y puesto que él mismo sufrió, habiendo sido probado y tentado, es perfectamente capaz de comprender y ayudar a los que se encuentran confundidos y afligidos.

\par 
%\textsuperscript{(1408.3)}
\textsuperscript{128:1.6} El carpintero de Nazaret comprendía ahora plenamente el trabajo que le esperaba, pero escogió dejar que su vida humana continuara su curso natural. En algunas de estas cuestiones es realmente un ejemplo para sus criaturas mortales, pues tal como está escrito: <<Tened dentro de vosotros el mismo espíritu que tenía también Cristo Jesús, el cual, siendo de la naturaleza de Dios, no consideraba extraño ser igual a Dios. Sin embargo, se dio poca importancia, y tomando la forma de una criatura, nació en la similitud de los hombres. Habiendo sido moldeado así como un hombre, se humilló y se hizo obediente hasta la muerte, incluso hasta la muerte en la cruz>>\footnote{\textit{Tened dentro de vosotros el mismo espíritu que tenía Jesús}: Flp 2:5-8.}.

\par 
%\textsuperscript{(1408.4)}
\textsuperscript{128:1.7} Vivió su vida mortal exactamente como todos los miembros de la familia humana pueden vivir la suya, como <<aquel que en los días de su encarnación elevaba con tanta frecuencia oraciones y súplicas, incluso con una gran emoción y lágrimas, a Aquel que es capaz de salvar de todo mal, y sus oraciones fueron eficaces porque creía>>\footnote{\textit{Jesús elevaba sus oraciones}: Heb 5:7.}. Por este motivo era necesario que se volviera \textit{en todos los aspectos} semejante a sus hermanos, para poder llegar a ser un soberano misericordioso y comprensivo para ellos.

\par 
%\textsuperscript{(1408.5)}
\textsuperscript{128:1.8} Nunca dudó de su naturaleza humana; era evidente por sí misma y siempre estaba presente en su conciencia. En cuanto a su naturaleza divina, siempre había lugar para las dudas y las conjeturas; al menos fue así hasta el acontecimiento que se produjo en su bautismo. La autoconciencia de su divinidad fue una lenta revelación, y desde el punto de vista humano, una revelación evolutiva natural. Esta revelación y esta autoconciencia de su divinidad empezaron en Jerusalén con el primer acontecimiento sobrenatural de su existencia humana, cuando aún no tenía trece años. La experiencia de realizar esta autoconciencia de su naturaleza divina se completó en el momento de la segunda experiencia sobrenatural de su encarnación; este episodio se produjo cuando Juan lo bautizó en el Jordán, acontecimiento que marcó el principio de su carrera pública de servicio y de enseñanza.

\par 
%\textsuperscript{(1408.6)}
\textsuperscript{128:1.9} Entre estas dos visitas celestiales, una a los trece años y la otra en su bautismo, no ocurrió nada sobrenatural ni sobrehumano en la vida de este Hijo Creador encarnado. A pesar de esto, el niño de Belén, el muchacho, el joven y el hombre de Nazaret, eran en realidad el Creador encarnado de un universo; pero en el transcurso de su vida humana hasta el día en que Juan lo bautizó, nunca utilizó ni una sola vez este poder, ni siguió las directrices de personalidades celestiales, exceptuando las de su serafín guardián. Nosotros que atestiguamos esto sabemos lo que decimos.

\par 
%\textsuperscript{(1408.7)}
\textsuperscript{128:1.10} Sin embargo, durante todos estos años de su vida en la carne, era realmente divino. Era en efecto un Hijo Creador del Padre Paradisiaco. Una vez que emprendió su carrera pública, después de completar técnicamente su experiencia puramente mortal para adquirir la soberanía, no dudó en admitir públicamente que era el Hijo de Dios. No dudó en declarar: <<Yo soy el Alfa y la Omega, el principio y el fin, el primero y el último>>\footnote{\textit{Yo soy el Alfa y la Omega}: Ap 1:8,11,17; 21:6; 22:13. \textit{El primero y el último}: Is 41:4,6; 44:6; 48:12; Ap 2:8.}. Años más tarde, no protestó cuando le llamaron Señor de la Gloria\footnote{\textit{Señor de la Gloria}: 1 Co 2:8; Stg 2:1.}, Gobernante de un Universo, el Señor Dios de toda la creación\footnote{\textit{Señor Dios de toda la creación}: Zac 6:5; Hch 10:36; Jos 3:11,13.}, el Santo de Israel\footnote{\textit{Santo de Israel}: 2 Re 19:22; Sal 71:22; Is 10:20; Jer 51:5.}, el Señor de todo, nuestro Señor y nuestro Dios\footnote{\textit{Nuestro Señor y nuestro Dios}: Jn 20:28.}, Dios con nosotros\footnote{\textit{Dios con nosotros}: Mt 1:23.}, el que tiene un nombre por encima de todos los nombres y en todos los mundos\footnote{\textit{El que tiene un nombre por encima de todos los nombres}: Flp 2:9.}, la Omnipotencia de un universo\footnote{\textit{La Omnipotencia}: Mt 28:18.}, la Mente Universal de esta creación, el Único en el que están ocultos todos los tesoros de la sabiduría y del conocimiento\footnote{\textit{Tesoros de la sabiduría y del conocimiento}: Ro 11:33; Col 2:3.}, la plenitud de Aquel que llena todas las cosas\footnote{\textit{Plenitud de Aquel que llena todas las cosas}: Ef 1:23.}, el Verbo eterno del Dios eterno\footnote{\textit{Verbo eterno del Dios eterno}: Jn 1:1.}, Aquel que era antes de todas las cosas y en quien todas las cosas consisten\footnote{\textit{Aquel que era antes de todas las cosas y en quien todas las cosas consisten}: Col 1:17.}, el Creador de los cielos y de la Tierra\footnote{\textit{Creador de los cielos y de la Tierra}: Gn 1:1; 2:4; Ex 20:11; 31:17; 2 Re 19:15; 2 Cr 2:12; Neh 9:6; Sal 115:15-16; 121:2; 124:8; 134:3; 146:6; Is 37:16; 42:5; 45:12,18; Jer 10:11-12; 32:17; 51:15-16; Hch 4:24; 14:15; Col 1:16; Ap 10:6; 14:7.}, el Sostén de un universo\footnote{\textit{Sostén de un universo}: Sal 37:17,24; 63:8; 145:14; Is 41:10,13; Heb 1:3.}, el Juez de toda la Tierra\footnote{\textit{Juez de toda la Tierra}: Gn 18:25; 1 Cr 16:33; Sal 94:2; 1 Sam 2:10.}, el Dador de la vida eterna\footnote{\textit{Dador de la vida eterna}: Dn 12:2; Mt 19:16,29; 25:46; Mc 10:17,30; Lc 10:25; 18:18,30; Jn 3:15-16,36; 4:14,36; 5:24,39; 6:27,49,47; 6:54,68; 8:51-52; 10:28; 11:25-26; 12:25,50; 17:2,3; Hch 13:46-48; Ro 2:7; 5:21; 6:22-23; Gl 6:8; 1 Ti 1:16; 6:12:19; Tit 1:2; 3:7; 1 Jn 1:2; 2:25; 3:15; 5:11,13,20; Jud 1:21; Ap 22:5.}, el Verdadero Pastor\footnote{\textit{Verdadero Pastor}: Sal 23:1; Jn 10:11,14; Heb 13:20.}, el Libertador de los mundos\footnote{\textit{Libertador de los mundos}: Sal 18:2; 2 Sam 22:2.} y el que Dirige nuestra salvación\footnote{\textit{El que dirige nuestra salvación}: Heb 2:10.}.

\par 
%\textsuperscript{(1409.1)}
\textsuperscript{128:1.11} Nunca puso objeción a ninguno de estos títulos cuando les fueron aplicados, después de emerger de su vida puramente humana para entrar en los años siguientes en los que tenía conciencia del ministerio de la divinidad en la humanidad, por la humanidad y para la humanidad, en este mundo y para todos los otros mundos. Jesús sólo puso objeción a un título que le aplicaron: cuando una vez le llamaron Emmanuel, simplemente replicó: <<No soy yo, es mi hermano mayor>>.

\par 
%\textsuperscript{(1409.2)}
\textsuperscript{128:1.12} Siempre, e incluso después de emerger a una vida más amplia en la Tierra, Jesús permaneció humildemente sometido a la voluntad del Padre que está en los cielos.

\par 
%\textsuperscript{(1409.3)}
\textsuperscript{128:1.13} Después de su bautismo, no tuvo inconveniente en permitir que los que creían sinceramente en él y sus seguidores agradecidos lo adoraran. Incluso cuando luchaba contra la pobreza y trabajaba con sus manos para proporcionar las necesidades básicas a su familia, su conciencia de ser un Hijo de Dios iba en aumento; sabía que era el autor de los cielos y de esta misma Tierra en la que ahora estaba viviendo su existencia humana. Y las huestes de seres celestiales de todo el enorme universo que lo observaba sabían igualmente que este hombre de Nazaret era su amado Soberano y su padre-Creador. Durante todos estos años, el universo de Nebadon permaneció en una profunda expectativa; todas las miradas celestiales estaban clavadas continuamente en Urantia ---en Palestina.

\par 
%\textsuperscript{(1409.4)}
\textsuperscript{128:1.14} Este año, Jesús se desplazó con José a Jerusalén para celebrar la Pascua. Como ya había llevado a Santiago al templo para la consagración, pensaba que tenía el deber de llevar a José. Jesús nunca mostró el menor grado de predilección en el trato con su familia. Fue con José a Jerusalén por la ruta habitual del valle del Jordán, pero regresó a Nazaret por el camino que pasaba por Amatus, al este del Jordán. Al bajar por el Jordán, Jesús le contó a José la historia judía, y en el viaje de vuelta, le habló de las experiencias de las famosas tribus de Rubén, Gad y Gilead que tradicionalmente habían vivido en estas regiones al este del río.

\par 
%\textsuperscript{(1409.5)}
\textsuperscript{128:1.15} José hizo muchas preguntas capitales a Jesús en relación con la misión de su vida, pero a la mayoría de ellas, Jesús se limitó a responder: <<Mi hora aún no ha llegado>>\footnote{\textit{Mi hora aún no ha llegado}: Jn 2:4; 7:30; 8:20.}. Sin embargo, en el transcurso de estas discusiones, Jesús dejó caer muchas palabras que José recordó durante los acontecimientos sensacionales de los años siguientes. Jesús pasó esta Pascua, acompañado de José, con sus tres amigos en Betania, como tenía la costumbre de hacer cuando estaba en Jerusalén asistiendo a estas fiestas conmemorativas.

\section*{2. El vigésimo segundo año (año 16 d. de J.C.)}
\par 
%\textsuperscript{(1409.6)}
\textsuperscript{128:2.1} Éste fue uno de los años durante los cuales los hermanos y hermanas de Jesús se enfrentaron con las pruebas y tribulaciones propias de los problemas y reajustes de la adolescencia. Jesús tenía ahora hermanos y hermanas entre los siete y los dieciocho años de edad, y estaba muy ocupado ayudándolos a adaptarse a los nuevos despertares de su vida intelectual y emocional. Así pues, tuvo que luchar con los problemas de la adolescencia a medida que se presentaban en la vida de sus hermanos y hermanas menores.

\par 
%\textsuperscript{(1410.1)}
\textsuperscript{128:2.2} Simón terminó sus estudios en la escuela este año y empezó a trabajar con Jacobo el albañil, el antiguo compañero de juegos de la infancia y el defensor siempre dispuesto de Jesús. Después de varias conversaciones familiares, llegaron a la conclusión de que no era prudente que todos los muchachos se dedicaran a la carpintería. Pensaban que si escogían oficios diferentes estarían en disposiciones de aceptar contratos para construir edificios enteros. Además, habían pasado por períodos de paro forzoso desde que tres de ellos trabajaban como carpinteros a jornada completa.

\par 
%\textsuperscript{(1410.2)}
\textsuperscript{128:2.3} Jesús continuó este año con la terminación de interiores y la ebanistería, pero dedicó la mayor parte de su tiempo al taller de reparaciones de las caravanas. Santiago empezaba a alternarse con él en el servicio del taller. Hacia finales de este año, cuando el trabajo de carpintería llegó a escasear en Nazaret, Jesús dejó a Santiago a cargo del taller de reparaciones y a José en el banco de carpintero de la casa, mientras que él se fue a Séforis para trabajar con un herrero. Estuvo trabajando seis meses en el metal y adquirió una habilidad considerable en el yunque.

\par 
%\textsuperscript{(1410.3)}
\textsuperscript{128:2.4} Antes de empezar en su nuevo empleo de Séforis, Jesús mantuvo una de sus conferencias familiares periódicas y nombró solemnemente a Santiago, que acababa de cumplir dieciocho años, como cabeza de familia. Prometió a su hermano un apoyo sincero y toda su cooperación, y exigió a cada miembro de la familia la promesa formal de obedecer a Santiago. A partir de este día, Santiago asumió toda la responsabilidad financiera de la familia, y Jesús entregaba a su hermano su paga semanal. Jesús nunca más recuperó de Santiago las riendas del hogar. Mientras trabajaba en Séforis podría haber regresado cada noche al hogar si hubiera sido necesario, pero permaneció ausente a propósito, echándole la culpa al tiempo y a otras causas, aunque su verdadero motivo era preparar a Santiago y a José para llevar la responsabilidad de la familia. Había empezado el lento proceso de separarse de su familia. Jesús volvía a Nazaret todos los sábados y a veces durante la semana cuando lo exigía la ocasión, para observar cómo funcionaba el nuevo plan, ofrecer consejos y aportar sugerencias útiles.

\par 
%\textsuperscript{(1410.4)}
\textsuperscript{128:2.5} El hecho de vivir la mayoría del tiempo en Séforis durante seis meses, proporcionó a Jesús una nueva oportunidad para conocer mejor el punto de vista que tenían los gentiles sobre la vida. Trabajó con ellos, vivió con ellos y de todas las maneras posibles estudió de cerca y con sumo cuidado los hábitos de vida y la mentalidad de los gentiles.

\par 
%\textsuperscript{(1410.5)}
\textsuperscript{128:2.6} Los niveles morales de esta ciudad natal de Herodes Antipas eran muy inferiores a los de incluso la zona para las caravanas de Nazaret, de tal manera que después de permanecer seis meses en Séforis, Jesús no dudó en encontrar un pretexto para regresar a Nazaret. El grupo para el que trabajaba iba a emprender unas obras públicas tanto en Séforis como en la nueva ciudad de Tiberiades, y Jesús estaba poco dispuesto a asumir cualquier tipo de empleo que estuviera bajo la supervisión de Herodes Antipas. También existían otras razones que hacían prudente, en opinión de Jesús, el regresar a Nazaret. Cuando volvió al taller de reparaciones, no asumió otra vez la dirección personal de los asuntos familiares. Trabajó en el taller en asociación con Santiago y, tanto como le fue posible, le permitió continuar supervisando el hogar. La gestión de los gastos familiares y la administración del presupuesto doméstico, que estaban en manos de Santiago, no sufrieron ningún cambio.

\par 
%\textsuperscript{(1410.6)}
\textsuperscript{128:2.7} Fue precisamente mediante esta planificación sabia y cuidadosa como Jesús preparó el camino para su retirada final de toda participación activa en los asuntos de su familia. Cuando Santiago tuvo dos años de experiencia como cabeza de familia ---y dos años antes de que se casara--- José fue encargado de los fondos de la casa y se le confió la dirección general del hogar.

\section*{3. El vigésimo tercer año (año 17 d. de J.C.)}
\par 
%\textsuperscript{(1411.1)}
\textsuperscript{128:3.1} La presión financiera cedió este año ligeramente, ya que cuatro miembros de la familia estaban trabajando. Miriam ganaba bastante con la venta de la leche y la mantequilla; Marta se había convertido en una tejedora experta. Habían pagado más de un tercio del precio de compra del taller de reparaciones. La situación era tal que Jesús dejó de trabajar durante tres semanas para llevar a Simón a la Pascua de Jerusalén; éste era el período más largo, libre de las faenas cotidianas, que había disfrutado desde la muerte de su padre.

\par 
%\textsuperscript{(1411.2)}
\textsuperscript{128:3.2} Viajaron a Jerusalén por el camino de la Decápolis y atravesaron Pella, Gerasa, Filadelfia, Hesbón y Jericó. Regresaron a Nazaret por la ruta costera, pasando por Lida, Jope, Cesarea, y desde allí, rodeando el Monte Carmelo, fueron a Tolemaida y Nazaret. Este viaje permitió a Jesús conocer bastante bien toda Palestina al norte de la región de Jerusalén.

\par 
%\textsuperscript{(1411.3)}
\textsuperscript{128:3.3} En Filadelfia, Jesús y Simón conocieron a un mercader de Damasco que experimentó tanta simpatía por los hermanos de Nazaret, que insistió para que se detuvieran con él en su sede de Jerusalén. Mientras Simón asistía al templo, Jesús pasó mucho tiempo conversando con este hombre de mundo bien educado y bastante viajero. Este mercader poseía más de cuatro mil camellos de caravanas; tenía intereses en todo el mundo romano y ahora estaba de camino hacia Roma. Le propuso a Jesús que viniera a Damasco para trabajar en su negocio de importaciones de oriente, pero Jesús le explicó que no tenía justificación para alejarse tanto de su familia en ese momento. Sin embargo, durante el camino de vuelta pensó mucho en aquellas ciudades lejanas y en los países aún más distantes del Lejano Occidente y del Lejano Oriente, países de los que había oído hablar con tanta frecuencia a los viajeros y conductores de las caravanas.

\par 
%\textsuperscript{(1411.4)}
\textsuperscript{128:3.4} Simón disfrutó mucho de su visita a Jerusalén. Fue admitido debidamente en la comunidad de Israel\footnote{\textit{Comunidad de Israel}: Ef 2:12.} durante la consagración pascual de los nuevos hijos del mandamiento. Mientras Simón asistía a las ceremonias pascuales, Jesús se mezcló con las multitudes de visitantes y emprendió muchas conversaciones personales interesantes con numerosos prosélitos gentiles.

\par 
%\textsuperscript{(1411.5)}
\textsuperscript{128:3.5} El más notable de todos estos contactos fue quizás con un joven helenista llamado Esteban. Este joven visitaba Jerusalén por primera vez y se encontró casualmente con Jesús el jueves por la tarde de la semana de la Pascua. Mientras los dos paseaban contemplando el palacio asmoneo, Jesús inició una conversación fortuita que tuvo como resultado el sentirse interesados el uno por el otro, lo que les llevó a una discusión de cuatro horas sobre la manera de vivir y el verdadero Dios y su culto. Esteban se quedó enormemente impresionado por lo que Jesús le dijo, y nunca olvidó sus palabras.

\par 
%\textsuperscript{(1411.6)}
\textsuperscript{128:3.6} Este mismo Esteban es el que posteriormente se hizo creyente en las enseñanzas de Jesús, y cuya intrepidez predicando este evangelio incipiente provocó la ira de los judíos, que lo apedrearon hasta morir\footnote{\textit{Apedreamiento de Esteban}: Hch 6:8-7:60.}. Una parte de la extraordinaria audacia de Esteban proclamando su visión del nuevo evangelio provenía directamente de esta primera conversación con Jesús. Pero Esteban nunca tuvo la menor sospecha de que el galileo con quien había hablado unos quince años antes era precisamente la misma persona que más tarde proclamaría como Salvador del mundo, y por quien tan pronto daría su vida, convirtiéndose así en el primer mártir de la nueva fe cristiana en evolución. Cuando Esteban dio su vida como precio por su ataque al templo judío y a sus prácticas tradicionales, un tal Saulo\footnote{\textit{Saulo de Tarso}: Hch 7:58.}, ciudadano de Tarso, se hallaba presente. Cuando Saulo vio cómo este griego podía morir por su fe, se despertaron en su corazón unos sentimientos que finalmente le llevaron a abrazar la causa por la que había muerto Esteban; más tarde se convirtió en el dinámico e indomable Pablo, el filósofo, si no el único fundador, de la religión cristiana.

\par 
%\textsuperscript{(1412.1)}
\textsuperscript{128:3.7} El domingo después de la semana pascual, Simón y Jesús emprendieron su viaje de regreso a Nazaret. Simón no olvidó nunca lo que Jesús le enseñó en este viaje. Siempre había amado a Jesús, pero ahora sentía que había empezado a conocer a su hermano-padre. Tuvieron muchas conversaciones íntimas y confidenciales mientras atravesaban el país y preparaban sus comidas al borde del camino. Llegaron a la casa el jueves a mediodía, y aquella noche Simón mantuvo despierta a la familia hasta tarde, contándoles sus experiencias.

\par 
%\textsuperscript{(1412.2)}
\textsuperscript{128:3.8} María se quedó trastornada cuando Simón le informó que Jesús había pasado la mayor parte del tiempo en Jerusalén <<conversando con los extranjeros, especialmente de los países lejanos>>. La familia de Jesús nunca pudo comprender su gran interés por la gente, su necesidad de hablar con ellos, de conocer su manera de vivir y de averiguar lo que pensaban.

\par 
%\textsuperscript{(1412.3)}
\textsuperscript{128:3.9} La familia de Nazaret estaba cada vez más absorbida por sus problemas inmediatos y humanos; no se mencionaba con frecuencia la futura misión de Jesús, y él mismo hablaba raras veces de su carrera futura. Su madre no se acordaba mucho de que era un hijo de la promesa. Poco a poco iba abandonando la idea de que Jesús tenía que cumplir una misión divina en la Tierra, pero a veces su fe se reavivaba cuando se detenía a recordar la visita de Gabriel antes de que el niño naciera.

\section*{4. El episodio de Damasco}
\par 
%\textsuperscript{(1412.4)}
\textsuperscript{128:4.1} Jesús pasó los cuatro últimos meses de este año en Damasco, como huésped del mercader que conoció por primera vez en Filadelfia, cuando iba camino de Jerusalén. Un representante de este mercader había buscado a Jesús al pasar por Nazaret y lo acompañó hasta Damasco. Este mercader, en parte judío, propuso consagrar una enorme cantidad de dinero para establecer una escuela de filosofía religiosa en Damasco. Proyectaba crear un centro de estudios que sobrepasara al de Alejandría. Le propuso a Jesús que emprendiera inmediatamente una larga gira por los centros de educación del mundo, como paso previo para convertirse en el director de este nuevo proyecto. Ésta fue una de las mayores tentaciones con las que Jesús tuvo que enfrentarse en el transcurso de su carrera puramente humana.

\par 
%\textsuperscript{(1412.5)}
\textsuperscript{128:4.2} Poco después, este mercader trajo ante Jesús a un grupo de doce mercaderes y banqueros que aceptaban financiar esta escuela recién proyectada. Jesús manifestó un profundo interés por la escuela que proponían y les ayudó a planificar su organización, pero siempre expresó el temor de que sus otras obligaciones anteriores, sin indicar cuáles, le impedirían aceptar la dirección de una empresa tan ambiciosa. El que deseaba ser su benefactor era obstinado y empleó provechosamente a Jesús en su casa haciendo algunas traducciones, mientras que él, su esposa y sus hijos e hijas trataban de persuadirlo para que aceptara el honor que se le ofrecía. Pero no se dejó convencer. Sabía muy bien que su misión en la Tierra no debía estar sostenida por instituciones de enseñanza; sabía que no debía comprometerse en lo más mínimo, para no ser dirigido por <<asambleas de hombres>>, por muy bien intencionadas que fueran.

\par 
%\textsuperscript{(1412.6)}
\textsuperscript{128:4.3} Él, que fue rechazado por los jefes religiosos de Jerusalén incluso después de haber demostrado su autoridad, fue reconocido y recibido como maestro instructor por los empresarios y banqueros de Damasco, y todo esto cuando era un carpintero oscuro y desconocido de Nazaret.

\par 
%\textsuperscript{(1412.7)}
\textsuperscript{128:4.4} Nunca habló de esta oferta a su familia, y al final de este año se encontraba de nuevo en Nazaret cumpliendo con sus deberes cotidianos, como si nunca hubiera sido tentado por las proposiciones halagadoras de sus amigos de Damasco. Estos hombres de Damasco tampoco asociaron nunca al futuro ciudadano de Cafarnaúm, que puso boca abajo a toda la sociedad judía, con el antiguo carpintero de Nazaret que había osado rechazar el honor que sus riquezas combinadas hubieran podido procurarle.

\par 
%\textsuperscript{(1413.1)}
\textsuperscript{128:4.5} Jesús se las ingenió con gran habilidad e intencionalidad para aislar diversos episodios de su vida con el fin de que, a los ojos del mundo, nunca fueran asociados y considerados como acciones realizadas por un mismo individuo. En los años posteriores escuchó muchas veces contar esta historia del extraño galileo que declinó la oportunidad de fundar en Damasco una escuela que rivalizara con Alejandría.

\par 
%\textsuperscript{(1413.2)}
\textsuperscript{128:4.6} Al tratar de aislar ciertos aspectos de su experiencia terrestre, uno de los objetivos que Jesús perseguía era evitar la reconstrucción de una carrera tan hábil y espectacular, que incitara a las futuras generaciones a venerar al maestro en lugar de someterse a la verdad que había vivido y enseñado. Jesús no quería que la reconstrucción de una historia humana tan sobresaliente desviara la atención de sus enseñanzas. Reconoció muy pronto que sus seguidores se sentirían tentados a formular una religión \textit{acerca} de él, que podría hacerle la competencia al evangelio del reino que tenía la intención de proclamar al mundo. Por consiguiente, durante toda su carrera extraordinaria, trató de suprimir convenientemente todo aquello que, en su opinión, pudiera favorecer esta tendencia humana natural a exaltar al maestro en lugar de proclamar sus enseñanzas.

\par 
%\textsuperscript{(1413.3)}
\textsuperscript{128:4.7} Este mismo motivo explica también por qué permitió que se le conociera por medio de nombres diferentes durante las diversas épocas de su variada vida en la Tierra. Además, no quería ejercer ninguna influencia indebida sobre su familia u otras personas, para no inducirles a creer en él en contra de sus sinceras convicciones. Siempre rehusó sacar una ventaja indebida o injusta de la mente humana. No quería que los hombres creyeran en él, a menos que sus corazones fueran sensibles a las realidades espirituales reveladas en sus enseñanzas.

\par 
%\textsuperscript{(1413.4)}
\textsuperscript{128:4.8} A finales de este año, las cosas marchaban bastante bien en el hogar de Nazaret. Los niños crecían y María se iba acostumbrando a las ausencias de Jesús del hogar. Éste continuaba enviando su salario a Santiago para el sostén de la familia, reservándose sólo una pequeña parte para sus gastos personales más inmediatos.

\par 
%\textsuperscript{(1413.5)}
\textsuperscript{128:4.9} A medida que pasaban los años, resultaba más difícil darse cuenta de que este hombre era un Hijo de Dios en la Tierra. Parecía que se estaba convirtiendo en un habitante más del planeta, en un hombre entre los hombres. El Padre que está en los cielos había ordenado que la donación se desarrollara precisamente de esta manera.

\section*{5. El vigésimo cuarto año (año 18 d. de J.C.)}
\par 
%\textsuperscript{(1413.6)}
\textsuperscript{128:5.1} Éste fue el primer año en que Jesús estuvo relativamente libre de responsabilidades familiares. Santiago administraba con mucho éxito los asuntos del hogar, ayudado por los consejos y las rentas de Jesús.

\par 
%\textsuperscript{(1413.7)}
\textsuperscript{128:5.2} A la semana siguiente de la Pascua de este año, un joven de Alejandría vino hasta Nazaret para organizar un encuentro entre Jesús y un grupo de judíos de Alejandría, que se celebraría en el transcurso del año y en algún lugar de la costa de Palestina. La conferencia se fijó para mediados de junio, y Jesús se desplazó hasta Cesarea para reunirse con cinco judíos eminentes de Alejandría, que le rogaron que se estableciera en su ciudad como instructor religioso, ofreciéndole como aliciente, para empezar, el puesto de ayudante del chazan en la sinagoga principal de la ciudad.

\par 
%\textsuperscript{(1414.1)}
\textsuperscript{128:5.3} Los portavoces de esta comisión explicaron a Jesús que Alejandría estaba destinada a convertirse en el centro principal de la cultura judía para el mundo entero; que la tendencia helenista de los asuntos judíos había sobrepasado probablemente a la escuela de pensamiento babilónica. Recordaron a Jesús los siniestros rumores de rebelión que corrían por Jerusalén y toda Palestina, y le aseguraron que cualquier sublevación de los judíos palestinos equivaldría a un suicidio nacional, que la mano de hierro de Roma aplastaría la rebelión en tres meses, y que Jerusalén sería destruida y el templo demolido hasta que no quedara piedra sobre piedra.

\par 
%\textsuperscript{(1414.2)}
\textsuperscript{128:5.4} Jesús escuchó todo lo que tenían que decir, les agradeció su confianza, y al declinar su invitación para ir a Alejandría, les dijo en esencia: <<Mi hora aún no ha llegado>>\footnote{\textit{Mi hora aún no ha llegado}: Jn 2:4; 7:30; 8:20.}. Se quedaron confundidos por su aparente indiferencia al honor que habían intentado conferirle. Antes de despedirse de Jesús le ofrecieron una bolsa de dinero como muestra de la estima de sus amigos de Alejandría, y en compensación por el tiempo y los gastos de venir hasta Cesarea para hablar con ellos. Pero rehusó también el dinero, diciendo: <<La casa de José nunca ha recibido limosnas, y no podemos comernos el pan de otra persona mientras yo tenga buenos brazos y mis hermanos puedan trabajar>>.

\par 
%\textsuperscript{(1414.3)}
\textsuperscript{128:5.5} Sus amigos de Egipto se embarcaron para su tierra; años después, cuando oyeron los rumores sobre el constructor de barcas de Cafarnaúm que estaba creando tanta conmoción en Palestina, pocos de ellos imaginaron que se trataba del niño de Belén ya adulto y del mismo galileo singular que había declinado sin ningún formalismo la invitación de convertirse en un gran maestro en Alejandría.

\par 
%\textsuperscript{(1414.4)}
\textsuperscript{128:5.6} Jesús regresó a Nazaret. Los seis meses restantes de este año fueron los más tranquilos de toda su carrera. Disfrutó de este respiro temporal en su programa habitual de problemas a resolver y de dificultades a superar. Comulgó mucho con su Padre que está en los cielos e hizo enormes progresos en el dominio de su mente humana.

\par 
%\textsuperscript{(1414.5)}
\textsuperscript{128:5.7} Pero los asuntos humanos en los mundos del tiempo y del espacio no transcurren con tranquilidad durante mucho tiempo. En diciembre, Santiago tuvo una conversación privada con Jesús para explicarle que estaba muy enamorado de Esta, una joven de Nazaret, y que les gustaría casarse pronto si fuera posible. Atrajo la atención sobre el hecho de que José pronto cumpliría dieciocho años, y que sería una buena experiencia para él tener la oportunidad de servir como cabeza de familia. Jesús dio su consentimiento para que Santiago se casara dos años más tarde, siempre que durante este intervalo preparara adecuadamente a José para asumir la dirección del hogar.

\par 
%\textsuperscript{(1414.6)}
\textsuperscript{128:5.8} Entonces se produjeron otros hechos ---los esponsales estaban en el ambiente. El éxito que tuvo Santiago al obtener el consentimiento de Jesús para casarse animó a Miriam a presentarse con sus proyectos ante su hermano-padre. Jacobo, el joven albañil, antiguo defensor voluntario de Jesús y ahora socio de Santiago y José en los negocios, hacía tiempo que había intentado obtener la mano de Miriam para casarse. Después de que Miriam expuso sus planes a Jesús, éste ordenó que Jacobo viniera a verle para pedir oficialmente la mano de Miriam, y prometió su bendición al matrimonio en cuanto ella estimara que Marta estaba preparada para asumir sus deberes de hija mayor.

\par 
%\textsuperscript{(1414.7)}
\textsuperscript{128:5.9} Cuando estaba en casa, Jesús continuaba enseñando en la escuela nocturna tres veces por semana, leía a menudo las escrituras los sábados en la sinagoga, conversaba con su madre, enseñaba a los niños y se comportaba en general como un ciudadano digno y respetable de Nazaret, dentro de la comunidad de Israel.

\section*{6. El vigésimo quinto año (año 19 d. de J.C.)}
\par 
%\textsuperscript{(1415.1)}
\textsuperscript{128:6.1} Este año empezó con toda la familia de Nazaret en buena salud y fue testigo del final de la escolaridad regular de todos los niños, a excepción de algunos trabajos que Marta tenía que hacer para Rut.

\par 
%\textsuperscript{(1415.2)}
\textsuperscript{128:6.2} Jesús era uno de los ejemplares humanos más vigorosos y refinados que habían aparecido en la Tierra desde la época de Adán. Su desarrollo físico era espléndido. Su mente era activa, aguda y penetrante ---comparada con la mentalidad media de sus contemporáneos, había alcanzado proporciones gigantescas--- y su espíritu era en verdad humanamente divino.

\par 
%\textsuperscript{(1415.3)}
\textsuperscript{128:6.3} El estado financiero de la familia se encontraba en las mejores condiciones desde que se liquidaron las propiedades de José. Se habían efectuado los últimos pagos del taller de reparaciones de las caravanas; no debían nada a nadie y, por primera vez en muchos años, contaban con algunos fondos. Por todo ello, y puesto que había llevado a sus otros hermanos a Jerusalén para que participaran en sus primeras ceremonias pascuales, Jesús decidió acompañar a Judá (que acababa de terminar sus estudios en la escuela de la sinagoga) en su primera visita al templo.

\par 
%\textsuperscript{(1415.4)}
\textsuperscript{128:6.4} Fueron a Jerusalén por el valle del Jordán y regresaron por el mismo camino, porque Jesús temía que podría tener algún problema si atravesaba Samaria con su joven hermano. En Nazaret, Judá ya había tenido varias veces pequeñas dificultades a causa de su carácter impulsivo, unido a sus violentos sentimientos patrióticos.

\par 
%\textsuperscript{(1415.5)}
\textsuperscript{128:6.5} Llegaron a Jerusalén a su debido tiempo e iban de camino para efectuar una primera visita al templo, cuya sola visión había excitado y entusiasmado a Judá hasta lo más profundo de su alma, cuando se encontraron por casualidad con Lázaro de Betania. Mientras Jesús charlaba con Lázaro y trataba de arreglar las cosas para celebrar juntos la Pascua, Judá inició un incidente muy serio para todos ellos. Cerca de allí se encontraba un guardia romano que hizo unos comentarios indecorosos sobre una muchacha judía que pasaba en ese momento. Judá enrojeció de indignación y no tardó en expresar su resentimiento por esta descortesía, haciéndolo de manera directa y al alcance del oído del soldado. Los legionarios romanos eran muy sensibles a todo lo que se pareciera a una falta de respeto por parte de los judíos; así pues, el guardia arrestó inmediatamente a Judá. Esto fue demasiado para el joven patriota, y antes de que Jesús pudiera prevenirlo con una mirada de advertencia, ya había dado rienda suelta a una voluble declaración de sentimientos antirromanos reprimidos, lo que no hizo más que empeorar la situación. Judá, con Jesús a su lado, fue llevado de inmediato a la prisión militar.

\par 
%\textsuperscript{(1415.6)}
\textsuperscript{128:6.6} Jesús trató de conseguir una audiencia inmediata para Judá, o bien que lo liberaran a tiempo para poder celebrar la Pascua aquella noche, pero fracasó en sus esfuerzos. Puesto que el día siguiente era un día de <<santa asamblea>>\footnote{\textit{Santa asamblea}: Ex 12:16; Lv 23:3-8,21,37; Nm 28:18,25-26; 29:1,7,12.} en Jerusalén, ni siquiera los romanos se atrevían a oír acusaciones contra un judío. En consecuencia, Judá continuó encarcelado hasta la mañana del segundo día después de su arresto, y Jesús permaneció con él en la prisión. No estuvieron presentes en el templo en la ceremonia de recepción de los hijos de la ley como plenos ciudadanos de Israel. Judá no participó en esta ceremonia oficial hasta varios años después, cuando se encontró de nuevo en Jerusalén durante otra Pascua, en conexión con su trabajo de propaganda a favor de los celotes, la organización patriótica a la que pertenecía y en la que era muy activo.

\par 
%\textsuperscript{(1415.7)}
\textsuperscript{128:6.7} A la mañana siguiente de su segundo día en la cárcel, Jesús compareció ante el magistrado militar en nombre de Judá. Presentó sus excusas por la juventud de su hermano y efectuó una exposición aclaratoria, pero juiciosa, de la naturaleza provocativa del incidente que había llevado al arresto de su hermano. Jesús manejó el asunto de tal manera, que el magistrado expresó la opinión de que el joven judío pudiera haber tenido alguna excusa válida que justificara su violenta explosión. Después de advertir a Judá que no se atreviera otra vez a ser culpable de semejante temeridad, dijo a Jesús al despedirlos: <<Harías bien en vigilar al muchacho; es capaz de crearos muchos problemas a todos>>. El juez romano tenía razón. Judá causó muchísimos problemas a Jesús, y siempre eran de la misma naturaleza: encontronazos con las autoridades civiles a causa de sus estallidos patrióticos imprudentes e insensatos.

\par 
%\textsuperscript{(1416.1)}
\textsuperscript{128:6.8} Jesús y Judá se desplazaron hasta Betania para pasar la noche, explicaron por qué no habían acudido a la cena pascual, y al día siguiente salieron para Nazaret. Jesús no contó a la familia el arresto de su joven hermano en Jerusalén, pero unas tres semanas después de su regreso, tuvo una larga conversación con Judá sobre este incidente. Después de esta conversación con Jesús, el mismo Judá contó el suceso a la familia. Nunca olvidó la paciencia y la indulgencia que manifestó su hermano-padre durante toda esta penosa experiencia.

\par 
%\textsuperscript{(1416.2)}
\textsuperscript{128:6.9} Ésta fue la última Pascua en la que Jesús acompañó a un miembro de su propia familia. El Hijo del Hombre iba a desligarse cada vez más de los estrechos lazos que le unían a los de su propia carne y sangre.

\par 
%\textsuperscript{(1416.3)}
\textsuperscript{128:6.10} Este año, sus períodos de profunda meditación fueron interrumpidos a menudo por Rut y sus compañeros de juego. Jesús siempre estaba dispuesto a aplazar sus reflexiones sobre su trabajo futuro para el mundo y el universo, a fin de compartir la alegría infantil y el regocijo juvenil de estos jóvenes, que nunca se cansaban de escucharle contar las experiencias de sus diversos viajes a Jerusalén. También disfrutaban mucho con sus historias sobre los animales y la naturaleza.

\par 
%\textsuperscript{(1416.4)}
\textsuperscript{128:6.11} Los niños siempre eran bienvenidos al taller de reparaciones. Jesús ponía arena, pedazos de madera y piedras al lado del taller, y los niños acudían en bandadas para entretenerse allí. Cuando se cansaban de sus juegos, los más atrevidos miraban a hurtadillas dentro del taller, y si el dueño no estaba ocupado, se arriesgaban a entrar diciendo: <<Tío Josué, sal y cuéntanos un largo cuento>>. Entonces lo hacían salir tirándole de las manos hasta que se sentaba en su piedra favorita junto a la esquina del taller, con los niños sentados en semicírculo en el suelo delante de él. ¡Cómo disfrutaban estos pequeñuelos con su tío Josué! Aprendían a reírse, y a reírse con ganas. Uno o dos de los más pequeños tenían la costumbre de trepar hasta sus rodillas y se sentaban allí, contemplando embelesados las expresiones de su rostro mientras narraba sus historias. Los niños amaban a Jesús, y Jesús amaba a los niños.

\par 
%\textsuperscript{(1416.5)}
\textsuperscript{128:6.12} A sus amigos les resultaba difícil comprender la amplitud de sus actividades intelectuales, cómo podía pasar de manera tan súbita y completa de las profundas discusiones sobre la política, la filosofía o la religión, a las travesuras alegres y gozosas de estos pequeños de cinco a diez años de edad. A medida que sus propios hermanos y hermanas crecían, a medida que disponía de más tiempo libre y antes de que llegaran los nietos, prestaba una gran atención a estos pequeños. Pero no vivió suficiente tiempo en la Tierra como para disfrutar mucho de los nietos.

\section*{7. El vigésimo sexto año (año 20 d. de J.C.)}
\par 
%\textsuperscript{(1416.6)}
\textsuperscript{128:7.1} Al empezar este año, Jesús de Nazaret se volvió poderosamente consciente de que poseía un poder potencial muy extenso. Pero también estaba totalmente persuadido de que este poder no debía ser empleado por su personalidad, como Hijo del Hombre, al menos hasta que llegara su hora.

\par 
%\textsuperscript{(1417.1)}
\textsuperscript{128:7.2} Por esta época reflexionó mucho sobre sus relaciones con su Padre que está en los cielos, aunque habló poco de ello. La conclusión de todas estas reflexiones la expresó una vez en su oración en la cima de la colina, cuando dijo: <<Independientemente de quién sea yo y del poder que pueda o no ejercer, siempre he estado y siempre estaré sometido a la voluntad de mi Padre Paradisiaco>>\footnote{\textit{Sometido a la voluntad de mi Padre}: Mt 26:39,42,44; Mc 14:36,39; Lc 22:42; Jn 4:34; 5:30; 6:38-40; 15:10; 17:4.}. Sin embargo, mientras este hombre iba y venía de su trabajo por Nazaret, era literalmente cierto ---en lo que se refiere a un enorme universo--- que <<en él estaban ocultos todos los tesoros de la sabiduría y del conocimiento>>\footnote{\textit{En él estaban ocultos todos los tesoros de la sabiduría}: Col 2:3.}.

\par 
%\textsuperscript{(1417.2)}
\textsuperscript{128:7.3} Los asuntos de la familia fueron bien todo este año, excepto en lo que se refiere a Judá. Santiago tuvo dificultades durante años con su hermano menor, que no estaba dispuesto a ponerse seriamente a trabajar ni se podía contar con él para que participara en los gastos del hogar. Aunque vivía en la casa, no era consciente de que tenía que ganar su parte para el mantenimiento de la familia.

\par 
%\textsuperscript{(1417.3)}
\textsuperscript{128:7.4} Jesús era un hombre de paz, y de vez en cuando se sentía apenado por las explosiones belicosas y los numerosos arrebatos patrióticos de Judá. Santiago y José estaban a favor de echarlo de la casa, pero Jesús no quiso consentirlo. Cada vez que llegaban al límite de su paciencia, Jesús sólo les aconsejaba: <<Tened paciencia. Sed sabios en vuestros consejos y elocuentes en vuestras vidas, para que vuestro hermano menor pueda conocer primero el mejor camino, y luego se sienta obligado a seguiros en él>>. El consejo sabio y afectuoso de Jesús evitó una ruptura en la familia. Permanecieron juntos, pero Judá nunca adquirió la sensatez hasta después de casarse.

\par 
%\textsuperscript{(1417.4)}
\textsuperscript{128:7.5} María hablaba rara vez de la futura misión de Jesús. Cada vez que se mencionaba este asunto, Jesús se limitaba a contestar: <<Mi hora aún no ha llegado>>\footnote{\textit{Mi hora aún no ha llegado}: Jn 2:4; 7:30; 8:20.}. Jesús casi había terminado la difícil tarea de destetar a su familia, para que no tuvieran que depender de la presencia inmediata de su personalidad. Se estaba preparando rápidamente para el día en que podría dejar convenientemente este hogar de Nazaret y empezar el preludio más activo de su verdadero ministerio hacia los hombres.

\par 
%\textsuperscript{(1417.5)}
\textsuperscript{128:7.6} No perdáis nunca de vista el hecho de que la misión principal de Jesús en su séptima donación consistía en adquirir la experiencia de las criaturas, lograr la soberanía de Nebadon. Y al mismo tiempo que acumulaba esta experiencia misma, efectuar la revelación suprema del Padre Paradisiaco a Urantia y a todo su universo local. Concomitante con estos objetivos, también se dedicó a desenredar los complicados asuntos de este planeta en la medida en que estaban relacionados con la rebelión de Lucifer.

\par 
%\textsuperscript{(1417.6)}
\textsuperscript{128:7.7} Jesús disfrutó este año de más horas libres de lo habitual, y consagró mucho tiempo a enseñar a Santiago la administración del taller de reparaciones, y a José la dirección de los asuntos del hogar. María presentía que se estaba preparando para dejarlos. ¿Dejarlos para ir adónde? ¿Para hacer qué? Casi había abandonado la idea de que Jesús era el Mesías. No podía comprenderlo; simplemente no podía sondear el interior de su hijo primogénito.

\par 
%\textsuperscript{(1417.7)}
\textsuperscript{128:7.8} Jesús pasó este año una gran parte de su tiempo con cada uno de los miembros de su familia. Salía con ellos para dar largos y frecuentes paseos por las colinas y a través del campo. Antes de la cosecha, llevó a Judá a casa de su tío granjero al sur de Nazaret, pero Judá no se quedó mucho tiempo después de la recolección. Huyó de allí y Simón lo encontró más tarde con los pescadores en el lago. Cuando Simón lo trajo de vuelta al hogar, Jesús mantuvo una conversación con el muchacho fugitivo y, puesto que quería ser pescador, fue con él hasta Magdala y lo puso en manos de un pariente que era pescador; desde aquel momento, Judá trabajó bastante bien y con regularidad hasta que contrajo matrimonio, y continuó como pescador después de casarse.

\par 
%\textsuperscript{(1418.1)}
\textsuperscript{128:7.9} Por fin había llegado el día en que todos los hermanos de Jesús habían elegido sus oficios y se habían establecido en ellos. El escenario se estaba preparando para que Jesús abandonara el hogar.

\par 
%\textsuperscript{(1418.2)}
\textsuperscript{128:7.10} En noviembre tuvo lugar una doble boda. Santiago se casó con Esta y Miriam se casó con Jacobo. Fue realmente un feliz acontecimiento. Incluso María estaba de nuevo feliz, excepto cuando se daba cuenta, de vez en cuando, que Jesús se estaba preparando para marcharse. Sufría el peso de una gran incertidumbre. Si Jesús quisiera sentarse y hablar francamente con ella de todo esto como cuando era niño... Pero se había vuelto muy reservado y mantenía un profundo silencio sobre el futuro.

\par 
%\textsuperscript{(1418.3)}
\textsuperscript{128:7.11} Santiago y su esposa Esta se instalaron en una linda casita, regalo del padre de ella, en la parte oeste de la ciudad. Aunque Santiago continuaba manteniendo el hogar de su madre, su contribución se redujo a la mitad a causa de su matrimonio, y José fue nombrado oficialmente por Jesús como cabeza de familia. Judá enviaba ahora fielmente su contribución mensual a la casa. Los enlaces de Santiago y de Miriam ejercieron una influencia muy beneficiosa sobre Judá, y al marcharse para la zona pesquera al día siguiente de la doble boda, le aseguró a José que podía confiar en él <<para cumplir con todo mi deber y más si es necesario>>. Y mantuvo su promesa.

\par 
%\textsuperscript{(1418.4)}
\textsuperscript{128:7.12} Miriam vivía en la casa de Jacobo, contigua a la de María, pues Jacobo padre había sido enterrado con sus antepasados. Marta ocupó el lugar de Miriam en el hogar, y la nueva organización funcionó sin problemas antes de que terminara el año.

\par 
%\textsuperscript{(1418.5)}
\textsuperscript{128:7.13} Al día siguiente de la doble boda, Jesús tuvo una importante conversación con Santiago. Le contó confidencialmente que se estaba preparando para dejar el hogar. Regaló a Santiago la escritura de propiedad del taller de reparaciones, dimitió de manera oficial y solemne como jefe de la casa de José, e instaló a su hermano Santiago de forma muy afectuosa como <<jefe y protector de la casa de mi padre>>. Redactó un pacto secreto, que luego firmaron los dos, en el que se estipulaba que a cambio de la donación del taller de reparaciones, Santiago asumiría en adelante toda la responsabilidad financiera de la familia, eximiendo a Jesús de cualquier obligación posterior en esta materia. Después de firmar el contrato y de arreglar el presupuesto de tal manera que la familia pudiera hacer frente a sus gastos reales sin ninguna contribución de Jesús, éste dijo a Santiago: <<Hijo mío, no obstante continuaré enviándote algo todos los meses hasta que haya llegado mi hora, pero utiliza lo que yo te envíe según se presenten las circunstancias. Emplea mis fondos para las necesidades o los placeres de la familia, como te parezca conveniente. Utilízalos en caso de enfermedad o para hacer frente a los incidentes inesperados que puedan sobrevenir a cualquier miembro de la familia>>.

\par 
%\textsuperscript{(1418.6)}
\textsuperscript{128:7.14} Así es como Jesús se preparaba para emprender la segunda fase de su vida adulta, separado de los suyos, antes de empezar a ocuparse públicamente de los asuntos de su Padre.


\chapter{Documento 129. Continuación de la vida adulta de Jesús}
\par 
%\textsuperscript{(1419.1)}
\textsuperscript{129:0.1} JESÚS se había separado de manera completa y definitiva de la administración de los asuntos domésticos de la familia de Nazaret y de la dirección inmediata de sus miembros. Hasta el día de su bautismo continuó contribuyendo a las finanzas familiares y tomándose un vivo interés personal por el bienestar espiritual de cada uno de sus hermanos y hermanas. Y siempre estaba dispuesto a hacer todo lo que fuera humanamente posible por el bienestar y la felicidad de su madre viuda.

\par 
%\textsuperscript{(1419.2)}
\textsuperscript{129:0.2} El Hijo del Hombre lo tenía ahora todo preparado para separarse de manera permanente del hogar de Nazaret; hacer esto no fue nada fácil para él. Jesús amaba de manera natural a su gente; quería a su familia, y este afecto natural había crecido enormemente debido a su extraordinaria dedicación a ellos. Cuanto más plenamente nos entregamos a nuestros semejantes, más llegamos a amarlos; puesto que Jesús se había dado tan completamente a su familia, los quería con un afecto grande y ferviente.

\par 
%\textsuperscript{(1419.3)}
\textsuperscript{129:0.3} Poco a poco, toda la familia había empezado a comprender que Jesús se estaba preparando para dejarlos. La tristeza de la separación que se avecinaba sólo estaba atenuada por esta manera gradual de prepararlos para anunciarles su intención de partir. Durante más de cuatro años observaron que estaba proyectando esta separación final.

\section*{1. El vigésimo séptimo año (año 21 d. de J.C.)}
\par 
%\textsuperscript{(1419.4)}
\textsuperscript{129:1.1} Una lluviosa mañana de domingo del mes de enero de este año
21, Jesús se despidió sin ceremonias de su familia, explicándoles solamente que iba a Tiberiades y luego a visitar otras ciudades alrededor del Mar de Galilea. Así se separó de ellos, y nunca más volvió a ser un miembro regular de este hogar.

\par 
%\textsuperscript{(1419.5)}
\textsuperscript{129:1.2} Pasó una semana en Tiberiades, la nueva ciudad que pronto iba a sustituir a Séforis como capital de Galilea. Como encontró pocas cosas que le interesaran, pasó sucesivamente por Magdala y Betsaida hasta llegar a Cafarnaúm, donde se detuvo para visitar a Zebedeo, el amigo de su padre. Los hijos de Zebedeo eran pescadores, y él mismo era constructor de barcas. Jesús de Nazaret era un experto en el arte de diseñar y en la construcción; era un maestro trabajando la madera, y Zebedeo conocía desde hacía tiempo la habilidad del artesano de Nazaret. Hacía mucho tiempo que Zebedeo tenía la intención de construir mejores barcas; expuso pues sus proyectos a Jesús, e invitó al carpintero visitante a que se uniera a él en esta empresa. Jesús aceptó con mucho gusto.

\par 
%\textsuperscript{(1419.6)}
\textsuperscript{129:1.3} Jesús sólo trabajó con Zebedeo poco más de un año, pero durante este tiempo creó un nuevo tipo de barcas y estableció métodos completamente nuevos para su fabricación. Gracias a una técnica superior y a unos métodos mucho mejores de tratar las tablas al vapor, Jesús y Zebedeo empezaron a construir barcas de un tipo muy superior; se trataba de unas embarcaciones mucho más seguras que los antiguos modelos para navegar por el lago. Zebedeo tuvo durante varios años más trabajo, fabricando este nuevo tipo de barcas, que el que su pequeña empresa podía producir; en menos de cinco años, prácticamente todas las embarcaciones que navegaban por el lago habían sido construidas en el taller de Zebedeo en Cafarnaúm. Jesús se hizo famoso entre los pescadores de Galilea como el diseñador de estas nuevas barcas.

\par 
%\textsuperscript{(1420.1)}
\textsuperscript{129:1.4} Zebedeo era un hombre medianamente adinerado; sus astilleros se encontraban al borde del lago al sur de Cafarnaúm y su casa estaba situada a la orilla del lago cerca del centro de pesca de Betsaida. Jesús vivió en la casa de Zebedeo durante su estancia de más de un año en Cafarnaúm. Durante mucho tiempo había trabajado solo en el mundo, es decir sin padre, y disfrutó mucho de este período de trabajo con un socio paternal.

\par 
%\textsuperscript{(1420.2)}
\textsuperscript{129:1.5} Salomé, la mujer de Zebedeo, era pariente de Anás, antiguo sumo sacerdote en Jerusalén, que había sido destituido hacía sólo ocho años, pero que seguía siendo el miembro más influyente del grupo de los saduceos. Salomé se convirtió en una gran admiradora de Jesús. Lo quería tanto como a sus propios hijos, Santiago, Juan y David, mientras que sus cuatro hijas lo consideraban como su hermano mayor. Jesús salía a menudo a pescar con Santiago, Juan y David, los cuales descubrieron que era tan buen pescador como experto constructor de barcas.

\par 
%\textsuperscript{(1420.3)}
\textsuperscript{129:1.6} Jesús envió dinero a Santiago todos los meses de este año. En octubre regresó a Nazaret para asistir a la boda de Marta, y durante más de dos años no volvió por Nazaret hasta poco antes de la doble boda de Simón y de Judá.

\par 
%\textsuperscript{(1420.4)}
\textsuperscript{129:1.7} Jesús construyó barcas durante todo este año y continuó observando cómo vivían los hombres en la Tierra. Iba a visitar con frecuencia la parada de las caravanas, pues la ruta directa de Damasco hacia el sur pasaba por Cafarnaúm. Cafarnaúm era un importante puesto militar romano, y el oficial que mandaba la guarnición era un gentil que creía en Yahvé, <<un hombre piadoso>>\footnote{\textit{Oficial romano, un hombre piadoso}: Mt 8:5-13; Lc 7:1-10.}, como los judíos solían designar a estos prosélitos. Este oficial pertenecía a una rica familia romana, y había asumido la responsabilidad de construir una hermosa sinagoga en Cafarnaúm, que había donado a los judíos poco antes de que Jesús viniera a vivir con Zebedeo. Jesús dirigió los oficios en esta nueva sinagoga más de la mitad de las veces este año, y algunos de los viajeros de las caravanas que asistieron por casualidad lo recordaban como el carpintero de Nazaret.

\par 
%\textsuperscript{(1420.5)}
\textsuperscript{129:1.8} Cuando llegó el momento de pagar los impuestos, Jesús se inscribió como <<artesano cualificado de Cafarnaúm>>. Desde este día hasta el final de su vida terrestre, fue conocido como habitante de Cafarnaúm. Nunca pretendió tener otra residencia legal, aunque permitió, por diversas razones, que otros fijaran su domicilio en Damasco, Betania, Nazaret e incluso en Alejandría.

\par 
%\textsuperscript{(1420.6)}
\textsuperscript{129:1.9} Encontró muchos libros nuevos en las arcas de la biblioteca de la sinagoga de Cafarnaúm, y pasaba al menos cinco noches por semana estudiando intensamente. Dedicaba una noche a la vida social con los adultos y pasaba otra con los jóvenes. En la personalidad de Jesús había algo de agradable e inspirador que atraía invariablemente a los jóvenes. Siempre hacía que se sintieran a gusto en su presencia. Quizás su gran secreto para permanecer entre ellos consistía en el doble hecho de que siempre se interesaba por lo que estaban haciendo, mientras que raramente les aconsejaba, a menos que se lo pidieran.

\par 
%\textsuperscript{(1420.7)}
\textsuperscript{129:1.10} La familia de Zebedeo casi adoraba a Jesús, y nunca dejaban de asistir a las charlas con preguntas y respuestas que dirigía cada noche después de la cena, antes de irse a estudiar a la sinagoga. Los jóvenes de la vecindad también acudían con frecuencia a estas reuniones tras la cena. A estas pequeñas asambleas, Jesús les impartía una enseñanza variada y avanzada, tan avanzada como podían comprender. Hablaba con ellos sin ninguna reserva y exponía sus ideas e ideales sobre la política, la sociología, la ciencia y la filosofía, pero nunca pretendía hablar con una autoridad final excepto cuando hablaba de religión ---de la relación del hombre con Dios.

\par 
%\textsuperscript{(1421.1)}
\textsuperscript{129:1.11} Una vez por semana, Jesús mantenía una reunión con toda la gente de la casa, el personal del taller y los ayudantes de la costa, pues Zebedeo tenía muchos empleados. Entre estos trabajadores es donde llamaron a Jesús por primera vez <<Maestro>>\footnote{\textit{El Maestro}: Mt 8:19; Mc 4:38; Lc 3:12; Jn 1:38.}. Todos lo querían. Le gustaba su trabajo en Cafarnaúm con Zebedeo, pero echaba de menos a los niños jugando al lado del taller de carpintería de Nazaret.

\par 
%\textsuperscript{(1421.2)}
\textsuperscript{129:1.12} De todos los hijos de Zebedeo, Santiago era el que más se interesaba por Jesús como maestro y como filósofo. Juan apreciaba más su enseñanza y sus opiniones sobre la religión. David lo respetaba como artesano, pero hacía poco caso de sus ideas religiosas y de sus enseñanzas filosóficas.

\par 
%\textsuperscript{(1421.3)}
\textsuperscript{129:1.13} Judá venía muchos sábados para escuchar lo que Jesús decía en la sinagoga, y se quedaba para charlar con él. Cuanto más veía a su hermano mayor, más se convencía de que Jesús era realmente un gran hombre.

\par 
%\textsuperscript{(1421.4)}
\textsuperscript{129:1.14} Jesús hizo este año grandes progresos en la dominación ascendente de su mente humana, y alcanzó niveles nuevos y elevados de contacto consciente con su Ajustador del Pensamiento interior.

\par 
%\textsuperscript{(1421.5)}
\textsuperscript{129:1.15} Éste fue su último año de vida estable. Jesús nunca más pasó un año entero en el mismo lugar o en la misma tarea. Se estaban acercando rápidamente los días de sus peregrinaciones terrestres. Los períodos de intensa actividad no estaban lejos en el futuro, pero entre su vida simple e intensamente activa del pasado y su ministerio público aún más intenso y arduo, iban a intercalarse ahora unos pocos años de grandes viajes y de actividad personal muy diversificada. Tenía que completar su formación como hombre del mundo antes de emprender su carrera de enseñanza y de predicación como hombre-Dios perfeccionado de las fases divina y posthumana de su donación en Urantia.

\section*{2. El vigésimo octavo año (año 22 d. de J.C.)}
\par 
%\textsuperscript{(1421.6)}
\textsuperscript{129:2.1} Jesús se despidió de Zebedeo y de Cafarnaúm en marzo del año 22 d.de J.C. Pidió una pequeña suma de dinero para costear sus gastos de viaje hasta Jerusalén. Mientras trabajaba con Zebedeo, sólo había cobrado las pequeñas cantidades de dinero que enviaba mensualmente a su familia de Nazaret. José venía un mes a Cafarnaúm para buscar el dinero, y al mes siguiente era Judá quien pasaba por Cafarnaúm para recibir el dinero de Jesús y llevarlo a Nazaret. El centro pesquero donde trabajaba Judá sólo estaba a unos kilómetros al sur de Cafarnaúm.

\par 
%\textsuperscript{(1421.7)}
\textsuperscript{129:2.2} Cuando Jesús se despidió de la familia de Zebedeo, acordó con ellos permanecer en Jerusalén hasta la Pascua, y todos prometieron estar presentes para este acontecimiento. Incluso convinieron en celebrar juntos la cena pascual. Todos se entristecieron cuando Jesús se marchó, especialmente las hijas de Zebedeo.

\par 
%\textsuperscript{(1421.8)}
\textsuperscript{129:2.3} Antes de dejar Cafarnaúm, Jesús tuvo una larga conversación con su nuevo amigo e íntimo compañero Juan Zebedeo. Le dijo que pensaba viajar mucho hasta que <<llegue mi hora>>, y le pidió que cada mes enviara en su nombre algún dinero a la familia de Nazaret, hasta que se agotaran los fondos que se le debían. Juan le hizo esta promesa: <<Maestro, dedícate a tus asuntos y haz tu trabajo en el mundo. Actuaré en tu lugar en éste y en cualquier otro asunto, y velaré por tu familia como si tuviera que mantener a mi propia madre y cuidar a mis propios hermanos y hermanas. Emplearé los fondos que te debe mi padre tal como has indicado y según se necesiten. Cuando tu dinero se haya agotado, si no recibo más de ti y tu madre se encontrara en la necesidad, entonces compartiré mi propio salario con ella. Puedes emprender tu camino en paz. Actuaré en tu lugar en todas estas cuestiones>>.

\par 
%\textsuperscript{(1422.1)}
\textsuperscript{129:2.4} Después de que Jesús partiera para Jerusalén, Juan consultó con su padre Zebedeo sobre el dinero que se le debía a Jesús, y se quedó sorprendido de que la suma fuera tan importante. Como Jesús había dejado el asunto completamente entre sus manos, acordaron que lo mejor sería invertir estos fondos en inmuebles y utilizar la renta para ayudar a la familia de Nazaret. Zebedeo conocía una casita de Cafarnaúm que estaba hipotecada y en venta, por lo que recomendó a Juan que la comprara con el dinero de Jesús, y guardara la escritura en depósito para su amigo. Juan hizo lo que su padre le aconsejó. Durante dos años, el arrendamiento de la casa se utilizó para pagar la hipoteca, y esto, unido a una importante cantidad de dinero que Jesús envió a Juan poco después para que la familia lo utilizara según sus necesidades, fue casi suficiente para cancelar esta deuda. Zebedeo añadió la diferencia, de manera que Juan pagó el resto de la hipoteca a su vencimiento, consiguiendo así una escritura libre de cargas para esta casa de dos piezas. De esta manera Jesús se convirtió, sin saberlo, en el propietario de una casa en Cafarnaúm.

\par 
%\textsuperscript{(1422.2)}
\textsuperscript{129:2.5} Cuando la familia de Nazaret se enteró de que Jesús se había marchado de Cafarnaúm, como no sabían nada de este arreglo financiero con Juan, creyeron que les había llegado la hora de salir adelante sin contar con su ayuda. Santiago se acordó de su pacto con Jesús y, con la ayuda de sus hermanos, asumió inmediatamente toda la responsabilidad de cuidar a la familia.

\par 
%\textsuperscript{(1422.3)}
\textsuperscript{129:2.6} Pero volvamos atrás para observar a Jesús en Jerusalén. Durante cerca de dos meses, pasó la mayor parte de su tiempo escuchando las discusiones en el templo, y realizando visitas ocasionales a las diversas escuelas de rabinos. La mayoría de los sábados los pasaba en Betania.

\par 
%\textsuperscript{(1422.4)}
\textsuperscript{129:2.7} Jesús había llevado consigo a Jerusalén una carta de la esposa de Zebedeo, dirigida al antiguo sumo sacerdote Anás, en la que Salomé lo presentaba como <<si fuera mi propio hijo>>. Anás pasó mucho tiempo con él, llevándolo personalmente a visitar las numerosas academias de los educadores religiosos de Jerusalén. Jesús inspeccionó a fondo estas escuelas y observó cuidadosamente sus métodos de enseñanza, pero no hizo ni una sola pregunta en público. Aunque Anás consideraba a Jesús como un gran hombre, no sabía bien cómo aconsejarle. Reconocía que sería una tontería sugerirle que ingresara como estudiante en una de las escuelas de Jerusalén, y sin embargo sabía muy bien que nunca concederían a Jesús la categoría de profesor titular, ya que nunca se había formado en estas escuelas.

\par 
%\textsuperscript{(1422.5)}
\textsuperscript{129:2.8} La época de la Pascua se estaba acercando, y junto con el gentío que venía de todas partes, Zebedeo y toda su familia llegaron a Jerusalén procedentes de Cafarnaúm. Todos se alojaron en la espaciosa casa de Anás, donde celebraron la Pascua como una familia unida y feliz.

\par 
%\textsuperscript{(1422.6)}
\textsuperscript{129:2.9} Antes de finalizar esta semana pascual, y aparentemente por casualidad, Jesús conoció a un rico viajero y a su hijo, un joven de unos diecisiete años. Estos viajeros procedían de la India, y mientras iban de camino para visitar Roma y otros diversos lugares del Mediterráneo, habían planeado llegar a Jerusalén durante la Pascua, con la esperanza de encontrar a alguien a quien poder contratar como intérprete para los dos y como preceptor para el hijo. El padre insistió para que Jesús consintiera en viajar con ellos. Jesús le habló de su familia y de que no era muy razonable marcharse por un período de casi dos años, durante los cuales podrían pasar necesidades. Entonces este viajero de Oriente le propuso a Jesús adelantarle el salario de un año, de manera que pudiera confiar estos fondos a sus amigos para proteger a su familia de la pobreza. Y Jesús aceptó hacer el viaje.

\par 
%\textsuperscript{(1423.1)}
\textsuperscript{129:2.10} Jesús entregó esta importante cantidad a Juan, el hijo de Zebedeo. Y ya sabéis cómo utilizó este dinero para liquidar la hipoteca de la propiedad de Cafarnaúm. Jesús confió a Zebedeo todo lo relacionado con este viaje por el Mediterráneo, pero le encargó que no se lo dijera a nadie, ni siquiera a los de su propia carne y sangre. Zebedeo no reveló nunca que conocía el paradero de Jesús durante este largo período de casi dos años. Antes de que Jesús regresara de este viaje, la familia de Nazaret estaba a punto de darlo por muerto. Solamente las aseveraciones de Zebedeo, que fue a Nazaret en diversas ocasiones con su hijo Juan, mantuvieron viva la esperanza en el corazón de María.

\par 
%\textsuperscript{(1423.2)}
\textsuperscript{129:2.11} Durante este período, la familia de Nazaret se las arregló bastante bien. Judá había aumentado considerablemente su cuota y mantuvo esta contribución adicional hasta que se casó. A pesar del poco apoyo que necesitaban, Juan Zebedeo adquirió la costumbre de presentarse cada mes con unos regalos para María y para Rut, de acuerdo con las instrucciones de Jesús.

\section*{3. El vigésimo noveno año (año 23 d. de J.C.)}
\par 
%\textsuperscript{(1423.3)}
\textsuperscript{129:3.1} Jesús pasó todo su vigésimo noveno año completando su periplo por el mundo mediterráneo. En la medida en que se nos ha permitido revelar estas experiencias, los principales acontecimientos de este viaje constituyen el tema de la narración que sigue inmediatamente a este documento.

\par 
%\textsuperscript{(1423.4)}
\textsuperscript{129:3.2} Durante todo este recorrido por el mundo romano, a Jesús se le conoció, por muchas razones, como el \textit{escriba de Damasco}. Sin embargo, en Corinto y en otras escalas del viaje de vuelta, fue conocido como el \textit{preceptor judío}.

\par 
%\textsuperscript{(1423.5)}
\textsuperscript{129:3.3} Éste fue un período extraordinario en la vida de Jesús. Durante este viaje efectuó muchos contactos con sus semejantes, pero esta experiencia es una fase de su vida que nunca reveló a ningún miembro de su familia y a ninguno de los apóstoles. Jesús vivió toda su vida en la carne y dejó este mundo sin que nadie supiera (excepto Zebedeo de Betsaida) que había hecho este gran viaje. Algunos de sus amigos pensaban que había vuelto a Damasco; otros creían que se había ido a la India. Su propia familia tendía a creer que estaba en Alejandría, porque sabían que una vez lo habían invitado a ir allí para convertirse en el ayudante del chazan.

\par 
%\textsuperscript{(1423.6)}
\textsuperscript{129:3.4} Cuando Jesús volvió a Palestina, no hizo nada por cambiar la opinión de su familia de que había ido desde Jerusalén hasta Alejandría; les dejó que continuaran creyendo que todo el tiempo que había estado fuera de Palestina lo había pasado en aquella ciudad de erudición y de cultura. Únicamente Zebedeo, el constructor de barcas de Betsaida, conocía los hechos sobre esta cuestión, y Zebedeo no se lo dijo a nadie.

\par 
%\textsuperscript{(1423.7)}
\textsuperscript{129:3.5} En todos vuestros esfuerzos por descifrar el significado de la vida de Jesús en Urantia, tenéis que recordar los motivos de la donación de Miguel. Si queréis comprender el significado de muchas de sus acciones aparentemente extrañas, tenéis que discernir el propósito de su estancia en vuestro mundo. Tuvo la constante cautela de no fabricar una carrera personal demasiado atractiva que acaparara toda la atención. No quería emplear recursos excepcionales o abrumadores con sus semejantes. Estaba dedicado al trabajo de revelar el Padre celestial a sus compañeros mortales, y al mismo tiempo se consagraba a la tarea sublime de vivir su vida terrestre mortal constantemente sometido a la voluntad de este mismo Padre Paradisiaco.

\par 
%\textsuperscript{(1424.1)}
\textsuperscript{129:3.6} Para comprender la vida de Jesús en la Tierra, siempre será útil también que todos los mortales que estudien esta donación divina recuerden que, aunque vivió esta vida de encarnación \textit{en} Urantia, la vivió \textit{para} todo su universo. En la vida que vivió en la carne de naturaleza mortal, había algo especial e inspirador para cada una de las esferas habitadas de todo el universo de Nebadon. Esto también es así para todos aquellos mundos que se han vuelto habitables después de la época memorable de su estancia en Urantia. Y esto mismo será igualmente cierto en todos los mundos que puedan ser habitados por criaturas volitivas, en toda la historia futura de este universo local.

\par 
%\textsuperscript{(1424.2)}
\textsuperscript{129:3.7} Gracias a las experiencias de este periplo por el mundo romano, y mientras duró el mismo, el Hijo del Hombre completó prácticamente su aprendizaje educativo por contacto con los pueblos tan diversos del mundo de su época y de su generación. En el momento de su regreso a Nazaret, y debido a lo que había aprendido viajando, ya conocía prácticamente cómo el hombre vivía y forjaba su existencia en Urantia.

\par 
%\textsuperscript{(1424.3)}
\textsuperscript{129:3.8} El verdadero objetivo de su recorrido alrededor de la cuenca del Mediterráneo era \textit{conocer a los hombres}. Estuvo en estrecho contacto con centenares de seres humanos en este viaje. Conoció y amó a toda clase de hombres, ricos y pobres, poderosos y humildes, negros y blancos, instruídos e iletrados, cultos e incultos, brutos y espirituales, religiosos e irreligiosos, morales e inmorales.

\par 
%\textsuperscript{(1424.4)}
\textsuperscript{129:3.9} En este viaje por el Mediterráneo, Jesús efectuó un gran avance en su tarea humana de dominar la mente material y mortal, y su Ajustador interior progresó mucho en la ascensión y la conquista espiritual de este mismo intelecto humano. Al finalizar este periplo, Jesús sabía implícitamente ---con toda certidumbre humana--- que era un Hijo de Dios, un Hijo Creador del Padre Universal. El Ajustador era cada vez más capaz de traer a la mente del Hijo del Hombre recuerdos nebulosos de su experiencia paradisiaca cuando estaba en asociación con su Padre divino, mucho antes de venir a organizar y administrar este universo local de Nebadon. Así, poco a poco, el Ajustador trajo a la conciencia humana de Jesús los recuerdos necesarios de su anterior existencia divina en las diversas épocas de un pasado casi eterno. El último episodio de su experiencia prehumana, puesto de manifiesto por el Ajustador, fue su conversación de despedida con Emmanuel de Salvington poco antes de abandonar su personalidad consciente para emprender su encarnación en Urantia. La imagen de este último recuerdo de su existencia prehumana apareció con toda claridad en la conciencia de Jesús el mismo día que Juan lo bautizó en el Jordán.

\section*{4. El Jesús humano}
\par 
%\textsuperscript{(1424.5)}
\textsuperscript{129:4.1} Para las inteligencias celestiales del universo local que lo observaban, este viaje por el Mediterráneo fue la más cautivadora de todas las experiencias terrestres de Jesús, al menos de toda su carrera hasta el momento de su crucifixión y de su muerte física. Éste fue el período fascinante de su \textit{ministerio personal}, en contraste con la época de ministerio público que pronto le seguiría. Este episodio único en su género fue aún más sobresaliente porque en aquel momento era todavía el carpintero de Nazaret, el constructor de barcas de Cafarnaúm, el escriba de Damasco; era todavía el Hijo del Hombre. Aún no había conseguido el dominio completo de su mente humana; el Ajustador no había dominado ni transcrito plenamente la identidad mortal. Era todavía un hombre entre los hombres.

\par 
%\textsuperscript{(1425.1)}
\textsuperscript{129:4.2} La experiencia religiosa puramente humana del Hijo del Hombre ---el crecimiento espiritual personal--- alcanzó casi la cima de lo accesible durante este año, el vigésimo noveno de su vida. Esta experiencia de desarrollo espiritual fue un crecimiento permanentemente gradual desde el momento en que llegó su Ajustador del Pensamiento hasta el día en que finalizó y se confirmó esta relación humana normal y natural entre la mente material del hombre y la dotación mental del espíritu. El fenómeno de fundir estas dos mentes en una sola fue una experiencia que el Hijo del Hombre alcanzó de manera completa y final, como mortal encarnado del mundo, el día de su bautismo en el Jordán.

\par 
%\textsuperscript{(1425.2)}
\textsuperscript{129:4.3} A través de todos estos años, aunque no parecía dedicarse a muchos períodos de comunión formal con su Padre celestial, perfeccionó unos métodos cada vez más eficaces para comunicarse personalmente con la presencia espiritual interior del Padre Paradisiaco. Vivió una vida real, una vida plena y una verdadera vida en la carne, normal, natural y corriente. Conoce por experiencia personal lo equivalente a la realidad de todo lo esencial de la vida que viven los seres humanos en los mundos materiales del tiempo y del espacio.

\par 
%\textsuperscript{(1425.3)}
\textsuperscript{129:4.4} El Hijo del Hombre experimentó la amplia gama de emociones humanas que van desde la alegría más espléndida al dolor más profundo. Era un niño alegre y un ser con un buen humor poco común; era igualmente un <<varón de dolores que conocía las aflicciones>>\footnote{\textit{Varón de dolores que conocía las aflicciones}: Is 53:3.}. En un sentido espiritual, atravesó la vida mortal desde el punto más bajo hasta el más elevado, desde el principio hasta el fin. Desde un punto de vista material, podría parecer que evitó vivir en los dos extremos sociales de la existencia humana, pero intelectualmente se familiarizó totalmente con la experiencia entera y completa de la humanidad.

\par 
%\textsuperscript{(1425.4)}
\textsuperscript{129:4.5} Jesús conoce los pensamientos y los sentimientos, los deseos y los impulsos, de los mortales evolutivos y ascendentes de los mundos, desde el nacimiento hasta la muerte. Ha vivido la vida humana desde los principios del yo físico, intelectual y espiritual, pasando por la infancia, la adolescencia, la juventud y la madurez, llegando incluso hasta la experiencia humana de la muerte\footnote{\textit{Experiencia humana de Jesús}: Heb 2:14-18; 4:15.}. No solamente pasó por estos períodos humanos, normales y conocidos, de avance intelectual y espiritual, sino que \textit{también} experimentó plenamente las fases superiores y más avanzadas de aproximación entre el ser humano y su Ajustador, que tan pocos mortales de Urantia consiguen alcanzar. Así pues, experimentó en su plenitud la vida del hombre mortal, no sólo tal como se vive en vuestro mundo, sino también tal como se vive en todos los demás mundos evolutivos del tiempo y del espacio, e incluso en los más elevados y avanzados de los mundos establecidos en la luz y la vida.

\par 
%\textsuperscript{(1425.5)}
\textsuperscript{129:4.6} Esta vida perfecta que vivió en la similitud de la carne mortal quizás no haya recibido la aprobación completa y universal de sus compañeros mortales, de aquellos que fueron casualmente sus contemporáneos en la Tierra; sin embargo, la vida encarnada que Jesús de Nazaret vivió en Urantia sí recibió la plena y completa aprobación del Padre Universal, porque constituía, al mismo tiempo y en una sola y misma vida de personalidad, la plenitud de la revelación del Dios eterno al hombre mortal, y la presentación de una personalidad humana perfeccionada que satisfacía plenamente al Creador Infinito.

\par 
%\textsuperscript{(1425.6)}
\textsuperscript{129:4.7} Éste fue su objetivo verdadero y supremo. No descendió para vivir en Urantia como el ejemplo perfecto y detallado a seguir por cualquier niño o adulto, por cualquier hombre o mujer, de aquella época o de cualquier otra. En verdad es cierto que todos podemos encontrar en su vida plena, rica, hermosa y noble, muchos elementos exquisitamente ejemplares y divinamente inspiradores, pero esto es así porque vivió una vida verdadera y auténticamente humana. Jesús no vivió su vida en la Tierra para establecer un ejemplo a imitar por todos los demás seres humanos. Vivió esta vida en la carne mediante el mismo ministerio de misericordia que todos vosotros podéis utilizar para vivir vuestra vida en la Tierra. Al vivir su vida mortal en su época y \textit{tal como él era}, estableció un ejemplo para que todos nosotros vivamos también la nuestra en nuestra época y \textit{tal como somos}. Quizás no aspiréis a vivir su vida, pero podéis decidir \textit{vivir la vuestra} como él vivió la suya, y por los mismos medios. Jesús puede que no sea el ejemplo técnico y detallado para todos los mortales de todos los tiempos en todos los planetas de este universo local, pero es eternamente la inspiración y guía de todos los peregrinos con destino paradisiaco procedentes de los mundos de ascensión inicial, que pasan a través del universo de universos y de Havona hasta el Paraíso. Jesús es el \textit{nuevo camino viviente}\footnote{\textit{Nuevo camino viviente}: Jn 14:6; Heb 10:20.} que va desde el hombre hasta Dios, de lo parcial a lo perfecto, de lo terrenal a lo celestial, del tiempo a la eternidad.

\par 
%\textsuperscript{(1426.1)}
\textsuperscript{129:4.8} Al final de su vigésimo noveno año, Jesús de Nazaret casi había terminado de vivir la vida que se exige a los mortales como residentes temporales en la carne. Trajo a la Tierra toda la plenitud de Dios que se puede manifestar al hombre; ahora casi se había convertido en la perfección del hombre que espera la ocasión para manifestarse a Dios. Y realizó todo esto antes de cumplir los treinta años.


\chapter{Documento 130. En el camino a Roma}
\par 
%\textsuperscript{(1427.1)}
\textsuperscript{130:0.1} El viaje por el mundo romano consumió la mayor parte del año veintiocho y todo el año veintinueve de la vida de Jesús en la Tierra. Jesús y los dos nativos de la India ---Gonod y su hijo Ganid--- salieron de Jerusalén el domingo por la mañana 26 de abril del año 22. Llevaron a cabo su viaje tal como lo habían programado, y Jesús se despidió del padre y del hijo en la ciudad de Charax, en el Golfo Pérsico, el 10 de diciembre del año siguiente, el año 23.

\par 
%\textsuperscript{(1427.2)}
\textsuperscript{130:0.2} Desde Jerusalén se dirigieron a Cesarea pasando por Jope. En Cesarea cogieron un barco para Alejandría. Desde Alejandría navegaron hasta Lasea, en Creta. Desde Creta siguieron por mar hasta Cartago, haciendo escala en Cirene. En Cartago tomaron un barco para Nápoles, deteniéndose en Malta, Siracusa y Mesina. Desde Nápoles fueron a Capua, y desde allí viajaron por la Vía Apia hasta Roma.

\par 
%\textsuperscript{(1427.3)}
\textsuperscript{130:0.3} Al terminar su estancia en Roma se dirigieron por vía terrestre a Tarento, donde se hicieron a la mar para Atenas en Grecia, deteniéndose en Nicópolis y Corinto. Desde Atenas fueron a Éfeso por la ruta de Troade. Desde Éfeso navegaron hacia Chipre, haciendo escala en Rodas. Pasaron mucho tiempo visitando Chipre y descansando, y luego se embarcaron para Antioquía en Siria. Desde Antioquía fueron hacia el sur hasta Sidón y pasaron después por Damasco. Desde Damasco viajaron en caravana hasta Mesopotamia, pasando por Tapsacos y Larisa. Permanecieron algún tiempo en Babilonia, visitaron Ur y otros lugares, y luego fueron a Susa. Desde Susa viajaron a Charax, donde Gonod y Ganid embarcaron para la India.

\par 
%\textsuperscript{(1427.4)}
\textsuperscript{130:0.4} Jesús había aprendido los rudimentos del idioma que hablaban Gonod y Ganid cuando estuvo trabajando cuatro meses en Damasco. Mientras estuvo allí, pasó la mayoría del tiempo haciendo traducciones del griego a una de las lenguas de la India, con la ayuda de un nativo de la región donde vivía Gonod.

\par 
%\textsuperscript{(1427.5)}
\textsuperscript{130:0.5} Durante su viaje por el Mediterráneo, Jesús pasó aproximadamente la mitad del día enseñando a Ganid y sirviendo de intérprete a Gonod en sus entrevistas de negocios y en sus relaciones sociales. El resto del día lo tenía a su disposición, y lo dedicaba a entablar esos estrechos contactos personales con sus semejantes, esas íntimas relaciones con los mortales de este mundo, que tanto caracterizaron sus actividades de estos años inmediatamente anteriores a su ministerio público.

\par 
%\textsuperscript{(1427.6)}
\textsuperscript{130:0.6} Gracias a estas observaciones de primera mano y a estos contactos reales, Jesús trabó conocimiento con la civilización material e intelectual superior de Occidente y del Levante. De Gonod y de su brillante hijo aprendió mucho sobre la civilización y la cultura de la India y de China, ya que Gonod, que era ciudadano de la India, había hecho tres grandes viajes al imperio de la raza amarilla.

\par 
%\textsuperscript{(1427.7)}
\textsuperscript{130:0.7} El joven Ganid aprendió mucho de Jesús durante esta larga e íntima asociación. Llegaron a tenerse un gran afecto mutuo, y el padre del muchacho trató muchas veces de persuadir a Jesús para que los acompañara a la India, pero él siempre declinó la invitación, alegando la necesidad de regresar con su familia de Palestina.

\section*{1. En Jope --- discurso sobre Jonás}
\par 
%\textsuperscript{(1428.1)}
\textsuperscript{130:1.1} Durante su estancia en Jope, Jesús conoció a Gadía\footnote{\textit{Gadía y Simón}: Hch 10:5-6.}, un intérprete filisteo que trabajaba para un curtidor llamado Simón. Los agentes de Gonod en Mesopotamia habían hecho muchos negocios con este Simón; por eso Gonod y su hijo deseaban visitarlo camino de Cesarea. Mientras permanecieron en Jope, Jesús y Gadía se hicieron buenos amigos. El joven filisteo era un buscador de la verdad. Jesús era un dador de la verdad; él \textit{era} la verdad para esa generación en Urantia. Cuando un gran buscador y un gran dador de la verdad se encuentran, se produce una gran iluminación liberadora surgida de la experiencia de la nueva verdad.

\par 
%\textsuperscript{(1428.2)}
\textsuperscript{130:1.2} Un día, después de la cena, Jesús y el joven filisteo paseaban por la orilla del mar y Gadía, sin saber que este <<escriba de Damasco>> estaba tan bien versado en las tradiciones hebreas, mostró a Jesús el lugar donde Jonás supuestamente había embarcado\footnote{\textit{Embarco de Jonás}: Jon 1:3.} para su funesto viaje a Tarsis. Cuando concluyó sus comentarios, hizo a Jesús la pregunta siguiente: <<¿Tú crees que el gran pez se tragó realmente a Jonás?>>\footnote{\textit{¿Tú crees que el gran pez se tragó realmente a Jonás?}: Jon 1:17.}. Jesús percibió que la vida del joven había estado enormemente influida por esta tradición, y que sus reflexiones al respecto le habían inculcado la locura de intentar huir del deber. Por lo tanto, Jesús no dijo nada que pudiera destruir repentinamente las motivaciones fundamentales que guiaban a Gadía en su vida práctica. En respuesta a la pregunta, Jesús dijo: <<Amigo mío, todos somos como Jonás, con una vida que vivir de acuerdo con la voluntad de Dios. Cada vez que tratamos de esquivar el deber de la vida diaria para ir en busca de tentaciones lejanas, nos ponemos inmediatamente bajo el dominio de influencias que no están dirigidas por los poderes de la verdad ni por las fuerzas de la rectitud. Huir del deber es sacrificar la verdad. Evadirse del servicio de la luz y la vida sólo puede llevar a esos conflictos angustiosos con las temibles ballenas del egoísmo, que al final conducen a las tinieblas y a la muerte, a menos que esos Jonases que han abandonado a Dios deseen, incluso estando en lo más profundo de su desesperación, volver su corazón hacia la búsqueda de Dios y su bondad. Cuando estas almas desalentadas buscan sinceramente a Dios ---con hambre de verdad y sed de rectitud--- no hay nada que pueda retenerlas por más tiempo en cautiverio. Por muy profundos que sean los abismos donde puedan haber caído, cuando buscan la luz de todo corazón, el espíritu del Señor Dios de los cielos las libera de sus cadenas; las tribulaciones de la vida las arrojan a la tierra firme de las nuevas oportunidades para un servicio renovado y una vida más sabia>>.

\par 
%\textsuperscript{(1428.3)}
\textsuperscript{130:1.3} Gadía se sintió muy conmovido por la enseñanza de Jesús. Siguieron conversando a la orilla del mar hasta muy entrada la noche, y antes de regresar a sus alojamientos, rezaron juntos y el uno por el otro. Este mismo Gadía escuchó las predicaciones posteriores de Pedro, se convirtió en un profundo creyente en Jesús de Nazaret, y tuvo una noche una memorable controversia con Pedro en casa de Dorcas\footnote{\textit{Casa de Dorcas en Jope}: Hch 9:36-42.}. Gadía también contribuyó mucho a que Simón, el rico mercader de cuero\footnote{\textit{Conversión de Simón el curtidor}: Hch 9:43.}, se decidiera a abrazar el cristianismo.

\par 
%\textsuperscript{(1428.4)}
\textsuperscript{130:1.4} (En este relato de la obra personal de Jesús con sus semejantes mortales durante su viaje por el Mediterráneo, y de acuerdo con el permiso que hemos recibido, traduciremos libremente sus palabras a la terminología moderna que se emplea en Urantia en el momento de esta presentación).

\par 
%\textsuperscript{(1429.1)}
\textsuperscript{130:1.5} La última conversación de Jesús con Gadía trató sobre el bien y el mal. Este joven filisteo estaba bastante desconcertado por el sentimiento de injusticia que le producía la presencia del mal conviviendo con el bien en el mundo. Dijo: <<Si Dios es infinitamente bueno, ¿cómo puede permitir que suframos las penas del mal?. Después de todo, ¿quién crea el mal?>> En aquellos tiempos, mucha gente creía todavía que Dios creaba a la vez el bien y el mal, pero Jesús nunca enseñó un error semejante. Al responder a esta pregunta, Jesús dijo: <<Hermano mío, Dios es amor, por lo tanto debe ser bueno, y su bondad es tan grande y real que no puede contener las cosas pequeñas e irreales del mal. Dios es tan positivamente bueno que no hay absolutamente ninguna cabida en él para el mal negativo. El mal es la elección inmadura y el paso en falso irreflexivo de los que se resisten a la bondad, rechazan la belleza y traicionan la verdad. El mal sólo es la inadaptación de la inmadurez o la influencia desintegradora y deformadora de la ignorancia. El mal es la inevitable oscuridad que sigue de cerca al rechazo imprudente de la luz. El mal es lo tenebroso y lo falso; cuando se abraza conscientemente y se aprueba voluntariamente, se convierte en pecado>>\footnote{\textit{Dios es amor}: 1 Jn 4:8,16.}.

\par 
%\textsuperscript{(1429.2)}
\textsuperscript{130:1.6} <<Al dotarte de la facultad de escoger entre la verdad y el error, tu Padre celestial ha creado el potencial negativo de la vía positiva de la luz y la vida; pero los errores del mal no existen realmente hasta el momento en que una criatura inteligente quiere que existan, por una mala elección de su manera de vivir. Estos males se elevan posteriormente a la categoría de pecado mediante la elección consciente y deliberada de esa misma criatura obstinada y rebelde. Por eso, nuestro Padre que está en los cielos permite que el bien y el mal continúen juntos su camino hasta el final de la vida, al igual que la naturaleza permite que el trigo y la cizaña crezcan juntos hasta el momento de la siega>>\footnote{\textit{El trigo y la cizaña}: Mt 13:24-30.}. Gadía quedó plenamente satisfecho con la respuesta de Jesús a su pregunta, después de que la discusión posterior clarificara en su mente el verdadero significado de estas importantes declaraciones.

\section*{2. En Cesarea}
\par 
%\textsuperscript{(1429.3)}
\textsuperscript{130:2.1} Jesús y sus amigos permanecieron en Cesarea más tiempo del que habían previsto, porque se descubrió que uno de los enormes remos que gobernaban la nave en la que pensaban embarcarse corría peligro de romperse. El capitán decidió permanecer en el puerto mientras fabricaban uno nuevo. Había escasez de carpinteros cualificados para esta tarea, y Jesús se ofreció voluntariamente para ayudar. Por las noches, Jesús y sus amigos paseaban por la hermosa muralla que servía de paseo alrededor del puerto. A Ganid le interesó mucho la explicación de Jesús sobre el sistema de canalización de las aguas de la ciudad y la técnica que utilizaban al emplear las mareas para lavar las calles y alcantarillas de la ciudad. El joven indio se sintió muy impresionado por el templo de Augusto, situado en una elevación y rematado con una estatua colosal del emperador romano. La segunda tarde de su estancia, los tres asistieron a un espectáculo en el enorme anfiteatro que podía contener veinte mil personas sentadas, y aquella misma noche fueron a ver una obra griega en el teatro. Estos eran los primeros espectáculos de este tipo que Ganid había visto en su vida, e hizo muchas preguntas a Jesús acerca de ellos. El tercer día por la mañana hicieron una visita oficial al palacio del gobernador, porque Cesarea era la capital de Palestina y la residencia del procurador romano.

\par 
%\textsuperscript{(1429.4)}
\textsuperscript{130:2.2} En su posada también estaba alojado un mercader de Mongolia, y como este oriental hablaba bastante bien el griego, Jesús mantuvo varias largas conversaciones con él. Este hombre se quedó muy impresionado con la filosofía de vida de Jesús y no olvidó nunca sus sabias palabras sobre <<la manera de vivir la vida celestial en la Tierra, sometiéndose diariamente a la voluntad del Padre celestial>>. Este mercader era taoísta, y por ello se había convertido en un firme creyente en la doctrina de una Deidad universal. Al regresar a Mongolia, empezó a enseñar estas verdades avanzadas a sus vecinos y a sus asociados en los negocios, y como resultado directo de estas actividades, su hijo mayor decidió hacerse sacerdote taoísta. Durante toda su vida, este joven ejerció una gran influencia en favor de la verdad avanzada; fue sucedido en esta vía por un hijo y un nieto, que también se consagraron fielmente a la doctrina del Dios Único ---el Soberano Supremo del Cielo.

\par 
%\textsuperscript{(1430.1)}
\textsuperscript{130:2.3} Aunque la rama oriental de la iglesia cristiana primitiva, que tenía su centro en Filadelfia, permaneció más fiel a las enseñanzas de Jesús que la hermandad de Jerusalén, es lamentable que no hubiera nadie como Pedro que fuera a China, o como Pablo que viajara a la India, donde el terreno espiritual era entonces tan favorable para plantar la semilla del nuevo evangelio del reino\footnote{\textit{Evangelio del reino}: Mt 4:3; 9:35; 24:14; Mc 1:14-15.}. Estas mismas enseñanzas de Jesús, tal como las sostenían los filadelfianos, hubieran suscitado en las mentes de los pueblos asiáticos espiritualmente hambrientos el mismo interés inmediato y efectivo que las predicaciones de Pedro y de Pablo suscitaron en occidente.

\par 
%\textsuperscript{(1430.2)}
\textsuperscript{130:2.4} Un día, uno de los jóvenes que trabajaban con Jesús en el remo del timón se mostró muy interesado por las palabras que este último dejaba caer de vez en cuando mientras trabajaban en el astillero. Cuando Jesús sugirió que el Padre que está en los cielos se interesaba por el bienestar de sus hijos en la Tierra, este joven griego llamado Anaxando dijo: <<Si los Dioses se interesan por mí, entonces ¿por qué no quitan al capataz cruel e injusto que dirige este taller?>>. Se quedó sorprendido cuando Jesús replicó: <<Puesto que conoces los caminos de la bondad y valoras la justicia, tal vez los Dioses han puesto a este hombre equivocado cerca de ti para que puedas guiarlo por ese camino mejor. Quizás tú eres la sal que puede hacer a este hermano más agradable para todos los demás hombres, es decir, si no has perdido tu sabor. Tal como están las cosas, este hombre es tu amo porque sus malos procedimientos te influyen desfavorablemente. ¿Por qué no afirmar tu dominio sobre el mal mediante el poder de la bondad, convirtiéndote así en el amo de todas las relaciones entre vosotros dos?. Puedo predecir que el bien que hay en ti podría vencer al mal que hay en él, si le dieras una oportunidad honrada y vivificante. En el transcurso de la existencia mortal no hay aventura más apasionante que la alegría de asociarse, en la vida material, con la energía espiritual y la verdad divina en una de sus luchas victoriosas contra el error y el mal. Es una experiencia maravillosa y transformadora la de convertirse en el canal viviente de la luz espiritual para los mortales que permanecen en las tinieblas espirituales. Si estás más favorecido por la verdad que este hombre, su necesidad debería ser un desafío para ti. ¡Seguramente no serás un cobarde, capaz de permanecer en la orilla del mar mirando cómo perece un compañero que no sabe nadar!. ¡Cuánto más valiosa es el alma de este hombre que se debate en las tinieblas, comparada con su cuerpo que se ahoga en el mar!>>.

\par 
%\textsuperscript{(1430.3)}
\textsuperscript{130:2.5} Anaxando se sintió profundamente conmovido por las palabras de Jesús. No tardó en contar a su superior lo que Jesús le había dicho, y aquella misma noche los dos pidieron a Jesús que les aconsejara sobre el bienestar de sus almas. Mucho más tarde, después de haberse proclamado en Cesarea el mensaje cristiano, estos dos hombres, uno griego y el otro romano, creyeron en la predicación de Felipe\footnote{\textit{Predicación de Felipe}: Hch 8:40.} y se convirtieron en miembros influyentes de la iglesia fundada por él. Posteriormente, este joven griego fue nombrado intendente de un centurión romano llamado Cornelio\footnote{\textit{Conversión de Cornelio}: Hch 10:1-48.}, que se hizo creyente a través del ministerio de Pedro. Anaxando continuó aportando la luz a los que estaban en las tinieblas hasta la época en que Pablo fue encarcelado en Cesarea\footnote{\textit{Encarcelamiento de Pablo}: Hch 23:31-24:27.}. Pereció accidentalmente mientras socorría a los heridos y moribundos, durante la gran masacre en la que murieron veinte mil judíos.

\par 
%\textsuperscript{(1431.1)}
\textsuperscript{130:2.6} Por esta época, Ganid empezó a darse cuenta de que su tutor empleaba sus ratos libres en este ministerio personal poco común hacia sus semejantes, y el joven indio decidió descubrir el motivo de estas actividades incesantes. Preguntó: <<¿Por qué te ocupas continuamente en hablar con extraños?>> Y Jesús respondió: <<Ganid, ningún hombre es un extraño para el que conoce a Dios. En la experiencia de encontrar al Padre que está en los cielos, descubres que todos los hombres son tus hermanos, y ¿no es normal que uno sienta alegría al encontrarse con un hermano recién descubierto?. Conocer a nuestros hermanos y hermanas, comprender sus problemas y aprender a amarlos, es la experiencia suprema de la vida>>.

\par 
%\textsuperscript{(1431.2)}
\textsuperscript{130:2.7} Fue una conversación que duró hasta bien entrada la noche, en el transcurso de la cual el joven pidió a Jesús que le explicara la diferencia entre la voluntad de Dios y el acto mental humano de elegir, que también se llama voluntad. En sustancia, Jesús dijo: La voluntad de Dios es el camino de Dios, el asociarse con la elección de Dios frente a cualquier alternativa potencial. En consecuencia, hacer la voluntad de Dios es la experiencia progresiva de parecerse cada vez más a Dios, y Dios es la fuente y el destino de todo lo que es bueno, bello y verdadero. La voluntad del hombre es el camino del hombre, la suma y la sustancia de lo que el mortal escoge ser y hacer. La voluntad es la elección deliberada de un ser auto-consciente, que conduce a una decisión y a un comportamiento basados en una reflexión inteligente.

\par 
%\textsuperscript{(1431.3)}
\textsuperscript{130:2.8} Aquella tarde, Jesús y Ganid habían disfrutado jugando con un perro pastor muy inteligente, y Ganid quiso saber si el perro tenía alma, si tenía voluntad. En respuesta a sus preguntas, Jesús dijo: <<El perro tiene una mente que puede conocer al hombre material, su dueño, pero no puede conocer a Dios, que es espíritu. Así pues, el perro no posee una naturaleza espiritual y no puede disfrutar de una experiencia espiritual. El perro puede tener una voluntad derivada de la naturaleza y acrecentada por el adiestramiento, pero este poder de la mente no es una fuerza espiritual, ni tampoco es comparable con la voluntad humana, porque no es \textit{reflexiva ---} no es el resultado de la discriminación de los significados superiores y morales, o de la elección de los valores espirituales y eternos. La posesión de estos poderes de discriminación espiritual y de elección de la verdad es lo que convierte al hombre mortal en un ser moral, en una criatura dotada de los atributos de la responsabilidad espiritual y del potencial de la supervivencia eterna>>. Jesús siguió explicando que la ausencia de estos poderes mentales en los animales es lo que hace imposible para siempre que el mundo animal pueda desarrollar un lenguaje en el tiempo, o experimentar algo equivalente a la supervivencia de la personalidad en la eternidad. Como consecuencia de la lección de este día, Ganid no creyó nunca más en la transmigración de las almas humanas a los cuerpos de los animales.

\par 
%\textsuperscript{(1431.4)}
\textsuperscript{130:2.9} Al día siguiente, Ganid discutió de todo esto con su padre, y en respuesta a una pregunta de Gonod, Jesús explicó que <<las voluntades humanas que se dedican exclusivamente a tomar decisiones temporales relacionadas con los problemas materiales de la existencia animal, están condenadas a perecer con el tiempo. Las que toman decisiones morales sinceras y efectúan elecciones espirituales incondicionales, se identifican así progresivamente con el espíritu interior y divino, y se van transformando cada vez más en valores de supervivencia eterna: una progresión sin fin de servicio divino>>.

\par 
%\textsuperscript{(1431.5)}
\textsuperscript{130:2.10} Fue este mismo día cuando oímos por primera vez esa verdad capital que, enunciada en términos modernos, significaría: <<La voluntad es esa manifestación de la mente humana que permite a la conciencia subjetiva expresarse objetivamente y experimentar el fenómeno de aspirar a ser semejante a Dios>>. Es en este mismo sentido como todo ser humano reflexivo e inclinado hacia el espíritu puede volverse \textit{creativo}.

\section*{3. En Alejandría}
\par 
%\textsuperscript{(1432.1)}
\textsuperscript{130:3.1} La estancia en Cesarea había estado llena de acontecimientos; cuando el barco estuvo listo, Jesús y sus dos amigos zarparon un día al mediodía hacia Alejandría en Egipto.

\par 
%\textsuperscript{(1432.2)}
\textsuperscript{130:3.2} La travesía fue sumamente agradable para los tres. Ganid estaba encantado con el viaje y mantenía ocupado a Jesús contestando a sus preguntas. Al acercarse al puerto de la ciudad, el joven se emocionó al ver el gran faro de Faros, situado en la isla que Alejandro había unido con la tierra firme a través de un dique, creando así dos magníficas ensenadas que hicieron de Alejandría la encrucijada comercial marítima de África, Asia y Europa. Este gran faro era una de las siete maravillas del mundo y el precursor de todos los faros posteriores. Por la mañana se levantaron temprano para contemplar este magnífico dispositivo salvavidas creado por el hombre, y en medio de las exclamaciones de Ganid, Jesús dijo: <<Y tú, hijo mío, te parecerás a este faro cuando regreses a la India, incluso cuando tu padre descanse en paz. Serás como la luz de la vida para los que estén a tu alrededor en las tinieblas, mostrando a todos los que lo deseen el camino seguro para llegar al puerto de la salvación>>. Estrechando la mano de Jesús, Ganid le dijo: <<Lo seré>>.

\par 
%\textsuperscript{(1432.3)}
\textsuperscript{130:3.3} Subrayamos de nuevo que los primeros maestros de la religión cristiana cometieron un grave error al concentrarse exclusivamente en la civilización occidental del mundo romano. Las enseñanzas de Jesús, tal como las conservaban los creyentes mesopotámicos del siglo primero, hubieran sido recibidas de buena gana por los diversos grupos religiosos de Asia.

\par 
%\textsuperscript{(1432.4)}
\textsuperscript{130:3.4} A las cuatro horas de desembarcar, ya estaban instalados cerca del extremo oriental de la gran avenida, de treinta metros de ancha y ocho kilómetros de larga, que llegaba hasta los límites occidentales de esta ciudad de un millón de habitantes. Después de echar una primera ojeada a las principales atracciones de la ciudad ---la universidad (museo), la biblioteca, el mausoleo real de Alejandro, el palacio, el templo de Neptuno, el teatro y el gimnasio--- Gonod se dedicó a sus negocios mientras que Jesús y Ganid se fueron a la biblioteca, la más grande del mundo. Aquí había cerca de un millón de manuscritos de todos los países civilizados: Grecia, Roma, Palestina, Partia, India, China e incluso Japón. En esta biblioteca, Ganid vio la mayor colección de literatura india de todo el mundo, y durante su estancia en Alejandría pasaron en este lugar un rato cada día. Jesús contó a Ganid que las escrituras hebreas habían sido traducidas al griego en este lugar. Discutieron una y otra vez de todas las religiones del mundo, y Jesús se esforzó en enseñar a esta mente joven la verdad que contenía cada una de ellas, añadiendo siempre: <<Pero Yahvé es el Dios que surgió de las revelaciones de Melquisedek y del pacto con Abraham. Los judíos eran los descendientes de Abraham y ocuparon posteriormente la misma tierra en la que Melquisedek había vivido y enseñado, y desde la cual envió maestros a todo el mundo; y su religión acabó describiendo al Señor Dios de Israel como Padre Universal que está en los cielos, reconociéndolo de manera más clara que cualquier otra religión del mundo>>.

\par 
%\textsuperscript{(1432.5)}
\textsuperscript{130:3.5} Bajo la dirección de Jesús, Ganid hizo una recopilación de las enseñanzas de todas las religiones del mundo que reconocían a una Deidad Universal, aunque pudieran admitir también otras deidades subordinadas. Después de muchas discusiones, Jesús y Ganid decidieron que los romanos no tenían ningún verdadero Dios en su religión, la cual no era mucho más que un culto al emperador. Llegaron a la conclusión de que los griegos tenían una filosofía, pero difícilmente una religión con un Dios personal. Descartaron los cultos de misterio debido a la confusión de su multiplicidad, y a que sus conceptos variados sobre la Deidad parecían derivarse de otras religiones y de religiones más antiguas.

\par 
%\textsuperscript{(1433.1)}
\textsuperscript{130:3.6} Aunque estas traducciones se hicieron en Alejandría, Ganid no arregló definitivamente esta selección y añadió sus propias conclusiones personales hasta finales de su estancia en Roma. Se sorprendió mucho al descubrir que los mejores autores de literatura sagrada del mundo reconocían todos, más o menos claramente, la existencia de un Dios eterno, y estaban en gran parte de acuerdo en cuanto al carácter de este Dios y sus relaciones con el hombre mortal.

\par 
%\textsuperscript{(1433.2)}
\textsuperscript{130:3.7} Jesús y Ganid pasaron mucho tiempo en el museo durante su estancia en Alejandría. Este museo no era una colección de objetos raros, sino más bien una universidad de bellas artes, ciencia y literatura. Profesores eruditos daban allí conferencias diarias, y en aquellos tiempos era el centro intelectual del mundo occidental. Día tras día, Jesús interpretaba las conferencias para Ganid. Cierto día, durante la segunda semana, el joven exclamó: <<Maestro Josué, tú sabes más que todos estos profesores; deberías levantarte y decirles las grandes cosas que me has enseñado. Están confundidos porque piensan demasiado. Hablaré con mi padre para que arregle esto>>. Jesús sonrió y le dijo: <<Eres un alumno admirativo, pero estos maestros no están dispuestos a que tú y yo les enseñemos nada. El orgullo de la erudición no espiritualizada es una trampa en la experiencia humana. El verdadero maestro mantiene su integridad intelectual permaneciendo siempre como un alumno>>.

\par 
%\textsuperscript{(1433.3)}
\textsuperscript{130:3.8} Alejandría era el lugar donde se mezclaban las culturas occidentales, y la ciudad más grande y magnífica del mundo después de Roma. Aquí se encontraba la sinagoga judía más grande del mundo, con la sede administrativa del sanedrín de Alejandría, los setenta ancianos dirigentes.

\par 
%\textsuperscript{(1433.4)}
\textsuperscript{130:3.9} Entre las muchas personas con quienes Gonod hizo transacciones mercantiles, había cierto banquero judío llamado Alejandro, cuyo hermano Filón era un famoso filósofo religioso de esta época. Filón había emprendido la tarea elogiable, pero extremadamente difícil, de armonizar la filosofía griega con la teología hebrea. Ganid y Jesús conversaron mucho sobre las enseñanzas de Filón y esperaban asistir a algunas de sus conferencias, pero durante toda su estancia en Alejandría este famoso judío helenista estuvo enfermo en la cama.

\par 
%\textsuperscript{(1433.5)}
\textsuperscript{130:3.10} Jesús elogió a Ganid muchos aspectos de la filosofía griega y de la doctrina de los estoicos, pero le inculcó la verdad de que estos sistemas de creencias, así como las enseñanzas imprecisas de algunos compatriotas de Ganid, sólo eran religiones en el sentido de que conducían a los hombres a encontrar a Dios y a disfrutar la experiencia viviente de conocer al Eterno.

\section*{4. El discurso sobre la realidad}
\par 
%\textsuperscript{(1433.6)}
\textsuperscript{130:4.1} La noche antes de partir de Alejandría, Ganid y Jesús tuvieron una larga conversación con uno de los profesores nombrados por el gobierno en la universidad, que daba una conferencia sobre las enseñanzas de Platón. Jesús hizo de intérprete para el erudito maestro griego, pero no insertó ninguna enseñanza propia que refutara la filosofía griega. Aquella noche, Gonod había salido para asuntos de negocios; por eso, después de la partida del profesor, el maestro y su alumno tuvieron una larga e íntima conversación sobre las doctrinas de Platón. Jesús aprobó de manera moderada algunas de las enseñanzas griegas sobre la teoría de que las cosas materiales del mundo eran vagos reflejos de las realidades espirituales invisibles, pero más sustanciales. Sin embargo, trató de establecer cimientos más sólidos para las reflexiones del joven, y por eso se embarcó en una larga disertación sobre la naturaleza de la realidad en el universo. He aquí en esencia y en lenguaje moderno lo que Jesús dijo a Ganid:

\par 
%\textsuperscript{(1434.1)}
\textsuperscript{130:4.2} La fuente de la realidad universal es el Infinito. Las cosas materiales de la creación finita son las repercusiones espacio-temporales del Arquetipo Paradisíaco y de la Mente Universal del Dios eterno. La causalidad en el mundo físico, la conciencia de sí en el mundo intelectual y el yo progresivo en el mundo espiritual ---estas realidades, proyectadas a escala universal, combinadas en una conexión eterna y experimentadas con cualidades perfectas y valores divinos--- constituyen \textit{la realidad del Supremo}. Pero en el universo siempre cambiante, la Personalidad Original de la causalidad, de la inteligencia y de la experiencia espiritual permanece inmutable, absoluta. Incluso en un universo eterno de valores ilimitados y de cualidades divinas, todas las cosas pueden cambiar y cambian con frecuencia, excepto los Absolutos y aquello que ha alcanzado el estado físico, el contenido intelectual o la identidad espiritual que sean absolutos.

\par 
%\textsuperscript{(1434.2)}
\textsuperscript{130:4.3} El nivel más alto que pueden alcanzar las criaturas finitas es el reconocimiento del Padre Universal y el conocimiento del Supremo. Incluso entonces, estos seres destinados a la finalidad continúan experimentando cambios en los movimientos del mundo físico y en sus fenómenos materiales. Asimismo, siguen siendo conscientes del progreso del yo en su continua ascensión por el universo espiritual, y experimentan una conciencia creciente de su apreciación cada vez más profunda del cosmos intelectual y de su reacción al mismo. La criatura solamente puede unificarse con el Creador mediante la perfección, la armonía y la unanimidad de la voluntad; este estado de divinidad sólo se puede alcanzar y mantener si la criatura continúa viviendo en el tiempo y en la eternidad conformando constantemente su voluntad personal finita a la voluntad divina del Creador. El deseo de hacer la voluntad del Padre siempre ha de ser supremo en el alma y debe dominar la mente de un hijo ascendente de Dios.

\par 
%\textsuperscript{(1434.3)}
\textsuperscript{130:4.4} Un tuerto nunca podrá percibir la profundidad de una perspectiva. De la misma manera, los científicos materialistas tuertos y los místicos y alegoristas espirituales tuertos tampoco pueden tener una visión correcta, ni pueden comprender adecuadamente las verdaderas profundidades de la realidad universal. Todos los valores auténticos de la experiencia de la criatura están ocultos en la profundidad del reconocimiento.

\par 
%\textsuperscript{(1434.4)}
\textsuperscript{130:4.5} Una causación desprovista de mente no puede transformar lo rudimentario y lo simple en elementos refinados y complejos; la experiencia sin el espíritu tampoco puede hacer que las mentes materiales de los mortales del tiempo se conviertan en caracteres divinos de supervivencia eterna. El único atributo del universo que caracteriza tan exclusivamente a la Deidad infinita es la perpetua donación creativa de la personalidad, que puede sobrevivir alcanzando progresivamente a la Deidad.

\par 
%\textsuperscript{(1434.5)}
\textsuperscript{130:4.6} La personalidad es esa dotación cósmica, esa fase de la realidad universal, que puede coexistir con unos cambios ilimitados y al mismo tiempo conservar su identidad en presencia misma de todos esos cambios, e indefinidamente después de ellos.

\par 
%\textsuperscript{(1434.6)}
\textsuperscript{130:4.7} La vida es una adaptación de la causalidad cósmica original a las exigencias y posibilidades de las situaciones universales; surge a la existencia mediante la acción de la Mente Universal y la activación de la chispa espiritual del Dios que es espíritu. El significado de la vida es su adaptabilidad; el valor de la vida es su capacidad para el progreso ---incluso hasta las alturas de la conciencia de Dios.

\par 
%\textsuperscript{(1434.7)}
\textsuperscript{130:4.8} La mala adaptación de la vida autoconsciente al universo produce la desarmonía cósmica. Si la voluntad de la personalidad diverge definitivamente de la tendencia de los universos, termina en el aislamiento intelectual, en la segregación de la personalidad. La pérdida del piloto espiritual interior sobreviene con el cese espiritual de la existencia. Así pues, la vida inteligente y progresiva es, en sí misma y por sí misma, una prueba incontrovertible de la existencia de un universo intencional que expresa la voluntad de un Creador divino. Y esta vida, en su conjunto, lucha por alcanzar los valores superiores, teniendo como meta final al Padre Universal.

\par 
%\textsuperscript{(1435.1)}
\textsuperscript{130:4.9} Aparte de los servicios superiores y casi espirituales del intelecto, la mente del hombre sólo sobrepasa el nivel animal en cuestión de grados. Por eso, los animales (que carecen de culto y de sabiduría) no pueden experimentar la superconciencia, la conciencia de la conciencia. La mente animal sólo es consciente del universo objetivo.

\par 
%\textsuperscript{(1435.2)}
\textsuperscript{130:4.10} El conocimiento es la esfera de la mente material, la que discierne los hechos. La verdad es el dominio del intelecto espiritualmente dotado que es consciente de conocer a Dios. El conocimiento se puede demostrar; la verdad se experimenta. El conocimiento es una posesión de la mente; la verdad una experiencia del alma, del yo que progresa. El conocimiento es una función del nivel no espiritual; la verdad es una fase del nivel mental-espiritual de los universos. La visión de la mente material percibe un mundo de conocimiento basado en hechos; la visión del intelecto espiritualizado discierne un mundo de valores verdaderos. Estos dos puntos de vista, sincronizados y armonizados, revelan el mundo de la realidad, en el cual la sabiduría interpreta los fenómenos del universo en términos de experiencia personal progresiva.

\par 
%\textsuperscript{(1435.3)}
\textsuperscript{130:4.11} El error (el mal) es la consecuencia de la imperfección. Las características de la imperfección, o los hechos de la mala adaptación, se revelan en el nivel material mediante la observación crítica y el análisis científico; en el nivel moral se revelan mediante la experiencia humana. La presencia del mal constituye la prueba de las inexactitudes de la mente y de la inmadurez del yo en evolución. Así pues, el mal es también una medida de la imperfección con que se interpreta el universo. La posibilidad de cometer errores es inherente a la adquisición de la sabiduría, el plan según el cual se progresa desde lo parcial y temporal a lo completo y eterno, desde lo relativo e imperfecto a lo definitivo y perfeccionado. El error es la sombra del estado incompleto relativo, que necesariamente debe proyectarse en medio del camino universal ascendente del hombre hacia la perfección del Paraíso. El error (el mal) no es una peculiaridad real del universo; es simplemente la observación de una relatividad en las relaciones entre la imperfección de lo finito incompleto y los niveles ascendentes del Supremo y del Último.

\par 
%\textsuperscript{(1435.4)}
\textsuperscript{130:4.12} Aunque Jesús expuso todo esto al joven en el lenguaje más apropiado para su comprensión, Ganid tenía los párpados pesados al final de la explicación y pronto cayó presa del sueño. A la mañana siguiente, se levantaron temprano para subir a bordo del barco con rumbo a Lasea, en la isla de Creta. Pero antes de embarcarse, el muchacho aún tenía que hacer más preguntas sobre el mal, a las cuales Jesús respondió:

\par 
%\textsuperscript{(1435.5)}
\textsuperscript{130:4.13} El mal es un concepto de la relatividad. Surge al observarse las imperfecciones que aparecen en la sombra proyectada por un universo finito de cosas y de seres, cuando este cosmos oscurece la luz viviente de la expresión universal de las realidades eternas del Uno Infinito.

\par 
%\textsuperscript{(1435.6)}
\textsuperscript{130:4.14} El mal potencial es inherente al estado necesariamente incompleto de la revelación de Dios como expresión, limitada por el espacio-tiempo, de la infinidad y de la eternidad. El hecho de lo parcial en presencia de lo completo constituye la relatividad de la realidad; crea la necesidad de escoger intelectualmente, y establece unos niveles de valores en nuestra capacidad para reconocer y responder al espíritu. El concepto incompleto y finito que la mente temporal y limitada de la criatura posee del Infinito es, en sí mismo y por sí mismo, \textit{el mal potencial}. Pero el error cada vez mayor de no efectuar, injustificadamente, una rectificación espiritual razonable de estas desarmonías intelectuales e insuficiencias espirituales, originalmente inherentes, equivale a cometer \textit{el mal efectivo}.

\par 
%\textsuperscript{(1436.1)}
\textsuperscript{130:4.15} Todos los conceptos estáticos y muertos son potencialmente malos. La sombra finita de la verdad relativa y viviente está en continuo movimiento. Los conceptos estáticos retrasan invariablemente la ciencia, la política, la sociedad y la religión. Los conceptos estáticos pueden representar cierto conocimiento, pero les falta sabiduría y están desprovistos de verdad. Sin embargo, no permitáis que el concepto de la relatividad os desoriente tanto que no podáis reconocer la coordinación del universo bajo la dirección de la mente cósmica, y su control estabilizado mediante la energía y el espíritu del Supremo.

\section*{5. En la isla de Creta}
\par 
%\textsuperscript{(1436.2)}
\textsuperscript{130:5.1} Al ir a Creta, los viajeros no tenían otra intención que la de distraerse, pasear por la isla y escalar las montañas. Los cretenses de esta época no disfrutaban de una reputación envidiable entre los pueblos vecinos. Sin embargo, Jesús y Ganid consiguieron que muchas almas realzaran sus niveles de pensamiento y de vida, estableciendo así las bases para la rápida aceptación de las enseñanzas evangélicas posteriores, cuando llegaron los primeros predicadores de Jerusalén. Jesús amaba a estos cretenses, a pesar de las duras palabras que Pablo pronunció más tarde sobre ellos, cuando envió a Tito a la isla para reorganizar sus iglesias\footnote{\textit{Duras palabras de Pablo}: Tit 1:10-16.}.

\par 
%\textsuperscript{(1436.3)}
\textsuperscript{130:5.2} En la ladera de una montaña de Creta, Jesús tuvo su primera larga conversación con Gonod sobre la religión. El padre se quedó muy impresionado y dijo: <<No me extraña que el chico se crea todo lo que le dices; pero yo no sabía que tuvieran una religión así en Jerusalén, y mucho menos en Damasco>>. Fue durante la estancia en esta isla cuando Gonod propuso por primera vez a Jesús que fuera con ellos a la India, y Ganid estuvo encantado con la idea de que Jesús pudiera aceptar este arreglo.

\par 
%\textsuperscript{(1436.4)}
\textsuperscript{130:5.3} Cierto día, Ganid preguntó a Jesús por qué no se había dedicado a enseñar públicamente, y éste le respondió: <<Hijo mío, todo debe aguardar su hora. Has nacido en el mundo, pero ninguna cantidad de ansiedad y ninguna manifestación de impaciencia te ayudarán a crecer. En todos estos asuntos hay que darle tiempo al tiempo. Sólo el tiempo hace que la fruta verde madure en el árbol. Una estación sucede a la otra y el atardecer sigue al amanecer únicamente con el paso del tiempo. Ahora estoy camino de Roma con tu padre y contigo, y esto es suficiente por hoy. Mi mañana esta enteramente en las manos de mi Padre celestial>>. Entonces contó a Ganid la historia de Moisés y de sus cuarenta años de espera vigilante y de preparación continua.

\par 
%\textsuperscript{(1436.5)}
\textsuperscript{130:5.4} Durante la visita a Buenos Puertos\footnote{\textit{Buenos Puertos}: Hch 27:8.} se produjo un incidente que Ganid no olvidó nunca. El recuerdo de este episodio siempre le despertó el deseo de hacer algo para cambiar el sistema de castas de su India natal. Un borracho degenerado estaba atacando a una joven esclava en la vía pública. Cuando Jesús vio el apuro de la chica, se abalanzó y alejó a la doncella del asalto del perturbado. Mientras la niña aterrorizada se agarraba a él, Jesús mantuvo al hombre enfurecido a una distancia prudencial con su poderoso brazo derecho extendido, hasta que el pobre tipo se agotó de tanto lanzar golpes furiosos en el aire. Ganid sintió el fuerte impulso de ayudar a Jesús a manejar este incidente, pero su padre se lo prohibió. Aunque no hablaban el idioma de la joven, ésta podía entender su acto de misericordia y les manifestó su profunda gratitud mientras los tres la acompañaban hasta su casa. En toda su vida encarnada, probablemente Jesús nunca estuvo tan cerca de pelearse con uno de sus contemporáneos como en esta ocasión. Aquella tarde le costó trabajo hacer entender a Ganid por qué no había golpeado al borracho. Ganid pensaba que este hombre debería haber recibido por lo menos tantos golpes como había dado a la joven.

\section*{6. El joven que tenía miedo}
\par 
%\textsuperscript{(1437.1)}
\textsuperscript{130:6.1} Mientras estaban en las montañas, Jesús tuvo una larga conversación con un joven que estaba temeroso y abatido. No pudiendo encontrar ánimo y consuelo en la relación con sus semejantes, este joven había buscado la soledad de las colinas; había crecido con un sentimiento de desamparo e inferioridad. Estas tendencias naturales se habían visto acrecentadas por las numerosas circunstancias difíciles que el muchacho había sufrido a medida que crecía, principalmente la pérdida de su padre cuando tenía doce años. Al encontrarse con él, Jesús le dijo: <<¡Saludos, amigo mío!, ¿por qué estás tan triste en un día tan hermoso?. Si ha sucedido algo que te aflija, quizás pueda ayudarte de alguna manera. En todo caso, es para mi un placer ofrecerte mis servicios>>.

\par 
%\textsuperscript{(1437.2)}
\textsuperscript{130:6.2} El joven estaba poco dispuesto a hablar, por lo que Jesús intentó otra manera de acercarse a su alma, diciendo: <<Comprendo que subas a estos montes para huir de la gente; por eso es natural que no quieras conversar conmigo, pero me gustaría saber si te son familiares estas colinas. ¿Conoces la dirección de estos senderos?. ¿Y podrías quizás indicarme cuál es el mejor camino para ir a Fénix?>>. El joven conocía muy bien aquellas montañas, y se interesó tanto en mostrar a Jesús el camino de Fénix, que dibujó en la tierra todos los senderos, explicándolos con todo detalle. Pero se quedó sorprendido y lleno de curiosidad cuando Jesús, después de decirle adiós y de hacer como el que se iba, se volvió repentinamente hacia él diciendo: <<Sé muy bien que deseas quedarte a solas con tu desconsuelo; pero no sería ni amable ni justo por mi parte recibir de ti una ayuda tan generosa para encontrar el mejor camino de llegar a Fénix, y luego alejarme despreocupadamente sin hacer el menor esfuerzo por responder a tu petición de ayuda y orientación para encontrar el mejor camino hacia el destino que buscas en tu corazón mientras permaneces aquí en la ladera de la montaña. Al igual que tú conoces muy bien los senderos que conducen a Fénix, por haberlos recorrido muchas veces, yo conozco bien el camino de la ciudad de tus esperanzas frustradas y de tus ambiciones contrariadas. Y puesto que me has pedido ayuda, no te decepcionaré>>. El joven se quedó prácticamente atónito, y apenas logró balbucear: <<Pero... si no te he pedido nada>>. Entonces Jesús, poniéndole suavemente la mano en el hombro, le dijo: <<No, hijo, no con palabras, pero apelaste a mi corazón con tu mirada anhelante. Hijo mío, para el que ama a sus semejantes hay una elocuente petición de ayuda en tu actitud de desaliento y desesperación. Siéntate a mi lado mientras te hablo de los senderos del servicio y de los caminos de la felicidad, que conducen desde las penas del yo a las alegrías de las actividades afectuosas en la fraternidad de los hombres y en el servicio del Dios del cielo>>.

\par 
%\textsuperscript{(1437.3)}
\textsuperscript{130:6.3} En aquel momento el joven sentía muchos deseos de hablar con Jesús, y se arrodilló a sus pies suplicándole que lo ayudara, que le mostrara el camino para escapar de su mundo de penas y fracasos personales. Jesús le dijo: <<Amigo mío, ¡levántate!. ¡Ponte de pie como un hombre!. Puedes estar rodeado de enemigos mezquinos y muchos obstáculos pueden retrasar tu marcha, pero las cosas importantes y reales de este mundo y del universo están de tu parte. El Sol sale todas las mañanas para saludarte exactamente igual que lo hace para el hombre más poderoso y próspero de la Tierra. Mira ---tienes un cuerpo fuerte y músculos poderosos--- tus facultades físicas son superiores a la media. Por supuesto, todo eso no sirve prácticamente para nada mientras te quedes aquí sentado en la ladera de la montaña lamentándote de tus desgracias, reales o imaginarias. Pero podrías hacer grandes cosas con tu cuerpo si quisieras apresurarte hacia los lugares donde hay grandes cosas por hacerse. Tratas de huir de tu yo infeliz, pero eso no es posible. Tú y los problemas de tu vida son reales; no puedes huir de ellos mientras estés vivo. Pero mira además, tu mente es clara y capaz. Tu cuerpo robusto tiene una mente inteligente que lo dirige. Pon tu mente a trabajar para resolver sus problemas; enseña a tu intelecto a trabajar para ti. No te dejes dominar por el miedo como un animal sin discernimiento. Tu mente debería ser tu valiente aliada en la resolución de los problemas de tu vida, en lugar de ser tú, como lo has sido, su abyecto esclavo atemorizado y el siervo de la depresión y de la derrota. Pero lo más valioso de todo, tu verdadero potencial de realización, es el espíritu que vive dentro de ti; él estimulará e inspirará tu mente para que se controle a sí misma y active a tu cuerpo si deseas liberarlo de las cadenas del miedo; así permitirás que tu naturaleza espiritual comience a liberarte de los males de la indolencia, gracias a la presencia y al poder de la fe viviente. Verás entonces cómo esta fe vencerá tu miedo a los hombres mediante la presencia irresistible de ese nuevo y predominante \textit{amor por tus semejantes} que pronto llenará tu alma hasta rebosar, porque en tu corazón habrá nacido la conciencia de que eres un hijo de Dios>>.

\par 
%\textsuperscript{(1438.1)}
\textsuperscript{130:6.4} <<En este día, hijo mío, has de nacer de nuevo, restablecido como un hombre de fe, de valor y de servicio consagrado a los hombres por amor a Dios. Cuando te hayas reajustado así a la vida, dentro de ti mismo, también te habrás reajustado con el universo; habrás nacido de nuevo ---nacido del espíritu--- y en adelante toda tu vida será una consecución victoriosa. Los problemas te fortificarán, las decepciones te espolearán, las dificultades serán un desafío y los obstáculos, un estímulo. ¡Levántate, joven!. Di adiós a la vida de temores serviles y de huidas cobardes. Regresa rápidamente a tu deber y vive tu vida en la carne como un hijo de Dios, como un mortal dedicado al servicio ennoblecedor del hombre en la Tierra, y destinado al magnífico y perpetuo servicio de Dios en la eternidad>>.

\par 
%\textsuperscript{(1438.2)}
\textsuperscript{130:6.5} Este joven, llamado Fortunato, se convirtió más tarde en el jefe de los cristianos de Creta y en el íntimo asociado de Tito en sus esfuerzos por elevar a los creyentes cretenses\footnote{\textit{Creyentes cretenses}: Tit 1:5.}.

\par 
%\textsuperscript{(1438.3)}
\textsuperscript{130:6.6} Los viajeros estaban realmente descansados y dispuestos cuando un buen día, a mediodía, se prepararon para zarpar hacia Cartago, en el norte de África, deteniéndose dos días en Cirene. Es aquí donde Jesús y Ganid prestaron sus primeros auxilios a un muchacho llamado Rufo, que había resultado herido al desplomarse una carreta de bueyes cargada. Lo llevaron a la casa de su madre, y en cuanto a su padre, Simón, jamás podía imaginar que el hombre cuya cruz llevaría más tarde, por orden de un soldado romano, era el mismo extranjero que en otro tiempo había socorrido a su hijo\footnote{\textit{Simón el portador de la cruz}: Mt 27:32; Mc 15:21; Lc 23:26.}.

\section*{7. En Cartago --- el discurso sobre el tiempo y el espacio}
\par 
%\textsuperscript{(1438.4)}
\textsuperscript{130:7.1} Durante la ruta hacia Cartago, Jesús pasó la mayoría del tiempo conversando con sus compañeros de viaje sobre temas sociales, políticos y comerciales, pero no se dijo casi nada sobre religión. Por primera vez, Gonod y Ganid descubrieron que Jesús era un buen narrador, y lo mantuvieron ocupado contando anécdotas de sus primeros años de vida en Galilea. También se enteraron de que se había criado en Galilea y no en Jerusalén ni en Damasco.

\par 
%\textsuperscript{(1438.5)}
\textsuperscript{130:7.2} Ganid había notado que la mayoría de las personas que habían encontrado por casualidad se sentían atraídas por Jesús, y por ello preguntó qué tenía uno que hacer para ganar amigos. Su maestro le dijo: <<Interésate por tus semejantes; aprende a amarlos y vigila la oportunidad de hacer algo por ellos que estás seguro que desean>>; luego citó el antiguo proverbio judío: <<Un hombre que quiere tener amigos debe mostrarse amistoso>>\footnote{\textit{Un hombre que quiere amigos debe ser amistoso}: Pr 18:24.}.

\par 
%\textsuperscript{(1439.1)}
\textsuperscript{130:7.3} En Cartago, Jesús tuvo una larga conversación memorable con un sacerdote mitríaco sobre la inmortalidad, el tiempo y la eternidad. Este persa se había educado en Alejandría y deseaba realmente aprender de Jesús. En respuesta a sus numerosas preguntas, y traducido a terminología moderna, Jesús dijo en sustancia lo siguiente:

\par 
%\textsuperscript{(1439.2)}
\textsuperscript{130:7.4} El tiempo es la corriente de los acontecimientos temporales que fluyen, percibidos por la conciencia de la criatura. El tiempo es un nombre que se ha dado al orden en que suceden los acontecimientos, que permite reconocerlos y separarlos. El universo del espacio es un fenómeno relacionado con el tiempo cuando es observado desde cualquier posición interior fuera de la morada fija del Paraíso. El movimiento del tiempo sólo se revela en relación con algo que no se mueve en el espacio como un fenómeno del tiempo. En el universo de universos, el Paraíso y sus Deidades trascienden tanto el tiempo como el espacio. En los mundos habitados, la personalidad humana (habitada y orientada por el espíritu del Padre Paradisiaco) es la única realidad relacionada con lo físico que puede trascender la secuencia material de los acontecimientos temporales.

\par 
%\textsuperscript{(1439.3)}
\textsuperscript{130:7.5} Los animales no perciben el tiempo como el hombre, e incluso para el hombre, debido a su punto de vista fragmentario y circunscrito, el tiempo aparece como una sucesión de acontecimientos; pero a medida que el hombre asciende, que progresa interiormente, su visión de esta procesión de acontecimientos aumenta de tal manera que la discierne cada vez más en su totalidad. Lo que anteriormente aparecía como una sucesión de acontecimientos se verá ahora como un ciclo completo y perfectamente relacionado; de esta manera, la simultaneidad circular desplazará cada vez más a la antigua conciencia de la secuencia lineal de los acontecimientos.

\par 
%\textsuperscript{(1439.4)}
\textsuperscript{130:7.6} Hay siete conceptos diferentes del espacio tal como está condicionado por el tiempo. El espacio se mide por el tiempo y no el tiempo por el espacio. La confusión de los científicos surge de que no logran reconocer la realidad del espacio. El espacio no es simplemente un concepto intelectual de la variación en la conexión de los objetos del universo. El espacio no está vacío, y la mente es la única cosa que el hombre conoce que puede trascender, aunque sea parcialmente, el espacio. La mente puede funcionar independientemente del concepto de la conexión espacial de los objetos materiales. El espacio es relativa y comparativamente finito para todos los seres con estatus de criatura. Cuanto más se aproxima la conciencia a la noción de las siete dimensiones cósmicas, el concepto de espacio potencial se aproxima más a la ultimidad. Pero el potencial del espacio sólo es realmente último en el nivel absoluto.

\par 
%\textsuperscript{(1439.5)}
\textsuperscript{130:7.7} Debe ser evidente que la realidad universal tiene un significado siempre relativo y en expansión en los niveles ascendentes y en vías de perfeccionamiento del cosmos. A fin de cuentas, los mortales sobrevivientes alcanzan la identidad en un universo de siete dimensiones.

\par 
%\textsuperscript{(1439.6)}
\textsuperscript{130:7.8} El concepto espacio-temporal de una mente de origen material está destinado a sufrir ampliaciones sucesivas a medida que la personalidad consciente que lo concibe asciende los niveles del universo. Cuando el hombre alcanza la mente que media entre los planos material y espiritual de existencia, sus ideas del espacio-tiempo se amplían enormemente en cuanto a la calidad de percepción y a la cantidad de experiencia. Los conceptos cósmicos crecientes de una personalidad espiritual que progresa se deben al aumento tanto de la profundidad de la perspicacia como del campo de la conciencia. A medida que la personalidad continúa su camino hacia arriba y hacia el interior hasta los niveles trascendentales de semejanza con la Deidad, el concepto del espacio-tiempo se acercará cada vez más a los conceptos sin tiempo y sin espacio de los Absolutos. Relativamente, y según sus logros trascendentales, los hijos con destino último llegarán a percibir estos conceptos del nivel absoluto.

\section*{8. En el camino a Neápolis y Roma}
\par 
%\textsuperscript{(1440.1)}
\textsuperscript{130:8.1} La primera escala en el camino de Italia era la isla de Malta. Jesús tuvo aquí una larga conversación con un joven abatido y desanimado llamado Claudo. Este muchacho había pensado en quitarse la vida, pero cuando terminó de conversar con el escriba de Damasco, dijo: <<Voy a afrontar la vida como un hombre; basta ya de hacer el cobarde. Voy a volver con mi gente y empezar de nuevo>>. Poco tiempo después se convirtió en un predicador entusiasta de los cínicos, y más tarde aún se unió a Pedro para proclamar el cristianismo en Roma y en Nápoles. Después de la muerte de Pedro fue a España a predicar el evangelio, pero no supo nunca que el hombre que lo había inspirado en Malta era el mismo Jesús a quien posteriormente proclamó como Liberador del mundo.

\par 
%\textsuperscript{(1440.2)}
\textsuperscript{130:8.2} En Siracusa pasaron una semana completa. El acontecimiento más notable de esta escala fue la rehabilitación de Esdras, el judío descarriado, que tenía la taberna donde Jesús y sus compañeros se habían hospedado. A Esdras le encantó la facilidad de trato de Jesús y le pidió que lo ayudara a volver a la fe de Israel. Expresó su desesperanza diciendo: <<Quiero ser un verdadero hijo de Abraham, pero no consigo encontrar a Dios>>. Jesús le dijo: \guillemotleft Si quieres realmente encontrar a Dios, ese deseo es en sí mismo la prueba de que ya lo has encontrado\footnote{\textit{Busca y encontrarás}: Mt 7:7-8; Lc 11:9-10.}. Tu problema no es que no puedas encontrar a Dios, porque el Padre ya te ha encontrado; tu problema es simplemente que no conoces a Dios. ¿Acaso no has leído en el profeta Jeremías: `Me buscarás y me encontrarás cuando me busques con todo tu corazón'?\footnote{\textit{Me buscarás y me encontrarás}: Jer 29:13.}. Y además, ¿no dice también este mismo profeta: `Te daré un corazón para que me conozcas, que yo soy el Señor, y tú pertenecerás a mi pueblo, y yo seré tu Dios'?\footnote{\textit{Te daré un corazón para que me conozcas}: Jer 24:7.}. ¿Y no has leído también en las escrituras donde dice: `Él mira a los hombres, y si alguno dijera: He pecado y he pervertido lo que era justo, y no me ha aprovechado, entonces Dios liberará de las tinieblas el alma de ese hombre, y verá la luz'?\footnote{\textit{Él mira a los hombres}: Job 33:27-28.}\guillemotright. Entonces Esdras encontró a Dios para satisfacción de su alma. Posteriormente, en asociación con un próspero prosélito griego, este judío construyó la primera iglesia cristiana de Siracusa.

\par 
%\textsuperscript{(1440.3)}
\textsuperscript{130:8.3} En Mesina se detuvieron un solo día, pero lo suficiente como para cambiar la vida de un muchacho, un vendedor de frutas; Jesús le compró frutas y a su vez lo alimentó con el pan de la vida. El muchacho no olvidó nunca las palabras de Jesús y la bondadosa mirada que las acompañó cuando, apoyando la mano sobre su hombro, le dijo: <<Adiós, hijo mío, sé valiente mientras te haces hombre, y después de alimentar el cuerpo, aprende también a alimentar el alma. Mi Padre que está en los cielos estará contigo y te guiará>>. El muchacho se hizo devoto de la religión mitríaca, y posteriormente se convirtió a la fe cristiana.

\par 
%\textsuperscript{(1440.4)}
\textsuperscript{130:8.4} Por fin llegaron a Nápoles, y tuvieron el sentimiento de que ya no estaban lejos de su destino final, Roma. Gonod tenía muchos negocios que tratar en Nápoles; aparte del tiempo en que Jesús era necesario como intérprete, él y Ganid dedicaron sus ratos libres a visitar y explorar la ciudad. Ganid se estaba haciendo experto en detectar a aquellos que parecían necesitar ayuda. Encontraron mucha pobreza en esta ciudad y distribuyeron muchas limosnas. Pero Ganid nunca comprendió el significado de las palabras de Jesús cuando le vio dar, en la calle, una moneda a un mendigo, y se negó a detenerse y consolar al hombre. Jesús dijo: <<¿Por qué malgastar palabras con alguien que no puede percibir el significado de lo que dices? El espíritu del Padre no puede enseñar y salvar a alguien que no tiene capacidad para la filiación>>. Jesús quería decir que el hombre no tenía una mente normal, que carecía de la facultad de responder a la guía del espíritu.

\par 
%\textsuperscript{(1441.1)}
\textsuperscript{130:8.5} En Nápoles no tuvo lugar ninguna experiencia sobresaliente; Jesús y el joven recorrieron toda la ciudad y repartieron buen ánimo con muchas sonrisas a centenares de hombres, mujeres y niños.

\par 
%\textsuperscript{(1441.2)}
\textsuperscript{130:8.6} Desde aquí siguieron hacia Roma por el camino de Capua, donde permanecieron tres días. Por la Vía Apia continuaron su viaje en dirección a Roma junto a sus animales de carga, ansiosos los tres por ver a esta dueña del imperio, la ciudad más grande del mundo entero.


\chapter{Documento 131. Las religiones del mundo}
\par 
%\textsuperscript{(1442.1)}
\textsuperscript{131:0.1} DURANTE la estancia de Jesús, Gonod y Ganid en Alejandría, el joven pasó una gran parte de su tiempo, y gastó no poca cantidad del dinero de su padre, recopilando las enseñanzas de las religiones del mundo sobre Dios y sus relaciones con el hombre mortal. Ganid empleó más de sesenta traductores eruditos para redactar este resumen de las doctrinas religiosas del mundo relativas a las Deidades. Y se debe poner de manifiesto en este relato que todas estas enseñanzas que describen al monoteísmo procedían en gran medida, directa o indirectamente, de las predicaciones de los misioneros de Maquiventa Melquisedek, que partieron de su sede en Salem para divulgar la doctrina de un Dios único ---el Altísimo--- hasta los confines de la Tierra.

\par 
%\textsuperscript{(1442.2)}
\textsuperscript{131:0.2} Presentamos aquí un resumen del manuscrito que Ganid preparó en Alejandría y Roma, y que se conservó en la India durante cientos de años después de su muerte. Organizó este material bajo los diez epígrafes siguientes:

\section*{1. El cinismo}
\par 
%\textsuperscript{(1442.3)}
\textsuperscript{131:1.1} Donde mejor se conservaron los residuos de las enseñanzas de los discípulos de Melquisedek fue en las doctrinas de los cínicos, con excepción de las que sobrevivieron en la religión judía. La selección de Ganid incluía los extractos siguientes:

\par 
%\textsuperscript{(1442.4)}
\textsuperscript{131:1.2} <<Dios es supremo; es el Altísimo del cielo y de la Tierra. Dios es el círculo perfeccionado de la eternidad, y gobierna el universo de universos. Es el único hacedor de los cielos y de la Tierra. Cuando decreta una cosa, esa cosa es. Nuestro Dios es un Dios único, y es compasivo y misericordioso. Todo lo que es elevado, santo, verdadero y bello es semejante a nuestro Dios. El Altísimo es la luz del cielo y de la Tierra; es el Dios del este, del oeste, del norte y del sur>>.

\par 
%\textsuperscript{(1442.5)}
\textsuperscript{131:1.3} <<Aunque la Tierra tuviera que desaparecer, la faz resplandeciente del Supremo permanecería en majestad y gloria. El Altísimo es el primero y el último, el principio y el fin de todas las cosas. No hay más que un solo Dios y su nombre es Verdad. Dios existe por sí mismo, y está exento de toda cólera y enemistad; es inmortal e infinito. Nuestro Dios es omnipotente y generoso. Aunque sus manifestaciones son numerosas, adoramos solamente a Dios mismo. Dios lo sabe todo ---nuestros secretos y nuestras proclamaciones; también sabe lo que merece cada uno de nosotros. No hay nada que sea semejante a su fuerza>>.

\par 
%\textsuperscript{(1442.6)}
\textsuperscript{131:1.4} <<Dios es un dador de paz y un protector fiel de todos los que le temen y confían en él. Da la salvación a todos los que le sirven. Toda la creación existe en el poder del Altísimo. Su amor divino brota de la santidad de su poder, y su afecto nace de la fuerza de su grandeza. El Altísimo ha decretado la unión del cuerpo y del alma y ha dotado al hombre de su propio espíritu. Lo que el hombre hace debe tener un final, pero lo que el Creador hace permanece para siempre. La experiencia humana nos aporta conocimiento, pero la contemplación del Altísimo nos da sabiduría>>.

\par 
%\textsuperscript{(1443.1)}
\textsuperscript{131:1.5} <<Dios derrama la lluvia sobre la tierra, hace brillar el Sol sobre el grano que germina, nos da la abundante cosecha de las cosas buenas de esta vida y la salvación eterna en el mundo por venir. Nuestro Dios goza de una gran autoridad; su nombre es Excelente y su naturaleza insondable. Cuando estáis enfermos, el Altísimo es quien os devuelve la salud. Dios está lleno de bondad hacia todos los hombres; no tenemos ningún amigo como el Altísimo. Su misericordia llena todos los lugares y su bondad abarca todas las almas. El Altísimo es inmutable y nos ayuda en los momentos de necesidad. Dondequiera que os dirijáis para orar, allí está la faz del Altísimo y el oído atento de nuestro Dios. Podéis esconderos de los hombres, pero no de Dios. Dios no está lejos de nosotros; es omnipresente. Dios llena todos los lugares y vive en el corazón del hombre que teme su santo nombre. La creación está en el Creador y el Creador en su creación. Buscamos al Altísimo y entonces lo encontramos en nuestro corazón. Vais en busca de un amigo querido, y luego lo descubrís en vuestra alma>>.

\par 
%\textsuperscript{(1443.2)}
\textsuperscript{131:1.6} <<El hombre que conoce a Dios considera a todos los hombres como sus iguales; son sus hermanos. Los egoístas, los que ignoran a sus hermanos en la carne, sólo reciben el hastío como recompensa. Los que aman a sus semejantes y tienen un corazón puro verán a Dios. Dios nunca olvida la sinceridad. Guiará a los sinceros de corazón hasta la verdad, porque Dios es la verdad>>.

\par 
%\textsuperscript{(1443.3)}
\textsuperscript{131:1.7} <<En vuestra vida, rechazad el error y venced el mal mediante el amor de la verdad viviente. En todas vuestras relaciones con los hombres, devolved bien por mal. El Señor Dios es misericordioso y amante; perdona las deudas. Amemos a Dios, porque él nos amó primero. Por el amor de Dios, y gracias a su misericordia, seremos salvados. Los pobres y los ricos son hermanos. Dios es su Padre. El mal que no queréis que os hagan, no lo hagáis a los demás>>.

\par 
%\textsuperscript{(1443.4)}
\textsuperscript{131:1.8} <<Invocad su nombre en todo momento, y en la medida en que creáis en su nombre, vuestra oración será escuchada. ¡Qué gran honor es adorar al Altísimo! Todos los mundos y todos los universos lo adoran. En todas vuestras oraciones, dad gracias ---elevaos a la adoración. La adoración piadosa evita el mal e impide el pecado. Alabemos en todo momento el nombre del Altísimo. El hombre que se refugia en el Altísimo oculta sus defectos al universo. Cuando os halláis ante Dios con un corazón puro, ya no tenéis miedo a nada en toda la creación. El Altísimo es como un padre y una madre amorosos; nos ama realmente a nosotros, sus hijos en la Tierra. Nuestro Dios nos perdonará y guiará nuestros pasos por el camino de la salvación. Nos cogerá de la mano y nos conducirá hasta él. Dios salva a los que confían en él; no obliga al hombre a servir su nombre>>.

\par 
%\textsuperscript{(1443.5)}
\textsuperscript{131:1.9} <<Si la fe del Altísimo ha penetrado en vuestro corazón, entonces viviréis libres de temor todos los días de vuestra vida. No os irritéis por la prosperidad de los impíos; no temáis a los que traman el mal; dejad que el alma se aparte del pecado y poned toda vuestra confianza en el Dios de la salvación. El alma cansada del mortal errante encuentra descanso eterno en los brazos del Altísimo; el hombre sabio ansía el abrazo divino; el hijo terrestre anhela la seguridad de los brazos del Padre Universal. El hombre noble busca ese estado superior en el que el alma del mortal se mezcla con el espíritu del Supremo. Dios es justo: el fruto que no recibimos por nuestros esfuerzos en este mundo, lo recibiremos en el próximo>>.

\section*{2. El judaísmo}
\par 
%\textsuperscript{(1444.1)}
\textsuperscript{131:2.1} Los kenitas de Palestina salvaron muchas enseñanzas de Melquisedek, y de aquellos archivos, tal como estaban conservados y modificados por los judíos, Jesús y Ganid escogieron los pasajes siguientes:

\par 
%\textsuperscript{(1444.2)}
\textsuperscript{131:2.2} <<En el principio, Dios creó los cielos y la Tierra y todas las cosas que contienen. Y he aquí que todo lo que había creado era muy bueno. Es el Señor el que es Dios; no hay nadie más que él, ni arriba en el cielo ni abajo en la Tierra. Por eso amarás al Señor tu Dios con todo tu corazón, con toda tu alma y con todas tus fuerzas. Al igual que las aguas cubren el mar, la Tierra se llenará con el conocimiento del Señor. Los cielos proclaman la gloria de Dios, y el firmamento muestra la obra de sus manos. Los días, uno tras otro, expresan su discurso, y las noches, una tras otra, muestran el conocimiento. No hay lenguaje o palabra donde no se oiga su voz. La obra del Señor es grande, y ha hecho todas las cosas con sabiduría; la grandeza del Señor es inescrutable. Conoce el número de las estrellas y las llama a todas por su nombre>>\footnote{\textit{Dios creó el Cielo y la Tierra}: Ex 31:17; 2 Re 19:15; 2 Cr 2:12; Sal 115:15-16; 121:2; 124:8; 134:3; Is 37:16; 45:12,18; Jer 10:11-12; 32:17; Ap 14:7. \textit{Dios creó la Tierra}: Is 40:26,28; Am 4:13. \textit{Dios creó mundos}: Heb 1:2. \textit{Dios creó al hombre y la mujer}: Gn 5:1-2. \textit{Dios creó todo}: Gn 1:1-27; 2:4-23; Ex 20:11; 31:17; Neh 9:6; Sal 146:6; Is 42:5; Jer 51:15-16; Mc 13:19; Jn 1:1-3; Hch 4:24; 14:15; Ef 3:9; Col 1:16; 1 P 4:19; Ap 4:11; 10:6. \textit{No hay nadie más que él}: Is 45:21; Dt 4:35,39; 1 Sam 2:2. \textit{Amarás al Señor tu Dios con todo tu corazón}: Dt 6:4-5; 10:12; 11:1,13,22; 13:3; 19:9; 30:6,16,20; Mt 22:37; Mc 12:30; Lc 10:27; Ro 8:28; Jos 22:5; 23:11. \textit{Los cielos proclaman la gloria de Dios}: Sal 19:1-3. \textit{La Tierra se llenará con el conocimiento del Señor}: Is 11:9; Hab 2:14. \textit{La obra del Señor es grande}: Sal 92:5; 111:2. \textit{Dios creó con sabiduría}: Eclo 1:1-4; Bar 3:32-36. \textit{Ha hecho todas las cosas con sabiduría}: Sal 104:24. \textit{La grandeza del Señor es inescrutable}: Sal 145:3. \textit{Conoce el número y nombre de las estrellas}: Sal 147:4.}.

\par 
%\textsuperscript{(1444.3)}
\textsuperscript{131:2.3} <<El poder del Señor es grande y su comprensión, infinita. Dice el Señor: `Así como los cielos son más elevados que la Tierra, mis caminos son más elevados que los vuestros y mis pensamientos más elevados que vuestros pensamientos'. Dios revela las cosas profundas y secretas porque la luz habita en él. El Señor es misericordioso y clemente; es paciente y abunda en bondad y verdad. El Señor es bueno y recto; guiará a los mansos en el juicio. ¡Probad y constatad que el Señor es bueno! Bendito sea el hombre que confía en Dios. Dios es nuestro refugio y nuestra fuerza, una ayuda muy presente en las dificultades>>\footnote{\textit{El poder del Señor es grande y su comprensión, infinita}: Sal 147:5. \textit{Los pensamientos de Dios son más elevados}: Is 55:9. \textit{Dios revela las cosas profundas y secretas}: Dn 2:22. \textit{El Señor es misericordioso y clemente}: Ex 34:6; Sal 103:8. \textit{El Señor es bueno y recto}: Sal 25:8-9. \textit{Probad y constatad que el Señor es bueno}: Sal 34:8. \textit{Bendito sea el hombre que confía en Dios}: Jer 17:7. \textit{Dios es nuestro refugio y nuestra ayuda}: Sal 46:1.}.

\par 
%\textsuperscript{(1444.4)}
\textsuperscript{131:2.4} <<La misericordia del Señor reposa de eternidad en eternidad en aquellos que le temen, y su rectitud llega hasta los hijos de nuestros hijos. El Señor es clemente y está lleno de compasión. El Señor es bueno con todos, y sus tiernas misericordias se extienden por toda su creación; cura a los apesadumbrados y venda sus heridas. ¿Adónde iré lejos del espíritu de Dios? ¿Adónde huiré de la presencia divina? Dice así el Alto y Sublime que vive en la eternidad, cuyo nombre es el Santo: `¡Vivo en el lugar alto y sagrado, y también en aquel que tiene el corazón contrito y el espíritu humilde!' Nadie puede esconderse de nuestro Dios, porque llena el cielo y la Tierra. Que los cielos se alegren y que la Tierra se regocije. ¡Que todas las naciones digan: el Señor reina! Dad gracias a Dios, porque su misericordia dura para siempre>>\footnote{\textit{La misericordia del Señor es eterna}: Sal 103:17. \textit{El Señor es clemente y está lleno de compasión}: Sal 111:4; 145:8. \textit{El Señor es bueno y extiende sus misericordias}: Sal 145:9. \textit{Cura a los apesadumbrados y venda sus heridas}: Sal 147:3. \textit{Vivo en el lugar alto y sagrado}: Is 57:15. \textit{Nadie puede esconderse de nuestro Señor}: Jer 23:24; Sal 139:7. \textit{Que los cielos se alegren}: Sal 96:11. \textit{El Señor reina}: 1 Cr 16:31; Sal 93:1; 96:10. \textit{Dad gracias a Dios por su misericordia}: 1 Cr 16:34. \textit{Su misericordia dura para siempre}: Sal 136:1-26.}.

\par 
%\textsuperscript{(1444.5)}
\textsuperscript{131:2.5} <<Los cielos proclaman la rectitud de Dios, y toda la gente ha visto su gloria. Dios es quien nos ha hecho, y no nosotros mismos; somos su pueblo, las ovejas de sus pastos. Su misericordia es perpetua, y su verdad permanece para todas las generaciones. Nuestro Dios gobierna entre las naciones. ¡Que la Tierra se llene con su gloria! ¡Oh, que los hombres alaben al Señor por su bondad y por sus dones maravillosos a los hijos de los hombres!>>\footnote{\textit{Los cielos proclaman su rectitud y su gloria}: Sal 97:6. \textit{Dios es quien nos ha hecho, no nosotros}: Sal 100:3. \textit{Su misericordia y su verdad es perpetua}: Sal 100:5. \textit{Nuestro Dios gobierna entre las naciones}: Sal 22:28; Dn 4:17,25,32; 5:21. \textit{Que la Tierra se llene con su gloria}: Sal 72:19. \textit{Alabad a Dios por su bondad}: Sal 107:8,15,21,31.}

\par 
%\textsuperscript{(1444.6)}
\textsuperscript{131:2.6} <<Dios ha hecho al hombre un poco menos que divino y lo ha coronado de amor y misericordia. El Señor conoce el camino de los justos, pero la vía de los impíos perecerá. El temor del Señor es el principio de la sabiduría; el conocimiento del Supremo es el entendimiento. Dice el Dios Todopoderoso: `Camina delante de mí y sé perfecto'. No olvidéis que el orgullo va por delante de la destrucción, y un espíritu altivo por delante de la caída. El que gobierna su propio espíritu es más poderoso que el que conquista una ciudad. Dice el Señor Dios, el Santo: `Cuando volváis a vuestro reposo espiritual seréis salvados; en la quietud y en la confianza encontraréis vuestra fuerza'. Los que esperan en el Señor renovarán sus fuerzas; se elevarán con alas como las águilas. Correrán y no se cansarán; caminarán y no desmayarán. El Señor apaciguará vuestro temor. Dice el Señor: `No temáis, porque estoy con vosotros. No desmayéis, porque soy vuestro Dios. Yo os fortaleceré; yo os ayudaré; sí, yo os sostendré con la diestra de mi justicia'>>\footnote{\textit{El hombre creado un poco menos que divino}: Sal 8:5. \textit{El Señor conoce el camino de los justos}: Sal 1:6. \textit{El temor del Señor es el principio de la sabiduría}: Sal 111:10; Pr 1:7; 9:10; Job 28:28. \textit{Sé perfecto}: Gn 17:1; 1 Re 8:61; Lv 19:2; Dt 18:13; Mt 5:48; 2 Co 13:11; Stg 1:4; 1 P 1:16. \textit{El orgullo va por delante de la destrucción}: Pr 16:18. \textit{El que gobierna su propio espíritu es más poderoso}: Pr 16:32. \textit{Cuando volváis a vuestro reposo espiritual}: Is 30:15. \textit{Los que esperan en el Señor renovarán sus fuerzas}: Is 40:31. \textit{Apaciguará vuestro temor}: Is 14:3. \textit{No temáis, porque estoy con vosotros}: Is 41:10.}.

\par 
%\textsuperscript{(1445.1)}
\textsuperscript{131:2.7} <<Dios es nuestro Padre; el Señor es nuestro redentor. Dios ha creado las huestes del universo y las preserva a todas. Su rectitud es como las montañas y su juicio como el gran abismo. Nos hace beber en el río de sus placeres, y en su luz veremos la luz. Es bueno dar gracias al Señor y cantar alabanzas al Altísimo, mostrar una benevolencia afectuosa por la mañana y una fidelidad divina cada noche. El reino de Dios es un reino perpetuo, y su dominio perdura a través de todas las generaciones. El Señor es mi pastor; nada me faltará. Me hace descansar en verdes pastos; me lleva junto a aguas tranquilas. Conforta mi alma. Me guía por las sendas de la rectitud. Sí, aunque camine por el valle de la sombra de la muerte, no temeré ningún mal, porque Dios está conmigo. La bondad y la misericordia me seguirán ciertamente todos los días de mi vida, y habitaré para siempre en la casa del Señor>>\footnote{\textit{Dios es nuestro Padre, nuestro redentor}: Is 63:16. \textit{Dios es creador y preservador}: Neh 9:6. \textit{Su rectitud es como las montañas}: Sal 36:6. \textit{Los abrevarás en el río de tus placeres ...}: Sal 36:8-9. \textit{Es bueno dar gracias al Señor}: Sal 92:1-2. \textit{El reino de Dios es un reino perpetuo}: Sal 145:13. \textit{El Señor es mi pastor; nada me faltará ...}: Sal 23:1-4.}.

\par 
%\textsuperscript{(1445.2)}
\textsuperscript{131:2.8} <<Yahvé es el Dios de mi salvación; por eso pondré mi confianza en el nombre divino. Confiaré en el Señor con todo mi corazón; no me apoyaré en mi propio entendimiento. En todos mis caminos lo reconoceré, y él dirigirá mis pasos. El Señor es fiel, mantiene su palabra con los que le sirven; el justo vivirá por su fe. Si no hacéis el bien, es porque el pecado está en la puerta; los hombres recogen el mal que plantan y el pecado que siembran. No os enojéis por culpa de los que hacen el mal. Si veneráis la iniquidad en vuestro corazón, el Señor no os escuchará; si pecáis contra Dios, perjudicaréis también a vuestra propia alma. Dios traerá a juicio la obra de cada hombre con todos sus secretos, buenos o malos. Tal como un hombre piensa en su corazón, así es él>>\footnote{\textit{Pondré mi confianza en el Dios de la salvación}: Is 12:2. \textit{Confiaré en el Señor con todo mi corazón}: Pr 3:5-6. \textit{El Señor es fiel}: Dt 7:9. \textit{El justo vivirá por su fe}: Hab 2:4. \textit{Si no hacéis el bien, es porque el pecado está en la puerta}: Gn 4:7. \textit{Los hombres recogen el pecado que siembran}: Job 4:8. \textit{No os enojéis por culpa de los que hacen el mal}: Sal 37:1. \textit{Si veneráis la iniquidad, el Señor no os escuchará}: Sal 66:18. \textit{Si pecáis contra Dios, perjudicaréis vuestra alma}: Pr 8:36. \textit{Dios traerá a juicio todas las obras secretas}: Ec 12:14. \textit{Tal como un hombre piensa, así es él}: Pr 23:7.}.

\par 
%\textsuperscript{(1445.3)}
\textsuperscript{131:2.9} <<El Señor está cercano a todos los que lo invocan con sinceridad y verdad. El llanto puede durar una noche, pero la alegría vendrá por la mañana. Un corazón alegre hace bien como una medicina. Dios no negará ninguna cosa buena a los que caminan con rectitud. Temed a Dios y guardad sus mandamientos, porque en esto reside todo el deber del hombre. Así se expresa el Señor que creó los cielos y formó la Tierra: `No hay más Dios que yo, un Dios justo y salvador. Desde todos los confines de la Tierra, miradme y sed salvados. Si me buscáis, me encontraréis, con tal que me busquéis de todo corazón'. Los mansos heredarán la Tierra y se regocijarán en la abundancia de la paz. Quien siembra la iniquidad cosechará la calamidad; los que siembran vientos recogerán tempestades>>\footnote{\textit{El Señor está cercano a todos los que lo invocan}: Sal 145:18. \textit{El llanto pasará}: Sal 30:5. \textit{Un corazón alegre hace bien como una medicina}: Pr 17:22. \textit{Dios no negará ninguna cosa buena}: Sal 84:11. \textit{Temed a Dios y guardad sus mandamientos}: Ec 12:13. \textit{Así se expresa el Señor}: Is 45:18. \textit{No hay más Dios que yo, un Dios justo y salvador}: Is 45:21-22. \textit{Si me buscáis, me encontraréis}: Jer 29:13. \textit{Los mansos heredarán la Tierra y se regocijarán en la paz}: Sal 37:11. \textit{Recogerás lo que siembres}: Job 4:8; Pr 22:8; Gl 6:7. \textit{Los que siembran vientos recogerán tempestades}: Os 8:7.}.

\par 
%\textsuperscript{(1445.4)}
\textsuperscript{131:2.10} <<`Venid ahora y razonemos juntos', dice el Señor, `aunque vuestros pecados sean como la escarlata, serán tan blancos como la nieve; aunque sean rojos como el carmesí, se volverán como la lana'. Pero no hay paz para los perversos; son vuestros propios pecados los que han apartado las buenas cosas de vosotros. Dios es la salud de mi semblante y la alegría de mi alma. El Dios eterno es mi fuerza; él es nuestra morada, y por debajo nos sostienen sus brazos eternos. El Señor está cerca de los afligidos, salva a todos los que tienen el espíritu como un niño. Las aflicciones del justo son numerosas, pero el Señor lo libera de todas. Encomendad vuestro camino al Señor ---confiad en él--- y él lo llevará a cabo. El que habita en el lugar secreto del Altísimo morará a la sombra del Todopoderoso>>\footnote{\textit{Venid ahora y razonemos juntos}: Is 1:18. \textit{No hay paz para los perversos}: Is 48:22; 57:21. \textit{Son vuestros pecados los que os han apartado de las cosas buenas}: Jer 5:25. \textit{Dios es la salud de mi semblante}: Sal 43:5. \textit{Dios es la alegría de mi alma}: Sal 35:9; Is 61:10. \textit{Nos sostienen sus brazos eternos}: Dt 33:27. \textit{El Señor está cerca de los afligidos}: Sal 34:18-19. \textit{Confiad en el Señor}: Sal 37:5. \textit{El que habita en el lugar secreto}: Sal 91:1.}.

\par 
%\textsuperscript{(1445.5)}
\textsuperscript{131:2.11} <<Ama a tu prójimo como a ti mismo; no guardes rencor a ningún hombre. No le hagas a nadie lo que tú aborreces. Ama a tu hermano, porque el Señor ha dicho: `Amaré a mis hijos sin restricción'. La senda del justo es como una luz resplandeciente que brilla cada vez más hasta el día perfecto. Los que son sabios brillarán como el resplandor del firmamento, y los que encaminan a muchos hombres hacia la justicia brillarán como las estrellas para siempre jamás. Que el perverso abandone su mal camino y el inicuo sus pensamientos rebeldes. Dice el Señor: `Que vuelvan a mí, y tendré misericordia de ellos; perdonaré en abundancia'>>\footnote{\textit{Ama a tu prójimo como a ti mismo}: Lv 19:18,34; Mt 5:43-44; 19:19b; 22:39; Mc 12:31,33; Lc 10:27; Ro 13:9b; Gl 5:14; Stg 2:8. \textit{La regla de oro negativa}: Tb 4:15. \textit{Los que son sabios brillarán como el firmamento}: Dn 12:3. \textit{Amaré a mis hijos sin restricción}: Os 14:4. \textit{La senda del justo es como una luz}: Pr 4:18. \textit{Que vuelvan a mí, y tendré misericordia de ellos}: Is 55:7.}.

\par 
%\textsuperscript{(1446.1)}
\textsuperscript{131:2.12} <<Dice Dios, el creador del cielo y de la Tierra:
`Los que aman mi ley gozan de una gran paz. Mis mandamientos son: Me amarás con todo tu corazón; no tendrás otros dioses ante mí; no pronunciarás mi nombre en vano; recuerda el día del sábado para santificarlo; honra a tu padre y a tu madre; no matarás; no cometerás adulterio; no robarás; no levantarás falso testimonio; no codiciarás'>>\footnote{\textit{Los que aman mi ley gozan de una gran paz}: Sal 119:165. \textit{Diez mandamientos}: Ex 20:3-17; Dt 5:7-21.}.

\par 
%\textsuperscript{(1446.2)}
\textsuperscript{131:2.13} <<Y a todos los que aman al Señor sobre todas las cosas y a sus prójimos como a sí mismos, el Dios del cielo dice: `Os rescataré de la tumba; os redimiré de la muerte. Seré misericordioso y justo con vuestros hijos. ¿No he dicho de mis criaturas de la Tierra: Sois los hijos del Dios viviente? ¿No os he amado con un amor perpetuo? ¿No os he invitado a que seáis como yo y a que viváis conmigo para siempre en el Paraíso?'>>\footnote{ \textit{Los que aman al Señor sobre todas las cosas}: Dt 6:4-5; 10:12; 11:1,13,22; 13:3; 19:9; 30:6,16,20; Mc 12:30; 22:37; Lc 10:27; Ro 8:28; Jos 22:5; 23:11. \textit{Os rescataré de la tumba}: Os 13:14. \textit{Misericordioso y justo con vuestros hijos}: Sal 103:17. \textit{Hijos del Dios viviente}: Os 1:10. \textit{Amados con un amor perpetuo}: Jer 31:3. \textit{Sed como yo y vivid conmigo}: Gn 17:1; Lv 19:2; Dt 18:13. \textit{Ir al Paraíso}: Sal 23:6.}

\section*{3. El budismo}
\par 
%\textsuperscript{(1446.3)}
\textsuperscript{131:3.1} Ganid se sorprendió al descubrir cuán cerca estaba el budismo de ser una religión grande y hermosa, pero sin Dios, sin una Deidad personal y universal. Sin embargo, encontró algún escrito de ciertas creencias anteriores que reflejaban un poco la influencia de las enseñanzas de los misioneros de Melquisedek, que continuaron su trabajo en la India incluso hasta la época de Buda. Jesús y Ganid reunieron las siguientes declaraciones de la literatura budista:

\par 
%\textsuperscript{(1446.4)}
\textsuperscript{131:3.2} <<La alegría brotará de un corazón puro hacia el Infinito; todo mi ser estará en paz con este regocijo supermortal. Mi alma está llena de satisfacción, y mi corazón desborda con la dicha de una confianza apacible. No tengo ningún temor; estoy libre de ansiedad. Me hallo en seguridad, y mis enemigos no pueden inquietarme. Estoy satisfecho con los frutos de mi confianza. He encontrado que es fácil acceder al Inmortal. Rezo para que la fe me sostenga en el largo viaje; sé que la fe del más allá no me faltará. Sé que mis hermanos prosperarán si llegan a imbuirse de la fe del Inmortal, la fe que crea la modestia, la rectitud, la sabiduría, la valentía, el conocimiento y la perseverancia. Abandonemos la tristeza y renunciemos al temor. Por medio de la fe, atrapemos la verdadera rectitud y la auténtica virilidad. Aprendamos a meditar sobre la justicia y la misericordia. La fe es la verdadera riqueza del hombre; es la dotación de virtud y de gloria>>.

\par 
%\textsuperscript{(1446.5)}
\textsuperscript{131:3.3} <<La injusticia es abyecta y el pecado es despreciable. El mal es degradante tanto de pensamiento como de obra. El dolor y la aflicción siguen al camino del mal como el polvo sigue al viento. La felicidad y la paz mental siguen al pensamiento puro y la vida virtuosa, como la sombra sigue a la sustancia de las cosas materiales. El mal es el fruto de un pensamiento mal dirigido. Es malo ver un pecado donde no lo hay, y no verlo donde sí lo hay. El mal es el sendero de las falsas doctrinas. Los que evitan el mal viendo las cosas tal como son, consiguen la alegría al abrazar así la verdad. Poned fin a vuestra miseria aborreciendo el pecado. Cuando elevéis vuestra mirada hacia el Noble, apartáos del pecado de todo corazón. No disculpéis el mal; no excuséis el pecado. Mediante vuestros esfuerzos por enmendar los pecados pasados, adquirís la fortaleza para resistir a la tendencia de recaer. El dominio de sí nace del arrepentimiento. No dejéis de confesar ninguna falta al Noble>>.

\par 
%\textsuperscript{(1447.1)}
\textsuperscript{131:3.4} <<La jovialidad y la alegría son las recompensas de las acciones bien hechas y son para la gloria del Inmortal. Nadie puede robaros la libertad de vuestra propia mente. Cuando la fe de vuestra religión ha emancipado vuestro corazón, cuando la mente está estabilizada e inmutable como una montaña, entonces la paz del alma fluye tranquilamente como las aguas de un río. Los que están seguros de la salvación, están liberados para siempre de la lujuria, la envidia, el odio y las ilusiones de las riquezas. Aunque la fe sea la energía de una vida mejor, sin embargo tenéis que conseguir con perseverancia vuestra propia salvación. Si queréis estar seguros de vuestra salvación final, aseguraos entonces de que tratáis sinceramente de ejecutar todo lo que es recto. Cultivad la seguridad del corazón, que procede del interior, y venid así a disfrutar del éxtasis de la salvación eterna>>.

\par 
%\textsuperscript{(1447.2)}
\textsuperscript{131:3.5} <<Ninguna persona religiosa puede esperar alcanzar la iluminación de la sabiduría inmortal si persiste en ser perezosa, indolente, débil, holgazana, desvergonzada y egoísta. Pero cualquiera que es cuidadoso, prudente, reflexivo, ferviente y serio ---aunque viva todavía en la Tierra--- puede alcanzar la iluminación suprema de la paz y la libertad de la sabiduría divina. Recordad que toda acción recibirá su recompensa. El mal acaba en aflicción y el pecado termina en dolor. La alegría y la felicidad son el resultado de una vida buena. Incluso el malhechor disfruta de un período de gracia antes de que llegue la completa maduración de sus malas acciones; pero la plena cosecha de la maldad llega inevitablemente. Que nadie piense con ligereza en el pecado, diciéndose en su corazón: `El castigo de las malas acciones no se acercará hasta mí'. Lo que hacéis os será hecho en el juicio de la sabiduría. La injusticia cometida con vuestros semejantes se volverá contra vosotros. La criatura no puede eludir el destino de sus actos>>.

\par 
%\textsuperscript{(1447.3)}
\textsuperscript{131:3.6} <<El insensato se ha dicho en su corazón: `El mal no me alcanzará'; pero sólo se encuentra la seguridad cuando el alma anhela la reprobación y la mente busca la sabiduría. El hombre sabio es un alma noble que sabe ser amistosa en medio de sus enemigos, tranquila entre los turbulentos y generosa entre los avariciosos. El amor de sí mismo es como las malas hierbas en un hermoso campo. El egoísmo conduce a la aflicción; la inquietud perpetua mata. La mente domada produce la felicidad. El guerrero más grande es aquel que se vence y subyuga a sí mismo. La moderación en todas las cosas es buena. Sólo es una persona superior aquella que estima la virtud y cumple con su deber. No dejéis que la cólera y el odio os dominen. No habléis duramente de nadie. El contentamiento es la mayor de las riquezas. Lo que se da con prudencia está bien economizado. No hagáis a los demás las cosas que no quisierais que os hicieran. Devolved bien por mal; venced el mal con el bien>>.

\par 
%\textsuperscript{(1447.4)}
\textsuperscript{131:3.7} <<Un alma justa es más deseable que la soberanía de toda la Tierra. La inmortalidad es la meta de la sinceridad; la muerte es el fin de una vida irreflexiva. Los diligentes no mueren; los irreflexivos ya están muertos. Benditos son aquellos que disciernen el estado inmortal. Los que torturan a los vivos hallarán poca felicidad después de la muerte. Los desinteresados van al cielo, donde gozan de la felicidad de una liberalidad infinita y continúan acrecentando su noble generosidad. Todo mortal que piense con rectitud, que hable noblemente y actúe desinteresadamente, no sólo disfrutará aquí de la virtud durante esta breve vida, sino que después de la disolución del cuerpo continuará disfrutando también de las delicias del cielo>>.

\section*{4. El hinduismo}
\par 
%\textsuperscript{(1447.5)}
\textsuperscript{131:4.1} Los misioneros de Melquisedek llevaron las enseñanzas del Dios único a todos los lugares por donde pasaron. Una gran parte de esta doctrina monoteísta, unida a otros conceptos anteriores, se incorporó en las enseñanzas posteriores del hinduismo. Jesús y Ganid efectuaron los extractos siguientes:

\par 
%\textsuperscript{(1448.1)}
\textsuperscript{131:4.2} <<Él es el gran Dios, supremo en todos los sentidos. Él es el Señor que abarca todas las cosas. Es el Creador y el controlador del universo de universos. Dios es un Dios único; está solo y existe por sí mismo; él es el único. Este Dios único es nuestro Hacedor y el destino último del alma. El Supremo brilla de una manera indescriptible; es la Luz de las Luces. Esta luz divina ilumina todos los corazones y todos los mundos. Dios es nuestro protector ---permanece al lado de sus criaturas--- y los que aprenden a conocerlo se vuelven inmortales. Dios es la gran fuente de la energía; es la Gran Alma. Ejerce una soberanía universal sobre todo. Este Dios único es amoroso, glorioso y adorable. Nuestro Dios tiene un poder supremo y habita en la morada suprema. Esta verdadera Persona es eterna y divina; es el Señor primordial del cielo. Todos los profetas lo han saludado, y él se ha revelado a nosotros. Nosotros lo adoramos. ¡Oh Persona Suprema, fuente de los seres, Señor de la creación y soberano del universo, revélanos a tus criaturas el poder por el que permaneces inmanente! Dios ha hecho el Sol y las estrellas; él es resplandeciente, puro y existe por sí mismo. Su conocimiento eterno es divinamente sabio. El mal no puede penetrar en el Eterno. Puesto que el universo surgió de Dios, él lo gobierna adecuadamente. Él es la causa de la creación, por eso todas las cosas están establecidas en él>>.

\par 
%\textsuperscript{(1448.2)}
\textsuperscript{131:4.3} <<Dios es el refugio seguro de todo hombre de bien que está necesitado; el Inmortal cuida de toda la humanidad. La salvación de Dios es poderosa y su bondad agradable. Es un protector amante y un defensor bendito. Dice el Señor: `Resido dentro de sus propias almas como una lámpara de sabiduría. Soy el esplendor de los espléndidos y la bondad de los buenos. Cuando dos o tres se reúnen, allí estoy yo también'. La criatura no puede eludir la presencia del Creador. El Señor cuenta incluso el parpadeo incesante de los ojos de todos los mortales; y adoramos a este Ser divino como nuestro compañero inseparable. Él es predominante, generoso, omnipresente e infinitamente bondadoso. El Señor es nuestro soberano, nuestro refugio y nuestro controlador supremo, y su espíritu primigenio reside dentro del alma mortal. El Testigo Eterno del vicio y de la virtud habita en el corazón del hombre. Meditemos largamente sobre el Vivificador adorable y divino; que su espíritu dirija plenamente nuestros pensamientos. ¡De este mundo irreal, condúcenos al real! ¡De las tinieblas, llévanos a la luz! ¡De la muerte, guíanos a la inmortalidad!>>

\par 
%\textsuperscript{(1448.3)}
\textsuperscript{131:4.4} <<Con nuestro corazón purificado de todo odio, adoremos al Eterno. Nuestro Dios es el Señor de la oración; escucha el clamor de sus hijos. Que todos los hombres sometan su voluntad al Resuelto. Deleitémonos con la liberalidad del Señor de la oración. Haced de la oración vuestra amiga más íntima, y de la adoración el sostén de vuestra alma. `Si quisierais darme un culto de amor', dice el Eterno, `os daría la sabiduría para alcanzarme, porque mi culto es la virtud común de todas las criaturas'. Dios es la iluminación de los abatidos y la fuerza de los que desfallecen. Puesto que Dios es nuestro amigo poderoso, ya no tenemos miedo. Alabamos el nombre del Conquistador nunca conquistado. Lo adoramos porque es el auxiliador fiel y eterno del hombre. Dios es nuestro director seguro y nuestro guía infalible. Es el gran autor del cielo y de la Tierra, poseedor de una energía ilimitada y de una sabiduría infinita. Su esplendor es sublime y su belleza divina. Es el refugio supremo del universo y el guardián inmutable de la ley perpetua. Nuestro Dios es el Señor de la vida y el Consolador de todos los hombres; ama a la humanidad y ayuda a los afligidos. Es el dador de nuestra vida y el Buen Pastor de los rebaños humanos. Dios es nuestro padre, nuestro hermano y nuestro amigo. Anhelamos conocer a este Dios en lo más profundo de nuestro ser>>.

\par 
%\textsuperscript{(1448.4)}
\textsuperscript{131:4.5} <<Hemos aprendido a conseguir la fe con el deseo ardiente de nuestro corazón. Hemos alcanzado la sabiduría refrenando nuestros sentidos, y por medio de la sabiduría, hemos experimentado la paz en el Supremo. El que está lleno de fe adora verdaderamente cuando su yo interno está absorto en Dios. Nuestro Dios usa los cielos como un manto; habita también en los otros seis universos esparcidos por todas partes. Es supremo sobre todo y en todo. Imploramos el perdón del Señor por todas nuestras ofensas a nuestros semejantes y eximimos a nuestro amigo del mal que nos ha hecho. Nuestro espíritu detesta todo mal; por lo tanto, oh Señor, líbranos de toda mancha de pecado. Oramos a Dios como consolador, protector y salvador ---como alguien que nos ama>>.

\par 
%\textsuperscript{(1449.1)}
\textsuperscript{131:4.6} <<El espíritu del Guardián del Universo entra en el alma de las criaturas simples. El hombre que adora al Dios Único es sabio. Los que se esfuerzan por llegar a la perfección deben conocer ciertamente al Señor Supremo. El que conoce la seguridad bienaventurada del Supremo nunca tiene miedo, porque el Supremo dice a los que le sirven, `No temáis porque estoy con vosotros'. El Dios de la providencia es nuestro Padre. Dios es la verdad. Y es el deseo de Dios que sus criaturas lo comprendan ---que lleguen a conocer plenamente la verdad. La verdad es eterna; sostiene el universo. Nuestro deseo supremo será unirnos con el Supremo. El Gran Controlador es el generador de todas las cosas--- todo evoluciona partiendo de él. Y he aquí la cima del deber: que ningún hombre haga a otro lo que le repugnaría a él mismo; no fomentad ninguna maldad, no castiguéis al que os castiga, conquistad la cólera con la misericordia, y venced el odio con la benevolencia. Deberíamos hacer todo esto porque Dios es un amigo cariñoso y un padre bondadoso que nos perdona todas nuestras ofensas terrenales>>.

\par 
%\textsuperscript{(1449.2)}
\textsuperscript{131:4.7} <<Dios es nuestro Padre, la Tierra es nuestra madre y el universo es el lugar donde hemos nacido. Sin Dios, el alma está prisionera; conocer a Dios libera el alma. La meditación sobre Dios y la unión con él producen la liberación de las ilusiones del mal y la salvación última de todas las trabas materiales. Cuando el hombre enrolle el espacio como un pedazo de cuero, entonces llegará el fin del mal, porque el hombre habrá encontrado a Dios. ¡Oh Dios, sálvanos de la triple ruina del infierno: la lujuria, la ira y la avaricia! ¡Oh alma, cíñete para la lucha espiritual de la inmortalidad! Cuando llegue el fin de la vida mortal, no dudes en abandonar este cuerpo por una forma más apropiada y hermosa, y despertarte en los dominios del Supremo y del Inmortal donde no existe el temor, la aflicción, el hambre, la sed ni la muerte. Conocer a Dios es cortar los lazos de la muerte. El alma que conoce a Dios se eleva en el universo como la crema aparece en la superficie de la leche. Adoramos a Dios, el hacedor de todo, la Gran Alma, que siempre está asentado en el corazón de sus criaturas. Los que saben que Dios está entronizado en el corazón humano, están destinados a volverse como él ---inmortales. El mal debe quedarse atrás en este mundo, pero la virtud acompaña al alma hasta el cielo>>.

\par 
%\textsuperscript{(1449.3)}
\textsuperscript{131:4.8} <<Sólo el perverso dice: El universo no posee ni verdad ni gobernante; sólo fue diseñado para satisfacer nuestra codicia. Estas almas están engañadas por la mezquindad de su intelecto. Por eso se abandonan a la satisfacción de su codicia, y privan a sus almas de las alegrías de la virtud y de los placeres de la rectitud. ¿Qué puede ser más grande que experimentar la salvación del pecado? El hombre que ha visto al Supremo es inmortal. Los amigos carnales del hombre no pueden sobrevivir a la muerte; sólo la virtud camina junto al hombre mientras viaja siempre adelante hacia los campos alegres y soleados del Paraíso>>.

\section*{5. El zoroastrismo}
\par 
%\textsuperscript{(1449.4)}
\textsuperscript{131:5.1} Zoroastro estuvo personalmente en contacto directo con los descendientes de los primeros misioneros de Melquisedek, y la doctrina del Dios único se convirtió en la enseñanza central de la religión que fundó en Persia. Aparte del judaísmo, ninguna religión de esta época contenía mayor cantidad de estas enseñanzas de Salem. Ganid sacó los extractos siguientes de los archivos de esta religión:

\par 
%\textsuperscript{(1450.1)}
\textsuperscript{131:5.2} <<Todas la cosas proceden del Dios Único y le pertenecen ---él es infinitamente sabio, bueno, justo, santo, resplandeciente y glorioso. Éste, nuestro Dios, es la fuente de toda luminosidad. Es el Creador, el Dios de todas las buenas intenciones y el protector de la justicia del universo. La conducta sabia en la vida consiste en actuar en armonía con el espíritu de la verdad. Dios lo ve todo y contempla tanto las malas acciones del perverso como las buenas obras del justo; nuestro Dios observa todas las cosas con una mirada destellante. Su toque es el toque de la curación. El Señor es un benefactor todopoderoso. Dios tiende su mano benéfica tanto al justo como al perverso. Dios estableció el mundo y ordenó las recompensas para el bien y para el mal. El Dios infinitamente sabio ha prometido la inmortalidad a las almas piadosas que piensan con pureza y actúan con rectitud. Llegaréis a ser aquello que deseáis de manera suprema. La luz del Sol es como la sabiduría para aquellos que disciernen a Dios en el universo>>.

\par 
%\textsuperscript{(1450.2)}
\textsuperscript{131:5.3} <<Alabad a Dios buscando lo que complace al Sabio. Adorad al Dios de la luz caminando alegremente en las sendas ordenadas por su religión revelada. No hay más que un Dios Supremo, el Señor de las Luces. Adoramos a aquel que hizo las aguas, las plantas, los animales, la Tierra y los cielos. Nuestro Dios es el Señor, el más benévolo. Adoramos al más hermoso, al Inmortal generoso dotado de la luz eterna. Dios está muy lejos de nosotros y al mismo tiempo muy cerca, porque reside en nuestras almas. Nuestro Dios es el divino y santísimo Espíritu del Paraíso, y sin embargo es más amistoso para el hombre que la más amistosa de todas las criaturas. Dios es de una gran ayuda para nosotros en la más grande de todas las ocupaciones, la de conocerlo a él mismo. Dios es nuestro amigo más adorable y justo; es nuestra sabiduría, nuestra vida y el vigor de nuestra alma y de nuestro cuerpo. Gracias a nuestros buenos pensamientos, el sabio Creador nos permitirá hacer su voluntad, consiguiendo así la realización de todo lo que es divinamente perfecto>>.

\par 
%\textsuperscript{(1450.3)}
\textsuperscript{131:5.4} <<Señor, enséñanos a vivir esta vida en la carne mientras nos preparamos para la próxima vida del espíritu. Háblanos, Señor, y haremos lo que nos ordenes. Enséñanos las buenas sendas, y caminaremos rectos. Concédenos el que podamos alcanzar la unión contigo. Sabemos que la religión es buena cuando conduce a la unión con la rectitud. Dios es nuestra naturaleza sabia, nuestro mejor pensamiento y nuestra acción justa. ¡Que Dios nos conceda la unidad con el espíritu divino y la inmortalidad en él mismo!>>

\par 
%\textsuperscript{(1450.4)}
\textsuperscript{131:5.5} <<Esta religión del Sabio purifica al creyente de todo mal pensamiento y de todo acto pecaminoso. Me inclino ante el Dios del cielo arrepintiéndome si he ofendido de pensamiento, palabra u obra ---intencionalmente o no--- y ofrezco oraciones por la misericordia y alabanzas por el perdón. Cuando me confieso, si me propongo no volver a hacer el mal, sé que el pecado será apartado de mi alma. Sé que el perdón disuelve las cadenas del pecado. Los que hacen el mal serán castigados, pero los que siguen la verdad gozarán de la felicidad de una salvación eterna. Cógenos mediante la gracia y dispensa un poder salvador a nuestra alma. Pedimos misericordia porque aspiramos a alcanzar la perfección; quisiéramos ser semejantes a Dios>>.

\section*{6. El suduanismo (el jainismo)}
\par 
%\textsuperscript{(1450.5)}
\textsuperscript{131:6.1} El tercer grupo de creyentes religiosos que preservó la doctrina de un Dios único en la India ---la supervivencia de las enseñanzas de Melquisedek--- era conocido en aquella época como los suduanistas. Estos creyentes se conocen más recientemente como los seguidores del jainismo. He aquí lo que enseñaban:

\par 
%\textsuperscript{(1450.6)}
\textsuperscript{131:6.2} <<El Señor del Cielo es supremo. Los que cometen pecado no ascenderán a las alturas, pero los que caminan por la senda de la rectitud encontrarán un lugar en el cielo. Estamos seguros de la vida en el estado futuro si conocemos la verdad. El alma del hombre puede ascender hasta el cielo más alto para desarrollar allí su verdadera naturaleza espiritual, para alcanzar la perfección. El estado celestial libera al hombre de la esclavitud del pecado y lo introduce en las bienaventuranzas finales; el hombre recto ya tiene la experiencia de haber terminado con el pecado y con todas las miserias que lo acompañan. El ego es el enemigo invencible del hombre, y se manifiesta en las cuatro pasiones más grandes del hombre: la ira, el orgullo, el engaño y la codicia. La victoria más grande del hombre es la conquista de sí mismo. Cuando el hombre se vuelve hacia Dios para ser perdonado, cuando tiene la audacia de disfrutar de esa libertad, eso mismo lo libera del temor. El hombre debería atravesar la vida tratando a sus semejantes como a él le gustaría ser tratado>>.

\section*{7. El sintoísmo}
\par 
%\textsuperscript{(1451.1)}
\textsuperscript{131:7.1} Hacía poco tiempo que los manuscritos de esta religión del Lejano Oriente se habían colocado en la biblioteca de Alejandría. Era la única religión del mundo de la que Ganid nunca había oído hablar. Esta creencia también contenía restos de las primeras enseñanzas de Melquisedek, tal como lo demuestran los extractos siguientes:

\par 
%\textsuperscript{(1451.2)}
\textsuperscript{131:7.2} <<Dice el Señor: `Todos sois receptores de mi divino poder; todos los hombres se benefician de mi ministerio de misericordia. Me complace mucho la multiplicación de los justos por todas las naciones. Tanto en las bellezas de la naturaleza como en la virtud de los hombres, el Príncipe del Cielo intenta revelarse y mostrar la rectitud de su naturaleza. Puesto que los pueblos antiguos no conocían mi nombre, me manifesté naciendo en el mundo como un ser visible, y soporté esa humillación para que ni siquiera los hombres olviden mi nombre. Soy el hacedor del cielo y de la Tierra; el Sol, la Luna y todas las estrellas obedecen a mi voluntad. Soy el soberano de todas las criaturas en la Tierra y en los cuatro mares. Aunque soy grande y supremo, sin embargo tengo consideración por la oración del más humilde de los hombres. Si una criatura quiere adorarme, escucharé su oración y le concederé el deseo de su corazón'>>.

\par 
%\textsuperscript{(1451.3)}
\textsuperscript{131:7.3} <<`Cada vez que el hombre cede a la ansiedad, se aleja un paso de la guía del espíritu de su corazón'. El orgullo oculta a Dios. Si queréis obtener la ayuda del cielo, apartad vuestro orgullo; cualquier indicio de orgullo intercepta la luz salvadora como si fuera una gran nube. Si no sois rectos por dentro, es inútil que oréis por las cosas de fuera. `Si escucho vuestras oraciones es porque os presentáis ante mí con un corazón puro, libre de falsedad y de hipocresía, con un alma que refleja la verdad como un espejo. Si queréis obtener la inmortalidad, renunciad al mundo y venid a mí'>>.

\section*{8. El taoísmo}
\par 
%\textsuperscript{(1451.4)}
\textsuperscript{131:8.1} Los mensajeros de Melquisedek penetraron muy dentro de China, y la doctrina del Dios único formó parte de las primeras enseñanzas de diversas religiones chinas; el taoísmo fue la que perduró más tiempo y contuvo la mayor cantidad de verdad monoteísta. Entre las enseñanzas de su fundador, Ganid reunió las siguientes:

\par 
%\textsuperscript{(1451.5)}
\textsuperscript{131:8.2} <<¡Cuán puro y sereno es el Supremo, y sin embargo cuán poderoso y fuerte, cuán profundo e insondable! Este Dios del cielo es el antecesor venerado de todas las cosas. Si conocéis al Eterno, estáis iluminados y sois sabios. Si no conocéis al Eterno, esa ignorancia se manifiesta entonces como mal, y así surgen las pasiones del pecado. Este Ser maravilloso existía antes que los cielos y la Tierra. Él es verdaderamente espiritual; está solo y no cambia. Él es en realidad la madre del mundo, y toda la creación gira a su alrededor. Este Gran Único se da a los hombres, permitiéndoles así superarse y sobrevivir. Aunque uno tenga pocos conocimientos, siempre puede caminar por las vías del Supremo; puede someterse a la voluntad del cielo>>.

\par 
%\textsuperscript{(1452.1)}
\textsuperscript{131:8.3} <<Todas las buenas obras de servicio sincero proceden del Supremo. Todas las cosas dependen de la Gran Fuente para vivir. El Gran Supremo no busca honores por sus dones. Aunque es supremo en poder, permanece oculto a nuestra mirada. Transmuta incesantemente sus atributos mientras perfecciona a sus criaturas. La Razón celestial es lenta y paciente en sus proyectos, pero está segura de sus realizaciones. El Supremo extiende el universo y lo sostiene por completo. ¡Cuán grandes y poderosos son su influencia desbordante y su poder de atracción! La verdadera bondad es como el agua, que todo lo bendice y no daña nada. Y al igual que el agua, la verdadera bondad busca los lugares inferiores, incluso aquellos niveles que evitan los demás, y lo hace así porque está emparentada con el Supremo. El Supremo crea todas las cosas, las alimenta en la naturaleza y las perfecciona en espíritu. Y es un misterio cómo el Supremo consigue nutrir, proteger y perfeccionar a las criaturas sin obligarlas. Guía y dirige, pero sin imponerse. Favorece el progreso, pero sin oprimir>>.

\par 
%\textsuperscript{(1452.2)}
\textsuperscript{131:8.4} <<El hombre sabio hace universal su corazón. Un poco de conocimiento es algo peligroso. Los que aspiran a la grandeza tienen que aprender a humillarse. En la creación, el Supremo se convirtió en la madre del mundo. Conocer a la madre de uno es reconocer su filiación. Es sabio el hombre que considera todas las partes desde el punto de vista de la totalidad. Relacionaos con cada hombre como si estuvierais en su lugar. Responded a las ofensas con la bondad. Si amáis a la gente, se sentirán atraídos hacia vosotros ---no tendréis ninguna dificultad para persuadirlos>>.

\par 
%\textsuperscript{(1452.3)}
\textsuperscript{131:8.5} <<El Gran Supremo lo penetra todo; está a la derecha y a la izquierda; sostiene toda la creación y habita en todos los seres sinceros. No podéis encontrar al Supremo, ni ir a un lugar donde no se encuentre. Si un hombre reconoce la maldad de sus acciones y se arrepiente de corazón de sus pecados, entonces puede buscar el perdón, librarse del castigo y transformar la calamidad en bendición. El Supremo es el refugio seguro para toda la creación; es el guardián y el salvador de la humanidad. Si lo buscáis diariamente, lo encontraréis. Puesto que puede perdonar los pecados, es en verdad el más apreciado por todos los hombres. Recordad siempre que Dios no recompensa al hombre por lo que hace, sino por lo que es; por ello, conceded vuestra ayuda a vuestros semejantes sin pensar en la recompensa. Haced el bien sin pensar en un beneficio egoísta>>.

\par 
%\textsuperscript{(1452.4)}
\textsuperscript{131:8.6} <<Los que conocen las leyes del Eterno son sabios. La ignorancia de la ley divina es una calamidad y un desastre. Los que conocen las leyes de Dios tienen una mentalidad liberal. Si conocéis al Eterno, aunque vuestro cuerpo perezca, vuestra alma sobrevivirá para el servicio del espíritu. Sois realmente sabios cuando reconocéis vuestra insignificancia. Si permanecéis en la luz del Eterno, gozaréis de la iluminación del Supremo. Los que dedican su persona al servicio del Supremo son felices en esta búsqueda del Eterno. Cuando el hombre muere, el espíritu empieza a desplegar su largo vuelo en el gran viaje de regreso al hogar>>.

\section*{9. El confucianismo}
\par 
%\textsuperscript{(1452.5)}
\textsuperscript{131:9.1} Entre las grandes religiones del mundo, incluso la que menos reconocía a Dios aceptó el monoteísmo de los misioneros de Melquisedek y de sus perseverantes sucesores. He aquí el resumen de Ganid sobre el confucianismo:

\par 
%\textsuperscript{(1452.6)}
\textsuperscript{131:9.2} <<Lo que el Cielo decreta está exento de error. La verdad es real y divina. Todas las cosas se originan en el Cielo, y el Gran Cielo no comete errores. El Cielo ha designado a numerosos subordinados para que ayuden a instruir y a elevar a las criaturas inferiores. Grande, muy grande es el Dios Único que dirige al hombre desde lo alto. Dios es majestuoso en su poder y terrible en su juicio. Pero este Gran Dios ha conferido un sentido moral incluso a muchos hombres inferiores. La generosidad del Cielo no se detiene jamás. La benevolencia es el don más precioso del Cielo a los hombres. El Cielo ha otorgado su nobleza al alma del hombre; las virtudes del hombre son el fruto de este don de la nobleza del Cielo. El Gran Cielo lo discierne todo y acompaña al hombre en todas sus acciones. Hacemos bien en llamar al Gran Cielo nuestro Padre y nuestra Madre. Si somos pues los servidores de nuestros antepasados divinos, entonces podemos rezar al Cielo con confianza. En todo momento y en todas las cosas, tengamos el temor reverencial de la majestad del Cielo. Reconocemos, oh Dios, Altísimo y soberano Potentado, que el juicio te pertenece, y que toda misericordia procede del corazón divino>>.

\par 
%\textsuperscript{(1453.1)}
\textsuperscript{131:9.3} <<Dios está con nosotros; por eso no sentimos ningún miedo en nuestro corazón. Si se encuentra alguna virtud en mí, se trata de la manifestación del Cielo que habita conmigo. Pero este Cielo dentro de mí efectúa a menudo unas demandas muy duras para mi fe. Si Dios está conmigo, he decidido no tener ninguna duda en mi corazón. La fe debe estar muy cerca de la verdad de las cosas, y no veo cómo un hombre puede vivir sin esta fe saludable. El bien y el mal no sobrevienen sin causa a los hombres. El Cielo trata al alma del hombre en consonancia con la intención de dicha alma. Cuando estéis equivocados, no dudéis en confesar vuestro error y apresuraos a enmendarlo>>.

\par 
%\textsuperscript{(1453.2)}
\textsuperscript{131:9.4} <<El sabio se ocupa de buscar la verdad, no simplemente de ganarse la vida. Alcanzar la perfección del Cielo es la meta del hombre. El hombre superior trata de adaptarse, y está libre de ansiedad y de temor. Dios está con vosotros, no lo dudéis en vuestro corazón. Toda buena acción tiene su recompensa. El hombre superior no murmura contra el Cielo ni guarda rencor a los hombres. Lo que no os gusta que os hagan, no lo hagáis a los demás. Que la compasión forme parte de todo castigo; de todas las maneras posibles procurad transformar el castigo en una bendición. Esta es la manera de obrar del Gran Cielo. Aunque todas las criaturas tienen que morir y regresar a la tierra, el espíritu del hombre noble se va para desplegarse en las alturas y ascender a la luz gloriosa del resplandor final>>.

\section*{10. <<Nuestra religión>>}
\par 
%\textsuperscript{(1453.3)}
\textsuperscript{131:10.1} Después del arduo trabajo de realizar esta compilación de las enseñanzas de las religiones del mundo relativas al Padre Paradisiaco, Ganid se puso a preparar lo que pensaba que era un resumen de la creencia a la que había llegado, con relación a Dios, como resultado de las enseñanzas de Jesús. Este joven había cogido la costumbre de denominar estas creencias <<nuestra religión>>, y he aquí lo que escribió:

\par 
%\textsuperscript{(1453.4)}
\textsuperscript{131:10.2} <<El Señor nuestro Dios es un Señor único, y deberíais amarlo con toda vuestra mente y con todo vuestro corazón, mientras que hacéis todo lo posible por amar a todos sus hijos como os amáis a vosotros mismos. Este Dios único es nuestro Padre celestial, en quien radican todas las cosas y que habita, por medio de su espíritu, en toda alma humana sincera. Nosotros, que somos los hijos de Dios, deberíamos aprender a confiarle la custodia de nuestra alma como a un fiel Creador. Con nuestro Padre celestial, todas las cosas son posibles. No podía ser de otra manera, puesto que él es el Creador que ha hecho todas las cosas y todos los seres. Aunque no podemos ver a Dios, podemos conocerlo. Viviendo diariamente la voluntad del Padre que está en los cielos, podemos revelarlo a nuestros semejantes>>\footnote{\textit{Un único Dios}: 2 Re 19:19; 1 Cr 17:20; Neh 9:6; Sal 86:10; Eclo 36:5; Is 37:16; 44:6,8; 45:5-6,21; Dt 4:35,39; 6:4; Mc 12:29,32; Jn 17:3; Ro 3:30; 1 Co 8:4-6; Gl 3:20; Ef 4:6; 1 Ti 2:5; Stg 2:19; 1 Sam 2:2; 2 Sam 7:22. \textit{Deberíais amarlo con toda la mente y corazón}: Dt 6:4-5; 10:12; 11:1,13,22; 13:3; 19:9; 30:6.16.20; Mt 22:37; Mc 12:30; Lc 10:27; Ro 8:28; Jos 22:5; 23:11. \textit{Amar al prójimo como a uno mismo}: Lv 19:18,34; Mt 5:43-44; 19:19b; 22:39; Mc 12:31,33; Lc 10:27; Ro 13:9b; Gl 5:14; Stg 2:8. \textit{Dios es nuestro Padre celestial}: Mt 5:16,16,45,48; 6:1,9,14; 6:26,32; 7:11,21; 10:32-33; 11:25; 12:50; 15:13; 16:17; 18:10,14,19,35; 23:9; Mc 11:25-26; Lc 10:21; 11:2,13. \textit{Todas las cosas radican en él}: Col 1:17. \textit{Somos los hijos de Dios}: 1 Cr 22:10; Sal 2:7; Is 56:5; Mt 5:9,16,45; Lc 20:36; Jn 1:12-13; 11:52; Hch 17:28-29; Ro 8:14-17,19,21; 9:26; 2 Co 6:18; Gl 3:26; 4:5-7; Ef 1:5; Flp 2:15; Heb 12:5-8; 1 Jn 3:1-2,10; 5:2; Ap 21:7; 2 Sam 7:14. \textit{El espíritu de Dios habita en nuestra alma}: Job 32:8,18; Is 63:10-11; Ez 37:14; Mt 10:20; Lc 17:21; Jn 17:21-23; Ro 8:9-11; 1 Co 3:16-17; 6:19; 2 Co 6:16; Gl 2:20; 1 Jn 3:24; 4:12-15; Ap 21:3. \textit{Deberíamos aprender a confiarle nuestra alma}: 1 P 4:19. \textit{Con Dios, todas las cosas son posibles}: Gn 18:14; Jer 32:27; Mt 19:26; Mc 10:27; 14:36; Lc 1:37; 18:27. \textit{Es el Creador de todas las cosas}: Gn 1:1; 2:4; 5:1-2; Ex 20:11; 31:17; 2 Re 19:15; 2 Cr 2:12; Neh 9:6; Sal 115:15-16; 121:2; 124:8; 134:3; 146:6; Eclo 1:1-4; 33:10; Is 37:16; 40:26,28; 42:5; 45:12,18; Jer 10:11-12; 32:17; 51:15; Bar 3:34-36; Am 4:13; Mal 2:10; Mc 13:19; Jn 1:1-3; Hch 4:24; 14:15; Ef 3:9; Col 1:16; Heb 1:2; 1 P 4:19; Ap 4:11; 10:6; 14::7. \textit{Podemos conocer a Dios}: Sal 46:10; Jn 14:7. \textit{Viviendo su voluntad podemos revelarlo}: Sal 143:10; Eclo 15:11-20; Mt 6:10; 7:21; 12:50; 26:39,42,44; Mc 3:35; 14:36,39; Lc 8:21; 11:2; 22:42; Jn 4:34; 5:30; 6:38-40; 7:16-17; 9:31; 14:21-24; 15:10,14-16; 17:4.}.

\par 
%\textsuperscript{(1453.5)}
\textsuperscript{131:10.3} <<Las riquezas divinas del carácter de Dios deben ser infinitamente profundas y eternamente sabias. No podemos encontrar a Dios por medio del conocimiento, pero podemos conocerlo en nuestro corazón por experiencia personal. Aunque su justicia puede sobrepasar nuestra capacidad de averiguación, su misericordia puede recibirla el ser más humilde de la Tierra. Aunque el Padre llena el universo, vive también en nuestro corazón. La mente del hombre es humana, mortal, pero el espíritu del hombre es divino, inmortal. Dios no es solamente todopoderoso sino también infinitamente sabio. Si nuestros padres terrenales, que tienen tendencia al mal, saben amar a sus hijos y concederles buenas cosas, cuánto más debe saber el buen Padre celestial amar sabiamente a sus hijos terrenales y otorgarles las bendiciones que les convienen>>\footnote{\textit{Las riquezas divinas del carácter de Dios}: Sal 92:5-6. \textit{No podemos encontrar a Dios por el conocimiento}: Job 9:10; Is 64:4; Ro 11:33-34; 1 Co 2:9. \textit{Podemos conocer a Dios en nuestro corazón}: 1 Co 2:10-16. \textit{Dios vive en nuestro corazón}: Job 32:8,18; Is 63:10-11; Ez 37:14; Mt 10:20; Lc 17:21; Jn 17:21-23; Ro 8:9-11; 1 Co 3:16-17; 6:19; 2 Co 6:16; Gl 2:20; 1 Jn 3:24; 4:12-15; Ap 21:3. \textit{La mente del hombre es mortal, el espíritu inmortal}: Job 32:8; 1 Co 2:10-16. \textit{Dios es todopoderoso}: Ex 9:16; 15:6; 1 Cr 29:11-12; Neh 1:10; Job 36:22; 37:23; Sal 59:16; 106:8; 111:6; 147:5; Jer 10:12; 27:5; 32:17; 51:15; Nm 14:17; Nah 1:3; Dt 9:29; Mt 28:18; 2 Sam 22:33. \textit{Dios es infinitamente sabio}: Jer 51:15; 1 Co 2:1-16. \textit{Si los padres terrenales aman, cuánto más Dios}: Mt 7:11; Lc 11:13.}.

\par 
%\textsuperscript{(1454.1)}
\textsuperscript{131:10.4} <<El Padre celestial no permitirá que perezca un solo hijo de la Tierra, si ese hijo tiene el deseo de encontrarle y anhela verdaderamente parecerse a él. Nuestro Padre ama incluso a los perversos y es siempre bondadoso con los ingratos. Si más seres humanos pudieran tan sólo enterarse de la bondad de Dios, se sentirían ciertamente motivados a arrepentirse de su mala conducta y a renunciar a todos los pecados conocidos. Todas las cosas buenas provienen del Padre de la luz, en quien no existe variabilidad ni sombra de cambio. El espíritu del Dios verdadero está en el corazón del hombre. Dios tiene la intención de que todos los hombres sean hermanos. Cuando los hombres empiezan a sentir el anhelo de Dios, esta es la prueba de que Dios los ha encontrado, y de que están a la búsqueda de conocimientos acerca de él. Vivimos en Dios y Dios habita en nosotros>>\footnote{\textit{No perecerá ninguno que desee encontrar a Dios}: Mt 18:14. \textit{Dios ama incluso a los perversos}: Ez 18:21-23,27; 33:11; Mt 18:11; Lc 19:10; Jn 3:16; 15:9-13; 17:22-23; Ro 5:8; Tit 3:4; 1 Jn 4:9-11,19. \textit{Si más pudieran enterarse de la bondad de Dios}: Ex 18:9; Zac 9:17; Ro 2:4. \textit{Todo proviene del Padre de la luz, que es invariable}: Stg 1:17. \textit{El espíritu de Dios está en el corazón del hombre}: Job 32:8,18; Is 63:10-11; Ez 37:14; Mt 10:20; Lc 17:21; Jn 17:21-23; Ro 8:9-11; 1 Co 3:16-17; 6:19; 2 Co 6:16; Gl 2:20; 1 Jn 3:24; 4:12-15; Ap 21:3. \textit{Todos los hombres son hermanos}: Mc 3:35; 1 Ts 4:9; 1 P 1:22. \textit{Vivimos en Dios}: Jn 15:4-7; 14:20.}.

\par 
%\textsuperscript{(1454.2)}
\textsuperscript{131:10.5} <<Ya no me basta con creer que Dios es el Padre de todo mi pueblo; en adelante creeré que es también \textit{mi} Padre. Siempre trataré de adorar a Dios con la ayuda del Espíritu de la Verdad, que será mi auxiliador cuando haya llegado realmente a conocer a Dios. Pero ante todo voy a practicar el culto de Dios aprendiendo a hacer su voluntad en la Tierra, es decir, que voy a hacer todo lo posible por tratar a cada uno de mis compañeros mortales tal como yo pienso que a Dios le gustaría que lo tratara. Cuando vivimos de esta manera en la carne, podemos pedir muchas cosas a Dios, y él nos concederá el deseo de nuestro corazón para que estemos bien preparados para servir a nuestros semejantes. Todo este servicio afectuoso con los hijos de Dios aumenta nuestra capacidad para recibir y experimentar las alegrías del cielo, los placeres superiores del ministerio del espíritu del cielo>>\footnote{\textit{Dios es ``mi'' padre}: 1 Cr 22:10; Sal 2:7; 89:26-27; Jer 3:19; Mt 5:9,16,45,48; 6:1,9,14; 6:26,32; 7:11; 10:32-33; 18:14; 23:9; Mc 11:25-26; Lc 6:36; 11:2,13; Jn 20:17b; Ro 1:7; 8:14 -15; 1 Co 1:3; 2 Co 1:2; 6:18; Gl 1:4; 4:6-7; Ef 1:2; Flp 1:2:; Col 1:2; 1 Ts 1:1,3; 2 Ts 1:1-2; 1 Ti 1:2; Flm 1:3; 1 Jn 3:1-2,10; 2 Sam 7:14. \textit{Adorar a Dios en espíritu y en verdad}: Jn 4:23-24. \textit{Haz a otros como Dios haría}: Mt 5:38-45; Lc 6:27-31; Tb 4:15. \textit{Pide y se te dará}: Mt 7:7; Mc 11:24; Lc 11:9; Jn 14:13-14; 15:7. \textit{Las alegrías del cielo}: Is 62:1-3; Jn 15:10-11.}.

\par 
%\textsuperscript{(1454.3)}
\textsuperscript{131:10.6} <<Todos los días daré gracias a Dios por sus dones inefables; lo alabaré por sus obras maravillosas para los hijos de los hombres. Para mí, es el Todopoderoso, el Creador, el Poder y la Misericordia, pero por encima de todo es mi Padre espiritual, y como su hijo terrenal, alguna vez llegaré a verlo. Mi preceptor me ha dicho que a medida que lo busque me volveré como él. Gracias a la fe en Dios, he alcanzado la paz con él. Esta nueva religión nuestra está llena de alegría y produce una felicidad duradera. Estoy seguro de que seré fiel hasta la muerte, y de que recibiré sin duda la corona de la vida eterna>>\footnote{\textit{Dar gracias a Dios por sus dones}: 2 Co 9:15. \textit{Alabar a Dios por sus obras}: Sal 92:1-2; 107:8,15; 107:21,31. \textit{El Todopoderoso}: Gn 17:1; Gn 28:3; Ex 6:3. \textit{El Creador}: Gn 1:1-27; 2:4-23; 5:1-2; Ex 20:11; 31:17; 2 Re 19:15; 2 Cr 2:12; Neh 9:6; Sal 115:15-16; 121:2; 124:8; 146:6; Eclo 1:1:4; 33:10; Is 37:16; 40:26,28; 42:5; 45::12,18; Jer 10:11-12; 32:17; 51:15; Bar 3:32-36; Am 4:13; Mal 2:11; Mc 13:19; Jn 1:1-3; Hch 4:24; 14:15; Ef 3:9; Col 1:16; Heb 1:2; 1 P 4:19; Ap 4:11; 10:6; 14:7. \textit{El Poder}: Ex 9:16; 15:6; 1 Cr 29:11-12; Neh 1:10; Job 36:22; 37:23; Sal 59:16; 106:8; 111:6; 147:5; Jer 10:12; 27:5; 32:17; 51:15; Nm 14:17; Nah 1:3; Dt 9:29; Mt 28:18; 2 Sam 22:33. \textit{La Misericordia}: Ex 20:6; 1 Cr 16:34; 2 Cr 5:13; 7:3,6; 30:9; Esd 3:11; Sal 25:6; 36:5; 86:5,13,15; 100:5; 103:8,17; 107:1; 116:5; 117:2; 118:1,4; 136:1-26; 145:8; Is 54:8; 55:7; Jer 3:12; Nm 14:18-19; Miq 7:18; Dt 4:31; 5:10; Heb 8:12. \textit{Relación entre padre e hijo}: 1 Cr 22:10; Sal 2:7; 89:26-27; Is 56:5; Jer 3:19; Mt 5:9,16,45; 6:1,9,14; 6:26,32; 7:11; 18:14; 23:9; Mc 11:25-26; Lc 6:36; 11:2,13; 20:36; Jn 1:12-13; 11:52; 20:17b; Hch 17:28-29; Ro 1:7; 8:14-17,19,21; 9:26; 1 Co 1:3; 2 Co 1:2; 6:18; Gl 3:26; 4:5-7; Ef 1:2,5; Flp 1:2; 2:15; Col 1:2; 1 Ts 1:1,3; 2 Ts 1:1-2; 1 Ti 1:2; Flm 1:3; Heb 12:5-8; 1 Jn 3:1-2,10; 5:2; Ap 21:7; 2 Sam 7:14. \textit{Nos volveremos como Dios}: 1 Jn 3:2. \textit{Por la fe alcanzaré la paz}: Is 32:17-18; Ro 5:1. \textit{Llena de alegría y felicidad}: Ro 15:13; 1 P 1:8. \textit{Corona de la vida eterna}: Stg 1:12; 1 P 5:4; Ap 2:10.}.

\par 
%\textsuperscript{(1454.4)}
\textsuperscript{131:10.7} <<Estoy aprendiendo a examinar todas las cosas y a adherirme a lo que es bueno. Haré a mis semejantes todo lo que yo quisiera que hicieran por mí. Por medio de esta nueva fe, sé que el hombre puede volverse el hijo de Dios, pero a veces me aterra ponerme a pensar que todos los hombres son mis hermanos, aunque debe ser verdad. No veo cómo podría regocijarme con la paternidad de Dios, si rehúso aceptar la fraternidad de los hombres. El que invoque el nombre del Señor será salvado. Si esto es verdad, entonces todos los hombres deben ser mis hermanos>>\footnote{\textit{Examinar todas las cosas y elegir lo bueno}: 1 Ts 5:21. \textit{La regla de oro}: Mt 7:12; Lc 6:31. \textit{Regla de oro negativa}: Tb 4:15. \textit{Hermandad espiritual}: Mt 12:50; Mc 3:35; Lc 8:21; Heb 2:11. \textit{Quien lo desee será salvado}: Sal 50:15; Jl 2:32; Zac 13:9; Mt 7:24; 10:32-33; 12:50; 16:24-25; Mc 3:35; 8:34-35; Lc 6:47; 9:23-24; 12:8; Jn 3:15-16; 4:13-14; 11:25-26; 12:46; Hch 2:21; 10:43; 13:26; Ro 9:33; 10:13; 1 Jn 2:23; 4:15; 5:1; Ap 22:17b.}.

\par 
%\textsuperscript{(1454.5)}
\textsuperscript{131:10.8} <<A partir de ahora haré mis buenas obras en secreto, y efectuaré mis oraciones principalmente cuando me encuentre solo. No juzgaré, para evitar ser injusto con mis semejantes. Voy a aprender a amar a mis enemigos; en verdad, aún no he dominado esta técnica de ser semejante a Dios. Aunque veo a Dios en las otras religiones, en `nuestra religión' lo encuentro más bello, más afectuoso, más misericordioso, más personal y más positivo. Pero por encima de todo, este Ser grande y glorioso es mi Padre espiritual, y yo soy su hijo. Únicamente por medio de mi deseo sincero de ser como él, terminaré por encontrarlo y por servirle eternamente. Por fin tengo una religión con un Dios, un Dios maravilloso, y es un Dios de salvación eterna>>\footnote{\textit{Haz el bien en secreto}: Mt 6:1-4. \textit{Ora cuando estés solo}: Mt 6:5-6. \textit{No juzgues}: Mt 7:1-2; Lc 6:37. \textit{Ama a tus enemigos}: Pr 24:17; Pr 25:21; Mt 5:44; Lc 6:27,35. \textit{Dios de la salvación eterna}: Heb 5:9; 1 Jn 5:20.}.


\chapter{Documento 132. La estancia en Roma}
\par 
%\textsuperscript{(1455.1)}
\textsuperscript{132:0.1} PUESTO que Gonod traía los saludos de los príncipes de la India para Tiberio, el soberano romano, los dos indios y Jesús se presentaron ante él al tercer día de llegar a Roma. El taciturno emperador estaba excepcionalmente alegre aquel día y charló largo rato con los tres. Cuando se retiraron de su presencia, el emperador, refiriéndose a Jesús, comentó al ayudante que estaba a su derecha: <<Si yo tuviera el porte real y los modales agradables de ese individuo, sería un verdadero emperador, ¿verdad?>>.

\par 
%\textsuperscript{(1455.2)}
\textsuperscript{132:0.2} Mientras estaba en Roma, Ganid tenía unas horas regulares para estudiar y para visitar los lugares de interés de la ciudad. Su padre tenía que tratar muchos negocios, y como deseaba que su hijo creciera para que fuera su digno sucesor en la dirección de sus vastos intereses comerciales, pensó que había llegado el momento de introducir al muchacho en el mundo de los negocios. En Roma había muchos ciudadanos de la India, y a menudo uno de los propios empleados de Gonod lo acompañaba como intérprete, de manera que Jesús disponía de días enteros para él; esto le proporcionó tiempo para conocer completamente esta ciudad de dos millones de habitantes. Se le encontraba con frecuencia en el foro, el centro de la vida política, jurídica y comercial. A menudo subía al Capitolio y mientras contemplaba este magnífico templo dedicado a Júpiter, Juno y Minerva, reflexionaba sobre la ignorancia servil en la que estaban sumidos los romanos. También pasaba mucho tiempo en el monte Palatino, donde se encontraban la residencia del emperador, el templo de Apolo y las bibliotecas griega y latina.

\par 
%\textsuperscript{(1455.3)}
\textsuperscript{132:0.3} En esta época, el Imperio Romano incluía todo el sur de Europa, Asia Menor, Siria, Egipto y el noroeste de África, y entre sus habitantes se contaban ciudadanos de todos los países del hemisferio oriental. La razón principal por la que Jesús había consentido en hacer este viaje era su deseo de estudiar este conjunto cosmopolita de mortales de Urantia, y de mezclarse con ellos.

\par 
%\textsuperscript{(1455.4)}
\textsuperscript{132:0.4} Durante su estancia en Roma, Jesús aprendió muchas cosas sobre los hombres, pero la más valiosa de todas las múltiples experiencias de sus seis meses de permanencia en esta ciudad fue su contacto con los dirigentes religiosos de la capital del imperio, y la influencia que ejerció sobre ellos. Antes del final de su primera semana en Roma, Jesús había buscado, y había conocido, a los principales dirigentes de los cínicos, los estoicos y los cultos de misterio, en particular los del grupo mitríaco. Para Jesús podía ser o no evidente que los judíos iban a rechazar su misión, pero preveía con toda seguridad que sus mensajeros no tardarían en venir a Roma para proclamar el reino de los cielos; por lo tanto se dedicó a preparar el camino, de la manera más sorprendente, para que su mensaje fuera recibido mejor y con más seguridad. Seleccionó a cinco dirigentes de los estoicos, a once de los cínicos y a dieciséis jefes del culto de los misterios, y pasó una gran parte de su tiempo libre, durante casi seis meses, en asociación íntima con estos educadores religiosos. He aquí el método que utilizó para instruirlos: ni una sola vez atacó sus errores ni tampoco mencionó nunca los defectos de sus enseñanzas. En cada caso seleccionaba la verdad que había en lo que enseñaban, y luego procedía a embellecer e iluminar esta verdad en sus mentes de tal manera que en muy poco tiempo este realzamiento de la verdad desplazaba eficazmente el error que la acompañaba; así es como estos hombres y mujeres enseñados por Jesús fueron preparados para reconocer posteriormente verdades adicionales y similares en las enseñanzas de los primeros misioneros cristianos. Esta pronta aceptación de las enseñanzas de los predicadores del evangelio fue lo que dio un impulso tan poderoso a la rápida difusión del cristianismo en Roma, y desde allí, a todo el imperio.

\par 
%\textsuperscript{(1456.1)}
\textsuperscript{132:0.5} Se puede comprender mejor el significado de esta actividad extraordinaria cuando observamos el hecho de que, de este grupo de treinta y dos dirigentes religiosos de Roma instruídos por Jesús, solamente dos fueron estériles; los otros treinta jugaron un papel central en el establecimiento del cristianismo en Roma, y algunos de ellos ayudaron también a que el principal templo mitríaco se convirtiera en la primera iglesia cristiana de esta ciudad. Nosotros, que contemplamos las actividades humanas desde los bastidores y a la luz de los diecinueve siglos transcurridos, reconocemos solamente tres factores con un valor fundamental que contribuyeron a preparar muy pronto el terreno para la rápida propagación del cristianismo por toda Europa, y son los siguientes:

\par 
%\textsuperscript{(1456.2)}
\textsuperscript{132:0.6} 1. La elección y el mantenimiento de Simón Pedro como apóstol\footnote{\textit{La selección de Pedro}: Mt 4:18-20.}.

\par 
%\textsuperscript{(1456.3)}
\textsuperscript{132:0.7} 2. La conversación en Jerusalén con Esteban, cuya muerte condujo a atraer a Saulo de Tarso\footnote{\textit{El apedreamiento de Esteban}: Hch 6:8-7:60.}.

\par 
%\textsuperscript{(1456.4)}
\textsuperscript{132:0.8} 3. La preparación preliminar de estos treinta romanos para que dirigieran posteriormente la nueva religión en Roma y en todo el imperio.

\par 
%\textsuperscript{(1456.5)}
\textsuperscript{132:0.9} En el transcurso de todas sus experiencias, ni Esteban ni los treinta escogidos se dieron cuenta nunca de que habían hablado una vez con el hombre cuyo nombre se había convertido en el tema de sus enseñanzas religiosas. La obra de Jesús a favor de estos primeros treinta y dos fue enteramente personal. En sus trabajos con estas personas, el escriba de Damasco nunca se reunió con más de tres a la vez, rara vez con más de dos, y la mayoría de las veces los enseñaba individualmente. Pudo llevar a cabo esta gran obra de educación religiosa porque estos hombres y mujeres no estaban atados a las tradiciones, no eran víctimas de ideas preconcebidas sobre todos los desarrollos religiosos del futuro.

\par 
%\textsuperscript{(1456.6)}
\textsuperscript{132:0.10} En los años que siguieron después, Pedro, Pablo y los otros cristianos que enseñaron en Roma oyeron hablar muchísimas veces de este escriba de Damasco que los había precedido, y que tan evidentemente había preparado el camino (sin darse cuenta, suponían ellos) para su llegada con el nuevo evangelio. Pablo nunca adivinó realmente la identidad de este escriba de Damasco, pero poco tiempo antes de su muerte, debido a la similitud de las descripciones de la persona, llegó a la conclusión de que el <<fabricante de tiendas de Antioquía>> era también el <<escriba de Damasco>>. En cierta ocasión, mientras predicaba en Roma, Simón Pedro sospechó, al escuchar una descripción del escriba de Damasco, que este individuo podría haber sido Jesús, pero rápidamente desechó la idea, sabiendo muy bien (eso creía él) que el Maestro nunca había estado en Roma.

\section*{1. Los verdaderos valores}
\par 
%\textsuperscript{(1456.7)}
\textsuperscript{132:1.1} Al principio de su estancia en Roma, Jesús tuvo una conversación de toda una noche con Angamón, el jefe de los estoicos. Este hombre se hizo posteriormente un gran amigo de Pablo y llegó a ser uno de los fervorosos seguidores de la iglesia cristiana en Roma. He aquí en esencia, y transcrito a un lenguaje moderno, lo que Jesús enseñó a Angamón:

\par 
%\textsuperscript{(1457.1)}
\textsuperscript{132:1.2} El modelo de los verdaderos valores ha de buscarse en el mundo espiritual y en los niveles divinos de la realidad eterna. Para un mortal ascendente, todas las normas más bajas y materiales deben ser consideradas como transitorias, parciales e inferiores. El científico, como tal, está limitado a descubrir la conexión entre los hechos materiales. Técnicamente, no tiene derecho a afirmar que es materialista o idealista, porque al hacerlo se supone que abandona la actitud de un verdadero científico, ya que todas y cada una de estas tomas de posición son la esencia misma de la filosofía.

\par 
%\textsuperscript{(1457.2)}
\textsuperscript{132:1.3} A menos que la perspicacia moral y el logro espiritual de la humanidad aumenten proporcionalmente, el progreso ilimitado de una cultura puramente materialista puede acabar transformándose en una amenaza para la civilización. Una ciencia puramente materialista alberga dentro de sí la semilla potencial de la destrucción de todo esfuerzo científico, porque este tipo de conducta es el presagio del colapso final de una civilización que ha abandonado su sentido de los valores morales y ha repudiado su meta de realización espiritual.

\par 
%\textsuperscript{(1457.3)}
\textsuperscript{132:1.4} El científico materialista y el idealista extremo están destinados a enfrentarse continuamente. Esto no es aplicable a aquellos científicos e idealistas que poseen un modelo común de valores morales elevados y de niveles de prueba espirituales. En todas las épocas, los científicos y las personas religiosas deben reconocer que pasan por el juicio del tribunal de las necesidades humanas. Deben evitar todo tipo de lucha entre ellos, mientras se esfuerzan valientemente por justificar su supervivencia mediante una mayor devoción al servicio del progreso humano. Si la pretendida ciencia o la pretendida religión de una época cualquiera es falsa, entonces deberá purificar sus actividades o bien desaparecer ante el surgimiento de una ciencia material o de una religión espiritual de un orden más auténtico y más digno.

\section*{2. El bien y el mal}
\par 
%\textsuperscript{(1457.4)}
\textsuperscript{132:2.1} Mardus era el jefe reconocido de los cínicos de Roma, y se hizo muy amigo del escriba de Damasco. Día tras día conversaba con Jesús, y noche tras noche escuchaba su enseñanza celestial. Entre las discusiones más importantes con Mardus, hubo una destinada a responder a la pregunta de este cínico sincero sobre el bien y el mal. Transcrito al lenguaje del siglo veinte, Jesús le dijo en esencia:

\par 
%\textsuperscript{(1457.5)}
\textsuperscript{132:2.2} Hermano mío, el bien y el mal son simplemente unas palabras que simbolizan los niveles relativos de comprensión humana del universo observable. Si eres éticamente perezoso y socialmente indiferente, puedes coger como modelo del bien las costumbres sociales corrientes. Si eres espiritualmente indolente y moralmente estático, puedes coger como modelo del bien las prácticas y tradiciones religiosas de tus contemporáneos. Pero el alma que sobrevive al tiempo y emerge en la eternidad debe efectuar una elección viviente y personal entre el bien y el mal, tal como éstos están determinados por los verdaderos valores de las normas espirituales establecidas por el espíritu divino que el Padre que está en los cielos ha enviado a residir en el corazón del hombre. Este espíritu interior es la norma de la supervivencia de la personalidad.

\par 
%\textsuperscript{(1457.6)}
\textsuperscript{132:2.3} La bondad, lo mismo que la verdad, siempre es relativa y contrasta infaliblemente con el mal. La percepción de estas cualidades de bondad y de verdad es lo que permite a las almas evolutivas de los hombres efectuar esas decisiones personales de elección que son esenciales para la supervivencia eterna.

\par 
%\textsuperscript{(1458.1)}
\textsuperscript{132:2.4} El individuo espiritualmente ciego que sigue lógicamente los dictados de la ciencia, las costumbres sociales y los dogmas religiosos, se encuentra en el grave peligro de sacrificar su independencia moral y de perder su libertad espiritual. Un alma así está destinada a convertirse en un papagayo intelectual, en un autómata social y en un esclavo de la autoridad religiosa.

\par 
%\textsuperscript{(1458.2)}
\textsuperscript{132:2.5} La bondad siempre está creciendo hacia nuevos niveles de mayor libertad para autorrealizarse moralmente y alcanzar la personalidad espiritual ---el descubrimiento del Ajustador interior y la identificación con él. Una experiencia es buena cuando eleva la apreciación de la belleza, aumenta la voluntad moral, realza el discernimiento de la verdad, aumenta la capacidad para amar y servir a nuestros semejantes, exalta los ideales espirituales y unifica los supremos motivos humanos del tiempo con los planes eternos del Ajustador interior. Todo esto conduce directamente a un mayor deseo de hacer la voluntad del Padre, alimentando así la pasión divina de encontrar a Dios y de parecerse más a él.

\par 
%\textsuperscript{(1458.3)}
\textsuperscript{132:2.6} A medida que ascendéis la escala universal de desarrollo de las criaturas, encontraréis una bondad creciente y una disminución del mal, en perfecta conformidad con vuestra capacidad para experimentar la bondad y discernir la verdad. La capacidad de mantener el error o de experimentar el mal no se perderá por completo hasta que el alma humana ascendente alcance los niveles espirituales finales.

\par 
%\textsuperscript{(1458.4)}
\textsuperscript{132:2.7} La bondad es viviente, relativa, siempre en progreso; es invariablemente una experiencia personal y está perpetuamente correlacionada con el discernimiento de la verdad y de la belleza. La bondad se encuentra en el reconocimiento de los valores positivos de verdad del nivel espiritual, que deben contrastar, en la experiencia humana, con su contrapartida negativa ---las sombras del mal potencial.

\par 
%\textsuperscript{(1458.5)}
\textsuperscript{132:2.8} Hasta que no alcancéis los niveles del Paraíso, la bondad siempre será más una búsqueda que una posesión, más una meta que una experiencia lograda. Pero cuando se tiene hambre y sed de rectitud, se experimenta una satisfacción creciente cuando se alcanza parcialmente la bondad. La presencia del bien y del mal en el mundo es, en sí misma, una prueba positiva de la existencia y de la realidad de la voluntad moral del hombre, de la personalidad, que identifica así estos valores y también es capaz de escoger entre ellos.

\par 
%\textsuperscript{(1458.6)}
\textsuperscript{132:2.9} En la época en que un mortal ascendente alcanza el Paraíso, su capacidad para identificar su yo con los verdaderos valores espirituales se ha ampliado tanto, que ha conseguido la posesión perfecta de la luz de la vida\footnote{\textit{Posesión de la luz de la vida}: Is 9:2; Jn 8:12; 1 Jn 2:8.}. Una personalidad espiritual así perfeccionada se unifica tan completa, divina y espiritualmente con las cualidades supremas y positivas de la bondad, de la belleza y de la verdad, que no queda ninguna posibilidad de que un espíritu así de recto pueda arrojar alguna sombra negativa de mal potencial cuando es expuesto a la luminosidad penetrante de la luz divina de los Soberanos infinitos del Paraíso. En todas estas personalidades espirituales, la bondad ha dejado de ser parcial, contrastante y comparativa; se ha vuelto divinamente completa y espiritualmente plena; se acerca a la pureza y a la perfección del Supremo.

\par 
%\textsuperscript{(1458.7)}
\textsuperscript{132:2.10} La \textit{posibilidad} del mal es necesaria para la elección moral, pero su realidad no lo es. Una sombra sólo tiene una realidad relativa. El mal real no es necesario como experiencia personal. El mal potencial funciona igual de bien como estímulo para tomar decisiones en el ámbito del progreso moral, en los niveles inferiores del desarrollo espiritual. El mal sólo se vuelve una realidad de la experiencia personal cuando una mente moral lo escoge deliberadamente.

\section*{3. La verdad y la fe}
\par 
%\textsuperscript{(1459.1)}
\textsuperscript{132:3.1} Nabon era un judío griego y el más importante de los dirigentes del principal culto de misterio en Roma, el culto mitríaco. Aunque este sumo sacerdote del mitracismo mantuvo muchas conversaciones con el escriba de Damasco, lo que le influyó de manera más permanente fue la discusión que tuvieron una noche sobre la verdad y la fe. Nabon había pensado en convertir a Jesús e incluso le había sugerido que regresara a Palestina como educador mitríaco. No sospechaba que Jesús lo estaba preparando para volverse uno de los primeros convertidos al evangelio del reino. Transcrito en una terminología moderna, he aquí en esencia lo que Jesús le enseñó:

\par 
%\textsuperscript{(1459.2)}
\textsuperscript{132:3.2} La verdad no se puede definir con palabras, sino solamente viviéndola. La verdad es siempre más que el conocimiento. El conocimiento se refiere a las cosas observadas, pero la verdad trasciende estos niveles puramente materiales en el sentido de que se asocia con la sabiduría y engloba unos imponderables tales como la experiencia humana e incluso las realidades espirituales y vivientes. El conocimiento se origina en la ciencia; la sabiduría, en la verdadera filosofía; la verdad, en la experiencia religiosa de la vida espiritual. El conocimiento trata de los hechos; la sabiduría, de las relaciones; la verdad, de los valores de la realidad.

\par 
%\textsuperscript{(1459.3)}
\textsuperscript{132:3.3} El hombre tiende a cristalizar la ciencia, a formular la filosofía y a dogmatizar la verdad, porque tiene pereza mental para adaptarse a las luchas progresivas de la vida, y porque tiene también un miedo terrible a lo desconocido. El hombre normal es lento en introducir cambios en sus hábitos de pensamiento y en sus técnicas de vida.

\par 
%\textsuperscript{(1459.4)}
\textsuperscript{132:3.4} La verdad revelada, la verdad descubierta personalmente, es la delicia suprema del alma humana; es la creación conjunta de la mente material y del espíritu interior. La salvación eterna de este alma que discierne la verdad y que ama la belleza, está asegurada por ese hambre y esa sed de bondad que conducen a este mortal a desarrollar una sola finalidad, la de hacer la voluntad del Padre, encontrar a Dios y volverse como él. Nunca existe conflicto entre el verdadero conocimiento y la verdad. Puede haber conflicto entre el conocimiento y las creencias humanas, las creencias teñidas de prejuicios, deformadas por el miedo y dominadas por el terror de tener que afrontar los nuevos hechos de los descubrimientos materiales o de los progresos espirituales.

\par 
%\textsuperscript{(1459.5)}
\textsuperscript{132:3.5} Pero el hombre nunca puede poseer la verdad sin el ejercicio de la fe. Esto es así porque los pensamientos, la sabiduría, la ética y los ideales del hombre nunca se elevarán por encima de su fe, de su esperanza sublime. Y toda verdadera fe de este tipo está basada en una reflexión profunda, en una autocrítica sincera y en una conciencia moral intransigente. La fe es la inspiración de la imaginación creativa impregnada de espíritu.

\par 
%\textsuperscript{(1459.6)}
\textsuperscript{132:3.6} La fe actúa para liberar las actividades superhumanas de la chispa divina, el germen inmortal que vive dentro de la mente del hombre, y que es el potencial de la supervivencia eterna. Las plantas y los animales sobreviven en el tiempo mediante la técnica de transmitir partículas idénticas de sí mismos de una generación a la siguiente. El alma humana del hombre (la personalidad) sobrevive a la muerte física asociando su identidad con esta chispa interior de divinidad, que es inmortal, y que actúa para perpetuar la personalidad humana en un nivel continuo y más elevado de existencia progresiva en el universo. La semilla oculta del alma humana es un espíritu inmortal. La segunda generación del alma es la primera de una serie de manifestaciones de la personalidad en existencias espirituales y progresivas, que sólo terminan cuando esta entidad divina alcanza la fuente de su existencia, la fuente personal de toda existencia, Dios, el Padre Universal.

\par 
%\textsuperscript{(1459.7)}
\textsuperscript{132:3.7} La vida humana continúa ---sobrevive--- porque tiene una función en el universo, la tarea de encontrar a Dios. El alma del hombre, activada por la fe, no puede detenerse hasta haber alcanzado esta meta de su destino; y una vez que ha conseguido esta meta divina, ya no puede tener fin porque se ha vuelto como Dios ---eterna.

\par 
%\textsuperscript{(1460.1)}
\textsuperscript{132:3.8} La evolución espiritual es una experiencia de la elección creciente y voluntaria de la bondad, acompañada de una disminución igual y progresiva de la posibilidad del mal. Cuando se alcanza la finalidad de elección de la bondad y la plena capacidad para apreciar la verdad, surge a la existencia una perfección de belleza y de santidad cuya rectitud inhibe eternamente la posibilidad de que emerja siquiera el concepto del mal potencial. El alma que conoce así a Dios no proyecta ninguna sombra de mal que ocasione dudas, cuando funciona en un nivel espiritual tan elevado de divina bondad.

\par 
%\textsuperscript{(1460.2)}
\textsuperscript{132:3.9} La presencia del espíritu del Paraíso en la mente del hombre constituye la promesa de la revelación y la garantía de la fe de una existencia eterna de progresión divina para todas las almas que tratan de identificarse con este fragmento espiritual interior e inmortal del Padre Universal.

\par 
%\textsuperscript{(1460.3)}
\textsuperscript{132:3.10} El progreso en el universo está caracterizado por una libertad creciente de la personalidad, porque está asociado con el logro progresivo de niveles cada vez más elevados de comprensión de sí mismo y del consiguiente dominio voluntario de sí mismo. Alcanzar la perfección del dominio espiritual de sí mismo equivale a consumar la independencia en el universo y la libertad personal. La fe alimenta y mantiene al alma del hombre en medio de la confusión de su orientación inicial en un universo tan vasto, mientras que la oración se convierte en el gran unificador de las diversas inspiraciones de la imaginación creativa y de los impulsos de fe de un alma que trata de identificarse con los ideales espirituales de la divina presencia interior y asociada.

\par 
%\textsuperscript{(1460.4)}
\textsuperscript{132:3.11} Nabon se quedó muy impresionado con estas palabras, tal como le sucedía con cada una de sus conversaciones con Jesús. Estas verdades continuaron ardiendo dentro de su corazón, y prestó una gran ayuda a los predicadores del evangelio de Jesús que llegaron más tarde.

\section*{4. Ministerio personal}
\par 
%\textsuperscript{(1460.5)}
\textsuperscript{132:4.1} Mientras estuvo en Roma, Jesús no dedicó todo su tiempo libre a esta tarea de preparar a hombres y mujeres para que se convirtieran en futuros discípulos del reino venidero. Pasó mucho tiempo adquiriendo un conocimiento íntimo de todas las razas y clases de hombres que vivían en esta ciudad, la más grande y cosmopolita del mundo. En cada uno de estos numerosos contactos humanos, Jesús tenía una doble finalidad: deseaba conocer la reacción de sus interlocutores ante la vida que estaban viviendo en la carne, y también era propenso a decir o a hacer algo que hiciera esta vida más rica y más digna de ser vivida. Durante estas semanas, sus enseñanzas religiosas no fueron diferentes de las que caracterizaron su vida posterior como educador de los doce y predicador para las multitudes.

\par 
%\textsuperscript{(1460.6)}
\textsuperscript{132:4.2} La idea central de su mensaje era siempre el hecho del amor del Padre celestial y la verdad de su misericordia, unido a la buena nueva de que el hombre es un hijo por la fe de este mismo Dios de amor. La técnica habitual que Jesús utilizaba en sus contactos sociales consistía en hacer preguntas a la gente para hacerles hablar y llevarlos a conversar con él. Al principio de la entrevista, él era el que habitualmente solía hacer las preguntas, y al final eran ellos los que le interrogaban. Tenía la misma habilidad para enseñar haciendo preguntas como contestándolas. Por regla general, a quienes más enseñaba es a quienes menos decía. Los que obtuvieron el mayor beneficio de su ministerio personal fueron los mortales agobiados, ansiosos y deprimidos, que encontraron mucho alivio en esta posibilidad de desahogar sus almas con un oyente compasivo y comprensivo, y él era todo esto y mucho más. Cuando estos seres humanos inadaptados habían contado sus problemas a Jesús, éste siempre estaba en condiciones de ofrecerles sugerencias prácticas e inmediatamente útiles para corregir sus verdaderas dificultades, y nunca dejaba de decirles palabras de alivio para el presente y de inmediato consuelo. A estos mortales afligidos les hablaba invariablemente del amor de Dios, y mediante métodos diversos y variados, les trasmitía el mensaje de que eran los hijos de este afectuoso Padre que está en los cielos.

\par 
%\textsuperscript{(1461.1)}
\textsuperscript{132:4.3} De esta manera, durante su estancia en Roma, Jesús tuvo personalmente un contacto afectuoso y edificante con más de quinientos mortales del mundo. Consiguió así un conocimiento de las diferentes razas de la humanidad que nunca hubiera podido adquirir en Jerusalén y quizás tampoco en Alejandría. Siempre consideró estos seis meses como uno de los períodos más ricos e instructivos de su vida terrestre.

\par 
%\textsuperscript{(1461.2)}
\textsuperscript{132:4.4} Como era de esperar, un hombre tan hábil y dinámico no podía vivir así durante seis meses en la metrópolis del mundo sin ser abordado por numerosas personas que deseaban obtener sus servicios para algún negocio o, más a menudo, para algún proyecto de enseñanza, de reforma social o de movimiento religioso. Recibió más de una docena de proposiciones de este tipo, y aprovechó cada una de ellas como una oportunidad para transmitir algún pensamiento de ennoblecimiento espiritual mediante palabras bien escogidas o por medio de algún favor servicial. A Jesús le encantaba hacer cosas ---incluso de poca importancia--- por toda clase de gente.

\par 
%\textsuperscript{(1461.3)}
\textsuperscript{132:4.5} Estuvo hablando con un senador romano sobre política y el arte de gobernar, y este único contacto con Jesús hizo tal impresión en este legislador que pasó el resto de su vida tratando en vano de persuadir a sus colegas para que cambiaran el curso de la política en vigor, sustituyendo la idea de un gobierno que mantenía y alimentaba al pueblo, por la de un pueblo que mantuviera al gobierno. Jesús pasó una noche con un rico propietario de esclavos y le habló del hombre como hijo de Dios; al día siguiente, este hombre llamado Claudio concedió la libertad a ciento diecisiete esclavos. Fue a cenar con un médico griego y le hizo saber que sus pacientes tenían una mente y un alma además de un cuerpo, induciendo así a este experto doctor a esforzarse por ayudar más ampliamente a sus semejantes. Conversó con todo tipo de personas de todos los ambientes y profesiones. El único lugar de Roma que no visitó fueron los baños públicos. Rehusó acompañar a sus amigos a los baños a causa de la promiscuidad sexual que predominaba allí.

\par 
%\textsuperscript{(1461.4)}
\textsuperscript{132:4.6} Mientras caminaba con un soldado romano a lo largo del Tiber, Jesús le dijo: <<Que tu corazón sea tan valiente como tu brazo. Atrévete a hacer justicia y sé lo bastante noble como para mostrar misericordia. Obliga a tu naturaleza inferior a obedecer a tu naturaleza superior, como tú obedeces a tus superiores. Venera la bondad y exalta la verdad. Escoge la belleza en lugar de la fealdad. Ama a tus semejantes y busca a Dios con todo tu corazón, porque Dios es tu Padre que está en los cielos>>.

\par 
%\textsuperscript{(1461.5)}
\textsuperscript{132:4.7} Al orador del foro le dijo: <<Tu elocuencia es placentera, tu lógica es admirable, tu voz es agradable, pero tu enseñanza no refleja la verdad. Si pudieras tan sólo disfrutar de la satisfacción inspiradora de conocer a Dios como tu Padre espiritual, entonces podrías emplear tu capacidad de orador para liberar a tus semejantes de la servidumbre de las tinieblas y de la esclavitud de la ignorancia>>. Éste fue el mismo Marcos\footnote{\textit{Marcos}: Col 4:10; Flm 1:24; 1 P 5:13.} que escuchó predicar a Pedro en Roma y se convirtió en su sucesor. Cuando crucificaron a Simón Pedro, este hombre fue el que desafió a los perseguidores romanos y continuó predicando audazmente el nuevo evangelio.

\par 
%\textsuperscript{(1462.1)}
\textsuperscript{132:4.8} Al encontrarse con un pobre hombre que había sido acusado falsamente, Jesús lo acompañó ante el magistrado y, una vez que le concedieron la autorización especial de comparecer en su nombre, pronunció un magnífico discurso en el cual dijo: <<La justicia engrandece a una nación, y cuanto más grande es una nación, más cuidado pondrá en que la injusticia no alcance ni al más humilde de sus ciudadanos. ¡Pobre de la nación en la que sólo los que poseen dinero e influencia pueden obtener una justicia pronta de sus tribunales! Un magistrado tiene el deber sagrado de absolver al inocente así como de castigar al culpable. La continuidad de una nación depende de la imparcialidad, de la equidad y de la integridad de sus tribunales. El gobierno civil está basado en la justicia, así como la verdadera religión está basada en la misericordia>>. El juez reconsideró el caso y después de examinar las pruebas, absolvió al acusado. De todas las actividades de Jesús durante este período de ministerio personal, ésta fue la que estuvo más cerca de ser una aparición pública.

\section*{5. Consejos para el hombre rico}
\par 
%\textsuperscript{(1462.2)}
\textsuperscript{132:5.1} Cierto hombre rico, ciudadano romano y estoico, llegó a interesarse mucho por las enseñanzas de Jesús, a quien había sido presentado por Angamón. Después de muchas conversaciones cordiales, este rico ciudadano preguntó a Jesús qué haría él con la riqueza si la tuviera, y Jesús le contestó: <<Dedicaría la riqueza material a mejorar la vida material, al igual que utilizaría el conocimiento, la sabiduría y el servicio espiritual para enriquecer la vida intelectual, ennoblecer la vida social y hacer progresar la vida espiritual. Administraría la riqueza material como un depositario prudente y eficaz de los recursos de una generación, para el beneficio y el ennoblecimiento de las generaciones próximas y sucesivas>>.

\par 
%\textsuperscript{(1462.3)}
\textsuperscript{132:5.2} Pero el hombre rico no estaba satisfecho del todo con la respuesta de Jesús, y se atrevió a preguntar de nuevo: <<¿Pero qué crees que debería hacer con su riqueza un hombre que estuviera en mi lugar? ¿Debería guardarla o repartirla?>> Cuando Jesús se dio cuenta de que este hombre deseaba realmente conocer mejor la verdad sobre su lealtad a Dios y su deber hacia los hombres, amplió su respuesta diciéndole: <<Mi buen amigo, discierno que buscas sinceramente la sabiduría y que amas honradamente la verdad; por eso me propongo exponerte mi punto de vista sobre la solución de tus problemas relacionados con las responsabilidades de la riqueza. Hago esto porque has \textit{pedido} mi consejo, y al ofrecerte esta reflexión, no me intereso por la riqueza de ningún otro hombre rico; mi consejo es sólo para ti y para tu conducta personal. Si deseas honradamente considerar tu riqueza como un depósito, si quieres realmente convertirte en un administrador prudente y eficaz de tu riqueza acumulada, entonces te aconsejaría que hicieras el siguiente análisis de los orígenes de tus riquezas. Pregúntate, y haz todo lo posible por encontrar la respuesta honrada, ¿de dónde procede esta riqueza? Para ayudarte a analizar los orígenes de tu gran fortuna, te sugeriría que recordaras los siguientes diez métodos diferentes de acumular bienes materiales>>:

\par 
%\textsuperscript{(1462.4)}
\textsuperscript{132:5.3} <<1. La riqueza heredada ---los bienes recibidos de los padres y de otros antepasados>>.

\par 
%\textsuperscript{(1462.5)}
\textsuperscript{132:5.4} <<2. La riqueza descubierta ---los bienes que proceden de los recursos no explotados de la madre Tierra>>.

\par 
%\textsuperscript{(1462.6)}
\textsuperscript{132:5.5} <<3. La riqueza comercial ---los bienes obtenidos como un beneficio justo en el intercambio y el trueque de las mercancías materiales>>.

\par 
%\textsuperscript{(1462.7)}
\textsuperscript{132:5.6} <<4. La riqueza injusta ---los bienes procedentes de la explotación injusta o de la esclavitud de nuestros semejantes>>.

\par 
%\textsuperscript{(1463.1)}
\textsuperscript{132:5.7} <<5. La riqueza del interés ---el beneficio derivado de las posibilidades de una ganancia justa y equitativa por los capitales invertidos>>.

\par 
%\textsuperscript{(1463.2)}
\textsuperscript{132:5.8} <<6. La riqueza debida al talento ---los bienes resultantes de las recompensas por los dones creativos e inventivos de la mente humana>>.

\par 
%\textsuperscript{(1463.3)}
\textsuperscript{132:5.9} <<7. la riqueza accidental ---los bienes procedentes de la generosidad de nuestros semejantes o que tienen su origen en las circunstancias de la vida>>.

\par 
%\textsuperscript{(1463.4)}
\textsuperscript{132:5.10} <<8. La riqueza robada ---los bienes obtenidos mediante la injusticia, la picardía, el robo o el fraude>>.

\par 
%\textsuperscript{(1463.5)}
\textsuperscript{132:5.11} <<9. Los fondos en depósito ---la riqueza colocada en tus manos por tus semejantes para una utilidad específica, presente o futura>>.

\par 
%\textsuperscript{(1463.6)}
\textsuperscript{132:5.12} <<10. La riqueza ganada ---los bienes que proceden directamente de tu propio trabajo personal, la recompensa justa y equitativa por tus propios esfuerzos diarios, mentales o físicos>>.

\par 
%\textsuperscript{(1463.7)}
\textsuperscript{132:5.13} <<Así pues, amigo mío, si quieres ser un administrador fiel y justo de tu gran fortuna, ante Dios y al servicio de los hombres, debes dividirla aproximadamente en estos diez grandes grupos, y luego administrar cada porción de acuerdo con la interpretación sabia y honrada de las leyes de la justicia, de la equidad, de la honradez y de la verdadera eficacia. No obstante, el Dios del cielo no te condenará si, en situaciones dudosas, a veces te equivocas a favor de una consideración misericordiosa y desinteresada por la aflicción de las víctimas que sufren las desgraciadas circunstancias de la vida mortal. Cuando tengas dudas honradas sobre la equidad y la justicia de una situación material, que tus decisiones favorezcan a los que están necesitados y ayuden a los que sufren la desdicha de unas penalidades inmerecidas>>.

\par 
%\textsuperscript{(1463.8)}
\textsuperscript{132:5.14} Después de discutir estas cuestiones durante varias horas, el hombre rico solicitó instrucciones más completas y detalladas, y Jesús amplió su consejo diciendo en sustancia: <<Al ofrecerte nuevas sugerencias relativas a tu actitud hacia la riqueza, te exhortaría a que recibieras mi consejo como destinado exclusivamente para ti y para tu conducta personal. Sólo hablo por cuenta propia y para ti como a un amigo que busca información. Te ruego que no dictes a otros hombres ricos cómo deben estimar su riqueza. Te aconsejaría que>>:

\par 
%\textsuperscript{(1463.9)}
\textsuperscript{132:5.15} <<1. Como administrador de una riqueza heredada, deberías considerar sus orígenes. Tienes la obligación moral de representar a la generación anterior en la transmisión honrada de una riqueza legítima a las generaciones siguientes, después de deducir una tasa justa para el beneficio de la generación presente. Pero no estás obligado a perpetuar cualquier fraude o injusticia implicados en la acumulación injusta de unas riquezas por parte de tus antepasados. Cualquier porción de tu riqueza heredada que resulte provenir del fraude o de la injusticia, puedes desembolsarla de acuerdo con tus convicciones de la justicia, de la generosidad y de la restitución. En cuanto al resto de tu riqueza legítimamente heredada, puedes utilizarla con equidad y trasmitirla con seguridad como depositario de una generación para la siguiente. Una sabia discriminación y un juicio sano deberían dictar tus decisiones en cuanto al legado de las riquezas a tus sucesores>>.

\par 
%\textsuperscript{(1463.10)}
\textsuperscript{132:5.16} <<2. Todo aquel que disfruta de la riqueza como resultado de un descubrimiento debería recordar que un individuo sólo puede vivir en la Tierra un corto período de tiempo; por consiguiente, debería tomar las disposiciones adecuadas para compartir estos descubrimientos de manera útil con el mayor número posible de sus semejantes. Aunque al descubridor no hay que negarle toda recompensa por sus esfuerzos de descubrimiento, tampoco debería atreverse egoístamente a reclamar todas las ventajas y bendiciones que se pueden obtener de la puesta al descubierto de los recursos atesorados por la naturaleza>>.

\par 
%\textsuperscript{(1464.1)}
\textsuperscript{132:5.17} <<3. Mientras que los hombres escojan concertar los negocios del mundo mediante el comercio y el trueque, tienen derecho a un beneficio justo y legítimo. Todo comerciante merece una remuneración por sus servicios; el negociante tiene derecho a su salario. La equidad comercial y el trato honrado que se otorga a los semejantes en los negocios organizados del mundo, crean muchos tipos diferentes de riquezas debidas a los beneficios, y todas estas fuentes de riqueza deben ser juzgadas según los principios más elevados de la justicia, la honradez y la equidad. El comerciante honrado no debería dudar en percibir el mismo beneficio que concedería gustosamente a un colega suyo por una operación similar. Aunque este tipo de riqueza, cuando los negocios se realizan a gran escala, no es idéntico a los ingresos ganados individualmente, al mismo tiempo, una riqueza acumulada así honradamente confiere a su poseedor un voto de una considerable equidad en el momento de repartirla posteriormente>>.

\par 
%\textsuperscript{(1464.2)}
\textsuperscript{132:5.18} <<4. Ningún mortal que conoce a Dios y trata de hacer la voluntad divina puede rebajarse hasta comprometerse con las opresiones de la riqueza. Ningún hombre noble se esforzará por acumular riquezas y amasar un poder financiero mediante la esclavización o la explotación injusta de sus hermanos en la carne. Cuando proceden del sudor de los mortales oprimidos, las riquezas son una maldición moral y una infamia espiritual. Toda riqueza de este tipo debería ser restituida a quienes han sido así desposeídos, o a sus hijos y a los hijos de sus hijos. No se puede construir una civilización duradera sobre la práctica de engañar al trabajador en su salario>>.

\par 
%\textsuperscript{(1464.3)}
\textsuperscript{132:5.19} <<5. La riqueza honrada tiene derecho a unos intereses. Mientras que los hombres pidan prestado y concedan préstamos, pueden percibir un interés equitativo siempre que el capital prestado proceda de una riqueza legítima. Purifica primero tu capital antes de reclamar los intereses. No te vuelvas tan despreciable y avaricioso como para rebajarte a practicar la usura. No te permitas nunca ser tan egoísta como para emplear el poder del dinero para obtener una ventaja injusta sobre tus semejantes que luchan. No cedas a la tentación de ser usurero con tu hermano que tiene apuros financieros>>.

\par 
%\textsuperscript{(1464.4)}
\textsuperscript{132:5.20} <<6. Si llegas a conseguir la riqueza mediante el despliegue de tu talento, si tus riquezas proceden de las remuneraciones por tus dotes inventivas, no reclames una porción injusta de dichas remuneraciones. El talento le debe algo tanto a sus antepasados como a sus descendientes; también tiene obligaciones con respecto a la raza, a la nación y a las circunstancias de sus descubrimientos ingeniosos; debería recordar también que trabajó y elaboró sus inventos como un hombre entre los hombres. Sería igualmente injusto impedir que una persona ingeniosa pueda incrementar su riqueza. A los hombres siempre les resultará imposible establecer leyes y reglas que se apliquen por igual a todos estos problemas de la distribución equitativa de la riqueza. Primero debes reconocer al hombre como hermano tuyo, y si deseas honradamente hacer por él lo que quisieras que hiciera por ti, los dictados elementales de la justicia, de la honradez y de la equidad te guiarán para arreglar de manera justa e imparcial todos los problemas recurrentes de las remuneraciones económicas y de la justicia social>>.

\par 
%\textsuperscript{(1464.5)}
\textsuperscript{132:5.21} <<7. Ningún hombre debería reclamar para sí una riqueza que el tiempo y la suerte pueden haber depositado entre sus manos, excepto los honorarios justos y legítimos obtenidos por administrarla. Las riquezas accidentales deberían considerarse un poco como un depósito para ser empleado en beneficio de nuestro grupo económico o social. Los poseedores de estas riquezas deberían tener el voto principal a la hora de determinar la distribución sabia y eficaz de estos recursos no ganados. El hombre civilizado no siempre considerará todo lo que controla como su propiedad personal y privada>>.

\par 
%\textsuperscript{(1465.1)}
\textsuperscript{132:5.22} <<8. Si una porción determinada de tu fortuna ha sido obtenida adrede por medio del fraude, si una fracción de tus bienes ha sido acumulada mediante prácticas fraudulentas o métodos no equitativos, si tus riquezas son el producto de negocios tratados injustamente con tus semejantes, apresúrate a restituir todas esas ganancias mal adquiridas a sus legítimos dueños. Efectúa todas las compensaciones necesarias y depura así tu fortuna de todos sus elementos indignos>>.

\par 
%\textsuperscript{(1465.2)}
\textsuperscript{132:5.23} <<9. La administración de los bienes que una persona realiza en beneficio de otras es una responsabilidad solemne y sagrada. No arriesgues ni pongas en peligro ese depósito. Coge únicamente para ti, de cualquier depósito, la fracción que aprobarían todos los hombres honrados>>.

\par 
%\textsuperscript{(1465.3)}
\textsuperscript{132:5.24} <<10. Aquella parte de tu fortuna que representa los ingresos de tus propios esfuerzos físicos y mentales ---si has trabajado con honradez y equidad--- es verdaderamente tuya. Nadie puede negarte el derecho a tener y a utilizar esa riqueza como lo estimes conveniente, siempre que el ejercicio de ese derecho no perjudique a tus semejantes>>.

\par 
%\textsuperscript{(1465.4)}
\textsuperscript{132:5.25} Cuando Jesús hubo terminado de darle estos consejos, el rico romano se levantó de su diván y, al desearle las buenas noches, le hizo esta promesa: <<Mi buen amigo, percibo que eres un hombre de gran sabiduría y bondad; mañana mismo empezaré a administrar todos mis bienes de acuerdo con tu consejo>>.

\section*{6. Ministerio social}
\par 
%\textsuperscript{(1465.5)}
\textsuperscript{132:6.1} Fue también aquí en Roma donde se produjo aquel incidente enternecedor durante el cual el Creador de un universo pasó varias horas devolviendo un niño perdido a su madre angustiada. Este chico se había extraviado al alejarse de su casa, y Jesús lo encontró llorando desconsoladamente. Jesús y Ganid iban camino de las bibliotecas, pero se consagraron a llevar al niño a su casa. Ganid nunca olvidó el comentario de Jesús: <<Sabes, Ganid, la mayoría de los seres humanos son como este niño perdido. Pasan mucho tiempo llorando de temor y sufriendo de aflicción, cuando en verdad se encuentran muy cerca del amparo y de la seguridad, de la misma manera que este niño no estaba lejos de su casa. Todos aquellos que conocen el camino de la verdad y gozan de la seguridad de conocer a Dios, deberían considerar como un privilegio, y no como un deber, ofrecer su orientación a sus semejantes en sus esfuerzos por encontrar las satisfacciones de la vida. ¿No hemos disfrutado de manera suprema con este servicio de devolver el niño a su madre? De la misma forma, los que conducen los hombres a Dios experimentan la satisfacción suprema del servicio humano>>. A partir de aquel día y durante el resto de su vida en la Tierra, Ganid siempre estuvo a la búsqueda de niños perdidos que pudiera devolver a su hogar.

\par 
%\textsuperscript{(1465.6)}
\textsuperscript{132:6.2} Había una viuda con cinco hijos cuyo marido había muerto en un accidente. Jesús contó a Ganid cómo él mismo había perdido a su padre en un accidente, y fueron muchas veces a consolar a esta madre y a sus hijos, mientras que Ganid solicitó dinero a su padre para proporcionarles alimento y ropa. No pararon en sus esfuerzos hasta que encontraron un empleo para el hijo mayor, de manera que pudiera ayudar a mantener a la familia.

\par 
%\textsuperscript{(1465.7)}
\textsuperscript{132:6.3} Aquella noche, mientras Gonod escuchaba el relato de estas experiencias, dijo cariñosamente a Jesús: <<Me propongo hacer de mi hijo un erudito o un hombre de negocios, y ahora empiezas a hacer de él un filósofo o un filántropo>>. Jesús replicó sonriendo: <<Quizás hagamos de él las cuatro cosas; podrá gozar entonces de una cuádruple satisfacción en la vida, porque su oído hecho para reconocer la melodía humana podrá apreciar cuatro tonos en vez de uno>>. Entonces dijo Gonod: <<Percibo que eres realmente un filósofo. Debes escribir un libro para las generaciones futuras>>. Y Jesús respondió: <<No un libro ---mi misión es vivir una vida en esta generación y para todas las generaciones. Yo..>>.. Pero se detuvo y le dijo a Ganid: <<Hijo mío, es hora de acostarse>>.

\section*{7. Viajes fuera de Roma}
\par 
%\textsuperscript{(1466.1)}
\textsuperscript{132:7.1} Jesús, Gonod y Ganid hicieron cinco viajes desde Roma hacia puntos interesantes del territorio circundante. Durante su visita a los lagos del norte de Italia, Jesús tuvo una larga conversación con Ganid sobre la imposibilidad de enseñarle a un hombre cosas sobre Dios, si ese hombre no desea conocer a Dios. Mientras viajaban hacia los lagos, se habían encontrado por casualidad con un pagano irreflexivo, y Ganid se sorprendió al ver que Jesús no utilizaba su técnica habitual de entablar una conversación con aquel hombre, que hubiera conducido de manera natural a discutir sobre cuestiones espirituales. Cuando Ganid preguntó a su maestro por qué mostraba tan poco interés por este pagano, Jesús respondió:

\par 
%\textsuperscript{(1466.2)}
\textsuperscript{132:7.2} <<Ganid, este hombre no tenía hambre de la verdad. No estaba descontento de sí mismo. No estaba preparado para pedir ayuda, y los ojos de su mente no estaban abiertos para recibir la luz destinada al alma. Este hombre no estaba maduro para la cosecha de la salvación. Hay que concederle más tiempo para que las pruebas y las dificultades de la vida lo preparen para recibir la sabiduría y el conocimiento superior. O bien, si pudiera venir a vivir con nosotros, podríamos mostrarle al Padre que está en los cielos con nuestra manera de vivir; nuestras vidas, como hijos de Dios, podrían atraerlo hasta el punto de que se vería obligado a preguntar sobre nuestro Padre. No se puede revelar a Dios a los que no lo buscan; no se puede conducir a las alegrías de la salvación a un alma que no lo desea. Es preciso que el hombre tenga hambre de la verdad como resultado de las experiencias de la vida, o que desee conocer a Dios como consecuencia del contacto con la vida de aquellos que conocen al Padre divino, antes de que otro ser humano pueda actuar como intermediario para conducir a ese compañero mortal hacia el Padre que está en los cielos. Si conocemos a Dios, nuestra verdadera tarea en la Tierra consiste en vivir de tal manera que permitamos al Padre revelarse en nuestra vida, y así todas las personas que buscan a Dios verán al Padre y solicitarán nuestra ayuda para averiguar más cosas sobre el Dios que logra expresarse de ese modo en nuestra vida>>.

\par 
%\textsuperscript{(1466.3)}
\textsuperscript{132:7.3} En el transcurso de la visita a Suiza, mientras estaban en las montañas, Jesús tuvo una conversación de un día entero con el padre y el hijo sobre el budismo. Ganid había hecho muchas veces preguntas directas a Jesús sobre Buda, pero siempre había recibido respuestas más o menos evasivas. Aquel día, en presencia de su hijo, el padre le hizo a Jesús una pregunta directa acerca de Buda, y recibió una respuesta directa. Gonod dijo: <<Me gustaría saber de verdad lo que piensas de Buda>>. Y Jesús contestó:

\par 
%\textsuperscript{(1466.4)}
\textsuperscript{132:7.4} <<Vuestro Buda fue mucho mejor que vuestro budismo. Buda fue un gran hombre e incluso un profeta para su pueblo, pero fue un profeta huérfano. Con esto quiero decir que perdió de vista muy pronto a su Padre espiritual, el Padre que está en los cielos. Su experiencia fue trágica. Intentó vivir y enseñar como mensajero de Dios, pero sin Dios. Buda dirigió su nave de salvación directamente hacia el puerto seguro, hasta la entrada de la ensenada de la salvación de los mortales, pero allí, a causa de unas cartas de navegación equivocadas, la buena nave encalló. Allí ha continuado durante muchas generaciones, inmóvil y casi desesperadamente varada. Y en este barco han permanecido muchos de vuestros compatriotas todos estos años. Viven a un tiro de piedra de las aguas seguras de la ensenada, pero se niegan a entrar porque la noble embarcación del buen Buda tuvo la desgracia de varar casi a la entrada del puerto. Los pueblos budistas nunca entrarán en esta ensenada a menos que abandonen la embarcación filosófica de su profeta y se agarren a su noble espíritu. Si vuestro pueblo hubiera permanecido fiel al espíritu de Buda, hace mucho tiempo que hubierais entrado en vuestro puerto de la tranquilidad de espíritu, del descanso del alma y de la seguridad de la salvación>>.

\par 
%\textsuperscript{(1467.1)}
\textsuperscript{132:7.5} <<Ya ves, Gonod, Buda conocía a Dios en espíritu, pero no logró descubrirlo claramente en su mente; los judíos descubrieron a Dios en la mente, pero olvidaron ampliamente conocerlo en espíritu. Hoy, los budistas chapotean en una filosofía sin Dios, mientras que mi pueblo está lastimosamente encadenado al temor de un Dios sin una filosofía salvadora de vida y de libertad. Vosotros tenéis una filosofía sin Dios; los judíos tienen un Dios, pero carecen ampliamente de una filosofía de vida que esté en relación con ello. Al no tener una visión de Dios como espíritu y como Padre, Buda no consiguió proporcionar en su enseñanza la energía moral y la fuerza motriz espiritual que debe poseer una religión para cambiar a una raza y elevar a una nación>>.

\par 
%\textsuperscript{(1467.2)}
\textsuperscript{132:7.6} Entonces Ganid exclamó: <<Maestro, elaboremos tú y yo una nueva religión, que sea lo bastante buena para la India y lo bastante grande para Roma, y quizás podamos ofrecérsela a los judíos a cambio de Yahvé>>. Jesús replicó: <<Ganid, las religiones no se elaboran. Las religiones de los hombres se desarrollan durante largos períodos de tiempo, mientras que las revelaciones de Dios brillan sobre la Tierra en la vida de los hombres que revelan a Dios a sus semejantes>>. Pero Gonod y Ganid no comprendieron el significado de estas palabras proféticas.

\par 
%\textsuperscript{(1467.3)}
\textsuperscript{132:7.7} Aquella noche, después de acostarse, Ganid no pudo dormir. Estuvo hablando mucho tiempo con su padre y finalmente le dijo, <<Sabes, padre, a veces pienso que Josué es un profeta>>. Su padre le respondió solamente, con tono somnoliento: <<Hijo mío, hay otros...>>.

\par 
%\textsuperscript{(1467.4)}
\textsuperscript{132:7.8} A partir de este día, y durante el resto de su vida terrestre, Ganid continuó desarrollando una religión propia. Mentalmente, se sentía poderosamente incitado por la amplitud de miras, la equidad y la tolerancia de Jesús. En todas sus conversaciones sobre filosofía y religión, este joven nunca experimentó resentimientos ni reacciones de antagonismo.

\par 
%\textsuperscript{(1467.5)}
\textsuperscript{132:7.9} ¡Qué escena para ser contemplada por las inteligencias celestiales, la de este espectáculo del joven indio proponiéndole al Creador de un universo que elaboraran una nueva religión! Aunque el joven no lo sabía, en aquel momento y lugar estaban elaborando una religión nueva y eterna ---un nuevo camino de salvación, la revelación de Dios al hombre a través de Jesús y en Jesús. Lo que el joven más deseaba hacer en el mundo, lo estaba haciendo inconscientemente en ese momento. Siempre fue y siempre es así. Aquello que una imaginación humana iluminada y reflexiva, instruida y guiada por el espíritu, desea ser y hacer desinteresadamente y de todo corazón, se vuelve sensiblemente creativo según el grado en que el mortal esté consagrado a hacer divinamente la voluntad del Padre. Cuando el hombre se asocia con Dios, grandes cosas pueden suceder, y de hecho suceden.


\chapter{Documento 133. El regreso de Roma}
\par 
%\textsuperscript{(1468.1)}
\textsuperscript{133:0.1} AL prepararse para dejar Roma, Jesús no se despidió de ninguno de sus amigos. El escriba de Damasco apareció en Roma sin anunciarse y desapareció de la misma manera. Tuvo que transcurrir un año entero para que los que lo conocían y lo amaban renunciaran a la esperanza de volverlo a ver. Antes del final del segundo año, pequeños grupos de los que lo habían conocido empezaron a juntarse debido a su interés común por sus enseñanzas y a los recuerdos mutuos de los buenos momentos pasados con él. Estos pequeños grupos de estoicos, cínicos y miembros de los cultos de misterio continuaron manteniendo estas reuniones irregulares e informales hasta el mismo momento en que los primeros predicadores de la religión cristiana aparecieron en Roma.

\par 
%\textsuperscript{(1468.2)}
\textsuperscript{133:0.2} Gonod y Ganid habían comprado tantas cosas en Alejandría y Roma que enviaron de antemano todas sus pertenencias a Tarento en una caravana de animales de carga, mientras que los tres viajeros caminaban cómodamente a través de Italia por la gran vía Apia. Durante este viaje se encontraron con toda clase de seres humanos. Muchos nobles ciudadanos romanos y colonos griegos vivían a lo largo de esta ruta, pero los descendientes de un gran número de esclavos inferiores ya empezaban a hacer su aparición.

\par 
%\textsuperscript{(1468.3)}
\textsuperscript{133:0.3} Un día mientras descansaban para almorzar, aproximadamente a medio camino de Tarento, Ganid le hizo a Jesús una pregunta directa para saber lo que pensaba del sistema de castas de la India. Jesús contestó: <<Aunque los seres humanos difieren unos de otros de muchas maneras, todos los mortales están en igualdad de condiciones ante Dios y el mundo espiritual. A los ojos de Dios sólo existen dos grupos de mortales: los que desean hacer su voluntad y los que no lo desean. Cuando el universo contempla un mundo habitado, discierne igualmente dos grandes clases: los que conocen a Dios y los que no lo conocen. Los que no pueden conocer a Dios son contados entre los animales de un mundo determinado. Los seres humanos se pueden dividir propiamente en muchas categorías según requisitos diferentes, pues se les puede considerar desde un punto de vista físico, mental, social, profesional o moral, pero cuando estas diferentes clases de mortales comparecen ante el tribunal de Dios, se presentan en igualdad de condiciones. En verdad Dios no hace acepción de personas. Aunque no se puede evitar reconocer las diferencias de aptitudes y dotaciones humanas en los terrenos intelectual, social y moral, no habría que hacer ninguna distinción de este tipo en la fraternidad espiritual de los hombres cuando se reúnen para adorar en la presencia de Dios>>\footnote{\textit{Dios no hace acepción de personas}: 2 Cr 19:7; Job 34:19; Eclo 35:12; Hch 10:34; Ro 2:11; Gl 2:6; 3:28; Ef 6:9; Col 3:11.}.

\section*{1. La misericordia y la justicia}
\par 
%\textsuperscript{(1468.4)}
\textsuperscript{133:1.1} Una tarde se produjo un incidente muy interesante al borde de la carretera, cuando se acercaban a Tarento. Observaron que un joven tosco y fanfarrón estaba atacando brutalmente a un muchacho más pequeño. Jesús se apresuró a socorrer al joven agredido, y una vez que lo hubo rescatado, mantuvo firmemente al agresor hasta que el muchacho más pequeño pudo huir. En cuanto Jesús soltó al pequeño peleón, Ganid se abalanzó sobre el muchacho y empezó a darle una buena paliza; ante el asombro de Ganid, Jesús intervino inmediatamente. Refrenó a Ganid y permitió que el asustado muchacho saliera huyendo. Tan pronto como recobró el aliento, Ganid exclamó con agitación: <<Maestro, no consigo comprenderte. Si la misericordia requiere que rescates al muchacho más pequeño, ¿no exige la justicia que se castigue al agresor más grande?>>. Jesús le respondió:

\par 
%\textsuperscript{(1469.1)}
\textsuperscript{133:1.2} <<Ganid, es bien cierto que no comprendes. El ministerio de la misericordia es siempre un trabajo individual, pero el castigo de la justicia es una función de los grupos administrativos de la sociedad, del gobierno o del universo. Como individuo estoy obligado a mostrar misericordia; tenía que ir a rescatar al muchacho agredido, y con toda lógica, debía emplear la fuerza suficiente para contener al agresor. Eso es exactamente lo que he hecho. He logrado liberar al muchacho agredido y ahí termina el ministerio de la misericordia. Luego he retenido por la fuerza al agresor el tiempo necesario para permitir que la parte más débil de la disputa pudiera huir, después de lo cual me he retirado del asunto. No me he puesto a juzgar al agresor, examinando sus motivos ---determinando todos los factores que entraban en juego en el ataque a su semejante--- para luego proceder a infligir el castigo que mi mente pudiera dictar como justa retribución por su mala acción. Ganid, la misericordia puede ser pródiga, pero la justicia es precisa. ¿No te das cuenta de que no hay dos personas que se pongan de acuerdo sobre el castigo que daría satisfacción a las exigencias de la justicia? Una querría imponer cuarenta latigazos, otra veinte, mientras que una tercera recomendaría la celda de aislamiento como justo castigo. ¿No puedes ver que en este mundo es mejor que tales responsabilidades recaigan sobre la colectividad, o sean administradas por los representantes escogidos de esa colectividad? En el universo, el acto de juzgar está a cargo de aquellos que conocen plenamente los antecedentes de todas las malas acciones, así como sus motivos. En una sociedad civilizada y en un universo organizado, la administración de la justicia presupone el pronunciamiento de una sentencia justa después de un juicio equitativo, y estas prerrogativas corresponden a los cuerpos judiciales de los mundos y a los administradores omniscientes de los universos superiores de toda la creación>>.

\par 
%\textsuperscript{(1469.2)}
\textsuperscript{133:1.3} Durante varios días conversaron sobre este problema de manifestar misericordia y de administrar justicia. Ganid comprendió, al menos en cierta medida, por qué Jesús se negaba a participar en las peleas personales. Pero Ganid hizo una última pregunta, a la que nunca recibió una respuesta plenamente satisfactoria; esta pregunta fue: <<Pero, Maestro, si una criatura de mal carácter y más fuerte te atacara y amenazara con destruirte, ¿qué harías? ¿No harías ningún esfuerzo por defenderte?>> Jesús no podía responder de una manera completa y satisfactoria a la pregunta del muchacho, puesto que no quería revelarle que él (Jesús) estaba viviendo en la Tierra para dar ejemplo del amor del Padre Paradisiaco a un universo que lo contemplaba. Sin embargo le dijo lo siguiente:

\par 
%\textsuperscript{(1469.3)}
\textsuperscript{133:1.4} <<Ganid, comprendo muy bien que algunos de estos problemas te dejen perplejo, y voy a procurar contestar a tu pregunta. Ante cualquier ataque que se pudiera hacer contra mi persona, primero determinaría si el agresor es o no un hijo de Dios ---mi hermano en la carne. Si yo estimara que esa criatura no posee juicio moral ni razón espiritual, me defendería sin vacilar hasta el límite de mi fuerza de resistencia, sin preocuparme por las consecuencias para el agresor. Pero no me comportaría así con un semejante que tuviera la condición de la filiación, ni siquiera en defensa propia. Es decir, no lo castigaría de antemano y sin juicio por haberme atacado. Mediante todas las estratagemas posibles, trataría de impedir y de disuadirlo de que lanzara su ataque, y de mitigarlo en caso de que no consiguiera abortarlo. Ganid, tengo una confianza absoluta en la protección de mi Padre celestial. Estoy consagrado a hacer la voluntad de mi Padre que está en los cielos. No creo que pueda sucederme ningún daño \textit{real;} no creo que la obra de mi vida pueda ser puesta en peligro realmente por cualquier cosa que mis enemigos pudieran desear hacerme, y es seguro que no tenemos que temer ninguna violencia por parte de nuestros amigos. Estoy absolutamente convencido de que el universo entero es amistoso conmigo ---insisto en creer en esta verdad todopoderosa con una confianza total, a pesar de todas las apariencias en contra>>.

\par 
%\textsuperscript{(1470.1)}
\textsuperscript{133:1.5} Pero Ganid no estaba satisfecho por completo. Conversaron muchas veces sobre estos temas, y Jesús le contó algunas de sus experiencias infantiles; le habló también de Jacobo, el hijo del albañil. Al enterarse de cómo Jacobo se había erigido a sí mismo en defensor de Jesús, Ganid dijo: <<¡Oh, empiezo a comprender! En primer lugar, sería muy raro que un ser humano normal quisiera atacar a una persona tan bondadosa como tú, e incluso si alguien fuera tan irreflexivo como para hacerlo, es casi seguro que algún otro mortal estaría a la mano para acudir en tu ayuda, como tú mismo te apresuras siempre a socorrer a cualquier persona que se encuentra en apuros. Maestro, estoy de acuerdo contigo en mi corazón, pero en mi cabeza continúo pensando que si yo hubiera sido Jacobo, hubiera disfrutado castigando a aquellos brutos que se atrevían a atacarte sólo porque pensaban que no te defenderías. Supongo que viajas con bastante seguridad a través de la vida, puesto que pasas mucho tiempo ayudando a otros y socorriendo a tus semejantes en apuros --- así pues, es muy probable que siempre haya alguien al alcance de la mano para defenderte>>. Y Jesús replicó: <<Esa prueba aún no ha llegado, Ganid, y cuando llegue, deberemos atenernos a la voluntad del Padre>>. Esto fue casi todo lo que el muchacho pudo sacarle a su maestro sobre el difícil tema de la defensa propia y de la no resistencia. En otra ocasión consiguió arrancarle a Jesús la opinión de que la sociedad organizada tenía todo el derecho a emplear la fuerza para hacer que se ejecuten sus justos mandatos.

\section*{2. El embarque en Tarento}
\par 
%\textsuperscript{(1470.2)}
\textsuperscript{133:2.1} Mientras que se demoraban en el embarcadero esperando que el barco descargara, los viajeros observaron a un hombre que estaba maltratando a su mujer. Como era su costumbre, Jesús intervino a favor de la persona agredida. Se acercó por detrás del marido enfurecido, y dándole una suave palmadita en el hombro, le dijo: <<Amigo mío, ¿puedo hablar contigo a solas un momento?>> El hombre irritado se quedó desconcertado por esta intervención, y después de un momento de vacilación embarazosa, balbuceó: <<¿Eh...por qué...sí, ¿qué quieres de mí?>> Jesús lo llevó aparte y le dijo: <<Amigo mío, supongo que ha debido sucederte algo terrible; tengo muchísimo deseo de que me cuentes qué le ha podido suceder a un hombre fuerte como tú para inducirle a atacar a su mujer, la madre de sus hijos, y además aquí a la vista de todo el mundo. Estoy seguro de que tienes la sensación de poseer alguna buena razón para esta agresión. ¿Qué ha hecho tu mujer para merecer un trato semejante por parte de su marido? Al observarte, creo discernir en tu rostro el amor por la justicia, si no el deseo de mostrar misericordia. Me aventuro a decir que, si me encontraras a un lado del camino, atacado por unos ladrones, te abalanzarías sin titubeos para socorrerme. Me atrevo a decir que has realizado muchas de estas acciones valientes en el transcurso de tu vida. Ahora, amigo mío, dime de qué se trata. ¿Ha hecho tu mujer algo malo, o has perdido tontamente la cabeza y la has agredido sin reflexionar?>> El corazón de este hombre se sintió conmovido, no tanto por las palabras de Jesús como por la mirada bondadosa y la simpática sonrisa que éste le ofreció al concluir sus observaciones. El hombre dijo: <<Veo que eres un sacerdote de los cínicos, y te agradezco que me hayas refrenado. Mi mujer no ha hecho nada realmente malo; es una buena mujer, pero me irrita por la manera que tiene de buscar camorra en público, y pierdo mi sangre fría. Lamento mi falta de autocontrol y prometo tratar de vivir de acuerdo con la antigua promesa que le hice a uno de tus hermanos, que me enseñó el mejor camino hace muchos años. Te lo prometo>>.

\par 
%\textsuperscript{(1471.1)}
\textsuperscript{133:2.2} Entonces, al decirle adiós, Jesús añadió: <<Hermano mío, recuerda siempre que el hombre no tiene ninguna autoridad legítima sobre la mujer, a menos que la mujer le haya dado de buena gana y voluntariamente esa autoridad. Tu esposa se ha comprometido a atravesar la vida contigo, a ayudarte en las luchas que comporta y a asumir la mayor parte de la carga consistente en dar a luz y criar a tus hijos; a cambio de este servicio especial, es simplemente equitativo que reciba de ti esa protección especial que el hombre puede dar a la mujer como a la compañera que tiene que llevar dentro de sí, dar a luz y alimentar a los hijos. La consideración y los cuidados afectuosos que un hombre está dispuesto a conceder a su esposa y a sus hijos, indican la medida en que ese hombre ha alcanzado los niveles superiores de la conciencia espiritual y creativa. ¿No sabes que los hombres y las mujeres están asociados con Dios, en el sentido de que cooperan para crear seres que crecen hasta poseer el potencial de almas inmortales? El Padre que está en los cielos trata como a un igual al Espíritu Madre de los hijos del universo. Es parecerse a Dios compartir tu vida y todo lo relacionado con ella en términos de igualdad con la compañera y madre que comparte contigo tan plenamente esa experiencia divina de reproduciros en las vidas de vuestros hijos. Si puedes amar a tus hijos como Dios te ama a ti, amarás y apreciarás a tu esposa como el Padre que está en los cielos honra y exalta al Espíritu Infinito, la madre de todos los hijos espirituales de un vasto universo>>.

\par 
%\textsuperscript{(1471.2)}
\textsuperscript{133:2.3} Al subir a bordo del barco, se volvieron para contemplar la escena de la pareja que, con lágrimas en los ojos, permanecía abrazada en silencio. Habiendo oído la última parte del mensaje de Jesús a aquel hombre, Gonod se pasó todo el día meditando en el tema, y decidió reorganizar su hogar cuando regresara a la India.

\par 
%\textsuperscript{(1471.3)}
\textsuperscript{133:2.4} El viaje hasta Nicópolis fue agradable pero lento, porque el viento no era favorable. Los tres pasaron muchas horas reviviendo sus experiencias en Roma y recordando todo lo que les había sucedido desde que se conocieron por primera vez en Jerusalén. Ganid se iba impregnando con el espíritu del ministerio personal. Empezó a ejercerlo con el despensero del barco, pero al segundo día, cuando se metió en las aguas profundas de la religión, llamó a Josué para que le echara una mano.

\par 
%\textsuperscript{(1471.4)}
\textsuperscript{133:2.5} Pasaron varios días en Nicópolis, la ciudad que Augusto había fundado unos cincuenta años antes como <<ciudad de la victoria>>, en conmemoración de la batalla de Actium, pues en este lugar había acampado con su ejército antes de la batalla. Se alojaron en la casa de un tal Jerami, un prosélito griego de la fe judía, a quien habían conocido a bordo del barco. El apóstol Pablo pasó todo el invierno con el hijo de Jerami en esta misma casa, en el transcurso de su tercer viaje misionero\footnote{\textit{Pablo invernando en Nicópolis}: Tit 3:12.}. Desde Nicópolis navegaron en el mismo barco hasta Corinto, la capital de la provincia romana de Acaya.

\section*{3. En Corinto}
\par 
%\textsuperscript{(1471.5)}
\textsuperscript{133:3.1} Por la época en que llegaron a Corinto, Ganid empezaba a interesarse mucho por la religión judía, así que no es extraño que al pasar un día por delante de la sinagoga y ver a la gente que entraba, le pidiera a Jesús que lo llevara al oficio. Aquel día escucharon a un rabino erudito discurrir sobre el <<Destino de Israel>>, y después del servicio religioso conocieron a un tal Crispo\footnote{\textit{Crispo}: Hch 18:8; 1 Co 1:14.}, el jefe principal de esta sinagoga. Regresaron muchas veces a los oficios de la sinagoga, pero principalmente para encontrarse con Crispo. Ganid le tomó un gran afecto a Crispo, a su mujer y a su familia de cinco hijos. Disfrutó mucho observando cómo un judío dirigía su vida familiar.

\par 
%\textsuperscript{(1472.1)}
\textsuperscript{133:3.2} Mientras que Ganid estudiaba la vida de familia, Jesús enseñaba a Crispo los mejores caminos de la vida religiosa. Jesús tuvo más de veinte reuniones con este judío progresista. Años más tarde, Pablo predicó en esta misma sinagoga, los judíos rechazaron su mensaje y votaron la prohibición de que continuara predicando en la sinagoga; entonces Pablo se dirigió hacia los gentiles, y no es sorprendente que Crispo y toda su familia abrazaran la nueva religión, convirtiéndose en uno de los pilares principales de la iglesia cristiana que Pablo organizó posteriormente en Corinto.

\par 
%\textsuperscript{(1472.2)}
\textsuperscript{133:3.3} Durante los dieciocho meses que Pablo predicó en Corinto, donde Silas y Timoteo\footnote{\textit{Silas y Timoteo}: Hch 18:5.} se reunieron con él más tarde, encontró a otras muchas personas que habían sido instruidas por <<el preceptor judío del hijo de un mercader indio>>.

\par 
%\textsuperscript{(1472.3)}
\textsuperscript{133:3.4} En Corinto se encontraron con gentes de todas las razas, procedentes de tres continentes. Después de Alejandría y Roma, ésta era la ciudad más cosmopolita del imperio mediterráneo. En esta ciudad había muchas cosas atractivas que ver, y Ganid nunca se cansó de visitar la ciudadela que se alzaba casi a seiscientos metros por encima del nivel del mar. También pasó una gran parte de su tiempo libre entre la sinagoga y la casa de Crispo. Al principio le escandalizó, y más tarde le encantó, la condición de la mujer en los hogares judíos; fue una revelación para este joven indio.

\par 
%\textsuperscript{(1472.4)}
\textsuperscript{133:3.5} Jesús y Ganid fueron a menudo los huéspedes de otro hogar judío, el de Justo\footnote{\textit{Justo}: Hch 18:7; Col 4:11.}, un piadoso mercader que vivía al lado de la sinagoga. Posteriormente, cuando el apóstol Pablo residió en esta casa, escuchó muchas veces el relato de estas visitas del muchacho indio y de su preceptor judío, y tanto Pablo como Justo se preguntaban qué habría sido de aquel sabio y brillante educador hebreo.

\par 
%\textsuperscript{(1472.5)}
\textsuperscript{133:3.6} Cuando estaban en Roma, Ganid había observado que Jesús rehusaba acompañarlos a los baños públicos. Después de aquello, el joven trató varias veces de persuadir a Jesús para que se explicara más ampliamente respecto a las relaciones entre los sexos. Aunque contestaba a las preguntas del muchacho, nunca parecía dispuesto a extenderse acerca de estos asuntos. Una noche, mientras paseaban por Corinto cerca del lugar donde la muralla de la ciudadela descendía hasta el mar, fueron abordados por dos mujeres públicas. Ganid estaba impregnado con la idea, por otra parte cierta, de que Jesús era un hombre de altos ideales, que aborrecía todo lo que sonara a impureza o tuviera sabor a mal; en consecuencia, se dirigió con sequedad a estas mujeres, indicándoles groseramente que se alejaran. Al ver esto, Jesús dijo a Ganid: <<Tienes buenas intenciones, pero no deberías atreverte a hablarle así a las hijas de Dios, aunque se trate de sus hijas desviadas. ¿Quiénes somos nosotros para juzgar a estas mujeres? ¿Acaso conoces todas las circunstancias que las han llevado a recurrir a estos métodos para ganarse la vida? Quédate aquí conmigo mientras hablamos de estas cosas>>. Al escuchar estas palabras, las prostitutas se quedaron aún más asombradas que Ganid.

\par 
%\textsuperscript{(1472.6)}
\textsuperscript{133:3.7} Mientras permanecían allí de pie, a la luz de la Luna, Jesús continuó diciendo: <<Dentro de cada mente humana vive un espíritu divino, el don del Padre que está en los cielos. Este buen espíritu se esfuerza continuamente por conducirnos a Dios, por ayudarnos a encontrar a Dios y a conocer a Dios. Pero dentro de los mortales existen también muchas tendencias físicas naturales que el Creador ha puesto allí para servir al bienestar del individuo y de la raza. Ahora bien, los hombres y las mujeres se desconciertan muchas veces al esforzarse por comprenderse a sí mismos y luchar con las múltiples dificultades que encuentran para ganarse la vida en un mundo tan ampliamente dominado por el egoísmo y el pecado. Ganid, percibo que ninguna de estas mujeres es voluntariamente mala. Puedo decir, por la expresión de sus rostros, que han padecido muchas penas; han sufrido mucho a manos de un destino aparentemente cruel; no han elegido intencionalmente este tipo de vida. En un desaliento que rozaba la desesperación, han sucumbido a la presión del momento y han aceptado esta manera desagradable de ganarse la vida como el mejor camino para salir de una situación que les parecía desesperada. Ganid, algunas personas son realmente perversas en su corazón, y escogen deliberadamente hacer cosas despreciables. Pero dime, al observar estos rostros ahora llenos de lágrimas, ¿ves algo malo o perverso?>> Mientras que Jesús esperaba su contestación, la voz de Ganid se ahogó al balbucear su respuesta: <<No, Maestro, no veo nada de eso, y me disculpo por mi grosería hacia ellas ---les ruego que me perdonen>>. Entonces dijo Jesús: <<Y yo te digo, en su nombre, que te han perdonado, como digo en nombre de mi Padre que está en los cielos que él las ha perdonado. Ahora venid todos conmigo a la casa de un amigo, donde recobraremos nuestras fuerzas y haremos planes para la vida nueva y mejor que está ante nosotros>>. Hasta ese momento, las asombradas mujeres no habían pronunciado una sola palabra; se miraron entre sí y siguieron silenciosamente a los hombres que mostraban el camino.

\par 
%\textsuperscript{(1473.1)}
\textsuperscript{133:3.8} Imagináos la sorpresa de la mujer de Justo cuando, a esta hora tardía, Jesús apareció con Ganid y estas dos extrañas, diciendo: <<Perdónanos por llegar a esta hora, pero Ganid y yo deseamos tomar un bocado, y quisiéramos compartirlo con estas nuevas amigas, que también necesitan alimentarse. Además de eso, venimos hacia ti con la idea de que estarás interesada en deliberar con nosotros sobre la mejor manera de ayudar a estas mujeres a emprender una nueva vida. Ellas pueden contarte su historia, pero supongo que han tenido muchas dificultades, y su misma presencia aquí en tu casa demuestra cuán seriamente desean conocer a gente de bien, y con cuánto placer aprovecharán la oportunidad de mostrarle a todo el mundo ---e incluso a los ángeles del cielo--- la clase de mujeres nobles y valientes que pueden llegar a ser>>.

\par 
%\textsuperscript{(1473.2)}
\textsuperscript{133:3.9} Cuando Marta, la esposa de Justo, hubo servido la comida en la mesa, Jesús se despidió de manera inesperada diciendo: <<Como se hace tarde y el padre del joven estará esperándonos, rogamos nos disculpen mientras os dejamos aquí juntas ---a tres mujeres--- las hijas amadas del Altísimo. Rogaré por vuestra orientación espiritual, mientras hacéis planes para una vida nueva y mejor en la Tierra y para la vida eterna en el gran más allá>>.

\par 
%\textsuperscript{(1473.3)}
\textsuperscript{133:3.10} Jesús y Ganid se despidieron así de las mujeres. Hasta ese momento, las dos prostitutas no habían dicho nada, y Ganid se quedó igualmente sin habla. A Marta le sucedió lo mismo durante unos instantes, pero pronto se puso a la altura de las circunstancias, e hizo por aquellas extrañas todo lo que Jesús había esperado. La mayor de las dos mujeres murió poco tiempo después, con brillantes esperanzas de supervivencia eterna; la más joven trabajó en el negocio de Justo, y más tarde se hizo miembro de por vida de la primera iglesia cristiana de Corinto.

\par 
%\textsuperscript{(1473.4)}
\textsuperscript{133:3.11} En la casa de Crispo, Jesús y Ganid se encontraron varias veces con un tal Gayo\footnote{\textit{Gayo}: Hch 19:29; 20:4; Ro 16:23; 1 Co 1:14; 3 Jn 1:1.}, que se convirtió posteriormente en un leal partidario de Pablo. Durante estos dos meses en Corinto, mantuvieron conversaciones íntimas con decenas de personas dignas de interés, y como resultado de estos contactos aparentemente casuales, más de la mitad de estas personas se hicieron miembros de la comunidad cristiana posterior.

\par 
%\textsuperscript{(1473.5)}
\textsuperscript{133:3.12} Cuando Pablo fue por primera vez a Corinto, no tenía la intención de quedarse mucho tiempo, pero no sabía hasta qué punto el preceptor judío había preparado bien el terreno para sus trabajos. Descubrió además que Aquila y Priscila\footnote{\textit{Aquila y Priscila}: Hch 18:2,18,26; Ro 16:3; 1 Co 16:19; 2 Ti 4:19.} ya habían despertado un gran interés por su doctrina. Aquila era uno de los cínicos con los que Jesús había entrado en contacto cuando estuvo en Roma. Esta pareja eran refugiados judíos de Roma, y aceptaron rápidamente las enseñanzas de Pablo, que vivió y trabajó con ellos, porque eran también fabricantes de tiendas. Fue debido a estas circunstancias por lo que Pablo prolongó su estancia en Corinto.

\section*{4. Trabajo personal en Corinto}
\par 
%\textsuperscript{(1474.1)}
\textsuperscript{133:4.1} Jesús y Ganid tuvieron otras muchas experiencias interesantes en Corinto. Tuvieron estrechas conversaciones con un gran número de personas, que se beneficiaron mucho de las instrucciones de Jesús.

\par 
%\textsuperscript{(1474.2)}
\textsuperscript{133:4.2} A un molinero le enseñó a moler los granos de la verdad en el molino de la experiencia viviente, para hacer que las cosas difíciles de la vida divina fueran fácilmente aceptables incluso por aquellos compañeros mortales que son frágiles y débiles. Jesús dijo: <<Da la leche de la verdad a aquellos que están en la infancia de la percepción espiritual. En tu ministerio viviente y amante, sirve el alimento espiritual de una manera atractiva y adaptada a la capacidad de recepción de cada uno de los que te pregunten>>\footnote{\textit{La leche de la verdad espiritual}: 1 Co 3:1-2; 1 P 2:2.}.

\par 
%\textsuperscript{(1474.3)}
\textsuperscript{133:4.3} Al centurión romano le dijo: <<Da al César lo que es del César y a Dios lo que es de Dios. No existe conflicto entre el sincero servicio de Dios y el leal servicio del César, a menos que el César se atreva a reclamar el homenaje que sólo puede ser reivindicado por la Deidad. La lealtad a Dios, si llegas a conocerlo, te hará aún más leal y fiel en tu devoción a un emperador digno>>\footnote{\textit{Da al César lo que es del César y a Dios lo que es de Dios}: Mt 22:21; Mc 12:17; Ro 13:7.}.

\par 
%\textsuperscript{(1474.4)}
\textsuperscript{133:4.4} Al jefe sincero del culto mitríaco le dijo: <<Haces bien en buscar una religión de salvación eterna, pero te equivocas al buscar esa gloriosa verdad entre los misterios elaborados por los hombres y en las filosofías humanas. ¿No sabes que el misterio de la salvación eterna reside dentro de tu propia alma? ¿No sabes que el Dios del cielo ha enviado a su espíritu para que viva dentro de ti, y que todos los mortales que aman la verdad y que sirven a Dios serán conducidos por este espíritu más allá de esta vida, a través de las puertas de la muerte, hasta las alturas eternas de la luz, donde Dios aguarda para recibir a sus hijos? Y no olvides nunca que vosotros, los que conocéis a Dios, sois los hijos de Dios si anheláis realmente pareceros a él>>\footnote{\textit{Dios habita en tu alma}: Job 32:8,18; Is 63:10-11; Ez 37:14; Mt 10:20; Lc 17:21; Jn 17:21-23; Ro 8:9-11; 1 Co 3:16-17; 6:19; 2 Co 6:16; Gl 2:20; 1 Jn 3:24; 4:12-15; Ap 21:3. \textit{Los hijos anhelan parecerse a Dios}: Sal 82:6; Jn 10:34-35.}.

\par 
%\textsuperscript{(1474.5)}
\textsuperscript{133:4.5} Al maestro epicúreo le dijo: <<Haces bien en elegir lo mejor y en apreciar lo bueno, pero ¿eres sabio cuando dejas de discernir las grandes cosas de la vida mortal que están incorporadas en los reinos del espíritu derivados de la conciencia de la presencia de Dios en el corazón humano? En toda experiencia humana, la cosa importante es la conciencia de conocer al Dios cuyo espíritu vive dentro de ti y trata de mostrarte el camino en el largo y casi interminable viaje para alcanzar la presencia personal de nuestro Padre común, el Dios de toda la creación, el Señor de los universos>>\footnote{\textit{El espíritu de Dios vive en nosotros}: Job 32:8,18; Is 63:10-11; Ez 37:14; Mt 10:20; Lc 17:21; Jn 17:21-23; Ro 8:9-11; 1 Co 3:16-17; 6:19; 2 Co 6:16; Gl 2:20; 1 Jn 3:24; 4:12-15; Ap 21:3.}.

\par 
%\textsuperscript{(1474.6)}
\textsuperscript{133:4.6} Al contratista y constructor griego le dijo: <<Amigo mío, al mismo tiempo que construyes los edificios materiales de los hombres, desarrolla un carácter espiritual a semejanza del espíritu divino interior de tu alma. No dejes que tus éxitos como constructor temporal sobrepasen a tus realizaciones como hijo espiritual del reino de los cielos. Mientras construyes las mansiones del tiempo para otros, no descuides asegurarte tu propio derecho a las mansiones de la eternidad. Recuerda siempre que existe una ciudad cuyos fundamentos son la rectitud y la verdad, y cuyo constructor y hacedor es Dios>>\footnote{\textit{Muchas mansiones}: Jn 14:2. \textit{La ciudad de Dios}: Heb 11:10. \textit{Lo temporal contra lo espiritual}: Mt 6:19-20; Lc 12:21; 1 Ti 6:17; 1 Jn 2:15-17.}.

\par 
%\textsuperscript{(1474.7)}
\textsuperscript{133:4.7} Al juez romano le dijo: <<Cuando juzgues a los hombres, recuerda que tú mismo comparecerás también algún día ante el tribunal de los Soberanos de un universo. Juzga con justicia e incluso con misericordia, al igual que algún día desearás ardientemente la consideración misericordiosa de las manos del Arbitro Supremo. Juzga como te gustaría ser juzgado en circunstancias semejantes, y así estarás guiado tanto por el espíritu de la ley como por su letra. De la misma manera que otorgas una justicia dominada por la equidad a la luz de las necesidades de los que son traídos ante ti, igualmente tendrás derecho a esperar una justicia templada por la misericordia, cuando algún día comparezcas ante el Juez de toda la Tierra>>\footnote{\textit{Judga justamente}: Mt 7:2; Lc 6:38; Stg 2:12-13. \textit{Juicio de todos}: Gn 18:24-25. \textit{Justicia, equidad, misericordia}: 1 Re 8:30,34,36,39; Sal 32:1; Is 1:18-20; Mt 6:12,14-15; Mc 11:25-26; Lc 6:36-38; 11:4.}.

\par 
%\textsuperscript{(1475.1)}
\textsuperscript{133:4.8} A la dueña de la posada griega le dijo: <<Ofrece tu hospitalidad como alguien que recibe a los hijos del Altísimo. Eleva la faena ingrata de tu trabajo diario hasta los niveles elevados de un arte refinado, mediante la conciencia creciente de que sirves a Dios en las personas en las que él habita por medio de su espíritu, el cual ha descendido para vivir en el corazón de los hombres, intentando así transformar sus mentes y conducir sus almas al conocimiento del Padre Paradisiaco que ha otorgado todos estos dones del espíritu divino>>\footnote{\textit{Hijos del Altísimo}: Sal 82:6.}.

\par 
%\textsuperscript{(1475.2)}
\textsuperscript{133:4.9} Jesús tuvo numerosos encuentros con un mercader chino. Al despedirse de él, le hizo estas advertencias: <<Adora sólo a Dios, que es tu verdadero antepasado espiritual. Recuerda que el espíritu del Padre vive siempre dentro de ti y orienta constantemente tu alma en dirección al cielo. Si sigues las directrices inconscientes de este espíritu inmortal, estarás seguro de perseverar en el camino elevado que conduce a encontrar a Dios. Cuando logres alcanzar al Padre que está en los cielos, será porque al buscarlo te habrás vuelto cada vez más semejante a él. Así pues, adiós, Chang, pero sólo por un tiempo, porque nos encontraremos de nuevo en los mundos de luz, donde el Padre de las almas espirituales ha preparado numerosos lugares de detención encantadores para los que se dirigen hacia el Paraíso>>\footnote{\textit{Adora sólo a Dios}: Ex 20:3; Dt 5:7; Mt 4:10. \textit{Muchas mansiones}: Jn 14:2.}.

\par 
%\textsuperscript{(1475.3)}
\textsuperscript{133:4.10} Al viajero que venía de Bretaña le dijo: <<Hermano mío, percibo que estás buscando la verdad, y sugiero que el espíritu del Padre de toda verdad tal vez resida dentro de ti. ¿Has probado sinceramente alguna vez hablar con el espíritu de tu propia alma? La cosa es ciertamente difícil y es raro que produzca la conciencia del éxito. Pero cualquier intento honrado de la mente material por comunicarse con su espíritu interior alcanza cierto éxito, aunque la mayoría de estas magníficas experiencias humanas deben permanecer mucho tiempo como registros superconscientes en el alma de esos mortales que conocen a Dios>>.

\par 
%\textsuperscript{(1475.4)}
\textsuperscript{133:4.11} Al muchacho fugitivo Jesús le dijo: <<Recuerda que hay dos seres de quienes no puedes escapar: Dios y tú mismo. Dondequiera que vayas, te llevas a ti mismo y al espíritu del Padre celestial que vive dentro de tu corazón. Hijo mío, no trates más de engañarte; asiéntate en la práctica valiente de enfrentarte a los hechos de la vida; aférrate a la seguridad de la filiación con Dios y a la certeza de la vida eterna, como te lo he indicado. Desde hoy en adelante, propónte ser un verdadero hombre, un hombre decidido a afrontar la vida con valentía e inteligencia>>.

\par 
%\textsuperscript{(1475.5)}
\textsuperscript{133:4.12} Al criminal condenado le dijo en su última hora: <<Hermano mío, has pasado por malos tiempos. Te has extraviado; te has enredado en las mallas del crimen. Basándome en lo que he hablado contigo, sé muy bien que no habías planeado hacer lo que ahora está a punto de costarte la vida temporal. Pero cometiste esa mala acción y tus semejantes te han encontrado culpable; han decidido que debes morir. Ni tú ni yo podemos negarle al Estado el derecho a defenderse como le parezca apropiado. Parece que no hay manera de escapar humanamente al castigo de tu delito. Tus semejantes están obligados a juzgarte por lo que has hecho, pero existe un Juez a quien puedes apelar para ser perdonado, y que te juzgará por tus verdaderos móviles y tus mejores intenciones. No debes temer hacer frente al juicio de Dios, si tu arrepentimiento es auténtico y tu fe sincera. El hecho de que tu error lleve consigo la pena de muerte impuesta por los hombres, no afecta a la oportunidad que tiene tu alma de obtener justicia y de gozar de misericordia ante los tribunales celestiales>>.

\par 
%\textsuperscript{(1476.1)}
\textsuperscript{133:4.13} Jesús disfrutó de muchas conversaciones íntimas con un gran número de almas hambrientas, demasiado numerosas para ser incluidas en esta narración. Los tres viajeros disfrutaron de su estancia en Corinto. A excepción de Atenas, que era más famosa como centro de educación, Corinto era la ciudad más importante de Grecia en esta época romana. Su estancia de dos meses en este centro comercial floreciente proporcionó a los tres la oportunidad de adquirir una experiencia valiosísima. Su estancia en esta ciudad fue una de las escalas más interesantes en el camino de regreso de Roma.

\par 
%\textsuperscript{(1476.2)}
\textsuperscript{133:4.14} Gonod tenía muchos intereses en Corinto, pero finalmente terminó sus negocios y se prepararon para navegar hacia Atenas. Viajaron en un pequeño barco que podía ser transportado por tierra sobre un carril desde uno de los puertos de Corinto hasta el otro, a una distancia de dieciséis kilómetros.

\section*{5. En Atenas --- discurso sobre la ciencia}
\par 
%\textsuperscript{(1476.3)}
\textsuperscript{133:5.1} Llegaron poco después al antiguo centro de la ciencia y del saber griegos. Ganid estaba muy emocionado con la idea de encontrarse en Atenas, de estar en Grecia, en el centro cultural del antiguo imperio de Alejandro, que había extendido sus fronteras hasta su propio país de la India. Había pocos negocios que tratar, de manera que Gonod pasó la mayor parte de su tiempo con Jesús y Ganid, visitando los numerosos lugares de interés y escuchando las atractivas discusiones entre el muchacho y su hábil maestro.

\par 
%\textsuperscript{(1476.4)}
\textsuperscript{133:5.2} Una gran universidad florecía aún en Atenas, y el trío hizo frecuentes visitas a sus salas de enseñanza. Jesús y Ganid habían discutido a fondo las enseñanzas de Platón cuando asistieron a las conferencias en el museo de Alejandría. Todos disfrutaron del arte de Grecia, cuyos ejemplos aún podían encontrarse aquí y allá por toda la ciudad.

\par 
%\textsuperscript{(1476.5)}
\textsuperscript{133:5.3} Tanto el padre como el hijo disfrutaron mucho con la discusión sobre la ciencia que tuvo lugar una noche en la posada entre Jesús y un filósofo griego. Después de que aquel pedante se llevara hablando cerca de tres horas y hubo terminado su discurso, Jesús dijo ---en términos adaptados al pensamiento moderno:

\par 
%\textsuperscript{(1476.6)}
\textsuperscript{133:5.4} Algún día, los científicos podrán medir la energía o las manifestaciones de fuerza de la gravedad, de la luz y de la electricidad, pero estos mismos científicos nunca podrán decir (científicamente) qué \textit{son} estos fenómenos del universo. La ciencia trata de las actividades de la energía física; la religión trata de los valores eternos. La verdadera filosofía procede de la sabiduría, que hace todo lo que puede por correlacionar estas observaciones cuantitativas y cualitativas. Siempre existe el peligro de que el científico que se ocupa de lo puramente físico pueda llegar a sufrir de orgullo matemático y de egoísmo estadístico, sin mencionar la ceguera espiritual.

\par 
%\textsuperscript{(1476.7)}
\textsuperscript{133:5.5} La lógica es válida en el mundo material, y las matemáticas son fiables cuando su aplicación se limita a las cosas físicas; pero ninguna de las dos puede considerarse enteramente digna de confianza o infalible cuando se aplican a los problemas de la vida. La vida contiene fenómenos que no son totalmente materiales. La aritmética dice que si un hombre puede esquilar una oveja en diez minutos, diez hombres pueden hacerlo en un minuto. Es un cálculo exacto, pero no es cierto, porque los diez hombres no podrían hacerlo; se estorbarían tanto los unos a los otros que el trabajo se retrasaría considerablemente.

\par 
%\textsuperscript{(1477.1)}
\textsuperscript{133:5.6} Las matemáticas afirman que si una persona representa cierta unidad de valor intelectual y moral, diez personas representarían diez veces ese valor. Pero al tratar de la personalidad humana, sería más exacto decir que una asociación semejante de personalidades es igual al cuadrado del número de personalidades que figuran en la ecuación, en lugar de su simple suma aritmética. Un grupo social de seres humanos que trabaja en armonía coordinada representa una fuerza mucho más grande que la simple suma de sus componentes.

\par 
%\textsuperscript{(1477.2)}
\textsuperscript{133:5.7} La cantidad puede ser identificada como un \textit{hecho}, convirtiéndose así en una uniformidad científica. La calidad, como está sujeta a la interpretación de la mente, representa una estimación de \textit{valores}, y por lo tanto, debe permanecer como una experiencia del individuo. Cuando la ciencia y la religión sean menos dogmáticas y toleren mejor la crítica, la filosofía empezará entonces a conseguir la \textit{unidad} en la comprensión inteligente del universo.

\par 
%\textsuperscript{(1477.3)}
\textsuperscript{133:5.8} Hay unidad en el universo cósmico, si tan sólo pudierais discernir su funcionamiento en su estado actual. El universo real es amistoso para cada hijo del Dios eterno. El verdadero problema es: ¿Cómo puede conseguir la mente finita del hombre una unidad de pensamiento lógica, verdadera y proporcionada? Este estado mental de conocimiento del universo sólo se puede obtener concibiendo la idea de que los hechos cuantitativos y los valores cualitativos tienen una causación común: el Padre Paradisiaco. Una concepción así de la realidad permite una comprensión más amplia de la unidad intencional de los fenómenos del universo; revela incluso una meta espiritual que la personalidad alcanza de manera progresiva. Éste es un concepto de unidad que puede percibir el trasfondo inmutable de un universo viviente donde las relaciones impersonales cambian sin cesar y donde las relaciones personales evolucionan continuamente.

\par 
%\textsuperscript{(1477.4)}
\textsuperscript{133:5.9} La materia, el espíritu y el estado intermedio entre ambos, son tres niveles interrelacionados e interasociados de la verdadera unidad del universo real. Por muy divergentes que puedan parecer los fenómenos universales de los hechos y de los valores, a fin de cuentas están unificados en el Supremo.

\par 
%\textsuperscript{(1477.5)}
\textsuperscript{133:5.10} La realidad de la existencia material está vinculada a la energía no reconocida así como a la materia visible. Cuando las energías del universo son frenadas hasta el punto de adquirir el grado requerido de movimiento, entonces, en condiciones favorables, estas mismas energías se convierten en masa. Y no olvidéis que la mente, la única que puede percibir la presencia de las realidades aparentes, es también real. La causa fundamental de este universo de energía-masa, de mente y de espíritu, es eterna ---existe y consiste en la naturaleza y en las reacciones del Padre Universal y de sus coordinados absolutos.

\par 
%\textsuperscript{(1477.6)}
\textsuperscript{133:5.11} Todos estaban más que asombrados por las palabras de Jesús, y cuando el griego se despidió de ellos, dijo: <<Por fin mis ojos han visto a un judío que piensa en algo más que en la superioridad racial, y que habla de algo más que de religión>>. Y se retiraron para pasar la noche.

\par 
%\textsuperscript{(1477.7)}
\textsuperscript{133:5.12} La estancia en Atenas fue agradable y provechosa, pero no particularmente fructífera en contactos humanos. Demasiados atenienses de aquellos tiempos, o estaban intelectualmente orgullosos de su reputación del pasado, o eran mentalmente estúpidos e ignorantes, pues descendían de los esclavos inferiores traídos en épocas anteriores, cuando había gloria en Grecia y sabiduría en la mente de sus habitantes. Sin embargo, aún se podían encontrar muchas mentes agudas entre los ciudadanos de Atenas.

\section*{6. En Éfeso --- discurso sobre el alma}
\par 
%\textsuperscript{(1477.8)}
\textsuperscript{133:6.1} Al partir de Atenas, los viajeros fueron por el camino de Tróades hasta Éfeso, la capital de la provincia romana de Asia. Efectuaron muchas visitas al célebre templo de Artemisa de los Efesios, a unos tres kilómetros de la ciudad. Artemisa era la diosa más famosa de toda Asia Menor y una perpetuación de la diosa madre aún más antigua de la Anatolia de épocas anteriores. Se decía que el tosco ídolo que se exhibía en el enorme templo dedicado a su culto había caído del cielo. A Ganid se le había enseñado muy pronto a respetar las imágenes como símbolos de la divinidad; no toda esta educación había sido erradicada, y pensó que lo mejor sería comprar un pequeño relicario de plata en honor de esta diosa de la fertilidad de Asia Menor. Aquella noche hablaron largo y tendido sobre la adoración de los objetos hechos con las manos humanas.

\par 
%\textsuperscript{(1478.1)}
\textsuperscript{133:6.2} Durante el tercer día de su estancia, caminaron río abajo para observar el dragado del puerto en su desembocadura. A mediodía conversaron con un joven fenicio muy desanimado y con nostalgia de su país, pero que sobre todo sentía envidia de un joven a quien habían ascendido por encima de él. Jesús le dirigió palabras de aliento y citó el antiguo proverbio hebreo: <<El talento de un hombre es el que le asegura una posición y le lleva ante los grandes hombres>>\footnote{\textit{El talento es el que asegura la posición}: Pr 18:16.}.

\par 
%\textsuperscript{(1478.2)}
\textsuperscript{133:6.3} De todas las grandes ciudades que visitaron en este viaje por el Mediterráneo, fue aquí donde menos pudieron hacer a favor del trabajo posterior de los misioneros cristianos. El cristianismo se estableció inicialmente en Éfeso gracias, en gran medida, a los esfuerzos de Pablo, que residió aquí más de dos años, fabricando tiendas para ganarse la vida y dando conferencias cada noche sobre religión y filosofía en el salón principal de la escuela de Tirano\footnote{\textit{Comienzos del cristianismo en Éfeso}: Hch 19:1-10.}.

\par 
%\textsuperscript{(1478.3)}
\textsuperscript{133:6.4} Había un pensador progresista que tenía relación con esta escuela local de filosofía, y Jesús tuvo varias reuniones provechosas con él. En el transcurso de estas conversaciones, Jesús utilizó repetidas veces la palabra <<alma>>. Este griego erudito acabó por preguntarle qué entendía él por <<alma>>, y Jesús respondió:

\par 
%\textsuperscript{(1478.4)}
\textsuperscript{133:6.5} <<El alma es la parte del hombre que refleja su yo, discierne la verdad y percibe el espíritu, y que eleva para siempre al ser humano por encima del nivel del mundo animal. La conciencia de sí, en sí misma y por sí misma, no es el alma. La autoconciencia moral es la verdadera autorrealización humana y constituye el fundamento del alma humana. El alma es esa parte del hombre que representa el valor potencial de supervivencia de la experiencia humana. La elección moral y la consecución espiritual, la capacidad para conocer a Dios y el impulso de ser semejante a él, son las características del alma. El alma del hombre no puede existir sin pensamiento moral y sin actividad espiritual. Un alma estancada es un alma moribunda. Pero el alma del hombre es distinta al espíritu divino que reside dentro de la mente. El espíritu divino llega al mismo tiempo que la mente humana efectúa su primera actividad moral, y en esa ocasión es cuando nace el alma.>>

\par 
%\textsuperscript{(1478.5)}
\textsuperscript{133:6.6} <<La salvación o la pérdida de un alma dependen de que la conciencia moral alcance o no el estado de supervivencia mediante una alianza eterna con el espíritu inmortal asociado que le ha sido dado. La salvación es la espiritualización de la autorrealización de la conciencia moral, que adquiere de este modo un valor de supervivencia. Todos los tipos de conflictos del alma consisten en la falta de armonía entre la conciencia de sí moral o espiritual, y la conciencia de sí puramente intelectual.>>

\par 
%\textsuperscript{(1478.6)}
\textsuperscript{133:6.7} <<Cuando el alma humana está madura, ennoblecida y espiritualizada, se acerca al estado celestial en el sentido de que casi llega a ser una entidad intermedia entre lo material y lo espiritual, entre el yo material y el espíritu divino. El alma evolutiva de un ser humano es difícil de describir y aun más difícil de demostrar, porque no puede ser descubierta por el método de la investigación material ni por el de la prueba espiritual. La ciencia material no puede demostrar la existencia de un alma, y la prueba puramente espiritual tampoco. A pesar de que la ciencia material y los criterios espirituales no puedan descubrir la existencia del alma humana, todo mortal moralmente consciente \textit{conoce} la existencia de \textit{su} alma como una experiencia personal \textit{real} y efectiva>>.

\section*{7. La estancia en Chipre --- discurso sobre la mente}
\par 
%\textsuperscript{(1479.1)}
\textsuperscript{133:7.1} Poco después, los viajeros se hicieron a la vela para Chipre, con una escala en Rodas. Disfrutaron de este largo viaje marítimo y llegaron a su isla de destino con el cuerpo descansado y el espíritu renovado.

\par 
%\textsuperscript{(1479.2)}
\textsuperscript{133:7.2} Habían planeado disfrutar de un período de verdadero descanso y esparcimiento durante esta visita a Chipre, pues su gira por el Mediterráneo estaba llegando a su fin. Desembarcaron en Pafos y empezaron enseguida a reunir las provisiones para su estancia de varias semanas en las montañas cercanas. Al tercer día de su llegada, partieron hacia las colinas con sus animales bien cargados.

\par 
%\textsuperscript{(1479.3)}
\textsuperscript{133:7.3} El trío pasó quince días sumamente agradables, y luego, de repente, el joven Ganid cayó gravemente enfermo. Durante dos semanas padeció una fiebre intensa, que a menudo lo llevaba hasta el delirio; tanto Jesús como Gonod se dedicaron de lleno a cuidar al muchacho enfermo. Jesús se ocupó del chico con habilidad y ternura, y el padre se quedó asombrado por la delicadeza y la pericia que Jesús demostró en todos sus cuidados hacia el joven enfermo. Estaban lejos de toda morada humana, y el muchacho se encontraba demasiado enfermo como para ser trasladado; así pues, se prepararon lo mejor que pudieron para cuidarlo hasta que se recuperara allí mismo en las montañas.

\par 
%\textsuperscript{(1479.4)}
\textsuperscript{133:7.4} Durante las tres semanas de la convalecencia de Ganid, Jesús le contó muchas cosas interesantes sobre la naturaleza y sus diversas manifestaciones. Se divirtieron mucho mientras correteaban por las montañas, con el muchacho haciendo preguntas, Jesús respondiéndolas y el padre maravillándose con toda la escena.

\par 
%\textsuperscript{(1479.5)}
\textsuperscript{133:7.5} La última semana de su estancia en las montañas, Jesús y Ganid tuvieron una larga conversación sobre las funciones de la mente humana. Después de varias horas de discusión, el joven hizo la pregunta siguiente: <<Pero, Maestro, ¿qué quieres decir cuando afirmas que el hombre experimenta una forma de conciencia de sí más elevada que la que experimentan los animales más evolucionados?>> Transcrito en un lenguaje moderno, Jesús le contestó:

\par 
%\textsuperscript{(1479.6)}
\textsuperscript{133:7.6} Hijo mío, ya te he hablado mucho de la mente del hombre y del espíritu divino que vive en ella, pero ahora, permíteme recalcar que la conciencia de sí es una \textit{realidad}. Cuando un animal se vuelve consciente de sí mismo, se convierte en un hombre primitivo. Este logro es el resultado de una coordinación de funciones entre la energía impersonal y la mente que concibe el espíritu; este fenómeno es el que justifica la donación de un punto focal absoluto a la personalidad humana: el espíritu del Padre que está en los cielos.

\par 
%\textsuperscript{(1479.7)}
\textsuperscript{133:7.7} Las ideas no son simplemente un registro de sensaciones; las ideas son sensaciones, más las interpretaciones reflexivas del yo personal; y el yo es más que la suma de sus sensaciones. En una individualidad que evoluciona empieza a haber un indicio de acercamiento a la unidad, y esa unidad se deriva de la presencia interior de un fragmento de la unidad absoluta, que activa espiritualmente a esa mente consciente de origen animal.

\par 
%\textsuperscript{(1479.8)}
\textsuperscript{133:7.8} Ningún simple animal puede poseer una conciencia del tiempo. Los animales poseen una coordinación fisiológica de sensaciones y reconocimientos asociados, y la memoria correspondiente; pero ninguno de ellos experimenta un reconocimiento de sensaciones que tenga un significado, ni muestra una asociación intencional de estas experiencias físicas combinadas, tal como se manifiestan en las conclusiones de las interpretaciones humanas inteligentes y reflexivas. Este hecho de la existencia autoconsciente, asociado a la realidad de su experiencia espiritual posterior, convierte al hombre en un hijo potencial del universo y prefigura que alcanzará finalmente a la Unidad Suprema del universo.

\par 
%\textsuperscript{(1480.1)}
\textsuperscript{133:7.9} El yo humano tampoco es simplemente la suma de sus estados sucesivos de conciencia. Sin el funcionamiento eficaz de un factor que ordena y asocia la conciencia, no existiría una unidad suficiente como para justificar la denominación de individualidad. Una mente no unificada de este tipo difícilmente podría alcanzar los niveles de conciencia del estado humano. Si las asociaciones de conciencia no fueran más que un accidente, la mente de todos los hombres manifestaría entonces las asociaciones incontroladas y desatinadas de ciertas fases de la locura mental.

\par 
%\textsuperscript{(1480.2)}
\textsuperscript{133:7.10} Una mente humana basada exclusivamente en la conciencia de las sensaciones físicas, nunca podría alcanzar los niveles espirituales; este tipo de mente material carecería totalmente del sentido de los valores morales y estaría desprovista del sentido director de dominación espiritual, que es tan esencial para conseguir la unidad armoniosa de la personalidad en el tiempo, y que es inseparable de la supervivencia de la personalidad en la eternidad.

\par 
%\textsuperscript{(1480.3)}
\textsuperscript{133:7.11} La mente humana empieza pronto a manifestar unas cualidades que son supermateriales; el intelecto humano verdaderamente reflexivo no está atado del todo por los límites del tiempo. El hecho de que los individuos sean tan diferentes en las acciones de su vida, no solamente indica las variadas dotaciones hereditarias y las diferentes influencias del entorno, sino también el grado de unificación que el yo ha conseguido con el espíritu interior del Padre, la medida en que están identificados el uno con el otro.

\par 
%\textsuperscript{(1480.4)}
\textsuperscript{133:7.12} La mente humana no soporta bien el conflicto de la doble fidelidad. Cuando un alma se esfuerza por servir al bien y al mal a la vez, experimenta una tensión extrema. La mente supremamente feliz y eficazmente unificada es la que está dedicada por entero a hacer la voluntad del Padre que está en los cielos. Los conflictos no resueltos destruyen la unidad y pueden terminar en el desquiciamiento mental. No obstante, el carácter de supervivencia de un alma no se favorece intentando asegurarse la paz mental a cualquier precio, mediante el abandono de las nobles aspiraciones o transigiendo con los ideales espirituales. Esta paz se alcanza más bien afirmando constantemente el triunfo de lo que es verdadero, y esta victoria se consigue venciendo al mal con la poderosa fuerza del bien.

\par 
%\textsuperscript{(1480.5)}
\textsuperscript{133:7.13} Al día siguiente partieron hacia Salamina, donde se embarcaron para Antioquía, en la costa de Siria.

\section*{8. En Antioquía}
\par 
%\textsuperscript{(1480.6)}
\textsuperscript{133:8.1} Antioquía era la capital de la provincia romana de Siria, y el gobernador imperial tenía aquí su residencia. Antioquía tenía medio millón de habitantes; era la tercera ciudad del imperio en importancia y la primera en perversidad y flagrante inmoralidad. Gonod tenía que tratar muchísimos negocios, de manera que Jesús y Ganid estuvieron a solas la mayoría del tiempo. Visitaron todas las cosas de esta ciudad políglota excepto el bosquecillo de Dafne. Gonod y Ganid fueron a visitar este notorio paraje de la indecencia, pero Jesús se negó a acompañarlos. Aquellas escenas no eran tan chocantes para los indios, pero eran repelentes para un hebreo idealista.

\par 
%\textsuperscript{(1480.7)}
\textsuperscript{133:8.2} Jesús se fue poniendo serio y pensativo a medida que se acercaba a Palestina y al final de su viaje. Conversó con poca gente en Antioquía y rara vez se paseó por la ciudad. Después de mucho preguntar por qué su maestro manifestaba tan poco interés por Antioquía, Ganid consiguió finalmente que Jesús dijera: <<Esta ciudad no está lejos de Palestina; quizás regrese aquí algún día>>.

\par 
%\textsuperscript{(1481.1)}
\textsuperscript{133:8.3} Ganid tuvo una experiencia muy interesante en Antioquía. Este joven había demostrado ser un alumno capaz y ya había empezado a llevar a la práctica algunas de las enseñanzas de Jesús. Había cierto indio relacionado con los negocios de su padre en Antioquía, que se había vuelto tan desagradable y enfadado que habían pensado en despedirlo. Cuando Ganid se enteró, se dirigió al centro de negocios de su padre y tuvo una larga conversación con su compatriota. Este hombre tenía el sentimiento de que le habían asignado la tarea equivocada. Ganid le habló del Padre que está en los cielos y le amplió de diversas maneras su visión de la religión. Pero de todo lo que dijo Ganid, lo que más le impactó fue la cita de un proverbio hebreo, cuyas palabras de sabiduría decían: <<Cualquier cosa que tu mano tenga que hacer, hazla con todas tus fuerzas>>\footnote{\textit{Haz todo con todas tus fuerzas}: Ec 9:10.}.

\par 
%\textsuperscript{(1481.2)}
\textsuperscript{133:8.4} Después de preparar su equipaje para la caravana de camellos, descendieron hasta Sidón y desde allí fueron a Damasco; tres días después se prepararon para el largo trayecto a través de las arenas del desierto.

\section*{9. En Mesopotamia}
\par 
%\textsuperscript{(1481.3)}
\textsuperscript{133:9.1} El viaje en caravana a través del desierto no era una experiencia nueva para estos grandes viajeros. Después de ver a su maestro ayudar a cargar sus veinte camellos, y al observar que se ofrecía como voluntario para conducir su propio animal, Ganid exclamó: <<Maestro, ¿hay algo que no sepas hacer?>> Jesús se limitó a sonreír, diciendo: <<Un maestro no deja de tener méritos a los ojos de un alumno aplicado>>. Y partieron así para la antigua ciudad de Ur.

\par 
%\textsuperscript{(1481.4)}
\textsuperscript{133:9.2} Jesús se interesó mucho por la historia antigua de Ur\footnote{\textit{Ur, lugar de nacimiento de Abraham}: Gn 11:27-31.}, lugar donde nació Abraham, y también se quedó fascinado con las ruinas y tradiciones de Susa\footnote{\textit{Historias de Ester y Susa}: Est 1:2ff.}, de tal manera que Gonod y Ganid prolongaron su estancia en estas regiones tres semanas más, con el fin de darle más tiempo a Jesús para que continuara sus investigaciones, y también para encontrar la mejor ocasión de persuadirlo para que regresara con ellos a la India.

\par 
%\textsuperscript{(1481.5)}
\textsuperscript{133:9.3} Fue en Ur donde Ganid tuvo una larga conversación con Jesús respecto a la diferencia entre el conocimiento, la sabiduría y la verdad. Se quedó encantado con el proverbio del sabio hebreo: <<La sabiduría es lo principal; por lo tanto, adquiere sabiduría. Junto a tu búsqueda del conocimiento, adquiere la comprensión. Exalta la sabiduría y ella te hará progresar. Te llevará hasta los honores con tal que la practiques>>\footnote{\textit{La sabiduría es lo principal}: Pr 4:7-8.}.

\par 
%\textsuperscript{(1481.6)}
\textsuperscript{133:9.4} Por fin llegó el día de la separación. Todos fueron valientes, especialmente el joven, pero fue una dura prueba. Tenían lágrimas en los ojos, pero valor en el corazón. Al despedirse de su maestro, Ganid le dijo: <<Adiós, Maestro, pero no para siempre. Cuando vuelva a Damasco, te buscaré. Te quiero, pues creo que el Padre que está en los cielos debe parecerse algo a ti; al menos sé que tú te pareces mucho a lo que me has contado de él. Recordaré tu enseñanza, pero por encima de todo, nunca te olvidaré>>. El padre dijo: <<Adiós a un gran maestro, a alguien que nos ha hecho mejores y que nos ha ayudado a conocer a Dios>>. Y Jesús respondió: <<Que la paz esté con vosotros, y que la bendición del Padre que está en los cielos permanezca siempre con vosotros>>. Y Jesús se quedó en la orilla, contemplando cómo la pequeña barca los llevaba hasta el barco anclado. El Maestro se separó así de sus amigos de la India en Charax, para no volver a verlos nunca más en este mundo; ellos tampoco supieron nunca, en este mundo, que el hombre que más tarde apareció como Jesús de Nazaret era este mismo amigo que acababan de dejar: Josué su instructor.

\par 
%\textsuperscript{(1481.7)}
\textsuperscript{133:9.5} En la India, Ganid creció y se volvió un hombre influyente, un digno sucesor de su eminente padre; divulgó por todas partes muchas de las nobles verdades que había aprendido de Jesús, su amado maestro. Más tarde en la vida, cuando Ganid oyó hablar del extraño educador de Palestina que terminó su carrera en una cruz, aunque reconoció la similitud entre el evangelio de este Hijo del Hombre y las enseñanzas de su preceptor judío, nunca se le ocurrió pensar que los dos eran de hecho la misma persona.

\par 
%\textsuperscript{(1482.1)}
\textsuperscript{133:9.6} Así terminó el capítulo de la vida del Hijo del Hombre que podría titularse: \textit{La misión de Josué el educador}.


\chapter{Documento 134. Los años de transición}
\par 
%\textsuperscript{(1483.1)}
\textsuperscript{134:0.1} DURANTE el viaje por el Mediterráneo, Jesús había estudiado cuidadosamente a las personas que fue encontrando y los países que fue atravesando, y aproximadamente por esta época llegó a su decisión final en cuanto al resto de su vida en la Tierra. Había examinado plenamente y entonces había aprobado finalmente el plan que estipulaba que nacería de padres judíos en Palestina. Por consiguiente, regresó deliberadamente a Galilea para esperar el comienzo de la obra de su vida como instructor público de la verdad. Empezó a hacer planes para una carrera pública en el país del pueblo de su padre José, y actuó así por su propio libre albedrío.

\par 
%\textsuperscript{(1483.2)}
\textsuperscript{134:0.2} Jesús había descubierto, por experiencia personal y humana, que de todo el mundo romano, Palestina era el mejor lugar para dar a conocer los últimos capítulos, y representar las escenas finales, de su vida en la Tierra. Por primera vez se sintió plenamente satisfecho con el programa de manifestar abiertamente su verdadera naturaleza y revelar su identidad divina entre los judíos y los gentiles de su Palestina natal. Decidió definitivamente terminar su vida en la Tierra y completar su carrera de existencia mortal en el mismo país donde había empezado su experiencia humana como un niño indefenso. Su carrera en Urantia había comenzado entre los judíos de Palestina, y escogió terminar su vida en Palestina y entre los judíos.

\section*{1. El trigésimo año (año 24 d. de J.C.)}
\par 
%\textsuperscript{(1483.3)}
\textsuperscript{134:1.1} Después de despedirse de Gonod y de Ganid en Charax (en diciembre del año 23) Jesús regresó por el camino de Ur a Babilonia, donde se unió a una caravana del desierto que se dirigía a Damasco. De Damasco fue a Nazaret, parándose sólo unas horas en Cafarnaúm, donde se detuvo para visitar a la familia de Zebedeo. Allí se encontró con su hermano Santiago, que desde hacía algún tiempo había venido a trabajar en su lugar en el astillero de Zebedeo. Después de charlar con Santiago y Judá (que también se encontraba por casualidad en Cafarnaúm) y después de transferir a su hermano Santiago la casita que Juan Zebedeo se había ingeniado para comprar, Jesús continuó su camino hacia Nazaret.

\par 
%\textsuperscript{(1483.4)}
\textsuperscript{134:1.2} Al final de su viaje por el Mediterráneo, Jesús había recibido dinero suficiente como para hacer frente a sus gastos diarios casi hasta el momento de empezar su ministerio público. Pero, aparte de Zebedeo de Cafarnaúm y de la gente que conoció en el transcurso de esta gira extraordinaria, el mundo nunca supo que había hecho este viaje. Su familia siempre creyó que había pasado este tiempo estudiando en Alejandría. Jesús nunca confirmó esta creencia, ni tampoco refutó abiertamente este malentendido.

\par 
%\textsuperscript{(1483.5)}
\textsuperscript{134:1.3} Durante su estancia de varias semanas en Nazaret, Jesús charló con su familia y sus amigos, pasó algún tiempo en el taller de reparaciones con su hermano José, pero consagró la mayor parte de su atención a María y a Rut. Rut estaba a punto de cumplir entonces los quince años, y ésta era la primera ocasión que Jesús tenía de conversar largamente con ella desde que se había convertido en una jovencita.

\par 
%\textsuperscript{(1484.1)}
\textsuperscript{134:1.4} Tanto Simón como Judá deseaban casarse desde hacía algún tiempo, pero les disgustaba hacerlo sin el consentimiento de Jesús; en consecuencia, habían retrasado estos acontecimientos, esperando el regreso de su hermano mayor. Aunque todos consideraban a Santiago como el cabeza de familia en la mayoría de los casos, cuando se trataba de casarse querían la bendición de Jesús. Así pues, Simón y Judá se casaron en una doble boda a principios de marzo de este año 24. Todos los hijos mayores estaban ahora casados; sólo Rut, la más joven, permanecía en casa con María.

\par 
%\textsuperscript{(1484.2)}
\textsuperscript{134:1.5} Jesús charlaba con toda naturalidad y normalidad con cada uno de los miembros de su familia, pero cuando estaban todos reunidos tenía tan pocas cosas que decir, que llegaron a comentarlo entre ellos. María en particular estaba desconcertada por este comportamiento excepcionalmente extraño de su hijo primogénito.

\par 
%\textsuperscript{(1484.3)}
\textsuperscript{134:1.6} Cuando Jesús se estaba preparando para dejar Nazaret, el guía de una gran caravana que pasaba por la ciudad cayó gravemente enfermo, y Jesús, que era políglota, se ofreció para reemplazarlo. Este viaje significaba que estaría ausente durante un año; puesto que todos sus hermanos estaban casados y su madre vivía en la casa con Rut, Jesús convocó un consejo de familia donde propuso que su madre y Rut se fueran a vivir a Cafarnaúm, a la casa que había cedido a Santiago tan recientemente. En consecuencia, pocos días después de que Jesús se marchara con la caravana, María y Rut se mudaron a Cafarnaúm, donde vivieron durante el resto de la vida de María en la casa que Jesús les había proporcionado. José y su familia se mudaron a la vieja casa de Nazaret.

\par 
%\textsuperscript{(1484.4)}
\textsuperscript{134:1.7} Éste fue uno de los años más excepcionales en la experiencia interior del Hijo del Hombre; hizo un gran progreso en la obtención de una armonía funcional entre su mente humana y el Ajustador interior. El Ajustador se había ocupado activamente de reorganizar el pensamiento y de preparar la mente para los grandes acontecimientos que se hallaban entonces en el futuro cercano. La personalidad de Jesús se estaba preparando para su gran cambio de actitud hacia el mundo. Éste fue el período intermedio, la etapa de transición de este ser que había empezado su vida como Dios que se manifiesta como hombre, y que ahora se estaba preparando para completar su carrera terrestre como hombre que se manifiesta como Dios.

\section*{2. El viaje en caravana hasta el Caspio}
\par 
%\textsuperscript{(1484.5)}
\textsuperscript{134:2.1} El primero de abril del año 24 fue cuando Jesús salió de Nazaret para emprender el viaje en caravana hasta la región del Mar Caspio. La caravana a la que Jesús se había unido como guía iba desde Jerusalén hasta la región sudoriental del Mar Caspio, pasando por Damasco y el Lago Urmia, y atravesando Asiria, Media y Partia. Antes de que regresara de este viaje habría de transcurrir un año entero.

\par 
%\textsuperscript{(1484.6)}
\textsuperscript{134:2.2} Para Jesús, este viaje en caravana era una nueva aventura de exploración y de ministerio personal. Tuvo una experiencia interesante con la familia que componía la caravana ---pasajeros, guardias y conductores de camellos. Decenas de hombres, mujeres y niños que residían a lo largo de la ruta seguida por la caravana vivieron una vida más rica como resultado de su contacto con Jesús, el guía extraordinario, para ellos, de una caravana ordinaria. No todos los que disfrutaron de su ministerio personal en estas ocasiones se beneficiaron de ello, pero la gran mayoría de los que lo conocieron y conversaron con él fueron mejores para el resto de su vida terrestre.

\par 
%\textsuperscript{(1484.7)}
\textsuperscript{134:2.3} De todos sus viajes por el mundo, éste que realizó al Mar Caspio fue el que llevó a Jesús más cerca de oriente, y le permitió adquirir una mejor comprensión de los pueblos del lejano oriente. Efectuó un contacto íntimo y personal con cada una de las razas sobrevivientes de Urantia, exceptuando la roja. Disfrutó con la misma intensidad realizando su ministerio personal para cada una de estas diversas razas y pueblos mezclados, y todos fueron receptivos a la verdad viviente que les aportaba. Los europeos del extremo occidente y los asiáticos del extremo oriente prestaron una atención idéntica a sus palabras de esperanza y de vida eterna, y fueron influídos de igual manera por la vida de servicio amoroso y de ministerio espiritual que vivió entre ellos con tanta benevolencia.

\par 
%\textsuperscript{(1485.1)}
\textsuperscript{134:2.4} El viaje de la caravana fue un éxito en todos los sentidos. Fue un episodio de lo más interesante en la vida humana de Jesús, pues durante este año desempeñó una tarea ejecutiva, siendo responsable del material confiado a su cargo y de la seguridad de los viajeros que integraban la caravana. Cumplió sus múltiples deberes con la mayor fidelidad, eficacia y sabiduría.

\par 
%\textsuperscript{(1485.2)}
\textsuperscript{134:2.5} A su regreso de la región caspia, Jesús renunció a la dirección de la caravana en el Lago Urmia, donde se detuvo poco más de dos semanas. Regresó como pasajero en una caravana posterior hasta Damasco, donde los propietarios de los camellos le rogaron que permaneciera a su servicio. Rehusó esta oferta y continuó su viaje con la procesión de la caravana hasta Cafarnaúm, donde llegó el primero de abril del año 25. Ya no consideraba a Nazaret como su hogar. Cafarnaúm se había convertido en el hogar de Jesús, de Santiago, de María y de Rut. Pero Jesús no vivió nunca más con su familia; cuando se encontraba en Cafarnaúm se alojaba en la casa de los Zebedeo.

\section*{3. Las conferencias de Urmia}
\par 
%\textsuperscript{(1485.3)}
\textsuperscript{134:3.1} Camino del Mar Caspio, Jesús se había detenido varios días en la vieja ciudad persa de Urmia, en la orilla occidental del Lago Urmia, para descansar y recuperarse. En la isla más grande de un pequeño archipiélago situado a corta distancia de la costa, cerca de Urmia, se encontraba un gran edificio ---un anfiteatro para conferencias--- dedicado al <<espíritu de la religión>>. Esta construcción era en realidad un templo de la filosofía de las religiones.

\par 
%\textsuperscript{(1485.4)}
\textsuperscript{134:3.2} Este templo de la religión había sido construido por un rico comerciante, ciudadano de Urmia, y sus tres hijos. Este hombre se llamaba Cimboitón, y entre sus antepasados se encontraban pueblos muy diversos.

\par 
%\textsuperscript{(1485.5)}
\textsuperscript{134:3.3} En esta escuela de religión, las conferencias y discusiones empezaban todos los días de la semana a las 10 de la mañana. Las sesiones de la tarde se iniciaban a las 3, y los debates nocturnos se abrían a las 8. Cimboitón o uno de sus tres hijos siempre presidían estas sesiones de enseñanza, de discusión y de debates. El fundador de esta singular escuela de religiones vivió y murió sin revelar nunca sus creencias religiosas personales.

\par 
%\textsuperscript{(1485.6)}
\textsuperscript{134:3.4} Jesús participó varias veces en estas discusiones, y antes de partir de Urmia, Cimboitón acordó con Jesús que en su viaje de regreso residiría dos semanas con ellos y daría veinticuatro conferencias sobre <<la fraternidad de los hombres>>; también dirigiría doce sesiones nocturnas de preguntas, discusiones y debates sobre sus conferencias en particular, y sobre la fraternidad de los hombres en general.

\par 
%\textsuperscript{(1485.7)}
\textsuperscript{134:3.5} En conformidad con este acuerdo, Jesús se detuvo en su viaje de vuelta y dio estas conferencias. De todas las enseñanzas del Maestro en Urantia, éstas fueron las más sistemáticas y formales. Nunca dijo tantas cosas sobre un mismo tema, ni antes ni después, como lo hizo en estas conferencias y discusiones sobre la fraternidad de los hombres. Estas conferencias trataron, en verdad, sobre el <<reino de Dios>> y los <<reinos de los hombres>>.

\par 
%\textsuperscript{(1486.1)}
\textsuperscript{134:3.6} Más de treinta religiones y cultos religiosos estaban representados en la facultad de este templo de filosofía religiosa. Los profesores eran elegidos, mantenidos y plenamente acreditados por sus grupos religiosos respectivos. En aquel momento había en la facultad unos setenta y cinco profesores, y vivían en casas de campo con capacidad para unas doce personas. Estos grupos se cambiaban cada Luna nueva echándolo a suertes. La intolerancia, el espíritu contencioso o cualquier otra tendencia que interfiriera con el funcionamiento apacible de la comunidad, suponía la destitución inmediata y sumaria del educador transgresor. Lo despedían sin ceremonias y su sustituto en espera era instalado inmediatamente en su lugar.

\par 
%\textsuperscript{(1486.2)}
\textsuperscript{134:3.7} Estos instructores de las diversas religiones hacían un gran esfuerzo para mostrar la similitud de sus religiones en cuanto a las cosas fundamentales de esta vida y de la siguiente. Para obtener una plaza en esta facultad bastaba con aceptar una sola doctrina ---cada profesor debía representar a una religión que reconociera a Dios--- a algún tipo de Deidad suprema. Había en la facultad cinco educadores independientes que no representaban a ninguna religión organizada, y Jesús apareció ante ellos bajo esta modalidad.

\par 
%\textsuperscript{(1486.3)}
\textsuperscript{134:3.8} [Cuando nosotros, los intermedios, preparamos por primera vez el resumen de las enseñanzas de Jesús en Urmia, surgió un desacuerdo entre los serafines de las iglesias y los serafines del progreso sobre la conveniencia de incluir estas enseñanzas en la Revelación de Urantia. Las condiciones que prevalecen tanto en las religiones como en los gobiernos humanos del siglo veinte son tan diferentes de las que predominaban en los tiempos de Jesús, que era difícil en verdad adaptar las enseñanzas del Maestro en Urmia a los problemas del reino de Dios y de los reinos de los hombres, tal como estas funciones mundiales existen en el siglo veinte. Nunca fuimos capaces de formular una exposición de las enseñanzas del Maestro que fuera aceptable para estos dos grupos de serafines del gobierno planetario. Finalmente, el Melquisedek presidente de la comisión reveladora nombró una comisión de tres de nosotros para que presentara nuestro punto de vista sobre las enseñanzas del Maestro en Urmia, adaptadas a las condiciones religiosas y políticas del siglo veinte en Urantia. En consecuencia, nosotros, los tres intermedios secundarios, completamos esta adaptación de las enseñanzas de Jesús, reexponiendo sus declaraciones tal como las aplicaríamos a las condiciones del mundo de hoy. Presentamos ahora estas exposiciones tal como están después de haber sido revisadas por el Melquisedek presidente de la comisión reveladora.]

\section*{4. La soberanía --- divina y humana}
\par 
%\textsuperscript{(1486.4)}
\textsuperscript{134:4.1} La fraternidad de los hombres está basada en la paternidad de Dios. La familia de Dios tiene su origen en el amor de Dios ---Dios es amor\footnote{\textit{Dios es amor}: 1 Jn 4:7-11,16,19.}. Dios Padre ama divinamente a sus hijos, a todos ellos.

\par 
%\textsuperscript{(1486.5)}
\textsuperscript{134:4.2} El reino de los cielos, el gobierno divino, está basado en el hecho de la soberanía divina ---Dios es espíritu\footnote{\textit{Dios es espíritu, su reino es espiritual}: Jn 3:5; 4:24.}. Puesto que Dios es espíritu, este reino es espiritual. El reino de los cielos no es material ni simplemente intelectual; es una relación espiritual entre Dios y el hombre.

\par 
%\textsuperscript{(1486.6)}
\textsuperscript{134:4.3} Si las diferentes religiones reconocen la soberanía espiritual de Dios Padre, entonces todas esas religiones permanecerán en paz. Sólo cuando una religión pretende ser de alguna manera superior a todas las demás, y poseer una autoridad exclusiva sobre las otras religiones, dicha religión se atreverá a ser intolerante con las demás religiones o tendrá la osadía de perseguir a otros creyentes religiosos.

\par 
%\textsuperscript{(1487.1)}
\textsuperscript{134:4.4} La paz religiosa ---la fraternidad--- nunca podrá existir a menos que todas las religiones estén dispuestas a despojarse por completo de toda autoridad eclesiástica, y a abandonar plenamente todo concepto de soberanía espiritual. Sólo Dios es el soberano espiritual.

\par 
%\textsuperscript{(1487.2)}
\textsuperscript{134:4.5} No podéis conseguir la igualdad entre las religiones (la libertad religiosa) sin guerras religiosas, a menos que todas las religiones estén dispuestas a transferir toda la soberanía religiosa a un nivel superhumano, a Dios mismo.

\par 
%\textsuperscript{(1487.3)}
\textsuperscript{134:4.6} El reino de los cielos en el corazón de los hombres creará la unidad religiosa (no necesariamente la uniformidad)\footnote{\textit{Unidad, no uniformidad}: Ro 14:17-19; 1 Co 1:10; 1 Co 10:17; 1 Co 12:17-31; Ef 4:3-6,11-13.} porque todos y cada uno de los grupos religiosos, compuestos por tales creyentes religiosos, estarán libres de toda noción de autoridad eclesiástica ---de soberanía religiosa.

\par 
%\textsuperscript{(1487.4)}
\textsuperscript{134:4.7} Dios es espíritu, y Dios confiere un fragmento de su ser espiritual para que resida en el corazón del hombre\footnote{\textit{El espíritu interior}: Job 32:8,18; Is 63:10-11; Ez 37:14; Mt 10:20; Lc 17:21; Jn 17:21-23; Ro 8:9-11; 1 Co 3:16-17; 6:19; 2 Co 6:16; Gl 2:20; 1 Jn 3:24; 4:12-15; Ap 21:3.}. Espiritualmente, todos los hombres son iguales\footnote{\textit{Espiritualmente, todos los hombres son iguales}: 2 Cr 19:7; Job 34:19; Eclo 35:12; Hch 10:34; Ro 2:11; Gl 2:6; 3:28; Ef 6:9; Col 3:11.}. El reino de los cielos está desprovisto de castas, de clases, de niveles sociales y de grupos económicos. Todos sois hermanos\footnote{\textit{Todos somos hermanos}: Mt 23:8.}.

\par 
%\textsuperscript{(1487.5)}
\textsuperscript{134:4.8} Pero en cuanto perdáis de vista la soberanía espiritual de Dios Padre, alguna religión empezará a afirmar su superioridad sobre las otras religiones. Entonces, en lugar de paz en la Tierra y de buena voluntad entre los hombres, empezarán las disensiones, las recriminaciones e incluso las guerras religiosas, o al menos las guerras entre los practicantes de la religión.

\par 
%\textsuperscript{(1487.6)}
\textsuperscript{134:4.9} Los seres dotados de libre albedrío que se consideran como iguales, a menos que reconozcan mutuamente estar sometidos a alguna soberanía superior, a alguna autoridad que esté por encima de ellos, tarde o temprano se sienten tentados a probar su capacidad para conseguir poder y autoridad sobre otras personas y grupos. El concepto de igualdad no aporta nunca la paz, excepto cuando se reconoce mutuamente una influencia supercontroladora de soberanía superior.

\par 
%\textsuperscript{(1487.7)}
\textsuperscript{134:4.10} Los hombres religiosos de Urmia vivían juntos en una paz y tranquilidad relativas porque habían renunciado plenamente a todas sus nociones de soberanía religiosa. Espiritualmente, todos creían en un Dios soberano; socialmente, la autoridad plena e indiscutible residía en su presidente Cimboitón. Todos sabían muy bien lo que le sucedería a cualquier educador que se atreviera a dominar a sus colegas. Ninguna paz religiosa duradera puede existir en Urantia hasta que todos los grupos religiosos no renuncien libremente a todas sus nociones de favor divino, de pueblo elegido y de soberanía religiosa. Sólo cuando Dios Padre se vuelva supremo, los hombres se volverán hermanos en religión y vivirán juntos en paz religiosa en la Tierra.

\section*{5. La soberanía política}
\par 
%\textsuperscript{(1487.8)}
\textsuperscript{134:5.1} [Aunque la enseñanza del Maestro referente a la soberanía de Dios es una verdad ---pero complicada por la aparición posterior de la religión acerca de su persona entre las religiones del mundo--- sus exposiciones relativas a la soberanía política se han complicado enormemente debido a la evolución política de la vida de las naciones durante los últimos mil novecientos y pico de años. En la época de Jesús sólo había dos grandes potencias mundiales: el Imperio Romano en occidente y el Imperio Han en oriente, y los dos estaban ampliamente separados por el reino de Partia y otras tierras intermedias de las regiones del Caspio y del Turquestán. Por lo tanto, en la exposición que viene a continuación nos hemos apartado aún más de la sustancia de las enseñanzas del Maestro en Urmia referentes a la soberanía política; al mismo tiempo, hemos intentado describir la importancia de dichas enseñanzas tal como son aplicables a la etapa particularmente crítica de la evolución de la soberanía política en el siglo veinte después de Cristo.]

\par 
%\textsuperscript{(1487.9)}
\textsuperscript{134:5.2} Nunca dejará de haber guerras en Urantia mientras las naciones se aferren a la noción ilusoria de la soberanía nacional ilimitada. Sólo existen dos niveles de soberanía relativa en un mundo habitado: el libre albedrío espiritual de cada mortal individual y la soberanía colectiva del conjunto de la humanidad. Entre el nivel del ser humano individual y el de la totalidad de la humanidad, todas las agrupaciones y asociaciones son relativas, transitorias y sólo tienen valor en la medida en que aumenten el bienestar, la felicidad y el progreso del individuo y del gran conjunto planetario ---del hombre y de la humanidad.

\par 
%\textsuperscript{(1488.1)}
\textsuperscript{134:5.3} Los educadores religiosos deben recordar siempre que la soberanía espiritual de Dios está por encima de todas las lealtades espirituales interpuestas e intermedias. Los gobernantes civiles aprenderán algún día que los Altísimos gobiernan en los reinos de los hombres\footnote{\textit{Los Altísimos gobiernan en los reinos de los hombres}: Dn 4:17,25,32; 5:21.}.

\par 
%\textsuperscript{(1488.2)}
\textsuperscript{134:5.4} Este gobierno de los Altísimos en los reinos de los hombres no está establecido para el beneficio especial de un grupo de mortales particularmente favorecido. No existe ningún tipo de <<pueblo elegido>>. El reinado de los Altísimos (los supercontroladores de la evolución política) está destinado a fomentar, entre \textit{todos} los hombres, el mayor bien para el mayor número de ellos y durante el mayor tiempo posible.

\par 
%\textsuperscript{(1488.3)}
\textsuperscript{134:5.5} La soberanía es el poder y crece mediante la organización. Este crecimiento de la organización del poder político es bueno y conveniente, porque tiende a englobar segmentos cada vez mayores del conjunto de la humanidad. Pero este mismo crecimiento de las organizaciones políticas crea un problema en cada etapa intermedia, entre la organización inicial y natural del poder político ---la familia--- y la consumación final del crecimiento político ---el gobierno de toda la humanidad, por toda la humanidad y para toda la humanidad.

\par 
%\textsuperscript{(1488.4)}
\textsuperscript{134:5.6} Partiendo del poder de los padres en el grupo familiar, la soberanía política evoluciona por medio de la organización a medida que las familias se superponen en clanes consanguíneos que se unen, por varias razones, en unidades tribales ---en agrupaciones políticas superconsanguíneas. A continuación, mediante el negocio, el comercio y la conquista, las tribus se unifican en una nación, mientras que las mismas naciones a veces se unifican en un imperio.

\par 
%\textsuperscript{(1488.5)}
\textsuperscript{134:5.7} A medida que la soberanía pasa de los grupos más pequeños a las colectividades mayores, las guerras disminuyen. Es decir, las guerras menores entre las naciones más pequeñas disminuyen, pero el potencial de las grandes guerras aumenta a medida que las naciones que ejercen la soberanía se vuelven cada vez más grandes. Finalmente, cuando todo el mundo haya sido explorado y ocupado, cuando las naciones sean pocas, fuertes y poderosas, cuando esas grandes naciones supuestamente soberanas lleguen a tener fronteras comunes, cuando sólo estén separadas por los océanos, entonces el escenario estará preparado para las guerras mayores, para los conflictos mundiales. Las llamadas naciones soberanas no pueden codearse sin generar conflictos y provocar guerras.

\par 
%\textsuperscript{(1488.6)}
\textsuperscript{134:5.8} La dificultad para que evolucione la soberanía política desde la familia hasta toda la humanidad reside en la inercia-resistencia que se manifiesta en todos los niveles intermedios. Las familias, en ocasiones, han desafiado a su clan, mientras que los clanes y las tribus han contrarrestado a menudo la soberanía del Estado territorial. Cada evolución nueva y progresiva de la soberanía política se encuentra (y siempre se ha encontrado) estorbada y entorpecida por las <<fases de andamiaje>> de los desarrollos anteriores de la organización política. Y esto es así porque las lealtades humanas, una vez que se han movilizado, son difíciles de modificar. La misma lealtad que hace posible la evolución de la tribu, hace difícil la evolución de la supertribu ---el Estado territorial. Y la misma lealtad (el patriotismo) que hace posible la evolución del Estado territorial, complica enormemente el desarrollo evolutivo del gobierno de toda la humanidad.

\par 
%\textsuperscript{(1488.7)}
\textsuperscript{134:5.9} La soberanía política se crea mediante la renuncia a la autodeterminación, primero por parte del individuo en el interior de la familia, y a continuación por las familias y los clanes en relación con la tribu y las agrupaciones más grandes. Este traspaso progresivo de la autodeterminación, desde las organizaciones políticas más pequeñas a otras cada vez más grandes, ha continuado en oriente generalmente sin interrupción desde el establecimiento de las dinastías Ming y Mogol. En occidente ha prevalecido durante más de mil años, hasta el final de la Guerra Mundial; después, un desacertado movimiento retrógrado invirtió temporalmente esta tendencia normal, restableciendo la soberanía política hundida de numerosa pequeñas colectividades europeas.

\par 
%\textsuperscript{(1489.1)}
\textsuperscript{134:5.10} Urantia no disfrutará de una paz duradera hasta que las llamadas naciones soberanas no entreguen sus poderes soberanos, de manera plena e inteligente, entre las manos de la fraternidad de los hombres ---del gobierno de la humanidad. El internacionalismo--- las ligas de naciones ---nunca podrá asegurar una paz permanente a la humanidad. Las confederaciones mundiales de naciones impedirán eficazmente las guerras menores y controlarán de manera aceptable a las naciones más pequeñas, pero no lograrán impedir las guerras mundiales ni controlarán a los tres, cuatro o cinco gobiernos más poderosos. En presencia de unos conflictos reales, una de estas potencias mundiales se retirará de la Liga y declarará la guerra. No se puede evitar que las naciones se lancen a la guerra mientras permanezcan infectadas con el virus ilusorio de la soberanía nacional. El internacionalismo es un paso en la dirección adecuada. Una fuerza de policía internacional impedirá muchas guerras menores, pero será ineficaz para impedir las guerras mayores, los conflictos entre los grandes gobiernos militares de la Tierra.

\par 
%\textsuperscript{(1489.2)}
\textsuperscript{134:5.11} A medida que disminuye el número de naciones verdaderamente soberanas (las grandes potencias), se acrecienta la oportunidad y la necesidad de un gobierno de la humanidad. Cuando sólo existan unas pocas (grandes) potencias realmente soberanas, o bien tendrán que embarcarse en una lucha a muerte por la supremacía nacional (imperial), o mediante la renuncia voluntaria a ciertas prerrogativas de la soberanía, tendrán que crear el núcleo esencial de un poder supernacional que sirva de comienzo para la soberanía real de toda la humanidad.

\par 
%\textsuperscript{(1489.3)}
\textsuperscript{134:5.12} La paz no llegará a Urantia hasta que todas las naciones llamadas soberanas no abandonen su poder de declarar la guerra entre las manos de un gobierno representativo de toda la humanidad. La soberanía política es innata en los pueblos del mundo. Cuando todos los pueblos de Urantia creen un gobierno mundial, tendrán el derecho y el poder de hacerlo SOBERANO; y cuando esa potencia mundial representativa o democrática controle las fuerzas terrestres, aéreas y navales del mundo, la paz en la Tierra y la buena voluntad entre los hombres podrán prevalecer ---pero no antes de ese momento.

\par 
%\textsuperscript{(1489.4)}
\textsuperscript{134:5.13} Podemos citar un ejemplo importante de los siglos diecinueve y veinte: Los cuarenta y ocho Estados de la Unión Federal Americana disfrutan de la paz desde hace mucho tiempo. Ya no tienen guerras entre ellos. Han cedido su soberanía al gobierno federal, y mediante el arbitraje de la guerra, han abandonado toda pretensión a las ilusiones de la autodeterminación. Aunque cada Estado regula sus asuntos internos, no se ocupa de las relaciones exteriores, de las tarifas, de la inmigración, de las cuestiones militares ni del comercio interestatal. Los Estados individuales tampoco se ocupan de las cuestiones de ciudadanía. Los cuarenta y ocho Estados sólo sufren los estragos de la guerra cuando la soberanía del gobierno federal se encuentra en algún peligro.

\par 
%\textsuperscript{(1489.5)}
\textsuperscript{134:5.14} Al haber abandonado los sofismas gemelos de la soberanía y de la autodeterminación, estos cuarenta y ocho Estados disfrutan de la paz y de la tranquilidad interestatal. De la misma manera, las naciones de Urantia empezarán a disfrutar de la paz cuando traspasen libremente sus soberanías respectivas a las manos de un gobierno global ---a la soberanía de la fraternidad de los hombres. En ese Estado mundial, las naciones pequeñas serán tan poderosas como las grandes, como sucede con el pequeño Estado de Rhode Island, que tiene sus dos senadores en el Congreso Americano, exactamente igual que el populoso Estado de Nueva York o el extenso Estado de Texas.

\par 
%\textsuperscript{(1490.1)}
\textsuperscript{134:5.15} La soberanía (estatal) limitada de estos cuarenta y ocho Estados fue creada por los hombres y para los hombres. La soberanía superestatal (nacional) de la Unión Federal Americana fue creada por los trece primeros de estos Estados en su propio beneficio y para el beneficio de los hombres. Algún día, las naciones crearán de manera similar la soberanía supernacional del gobierno planetario de la humanidad, en su propio beneficio y para el beneficio de todos los hombres.

\par 
%\textsuperscript{(1490.2)}
\textsuperscript{134:5.16} Los ciudadanos no nacen para el beneficio de los gobiernos; los gobiernos son organizaciones pensadas y creadas para el beneficio de los hombres. La evolución de la soberanía política no puede tener otro destino que la aparición del gobierno de la soberanía de todos los hombres. Todas las demás soberanías tienen un valor relativo, un significado intermedio y una condición subordinada.

\par 
%\textsuperscript{(1490.3)}
\textsuperscript{134:5.17} Con el progreso científico, las guerras se van a volver cada vez más devastadoras, hasta que se conviertan prácticamente en un suicidio racial. ¿Cuántas guerras mundiales tendrán que producirse y cuántas ligas de naciones tendrán que fracasar antes de que los hombres estén dispuestos a establecer el gobierno de la humanidad y empiecen a disfrutar de las bendiciones de una paz permanente y a desarrollarse con la tranquilidad de la buena voluntad ---de la buena voluntad mundial--- entre los hombres?

\section*{6. La ley, la libertad y la soberanía}
\par 
%\textsuperscript{(1490.4)}
\textsuperscript{134:6.1} Si un hombre desea ardientemente su independencia ---la libertad--- debe recordar que \textit{todos} los demás hombres anhelan la misma independencia. Los grupos de mortales que aman así la libertad no pueden convivir en paz a menos que se sometan a las leyes, reglas y reglamentos que conceden a cada persona el mismo grado de independencia, salvaguardando al mismo tiempo un grado igual de independencia para todos sus semejantes mortales. Si un hombre ha de ser absolutamente libre, entonces otro tendrá que convertirse en un esclavo absoluto. La naturaleza relativa de la libertad es verdadera en el terreno social, económico y político. La libertad es el don de la civilización, hecho posible por la fuerza de la LEY.

\par 
%\textsuperscript{(1490.5)}
\textsuperscript{134:6.2} La religión hace espiritualmente posible realizar la fraternidad de los hombres, pero se necesitará un gobierno de la humanidad para que regule los problemas sociales, económicos y políticos asociados a ese objetivo de la felicidad y de la eficacia humanas.

\par 
%\textsuperscript{(1490.6)}
\textsuperscript{134:6.3} Habrá guerras y rumores de guerras\footnote{\textit{Guerras y rumores de guerras}: Mt 24:6-7; Mc 13:7-8; Lc 21:9-10.} ---una nación se levantará contra otra--- mientras que la soberanía política del mundo esté dividida e injustamente mantenida por un grupo de Estados nacionales. Inglaterra, Escocia y Gales siempre estuvieron luchando entre sí hasta que renunciaron a sus respectivas soberanías y las confiaron al Reino Unido.

\par 
%\textsuperscript{(1490.7)}
\textsuperscript{134:6.4} Una nueva guerra mundial enseñará a las naciones llamadas soberanas a formar una especie de federación, creando así el mecanismo para evitar las guerras menores, las guerras entre las naciones más pequeñas. Pero las guerras globales continuarán hasta que se cree el gobierno de la humanidad. La soberanía global impedirá las guerras globales ---ninguna otra cosa puede hacerlo.

\par 
%\textsuperscript{(1490.8)}
\textsuperscript{134:6.5} Los cuarenta y ocho Estados americanos libres conviven en paz. Entre los ciudadanos de estos cuarenta y ocho Estados se encuentran todas las razas y nacionalidades diversas que viven en las naciones de Europa, donde siempre están en guerra. Estos americanos representan a casi todas las religiones, sectas y cultos religiosos de todo el ancho mundo, y sin embargo conviven en paz aquí en Norteamérica. Todo esto es posible porque estos cuarenta y ocho Estados han renunciado a su soberanía y han abandonado toda noción de supuestos derechos a la autodeterminación.

\par 
%\textsuperscript{(1490.9)}
\textsuperscript{134:6.6} No es una cuestión de armamento o de desarme. La cuestión del servicio militar obligatorio o voluntario tampoco influye en estos problemas de mantener la paz mundial. Si se le quitaran a las naciones poderosas todas las formas de armamento mecánico moderno y todos los tipos de explosivos, lucharían con los puños, las piedras y las mazas mientras siguieran aferradas a las ilusiones de su derecho divino a la soberanía nacional.

\par 
%\textsuperscript{(1491.1)}
\textsuperscript{134:6.7} La guerra no es una enfermedad grande y terrible del hombre; la guerra es un síntoma, un resultado. La verdadera enfermedad es el virus de la soberanía nacional.

\par 
%\textsuperscript{(1491.2)}
\textsuperscript{134:6.8} Las naciones de Urantia no han poseído una verdadera soberanía; nunca han tenido una soberanía que pudiera protegerlas de los estragos y las devastaciones de las guerras mundiales. Al crear el gobierno global de la humanidad, las naciones no abandonan su soberanía, sino más bien están creando de hecho una soberanía mundial, real, duradera y de buena fe, que en adelante será plenamente capaz de protegerlas de todas las guerras. Los asuntos locales serán tratados por los gobiernos locales, y los asuntos nacionales por los gobiernos nacionales; los asuntos internacionales serán administrados por el gobierno mundial.

\par 
%\textsuperscript{(1491.3)}
\textsuperscript{134:6.9} La paz mundial no se puede mantener mediante tratados, diplomacia, políticas exteriores, alianzas, equilibrios de poder o cualquier otro tipo de juegos malabares improvisados con las soberanías de los nacionalismos. Hay que crear una ley mundial y debe ser aplicada por un gobierno mundial ---la soberanía de toda la humanidad.

\par 
%\textsuperscript{(1491.4)}
\textsuperscript{134:6.10} Con un gobierno mundial, los individuos gozarán de una libertad mucho más amplia. Hoy, los ciudadanos de las grandes potencias están cargados de impuestos, reglamentados y controlados de una manera casi opresiva. Una gran parte de esta intromisión actual en las libertades individuales desaparecerá cuando los gobiernos nacionales estén dispuestos a depositar su soberanía, en materia de asuntos internacionales, entre las manos de un gobierno global.

\par 
%\textsuperscript{(1491.5)}
\textsuperscript{134:6.11} Bajo un gobierno mundial, las colectividades nacionales tendrán una verdadera oportunidad para realizar y disfrutar las libertades personales de una auténtica democracia. La falacia de la autodeterminación habrá terminado. Con la reglamentación global del dinero y del comercio llegará la nueva era de una paz a escala mundial. Pronto podría surgir un idioma mundial, y al menos habrá alguna esperanza de que algún día exista una religión mundial ---o unas religiones con un punto de vista global.

\par 
%\textsuperscript{(1491.6)}
\textsuperscript{134:6.12} La seguridad colectiva nunca proporcionará la paz hasta que la colectividad incluya a toda la humanidad.

\par 
%\textsuperscript{(1491.7)}
\textsuperscript{134:6.13} La soberanía política del gobierno representativo de la humanidad traerá una paz duradera a la Tierra, y la fraternidad espiritual del hombre asegurará para siempre la buena voluntad entre todos los hombres. No existe ningún otro camino para conseguir la paz en la Tierra y la buena voluntad entre los hombres\footnote{\textit{Para conseguir paz, la buena voluntad}: Lc 2:14.}.

\par 
%\textsuperscript{(1491.8)}
\textsuperscript{134:6.14} Después de la muerte de Cimboitón, sus hijos encontraron grandes dificultades para mantener la paz en la facultad. Las repercusiones de las enseñanzas de Jesús hubieran sido mucho mayores si los educadores cristianos posteriores que se incorporaron a la facultad de Urmia hubieran mostrado más sabiduría y hubieran ejercido más tolerancia.

\par 
%\textsuperscript{(1491.9)}
\textsuperscript{134:6.15} El hijo mayor de Cimboitón recurrió a Abner, de Filadelfia, para que le ayudara, pero Abner tuvo muy poco acierto en la elección de los educadores, en el sentido de que resultaron ser inflexibles e intransigentes. Estos instructores trataron de que su religión dominara a las otras creencias. Nunca sospecharon que las conferencias del conductor de caravanas, a las que se aludía con tanta frecuencia, habían sido dadas por el mismo Jesús.

\par 
%\textsuperscript{(1491.10)}
\textsuperscript{134:6.16} Al aumentar la confusión dentro de la facultad, los tres hermanos retiraron su apoyo financiero, y al cabo de cinco años la escuela cerró. Más tarde se abrió de nuevo como templo mitríaco, y finalmente se incendió en conjunción con una de sus celebraciones orgiásticas.

\section*{7. El trigésimo primer año (año 25 d. de J.C.)}
\par 
%\textsuperscript{(1492.1)}
\textsuperscript{134:7.1} Cuando Jesús volvió de su viaje al Mar Caspio, sabía que sus desplazamientos por el mundo prácticamente habían terminado. Sólo hizo un viaje más fuera de Palestina, y fue para ir a Siria. Después de una breve visita a Cafarnaúm, se dirigió a Nazaret, donde se quedó unos días haciendo visitas. A mediados de abril salió de Nazaret para Tiro. Desde allí viajó hacia el norte, deteniéndose unos días en Sidón, pero su destino era Antioquía.

\par 
%\textsuperscript{(1492.2)}
\textsuperscript{134:7.2} Éste es el año de los recorridos solitarios de Jesús a través de Palestina y Siria. Durante todo este año de viajes, fue conocido por diversos nombres en distintas partes del país: el carpintero de Nazaret, el constructor de barcas de Cafarnaúm, el escriba de Damasco y el educador de Alejandría.

\par 
%\textsuperscript{(1492.3)}
\textsuperscript{134:7.3} En Antioquía, el Hijo del Hombre vivió más de dos meses, trabajando, observando, estudiando, visitando, ayudando y, durante todo este tiempo, aprendiendo cómo viven los hombres, cómo piensan, sienten y reaccionan al entorno de la existencia humana. Durante tres semanas de este período trabajó como fabricante de tiendas. En Antioquía permaneció más tiempo que en cualquiera de los otros lugares que visitó en este viaje. Diez años después, cuando el apóstol Pablo predicó en Antioquía\footnote{\textit{Pablo en Antioquía}: Hch 11:25-26.} y oyó hablar a sus discípulos de las doctrinas del \textit{escriba de Damasco}, no sospechó que sus alumnos habían oído la voz y escuchado las enseñanzas del propio Maestro.

\par 
%\textsuperscript{(1492.4)}
\textsuperscript{134:7.4} Desde Antioquía, Jesús viajó hacia el sur a lo largo de la costa hasta Cesarea, donde se detuvo unas semanas, continuando luego por la costa hasta Jope. Desde Jope viajó tierra adentro hasta Jamnia, Asdod y Gaza. Desde Gaza cogió la ruta interior hasta Beerseba, donde permaneció una semana.

\par 
%\textsuperscript{(1492.5)}
\textsuperscript{134:7.5} Jesús emprendió entonces su periplo final, como individuo particular, a través del corazón de Palestina, desplazándose desde Beerseba en el sur hasta Dan en el norte. En este viaje hacia el norte se detuvo en Hebrón, Belén (donde vio su lugar de nacimiento), Jerusalén (no visitó Betania), Beerot, Lebona, Sicar, Siquem, Samaria, Geba, En-Ganim, Endor y Madón. Atravesando Magdala y Cafarnaúm, continuó hacia el norte, pasando al este de las Aguas de Merom, y se dirigió por Cárata hasta Dan o Cesarea de Filipo.

\par 
%\textsuperscript{(1492.6)}
\textsuperscript{134:7.6} El Ajustador del Pensamiento interior condujo entonces a Jesús a apartarse de los lugares habitados por los hombres, y a subir al Monte Hermón\footnote{\textit{El ajustador conduce a Jesús a la montaña}: Mt 4:1; Mc 1:12; Lc 4:1.} para poder terminar allí el trabajo de dominar su mente humana, y completar la tarea de efectuar su consagración total al resto de la obra de su vida en la Tierra.

\par 
%\textsuperscript{(1492.7)}
\textsuperscript{134:7.7} Ésta fue una de las épocas excepcionales y extraordinarias de la vida terrestre del Maestro en Urantia. Atravesó otra experiencia muy similar cuando estuvo solo en las colinas cercanas a Pella, inmediatamente después de su bautismo. Este período de aislamiento en el Monte Hermón marcó el final de su carrera puramente humana, es decir, la terminación técnica de su donación como mortal, mientras que el aislamiento posterior señaló el comienzo de la fase más divina de su donación. Jesús vivió a solas con Dios durante seis semanas en las pendientes del Monte Hermón.

\section*{8. La estancia en el monte Hermón}
\par 
%\textsuperscript{(1492.8)}
\textsuperscript{134:8.1} Después de pasar algún tiempo en las proximidades de Cesarea de Filipo, Jesús preparó sus provisiones, adquirió una bestia de carga, contrató a un muchacho llamado Tiglat y se dirigió por el camino de Damasco hasta un pueblo conocido en otro tiempo como Beit Jenn, en los cerros al pie del Monte Hermón. Aquí, poco antes de mediados de agosto del año 25, estableció su campamento, dejó sus provisiones al cuidado de Tiglat y ascendió las laderas solitarias de la montaña. Durante este primer día, Tiglat acompañó a Jesús en su subida hasta un punto determinado a unos 2000 metros sobre el nivel del mar, donde construyeron un receptáculo de piedra en el que Tiglat tenía que depositar los alimentos dos veces por semana.

\par 
%\textsuperscript{(1493.1)}
\textsuperscript{134:8.2} El primer día después de dejar a Tiglat, Jesús sólo había ascendido un corto trayecto de la montaña cuando se detuvo para orar. Entre otras cosas, pidió a su Padre que hiciera volver a su serafín guardián para que <<acompañara a Tiglat>>. Solicitó que se le permitiera subir solo hacia su última contienda con las realidades de la existencia mortal, y esta petición le fue concedida. Participó en la gran prueba con la única ayuda y apoyo de su Ajustador interior.

\par 
%\textsuperscript{(1493.2)}
\textsuperscript{134:8.3} Jesús comió frugalmente mientras estuvo en la montaña; sólo se abstuvo de todo alimento un día o dos a la vez\footnote{\textit{¿Jesus ayunó o comió?}: Mt 4:2; Mc 1:13; Lc 4:2.}. Los seres superhumanos que se enfrentaron con él en esta montaña, con quienes luchó en espíritu y a quienes derrotó en poder, eran \textit{reales;} eran sus mayores enemigos del sistema de Satania; no eran fantasmas de la imaginación, producidos por los desvaríos intelectuales de un mortal debilitado y hambriento que no pudiera distinguir la realidad de las visiones de una mente enajenada.

\par 
%\textsuperscript{(1493.3)}
\textsuperscript{134:8.4} Jesús pasó las tres últimas semanas de agosto y las tres primeras de septiembre en el Monte Hermón. Durante estas semanas, terminó la tarea mortal de alcanzar los círculos de comprensión mental y de control de la personalidad. Durante todo este período de comunión con su Padre celestial, el Ajustador interior también finalizó los servicios que se le habían asignado. La meta mortal de esta criatura terrestre fue alcanzada allí. Sólo quedaba por consumar la fase final de armonización entre su mente y el Ajustador.

\par 
%\textsuperscript{(1493.4)}
\textsuperscript{134:8.5} Después de más de cinco semanas de comunión ininterrumpida con su Padre Paradisiaco, Jesús estuvo absolutamente seguro de su naturaleza y de la certeza de su triunfo sobre los niveles materiales de manifestación de la personalidad en el espacio-tiempo. Creía plenamente en el predominio de su naturaleza divina sobre su naturaleza humana, y no dudó en afirmarlo.

\par 
%\textsuperscript{(1493.5)}
\textsuperscript{134:8.6} Hacia el final de su estancia en la montaña, Jesús pidió a su Padre que se le permitiera celebrar una conferencia con sus enemigos de Satania en su calidad de Hijo del Hombre, como Josué ben José. Esta petición le fue concedida. La gran tentación, la prueba del universo, tuvo lugar durante la última semana en el Monte Hermón. Satanás (en representación de Lucifer) y Caligastia, el Príncipe Planetario rebelde, estaban presentes junto a Jesús y fueron hechos plenamente visibles para él. Esta <<tentación>>\footnote{\textit{Tentaciones de Jesús}: Mt 4:3-11; Mc 1:13; Lc 4:2-13.}, esta prueba final de lealtad humana frente a las falsedades de las personalidades rebeldes, no tenía que ver con el alimento, los pináculos del templo o los actos presuntuosos. No tenía que ver con los reinos de este mundo, sino con la soberanía de un poderoso y glorioso universo. El simbolismo de vuestras escrituras estaba destinado a las épocas atrasadas del pensamiento infantil del mundo. Las generaciones siguientes deberían comprender la gran lucha que mantuvo el Hijo del Hombre aquel día memorable en el Monte Hermón.

\par 
%\textsuperscript{(1493.6)}
\textsuperscript{134:8.7} A las numerosas proposiciones y contraproposiciones de los emisarios de Lucifer, Jesús se limitó a responder: <<Que prevalezca la voluntad de mi Padre Paradisiaco, y a ti, mi hijo rebelde, que los Ancianos de los Días te juzguen divinamente. Soy tu Creador-padre; difícilmente puedo juzgarte con justicia, y ya has despreciado mi misericordia. Te confío a la decisión de los Jueces de un universo más grande>>.

\par 
%\textsuperscript{(1494.1)}
\textsuperscript{134:8.8} A todos los arreglos y artimañas sugeridos por Lucifer, a todas las proposiciones engañosas relativas a la donación de la encarnación, Jesús se limitó a responder: <<Que se haga la voluntad de mi Padre Paradisiaco>>. Cuando la dura prueba terminó, el serafín guardián que se mantenía apartado volvió al lado de Jesús y le aportó su servicio.

\par 
%\textsuperscript{(1494.2)}
\textsuperscript{134:8.9} Una tarde a finales del verano, en medio de los árboles y del silencio de la naturaleza, Miguel de Nebadon ganó la soberanía incontestable de su universo. Aquel día concluyó la tarea establecida para los Hijos Creadores de vivir hasta la saciedad la vida encarnada en la similitud de la carne mortal, en los mundos evolutivos del tiempo y del espacio. Esta proeza importantísima no se anunció al universo hasta el día de su bautismo, meses más tarde, pero en verdad tuvo lugar aquel día en la montaña. Cuando Jesús descendió de su estancia en el Monte Hermón, la rebelión de Lucifer en Satania y la secesión de Caligastia en Urantia estaban prácticamente arregladas. Jesús había pagado el último precio que se le exigía para obtener la soberanía de su universo, que en sí misma regula el estado de todos los rebeldes y determina que toda sublevación futura de este tipo (si llega a producirse alguna vez) puede ser tratada de manera sumaria y eficaz. En consecuencia, se puede observar que la llamada <<gran tentación>> de Jesús tuvo lugar algún tiempo antes de su bautismo, y no inmediatamente después.

\par 
%\textsuperscript{(1494.3)}
\textsuperscript{134:8.10} Al final de su estancia en la montaña, mientras Jesús descendía se encontró con Tiglat, que subía para acudir a la cita con los alimentos. Al indicarle que se volviera, solamente le dijo: <<El período de descanso ha terminado; tengo que volver a los asuntos de mi Padre>>. Se mantuvo silencioso y muy cambiado durante el viaje de regreso hacia Dan, donde se despidió del muchacho, regalándole el asno. Luego se dirigió hacia el sur por el mismo camino que había venido, hasta llegar a Cafarnaúm.

\section*{9. El período de espera}
\par 
%\textsuperscript{(1494.4)}
\textsuperscript{134:9.1} Ahora estaba próximo el final del verano, cerca de la época del día de la expiación y de la fiesta de los tabernáculos. El sábado, Jesús tuvo una reunión familiar en Cafarnaúm y al día siguiente partió para Jerusalén con Juan, el hijo de Zebedeo, dirigiéndose por el este del lago y por Gerasa, y descendiendo por el valle del Jordán. Aunque charló de vez en cuando con su compañero durante el camino, Juan notó un gran cambio en Jesús.

\par 
%\textsuperscript{(1494.5)}
\textsuperscript{134:9.2} Jesús y Juan se detuvieron en Betania para pasar la noche con Lázaro y sus hermanas, y a la mañana siguiente salieron temprano para Jerusalén. Estuvieron casi tres semanas en la ciudad y sus alrededores, al menos así lo hizo Juan. Muchos días, Juan fue solo a Jerusalén mientras Jesús deambulaba por las colinas cercanas y se dedicaba a numerosos períodos de comunión espiritual con su Padre celestial.

\par 
%\textsuperscript{(1494.6)}
\textsuperscript{134:9.3} Los dos asistieron a los oficios solemnes del día de la expiación. Juan estaba muy impresionado con las ceremonias de este día importante en el ritual religioso judío, pero Jesús permaneció como un espectador pensativo y silencioso. Para el Hijo del Hombre, este espectáculo resultaba lastimoso y patético. Lo veía todo como una falsa representación del carácter y de los atributos de su Padre celestial. Consideraba los acontecimientos de este día como una parodia de los hechos de la justicia divina y de la verdad de la misericordia infinita. Ardía en deseos de proclamar la auténtica verdad sobre el carácter amoroso y el comportamiento misericordioso de su Padre en el universo, pero su fiel Monitor le advirtió que su hora aún no había llegado. Sin embargo, aquella noche en Betania, Jesús dejó caer numerosos comentarios que perturbaron mucho a Juan, el cual nunca comprendió por completo el verdadero significado de lo que Jesús dijo en la conversación que tuvieron aquella noche.

\par 
%\textsuperscript{(1495.1)}
\textsuperscript{134:9.4} Jesús planeó quedarse con Juan toda la semana de la fiesta de los tabernáculos. Esta fiesta era la festividad anual de toda Palestina, la época de las vacaciones de los judíos. Aunque Jesús no participó en el júbilo de la ocasión, era evidente que le causaba placer y experimentaba satisfacción al contemplar cómo los jóvenes y los mayores se entregaban a la alegría y al gozo.

\par 
%\textsuperscript{(1495.2)}
\textsuperscript{134:9.5} A mediados de la semana de esta celebración y antes de que terminaran las festividades, Jesús se despidió de Juan diciendo que deseaba retirarse a las colinas, donde podría comulgar mejor con su Padre Paradisiaco. Juan hubiera querido acompañarlo, pero Jesús insistió para que se quedara hasta el fin de las festividades, diciendo: <<No se te exige que lleves el peso del Hijo del Hombre; sólo el vigilante debe estar en vela mientras la ciudad duerme en paz>>. Jesús no regresó a Jerusalén. Después de pasar casi una semana solo en las colinas cercanas a Betania, partió para Cafarnaúm. Camino del hogar, pasó un día y una noche a solas en las laderas del Gilboa\footnote{\textit{Gilboa, donde murió Saúl}: 1 Cr 10:1-5; 1 Sam 31:1-4; 2 Sam 1:1-10.}, cerca del lugar donde el rey Saúl se había quitado la vida; cuando llegó a Cafarnaúm, parecía más alegre que en el momento de dejar a Juan en Jerusalén.

\par 
%\textsuperscript{(1495.3)}
\textsuperscript{134:9.6} A la mañana siguiente, Jesús fue al arca que contenía sus efectos personales, que se habían quedado en el taller de Zebedeo, se puso su delantal y se presentó al trabajo, diciendo: <<Es conveniente que permanezca ocupado mientras espero a que llegue mi hora>>. Y trabajó varios meses en el astillero, al lado de su hermano Santiago, hasta enero del año siguiente. Después de este período de trabajo con Jesús, Santiago nunca más abandonó real y totalmente su fe en la misión de Jesús, a pesar de las dudas que oscurecían su comprensión del trabajo de la vida del Hijo del Hombre.

\par 
%\textsuperscript{(1495.4)}
\textsuperscript{134:9.7} Durante este período final de trabajo en el astillero, Jesús pasó la mayor parte de su tiempo acabando los interiores de algunas grandes embarcaciones. Ponía un gran cuidado en toda su obra manual, y parecía experimentar la satisfacción del logro humano cada vez que terminaba una pieza digna de elogio. Aunque no perdía el tiempo con pequeñeces, era un artesano cuidadoso cuando confeccionaba los detalles esenciales de un encargo determinado.

\par 
%\textsuperscript{(1495.5)}
\textsuperscript{134:9.8} A medida que pasaba el tiempo, llegaron rumores a Cafarnaúm sobre un tal Juan que predicaba mientras bautizaba a los penitentes en el Jordán\footnote{\textit{Predicación de Juan}: Mt 3:1-2,5-6.}. La predicación de Juan era: <<El reino de los cielos está cerca; arrepentíos y sed bautizados>>\footnote{\textit{El reino está cerca, arrepentíos}: Mt 3:2; Lc 3:3.}. Jesús escuchó estos informes a medida que Juan remontaba lentamente el valle del Jordán desde el vado del río más cercano a Jerusalén. Pero Jesús continuó trabajando construyendo barcas, hasta que Juan llegó río arriba a un lugar cercano a Pella, en el mes de enero del año siguiente, el año 26. Entonces dejó sus herramientas, declarando <<Ha llegado mi hora>>, y poco después se presentó ante Juan para ser bautizado.

\par 
%\textsuperscript{(1495.6)}
\textsuperscript{134:9.9} Un gran cambio se había producido en Jesús. De la gente que había disfrutado de sus visitas y servicios mientras recorría el país de arriba abajo, pocos reconocieron después, en el maestro público, a la misma persona que habían conocido y amado como individuo particular en años anteriores. Había una razón que impedía a sus primeros beneficiarios reconocerlo en su papel posterior como educador público lleno de autoridad: La transformación de su mente y de su espíritu se había estado desarrollando a lo largo de muchos años, y había finalizado durante la permanencia extraordinaria en el Monte Hermón.


\chapter{Documento 135. Juan el Bautista}
\par 
%\textsuperscript{(1496.1)}
\textsuperscript{135:0.1} JUAN el Bautista nació el 25 de marzo del año 7 a. de J. C., según la promesa que Gabriel le había hecho a Isabel en junio del año anterior. Durante cinco meses, Isabel guardó en secreto la visita de Gabriel\footnote{\textit{El secreto de Isabel}: Lc 1:24.}; cuando se lo dijo a su marido Zacarías, éste se quedó muy preocupado y sólo creyó plenamente en su relato después de tener un sueño insólito\footnote{\textit{El sueño de Zacarías}: Lc 1:11-23.}, unas seis semanas antes del nacimiento de Juan. Aparte de la visita de Gabriel a Isabel y del sueño de Zacarías, no hubo nada extraño ni sobrenatural en relación con el nacimiento de Juan el Bautista\footnote{\textit{Nacimiento de Juan el Bautista}: Lc 1:57.}.

\par 
%\textsuperscript{(1496.2)}
\textsuperscript{135:0.2} Al octavo día Juan fue circuncidado\footnote{\textit{La circuncisión de Juan}: Lc 1:59.} de acuerdo con la costumbre judía. Día tras día y año tras año, creció como un niño normal en el pueblecito conocido en aquella época con el nombre de Ciudad de Judá, a unos seis kilómetros al oeste de Jerusalén.

\par 
%\textsuperscript{(1496.3)}
\textsuperscript{135:0.3} El acontecimiento más sobresaliente del principio de la infancia de Juan fue la visita que hizo, en compañía de sus padres, a Jesús y a la familia de Nazaret. Esta visita tuvo lugar en el mes de junio del año 1 a. de J. C., cuando tenía poco más de seis años de edad.

\par 
%\textsuperscript{(1496.4)}
\textsuperscript{135:0.4} Después de regresar de Nazaret, los padres de Juan empezaron la educación sistemática del muchacho. En este pueblecito no había escuela de la sinagoga; sin embargo, como Zacarías era sacerdote, estaba bastante bien instruido e Isabel era mucho más culta que el promedio de las mujeres de Judea; ella también pertenecía al estado eclesiástico, puesto que era una descendiente de las <<hijas de Aarón>>. Como Juan era hijo único, sus padres consagraron mucho tiempo a su educación mental y espiritual. Zacarías sólo tenía cortos períodos de servicio en el templo de Jerusalén, de manera que dedicó una gran parte de su tiempo a instruir a su hijo.

\par 
%\textsuperscript{(1496.5)}
\textsuperscript{135:0.5} Zacarías e Isabel poseían una pequeña granja donde criaban ovejas. Apenas tenían para vivir con esta propiedad, pero Zacarías percibía un salario regular de los fondos del templo dedicados a los sacerdotes.

\section*{1. Juan se hace nazareo}
\par 
%\textsuperscript{(1496.6)}
\textsuperscript{135:1.1} No había escuela donde Juan pudiera graduarse a la edad de catorce años, pero sus padres habían elegido este año como el más apropiado para que pronunciara sus votos oficiales de nazareo. En consecuencia, Zacarías e Isabel llevaron a su hijo a En-Gedi, cerca del Mar Muerto. Esta era la sede de la hermandad nazarea en el sur, y es allí donde el muchacho fue debidamente admitido en esta orden de manera solemne y para toda la vida. Después de las ceremonias y de hacer los votos de abstenerse de toda bebida embriagadora\footnote{\textit{Juan se convierte en nazareo}: Lc 1:15.}, dejarse crecer el pelo y no tocar a los muertos, la familia se dirigió a Jerusalén donde Juan completó, delante del templo, las ofrendas que se exigían a los que pronunciaban los votos nazareos.

\par 
%\textsuperscript{(1496.7)}
\textsuperscript{135:1.2} Juan hizo los mismos votos vitalicios que habían efectuado sus ilustres predecesores, Sansón y el profeta Samuel. Un nazareo de por vida estaba considerado como una personalidad sacrosanta. Los judíos concedían a un nazareo casi el mismo respeto y veneración que al sumo sacerdote, lo que no era de extrañar, puesto que los nazareos consagrados para toda la vida eran las únicas personas, además de los sumos sacerdotes, a quienes se les permitía entrar en el santo de los santos del templo.

\par 
%\textsuperscript{(1497.1)}
\textsuperscript{135:1.3} Juan regresó de Jerusalén a su casa para cuidar las ovejas de su padre. Creció y se convirtió en un hombre fuerte con un carácter noble.

\par 
%\textsuperscript{(1497.2)}
\textsuperscript{135:1.4} A los dieciséis años, debido a unas lecturas acerca de Elías, Juan se quedó muy impresionado con el profeta del Monte Carmelo y decidió adoptar su manera de vestir\footnote{\textit{La forma de vestir de Juan}: Mt 3:4a; Mc 1:6a; Lc 1:80.}. A partir de aquel día, Juan llevó siempre una prenda de vestir cubierta de pelo con un cinturón de cuero. A los dieciséis años ya medía más de un metro ochenta y casi había alcanzado su pleno desarrollo. Con sus cabellos sueltos y su manera peculiar de vestir, resultaba en verdad un joven pintoresco. Sus padres esperaban grandes cosas de su único descendiente, un hijo de la promesa y nazareo para toda la vida.

\section*{2. La muerte de Zacarías}
\par 
%\textsuperscript{(1497.3)}
\textsuperscript{135:2.1} Después de una enfermedad que duró varios meses, Zacarías murió en julio del año 12, cuando Juan acababa de cumplir los dieciocho años. Fue un momento de gran desconcierto para Juan, pues el voto nazareo prohibía el contacto con los muertos, incluídos los de su propia familia. Aunque Juan había procurado cumplir con las restricciones de su voto respecto a la contaminación con los muertos, no estaba seguro de haberse sometido totalmente a los requisitos de la orden nazarea. Por esta razón, después del entierro de su padre fue a Jerusalén, y en el rincón nazareo del atrio de las mujeres ofreció los sacrificios requeridos para su purificación.

\par 
%\textsuperscript{(1497.4)}
\textsuperscript{135:2.2} En septiembre de este año, Isabel y Juan hicieron un viaje a Nazaret para visitar a María y a Jesús. Juan estaba casi decidido a empezar la obra de su vida, pero se sintió inducido, no sólo por las palabras de Jesús sino también por su ejemplo, a regresar al hogar, cuidar a su madre y esperar <<a que llegara la hora del Padre>>. Después de despedirse de Jesús y de María al final de esta agradable visita, Juan no volvió a ver a Jesús hasta el momento de su bautismo en el Jordán.

\par 
%\textsuperscript{(1497.5)}
\textsuperscript{135:2.3} Juan e Isabel regresaron a su hogar y empezaron a hacer planes para el futuro. Como Juan se negaba a aceptar la renta de sacerdote que le correspondía de los fondos del templo, al cabo de dos años lo habían perdido todo menos su casa; así pues, decidieron dirigirse hacia el sur con su rebaño de ovejas. En consecuencia, Juan se trasladó a Hebrón el verano en que cumplió los veinte años. Cuidó de sus ovejas en el llamado <<desierto de Judea>>\footnote{\textit{``Desierto'' de Judea}: Mt 3:1; Lc 3:2.}, cerca de un arroyo que era tributario de un torrente mayor, que desembocaba en el Mar Muerto a la altura de En-Gedi. La colonia de En-Gedi incluía no solamente a los nazareos consagrados de por vida o por un período determinado, sino también a otros numerosos pastores ascéticos que se congregaban en esta región con sus rebaños y fraternizaban con la hermandad de los nazareos. Vivían de la cría de las ovejas y de las donaciones que los ricos judíos hacían a la orden.

\par 
%\textsuperscript{(1497.6)}
\textsuperscript{135:2.4} A medida que pasaba el tiempo, Juan regresaba cada vez menos a Hebrón y visitaba En-Gedi con mayor frecuencia. Era tan absolutamente diferente a la mayoría de los nazareos, que le resultaba muy difícil fraternizar plenamente con la hermandad. Pero tenía un gran afecto por Abner, el jefe y dirigente reconocido de la colonia de En-Gedi.

\section*{3. La vida de un pastor}
\par 
%\textsuperscript{(1497.7)}
\textsuperscript{135:3.1} A lo largo del valle de este pequeño arroyo, Juan construyó no menos de una docena de refugios de piedra y de corrales para la noche, a base de piedras apiladas, en los cuales podía vigilar y proteger a sus rebaños de ovejas y cabras. La vida de pastor le dejaba mucho tiempo libre para pensar. Hablaba mucho con Ezda, un niño huérfano de Bet-sur, a quien en cierto modo había adoptado, y que cuidaba de los rebaños cuando Juan iba a Hebrón para ver a su madre y vender ovejas, y también cuando bajaba a En-Gedi para los oficios del sábado. Juan y el muchacho vivían de manera muy simple, alimentándose de carne de cordero, leche de cabra, miel silvestre y las langostas comestibles de esta región\footnote{\textit{La dieta de Juan}: Mt 3:4; Mc 1:6.}. Esta dieta habitual la completaban con las provisiones que traían de Hebrón y En-Gedi de vez en cuando.

\par 
%\textsuperscript{(1498.1)}
\textsuperscript{135:3.2} Isabel mantenía informado a Juan de los asuntos de Palestina y del mundo. Él estaba cada vez más profundamente convencido de que se acercaba rápidamente el momento en que el antiguo orden de cosas iba a terminar, de que él se convertiría en el precursor de la llegada de una nueva era, <<el reino de los cielos>>. Este rudo pastor tenía una gran predilección por los escritos del profeta Daniel. Había leído mil veces la descripción que Daniel hacía de la gran estatua\footnote{\textit{La ``gran estatua'' de Daniel}: Dn 2:31-33.}; Zacarías le había dicho que ésta representaba la historia de los grandes reinos del mundo, empezando por Babilonia, luego Persia, Grecia y finalmente Roma. Juan se daba cuenta de que Roma ya estaba compuesta por unos pueblos y razas tan políglotas, que nunca podría convertirse en un imperio con unos cimientos sólidos y firmemente consolidados. Creía que Roma ya estaba entonces dividida en Siria, Egipto, Palestina y otras provincias. Luego continuó leyendo que <<en los días de estos reyes, el Dios del cielo establecerá un reino que nunca será destruido. Y este reino no será entregado a otros pueblos, sino que romperá en pedazos y destruirá a todos esos reinos, y subsistirá para siempre>>\footnote{\textit{Reino eterno}: Dn 2:44.}. <<Y le entregaron un dominio, gloria y un reino, para que todos los pueblos, naciones y lenguas le sirvieran. Su dominio es un dominio perpetuo que nunca perecerá, y su reino nunca será destruido>>. <<Y el reino, el dominio y la grandeza del reino que están por debajo de todos los cielos, serán entregados al pueblo de los santos del Altísimo, cuyo reino es un reino eterno, y todos los dominios le servirán y le obedecerán>>\footnote{\textit{Dominio sin fni}: Dn 7:14. \textit{Dar el reino a los Altísimos}: Dn 7:27.}.

\par 
%\textsuperscript{(1498.2)}
\textsuperscript{135:3.3} Juan nunca fue completamente capaz de elevarse por encima de la confusión que le producía lo que había oído decir a sus padres sobre Jesús y estos pasajes que leía en las escrituras. En el libro de Daniel leía: <<Tuve unas visiones nocturnas, y contemplé a alguien semejante al Hijo del Hombre que venía con las nubes del cielo, y le entregaron un dominio, la gloria y un reino>>\footnote{\textit{Visión de la venida del Hijo del Hombre}: Dn 7:13-14.}. Pero estas palabras del profeta no concordaban con lo que sus padres le habían enseñado. Su conversación con Jesús, cuando fue a visitarlo a la edad de dieciocho años, tampoco se correspondía con estas declaraciones de las escrituras. A pesar de esta confusión, su madre le aseguró todo el tiempo que duró su perplejidad que su primo lejano, Jesús de Nazaret, era el verdadero Mesías, que había venido para sentarse en el trono de David, y que él (Juan) se convertiría en su primer precursor y en su principal apoyo.

\par 
%\textsuperscript{(1498.3)}
\textsuperscript{135:3.4} Debido a todo lo que había escuchado sobre el vicio y la perversidad de Roma y el libertinaje y la esterilidad moral del imperio, por todo lo que había oído de las maldades de Herodes Antipas y de los gobernadores de Judea, Juan tendía a creer que el final de la era estaba próximo. A este noble y rudo hijo de la naturaleza le parecía que el mundo estaba maduro para el final de la era del hombre y el amanecer de la era nueva y divina ---el reino de los cielos. En el corazón de Juan creció el sentimiento de que iba a ser el último de los antiguos profetas y el primero de los nuevos. Vibraba honradamente con el impulso creciente de salir fuera y proclamar a todos los hombres: <<¡Arrepentíos! ¡Poneos bien con Dios! Disponeos para el fin; preparaos para la aparición del orden nuevo y eterno de las cosas terrestres, el reino de los cielos>>.

\section*{4. La muerte de Isabel}
\par 
%\textsuperscript{(1499.1)}
\textsuperscript{135:4.1} El 17 de agosto del año 22, cuando Juan tenía veintiocho años, su madre falleció repentinamente. Como los amigos de Isabel conocían las restricciones nazareas respecto al contacto con los muertos, incluídos los de la propia familia, hicieron todos los arreglos para el entierro de Isabel antes de mandar a buscar a Juan. Cuando recibió la noticia de la muerte de su madre, Juan ordenó a Ezda que llevara sus rebaños a En-Gedi y partió para Hebrón.

\par 
%\textsuperscript{(1499.2)}
\textsuperscript{135:4.2} Al regresar a En-Gedi después del funeral de su madre, donó sus rebaños a la hermandad y se apartó del mundo exterior durante una temporada para ayunar y orar. Juan sólo conocía los métodos antiguos para acercarse a la divinidad; sólo conocía las historias de Elías, Samuel y Daniel. Elías era su ideal como profeta. Elías era el primer educador de Israel que fue considerado como un profeta, y Juan creía sinceramente que él mismo sería el último de este largo e ilustre linaje de mensajeros del cielo.

\par 
%\textsuperscript{(1499.3)}
\textsuperscript{135:4.3} Juan vivió en En-Gedi durante dos años y medio, y persuadió a la mayoría de la hermandad de que <<se acercaba el fin de la era>>, de que <<el reino de los cielos estaba a punto de aparecer>>. Todas sus primeras enseñanzas estaban basadas en la idea y el concepto corrientes que tenían los judíos de un Mesías prometido a la nación judía para liberarla de la dominación de sus gobernantes gentiles.

\par 
%\textsuperscript{(1499.4)}
\textsuperscript{135:4.4} Durante todo este período, Juan leyó asiduamente los escritos sagrados que encontró en el hogar de los nazareos de En-Gedi. Le impresionó de manera especial Isaías y también Malaquías, el último de los profetas hasta aquel momento. Leyó y releyó los últimos cinco capítulos de Isaías, y creyó en aquellas profecías. Entonces se puso a leer en Malaquías: <<He aquí, os enviaré a Elías, el profeta, antes de que llegue el gran y terrible día del Señor; él orientará el corazón de los padres hacia los hijos y el corazón de los hijos hacia sus padres, para que yo no venga a castigar la Tierra con una maldición>>\footnote{\textit{Envío previo de Elías}: Mal 4:5-6.}. Esta promesa del regreso de Elías, hecha por Malaquías, fue lo único que impidió a Juan salir fuera a predicar sobre el reino venidero, y exhortar a sus compatriotas judíos a evitar la ira por venir. Juan estaba maduro para proclamar el mensaje del reino venidero, pero esta esperanza de que Elías regresaría lo retuvo durante más de dos años. Sabía que él no era Elías. ¿Qué quería decir Malaquías? ¿Su profecía era literal o figurada? ¿Cómo podía saber la verdad? Finalmente se atrevió a pensar que, puesto que el primer profeta se había llamado Elías, el último también debía ser conocido finalmente por el mismo nombre. Sin embargo, tenía dudas, unas dudas suficientes como para impedirle llamarse a sí mismo Elías.

\par 
%\textsuperscript{(1499.5)}
\textsuperscript{135:4.5} Fue la influencia de Elías la que hizo que Juan adoptara sus métodos de ataque directo y áspero contra los pecados y los vicios de sus contemporáneos. Intentó vestirse como Elías y procuró hablar como Elías; en todo su aspecto exterior se parecía al antiguo profeta. Era un hijo de la naturaleza igual de fuerte y pintoresco, un predicador de la rectitud igual de intrépido y temerario. Juan no era analfabeto, conocía bien las sagradas escrituras judías, pero tenía poca cultura. Era un pensador de ideas claras, un orador poderoso y un acusador ardiente. No se puede decir que fuera un ejemplo para su época, pero sí era una censura elocuente.

\par 
%\textsuperscript{(1499.6)}
\textsuperscript{135:4.6} Finalmente elaboró un método para proclamar la nueva era, el reino de Dios. Decidió que se iba a convertir en el precursor del Mesías. Barrió todas las dudas y partió de En-Gedi, un día de marzo del año 25, para empezar su corta pero brillante carrera como predicador público.

\section*{5. El reino de Dios}
\par 
%\textsuperscript{(1500.1)}
\textsuperscript{135:5.1} Para comprender el mensaje de Juan hay que tener en cuenta el estado del pueblo judío en el momento en que apareció en escena. Durante cerca de cien años, todo Israel se había encontrado en un laberinto; no acertaban a explicar su continuo sometimiento a los soberanos gentiles. ¿No había enseñado Moisés que la rectitud siempre era recompensada con la prosperidad y el poder? ¿Acaso no eran el pueblo elegido de Dios? ¿Por qué el trono de David estaba abandonado y vacante? A la luz de las doctrinas mosaicas y de los preceptos de los profetas, a los judíos les resultaba difícil explicar su prolongada aflicción nacional.

\par 
%\textsuperscript{(1500.2)}
\textsuperscript{135:5.2} Unos cien años antes de los tiempos de Jesús y de Juan, una nueva escuela de educadores religiosos había surgido en Palestina, la de los apocalípticos. Estos nuevos instructores desarrollaron un sistema de creencias que explicaba los sufrimientos y la humillación de los judíos sobre la base de que estaban pagando las consecuencias de los pecados de la nación. Recurrían a las razones bien conocidas destinadas a explicar la cautividad en Babilonia y en otros lugares en tiempos pasados. Pero, según enseñaban los apocalípticos, Israel debía recobrar el ánimo; los días de su aflicción casi habían terminado; el castigo disciplinario del pueblo elegido de Dios estaba llegando a su fin; la paciencia de Dios con los extranjeros gentiles se estaba agotando. El final del poder de Roma era sinónimo del final de la era y, en cierto sentido, del fin del mundo. Estos nuevos educadores se apoyaban ampliamente en las predicciones de Daniel, y en consecuencia enseñaban que la creación estaba a punto de entrar en su etapa final; los reinos de este mundo estaban a punto de convertirse en el reino de Dios. Para la mente de los judíos de aquella época, éste era el significado de la expresión <<el reino de los cielos>> que figura en todas las enseñanzas de Juan y de Jesús. Para los judíos de Palestina, la frase <<el reino de los cielos>> sólo tenía un significado: un estado absolutamente justo en el que Dios (el Mesías) gobernaría las naciones de la Tierra con la misma perfección de poder con que gobernaba en el cielo\footnote{\textit{El Mesías gobierna cielo y tierra}: Mt 6:10; Lc 11:2.} ---<<Hágase tu voluntad en la Tierra como en el cielo>>.

\par 
%\textsuperscript{(1500.3)}
\textsuperscript{135:5.3} En los tiempos de Juan, todos los judíos se preguntaban con ansiedad: <<¿Cuánto tardará en llegar el reino?>> Existía el sentimiento general de que el poder de las naciones gentiles se acercaba a su fin. En todo el mundo judío estaba presente la viva esperanza y la expectación ansiosa de que la consumación del deseo de todos los siglos se produciría durante la vida de aquella generación.

\par 
%\textsuperscript{(1500.4)}
\textsuperscript{135:5.4} Aunque había grandes diferencias de opinión entre los judíos sobre la naturaleza del reino venidero, todos compartían la creencia de que el acontecimiento era inminente, de que estaba próximo, e incluso a punto de suceder. Muchos de los que leían el Antiguo Testamento de manera literal esperaban con expectación a un nuevo rey en Palestina, a una nación judía regenerada, liberada de sus enemigos y gobernada por el sucesor del rey David, el Mesías, que sería rápidamente reconocido como el soberano justo y legítimo del mundo entero. Otro grupo de judíos piadosos, más pequeño, tenía una visión muy distinta de este reino de Dios. Enseñaban que el reino venidero no era de este mundo, que el mundo se acercaba a su fin evidente, y que <<un nuevo cielo y una nueva Tierra>>\footnote{\textit{Un nuevo cielo y una nueva Tierra}: Is 65:17; Is 66:22; Ap 21:1.} anunciarían el establecimiento del reino de Dios; que este reino sería un dominio perpetuo, que se pondría fin al pecado y que los ciudadanos del nuevo reino se volverían inmortales disfrutando de esta felicidad sin fin.

\par 
%\textsuperscript{(1500.5)}
\textsuperscript{135:5.5} Todos estaban de acuerdo en que una purga drástica o una corrección purificadora tenía que preceder necesariamente al establecimiento del nuevo reino en la Tierra. Los que se adherían al sentido literal enseñaban que seguiría una guerra mundial que destruiría a todos los incrédulos, mientras que los fieles conseguirían una victoria universal y eterna. Los espiritualistas enseñaban que el reino se anunciaría con el gran juicio de Dios, que relegaría a los inicuos a su juicio de castigo y de destrucción final bien merecido, y al mismo tiempo elevaría a los santos creyentes del pueblo elegido a los tronos de honor y de autoridad junto al Hijo del Hombre, el cual reinaría en nombre de Dios sobre las naciones redimidas. Este último grupo creía incluso que muchos gentiles piadosos podrían ser admitidos en la hermandad del nuevo reino.

\par 
%\textsuperscript{(1501.1)}
\textsuperscript{135:5.6} Algunos judíos mantenían la opinión de que Dios quizás podría establecer este nuevo reino mediante una intervención directa y divina, pero la gran mayoría creía que interpondría a un intermediario que lo representara, el Mesías. Y éste era el único significado posible que la palabra Mesías podía tener en la mente de los judíos de la generación de Juan y de Jesús. \textit{Mesías} no podía referirse de ninguna manera a alguien que se limitara a enseñar la voluntad de Dios o a proclamar la necesidad de una vida de rectitud. A todas las personas santas de este tipo, los judíos les daban el nombre de \textit{profetas}. El Mesías debía ser más que un profeta; el Mesías debía traer el establecimiento del nuevo reino, el reino de Dios. Nadie que dejara de hacer esto podía ser el Mesías en el sentido tradicional judío.

\par 
%\textsuperscript{(1501.2)}
\textsuperscript{135:5.7} ¿Quién sería este Mesías? De nuevo, los educadores judíos tenían opiniones diferentes. Los más viejos se aferraban a la doctrina del hijo de David. Los más jóvenes enseñaban que, puesto que el nuevo reino era un reino celestial, el nuevo soberano podría ser también una personalidad divina, alguien que hubiera estado sentado mucho tiempo a la diestra de Dios en el cielo. Por muy extraño que parezca, los que concebían así al soberano del nuevo reino no lo imaginaban como un Mesías humano, no como un simple \textit{hombre}, sino como <<el Hijo del Hombre>>\footnote{\textit{El Hijo del Hombre}: Ez 2:1,3,6,8; 3:1-4,10,17; Dn 7:13,14; Mt 8:20; Mc 2:10; Lc 5:24; Jn 1:51; Ap 1:13; 14:14; 1 Hen 46:1-6; 48:1-7; 60:10; 62:1-14; 63:11; 69:26-29; 70:1-2; 71:14,16.} ---un Hijo de Dios--- un Príncipe celestial, que había estado mucho tiempo esperando asumir así la soberanía de la Tierra renovada. Éste era el trasfondo religioso del mundo judío cuando Juan salió a proclamar: <<¡Arrepentíos, porque el reino de los cielos está cerca!>>\footnote{\textit{Juan predica el arrepentimiento}: Mt 3:2; Mc 1:4.}

\par 
%\textsuperscript{(1501.3)}
\textsuperscript{135:5.8} Está claro pues que el anuncio que Juan\footnote{\textit{Anuncio de Juan}: Mt 3:3; Mc 1:2-8; Lc 3:2-6.} hacía del reino venidero tenía no menos de media docena de significados diferentes en la mente de los que escuchaban su predicación apasionada. Pero cualquiera que fuera el significado que atribuían a las palabras empleadas por Juan, cada uno de estos diversos grupos que esperaban el reino de los judíos estaba intrigado por las proclamaciones de este predicador de la rectitud y del arrepentimiento, sincero, entusiasta y tosco, pero eficaz, que exhortaba tan solemnemente a sus oyentes a <<huir de la ira venidera>>\footnote{\textit{Huir de la ira venidera}: Mt 3:7; Lc 3:7.}.

\section*{6. Juan empieza a predicar}
\par 
%\textsuperscript{(1501.4)}
\textsuperscript{135:6.1} A principios del mes de marzo del año 25, Juan rodeó la costa occidental del Mar Muerto y subió por el río Jordán hasta llegar frente a Jericó, al antiguo vado por el que pasaron Josué y los hijos de Israel cuando entraron por primera vez en la tierra prometida. Cruzó al otro lado del río, se instaló cerca de la entrada del vado y empezó a predicar\footnote{\textit{Juan comienza a predicar}: Mt 3:1; Mc 1:4,7-8; Lc 3:2-3.} a la gente que atravesaba el río en ambas direcciones. Éste era el cruce más frecuentado de todos los que tenía el Jordán.

\par 
%\textsuperscript{(1501.5)}
\textsuperscript{135:6.2} Todos los que oían a Juan se daban cuenta de que era más que un predicador. La gran mayoría de los que escuchaban a este hombre extraño que había surgido del desierto de Judea se alejaban con la creencia de que habían oído la voz de un profeta. No es de extrañar que el alma de estos judíos, cansados y esperanzados, se agitara profundamente ante un fenómeno como éste. En toda la historia judía, los piadosos hijos de Abraham nunca habían deseado tanto <<el consuelo de Israel>>\footnote{\textit{El consuelo de Israel}: Lc 2:25.} ni esperado más ardientemente <<la restauración del reino>>\footnote{\textit{La restauración del reino}: Hch 1:6.}. En toda la historia judía, el mensaje de Juan <<el reino de los cielos está cerca>>\footnote{\textit{El reino de los cielos está cerca}: Mt 3:2; Mc 1:15.} nunca hubiera podido ejercer un impacto tan profundo y universal como en el momento en que apareció tan misteriosamente en la orilla de este vado meridional del Jordán.

\par 
%\textsuperscript{(1502.1)}
\textsuperscript{135:6.3} Era pastor, como Amós. Estaba vestido como el antiguo Elías; fulminaba con sus amonestaciones y lanzaba sus advertencias con el <<espíritu y el poder de Elías>>\footnote{\textit{El espíritu y el poder de Elías}: Lc 1:17.}. No es de sorprender que este extraño predicador creara una poderosa conmoción en toda Palestina, a medida que los viajeros llevaban por todas partes la noticia de su predicación al borde del Jordán.

\par 
%\textsuperscript{(1502.2)}
\textsuperscript{135:6.4} El trabajo de este predicador nazareo contenía además una característica \textit{nueva:} bautizaba a cada uno de sus creyentes en el Jordán <<para la remisión de los pecados>>\footnote{\textit{Bautizo para remisión de los pecados}: Mt 3:5-6; Mc 1:4-5; Lc 3:3.}. Aunque el bautismo no era una ceremonia nueva para los judíos, nunca habían visto emplearlo como Juan lo hacía ahora. Durante mucho tiempo, habían tenido la costumbre de bautizar así a los prosélitos gentiles para admitirlos en la comunidad del patio exterior del templo, pero nunca se había pedido a los mismos judíos que se sometieran al bautismo de arrepentimiento. Sólo transcurrieron quince meses entre el momento en que Juan empezó a predicar y a bautizar, y su arresto y encarcelamiento a instigación de Herodes Antipas, pero en este corto período de tiempo bautizó a mucho más de cien mil penitentes.

\par 
%\textsuperscript{(1502.3)}
\textsuperscript{135:6.5} Juan predicó\footnote{\textit{Predicación de Juan}: Jn 1:28.} cuatro meses en el vado de Betania, antes de partir hacia el norte remontando el Jordán. Decenas de miles de oyentes, algunos por curiosidad, pero muchos con sinceridad y seriedad, vinieron a escucharlo de todas partes de Judea, Perea y Samaria. Unos cuantos vinieron incluso desde Galilea.

\par 
%\textsuperscript{(1502.4)}
\textsuperscript{135:6.6} En mayo de este año, mientras que aún se demoraba en el vado de Betania, los sacerdotes y los levitas enviaron una delegación para preguntar a Juan si pretendía ser el Mesías, y en virtud de qué autoridad predicaba\footnote{\textit{Las preguntas de la delegación}: Jn 1:19-27.}. Juan respondió a estos interrogadores diciendo: <<Id a decir a vuestros jefes que habéis oído `la voz de aquel que clama en el desierto', como lo expresó el profeta diciendo: `Preparad el camino del Señor, enderezad una senda para nuestro Dios. Todo valle será colmado, toda montaña y toda colina serán allanadas; el terreno accidentado se volverá llano, y los lugares rocosos se convertirán en un valle liso; y todo el género humano verá la salvación de Dios'.>>\footnote{\textit{Respuesta de Juan}: Is 40:3-5; Mt 3:3; Mc 1:2-3; Lc 3:4-5; Jn 1:23.}

\par 
%\textsuperscript{(1502.5)}
\textsuperscript{135:6.7} Juan era un predicador heroico, pero carente de tacto\footnote{\textit{El mensaje sin tacto de Juan}: Mt 3:7-10; Lc 3:7-9.}. Un día que estaba predicando y bautizando en la orilla occidental del Jordán, un grupo de fariseos y cierto número de saduceos se adelantaron y se presentaron para ser bautizados. Antes de conducirlos hasta el agua, Juan se dirigió a ellos como grupo diciendo: <<¿Quién os ha avisado para que huyáis de la ira venidera, como las víboras ante el fuego? Yo os bautizaré, pero os advierto que tenéis que producir los frutos dignos de un arrepentimiento sincero, si queréis recibir la remisión de vuestros pecados. No me digáis que Abraham es vuestro padre. Os declaro que de estas doce piedras que están ante vosotros, Dios es capaz de hacer surgir unos hijos dignos de Abraham. El hacha ya está puesta en las raíces mismas de los árboles. Todo árbol que no dé buen fruto está destinado a ser cortado y echado al fuego>>\footnote{\textit{Dar buen fruto}: Mt 7:16-20; Mt 12:33-34; Lc 3:9-10; Lc 6:43-44; Jn 15:2.}. (Las doce piedras a las que se refería eran las famosas piedras conmemorativas erigidas por Josué para recordar el paso de las <<doce tribus>> por este mismo vado cuando entraron por primera vez en la tierra prometida\footnote{\textit{Doce piedras conmemorativas}: Jos 4:1-9.}.)

\par 
%\textsuperscript{(1502.6)}
\textsuperscript{135:6.8} Juan daba clases a sus discípulos\footnote{\textit{Juan daba clases a sus discípulos}: Lc 3:10-11.}, en el transcurso de las cuales los instruía sobre los detalles de su nueva vida y procuraba responder a sus numerosas preguntas. Aconsejaba a los educadores que enseñaran el espíritu así como la letra de la ley. Ordenaba a los ricos que alimentaran a los pobres. A los recaudadores de impuestos les decía: <<No percibáis más de lo que os han asignado>>\footnote{\textit{Lección para los recaudadores}: Lc 3:12-13.}. A los soldados les decía: <<No ejerzáis la violencia y no exijáis nada injustamente ---contentaos con vuestro salario>>\footnote{\textit{Lección para los soldados}: Lc 3:14.}. Y a todo el mundo aconsejaba: <<Preparaos para el final de la era--- el reino de los cielos está cerca>>\footnote{\textit{Lección para todos}: Mt 3:2; Mc 1:15.}.

\section*{7. Juan viaja hacia el norte}
\par 
%\textsuperscript{(1503.1)}
\textsuperscript{135:7.1} Juan tenía todavía ideas confusas sobre el reino venidero y su rey\footnote{\textit{Juan confuso acerca del reino}: Jn 1:31a.}. Cuanto más predicaba, más confuso se sentía, pero esta incertidumbre intelectual sobre la naturaleza del reino venidero nunca disminuyó en lo más mínimo su convencimiento de que la aparición inmediata del reino era indudable. Juan podía estar confuso en su mente, pero nunca en su espíritu. No tenía ninguna duda sobre la llegada del reino, pero distaba de estar seguro si Jesús iba a ser o no el soberano de este reino. Cuando Juan se aferraba a la idea del restablecimiento del trono de David, las enseñanzas de sus padres de que Jesús, nacido en la Ciudad de David, iba a ser el libertador tanto tiempo esperado, le parecían consistentes. Pero en los momentos en que se inclinaba más hacia la doctrina de un reino espiritual y el final de la era temporal en la Tierra, tenía grandes dudas sobre el papel que Jesús jugaría en tales acontecimientos. A veces lo ponía todo en tela de juicio, pero no por mucho tiempo. Deseaba realmente poder hablar de todo esto con su primo, pero eso era contrario al acuerdo establecido entre ellos.

\par 
%\textsuperscript{(1503.2)}
\textsuperscript{135:7.2} A medida que Juan viajaba hacia el norte, pensaba mucho en Jesús. Se detuvo en más de una docena de lugares mientras remontaba el Jordán. Fue en Adán donde, en respuesta a la pregunta directa que sus discípulos le hicieron <<¿Eres tú el Mesías?>>, hizo referencia por primera vez a <<otro que ha de venir después de mí>>\footnote{\textit{Juan como precursor del Mesías}: Mt 3:11-12; Mc 1:7-8; Lc 3:16-17; Jn 1:26-27.}. Y continuó diciendo: <<Después de mí vendrá uno que es más grande que yo, ante quien no soy digno de inclinarme para desatar las correas de sus sandalias. Yo os bautizo con agua, pero él os bautizará con el Espíritu Santo. Tiene en su mano la pala para limpiar completamente su era; recogerá el trigo en su granero, pero quemará la paja con el fuego del juicio>>\footnote{\textit{Juan niega ser el Mesías}: Mt 3:11; Mc 1:7-8; Lc 3:15-17; Jn 1:24-26.}.

\par 
%\textsuperscript{(1503.3)}
\textsuperscript{135:7.3} En contestación a las preguntas de sus discípulos, Juan continuó ampliando sus enseñanzas, añadiendo día tras día más información útil y confortante, en comparación con su enigmático mensaje inicial: <<Arrepentíos y sed bautizados>>\footnote{\textit{Arrepentíos y bautizaos}: Lc 3:3.}. Por esta época empezó a llegar mucha gente de Galilea y de la Decápolis. Decenas de creyentes sinceros permanecían, día tras día, junto a su adorado maestro.

\section*{8. Encuentro de Jesús y de Juan}
\par 
%\textsuperscript{(1503.4)}
\textsuperscript{135:8.1} En el mes de diciembre del año 25, cuando Juan llegó a las proximidades de Pella\footnote{\textit{Juan llega a Pella}: Jn 1:28.} en su viaje remontando el Jordán, su fama se había extendido por toda Palestina, y su obra se había convertido en el tema principal de conversación de todas las ciudades cercanas al Lago de Galilea. Jesús había hablado favorablemente del mensaje de Juan, lo que hizo que muchos habitantes de Cafarnaúm se unieran al culto de arrepentimiento y de bautismo de Juan. Santiago y Juan, los hijos pescadores de Zebedeo, habían ido al vado en diciembre, poco después de que Juan se instalara a predicar cerca de Pella, y se habían ofrecido para ser bautizados. Iban a ver a Juan una vez por semana, y traían a Jesús las noticias directas y recientes de la obra del evangelista.

\par 
%\textsuperscript{(1503.5)}
\textsuperscript{135:8.2} Santiago y Judá, los hermanos de Jesús, habían hablado de ir a ver a Juan para ser bautizados. Ahora que Judá había venido a Cafarnaúm para los oficios del sábado, después de escuchar el discurso de Jesús en la sinagoga, tanto él como Santiago decidieron pedirle consejo con respecto a sus planes. Esto sucedía el sábado 12 de enero del año 26 por la noche. Jesús les pidió que aplazaran la discusión hasta el día siguiente, y entonces les daría su respuesta. Durmió muy poco aquella noche, pues estuvo en estrecha comunión con el Padre celestial. Había acordado almorzar con sus hermanos a mediodía y aconsejarles con respecto al bautismo de Juan. Aquel domingo por la mañana, Jesús estaba trabajando como de costumbre en el astillero. Santiago y Judá habían llegado con el almuerzo y lo estaban esperando en el almacén de madera, pues aún no había llegado la hora del descanso de mediodía, y sabían que Jesús era muy formal en estas cuestiones.

\par 
%\textsuperscript{(1504.1)}
\textsuperscript{135:8.3} Poco antes de la pausa del mediodía, Jesús dejó sus herramientas, se quitó su delantal de trabajo y anunció simplemente a los tres trabajadores que estaban con él en el taller: <<Ha llegado mi hora>>. Fue en busca de sus hermanos Santiago y Judá, repitiendo: <<Ha llegado mi hora ---vamos a ver a Juan>>. Partieron inmediatamente para Pella, tomándose el almuerzo mientras viajaban. Esto ocurría el domingo 13 de enero. Se detuvieron para pasar la noche en el valle del Jordán y llegaron al lugar donde Juan estaba bautizando hacia el mediodía del día siguiente\footnote{\textit{Jesús viaja a Pella}: Mc 1:9; Lc 3:21; Jn 1:29.}.

\par 
%\textsuperscript{(1504.2)}
\textsuperscript{135:8.4} Juan acababa de empezar a bautizar a los candidatos del día. Decenas de penitentes formaban cola esperando su turno, cuando Jesús y sus dos hermanos ocuparon su lugar en esta fila de hombres y mujeres sinceros que se habían hecho creyentes en la predicación de Juan sobre el reino venidero. Juan había preguntado por Jesús a los hijos de Zebedeo. Estaba enterado de los comentarios de Jesús sobre su predicación, y día tras día esperaba verlo llegar a aquel lugar, pero no había imaginado encontrarlo en la cola de los candidatos al bautismo.

\par 
%\textsuperscript{(1504.3)}
\textsuperscript{135:8.5} Como estaba absorto con los detalles de bautizar rápidamente a un número tan elevado de conversos, Juan no levantó los ojos para ver a Jesús hasta que el Hijo del Hombre no estuvo delante de él. Cuando Juan reconoció a Jesús, interrumpió las ceremonias unos momentos mientras saludaba a su primo carnal y le preguntaba: <<Pero ¿por qué bajas hasta el agua para saludarme?>> Jesús respondió: <<Para someterme a tu bautismo>>. Y Juan replicó: <<Pero soy yo quien necesita ser bautizado por ti. ¿Por qué vienes hasta mí?>> Y Jesús le susurró a Juan: <<Sé indulgente conmigo ahora, pues conviene que demos este ejemplo a mis hermanos que están aquí conmigo, y para que la gente pueda saber que ha llegado mi hora>>\footnote{\textit{Objeciones de Juan}: Mt 3:14-15.}.

\par 
%\textsuperscript{(1504.4)}
\textsuperscript{135:8.6} La voz de Jesús tenía un tono firme y terminante. Juan temblaba de emoción al prepararse para bautizar a Jesús de Nazaret en el Jordán, a mediodía del lunes 14 de enero del año 26. Así fue como Juan bautizó a Jesús y a sus dos hermanos, Santiago y Judá. Y cuando Juan hubo bautizado a los tres, despidió a los demás hasta el día siguiente, anunciando que reanudaría los bautismos al mediodía. Mientras la gente se marchaba, los cuatro hombres, que aún permanecían en el agua, oyeron un sonido extraño, y acto seguido se produjo una aparición durante unos instantes inmediatamente por encima de la cabeza de Jesús, y oyeron una voz que decía: <<Éste es mi hijo amado en quien me siento muy complacido>>\footnote{\textit{Bautismo de Jesús y voz divina}: Mt 3:16-17; Mc 1:10-11; Lc 3:22; Jn 1:32-34.}. Un gran cambio se produjo en el semblante de Jesús; salió del agua en silencio y se despidió de ellos, dirigiéndose hacia las colinas del este. Nadie lo volvió a ver durante cuarenta días\footnote{\textit{Cuarenta días solo}: Mt 4:1-11; Mc 1:12-13; Lc 4:1-13.}.

\par 
%\textsuperscript{(1504.5)}
\textsuperscript{135:8.7} Juan siguió a Jesús la distancia suficiente como para contarle la historia de la visita de Gabriel a su madre antes de que nacieran los dos\footnote{\textit{Visita de Gabriel a Isabel}: Lc 1:11-19,24-25.}, tal como lo había escuchado tantas veces de labios de su madre. Dejó que Jesús continuara su camino, después de haberle dicho: <<Ahora sé con seguridad que tú eres el Libertador>>\footnote{\textit{Ahora sé con seguridad}: Jn 1:34.}. Pero Jesús no respondió.

\section*{9. Cuarenta días de predicación}
\par 
%\textsuperscript{(1505.1)}
\textsuperscript{135:9.1} Cuando Juan regresó junto a sus discípulos (ahora tenía unos veinticinco o treinta que vivían constantemente con él), los encontró conversando seriamente, discutiendo lo que acababa de suceder en relación con el bautismo de Jesús. Se quedaron mucho más asombrados cuando Juan les contó ahora la historia de la visita de Gabriel a María antes del nacimiento de Jesús\footnote{\textit{Visita de Gabriel a María}: Lc 1:26-38.}, y también el hecho de que Jesús no le dijera ni una palabra después de hablarle de ello. Aquella noche no llovió, y este grupo de treinta personas o más conversó largamente bajo la noche estrellada. Se preguntaban dónde había ido Jesús y cuándo lo volverían a ver.

\par 
%\textsuperscript{(1505.2)}
\textsuperscript{135:9.2} Después del incidente de este día, la predicación de Juan adquirió un nuevo tono de certidumbre en sus proclamaciones respecto al reino venidero y al Mesías esperado. Estos cuarenta días de espera, aguardando el regreso de Jesús, fueron un período de tensión. Pero Juan continuó predicando con gran fuerza, y sus discípulos empezaron a predicar aproximadamente por esta época a las multitudes desbordantes que se amontonaban alrededor de Juan a orillas del Jordán.

\par 
%\textsuperscript{(1505.3)}
\textsuperscript{135:9.3} En el transcurso de estos cuarenta días de espera, numerosos rumores se esparcieron por el país, llegando incluso hasta Tiberiades y Jerusalén. Miles de personas pasaban por el campamento de Juan para ver la nueva atracción, el famoso Mesías, pero Jesús no estaba a la vista. Cuando los discípulos de Juan afirmaban que el extraño hombre de Dios se había marchado a las colinas, muchos dudaban de toda la historia.

\par 
%\textsuperscript{(1505.4)}
\textsuperscript{135:9.4} Unas tres semanas después de la partida de Jesús, una nueva delegación de los sacerdotes y fariseos de Jerusalén llegó hasta aquel lugar de Pella. Preguntaron directamente a Juan si él era Elías o el profeta que Moisés había prometido\footnote{\textit{Preguntas de la delegación}: Jn 1:19-21; Jn 1:24-26.}. Cuando Juan les dijo, <<Yo no soy>>, se atrevieron a preguntarle, <<¿Eres el Mesías?>>, y Juan respondió: <<No lo soy>>. Entonces, estos hombres de Jerusalén le dijeron: <<Si no eres Elías, ni el profeta, ni el Mesías, entonces ¿por qué bautizas a la gente, creando todo este alboroto?>> Y Juan replicó: <<Aquellos que me han escuchado y han recibido mi bautismo os pueden decir quién soy yo, pero os afirmo que si bien yo bautizo con agua, ha estado entre nosotros aquel que volverá para bautizaros con el Espíritu Santo>>.

\par 
%\textsuperscript{(1505.5)}
\textsuperscript{135:9.5} Estos cuarenta días fueron un período difícil para Juan y sus discípulos. ¿Cuales iban a ser las relaciones entre Juan y Jesús? Se planteaban cientos de interrogantes. La política y las preferencias egoístas empezaron a hacer su aparición. Brotaron violentas discusiones alrededor de las diversas ideas y conceptos del Mesías. ¿Se convertiría en un jefe militar y en un rey como David? ¿Destruiría a los ejércitos romanos como Josué había hecho con los cananeos?\footnote{\textit{Invasión de Josué}: Jos 12:7-24.} ¿O vendría para establecer un reino espiritual? Juan se definió más bien por la opinión de la minoría, de que Jesús había venido para establecer el reino de los cielos, aunque no tenía del todo claro en su propia mente qué debería de incluirse exactamente dentro de esta misión de establecer el reino de los cielos.

\par 
%\textsuperscript{(1505.6)}
\textsuperscript{135:9.6} Fueron días arduos en la experiencia de Juan, y oró para que Jesús regresara. Algunos discípulos de Juan organizaron grupos de reconocimiento para ir en busca de Jesús, pero Juan lo prohibió diciendo: <<El tiempo de cada uno de nosotros está en las manos del Dios del cielo; él guiará a su Hijo elegido>>.

\par 
%\textsuperscript{(1505.7)}
\textsuperscript{135:9.7} El sábado 23 de febrero por la mañana temprano, cuando los compañeros de Juan, que estaban tomando su desayuno, levantaron la mirada hacia el norte, vieron a Jesús que venía hacia ellos. Mientras se acercaba, Juan se subió a una gran roca, elevó su voz sonora y dijo: <<¡Mirad al Hijo de Dios, el libertador del mundo! Es de él de quien he dicho, `Detrás de mí vendrá aquel que ha sido elegido antes que yo, porque existía antes que yo'. Por esta razón he salido del desierto para predicar el arrepentimiento y bautizar con agua, proclamando que el reino de los cielos está cerca. Ahora viene aquel que os bautizará con el Espíritu Santo. Yo he visto al espíritu divino descender sobre este hombre, y he oído la voz de Dios afirmar: `Éste es mi hijo amado en quien me siento muy complacido'.>>\footnote{\textit{Declaración de Juan}: Mc 1:7; Jn 1:29-32. \textit{Descenso del espíritu y voz}: Mt 3:16-17; Mc 1:10-11; Lc 3:21-22.}

\par 
%\textsuperscript{(1506.1)}
\textsuperscript{135:9.8} Jesús les rogó que continuaran desayunando, mientras se sentaba para comer con Juan, pues sus hermanos Santiago y Judá habían regresado a Cafarnaúm.

\par 
%\textsuperscript{(1506.2)}
\textsuperscript{135:9.9} Al día siguiente por la mañana temprano, se despidió de Juan y de sus discípulos y emprendió el regreso a Galilea\footnote{\textit{Jesús vuelve a Galilea}: Lc 4:14; Jn 1:43a.}. No les dio ninguna indicación sobre cuándo volverían a verlo. A las preguntas de Juan acerca de su propia predicación y de su misión, Jesús dijo solamente: <<Mi Padre te guiará ahora y en el futuro como lo ha hecho en el pasado>>. Y estos dos grandes hombres se separaron aquella mañana a orillas del Jordán, para no volverse a ver nunca más en la carne.

\section*{10. Juan viaja hacia el sur}
\par 
%\textsuperscript{(1506.3)}
\textsuperscript{135:10.1} Puesto que Jesús había ido en dirección norte hacia Galilea, Juan se sintió inducido a volver sobre sus pasos hacia el sur. En consecuencia, el domingo 3 de marzo por la mañana, Juan y el resto de sus discípulos emprendieron su viaje hacia el sur. Mientras tanto, aproximadamente una cuarta parte de los seguidores inmediatos de Juan habían partido para Galilea en busca de Jesús. La tristeza de la confusión envolvía a Juan. Nunca más volvió a predicar como lo había hecho antes de bautizar a Jesús. Sentía de alguna manera que la responsabilidad del reino venidero ya no descansaba sobre sus hombros. Sentía que su obra estaba casi terminada; estaba desconsolado y solitario. Pero predicaba, bautizaba y continuaba viajando hacia el sur.

\par 
%\textsuperscript{(1506.4)}
\textsuperscript{135:10.2} Juan se detuvo varias semanas cerca del pueblo de Adán, y fue aquí donde lanzó su ataque memorable contra Herodes Antipas por haberse apoderado ilegalmente de la mujer de otro hombre\footnote{\textit{Juan ataca a Herodes}: Mt 14:3b-4; Mc 6:17b-19; Lc 3:19.}. En junio de este año 26, Juan estaba de vuelta en el vado del Jordán en Betania, donde había empezado su predicación del reino venidero más de un año antes. Durante las semanas que siguieron al bautismo de Jesús, el carácter de la predicación de Juan fue cambiando paulatinamente; ahora proclamaba la misericordia para la gente común, mientras que denunciaba con renovada vehemencia la corrupción de los dirigentes políticos y religiosos.

\par 
%\textsuperscript{(1506.5)}
\textsuperscript{135:10.3} Juan había estado predicando en el territorio de Herodes Antipas. Éste se alarmó por temor a que Juan y sus discípulos provocaran una rebelión. Herodes también estaba ofendido por las críticas que Juan hacía en público de sus asuntos familiares. En vista de todo esto, Herodes decidió meter a Juan en la cárcel\footnote{\textit{Juan encarcelado}: Mt 14:3; Mc 1:14,6:17; Lc 3:19.}. En consecuencia, el 12 de junio por la mañana muy temprano, antes de que llegaran las multitudes para escuchar la predicación y presenciar los bautismos, los agentes de Herodes arrestaron a Juan. Como pasaban las semanas sin que fuera liberado, sus discípulos se dispersaron por toda Palestina; muchos de ellos fueron a Galilea para unirse a los seguidores de Jesús.

\section*{11. Juan en la cárcel}
\par 
%\textsuperscript{(1506.6)}
\textsuperscript{135:11.1} Juan tuvo una experiencia solitaria y un poco amarga en la cárcel. Pocos discípulos suyos fueron autorizados para visitarlo. Anhelaba ver a Jesús, pero tuvo que contentarse con oír hablar de su obra a través de aquellos discípulos suyos que se habían hecho creyentes en el Hijo del Hombre. A menudo se sentía tentado a dudar de Jesús y de su misión divina. Si Jesús era el Mesías, ¿por qué no hacía nada para liberarlo de esta intolerable reclusión? Durante más de año y medio, este hombre robusto habituado al aire libre de Dios languideció en aquella despreciable prisión. Esta experiencia fue una gran prueba para su fe en Jesús y para su lealtad hacia él. En verdad, toda esta experiencia fue una gran prueba incluso para la fe de Juan en Dios. Muchas veces tuvo la tentación de dudar hasta de la autenticidad de su propia misión y experiencia.

\par 
%\textsuperscript{(1507.1)}
\textsuperscript{135:11.2} Después de pasar varios meses en la cárcel, un grupo de sus discípulos vino a verle, y después de informarle de las actividades públicas de Jesús, le dijeron: <<Así que ya ves, Maestro, aquel que estuvo contigo en el alto Jordán, prospera y recibe a todos los que vienen hasta él. Incluso come en los festines con los publicanos y los pecadores. Tú has dado testimonio valientemente por él, y sin embargo, él no hace nada por conseguir tu liberación>>\footnote{\textit{Preguntas de los discípulos de Juan}: Jn 3:25-36a.}. Pero Juan contestó a sus amigos: <<Este hombre no puede hacer nada a menos que le sea dado por su Padre que está en los cielos. Recordad bien que he dicho, `Yo no soy el Mesías, pero he sido enviado delante de él para preparar su camino'. Y eso es lo que he hecho. El que tiene la novia es el novio, pero el amigo del novio, que permanece cerca, se regocija mucho cuando escucha la voz del novio. Mi alegría es pues completa. Él debe aumentar y yo disminuir. Yo pertenezco a esta Tierra y he proclamado mi mensaje. Jesús de Nazaret ha venido del cielo a la Tierra y está por encima de todos nosotros. El Hijo del Hombre ha descendido de Dios, y os proclamará las palabras de Dios. Porque el Padre que está en los cielos no escatima el espíritu a su propio Hijo. El Padre ama a su Hijo y pronto pondrá todas las cosas en las manos de este Hijo. El que cree en el Hijo tiene la vida eterna. Y estas palabras que digo son verdaderas y permanentes>>.

\par 
%\textsuperscript{(1507.2)}
\textsuperscript{135:11.3} Estos discípulos se quedaron tan sorprendidos con la declaración de Juan que se marcharon en silencio. Juan también estaba muy agitado, pues percibía que acababa de pronunciar una profecía. Nunca más dudó por completo de la misión y de la divinidad de Jesús. Pero fue una dolorosa desilusión para Juan el que Jesús no le enviara ningún mensaje, no viniera a verlo y no utilizara ninguno de sus grandes poderes para liberarlo de la cárcel. Pero Jesús estaba al corriente de todo esto. Quería mucho a Juan, pero ahora que estaba enterado de su naturaleza divina, sabiendo plenamente las grandes cosas que se preparaban para Juan cuando partiera de este mundo, y sabiendo también que la obra de Juan en la Tierra había terminado, se contuvo para no intervenir en el desarrollo natural de la carrera de este gran predicador y profeta.

\par 
%\textsuperscript{(1507.3)}
\textsuperscript{135:11.4} Esta larga incertidumbre en la prisión era humanamente insoportable. Muy pocos días antes de su muerte, Juan envió de nuevo a unos mensajeros de confianza para que le preguntaran a Jesús: <<¿Está concluida mi obra? ¿Por qué languidezco en la cárcel? ¿Eres realmente el Mesías o tenemos que esperar a otro?>>\footnote{\textit{Más preguntas de Juan}: Mt 11:2-6; Lc 7:19-23.} Cuando estos dos discípulos entregaron el mensaje a Jesús, el Hijo del Hombre respondió: <<Volved a Juan y decidle que no he olvidado, pero que lleve esto también con paciencia, porque corresponde que cumplamos con toda la rectitud. Contadle a Juan lo que habéis visto y oído ---que la buena nueva se predica a los pobres--- y finalmente, decidle al amado precursor de mi misión terrenal que será abundantemente bendecido en la era por venir, si procura no dudar y tropezar por mi causa>>. Éstas fueron las últimas palabras que Juan recibió de Jesús. Este mensaje lo animó ampliamente y contribuyó mucho a estabilizar su fe y a prepararlo para el trágico final de su vida en la carne, que siguió tan de cerca a esta memorable ocasión.

\section*{12. La muerte de Juan el Bautista}
\par 
%\textsuperscript{(1508.1)}
\textsuperscript{135:12.1} Como Juan estaba trabajando en el sur de Perea en el momento de ser arrestado, fue llevado inmediatamente a la prisión de la fortaleza de Macaerus, donde permaneció encarcelado hasta su ejecución. Herodes gobernaba en Perea y Galilea, y en esta época mantenía su residencia en Perea tanto en Julias como en Macaerus. Su residencia oficial de Galilea la había trasladado de Séforis a Tiberiades, la nueva capital.

\par 
%\textsuperscript{(1508.2)}
\textsuperscript{135:12.2} Herodes tenía miedo de liberar a Juan por temor a que provocara una rebelión\footnote{\textit{Herodes y Juan}: Mt 14:3-5; Mc 6:17-19.}. Temía ejecutarlo por miedo a que la multitud se amotinara en la capital, pues miles de pereanos creían que Juan era un santo, un profeta. Por esta razón, Herodes mantenía en la cárcel al predicador nazareo, sin saber qué hacer con él. Juan había comparecido varias veces ante Herodes, pero nunca aceptó marcharse de sus dominios ni abstenerse de toda actividad pública si era puesto en libertad. Y la nueva agitación en constante aumento relacionada con Jesús de Nazaret advertía a Herodes que no era el momento adecuado para poner en libertad a Juan. Además, Juan era víctima también del odio intenso y amargo de Herodías, la mujer ilegítima de Herodes.

\par 
%\textsuperscript{(1508.3)}
\textsuperscript{135:12.3} Herodes habló con Juan en numerosas ocasiones sobre el reino de los cielos, y aunque a veces se quedó seriamente impresionado con su mensaje, tenía miedo de liberarlo de la prisión\footnote{\textit{Juan ante Herodes}: Mc 6:20.}.

\par 
%\textsuperscript{(1508.4)}
\textsuperscript{135:12.4} Como aún se estaban construyendo muchos edificios en Tiberiades, Herodes pasaba la mayor parte del tiempo en sus residencias de Perea, y tenía predilección por la fortaleza de Macaerus. Tuvieron que pasar varios años antes de que se terminaran por completo todos los edificios públicos y la residencia oficial de Tiberiades.

\par 
%\textsuperscript{(1508.5)}
\textsuperscript{135:12.5} Para celebrar su cumpleaños\footnote{\textit{Fiesta de cumpleaños de Herodes}: Mt 14:6a; Mc 6:21.}, Herodes organizó una gran fiesta en el palacio de Macaerus para sus oficiales principales y otras personalidades ilustres de los consejos de gobierno de Galilea y de Perea. Como Herodías no había conseguido llevar a cabo la ejecución de Juan pidiéndoselo directamente a Herodes, se dedicó ahora a la tarea de hacerle morir mediante un astuto plan.

\par 
%\textsuperscript{(1508.6)}
\textsuperscript{135:12.6} En el transcurso de las festividades y diversiones de la velada, Herodías presentó a su hija para que bailara ante los comensales. Herodes quedó muy complacido con la actuación de la doncella y, llamándola ante él, le dijo: <<Eres encantadora. Estoy muy contento contigo. Pídeme en mi cumpleaños todo lo que desees y yo te lo daré, aunque sea la mitad de mi reino>>. Cuando Herodes dijo esto, se encontraba bajo el influjo de todo lo que había bebido. La joven se retiró y le preguntó a su madre qué debía pedirle a Herodes. Herodías le dijo: <<Ve a Herodes y pídele la cabeza de Juan el Bautista>>. La joven regresó a la mesa del banquete y le dijo a Herodes: <<Te pido que me des inmediatamente la cabeza de Juan el Bautista en una bandeja>>.

\par 
%\textsuperscript{(1508.7)}
\textsuperscript{135:12.7} Herodes se llenó de temor y de tristeza, pero a causa de su promesa y de todos los testigos que estaban en el banquete con él, no quiso rechazar la petición. Herodes Antipas envió a un soldado, ordenándole que trajera la cabeza de Juan. Así es como Juan fue decapitado aquella noche en la prisión; el soldado trajo la cabeza del profeta en una bandeja y se la dio a la joven detrás de la sala del banquete. Y la doncella entregó la bandeja a su madre. Cuando los discípulos de Juan se enteraron de esto, vinieron a la prisión para recoger su cuerpo, y después de darle sepultura, fueron a decírselo a Jesús.


\chapter{Documento 136. El bautismo y los cuarenta días}
\par 
%\textsuperscript{(1509.1)}
\textsuperscript{136:0.1} JESÚS comenzó su ministerio público cuando el interés popular por la predicación de Juan estaba en su apogeo y en la época en que el pueblo judío de Palestina esperaba ansiosamente la aparición del Mesías\footnote{\textit{El pueblo expectante del Mesías}: Lc 3:15; Jn 1:25.}. Había un gran contraste entre Juan y Jesús. Juan era un obrero ardiente y severo, mientras que Jesús era un trabajador tranquilo y feliz; en toda su vida, sólo unas pocas veces se le vio apresurarse. Jesús era un consuelo reconfortante para el mundo y en cierto modo un ejemplo. Juan apenas era un consuelo o un ejemplo; predicaba el reino de los cielos, pero no participaba mucho de su felicidad. Aunque Jesús se refería a Juan como el más grande de los profetas\footnote{\textit{Juan, el mayor profeta}: Mt 11:11; Lc 7:28.} del antiguo orden, también decía que el más humilde de los que vieran la gran luz del nuevo camino, y entrara por allí en el reino de los cielos, era en verdad más grande que Juan.

\par 
%\textsuperscript{(1509.2)}
\textsuperscript{136:0.2} Cuando Juan predicaba el reino venidero, lo esencial de su mensaje era: <<¡Arrepentíos!. Huid de la cólera inminente>>. Cuando Jesús empezó a predicar, mantuvo la exhortación al arrepentimiento, pero este mensaje estaba siempre ligado al evangelio, a la buena nueva de la alegría y de la libertad del nuevo reino\footnote{\textit{El mensaje del evangelio}: Lc 8:1.}.

\section*{1. Los conceptos del Mesías esperado}
\par 
%\textsuperscript{(1509.3)}
\textsuperscript{136:1.1} Los judíos poseían diversas ideas sobre el libertador esperado, y cada una de estas diferentes escuelas de enseñanza mesiánica podía citar pasajes de las escrituras hebreas como prueba de sus argumentos. De manera general, los judíos consideraban que su historia nacional empezaba con Abraham y culminaría con el Mesías y la nueva era del reino de Dios. En los siglos anteriores habían concebido a este libertador como <<el siervo del Señor>>, luego como <<el Hijo del Hombre>>, mientras que más recientemente algunos incluso habían llegado a referirse al Mesías como el <<Hijo de Dios>>. Pero, sin importar que le llamaran <<la semilla de Abraham>> o <<el hijo de David>>, todos estaban de acuerdo en que tenía que ser el Mesías, el <<ungido>>. Así pues, el concepto evolucionó desde <<siervo del Señor>> a <<hijo de David>>, y de <<Hijo del Hombre>> a <<Hijo de Dios>>\footnote{\textit{El Siervo del Señor}: Is 42:1; 53:11; Ez 34:22-31. \textit{El Hijo del Hombre}: Ez 2:1,3,6,8; 3:1-4,10,17; Mt 8:20; Dn 7:13,14; Mc 2:10; Lc 5:24; Jn 1:51; Ap 1:13; 14:14; 1 Hen 46:1-6; 48:1-7; 60:10; 62:1-14; 63:11; 69:26-29; 70:1-2; 71:14-16. \textit{Semilla de Abraham}: Gn 18:18; Hch 3:25; Ro 4:13. \textit{Hijo de David}: Mt 1:1; 2 Sam 7:12-13. \textit{El Mesías}: Dn 9:25-26; Jn 1:41; 4:25-26. \textit{El Ungido}: Is 61:1; Lc 4:18; Hch 10:38. \textit{Jesús, Hijo de Dios}: Mt 8:29; 14:33; 16:15-16; 27:54; Mc 1:1; 3:11; 15:39; Lc 1:35; 4:41; Jn 1:34,49; 3:16-18; 10:36; 20:31; Hch 8:37.}.

\par 
%\textsuperscript{(1509.4)}
\textsuperscript{136:1.2} En los tiempos de Juan y de Jesús, los judíos más cultos habían desarrollado la idea del Mesías venidero como que sería un israelita perfeccionado y representativo, que reuniría en sí mismo como <<siervo del Señor>> el triple cargo de profeta, sacerdote y rey.

\par 
%\textsuperscript{(1509.5)}
\textsuperscript{136:1.3} Los judíos creían devotamente que, al igual que Moisés había liberado a sus padres de la esclavitud egipcia mediante prodigios milagrosos, el Mesías esperado liberaría al pueblo judío de la dominación romana mediante milagros de poder aún más grandes y maravillas de triunfo racial. Los rabinos habían reunido casi quinientos pasajes de las Escrituras que, a pesar de sus contradicciones aparentes, eran profecías, según ellos, del advenimiento del Mesías. En medio de todos estos detalles de tiempo, de técnicas y de funciones, casi perdieron de vista por completo la \textit{personalidad} del Mesías prometido. Esperaban el restablecimiento de la gloria nacional judía ---la exaltación temporal de Israel--- en lugar de la salvación del mundo. Es evidente pues que Jesús de Nazaret no podría satisfacer nunca este concepto mesiánico materialista de la mente judía. Si los judíos hubieran sabido ver estos pronunciamientos proféticos bajo una luz diferente, muchas de sus predicciones supuestamente mesiánicas hubieran preparado sus mentes de manera muy natural para reconocer en Jesús a aquel que cerraría una era e inauguraría una dispensación de misericordia y de salvación nueva y mejor para todas las naciones.

\par 
%\textsuperscript{(1510.1)}
\textsuperscript{136:1.4} Los judíos habían sido educados en la creencia de la doctrina de la \textit{Shekinah}. Pero este pretendido símbolo de la Presencia Divina no estaba visible en el templo. Creían que la venida del Mesías efectuaría su restablecimiento. Tenían ideas confusas sobre el pecado racial y la supuesta naturaleza maligna del hombre. Algunos enseñaban que el pecado de Adán había causado la maldición de la raza humana, y que el Mesías destruiría esa maldición y restituiría al hombre en el favor divino. Otros enseñaban que al crear al hombre, Dios había puesto dentro de su ser una naturaleza buena y otra mala; que cuando observó el resultado de esta combinación, se había desilusionado mucho, y que <<se arrepintió de haber creado así al hombre>>\footnote{\textit{Dios arrepentido de crear al hombre}: Gn 6:6.}. Los que enseñaban esto creían que el Mesías tenía que venir para redimir al hombre de esta naturaleza maligna innata.

\par 
%\textsuperscript{(1510.2)}
\textsuperscript{136:1.5} La mayoría de los judíos creía que continuaban languideciendo bajo el poder romano debido a sus pecados nacionales y a la frialdad de los prosélitos gentiles. La nación judía no se había \textit{arrepentido} de todo corazón; por eso el Mesías retrasaba su llegada. Se hablaba mucho de arrepentimiento, lo que explica la atracción poderosa e inmediata de la predicación de Juan: <<Arrepentíos y sed bautizados, porque el reino de los cielos está cerca>>\footnote{\textit{Arrepentíos y bautizaos}: Mt 3:2.}. Y para cualquier judío piadoso, el reino de los cielos sólo podía significar una cosa: la venida del Mesías.

\par 
%\textsuperscript{(1510.3)}
\textsuperscript{136:1.6} La donación de Miguel contenía una característica que era completamente ajena al concepto judío del Mesías; esta característica era la \textit{unión} de las dos naturalezas: la humana y la divina. Los judíos habían concebido al Mesías de distintas maneras: como humano perfeccionado, como superhumano e incluso como divino, pero nunca habían pensado en el concepto de la \textit{unión} de lo humano y lo divino. Este fue el gran escollo de los primeros discípulos de Jesús. Captaban el concepto humano del Mesías como hijo de David\footnote{\textit{El Mesías como hijo de David}: 2 Sam 7:12-13.}, tal como había sido presentado por los primeros profetas; también comprendían al Mesías como Hijo del Hombre\footnote{\textit{El Mesías como Hijo del Hombre}: Dn 7:13-14.}, la idea superhumana de Daniel y de algunos de los últimos profetas, e incluso como Hijo de Dios, tal como lo habían descrito el autor del Libro de Enoc y algunos de sus contemporáneos. Pero nunca llegaron a considerar, ni por un solo instante, el verdadero concepto de la unión, en una sola personalidad terrestre, de las dos naturalezas: la humana y la divina. La encarnación del Creador en forma de criatura no había sido revelada de antemano. Sólo fue revelada en Jesús\footnote{\textit{Dios revelado cuando se encarnó}: Jn 1:1-5.}. El mundo no sabía nada de estas cosas hasta que el Hijo Creador se hizo carne y habitó entre los mortales del planeta.

\section*{2. El bautismo de Jesús}
\par 
%\textsuperscript{(1510.4)}
\textsuperscript{136:2.1} Jesús fue bautizado\footnote{\textit{El bautismo de Jesús}: Mt 3:13-17; Mc 1:9; Lc 3:21.} en el apogeo de la predicación de Juan, cuando Palestina estaba inflamada con la esperanza de su mensaje ---<<el reino de Dios está cerca>>\footnote{\textit{El reino de Dios está cerca}: Dn 2:44; Mt 3:2.}--- y todo el pueblo judío se dedicaba a un análisis de sí mismo serio y solemne. El sentido judío de la solidaridad racial era muy profundo. Los judíos no sólo creían que los pecados de un padre podían afectar a sus hijos, sino que también creían firmemente que el pecado de un individuo podía maldecir a la nación. Por consiguiente, no todos los que se sometían al bautismo de Juan se consideraban culpables de los pecados específicos que Juan denunciaba. Muchas almas piadosas eran bautizadas por Juan para el bien de Israel; temían que un pecado de ignorancia por su parte pudiera retrasar la venida del Mesías. Sentían que pertenecían a una nación culpable y maldita por el pecado\footnote{\textit{Se sentían como una nación culpable}: Dn 9:11.}, y se sometían al bautismo para manifestar de este modo los frutos de una penitencia racial. Por lo tanto, es evidente que Jesús no recibió de ninguna manera el bautismo de Juan como rito de arrepentimiento o para la remisión de los pecados. Al aceptar el bautismo de manos de Juan, Jesús no hacía más que seguir el ejemplo de muchos israelitas piadosos.

\par 
%\textsuperscript{(1511.1)}
\textsuperscript{136:2.2} Cuando Jesús de Nazaret bajó al Jordán para ser bautizado, era un mortal del mundo que había alcanzado el pináculo de la ascensión evolutiva humana en todos los aspectos relacionados con la conquista de la mente y la identificación del yo con el espíritu. Ese día, estuvo de pie en el Jordán como un mortal perfeccionado de los mundos evolutivos del tiempo y del espacio. Una sincronía perfecta y una comunicación plena se habían establecido entre la mente mortal de Jesús y su Ajustador espiritual interior, el don divino de su Padre Paradisiaco. Desde la ascensión de Miguel a la jefatura de su universo, un Ajustador como éste reside en todos los seres normales que viven en Urantia, excepto que el Ajustador de Jesús había sido preparado previamente para esta misión especial, habiendo habitado de manera similar en Maquiventa Melquisedek, otro superhumano encarnado en la similitud de la carne mortal.

\par 
%\textsuperscript{(1511.2)}
\textsuperscript{136:2.3} Ordinariamente, cuando un mortal del planeta alcanza estos altos niveles de perfección de la personalidad, se producen esos fenómenos preliminares de elevación espiritual que culminan finalmente en la fusión definitiva del alma madura del mortal con su Ajustador divino asociado. Aparentemente, un cambio de esta naturaleza debía producirse en la experiencia de la personalidad de Jesús de Nazaret el mismo día que descendió al Jordán con sus dos hermanos para ser bautizado por Juan. Esta ceremonia era el acto final de su vida puramente humana en Urantia, y muchos observadores superhumanos esperaban presenciar la fusión del Ajustador con la mente que habitaba, pero todos estaban destinados a sufrir una desilusión. Ocurrió algo nuevo y mucho más grandioso. Mientras Juan imponía sus manos sobre Jesús para bautizarlo, el Ajustador residente se despidió para siempre del alma humana perfeccionada de Josué ben José. Unos instantes después, esta entidad divina regresó de Divinington como Ajustador Personalizado\footnote{\textit{Ajustador Personalizado}: Mt 3:16-17; Mc 1:10-11; Lc 3:22; Jn 1:32-33.} y jefe de sus semejantes en todo el universo local de Nebadon. Jesús pudo así observar a su propio espíritu divino anterior regresar y descender sobre él de forma personalizada. Y entonces oyó hablar a este mismo espíritu originario del Paraíso, que decía: <<Éste es mi Hijo amado en quien tengo complacencia>>\footnote{\textit{Éste es mi Hijo amado}: Mt 3:17; Mc 1:11; Lc 3:22.}. Juan y los dos hermanos de Jesús también oyeron estas palabras. Los discípulos de Juan, que estaban al borde del agua, no las oyeron ni tampoco vieron la aparición del Ajustador Personalizado. Sólo los ojos de Jesús contemplaron al Ajustador Personalizado.

\par 
%\textsuperscript{(1511.3)}
\textsuperscript{136:2.4} Cuando el Ajustador Personalizado ahora ensalzado que había regresado hubo hablado así, todo fue silencio. Y mientras los cuatro interesados permanecían en el agua, Jesús levantó la mirada hacia el cercano Ajustador y oró: <<Padre mío que reinas en el cielo, santificado sea tu nombre. ¡Que venga tu reino!. Que tu voluntad se haga en la Tierra, así como se hace en el cielo>>\footnote{\textit{La oración de Jesús}: Mt 6:9-10; Lc 11:2.}. Cuando terminó de orar, <<se abrieron los cielos>>\footnote{\textit{Se abrieron los cielos}: Mt 3:16; Mc 1:10; Lc 3:21.}, y el Hijo del Hombre contempló la imagen de sí mismo como Hijo de Dios, presentada por el Ajustador ahora Personalizado, tal como era antes de venir a la Tierra en la similitud de la carne mortal, y tal como volvería a ser cuando terminara su vida encarnada. Jesús fue el único que presenció esta visión celestial.

\par 
%\textsuperscript{(1512.1)}
\textsuperscript{136:2.5} Lo que Juan y Jesús oyeron fue la voz del Ajustador Personalizado\footnote{\textit{Voz del Ajustador Personalizado}: Mt 3:17; Mc 1:11; Lc 3:22.}, hablando en nombre del Padre Universal, porque el Ajustador proviene del Padre Paradisiaco y es semejante a él. Durante el resto de la vida terrenal de Jesús, este Ajustador Personalizado estuvo asociado con él en todas sus obras; Jesús permaneció en constante comunión con este Ajustador ensalzado.

\par 
%\textsuperscript{(1512.2)}
\textsuperscript{136:2.6} Cuando Jesús fue bautizado, no se arrepintió de ninguna mala acción y no hizo ninguna confesión de pecado. Se trataba de un bautismo de consagración a la realización de la voluntad del Padre celestial. En su bautismo escuchó la llamada inequívoca de su Padre, la citación final para que se ocupara de los asuntos de su Padre, y se retiró a solas durante cuarenta días para meditar sobre estos múltiples problemas. Al retirarse así durante cierto tiempo de todo contacto personal activo con sus asociados terrenales, Jesús, tal como era y en Urantia, estaba siguiendo el mismo procedimiento que prevalece en los mundos morontiales, cuando un mortal ascendente fusiona con la presencia interior del Padre Universal.

\par 
%\textsuperscript{(1512.3)}
\textsuperscript{136:2.7} Este día de bautismo marcó el final de la vida puramente humana de Jesús. El Hijo divino ha encontrado a su Padre, el Padre Universal ha encontrado a su Hijo encarnado, y hablan el uno con el otro.

\par 
%\textsuperscript{(1512.4)}
\textsuperscript{136:2.8} (Jesús tenía casi treinta y un años y medio cuando fue bautizado. Aunque Lucas dice que fue bautizado en el decimoquinto año del reinado de Tiberio César, lo que nos daría el año 29 puesto que Augusto murió en el año 14, hay que recordar que Tiberio fue coemperador con Augusto durante dos años y medio antes de la muerte de este último, habiéndose acuñado monedas en su honor en octubre del año 11. El decimoquinto año de su reinado efectivo fue, por tanto, este mismo año 26, el del bautismo de Jesús. Éste fue también el año en que Poncio Pilatos empezó a mandar como gobernador de Judea.)\footnote{\textit{Cronología de eventos}: Lc 3:1.}

\section*{3. Los cuarenta días}
\par 
%\textsuperscript{(1512.5)}
\textsuperscript{136:3.1} Jesús había soportado la gran tentación de su donación como mortal antes de su bautismo cuando el rocío del Monte Hermón lo había mojado durante seis semanas. Allá en el Monte Hermón, como un mortal del planeta sin ayuda ninguna, se había enfrentado con Caligastia, el pretendiente de Urantia, el príncipe de este mundo, y lo había derrotado. En este día memorable, según los archivos del universo, Jesús de Nazaret se convirtió en el Príncipe Planetario de Urantia. Este Príncipe de Urantia, que muy pronto sería proclamado Soberano supremo de Nebadon, iniciaba ahora cuarenta días\footnote{\textit{Los ``cuarenta días''}: Mt 4:1-11; Mc 1:12-13; Lc 4:1-13.} de retiro para elaborar los planes y determinar la técnica que utilizaría para proclamar el nuevo reino de Dios en el corazón de los hombres.

\par 
%\textsuperscript{(1512.6)}
\textsuperscript{136:3.2} Después de su bautismo, consagró estos cuarenta días a adaptarse a los cambios de relaciones con el mundo y el universo, ocasionados por la personalización de su Ajustador. Durante su aislamiento en las colinas de Perea, Jesús determinó la política a seguir y los métodos que emplearía en la nueva fase modificada de la vida terrenal que estaba a punto de inaugurar.

\par 
%\textsuperscript{(1512.7)}
\textsuperscript{136:3.3} Jesús no efectuó este retiro para ayunar ni tampoco para afligir su alma. No era un asceta, y había venido para destruir definitivamente todas estas ideas sobre cómo acercarse a Dios. Sus razones para buscar esta soledad eran totalmente diferentes de las que habían motivado a Moisés y a Elías, e incluso a Juan el Bautista. Jesús estaba entonces plenamente consciente de sus relaciones con el universo creado por él, así como con el universo de universos supervisado por el Padre Paradisiaco, su Padre celestial. Ahora recordaba plenamente su misión de donación y las instrucciones que le diera su hermano mayor Emmanuel antes de empezar su encarnación en Urantia. Ahora comprendía clara y plenamente todas estas vastas relaciones y deseaba encontrarse a solas durante un período de meditación tranquila, para poder elaborar los planes y decidir el procedimiento a seguir en la continuación de su obra pública a favor de este mundo y de todos los demás mundos de su universo local.

\par 
%\textsuperscript{(1513.1)}
\textsuperscript{136:3.4} Mientras deambulaba por las colinas en busca de un refugio apropiado, Jesús se encontró con el jefe ejecutivo de su universo, Gabriel, la Radiante Estrella Matutina de Nebadon. Gabriel restableció ahora sus comunicaciones personales con el Hijo Creador del universo; era su primer contacto directo desde que Miguel se despidió de sus asociados en Salvington para ir a Edentia con objeto de prepararse para su donación en Urantia. Siguiendo las instrucciones de Emmanuel, y autorizado por los Ancianos de los Días de Uversa, Gabriel mostró ahora a Jesús la información que indicaba que la experiencia de su donación en Urantia estaba prácticamente terminada en lo referente a la adquisición de la soberanía perfeccionada de su universo y a la finalización de la rebelión de Lucifer. Lo primero lo había conseguido el día de su bautismo, cuando la personalización de su Ajustador demostró la perfección y la plenitud de su donación en la similitud de la carne mortal, y lo segundo se volvió un hecho histórico el día que descendió del Monte Hermón para reunirse con el joven Tiglat que lo esperaba. Jesús recibió ahora la noticia, proveniente de la autoridad más alta del universo local y del superuniverso, de que su obra donadora había terminado en lo que afectaba a su estado personal en relación con la soberanía y la rebelión. Ya había recibido esta garantía directamente del Paraíso en su visión bautismal y en el fenómeno de la personalización de su Ajustador del Pensamiento interior.

\par 
%\textsuperscript{(1513.2)}
\textsuperscript{136:3.5} Mientras permanecía en la montaña conversando con Gabriel, el Padre de Edentia, el de la Constelación, apareció en persona ante Jesús y Gabriel, diciendo: <<Los registros han finalizado. La soberanía del Miguel n{\textordmasculine} 611.121 sobre su universo de Nebadon descansa consumada a la diestra del Padre Universal. Te libero de tu donación de parte de Emmanuel, tu hermano y patrocinador de tu encarnación en Urantia. Eres libre de dar por terminada tu donación de encarnación ahora o en cualquier otro momento, de la manera que tú mismo escojas, ascender a la diestra de tu Padre, recibir tu soberanía y asumir el gobierno incondicional bien merecido de todo Nebadon. También doy fe de que por autorización de los Ancianos de los Días, se han completado las formalidades superuniversales relacionadas con la terminación de toda rebelión pecaminosa en tu universo; se te ha otorgado una autoridad plena e ilimitada para intervenir en cualquier posible sublevación de este tipo en el futuro. Tu obra en Urantia y en la carne de una criatura mortal está formalmente terminada. De ahora en adelante, todo lo que hagas dependerá de tu propia elección>>.

\par 
%\textsuperscript{(1513.3)}
\textsuperscript{136:3.6} Cuando el Altísimo Padre de Edentia se hubo despedido, Jesús conversó largo rato con Gabriel sobre el bienestar del universo y, al enviar sus saludos a Emmanuel, le aseguró que en la obra que estaba por emprender en Urantia, siempre recordaría los consejos recibidos en Salvington antes de comenzar su misión donadora.

\par 
%\textsuperscript{(1514.1)}
\textsuperscript{136:3.7} Durante estos cuarenta días de aislamiento, Santiago y Juan, los hijos de Zebedeo, estuvieron ocupados buscando a Jesús. Muchas veces estuvieron a poca distancia del lugar donde residía, pero nunca llegaron a encontrarlo.

\section*{4. Los planes para la obra pública}
\par 
%\textsuperscript{(1514.2)}
\textsuperscript{136:4.1} Día tras día, en las colinas, Jesús elaboró los planes para el resto de su donación en Urantia. En primer lugar decidió que no enseñaría al mismo tiempo que Juan. Proyectó permanecer en un retiro relativo hasta que la obra de Juan consiguiera su propósito, o fuera interrumpida súbitamente por su encarcelamiento. Jesús sabía muy bien que los sermones de Juan, intrépidos y desprovistos de tacto, pronto suscitarían el temor y la enemistad de los gobernantes civiles. En vista de la situación precaria de Juan, Jesús empezó definitivamente a preparar su programa de trabajo público a favor de su pueblo y del mundo, a favor de cada mundo habitado de todo su vasto universo. La donación de Miguel como mortal tuvo lugar \textit{en} Urantia, pero \textit{para} todos los mundos de Nebadon.

\par 
%\textsuperscript{(1514.3)}
\textsuperscript{136:4.2} Después de concebir el plan general para coordinar su programa con el movimiento de Juan, lo primero que hizo Jesús fue repasar mentalmente las instrucciones de Emmanuel. Reflexionó profundamente sobre los consejos que le habían dado relativos a sus métodos de trabajo, y a que no dejara escritos perdurables en el planeta. Jesús nunca más volvió a escribir, salvo en la arena. En su visita posterior a Nazaret, y con gran pena por parte de su hermano José, Jesús destruyó todos los escritos suyos que se conservaban en las tablillas del taller de carpintería, o estaban colgados en las paredes de la vieja casa. Jesús también reflexionó mucho sobre los consejos de Emmanuel relacionados con su comportamiento en materia económica, social y política hacia el mundo que encontraría en esta época.

\par 
%\textsuperscript{(1514.4)}
\textsuperscript{136:4.3} Jesús no ayunó durante estos cuarenta días de aislamiento\footnote{\textit{Jesús no ayunó}: Mt 4:2; Mc 1:13; Lc 4:2.}. El período más largo que estuvo sin alimentarse fue los dos primeros días que pasó en las colinas, pues estaba tan ensimismado en sus pensamientos que se olvidó por completo de comer. Pero al tercer día se puso a buscar alimentos. Durante este período, tampoco fue \textit{tentado} por espíritus malignos ni por personalidades rebeldes estacionadas en este mundo o procedentes de cualquier otro mundo\footnote{\textit{Las ``tentaciones'' ocurrieron anteriormente}: Mt 4:3-11; Mc 1:12-13; Lc 4:13.}.

\par 
%\textsuperscript{(1514.5)}
\textsuperscript{136:4.4} Estos cuarenta días fueron la ocasión para el diálogo final entre su mente humana y su mente divina, o más bien para el primer funcionamiento real de estas dos mentes reunidas ahora en una sola. Los resultados de este importante período de meditación demostraban de manera concluyente que su mente divina había dominado triunfal y espiritualmente a su intelecto humano. De ahora en adelante, la mente del hombre se ha convertido en la mente de Dios, y aunque la individualidad de la mente del hombre está siempre presente, esta mente humana espiritualizada dice siempre: <<Que no se haga mi voluntad sino la tuya>>\footnote{\textit{Que no se haga mi voluntad sino la tuya}: Mt 26:39,42,44; Mc 14:36,39; Lc 22:42; Jn 4:34; Jn 5:30; Jn 6:38-40; Jn 15:10; Jn 17:4.}.

\par 
%\textsuperscript{(1514.6)}
\textsuperscript{136:4.5} Los acontecimientos de este período extraordinario no fueron las visiones fantásticas de una mente hambrienta y debilitada, ni tampoco fueron los simbolismos confusos y pueriles que más tarde se transmitieron como las <<tentaciones de Jesús en el desierto>>\footnote{\textit{Razón de la idea de los ``cuarenta días''}: Mt 4:1-11; Mc 1:12-13; Lc 4:1-13.}. Fue más bien un período para meditar sobre toda la carrera memorable y variada de la donación en Urantia, y para preparar cuidadosamente los planes del ministerio ulterior que fuera más útil para este mundo, y a la vez contribuyera también un poco al mejoramiento de todas las otras esferas aisladas por la rebelión. Jesús examinó toda la historia de la vida humana en Urantia, desde los días de Andón y Fonta, pasando por la falta de Adán, hasta el ministerio de Melquisedek de Salem.

\par 
%\textsuperscript{(1514.7)}
\textsuperscript{136:4.6} Gabriel había recordado a Jesús que podía manifestarse al mundo de dos maneras diferentes, en el caso de que decidiera permanecer algún tiempo en Urantia. También se le indicó claramente a Jesús que su elección en esta materia no tendría nada que ver con su soberanía universal ni con el final de la rebelión de Lucifer. Las dos maneras de servir al mundo eran las siguientes:

\par 
%\textsuperscript{(1515.1)}
\textsuperscript{136:4.7} 1. Su propia vía ---La vía que pudiera parecerle más agradable y útil, desde el punto de vista de las necesidades inmediatas de este mundo y de la edificación en curso de su propio universo.

\par 
%\textsuperscript{(1515.2)}
\textsuperscript{136:4.8} 2. La vía del Padre ---La demostración con el ejemplo de un ideal, a largo plazo, de vida como criatura, según lo ven las altas personalidades de la administración paradisíaca del universo de universos.

\par 
%\textsuperscript{(1515.3)}
\textsuperscript{136:4.9} Se le indicó claramente a Jesús que tenía dos maneras de ordenar el resto de su vida terrestre. Tal como se podían observar a la luz de la situación inmediata, cada una de estas vías tenía puntos a favor. El Hijo del Hombre vio claramente que su elección entre estas dos líneas de conducta no tendría ninguna repercusión sobre la recepción de la soberanía de su universo; éste era un asunto que ya estaba arreglado y sellado en los archivos del universo de universos y sólo estaba pendiente de su petición personal. Pero se le indicó a Jesús que su hermano paradisíaco, Emmanuel, sentiría una gran satisfacción si Jesús juzgara conveniente terminar su carrera terrenal de encarnación tan noblemente como la había empezado, siempre sometido a la voluntad del Padre. Al tercer día de este aislamiento, Jesús se prometió a sí mismo que volvería al mundo para terminar su carrera terrenal, y que en cualquier situación que implicara los dos caminos, siempre escogería la voluntad del Padre. Y vivió el resto de su vida terrestre permaneciendo siempre fiel a esta resolución. Incluso hasta el amargo final, subordinó invariablemente su voluntad soberana a la de su Padre celestial.

\par 
%\textsuperscript{(1515.4)}
\textsuperscript{136:4.10} Los cuarenta días en el desierto montañoso no fueron un período de grandes tentaciones, sino más bien el período de las \textit{grandes decisiones}\footnote{\textit{Las grandes decisiones}: Mt 4:1; Mc 1:12-13; Lc 4:1-2.} del Maestro. Durante estos días de solitaria comunión consigo mismo y con la presencia inmediata de su Padre ---el Ajustador Personalizado (pues ya no tenía un guardián seráfico personal)--- tomó una tras otra las grandes decisiones que regirían su política y su conducta durante el resto de su carrera terrenal. La tradición de una gran tentación fue conectada posteriormente con este período de aislamiento debido a una confusión con los relatos fragmentarios de las luchas en el Monte Hermón, y además porque era costumbre que todos los grandes profetas y líderes humanos empezaran su carrera pública sometiéndose a estos supuestos períodos de ayuno y oración. Cada vez que Jesús se enfrentaba con una decisión nueva o importante, siempre tenía la costumbre de retirarse para comulgar con su propio espíritu y tratar así de conocer la voluntad de Dios.

\par 
%\textsuperscript{(1515.5)}
\textsuperscript{136:4.11} En todos estos proyectos para el resto de su vida terrenal, Jesús siempre estuvo dividido, en su corazón humano, entre dos líneas opuestas de conducta:

\par 
%\textsuperscript{(1515.6)}
\textsuperscript{136:4.12} 1. Albergaba un intenso deseo de conseguir que su pueblo ---y el mundo entero--- creyera en él y aceptara su nuevo reino espiritual. Y conocía muy bien las ideas de sus compatriotas sobre el Mesías venidero.

\par 
%\textsuperscript{(1515.7)}
\textsuperscript{136:4.13} 2. Vivir y actuar de la manera que sabía que su Padre aprobaría, llevar a cabo su trabajo a favor de otros mundos necesitados, y continuar, en el establecimiento del reino, revelando al Padre y manifestando su divino carácter de amor.

\par 
%\textsuperscript{(1515.8)}
\textsuperscript{136:4.14} Durante estos días extraordinarios, Jesús vivió en una antigua caverna rocosa, un refugio en la ladera de las colinas, cerca de una aldea llamada en otro tiempo Beit Adis. Bebía en el pequeño manantial que brotaba en la falda de la colina cerca de este refugio rocoso.

\section*{5. La primera gran decisión}
\par 
%\textsuperscript{(1516.1)}
\textsuperscript{136:5.1} Al tercer día de empezar esta conversación consigo mismo y con su Ajustador Personalizado, Jesús fue gratificado con la visión de las huestes celestiales de Nebadon, reunidas y enviadas por sus comandantes para aguardar los mandatos de su amado Soberano. Este ejército poderoso comprendía doce legiones de serafines\footnote{\textit{Doce legiones de ángeles y otros}: Mt 26:53.} y cantidades proporcionales de todas las órdenes de inteligencias del universo. La primera gran decisión de Jesús en su aislamiento consistió en determinar si utilizaría o no estas poderosas personalidades en el programa posterior de su obra pública en Urantia.

\par 
%\textsuperscript{(1516.2)}
\textsuperscript{136:5.2} Jesús decidió que \textit{no} utilizaría ni una sola personalidad de esta vasta asamblea, a menos que resultara evidente que se trataba de la \textit{voluntad de su Padre}. A pesar de esta decisión de tipo general, este enorme ejército permaneció con él durante el resto de su vida terrestre, siempre dispuesto a obedecer a la menor expresión de la voluntad de su Soberano. Jesús no contemplaba constantemente, con sus ojos humanos, estas personalidades acompañantes, pero su Ajustador Personalizado asociado las veía permanentemente y podía comunicarse con todas ellas.

\par 
%\textsuperscript{(1516.3)}
\textsuperscript{136:5.3} Antes de descender de su retiro de cuarenta días en las montañas, Jesús confió el mando inmediato de este ejército acompañante de personalidades universales a su Ajustador recientemente Personalizado. Durante más de cuatro años del tiempo de Urantia, estas personalidades seleccionadas de todas las divisiones de inteligencias universales funcionaron con obediencia y respeto bajo la sabia dirección de este Monitor de Misterio Personalizado, ensalzado y experimentado. Al asumir el mando de esta poderosa asamblea, el Ajustador, que había sido en otro tiempo parte y esencia del Padre Paradisiaco, aseguró a Jesús que en ningún caso se permitiría a estos agentes superhumanos servir o manifestarse en conexión con su carrera terrestre, o a favor de ella, a menos que fuera patente que el Padre deseaba dicha intervención. Así pues, mediante una sola gran decisión, Jesús se privó voluntariamente de toda cooperación sobrehumana en todos los asuntos relacionados con el resto de su carrera como mortal, a menos que el Padre eligiera por su cuenta participar en un acto o episodio determinado de los trabajos terrestres del Hijo.

\par 
%\textsuperscript{(1516.4)}
\textsuperscript{136:5.4} Al aceptar el mando de las huestes universales al servicio de Cristo Miguel, el Ajustador Personalizado se esmeró en señalar a Jesús que, aunque las actividades \textit{espaciales} de esta asamblea de criaturas universales podían ser limitadas por la autoridad delegada de su Creador, estas restricciones no tendrían efecto en cuanto a las funciones de estas criaturas en el \textit{tiempo}. Esta limitación se debía al hecho de que los Ajustadores son seres independientes del tiempo una vez que han sido personalizados. Por consiguiente, a Jesús se le advirtió que, aunque el control de todas las inteligencias vivientes colocadas bajo el mando del Ajustador sería completo y perfecto en todo lo relacionado con el \textit{espacio}, no se podrían imponer unas limitaciones tan perfectas en lo concerniente al \textit{tiempo}. El Ajustador le dijo: <<Tal como has ordenado, impediré que este ejército acompañante de inteligencias universales intervenga en cualquier cuestión relacionada con tu carrera terrestre, excepto en los casos en que el Padre Paradisiaco me ordene dejar actuar a estos agentes para que se cumpla su voluntad divina, tal como tú la hayas elegido, y en aquellos otros casos en que tu voluntad divina y humana pueda emprender una elección o una acción que implique desviaciones del orden terrestre natural, relacionadas exclusivamente con el \textit{tiempo}. En todos estos casos soy impotente, y tus criaturas aquí reunidas en perfección y unidad de poder son igualmente impotentes. Si tus dos naturalezas unidas albergan alguna vez tales deseos, esos mandatos tuyos serán ejecutados inmediatamente. En todos esos asuntos, tu deseo constituirá la abreviación del tiempo, y la cosa proyectada \textit{existirá}. Bajo mi autoridad, esto constituye la mayor limitación que puede imponerse a tu soberanía potencial. En mi propia conciencia el tiempo no existe, y por esta razón no puedo limitar a tus criaturas en ninguna cuestión relacionada con el tiempo>>.

\par 
%\textsuperscript{(1517.1)}
\textsuperscript{136:5.5} Jesús fue así informado de las consecuencias de su decisión de seguir viviendo como un hombre entre los hombres. Mediante una sola decisión, había excluido a todas sus huestes universales presentes de inteligencias diversas de participar en su próximo ministerio público, excepto en los asuntos relacionados exclusivamente con el \textit{tiempo}. Es pues evidente que cualquier posible manifestación sobrenatural o supuestamente superhumana que acompañara al ministerio de Jesús sólo concerniría a la eliminación del tiempo, a menos que el Padre celestial dictaminara específicamente lo contrario. Ningún milagro, ningún ministerio de misericordia, ningún otro acontecimiento posible que ocurriera en relación con el resto de la obra terrestre de Jesús, podría tener la naturaleza o el carácter de una acción que trascendiera las leyes naturales establecidas, que rigen normalmente los asuntos de los hombres tal como viven en Urantia, \textit{excepto} en esta cuestión expresamente mencionada del \textit{tiempo}. Por supuesto, ningún límite podía ser impuesto a las manifestaciones de <<la voluntad del Padre>>. La eliminación del tiempo, en conexión con el deseo expreso de este Soberano potencial de un universo, sólo podía evitarse mediante la acción directa y explícita de la \textit{voluntad} de este hombre-Dios en el sentido de que el tiempo, relacionado con el acto o el acontecimiento en cuestión, \textit{no debía ser acortado o eliminado}. A fin de impedir la aparición de \textit{milagros temporales} aparentes, Jesús tenía que permanecer constantemente consciente del tiempo. Cualquier lapsus en su conciencia del tiempo, en conexión con el mantenimiento de un deseo concreto, equivaldría a hacer efectiva la cosa concebida en la mente de este Hijo Creador, y todo ello sin la intervención del tiempo.

\par 
%\textsuperscript{(1517.2)}
\textsuperscript{136:5.6} Gracias al control supervisor de su Ajustador Personalizado y asociado, Miguel podía limitar perfectamente sus actividades terrestres personales en lo relacionado con el espacio, pero no le era posible al Hijo del Hombre limitar así su nuevo estado terrestre como Soberano potencial de Nebadon en lo referente al \textit{tiempo}. Este era el estado real de Jesús de Nazaret cuando salió para comenzar su ministerio público en Urantia.

\section*{6. La segunda decisión}
\par 
%\textsuperscript{(1517.3)}
\textsuperscript{136:6.1} Habiendo fijado su política respecto a todas las personalidades de todas las clases de inteligencias por él creadas, en la medida en que esto podía determinarse a la vista del potencial inherente a su nuevo estado de divinidad, Jesús orientó luego sus pensamientos sobre sí mismo. Ahora que era plenamente consciente de ser el creador de todas las cosas y de todos los seres existentes en este universo, ¿qué iba a hacer con estas prerrogativas de creador en las situaciones recurrentes de la vida que tendría que afrontar en cuanto regresara a Galilea para reanudar su trabajo entre los hombres?. De hecho, allí mismo donde se encontraba, en estas colinas solitarias, ya se le había presentado poderosamente este problema mediante la necesidad de conseguir comida. Al tercer día de sus meditaciones solitarias, su cuerpo humano sintió hambre. ¿Debía ir en busca de alimento como cualquier hombre común, o debía ejercer simplemente sus poderes creadores normales y producir un alimento corporal apropiado y al alcance de la mano?. Esta gran decisión del Maestro os ha sido descrita como una tentación ---como un reto de unos supuestos enemigos para que <<mande que estas piedras se conviertan en panes>>\footnote{\textit{La ``tentación'' para obtener comida}: Mt 4:3; Lc 4:3.}.

\par 
%\textsuperscript{(1518.1)}
\textsuperscript{136:6.2} Jesús estableció pues una nueva política coherente para el resto de su obra terrenal. En lo que se refería a sus necesidades personales, e incluso en general en sus relaciones con otras personalidades, eligió deliberadamente en ese momento seguir el camino de la existencia terrestre normal; se pronunció firmemente contra una línea de conducta que trascendiera, violara o ultrajara las leyes naturales establecidas por él. Pero tal como ya le había advertido su Ajustador Personalizado, no podía asegurar que en ciertas circunstancias concebibles, estas leyes naturales no pudieran resultar considerablemente \textit{aceleradas}. En principio, Jesús decidió que la obra de su vida sería organizada y continuada conforme a las leyes de la naturaleza y en armonía con la organización social existente. El Maestro eligió así un programa de vida que equivalía a la decisión de estar en contra de los milagros y de los prodigios. Una vez más se pronunció a favor de <<la voluntad del Padre>>; una vez más puso todas las cosas entre las manos de su Padre Paradisiaco.

\par 
%\textsuperscript{(1518.2)}
\textsuperscript{136:6.3} La naturaleza humana de Jesús le dictaba que su primer deber era preservar su vida; es el comportamiento normal del hombre físico en los mundos del tiempo y del espacio, y por consiguiente, la reacción legítima de un mortal de Urantia. Pero las preocupaciones de Jesús no se limitaban sólo a este mundo y a sus criaturas; estaba viviendo una vida destinada a instruir e inspirar a las múltiples criaturas de un vastísimo universo.

\par 
%\textsuperscript{(1518.3)}
\textsuperscript{136:6.4} Antes de la iluminación de su bautismo, había vivido en perfecta sumisión a la voluntad y a la orientación de su Padre celestial. Tomó la enérgica decisión de continuar viviendo con la misma dependencia implícita y humana de la voluntad del Padre. Se propuso seguir una línea de conducta antinatural ---decidió que no trataría de preservar su vida. Escogió continuar su política de negarse a defenderse. Expresó sus conclusiones con las palabras de las Escrituras, familiares para su mente humana: <<No sólo de pan vivirá el hombre, sino de toda palabra que sale de la boca de Dios>>\footnote{\textit{No sólo de pan vivirá el hombre}: Dt 8:3; Mt 4:3-4; Lc 4:3-4.}. Al llegar a esta conclusión sobre el apetito de la naturaleza física que se manifiesta como hambre, el Hijo del Hombre efectuó su declaración final sobre todas las demás necesidades de la carne y de los impulsos naturales de la naturaleza humana.

\par 
%\textsuperscript{(1518.4)}
\textsuperscript{136:6.5} Quizás podría utilizar su poder sobrehumano para ayudar a otros, pero nunca para sí mismo. Y se mantuvo fiel a esta línea de conducta hasta el final, cuando dijeron mofándose de él: <<Ha salvado a los demás, pero no puede salvarse a sí mismo>>\footnote{\textit{Salvó a otros pero no a sí mismo}: Mt 27:42a; Mc 15:31; Lc 23:35a.} ---porque no quiso hacerlo.

\par 
%\textsuperscript{(1518.5)}
\textsuperscript{136:6.6} Los judíos esperaban a un Mesías que realizara maravillas aún más grandes que Moisés, de quien se decía que había hecho manar agua de la roca en un lugar árido y que había alimentado con maná a sus antepasados en el desierto. Jesús conocía la clase de Mesías que esperaban sus compatriotas, y disponía de todos los poderes y prerrogativas para estar a la altura de sus más ardientes esperanzas, pero tomó la decisión de ponerse en contra de este magnífico programa de poder y de gloria. Jesús consideraba esta conducta de esperar acciones milagrosas como un retroceso a los antiguos tiempos de la magia ignorante y de las prácticas degeneradas de los curanderos salvajes. Quizás, para la salvación de sus criaturas, consintiera en acelerar la ley natural, pero trascender sus propias leyes, ya sea en su propio beneficio o para deslumbrar a sus semejantes, eso no lo haría. Y esta decisión del Maestro fue definitiva.

\par 
%\textsuperscript{(1518.6)}
\textsuperscript{136:6.7} Jesús se entristecía por su pueblo; comprendía plenamente cómo habían llegado a esperar al Mesías venidero, la época en que <<la tierra producirá diez mil veces más frutos, y una vid tendrá mil sarmientos, y cada sarmiento producirá mil racimos, y cada racimo producirá mil uvas, y cada uva producirá un barril de vino>>\footnote{\textit{La tierra dará frutos}: Is 4:2; Is 5:10; 1 Hen 10:18-20a.}. Los judíos creían que el Mesías inauguraría una era de abundancia milagrosa. Los hebreos se habían alimentado durante mucho tiempo de tradiciones de milagros y de leyendas de prodigios.

\par 
%\textsuperscript{(1519.1)}
\textsuperscript{136:6.8} Jesús no era un Mesías que venía para multiplicar el pan y el vino. No venía para abastecer exclusivamente las necesidades temporales; venía para hacer una revelación de su Padre celestial a sus hijos terrestres, mientras intentaba que sus hijos terrestres se unieran a él en un esfuerzo sincero por vivir según la voluntad del Padre que está en los cielos.

\par 
%\textsuperscript{(1519.2)}
\textsuperscript{136:6.9} Con esta decisión, Jesús de Nazaret describía a los espectadores de un universo la locura y el pecado de prostituir los talentos divinos y las aptitudes dadas por Dios para el engrandecimiento personal o para el beneficio y la glorificación puramente egoístas. Éste era el pecado de Lucifer y Caligastia.

\par 
%\textsuperscript{(1519.3)}
\textsuperscript{136:6.10} Esta gran decisión de Jesús ilustra dramáticamente la verdad de que la satisfacción egoísta y la gratificación sensual, solas y por sí mismas, son incapaces de aportar la felicidad a los seres humanos que evolucionan. En la existencia mortal, existen valores más elevados ---la maestría intelectual y el perfeccionamiento espiritual--- que trascienden con mucho la gratificación necesaria de los apetitos e impulsos puramente físicos del hombre. Los dones naturales del hombre, sus talentos y aptitudes, deberían emplearse principalmente para desarrollar y ennoblecer los poderes superiores de la mente y del espíritu.

\par 
%\textsuperscript{(1519.4)}
\textsuperscript{136:6.11} Jesús reveló así, a las criaturas de su universo, la técnica del camino nuevo y mejor, los valores morales superiores de la vida, y las satisfacciones espirituales más profundas de la existencia humana evolutiva en los mundos del espacio.

\section*{7. La tercera decisión}
\par 
%\textsuperscript{(1519.5)}
\textsuperscript{136:7.1} Después de tomar sus decisiones respecto a los asuntos relacionados con el alimento y el suministro físico para las necesidades de su cuerpo material, el cuidado de su salud y la de sus asociados, aún quedaban otros problemas por resolver. ¿Cómo se comportaría ante un peligro personal?. Decidió ejercer una vigilancia normal sobre su seguridad física, y tomar precauciones razonables para evitar el fin prematuro de su carrera en la carne, pero decidió abstenerse de toda intervención superhumana cuando sobreviniera la crisis de su vida en la carne. Mientras tomaba esta decisión, Jesús estaba sentado a la sombra de un árbol en un saliente rocoso, con un precipicio que se abría ante él. Se daba perfectamente cuenta que desde este saliente podía arrojarse al vacío sin sufrir ningún daño\footnote{\textit{La ``tentación'' de saltar al vacío}: Mt 4:5-6; Lc 4:9-11.}, siempre que revocara su primera gran decisión de no invocar la intervención de sus inteligencias celestiales para continuar la obra de su vida en Urantia, y siempre que anulara su segunda decisión sobre su comportamiento respecto a la preservación de su vida.

\par 
%\textsuperscript{(1519.6)}
\textsuperscript{136:7.2} Jesús sabía que sus compatriotas esperaban un Mesías que estuviera por encima de las leyes naturales. Le habían enseñado bien aquel pasaje de las Escrituras: <<No te sucederá ningún mal, y ninguna plaga se acercará a tu morada. Pues te confiará al cuidado de sus ángeles para que te guarden en todos tus caminos. Te llevarán en sus manos, para que tu pie no tropiece contra una piedra>>\footnote{\textit{No te sucederá ningún mal}: Sal 91:10-12.}. Esta especie de presunción, este desafío a las leyes de la gravedad de su Padre, ¿estarían justificados para protegerse de un posible daño o quizás para ganarse la confianza de su pueblo mal enseñado y desorientado?. Esta línea de conducta, por muy satisfactoria que fuera para los judíos en busca de signos, no sería una revelación de su Padre, sino una dudosa manipulación de las leyes establecidas en el universo de universos.

\par 
%\textsuperscript{(1519.7)}
\textsuperscript{136:7.3} Comprendiendo todo esto y sabiendo que el Maestro se negaba a trabajar desafiando sus leyes establecidas de la naturaleza en lo que concernía a su conducta personal, sabéis con certidumbre que nunca caminó sobre las aguas\footnote{\textit{Jesús nunca caminó sobre las aguas}: Mt 14:25-27; Mc 6:48-50; Jn 6:19-21.} y que nunca hizo nada que violara su orden material de administrar el mundo\footnote{\textit{Ningún signo sobrenatural}: Mt 12:38; Mt 16:1; Mc 8:11.}. Por supuesto, recordad siempre que aún no se había encontrado la manera de librarlo por completo de la falta de control sobre el elemento tiempo en conexión con los asuntos entregados a la jurisdicción del Ajustador Personalizado.

\par 
%\textsuperscript{(1520.1)}
\textsuperscript{136:7.4} Durante toda su vida terrenal, Jesús permaneció constantemente fiel a esta decisión. Aunque los fariseos le provocaron pidiéndole un signo, y los espectadores en el Calvario le desafiaron a que descendiera de la cruz, mantuvo firmemente la decisión que tomó en esta hora en la ladera de la montaña\footnote{\textit{Incluso cuando quienes le veían le provocaron}: Mt 27:39-44; Mc 15:29-32; Lc 23:35-37.}.

\section*{8. La cuarta decisión}
\par 
%\textsuperscript{(1520.2)}
\textsuperscript{136:8.1} El gran problema siguiente con el que tuvo que luchar este hombre-Dios y que pronto resolvió de acuerdo con la voluntad del Padre celestial consistía en saber si debía o no emplear algunos de sus poderes sobrehumanos para atraer la atención y conseguir la adhesión de sus semejantes. ¿Debía emplear, de alguna manera, sus poderes universales para satisfacer la inclinación de los judíos por lo espectacular y lo maravilloso?. Decidió que no haría nada semejante. Se ratificó en una línea de conducta que eliminaba todas estas prácticas como método para llevar su misión al conocimiento de los hombres. Y vivió constantemente de acuerdo con esta gran decisión. Incluso en los numerosos casos en que permitió manifestaciones de misericordia que comportaron un acortamiento del tiempo\footnote{\textit{Que no contaran nada sobre el acortamiento del tiempo}: Mt 8:4; Mt 9:30; Mt 12:16; Mc 1:44; Mc 5:43; Mc 7:36; Mc 8:26; Lc 5:14; Lc 8:56.}, casi invariablemente recomendó a los que recibieron su ministerio curativo que no contaran a nadie los beneficios que habían recibido. Siempre rechazó el desafío sarcástico de sus enemigos cuando le pedían <<muéstranos un signo>>\footnote{\textit{Jesús rehusó mostrar signos}: Mt 12:38-39; Mt 16:1-4; Mc 8:11-12; Lc 11:16,29-30; Jn 2:18-20; 6:30.} como prueba y demostración de su divinidad.

\par 
%\textsuperscript{(1520.3)}
\textsuperscript{136:8.2} Jesús preveía muy sabiamente que la realización de milagros y la ejecución de prodigios sólo produciría una lealtad superficial mediante la intimidación de la mente material; tales acciones no revelarían a Dios ni salvarían a los hombres. Se negó a ser simplemente un hacedor de prodigios. Resolvió que se ocuparía de una sola tarea: el establecimiento del reino de los cielos.

\par 
%\textsuperscript{(1520.4)}
\textsuperscript{136:8.3} Durante todo este importante diálogo de Jesús en comunión consigo mismo, el elemento humano que interroga y casi duda estaba presente, porque Jesús era hombre a la vez que Dios. Era evidente que los judíos nunca lo aceptarían como Mesías si no hacía prodigios. Además, si consentía en hacer una sola cosa no natural, la mente humana sabría con certidumbre que era por subordinación a una mente verdaderamente divina. Para la mente divina, ¿sería compatible con <<la voluntad del Padre>>\footnote{\textit{Jesús decide vivir la voluntad del Padre}: Mt 26:39,42,44; Mc 14:36,39; Lc 22:42; Jn 4:34; 5:30; 6:38-40; 15:10; 17:4.} hacer esta concesión a la naturaleza dubitativa de la mente humana?. Jesús decidió que sería incompatible, y citó la presencia del Ajustador Personalizado como prueba suficiente de la divinidad asociada con la humanidad.

\par 
%\textsuperscript{(1520.5)}
\textsuperscript{136:8.4} Jesús había viajado mucho; recordaba Roma, Alejandría y Damasco. Conocía los modos de obrar del mundo ---cómo la gente conseguía sus propósitos en la política y en el comercio por medio de compromisos y diplomacia. ¿Utilizaría este conocimiento para hacer avanzar su misión en la Tierra?. ¡No!. Se pronunció igualmente contra todo compromiso con la sabiduría del mundo y la influencia de las riquezas para establecer el reino. De nuevo escogió depender exclusivamente de la voluntad del Padre.

\par 
%\textsuperscript{(1520.6)}
\textsuperscript{136:8.5} Jesús se daba perfectamente cuenta de los atajos que se abrían para alguien con sus poderes. Conocía muchas maneras de atraer la atención inmediata de la nación y del mundo entero sobre su persona. Pronto se celebraría la Pascua en Jerusalén; la ciudad estaría llena de visitantes. Podía ascender al pináculo del templo y, ante las multitudes asombradas, caminar por el aire\footnote{\textit{Decide no caminar en el aire desde el templo}: Mt 4:5-6; Lc 4:9-11.}; éste era el tipo de Mesías que la gente esperaba. Pero después los desilusionaría, puesto que no había venido para volver a establecer el trono de David. Y conocía la futilidad del método de Caligastia, consistente en tratar de adelantarse a la manera natural, lenta y segura de llevar a cabo el propósito divino. Una vez más, el Hijo del Hombre se inclinó con obediencia ante la vía del Padre, la voluntad del Padre.

\par 
%\textsuperscript{(1521.1)}
\textsuperscript{136:8.6} Jesús escogió establecer el reino de los cielos en el corazón de los hombres mediante métodos naturales, normales, difíciles y penosos, los mismos procedimientos que sus hijos terrestres tendrían que seguir posteriormente en su trabajo de ampliar y extender este reino celestial. El Hijo del Hombre sabía muy bien que sería <<a través de muchas tribulaciones como muchos hijos de todos los tiempos entrarían en el reino>>\footnote{\textit{Entrar al reino a través de muchas tribulaciones}: Hch 14:22.}. Jesús estaba pasando ahora por la gran prueba de los hombres civilizados, la de tener el poder y negarse firmemente a utilizarlo para fines puramente egoístas o personales.

\par 
%\textsuperscript{(1521.2)}
\textsuperscript{136:8.7} Al estudiar la vida y la experiencia del Hijo del Hombre, deberíais tener siempre presente el hecho de que el Hijo de Dios estaba encarnado en la mente de un ser humano del siglo primero, y no en la mente de un mortal del siglo veinte o de otro siglo. Con esto deseamos transmitiros la idea de que los dones humanos de Jesús habían sido adquiridos por vía natural. Él era el producto de los factores hereditarios y ambientales de su época, unidos a la influencia de su instrucción y de su educación. Su humanidad era auténtica, natural, totalmente derivada y alimentada por los antecedentes de la situación intelectual real y de las condiciones económicas y sociales de aquella época y de aquella generación. Aunque en la experiencia de este hombre-Dios siempre existía la posibilidad de que la mente divina trascendiera al intelecto humano, sin embargo, siempre que funcionaba su mente humana, lo hacía como lo haría una verdadera mente mortal en las condiciones del entorno humano de aquella época.

\par 
%\textsuperscript{(1521.3)}
\textsuperscript{136:8.8} Jesús ilustró para todos los mundos de su vasto universo la locura de crear situaciones artificiales con el propósito de mostrar una autoridad arbitraria, o de permitirse un poder excepcional, para realzar los valores morales o acelerar el progreso espiritual. Jesús decidió que, durante su misión en la Tierra, no se prestaría a repetir la decepción del reinado de los Macabeos. Se negó a prostituir sus atributos divinos para adquirir una popularidad no merecida o para conseguir un prestigio político. No consentiría en transmutar la energía divina y creativa en poder nacional o en prestigio internacional. Jesús de Nazaret se negó a hacer compromisos con el \textit{mal}, y mucho menos a asociarse con el pecado. El Maestro colocó triunfalmente la fidelidad a la voluntad de su Padre por encima de cualquier otra consideración terrestre y temporal.

\section*{9. La quinta decisión}
\par 
%\textsuperscript{(1521.4)}
\textsuperscript{136:9.1} Habiendo establecido el criterio a seguir en lo referente a sus relaciones individuales con las leyes naturales y el poder espiritual, dirigió su atención hacia la elección de los métodos que emplearía para proclamar y establecer el reino de Dios. Juan ya había iniciado este trabajo; ¿cómo podría Jesús continuar el mensaje?. ¿Cómo debería seguir con la misión de Juan?. ¿Cómo debería organizar a sus seguidores para que el esfuerzo resultara eficaz y la cooperación inteligente?. Jesús estaba llegando ahora a la decisión final que le impediría seguir considerándose el Mesías judío, al menos tal como la población concebía al Mesías en aquella época.

\par 
%\textsuperscript{(1522.1)}
\textsuperscript{136:9.2} Los judíos imaginaban a un libertador que llegaría con un poder milagroso para derribar a los enemigos de Israel y establecer a los judíos como gobernantes del mundo, libres de la miseria y de la opresión. Jesús sabía que esta esperanza no se materializaría nunca. Sabía que el reino de los cielos concernía a la victoria sobre el mal en el corazón de los hombres, y que se trataba de un asunto puramente espiritual. Reflexionó sobre la conveniencia de inaugurar el reino espiritual con una brillante y deslumbrante demostración de poder ---esta línea de conducta hubiera sido permisible y estaba totalmente dentro de la jurisdicción de Miguel--- pero adoptó una posición totalmente contraria a este plan. No transigiría con las técnicas revolucionarias de Caligastia. Había ganado potencialmente el mundo sometiéndose a la voluntad del Padre, y se propuso terminar su obra como la había empezado, y como Hijo del Hombre.

\par 
%\textsuperscript{(1522.2)}
\textsuperscript{136:9.3} ¡Es difícil que podáis imaginar lo que hubiera sucedido en Urantia si este hombre-Dios, ahora en posesión potencial de todos los poderes en el cielo y en la Tierra, hubiera decidido desplegar una sola vez el estandarte de la soberanía, formar su prodigioso ejército en orden de batalla!. Pero no transigiría. No serviría al mal para que se pudiera suponer que la adoración de Dios provenía de ello. Permaneció fiel a la voluntad del Padre. Proclamaría a un universo que lo observaba: <<Adoraréis al Señor vuestro Dios, y a él solo serviréis>>\footnote{\textit{Servir sólo al Señor Dios}: Ex 20:3-5a; Dt 5:7-9a; 6:13-14; 10:20; Mt 4:10; Lc 4:8.}.

\par 
%\textsuperscript{(1522.3)}
\textsuperscript{136:9.4} A medida que pasaban los días, Jesús percibía con mayor claridad la clase de revelador de la verdad que iba a ser. Discernía que el camino de Dios no iba a ser un camino fácil. Empezó a darse cuenta de que el resto de su experiencia humana podría ser un amargo cáliz\footnote{\textit{Beber el cáliz amargo}: Mt 20:22-23; Mt 26:39,42; Mc 10:38-39; Mc 14:36; Lc 22:42; Jn 18:11.}, pero decidió beberlo.

\par 
%\textsuperscript{(1522.4)}
\textsuperscript{136:9.5} Incluso su mente humana dice adiós al trono de David. Paso a paso, esta mente humana se mueve en el sendero de lo divino. La mente humana todavía hace preguntas, pero acepta invariablemente las respuestas divinas como regla final, en esta existencia combinada de vivir como un hombre en el mundo mientras se somete todo el tiempo, de forma incondicional, a hacer la voluntad eterna y divina del Padre.

\par 
%\textsuperscript{(1522.5)}
\textsuperscript{136:9.6} Roma era la dueña del mundo occidental. El Hijo del Hombre, ahora en su aislamiento, tomando estas importantes decisiones, con las huestes del cielo a sus órdenes, representaba la última oportunidad de los judíos para conseguir el dominio del mundo; pero este judío de nacimiento, dotado de una sabiduría y de un poder tan extraordinarios, no quiso emplear sus dones universales para encumbrarse personalmente ni para entronizar a su pueblo\footnote{\textit{``Tentación'' para controlar los reinos}: Mt 4:8; Lc 4:5.}. Veía, por decirlo así, <<los reinos de este mundo>>, y poseía el poder para apoderarse de ellos. Los Altísimos de Edentia habían puesto estos poderes en sus manos, pero no los quería. Los reinos de la Tierra eran cosas mezquinas, indignas del interés del Creador y Soberano de un universo. Sólo tenía un objetivo: la revelación posterior de Dios al hombre, el establecimiento del reino, la soberanía del Padre celestial en el corazón de los hombres.

\par 
%\textsuperscript{(1522.6)}
\textsuperscript{136:9.7} Las ideas de batallas, contiendas y masacres repugnaban a Jesús; no quería nada de eso. Aparecería en la Tierra como el Príncipe de la Paz para revelar a un Dios de amor. Antes de su bautismo había rechazado de nuevo otra oferta de los celotes para encabezar su rebelión contra los opresores romanos. Ahora, tomó la decisión final con respecto a los pasajes de las Escrituras que su madre le había enseñado, tales como: <<El Señor me ha dicho: `Tú eres mi Hijo; te he engendrado hoy. Pídeme, y te daré a los paganos por herencia y los confines de la Tierra como posesión. Los quebrantarás con mano de hierro; los despedazarás como una vasija de alfarero'>>\footnote{\textit{Tú eres mi Hijo; te he engendrado hoy}: Sal 2:7-9; Hch 13:33.}.

\par 
%\textsuperscript{(1522.7)}
\textsuperscript{136:9.8} Jesús de Nazaret llegó a la conclusión de que estas citas no se referían a él. Por último, y de una vez por todas, la mente humana del Hijo del Hombre barrió por completo todas estas dificultades y contradicciones mesiánicas ---las escrituras hebreas, la educación de los padres, la enseñanza del chazan, las expectativas de los judíos y los ambiciosos deseos humanos. Decidió su línea de conducta de manera definitiva. Regresaría a Galilea y empezaría tranquilamente la proclamación del reino, confiando en su Padre (el Ajustador Personalizado) para elaborar los detalles cotidianos de actuación.

\par 
%\textsuperscript{(1523.1)}
\textsuperscript{136:9.9} Con estas decisiones, Jesús sentó un digno ejemplo para todas las personas de todos los mundos de un vasto universo al negarse a aplicar pruebas materiales para demostrar los problemas espirituales, al negarse a desafiar presuntuosamente las leyes naturales. Y dio un ejemplo inspirador de lealtad universal y de nobleza moral cuando se negó a coger el poder temporal como preludio de la gloria espiritual.

\par 
%\textsuperscript{(1523.2)}
\textsuperscript{136:9.10} Si el Hijo del Hombre tenía dudas acerca de su misión y de la naturaleza de ésta cuando subió a las colinas después de su bautismo, ya no tenía ninguna cuando volvió entre sus compañeros después de los cuarenta días de aislamiento y de decisiones.

\par 
%\textsuperscript{(1523.3)}
\textsuperscript{136:9.11} Jesús ha elaborado un programa para establecer el reino del Padre. No alimentará las satisfacciones físicas de la gente. No distribuirá pan a las multitudes como vio hacer tan recientemente en Roma. No atraerá la atención sobre sí mismo haciendo prodigios, a pesar de que los judíos esperan precisamente un libertador de esta índole. Tampoco intentará que acepten su mensaje espiritual mediante una exhibición de autoridad política o de poder temporal.

\par 
%\textsuperscript{(1523.4)}
\textsuperscript{136:9.12} Al rechazar estos métodos que realzarían el reino venidero a los ojos de los judíos que lo esperaban, Jesús contaba con que estos mismos judíos rechazarían a fin de cuentas y con seguridad todos sus derechos a la autoridad y a la divinidad. Sabiendo todo esto, Jesús trató de evitar durante mucho tiempo que sus primeros discípulos hablaran de él como si fuera el Mesías.

\par 
%\textsuperscript{(1523.5)}
\textsuperscript{136:9.13} Durante todo su ministerio público, tuvo que enfrentarse constantemente con tres situaciones recurrentes: el clamor para ser alimentados, la insistencia en ver milagros, y la petición final de que permitiera a sus seguidores coronarlo rey. Pero Jesús no se apartó nunca de las decisiones que había tomado durante estos días de aislamiento en las colinas de Perea.

\section*{10. La sexta decisión}
\par 
%\textsuperscript{(1523.6)}
\textsuperscript{136:10.1} El último día de este retiro memorable, antes de bajar de la montaña para reunirse con Juan y sus discípulos, el Hijo del Hombre tomó su decisión final. Y la comunicó al Ajustador Personalizado en estos términos: <<En todas las demás cuestiones, al igual que en estas decisiones ya registradas, te prometo que me someteré a la voluntad de mi Padre>>\footnote{\textit{Someterse a la voluntad del Padre}: Mt 26:39,42,44; Mc 14:36,39; Lc 22:42; Jn 4:34; 5:30; 6:38-40; 15:10; 17:4.}. Después de haber dicho esto, descendió de la montaña. Y su faz resplandecía con la gloria de las victorias espirituales y de las proezas morales.


\chapter{Documento 137. El tiempo de espera en Galilea}
\par 
%\textsuperscript{(1524.1)}
\textsuperscript{137:0.1} EL SÁBADO 23 de febrero del año 26, por la mañana temprano, Jesús descendió de las colinas para reunirse con los compañeros de Juan que acampaban en Pella. Todo este día Jesús se mezcló con la multitud. Atendió a un chico que se había lastimado en una caída y se desplazó hasta el cercano pueblo de Pella para poner al niño a salvo en manos de sus padres.

\section*{1. La elección de los cuatro primeros apóstoles}
\par 
%\textsuperscript{(1524.2)}
\textsuperscript{137:1.1} Durante este sábado, dos de los principales discípulos de Juan pasaron mucho tiempo con Jesús. De todos los seguidores de Juan, uno llamado Andrés es el que estaba más profundamente impresionado por Jesús. Lo acompañó hasta Pella con el muchacho lesionado, y por el camino de vuelta al campamento de Juan le hizo muchas preguntas a Jesús; poco antes de llegar a su destino, los dos se detuvieron para tener una breve conversación, durante la cual Andrés dijo: <<Te he estado observando desde que viniste a Cafarnaúm, y creo que eres el nuevo Instructor; aunque no comprendo toda tu enseñanza, estoy plenamente decidido a seguirte. Quisiera sentarme a tus pies para aprender toda la verdad sobre el nuevo reino>>. Con una cordial resolución, Jesús acogió a Andrés como el primer apóstol\footnote{\textit{Selección de Andrés}: Mt 4:18-20; Mc 1:16-18; Jn 1:40.} de aquel grupo de doce que iba a trabajar con él en la obra de establecer el nuevo reino de Dios en el corazón de los hombres.

\par 
%\textsuperscript{(1524.3)}
\textsuperscript{137:1.2} Andrés había observado en silencio la labor de Juan y creía sinceramente en ella. Tenía un hermano muy capaz y entusiasta, llamado Simón, que era uno de los principales discípulos de Juan. No sería impropio decir que Simón era uno de los apoyos más importantes de Juan.

\par 
%\textsuperscript{(1524.4)}
\textsuperscript{137:1.3} Poco después de que Jesús y Andrés regresaran al campamento, Andrés buscó a su hermano Simón y llevándolo aparte le comunicó que estaba convencido de que Jesús era el gran Instructor, y que se había comprometido a ser su discípulo. Continuó diciendo que Jesús había aceptado su propuesta de servicio, y le sugirió que él (Simón) fuera también a Jesús y se ofreciera para unirse al servicio del nuevo reino. Simón dijo: <<Desde que ese hombre vino a trabajar al taller de Zebedeo, he creído que había sido enviado por Dios, pero ¿qué hacemos con Juan? ¿Vamos a abandonarlo? ¿Es esto lo que debemos hacer?>> Con lo cual, acordaron ir enseguida a consultar a Juan. Juan se entristeció con la idea de perder a dos de sus capaces consejeros y más prometedores discípulos, pero contestó valientemente a sus preguntas diciendo: <<Esto sólo es el principio; mi obra terminará dentro de poco y todos nos convertiremos en sus discípulos>>. Entonces Andrés le hizo señas a Jesús y le anunció aparte que su hermano deseaba entrar al servicio del nuevo reino. Al acoger a Simón como su segundo apóstol, Jesús le dijo: <<Simón, tu entusiasmo es loable, pero peligroso para el trabajo del reino. Te recomiendo que seas más cuidadoso con tus palabras. Desearía cambiar tu nombre por el de Pedro>>\footnote{\textit{Selección de Simón Pedro}: Mt 4:18-20; Mc 1:16-18; Lc 5:1-9; Jn 1:41-42.}.

\par 
%\textsuperscript{(1525.1)}
\textsuperscript{137:1.4} Los padres del chico lastimado, que vivían en Pella, habían rogado a Jesús que pasara la noche con ellos, que se considerara como en su casa, y él había prometido volver. Antes de separarse de Andrés y de su hermano, Jesús les dijo: <<Mañana temprano iremos a Galilea>>.

\par 
%\textsuperscript{(1525.2)}
\textsuperscript{137:1.5} Después de que Jesús hubiera regresado a Pella para pasar la noche, y mientras que Andrés y Simón discutían todavía sobre la naturaleza de su servicio en el establecimiento del reino por venir, Santiago y Juan, los hijos de Zebedeo, llegaron al lugar. Acababan de regresar de su larga e inútil búsqueda de Jesús en las colinas. Cuando oyeron contar a Simón Pedro cómo él y su hermano Andrés se habían convertido en los primeros consejeros aceptados del nuevo reino, y que iban a partir a la mañana siguiente con su nuevo Maestro para Galilea, Santiago y Juan se entristecieron. Conocían a Jesús desde hacía tiempo y lo amaban. Lo habían buscado durante muchos días en las colinas, y ahora regresaban para enterarse de que otros habían sido elegidos antes que ellos. Preguntaron adónde había ido Jesús y se dieron prisa en encontrarlo.

\par 
%\textsuperscript{(1525.3)}
\textsuperscript{137:1.6} Jesús estaba durmiendo cuando llegaron a su habitación, pero lo despertaron diciendo: <<Mientras nosotros, que hemos vivido tanto tiempo contigo, te buscábamos en las colinas, ¿cómo es que prefieres a otros antes que a nosotros, y escoges a Andrés y a Simón como tus primeros asociados en el nuevo reino?>> Jesús les respondió: <<Serenad vuestro corazón y preguntaos, `¿quién os ha ordenado buscar al Hijo del Hombre mientras se dedicaba a los asuntos de su Padre?'>>. Después de contar los detalles de su larga búsqueda en las colinas, Jesús continuó enseñándoles: <<Deberíais aprender a buscar el secreto del nuevo reino en vuestro corazón, y no en las colinas. Aquello que buscabais ya estaba presente en vuestra alma. En verdad sois mis hermanos ---no necesitabais que yo os aceptara--- ya pertenecíais al reino. Tened buen ánimo y preparaos también para acompañarnos mañana a Galilea>>. Juan se atrevió entonces a preguntar: <<Pero, Maestro, ¿Santiago y yo seremos tus asociados en el nuevo reino, como lo son Andrés y Simón?>> Jesús puso una mano en el hombro de cada uno de ellos y dijo: <<Hermanos míos, ya estabais conmigo en el espíritu del reino, incluso antes de que los otros solicitaran ser admitidos. Vosotros, mis hermanos, no tenéis necesidad de presentar una petición para entrar en el reino; habéis estado conmigo en el reino desde el principio. Ante los hombres, otros pueden tener prioridad sobre vosotros, pero en mi corazón ya contaba con vosotros para los consejos del reino, incluso antes de que pensarais en pedírmelo. También podríais haber sido los primeros ante los hombres, si no os hubierais ausentado para dedicaros a la tarea bien intencionada, pero impuesta por vosotros mismos, de buscar a alguien que no estaba perdido. En el reino venidero, no os preocupéis por las cosas que alimentan vuestra ansiedad, sino más bien interesaos en hacer solamente, en todo momento, la voluntad del Padre que está en los cielos>>\footnote{\textit{Selección de Santiago y Juan}: Mt 4:21-22; Mc 1:19-20; Lc 5:10-11.}.

\par 
%\textsuperscript{(1525.4)}
\textsuperscript{137:1.7} Santiago y Juan aceptaron la reprimenda de buena gana; nunca más tuvieron envidia de Andrés y de Simón. Se prepararon para salir a la mañana siguiente para Galilea con los otros dos apóstoles asociados. A partir de este día, la palabra `apóstol' fue empleada para diferenciar a la familia elegida de consejeros de Jesús de la vasta multitud de discípulos creyentes que le siguieron posteriormente.

\par 
%\textsuperscript{(1525.5)}
\textsuperscript{137:1.8} Avanzada la noche, Santiago, Juan, Andrés y Simón mantuvieron una conversación con Juan el Bautista. Con lágrimas en los ojos pero con voz firme, el fornido profeta judeo renunció a dos de sus principales discípulos para que fueran los apóstoles del Príncipe galileo del reino por venir.

\section*{2. La elección de Felipe y de Natanael}
\par 
%\textsuperscript{(1526.1)}
\textsuperscript{137:2.1} El domingo por la mañana 24 de febrero del año 26, Jesús se despidió de Juan el Bautista al borde del río cerca de Pella, para no volverlo a ver nunca más en la carne.

\par 
%\textsuperscript{(1526.2)}
\textsuperscript{137:2.2} Aquel día, mientras Jesús y sus cuatro discípulos-apóstoles partían para Galilea\footnote{\textit{Jesús en Galilea}: Mt 4:23; Mc 1:21; Lc 4:14; Jn 1:43.}, un gran alboroto tuvo lugar en el campamento de los seguidores de Juan. La primera gran división estaba a punto de producirse. El día anterior, Juan había dicho explícitamente a Andrés y a Esdras que Jesús era el Libertador\footnote{\textit{Pronunciamiento de Juan}: Jn 1:35-37.}. Andrés decidió seguir a Jesús, pero Esdras rechazó al apacible carpintero de Nazaret, proclamando a sus asociados: <<El profeta Daniel afirma que el Hijo del Hombre vendrá con las nubes del cielo, lleno de poder y gran gloria. Este carpintero galileo, este constructor de barcas de Cafarnaúm, no puede ser el Libertador. Un don semejante de Dios, ¿puede salir de Nazaret? Ese Jesús es un pariente de Juan, y nuestro maestro se ha dejado engañar por la gran bondad de su corazón. Mantengámonos apartados de ese falso Mesías>>\footnote{\textit{Cita de Daniel}: Dn 7:13.}. Cuando Juan le regañó por estas declaraciones, Esdras se retiró llevándose a muchos discípulos y se dirigió apresuradamente hacia el sur. Este grupo continuó bautizando en nombre de Juan y fundó finalmente una secta con aquellos que creían en Juan pero rehusaban aceptar a Jesús. Un resto de este grupo aún sobrevive en Mesopotamia en la actualidad.

\par 
%\textsuperscript{(1526.3)}
\textsuperscript{137:2.3} Mientras estos disturbios se fraguaban entre los seguidores de Juan, Jesús y sus cuatro discípulos-apóstoles avanzaban a buen paso hacia Galilea. Antes de cruzar el Jordán para ir a Nazaret por el camino de Naín, Jesús miró hacia adelante y vio por la carretera a un tal Felipe de Betsaida que venía hacia ellos con un amigo. Jesús había conocido a Felipe anteriormente, y los cuatro nuevos apóstoles también lo conocían bien. Iba de camino con su amigo Natanael para ver a Juan en Pella a fin de informarse mejor sobre la llegada anunciada del reino de Dios, y se sintió encantado de saludar a Jesús. Felipe había admirado a Jesús desde que vino por primera vez a Cafarnaúm. Pero Natanael, que vivía en Caná de Galilea, no conocía a Jesús. Felipe se adelantó para saludar a sus amigos, mientras Natanael descansaba a la sombra de un árbol al borde del camino.

\par 
%\textsuperscript{(1526.4)}
\textsuperscript{137:2.4} Pedro llevó aparte a Felipe y procedió a explicarle que todos ellos, refiriéndose a él mismo, Andrés, Santiago y Juan, se habían vuelto compañeros de Jesús en el nuevo reino, e incitó vivamente a Felipe a que se ofreciera para este servicio. Felipe se encontró en un aprieto. ¿Qué debía hacer? Aquí, sin el menor preaviso ---al borde del camino cerca del Jordán--- había surgido la cuestión más importante de toda una vida, y tenía que tomar una decisión inmediata. Mientras Felipe conversaba seriamente con Pedro, Andrés y Juan, Jesús describía a Santiago el camino a seguir a través de Galilea hasta Cafarnaúm. Finalmente, Andrés sugirió a Felipe: <<¿Por qué no le preguntas al Maestro?>>.

\par 
%\textsuperscript{(1526.5)}
\textsuperscript{137:2.5} Felipe se dio cuenta repentinamente de que Jesús era realmente un gran hombre, posiblemente el Mesías, y decidió atenerse a lo que Jesús decidiera en este asunto. Fue directamente hacia él y le preguntó: <<Maestro, ¿debo ir hasta Juan o unirme a mis amigos que te siguen?>> Y Jesús respondió: <<Sígueme>>. Felipe se emocionó con la certidumbre de haber encontrado al Libertador\footnote{\textit{Selección de Felipe}: Jn 1:43-44.}.

\par 
%\textsuperscript{(1526.6)}
\textsuperscript{137:2.6} Entonces Felipe le hizo señas al grupo para que permanecieran donde estaban, mientras se apresuraba a revelar su decisión a su amigo Natanael\footnote{\textit{Felipe y Natanael}: Jn 1:45-46.}, que aún continuaba debajo de la morera reflexionando sobre todas las cosas que había oído respecto a Juan el Bautista, el reino por venir y el Mesías esperado. Felipe interrumpió esta meditación, exclamando: <<He encontrado al Libertador, aquel de quien han escrito Moisés y los profetas y a quien Juan ha proclamado>>. Natanael levantó la vista e inquirió: <<¿De dónde viene ese maestro?>> Y Felipe replicó: <<Es Jesús de Nazaret, el hijo de José, el carpintero, que reside desde hace poco en Cafarnaúm>>. Entonces Natanael, un poco sobresaltado, preguntó: <<¿Una cosa tan buena puede salir de Nazaret?>> Pero Felipe, cogiéndolo por el brazo, le dijo: <<Ven a ver>>.

\par 
%\textsuperscript{(1527.1)}
\textsuperscript{137:2.7} Felipe condujo a Natanael hasta Jesús, el cual, mirando bondadosamente de frente a este hombre sincero que dudaba, dijo: <<He aquí a un auténtico israelita, en quien no hay falsedad. Sígueme>>. Y Natanael, volviéndose hacia Felipe, le dijo: <<Tienes razón. Es en verdad un maestro de hombres. Yo también le seguiré, si soy digno>>. Jesús hizo un gesto afirmativo con la cabeza a Natanael, diciéndole de nuevo: <<Sígueme>>\footnote{\textit{La selección de Natanael}: Jn 1:47-51.}.

\par 
%\textsuperscript{(1527.2)}
\textsuperscript{137:2.8} Jesús ya había reunido a la mitad de su futuro cuerpo de asociados íntimos, cinco que lo conocían desde hacía algún tiempo más un extraño, Natanael. Sin más dilación, atravesaron el Jordán, pasaron por el pueblo de Naín y al final de la tarde llegaron a Nazaret.

\par 
%\textsuperscript{(1527.3)}
\textsuperscript{137:2.9} Todos pasaron la noche con José, en la casa de la infancia de Jesús. Los compañeros de Jesús no entendieron muy bien por qué su maestro recién descubierto estaba tan preocupado por destruir completamente todos los vestigios de su escritura que permanecían en la casa, tales como los Diez Mandamientos y otras sentencias y refranes. Pero esta conducta, unida al hecho de que nunca más lo vieron escribir ---excepto en el polvo o en la arena--- hizo una profunda impresión en sus mentes.

\section*{3. La visita a Cafarnaúm}
\par 
%\textsuperscript{(1527.4)}
\textsuperscript{137:3.1} Al día siguiente, Jesús envió a sus apóstoles a Caná, ya que todos ellos estaban invitados a la boda\footnote{\textit{La boda en Caná}: Jn 2:1-2.} de una joven sobresaliente de aquella ciudad, mientras él se preparaba para hacerle una breve visita a su madre en Cafarnaúm, deteniéndose en Magdala para ver a su hermano Judá.

\par 
%\textsuperscript{(1527.5)}
\textsuperscript{137:3.2} Antes de salir de Nazaret, los nuevos asociados de Jesús contaron a José y a otros miembros de la familia de Jesús los acontecimientos maravillosos del entonces pasado reciente, y expresaron francamente su creencia de que Jesús era el libertador tanto tiempo esperado. Estos miembros de la familia de Jesús discutieron sobre todo esto, y José dijo: <<Después de todo, quizás mamá tenía razón ---quizás nuestro extraño hermano sea el futuro rey>>.

\par 
%\textsuperscript{(1527.6)}
\textsuperscript{137:3.3} Judá había estado presente en el bautismo de Jesús y, con su hermano Santiago, se había vuelto un firme creyente en la misión de Jesús en la Tierra. Aunque tanto Santiago como Judá estaban muy perplejos respecto a la naturaleza de la misión de su hermano, su madre había resucitado todas sus antiguas esperanzas de que Jesús sería el Mesías, el hijo de David, y animaba a sus hijos a que tuvieran fe en su hermano como libertador de Israel.

\par 
%\textsuperscript{(1527.7)}
\textsuperscript{137:3.4} Jesús llegó a Cafarnaúm el lunes por la noche, pero no fue a su propia casa, donde vivían Santiago y su madre; fue directamente a la casa de Zebedeo. Todos sus amigos de Cafarnaúm advirtieron un cambio grande y agradable en él. Una vez más parecía relativamente contento y más semejante a como había sido durante sus primeros años en Nazaret. En los años anteriores a su bautismo y a los períodos de aislamiento justo antes y después del mismo, se había vuelto cada vez más serio y reservado. Ahora, a todos ellos les parecía que volvía a ser como antes. Había en él algo de importancia majestuosa y de aspecto sublime, pero estaba nuevamente desenfadado y alegre.

\par 
%\textsuperscript{(1528.1)}
\textsuperscript{137:3.5} María se estremecía de esperanza. Preveía que la promesa de Gabriel iba a cumplirse próximamente. Esperaba que pronto toda Palestina se quedaría sorprendida y pasmada ante la revelación milagrosa de su hijo como rey sobrenatural de los judíos. Pero a las numerosas preguntas que le hicieron su madre, Santiago, Judá y Zebedeo, Jesús se limitó a responder sonriendo: <<Es mejor que me quede aquí durante algún tiempo; debo hacer la voluntad de mi Padre que está en los cielos>>.

\par 
%\textsuperscript{(1528.2)}
\textsuperscript{137:3.6} Al día siguiente, martes, todos fueron a Caná para asistir a la boda de Noemí, que iba a celebrarse al otro día. A pesar de las advertencias reiteradas de Jesús de que no hablaran a nadie de él <<hasta que llegara la hora del Padre>>, ellos insistieron en divulgar discretamente la noticia de que habían encontrado al Libertador. Cada uno de ellos esperaba con confianza que Jesús inauguraría la toma de posesión de su autoridad mesiánica en la próxima boda de Caná, y que lo haría con un gran poder y una grandeza sublime. Recordaban lo que les habían dicho sobre los fenómenos que acompañaron a su bautismo, y creían que su carrera futura en la Tierra estaría marcada de manifestaciones crecientes de maravillas sobrenaturales y de demostraciones milagrosas. En consecuencia, toda la región se preparó para reunirse en Caná para la fiesta nupcial de Noemí y Johab, el hijo de Natán.

\par 
%\textsuperscript{(1528.3)}
\textsuperscript{137:3.7} Hacía años que María no estaba tan alegre. Viajó hasta Caná con el ánimo de una reina madre que va a presenciar la coronación de su hijo. Desde que Jesús tenía trece años, su familia y sus amigos no lo habían visto tan despreocupado y feliz, tan atento y comprensivo con los anhelos y deseos de sus asociados, tan tiernamente compasivo. Así que todos cuchicheaban entre ellos, en pequeños grupos, preguntándose qué iba a suceder. ¿Cuál sería el próximo acto de este extraño personaje? ¿Cómo anunciaría la gloria del reino venidero? Todos estaban emocionados con la idea de que iban a estar presentes para contemplar la revelación de la fuerza y del poder del Dios de Israel.

\section*{4. Las bodas de Caná}
\par 
%\textsuperscript{(1528.4)}
\textsuperscript{137:4.1} Hacia el mediodía del miércoles, cerca de mil convidados habían llegado a Caná, más de cuatro veces el número de invitados a la fiesta nupcial. Los judíos tenían la costumbre de celebrar los casamientos los miércoles, y las invitaciones habían sido enviadas con un mes de antelación. Durante la mañana y el principio de la tarde, aquello se parecía más a una recepción pública para Jesús que a una boda. Todo el mundo quería saludar a este galileo casi famoso, y él era sumamente cordial con todos, jóvenes y adultos, judíos y gentiles. Todos se regocijaron cuando Jesús accedió a encabezar la procesión nupcial preliminar.

\par 
%\textsuperscript{(1528.5)}
\textsuperscript{137:4.2} Jesús era ahora enteramente consciente de su existencia humana, de su preexistencia divina, y del estado de sus naturalezas humana y divina combinadas o fusionadas. Con un equilibrio perfecto podía jugar en todo momento su papel humano o asumir inmediatamente las prerrogativas de la personalidad de su naturaleza divina.

\par 
%\textsuperscript{(1528.6)}
\textsuperscript{137:4.3} A medida que pasaba el día, Jesús se fue haciendo cada vez más consciente de que la gente esperaba que efectuara algún prodigio; comprendió especialmente que su familia y sus seis discípulos-apóstoles esperaban que anunciara su futuro reino de una manera apropiada mediante alguna manifestación sorprendente y sobrenatural.

\par 
%\textsuperscript{(1529.1)}
\textsuperscript{137:4.4} Al principio de la tarde, María llamó a Santiago y juntos se atrevieron a acercarse a Jesús para preguntarle si estaría dispuesto a confiar en ellos hasta el punto de informarles en qué momento y lugar de las ceremonias de la boda había planeado manifestarse como un <<ser sobrenatural>>. En cuanto abordaron esta cuestión con Jesús, vieron que habían suscitado su indignación característica. Él se limitó a decir: <<Si me amáis, entonces disponeos a aguardar conmigo mientras espero la voluntad de mi Padre que está en los cielos>>. Pero la elocuencia de su reproche residía en la expresión de su rostro.

\par 
%\textsuperscript{(1529.2)}
\textsuperscript{137:4.5} El Jesús humano se sintió muy decepcionado por esta acción de su madre, y se quedó muy pensativo ante su propia reacción a la propuesta insinuante de ella de que se permitiera darse el gusto de alguna demostración exterior de su divinidad. Ésta era precisamente una de las cosas que había decidido no hacer cuando estuvo recientemente aislado en las colinas. María estuvo muy deprimida durante varias horas. Le dijo a Santiago: <<No puedo comprenderlo. ¿Qué significa todo esto? ¿Su extraña conducta nunca tendrá fin?>> Santiago y Judá trataron de consolar a su madre, mientras que Jesús se retiraba para estar a solas durante una hora. Pero volvió a la reunión, mostrándose una vez más alegre y desenfadado.

\par 
%\textsuperscript{(1529.3)}
\textsuperscript{137:4.6} El casamiento tuvo lugar en medio de un silencio expectante, pero toda la ceremonia finalizó y el huésped de honor no hizo un solo gesto, no pronunció una sola palabra. Entonces se empezó a cuchichear que el carpintero y constructor de barcas, proclamado por Juan como <<el Libertador>>, descubriría su juego durante las fiestas de la noche, quizás en la cena nupcial. Pero Jesús apartó eficazmente de la mente de sus seis discípulos-apóstoles toda esperanza de una demostración de este tipo, cuando los reunió un poco antes de la cena nupcial y les dijo muy seriamente: <<No creáis que he venido a este lugar para efectuar algún prodigio que satisfaga a los curiosos o que convenza a los que dudan. Estamos aquí más bien para esperar la voluntad de nuestro Padre que está en los cielos>>. Cuando María y los demás lo vieron deliberando con sus asociados, estuvieron plenamente persuadidos en su propia mente de que algo extraordinario estaba a punto de suceder. Y todos se sentaron para disfrutar en buena compañía de la cena nupcial y de la noche de fiesta.

\par 
%\textsuperscript{(1529.4)}
\textsuperscript{137:4.7} El padre del novio había suministrado vino en abundancia para todos los huéspedes invitados a la fiesta nupcial, pero ¿cómo iba a suponer que la boda de su hijo se iba a convertir en un acontecimiento tan íntimamente asociado con la esperada manifestación de Jesús como libertador mesiánico? Estaba encantado de tener el honor de contar entre sus huéspedes al célebre galileo, pero antes de que terminara la cena nupcial, los criados le trajeron la noticia desconcertante de que el vino se estaba acabando. Cuando la cena oficial hubo terminado y los invitados se paseaban por el jardín, la madre del novio le confió a María que la provisión de vino se había agotado. Y María le dijo en confianza: <<No se preocupe ---hablaré con mi hijo. Él nos ayudará>>. Y se atrevió a hablar así, a pesar de la reprimenda recibida pocas horas antes.

\par 
%\textsuperscript{(1529.5)}
\textsuperscript{137:4.8} Durante muchos años, María siempre se había dirigido a Jesús para que la ayudara en cada una de las crisis de su vida familiar en Nazaret, de manera que fue muy natural para ella pensar en él en este momento. Pero esta madre con aspiraciones tenía también otros motivos para acudir a su hijo mayor en esta ocasión. Jesús estaba solo en un rincón del jardín, y su madre se le acercó diciendo: <<Hijo mío, no tienen vino>>. Y Jesús contestó: <<Mi buena mujer, ¿en qué me concierne ese asunto?>> María dijo: <<Pero yo creo que ha llegado tu hora. ¿No puedes ayudarnos?>> Jesús replicó: <<Afirmo de nuevo que no he venido para actuar de esa manera. ¿Por qué me molestas otra vez con esos asuntos?>> Entonces, echándose a llorar, María le suplicó: <<Pero, hijo mío, les he prometido que nos ayudarías. ¿No querrías hacer algo por mí, por favor?>> Entonces dijo Jesús: <<Mujer, ¿quién te ha dicho que hagas ese tipo de promesas? Cuídate de no volverlo a hacer. En todas las cosas debemos servir la voluntad del Padre que está en los cielos>>\footnote{\textit{La petición de María}: Jn 2:3-4.}.

\par 
%\textsuperscript{(1530.1)}
\textsuperscript{137:4.9} María, la madre de Jesús, se sintió abatida; ¡estaba aturdida! Mientras permanecía allí inmóvil delante de él, con el rostro lleno de lágrimas, el corazón humano de Jesús se rindió de compasión por la mujer que lo había llevado en su seno. Se inclinó hacia ella, puso tiernamente la mano sobre su cabeza, y le dijo: <<Vamos, vamos, madre María, no te aflijas por mis palabras aparentemente duras. ¿No te he dicho muchas veces que he venido solamente para hacer la voluntad de mi Padre celestial? Con mucho gusto haría lo que me pides si formara parte de la voluntad del Padre..>>. Y Jesús se detuvo en seco, vacilando. María pareció percibir que algo estaba sucediendo. Dando un salto, arrojó sus brazos alrededor del cuello de Jesús, lo besó, y se precipitó hacia la sala de los criados, diciendo: <<Cualquier cosa que mi hijo os diga, hacedla>>\footnote{\textit{Instrucciones a los sirvientes}: Jn 2:5.}. Pero Jesús no dijo nada. Ahora se daba cuenta de que ya había dicho demasiado ---o más bien que había deseado demasiado con su pensamiento.

\par 
%\textsuperscript{(1530.2)}
\textsuperscript{137:4.10} María saltaba de alegría. No sabía cómo se produciría el vino, pero creía confiadamente de que por fin había persuadido a su hijo primogénito para que afirmara su autoridad, para que se atreviera a presentarse resueltamente, reclamara su posición y mostrara su poder mesiánico. A causa de la presencia y de la asociación de ciertos poderes y personalidades universales, que todos los allí presentes ignoraban por completo, ella no iba a ser defraudada. El vino que María deseaba y que Jesús, el Dios-hombre, anhelaba humanamente por simpatía, estaba en camino.

\par 
%\textsuperscript{(1530.3)}
\textsuperscript{137:4.11} Cerca de allí había seis grandes vasijas de piedra, llenas de agua, con unos ochenta litros cada una. Este agua estaba destinada a utilizarse posteriormente en las ceremonias finales de purificación de la celebración matrimonial. La agitación de los criados alrededor de estas enormes vasijas de piedra, bajo la activa dirección de su madre, atrajo la atención de Jesús. Al acercarse, observó que estaban sacando vino a cántaros llenos\footnote{\textit{Conversión del agua en vino}: Jn 2:6-8.}.

\par 
%\textsuperscript{(1530.4)}
\textsuperscript{137:4.12} Jesús se fue dando cuenta gradualmente de lo que había sucedido. De todas las personas presentes en la fiesta matrimonial de Caná, Jesús era el más sorprendido. Los otros habían esperado que efectuara un prodigio, pero eso era precisamente lo que se había propuesto no hacer. Entonces, el Hijo del Hombre recordó la advertencia que su Ajustador del Pensamiento Personalizado le había hecho en las colinas. Recordó cómo el Ajustador le había avisado que ningún poder o personalidad podía privarlo de su prerrogativa como creador de ser independiente del tiempo. En esta ocasión, los transformadores del poder, los intermedios y todas las demás personalidades que se requerían, estaban reunidos cerca del agua y de los otros elementos necesarios, y en presencia del deseo expresado por el Soberano Creador del Universo, no había manera de evitar la aparición instantánea del \textit{vino}. La producción de este incidente estaba asegurada de manera doble, pues el Ajustador Personalizado había notificado que la ejecución del deseo del Hijo no infringía de ninguna manera la voluntad del Padre.

\par 
%\textsuperscript{(1530.5)}
\textsuperscript{137:4.13} Pero esto no fue un milagro en ningún sentido. Ninguna ley de la naturaleza fue modificada, abolida o ni siquiera trascendida. Lo único que se produjo fue la anulación del \textit{tiempo} en asociación con la reunión celestial de los elementos químicos indispensables para la elaboración del vino. En Caná, en esta ocasión, los agentes del Creador hicieron el vino exactamente tal como lo hacen mediante los procesos naturales ordinarios, \textit{salvo} que lo hicieron con independencia del tiempo y con la intervención de agentes sobrehumanos para reunir en el espacio los ingredientes químicos necesarios.

\par 
%\textsuperscript{(1531.1)}
\textsuperscript{137:4.14} Además, era evidente que la realización de este pretendido milagro no era contraria a la voluntad del Padre Paradisiaco, pues de otra manera no se habría producido, ya que Jesús se había sometido en todas las cosas a la voluntad del Padre.

\par 
%\textsuperscript{(1531.2)}
\textsuperscript{137:4.15} Cuando los criados sacaron este nuevo vino y lo llevaron al padrino de boda, el <<maestro de ceremonias>>, y éste lo hubo probado, llamó al novio, diciéndole: <<Es costumbre servir primero el buen vino, y cuando los convidados han bebido bien, se trae el fruto inferior de la vid; pero tú has guardado el mejor vino para el final de la fiesta>>\footnote{\textit{Comentarios del maestresala}: Jn 2:9-10.}.

\par 
%\textsuperscript{(1531.3)}
\textsuperscript{137:4.16} María y los discípulos de Jesús se regocijaron mucho con el supuesto milagro, pensando que Jesús lo había efectuado intencionalmente, pero Jesús se retiró a un rincón abrigado del jardín y se puso a meditar seriamente durante breves momentos. Finalmente concluyó que, dadas las circunstancias, el incidente estaba más allá de su control personal, y al no ser contrario a la voluntad de su Padre, era inevitable. Cuando regresó entre los invitados, éstos lo miraron con temor; todos creían que era el Mesías. Pero Jesús estaba dolorosamente perplejo; sabía que sólo creían en él a causa del extraño suceso que accidentalmente habían contemplado\footnote{\textit{Creencias de los discípulos}: Jn 2:11.}. Jesús se retiró de nuevo durante un rato a la azotea de la casa para meditar sobre todo aquello.

\par 
%\textsuperscript{(1531.4)}
\textsuperscript{137:4.17} Jesús comprendió entonces plenamente que debía mantenerse continuamente alerta para que su inclinación a la simpatía y a la compasión no fuera responsable de otros incidentes de este tipo. Sin embargo, muchos acontecimientos similares se produjeron antes de que el Hijo del Hombre se despidiera definitivamente de su vida mortal en la carne.

\section*{5. De regreso a Cafarnaúm}
\par 
%\textsuperscript{(1531.5)}
\textsuperscript{137:5.1} Aunque muchos de los invitados se quedaron durante toda la semana de las festividades nupciales, Jesús, con sus discípulos-apóstoles recién elegidos ---Santiago, Juan, Andrés, Pedro, Felipe y Natanael--- partió a la mañana siguiente muy temprano para Cafarnaúm\footnote{\textit{Partida del grupo a Cafarnaúm}: Jn 2:12.}, marchándose sin despedirse de nadie. La familia de Jesús y todos sus amigos de Caná estaban muy apenados por su partida tan repentina, y Judá, su hermano menor, salió en su búsqueda. Jesús y sus apóstoles fueron directamente a la casa de Zebedeo en Betsaida. Durante este viaje, Jesús habló con sus asociados recién elegidos de muchas cosas importantes para el reino venidero, y les advirtió especialmente que no mencionaran la transformación del agua en vino. También les aconsejó que evitaran, en su futuro trabajo, las ciudades de Séforis y Tiberiades.

\par 
%\textsuperscript{(1531.6)}
\textsuperscript{137:5.2} Aquella noche, después de la cena, en el hogar de Zebedeo y Salomé, Jesús celebró una de las conferencias más importantes de toda su carrera terrestre. En esta reunión sólo estuvieron presentes los seis apóstoles; Judá llegó cuando estaban a punto de separarse. Estos seis hombres escogidos habían viajado con Jesús desde Caná hasta Betsaida caminando, por así decirlo, sobre las nubes. Estaban llenos de expectación y emocionados con la idea de haber sido elegidos como asociados inmediatos del Hijo del Hombre. Pero cuando Jesús empezó a decirles claramente quién era él, cuál iba a ser su misión en la Tierra y cómo podría terminar quizás, se quedaron aturdidos. No podían comprender lo que les estaba diciendo. Se quedaron sin habla; el mismo Pedro estaba más anonadado de lo que se puede expresar. Sólo Andrés, el profundo pensador, se atrevió a contestar a las recomendaciones de Jesús. Cuando Jesús percibió que no comprendían su mensaje, cuando vio que sus ideas sobre el Mesías judío estaban tan completamente cristalizadas, los envió a descansar mientras él caminaba y conversaba con su hermano Judá. Antes de despedirse de Jesús, Judá le dijo con mucha emoción: <<Mi hermano-padre, nunca te he comprendido. No sé con certidumbre si eres lo que mi madre nos ha enseñado, y no comprendo plenamente el reino venidero, pero sí sé que eres un poderoso hombre de Dios. He oído la voz en el Jordán y creo en ti, sin importarme quien seas>>. Después de hablar así, Judá se marchó para su propio hogar en Magdala.

\par 
%\textsuperscript{(1532.1)}
\textsuperscript{137:5.3} Aquella noche Jesús no durmió. Envolviéndose en sus mantas, se sentó a la orilla del lago para reflexionar, y reflexionó hasta el alba del día siguiente. Durante las largas horas de esta noche de meditación, Jesús llegó a comprender claramente que nunca conseguiría que sus discípulos lo vieran bajo otra forma que no fuera la del Mesías largo tiempo esperado. Al final reconoció que no había manera de emprender su mensaje del reino excepto como cumplimiento de la predicción de Juan, y como aquel que los judíos estaban esperando. Después de todo, aunque él no era el Mesías de tipo davídico, sí era en verdad el cumplimiento de las declaraciones proféticas de los videntes del pasado con mayores inclinaciones espirituales. Nunca más negó por completo que fuera el Mesías. La tarea de desenredar finalmente esta complicada situación decidió dejarla a la manifestación de la voluntad del Padre.

\par 
%\textsuperscript{(1532.2)}
\textsuperscript{137:5.4} A la mañana siguiente, Jesús se reunió con sus amigos en el desayuno, pero formaban un grupo melancólico. Charló con ellos y al final de la comida los reunió a su alrededor, diciendo: <<Es voluntad de mi Padre que nos quedemos por aquí durante una temporada. Habéis oído decir a Juan que había venido a preparar el camino para el reino; por lo tanto, nos conviene esperar a que Juan termine su predicación. Cuando el precursor del Hijo del Hombre haya terminado su obra, empezaremos a proclamar la buena nueva del reino>>. Ordenó a sus apóstoles que volvieran a sus redes, mientras él se preparaba para ir con Zebedeo al astillero. Les prometió que los vería al día siguiente en la sinagoga, donde iba a hablar, y los citó para reunirse con ellos aquel sábado por la tarde.

\section*{6. Los acontecimientos de un sábado}
\par 
%\textsuperscript{(1532.3)}
\textsuperscript{137:6.1} La primera aparición pública de Jesús, después de su bautismo, tuvo lugar en la sinagoga de Cafarnaúm el sábado 2 de marzo del año 26. La sinagoga estaba atestada de gente. A la historia del bautismo en el Jordán se añadían ahora las recientes noticias de Caná sobre el agua y el vino. Jesús dio asientos de honor a sus seis apóstoles, y junto a ellos estaban sentados sus hermanos carnales Santiago y Judá. Su madre había regresado con Santiago a Cafarnaúm la noche anterior, y también se hallaba presente\footnote{\textit{La madre de Jesús está presente}: Jn 2:12.}, sentada en la sección de la sinagoga destinada a las mujeres. Todo el auditorio tenía los nervios de punta; esperaban contemplar alguna manifestación extraordinaria de poder sobrenatural que fuera un testimonio apropiado de la naturaleza y la autoridad de aquel que iba a hablarles aquel día. Pero estaban destinados a sufrir una decepción.

\par 
%\textsuperscript{(1532.4)}
\textsuperscript{137:6.2} Cuando Jesús se levantó, el jefe de la sinagoga le tendió el rollo de las Escrituras, y leyó en el profeta Isaías: <<Así dice el Señor: `El cielo es mi trono, y la Tierra mi escabel. ¿Dónde está la casa que habéis construido para mí? ¿Y dónde está el lugar de mi morada? Todas estas cosas las han hecho mis manos', dice el Señor. `Pero me fijaré en el hombre que es humilde y de espíritu contrito, y que tiembla con mi palabra'. Oíd la voz del Señor, vosotros que tembláis y tenéis miedo: `Vuestros hermanos os han odiado y desechado a causa de mi nombre'. Pero el Señor sea glorificado. Él aparecerá ante vosotros con alegría y todos los demás serán avergonzados. Una voz de la ciudad, una voz del templo, una voz del Señor dice: `Antes de estar de parto, dio a luz; antes de venirle los dolores, dio a luz un hijo varón'. ¿Quién ha oído una cosa semejante? ¿Producirá la tierra en un solo día? ¿O puede una nación nacer de un golpe? Pero así dice el Señor: `He aquí que extenderé la paz como un río, e incluso la gloria de los gentiles se parecerá a un torrente que fluye. Como alguien a quien su madre consuela, así os consolaré yo. Seréis consolados incluso en Jerusalén. Y cuando veáis estas cosas, se alegrará vuestro corazón'.>>\footnote{\textit{Versículos de Isaías leídos por Jesús}: Is 66:1-2; 66:5-8; 66:12-14.}

\par 
%\textsuperscript{(1533.1)}
\textsuperscript{137:6.3} Cuando terminó esta lectura, Jesús devolvió el rollo a su guardián. Antes de sentarse, dijo simplemente: <<Sed pacientes y veréis la gloria de Dios; así es como será para todos aquellos que aguardan conmigo y aprenden así a hacer la voluntad de mi Padre que está en los cielos>>. Y la gente se fue a sus casas, preguntándose por el significado de todo esto.

\par 
%\textsuperscript{(1533.2)}
\textsuperscript{137:6.4} Aquella tarde, Jesús y sus apóstoles, con Santiago y Judá, se subieron en una barca y se alejaron un poco de la orilla, donde echaron el ancla mientras Jesús les hablaba del reino venidero. Y comprendieron más cosas de las que habían entendido la noche del jueves.

\par 
%\textsuperscript{(1533.3)}
\textsuperscript{137:6.5} Jesús les mandó que se ocuparan de sus deberes regulares hasta que <<llegue la hora del reino>>. Y para animarlos, él mismo dio ejemplo volviendo a trabajar regularmente en el astillero. Al explicarles que deberían pasar tres horas cada noche estudiando y preparándose para su trabajo futuro, Jesús añadió: \guillemotleft Todos nos quedaremos por aquí hasta que el Padre me pida que os llame. Cada uno de vosotros debe regresar ahora a su trabajo de costumbre como si nada hubiera ocurrido. No habléis a nadie de mí y recordad que mi reino no ha de venir con estruendo y fascinación, sino más bien debe venir a través del gran cambio que mi Padre habrá efectuado en vuestro corazón y en el corazón de aquellos que serán llamados para unirse a vosotros en los consejos del reino\footnote{\textit{El reino no llegará con estruendo, sino en los corazones}: Lc 17:20-21.}. Ahora sois mis amigos\footnote{\textit{Sois mis amigos}: Jn 15:14-15.}; confío en vosotros y os amo; pronto os convertiréis en mis asociados personales. Sed pacientes, sed dulces. Obedeced siempre a la voluntad del Padre. Preparaos para la llamada del reino. Aunque experimentaréis una gran alegría al servicio de mi Padre, también debéis prepararos para las dificultades, porque os advierto que muchos sólo entrarán en el reino pasando por grandes tribulaciones\footnote{\textit{El reino de la tribulación y la alegría}: Jn 16:33; Hch 14:22; Ap 7:14.}. Para aquellos que han encontrado el reino, su alegría será completa, y serán llamados los bienaventurados de toda la Tierra. Pero no alimentéis falsas esperanzas; el mundo tropezará con mis palabras. Incluso vosotros, mis amigos, no percibís plenamente lo que estoy revelando a vuestras mentes confusas. No os engañéis; saldremos a trabajar para una generación que busca signos. Exigirán la realización de prodigios como prueba de que soy el enviado de mi Padre, y serán lentos en reconocer, en la revelación del \textit{amor} de mi Padre, las cartas credenciales de mi misión\guillemotright.

\par 
%\textsuperscript{(1533.4)}
\textsuperscript{137:6.6} Aquella noche, cuando volvieron a tierra y antes de separarse, Jesús oró de pie al borde del agua: <<Padre mío, te doy las gracias por estos pequeños que ya creen, a pesar de sus dudas. Por amor a ellos, me he apartado para hacer tu voluntad. Ojalá aprendan ahora a ser uno, como nosotros somos uno>>.

\section*{7. Cuatro meses de formación}
\par 
%\textsuperscript{(1533.5)}
\textsuperscript{137:7.1} Durante cuatro largos meses ---marzo, abril, mayo y junio--- continuó este tiempo de espera; Jesús mantuvo más de cien reuniones largas y serias, aunque alegres y animadas, con estos seis asociados y su propio hermano Santiago. Debido a enfermedades en su familia, Judá rara vez pudo asistir a estas clases. Santiago no perdió la fe en su hermano Jesús, pero durante estos meses de pausa y de inacción, María casi llegó a desesperar de su hijo. Su fe, que se había elevado a tales alturas en Caná, se hundió ahora hasta niveles muy bajos. Lo único que hacía era recurrir a su exclamación tantas veces repetida: <<No consigo comprenderlo. No consigo descifrar qué significa todo esto>>. Pero la mujer de Santiago contribuyó mucho a sostener el ánimo de María.

\par 
%\textsuperscript{(1534.1)}
\textsuperscript{137:7.2} Durante estos cuatro meses, estos siete creyentes, uno de ellos su propio hermano carnal, aprendieron a conocer a Jesús; estuvieron acostumbrándose a la idea de vivir con este Dios-hombre. Aunque lo llamaban Rabino\footnote{\textit{Le llamaban Rabbí}: Jn 1:38,49; Jn 3:2,26; Jn 6:25.}, estaban aprendiendo a no temerle. Jesús poseía esa gracia incomparable en su personalidad que le permitía vivir entre ellos de tal manera que no se sentían desalentados por su divinidad. Encontraban sumamente fácil ser <<amigos de Dios>>\footnote{\textit{Amigos de Dios}: Jn 15:14-15.}, Dios encarnado en la similitud de la carne mortal. Este compás de espera fue una dura prueba para todo el grupo de creyentes. Nada milagroso sucedió, absolutamente nada. Día tras día se ponían a hacer su trabajo ordinario, y noche tras noche se sentaban a los pies de Jesús. Se mantenían unidos gracias a su personalidad sin igual y a las atractivas palabras que les dirigía noche tras noche.

\par 
%\textsuperscript{(1534.2)}
\textsuperscript{137:7.3} Este período de espera y de enseñanza fue especialmente duro para Simón Pedro. Intentó repetidas veces persuadir a Jesús para que emprendiera la predicación del reino en Galilea mientras Juan continuaba predicando en Judea. Pero Jesús siempre respondía a Pedro: <<Ten paciencia, Simón. Haz progresos. No estaremos de ningún modo demasiado preparados cuando el Padre nos llame>>. Y Andrés tranquilizaba a Pedro de vez en cuando con sus consejos más moderados y filosóficos. Andrés estaba enormemente impresionado por la naturalidad humana de Jesús. Nunca se cansaba de contemplar cómo alguien que podía vivir tan cerca de Dios, podía ser tan amistoso y considerado con los hombres.

\par 
%\textsuperscript{(1534.3)}
\textsuperscript{137:7.4} A lo largo de todo este período, Jesús no habló en la sinagoga más que dos veces. Hacia el final de estas numerosas semanas de espera, los comentarios sobre su bautismo y el vino de Caná habían empezado a calmarse. Y Jesús tuvo cuidado de que no se produjeran más milagros aparentes durante este período. Pero aunque vivían de manera tan tranquila en Betsaida, las extrañas acciones de Jesús habían sido comunicadas a Herodes Antipas, quien a su vez envió a unos espías para averiguar lo que estaba pasando. Pero Herodes estaba más preocupado por la predicación de Juan. Decidió no molestar a Jesús, cuya obra proseguía tan sosegadamente en Cafarnaúm.

\par 
%\textsuperscript{(1534.4)}
\textsuperscript{137:7.5} Durante este tiempo de espera, Jesús se esforzó por enseñar a sus asociados la actitud que debían adoptar con respecto a los diversos grupos religiosos y partidos políticos de Palestina. Jesús siempre decía: <<Tratamos de ganarlos a todos, pero no \textit{pertenecemos} a ninguno de ellos>>.

\par 
%\textsuperscript{(1534.5)}
\textsuperscript{137:7.6} A los escribas y rabinos, en conjunto, se les llamaba fariseos. Ellos se denominaban a sí mismos los <<asociados>>. Eran, en muchos aspectos, el grupo progresista entre todos los judíos, pues habían adoptado muchas enseñanzas que no figuraban claramente en las escrituras hebreas, como la creencia en la resurrección de los muertos\footnote{\textit{La resurrección de los muertos}: Dn 12:1-2.}, una doctrina que sólo había sido mencionada por Daniel, un profeta reciente.

\par 
%\textsuperscript{(1534.6)}
\textsuperscript{137:7.7} Los saduceos estaban compuestos por el clero y ciertos judíos ricos. No daban tanta importancia a los detalles de la aplicación de la ley. Los fariseos y los saduceos eran en realidad partidos religiosos en lugar de sectas.

\par 
%\textsuperscript{(1534.7)}
\textsuperscript{137:7.8} Los esenios eran una verdadera secta religiosa, que había nacido durante la revuelta de los Macabeos. En algunos aspectos, sus normas eran más exigentes que las de los fariseos. Habían adoptado muchas creencias y prácticas persas, vivían en hermandad en monasterios, practicaban el celibato y lo poseían todo en común. Se especializaban en las enseñanzas sobre los ángeles.

\par 
%\textsuperscript{(1535.1)}
\textsuperscript{137:7.9} Los celotes eran un grupo de fervientes patriotas judíos. Sostenían que todos los métodos estaban justificados en la lucha para liberarse de la esclavitud del yugo romano.

\par 
%\textsuperscript{(1535.2)}
\textsuperscript{137:7.10} Los herodianos eran un partido puramente político que abogaba por la emancipación del gobierno directo de Roma mediante la restauración de la dinastía de Herodes.

\par 
%\textsuperscript{(1535.3)}
\textsuperscript{137:7.11} En el centro mismo de Palestina vivían los samaritanos, con quienes <<los judíos no tenían relaciones>>\footnote{\textit{No tenían trato con los samaritanos}: Jn 4:9.}, a pesar de que tenían muchos puntos de vista similares con las enseñanzas judías.

\par 
%\textsuperscript{(1535.4)}
\textsuperscript{137:7.12} Todos estos partidos y sectas, incluyendo la pequeña hermandad nazarea, creían que el Mesías llegaría algún día. Todos esperaban a un libertador nacional. Pero Jesús fue muy preciso al aclarar que él y sus discípulos no se aliarían con ninguna de estas escuelas de pensamiento o de práctica. El Hijo del Hombre no debía ser ni un nazareo ni un esenio.

\par 
%\textsuperscript{(1535.5)}
\textsuperscript{137:7.13} Aunque más adelante Jesús ordenó a los apóstoles que salieran, como había hecho Juan, a predicar el evangelio e instruir a los creyentes, hizo hincapié en la proclamación de la <<buena nueva del reino de los cielos>>\footnote{\textit{Predicar buenas noticias}: Mt 3:2; Mc 1:14; Lc 8:1; Jn 3:3,5.}. Inculcó incansablemente a sus asociados que debían <<mostrar amor, compasión y simpatía>>. Desde el principio enseñó a sus seguidores que el reino de los cielos era una experiencia espiritual que tenía que ver con la entronización de Dios en el corazón de los hombres.

\par 
%\textsuperscript{(1535.6)}
\textsuperscript{137:7.14} Mientras que Jesús y los siete se demoraban así antes de lanzarse a su predicación pública activa, pasaban dos noches por semana en la sinagoga estudiando las escrituras hebreas. Años más tarde, después de intensos períodos de trabajo público, los apóstoles recordarían estos cuatro meses como los más preciosos y provechosos de toda su asociación con el Maestro. Jesús enseñó a estos hombres todo lo que podían asimilar. No cometió el error de enseñarles con exceso. No los precipitó en la confusión presentándoles una verdad que sobrepasara demasiado su capacidad de comprensión.

\section*{8. El sermón sobre el reino}
\par 
%\textsuperscript{(1535.7)}
\textsuperscript{137:8.1} El sábado 22 de junio, poco antes de partir para su primera gira de predicación, y unos diez días después del arresto de Juan, Jesús ocupó el púlpito de la sinagoga por segunda vez desde que trajo a sus apóstoles a Cafarnaúm.

\par 
%\textsuperscript{(1535.8)}
\textsuperscript{137:8.2} Unos días antes de predicar este sermón sobre <<el Reino>>\footnote{\textit{Evangelio del reino}: Mt 4:23; 9:35; 24:14; Mc 1:14-15.}, mientras Jesús trabajaba en el astillero, Pedro le trajo la noticia del arresto de Juan\footnote{\textit{Jesús oye que Juan está en prisión}: Mt 4:12,17; Mc 1:14-15.}. Jesús dejó sus herramientas una vez más, se quitó el delantal y le dijo a Pedro: <<La hora del Padre ha llegado. Preparémonos para proclamar el evangelio del reino>>.

\par 
%\textsuperscript{(1535.9)}
\textsuperscript{137:8.3} Este martes 18 de junio del año 26 fue el último día que Jesús trabajó en un banco de carpintería. Pedro se precipitó fuera del taller, y hacia media tarde había reunido a todos sus compañeros; los dejó en un bosquecillo cercano a la costa, y fue en busca de Jesús. Pero no pudo encontrarlo, porque el Maestro había ido a otro bosquecillo para orar. No lo vieron hasta una hora avanzada de aquella noche, cuando regresó a la casa de Zebedeo y pidió de comer. Al día siguiente, envió a su hermano Santiago para que solicitara el privilegio de hablar en la sinagoga el sábado siguiente. El jefe de la sinagoga se alegró mucho de que Jesús estuviera dispuesto de nuevo a dirigir los oficios.

\par 
%\textsuperscript{(1536.1)}
\textsuperscript{137:8.4} Antes de que Jesús predicara este memorable sermón sobre el reino de Dios, el primer esfuerzo con pretensiones de su carrera pública, leyó en las Escrituras los pasajes siguientes: <<Seréis para mí un reino de sacerdotes, un pueblo santo. Yahvé es nuestro juez, Yahvé es nuestro legislador, Yahvé es nuestro rey; él nos salvará. Yahvé es mi rey y mi Dios. Él es un gran rey sobre toda la Tierra. La misericordia está sobre Israel en este reino. Bendita sea la gloria del Señor, porque él es nuestro Rey>>\footnote{\textit{Un pueblo santo}: Ex 19:6. \textit{Yahvé en nuestro juez}: Is 33:22. \textit{Yahvé es nuestro rey y Dios}: Sal 84:3. \textit{Él es un gran rey}: Sal 47:2. \textit{La misericordia}: Sal 138:2. \textit{Bendita sea la gloria del Señor}: Ez 3:12. \textit{Porque él es nuestro Rey}: Sal 89:18.}.

\par 
%\textsuperscript{(1536.2)}
\textsuperscript{137:8.5} Cuando terminó de leer, Jesús dijo:

\par 
%\textsuperscript{(1536.3)}
\textsuperscript{137:8.6} <<He venido para proclamar el establecimiento del reino del Padre. Este reino incluirá a las almas adoradoras de los judíos y de los gentiles, de los ricos y de los pobres, de los hombres libres y de los esclavos, porque mi Padre no hace acepción de personas; su amor y su misericordia son para todos>>\footnote{\textit{Proclamar el establecimiento del reino del Padre}: Mt 3:2; 4:17,23; 5:3,10,19-20; 6:33; 7:21; 8:11; 9:35; 10:7; 11:11-12; 12:28; 13:11,14,31-52; 16:19; 18:1-4,23; 19:14,23-24; 20:1; 21:31,43; 22:2; 23:13; 24:14; 25:1,14; Mc 1:14-15; 4:11,26,30; 9:1,47; 10:14-15,23-25; 12:34; 14:25; 15:43; Lc 4:43; 6:20; 7:28; 8:1,10; 9:2,11,27; 9:60,62; 10:9-11; 11:20; 12:31-32; 13:18,20,28,29; 14:15; 16:16; 17:20-21; 18:16-17,24-25; 19:11; 21:31; 22:16,18; 23:51; Jn 3:3,5; Ro 14:17; 1 Co 4:20; 6:9-10. \textit{No hace acepción de personas}: 2 Cr 19:7; Job 34:19; Eclo 35:12; Hch 10:34; Ro 2:11; Gl 2:6; 3:28; Ef 6:9; Col 3:11. \textit{Su amor y su misericordia son para todos}: Ef 2:4. \textit{Reino universal}: 1 Co 12:33.}.

\par 
%\textsuperscript{(1536.4)}
\textsuperscript{137:8.7} <<El Padre que está en los cielos envía su espíritu para que habite en la mente de los hombres, y cuando yo haya terminado mi obra en la Tierra, el Espíritu de la Verdad será igualmente derramado sobre todo el género humano. El espíritu de mi Padre y el Espíritu de la Verdad os establecerán en el reino venidero de comprensión espiritual y de rectitud divina. Mi reino no es de este mundo. El Hijo del Hombre no conducirá los ejércitos a la batalla para establecer un trono de poder o un reino de gloria terrenal. Cuando llegue mi reino, conoceréis al Hijo del Hombre como el Príncipe de la Paz, como la revelación del Padre eterno. Los hijos de este mundo luchan por establecer y ampliar los reinos de este mundo, pero mis discípulos entrarán en el reino de los cielos por medio de sus decisiones morales y de sus victorias espirituales; y una vez que hayan entrado, encontrarán la alegría, la rectitud y la vida eterna>>\footnote{\textit{El espíritu de mi Padre y el Espíritu de la Verdad}: Jn 17:21-23. \textit{Mi reino no es de este mundo}: Jn 18:36. \textit{Espíritu de la Verdad}: Ez 11:19; 18:31; 36:26-27; Jl 2:28-29; Lc 24:49; Jn 7:39; 14:16-18,23,26; 15:4,26; 16:6-8,13-14; 17:21-23; Hch 1:5,8a; 2:1-4,16-18; 2:33; 2 Co 13:5; Gl 2:20; 4:6; Ef 1:13; 4:30; 1 Jn 4:12-15.}.

\par 
%\textsuperscript{(1536.5)}
\textsuperscript{137:8.8} <<Aquellos que intentan en primer lugar entrar en el reino, y empiezan así a esforzarse por conseguir una nobleza de carácter semejante a la de mi Padre, pronto poseerán todas las demás cosas que necesitan. Pero os lo digo con toda sinceridad: a menos que tratéis de entrar en el reino con la fe y la dependencia confiada de un niño pequeño, no seréis admitidos de ninguna manera>>\footnote{\textit{Buscar primero el reino}: Mt 6:33; Lc 12:31. \textit{La fe confiada de un niño}: Mt 18:2-4; 19:13-14; Mc 9:36-37; 10:13-15; Lc 9:47-48; 18:17.}.

\par 
%\textsuperscript{(1536.6)}
\textsuperscript{137:8.9} <<No os dejéis engañar por aquellos que vienen diciendo: el reino está aquí o el reino está allá, porque el reino de mi Padre no tiene nada que ver con las cosas visibles y materiales. Este reino ya se encuentra ahora entre vosotros, porque allí donde el espíritu de Dios enseña y dirige el alma del hombre, allí está en realidad el reino de los cielos. Y este reino de Dios es rectitud, paz y alegría en el Espíritu Santo>>\footnote{\textit{No os dejéis engañar}: Mt 24:23; Mc 13:21; Lc 17:23; Lc 21:8. \textit{Reino de Dios}: Ro 14:17.}.

\par 
%\textsuperscript{(1536.7)}
\textsuperscript{137:8.10} <<Juan os ha bautizado verdaderamente en señal de arrepentimiento y para la remisión de vuestros pecados, pero cuando entréis en el reino celestial, seréis bautizados con el Espíritu Santo>>\footnote{\textit{Bautismo de Juan en señal de arrepentimiento}: Mt 3:11; Lc 3:16; Hch 1:5; Hch 11:16.}.

\par 
%\textsuperscript{(1536.8)}
\textsuperscript{137:8.11} <<En el reino de mi Padre no habrá ni judíos ni gentiles, sino únicamente aquellos que buscan la perfección a través del servicio, porque declaro que aquel que quiera ser grande en el reino de mi Padre, deberá convertirse primero en el servidor de todos. Si estáis dispuestos a servir a vuestros semejantes, os sentaréis conmigo en mi reino, al igual que yo me sentaré dentro de poco con mi Padre en su reino por haber servido en la similitud de la criatura>>\footnote{\textit{No habrá judíos ni gentiles}: 2 Cr 19:7; Job 34:19; Eclo 35:12; Hch 10:34; Ro 2:9-11; 9:24; 10:12; Gl 2:6; 3:28; Ef 6:9; Col 3:11. \textit{El grande es el servidor de todos}: Mt 20:26-27; 23:11-12; Mc 9:35; 10:43-44; Lc 22:26.}.

\par 
%\textsuperscript{(1536.9)}
\textsuperscript{137:8.12} <<Este nuevo reino es igual a una semilla que crece en la tierra fértil de un campo. No alcanza rápidamente su plena fructificación. Hay un intervalo de tiempo entre el establecimiento del reino en el alma del hombre y el momento en que el reino madura hasta su plena fructificación de rectitud perpetua y de salvación eterna>>\footnote{\textit{El reino como una semilla}: Mt 13:8,23; Mc 4:8,20; Lc 8:8,15.}.

\par 
%\textsuperscript{(1536.10)}
\textsuperscript{137:8.13} <<Este reino que os proclamo no es un reinado de poder y de abundancia. El reino de los cielos no es un asunto de comida y de bebida, sino más bien una vida de rectitud progresiva y de alegría creciente en el servicio cada vez más perfecto de mi Padre que está en los cielos. Porque ¿no ha dicho el Padre refiriéndose a sus hijos del mundo: `es mi voluntad que sean finalmente perfectos, como yo soy perfecto'?>>\footnote{\textit{El reino es una vida de servicio}: Ro 14:17. \textit{Sed perfectos}: Gn 17:1; 1 Re 8:61; Lv 19:2; Dt 18:13; Mt 5:48; 2 Co 13:11; Stg 1:4; 1 P 1:16.}

\par 
%\textsuperscript{(1537.1)}
\textsuperscript{137:8.14} <<He venido a predicar la buena nueva del reino. No he venido a aumentar las cargas pesadas de los que quieran entrar en este reino. Proclamo un camino nuevo y mejor, y aquellos que sean capaces de entrar en el reino venidero disfrutarán del descanso divino. Todo lo que os cueste en cosas del mundo, cualquier precio que paguéis por entrar en el reino de los cielos, lo recibiréis multiplicado en alegría y en progreso espiritual en este mundo, y la vida eterna en la era por venir>>\footnote{\textit{Jesús predicó buenas nuevas}: Lc 8:1. \textit{El reino merece el coste}: Mt 19:29.}.

\par 
%\textsuperscript{(1537.2)}
\textsuperscript{137:8.15} <<La entrada en el reino del Padre no depende de los ejércitos en marcha, de los reinos derrocados de este mundo, ni de la ruptura del yugo de los cautivos. El reino de los cielos está cerca, y todos los que entren en él encontrarán una libertad abundante y una gozosa salvación>>\footnote{\textit{El reino está cerca}: Mt 4:17; 10:7; Mc 1:15; Lc 21:31.}.

\par 
%\textsuperscript{(1537.3)}
\textsuperscript{137:8.16} <<Este reino es un dominio perpetuo. Los que entren en el reino ascenderán hasta mi Padre; alcanzarán ciertamente la diestra de su gloria en el Paraíso. Todos los que entren en el reino de los cielos se convertirán en los hijos de Dios, y en la era venidera ascenderán hasta el Padre. No he venido a llamar a los supuestos justos, sino a los pecadores y a todos los que tienen hambre y sed de la rectitud de la perfección divina>>\footnote{\textit{El reino es un dominio perpetuo}: 2 P 1:11. \textit{Alcanzarán la diestra de su gloria}: Mt 25:33-34; Mc 10:37,40; 16:19. \textit{Jesús vino a llamar a los pecadores}: Mt 9:13; Mc 2:17; Lc 5:32.}.

\par 
%\textsuperscript{(1537.4)}
\textsuperscript{137:8.17} <<Juan ha venido a predicar el arrepentimiento para prepararos para el reino; ahora vengo yo para proclamar que la fe, el regalo de Dios, es el precio para entrar en el reino de los cielos. Con que sólo creáis que mi Padre os ama con un amor infinito, ya estáis en el reino de Dios>>\footnote{\textit{La fe es el precio para entrar}: Hab 2:4; Ro 1:17; Gl 3:11; Ef 2:8; 1 P 1:9.}.

\par 
%\textsuperscript{(1537.5)}
\textsuperscript{137:8.18} Cuando terminó de hablar así, Jesús se sentó. Todos los que le oyeron se quedaron asombrados con sus palabras. Sus discípulos se maravillaron. Pero la gente no estaba preparada para recibir la buena nueva de labios de este Dios-hombre. Aproximadamente un tercio de sus oyentes creyó en el mensaje, aunque no pudieran comprenderlo por completo; otro tercio aproximadamente se preparó en su fuero interno para rechazar este concepto puramente espiritual del reino esperado, mientras que el tercio restante no pudo captar su enseñanza, y muchos de éstos creyeron sinceramente que <<había perdido el juicio>>\footnote{\textit{Muchos creían que Jesús estaba loco}: Mc 3:21.}.


\chapter{Documento 138. La formación de los mensajeros del reino}
\par 
%\textsuperscript{(1538.1)}
\textsuperscript{138:0.1} DESPUÉS de predicar el sermón sobre <<el Reino>>, Jesús reunió a los seis apóstoles aquella tarde y empezó a exponerles sus planes para visitar las ciudades situadas alrededor y en las proximidades del Mar de Galilea. Sus hermanos Santiago y Judá estaban muy molestos porque no habían sido llamados para participar en esta conferencia. Hasta ese momento se habían considerado como pertenecientes al círculo interno de los asociados de Jesús. Pero Jesús había decidido no tener parientes cercanos entre los miembros de este cuerpo de directores apostólicos del reino. El hecho de no incluir a Santiago y a Judá entre los pocos elegidos, así como su aparente alejamiento de su madre desde la experiencia de Caná, fue el punto de partida de un abismo cada vez más profundo entre Jesús y su familia. Esta situación continuó durante todo su ministerio público ---los suyos llegaron casi a rechazarlo--- y estas diferencias no desaparecieron por completo hasta después de su muerte y resurrección. Su madre oscilaba constantemente entre actitudes de fe y esperanza fluctuantes, y emociones crecientes de desilusión, humillación y desesperación. Sólo Rut, la más joven, permaneció inquebrantablemente fiel a su hermano-padre.

\par 
%\textsuperscript{(1538.2)}
\textsuperscript{138:0.2} Hasta después de la resurrección, toda la familia de Jesús participó muy poco en su ministerio. Un profeta siempre recibe honores, excepto en su propia tierra, y siempre goza de una estima comprensiva, salvo en su propia familia\footnote{\textit{No hay profeta sin honra excepto en su casa}: Mt 13:57; Mc 6:4; Lc 4:24; Jn 4:44.}.

\section*{1. Las instrucciones finales}
\par 
%\textsuperscript{(1538.3)}
\textsuperscript{138:1.1} Al día siguiente, el domingo 23 de junio del año 26, Jesús comunicó a los seis sus instrucciones finales. Les ordenó que salieran de dos en dos para enseñar la buena nueva del reino. Les prohibió que bautizaran y les aconsejó que no predicaran públicamente. Continuó explicándoles que más adelante les permitiría predicar en público, pero que durante una temporada, y por muchas razones, deseaba que adquirieran una experiencia práctica en el trato personal con sus semejantes. Jesús se proponía que su primera gira fuera enteramente de \textit{trabajo personal}. Aunque esta declaración desilusionó un poco a los apóstoles, sin embargo percibieron, al menos en parte, la razón que tenía Jesús para empezar así la proclamación del reino, y se marcharon con buen ánimo y un entusiasmo confiado. Los envió por parejas: Santiago y Juan fueron a Jeresa, Andrés y Pedro a Cafarnaúm, mientras que Felipe y Natanael se dirigieron a Tariquea.

\par 
%\textsuperscript{(1538.4)}
\textsuperscript{138:1.2} Antes de que empezaran estas dos primeras semanas de servicio, Jesús les anunció que deseaba ordenar a doce apóstoles para que continuaran el trabajo del reino después de su partida, y autorizó a cada uno de ellos para que escogiera, entre sus primeros conversos, a un hombre destinado a formar parte del cuerpo apostólico en proyecto. Juan tomó la palabra para preguntar: <<Pero, Maestro, ¿esos seis hombres estarán entre nosotros y compartirán todas las cosas en igualdad con nosotros, que hemos estado contigo desde el Jordán y hemos escuchado todas tus enseñanzas de preparación para nuestro primer trabajo a favor del reino?>> Y Jesús replicó: <<Sí, Juan, los hombres que escojáis formarán uno solo con nosotros, y vosotros les enseñaréis todo lo relacionado con el reino, como yo os lo he enseñado>>. Después de decirles esto, Jesús los dejó.

\par 
%\textsuperscript{(1539.1)}
\textsuperscript{138:1.3} Los seis no se separaron para cumplir su misión hasta después de haber discutido largamente la orden de Jesús de que cada uno de ellos tenía que escoger a un nuevo apóstol. El dictamen de Andrés acabó por prevalecer, y se marcharon a sus tareas. Andrés dijo en esencia: <<El Maestro tiene razón; somos demasiado pocos para abarcar este trabajo. Se necesitan más instructores, y el Maestro nos ha demostrado una gran confianza puesto que nos ha encargado la elección de estos seis nuevos apóstoles>>. Aquella mañana, al separarse para cumplir con su trabajo, había un poquito de depresión oculta en el corazón de cada uno de ellos. Sabían que iban a echar de menos a Jesús, y además de su temor y de su timidez, ésta no era la manera en que habían imaginado que se inauguraría el reino de los cielos.

\par 
%\textsuperscript{(1539.2)}
\textsuperscript{138:1.4} Se había dispuesto que los seis trabajarían dos semanas, después de lo cual regresarían al hogar de Zebedeo para tener una conferencia. Mientras tanto, Jesús fue a Nazaret para charlar con José, Simón y otros miembros de su familia que vivían en las inmediaciones. Para conservar la confianza y el afecto de su familia, Jesús hizo todo lo que era humanamente posible y compatible con su dedicación a hacer la voluntad de su Padre. En esta cuestión cumplió plenamente con su deber, e incluso más.

\par 
%\textsuperscript{(1539.3)}
\textsuperscript{138:1.5} Mientras que los apóstoles realizaban esta misión, Jesús pensó mucho en Juan, que ahora estaba en la cárcel. Era una gran tentación utilizar sus poderes potenciales para liberarlo, pero una vez más se resignó a <<servir la voluntad del Padre>>.

\section*{2. La elección de los seis}
\par 
%\textsuperscript{(1539.4)}
\textsuperscript{138:2.1} Esta primera gira misionera de los seis fue todo un éxito. Todos descubrieron el gran valor del contacto directo y personal con los hombres. Volvieron a Jesús comprendiendo mucho mejor que, después de todo, la religión es pura y totalmente un asunto de \textit{experiencia personal}. Empezaron a sentir hasta qué punto la gente del pueblo tenía hambre de oír palabras de consuelo religioso y de aliento espiritual. Cuando se reunieron alrededor de Jesús, todos quisieron hablar a la vez, pero Andrés asumió el mando y a medida que los fue llamando uno a uno, presentaron su informe oficial al Maestro y propusieron sus nombramientos para los seis nuevos apóstoles.

\par 
%\textsuperscript{(1539.5)}
\textsuperscript{138:2.2} Después de que cada uno hubiera presentado al nuevo apóstol de su elección, Jesús pidió a todos los demás que votaran su nombramiento; y así, los seis nuevos apóstoles fueron debidamente aceptados, de manera unánime, por los seis primeros. Después, Jesús anunció que todos irían a visitar a estos candidatos para confirmarles el llamamiento al servicio.

\par 
%\textsuperscript{(1539.6)}
\textsuperscript{138:2.3} Los apóstoles recién elegidos eran\footnote{\textit{Nombres bíblicos de los apóstoles}: Mt 10:2-4; Mc 3:16-19; Lc 6:14-16; Hch 1:13.}:

\par 
%\textsuperscript{(1539.7)}
\textsuperscript{138:2.4} 1. \textit{Mateo Leví}, el recaudador de derechos de aduana de Cafarnaúm, que tenía su oficina exactamente al este de la ciudad, cerca de los límites de Batanea. Había sido elegido por Andrés.

\par 
%\textsuperscript{(1539.8)}
\textsuperscript{138:2.5} 2. \textit{Tomás Dídimo}, pescador de Tariquea y en otro tiempo carpintero y albañil en Gadara. Había sido elegido por Felipe.

\par 
%\textsuperscript{(1539.9)}
\textsuperscript{138:2.6} 3. \textit{Santiago Alfeo}, pescador y agricultor de Jeresa, había sido elegido por Santiago Zebedeo.

\par 
%\textsuperscript{(1539.10)}
\textsuperscript{138:2.7} 4. \textit{Judas Alfeo}, el hermano gemelo de Santiago Alfeo, y también pescador, había sido elegido por Juan Zebedeo.

\par 
%\textsuperscript{(1540.1)}
\textsuperscript{138:2.8} 5. \textit{Simón Celotes} era un alto funcionario de la organización patriótica de los celotes, un puesto que abandonó para unirse a los apóstoles de Jesús. Antes de unirse a los celotes, Simón había sido comerciante. Fue elegido por Pedro.

\par 
%\textsuperscript{(1540.2)}
\textsuperscript{138:2.9} 6. \textit{Judas Iscariote} era el hijo único de unos padres judíos ricos que vivían en Jericó. Se había apegado a Juan el Bautista, y sus padres saduceos lo habían repudiado. Estaba buscando trabajo por estas regiones cuando lo encontraron los apóstoles de Jesús. Natanael lo invitó a unirse a sus filas, especialmente a causa de su experiencia financiera. Judas Iscariote era el único judeo entre los doce apóstoles.

\par 
%\textsuperscript{(1540.3)}
\textsuperscript{138:2.10} Jesús pasó un día entero con los seis, respondiendo a sus preguntas y escuchando los detalles de sus informes, pues tenían muchas experiencias interesantes y provechosas que contar. Ahora percibían la sabiduría del plan del Maestro de enviarlos a trabajar de una manera tranquila y personal antes de lanzarse a unos esfuerzos públicos más ambiciosos.

\section*{3. El llamamiento de Mateo y de Simón}
\par 
%\textsuperscript{(1540.4)}
\textsuperscript{138:3.1} Al día siguiente, Jesús y los seis fueron a ver a Mateo\footnote{\textit{Selección de Mateo}: Mt 9:9; Mc 2:14; Lc 5:27-28.}, el recaudador de aduanas. Mateo los estaba esperando; había saldado sus libros y se había preparado para traspasar los asuntos de su oficina a su hermano. Al acercarse a la oficina de peajes, Andrés se adelantó con Jesús, que miró de frente a Mateo y le dijo: <<Sígueme>>. Mateo se levantó y llevó a Jesús y a los apóstoles a su casa.

\par 
%\textsuperscript{(1540.5)}
\textsuperscript{138:3.2} Mateo le habló a Jesús del banquete que había organizado para aquella noche, diciendo que deseaba al menos ofrecer esta cena a su familia y a sus amigos, si Jesús estaba de acuerdo y accedía a ser el invitado de honor. Jesús asintió con la cabeza. Entonces Pedro cogió a Mateo aparte y le explicó que había invitado a un tal Simón a unirse a los apóstoles, y se aseguró el consentimiento de Mateo para que Simón también fuera convidado a esta fiesta.

\par 
%\textsuperscript{(1540.6)}
\textsuperscript{138:3.3} Después de almorzar a mediodía en la casa de Mateo, todos fueron con Pedro a visitar a Simón el Celote. Lo encontraron en su antigua oficina de negocios, que ahora dirigía su sobrino. Cuando Pedro condujo a Jesús hasta Simón, el Maestro saludó al ardiente patriota y sólo le dijo: <<Sígueme>>.

\par 
%\textsuperscript{(1540.7)}
\textsuperscript{138:3.4} Todos regresaron a la casa de Mateo, donde hablaron mucho sobre política y religión hasta la hora de la cena\footnote{\textit{El banquete de Mateo}: Mt 9:10; Mc 2:15; Lc 5:29.}. La familia Leví se dedicaba desde hacía mucho tiempo a los negocios y a la recaudación de impuestos; por ello, muchos de los convidados invitados por Mateo a este banquete habrían sido calificados de <<publicanos y pecadores>>\footnote{\textit{Publicanos y pecadores}: Mt 9:10-11; Mt 11:19; Mc 2:15-16; Lc 5:30; Lc 7:34; Lc 15:1.} por los fariseos.

\par 
%\textsuperscript{(1540.8)}
\textsuperscript{138:3.5} En aquellos tiempos, cuando un banquete-recepción de este tipo se ofrecía a un individuo sobresaliente, todas las personas interesadas tenían la costumbre de merodear por la sala del banquete para ver comer a los convidados y escuchar la conversación y los discursos de los invitados de honor. Por consiguiente, la mayoría de los fariseos de Cafarnaúm se encontraban presentes en esta ocasión para observar la conducta de Jesús en esta reunión social poco común.

\par 
%\textsuperscript{(1540.9)}
\textsuperscript{138:3.6} A medida que avanzaba la cena, la alegría de los convidados se elevó a alturas de fiesta; todos estaban pasando un rato tan espléndido que los espectadores fariseos empezaron a criticar a Jesús, en su fuero interno, por su participación en un acontecimiento tan frívolo y desenfadado. Más avanzada la noche, durante los discursos, uno de los fariseos más maliciosos llegó hasta el punto de criticar la conducta de Jesús delante de Pedro, diciendo: <<Cómo te atreves a enseñar que este hombre es justo, cuando come con publicanos y pecadores, prestando así su presencia a estas escenas de abandono a los placeres>>. Pedro le susurró esta crítica a Jesús antes de que éste pronunciara la bendición de despedida a todos los reunidos. Cuando Jesús empezó a hablar, dijo: <<Al venir aquí esta noche para acoger a Mateo y a Simón en nuestra hermandad, me complace presenciar vuestra alegría y vuestro regocijo social, pero deberíais regocijaros aún más porque muchos de vosotros entraréis en el reino del espíritu por venir, donde disfrutaréis más abundantemente de las buenas cosas del reino de los cielos. A los que estáis entre nosotros, criticándome en vuestro fuero interno porque he venido aquí para divertirme con estos amigos, permitidme decir que he venido para proclamar la alegría a los oprimidos de la sociedad y la libertad espiritual a los cautivos morales. ¿Necesito recordaros que los que están sanos no necesitan al médico, sino más bien los que están enfermos? He venido, no para llamar a los justos, sino a los pecadores>>\footnote{\textit{Los sanos no necesitan médico}: Mt 9:11-13; Mc 2:16-17; Lc 5:30-32.}.

\par 
%\textsuperscript{(1541.1)}
\textsuperscript{138:3.7} En verdad era un extraño espectáculo para la sociedad judía el ver a un hombre de carácter recto y de sentimientos nobles, mezclarse de manera libre y alegre con la gente corriente, e incluso con una muchedumbre irreligiosa y amiga de los placeres, compuesta de publicanos y de supuestos pecadores. Simón Celotes deseaba dar un discurso en esta reunión en casa de Mateo, pero Andrés, sabiendo que Jesús no quería que el reino venidero se confundiera con el movimiento de los celotes, lo persuadió para que se abstuviera de hacer comentarios en público.

\par 
%\textsuperscript{(1541.2)}
\textsuperscript{138:3.8} Jesús y los apóstoles pasaron la noche en casa de Mateo, y mientras la gente regresaba a sus hogares, sólo hablaban de una cosa: de la bondad y la amabilidad de Jesús.

\section*{4. El llamamiento de los gemelos}
\par 
%\textsuperscript{(1541.3)}
\textsuperscript{138:4.1} Al día siguiente, los nueve fueron en barca hasta Jeresa para efectuar el llamamiento formal de los dos apóstoles siguientes, Santiago y Judas, los hijos gemelos de Alfeo, los candidatos propuestos por Santiago y Juan Zebedeo. Los gemelos pescadores contaban con la venida de Jesús y sus apóstoles, y por ello los estaban esperando en la orilla. Santiago Zebedeo presentó al Maestro a los pescadores de Jeresa; Jesús los miró fijamente, asintió con la cabeza y dijo: <<Seguidme>>.

\par 
%\textsuperscript{(1541.4)}
\textsuperscript{138:4.2} Aquella tarde, que la pasaron juntos, Jesús los instruyó plenamente respecto a la asistencia a las reuniones festivas; concluyó sus comentarios diciendo: <<Todos los hombres son mis hermanos. Mi Padre celestial no desprecia a ninguna de las criaturas que hemos hecho. El reino de los cielos está abierto a todos los hombres y a todas las mujeres. Nadie puede cerrar la puerta de la misericordia en la cara de un alma hambrienta que está intentando entrar. Nos sentaremos a comer con todos los que deseen oír hablar del reino. Cuando nuestro Padre celestial contempla a los hombres desde arriba, todos son iguales. Así pues, no os neguéis a partir el pan con un fariseo o un pecador, con un saduceo o un publicano, con un romano o un judío, con un rico o un pobre, con un hombre libre o un esclavo. La puerta del reino está abierta de par en par para todos los que deseen conocer la verdad y encontrar a Dios>>\footnote{\textit{Igualdad de todos los que buscan a Dios}: 2 Cr 19:7; Job 34:19; Eclo 35:12; Hch 10:34; Ro 2:11; Gl 2:6; 3:28; Ef 6:9; Col 3:11.}.

\par 
%\textsuperscript{(1541.5)}
\textsuperscript{138:4.3} Aquella noche, en una simple cena en la casa de Alfeo, los hermanos gemelos fueron recibidos en la familia apostólica. Más avanzada la noche, Jesús dio a sus apóstoles su primera lección sobre el origen, la naturaleza y el destino de los espíritus impuros, pero no pudieron comprender el sentido de lo que les decía. Les resultaba muy fácil amar y admirar a Jesús, pero muy difícil comprender muchas de sus enseñanzas.

\par 
%\textsuperscript{(1542.1)}
\textsuperscript{138:4.4} Después de una noche de descanso, todo el grupo, ahora compuesto de once miembros, fue en barca hasta Tariquea.

\section*{5. El llamamiento de Tomás y de Judas}
\par 
%\textsuperscript{(1542.2)}
\textsuperscript{138:5.1} Tomás el pescador y Judas el errante se encontraron con Jesús y los apóstoles en el desembarcadero de las barcas de pesca de Tariquea, y Tomás condujo al grupo hasta su casa cercana. Felipe presentó entonces a Tomás como su candidato para el apostolado y Natanael presentó a Judas Iscariote, el judeo, para un honor similar. Jesús miró a Tomás y le dijo: <<Tomás, te falta fe; sin embargo, te recibo. Sígueme>>. A Judas Iscariote, el Maestro le dijo: <<Judas, todos somos de la misma carne, y al recibirte entre nosotros, ruego porque seas siempre leal con tus hermanos galileos. Sígueme>>.

\par 
%\textsuperscript{(1542.3)}
\textsuperscript{138:5.2} Una vez que hubieron descansado, Jesús se llevó a los doce durante un rato a un lugar apartado, para orar con ellos y para instruirlos sobre la naturaleza y el trabajo del Espíritu Santo; pero de nuevo no lograron comprender plenamente el significado de las maravillosas verdades que el Maestro se esforzaba por enseñarles. Uno captaba un detalle y su vecino comprendía otro, pero ninguno conseguía abarcar el conjunto de su enseñanza. Siempre cometían el error de intentar adaptar el nuevo evangelio de Jesús a sus viejas formas de creencia religiosa. No podían captar la idea de que Jesús había venido para proclamar un nuevo evangelio de salvación y para establecer una nueva manera de encontrar a Dios; no percibían que él \textit{era} una nueva revelación del Padre celestial.

\par 
%\textsuperscript{(1542.4)}
\textsuperscript{138:5.3} Al día siguiente, Jesús dejó completamente solos a sus doce apóstoles; quería que se conocieran y deseaba que estuvieran a solas para que comentaran lo que les había enseñado. El Maestro regresó para la cena, y durante la sobremesa les habló del ministerio de los serafines, y algunos de los apóstoles comprendieron su enseñanza. Descansaron esa noche y al día siguiente partieron en barca para Cafarnaúm.

\par 
%\textsuperscript{(1542.5)}
\textsuperscript{138:5.4} Zebedeo y Salomé se habían ido a vivir con su hijo David, para que su amplia casa pudiera estar a la disposición de Jesús y de sus doce apóstoles. Jesús pasó aquí un sábado tranquilo con sus mensajeros escogidos; les describió cuidadosamente los planes para proclamar el reino y les explicó plenamente la importancia de evitar todo conflicto con las autoridades civiles, diciendo: <<Si es necesario censurar a los gobernantes civiles, dejadme a mí esa tarea. Procurad no hacer acusaciones contra el César o sus servidores>>. Fue esta misma noche cuando Judas Iscariote llevó a Jesús aparte para preguntarle por qué no se hacía nada para sacar a Juan de la cárcel. Y Judas no se quedó totalmente satisfecho con la actitud de Jesús.

\section*{6. La semana de formación intensiva}
\par 
%\textsuperscript{(1542.6)}
\textsuperscript{138:6.1} La semana siguiente fue consagrada a un programa de intensa formación. Cada día, los seis nuevos apóstoles se ponían en manos de quienes los habían propuesto respectivamente para efectuar un repaso completo de todo lo que habían aprendido y experimentado como preparación para el trabajo del reino. Los primeros apóstoles analizaban cuidadosamente, en beneficio de los seis más nuevos, las enseñanzas dadas por Jesús hasta ese momento. Por la noche, todos se reunían en el jardín de Zebedeo para recibir la instrucción de Jesús.

\par 
%\textsuperscript{(1542.7)}
\textsuperscript{138:6.2} Fue en esta época cuando Jesús estableció un día de fiesta a mitad de la semana para descansar y divertirse. Y continuaron con este programa de relajarse un día por semana durante el resto de la vida material del Maestro. Por regla general, el miércoles nunca realizaban sus actividades regulares. En este día de fiesta semanal, Jesús tenía la costumbre de dejarlos solos, diciendo: <<Hijos míos, coged un día de asueto. Descansad de las arduas tareas del reino y disfrutad del alivio que procura el volver a vuestras antiguas vocaciones o el descubrir nuevos tipos de actividades recreativas>>. Durante este período de su vida terrestre, Jesús no necesitaba realmente este día de descanso, pero se amoldó a este plan porque sabía que era mejor para sus asociados humanos. Jesús era el instructor ---el Maestro; sus compañeros eran sus alumnos--- sus discípulos.

\par 
%\textsuperscript{(1543.1)}
\textsuperscript{138:6.3} Jesús se esforzó por aclarar a sus apóstoles la diferencia entre sus enseñanzas y su \textit{vida entre ellos}, y las enseñanzas que podrían surgir posteriormente \textit{acerca de} él. Jesús les dijo: <<Mi reino y el evangelio relacionado con él serán lo esencial de vuestro mensaje. No os desviéis del tema predicando \textit{sobre} mí y \textit{sobre} mis enseñanzas. Proclamad el evangelio del reino y describid mi revelación del Padre celestial, pero no os extraviéis por las sendas descarriadas de crear leyendas y de construir un culto relacionados con creencias y enseñanzas \textit{acerca de} mis creencias y enseñanzas>>\footnote{\textit{El evangelio del reino}: Mt 4:23; Mt 9:35; Mt 24:14; Mc 1:14-15.}. Pero, de nuevo, no comprendieron por qué hablaba así, y ninguno se atrevió a preguntar por qué les enseñaba de esta manera.

\par 
%\textsuperscript{(1543.2)}
\textsuperscript{138:6.4} En estas primeras enseñanzas, Jesús trató de evitar en lo posible las controversias con sus apóstoles, salvo aquellas que implicaban conceptos erróneos sobre su Padre que está en el cielo. En todas estas cuestiones, nunca dudaba en corregir las creencias erróneas. Había \textit{una sola} motivación en la vida de Jesús en Urantia después de su bautismo, y era efectuar una revelación mejor y más verdadera de su Padre Paradisiaco; él era el pionero del camino nuevo y mejor hacia Dios\footnote{\textit{Un camino nuevo y mejor}: Jn 14:6; Heb 10:20.}, el camino de la fe y del amor. Su exhortación a los apóstoles era siempre: <<Buscad a los pecadores; encontrad a los abatidos y confortad a los que están llenos de preocupaciones>>.

\par 
%\textsuperscript{(1543.3)}
\textsuperscript{138:6.5} Jesús captaba perfectamente la situación. Poseía un poder ilimitado que podía haber sido utilizado para impulsar su misión, pero estaba plenamente satisfecho con unos medios y unas personalidades que la mayoría de la gente hubiera calificado de inadecuados y los habría estimado como insignificantes. Estaba embarcado en una misión con enormes posibilidades dramáticas, pero insistió en dedicarse a los asuntos de su Padre de la manera más discreta y menos espectacular; evitó cuidadosamente toda exhibición de poder. Ahora se proponía trabajar tranquilamente con sus doce apóstoles, al menos durante varios meses, en las proximidades del Mar de Galilea.

\section*{7. Una nueva desilusión}
\par 
%\textsuperscript{(1543.4)}
\textsuperscript{138:7.1} Jesús había proyectado una tranquila campaña misionera de cinco meses de trabajo personal. No había dicho a los apóstoles cuánto tiempo iba a durar; trabajaban de semana en semana. Al principio de este primer día de la semana, precisamente cuando estaba a punto de anunciar este plan a sus doce apóstoles, Simón Pedro, Santiago Zebedeo y Judas Iscariote vinieron para hablarle en privado. Llevando aparte a Jesús, Pedro se atrevió a decir: <<Maestro, venimos a petición de nuestros compañeros para preguntar si no es ya el momento adecuado para entrar en el reino. ¿Vas a proclamar el reino en Cafarnaúm o nos trasladaremos a Jerusalén? Y cuándo sabremos, cada uno de nosotros, los puestos que vamos a ocupar contigo en el establecimiento del reino..>>. Y Pedro hubiera continuado haciendo otras preguntas, pero Jesús levantó una mano amonestadora y lo interrumpió. Haciendo señas a los otros apóstoles, que se hallaban cerca, para que se unieran a ellos, Jesús les dijo: <<Hijos míos, ¡cuánto tiempo seré indulgente con vosotros! ¿No os he aclarado que mi reino no es de este mundo? Os he dicho muchas veces que no he venido para sentarme en el trono de David; entonces, ¿cómo es que me preguntáis cuál es el lugar que ocupará cada uno de vosotros en el reino del Padre? ¿No podéis percibir que os he llamado como embajadores de un reino espiritual? ¿No comprendéis que pronto, muy pronto, vais a representarme en el mundo y en la proclamación del reino, como yo represento ahora a mi Padre que está en los cielos? ¿Es posible que os haya elegido e instruido como mensajeros del reino, y que sin embargo no comprendáis la naturaleza y la trascendencia de este reino venidero de supremacía divina en el corazón de los hombres? Amigos míos, escuchadme una vez más. Desterrad de vuestra mente la idea de que mi reino es un gobierno de poder o un reinado de gloria. En verdad, todos los poderes en el cielo y en la Tierra pronto serán puestos entre mis manos, pero no es voluntad del Padre que utilicemos esta dotación divina para glorificarnos durante esta era. En otra era, os sentaréis verdaderamente conmigo en poder y en gloria, pero ahora es nuestro deber someternos a la voluntad del Padre y obedecer humildemente saliendo a ejecutar su mandato en la Tierra>>.

\par 
%\textsuperscript{(1544.1)}
\textsuperscript{138:7.2} Una vez más, sus compañeros se quedaron horrorizados, atónitos. Jesús los envió de dos en dos para orar, pidiéndoles que regresaran a verlo al mediodía. En esta mañana decisiva, cada uno de ellos trató de encontrar a Dios, y cada uno se esforzó por animar y fortalecer al otro; luego volvieron para ver a Jesús tal como éste les había ordenado.

\par 
%\textsuperscript{(1544.2)}
\textsuperscript{138:7.3} Jesús les contó entonces la venida de Juan, el bautismo en el Jordán, la fiesta nupcial de Caná, la reciente elección de los seis y la separación de sus propios hermanos carnales. Les advirtió que el enemigo del reino trataría también de separarlos. Después de esta conversación breve pero seria, todos los apóstoles se levantaron, bajo la dirección de Pedro, para declarar su devoción imperecedera a su Maestro y prometer su lealtad inconmovible al reino, según palabras de Tomás, <<a ese reino por venir, sea lo que sea, y aunque no lo comprenda por completo>>. Todos \textit{creían en Jesús} sinceramente, aunque no comprendieran plenamente su enseñanza.

\par 
%\textsuperscript{(1544.3)}
\textsuperscript{138:7.4} Jesús les preguntó entonces cuánto dinero tenían entre todos; también se interesó por las medidas qué habían tomado para mantener a sus familias. Cuando se vio que apenas tenían fondos suficientes para mantenerse durante dos semanas, Jesús dijo: <<No es la voluntad de mi Padre que empecemos a trabajar en estas condiciones. Nos quedaremos aquí dos semanas junto al mar para pescar o hacer cualquier cosa que encontremos; mientras tanto, bajo la dirección de Andrés, el primer apóstol elegido, os organizaréis de tal manera que podáis disponer de todo lo necesario para vuestro futuro trabajo, tanto en el ministerio personal actual como cuando os ordene posteriormente predicar el evangelio e instruir a los creyentes>>. Todos se alegraron mucho con estas palabras; ésta era la primera indicación clara y positiva que tenían de que Jesús proyectaba emprender en el futuro unos esfuerzos públicos más dinámicos y pretenciosos.

\par 
%\textsuperscript{(1544.4)}
\textsuperscript{138:7.5} Los apóstoles pasaron el resto del día perfeccionando su organización y preparando las barcas y las redes para salir a pescar al día siguiente, pues todos habían decidido que se dedicarían a la pesca; la mayoría de ellos habían sido pescadores, y el mismo Jesús era un barquero y un pescador experto. Muchas de las barcas que utilizaron en los pocos años siguientes habían sido construidas por Jesús con sus propias manos. Y eran unas barcas buenas y dignas de confianza.

\par 
%\textsuperscript{(1544.5)}
\textsuperscript{138:7.6} Jesús les encargó que se consagraran a la pesca durante dos semanas, añadiendo: <<Y luego partiréis para convertiros en pescadores de hombres>>\footnote{\textit{Los apóstoles como pescadores de hombres}: Mt 4:19; Mc 1:17; Lc 5:10b.}. Pescaron en tres grupos, y Jesús salía cada noche con un grupo diferente. ¡Cuánto disfrutaban todos con la compañía de Jesús! Era un buen pescador, un compañero alegre y un amigo inspirador; cuanto más trabajaban con él, más lo amaban. Mateo dijo un día: <<Cuanto más se comprende a alguna gente, menos se les admira; pero con este hombre, cuanto menos lo comprendo, más lo amo>>.

\par 
%\textsuperscript{(1545.1)}
\textsuperscript{138:7.7} Este plan de pescar dos semanas y de salir dos semanas a hacer un trabajo personal a favor del reino lo efectuaron durante más de cinco meses hasta el final de este año 26, hasta después de que cesaran las persecuciones especialmente dirigidas contra los discípulos de Juan tras el arresto de éste.

\section*{8. El primer trabajo de los doce}
\par 
%\textsuperscript{(1545.2)}
\textsuperscript{138:8.1} Después de vender las capturas de la pesca de dos semanas, Judas Iscariote, que había sido elegido como tesorero de los doce, dividió los fondos apostólicos en seis partes iguales, una vez deducidos los fondos para el cuidado de las familias que dependían de los apóstoles. Luego, hacia mediados de agosto del año 26, se marcharon de dos en dos a las campañas de trabajo asignadas por Andrés. Las dos primeras semanas Jesús salió con Andrés y Pedro, las dos segundas con Santiago y Juan, y así sucesivamente con las otras parejas en el orden en que habían sido escogidos. De esta manera pudo salir al menos una vez con cada pareja, antes de reunirlos para empezar su ministerio público.

\par 
%\textsuperscript{(1545.3)}
\textsuperscript{138:8.2} Jesús les enseñó a predicar el perdón de los pecados mediante la \textit{fe en Dios}, sin penitencias ni sacrificios, y que el Padre que está en los cielos ama a todos sus hijos con el mismo amor eterno. Ordenó a sus apóstoles que se abstuvieran de discutir sobre:

\par 
%\textsuperscript{(1545.4)}
\textsuperscript{138:8.3} 1. El trabajo y el encarcelamiento de Juan el Bautista.

\par 
%\textsuperscript{(1545.5)}
\textsuperscript{138:8.4} 2. La voz que se escuchó en su bautismo. Jesús dijo: <<Sólo aquellos que oyeron la voz pueden referirse a ella. Proclamad solamente las cosas que me habéis oído decir; no habléis por rumores>>\footnote{\textit{Decid lo que habéis oído; no rumores}: Jn 8:26; Hch 4:20.}.

\par 
%\textsuperscript{(1545.6)}
\textsuperscript{138:8.5} 3. La transformación del agua en vino, en Caná. Jesús les encomendó seriamente: <<No le contéis a nadie lo del agua y el vino>>\footnote{\textit{No habléis del agua convertida en vino}: Jn 2:1-11.}.

\par 
%\textsuperscript{(1545.7)}
\textsuperscript{138:8.6} Pasaron momentos maravillosos a lo largo de estos cinco o seis meses, durante los cuales trabajaron como pescadores cada dos semanas alternativas, ganando así el dinero suficiente como para mantenerse en campaña las dos semanas siguientes de trabajo misionero para el reino.

\par 
%\textsuperscript{(1545.8)}
\textsuperscript{138:8.7} La gente corriente se maravillaba con las enseñanzas y el ministerio de Jesús y sus apóstoles. Los rabinos habían enseñado durante mucho tiempo a los judíos que los ignorantes no podían ser ni piadosos ni justos. Pero los apóstoles de Jesús eran piadosos y justos, y sin embargo ignoraban alegremente una gran parte de la erudición de los rabinos y de la sabiduría del mundo.

\par 
%\textsuperscript{(1545.9)}
\textsuperscript{138:8.8} Jesús explicó claramente a sus apóstoles la diferencia entre el arrepentimiento\footnote{\textit{El arrepentimiento, huir de la ira venidera}: Mt 3:2,7; Lc 3:3,7.} mediante las supuestas buenas obras, como enseñaban los judíos, y el cambio mental por la fe ---el nuevo nacimiento\footnote{\textit{Cambio de la mente por la fe, el nuevo nacimiento}: Jn 3:3-8; Gl 2:16; 3:2; Stg 2:14.}--- que él exigía como precio de admisión en el reino. Enseñó a sus apóstoles que la \textit{fe} era el único requisito para entrar en el reino del Padre. Juan les había enseñado el <<arrepentimiento ---a huir de la ira venidera>>. Jesús enseñaba que <<la fe es la puerta abierta para entrar en el amor presente, perfecto y eterno de Dios>>\footnote{\textit{La fe es la puerta abierta}: Mt 17:20; Mt 21:11; Lc 17:6; Hch 14:27.}. Jesús no hablaba como un profeta, como alguien que viene a proclamar la palabra de Dios. Parecía hablar de sí mismo como alguien que tiene autoridad. Jesús trataba de desviar sus mentes de la búsqueda de milagros hacia el descubrimiento de una experiencia auténtica y personal en la satisfacción y la seguridad de que el espíritu de amor y de gracia salvadora de Dios residía en ellos.

\par 
%\textsuperscript{(1545.10)}
\textsuperscript{138:8.9} Los discípulos aprendieron muy pronto que el Maestro tenía un profundo respeto y una consideración compasiva por \textit{cada} ser humano con quien se encontraba, y estaban enormemente impresionados por esta consideración uniforme e invariable que concedía de manera permanente a toda clase de hombres, mujeres y niños. Se detenía a la mitad de un profundo discurso para salir a la carretera y decirle unas palabras de aliento a una mujer que pasaba cargada con el peso de su cuerpo y de su alma. Interrumpía una importante conferencia con sus apóstoles para fraternizar con un niño inoportuno. Nada parecía nunca tan importante para Jesús como el ser \textit{humano individual} que se encontraba por casualidad en su presencia inmediata. Era maestro e instructor, pero era aún más ---era también un amigo y un vecino, un compañero comprensivo.

\par 
%\textsuperscript{(1546.1)}
\textsuperscript{138:8.10} Aunque la enseñanza pública de Jesús consistía principalmente en parábolas y en discursos breves, instruía invariablemente a sus apóstoles mediante preguntas y respuestas. Durante sus discursos públicos posteriores, siempre se interrumpía para responder a las preguntas sinceras.

\par 
%\textsuperscript{(1546.2)}
\textsuperscript{138:8.11} Al principio los apóstoles se escandalizaron por la manera en que Jesús trataba a las mujeres, pero pronto se acostumbraron; les explicó muy claramente que, en el reino, había que conceder a las mujeres los mismos derechos que a los hombres.

\section*{9. Cinco meses de prueba}
\par 
%\textsuperscript{(1546.3)}
\textsuperscript{138:9.1} Este período un poco monótono en el que se alternaba la pesca con el trabajo personal resultó ser una experiencia agotadora para los doce apóstoles, pero soportaron la prueba. A pesar de todas sus quejas, dudas y descontentos pasajeros, permanecieron fieles a su promesa de devoción y de lealtad al Maestro. Su asociación personal con Jesús durante estos meses de prueba les hizo quererle tanto, que todos
(salvo Judas Iscariote) permanecieran leales y fieles a su persona incluso en las horas sombrías del juicio y la crucifixión. Unos hombres auténticos sencillamente no podían abandonar de verdad a un educador venerado que había vivido tan cerca de ellos y que tanto se había consagrado a ellos como lo hizo Jesús. Durante las horas sombrías de la muerte del Maestro, toda razón, todo juicio y toda lógica se anularon en el corazón de estos apóstoles, para dar paso a una sola emoción humana extraordinaria ---el sentimiento supremo de amistad y de fidelidad. Estos cinco meses de trabajo con Jesús indujeron a estos apóstoles, a cada uno de ellos, a considerarlo como el mejor \textit{amigo} que tenían en el mundo. Fue este sentimiento humano, y no sus enseñanzas grandiosas o sus actos maravillosos, lo que los mantuvo unidos hasta después de la resurrección y de la reanudación de la proclamación del evangelio del reino.

\par 
%\textsuperscript{(1546.4)}
\textsuperscript{138:9.2} Estos meses de trabajo apacible no solamente fueron una gran prueba para los apóstoles, a la cual sobrevivieron, sino que esta temporada de inactividad pública fue una gran prueba para la familia de Jesús. Hacia la época en que Jesús estuvo preparado para empezar su obra pública, toda su familia (excepto Rut) prácticamente lo había abandonado. Sólo trataron de ponerse en contacto con él en pocas ocasiones posteriores, y fue para persuadirlo de que regresara con ellos al hogar, pues casi habían llegado a creer que estaba fuera de sí\footnote{\textit{La familia de Jesús creía que estaba fuera de sí}: Mc 3:21.}. Eran sencillamente incapaces de sondear su filosofía o de captar su enseñanza; todo esto era demasiado para los de su propia carne y sangre.

\par 
%\textsuperscript{(1546.5)}
\textsuperscript{138:9.3} Los apóstoles continuaron su trabajo personal en Cafarnaúm, Betsaida-Julias, Corazín, Gerasa, Hipos, Magdala, Caná, Belén de Galilea, Jotapata, Ramá, Safed, Giscala, Gadara y Abila. Además de estas ciudades, trabajaron en muchos pueblos así como en el campo. Hacia el final de este período, los doce habían elaborado unos planes bastante satisfactorios para cuidar de sus familias respectivas. La mayoría de los apóstoles estaban casados, y algunos tenían varios hijos, pero habían tomado tales medidas para el sostén de sus hogares que, con un poco de ayuda de los fondos apostólicos, podían consagrar todas sus energías a la obra del Maestro sin tener que preocuparse por el bienestar financiero de sus familias.

\section*{10. La organización de los doce}
\par 
%\textsuperscript{(1547.1)}
\textsuperscript{138:10.1} Los apóstoles se organizaron muy pronto de la manera siguiente:

\par 
%\textsuperscript{(1547.2)}
\textsuperscript{138:10.2} 1. Andrés, el primer apóstol elegido, fue nombrado presidente y director general de los doce.

\par 
%\textsuperscript{(1547.3)}
\textsuperscript{138:10.3} 2. Pedro, Santiago y Juan fueron nombrados compañeros personales de Jesús. Tenían que atenderlo día y noche, cuidar de sus necesidades materiales y diversas, y acompañarlo en las vigilias nocturnas de oración y de comunión misteriosa con el Padre celestial.

\par 
%\textsuperscript{(1547.4)}
\textsuperscript{138:10.4} 3. A Felipe lo hicieron administrador del grupo. Tenía el deber de proporcionar los alimentos y de vigilar que los visitantes, y a veces incluso las multitudes de oyentes, tuvieran algo que comer.

\par 
%\textsuperscript{(1547.5)}
\textsuperscript{138:10.5} 4. Natanael velaba por las necesidades de las familias de los doce. Recibía informes regulares sobre las demandas de la familia de cada apóstol, y cada semana enviaba fondos a quienes los necesitaban, después de pedirlos a Judas.

\par 
%\textsuperscript{(1547.6)}
\textsuperscript{138:10.6} 5. Mateo era el agente fiscal del cuerpo apostólico. Tenía el deber de vigilar que el presupuesto estuviera equilibrado y que la tesorería estuviera abastecida. Si no había fondos disponibles para el sostén mutuo, si no se recibían donaciones suficientes para mantener al grupo, Mateo tenía la autoridad de ordenar a los doce que regresaran a sus redes durante cierto tiempo. Pero nunca fue necesario hacerlo después de que empezaron su trabajo público; siempre tenía suficientes fondos en la tesorería para financiar sus actividades.

\par 
%\textsuperscript{(1547.7)}
\textsuperscript{138:10.7} 6. Tomás era el encargado del itinerario. A él le incumbía planear el alojamiento y, de una manera general, seleccionar los lugares para la enseñanza y la predicación, asegurando así un programa de viajes sin variaciones ni contratiempos.

\par 
%\textsuperscript{(1547.8)}
\textsuperscript{138:10.8} 7. Santiago y Judas, los hijos gemelos de Alfeo, fueron designados para dirigir a las multitudes. Tenían la tarea de delegar en un número suficiente de acomodadores asistentes que les permitieran mantener el orden entre las masas durante la predicación.

\par 
%\textsuperscript{(1547.9)}
\textsuperscript{138:10.9} 8. A Simón Celotes se le encargó de los entretenimientos y de la diversión. Preparaba los programas de los miércoles y también trataba de proporcionar cada día unas horas de distracción y diversión.

\par 
%\textsuperscript{(1547.10)}
\textsuperscript{138:10.10} 9. Judas Iscariote fue nombrado tesorero. Llevaba la bolsa\footnote{\textit{Judas llevaba la bolsa}: Jn 12:6; 13:29.}, pagaba todos los gastos y llevaba los libros de la contabilidad. Cada semana hacía un proyecto de presupuesto para Mateo y también presentaba sus informes semanales a Andrés. Judas desembolsaba los fondos con la autorización de Andrés.

\par 
%\textsuperscript{(1547.11)}
\textsuperscript{138:10.11} Los doce funcionaron de esta forma desde su organización primitiva hasta el momento en que tuvieron necesidad de reorganizarse debido a la deserción de Judas, el traidor. El Maestro y sus discípulos-apóstoles continuaron viviendo de esta manera sencilla hasta el domingo 12 de enero del año 27, día en que los reunió y los ordenó formalmente como embajadores del reino y predicadores de su buena nueva. Inmediatamente después de esto, se prepararon para salir hacia Jerusalén y Judea en su primera gira de predicación pública.


\chapter{Documento 139. Los doce apóstoles}
\par 
%\textsuperscript{(1548.1)}
\textsuperscript{139:0.1} UN testimonio elocuente del encanto y la rectitud de la vida terrestre de Jesús es el siguiente: aunque a menudo hizo pedazos las esperanzas de sus apóstoles\footnote{\textit{Los doce apóstoles}: Mt 10:2-4; Mc 3:16-19; Lc 6:14-16; Hch 1:13.} y destrozó cada una de sus ambiciones de elevación personal, sólo uno de ellos lo abandonó.

\par 
%\textsuperscript{(1548.2)}
\textsuperscript{139:0.2} Los apóstoles aprendieron de Jesús sobre el reino de los cielos, y Jesús aprendió mucho de ellos sobre el reino de los hombres, sobre cómo vive la naturaleza humana en Urantia y en los otros mundos evolutivos del tiempo y del espacio. Estos doce hombres representaban muchos tipos diferentes de temperamentos humanos, y la instrucción recibida no los había hecho \textit{semejantes}. Muchos de estos pescadores galileos tenían una fuerte proporción de sangre gentil a consecuencia de la conversión forzosa de la población no judía de Galilea cien años antes.

\par 
%\textsuperscript{(1548.3)}
\textsuperscript{139:0.3} No cometáis el error de considerar a los apóstoles como totalmente ignorantes e incultos. Todos, salvo los gemelos Alfeo, se habían graduado en las escuelas de la sinagoga, habiendo sido educados a fondo en las escrituras hebreas y en gran parte de los conocimientos corrientes de aquella época. Siete de ellos se habían graduado en las escuelas de la sinagoga de Cafarnaúm, y no existían mejores escuelas judías en toda Galilea.

\par 
%\textsuperscript{(1548.4)}
\textsuperscript{139:0.4} Cuando vuestros escritos califican a estos mensajeros del reino de <<ignorantes e iletrados>>\footnote{\textit{Hombres ignorantes e iletrados}: Hch 4:13.}, tenían la intención de transmitir la idea de que se trataba de laicos no instruidos en la ciencia de los rabinos, ni educados en los métodos de interpretación rabínica de las Escrituras. Carecían de la llamada educación superior. En los tiempos modernos se les consideraría seguramente como ineducados, e incluso en algunos círculos sociales como incultos. Una cosa es segura: no todos habían pasado por el mismo programa educativo rígido y estereotipado. Desde la adolescencia en adelante, habían disfrutado de experiencias diferentes en el aprendizaje de la vida.

\section*{1. Andrés, el primer escogido}
\par 
%\textsuperscript{(1548.5)}
\textsuperscript{139:1.1} Andrés\footnote{\textit{Andrés}: Mt 4:18; Mc 1:16; Jn 1:40.}, el presidente del cuerpo apostólico del reino, nació en Cafarnaúm. Era el hijo mayor de una familia de cinco: él mismo, su hermano Simón y tres hermanas. Su padre, ya fallecido, había sido socio de Zebedeo en un negocio de desecación de pescado en Betsaida, el puerto pesquero de Cafarnaúm. Cuando se convirtió en apóstol, Andrés era soltero pero vivía en casa de su hermano casado, Simón Pedro. Ambos eran pescadores y socios de Santiago y Juan, los hijos de Zebedeo.

\par 
%\textsuperscript{(1548.6)}
\textsuperscript{139:1.2} Cuando fue elegido como apóstol en el año 26, Andrés tenía 33 años, un año completo más que Jesús, y era el mayor de los apóstoles. Provenía de una excelente línea de antepasados y era el más capaz de los doce. A excepción de la oratoria, era igual a sus compañeros en casi todas las aptitudes imaginables. Jesús nunca le puso a Andrés un apodo, una designación fraternal. Pero al igual que los apóstoles pronto empezaron a llamar Maestro a Jesús, también designaron a Andrés con un nombre que equivalía a Jefe.

\par 
%\textsuperscript{(1549.1)}
\textsuperscript{139:1.3} Andrés era un buen organizador y un administrador aún mejor. Era uno de los cuatro apóstoles que formaban parte del círculo íntimo, pero al ser nombrado por Jesús como jefe del grupo apostólico, tenía que permanecer en su puesto con sus hermanos mientras que los otros tres disfrutaban de una comunión muy estrecha con el Maestro. Andrés siguió siendo el decano del cuerpo apostólico hasta el final.

\par 
%\textsuperscript{(1549.2)}
\textsuperscript{139:1.4} Aunque Andrés no fue nunca un predicador eficaz, era un trabajador personal eficiente; era el misionero pionero del reino, en el sentido de que al ser el primer apóstol elegido, llevó inmediatamente ante Jesús a su hermano Simón\footnote{\textit{Andrés lleva a Jesús a su hermano Simón}: Mt 4:18-19; Mc 1:16-18; Jn 1:40-42.}, el cual se convirtió posteriormente en uno de los mejores predicadores del reino. Andrés fue el defensor principal de la política de Jesús consistente en utilizar el programa del trabajo personal como medio de educar a los doce como mensajeros del reino.

\par 
%\textsuperscript{(1549.3)}
\textsuperscript{139:1.5} Si Jesús enseñaba a los apóstoles en privado o predicaba a las multitudes, Andrés conocía generalmente lo que estaba ocurriendo; era un ejecutivo inteligente y un administrador eficaz. Tomaba decisiones inmediatas en todos los asuntos que le comunicaban, salvo cuando estimaba que el problema sobrepasaba el ámbito de su autoridad, en cuyo caso lo consultaba directamente a Jesús.

\par 
%\textsuperscript{(1549.4)}
\textsuperscript{139:1.6} Andrés y Pedro tenían un carácter y un temperamento muy distintos, pero hay que indicar eternamente en su favor que se llevaban maravillosamente bien. Andrés nunca tuvo celos de la capacidad oratoria de Pedro. Pocas veces se verá a un hombre de más edad del tipo de Andrés ejercer una influencia tan profunda sobre un hermano más joven y talentoso. Andrés y Pedro nunca parecían estar celosos, en lo más mínimo, de las aptitudes o de los éxitos del otro. Avanzada la noche del día de Pentecostés, cuando dos mil almas fueron añadidas al reino\footnote{\textit{Dos miles almas añadidas al reino}: Hch 2:41.} a causa principalmente de la predicación enérgica e inspiradora de Pedro, Andrés le dijo a su hermano: <<Yo no podría haberlo hecho, pero estoy contento de tener un hermano que sí puede hacerlo>>. A lo cual Pedro respondió: <<Si tú no me hubieras traído hasta el Maestro, y sin tu perseverancia para \textit{mantenerme} a su lado, yo no hubiera estado aquí para hacerlo>>. Andrés y Pedro eran las excepciones a la regla, una prueba de que incluso los hermanos pueden convivir pacíficamente y trabajar juntos con eficacia.

\par 
%\textsuperscript{(1549.5)}
\textsuperscript{139:1.7} Después de Pentecostés, Pedro fue famoso, pero a Andrés el mayor nunca le irritó pasar el resto de su vida siendo presentado como <<el hermano de Simón Pedro>>\footnote{\textit{El hermano de Pedro}: Lc 6:14; Jn 1:40; Jn 6:8.}.

\par 
%\textsuperscript{(1549.6)}
\textsuperscript{139:1.8} De todos los apóstoles, Andrés era el que mejor juzgaba a los hombres. Sabía que en el corazón de Judas Iscariote se estaban fraguando problemas antes de que ninguno de los otros sospechara que algo iba mal en el tesorero; pero no le habló a nadie de sus temores. El gran servicio que Andrés hizo por el reino consistió en aconsejar a Pedro, Santiago y Juan sobre la elección de los primeros misioneros que se enviaron para proclamar el evangelio, y también en asesorar a estos primeros dirigentes sobre la organización de los asuntos administrativos del reino. Andrés tenía un don especial para descubrir los recursos ocultos y los talentos latentes de los jóvenes.

\par 
%\textsuperscript{(1549.7)}
\textsuperscript{139:1.9} Poco después de la ascensión de Jesús a las alturas, Andrés empezó a escribir un relato personal de muchos de los dichos y hechos de su difunto Maestro. Después de la muerte de Andrés se hicieron otras copias de este relato privado, que circularon libremente entre los primeros educadores de la iglesia cristiana. Estas notas provisionales de Andrés fueron posteriormente corregidas, enmendadas, alteradas y aumentadas hasta convertirse en una narración bastante consecutiva de la vida del Maestro en la Tierra. La última de estas pocas copias alteradas y enmendadas fue destruida por el fuego en Alejandría, unos cien años después de que el original hubiera sido escrito por el primer elegido de los doce apóstoles.

\par 
%\textsuperscript{(1550.1)}
\textsuperscript{139:1.10} Andrés era un hombre de perspicacia clara, de pensamiento lógico y de decisión firme; la gran fuerza de su carácter residía en su magnífica estabilidad. La desventaja de su temperamento era su falta de entusiasmo; muchas veces omitía animar a sus compañeros con alabanzas juiciosas. Esta reticencia a elogiar las habilidades meritorias de sus amigos provenía de su odio por la adulación y la hipocresía. Andrés era uno de esos hombres de empresas modestas, experto, de humor estable, que se ha formado por su propio esfuerzo y que consigue el éxito.

\par 
%\textsuperscript{(1550.2)}
\textsuperscript{139:1.11} Todos los apóstoles amaban a Jesús, pero es verdad que cada uno de los doce se sentía atraído por él debido a una característica determinada de su personalidad que ejercía una atracción especial sobre ese apóstol en particular. Andrés admiraba a Jesús a causa de su constante sinceridad, de su dignidad sin afectación. Una vez que los hombres conocían a Jesús, sentían la necesidad de compartirlo con sus amigos; deseaban realmente que todo el mundo lo conociera.

\par 
%\textsuperscript{(1550.3)}
\textsuperscript{139:1.12} Cuando las persecuciones posteriores dispersaron finalmente a los apóstoles fuera de Jerusalén, Andrés viajó por Armenia, Asia Menor y Macedonia; después de atraer a miles de almas al reino, fue finalmente detenido y crucificado en Patras, en Acaya. Este hombre robusto pasó dos días completos en la cruz antes de expirar, y durante estas horas trágicas continuó proclamando eficazmente la buena nueva de la salvación del reino de los cielos.

\section*{2. Simón Pedro}
\par 
%\textsuperscript{(1550.4)}
\textsuperscript{139:2.1} Simón tenía treinta años cuando se unió a los apóstoles\footnote{\textit{Simón se une a los apóstoles}: Mt 4:18-20; Mc 1:16-18; Lc 5:1-11; Jn 1:40-42.}. Estaba casado, tenía tres hijos y vivía en Betsaida, cerca de Cafarnaúm. Su hermano Andrés y la madre de su mujer vivían con él\footnote{\textit{La mujer de Simón y su suegra}: Mt 8:14; Mc 1:30; Lc 4:38.}. Tanto Pedro como Andrés estaban asociados en la pesca con los hijos de Zebedeo\footnote{\textit{Socios en la pesca}: Mc 1:16; Lc 5:10.}.

\par 
%\textsuperscript{(1550.5)}
\textsuperscript{139:2.2} El Maestro conocía a Simón desde hacía algún tiempo, antes de que Andrés lo presentara\footnote{\textit{Andrés presenta a Simón}: Jn 1:40-42.} como segundo apóstol\footnote{\textit{Segundo apóstol}: Mt 10:2; Jn 1:40-42.}. Cuando Jesús le dio a Simón el nombre de Pedro\footnote{\textit{Simón renombrado como Pedro}: Mc 3:13; Lc 6:14; Jn 1:42.}, lo hizo con una sonrisa; iba a ser una especie de apodo. Simón era bien conocido entre todos sus amigos como un tipo imprevisible e impulsivo. Es verdad que, más tarde, Jesús concedió una importancia nueva y significativa a este apodo dado a la ligera\footnote{\textit{Nuevo significado}: Mt 16:18.}.

\par 
%\textsuperscript{(1550.6)}
\textsuperscript{139:2.3} Simón Pedro era un hombre impulsivo, un optimista. Había crecido permitiéndose expresar libremente sus fuertes sentimientos; se metía constantemente en dificultades porque persistía en hablar sin reflexionar. Esta especie de atolondramiento también causaba problemas incesantes a todos sus amigos y asociados, y fue la causa de las numerosas reprimendas suaves que recibió de su Maestro. La única razón que impidió a Pedro meterse en más problemas por motivo de sus palabras irreflexivas fue que aprendió muy pronto a contarle a su hermano Andrés muchos de sus planes y proyectos, antes de aventurarse a proponerlos en público.

\par 
%\textsuperscript{(1550.7)}
\textsuperscript{139:2.4} Pedro era un orador desenvuelto, elocuente y teatral. Era también un conductor de hombres nato e inspirador, un pensador rápido pero no un razonador profundo. Hacía muchas preguntas, más que todos los apóstoles juntos, y aunque la mayoría de ellas eran buenas y pertinentes, muchas eran irreflexivas y tontas. Pedro no tenía una mente profunda, pero conocía su mente bastante bien. Por lo tanto, era un hombre de decisión rápida y de acción repentina. Mientras que los demás hablaban asombrados al ver a Jesús en la playa, Pedro saltó al agua y nadó hacia la tierra para reunirse con el Maestro\footnote{\textit{El impetuoso Pedro se tira al agua}: Jn 21:7.}.

\par 
%\textsuperscript{(1551.1)}
\textsuperscript{139:2.5} La característica que Pedro más admiraba de Jesús era su ternura suprema\footnote{\textit{Ternura, misericordia}: Mt 6:14; 18:21-22; Lc 17:4.}. Pedro nunca se cansaba de contemplar la indulgencia de Jesús. Nunca olvidó la lección de perdonar a los malhechores no solamente siete veces, sino setenta veces más siete. Reflexionó mucho sobre estas marcas del carácter misericordioso del Maestro durante los días sombríos y tristes que siguieron a su negación irreflexiva y no deliberada de Jesús en el patio del sumo sacerdote\footnote{\textit{Negación de Pedro}: Mt 26:69-75; Mc 14:66-72; Lc 22:55-62; Jn 18:17,25-27.}.

\par 
%\textsuperscript{(1551.2)}
\textsuperscript{139:2.6} Simón Pedro vacilaba de manera angustiosa\footnote{\textit{La vacilación de Pedro}: Jn 13:8-9.}; pasaba repentinamente de un extremo al otro. Primero se negó a que Jesús le lavara los pies, y luego, al escuchar la réplica del Maestro, le rogó que le lavara todo el cuerpo. Después de todo, Jesús sabía que las faltas de Pedro provenían de la cabeza y no del corazón. Pedro representaba una de las combinaciones más inexplicables de coraje y cobardía que se hayan visto nunca sobre la Tierra. La gran fuerza de su carácter era la lealtad, la amistad. Pedro amaba real y sinceramente a Jesús, y sin embargo, a pesar de esta sublime fuerza de devoción, era tan inestable y variable que permitió que una criada le importunara hasta el punto de renegar de su Señor y Maestro\footnote{\textit{Negación de Pedro}: Mt 26:69-75; Mc 14:66-72; Lc 22:55-62; Jn 18:17,25-27.}. Pedro podía soportar la persecución y cualquier otra forma de ataque directo, pero se avergonzaba y encogía ante el ridículo. Era un soldado valiente cuando lo atacaban de frente, pero un cobarde miedoso y vil cuando era sorprendido por la retaguardia.

\par 
%\textsuperscript{(1551.3)}
\textsuperscript{139:2.7} Pedro fue el primer apóstol de Jesús que se adelantó para defender la obra de Felipe entre los samaritanos y la de Pablo entre los gentiles\footnote{\textit{Defensa de Pedro de extender la predicación a los gentiles}: Hch 8:14-25; Hch 15:5-12.}; sin embargo más tarde, en Antioquía, dio marcha atrás\footnote{\textit{Cambio de decisión de Pedro}: Gl 2:11-14.} cuando se enfrentó con unos judaizantes que lo ridiculizaban, y se alejó temporalmente de los gentiles atrayendo así la audaz censura de Pablo sobre su cabeza.

\par 
%\textsuperscript{(1551.4)}
\textsuperscript{139:2.8} Fue el primero de los apóstoles que reconoció de todo corazón la humanidad y la divinidad combinadas de Jesús, y el primero ---salvo Judas--- que renegó de él. Pedro no tenía mucho de soñador, pero le disgustaba descender de las nubes del éxtasis y del entusiasmo de su inclinación teatral al mundo de la realidad simple y vulgar.

\par 
%\textsuperscript{(1551.5)}
\textsuperscript{139:2.9} Cuando seguía a Jesús, de manera literal y figurada, o bien encabezaba la procesión o se quedaba rezagado ---<<siguiéndola de lejos>>\footnote{\textit{Siguiendo de lejos}: Mt 26:58; Mc 14:54; Lc 22:54.}. Pero era el predicador más destacado de los doce; contribuyó más que cualquier otra persona, aparte de Pablo, a establecer el reino y a enviar a sus mensajeros, en una sola generación, a los cuatro puntos cardinales de la Tierra.

\par 
%\textsuperscript{(1551.6)}
\textsuperscript{139:2.10} Después de renegar atolondradamente del Maestro, se encontró a sí mismo, y bajo la dirección cariñosa y comprensiva de Andrés, fue de nuevo el primero en regresar a las redes de pesca mientras los apóstoles se quedaban para averiguar qué iba a suceder después de la crucifixión. Cuando estuvo completamente seguro de que Jesús lo había perdonado y supo que había sido reintegrado en el seno del Maestro, las llamas del reino ardieron tan vivamente en su alma que se convirtió en una gran luz salvadora para miles de personas que vivían en las tinieblas.

\par 
%\textsuperscript{(1551.7)}
\textsuperscript{139:2.11} Después de partir de Jerusalén y antes de que Pablo se convirtiera en el espíritu dirigente de las iglesias cristianas de los gentiles, Pedro viajó mucho, visitando todas las iglesias desde Babilonia hasta Corinto. Incluso visitó y atendió a muchas iglesias fundadas por Pablo. Aunque Pedro y Pablo diferían mucho en temperamento y educación, e incluso en teología, durante sus últimos años trabajaron juntos en armonía para la edificación de las iglesias.

\par 
%\textsuperscript{(1552.1)}
\textsuperscript{139:2.12} El estilo y la enseñanza de Pedro se manifiestan un poco en los sermones parcialmente transcritos por Lucas, y en el Evangelio de Marcos. Su estilo vigoroso aparece mejor en su carta conocida como la Primera Epístola de Pedro; al menos era así antes de que fuera alterada posteriormente por un discípulo de Pablo.

\par 
%\textsuperscript{(1552.2)}
\textsuperscript{139:2.13} Pero Pedro persistió en cometer el error de intentar convencer a los judíos de que, después de todo, Jesús era real y verdaderamente el Mesías judío. Hasta el día de su muerte, Simón Pedro continuó confundiendo en su mente los conceptos de: Jesús como Mesías judío, Cristo como redentor del mundo, y el Hijo del Hombre como revelación de Dios, el Padre amoroso de toda la humanidad.

\par 
%\textsuperscript{(1552.3)}
\textsuperscript{139:2.14} La esposa de Pedro era una mujer muy capaz. Durante años trabajó de manera aceptable como miembro del cuerpo evangélico femenino, y cuando Pedro fue expulsado de Jerusalén, lo acompañó en todos sus viajes a las iglesias y en todos sus recorridos misioneros. El día en que su ilustre marido dejó la vida, ella fue arrojada a las bestias salvajes en la arena de Roma.

\par 
%\textsuperscript{(1552.4)}
\textsuperscript{139:2.15} Así es como este hombre, Pedro, un amigo íntimo de Jesús, un miembro del círculo interno, partió de Jerusalén y proclamó la buena nueva del reino con poder y gloria hasta que la plenitud de su ministerio llegó a su fin. Consideró que le hacían un gran honor cuando sus captores le informaron que moriría como había muerto su Maestro ---en la cruz. Así pues, Simón Pedro fue crucificado en Roma.

\section*{3. Santiago Zebedeo}
\par 
%\textsuperscript{(1552.5)}
\textsuperscript{139:3.1} Santiago\footnote{\textit{Santiago Zebedeo}: Mt 4:21-22; Mc 1:19-20; Lc 5:10.}, el mayor de los dos hijos apóstoles de Zebedeo, a quienes Jesús apodó <<los hijos del trueno>>\footnote{\textit{Hijos del trueno}: Mc 3:17.}, tenía treinta años cuando se convirtió en apóstol. Estaba casado, tenía cuatro hijos y vivía cerca de sus padres en Betsaida, en las afueras de Cafarnaúm. Era pescador, y ejercía su profesión en compañía de su hermano menor Juan, y en asociación con Andrés y Simón. Santiago y su hermano Juan disfrutaban de la ventaja de haber conocido a Jesús mucho antes que todos los demás apóstoles.

\par 
%\textsuperscript{(1552.6)}
\textsuperscript{139:3.2} Este apóstol competente tenía un temperamento contradictorio; parecía poseer realmente dos naturalezas, ambas activadas por fuertes sentimientos. Era particularmente vehemente cuando se despertaba toda su indignación. Tenía un genio furibundo cuando se le provocaba suficientemente, y cuando pasaba la tormenta, siempre tenía la costumbre de justificar y excusar su enfado con el pretexto de que sólo era una manifestación de justa indignación. Aparte de estos arrebatos periódicos de ira, la personalidad de Santiago se parecía mucho a la de Andrés. No poseía la discreción ni la perspicacia de Andrés para penetrar en la naturaleza humana, pero hablaba en público mucho mejor que él. Después de Pedro, o quizás de Mateo, Santiago era el mejor orador público de los doce.

\par 
%\textsuperscript{(1552.7)}
\textsuperscript{139:3.3} Aunque Santiago no era en ningún sentido voluble, un día podía estar callado y taciturno, y al día siguiente muy conversador y narrador. Habitualmente hablaba abiertamente con Jesús, pero era, de los doce, aquel que permanecía en silencio durante días seguidos. Estos períodos de silencio inexplicable constituían su gran debilidad.

\par 
%\textsuperscript{(1552.8)}
\textsuperscript{139:3.4} El aspecto más destacado de la personalidad de Santiago era su aptitud para ver todas las facetas de un problema. Él fue, de los doce, el que estuvo más cerca de captar la importancia y la significación reales de la enseñanza de Jesús. Al principio también fue lento en comprender lo que decía el Maestro, pero antes de finalizar su preparación, había adquirido un concepto superior del mensaje de Jesús. Santiago era capaz de entender un amplio abanico de la naturaleza humana. Se llevaba bien con el talentoso Andrés, con el impetuoso Pedro y con su reservado hermano Juan.

\par 
%\textsuperscript{(1553.1)}
\textsuperscript{139:3.5} Aunque Santiago y Juan tenían sus problemas cuando intentaban trabajar juntos, era inspirador observar lo bien que se llevaban. No lo lograban tan bien como Andrés y Pedro, pero se llevaban mucho mejor de lo que se puede esperar habitualmente de dos hermanos, sobre todo de dos hermanos tan testarudos y decididos. Pero, por muy extraño que parezca, estos dos hijos de Zebedeo eran mucho más tolerantes el uno con el otro que con los desconocidos. Se tenían un gran afecto mutuo; siempre habían sido buenos compañeros de juego. Fueron estos <<hijos del trueno>> los que quisieron pedir que bajara fuego del cielo para aniquilar a los samaritanos\footnote{\textit{Caer fuego del cielo sobre los samaritanos}: Lc 9:54.} que se habían atrevido a ser irrespetuosos con su Maestro. Pero la muerte prematura de Santiago modificó enormemente el temperamento vehemente de su hermano menor Juan.

\par 
%\textsuperscript{(1553.2)}
\textsuperscript{139:3.6} La característica que Santiago más admiraba en Jesús era el afecto compasivo del Maestro. El interés comprensivo de Jesús por los pequeños y los grandes, los ricos y los pobres, le llamaba poderosamente la atención.

\par 
%\textsuperscript{(1553.3)}
\textsuperscript{139:3.7} Santiago Zebedeo era un pensador y un planificador bien equilibrado. Junto con Andrés, era uno de los miembros más sensatos del grupo apostólico. Era un individuo enérgico, pero nunca tenía prisa. Era un excelente contrapeso de Pedro.

\par 
%\textsuperscript{(1553.4)}
\textsuperscript{139:3.8} Era sencillo y poco dramático, un servidor cotidiano, un trabajador modesto, que no buscaba ninguna recompensa especial después de haber captado una parte del verdadero significado del reino. Incluso en la historia de la madre de Santiago y Juan, que pidió que se concediera un puesto a sus hijos a la derecha y a la izquierda de Jesús, no hay que olvidar que fue la madre quien efectuó esta petición\footnote{\textit{La petición de la madre}: Mt 20:20-21,23; Mc 10:35-37,40.}. Cuando declararon que estaban preparados para asumir esas responsabilidades, hay que reconocer que estaban enterados de los peligros que acompañaban a la supuesta revuelta del Maestro contra el poder de Roma, y que también estaban dispuestos a pagar el precio. Cuando Jesús les preguntó si estaban preparados para beber la copa\footnote{\textit{Preparado para beber la copa}: Mt 20:22-23; Mc 10:38-39.}, respondieron que sí. En lo que se refiere a Santiago, esto fue literalmente cierto ---bebió la copa con el Maestro, ya que fue el primer apóstol que sufrió el martirio\footnote{\textit{El martirio de Santiago}: Hch 12:1-2.}, pues Herodes Agripa pronto lo hizo ejecutar con la espada. Santiago fue así el primero de los doce que sacrificó su vida en el nuevo frente de batalla del reino. Herodes Agripa temía más a Santiago que a todos los demás apóstoles. Sí, es verdad que a menudo permanecía tranquilo y silencioso, pero era valiente y decidido cuando despertaban y desafiaban sus convicciones.

\par 
%\textsuperscript{(1553.5)}
\textsuperscript{139:3.9} Santiago vivió su vida de manera plena, y cuando llegó el final, se comportó con tanta gracia y entereza que incluso su acusador y delator, que asistió a su juicio y ejecución, se conmovió hasta tal punto que abandonó precipitadamente el espectáculo de la muerte de Santiago para unirse a los discípulos de Jesús.

\section*{4. Juan Zebedeo}
\par 
%\textsuperscript{(1553.6)}
\textsuperscript{139:4.1} Cuando Juan\footnote{\textit{Juan Zebedeo}: Mt 4:21-22; Mc 1:19-20; Lc 5:10-11.} se convirtió en apóstol, tenía veinticuatro años y era el más joven de los doce. Estaba soltero y vivía con sus padres en Betsaida; era pescador y trabajaba con su hermano Santiago en asociación con Andrés y Pedro. Antes y después de convertirse en apóstol, Juan ejerció como representante personal de Jesús en las relaciones con la familia del Maestro, y continuó llevando esta responsabilidad mientras vivió María, la madre de Jesús.

\par 
%\textsuperscript{(1553.7)}
\textsuperscript{139:4.2} Puesto que Juan era el más joven de los doce, y estaba tan estrechamente unido a Jesús por los asuntos de su familia, era muy querido por el Maestro, pero no se puede decir en verdad que era <<el discípulo que Jesús amaba>>\footnote{\textit{El discípulo que Jesús amaba}: Jn 13:23; Jn 19:26; Jn 20:2; Jn 21:7,20.}. Difícilmente se puede imaginar que una personalidad tan magnánima como la de Jesús fuera culpable de mostrar favoritismos, de amar a uno de sus apóstoles más que a los otros. El hecho de que Juan fue uno de los tres ayudantes personales de Jesús dio más credibilidad a esta idea errónea, sin mencionar que Juan, así como su hermano Santiago, había conocido a Jesús desde hacía más tiempo que los otros apóstoles.

\par 
%\textsuperscript{(1554.1)}
\textsuperscript{139:4.3} Pedro, Santiago y Juan fueron asignados como ayudantes personales de Jesús poco después de convertirse en apóstoles. Poco después de la elección de los doce, cuando Jesús nombró a Andrés como director del grupo, le dijo: <<Ahora deseo que designes a dos o tres de tus compañeros para que estén conmigo y permanezcan a mi lado, para que me conforten y atiendan mis necesidades diarias>>. Andrés pensó que, para este deber especial, lo mejor sería seleccionar a los tres primeros apóstoles escogidos después de él. A él mismo le hubiera gustado ofrecerse como voluntario para este bendito servicio, pero el Maestro ya le había dado su cometido; así que ordenó inmediatamente que Pedro, Santiago y Juan acompañaran a Jesús.

\par 
%\textsuperscript{(1554.2)}
\textsuperscript{139:4.4} Juan Zebedeo tenía un carácter con muchos rasgos agradables, pero uno que no era tan agradable era su vanidad desmedida, aunque habitualmente bien disimulada. Su prolongada asociación con Jesús produjo muchos y grandes cambios en su carácter. Su vanidad disminuyó considerablemente, pero cuando envejeció y se volvió un poco infantil, este amor propio volvió a aparecer en cierta medida, de tal manera que, cuando estaba ocupado guiando a Natán en la redacción del evangelio que ahora lleva su nombre, el anciano apóstol no dudó en referirse a menudo a sí mismo como el <<discípulo que Jesús amaba>>. En vista del hecho de que Juan casi llegó a ser, más que ningún otro mortal terrestre, el camarada de Jesús, de que era su representante personal elegido para tantos asuntos, no es de extrañar que llegara a considerarse como el <<discípulo que Jesús amaba>>, pues sabía perfectamente que era el discípulo en quien Jesús confiaba con mucha frecuencia.

\par 
%\textsuperscript{(1554.3)}
\textsuperscript{139:4.5} El rasgo más sobresaliente del carácter de Juan era su formalidad; era puntual y valiente, fiel y entregado. Su mayor debilidad era su vanidad característica. Era el miembro más joven de la familia de su padre y el más joven del grupo apostólico. Quizás estaba un poco mimado; tal vez lo habían complacido con exceso. Pero el Juan de los años posteriores fue un tipo de persona muy diferente al joven arbitrario y satisfecho de sí mismo que se incorporó a las filas de los apóstoles de Jesús cuando tenía veinticuatro años.

\par 
%\textsuperscript{(1554.4)}
\textsuperscript{139:4.6} Las características de Jesús que Juan apreciaba más eran el amor y el altruismo del Maestro; estos rasgos le impresionaron tanto que toda su vida posterior estuvo dominada por un sentimiento de amor y de devoción fraternal. Habló de amor y escribió sobre el amor. Este <<hijo del trueno>> se convirtió en el <<apóstol del amor>>\footnote{\textit{El apóstol del amor}: 1 Jn 3:11,14,23; 4:7-12,16-21; 5:1-2.}. En Éfeso, siendo ya un obispo anciano que no se podía mantener de pie en el púlpito para predicar, y tenían que llevarlo a la iglesia en una silla, cuando al final de los oficios le pedían que dijera algunas palabras para los creyentes, durante años se limitó a repetir: <<Hijos míos, amaos los unos a los otros>>\footnote{\textit{Hijos míos, amaos los unos a los otros}: 1 Jn 3:18.}.

\par 
%\textsuperscript{(1554.5)}
\textsuperscript{139:4.7} Juan era un hombre de pocas palabras, salvo cuando despertaban su mal genio. Pensaba mucho pero hablaba poco. Con la edad, su mal genio se volvió más suave, mejor controlado, pero nunca superó su aversión a hablar; nunca dominó por completo esta reticencia. Sin embargo, estaba dotado de una extraordinaria imaginación creativa.

\par 
%\textsuperscript{(1555.1)}
\textsuperscript{139:4.8} Juan tenía otra faceta que uno no esperaría encontrar en este tipo de hombre tranquilo e introspectivo. Era un poco fanático y extremadamente intolerante. En este aspecto se parecía mucho a Santiago ---los dos querían pedir que bajara fuego del cielo sobre las cabezas de los samaritanos irrespetuosos\footnote{\textit{Juan pide caer fuego sobre los samaritanos}: Lc 9:54.}. Cuando Juan se encontraba con algunos desconocidos que enseñaban en nombre de Jesús, se lo prohibía inmediatamente\footnote{\textit{Juan prohibió predicar a un extraño}: Mc 9:38; Lc 9:49.}. Pero no era el único de los doce que estaba infectado con esta clase de amor propio y de conciencia de superioridad.

\par 
%\textsuperscript{(1555.2)}
\textsuperscript{139:4.9} La vida de Juan sufrió una enorme influencia al ver a Jesús circulando sin hogar, pues sabía con cuánta fidelidad había asegurado el porvenir de su madre y de su familia. Juan también simpatizaba profundamente con Jesús al ver que su familia no le comprendía, siendo consciente de que se iban distanciando gradualmente de él. Toda esta situación, unida al hecho de que Jesús siempre sometía sus más pequeños deseos a la voluntad del Padre que está en el cielo y el observar su vida diaria de confianza implícita, hicieron en Juan una impresión tan profunda que produjo unos cambios marcados y permanentes en su carácter, unos cambios que se manifestaron a lo largo de toda su vida posterior.

\par 
%\textsuperscript{(1555.3)}
\textsuperscript{139:4.10} Juan tenía un valor frío y temerario que pocos de los otros apóstoles poseían. Fue el único apóstol que siguió a Jesús sin cesar la noche de su arresto y se atrevió a acompañar a su Maestro hasta las mismas puertas de la muerte. Estuvo presente y al alcance de la mano hasta la última hora terrestre de Jesús, realizando fielmente su misión de confianza respecto a la madre de Jesús\footnote{\textit{Juan cuidó de María}: Jn 19:26-27.}, y dispuesto a recibir las instrucciones adicionales que pudieran dársele durante los últimos momentos de la existencia mortal del Maestro. Una cosa es indudable: Juan era completamente digno de confianza. Se sentaba habitualmente a la derecha de Jesús cuando los doce estaban comiendo. Fue el primero de los doce que creyó real y plenamente en la resurrección, y el primero que reconoció al Maestro\footnote{\textit{El primero que reconoció a Jesús}: Jn 21:7.} cuando venía hacia ellos por la orilla del mar después de su resurrección.

\par 
%\textsuperscript{(1555.4)}
\textsuperscript{139:4.11} Este hijo de Zebedeo estuvo asociado muy estrechamente con Pedro en las primeras actividades del movimiento cristiano, convirtiéndose en uno de los pilares principales de la iglesia de Jerusalén. Fue el brazo derecho de Pedro el día de Pentecostés.

\par 
%\textsuperscript{(1555.5)}
\textsuperscript{139:4.12} Varios años después del martirio de Santiago, Juan se casó con la viuda de su hermano. Una nieta amorosa le cuidó durante los últimos veinte años de su vida.

\par 
%\textsuperscript{(1555.6)}
\textsuperscript{139:4.13} Juan estuvo varias veces en la cárcel y fue desterrado a la Isla de Patmos\footnote{\textit{Desterrado a Patmos}: Ap 1:9.} por un período de cuatro años, hasta que otro emperador subió al poder en Roma. Si Juan no hubiera tenido tanto tacto y sagacidad, indudablemente lo hubieran matado como a su hermano Santiago, que decía lo que pensaba con mayor claridad. A medida que pasaron los años, Juan, así como Santiago, el hermano del Señor, aprendieron a practicar una prudente conciliación cuando comparecían ante los magistrados civiles. Descubrieron que una <<respuesta dulce desvía el furor>>\footnote{\textit{La respuesta dulce desvía el furor}: Pr 15:1.}. Aprendieron también a presentar la iglesia como una <<hermandad espiritual dedicada al servicio social de la humanidad>>, en lugar de hacerlo como <<el reino de los cielos>>. Enseñaron el servicio amoroso en lugar del poder soberano ---con reino y rey.

\par 
%\textsuperscript{(1555.7)}
\textsuperscript{139:4.14} Durante su exilio temporal en Patmos, Juan escribió el libro del Apocalipsis, que actualmente poseéis de una manera muy abreviada y deformada. Este libro del Apocalipsis contiene los fragmentos sobrevivientes de una gran revelación, porque después de que Juan lo escribiera, se perdieron muchas partes del mismo y otras fueron eliminadas. Sólo se conserva de manera fragmentaria y adulterada.

\par 
%\textsuperscript{(1555.8)}
\textsuperscript{139:4.15} Juan viajó mucho, trabajó sin cesar y después de convertirse en obispo de las iglesias de Asia, se estableció en Éfeso. Cuando tenía noventa y nueve años, estando en Éfeso, dirigió a su asociado Natán en la redacción del llamado <<Evangelio según Juan>>. Juan Zebedeo se convirtió finalmente en el teólogo más sobresaliente de los doce apóstoles. Murió de muerte natural en Éfeso en el año 103, a los ciento un años de edad.

\section*{5. Felipe el Curioso}
\par 
%\textsuperscript{(1556.1)}
\textsuperscript{139:5.1} Felipe fue el quinto apóstol en ser elegido, habiendo sido llamado cuando Jesús y sus cuatro primeros apóstoles se dirigían desde el lugar de reunión de Juan en el Jordán hacia Caná de Galilea. Como vivía en Betsaida, Felipe había oído hablar de Jesús desde hacía algún tiempo, pero no se le había ocurrido que fuera realmente un gran hombre hasta aquel día, en el valle del Jordán, cuando Jesús le dijo: <<Sígueme>>. Felipe también se sintió un poco influido por el hecho de que Andrés, Pedro, Santiago y Juan habían aceptado a Jesús como el Libertador.

\par 
%\textsuperscript{(1556.2)}
\textsuperscript{139:5.2} Felipe\footnote{\textit{Felipe elegido como apóstol}: Jn 1:43-44.} tenía veintisiete años cuando se unió a los apóstoles; se había casado hacía poco tiempo, pero no tenía hijos en aquellos momentos. El apodo que los apóstoles le dieron significaba <<curiosidad>>. Felipe siempre quería que le mostraran. Nunca parecía ver muy lejos en un asunto cualquiera. No era necesariamente torpe, pero carecía de imaginación. Esta falta de imaginación era la gran debilidad de su carácter. Era un individuo corriente y vulgar.

\par 
%\textsuperscript{(1556.3)}
\textsuperscript{139:5.3} Cuando los apóstoles se organizaron para el servicio, a Felipe lo hicieron administrador; tenía el deber de velar para que no les faltaran las provisiones en ningún momento. Y fue un buen administrador. Su característica más destacada era su minuciosidad metódica; era matemático y sistemático al mismo tiempo.

\par 
%\textsuperscript{(1556.4)}
\textsuperscript{139:5.4} Felipe era el segundo de una familia de siete hermanos, tres niños y cuatro niñas. Después de la resurrección, bautizó a toda su familia para que entrara en el reino. Los miembros de la familia de Felipe eran pescadores. Su padre era un hombre muy capacitado, un profundo pensador, pero su madre procedía de una familia muy mediocre. Felipe no era un hombre de quien se podía esperar que hiciera grandes cosas, pero podía hacer pequeñas cosas a lo grande, hacerlas bien y de manera aceptable. Muy pocas veces, en cuatro años, dejó de tener provisiones al alcance de la mano para satisfacer las necesidades de todos. Incluso las numerosas situaciones de emergencia que surgían a causa de la vida que llevaban, rara vez lo cogieron desprevenido. El departamento de intendencia de la familia apostólica estaba administrado con inteligencia y eficacia.

\par 
%\textsuperscript{(1556.5)}
\textsuperscript{139:5.5} El punto fuerte de Felipe era su formalidad metódica; el punto débil de su modo de ser era su falta casi total de imaginación, la ausencia de aptitud para reunir dos y dos y obtener cuatro. Era matemático en lo abstracto, pero no constructivo en su imaginación. Carecía casi por completo de cierto tipo de imaginación. Era el típico hombre medio y corriente de la calle. Había una gran cantidad de hombres y mujeres de esta clase entre las multitudes que acudían para escuchar las enseñanzas y predicaciones de Jesús, y obtenían un gran consuelo al observar que uno semejante a ellos había sido elevado a una posición de honor en los consejos del Maestro; les animaba el hecho de que alguien como ellos ocupara ya un alto puesto en los asuntos del reino. Y Jesús aprendió mucho sobre cómo funcionan algunas mentes humanas mientras escuchaba con tanta paciencia las preguntas tontas de Felipe, y condescendía tantas veces con la petición de su administrador para que <<le mostraran>>.

\par 
%\textsuperscript{(1556.6)}
\textsuperscript{139:5.6} La cualidad principal que Felipe admiraba continuamente en Jesús era la generosidad inagotable del Maestro. Felipe nunca pudo encontrar en Jesús algo que fuera pequeño, mezquino o avaro, y veneraba esta dadivosidad permanente e inagotable.

\par 
%\textsuperscript{(1557.1)}
\textsuperscript{139:5.7} La personalidad de Felipe tenía poco de notable. A menudo le llamaban <<Felipe de Betsaida, la ciudad donde viven Andrés y Pedro>>\footnote{\textit{Felipe de Betsaida}: Jn 1:44; 12:21.}. Estaba casi desprovisto de discernimiento en su visión de las cosas; era incapaz de captar las posibilidades dramáticas de una situación determinada. No era pesimista, sino simplemente prosaico. También carecía en gran medida de perspicacia espiritual. No dudaba en interrumpir a Jesús en medio de uno de sus más profundos discursos para hacer una pregunta aparentemente tonta. Pero Jesús nunca le regañaba por estos atolondramientos; era paciente con él y tomaba en consideración su incapacidad para captar los significados más profundos de la enseñanza. Jesús sabía muy bien que si reprendía una sola vez a Felipe por hacer estas preguntas inoportunas, no solamente heriría a esta alma honrada, sino que tal reprimenda ofendería tanto a Felipe, que nunca más se sentiría libre para hacer preguntas. Jesús sabía que en sus mundos del espacio había miles de millones de mortales de este tipo con lentitud para pensar, y quería animarlos a todos para que acudieran a él y siempre se sintieran libres de someterle sus preguntas y problemas. Después de todo, a Jesús le interesaban realmente más las preguntas tontas de Felipe que el sermón que pudiera estar predicando. Jesús se interesaba de manera suprema por los \textit{hombres}, por todas las clases de hombres.

\par 
%\textsuperscript{(1557.2)}
\textsuperscript{139:5.8} El administrador apostólico no hablaba bien en público, pero era un trabajador personal muy persuasivo y con éxito. No se desanimaba fácilmente; trabajaba con dedicación y tenacidad en todo lo que emprendía. Poseía el gran don excepcional de saber decir: <<Ven>>. Cuando Natanael, su primer converso, quiso discutir sobre los méritos y deméritos de Jesús y de Nazaret, la respuesta eficaz de Felipe fue: <<Ven y ve>>\footnote{\textit{Ven y ve}: Jn 1:45-46.}. No era un predicador dogmático que exhortaba a sus oyentes a que <<fueran>> ---a hacer esto o aquello. Se enfrentaba con todas las situaciones, a medida que surgían en su trabajo, diciendo: <<Ven ---ven conmigo, te mostraré el camino>>. Ésta es siempre la técnica más eficaz en todas las formas y fases de la enseñanza. Incluso los padres pueden aprender de Felipe la mejor manera de decir a sus hijos, \textit{no} <<Id a hacer esto o aquello>>, sino más bien: <<Venid con nosotros, vamos a mostraros y a compartir con vosotros el mejor camino>>.

\par 
%\textsuperscript{(1557.3)}
\textsuperscript{139:5.9} La incapacidad de Felipe para adaptarse a una nueva situación quedó bien ilustrada cuando los griegos se dirigieron a él, en Jerusalén, diciéndole: <<Señor, deseamos ver a Jesús>>. A cualquier judío que hubiera hecho esta petición, Felipe le habría dicho: <<Ven>>. Pero aquellos hombres eran extranjeros, y Felipe no recordaba ninguna instrucción de sus superiores sobre este tema; así pues, lo único que se le ocurrió fue consultar con el jefe Andrés, y a continuación los dos acompañaron a los griegos indagadores hasta Jesús\footnote{\textit{El proceso de decisión}: Jn 12:20-22.}. De la misma manera, cuando fue a Samaria para predicar y bautizar a los creyentes, como su Maestro le había encargado, se abstuvo de imponer las manos sobre sus conversos como símbolo de que habían recibido el Espíritu de la Verdad\footnote{\textit{Felipe no imponía las manos}: Hch 8:5-6,12-16.}. Esto tuvieron que hacerlo Pedro y Juan\footnote{\textit{Imposición de Pedro y Juan}: Hch 8:14-17.}, que vinieron poco después de Jerusalén para observar su labor en nombre de la iglesia madre.

\par 
%\textsuperscript{(1557.4)}
\textsuperscript{139:5.10} Felipe pasó por el penoso período de la muerte del Maestro, participó en la reorganización de los doce, y fue el primero que partió para ganar almas para el reino fuera de la comunidad judía inmediata; tuvo bastante éxito en su labor con los samaritanos y en todos sus trabajos posteriores a favor del evangelio.

\par 
%\textsuperscript{(1557.5)}
\textsuperscript{139:5.11} La esposa de Felipe, que era un miembro eficiente del cuerpo evangélico femenino, se unió activamente a su marido en su trabajo evangélico después de huir de las persecuciones de Jerusalén. Su esposa era una mujer audaz. Permaneció al pie de la cruz de Felipe estimulándolo para que proclamara la buena nueva incluso a sus asesinos; cuando se debilitaron las fuerzas de Felipe, ella empezó a contar la historia de la salvación por medio de la fe en Jesús, y sólo pudieron silenciarla cuando los airados judíos se precipitaron sobre ella y la apedrearon hasta morir. Su hija mayor, Lea, continuó la obra de ambos, convirtiéndose más tarde en la famosa profetisa de Hierápolis.

\par 
%\textsuperscript{(1558.1)}
\textsuperscript{139:5.12} Felipe, el antiguo administrador de los doce, fue un hombre poderoso en el reino, que ganó almas por dondequiera que pasó. Finalmente, fue crucificado por su fe y enterrado en Hierápolis.

\section*{6. El honrado Natanael}
\par 
%\textsuperscript{(1558.2)}
\textsuperscript{139:6.1} Natanael\footnote{\textit{Natanael, traído por Felipe}: Jn 1:45-49.}, el sexto y último apóstol elegido personalmente por el Maestro, fue llevado hasta Jesús por su amigo Felipe. Había estado asociado con Felipe en varias empresas comerciales, e iba de camino con él para ver a Juan el Bautista cuando se encontraron con Jesús.

\par 
%\textsuperscript{(1558.3)}
\textsuperscript{139:6.2} Cuando Natanael se unió a los apóstoles tenía veinticinco años y era el segundo más joven del grupo. Era el hijo menor de una familia de siete, estaba soltero y era el único sostén de sus padres ancianos y enfermos, con quienes vivía en Caná\footnote{\textit{Vivió en Caná}: Jn 21:2.}; sus hermanos y su hermana estaban casados o habían fallecido, y ninguno de ellos vivía allí. Natanael y Judas Iscariote eran los dos hombres más instruídos de los doce. Natanael había pensado en hacerse comerciante.

\par 
%\textsuperscript{(1558.4)}
\textsuperscript{139:6.3} Jesús, personalmente, no le puso un apodo a Natanael, pero los doce pronto empezaron a hablar de él en términos que significaban honradez, sinceridad. Era un hombre <<sin engaño>>\footnote{\textit{Sin engaño}: Jn 1:47.}, y ésta era su gran virtud; era honrado y sincero a la vez. La debilidad de su carácter era su orgullo; estaba muy orgulloso de su familia, de su ciudad, de su reputación y de su país, todo lo cual es loable si no se exagera demasiado. Pero en sus prejuicios personales, Natanael era propenso a extremar las cosas. Tenía la tendencia de prejuzgar a los individuos según sus opiniones personales. Antes incluso de conocer a Jesús, no tardó en preguntar: <<¿Puede algo bueno salir de Nazaret?>>\footnote{\textit{¿Puede algo bueno salir de Nazaret?}: Jn 1:46.} Pero Natanael no era testarudo, aunque fuera orgulloso. Cambió inmediatamente de opinión en cuanto contempló el rostro de Jesús.

\par 
%\textsuperscript{(1558.5)}
\textsuperscript{139:6.4} Natanael era, en muchos aspectos, el genio excéntrico de los doce. Era el filósofo y el soñador apostólico, pero era un tipo de soñador muy práctico. Alternaba entre momentos de profunda filosofía y períodos de un humor excepcional y divertido; cuando tenía la disposición de ánimo apropiada, probablemente era el mejor narrador de historias de los doce. Jesús disfrutaba enormemente escuchando las disertaciones de Natanael sobre las cosas serias y las frívolas. Poco a poco, Natanael fue considerando a Jesús y al reino con más seriedad, pero nunca se tomó en serio a sí mismo.

\par 
%\textsuperscript{(1558.6)}
\textsuperscript{139:6.5} Todos los apóstoles amaban y respetaban a Natanael, y él se llevaba magníficamente bien con todos ellos, excepto con Judas Iscariote. Judas creía que Natanael no se tomaba su apostolado con la suficiente seriedad, y una vez tuvo la temeridad de ir en secreto a Jesús para dar sus quejas contra él. Jesús le dijo: <<Judas, vigila tus pasos con cuidado; no exageres tu cargo. ¿Quién de nosotros está calificado para juzgar a su hermano? No es voluntad del Padre que sus hijos participen solamente en las cosas serias de la vida. Permíteme repetirte que he venido para que mis hermanos en la carne puedan tener un gozo, una alegría y una vida más abundantes. Vete pues, Judas, y haz bien lo que te han confiado, pero deja que tu hermano Natanael dé cuenta de sí mismo a Dios>>\footnote{\textit{He venido a traer gozo}: Jn 15:11. \textit{He venido a traer alegría}: Mc 4:16; Hch 2:46; 14:17; Heb 1:8-9. \textit{He venido a traer vida abundante}: Jn 10:10.}. El recuerdo de esta experiencia, unido al de otras muchas similares, vivió durante mucho tiempo en el corazón engañado de Judas Iscariote.

\par 
%\textsuperscript{(1559.1)}
\textsuperscript{139:6.6} Muchas veces, cuando Jesús estaba en la montaña con Pedro, Santiago y Juan, y la situación se ponía tensa y confusa entre los apóstoles, cuando el mismo Andrés tenía dudas sobre qué decir a sus hermanos entristecidos, Natanael suavizaba la tensión con un poco de filosofía o un golpe de humor; además, un humor de calidad.

\par 
%\textsuperscript{(1559.2)}
\textsuperscript{139:6.7} Natanael tenía el deber de ocuparse de las familias de los doce. A menudo estaba ausente de los consejos apostólicos, porque en cuanto se enteraba de que la enfermedad o algún acontecimiento fuera de lo común afectaba a una de las personas a su cargo, no perdía tiempo en presentarse en el hogar en cuestión. Los doce estaban tranquilos porque sabían que el bienestar de sus familias estaba seguro en las manos de Natanael.

\par 
%\textsuperscript{(1559.3)}
\textsuperscript{139:6.8} Natanael veneraba sobre todo a Jesús por su tolerancia. Nunca se cansaba de contemplar la amplitud de miras y la compasión generosa del Hijo del Hombre.

\par 
%\textsuperscript{(1559.4)}
\textsuperscript{139:6.9} El padre de Natanael (Bartolomé)\footnote{\textit{Natanael Bartolomé}: Mt 10:3; Mc 3:18; Lc 6:14; Hch 1:13.} murió poco después de Pentecostés; a continuación, este apóstol se dirigió a Mesopotamia y a la India para proclamar la buena nueva del reino y bautizar a los creyentes. Sus hermanos no supieron nunca qué había sido de su antiguo filósofo, poeta y humorista. Pero él también fue un gran hombre en el reino y contribuyó mucho a divulgar las enseñanzas de su Maestro, aunque no participó en la organización de la iglesia cristiana posterior. Natanael murió en la India.

\section*{7. Mateo Leví}
\par 
%\textsuperscript{(1559.5)}
\textsuperscript{139:7.1} Mateo\footnote{\textit{Mateo Leví, publicano}: Mt 9:9; Mt 10:3; Mc 2:14; Mc 3:18; Lc 5:27-28; Lc 6:15.}, el séptimo apóstol, fue elegido por Andrés. Mateo pertenecía a una familia de cobradores de impuestos, o publicanos, y él mismo era recaudador de aduanas en Cafarnaúm, donde vivía. Tenía treinta y un años, estaba casado y tenía cuatro hijos. Era un hombre que poseía una riqueza moderada, el único miembro del cuerpo apostólico que contaba con ciertos recursos. Era un buen hombre de negocios, una persona muy sociable, y estaba dotado de la habilidad de hacer amigos y de llevarse muy bien con una gran variedad de personas.

\par 
%\textsuperscript{(1559.6)}
\textsuperscript{139:7.2} Andrés nombró a Mateo representante financiero de los apóstoles. Era en cierto modo el agente fiscal y el portavoz publicitario de la organización apostólica. Era un juez agudo de la naturaleza humana y un propagandista muy eficaz. Es difícil hacerse una idea de su personalidad, pero era un discípulo muy formal y creyó cada vez más en la misión de Jesús y en la certeza del reino. Jesús nunca le puso un apodo a Leví, pero sus compañeros apóstoles se referían a él con frecuencia como el <<que consigue dinero>>.

\par 
%\textsuperscript{(1559.7)}
\textsuperscript{139:7.3} El punto fuerte de Leví era su devoción entusiasta a la causa. El hecho de que él, un publicano, hubiera sido aceptado por Jesús y sus apóstoles, llenaba de gratitud a este antiguo recaudador de impuestos. Sin embargo, el resto de los apóstoles necesitó un poco de tiempo, sobre todo Simón Celotes y Judas Iscariote, para admitir la presencia del publicano entre ellos. La debilidad de Mateo era su visión miope y materialista de la vida, pero a medida que pasaron los meses hizo grandes progresos en todas estas cuestiones. Como tenía el deber de surtir la tesorería, es natural que no pudiera estar presente en muchos de los períodos más preciosos de la instrucción.

\par 
%\textsuperscript{(1559.8)}
\textsuperscript{139:7.4} Lo que Mateo apreciaba más del Maestro era su tendencia a perdonar. Nunca dejaba de repetir que la fe era lo único que se necesitaba en el asunto de encontrar a Dios. Siempre le gustaba hablar del reino como <<este asunto de encontrar a Dios>>.

\par 
%\textsuperscript{(1560.1)}
\textsuperscript{139:7.5} Aunque Mateo era un hombre que tenía su pasado, daba una excelente impresión de sí mismo, y a medida que pasó el tiempo, sus compañeros se enorgullecieron de las acciones del publicano. Fue uno de los apóstoles que tomó amplias notas de los dichos de Jesús, y estas notas se utilizaron posteriormente como base para la narración que hizo Isador de los dichos y hechos de Jesús, que ha llegado a conocerse como el Evangelio según Mateo.

\par 
%\textsuperscript{(1560.2)}
\textsuperscript{139:7.6} La vida grande y útil de Mateo, el hombre de negocios y recaudador de aduanas de Cafarnaúm, ha servido para conducir a miles y miles de otros hombres de negocios\footnote{\textit{Ejemplo para hombres de negocios}: Mt 9:9; Mc 2:14; Lc 5:27-28.}, funcionarios públicos y políticos, durante los siglos siguientes, a escuchar también la atractiva voz del Maestro diciendo: <<Sígueme>>. Mateo era realmente un político sagaz, pero era intensamente fiel a Jesús y estaba dedicado de manera suprema a la tarea de cuidar que los mensajeros del reino venidero estuvieran financiados adecuadamente.

\par 
%\textsuperscript{(1560.3)}
\textsuperscript{139:7.7} La presencia de Mateo entre los doce fue el medio de mantener las puertas del reino abiertas de par en par para una multitud de almas desanimadas y proscritas que se habían considerado desde hacía mucho tiempo excluidas de los consuelos de la religión. Hombres y mujeres repudiados y desesperados se congregaban para escuchar a Jesús, que nunca rechazó a uno solo de ellos.

\par 
%\textsuperscript{(1560.4)}
\textsuperscript{139:7.8} Mateo recibía las donaciones ofrecidas libremente por los discípulos creyentes y los oyentes directos de las enseñanzas del Maestro, pero nunca solicitó abiertamente la contribución de las multitudes. Efectuó todo su trabajo financiero de una manera tranquila y personal, y recaudó la mayor parte del dinero entre la clase más acomodada de los creyentes interesados. Entregó prácticamente la totalidad de su modesta fortuna a la obra del Maestro y sus apóstoles, pero ellos nunca se enteraron de esta generosidad, salvo Jesús, que estaba al corriente de todo. Mateo dudaba en contribuir abiertamente a los fondos apostólicos por temor a que Jesús y sus asociados pudieran considerar que su dinero estaba manchado; en consecuencia, hizo muchas aportaciones en nombre de otros creyentes. Durante los primeros meses, cuando Mateo se daba cuenta de que su presencia entre ellos era más o menos una prueba, sentía la fuerte tentación de hacerles saber que con su dinero se compraba a menudo su pan cotidiano, pero no lo hizo. Cuando la prueba del desdén por el publicano se hacía manifiesta, Leví ardía en deseos de revelarles su generosidad, pero siempre se las arregló para guardar silencio.

\par 
%\textsuperscript{(1560.5)}
\textsuperscript{139:7.9} Cuando los fondos para las necesidades previstas de la semana eran insuficientes, Leví sacaba a menudo cantidades importantes de sus propios recursos personales. A veces también, cuando la enseñanza de Jesús le interesaba mucho, prefería quedarse y escuchar la doctrina, aún sabiendo que tendría que compensar personalmente los fondos necesarios que no había ido a solicitar. ¡Pero Leví deseaba tanto que Jesús supiera que una buena parte del dinero procedía de su bolsillo! Poco podía suponer que el Maestro estaba al corriente de todo. Todos los apóstoles murieron sin saber que Mateo fue su benefactor hasta tal extremo, que cuando partió para proclamar el evangelio del reino, después del comienzo de las persecuciones, estaba prácticamente en la pobreza.

\par 
%\textsuperscript{(1560.6)}
\textsuperscript{139:7.10} Cuando estas persecuciones obligaron a los creyentes a abandonar Jerusalén, Mateo viajó hacia el norte, predicando el evangelio del reino y bautizando a los creyentes. Sus antiguos asociados apostólicos perdieron todo contacto con él, pero continuó predicando y bautizando en Siria, Capadocia, Galacia, Bitinia y Tracia. Fue en Tracia, en Lisimaquia, donde ciertos judíos increyentes conspiraron con los soldados romanos para provocar su muerte. Este publicano regenerado murió triunfante en la fe de una salvación que había adquirido con tanta seguridad de las enseñanzas del Maestro durante su reciente estancia en la Tierra.

\section*{8. Tomás Dídimo}
\par 
%\textsuperscript{(1561.1)}
\textsuperscript{139:8.1} Tomás era el octavo apóstol y fue elegido por Felipe. En siglos posteriores se le ha conocido como <<Tomás el incrédulo>>\footnote{\textit{Tomás el incrédulo}: Jn 20:24-25.}, pero sus hermanos apóstoles apenas lo consideraban como un incrédulo crónico. Es cierto que tenía un tipo de mente lógica y escéptica, pero poseía una forma de lealtad valiente que impedía a los que lo conocían íntimamente considerarlo como un escéptico vano.

\par 
%\textsuperscript{(1561.2)}
\textsuperscript{139:8.2} Cuando Tomás se unió a los apóstoles tenía veintinueve años, estaba casado y tenía cuatro hijos. Anteriormente había sido carpintero y albañil, pero después se convirtió en pescador y residía en Tariquea, población situada en la orilla occidental del Jordán, donde el río sale del Mar de Galilea, y estaba considerado como el ciudadano más importante de este pueblecito. Tenía poca instrucción, pero poseía una mente aguda y racional; era hijo de unos padres excelentes que vivían en Tiberiades. Tomás poseía la única mente realmente analítica de los doce; era el verdadero científico del grupo apostólico.

\par 
%\textsuperscript{(1561.3)}
\textsuperscript{139:8.3} Los primeros años de la vida familiar de Tomás habían sido desdichados; sus padres no eran plenamente felices en su vida matrimonial, y esto repercutió en la experiencia adulta de Tomás. Creció con un carácter muy desagradable y pendenciero. Incluso su esposa se alegró de que se uniera a los apóstoles; se sintió aliviada con la idea de que su pesimista marido estaría lejos del hogar la mayor parte del tiempo. Tomás tenía también una vena de desconfianza que hacía muy difícil llevarse pacíficamente con él. Pedro se contrarió mucho al principio por la presencia de Tomás, y se quejaba a su hermano Andrés de que Tomás era <<mezquino, mal parecido y siempre desconfiado>>. Pero cuanto más conocieron sus compañeros a Tomás, más lo quisieron. Descubrieron que era extremadamente honrado y resueltamente leal. Era perfectamente sincero e incuestionablemente veraz, pero era un crítico nato y había crecido convirtiéndose en un auténtico pesimista. Su mente analítica estaba afligida por la desconfianza. Estaba perdiendo rápidamente la fe en sus semejantes cuando se asoció con los doce y entró así en contacto con el noble carácter de Jesús. Esta asociación con el Maestro empezó a transformar inmediatamente todo el modo de ser de Tomás, y a efectuar grandes cambios en sus reacciones mentales hacia sus semejantes.

\par 
%\textsuperscript{(1561.4)}
\textsuperscript{139:8.4} La gran fuerza de Tomás era su extraordinaria mente analítica unida a su valor resuelto ---una vez que había tomado una decisión. Su gran debilidad era su duda suspicaz, que nunca venció por completo en toda su vida en la carne.

\par 
%\textsuperscript{(1561.5)}
\textsuperscript{139:8.5} En la organización de los doce, Tomás estaba encargado de preparar y dirigir el itinerario, y fue un director capacitado del trabajo y de los desplazamientos del cuerpo apostólico. Era un buen ejecutivo, un excelente hombre de negocios, pero estaba limitado por sus numerosos cambios de humor; no era el mismo hombre de un día para el siguiente. Cuando se unió a los apóstoles tenía inclinación por las cavilaciones melancólicas, pero el contacto con Jesús y los apóstoles lo curó en gran medida de esta morbosa introspección.

\par 
%\textsuperscript{(1561.6)}
\textsuperscript{139:8.6} Jesús disfrutaba mucho con la compañía de Tomás y tuvo muchas conversaciones largas y personales con él. La presencia de Tomás entre los apóstoles era un gran consuelo para todos los escépticos honrados y animó a muchas mentes afligidas a entrar en el reino, aunque no pudieran comprender íntegramente todos los aspectos espirituales y filosóficos de las enseñanzas de Jesús. La presencia de Tomás entre los doce era una declaración permanente de que Jesús amaba incluso a los escépticos honrados.

\par 
%\textsuperscript{(1562.1)}
\textsuperscript{139:8.7} Los otros apóstoles tenían veneración por Jesús a causa de algún rasgo especial y destacado de su personalidad tan rica, pero Tomás veneraba a su Maestro por su carácter magníficamente equilibrado. Tomás admiraba y honraba cada vez más a aquel que era tan afectuosamente misericordioso y sin embargo justo y equitativo de manera tan inflexible; que era tan firme pero nunca testarudo; tan tranquilo, pero nunca indiferente; tan socorrido y tan compasivo, pero nunca entrometido ni dictatorial; tan fuerte pero al mismo tiempo tan dulce; tan positivo, pero nunca tosco ni brusco; tan tierno pero nunca vacilante; tan puro e inocente, pero al mismo tiempo tan viril, dinámico y enérgico; tan verdaderamente valiente, pero nunca temerario ni imprudente; tan amante de la naturaleza, pero tan libre de toda tendencia a venerarla; tan lleno de humor y tan jovial, pero tan libre de ligereza y de frivolidad. Esta incomparable simetría de su personalidad era lo que tanto encantaba a Tomás. De los doce, él era probablemente el que mejor comprendía intelectualmente a Jesús y apreciaba mejor su personalidad.

\par 
%\textsuperscript{(1562.2)}
\textsuperscript{139:8.8} En los consejos de los doce, Tomás era siempre precavido y defendía la política de <<primero la seguridad>>, pero si se votaba en contra de su conservadurismo o se rechazaba, siempre era el primero en lanzarse intrépidamente a ejecutar el programa que se había aprobado. Una y otra vez se oponía a un proyecto determinado por considerarlo arriesgado y temerario, y lo debatía encarnizadamente hasta el final; pero cuando Andrés sometía la proposición a votación, y cuando los doce escogían hacer aquello contra lo que se había opuesto tan enérgicamente, Tomás era el primero en decir: <<¡Vamos!>>\footnote{\textit{El primero en decir ``¡Vamos!''}: Jn 11:16.}. Era un buen perdedor. No guardaba rencor ni alimentaba resentimientos. Una y otra vez se opuso a dejar que Jesús se expusiera a un peligro, pero cuando el Maestro decidía correr ese riesgo, siempre era Tomás el que reunía a los apóstoles con sus valientes palabras: <<Venid, camaradas, vamos a morir con él>>\footnote{\textit{Vayamos a morir con él}: Jn 11:16.}.

\par 
%\textsuperscript{(1562.3)}
\textsuperscript{139:8.9} En algunos aspectos, Tomás era como Felipe, también quería <<que le mostraran>>; pero sus expresiones exteriores de duda se basaban en mecanismos intelectuales completamente diferentes. Tomás era analítico, y no simplemente escéptico. En cuanto al valor físico personal, era uno de los más valientes de los doce.

\par 
%\textsuperscript{(1562.4)}
\textsuperscript{139:8.10} Tomás tenía algunos días muy malos; a veces estaba triste y abatido. La pérdida de su hermana gemela\footnote{\textit{Tomás Dídimo (Mellizo)}: Mt 10:3; Mc 3:18; Lc 6:15; Jn 21:2; Hch 1:13.}, cuando él tenía nueve años, le había producido mucha pena juvenil y había aumentado los problemas temperamentales de su vida posterior. Cuando Tomás se desalentaba, a veces era Natanael quien le ayudaba a recuperarse, otras veces Pedro, y con frecuencia uno de los gemelos Alfeo. Desgraciadamente, cuando estaba más deprimido siempre trataba de evitar el contacto directo con Jesús. Pero el Maestro estaba al corriente de todo esto y tenía una simpatía comprensiva por su apóstol cuando estaba así de afligido por la depresión y acosado por las dudas.

\par 
%\textsuperscript{(1562.5)}
\textsuperscript{139:8.11} Tomás conseguía a veces el permiso de Andrés para marcharse a solas durante un día o dos. Pero pronto aprendió que este modo de obrar era poco sabio; pronto descubrió que cuando estaba abatido era mejor aferrarse a su trabajo y permanecer cerca de sus compañeros. Pero independientemente de lo que sucediera en su vida emocional, continuaba siendo firmemente un apóstol. Cuando realmente llegaba el momento de ir hacia adelante, siempre era Tomás el que decía: <<¡Vamos!>>.

\par 
%\textsuperscript{(1562.6)}
\textsuperscript{139:8.12} Tomás es el gran ejemplo de un ser humano que tiene dudas, se enfrenta con ellas y las vence. Tenía una mente poderosa y no era un crítico mordaz. Era un pensador lógico; era la prueba decisiva para Jesús y sus compañeros apóstoles. Si Jesús y su obra no hubieran sido auténticos, no hubieran podido retener, desde el principio hasta el fin, a un hombre como Tomás. Tenía un sentido agudo y seguro de los \textit{hechos}. Al primer síntoma de fraude o de engaño, Tomás los hubiera abandonado a todos. Los científicos pueden no comprender plenamente todo lo concerniente a Jesús y su obra en la Tierra, pero allí había un hombre que vivió y trabajó con el Maestro y sus asociados humanos, cuya mente era la de un verdadero científico ---Tomás Dídimo--- y él creía en Jesús de Nazaret.

\par 
%\textsuperscript{(1563.1)}
\textsuperscript{139:8.13} Tomás pasó por momentos difíciles durante los días del juicio y la crucifixión. Estuvo sumido algún tiempo en los abismos de la desesperación, pero recobró su valor, se pegó tenazmente a los apóstoles y estuvo presente con ellos para acoger a Jesús en el Mar de Galilea\footnote{\textit{Acogió a Jesús en Galilea}: Jn 21:1-2.}. Sucumbió por algún tiempo a la depresión de su incredulidad, pero finalmente recuperó su fe y su valor. Aconsejó sabiamente a los apóstoles después de Pentecostés y cuando la persecución dispersó a los creyentes, fue a Chipre, Creta, la costa norteafricana y Sicilia, predicando la buena nueva del reino y bautizando a los creyentes. Tomás continuó predicando y bautizando hasta que fue capturado por los agentes del gobierno romano y ejecutado en Malta. Sólo unas semanas antes de su muerte había empezado a escribir la vida y las enseñanzas de Jesús.

\section*{9. y 10. Santiago y Judas Alfeo}
\par 
%\textsuperscript{(1563.2)}
\textsuperscript{139:9.1} Santiago y Judas, los hijos de Alfeo\footnote{\textit{Los gemelos Alfeo, Santiago y Judas}: Mt 10:3; Mc 3:18; Lc 6:15-16; Hch 1:13.}, los pescadores gemelos que vivían cerca de Jeresa, fueron los noveno y décimo apóstoles, y fueron elegidos por Santiago y Juan Zebedeo. Tenían veintiséis años y estaban casados; Santiago tenía tres hijos y Judas dos.

\par 
%\textsuperscript{(1563.3)}
\textsuperscript{139:9.2} No hay mucho que decir sobre estos dos pescadores corrientes. Amaban a su Maestro y Jesús los amaba, pero nunca interrumpían sus discursos con preguntas. Comprendían muy poca cosa de las discusiones filosóficas o de los debates teológicos de sus compañeros apóstoles, pero les alegraba encontrarse entre los miembros de este grupo de hombres importantes. Estos dos hombres eran casi idénticos en su apariencia personal, en sus características mentales y en el alcance de su percepción espiritual. Lo que puede decirse de uno se puede aplicar al otro.

\par 
%\textsuperscript{(1563.4)}
\textsuperscript{139:9.3} Andrés les asignó el trabajo de mantener el orden entre las multitudes. Eran los celadores principales durante las horas de predicación y, de hecho, los servidores generales y los recaderos de los doce. Ayudaban a Felipe con los víveres, llevaban el dinero de Natanael a las familias, y siempre estaban dispuestos a prestar ayuda a cualquiera de los apóstoles.

\par 
%\textsuperscript{(1563.5)}
\textsuperscript{139:9.4} Las multitudes de gente común y corriente se sentían muy estimuladas al ver a dos personas como ellas honradas con un puesto entre los apóstoles. Mediante su admisión como apóstoles, estos gemelos mediocres fueron el medio de atraer al reino a numerosos creyentes pusilánimes. Además, la gente común y corriente aceptaba mejor la idea de ser conducida y dirigida por unos celadores oficiales que se parecían mucho a ellos mismos.

\par 
%\textsuperscript{(1563.6)}
\textsuperscript{139:9.5} Santiago y Judas, a quienes también se les llamaba Tadeo y Lebeo\footnote{\textit{También llamados Tadeo y Lebeo}: Mt 10:3; Mc 3:18.}, no tenían puntos fuertes ni débiles. Los apodos que les dieron los discípulos eran designaciones bondadosas de mediocridad. Eran <<los menores de todos los apóstoles>>\footnote{\textit{Los menores de los apóstoles}: 1 Co 15:9.}; lo sabían y se sentían complacidos con ello.

\par 
%\textsuperscript{(1563.7)}
\textsuperscript{139:9.6} Santiago Alfeo amaba especialmente a Jesús por la sencillez del Maestro. Estos gemelos no podían comprender la mente de Jesús, pero captaban el vínculo de simpatía entre ellos y el corazón de su Maestro. Su mente no era de un orden elevado; incluso se les podría calificar respetuosamente de tontos, pero efectuaron una experiencia real en su naturaleza espiritual. Creían en Jesús; eran hijos de Dios y miembros del reino.

\par 
%\textsuperscript{(1564.1)}
\textsuperscript{139:9.7} Judas Alfeo se sentía atraído por Jesús debido a la humildad sin ostentación del Maestro. Una humildad así, unida a una dignidad personal semejante, ejercía una gran atracción sobre Judas. El hecho de que Jesús recomendara siempre que no mencionaran sus actos extraordinarios causaba una gran impresión a este sencillo hijo de la naturaleza.

\par 
%\textsuperscript{(1564.2)}
\textsuperscript{139:9.8} Los gemelos eran unos asistentes bondadosos y simples, y todo el mundo los quería. Jesús acogió a estos jóvenes, dotados de un solo talento, en puestos de honor de su plana mayor personal en el reino porque existen miles de millones de otras almas semejantes, simples y temerosas, en los mundos del espacio, a quienes el Maestro desea acoger igualmente en una comunión activa y creyente con él y con su Espíritu de la Verdad efusionado. Jesús no desprecia la pequeñez, sino sólo el mal y el pecado. Santiago y Judas eran \textit{limitados}, pero también \textit{fieles}. Eran simples e ignorantes, pero también generosos, cariñosos y desprendidos.

\par 
%\textsuperscript{(1564.3)}
\textsuperscript{139:9.9} Qué orgullo más grato sintieron estos hombres humildes el día en que el Maestro se negó a aceptar a cierto hombre rico como evangelista\footnote{\textit{Jesús rechazó a un hombre rico}: Mt 19:21-22; Mc 10:21-22; Lc 18:22-23.}, a menos que vendiera sus bienes y ayudara a los pobres. Cuando la gente escuchó esto y contempló a los gemelos entre sus consejeros, supieron con seguridad que Jesús no hacía acepción de personas\footnote{\textit{Jesús no hacía acepción de personas}: 2 Cr 19:7; Job 34:19; Eclo 35:12; Mt 22:16; Mc 12:14; Lc 20:21; Hch 10:34; Ro 2:11; Gl 2:6; 3:28; Ef 6:9; Col 3:11.}. ¡Sólo una institución divina ---el reino de los cielos--- podía construírse sobre unos fundamentos humanos tan mediocres!

\par 
%\textsuperscript{(1564.4)}
\textsuperscript{139:9.10} En toda su asociación con Jesús, los gemelos sólo se atrevieron una o dos veces a hacer preguntas en público. Cierta vez, Judas se sintió intrigado hasta el punto de hacerle una pregunta a Jesús cuando el Maestro habló de revelarse abiertamente al mundo. Se sintió un poco decepcionado de que ya no hubiera secretos que pertenecieran a los doce, y se atrevió a preguntar: <<Pero, Maestro, cuando te proclames así al mundo, ¿cómo nos favorecerás con manifestaciones especiales de tu bondad?>>\footnote{\textit{La rara pregunta de Judas}: Jn 14:22.}.

\par 
%\textsuperscript{(1564.5)}
\textsuperscript{139:9.11} Los gemelos sirvieron fielmente hasta el fin, hasta los días sombríos del juicio, la crucifixión y la desesperación. Nunca perdieron la fe de su corazón en Jesús y (con excepción de Juan) fueron los primeros en creer en su resurrección. Pero no pudieron comprender el establecimiento del reino. Poco después de que su Maestro fuera crucificado, regresaron a sus familias y a sus redes; su trabajo había concluido. No estaban capacitados para proseguir en las batallas más complejas del reino. Pero vivieron y murieron conscientes de haber sido honrados y bendecidos con cuatro años de asociación estrecha y personal con un Hijo de Dios, el autor soberano de un universo.

\section*{11. Simón el Celote}
\par 
%\textsuperscript{(1564.6)}
\textsuperscript{139:11.1} Simón Celotes\footnote{\textit{La admisión de Simón Celotes}: Mt 10:4; Mc 3:18; Lc 6:15; Hch 1:13.}, el undécimo apóstol, fue elegido por Simón Pedro. Era un hombre capacitado, de buen linaje, que vivía con su familia en Cafarnaúm. Tenía veintiocho años cuando se unió a los apóstoles. Era un ardiente agitador y también un hombre que hablaba mucho sin reflexionar. Había sido comerciante en Cafarnaúm antes de dirigir toda su atención a la organización patriótica de los celotes.

\par 
%\textsuperscript{(1564.7)}
\textsuperscript{139:11.2} A Simón Celotes lo encargaron de las diversiones y de la distracción del grupo apostólico, y fue un organizador muy eficaz del entretenimiento y las actividades recreativas de los doce.

\par 
%\textsuperscript{(1564.8)}
\textsuperscript{139:11.3} La fuerza de Simón radicaba en su lealtad inspiradora. Cuando los apóstoles se encontraban con un hombre o una mujer que vacilaba en la indecisión de entrar en el reino, enviaban a buscar a Simón. Habitualmente, este defensor entusiasta de la salvación mediante la fe en Dios sólo necesitaba unos quince minutos para aclarar todas las dudas y eliminar toda indecisión, para ver cómo nacía una nueva alma a la <<libertad de la fe y la alegría de la salvación>>.

\par 
%\textsuperscript{(1565.1)}
\textsuperscript{139:11.4} La gran debilidad de Simón era su mentalidad materialista. Este judío nacionalista no podía convertirse rápidamente en un internacionalista con inclinaciones espirituales. Cuatro años eran insuficientes para efectuar una transformación intelectual y emocional semejante, pero Jesús siempre fue paciente con él.

\par 
%\textsuperscript{(1565.2)}
\textsuperscript{139:11.5} Lo que Simón más admiraba de Jesús era la calma del Maestro, su seguridad, su equilibrio y su inexplicable serenidad.

\par 
%\textsuperscript{(1565.3)}
\textsuperscript{139:11.6} Aunque Simón era un rabioso revolucionario, un agitador audaz, subyugó gradualmente su ardiente naturaleza hasta convertirse en un predicador poderoso y eficaz de <<la paz en la Tierra y la buena voluntad entre los hombres>>\footnote{\textit{Predicar la paz y la buena voluntad}: Lc 2:14.}. Simón era un gran polemista; le gustaba discutir. Cuando había que tratar con las mentes legalistas de los judíos cultos o con los sofismas intelectuales de los griegos, esta tarea siempre se asignaba a Simón.

\par 
%\textsuperscript{(1565.4)}
\textsuperscript{139:11.7} Era un rebelde por naturaleza y un iconoclasta por su formación, pero Jesús lo conquistó para los conceptos superiores del reino de los cielos. Siempre se había identificado con el partido de la protesta, pero ahora se unía al partido del progreso, el de la evolución ilimitada y eterna del espíritu y de la verdad. Simón era un hombre de lealtades intensas y de ardientes devociones personales, y amaba profundamente a Jesús.

\par 
%\textsuperscript{(1565.5)}
\textsuperscript{139:11.8} Jesús no tenía miedo de identificarse con los hombres de negocios, los obreros, los optimistas, los pesimistas, los filósofos, los escépticos, los publicanos, los políticos y los patriotas.

\par 
%\textsuperscript{(1565.6)}
\textsuperscript{139:11.9} El Maestro tuvo muchas conversaciones con Simón, pero nunca logró transformar plenamente a este ardiente nacionalista judío en un internacionalista. Jesús le dijo a menudo a Simón que era correcto desear la mejora del orden social, económico y político, pero siempre añadía: <<Eso no es asunto del reino de los cielos. Debemos dedicarnos a hacer la voluntad del Padre. Nuestro trabajo consiste en ser los embajadores de un gobierno espiritual de arriba, y no debemos ocuparnos inmediatamente de otra cosa que no sea representar la voluntad y el carácter del Padre divino que dirige ese gobierno, cuyas cartas credenciales aportamos>>. Todo esto era difícil de comprender para Simón, pero empezó gradualmente a captar una parte del significado de la enseñanza del Maestro.

\par 
%\textsuperscript{(1565.7)}
\textsuperscript{139:11.10} Después de la dispersión ocasionada por las persecuciones en Jerusalén, Simón se retiró de forma temporal. Estaba literalmente deshecho. Había renunciado como patriota nacionalista por deferencia a las enseñanzas de Jesús; y ahora todo estaba perdido. Estaba desesperado, pero al cabo de unos años recobró sus esperanzas y salió a proclamar el evangelio del reino.

\par 
%\textsuperscript{(1565.8)}
\textsuperscript{139:11.11} Fue a Alejandría, y después de trabajar Nilo arriba penetró en el corazón de África, predicando por todas partes el evangelio de Jesús y bautizando a los creyentes. Así estuvo trabajando hasta que fue viejo y débil. Cuando murió fue enterrado en el corazón de
África.

\section*{12. Judas Iscariote}
\par 
%\textsuperscript{(1565.9)}
\textsuperscript{139:12.1} Judas Iscariote\footnote{\textit{La elección de Judas Iscariote}: Mt 10:4; Mc 3:19; Lc 6:16.}, el duodécimo apóstol, fue elegido por Natanael. Había nacido en Queriot, una pequeña ciudad del sur de Judea. Cuando era un muchacho, sus padres se mudaron a Jericó, donde vivió y estuvo trabajando en las diversas empresas comerciales de su padre, hasta que se interesó por la predicación y la obra de Juan el Bautista. Los padres de Judas eran saduceos, y repudiaron a su hijo cuando éste se unió a los discípulos de Juan.

\par 
%\textsuperscript{(1566.1)}
\textsuperscript{139:12.2} Cuando Natanael lo encontró en Tariquea, Judas estaba buscando trabajo en una empresa desecadora de pescado en el extremo sur del Mar de Galilea. Tenía treinta años y estaba soltero cuando se unió a los apóstoles. Era probablemente el más instruido de los doce y el único judeo de la familia apostólica del Maestro. Judas no tenía ningún rasgo destacado de virtud personal, aunque poseía exteriormente muchas características aparentes de cultura y de buena educación. Era un buen pensador, pero no siempre un pensador verdaderamente \textit{honrado}. Judas no se comprendía en realidad a sí mismo; no era realmente sincero consigo mismo.

\par 
%\textsuperscript{(1566.2)}
\textsuperscript{139:12.3} Andrés nombró a Judas tesorero de los doce, un puesto para el que estaba eminentemente preparado, y hasta el momento de traicionar a su Maestro, cumplió con las responsabilidades de su cargo de manera honesta, fiel y con la mayor eficacia.

\par 
%\textsuperscript{(1566.3)}
\textsuperscript{139:12.4} Judas no admiraba ningún rasgo especial de Jesús, aparte de la personalidad generalmente atractiva y exquisitamente encantadora del Maestro. Judas nunca fue capaz de superar sus prejuicios de judeo contra sus compañeros galileos; llegó incluso a criticar, en su mente, muchas cosas de Jesús. Este judeo satisfecho de sí mismo se atrevía a criticar a menudo, en su propio fuero interno, a aquel a quien once de los apóstoles consideraban como el hombre perfecto, como <<el único enteramente amable y el más sobresaliente entre diez mil>>\footnote{\textit{El único enteramente amable}: Cnt 5:16. \textit{El más sobresaliente entre diez mil}: Cnt 5:10.}. Albergaba realmente la noción de que Jesús era tímido y de que tenía cierto miedo a afirmar su propio poder y autoridad.

\par 
%\textsuperscript{(1566.4)}
\textsuperscript{139:12.5} Judas era un hombre de negocios sobresaliente. Se necesitaba tacto, habilidad y paciencia, así como una devoción concienzuda, para administrar los asuntos financieros de un idealista como Jesús, sin mencionar la lucha contra los métodos desordenados de algunos de sus apóstoles en el tema de los negocios. Judas era realmente un gran ejecutivo, un financiero previsor y capaz, y un defensor de la organización. Ninguno de los doce criticó nunca a Judas. Hasta donde eran capaces de percibir, Judas Iscariote era un tesorero incomparable, un hombre culto, un apóstol leal (aunque crítico a veces) y un gran acierto en todos los sentidos de la palabra. Los apóstoles amaban a Judas; era realmente uno de ellos. Debe haber \textit{creído} en Jesús, pero dudamos de que \textit{amara} realmente al Maestro con todo su corazón. El caso de Judas ilustra la verdad del proverbio: <<Hay un camino que le parece justo a un hombre, pero cuyo final es la muerte>>\footnote{\textit{El camino que parece correcto}: Pr 14:12; 16:25.}. Es completamente posible caer víctima del engaño sosegado de la agradable adaptación a los caminos del pecado y de la muerte. Estad seguros de que, en el aspecto financiero, Judas siempre fue leal a su Maestro y a sus compañeros apóstoles. El dinero nunca hubiera podido ser el motivo de su traición al Maestro.

\par 
%\textsuperscript{(1566.5)}
\textsuperscript{139:12.6} Judas era el hijo único de unos padres poco sabios, que lo consintieron y mimaron cuando era pequeño; fue un niño malcriado. Creció con una idea exagerada de su propia importancia. No era un buen perdedor. Tenía ideas vagas y retorcidas sobre la justicia; era dado a entregarse al odio y a la desconfianza. Era un experto en tergiversar las palabras y las acciones de sus amigos. Durante toda su vida, Judas había cultivado el hábito de desquitarse con aquellos que suponía que lo habían maltratado. Su sentido de los valores y de las lealtades era defectuoso.

\par 
%\textsuperscript{(1566.6)}
\textsuperscript{139:12.7} Para Jesús, Judas era una aventura de la fe. El Maestro comprendió plenamente desde el principio la debilidad de este apóstol y conocía muy bien los peligros de admitirlo en la confraternidad. Pero es propio de la naturaleza de los Hijos de Dios el dar a todos los seres creados una oportunidad plena e igual de salvación y supervivencia. Jesús quería que no sólo los mortales de este mundo, sino también los observadores de otros innumerables mundos, supieran que si existen dudas sobre la sinceridad y el entusiasmo de la devoción de una criatura hacia el reino, los Jueces de los hombres tienen la costumbre invariable de aceptar plenamente al candidato dudoso. La puerta de la vida eterna está abierta de par en par para todos; <<todo el que quiera puede venir>>\footnote{\textit{Todo el que quiera puede venir}: Sal 50:15; Jl 2:32; Zac 13:9; Mt 7:24; 10:32-33; 12:50; 16:24-25; Mc 3:35; 8:34-35; Lc 6:47; 9:23-24; 12:8; Jn 3:15-16; 4:13-14; 11:25-26; 12:46; Hch 2:21; 10:43; 13:26; Ro 9:33; 10:13; 1 Jn 2:23; 4:15; 5:1; Ap 22:17b.}; no hay restricciones ni limitaciones, salvo la \textit{fe} del que viene.

\par 
%\textsuperscript{(1567.1)}
\textsuperscript{139:12.8} Ésta es precisamente la razón por la cual Jesús permitió que Judas continuara hasta el fin, haciendo siempre todo lo posible por transformar y salvar a este apóstol débil y confundido. Pero cuando la luz no se recibe con honradez ni se vive en conformidad con ella, tiende a convertirse en tinieblas dentro del alma. Judas creció intelectualmente en cuanto a las enseñanzas de Jesús sobre el reino, pero no progresó en la adquisición de un carácter espiritual, como lo hicieron los otros apóstoles. No consiguió realizar un progreso personal satisfactorio en su experiencia espiritual.

\par 
%\textsuperscript{(1567.2)}
\textsuperscript{139:12.9} Judas se dedicó a cavilar cada vez más sobre sus desilusiones personales, y finalmente se convirtió en una víctima del resentimiento. Sus sentimientos habían sido heridos muchas veces, y se volvió anormalmente desconfiado con sus mejores amigos, e incluso con el Maestro. Pronto se obsesionó con la idea de desquitarse, de hacer lo que fuera para vengarse, sí, incluso traicionando a sus compañeros y a su Maestro.

\par 
%\textsuperscript{(1567.3)}
\textsuperscript{139:12.10} Pero estas ideas perversas y peligrosas no cobraron forma definitiva hasta el día en que una mujer agradecida rompió un costoso frasco de incienso\footnote{\textit{Frasco de incienso}: Mt 26:6-7; Mc 14:3; Lc 7:37-38; Jn 11:2; 12:3.} a los pies de Jesús. Esto le pareció a Judas un despilfarro, y cuando Jesús rechazó tan radicalmente su protesta pública allí mismo en presencia de todos, aquello fue demasiado para él. Este suceso desencadenó la movilización de todo el odio, el daño, la maldad, los prejuicios, los celos y los deseos de revancha acumulados durante toda una vida, y decidió desquitarse con quien fuera. Pero cristalizó toda la maldad de su naturaleza sobre la \textit{única} persona inocente de todo el drama sórdido de su vida desgraciada, simplemente porque dio la casualidad de que Jesús era el actor principal en el episodio que marcó su pasaje desde el reino progresivo de la luz al dominio de las tinieblas escogido por él mismo.

\par 
%\textsuperscript{(1567.4)}
\textsuperscript{139:12.11} En muchas ocasiones, tanto en público como en privado, el Maestro había advertido a Judas que se estaba desviando, pero las advertencias divinas son generalmente inútiles cuando se dirigen a una naturaleza humana amargada. Jesús hizo todo lo posible y compatible con la libertad moral del hombre para evitar que Judas escogiera el camino equivocado. La gran prueba acabó por llegar. El hijo del resentimiento fracasó; cedió a los dictados agrios y sórdidos de una mente orgullosa y vengativa que exageraba su propia importancia, y se hundió rápidamente en la confusión, la desesperación y la depravación.

\par 
%\textsuperscript{(1567.5)}
\textsuperscript{139:12.12} Judas dio comienzo entonces a la intriga vil y vergonzosa de traicionar\footnote{\textit{La traición de Judas}: Mt 26:14-16,47-49; Mc 14:10-11,43-45; Lc 22:3-5,47-48; Jn 13:2,26-27; 18:2-5.} a su Señor y Maestro, y rápidamente llevó a cabo su nefasto proyecto. Durante la ejecución de sus planes de pérfida traición, concebidos en la cólera, experimentó momentos de pesar y de verg\"uenza, y en esos intervalos de lucidez concebía tímidamente la idea, para justificarse en su propia mente, de que Jesús quizás podría ejercer su poder y salvarse en el último momento.

\par 
%\textsuperscript{(1567.6)}
\textsuperscript{139:12.13} Cuando este asunto sórdido y pecaminoso hubo terminado, este mortal renegado, que con tanta ligereza había vendido a su amigo por treinta monedas de plata para satisfacer las ansias de venganza que había alimentado durante tanto tiempo, salió precipitadamente y cometió el acto final del drama consistente en huir de las realidades de la existencia mortal ---se suicidó\footnote{\textit{El suicidio de Judas}: Mt 27:3-5; Hch 1:18.}.

\par 
%\textsuperscript{(1567.7)}
\textsuperscript{139:12.14} Los once apóstoles se quedaron horrorizados, anonadados. Jesús se limitó a mirar con lástima al traidor. Los mundos han encontrado difícil perdonar a Judas, y se evita pronunciar su nombre en todo un vasto universo.


\chapter{Documento 140. La ordenación de los doce}
\par 
%\textsuperscript{(1568.1)}
\textsuperscript{140:0.1} El domingo 12 de enero del año 27, un poco antes del mediodía, Jesús reunió a los apóstoles para su ordenación\footnote{\textit{Sermón de la ordenación}: Mt 5:1; Lc 6:17.} como predicadores públicos del evangelio del reino\footnote{\textit{El evangelio del reino}: Mt 4:23; 9:35; 24:14; Mc 1:14-15.}. Los doce esperaban ser llamados de un día a otro; por eso, aquella mañana no se alejaron mucho de la costa para pescar. Algunos de ellos se habían quedado cerca de la orilla reparando sus redes y remendando sus atavíos de pesca.

\par 
%\textsuperscript{(1568.2)}
\textsuperscript{140:0.2} Cuando Jesús bajó a la playa para convocar a los apóstoles, primero llamó a Andrés y Pedro, que estaban pescando cerca de la orilla; luego hizo señas a Santiago y Juan, que se encontraban cerca en una barca charlando con su padre Zebedeo y reparando sus redes. Reunió de dos en dos a los otros apóstoles, y cuando los doce estuvieron congregados, se dirigió con ellos hacia las tierras montañosas del norte de Cafarnaúm, donde procedió a instruirlos como preparación para su ordenación formal.

\par 
%\textsuperscript{(1568.3)}
\textsuperscript{140:0.3} Por una vez, los doce apóstoles estaban silenciosos; incluso Pedro se hallaba pensativo. ¡Por fin había llegado la hora tanto tiempo esperada! Partían a solas con el Maestro para participar en algún tipo de ceremonia solemne de consagración personal y de dedicación colectiva al trabajo sagrado de representar a su Maestro en la proclamación del advenimiento del reino de su Padre.

\section*{1. La instrucción preliminar}
\par 
%\textsuperscript{(1568.4)}
\textsuperscript{140:1.1} Antes del servicio formal de ordenación, Jesús dijo a los doce que estaban sentados a su alrededor: <<Hermanos míos, la hora del reino ha llegado. Os he traído aquí, a solas conmigo, para presentaros al Padre como embajadores del reino. Algunos de vosotros me habéis oído hablar de este reino en la sinagoga cuando fuisteis llamados por primera vez. Cada uno de vosotros ha aprendido más sobre el reino del Padre desde que habéis estado trabajando conmigo en las ciudades cercanas al Mar de Galilea. Pero en este momento tengo algo más que deciros con respecto a este reino>>.

\par 
%\textsuperscript{(1568.5)}
\textsuperscript{140:1.2} <<El nuevo reino que mi Padre está a punto de establecer en el corazón de sus hijos terrestres está destinado a ser un dominio eterno. Este gobierno de mi Padre en el corazón de aquellos que desean hacer su voluntad divina no tendrá fin. Os declaro que mi Padre no es el Dios de los judíos o de los gentiles. Muchos vendrán del este y del oeste para sentarse con nosotros en el reino del Padre, mientras que muchos hijos de Abraham se negarán a entrar en esta nueva fraternidad, en la que el espíritu del Padre reina en el corazón de los hijos de los hombres>>\footnote{\textit{El reino eterno}: Lc 1:33. \textit{No es el Dios de los judíos o de los gentiles}: 2 Cr 19:7; Job 34:19; Eclo 35:12; Hch 10:34; Ro 2:9-11; 9:24; 10:12; Gl 2:6; 3:28; Ef 6:9; Col 3:11. \textit{Muchos vendrán}: Mt 8:11-12; Lc 13:28-29.}.

\par 
%\textsuperscript{(1568.6)}
\textsuperscript{140:1.3} <<El poder de este reino no consistirá en la fuerza de los ejércitos ni en la importancia de las riquezas, sino más bien en la gloria del espíritu divino que vendrá a enseñar la mente y dirigir el corazón de los ciudadanos renacidos de este reino celestial ---los hijos de Dios. Ésta es la fraternidad del amor donde reina la rectitud y cuyo grito de guerra será: Paz en la Tierra y buena voluntad entre todos los hombres. Este reino, que muy pronto vais a proclamar, es el deseo de los hombres de bien de todos los tiempos, la esperanza de toda la Tierra y el cumplimiento de las sabias promesas de todos los profetas>>\footnote{\textit{Paz en la Tierra}: Lc 2:14.}.

\par 
%\textsuperscript{(1569.1)}
\textsuperscript{140:1.4} <<Pero para vosotros, hijos míos, y para todos los demás que quieran seguiros en este reino, una dura prueba se prepara. Sólo la fe os permitirá atravesar sus puertas, pero tendréis que producir los frutos del espíritu de mi Padre si queréis continuar ascendiendo en la vida progresiva de la comunidad divina. En verdad, en verdad os digo que no todo el que dice `Señor, Señor' entrará en el reino de los cielos, sino más bien aquel que hace la voluntad de mi Padre que está en los cielos>>\footnote{\textit{Sólo la fe os atravesará la puerta}: Ro 1:17; Ro 3:28; Ef 2:8. \textit{Producir los frutos del espíritu para continuar}: Gl 5:22-23; Ef 5:9; Stg 2:17,20,26. \textit{El que hace la voluntad de mi Padre}: Mt 7:21.}.

\par 
%\textsuperscript{(1569.2)}
\textsuperscript{140:1.5} <<Vuestro mensaje para el mundo será: Buscad primero el reino de Dios y su rectitud, y cuando los hayáis encontrado, todas las demás cosas esenciales para la supervivencia eterna estarán aseguradas por añadidura. Ahora quisiera dejar claro para vosotros que este reino de mi Padre no vendrá con una exhibición exterior de poder ni con una demostración indecorosa. No debéis salir de aquí para proclamar el reino diciendo: `está aquí' o `está allí', porque este reino que predicaréis es Dios dentro de vosotros>>\footnote{\textit{Buscad primero el reino}: Mt 6:33; Lc 12:31. \textit{No con una demostración indecorosa}: Mt 24:30-31; Mc 13:26; Lc 21:21-27; Hch 1:7-8. \textit{El reino es Dios dentro de vosotros}: Job 32:8,18; Is 63:10-11; Ez 37:14; Mt 10:29; Lc 17:21; Jn 17:21-23; Ro 8:9-11; 1 Co 3:16-17; 6:19; 2 Co 6:16; Gl 2:20; 1 Jn 3:24; 4:12-15; Ap 21:3.}.

\par 
%\textsuperscript{(1569.3)}
\textsuperscript{140:1.6} <<Quien quiera ser grande en el reino de mi Padre, deberá volverse un ministro para todos; y si alguien quiere ser el primero entre vosotros, que se convierta en el servidor de sus hermanos. Una vez que hayáis sido recibidos realmente como ciudadanos del reino celestial, ya no seréis servidores, sino hijos, hijos del Dios viviente. Así es como este reino progresará en el mundo, hasta que destruya todas las barreras y conduzca a todos los hombres a conocer a mi Padre y a creer en la verdad salvadora que he venido a proclamar. Incluso ahora mismo el reino está cerca, y algunos de vosotros no moriréis hasta que hayáis visto llegar el reino de Dios con gran poder>>\footnote{\textit{Ser grande por el ministerio}: Mt 20:26-27; 23:11; Mc 9:35; 10:43-44; Lc 22:26. \textit{Hijos del Dios viviente}: 1 Cr 22:10; Sal 2:7; Is 56:5; Mt 5:9,16,45; Lc 20:36; Jn 1:12-13; 11:52; Hch 17:28-29; Ro 8:14-17,19,21; 9:26; 2 Co 6:18; Gl 3:26; 4:5-7; Ef 1:5; Flp 2:15; Heb 12:5-8; 1 Jn 3:1-2,10; 5:2; Ap 21:7; 2 Sam 7:14. \textit{El reino está cerca}: Mt 3:2; 4:17; 10:7; Mc 1:15; Lc 10:9-11; 17:20-21; 21:31. \textit{El reino de Dios con gran poder}: Hch 2:1-4.}.

\par 
%\textsuperscript{(1569.4)}
\textsuperscript{140:1.7} <<Esto que vuestros ojos contemplan ahora, este pequeño comienzo de doce hombres comunes, se multiplicará y crecerá hasta que, finalmente, toda la Tierra se llene con las alabanzas de mi Padre. Y no será tanto por las palabras que diréis, sino más bien por la vida que viviréis, como los hombres sabrán que habéis estado conmigo y que habéis aprendido las realidades del reino. Aunque no quisiera colocar ninguna carga pesada sobre vuestra mente, estoy a punto de depositar sobre vuestra alma la solemne responsabilidad de representarme en el mundo cuando os deje dentro de poco, como yo represento ahora a mi Padre en esta vida que estoy viviendo en la carne>>. Cuando Jesús terminó de hablar, se levantó.

\section*{2. La ordenación}
\par 
%\textsuperscript{(1569.5)}
\textsuperscript{140:2.1} Jesús indicó entonces a los doce mortales que acababan de escuchar su declaración sobre el reino que se arrodillaran en círculo alrededor de él. Luego, el Maestro puso sus manos sobre la cabeza de cada apóstol, empezando por Judas Iscariote y terminando por Andrés. Después de haberlos bendecido, extendió las manos y oró:

\par 
%\textsuperscript{(1569.6)}
\textsuperscript{140:2.2} <<Padre mío, aquí te traigo a estos hombres, mis mensajeros. Entre nuestros hijos de la Tierra, he escogido a estos doce para que vayan a representarme como yo he venido para representarte. Ámalos y acompáñalos como tú me has amado y acompañado. Y ahora, Padre mío, concédeles sabiduría a estos hombres, mientras deposito todos los asuntos del reino venidero entre sus manos. Desearía, si es tu voluntad, permanecer algún tiempo en la Tierra para ayudarlos en sus trabajos por el reino. De nuevo, Padre mío, te doy las gracias por estos hombres, y los confío a tu cuidado mientras me dedico a terminar la obra que me has encomendado>>.

\par 
%\textsuperscript{(1570.1)}
\textsuperscript{140:2.3} Cuando Jesús terminó de orar, cada uno de los apóstoles permaneció inclinado en su sitio. Transcurrieron muchos minutos antes de que el mismo Pedro se atreviera a levantar los ojos para mirar al Maestro. Uno tras otro abrazaron a Jesús, pero nadie dijo nada. Un gran silencio invadió el lugar, mientras que una multitud de seres celestiales contemplaba desde arriba esta escena solemne y sagrada ---el Creador de un universo poniendo los asuntos de la fraternidad divina de los hombres bajo la dirección de unas mentes humanas.

\section*{3. El sermón de ordenación}
\par 
%\textsuperscript{(1570.2)}
\textsuperscript{140:3.1} Jesús tomó entonces la palabra y dijo: <<Ahora que sois embajadores del reino de mi Padre, os habéis convertido así en una clase de hombres separada y distinta de todos los demás hombres de la Tierra. Ahora ya no sois como unos hombres entre los hombres, sino como unos ciudadanos iluminados de otro país celestial entre las criaturas ignorantes de este mundo tenebroso. Ya no es suficiente con que viváis como habéis hecho hasta ahora, sino que de aquí en adelante deberéis de vivir como aquellos que han saboreado las glorias de una vida mejor, y han sido enviados de vuelta a la Tierra como embajadores del Soberano de ese mundo nuevo y mejor. Se espera más del profesor que del alumno; al amo se le exige más que al servidor. A los ciudadanos del reino celestial se les pide más que a los ciudadanos del gobierno terrestre. Algunas de las cosas que estoy a punto de deciros os parecerán duras, pero habéis elegido representarme en el mundo como yo ahora represento al Padre. Y como agentes míos en la Tierra, estaréis obligados a acatar las enseñanzas y las prácticas que reflejan mis ideales de vida mortal en los mundos del espacio, lo que ejemplifico en mi vida terrestre revelando al Padre que está en los cielos>>\footnote{\textit{Jesús comienza el sermón de la ordenación}: Mt 5:2; Lc 6:20. \textit{Sois embajadores}: 2 Co 5:20. \textit{Ciudadanos del cielo}: Heb 11:16. \textit{Se te pedirá más}: Lc 12:48.}.

\par 
%\textsuperscript{(1570.3)}
\textsuperscript{140:3.2} <<Os envío para proclamar la libertad a los cautivos espirituales, la alegría a los esclavos del miedo, y para curar a los enfermos de acuerdo con la voluntad de mi Padre que está en los cielos. Cuando encontréis a mis hijos en la aflicción, decidles palabras de estímulo como éstas:>>\footnote{\textit{Liberar a los cautivos espirituales}: Is 61:1; Lc 4:18.}

\par 
%\textsuperscript{(1570.4)}
\textsuperscript{140:3.3} <<Bienaventurados los pobres de espíritu, los humildes, porque de ellos son los tesoros del reino de los cielos>>\footnote{\textit{Bienaventurados los pobres de espíritu}: Mt 5:3; Lc 6:20b.}.

\par 
%\textsuperscript{(1570.5)}
\textsuperscript{140:3.4} <<Bienaventurados los que tienen hambre y sed de rectitud, porque ellos serán saciados>>\footnote{\textit{Bienaventurados los que tienen hambre y sed de rectitud}: Mt 5:6; Lc 6:21a.}.

\par 
%\textsuperscript{(1570.6)}
\textsuperscript{140:3.5} <<Bienaventurados los mansos, porque ellos heredarán la Tierra>>\footnote{\textit{Bienaventurados los mansos}: Mt 5:5.}.

\par 
%\textsuperscript{(1570.7)}
\textsuperscript{140:3.6} <<Bienaventurados los limpios de corazón, porque ellos verán a Dios>>\footnote{\textit{Bienaventurados los limpios de corazón}: Mt 5:8.}.

\par 
%\textsuperscript{(1570.8)}
\textsuperscript{140:3.7} <<Y decid también a mis hijos estas palabras adicionales de consuelo espiritual y de promesa:>>

\par 
%\textsuperscript{(1570.9)}
\textsuperscript{140:3.8} <<Bienaventurados los afligidos, porque ellos serán consolados. Bienaventurados los que lloran, porque ellos recibirán el espíritu de la alegría>>\footnote{\textit{Bienaventurados los afligidos}: Mt 5:4. \textit{Bienaventurados los que lloran}: Lc 6:21b.}.

\par 
%\textsuperscript{(1570.10)}
\textsuperscript{140:3.9} <<Bienaventurados los misericordiosos, porque ellos alcanzarán misericordia>>\footnote{\textit{Bienaventurados los misericordiosos}: Mt 5:7.}.

\par 
%\textsuperscript{(1570.11)}
\textsuperscript{140:3.10} <<Bienaventurados los pacificadores, porque ellos serán llamados hijos de Dios>>\footnote{\textit{Bienaventurados los pacificadores}: Mt 5:9.}.

\par 
%\textsuperscript{(1570.12)}
\textsuperscript{140:3.11} <<Bienaventurados los perseguidos a causa de su rectitud, porque de ellos es el reino de los cielos. Consideraos bienaventurados cuando los hombres os injurien y os persigan, y digan falsamente toda clase de mal contra vosotros. Regocijaos y alegraos en extremo, porque vuestra recompensa será grande en los cielos>>\footnote{\textit{Bienaventurados los perseguidos}: Mt 5:10-12; Lc 6:22-23a.}.

\par 
%\textsuperscript{(1570.13)}
\textsuperscript{140:3.12} <<Hermanos míos, mientras os envío fuera, vosotros sois la sal de la Tierra, una sal con sabor de salvación. Pero si esta sal ha perdido su sabor, ¿con qué se sazonará? En lo sucesivo ya no sirve más que para ser arrojada y pisoteada por los hombres>>\footnote{\textit{Sois la sal de la Tierra}: Mt 5:13; Mc 9:50; Lc 14:34-35.}.

\par 
%\textsuperscript{(1570.14)}
\textsuperscript{140:3.13} <<Vosotros sois la luz del mundo. Una ciudad situada en una colina no se puede ocultar. Los hombres tampoco encienden una luz para ponerla debajo de un almud, sino en un candelero; y da luz a todos los que están en la casa. Que vuestra luz brille ante los hombres de tal manera que puedan ver vuestras buenas obras y sean inducidos a glorificar a vuestro Padre que está en los cielos>>\footnote{\textit{Vosotros sois la luz del mundo}: Mt 5:14-16; Mc 4:21; Lc 8:16; 11:33.}.

\par 
%\textsuperscript{(1571.1)}
\textsuperscript{140:3.14} <<Os envío al mundo para que me representéis y actuéis como embajadores del reino de mi Padre. Cuando salgáis a proclamar la buena nueva, poned vuestra confianza en el Padre, de quien sois mensajeros. No resistáis a la injusticia por medio de la fuerza; no pongáis vuestra confianza en el vigor corporal. Si vuestro prójimo os golpea en la mejilla derecha, ofrecedle también la izquierda. Estad dispuestos a sufrir una injusticia en lugar de acudir a la ley entre vosotros. Atended con bondad y misericordia a todos los que están afligidos y necesitados>>\footnote{\textit{Embajadores del reino}: 2 Co 5:20. \textit{Estad dispuestos a la injusticia}: Mt 5:39; Lc 6:29. \textit{No litiguéis unos con otros}: 1 Co 6:1-7.}.

\par 
%\textsuperscript{(1571.2)}
\textsuperscript{140:3.15} <<Os lo digo: amad a vuestros enemigos, haced el bien a los que os odian, bendecid a los que os maldicen, y orad por los que os utilizan con malicia. Haced por los hombres todo lo que creéis que yo haría por ellos>>\footnote{\textit{Ama a tus enemigos}: Mt 5:44; Lc 6:27-28. \textit{La regla de oro}: Mt 7:12; Lc 6:31.}.

\par 
%\textsuperscript{(1571.3)}
\textsuperscript{140:3.16} <<Vuestro Padre que está en los cielos hace que el Sol brille sobre los malos al igual que sobre los buenos; asimismo, envía la lluvia sobre los justos y los injustos. Vosotros sois los hijos de Dios; aún más, ahora sois los embajadores del reino de mi Padre. Sed misericordiosos como Dios es misericordioso, y en el eterno futuro del reino, seréis perfectos como vuestro Padre celestial es perfecto>>\footnote{\textit{El Padre trae el sol y la lluvia}: Mt 5:45. \textit{Sed misericordiosos como Dios lo es}: Lc 6:36. \textit{Sed perfectos}: Gn 17:1; 1 Re 8:61; Lv 19:2; Dt 18:13; Mt 5:48; 2 Co 13:11; Stg 1:4; 1 P 1:16.}.

\par 
%\textsuperscript{(1571.4)}
\textsuperscript{140:3.17} <<Estáis encargados de salvar a los hombres, no de juzgarlos. Al final de vuestra vida terrestre, todos esperaréis misericordia; por eso os pido que durante vuestra vida mortal mostréis misericordia a todos vuestros hermanos en la carne. No cometáis el error de intentar quitar una mota del ojo de vuestro hermano, cuando hay una viga en el vuestro. Después de sacar primero la viga de vuestro propio ojo, podréis ver mejor para quitar la mota del ojo de vuestro hermano>>\footnote{\textit{Salvad a los hombres, no los juzguéis}: Mt 7:1-2; Lc 6:37. \textit{La mota en el ojo ajeno y la viga en el propio}: Mt 7:3-5; Lc 6:41-42.}.

\par 
%\textsuperscript{(1571.5)}
\textsuperscript{140:3.18} <<Discernid claramente la verdad; vivid con audacia la vida recta; así seréis mis apóstoles y los embajadores de mi Padre. Habéis oído decir que: `Si el ciego conduce al ciego, los dos se caerán al precipicio'. Si queréis guiar a otras personas hacia el reino, vosotros mismos tenéis que caminar en la clara luz de la verdad viviente. En todos los asuntos del reino, os exhorto a que mostréis un juicio justo y una sabiduría penetrante. No ofrezcáis las cosas santas a los perros, ni arrojéis vuestras perlas delante de los cerdos, no sea que pisoteen vuestras joyas y se vuelvan para despedazaros>>\footnote{\textit{Si el ciego conduce al ciego}: Mt 15:14; Lc 6:39. \textit{No arrojéis vuestras perlas a los cerdos}: Mt 7:6.}.

\par 
%\textsuperscript{(1571.6)}
\textsuperscript{140:3.19} <<Os pongo en guardia contra los falsos profetas que vendrán hacia vosotros vestidos de cordero, mientras que por dentro son como lobos voraces. Por sus frutos los conoceréis. ¿Recogen los hombres uvas de las espinas o higos de los cardos? Así pues, todo buen árbol produce buenos frutos, pero el árbol corrompido da malos frutos. Un buen árbol no puede producir malos frutos, ni un árbol corrompido buenos frutos. Todo árbol que no da buenos frutos pronto es derribado y arrojado al fuego. Para conseguir entrar en el reino de los cielos, lo que cuenta es el móvil. Mi Padre mira dentro del corazón de los hombres y los juzga por sus deseos internos y sus intenciones sinceras>>\footnote{\textit{Advertencia contra los falsos profetas}: Mt 7:15,22-23; Mt 24:11,24; Mc 13:22; 1 Jn 4:1. \textit{Por sus frutos los conoceréis}: Mt 7:16-20; Lc 6:43-44. \textit{Frutos del espíritu}: Gl 5:22-23; Ef 5:9. \textit{Requerimientos para dar buenos frutos}: Mt 3:10; Mt 12:33; Lc 3:9; Lc 6:43-44; Lc 13:6,9; Jn 15:7-8,16. \textit{Dios juzga el corazón}: Lc 16:15; Hch 1:24; 1 Ts 2:4.}.

\par 
%\textsuperscript{(1571.7)}
\textsuperscript{140:3.20} <<En el gran día del juicio del reino, muchos me dirán: `¿No hemos profetizado en tu nombre y hemos hecho muchas obras maravillosas por tu nombre?' Pero yo me veré obligado a decirles, `Nunca os he conocido; apartaos de mí, vosotros que sois unos falsos educadores'. Pero todo el que escuche esta instrucción y ejecute sinceramente su misión de representarme ante los hombres, como yo he representado a mi Padre ante vosotros, encontrará una entrada abundante a mi servicio y en el reino del Padre celestial>>\footnote{\textit{Falsas presunciones de fe}: Mt 7:22-23. \textit{Recompensa de los creyentes}: Mt 7:24; Lc 6:47.}.

\par 
%\textsuperscript{(1571.8)}
\textsuperscript{140:3.21} Los apóstoles nunca habían escuchado antes a Jesús expresarse de esta manera, pues les había hablado como alguien que posee una autoridad suprema\footnote{\textit{Jesús hablaba con autoridad}: Mt 7:28-29; Mc 1:22.}. Descendieron de la montaña casi al ponerse el Sol, pero ninguno le hizo preguntas a Jesús.

\section*{4. Vosotros sois la sal de la Tierra}
\par 
%\textsuperscript{(1572.1)}
\textsuperscript{140:4.1} El llamado <<Sermón de la Montaña>> no es el evangelio de Jesús. Contiene de hecho muchas enseñanzas útiles, pero eran las instrucciones de ordenación de Jesús a los doce apóstoles. Era el encargo personal del Maestro a los que iban a continuar predicando el evangelio y que aspiraban a representarlo en el mundo de los hombres, como él representaba a su Padre con tanta elocuencia y perfección.

\par 
%\textsuperscript{(1572.2)}
\textsuperscript{140:4.2} \textit{<<Vosotros sois la sal de la Tierra, una sal con sabor de salvación. Pero si esta sal ha perdido su sabor, ¿con qué se sazonará? En lo sucesivo ya no sirve más que para ser arrojada y pisoteada por los hombres>>.}

\par 
%\textsuperscript{(1572.3)}
\textsuperscript{140:4.3} En los tiempos de Jesús, la sal era un elemento precioso. Se utilizaba incluso como moneda. La palabra moderna <<salario>> se deriva de sal. La sal no sólo condimenta los alimentos, sino que también los conserva. Hace que otras cosas sean más sabrosas, y sirve así a medida que se gasta.

\par 
%\textsuperscript{(1572.4)}
\textsuperscript{140:4.4} \textit{<<Vosotros sois la luz del mundo. Una ciudad situada en una colina no se puede ocultar. Los hombres tampoco encienden una luz para ponerla debajo de un almud, sino en un candelero; y da luz a todos los que están en la casa. Que vuestra luz brille ante los hombres de tal manera que puedan ver vuestras buenas obras y sean inducidos a glorificar a vuestro Padre que está en los cielos>>.}\footnote{\textit{Vosotros sois la luz del mundo}: Mt 5:14-16; Mc 4:21; Lc 8:16; 11:33.}

\par 
%\textsuperscript{(1572.5)}
\textsuperscript{140:4.5} Aunque la luz disipa las tinieblas, también puede ser tan <<cegadora>> como para confundir y frustrar. Se nos exhorta a que dejemos brillar nuestra luz \textit{de tal manera} que nuestros semejantes se sientan guiados hacia unos caminos nuevos y divinos de vida realzada. Nuestra luz no debe brillar como para atraer la atención sobre nosotros mismos. También podemos utilizar nuestra propia profesión como un <<reflector>> eficaz para diseminar esta luz de la vida.

\par 
%\textsuperscript{(1572.6)}
\textsuperscript{140:4.6} Los caracteres fuertes no se forman \textit{evitando} hacer el mal, sino más bien haciendo realmente el bien. El altruismo es la insignia de la grandeza humana. Los niveles más altos de autorrealización se alcanzan mediante la adoración y el servicio. La persona feliz y eficaz está motivada por el amor de hacer el bien, y no por el temor de hacer el mal.

\par 
%\textsuperscript{(1572.7)}
\textsuperscript{140:4.7} \textit{<<Por sus frutos los conoceréis>>.}\footnote{\textit{Por sus frutos los conoceréis}: Mt 7:16-20; Lc 6:43-44. \textit{Frutos del espíritu}: Gl 5:22-23; Ef 5:9.} La personalidad es básicamente invariable. Lo que cambia ---lo que crece--- es el carácter moral. El error principal de las religiones modernas es el negativismo. El árbol que no produce frutos es <<derribado y arrojado al fuego>>\footnote{\textit{El árbol estéril será talado}: Mt 7:19.}. El valor moral no puede provenir de la simple represión ---de la obediencia al mandato <<No harás>>. El miedo y la verg\"uenza son motivaciones sin valor para la vida religiosa. La religión solamente es válida cuando revela la paternidad de Dios y realza la fraternidad de los hombres.

\par 
%\textsuperscript{(1572.8)}
\textsuperscript{140:4.8} Una persona se forma una filosofía eficaz de la vida combinando la perspicacia cósmica con la suma de sus propias reacciones emocionales ante el entorno social y económico. Recordad: aunque los impulsos hereditarios no se pueden modificar fundamentalmente, las reacciones emocionales a esos impulsos sí se pueden cambiar; por consiguiente, la naturaleza moral se puede modificar, el carácter se puede mejorar. En un carácter fuerte, las reacciones emocionales están integradas y coordinadas, generando así una personalidad unificada. La falta de unificación debilita la naturaleza moral y engendra la desdicha.

\par 
%\textsuperscript{(1572.9)}
\textsuperscript{140:4.9} Sin una meta que merezca la pena, la vida carece de objetivo y de provecho, lo que ocasiona mucha infelicidad. El discurso de Jesús en la ordenación de los doce constituye una filosofía magistral de la vida. Jesús exhortó a sus seguidores a que ejercitaran una fe experiencial. Les advirtió que no se limitaran a depender de un asentimiento intelectual, de la credulidad o de la autoridad establecida.

\par 
%\textsuperscript{(1573.1)}
\textsuperscript{140:4.10} La educación debería ser una técnica para aprender (para descubrir) los mejores métodos de satisfacer nuestros impulsos naturales y hereditarios, y la felicidad es el resultado final de estas técnicas mejores de satisfacción emocional. La felicidad depende poco del entorno, aunque un ambiente agradable puede contribuir mucho a ella.

\par 
%\textsuperscript{(1573.2)}
\textsuperscript{140:4.11} Todo mortal ansía realmente ser una persona completa, ser perfecto\footnote{\textit{Sed perfectos}: Gn 17:1; 1 Re 8:61; Lv 19:2; Dt 18:13; Mt 5:48; 2 Co 13:11; Stg 1:4; 1 P 1:16.} como el Padre que está en los cielos es perfecto, y este logro es posible porque, a fin de cuentas, el <<universo es verdaderamente paternal>>.

\section*{5. Amor paternal y amor fraternal}
\par 
%\textsuperscript{(1573.3)}
\textsuperscript{140:5.1} Desde el Sermón de la Montaña hasta el discurso de la
Última Cena, Jesús enseñó a sus discípulos a manifestar un amor \textit{paternal}\footnote{\textit{Amor paternal}: Jn 14:21,23; 15:9-12; 17:23,26; 1 Jn 3:1.} en lugar de un amor \textit{fraternal}. El amor fraternal consiste en amar al prójimo como a sí mismo, lo que sería una aplicación adecuada de la <<regla de oro>>. Pero el afecto paternal exige que améis a vuestros compañeros mortales como Jesús os ama.

\par 
%\textsuperscript{(1573.4)}
\textsuperscript{140:5.2} Jesús ama a la humanidad con un afecto doble. Vivió en la Tierra bajo una doble personalidad ---humana y divina. Como Hijo de Dios, ama al hombre con un amor paternal ---es el Creador del hombre, su Padre en el universo. Como Hijo del Hombre, Jesús ama a los mortales como un hermano ---fue realmente un hombre entre los hombres.

\par 
%\textsuperscript{(1573.5)}
\textsuperscript{140:5.3} Jesús no esperaba que sus discípulos consiguieran una manifestación imposible de amor fraternal, pero sí contaba con que se esforzarían tanto por parecerse a Dios ---por ser perfectos como el Padre que está en los cielos es perfecto--- que podrían empezar a considerar a los hombres como Dios considera a sus criaturas, y así podrían empezar a amar a los hombres como Dios los ama ---a manifestar los principios de un afecto paternal. En el transcurso de estas exhortaciones a los doce apóstoles, Jesús trató de revelar este nuevo concepto de \textit{amor paternal}, tal como está relacionado con ciertas actitudes emocionales involucradas cuando se efectúan numerosos ajustes sociales al entorno.

\par 
%\textsuperscript{(1573.6)}
\textsuperscript{140:5.4} El Maestro inició este importante discurso llamando la atención sobre cuatro actitudes de \textit{fe}, como preludio a la descripción posterior de sus cuatro reacciones trascendentales y supremas de amor paternal, en contraste con las limitaciones del simple amor fraternal.

\par 
%\textsuperscript{(1573.7)}
\textsuperscript{140:5.5} Primero habló de los que eran pobres de espíritu, de los que tenían hambre de rectitud, de los que perseveraban en la mansedumbre y de los limpios de corazón. Se podría esperar que estos mortales que disciernen el espíritu alcanzarían los niveles suficientes de desinterés divino como para ser capaces de intentar el extraordinario ejercicio del afecto \textit{paternal;} que, incluso en la aflicción, estarían facultados para mostrar misericordia, promover la paz y soportar las persecuciones. Y que a lo largo de todas estas penosas situaciones, amarían con un amor paternal incluso a una humanidad poco amable. El afecto de un padre puede alcanzar unos niveles de devoción que trascienden inmensamente el afecto de un hermano.

\par 
%\textsuperscript{(1573.8)}
\textsuperscript{140:5.6} La fe y el amor de estas beatitudes fortalecen el carácter moral y crean la felicidad. El miedo y la ira debilitan el carácter y destruyen la felicidad. Este sermón importante se inició con una nota de felicidad.

\par 
%\textsuperscript{(1573.9)}
\textsuperscript{140:5.7} 1. \textit{<<Bienaventurados los pobres de espíritu ---los humildes>>}\footnote{\textit{Los pobres de espíritu, los humildes}: Mt 5:3; Lc 6:20b.}. Para un niño, la felicidad es la satisfacción de una ansia inmediata de placer. El adulto está dispuesto a sembrar las semillas de la abnegación, con el fin de obtener las cosechas posteriores de una felicidad mayor. En los tiempos de Jesús y después de ellos, la felicidad ha sido asociada demasiado a menudo con la idea de poseer riquezas. En la historia del fariseo y del publicano que oraban en el templo\footnote{\textit{La oración del publicano}: Lc 18:10-14.}, uno se sentía rico de espíritu ---egotista; el otro se sentía <<pobre de espíritu>> ---humilde. Uno era autosuficiente; el otro era enseñable y buscaba la verdad. Los pobres de espíritu buscan metas de riqueza espiritual ---buscan a Dios. Estos buscadores de la verdad no tienen que esperar sus recompensas en un futuro lejano; son recompensados \textit{ahora}. Encuentran el reino de los cielos en su propio corazón, y experimentan esa felicidad \textit{ahora}.

\par 
%\textsuperscript{(1574.1)}
\textsuperscript{140:5.8} 2. \textit{<<Bienaventurados los que tienen hambre y sed de rectitud, porque ellos serán saciados>>}\footnote{\textit{Los hambrientos y sedientos de rectitud}: Mt 5:6; Lc 6:21a.}. Sólo aquellos que se sienten pobres de espíritu tienen sed de rectitud. Sólo los humildes buscan la fuerza divina y anhelan el poder espiritual. Sin embargo, es sumamente peligroso practicar a sabiendas el ayuno espiritual con el fin de aumentar nuestro apetito de los dones espirituales. El ayuno físico se vuelve peligroso después de cuatro o cinco días; uno puede perder todo deseo de alimentarse. El ayuno prolongado, tanto físico como espiritual, tiende a destruir el apetito.

\par 
%\textsuperscript{(1574.2)}
\textsuperscript{140:5.9} La rectitud experiencial es un placer, no un deber. La rectitud de Jesús es un amor dinámico ---un afecto paterno-fraternal. No es una rectitud negativa del tipo <<no harás>>. ¿Cómo podría alguien tener hambre de algo negativo ---de algo a <<no hacer>>?

\par 
%\textsuperscript{(1574.3)}
\textsuperscript{140:5.10} No es fácil enseñar estas dos primeras beatitudes a una mente infantil, pero la mente madura debería captar su significado.

\par 
%\textsuperscript{(1574.4)}
\textsuperscript{140:5.11} 3. \textit{<<Bienaventurados los mansos, porque ellos heredarán la Tierra>>}\footnote{\textit{Los mansos heredarán la Tierra}: Mt 5:5.}. La mansedumbre auténtica no tiene ninguna relación con el miedo. Es más bien una actitud del hombre cooperando con Dios ---<<Hágase tu voluntad>>\footnote{\textit{Hágase tu voluntad}: Mt 6:10; 26:39,42,44; Mc 14:36,39; Lc 11:2; 22:42.}. Engloba la paciencia y la indulgencia, y está motivada por una fe imperturbable en un universo justo y amistoso. Domina todas las tentaciones de rebelarse contra el gobierno divino. Jesús fue el hombre manso ideal de Urantia, y heredó un vasto universo.

\par 
%\textsuperscript{(1574.5)}
\textsuperscript{140:5.12} 4. \textit{<<Bienaventurados los limpios de corazón, porque ellos verán a Dios>>}\footnote{\textit{Los puros de corazón}: Mt 5:8.}. La pureza espiritual no es una cualidad negativa, salvo que carece de recelo y de revancha. Al hablar de la pureza, Jesús no tenía la intención de tratar exclusivamente de las actitudes sexuales humanas. Se refería más bien a esa fe que los hombres deberían tener en sus semejantes; a esa fe que los padres tienen en sus hijos, y que les permite amar a sus semejantes como un padre los amaría. El amor de un padre no tiene necesidad de mimar, y no perdona el mal, pero siempre se opone al cinismo. El amor paternal tiene una única finalidad, y siempre busca lo mejor que hay en el hombre; ésta es la actitud de un verdadero padre.

\par 
%\textsuperscript{(1574.6)}
\textsuperscript{140:5.13} Ver a Dios ---por la fe--- significa adquirir la verdadera perspicacia espiritual. La perspicacia espiritual intensifica el gobierno del Ajustador, y los dos reunidos terminan por aumentar la conciencia de Dios. Cuando conocéis al Padre, os sentís confirmados en la seguridad de vuestra filiación divina, y podéis amar cada vez más a vuestros hermanos en la carne, no sólo como un hermano ---con un amor fraternal--- sino también como un padre ---con un afecto paternal\footnote{\textit{Amor paternal}: Jn 3:16; Jn 17:23,26; 1 Jn 3:1,16; 1 Jn 4:9-11,19.}.

\par 
%\textsuperscript{(1574.7)}
\textsuperscript{140:5.14} Esta exhortación es fácil de enseñar incluso a un niño. Los niños son confiados por naturaleza, y los padres deberían cuidar de que no pierdan esta fe sencilla. Al tratar con los niños, evitad todo engaño y absteneos de sugerir la desconfianza. Ayudadlos juiciosamente a escoger a sus héroes y a seleccionar el trabajo de su vida.

\par 
%\textsuperscript{(1574.8)}
\textsuperscript{140:5.15} Luego, Jesús continuó instruyendo a sus discípulos sobre cómo conseguir el objetivo principal de todas las luchas humanas ---la perfección--- e incluso la consecución divina. Siempre les recomendaba: <<Sed perfectos como vuestro Padre que está en los cielos es perfecto>>\footnote{\textit{Sed perfectos}: Gn 17:1; 1 Re 8:61; Lv 19:2; Dt 18:13; Mt 5:48; 2 Co 13:11; Stg 1:4; 1 P 1:16.}. No exhortaba a los doce a que amaran al prójimo como se amaban a sí mismos. Esto hubiera sido un logro meritorio, que hubiera indicado la realización del amor fraternal. Recomendaba más bien a sus apóstoles que amaran a los hombres como él los había amado ---con un afecto \textit{paternal} así como fraternal. Y esto lo ilustró indicando cuatro reacciones supremas de amor paternal:

\par 
%\textsuperscript{(1575.1)}
\textsuperscript{140:5.16} 1. \textit{<<Bienaventurados los afligidos, porque ellos serán consolados>>}\footnote{\textit{Los afligidos serán confortados}: Mt 5:4; Lc 6:21b.}. El llamado sentido común o la lógica más superior nunca sugerirían que la felicidad puede surgir de la aflicción. Pero Jesús no se refería a la aflicción externa u ostentatoria. Hacía alusión a una actitud emotiva de ternura de corazón. Es un gran error enseñar a los niños y a los jóvenes que no es varonil mostrar ternura\footnote{\textit{La ternura es varonil}: Mc 14:72; Lc 7:32; Lc 19:41; Lc 22:62; Jn 11:35; Hch 20:37; Ap 5:4.} o, por otra parte, dar testimonio de sentimientos emotivos o de sufrimientos físicos. La compasión es un atributo valioso tanto en el hombre como en la mujer. No es necesario ser insensible para ser varonil. Ésta es la manera equivocada de crear hombres valientes. Los grandes hombres de este mundo no han tenido miedo de afligirse. Moisés, el afligido\footnote{\textit{Moisés, el afligido}: Nm 12:3.}, fue un hombre más grande que Sansón o Goliat. Moisés fue un guía extraordinario, pero también estaba lleno de mansedumbre. Ser sensible y reaccionar antes las necesidades humanas crea una felicidad auténtica y duradera, y al mismo tiempo estas actitudes benévolas protegen el alma contra las influencias destructivas de la ira, el odio y la desconfianza.

\par 
%\textsuperscript{(1575.2)}
\textsuperscript{140:5.17} 2. \textit{<<Bienaventurados los misericordiosos, porque ellos conseguirán misericordia>>}\footnote{\textit{Los misericordiosos obtendrán misericordia}: Mt 5:7.}. La misericordia denota aquí la altura, la profundidad y la anchura de la amistad más sincera ---la bondad. A veces, la misericordia puede ser pasiva, pero aquí es activa y dinámica--- la ternura paternal suprema. Un padre amoroso tiene pocas dificultades para perdonar a su hijo, incluso muchas veces. En un niño no mimado, el impulso de aliviar el sufrimiento es natural. Los niños son normalmente bondadosos y compasivos cuando tienen la edad suficiente para apreciar las situaciones reales.

\par 
%\textsuperscript{(1575.3)}
\textsuperscript{140:5.18} 3. \textit{<<Bienaventurados los pacificadores, porque ellos serán llamados hijos de Dios>>}\footnote{\textit{Los pacificadores serán llamados hijos de Dios}: Mt 5:9.}. Los oyentes de Jesús deseaban ardientemente una liberación militar, no unos pacificadores. Pero la paz de Jesús\footnote{\textit{La paz de Jesús}: Jn 14:27a.} no es de tipo pacífico y negativo. En presencia de las pruebas y de las persecuciones, decía: <<Mi paz os dejo>>. <<Que vuestro corazón no se perturbe, y no tengáis miedo>>\footnote{\textit{La paz cura el corazón perturbado}: Jn 14:1. \textit{No tengáis miedo}: Jn 14:27b.}. Ésta es la paz que impide los conflictos ruinosos. La paz personal integra la personalidad. La paz social impide el miedo, la codicia y la ira. La paz política impide los antagonismos raciales, las desconfianzas nacionales y la guerra. La pacificación es el remedio para la desconfianza y la sospecha.

\par 
%\textsuperscript{(1575.4)}
\textsuperscript{140:5.19} Es fácil enseñar a los niños a trabajar como pacificadores. Disfrutan con las actividades de equipo; les gusta jugar juntos. El Maestro dijo en otra ocasión: <<Quien quiera salvar su vida la perderá, pero quien esté dispuesto a perderla, la encontrará>>\footnote{\textit{Quien quiera salvar su vida, la perderá}: Mt 10:39; 16:25; Mc 8:35; 9:24; Lc 17:33; Jn 12:25.}.

\par 
%\textsuperscript{(1575.5)}
\textsuperscript{140:5.20} 4. \textit{<<Bienaventurados los perseguidos a causa de su rectitud, porque de ellos es el reino de los cielos. Consideraos bienaventurados cuando los hombres os injurien y os persigan, y digan falsamente toda clase de mal contra vosotros. Regocijaos y alegraos en extremo, porque vuestra recompensa será grande en los cielos>>}\footnote{\textit{Los perseguidos por su rectitud}: Mt 5:10-12a; Lc 6:22-23a.}.

\par 
%\textsuperscript{(1575.6)}
\textsuperscript{140:5.21} Muy a menudo, la persecución sigue de hecho a la paz. Pero los jóvenes y los adultos valientes no huyen nunca de las dificultades o del peligro. <<No existe un amor más grande que el de dar la vida por sus amigos>>\footnote{\textit{No existe un amor más grande}: Jn 15:13.}. Un amor paternal puede hacer libremente todas estas cosas ---unas cosas que el amor fraternal difícilmente puede abarcar. El progreso ha sido siempre la cosecha final de la persecución.

\par 
%\textsuperscript{(1575.7)}
\textsuperscript{140:5.22} Los niños responden siempre al desafío de la valentía. La juventud siempre está dispuesta a <<aceptar un desafío>>. Todos los niños deberían aprender pronto a sacrificarse.

\par 
%\textsuperscript{(1575.8)}
\textsuperscript{140:5.23} Se descubre pues que las bienaventuranzas del Sermón de la Montaña están basadas en la fe y el amor, y no en la ley ---en la ética y el deber.

\par 
%\textsuperscript{(1575.9)}
\textsuperscript{140:5.24} El amor paternal se complace en devolver el bien por el mal ---en hacer el bien como pago a la injusticia.

\section*{6. La noche de la ordenación}
\par 
%\textsuperscript{(1576.1)}
\textsuperscript{140:6.1} El domingo por la noche, al llegar de las tierras altas del norte de Cafarnaúm a la casa de Zebedeo, Jesús y los doce compartieron una cena sencilla. Más tarde, mientras Jesús se fue a pasear por la playa, los doce hablaron entre ellos. Después de una breve conversación, mientras los gemelos encendían un pequeño fuego para calentarse y tener más luz, Andrés salió a buscar a Jesús; cuando le dio alcance, le dijo: <<Maestro, mis hermanos son incapaces de comprender lo que has dicho sobre el reino. No nos sentimos en condiciones de empezar este trabajo hasta que nos hayas dado más enseñanzas. He venido para pedirte que te reúnas con nosotros en el jardín y nos ayudes a comprender el significado de tus palabras>>. Y Jesús fue con Andrés para reunirse con los apóstoles.

\par 
%\textsuperscript{(1576.2)}
\textsuperscript{140:6.2} Cuando hubo entrado en el jardín, congregó a los apóstoles a su alrededor y continuó enseñándoles, diciendo: <<Encontráis difícil recibir mi mensaje porque quisierais construir la nueva enseñanza directamente sobre la antigua, pero os afirmo que tenéis que renacer. Tenéis que comenzar de nuevo como niños pequeños y estar dispuestos a confiar en mi enseñanza y a creer en Dios. El nuevo evangelio del reino no se puede amoldar a lo que existe. Tenéis ideas equivocadas sobre el Hijo del Hombre y su misión en la Tierra. Pero no cometáis el error de pensar que he venido para rechazar la ley y los profetas; no he venido para destruir, sino para completar, ampliar e iluminar. No he venido para transgredir la ley, sino más bien para escribir estos nuevos mandamientos en las tablas de vuestro corazón>>\footnote{\textit{Tenéis que renacer}: Jn 3:3,7; 1 P 1:23. \textit{Comenzar como niños pequeños}: Mt 18:2-4; 19:13-14; Mc 9:36-37; 10:13-15; Lc 9:47-48; 18:16-17. \textit{No vengo a transgredir la ley, sino a completarla}: Mt 5:17.}.

\par 
%\textsuperscript{(1576.3)}
\textsuperscript{140:6.3} <<Exijo de vosotros una rectitud que sobrepasará a la de aquellos que intentan obtener el favor del Padre con la limosna, la oración y el ayuno. Si queréis entrar en el reino, habréis de tener una rectitud que consiste en el amor, la misericordia y la verdad ---el deseo sincero de hacer la voluntad de mi Padre que está en los cielos>>\footnote{\textit{Una rectitud mayor}: Mt 5:20.}.

\par 
%\textsuperscript{(1576.4)}
\textsuperscript{140:6.4} Entonces, Simón Pedro dijo: <<Maestro, si tienes un nuevo mandamiento, quisiéramos oírlo. Revélanos el nuevo camino>>\footnote{\textit{Un nuevo mandamiento}: Mt 5:21-22.}. Jesús le contestó a Pedro: <<Habéis oído decir a los que enseñan la ley: `No matarás; y cualquiera que mate estará sujeto a juicio'. Pero yo miro más allá del acto para descubrir el móvil. Os declaro que todo aquel que está irritado contra su hermano está en peligro de ser condenado. El que alimenta el odio en su corazón y planea la venganza en su mente, corre el peligro de ser juzgado. Tenéis que juzgar a vuestros compañeros por sus actos; el Padre que está en los cielos juzga según las intenciones>>\footnote{\textit{Habéis oído ``No matarás''}: Ex 20:13; Dt 5:17; Mt 5:21. \textit{Judgaz los motivos, no los actos}: 1 Sam 16:7. \textit{Corre peligro de ser juzgado}: Lv 24:17,21; Nm 35:30.}.

\par 
%\textsuperscript{(1576.5)}
\textsuperscript{140:6.5} <<Habéis oído decir a los maestros de la ley: `No cometerás adulterio'. Pero yo os digo que todo hombre que mira a una mujer con intenciones de lujuria, ya ha cometido adulterio con ella en su corazón. Sólo podéis juzgar a los hombres por sus actos, pero mi Padre mira dentro del corazón de sus hijos y los juzga con misericordia según sus intenciones y deseos reales>>\footnote{\textit{No cometáis adulterio}: Ex 20:14; Dt 5:18. \textit{Pero yo os digo ``La lujuria es adulterio''}: Mt 5:27-28.}.

\par 
%\textsuperscript{(1576.6)}
\textsuperscript{140:6.6} Jesús estaba dispuesto a continuar examinando los otros mandamientos, cuando Santiago Zebedeo le interrumpió para preguntar: <<Maestro, ¿qué vamos a enseñar a la gente sobre el divorcio? ¿Hemos de permitir que un hombre se divorcie de su mujer como Moisés lo ordenó?>>\footnote{\textit{Respecto al divorcio}: Mt 5:31-32; 19:3-9; Mc 10:2-12; Lc 16:18.} Cuando Jesús escuchó esta pregunta, dijo: <<No he venido para legislar, sino para iluminar. No he venido para reformar los reinos de este mundo, sino más bien para establecer el reino de los cielos. No es voluntad del Padre que ceda a la tentación de enseñaros reglas de gobierno, de comercio o de conducta social; aunque pudieran ser buenas para hoy, estarían lejos de ser convenientes para la sociedad de otra época. Estoy en la Tierra únicamente para confortar la mente, liberar el espíritu y salvar el alma de los hombres. Pero sobre esta cuestión del divorcio os diré que, aunque Moisés consideraba favorablemente estas cosas, no era así en los tiempos de Adán ni en el Jardín>>.

\par 
%\textsuperscript{(1577.1)}
\textsuperscript{140:6.7} Después de que los apóstoles hubieron hablado entre ellos durante unos momentos, Jesús continuó diciendo: <<Tenéis que reconocer siempre los dos puntos de vista de toda conducta de los mortales ---el humano y el divino; los caminos de la carne y la senda del espíritu; la opinión del tiempo y el punto de vista de la eternidad>>\footnote{\textit{Dos visiones, la carne y el espíritu}: Jn 3:6; Ro 8:4-5; Gl 5:16-17.}. Aunque los doce no podían comprender todo lo que les enseñaba, esta instrucción les ayudó realmente mucho.

\par 
%\textsuperscript{(1577.2)}
\textsuperscript{140:6.8} Entonces Jesús dijo: <<Pero vais a tropezar con mis enseñanzas porque estáis acostumbrados a interpretar mi mensaje literalmente; sois lentos en discernir el espíritu de mi enseñanza. Debéis recordar otra vez que sois mis mensajeros; estáis obligados a vivir vuestra vida como yo he vivido la mía en espíritu. Sois mis representantes personales; pero no cometáis el error de esperar que todos los hombres vivan como vosotros en todos los aspectos. También debéis recordar que tengo ovejas que no pertenecen a este rebaño, y que también estoy en deuda con ellos, ya que he de proporcionarles el modelo para hacer la voluntad de Dios, mientras vivo la vida de la naturaleza mortal>>\footnote{\textit{Tengo ovejas de otro rebaño}: Jn 10:16.}.

\par 
%\textsuperscript{(1577.3)}
\textsuperscript{140:6.9} Entonces Natanael preguntó: <<Maestro, ¿no vamos a dejar ningún lugar para la justicia? La ley de Moisés dice: `ojo por ojo y diente por diente'. ¿Qué vamos a decir nosotros?>>\footnote{\textit{Ojo por ojo y diente por diente}: Ex 21:24; Lv 24:20; Dt 19:21; Mt 5:38.} Y Jesús contestó: <<Vosotros devolveréis el bien por el mal. Mis mensajeros no deben luchar con los hombres, sino ser dulces con todos. Vuestra regla no será medida por medida. Los gobernantes de los hombres pueden tener tales leyes, pero no es así en el reino; la misericordia determinará siempre vuestro juicio, y el amor vuestra conducta. Y si estas afirmaciones os parecen duras, aun podéis echaros atrás. Si los requisitos del apostolado los encontráis demasiado duros, podéis volver al camino menos riguroso de los discípulos>>\footnote{\textit{Devolver el bien por el mal}: Mt 5:38-42. \textit{Mayor exigencia a los apóstoles}: Lc 14:25-72; Jn 6:60-69.}.

\par 
%\textsuperscript{(1577.4)}
\textsuperscript{140:6.10} Al escuchar estas palabras sorprendentes, los apóstoles se alejaron entre ellos un momento, pero no tardaron en volver, y Pedro dijo: <<Maestro, queremos seguir contigo; ninguno de nosotros quiere volverse atrás. Estamos plenamente preparados para pagar el precio adicional; beberemos la copa. Queremos ser apóstoles, no simplemente discípulos>>.

\par 
%\textsuperscript{(1577.5)}
\textsuperscript{140:6.11} Cuando Jesús oyó esto, dijo: <<Estad dispuestos entonces a asumir vuestras responsabilidades y a seguirme. Haced vuestras buenas acciones en secreto; cuando deis una limosna, que la mano izquierda no sepa lo que hace la mano derecha. Cuando oréis, hacedlo a solas y no utilicéis vanas repeticiones y frases sin sentido. Recordad siempre que el Padre sabe lo que necesitáis incluso antes de que se lo pidáis. Y no os pongáis a ayunar con un aspecto triste para que os vean los hombres. Como mis apóstoles escogidos, reservados ahora para el servicio del reino, no acumuléis tesoros en la Tierra, sino que, mediante vuestro servicio desinteresado, guardad tesoros en el cielo, porque allí donde estén vuestros tesoros estará también vuestro corazón>>\footnote{\textit{Asumid vuestras responsabilidades y seguidme}: Mt 16:24; Mc 8:34; 10:21; Lc 9:23. \textit{Haced buenas acciones en secreto}: Mt 6:1-4. \textit{Orad a solas}: Mt 6:5-7. \textit{Dios sabe lo que necesitamos antes de pedírselo}: Is 65:24; Mt 6:32. \textit{No ayunéis con aspecto triste}: Mt 6:16-18. \textit{Guardad tesoros en el cielo}: Mt 6:19-21; Lc 12:33-34.}.

\par 
%\textsuperscript{(1577.6)}
\textsuperscript{140:6.12} <<El ojo es la lámpara del cuerpo; por lo tanto, si vuestro ojo es generoso, todo vuestro cuerpo estará lleno de luz. Pero si vuestro ojo es egoísta, todo vuestro cuerpo estará lleno de tinieblas. Si la luz misma que está en vosotros se convierte en tinieblas, ¡cuán profundas serán esas tinieblas!>>\footnote{\textit{El ojo es la lámpara del cuerpo}: Mt 6:22-23; Lc 11:34-36.}

\par 
%\textsuperscript{(1577.7)}
\textsuperscript{140:6.13} Entonces Tomás preguntó a Jesús si debían <<continuar teniéndolo todo en común>>. El Maestro contestó: <<Sí, hermanos míos, quisiera que viviéramos juntos como una familia comprensiva. Una gran obra se os ha confiado, y deseo vuestro servicio indiviso. Sabéis que se ha dicho muy bien: `Nadie puede servir a dos señores a la vez'. No podéis adorar sinceramente a Dios, y al mismo tiempo servir al Dinero de todo corazón. Ahora que os habéis enrolado sin reservas en el trabajo del reino, no os inquietéis por vuestra vida, y preocupaos mucho menos por lo que vais a comer o beber, o con qué vestiréis vuestro cuerpo. Ya habéis aprendido que unas manos serviciales y unos corazones diligentes no pasan hambre. Ahora que os estáis preparando para consagrar todas vuestras energías al trabajo del reino, estad seguros de que el Padre no se olvidará de vuestras necesidades. Buscad primero el reino de Dios, y cuando hayáis encontrado la entrada, todas las cosas necesarias las recibiréis por añadidura. Por eso, no os preocupéis indebidamente por el mañana. A cada día le basta su propio afán>>\footnote{\textit{Ningún hombre puede servir a dos señores}: Mt 6:24; Lc 16:13. \textit{No os angustiéis por el mañana}: Mt 6:25,31-32; Lc 12:22-23. \textit{Buscad primero el reino}: Mt 6:33-34.}.

\par 
%\textsuperscript{(1578.1)}
\textsuperscript{140:6.14} Cuando vio que estaban dispuestos a permanecer levantados toda la noche para hacerle preguntas, Jesús les dijo: <<Hermanos míos, sois vasijas de barro; es mejor que vayáis a descansar con el fin de estar dispuestos para el trabajo de mañana>>. Pero el sueño se había alejado de sus párpados. Pedro se atrevió a pedir a su Maestro <<sólo una breve conversación privada contigo. No es que yo tenga secretos para mis hermanos, pero estoy confundido y, si acaso mereciera una reprimenda de mi Maestro, podría soportarla mejor a solas contigo>>. Jesús le dijo: <<Ven conmigo, Pedro>> ---mostrando el camino hacia la casa. Cuando Pedro regresó de encontrarse con su Maestro, muy alentado y bastante estimulado, Santiago decidió entrar para hablar con Jesús. Y así sucesivamente, hasta las primeras horas de la mañana, los demás apóstoles entraron de uno en uno para hablar con el Maestro. Cuando todos hubieron conversado personalmente con él, salvo los gemelos, que se habían dormido, Andrés entró a ver a Jesús y le dijo: <<Maestro, los gemelos se han dormido cerca del fuego en el jardín; ¿debo despertarlos para preguntarles si quieren hablar también contigo?>> Y Jesús le dijo a Andrés, sonriendo: <<Hacen bien ---no los molestes>>. La noche ya había pasado y despuntaba la luz de un nuevo día.

\section*{7. La semana después de la ordenación}
\par 
%\textsuperscript{(1578.2)}
\textsuperscript{140:7.1} Después de unas horas de sueño, cuando los doce estaban reunidos tomando un desayuno tardío con Jesús, éste les dijo: <<Ahora debéis empezar vuestro trabajo de predicación de la buena nueva y de instrucción de los creyentes. Preparaos para ir a Jerusalén>>. Después de que Jesús hubiera hablado, Tomás reunió el valor suficiente para decir: <<Ya sé, Maestro, que deberíamos estar preparados para emprender el trabajo, pero me temo que aún no somos capaces de llevar a cabo esta gran empresa. ¿Permitirías que nos quedáramos por aquí cerca unos días más, antes de empezar el trabajo del reino?>> Cuando vio que todos sus apóstoles estaban dominados por el mismo temor, Jesús dijo: <<Será como habéis pedido; permaneceremos aquí hasta después del sábado>>.

\par 
%\textsuperscript{(1578.3)}
\textsuperscript{140:7.2} Durante semanas y semanas, pequeños grupos de activos buscadores de la verdad, así como espectadores curiosos, habían venido a Betsaida para ver a Jesús. Las noticias sobre él ya se habían difundido más allá de la región; habían venido grupos de investigadores desde ciudades tan lejanas como Tiro, Sidón, Damasco, Cesarea y Jerusalén. Hasta ese momento, Jesús había acogido a esta gente y los había instruido sobre el reino, pero el Maestro traspasó ahora esta tarea a los doce. Andrés escogía a uno de los apóstoles y le asignaba un grupo de visitantes; a veces, los doce estaban todos ocupados con esta misión.

\par 
%\textsuperscript{(1578.4)}
\textsuperscript{140:7.3} Trabajaron durante dos días, enseñando de día y manteniendo sus conversaciones privadas hasta horas avanzadas de la noche. Al tercer día, Jesús se fue a charlar con Zebedeo y Salomé, después de despedir a sus apóstoles diciendo: <<Id a pescar, tratad de hacer algo distinto sin preocupaciones, o visitad quizás a vuestras familias>>. El jueves regresaron para tres días más de enseñanza.

\par 
%\textsuperscript{(1578.5)}
\textsuperscript{140:7.4} Durante esta semana de repaso, Jesús repitió muchas veces a sus apóstoles los dos grandes motivos de su misión en la Tierra después de su bautismo:

\par 
%\textsuperscript{(1578.6)}
\textsuperscript{140:7.5} 1. Revelar el Padre a los hombres.

\par 
%\textsuperscript{(1578.7)}
\textsuperscript{140:7.6} 2. Conducir a los hombres a hacerse conscientes de su filiación ---a comprender por la fe que son los hijos del Altísimo.

\par 
%\textsuperscript{(1579.1)}
\textsuperscript{140:7.7} Una semana así de experiencias variadas hizo mucho bien a los doce; algunos incluso empezaron a tener demasiada confianza en sí mismos. En la última conferencia, la noche después del sábado, Pedro y Santiago se acercaron a Jesús, diciendo: <<Estamos preparados; salgamos ahora para conquistar el reino>>. A lo cual Jesús replicó: <<Que vuestra sabiduría iguale a vuestro entusiasmo y vuestra valentía compense vuestra ignorancia>>.

\par 
%\textsuperscript{(1579.2)}
\textsuperscript{140:7.8} Aunque los apóstoles no lograban comprender muchas de sus enseñanzas, no dejaban de captar el significado de la vida maravillosamente hermosa que vivía con ellos.

\section*{8. El jueves por la tarde, en el lago}
\par 
%\textsuperscript{(1579.3)}
\textsuperscript{140:8.1} Jesús sabía muy bien que sus apóstoles no asimilaban plenamente sus enseñanzas. Decidió impartir una instrucción especial a Pedro, Santiago y Juan, con la esperanza de que fueran capaces de clarificar las ideas de sus compañeros. Veía que los doce captaban algunas características de la idea de un reino espiritual, pero persistían con obstinación en relacionar directamente estas nuevas enseñanzas espirituales con sus antiguos conceptos literales y arraigados del reino de los cielos como restauración del trono de David y restablecimiento de Israel como potencia temporal en la Tierra. En consecuencia, el jueves por la tarde, Jesús se alejó de la costa en una barca con Pedro, Santiago y Juan, para hablarles de los asuntos del reino. Fue una conversación educativa de cuatro horas que abarcó decenas de preguntas y respuestas, y se puede incluir de manera muy provechosa en este relato, reorganizando el resumen de esta tarde importante que Simón Pedro ofreció a su hermano Andrés a la mañana siguiente:

\par 
%\textsuperscript{(1579.4)}
\textsuperscript{140:8.2} 1. \textit{Hacer la voluntad del Padre}. La enseñanza de Jesús sobre confiar en los cuidados del Padre celestial no era un fatalismo ciego y pasivo. Aquella tarde citó, dándolo por bueno, un viejo refrán hebreo: <<El que no trabaje no comerá>>\footnote{\textit{El que no trabaje no comerá}: 2 Ts 3:10.}. Señaló su propia experiencia como comentario suficiente sobre sus enseñanzas. Sus preceptos sobre la confianza en el Padre no deben juzgarse según las condiciones sociales o económicas de los tiempos modernos o de cualquier otra época. Su enseñanza abarca los principios ideales de una vida cercana a Dios, en todas las épocas y en todos los mundos.

\par 
%\textsuperscript{(1579.5)}
\textsuperscript{140:8.3} Jesús aclaró a los tres la diferencia que había entre las exigencias de ser apóstol y las de ser discípulo. Incluso entonces no prohibió a los doce que ejercitaran la prudencia y la previsión. Él no predicaba contra la previsión, sino contra la ansiedad y la preocupación. Enseñaba la sumisión activa y alerta a la voluntad de Dios. En respuesta a las numerosas preguntas sobre la frugalidad y el ahorro, simplemente llamó la atención sobre su vida de carpintero, de fabricante de barcas y de pescador, y sobre su cuidadosa organización de los doce. Trató de aclararles que el mundo no debe ser considerado como un enemigo; que las circunstancias de la vida constituyen un designio divino que trabaja con los hijos de Dios.

\par 
%\textsuperscript{(1579.6)}
\textsuperscript{140:8.4} Jesús tuvo grandes dificultades para hacerles comprender su práctica personal de la no resistencia. Se negaba absolutamente a defenderse, y a los apóstoles les pareció que le hubiera gustado que ellos hubieran seguido la misma política. Les enseñó que no se opusieran al mal, que no combatieran las injusticias o las injurias, pero no les enseñó que toleraran pasivamente la maldad. Aquella tarde dejó muy claro que aprobaba el castigo social para los malhechores y los criminales, y que a veces el gobierno civil tiene que emplear la fuerza para mantener el orden social y aplicar la justicia.

\par 
%\textsuperscript{(1579.7)}
\textsuperscript{140:8.5} Nunca dejó de prevenir a sus discípulos contra la práctica perniciosa de las \textit{represalias}; no soportaba la revancha, la idea de desquitarse. Deploraba que se guardara rencor. Rechazaba la idea del ojo por ojo y diente por diente\footnote{\textit{Ojo por ojo y diente por diente}: Ex 21:24; Lv 24:20; Dt 19:21; Mt 5:38.}. Desaprobaba todo el concepto de la revancha privada y personal\footnote{\textit{Advertencia contra la venganza}: Pr 20:22; Dt 32:35; Ro 12:19.}, dejando estas cuestiones al gobierno civil, por un lado, y al juicio de Dios, por otro. Aclaró a los tres que sus enseñanzas se aplicaban al \textit{individuo}, no al Estado. Las instrucciones que había dado hasta ese momento sobre estas cuestiones las resumió como sigue:

\par 
%\textsuperscript{(1580.1)}
\textsuperscript{140:8.6} Amad a vuestros enemigos\footnote{\textit{Amad a vuestros enemigos}: Mt 5:43-44; Lc 6:27,35.} ---recordad las demandas morales de la fraternidad humana.

\par 
%\textsuperscript{(1580.2)}
\textsuperscript{140:8.7} La futilidad del mal: un agravio no se repara con la venganza\footnote{\textit{Futilidad de devolver mal con mal}: Mt 5:39-42; Ro 12:17-21; 1 Ts 5:15; 1 P 3:9-12.}. No cometáis el error de combatir el mal con sus propias armas.

\par 
%\textsuperscript{(1580.3)}
\textsuperscript{140:8.8} Tened fe\footnote{\textit{Tened fe}: Mt 6:25-34; Lc 12:22-32.} ---tened confianza en el triunfo final de la justicia divina y de la bondad eterna.

\par 
%\textsuperscript{(1580.4)}
\textsuperscript{140:8.9} 2. \textit{Actitud política}. Advirtió a sus apóstoles que fueran discretos en sus comentarios sobre las tensas relaciones que existían entonces entre el pueblo judío y el gobierno romano; les prohibió que se enredaran de alguna manera en estas dificultades. Siempre tenía el cuidado de evitar las trampas políticas de sus enemigos, respondiendo siempre: <<Dad al César lo que es del César, y a Dios lo que es de Dios>>\footnote{\textit{Dad al César lo que es del César y a Dios lo que es de Dios}: Mt 22:21; Mc 12:17; Lc 20:25.}. Se negaba a desviar su atención de su misión, que era la de establecer un nuevo camino de salvación; no se permitía a sí mismo preocuparse por otra cosa. En su vida personal, siempre acataba debidamente todas las leyes y reglas civiles; en todas sus enseñanzas públicas, hacía caso omiso de las cuestiones cívicas, sociales y económicas. Dijo a los tres apóstoles que sólo se preocupaba por los principios de la vida espiritual interior y personal del hombre.

\par 
%\textsuperscript{(1580.5)}
\textsuperscript{140:8.10} Jesús no era pues un reformador político. No venía para reorganizar el mundo; aunque lo hubiera hecho, sólo hubiera sido aplicable a aquella época y a aquella generación. Sin embargo, mostró al hombre la mejor manera de vivir, y ninguna generación está exenta de la tarea de descubrir la mejor manera de adaptar la vida de Jesús a sus propios problemas. Pero no cometáis nunca el error de identificar las enseñanzas de Jesús con alguna teoría política o económica, con algún sistema social o industrial.

\par 
%\textsuperscript{(1580.6)}
\textsuperscript{140:8.11} 3. \textit{Actitud social}. Durante mucho tiempo, los rabinos judíos habían debatido la cuestión: ¿Quién es mi prójimo?\footnote{\textit{¿Quién es mi prójimo?}: Lc 10:29-37.} Jesús vino a presentar la idea de una bondad activa y espontánea, de un amor tan sincero por los semejantes, que ampliaba el concepto de vecindad hasta incluir al mundo entero, convirtiendo así en prójimos a todos los hombres. Pero a pesar de todo esto, Jesús se interesaba únicamente por el individuo, no por la masa. Jesús no era un sociólogo, pero trabajó para destruir todas las formas de aislamiento egoísta. Enseñaba la simpatía pura, la compasión. Miguel de Nebadon es un Hijo dominado por la misericordia; la compasión es su verdadera naturaleza.

\par 
%\textsuperscript{(1580.7)}
\textsuperscript{140:8.12} El Maestro no dijo que los hombres nunca debían convidar a comer a sus amigos, pero sí dijo que sus discípulos deberían organizar festines para los pobres y los desgraciados\footnote{\textit{Convidar a los pobres}: Lc 14:12-14.}. Jesús tenía un sólido sentido de la justicia, pero siempre estaba templada por la misericordia. No enseñó a sus apóstoles que se dejaran engañar por los parásitos sociales o los mendigos profesionales. El momento en que estuvo más cerca de efectuar unas declaraciones sociológicas fue cuando dijo: <<No juzguéis, para no ser juzgados>>\footnote{\textit{No juzguéis, para no ser juzgados}: Mt 7:1; Lc 6:37.}.

\par 
%\textsuperscript{(1580.8)}
\textsuperscript{140:8.13} Indicó claramente que la beneficencia sin distinción puede ser acusada de muchos males sociales. Al día siguiente, Jesús ordenó definitivamente a Judas que no se entregara ningún fondo apostólico como limosna, a menos que él lo pidiera o que dos de los apóstoles lo solicitaran conjuntamente. En todas estas cuestiones, Jesús siempre tenía la costumbre de decir: <<Sed tan prudentes como las serpientes, pero tan inofensivos como las palomas>>\footnote{\textit{Sed sabios como serpientes}: Mt 10:16b.}. En todas las situaciones sociales, parecía tener el propósito de enseñar la paciencia, la tolerancia y el perdón.

\par 
%\textsuperscript{(1581.1)}
\textsuperscript{140:8.14} Para Jesús, la familia ocupaba el centro mismo de la filosofía de la vida ---aquí y en el más allá. Sus enseñanzas sobre Dios las basó en la familia, tratando al mismo tiempo de corregir la tendencia de los judíos a honrar excesivamente a sus antepasados. Alabó la vida familiar como el deber humano más alto, pero indicó claramente que las relaciones familiares no deben interferir con las obligaciones religiosas. Llamó la atención sobre el hecho de que la familia es una institución temporal que no sobrevive a la muerte. Jesús no dudó en abandonar a su familia cuando ésta se opuso a la voluntad del Padre. Enseñó la nueva y más amplia fraternidad de los hombres ---los hijos de Dios. En la época de Jesús, las costumbres relacionadas con el divorcio eran relajadas en Palestina y en todo el imperio romano. Se negó repetidas veces a establecer leyes sobre el matrimonio y el divorcio, pero muchos de los primeros seguidores de Jesús tenían opiniones arraigadas sobre el divorcio, y no dudaron en atribuírselas a él. Todos los escritores del Nuevo Testamento, exceptuando a Juan Marcos, se adhirieron a estas ideas más estrictas y avanzadas sobre el divorcio.

\par 
%\textsuperscript{(1581.2)}
\textsuperscript{140:8.15} 4. \textit{Actitud económica}. Jesús trabajó, vivió y comerció en el mundo tal como lo encontró. No era un reformador económico, aunque llamó frecuentemente la atención sobre la injusticia de la distribución desigual de la riqueza; pero no ofreció ninguna sugerencia como remedio. Indicó claramente a los tres que, aunque sus apóstoles no debían poseer bienes, no predicaba contra la riqueza y la propiedad, sino únicamente contra su distribución desigual e injusta. Reconocía la necesidad de la justicia social y de la equidad industrial, pero no ofreció ninguna regla para conseguirlas.

\par 
%\textsuperscript{(1581.3)}
\textsuperscript{140:8.16} Nunca enseñó a sus discípulos que evitaran las posesiones terrenales; sólo a sus doce apóstoles. Lucas, el médico, creía firmemente en la igualdad social, y contribuyó mucho a interpretar las palabras de Jesús en consonancia con sus creencias personales. Jesús nunca ordenó personalmente a sus seguidores que adoptaran un modo de vida comunitario; no hizo ninguna declaración de ningún tipo sobre estas cuestiones.

\par 
%\textsuperscript{(1581.4)}
\textsuperscript{140:8.17} Jesús previno con frecuencia a sus oyentes contra la codicia, declarando que <<la felicidad de un hombre no consiste en la abundancia de sus posesiones materiales>>\footnote{\textit{La felicidad no consiste en posesiones}: Lc 12:15.}. Reiteraba constantemente: <<¿De qué le sirve a un hombre ganar el mundo entero, si pierde su propia alma?>>\footnote{\textit{¿De qué le sirve al hombre ganar el mundo entero?}: Mt 16:26; Mc 8:36; Lc 9:25.} No lanzó ataques directos contra la posesión de bienes, pero sí insistió en que es eternamente esencial el dar la prioridad a los valores espirituales. En sus enseñanzas posteriores trató de corregir muchas opiniones erróneas urantianas sobre la vida, contando numerosas parábolas que dio a conocer en el transcurso de su ministerio público. Jesús nunca tuvo la intención de formular teorías económicas; sabía muy bien que cada época debe desarrollar sus propios remedios para los problemas existentes. Si Jesús estuviera hoy en la Tierra, viviendo su vida en la carne, desilusionaría mucho a la mayoría de los hombres y mujeres de bien, por la sencilla razón de que no tomaría partido en los debates políticos, sociales o económicos del día. Permanecería sublimemente al margen, mientras que os enseñaría a perfeccionar vuestra vida espiritual interior, con el fin de haceros mucho más competentes para atacar la solución de vuestros problemas puramente humanos.

\par 
%\textsuperscript{(1581.5)}
\textsuperscript{140:8.18} Jesús quería hacer a todos los hombres semejantes a Dios, y luego permanecer cerca con simpatía mientras estos hijos de Dios resuelven sus propios problemas políticos, sociales y económicos. No era la riqueza lo que denunciaba, sino lo que hace la riqueza a la mayoría de sus adictos. Este jueves por la tarde, Jesús dijo por primera vez a sus discípulos que <<es más bienaventurado dar que recibir>>\footnote{\textit{Es más bienaventurado dar que recibir}: Hch 20:35.}.

\par 
%\textsuperscript{(1581.6)}
\textsuperscript{140:8.19} 5. \textit{Religión personal}. Vosotros, al igual que hicieron sus apóstoles, deberíais comprender mejor las enseñanzas de Jesús a través de su vida. Vivió una vida perfeccionada en Urantia, y sus enseñanzas excepcionales sólo se pueden comprender cuando se visualiza esa vida en su trasfondo inmediato. Es su vida, y no sus lecciones a los doce o sus sermones a las multitudes, lo que ayudará mejor a revelar el carácter divino y la personalidad amorosa del Padre.

\par 
%\textsuperscript{(1582.1)}
\textsuperscript{140:8.20} Jesús no atacó las enseñanzas de los profetas hebreos o de los moralistas griegos. El Maestro reconocía las numerosas cosas buenas que defendían estos grandes pensadores, pero había venido a la Tierra para enseñar algo \textit{adicional:} <<la conformidad voluntaria de la voluntad del hombre a la voluntad de Dios>>\footnote{\textit{Haced voluntariamente la ``voluntad'' de Dios}: Sal 143:10; Eclo 15:11-20; Mt 6:10; 7:21; 12:50; Mc 3:35; Lc 8:21; 11:2; Jn 7:16-17; 9:31; 14:21-24; 15:10,14-16.}. Jesús no quería limitarse a producir \textit{hombres religiosos}, unos mortales enteramente ocupados en sentimientos religiosos y animados exclusivamente por impulsos espirituales. Si hubierais podido echar una sola mirada sobre él, hubierais sabido que Jesús era realmente un hombre de gran experiencia en las cosas de este mundo. Las enseñanzas de Jesús en este sentido han sido groseramente falseadas y muy mal presentadas a lo largo de todos los siglos de la era cristiana; también habéis tenido ideas tergiversadas sobre la mansedumbre y la humildad del Maestro. La meta que perseguía en su vida parece haber sido un \textit{magnífico respeto de sí mismo}. Sólo aconsejaba a los hombres que se humillaran para que pudieran ser verdaderamente ensalzados; lo que en realidad perseguía era una humildad auténtica ante Dios. Atribuía un gran valor a la sinceridad ---al corazón puro. La fidelidad era una virtud cardinal en su evaluación del carácter, mientras que la \textit{valentía} era el centro mismo de sus enseñanzas. Su consigna era <<No temáis>>\footnote{\textit{Jesús decía ``No temáis''}: Mt 10:28,31; 14:27; 17:7; 28:10; Mc 5:36; 6:50; Lc 5:10; 8:50; 12:4,4,7,32; Jn 6:20; 14:27.}, y el aguante paciente era su ideal de la fuerza de carácter. Las enseñanzas de Jesús constituyen una religión de valor, de coraje y de heroísmo. Y precisamente por eso escogió, como representantes personales suyos, a doce hombres corrientes que eran en su mayoría pescadores toscos, viriles y valerosos.

\par 
%\textsuperscript{(1582.2)}
\textsuperscript{140:8.21} Jesús tenía poco que decir sobre los vicios sociales de su época; rara vez se refirió a la delincuencia moral. Era un educador positivo de la verdadera virtud. Evitó cuidadosamente el método negativo de impartir la enseñanza; rehusó darle publicidad al mal. No era siquiera un reformador moral. Sabía muy bien, y así lo enseñó a sus apóstoles, que los impulsos sensuales de la humanidad no se suprimen con los reproches religiosos ni con las prohibiciones legales. Sus pocas denuncias estaban dirigidas sobre todo contra el orgullo, la crueldad, la opresión y la hipocresía.

\par 
%\textsuperscript{(1582.3)}
\textsuperscript{140:8.22} Jesús no denunció con vehemencia ni siquiera a los fariseos, como había hecho Juan. Sabía que muchos escribas y fariseos tenían un corazón honesto; comprendía que eran esclavos serviles de las tradiciones religiosas. Jesús insistía mucho en <<empezar por sanar el árbol>>\footnote{\textit{Empezad sanando el árbol}: Mt 12:33.}. Fijó en el ánimo de los tres que valoraba la vida en su totalidad, y no sólo algunas virtudes particulares.

\par 
%\textsuperscript{(1582.4)}
\textsuperscript{140:8.23} La única lección que Juan aprendió de la enseñanza de este día fue que el fondo de la religión de Jesús consistía en adquirir un carácter compasivo, acoplado con una personalidad motivada por hacer la voluntad del Padre que está en los cielos.

\par 
%\textsuperscript{(1582.5)}
\textsuperscript{140:8.24} Pedro captó la idea de que el evangelio que estaban a punto de proclamar era realmente un nuevo punto de partida para toda la raza humana. Más tarde transmitió esta impresión a Pablo, que la utilizó para formular su doctrina de Cristo como <<el segundo Adán>>\footnote{\textit{Cristo como segundo Adán}: 1 Co 15:45-49.}.

\par 
%\textsuperscript{(1582.6)}
\textsuperscript{140:8.25} Santiago comprendió la emocionante verdad de que Jesús deseaba que sus hijos de la Tierra vivieran como si ya fueran ciudadanos del reino celestial acabado.

\par 
%\textsuperscript{(1582.7)}
\textsuperscript{140:8.26} Jesús sabía que los hombres son diferentes, y así lo enseñó a sus apóstoles. Los exhortaba constantemente a que se abstuvieran de intentar moldear a los discípulos y a los creyentes según un modelo predeterminado. Intentaba dejar que cada alma se desarrollara según su propia manera, como un individuo distinto que se perfecciona ante Dios. En respuesta a una de las numerosas preguntas de Pedro, el Maestro dijo: <<Quiero liberar a los hombres para que puedan empezar de nuevo como niños pequeños en una vida nueva y mejor>>. Jesús insistía siempre en que la verdadera bondad debe ser inconsciente\footnote{\textit{La verdadera bondad es inconsciente}: Mt 6:1-3.}, que al hacer caridad no hay que dejar que la mano izquierda se entere de lo que hace la derecha.

\par 
%\textsuperscript{(1583.1)}
\textsuperscript{140:8.27} Aquella tarde, los tres apóstoles se escandalizaron cuando se dieron cuenta de que la religión de su Maestro no preveía el examen espiritual de sí mismo. Todas las religiones anteriores y posteriores a los tiempos de Jesús, incluido el cristianismo, prevén cuidadosamente un examen concienzudo de sí mismo. Pero no es así con la religión de Jesús de Nazaret; su filosofía de la vida carece de introspección religiosa. El hijo del carpintero nunca enseñó la \textit{formación} del carácter; enseñó el \textit{crecimiento} del carácter\footnote{\textit{Crecimiento del carácter}: Mt 13:31-32; Mc 4:31-32; Lc 13:18-19.}, declarando que el reino de los cielos se parece a un grano de mostaza. Pero Jesús no dijo nada que proscribiera el análisis de sí mismo como medio de prevenir el egotismo presuntuoso.

\par 
%\textsuperscript{(1583.2)}
\textsuperscript{140:8.28} El derecho a entrar en el reino está condicionado por la fe, por la creencia personal. Lo que hay que pagar para permanecer en la ascensión progresiva del reino es la perla de gran precio\footnote{\textit{La perla de gran precio}: Mt 13:45-46.}; para poseerla, el hombre vende todo lo que tiene.

\par 
%\textsuperscript{(1583.3)}
\textsuperscript{140:8.29} La enseñanza de Jesús es una religión para todos, no solamente para los débiles y los esclavos. Su religión nunca se cristalizó (en su época) en credos y en leyes teológicas; no dejó ni una línea escrita detrás de él. Su vida y sus enseñanzas fueron legadas al universo como una herencia inspiradora e ideal, adecuada para la orientación espiritual y la instrucción moral de todas las épocas en todos los mundos. Incluso hoy en día, las enseñanzas de Jesús se mantienen apartadas de todas las religiones, como tales, aunque son la esperanza viviente de cada una de ellas.

\par 
%\textsuperscript{(1583.4)}
\textsuperscript{140:8.30} Jesús no enseñó a sus apóstoles que la religión es la única ocupación del hombre en la Tierra; ésta era la idea que tenían los judíos del servicio de Dios. Pero sí insistió en que la religión sería la ocupación exclusiva de los doce. Jesús no enseñó nada que desviara a sus creyentes de la búsqueda de una cultura auténtica; sólo le quitó mérito a las escuelas religiosas de Jerusalén, atadas a la tradición. Era liberal, generoso, culto y tolerante. La piedad retraída no ocupaba ningún lugar en su filosofía de la manera recta de vivir.

\par 
%\textsuperscript{(1583.5)}
\textsuperscript{140:8.31} El Maestro no ofreció soluciones para los problemas no religiosos de su propia época ni de ninguna época posterior. Jesús deseaba desarrollar la comprensión espiritual de las realidades eternas y estimular la iniciativa en la originalidad de la manera de vivir; se ocupó exclusivamente de las necesidades espirituales subyacentes y permanentes de la raza humana. Reveló una bondad igual a la de Dios. Exaltó el amor ---la verdad, la belleza y la bondad--- como el ideal divino y la realidad eterna.

\par 
%\textsuperscript{(1583.6)}
\textsuperscript{140:8.32} El Maestro vino para crear un nuevo espíritu en el hombre, una nueva voluntad ---para conferirle una capacidad nueva para conocer la verdad, experimentar la compasión y escoger la bondad--- la voluntad de estar en armonía con la voluntad de Dios, unida al impulso eterno de volverse perfecto\footnote{\textit{Sed perfectos}: Gn 17:1; 1 Re 8:61; Lv 19:2; Dt 18:13; Mt 5:48; 2 Co 13:11; Stg 1:4; 1 P 1:16.} como el Padre que está en los cielos es perfecto.

\section*{9. El día de la consagración}
\par 
%\textsuperscript{(1583.7)}
\textsuperscript{140:9.1} Jesús dedicó el sábado siguiente a sus apóstoles, regresando a las tierras altas donde los había ordenado. Allí, después de un largo mensaje personal de estímulo, hermosamente conmovedor, emprendió el acto solemne de la consagración de los doce. Aquel sábado por la tarde, Jesús reunió a los apóstoles a su alrededor, en la ladera de la colina, y los puso en manos de su Padre celestial como preparación para el día en que se vería obligado a dejarlos solos en el mundo. No hubo ninguna enseñanza nueva en esta ocasión, sólo conversación y comunión.

\par 
%\textsuperscript{(1584.1)}
\textsuperscript{140:9.2} Jesús analizó muchos aspectos del sermón de ordenación, pronunciado en este mismo lugar; luego los llamó ante él, uno a uno, y les encargó que salieran al mundo como sus representantes. La misión de consagración\footnote{\textit{Consagración y comisión}: Mt 10:1; Mc 3:13-14; Lc 9:1-2.} del Maestro fue: <<Id por todo el mundo y predicad la buena nueva del reino. Liberad a los cautivos espirituales, confortad a los oprimidos y ayudad a los afligidos. Habéis recibido gratuitamente, dad gratuitamente>>\footnote{\textit{Encargo de predicar por el mundo}: Mt 24:14; 28:19-20a; Mc 13:10; 16:15; Lc 24:47; Jn 17:18; Hch 1:8b. \textit{Liberad a los cautivos espirituales}: Is 61:1; Lc 4:18. \textit{Confortad a los oprimidos}: Is 40:1; 2 Co 1:3-4. \textit{Ayudad a los afligios}: Mt 20:26; Mc 10:43. \textit{Habéis recibido gratuitamente, dad gratuitamente}: Mt 10:8.}.

\par 
%\textsuperscript{(1584.2)}
\textsuperscript{140:9.3} Jesús les aconsejó que no llevaran dinero ni ropa adicional\footnote{\textit{No llevéis dinero ni ropa adicional}: Mt 10:9-10; Mc 6:8-9; Lc 9:3; Lc 10:4.}, diciendo: <<El obrero merece su salario>>\footnote{\textit{El obrero merece su salario}: Lc 10:7.}. Y finalmente dijo: <<Mirad, os envío como ovejas en medio de los lobos; sed pues tan prudentes como las serpientes y tan inofensivos como las palomas. Pero prestad atención, porque vuestros enemigos os llevarán ante sus consejos, y os criticarán severamente en sus sinagogas. Seréis llevados ante los gobernadores y los jefes porque creéis en este evangelio, y vuestro testimonio mismo será mi propio testimonio ante ellos. Cuando os lleven a juicio, no os inquietéis por lo que tendréis que decir, porque el espíritu de mi Padre vive en vosotros y en esos momentos hablará por vosotros. Algunos de vosotros seréis ejecutados, y antes de que establezcáis el reino en la Tierra, seréis odiados por muchos pueblos a causa de este evangelio; pero no temáis, yo estaré con vosotros y mi espíritu os precederá en el mundo entero. La presencia de mi Padre permanecerá en vosotros mientras que os dirigís primero hacia los judíos y luego hacia los gentiles>>\footnote{\textit{Os envío como ovejas entre lobos}: Lc 10:3. \textit{Sed sabios como serpientes}: Mt 10:16. \textit{Os llevarán ante los consejos}: Lc 21:12. \textit{Os criticarán en las sinagogas}: Mt 10:17-21; Mc 13:9. \textit{Seréis vilipendiados y ejecutados}: Mt 24:9. \textit{Testimonio del espíritu interior}: Mt 10:19-20; Mc 13:11; Lc 12:11-12; Lc 21:13-15. \textit{Primero a los judíos, luego a los gentiles}: Ro 1:16.}.

\par 
%\textsuperscript{(1584.3)}
\textsuperscript{140:9.4} Cuando descendieron de la montaña, regresaron a su hogar en la casa de Zebedeo.

\section*{10. La noche después de la consagración}
\par 
%\textsuperscript{(1584.4)}
\textsuperscript{140:10.1} Aquella noche, Jesús enseñó dentro de la casa porque había empezado a llover; habló muy extensamente a los doce para tratar de mostrarles lo que debían \textit{ser}, y no lo que debían \textit{hacer}. Sólo conocían una religión que imponía \textit{hacer} ciertas cosas para poder alcanzar la rectitud ---la salvación. Pero Jesús les repetía: <<En el reino, tenéis que \textit{ser} rectos para hacer el trabajo>>. Muchas veces reiteró: <<\textit{Sed} pues perfectos, como vuestro Padre que está en los cielos es perfecto>>\footnote{\textit{Sed perfectos}: Gn 17:1; 1 Re 8:61; Lv 19:2; Dt 18:13; Mt 5:48; 2 Co 13:11; Stg 1:4; 1 P 1:16.}. El Maestro explicaba todo el tiempo a sus apóstoles aturdidos que la salvación que había venido a traer al mundo sólo se podía obtener \textit{creyendo}, con una fe simple y sincera. Jesús dijo: <<Juan ha predicado un bautismo de arrepentimiento, de aflicción por la vieja manera de vivir. Vosotros vais a proclamar el bautismo de la comunión con Dios. Predicad el arrepentimiento a los que necesitan esa enseñanza, pero a los que ya buscan entrar sinceramente en el reino, abridles las puertas de par en par y pedidles que entren en la jubilosa hermandad de los hijos de Dios>>\footnote{\textit{Juan os dió un bautismo de arrepentimiento}: Mt 3:2; Lc 3:3; Hch 13:24. \textit{La hermandad de los hijos de Dios}: Mt 12:50; Mc 3:35; Lc 8:21; Hch 2:42; 1 Co 1:9; Ef 3:9.}. Pero era una tarea difícil la de persuadir a estos pescadores galileos de que, en el reino, primero hay que \textit{ser} recto por la fe, antes de \textit{obrar} con rectitud en la vida cotidiana de los mortales de la Tierra.

\par 
%\textsuperscript{(1584.5)}
\textsuperscript{140:10.2} Otro gran obstáculo en este trabajo de enseñar a los doce era su tendencia a aceptar los principios altamente idealistas y espirituales de la verdad religiosa, y transformarlos en reglas concretas de conducta personal. Jesús les presentaba el hermoso espíritu de la actitud del alma, pero ellos insistían en traducir estas enseñanzas a reglas de comportamiento personal. Muchas veces, cuando estaban seguros de recordar lo que el Maestro había dicho, casi no podían dejar de olvidar lo que \textit{no} había dicho. Pero asimilaron lentamente su enseñanza, porque Jesús \textit{era} todo lo que enseñaba. Lo que no pudieron obtener con sus instrucciones verbales, lo adquirieron paulatinamente viviendo con él.

\par 
%\textsuperscript{(1585.1)}
\textsuperscript{140:10.3} Los apóstoles no percibían que su Maestro estaba ocupado en vivir una vida de inspiración espiritual para todas las personas de todas las épocas en todos los mundos de un vasto universo. A pesar de lo que Jesús les decía de vez en cuando, los apóstoles no captaban la idea de que estaba efectuando una labor \textit{en} este mundo, pero \textit{para} todos los otros mundos de su inmensa creación. Jesús vivió su vida terrestre en Urantia, no para establecer un ejemplo personal de vida mortal para los hombres y mujeres de este mundo, sino más bien para crear \textit{un ideal altamente espiritual e inspirador} para todos los seres mortales de todos los mundos.

\par 
%\textsuperscript{(1585.2)}
\textsuperscript{140:10.4} Esta misma noche, Tomás le preguntó a Jesús: <<Maestro, tú dices que debemos volvernos como niños pequeños antes de poder entrar en el reino del Padre, y sin embargo nos has advertido que no nos dejemos engañar por los falsos profetas, ni que nos hagamos culpables de arrojar nuestras perlas a los cerdos. Pues bien, estoy francamente desconcertado. No consigo comprender tu enseñanza>>\footnote{\textit{Volverse como un niño}: Mt 18:2-4; Mt 19:13-14; Mc 9:36-37; Mc 10:13-15; Lc 9:47-48; Lc 18:16-17. \textit{Los falsos profetas}: Mt 7:15,22-23; Mt 24:11; Mc 13:22; 1 Jn 4:1. \textit{Arrojar nuestras perlas a los cerdos}: Mt 7:6.}. Jesús le contestó a Tomás: <<¡Cuánto tiempo seré indulgente con vosotros! Siempre insistís en entender literalmente todo lo que enseño. Cuando os he pedido que os volváis como niños pequeños, como precio de entrada en el reino, no me refería a la facilidad de dejarse engañar, a la simple buena voluntad de creer, ni a la rapidez para confiar en los extraños agradables. Lo que deseaba que pudierais deducir con este ejemplo era la relación entre un niño y su padre. Tú eres el hijo, y es en el reino de \textit{tu} padre donde pretendes entrar. Entre todo niño normal y su padre existe ese afecto natural que asegura una relación comprensiva y amorosa, y que excluye para siempre toda tendencia al regateo para obtener el amor y la misericordia del Padre. Y el evangelio que vais a predicar tiene que ver con una salvación que se origina cuando se comprende, por la fe, esta misma relación eterna entre el niño y su padre>>.

\par 
%\textsuperscript{(1585.3)}
\textsuperscript{140:10.5} La característica principal de la enseñanza de Jesús consistía en que la \textit{moralidad} de su filosofía se originaba en la relación personal del individuo con Dios ---la misma relación que entre el niño y su padre. Jesús hacía hincapié en el \textit{individuo}, y no en la raza o en la nación. Mientras cenaban, Jesús tuvo una conversación con Mateo en la que le explicó que la moralidad de un acto cualquiera está determinada por el móvil del individuo. La moralidad de Jesús era siempre positiva. La regla de oro\footnote{\textit{La regla de oro de Jesús, en modo positivo}: Mt 7:12; Lc 6:31.}, tal como Jesús la expuso de nuevo con más claridad, exige un contacto social activo; la antigua regla negativa\footnote{\textit{La antigua regla de oro, en modo negativo}: Tb 4:15.} podía ser obedecida en la soledad. Jesús despojó a la moralidad de todas las reglas y ceremonias, y la elevó a los niveles majestuosos del pensamiento espiritual y de la vida verdaderamente recta.

\par 
%\textsuperscript{(1585.4)}
\textsuperscript{140:10.6} Esta nueva religión de Jesús no estaba desprovista de implicaciones prácticas, pero todo lo que se puede encontrar en su enseñanza con un valor práctico, en el aspecto político, social o económico, es la consecuencia natural de esta experiencia interior del alma, que manifiesta los frutos del espíritu en el ministerio diario espontáneo de una experiencia religiosa personal auténtica.

\par 
%\textsuperscript{(1585.5)}
\textsuperscript{140:10.7} Después de que Jesús y Mateo terminaran de hablar, Simón Celotes preguntó: <<Pero, Maestro, ¿\textit{todos} los hombres son hijos de Dios?>> Y Jesús contestó: <<Sí, Simón, todos los hombres son hijos de Dios, y ésa es la buena nueva que vais a proclamar>>. Pero los apóstoles no conseguían comprender esta doctrina; era una declaración nueva, extraña y sorprendente. A causa de su deseo de inculcar esta verdad a sus discípulos, Jesús les enseñó a tratar a todos los hombres como hermanos.

\par 
%\textsuperscript{(1585.6)}
\textsuperscript{140:10.8} En respuesta a una pregunta de Andrés, el Maestro indicó claramente que la moralidad implícita en su enseñanza era inseparable de la religión implícita en su manera de vivir. Enseñaba la moralidad, no partiendo de la \textit{naturaleza} del hombre, sino partiendo de la \textit{relación} del hombre con Dios.

\par 
%\textsuperscript{(1585.7)}
\textsuperscript{140:10.9} Juan le preguntó a Jesús: <<Maestro, ¿qué es el reino de los cielos?>> Y Jesús respondió: <<El reino de los cielos consiste en estas tres cosas esenciales: primero, el reconocimiento del hecho de la soberanía de Dios; segundo, la creencia en la verdad de la filiación con Dios; y tercero, la fe en la eficacia del deseo supremo humano de hacer la voluntad de Dios ---de ser semejante a Dios. Y he aquí la buena nueva del evangelio: por medio de la fe, cada mortal puede poseer todas estas cosas esenciales para la salvación>>.

\par 
%\textsuperscript{(1586.1)}
\textsuperscript{140:10.10} Ahora que la semana de espera había terminado, se prepararon para partir al día siguiente hacia Jerusalén.


\chapter{Documento 141. El comienzo de la obra pública}
\par 
%\textsuperscript{(1587.1)}
\textsuperscript{141:0.1} EL 19 de enero del año 27, primer día de la semana, Jesús y los doce apóstoles se prepararon para marcharse de su cuartel general de Betsaida. Los doce no sabían nada de los planes de su Maestro, excepto que subirían a Jerusalén para asistir a la fiesta de la Pascua en abril, y que se tenía la intención de viajar por el camino del valle del Jordán. No salieron de la casa de Zebedeo hasta cerca del mediodía, porque las familias de los apóstoles y de otros discípulos habían venido para despedirlos y desearles buena suerte en la nueva tarea que estaban a punto de empezar.

\par 
%\textsuperscript{(1587.2)}
\textsuperscript{141:0.2} Poco antes de partir, los apóstoles no vieron al Maestro, y Andrés salió a buscarlo. No tardó en encontrarlo sentado en una barca en la playa, y Jesús estaba llorando. Los doce habían visto a menudo a su Maestro cuando parecía apesadumbrado, y habían contemplado sus breves períodos de graves preocupaciones mentales, pero ninguno de ellos lo había visto nunca llorar. Andrés se quedó un poco sorprendido al ver al Maestro así de afectado en vísperas de su partida hacia Jerusalén, y se atrevió a acercarse a Jesús para preguntarle: <<En este gran día, Maestro, cuando estamos a punto de partir hacia Jerusalén para proclamar el reino del Padre, ¿por qué lloras? ¿Quién de nosotros te ha ofendido?>> Y Jesús, regresando con Andrés para reunirse con los doce, le respondió: <<Ninguno de vosotros me ha causado pena. Estoy triste solamente porque ningún miembro de la familia de mi padre José se ha acordado de venir para desearnos buena suerte>>. En aquel momento, Rut estaba de visita en casa de su hermano José, en Nazaret. Otros miembros de su familia se mantenían alejados por orgullo, decepción, incomprensión y pequeños resentimientos a los que habían cedido porque sus sentimientos habían sido heridos.

\section*{1. La salida de Galilea}
\par 
%\textsuperscript{(1587.3)}
\textsuperscript{141:1.1} Cafarnaúm no estaba lejos de Tiberiades, y la fama de Jesús había empezado a propagarse ampliamente por toda Galilea, e incluso más allá. Jesús sabía que Herodes empezaría pronto a prestar atención a su obra; por eso pensó que sería mejor viajar hacia el sur y entrar en Judea con sus apóstoles. Un grupo de más de cien creyentes deseaba ir con ellos, pero Jesús les habló y les rogó que no acompañaran al grupo apostólico en su descenso por el Jordán. Aunque consintieron en quedarse atrás, muchos de ellos siguieron al Maestro pocos días después.

\par 
%\textsuperscript{(1587.4)}
\textsuperscript{141:1.2} El primer día, Jesús y los apóstoles sólo llegaron hasta Tariquea, donde descansaron durante la noche. Al día siguiente viajaron hasta un punto del Jordán, cerca de Pella, donde Juan había predicado aproximadamente un año antes, y donde Jesús había recibido el bautismo. Se detuvieron allí durante más de dos semanas, enseñando y predicando. Hacia el final de la primera semana, varios cientos de personas se habían reunido en un campamento, cerca del lugar donde residían Jesús y los doce; habían venido de Galilea, Fenicia, Siria, la Decápolis, Perea y Judea.

\par 
%\textsuperscript{(1588.1)}
\textsuperscript{141:1.3} Jesús no efectuó ninguna predicación pública. Andrés dividía la multitud y designaba los predicadores para las asambleas de la mañana y de la tarde. Después de la cena, Jesús conversaba con los doce. No les enseñaba nada nuevo, pero repasaba su enseñanza anterior y contestaba a sus numerosas preguntas. Durante una de aquellas noches, contó a los doce algunas cosas sobre los cuarenta días que había pasado en las colinas, cerca de este lugar.

\par 
%\textsuperscript{(1588.2)}
\textsuperscript{141:1.4} Muchas de las personas que venían de Perea y de Judea habían sido bautizadas por Juan y estaban interesadas en saber más cosas sobre las enseñanzas de Jesús. Los apóstoles hicieron muchos progresos enseñando a los discípulos de Juan, ya que no desacreditaban de ninguna manera la predicación de Juan, y además, en aquella época ni siquiera bautizaban a sus nuevos discípulos. Pero siempre fue un escollo para los seguidores de Juan el ver que Jesús, si era todo lo que Juan había anunciado, no hacía nada por sacarlo de la cárcel. Los discípulos de Juan nunca pudieron comprender por qué Jesús no impidió la muerte cruel de su amado jefe.

\par 
%\textsuperscript{(1588.3)}
\textsuperscript{141:1.5} Noche tras noche, Andrés enseñaba cuidadosamente a sus compañeros apóstoles la tarea delicada y difícil de llevarse bien con los seguidores de Juan el Bautista. Durante este primer año del ministerio público de Jesús, más de las tres cuartas partes de sus discípulos habían seguido previamente a Juan y habían recibido su bautismo. Todo este año 27 lo pasaron haciéndose cargo tranquilamente de la obra de Juan en Perea y Judea.

\section*{2. La ley de Dios y la voluntad del Padre}
\par 
%\textsuperscript{(1588.4)}
\textsuperscript{141:2.1} La noche antes de partir de Pella, Jesús dio a los apóstoles algunas enseñanzas adicionales sobre el nuevo reino. El Maestro dijo: <<Se os ha enseñado a esperar la venida del reino de Dios, y ahora vengo para anunciar que este reino tanto tiempo esperado está cerca, que incluso ya está aquí, en medio de nosotros. En todo reino ha de haber un rey sentado en su trono, decretando las leyes del reino. Por eso habéis desarrollado un concepto del reino de los cielos consistente en el gobierno glorificado del pueblo judío sobre todos los pueblos de la Tierra, con el Mesías sentado en el trono de David, promulgando, desde ese lugar de poder milagroso, las leyes del mundo entero. Pero, hijos míos, no veis con los ojos de la fe, y no oís con el entendimiento del espíritu. Declaro que el reino de los cielos es la comprensión y el reconocimiento del gobierno de Dios en el corazón de los hombres. Es verdad que hay un Rey en este reino, y ese Rey es mi Padre y vuestro Padre. Somos en verdad sus súbditos leales, pero mucho más allá de este hecho se encuentra la verdad transformadora de que somos sus \textit{hijos}. En mi vida, esta verdad ha de volverse manifiesta para todos. Nuestro Padre también está sentado en un trono, pero ninguna mano lo ha hecho. El trono del Infinito es la residencia eterna del Padre en el cielo de los cielos; él llena todas las cosas y proclama sus leyes a unos universos tras otros. Y el Padre reina también en el corazón de sus hijos de la Tierra por medio del espíritu que ha enviado a vivir dentro del alma de los hombres mortales>>\footnote{\textit{El reino de Dios ya está aquí}: Mt 3:2; 4:17,23; 6:33; 9:35; 10:7; 24:14; Mc 1:14-15; Lc 4:43; 10:9,11; 17:21; 21:29-32; Jn 3:3,5. \textit{El trono del rey David}: Lc 1:32. \textit{El espíritu de Dios dentro de nosotros}: Job 32:8,18; Is 63:10-11; Ez 37:14; Mt 10:20; Lc 17:20-21; Jn 17:21-23; Ro 8:9-11; 1 Co 3:16-17; 6:19; 2 Co 6:16; Gl 2:20; 1 Jn 3:24; 4:12-15; Ap 21:3.}.

\par 
%\textsuperscript{(1588.5)}
\textsuperscript{141:2.2} <<Cuando sois los súbditos de este reino, debéis oír en verdad la ley del Soberano Universal; pero cuando, a causa del evangelio del reino que he venido a proclamar, descubrís por la fe que sois hijos, ya no seguís considerándoos como criaturas sujetas a la ley de un rey todopoderoso, sino como los hijos privilegiados de un Padre amoroso y divino. En verdad, en verdad os digo que cuando la voluntad del Padre es vuestra \textit{ley}, difícilmente estáis en el reino. Pero cuando la voluntad del Padre se convierte realmente en vuestra \textit{voluntad}, entonces estáis de verdad en el reino, porque el reino se ha vuelto así una experiencia establecida en vosotros. Cuando la voluntad de Dios es vuestra ley, sois unos nobles súbditos esclavos; pero cuando creéis en este nuevo evangelio de filiación divina, la voluntad de mi Padre se convierte en vuestra voluntad, y sois elevados a la alta posición de los hijos libres de Dios, los hijos liberados del reino>>\footnote{\textit{Evangelio del reino}: Mt 4:23; 9:35; 24:14; Mc 1:14-15. \textit{Hijos de Dios por la fe}: 1 Cr 22:10; Sal 2:7; Is 56:5; Mt 5:9,16,45; Lc 20:36; Jn 1:12-13; 11:52; Hch 17:28-29; Ro 8:14-17,19,21; 9:26; 2 Co 6:18; Gl 3:26; 4:5-7; Ef 1:5; Flp 2:15; Heb 12:5-8; 1 Jn 3:1-2,10; 5:2; Ap 21:7; 2 Sam 7:14. \textit{Cuando la voluntad del Padre se convierta en vuestra voluntad}: Sal 143:10; Eclo 15:11-20; Mt 6:10; 7:21; 12:50; Mc 3:35; Lc 8:21; 11:2; Jn 7:16-17; 9:31; 14:21-24; 15:10,14-16.}.

\par 
%\textsuperscript{(1589.1)}
\textsuperscript{141:2.3} Algunos apóstoles captaron algo de esta enseñanza, pero ninguno de ellos comprendió el significado completo de esta formidable declaración, a excepción quizás de Santiago Zebedeo. Sin embargo, estas palabras se grabaron en su corazón y emergieron para alegrar su ministerio durante los años posteriores de servicio.

\section*{3. La estancia en Amatus}
\par 
%\textsuperscript{(1589.2)}
\textsuperscript{141:3.1} El Maestro y sus apóstoles permanecieron cerca de Amatus casi tres semanas. Los apóstoles continuaron predicando a la multitud dos veces al día, y Jesús predicó todos los sábados por la tarde. Resultó imposible continuar con el recreo de los miércoles; por eso, Andrés decidió que dos apóstoles descansarían cada día durante seis días de la semana, y que todos estarían de servicio durante los oficios del sábado.

\par 
%\textsuperscript{(1589.3)}
\textsuperscript{141:3.2} Pedro, Santiago y Juan hicieron la mayor parte de la predicación pública. Felipe, Natanael, Tomás y Simón hicieron una gran parte del trabajo personal y dirigieron clases para grupos especiales de investigadores; los gemelos continuaron con su supervisión general de vigilancia, mientras que Andrés, Mateo y Judas se organizaron en un comité de administración general de tres miembros, aunque cada uno de ellos también realizó un considerable trabajo religioso.

\par 
%\textsuperscript{(1589.4)}
\textsuperscript{141:3.3} Andrés estaba muy ocupado con la tarea de arreglar los malentendidos y desacuerdos que se repetían continuamente entre los discípulos de Juan y los discípulos más recientes de Jesús. Cada pocos días se producían situaciones graves, pero Andrés, con la ayuda de sus colegas apostólicos, se las ingeniaba para persuadir a las partes en conflicto para que llegaran a algún tipo de acuerdo, aunque fuera temporal. Jesús rehusó participar en ninguna de estas conferencias; tampoco quiso dar ningún consejo sobre la manera de arreglar adecuadamente estas dificultades. Ni una sola vez ofreció sugerencias a los apóstoles sobre cómo resolver estos confusos problemas. Cuando Andrés se presentaba con estas cuestiones, Jesús siempre le decía: <<No es prudente que el anfitrión participe en las querellas familiares de sus huéspedes; un padre sabio nunca toma partido en las desavenencias menores de sus propios hijos>>.

\par 
%\textsuperscript{(1589.5)}
\textsuperscript{141:3.4} El Maestro mostraba una gran sabiduría y manifestaba una equidad perfecta en todas sus relaciones con sus apóstoles y con todos sus discípulos. Jesús era realmente un maestro de hombres; ejercía una gran influencia sobre sus semejantes a causa de la fuerza y el encanto combinados de su personalidad. Su vida ruda, nómada y sin hogar producía una sutil influencia dominante. Había un atractivo intelectual y un poder persuasivo espiritual en su manera de enseñar llena de autoridad, en su lógica lúcida, en la fuerza de su razonamiento, en su perspicacia sagaz, en su viveza mental, en su serenidad incomparable y en su sublime tolerancia. Era sencillo, varonil, honrado e intrépido. Junto a toda esta influencia física e intelectual que manifestaba la presencia del Maestro, también se encontraban todos los encantos espirituales del ser que se habían asociado con su personalidad ---la paciencia, la ternura, la mansedumbre, la dulzura y la humildad.

\par 
%\textsuperscript{(1589.6)}
\textsuperscript{141:3.5} Jesús de Nazaret era en verdad una personalidad fuerte y enérgica; era una potencia intelectual y una fortaleza espiritual. Su personalidad no atraía solamente, entre sus discípulos, a las mujeres propensas a la espiritualidad, sino también al culto e intelectual Nicodemo y al endurecido soldado romano, el capitán que estaba de guardia en la cruz, que después de ver morir al Maestro, dijo: <<En verdad, era un Hijo de Dios>>\footnote{\textit{Evangelio del Hijo divino de Dios}: Mt 8:29; 14:33; 16:15-16; 27:54; Mc 1:1; 3:11; 15:39; Lc 1:35; 4:41; Jn 1:34,49; 3:16-18; 10:36; 20:31; Hch 8:37.}. Y los enérgicos y robustos pescadores galileos le llamaban Maestro\footnote{\textit{Le llamaban ``Maestro''}: Mc 9:38; 13:1; Lc 5:5; Jn 4:31.}.

\par 
%\textsuperscript{(1590.1)}
\textsuperscript{141:3.6} Los retratos de Jesús han sido muy desacertados. Esas pinturas de Cristo han ejercido una influencia perjudicial sobre la juventud; los mercaderes del templo difícilmente hubieran huido delante de Jesús si éste hubiera sido el tipo de hombre que vuestros artistas han representado generalmente. Su masculinidad estaba llena de dignidad; era bueno, pero natural. Jesús no tenía la actitud de un místico apacible, dulce, suave y amable. Su enseñanza era conmovedoramente dinámica. No solamente tenía \textit{buenas intenciones}, sino que iba de un sitio para otro \textit{haciendo} realmente \textit{el bien}\footnote{\textit{Jesús pasó haciendo el bien}: Hch 10:38.}.

\par 
%\textsuperscript{(1590.2)}
\textsuperscript{141:3.7} El Maestro nunca dijo: <<Venid a mí todos los que sois indolentes y todos los soñadores>>. Pero sí dijo muchas veces: <<Venid a mí todos los que os \textit{esforzáis}, y yo os daré descanso ---fuerza espiritual>>\footnote{\textit{Venid y os daré descanso}: Mt 11:28.}. En verdad, el yugo del Maestro es ligero\footnote{\textit{El yugo es ligero}: Mt 11:29-30.}, pero incluso así, nunca lo impone; cada persona debe coger ese yugo por su propia voluntad.

\par 
%\textsuperscript{(1590.3)}
\textsuperscript{141:3.8} Jesús describió la conquista como fruto del sacrificio, el sacrificio del orgullo y del egoísmo. Al mostrar misericordia, pretendía ilustrar la liberación espiritual de todos los rencores, agravios, ira y ansias de poder y de venganza egoístas. Cuando dijo: <<No resistáis al mal>>\footnote{\textit{No resistáis al mal}: Mt 5:39-42; Lc 6:28-31.}, explicó más adelante que no quería decir que excusara el pecado o que aconsejara fraternizar con la iniquidad. Intentaba más bien enseñar a perdonar, a <<no resistirse a los malos tratos contra nuestra personalidad, al perjuicio dañino contra nuestros sentimientos de dignidad personal>>.

\section*{4. La enseñanza sobre el Padre}
\par 
%\textsuperscript{(1590.4)}
\textsuperscript{141:4.1} Durante su estancia en Amatus, Jesús pasó mucho tiempo enseñando a los apóstoles el nuevo concepto de Dios; les inculcó una y otra vez que \textit{Dios es un Padre}, y no un contable grande y supremo que se ocupa principalmente de efectuar asientos perjudiciales contra sus hijos desviados de la Tierra, registrando sus pecados y maldades para luego utilizarlos contra ellos cuando se siente a juzgarlos como justo Juez de toda la creación. Desde hacía mucho tiempo, los judíos habían concebido a Dios como un rey por encima de todo\footnote{\textit{Dios como rey por encima de todo}: Sal 22:27; Is 2:2-5; Dn 2:44; Zac 14:7; Mal 1:14.}, e incluso como Padre de la nación\footnote{\textit{Dios como padre de la nación judía}: Is 63:16; 64:8.}, pero nunca antes un gran número de hombres mortales había mantenido la idea de Dios como Padre amoroso del \textit{individuo}.

\par 
%\textsuperscript{(1590.5)}
\textsuperscript{141:4.2} En respuesta a la pregunta de Tomás: <<¿Quién es este Dios del reino?>>, Jesús replicó: <<Dios es \textit{tu} Padre, y la religión ---mi evangelio--- no es ni más ni menos que reconocer la verdad, creyéndolo, de que tú eres su hijo. Y yo estoy aquí, viviendo en la carne entre vosotros, para clarificar estas dos ideas con mi vida y mis enseñanzas>>footnote{\textit{Dios es nuestro Padre}: 1 Cr 22:10; Sal 2:7; 89:26-27; Jer 3:19; Mt 5:16,45,48; 6:1,9,14; 6:26:32; 7:11; 10:32-33; 18:14; 23:9; Mc 11:25-26; Lc 6:36; 11:2,13; Jn 20:17b; Ro 1:7; 8:14-15; 1 Co 1:3; 2 Co 1:2; 6:18; Gl 1:4; 4:6-7; Ef 1:2; Flp 1:2; Col 1:2; 1 Ts 1:1,3; 2 Ts 1:1-2; 1 Ti 1:2; Flm 1:2; 2 Sam 7:14. \textit{Somos los hijos de Dios}: 1 Cr 22:10; Sal 2:7; Is 56:5; Mt 5:9,16,45; Lc 20:36; Jn 1:12-13; 11:52; Hch 17:28-29; Ro 8:14-17,19,21; 9:26; 2 Co 6:18; Gl 3:26; 4:5-7; Ef 1:5; Flp 2:15; Heb 12:5-8; 1 Jn 3:1-2,10; 5:2; Ap 21:7; 2 Sam 7:14.}.

\par 
%\textsuperscript{(1590.6)}
\textsuperscript{141:4.3} Jesús también intentó liberar la mente de sus apóstoles de la idea de que ofrecer sacrificios de animales era un deber religioso. Pero estos hombres, educados en la religión del sacrificio diario, eran lentos en comprender lo que les quería decir. Sin embargo, el Maestro no se cansó de enseñarles. Cuando no conseguía llegar a la mente de todos los apóstoles mediante un solo ejemplo, volvía a repetir su mensaje empleando otro tipo de parábola con objeto de iluminarlos.

\par 
%\textsuperscript{(1590.7)}
\textsuperscript{141:4.4} Por esta misma época, Jesús empezó a enseñar más plenamente a los doce sobre su misión\footnote{\textit{La misión de los doce}: Mt 10:1,8; Lc 9:2; 10:9.} de <<consolar a los afligidos y de cuidar a los enfermos>>. El Maestro les enseñó muchas cosas sobre el hombre completo ---la unión del cuerpo, la mente y el espíritu para formar el individuo, hombre o mujer. Jesús expuso a sus asociados los tres tipos de aflicción que iban a encontrar, y luego les explicó cómo deberían ayudar a todos los que sufren los dolores de las enfermedades humanas. Les enseñó a reconocer:

\par 
%\textsuperscript{(1591.1)}
\textsuperscript{141:4.5} 1. Las enfermedades de la carne ---las aflicciones generalmente consideradas como enfermedades físicas.

\par 
%\textsuperscript{(1591.2)}
\textsuperscript{141:4.6} 2. Las mentes perturbadas ---las aflicciones no físicas, posteriormente consideradas como dificultades y desórdenes emocionales y mentales.

\par 
%\textsuperscript{(1591.3)}
\textsuperscript{141:4.7} 3. La posesión por los malos espíritus.

\par 
%\textsuperscript{(1591.4)}
\textsuperscript{141:4.8} En diversas ocasiones, Jesús explicó a sus apóstoles la naturaleza de estos malos espíritus, y les dijo algunas cosas sobre su origen; en aquella época también se les llamaba a menudo espíritus impuros. El Maestro conocía bien la diferencia entre la posesión por los malos espíritus y la demencia, pero los apóstoles lo ignoraban. En vista de su conocimiento limitado de la historia primitiva de Urantia, Jesús tampoco podía emprender la tarea de hacerles comprender plenamente esta cuestión. Pero les dijo muchas veces, aludiendo a estos malos espíritus: <<No volverán a molestar a los hombres cuando yo haya ascendido hasta mi Padre que está en los cielos, y después de que haya derramado mi espíritu sobre todo el género humano, en la época en que el reino vendrá con gran poder y gloria espiritual>>\footnote{\textit{Jesús derramando su espíritu}: Ez 11:19; 18:31; 36:26-27; Jl 2:28-29; Lc 24:49; Jn 7:39; 14:16-18,23,26; 15:4,26; 16:7-8,13-14; 17:21-23; Hch 1:5,8a; 2:1-4,16-18; 2:33; 2 Co 13:5; Gl 2:20; 4:6; Ef 1:13; 4:30; 1 Jn 4:12-15. \textit{Llegada del reino en poder y en gloria}: Mt 24:30-31; Mc 13:26; Lc 21:27; Hch 1:7-8.}.

\par 
%\textsuperscript{(1591.5)}
\textsuperscript{141:4.9} Semana tras semana y un mes tras otro, a lo largo de todo este año, los apóstoles prestaron cada vez más atención a la tarea de curar a los enfermos.

\section*{5. La unidad espiritual}
\par 
%\textsuperscript{(1591.6)}
\textsuperscript{141:5.1} Una de las conferencias nocturnas más extraordinarias de Amatus fue la sesión en la que se discutió sobre la unidad espiritual. Santiago Zebedeo había preguntado: <<Maestro, ¿cómo podemos aprender a tener el mismo punto de vista, y a disfrutar así de una mayor armonía entre nosotros?>>\footnote{\textit{Armonía apostólica}: Zac 3:10; Jn 17:21; Ef 4:13-16.} Cuando Jesús escuchó esta pregunta, su espíritu se alteró de tal manera que replicó: <<Santiago, Santiago, ¿cuándo te he enseñado que todos debéis tener el mismo punto de vista? He venido al mundo para proclamar la libertad espiritual, con el fin de que los mortales puedan tener la facultad de vivir una vida individual original y libre ante Dios. No deseo que la armonía social y la paz fraternal se adquieran a costa del sacrificio de la personalidad libre y de la originalidad espiritual. Lo que yo os pido, a mis apóstoles, es la \textit{unidad espiritual} ---y eso lo podéis experimentar en la alegría de vuestra dedicación unida a hacer de todo corazón la voluntad de mi Padre que está en los cielos. No necesitáis tener el mismo punto de vista, sentir de la misma manera o ni siquiera pensar de la misma manera, para \textit{ser iguales} espiritualmente. La unidad espiritual procede de la conciencia de que cada uno de vosotros está habitado, y cada vez más gobernado, por el don espiritual del Padre celestial. Vuestra armonía apostólica debe originarse en el hecho de que la esperanza espiritual de cada uno de vosotros es idéntica en su origen, naturaleza y destino>>\footnote{\textit{Se requiere unidad espiritual}: Ef 4:3-12. \textit{Unidad en la diversidad}: 1 Co 12:4-31. \textit{No sacar las creencias fuera de su esencia}: Mt 12:43-45; Lc 11:24-26.}.

\par 
%\textsuperscript{(1591.7)}
\textsuperscript{141:5.2} <<De esta manera podéis experimentar una unidad perfeccionada de intención espiritual y de comprensión espiritual, que tiene su origen en la conciencia mutua de la identidad de cada uno de vuestros espíritus paradisiacos internos; y podéis disfrutar toda esta profunda unidad espiritual en presencia misma de la extrema diversidad de vuestras actitudes individuales en lo referente a la reflexión intelectual, a los sentimientos propios de vuestro temperamento y a la conducta social. Vuestras personalidades pueden ser agradablemente variadas y notablemente diferentes, pero vuestras naturalezas espirituales y los frutos espirituales de vuestra adoración divina y de vuestro amor fraternal pueden estar tan unificados, que todos los que contemplen vuestra vida reconocerán con toda seguridad esta identidad de espíritu y esta unidad de alma. Reconocerán que habéis estado conmigo y que habéis aprendido así a hacer, de una manera aceptable, la voluntad del Padre que está en los cielos. Podéis conseguir la unidad en el servicio de Dios, aunque cada uno de vosotros cumpla ese servicio siguiendo la técnica de sus propias dotaciones originales de mente, de cuerpo y de alma>>.

\par 
%\textsuperscript{(1592.1)}
\textsuperscript{141:5.3} <<Vuestra unidad espiritual implica dos factores, que siempre se armonizarán en la vida de los creyentes individuales: En primer lugar, poseéis un motivo común para una vida de servicio; todos deseáis por encima de todo hacer la voluntad del Padre que está en los cielos. Y en segundo lugar, todos tenéis una meta común en la existencia; todos os proponéis encontrar al Padre que está en los cielos, mostrando así al universo que os habéis vuelto como él>>.

\par 
%\textsuperscript{(1592.2)}
\textsuperscript{141:5.4} Jesús volvió muchas veces sobre este tema durante la preparación de los doce. Les dijo repetidamente que no deseaba que los que creían en él se volvieran dogmatizados y uniformizados según las interpretaciones religiosas incluso de los hombres de bien. Una y otra vez previno a sus apóstoles contra la elaboración de credos y el establecimiento de tradiciones como medio de guiar y controlar a los creyentes en el evangelio del reino.

\section*{6. La última semana en Amatus}
\par 
%\textsuperscript{(1592.3)}
\textsuperscript{141:6.1} Hacia el final de la última semana en Amatus, Simón Celotes llevó ante Jesús a un tal Tejerma, un persa que hacía negocios en Damasco. Tejerma había oído hablar de Jesús y había venido a Cafarnaúm para verlo. Al enterarse de que Jesús se había ido con sus apóstoles bajando por el Jordán hacia Jerusalén, partió en su búsqueda. Andrés había presentado Tejerma a Simón para que lo instruyera. Simón consideraba al persa como un <<adorador del fuego>>, aunque Tejerma se esmeró en explicarle que el fuego sólo era el símbolo visible del Único Puro y Santo. Después de hablar con Jesús, el persa manifestó su intención de permanecer varios días para oír la enseñanza y escuchar la predicación.

\par 
%\textsuperscript{(1592.4)}
\textsuperscript{141:6.2} Cuando Simón Celotes y Jesús se quedaron solos, Simón le preguntó al Maestro: <<¿Por qué no he podido persuadirlo? ¿Por qué se ha resistido tanto conmigo y te ha escuchado tan rápidamente?>> Jesús respondió: <<Simón, Simón, ¿cuántas veces te he enseñado que dejes de esforzarte por \textit{extraer} algo del corazón de los que buscan la salvación? ¿Cuántas veces te he dicho que trabajes solamente para \textit{introducir} algo dentro de esas almas hambrientas? Conduce a los hombres hasta el reino, y las grandes verdades vivientes del reino pronto expulsarán todo error grave. Cuando hayas dado a conocer al hombre mortal la buena nueva de que Dios es su Padre, podrás persuadirlo más fácilmente de que es en realidad un hijo de Dios. Una vez hecho esto, habrás llevado la luz de la salvación a un ser que está en las tinieblas. Simón, cuando el Hijo del Hombre vino a ti por primera vez, ¿llegó acusando a Moisés y a los profetas para proclamar una manera de vivir nueva y mejor? No. No he venido para eliminar lo que poseéis de vuestros antepasados, sino para mostraros la visión completa de lo que vuestro padres sólo vieron en parte. Así pues Simón, ve a enseñar y a predicar el reino, y cuando tengas a un hombre a salvo y seguro en el reino, entonces será momento, si se acerca a ti con sus preguntas, de impartirle una enseñanza relacionada con el avance progresivo del alma dentro del reino divino>>.

\par 
%\textsuperscript{(1592.5)}
\textsuperscript{141:6.3} Simón se quedó asombrado con estas palabras, pero hizo lo que Jesús le había enseñado, y Tejerma el persa fue contado entre los que entraron en el reino.

\par 
%\textsuperscript{(1592.6)}
\textsuperscript{141:6.4} Aquella noche, Jesús dio un discurso a los apóstoles sobre la nueva vida en el reino. Dijo en parte: <<Cuando entráis en el reino, nacéis de nuevo. No podéis enseñar las cosas profundas del espíritu a los que sólo han nacido de la carne; primero cuidad de que los hombres nazcan de espíritu, antes de intentar instruirlos en los caminos avanzados del espíritu. No empecéis a mostrar a los hombres las bellezas del templo hasta que no hayan entrado primero dentro del templo. Presentad los hombres a Dios, \textit{como} hijos de Dios, antes de discurrir sobre las doctrinas de la paternidad de Dios y de la filiación de los hombres. No rivalicéis con los hombres ---sed siempre pacientes. El reino no es vuestro, sólo sois sus embajadores. Salid simplemente a proclamar: He aquí el reino de los cielos ---Dios es vuestro Padre y vosotros sois sus hijos, y si creéis de todo corazón, esta buena nueva \textit{es} vuestra salvación eterna>>\footnote{\textit{Nacer de nuevo}: Jn 3:1-12. \textit{No rivalicéis con los hombres}: 2 Ti 2:23-26.}.

\par 
%\textsuperscript{(1593.1)}
\textsuperscript{141:6.5} Los apóstoles hicieron grandes progresos durante la estancia en Amatus. Pero se sintieron muy decepcionados de que Jesús no les diera ninguna sugerencia sobre las relaciones con los discípulos de Juan. Incluso en la importante cuestión del bautismo, Jesús se limitó a decir: <<En verdad, Juan ha bautizado con agua, pero cuando entréis en el reino de los cielos, seréis bautizados con el Espíritu>>\footnote{\textit{Bautismo con el espíritu}: Mt 3:11; 28:19; Mc 1:8; Lc 3:16; Jn 1:32-33; Hch 1:5; 2:1-4,38; 10:47; 11:16.}.

\section*{7. En Betania más allá del Jordán}
\par 
%\textsuperscript{(1593.2)}
\textsuperscript{141:7.1} El 26 de febrero, Jesús, sus apóstoles y un grupo numeroso de discípulos viajaron siguiendo el Jordán hasta el vado cerca de Betania en Perea, el lugar donde Juan había proclamado por primera vez el reino venidero. Jesús permaneció allí con sus apóstoles, enseñando y predicando durante cuatro semanas, antes de partir para subir a Jerusalén.

\par 
%\textsuperscript{(1593.3)}
\textsuperscript{141:7.2} Durante la segunda semana de su estancia en Betania más allá del Jordán, Jesús se llevó a Pedro, Santiago y Juan para descansar tres días en las colinas situadas al otro lado del río, al sur de Jericó. El Maestro enseñó a estos tres hombres muchas verdades nuevas y avanzadas sobre el reino de los cielos. Dichas enseñanzas las hemos reorganizado y clasificado de la manera siguiente a efectos de este relato:

\par 
%\textsuperscript{(1593.4)}
\textsuperscript{141:7.3} Jesús procuró dejar muy claro que deseaba que sus discípulos, una vez que hubieran probado las buenas realidades espirituales del reino, vivieran de tal manera en el mundo que cuando los hombres \textit{vieran} sus vidas se volvieran conscientes del reino, y se sintieran así inducidos a preguntar a los creyentes sobre los caminos del reino. Todos estos buscadores sinceros de la verdad se alegran siempre de \textit{escuchar} la buena nueva del don de la fe que asegura la admisión en el reino, con sus realidades espirituales eternas y divinas.

\par 
%\textsuperscript{(1593.5)}
\textsuperscript{141:7.4} El Maestro intentó imprimir en el ánimo de todos los educadores del evangelio del reino que lo único que tenían que hacer era revelar al hombre individual que Dios es su Padre ---llevar a ese hombre individual a hacerse consciente de su filiación; y luego, presentar este mismo hombre a Dios como su hijo por la fe. Estas dos revelaciones esenciales se cumplían en Jesús. Él se convirtió, efectivamente, en <<el camino, la verdad y la vida>>\footnote{\textit{Jesús es el camino, la verdad y la vida}: Jn 14:6.}. La religión de Jesús estaba enteramente basada en la manera de vivir su vida de donación en la Tierra. Cuando Jesús se marchó de este mundo, no dejó detrás de él ni libros, ni leyes, ni otras formas de organización humana que afectaran la vida religiosa del individuo.

\par 
%\textsuperscript{(1593.6)}
\textsuperscript{141:7.5} Jesús indicó francamente que había venido para establecer unas relaciones personales y eternas con los hombres, que siempre tendrían prioridad sobre todas las demás relaciones humanas. Y recalcó que esta hermandad espiritual íntima debía extenderse a todos los hombres de todas las épocas y de todas las condiciones sociales, en todos los pueblos. La única recompensa que ofrecía a sus hijos era: en este mundo, la alegría espiritual y la comunión divina; y en el mundo siguiente, la vida eterna en el desarrollo de las realidades espirituales divinas del Padre Paradisiaco.

\par 
%\textsuperscript{(1593.7)}
\textsuperscript{141:7.6} Jesús hizo mucho hincapié en lo que él llamaba las dos verdades de primera importancia en las enseñanzas del reino, que son las siguientes: conseguir la salvación por medio de la fe, y de la fe solamente, asociada con la enseñanza revolucionaria de conseguir la libertad humana mediante el reconocimiento sincero de la verdad. <<Conoceréis la verdad y la verdad os hará libres>>\footnote{\textit{La verdad os hará libres}: Jn 8:32.}. Jesús era la verdad manifestada en la carne\footnote{\textit{Jesús era la verdad hecha manifiesta}: Jn 1:14.}, y prometió enviar a su Espíritu de la Verdad al corazón de todos sus hijos después de regresar al Padre que está en los cielos.

\par 
%\textsuperscript{(1594.1)}
\textsuperscript{141:7.7} El Maestro enseñaba a estos apóstoles los elementos esenciales de la verdad para toda una era de la Tierra. A menudo escuchaban sus enseñanzas, aunque lo que decía estaba destinado en realidad a inspirar y edificar a otros mundos. Dio ejemplo de un plan de vida nuevo y original. Desde el punto de vista humano era en verdad un judío, pero vivió su vida para todo el planeta como un mortal del mundo.

\par 
%\textsuperscript{(1594.2)}
\textsuperscript{141:7.8} Para estar seguro de que su Padre sería reconocido durante el desarrollo del plan del reino, Jesús explicó que había ignorado adrede a los <<grandes de la Tierra>>. Empezó su trabajo con los pobres\footnote{\textit{Empezar su trabajo con los pobres}: Mt 11:5; Lc 4:18; 7:22; 14:13.}, la clase que precisamente había sido tan desdeñada por la mayoría de las religiones evolutivas de las épocas anteriores. No despreciaba a ninguna persona; su plan era mundial, e incluso universal. Fue tan audaz y enérgico en estas declaraciones, que incluso Pedro, Santiago y Juan estuvieron tentados de creer que quizás había perdido el juicio\footnote{\textit{Jesús visto como fuera de sí}: Mc 3:21.}.

\par 
%\textsuperscript{(1594.3)}
\textsuperscript{141:7.9} Intentó impartir suavemente a estos apóstoles la verdad de que había venido a esta misión donadora, no para dar un ejemplo a algunas criaturas de la Tierra, sino para establecer y demostrar un modelo de vida humana para todos los pueblos de todos los mundos en todo su universo. Este modelo de vida se acercaba a la perfección más alta, incluso a la bondad final del Padre Universal. Pero los apóstoles no podían comprender el significado de sus palabras.

\par 
%\textsuperscript{(1594.4)}
\textsuperscript{141:7.10} Declaró que había venido para ejercer como instructor, un instructor enviado del cielo\footnote{\textit{Instructor venido del cielo}: Mt 11:1; 22:16; Mc 4:1; 6:2,34; 8:31; Jn 3:2; Hch 1:1.} para presentar la verdad espiritual a la mente material. Y esto es exactamente lo que hizo. Era un instructor, no un predicador. Desde el punto de vista humano, Pedro era un predicador mucho más eficaz que Jesús. Si la predicación de Jesús era tan eficaz, se debía más a su personalidad excepcional que a una irresistible atracción oratoria o emocional. Jesús hablaba directamente al alma de los hombres. Instruía al espíritu del hombre, pero a través de la mente. Vivía con los hombres.

\par 
%\textsuperscript{(1594.5)}
\textsuperscript{141:7.11} Fue en esta ocasión cuando Jesús insinuó a Pedro, Santiago y Juan que su trabajo en la Tierra estaba limitado en algunos aspectos por encargo de su <<asociado de arriba>>, refiriéndose a las instrucciones recibidas de su hermano paradisiaco Emmanuel antes de la donación. Les dijo que había venido para hacer la voluntad de su Padre\footnote{\textit{Jesús vivió la voluntad del Padre}: Mt 26:39,42,44; Mc 14:36,39; Lc 22:42; Jn 4:34; 5:30; 6:38-40; 15:10; 17:4.}, y únicamente la voluntad de su Padre. Como estaba motivado así por una sola intención sincera, no se preocupaba ansiosamente por el mal en el mundo.

\par 
%\textsuperscript{(1594.6)}
\textsuperscript{141:7.12} Los apóstoles empezaban a reconocer la amistad sin afectación de Jesús. Aunque era fácil acercarse al Maestro, siempre vivía independientemente de todos los seres humanos, y por encima de ellos. Nunca estuvo dominado ni un solo momento por una influencia puramente humana, o sujeto al frágil juicio humano. No prestaba ninguna atención a la opinión pública y no se dejaba influir por los elogios. Rara vez se interrumpió para corregir malentendidos o para ofenderse por una tergiversación. Nunca le pidió consejo a nadie; nunca solicitó oraciones.

\par 
%\textsuperscript{(1594.7)}
\textsuperscript{141:7.13} Santiago estaba asombrado por la manera en que Jesús parecía ver el fin desde el principio. El Maestro rara vez parecía sorprenderse. Nunca estaba excitado, enojado o desconcertado. Nunca pidió disculpas a nadie. A veces estaba triste, pero nunca desanimado.

\par 
%\textsuperscript{(1594.8)}
\textsuperscript{141:7.14} Juan percibió más claramente que, a pesar de todos sus atributos divinos, después de todo Jesús era humano\footnote{\textit{Jesús era ``humano''}: Jn 1:14; 1 Ti 2:5.}. Jesús vivía como un hombre entre los hombres, y los comprendía, los amaba y sabía cómo dirigirlos. En su vida personal era tan humano, y sin embargo tan irreprochable. Y siempre era desinteresado.

\par 
%\textsuperscript{(1595.1)}
\textsuperscript{141:7.15} Aunque Pedro, Santiago y Juan no pudieron comprender gran cosa de lo que Jesús dijo en esta ocasión, sus palabras bondadosas se grabaron en sus corazones, y después de la crucifixión y la resurrección, surgieron abundantemente para enriquecer y alegrar su ministerio posterior. No es de extrañar que estos apóstoles no comprendieran plenamente las palabras del Maestro, porque estaba delineando ante ellos el plan de una nueva era.

\section*{8. Trabajo en Jericó}
\par 
%\textsuperscript{(1595.2)}
\textsuperscript{141:8.1} Durante las cuatro semanas de estancia en Betania más allá del Jordán, Andrés designó varias veces por semana a unas parejas apostólicas para que subieran uno o dos días a Jericó. Juan tenía muchos creyentes en Jericó, y la mayoría de ellos acogieron con placer las enseñanzas más avanzadas de Jesús y sus apóstoles. Durante estas visitas a Jericó, los apóstoles empezaron a llevar a cabo más expresamente las instrucciones de Jesús de ayudar a los enfermos\footnote{\textit{Los apóstoles ayudando a los enfermos}: Mt 10:8; Lc 10:9.}; visitaron cada casa de la ciudad y trataron de confortar a todas las personas afligidas.

\par 
%\textsuperscript{(1595.3)}
\textsuperscript{141:8.2} Los apóstoles efectuaron alguna labor pública en Jericó, pero sus esfuerzos fueron principalmente de naturaleza más tranquila y personal. Ahora hicieron el descubrimiento de que la buena nueva del reino reconfortaba mucho a los enfermos, que su mensaje llevaba la curación a los afligidos. Fue en Jericó donde los doce pusieron en práctica, por primera vez, el encargo de Jesús de predicar la buena nueva del reino y de atender a los afligidos.

\par 
%\textsuperscript{(1595.4)}
\textsuperscript{141:8.3} Se detuvieron en Jericó, de camino hacia Jerusalén, y fueron alcanzados por una delegación de Mesopotamia que había venido para hablar con Jesús. Los apóstoles habían proyectado pasar un solo día allí, pero cuando llegaron estos buscadores orientales de la verdad, Jesús pasó tres días con ellos. Éstos últimos regresaron a sus diversos hogares, a lo largo del Éufrates, con la felicidad de conocer las nuevas verdades del reino de los cielos.

\section*{9. La partida hacia Jerusalén}
\par 
%\textsuperscript{(1595.5)}
\textsuperscript{141:9.1} El último día de marzo, un lunes, Jesús y los apóstoles emprendieron la subida de las colinas hacia Jerusalén. Lázaro de Betania había bajado dos veces al Jordán para ver a Jesús, y se habían tomado todas las disposiciones necesarias para que el Maestro y sus apóstoles instalaran su cuartel general en la casa de Lázaro y sus hermanas, en Betania, durante todo el tiempo que desearan quedarse en Jerusalén.

\par 
%\textsuperscript{(1595.6)}
\textsuperscript{141:9.2} Los discípulos de Juan permanecieron en Betania más allá del Jordán, enseñando y bautizando a las multitudes, de manera que Jesús sólo iba acompañado de los doce cuando llegó a casa de Lázaro. Jesús y los apóstoles se detuvieron allí durante cinco días, descansando y reponiéndose, antes de continuar hacia Jerusalén para la Pascua. Fue un gran acontecimiento en la vida de Marta y María tener al Maestro y a sus apóstoles en el hogar de su hermano, donde pudieron atender sus necesidades.

\par 
%\textsuperscript{(1595.7)}
\textsuperscript{141:9.3} El domingo 6 de abril por la mañana, Jesús y los apóstoles bajaron a Jerusalén\footnote{\textit{A Jerusalén}: Jn 2:23.}; ésta era la primera vez que el Maestro y los doce se encontraban allí todos juntos.


\chapter{Documento 142. La pascua en Jerusalén}
\par 
%\textsuperscript{(1596.1)}
\textsuperscript{142:0.1} DURANTE el mes de abril, Jesús y los apóstoles trabajaron en Jerusalén, saliendo de la ciudad todas las tardes para pasar la noche en Betania. El mismo Jesús pasó una o dos noches por semana en Jerusalén en la casa de Flavio, un judío griego, donde muchos judíos eminentes venían en secreto para entrevistarse con él.

\par 
%\textsuperscript{(1596.2)}
\textsuperscript{142:0.2} El primer día en Jerusalén, Jesús visitó al antiguo sumo sacerdote Anás, su amigo de años atrás y pariente de Salomé, la esposa de Zebedeo. Anás había oído hablar de Jesús y de sus enseñanzas, y cuando Jesús llamó a la casa del sumo sacerdote, fue recibido con mucha reserva. Cuando Jesús percibió la frialdad de Anás, se despidió inmediatamente, diciéndole al marcharse: <<El miedo es el principal tirano del hombre, y el orgullo, su mayor debilidad; ¿te entregarás tú mismo a la esclavitud de estos dos destructores de la alegría y de la libertad?>> Pero Anás no respondió. El Maestro no lo volvió a ver hasta el momento en que Anás se sentó con su yerno para juzgar al Hijo del Hombre.

\section*{1. La enseñanza en el templo}
\par 
%\textsuperscript{(1596.3)}
\textsuperscript{142:1.1} Durante todo este mes, Jesús o uno de los apóstoles enseñaron diariamente en el templo. Cuando el gentío pascual era demasiado numeroso como para entrar en el templo y escuchar la enseñanza, los apóstoles dirigían muchos grupos educativos fuera de los recintos sagrados. Lo esencial de su mensaje era:

\par 
%\textsuperscript{(1596.4)}
\textsuperscript{142:1.2} 1. El reino de los cielos está cerca\footnote{\textit{El Reino de Dios está cerca}: Mt 3:2; 4:17; 10:7; Mc 1:15; Lc 10:9,11; 17:20-21; 21:31.}.

\par 
%\textsuperscript{(1596.5)}
\textsuperscript{142:1.3} 2. Podéis entrar en el reino de los cielos mediante vuestra fe en la paternidad de Dios, convirtiéndoos así en los hijos de Dios\footnote{\textit{Hijos de Dios por la fe}: 1 Cr 22:10; Sal 2:7; Is 56:5; Mt 5:9,16,45; Lc 20:36; Jn 1:12-13; 11:52; Hch 17:28-29; Ro 8:14-17,19,21; 9:26; 2 Co 6:18; Gl 3:26; 4:5-7; Ef 1:5; Flp 2:15; Heb 12:5-8; 1 Jn 3:1-2,10; 5:2; Ap 21:7; 2 Sam 7:14.}.

\par 
%\textsuperscript{(1596.6)}
\textsuperscript{142:1.4} 3. El amor es la regla de vida dentro del reino ---la suprema devoción a Dios mientras que amáis a vuestro prójimo como a vosotros mismos\footnote{\textit{Amar a Dios y al prójimo como a uno mismo}: Mt 5:43-44; 19:19; 22:36-39; Mc 12:28-33; Lc 10:25-27.}.

\par 
%\textsuperscript{(1596.7)}
\textsuperscript{142:1.5} 4. La ley del reino es la obediencia a la voluntad del Padre, la cual produce los frutos del espíritu en vuestra vida personal\footnote{\textit{Frutos del espíritu}: Gl 5:22-23; Ef 5:9.}.

\par 
%\textsuperscript{(1596.8)}
\textsuperscript{142:1.6} Las multitudes que vinieron a celebrar la Pascua escucharon esta enseñanza de Jesús, y centenares de ellos se regocijaron con la buena nueva\footnote{\textit{Las multitudes se regocijan con la buena nueva}: Jn 2:23.}. Los principales sacerdotes y dirigentes de los judíos empezaron a interesarse mucho por Jesús y sus apóstoles, y discutieron entre sí sobre lo que debían hacer con ellos.

\par 
%\textsuperscript{(1596.9)}
\textsuperscript{142:1.7} Además de enseñar dentro y fuera del templo, los apóstoles y otros creyentes se ocupaban de hacer mucho trabajo personal entre las multitudes de la Pascua. Estos hombres y mujeres interesados en el mensaje de Jesús llevaron las nuevas que escucharon durante esta celebración pascual hasta los lugares más alejados del imperio romano, y también a oriente. Éste fue el principio de la difusión del evangelio del reino en el mundo exterior. El trabajo de Jesús ya no iba a limitarse a Palestina.

\section*{2. La ira de Dios}
\par 
%\textsuperscript{(1597.1)}
\textsuperscript{142:2.1} Se encontraba en Jerusalén, asistiendo a las festividades de la Pascua, un rico negociante judío de Creta llamado Jacobo, que fue hasta Andrés para pedirle ver a Jesús en privado. Andrés arregló este encuentro secreto con Jesús en la casa de Flavio para el día siguiente al anochecer. Este hombre no podía comprender las enseñanzas del Maestro, y venía porque deseaba indagar más plenamente sobre el reino de Dios. Jacobo le dijo a Jesús: <<Pero, Rabino, Moisés y los antiguos profetas nos dicen que Yahvé es un Dios celoso, un Dios con una gran ira y un intenso furor. Los profetas dicen que odia a los malhechores y que se venga de los que no obedecen su ley. Tú y tus discípulos nos enseñáis que Dios es un Padre benévolo y compasivo que ama tanto a todos los hombres que los acogería con agrado en este nuevo reino de los cielos que tú proclamas tan cercano>>\footnote{\textit{Moisés dijo que Dios es celoso}: Ex 20:5; 34:14; Nah 1:2; Dt 6:15; 32:16,21; Jos 24:19. \textit{Un Dios de gran ira e intenso furor}: Ex 22:24; 2 Re 22:13; Lv 10:6; Nm 11:33; Dt 9:7-8,22. \textit{Que se venga de los pecadores}: Sal 58:10; Is 34:8; Dt 32:35,41,43; Jue 11:36.}.

\par 
%\textsuperscript{(1597.2)}
\textsuperscript{142:2.2} Cuando Jacobo terminó de hablar, Jesús contestó: <<Jacobo, has expuesto muy bien las enseñanzas de los antiguos profetas, que instruyeron a los hijos de su generación de acuerdo con las luces de su tiempo. Nuestro Padre que está en el Paraíso es invariable. Pero el concepto de su naturaleza se ha ampliado y ha crecido desde la época de Moisés hasta los tiempos de Amós, e incluso hasta la generación del profeta Isaías. Ahora, yo he venido en forma carnal para revelar el Padre con una nueva gloria y dar a conocer su amor y su misericordia a todos los hombres de todos los mundos. A medida que el evangelio de este reino se divulgue por el mundo con su mensaje de felicidad y de buena voluntad para todos los hombres, nacerán unas relaciones mejores y superiores entre las familias de todas las naciones. A medida que pase el tiempo, los padres y sus hijos se amarán más los unos a los otros, y esto producirá una mayor comprensión del amor del Padre que está en los cielos por sus hijos de la Tierra. Recuerda, Jacobo, que un padre bueno y verdadero no solamente ama a su familia como un todo ---como una familia--- sino que también ama de verdad y cuida con afecto a cada miembro \textit{individual}>>.

\par 
%\textsuperscript{(1597.3)}
\textsuperscript{142:2.3} Después de mucho discutir sobre el carácter del Padre celestial, Jesús se detuvo para decir: <<Tú, Jacobo, como eres padre de una familia numerosa, conoces bien la verdad de mis palabras>>. Y Jacobo dijo: <<Pero Maestro, ¿quién te ha dicho que soy padre de seis hijos? ¿Cómo sabías esto de mí?>> Y el Maestro contestó: <<Basta con decir que el Padre y el Hijo conocen todas las cosas, porque en verdad lo ven todo. Puesto que amas a tus hijos como un padre terrestre, ahora debes aceptar como una realidad el amor del Padre celestial por \textit{ti} ---no solamente por todos los hijos de Abraham, sino por ti, por tu alma individual>>.

\par 
%\textsuperscript{(1597.4)}
\textsuperscript{142:2.4} Jesús continuó diciendo: <<Cuando tus hijos son muy jóvenes e inmaduros, y has de castigarlos, pueden pensar que su padre está enojado y lleno de ira resentida. Su inmadurez no les permite penetrar más allá del castigo para discernir el afecto previsor y correctivo del padre. Pero cuando estos mismos hijos se vuelven hombres y mujeres adultos, ¿no sería insensato por su parte agarrarse a estos conceptos antiguos y equivocados sobre su padre? Como hombres y mujeres, deberían discernir ahora el amor de su padre en todas estas correcciones de los primeros años. A medida que transcurren los siglos, ¿no debería la humanidad llegar a comprender mejor la verdadera naturaleza y el carácter amoroso del Padre que está en los cielos? ¿Qué provecho sacáis de la iluminación espiritual de las generaciones sucesivas, si persistís en ver a Dios como lo veían Moisés y los profetas? Te digo, Jacobo, que a la brillante luz de esta hora, deberías ver al Padre como ninguno de tus antecesores lo han contemplado nunca. Al verlo de esta manera, deberías regocijarte por entrar en un reino donde gobierna un Padre tan misericordioso, y deberías procurar que su voluntad de amor domine tu vida de aquí en adelante>>.

\par 
%\textsuperscript{(1598.1)}
\textsuperscript{142:2.5} Y Jacobo contestó: <<Rabino, yo creo; deseo que me conduzcas al reino del Padre>>.

\section*{3. El concepto de Dios}
\par 
%\textsuperscript{(1598.2)}
\textsuperscript{142:3.1} La mayoría de los doce apóstoles habían escuchado este debate sobre el carácter de Dios, y aquella noche hicieron muchas preguntas a Jesús sobre el Padre que está en los cielos. La mejor manera de presentar las respuestas del Maestro a estas preguntas consiste en resumirlas de la manera siguiente con un lenguaje moderno:

\par 
%\textsuperscript{(1598.3)}
\textsuperscript{142:3.2} Jesús reprendió suavemente a los doce, diciéndoles en esencia: ¿No conocéis las tradiciones de Israel relacionadas con el crecimiento de la idea de Yahvé, e ignoráis la enseñanza de las Escrituras sobre la doctrina de Dios? Luego el Maestro empezó a instruir a los apóstoles sobre la evolución del concepto de la Deidad a lo largo de todo el desarrollo del pueblo judío. Llamó su atención sobre las siguientes fases del crecimiento de la idea de Dios:

\par 
%\textsuperscript{(1598.4)}
\textsuperscript{142:3.3} 1. \textit{Yahvé} ---El dios de los clanes del Sinaí\footnote{\textit{Yahvé, el dios de los clanes del Sinaí}: Gn 22:14; Ex 6:3; Sal 83:18; Is 12:2; 26:4.}. Éste era el concepto primitivo de la Deidad, que Moisés elevó al nivel superior de Señor Dios de Israel. El Padre que está en los cielos nunca deja de aceptar la adoración sincera de sus hijos de la Tierra, por muy tosco que sea su concepto de la Deidad o el nombre con que simbolizan su naturaleza divina.

\par 
%\textsuperscript{(1598.5)}
\textsuperscript{142:3.4} 2. \textit{El Altísimo}. Este concepto del Padre que está en los cielos fue proclamado por Melquisedek a Abraham, y desde Salem fue llevado muy lejos por aquellos que creyeron posteriormente en esta idea ampliada y expandida de la Deidad. Abraham y su hermano se fueron de Ur\footnote{\textit{Abraham abandona Ur}: Gn 11:31.} porque se había establecido allí la adoración del Sol, y se volvieron creyentes en las enseñanzas de Melquisedek sobre El Elyón ---el Dios Altísimo\footnote{\textit{El Elyón, el Dios Altísimo}: Gn 14:18-22; Heb 7:1.}. Tenían un concepto compuesto de Dios, consistente en una mezcla de sus antiguas ideas mesopotámicas y de la doctrina del Altísimo.

\par 
%\textsuperscript{(1598.6)}
\textsuperscript{142:3.5} 3. \textit{El Shaddai}\footnote{\textit{El Shaddai, Deidad creadora}: Gn 17:1; 28:3; Ex 6:3.}. Durante aquellos tiempos primitivos, muchos hebreos adoraban a El Shaddai, el concepto egipcio del Dios del cielo, que habían aprendido durante su cautiverio en la tierra del Nilo. Mucho tiempo después de la época de Melquisedek, estos tres conceptos de Dios se fundieron en uno solo para formar la doctrina de la Deidad creadora, el Señor Dios de Israel.

\par 
%\textsuperscript{(1598.7)}
\textsuperscript{142:3.6} 4. \textit{Elohim}\footnote{\textit{Elohim (los Dioses) crearon}: Gn 1:1; Ex 6:2.}. La enseñanza sobre la Trinidad del Paraíso ha sobrevivido desde los tiempos de Adán. ¿No recordáis que las Escrituras empiezan afirmando que <<En el principio, los Dioses crearon los cielos y la Tierra>>? Esto indica que cuando se escribió este pasaje, el concepto trinitario de tres Dioses en uno había encontrado su lugar en la religión de nuestros antepasados.

\par 
%\textsuperscript{(1598.8)}
\textsuperscript{142:3.7} 5. \textit{El Yahvé Supremo}. En los tiempos de Isaías, estas creencias sobre Dios se habían ampliado hasta el concepto de un Creador Universal\footnote{\textit{Creador Universal}: Gn 1:1-27; 2:4-23; 5:1-2; Ex 20:11; 31:17; 2 Re 19:15; 2 Cr 2:12; Neh 9:6; Sal 115:15; 121:2; 124:8; 134:3; 146:6; Eclo 1:1-4; 33:10; Is 37:16; 40:26,28; 42:5; 45:12,18; Jer 10:11-12; 32:17; 51:15; Bar 3:32-36; Am 4:13; Mal 2:10; Mc 13:19; Jn 1:10-3; Hch 4:24; 14:15; Ef 3:9; Col 1:16; Heb 1:2; 1 P 4:19; Ap 4:11; 10:6; 14:7.} que era a la vez todopoderoso\footnote{\textit{Todopoderoso}: Ex 9:16; 15:6; 1 Cr 29:11-12; Neh 1:10; Job 36:22; 37:23; Sal 59:16; 106:8; 111:6; 147:5; Jer 10:12; 27:5; 32:17; 51:15; Nm 14:17; Nah 1:3; Dt 9:29; Mt 28:18; 2 Sam 22:33.} y totalmente misericordioso\footnote{\textit{Totalmente misericordioso}: Ex 20:6; 1 Cr 16:34,41; 2 Cr 5:13; 7:3,6; 30:9; Esd 3:11; Sal 25:6; 36:o5; 86:5,13,15; 100:5; 103:8,11,17; 107:1; 116:5; 117:2; 118:1,4; 136:1-26; 145:8; Is 54:8; 55:7; Jer 3:12; Nm 14:18-19; Miq 7:18; Dt 4:31; 5:10; Heb 8:12.}. Este concepto de Dios, en vías de evolución y ampliación, suplantó en la práctica todas las ideas anteriores que la religión de nuestros padres tenía sobre la Deidad.

\par 
%\textsuperscript{(1598.9)}
\textsuperscript{142:3.8} 6. \textit{El Padre que está en los cielos}\footnote{\textit{El Padre que está en los Cielos}: Mt 5:9,16,45,48; 6:1,9,14; 6:26,32; 7:11,21; 10:32-33; 11:25; 12:50; 15:13; 16:17; 18:10,14,19,35; 23:9; Mc 11:25-26; Lc 10:21; 11:2,13.}. Y ahora, conocemos a Dios como nuestro Padre que está en los cielos. Nuestra enseñanza proporciona una religión en la que el creyente \textit{es} un hijo de Dios. Ésta es la buena nueva del evangelio del reino de los cielos. El Hijo y el Espíritu coexisten con el Padre, y la revelación de la naturaleza y del ministerio de estas Deidades del Paraíso continuará ampliándose y clarificándose a lo largo de las eras sin fin de la progresión espiritual eterna de los hijos ascendentes de Dios. En todos los tiempos y durante todas las épocas, la adoración verdadera de cualquier ser humano ---respecto al progreso espiritual individual--- es reconocida por el espíritu interior como un homenaje que se rinde al Padre que está en los cielos.

\par 
%\textsuperscript{(1599.1)}
\textsuperscript{142:3.9} Los apóstoles nunca se habían sentido antes tan conmocionados como al escuchar este relato del crecimiento del concepto de Dios en la mente judía de las generaciones anteriores; estaban demasiado aturdidos como para hacer preguntas. Mientras permanecían sentados en silencio delante de Jesús, el Maestro continuó: <<Habríais conocido estas verdades si hubierais leído las Escrituras. ¿No habéis leído lo que se dice en Samuel: `Y la ira del Señor se encendió contra Israel, de tal manera que incitó a David contra ellos, diciéndole que fuera a contar a Israel y a Judá'? Esto no era de extrañar, porque en la época de Samuel, los hijos de Abraham creían realmente que Yahvé creaba tanto el bien como el mal. Pero cuando un escritor posterior narró estos acontecimientos, después de la ampliación del concepto judío sobre la naturaleza de Dios, no se atrevió a atribuir el mal a Yahvé, y por esta razón dijo: `Y Satanás se levantó contra Israel, e incitó a David para que contara a Israel'. ¿No podéis discernir que estos relatos de las Escrituras muestran claramente cómo continuó creciendo el concepto de la naturaleza de Dios de una generación a la siguiente?>>\footnote{\textit{El Señor pidió a David hacer recuento del pueblo}: 2 Sam 24:1. \textit{Satanás pidió a David hacer recuento del pueblo}: 1 Cr 21:1.}

\par 
%\textsuperscript{(1599.2)}
\textsuperscript{142:3.10} <<También deberíais haber percibido el crecimiento de la comprensión de la ley divina, en perfecta congruencia con estos conceptos ampliados de la divinidad. Cuando los hijos de Israel salieron de Egipto, en una fecha anterior a la revelación ampliada de Yahvé, tenían diez mandamientos que les sirvieron de ley hasta la época en que acamparon delante del Sinaí. Estos diez mandamientos eran:>>

\par 
%\textsuperscript{(1599.3)}
\textsuperscript{142:3.11} <<1. No adoraréis a ningún otro dios, porque el Señor es un Dios celoso>>\footnote{\textit{No adorarás a ningún otro dios}: Ex 20:3; Ex 34:14; Dt 5:7.}.

\par 
%\textsuperscript{(1599.4)}
\textsuperscript{142:3.12} <<2. No fundiréis imágenes de dioses>>\footnote{\textit{No fundiréis imágenes de los dioses}: Ex 20:4; Ex 34:17; Dt 5:8.}.

\par 
%\textsuperscript{(1599.5)}
\textsuperscript{142:3.13} <<3. No dejaréis de guardar la fiesta del pan ázimo>>\footnote{\textit{Guardaréis la fiesta de los ácimos}: Ex 12:6-11; Ex 23:15; Ex 34:18; Dt 16:1-4.}.

\par 
%\textsuperscript{(1599.6)}
\textsuperscript{142:3.14} <<4. Todos los machos primogénitos de los hombres y de los animales me pertenecen, dice el Señor>>\footnote{\textit{Todos los machos primogénitos son del Señor}: Ex 13:2,12; 34:19-20; Nm 18:15.}.

\par 
%\textsuperscript{(1599.7)}
\textsuperscript{142:3.15} <<5. Podéis trabajar seis días, pero el séptimo descansaréis>>\footnote{\textit{El séptimo día descansaréis}: Ex 20:8-11; 34:21; Dt 5:12-15.}.

\par 
%\textsuperscript{(1599.8)}
\textsuperscript{142:3.16} <<6. No dejaréis de guardar la fiesta de las primeras frutas y la fiesta de la cosecha a final de año>>\footnote{\textit{Celebraréis dos fiestas}: Ex 23:16; 34:22.}.

\par 
%\textsuperscript{(1599.9)}
\textsuperscript{142:3.17} <<7. No ofreceréis la sangre de ningún sacrificio con pan fermentado>>\footnote{\textit{No sacrificaréis con pan fermentado}: Ex 23:18a; 34:25a.}.

\par 
%\textsuperscript{(1599.10)}
\textsuperscript{142:3.18} <<8. El sacrificio de la fiesta de la Pascua no se dejará allí hasta por la mañana>>\footnote{\textit{No dejáreis comida pascual para el día siguiente}: Ex 23:18b; 34:25b.}.

\par 
%\textsuperscript{(1599.11)}
\textsuperscript{142:3.19} <<9. Llevaréis a la casa del Señor vuestro Dios las primicias de los primeros frutos de la tierra>>\footnote{\textit{Llevaréis las primicias de los frutos}: Ex 23:16,19a; 34:22,26a.}.

\par 
%\textsuperscript{(1599.12)}
\textsuperscript{142:3.20} <<10. No herviréis un cabrito en la leche de su madre>>\footnote{\textit{No herviréis un cabrito en la leche de su madre}: Ex 23:19b; 34:26b; Dt 14:21.}.

\par 
%\textsuperscript{(1599.13)}
\textsuperscript{142:3.21} <<Luego, en medio de los truenos y los relámpagos del Sinaí, Moisés les dio los nuevos diez mandamientos, y todos admitiréis que son unas expresiones más dignas de acompañar los conceptos ampliados de la Deidad, representados como Yahvé. ¿No habéis observado nunca que estos mandamientos están registrados dos veces en las Escrituras? En el primer caso, la liberación de Egipto se señala como razón para guardar el sábado, mientras que en un escrito posterior, las creencias religiosas en progreso de nuestros antepasados exigieron que este texto fuera cambiado para reconocer el hecho de la creación como motivo para respetar el sábado>>\footnote{\textit{El monte Sinaí}: Ex 19:16-18; Dt 5:4-5. \textit{Los diez mandamientos}: Ex 20:1-17; Dt 5:6-21. \textit{Salida de Egipto}: Dt 5:15. \textit{Creación en seis días}: Ex 20:11.}.

\par 
%\textsuperscript{(1599.14)}
\textsuperscript{142:3.22} <<Y luego, recordaréis que una vez más ---en la época de Isaías, cuando había una mayor iluminación espiritual--- estos diez mandamientos negativos fueron cambiados por la gran ley positiva del amor, por el precepto de amar a Dios de manera suprema y a vuestro prójimo como a vosotros mismos. Yo también os declaro que esta ley suprema del amor a Dios y a los hombres constituye todo el deber de los hombres>>\footnote{\textit{Cambiados por la ley del amor}: Is 38:17; 55:4-7; 63:7-9. \textit{Amar a Dios y al prójimo}: Mt 5:43-44; 19:19; 22:36-40; Mc 12:28-33; Lc 10:25-27.}.

\par 
%\textsuperscript{(1600.1)}
\textsuperscript{142:3.23} Cuando terminó de hablar, nadie le hizo ninguna pregunta. Y cada uno de ellos se retiró para descansar.

\section*{4. Flavio y la cultura griega}
\par 
%\textsuperscript{(1600.2)}
\textsuperscript{142:4.1} Flavio, el judío griego, era un prosélito sin acceso al templo, pues no había sido circuncidado ni bautizado. Como apreciaba mucho la belleza en el arte y la escultura, la casa que ocupaba durante su estancia en Jerusalén era un hermoso edificio. Este hogar estaba exquisitamente adornado con tesoros inapreciables que había rebuscado aquí y allá en sus viajes por el mundo. Cuando pensó por primera vez en invitar a Jesús a su casa, temía que el Maestro pudiera ofenderse al ver aquellas pretendidas imágenes. Pero cuando Jesús entró en la casa, Flavio se quedó agradablemente sorprendido ya que, en lugar de reprenderle por tener aquellos objetos supuestamente idólatras esparcidos por toda la casa, manifestó un gran interés por toda la colección, y mostró su aprecio haciendo muchas preguntas sobre cada objeto, mientras que Flavio lo acompañaba de una habitación a otra, mostrándole sus estatuas favoritas.

\par 
%\textsuperscript{(1600.3)}
\textsuperscript{142:4.2} El Maestro vio que su anfitrión estaba aturdido por su actitud favorable hacia el arte; por consiguiente, cuando terminaron de examinar toda la colección, Jesús dijo: <<Puesto que sabes apreciar la belleza de las cosas creadas por mi Padre y modeladas por las manos artísticas del hombre, ¿por qué esperabas recibir una reprimenda? Porque Moisés intentó en otra época combatir la idolatría y la adoración de los falsos dioses, ¿por qué todos los hombres han de rechazar la reproducción de la gracia y de la belleza? Te digo, Flavio, que los hijos de Moisés lo han comprendido mal, y ahora convierten en falsos dioses hasta sus prohibiciones de las imágenes y de los retratos de las cosas del cielo y de la tierra. Pero, aunque Moisés enseñara estas restricciones a las mentes ignorantes de aquellos tiempos, ¿qué tienen que ver con nuestra época, en la que el Padre que está en los cielos es revelado como el Soberano Espiritual universal por encima de todo? Flavio, te aseguro que en el reino venidero ya no continuarán enseñando `No adoréis esto y no adoréis aquello'; ya no se ocuparán de ordenar que os abstengáis de esto y que tengáis cuidado de no hacer aquello, sino que todos se ocuparán más bien de un solo deber supremo. Y este deber de los hombres está expresado en dos grandes privilegios: la adoración sincera del Creador infinito, el Padre Paradisiaco, y el servicio amoroso otorgado a nuestros semejantes. Si amas a tu prójimo como a ti mismo, sabes realmente que eres un hijo de Dios>>\footnote{\textit{Moisés combatió la idolatría}: Ex 20:4; Dt 5:8. \textit{Cambiados por la ley del amor}: Is 38:17; 55:4-7; 63:7-9. \textit{Amar a Dios y al prójimo}: Mt 5:43-44; 19:19; 22:36-40; Mc 12:28-33; Lc 10:25-27.}.

\par 
%\textsuperscript{(1600.4)}
\textsuperscript{142:4.3} <<En una época en que mi Padre no era bien comprendido, las tentativas de Moisés por oponerse a la idolatría estaban justificadas, pero en la era por venir, el Padre habrá sido revelado en la vida del Hijo; y esta nueva revelación de Dios hará que sea perpetuamente inútil confundir al Padre Creador con los ídolos de piedra o las imágenes de oro y plata. En lo sucesivo, los hombres inteligentes podrán disfrutar de los tesoros del arte, sin confundir esta apreciación material de la belleza con la adoración y el servicio del Padre Paradisiaco, el Dios de todas las cosas y de todos los seres>>.

\par 
%\textsuperscript{(1600.5)}
\textsuperscript{142:4.4} Flavio creyó todo lo que Jesús le enseñó. Al día siguiente se dirigió a Betania más allá del Jordán y fue bautizado por los discípulos de Juan. Hizo esto porque los apóstoles de Jesús aún no bautizaban a los creyentes. Cuando Flavio regresó a Jerusalén, dio una gran fiesta para Jesús e invitó a sesenta de sus amigos. Muchos de estos convidados también se hicieron creyentes en el mensaje del reino venidero.

\section*{5. El discurso sobre la seguridad}
\par 
%\textsuperscript{(1601.1)}
\textsuperscript{142:5.1} Uno de los grandes sermones que Jesús predicó en el templo, durante esta semana de la Pascua, fue en respuesta a una pregunta que hizo uno de sus oyentes, un hombre de Damasco. Este hombre preguntó a Jesús: <<Pero, Rabino, ¿cómo sabremos con certidumbre que has sido enviado por Dios, y que podemos entrar realmente en ese reino que tú y tus discípulos afirmáis que está cerca?>> Y Jesús contestó:

\par 
%\textsuperscript{(1601.2)}
\textsuperscript{142:5.2} <<En cuanto a mi mensaje y a las enseñanzas de mis discípulos, debéis juzgarlos por sus frutos. Si os proclamamos las verdades del espíritu, el espíritu atestiguará en vuestro corazón que nuestro mensaje es auténtico. En lo referente al reino y a vuestra seguridad de que seréis aceptados por el Padre celestial, permitidme preguntaros ¿habría entre vosotros algún padre, digno de ese nombre y de buen corazón, que mantuviera a su hijo en la ansiedad o la duda en cuanto a su posición dentro de la familia o a su grado de seguridad en el afecto del corazón de su padre? ¿Acaso vosotros, los padres terrestres, disfrutáis torturando a vuestros hijos con incertidumbres sobre el lugar que ocupan en el amor permanente de vuestro corazón humano? Vuestro Padre que está en los cielos tampoco deja a sus hijos, nacidos del espíritu por la fe, en una ambigua incertidumbre sobre su posición en el reino. Si recibís a Dios como vuestro Padre, entonces sí que sois en verdad los hijos de Dios. Y si sois sus hijos, entonces estáis seguros de la posición y del lugar de todo lo que concierne a la filiación eterna y divina. Si creéis en mis palabras, creéis de ese modo en Aquel que me ha enviado, y al creer así en el Padre, os habéis asegurado vuestra posición en la ciudadanía celestial. Si hacéis la voluntad del Padre que está en los cielos, nunca dejaréis de conseguir la vida eterna de progreso en el reino divino>>\footnote{\textit{Judgad por los frutos del espíritu}: Mt 7:15-19; Mt 12:33; Lc 3:9; Lc 6:43-45; Gl 5:22-23; Ef 5:9. \textit{Hijos de Dios}: 1 Cr 22:10; Sal 2:7; Is 56:5; Mt 5:9,16,45; Lc 20:36; Jn 1:12-13; 11:52; Hch 17:28-29; Ro 8:14-17,19,21; 9:26; 2 Co 6:18; Gl 3:26; 4:5-7; Ef 1:5; Flp 2:15; Heb 12:5-8; 1 Jn 3:1-2,10; 5:2; Ap 21:7; 2 Sam 7:14.}.

\par 
%\textsuperscript{(1601.3)}
\textsuperscript{142:5.3} <<El Espíritu Supremo dará testimonio con vuestro espíritu de que sois realmente los hijos de Dios. Si sois los hijos de Dios, entonces habéis nacido del espíritu de Dios; y cualquiera que ha nacido del espíritu, tiene dentro de sí el poder de vencer todas las dudas, y ésta es la victoria que supera todas las incertidumbres, vuestra propia fe>>\footnote{\textit{El espíritu da testimonio del espíritu}: Ro 8:14-17; 1 Jn 5:6,8. \textit{Sobreponerse a la duda con la fe}: 1 Jn 5:4.}.

\par 
%\textsuperscript{(1601.4)}
\textsuperscript{142:5.4} <<El profeta Isaías ha dicho, al hablar de esta época: `Cuando el espíritu se derrame sobre nosotros desde arriba, entonces la labor de la rectitud se convertirá en paz, tranquilidad y seguridad para siempre'. Para todos los que creen de verdad en este evangelio, yo seré la garantía de su admisión en la felicidad eterna y en la vida perpetua del reino de mi Padre. Así pues, vosotros que oís este mensaje y creéis en este evangelio del reino, sois los hijos de Dios y tenéis la vida eterna. La prueba para el mundo entero de que habéis nacido del espíritu es que os amáis sinceramente los unos a los otros>>\footnote{\textit{Cuando el espíritu se derrame vendrá la paz}: Is 32:15-17. \textit{Evidencia del espíritu: el amor}: Jn 13:35.}.

\par 
%\textsuperscript{(1601.5)}
\textsuperscript{142:5.5} La multitud de oyentes permaneció muchas horas con Jesús, haciéndole preguntas y escuchando atentamente sus respuestas confortantes. La enseñanza de Jesús también animó a los apóstoles a predicar el evangelio del reino con más fuerza y seguridad. Esta experiencia en Jerusalén fue una gran inspiración para los doce. Era su primer contacto con un gentío tan enorme, y aprendieron muchas lecciones valiosas que les resultaron de gran ayuda en su trabajo posterior.

\section*{6. La conversación con Nicodemo}
\par 
%\textsuperscript{(1601.6)}
\textsuperscript{142:6.1} Una tarde, en la casa de Flavio, un tal Nicodemo vino a ver a Jesús\footnote{\textit{Nicodemo visita a Jesús}: Jn 3:1-2a.}; era un miembro rico y anciano del sanedrín judío. Había oído hablar mucho de las enseñanzas de este galileo, y por eso fue a escucharlo una tarde mientras enseñaba en los patios del templo. Hubiera querido ir a menudo a escuchar las lecciones de Jesús, pero temía ser visto por la gente que asistía a su enseñanza, porque los dirigentes de los judíos estaban ya tan en desacuerdo con Jesús, que ningún miembro del sanedrín quería que se le identificara abiertamente de alguna manera con él. En consecuencia, Nicodemo había convenido con Andrés que vería a Jesús aquella tarde concreta, en privado y después del anochecer. Pedro, Santiago y Juan se encontraban en el jardín de Flavio cuando empezó la entrevista, pero más tarde todos entraron en la casa, donde continuó la conversación.

\par 
%\textsuperscript{(1602.1)}
\textsuperscript{142:6.2} Al recibir a Nicodemo, Jesús no mostró ninguna deferencia especial; al hablar con él, no hubo concesiones ni intentos indebidos de persuasión. El Maestro no trató de rechazar a su clandestino visitante, ni fue sarcástico con él. En todo su trato con el distinguido visitante, Jesús se mostró tranquilo, serio y digno. Nicodemo no era un delegado oficial del sanedrín; vino a ver a Jesús solamente debido a su interés personal y sincero por las enseñanzas del Maestro.

\par 
%\textsuperscript{(1602.2)}
\textsuperscript{142:6.3} Después de ser presentado por Flavio, Nicodemo dijo: <<Rabino, sabemos que eres un instructor enviado por Dios, porque ningún simple hombre podría enseñar así a menos que Dios estuviera con él. Y estoy deseoso de saber más cosas sobre tus enseñanzas relacionadas con el reino venidero>>\footnote{\textit{La pregunta de Nicodemo}: Jn 3:2b.}.

\par 
%\textsuperscript{(1602.3)}
\textsuperscript{142:6.4} Jesús respondió a Nicodemo: <<En verdad, en verdad te digo, Nicodemo, que a menos que un hombre nazca de lo alto, no puede ver el reino de Dios>>\footnote{\textit{Jesús responde ``Naciendo de nuevo''}: Jn 3:3.}. Entonces Nicodemo contestó: <<Pero, ¿cómo puede un hombre nacer de nuevo cuando es viejo? No puede entrar por segunda vez en el seno de su madre para nacer>>\footnote{\textit{¿Cómo ``nacer de nuevo''?}: Jn 3:4.}.

\par 
%\textsuperscript{(1602.4)}
\textsuperscript{142:6.5} Jesús dijo: <<Sin embargo, te aseguro que a menos que un hombre nazca del espíritu, no podrá entrar en el reino de Dios. Lo que ha nacido de la carne, es carne, y lo que ha nacido del espíritu, es espíritu. Pero no deberías asombrarte porque he dicho que debes nacer de lo alto. Cuando sopla el viento, oyes el susurro de las hojas, pero no ves el viento ---de dónde viene o adónde va--- y lo mismo sucede con todo aquel que ha nacido del espíritu. Con los ojos de la carne puedes contemplar las manifestaciones del espíritu, pero no puedes discernir realmente al espíritu>>\footnote{\textit{Nacido del espíritu}: Jn 3:5-8.}.

\par 
%\textsuperscript{(1602.5)}
\textsuperscript{142:6.6} Nicodemo respondió: <<Pero no comprendo ---¿cómo puede ser eso?>>\footnote{\textit{Nicodemo repite su pregunta}: Jn 3:9.} Jesús dijo: <<¿Es posible que seas un educador de Israel y que sin embargo ignores todo esto? Los que conocen las realidades del espíritu tienen pues el deber de revelar estas cosas a los que disciernen solamente las manifestaciones del mundo material. Pero ¿nos creerás si te hablamos de las verdades celestiales? ¿Tienes el coraje de creer, Nicodemo, en alguien que ha descendido del cielo, en el mismo Hijo del Hombre?>>\footnote{\textit{Realidades del espíritu}: Jn 3:10-13.}

\par 
%\textsuperscript{(1602.6)}
\textsuperscript{142:6.7} Y Nicodemo dijo: <<Pero ¿cómo puedo empezar a captar ese espíritu que ha de rehacerme como preparación para entrar en el reino?>> Jesús respondió: <<El espíritu del Padre que está en los cielos ya reside dentro de ti. Si quieres dejarte conducir por este espíritu que viene de arriba, muy pronto empezarás a ver con los ojos del espíritu; a continuación, si escoges de todo corazón seguir la orientación del espíritu, nacerás del espíritu, porque el único propósito de tu vida será hacer la voluntad de tu Padre que está en los cielos. Al encontrarte así, nacido del espíritu y feliz en el reino de Dios, empezarás a producir en tu vida diaria los frutos abundantes del espíritu>>.

\par 
%\textsuperscript{(1602.7)}
\textsuperscript{142:6.8} Nicodemo era completamente sincero. Estaba profundamente impresionado, pero se fue desconcertado. Era un hombre realizado en cuanto al desarrollo personal, al dominio de sí mismo e incluso a las altas cualidades morales. Era refinado, egoísta y altruista, pero no sabía cómo \textit{someter} su voluntad a la voluntad del Padre divino, como un niño pequeño está dispuesto a someterse a la guía y dirección de un padre terrestre sabio y amoroso, convirtiéndose así en realidad en un hijo de Dios, en un heredero progresivo del reino eterno.

\par 
%\textsuperscript{(1603.1)}
\textsuperscript{142:6.9} Pero Nicodemo supo reunir la suficiente fe como para apoderarse del reino. Protestó tímidamente cuando sus colegas del sanedrín intentaron condenar a Jesús sin juicio. Más tarde, con José de Arimatea, reconoció audazmente su fe\footnote{\textit{Reconocimiento público}: Jn 19:38-42.} y reclamó el cuerpo de Jesús, incluso cuando la mayoría de los discípulos habían huido atemorizados del escenario del sufrimiento y de la muerte final de su Maestro.

\section*{7. La lección sobre la familia}
\par 
%\textsuperscript{(1603.2)}
\textsuperscript{142:7.1} Después del activo período de enseñanza y de trabajo personal durante la semana pascual en Jerusalén, Jesús pasó el miércoles siguiente descansando con sus apóstoles en Betania. Aquella tarde, Tomás hizo una pregunta que atrajo una respuesta larga e instructiva. Tomás dijo: <<Maestro, el día que fuimos seleccionados como embajadores del reino, nos dijiste muchas cosas; nos instruiste sobre nuestra manera personal de vivir, pero, ¿qué le enseñaremos a la multitud? ¿Cómo deberá vivir esa gente después de que el reino llegue más plenamente? ¿Tus discípulos poseerán esclavos? ¿Tus fieles buscarán la pobreza y huirán de la riqueza? ¿Prevalecerá solamente la misericordia, de tal manera que ya no tendremos ni ley ni justicia?>> Jesús y los doce pasaron toda la tarde y toda aquella noche, después de la cena, discutiendo las preguntas de Tomás. Para los propósitos de esta narración, presentamos el siguiente resumen de las instrucciones del Maestro:

\par 
%\textsuperscript{(1603.3)}
\textsuperscript{142:7.2} En primer lugar, Jesús intentó aclarar a sus apóstoles que él mismo estaba en la Tierra viviendo una vida excepcional en la carne, y que ellos doce habían sido llamados para participar en esta experiencia donadora del Hijo del Hombre; como tales colaboradores, también tenían que compartir muchas de las restricciones y obligaciones especiales de toda esta experiencia de donación. Hubo una insinuación velada a que el Hijo del Hombre era la única persona que había vivido en la Tierra, capaz de ver simultáneamente dentro del corazón mismo de Dios y en las profundidades del alma humana.

\par 
%\textsuperscript{(1603.4)}
\textsuperscript{142:7.3} Jesús explicó muy claramente que el reino de los cielos era una experiencia evolutiva que empezaba aquí, en la Tierra, y progresaba por medio de etapas sucesivas de vida hasta el Paraíso. En el transcurso de la noche indicó con precisión que en alguna fase futura del desarrollo del reino, volvería a visitar este mundo con poder espiritual y gloria divina.

\par 
%\textsuperscript{(1603.5)}
\textsuperscript{142:7.4} Luego explicó que la <<idea del reino>> no era la mejor manera de ilustrar la relación del hombre con Dios; que empleaba esta metáfora porque el pueblo judío estaba esperando el reino, y porque Juan había predicado refiriéndose al reino por venir. Jesús dijo: <<La gente de otra época comprenderá mejor el evangelio del reino cuando éste sea presentado en unos términos que expresen la relación familiar ---cuando el hombre comprenda la religión como la enseñanza de la paternidad de Dios y la fraternidad de los hombres, la filiación con Dios>>\footnote{\textit{Analogía de ``la familia''}: 1 Jn 3:1-2a.}. Después, el Maestro disertó con cierta amplitud sobre la familia terrenal, como una ilustración de la familia celestial, exponiendo de nuevo las dos leyes fundamentales de la vida: el primer mandamiento de amor por el padre, el cabeza de familia, y el segundo mandamiento de amor mutuo entre los hijos, el de amar al hermano como a sí mismo. Luego explicó que esta cualidad del afecto fraternal se manifestaría invariablemente en el servicio social desinteresado y amoroso.

\par 
%\textsuperscript{(1603.6)}
\textsuperscript{142:7.5} A esto le siguió el debate memorable sobre las características fundamentales de la vida familiar, y su aplicación a la relación existente entre Dios y el hombre. Jesús declaró que una verdadera familia está fundada en los siete hechos siguientes:

\par 
%\textsuperscript{(1604.1)}
\textsuperscript{142:7.6} 1. \textit{El hecho de la existencia}. Las relaciones de la naturaleza y los fenómenos del parecido físico están ligados en la familia: los hijos heredan ciertas características parentales. Los hijos tienen su origen en sus padres; la existencia de su personalidad depende del acto de los padres. La relación de padre a hijo es inherente a toda la naturaleza e impregna todas las existencias vivientes.

\par 
%\textsuperscript{(1604.2)}
\textsuperscript{142:7.7} 2. \textit{La seguridad y el placer}. Los padres auténticos experimentan un gran placer satisfaciendo las necesidades de sus hijos. Muchos padres no se contentan con abastecer simplemente las necesidades de sus hijos, sino que disfrutan también asegurándoles sus placeres.

\par 
%\textsuperscript{(1604.3)}
\textsuperscript{142:7.8} 3. \textit{La educación y la preparación}. Los padres sabios planean cuidosamente la educación y la preparación adecuada de sus hijos e hijas. Se les prepara desde que son jóvenes para las responsabilidades mayores de la vida adulta.

\par 
%\textsuperscript{(1604.4)}
\textsuperscript{142:7.9} 4. \textit{La disciplina y la restricción}. Los padres previsores también toman medidas para la disciplina, la dirección, la corrección y a veces la restricción necesarias de sus descendientes jóvenes e inmaduros.

\par 
%\textsuperscript{(1604.5)}
\textsuperscript{142:7.10} 5. \textit{La camaradería y la lealtad}. El padre afectuoso mantiene una relación íntima y amorosa con sus hijos. Siempre está dispuesto a escuchar sus peticiones; siempre está preparado para compartir sus penalidades y ayudarlos en sus dificultades. El padre se interesa de manera suprema por el bienestar progresivo de su descendencia.

\par 
%\textsuperscript{(1604.6)}
\textsuperscript{142:7.11} 6. \textit{El amor y la misericordia}. Un padre compasivo perdona espontáneamente; los padres no alimentan ideas de venganza contra sus hijos. Los padres no son como los jueces, los enemigos o los acreedores. Las familias verdaderas están construidas sobre la tolerancia, la paciencia y el perdón.

\par 
%\textsuperscript{(1604.7)}
\textsuperscript{142:7.12} 7. \textit{Las disposiciones para el futuro}. A los padres temporales les gusta dejar una herencia para sus hijos. La familia continúa de una generación a la siguiente. La muerte sólo acaba con una generación para marcar el comienzo de la siguiente. La muerte pone término a una vida individual, pero no necesariamente a la familia.

\par 
%\textsuperscript{(1604.8)}
\textsuperscript{142:7.13} El Maestro examinó durante horas la aplicación de estas características de la vida familiar a las relaciones del hombre ---el hijo terrestre--- con Dios ---el Padre Paradisiaco. Y ésta fue su conclusión: <<Conozco a la perfección la totalidad de esta relación de un hijo con el Padre, porque ya he alcanzado ahora, en el terreno de la filiación, todo lo que tendréis que alcanzar en el eterno futuro. El Hijo del Hombre está preparado para ascender a la diestra del Padre, de manera que, en mí, el camino está ahora aún más abierto para que todos vosotros veáis a Dios y, antes de que hayáis terminado la gloriosa progresión, os volváis perfectos como vuestro Padre que está en los cielos es perfecto>>\footnote{\textit{Sed perfectos}: Gn 17:1; 1 Re 8:61; Lv 19:2; Dt 18:13; Mt 5:48; 2 Co 13:11; Stg 1:4; 1 P 1:16.}.

\par 
%\textsuperscript{(1604.9)}
\textsuperscript{142:7.14} Cuando los apóstoles escucharon estas palabras sorprendentes, recordaron las declaraciones que Juan había hecho en la época del bautismo de Jesús; también se acordaron vívidamente de esta experiencia en conexión con sus predicaciones y enseñanzas, después de la muerte y resurrección del Maestro.

\par 
%\textsuperscript{(1604.10)}
\textsuperscript{142:7.15} Jesús es un Hijo divino que cuenta con toda la confianza del Padre Universal. Había estado con el Padre y lo comprendía plenamente. Ahora había vivido su vida terrestre a la entera satisfacción del Padre, y esta encarnación en la carne le había permitido comprender plenamente al hombre. Jesús era la perfección del hombre; había alcanzado la misma perfección que todos los verdaderos creyentes están destinados a alcanzar en él y a través de él. Jesús reveló al hombre un Dios de perfección, y presentó a Dios, en su propia persona, al hijo perfeccionado de los mundos.

\par 
%\textsuperscript{(1605.1)}
\textsuperscript{142:7.16} Aunque Jesús estuvo hablando durante varias horas, Tomás aún no estaba satisfecho, puesto que dijo: <<Pero, Maestro, no nos parece que el Padre que está en los cielos nos trate siempre con bondad y misericordia. Muchas veces sufrimos enormemente en la Tierra, y nuestras oraciones no siempre son contestadas. ¿En qué punto no conseguimos captar el significado de tu enseñanza?>>

\par 
%\textsuperscript{(1605.2)}
\textsuperscript{142:7.17} Jesús replicó: <<Tomás, Tomás, ¿cuánto tiempo necesitarás para adquirir la aptitud de escuchar con el oído del espíritu? ¿Cuánto tiempo pasará antes de que disciernas que este reino es un reino espiritual, y que mi Padre es también un ser espiritual? ¿No comprendes que os enseño como hijos espirituales de la familia espiritual del cielo, cuyo jefe paterno es un espíritu infinito y eterno? ¿No me permitiréis que utilice la familia terrestre para ilustrar las relaciones divinas, sin aplicar mi enseñanza tan literalmente a los asuntos materiales? ¿No podéis separar en vuestra mente las realidades espirituales del reino, de los problemas materiales, sociales, económicos y políticos de esta época? Cuando hablo el lenguaje del espíritu, ¿por qué insistís en traducir mi intención al lenguaje de la carne, simplemente porque me tomo la libertad de emplear las relaciones vulgares y literales con una finalidad ilustrativa? Hijos míos, os ruego que dejéis de aplicar la enseñanza del reino del espíritu a los sórdidos asuntos de la esclavitud, la pobreza, las casas y las tierras, y a los problemas materiales de la equidad y la justicia humanas. Estas cuestiones temporales interesan a los hombres de este mundo, y aunque en cierto modo afectan a todos los hombres, habéis sido llamados para representarme en el mundo como yo represento a mi Padre. Sois los embajadores espirituales de un reino espiritual, los representantes especiales del Padre del espíritu. A estas alturas, ya debería poder instruiros como hombres maduros del reino del espíritu. ¿Tendré que seguir hablándoos como si fuerais niños? ¿No creceréis nunca en percepción espiritual? Sin embargo, os amo y seré indulgente con vosotros hasta el fin de nuestra asociación en la carne. E incluso entonces, mi espíritu os precederá en el mundo entero>>.

\section*{8. En Judea del sur}
\par 
%\textsuperscript{(1605.3)}
\textsuperscript{142:8.1} A finales de abril, la oposición de los fariseos y saduceos se había vuelto tan pronunciada contra Jesús, que el Maestro y sus apóstoles decidieron dejar Jerusalén por un tiempo, y se dirigieron hacia el sur para trabajar en Belén y Hebrón\footnote{\textit{Regreso a Judea}: Jn 3:22.}. Pasaron todo el mes de mayo efectuando un trabajo personal en estas ciudades y entre la gente de los pueblos vecinos. Durante este viaje no hicieron ninguna predicación pública, sino solamente visitas de casa en casa. Mientras los apóstoles enseñaban el evangelio y cuidaban a los enfermos, Jesús y Abner pasaron una parte de este tiempo en En-Gedi, visitando la colonia nazarea. Juan el Bautista había salido de este lugar, y Abner había sido jefe de este grupo. Muchos miembros de la fraternidad nazarea se hicieron creyentes en Jesús, pero la mayoría de estos hombres ascéticos y extravagantes rehusó aceptarlo como un instructor enviado del cielo, porque no enseñaba el ayuno ni otras formas de abnegación.

\par 
%\textsuperscript{(1605.4)}
\textsuperscript{142:8.2} La gente que vivía en esta región no sabía que Jesús había nacido en Belén. Al igual que la gran mayoría de sus discípulos, siempre habían supuesto que el Maestro había nacido en Nazaret, pero los doce conocían la verdad.

\par 
%\textsuperscript{(1605.5)}
\textsuperscript{142:8.3} Esta estancia en el sur de Judea fue un período de trabajo reposado y fructífero; muchas almas se añadieron al reino. A primeros de junio, la agitación contra Jesús se había calmado tanto en Jerusalén, que el Maestro y los apóstoles regresaron para instruir y alentar a los creyentes.

\par 
%\textsuperscript{(1606.1)}
\textsuperscript{142:8.4} Aunque Jesús y los apóstoles pasaron todo el mes de junio en Jerusalén o en las proximidades, no efectuaron ninguna enseñanza pública durante este período. Vivieron la mayor parte del tiempo en las tiendas que montaron en un parque o jardín sombreado conocido en aquella época con el nombre de Getsemaní. Este parque estaba situado en la ladera occidental del Monte de los Olivos, no lejos del arroyo Cedrón. Los sábados del fin de semana los pasaban habitualmente con Lázaro y sus hermanas en Betania. Jesús entró pocas veces dentro de los muros de Jerusalén, pero un gran número de investigadores interesados fueron hasta Getsemaní para charlar con él. Un viernes por la noche, Nicodemo y un tal José de Arimatea se atrevieron a salir para visitar a Jesús, pero cuando estaban delante de la entrada de la tienda del Maestro, se volvieron atrás por miedo. Por supuesto, no se percataban de que Jesús conocía todo lo que hacían.

\par 
%\textsuperscript{(1606.2)}
\textsuperscript{142:8.5} Cuando los dirigentes de los judíos se enteraron de que Jesús había regresado a Jerusalén, se prepararon para arrestarlo; pero al observar que no predicaba en público, concluyeron que se había asustado con el alboroto que habían causado anteriormente, y decidieron permitirle que continuara enseñando de esta manera privada, sin molestarlo más. Así es como las cosas siguieron desarrollándose tranquilamente hasta los últimos días de junio, cuando un tal Simón, miembro del sanedrín, abrazó públicamente las enseñanzas de Jesús, después de decírselo en persona a los jefes de los judíos. Inmediatamente se produjo un nuevo alboroto para capturar a Jesús, y tomó tal importancia, que el Maestro decidió retirarse a las ciudades de Samaria y la Decápolis\footnote{\textit{Regreso a la Decápolis}: Jn 4:1-3.}.


\chapter{Documento 143. La travesía de Samaria}
\par 
%\textsuperscript{(1607.1)}
\textsuperscript{143:0.1} A FINALES de junio del año 27, debido a la oposición creciente de los dirigentes religiosos judíos, Jesús y los doce partieron de Jerusalén después de enviar sus tiendas y sus escasos efectos personales para que fueran guardados en la casa de Lázaro, en Betania. Se dirigieron al norte hacia Samaria, y el sábado se detuvieron en Betel. Predicaron allí durante varios días a la gente que venía de Gofna y Efraín. Un grupo de ciudadanos de Arimatea y Tamna vino para invitar a Jesús a que visitara sus pueblos. El Maestro y sus apóstoles pasaron más de dos semanas enseñando a los judíos y samaritanos de esta región, muchos de los cuales venían de lugares tan lejanos como Antípatris para escuchar la buena nueva del reino.

\par 
%\textsuperscript{(1607.2)}
\textsuperscript{143:0.2} Los habitantes del sur de Samaria escucharon con placer a Jesús, y los apóstoles, a excepción de Judas Iscariote, consiguieron vencer muchos de los prejuicios que tenían contra los samaritanos. A Judas le resultaba muy difícil amar a estos samaritanos. La última semana de julio, Jesús y sus compañeros se prepararon para partir hacia las nuevas ciudades griegas de Fasaelis y Arquelais, cerca del Jordán.

\section*{1. La predicación en Arquelais}
\par 
%\textsuperscript{(1607.3)}
\textsuperscript{143:1.1} Durante la primera mitad del mes de agosto, el grupo apostólico estableció su cuartel general en las ciudades griegas de Arquelais y Fasaelis, donde efectuaron su primera experiencia de predicación a una concurrencia compuesta casi exclusivamente de gentiles ---griegos, romanos y sirios--- ya que pocos judíos residían en estas dos ciudades griegas. Al ponerse en contacto con estos ciudadanos romanos, los apóstoles encontraron nuevas dificultades para proclamar el mensaje del reino venidero, y tropezaron con nuevas objeciones a las enseñanzas de Jesús. En una de las muchas conversaciones nocturnas con sus apóstoles, Jesús escuchó atentamente estas objeciones al evangelio del reino mientras los doce repasaban sus experiencias con la gente que se había beneficiado de su trabajo personal.

\par 
%\textsuperscript{(1607.4)}
\textsuperscript{143:1.2} Felipe hizo una pregunta que fue representativa de sus dificultades. Felipe dijo: <<Maestro, estos griegos y romanos menosprecian nuestro mensaje, pues dicen que estas enseñanzas sólo son adecuadas para los débiles y los esclavos. Aseguran que la religión de los paganos es superior a nuestra enseñanza, porque estimula a adquirir un carácter fuerte, robusto y dinámico. Afirman que queremos convertir a todos los hombres en unos especímenes debilitados de no resistentes pasivos, que desaparecerían rápidamente de la faz de la Tierra. A ti te aprecian, Maestro, y admiten francamente que tu enseñanza es celestial e ideal, pero no quieren tomarnos en serio. Afirman que tu religión no es para este mundo, que los hombres no pueden vivir según lo que enseñas. Y ahora, Maestro, ¿qué les vamos a decir a estos gentiles?>>

\par 
%\textsuperscript{(1607.5)}
\textsuperscript{143:1.3} Después de haber escuchado otras objeciones similares al evangelio del reino presentadas por Tomás, Natanael, Simón Celotes y Mateo, Jesús dijo a los doce:

\par 
%\textsuperscript{(1608.1)}
\textsuperscript{143:1.4} <<He venido a este mundo para hacer la voluntad de mi Padre y para revelar su carácter afectuoso a toda la humanidad. Ésta es, hermanos míos, mi misión. Y ésta es la única cosa que haré, independientemente de que mis enseñanzas sean mal comprendidas por los judíos o los gentiles de esta época o de otra generación. Pero no deberíais pasar por alto el hecho de que el amor divino también tiene sus disciplinas severas. El amor de un padre por su hijo obliga muchas veces al padre a refrenar las acciones imprudentes de su atolondrado descendiente. El hijo no siempre comprende los motivos sabios y afectuosos de la disciplina restrictiva del padre. Pero os aseguro que mi Padre Paradisiaco gobierna de hecho un universo de universos con el poder predominante de su amor. El amor es la más grande de todas las realidades espirituales. La verdad es una revelación liberadora, pero el amor es la relación suprema. Cualesquiera que sean los desatinos que vuestros compañeros humanos puedan cometer en la administración del mundo de hoy, el evangelio que os proclamo gobernará este mismo mundo en una era por venir. La meta última del progreso humano consiste en reconocer respetuosamente la paternidad de Dios y en materializar con amor la fraternidad de los hombres>>\footnote{\textit{Jesús vino a hacer la voluntad de Dios}: Mt 26:39,42,44; Mc 14:36,39; Lc 22:42; Jn 4:34; 5:30; 6:38-40; 15:10; 17:4. \textit{Amorosa disciplina de los niños}: Heb 12:5-11. \textit{Dios gobierna mediante el amor}: Mt 5:43-48; 22:37-40; Mc 12:29-33; Lc 10:27; Jn 3:16; 13:34-35; 14:21-23; 15:9-13,17; 16:27; 17:22-23; Ro 5:8; 1 Co 13:1-8; 2 Co 13:11; Tit 3:4; 1 Jn 3:1; 4:7-19.}.

\par 
%\textsuperscript{(1608.2)}
\textsuperscript{143:1.5} <<¿Quién os ha dicho que mi evangelio sólo está destinado a los esclavos y a los débiles? ¿Acaso vosotros, mis apóstoles elegidos, parecéis débiles? ¿Tenía Juan aspecto de endeble? ¿Observáis que yo sea esclavo del miedo? Es verdad que el evangelio se predica a los pobres y a los oprimidos de esta generación. Las religiones de este mundo han olvidado a los pobres, pero mi Padre no hace acepción de personas. Además, los pobres de hoy son los primeros en hacer caso de la llamada al arrepentimiento y a aceptar la filiación. El evangelio del reino debe ser predicado a todos los hombres ---judíos y gentiles, griegos y romanos, ricos y pobres, libres y esclavos--- e igualmente a los jóvenes y a los viejos, a los hombres y a las mujeres>>\footnote{\textit{Id predicad el evangelio a todos}: Mt 24:14; 28:19-20a; Mc 13:10; 16:15; Lc 24:47; Jn 17:18; Hch 1:8b. \textit{El evangelio del reino}: Mt 3:2; 4:17,23; 5:3,10,19-20; 6:33; 7:21; 8:11; 9:35; 10:7; 11:11-12; 12:28; 13:11,24,31-52; 16:19; 18:1-4,23; 19:14,23-24; 20:1; 21:31,43; 22:2; 23:13; 24:14; 25:1,14; Mc 1:14-15; 4:11,26,30; 9:1,47; 10:14-15,23-25; 12:34; 14:25; 15:43; Lc 4:43; 6:20; 7:28; 8:1,10; 9:2,11,27; 9:60,62; 10:9-11; 11:20; 12:31-32; 13:18,20,28,29; 14:15; 16:16; 17:20-21; 18:16-17,24-25; 19:11; 21:31; 22:16,18; 23:51; Jn 3:3,5; Ro 14:17; 1 Co 4:20; 6:9-10. \textit{El evangelio predicado a los pobres}: Is 61:1-2; Mt 11:4-5; Lc 7:22. \textit{¿Acaso Juan era un débil?}: Mt 11:7-9; Lc 7:24-26. \textit{Mi padre no hace acepción de personas}: 2 Cr 19:7; Job 34:19; Eclo 35:12; Hch 10:34; Ro 2:11; Gl 2:6; 3:28; Ef 6:9; Col 3:11.}.

\par 
%\textsuperscript{(1608.3)}
\textsuperscript{143:1.6} <<Aunque mi Padre es un Dios de amor y se deleita practicando la misericordia, no os impregnéis de la idea de que el servicio del reino debe ser de una facilidad monótona. La ascensión al Paraíso es la aventura suprema de todos los tiempos, la dura obtención de la eternidad. El servicio del reino en la Tierra exigirá toda la valiente virilidad que vosotros y vuestros colaboradores podáis reunir. Muchos de vosotros seréis ejecutados por vuestra lealtad al evangelio de este reino. Es fácil morir en el campo de batalla cuando la presencia de vuestros camaradas de combate fortalece vuestra valentía, pero se requiere una forma superior y más profunda de valentía y de devoción humanas para dar la vida con serenidad y en solitario por el amor de una verdad guardada en vuestro corazón mortal>>\footnote{\textit{Dios de amor}: Mt 5:43-48; 22:37-40; Mc 12:29-33; Lc 10:27; Jn 3:16; 13:34-35; 14:21-23; 15:9-13,17; 16:27; 17:22-23; Ro 5:8; 1 Co 13:1-8; 2 Co 13:11; Tit 3:4; 1 Jn 3:1; 4:7-19. \textit{Dios se deleita en la misericordia}: Ex 20:6; 1 Cr 16:34; 2 Cr 5:13; 7:3,6; 30:9; Esd 3:11; Sal 25:6; 36:5; 86:5,13,15; 100:5:; 103:8,11,17; 107:1; 116:5; 117:2; 118:1,4; 136:1-26; 145;8; Is 55:7; Jer 3:12; Nm 14:18-19; Miq 7:18; Dt 5:10; Heb 6:12. \textit{Exigencia de un servicio valiente}: Mt 5:10; Lc 21:12,16-17; Ro 5:3-5; 8:35-39; 2 Co 6:3-10; 1 P 4:12-14.}.

\par 
%\textsuperscript{(1608.4)}
\textsuperscript{143:1.7} <<Hoy, los incrédulos pueden mofarse de vosotros porque predicáis un evangelio de no resistencia y porque vivís una vida sin violencia, pero sois los primeros voluntarios de una larga serie de creyentes sinceros en el evangelio de este reino, que asombrarán a toda la humanidad por su consagración heroica a estas enseñanzas. Ningún ejército del mundo ha desplegado nunca más coraje y bravura que los que mostraréis vosotros y vuestros leales sucesores cuando salgáis para proclamar al mundo entero la buena nueva ---la paternidad de Dios y la fraternidad de los hombres. La valentía de la carne es la forma más baja de bravura. La bravura mental es un tipo más elevado de valentía humana, pero la bravura superior y suprema consiste en la fidelidad inflexible a las convicciones iluminadas de las realidades espirituales profundas. Una valentía así constituye el heroísmo del hombre que conoce a Dios. Y todos vosotros sois hombres que conocéis a Dios; sois, en verdad, los asociados personales del Hijo del Hombre>>.

\par 
%\textsuperscript{(1608.5)}
\textsuperscript{143:1.8} Esto no es todo lo que Jesús dijo en esta ocasión, pero es la introducción de su discurso. Luego continuó hablando largamente para ampliar e ilustrar esta declaración. Éste fue uno de los discursos más apasionados que Jesús pronunció nunca ante los doce. El Maestro rara vez hablaba a sus apóstoles mostrando unos poderosos sentimientos, pero ésta fue una de las pocas ocasiones en las que se expresó con una seriedad manifiesta, acompañada de una marcada emoción.

\par 
%\textsuperscript{(1609.1)}
\textsuperscript{143:1.9} El efecto sobre la predicación pública y el ministerio personal de los apóstoles fue inmediato; a partir de aquel mismo día, su mensaje adquirió un nuevo matiz en el que predominaba la valentía. Los doce continuaron adquiriendo el espíritu positivamente dinámico del nuevo evangelio del reino. Desde aquel día en adelante, ya no se ocuparon tanto de predicar las virtudes negativas y los preceptos pasivos de la enseñanza multifacética de su Maestro.

\section*{2. La lección sobre el dominio de sí mismo}
\par 
%\textsuperscript{(1609.2)}
\textsuperscript{143:2.1} El Maestro era un ejemplo perfeccionado de un hombre dueño de sí mismo\footnote{\textit{El dominio de sí mismo de Jesús}: 1 P 2:23-24.}. Cuando fue injuriado, no injurió; cuando sufrió, no profirió ninguna amenaza contra sus torturadores; cuando fue acusado por sus enemigos, simplemente se encomendó al juicio justo del Padre que está en los cielos.

\par 
%\textsuperscript{(1609.3)}
\textsuperscript{143:2.2} En una de las conferencias nocturnas, Andrés le preguntó a Jesús: <<Maestro, ¿debemos practicar la abnegación como Juan nos ha enseñado, o debemos procurar adquirir el autocontrol que tú enseñas? ¿En qué se diferencia tu enseñanza de la de Juan?>> Jesús respondió: <<En verdad, Juan os ha enseñado el camino de la rectitud de acuerdo con las luces y las leyes de sus antepasados; era la religión del examen de conciencia y de la abnegación. Pero yo vengo con un nuevo mensaje de olvido de sí mismo y de dominio de sí mismo. Os muestro el camino de la vida tal como mi Padre que está en los cielos me lo ha revelado>>.

\par 
%\textsuperscript{(1609.4)}
\textsuperscript{143:2.3} <<En verdad, en verdad os digo que aquel que se gobierna a sí mismo es más grande que el que conquista una ciudad. El dominio de sí mismo es la medida de la naturaleza moral de un hombre, y el indicador de su desarrollo espiritual. En el antiguo orden practicabais el ayuno y la oración. Como criaturas nuevas renacidas del espíritu, se os enseña a creer y a regocijaros. En el reino del Padre, debéis convertiros en criaturas nuevas; las cosas viejas deben desaparecer; observad que os muestro cómo todas las cosas deben renovarse. Por medio de vuestro amor recíproco vais a convencer al mundo de que habéis pasado de la esclavitud a la libertad, de la muerte a la vida eterna>>\footnote{\textit{Aquel que se gobierna a sí mismo es grande}: Pr 16:32. \textit{El antiguo orden era ayuno y oración}: Mt 9:14; Mc 2:18; Lc 5:33. \textit{El orden antiguo y el nuevo}: Mt 9:16-17; Mc 2:21-22; Lc 5:36-38; Ro 7:6; 2 Co 5:17. \textit{Testigos a través del amor}: Jn 13:35. \textit{De la esclavitud a la libertad}: Ro 8:21.}.

\par 
%\textsuperscript{(1609.5)}
\textsuperscript{143:2.4} <<En el antiguo camino, intentáis suprimir, obedecer y conformaros a unas reglas de vida; en el nuevo camino, primero sois \textit{transformados} por el Espíritu de la Verdad y, por ello, fortalecidos en vuestra alma interior mediante la constante renovación espiritual de vuestra mente; así estáis dotados con el poder de ejecutar, con certeza y alegría, la voluntad misericordiosa, aceptable y perfecta de Dios. No lo olvidéis ---vuestra fe personal en las promesas extremadamente grandes y preciosas de Dios es la que os garantiza que os convertiréis en partícipes de la naturaleza divina. Así, mediante vuestra fe y la transformación del espíritu, os convertís en realidad en los templos de Dios, y su espíritu vive efectivamente dentro de vosotros. Así pues, si el espíritu reside dentro de vosotros, ya no sois unos esclavos ligados a la carne, sino unos hijos del espíritu, independientes y liberados. La nueva ley del espíritu os dota de la libertad del dominio de sí mismo, reemplazando la antigua ley del miedo, basada en la autoesclavitud y en el yugo de la abnegación>>\footnote{\textit{Transformación}: Sal 51:10; Ez 18:31; 36:26; Ro 12:2; 2 Co 5:17-19. \textit{Fe en las promesas}: 2 P 1:4. \textit{Seréis los templos de Dios}: Lc 17:21; Ro 8:9-11; 1 Co 3:16-17; 6:19-20; 2 Co 6:16; 2 Ti 1:14; 1 Jn 4:12-15; Ap 21:3. \textit{Hijos liberados del Espíritu}: Ro 8:2,15-16; 2 Co 3:17; Gl 4:6-7.}.

\par 
%\textsuperscript{(1609.6)}
\textsuperscript{143:2.5} <<Muchas veces, cuando habéis hecho el mal, habéis pensado en imputar vuestros actos a la influencia del demonio, cuando en realidad simplemente os habéis descarriado a causa de vuestras propias tendencias naturales. ¿No os ha dicho el profeta Jeremías hace mucho tiempo que el corazón humano es más engañoso que nada, e incluso a veces desesperadamente perverso? ¡Qué fácil es engañaros a vosotros mismos y caer así en unos temores tontos, en deseos de todo tipo, placeres esclavizantes, malicia, envidia e incluso en un odio vengativo!>>\footnote{\textit{El corazón humano es engañoso}: Jer 17:9.}

\par 
%\textsuperscript{(1610.1)}
\textsuperscript{143:2.6} <<La salvación se obtiene por la regeneración del espíritu y no por las acciones presuntuosas de la carne. Estáis justificados por la fe y sois aceptados por la gracia, no por el temor y la abnegación de la carne, aunque los hijos del Padre, que han nacido del espíritu, son siempre y para siempre \textit{dueños} de su yo y de todo lo que se refiere a los deseos de la carne. Cuando sabéis que es la fe la que os salva, tenéis una verdadera paz con Dios. Y todos los que siguen el camino de esta paz celestial están destinados a ser santificados en el servicio eterno de los hijos, en constante progreso, del Dios eterno. En lo sucesivo, ya no es un deber, sino que es más bien vuestro elevado privilegio el purificaros de todos los males de la mente y del cuerpo, mientras buscáis la perfección en el amor de Dios>>\footnote{\textit{Salvados por la fe, no por las obras}: Mc 16:16; Jn 3:36; Ro 3:27-30; Gl 2:16. \textit{Los hijos de la fe son dueños de sí mismos}: Ro 8:5,14; Gl 5:16. \textit{Paz con Dios}: Jn 14:27; 16:33; Ro 5:1-2; 1 Ts 5:23. \textit{Purificarse}: Sal 51:10; 2 Co 7:1; Stg 4:8.}.

\par 
%\textsuperscript{(1610.2)}
\textsuperscript{143:2.7} <<Vuestra filiación está fundada en la fe, y debéis permanecer impasibles ante el miedo. Vuestra alegría nace de la confianza en la palabra divina, y por consiguiente, no pondréis en duda la realidad del amor y de la misericordia del Padre. La bondad misma de Dios es la que conduce a los hombres a un arrepentimiento sincero y auténtico. Vuestro secreto para dominar el yo está ligado a vuestra fe en el espíritu interno, que siempre actúa por amor. Incluso esta fe salvadora no la tenéis por vosotros mismos; es también un regalo de Dios. Si sois los hijos de esta fe viviente, ya no sois los esclavos del yo, sino más bien los dueños triunfantes de vosotros mismos, los hijos liberados de Dios>>\footnote{\textit{Filiación por la fe}: Jn 1:12; Ro 8:14; Gl 3:26. \textit{La bondad de Dios}: Sal 86:5; Mt 19:17; Mc 10:18; Lc 18:19; Ro 2:4. \textit{La fe es un regalo de Dios}: Jn 6:40,65; Ef 2:8; Stg 1:17. \textit{Los hijos liberados de Dios}: 1 Cr 22:10; Sal 2:7; Is 56:5; Mt 5:9,16,45; Lc 20:36; Jn 1:12-13; 11:52; Hch 17:28-29; Ro 8:14-15,19,21; 9:26; 2 Co 6:18; Gl 3:26; 4:5-7; Ef 1:5; Flp 2:15; Heb 12:5-8; 1 Jn 3:1-2,10; 5:2; Ap 21:7; 2 Sam 7:14.}.

\par 
%\textsuperscript{(1610.3)}
\textsuperscript{143:2.8} <<Así pues, hijos míos, si habéis nacido del espíritu, estáis liberados para siempre de la esclavitud consciente de una vida de abnegación y de vigilancia continua de los deseos de la carne, y sois trasladados al alegre reino del espíritu, en el que manifestáis espontáneamente los frutos del espíritu en vuestra vida diaria. Los frutos del espíritu son la esencia del tipo más elevado de autocontrol agradable y ennoblecedor, e incluso lo máximo que un mortal terrestre puede alcanzar ---el verdadero dominio de sí mismo>>\footnote{\textit{Liberados de la abnegación}: Ro 8:1-17. \textit{Los frutos del espíritu}: Gl 5:22-23; Ef 5:9.}.

\section*{3. La diversión y el esparcimiento}
\par 
%\textsuperscript{(1610.4)}
\textsuperscript{143:3.1} Por esta época se desarrolló un estado de gran tensión nerviosa y emocional entre los apóstoles y sus discípulos asociados inmediatos. Aún no se habían acostumbrado a convivir y a trabajar juntos. Cada vez tenían más dificultades para mantener relaciones armoniosas con los discípulos de Juan. El contacto con los gentiles y los samaritanos era una gran prueba para estos judíos. Y además de todo esto, las recientes declaraciones de Jesús habían aumentado la alteración de su estado mental. Andrés estaba casi fuera de sí; ya no sabía qué hacer, y por eso acudió al Maestro con sus problemas y perplejidades. Cuando Jesús terminó de escuchar el relato de las dificultades de su jefe apostólico, dijo: <<Andrés, no puedes disuadir a los hombres de sus confusiones cuando llegan a un grado semejante de complicación, y cuando tantas personas con fuertes sentimientos están implicadas. No puedo hacer lo que me pides ---no deseo participar en esas dificultades sociales personales--- pero me uniré a vosotros para disfrutar de un período de tres días de descanso y esparcimiento. Dirígete a tus hermanos y anúnciales que todos vais a subir conmigo al Monte Sartaba, donde deseo descansar un día o dos>>.

\par 
%\textsuperscript{(1610.5)}
\textsuperscript{143:3.2} <<Ahora deberías dirigirte a cada uno de tus once hermanos y decirles en privado: `El Maestro desea que pasemos a solas con él un período de descanso y esparcimiento. Puesto que todos hemos experimentado recientemente mucha inquietud espiritual y tensión mental, sugiero que durante estas vacaciones no mencionemos para nada nuestras pruebas y dificultades. ¿Puedo contar contigo para que cooperes conmigo en este asunto?' Contacta así con cada uno de tus hermanos de manera privada y personal>>. Y Andrés hizo lo que el Maestro le había ordenado.

\par 
%\textsuperscript{(1611.1)}
\textsuperscript{143:3.3} Éste fue un acontecimiento maravilloso en la experiencia de cada uno de ellos; jamás olvidaron el día que subieron a la montaña. A lo largo de todo el trayecto apenas dijeron una sola palabra de sus dificultades. Al llegar a la cima de la montaña, Jesús los sentó a su alrededor mientras les decía: <<Hermanos míos, todos debéis aprender el valor del descanso y la eficacia del esparcimiento. Debéis comprender que el mejor método para resolver algunos problemas embrollados consiste en alejarse de ellos durante algún tiempo. Luego, cuando volvéis renovados por el descanso o la adoración, sois capaces de atacar vuestras dificultades con una cabeza más despejada y una mano más firme, sin mencionar un corazón más resuelto. Además, muchas veces encontraréis que el tamaño y las proporciones de vuestro problema ha disminuido mientras descansabais vuestra mente y vuestro cuerpo>>.

\par 
%\textsuperscript{(1611.2)}
\textsuperscript{143:3.4} Al día siguiente, Jesús asignó un tema de discusión a cada uno de los doce. Consagraron todo el día a los recuerdos y a hablar de asuntos no relacionados con su trabajo religioso. Se quedaron anonadados durante unos momentos cuando Jesús incluso descuidó dar las gracias ---verbalmente--- al romper el pan para su almuerzo del mediodía. Era la primera vez que lo veían omitir esta formalidad.

\par 
%\textsuperscript{(1611.3)}
\textsuperscript{143:3.5} Cuando subieron a la montaña, la cabeza de Andrés estaba llena de problemas. El corazón de Juan estaba excesivamente perplejo. El alma de Santiago estaba dolorosamente perturbada. Mateo tenía mucha necesidad de fondos debido a la estancia del grupo entre los gentiles. Pedro estaba fatigado y había estado recientemente más temperamental que de costumbre. Judas sufría uno de sus ataques periódicos de susceptibilidad y egoísmo. Simón estaba excepcionalmente trastornado debido a sus esfuerzos por conciliar su patriotismo con el amor de la fraternidad de los hombres. Felipe estaba cada vez más confundido por la manera en que se desarrollaban los acontecimientos. El humor de Natanael había disminuido desde que habían entrado en contacto con las poblaciones gentiles, y Tomás se encontraba en medio de un grave período de depresión. Sólo los gemelos estaban en un estado normal y sin inquietudes. Todos se sentían extremadamente confusos en cuanto a la manera de llevarse pacíficamente con los discípulos de Juan.

\par 
%\textsuperscript{(1611.4)}
\textsuperscript{143:3.6} Al tercer día, cuando empezaron a bajar de la montaña para regresar a su campamento, un gran cambio se había producido en ellos. Habían hecho el importante descubrimiento de que muchas perplejidades humanas no existen en realidad, de que muchas dificultades angustiosas son creadas por un miedo exagerado y producidas por un recelo desmedido. Habían aprendido que la mejor manera de tratar todas las confusiones de este tipo era alejarse de ellas; al irse, habían dejado que estos problemas se resolvieran por sí mismos.

\par 
%\textsuperscript{(1611.5)}
\textsuperscript{143:3.7} El regreso de este descanso marcó el principio de un período de relaciones considerablemente mejores con los seguidores de Juan. Una gran parte de los doce cedió realmente a la hilaridad cuando notaron el cambio del estado mental de cada uno y observaron la ausencia de irritabilidad nerviosa que disfrutaban como consecuencia de sus tres días de vacaciones, alejados de los deberes rutinarios de la vida. Siempre existe el peligro de que la monotonía de los contactos humanos multiplique considerablemente las perplejidades y aumente las dificultades.

\par 
%\textsuperscript{(1611.6)}
\textsuperscript{143:3.8} Pocos gentiles de las dos ciudades griegas de Arquelais y Fasaelis creyeron en el evangelio, pero los doce apóstoles adquirieron una valiosa experiencia con este extenso trabajo, el primero que realizaban con unas poblaciones compuestas exclusivamente de gentiles. Un lunes por la mañana hacia mediados de mes, Jesús le dijo a Andrés: <<Entremos en Samaria>>\footnote{\textit{A través de Samaria}: Jn 4:3-5.}. Y se pusieron en camino inmediatamente hacia la ciudad de Sicar, cerca del pozo de Jacob.

\section*{4. Los judíos y los samaritanos}
\par 
%\textsuperscript{(1612.1)}
\textsuperscript{143:4.1} Durante más de seiscientos años, los judíos de Judea, y más tarde también los de Galilea, habían estado enemistados con los samaritanos. Este sentimiento nocivo entre los judíos y los samaritanos surgió de la manera siguiente: Unos setecientos años a. de J.C., Sargón, rey de Asiria, aplastó una revuelta en Palestina central y se llevó como cautivos a más de veinticinco mil judíos del reino septentrional de Israel, instalando en su lugar a un número casi igual de descendientes de los cutitas, sefarvitas y amatitas. Más tarde, Asurbanipal envió también otras colonias para que vivieran en Samaria.

\par 
%\textsuperscript{(1612.2)}
\textsuperscript{143:4.2} La enemistad religiosa entre los judíos y los samaritanos databa desde el regreso de los judíos de su cautividad en Babilonia, cuando los samaritanos se esforzaron por impedir la reconstrucción de Jerusalén. Más adelante ofendieron a los judíos prestando su ayuda amistosa a los ejércitos de Alejandro. En agradecimiento por su amistad, Alejandro concedió un permiso a los samaritanos para que construyeran un templo en el Monte Gerizim, donde adoraron a Yahvé y a sus dioses tribales, y ofrecieron sacrificios muy semejantes a los de los servicios del templo de Jerusalén. Con este culto continuaron por lo menos hasta la época de los macabeos, cuando Juan Hircano destruyó su templo del Monte Gerizim. Durante sus trabajos a favor de los samaritanos después de la muerte de Jesús, el apóstol Felipe mantuvo numerosas reuniones en el lugar de este antiguo templo samaritano.

\par 
%\textsuperscript{(1612.3)}
\textsuperscript{143:4.3} Los antagonismos entre los judíos y los samaritanos eran históricos y se habían afianzado con el paso del tiempo; desde la época de Alejandro, los dos grupos se habían relacionado cada vez menos. Los doce apóstoles no se oponían a predicar en las ciudades griegas y en otras ciudades gentiles de la Decápolis y Siria, pero fue una dura prueba para su fidelidad al Maestro cuando éste les dijo: <<Entremos en Samaria>>. Sin embargo, durante el año y pico que habían pasado con Jesús, habían desarrollado una forma de lealtad personal que trascendía incluso su fe en las enseñanzas del Maestro y sus prejuicios contra los samaritanos.

\section*{5. La mujer de Sicar}
\par 
%\textsuperscript{(1612.4)}
\textsuperscript{143:5.1} Cuando el Maestro y los doce llegaron al pozo de Jacob\footnote{\textit{El pozo de Jacob}: Jn 4:6.}, Jesús estaba cansado del viaje y se detuvo cerca del pozo, mientras Felipe se llevaba a los apóstoles a Sicar para que le ayudaran a traer la comida y las tiendas, pues tenían la intención de permanecer algún tiempo en aquellos parajes\footnote{\textit{Los apóstoles van a por comida}: Jn 4:8.}. Pedro y los hijos de Zebedeo se hubieran quedado con Jesús, pero éste les rogó que se fueran con sus hermanos, diciendo: <<No temáis por mí, estos samaritanos serán amistosos; sólo nuestros hermanos, los judíos, intentan hacernos daño>>. Eran casi las seis de aquella tarde de verano, cuando Jesús se sentó cerca del pozo para esperar el regreso de los apóstoles.

\par 
%\textsuperscript{(1612.5)}
\textsuperscript{143:5.2} El agua del pozo de Jacob contenía menos minerales que la de los pozos de Sicar, y por eso era más apreciada como agua potable. Jesús tenía sed, pero no disponía de ningún medio para sacar el agua. Por eso, cuando una mujer de Sicar llegó con su cántaro y se dispuso a sacar agua del pozo, Jesús le dijo: <<Dame de beber>>\footnote{\textit{Jesús pide de beber a una mujer}: Jn 4:7.}. Esta mujer de Samaria sabía que Jesús era judío debido a su apariencia y a su vestido, y supuso que era un judío de Galilea a causa de su acento. Se llamaba Nalda y era una hermosa criatura. Se quedó muy sorprendida de que un hombre judío le hablara así al lado del pozo y le pidiera de beber, porque en aquellos tiempos no se consideraba correcto que un hombre que se preciara hablara en público con una mujer, y mucho menos que un judío conversara con una samaritana. Por eso Nalda le preguntó a Jesús: <<¿Cómo es que tú, siendo judío, me pides de beber a mí, a una mujer samaritana?>> Jesús contestó: <<En verdad te he pedido de beber, pero si solamente pudieras comprender, me pedirías un trago de agua viva>>\footnote{\textit{Jesús habla del ``agua de vida''}: Jn 4:10.}. Entonces, Nalda dijo\footnote{\textit{La contestación de Nalda}: Jn 4:9.}: <<Pero Señor, no tienes con qué sacarla, y el pozo es profundo; ¿de dónde tienes pues ese agua viva? ¿Eres más grande que nuestro padre Jacob que nos dio este pozo, del que bebió él mismo y también sus hijos y su ganado?>>\footnote{\textit{¿Cómo puedo sacar ``agua viva''?}: Jn 4:11-12.}

\par 
%\textsuperscript{(1613.1)}
\textsuperscript{143:5.3} Jesús respondió: <<Todo el que bebe de este agua volverá a tener sed, pero cualquiera que beba el agua del espíritu vivo nunca tendrá sed. Este agua viva se volverá en él un manantial refrescante que brotará hasta la vida eterna>>. Nalda dijo entonces: <<Dame de ese agua para no tener más sed, ni tener que venir hasta aquí para sacarla. Además, todo lo que una samaritana pueda recibir de un judío tan digno de elogios será un placer>>\footnote{\textit{Conversación con Nalda}: Jn 4:13-15.}.

\par 
%\textsuperscript{(1613.2)}
\textsuperscript{143:5.4} Nalda no sabía cómo interpretar la buena disposición de Jesús para hablar con ella. Veía en el rostro del Maestro la expresión de un hombre recto y santo, pero tomó su cordialidad por una familiaridad ordinaria, y malinterpretó su simbolismo como una manera de hacerle insinuaciones. Como era una mujer de moral descuidada, estaba dispuesta a volverse abiertamente coqueta cuando Jesús, mirándola directamente a los ojos, le dijo con una voz imperativa: <<Mujer, ve a buscar a tu marido y traelo hasta aquí>>. Esta orden devolvió a Nalda su sentido común. Vio que había juzgado mal la bondad del Maestro; percibió que había interpretado mal el sentido de sus palabras. Estaba asustada; empezó a darse cuenta de que estaba en presencia de una persona excepcional, y buscando a ciegas en su mente una respuesta apropiada, dijo con gran confusión: <<Pero Señor, no puedo llamar a mi marido, porque no tengo marido>>. Entonces dijo Jesús: <<Has dicho la verdad porque, aunque una vez tuviste un marido, el hombre con quien vives ahora no es tu marido. Sería mejor que dejaras de jugar con mis palabras, y buscaras el agua viva que te he ofrecido hoy>>\footnote{\textit{Conversación posterior}: Jn 4:16-18.}.

\par 
%\textsuperscript{(1613.3)}
\textsuperscript{143:5.5} Ahora Nalda había recobrado la seriedad, y su lado bueno se había despertado. No era una mujer inmoral por haberlo elegido así plenamente. Había sido repudiada cruel e injustamente por su marido y, en esta situación desesperada, había consentido en vivir como esposa de cierto griego, pero sin casarse. Nalda se sentía ahora muy avergonzada por haberle hablado a Jesús con tanta ligereza, y se dirigió al Maestro muy arrepentida, diciendo: <<Señor, me arrepiento de la manera en que te he hablado, pues percibo que eres un hombre santo o quizás un profeta>>\footnote{\textit{Un ``hombre santo'' o un profeta}: Jn 4:19.}. Y estaba a punto de solicitar al Maestro una ayuda directa y personal, cuando hizo lo que tantas personas han hecho antes y después de ella ---eludió la cuestión de la salvación personal, orientándose hacia una discusión sobre teología y filosofía. Desvió rápidamente la conversación sobre sus propias necesidades espirituales hacia un debate teológico. Señalando al Monte Gerizim, continuó: <<Nuestros padres adoraban en esta montaña, pero sin embargo, \textit{tú} dirías que el lugar donde los hombres deberían adorar se encuentra en Jerusalén; ¿cuál es pues el lugar apropiado para adorar a Dios?>>\footnote{\textit{¿Cuál es el lugar correcto para adorar a Dios?}: Jn 4:20.}

\par 
%\textsuperscript{(1613.4)}
\textsuperscript{143:5.6} Jesús percibió la tentativa del alma de la mujer por evitar un contacto directo y escrutador con su Hacedor, pero también vio que en su alma estaba presente el deseo de conocer la mejor manera de vivir. Después de todo, en el corazón de Nalda había una verdadera sed de agua viva; la trató pues con paciencia, diciéndole: <<Mujer, déjame decirte que se acerca el día en que no adorarás al Padre ni en esta montaña ni en Jerusalén. Actualmente adoráis aquello que no conocéis, una mezcla de la religión de numerosos dioses paganos y de las filosofías gentiles. Los judíos saben al menos a quien adoran; han eliminado toda confusión, concentrando su adoración en un solo Dios, Yahvé. Deberías creerme cuando digo que se acerca la hora ---e incluso ya está aquí--- en que todos los adoradores sinceros adorarán al Padre en espíritu y en verdad, porque estos son precisamente los adoradores que busca el Padre. Dios es espíritu, y aquellos que lo adoran deben adorarlo en espíritu y en verdad. Tu salvación proviene no de saber cómo deberían adorar los demás o dónde deberían hacerlo, sino de recibir en tu propio corazón este agua viva que te ofrezco en este mismo momento>>\footnote{\textit{En ninguna montaña}: Jn 4:21-22. \textit{Adorar en el espíritu y la verdad}: Jn 4:23-24.}.

\par 
%\textsuperscript{(1614.1)}
\textsuperscript{143:5.7} Pero Nalda haría un esfuerzo más por esquivar la discusión del embarazoso problema de su vida personal en la Tierra y del estado de su alma ante Dios. Una vez más recurrió a cuestiones sobre la religión en general, diciendo: <<Sí, ya sé, Señor, que Juan ha predicado sobre la venida del Convertidor, aquel que será llamado el Libertador, y que cuando venga, nos proclamará todas las cosas..>>. y Jesús, interrumpiendo a Nalda, le dijo con una seguridad sorprendente: <<Yo, que te hablo, soy esa persona>>\footnote{\textit{Jesús es el Mesías}: Jn 4:25-26.}.

\par 
%\textsuperscript{(1614.2)}
\textsuperscript{143:5.8} Ésta era la primera declaración directa, positiva y sin disfraz de su naturaleza y filiación divinas que Jesús hacía en la Tierra; y la hizo a una mujer, a una samaritana, a una mujer de reputación dudosa hasta ese momento a los ojos de los hombres. Pero los ojos divinos veían más a esta mujer como una víctima del pecado de los demás que como una pecadora por su propio deseo, y \textit{ahora} la veían como un alma humana que deseaba la salvación, la deseaba sinceramente y de todo corazón, y con eso bastaba.

\par 
%\textsuperscript{(1614.3)}
\textsuperscript{143:5.9} Cuando Nalda estaba a punto de expresar su anhelo real y personal por las cosas mejores y por una manera más noble de vivir, en el momento en que se disponía a hablar del verdadero deseo de su corazón, los doce apóstoles regresaron de Sicar\footnote{\textit{Los discípulos regresan}: Jn 4:27.}. Al encontrarse con esta escena, la de Jesús hablando tan íntimamente con esta mujer ---esta mujer samaritana, y a solas--- se quedaron más que sorprendidos. Depositaron rápidamente sus provisiones y se apartaron a un lado, sin que nadie se atreviera a censurarlo, mientras Jesús le decía a Nalda: <<Mujer, continúa tu camino; Dios te ha perdonado. De ahora en adelante vivirás una nueva vida. Has recibido el agua viva; una nueva alegría brotará dentro de tu alma, y te convertirás en una hija del Altísimo>>. Al percibir la desaprobación de los apóstoles, la mujer abandonó su cántaro y huyó hacia la ciudad.

\par 
%\textsuperscript{(1614.4)}
\textsuperscript{143:5.10} Al entrar en la ciudad, fue diciendo a todo el que encontró: <<Ve al pozo de Jacob, y date prisa, pues allí verás a un hombre que me ha dicho todo lo que he hecho. ¿Podría ser el Convertidor?>> Antes de ponerse el Sol, un gran gentío se había reunido en el pozo de Jacob para escuchar a Jesús. Y el Maestro les contó más cosas sobre el agua de la vida, el don del espíritu interior\footnote{\textit{Jesús enseña a la multitud en el pozo}: Jn 4:28-30.}.

\par 
%\textsuperscript{(1614.5)}
\textsuperscript{143:5.11} Los apóstoles nunca dejaron de escandalizarse por la buena disposición de Jesús para hablar con las mujeres, con unas mujeres de reputación dudosa, e incluso con mujeres inmorales. A Jesús le resultaba muy difícil enseñar a sus apóstoles que las mujeres, incluso las calificadas de inmorales, tienen un alma que puede escoger a Dios como Padre suyo, y convertirse así en las hijas de Dios y en candidatas a la vida eterna. Incluso diecinueve siglos más tarde, mucha gente muestra la misma aversión a captar las enseñanzas del Maestro. La misma religión cristiana ha sido construida insistentemente alrededor del hecho de la muerte de Cristo, en lugar de hacerlo alrededor de la verdad de su vida. El mundo debería interesarse más por su vida feliz, reveladora de Dios, que por su muerte trágica y triste.

\par 
%\textsuperscript{(1614.6)}
\textsuperscript{143:5.12} Al día siguiente, Nalda contó toda esta historia al apóstol Juan, pero éste nunca la reveló íntegramente a los otros apóstoles, y Jesús no habló detalladamente de esto a los doce.

\par 
%\textsuperscript{(1615.1)}
\textsuperscript{143:5.13} Nalda le contó a Juan que Jesús le había dicho <<todo lo que había hecho>>. Juan quiso muchas veces preguntarle a Jesús sobre esta charla con Nalda, pero nunca lo hizo. Jesús sólo le había dicho a Nalda una cosa sobre sí misma\footnote{\textit{Jesús sólo le dijo una cosa sobre sí misma}: Jn 4:18.}, pero su mirada clavada en sus ojos y la manera de tratarla trajeron en un instante a su mente una revisión panorámica de toda su variada vida, de tal forma que asoció toda esta autorrevelación de su vida pasada con la mirada y las palabras del Maestro. Jesús nunca le dijo que había tenido cinco maridos. Había vivido con cuatro hombres diferentes desde que su marido la había repudiado, y este hecho, junto con todo su pasado, surgió tan vívidamente en su mente cuando se dio cuenta de que Jesús era un hombre de Dios, que posteriormente le repitió a Juan que Jesús le había dicho realmente todo sobre sí misma\footnote{\textit{Jesús me dijo todo lo que había hecho}: Jn 4:29.}.

\section*{6. El renacimiento religioso en Samaria}
\par 
%\textsuperscript{(1615.2)}
\textsuperscript{143:6.1} La tarde que Nalda hizo salir a la muchedumbre de Sicar para ver a Jesús, los doce acababan de regresar con los alimentos y rogaron a Jesús que comiera con ellos en lugar de hablarle a la gente, pues llevaban todo el día sin comer y tenían hambre. Pero Jesús sabía que pronto les envolvería la oscuridad, y por ello persistió en su determinación de hablarle a la gente antes de despedirla. Cuando Andrés intentó persuadirlo para que comiera algo antes de dirigirse a la multitud\footnote{\textit{Le piden a Jesús que coma algo}: Jn 4:31.}, Jesús le dijo: <<Tengo un alimento para comer que vosotros no conocéis>>\footnote{\textit{Jesús tiene un alimento que nadie conoce}: Jn 4:32-38.}. Cuando los apóstoles escucharon esto, se dijeron entre ellos: <<¿Alguien le ha traído algo de comer? ¿Puede ser que la mujer le haya dado alimentos además de bebida?>> Cuando Jesús los escuchó hablando entre ellos, antes de dirigirse a la gente se volvió hacia los doce y les dijo: <<Mi alimento es hacer la voluntad de Aquel que me ha enviado y realizar su obra. Deberíais dejar de decir que falta tanto o tanto tiempo para la cosecha. Contemplad a esta gente que sale de una ciudad samaritana para escucharnos; os digo que los campos ya se han puesto blancos para la cosecha. El que siega recibe su salario y recoge este fruto para la vida eterna; en consecuencia, los sembradores y los segadores se regocijan juntos, porque en esto reside la verdad del refrán: `uno siembra y el otro cosecha'. Ahora os envío a cosechar algo que no habéis trabajado; otros han trabajado, y vosotros estáis a punto de formar parte de su trabajo>>. Dijo esto refiriéndose a la predicación de Juan el Bautista.

\par 
%\textsuperscript{(1615.3)}
\textsuperscript{143:6.2} Jesús y los apóstoles fueron a Sicar y predicaron dos días antes de establecer su campamento en el Monte Gerizim\footnote{\textit{Predicación en el Gerizim}: Jn 4:39-41.}. Muchos habitantes de Sicar creyeron en el evangelio y pidieron ser bautizados, pero los apóstoles de Jesús aún no bautizaban\footnote{\textit{Los apóstoles de Jesús no bautizaban}: Jn 3:22b,26b.}.

\par 
%\textsuperscript{(1615.4)}
\textsuperscript{143:6.3} La primera noche de campamento en el Monte Gerizim, los apóstoles suponían que Jesús les regañaría por su actitud hacia la mujer en el pozo de Jacob, pero él no hizo ninguna referencia a este asunto. En lugar de eso, les dio una charla memorable sobre <<las realidades que son centrales en el reino de Dios>>. En cualquier religión, es muy fácil consentir que los valores se vuelvan desproporcionados y permitir que los hechos ocupen el lugar de la verdad en la teología personal. El hecho de la cruz se volvió el centro mismo del cristianismo posterior, pero ésta no es la verdad central de la religión que se puede deducir de la vida y de las enseñanzas de Jesús de Nazaret.

\par 
%\textsuperscript{(1615.5)}
\textsuperscript{143:6.4} El tema de la enseñanza de Jesús en el Monte Gerizim fue el siguiente: deseaba que todos los hombres vieran a Dios como un Padre-amigo, así como él (Jesús) es un hermano-amigo. Les inculcó una y otra vez que el amor es la relación más grande en el mundo ---en el universo---, al igual que la verdad es la proclamación más grande de la observación de estas relaciones divinas.

\par 
%\textsuperscript{(1616.1)}
\textsuperscript{143:6.5} Jesús se manifestó tan plenamente a los samaritanos porque podía hacerlo sin peligro, y porque sabía que no volvería a visitar el corazón de Samaria para predicar el evangelio del reino.

\par 
%\textsuperscript{(1616.2)}
\textsuperscript{143:6.6} Jesús y los doce acamparon en el Monte Gerizim hasta finales de agosto. Durante el día predicaban la buena nueva del reino ---la paternidad de Dios--- a los samaritanos en las ciudades, y pasaban la noche en el campamento. El trabajo que Jesús y los doce efectuaron en estas ciudades samaritanas dio muchas almas al reino y contribuyó ampliamente a preparar el terreno para la obra maravillosa de Felipe\footnote{\textit{La obra de Felipe}: Hch 8:5-8.} en estas regiones, después de la muerte y resurrección de Jesús, y después de que los apóstoles se dispersaran hasta los confines de la Tierra debido a la persecución encarnizada contra los creyentes en Jerusalén.

\section*{7. Las enseñanzas sobre la oración y la adoración}
\par 
%\textsuperscript{(1616.3)}
\textsuperscript{143:7.1} En las conferencias nocturnas en el Monte Gerizim, Jesús enseñó muchas grandes verdades y recalcó particularmente las siguientes:

\par 
%\textsuperscript{(1616.4)}
\textsuperscript{143:7.2} La verdadera religión es la actuación de un alma individual en sus relaciones conscientes con el Creador; la religión organizada es el intento del hombre por \textit{socializar} la adoración de los practicantes individuales de la religión.

\par 
%\textsuperscript{(1616.5)}
\textsuperscript{143:7.3} La adoración ---la contemplación de lo espiritual--- debe alternar con el servicio, el contacto con la realidad material. El trabajo debería alternar con el esparcimiento; la religión debería estar equilibrada con el humor. La filosofía profunda debería ser aliviada con la poesía rítmica. El esfuerzo por vivir ---la tensión de la personalidad en el tiempo--- debería ser mitigado con el reposo de la adoración. Las sensaciones de inseguridad procedentes del miedo al aislamiento de la personalidad en el universo deberían ser contrarrestadas con la contemplación del Padre, a través de la fe, y con el intento de comprender al Supremo.

\par 
%\textsuperscript{(1616.6)}
\textsuperscript{143:7.4} La oración está destinada a hacer que el hombre piense menos y \textit{comprenda} más; no está destinada a incrementar el conocimiento, sino más bien a ampliar la perspicacia.

\par 
%\textsuperscript{(1616.7)}
\textsuperscript{143:7.5} La adoración tiene el propósito de anticiparse a la vida mejor del futuro, y luego reflejar estas nuevas significaciones espirituales sobre la vida presente. La oración es un sostén espiritual, pero la adoración es divinamente creativa.

\par 
%\textsuperscript{(1616.8)}
\textsuperscript{143:7.6} La adoración es la técnica de buscar en el \textit{Uno} la inspiración para servir a la \textit{multitud}. La adoración es la vara que mide el grado en que el alma se ha desprendido del universo material, y se ha adherido de manera simultánea y segura a las realidades espirituales de toda la creación.

\par 
%\textsuperscript{(1616.9)}
\textsuperscript{143:7.7} La oración es recordarse a sí mismo ---un pensamiento sublime; la adoración es olvidarse de sí mismo--- un superpensamiento. La adoración es una atención sin esfuerzo, el verdadero descanso ideal del alma, una forma de ejercicio espiritual sosegado.

\par 
%\textsuperscript{(1616.10)}
\textsuperscript{143:7.8} La adoración es el acto de un fragmento que se identifica con el Todo, lo finito con lo Infinito, el hijo con el Padre, el tiempo en la operación de ajustarse al ritmo de la eternidad. La adoración es el acto de la comunión personal del hijo con el Padre divino, la aceptación de unas actitudes vivificantes, creativas, fraternales y románticas por parte del alma-espíritu del hombre.

\par 
%\textsuperscript{(1616.11)}
\textsuperscript{143:7.9} Aunque los apóstoles sólo comprendieron una pequeña parte de las enseñanzas del Maestro en el campamento, otros mundos las comprendieron, y otras generaciones de la Tierra las comprenderán.


\chapter{Documento 144. En el Gilboa y la Decápolis}
\par 
%\textsuperscript{(1617.1)}
\textsuperscript{144:0.1} DURANTE los meses de septiembre y octubre se retiraron a un campamento aislado en las laderas del Monte Gilboa. Jesús pasó aquí el mes de septiembre a solas con sus apóstoles, enseñándoles e instruyéndoles en las verdades del reino.

\par 
%\textsuperscript{(1617.2)}
\textsuperscript{144:0.2} Había varias razones para que Jesús y sus apóstoles se retiraran en aquel momento a la frontera de Samaria y la Decápolis. Los dirigentes religiosos de Jerusalén eran muy hostiles; Herodes Antipas aún mantenía a Juan en la cárcel, temiendo tanto ponerlo en libertad como ejecutarlo, y continuaba sospechando que existía algún tipo de complicidad entre Juan y Jesús. En estas condiciones, no era prudente planear una labor dinámica en Judea o en Galilea. Y había una tercera razón: la tensión lentamente creciente entre los jefes de los discípulos de Juan y los apóstoles de Jesús, que empeoraba a medida que aumentaba el número de creyentes.

\par 
%\textsuperscript{(1617.3)}
\textsuperscript{144:0.3} Jesús sabía que el período de trabajo preliminar de enseñanza y predicación casi había terminado, que el paso siguiente sería el comienzo del pleno esfuerzo final de su vida en la Tierra; no deseaba que la puesta en marcha de esta empresa fuera de ninguna manera penosa o embarazosa para Juan el Bautista. Por eso Jesús había decidido pasar algún tiempo aislado, repasando la enseñanza con sus apóstoles, y luego efectuar algún trabajo discreto en las ciudades de la Decápolis, hasta que Juan fuera ejecutado o puesto en libertad para unirse a ellos en un esfuerzo común.

\section*{1. El campamento de Gilboa}
\par 
%\textsuperscript{(1617.4)}
\textsuperscript{144:1.1} A medida que pasaba el tiempo, los doce se consagraban más a Jesús y estaban más comprometidos con el trabajo del reino. Su devoción era en gran parte una cuestión de lealtad personal. No captaban su enseñanza polifacética; no comprendían plenamente la naturaleza de Jesús ni el significado de su donación en la Tierra.

\par 
%\textsuperscript{(1617.5)}
\textsuperscript{144:1.2} Jesús indicó claramente a sus apóstoles que se habían retirado por tres razones:

\par 
%\textsuperscript{(1617.6)}
\textsuperscript{144:1.3} 1. Para confirmar la comprensión que ellos tenían del evangelio del reino, y su fe en el mismo.

\par 
%\textsuperscript{(1617.7)}
\textsuperscript{144:1.4} 2. Para permitir que se calmara la oposición a la obra de ellos, tanto en Judea como en Galilea.

\par 
%\textsuperscript{(1617.8)}
\textsuperscript{144:1.5} 3. Para esperar cuál sería el destino de Juan el Bautista.

\par 
%\textsuperscript{(1617.9)}
\textsuperscript{144:1.6} Mientras se demoraban en el Gilboa, Jesús contó muchas cosas a los doce sobre sus primeros años de vida y sus experiencias en el Monte Hermón; también les reveló algo de lo sucedido en las colinas durante los cuarenta días que siguieron inmediatamente a su bautismo. Y les encargó formalmente que no contaran a nadie estas experiencias hasta después de que hubiera regresado al Padre.

\par 
%\textsuperscript{(1618.1)}
\textsuperscript{144:1.7} Durante estas semanas de septiembre, descansaron, conversaron, relataron sus experiencias desde que Jesús les había llamado por primera vez al servicio, y emprendieron un esfuerzo serio por coordinar lo que el Maestro les había enseñado hasta ese momento. En cierta medida, todos tenían el sentimiento de que ésta sería su última oportunidad para descansar de manera prolongada. Se daban cuenta de que su próximo esfuerzo público, en Judea o en Galilea, marcaría el principio de la proclamación final del reino venidero, pero tenían poca o ninguna idea concreta sobre lo que este reino sería cuando llegara. Juan y Andrés pensaban que el reino ya había llegado. Pedro y Santiago creían que aún estaba por venir. Natanael y Tomás confesaban francamente que estaban perplejos. Mateo, Felipe y Simón Celotes estaban indecisos y confusos. Los gemelos se mantenían felizmente ignorantes de la controversia, y Judas Iscariote guardaba silencio, evasivo.

\par 
%\textsuperscript{(1618.2)}
\textsuperscript{144:1.8} La mayor parte de este tiempo, Jesús estuvo a solas en la montaña, cerca del campamento. De vez en cuando se llevaba a Pedro, Santiago o Juan, pero muy a menudo se iba solo para orar o comulgar. Después del bautismo de Jesús y de los cuarenta días en las colinas de Perea, no es muy exacto calificar de oración estos períodos de comunión con su Padre, y tampoco es consistente decir que Jesús estaba adorando; pero es totalmente correcto sugerir que en estos períodos estaba en comunión personal con su Padre.

\par 
%\textsuperscript{(1618.3)}
\textsuperscript{144:1.9} El tema central de las discusiones, a lo largo de todo el mes de septiembre, fue la oración y la adoración. Después de haber hablado de la adoración durante varios días, Jesús terminó pronunciando su memorable discurso sobre la oración en respuesta a la petición de Tomás: <<Maestro, enséñanos a orar>>\footnote{\textit{Enséñanos a orar}: Lc 11:1.}.

\par 
%\textsuperscript{(1618.4)}
\textsuperscript{144:1.10} Juan había enseñado una oración a sus discípulos, una oración para la salvación en el reino por venir\footnote{\textit{La oración de los discípulos de Juan}: Lc 11:1b.}. Aunque Jesús nunca prohibió a sus seguidores que utilizaran la forma de oración de Juan, los apóstoles percibieron muy pronto que su Maestro no aprobaba plenamente la práctica de expresar oraciones establecidas y formales. Sin embargo, los creyentes solicitaban constantemente que se les enseñara a orar. Los doce anhelaban saber el tipo de súplica que Jesús aprobaría. Debido principalmente a esta necesidad de una súplica sencilla para la gente corriente, Jesús consintió entonces en enseñarles, en respuesta a la petición de Tomás, una forma sugerente de oración. Jesús dio esta lección una tarde durante la tercera semana de la estancia del grupo en el Monte Gilboa.

\section*{2. El discurso sobre la oración}
\par 
%\textsuperscript{(1618.5)}
\textsuperscript{144:2.1} <<En verdad, Juan os ha enseñado una forma sencilla de oración: `¡Oh Padre, límpianos del pecado, muéstranos tu gloria, revélanos tu amor y deja que tu espíritu santifique para siempre nuestro corazón. Amén!' Enseñó esta oración para que tuvierais algo que enseñar a las multitudes. No era su intención que utilizarais esta súplica establecida y formal como expresión de vuestra propia alma en oración>>.

\par 
%\textsuperscript{(1618.6)}
\textsuperscript{144:2.2} <<La oración es una expresión enteramente personal y espontánea de la actitud del alma hacia el espíritu; la oración debería ser la comunión de la filiación y la expresión de la hermandad. Cuando la oración es dictada por el espíritu, conduce al progreso espiritual cooperativo. La oración ideal es una forma de comunión espiritual que conduce a la adoración inteligente. La verdadera oración es la actitud sincera de tender la mano hacia el cielo para conseguir vuestros ideales>>.

\par 
%\textsuperscript{(1619.1)}
\textsuperscript{144:2.3} <<La oración es el aliento del alma y debería induciros a perseverar en vuestro intento por descubrir la voluntad del Padre. Si cualquiera de vosotros tiene un vecino y vais a verle a media noche, diciéndole: `Amigo, préstame tres panes, porque un amigo mío que está de viaje ha venido a verme, y no tengo nada que ofrecerle'; y si vuestro vecino responde, `No me molestes, porque la puerta ya está cerrada y mis hijos y yo estamos acostados; por eso no puedo levantarme para darte el pan', vosotros insistiréis explicándole que vuestro amigo tiene hambre, y que no tenéis ninguna comida que ofrecerle. Os digo que si vuestro vecino no quiere levantarse para daros el pan por amistad hacia vosotros, se levantará a causa de vuestra importunidad y os dará tantos panes como necesitéis. Así pues, si la perseverancia obtiene incluso los favores del hombre mortal, cuánto más vuestra perseverancia en el espíritu conseguirá para vosotros el pan de la vida de las manos complacientes del Padre que está en los cielos. Os lo digo otra vez: Pedid y se os dará; buscad y encontraréis; llamad y se os abrirá. Porque todo el que pide recibe; el que busca encuentra; y al que llama a la puerta de la salvación se le abrirá>>\footnote{\textit{Parábola del vecino y los panes}: Lc 11:5-8. \textit{Pedid, buscad, llamad}: Mt 21:22; Lc 11:5-10. \textit{El que pide, recibe}: Mt 7:7-8; Mc 11:24; Jn 14:13-14; 16:24.}.

\par 
%\textsuperscript{(1619.2)}
\textsuperscript{144:2.4} <<¿Qué padre de entre vosotros, si su hijo le hace una petición imprudente, dudaría en darle según la sabiduría paternal, en lugar de hacerlo en los términos de la demanda errónea del hijo? Si el niño necesita pan, ¿le daréis una piedra simplemente porque la ha pedido tontamente? Si vuestro hijo necesita un pez, ¿le daréis una serpiente de agua simplemente porque ha aparecido una en la red con el pescado, y el niño la pide neciamente? Si vosotros, que sois mortales y finitos, sabéis cómo responder a las peticiones y dar a vuestros hijos unos dones buenos y apropiados, ¿cuánto más, vuestro Padre celestial, dará el espíritu y numerosas bendiciones adicionales a aquellos que se lo pidan? Los hombres deberían orar siempre sin dejarse desanimar>>\footnote{\textit{Regalos apropiados para un hijo}: Mt 7:9-11; Lc 11:11-13.}.

\par 
%\textsuperscript{(1619.3)}
\textsuperscript{144:2.5} <<Dejadme que os cuente la historia de cierto juez que vivía en una ciudad perversa. Este juez no temía a Dios ni tenía respeto por los hombres. Ahora bien, había en esta ciudad una viuda necesitada que iba continuamente a la casa de este juez injusto, diciendo: `Protéjeme de mi adversario'. Durante algún tiempo no quiso prestarle atención, pero pronto se dijo para sus adentros: `Aunque no temo a Dios ni tengo consideración con los hombres, como esta viuda no deja de molestarme, la defenderé para que deje de cansarme con sus continuas visitas'. Os cuento estas historias para animaros a perseverar en la oración, y no para daros a entender que vuestras súplicas modificarán al Padre justo y recto del cielo. En todo caso, vuestra insistencia no es para ganar el favor de Dios, sino para cambiar vuestra actitud terrestre y aumentar la capacidad de vuestra alma para recibir el espíritu>>\footnote{\textit{Parábola del juez altivo}: Lc 18:1-5. \textit{La perseverancia da frutos}: Lc 18:6-8.}.

\par 
%\textsuperscript{(1619.4)}
\textsuperscript{144:2.6} <<Pero cuando oráis, empleáis tan poca fe. Una fe auténtica desplazará las montañas de dificultades materiales que puedan encontrarse en el sendero de la expansión del alma y del progreso espiritual>>\footnote{\textit{Orad con fe}: Mt 17:20; 21:21; 1 Co 13:2.}.

\section*{3. La oración del creyente}
\par 
%\textsuperscript{(1619.5)}
\textsuperscript{144:3.1} Pero los apóstoles aún no estaban satisfechos; deseaban que Jesús les ofreciera una oración modelo que pudieran enseñar a los nuevos discípulos. Después de escuchar este discurso sobre la oración, Santiago Zebedeo dijo: <<Muy bien, Maestro, pero esa forma de oración no la deseamos tanto para nosotros como para los nuevos creyentes que nos piden tan a menudo: `Enseñadnos a orar de manera aceptable al Padre que está en los cielos'.>>\footnote{\textit{La petición de los apóstoles de una oración}: Lc 11:1.}

\par 
%\textsuperscript{(1619.6)}
\textsuperscript{144:3.2} Cuando Santiago terminó de hablar, Jesús dijo: <<Si aún continuáis deseando una oración así, os daré a conocer la que enseñé a mis hermanos y hermanas en Nazaret>>\footnote{\textit{Jesús responde}: Mt 6:9a; Lc 11:2a.}:

\par 
%\textsuperscript{(1620.1)}
\textsuperscript{144:3.3} Padre nuestro que estás en los cielos,

\par 
%\textsuperscript{(1620.2)}
\textsuperscript{144:3.4} Santificado sea tu nombre.

\par 
%\textsuperscript{(1620.3)}
\textsuperscript{144:3.5} Que venga tu reino; que se haga tu voluntad

\par 
%\textsuperscript{(1620.4)}
\textsuperscript{144:3.6} En la Tierra al igual que en el cielo.

\par 
%\textsuperscript{(1620.5)}
\textsuperscript{144:3.7} Danos hoy nuestro pan para mañana;

\par 
%\textsuperscript{(1620.6)}
\textsuperscript{144:3.8} Vivifica nuestra alma con el agua de la vida.

\par 
%\textsuperscript{(1620.7)}
\textsuperscript{144:3.9} Y perdónanos nuestras deudas

\par 
%\textsuperscript{(1620.8)}
\textsuperscript{144:3.10} Como nosotros también hemos perdonado a nuestros deudores.

\par 
%\textsuperscript{(1620.9)}
\textsuperscript{144:3.11} Sálvanos de la tentación, líbranos del mal,

\par 
%\textsuperscript{(1620.10)}
\textsuperscript{144:3.12} Y haznos cada vez más perfectos como tú mismo\footnote{\textit{La oración del Señor}: Mt 6:9-13; Lc 11:2-4.}.
\bigbreak
\par 
%\textsuperscript{(1620.11)}
\textsuperscript{144:3.13} No es de extrañar que los apóstoles desearan que Jesús les enseñara una oración modelo para los creyentes. Juan el Bautista había enseñado varias oraciones a sus seguidores; todos los grandes instructores habían formulado oraciones para sus alumnos. Los educadores religiosos de los judíos tenían unas veinticinco o treinta oraciones establecidas, que recitaban en las sinagogas e incluso en las esquinas de la calle. Jesús era particularmente contrario a orar en público. Hasta ese momento, los doce sólo lo habían escuchado rezar unas pocas veces. Observaban que pasaba las noches enteras orando o adorando, y tenían mucha curiosidad por conocer el método o la forma de sus súplicas. Se sentían acosados y sin saber qué contestar a las multitudes cuando éstas les pedían que les enseñaran a rezar, como Juan había enseñado a sus discípulos.

\par 
%\textsuperscript{(1620.12)}
\textsuperscript{144:3.14} Jesús enseñó a los doce a orar siempre en secreto\footnote{\textit{Orad en secreto}: Mt 6:6.}; a salir a solas en medio de los tranquilos contornos de la naturaleza, o a entrar en sus habitaciones y cerrar las puertas cuando se pusieran a orar.

\par 
%\textsuperscript{(1620.13)}
\textsuperscript{144:3.15} Después de la muerte de Jesús y de su ascensión hacia el Padre, muchos creyentes adoptaron la costumbre de terminar este llamado Padre nuestro, añadiendo: <<En el nombre del Señor Jesucristo>>. Más tarde aún, dos líneas se perdieron al copiarse esta oración, y se añadió una cláusula adicional que decía: <<Porque tuyo es el reino, el poder y la gloria, para siempre>>\footnote{\textit{Adición de ``tuyo es el poder y la gloria''}: Mt 6:13b.}.

\par 
%\textsuperscript{(1620.14)}
\textsuperscript{144:3.16} Jesús ofreció esta oración a los apóstoles, de manera colectiva, tal como la rezaban en el hogar de Nazaret. Nunca enseñó una oración personal formalista, sino únicamente súplicas colectivas, familiares o sociales. Y nunca lo hizo por su propia voluntad.

\par 
%\textsuperscript{(1620.15)}
\textsuperscript{144:3.17} Jesús enseñó que la oración eficaz debe ser:

\par 
%\textsuperscript{(1620.16)}
\textsuperscript{144:3.18} 1. Altruista ---no solamente para sí mismo.

\par 
%\textsuperscript{(1620.17)}
\textsuperscript{144:3.19} 2. Creyente ---conforme a la fe.

\par 
%\textsuperscript{(1620.18)}
\textsuperscript{144:3.20} 3. Sincera ---honrada de corazón.

\par 
%\textsuperscript{(1620.19)}
\textsuperscript{144:3.21} 4. Inteligente ---conforme a la luz.

\par 
%\textsuperscript{(1620.20)}
\textsuperscript{144:3.22} 5. Confiada ---sometida a la voluntad infinitamente sabia del Padre.

\par 
%\textsuperscript{(1620.21)}
\textsuperscript{144:3.23} Cuando Jesús pasaba noches enteras rezando en la montaña, lo hacía principalmente por sus discípulos, y en particular por los doce. El Maestro oraba muy poco para sí mismo, aunque practicaba mucho la adoración, cuya naturaleza era una comunión comprensiva con su Padre Paradisiaco.

\section*{4. Más cosas sobre la oración}
\par 
%\textsuperscript{(1620.22)}
\textsuperscript{144:4.1} Durante los días siguientes al discurso sobre la oración, los apóstoles continuaron haciéndole preguntas al Maestro sobre esta práctica cultual importantísima. Las instrucciones que Jesús impartió a los apóstoles durante aquellos días sobre la oración y la adoración se pueden resumir y exponer en un lenguaje moderno de la manera siguiente:

\par 
%\textsuperscript{(1621.1)}
\textsuperscript{144:4.2} La repetición seria y anhelante de una súplica cualquiera, cuando esa oración es la expresión sincera de un hijo de Dios y es manifestada con fe, por muy descaminada que esté o por muy imposible que sea de responder directamente, nunca deja de aumentar la capacidad de recepción espiritual del alma.

\par 
%\textsuperscript{(1621.2)}
\textsuperscript{144:4.3} En todas las oraciones, recordad que la filiación es un \textit{don}. Ningún niño tiene que hacer nada para \textit{conseguir} la condición de hijo o de hija. El hijo terrestre surge a la existencia por voluntad de sus padres. De la misma manera, el hijo de Dios llega a la gracia y a la nueva vida del espíritu por voluntad del Padre que está en los cielos. Por eso, el reino de los cielos ---la filiación divina--- debe \textit{recibirse} como lo recibiría un niño pequeño. La rectitud --- el desarrollo progresivo del carácter ---se adquiere, pero la filiación se recibe por la gracia y a través de la fe\footnote{\textit{La filiación se recibe al igual que un niño}: Mt 18:2-4; 19:13-14; Mc 9:36-37; 10:13-16; Lc 9:47-48; 18:16-17.}.

\par 
%\textsuperscript{(1621.3)}
\textsuperscript{144:4.4} La oración condujo a Jesús a la supercomunión de su alma con los Gobernantes Supremos del universo de universos. La oración conducirá a los mortales de la Tierra a la comunión de la verdadera adoración. La capacidad espiritual de recepción del alma determina la cantidad de bendiciones celestiales que uno puede apropiarse personalmente, y comprender conscientemente, como respuesta a la oración.

\par 
%\textsuperscript{(1621.4)}
\textsuperscript{144:4.5} La oración, y la adoración que la acompaña, es una técnica para apartarse de la rutina diaria de la vida, de los agobios monótonos de la existencia material. Es una vía para acercarse a la autorrealización espiritualizada y para conseguir la individualidad intelectual y religiosa.

\par 
%\textsuperscript{(1621.5)}
\textsuperscript{144:4.6} La oración es un antídoto contra la introspección nociva. La oración, al menos tal como la enseñó el Maestro, es una ayuda benéfica para el alma. Jesús empleó convenientemente la influencia benéfica de la oración para sus propios semejantes. El Maestro oraba generalmente en plural, no en singular. Jesús solamente oró para sí mismo en las grandes crisis de su vida terrestre.

\par 
%\textsuperscript{(1621.6)}
\textsuperscript{144:4.7} La oración es el aliento de la vida del espíritu en medio de la civilización material de las razas de la humanidad. La adoración es la salvación para las generaciones de mortales que persiguen los placeres.

\par 
%\textsuperscript{(1621.7)}
\textsuperscript{144:4.8} Al igual que la oración se puede asemejar a la recarga de las baterías espirituales del alma, la adoración se puede comparar al acto de sintonizar el alma para captar las emisiones universales del espíritu infinito del Padre Universal.

\par 
%\textsuperscript{(1621.8)}
\textsuperscript{144:4.9} La oración es la mirada sincera y anhelante que el hijo dirige a su Padre espiritual; es un proceso psicológico que consiste en intercambiar la voluntad humana por la voluntad divina. La oración es una parte del plan divino para transformar lo que es en lo que debería ser.

\par 
%\textsuperscript{(1621.9)}
\textsuperscript{144:4.10} Una de las razones por las cuales Pedro, Santiago y Juan, que con tanta frecuencia acompañaron a Jesús en sus largas vigilias nocturnas, nunca lo escucharon rezar, es porque su Maestro raramente expresaba sus oraciones en un lenguaje hablado. Jesús efectuaba prácticamente todas sus oraciones en espíritu y en su corazón ---en silencio.

\par 
%\textsuperscript{(1621.10)}
\textsuperscript{144:4.11} De todos los apóstoles, Pedro y Santiago son los que estuvieron más cerca de comprender las enseñanzas del Maestro sobre la oración y la adoración.

\section*{5. Otras formas de oración}
\par 
%\textsuperscript{(1621.11)}
\textsuperscript{144:5.1} De vez en cuando, durante el resto de su estancia en la Tierra, Jesús atrajo la atención de los apóstoles sobre diversas formas adicionales de oración, pero sólo lo hizo para ilustrar otras cuestiones, y les recomendó que no enseñaran a las multitudes estas <<oraciones en parábolas>>. Muchas de ellas procedían de otros planetas habitados, pero Jesús no reveló este hecho a los doce. Entre estas oraciones se encontraban las siguientes:
\begin{center}
\par 
%\textsuperscript{(1622.1)}
\textsuperscript{144:5.2} Padre nuestro en quien consisten los reinos del universo,

\par 
%\textsuperscript{(1622.2)}
\textsuperscript{144:5.3} Que tu nombre sea elevado y tu carácter glorificado.

\par 
%\textsuperscript{(1622.3)}
\textsuperscript{144:5.4} Tu presencia nos rodea, y tu gloria se manifiesta

\par 
%\textsuperscript{(1622.4)}
\textsuperscript{144:5.5} Imperfectamente a través de nosotros, así como se muestra en perfección en el cielo.

\par 
%\textsuperscript{(1622.5)}
\textsuperscript{144:5.6} Danos hoy las fuerzas vivificantes de la luz,

\par 
%\textsuperscript{(1622.6)}
\textsuperscript{144:5.7} Y no dejes que nos desviemos por las sendas perversas de nuestra imaginación,

\par 
%\textsuperscript{(1622.7)}
\textsuperscript{144:5.8} Porque tuya es la gloriosa presencia interior, el poder eterno,

\par 
%\textsuperscript{(1622.8)}
\textsuperscript{144:5.9} Y para nosotros, el don eterno del amor infinito de tu Hijo.

\par 
%\textsuperscript{(1622.9)}
\textsuperscript{144:5.10} Así sea, y es eternamente verdad.
\end{center}

\begin{center}
	\par * * *
\end{center}

\begin{center}
\par 
%\textsuperscript{(1622.10)}
\textsuperscript{144:5.11} Padre nuestro creador, que estás en el centro del universo,

\par 
%\textsuperscript{(1622.11)}
\textsuperscript{144:5.12} Otórganos tu naturaleza y danos tu carácter.

\par 
%\textsuperscript{(1622.12)}
\textsuperscript{144:5.13} Haz de nosotros tus hijos e hijas por la gracia

\par 
%\textsuperscript{(1622.13)}
\textsuperscript{144:5.14} Y glorifica tu nombre a través de nuestro perfeccionamiento eterno.

\par 
%\textsuperscript{(1622.14)}
\textsuperscript{144:5.15} Danos tu espíritu ajustador y controlador para que viva y resida en nosotros

\par 
%\textsuperscript{(1622.15)}
\textsuperscript{144:5.16} Para que podamos hacer tu voluntad en esta esfera, como los ángeles ejecutan tus órdenes en la luz.

\par 
%\textsuperscript{(1622.16)}
\textsuperscript{144:5.17} Sosténnos hoy en nuestro progreso a lo largo del camino de la verdad.

\par 
%\textsuperscript{(1622.17)}
\textsuperscript{144:5.18} Líbranos de la inercia, del mal y de toda transgresión pecaminosa.

\par 
%\textsuperscript{(1622.18)}
\textsuperscript{144:5.19} Sé paciente con nosotros, como nosotros mostramos misericordia a nuestros semejantes.

\par 
%\textsuperscript{(1622.19)}
\textsuperscript{144:5.20} Derrama ampliamente el espíritu de tu misericordia en nuestros corazones de criaturas.

\par 
%\textsuperscript{(1622.20)}
\textsuperscript{144:5.21} Guíanos con tu propia mano, paso a paso, por el incierto laberinto de la vida,

\par 
%\textsuperscript{(1622.21)}
\textsuperscript{144:5.22} Y cuando llegue nuestro fin, recibe en tu propio seno nuestro espíritu fiel.

\par 
%\textsuperscript{(1622.22)}
\textsuperscript{144:5.23} Así sea, que se haga tu voluntad y no nuestros deseos.
\end{center}

\begin{center}
	\par * * *
\end{center}

\begin{center}
\par 
%\textsuperscript{(1622.23)}
\textsuperscript{144:5.24} Padre nuestro celestial, perfecto y justo,

\par 
%\textsuperscript{(1622.24)}
\textsuperscript{144:5.25} Guía y dirige hoy nuestro viaje.

\par 
%\textsuperscript{(1622.25)}
\textsuperscript{144:5.26} Santifica nuestros pasos y coordina nuestros pensamientos.

\par 
%\textsuperscript{(1622.26)}
\textsuperscript{144:5.27} Condúcenos siempre por los caminos del progreso eterno.

\par 
%\textsuperscript{(1622.27)}
\textsuperscript{144:5.28} Llénanos de sabiduría hasta la plenitud del poder

\par 
%\textsuperscript{(1622.28)}
\textsuperscript{144:5.29} Y vivifícanos con tu energía infinita.

\par 
%\textsuperscript{(1622.29)}
\textsuperscript{144:5.30} Inspíranos con la conciencia divina de

\par 
%\textsuperscript{(1622.30)}
\textsuperscript{144:5.31} La presencia y la guía de las huestes seráficas.

\par 
%\textsuperscript{(1622.31)}
\textsuperscript{144:5.32} Guíanos siempre hacia arriba por el sendero de la luz;

\par 
%\textsuperscript{(1622.32)}
\textsuperscript{144:5.33} Justifícanos plenamente el día del gran juicio.

\par 
%\textsuperscript{(1622.33)}
\textsuperscript{144:5.34} Haznos semejantes a ti en gloria eterna

\par 
%\textsuperscript{(1622.34)}
\textsuperscript{144:5.35} Y recíbenos a tu servicio perpetuo en el cielo.
\end{center}

\begin{center}
	\par * * *
\end{center}

\begin{center}
\par 
%\textsuperscript{(1622.35)}
\textsuperscript{144:5.36} Padre nuestro, que permaneces en el misterio,

\par 
%\textsuperscript{(1622.36)}
\textsuperscript{144:5.37} Revélanos tu santo carácter.

\par 
%\textsuperscript{(1622.37)}
\textsuperscript{144:5.38} Concede hoy a tus hijos de la Tierra

\par 
%\textsuperscript{(1622.38)}
\textsuperscript{144:5.39} Que vean el camino, la luz y la verdad.

\par 
%\textsuperscript{(1622.39)}
\textsuperscript{144:5.40} Muéstranos el sendero del progreso eterno,

\par 
%\textsuperscript{(1622.40)}
\textsuperscript{144:5.41} Y danos la voluntad de caminar en él.

\par 
%\textsuperscript{(1622.41)}
\textsuperscript{144:5.42} Establece dentro de nosotros tu soberanía divina

\par 
%\textsuperscript{(1622.42)}
\textsuperscript{144:5.43} Y otórganos así el completo dominio del yo.

\par 
%\textsuperscript{(1622.43)}
\textsuperscript{144:5.44} No dejes que nos desviemos por los senderos de las tinieblas y de la muerte;

\par 
%\textsuperscript{(1622.44)}
\textsuperscript{144:5.45} Condúcenos perpetuamente cerca de las aguas de la vida.

\par 
%\textsuperscript{(1622.45)}
\textsuperscript{144:5.46} Escucha estas oraciones nuestras por tu propio bien;

\par 
%\textsuperscript{(1622.46)}
\textsuperscript{144:5.47} Complácete en hacernos cada vez más semejantes a ti.

\par 
%\textsuperscript{(1623.1)}
\textsuperscript{144:5.48} Al final, por el amor del Hijo divino,

\par 
%\textsuperscript{(1623.2)}
\textsuperscript{144:5.49} Recíbenos en los brazos eternos.

\par 
%\textsuperscript{(1623.3)}
\textsuperscript{144:5.50} Así sea, que se haga tu voluntad y no la nuestra.
\end{center}

\begin{center}
	\par * * *
\end{center}

\begin{center}
\par 
%\textsuperscript{(1623.4)}
\textsuperscript{144:5.51} Glorioso Padre y Madre, fundidos en un solo ascendiente,

\par 
%\textsuperscript{(1623.5)}
\textsuperscript{144:5.52} Quisiéramos ser fieles a tu naturaleza divina.

\par 
%\textsuperscript{(1623.6)}
\textsuperscript{144:5.53} Que tu propio yo viva de nuevo en nosotros y a través de nosotros

\par 
%\textsuperscript{(1623.7)}
\textsuperscript{144:5.54} Mediante el don y el otorgamiento de tu espíritu divino,

\par 
%\textsuperscript{(1623.8)}
\textsuperscript{144:5.55} Reproduciéndote así imperfectamente en esta esfera

\par 
%\textsuperscript{(1623.9)}
\textsuperscript{144:5.56} Como te muestras de manera perfecta y majestuosa en el cielo.

\par 
%\textsuperscript{(1623.10)}
\textsuperscript{144:5.57} Danos día tras día tu dulce ministerio de fraternidad

\par 
%\textsuperscript{(1623.11)}
\textsuperscript{144:5.58} Y condúcenos en todo momento por el sendero del servicio afectuoso.

\par 
%\textsuperscript{(1623.12)}
\textsuperscript{144:5.59} Sé siempre e incansablemente paciente con nosotros

\par 
%\textsuperscript{(1623.13)}
\textsuperscript{144:5.60} Como nosotros mostramos tu paciencia a nuestros hijos.

\par 
%\textsuperscript{(1623.14)}
\textsuperscript{144:5.61} Danos la sabiduría divina que hace bien todas las cosas

\par 
%\textsuperscript{(1623.15)}
\textsuperscript{144:5.62} Y el amor infinito que es bondadoso con todas las criaturas.

\par 
%\textsuperscript{(1623.16)}
\textsuperscript{144:5.63} Otórganos tu paciencia y tu misericordia,

\par 
%\textsuperscript{(1623.17)}
\textsuperscript{144:5.64} Para que nuestra caridad envuelva a los débiles del mundo.

\par 
%\textsuperscript{(1623.18)}
\textsuperscript{144:5.65} Y cuando termine nuestra carrera, haz de ella un honor para tu nombre,

\par 
%\textsuperscript{(1623.19)}
\textsuperscript{144:5.66} Un placer para tu buen espíritu, y una satisfacción para los que ayudan a nuestra alma.

\par 
%\textsuperscript{(1623.20)}
\textsuperscript{144:5.67} Que el bien eterno de tus hijos mortales no sea el que nosotros anhelamos, afectuoso Padre nuestro, sino el que tú deseas.

\par 
%\textsuperscript{(1623.21)}
\textsuperscript{144:5.68} Que así sea.
\end{center}

\begin{center}
	\par * * *
\end{center}

\begin{center}
\par 
%\textsuperscript{(1623.22)}
\textsuperscript{144:5.69} Fuente nuestra totalmente fiel y Centro todopoderoso nuestro,

\par 
%\textsuperscript{(1623.23)}
\textsuperscript{144:5.70} Que el nombre de tu Hijo lleno de bondad sea santificado y venerado.

\par 
%\textsuperscript{(1623.24)}
\textsuperscript{144:5.71} Tus generosidades y tus bendiciones han descendido sobre nosotros,

\par 
%\textsuperscript{(1623.25)}
\textsuperscript{144:5.72} Dándonos fuerza para hacer tu voluntad y ejecutar tus mandatos.

\par 
%\textsuperscript{(1623.26)}
\textsuperscript{144:5.73} Danos en todo momento el sustento del árbol de la vida;

\par 
%\textsuperscript{(1623.27)}
\textsuperscript{144:5.74} Refréscanos día tras día con las aguas vivas del río de la vida.

\par 
%\textsuperscript{(1623.28)}
\textsuperscript{144:5.75} Condúcenos paso a paso fuera de las tinieblas y hacia la luz divina.

\par 
%\textsuperscript{(1623.29)}
\textsuperscript{144:5.76} Renueva nuestra mente mediante las transformaciones del espíritu interior,

\par 
%\textsuperscript{(1623.30)}
\textsuperscript{144:5.77} Y cuando llegue finalmente nuestro fin mortal,

\par 
%\textsuperscript{(1623.31)}
\textsuperscript{144:5.78} Recíbenos contigo y envíanos a la eternidad.

\par 
%\textsuperscript{(1623.32)}
\textsuperscript{144:5.79} Corónanos con las diademas celestiales del servicio fructífero,

\par 
%\textsuperscript{(1623.33)}
\textsuperscript{144:5.80} Y glorificaremos al Padre, al Hijo y a la Santa Influencia.

\par 
%\textsuperscript{(1623.34)}
\textsuperscript{144:5.81} Que así sea, en todo un universo sin fin.
\end{center}

\begin{center}
	\par * * *
\end{center}

\begin{center}
\par 
%\textsuperscript{(1623.35)}
\textsuperscript{144:5.82} Padre nuestro que resides en los lugares secretos del universo,

\par 
%\textsuperscript{(1623.36)}
\textsuperscript{144:5.83} Que tu nombre sea honrado, tu misericordia venerada, y tu juicio respetado.

\par 
%\textsuperscript{(1623.37)}
\textsuperscript{144:5.84} Que el Sol de la rectitud brille sobre nosotros a mediodía,

\par 
%\textsuperscript{(1623.38)}
\textsuperscript{144:5.85} Mientras te suplicamos que guíes nuestros pasos descarriados en el crepúsculo.

\par 
%\textsuperscript{(1623.39)}
\textsuperscript{144:5.86} Llévanos de la mano por los caminos que tú mismo has escogido,

\par 
%\textsuperscript{(1623.40)}
\textsuperscript{144:5.87} Y no nos abandones cuando la senda sea dura y las horas sombrías.

\par 
%\textsuperscript{(1623.41)}
\textsuperscript{144:5.88} No nos olvides como nosotros te olvidamos y abandonamos tan a menudo.

\par 
%\textsuperscript{(1623.42)}
\textsuperscript{144:5.89} Pero sé misericordioso y ámanos como nosotros deseamos amarte.

\par 
%\textsuperscript{(1623.43)}
\textsuperscript{144:5.90} Míranos desde arriba con benevolencia y perdónanos con misericordia

\par 
%\textsuperscript{(1623.44)}
\textsuperscript{144:5.91} Como nosotros perdonamos en justicia a los que nos afligen y nos perjudican.

\par 
%\textsuperscript{(1624.1)}
\textsuperscript{144:5.92} Que el amor, la devoción y la donación del Hijo majestuoso,

\par 
%\textsuperscript{(1624.2)}
\textsuperscript{144:5.93} Nos proporcionen la vida eterna con tu misericordia y amor sin fin.

\par 
%\textsuperscript{(1624.3)}
\textsuperscript{144:5.94} Que el Dios de los universos nos otorgue la plena medida de su espíritu;

\par 
%\textsuperscript{(1624.4)}
\textsuperscript{144:5.95} Danos la gracia de someternos a las directrices de este espíritu.

\par 
%\textsuperscript{(1624.5)}
\textsuperscript{144:5.96} Por el ministerio afectuoso de las leales huestes seráficas

\par 
%\textsuperscript{(1624.6)}
\textsuperscript{144:5.97} Que el Hijo nos guíe y nos conduzca hasta el final de la era.

\par 
%\textsuperscript{(1624.7)}
\textsuperscript{144:5.98} Haznos siempre cada vez más semejantes a ti mismo

\par 
%\textsuperscript{(1624.8)}
\textsuperscript{144:5.99} Y cuando llegue nuestro fin, recíbenos en el abrazo eterno del Paraíso.

\par 
%\textsuperscript{(1624.9)}
\textsuperscript{144:5.100} Que así sea, en nombre del Hijo donador

\par 
%\textsuperscript{(1624.10)}
\textsuperscript{144:5.101} Para el honor y la gloria del Padre Supremo.
\end{center}

\par 
%\textsuperscript{(1624.11)}
\textsuperscript{144:5.102} Aunque los apóstoles no tenían la libertad de exponer estas lecciones sobre la oración en sus enseñanzas públicas, todas estas revelaciones les resultaron muy provechosas en sus experiencias religiosas personales. Jesús utilizó como ejemplos estos modelos de oración y otros más en conexión con la instrucción íntima de los doce, y se ha concedido un permiso expreso para transcribir estos siete modelos de oración en este relato.

\section*{6. La conferencia con los apóstoles de Juan}
\par 
%\textsuperscript{(1624.12)}
\textsuperscript{144:6.1} Hacia primeros de octubre, Felipe y algunos de sus compañeros apóstoles estaban en un pueblo cercano comprando provisiones, cuando se encontraron con algunos de los apóstoles de Juan el Bautista. Este encuentro fortuito en la plaza del mercado tuvo como resultado una conferencia de tres semanas, en el campamento de Gilboa, entre los apóstoles de Jesús y los apóstoles de Juan, porque Juan, siguiendo el ejemplo de Jesús, había nombrado recientemente como apóstoles a doce de sus principales discípulos. Juan había hecho esto debido a la insistencia de Abner, el jefe de sus leales partidarios. Jesús estuvo presente en el campamento de Gilboa toda la primera semana de esta conferencia conjunta, pero se ausentó durante las dos últimas.

\par 
%\textsuperscript{(1624.13)}
\textsuperscript{144:6.2} A principios de la segunda semana de este mes, Abner había reunido a todos sus compañeros en el campamento de Gilboa y estaba preparado para deliberar con los apóstoles de Jesús. Durante tres semanas, estos veinticuatro hombres celebraron sus sesiones tres veces al día y seis días por semana. La primera semana, Jesús se mezcló con ellos en sus sesiones de la mañana, de la tarde y de la noche. Querían que el Maestro se reuniera con ellos y presidiera sus deliberaciones conjuntas, pero él se negó firmemente a participar en sus discusiones, aunque consintió en hablarles en tres ocasiones. Estas charlas de Jesús a los veinticuatro trataron de la comprensión, la cooperación y la tolerancia.

\par 
%\textsuperscript{(1624.14)}
\textsuperscript{144:6.3} Andrés y Abner presidieron alternativamente estas reuniones conjuntas de los dos grupos apostólicos. Estos hombres tenían muchas dificultades que tratar y numerosos problemas que resolver. Una y otra vez quisieron someter sus inquietudes a Jesús, sin otro resultado que oírle decir: <<Sólo me ocupo de vuestros problemas personales y puramente religiosos. Soy el representante del Padre para los \textit{individuos}, no para los grupos. Si tenéis dificultades personales en vuestras relaciones con Dios, venid a mí; os escucharé y os aconsejaré para que solucionéis vuestro problema. Pero si os ponéis a coordinar las interpretaciones humanas divergentes de las cuestiones religiosas, y a socializar la religión, estáis destinados a solucionar todos esos problemas con vuestras propias decisiones. Sin embargo, contad siempre con mi simpatía y mi interés. Cuando lleguéis a vuestras conclusiones en relación con estos temas sin importancia espiritual, con tal que estéis todos de acuerdo, os prometo de antemano toda mi aprobación y mi cooperación sincera. Y ahora, para no estorbaros en vuestras deliberaciones, os dejo durante dos semanas. No os inquietéis por mí, pues regresaré a vosotros. Estaré ocupado en los asuntos de mi Padre, porque tenemos otros reinos además de éste>>\footnote{\textit{Los asuntos de mi Padre, otros reinos}: Lc 2:49; Jn 10:16.}.

\par 
%\textsuperscript{(1625.1)}
\textsuperscript{144:6.4} Después de hablar así, Jesús descendió por la ladera de la montaña y no lo volvieron a ver durante dos semanas completas. No supieron nunca dónde había ido ni qué había hecho durante aquellos días. Se quedaron tan desconcertados por la ausencia del Maestro, que los veinticuatro necesitaron algún tiempo para ponerse a considerar seriamente sus problemas. Sin embargo, al cabo de una semana estaban sumergidos de nuevo en sus discusiones, y no podían recurrir a Jesús para que les ayudara.

\par 
%\textsuperscript{(1625.2)}
\textsuperscript{144:6.5} El primer asunto que el grupo acordó fue adoptar la oración que Jesús les había enseñado tan recientemente. Votaron por unanimidad que aceptaban esta oración como la única que los dos grupos de apóstoles enseñarían a los creyentes.

\par 
%\textsuperscript{(1625.3)}
\textsuperscript{144:6.6} A continuación decidieron que mientras Juan viviera, ya sea en la cárcel o fuera de ella, ambos grupos de doce apóstoles continuarían con su propio trabajo, y que cada tres meses celebrarían reuniones conjuntas de una semana en lugares a convenir de vez en cuando.

\par 
%\textsuperscript{(1625.4)}
\textsuperscript{144:6.7} Pero el más grave de todos sus problemas era la cuestión del bautismo\footnote{\textit{Disputa sobre el bautismo}: Jn 3:22-26.}. Sus dificultades se habían agravado mucho más porque Jesús se había negado a pronunciarse sobre el tema. Finalmente acordaron lo siguiente\footnote{\textit{El acuerdo sobre el bautismo}: Jn 4:2.}: Mientras Juan viviera, o hasta que modificaran esta decisión de manera conjunta, sólo los apóstoles de Juan bautizarían a los creyentes, y sólo los apóstoles de Jesús completarían la instrucción de los nuevos discípulos. En consecuencia, desde aquel momento hasta después de la muerte de Juan, dos apóstoles de Juan acompañaron a Jesús y sus apóstoles para bautizar a los creyentes, pues el consejo conjunto había votado por unanimidad que el bautismo se convertiría en el paso inicial de la alianza exterior con los asuntos del reino.

\par 
%\textsuperscript{(1625.5)}
\textsuperscript{144:6.8} A continuación acordaron que, si Juan moría, sus apóstoles se presentarían ante Jesús y se someterían a su dirección, y que dejarían de bautizar a menos que fueran autorizados por Jesús o sus apóstoles.

\par 
%\textsuperscript{(1625.6)}
\textsuperscript{144:6.9} Después votaron que, en el caso de que Juan muriera, los apóstoles de Jesús empezarían a bautizar con agua como símbolo del bautismo del Espíritu divino\footnote{\textit{El significado del bautismo}: Mt 3:11; Mc 1:4,8; Lc 3:3,16; Jn 1:26-27,33.}. La cuestión de si el \textit{arrepentimiento} debía ligarse o no a la predicación del bautismo se dejó opcional; no se tomó ninguna decisión obligatoria para el grupo. Los apóstoles de Juan predicaban: <<Arrepentíos y sed bautizados>>, y los apóstoles de Jesús proclamaban: <<Creed y sed bautizados>>.

\par 
%\textsuperscript{(1625.7)}
\textsuperscript{144:6.10} Ésta es la historia del primer intento de los seguidores de Jesús por coordinar los esfuerzos divergentes, ajustar las diferencias de opinión, organizar las empresas colectivas, regular las observancias externas y socializar las prácticas religiosas personales.

\par 
%\textsuperscript{(1625.8)}
\textsuperscript{144:6.11} Examinaron otras muchas cuestiones menores y llegaron a un acuerdo unánime para solucionarlas. Estos veinticuatro hombres tuvieron una experiencia verdaderamente notable durante las dos semanas que se vieron obligados a enfrentarse con los problemas y a arreglar las dificultades sin Jesús. Aprendieron a discrepar, a discutir, a litigar, a orar y a transigir, y desde el principio al fin, a experimentar simpatía por el punto de vista de la otra persona y a mantener al menos cierto grado de tolerancia por sus opiniones sinceras.

\par 
%\textsuperscript{(1625.9)}
\textsuperscript{144:6.12} Jesús regresó la tarde de la discusión final sobre los asuntos financieros; se enteró de sus deliberaciones, escuchó sus decisiones y dijo: <<Éstas son pues vuestras conclusiones; ayudaré a cada uno de vosotros a llevar a cabo el espíritu de vuestras decisiones conjuntas>>.

\par 
%\textsuperscript{(1626.1)}
\textsuperscript{144:6.13} Juan fue ejecutado dos meses y medio después, y durante todo este tiempo sus apóstoles permanecieron con Jesús y los doce. Todos trabajaron juntos y bautizaron a los creyentes durante este período de actividad en las ciudades de la Decápolis. El campamento de Gilboa se levantó el 2 de noviembre del año 27.

\section*{7. En las ciudades de la Decápolis}
\par 
%\textsuperscript{(1626.2)}
\textsuperscript{144:7.1} Durante los meses de noviembre y diciembre, Jesús y los veinticuatro trabajaron tranquilamente en las ciudades griegas de la Decápolis, principalmente en Escitópolis, Gerasa, Abila y Gadara. Éste fue realmente el final del período preliminar durante el cual se hicieron cargo del trabajo y de la organización de Juan. La religión de una nueva revelación, al socializarse, siempre paga el precio de un compromiso con las formas y costumbres establecidas de la religión precedente que trata de salvar. El bautismo fue el precio que pagaron los discípulos de Jesús para incluir entre ellos, como grupo religioso socializado, a los seguidores de Juan el Bautista. Los discípulos de Juan, al unirse con los de Jesús, renunciaron a casi todo, excepto al bautismo con agua.

\par 
%\textsuperscript{(1626.3)}
\textsuperscript{144:7.2} Jesús enseñó poco en público durante esta misión en las ciudades de la Decápolis. Pasó un tiempo importante enseñando a los veinticuatro y tuvo muchas sesiones especiales con los doce apóstoles de Juan. Con el tiempo llegaron a comprender mejor por qué Jesús no iba a visitar a Juan en la cárcel, y por qué no hacía ningún esfuerzo por conseguir su liberación. Pero nunca pudieron comprender por qué Jesús no realizaba obras milagrosas, por qué se negaba a manifestar los signos exteriores de su autoridad divina. Antes de venir al campamento de Gilboa, habían creído en Jesús principalmente a causa del testimonio de Juan, pero pronto empezaron a creer como resultado de su propio contacto con el Maestro y sus enseñanzas.

\par 
%\textsuperscript{(1626.4)}
\textsuperscript{144:7.3} Durante estos dos meses, el grupo trabajó la mayoría del tiempo en parejas; uno de los apóstoles de Jesús salía con uno de los de Juan. El apóstol de Juan bautizaba, el apóstol de Jesús instruía, y los dos predicaban el evangelio del reino tal como ellos lo comprendían. Y conquistaron muchas almas entre estos gentiles y judíos apóstatas.

\par 
%\textsuperscript{(1626.5)}
\textsuperscript{144:7.4} Abner, el jefe de los apóstoles de Juan, se convirtió en un fervoroso creyente en Jesús, y más tarde fue nombrado director de un grupo de setenta educadores, a quienes el Maestro encargó la predicación del evangelio.

\section*{8. En el campamento cerca de Pella}
\par 
%\textsuperscript{(1626.6)}
\textsuperscript{144:8.1} A finales de diciembre, todos se trasladaron cerca del Jordán, en las proximidades de Pella, donde reanudaron la enseñanza y la predicación. Tanto los judíos como los gentiles acudían a este campamento para escuchar el evangelio. Una tarde, mientras Jesús enseñaba a la multitud, unos amigos íntimos de Juan trajeron al Maestro el último mensaje que recibiría del Bautista.

\par 
%\textsuperscript{(1626.7)}
\textsuperscript{144:8.2} Juan llevaba ya un año y medio en la cárcel, y la mayor parte de este tiempo Jesús había trabajado de manera muy discreta; por eso no era de extrañar que Juan se sintiera inducido a preguntarse qué pasaba con el reino. Los amigos de Juan interrumpieron la enseñanza de Jesús para decirle: <<Juan el Bautista nos ha enviado para preguntarte: ¿Eres realmente el Libertador, o tenemos que esperar a otro?>>\footnote{\textit{¿Eres tú el Libertador?}: Mt 11:2-3; Lc 7:19-20.}

\par 
%\textsuperscript{(1626.8)}
\textsuperscript{144:8.3} Jesús hizo una pausa para decir a los amigos de Juan: <<Volved y haced saber a Juan que no ha sido olvidado. Contadle lo que habéis visto y oído, que la buena nueva se predica a los pobres>>\footnote{\textit{La buena nueva se predica a los pobres}: Is 61:1-2; Mt 11:4-5; Lc 4:18; 7:21-22.}. Después de hablar un poco más con los mensajeros de Juan, Jesús se volvió de nuevo hacia la multitud y dijo: <<No creáis que Juan duda del evangelio del reino. Sólo hace averiguaciones para tranquilizar a sus discípulos, que son también mis discípulos. Juan no es débil. A vosotros que habéis escuchado predicar a Juan antes de que Herodes lo encarcelara, dejadme que os pregunte: ¿Qué habéis visto en Juan ---a una caña sacudida por el viento? ¿A un hombre de humor cambiante, vestido con prendas suaves? Por regla general, los que están vestidos de manera suntuosa y viven exquisitamente están en las cortes de los reyes y en las mansiones de los ricos. Pero ¿qué habéis visto al contemplar a Juan? ¿A un profeta? Sí, os lo digo, y mucho más que un profeta. De Juan estaba escrito: `He aquí que envío a mi mensajero por delante de tu presencia; él preparará el camino delante de ti'.>>\footnote{\textit{Jesús habla sobre Juan Bautista}: Mt 11:7-10; Lc 7:24-27. \textit{Juan, el mensajero prometido}: Jn 1:6-7,15. \textit{Enviado a preparar el camino}: Mal 3:1.}

\par 
%\textsuperscript{(1627.1)}
\textsuperscript{144:8.4} <<En verdad, en verdad os digo que de aquellos que han nacido de mujer no ha surgido ninguno más grande que Juan el Bautista; sin embargo, incluso el más pequeño en el reino de los cielos es más grande que él, porque ha nacido del espíritu y sabe que se ha convertido en un hijo de Dios>>\footnote{\textit{Juan el más grande}: Mt 11:11; Lc 7:28.}.

\par 
%\textsuperscript{(1627.2)}
\textsuperscript{144:8.5} Muchos de los que escucharon\footnote{\textit{Muchos creyeron}: Lc 7:29.} a Jesús aquel día se sometieron al bautismo de Juan, manifestando así públicamente su entrada en el reino. Desde aquel día en adelante, los apóstoles de Juan permanecieron firmemente unidos a Jesús. Este suceso marcó la verdadera unión de los seguidores de Juan y de Jesús.

\par 
%\textsuperscript{(1627.3)}
\textsuperscript{144:8.6} Después de conversar con Abner, los mensajeros se marcharon hacia Macaerus para contar todo esto a Juan. Éste se sintió muy confortado, y su fe se fortaleció con las palabras de Jesús y el mensaje de Abner.

\par 
%\textsuperscript{(1627.4)}
\textsuperscript{144:8.7} Aquella tarde, Jesús continuó su enseñanza, diciendo: <<¿Con qué compararé a esta generación? Muchos de vosotros no recibiréis ni el mensaje de Juan ni mi enseñanza. Sois como los niños que juegan en la plaza del mercado, que llaman a sus compañeros para decirles: `Hemos tocado la flauta para vosotros y no habéis bailado; hemos gemido y no os habéis afligido'. Lo mismo sucede con algunos de vosotros. Juan ha venido, sin comer ni beber, y han dicho que tenía al demonio. El Hijo del Hombre viene, comiendo y bebiendo, y esas mismas personas dicen: `¡Observad, es un comilón y un bebedor de vino, un amigo de los publicanos y de los pecadores!' En verdad, la sabiduría es justificada por sus hijos>>\footnote{\textit{Comentarios de Jesús}: Mt 11:16-19; Lc 7:31-35.}.

\par 
%\textsuperscript{(1627.5)}
\textsuperscript{144:8.8} <<Parecería que el Padre que está en los cielos ha ocultado algunas de estas verdades a los sabios y a los arrogantes, mientras que las ha revelado a los niños. Pero el Padre hace bien todas las cosas; el Padre se revela al universo mediante los métodos de su propia elección. Venid pues, todos los que os afanáis y lleváis una carga pesada, y encontraréis descanso para vuestra alma. Haced vuestro el yugo divino, y experimentaréis la paz de Dios, que sobrepasa toda comprensión>>\footnote{\textit{Verdades ocultas reveladas a los niños}: Mt 11:25-26; Lc 10:21. \textit{Venid todos los que os afanáis}: Mt 11:28-29. \textit{La paz de Dios sobrepasa toda comprensión}: Flp 4:7.}.

\section*{9. La muerte de Juan el Bautista}
\par 
%\textsuperscript{(1627.6)}
\textsuperscript{144:9.1} Juan el Bautista fue ejecutado, por orden de Herodes Antipas, la noche del 10 de enero del año 28. Al día siguiente, algunos discípulos de Juan, que habían ido a Macaerus, oyeron hablar de su ejecución; se presentaron ante Herodes y solicitaron su cuerpo, que colocaron en un sepulcro\footnote{\textit{Juan Bautista ejecutado y enterrado}: Mt 14:9-12; Mc 6:27-29.}, y lo enterraron más tarde en Sebaste, la patria de Abner. Al día siguiente, 12 de enero, partieron hacia el norte en dirección al campamento de los apóstoles de Juan y de Jesús, cerca de Pella, y contaron a Jesús la muerte de Juan. Cuando Jesús escuchó su informe, despidió a la multitud, convocó a los veinticuatro y les dijo: <<Juan ha muerto. Herodes lo ha hecho decapitar. Esta noche, reuníos en consejo y arreglad vuestros asuntos convenientemente. Ya no habrá más dilaciones. Ha llegado la hora de proclamar el reino abiertamente y con poder. Mañana iremos a Galilea>>.

\par 
%\textsuperscript{(1627.7)}
\textsuperscript{144:9.2} En consecuencia, el 13 de enero del año 28 por la mañana temprano, Jesús y los apóstoles, acompañados por unos veinticinco discípulos, se dirigieron a Cafarnaúm y aquella noche se alojaron en la casa de Zebedeo.


\chapter{Documento 145. Cuatro días memorables en Cafarnaúm}
\par 
%\textsuperscript{(1628.1)}
\textsuperscript{145:0.1} JESÚS y los apóstoles llegaron a Cafarnaúm\footnote{\textit{Jesús y sus apóstoles en Cafarnaúm}: Mc 1:21a; Lc 4:31.} el martes 13 de enero al anochecer. Como de costumbre, establecieron su cuartel general en la casa de Zebedeo, en Betsaida. Ahora que Juan el Bautista había sido ejecutado, Jesús se preparó para lanzarse abiertamente a su primera gira de predicación pública en Galilea. La noticia del regreso de Jesús se difundió rápidamente por toda la ciudad, y a primeras horas del día siguiente, María, la madre de Jesús, salió apresuradamente hacia Nazaret para visitar a su hijo José.

\par 
%\textsuperscript{(1628.2)}
\textsuperscript{145:0.2} Jesús pasó el miércoles, el jueves y el viernes en la casa de Zebedeo instruyendo a sus apóstoles como preparación para su primera gran gira de predicación pública. También recibió y enseñó, tanto individualmente como en grupo, a muchos investigadores serios. Por medio de Andrés, arregló las cosas para hablar en la sinagoga el sábado siguiente.

\par 
%\textsuperscript{(1628.3)}
\textsuperscript{145:0.3} Al final de la tarde del viernes, Rut, la hermana menor de Jesús, le hizo una visita en secreto. Pasaron casi una hora juntos en una barca anclada a poca distancia de la costa. Ningún ser humano se enteró nunca de esta visita, salvo Juan Zebedeo, a quien se le recomendó que no se lo dijera a nadie. Rut era el único miembro de la familia de Jesús que creía, de manera firme y constante, en la divinidad de la misión terrestre de su hermano; y lo creyó desde su más temprana conciencia espiritual, pasando por todo el ministerio extraordinario de Jesús, su muerte, su resurrección y su ascensión. Finalmente, Rut pasó a los mundos del más allá sin haber dudado nunca del carácter sobrenatural de la misión en la carne de su hermano-padre. En lo que respecta a su familia terrestre, la pequeña Rut fue el principal consuelo de Jesús durante las penosas pruebas de su juicio, su rechazo y su crucifixión.

\section*{l. La redada de peces}
\par 
%\textsuperscript{(1628.4)}
\textsuperscript{145:1.1} El viernes por la mañana de esta misma semana, cuando Jesús estaba enseñando al lado de la playa, la gente se apiñó junto a él tan cerca del borde del agua, que hizo señas a unos pescadores que estaban en una barca cercana para que vinieran a rescatarlo. Subió a la barca y continuó enseñando durante más de dos horas a la multitud reunida\footnote{\textit{Jesús habla desde una barca}: Lc 5:1-3.}. Esta barca tenía el nombre de <<Simón>>; era la antigua embarcación de pesca de Simón Pedro y había sido construida por las mismas manos de Jesús. Aquella precisa mañana, la barca estaba siendo utilizada por David Zebedeo y dos socios\footnote{\textit{Simón y David}: Mt 4:18; Mc 1:16.}, que acababan de volver a la costa después de una noche de pesca infructuosa en el lago. Estaban limpiando y reparando sus redes cuando Jesús les pidió que vinieran en su ayuda.

\par 
%\textsuperscript{(1628.5)}
\textsuperscript{145:1.2} Después de que Jesús hubo terminado de enseñar a la gente, dijo a David: <<Como os habéis retrasado por venir a ayudarme, permitidme ahora trabajar con vosotros. Vamos a pescar. Dirigíos hacia esa parte profunda y dejad caer vuestras redes para hacer una captura>>. Pero Simón, uno de los ayudantes de David, respondió: <<Maestro, es inútil. Hemos faenado toda la noche y no hemos cogido nada; sin embargo, puesto que tú lo ordenas, vamos a salir y arrojaremos las redes>>. Simón consintió en seguir las instrucciones de Jesús porque David, su patrón, se lo indicó con un gesto. Cuando llegaron al lugar señalado por Jesús, lanzaron sus redes y reunieron tal cantidad de peces que tuvieron miedo de que se rompieran las redes; tanto fue así que hicieron señas a sus asociados de la costa para que vinieran a ayudarlos. Cuando llenaron totalmente las tres barcas de peces, casi hasta el punto de hundirse, el tal Simón se postró a los pies de Jesús, diciendo: <<Apártate de mí, Maestro, porque soy un pecador>>. Simón y todos los implicados en este episodio se quedaron atónitos con esta redada de peces\footnote{\textit{La redada de peces}: Lc 5:4-11.}. A partir de aquel día, David Zebedeo, este Simón, y sus asociados abandonaron sus redes y siguieron a Jesús\footnote{\textit{Pescadores de hombres}: Mt 4:19-20; Mc 1:17-18.}.

\par 
%\textsuperscript{(1629.1)}
\textsuperscript{145:1.3} Pero ésta no fue en ningún sentido una pesca milagrosa. Jesús era un atento observador de la naturaleza; era un pescador experto y conocía las costumbres de los peces en el Mar de Galilea. En esta ocasión, se limitó a dirigir a estos hombres hacia el lugar donde los peces se encontraban a aquella hora del día. Pero los seguidores de Jesús siempre consideraron este suceso como un milagro.

\section*{2. La tarde en la sinagoga}
\par 
%\textsuperscript{(1629.2)}
\textsuperscript{145:2.1} El sábado siguiente, en los oficios de la tarde en la sinagoga\footnote{\textit{Jesús predica en la sinagoga}: Mc 1:21; Lc 4:31.}, Jesús predicó su sermón sobre <<La voluntad del Padre que está en los cielos>>. Por la mañana, Simón Pedro había predicado sobre <<El reino>>. En la reunión del jueves por la noche en la sinagoga, Andrés había enseñado sobre el tema <<El nuevo camino>>. En aquel momento concreto, la gente que creía en Jesús era más numerosa en Cafarnaúm que en cualquier otra ciudad de la Tierra.

\par 
%\textsuperscript{(1629.3)}
\textsuperscript{145:2.2} Cuando Jesús enseñó en la sinagoga aquel sábado por la tarde, siguiendo la costumbre cogió su primer texto en la ley y leyó en el Libro del Éxodo: <<Servirás al Señor tu Dios, y él bendecirá tu pan y tu agua, y toda enfermedad será apartada de ti>>\footnote{\textit{Servirás al Señor, tu Dios}: Ex 23:25.}. El segundo texto lo escogió en los Profetas, leyendo en Isaías: <<Levántate y resplandece, porque ha venido tu luz, y la gloria del Señor se ha levantado sobre ti. La oscuridad puede cubrir la Tierra y las profundas tinieblas envolver a la gente, pero el espíritu del Señor se levantará sobre ti y verán que la gloria divina te acompaña. Incluso los gentiles vendrán hacia esta luz, y muchos grandes pensadores se rendirán ante su resplandor>>\footnote{\textit{Levántate, resplandece, ha venido la luz}: Is 60:1-3.}.

\par 
%\textsuperscript{(1629.4)}
\textsuperscript{145:2.3} Este sermón fue un esfuerzo por parte de Jesús para exponer claramente el hecho de que la religión es una \textit{experiencia personal}. Entre otras cosas, el Maestro dijo:

\par 
%\textsuperscript{(1629.5)}
\textsuperscript{145:2.4} <<Sabéis bien que, aunque un padre de buen corazón ama a su familia como un todo, los considera así como grupo a causa de su sólido afecto por cada miembro individual de esa familia. Hay que dejar de acercarse al Padre que está en los cielos como un hijo de Israel, y hacerlo como un \textit{hijo de Dios}. Como grupo, sois en efecto los hijos de Israel, pero como individuos, cada uno de vosotros es un hijo de Dios. He venido, no para revelar el Padre a los hijos de Israel, sino más bien para traer al creyente individual este conocimiento de Dios y la revelación de su amor y de su misericordia como una experiencia personal auténtica. Todos los profetas os han enseñado que Yahvé cuida a su pueblo, que Dios ama a Israel. Pero yo he venido en medio de vosotros para proclamar una verdad más grande, una verdad que muchos de los últimos profetas también captaron, la verdad de que Dios \textit{os} ama ---a cada uno de vosotros--- como individuos. Durante todas estas generaciones, habéis tenido una religión nacional o racial; yo he venido ahora para daros una religión personal>>.

\par 
%\textsuperscript{(1630.1)}
\textsuperscript{145:2.5} <<Pero incluso esto no es una idea nueva. Muchos de los que tenéis inclinaciones espirituales habéis conocido esta verdad, puesto que algunos profetas así os lo han enseñado. ¿No habéis leído en las Escrituras lo que dice el profeta Jeremías?: `En aquellos días ya no volverán a decir: los padres han comido uvas verdes y son los hijos los que tienen la dentera. Cada cual morirá por su propia iniquidad; todo hombre que coma uvas verdes tendrá dentera. Mirad, se acercan los días en que haré un nuevo pacto con mi pueblo, no según el pacto que hice con sus padres cuando los saqué de la tierra de Egipto, sino según el nuevo camino. Incluso escribiré mi ley en sus corazones. Yo seré su Dios, y ellos serán mi pueblo. Cuando llegue ese día, los hombres ya no dirán a sus vecinos: ¿conoces al Señor? ¡No! Porque todos me conocerán personalmente, desde el más humilde hasta el más grande'.>>\footnote{\textit{Uvas verdes, hijos con dentera}: Jer 31:29-34.}

\par 
%\textsuperscript{(1630.2)}
\textsuperscript{145:2.6} <<¿No habéis leído estas promesas? ¿No creéis en las Escrituras? ¿No comprendéis que las palabras del profeta se están cumpliendo en lo que contempláis hoy mismo? ¿No os ha exhortado Jeremías a que hagáis de la religión un asunto del corazón, a que os relacionéis con Dios como individuos? ¿No os ha dicho el profeta que el Dios de los cielos sondearía vuestros corazones individuales? ¿Y no se os ha advertido que el corazón humano es, por naturaleza, más engañoso que nada, y con mucha frecuencia desesperadamente perverso?>>\footnote{\textit{La religión un asunto del corazón}: Jer 24:7. \textit{Sondeará vuestros corazones individuales}: Jer 17:9-10.}

\par 
%\textsuperscript{(1630.3)}
\textsuperscript{145:2.7} <<¿No habéis leído también el pasaje donde Ezequiel enseñó a vuestros padres que la religión debe convertirse en una realidad en vuestra experiencia individual? Ya no utilizaréis el proverbio que dice: `Los padres han comido uvas verdes y son los hijos los que tienen la dentera'. `Tan cierto como que estoy vivo', dice el Señor Dios, `he aquí que todas las almas me pertenecen; tanto el alma del padre como el alma del hijo. Sólo el alma que peque morirá'. Y luego, Ezequiel predijo incluso el día de hoy cuando habló en nombre de Dios, diciendo: `Os daré también un nuevo corazón, y pondré dentro de vosotros un espíritu nuevo'.>>\footnote{\textit{Sólo el alma pecadora morirá}: Jer 31:29-30; Ez 18:2-4. \textit{Dios dará un nuevo corazón, un nuevo espíritu}: Ez 36:26.}

\par 
%\textsuperscript{(1630.4)}
\textsuperscript{145:2.8} <<Debéis dejar de temer que Dios castiga a una nación por el pecado de un individuo. El Padre que está en los cielos tampoco castigará a uno de sus hijos creyentes por los pecados de una nación, aunque un miembro individual de una familia pueda sufrir a menudo las consecuencias materiales de los errores familiares y de las transgresiones colectivas. ¿No os dais cuenta de que la esperanza de tener una nación mejor ---o un mundo mejor--- está ligada al progreso y a la iluminación del individuo?>>

\par 
%\textsuperscript{(1630.5)}
\textsuperscript{145:2.9} Luego el Maestro describió que, una vez que los hombres disciernen esta libertad espiritual, el Padre que está en los cielos quiere que sus hijos de la Tierra empiecen la ascensión eterna de la carrera hacia el Paraíso, que consiste en una respuesta consciente de la criatura al impulso divino del espíritu interior por encontrar al Creador, conocer a Dios y tratar de volverse semejante a él.

\par 
%\textsuperscript{(1630.6)}
\textsuperscript{145:2.10} Este sermón fue de una gran ayuda para los apóstoles. Todos comprendieron mucho mejor que el evangelio del reino es un mensaje destinado al individuo, no a la nación.

\par 
%\textsuperscript{(1630.7)}
\textsuperscript{145:2.11} Aunque los habitantes de Cafarnaúm estaban familiarizados con las enseñanzas de Jesús, se quedaron asombrados con su sermón de este sábado. Enseñó, en verdad, como alguien que tiene autoridad, y no como los escribas\footnote{\textit{Jesús enseñó con autoridad}: Mt 7:28-29; Mc 1:21-22; Lc 4:31-32.}.

\par 
%\textsuperscript{(1630.8)}
\textsuperscript{145:2.12} En el preciso momento en que Jesús terminaba de hablar, un joven de la asamblea que se había perturbado mucho con sus palabras cayó víctima de un violento ataque epiléptico, acompañado de grandes gritos. Al final de la crisis, cuando estaba recobrando la conciencia, habló en un estado de ensueño, diciendo: <<¿Qué vamos a hacer contigo, Jesús de Nazaret? Eres el santo de Dios; ¿has venido para destruirnos?>> Jesús pidió a la gente que permaneciera tranquila, cogió al joven por la mano, y le dijo: <<Sal de ese estado>>; y se despertó inmediatamente\footnote{\textit{Curación del epiléptico}: Mc 1:23-26; Lc 4:33-35.}.

\par 
%\textsuperscript{(1631.1)}
\textsuperscript{145:2.13} Este joven no estaba poseído por un espíritu impuro o un demonio; era víctima de una epilepsia corriente. Pero le habían enseñado que su afección se debía a que estaba poseído por un espíritu maligno. Creía en lo que le habían dicho y se comportaba de acuerdo con ello en todo lo que pensaba o decía sobre su enfermedad. Toda la gente creía que estos fenómenos estaban causados directamente por la presencia de los espíritus impuros. En consecuencia, creyeron que Jesús había echado un demonio de este hombre. Pero Jesús no lo curó de su epilepsia en aquel momento. Este joven no se curó realmente hasta más tarde, aquel mismo día, después de la puesta del Sol. Mucho después del día de Pentecostés, el apóstol Juan, que fue el último que escribió sobre las actividades de Jesús, evitó toda referencia a estas pretendidas <<expulsiones de demonios>>, y lo hizo así debido al hecho de que estos casos de posesión demoníaca no volvieron a producirse después de Pentecostés.

\par 
%\textsuperscript{(1631.2)}
\textsuperscript{145:2.14} Como resultado de este vulgar incidente, por todo Cafarnaúm se divulgó rápidamente la noticia de que Jesús había echado un demonio de un hombre\footnote{\textit{Noticias de la curación}: Mc 1:27-28; Lc 4:36-37.}, y que lo había curado milagrosamente en la sinagoga al final de su sermón de la tarde. El sábado era el momento propicio para que este rumor sorprendente se propagara de manera rápida y eficaz. Esta noticia llegó también a todas las poblaciones más pequeñas que rodeaban a Cafarnaúm, y mucha gente se la creyó.

\par 
%\textsuperscript{(1631.3)}
\textsuperscript{145:2.15} La esposa y la suegra de Simón Pedro hacían la mayor parte de la cocina y del trabajo doméstico en la gran casa de Zebedeo, donde Jesús y los doce habían establecido su cuartel general. La casa de Pedro estaba cerca de la de Zebedeo. Jesús y sus amigos se detuvieron allí al regresar de la sinagoga porque la madre de la esposa de Pedro llevaba varios días enferma con fiebre y escalofríos. Sucedió por casualidad que la fiebre se le quitó mientras Jesús estaba de pie al lado de la enferma, sosteniendo su mano, acariciándole la frente y diciéndole palabras de consuelo y de aliento\footnote{\textit{La suegra de Pedro, enferma}: Mt 8:14-15; Mc 8:30-31; Lc 4:38-39.}. Jesús aún no había tenido tiempo de explicar a sus apóstoles que no se había producido ningún milagro en la sinagoga; con este incidente tan reciente y vívido en su memoria, y al recordar el agua y el vino de Caná, tomaron esta coincidencia como otro milagro, y algunos de ellos salieron precipitadamente para difundir la noticia por toda la ciudad.

\par 
%\textsuperscript{(1631.4)}
\textsuperscript{145:2.16} Amata, la suegra de Pedro, padecía de paludismo. En aquel momento no fue curada milagrosamente por Jesús. Su curación no se realizó hasta varias horas más tarde, después de la puesta del Sol, en conexión con el extraordinario acontecimiento que se produjo en el patio delantero de la casa de Zebedeo.

\par 
%\textsuperscript{(1631.5)}
\textsuperscript{145:2.17} Estos casos son típicos de la manera en que una generación en busca de prodigios, y un pueblo propenso a ver milagros, se aferraban indefectiblemente a todas estas coincidencias como pretexto para proclamar que Jesús había efectuado otro milagro.

\section*{3. La curación a la puesta del Sol}
\par 
%\textsuperscript{(1631.6)}
\textsuperscript{145:3.1} En el momento en que Jesús y sus apóstoles se disponían a compartir su cena, casi al final de este sábado memorable, todo Cafarnaúm y sus alrededores estaban alborotados a causa de estas pretendidas curaciones milagrosas; y todos los que estaban enfermos o afligidos empezaron a prepararse para ir a ver a Jesús, o para que sus amigos los transportaran hasta allí, en cuanto se pusiera el Sol. Según las enseñanzas judías, ni siquiera estaba permitido buscar la salud durante las horas sagradas del sábado.

\par 
%\textsuperscript{(1632.1)}
\textsuperscript{145:3.2} Así pues, tan pronto como el Sol desapareció por el horizonte, decenas de hombres, mujeres y niños afligidos empezaron a dirigirse hacia la casa de Zebedeo en Betsaida. Un hombre salió con su hija paralítica en cuanto el Sol se ocultó por detrás de la casa de su vecino\footnote{\textit{La multitud a la puesta de sol}: Mt 8:16; Mc 1:32-33; Lc 4:40.}.

\par 
%\textsuperscript{(1632.2)}
\textsuperscript{145:3.3} Los acontecimientos de todo este día habían preparado el escenario para este espectáculo extraordinario a la puesta del Sol. Incluso el texto que Jesús había utilizado en su sermón de la tarde daba a entender que la enfermedad debía ser desterrada; ¡y había hablado con un poder y una autoridad sin precedentes! ¡Su mensaje era tan apremiante! Aunque no había apelado a la autoridad humana, había hablado directamente a la conciencia y al alma de los hombres. Aún cuando no había recurrido a la lógica, a las argucias legales o a las aserciones ingeniosas, había efectuado un poderoso llamamiento directo, claro y personal al corazón de sus oyentes.

\par 
%\textsuperscript{(1632.3)}
\textsuperscript{145:3.4} Este sábado fue un gran día en la vida terrestre de Jesús, y en la vida de un universo. Para todo el universo local, la pequeña ciudad judía de Cafarnaúm fue, en todos los sentidos, la verdadera capital de Nebadon. El puñado de judíos de la sinagoga de Cafarnaúm no eran los únicos seres que escucharon esta importante declaración con la que Jesús concluyó su sermón: <<El odio es la sombra del miedo, y la venganza, la máscara de la cobardía>>. Sus oyentes tampoco podrían olvidar sus palabras benditas, cuando declaró: <<El hombre es el hijo de Dios, y no un hijo del diablo>>.

\par 
%\textsuperscript{(1632.4)}
\textsuperscript{145:3.5} Poco después de la puesta del Sol, mientras Jesús y los apóstoles permanecían todavía alrededor de la mesa de la cena, la esposa de Pedro escuchó voces en el patio delantero y, al acercarse a la puerta, vio que se estaba congregando un gran número de enfermos, y que el camino de Cafarnaúm estaba atestado de gente que venía a buscar la curación de manos de Jesús. Al contemplar este espectáculo, fue inmediatamente a informar a su marido, el cual se lo dijo a Jesús.

\par 
%\textsuperscript{(1632.5)}
\textsuperscript{145:3.6} Cuando el Maestro salió a la entrada principal de la casa de Zebedeo, sus ojos se encontraron con una masa humana aquejada y afligida. Contempló a casi mil seres humanos enfermos y doloridos; al menos éste era el número de personas reunidas delante de él. Pero no todos los presentes estaban afligidos; algunos habían venido para ayudar a sus seres queridos en este esfuerzo por conseguir la curación.

\par 
%\textsuperscript{(1632.6)}
\textsuperscript{145:3.7} El espectáculo de estos mortales afligidos, hombres, mujeres y niños, que sufrían en gran parte a consecuencia de las equivocaciones y transgresiones de sus propios Hijos a quienes había confiado la administración del universo, conmovió particularmente el corazón humano de Jesús y puso a prueba la misericordia divina de este benévolo Hijo Creador. Pero Jesús sabía bien que nunca podría construir un movimiento espiritual duradero sobre la base de unos prodigios puramente materiales. Había seguido la conducta permanente de abstenerse de exhibir sus prerrogativas de creador. Lo sobrenatural o lo milagroso no habían acompañado a su enseñanza desde el episodio de Caná; sin embargo, esta multitud afligida conmovió su corazón compasivo y apeló poderosamente a su afecto comprensivo.

\par 
%\textsuperscript{(1632.7)}
\textsuperscript{145:3.8} Una voz procedente del patio delantero exclamó: <<Maestro, pronuncia la palabra, devuélvenos la salud, cura nuestras enfermedades y salva nuestras almas>>. Apenas se habían pronunciado estas palabras cuando una inmensa comitiva de serafines, controladores físicos, Portadores de Vida e intermedios, que siempre acompañaban a este Creador encarnado de un universo, se prepararon para actuar con poder creativo si su Soberano daba la señal. Éste fue uno de esos momentos, en la carrera terrestre de Jesús, en los que la sabiduría divina y la compasión humana estaban tan entrelazadas en el juicio del Hijo del Hombre, que buscó refugio recurriendo a la voluntad de su Padre.

\par 
%\textsuperscript{(1632.8)}
\textsuperscript{145:3.9} Cuando Pedro imploró al Maestro que atendiera aquellas peticiones de ayuda, Jesús paseó su mirada sobre la muchedumbre de afligidos, y contestó: <<He venido al mundo para revelar al Padre y establecer su reino. He vivido mi vida hasta este momento con esa finalidad. Por lo tanto, si fuera la voluntad de Aquel que me ha enviado, y si no es incompatible con mi dedicación a proclamar el evangelio del reino de los cielos, desearía que mis hijos se curaran... y..>>. pero las demás palabras de Jesús se perdieron en el alboroto.

\par 
%\textsuperscript{(1633.1)}
\textsuperscript{145:3.10} Jesús había transferido la responsabilidad de esta decisión curativa a la autoridad de su Padre. Es evidente que la voluntad del Padre no interpuso ninguna objeción, pues apenas había pronunciado el Maestro estas palabras, el conjunto de personalidades celestiales que servían bajo las órdenes del Ajustador del Pensamiento Personalizado de Jesús se puso poderosamente en movimiento. La enorme comitiva descendió en medio de aquella multitud abigarrada de mortales afligidos, y en unos instantes, 683 hombres, mujeres y niños recuperaron la salud, fueron perfectamente curados de todas sus enfermedades físicas y de otros desórdenes materiales\footnote{\textit{La curación a la puesta de sol}: Mt 8:6b; Mc 1:34; Lc 4:40-41.}. Una escena semejante no se había visto nunca en la Tierra antes de aquel día, ni tampoco después. Para aquellos de nosotros que estaban presentes y contemplaron esta oleada creativa de curaciones, fue en verdad un espectáculo conmovedor.

\par 
%\textsuperscript{(1633.2)}
\textsuperscript{145:3.11} Pero de todos los seres que se quedaron asombrados con esta explosión repentina e inesperada de curación sobrenatural, Jesús era el más sorprendido. En el instante en que su interés y su compasión humanos estaban centrados en la escena de sufrimiento y aflicción desplegada allí ante sus ojos, olvidó tener en cuenta en su mente humana las advertencias exhortatorias de su Ajustador Personalizado; éste le había advertido que, bajo ciertas condiciones y en ciertas circunstancias, era imposible limitar el elemento tiempo en las prerrogativas creadoras de un Hijo Creador. Jesús deseaba que estos mortales que sufrían se curaran, si no se infringía con ello la voluntad de su Padre. El Ajustador Personalizado de Jesús decidió instantáneamente que un acto así de energía creativa no transgrediría en aquel momento la voluntad del Padre Paradisiaco; con esta decisión ---y teniendo en cuenta que Jesús había expresado previamente el deseo curativo--- el acto creativo \textit{existió}. Aquello que un \textit{Hijo Creador} desea y su Padre lo \textit{quiere}, EXISTE. Una curación física y masiva de mortales como ésta no volvió a producirse en toda la vida posterior de Jesús en la Tierra.

\par 
%\textsuperscript{(1633.3)}
\textsuperscript{145:3.12} Como era de esperar, la noticia de esta curación a la puesta del Sol, en Betsaida de Cafarnaúm, se difundió por toda Galilea y Judea, y por regiones más lejanas. Los temores de Herodes se despertaron una vez más; envió a unos observadores para que le informaran sobre la obra y las enseñanzas de Jesús, y para que averiguaran si se trataba del antiguo carpintero de Nazaret o de Juan el Bautista resucitado de entre los muertos.

\par 
%\textsuperscript{(1633.4)}
\textsuperscript{145:3.13} Durante el resto de su carrera terrestre, y a causa principalmente de esta demostración involuntaria de curación física, Jesús se convirtió en lo sucesivo tanto en médico como en predicador. Es cierto que continuó enseñando, pero su trabajo personal consistía sobre todo en ayudar a los enfermos y a los afligidos, mientras que sus apóstoles se ocupaban de predicar en público y de bautizar a los creyentes.

\par 
%\textsuperscript{(1633.5)}
\textsuperscript{145:3.14} Pero la mayoría de los que recibieron la curación física sobrenatural, o creativa, durante esta demostración de energía divina después de ponerse el Sol, no obtuvieron un beneficio espiritual permanente de esta extraordinaria manifestación de misericordia. Unos pocos fueron edificados realmente gracias a este ministerio físico, pero esta asombrosa erupción de curación creativa, independiente del tiempo, no hizo avanzar el reino espiritual en el corazón de los hombres.

\par 
%\textsuperscript{(1633.6)}
\textsuperscript{145:3.15} Las curaciones milagrosas que acompañaron de vez en cuando la misión de Jesús en la Tierra no formaban parte de su plan para proclamar el reino. Fueron accidentalmente inherentes a la presencia en la Tierra de un ser divino con unas prerrogativas creadoras casi ilimitadas, en asociación con una combinación sin precedentes de misericordia divina y de compasión humana. Pero estos pretendidos milagros dieron muchos problemas a Jesús, en el sentido de que le proporcionaron una publicidad que ocasionaba prejuicios y le aportaron una notoriedad que no deseaba.

\section*{4. La noche siguiente}
\par 
%\textsuperscript{(1634.1)}
\textsuperscript{145:4.1} Durante toda la noche que siguió a esta gran explosión de curaciones, la multitud alegre y feliz invadió la casa de Zebedeo, y el entusiasmo emotivo de los apóstoles de Jesús alcanzó los niveles más altos. Desde el punto de vista humano, éste fue probablemente el día más grande de todos los días inolvidables de su asociación con Jesús. En ningún momento anterior ni posterior se elevaron sus esperanzas hasta tales alturas de expectativa confiada. Sólo unos días antes, cuando aún se encontraban en el interior de las fronteras de Samaria, Jesús les había dicho que había llegado la hora en que el reino debía ser proclamado con \textit{poderío}, y ahora sus ojos habían contemplado lo que suponían que era la realización de esta promesa. Estaban emocionados con la idea de lo que vendría después si esta asombrosa manifestación de poder curativo no era más que el principio. Habían desterrado sus dudas prolongadas sobre la divinidad de Jesús. Estaban literalmente embriagados con el éxtasis de su aturdido encantamiento.

\par 
%\textsuperscript{(1634.2)}
\textsuperscript{145:4.2} Pero cuando buscaron a Jesús, no pudieron encontrarlo. El Maestro estaba muy perturbado por lo que había sucedido. Estos hombres, mujeres y niños que habían sido curados de diversas enfermedades se quedaron hasta horas avanzadas de la noche, esperando que Jesús regresara para poder expresarle su gratitud. A medida que pasaban las horas y el Maestro permanecía recluido, los apóstoles no podían comprender su conducta; su alegría hubiera sido completa y perfecta si no hubiera sido por esta ausencia continuada. Cuando Jesús regresó entre ellos, ya era tarde, y prácticamente todos los beneficiarios del episodio curativo se habían ido a sus casas. Jesús rehusó las felicitaciones y la adoración de los doce y de los demás que se habían quedado para saludarlo, limitándose a decir: <<No os regocijéis porque mi Padre tenga el poder de curar el cuerpo, sino más bien porque tiene la fuerza de salvar el alma. Vamos a descansar, pues mañana tenemos que ocuparnos de los asuntos del Padre>>.

\par 
%\textsuperscript{(1634.3)}
\textsuperscript{145:4.3} Una vez más, doce hombres decepcionados, perplejos y con el corazón entristecido se fueron a descansar; pocos de ellos, exceptuando a los gemelos, durmieron mucho aquella noche. Tan pronto como el Maestro hacía algo que alegraba el alma y regocijaba el corazón de sus apóstoles, parecía que inmediatamente hacía añicos sus esperanzas y demolía completamente los fundamentos de su coraje y entusiasmo. Cuando estos pescadores desconcertados se miraban entre sí a los ojos, sólo tenían un pensamiento: <<No podemos comprenderlo. ¿Qué significa todo esto?>>

\section*{5. El domingo por la mañana temprano}
\par 
%\textsuperscript{(1634.4)}
\textsuperscript{145:5.1} Jesús tampoco durmió mucho aquel sábado por la noche. Se dio cuenta de que el mundo estaba lleno de sufrimiento físico y repleto de dificultades materiales. Meditaba sobre el grave peligro de verse obligado a consagrar tal cantidad de su tiempo al cuidado de los enfermos y afligidos, que su misión de establecer el reino espiritual en el corazón de los hombres se vería obstaculizada por el ministerio de las cosas físicas, o al menos subordinada a dicho ministerio. Debido a que estos pensamientos y otros similares ocuparon la mente mortal de Jesús durante la noche, aquel domingo por la mañana se levantó mucho antes del amanecer y se fue solo\footnote{\textit{El retiro de Jesús}: Mc 1:35; Lc 4:42a.} a uno de sus lugares favoritos para comulgar con el Padre. En esta mañana temprano, Jesús escogió como tema de oración la sabiduría y el juicio para impedir que su compasión humana, unida a su misericordia divina, se sintieran tan influidas en presencia del sufrimiento humano, que todo su tiempo estuviera ocupado con el ministerio físico, descuidando el ministerio espiritual. Aunque no deseaba evitar por completo ayudar a los enfermos, sabía que también tenía que hacer un trabajo más importante, el de la enseñanza espiritual y la educación religiosa.

\par 
%\textsuperscript{(1635.1)}
\textsuperscript{145:5.2} Jesús salía tan a menudo a orar en las colinas porque no había habitaciones privadas donde poder llevar a cabo sus devociones personales.

\par 
%\textsuperscript{(1635.2)}
\textsuperscript{145:5.3} Pedro no pudo dormir aquella noche; por eso, poco después de que Jesús se hubiera ido a orar, despertó muy temprano a Santiago y a Juan, y los tres salieron para buscar a su Maestro\footnote{\textit{Buscando a Jesús}: Mc 1:36-37a.}. Después de buscarlo durante más de una hora, encontraron a Jesús y le suplicaron que les contara la razón de su extraña conducta. Deseaban saber por qué parecía estar disgustado por la poderosa efusión del espíritu de curación, cuando toda la gente estaba encantada y sus apóstoles tan llenos de alegría.

\par 
%\textsuperscript{(1635.3)}
\textsuperscript{145:5.4} Durante más de cuatro horas, Jesús se esforzó por explicar a estos tres apóstoles lo que había sucedido. Les enseñó lo que había acontecido y les explicó los peligros de este tipo de manifestaciones. Jesús les confió el motivo por el que había salido a orar. Intentó indicar claramente a sus asociados personales las verdaderas razones por las cuales el reino del Padre no se podía construir sobre la realización de prodigios y las curaciones físicas. Pero no podían comprender su enseñanza.

\par 
%\textsuperscript{(1635.4)}
\textsuperscript{145:5.5} Mientras tanto, el domingo por la mañana temprano, otra multitud de almas afligidas y muchos curiosos empezaron a congregarse alrededor de la casa de Zebedeo. Gritaban que querían ver a Jesús. Andrés y los apóstoles estaban tan perplejos que, mientras Simón Celotes hablaba a la asamblea, Andrés salió a buscar a Jesús con algunos de sus compañeros. Cuando hubo localizado a Jesús en compañía de los tres, Andrés dijo: <<Maestro, ¿por qué nos dejas solos con la multitud? Mira, todo el mundo te busca; nunca han buscado antes tantas personas tu enseñanza. En este mismo momento, la casa está rodeada de gente que ha venido de cerca y de lejos a causa de tus obras poderosas. ¿No vas a volver con nosotros para aportarles tu ministerio?>>\footnote{\textit{La multitud busca a Jesús para curarse}: Mc 1:37b; Lc 4:42b.}

\par 
%\textsuperscript{(1635.5)}
\textsuperscript{145:5.6} Cuando Jesús escuchó esto, contestó: <<Andrés, ¿no te he enseñado a ti y a los demás que mi misión en la Tierra es revelar al Padre, y que mi mensaje es proclamar el reino de los cielos? ¿Entonces cómo puede ser que quieras que me desvíe de mi trabajo para contentar a los curiosos y satisfacer a los que buscan signos y prodigios? ¿No hemos estado entre esa gente todos estos meses? ¿Y se han congregado en multitudes para escuchar la buena nueva del reino? ¿Por qué vienen ahora a acosarnos? ¿No es para buscar la curación de su cuerpo físico, en vez de venir porque han recibido la verdad espiritual para la salvación de su alma? Cuando los hombres se sienten atraídos hacia nosotros a causa de las manifestaciones extraordinarias, muchos no vienen buscando la verdad y la salvación sino más bien la curación de sus dolencias físicas, y para conseguir la liberación de sus dificultades materiales>>.

\par 
%\textsuperscript{(1635.6)}
\textsuperscript{145:5.7} <<Todo este tiempo he estado en Cafarnaúm, y tanto en la sinagoga como al lado del mar, he proclamado la buena nueva del reino a todos los que tenían oídos para oír y un corazón para recibir la verdad. No es voluntad de mi Padre que vuelva con vosotros para entretener a esos curiosos y dedicarme al ministerio de las cosas materiales, con exclusión de las espirituales. Os he ordenado para que prediquéis el evangelio y ayudéis a los enfermos, pero no debo dejarme absorber por las curaciones, dejando de lado mi enseñanza. No, Andrés, no voy a volver con vosotros. Id y decidle a la gente que crean en lo que les hemos enseñado, y que se regocijen en la libertad de los hijos de Dios. Y preparaos para nuestra partida hacia las otras ciudades de Galilea, donde el camino ya ha sido preparado para la predicación de la buena nueva del reino. Ésta es la finalidad para la que he venido desde donde se encuentra el Padre. Así pues, id y preparad nuestra partida inmediata, mientras espero aquí vuestro regreso>>\footnote{\textit{Jesús viene a predicar, no a curar}: Mc 1:38; Lc 4:43.}.

\par 
%\textsuperscript{(1636.1)}
\textsuperscript{145:5.8} Cuando Jesús hubo hablado, Andrés y sus compañeros apóstoles emprendieron tristemente el camino de vuelta a la casa de Zebedeo, despidieron a la multitud reunida y se prepararon rápidamente para el viaje, como Jesús les había ordenado. Así pues, el domingo por la tarde 18 de enero del año 28, Jesús y los apóstoles empezaron su primera gira de predicación realmente pública y manifiesta en las ciudades de Galilea. Durante este primer periplo, predicaron el evangelio del reino en muchas ciudades, pero no visitaron Nazaret\footnote{\textit{Comienzo de la primera gira de predicación}: Mt 4:23; 9:35b; Mc 1:39; Lc 4:44.}.

\par 
%\textsuperscript{(1636.2)}
\textsuperscript{145:5.9} Aquel domingo por la tarde, poco después de que Jesús y sus apóstoles hubieran salido para Rimón, sus hermanos Santiago y Judá se presentaron en la casa de Zebedeo para verlo. Hacia el mediodía de aquel día, Judá había buscado por todas partes a su hermano Santiago y le había insistido para que fueran a ver a Jesús. Pero cuando Santiago consintió por fin en acompañar a Judá, Jesús ya se había marchado.

\par 
%\textsuperscript{(1636.3)}
\textsuperscript{145:5.10} Los apóstoles eran reacios a abandonar el gran interés que se había despertado en Cafarnaúm. Pedro calculó que no menos de mil creyentes podían haber sido bautizados en el reino. Jesús los escuchó con paciencia, pero no consintió en volver. Durante un rato prevaleció el silencio, y luego Tomás se dirigió a sus compañeros apóstoles diciendo: <<¡Vamos! El Maestro ha hablado. No importa que no podamos comprender plenamente los misterios del reino de los cielos, pues de una cosa estamos seguros: Seguimos a un instructor que no busca ninguna gloria para sí mismo>>. Y, a regañadientes, salieron a predicar la buena nueva en las ciudades de Galilea.


\chapter{Documento 146. La primera gira de predicación en Galilea}
\par 
%\textsuperscript{(1637.1)}
\textsuperscript{146:0.1} LA PRIMERA gira de predicación pública en Galilea empezó el domingo 18 de enero del año 28 y continuó durante unos dos meses, finalizando con el regreso a Cafarnaúm el 17 de marzo. A lo largo de esta gira, Jesús y los doce apóstoles, con la ayuda de los antiguos apóstoles de Juan, predicaron el evangelio y bautizaron a los creyentes en Rimón, Jotapata, Ramá, Zabulón, Irón, Giscala, Corazín, Madón, Caná, Naín y Endor. En estas ciudades se detuvieron para enseñar, mientras que en otras muchas ciudades más pequeñas proclamaron el evangelio del reino a medida que pasaban por ellas.

\par 
%\textsuperscript{(1637.2)}
\textsuperscript{146:0.2} Ésta fue la primera vez que Jesús permitió a sus asociados predicar sin restricciones. En el transcurso de esta gira, sólo les hizo advertencias en tres ocasiones; les recomendó que permanecieran lejos de Nazaret y que fueran discretos cuando pasaran por Cafarnaúm y Tiberiades. Para los apóstoles fue una causa de gran satisfacción sentir que por fin tenían la libertad de predicar y enseñar sin restricciones, y se lanzaron con una gran seriedad y alegría a la tarea de predicar el evangelio, atender a los enfermos y bautizar a los creyentes.

\section*{1. La predicación en Rimón}
\par 
%\textsuperscript{(1637.3)}
\textsuperscript{146:1.1} La pequeña ciudad de Rimón había estado dedicada en otro tiempo a la adoración de Ramán, un dios babilónico del aire. Las creencias de los rimonitas contenían todavía muchas enseñanzas babilónicas primitivas y enseñanzas posteriores de Zoroastro; por esta razón, Jesús y los veinticuatro consagraron mucho tiempo a la tarea de indicarles claramente la diferencia entre estas antiguas creencias y el nuevo evangelio del reino. Pedro predicó aquí sobre <<Aarón y el becerro de oro>>\footnote{\textit{Aarón y el becerro de oro}: Ex 32:1-35; Dt 9:16-21.}, uno de los grandes sermones del principio de su carrera.

\par 
%\textsuperscript{(1637.4)}
\textsuperscript{146:1.2} Aunque muchos ciudadanos de Rimón se convirtieron en creyentes de las enseñanzas de Jesús, en años posteriores causaron grandes dificultades a sus hermanos. En el corto espacio de una sola vida, es difícil convertir a unos adoradores de la naturaleza a la plena comunión de la adoración de un ideal espiritual.

\par 
%\textsuperscript{(1637.5)}
\textsuperscript{146:1.3} Muchos de los mejores conceptos babilónicos y persas sobre la luz y las tinieblas, el bien y el mal, el tiempo y la eternidad, fueron incorporados más tarde en las doctrinas del llamado cristianismo; esta inclusión hizo que los pueblos del Cercano Oriente aceptaran más rápidamente las enseñanzas cristianas. De la misma manera, la inclusión de muchas teorías de Platón sobre el espíritu ideal o los arquetipos invisibles de todas las cosas visibles y materiales, tal como Filón las adaptó más tarde a la teología hebrea, hizo que las enseñanzas cristianas de Pablo fueran aceptadas más fácilmente por los griegos occidentales.

\par 
%\textsuperscript{(1637.6)}
\textsuperscript{146:1.4} Fue en Rimón donde Todán escuchó por primera vez el evangelio del reino, y más tarde llevó este mensaje a Mesopotamia y mucho más allá. Fue uno de los primeros que predicó la buena nueva a los habitantes de más allá del Éufrates.

\section*{2. En Jotapata}
\par 
%\textsuperscript{(1638.1)}
\textsuperscript{146:2.1} Aunque la gente común y corriente de Jotapata escuchó con gusto a Jesús y sus apóstoles, y muchas personas aceptaron el evangelio del reino, lo más sobresaliente de esta misión en Jotapata fue el discurso de Jesús a los veinticuatro durante la segunda noche de su estancia en esta pequeña ciudad. Natanael tenía ideas confusas sobre las enseñanzas del Maestro respecto a la oración, la acción de gracias y la adoración. En respuesta a su pregunta, Jesús habló muy extensamente para explicar mejor su enseñanza. Resumido en un lenguaje moderno, este discurso se puede presentar para hacer hincapié en los puntos siguientes:

\par 
%\textsuperscript{(1638.2)}
\textsuperscript{146:2.2} 1. Cuando el corazón del hombre alberga una consideración consciente y persistente por la iniquidad, se va destruyendo gradualmente la conexión que el alma humana ha establecido, mediante la oración, con los circuitos espirituales de comunicación entre el hombre y su Hacedor\footnote{\textit{El corazón inicuo destruye la conexión de la oración}: Jer 17:9.}. Naturalmente, Dios escucha la súplica de su hijo, pero cuando el corazón humano alberga los conceptos de la iniquidad de manera deliberada y permanente, la comunión personal entre el hijo terrenal y su Padre celestial se pierde gradualmente.

\par 
%\textsuperscript{(1638.3)}
\textsuperscript{146:2.3} 2. Una oración que es incompatible con las leyes de Dios conocidas y establecidas, es una abominación para las Deidades del Paraíso. Si el hombre no quiere escuchar a los Dioses que hablan a su creación mediante las leyes del espíritu, de la mente y de la materia, un acto así de desprecio deliberado y consciente por parte de la criatura impide que las personalidades espirituales presten atención a las súplicas personales de esos mortales anárquicos y desobedientes. Jesús citó a sus apóstoles las palabras del profeta Zacarías: <<Pero se negaron a escuchar, se volvieron de espaldas y se taparon los oídos para no oír. Sí, endurecieron su corazón como una piedra, para no tener que oír mi ley ni las palabras que yo les enviaba por medio de mi espíritu a través de los profetas; por eso, los resultados de sus malos pensamientos recaen como una gran ira sobre sus cabezas culpables. Y sucedió que gritaron para recibir misericordia, pero ningún oído estaba abierto para escucharlos>>\footnote{\textit{Endurecieron su corazón}: Zac 7:11-13.}. Jesús citó a continuación el proverbio del sabio que decía: <<Si alguien desvía su oído para no escuchar la ley divina, incluso su oración será una abominación>>\footnote{\textit{Desviaron su oído y su oración fue una abominación}: Pr 28:9.}.

\par 
%\textsuperscript{(1638.4)}
\textsuperscript{146:2.4} 3. Al abrir el terminal humano del canal de comunicación entre Dios y el hombre, los mortales ponen inmediatamente a su disposición la corriente constante del ministerio divino para con las criaturas de los mundos. Cuando el hombre escucha hablar al espíritu de Dios dentro de su corazón humano, en esa experiencia se encuentra inherente el hecho de que Dios escucha simultáneamente la oración de ese hombre. El perdón de los pecados\footnote{\textit{El perdón}: Eclo 28:1-5; Mt 6:12,14-15; 18:21-35; Mc 11:25-26; Lc 6:37b; 11:4a; 17:3-4; Ef 4:32; 1 Jn 2:12.} también funciona de esta misma manera infalible. El Padre que está en los cielos os ha perdonado incluso antes de que hayáis pensado en pedírselo, pero dicho perdón no está disponible en vuestra experiencia religiosa personal hasta el momento en que perdonáis a vuestros semejantes. El perdón de Dios no está condicionado, de \textit{hecho}, por vuestro perdón a vuestros semejantes, pero como \textit{experiencia} está sometido exactamente a esa condición. Este hecho de la sincronización entre el perdón divino y el perdón humano estaba reconocido e incluido en la oración que Jesús enseñó a los apóstoles.

\par 
%\textsuperscript{(1638.5)}
\textsuperscript{146:2.5} 4. Existe una ley fundamental de justicia en el universo que la misericordia no tiene poder para burlar. Las glorias desinteresadas del Paraíso no pueden ser recibidas por una criatura totalmente egoísta de los reinos del tiempo y del espacio. Ni siquiera el amor infinito de Dios puede imponer la salvación de la supervivencia eterna a una criatura mortal que no escoge sobrevivir. La misericordia dispone de una gran libertad de donación, pero después de todo, hay mandatos de la justicia que ni siquiera el amor combinado con la misericordia pueden revocar eficazmente. Jesús citó de nuevo las escrituras hebreas: <<He llamado y habéis rehusado escuchar; he tendido mi mano, pero nadie ha prestado atención. Habéis despreciado todos mis consejos, y habéis rechazado mi desaprobación; debido a esta actitud rebelde, es inevitable que cuando me invoquéis no recibáis respuesta. Como habéis rechazado el camino de la vida, podéis buscarme con diligencia en vuestros momentos de sufrimiento, pero no me encontraréis>>\footnote{\textit{Efecto de rechaza a Dios}: Pr 1:24-28.}.

\par 
%\textsuperscript{(1639.1)}
\textsuperscript{146:2.6} 5. Los que quieran recibir misericordia, deberán mostrar misericordia; no juzguéis, para no ser juzgados\footnote{\textit{No juzguéis, para no ser juzgados}: Mt 7:1-2.}. Con el espíritu con que juzguéis a los demás también seréis juzgados. La misericordia no anula totalmente la justicia universal. Al final será cierto que: <<Cualquiera que cierra sus oídos al lamento del pobre, también pedirá ayuda algún día, y nadie lo escuchará>>\footnote{\textit{A quien desoiga el lamento del podre}: Pr 21:13.}. La sinceridad de cualquier oración es la garantía de que será escuchada; la sabiduría espiritual y la compatibilidad universal de cualquier petición determinan el momento, la manera y el grado de la respuesta. Un padre sabio no responde \textit{literalmente} a las oraciones tontas de sus hijos ignorantes e inexpertos, aunque dichos hijos puedan obtener mucho placer y una satisfacción real para su alma efectuando ese tipo de peticiones absurdas.

\par 
%\textsuperscript{(1639.2)}
\textsuperscript{146:2.7} 6. Cuando estéis totalmente consagrados a hacer la voluntad del Padre que está en los cielos, todas vuestras súplicas serán contestadas\footnote{\textit{Cuándo son contestadas las súplicas}: Mt 7:7-11; 21:22; Mc 11:24-26; Lc 11:9-13; Jn 14:13-14; 15:7,16; 16:23-24.}, porque vuestras oraciones estarán plenamente de acuerdo con la voluntad del Padre, y la voluntad del Padre se manifiesta constantemente en todo su inmenso universo. Aquello que un verdadero hijo desea y el Padre infinito lo quiere, EXISTE. Una oración así no puede permanecer sin respuesta, y es posible que ningún otro tipo de petición pueda ser contestada plenamente.

\par 
%\textsuperscript{(1639.3)}
\textsuperscript{146:2.8} 7. El grito del justo es el acto de fe del hijo de Dios que abre la puerta del almacén de bondad, de verdad y de misericordia del Padre; estos dones preciados han estado esperando mucho tiempo a que el hijo se acerque y se los apropie personalmente. La oración no cambia la actitud divina hacia el hombre, pero sí cambia la actitud del hombre hacia el Padre invariable. Es el \textit{móvil} de la oración lo que le da el derecho de acceso al oído divino, y no el estado social, económico o religioso exterior de aquel que ora.

\par 
%\textsuperscript{(1639.4)}
\textsuperscript{146:2.9} 8. La oración no se puede emplear para evitar las demoras del tiempo ni para trascender los obstáculos del espacio. La oración no es una técnica diseñada para engrandecer el yo ni para conseguir una ventaja injusta sobre los semejantes. Un alma totalmente egoísta es incapaz de orar en el verdadero sentido de la palabra. Jesús dijo: <<Que vuestra delicia suprema esté en el carácter de Dios, y él os concederá con seguridad los sinceros deseos de vuestro corazón>>\footnote{\textit{Delicia en Dios}: Sal 37:4.}. <<Encomendad vuestro camino al Señor; confiad en él, y él actuará>>\footnote{\textit{Confiad en Dios}: Sal 37:5.}. <<Porque el Señor escucha el lamento del indigente y atenderá la oración del desamparado>>\footnote{\textit{Dios escucha el lamento del necesitado}: Sal 72:12; 102:17.}.

\par 
%\textsuperscript{(1639.5)}
\textsuperscript{146:2.10} 9. <<Yo he salido del Padre; por lo tanto, si alguna vez tenéis dudas sobre lo que debéis pedirle al Padre, pedidlo en mi nombre, y yo presentaré vuestra petición de acuerdo con vuestras necesidades y deseos reales y en conformidad con la voluntad de mi Padre>>. Guardaos contra el grave peligro de volveros egocéntricos en vuestras oraciones. Evitad orar mucho por vosotros mismos; orad más por el progreso espiritual de vuestros hermanos. Evitad las oraciones materialistas; orad en espíritu y por la abundancia de los dones del espíritu\footnote{\textit{Orad por necesidades y deseos reales}: Jn 14:13-14.}.

\par 
%\textsuperscript{(1639.6)}
\textsuperscript{146:2.11} 10. Cuando oréis por los enfermos y los afligidos, no esperéis que vuestras súplicas reemplacen los cuidados afectuosos e inteligentes que necesitan esos afligidos. Orad por el bienestar de vuestras familias, amigos y compañeros, pero orad especialmente por aquellos que os maldicen, y efectuad súplicas afectuosas por aquellos que os persiguen\footnote{\textit{Orad por vuestros enemigos}: Mt 5:44; Lc 6:28.}. <<En cuanto al momento en que debéis orar, no os lo indicaré. Sólo el espíritu que reside en vosotros puede incitaros a manifestar las peticiones que expresen vuestra relación interior con el Padre de los espíritus>>\footnote{\textit{Padre de los espíritus}: Heb 12:9.}.

\par 
%\textsuperscript{(1640.1)}
\textsuperscript{146:2.12} 11. Mucha gente sólo recurre a la oración cuando tiene dificultades. Una práctica así es irreflexiva y descaminada. Es verdad que hacéis bien en orar cuando estáis agobiados, pero también deberíais acordaros de hablar con vuestro Padre como un hijo, incluso cuando todo va bien para vuestra alma. Que vuestras súplicas reales sean siempre en secreto\footnote{\textit{Orad en secreto}: Mt 6:6.}. No permitáis que los hombres escuchen vuestras oraciones personales. Las oraciones de acción de gracias son apropiadas para los grupos de adoradores, pero la oración del alma es un asunto personal. Sólo existe una forma de oración que es apropiada para todos los hijos de Dios, y es: <<Sin embargo, que se haga tu voluntad>>.

\par 
%\textsuperscript{(1640.2)}
\textsuperscript{146:2.13} 12. Todos los que creen en este evangelio deberían orar sinceramente por la expansión del reino de los cielos. De todas las oraciones de las Escrituras hebreas, Jesús hizo un comentario muy favorable sobre esta súplica del salmista: <<Crea en mí un corazón limpio, oh Dios, y renueva un espíritu recto dentro de mí. Purifícame de los pecados secretos y preserva a tu servidor de las transgresiones presuntuosas>>\footnote{\textit{Crea un corazón limpio}: Sal 51:10. \textit{Purifícame de mis pecados}: Sal 19:12-13.}. Jesús hizo un extenso comentario sobre la relación entre la oración y el lenguaje descuidado y ofensivo, citando el pasaje: <<Oh Señor, pon un vigilante delante de mi boca, y guarda la puerta de mis labios>>\footnote{\textit{Pon un vigilante en mi boca}: Sal 141:3.}. Jesús dijo: <<La lengua humana es un órgano que muy pocos hombres saben domar; pero el espíritu interior puede transformar este miembro indómito en una suave voz de tolerancia y en un ministro inspirador de misericordia>>\footnote{\textit{Lengua indómita}: Stg 3:8.}.

\par 
%\textsuperscript{(1640.3)}
\textsuperscript{146:2.14} 13. Jesús enseñó que la oración para recibir la guía divina en el sendero de la vida terrestre seguía en importancia a la súplica para conocer la voluntad del Padre. Esto significa, en realidad, orar para obtener la sabiduría divina. Jesús no enseñó nunca que pudieran obtenerse conocimientos humanos y habilidades especiales por medio de la oración. Pero sí enseñó que la oración es un factor en la ampliación de nuestra capacidad para recibir la presencia del espíritu divino. Cuando Jesús enseñó a sus asociados que oraran en espíritu y en verdad\footnote{\textit{Orar en espíritu y en verdad}: Jn 4:24.}, explicó que se refería a que oraran con sinceridad y de acuerdo con las luces que poseía cada cual, que oraran de todo corazón y con inteligencia, seriedad y constancia.

\par 
%\textsuperscript{(1640.4)}
\textsuperscript{146:2.15} 14. Jesús previno a sus discípulos contra la idea de que sus oraciones serían más eficaces utilizando repeticiones adornadas\footnote{\textit{Oraciones elocuentes y adornadas}: Mt 6:7-8a.}, una fraseología elocuente, el ayuno, la penitencia o los sacrificios. Pero sí exhortó a sus creyentes a que emplearan la oración como un medio de elevarse a la verdadera adoración a través de la acción de gracias. Jesús deploraba que se encontrara tan poco espíritu de acción de gracias en las oraciones y el culto de sus seguidores. En esta ocasión citó las Escrituras, diciendo: <<Es bueno dar gracias al Señor y cantar alabanzas al nombre del Altísimo, reconocer su misericordia cada mañana y su fidelidad cada noche, porque Dios me ha hecho feliz con su obra. Daré gracias por todas las cosas en conformidad con la voluntad de Dios>>\footnote{\textit{Es bueno dar gracias al Señor}: Sal 92:1-2. \textit{Dios me ha hecho feliz con su obra}: Sal 92:4. \textit{Daré gracias por todas las cosas}: 1 Ts 5:18.}.

\par 
%\textsuperscript{(1640.5)}
\textsuperscript{146:2.16} 15. Jesús dijo a continuación: <<No os preocupéis constantemente por vuestras necesidades ordinarias. No sintáis aprensión por los problemas de vuestra existencia terrestre; en todas estas cosas, mediante la oración y la súplica, con un espíritu sincero de acción de gracias, exponed vuestras necesidades ante vuestro Padre que está en los cielos>>\footnote{\textit{No os preocupéis constantemente}: Flp 4:6.}. Luego citó de las Escrituras: <<Alabaré el nombre de Dios con un cántico y lo ensalzaré con mi acción de gracias. Esto agradará más al Señor que el sacrificio de un buey o de un becerro con cuernos y pezuñas>>\footnote{\textit{Alabaré a Dios con un cántico}: Sal 69:30-31.}.

\par 
%\textsuperscript{(1641.1)}
\textsuperscript{146:2.17} 16. Jesús enseñó a sus seguidores que, después de haber hecho sus oraciones al Padre, deberían permanecer algún tiempo en un estado de receptividad silenciosa para proporcionar al espíritu interior las mejores posibilidades de hablarle al alma atenta. El espíritu del Padre le habla mejor al hombre cuando la mente humana se encuentra en una actitud de verdadera adoración. Adoramos a Dios con la ayuda del espíritu interior del Padre y mediante la iluminación de la mente humana a través del ministerio de la verdad. Jesús enseñó que la adoración hace al adorador cada vez más semejante al ser que adora. La adoración es una experiencia transformadora por medio de la cual el finito se acerca gradualmente a la presencia del Infinito, y finalmente la alcanza.

\par 
%\textsuperscript{(1641.2)}
\textsuperscript{146:2.18} Jesús contó a sus apóstoles otras muchas verdades sobre la comunión del hombre con Dios, pero pocos de ellos pudieron abarcar plenamente su enseñanza.

\section*{3. La parada en Ramá}
\par 
%\textsuperscript{(1641.3)}
\textsuperscript{146:3.1} Jesús tuvo en Ramá el debate memorable con el anciano filósofo griego que enseñaba que la ciencia y la filosofía eran suficientes para satisfacer las necesidades de la experiencia humana. Jesús escuchó con paciencia y simpatía a este educador griego, aceptando la verdad de muchas de las cosas que dijo. Pero cuando terminó de hablar, Jesús le señaló que en su examen de la existencia humana había omitido explicar <<de dónde, por qué, y hacia dónde>>, y añadió: <<Allí donde tú terminas, empezamos nosotros. La religión es una revelación al alma humana que trata con unas realidades espirituales que la mente sola nunca podría descubrir ni sondear por completo. Los esfuerzos intelectuales pueden revelar los hechos de la vida, pero el evangelio del reino descubre las \textit{verdades} de la existencia. Tú has hablado de las sombras materiales de la verdad; ¿quieres escucharme ahora mientras te hablo de las realidades eternas y espirituales que proyectan esas sombras temporales transitorias de los hechos materiales de la existencia mortal?>> Durante más de una hora, Jesús enseñó a este griego las verdades salvadoras del evangelio del reino. Al anciano filósofo le conmovió el modo de acercarse del Maestro, y como era sinceramente honrado de corazón, creyó rápidamente en este evangelio de salvación.

\par 
%\textsuperscript{(1641.4)}
\textsuperscript{146:3.2} Los apóstoles estaban un poco desconcertados por la manera evidente con que Jesús aprobaba muchas de las proposiciones del griego, pero Jesús les dijo más tarde en privado: <<Hijos míos, no os asombréis por mi tolerancia con la filosofía del griego. La certidumbre interior verdadera y auténtica no teme en absoluto el análisis exterior, ni la verdad se resiente por una crítica honesta. No deberíais olvidar nunca que la intolerancia es la máscara que cubre las dudas que se mantienen en secreto sobre la autenticidad de las creencias que uno tiene. A nadie le inquieta en ningún momento la actitud de su vecino, cuando tiene una confianza total en la verdad de lo que cree de todo corazón. El coraje es la confianza completamente honesta en las cosas que uno profesa creer. Los hombres sinceros no temen el examen crítico de sus verdaderas convicciones y de sus nobles ideales>>.

\par 
%\textsuperscript{(1641.5)}
\textsuperscript{146:3.3} La segunda noche en Ramá, Tomás le hizo a Jesús la pregunta siguiente: <<Maestro, un nuevo creyente en tus enseñanzas ¿cómo puede saber realmente, estar realmente seguro, de la verdad de este evangelio del reino?>>

\par 
%\textsuperscript{(1641.6)}
\textsuperscript{146:3.4} Jesús le dijo a Tomás: <<Tu seguridad de que has entrado en la familia del reino del Padre y de que sobrevivirás eternamente con los hijos del reino es enteramente un asunto de experiencia personal ---de fe en la palabra de la verdad. La seguridad espiritual equivale a tu experiencia religiosa personal con las realidades eternas de la verdad divina; dicho de otra manera, es igual a tu comprensión inteligente de las realidades de la verdad, más tu fe espiritual y menos tus dudas sinceras>>.

\par 
%\textsuperscript{(1642.1)}
\textsuperscript{146:3.5} <<El Hijo está dotado por naturaleza de la vida del Padre. Como habéis sido dotados del espíritu viviente del Padre, sois por tanto hijos de Dios. Sobrevivís a vuestra vida en el mundo material de la carne porque estáis identificados con el espíritu viviente del Padre, el don de la vida eterna. En verdad, muchas personas tenían esta vida antes de que yo viniera del Padre, y muchos más han recibido este espíritu porque han creído en mis palabras; pero os aseguro que, cuando yo regrese al Padre, él enviará su espíritu al corazón de todos los hombres>>\footnote{\textit{Hijo dotado de la vida del Padre}: Jn 5:26. \textit{Espíritu viviente del Padre}: Job 32:8,18; Is 63:10-11; Ez 37:14; Mt 10:20; Lc 17:21; Jn 17:21-23; Ro 8:9-11; 1 Co 3:16-17; 6:19; 2 Co 6:16; Gl 2:20; 1 Jn 3:24; 4:12-15; Ap 21:3. \textit{Los creyentes reciben el espíritu}: Jn 5:24.}.

\par 
%\textsuperscript{(1642.2)}
\textsuperscript{146:3.6} <<Aunque no podéis observar al espíritu divino trabajando en vuestra mente, existe un método práctico para descubrir hasta qué punto habéis cedido el control de los poderes de vuestra alma a la enseñanza y a la dirección de este espíritu interior del Padre celestial: es el grado de vuestro amor por vuestros semejantes humanos. Este espíritu del Padre participa del amor del Padre, y a medida que domina al hombre, lo conduce infaliblemente en la dirección de la adoración divina y de la consideración afectuosa por los semejantes. Al principio, creéis que sois los hijos de Dios porque mi enseñanza os ha hecho más conscientes de las directrices internas de la presencia de nuestro Padre que reside en vosotros; pero el Espíritu de la Verdad será derramado dentro de poco sobre todo el género humano, y vivirá entre los hombres y los enseñará a todos, como yo ahora vivo entre vosotros y os digo las palabras de la verdad. Este Espíritu de la Verdad, que habla para los dones espirituales de vuestra alma, os ayudará a saber que sois los hijos de Dios. Dará testimonio de manera infalible con la presencia interior del Padre, vuestro espíritu, que entonces residirá en todos los hombres, como ahora reside en algunos, y os dirá que sois en realidad los hijos de Dios>>\footnote{\textit{El espíritu de Dios ayuda a conocer la filiación}: Ro 8:16. \textit{El espíritu interior confirma}: Ro 8:14. \textit{El Espíritu de la Verdad}: Ez 11:19; 18:31; 36:26-27; Jl 2:28-29; Lc 24:49; Jn 7:39; 14:16-18,23,26; 15:4,26; 16:7-8,13-14; 17:21-23; Hch 1:5,8a; 2:1-4,16-18; 2:33; 2 Co 13:5; Gl 2:20; 4:6; Ef 1:13; 4:30; 1 Jn 4:12-15.}.

\par 
%\textsuperscript{(1642.3)}
\textsuperscript{146:3.7} <<Todo hijo terrestre que sigue las directrices de este espíritu terminará conociendo la voluntad de Dios, y aquel que se abandona a la voluntad de mi Padre vivirá para siempre. El camino que va de la vida terrestre al estado eterno no se os ha indicado claramente; sin embargo hay un camino, siempre lo ha habido, y yo he venido para hacerlo nuevo y viviente. Aquel que entra en el reino ya tiene la vida eterna ---no perecerá nunca. Pero muchas de estas cosas las comprenderéis mejor cuando yo haya regresado al Padre, y seáis capaces de contemplar retrospectivamente vuestras experiencias de ahora>>\footnote{\textit{Quien siga al espíritu vivirá para siempre}: Jn 10:27-28; 17:2-3. \textit{La vida eterna}: Dn 12:2; Mt 19:16,29; 25:46; Mc 10:17,30; Lc 10:25; 18:18,30; Jn 3:15-16,36; 4:14,36; 5:24,39; 6:27,40,47; 6:54,68; 8:51-52; 10:28; 11:25-26; 12:25,50; 17:2-3; Hch 13:46-48; Ro 2:7; 5:21; 6:22-23; Gl 6:8; 1 Ti 1:16; 6:12,19; Tit 1:2; 3:7; 1 Jn 1:2; 2:25; 3:15; 5:11,13,20; Jud 1:21; Ap 22:5. \textit{Un camino nuevo y viviente}: Heb 10:20.}.

\par 
%\textsuperscript{(1642.4)}
\textsuperscript{146:3.8} Todos los que escucharon estas palabras bienaventuradas se llenaron de regocijo. Las enseñanzas judías sobre la supervivencia de los justos eran confusas e inciertas, y para los discípulos de Jesús resultaba vivificante e inspirador escuchar estas palabras tan precisas y positivas, asegurando la supervivencia eterna para todos los creyentes sinceros.

\par 
%\textsuperscript{(1642.5)}
\textsuperscript{146:3.9} Los apóstoles continuaron predicando y bautizando a los creyentes, conservando la costumbre de ir de casa en casa para confortar a los deprimidos y atender a los enfermos y afligidos. La organización apostólica se había ampliado, en el sentido de que cada apóstol de Jesús tenía ahora como asociado a un apóstol de Juan; Abner era el asociado de Andrés; y este plan prevaleció hasta que bajaron a Jerusalén para la Pascua siguiente.

\par 
%\textsuperscript{(1642.6)}
\textsuperscript{146:3.10} Durante su estancia en Zabulón, la instrucción especial que Jesús les dio consistió principalmente en nuevas discusiones sobre las obligaciones recíprocas en el reino, y englobó una enseñanza destinada a clarificar las diferencias entre la experiencia religiosa personal y las buenas relaciones en las obligaciones religiosas sociales. Ésta fue una de las pocas veces que el Maestro discurrió sobre los aspectos sociales de la religión. A lo largo de toda su vida en la Tierra, Jesús dio a sus discípulos muy pocas instrucciones sobre la socialización de la religión.

\par 
%\textsuperscript{(1643.1)}
\textsuperscript{146:3.11} La población de Zabulón era de raza mixta, ni judía ni gentil, y pocos de ellos creyeron realmente en Jesús, a pesar de que habían oído hablar de la curación de los enfermos en Cafarnaúm.

\section*{4. El evangelio en Irón}
\par 
%\textsuperscript{(1643.2)}
\textsuperscript{146:4.1} En Irón, como también en muchas de las ciudades más pequeñas de Galilea y Judea, había una sinagoga, y durante los primeros tiempos de su ministerio, Jesús tenía la costumbre de hablar los sábados en estas sinagogas. A veces hablaba durante los oficios de la mañana, y Pedro o uno de los otros apóstoles predicaba por la tarde. Jesús y los apóstoles también enseñaban y predicaban a menudo en las asambleas vespertinas de la sinagoga durante los días de la semana. Aunque los jefes religiosos de Jerusalén eran cada vez más hostiles hacia Jesús, no ejercían ningún control directo sobre las sinagogas exteriores a la ciudad. Sólo en una época más tardía del ministerio público de Jesús, consiguieron crear un sentimiento tan generalizado en contra de él que provocaron casi el cierre total de las sinagogas a su enseñanza. Pero en estos momentos, todas las sinagogas de Galilea y Judea estaban abiertas para él\footnote{\textit{Sinagogas abiertas}: Mc 1:39.}.

\par 
%\textsuperscript{(1643.3)}
\textsuperscript{146:4.2} En Irón se encontraban unas minas muy importantes para aquella época, y como Jesús nunca había compartido la vida de los mineros, durante su estancia en Irón pasó la mayor parte de su tiempo en las minas. Mientras los apóstoles visitaban los hogares y predicaban en los lugares públicos, Jesús trabajaba en las minas con estos obreros subterráneos. La fama de Jesús como sanador se había propagado hasta este pueblo remoto, y muchos enfermos y afligidos buscaron su ayuda; la gente se benefició ampliamente de su ministerio curativo. Pero el Maestro no efectuó, en ninguno de estos casos, un pretendido milagro de curación, exceptuando el del leproso.

\par 
%\textsuperscript{(1643.4)}
\textsuperscript{146:4.3} Al final de la tarde del tercer día en Irón, cuando Jesús regresaba de las minas, pasó por casualidad por una angosta calle lateral en dirección a su alojamiento. Al acercarse a la choza miserable de cierto leproso, el afectado, que había oído hablar de la fama de Jesús como sanador, se atrevió a abordarlo cuando pasaba por su puerta, y se arrodilló delante de él, diciendo: <<Señor, si tan sólo quisieras, podrías purificarme. He oído el mensaje de tus instructores y quisiera entrar en el reino si pudiera ser purificado>>. El leproso se expresó de esta manera porque, entre los judíos, a los leprosos se les prohibía incluso asistir a la sinagoga o practicar otro tipo de culto en público. Este hombre creía realmente que no sería recibido en el reino venidero a menos que pudiera curarse de su lepra. Cuando Jesús lo vio así de afligido y escuchó sus palabras impregnadas de fe, su corazón humano se conmovió y su mente divina se enterneció de compasión. Mientras Jesús lo contemplaba, el hombre se echó de bruces y lo adoró. Entonces, el Maestro alargó su mano, lo tocó y le dijo: <<Sí quiero ---queda purificado>>. Y el hombre se curó de inmediato; la lepra había dejado de afligirlo\footnote{\textit{Curación del leproso}: Mt 8:1-3; Mc 1:40-42; Lc 5:12-13.}.

\par 
%\textsuperscript{(1643.5)}
\textsuperscript{146:4.4} Cuando Jesús hubo levantado al hombre del suelo, le encargó: <<Cuida de no hablarle a nadie de tu curación, sino más bien dirígete tranquilamente a tus asuntos, preséntate ante el sacerdote y ofrece los sacrificios ordenados por Moisés en testimonio de tu purificación>>\footnote{\textit{``No hables de tu curación''}: Mt 8:4.}. Pero este hombre no hizo lo que Jesús le había indicado. En lugar de eso, empezó a anunciar por toda la localidad que Jesús lo había curado de su lepra\footnote{\textit{El leproso habla}: Mc 1:43-45; Lc 5:14-15.}, y como todo el pueblo lo conocía, la gente pudo ver claramente que había sido librado de su enfermedad. No fue a ver a los sacerdotes como Jesús le había recomendado. Como consecuencia de haber divulgado la noticia de que Jesús lo había curado, el Maestro fue tan asediado por los enfermos que se vio obligado a levantarse temprano al día siguiente y dejar el pueblo. Aunque Jesús no volvió a entrar en la ciudad, permaneció dos días en las afueras cerca de las minas, donde continuó enseñando más cosas a los mineros creyentes sobre el evangelio del reino\footnote{\textit{El evangelio del reino}: Mt 4:23; 9:35; 24:14; Mc 1:14-15.}.

\par 
%\textsuperscript{(1644.1)}
\textsuperscript{146:4.5} Esta purificación del leproso era el primer supuesto milagro que Jesús había realizado intencional y deliberadamente hasta ese momento. Y se trataba de un auténtico caso de lepra.

\par 
%\textsuperscript{(1644.2)}
\textsuperscript{146:4.6} Desde Irón fueron a Giscala, donde pasaron dos días proclamando el evangelio, y luego partieron hacia Corazín, donde estuvieron casi una semana predicando la buena nueva, pero en esta ciudad fueron incapaces de conseguir muchos creyentes para el reino. En ningún lugar donde Jesús había enseñado había encontrado un rechazo tan general de su mensaje. La estancia en Corazín fue muy deprimente para la mayoría de los apóstoles; Andrés y Abner tuvieron muchas dificultades para levantar el ánimo de sus asociados. Así pues, atravesaron tranquilamente Cafarnaúm, y continuaron hasta el pueblo de Madón, donde no tuvieron mucho más éxito. En la mente de la mayoría de los apóstoles prevalecía la idea de que su falta de éxito en estas ciudades que habían visitado tan recientemente se debía a la insistencia de Jesús de que, en sus enseñanzas y predicaciones, se abstuvieran de hablar de él como sanador. ¡Cuánto hubieran deseado que purificara a otro leproso o que manifestara su poder de alguna otra manera para atraer la atención de la gente! Pero el Maestro se mantuvo impasible ante sus ardientes deseos.

\section*{5. De vuelta en Caná}
\par 
%\textsuperscript{(1644.3)}
\textsuperscript{146:5.1} El grupo apostólico se alegró enormemente cuando Jesús anunció: <<Mañana iremos a Caná>>\footnote{\textit{A Galilea, a Caná}: Jn 4:43.}. Sabían que en Caná los escucharían con simpatía, porque Jesús era bien conocido allí. Iban prosperando en su trabajo de atraer a la gente al reino cuando, al tercer día, cierto ciudadano destacado de Cafarnaúm, llamado Tito, se presentó en Caná; era un creyente a medias y su hijo estaba gravemente enfermo\footnote{\textit{El hijo enfermo del noble}: Jn 4:46.}. Había oído que Jesús estaba en Caná, por lo que se apresuró a ir a verlo. Los creyentes de Cafarnaúm consideraban que Jesús podía curar cualquier enfermedad.

\par 
%\textsuperscript{(1644.4)}
\textsuperscript{146:5.2} Cuando este noble hubo localizado a Jesús en Caná, le suplicó que fuera rápidamente a Cafarnaúm para curar a su hijo afligido. Mientras los apóstoles permanecían cerca con la respiración cortada por la expectación, Jesús, mirando al padre del muchacho enfermo, dijo: <<¿Cuánto tiempo seré indulgente con vosotros? El poder de Dios está en medio de vosotros, pero a menos que veáis signos y contempléis prodigios, os negáis a creer>>. Pero el noble le suplicó a Jesús, diciendo: <<Señor mío, yo sí creo, pero ven antes de que mi hijo perezca, porque cuando lo dejé ya estaba a punto de morir>>. Después de inclinar la cabeza unos momentos, en una meditación silenciosa, Jesús dijo súbitamente: <<Vuelve a tu hogar; tu hijo vivirá>>. Tito creyó en la palabra de Jesús y se apresuró a regresar a Cafarnaúm. Cuando iba de vuelta, sus sirvientes salieron a su encuentro, diciendo: <<Regocíjate, pues tu hijo ha mejorado ---vive>>. Entonces Tito les preguntó a qué hora había empezado a mejorar el muchacho, y cuando los criados contestaron <<ayer, hacia la hora séptima, desapareció la fiebre>>, el padre recordó que era aproximadamente esa hora cuando Jesús había dicho: <<Tu hijo vivirá>>. A partir de entonces Tito creyó de todo corazón, y toda su familia también creyó. Su hijo se convirtió en un poderoso ministro del reino y más tarde sacrificó su vida con los que sufrían en Roma. Toda la familia de Tito, sus amigos, e incluso los apóstoles, consideraron este episodio como un milagro, pero no lo fue. Al menos éste no fue un milagro de curación de una enfermedad física\footnote{\textit{Curación no milagrosa}: Jn 4:54.}. Fue simplemente un caso de preconocimiento respecto al proceso de la ley natural, precisamente el tipo de conocimiento al que Jesús recurrió con frecuencia después de su bautismo\footnote{\textit{Jesús previó la recuperación}: Jn 4:47-53.}.

\par 
%\textsuperscript{(1645.1)}
\textsuperscript{146:5.3} Jesús se vio de nuevo forzado a salir apresuradamente de Caná debido a que el segundo episodio de este tipo que acompañó a su ministerio en esta población había llamado excesivamente la atención. Los vecinos del pueblo se acordaban del agua y del vino, y ahora que suponían que Jesús había curado al hijo del noble a una distancia tan grande, acudían a él no solamente para traerle a los enfermos y a los afligidos, sino también para enviarle mensajeros con el ruego de que curara a los pacientes a distancia. Cuando Jesús vio que toda la región estaba alborotada, dijo: <<Vamos a Naín>>.

\section*{6. Naín y el hijo de la viuda}
\par 
%\textsuperscript{(1645.2)}
\textsuperscript{146:6.1} Esta gente creía en los signos; era una generación que buscaba prodigios. Por esta época, los habitantes de la Galilea central y meridional pensaban en Jesús y en su ministerio personal en términos de milagros. Decenas, centenares de personas honradas que sufrían de desórdenes puramente nerviosos y que estaban afligidas por trastornos emocionales, se presentaban delante de Jesús, y luego volvían a sus casas anunciando a sus amigos que Jesús las había curado. Esta gente ignorante y simple consideraba estos casos de curación mental como curaciones físicas, como curas milagrosas.

\par 
%\textsuperscript{(1645.3)}
\textsuperscript{146:6.2} Cuando Jesús intentó alejarse de Caná para ir a Naín, una gran multitud de creyentes y muchos curiosos se fueron detrás de él. Estaban decididos a contemplar milagros y prodigios, y no iban a quedar decepcionados. Cuando Jesús y sus apóstoles se acercaban a la puerta de la ciudad, se encontraron con una procesión fúnebre que se dirigía al cementerio cercano para llevar al hijo único de una madre viuda de Naín\footnote{\textit{El hijo de la viuda}: Lc 7:11-15.}. Esta mujer era muy respetada, y la mitad del pueblo iba detrás de los que llevaban el féretro de este muchacho supuestamente muerto. Cuando la procesión fúnebre llegó a la altura de Jesús y sus seguidores, la viuda y sus amigos reconocieron al Maestro, y le suplicaron que devolviera el hijo a la vida. Sus expectativas de un milagro se habían despertado hasta tal extremo que creían que Jesús podía curar cualquier enfermedad humana y, ¿por qué este sanador no podría incluso revivir a los muertos? Al ser importunado de esta manera, Jesús se adelantó, levantó la tapa del ataúd y examinó al muchacho. Al descubrir que el joven no estaba realmente muerto, percibió la tragedia que su presencia podía evitar. Así pues, se volvió hacia la madre y le dijo: <<No llores. Tu hijo no está muerto; está dormido. Te será devuelto>>. Luego cogió al joven de la mano y le dijo: <<Despiértate y levántate>>. Y el joven supuestamente muerto se incorporó enseguida y empezó a hablar, y Jesús los envió de vuelta a sus casas.

\par 
%\textsuperscript{(1645.4)}
\textsuperscript{146:6.3} Jesús se esforzó por calmar a la multitud y trató en vano de explicarles que el muchacho no estaba realmente muerto, que él no lo había traído de la tumba, pero fue inútil. La multitud que lo seguía, y todo el pueblo de Naín, habían llegado al máximo grado de frenesí emotivo\footnote{\textit{La gente asombrada}: Lc 7:16.}. Muchos fueron dominados por el miedo, otros por el pánico, mientras que otros aún empezaron a rezar y a lamentarse por sus pecados. No se pudo dispersar a la ruidosa multitud hasta mucho después de la caída de la noche. Naturalmente, a pesar de la afirmación de Jesús de que el muchacho no estaba muerto, todos insistían en que se había producido un milagro, que el muerto había sido resucitado. Aunque Jesús les dijo que el muchacho estaba simplemente en un estado de sueño profundo, explicaron que ésa era su manera de hablar, y llamaron la atención sobre el hecho de que siempre trataba de ocultar sus milagros con mucha modestia.

\par 
%\textsuperscript{(1646.1)}
\textsuperscript{146:6.4} Así pues, la noticia de que Jesús había resucitado de entre los muertos al hijo de la viuda se divulgó por toda Galilea y Judea, y muchos de los que la escucharon se la creyeron. Jesús nunca pudo hacer entender por completo, ni siquiera a todos sus apóstoles, que el hijo de la viuda no estaba realmente muerto cuando le ordenó que se despertara y se levantara. Pero sí los convenció lo suficiente como para evitar que este suceso se incluyera en todos los escritos posteriores, salvo en el de Lucas\footnote{\textit{Un supuesto milagro}: Lc 7:17.}, que relató el episodio tal como se lo habían contado. Una vez más Jesús fue tan asediado como médico, que al día siguiente temprano partió para Endor.

\section*{7. En Endor}
\par 
%\textsuperscript{(1646.2)}
\textsuperscript{146:7.1} En Endor, Jesús eludió durante unos días a las ruidosas multitudes que buscaban la curación física. Durante su estancia en este lugar, el Maestro refirió, para instrucción de los apóstoles, la historia del rey Saúl y la bruja de Endor\footnote{\textit{La bruja de Endor}: 1 Sam 28:7-25.}. Jesús indicó claramente a sus apóstoles que los intermedios desviados y rebeldes que habían personificado con frecuencia a los supuestos espíritus de los muertos, pronto serían puestos bajo control de manera que ya no podrían volver a hacer estas cosas extrañas. Dijo a sus discípulos que, después de que volviera al Padre, y después de que hubieran derramado su espíritu sobre todo el género humano, estos seres semiespirituales ---llamados espíritus impuros--- ya no podrían poseer a los débiles mentales ni a los mortales malintencionados.

\par 
%\textsuperscript{(1646.3)}
\textsuperscript{146:7.2} Jesús explicó además a sus apóstoles que los espíritus de los seres humanos fallecidos no regresan a su mundo de origen para comunicarse con sus semejantes vivos. Al espíritu en progreso del hombre mortal sólo le sería posible volver a la Tierra después de haber transcurrido una época dispensacional, e incluso entonces, sólo sería en casos excepcionales y como parte de la administración espiritual del planeta.

\par 
%\textsuperscript{(1646.4)}
\textsuperscript{146:7.3} Después de haber descansado dos días, Jesús dijo a sus apóstoles: <<Regresemos mañana a Cafarnaúm para quedarnos allí y enseñar mientras se calman los alrededores. A estas alturas, en nuestro pueblo ya se habrán recuperado en parte de esta especie de agitación>>.


\chapter{Documento 147. El paréntesis de la visita a Jerusalén}
\par 
%\textsuperscript{(1647.1)}
\textsuperscript{147:0.1} JESÚS y los apóstoles llegaron a Cafarnaúm el miércoles 17 de marzo y pasaron dos semanas en su cuartel general de Betsaida antes de partir para Jerusalén. Durante estas dos semanas, los apóstoles enseñaron a la gente en la orilla del mar, mientras que Jesús pasó mucho tiempo a solas en las colinas, ocupado en los asuntos de su Padre. En el transcurso de este período, Jesús, acompañado de Santiago y Juan Zebedeo, hizo dos viajes secretos a Tiberiades, donde se encontraron con los creyentes y los instruyeron en el evangelio del reino.

\par 
%\textsuperscript{(1647.2)}
\textsuperscript{147:0.2} Muchos miembros de la casa de Herodes creían en Jesús y asistieron a estas reuniones. La influencia de estos creyentes dentro de la familia oficial de Herodes fue la que había contribuido a que disminuyera la enemistad de este gobernador hacia Jesús. Estos creyentes de Tiberiades habían explicado plenamente a Herodes que el <<reino>> que Jesús proclamaba era de naturaleza espiritual, y no una aventura política. Herodes daba bastante crédito a estos miembros de su propia casa y por eso no llegó a alarmarse indebidamente por la divulgación de las noticias sobre las enseñanzas y las curaciones de Jesús. No tenía objeciones al trabajo de Jesús como sanador o instructor religioso. A pesar de la actitud favorable de muchos consejeros de Herodes, e incluso del mismo Herodes, había un grupo de subordinados suyos que estaban tan influídos por los jefes religiosos de Jerusalén, que continuaron siendo enemigos encarnizados y amenazadores de Jesús y de los apóstoles; más tarde, este grupo contribuyó mucho a impedir sus actividades públicas. El peligro más grande para Jesús residía en los dirigentes religiosos de Jerusalén, y no en Herodes. Precisamente por esta razón, Jesús y los apóstoles pasaron tanto tiempo en Galilea e hicieron allí la mayor parte de su predicación pública, en lugar de hacerlo en Jerusalén y en Judea.

\section*{1. El servidor del centurión}
\par 
%\textsuperscript{(1647.3)}
\textsuperscript{147:1.1} El día antes de prepararse para ir a Jerusalén a la fiesta de la Pascua, Mangus, un centurión o capitán de la guardia romana estacionada en Cafarnaúm, fue a ver a los jefes de la sinagoga, diciendo: <<Mi fiel ordenanza está enfermo y a punto de morir. ¿Podríais ir a ver a Jesús en mi nombre para suplicarle que cure a mi servidor?>>\footnote{\textit{El sirviente del centurión}: Mt 8:5-6; Lc 7:1-5.} El capitán romano actuó así porque pensaba que los dirigentes judíos tendrían más influencia sobre Jesús. Así pues, los ancianos fueron a ver a Jesús y su portavoz le dijo: <<Maestro, te rogamos encarecidamente que vayas a Cafarnaúm para salvar al servidor favorito del centurión romano; este capitán es digno de tu atención porque ama a nuestra nación e incluso nos ha construido la sinagoga donde has hablado tantas veces>>.

\par 
%\textsuperscript{(1647.4)}
\textsuperscript{147:1.2} Después de haberlos escuchado, Jesús les dijo: <<Iré con vosotros>>. Cuando llegó con ellos a la casa del centurión, y antes de que hubieran entrado en su patio, el soldado romano envió a sus amigos para que saludaran a Jesús, con la instrucción de decirle: <<Señor, no te molestes en entrar en mi casa, porque no soy digno de que vengas bajo mi techo. Tampoco me he considerado digno de ir a verte; por eso te he enviado a los ancianos de tu propio pueblo. Pero sé que puedes pronunciar la palabra allí mismo donde estás y que mi servidor se curará. Porque yo mismo estoy bajo las órdenes de otros, y tengo soldados a mis órdenes, y le digo a éste que vaya, y va; le digo a otro que venga, y viene, y a mis criados que hagan esto o aquello, y lo hacen>>\footnote{\textit{La fe del centurión}: Mt 8:7-9; Lc 7:6-8.}.

\par 
%\textsuperscript{(1648.1)}
\textsuperscript{147:1.3} Cuando Jesús oyó estas palabras, se volvió y dijo a sus apóstoles y a los que estaban con ellos: <<Me maravilla la creencia de este gentil. En verdad, en verdad os digo que no he encontrado una fe tan grande, no, ni siquiera en Israel>>\footnote{\textit{Jesús maravillado de la fe}: Lc 7:9-10.}. Jesús le dio la espalda a la casa, y dijo: <<Vámonos de aquí>>. Los amigos del centurión entraron en la casa y le contaron a Mangus lo que Jesús había dicho. A partir de aquel momento, el servidor empezó a mejorar y finalmente recuperó su salud y utilidad normales\footnote{\textit{La curación del sirviente}: Mt 8:13.}.

\par 
%\textsuperscript{(1648.2)}
\textsuperscript{147:1.4} Nunca hemos sabido exactamente qué es lo que sucedió en esta ocasión. Éste es simplemente el relato del suceso; en cuanto a si los seres invisibles contribuyeron o no a la curación del servidor del centurión, eso es algo que no se reveló a los que acompañaban a Jesús. Sólo conocemos el hecho de que el servidor se recuperó por completo.

\section*{2. El viaje a Jerusalén}
\par 
%\textsuperscript{(1648.3)}
\textsuperscript{147:2.1} El martes 30 de marzo, por la mañana temprano, Jesús y el grupo apostólico iniciaron su viaje a Jerusalén para la Pascua, tomando el camino del valle del Jordán. Llegaron el viernes 2 de abril por la tarde, y como de costumbre, establecieron su cuartel general en Betania. Al pasar por Jericó, se detuvieron para descansar mientras que Judas depositaba una parte de los fondos comunes en el banco de un amigo de su familia. Era la primera vez que Judas transportaba un excedente de dinero, y este depósito permaneció intacto hasta que pasaron de nuevo por Jericó durante el último viaje memorable a Jerusalén, poco antes del juicio y la muerte de Jesús.

\par 
%\textsuperscript{(1648.4)}
\textsuperscript{147:2.2} El grupo tuvo un viaje tranquilo hasta Jerusalén, pero apenas se habían instalado en Betania cuando empezaron a congregarse, de cerca y de lejos, personas que buscaban la curación para su cuerpo, el consuelo para su mente confusa y la salvación para su alma; eran tan numerosas que Jesús tuvo poco tiempo para descansar. Por esta razón, montaron las tiendas en Getsemaní, y el Maestro iba y venía de Betania a Getsemaní para evitar la multitud que lo asediaba constantemente. El grupo apostólico pasó casi tres semanas en Jerusalén, pero Jesús les ordenó que no predicaran en público, que se limitaran a la enseñanza en privado y al trabajo personal.

\par 
%\textsuperscript{(1648.5)}
\textsuperscript{147:2.3} Celebraron la Pascua tranquilamente en Betania. Era la primera vez que Jesús y la totalidad de los doce compartían la fiesta pascual sin derramamiento de sangre. Los apóstoles de Juan no comieron la Pascua con Jesús y sus apóstoles; celebraron la fiesta con Abner y muchos de los primeros creyentes en las predicaciones de Juan. Ésta era la segunda Pascua que Jesús celebraba con sus apóstoles en Jerusalén.

\par 
%\textsuperscript{(1648.6)}
\textsuperscript{147:2.4} Cuando Jesús y los doce partieron para Cafarnaúm, los apóstoles de Juan no regresaron con ellos. Se quedaron en Jerusalén y sus alrededores bajo la dirección de Abner, trabajando discretamente por la expansión del reino, mientras que Jesús y los doce regresaban para efectuar su labor en Galilea. Los veinticuatro nunca más volvieron a estar todos juntos hasta poco antes de que los setenta evangelistas recibieran su misión y su orden de partir. Pero los dos grupos cooperaban entre sí y prevalecían los mejores sentimientos, a pesar de sus diferencias de opinión.

\section*{3. En el estanque de Betesda}
\par 
%\textsuperscript{(1649.1)}
\textsuperscript{147:3.1} Durante la tarde del segundo sábado en Jerusalén, mientras el Maestro y los apóstoles estaban a punto de participar en los servicios del templo, Juan le dijo a Jesús: <<Ven conmigo, quisiera mostrarte algo>>. Juan llevó a Jesús por una de las puertas de Jerusalén hasta un estanque de agua llamado Betesda\footnote{\textit{El estanque de Betesda}: Jn 5:1-4.}. Alrededor de este estanque había una estructura de cinco pórticos, bajo los cuales permanecía un gran número de enfermos en busca de curación. Se trataba de un manantial caliente cuyas aguas rojizas burbujeaban a intervalos irregulares a causa de las acumulaciones de gases en las cavernas rocosas que se encontraban debajo del estanque. Muchos creían que esta perturbación periódica de las aguas calientes se debía a influencias sobrenaturales, y era creencia popular de que la primera persona que entrara en el agua después de una de estas perturbaciones se curaría de cualquier enfermedad que tuviera.

\par 
%\textsuperscript{(1649.2)}
\textsuperscript{147:3.2} Los apóstoles estaban un poco inquietos por las restricciones impuestas por Jesús, y Juan, el más joven de los doce, se sentía particularmente impaciente por esta prohibición. Había llevado a Jesús al estanque pensando que el espectáculo de los enfermos allí reunidos conmovería tanto la compasión del Maestro que lo incitaría a efectuar un milagro de curación, y así todo Jerusalén se quedaría asombrado y pronto se pondría a creer en el evangelio del reino. Juan le dijo a Jesús: <<Maestro, mira toda esta gente que sufre; ¿no hay nada que podamos hacer por ellos?>> Y Jesús replicó: <<Juan, ¿por qué me tientas para que me desvíe del camino que he escogido? ¿Por qué continúas deseando sustituir la proclamación del evangelio de la verdad eterna por la realización de prodigios y la curación de los enfermos? Hijo mío, no me está permitido hacer lo que deseas, pero reúne a esos enfermos y afligidos para que pueda dirigirles unas palabras de aliento y de consuelo eterno>>.

\par 
%\textsuperscript{(1649.3)}
\textsuperscript{147:3.3} Al dirigirse a los allí reunidos, Jesús les dijo: <<Muchos de vosotros estáis aquí, enfermos y afligidos, porque habéis vivido muchos años en el camino equivocado. Algunos sufren por los accidentes del tiempo, otros a consecuencia de los errores de sus antepasados, mientras que algunos de vosotros lucháis contra los obstáculos de las condiciones imperfectas de vuestra existencia temporal. Pero mi Padre trabaja, y yo quisiera trabajar, para mejorar vuestra condición en la Tierra, y más especialmente para asegurar vuestro estado eterno. Ninguno de nosotros puede hacer gran cosa por cambiar las dificultades de la vida, a menos que descubramos que el Padre que está en los cielos así lo quiere. Después de todo, todos estamos obligados a hacer la voluntad del Eterno. Si todos os pudierais curar de vuestras aflicciones físicas, indudablemente os admiraríais, pero es aun más importante que seáis purificados de toda enfermedad espiritual y que os encontréis curados de todas las dolencias morales. Todos sois hijos de Dios; sois los hijos del Padre celestial. Las trabas del tiempo pueden parecer afligiros, pero el Dios de la eternidad os ama. Cuando llegue la hora del juicio, no temáis, pues todos encontraréis no solamente justicia, sino una abundante misericordia. En verdad, en verdad os lo digo: Aquel que escucha el evangelio del reino y cree en esta enseñanza de la filiación con Dios, posee la vida eterna; esos creyentes pasan ya del juicio y de la muerte a la luz y a la vida. Y se acerca la hora en que incluso aquellos que están en la tumba escucharán la voz de la resurrección>>\footnote{\textit{Mi Padre trabaja y yo trabajo}: Jn 5:17. \textit{Quienes escuchan y creen tienen vida eterna}: Jn 5:24-28; 6:40.}.

\par 
%\textsuperscript{(1649.4)}
\textsuperscript{147:3.4} Muchos de los que lo escucharon creyeron en el evangelio del reino. Algunos de los afligidos se sintieron tan inspirados y revivificados espiritualmente, que anduvieron proclamando de acá para allá que también habían sido curados de sus dolencias físicas.

\par 
%\textsuperscript{(1649.5)}
\textsuperscript{147:3.5} Un hombre que había estado muchos años deprimido y gravemente afligido con las dolencias de su mente perturbada, se regocijó con las palabras de Jesús, recogió su lecho y salió hacia su casa, aunque era el día del sábado. Este hombre angustiado había esperado todos estos años que \textit{alguien} le ayudara; era tan víctima del sentimiento de su propia impotencia que ni una sola vez había concebido la idea de ayudarse a sí mismo, aunque ésta era la única cosa que tenía que hacer para recuperarse ---recoger su lecho y salir caminando\footnote{\textit{Recoger su lecho y caminar}: Jn 5:5-9.}.

\par 
%\textsuperscript{(1650.1)}
\textsuperscript{147:3.6} Jesús le dijo entonces a Juan: <<Vámonos de aquí antes de que los principales sacerdotes y los escribas se encuentren con nosotros y se ofendan porque hemos dirigido unas palabras de vida a estos afligidos>>\footnote{\textit{Temores de los sacerdotes y escribas}: Jn 5:10-15.}. Volvieron al templo para reunirse con sus compañeros, y todos partieron enseguida para pasar la noche en Betania. Juan nunca contó a los otros apóstoles la visita que había hecho con Jesús, este sábado por la tarde, al estanque de Betesda.

\section*{4. La regla de vida}
\par 
%\textsuperscript{(1650.2)}
\textsuperscript{147:4.1} Al anochecer de este mismo sábado, en Betania, mientras que Jesús, los doce y un grupo de creyentes estaban reunidos alrededor del fuego en el jardín de Lázaro, Natanael le hizo a Jesús la pregunta siguiente: <<Maestro, aunque nos has enseñado la versión positiva de la antigua regla de vida, indicándonos que deberíamos hacer a los demás lo que deseamos que nos hagan a nosotros, no discierno plenamente cómo podremos obrar siempre de acuerdo con este mandato. Permíteme ilustrar mi opinión citando el ejemplo de un hombre lascivo que mira con inmoralidad a su futura compañera de pecado. ¿Cómo podemos enseñar que este hombre malintencionado debería hacer a los demás lo que quisiera que le hicieran a él?>>\footnote{\textit{La regla de oro}: Mt 7:12; Lc 6:31. \textit{La regla de oro (negativa)}: Tb 4:15.}

\par 
%\textsuperscript{(1650.3)}
\textsuperscript{147:4.2} Cuando Jesús escuchó la pregunta de Natanael, se puso inmediatamente de pie, señaló al apóstol con el dedo, y dijo: <<¡Natanael, Natanael! ¿Qué tipo de pensamientos mantienes en tu corazón? ¿No recibes mis enseñanzas como alguien que ha nacido del espíritu? ¿No escucháis la verdad como hombres con sabiduría y comprensión espiritual? Cuando os recomendé que hicierais por los demás lo que quisierais que hicieran por vosotros, me dirigía a unos hombres con ideales elevados, y no a unos que sentirían la tentación de tergiversar mi enseñanza, convirtiéndola en una licencia para estimular las malas acciones>>.

\par 
%\textsuperscript{(1650.4)}
\textsuperscript{147:4.3} Cuando el Maestro hubo hablado, Natanael se levantó y dijo: <<Pero Maestro, no deberías pensar que apruebo semejante interpretación de tu enseñanza. He hecho esta pregunta porque he supuesto que muchos hombres de este tipo podrían juzgar mal tus recomendaciones, y esperaba que nos darías una enseñanza adicional sobre estas cuestiones>>. Una vez que Natanael se hubo sentado, Jesús continuó hablando: <<Sé bien, Natanael, que tu mente no aprueba ninguna idea de maldad de este tipo, pero me desilusiona que todos vosotros olvidéis con tanta frecuencia darle una interpretación auténticamente espiritual a mis enseñanzas corrientes, a unas instrucciones que debo daros en lenguaje humano y a la manera en que hablan los hombres. Permitidme ahora que os enseñe sobre los diversos niveles de significado ligados a la interpretación de esta regla de vida, a esta recomendación de `hacer por los demás lo que deseáis que ellos hagan por vosotros':>>

\par 
%\textsuperscript{(1650.5)}
\textsuperscript{147:4.4} <<1. \textit{El nivel de la carne}. Esta interpretación puramente egoísta y lasciva tendría un buen ejemplo en la hipótesis de tu pregunta>>.

\par 
%\textsuperscript{(1650.6)}
\textsuperscript{147:4.5} <<2. \textit{El nivel de los sentimientos}. Este plano se encuentra un nivel por encima del de la carne, e implica que la compasión y la piedad realzan nuestra interpretación de esta regla de vida>>.

\par 
%\textsuperscript{(1650.7)}
\textsuperscript{147:4.6} <<3. \textit{El nivel de la mente}. Ahora entran en acción la razón de la mente y la inteligencia de la experiencia. El buen juicio dicta que esta regla de vida debería ser interpretada en consonancia con el idealismo más elevado, incorporado en la nobleza de un profundo respeto de sí mismo>>.

\par 
%\textsuperscript{(1651.1)}
\textsuperscript{147:4.7} <<4. \textit{El nivel del amor fraternal}. Aún más arriba se descubre el nivel de la consagración desinteresada al bienestar de nuestros semejantes. En este plano más elevado del servicio social entusiasta, que nace de la conciencia de la paternidad de Dios y del reconocimiento consiguiente de la fraternidad de los hombres, se descubre una interpretación nueva y mucho más hermosa de esta regla de vida fundamental>>.

\par 
%\textsuperscript{(1651.2)}
\textsuperscript{147:4.8} <<5. \textit{El nivel moral}. Luego, cuando alcancéis unos verdaderos niveles filosóficos de interpretación, cuando tengáis una verdadera comprensión de la \textit{rectitud} y de la \textit{maldad} en los acontecimientos, cuando percibáis la idoneidad eterna de las relaciones humanas, empezaréis a considerar este problema de interpretación como imaginaríais que una tercera persona de pensamientos elevados, idealista, sabia e imparcial consideraría e interpretaría este mandato, pero aplicado a vuestros problemas personales de adaptación a los incidentes de vuestra vida>>.

\par 
%\textsuperscript{(1651.3)}
\textsuperscript{147:4.9} <<6. \textit{El nivel espiritual}. En último lugar alcanzamos el nivel de la perspicacia del espíritu y de la interpretación espiritual, el nivel más elevado de todos, que nos impulsa a reconocer en esta regla de vida el mandamiento divino de tratar a todos los hombres como concebimos que Dios los trataría. Éste es el ideal universal de las relaciones humanas, y ésta es vuestra actitud ante todos estos problemas cuando vuestro deseo supremo es hacer siempre la voluntad del Padre. Quisiera pues que hicierais por todos los hombres lo que sabéis que yo haría por ellos en circunstancias semejantes>>.

\par 
%\textsuperscript{(1651.4)}
\textsuperscript{147:4.10} Nada de lo que Jesús había dicho a los apóstoles hasta ese momento les había impresionado tanto. Continuaron discutiendo las palabras del Maestro hasta mucho después de que éste se hubiera retirado. Aunque Natanael tardó en recobrarse de la hipótesis de que Jesús no había interpretado bien el espíritu de su pregunta, los demás estaban más que agradecidos a su colega filosófico por haber tenido el valor de hacer una pregunta que incitaba tanto a la reflexión.

\section*{5. La visita a Simón el fariseo}
\par 
%\textsuperscript{(1651.5)}
\textsuperscript{147:5.1} Aunque Simón no era un miembro del sanedrín judío, era un fariseo influyente de Jerusalén\footnote{\textit{Simón el fariseo}: Lc 7:36a.}. Era un creyente poco entusiasta, y a pesar de que podría ser criticado severamente por ello, se atrevió a invitar a Jesús y a sus asociados personales Pedro, Santiago y Juan a un banquete en su casa. Simón había observado al Maestro desde hacía mucho tiempo; estaba muy impresionado por sus enseñanzas y aun más por su personalidad.

\par 
%\textsuperscript{(1651.6)}
\textsuperscript{147:5.2} Los fariseos ricos eran adictos a dar limosnas, y no evitaban la publicidad relacionada con su filantropía. A veces incluso hubieran tocado las trompetas cuando se disponían a dispensar la caridad a un mendigo. Cuando estos fariseos ofrecían un banquete a unos invitados distinguidos, tenían la costumbre de dejar abiertas las puertas de la casa para que incluso los mendigos de la calle pudieran entrar; éstos permanecían de pie junto a las paredes de la sala, detrás de los lechos de los convidados, para estar en condiciones de recibir los trozos de comida que los comensales pudieran lanzarles.

\par 
%\textsuperscript{(1651.7)}
\textsuperscript{147:5.3} En esta ocasión particular, en la casa de Simón, entre la gente que entraba de la calle había una mujer de mala reputación que recientemente se había vuelto creyente en la buena nueva del evangelio del reino. Esta mujer era bien conocida en todo Jerusalén como la antigua dueña de un burdel considerado de alta categoría, situado muy cerca del patio de los gentiles del templo. Al aceptar las enseñanzas de Jesús, había cerrado su abominable negocio, y había persuadido a la mayoría de las mujeres que estaban asociadas con ella a que aceptaran el evangelio y cambiaran su forma de vida. A pesar de esto, los fariseos continuaban despreciándola mucho y estaba obligada a llevar el cabello suelto ---el distintivo de la prostitución. Esta mujer anónima había traído consigo un gran frasco de loción perfumada para ungir; permanecía de pie detrás de Jesús, que estaba recostado para comer, y empezó a ungirle los pies, al mismo tiempo que se los mojaba con sus lágrimas de gratitud, secándoselos con sus cabellos\footnote{\textit{Jesús ungido por una antigua prostituta}: Lc 7:36b-38.}. Cuando hubo terminado la unción, continuó llorando y besándole los pies.

\par 
%\textsuperscript{(1652.1)}
\textsuperscript{147:5.4} Cuando Simón vio todo esto, se dijo para sus adentros: <<Si este hombre fuera un profeta, hubiera percibido quién lo está tocando así y de qué tipo de mujer se trata; de una pecadora de mala fama>>. Sabiendo lo que pasaba por la mente de Simón, Jesús tomó la palabra y dijo: <<Simón, hay algo que me gustaría decirte>>. Simón respondió: <<Maestro, dilo>>. Entonces Jesús dijo: <<Un rico prestamista tenía dos deudores. Uno le debía quinientos denarios y el otro cincuenta. Entonces, como ninguno de ellos tenía con qué pagarle, les perdonó la deuda a los dos. Según tú, Simón, ¿cuál de ellos lo amará más?>> Simón contestó: <<Supongo que aquel a quien más le perdonó>>\footnote{\textit{Parábola de los dos deudores}: Lc 7:39-48.}. Y Jesús le dijo: <<Has juzgado bien>>, y señalando a la mujer, continuó: <<Simón, mira bien a esta mujer. He entrado en tu casa como invitado, y sin embargo no me has dado agua para mis pies. Esta mujer agradecida me ha lavado los pies con sus lágrimas y los ha secado con sus cabellos. No me has dado un beso amistoso de bienvenida, pero esta mujer, desde que entró, no ha dejado de besarme los pies. Has olvidado ungirme la cabeza con aceite, pero ella ha ungido mis pies con lociones costosas. ¿Cuál es el significado de todo esto? Simplemente que sus numerosos pecados le han sido perdonados, y esto la ha llevado a amar mucho. Pero los que sólo han recibido un poco de perdón a veces sólo aman un poco>>. Jesús se volvió hacia la mujer, la cogió de la mano, la levantó y le dijo: <<En verdad te has arrepentido de tus pecados, y están perdonados. No te desanimes por la actitud irreflexiva y severa de tus semejantes; continúa tu camino en la alegría y la libertad del reino de los cielos>>.

\par 
%\textsuperscript{(1652.2)}
\textsuperscript{147:5.5} Cuando Simón y sus amigos que estaban sentados comiendo con él escucharon estas palabras, se quedaron más que sorprendidos y empezaron a cuchichear entre ellos: <<¿Quién es este hombre que se atreve incluso a perdonar los pecados?>> Cuando Jesús los escuchó murmurar así, se volvió para despedir a la mujer, diciendo: <<Mujer, vete en paz; tu fe te ha salvado>>\footnote{\textit{Fe y perdón de los pecados}: Lc 7:49-50.}.

\par 
%\textsuperscript{(1652.3)}
\textsuperscript{147:5.6} Cuando Jesús se levantó con sus amigos para irse, se volvió hacia Simón y le dijo: <<Conozco tu corazón, Simón. Sé cómo estás desgarrado entre la fe y la duda, cómo estás desconcertado por el miedo y confundido por el orgullo; pero ruego por ti, para que te abandones a la luz y puedas experimentar en tu situación en la vida esas poderosas transformaciones de mente y de espíritu comparables a los cambios enormes que el evangelio del reino ya ha producido en el corazón de tu visitante no invitada ni bienvenida. Os declaro a todos que el Padre ha abierto las puertas del reino celestial a todos los que tienen la fe necesaria para entrar, y ningún hombre o asociación de hombres podrán cerrar esas puertas ni siquiera al alma más humilde o al pecador supuestamente más flagrante de la Tierra, si sinceramente aspiran a entrar>>\footnote{\textit{La puerta abierta para entrar por la fe}: Ap 3:8.}. Y Jesús, con Pedro, Santiago y Juan, se despidieron de su anfitrión y fueron a reunirse con el resto de los apóstoles en el campamento del jardín de Getsemaní.

\par 
%\textsuperscript{(1653.1)}
\textsuperscript{147:5.7} Aquella misma noche, Jesús dio a los apóstoles el inolvidable discurso sobre el valor relativo del estatus ante Dios y del progreso en la ascensión eterna hacia el Paraíso. Jesús dijo: <<Hijos míos, si existe una verdadera conexión viviente entre el hijo y el Padre, el hijo está seguro de progresar continuamente hacia los ideales del Padre. Es verdad que al principio el hijo puede progresar lentamente, pero su progreso no es por ello menos seguro. Lo importante no es la rapidez de vuestro progreso, sino su certidumbre. Vuestros logros actuales no son tan importantes como el hecho de que la \textit{dirección} de vuestro progreso es hacia Dios. Aquello en lo que os estáis convirtiendo, día tras día, tiene infinitamente más importancia que lo que sois hoy>>.

\par 
%\textsuperscript{(1653.2)}
\textsuperscript{147:5.8} <<Esta mujer transformada, que algunos de vosotros habéis visto hoy en la casa de Simón, vive en este momento en un nivel muy inferior al de Simón y sus asociados bien intencionados. Pero estos fariseos están ocupados en el falso progreso de la ilusión de atravesar los círculos engañosos de los servicios ceremoniales sin sentido, mientras que esta mujer ha empezado, con una seriedad total, la larga y extraordinaria búsqueda de Dios; y su camino hacia el cielo no está bloqueado por el orgullo espiritual ni por la satisfacción moral de sí misma. Humanamente hablando, esta mujer está mucho más lejos de Dios que Simón, pero su alma sigue un movimiento progresivo; esta mujer está en camino hacia una meta eterna. En esta mujer están presentes unas enormes posibilidades espirituales para el futuro. Algunos de vosotros pueden no encontrarse en unos niveles realmente elevados de alma y de espíritu, pero estáis efectuando progresos diarios hacia Dios en el camino viviente que vuestra fe ha abierto. En cada uno de vosotros existen unas enormes posibilidades para el futuro. Es mucho mejor tener una fe limitada, pero viva y creciente, que poseer un gran intelecto con sus depósitos muertos de sabiduría mundana y de incredulidad espiritual>>\footnote{\textit{Jesús, el camino viviente}: Jn 14:6; Heb 10:20.}.

\par 
%\textsuperscript{(1653.3)}
\textsuperscript{147:5.9} Jesús previno seriamente a sus apóstoles contra la necedad del hijo de Dios que abusa del amor del Padre. Declaró que el Padre celestial no es un padre descuidado, negligente o tontamente indulgente, que siempre está dispuesto a indultar el pecado y a perdonar la imprudencia. Advirtió a sus oyentes que no aplicaran erróneamente sus ejemplos del padre y el hijo de manera que pudiera parecer que Dios es como uno de esos padres demasiado indulgentes y nada sabios, que conspiran con la necedad de la Tierra para provocar la ruina moral de sus hijos irreflexivos, contribuyendo así de manera cierta y directa a la delincuencia y a la pronta corrupción de sus propios descendientes. Jesús dijo: <<Mi Padre no aprueba con indulgencia los actos y las prácticas de sus hijos que conducen a la destrucción y a la ruina de todo crecimiento moral y de todo progreso espiritual. Esas prácticas pecaminosas son una abominación a los ojos de Dios>>.

\par 
%\textsuperscript{(1653.4)}
\textsuperscript{147:5.10} Jesús asistió a otras muchas reuniones y banquetes semiprivados con los grandes y los humildes, los ricos y los pobres de Jerusalén, antes de partir finalmente con sus apóstoles hacia Cafarnaúm. Muchos, en verdad, se hicieron creyentes en el evangelio del reino y fueron bautizados posteriormente por Abner y sus asociados, que se quedaron atrás para fomentar los intereses del reino en Jerusalén y sus alrededores.

\section*{6. El regreso a Cafarnaúm}
\par 
%\textsuperscript{(1653.5)}
\textsuperscript{147:6.1} La última semana de abril, Jesús y los doce salieron de su cuartel general de Betania cerca de Jerusalén, y emprendieron su viaje de regreso a Cafarnaúm por el camino de Jericó y el Jordán.

\par 
%\textsuperscript{(1654.1)}
\textsuperscript{147:6.2} Los sacerdotes principales y los jefes religiosos de los judíos tuvieron muchas reuniones secretas con el fin de decidir qué iban a hacer con Jesús. Todos estaban de acuerdo en que había que hacer algo para poner fin a su enseñanza, pero no se ponían de acuerdo en el método a emplear. Habían tenido la esperanza de que las autoridades civiles dispondrían de él como Herodes había puesto fin a la carrera de Juan, pero descubrieron que Jesús llevaba su actividad de tal manera que los funcionarios romanos no estaban muy alarmados por sus predicaciones. En consecuencia, en una reunión celebrada el día antes de la partida de Jesús para Cafarnaúm, decidieron que tenía que ser capturado bajo la acusación de un delito religioso, y ser juzgado por el sanedrín. Por esta razón, nombraron una comisión de seis espías secretos para que siguieran a Jesús y observaran sus palabras y sus actos; cuando hubieran acumulado suficientes pruebas de infracciones a la ley y de blasfemias, tenían que regresar con su informe a Jerusalén. Estos seis judíos alcanzaron en Jericó al grupo apostólico, que constaba de unos treinta miembros, y con el pretexto de que deseaban convertirse en discípulos, se unieron a la familia de seguidores de Jesús, permaneciendo con el grupo hasta el comienzo de la segunda gira de predicación en Galilea. En ese momento, tres de ellos volvieron a Jerusalén para presentar su informe a los principales sacerdotes y al sanedrín.

\par 
%\textsuperscript{(1654.2)}
\textsuperscript{147:6.3} Pedro predicó a la multitud reunida en el vado del Jordán, y a la mañana siguiente se dirigieron río arriba hacia Amatus. Querían continuar directamente hasta Cafarnaúm, pero se había congregado tanta gente que se quedaron tres días, predicando, enseñando y bautizando. No se marcharon para casa hasta el sábado por la mañana temprano, primer día de mayo. Los espías de Jerusalén estaban seguros de que ahora podrían obtener la primera acusación contra Jesús ---la de violar el sábado--- puesto que se había atrevido a emprender su viaje el día del sábado. Pero iban a sufrir una desilusión porque, justo antes de partir, Jesús llamó a Andrés y le dio instrucciones, delante de todos ellos, para que sólo avanzaran unos mil metros, la distancia legal que los judíos podían recorrer el día del sábado.

\par 
%\textsuperscript{(1654.3)}
\textsuperscript{147:6.4} Pero los espías no tuvieron que esperar mucho para tener la oportunidad de acusar a Jesús y a sus compañeros de violar el sábado. Al pasar el grupo por un camino estrecho, a ambos lados y al alcance de la mano se encontraba el trigo ondulante, que en esa época estaba madurando; como algunos de los apóstoles tenían hambre, arrancaron el grano maduro y se lo comieron. Entre los viajeros existía la costumbre de servirse grano mientras pasaban por la carretera, y por esta razón no se atribuía ninguna idea de maldad a esta conducta. Pero los espías cogieron esto como pretexto para atacar a Jesús. Cuando vieron a Andrés restregando el grano en su mano, se acercaron y le dijeron: <<¿No sabes que es ilegal arrancar y restregar el grano el día del sábado?>>\footnote{\textit{Los apóstoles trituran grano en sábado}: Mt 12:1-2; Mc 2:23-24; Lc 6:1-2.} Andrés respondió: <<Pero tenemos hambre y sólo restregamos la cantidad suficiente para nuestras necesidades; ¿desde cuándo es un pecado comer grano el día del sábado?>> Pero los fariseos replicaron: <<No haces mal en comerlo, pero violas la ley al arrancar y restregar el grano entre tus manos; tu Maestro seguramente no aprobaría esa conducta>>. Entonces, Andrés dijo: <<Si no es malo comerse el grano, seguramente restregarlo entre nuestras manos no es mucho más trabajo que masticarlo, cosa que permitís; ¿por qué hacéis un problema por estas nimiedades?>> Cuando Andrés insinuó que eran unos sofistas, se indignaron y se precipitaron hacia Jesús, que caminaba detrás charlando con Mateo, y protestaron diciendo: <<Mira, Maestro, tus apóstoles hacen lo que es ilegal el día del sábado; arrancan, restriegan y se comen el grano. Estamos seguros de que les vas a ordenar que dejen de hacerlo>>. Jesús dijo entonces a los acusadores: <<En verdad sois celosos de la ley, y hacéis bien en recordar el sábado para santificarlo. Pero ¿no habéis leído nunca en las Escrituras que un día que David tenía hambre entró en la casa de Dios con sus compañeros, y se comieron el pan de la proposición, que nadie estaba autorizado a comer excepto los sacerdotes? Y David también dio de este pan a los que estaban con él. ¿Y no habéis leído en nuestra ley que es legal hacer muchas cosas necesarias el sábado? ¿Y no voy a veros comer, antes de que termine el día, lo que habéis traído para vuestras necesidades de hoy? Amigos míos, hacéis bien en defender el sábado, pero haríais mejor en proteger la salud y el bienestar de vuestros semejantes. Afirmo que el sábado ha sido hecho para el hombre, y no el hombre para el sábado. Y si estáis aquí con nosotros para vigilar mis palabras, entonces proclamaré abiertamente que el Hijo del Hombre es dueño incluso del sábado>>\footnote{\textit{David comió pan de la proposición}: 1 Sam 21:3-6. \textit{Jesús señor del sábado}: Mt 12:3-8; Mc 2:25-28; Lc 6:3-5.}.

\par 
%\textsuperscript{(1655.1)}
\textsuperscript{147:6.5} Los fariseos se quedaron asombrados y confundidos ante sus palabras de discernimiento y de sabiduría. Durante el resto del día se mantuvieron apartados y no se atrevieron a hacer más preguntas.

\par 
%\textsuperscript{(1655.2)}
\textsuperscript{147:6.6} El antagonismo de Jesús hacia las tradiciones judías y los ceremoniales serviles era siempre \textit{positivo}. Consistía en lo que él hacía y afirmaba. El Maestro pasaba poco tiempo haciendo denuncias negativas. Enseñaba que los que conocen a Dios pueden gozar de la libertad de vivir sin engañarse a sí mismos con los desenfrenos del pecado. Jesús dijo a sus apóstoles: <<Amigos, si estáis iluminados por la verdad y si sabéis realmente lo que hacéis, sois bienaventurados; pero si no conocéis el camino divino, sois desgraciados y ya quebrantáis la ley>>.

\section*{7. De regreso en Cafarnaúm}
\par 
%\textsuperscript{(1655.3)}
\textsuperscript{147:7.1} El lunes 3 de mayo, alrededor del mediodía, Jesús y los doce llegaron en barco a Betsaida, procedentes de Tariquea. Viajaron en barco para eludir a los que los acompañaban. Pero al día siguiente, todos ellos, incluyendo a los espías oficiales de Jerusalén, habían encontrado de nuevo a Jesús.

\par 
%\textsuperscript{(1655.4)}
\textsuperscript{147:7.2} El martes por la tarde, Jesús estaba dirigiendo una de sus clases habituales de preguntas y respuestas, cuando el jefe de los seis espías le dijo: <<Hoy estaba hablando con uno de los discípulos de Juan, que está aquí asistiendo a tu enseñanza, y no acertábamos a comprender por qué nunca ordenas a tus discípulos que ayunen y recen, como nosotros los fariseos ayunamos, y como Juan lo mandó a sus discípulos>>. Refiriéndose a una declaración de Juan, Jesús respondió a este interrogador: \guillemotleft¿Acaso ayunan los pajes de honor cuando el novio está con ellos? Mientras el novio permanece con ellos, difícilmente pueden ayunar. Pero se acerca la hora en que el novio será apartado de allí, y entonces los pajes de honor ayunarán y orarán indudablemente. La oración es algo natural para los hijos de la luz\footnote{\textit{Hijos de la luz}: Lc 16:8; Jn 12:36; Ef 5:8; 1 Ts 5:5.}, pero el ayuno no forma parte del evangelio del reino de los cielos. Recordad que un sastre sabio no cose un trozo de tela nueva y sin encoger en un vestido viejo, por temor a que cuando se moje, encoja, y produzca un desgarrón aún mayor. Los hombres tampoco ponen el vino nuevo en odres viejos, para que el vino nuevo no reviente los odres y se pierdan tanto el vino como los odres. El hombre sabio pone el vino nuevo en odres nuevos. Por eso mis discípulos muestran sabiduría al no incorporar demasiadas cosas del viejo orden en la nueva enseñanza del evangelio del reino. Vosotros, que habéis perdido a vuestro instructor, podéis estar justificados si ayunáis durante un tiempo. El ayuno\footnote{\textit{Discurso sobre el ayuno}: Mt 9:14-17; Mc 2:18-22; Lc 5:33-38.} puede ser una parte apropiada de la ley de Moisés, pero en el reino venidero, los hijos de Dios estarán liberados del miedo y experimentarán la alegría en el espíritu divino\guillemotright. Cuando escucharon estas palabras, los discípulos de Juan se sintieron confortados mientras que los fariseos, por su parte, se quedaron aún más confundidos.

\par 
%\textsuperscript{(1656.1)}
\textsuperscript{147:7.3} El Maestro procedió entonces a prevenir a sus oyentes contra el mantenimiento de la idea de que todas las antiguas enseñanzas tenían que ser totalmente reemplazadas por las nuevas doctrinas. Jesús dijo: <<Lo que es antiguo, pero también \textit{verdadero}, debe permanecer. De la misma manera, lo que es nuevo, pero falso, debe ser rechazado. Tened la fe y el valor de aceptar lo que es nuevo y también verdadero. Recordad que está escrito: `No abandonéis a un viejo amigo, porque el nuevo no es comparable con él. Un amigo nuevo es como el vino nuevo; si se vuelve viejo, lo beberéis con alegría'.>>\footnote{\textit{El viejo y el nuevo amigo}: Pr 27:10. \textit{El vino viejo y el nuevo}: Lc 5:39.}

\section*{8. La fiesta de la bondad espiritual}
\par 
%\textsuperscript{(1656.2)}
\textsuperscript{147:8.1} Aquella noche, mucho después de que los oyentes habituales se hubieran retirado, Jesús continuó enseñando a sus apóstoles. Empezó esta lección especial citando al profeta Isaías:

\par 
%\textsuperscript{(1656.3)}
\textsuperscript{147:8.2} <<`¿Por qué habéis ayunado? ¿Por qué razón afligís vuestras almas mientras que continuáis encontrando placer en la opresión y deleitándoos con la injusticia? He aquí que ayunáis por amor a la contienda y a la disputa, y para golpear con el puño de la maldad. Pero ayunando de esta manera no haréis oír vuestras voces en el cielo>>\footnote{\textit{¿Por qué habéis ayunado?}: Is 58:3-4.}.

\par 
%\textsuperscript{(1656.4)}
\textsuperscript{147:8.3} <<`¿Es éste el ayuno que he elegido ---un día para que el hombre aflija su alma? ¿Es para que incline la cabeza como un junco, para que se arrastre vestido de penitente? ¿Os atreveréis a decir que esto es un ayuno y un día aceptable a los ojos del Señor? ¿No es éste el ayuno que yo escogería: desatar las cadenas de la maldad, deshacer los nudos de las cargas pesadas, dejar libres a los oprimidos y romper todos los yugos? ¿No es compartir mi pan con el hambriento y traer a mi casa a los pobres sin hogar? Y cuando vea a los que están desnudos, los vestiré>>\footnote{\textit{¿Por qué no hacer el bien?}: Is 58:5-7.}.

\par 
%\textsuperscript{(1656.5)}
\textsuperscript{147:8.4} <<`Entonces vuestra luz brotará como la mañana y vuestra salud crecerá con rapidez. Vuestra rectitud os precederá, mientras que la gloria del Señor será vuestra retaguardia. Entonces invocaréis al Señor y él os responderá; gritaréis con fuerza y él dirá: Aquí estoy. Hará todo esto si dejáis de oprimir, de condenar y de mostrar vanidad. El Padre desea más bien que extendáis vuestro corazón a los hambrientos y que ayudéis a las almas afligidas; entonces vuestra luz brillará en las tinieblas, e incluso vuestra obscuridad será como el mediodía. Entonces el Señor os guiará contínuamente, satisfaciendo vuestra alma y renovando vuestra fortaleza. Os volveréis como un jardín regado, como un manantial cuyas aguas no se agotan. Los que hacen estas cosas restablecerán las glorias perdidas; levantarán los cimientos de muchas generaciones; serán llamados los reconstructores de los muros rotos, los restauradores de los caminos seguros por los que se puede transitar'>>\footnote{\textit{Vuestra luz brotará}: Is 58:8-12.}.

\par 
%\textsuperscript{(1656.6)}
\textsuperscript{147:8.5} Luego, hasta muy entrada la noche, Jesús expuso a sus apóstoles la verdad de que era su fe la que les daba seguridad en el reino del presente y del futuro, y no la aflicción de su alma ni el ayuno del cuerpo. Exhortó a los apóstoles a que vivieran al menos a la altura de las ideas del profeta de antaño, y expresó la esperanza de que progresarían mucho, incluso más allá de los ideales de Isaías y de los antiguos profetas. Las últimas palabras que pronunció aquella noche fueron: <<Creced en la gracia por medio de esa fe viviente que capta el hecho de que sois hijos de Dios, y al mismo tiempo reconoce a cada hombre como un hermano>>\footnote{\textit{Creced en la gracia}: 2 P 3:18.}.

\par 
%\textsuperscript{(1656.7)}
\textsuperscript{147:8.6} Eran más de las dos de la madrugada cuando Jesús dejó de hablar, y cada cual se retiró a descansar.


\chapter{Documento 148. La preparación de los evangelistas en Betsaida}
\par 
%\textsuperscript{(1657.1)}
\textsuperscript{148:0.1} DESDE el 3 de mayo hasta el 3 de octubre del año 28, Jesús y el cuerpo apostólico estuvieron residiendo en la casa de Zebedeo en Betsaida. Durante todo este período de cinco meses de la estación seca, un enorme campamento se mantuvo al lado del mar, cerca de la residencia de Zebedeo, la cual había sido considerablemente ampliada para alojar a la familia creciente de Jesús. Este campamento junto a la playa, que contaba entre quinientas y mil quinientas personas, estuvo ocupado por una población en constante cambio de buscadores de la verdad, de candidatos a la curación y de adictos a la curiosidad. Esta ciudad cubierta de tiendas estaba bajo la supervisión general de David Zebedeo, asistido por los gemelos Alfeo. El campamento era un modelo de orden y de higiene, así como de administración general. Los enfermos de diversos tipos estaban separados y bajo la supervisión de un médico creyente, un sirio llamado Elman.

\par 
%\textsuperscript{(1657.2)}
\textsuperscript{148:0.2} Durante todo este período, los apóstoles iban a pescar al menos un día por semana, y vendían sus capturas a David para su consumo en el campamento al lado del mar. Los fondos que se obtenían así eran entregados al tesorero del grupo. Los doce tenían permiso para pasar una semana cada mes con sus familiares o amigos.

\par 
%\textsuperscript{(1657.3)}
\textsuperscript{148:0.3} Aunque Andrés continuaba con la responsabilidad general de las actividades apostólicas, Pedro tenía enteramente a su cargo la escuela de los evangelistas. Cada mañana, todos los apóstoles contribuían a enseñar a los grupos de evangelistas, y por la tarde, tanto los instructores como los alumnos enseñaban a la gente. Después de la cena, cinco noches por semana, los apóstoles dirigían unas clases de preguntas y respuestas en beneficio de los evangelistas. Una vez por semana, Jesús presidía estas clases y contestaba las preguntas que habían quedado pendientes en las sesiones anteriores.

\par 
%\textsuperscript{(1657.4)}
\textsuperscript{148:0.4} En cinco meses, varios miles de personas pasaron por este campamento. Se veía con frecuencia a personas interesadas procedentes de todos los rincones del Imperio Romano y de los países situados al este del Éufrates. Éste fue el período estable y bien organizado más prolongado de la enseñanza del Maestro. La familia directa de Jesús pasó la mayor parte de este tiempo en Nazaret o en Caná.

\par 
%\textsuperscript{(1657.5)}
\textsuperscript{148:0.5} El campamento no estaba dirigido como una colectividad de intereses comunes, a la manera de la familia apostólica. David Zebedeo gobernó esta gran ciudad de tiendas de tal manera que se convirtió en una empresa capaz de autoabastecerse, aunque nunca se rechazó a nadie. Este campamento en constante cambio fue un aspecto indispensable de la escuela de instrucción evangélica de Pedro.

\section*{1. Una nueva escuela de profetas}
\par 
%\textsuperscript{(1657.6)}
\textsuperscript{148:1.1} Pedro, Santiago y Andrés formaban el comité nombrado por Jesús para evaluar a los aspirantes que deseaban ingresar en la escuela de evangelistas. Todas las razas y nacionalidades del mundo romano y de oriente, hasta la India incluida, estaban representadas entre los estudiantes de esta nueva escuela de profetas. El método de esta escuela consistía en aprender y en practicar. Aquello que los estudiantes aprendían por la mañana, lo enseñaban a la asamblea por la tarde al lado del mar. Después de la cena, discutían libremente tanto de lo aprendido por la mañana como de lo que habían enseñado por la tarde.

\par 
%\textsuperscript{(1658.1)}
\textsuperscript{148:1.2} Cada instructor apostólico enseñaba su propio punto de vista sobre el evangelio del reino. No se esforzaban por enseñar exactamente de la misma manera; no existía ninguna formulación uniforme o dogmática de las doctrinas teológicas. Aunque todos enseñaban la \textit{misma verdad}, cada apóstol presentaba su propia interpretación personal de las enseñanzas del Maestro. Jesús apoyaba esta presentación de la diversidad de experiencias personales en las cosas del reino; durante la sesión semanal de preguntas, armonizaba y coordinaba infaliblemente estos numerosos puntos de vista divergentes del evangelio. A pesar de este alto grado de libertad personal en materia de enseñanza, Simón Pedro tendía a dominar la teología de la escuela evangelista. Después de Pedro, Santiago Zebedeo era quien ejercía la mayor influencia personal.

\par 
%\textsuperscript{(1658.2)}
\textsuperscript{148:1.3} Los más de cien evangelistas instruídos durante estos cinco meses al lado del mar representaron el material del que se obtuvieron más tarde (a excepción de Abner y de los apóstoles de Juan) los setenta instructores y predicadores del evangelio. La escuela de evangelistas no lo poseía todo en común al mismo nivel que los doce.

\par 
%\textsuperscript{(1658.3)}
\textsuperscript{148:1.4} Estos evangelistas enseñaron y predicaron el evangelio, pero no bautizaron a los creyentes hasta que posteriormente Jesús los ordenó y les dio la misión de ser los setenta mensajeros del reino. Del gran número de personas que habían sido curadas en este lugar durante el incidente a la puesta del Sol, únicamente siete llegaron a contarse entre estos estudiantes evangelistas. El hijo del noble de Cafarnaúm fue uno de los que fueron preparados para el servicio evangélico en la escuela de Pedro.

\section*{2. El hospital de Betsaida}
\par 
%\textsuperscript{(1658.4)}
\textsuperscript{148:2.1} En conexión con el campamento al lado del mar, Elman, el médico sirio, con la ayuda de un grupo de veinticinco mujeres jóvenes y doce hombres, organizó y dirigió durante cuatro meses lo que se puede considerar como el primer hospital del reino. En esta enfermería, situada a corta distancia al sur de la principal ciudad de tiendas, trataron a los enfermos según todos los métodos materiales conocidos, así como también por medio de las prácticas espirituales de la oración y el estímulo de la fe. Jesús visitaba a los enfermos de este campamento al menos tres veces por semana, y se ponía en contacto personal con cada uno de ellos. Según lo que sabemos, no se produjo ningún pretendido milagro de curación sobrenatural entre las mil personas afligidas y doloridas que salieron mejoradas o curadas de esta enfermería. Sin embargo, la gran mayoría de estas personas beneficiadas no dejó de proclamar que Jesús las había curado.

\par 
%\textsuperscript{(1658.5)}
\textsuperscript{148:2.2} Muchas de las curas efectuadas por Jesús en conexión con su ministerio a favor de los pacientes de Elman se parecían en verdad a obras milagrosas, pero se nos ha indicado que se trataba únicamente de esas transformaciones de mente y de espíritu que a veces se producen en la experiencia de las personas expectantes y dominadas por la fe, cuando se encuentran bajo la influencia inmediata e inspiradora de una personalidad fuerte, positiva y benéfica, cuyo ministerio destierra el temor y destruye la ansiedad.

\par 
%\textsuperscript{(1658.6)}
\textsuperscript{148:2.3} Elman y sus asociados se esforzaron por enseñar la verdad, a estos enfermos, sobre la <<posesión por los malos espíritus>>, pero tuvieron poco éxito. La creencia de que la enfermedad física y los desórdenes mentales podían ser causados por la presencia de un espíritu, llamado impuro, en la mente o en el cuerpo de la persona afligida, era casi universal.

\par 
%\textsuperscript{(1659.1)}
\textsuperscript{148:2.4} En todos sus contactos con los enfermos y los afligidos, cuando se trataba de la técnica de tratamiento o de revelar las causas desconocidas de una enfermedad, Jesús no pasaba por alto las instrucciones que le había dado Emmanuel, su hermano paradisíaco, antes de embarcarse en la aventura de la encarnación en Urantia. A pesar de esto, los que cuidaban a los enfermos aprendieron muchas lecciones útiles observando la manera en que Jesús inspiraba la fe y la confianza a los enfermos y a los que sufrían.

\par 
%\textsuperscript{(1659.2)}
\textsuperscript{148:2.5} El campamento se dispersó un poco antes de que se acercara la estación en que aumentaban los enfriamientos y las fiebres.

\section*{3. Los asuntos del Padre}
\par 
%\textsuperscript{(1659.3)}
\textsuperscript{148:3.1} Durante todo este período, Jesús dirigió menos de una docena de ceremonias públicas en el campamento y habló una sola vez en la sinagoga de Cafarnaúm, el segundo sábado antes de partir con los evangelistas recién instruídos para la segunda gira de predicación pública en Galilea.

\par 
%\textsuperscript{(1659.4)}
\textsuperscript{148:3.2} Desde su bautismo, el Maestro no había estado tanto tiempo solo como durante este período de instrucción de los evangelistas en el campamento de Betsaida. Cada vez que uno de los apóstoles se atrevía a preguntarle por qué los dejaba con tanta frecuencia, Jesús contestaba invariablemente que estaba ocupado <<en los asuntos del Padre>>.

\par 
%\textsuperscript{(1659.5)}
\textsuperscript{148:3.3} Durante estos períodos de ausencia, Jesús sólo iba acompañado de dos apóstoles. Había liberado temporalmente a Pedro, Santiago y Juan de sus obligaciones como asistentes personales, para que también pudieran participar en la tarea de instruir a los nuevos candidatos evangelistas, cuyo número superaba el centenar. Cuando el Maestro deseaba ir a las colinas para ocuparse de los asuntos del Padre, llamaba a dos apóstoles cualquiera que se encontraran libres para que lo acompañaran. De esta manera, cada uno de los doce tuvo la oportunidad de disfrutar de una asociación estrecha y de un contacto íntimo con Jesús.

\par 
%\textsuperscript{(1659.6)}
\textsuperscript{148:3.4} Aunque no ha sido revelado para los efectos de esta narración, hemos llegado a la conclusión de que durante muchos de estos períodos solitarios en las colinas, el Maestro estaba en asociación directa y ejecutiva con un gran número de los principales administradores de los asuntos de su universo. Desde la época de su bautismo, este Soberano encarnado de nuestro universo había tomado conscientemente una parte cada vez más activa en la dirección de ciertas fases de la administración universal. Siempre hemos mantenido la opinión de que durante estas semanas de menor participación en los asuntos terrestres, y de una manera no revelada a sus compañeros inmediatos, estaba ocupado en dirigir a las altas inteligencias espirituales encargadas del funcionamiento de un vasto universo, y el Jesús humano eligió llamar a estas actividades suyas <<ocuparse de los asuntos de su Padre>>.

\par 
%\textsuperscript{(1659.7)}
\textsuperscript{148:3.5} Cuando Jesús estaba solo durante horas, pero dos de sus apóstoles se encontraban cerca, muchas veces observaron que sus rasgos experimentaban unos cambios rápidos y múltiples, aunque no le escucharon articular palabra. Tampoco observaron ninguna manifestación visible de seres celestiales que pudieran haber estado en comunicación con su Maestro, como los que vieron algunos apóstoles en una ocasión posterior.

\section*{4. El mal, el pecado y la iniquidad}
\par 
%\textsuperscript{(1659.8)}
\textsuperscript{148:4.1} En un rincón aislado y protegido del jardín de Zebedeo, Jesús tenía la costumbre de mantener conversaciones particulares, dos noches por semana, con las personas que deseaban hablar con él. En una de estas conversaciones vespertinas en privado, Tomás le hizo al Maestro la siguiente pregunta: <<¿Por qué es necesario que los hombres nazcan del espíritu para entrar en el reino? ¿Es necesario el renacimiento para evitar el control del maligno? Maestro, ¿qué es el mal?>>\footnote{\textit{¿Por qué debemos nacer del espíritu?}: Jn 3:3-6.} Después de escuchar estas preguntas, Jesús le dijo a Tomás:

\par 
%\textsuperscript{(1660.1)}
\textsuperscript{148:4.2} <<No cometas el error de confundir el \textit{mal} con el \textit{maligno}, llamado con más exactitud el \textit{inicuo}. Aquel que llamas el maligno es el hijo del amor de sí mismo, el alto administrador que se rebeló deliberadamente contra el gobierno de mi Padre y de sus Hijos leales. Pero ya he vencido a estos rebeldes pecaminosos. Clarifica en tu mente estas actitudes diferentes hacia el Padre y su universo. No olvides nunca estas leyes que regulan las relaciones con la voluntad del Padre:>>

\par 
%\textsuperscript{(1660.2)}
\textsuperscript{148:4.3} <<El mal es la transgresión inconsciente o involuntaria de la ley divina, de la voluntad del Padre. El mal es igualmente la medida de la imperfección con que se obedece a la voluntad del Padre>>.

\par 
%\textsuperscript{(1660.3)}
\textsuperscript{148:4.4} <<El pecado es la transgresión consciente, conocida y deliberada de la ley divina, de la voluntad del Padre. El pecado es la medida de la aversión a dejarse conducir divinamente y dirigir espiritualmente>>.

\par 
%\textsuperscript{(1660.4)}
\textsuperscript{148:4.5} <<La iniquidad es la transgresión premeditada, determinada y persistente de la ley divina, de la voluntad del Padre. La iniquidad es la medida del rechazo continuo del plan amoroso del Padre para la supervivencia de la personalidad, y del ministerio misericordioso de salvación de los Hijos>>.

\par 
%\textsuperscript{(1660.5)}
\textsuperscript{148:4.6} <<Antes de renacer del espíritu, el hombre mortal está sujeto a las malas tendencias inherentes a su naturaleza, pero estas imperfecciones naturales de conducta no son ni el pecado ni la iniquidad. El hombre mortal acaba de empezar su larga ascensión hacia la perfección del Padre que está en el Paraíso. Ser imperfecto o parcial por dotación natural no es un pecado. Es verdad que el hombre está sometido al mal, pero no es en ningún sentido el hijo del maligno, a menos que haya escogido a sabiendas y deliberadamente los caminos del pecado y una vida de iniquidad. El mal es inherente al orden natural de este mundo, pero el pecado es una actitud de rebelión consciente que fue traída a este mundo por aquellos que cayeron desde la luz espiritual hasta las densas tinieblas>>.

\par 
%\textsuperscript{(1660.6)}
\textsuperscript{148:4.7} <<Tomás, estás confundido por las doctrinas de los griegos y los errores de los persas. No comprendes las relaciones entre el mal y el pecado porque consideras que la humanidad empezó en la Tierra con un Adán perfecto, y fue degenerando rápidamente, a través del pecado, hasta el deplorable estado actual del hombre. Pero, ¿por qué te niegas a comprender el significado del relato que revela cómo Caín, el hijo de Adán, fue a la tierra de Nod y allí consiguió una esposa? ¿Por qué te niegas a interpretar el significado del relato que describe cómo los hijos de Dios encontraron esposas entre las hijas de los hombres?>>\footnote{\textit{La supuesta caída del hombre}: Gn 3:17-19; Ro 5:12-19. \textit{Caín tomó esposa en Nod}: Gn 4:16-17. \textit{Los hijos de Dios tomaron esposas}: Gn 6:1-2.}

\par 
%\textsuperscript{(1660.7)}
\textsuperscript{148:4.8} <<Es verdad que los hombres son malos por naturaleza, pero no necesariamente pecadores. El nuevo nacimiento ---el bautismo del espíritu--- es esencial para liberarse del mal y necesario para entrar en el reino de los cielos, pero nada de esto disminuye el hecho de que el hombre es un hijo de Dios. Esta presencia inherente del mal potencial tampoco significa que el hombre esté separado, de alguna manera misteriosa, del Padre que está en los cielos, de tal forma que, como si fuera un extraño, un extranjero o un hijastro, tiene que intentar de alguna manera que el Padre lo adopte legalmente. Todas estas ideas han nacido, en primer lugar, de vuestra mala comprensión del Padre, y en segundo lugar, de vuestra ignorancia sobre el origen, la naturaleza y el destino del hombre>>\footnote{\textit{Malos por naturaleza, pero no pecadores}: Ec 7:20; Ro 3:23; 1 Jn 1:8.}.

\par 
%\textsuperscript{(1660.8)}
\textsuperscript{148:4.9} <<Los griegos y otros os han enseñado que el hombre va descendiendo continuamente desde la perfección divina hacia el olvido o la destrucción; yo he venido para mostrar que el hombre, gracias a su entrada en el reino, asciende de manera cierta y segura hacia Dios y la perfección divina. Cualquier ser que, de alguna manera, no alcanza los ideales divinos y espirituales de la voluntad del Padre eterno, es potencialmente malo, pero ese ser no es en ningún sentido un pecador, y mucho menos inicuo>>\footnote{\textit{Los griegos enseñan la degradación del hombre}: Ro 5:12-19.}.

\par 
%\textsuperscript{(1661.1)}
\textsuperscript{148:4.10} \guillemotleft Tomás, ¿no has leído acerca de esto en las Escrituras, donde está escrito: `Vosotros sois los hijos del Señor vuestro Dios'. `Yo seré su Padre y él será mi hijo'\footnote{\textit{Yo seré su Padre, él será mi hijo}: 2 Sam 7:14.}. `Lo he escogido para que sea mi hijo\footnote{\textit{Lo he elegido como mi hijo}: 1 Cr 28:6.} ---yo seré su Padre'. `Trae a mis hijos desde lejos y a mis hijas desde los confines de la Tierra, e incluso a todos los que son llamados por mi nombre, porque los he creado para gloria mía'\footnote{\textit{Los he creado para mi gloria}: Is 43:6-7.}. `Sois los hijos del Dios viviente'\footnote{\textit{Sois los hijos de Dios}: Sal 82:6. \textit{Sois los hijos del Dios viviente}: Os 1:10.}. `Los que tienen el espíritu de Dios son en verdad los hijos de Dios?'\footnote{\textit{Los que tienen el espíritu son hijos de Dios}: Ro 8:14.} De la misma manera que el hijo terrestre posee un fragmento material de su padre humano, existe un fragmento espiritual del Padre celestial en cada hijo del reino por la fe\guillemotright.

\par 
%\textsuperscript{(1661.2)}
\textsuperscript{148:4.11} Jesús dijo todo esto y mucho más a Tomás, y el apóstol comprendió una gran parte de ello; sin embargo, Jesús le recomendó: <<no hables con los demás sobre estas cuestiones hasta después de que yo haya regresado al Padre>>. Y Tomás no mencionó esta entrevista hasta después de que el Maestro hubo partido de este mundo.

\section*{5. La finalidad de la aflicción}
\par 
%\textsuperscript{(1661.3)}
\textsuperscript{148:5.1} En otra de estas entrevistas privadas en el jardín, Natanael le preguntó a Jesús: <<Maestro, aunque empiezo a comprender por qué rehúsas practicar la curación de manera indiscriminada, aún no logro comprender por qué el Padre amoroso que está en los cielos permite que tantos hijos suyos de la Tierra sufran tantas aflicciones>>. El Maestro respondió a Natanael, diciendo:

\par 
%\textsuperscript{(1661.4)}
\textsuperscript{148:5.2} <<Natanael, tú y otras muchas personas estáis así de perplejos porque no comprendéis que el orden natural de este mundo ha sido alterado muchas veces a causa de las aventuras pecaminosas de ciertos traidores rebeldes a la voluntad del Padre. Yo he venido para empezar a poner orden en estas cosas. Pero se necesitarán muchos siglos para devolver esta parte del universo a su antigua conducta, y liberar así a los hijos de los hombres de las cargas adicionales del pecado y de la rebelión. La sola presencia del mal es una prueba suficiente para la ascensión del hombre ---el pecado no es esencial para la supervivencia>>.

\par 
%\textsuperscript{(1661.5)}
\textsuperscript{148:5.3} <<Pero hijo mío, deberías saber que el Padre no aflige deliberadamente a sus hijos. El hombre atrae sobre sí mismo aflicciones innecesarias como resultado de su negativa persistente a caminar en los senderos mejores de la voluntad divina. La aflicción está en potencia en el mal, pero una gran parte de ella ha sido producida por el pecado y la iniquidad. En este mundo han tenido lugar muchos acontecimientos insólitos, y no es de extrañar que todos los hombres que reflexionan se queden perplejos ante las escenas de sufrimiento y de aflicción que contemplan. Pero puedes estar seguro de una cosa: el Padre no envía la aflicción como un castigo arbitrario por haber obrado mal. Las imperfecciones y los obstáculos del mal son inherentes; los castigos del pecado son inevitables; las consecuencias destructivas de la iniquidad son inexorables. El hombre no debería acusar a Dios por las calamidades que son el resultado natural de la vida que ha escogido vivir; el hombre tampoco debería quejarse de las experiencias que forman parte de la vida, tal como ésta se vive en este mundo. Es voluntad del Padre que el hombre mortal trabaje con perseverancia y firmeza para mejorar su condición en la Tierra. La aplicación inteligente debería capacitar al hombre para superar una gran parte de su miseria terrestre>>\footnote{\textit{No hay aflicción a propósito}: Heb 12:5-11.}.

\par 
%\textsuperscript{(1662.1)}
\textsuperscript{148:5.4} <<Natanael, nuestra misión consiste en ayudar a los hombres a resolver sus problemas espirituales y, de esta manera, vivificar su mente de tal forma que se encuentren mejor preparados e inspirados para intentar resolver sus múltiples problemas materiales. Sé que estás confundido después de haber leído las Escrituras. La tendencia de atribuir a Dios la responsabilidad de todo lo que el hombre ignorante no logra comprender ha prevalecido demasiado a menudo. El Padre no es personalmente responsable de todo lo que no podáis comprender. No dudes del amor del Padre simplemente porque te aflija alguna ley justa y sabia decretada por él, porque has transgredido inocente o deliberadamente ese mandato divino>>.

\par 
%\textsuperscript{(1662.2)}
\textsuperscript{148:5.5} \guillemotleft Pero Natanael, hay muchas cosas en las Escrituras que podrían haberte instruido si las hubieras leído con discernimiento. ¿No recuerdas que está escrito: `Hijo mío, no desprecies el castigo del Señor\footnote{\textit{No desprecies el castigo}: Pr 3:11-12.}, ni te canses de su reprimenda, porque el Señor corrige al que ama, como un padre corrige al hijo en quien tiene su complacencia'\footnote{\textit{Corrección en el sufrimiento}: Job 5:17-18.}. `El Señor no aflige de buena gana'\footnote{\textit{El Señor no aflige de buena gana}: Lm 3:33.}. `Antes de estar afligido me había desviado\footnote{\textit{Antes de estar afligido me había desviado}: Sal 119:67.}, pero ahora cumplo la ley. La aflicción ha sido buena para mí, pues me ha permitido aprender los estatutos divinos'. `Conozco vuestros pesares\footnote{\textit{Conozco vuestros pesares}: Ex 3:7.}. El Dios eterno es vuestro refugio\footnote{\textit{Dios es mi refugio}: Dt 33:27.}, y por debajo se encuentran los brazos eternos'. `El Señor es también un refugio para los oprimidos, un puerto de descanso en los momentos de confusión'\footnote{\textit{Dios es un puerto de descanso}: Sal 9:9.}. `El Señor lo fortalecerá en el lecho de la aflicción; el Señor no olvidará a los enfermos'\footnote{\textit{Dios no se olvidará de los enfermos}: Sal 41:3.}. `De la misma manera que un padre muestra compasión por sus hijos, el Señor se compadece de aquellos que le temen. Él conoce vuestro cuerpo; se acuerda de que sois polvo'\footnote{\textit{Él recuerda que sois polvo}: Sal 103:13-14.}. `Cura a los abatidos\footnote{\textit{Cura a los abatidos}: Sal 147:3.} y venda sus heridas'. `Él es la esperanza del pobre\footnote{\textit{Él es la esperanza del pobre}: Is 25:4.}, la fuerza del indigente en su desdicha, un refugio contra la tempestad y una sombra contra el calor sofocante'. `Da poder al extenuado\footnote{\textit{Da poder al extenuado}: Is 40:29.} y acrecienta las fuerzas de los que no tienen ninguna potencia'. `No quebrará la caña cascada, y no apagará el lino humeante'\footnote{\textit{No romperá la caña quebrada}: Is 42:3.}. `Cuando atraveséis las aguas de la aflicción, yo estaré con vosotros, y cuando los ríos de la adversidad os inunden\footnote{\textit{Cuando los ríos de la aflicción os inunden}: Is 43:2.}, no os abandonaré'. `Él me ha enviado para vendar los corazones rotos\footnote{\textit{Enviado para vendar los corazones rotos}: Is 61:1-2.}, para proclamar la libertad a los cautivos y para consolar a todos los enlutados'. `El sufrimiento contiene la enmienda\footnote{\textit{La aflicción es buena para mí}: Sal 119:71.}; la aflicción no nace del polvo?'\footnote{\textit{La aflicción no nace del polvo}: Job 5:6.}\guillemotright

\section*{6. El malentendido sobre el sufrimiento ---El discurso sobre Job}
\par 
%\textsuperscript{(1662.3)}
\textsuperscript{148:6.1} Aquella misma tarde, en Betsaida, Juan también le preguntó a Jesús por qué tanta gente aparentemente inocente sufría tantas enfermedades y experimentaba tantas aflicciones. Al responder a las preguntas de Juan, entre otras muchas cosas, el Maestro dijo:

\par 
%\textsuperscript{(1662.4)}
\textsuperscript{148:6.2} <<Hijo mío, no comprendes el significado de la adversidad ni la misión del sufrimiento. ¿No has leído esa obra maestra de la literatura semita ---la historia que está en las Escrituras sobre las aflicciones de Job? ¿No recuerdas que esta maravillosa parábola empieza con la narración de la prosperidad material del servidor del Señor? Recuerdas bien que Job gozaba de la bendición de tener hijos, riqueza, dignidad, posición, salud y todas las demás cosas que los hombres valoran en esta vida temporal. Según las enseñanzas tradicionalmente aceptadas por los hijos de Abraham, esta prosperidad material era una prueba más que suficiente del favor divino. Sin embargo, las posesiones materiales y la prosperidad temporal no indican el favor de Dios. Mi Padre que está en los cielos ama a los pobres tanto como a los ricos; él no hace acepción de personas>>\footnote{\textit{Libro de Job}: Job. \textit{Job tenía riquez, salud, posición}: Job 1:1-3. \textit{Dios bendice al fiel}: Gn 49:25; Sal 1:1-3; Pr 3:33; Pr 10:6; Dt 28:1-8. \textit{No hace acepción de personas}: 2 Cr 19:7; Job 34:19; Eclo 35:12; Hch 10:34; Ro 2:11; Gl 2:6; 3:28; Ef 6:9; Col 3:11.}.

\par 
%\textsuperscript{(1663.1)}
\textsuperscript{148:6.3} <<Aunque a la transgresión de la ley divina le sigue, tarde o temprano, la cosecha del castigo, y aunque los hombres terminan sin duda por recoger aquello que han sembrado, sin embargo deberías saber que el sufrimiento humano no siempre es un castigo por un pecado anterior. Tanto Job como sus amigos no lograron encontrar la verdadera respuesta a sus perplejidades. Con los conocimientos que disfrutas en la actualidad, difícilmente atribuirías a Satanás o a Dios los papeles que interpretan en esta parábola singular. Job no encontró, por medio del sufrimiento, la explicación de sus problemas intelectuales ni la solución de sus dificultades filosóficas, pero sí consiguió grandes victorias. Incluso en presencia misma del derrumbamiento de sus defensas teológicas, se elevó a unas alturas espirituales en las que pudo decir con sinceridad: `Me aborrezco a mí mismo'; entonces se le concedió la salvación de una \textit{visión de Dios}. Así pues, incluso a través de un sufrimiento mal comprendido, Job se elevó a un plano sobrehumano de comprensión moral y de perspicacia espiritual. Cuando el servidor que sufre obtiene una visión de Dios, se produce una paz en el alma que sobrepasa toda comprensión humana>>\footnote{\textit{Los hombres cosechan lo que siembran}: Job 4:8; Gl 6:7. \textit{Me aborrezco a mí mismo}: Job 42:6. \textit{Paz que sobrepasa toda comprensión}: Flp 4:7.}.

\par 
%\textsuperscript{(1663.2)}
\textsuperscript{148:6.4} \guillemotleft El primer amigo de Job, Elifaz\footnote{\textit{Elifaz, el primer amigo}: Job 4:1--5:27.}, exhortó al sufridor a que mostrara en sus aflicciones la misma entereza que había recomendado a otras personas en la época de su prosperidad. Este falso consolador dijo: `Confía en tu religión, Job; recuerda que son los perversos los que sufren, no los justos. Debes merecer este castigo, pues de lo contrario no estarías afligido. Sabes bien que ningún hombre puede ser justo a los ojos de Dios. Sabes que los malvados nunca prosperan realmente. De cualquier forma, el hombre parece predestinado a sufrir, y quizás el Señor sólo te castiga por tu propio bien'. No es de extrañar que el pobre Job no se sintiera muy consolado con esta interpretación del problema del sufrimiento humano\guillemotright.

\par 
%\textsuperscript{(1663.3)}
\textsuperscript{148:6.5} \guillemotleft Pero el consejo de su segundo amigo, Bildad\footnote{\textit{Bildad, el segundo amigo}: Job 8:1-22.}, fue aún más deprimente, a pesar de su acierto desde el punto de vista de la teología aceptada en aquella época. Bildad dijo: `Dios no puede ser injusto. Tus hijos han debido ser unos pecadores, puesto que han perecido; debes estar en un error, pues de lo contrario no estarías así de afligido. Si eres realmente justo, Dios te liberará seguramente de tus aflicciones. La historia de las relaciones de Dios con el hombre debería enseñarte que el Todopoderoso sólo destruye a los perversos'.\guillemotright

\par 
%\textsuperscript{(1663.4)}
\textsuperscript{148:6.6} <<A continuación, recuerdas cómo Job respondió a sus amigos, diciendo: `Sé bien que Dios no escucha mi llamada de auxilio. ¿Cómo Dios puede ser justo y al mismo tiempo no hacer caso en absoluto de mi inocencia? Estoy aprendiendo que no puedo obtener satisfacción apelando al Todopoderoso. ¿No podéis percibir que Dios tolera la persecución de los buenos por parte de los malos? Y puesto que el hombre es tan débil, ¿qué posibilidades tiene de encontrar consideración entre las manos de un Dios omnipotente? Dios me ha hecho como soy, y cuando se vuelve así contra mí, estoy sin defensa. ¿Por qué Dios me ha creado, simplemente para sufrir de esta manera miserable?'>>\footnote{\textit{La desesperanza de Job}: Job 9:1--10:22.}

\par 
%\textsuperscript{(1663.5)}
\textsuperscript{148:6.7} <<¿Quién puede criticar la actitud de Job, en vista de los consejos de sus amigos y de las ideas erróneas sobre Dios que ocupaban su propia mente? ¿No ves que Job deseaba ardientemente un Dios \textit{humano}, que tenía sed de comunicarse con un Ser divino que conociera la condición mortal del hombre y comprendiera que los justos han de sufrir a menudo, siendo inocentes, como parte de esta primera vida en la larga ascensión hacia el Paraíso? Por eso el Hijo del Hombre ha venido desde el Padre para vivir una vida tal en la carne, que sea capaz de consolar y socorrer a todos aquellos que de aquí en adelante van a ser llamados a soportar las aflicciones de Job>>.

\par 
%\textsuperscript{(1663.6)}
\textsuperscript{148:6.8} \guillemotleft El tercer amigo de Job, Zofar\footnote{\textit{Zofar, el tercer amigo}: Job 11:1-20.}, pronunció entonces unas palabras aún menos confortantes cuando dijo: `Eres un necio al pretender que eres justo, puesto que estás así de afligido. Pero admito que es imposible comprender los caminos de Dios. Quizás haya un propósito oculto en todos tus sufrimientos'. Después de haber escuchado a sus tres amigos, Job apeló directamente a Dios para que lo ayudara, alegando el hecho de que `el hombre, nacido de mujer, vive pocos días y está lleno de problemas'.\guillemotright\footnote{\textit{Vive pocos días y está lleno de problemas}: Job 14:1.}

\par 
%\textsuperscript{(1664.1)}
\textsuperscript{148:6.9} <<Entonces empezó la segunda sesión con sus amigos. Elifaz se volvió más severo, acusador y sarcástico. Bildad se indignó por el desprecio de Job por sus amigos. Zofar reiteró sus consejos melancólicos. A estas alturas, Job se había disgustado con sus amigos y apeló de nuevo a Dios; ahora apelaba a un Dios justo, contra el Dios de injusticia incorporado en la filosofía de sus amigos e incluido también en su propia actitud religiosa. A continuación, Job buscó refugio en el consuelo de una vida futura, en la que las injusticias de la existencia mortal pudieran ser rectificadas de manera más justa. A falta de recibir la ayuda de los hombres, Job es impulsado hacia Dios. Luego sobreviene en su corazón la gran lucha entre la fe y la duda. Finalmente, el humano afligido empieza a percibir la luz de la vida. Su alma torturada se eleva a nuevas alturas de esperanza y valentía; puede ser que continúe sufriendo e incluso que muera, pero su alma iluminada pronuncia ahora este grito de triunfo, `¡Mi Protector vive!'>>

\par 
%\textsuperscript{(1664.2)}
\textsuperscript{148:6.10} <<Job tenía totalmente razón cuando desafió la doctrina de que Dios aflige a los hijos para castigar a sus padres. Job estaba preparado para admitir que Dios es justo, pero anhelaba una revelación del carácter personal del Eterno que satisfaciera su alma. Y ésa es nuestra misión en la Tierra. A los mortales que sufren ya no se les volverá a negar el consuelo de conocer el amor de Dios y de comprender la misericordia del Padre que está en los cielos. El discurso de Dios pronunciado desde el torbellino era un concepto majestuoso para la época en que fue expresado, pero tú ya has aprendido que el Padre no se revela de esa manera, sino que habla más bien dentro del corazón humano como una vocecita suave, que dice: `Éste es el camino; síguelo'. ¿No comprendes que Dios reside dentro de ti, que se ha vuelto como tú eres para poder hacerte como él es?>>

\par 
%\textsuperscript{(1664.3)}
\textsuperscript{148:6.11} Luego, Jesús hizo su declaración final: <<El Padre que está en los cielos no aflige voluntariamente a los hijos de los hombres. El hombre sufre, en primer lugar, por los accidentes del tiempo y las imperfecciones de la desdicha de una existencia física desprovista de madurez. En segundo lugar, sufre las consecuencias inexorables del pecado ---de la transgresión de las leyes de la vida y de la luz. Y finalmente, el hombre recoge la cosecha de su propia persistencia inicua en la rebelión contra la justa soberanía del cielo sobre la Tierra. Pero las miserias del hombre no son un azote \textit{personal} del juicio divino. El hombre puede hacer, y hará, muchas cosas para disminuir sus sufrimientos temporales. Pero libérate de una vez por todas de la superstición de que Dios aflige al hombre a instancias del maligno. Estudia el Libro de Job sólo para descubrir cuántas ideas erróneas sobre Dios pueden albergar honradamente incluso unos hombres de bien; y luego observa cómo el mismo Job, dolorosamente afligido, encontró al Dios del consuelo y de la salvación, a pesar de estas enseñanzas erróneas. Al final, su fe traspasó las nubes del sufrimiento para discernir la luz de la vida derramada por el Padre como misericordia curativa y rectitud eterna>>.

\par 
%\textsuperscript{(1664.4)}
\textsuperscript{148:6.12} Juan meditó estas afirmaciones en su corazón durante muchos días. Esta conversación con el Maestro en el jardín provocó un cambio considerable en toda su vida posterior, y más tarde contribuyó mucho a que los otros apóstoles cambiaran su punto de vista en cuanto a la fuente, la naturaleza y la finalidad de las aflicciones humanas comunes. Pero Juan no habló nunca de esta conversación hasta después de la partida del Maestro.

\section*{7. El hombre de la mano seca}
\par 
%\textsuperscript{(1664.5)}
\textsuperscript{148:7.1} El segundo sábado antes de la partida de los apóstoles y del nuevo cuerpo de evangelistas para la segunda gira de predicación por Galilea, Jesús habló en la sinagoga de Cafarnaúm sobre <<Las alegrías de una vida de rectitud>>. Cuando Jesús terminó de hablar, un amplio grupo de mutilados, lisiados, enfermos y afligidos se agolpó a su alrededor buscando la curación. En este grupo también se encontraban los apóstoles, muchos de los nuevos evangelistas y los espías fariseos de Jerusalén. A cualquier parte que fuera Jesús (excepto cuando iba a las colinas a los asuntos de su Padre) los seis espías de Jerusalén lo seguían con toda seguridad.

\par 
%\textsuperscript{(1665.1)}
\textsuperscript{148:7.2} Mientras Jesús estaba hablándole a la gente, el jefe de los espías fariseos incitó a un hombre que tenía una mano seca a que se acercara al Maestro y le preguntara si era legal ser curado el día del sábado, o si debía buscar el remedio otro día. Cuando Jesús vio al hombre, escuchó sus palabras y percibió que había sido enviado por los fariseos, dijo: <<Acércate, que voy a hacerte una pregunta. Si tuvieras una oveja y se cayera en un hoyo el día del sábado, ¿bajarías para cogerla y sacarla de allí? ¿Es lícito hacer estas cosas el día del sábado?>> Y el hombre respondió: <<Sí, Maestro, sería lícito hacer esta buena acción el día del sábado>>. Entonces, dirigiéndose a todos ellos, Jesús dijo: <<Sé por qué habéis enviado a este hombre a mi presencia. Quisierais encontrar en mí un motivo de culpa si pudierais tentarme para que muestre misericordia el día del sábado. Todos aceptáis en silencio que era lícito sacar del hoyo a la desgraciada oveja, aunque sea sábado, y os pongo por testigos de que es lícito mostrar una bondad afectuosa el día del sábado no sólo a los animales, sino también a los hombres. ¡Cuánto más valioso es un hombre que una oveja! Proclamo que es legal hacer el bien a los hombres el día del sábado>>. Mientras todos permanecían delante de él en silencio, Jesús se dirigió al hombre de la mano seca y le dijo: <<Ponte aquí a mi lado para que todos puedan verte. Y ahora, para que puedas saber que es voluntad de mi Padre que hagáis el bien el día del sábado, si tienes fe para ser curado, te ruego que extiendas la mano>>.

\par 
%\textsuperscript{(1665.2)}
\textsuperscript{148:7.3} Cuando este hombre alargaba su mano seca, ésta quedó curada. La gente estuvo a punto de revolverse contra los fariseos, pero Jesús les pidió que se calmaran, diciendo: <<Acabo de deciros que es lícito hacer el bien el sábado, salvar una vida, pero no os he enseñado que hagáis el mal y que cedáis al deseo de matar>>. Los fariseos se fueron encolerizados, y a pesar de que era sábado, se dieron mucha prisa en llegar a Tiberiades para pedirle consejo a Herodes; hicieron todo lo que estuvo en su poder por despertar su preocupación, con objeto de asegurarse la alianza de los herodianos en contra de Jesús. Pero Herodes se negó a tomar medidas contra Jesús, aconsejándoles que llevaran sus quejas a Jerusalén.

\par 
%\textsuperscript{(1665.3)}
\textsuperscript{148:7.4} Éste es el primer caso en el que Jesús realizó un milagro en respuesta al desafío de sus enemigos. El Maestro llevó a cabo este supuesto milagro, no para demostrar su poder curativo, sino para protestar eficazmente contra la transformación del descanso religioso del sábado en una verdadera esclavitud de restricciones sin sentido para toda la humanidad. Este hombre regresó a su trabajo como albañil, demostrando ser una de las personas cuya curación fue seguida por una vida de acción de gracias y de rectitud.

\section*{8. La última semana en Betsaida}
\par 
%\textsuperscript{(1665.4)}
\textsuperscript{148:8.1} Durante la última semana de la estancia en Betsaida, los espías de Jerusalén tuvieron una actitud muy dividida con respecto a Jesús y sus enseñanzas. Tres de estos fariseos estaban enormemente impresionados por lo que habían visto y oído. Mientras tanto, en Jerusalén, un joven miembro influyente del sanedrín, llamado Abraham, adoptó públicamente las enseñanzas de Jesús y fue bautizado por Abner en el estanque de Siloam. Todo Jerusalén estaba convulsionado por este acontecimiento, y unos mensajeros fueron enviados inmediatamente a Betsaida para hacer volver a los seis espías fariseos.

\par 
%\textsuperscript{(1666.1)}
\textsuperscript{148:8.2} El filósofo griego que había sido ganado para el reino durante la gira anterior por Galilea, regresó con algunos judíos ricos de Alejandría, y una vez más invitaron a Jesús para que fuera a su ciudad con objeto de establecer una escuela conjunta de filosofía y religión, así como un hospital para los enfermos. Pero Jesús declinó cortésmente la invitación.

\par 
%\textsuperscript{(1666.2)}
\textsuperscript{148:8.3} Aproximadamente por esta época, un profeta llamado Quirmet, que se ponía en trance, llegó al campamento de Betsaida procedente de Bagdad. Este supuesto profeta tenía unas visiones peculiares cuando estaba en trance y unos sueños fantásticos cuando se perturbaba su sueño. Creó un alboroto considerable en el campamento, y Simón Celotes opinaba que había que tratar más bien con rudeza a este farsante que se engañaba a sí mismo, pero Jesús intervino para dejarle total libertad de acción durante unos días. Todos los que escucharon su predicación reconocieron pronto que, utilizando el criterio del evangelio del reino, su enseñanza no era válida. Quirmet regresó poco después a Bagdad, llevándose consigo solamente a media docena de almas inestables y erráticas. Pero antes de que Jesús intercediera por el profeta de Bagdad, David Zebedeo, con la ayuda de un comité nombrado por sí mismo, había llevado a Quirmet al lago y, después de zambullirlo repetidas veces en el agua, le aconsejaron que se fuera de allí ---que organizara y construyera su propio campamento.

\par 
%\textsuperscript{(1666.3)}
\textsuperscript{148:8.4} Aquel mismo día, una mujer fenicia llamada Bet-Marión se volvió tan fanática que perdió la cabeza, y sus amigos la despidieron después de haberle faltado poco para ahogarse al intentar caminar por el agua.

\par 
%\textsuperscript{(1666.4)}
\textsuperscript{148:8.5} Abraham el fariseo, el nuevo converso de Jerusalén, donó todos sus bienes terrenales al tesoro apostólico, y esta contribución ayudó mucho a que se pudieran enviar inmediatamente los cien evangelistas recién instruídos. Andrés ya había anunciado el cierre del campamento, y todos se prepararon para irse a sus casas o para acompañar a los evangelistas a Galilea.

\section*{9. La curación del paralítico}
\par 
%\textsuperscript{(1666.5)}
\textsuperscript{148:9.1} El viernes por la tarde, 1 de octubre, Jesús estaba celebrando su última reunión con los apóstoles, los evangelistas y otros líderes del campamento en vías de disolverse; los seis fariseos de Jerusalén estaban sentados en la primera fila de esta asamblea, en la espaciosa habitación agrandada de la parte delantera de la casa de Zebedeo. Entonces se produjo uno de los episodios más extraños y singulares de toda la vida terrestre de Jesús. En aquel momento, el Maestro estaba hablando de pie en esta gran habitación, que había sido construida para acoger estas reuniones durante la estación de las lluvias. La casa estaba totalmente rodeada por una gran muchedumbre que aguzaba el oído para captar algunas palabras del discurso de Jesús.

\par 
%\textsuperscript{(1666.6)}
\textsuperscript{148:9.2} Mientras la casa estaba abarrotada de gente y totalmente rodeada de oyentes entusiastas, un hombre que llevaba mucho tiempo afligido de parálisis fue traído por sus amigos desde Cafarnaúm en una pequeña litera. Este paralítico había oído que Jesús estaba a punto de marcharse de Betsaida, y después de hablar con Aarón el albañil, que había sido curado tan recientemente, decidió que le llevaran a la presencia de Jesús, donde podría buscar la curación. Sus amigos trataron de entrar en la casa de Zebedeo por la puerta de delante y por la de atrás, pero el gentío era demasiado compacto. Sin embargo, el paralítico se negó a darse por vencido; pidió a sus amigos que consiguieran unas escaleras, con las cuales subieron al tejado de la habitación en la que Jesús estaba hablando, y después de aflojar las tejas, bajaron audazmente al enfermo en su litera con unas cuerdas hasta que el afligido se encontró en el suelo directamente delante del Maestro. Cuando Jesús vio lo que habían hecho, dejó de hablar, mientras que los que estaban con él en la habitación se maravillaron de la perseverancia del enfermo y sus amigos. El paralítico dijo: <<Maestro, no quisiera interrumpir tu enseñanza, pero estoy decidido a curarme. No soy como aquellos que recibieron la curación y se olvidaron enseguida de tu enseñanza. Quisiera curarme para poder servir en el reino de los cielos>>. A pesar de que la aflicción de este hombre se la había producido su propia vida disipada, al ver su fe, Jesús le dijo al paralítico: <<Hijo, no temas; tus pecados están perdonados. Tu fe te salvará>>.

\par 
%\textsuperscript{(1667.1)}
\textsuperscript{148:9.3} Cuando los fariseos de Jerusalén, junto con otros escribas y juristas que estaban sentados con ellos, escucharon esta declaración de Jesús, empezaron a decirse entre ellos: <<¿Cómo se atreve este hombre a hablar así? ¿No comprende que esas palabras son una blasfemia? ¿Quién puede perdonar los pecados si no Dios?>> Al percibir en su espíritu que razonaban de esta manera en su propia mente y entre ellos, Jesús les dirigió la palabra, diciendo: <<¿Por qué razonáis así en vuestro corazón? ¿Quiénes sois vosotros para juzgarme? ¿Qué diferencia hay entre decirle a este paralítico: tus pecados están perdonados, o decirle: levántate, coge tu litera y anda? Pero para que vosotros, que presenciáis todo esto, podáis saber definitivamente que el Hijo del Hombre tiene autoridad y poder en la Tierra para perdonar los pecados, le diré a este hombre afligido: Levántate, recoge tu litera y vete a tu propia casa>>. Cuando Jesús hubo hablado así, el paralítico se levantó, los que estaban presentes le abrieron paso, y salió delante de todos ellos. Aquellos que vieron estas cosas se quedaron asombrados. Pedro disolvió la asamblea, mientras que muchos oraban y glorificaban a Dios, confesando que nunca habían visto antes unos acontecimientos tan extraordinarios.

\par 
%\textsuperscript{(1667.2)}
\textsuperscript{148:9.4} Los mensajeros del sanedrín llegaron más o menos en aquel momento para ordenar a los seis espías que regresaran a Jerusalén. Cuando escucharon este mensaje, emprendieron un serio debate entre ellos; una vez que terminaron de discutir, el jefe y dos de sus asociados regresaron con los mensajeros a Jerusalén, mientras que los otros tres espías fariseos confesaron su fe en Jesús y se dirigieron inmediatamente al lago, donde fueron bautizados por Pedro y admitidos por los apóstoles en la comunidad como hijos del reino.


\chapter{Documento 149. La segunda gira de predicación}
\par 
%\textsuperscript{(1668.1)}
\textsuperscript{149:0.1} LA SEGUNDA gira de predicación pública por Galilea empezó el domingo 3 de octubre del año 28, y continuó durante cerca de tres meses, finalizando el 30 de diciembre. En este esfuerzo participaron Jesús y sus doce apóstoles, asistidos por el grupo recién reclutado de 117 evangelistas y por otras numerosas personas interesadas. Durante esta gira visitaron Gadara, Tolemaida, Jafia, Dabarita, Meguido, Jezreel, Escitópolis, Tariquea, Hipos, Gamala, Betsaida-Julias, y otras muchas ciudades y pueblos.

\par 
%\textsuperscript{(1668.2)}
\textsuperscript{149:0.2} Antes de partir este domingo por la mañana, Andrés y Pedro pidieron a Jesús que asignara las obligaciones definitivas a los nuevos evangelistas, pero el Maestro rehusó diciendo que no era de su incumbencia hacer unas cosas que otros podían ejecutar de manera aceptable. Después de deliberar convenientemente, se decidió que Santiago Zebedeo asignaría las obligaciones. Cuando Santiago concluyó sus comentarios, Jesús dijo a los evangelistas: <<Salid ahora a efectuar el trabajo que se os ha encomendado, y más adelante, cuando hayáis demostrado vuestra competencia y fidelidad, os ordenaré para que prediquéis el evangelio del reino>>.

\par 
%\textsuperscript{(1668.3)}
\textsuperscript{149:0.3} A lo largo de esta gira, sólo Santiago y Juan viajaron con Jesús. Pedro y los demás apóstoles se llevaron cada uno a unos doce evangelistas, y mantuvieron un estrecho contacto con ellos mientras efectuaron su obra de predicación y enseñanza. Tan pronto como los creyentes estaban preparados para entrar en el reino, los apóstoles les administraban el bautismo. Jesús y sus dos compañeros viajaron mucho durante estos tres meses, visitando a menudo dos ciudades en un solo día para observar el trabajo de los evangelistas y para estimularlos en sus esfuerzos por establecer el reino. Toda esta segunda gira de predicación fue principalmente un esfuerzo por proporcionar una experiencia práctica a este cuerpo de 117 evangelistas recién instruidos.

\par 
%\textsuperscript{(1668.4)}
\textsuperscript{149:0.4} Durante todo este período y posteriormente, hasta el momento en que Jesús y los doce partieron finalmente para Jerusalén, David Zebedeo mantuvo un cuartel general permanente para la obra del reino en la casa de su padre en Betsaida. Éste era el centro de intercambio de información para el trabajo de Jesús en la Tierra, y la estación de relevo para el servicio de mensajeros que David mantenía entre los que trabajaban en las diversas partes de Palestina y regiones adyacentes. Todo esto lo hizo por su propia iniciativa, pero con la aprobación de Andrés. David empleó entre cuarenta y cincuenta mensajeros en este departamento de información para la obra del reino, la cual se ampliaba y extendía rápidamente. Mientras efectuaba este servicio, se ganaba parcialmente la vida dedicando una parte de su tiempo a su antiguo oficio de pescador.

\section*{1. La extensa fama de Jesús}
\par 
%\textsuperscript{(1668.5)}
\textsuperscript{149:1.1} En la época en que se levantó el campamento de Betsaida, la fama de Jesús, en particular como sanador, se había propagado por todas las regiones de Palestina y a través de toda Siria y los países limítrofes. Después de partir de Betsaida, los enfermos siguieron llegando durante semanas, y como no encontraban al Maestro, al enterarse por David dónde estaba, salían en su búsqueda. Durante esta gira, Jesús no realizó deliberadamente ningún supuesto milagro de curación. Sin embargo, docenas de afligidos recuperaron la salud y la felicidad como resultado del poder reconstructor de la intensa fe que los impulsaba a buscar la curación.

\par 
%\textsuperscript{(1669.1)}
\textsuperscript{149:1.2} Aproximadamente por la época de esta misión empezó a producirse una serie peculiar e inexplicable de fenómenos de curación que continuaron durante el resto de la vida de Jesús en la Tierra. En el transcurso de esta gira de tres meses, más de cien hombres, mujeres y niños de Judea, Idumea, Galilea, Siria, Tiro y Sidón, y del otro lado del Jordán, se beneficiaron de esta curación inconsciente por parte de Jesús y, al regresar a sus casas, contribuyeron a aumentar la fama del Maestro. Y lo hicieron a pesar de que Jesús, cada vez que observaba uno de estos casos de curación espontánea, encargaba directamente al beneficiario que <<no se lo contara a nadie>>.

\par 
%\textsuperscript{(1669.2)}
\textsuperscript{149:1.3} Nunca se nos ha revelado qué es lo que sucedía exactamente en estos casos de curación espontánea o inconsciente. El Maestro nunca explicó a sus apóstoles cómo se efectuaban estas curaciones, salvo que en diversas ocasiones se limitó a decir: <<Percibo que una energía ha salido de mí>>. En una ocasión que fue tocado por un niño enfermo, comentó: <<Percibo que la vida ha salido de mí>>.

\par 
%\textsuperscript{(1669.3)}
\textsuperscript{149:1.4} En ausencia de una explicación directa del Maestro sobre la naturaleza de estos casos de curación espontánea, sería una presunción por nuestra parte intentar explicar cómo se efectuaban, pero se nos ha permitido indicar nuestra opinión sobre todos estos fenómenos de curación. Creemos que muchos de estos milagros aparentes de curación, que se produjeron en el transcurso del ministerio terrestre de Jesús, fueron el resultado de la coexistencia de las tres siguientes influencias poderosas, potentes y asociadas:

\par 
%\textsuperscript{(1669.4)}
\textsuperscript{149:1.5} 1. La presencia de una fe sólida, dominante y viviente en el corazón del ser humano que buscaba con insistencia la curación, junto con el hecho de que deseaba esta curación por sus beneficios espirituales más bien que por un restablecimiento puramente físico.

\par 
%\textsuperscript{(1669.5)}
\textsuperscript{149:1.6} 2. La existencia, concomitante con esta fe humana, de la gran simpatía y compasión del Hijo Creador de Dios, encarnado y dominado por la misericordia, que poseía realmente en su persona unos poderes y unas prerrogativas creativos de curación casi ilimitados e independientes del tiempo.

\par 
%\textsuperscript{(1669.6)}
\textsuperscript{149:1.7} 3. Al mismo tiempo que la fe de la criatura y la vida del Creador, también hay que señalar que este Dios-hombre era la expresión personificada de la voluntad del Padre. Si en el contacto entre la necesidad humana y el poder divino capaz de satisfacerla, el Padre no deseaba lo contrario, los dos se convertían en uno solo, y la curación se producía sin que el Jesús humano fuera consciente de ello, pero era inmediatamente reconocida por su naturaleza divina. Así pues, la explicación de muchos de estos casos de curación se encuentra en una gran ley que conocemos desde hace mucho tiempo, a saber: Aquello que el Hijo Creador desea y el Padre eterno lo quiere, EXISTE.

\par 
%\textsuperscript{(1669.7)}
\textsuperscript{149:1.8} Tenemos pues la opinión de que, ante la presencia personal de Jesús, ciertas formas de profunda fe humana \textit{forzaban}, literal y realmente, la manifestación de la curación por medio de ciertas fuerzas y personalidades creativas del universo que en ese momento estaban tan íntimamente asociadas con el Hijo del Hombre. Por lo tanto, es un hecho registrado que Jesús permitía con frecuencia que los hombres se curaran a sí mismos, en su presencia, gracias a su poderosa fe personal.

\par 
%\textsuperscript{(1670.1)}
\textsuperscript{149:1.9} Otras muchas personas buscaban la curación por motivos totalmente egoístas. Una rica viuda de Tiro vino con su séquito buscando la curación de sus numerosas enfermedades; a medida que seguía a Jesús por toda Galilea, continuó ofreciéndole cada vez más dinero, como si el poder de Dios fuera algo que se pudiera vender al mejor postor. Pero ella nunca llegó a interesarse por el evangelio del reino; sólo buscaba la curación de sus dolencias físicas.

\section*{2. La actitud de la gente}
\par 
%\textsuperscript{(1670.2)}
\textsuperscript{149:2.1} Jesús comprendía la mente de los hombres. Conocía el contenido del corazón del hombre, y si sus enseñanzas hubieran sido legadas tal como él las presentó, sin más comentario que la interpretación inspiradora proporcionada por su vida terrestre, todas las naciones y todas las religiones del mundo hubieran abrazado rápidamente el evangelio del reino. Los esfuerzos bien intencionados de los primeros seguidores de Jesús por reformular sus enseñanzas a fin de hacerlas más aceptables para ciertas naciones, razas y religiones, sólo tuvieron como resultado que dichas enseñanzas fueran menos aceptables por todas las demás naciones, razas y religiones.

\par 
%\textsuperscript{(1670.3)}
\textsuperscript{149:2.2} En sus esfuerzos por atraer la atención favorable de ciertos grupos de su época hacia las enseñanzas de Jesús, el apóstol Pablo escribió muchas cartas de instrucciones y recomendaciones. Otros instructores del evangelio de Jesús hicieron lo mismo, pero ninguno de ellos pensó que algunos de estos escritos serían reunidos posteriormente por aquellos que los presentarían como un compendio de las enseñanzas de Jesús. Así pues, aunque el llamado cristianismo contiene más elementos del evangelio del Maestro que ninguna otra religión, también contiene muchas cosas que Jesús no enseñó. Además de la incorporación, en el cristianismo primitivo, de muchas enseñanzas de los misterios persas y de muchos elementos de la filosofía griega, se cometieron dos grandes errores:

\par 
%\textsuperscript{(1670.4)}
\textsuperscript{149:2.3} 1. El esfuerzo por conectar directamente la enseñanza del evangelio con la teología judía, tal como lo ilustran las doctrinas cristianas de la expiación ---la enseñanza de que Jesús era el Hijo sacrificado que satisfaría la justicia inflexible del Padre y aplacaría la ira divina. Estas enseñanzas tuvieron su origen en el esfuerzo loable por hacer más aceptable el evangelio del reino entre los judíos incrédulos. Aunque estos esfuerzos fracasaron en lo referente a atraer a los judíos, no dejaron de confundir y de apartar a muchas almas sinceras de todas las generaciones posteriores.

\par 
%\textsuperscript{(1670.5)}
\textsuperscript{149:2.4} 2. La segunda gran equivocación de los primeros seguidores del Maestro, un error que todas las generaciones posteriores han insistido en perpetuar, fue la de organizar tan completamente la doctrina cristiana alrededor de la \textit{persona} de Jesús. Este énfasis excesivo que se ha dado a la personalidad de Jesús, dentro de la teología del cristianismo, ha contribuido a oscurecer sus enseñanzas. Todo esto ha hecho que los judíos, los mahometanos, los hindúes y otras personas religiosas orientales encuentren cada vez más difícil aceptar las enseñanzas de Jesús. No quisiéramos restar importancia al lugar que ocupa la personalidad de Jesús en una religión que puede llevar su nombre, pero tampoco quisiéramos permitir que esta consideración eclipse su vida inspiradora o sustituya su mensaje salvador: la paternidad de Dios y la fraternidad de los hombres.

\par 
%\textsuperscript{(1670.6)}
\textsuperscript{149:2.5} Los que enseñan la religión de Jesús deberían acercarse a las otras religiones reconociendo las verdades que tienen en común (muchas de las cuales provienen directa o indirectamente del mensaje de Jesús) absteniéndose al mismo tiempo de recalcar demasiado las diferencias.

\par 
%\textsuperscript{(1671.1)}
\textsuperscript{149:2.6} En aquel momento concreto, la fama de Jesús se basaba principalmente en su reputación como sanador, pero esto no significa que continuara siendo así. A medida que pasaba el tiempo, se le buscaba cada vez más por su ayuda espiritual. Pero eran las curaciones físicas las que ejercían el atractivo más directo e inmediato sobre la gente común. A Jesús lo buscaban cada vez más las víctimas de la esclavitud moral y del agobio mental, y él les enseñaba invariablemente el camino de la liberación. Los padres buscaban su consejo sobre la manera de dirigir a sus hijos, y las madres le pedían ayuda para guiar a sus hijas. Los que estaban en las tinieblas acudían a él, y él les revelaba la luz de la vida. Siempre prestaba atención a las penas de la humanidad, y siempre ayudaba a los que buscaban su ministerio.

\par 
%\textsuperscript{(1671.2)}
\textsuperscript{149:2.7} Mientras que el Creador mismo estaba en la Tierra, encarnado en la similitud de la carne mortal, era inevitable que se produjeran algunas cosas extraordinarias. Pero nunca deberíais acercaros a Jesús a través de estos incidentes llamados milagrosos. Aprended a acercaros al milagro a través de Jesús, pero no cometáis el error de acercaros a Jesús a través del milagro. Esta recomendación está justificada, a pesar de que Jesús de Nazaret es el único fundador de una religión que ha realizado actos supermateriales en la Tierra.

\par 
%\textsuperscript{(1671.3)}
\textsuperscript{149:2.8} El rasgo más sorprendente y más revolucionario de la misión de Miguel en la Tierra fue su actitud hacia las mujeres. En una época y en una generación en las que se suponía que un hombre no podía saludar en un lugar público ni siquiera a su propia esposa, Jesús se atrevió a llevar consigo a mujeres como instructoras del evangelio durante su tercera gira por Galilea. Y tuvo el valor consumado de hacerlo a pesar de la enseñanza rabínica que proclamaba que <<era mejor quemar las palabras de la ley antes que entregárselas a las mujeres>>.

\par 
%\textsuperscript{(1671.4)}
\textsuperscript{149:2.9} En una sola generación, Jesús sacó a las mujeres del olvido irrespetuoso y de las faenas serviles de todos los siglos anteriores. Y es algo vergonzoso para la religión que se atrevió a llevar el nombre de Jesús que le haya faltado el valor moral de seguir este noble ejemplo en su actitud posterior hacia las mujeres.

\par 
%\textsuperscript{(1671.5)}
\textsuperscript{149:2.10} Cuando Jesús se mezclaba con la gente, todos lo encontraban completamente liberado de las supersticiones de la época. Estaba libre de prejuicios religiosos y nunca era intolerante. No había nada en su corazón que se pareciera al antagonismo social. Aunque se conformaba con lo que había de bueno en la religión de sus antepasados, no dudaba en hacer caso omiso de las tradiciones supersticiosas y esclavizantes inventadas por el hombre. Se atrevió a enseñar que las catástrofes de la naturaleza, los accidentes del tiempo y otros acontecimientos calamitosos no son azotes del juicio divino ni designios misteriosos de la Providencia. Denunció la devoción servil a las ceremonias sin sentido y mostró la falacia del culto materialista. Proclamó audazmente la libertad espiritual del hombre y se atrevió a enseñar que los mortales que viven en la carne son, de hecho y en verdad, hijos del Dios viviente.

\par 
%\textsuperscript{(1671.6)}
\textsuperscript{149:2.11} Jesús trascendió todas las enseñanzas de sus antepasados cuando sustituyó audazmente las manos limpias por los corazones puros como signo de la verdadera religión. Instaló la realidad en el lugar de la tradición y barrió todas las pretensiones de la vanidad y de la hipocresía. Y sin embargo, este intrépido hombre de Dios no dio rienda suelta a las críticas destructivas ni manifestó un completo desdén por las costumbres religiosas, sociales, económicas y políticas de su época. No era un revolucionario militante; era un evolucionista progresista. Sólo emprendía la destrucción de algo que \textit{existía} cuando ofrecía simultáneamente a sus semejantes la cosa superior que \textit{debía existir}.

\par 
%\textsuperscript{(1672.1)}
\textsuperscript{149:2.12} Jesús obtenía la obediencia de sus seguidores sin exigirla. De todos los hombres que recibieron su llamamiento personal, sólo tres rehusaron aceptar esta invitación a convertirse en sus discípulos. Ejercía un poder de atracción particular sobre los hombres, pero no era dictatorial. Inspiraba confianza, y nadie se sintió nunca ofendido por recibir una orden suya. Poseía una autoridad absoluta sobre sus discípulos, pero ninguno puso nunca objeciones. Permitía que sus seguidores le llamaran Maestro.

\par 
%\textsuperscript{(1672.2)}
\textsuperscript{149:2.13} El Maestro era admirado por todos los que se encontraban con él, excepto por los que tenían prejuicios religiosos muy arraigados o los que creían discernir un peligro político en sus enseñanzas. Los hombres se asombraban por la originalidad y el tono de autoridad de su enseñanza. Se maravillaban de su paciencia cuando trataba con los retrasados y los inoportunos que lo interrogaban. Inspiraba esperanza y confianza en el corazón de todos los que recibían su ministerio. Sólo le temían aquellos que no lo conocían, y sólo le odiaban aquellos que lo consideraban como el campeón de una verdad destinada a destruir el mal y el error que habían decidido mantener a toda costa en su corazón.

\par 
%\textsuperscript{(1672.3)}
\textsuperscript{149:2.14} Ejercía una influencia poderosa y particularmente fascinante tanto sobre sus amigos como sobre sus enemigos. Las multitudes lo seguían durante semanas enteras, únicamente para escuchar sus palabras benévolas y para observar su vida sencilla. Los hombres y las mujeres leales amaban a Jesús con un afecto casi sobrehumano, y cuanto más lo conocían, más lo amaban. Y todo esto sigue siendo verdad; incluso hoy y en todas las épocas futuras, cuanto más conozca el hombre a este Dios-hombre, más lo amará y lo seguirá.

\section*{3. La hostilidad de los jefes religiosos}
\par 
%\textsuperscript{(1672.4)}
\textsuperscript{149:3.1} A pesar de que la gente común acogía favorablemente a Jesús y sus enseñanzas, los jefes religiosos de Jerusalén estaban cada vez más alarmados y hostiles. Los fariseos habían formulado una teología sistemática y dogmática. Jesús era un instructor que enseñaba a medida que se presentaba la ocasión; no era un educador sistemático. Jesús enseñaba mediante parábolas, basándose más en la vida que en la ley. (Y cuando empleaba una parábola para ilustrar su mensaje, tenía la intención de utilizar \textit{una} sola característica de la historia con esa finalidad. Se pueden obtener muchas ideas falsas sobre las enseñanzas de Jesús cuando se intentan transformar sus parábolas en alegorías.)

\par 
%\textsuperscript{(1672.5)}
\textsuperscript{149:3.2} Los jefes religiosos de Jerusalén se estaban poniendo casi frenéticos a causa de la reciente conversión del joven Abraham y de la deserción de los tres espías, que habían sido bautizados por Pedro, y que ahora acompañaban a los evangelistas en esta segunda gira de predicación por Galilea. Los dirigentes judíos estaban cada vez más cegados por el miedo y los prejuicios, mientras que sus corazones se endurecían debido al rechazo continuo de las atractivas verdades del evangelio del reino. Cuando los hombres se niegan a recurrir al espíritu que reside en ellos, poco se puede hacer por modificar su actitud.

\par 
%\textsuperscript{(1672.6)}
\textsuperscript{149:3.3} Cuando Jesús se reunió por primera vez con los evangelistas en el campamento de Betsaida, al terminar su alocución les dijo: <<Debéis recordar que tanto física como mentalmente ---emocionalmente--- los hombres reaccionan de manera individual. La única cosa \textit{uniforme} que tienen los hombres es el espíritu interior. Aunque los espíritus divinos pueden variar un poco en la naturaleza y la magnitud de su experiencia, reaccionan de manera uniforme a todas las peticiones espirituales. La humanidad sólo podrá alcanzar la unidad y la fraternidad a través de este espíritu, y apelando a él>>. Pero muchos líderes de los judíos habían cerrado las puertas de su corazón al llamamiento espiritual del evangelio. A partir de este día, no dejaron de hacer planes y de conspirar para destruir al Maestro. Estaban convencidos de que Jesús tenía que ser detenido, condenado y ejecutado como delincuente religioso, como un violador de las enseñanzas cardinales de la sagrada ley judía.

\section*{4. El desarrollo de la gira de predicación}
\par 
%\textsuperscript{(1673.1)}
\textsuperscript{149:4.1} Jesús hizo muy poco trabajo público durante esta gira de predicación, pero dirigió muchas clases vespertinas para los creyentes en la mayoría de las ciudades y pueblos en los que residió ocasionalmente con Santiago y Juan. En una de estas sesiones vespertinas, uno de los evangelistas más jóvenes le hizo una pregunta a Jesús sobre la ira, y en su respuesta, el Maestro dijo entre otras cosas:

\par 
%\textsuperscript{(1673.2)}
\textsuperscript{149:4.2} <<La ira es una manifestación material que representa, de una manera general, la medida en que la naturaleza espiritual no ha logrado dominar las naturalezas intelectual y física combinadas. La ira indica vuestra falta de amor fraternal tolerante, más vuestra falta de dignidad y de autocontrol. La ira merma la salud, envilece la mente, y obstaculiza al instructor espiritual del alma del hombre. ¿No habéis leído en las Escrituras que `la ira mata al hombre necio' y que el hombre `se desgarra a sí mismo en su ira'? ¿Que `el que es lento en encolerizarse posee una gran comprensión,' mientras que `el que se irrita fácilmente exalta la insensatez'? Todos sabéis que `una respuesta dulce desvía el furor,' y que `las palabras ásperas despiertan la cólera'. `La discreción difiere la cólera' mientras que `el que no controla su propio yo se parece a una ciudad sin defensa y sin murallas'. `La ira es cruel y la cólera es ultrajante'. `Los hombres airados incitan a la disputa, mientras que los furiosos multiplican sus transgresiones'. `No seáis ligeros de espíritu, porque la cólera reposa en el seno de los necios'.>> Antes de terminar de hablar, Jesús dijo además: <<Que vuestro corazón esté tan dominado por el amor, que vuestro guía espiritual tenga pocas dificultades para liberaros de la tendencia a dejaros llevar por esos arranques de ira animal que son incompatibles con el estado de la filiación divina>>.

\par 
%\textsuperscript{(1673.3)}
\textsuperscript{149:4.3} En esta misma ocasión, el Maestro le habló al grupo sobre la conveniencia de poseer un carácter bien equilibrado. Reconoció que la mayoría de los hombres necesitaba consagrarse al dominio de alguna profesión, pero deploraba toda tendencia a la especialización excesiva, a volverse estrecho de ideas y limitado en las actividades de la vida. Llamó la atención sobre el hecho de que toda virtud, si es llevada al extremo, se puede convertir en un vicio. Jesús siempre predicó la moderación y enseñó la coherencia ---el ajuste de los problemas de la vida en su debida proporción. Señaló que un exceso de compasión y de piedad puede degenerar en una grave inestabilidad emocional; que el entusiasmo puede llevar al fanatismo. Mencionó a uno de sus antiguos asociados, cuya imaginación lo había llevado a empresas visionarias e irrealizables. Al mismo tiempo, los previno contra los peligros de la monotonía de una mediocridad demasiado conservadora.

\par 
%\textsuperscript{(1673.4)}
\textsuperscript{149:4.4} Luego, Jesús discurrió sobre los peligros de la valentía y de la fe, de cómo estas cualidades a veces conducen a las almas irreflexivas a la temeridad y a la presunción. También mostró cómo la prudencia y la discreción, llevadas demasiado lejos, conducen a la cobardía y al fracaso. Exhortó a sus oyentes a que se esforzaran por ser originales, pero evitando toda tendencia a la excentricidad. Abogó por una simpatía desprovista de sentimentalismo, y por una piedad sin beatería. Enseñó un respeto libre del miedo y de la superstición.

\par 
%\textsuperscript{(1674.1)}
\textsuperscript{149:4.5} Lo que impresionaba a sus compañeros no era tanto lo que Jesús enseñaba sobre el carácter equilibrado como el hecho de que su propia vida era una ilustración tan elocuente de su enseñanza. Vivió en medio de la tensión y de la tempestad, pero nunca vaciló. Sus enemigos le tendieron trampas continuamente, pero nunca lo cogieron. Los sabios y los eruditos intentaron ponerle zancadillas, pero no tropezó. Procuraron enredarlo en discusiones, pero sus respuestas eran siempre esclarecedoras, dignas y definitivas. Cuando interrumpían sus discursos con múltiples preguntas, sus respuestas eran siempre significativas y concluyentes. Nunca recurrió a tácticas indignas para enfrentarse a la continua presión de sus enemigos, que no dudaban en emplear todo tipo de mentiras, de injusticias y de iniquidades en sus ataques contra él.

\par 
%\textsuperscript{(1674.2)}
\textsuperscript{149:4.6} Aunque es verdad que muchos hombres y mujeres han de emplearse asiduamente en un oficio determinado para ganarse la vida, sin embargo es enteramente deseable que los seres humanos cultiven una amplia gama de conocimientos sobre la vida tal como se vive en la Tierra. Las personas realmente educadas no se conforman con permanecer en la ignorancia sobre la vida y las actividades de sus semejantes.

\section*{5. La lección sobre el contentamiento}
\par 
%\textsuperscript{(1674.3)}
\textsuperscript{149:5.1} Un día que Jesús estaba visitando al grupo de evangelistas que trabajaba bajo la supervisión de Simón Celotes, éste le preguntó al Maestro durante la conferencia nocturna: <<¿Por qué algunas personas están mucho más felices y contentas que otras? ¿Es el contentamiento un asunto de experiencia religiosa?>> En respuesta a la pregunta de Simón, Jesús dijo entre otras cosas:

\par 
%\textsuperscript{(1674.4)}
\textsuperscript{149:5.2} <<Simón, algunas personas son por naturaleza más felices que otras. Eso depende muchísimo de la buena voluntad del hombre a dejarse conducir y dirigir por el espíritu del Padre que vive dentro de él. ¿No has leído en las Escrituras las palabras del sabio: `El espíritu del hombre es la vela del Señor que examina todo su interior'? Y también que estos mortales conducidos así por el espíritu dicen: `Me conformo gustosamente con lo que tengo; sí, poseo una herencia excelente'. `Lo poco que posee un justo es mejor que las riquezas de muchos malvados,' porque `un hombre bueno obtiene la satisfacción de su propio interior'. `Un corazón alegre produce un semblante jovial y es una fiesta contínua. Es mejor tener un poco con veneración al Señor, que un gran tesoro con sus problemas incluídos. Es mejor una comida de legumbres con amor, que un buey engordado acompañado de odio. Es mejor poseer un poco con justicia, que grandes ingresos sin rectitud'. `Un corazón alegre hace bien como un medicamento'. `Es mejor tener un puñado con serenidad, que una gran abundancia con penas y vejación de espíritu'.>>

\par 
%\textsuperscript{(1674.5)}
\textsuperscript{149:5.3} <<Una gran parte de las penas del hombre provienen de la frustración de sus ambiciones y de las ofensas a su orgullo. Aunque los hombres tienen consigo mismos el deber de llevar la mejor vida posible en la Tierra, una vez que han hecho ese esfuerzo sincero, deberían aceptar su suerte con alegría y ejercitar su ingenio para sacar el mejor partido a lo que tienen entre sus manos. Demasiadas dificultades de los hombres tienen su origen en el temor que alberga su propio corazón. `El perverso huye sin que nadie lo persiga'. `Los perversos se parecen a un mar agitado, pues no puede detenerse, pero sus aguas arrojan cieno y lodo; no hay paz, dice Dios, para los perversos'.>>

\par 
%\textsuperscript{(1674.6)}
\textsuperscript{149:5.4} <<No busquéis pues una paz falsa y una alegría pasajera, sino más bien la seguridad de la fe y las garantías de la filiación divina, que dan la serenidad, el contentamiento y la alegría suprema en el espíritu>>.

\par 
%\textsuperscript{(1675.1)}
\textsuperscript{149:5.5} Jesús difícilmente consideraba este mundo como un <<valle de lágrimas>>. Más bien lo consideraba como <<el valle donde se forjan las almas>>, la esfera de nacimiento de los espíritus eternos e inmortales destinados a ascender al Paraíso.

\section*{6. El <<temor al Señor>>}
\par 
%\textsuperscript{(1675.2)}
\textsuperscript{149:6.1} Fue en Gamala, durante la conferencia de la tarde, donde Felipe dijo a Jesús: <<Maestro, ¿por qué las Escrituras nos enseñan que `temamos al Señor,' mientras que tú desearías que miráramos sin temor al Padre que está en los cielos? ¿Cómo podemos armonizar estas enseñanzas?>> Jesús contestó a Felipe, diciendo:

\par 
%\textsuperscript{(1675.3)}
\textsuperscript{149:6.2} <<Hijos míos, no me sorprende que hagáis estas preguntas. Al principio, el hombre sólo podía aprender el respeto a través del miedo, pero yo he venido para revelar el amor del Padre con el fin de que os sintáis inducidos a adorar al Eterno por el atractivo del reconocimiento afectuoso de un hijo, y la reciprocidad del amor profundo y perfecto del Padre. Quisiera liberaros de la esclavitud de poneros, por miedo servil, al servicio fastidioso de un Dios-Rey celoso e iracundo. Quisiera enseñaros la relación de Padre a hijo entre Dios y el hombre, para que os sintáis conducidos alegremente a la libre adoración, sublime y celeste, de un Padre-Dios amoroso, justo y misericordioso>>.

\par 
%\textsuperscript{(1675.4)}
\textsuperscript{149:6.3} <<El `temor al Señor' ha tenido diferentes significados a través de los tiempos; empezó con el miedo, ha pasado por la angustia y el terror, y ha llegado hasta el temor y el respeto. Partiendo del respeto, ahora quisiera elevaros, a través del reconocimiento, de la comprensión y de la apreciación, hasta el \textit{amor}. Cuando el hombre sólo reconoce las obras de Dios, es inducido a temer al Supremo; pero cuando el hombre empieza a comprender y a experimentar la personalidad y el carácter del Dios viviente, se siente inducido a amar cada vez más a este bueno y perfecto Padre universal y eterno. Este cambio de relación entre el hombre y Dios es precisamente lo que constituye la misión del Hijo del Hombre en la Tierra>>.

\par 
%\textsuperscript{(1675.5)}
\textsuperscript{149:6.4} <<Los hijos inteligentes no temen a su padre a fin de poder recibir buenos dones de sus manos; pero una vez que ya han recibido abundantemente las buenas cosas otorgadas por los dictados del afecto del padre por sus hijos e hijas, estos hijos muy amados se sienten inducidos a amar a su padre en respuesta al reconocimiento y a la apreciación de tan generosa beneficencia. La bondad de Dios conduce al arrepentimiento; la beneficencia de Dios conduce al servicio; la misericordia de Dios conduce a la salvación; mientras que el amor de Dios conduce a la adoración inteligente y generosa>>.

\par 
%\textsuperscript{(1675.6)}
\textsuperscript{149:6.5} <<Vuestros antepasados temían a Dios porque era poderoso y misterioso. Vosotros lo adoraréis porque es magnífico en amor, abundante en misericordia y glorioso en verdad. El poder de Dios engendra el temor en el corazón del hombre, pero la nobleza y la rectitud de su personalidad producen la veneración, el amor y la adoración voluntaria. Un hijo obediente y afectuoso no le tiene miedo ni terror a su padre, aunque sea poderoso y noble. He venido al mundo para sustituir el miedo por el amor, la tristeza por la alegría, el temor por la confianza, la esclavitud servil y las ceremonias sin significado por el servicio amoroso y la adoración agradecida. Pero continúa siendo cierto para los que se encuentran en las tinieblas que `el temor al Señor es el comienzo de la sabiduría'. Cuando la luz brille más plenamente, los hijos de Dios se sentirán inducidos a alabar al Infinito por lo que él \textit{es}, en lugar de temerlo por lo que \textit{hace}>>.

\par 
%\textsuperscript{(1675.7)}
\textsuperscript{149:6.6} <<Cuando los hijos son jóvenes e irreflexivos, se les debe reprender necesariamente para que honren a sus padres; pero cuando crecen y empiezan a apreciar mejor los beneficios del ministerio y de la protección de sus padres, un respeto comprensivo y un afecto creciente los eleva a ese nivel de experiencia en el que aman realmente a sus padres por lo que son, más que por lo que han hecho. El padre ama de manera natural a su hijo, pero el hijo debe desarrollar su amor por el padre, empezando por el miedo de lo que el padre puede hacer, y continuando por el temor, el terror, la dependencia y el respeto, hasta la consideración agradecida y afectuosa del amor>>.

\par 
%\textsuperscript{(1676.1)}
\textsuperscript{149:6.7} <<Se os ha enseñado que debéis `temer a Dios y guardar sus mandamientos, porque en eso reside todo el deber del hombre'. Pero yo he venido para daros un mandamiento nuevo y superior. Quisiera enseñaros a `amar a Dios y a aprender a hacer su voluntad, porque éste es el privilegio más elevado de los hijos liberados de Dios'. A vuestros padres les enseñaron a `temer a Dios ---al Rey Todopoderoso'. Y yo os enseño: `Amad a Dios ---al Padre totalmente misericordioso'.>>

\par 
%\textsuperscript{(1676.2)}
\textsuperscript{149:6.8} <<En el reino de los cielos, que he venido a proclamar, no hay un rey elevado y poderoso; este reino es una familia divina. El centro y el jefe, universalmente reconocido y adorado sin reservas, de esta extensa fraternidad de seres inteligentes, es mi Padre y vuestro Padre. Yo soy su Hijo, y vosotros también sois sus hijos. Por consiguiente, es eternamente cierto que vosotros y yo somos hermanos en el estado celestial, y mucho más desde que nos hemos vuelto hermanos en la carne, en la vida terrenal. Dejad pues de temer a Dios como a un rey o de servirle como a un amo; aprended a venerarlo como Creador; a honrarlo como al Padre de vuestra juventud espiritual; a amarlo como a un defensor misericordioso; y finalmente, a adorarlo como al Padre amoroso y omnisapiente de vuestra comprensión y apreciación espirituales más maduras>>.

\par 
%\textsuperscript{(1676.3)}
\textsuperscript{149:6.9} <<Vuestros conceptos erróneos del Padre que está en los cielos dan origen a vuestras ideas falsas sobre la humildad y a una gran parte de vuestra hipocresía. El hombre puede ser un gusano de tierra por su naturaleza y origen, pero cuando está habitado por el espíritu de mi Padre, ese hombre se vuelve divino en su destino. El espíritu que mi Padre ha otorgado regresará con toda seguridad a la fuente divina y al nivel universal de su origen, y el alma humana del hombre mortal, que se habrá convertido en la hija renacida de este espíritu interior, se elevará ciertamente con el espíritu divino hasta la presencia misma del Padre eterno>>.

\par 
%\textsuperscript{(1676.4)}
\textsuperscript{149:6.10} <<En verdad, la humildad le conviene al hombre mortal que recibe todos estos dones del Padre que está en los cielos, aunque hay una dignidad divina que está ligada a todos estos candidatos, por la fe, a la ascensión eterna del reino celestial. Las prácticas sin sentido y serviles de una humildad ostentosa y falsa son incompatibles con la apreciación de la fuente de vuestra salvación y con el reconocimiento del destino de vuestras almas nacidas del espíritu. La humildad ante Dios es totalmente apropiada en el fondo de vuestro corazón; la mansedumbre delante de los hombres es loable; pero la hipocresía de una humildad consciente y deseosa de llamar la atención es infantil e indigna de los hijos iluminados del reino>>.

\par 
%\textsuperscript{(1676.5)}
\textsuperscript{149:6.11} <<Hacéis bien en ser dóciles ante Dios y en controlaros delante de los hombres, pero que vuestra mansedumbre sea de origen espiritual, y no la exhibición autoengañosa de un sentido consciente de superioridad presuntuosa. El profeta habló juiciosamente cuando dijo: `Caminad humildemente con Dios' porque, aunque el Padre celestial es el Infinito y el Eterno, también habita `en aquel que tiene una mente contrita y un espíritu humilde'. Mi Padre desdeña el orgullo, detesta la hipocresía y aborrece la iniquidad. Para recalcar el valor de la sinceridad y la confianza perfecta en el sostén amoroso y en la guía fiel del Padre celestial, me he referido con mucha frecuencia a los niños, con el fin de ilustrar la actitud mental y la reacción espiritual que son tan esenciales para que el hombre mortal acceda a las realidades espirituales del reino de los cielos>>.

\par 
%\textsuperscript{(1677.1)}
\textsuperscript{149:6.12} <<El profeta Jeremías describió bien a muchos mortales cuando dijo: `Estáis cerca de Dios en la boca, pero lejos de él en el corazón'. ¿Y no habéis leído también esa terrible advertencia del profeta que dijo: `Sus sacerdotes enseñan por un salario y sus profetas adivinan por dinero. Al mismo tiempo, manifiestan piedad y proclaman que el Señor está con ellos'? ¿No habéis sido bien advertidos contra los que `hablan de paz con sus vecinos, estando la maldad en su corazón', contra los que `adulan con los labios, mientras que su corazón actúa con doblez'? De todas las penas de un hombre confiado, ninguna es más terrible que la de ser `herido en la casa de un amigo en quien confía'.>>

\section*{7. El regreso a Betsaida}
\par 
%\textsuperscript{(1677.2)}
\textsuperscript{149:7.1} Después de consultar con Simón Pedro y de recibir la aprobación de Jesús, Andrés había indicado a David, en Betsaida, que enviara a unos mensajeros a los diversos grupos de predicadores con la instrucción de que finalizaran la gira y regresaran a Betsaida durante la jornada del jueves 30 de diciembre. A la hora de la cena de este día lluvioso, todo el grupo apostólico y los educadores evangelistas habían llegado a la casa de Zebedeo.

\par 
%\textsuperscript{(1677.3)}
\textsuperscript{149:7.2} El grupo permaneció junto hasta el sábado, alojándose en los hogares de Betsaida y de la ciudad cercana de Cafarnaúm; después, a todo el grupo se le concedió dos semanas de vacaciones para ir a ver a sus familias, visitar a sus amigos o ir a pescar. Los dos o tres días que estuvieron juntos en Betsaida fueron verdaderamente divertidos e inspiradores; incluso los educadores más antiguos se sintieron edificados escuchando a los jóvenes predicadores relatar sus experiencias.

\par 
%\textsuperscript{(1677.4)}
\textsuperscript{149:7.3} De los 117 evangelistas que participaron en esta segunda gira de predicación por Galilea, unos setenta y cinco solamente sobrevivieron a la prueba de la experiencia real, y estuvieron disponibles para que se les asignara una tarea al final de las dos semanas de descanso. Jesús permaneció en la casa de Zebedeo con Andrés, Pedro, Santiago y Juan, y pasó mucho tiempo conferenciando con ellos sobre el bienestar y la expansión del reino.


\chapter{Documento 150. La tercera gira de predicación}
\par 
%\textsuperscript{(1678.1)}
\textsuperscript{150:0.1} EL DOMINGO por la tarde 16 de enero del año 29, Abner llegó a Betsaida con los apóstoles de Juan, y al día siguiente mantuvo una conferencia conjunta con Andrés y los apóstoles de Jesús. Abner y sus asociados establecieron su cuartel general en Hebrón y cogieron la costumbre de venir periódicamente a Betsaida para este tipo de conferencias.

\par 
%\textsuperscript{(1678.2)}
\textsuperscript{150:0.2} Entre las numerosas cuestiones que se consideraron en esta conferencia conjunta se encontraba la práctica de ungir a los enfermos con ciertos tipos de aceite en unión con unas oraciones para la curación. Jesús rehusó de nuevo participar en estas discusiones o expresar su opinión sobre las conclusiones. Los apóstoles de Juan siempre habían utilizado el aceite de ungir en su ministerio hacia los enfermos y los afligidos, y trataron de establecer que esta práctica fuera uniforme para ambos grupos, pero los apóstoles de Jesús se negaron a someterse a esta regla.

\par 
%\textsuperscript{(1678.3)}
\textsuperscript{150:0.3} El martes 18 de enero, los evangelistas que habían pasado la prueba, unos setenta y cinco en total, se reunieron con los veinticuatro en la casa de Zebedeo en Betsaida, antes de ser enviados a la tercera gira de predicación por Galilea. Esta tercera misión se prolongó durante siete semanas.

\par 
%\textsuperscript{(1678.4)}
\textsuperscript{150:0.4} Los evangelistas fueron enviados en grupos de cinco, mientras que Jesús y los doce viajaron juntos la mayor parte del tiempo; los apóstoles salían de dos en dos para bautizar a los creyentes cuando lo requería la ocasión. Durante un período de casi tres semanas, Abner y sus asociados trabajaron también con los grupos de evangelistas, aconsejándolos y bautizando a los creyentes. Visitaron Magdala, Tiberiades, Nazaret y todas las principales ciudades y pueblos del centro y sur de Galilea, todos los lugares visitados anteriormente y muchos más. Éste fue su último mensaje para Galilea, exceptuando las regiones del norte.

\section*{1. El cuerpo de mujeres evangelistas}
\par 
%\textsuperscript{(1678.5)}
\textsuperscript{150:1.1} De todos los actos audaces que Jesús efectuó en relación con su carrera terrestre, el más asombroso fue su anuncio repentino la tarde del 16 de enero: <<Mañana seleccionaremos a diez mujeres para trabajar en el ministerio del reino>>. Al empezar el período de dos semanas durante las cuales los apóstoles y los evangelistas iban a estar ausentes de Betsaida debido a sus vacaciones, Jesús le rogó a David que llamara a sus padres para que regresaran a su hogar, y que enviara a unos mensajeros para convocar en Betsaida a diez mujeres devotas que habían servido en la administración del antiguo campamento y la enfermería de tiendas. Todas estas mujeres habían escuchado la enseñanza impartida a los jóvenes evangelistas, pero nunca se les había ocurrido, ni a ellas ni a sus instructores, que Jesús se atrevería a encargar a unas mujeres la enseñanza del evangelio del reino y la atención a los enfermos. Estas diez mujeres escogidas y autorizadas por Jesús eran: Susana, la hija del antiguo chazán de la sinagoga de Nazaret; Juana, la esposa de Chuza, el administrador de Herodes Antipas; Isabel, la hija de un judío rico de Tiberiades y Séforis; Marta, la hermana mayor de Andrés y Pedro; Raquel, la cuñada de Judá, el hermano carnal del Maestro; Nasanta, la hija de Elman, el médico sirio; Milca, una prima del apóstol Tomás; Rut, la hija mayor de Mateo Leví; Celta, la hija de un centurión romano; y Agaman, una viuda de Damasco. Posteriormente, Jesús añadió dos mujeres más a este grupo: María Magdalena y Rebeca, la hija de José de Arimatea.

\par 
%\textsuperscript{(1679.1)}
\textsuperscript{150:1.2} Jesús autorizó a estas mujeres para que establecieran su propia organización, y ordenó a Judas que les proporcionara fondos para equiparse y comprar animales de carga. Las diez eligieron a Susana como jefa y a Juana como tesorera. A partir de este momento se procuraron sus propios fondos; nunca más recurrieron a la ayuda de Judas.

\par 
%\textsuperscript{(1679.2)}
\textsuperscript{150:1.3} En una época como ésta, en la que ni siquiera se permitía a las mujeres permanecer en el piso principal de la sinagoga (estaban confinadas a la galería de las mujeres), era más que sorprendente observar que se las reconocía como instructoras autorizadas del nuevo evangelio del reino. El encargo que Jesús confió a estas diez mujeres, al seleccionarlas para la enseñanza y el ministerio del evangelio, fue la proclamación de emancipación que liberaba a todas las mujeres para todos los tiempos; los hombres ya no debían considerar a las mujeres como espiritualmente inferiores a ellos. Fue una auténtica conmoción, incluso para los doce apóstoles. A pesar de que habían escuchado muchas veces decir al Maestro que <<en el reino de los cielos no hay ni ricos ni pobres, ni libres ni esclavos, ni hombres ni mujeres, sino que todos son igualmente los hijos e hijas de Dios>>, se quedaron literalmente pasmados cuando Jesús propuso autorizar formalmente a estas diez mujeres como instructoras religiosas, e incluso permitirles que viajaran con ellos. Todo el país se conmovió por esta manera de proceder, y los enemigos de Jesús sacaron un gran provecho de esta decisión; pero por todas partes, las mujeres que creían en la buena nueva respaldaron firmemente a sus hermanas escogidas, y expresaron su más plena aprobación a este reconocimiento tardío del lugar de la mujer en el trabajo religioso. Inmediatamente después de la partida del Maestro, los apóstoles pusieron en práctica esta liberación de las mujeres, otorgándoles el debido reconocimiento, pero las generaciones posteriores volvieron a caer en las antiguas costumbres. Durante los primeros tiempos de la iglesia cristiana, las mujeres instructoras y ministras fueron llamadas \textit{diaconisas}, y se les concedió un reconocimiento general. Pero Pablo, a pesar del hecho de que admitía todo esto en teoría, nunca lo incorporó realmente en su propia actitud y le resultó personalmente difícil ponerlo en práctica.

\section*{2. La parada en Magdala}
\par 
%\textsuperscript{(1679.3)}
\textsuperscript{150:2.1} Cuando el grupo apostólico salió de Betsaida, las mujeres viajaron en la retaguardia. Durante las conferencias, siempre se sentaban en grupo enfrente y a la derecha del orador. Cada vez más mujeres se habían hecho creyentes en el evangelio del reino, y cuando habían deseado mantener una conversación personal con Jesús o con uno de los apóstoles, se habían originado muchas dificultades y un sin fin de situaciones embarazosas. Ahora, todo esto había cambiado. Cuando cualquier mujer creyente deseaba ver al Maestro o entrevistarse con los apóstoles, iba a ver a Susana, y acompañada por una de las doce mujeres evangelistas, se dirigían enseguida a la presencia del Maestro o de uno de sus apóstoles.

\par 
%\textsuperscript{(1680.1)}
\textsuperscript{150:2.2} Fue en Magdala donde las mujeres demostraron por primera vez su utilidad y justificaron la sabiduría de haberlas escogido. Andrés había impuesto a sus asociados unas reglas más bien estrictas en lo referente al trabajo personal con las mujeres, especialmente con aquellas de conducta dudosa. Cuando el grupo llegó a Magdala, estas diez mujeres evangelistas pudieron entrar libremente en los lugares depravados y predicar directamente la buena nueva a todas sus inquilinas. Y cuando visitaban a los enfermos, estas mujeres eran capaces de acercarse íntimamente, en su ministerio, a sus hermanas afligidas. A consecuencia del servicio efectuado en este lugar por estas diez mujeres (más tarde conocidas como las doce mujeres), María Magdalena fue ganada para el reino. A través de una serie de desventuras, y como consecuencia de la actitud de la sociedad respetable hacia las mujeres que cometían estos errores de juicio, esta mujer había ido a parar a uno de los lugares ignominiosos de Magdala. Marta y Raquel fueron las que indicaron claramente a María que las puertas del reino estaban abiertas incluso para las personas como ella. María creyó en la buena nueva y fue bautizada por Pedro al día siguiente.

\par 
%\textsuperscript{(1680.2)}
\textsuperscript{150:2.3} María Magdalena se convirtió en la educadora más eficaz del evangelio, dentro de este grupo de doce mujeres evangelistas. Fue seleccionada para este servicio en Jotapata, junto con Rebeca, unas cuatro semanas después de su conversión. Durante el resto de la vida de Jesús en la Tierra, María, Rebeca y sus compañeras de grupo continuaron trabajando fiel y eficazmente para iluminar y elevar a sus hermanas oprimidas. Y cuando el último y trágico episodio del drama de la vida de Jesús se estaba representando, a pesar de que todos los apóstoles, salvo uno, habían huido, todas estas mujeres estuvieron presentes, y ninguna de ellas lo negó ni lo traicionó.

\section*{3. Un sábado en Tiberiades}
\par 
%\textsuperscript{(1680.3)}
\textsuperscript{150:3.1} Andrés, siguiendo las instrucciones de Jesús, había responsabilizado a las mujeres de los oficios del grupo apostólico para el sábado. Esto significaba, naturalmente, que no se podían celebrar en la nueva sinagoga. Las mujeres eligieron a Juana para que se encargara de esta contingencia, y la reunión se celebró en la sala de banquetes del nuevo palacio de Herodes, ya que Herodes se encontraba residiendo en Julias, en Perea. Juana leyó en las Escrituras unos pasajes sobre la obra de la mujer en la vida religiosa de Israel, haciendo referencia a Miriam, Débora, Ester y otras.

\par 
%\textsuperscript{(1680.4)}
\textsuperscript{150:3.2} A una hora avanzada de aquella noche, Jesús dio al grupo reunido una charla memorable sobre <<La magia y la superstición>>. En aquellos tiempos, la aparición de una estrella brillante y supuestamente nueva era considerada como el signo de que un gran hombre había nacido en la Tierra. Como se había observado recientemente una estrella de este tipo, Andrés le preguntó a Jesús si estas creencias estaban bien fundadas. En su larga respuesta a la pregunta de Andrés, el Maestro emprendió un examen completo de todo el tema de la superstición humana. La exposición que Jesús efectuó en esta ocasión se puede resumir, en lenguaje moderno, de la manera siguiente:

\par 
%\textsuperscript{(1680.5)}
\textsuperscript{150:3.3} 1. El camino que siguen las estrellas en el cielo no tiene absolutamente nada que ver con los acontecimientos de la vida humana en la Tierra. La astronomía es una ocupación adecuada de la ciencia, pero la astrología es una masa de errores supersticiosos que no tienen ningún sitio en el evangelio del reino.

\par 
%\textsuperscript{(1680.6)}
\textsuperscript{150:3.4} 2. El examen de los órganos internos de un animal recién degollado no puede revelar nada sobre el tiempo atmosférico, los acontecimientos futuros o el resultado de los asuntos humanos.

\par 
%\textsuperscript{(1680.7)}
\textsuperscript{150:3.5} 3. Los espíritus de los muertos no regresan para comunicarse con sus familiares o con sus antiguos amigos todavía vivos.

\par 
%\textsuperscript{(1681.1)}
\textsuperscript{150:3.6} 4. Los amuletos y las reliquias son impotentes para curar las enfermedades, evitar los desastres o influir en los malos espíritus; la creencia en todos estos medios materiales para influir sobre el mundo espiritual no es más que una vulgar superstición.

\par 
%\textsuperscript{(1681.2)}
\textsuperscript{150:3.7} 5. Echarlo a suertes quizás sea una manera útil de resolver muchas dificultades menores, pero no es un método destinado a descubrir la voluntad divina. Los resultados que se obtienen así son simplemente el producto de la casualidad material. El único medio de comulgar con el mundo espiritual está incluido en la dotación espiritual de la humanidad, el espíritu interior del Padre, junto con el espíritu derramado por el Hijo y la influencia omnipresente del Espíritu Infinito.

\par 
%\textsuperscript{(1681.3)}
\textsuperscript{150:3.8} 6. La adivinación, la hechicería y la brujería son supersticiones de las mentes ignorantes, como también lo son las ilusiones de la magia. La creencia en los números mágicos, en los pronósticos de buena suerte y en los presagios de mala suerte, es una pura superstición sin ningún fundamento.

\par 
%\textsuperscript{(1681.4)}
\textsuperscript{150:3.9} 7. La interpretación de los sueños es ampliamente un sistema supersticioso e infundado de especulaciones ignorantes y fantásticas. El evangelio del reino no ha de tener nada en común con los sacerdotes adivinos de la religión primitiva.

\par 
%\textsuperscript{(1681.5)}
\textsuperscript{150:3.10} 8. Los espíritus del bien o del mal no pueden residir dentro de los símbolos materiales de arcilla, madera o metal; los ídolos no son nada más que el material con el que están fabricados.

\par 
%\textsuperscript{(1681.6)}
\textsuperscript{150:3.11} 9. Las prácticas de los encantadores, los brujos, los magos y los hechiceros provienen de las supersticiones de los egipcios, los asirios, los babilonios y los antiguos cananeos. Los amuletos y todas las clases de encantamientos son inútiles tanto para conseguir la protección de los buenos espíritus como para desviar a los supuestos espíritus impuros.

\par 
%\textsuperscript{(1681.7)}
\textsuperscript{150:3.12} 10. Jesús desenmascaró y censuró la creencia de sus oyentes en los encantamientos, las ordalías, los hechizos, las maldiciones, los signos, las mandrágoras, las cuerdas anudadas y todas las demás formas de superstición ignorante y esclavizante.

\section*{4. El envío de los apóstoles de dos en dos}
\par 
%\textsuperscript{(1681.8)}
\textsuperscript{150:4.1} A la tarde siguiente, después de reunir a los doce apóstoles, a los apóstoles de Juan y al grupo recién autorizado de mujeres, Jesús dijo: <<Podéis ver por vosotros mismos que la cosecha es abundante, pero que los obreros son pocos. Así pues, oremos todos al Señor de la cosecha para que envíe aún más obreros a sus campos. Mientras yo me quedo aquí para animar e instruir a los educadores más jóvenes, quisiera enviar a los más antiguos de dos en dos para que pasen rápidamente por toda Galilea predicando el evangelio del reino, mientras que aún se puede hacer de manera cómoda y pacífica>>. Luego designó a las parejas de apóstoles tal como él deseaba que salieran, y fueron las siguientes: Andrés y Pedro, Santiago y Juan Zebedeo, Felipe y Natanael, Tomás y Mateo, Santiago y Judas Alfeo, Simón Celotes y Judas Iscariote.

\par 
%\textsuperscript{(1681.9)}
\textsuperscript{150:4.2} Jesús fijó la fecha en que se encontraría con los doce en Nazaret, y al separarse dijo: <<Durante esta misión, no vayáis a ninguna ciudad de los gentiles ni tampoco a Samaria; id más bien donde están las ovejas perdidas de la casa de Israel. Predicad el evangelio del reino y proclamad la verdad salvadora de que el hombre es un hijo de Dios. Recordad que el discípulo difícilmente está por encima de su maestro y que un siervo no es más grande que su señor. Es suficiente con que el discípulo sea igual a su maestro y el siervo llegue a ser como su señor. Si alguna gente se ha atrevido a calificar al dueño de la casa de asociado de Belcebú, ¡con cuánta más razón considerarán de esa manera a la gente de su casa! Pero no tenéis que temer a estos enemigos incrédulos. Os aseguro que no hay nada tan encubierto que no se pueda revelar; no hay nada oculto que no se pueda conocer. Lo que os he enseñado en privado, predicadlo en público con sabiduría. Lo que os he revelado dentro de la casa, proclamadlo a su debido tiempo desde los tejados. Os lo digo, amigos y discípulos míos, no temáis a los que pueden matar el cuerpo, pero no son capaces de destruir el alma; poned más bien vuestra confianza en Aquel que es capaz de sostener el cuerpo y de salvar el alma>>.

\par 
%\textsuperscript{(1682.1)}
\textsuperscript{150:4.3} <<¿No se venden dos gorriones por un céntimo? Y sin embargo os declaro que ninguno de ellos está olvidado a los ojos de Dios. ¿No sabéis que incluso los cabellos de vuestras cabezas están todos contados? Así pues, no temáis; vosotros valéis más que una gran cantidad de gorriones. No os avergoncéis de mi enseñanza; salid a proclamar la paz y la buena voluntad, pero no os engañéis ---la paz no siempre acompañará vuestra predicación. He venido para traer la paz a la Tierra, pero cuando los hombres rechazan mi regalo, se producen divisiones y disturbios. Cuando toda una familia recibe el evangelio del reino, la paz permanece realmente en esa casa; pero cuando algunos miembros de la familia entran en el reino y otros rechazan el evangelio, una división así sólo puede producir pena y tristeza. Trabajad seriamente para salvar a la familia entera, a fin de que un hombre no tenga por enemigos a los miembros de su propia casa. Pero cuando hayáis hecho todo lo posible por todos los miembros de cada familia, os declaro que cualquiera que ame a su padre o a su madre más que a este evangelio, no es digno del reino>>.

\par 
%\textsuperscript{(1682.2)}
\textsuperscript{150:4.4} Después de haber escuchado estas palabras, los doce se prepararon para partir. No volvieron a verse hasta el momento en que se reunieron en Nazaret para encontrarse con Jesús y los otros discípulos, tal como el Maestro lo había dispuesto.

\section*{5. ¿Qué debo hacer para salvarme?}
\par 
%\textsuperscript{(1682.3)}
\textsuperscript{150:5.1} Una tarde en Sunem, después de que los apóstoles de Juan hubieran regresado a Hebrón y los apóstoles de Jesús hubieran sido enviados de dos en dos, el Maestro estaba ocupado en enseñar a un grupo de doce de los evangelistas más jóvenes que trabajaban bajo la dirección de Jacobo, junto con las doce mujeres, cuando Raquel le hizo a Jesús la pregunta siguiente: <<Maestro, ¿qué debemos responder cuando las mujeres nos preguntan: Qué debo hacer para salvarme?>> Cuando Jesús escuchó esta pregunta, respondió:

\par 
%\textsuperscript{(1682.4)}
\textsuperscript{150:5.2} <<Cuando los hombres y las mujeres os pregunten qué deben hacer para salvarse, vosotras contestaréis: Creed en este evangelio del reino; aceptad el perdón divino. Reconoced, por la fe, al espíritu interno de Dios, cuya aceptación os convierte en hijos de Dios. ¿No habéis leído en las Escrituras el pasaje que dice: `Mi rectitud y mi fuerza residen en el Señor?' Y también allí donde el Padre dice: `Mi justicia se acerca; mi salvación se ha hecho pública y mis brazos envolverán a mi pueblo'. `Mi alma se regocijará en el amor de mi Dios, porque me ha vestido con las vestiduras de la salvación y me ha cubierto con la túnica de su rectitud'. ¿No habéis leído también, refiriéndose al Padre, que su nombre `será llamado el Señor de nuestra rectitud?' `Quitaos los harapos sucios de la presunción y vestid a mi hijo con la túnica de la rectitud divina y de la salvación eterna'. Es eternamente cierto que `el justo vivirá por su fe'. La entrada en el reino del Padre es totalmente libre, pero el progreso ---el crecimiento en la gracia--- es indispensable para permanecer allí>>.

\par 
%\textsuperscript{(1682.5)}
\textsuperscript{150:5.3} <<La salvación es el don del Padre y es revelada por sus Hijos. Su aceptación, por la fe, os convierte en partícipes de la naturaleza divina, en hijos o hijas de Dios. Por la fe, estáis justificadas; por la fe, sois salvadas; y por esta misma fe, avanzaréis eternamente en el camino de la perfección progresiva y divina. Abraham fue justificado por la fe y tomó conciencia de la salvación gracias a las enseñanzas de Melquisedek. A lo largo de todos los tiempos, esta misma fe ha salvado a los hijos de los hombres, pero ahora un Hijo ha venido del Padre para hacer más real y aceptable la salvación>>.

\par 
%\textsuperscript{(1683.1)}
\textsuperscript{150:5.4} Cuando Jesús terminó de hablar, los que habían escuchado estas palabras benévolas sintieron un gran regocijo, y en los días que siguieron, todos continuaron proclamando el evangelio del reino con una nueva fuerza y con una energía y un entusiasmo renovados. Las mujeres se regocijaron aún más al saber que estaban incluidas en estos planes para establecer el reino en la Tierra.

\par 
%\textsuperscript{(1683.2)}
\textsuperscript{150:5.5} Al resumir su declaración final, Jesús dijo: <<No podéis comprar la salvación; no podéis ganar la rectitud. La salvación es un don de Dios, y la rectitud es el fruto natural de la vida nacida del espíritu, la vida de filiación en el reino. No vais a salvaros porque viváis una vida de rectitud, sino que viviréis una vida de rectitud porque ya habéis sido salvados, porque habéis reconocido la filiación como un don de Dios, y el servicio en el reino como la delicia suprema de la vida en la Tierra. Cuando los hombres creen en este evangelio, que es una revelación de la bondad de Dios, se sienten inducidos a arrepentirse voluntariamente de todos los pecados conocidos. La realización de la filiación es incompatible con el deseo de pecar. Los creyentes en el reino tienen hambre de rectitud y sed de perfección divina>>.

\section*{6. Las lecciones vespertinas}
\par 
%\textsuperscript{(1683.3)}
\textsuperscript{150:6.1} En las discusiones de la tarde, Jesús habló de muchos temas. Durante el resto de esta gira ---antes de que todos se reunieran en Nazaret--- trató de <<El amor de Dios>>, <<Los sueños y las visiones>>, <<La malicia>>, <<La humildad y la mansedumbre>>, <<El coraje y la lealtad>>, <<La música y la adoración>>, <<El servicio y la obediencia>>, <<El orgullo y la presunción>>, <<La relación entre el perdón y el arrepentimiento>>, <<La paz y la perfección>>, <<La calumnia y la envidia>>, <<El mal, el pecado y la tentación>>, <<Las dudas y la incredulidad>>, <<La sabiduría y la adoración>>. Como los apóstoles más antiguos estaban ausentes, estos grupos más jóvenes de hombres y mujeres participaron más libremente en estos debates con el Maestro.

\par 
%\textsuperscript{(1683.4)}
\textsuperscript{150:6.2} Después de pasar dos o tres días con un grupo de doce evangelistas, Jesús se desplazaba para reunirse con otro grupo, y los mensajeros de David le informaban del paradero y de los movimientos de todos estos trabajadores. Como ésta era su primera gira, las mujeres permanecieron una buena parte del tiempo con Jesús. Cada uno de estos grupos estaba plenamente informado del desarrollo de la gira gracias al servicio de los mensajeros, y la recepción de noticias de los otros grupos siempre era una fuente de estímulo para estos trabajadores dispersos y separados.

\par 
%\textsuperscript{(1683.5)}
\textsuperscript{150:6.3} Antes de separarse, se había acordado que los doce apóstoles, junto con los evangelistas y el cuerpo de mujeres, se congregarían en Nazaret el viernes 4 de marzo para reunirse con el Maestro. En consecuencia, alrededor de esta fecha, los diversos grupos de apóstoles y de evangelistas empezaron a dirigirse desde todas las partes de la Galilea central y meridional hacia Nazaret. A media tarde, Andrés y Pedro, los últimos en llegar, habían entrado en el campamento preparado por los primeros que llegaron y situado en las altas tierras al norte de la ciudad. Ésta era la primera vez que Jesús visitaba Nazaret desde el comienzo de su ministerio público.

\section*{7. La estancia en Nazaret}
\par 
%\textsuperscript{(1683.6)}
\textsuperscript{150:7.1} Este viernes por la tarde, Jesús se paseó por Nazaret totalmente desapercibido y sin ser reconocido. Pasó por la casa de su infancia y por el taller de carpintería y permaneció media hora en la colina donde tanto disfrutaba cuando era un muchacho. Desde el día en que Juan lo bautizó en el Jordán, el Hijo del Hombre no había sentido conmoverse en su alma tal cantidad de emociones humanas. Mientras bajaba de la montaña, escuchó los sonidos familiares del toque de trompeta que anunciaba la puesta del Sol, tal como los había escuchado tantísimas veces cuando era un niño que crecía en Nazaret. Antes de volver al campamento, pasó por la sinagoga donde había ido a la escuela, y se abandonó mentalmente a numerosas reminiscencias de la época de su infancia. Horas antes, Jesús había enviado a Tomás para que se pusiera de acuerdo con el jefe de la sinagoga a fin de poder predicar en los oficios matutinos del sábado.

\par 
%\textsuperscript{(1684.1)}
\textsuperscript{150:7.2} La gente de Nazaret nunca había sido famosa por su religiosidad ni por su manera recta de vivir. Con el transcurso de los años, este pueblo se había contaminado cada vez más con los bajos criterios morales de la cercana ciudad de Séforis. Durante toda la juventud y los primeros años de la vida adulta de Jesús, las opiniones sobre él habían estado divididas en Nazaret; su decisión de mudarse a Cafarnaúm había producido mucho resentimiento. Los habitantes de Nazaret habían oído hablar mucho de las actividades de su antiguo carpintero, pero estaban ofendidos porque nunca había incluído a su pueblo natal en ninguna de sus anteriores giras de predicación. Habían oído hablar, por supuesto, de la fama de Jesús, pero la mayoría de los ciudadanos estaban enojados porque no había realizado ninguna de sus grandes obras en la ciudad de su juventud. Durante meses, la gente de Nazaret había discutido mucho sobre Jesús, pero sus opiniones eran, en general, desfavorables hacia él.

\par 
%\textsuperscript{(1684.2)}
\textsuperscript{150:7.3} El Maestro se encontró pues, no en un ambiente de bienvenida al hogar, sino en medio de una atmósfera decididamente hostil e hipercrítica. Pero esto no era todo. Sabiendo que iba a pasar este sábado en Nazaret y suponiendo que hablaría en la sinagoga, sus enemigos habían sobornado a un buen número de hombres rudos y groseros para que lo hostigaran y provocaran dificultades de todas las maneras posibles.

\par 
%\textsuperscript{(1684.3)}
\textsuperscript{150:7.4} La mayoría de los antiguos amigos de Jesús, incluído el chazán que lo adoraba y que había sido su profesor en la adolescencia, habían muerto o se habían marchado de Nazaret, y la generación más joven era propensa a sentirse muy recelosa con su fama. Ya no se acordaban de su dedicación, siendo adolescente, a la familia de su padre, y lo criticaban severamente por su negligencia en no visitar a su hermano y a sus hermanas casadas que vivían en Nazaret. La actitud de la familia de Jesús hacia él también había contribuido a acrecentar este sentimiento desfavorable de los ciudadanos. Los judíos más ortodoxos se atrevieron incluso a criticar a Jesús por haber caminado demasiado deprisa cuando iba a la sinagoga aquel sábado por la mañana.

\section*{8. Los oficios del sábado}
\par 
%\textsuperscript{(1684.4)}
\textsuperscript{150:8.1} Aquel sábado hacía un día magnífico, y todo Nazaret, amigos y enemigos, salió para escuchar lo que este antiguo habitante de su ciudad iba a decir en la sinagoga. Una gran parte del séquito apostólico tuvo que permanecer fuera de la sinagoga, pues no había sitio para todos los que habían venido a escucharlo. Cuando era joven, Jesús había hablado con frecuencia en este lugar de culto. Aquella mañana, cuando el jefe de la sinagoga le pasó el rollo de los escritos sagrados donde iba a leer la lección de las Escrituras, ninguno de los presentes pareció recordar que éste era el mismo manuscrito que Jesús había regalado a esta sinagoga.

\par 
%\textsuperscript{(1684.5)}
\textsuperscript{150:8.2} Los oficios de este día se celebraron exactamente igual que cuando Jesús asistía siendo niño. Subió al estrado de los oradores con el jefe de la sinagoga, y el oficio empezó recitándose dos oraciones: <<Bendito sea el Señor, Rey del mundo, que forma la luz y crea las tinieblas, que hace la paz y crea todas las cosas; que en su misericordia da la luz a la Tierra y a los que viven en ella, y que en su bondad renueva las obras de la creación día tras día y cada día. Bendito sea el Señor nuestro Dios por la gloria de las obras de sus manos y por las luces iluminadoras que ha hecho para su alabanza. Selá. Bendito sea el Señor nuestro Dios que ha creado las luces>>.

\par 
%\textsuperscript{(1685.1)}
\textsuperscript{150:8.3} Después de una breve pausa, siguieron rezando: <<El Señor nuestro Dios nos ha amado con un gran amor, y se ha compadecido de nosotros con una piedad desbordante, nuestro Padre y nuestro Rey, por amor a nuestros padres que confiaron en él. Tú les enseñaste las reglas de la vida; ten misericordia de nosotros y enséñanos. Ilumina nuestros ojos con la ley; haz que nuestros corazones se ajusten a tus mandamientos; une nuestros corazones para que amemos y temamos tu nombre, y no nos avergonzaremos por los siglos de los siglos. Porque tú eres un Dios que prepara la salvación, y nos has escogido entre todas las naciones y lenguas, y en verdad nos has acercado a tu gran nombre ---selá--- para que podamos alabar tu unidad con amor. Bendito sea el Señor que, en su amor, ha elegido a su pueblo Israel>>.

\par 
%\textsuperscript{(1685.2)}
\textsuperscript{150:8.4} La congregación recitó luego el Semá, el credo de la fe judía. Este ritual consistía en repetir numerosos pasajes de la ley, e indicaba que los creyentes aceptaban el yugo del reino de los cielos, y también el yugo de los mandamientos tal como debían aplicarlos de día y de noche.

\par 
%\textsuperscript{(1685.3)}
\textsuperscript{150:8.5} Luego continuaron con la tercera oración: <<Es verdad que tú eres Yahvé, nuestro Dios y el Dios de nuestros padres, nuestro Rey y el Rey de nuestros padres; nuestro Salvador y el Salvador de nuestros padres; nuestro Creador y la roca de nuestra salvación; nuestra ayuda y nuestro libertador. Tu nombre existe desde la eternidad, y no hay más Dios que tú. Los que fueron liberados cantaron un nuevo cántico a tu nombre a la orilla del mar; todos juntos te alabaron y te reconocieron como Rey, diciendo: Yahvé reinará por los siglos de los siglos. Bendito sea el Señor que salva a Israel>>.

\par 
%\textsuperscript{(1685.4)}
\textsuperscript{150:8.6} El jefe de la sinagoga se situó entonces en su puesto delante del arca, o cofre, que contenía las escrituras sagradas, y empezó a recitar las diecinueve oraciones de elogio, o bendiciones. Pero en esta ocasión era conveniente acortar el oficio a fin de que el invitado de honor dispusiera de más tiempo para su discurso; por consiguiente, sólo se recitaron la primera y la última bendiciones. La primera era: <<Bendito sea el Señor nuestro Dios y el Dios de nuestros padres, el Dios de Abraham, el Dios de Isaac y el Dios de Jacob; el grande, el poderoso y el terrible Dios, que muestra misericordia y benevolencia, que crea todas las cosas, que recuerda sus bondadosas promesas a nuestros padres y envía con amor un salvador a los hijos de sus hijos para gloria de su propio nombre. Oh Rey, favorecedor, salvador y protector. Bendito eres tú, oh Yahvé, protector de Abraham>>.

\par 
%\textsuperscript{(1685.5)}
\textsuperscript{150:8.7} Después siguió la última bendición: <<Oh, concede a tu pueblo Israel una gran paz perpetua, pues tú eres el Rey y el Señor de toda paz. Y ves con buenos ojos bendecir con la paz a Israel en todo tiempo y a todas horas. Bendito seas, Yahvé, que bendices con la paz a tu pueblo Israel>>. La asamblea no miraba al jefe mientras éste recitaba las bendiciones. Después de las bendiciones, ofreció una oración no oficial, adecuada a la circunstancia, y cuando concluyó, toda la congregación se unió para decir amén.

\par 
%\textsuperscript{(1685.6)}
\textsuperscript{150:8.8} Luego, el chazán se dirigió al arca y sacó un rollo que entregó a Jesús para que éste pudiera leer la lección de las Escrituras. Era habitual llamar a siete personas para que leyeran por lo menos tres versos de la ley, pero en esta ocasión se renunció a esta práctica para que el visitante pudiera leer la lección que él mismo había escogido. Jesús cogió el rollo, se puso de pie y empezó a leer en el Deuteronomio: <<Pues este mandamiento que hoy te doy no es un secreto para ti, ni está lejos. No está en el cielo, para que no digas: ¿quién subirá al cielo por nosotros y nos lo traerá para que podamos oírlo y ponerlo en práctica? Tampoco está al otro lado del mar, para que no digas: ¿quién atravesará el mar por nosotros para que nos traiga el mandamiento a fin de que podamos oírlo y ponerlo en práctica? No, la palabra de vida está muy cerca de ti, incluso en tu presencia y en tu corazón, para que puedas conocerla y obedecerla>>.

\par 
%\textsuperscript{(1686.1)}
\textsuperscript{150:8.9} Cuando terminó de leer en el libro de la ley, pasó a Isaías donde empezó a leer: <<El espíritu del Señor está sobre mí, porque me ha ungido para que predique la buena nueva a los pobres. Me ha enviado para que proclame la libertad a los cautivos y la recuperación de la vista a los ciegos, para poner en libertad a los que se sienten heridos y proclamar el año favorable del Señor>>.

\par 
%\textsuperscript{(1686.2)}
\textsuperscript{150:8.10} Jesús cerró el libro y, después de devolverlo al jefe de la sinagoga, se sentó y empezó a hablarle a la gente. Comenzó diciendo: <<Hoy, estas Escrituras se han cumplido>>. Y luego habló cerca de quince minutos sobre <<Los hijos y las hijas de Dios>>. Su discurso agradó a muchos de los asistentes, que se maravillaron de su gracia y de su sabiduría.

\par 
%\textsuperscript{(1686.3)}
\textsuperscript{150:8.11} Después de concluir los oficios formales, existía la costumbre de que el orador permaneciera en la sinagoga para que las personas interesadas pudieran hacerle preguntas. En consecuencia, este sábado por la mañana, Jesús descendió para mezclarse con la multitud que se adelantaba para hacerle preguntas. En este grupo había muchos individuos violentos con intenciones dañinas, mientras que alrededor del gentío circulaban aquellos degenerados que habían sido sobornados para causarle problemas a Jesús. Muchos discípulos y evangelistas que habían permanecido fuera avanzaron ahora para entrar en la sinagoga y se dieron cuenta enseguida de que se estaba fraguando un disturbio. Trataron de llevarse al Maestro, pero éste no quiso ir con ellos.

\section*{9. Nazaret rechaza a Jesús}
\par 
%\textsuperscript{(1686.4)}
\textsuperscript{150:9.1} Jesús se encontró rodeado en la sinagoga por una gran multitud de enemigos y muy pocos de sus propios seguidores. En respuesta a las preguntas groseras y a las burlas siniestras, comentó medio en broma: <<Sí, soy el hijo de José; soy el carpintero, y no me sorprende que me recordéis el proverbio `Médico, cúrate a ti mismo', ni que me desafiéis para que haga en Nazaret lo que habéis oído decir que realicé en Cafarnaúm; pero os pongo por testigos de que las mismas Escrituras afirman que `a un profeta no le faltan honores, salvo en su propio país y entre su propia gente'.>>

\par 
%\textsuperscript{(1686.5)}
\textsuperscript{150:9.2} Pero lo empujaron y, señalándolo con un dedo acusador, le dijeron: <<Crees que eres mejor que la gente de Nazaret; te fuiste de aquí, pero tu hermano es un obrero común y tus hermanas viven todavía entre nosotros. Conocemos a tu madre, María. ¿Donde se encuentran hoy? Hemos escuchado grandes cosas sobre ti, pero observamos que no haces ningún prodigio a tu regreso>>. Jesús les contestó: <<Amo a la gente que vive en la ciudad donde crecí, y me regocijaría veros entrar a todos en el reino de los cielos, pero no me corresponde determinar la realización de las obras de Dios. Las transformaciones de la gracia se forjan como respuesta a la fe viviente de aquellos que son sus beneficiarios>>.

\par 
%\textsuperscript{(1686.6)}
\textsuperscript{150:9.3} Jesús hubiera manejado amablemente a la multitud y hubiera desarmado eficazmente incluso a sus enemigos más violentos, si uno de sus propios apóstoles, Simón Celotes, no hubiera cometido un grave error táctico. Con la ayuda de Nacor, uno de los evangelistas más jóvenes, Simón había reunido entretanto a un grupo de amigos de Jesús que estaban entre el gentío y, con una actitud agresiva, advirtieron a los enemigos del Maestro que se fueran de allí. Hacía tiempo que Jesús había enseñado a los apóstoles que una respuesta dulce desvía el furor, pero sus partidarios no estaban acostumbrados a que trataran a su amado instructor, a quien tan gustosamente llamaban Maestro, con tanta descortesía y desdén. Aquello fue demasiado para ellos y se pusieron a expresar su resentimiento apasionado y vehemente, lo cual no hizo más que encender los ánimos alborotadores de esta asamblea impía y grosera. Y así, bajo la dirección de los mercenarios, aquellos rufianes agarraron a Jesús y lo sacaron precipitadamente de la sinagoga hasta la cima de una escarpada colina cercana, donde estaban dispuestos a empujarlo al vacío para que se estrellara abajo. Pero cuando estaban a punto de empujarlo por el borde del acantilado, Jesús se revolvió de pronto sobre sus captores y, haciéndoles frente, se cruzó tranquilamente de brazos. No dijo nada, pero sus amigos se quedaron más que asombrados cuando empezó a caminar hacia adelante, mientras que el populacho se apartaba y lo dejaba pasar sin molestarlo.

\par 
%\textsuperscript{(1687.1)}
\textsuperscript{150:9.4} Jesús, seguido de sus discípulos, se dirigió al campamento, donde refirieron todo lo sucedido. Aquella tarde se prepararon para volver al día siguiente temprano a Cafarnaúm, tal como Jesús lo había ordenado. Este final turbulento de la tercera gira de predicación pública tuvo un efecto de moderación sobre todos los seguidores de Jesús. Empezaron a darse cuenta del significado de algunas enseñanzas del Maestro; estaban despertando al hecho de que el reino sólo se establecería mediante muchas tristezas y amargas desilusiones.

\par 
%\textsuperscript{(1687.2)}
\textsuperscript{150:9.5} Aquel domingo por la mañana abandonaron Nazaret, y después de viajar por caminos diferentes, todos se congregaron finalmente en Betsaida el jueves 10 de marzo al mediodía. Se reunieron como un grupo sobrio y serio de predicadores desilusionados del evangelio de la verdad, y no como un conjunto, entusiasta y conquistador, de cruzados triunfantes.


\chapter{Documento 151. Estancia y enseñanza a la orilla del mar}
\par 
%\textsuperscript{(1688.1)}
\textsuperscript{151:0.1} EL 10 DE MARZO, todos los grupos de predicadores y de instructores se habían reunido en Betsaida. El jueves por la noche y el viernes, muchos de ellos salieron a pescar, mientras que el día del sábado asistieron a la sinagoga para escuchar a un anciano judío de Damasco discurrir sobre la gloria del padre Abraham. Jesús pasó la mayor parte de este sábado a solas en las colinas. Este sábado por la noche, el Maestro habló durante más de una hora a los grupos reunidos sobre <<la misión de la adversidad y el valor espiritual de las decepciones>>. Fue un acontecimiento memorable y sus oyentes no olvidaron nunca la lección que les impartió.

\par 
%\textsuperscript{(1688.2)}
\textsuperscript{151:0.2} Jesús no se había recuperado por completo del disgusto de haber sido rechazado recientemente en Nazaret; los apóstoles observaron que en su comportamiento habitualmente jovial había una mezcla de tristeza particular. Santiago y Juan permanecieron con él la mayor parte del tiempo, pues Pedro estaba muy ocupado con las numerosas responsabilidades relacionadas con el bienestar y la dirección del nuevo cuerpo de evangelistas. Las mujeres pasaron este compás de espera, antes de partir para la Pascua en Jerusalén, visitando casa por casa, enseñando el evangelio, y cuidando a los enfermos en Cafarnaúm y en las ciudades y pueblos cercanos.

\section*{1. La parábola del sembrador}
\par 
%\textsuperscript{(1688.3)}
\textsuperscript{151:1.1} Aproximadamente por esta época, Jesús empezó a emplear por primera vez el método de las parábolas para enseñar a las multitudes que se congregaban con tanta frecuencia a su alrededor. Como Jesús había conversado con los apóstoles y otras personas hasta muy entrada la madrugada, aquel domingo por la mañana muy pocos del grupo se habían levantado para el desayuno; así pues, se fue a la orilla del mar y se sentó solo en una barca, en la vieja barca de pesca de Andrés y Pedro, que siempre se mantenía a su disposición; y se puso a meditar sobre el paso siguiente a dar en la tarea de difundir el reino. Pero el Maestro no iba a estar solo durante mucho tiempo. Muy pronto, la gente de Cafarnaúm y de los pueblos vecinos empezó a llegar, y hacia las diez de la mañana, casi mil personas se habían congregado en la playa cerca de la barca de Jesús, dando gritos para llamar su atención. Pedro ya se había levantado y, abriéndose paso hasta la barca, le dijo a Jesús: <<Maestro, ¿les hablo?>> Pero Jesús contestó: <<No, Pedro, les voy a contar una historia>>. Entonces Jesús empezó la narración de la parábola del sembrador, una de las primeras de una larga serie de parábolas similares que enseñó a las multitudes que lo seguían. Esta barca tenía un asiento elevado en el que Jesús se sentó (ya que era costumbre estar sentado para enseñar) mientras le hablaba a la muchedumbre congregada a lo largo de la playa. Después de que Pedro hubiera pronunciado unas palabras, Jesús dijo:

\par 
%\textsuperscript{(1688.4)}
\textsuperscript{151:1.2} <<Un sembrador salió a sembrar y sucedió que mientras sembraba, algunas semillas cayeron al borde del camino, donde fueron pisoteadas y devoradas por los pájaros del cielo. Otras semillas cayeron en lugares rocosos donde había poca tierra, y brotaron inmediatamente porque la tierra no tenía profundidad, pero tan pronto como brilló el Sol se marchitaron, porque no tenían raíces para absorber la humedad. Otras semillas cayeron entre los espinos, y cuando los espinos crecieron, las ahogaron, de manera que no produjeron ningún grano. Pero otras semillas cayeron en una buena tierra, y cuando crecieron, algunas produjeron treinta, otras sesenta y otras cien granos>>. Cuando terminó de contar esta parábola, dijo a la multitud: <<El que tenga oídos para oír, que oiga>>.

\par 
%\textsuperscript{(1689.1)}
\textsuperscript{151:1.3} Cuando escucharon a Jesús enseñar a la gente de esta manera, los apóstoles y aquellos que estaban con ellos se quedaron enormemente perplejos; después de hablar mucho entre ellos aquella tarde en el jardín de Zebedeo, Mateo le dijo a Jesús: <<Maestro, ¿cuál es el significado de las oscuras palabras que ofreces a la multitud? ¿Por qué hablas en parábolas a los que buscan la verdad?>> Y Jesús contestó:

\par 
%\textsuperscript{(1689.2)}
\textsuperscript{151:1.4} <<Todo este tiempo os he enseñado con paciencia. A vosotros os ha sido dado conocer los misterios del reino de los cielos, pero a las multitudes sin discernimiento y a aquellos que buscan nuestra destrucción, desde ahora en adelante los misterios del reino les serán presentados en parábolas. Y actuaremos así para que aquellos que desean entrar realmente en el reino puedan discernir el significado de la enseñanza y encontrar así la salvación, mientras que los que escuchan únicamente para atraparnos se quedarán aún más confundidos, en el sentido de que verán sin ver y oirán sin oír. Hijos míos, ¿no percibís la ley del espíritu, que establece que al que tiene se le dará para que posea en abundancia, pero al que no tiene, incluso lo poco que tiene se le quitará? Por eso, de aquí en adelante le hablaré mucho a la gente en parábolas, para que nuestros amigos y aquellos que desean conocer la verdad puedan encontrar lo que buscan, mientras que nuestros enemigos y aquellos que no aman la verdad puedan escuchar sin comprender. Mucha de esta gente no sigue el camino de la verdad. El profeta supo describir en verdad a todas estas almas sin discernimiento, cuando dijo: `Porque el corazón de este pueblo se ha embrutecido, son duros de oído y han cerrado los ojos por temor a discernir la verdad y a entenderla en su corazón'.>>

\par 
%\textsuperscript{(1689.3)}
\textsuperscript{151:1.5} Los apóstoles no comprendieron por completo el significado de las palabras del Maestro. Mientras Andrés y Tomás siguieron hablando con Jesús, Pedro y los otros apóstoles se retiraron a otra parte del jardín, donde emprendieron una larga y seria discusión.

\section*{2. La interpretación de la parábola}
\par 
%\textsuperscript{(1689.4)}
\textsuperscript{151:2.1} Pedro y el grupo que le rodeaba llegaron a la conclusión de que la parábola del sembrador era una alegoría, que cada uno de sus elementos tenía un significado oculto; así pues, decidieron ir a ver a Jesús para solicitarle una explicación. En consecuencia, Pedro se acercó al Maestro, diciendo: <<Somos incapaces de penetrar el significado de esta parábola, y deseamos que nos la expliques, puesto que dices que se nos ha dado conocer los misterios del reino>>. Cuando escuchó esto, Jesús le dijo a Pedro: <<Hijo mío, no deseo ocultarte nada, pero supongamos que me cuentas primero lo que habéis estado hablando; ¿cuál es tu interpretación de la parábola?>>

\par 
%\textsuperscript{(1689.5)}
\textsuperscript{151:2.2} Después de un momento de silencio, Pedro dijo: <<Maestro, hemos hablado mucho sobre la parábola, y ésta es la interpretación a la que he llegado: El sembrador es el predicador del evangelio; la semilla es la palabra de Dios. Las semillas que cayeron al borde del camino representan a los que no comprenden la enseñanza del evangelio. Los pájaros que atraparon rápidamente las semillas que cayeron en el suelo endurecido representan a Satanás, o al maligno, que esconde lo que se ha sembrado en el corazón de esos ignorantes. Las semillas que cayeron en los lugares rocosos y que brotaron con tanta rapidez representan a esas personas superficiales e irreflexivas que, cuando escuchan la buena nueva, reciben el mensaje con alegría, pero como la verdad no tiene ninguna raíz verdadera en su comprensión más profunda, su devoción dura poco ante las tribulaciones y las persecuciones. Estos creyentes tropiezan cuando llegan las dificultades, y cuando son tentados, desfallecen. Las semillas que cayeron entre los espinos representan a los que escuchan la palabra con agrado, pero permiten que las inquietudes del mundo y la falsedad de las riquezas ahoguen la palabra de la verdad, de tal manera que se vuelve estéril. Pero las semillas que cayeron en una buena tierra y crecieron hasta que unas produjeron treinta, otras sesenta y otras cien granos, representan a los que han escuchado la verdad, la han recibido con diversos grados de apreciación ---debido a sus diferentes dotes intelectuales--- y por eso manifiestan esos diversos grados de experiencia religiosa>>.

\par 
%\textsuperscript{(1690.1)}
\textsuperscript{151:2.3} Después de escuchar la interpretación que Pedro hizo de la parábola, Jesús preguntó a los otros apóstoles si no tenían también alguna sugerencia que ofrecer. Natanael fue el único que respondió a esta invitación, diciendo: <<Maestro, reconozco que hay muchas cosas buenas en la interpretación que Simón Pedro ha hecho de la parábola, pero no estoy totalmente de acuerdo con él. Mi idea de esta parábola sería la siguiente: La semilla representa al evangelio del reino, mientras que el sembrador simboliza los mensajeros del reino. Las semillas que cayeron al borde del camino en la tierra endurecida representan a los que han escuchado poca cosa del evangelio, junto con aquellos que son indiferentes al mensaje y que han endurecido su corazón. Los pájaros del cielo que atraparon rápidamente las semillas que cayeron al borde del camino representan los hábitos que tenemos en la vida, la tentación del mal y los deseos de la carne. Las semillas que cayeron entre las rocas simbolizan las almas emotivas que reciben rápidamente la nueva enseñanza, y que abandonan la verdad con la misma rapidez cuando tienen que enfrentarse con las dificultades y las realidades de vivir a la altura de esa verdad; carecen de percepción espiritual. Las semillas que cayeron entre los espinos representan a los que se sienten atraídos por las verdades del evangelio; están dispuestos a seguir sus enseñanzas, pero el orgullo del mundo, los celos, la envidia y las ansiedades de la existencia humana se lo impiden. Las semillas que cayeron en la buena tierra y crecieron hasta que unas produjeron treinta, otras sesenta y otras cien granos, representan los diferentes grados naturales de aptitud para comprender la verdad y responder a sus enseñanzas espirituales, por parte de unos hombres y mujeres que poseen unos dones diversos de iluminación espiritual>>.

\par 
%\textsuperscript{(1690.2)}
\textsuperscript{151:2.4} Cuando Natanael terminó de hablar, los apóstoles y sus compañeros emprendieron una seria discusión y se metieron en un ardiente debate; algunos sostenían que la interpretación de Pedro era correcta, mientras que otro número casi igual trataba de defender la explicación que Natanael había dado de la parábola. Mientras tanto, Pedro y Natanael se habían retirado a la casa, donde se enredaron en un esfuerzo enérgico y decidido por convencer al otro y cambiar su opinión.

\par 
%\textsuperscript{(1690.3)}
\textsuperscript{151:2.5} El Maestro permitió que esta confusión alcanzara su máxima intensidad de expresión; luego dio unas palmadas y los llamó para que se acercaran. Cuando todos estuvieron reunidos de nuevo a su alrededor, dijo: <<Antes de que os hable de esta parábola, ¿alguno de vosotros tiene algo que decir?>> Después de un momento de silencio, Tomás dijo: <<Sí, Maestro, deseo decir unas palabras. Recuerdo que una vez nos dijiste que tuviéramos cuidado con esto mismo. Nos indicaste que, cuando utilizáramos unos ejemplos para nuestra predicación, debíamos emplear historias verdaderas, y no fábulas. Debíamos escoger la historia que mejor conviniera para ilustrar la única verdad central y esencial que deseábamos enseñar a la gente, y que, después de haber utilizado así dicha historia, no debíamos intentar hacer una aplicación espiritual de todos los detalles menores involucrados en la historia que habíamos contado. Estimo que tanto Pedro como Natanael se equivocan al intentar interpretar esta parábola. Admiro la habilidad que tienen para hacer estas cosas, pero estoy igualmente seguro de que todas esas tentativas para hacer que una parábola natural arroje analogías espirituales en todos sus aspectos, sólo pueden llevar a la confusión y a una idea gravemente falsa de la verdadera finalidad de dicha parábola. La prueba de que llevo razón lo demuestra plenamente el hecho de que hace una hora todos estábamos de acuerdo, y ahora estamos divididos en dos grupos separados que mantienen opiniones diferentes sobre esta parábola, y sostienen esas opiniones con tanto ahínco que, en mi opinión, obstaculiza nuestra capacidad para captar plenamente la gran verdad que tenías en la mente cuando presentaste esta parábola a la muchedumbre y nos pediste posteriormente que la comentáramos>>.

\par 
%\textsuperscript{(1691.1)}
\textsuperscript{151:2.6} Las palabras de Tomás tuvieron un efecto tranquilizador sobre todos ellos. Tomás hizo que recordaran lo que Jesús les había enseñado en ocasiones anteriores, y antes de que Jesús continuara hablando, Andrés se levantó y dijo: <<Estoy persuadido de que Tomás tiene razón, y me gustaría que nos dijera el significado que le atribuye a la parábola del sembrador>>. Jesús le hizo señas a Tomás para que hablara, y éste dijo: <<Hermanos míos, no deseaba prolongar esta discusión, pero si así lo deseáis, diré que creo que esta parábola ha sido contada para enseñarnos una gran verdad, que es la siguiente: Por muy fiel y eficazmente que ejecutemos nuestra misión divina, nuestra enseñanza del evangelio del reino estará acompañada de diferentes grados de éxito; y todas esas diferencias de resultados se deberán directamente a las condiciones inherentes a las circunstancias de nuestro ministerio, unas condiciones sobre las que tenemos poco o ningún control>>.

\par 
%\textsuperscript{(1691.2)}
\textsuperscript{151:2.7} Cuando Tomás terminó de hablar, la mayoría de sus compañeros predicadores estaban dispuestos a darle la razón, e incluso Pedro y Natanael estaban a punto de hablar con él, cuando Jesús se levantó y dijo: <<Bien hecho, Tomás; has discernido el verdadero significado de las parábolas; pero tanto Pedro como Natanael os han hecho a todos el mismo bien, en el sentido de que han mostrado plenamente el peligro de aventurarse a convertir mis parábolas en alegorías. En vuestro propio fuero interno, podéis ocuparos a menudo de manera provechosa en estos vuelos de la imaginación especulativa, pero cometéis un error cuando intentáis incorporar esas conclusiones en vuestra enseñanza pública>>.

\par 
%\textsuperscript{(1691.3)}
\textsuperscript{151:2.8} Ahora que la tensión había desaparecido, Pedro y Natanael se felicitaron mutuamente por sus interpretaciones, y a excepción de los gemelos Alfeo, cada uno de los apóstoles se aventuró a hacer una interpretación de la parábola del sembrador antes de retirarse para dormir. Incluso Judas Iscariote ofreció una interpretación muy plausible. Los doce intentaron a menudo descifrar entre ellos las parábolas del Maestro como lo hubieran hecho con una alegoría, pero nunca más se tomaron en serio estas especulaciones. Fue una sesión muy provechosa para los apóstoles y sus compañeros, especialmente porque a partir de este momento Jesús empleó cada vez más parábolas en su enseñanza pública.

\section*{3. Más cosas sobre las parábolas}
\par 
%\textsuperscript{(1691.4)}
\textsuperscript{151:3.1} Los apóstoles tenían predilección por las parábolas, de tal manera que toda la tarde siguiente la consagraron a seguir discutiendo sobre las parábolas. Jesús empezó la conferencia de la tarde, diciendo: <<Amados míos, en el momento de enseñar siempre debéis hacer una diferencia para adaptar vuestra presentación de la verdad a la mente y al corazón de los que os escuchan. Cuando os encontráis delante de una muchedumbre de intelectos y de temperamentos variados, no podéis decir palabras diferentes para cada tipo de oyente, pero podéis contar una historia para transmitir vuestra enseñanza. Cada grupo, e incluso cada individuo, podrá interpretar vuestra parábola a su manera, según sus propios dones intelectuales y espirituales. Debéis dejar que vuestra luz brille, pero hacedlo con sabiduría y discreción. Nadie enciende un candil para cubrirlo con una vasija o colocarlo debajo de la cama, sino que pone su candil sobre un pedestal donde todos puedan contemplar la luz. Permitidme que os diga que, en el reino de los cielos, no hay nada oculto que no se pueda manifestar; ni tampoco hay secretos que finalmente no se puedan conocer. Todas esas cosas acabarán por salir a la luz. No penséis solamente en las multitudes y en la manera en que escuchan la verdad; prestad atención también a la manera en que vosotros mismos escucháis. Recordad que os he dicho muchas veces: A aquel que tiene se le dará más, mientras que al que no tiene se le quitará incluso lo que cree tener>>.

\par 
%\textsuperscript{(1692.1)}
\textsuperscript{151:3.2} La prolongada discusión sobre las parábolas y las instrucciones adicionales en cuanto a su interpretación, se pueden resumir y expresar en un lenguaje moderno de la manera siguiente:

\par 
%\textsuperscript{(1692.2)}
\textsuperscript{151:3.3} 1. Jesús aconsejó que no se emplearan las fábulas ni las alegorías para enseñar las verdades del evangelio. Sí que recomendó la libre utilización de las parábolas, en especial las parábolas relacionadas con la naturaleza. Recalcó el valor de utilizar la \textit{analogía} existente entre los mundos natural y espiritual como un medio de enseñar la verdad. Aludió con frecuencia a lo natural como <<la sombra irreal y fugaz de las realidades del espíritu>>.

\par 
%\textsuperscript{(1692.3)}
\textsuperscript{151:3.4} 2. Jesús contó tres o cuatro parábolas de las escrituras hebreas, y llamó la atención sobre el hecho de que este método de enseñanza no era totalmente nuevo. Sin embargo, se convirtió casi en un método nuevo por la manera en que lo empleó desde entonces en adelante.

\par 
%\textsuperscript{(1692.4)}
\textsuperscript{151:3.5} 3. Al enseñar a los apóstoles el valor de las parábolas, Jesús llamó la atención sobre los puntos siguientes:

\par 
%\textsuperscript{(1692.5)}
\textsuperscript{151:3.6} La parábola apela simultáneamente a unos niveles extremadamente diferentes de la mente y del espíritu. La parábola estimula la imaginación, desafía el discernimiento y provoca el pensamiento crítico; promueve la simpatía sin despertar el antagonismo.

\par 
%\textsuperscript{(1692.6)}
\textsuperscript{151:3.7} La parábola pasa de las cosas conocidas al discernimiento de lo desconocido. La parábola utiliza lo material y lo natural como un medio para presentar lo espiritual y lo supermaterial.

\par 
%\textsuperscript{(1692.7)}
\textsuperscript{151:3.8} Las parábolas favorecen la toma de decisiones morales imparciales. La parábola evita numerosos prejuicios e introduce con elegancia las nuevas verdades en la mente, y hace todo esto despertando un mínimo de defensas propias en el resentimiento personal.

\par 
%\textsuperscript{(1692.8)}
\textsuperscript{151:3.9} Rechazar la verdad contenida en una analogía parabólica requiere una acción intelectual consciente que menosprecie directamente el juicio honesto y la decisión justa de la persona. La parábola conduce a forzar el pensamiento a través del sentido del oído.

\par 
%\textsuperscript{(1692.9)}
\textsuperscript{151:3.10} El uso de la parábola como medio de enseñanza permite al instructor presentar verdades nuevas, e incluso sorprendentes, mientras que al mismo tiempo evita ampliamente toda controversia y todo conflicto exterior con la tradición y la autoridad establecida.

\par 
%\textsuperscript{(1693.1)}
\textsuperscript{151:3.11} La parábola posee también la ventaja de avivar la memoria de la verdad enseñada, cuando se encuentran posteriormente las mismas escenas familiares.

\par 
%\textsuperscript{(1693.2)}
\textsuperscript{151:3.12} Jesús intentó de esta manera poner al corriente a sus discípulos de las diversas razones que apoyaban su práctica de emplear cada vez más parábolas en su enseñanza pública.

\par 
%\textsuperscript{(1693.3)}
\textsuperscript{151:3.13} Hacia el final de la lección de la tarde, Jesús hizo su primer comentario sobre la parábola del sembrador. Dijo que la parábola se refería a dos cosas: En primer lugar, era una revisión de su propio ministerio hasta ese momento, y una previsión de lo que le esperaba durante el resto de su vida en la Tierra. Y en segundo lugar, también era una alusión a lo que los apóstoles y otros mensajeros del reino podían esperar en su ministerio, de generación en generación, a medida que pasara el tiempo.

\par 
%\textsuperscript{(1693.4)}
\textsuperscript{151:3.14} Jesús recurrió también al empleo de las parábolas para refutar lo mejor posible el esfuerzo premeditado de los jefes religiosos de Jerusalén, que enseñaban que toda su obra se efectuaba gracias a la ayuda de los demonios y del príncipe de los diablos. La apelación a la naturaleza contradecía esta enseñanza, ya que la gente de aquella época consideraba que todos los fenómenos naturales eran producidos directamente por los seres espirituales y las fuerzas supernaturales. También se decidió a utilizar este método de enseñanza porque le permitía proclamar verdades esenciales a los que deseaban conocer el mejor camino, y al mismo tiempo proporcionaba a sus enemigos menos oportunidades de encontrar motivos para sentirse ofendidos y acusarlo.

\par 
%\textsuperscript{(1693.5)}
\textsuperscript{151:3.15} Antes de despedir al grupo para pasar la noche, Jesús dijo: <<Ahora os voy a contar lo último de la parábola del sembrador. Quiero probaros para saber cómo recibiréis esto: El reino de los cielos se parece también a un hombre que echa una buena semilla en la tierra; mientras dormía por la noche y se ocupaba de sus asuntos durante el día, la semilla brotó y creció, y aunque no sabía cómo sucedió, la planta fructificó. Primero fue la hoja, luego la espiga y luego el grano completo en la espiga. Y cuando el grano estuvo maduro, empleó la hoz y fue el final de la cosecha. El que tenga oídos para oír, que oiga>>.

\par 
%\textsuperscript{(1693.6)}
\textsuperscript{151:3.16} Los apóstoles le dieron muchas vueltas a estas palabras en su mente, pero el Maestro nunca volvió a mencionar este añadido a la parábola del sembrador.

\section*{4. Más parábolas al lado del mar}
\par 
%\textsuperscript{(1693.7)}
\textsuperscript{151:4.1} Al día siguiente, Jesús volvió a enseñar a la gente desde la barca, diciendo: <<El reino de los cielos se parece a un hombre que sembró una buena semilla en su campo; pero mientras dormía, su enemigo vino y sembró cizaña en medio del trigo, huyendo apresuradamente. Y así, cuando los jóvenes tallos brotaron y más tarde estuvieron a punto de producir su fruto, apareció también la cizaña. Entonces, los servidores de este propietario fueron a decirle: `Señor, ¿no sembraste buena semilla en tu campo? ¿de dónde ha salido entonces esa cizaña?' El dueño respondió a sus servidores: `Algún enemigo lo ha hecho'. Entonces los servidores le preguntaron: `¿Quieres que vayamos a arrancar la cizaña?' Pero él les contestó diciendo: `No, no sea que al arrancarla desarraiguéis también el trigo. Lo mejor es dejarlos que crezcan juntos hasta el momento de la cosecha, y entonces diré a los segadores: Primero recoged la cizaña y atadla en fardos para quemarla, y luego recoged el trigo para almacenarlo en mi granero'.>>

\par 
%\textsuperscript{(1693.8)}
\textsuperscript{151:4.2} Después de algunas preguntas de la gente, Jesús contó otra parábola: <<El reino de los cielos se parece a un grano de mostaza que un hombre sembró en su campo. Ahora bien, un grano de mostaza es la más pequeña de todas las semillas, pero cuando está maduro, se convierte en la hierba más grande de todas y se parece a un árbol, de manera que los pájaros del cielo pueden venir y reposar en sus ramas>>.

\par 
%\textsuperscript{(1694.1)}
\textsuperscript{151:4.3} <<El reino de los cielos se parece también a la levadura que una mujer cogió para esconderla en tres medidas de harina, y sucedió de esta manera que toda la masa fermentó>>.

\par 
%\textsuperscript{(1694.2)}
\textsuperscript{151:4.4} <<El reino de los cielos se parece también a un tesoro escondido en un campo, que un hombre descubrió. En su alegría, salió a vender todo lo que poseía a fin de tener el dinero para comprar el campo>>.

\par 
%\textsuperscript{(1694.3)}
\textsuperscript{151:4.5} <<El reino de los cielos se parece también a un comerciante que busca perlas finas; y habiendo encontrado una perla de gran valor, salió a vender todo lo que poseía para poder comprar la perla extraordinaria>>.

\par 
%\textsuperscript{(1694.4)}
\textsuperscript{151:4.6} <<Y además, el reino de los cielos se parece a una red barredera que fue arrojada al mar y recogió todo tipo de peces. Cuando la red estuvo llena, los pescadores la sacaron a la playa, donde se sentaron para distribuir el pescado; recogieron los buenos en unos recipientes y arrojaron los malos>>.

\par 
%\textsuperscript{(1694.5)}
\textsuperscript{151:4.7} Jesús contó a las multitudes otras muchas parábolas. De hecho, a partir de esta época, rara vez empleó otro método para enseñar a las masas. Después de hablar en parábolas a un auditorio público, explicaba sus enseñanzas a los apóstoles y a los evangelistas con más plenitud y claridad durante las clases vespertinas.

\section*{5. La visita a Jeresa}
\par 
%\textsuperscript{(1694.6)}
\textsuperscript{151:5.1} La multitud continuó aumentando durante toda la semana. El sábado, Jesús se apresuró a partir hacia las colinas, pero cuando llegó el domingo por la mañana, la muchedumbre volvió. Jesús les habló a primera hora de la tarde después de la predicación de Pedro, y cuando hubo terminado, dijo a sus apóstoles: <<Estoy cansado de las multitudes; crucemos a la otra orilla para poder descansar un día>>.

\par 
%\textsuperscript{(1694.7)}
\textsuperscript{151:5.2} Durante la travesía del lago, se encontraron con una de esas violentas y repentinas tempestades que son características del mar de Galilea, sobre todo en esta época del año. Esta extensión de agua se encuentra a unos doscientos metros por debajo del nivel del mar, y está rodeada por unos altos márgenes, especialmente al oeste. Hay gargantas escarpadas que van desde el lago hasta las colinas; durante el día, una bolsa de aire caliente se eleva por encima del lago, y después de la puesta del Sol, el aire frío de las gargantas tiene tendencia a precipitarse sobre el lago. Estos vendavales llegan con rapidez y a veces se desvanecen de la misma forma repentina.

\par 
%\textsuperscript{(1694.8)}
\textsuperscript{151:5.3} Uno de estos vendavales vespertinos fue precisamente el que sorprendió a la barca que llevaba a Jesús a la otra orilla este domingo por la tarde. Otras tres barcas con algunos de los evangelistas más jóvenes seguían detrás. La tempestad era violenta, aunque limitada a esta región del lago, pues no había signos de tormenta en la orilla occidental. El viento era tan fuerte que las olas empezaron a inundar la barca. El fuerte viento había arrancado la vela antes de que los apóstoles pudieran recogerla, y ahora dependían totalmente de sus remos mientras bogaban penosamente hacia la costa, a unos dos kilómetros y medio de distancia.

\par 
%\textsuperscript{(1694.9)}
\textsuperscript{151:5.4} Mientras tanto, Jesús permanecía dormido en la popa de la barca debajo de un pequeño cobertizo. El Maestro estaba cansado cuando partieron de Betsaida, y para conseguir descansar, les había ordenado que lo llevaran en una embarcación hasta la otra orilla. Estos antiguos pescadores eran unos remeros vigorosos y experimentados, pero éste era uno de los peores temporales con que se habían encontrado nunca. Aunque el viento y las olas sacudían su barca como si fuera de juguete, Jesús continuaba durmiendo tranquilamente. Pedro estaba en el remo de la derecha, cerca de la popa. Cuando la barca empezó a llenarse de agua, dejó su remo y se precipitó hacia Jesús, sacudiéndolo vigorosamente para despertarlo. Cuando estuvo despierto, Pedro le dijo: <<Maestro, ¿no sabes que estamos en medio de una violenta tormenta? Si no nos salvas, todos pereceremos>>.

\par 
%\textsuperscript{(1695.1)}
\textsuperscript{151:5.5} Jesús salió en medio de la lluvia y primero miró a Pedro, luego escudriñó en la oscuridad a los remeros que se esforzaban, y de nuevo volvió la vista hacia Simón Pedro, que, en su agitación, aún no había regresado a su remo, y le dijo: <<¿Por qué tenéis todos tanto miedo? ¿Dónde está vuestra fe? Paz, permaneced tranquilos>>. Apenas había expresado Jesús esta reprimenda a Pedro y a los otros apóstoles, apenas le había pedido a Pedro que buscara la paz para calmar su alma inquieta, la atmósfera perturbada restableció su equilibrio y se asentó en una gran calma. Las olas irritadas se apaciguaron casi inmediatamente, mientras que los oscuros nubarrones que se habían extinguido en un corto aguacero, se desvanecieron, y las estrellas del cielo brillaron en lo alto. En la medida en que podemos juzgar esto, todo fue una pura coincidencia; pero los apóstoles, y en particular Simón Pedro, nunca dejaron de considerar el episodio como un milagro de la naturaleza. Para los hombres de aquella época era muy fácil creer en los milagros de la naturaleza, puesto que creían firmemente que toda la naturaleza era un fenómeno directamente controlado por las fuerzas espirituales y los seres sobrenaturales.

\par 
%\textsuperscript{(1695.2)}
\textsuperscript{151:5.6} Jesús explicó claramente a los doce que había hablado a sus espíritus perturbados y que se había dirigido a sus mentes agitadas por el miedo, y que no había mandado a los elementos que obedecieran a su palabra, pero fue en vano. Los seguidores del Maestro siempre se empeñaron en interpretar a su propia manera todas estas coincidencias. A partir de este día, insistieron en considerar que el Maestro poseía un poder absoluto sobre los elementos naturales. Pedro no se cansó nunca de contar que <<incluso los vientos y las olas le obedecían>>.

\par 
%\textsuperscript{(1695.3)}
\textsuperscript{151:5.7} Ya era casi de noche cuando Jesús y sus asociados llegaron a la orilla, y como era una noche tranquila y hermosa, todos descansaron en las barcas y no desembarcaron hasta la mañana siguiente, poco después de salir el Sol. Cuando se hubieron reunido, unos cuarenta en total, Jesús dijo: <<Subamos a aquellas colinas y permanezcamos allí unos días mientras reflexionamos sobre los problemas del reino del Padre>>.

\section*{6. El lunático de Jeresa}
\par 
%\textsuperscript{(1695.4)}
\textsuperscript{151:6.1} Aunque la mayor parte de la cercana ribera oriental del lago subía en pendiente suave hasta las tierras altas que estaban detrás, en este lugar concreto había una ladera empinada donde, en algunos puntos, la costa descendía de golpe hasta el lago. Señalando la ladera de la colina cercana, Jesús dijo: <<Subamos a esa ladera para desayunar y descansemos mientras hablamos debajo de algún refugio>>.

\par 
%\textsuperscript{(1695.5)}
\textsuperscript{151:6.2} Toda esta ladera estaba llena de cavernas que habían sido labradas en la roca. Muchos de estos nichos eran antiguos sepulcros. Hacia la mitad de esta pendiente, en un lugar pequeño relativamente llano, se encontraba el cementerio del pueblecito de Jeresa. Cuando Jesús y sus asociados pasaban cerca de este cementerio, un lunático que vivía en estas cuevas de la ladera se precipitó hacia ellos. Este demente era muy conocido en aquellos parajes; en otra época había estado amarrado con grilletes y cadenas, y confinado en una de las grutas. Hacía tiempo que había roto sus cadenas y ahora vagaba a su antojo entre las tumbas y los sepulcros abandonados.

\par 
%\textsuperscript{(1696.1)}
\textsuperscript{151:6.3} Este hombre, que se llamaba Amós, estaba afligido por una forma periódica de locura. Había períodos considerablemente largos durante los cuales buscaba con qué vestirse y se comportaba razonablemente bien entre sus semejantes. Durante uno de estos intervalos de lucidez, había ido a Betsaida, donde había escuchado la predicación de Jesús y de los apóstoles, y en aquel momento se había puesto a creer a medias en el evangelio del reino. Pero pronto reapareció una fase tormentosa de su enfermedad, y huyó hacia las tumbas, donde gemía, clamaba a gritos y se comportaba de tal manera que aterrorizaba a todos los que lo encontraban por casualidad.

\par 
%\textsuperscript{(1696.2)}
\textsuperscript{151:6.4} Cuando Amós reconoció a Jesús, cayó a sus pies y exclamó: <<Te conozco, Jesús, pero estoy poseído por muchos demonios, y te suplico que no me atormentes>>. Este hombre creía sinceramente que su periódica aflicción mental se debía al hecho de que, en los momentos de crisis, los espíritus malignos o impuros entraban en él y dominaban su mente y su cuerpo. Sus trastornos eran principalmente emocionales ---su cerebro no estaba gravemente enfermo.

\par 
%\textsuperscript{(1696.3)}
\textsuperscript{151:6.5} Jesús bajó la mirada sobre el hombre que estaba agachado como un animal a sus pies, se inclinó, lo cogió de la mano, lo levantó y le dijo: <<Amós, no estás poseído por un demonio; ya has oído la buena nueva de que eres un hijo de Dios. Te ordeno que salgas de ese estado>>. Cuando Amós oyó a Jesús decir estas palabras, se produjo tal transformación en su intelecto, que recobró inmediatamente su entero juicio y el control normal de sus emociones. En ese momento, una multitud considerable procedente del pueblo vecino se había congregado, y esta gente, unida a los porqueros que venían de las tierras altas situadas más arriba, se sorprendieron al ver al lunático sentado con Jesús y sus discípulos en posesión de su entero juicio y conversando espontáneamente con ellos.

\par 
%\textsuperscript{(1696.4)}
\textsuperscript{151:6.6} Mientras los porqueros se precipitaban hacia el pueblo para divulgar la noticia de que el lunático había sido domado, los perros cargaron contra una pequeña piara de unos treinta cerdos que habían quedado abandonados, y empujaron a la mayoría por encima de un precipicio hasta el mar. Este incidente, unido a la presencia de Jesús y a la curación supuestamente milagrosa del lunático, fue lo que dio origen a la leyenda de que Jesús había curado a Amós arrojando a una legión de demonios fuera de él, y que esos demonios se habían metido en la piara de cerdos, induciéndoles en el acto a que se precipitaran de cabeza hacia su destrucción en el mar. Antes de que terminara el día, los cuidadores de cerdos habían difundido este episodio por todas partes, y el pueblo entero se lo creyó. Amós creyó sin ninguna duda en esta historia; había visto caer a los cerdos por encima del borde de la colina poco después de que su mente perturbada hubiera recuperado la tranquilidad, y siempre creyó que los cerdos se habían llevado consigo a los mismos espíritus malignos que durante tanto tiempo lo habían atormentado y afligido. Esto contribuyó mucho a que su curación fuera permanente. Es igualmente cierto que todos los apóstoles de Jesús (salvo Tomás) creyeron que el episodio de los cerdos estaba directamente relacionado con la curación de Amós.

\par 
%\textsuperscript{(1696.5)}
\textsuperscript{151:6.7} Jesús no consiguió el descanso que iba buscando. La mayor parte de aquel día estuvo asediado por la gente que venía en respuesta a la noticia de que Amós había sido curado, y atraída por la historia de que los demonios habían salido del lunático metiéndose en la piara de cerdos. Y así, el martes por la mañana temprano, después de una sola noche de descanso, Jesús y sus amigos fueron despertados por una delegación de estos gentiles criadores de cerdos que venía para exigirles que se fueran de su región. Su portavoz dijo a Pedro y a Andrés: <<Pescadores de Galilea, iros de aquí y llevaos a vuestro profeta. Sabemos que es un hombre santo, pero los dioses de nuestro país no lo conocen, y corremos el riesgo de perder muchos cerdos. Tenemos miedo de vosotros, y por eso os rogamos que os vayáis de aquí>>. Cuando Jesús los escuchó, le dijo a Andrés: <<Volvamos a nuestro hogar>>.

\par 
%\textsuperscript{(1697.1)}
\textsuperscript{151:6.8} Cuando estaban a punto de partir, Amós le suplicó a Jesús que le permitiera ir con ellos, pero el Maestro no quiso consentirlo. Jesús le dijo a Amós: <<No olvides que eres un hijo de Dios. Vuelve con tu propia gente y muéstrales las grandes cosas que Dios ha hecho por ti>>. Y Amós se puso a divulgar por todas partes que Jesús había echado a una legión de demonios de su alma perturbada, y que estos espíritus malignos se habían metido en una piara de cerdos, que los habían llevado rápidamente a la destrucción. Y no se detuvo hasta que hubo recorrido todas las ciudades de la Decápolis, proclamando las grandes cosas que Jesús había hecho por él.


\chapter{Documento 152. Los acontecimientos que condujeron a la crisis de Cafarnaúm}
\par 
%\textsuperscript{(1698.1)}
\textsuperscript{152:0.1} LA HISTORIA de la curación de Amós, el lunático de Jeresa, ya había llegado hasta Betsaida y Cafarnaúm, de manera que una gran multitud esperaba a Jesús cuando su barca arribó aquel martes por la mañana. En esta multitud se encontraban los nuevos observadores enviados por el sanedrín de Jerusalén, que habían bajado a Cafarnaúm con el fin de encontrar un pretexto para arrestar e inculpar al Maestro. Mientras Jesús hablaba con la gente que se había reunido para saludarle, Jairo, uno de los jefes de la sinagoga, se abrió paso entre la muchedumbre, cayó a sus pies, lo cogió de la mano y le suplicó que se apresurara a ir con él, diciendo: <<Maestro, mi hijita, mi única hija, yace en mi casa a punto de morir. Te ruego que vengas a curarla>>. Cuando Jesús escuchó la petición de este padre, dijo: <<Iré contigo>>.

\par 
%\textsuperscript{(1698.2)}
\textsuperscript{152:0.2} Mientras Jesús acompañaba a Jairo, la gran multitud, que había escuchado la súplica del padre, los siguió para ver qué iba a suceder. Poco antes de llegar a la casa del jefe, mientras pasaban rápidamente por una calle estrecha con la muchedumbre empujándolo, Jesús se detuvo de pronto y exclamó: <<Alguien me ha tocado>>. Y cuando aquellos que estaban cerca de él negaron haberle tocado, Pedro dijo: <<Maestro, puedes ver que este gentío te apretuja, amenaza con aplastarnos, y sin embargo dices que `alguien me ha tocado'. ¿Qué quieres decir?>> Entonces Jesús dijo: <<He preguntado quién me ha tocado, porque he percibido que una energía viviente ha salido de mí>>. Jesús miró a su alrededor, y sus ojos se posaron en una mujer cercana, que se adelantó, se arrodilló a sus pies y dijo: <<Durante años he estado afligida con una hemorragia mortificante. Muchos médicos me han hecho sufrir mucho; he gastado todos mis bienes, pero ninguno ha podido curarme. Entonces oí hablar de ti, y pensé que si pudiera tocar solamente el borde de tu manto, seguramente me curaría. Así pues, apreté el paso con la gente a medida que caminaban hasta que, al estar cerca de ti, Maestro, he tocado el borde de tu manto, y he recuperado la salud; sé que me he curado de mi aflicción>>.

\par 
%\textsuperscript{(1698.3)}
\textsuperscript{152:0.3} Cuando Jesús escuchó esto, cogió a la mujer de la mano, la levantó y le dijo: <<Hija, tu fe te ha curado; ve en paz>>. Era su \textit{fe}, y no su \textit{contacto}, lo que la había curado. Este caso es un buen ejemplo de las muchas curaciones aparentemente milagrosas que acompañaron la carrera terrestre de Jesús, pero que él, en ningún sentido, deseó conscientemente. El paso del tiempo demostró que esta mujer se había curado realmente de su enfermedad. Su fe era del tipo que atrapaba directamente el poder creativo que residía en la persona del Maestro. Con la fe que tenía, sólo necesitaba acercarse a la persona del Maestro. No era necesario en absoluto que tocara su manto; eso era simplemente la parte supersticiosa de su creencia. Jesús llamó a su presencia a esta mujer de Cesarea de Filipo, llamada Verónica, para corregir dos errores que podrían haber permanecido en su mente, o que podrían haber perdurado en la mente de los que habían presenciado esta curación: No quería que Verónica se marchara pensando que su miedo por intentar robar su curación había sido premiado, o que su superstición de asociar el toque del vestido de Jesús con su curación había sido eficaz. Deseaba que todos supieran que era su \textit{fe} pura y viviente la que había efectuado la curación.

\section*{1. En la casa de Jairo}
\par 
%\textsuperscript{(1699.1)}
\textsuperscript{152:1.1} Jairo estaba, por supuesto, enormemente impaciente por esta demora en llegar a su casa; por eso ahora siguieron caminando con paso acelerado. Incluso antes de que entraran en el patio del jefe, uno de sus sirvientes salió diciendo: <<No molestes al Maestro; tu hija ha muerto>>. Pero Jesús no pareció prestar atención a las palabras del sirviente, porque, llevándose consigo a Pedro, Santiago y Juan, se volvió hacia el padre desconsolado y le dijo: <<No temas; limítate a creer>>. Cuando entró en la casa, encontró que los flautistas ya estaban allí con las plañideras formando un alboroto indecente; los parientes ya se habían puesto a llorar y a lamentarse. Después de echar a todas las plañideras de la habitación, entró con el padre, la madre y sus tres apóstoles. Había dicho a las plañideras que la doncella no estaba muerta, pero se rieron de él con desprecio. Jesús se volvió entonces hacia la madre, diciéndole: <<Tu hija no está muerta; sólo está dormida>>. Cuando la casa recuperó la tranquilidad, Jesús se acercó al lecho de la niña, la cogió de la mano y le dijo: <<Hija, yo te lo digo, ¡despierta y levántate!>> Cuando la chica escuchó estas palabras, se levantó inmediatamente y caminó por la habitación. Luego, cuando se hubo recuperado de su aturdimiento, Jesús ordenó que le dieran algo de comer, pues había estado mucho tiempo sin tomar alimento.

\par 
%\textsuperscript{(1699.2)}
\textsuperscript{152:1.2} Como había mucha agitación en Cafarnaúm en contra de Jesús, éste reunió a la familia y les explicó que la joven había caído en un estado de coma después de una fiebre prolongada, y que él se había limitado a despertarla, que no la había resucitado de entre los muertos. También explicó todo esto a sus apóstoles, pero fue en vano; todos creían que había resucitado a la chiquilla de entre los muertos. Todo lo que Jesús decía para explicar muchos de estos milagros aparentes, tenía poco efecto sobre sus seguidores. Eran propensos a ver milagros, y no perdían ni una oportunidad para atribuirle un nuevo prodigio a Jesús. Jesús y los apóstoles regresaron a Betsaida, después de haber encargado específicamente a todos que no se lo contaran a nadie.

\par 
%\textsuperscript{(1699.3)}
\textsuperscript{152:1.3} Cuando salió de la casa de Jairo, dos ciegos, guiados por un niño mudo, lo siguieron dando gritos para que los curara. Aproximadamente por esta época, la reputación de Jesús como sanador estaba en su apogeo. Por todas partes donde iba, los enfermos y los afligidos lo estaban esperando. El Maestro parecía ahora muy cansado, y todos sus amigos empezaban a preocuparse, pues si continuaba con su labor de enseñanza y de curación, acabaría por desplomarse.

\par 
%\textsuperscript{(1699.4)}
\textsuperscript{152:1.4} Los apóstoles de Jesús, sin contar a la gente común y corriente, no podían comprender la naturaleza y los atributos de este Dios-hombre. Ninguna generación posterior tampoco ha sido capaz de evaluar lo que sucedió en la Tierra en la persona de Jesús de Nazaret. Y la ciencia o la religión nunca tendrán la oportunidad de examinar estos acontecimientos notables, por la sencilla razón de que una situación así de extraordinaria no volverá a producirse nunca más en este mundo ni en ningún otro mundo de Nebadon. Nunca más volverá a aparecer, en ningún mundo de todo este universo, un ser en la similitud de la carne mortal que incorpore al mismo tiempo todos los atributos de la energía creativa, combinados con los dones espirituales que trascienden el tiempo y la mayoría de las otras limitaciones materiales.

\par 
%\textsuperscript{(1700.1)}
\textsuperscript{152:1.5} Antes de que Jesús estuviera en la Tierra, o después de entonces, nunca ha sido posible obtener de manera tan directa y gráfica los resultados que acompañan a la fe sólida y viviente de los hombres y las mujeres mortales. Para repetir estos fenómenos, tendríamos que ir a la presencia inmediata de Miguel, el Creador, y encontrarlo tal como era en aquella época ---el Hijo del Hombre. Asimismo, aunque su ausencia impide que estas manifestaciones materiales se produzcan hoy en día, deberíais absteneros de fijar cualquier tipo de limitación a la posible manifestación de su \textit{poder espiritual}. Aunque el Maestro está ausente como ser material, se encuentra presente como influencia espiritual en el corazón de los hombres. Al marcharse de este mundo, Jesús ha hecho posible que su espíritu viva al lado del de su Padre, que reside en la mente de todo el género humano.

\section*{2. La alimentación de los cinco mil}
\par 
%\textsuperscript{(1700.2)}
\textsuperscript{152:2.1} Jesús continuó enseñando a la gente durante el día, e instruyendo a los apóstoles y a los evangelistas por la noche. El viernes decretó una semana de vacaciones para que todos sus seguidores pudieran pasar unos días en sus casas o con sus amigos, antes de prepararse a subir a Jerusalén para la Pascua. Pero más de la mitad de sus discípulos se negaron a abandonarlo, y la multitud aumentaba diariamente hasta tal punto que David Zebedeo deseaba establecer un nuevo campamento, pero Jesús se negó a darle su consentimiento. El Maestro había descansado tan poco durante el sábado, que el domingo 27 de marzo por la mañana intentó alejarse de la gente. Algunos evangelistas se quedaron allí para hablarle a la multitud, mientras que Jesús y los doce planeaban escaparse, sin ser vistos, a la orilla opuesta del lago, donde pensaban encontrar el descanso que tanto necesitaban en un hermoso parque al sur de Betsaida-Julias. Esta región era un lugar de recreo favorito para los habitantes de Cafarnaúm; todos conocían bien estos parques de la costa oriental.

\par 
%\textsuperscript{(1700.3)}
\textsuperscript{152:2.2} Pero la gente no les dejó salirse con la suya. Vieron la dirección que tomaba la barca de Jesús, alquilaron todas las embarcaciones disponibles y salieron en su persecución. Los que no pudieron conseguir una barca se pusieron en camino para rodear a pie el extremo septentrional del lago.

\par 
%\textsuperscript{(1700.4)}
\textsuperscript{152:2.3} Al caer la tarde, más de mil personas habían localizado al Maestro en uno de los parques; él les habló brevemente, y Pedro lo hizo después. Mucha de esta gente había traído su comida, y después de cenar, se reunieron en pequeños grupos mientras los apóstoles y los discípulos de Jesús les enseñaban.

\par 
%\textsuperscript{(1700.5)}
\textsuperscript{152:2.4} El lunes por la tarde, la multitud había aumentado a más de tres mil personas. Y además ---ya entrada la noche--- la gente continuaba afluyendo, trayendo con ellos todo tipo de enfermos. Cientos de personas interesadas habían planeado detenerse en Cafarnaúm, en su camino hacia la Pascua, para ver y escuchar a Jesús, y se negaban sencillamente a sufrir un desengaño. El miércoles a mediodía, unos cinco mil hombres, mujeres y niños se habían congregado aquí, en este parque al sur de Betsaida-Julias. El tiempo era agradable, pues se acercaba el final de la estación de las lluvias en esta región.

\par 
%\textsuperscript{(1700.6)}
\textsuperscript{152:2.5} Felipe había traído provisiones para alimentar a Jesús y a los doce durante tres días, y estaban al cuidado del joven Marcos, su recadero. Este día por la tarde, el tercero para casi la mitad de esta multitud, los víveres que la gente había traído consigo estaban a punto de agotarse. David Zebedeo no contaba aquí con una ciudad de tiendas para alimentar y alojar a las multitudes. Felipe tampoco había previsto alimentos para una muchedumbre tan grande. Pero aunque la gente tenía hambre, no quería irse. Se cuchicheaba en voz baja que, como Jesús deseaba evitar dificultades tanto con Herodes como con los dirigentes de Jerusalén, había elegido este sitio tranquilo, fuera de la jurisdicción de todos sus enemigos, como el lugar adecuado para ser coronado rey. El entusiasmo de la gente aumentaba de hora en hora. A Jesús no le decían ni una palabra, aunque, por supuesto, sabía todo lo que estaba pasando. Incluso los doce apóstoles también estaban contaminados con estas ideas, y en especial los evangelistas más jóvenes. Los apóstoles que estaban a favor de esta tentativa para proclamar rey a Jesús eran Pedro, Juan, Simón Celotes y Judas Iscariote. Andrés, Santiago, Natanael y Tomás se oponían a este proyecto. Mateo, Felipe y los gemelos Alfeo no opinaban. El cabecilla de esta conspiración para hacerlo rey era Joab, uno de los jóvenes evangelistas.

\par 
%\textsuperscript{(1701.1)}
\textsuperscript{152:2.6} Ésta era la situación el miércoles hacia las cinco de la tarde, cuando Jesús le pidió a Santiago Alfeo que llamara a Andrés y a Felipe. Jesús dijo: <<¿Qué vamos a hacer con la multitud? Hace ya tres días que están con nosotros, y muchos de ellos tienen hambre. No tienen comida>>. Felipe y Andrés intercambiaron una mirada, y luego Felipe contestó: <<Maestro, deberías despedir a esta gente para que fueran a los pueblos de los alrededores a comprar comida>>. Andrés temía que se materializara la intriga para coronarlo rey, por lo que apoyó rápidamente a Felipe, diciendo: <<Sí, Maestro, creo que es mejor que despidas a la multitud para que se vayan por su camino y compren comida, y así consigues descansar algún tiempo>>. Mientras tanto, otros apóstoles se habían unido a la conversación. Jesús dijo entonces: <<Pero no deseo despedirlos hambrientos; ¿no podéis alimentarlos?>> Esto fue demasiado para Felipe, que dijo inmediatamente: <<Maestro, aquí en pleno campo, ¿dónde podemos comprar pan para esta multitud? Con doscientos denarios no tendríamos suficiente para un almuerzo>>.

\par 
%\textsuperscript{(1701.2)}
\textsuperscript{152:2.7} Antes de que los apóstoles tuvieran la posibilidad de expresarse, Jesús se volvió hacia Andrés y Felipe, diciendo: <<No quiero despedir a esta gente. Están aquí como ovejas sin pastor. Me gustaría alimentarlos. ¿De cuánta comida disponemos?>> Mientras Felipe conversaba con Mateo y Judas, Andrés buscó al joven Marcos para averiguar cuántas provisiones quedaban. Volvió hacia Jesús, diciendo: <<Al muchacho sólo le quedan cinco panes de cebada y dos pescados secos>> ---y Pedro añadió inmediatamente: <<Y aún tenemos que comer esta noche>>.

\par 
%\textsuperscript{(1701.3)}
\textsuperscript{152:2.8} Jesús permaneció en silencio durante un momento. Había en sus ojos una mirada lejana. Los apóstoles no decían nada. Jesús se volvió repentinamente hacia Andrés y dijo: <<Tráeme los panes y los peces>>. Cuando Andrés le trajo la canasta, el Maestro dijo: <<Ordenad a la gente que se siente en la hierba en grupos de cien, y que designen a un jefe para cada grupo, mientras traéis a todos los evangelistas aquí con nosotros>>.

\par 
%\textsuperscript{(1701.4)}
\textsuperscript{152:2.9} Jesús cogió los panes en sus manos y, después de dar las gracias, partió el pan y lo dio a sus apóstoles, que lo pasaron a sus compañeros, quienes a su vez lo llevaron a la multitud. Jesús partió y distribuyó los peces de la misma manera. Y aquella multitud comió hasta saciarse. Cuando hubieron terminado de comer, Jesús dijo a los discípulos: <<Recoged los trozos que quedan para que no se pierda nada>>. Cuando terminaron de recoger los pedazos, tenían doce canastas llenas. Unos cinco mil hombres, mujeres y niños habían comido en este banquete extraordinario.

\par 
%\textsuperscript{(1702.1)}
\textsuperscript{152:2.10} Éste fue el primero y el único milagro natural que Jesús efectuó después de haberlo planeado conscientemente. Es verdad que sus discípulos tenían tendencia a calificar de milagros muchas cosas que no lo eran, pero éste fue un auténtico ministerio sobrenatural. Se nos ha enseñado que, en este caso, Miguel multiplicó los elementos nutritivos como siempre lo hace, salvo que eliminó el factor tiempo y el encauzamiento vital observable.

\section*{3. El episodio de la coronación}
\par 
%\textsuperscript{(1702.2)}
\textsuperscript{152:3.1} La alimentación de los cinco mil por medio de la energía sobrenatural fue otro de esos casos en los que la compasión humana unida al poder creativo dieron como resultado lo que sucedió. Ahora que la multitud había sido saciada, y puesto que la fama de Jesús había aumentado aquí y ahora debido a este prodigio asombroso, el proyecto de apoderarse del Maestro y proclamarlo rey ya no necesitaba la dirección de nadie. La idea pareció propagarse entre la muchedumbre como un contagio. La reacción de la multitud ante esta satisfacción repentina y espectacular de sus necesidades físicas fue profunda e irresistible. A los judíos se les había enseñado durante mucho tiempo que cuando viniera el Mesías, el hijo de David, haría que la leche y la miel fluyeran de nuevo por la tierra, y que el pan de la vida les sería otorgado, tal como se suponía que el maná del cielo había caído sobre sus antepasados en el desierto. Todas estas expectativas, ¿no se habían cumplido ahora precisamente delante de sus ojos? Cuando esta multitud hambrienta y desnutrida hubo terminado de saciarse con el alimento milagroso, sólo tuvo una reacción unánime: <<Éste es nuestro rey>>. El libertador de Israel, obrador de prodigios, había llegado. A los ojos de esta gente sencilla, el poder de alimentar llevaba consigo el derecho a gobernar. Así pues, no es de extrañar que en cuanto la multitud hubo terminado de comer opíparamente, se levantara como un solo hombre, vociferando: <<¡Hacedlo rey!>>

\par 
%\textsuperscript{(1702.3)}
\textsuperscript{152:3.2} Este griterío poderoso entusiasmó a Pedro y a aquellos apóstoles que aún conservaban la esperanza de que Jesús afirmara su derecho a gobernar. Pero estas falsas esperanzas no iban a durar mucho tiempo. Apenas había dejado de resonar este poderoso griterío de la multitud en las rocas cercanas, cuando Jesús subió a una enorme piedra, levantó su mano derecha para atraer la atención, y dijo: <<Hijos míos, vuestras intenciones son buenas, pero tenéis la vista corta y tendencias materialistas>>. Hubo una breve pausa; este fornido galileo estaba allí plantado de manera majestuosa en el resplandor encantador de aquel crepúsculo oriental. Parecía un rey de pies a cabeza mientras continuó hablándole a esta multitud que retenía el aliento: <<Queréis hacerme rey, no porque vuestras almas hayan sido iluminadas por una gran verdad, sino porque vuestros estómagos han sido llenados de pan. ¿Cuántas veces os he dicho que mi reino no es de este mundo? El reino de los cielos que nosotros proclamamos es una fraternidad espiritual, y ningún hombre lo gobierna sentado en un trono material. Mi Padre que está en los cielos es el Soberano omnisapiente y todopoderoso de esta fraternidad espiritual de los hijos de Dios en la Tierra. ¿De tal manera he fallado en revelaros al Padre de los espíritus que queréis hacer rey a su Hijo en la carne? Ahora iros todos de aquí a vuestras propias casas. Si necesitáis a un rey, que el Padre de las luces sea entronizado en el corazón de cada uno de vosotros como Soberano espiritual de todas las cosas>>.

\par 
%\textsuperscript{(1702.4)}
\textsuperscript{152:3.3} Estas palabras de Jesús despidieron a la multitud atónita y descorazonada. Muchos de los que habían creído en él cambiaron de parecer y a partir de aquel día dejaron de seguirlo. Los apóstoles permanecían mudos, reunidos en silencio alrededor de las doce canastas con los restos de comida; sólo el joven Marcos, el chico de los recados, dijo: <<Y se negó a ser nuestro rey>>. Antes de marcharse para estar solo en las colinas, Jesús se volvió hacia Andrés y le dijo: <<Lleva a tus hermanos de regreso a la casa de Zebedeo y reza con ellos, especialmente por tu hermano Simón Pedro>>.

\section*{4. La visión nocturna de Simón Pedro}
\par 
%\textsuperscript{(1703.1)}
\textsuperscript{152:4.1} Los apóstoles sin su Maestro ---que los había hecho partir solos--- se montaron en la barca y empezaron a remar en silencio hacia Betsaida, en la orilla occidental del lago. Ninguno de los doce estaba tan abrumado y abatido como Simón Pedro. Apenas si pronunciaron una palabra; todos estaban pensando en el Maestro que se encontraba solo en las colinas. ¿Los había abandonado? Nunca antes los había despedido a todos, negándose a ir con ellos. ¿Qué podía significar todo esto?

\par 
%\textsuperscript{(1703.2)}
\textsuperscript{152:4.2} Se había levantado un fuerte viento contrario que casi les impedía avanzar, y la oscuridad cayó sobre ellos. A medida que pasaban las horas de oscuridad remando penosamente, Pedro, cada vez más cansado, cayó en un profundo sueño de agotamiento. Andrés y Santiago lo pusieron a descansar en el asiento acolchado de la popa de la barca. Mientras los otros apóstoles luchaban contra el viento y las olas, Pedro tuvo un sueño, una visión de Jesús que venía hacia ellos caminando por el mar. Cuando el Maestro pareció pasar cerca de la barca, Pedro gritó: <<Sálvanos, Maestro, sálvanos>>. Los que se encontraban en la parte posterior de la barca le oyeron decir algunas de estas palabras. Mientras esta aparición nocturna continuaba en la mente de Pedro, soñó que Jesús decía: <<Tened buen ánimo; soy yo; no temáis>>. Esto fue como un bálsamo de Galaad para el alma perturbada de Pedro; calmó su espíritu confuso, de manera que (en su sueño) gritó al Maestro: <<Señor, si eres tú realmente, ordéname venir y caminar contigo por el agua>>. Y cuando Pedro empezó a caminar por el agua, las olas turbulentas lo asustaron, y cuando estaba a punto de hundirse, gritó: <<Señor, ¡sálvame!>> La mayor parte de los doce lo escucharon proferir este grito. Entonces Pedro soñó que Jesús venía a rescatarlo, alargaba su mano, lo agarraba y lo levantaba, diciendo: <<Oh, hombre de poca fe, ¿por qué has dudado?>>

\par 
%\textsuperscript{(1703.3)}
\textsuperscript{152:4.3} En conexión con la última parte de su sueño, Pedro se levantó del asiento donde dormía, salió de la barca y cayó realmente al agua. Y se despertó de su sueño en el momento en que Andrés, Santiago y Juan se inclinaban y lo sacaban del mar.

\par 
%\textsuperscript{(1703.4)}
\textsuperscript{152:4.4} Para Pedro esta experiencia siempre fue real. Creía sinceramente que Jesús había venido hacia ellos aquella noche. Sólo convenció parcialmente a Juan Marcos, lo que explica por qué Marcos omitió una parte de la historia en su narración. Lucas, el médico, investigó cuidadosamente este asunto, y concluyó que el episodio era una visión de Pedro; por consiguiente, rehusó incorporar esta historia en el relato que estaba preparando.

\section*{5. De regreso en Betsaida}
\par 
%\textsuperscript{(1703.5)}
\textsuperscript{152:5.1} El jueves por la mañana, antes del amanecer, anclaron su barca cerca de la casa de Zebedeo y procuraron dormir hasta alrededor del mediodía. Andrés fue el primero que se levantó; se fue a dar un paseo cerca del mar, y encontró a Jesús en compañía del chico de los recados, sentado en una piedra al borde del agua. Un gran número de gente y de jóvenes evangelistas pasaron toda la noche y gran parte del día siguiente buscando a Jesús por las colinas orientales; pero poco después de la medianoche, Jesús y el joven Marcos habían partido a pie para rodear el lago y cruzar el río de regreso a Betsaida.

\par 
%\textsuperscript{(1704.1)}
\textsuperscript{152:5.2} De las cinco mil personas que habían sido alimentadas milagrosamente y que, con el estómago lleno y el corazón vacío, habían querido proclamarlo rey, sólo unas quinientas insistieron en seguirlo. Pero antes de que se enteraran de que había regresado a Betsaida, Jesús le pidió a Andrés que congregara a los doce apóstoles y a sus asociados, incluyendo a las mujeres, diciendo: <<Deseo hablar con ellos>>. Cuando todos estuvieron dispuestos, Jesús dijo:

\par 
%\textsuperscript{(1704.2)}
\textsuperscript{152:5.3} <<¿Cuánto tiempo seré indulgente con vosotros? ¿Sois todos torpes en comprender espiritualmente y estáis faltos de fe viviente? Todos estos meses os he enseñado las verdades del reino, y sin embargo estáis dominados por los móviles materiales en lugar de estarlo por las consideraciones espirituales. ¿No habéis leído siquiera en las Escrituras el pasaje donde Moisés exhorta a los hijos incrédulos de Israel, diciendo: `No temáis, permaneced tranquilos y contemplad la salvación del Señor'? El cantor dijo: `Poned vuestra confianza en el Señor'. `Sed pacientes, esperad al Señor y tened buen ánimo. Él fortalecerá vuestro corazón'. `Echad vuestra carga sobre el Señor, y él os sostendrá. Confiad en él en todo momento y desahogaos con él, porque Dios es vuestro refugio'. `El que reside en el lugar secreto del Altísimo, permanecerá a la sombra del Todopoderoso'. `Es mejor fiarse del Señor que poner la confianza en los príncipes humanos'.>>

\par 
%\textsuperscript{(1704.3)}
\textsuperscript{152:5.4} <<¿Comprendéis todos ahora que la producción de milagros y la ejecución de prodigios materiales no conquistarán almas para el reino espiritual? Hemos alimentado a la multitud, pero eso no los ha inducido a tener hambre del pan de la vida ni sed de las aguas de la rectitud espiritual. Una vez satisfecha su hambre, no trataron de entrar en el reino de los cielos, sino que intentaron proclamar rey al Hijo del Hombre a la manera de los reyes de este mundo, sólo para poder seguir comiendo pan sin tener que trabajar para ganarlo. Todo esto, en lo que muchos de vosotros habéis más o menos participado, no contribuye en nada a revelar el Padre celestial ni a hacer avanzar su reino en la Tierra. ¿No tenemos enemigos suficientes entre los jefes religiosos del país como para hacer lo posible por indisponer también a los gobernantes civiles? Ruego al Padre que unja vuestros ojos para que podáis ver y abra vuestros oídos para que podáis oír, a fin de que tengáis una fe plena en el evangelio que os he enseñado>>.

\par 
%\textsuperscript{(1704.4)}
\textsuperscript{152:5.5} Jesús anunció después que deseaba retirarse unos días para descansar con sus apóstoles, antes de que se prepararan a subir a Jerusalén para la Pascua, y a todos los discípulos y a la multitud les prohibió que lo siguieran. En consecuencia, salieron en barca hacia la región de Genesaret para descansar y dormir durante dos o tres días. Jesús se estaba preparando para una gran crisis de su vida en la Tierra, y por esta razón pasó mucho tiempo en comunión con el Padre que está en los cielos.

\par 
%\textsuperscript{(1704.5)}
\textsuperscript{152:5.6} La noticia de la alimentación de los cinco mil y del intento de convertir a Jesús en rey despertó una amplia curiosidad y suscitó los temores de los jefes religiosos y de los gobernantes civiles de toda Galilea y Judea. Este gran milagro no contribuyó en nada a fomentar el evangelio del reino en el alma de los creyentes propensos al materialismo y poco entusiastas, pero sí cumplió el objetivo de poner fin a las tendencias de la familia inmediata de Jesús, compuesta por los apóstoles y los discípulos íntimos, consistentes en buscar milagros y en desear ardientemente un rey. Este episodio espectacular puso fin a la primera época de enseñanza, instrucción y curación, preparando así el camino para la inauguración de este último año de proclamación de las fases superiores y más espirituales del nuevo evangelio del reino ---la filiación divina, la libertad espiritual y la salvación eterna.

\section*{6. En Genesaret}
\par 
%\textsuperscript{(1705.1)}
\textsuperscript{152:6.1} Mientras descansaba en la casa de un rico creyente de la región de Genesaret, Jesús mantuvo conversaciones informales con los doce todas las tardes. Los embajadores del reino formaban un grupo serio, sobrio y escarmentado de hombres desilusionados. Pero incluso después de todo lo que había sucedido, los acontecimientos posteriores revelaron que estos doce hombres no estaban todavía completamente liberados de sus ideas innatas y largo tiempo acariciadas sobre la venida del Mesías judío. Los acontecimientos de algunas semanas antes se habían desarrollado demasiado rápidamente como para que estos pescadores asombrados pudieran comprender todo su significado. Los hombres y las mujeres necesitan tiempo para efectuar cambios radicales y amplios en sus conceptos básicos y fundamentales sobre la conducta social, las actitudes filosóficas y las convicciones religiosas.

\par 
%\textsuperscript{(1705.2)}
\textsuperscript{152:6.2} Mientras Jesús y los doce descansaban en Genesaret, las multitudes se dispersaron; algunos regresaron a sus casas y otros se fueron a Jerusalén para la Pascua. En menos de un mes, los seguidores entusiastas y declarados de Jesús, que ascendían a más de cincuenta mil solamente en Galilea, se redujeron a menos de quinientos. Jesús deseaba que sus apóstoles pasaran por esta experiencia con la inconstancia de las aclamaciones populares, para que no se sintieran tentados a fiarse de estas manifestaciones de histeria religiosa transitoria después de que los hubiera dejado solos con el trabajo del reino; pero sólo consiguió un éxito parcial en este esfuerzo.

\par 
%\textsuperscript{(1705.3)}
\textsuperscript{152:6.3} La segunda noche de su estancia en Genesaret, el Maestro contó de nuevo a los apóstoles la parábola del sembrador y añadió estas palabras: <<Ya veis, hijos míos, que recurrir a los sentimientos humanos es transitorio y totalmente decepcionante; apelar exclusivamente al intelecto del hombre es igualmente vacío y estéril; sólo dirigiendo vuestro llamamiento al espíritu que vive dentro de la mente humana, podéis esperar conseguir un éxito duradero y efectuar esas maravillosas transformaciones del carácter humano que pronto se manifiestan mediante la producción abundante de los auténticos frutos del espíritu en la vida diaria de todos aquellos que se encuentran liberados así de las tinieblas de la duda mediante el nacimiento del espíritu en la luz de la fe ---el reino de los cielos>>.

\par 
%\textsuperscript{(1705.4)}
\textsuperscript{152:6.4} Jesús enseñó el recurso a las emociones como técnica para detener y concentrar la atención intelectual. A esa mente así despierta y avivada la calificó de puerta de entrada al alma, donde reside esa naturaleza espiritual del hombre que debe reconocer la verdad y responder al llamamiento espiritual del evangelio, a fin de producir los resultados permanentes de las verdaderas transformaciones del carácter.

\par 
%\textsuperscript{(1705.5)}
\textsuperscript{152:6.5} Jesús se esforzó así por preparar a los apóstoles para la conmoción inminente ---la crisis de la actitud del público hacia él, que iba a producirse pocos días después. Explicó a los doce que los dirigentes religiosos de Jerusalén conspirarían con Herodes Antipas para destruirlos. Los doce empezaron a comprender más plenamente (aunque no de manera definitiva) que Jesús no iba a sentarse en el trono de David. Percibieron más plenamente que los prodigios materiales no harían progresar la verdad espiritual. Empezaron a darse cuenta de que la alimentación de los cinco mil y el movimiento popular para hacer rey a Jesús fueron el apogeo de las expectativas del pueblo, que buscaba milagros y esperaba prodigios, y el punto culminante de las aclamaciones que Jesús recibía de la plebe. Discernían vagamente y entreveían débilmente los tiempos de la criba espiritual y de la cruel adversidad que se acercaban. Estos doce hombres se despertaban lentamente a la comprensión de la verdadera naturaleza de su tarea como embajadores del reino, y empezaron a prepararse para las pruebas difíciles y severas del último año del ministerio del Maestro en la Tierra.

\par 
%\textsuperscript{(1706.1)}
\textsuperscript{152:6.6} Antes de salir de Genesaret, Jesús les informó respecto a la alimentación milagrosa de los cinco mil, diciéndoles exactamente por qué había emprendido esta manifestación extraordinaria de poder creativo, y también les aseguró que no había cedido a su compasión por la multitud hasta que no hubo averiguado que aquello era <<conforme a la voluntad del Padre>>.

\section*{7. En Jerusalén}
\par 
%\textsuperscript{(1706.2)}
\textsuperscript{152:7.1} El domingo 3 de abril, Jesús partió de Betsaida para dirigirse a Jerusalén, acompañado únicamente por los doce apóstoles. Para evitar las multitudes y atraer el mínimo de atención posible, viajaron por el camino de Gerasa y Filadelfia. Les prohibió que hicieran cualquier tipo de enseñanza pública durante este viaje; tampoco les permitió que enseñaran o predicaran mientras estuvieran en Jerusalén. Llegaron a Betania, cerca de Jerusalén, el miércoles 6 de abril al anochecer. Aquella fue la única noche que se detuvieron en la casa de Lázaro, Marta y María, pues al día siguiente se separaron. Jesús se hospedó con Juan en la casa de un creyente llamado Simón, cerca de la casa de Lázaro en Betania. Judas Iscariote y Simón Celotes se quedaron con unos amigos en Jerusalén, mientras que el resto de los apóstoles residió, de dos en dos, en diferentes hogares.

\par 
%\textsuperscript{(1706.3)}
\textsuperscript{152:7.2} Durante esta Pascua, Jesús sólo entró una vez en Jerusalén, y lo hizo el gran día de la fiesta. Abner llevó a muchos creyentes de Jerusalén para que se reunieran con Jesús en Betania. Durante esta estancia en Jerusalén, los doce aprendieron cuán amargos se estaban volviendo los sentimientos hacia su Maestro. Todos partieron de Jerusalén convencidos de que una crisis era inminente.

\par 
%\textsuperscript{(1706.4)}
\textsuperscript{152:7.3} El domingo 24 de abril, Jesús y los apóstoles salieron de Jerusalén hacia Betsaida, pasando por las ciudades costeras de Jope, Cesarea y Tolemaida. Desde allí fueron por el interior a Ramá y Corazín, llegando a Betsaida el viernes 29 de abril. En cuanto estuvieron en casa, Jesús envió a Andrés a pedirle permiso al jefe de la sinagoga para hablar al día siguiente, sábado, en los oficios de la tarde. Jesús sabía muy bien que ésta era la última vez que le permitirían hablar en la sinagoga de Cafarnaúm.

\newpage
\pagestyle{empty}

\par {\huge Abreviaturas}
\bigbreak
\bigbreak
\begin{multicols}{2}
	\par LU \textit{(El Libro de Urantia)}
	\bigbreak
	\par Libros bíblicos:
	\bigbreak
	\par Abd \textit{(Abdías)}
	\par Am \textit{(Amós)}
	\par Ap \textit{(Apocalipsis)}
	\par Bar \textit{(Baruc)}
	\par Co \textit{(Epístola a los Corintios)}
	\par Cnt \textit{(El Cantar de los Cantares)}
	\par Col \textit{(Epístola a los Colosenses)}
	\par Cr \textit{(Crónicas)}
	\par Dn \textit{(Daniel)}
	\par Dt \textit{(Deuteronomio)}
	\par Ec \textit{(Eclesiastés)}
	\par Eclo \textit{(Ecclesiástico)}
	\par Ef \textit{(Epístola a los Efesios)}
	\par Esd \textit{(Esdras)}
	\par Est \textit{(Ester)}
	\par Ex \textit{(Éxodo)}
	\par Ez \textit{(Ezequiel)} 
	\par Flm \textit{(Epístola a Filemón)}
	\par Flp \textit{(Epístola a los Filipenses)}
	\par Gl \textit{(Epítosla a los Gálatas)}
	\par Gn \textit{(Génesis)}
	\par Hab \textit{(Habacuc)} 
	\par Hag \textit{(Ageo)}
	\par Hch \textit{(Hechos de los Apóstoles)}
	\par Heb \textit{(Epístola a los Hebreos)}
	\par Is \textit{(Isaías)}
	\par Jer \textit{(Jeremías)}
	\par Jl \textit{(Joel)}
	\par Jn \textit{(Juan, evangelio y epístolas)}
	\par Job \textit{(Job)}
	\par Jon \textit{(Jonás)}
	\par Jos \textit{(Josué)}
	\par Jud \textit{(Epístola de Judas)}
	\par Jue \textit{(Jueces)}
	\par Lc \textit{(Lucas)}
	\par Lm \textit{(Lamentaciones)}
	\par Lv \textit{(Levítico)}
	\par Mac \textit{(Macabeos)}
	\par Mal \textit{(Malaquías)}
	\par Mc \textit{(Marcos)}
	\par Miq \textit{(Miqueas)} 
	\par Mt \textit{(Mateo)}
	\par Nah \textit{(Nahúm)}
	\par Neh \textit{(Nehemías)} 
	\par Nm \textit{(Números)}
	\par Os \textit{(Oseas)}
	\par P \textit{(Epístola de Pedro)}
	\par Pr \textit{(Proverbios)}
	\par Re \textit{(Reyes)}
	\par Ro \textit{(Epístola a los Romanos)}
	\par Rt \textit{(Rut)}
	\par Sab \textit{(Sabiduría)}
	\par Sal \textit{(Salmos)}
	\par Sam \textit{(Samuel)}
	\par Sof \textit{(Sofonías)}
	\par Stg \textit{(Epístola a Santiago)}
	\par Ti \textit{(Epístola a Timoteo)}
	\par Tit \textit{(Epítosla a Tito)}
	\par Ts \textit{(Epístola a los Tesalonicenses)}
	\par Zac \textit{(Zacarías)}
	\bigbreak
	\par Libros bíblicos apócrifos:
	\bigbreak 
	\par AsMo \textit{(Asunción de Moisés)}
	\par Bel \textit{(Bel y el Dragón)} 
	\par Hen \textit{(Enoc)} 
	\par Man \textit{(Oración de Manasés)} 
	\par Tb \textit{(Tobit)}
	\bigbreak
	\par Libros de otras religiones: 
	\bigbreak
	\par XXX \textit{(YYYY)}
	
	
\end{multicols}

\end{document}
