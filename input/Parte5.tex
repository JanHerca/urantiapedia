% Author of this conversion to LaTeX format: Jan Herca, 2017
\documentclass[twoside, 11pt]{book}
\usepackage[T1]{fontenc} % indica al procesador cómo imprimir los caracteres
\usepackage{fontspec} % permite definir fuentes a partir de las instaladas en el SO
\usepackage{geometry}
\usepackage{graphicx}
\usepackage{float}
\usepackage{tocloft}
\usepackage{titleps}
\usepackage{emptypage}
\usepackage[spanish]{babel}
\usepackage{multicol}
% Text styles
\geometry{paperwidth=16cm, paperheight=24cm, top=2.5cm, bottom=1.7cm, inner=2.5cm, outer=1.2cm}

\makeatletter
\def\@makechapterhead#1{%
	\vspace*{50\p@}%
	{\parindent \z@ \raggedright \normalfont
		\interlinepenalty\@M
		\huge \bfseries #1\par\nobreak
		\vskip 40\p@
}}
\def\@makeschapterhead#1{%
	\vspace*{50\p@}%
	{\parindent \z@ \raggedright
		\normalfont
		\interlinepenalty\@M
		\huge \bfseries  #1\par\nobreak
		\vskip 40\p@
}}
\makeatother

\renewcommand{\cftchapleader}{\cftdotfill{\cftdotsep}}
\renewcommand{\thechapter}{}
\renewcommand{\cftchapfont}{\large}
\cftsetpnumwidth{3em}
\renewcommand{\cftchappagefont}{\large}


\title{La Quinta Revelación \newline Quinto Volumen \newline La religión, la sobrevivencia a la muerte y la Deidad experiencial}
\date{}
\begin{document}
	
\begin{titlepage}
	\centering
	{\Huge\bfseries El Libro de Urantia\par}
	{\huge\bfseries La Quinta Revelación\par}
	\vspace{1cm}
	{\huge\bfseries Quinto Volumen\par}
	\vspace{1cm}
	{\huge\bfseries La religión,\par}
	{\huge\bfseries la sobrevivencia a la muerte\par}
	{\huge\bfseries y la Deidad experiencial\par}
	\vfill
	{\scshape\Large URANTIA FOUNDATION\par}
	{\scshape\Large CHICAGO ILLINOIS\par}
	{\Large 2009 Traducción al español Europea\par}
\end{titlepage}
	
	
\par {\textcopyright} 2019 Jan Herca, de la edición
\par {\textcopyright} 2009 Urantia Foundation, de la traducción
\par {\textcopyright} 1993 Urantia Foundation, de otros materiales
\bigbreak
\par Jan Herca
\par Correo electrónico: janherca@gmail.com
\bigbreak
\par Urantia Foundation
\par 533 West Diversey Parkway
\par Chicago, IL 60614 EE.UU.A
\par Oficina: 1+(773) 525-3319
\par Fax: 1 +(773) 525-7739
\par Website: http://www.urantia.org
\par Correo electrónico: urantia@urantia.org
\bigbreak
\par Todos los derechos reservados, incluyendo el de traducción en los Estados Unidos de América, Canadá y en los demás países de la Unión Internacional de copyright. Todos los derechos reservados en los paises firmantes de la Union Panamericana de la Union internacional de copyright.
\par No todo el libro ni parte de él pueden ser copiados, reproducidos o traducidos en forma alguna, ya sea por medio electrónico, mecánico u otra forma, como fotocopia, grabación o archivo computerizado sin autorización por escrito del editor.
\par URANTIA,'' ``URANTIAN,'' ``EL LIBRO DE URANTIA'' y son marcas registradas de Urantia Foundation y su uso está sujeto a licencia.
\bigbreak
\par La Quinta Revelación es una reedición de El Libro de Urantia (Edición Europea). Está dividido en siete volúmenes para hacerlo más manejable y dispone de contenido adicional en forma de ayudas a la lectura integradas en el texto. El Libro de Urantia (Edición Europea) es una traducción de The Urantia Book realizada por la Fundación Urantia en 2009. 
\newpage

\begin{center}
	{\huge\bfseries Las partes del libro\par}
	\vspace{1cm}
	{\scshape\large PRIMER VOLUMEN\par}
	{\scshape\Large DIOS, EL UNIVERSO CENTRAL Y LOS SUPERUNIVERSOS\par}
	\vspace{1cm}
	
	{\scshape\large SEGUNDO VOLUMEN \par}
	{\scshape\Large EL UNIVERSO LOCAL\par}
	\vspace{1cm}
	
	{\scshape\large TERCER VOLUMEN \par}
	{\scshape\Large LA HISTORIA DE NUESTRO PLANETA, URANTIA\par}
	\vspace{1cm}
	
	{\scshape\large CUARTO VOLUMEN \par}
	{\scshape\Large LA EVOLUCIÓN DE LA CIVILIZACIÓN HUMANA\par}
	\vspace{1cm}
	
	{\scshape\large QUINTO VOLUMEN \par}
	{\scshape\Large LA RELIGIÓN, LA SOBREVIVENCIA A LA MUERTE Y LA DEIDAD EXPERIENCIAL\par}
	\vspace{1cm}
	
	{\scshape\large SEXTO VOLUMEN \par}
	{\scshape\Large LA VIDA Y LAS ENSEÑANZAS DE JESÚS - I\par}
	\vspace{1cm}
	
	{\scshape\large SÉPTIMO VOLUMEN \par}
	{\scshape\Large LA VIDA Y LAS ENSEÑANZAS DE JESÚS - II\par}
\end{center}
	
\newpage
\begin{center}
	{\small \textit {Intencionadamente en blanco}\par}
\end{center}
\newpage

\pagestyle{empty}


\tableofcontents

\newpagestyle{main}{
	%\setheadrule{1pt}% Header rule
	%\setfootrule{.4pt}% Footer rule
	\sethead[\small \thepage]% odd-left
	[]% odd-center
	[\begin{minipage}{0.9\textwidth}\begin{flushright}\scriptsize \MakeUppercase{\chaptertitle}\end{flushright}\end{minipage}]% odd-right
	{\begin{minipage}{0.9\textwidth}\scriptsize \MakeUppercase{\chaptertitle}\end{minipage}}% even-left
	{}% even-center
	{\small \thepage}% even-right
	\setfoot[]% odd-left
	[]% odd-center
	[]% odd-right
	{}% even-left
	{}% even-center
	{}% even-right
}

\pagestyle{main}
\renewcommand{\makeheadrule}{\rule[-.6\baselineskip]{\linewidth}{.4pt}}



\chapter{Documento 99. Los problemas sociales de la religión}
\par
%\textsuperscript{(1086.1)}
\textsuperscript{99:0.1} LA RELIGIÓN consigue aportar su ministerio social más elevado cuando posee una conexión mínima con las instituciones laicas de la sociedad. En las épocas pasadas, puesto que las reformas sociales estaban limitadas principalmente al terreno moral, la religión no tenía que ajustar su actitud a los grandes cambios de los sistemas económicos y políticos. El problema principal de la religión consistía en intentar reemplazar el mal por el bien dentro del orden social existente de la cultura política y económica. La religión ha tendido así a perpetuar indirectamente el orden establecido de la sociedad, a fomentar el mantenimiento del tipo de civilización existente.

\par
%\textsuperscript{(1086.2)}
\textsuperscript{99:0.2} Pero la religión no debería ocuparse directamente de crear nuevos órdenes sociales ni de conservar los antiguos. La verdadera religión se opone a la violencia como técnica de evolución social, pero no se opone a los esfuerzos inteligentes de la sociedad por adaptar sus costumbres y ajustar sus instituciones a las nuevas condiciones económicas y exigencias culturales.

\par
%\textsuperscript{(1086.3)}
\textsuperscript{99:0.3} La religión aprobó las reformas sociales ocasionales de los siglos pasados, pero en el siglo veinte está obligada a enfrentarse con los ajustes que ha de realizar ante una reconstrucción social amplia y continuada. Las condiciones de vida cambian con tanta rapidez que hay que acelerar enormemente las modificaciones institucionales y, por consiguiente, la religión debe apresurar su adaptación a este nuevo orden social en constante cambio.

\section*{1. La religión y la reconstrucción social}
\par
%\textsuperscript{(1086.4)}
\textsuperscript{99:1.1} Las invenciones mecánicas y la diseminación del conocimiento están modificando la civilización; si se quiere evitar un desastre cultural, es imperioso efectuar ciertos ajustes económicos y cambios sociales. Este nuevo orden social que se aproxima no se establecerá afablemente durante un milenio. La raza humana debe aceptar una serie de cambios, ajustes y reajustes. La humanidad está en marcha hacia un nuevo destino planetario no revelado.

\par
%\textsuperscript{(1086.5)}
\textsuperscript{99:1.2} La religión debe ejercer una poderosa influencia a favor de la estabilidad moral y del progreso espiritual, desempeñando dinámicamente sus funciones en medio de estas condiciones cambiantes y de estos ajustes económicos sin fin.

\par
%\textsuperscript{(1086.6)}
\textsuperscript{99:1.3} La sociedad de Urantia nunca puede esperar estabilizarse como en las épocas pasadas. El navío social ha zarpado de las bahías abrigadas de la tradición establecida, y ha empezado a navegar en el alta mar del destino evolutivo; el alma del hombre necesita, como nunca antes en toda la historia del mundo, escudriñar cuidadosamente sus mapas de moralidad y observar esmeradamente la brújula de su orientación religiosa. La misión suprema de la religión, como influencia social, consiste en estabilizar los ideales de la humanidad durante esos peligrosos períodos de transición entre una fase de civilización y la siguiente, entre un nivel de cultura y el siguiente.

\par
%\textsuperscript{(1087.1)}
\textsuperscript{99:1.4} La religión no tiene ningún deber nuevo que cumplir, pero se le pide que actúe urgentemente como guía sabia y consejera experimentada en todas estas nuevas situaciones humanas que cambian con rapidez. La sociedad se está volviendo más mecánica, más compacta, más compleja y más críticamente interdependiente. La religión debe ejercer su actividad para impedir que estas nuevas interasociaciones íntimas se vuelvan mutuamente retrógradas o incluso destructivas. La religión debe actuar como la sal cósmica que impide que los fermentos del progreso destruyan el sabor cultural de la civilización.
Únicamente el ministerio de la religión puede conducir a estas relaciones sociales y agitaciones económicas nuevas hacia una fraternidad duradera.

\par
%\textsuperscript{(1087.2)}
\textsuperscript{99:1.5} Humanamente hablando, un humanitarismo ateo es un noble gesto, pero la verdadera religión es la única fuerza que puede acrecentar de forma duradera la sensibilidad de un grupo social hacia las necesidades y los sufrimientos de otros grupos. En el pasado, la religión institucional podía permanecer pasiva mientras las capas superiores de la sociedad hacían oídos sordos a los sufrimientos y la opresión de las capas inferiores desamparadas, pero en los tiempos modernos, estas clases sociales inferiores ya no son tan abyectamente ignorantes ni están políticamente tan indefensas.

\par
%\textsuperscript{(1087.3)}
\textsuperscript{99:1.6} La religión no debe implicarse orgánicamente en el trabajo laico de la reconstrucción social ni de la reorganización económica. Pero debe seguir activamente el mismo ritmo que todos estos progresos de la civilización, repitiendo con claridad y energía sus mandatos morales y sus preceptos espirituales, su filosofía progresiva de la vida humana y de la supervivencia trascendente. El espíritu de la religión es eterno, pero la forma de expresarlo debe ser expuesta de nuevo cada vez que se revise el diccionario de la lengua humana.

\section*{2. La debilidad de la religión institucional}
\par
%\textsuperscript{(1087.4)}
\textsuperscript{99:2.1} La religión institucional no puede proporcionar inspiración ni ofrecer directrices para esta reconstrucción social y esta reorganización económica inminentes a escala mundial, porque se ha vuelto desgraciadamente una parte más o menos orgánica del orden social y del sistema económico que están destinados a ser reconstruidos. Sólo la verdadera religión de la experiencia espiritual personal puede ejercer sus funciones de manera útil y creativa en la crisis actual de la civilización.

\par
%\textsuperscript{(1087.5)}
\textsuperscript{99:2.2} La religión institucional está ahora atrapada en el punto muerto de un círculo vicioso. No puede reconstruir la sociedad sin reconstruirse primero a sí misma; y como es una parte integrante tan grande del orden establecido, no puede reconstruirse a sí misma hasta que la sociedad haya sido radicalmente reconstruida.

\par
%\textsuperscript{(1087.6)}
\textsuperscript{99:2.3} Las personas religiosas deben ejercer su actividad en la sociedad, en la industria y en la política como individuos, no como grupos, partidos o instituciones. Un grupo religioso que se permite actuar como tal fuera de sus actividades religiosas, se convierte inmediatamente en un partido político, una organización económica o una institución social. El colectivismo religioso debe limitar sus esfuerzos a fomentar las causas religiosas.

\par
%\textsuperscript{(1087.7)}
\textsuperscript{99:2.4} Las personas religiosas no tienen más valor que las personas no religiosas en las tareas de la reconstrucción social, excepto en la medida en que su religión les haya conferido una mayor previsión cósmica y las haya dotado de esa sabiduría social superior nacida del deseo sincero de amar a Dios de manera suprema, y de amar a cada hombre como a un hermano en el reino celestial. El orden social ideal es aquél en el que cada hombre ama a su prójimo tal como se ama a sí mismo.

\par
%\textsuperscript{(1087.8)}
\textsuperscript{99:2.5} La iglesia institucionalizada puede dar la apariencia de haber servido a la sociedad en el pasado glorificando el orden político y económico establecido, pero si desea sobrevivir, debe poner fin rápidamente a toda actividad de este tipo. Su única actitud adecuada consiste en enseñar la no violencia, la doctrina de la evolución pacífica en lugar de la revolución violenta ---la paz en la Tierra y la buena voluntad entre todos los hombres.

\par
%\textsuperscript{(1088.1)}
\textsuperscript{99:2.6} Si la religión moderna encuentra difícil ajustar su actitud a las transformaciones sociales que varían con rapidez, es únicamente porque se ha permitido volverse completamente tradicional, dogmatizada e institucionalizada. La religión de la experiencia viviente no encuentra ninguna dificultad en mantenerse por delante de todos esos desarrollos sociales y agitaciones económicas, desempeñando siempre su actividad en medio de ellos como estabilizadora moral, guía social y piloto espiritual. La verdadera religión transporta de una época a la siguiente la cultura que merece la pena y esa sabiduría que ha nacido de la experiencia de conocer a Dios y de esforzarse por parecerse a él.

\section*{3. La religión y las personas religiosas}
\par
%\textsuperscript{(1088.2)}
\textsuperscript{99:3.1} El cristianismo primitivo estaba totalmente libre de los enredos civiles, los compromisos sociales y las alianzas económicas. Sólo el cristianismo institucionalizado posterior se convirtió en una parte orgánica de la estructura política y social de la civilización occidental.

\par
%\textsuperscript{(1088.3)}
\textsuperscript{99:3.2} El reino de los cielos no es ni un orden social ni un orden económico; es una fraternidad exclusivamente espiritual de individuos que conocen a Dios. Es verdad que esta fraternidad constituye en sí misma un fenómeno social nuevo y sorprendente, que va acompañado de unas repercusiones políticas y económicas asombrosas.

\par
%\textsuperscript{(1088.4)}
\textsuperscript{99:3.3} La persona religiosa no es indiferente al sufrimiento social, ni hace caso omiso de la injusticia civil, ni está aislada del pensamiento económico, ni es insensible a la tiranía política. La religión influye directamente sobre la reconstrucción social porque espiritualiza y proporciona unos ideales al ciudadano individual. La civilización cultural está influida indirectamente por la actitud de estas personas religiosas individuales a medida que se convierten en miembros activos e influyentes de los diversos grupos sociales, morales, económicos y políticos.

\par
%\textsuperscript{(1088.5)}
\textsuperscript{99:3.4} Para conseguir una civilización cultural elevada se necesita, en primer lugar, el tipo ideal de ciudadano, y a continuación unos mecanismos sociales ideales y adecuados con los que estos ciudadanos puedan controlar las instituciones económicas y políticas de esa sociedad humana avanzada.

\par
%\textsuperscript{(1088.6)}
\textsuperscript{99:3.5} Debido a un exceso de falso sentimentalismo, la iglesia ha socorrido durante mucho tiempo a los desvalidos y a los infelices, y todo eso ha estado muy bien, pero este mismo sentimentalismo ha conducido a la perpetuación imprudente de unos linajes racialmente degenerados que han retrasado enormemente el progreso de la civilización.

\par
%\textsuperscript{(1088.7)}
\textsuperscript{99:3.6} Aunque muchos reconstructores sociales individuales rechazan con vehemencia la religión institucionalizada, son, después de todo, unos religiosos entusiastas a la hora de propagar sus reformas sociales. Así es como una motivación religiosa personal y más o menos no reconocida juega un papel importante en el programa actual de reconstrucción social.

\par
%\textsuperscript{(1088.8)}
\textsuperscript{99:3.7} La gran debilidad de todo este tipo de actividad religiosa no reconocida e inconsciente reside en que es incapaz de sacar provecho de una crítica religiosa abierta y de alcanzar, por medio de ella, unos niveles beneficiosos de autocorrección. Es un hecho que la religión no progresa a menos que esté disciplinada por la crítica constructiva, ampliada por la filosofía, purificada por la ciencia y alimentada por una camaradería leal.

\par
%\textsuperscript{(1088.9)}
\textsuperscript{99:3.8} Siempre existe el gran peligro de que la religión se deforme y se desnaturalice y empiece a perseguir metas erróneas, como sucede en los tiempos de guerra, cuando cada nación en conflicto prostituye su religión transformándola en propaganda militar. El fervor sin amor siempre es perjudicial para la religión, mientras que la persecución desvía las actividades de la religión hacia la realización de alguna campaña sociológica o teológica.

\par
%\textsuperscript{(1089.1)}
\textsuperscript{99:3.9} La religión sólo puede mantenerse libre de las alianzas laicas profanas por medio de:

\par
%\textsuperscript{(1089.2)}
\textsuperscript{99:3.10} 1. Una filosofía críticamente correctiva.

\par
%\textsuperscript{(1089.3)}
\textsuperscript{99:3.11} 2. La independencia de toda alianza social, económica y política.

\par
%\textsuperscript{(1089.4)}
\textsuperscript{99:3.12} 3. Unas comunidades creativas, reconfortantes y que expandan el amor.

\par
%\textsuperscript{(1089.5)}
\textsuperscript{99:3.13} 4. El aumento progresivo de la perspicacia espiritual y de la apreciación de los valores cósmicos.

\par
%\textsuperscript{(1089.6)}
\textsuperscript{99:3.14} 5. La prevención del fanatismo mediante las compensaciones que ofrece una actitud mental científica.

\par
%\textsuperscript{(1089.7)}
\textsuperscript{99:3.15} Las personas religiosas, como grupo, nunca deben ocuparse de otra cosa que no sea de \textit{religión}, aunque cada una de estas personas, como ciudadano individual, puede convertirse en el dirigente destacado de algún movimiento de reconstrucción social, económica o política.

\par
%\textsuperscript{(1089.8)}
\textsuperscript{99:3.16} La tarea de la religión consiste en crear, sostener e inspirar en el ciudadano individual la lealtad cósmica que lo dirija a lograr el éxito en el progreso de todos estos servicios sociales difíciles, pero deseables.

\section*{4. Dificultades de transición}
\par
%\textsuperscript{(1089.9)}
\textsuperscript{99:4.1} La religión auténtica hace que la persona religiosa resulte socialmente fragante y crea la comprensión de la hermandad humana. Pero la formalización de los grupos religiosos destruye muchas veces los valores mismos para la promoción de los cuales el grupo se había organizado. La amistad humana y la religión divina se ayudan mutuamente y se iluminan de modo significativo si cada una de ellas crece con equilibrio y armonía. La religión da un nuevo sentido a todas las asociaciones de grupo ---familias, escuelas y clubes. Confiere nuevos valores a las diversiones y ensalza el verdadero humor.

\par
%\textsuperscript{(1089.10)}
\textsuperscript{99:4.2} La perspicacia espiritual transforma a los dirigentes sociales; la religión impide que todos los movimientos colectivos pierdan de vista sus verdaderos objetivos. La religión, junto con los niños, es la gran unificadora de la vida familiar, a condición de que se trate de una fe viviente y creciente. No se puede tener una vida familiar sin niños; una vida así se puede vivir sin religión, pero esta desventaja multiplica enormemente las dificultades de esta íntima asociación humana. Durante las primeras décadas del siglo veinte, la vida familiar, junto con la experiencia religiosa personal, es la que más sufre la decadencia resultante de la transición entre las antiguas lealtades religiosas y los nuevos significados y valores emergentes.

\par
%\textsuperscript{(1089.11)}
\textsuperscript{99:4.3} La verdadera religión es una manera significativa de vivir dinámicamente enfrentándose a las realidades corrientes de la vida diaria. Pero si la religión ha de estimular el desarrollo individual del carácter y acrecentar la integración de la personalidad, no debe ser uniformizada. Si ha de alentar la evaluación de la experiencia y servir como un aliciente de valor, no debe ser estereotipada. Si la religión ha de fomentar las lealtades supremas, no debe ser formalista.

\par
%\textsuperscript{(1089.12)}
\textsuperscript{99:4.4} Cualesquiera que sean los trastornos que puedan acompañar al crecimiento social y económico de la civilización, la religión es auténtica y valiosa si fomenta en el individuo una experiencia en la que prevalece la soberanía de la verdad, la belleza y la bondad, porque éste es el verdadero concepto espiritual de la realidad suprema. Y a través del amor y de la adoración, todo esto adquiere significado bajo la forma de la hermandad con los hombres y la filiación con Dios.

\par
%\textsuperscript{(1090.1)}
\textsuperscript{99:4.5} Después de todo, lo que uno cree, más bien que lo que uno sabe, es lo que determina la conducta y domina las acciones personales. El conocimiento basado puramente en los hechos ejerce muy poca influencia sobre el hombre medio, a menos que sea activado emocionalmente. Pero la activación de la religión es superemocional, unificando toda la experiencia humana en unos niveles trascendentes por medio del contacto y la liberación de las energías espirituales en la vida mortal.

\par
%\textsuperscript{(1090.2)}
\textsuperscript{99:4.6} Durante los tiempos psicológicamente agitados del siglo veinte, en medio de los trastornos económicos, las contracorrientes morales y las mareas sociológicas desgarradoras de las transiciones ciclónicas de una era científica, miles y miles de hombres y de mujeres se han dislocado humanamente; están ansiosos, inquietos, temerosos, inseguros e inestables; necesitan, más que nunca en la historia del mundo, el consuelo y la estabilidad de una religión sana. Existe un estancamiento espiritual y un caos filosófico en presencia de unos logros científicos y de unos desarrollos mecánicos sin precedentes.

\par
%\textsuperscript{(1090.3)}
\textsuperscript{99:4.7} No existe ningún peligro en que la religión se vuelva cada vez más un asunto privado ---una experiencia personal--- con tal que no pierda de vista su motivación de servicio social desinteresado y amoroso. La religión ha sufrido muchas influencias secundarias: la mezcla repentina de las culturas, la entremezcla de los credos, la disminución de la autoridad eclesiástica, la modificación de la vida familiar, así como la urbanización y la mecanización.

\par
%\textsuperscript{(1090.4)}
\textsuperscript{99:4.8} El mayor peligro espiritual para el hombre consiste en el progreso parcial, en la difícil situación de un crecimiento incompleto: en abandonar las religiones evolutivas del miedo sin aferrarse inmediatamente a la religión revelada del amor. La ciencia moderna, y en particular la psicología, sólo ha debilitado a aquellas religiones que dependen tan ampliamente del miedo, la superstición y las emociones.

\par
%\textsuperscript{(1090.5)}
\textsuperscript{99:4.9} Una transición siempre va acompañada de confusión, y el mundo religioso disfrutará de poca tranquilidad hasta que no finalice la gran lucha entre las tres filosofías de la religión que están en conflicto:

\par
%\textsuperscript{(1090.6)}
\textsuperscript{99:4.10} 1. La creencia espiritista (en una Deidad providencial) de muchas religiones.

\par
%\textsuperscript{(1090.7)}
\textsuperscript{99:4.11} 2. La creencia humanista e idealista de muchas filosofías.

\par
%\textsuperscript{(1090.8)}
\textsuperscript{99:4.12} 3. Las ideas mecanicistas y naturalistas de muchas ciencias.

\par
%\textsuperscript{(1090.9)}
\textsuperscript{99:4.13} Estas tres aproximaciones parciales a la realidad del cosmos deberán armonizarse finalmente gracias a la presentación revelatoria de la religión, la filosofía y la cosmología, que describe la existencia trina del espíritu, la mente y la energía que provienen de la Trinidad del Paraíso y que alcanzan su unificación espacio-temporal dentro de la Deidad del Supremo.

\section*{5. Los aspectos sociales de la religión}
\par
%\textsuperscript{(1090.10)}
\textsuperscript{99:5.1} Aunque la religión es exclusivamente una experiencia espiritual personal ---conocer a Dios como Padre--- el corolario de esta experiencia ---conocer al hombre como hermano--- implica la adaptación del yo a otros yoes, y esto supone el aspecto social o colectivo de la vida religiosa. La religión es en primer lugar una adaptación interior o personal, y luego se convierte en un asunto de servicio social o de adaptación a un grupo. El hecho de la tendencia gregaria del hombre provoca forzosamente el nacimiento de los grupos religiosos. Lo que les suceda a esos grupos religiosos depende mucho de la inteligencia de sus dirigentes. En las sociedades primitivas, el grupo religioso no siempre es muy diferente de los grupos económicos o políticos. La religión ha sido siempre una conservadora de la moral y una estabilizadora de la sociedad. Y esto continua siendo cierto a pesar de que muchos socialistas y humanistas modernos enseñan lo contrario.

\par
%\textsuperscript{(1091.1)}
\textsuperscript{99:5.2} Recordad siempre que la verdadera religión consiste en conocer a Dios como vuestro Padre y al hombre como vuestro hermano. La religión no es una creencia servil en unas amenazas de castigo o en las promesas mágicas de unas recompensas místicas futuras.

\par
%\textsuperscript{(1091.2)}
\textsuperscript{99:5.3} La religión de Jesús es la influencia más dinámica que haya activado nunca a la raza humana. Jesús hizo pedazos las tradiciones, destruyó los dogmas e invitó a la humanidad a que realizara sus ideales más elevados en el tiempo y en la eternidad ---a ser perfecta como el Padre que está en los cielos es perfecto\footnote{\textit{Sed perfectos}: Gn 17:1; 1 Re 8:61; Lv 19:2; Dt 18:13; Mt 5:48; 2 Co 13:11; Stg 1:4; 1 P 1:16.}.

\par
%\textsuperscript{(1091.3)}
\textsuperscript{99:5.4} La religión tiene pocas posibilidades de ejercer su actividad hasta que el grupo religioso no se separe de todos los demás grupos ---hasta que forme la asociación social de los miembros espirituales del reino de los cielos.

\par
%\textsuperscript{(1091.4)}
\textsuperscript{99:5.5} La doctrina de la depravación total del hombre ha destruido una gran parte del potencial que tenía la religión para llevar a cabo unas repercusiones sociales de naturaleza elevadora y de valor inspirador. Jesús trató de restablecer la dignidad del hombre cuando declaró que todos los hombres son hijos de Dios\footnote{\textit{Todos los hombres son hijos de Dios}: 1 Cr 22:10; Sal 2:7; Is 56:5; Mt 5:9,16,45; Lc 20:36; Jn 1:12-13; 11:52; Hch 17:28-29; Ro 8:14-17,19,21; 9:26; 2 Co 6:18; Gl 3:26; 4:5-7; Ef 1:5; Flp 2:15; Heb 12:5-8; 1 Jn 3:1-2,10; 5:2; Ap 21:7; 2 Sam 7:14.}.

\par
%\textsuperscript{(1091.5)}
\textsuperscript{99:5.6} Cualquier creencia religiosa que logre espiritualizar al creyente no dejará de producir unas repercusiones poderosas en la vida social de esa persona. La experiencia religiosa produce infaliblemente los <<frutos del espíritu>>\footnote{\textit{Frutos del espíritu}: Gl 5:22-23; Ef 5:9.} en la vida diaria del mortal dirigido por el espíritu.

\par
%\textsuperscript{(1091.6)}
\textsuperscript{99:5.7} Con la misma seguridad con que los hombres comparten sus creencias religiosas, crean también un grupo religioso de algún tipo que acaba creando unas metas comunes. Las personas religiosas se unirán algún día y se pondrán a cooperar realmente sobre la base de la unidad de los ideales y los objetivos, en lugar de intentar hacerlo sobre la base de las opiniones psicológicas y de las creencias teológicas. Son las metas, en lugar de los credos, las que deberían unir a las personas religiosas. Puesto que la verdadera religión es un asunto de experiencia espiritual personal, es inevitable que cada persona religiosa individual posea su propia interpretación personal sobre la manera de efectuar esta experiencia espiritual. La palabra <<fe>> debería representar la relación del individuo con Dios, en lugar de ser la expresión de un credo sobre el que un grupo de mortales ha conseguido ponerse de acuerdo como actitud religiosa común. <<¿Tenéis fe? Entonces tenedla por vosotros mismos>>\footnote{\textit{¿Tenéis fe?}: Ro 14:22.}.

\par
%\textsuperscript{(1091.7)}
\textsuperscript{99:5.8} La fe sólo se ocupa de captar los valores ideales, y esto queda demostrado en la definición del Nuevo Testamento donde se afirma que la fe es la sustancia de las cosas que se esperan y la prueba de las que no se ven\footnote{\textit{La fe definida}: Heb 11:1.}.

\par
%\textsuperscript{(1091.8)}
\textsuperscript{99:5.9} El hombre primitivo hacía pocos esfuerzos por expresar en palabras sus convicciones religiosas. Su religión era danzada más que pensada. Los hombres modernos han elaborado muchas creencias y han creado muchas pruebas de la fe religiosa. Las personas religiosas futuras deberán vivir su religión, dedicarse al servicio sincero de la fraternidad de los hombres. Ya es hora de que los hombres tengan una experiencia religiosa tan personal y tan sublime, que sólo se pueda comprender y expresar mediante unos <<sentimientos que se encuentran demasiado profundos como para ser dichos con palabras>>.

\par
%\textsuperscript{(1091.9)}
\textsuperscript{99:5.10} Jesús no exigía a sus seguidores que se reunieran periódicamente para recitar un conjunto de palabras que indicaran sus creencias comunes. Sólo les ordenó que se reunieran para \textit{hacer algo} concreto ---participar en una cena común en recuerdo de su vida de donación en Urantia.

\par
%\textsuperscript{(1091.10)}
\textsuperscript{99:5.11} ¡Qué error cometen los cristianos cuando, después de presentar a Cristo como el guía espiritual ideal y supremo, se atreven a exigir a los hombres y a las mujeres conscientes de Dios que rechacen el liderazgo histórico de los hombres que conocían a Dios y que han contribuido a iluminar a su nación o a su raza particular durante las épocas pasadas!

\section*{6. La religión institucional}
\par
%\textsuperscript{(1092.1)}
\textsuperscript{99:6.1} El sectarismo es una enfermedad de la religión institucional, y el dogmatismo es una esclavitud de la naturaleza espiritual. Es mucho mejor tener una religión sin iglesia que una iglesia sin religión. El desorden religioso del siglo veinte no denota, en sí mismo y por sí mismo, una decadencia espiritual. La confusión aparece tanto antes del crecimiento como antes de la destrucción.

\par
%\textsuperscript{(1092.2)}
\textsuperscript{99:6.2} La socialización de la religión posee un objetivo real. La finalidad de las actividades religiosas colectivas consiste en representar dramáticamente la lealtad hacia la religión; magnificar los atractivos de la verdad, la belleza y la bondad; fomentar la atracción de los valores supremos; realzar el servicio de una hermandad desinteresada; glorificar los potenciales de la vida familiar; promover la educación religiosa; proporcionar consejos sabios y orientación espiritual; y estimular el culto colectivo. Todas las religiones vivientes estimulan la amistad humana, conservan la moralidad, promueven el bienestar de la vecindad y facilitan la difusión del evangelio esencial contenido en sus respectivos mensajes de salvación eterna.

\par
%\textsuperscript{(1092.3)}
\textsuperscript{99:6.3} Pero a medida que la religión se institucionaliza, su poder para hacer el bien se reduce mientras que las posibilidades de hacer el mal se multiplican enormemente. Los peligros de una religión formalista son los siguientes: fijación de las creencias y cristalización de los sentimientos; acumulación de los derechos adquiridos con un incremento de la secularización; tendencia a uniformizar y a fosilizar la verdad; la religión se desvía del servicio a Dios hacia el servicio a la iglesia; inclinación de los dirigentes a convertirse en administradores en lugar de ministros; tendencia a formar sectas y divisiones competitivas; establecimiento de una autoridad eclesiástica opresiva; creación de la actitud aristocrática de <<pueblo elegido>>; fomento de las ideas falsas y exageradas sobre la santidad; rutinización de la religión y petrificación del culto; tendencia a venerar el pasado ignorando las necesidades del presente; incapacidad para dar una interpretación moderna de la religión; enredos con las funciones de las instituciones laicas; la religión formalista crea la discriminación nefasta de las castas religiosas; se convierte en un juez intolerante de la ortodoxia; no logra conservar el interés de la juventud aventurera, y pierde gradualmente el mensaje salvador del evangelio de la salvación eterna.

\par
%\textsuperscript{(1092.4)}
\textsuperscript{99:6.4} La religión oficial frena a los hombres en sus actividades espirituales personales, en lugar de liberarlos para un servicio más elevado como constructores del reino.

\section*{7. Las aportaciones de la religión}
\par
%\textsuperscript{(1092.5)}
\textsuperscript{99:7.1} Aunque las iglesias y todos los demás grupos religiosos deberían mantenerse apartados de toda actividad laica, al mismo tiempo la religión no debe hacer nada por entorpecer o retrasar la coordinación social de las instituciones humanas. El significado de la vida debe continuar creciendo; el hombre debe seguir adelante con su reforma de la filosofía y su clarificación de la religión.

\par
%\textsuperscript{(1092.6)}
\textsuperscript{99:7.2} La ciencia política debe llevar a cabo la reconstrucción de la economía y de la industria mediante las técnicas que aprende de las ciencias sociales, y mediante la perspicacia y los móviles suministrados por la vida religiosa. En toda reconstrucción social, la religión proporciona una lealtad estabilizadora hacia un objeto trascendente, hacia una meta estable situada más allá y por encima del objetivo inmediato y temporal. En medio de la confusión de un entorno que cambia rápidamente, el hombre mortal necesita el apoyo de una amplia perspectiva cósmica.

\par
%\textsuperscript{(1093.1)}
\textsuperscript{99:7.3} La religión inspira al hombre a vivir con valentía y alegría sobre la faz de la Tierra; une la paciencia a la pasión, la perspicacia al entusiasmo, la compasión al poder y los ideales a la energía.

\par
%\textsuperscript{(1093.2)}
\textsuperscript{99:7.4} El hombre nunca puede tomar una decisión sabia sobre los asuntos temporales, ni trascender el egoísmo de los intereses personales, a menos que medite en presencia de la soberanía de Dios y tenga en cuenta las realidades de los significados divinos y de los valores espirituales.

\par
%\textsuperscript{(1093.3)}
\textsuperscript{99:7.5} La interdependencia económica y la hermandad social conducirán finalmente a la fraternidad. El hombre es un soñador por naturaleza, pero la ciencia lo está aleccionando, de manera que la religión podrá pronto activarlo con mucho menos peligro de precipitar unas reacciones fanáticas. Las necesidades económicas atan al hombre a la realidad, y la experiencia religiosa personal conduce a este mismo hombre a enfrentarse con las realidades eternas de una ciudadanía cósmica en constante expansión y progreso.

\par
%\textsuperscript{(1093.4)}
\textsuperscript{99:7.6} [Presentado por un Melquisedek de Nebadon.]


\chapter{Documento 100. La religión en la experiencia humana}
\par
%\textsuperscript{(1094.1)}
\textsuperscript{100:0.1} LA EXPERIENCIA de una vida religiosa dinámica transforma a un individuo mediocre en una personalidad con un poder idealista. La religión contribuye al progreso de todos fomentando el progreso de cada individuo, y el progreso de cada uno aumenta con el logro de todos.

\par
%\textsuperscript{(1094.2)}
\textsuperscript{100:0.2} La asociación íntima con otras personas religiosas estimula mutuamente el crecimiento espiritual. El amor suministra el terreno para el crecimiento religioso ---una atracción objetiva en lugar de una satisfacción subjetiva--- y sin embargo proporciona la satisfacción subjetiva suprema. La religión ennoblece el pesado trabajo común de la vida diaria.

\section*{1. El crecimiento religioso}
\par
%\textsuperscript{(1094.3)}
\textsuperscript{100:1.1} Aunque la religión produce el crecimiento de los significados y el realce de los valores, cuando las evaluaciones puramente personales son elevadas a unos niveles absolutos, el resultado siempre es un mal. El niño evalúa la experiencia con arreglo a su contenido de placer; la madurez es proporcional a la sustitución del placer personal por los significados superiores, e incluso por la lealtad a los conceptos más elevados de las situaciones diversificadas de la vida y de las relaciones cósmicas.

\par
%\textsuperscript{(1094.4)}
\textsuperscript{100:1.2} Algunas personas están demasiado ocupadas para crecer y se encuentran por tanto en un grave peligro de inmovilismo espiritual. Se deben tomar disposiciones para el crecimiento de los significados en las distintas edades, en las culturas sucesivas y en las etapas pasajeras de la civilización progresiva. Los principales inhibidores del crecimiento son los prejuicios y la ignorancia.

\par
%\textsuperscript{(1094.5)}
\textsuperscript{100:1.3} Concededle a cada niño que crece la oportunidad de desarrollar su propia experiencia religiosa; no le impongáis una experiencia adulta ya hecha. Recordad que el progreso, año tras año, a través de un régimen de enseñanza establecido, no significa necesariamente progreso intelectual y mucho menos crecimiento espiritual. Ampliación del vocabulario no quiere decir desarrollo del carácter. El crecimiento no está indicado realmente por los simples resultados, sino más bien por el progreso. El verdadero desarrollo educativo está indicado por el realce de los ideales, la apreciación creciente de los valores, los nuevos significados de los valores y una lealtad mayor a los valores supremos.

\par
%\textsuperscript{(1094.6)}
\textsuperscript{100:1.4} A los niños sólo les impresiona de manera permanente la lealtad de sus compañeros adultos; los preceptos, e incluso el ejemplo, no les influye de manera duradera. Las personas leales son personas que crecen, y el crecimiento es una realidad que impresiona e inspira. Vivid lealmente hoy ---creced--- y mañana será otro día. La manera más rápida que tiene un renacuajo de convertirse en una rana consiste en vivir lealmente cada instante como un renacuajo.

\par
%\textsuperscript{(1094.7)}
\textsuperscript{100:1.5} El terreno fundamental para el crecimiento religioso presupone una vida progresiva de autorrealización, la coordinación de las tendencias naturales, el ejercicio de la curiosidad y el placer de las aventuras razonables, el experimentar sentimientos de satisfacción, el funcionamiento del miedo para estimular la atención y la conciencia, la atracción de lo maravilloso, y una conciencia normal de nuestra pequeñez, la humildad. El crecimiento también está basado en el descubrimiento del yo, acompañado de autocrítica ---de conciencia--- pues la conciencia es realmente la crítica de uno mismo por nuestra propia escala de valores, los ideales personales.

\par
%\textsuperscript{(1095.1)}
\textsuperscript{100:1.6} La salud física, el temperamento heredado y el entorno social influyen notablemente sobre la experiencia religiosa. Pero estas condiciones temporales no impiden el progreso espiritual interior de un alma dedicada a hacer la voluntad del Padre que está en los cielos. En todos los mortales normales existen ciertos impulsos innatos hacia el crecimiento y la autorrealización, que funcionan si no están específicamente reprimidos. La técnica segura para fomentar esta dotación constitutiva del potencial del crecimiento espiritual consiste en mantener una actitud de consagración sincera a los valores supremos.

\par
%\textsuperscript{(1095.2)}
\textsuperscript{100:1.7} La religión no se puede dar, recibir, prestar, aprender o perder. Es una experiencia personal que crece en proporción a la búsqueda creciente de los valores finales. El crecimiento cósmico acompaña pues a la acumulación de los significados y a la constante elevación de los valores. Pero la nobleza misma siempre es un crecimiento inconsciente.

\par
%\textsuperscript{(1095.3)}
\textsuperscript{100:1.8} La manera religiosa de pensar y de actuar contribuye a la economía del crecimiento espiritual. Uno puede desarrollar unas predisposiciones religiosas para reaccionar favorablemente a los estímulos espirituales, una especie de reflejo espiritual condicionado. Los hábitos que favorecen el crecimiento religioso engloban: el cultivo de la sensibilidad a los valores divinos, el reconocimiento de la vida religiosa de los demás, la meditación reflexiva sobre los significados cósmicos, la solución de los problemas utilizando la adoración, compartir vuestra vida espiritual con vuestros semejantes, evitar el egoísmo, negarse a abusar de la misericordia divina, y vivir como si se estuviera en presencia de Dios. Los factores del crecimiento religioso pueden ser intencionales, pero el crecimiento mismo es invariablemente inconsciente.

\par
%\textsuperscript{(1095.4)}
\textsuperscript{100:1.9} Sin embargo, la naturaleza inconsciente del crecimiento religioso no significa que se trate de una actividad que se desarrolla en el ámbito supuestamente subconsciente del intelecto humano; significa más bien que las actividades creativas tienen lugar en los niveles superconscientes de la mente mortal. La experiencia de comprender la realidad de que el crecimiento religioso es inconsciente, es la única prueba positiva de la existencia funcional de la superconciencia.

\section*{2. El crecimiento espiritual}
\par
%\textsuperscript{(1095.5)}
\textsuperscript{100:2.1} El desarrollo espiritual depende, en primer lugar, del mantenimiento de una conexión espiritual viviente con las verdaderas fuerzas espirituales y, en segundo lugar, de la producción continua de los frutos espirituales, ofreciendo a vuestros semejantes la ayuda que habéis recibido de vuestros benefactores espirituales. El progreso espiritual está basado en el reconocimiento intelectual de nuestra pobreza espiritual, unido a la conciencia personal del hambre de perfección, el deseo de conocer a Dios y de parecerse a él, la intención sincera de hacer la voluntad del Padre que está en los cielos.

\par
%\textsuperscript{(1095.6)}
\textsuperscript{100:2.2} El crecimiento espiritual es, en primer lugar, un despertar a las necesidades, luego un discernimiento de los significados, y finalmente un descubrimiento de los valores. La prueba del verdadero desarrollo espiritual consiste en la manifestación de una personalidad humana motivada por el amor, activada por el servicio desinteresado y dominada por la adoración sincera de los ideales de perfección de la divinidad. Toda esta experiencia constituye la realidad de la religión, en contraste con las simples creencias teológicas.

\par
%\textsuperscript{(1095.7)}
\textsuperscript{100:2.3} La religión puede progresar hasta ese nivel de experiencia en el que se convierte en una técnica sabia e iluminada de reacción espiritual al universo. Esa religión glorificada puede ejercer su actividad en tres niveles de la personalidad humana: el intelectual, el morontial y el espiritual; en la mente, en el alma evolutiva y con el espíritu interior.

\par
%\textsuperscript{(1096.1)}
\textsuperscript{100:2.4} La espiritualidad indica inmediatamente vuestra proximidad a Dios y la medida de vuestra utilidad para vuestros semejantes. La espiritualidad realza la aptitud para descubrir la belleza en las cosas, para reconocer la verdad en los significados y para descubrir la bondad en los valores. El desarrollo espiritual está determinado por la capacidad para llevarlo a cabo y es directamente proporcional a la eliminación de los elementos egoístas del amor.

\par
%\textsuperscript{(1096.2)}
\textsuperscript{100:2.5} El verdadero estado espiritual representa la medida en que se ha alcanzado la Deidad, la armonización con el Ajustador. Conseguir la finalidad de la espiritualidad equivale a alcanzar el máximo de realidad, el máximo de semejanza con Dios. La vida eterna es la búsqueda interminable de los valores infinitos.

\par
%\textsuperscript{(1096.3)}
\textsuperscript{100:2.6} La meta de la autorrealización humana debería ser espiritual, no material. Las únicas realidades por las que vale la pena luchar son divinas, espirituales y eternas. El hombre mortal tiene derecho al disfrute de los placeres físicos y a la satisfacción de los afectos humanos; se beneficia de la lealtad a las asociaciones humanas y a las instituciones temporales; pero éstos no son los cimientos eternos sobre los que ha de construir la personalidad inmortal que deberá trascender el espacio, vencer el tiempo y alcanzar el destino eterno de la perfección divina y del servicio como finalitario.

\par
%\textsuperscript{(1096.4)}
\textsuperscript{100:2.7} Jesús describió la profunda seguridad del mortal que conoce a Dios cuando dijo: <<Para un creyente en el reino que conoce a Dios, ¿que importa si todas las cosas terrenales se derrumban?>>\footnote{\textit{Para un creyente que conoce a Dios}: Mt 6:25-34; 10:28; Lc 12:4; Heb 13:6.} Las seguridades temporales son vulnerables, pero las certezas espirituales son inquebrantables. Cuando las mareas de la adversidad, el egoísmo, la crueldad, el odio, la maldad y los celos humanos sacuden el alma de los mortales, podéis tener la seguridad de que existe un bastión interior, la ciudadela del espíritu, que es absolutamente inatacable; al menos esto es cierto para todo ser humano que ha confiado la custodia de su alma al espíritu interior del Dios eterno.

\par
%\textsuperscript{(1096.5)}
\textsuperscript{100:2.8} Después de este logro espiritual, conseguido por medio de un crecimiento gradual o de una crisis específica, se produce una nueva orientación de la personalidad así como el desarrollo de una nueva escala de valores. Estas personas nacidas del espíritu tienen tales motivaciones nuevas en la vida que pueden mantenerse tranquilamente al margen mientras perecen sus ambiciones más queridas y se derrumban sus esperanzas más profundas; saben positivamente que estas catástrofes no son más que cataclismos rectificadores que destruyen nuestras creaciones temporales, preludiando la construcción de las realidades más nobles y duraderas de un nivel nuevo y más sublime de consecución universal.

\section*{3. Los conceptos de valor supremo}
\par
%\textsuperscript{(1096.6)}
\textsuperscript{100:3.1} La religión no es una técnica para conseguir una paz mental estática y feliz; es un impulso destinado a organizar el alma para un servicio dinámico. Es el reclutamiento de la totalidad del yo para el servicio leal de amar a Dios y servir a los hombres. La religión paga cualquier precio que sea necesario para conseguir la meta suprema, la recompensa eterna. La lealtad religiosa conlleva una consagración tan completa que es magníficamente sublime. Y esta lealtad es socialmente eficaz y espiritualmente progresiva.

\par
%\textsuperscript{(1096.7)}
\textsuperscript{100:3.2} Para la persona religiosa, la palabra Dios se convierte en un símbolo que significa el acercamiento a la realidad suprema y el reconocimiento del valor divino. Las preferencias y las aversiones humanas no son las que determinan el bien y el mal; los valores morales no tienen su origen en la satisfacción de los deseos o en las frustraciones emocionales.

\par
%\textsuperscript{(1096.8)}
\textsuperscript{100:3.3} Cuando meditéis sobre los valores, debéis distinguir entre lo que \textit{es} un valor y lo que \textit{tiene} un valor. Debéis reconocer la relación que existe entre las actividades agradables y su sensata integración así como su creciente realización en los niveles progresivamente más elevados de la experiencia humana.

\par
%\textsuperscript{(1097.1)}
\textsuperscript{100:3.4} El significado es algo que la experiencia añade al valor; es la conciencia apreciativa de los valores. Un placer aislado y puramente egoísta puede connotar una verdadera desvalorización de los significados, un disfrute sin sentido que linda con el mal relativo. Los valores son experienciales cuando las realidades son significativas y están mentalmente asociadas, cuando tales relaciones son reconocidas y apreciadas por la mente.

\par
%\textsuperscript{(1097.2)}
\textsuperscript{100:3.5} Los valores nunca pueden ser estáticos; la realidad significa cambio, crecimiento. El cambio sin crecimiento, sin expansión de los significados y sin exaltación de los valores, no tiene ningún valor ---es un mal potencial. Cuanto mayor sea la calidad de la adaptación cósmica, más significado posee una experiencia cualquiera. Los valores no son ilusiones conceptuales; son reales, pero siempre dependen del hecho de las relaciones. Los valores son siempre tanto actuales como potenciales ---no representan lo que era, sino lo que es y lo que será.

\par
%\textsuperscript{(1097.3)}
\textsuperscript{100:3.6} La asociación de los actuales con los potenciales equivale al crecimiento, a la realización experiencial de los valores. Pero el crecimiento no es el simple progreso. El progreso siempre es significativo, pero no tiene relativamente ningún valor en ausencia de crecimiento. El valor supremo de la vida humana consiste en el crecimiento de los valores, en el progreso en los significados y en la realización de la correlación cósmica entre estas dos experiencias. Una experiencia así equivale a tener conciencia de Dios. Un mortal así, aunque no es sobrenatural, se está volviendo realmente sobrehumano; un alma inmortal está evolucionando.

\par
%\textsuperscript{(1097.4)}
\textsuperscript{100:3.7} El hombre no puede provocar el crecimiento, pero puede suministrar las condiciones favorables. El crecimiento siempre es inconsciente, ya sea físico, intelectual o espiritual. El amor crece así; no se puede crear, ni fabricar ni comprar; debe crecer. La evolución es una técnica cósmica de crecimiento. El crecimiento social no se puede conseguir por medio de la legislación, y el crecimiento moral no se obtiene mediante una administración mejor. El hombre puede fabricar una máquina, pero su valor real debe provenir de la cultura humana y de la apreciación personal. La única contribución que el hombre puede hacer al crecimiento es la movilización de todos los poderes de su personalidad ---su fe viviente.

\section*{4. Problemas de crecimiento}
\par
%\textsuperscript{(1097.5)}
\textsuperscript{100:4.1} Una vida religiosa es una vida dedicada, y una vida dedicada es una vida creativa, original y espontánea. Aquellos conflictos que ponen en marcha la elección de unas maneras de reaccionar nuevas y mejores, en lugar de las antiguas formas inferiores de reaccionar, son los que hacen surgir las nuevas perspicacias religiosas. Los nuevos significados sólo emergen en medio de los conflictos; y un conflicto sólo persiste cuando nos negamos a adoptar los valores más elevados implicados en los significados superiores.

\par
%\textsuperscript{(1097.6)}
\textsuperscript{100:4.2} Las perplejidades religiosas son inevitables; no puede existir ningún crecimiento sin conflicto psíquico y sin agitación espiritual. La organización de un modelo filosófico de vida ocasiona una conmoción considerable en el terreno filosófico de la mente. La lealtad hacia lo grande, lo bueno, lo verdadero y lo noble no se ejerce sin lucha. La clarificación de la visión espiritual y el realce de la perspicacia cósmica van acompañados de esfuerzo. Y el intelecto humano protesta cuando se le quita el sustento de las energías no espirituales de la existencia temporal. La mente indolente animal se rebela ante el esfuerzo que exige la lucha para resolver los problemas cósmicos.

\par
%\textsuperscript{(1097.7)}
\textsuperscript{100:4.3} Pero el gran problema de la vida religiosa consiste en la tarea de unificar los poderes del alma, inherentes a la personalidad, mediante el dominio del AMOR. La salud, la eficacia mental y la felicidad resultan de la unificación de los sistemas físicos, de los sistemas mentales y de los sistemas espirituales. El hombre entiende mucho de salud y de juicio, pero ha comprendido realmente muy pocas cosas sobre la felicidad. La felicidad más grande está indisolublemente enlazada con el progreso espiritual. El crecimiento espiritual produce una alegría duradera, una paz que sobrepasa toda comprensión.

\par
%\textsuperscript{(1098.1)}
\textsuperscript{100:4.4} En la vida física, los sentidos comunican la existencia de las cosas; la mente descubre la realidad de los significados; pero la experiencia espiritual revela al individuo los verdaderos valores de la vida. Estos niveles elevados de vida humana se alcanzan mediante el amor supremo a Dios y el amor desinteresado a los hombres. Si amáis a vuestros semejantes, es porque habéis descubierto sus valores. Jesús amaba tanto a los hombres porque les atribuía un alto valor. Podéis descubrir mejor los valores de vuestros compañeros descubriendo sus motivaciones. Si alguien os irrita, os produce sentimientos de rencor, deberíais tratar de discernir con simpatía su punto de vista, las razones de su comportamiento censurable. En cuanto comprendéis a vuestro prójimo, os volvéis tolerantes, y esta tolerancia se convierte en amistad y madura en amor.

\par
%\textsuperscript{(1098.2)}
\textsuperscript{100:4.5} Tratad de ver con los ojos de la imaginación el retrato de uno de vuestros antepasados primitivos de los tiempos de las cavernas ---un hombre bajo, contrahecho, sucio, corpulento y gruñón, que permanece con las piernas abiertas, levantando un garrote, respirando odio y animosidad, mientras mira ferozmente delante de él. Esta imagen difícilmente representa la dignidad divina del hombre. Pero ampliemos el cuadro. Delante de este humano animado se encuentra agazapado un tigre con dientes de sable. Detrás del hombre hay una mujer y dos niños. Reconocéis inmediatamente que esta imagen representa los principios de muchas cosas hermosas y nobles de la raza humana, pero el hombre es el mismo en los dos cuadros. Sólo que en el segundo esbozo contáis con la ayuda de un horizonte más amplio. En él discernís la motivación de este mortal evolutivo. Su actitud se vuelve digna de elogio porque lo comprendéis. Si tan sólo pudierais sondear los móviles de vuestros compañeros, cuánto mejor los comprenderíais. Si tan sólo pudierais conocer a vuestros semejantes, terminaríais por enamoraros de ellos.

\par
%\textsuperscript{(1098.3)}
\textsuperscript{100:4.6} No podéis amar realmente a vuestros compañeros con un simple acto de voluntad. El amor sólo nace de una comprensión completa de los móviles y sentimientos de vuestros semejantes. Amar hoy a todos los hombres no es tan importante como aprender cada día a amar a un ser humano más. Si cada día o cada semana lográis comprender a uno más de vuestros compañeros, y si éste es el límite de vuestra capacidad, entonces estáis sin duda haciendo sociable y espiritualizando realmente vuestra personalidad. El amor es contagioso, y cuando la devoción humana es inteligente y sabia, el amor es más contagioso que el odio. Pero sólo el amor auténtico y desinteresado es verdaderamente contagioso. Si tan sólo cada mortal pudiera convertirse en un foco de afecto dinámico, este virus benigno del amor pronto impregnaría la corriente de emoción sentimental de la humanidad hasta tal punto que toda la civilización quedaría envuelta en el amor, y ésta sería la realización de la fraternidad de los hombres.

\section*{5. La conversión y el misticismo}
\par
%\textsuperscript{(1098.4)}
\textsuperscript{100:5.1} El mundo está lleno de almas perdidas, no perdidas en el sentido teológico, sino perdidas en el sentido de la dirección, vagando confusas entre las doctrinas en ismo y los cultos de una era filosófica frustrada. Muy pocas de ellas han aprendido a instalar una filosofía de vida en el lugar de la autoridad religiosa. (Los símbolos de la religión socializada no deben ser menospreciados como canales de crecimiento, aunque el lecho del río no sea el río mismo.)

\par
%\textsuperscript{(1098.5)}
\textsuperscript{100:5.2} La evolución del crecimiento religioso conduce, por medio del conflicto, del estancamiento a la coordinación, de la inseguridad a la fe convencida, de la confusión de la conciencia cósmica a la unificación de la personalidad, del objetivo temporal al objetivo eterno, de la esclavitud del miedo a la libertad de la filiación divina.

\par
%\textsuperscript{(1099.1)}
\textsuperscript{100:5.3} Debemos indicar claramente que las declaraciones de lealtad a los ideales supremos ---el darse cuenta psíquica, emocional y espiritualmente de tener conciencia de Dios--- pueden ser el resultado de un crecimiento natural y gradual, o a veces se pueden experimentar en ciertas coyunturas tales como una crisis. El apóstol Pablo experimentó precisamente una conversión repentina y espectacular de este tipo aquel día memorable en el camino de Damasco\footnote{\textit{La conversión de Pablo}: Hch 9:1-9,20.}. Siddharta Gautama tuvo una experiencia similar la noche en que se sentó a solas para tratar de penetrar en el misterio de la verdad final. Otras muchas personas han tenido experiencias similares, y muchos creyentes sinceros han progresado en el espíritu sin conversión repentina.

\par
%\textsuperscript{(1099.2)}
\textsuperscript{100:5.4} La mayoría de los fenómenos espectaculares relacionados con las conversiones llamadas religiosas son de naturaleza totalmente psicológica, pero de vez en cuando se producen experiencias que tienen también un origen espiritual. Cuando la movilización mental es absolutamente total en un nivel cualquiera de la expansión psíquica hacia la consecución espiritual, cuando las motivaciones humanas de lealtad a la idea divina son perfectas, entonces se produce con mucha frecuencia un descenso repentino del espíritu interior para sincronizarse con el objetivo concentrado y consagrado de la mente superconsciente del mortal creyente. Estas experiencias de unificación de los fenómenos intelectuales y espirituales son las que constituyen la conversión, la cual consiste en unos factores que sobrepasan las implicaciones puramente psicológicas.

\par
%\textsuperscript{(1099.3)}
\textsuperscript{100:5.5} Pero la emoción sola es una conversión falsa; hace falta tanto la fe como el sentimiento. En el grado en que esta movilización psíquica sea parcial, y en la medida en que estos móviles de la lealtad humana sean incompletos, la experiencia de la conversión será una realidad intelectual, emocional y espiritual mixta.

\par
%\textsuperscript{(1099.4)}
\textsuperscript{100:5.6} Si uno está dispuesto a admitir, como hipótesis práctica de trabajo, la existencia de una mente subconsciente teórica en la vida intelectual por lo demás unificada, entonces, para ser coherente, uno debería dar por sentado la existencia de un nivel superconsciente similar y correspondiente de actividad intelectual ascendente, la zona de contacto inmediato con la entidad espiritual interior, el Ajustador del Pensamiento. El gran peligro de todas estas especulaciones psíquicas consiste en que las visiones y otras experiencias llamadas místicas, así como los sueños extraordinarios, pueden ser considerados como comunicaciones divinas a la mente humana. En los tiempos pasados, los seres divinos se han revelado a ciertas personas que conocían a Dios, no a causa de sus trances místicos o de sus visiones enfermizas, sino a pesar de todos esos fenómenos.

\par
%\textsuperscript{(1099.5)}
\textsuperscript{100:5.7} En contraste con la búsqueda de la conversión, la mejor manera de acercarse a las zonas morontiales de posible contacto con el Ajustador del Pensamiento debería ser a través de la fe viviente y de la adoración sincera, de una oración incondicional y desinteresada. En conjunto, una parte demasiado grande de los recuerdos que afluyen desde los niveles inconscientes de la mente humana ha sido confundida con revelaciones divinas y directrices espirituales.

\par
%\textsuperscript{(1099.6)}
\textsuperscript{100:5.8} La práctica habitual del ensueño religioso va acompañada de un gran peligro; el misticismo puede convertirse en una técnica para eludir la realidad, aunque a veces ha sido un medio de comunión espiritual auténtica. Los cortos períodos de retiro del escenario activo de la vida pueden no ser gravemente peligrosos, pero el aislamiento prolongado de la personalidad es sumamente indeseable. El estado de conciencia visionaria semejante al trance no debería cultivarse en ninguna circunstancia como experiencia religiosa.

\par
%\textsuperscript{(1099.7)}
\textsuperscript{100:5.9} La característica del estado místico consiste en una conciencia difusa, con islotes intensos de atención focalizada que operan en un intelecto relativamente pasivo. Todo esto hace que la conciencia gravite hacia el subconsciente, en lugar de dirigirse hacia la zona del contacto espiritual, el superconsciente. Muchos místicos han llevado su disociación mental hasta el nivel de las manifestaciones mentales anormales.

\par
%\textsuperscript{(1100.1)}
\textsuperscript{100:5.10} La actitud más sana de meditación espiritual se halla en la adoración reflexiva y en la oración de acción de gracias. La comunión directa con el Ajustador del Pensamiento, tal como sucedió en los últimos años de la vida de Jesús en la carne, no debería confundirse con estas experiencias llamadas místicas. Los factores que contribuyen al inicio de la comunión mística indican el peligro de estos estados psíquicos. El estado místico es favorecido por circunstancias tales como el cansancio físico, el ayuno, la disociación psíquica, las experiencias estéticas profundas, los impulsos sexuales intensos, el miedo, la ansiedad, la furia y el baile frenético. Muchos elementos que aparecen como resultado de esta preparación preliminar tienen su origen en la mente subconsciente.

\par
%\textsuperscript{(1100.2)}
\textsuperscript{100:5.11} Por muy favorables que pudieran ser las condiciones para los fenómenos místicos, se debería comprender claramente que Jesús de Nazaret no recurrió nunca a estos métodos para comunicarse con el Padre Paradisiaco. Jesús no tenía alucinaciones subconscientes ni ilusiones superconscientes.

\section*{6. Los signos de una vida religiosa}
\par
%\textsuperscript{(1100.3)}
\textsuperscript{100:6.1} Las religiones evolutivas y las religiones reveladas pueden diferir notablemente en cuanto a sus métodos, pero sus móviles tienen una gran similitud. La religión no es una función específica de la vida; es más bien una manera de vivir. La verdadera religión es una devoción incondicional hacia una realidad que la persona religiosa considera que tiene un valor supremo para él y para toda la humanidad. Las características sobresalientes de todas las religiones son: una lealtad incondicional y una devoción sincera hacia los valores supremos. Esta devoción religiosa hacia los valores supremos se manifiesta en la relación de una madre supuestamente irreligiosa con su hijo, y en la lealtad ferviente de las personas no religiosas hacia la causa que han abrazado.

\par
%\textsuperscript{(1100.4)}
\textsuperscript{100:6.2} El valor supremo aceptado por la persona religiosa puede ser degradante o incluso falso, pero no obstante es religioso. Una religión es auténtica en la medida exacta en que el valor que considera supremo es verdaderamente una realidad cósmica con un valor espiritual auténtico.

\par
%\textsuperscript{(1100.5)}
\textsuperscript{100:6.3} Los signos de la reacción humana a los impulsos religiosos abarcan las cualidades de la nobleza y la grandeza. La persona religiosa sincera tiene conciencia de ser ciudadana del universo y se da cuenta de que se pone en contacto con unas fuentes de poder sobrehumano. Se siente emocionada y estimulada ante la seguridad de pertenecer a una hermandad superior y ennoblecida de hijos de Dios. La conciencia de la propia valía se ha acrecentado mediante el estímulo de la búsqueda de los objetivos universales más elevados ---las metas supremas.

\par
%\textsuperscript{(1100.6)}
\textsuperscript{100:6.4} El yo se ha abandonado al impulso misterioso de una motivación que lo abarca todo, que impone una autodisciplina más intensa, disminuye los conflictos emocionales y hace que la vida mortal sea digna de ser vivida. El reconocimiento pesimista de las limitaciones humanas se transforma en una conciencia natural de los defectos humanos, unida a la determinación moral y a la aspiración espiritual de alcanzar las metas más elevadas del universo y del superuniverso. Esta intensa lucha por alcanzar los ideales supermortales está siempre caracterizada por un aumento de la paciencia, la indulgencia, la fortaleza y la tolerancia.

\par
%\textsuperscript{(1100.7)}
\textsuperscript{100:6.5} Pero la verdadera religión es un amor viviente, una vida de servicio. El desapego de la persona religiosa hacia muchas cosas que son puramente temporales y banales no conduce nunca al aislamiento social, y no debería destruir el sentido del humor. La auténtica religión no le quita nada a la existencia humana, sino que añade de hecho unos nuevos significados al conjunto de la vida; genera nuevos tipos de entusiasmo, fervor y valentía. Puede incluso engendrar el espíritu de cruzada, que es más que peligroso si no está controlado por la perspicacia espiritual y la consagración leal a las obligaciones sociales comunes de las lealtades humanas.

\par
%\textsuperscript{(1101.1)}
\textsuperscript{100:6.6} Una de las características más asombrosas de la vida religiosa es esa paz dinámica y sublime, esa paz que sobrepasa toda comprensión humana, esa serenidad cósmica que revela la ausencia de toda duda y de toda agitación\footnote{\textit{Paz perfecta}: Is 26:3; Lc 1:14; 2:14; Jn 14:27; 16:33; Flp 4:7.}. Esos niveles de estabilidad espiritual son inmunes a la decepción. Tales personas religiosas se parecen al apóstol Pablo, que decía: <<Estoy convencido de que ni la muerte, ni la vida, ni los ángeles, ni los principados, ni los poderes, ni las cosas presentes, ni las cosas por venir, ni lo alto, ni lo profundo, ni ninguna otra cosa podrá separarnos del amor de Dios>>\footnote{\textit{Estoy convencido de que ni la muerte}: Ro 8:38-39.}.

\par
%\textsuperscript{(1101.2)}
\textsuperscript{100:6.7} Existe un sentimiento de seguridad, unido al reconocimiento de una gloria triunfante, que reside en la conciencia de la persona religiosa que ha captado la realidad del Supremo y que persigue la meta del Último.

\par
%\textsuperscript{(1101.3)}
\textsuperscript{100:6.8} Incluso la religión evolutiva posee esta misma lealtad y grandeza porque es una experiencia auténtica. Pero la religión revelada es \textit{excelente} a la vez que auténtica. Las nuevas lealtades debidas a una visión espiritual más amplia crean nuevos niveles de amor y de devoción, de servicio y de hermandad; y toda esta perspectiva social realzada produce una mayor conciencia de la Paternidad de Dios y de la fraternidad de los hombres.

\par
%\textsuperscript{(1101.4)}
\textsuperscript{100:6.9} La diferencia característica entre la religión evolutiva y la religión revelada consiste en una nueva calidad de sabiduría divina que se añade a la sabiduría humana puramente experiencial. Pero la experiencia en y con las religiones humanas es la que desarrolla la capacidad para recibir posteriormente los dones crecientes de la sabiduría divina y de la perspicacia cósmica.

\section*{7. El apogeo de la vida religiosa}
\par
%\textsuperscript{(1101.5)}
\textsuperscript{100:7.1} Aunque el mortal medio de Urantia no puede esperar alcanzar la elevada perfección de carácter que adquirió Jesús de Nazaret mientras permaneció en la carne, a todo creyente mortal le es totalmente posible desarrollar una personalidad fuerte y unificada según el modelo perfeccionado de la personalidad de Jesús. La característica incomparable de la personalidad del Maestro no era tanto su perfección como su simetría, su exquisita unificación equilibrada. La presentación más eficaz de Jesús consiste en seguir el ejemplo de aquel que dijo, mientras hacía un gesto hacia el Maestro que permanecía de pie delante de sus acusadores: <<¡He aquí al hombre!>>\footnote{\textit{¡He aquí al hombre!}: Jn 19:5.}

\par
%\textsuperscript{(1101.6)}
\textsuperscript{100:7.2} La amabilidad constante de Jesús conmovía el corazón de los hombres, pero la firmeza de su fuerza de carácter asombraba a sus seguidores. Era realmente sincero; no había nada de hipócrita en él. Estaba exento de simulación; era siempre tan refrescantemente auténtico. Nunca se rebajó a fingir, y nunca recurrió a la impostura. Vivía la verdad tal como la enseñaba. Él era la verdad\footnote{\textit{Él era y es la verdad}: Jn 1:17; 14:6.}. Estaba obligado a proclamar la verdad salvadora a su generación, aunque esta sinceridad a veces causara sufrimiento. Era incondicionalmente leal a toda verdad.

\par
%\textsuperscript{(1101.7)}
\textsuperscript{100:7.3} Pero el Maestro era tan razonable, tan accesible. Era tan práctico en todo su ministerio, mientras que todos sus planes estaban caracterizados por un sentido común santificado. Estaba libre de toda tendencia extravagante, errática y excéntrica. Nunca era caprichoso, antojadizo o histérico. En toda su enseñanza y en todas las cosas que hacía siempre había una discriminación exquisita, asociada a un extraordinario sentido de la corrección.

\par
%\textsuperscript{(1102.1)}
\textsuperscript{100:7.4} El Hijo del Hombre siempre fue una personalidad bien equilibrada. Incluso sus enemigos le tenían un respeto saludable; temían incluso su presencia. Jesús no tenía miedo. Estaba sobrecargado de entusiasmo divino, pero nunca se volvió fanático. Era emocionalmente activo, pero nunca caprichoso. Era imaginativo pero siempre práctico. Se enfrentaba con franqueza a las realidades de la vida, pero nunca era insulso ni prosaico. Era valiente pero nunca temerario; prudente, pero nunca cobarde. Era compasivo pero no sensiblero; excepcional pero no excéntrico. Era piadoso pero no beato. Estaba tan bien equilibrado porque estaba perfectamente unificado.

\par
%\textsuperscript{(1102.2)}
\textsuperscript{100:7.5} Jesús no reprimía su originalidad. No estaba atado a la tradición ni obstaculizado por la esclavitud a los convencionalismos estrechos. Hablaba con una confianza indudable y enseñaba con una autoridad absoluta. Pero su magnífica originalidad no le inducía a pasar por alto las perlas de verdad contenidas en las enseñanzas de sus predecesores o de sus contemporáneos. Y la más original de sus enseñanzas fue el énfasis que puso en el amor y la misericordia, en lugar del miedo y el sacrificio.

\par
%\textsuperscript{(1102.3)}
\textsuperscript{100:7.6} Jesús tenía un punto de vista muy amplio. Exhortaba a sus seguidores a que predicaran el evangelio a todos los pueblos. Estaba exento de toda estrechez de miras. Su corazón compasivo abarcaba a toda la humanidad e incluso a un universo. Su invitación siempre era: <<Quienquiera que lo desee, puede venir>>\footnote{\textit{Quienquiera que lo desee, puede venir}: Sal 50:15; Jl 2:32; Zac 13:9; Mt 7:24; 10:32-33; 12:50; 16:24-25; Mc 3:35; 8:34-35; Lc 6:47; 9:23-24; 12:8; Jn 3:15-16; 4:13-14; 11:25-26; 12:46; Hch 2:21; 10:43; 13:26; Ro 9:33; 10:13; 1 Jn 2:23; 4:15;  5:1; Ap 22:17b.}.

\par
%\textsuperscript{(1102.4)}
\textsuperscript{100:7.7} De Jesús se ha dicho en verdad: <<Confiaba en Dios>>\footnote{\textit{Confiaba en Dios}: Mt 27:43.}. Como hombre entre los hombres, confiaba de la manera más sublime en el Padre que está en los cielos. Confiaba en su Padre como un niño pequeño confía en su padre terrenal. Su fe era perfecta pero nunca presuntuosa. Por muy cruel o indiferente que la naturaleza pareciera ser para el bienestar de los hombres en la Tierra, Jesús no titubeó nunca en su fe. Era inmune a las decepciones e insensible a las persecuciones. Los fracasos aparentes no le afectaban.

\par
%\textsuperscript{(1102.5)}
\textsuperscript{100:7.8} Amaba a los hombres como hermanos, reconociendo al mismo tiempo cuánto diferían en dones innatos y en cualidades adquiridas. <<Iba de un sitio para otro haciendo el bien>>\footnote{\textit{Iba de un sitio para otro haciendo el bien}: Hch 10:38.}.

\par
%\textsuperscript{(1102.6)}
\textsuperscript{100:7.9} Jesús era una persona excepcionalmente alegre, pero no era un optimista ciego e irracional. Sus palabras constantes de exhortación eran: <<Tened buen ánimo>>\footnote{\textit{Tened buen ánimo}: Mt 9:2; 14:27; Mc 6:50; Jn 16:33; Hch 23:11.}. Podía mantener esta actitud convencida debido a su confianza inquebrantable en Dios y a su fe férrea en los hombres. Siempre manifestaba una consideración conmovedora a todos los hombres porque los amaba y creía en ellos. Pero siempre se mantuvo fiel a sus convicciones y magníficamente firme en su consagración a hacer la voluntad de su Padre.

\par
%\textsuperscript{(1102.7)}
\textsuperscript{100:7.10} El Maestro siempre fue generoso. Nunca se cansó de decir: <<Es más bienaventurado dar que recibir>>\footnote{\textit{Es más bienaventurado dar que recibir}: Hch 20:35.}. Y también: <<Habéis recibido gratuitamente, dad gratuitamente>>\footnote{\textit{Habéis recibido gratuitamente, dad gratuitamente}: Mt 10:8.}. Y sin embargo, a pesar de su generosidad ilimitada, nunca fue derrochador ni extravagante. Enseñó que tenéis que creer para recibir la salvación. <<Pues todo aquel que busca, recibirá>>\footnote{\textit{Pues todo aquel que busca, recibirá}: Mt 7:8; Lc 11:10.}.

\par
%\textsuperscript{(1102.8)}
\textsuperscript{100:7.11} Era sincero, pero siempre amable. Decía: <<Si no fuera así, os lo habría dicho>>\footnote{\textit{Si no fuera así, os lo habría dicho}: Jn 14:2.}. Era franco, pero siempre amistoso. Expresaba claramente su amor por los pecadores y su odio por el pecado. Pero en toda esta franqueza sorprendente, era infaliblemente \textit{equitativo}.

\par
%\textsuperscript{(1102.9)}
\textsuperscript{100:7.12} Jesús siempre estaba alegre, a pesar de que a veces bebió profundamente en la copa de las tristezas humanas. Se enfrentó con intrepidez a las realidades de la existencia, y sin embargo estaba lleno de entusiasmo por el evangelio del reino\footnote{\textit{Evangelio del reino}: Mt 4:23; 9:35; 24:14; Mc 1:14-15.}. Pero controlaba su entusiasmo; éste nunca lo dominó a él. Estaba consagrado sin reservas a <<los asuntos del Padre>>\footnote{\textit{Los asuntos del Padre}: Lc 2:49.}. Este entusiasmo divino condujo a sus hermanos no espirituales a pensar que estaba fuera de sí, pero el universo que lo contemplaba lo valoraba como el modelo de la cordura y el arquetipo de la devoción mortal suprema a los criterios elevados de la vida espiritual. Su entusiasmo controlado era contagioso; sus compañeros se veían obligados a compartir su divino optimismo.

\par
%\textsuperscript{(1103.1)}
\textsuperscript{100:7.13} Este hombre de Galilea no era un hombre de tristezas\footnote{\textit{No era un hombre de tristezas}: Is 53:3.}; era un alma de alegría. Siempre estaba diciendo: <<Regocijaos y estad llenos de alegría>>\footnote{\textit{Regocijaos y estad llenos de alegría}: Mt 5:12.}. Pero cuando el deber lo exigió, estuvo dispuesto a atravesar valientemente el <<valle de la sombra de la muerte>>\footnote{\textit{Valle de la sombra de la muerte}: Sal 23:4.}. Era alegre pero al mismo tiempo humilde.

\par
%\textsuperscript{(1103.2)}
\textsuperscript{100:7.14} Su valor sólo era igualado por su paciencia. Cuando le presionaban para que actuara prematuramente, se limitaba a responder: <<Mi hora aún no ha llegado>>\footnote{\textit{Mi hora aún no ha llegado}: Jn 2:4.}. Nunca tenía prisa; su serenidad era sublime. Pero a menudo se indignaba contra el mal, no toleraba el pecado. Con frecuencia se sintió impulsado a oponerse enérgicamente a aquello que iba en contra del bienestar de sus hijos en la Tierra. Pero su indignación contra el pecado nunca le condujo a enojarse con los pecadores.

\par
%\textsuperscript{(1103.3)}
\textsuperscript{100:7.15} Su valor era magnífico, pero nunca fue temerario. Su lema era: <<No temáis>>\footnote{\textit{No temáis}: Mt 10:28,31; 14:27; 17:7; 28:5,10; Mc 5:36; 6:50; Lc 5:10; 8:50; 12:4-7,32; Jn 6:20; 14:27.}. Su valentía era altiva y su coraje a menudo heroico. Pero su coraje estaba unido a la discreción y controlado por la razón. Era un coraje nacido de la fe, no la temeridad de una presunción ciega. Era realmente valiente pero nunca atrevido.

\par
%\textsuperscript{(1103.4)}
\textsuperscript{100:7.16} El Maestro era un modelo de veneración. Su oración, incluso en su juventud, empezaba por: <<Padre nuestro que estás en los cielos, santificado sea tu nombre>>\footnote{\textit{Padre nuestro que estás en los cielos}: Mt 5:16,45,48; 6:1,9,14,26,32; 7:11,21; 10:32-33; 11:25; 12:50; 15:13; 16:17; 18:10,14,19,35; 23:9; Mc 11:25-26; Lc 11:2.13.}. Respetaba incluso el culto erróneo de sus semejantes. Pero esto no le impedía luchar contra las tradiciones religiosas o atacar los errores de las creencias humanas. Veneraba la verdadera santidad, y sin embargo podía apelar con razón a sus semejantes, diciendo: <<¿Quien de vosotros me declarará culpable de pecado?>>\footnote{\textit{¿Quien me declarará culpable de pecado?}: Jn 8:46.}.

\par
%\textsuperscript{(1103.5)}
\textsuperscript{100:7.17} Jesús era grande porque era bueno, y sin embargo fraternizaba con los niños pequeños. Era amable y modesto en su vida personal, y sin embargo era el hombre perfeccionado de un universo. Sus compañeros le llamaban Maestro por propia iniciativa.

\par
%\textsuperscript{(1103.6)}
\textsuperscript{100:7.18} Jesús era la personalidad humana perfectamente unificada. Y hoy, como en Galilea, continúa unificando la experiencia mortal y coordinando los esfuerzos humanos. Unifica la vida, ennoblece el carácter y simplifica la experiencia. Entra en la mente humana para elevarla, transformarla y transfigurarla. Es literalmente cierto que: <<Si un hombre tiene a Cristo Jesús dentro de él, es una criatura nueva; las cosas viejas van desapareciendo; y mirad, todas las cosas se vuelven nuevas>>\footnote{\textit{Si un hombre tiene a Cristo Jesús dentro de él}: 2 Co 5:17.}.

\par
%\textsuperscript{(1103.7)}
\textsuperscript{100:7.19} [Presentado por un Melquisedek de Nebadon.]


\chapter{Documento 101. La naturaleza real de la religión}
\par
%\textsuperscript{(1104.1)}
\textsuperscript{101:0.1} LA RELIGIÓN, como experiencia humana, se extiende desde la esclavitud del miedo primitivo de los salvajes en evolución hasta la libertad sublime y admirable de la fe de los mortales civilizados que son magníficamente conscientes de su filiación con el Dios eterno.

\par
%\textsuperscript{(1104.2)}
\textsuperscript{101:0.2} La religión es la antecesora de la ética y de la moral avanzadas de la evolución social progresiva. Pero la religión, como tal, no es simplemente un movimiento moral, aunque sus manifestaciones exteriores y sociales estén poderosamente influidas por el impulso ético y moral de la sociedad humana. La religión es siempre la inspiradora de la naturaleza evolutiva del hombre, pero no es el secreto de dicha evolución.

\par
%\textsuperscript{(1104.3)}
\textsuperscript{101:0.3} La religión, la fe-convencimiento de la personalidad, siempre puede triunfar sobre la lógica superficialmente contradictoria de la desesperación, una lógica nacida en la mente material no creyente. Existe realmente una voz interior verdadera y auténtica, esa <<luz verdadera que ilumina a todo hombre que viene al mundo>>\footnote{\textit{Luz verdadera que ilumina a todo hombre}: Jn 1:9.}. Y esta guía espiritual es distinta de las incitaciones éticas de la conciencia humana. La sensación de la seguridad religiosa es más que un sentimiento emotivo. La seguridad de la religión trasciende la razón de la mente e incluso la lógica de la filosofía. La religión \textit{es} fe, confianza y seguridad.

\section*{1. La verdadera religión}
\par
%\textsuperscript{(1104.4)}
\textsuperscript{101:1.1} La verdadera religión no es un sistema de creencias filosóficas que se pueda entender y justificar mediante pruebas naturales, y tampoco es una experiencia fantástica y mística de indescriptibles sentimientos de éxtasis que sólo puedan disfrutar los adeptos románticos del misticismo. La religión no es el producto de la razón, pero vista desde dentro, es totalmente razonable. La religión no proviene de la lógica de la filosofía humana, pero como experiencia de los mortales es totalmente lógica. La religión es la experimentación de la divinidad en la conciencia de un ser moral de origen evolutivo; representa una experiencia auténtica con las realidades eternas en el tiempo, la realización de las satisfacciones espirituales mientras se vive todavía en la carne.

\par
%\textsuperscript{(1104.5)}
\textsuperscript{101:1.2} El Ajustador del Pensamiento no posee ningún mecanismo especial para poder expresarse; no existe ninguna facultad religiosa mística para recibir o expresar las emociones religiosas. Estas experiencias son asequibles a través del mecanismo naturalmente ordenado de la mente mortal. Y en esto se halla una explicación de las dificultades que encuentra el Ajustador para ponerse en comunicación directa con la mente material donde reside constantemente.

\par
%\textsuperscript{(1104.6)}
\textsuperscript{101:1.3} El espíritu divino no se pone en contacto con el hombre mortal por medio de los sentimientos o las emociones, sino en el ámbito de los pensamientos más elevados y más espiritualizados. Son vuestros \textit{pensamientos}, y no vuestros sentimientos, los que os conducen hacia Dios. La naturaleza divina sólo se puede percibir con los ojos de la mente. Pero la mente que discierne realmente a Dios, que escucha al Ajustador interior, es la mente pura. <<Sin santidad, ningún hombre puede ver a Dios>>\footnote{\textit{Sin santidad, ningún hombre puede ver a Dios}: Heb 12:14.}. Toda comunión interna y espiritual de este tipo se califica de perspicacia espiritual. Estas experiencias religiosas son el resultado de la impresión producida en la mente del hombre por las operaciones combinadas del Ajustador y del Espíritu de la Verdad, a medida que actúan entre y sobre las ideas, los ideales, las percepciones y los esfuerzos espirituales de los hijos evolutivos de Dios.

\par
%\textsuperscript{(1105.1)}
\textsuperscript{101:1.4} Así pues, la religión no vive y prospera mediante la vista y los sentimientos, sino más bien mediante la fe y la perspicacia. La religión no consiste en el descubrimiento de nuevos hechos o en el hallazgo de una experiencia excepcional, sino más bien en el descubrimiento de nuevos \textit{significados} espirituales en los hechos ya bien conocidos por la humanidad. La experiencia religiosa más elevada no depende de unos actos previos guiados por la creencia, la tradición y la autoridad; la religión no es tampoco el fruto de unos sentimientos sublimes y de unas emociones puramente místicas. Es más bien una experiencia profundamente grande y real de comunión espiritual con las influencias espirituales que residen en la mente humana. Y en la medida en que esta experiencia se puede definir en términos psicológicos, consiste simplemente en la experiencia de sentir que la realidad de creer en Dios es la realidad de esa experiencia puramente personal.

\par
%\textsuperscript{(1105.2)}
\textsuperscript{101:1.5} Aunque la religión no es el producto de las especulaciones racionalistas de una cosmología material, sin embargo es la creación de una perspicacia totalmente racional que se origina en la experiencia mental del hombre. La religión no nace ni de las meditaciones místicas ni de las contemplaciones solitarias, aunque sea siempre más o menos misteriosa y siempre indefinible e inexplicable en términos de la razón puramente intelectual y de la lógica filosófica. Los gérmenes de la verdadera religión se originan en el ámbito de la conciencia moral del hombre, y se revelan en el crecimiento de la perspicacia espiritual del hombre, esa facultad de la personalidad humana que se adquiere como consecuencia de la presencia del Ajustador del Pensamiento que revela a Dios en la mente mortal hambrienta de Dios.

\par
%\textsuperscript{(1105.3)}
\textsuperscript{101:1.6} La fe une la perspicacia moral al discernimiento concienzudo de los valores, y el sentido evolutivo preexistente del deber completa el linaje de la verdadera religión. La experiencia de la religión produce finalmente la conciencia cierta de Dios y la seguridad indudable de la supervivencia de la personalidad creyente.

\par
%\textsuperscript{(1105.4)}
\textsuperscript{101:1.7} Se puede ver así que los anhelos religiosos y los impulsos espirituales no son de tal naturaleza que se limiten a conducir a los hombres a \textit{querer} creer en Dios, sino que son más bien de tal naturaleza y poder que inculcan profundamente en los hombres el convencimiento de que \textit{deberían} creer en Dios. El sentido del deber evolutivo y las obligaciones resultantes de la iluminación de la revelación producen una impresión tan profunda en la naturaleza moral del hombre que éste llega finalmente a esa situación mental y a esa actitud del alma en las que concluye que \textit{no tiene ningún derecho a no creer en Dios}. La sabiduría elevada y superfilosófica de estas personas iluminadas y disciplinadas les enseña finalmente que dudar de Dios o desconfiar de su bondad sería mostrarse infieles hacia el objeto \textit{más real} y \textit{más profundo} que reside en la mente y el alma humanas --- el Ajustador divino.

\section*{2. El hecho de la religión}
\par
%\textsuperscript{(1105.5)}
\textsuperscript{101:2.1} El hecho de la religión consiste enteramente en la experiencia religiosa de los seres humanos racionales y corrientes. Éste es el único sentido en el que la religión puede ser considerada como científica o incluso psicológica. La prueba de que la revelación es revelación consiste en este mismo hecho de la experiencia humana: el hecho de que la revelación sintetiza las ciencias aparentemente divergentes de la naturaleza y la teología de la religión en una filosofía del universo coherente y lógica, en una explicación coordinada e ininterrumpida tanto de la ciencia como de la religión, creando así una armonía mental y una satisfacción espiritual que contesta, en la experiencia humana, a aquellos interrogantes de la mente mortal que ansía saber \textit{de qué manera} el Infinito pone en práctica su voluntad y realiza sus planes en la materia, con las mentes y sobre el espíritu.

\par
%\textsuperscript{(1106.1)}
\textsuperscript{101:2.2} La razón es el método de la ciencia; la fe es el método de la religión; la lógica es la técnica que intenta utilizar la filosofía. La revelación compensa la ausencia del punto de vista morontial, proporcionando una técnica para conseguir unificar la comprensión de la realidad y de las relaciones entre la materia y el espíritu por mediación de la mente. La verdadera revelación nunca hace antinatural a la ciencia, irrazonable a la religión o ilógica a la filosofía.

\par
%\textsuperscript{(1106.2)}
\textsuperscript{101:2.3} Por medio del estudio de la ciencia, la razón puede conducir, a través de la naturaleza, hacia una Causa Primera, pero se necesita la fe religiosa para transformar la Causa Primera de la ciencia en un Dios de salvación; y la revelación se necesita además para validar esta fe, esta perspicacia espiritual.

\par
%\textsuperscript{(1106.3)}
\textsuperscript{101:2.4} Existen dos razones fundamentales para creer en un Dios que fomenta la supervivencia humana:

\par
%\textsuperscript{(1106.4)}
\textsuperscript{101:2.5} 1. La experiencia humana, la seguridad personal, la esperanza y la confianza que se reflejan de una u otra forma y que son desencadenadas por el Ajustador del Pensamiento interior.

\par
%\textsuperscript{(1106.5)}
\textsuperscript{101:2.6} 2. La revelación de la verdad, ya sea mediante el ministerio personal directo del Espíritu de la Verdad, mediante la donación de los Hijos divinos en el mundo, o a través de las revelaciones escritas.

\par
%\textsuperscript{(1106.6)}
\textsuperscript{101:2.7} La ciencia termina su investigación, por medio de la razón, en la hipótesis de una Causa Primera. La religión no se detiene en su trayectoria de fe hasta estar segura de la existencia de un Dios de salvación. Los estudios discriminatorios de la ciencia sugieren lógicamente la realidad y la existencia de un Absoluto. La religión cree sin reservas en la existencia y en la realidad de un Dios que fomenta la supervivencia de la personalidad. Aquello que la metafísica no logra hacer de ninguna manera, y aquello que incluso la filosofía sólo logra hacer parcialmente, la revelación lo consigue: es decir, afirmar que esta Causa Primera de la ciencia y que el Dios de salvación de la religión son \textit{una sola y misma Deidad}.

\par
%\textsuperscript{(1106.7)}
\textsuperscript{101:2.8} La razón es la prueba de la ciencia, la fe es la prueba de la religión, la lógica es la prueba de la filosofía, pero la revelación sólo es validada por la \textit{experiencia} humana. La ciencia proporciona el conocimiento; la religión proporciona la felicidad; la filosofía proporciona la unidad; la revelación confirma la armonía experiencial de este acercamiento trino a la realidad universal.

\par
%\textsuperscript{(1106.8)}
\textsuperscript{101:2.9} La contemplación de la naturaleza sólo puede revelar a un Dios de la naturaleza, a un Dios de movimiento. La naturaleza sólo muestra la materia, el movimiento y la animación ---la vida. Bajo ciertas condiciones, la materia más la energía se manifiestan como formas vivientes, pero mientras que la vida natural es así un fenómeno relativamente continuo, es totalmente transitorio para los individuos. La naturaleza no proporciona una base para una creencia lógica en la supervivencia de la personalidad humana. El hombre religioso que encuentra a Dios en la naturaleza ya ha encontrado primero a este mismo Dios personal en su propia alma.

\par
%\textsuperscript{(1106.9)}
\textsuperscript{101:2.10} La fe revela a Dios en el alma. La revelación, sustituta de la perspicacia morontial en un mundo evolutivo, permite al hombre ver en la naturaleza al mismo Dios que la fe le muestra en su alma. La revelación consigue así colmar con éxito el abismo existente entre lo material y lo espiritual, e incluso entre la criatura y el Creador, entre el hombre y Dios.

\par
%\textsuperscript{(1107.1)}
\textsuperscript{101:2.11} La contemplación de la naturaleza señala lógicamente hacia la existencia de una dirección inteligente, e incluso de una supervisión viviente, pero no revela de ninguna manera satisfactoria a un Dios personal. Por otra parte, la naturaleza no revela nada que impida considerar al universo como la obra del Dios de la religión\footnote{\textit{Las obras de Dios}: Sal 19:1.}. No se puede encontrar a Dios a través de la naturaleza sola, pero una vez que el hombre lo ha encontrado de otra manera, el estudio de la naturaleza se vuelve totalmente coherente con una interpretación más elevada y más espiritual del universo.

\par
%\textsuperscript{(1107.2)}
\textsuperscript{101:2.12} La revelación, como fenómeno que hace época, es periódica; como experiencia personal humana, es continua. La divinidad actúa en la personalidad de los mortales bajo la forma del Ajustador, el don del Padre, bajo la forma del Espíritu de la Verdad del Hijo, y bajo la forma del Espíritu Santo del Espíritu del Universo, mientras que estas tres dotaciones supermortales están unificadas en la evolución experiencial humana bajo la forma del ministerio del Supremo.

\par
%\textsuperscript{(1107.3)}
\textsuperscript{101:2.13} La verdadera religión es hacerse una idea de la realidad, el producto por la fe de la conciencia moral, y no un simple asentimiento intelectual a un conjunto cualquiera de doctrinas dogmáticas. La verdadera religión consiste en la experiencia de que <<el Espíritu mismo da testimonio con nuestro espíritu de que somos hijos de Dios>>\footnote{\textit{El Espíritu mismo da testimonio}: Ro 8:16.}. La religión no consiste en proposiciones teológicas, sino en la perspicacia espiritual y en la sublimidad de la confianza del alma.

\par
%\textsuperscript{(1107.4)}
\textsuperscript{101:2.14} Vuestra naturaleza más profunda ---el Ajustador divino--- crea dentro de vosotros un hambre y una sed de rectitud, cierto anhelo de perfección divina. La religión es el acto de fe por el cual se reconoce este impulso interior por alcanzar la divinidad; y así se originan esa confianza y esa seguridad del alma de las que tomáis conciencia como el camino de la salvación, la técnica para la supervivencia de la personalidad y de todos aquellos valores que habéis llegado a considerar como verdaderos y buenos.

\par
%\textsuperscript{(1107.5)}
\textsuperscript{101:2.15} La comprensión de la religión no ha dependido nunca, y nunca dependerá, de un gran saber o de una lógica ingeniosa. La religión es una perspicacia espiritual, y ésta es precisamente la razón por la que algunos de los más grandes educadores religiosos del mundo, e incluso los profetas, han poseído a veces tan poca sabiduría del mundo. La fe religiosa está al alcance tanto de los eruditos como de los ignorantes.

\par
%\textsuperscript{(1107.6)}
\textsuperscript{101:2.16} La religión debe ser siempre su propio crítico y su propio juez; nunca puede ser observada, y mucho menos comprendida, desde el exterior. Vuestra única seguridad acerca de un Dios personal consiste en vuestra propia perspicacia sobre vuestra creencia en las cosas espirituales, así como vuestra experiencia con ellas. Para todos vuestros semejantes que han tenido una experiencia similar, no es necesario ningún argumento sobre la personalidad o la realidad de Dios, mientras que para todos los demás hombres que no tienen esta seguridad de Dios, ningún argumento posible será nunca realmente convincente.

\par
%\textsuperscript{(1107.7)}
\textsuperscript{101:2.17} La psicología puede en verdad intentar estudiar los fenómenos de las reacciones religiosas ante el entorno social, pero nunca puede esperar penetrar en los móviles y en los efectos reales e internos de la religión. Únicamente la teología, la esfera de la fe y la técnica de la revelación, puede proporcionar algún tipo de explicación inteligente sobre la naturaleza y el contenido de la experiencia religiosa.

\section*{3. Las características de la religión}
\par
%\textsuperscript{(1107.8)}
\textsuperscript{101:3.1} La religión es tan vital que sobrevive en ausencia de erudición. Vive a pesar de contaminarse con cosmologías erróneas y falsas filosofías; sobrevive incluso a la confusión de la metafísica. A través de todas las vicisitudes históricas de la religión, siempre sobrevive aquello que es indispensable para el progreso y la supervivencia humanos: la conciencia ética y el conocimiento moral.

\par
%\textsuperscript{(1108.1)}
\textsuperscript{101:3.2} La perspicacia de la fe, o intuición espiritual, es la dotación de la mente cósmica en asociación con el Ajustador del Pensamiento, que es el regalo del Padre al hombre. La razón espiritual, la inteligencia del alma, es la dotación del Espíritu Santo, el regalo del Espíritu Creativo al hombre. La filosofía espiritual, la sabiduría de las realidades espirituales, es la dotación del Espíritu de la Verdad, el regalo combinado de los Hijos donadores a los hijos de los hombres. La coordinación y la interasociación de estas dotaciones espirituales hacen que el hombre tenga un destino potencial como personalidad espiritual.

\par
%\textsuperscript{(1108.2)}
\textsuperscript{101:3.3} Esta misma personalidad espiritual, bajo una forma primitiva y embrionaria, es la que, poseída por el Ajustador, sobrevive a la muerte natural en la carne. Por medio del camino viviente proporcionado por los Hijos divinos, esta entidad combinada de origen espiritual, en asociación con la experiencia humana, está capacitada para sobrevivir (bajo la custodia del Ajustador) a la disolución del yo físico compuesto de mente y de materia, cuando esta asociación transitoria de lo material y lo espiritual se destruye debido al cese del movimiento vital.

\par
%\textsuperscript{(1108.3)}
\textsuperscript{101:3.4} El alma del hombre se revela por medio de la fe religiosa, y demuestra la divinidad potencial de su naturaleza emergente por la manera característica en que induce a la personalidad mortal a reaccionar ante ciertas situaciones intelectuales y sociales duras y difíciles. La fe espiritual auténtica (la verdadera conciencia moral) se revela en que:

\par
%\textsuperscript{(1108.4)}
\textsuperscript{101:3.5} 1. Provoca el progreso de la ética y de la moral a pesar de las tendencias animales inherentes y adversas.

\par
%\textsuperscript{(1108.5)}
\textsuperscript{101:3.6} 2. Produce una confianza sublime en la bondad de Dios, en medio incluso de amargas decepciones y de derrotas aplastantes.

\par
%\textsuperscript{(1108.6)}
\textsuperscript{101:3.7} 3. Genera un valor y una confianza profundos a pesar de las adversidades naturales y de las calamidades físicas.

\par
%\textsuperscript{(1108.7)}
\textsuperscript{101:3.8} 4. Muestra una serenidad inexplicable y una tranquilidad continua a pesar de las enfermedades desconcertantes e incluso de los sufrimientos físicos agudos.

\par
%\textsuperscript{(1108.8)}
\textsuperscript{101:3.9} 5. Mantiene a la personalidad en una calma y un equilibrio misteriosos en medio de los malos tratos y de las injusticias más flagrantes.

\par
%\textsuperscript{(1108.9)}
\textsuperscript{101:3.10} 6. Mantiene una confianza divina en la victoria final, a pesar de las crueldades de un destino aparentemente ciego y de la aparente indiferencia total de las fuerzas naturales hacia el bienestar humano.

\par
%\textsuperscript{(1108.10)}
\textsuperscript{101:3.11} 7. Insiste en creer inquebrantablemente en Dios a pesar de todas las demostraciones contrarias de la lógica, y resiste con éxito a todos los demás sofismas intelectuales.

\par
%\textsuperscript{(1108.11)}
\textsuperscript{101:3.12} 8. Continúa mostrando una fe intrépida en la supervivencia del alma, sin tener en cuenta las enseñanzas engañosas de la falsa ciencia ni las ilusiones persuasivas de una filosofía errónea.

\par
%\textsuperscript{(1108.12)}
\textsuperscript{101:3.13} 9. Vive y triunfa a pesar de la sobrecarga abrumadora de las civilizaciones complejas y parciales de los tiempos modernos.

\par
%\textsuperscript{(1108.13)}
\textsuperscript{101:3.14} 10. Contribuye a la supervivencia continua del altruismo a pesar del egoísmo humano, los antagonismos sociales, las avaricias industriales y los desajustes políticos.

\par
%\textsuperscript{(1108.14)}
\textsuperscript{101:3.15} 11. Se adhiere firmemente a una creencia sublime en la unidad universal y en la guía divina, sin tener en cuenta la presencia desconcertante del mal y del pecado.

\par
%\textsuperscript{(1108.15)}
\textsuperscript{101:3.16} 12. Continúa muy acertadamente adorando a Dios a pesar de todo y por encima de todo. Se atreve a declarar: <<Aunque Él me mate, seguiré sirviéndole>>\footnote{\textit{Aunque Él me mate}: Job 13:15.}.

\par
%\textsuperscript{(1108.16)}
\textsuperscript{101:3.17} Sabemos pues, por tres fenómenos, que el hombre posee un espíritu o unos espíritus divinos que residen dentro de él: primero, por la experiencia personal ---la fe religiosa; segundo, por la revelación ---personal y racial; y tercero, por la manifestación asombrosa de unas reacciones extraordinarias y poco naturales hacia el entorno material, tal como ha quedado ilustrado en la relación anterior de doce comportamientos de tipo espiritual en presencia de unas situaciones concretas y difíciles de la existencia humana real. Y aún hay otros más.

\par
%\textsuperscript{(1109.1)}
\textsuperscript{101:3.18} Esta actuación esencial y vigorosa de la fe en el ámbito de la religión es precisamente la que le da al hombre mortal el derecho de aseverar la posesión personal y la realidad espiritual de este don supremo de la naturaleza humana: la experiencia religiosa.

\section*{4. Las limitaciones de la revelación}
\par
%\textsuperscript{(1109.2)}
\textsuperscript{101:4.1} Puesto que vuestro mundo ignora generalmente el origen de las cosas, incluso de las cosas físicas, ha parecido sabio proporcionarle de vez en cuando conocimientos de cosmología. Esto siempre ha causado problemas para el futuro. Las leyes de la revelación nos obstaculizan enormemente porque prohíben comunicar conocimientos inmerecidos o prematuros. Toda cosmología presentada como parte de una religión revelada está destinada a quedarse atrás en muy poco tiempo. Por consiguiente, los estudiosos futuros de esa revelación se sienten tentados a desechar cualquier elemento de verdad religiosa auténtica que pueda contener, porque descubren errores a primera vista en las cosmologías asociadas que se presentan en ella.

\par
%\textsuperscript{(1109.3)}
\textsuperscript{101:4.2} La humanidad debería comprender que nosotros, que participamos en la revelación de la verdad, estamos muy rigurosamente limitados por las instrucciones de nuestros superiores. No tenemos libertad para anticipar los descubrimientos científicos que se producirán en mil años. Los reveladores deben actuar con arreglo a las instrucciones que forman parte del mandato de revelar. No vemos ninguna manera de salvar esta dificultad, ni ahora ni en ningún momento del futuro. Sabemos muy bien que los hechos históricos y las verdades religiosas de esta serie de presentaciones revelatorias permanecerán en los anales de las épocas venideras, pero dentro de pocos años muchas de nuestras afirmaciones relacionadas con las ciencias físicas necesitarán una revisión a consecuencia de los desarrollos científicos adicionales y de los nuevos descubrimientos. Estos nuevos desarrollos los prevemos incluso desde ahora, pero se nos prohíbe incluir en nuestros escritos revelatorios esos hechos aún no descubiertos por la humanidad. Que quede muy claro que las revelaciones no son necesariamente inspiradas. La cosmología que figura en estas revelaciones \textit{no es inspirada}. Está limitada por el permiso que nos han concedido para coordinar y clasificar el conocimiento de hoy en día. Aunque la perspicacia divina o espiritual sea un don, \textit{la sabiduría humana tiene que evolucionar}.

\par
%\textsuperscript{(1109.4)}
\textsuperscript{101:4.3} La verdad siempre es una revelación: es una autorrevelación cuando emerge como resultado del trabajo del Ajustador interior, y es una revelación que hace época cuando es presentada mediante la actuación de algún otro agente, grupo o personalidad celestial.

\par
%\textsuperscript{(1109.5)}
\textsuperscript{101:4.4} A fin de cuentas, la religión ha de ser juzgada por sus frutos, con arreglo a la manera y a la amplitud en que manifiesta su propia excelencia inherente y divina.

\par
%\textsuperscript{(1109.6)}
\textsuperscript{101:4.5} La verdad puede ser sólo relativamente inspirada, aunque la revelación sea invariablemente un fenómeno espiritual. Las afirmaciones referentes a la cosmología nunca son inspiradas, pero estas revelaciones tienen un inmenso valor ya que al menos clarifican transitoriamente los conocimientos mediante:

\par
%\textsuperscript{(1109.7)}
\textsuperscript{101:4.6} 1. La reducción de la confusión, eliminando con autoridad los errores.

\par
%\textsuperscript{(1109.8)}
\textsuperscript{101:4.7} 2. La coordinación de los hechos y de las observaciones conocidos o a punto de ser conocidos.

\par
%\textsuperscript{(1110.1)}
\textsuperscript{101:4.8} 3. El restablecimiento de importantes fragmentos de conocimientos perdidos relacionados con acontecimientos históricos del pasado lejano.

\par
%\textsuperscript{(1110.2)}
\textsuperscript{101:4.9} 4. El suministro de una información que colma las lagunas vitales existentes en los conocimientos adquiridos de otras maneras.

\par
%\textsuperscript{(1110.3)}
\textsuperscript{101:4.10} 5. La presentación de unos datos cósmicos de tal forma que ilumine las enseñanzas espirituales contenidas en la revelación que las acompaña.

\section*{5. La religión ampliada por revelación}
\par
%\textsuperscript{(1110.4)}
\textsuperscript{101:5.1} La revelación es una técnica que permite ahorrar grandes períodos de tiempo en el trabajo necesario de clasificar y separar los errores de la evolución de las verdades conseguidas por medio del espíritu.

\par
%\textsuperscript{(1110.5)}
\textsuperscript{101:5.2} La ciencia se ocupa de los \textit{hechos}; la religión sólo se interesa por los \textit{valores}. A través de una filosofía iluminada, la mente se esfuerza por unir los significados de los hechos y de los valores para llegar así a un concepto de la \textit{realidad} total. Recordad que la ciencia es el ámbito del conocimiento, la filosofía el campo de la sabiduría y la religión la esfera de la experiencia de la fe. Pero la religión presenta sin embargo dos fases de manifestación:

\par
%\textsuperscript{(1110.6)}
\textsuperscript{101:5.3} 1. La religión evolutiva. La experiencia de la adoración primitiva, la religión que procede de la mente.

\par
%\textsuperscript{(1110.7)}
\textsuperscript{101:5.4} 2. La religión revelada. La actitud hacia el universo que procede del espíritu; la seguridad y la creencia de que las realidades eternas se conservan, de que la personalidad sobrevive y de que finalmente se alcanza la Deidad cósmica, cuyo propósito ha hecho posible todo esto. Tarde o temprano, la religión evolutiva está destinada a recibir la expansión espiritual de la revelación; esto forma parte del plan del universo.

\par
%\textsuperscript{(1110.8)}
\textsuperscript{101:5.5} Tanto la ciencia como la religión emprenden su camino suponiendo ciertas bases generalmente aceptadas para poder hacer deducciones lógicas. Así pues, la filosofía debe empezar también su carrera suponiendo la realidad de tres cosas:

\par
%\textsuperscript{(1110.9)}
\textsuperscript{101:5.6} 1. El cuerpo material.

\par
%\textsuperscript{(1110.10)}
\textsuperscript{101:5.7} 2. La fase supermaterial del ser humano, el alma o incluso el espíritu interior.

\par
%\textsuperscript{(1110.11)}
\textsuperscript{101:5.8} 3. La mente humana, el mecanismo para la intercomunicación y la interasociación entre el espíritu y la materia, entre lo material y lo espiritual.

\par
%\textsuperscript{(1110.12)}
\textsuperscript{101:5.9} Los científicos reúnen los hechos, los filósofos coordinan las ideas, mientras que los profetas ensalzan los ideales. Los sentimientos y las emociones acompañan invariablemente a la religión, pero no son la religión. La religión puede ser el sentimiento de la experiencia, pero es difícilmente la experiencia de los sentimientos. Ni la lógica (la racionalización) ni las emociones (los sentimientos) son una parte esencial de la experiencia religiosa, aunque las dos pueden estar diversamente asociadas al ejercicio de la fe para favorecer la perspicacia espiritual de la realidad, todo ello de acuerdo con el estado y las tendencias temperamentales de la mente individual.

\par
%\textsuperscript{(1110.13)}
\textsuperscript{101:5.10} La religión evolutiva es la manifestación exterior del don del ayudante mental del universo local encargado de crear y de fomentar la característica de la adoración en el hombre evolutivo. Estas religiones primitivas se interesan directamente por la ética y la moral, por el sentido del \textit{deber} humano. Estas religiones están basadas en la seguridad de la conciencia y conducen a la estabilización de unas civilizaciones relativamente éticas.

\par
%\textsuperscript{(1111.1)}
\textsuperscript{101:5.11} Las religiones personalmente reveladas están patrocinadas por los espíritus donados que representan a las tres personas de la Trinidad del Paraíso, y se ocupan especialmente de la expansión de la \textit{verdad}. La religión evolutiva introduce a fondo en el individuo la idea del deber personal; la religión revelada hace cada vez más hincapié en el amor, en la regla de oro.

\par
%\textsuperscript{(1111.2)}
\textsuperscript{101:5.12} La religión evolutiva descansa enteramente sobre la fe. La revelación posee la seguridad adicional de presentar extensamente las verdades de la divinidad y de la realidad, y el testimonio aun más valioso de la experiencia real que se acumula como consecuencia de la unión práctica activa entre la fe de la evolución y la verdad de la revelación. Esta unión activa entre la fe humana y la verdad divina constituye la posesión de un carácter que está bien encaminado hacia la adquisición efectiva de una personalidad morontial.

\par
%\textsuperscript{(1111.3)}
\textsuperscript{101:5.13} La religión evolutiva sólo proporciona la certidumbre basada en la fe y la confirmación de la conciencia; la religión revelada proporciona la certidumbre basada en la fe más la verdad de una experiencia viviente con las realidades de la revelación. La tercera etapa de la religión, o tercera fase de la experiencia religiosa, está relacionada con el estado morontial, con la comprensión más firme de la mota. Durante la progresión morontial, las verdades de la religión revelada se amplían de manera creciente; conoceréis cada vez mejor la verdad de los valores supremos, las bondades divinas, las relaciones universales, las realidades eternas y los destinos finales.

\par
%\textsuperscript{(1111.4)}
\textsuperscript{101:5.14} A lo largo de la progresión morontial, la seguridad de la verdad reemplaza cada vez más a la seguridad de la fe. Cuando seáis enrolados finalmente en el verdadero mundo espiritual, entonces las seguridades de la pura perspicacia espiritual actuarán en lugar de la fe y de la verdad, o más bien conjuntamente con ellas y superponiéndose a estas antiguas técnicas de seguridad de la personalidad.

\section*{6. La experiencia religiosa progresiva}
\par
%\textsuperscript{(1111.5)}
\textsuperscript{101:6.1} La fase morontial de la religión revelada está relacionada con la \textit{experienciade la supervivencia}, y su gran motivación consiste en alcanzar la perfección del espíritu. También se encuentra presente el estímulo superior de la adoración, unido a la llamada impelente de un servicio ético creciente. La perspicacia morontial trae consigo una conciencia cada vez mayor del Séptuple, del Supremo e incluso del Último.

\par
%\textsuperscript{(1111.6)}
\textsuperscript{101:6.2} A lo largo de toda la experiencia religiosa, desde sus primeros comienzos en el nivel material hasta el momento en que se alcanza el pleno estado espiritual, el Ajustador es el secreto para la comprensión personal de la realidad de la existencia del Supremo; y este mismo Ajustador posee también los secretos de vuestra fe en el logro trascendental del Último. La personalidad experiencial del hombre en evolución, unida a la esencia bajo la forma de Ajustador procedente del Dios existencial, constituye la culminación potencial de la existencia suprema, y es por naturaleza la base para la existenciación superfinita de la personalidad trascendental.

\par
%\textsuperscript{(1111.7)}
\textsuperscript{101:6.3} La voluntad moral engloba las decisiones basadas en el conocimiento razonado, acrecentadas por la sabiduría y aprobadas por la fe religiosa. Estas elecciones son actos de naturaleza moral y prueban la existencia de una personalidad moral, la precursora de la personalidad morontial y, finalmente, del verdadero estado espiritual.

\par
%\textsuperscript{(1111.8)}
\textsuperscript{101:6.4} El tipo evolutivo de conocimiento no es más que la acumulación del material protoplásmico de la memoria; ésta es la forma más primitiva de conciencia que tienen las criaturas. La sabiduría engloba las ideas formuladas a partir de la memoria protoplásmica mediante un proceso de asociaciones y recombinaciones, y estos fenómenos son los que diferencian a la mente humana de la simple mente animal. Los animales tienen conocimientos, pero sólo el hombre posee capacidad para la sabiduría. La verdad se vuelve accesible para el individuo dotado de sabiduría porque a dicha mente se le conceden los espíritus del Padre y de los Hijos: el Ajustador del Pensamiento y el Espíritu de la Verdad.

\par
%\textsuperscript{(1112.1)}
\textsuperscript{101:6.5} Cuando Cristo Miguel se donó en Urantia, vivió bajo el reinado de la religión evolutiva hasta la época de su bautismo. Desde aquel momento hasta el acontecimiento de su crucifixión incluido, llevó adelante su obra mediante la guía conjunta de la religión evolutiva y de la religión revelada. Desde la mañana de su resurrección hasta su ascensión, atravesó las múltiples fases de la vida morontial de transición humana desde el mundo de la materia hasta el mundo del espíritu. Después de su ascensión, Miguel adquirió el dominio de la experiencia de la Supremacía, la comprensión del Supremo; y como era la única persona de Nebadon que poseía una capacidad ilimitada para experimentar la realidad del Supremo, alcanzó inmediatamente el estado de la soberanía de supremacía en, y sobre, su universo local.

\par
%\textsuperscript{(1112.2)}
\textsuperscript{101:6.6} En el hombre, la fusión final con el Ajustador interior y la unidad resultante ---la síntesis del hombre y de la esencia de Dios en una personalidad--- hacen de él, en potencia, una parte viviente del Supremo, y aseguran a este antiguo ser mortal el derecho de nacimiento eterno a perseguir interminablemente la finalidad del servicio universal con y para el Supremo.

\par
%\textsuperscript{(1112.3)}
\textsuperscript{101:6.7} La revelación enseña al hombre mortal que para emprender esta aventura tan magnífica y fascinante a través del espacio y por medio de la progresión del tiempo, debe empezar por organizar sus conocimientos en ideas-decisiones; luego debe ordenarle a la sabiduría que trabaje sin cesar en su noble tarea de transformar las ideas que posee en ideales cada vez más prácticos, pero no obstante celestiales, e incluso en aquellos conceptos que son tan razonables como ideas, y tan lógicos como ideales, que el Ajustador se atreva a combinarlos y espiritualizarlos de tal manera que se encuentren disponibles para esa asociación, en la mente finita, que los convertirá en el verdadero complemento humano ya preparado para la actividad del Espíritu de la Verdad de los Hijos, las manifestaciones espacio-temporales de la verdad del Paraíso ---de la verdad universal. La coordinación de las ideas-decisiones, de los ideales lógicos y de la verdad divina constituye la posesión de un carácter justo, el requisito previo para que un mortal sea admitido en las realidades en constante expansión y cada vez más espirituales de los mundos morontiales.

\par
%\textsuperscript{(1112.4)}
\textsuperscript{101:6.8} Las enseñanzas de Jesús constituyeron la primera religión urantiana que abarcó tan plenamente una coordinación armoniosa de conocimiento, sabiduría, fe, verdad y amor, que proporcionó de manera total y simultánea la tranquilidad temporal, la certidumbre intelectual, la iluminación moral, la estabilidad filosófica, la sensibilidad ética, la conciencia de Dios y la firme seguridad de la supervivencia personal. La fe de Jesús señalaba el camino hacia la finalidad de la salvación humana, hacia lo máximo que pueden alcanzar los mortales en el universo, puesto que aseguraba:

\par
%\textsuperscript{(1112.5)}
\textsuperscript{101:6.9} 1. La liberación de las trabas materiales mediante la comprensión personal de la filiación con Dios, que es espíritu\footnote{\textit{Dios es espíritu}: Jn 4:24.}.

\par
%\textsuperscript{(1112.6)}
\textsuperscript{101:6.10} 2. La liberación de la esclavitud intelectual: el hombre conocerá la verdad, y la verdad lo hará libre\footnote{\textit{Conocerás la verdad, y la verdad te hará libre}: Jn 8:32.}.

\par
%\textsuperscript{(1112.7)}
\textsuperscript{101:6.11} 3. La liberación de la ceguera espiritual\footnote{\textit{Cegera espiritual}: Jn 9:39.}, la comprensión humana de la fraternidad de los seres mortales y la conciencia morontial de la hermandad de todas las criaturas del universo\footnote{\textit{Realización de la hermandad de Dios}: Jn 1:12.}; el descubrimiento de la realidad espiritual a través del servicio, y la revelación de la bondad de los valores espirituales por medio del ministerio.

\par
%\textsuperscript{(1113.1)}
\textsuperscript{101:6.12} 4. La liberación del estado incompleto del yo mediante el hecho de alcanzar los niveles espirituales del universo y a través de la comprensión final de la armonía de Havona y de la perfección del Paraíso.

\par
%\textsuperscript{(1113.2)}
\textsuperscript{101:6.13} 5. La liberación del yo, escapando a las limitaciones de la conciencia de sí mismo mediante el hecho de alcanzar los niveles cósmicos de la mente Suprema y gracias a la coordinación con los logros de todos los demás seres conscientes de sí mismos.

\par
%\textsuperscript{(1113.3)}
\textsuperscript{101:6.14} 6. La liberación del tiempo, la consecución de una vida eterna\footnote{\textit{Vida eterna}: Dn 12:2; Mt 19:16,29; 25:46; Mc 10:17,30; Lc 10:25; 18:18,30; Jn 3:15-16,36; 4:14,36; 5:24,39; 6:27,40,47; 6:54,68; 8:51-52; 10:28; 11:25-26; 12:25,50; 17:2-3; Hch 13:46-48; Ro 2:7; 5:21; 6:22-23; Gl 6:8; 1 Ti 1:16; 6:12,19; Tit 1:2; 3:7; 1 Jn 1:2; 2:25; 3:15; 5:11,13,20; Jud 1:21; Ap 22:5.} de progreso sin fin para reconocer a Dios y al servicio de Dios.

\par
%\textsuperscript{(1113.4)}
\textsuperscript{101:6.15} 7. La liberación de lo finito, la unión perfeccionada con la Deidad en el Supremo y a través de él, mediante la cual la criatura intenta descubrir trascendentalmente al Último en los niveles postfinalitarios de lo absonito.

\par
%\textsuperscript{(1113.5)}
\textsuperscript{101:6.16} Esta liberación séptuple equivale a realizar de manera completa y perfecta la experiencia última del Padre Universal. Todo esto está contenido en potencia dentro de la realidad de la fe de la experiencia religiosa humana. Y puede estar contenido así, ya que la fe de Jesús\footnote{\textit{La fe de Jesús}: Ro 3:22; Gl 2:16; 3:22; Ap 14:12.} estaba alimentada por unas realidades que se encuentran incluso más allá de lo último, y su fe revelaba dichas realidades; la fe de Jesús se acercaba a la categoría de un absoluto universal en la medida en que esto se puede manifestar en el cosmos espacio-temporal en evolución.

\par
%\textsuperscript{(1113.6)}
\textsuperscript{101:6.17} El hombre mortal, cuando se apropia de la fe de Jesús, puede probar de antemano, en el tiempo, las realidades de la eternidad. Jesús descubrió en la experiencia humana al Padre Final, y sus hermanos encarnados en la vida mortal pueden seguirlo en esta misma experiencia de descubrimiento del Padre. En esta experiencia con el Padre pueden incluso conseguir, tal como son, la misma satisfacción que Jesús consiguió tal como él era. En el universo de Nebadon se actualizaron unos nuevos potenciales a consecuencia de la donación final de Miguel, y uno de ellos fue la nueva iluminación del camino de la eternidad\footnote{\textit{Un nuevo camino viviente}: Jn 14:6; Heb 10:20.} que conduce al Padre de todos, y que puede ser recorrido incluso por los mortales materiales de carne y hueso durante su vida inicial en los planetas del espacio. Jesús era y es la nueva vía viviente por la que el hombre puede recibir la herencia divina\footnote{\textit{Herencia divina}: 1 P 1:4.} que el Padre ha decretado que será suya con tal que la pida. En Jesús se encuentran abundantemente demostrados tanto los comienzos como las finalizaciones de la experiencia con la fe de la humanidad, incluso de la humanidad divina.

\section*{7. Una filosofía personal de la religión}
\par
%\textsuperscript{(1113.7)}
\textsuperscript{101:7.1} Una idea no es más que un plan teórico de acción, mientras que una decisión firme es un plan de acción validado. Un estereotipo es un plan de acción aceptado sin validación. Los materiales con los que se puede construir una filosofía personal de la religión proceden tanto de la experiencia interior como de la experiencia del individuo con su entorno. La posición social, las condiciones económicas, las oportunidades educativas, las inclinaciones morales, las influencias institucionales, los desarrollos políticos, las tendencias raciales y las enseñanzas religiosas de la época y del lugar donde uno vive se convierten todos en factores que afectan a la formulación de una filosofía personal de la religión. Incluso el temperamento inherente y las inclinaciones intelectuales determinan notablemente el tipo de filosofía religiosa. La vocación, el matrimonio y los parientes influyen todos sobre la evolución de las normas de vida personales.

\par
%\textsuperscript{(1113.8)}
\textsuperscript{101:7.2} Una filosofía de la religión se desarrolla a partir de un crecimiento básico de las ideas, más la vida experimental, siendo ambos modificados por la tendencia a imitar a los semejantes. La validez de las conclusiones filosóficas depende de una manera de pensar aguda, honrada y juiciosa, en unión con la sensibilidad a los significados y la exactitud en la evaluación. Las personas moralmente cobardes nunca consiguen unos niveles elevados de pensamiento filosófico; hace falta valor para meterse en nuevos niveles de experiencia e intentar explorar los terrenos desconocidos de la vida intelectual.

\par
%\textsuperscript{(1114.1)}
\textsuperscript{101:7.3} Dentro de poco aparecerán nuevos sistemas de valores; se conseguirán nuevas formulaciones de principios y criterios; se reformarán las costumbres y los ideales; se alcanzará cierta idea de un Dios personal, seguida de unos conceptos más amplios sobre las relaciones con esta idea.

\par
%\textsuperscript{(1114.2)}
\textsuperscript{101:7.4} La gran diferencia entre una filosofía religiosa y una filosofía no religiosa de la vida consiste en la naturaleza y el nivel de los valores reconocidos, y en el objeto de las lealtades. La evolución de la filosofía religiosa comporta cuatro fases: Una experiencia así puede volverse simplemente conformista, resignada a someterse a la tradición y a la autoridad. O puede satisfacerse con pequeños logros, los suficientes como para estabilizar la vida diaria, por lo que pronto se queda detenida en este nivel atrasado. Estos mortales creen que es mejor dejar las cosas como están. Un tercer grupo progresa hasta el nivel de la intelectualidad lógica, pero se estancan allí a consecuencia de la esclavitud cultural. Es verdaderamente lamentable contemplar a unos intelectos gigantes totalmente sometidos al dominio cruel de la servidumbre cultural. Es igualmente patético observar a aquellos que cambian su esclavitud cultural por las cadenas materialistas de una ciencia calificada erróneamente de esta manera. El cuarto nivel de la filosofía consigue liberarse de todos los obstáculos convencionales y tradicionales, y se atreve a pensar, actuar y vivir de manera honrada, leal, intrépida y veraz.

\par
%\textsuperscript{(1114.3)}
\textsuperscript{101:7.5} La prueba decisiva para cualquier filosofía religiosa consiste en saber si distingue o no entre las realidades del mundo material y las del mundo espiritual, reconociendo al mismo tiempo su unificación en el esfuerzo intelectual y el servicio social. Una buena filosofía religiosa no confunde las cosas de Dios con las cosas del César\footnote{\textit{Separar las cosas del César de las de Dios}: Mt 22:21; Mc 12:17; Lc 20:25.}. Y tampoco reconoce que el culto estético a las puras maravillas sea un sustituto de la religión.

\par
%\textsuperscript{(1114.4)}
\textsuperscript{101:7.6} La filosofía transforma la religión primitiva, que era principalmente un cuento de hadas de la conciencia, en una experiencia viviente de los valores ascendentes de la realidad cósmica.

\section*{8. La fe y la creencia}
\par
%\textsuperscript{(1114.5)}
\textsuperscript{101:8.1} La creencia alcanza el nivel de la fe cuando motiva la vida y modela la manera de vivir. La aceptación de una enseñanza como verdadera no es la fe; es una simple creencia. La certidumbre y la convicción tampoco son la fe. Un estado mental sólo alcanza los niveles de la fe cuando domina realmente la manera de vivir. La fe es un atributo viviente de la experiencia religiosa personal auténtica. Uno cree en la verdad, admira la belleza y respeta la bondad, pero no las adora; una actitud así de fe salvadora está centrada solamente en Dios, que es la personificación de todas estas cosas e infinitamente más.

\par
%\textsuperscript{(1114.6)}
\textsuperscript{101:8.2} La creencia limita y ata siempre; la fe expande y desata. La creencia fija, la fe libera. Pero la fe religiosa viviente es más que una asociación de creencias nobles; es más que un sistema elevado de filosofía; es una experiencia viviente que se interesa por los significados espirituales, los ideales divinos y los valores supremos; conoce a Dios y sirve a los hombres. Las creencias pueden llegar a ser propiedad de un grupo, pero la fe ha de ser personal. Las creencias teológicas se pueden sugerir a un grupo, pero la fe sólo puede surgir en el corazón de la persona religiosa individual.

\par
%\textsuperscript{(1114.7)}
\textsuperscript{101:8.3} La fe falsifica su misión de confianza cuando se atreve a negar las realidades y a conferir a sus adeptos un conocimiento ficticio. La fe se vuelve traidora cuando fomenta la traición de la integridad intelectual y desprecia la lealtad a los valores supremos y a los ideales divinos. La fe nunca rehuye el deber de resolver los problemas de la vida mortal. La fe viviente no fomenta el fanatismo, la persecución o la intolerancia.

\par
%\textsuperscript{(1115.1)}
\textsuperscript{101:8.4} La fe no encadena la imaginación creadora ni tampoco mantiene prejuicios irrazonables hacia los descubrimientos de la investigación científica. La fe vitaliza la religión y obliga a la persona religiosa a vivir heroicamente la regla de oro. El fervor de la fe está en armonía con el conocimiento, y sus esfuerzos son el preludio de una paz sublime.

\section*{9. La religión y la moralidad}
\par
%\textsuperscript{(1115.2)}
\textsuperscript{101:9.1} Ninguna supuesta revelación de la religión puede ser considerada como auténtica si no logra reconocer las exigencias del deber de las obligaciones éticas que han sido creadas y fomentadas por la religión evolutiva anterior. La revelación amplía infaliblemente el horizonte ético de la religión evolutiva, extendiendo simultánea e indefectiblemente las obligaciones morales de todas las revelaciones anteriores.

\par
%\textsuperscript{(1115.3)}
\textsuperscript{101:9.2} Cuando os atrevéis a hacer un juicio crítico sobre la religión primitiva del hombre (o sobre la religión del hombre primitivo), deberíais recordar que hay que juzgar a aquellos salvajes, y evaluar su experiencia religiosa, de acuerdo con sus luces y su nivel de conciencia. No cometáis el error de juzgar la religión de otras personas según vuestros propios criterios sobre el conocimiento y la verdad.

\par
%\textsuperscript{(1115.4)}
\textsuperscript{101:9.3} La verdadera religión es ese convencimiento sublime y profundo, dentro del alma, que advierte irresistiblemente al hombre que sería malo para él no creer en esas realidades morontiales que constituyen sus conceptos éticos y morales más elevados, su interpretación más elevada de los valores más grandes de la vida y de las realidades más profundas del universo. Una religión así es simplemente la experiencia de abandonar la lealtad intelectual a los dictados más elevados de la conciencia espiritual.

\par
%\textsuperscript{(1115.5)}
\textsuperscript{101:9.4} La búsqueda de la belleza sólo forma parte de la religión en la medida en que es ética y en el grado en que enriquece el concepto de la moral. El arte sólo es religioso cuando se difunde con una intención derivada de una elevada motivación espiritual.

\par
%\textsuperscript{(1115.6)}
\textsuperscript{101:9.5} La conciencia espiritual iluminada del hombre civilizado no se interesa tanto por una creencia intelectual específica, o por una manera particular de vivir, como por descubrir la verdad de la vida, la técnica buena y correcta de reaccionar ante las situaciones constantemente recurrentes de la existencia mortal. La conciencia moral es simplemente un nombre que se aplica al reconocimiento y al conocimiento humanos de esos valores éticos y de esos valores morontiales emergentes respecto a los cuales el sentido del deber exige que el hombre se atenga a ellos para controlar y dirigir su conducta diaria.

\par
%\textsuperscript{(1115.7)}
\textsuperscript{101:9.6} Aunque reconocemos que la religión es imperfecta, existen al menos dos manifestaciones prácticas de su naturaleza y de su función:

\par
%\textsuperscript{(1115.8)}
\textsuperscript{101:9.7} 1. El impulso espiritual y la presión filosófica de la religión tienden a hacer que el hombre proyecte su apreciación de los valores morales directamente hacia afuera, hacia los asuntos de sus semejantes ---la reacción ética de la religión.

\par
%\textsuperscript{(1115.9)}
\textsuperscript{101:9.8} 2. La religión crea para la mente humana una conciencia espiritualizada de la realidad divina, basada en unos conceptos precedentes de los valores morales, derivada por la fe de dichos conceptos, y coordinada con unos conceptos superpuestos de los valores espirituales. La religión se vuelve así una censora de los asuntos humanos, una forma de esperanza y de confianza moral glorificada en la realidad, en las realidades elevadas del tiempo y en las realidades más duraderas de la eternidad.

\par
%\textsuperscript{(1116.1)}
\textsuperscript{101:9.9} La fe se convierte en la conexión entre la conciencia moral y el concepto espiritual de la realidad duradera. La religión se vuelve el camino por el que el hombre escapa de las limitaciones materiales del mundo temporal y natural hacia las realidades celestiales del mundo eterno y espiritual por medio de la técnica de la salvación, de la transformación morontial progresiva.

\section*{10. La religión como liberadora del hombre}
\par
%\textsuperscript{(1116.2)}
\textsuperscript{101:10.1} El hombre inteligente sabe que es un hijo de la naturaleza, una parte del universo material; asimismo, no discierne ninguna supervivencia de la personalidad individual en los movimientos y tensiones del nivel matemático del universo energético. El hombre tampoco puede discernir nunca la realidad espiritual a través del examen de las causas y de los efectos físicos.

\par
%\textsuperscript{(1116.3)}
\textsuperscript{101:10.2} Un ser humano se da cuenta también de que es una parte del cosmos ideacional, pero aunque un concepto puede perdurar más allá de la duración de la vida de un mortal, no hay nada inherente al concepto que indique la supervivencia personal de la personalidad que lo concibe. El agotamiento de las posibilidades de la lógica y de la razón tampoco revelará nunca al lógico o al razonador la verdad eterna de la supervivencia de la personalidad.

\par
%\textsuperscript{(1116.4)}
\textsuperscript{101:10.3} El nivel material de la ley asegura la continuidad de la causalidad, la reacción interminable de los efectos a unas acciones precedentes; el nivel mental sugiere la perpetuación de la continuidad de las ideas, el flujo incesante de la potencialidad conceptual procedente de las ideas preexistentes. Pero ninguno de estos niveles del universo revela al mortal inquisitivo una vía por donde poder escapar de su estado parcial y de la intolerable incertidumbre de ser una realidad transitoria en el universo, una personalidad temporal condenada a extinguirse cuando se agoten las energías limitadas de la vida.

\par
%\textsuperscript{(1116.5)}
\textsuperscript{101:10.4} Sólo a través del camino morontial, que conduce a la perspicacia espiritual, es como el hombre podrá romper alguna vez las cadenas inherentes a su estado mortal en el universo. La energía y la mente sí conducen de vuelta hacia el Paraíso y la Deidad, pero ni la dotación energética ni la dotación mental del hombre proceden directamente de esta Deidad del Paraíso. El hombre sólo es un hijo de Dios en el sentido espiritual. Y esto es así porque sólo en el sentido espiritual es como el hombre está dotado y habitado en este momento por el Padre Paradisiaco. La humanidad nunca podrá descubrir a la divinidad salvo a través del camino de la experiencia religiosa y mediante el ejercicio de la fe verdadera. La aceptación, por la fe, de la verdad de Dios, permite al hombre escapar de las fronteras circunscritas de las limitaciones materiales, y le proporciona una esperanza racional de conseguir un salvoconducto para salir del mundo material, donde existe la muerte, hacia el mundo espiritual, donde está la vida eterna.

\par
%\textsuperscript{(1116.6)}
\textsuperscript{101:10.5} La finalidad de la religión no es satisfacer la curiosidad sobre Dios, sino más bien proporcionar la constancia intelectual y la seguridad filosófica, estabilizar y enriquecer la vida humana mezclando lo mortal con lo divino, lo parcial con lo perfecto, el hombre y Dios. Es a través de la experiencia religiosa como los conceptos humanos de la idealidad son dotados de realidad.

\par
%\textsuperscript{(1116.7)}
\textsuperscript{101:10.6} Nunca podrá haber pruebas científicas o lógicas de la divinidad. La razón por sí sola nunca podrá validar los valores y las bondades de la experiencia religiosa. Pero siempre seguirá siendo cierto que cualquiera que desee hacer la voluntad de Dios comprenderá la validez de los valores espirituales. Ésta es la mayor aproximación que se puede efectuar en el nivel mortal en el sentido de ofrecer una prueba de la realidad de la experiencia religiosa. Una fe así proporciona la única manera de escapar de las garras mecánicas del mundo material y de las deformaciones causadas por los errores que se encuentran en el estado incompleto del mundo intelectual; es la única solución que se ha descubierto para salir del atolladero en que se encuentra el pensamiento mortal en lo que se refiere a la supervivencia continua de la personalidad individual. Es el único pasaporte para culminar la realidad y para la eternidad de vida en una creación universal de amor, ley, unidad y alcance progresivo de la Deidad.

\par
%\textsuperscript{(1117.1)}
\textsuperscript{101:10.7} La religión cura eficazmente el sentimiento humano de aislamiento idealista o de soledad espiritual; concede al creyente el derecho de hijo de Dios, de ciudadano de un universo nuevo y significativo. La religión le asegura al hombre que, cuando sigue el destello de rectitud discernible en su alma, se identifica de este modo con el plan del Infinito y el objetivo del Eterno. Un alma así liberada empieza a sentirse inmediatamente como en su casa en este nuevo universo, su universo.

\par
%\textsuperscript{(1117.2)}
\textsuperscript{101:10.8} Cuando experimentáis esta transformación por la fe, ya no sois una parte servil del cosmos matemático, sino más bien un hijo volitivo liberado del Padre Universal. Este hijo liberado ya no lucha solo contra el destino inexorable que pone fin a la existencia temporal; ya no combate contra toda la naturaleza, con las probabilidades totalmente en contra suya; ya no se tambalea debido al miedo paralizante de que quizás haya puesto su confianza en una ilusión sin esperanzas, o colocado su fe en un error de su fantasía.

\par
%\textsuperscript{(1117.3)}
\textsuperscript{101:10.9} Ahora, los hijos de Dios se han alistado juntos para librar la batalla del triunfo de la realidad sobre las sombras parciales de la existencia. Por fin todas las criaturas se vuelven conscientes del hecho de que Dios y todas las huestes divinas de un universo casi ilimitado están de su lado en la lucha celestial por alcanzar la vida eterna y el estado divino. Por supuesto, estos hijos liberados por la fe se han alistado en las luchas del tiempo al lado de las fuerzas supremas y de las personalidades divinas de la eternidad; incluso las estrellas en su trayectoria combaten ahora por ellos; por fin contemplan el universo desde dentro, desde el punto de vista de Dios, y las incertidumbres del aislamiento material se transforman en las certezas de la progresión espiritual eterna. Incluso el tiempo mismo se vuelve una mera sombra de la eternidad, proyectada por las realidades del Paraíso sobre la panoplia móvil del espacio.

\par
%\textsuperscript{(1117.4)}
\textsuperscript{101:10.10} [Presentado por un Melquisedek de Nebadon.]


\chapter{Documento 102. Los fundamentos de la fe religiosa}
\par
%\textsuperscript{(1118.1)}
\textsuperscript{102:0.1} PARA el materialista no creyente, el hombre es simplemente un accidente evolutivo. Sus esperanzas de supervivencia están engarzadas en una ficción de su imaginación como ser mortal; sus miedos, amores, anhelos y creencias no son más que la reacción de la yuxtaposición fortuita de ciertos átomos de materia sin vida. Ningún despliegue de energía y ninguna expresión de confianza pueden transportarlo más allá de la tumba. Las obras piadosas y el talento inspirador de los mejores hombres están condenados a perecer en la muerte, en esa larga noche solitaria del olvido eterno y de la extinción del alma. Una desesperación sin nombre es la única recompensa que recibe el hombre por vivir y trabajar sin descanso bajo el sol temporal de la existencia mortal. Cada día de la vida aprieta de manera lenta y segura el nudo de un destino despiadado que un universo de materia, hostil e implacable, ha decretado como insulto supremo para todo lo que es hermoso, noble, elevado y bueno en los deseos humanos\footnote{\textit{La muerte inevitable, no desesperéis}: Ec 1:1-8; 2:11-23.}.

\par
%\textsuperscript{(1118.2)}
\textsuperscript{102:0.2} Pero éste no es el fin ni el destino eterno del hombre; esta visión no es más que el grito de desesperación lanzado por un alma errante que se ha perdido en las tinieblas espirituales, y que continúa luchando valientemente en medio de los sofismas mecanicistas de una filosofía material cegada por la confusión y la deformación de una erudición compleja. Toda esta condena a las tinieblas y todo este destino de desesperación se disipan\footnote{\textit{La fe disipa la desesperación}: Ro 1:17.} para siempre mediante un valiente despliegue de fe por parte del hijo de Dios más humilde e inculto que viva en la Tierra.

\par
%\textsuperscript{(1118.3)}
\textsuperscript{102:0.3} Esta fe salvadora nace en el corazón humano cuando la conciencia moral del hombre se da cuenta de que, en la experiencia mortal, los valores humanos pueden ser trasladados de lo material a lo espiritual, de lo humano a lo divino, del tiempo a la eternidad.

\section*{1. Las seguridades de la fe}
\par
%\textsuperscript{(1118.4)}
\textsuperscript{102:1.1} El trabajo del Ajustador del Pensamiento explica la transformación del sentido primitivo y evolutivo del deber del hombre en una fe superior y más segura en las realidades eternas de la revelación. El corazón del hombre ha de tener hambre de perfección para que le asegure la capacidad de comprender los caminos de la fe que conducen al logro supremo. Si un hombre elige hacer la voluntad divina, conocerá el camino de la verdad. Es literalmente cierto que <<hay que conocer las cosas humanas para poder amarlas, pero hay que amar las cosas divinas para poder conocerlas>>\footnote{\textit{Hay que amar las cosas divinas}: 1 Jn 4:7-8.}. Las dudas honradas y las preguntas sinceras no son un pecado; estas actitudes representan simplemente un retraso en el viaje progresivo hacia el logro de la perfección. La confianza semejante a la de un niño\footnote{\textit{Tener la confianza de un niño}: Mt 18:3; Mc 10:15; Lc 18:17.} le asegura al hombre su entrada en el reino de la ascensión celestial, pero el progreso depende enteramente del ejercicio vigoroso de la fe robusta y convencida del hombre adulto.

\par
%\textsuperscript{(1119.1)}
\textsuperscript{102:1.2} La razón de la ciencia está basada en los hechos observables del tiempo; la fe de la religión presenta sus razonamientos basándose en el programa espiritual de la eternidad. Lo que el conocimiento y la razón no pueden hacer por nosotros, la verdadera sabiduría nos exhorta a que permitamos que la fe lo realice a través de la perspicacia religiosa y la transformación espiritual.

\par
%\textsuperscript{(1119.2)}
\textsuperscript{102:1.3} Debido al aislamiento causado por la rebelión, la revelación de la verdad en Urantia se ha mezclado demasiado a menudo con las declaraciones de cosmologías parciales y transitorias. La verdad permanece invariable de generación en generación, pero las enseñanzas que la acompañan concernientes al mundo físico varían de día en día y de año en año. La verdad eterna no debería ser despreciada porque se la encuentre por casualidad en compañía de ideas obsoletas sobre el mundo material. Cuanta más ciencia conocéis, menos seguros estáis; cuanto más religión \textit{poseéis}, más certidumbre tenéis.

\par
%\textsuperscript{(1119.3)}
\textsuperscript{102:1.4} Las certidumbres de la ciencia proceden totalmente del intelecto; las certezas de la religión se originan en los fundamentos mismos de la \textit{totalidad de la personalidad}. La ciencia apela a la comprensión de la mente; la religión apela a la lealtad y a la devoción del cuerpo, la mente y el espíritu, e incluso de toda la personalidad.

\par
%\textsuperscript{(1119.4)}
\textsuperscript{102:1.5} Dios es tan real y absoluto que no se puede ofrecer, como testimonio de su realidad, ningún signo material de prueba ni ninguna demostración de supuestos milagros. Siempre llegaremos a conocerlo porque confiamos en él, y nuestra creencia en él está totalmente basada en nuestra participación personal en las manifestaciones divinas de su realidad infinita.

\par
%\textsuperscript{(1119.5)}
\textsuperscript{102:1.6} El Ajustador del Pensamiento interior despierta infaliblemente en el alma humana una auténtica hambre de búsqueda de la perfección así como una enorme curiosidad, que sólo se pueden satisfacer adecuadamente mediante la comunión con Dios, la fuente divina de ese Ajustador. El alma hambrienta del hombre se niega a satisfacerse con cualquier otra cosa que sea inferior a la comprensión personal del Dios viviente. Aunque Dios pueda ser mucho más que una personalidad moral elevada y perfecta, en nuestro concepto hambriento y finito no puede ser nada menos.

\section*{2. La religión y la realidad}
\par
%\textsuperscript{(1119.6)}
\textsuperscript{102:2.1} Las mentes observadoras y las almas exigentes conocen la religión cuando la encuentran en la vida de sus semejantes. La religión no necesita ninguna definición; todos conocemos sus frutos sociales, intelectuales, morales y espirituales. Todo esto se deriva del hecho de que la religión es propiedad de la raza humana; no es un producto de la cultura. Es verdad que la percepción de la religión sigue siendo humana y que está sujeta por ello a la servidumbre de la ignorancia, a la esclavitud de la superstición, a los engaños de la sofisticación y a las ilusiones de las falsas filosofías.

\par
%\textsuperscript{(1119.7)}
\textsuperscript{102:2.2} Una de las peculiaridades características de la auténtica seguridad religiosa consiste en que, a pesar del carácter absoluto de sus afirmaciones y de la firmeza de su actitud, el espíritu de su expresión es tan equilibrado y templado que nunca transmite la menor impresión de presunción o de exaltación egoísta. La sabiduría de la experiencia religiosa es en cierto modo una paradoja, ya que es de origen humano y procede al mismo tiempo del Ajustador. La fuerza religiosa no es producto de las prerrogativas personales del individuo, sino más bien la manifestación de la asociación sublime entre el hombre y la fuente eterna de toda sabiduría. Así es como las palabras y los actos de la religión verdadera y no contaminada poseen una autoridad irresistible para todos los mortales iluminados.

\par
%\textsuperscript{(1119.8)}
\textsuperscript{102:2.3} Es difícil identificar y analizar los factores de una experiencia religiosa, pero no es difícil observar que los practicantes religiosos viven y se comportan como si ya estuvieran en presencia del Eterno. Los creyentes reaccionan ante esta vida temporal como si la inmortalidad estuviera ya al alcance de sus manos. En la vida de estos mortales se puede observar una originalidad válida y una espontaneidad de expresión que los separa para siempre de aquellos semejantes suyos que sólo se han impregnado de la sabiduría del mundo. Las personas religiosas parecen vivir eficazmente liberadas del acoso de la prisa y de la tensión dolorosa de las vicisitudes inherentes a las corrientes transitorias del tiempo; manifiestan una estabilidad en su personalidad y una tranquilidad de carácter que las leyes de la fisiología, la psicología y la sociología no pueden explicar.

\par
%\textsuperscript{(1120.1)}
\textsuperscript{102:2.4} El tiempo es un elemento invariable para adquirir el conocimiento; la religión hace que sus dones sean inmediatamente asequibles, aunque existe el factor importante del crecimiento en la gracia, de un progreso preciso en todas las fases de la experiencia religiosa. El conocimiento es una búsqueda eterna; siempre estaréis aprendiendo, pero nunca seréis capaces de llegar al conocimiento completo de la verdad absoluta. El conocimiento por sí solo nunca puede proporcionar una certeza absoluta, sino únicamente una probabilidad aproximada creciente; pero el alma religiosa espiritualmente iluminada \textit{sabe}, y sabe \textit{ahora}\footnote{\textit{El alma sabe, y sabe ahora}: Sal 139:14.}. Y sin embargo, esta certidumbre profunda y positiva no conduce a esta persona religiosa mentalmente sana a interesarse menos por los altibajos del progreso de la sabiduría humana, la cual está unida en sus objetivos materiales a los desarrollos de una ciencia que avanza lentamente.

\par
%\textsuperscript{(1120.2)}
\textsuperscript{102:2.5} Incluso los descubrimientos de la ciencia no son verdaderamente \textit{reales} en la conciencia de la experiencia humana hasta que no son desenmarañados y correlacionados, hasta que sus hechos pertinentes no tienen un \textit{significado} efectivo gracias a su inclusión en las corrientes de pensamiento de la mente. El hombre mortal percibe incluso su entorno físico desde el nivel mental, desde la perspectiva de su registro psicológico. Por eso no es de extrañar que el hombre interprete el universo de una manera extremadamente unificada, y luego intente identificar esta unidad energética de su ciencia con la unidad espiritual de su experiencia religiosa. La mente es unidad; la conciencia mortal vive en el nivel mental y percibe las realidades universales a través de los ojos de la dotación mental. La perspectiva mental no proporcionará la unidad existencial de la fuente de la realidad, la Fuente-Centro Primera, pero puede presentar, y alguna vez presentará al hombre, la síntesis experiencial de la energía, la mente y el espíritu en el Ser Supremo y como Ser Supremo. Pero la mente nunca podrá conseguir esta unificación de la diversidad de la realidad, a menos que dicha mente sea firmemente consciente de las cosas materiales, los significados intelectuales y los valores espirituales; sólo existe unidad en la armonía de la trinidad de la realidad funcional, y la satisfacción que proporciona a la personalidad la comprensión de la constancia y de la coherencia cósmicas sólo se hallan en la unidad.

\par
%\textsuperscript{(1120.3)}
\textsuperscript{102:2.6} En la experiencia humana, la unidad se encuentra mejor a través de la filosofía. Y aunque el conjunto del pensamiento filosófico debe estar basado siempre en los hechos materiales, la perspicacia espiritual humana es el alma y la energía de la verdadera dinámica filosófica.

\par
%\textsuperscript{(1120.4)}
\textsuperscript{102:2.7} Al hombre evolutivo no le entusiasma por naturaleza el trabajo duro. En la experiencia de su vida, para mantenerse al mismo ritmo que las exigencias impelentes y los impulsos irresistibles de una experiencia religiosa creciente, necesita tener una actividad incesante en el crecimiento espiritual, la expansión intelectual, el desarrollo basado en los hechos y el servicio social. No existe ninguna verdadera religión sin una personalidad extremadamente activa. Por eso los hombres más indolentes intentan a menudo evitar los rigores de las actividades verdaderamente religiosas mediante una especie de autoengaño ingenioso, recurriendo a retirarse al falso refugio de las doctrinas y de los dogmas religiosos estereotipados. Pero la verdadera religión está viva. La cristalización intelectual de los conceptos religiosos equivale a la muerte espiritual. No podéis concebir una religión sin ideas, pero una vez que la religión se reduce únicamente a una \textit{idea}, ya no es una religión; se ha convertido simplemente en una especie de filosofía humana.

\par
%\textsuperscript{(1121.1)}
\textsuperscript{102:2.8} Además, existen otros tipos de almas inestables y mal disciplinadas que suelen utilizar las ideas sentimentales de la religión como camino para eludir las exigencias enojosas de la vida. Cuando ciertos mortales vacilantes y asustadizos intentan escapar de la presión incesante de la vida evolutiva, la religión, tal como ellos la conciben, parece ofrecerles el refugio más cercano, la mejor escapatoria. Pero la religión tiene la misión de preparar al hombre para enfrentarse de manera valiente, e incluso heroica, a las vicisitudes de la vida. La religión es el don supremo del hombre evolutivo, la única cosa que le permite seguir adelante y <<aguantar como si viera a Aquel que es invisible>>\footnote{\textit{Vivir como si se viera al Invisible}: Heb 11:27.}. Sin embargo, el misticismo es a menudo una especie de retirada de la vida, siendo abrazado por aquellos humanos que no disfrutan con las actividades más vigorosas de una vida religiosa vivida en las esferas abiertas de la sociedad y del comercio humanos. La verdadera religión debe \textit{actuar}. El comportamiento es una consecuencia de la religión cuando el hombre tiene realmente una, o más bien cuando el hombre permite que la religión lo posea verdaderamente. La religión nunca se sentirá satisfecha con unos simples pensamientos o con unos sentimientos pasivos.

\par
%\textsuperscript{(1121.2)}
\textsuperscript{102:2.9} No ignoramos el hecho de que la religión actúa a menudo de manera insensata e incluso irreligiosa, pero \textit{actúa}. Las aberraciones de algunas convicciones religiosas han conducido a persecuciones sangrientas, pero la religión siempre hace algo; ¡es dinámica!

\section*{3. El conocimiento, la sabiduría y la perspicacia}
\par
%\textsuperscript{(1121.3)}
\textsuperscript{102:3.1} Las deficiencias intelectuales o las carencias educativas obstaculizan inevitablemente los logros religiosos más elevados, porque un entorno de naturaleza espiritual tan empobrecido le roba a la religión su canal principal de contacto filosófico con el mundo de los conocimientos científicos. Los factores intelectuales de la religión son importantes, pero a veces su desarrollo excesivo es del mismo modo muy perjudicial y embarazoso. La religión debe trabajar continuamente bajo una necesidad paradójica: la necesidad de emplear eficazmente el pensamiento, y al mismo tiempo no hacer caso de la utilidad espiritual de todo pensamiento.

\par
%\textsuperscript{(1121.4)}
\textsuperscript{102:3.2} Las especulaciones religiosas son inevitables, pero siempre son perjudiciales; la especulación desvirtúa invariablemente su objeto. La especulación tiende a transformar la religión en algo material o humanista, y así, a la vez que interfiere directamente con la claridad del pensamiento lógico, hace indirectamente que la religión aparezca como una función del mundo temporal, del mundo mismo con el que debería estar en eterna contraposición. Por consiguiente, la religión siempre estará caracterizada por las paradojas, las paradojas ocasionadas por la ausencia de conexión experiencial entre el nivel material y el nivel espiritual del universo ---de la mota morontial, la sensibilidad superfilosófica que permite discernir la verdad y percibir la unidad.

\par
%\textsuperscript{(1121.5)}
\textsuperscript{102:3.3} Los sentimientos materiales, las emociones humanas, conducen directamente a las acciones materiales, a los actos egoístas. La perspicacia religiosa, las motivaciones espirituales, conducen directamente a las acciones religiosas, a los actos desinteresados de servicio social y de generosidad altruista.

\par
%\textsuperscript{(1121.6)}
\textsuperscript{102:3.4} El deseo religioso es la búsqueda ávida de la realidad divina. La experiencia religiosa es tener conciencia de haber encontrado a Dios. Y cuando un ser humano encuentra a Dios, el alma de ese ser experimenta tal agitación indescriptible por el triunfo de su descubrimiento, que se ve impulsado a buscar un contacto de servicio afectuoso con sus semejantes menos iluminados, no para revelar que ha encontrado a Dios, sino más bien para permitir que el desbordamiento de bondad eterna que brota de su propia alma refresque y ennoblezca a sus semejantes. La auténtica religión conduce a un servicio social cada vez mayor.

\par
%\textsuperscript{(1122.1)}
\textsuperscript{102:3.5} La ciencia, el conocimiento, conduce a la conciencia de los \textit{hechos}; la religión, la experiencia, conduce a la conciencia de los \textit{valores}; la filosofía, la sabiduría, conduce a la conciencia \textit{coordinada}; la revelación (la sustituta de la mota morontial) conduce a la conciencia de la \textit{verdadera realidad}; mientras que la coordinación de la conciencia de los hechos, los valores y la verdadera realidad constituye el tener conciencia de la realidad de la personalidad, lo máximo del ser, junto con la creencia en la posibilidad de la supervivencia de esta misma personalidad.

\par
%\textsuperscript{(1122.2)}
\textsuperscript{102:3.6} El conocimiento conduce a situar a los hombres, a originar las capas y las castas sociales. La religión conduce a servir a los hombres, creando así la ética y el altruismo. La sabiduría conduce a una asociación mejor y más elevada tanto de las ideas como con los semejantes. La revelación libera a los hombres y los pone en camino hacia la aventura eterna.

\par
%\textsuperscript{(1122.3)}
\textsuperscript{102:3.7} La ciencia clasifica a los hombres; la religión ama a los hombres, incluso como a vosotros mismos; la sabiduría hace justicia a los distintos hombres; pero la revelación glorifica al hombre y revela su capacidad para asociarse con Dios.

\par
%\textsuperscript{(1122.4)}
\textsuperscript{102:3.8} La ciencia se esfuerza en vano por crear la fraternidad de la cultura; la religión engendra la fraternidad del espíritu. La filosofía lucha por la fraternidad de la sabiduría; la revelación describe la fraternidad eterna, el Cuerpo Paradisiaco de la Finalidad.

\par
%\textsuperscript{(1122.5)}
\textsuperscript{102:3.9} El conocimiento produce orgullo en el hecho de la personalidad; la sabiduría es la conciencia del significado de la personalidad; la religión es la experiencia del conocimiento del valor de la personalidad; la revelación es la seguridad de la supervivencia de la personalidad.

\par
%\textsuperscript{(1122.6)}
\textsuperscript{102:3.10} La ciencia trata de identificar, analizar y clasificar las partes segmentadas del cosmos ilimitado. La religión capta la idea del todo, el cosmos total. La filosofía intenta identificar los segmentos materiales de la ciencia con el concepto del todo basado en la perspicacia espiritual del todo. Allí donde la filosofía fracasa en este intento, la revelación tiene éxito, afirmando que el círculo cósmico es universal, eterno, absoluto e infinito. Este cosmos del Infinito YO SOY\footnote{\textit{El Infinito YO SOY}: Ex 3:13-14.} es por tanto interminable, ilimitado, y lo incluye todo ---sin tiempo, sin espacio e incalificado. Y atestiguamos que el Infinito YO SOY es también el Padre de Miguel de Nebadon y el Dios de la salvación humana.

\par
%\textsuperscript{(1122.7)}
\textsuperscript{102:3.11} La ciencia alude a la Deidad como un \textit{hecho}; la filosofía presenta la \textit{idea} de un Absoluto; la religión presenta la imagen de Dios como una \textit{personalidadespiritual} amorosa. La revelación afirma que existe \textit{unidad} entre el hecho de la Deidad, la idea del Absoluto y la personalidad espiritual de Dios; y además presenta este concepto bajo la forma de nuestro Padre ---el hecho universal de la existencia, la idea eterna de la mente y el espíritu infinito de la vida.

\par
%\textsuperscript{(1122.8)}
\textsuperscript{102:3.12} La persecución del conocimiento constituye la ciencia; la búsqueda de la sabiduría es la filosofía; el amor a Dios es la religión; el hambre de la verdad \textit{es} una revelación. Pero el Ajustador del Pensamiento interior es el que conecta el sentimiento de la realidad con la perspicacia espiritual humana del cosmos.

\par
%\textsuperscript{(1122.9)}
\textsuperscript{102:3.13} En la ciencia, la idea precede a la expresión de su realización; en la religión, la experiencia de la realización precede a la expresión de la idea. Existe una inmensa diferencia entre la voluntad evolutiva de creer y el producto de la razón iluminada, la perspicacia religiosa y la revelación ---la \textit{voluntad que cree}.

\par
%\textsuperscript{(1122.10)}
\textsuperscript{102:3.14} En la evolución, la religión conduce con frecuencia al hombre a crear sus conceptos de Dios; la revelación manifiesta el fenómeno de Dios haciendo evolucionar al hombre mismo, mientras que en la vida terrestre de Cristo Miguel contemplamos el fenómeno de Dios revelándose al hombre. La evolución tiende a hacer a Dios semejante al hombre; la revelación tiende a hacer al hombre semejante a Dios.

\par
%\textsuperscript{(1122.11)}
\textsuperscript{102:3.15} La ciencia sólo se satisface con las causas primeras, la religión con la personalidad suprema, y la filosofía con la unidad. La revelación afirma que las tres son una sola, y que todas son buenas. Lo \textit{real eterno} es el bien del universo, y no las ilusiones temporales del mal espacial. En la experiencia espiritual de todas las personalidades, siempre es cierto que lo real es el bien y que el bien es lo real.

\section*{4. El hecho de la experiencia}
\par
%\textsuperscript{(1123.1)}
\textsuperscript{102:4.1} Debido a la presencia del Ajustador del Pensamiento en vuestra mente, para vosotros no es más misterioso conocer la mente de Dios que estar seguros de que tenéis conciencia de conocer cualquier otra mente, humana o superhumana. La religión y la conciencia social tienen esto en común: están basadas en la conciencia de que existen otras mentes. La técnica que utilizáis para aceptar como vuestra la idea de otra persona, es la misma que podéis emplear para <<dejar que la mente que estaba en Cristo esté también en vosotros>>\footnote{\textit{Dejar la mente de Cristo estar en vosotros}: 1 Co 2:16; Flp 2:5.}.

\par
%\textsuperscript{(1123.2)}
\textsuperscript{102:4.2} ¿Qué es la experiencia humana? Es simplemente cualquier interacción entre un yo activo e inquisitivo y cualquier otra realidad activa y externa. La cantidad de experiencia está determinada por la profundidad de los conceptos más la totalidad del reconocimiento de la realidad de lo exterior. El movimiento de la experiencia es igual a la fuerza de la imaginación expectante más la agudeza del descubrimiento sensorial de las cualidades externas de la realidad contactada. El hecho de la experiencia se encuentra en la conciencia de sí mismo y de que hay otras existencias ---otras cosas, otras mentes y otros espíritus.

\par
%\textsuperscript{(1123.3)}
\textsuperscript{102:4.3} El hombre se vuelve muy pronto consciente de que no está solo en el mundo o en el universo. Se desarrolla una conciencia natural y espontánea de que existen otras mentes en el entorno del individuo. La fe transforma esta experiencia natural en religión, en el reconocimiento de Dios como realidad ---como fuente, naturaleza y destino--- de las \textit{otras mentes}. Pero este conocimiento de Dios siempre es una realidad de la experiencia personal. Si Dios no fuera una personalidad, no podría convertirse en una parte viviente de la experiencia religiosa real de una personalidad humana.

\par
%\textsuperscript{(1123.4)}
\textsuperscript{102:4.4} El elemento de error presente en la experiencia religiosa humana es directamente proporcional al contenido de materialismo que contamina el concepto espiritual del Padre Universal. La progresión pre-espiritual del hombre en el universo consiste en la experiencia de despojarse de estas ideas erróneas sobre la naturaleza de Dios y sobre la realidad del espíritu puro y verdadero. La Deidad es más que espíritu, pero el acercamiento espiritual es el único posible para el hombre ascendente.

\par
%\textsuperscript{(1123.5)}
\textsuperscript{102:4.5} La oración es en verdad una parte de la experiencia religiosa, pero las religiones modernas han hecho hincapié erróneamente en ella, descuidando en gran parte la comunión más esencial de la adoración. La adoración intensifica y amplía los poderes reflexivos de la mente. La oración puede enriquecer la vida, pero la adoración ilumina el destino.

\par
%\textsuperscript{(1123.6)}
\textsuperscript{102:4.6} La religión revelada es el elemento unificador de la existencia humana. La revelación unifica la historia, coordina la geología, la astronomía, la física, la química, la biología, la sociología y la psicología. La experiencia espiritual es la verdadera alma del cosmos del hombre.

\section*{5. La supremacía del potencial intencional}
\par
%\textsuperscript{(1123.7)}
\textsuperscript{102:5.1} Aunque el establecimiento del hecho de la creencia no equivale a establecer el hecho de aquello en lo que se cree, sin embargo, la progresión evolutiva desde las formas simples de vida hasta el estado de la personalidad demuestra bien el hecho de la existencia, desde un principio, del potencial de la personalidad. Y en los universos del tiempo, lo potencial siempre es supremo con respecto a lo manifestado. En el cosmos evolutivo, lo potencial es lo que va a ser, y lo que va a ser es el desarrollo de los mandatos deliberados de la Deidad.

\par
%\textsuperscript{(1124.1)}
\textsuperscript{102:5.2} Esta misma supremacía intencional está expresada en la evolución de la ideación mental cuando el miedo animal primitivo se transmuta en una veneración constantemente más profunda hacia Dios y en un temor creciente hacia el universo. El hombre primitivo tenía más miedo religioso que fe, y la supremacía de los potenciales espirituales sobre los actuales mentales queda demostrada cuando este miedo cobarde se transforma en una fe viviente en las realidades espirituales.

\par
%\textsuperscript{(1124.2)}
\textsuperscript{102:5.3} Podéis interpretar psicológicamente la religión evolutiva, pero no la religión de origen espiritual basada en la experiencia personal. La moralidad humana puede reconocer los valores, pero sólo la religión puede conservar, ensalzar y espiritualizar esos valores. Pero a pesar de estas acciones, la religión es algo más que una moralidad basada en las emociones. La religión es a la moral lo que el amor es al deber, lo que la filiación es a la servidumbre, lo que la esencia es a la sustancia. La moralidad revela a un Controlador todopoderoso, a una Deidad a quien servir; la religión revela a un Padre lleno de amor, a un Dios a quien adorar y amar. Y esto se debe una vez más a que el potencial espiritual de la religión domina a la moralidad evolutiva basada en el sentido del deber.

\section*{6. La certidumbre de la fe religiosa}
\par
%\textsuperscript{(1124.3)}
\textsuperscript{102:6.1} La eliminación filosófica del miedo religioso y el progreso continuo de la ciencia aumentan enormemente la mortandad de los falsos dioses; y aunque esta desaparición de las deidades creadas por los hombres pueda nublar momentáneamente la visión espiritual, termina por destruir la ignorancia y la superstición que tanto tiempo ocultaron al Dios viviente del amor eterno. La relación entre la criatura y el Creador es una experiencia viviente, una fe religiosa dinámica, que no está sujeta a una definición precisa. Aislar una parte de la vida y llamarla religión es desintegrar la vida y desvirtuar la religión. Ésta es precisamente la razón por la que el Dios de la adoración exige una fidelidad total, o ninguna.

\par
%\textsuperscript{(1124.4)}
\textsuperscript{102:6.2} Los dioses de los hombres primitivos puede que no fueran más que las sombras de aquellos mismos hombres; el Dios viviente es la luz divina cuyas interrupciones forman las sombras de la creación en todo el espacio.

\par
%\textsuperscript{(1124.5)}
\textsuperscript{102:6.3} La persona religiosa con alcance filosófico tiene fe en un Dios personal de salvación personal, en algo más que una realidad, un valor, un nivel de consecución, un proceso elevado, una trasmutación, el último del espacio-tiempo, una idealización, la personificación de la energía, la entidad de la gravedad, una proyección humana, la idealización del yo, el ensalzamiento de la naturaleza, la tendencia a la bondad, el impulso hacia adelante de la evolución, o una hipótesis sublime. La persona religiosa tiene fe en un Dios de amor\footnote{\textit{La persona religiosa tiene fe en un Dios de amor}: 1 Co 13:1-13.}. El amor es la esencia de la religión y el manantial de las civilizaciones superiores.

\par
%\textsuperscript{(1124.6)}
\textsuperscript{102:6.4} La fe transforma al Dios filosófico de la probabilidad en el Dios salvador de la seguridad en la experiencia religiosa personal. El escepticismo puede desafiar las teorías de la teología, pero la confianza en la fiabilidad de la experiencia personal afirma la verdad de esa creencia que se ha convertido en fe.

\par
%\textsuperscript{(1124.7)}
\textsuperscript{102:6.5} Se puede llegar a convicciones sobre Dios a través de un sabio razonamiento, pero el individuo sólo llega a conocer a Dios por medio de la fe, a través de la experiencia personal. Hay que contar con las probabilidades en muchas cosas relacionadas con la vida, pero se puede experimentar la certeza cuando, al contactar con la realidad cósmica, uno se acerca a esos significados y valores por medio de la fe viviente. El alma que conoce a Dios se atreve a decir <<yo sé>>, incluso cuando este conocimiento de Dios es puesto en duda por el no creyente, que niega esta certeza porque no está totalmente respaldada por la lógica intelectual. El creyente se limita a contestar a todos estos escépticos: <<¿Cómo sabes que yo no sé?>>.

\par
%\textsuperscript{(1125.1)}
\textsuperscript{102:6.6} Aunque la razón siempre puede dudar de la fe, la fe puede siempre complementar tanto a la razón como a la lógica. La razón crea esa probabilidad que la fe puede transformar en una certeza moral, e incluso en una experiencia espiritual. Dios es la primera verdad y el último hecho; por eso toda verdad tiene su origen en él, mientras que todos los hechos existen en relación con él. Dios es la verdad absoluta. Uno puede conocer a Dios bajo la forma de verdad, pero para comprender a Dios ---para explicarlo--- hay que explorar el hecho del universo de universos. El inmenso abismo que existe entre la experiencia de la verdad de Dios y la ignorancia del hecho de Dios sólo se puede colmar mediante la fe viviente. La razón sola no puede llevar a cabo la armonía entre la verdad infinita y los hechos universales.

\par
%\textsuperscript{(1125.2)}
\textsuperscript{102:6.7} La creencia puede ser incapaz de resistir a la duda y de soportar el miedo, pero la fe siempre triunfa sobre la duda, porque la fe es a la vez positiva y viviente. Lo positivo siempre tiene ventaja sobre lo negativo, la verdad sobre el error, la experiencia sobre la teoría, las realidades espirituales sobre los hechos aislados del tiempo y del espacio. La prueba convincente de esta certeza espiritual consiste en los frutos sociales del espíritu que estos creyentes, las personas con fe, producen como resultado de esta experiencia espiritual auténtica. Jesús dijo: <<Si amáis a vuestros semejantes como yo os he amado, entonces todos los hombres sabrán que sois mis discípulos>>\footnote{\textit{Amad a vuestros semejantes como yo os amo}: Jn 13:34-35; 15:12.}.

\par
%\textsuperscript{(1125.3)}
\textsuperscript{102:6.8} Para la ciencia, Dios es una posibilidad; para la psicología, una cosa deseable; para la filosofía, una probabilidad; para la religión, una certeza, una realidad de la experiencia religiosa. La razón exige que una filosofía que no puede encontrar al Dios de la probabilidad debería ser muy respetuosa con esa fe religiosa que puede, y encuentra, al Dios de la certidumbre. La ciencia tampoco debería descartar la experiencia religiosa por motivos de credulidad, al menos mientras se aferre a la suposición de que los dones intelectuales y filosóficos del hombre surgieron de unas inteligencias cada vez menores a medida que se alejan más en el pasado, teniendo finalmente su origen en la vida primitiva que estaba totalmente desprovista de todo pensamiento y de todo sentimiento.

\par
%\textsuperscript{(1125.4)}
\textsuperscript{102:6.9} Los hechos de la evolución no se deben utilizar en contra de la verdad de que la experiencia espiritual de la vida religiosa de un mortal que conoce a Dios es realmente una certeza. Los hombres inteligentes deberían dejar de razonar como niños e intentar utilizar la lógica coherente de los adultos ---la lógica que tolera el concepto de la verdad al lado de la observación de los hechos. El materialismo científico se declara en quiebra cuando, en presencia de cada fenómeno universal recurrente, se empeña en consolidar sus objeciones habituales achacando aquello que está admitido como superior a aquello que está admitido como inferior. La coherencia exige que se reconozcan las actividades de un Creador intencional.

\par
%\textsuperscript{(1125.5)}
\textsuperscript{102:6.10} La evolución orgánica es un hecho; la evolución intencional o progresiva es una verdad que vuelve coherentes los fenómenos, de otra manera contradictorios, de los logros siempre ascendentes de la evolución. Cuanto más progresa un científico en la ciencia que ha escogido, más abandona las teorías de los hechos materialistas a favor de la verdad cósmica del predominio de la Mente Suprema. El materialismo degrada la vida humana; el evangelio de Jesús realza enormemente a todos los mortales y los eleva de manera celestial. Hay que imaginar que la existencia mortal consiste en la experiencia misteriosa y fascinante de llevar a cabo la realidad del encuentro entre el ser humano que tiende su mano hacia arriba y la divinidad que tiende su mano salvadora hacia abajo.

\section*{7. La certidumbre de lo divino}
\par
%\textsuperscript{(1126.1)}
\textsuperscript{102:7.1} Puesto que el Padre Universal existe por sí mismo, también se explica por sí mismo; vive realmente en todo mortal racional. Pero no podéis estar seguros de Dios a menos que lo conozcáis; la filiación es la única experiencia que asegura la paternidad. El universo está sufriendo cambios por todas partes. Un universo que cambia es un universo dependiente; una creación así no puede ser final ni absoluta. Un universo finito depende totalmente del Último y del Absoluto. El universo y Dios no son idénticos; uno es la causa y el otro el efecto. La causa es absoluta, infinita, eterna e invariable; el efecto es espacio-temporal y trascendental, pero siempre cambiante, siempre en crecimiento.

\par
%\textsuperscript{(1126.2)}
\textsuperscript{102:7.2} Dios es el solo y único hecho en el universo causado por sí mismo. Él es el secreto del orden, del plan y de la finalidad de toda la creación de cosas y de seres. El universo que cambia por todas partes está regulado y estabilizado por unas leyes absolutamente invariables, los hábitos de un Dios invariable. El hecho de Dios, la ley divina, no cambia; la verdad de Dios, su relación con el universo, es una revelación relativa que siempre es adaptable al universo en constante evolución.

\par
%\textsuperscript{(1126.3)}
\textsuperscript{102:7.3} Aquellos que desearían inventar una religión sin Dios se parecen a los que quisieran cosechar frutos sin árboles, o tener hijos sin padres. No se pueden obtener efectos sin causas; sólo el YO SOY carece de causa. El hecho de la experiencia religiosa implica un Dios, y este Dios de la experiencia personal debe ser una Deidad personal. No podéis orar a una fórmula química, suplicar a una ecuación matemática, adorar a una hipótesis, confiar en un postulado, comulgar con un proceso, servir a una abstracción o mantener una camaradería afectuosa con una ley.

\par
%\textsuperscript{(1126.4)}
\textsuperscript{102:7.4} Es verdad que muchas características aparentemente religiosas pueden tener su origen en raíces no religiosas. Un hombre puede negar a Dios intelectualmente y, sin embargo, ser moralmente bueno, leal, filial, honrado e incluso idealista. El hombre puede injertar muchas ramas puramente humanistas en su naturaleza espiritual básica, y probar así aparentemente sus opiniones a favor de una religión sin Dios, pero esta experiencia está desprovista de valores de supervivencia, de conocimiento de Dios y de ascensión hacia Dios. En una experiencia humana de este tipo sólo se producen frutos sociales, no espirituales. El injerto determina la naturaleza del fruto, a pesar de que el alimento viviente se extraiga de las raíces de la dotación divina original tanto mental como espiritual.

\par
%\textsuperscript{(1126.5)}
\textsuperscript{102:7.5} La marca distintiva intelectual de la religión es la certeza; su característica filosófica es la coherencia; sus frutos sociales son el amor y el servicio\footnote{\textit{Frutos sociales del espíritu}: Gl 5:22-23; Ef 5:9.}.

\par
%\textsuperscript{(1126.6)}
\textsuperscript{102:7.6} La persona que conoce a Dios no es alguien que no vea las dificultades o que no piense en los obstáculos que se alzan en el camino para encontrar a Dios en el laberinto de las supersticiones, las tradiciones y las tendencias materialistas de los tiempos modernos. Ha encontrado todos esos frenos y ha triunfado sobre ellos, los ha superado mediante la fe viviente, y ha alcanzado las tierras altas de la experiencia espiritual a pesar de ellos. Pero es cierto que muchas personas interiormente seguras de Dios temen afirmar estos sentimientos de certeza a causa de la multiplicidad y la habilidad de aquellos que acumulan objeciones y exageran las dificultades sobre el hecho de creer en Dios. No se necesita una gran profundidad intelectual para encontrar fallos, hacer preguntas o poner objeciones. Pero sí hace falta una mente brillante para contestar esas preguntas y resolver esas dificultades; la certeza de la fe es la mejor técnica para tratar todas esas opiniones superficiales.

\par
%\textsuperscript{(1127.1)}
\textsuperscript{102:7.7} Si la ciencia, la filosofía o la sociología se atreven a volverse dogmáticas en su enfrentamiento con los profetas de la verdadera religión, entonces los hombres que conocen a Dios deberían replicar a ese dogmatismo injustificado con el dogmatismo más clarividente de la certeza de la experiencia espiritual personal: <<Sé lo que he experimentado porque soy un hijo del YO SOY>>. Si la experiencia personal de una persona que tiene fe es puesta en duda por un dogma, entonces ese hijo del Padre experimentable, nacido por la fe, puede contestar con este dogma indiscutible, la declaración de su filiación real con el Padre Universal.

\par
%\textsuperscript{(1127.2)}
\textsuperscript{102:7.8} Sólo una realidad incalificada, un absoluto, puede atreverse a ser coherentemente dogmática. Aquellos que pretenden ser dogmáticos, si son coherentes, deben ser conducidos tarde o temprano a los brazos del Absoluto de la energía, del Universal de la verdad, y del Infinito del amor.

\par
%\textsuperscript{(1127.3)}
\textsuperscript{102:7.9} Si los enfoques no religiosos de la realidad cósmica se atreven a poner en duda la certidumbre de la fe a causa de su estado no demostrado, entonces aquel que experimenta el espíritu puede recurrir también a poner dogmáticamente en tela de juicio los hechos de la ciencia y las creencias de la filosofía por las razones de que éstos tampoco están demostrados, ya que se trata igualmente de unas experiencias que tienen lugar en la conciencia del científico o del filósofo.

\par
%\textsuperscript{(1127.4)}
\textsuperscript{102:7.10} Dios es la más ineludible de todas las presencias, el más real de todos los hechos, la más viva de todas las verdades, el más afectuoso de todos los amigos y el más divino de todos los valores; de Dios tenemos derecho a estar más seguros que de cualquier otra experiencia universal.

\section*{8. Las pruebas de la religión}
\par
%\textsuperscript{(1127.5)}
\textsuperscript{102:8.1} La mejor prueba de la realidad y de la eficacia de la religión consiste en el \textit{hecho de la experiencia humana}; a saber, que el hombre, temeroso y desconfiado por naturaleza, dotado de forma innata de un fuerte instinto de conservación y anhelando sobrevivir después de la muerte, está dispuesto a confiar plenamente los intereses más profundos de su presente y de su futuro al cuidado y a la dirección de ese poder y de esa persona que su fe designa como Dios. Ésta es la única verdad central de toda religión. En cuanto a lo que ese poder o esa persona exige al hombre a cambio de este cuidado y de esta salvación final, no existen dos religiones que estén de acuerdo; de hecho, todas están más o menos en desacuerdo.

\par
%\textsuperscript{(1127.6)}
\textsuperscript{102:8.2} En lo que se refiere a la situación de cualquier religión en la escala evolutiva, la mejor manera de considerarla es por sus juicios morales y sus normas éticas. Cuanto más elevada es la naturaleza de cualquier religión, más alienta una moralidad social y una cultura ética en constante progreso, y más alentada es por ellas. No podemos juzgar a una religión por el estado de la civilización que la acompaña; es mejor que apreciemos la verdadera naturaleza de una civilización por la pureza y la nobleza de su religión. Muchos de los educadores religiosos más notables del mundo fueron prácticamente incultos. La sabiduría del mundo no es necesaria para ejercer una fe salvadora en las realidades eternas.

\par
%\textsuperscript{(1127.7)}
\textsuperscript{102:8.3} La diferencia entre las religiones de las diversas épocas depende totalmente de la manera diferente en que los hombres comprenden la realidad, y de la forma distinta en que reconocen los valores morales, las relaciones éticas y las realidades espirituales.

\par
%\textsuperscript{(1127.8)}
\textsuperscript{102:8.4} La ética es el eterno espejo social o racial que refleja fielmente el progreso, por otra parte inobservable, de los desarrollos espirituales y religiosos internos. El hombre siempre ha pensado en Dios en función de lo mejor que conocía, de sus ideas más profundas y de sus ideales más elevados. Incluso la religión histórica siempre ha creado sus conceptos de Dios a partir de sus valores reconocidos más elevados. Toda criatura inteligente da el nombre de Dios al ser más elevado y mejor que conoce.

\par
%\textsuperscript{(1128.1)}
\textsuperscript{102:8.5} Cuando la religión ha quedado reducida a los términos de la razón y de la expresión intelectual, siempre se ha atrevido a criticar la civilización y el progreso evolutivo, juzgándolos con sus propios criterios sobre la cultura ética y el progreso moral.

\par
%\textsuperscript{(1128.2)}
\textsuperscript{102:8.6} Aunque la religión personal precede a la evolución de la moral humana, hay que indicar lamentablemente que la religión institucional se ha quedado invariablemente rezagada detrás de las costumbres lentamente cambiantes de las razas humanas. La religión organizada ha demostrado ser conservadoramente lenta. Los profetas han conducido generalmente a los pueblos hacia un desarrollo religioso; los teólogos habitualmente los han frenado. Puesto que la religión es un asunto de experiencia interior o personal, nunca puede desarrollarse con mucha anticipación sobre la evolución intelectual de las razas.

\par
%\textsuperscript{(1128.3)}
\textsuperscript{102:8.7} Pero la religión nunca es realzada cuando se recurre a los pretendidos milagros. La búsqueda de los milagros es un retroceso a las religiones primitivas de la magia. La verdadera religión no tiene nada que ver con los supuestos milagros, y la religión revelada nunca se apoya en los milagros como prueba de su autoridad. La religión está siempre arraigada y basada en la experiencia personal. Y vuestra religión más elevada, la vida de Jesús, fue precisamente una experiencia personal de este tipo: el hombre, el hombre mortal, buscando a Dios y encontrándolo plenamente en el transcurso de una corta vida en la carne, mientras que en esta misma experiencia humana Dios se manifestó buscando al hombre y encontrándolo, para la plena satisfacción del alma perfecta de la supremacía infinita. Esto es la religión, la más elevada que se haya revelado hasta ahora en el universo de Nebadon ---la vida terrestre de Jesús de Nazaret.

\par
%\textsuperscript{(1128.4)}
\textsuperscript{102:8.8} [Presentado por un Melquisedek de Nebadon.]


\chapter{Documento 103. La realidad de la experiencia religiosa}
\par
%\textsuperscript{(1129.1)}
\textsuperscript{103:0.1} TODAS las reacciones verdaderamente religiosas del hombre están patrocinadas por el ministerio inicial del ayudante de la adoración, y censuradas por el ayudante de la sabiduría. La primera dotación supermental del hombre es la de la inclusión de su personalidad en el circuito del Espíritu Santo del Espíritu Creativo del Universo; y mucho antes de las donaciones de los Hijos divinos o de la donación universal de los Ajustadores, esta influencia actúa para ampliar el punto de vista del hombre sobre la ética, la religión y la espiritualidad. Después de las donaciones de los Hijos Paradisiacos, el Espíritu de la Verdad liberado contribuye poderosamente a aumentar la capacidad humana para percibir las verdades religiosas. A medida que progresa la evolución en un mundo habitado, los Ajustadores del Pensamiento participan cada vez más en el desarrollo de los tipos superiores de perspicacia religiosa humana. El Ajustador del Pensamiento es la ventana cósmica a través de la cual la criatura finita puede vislumbrar, por la fe, las certidumbres y divinidades de la Deidad ilimitada, el Padre Universal.

\par
%\textsuperscript{(1129.2)}
\textsuperscript{103:0.2} Las tendencias religiosas de las razas humanas son innatas; se manifiestan universalmente y tienen un origen aparentemente natural; las religiones primitivas son siempre evolutivas en su génesis. A medida que la experiencia religiosa natural continúa progresando, las revelaciones periódicas de la verdad se intercalan en el curso, por otra parte lento, de la evolución planetaria.

\par
%\textsuperscript{(1129.3)}
\textsuperscript{103:0.3} En Urantia existen actualmente cuatro tipos de religión:

\par
%\textsuperscript{(1129.4)}
\textsuperscript{103:0.4} 1. La religión natural o evolutiva.

\par
%\textsuperscript{(1129.5)}
\textsuperscript{103:0.5} 2. La religión sobrenatural o revelatoria.

\par
%\textsuperscript{(1129.6)}
\textsuperscript{103:0.6} 3. La religión práctica o corriente, una mezcla en mayor o menor grado de religiones naturales y sobrenaturales.

\par
%\textsuperscript{(1129.7)}
\textsuperscript{103:0.7} 4. Las religiones filosóficas, las doctrinas teológicas fabricadas por el hombre o elaboradas por la filosofía, y las religiones creadas por la razón.

\section*{1. La filosofía de la religión}
\par
%\textsuperscript{(1129.8)}
\textsuperscript{103:1.1} La unidad de la experiencia religiosa de un grupo social o racial proviene de la naturaleza idéntica del fragmento de Dios que reside en el individuo. Esta partícula divina en el hombre es la que origina su interés generoso por el bienestar de los demás hombres. Pero, puesto que la personalidad es única ---no hay dos mortales que sean iguales--- la consecuencia inevitable es que no hay dos seres humanos que puedan interpretar de la misma manera las directrices y los impulsos del espíritu de la divinidad que vive en sus mentes. Un grupo de mortales puede experimentar la unidad espiritual, pero nunca podrá alcanzar la uniformidad filosófica. Esta diversidad de interpretación del pensamiento y de la experiencia religiosos está demostrada en el hecho de que los teólogos y los filósofos del siglo veinte han formulado más de quinientas definiciones diferentes de la religión. En realidad, cada ser humano define la religión desde el punto de vista de su propia interpretación experiencial de los impulsos divinos que emanan del espíritu de Dios que reside en él, y por lo tanto esta interpretación ha de ser única y totalmente diferente de la filosofía religiosa de todos los demás seres humanos.

\par
%\textsuperscript{(1130.1)}
\textsuperscript{103:1.2} Cuando un mortal está plenamente de acuerdo con la filosofía religiosa de otro compañero mortal, ese fenómeno indica que estos dos seres han tenido una \textit{experiencia religiosa} similar en lo referente a las materias implicadas en su interpretación filosófica semejante de la religión.

\par
%\textsuperscript{(1130.2)}
\textsuperscript{103:1.3} Aunque vuestra religión es un asunto de experiencia personal, es sumamente importante que lleguéis a conocer una gran cantidad de otras experiencias religiosas (las diversas interpretaciones de otros mortales diferentes) a fin de que podáis impedir que vuestra vida religiosa se vuelva egocéntrica --- circunscrita, egoísta e insociable.

\par
%\textsuperscript{(1130.3)}
\textsuperscript{103:1.4} El racionalismo se equivoca cuando supone que la religión es, en primer lugar, una creencia primitiva en algo, que va seguida después de la búsqueda de los valores. La religión es ante todo una búsqueda de los valores, y luego formula un sistema de creencias interpretativas. Para los hombres es mucho más fácil ponerse de acuerdo sobre los valores religiosos ---las metas--- que sobre las creencias ---las interpretaciones. Esto explica cómo la religión puede coincidir en los valores y las metas, y mostrar al mismo tiempo el fenómeno desconcertante de mantener una creencia en cientos de creencias contrarias ---los credos. Esto explica también por qué una persona determinada puede mantener su experiencia religiosa a pesar de abandonar o de cambiar muchas de sus creencias religiosas. La religión subsiste a pesar de los cambios revolucionarios en las creencias religiosas. La teología no engendra la religión; es la religión la que da nacimiento a la filosofía teológica.

\par
%\textsuperscript{(1130.4)}
\textsuperscript{103:1.5} El hecho de que las personas religiosas hayan creído en tantas cosas falsas no invalida la religión, porque la religión está basada en el reconocimiento de los valores y es validada por la fe de la experiencia religiosa personal. La religión se basa pues en la experiencia y en el pensamiento religioso; la teología, la filosofía de la religión, es un intento sincero por interpretar esa experiencia. Estas creencias interpretativas pueden ser correctas o erróneas, o una mezcla de verdad y de error.

\par
%\textsuperscript{(1130.5)}
\textsuperscript{103:1.6} Llevar a cabo el reconocimiento de los valores espirituales es una experiencia que sobrepasa la ideación. Ningún idioma humano posee una palabra que se pueda emplear para designar esa <<sensación>>, <<sentimiento>>, <<intuición>> o <<experiencia>> que hemos elegido llamar la conciencia de Dios. El espíritu de Dios que reside en el hombre no es personal ---el Ajustador es prepersonal--- pero este Monitor presenta un valor, exhala un aroma de divinidad, que es personal en el sentido más elevado e infinito. Si Dios no fuera al menos personal, no podría ser consciente, y si no fuera consciente, entonces sería infrahumano\footnote{\textit{El espíritu en el hombre (Espíritu Santo)}: Gn 1:2; Ex 31:3; 35:31; Job 33:4; Sal 51:10-11; 139:7; Pr 1:23; Is 44:3; 59:21; 61:1; 63:10-11; Lc 4:1; 11:13; Jn 1:33; 3:5; 2 Ti 1:14. \textit{El espíritu en el hombre (Espíritu de la Verdad)}: Ez 11:19; 18:31; 36:26-27; Jl 2:28-29; Lc 24:49; Jn 7:39; 14:16-18,23,26; 15:4,26; 16:7,13-14; 17:21-23; Hch 1:5,8a; 2:1-4,16-18; 2:33; 2 Co 13:5; Gl 2:20; 4:6; Ef 1:13; 4:30; 1 Jn 4:12-15. \textit{El espíritu en el hombre (Ajustador del Pensamiento)}: Job 32:8,18; Is 63:10-11; Ez 37:14; Mt 10:20; Lc 17:21; Jn 17:21-23; Ro 8:9-11; 1 Co 3:16-17; 6:19; 2 Co 6:16; Gl 2:20; 1 Jn 3:24; 4:12-15; Ap 21:3.}.

\section*{2. La religión y el individuo}
\par
%\textsuperscript{(1130.6)}
\textsuperscript{103:2.1} La religión es funcional en la mente humana y se lleva a cabo en la experiencia antes de aparecer en la conciencia humana. Un niño existe durante cerca de nueve meses antes de experimentar el \textit{nacimiento}. Pero el <<nacimiento>> de la religión no es repentino, es más bien una aparición gradual. Sin embargo, tarde o temprano hay un <<día de nacimiento>>\footnote{\textit{Día del nacimiento espiritual}: Jn 1:13; 3:3-8.}. No entráis en el reino de los cielos a menos que hayáis <<nacido de nuevo>> ---nacido del Espíritu. Muchos nacimientos espirituales van acompañados de una gran angustia espiritual y de perturbaciones psicológicas acentuadas, al igual que muchos nacimientos físicos están caracterizados por un <<parto difícil>> y otras anormalidades del <<alumbramiento>>. Otros nacimientos espirituales suponen un crecimiento normal y natural del reconocimiento de los valores supremos con un incremento de la experiencia espiritual, aunque no se produce ningún desarrollo religioso sin un esfuerzo consciente y unas resoluciones positivas e individuales. La religión nunca es una experiencia pasiva, una actitud negativa. Lo que se llama el <<nacimiento de la religión>> no está directamente relacionado con las experiencias llamadas de conversión que caracterizan habitualmente a los episodios religiosos que se producen más tarde en la vida a consecuencia de conflictos mentales, represiones emocionales y trastornos temperamentales.

\par
%\textsuperscript{(1131.1)}
\textsuperscript{103:2.2} Pero aquellas personas que han sido criadas por sus padres de tal manera que han crecido con la conciencia de ser los hijos de un Padre celestial amoroso, no deberían mirar con recelo a sus compañeros mortales que sólo han podido alcanzar esta conciencia de comunión con Dios a través de una crisis psicológica, de un trastorno emocional.

\par
%\textsuperscript{(1131.2)}
\textsuperscript{103:2.3} El terreno evolutivo de la mente del hombre donde germina la semilla de la religión revelada es la naturaleza moral que da origen tan pronto a una conciencia social. Las primeras incitaciones de la naturaleza moral de un niño no están relacionadas con el sexo, la culpa o el orgullo personal, sino más bien con los impulsos de justicia, equidad y unos vivos deseos de bondad ---de servicio eficaz hacia sus semejantes. Cuando se alimentan estos despertares morales iniciales, se produce un desarrollo gradual de la vida religiosa que está relativamente libre de conflictos, trastornos y crisis.

\par
%\textsuperscript{(1131.3)}
\textsuperscript{103:2.4} Todo ser humano experimenta muy pronto algún tipo de conflicto entre sus impulsos egoístas y sus impulsos altruistas, y muchas veces, la primera experiencia de tener conciencia de Dios se puede alcanzar como resultado de buscar una ayuda superhumana para la tarea de resolver estos conflictos morales.

\par
%\textsuperscript{(1131.4)}
\textsuperscript{103:2.5} La psicología de un niño es positiva por naturaleza, no negativa. Hay tantos mortales que son negativos porque han sido educados así. Cuando decimos que los niños son positivos nos referimos a sus impulsos morales, a esos poderes mentales cuya aparición señala la llegada del Ajustador del Pensamiento.

\par
%\textsuperscript{(1131.5)}
\textsuperscript{103:2.6} Cuando surge la conciencia religiosa con ausencia de enseñanzas erróneas, la mente del niño normal avanza positivamente hacia la rectitud moral y el servicio social, en lugar de alejarse negativamente del pecado y la culpa. Puede o no haber conflicto en el desarrollo de la experiencia religiosa, pero siempre están presentes las inevitables decisiones, esfuerzos y actuaciones de la voluntad humana.

\par
%\textsuperscript{(1131.6)}
\textsuperscript{103:2.7} La elección moral está normalmente acompañada de un mayor o menor conflicto moral. Este primer conflicto de la mente infantil tiene lugar entre los vivos deseos del egoísmo y los impulsos del altruismo. El Ajustador del Pensamiento no desprecia los valores que los móviles egoístas tienen para la personalidad, pero trabaja para conceder una ligera preferencia a los impulsos altruistas que conducen a la meta de la felicidad humana y a las alegrías del reino de los cielos.

\par
%\textsuperscript{(1131.7)}
\textsuperscript{103:2.8} Cuando un ser moral escoge ser desinteresado al enfrentarse con el impulso de ser egoísta, lleva a cabo una experiencia religiosa primitiva. Ningún animal puede hacer esta elección; esta decisión es a la vez humana y religiosa. Abarca el hecho de la conciencia de Dios y manifiesta el impulso hacia el servicio social, la base de la fraternidad de los hombres. Cuando la mente escoge, mediante un acto de libre albedrío, un juicio moral justo, esta decisión constituye una experiencia religiosa.

\par
%\textsuperscript{(1131.8)}
\textsuperscript{103:2.9} Pero antes de que un niño se haya desarrollado lo suficiente como para adquirir una capacidad moral y, por lo tanto, ser capaz de escoger el servicio altruista, ya ha desarrollado una naturaleza egoísta fuerte y bien unificada. Esta situación de hecho es la que da origen a la teoría de la lucha entre la naturaleza <<superior>> y la naturaleza <<inferior>>\footnote{\textit{Naturaleza superior e inferior}: Ro 6:6; Ef 4:22-24; Col 3:9-10.}, entre el <<antiguo hombre pecador>> y la <<nueva naturaleza>> de la gracia. Un niño normal empieza a aprender muy pronto en la vida que es <<más bienaventurado dar que recibir>>\footnote{\textit{Más bienaventurado dar que recibir}: Hch 20:35.}.

\par
%\textsuperscript{(1131.9)}
\textsuperscript{103:2.10} El hombre tiende a identificar el impulso de servirse a sí mismo con su ego ---con su yo. Por contraste, se siente inclinado a identificar la voluntad de ser altruista con alguna influencia exterior a él ---Dios. Y en verdad este juicio es correcto, pues todos estos deseos altruistas tienen realmente su origen en las directrices del Ajustador del Pensamiento interior, y este Ajustador es un fragmento de Dios. La conciencia humana reconoce el impulso del Monitor espiritual como la incitación a ser altruista, a preocuparse por los semejantes. Ésta es al menos la experiencia inicial y fundamental de la mente del niño. Cuando el niño que crece no consigue unificar su personalidad, el impulso altruista puede superdesarrollarse hasta el punto de perjudicar seriamente el bienestar del yo. Una conciencia descaminada puede volverse responsable de muchos conflictos, preocupaciones, tristezas y un sinfín de desgracias humanas.

\section*{3. La religión y la raza humana}
\par
%\textsuperscript{(1132.1)}
\textsuperscript{103:3.1} Aunque todas las creencias en los espíritus, los sueños y otras diversas supersticiones han jugado un papel en el origen evolutivo de las religiones primitivas, no deberíais pasar por alto la influencia del espíritu de solidaridad del clan o de la tribu. En las relaciones de grupo estaba presente la situación social exacta que proporcionaba el estímulo para el conflicto entre el egoísmo y el altruismo en la naturaleza moral de la mente humana primitiva. A pesar de su creencia en los espíritus, los australianos primitivos centran todavía su religión en el clan. Con el tiempo, estos conceptos religiosos tienden a personalizarse, primero como animales, y más tarde bajo la forma de un superhombre o un Dios. Incluso las razas inferiores como los bosquimanos de África, que ni siquiera creen en los tótemes, reconocen la diferencia entre el interés personal y el interés colectivo, una distinción primitiva entre los valores seculares y los valores sagrados. Pero el grupo social no es la fuente de la experiencia religiosa. Independientemente de la influencia de todas estas contribuciones primitivas a la religión inicial del hombre, sigue siendo un hecho que el verdadero impulso religioso tiene su origen en las presencias espirituales auténticas que activan la voluntad de ser desinteresado.

\par
%\textsuperscript{(1132.2)}
\textsuperscript{103:3.2} La religión ulterior se presagia en la creencia primitiva en las maravillas y los misterios naturales, el mana impersonal. Pero tarde o temprano, la religión en evolución exige que el individuo haga algún sacrificio personal por el bien de su grupo social, haga algo para que otras personas sean más felices y mejores. Al final, la religión está destinada a convertirse en el servicio de Dios y de los hombres.

\par
%\textsuperscript{(1132.3)}
\textsuperscript{103:3.3} La religión está diseñada para cambiar el entorno del hombre, pero una gran parte de la religión que poseen los mortales de hoy se ha vuelto incapaz de hacerlo. El entorno es el que ha dominado con demasiada frecuencia a la religión.

\par
%\textsuperscript{(1132.4)}
\textsuperscript{103:3.4} Recordad que en la religión de todas las épocas, la experiencia más importante es el sentimiento relacionado con los valores morales y los significados sociales, y no el pensamiento relativo a los dogmas teológicos o a las teorías filosóficas. La religión evoluciona favorablemente a medida que el elemento de la magia es reemplazado por el concepto de la moral.

\par
%\textsuperscript{(1132.5)}
\textsuperscript{103:3.5} El hombre ha evolucionado desde las supersticiones del mana, la magia, la adoración de la naturaleza, el miedo a los espíritus y la adoración de los animales, hasta los diversos ceremoniales mediante los cuales las actitudes religiosas del individuo se convirtieron en las reacciones colectivas del clan. Luego estas ceremonias se focalizaron y cristalizaron en las creencias tribales, y finalmente estos miedos y credos se personalizaron en dioses. Pero en toda esta evolución religiosa, el elemento moral nunca ha estado totalmente ausente. El impulso de Dios dentro del hombre siempre ha sido fuerte. Estas poderosas influencias ---una humana y la otra divina--- aseguraron la supervivencia de la religión a través de las vicisitudes de los siglos, a pesar de que muy a menudo estuvo amenazada de extinción debido a cientos de tendencias subversivas y antagonismos hostiles.

\section*{4. La comunión espiritual}
\par
%\textsuperscript{(1133.1)}
\textsuperscript{103:4.1} La diferencia característica entre una reunión social y una asamblea religiosa consiste en que, en contraste con la mundana, la religiosa está impregnada de una atmósfera de \textit{comunión}. De esta manera, la asociación humana engendra un sentimiento de compañerismo con lo divino, y éste es el comienzo del culto colectivo. Compartir una comida común fue el primer tipo de comunión social, y las religiones primitivas estipularon así que una parte del sacrificio ceremonial fuera consumida por los fieles. Incluso en el cristianismo, el pan eucarístico conserva esta forma de comunión. La atmósfera de la comunión proporciona un período de tregua reconfortante y reparador en el conflicto entre el ego egoísta y el impulso altruista del Monitor espiritual interior. Éste es el preludio de la verdadera adoración ---la práctica de la presencia de Dios, que conduce a la aparición de la fraternidad de los hombres.

\par
%\textsuperscript{(1133.2)}
\textsuperscript{103:4.2} Cuando el hombre primitivo sentía que su comunión con Dios se había interrumpido, recurría a algún tipo de sacrificio en un esfuerzo por expiar su falta, por restablecer las relaciones amistosas. El hambre y la sed de rectitud conducen al descubrimiento de la verdad, y la verdad acrecienta los ideales, y esto crea nuevos problemas para las personas religiosas individuales, pues nuestros ideales tienden a crecer en progresión geométrica, mientras que nuestra capacidad para vivir a su altura sólo aumenta en progresión aritmética.

\par
%\textsuperscript{(1133.3)}
\textsuperscript{103:4.3} El sentimiento de culpa (no la conciencia del pecado) proviene, o bien de la interrupción de la comunión espiritual, o de la disminución de los ideales morales. Uno sólo puede liberarse de esta difícil situación comprendiendo bien que nuestros ideales morales más elevados no son necesariamente sinónimos de la voluntad de Dios. El hombre no puede esperar vivir a la altura de sus ideales más elevados, pero puede ser fiel a su intención de encontrar a Dios y de parecerse cada vez más a él.

\par
%\textsuperscript{(1133.4)}
\textsuperscript{103:4.4} Jesús suprimió todas las ceremonias de sacrificios y de expiación. Destruyó las bases de toda esta culpabilidad ficticia y de este sentimiento de aislamiento en el universo al afirmar que el hombre es un hijo de Dios; la relación entre la criatura y el Creador fue puesta sobre la base de una relación entre padre e hijo. Dios se convierte en un Padre amoroso para sus hijos e hijas mortales. Todas las ceremonias que no formen parte legítima de esta relación familiar íntima están abolidas para siempre.

\par
%\textsuperscript{(1133.5)}
\textsuperscript{103:4.5} Dios Padre no se relaciona con el hombre, su hijo, sobre la base de sus virtudes o de sus méritos reales, sino sobre el reconocimiento de los móviles del hijo ---el propósito y la intención de la criatura. Esta relación es una asociación entre padre e hijo, y está impulsada por el amor divino.

\section*{5. El origen de los ideales}
\par
%\textsuperscript{(1133.6)}
\textsuperscript{103:5.1} La mente evolutiva primitiva da origen a un sentimiento de deber social y de obligación moral derivado principalmente del miedo emocional. El deseo más positivo de servicio social y el idealismo altruista proceden del impulso directo del espíritu divino que reside en la mente humana\footnote{\textit{La Regla de Oro procedente del Ajustador}: Mt 7:12; Lc 6:31.}.

\par
%\textsuperscript{(1133.7)}
\textsuperscript{103:5.2} Esta idea-ideal de hacer el bien a los demás\footnote{\textit{Hacer el bien a los demás}: Gl 6:10.} ---el impulso de negarle algo al ego en beneficio de nuestro prójimo--- está al principio muy circunscrita. El hombre primitivo sólo considera como prójimos a las personas más cercanas a él\footnote{\textit{Amar al prójimo como a sí mismo}: Lv 19:18,34; Mt 5:43-44; 19:19; 22:39; Mc 12:31,33; Lc 10:27; Ro 13:9b; Gl 5:13-14; Stg 2:8.}, a aquellos que lo tratan con amistad; a medida que avanza la civilización religiosa, el concepto de prójimo se expande hasta abarcar el clan, la tribu, o la nación\footnote{\textit{Amarse unos a otros}: 1 Ts 4:9; 1 P 1:22; 1 Jn 3:11,23; 4:7,11-12,21; 2 Jn 1:5.}. Luego, Jesús amplió el ámbito del prójimo hasta englobar al conjunto de la humanidad\footnote{\textit{Amar a los semejantes}: Jn 13:34-35; 15:12,17.}, y que deberíamos amar incluso a nuestros enemigos\footnote{\textit{Amar a los enemigos}: Mt 5:44; Lc 6:27,35.}. Hay algo en el interior de cada ser humano normal que le dice que esta enseñanza es moral ---es justa. Incluso aquellos que practican menos este ideal admiten que es justo en teoría.

\par
%\textsuperscript{(1134.1)}
\textsuperscript{103:5.3} Todos los hombres reconocen la moralidad de este impulso humano universal a ser desinteresados y altruistas. El humanista atribuye el origen de este impulso al funcionamiento natural de la mente material; la persona religiosa reconoce más correctamente que este impulso verdaderamente desinteresado de la mente mortal es una respuesta a las directrices espirituales internas del Ajustador del Pensamiento.

\par
%\textsuperscript{(1134.2)}
\textsuperscript{103:5.4} Pero la interpretación que el hombre hace de estos conflictos iniciales entre la voluntad que busca el bien del yo y la voluntad que busca el bien de los demás no siempre es fiable. Sólo una personalidad bastante bien unificada puede arbitrar las controversias multiformes entre los anhelos del ego y la conciencia social en ciernes. Nuestro yo tiene sus derechos así como nuestros prójimos tienen los suyos. Ninguno de los dos debe reclamar en exclusiva la atención y el servicio del individuo. La incapacidad para resolver este problema da origen al tipo más primitivo de sentimientos humanos de culpa.

\par
%\textsuperscript{(1134.3)}
\textsuperscript{103:5.5} La felicidad humana sólo se consigue cuando el deseo egoísta del yo y el impulso altruista del yo superior (del espíritu divino) están coordinados y conciliados mediante la voluntad unificada de la personalidad que integra y supervisa. La mente del hombre evolutivo se enfrenta constantemente al complejo problema de arbitrar el combate entre la expansión natural de los impulsos emocionales y el crecimiento moral de las incitaciones altruistas basadas en la perspicacia espiritual ---en la reflexión religiosa auténtica.

\par
%\textsuperscript{(1134.4)}
\textsuperscript{103:5.6} El intento por conseguir la misma cantidad de bien para el yo que para el mayor número de otros yoes representa un problema que no siempre se puede resolver satisfactoriamente dentro de un marco espacio-temporal. En el transcurso de una vida eterna, estos antagonismos se pueden resolver, pero en una corta vida humana es imposible solucionarlos. Jesús se refirió a esta paradoja cuando dijo: <<Aquel que salve su vida la perderá, pero aquel que pierda su vida por amor al reino, la encontrará>>\footnote{\textit{Quien salve la vida la perderá}: Mt 10:39; 16:25; Mc 8:35; Lc 9:24; 17:33; Jn 12:25.}.

\par
%\textsuperscript{(1134.5)}
\textsuperscript{103:5.7} La persecución del ideal ---la lucha por parecerse a Dios--- es un esfuerzo continuo antes y después de la muerte. La vida después de la muerte no es diferente, en sus aspectos esenciales, a la existencia mortal. Todo lo bueno que hacemos en esta vida contribuye directamente a realzar la vida futura. La verdadera religión no favorece la indolencia moral ni la pereza espiritual fomentando la vana esperanza de recibir todas las virtudes de un carácter noble por el simple hecho de atravesar las puertas de la muerte natural. La verdadera religión no minimiza los esfuerzos del hombre por progresar durante su estancia en la vida como arrendatario mortal. Todo logro humano contribuye directamente a enriquecer las primeras etapas de la experiencia de la supervivencia inmortal.

\par
%\textsuperscript{(1134.6)}
\textsuperscript{103:5.8} Es funesto para el idealismo humano enseñarle al hombre que todos sus impulsos altruistas son simplemente el desarrollo de sus instintos gregarios naturales. Pero el hombre se siente ennoblecido y poderosamente estimulado cuando se entera de que estos impulsos superiores de su alma emanan de las fuerzas espirituales que residen en su mente mortal.

\par
%\textsuperscript{(1134.7)}
\textsuperscript{103:5.9} Una vez que el hombre comprende plenamente que algo eterno y divino vive y se esfuerza dentro de él, esto lo eleva por encima y más allá de sí mismo. Así es como una fe viviente en el origen superhumano de nuestros ideales valida nuestra creencia de que somos hijos de Dios y hace reales nuestras convicciones altruistas, los sentimientos de la fraternidad de los hombres.

\par
%\textsuperscript{(1134.8)}
\textsuperscript{103:5.10} El hombre, en su ámbito espiritual, posee realmente un libre albedrío. El hombre mortal no es un esclavo desamparado de la soberanía inflexible de un Dios todopoderoso, ni una víctima de la fatalidad desesperante de un determinismo cósmico mecanicista. El hombre es verdaderamente el arquitecto de su propio destino eterno.

\par
%\textsuperscript{(1135.1)}
\textsuperscript{103:5.11} Pero las presiones no salvan ni ennoblecen al hombre. El crecimiento espiritual surge del interior del alma en evolución. La presión puede deformar la personalidad, pero nunca estimula el crecimiento. Incluso la presión educativa sólo es negativamente útil, en el sentido de que puede ayudar a impedir las experiencias desastrosas. El crecimiento espiritual es mucho mayor cuando todas las presiones externas se reducen al mínimo. <<Allí donde está el espíritu del Señor, hay libertad>>\footnote{\textit{Donde está el espíritu del Señor, hay libertad}: 2 Co 3:17.}. El hombre se desarrolla mejor cuando las presiones del hogar, la comunidad, la iglesia y el Estado son menores. Pero no se debe interpretar que esto signifique que en una sociedad progresiva no haya cabida para el hogar, las instituciones sociales, la iglesia y el Estado.

\par
%\textsuperscript{(1135.2)}
\textsuperscript{103:5.12} Cuando un miembro de un grupo social religioso ha cumplido con los requisitos de dicho grupo, se le debería animar a disfrutar de la libertad religiosa, expresando plenamente su propia interpretación personal de las verdades de la creencia religiosa y de los hechos de la experiencia religiosa. La seguridad de un grupo religioso depende de su unidad espiritual, no de su uniformidad teológica. Los miembros de un grupo religioso deberían poder disfrutar de la libertad de pensar libremente, sin tener que convertirse en <<librepensadores>>. Existe una gran esperanza para toda iglesia que adore al Dios viviente, valide la fraternidad de los hombres y se atreva a suprimir la presión de todo credo entre sus miembros.

\section*{6. La coordinación filosófica}
\par
%\textsuperscript{(1135.3)}
\textsuperscript{103:6.1} La teología es el estudio de las acciones y reacciones del espíritu humano; nunca podrá convertirse en una ciencia, ya que siempre deberá estar más o menos combinada con la psicología para expresarse de forma personal, y con la filosofía para ser descrita de manera sistemática. La teología es siempre el estudio de \textit{vuestra} religión; el estudio de la religión de los demás es la psicología.

\par
%\textsuperscript{(1135.4)}
\textsuperscript{103:6.2} Cuando el hombre aborda el estudio y el examen de su universo desde el \textit{exterior}, da nacimiento a las diversas ciencias físicas; cuando aborda la investigación de sí mismo y del universo desde el \textit{interior}, da origen a la teología y a la metafísica. El arte posterior de la filosofía se desarrolla en un esfuerzo por armonizar las numerosas discrepancias que al principio están destinadas a aparecer entre los hallazgos y las enseñanzas de estas dos maneras diametralmente opuestas de acercarse al universo de cosas y de seres.

\par
%\textsuperscript{(1135.5)}
\textsuperscript{103:6.3} La religión tiene que ver con el punto de vista espiritual, con la conciencia de la \textit{interioridad} de la experiencia humana. La naturaleza espiritual del hombre le proporciona a éste la oportunidad de darle la vuelta al universo desde fuera hacia dentro. Por lo tanto es cierto que, vista exclusivamente desde la interioridad de la experiencia de la personalidad, toda la creación parece ser de naturaleza espiritual.

\par
%\textsuperscript{(1135.6)}
\textsuperscript{103:6.4} Cuando el hombre inspecciona analíticamente el universo a través de los dones materiales de sus sentidos físicos y de su percepción mental asociada, el cosmos parece ser mecánico y energético-material. Esta técnica para estudiar la realidad consiste en darle la vuelta al universo desde dentro hacia fuera.

\par
%\textsuperscript{(1135.7)}
\textsuperscript{103:6.5} No se puede construir un concepto filosófico lógico y coherente del universo sobre los postulados del materialismo o del espiritismo, pues estos dos sistemas de pensamiento, cuando se aplican de forma universal, se ven obligados a ver el cosmos de manera deformada, ya que el primero contacta con un universo vuelto desde dentro hacia fuera, y el segundo reconoce la naturaleza de un universo vuelto desde fuera hacia dentro. Así pues, ni la ciencia ni la religión solas, en sí mismas y por sí mismas, nunca podrán esperar conseguir una comprensión adecuada de las verdades y las relaciones universales sin la guía de la filosofía humana y la iluminación de la revelación divina.

\par
%\textsuperscript{(1136.1)}
\textsuperscript{103:6.6} El espíritu interior del hombre tendrá que depender siempre, para poder expresarse y autorrealizarse, del mecanismo y la técnica de la mente. La experiencia exterior del hombre con la realidad material deberá basarse igualmente en la conciencia mental de la personalidad que experimenta. Por esta razón, las experiencias humanas espirituales y materiales ---interiores y exteriores--- están siempre correlacionadas con la función mental, y condicionadas, en cuanto a su comprensión consciente, por la actividad de la mente. El hombre experimenta la materia en su mente; experimenta la realidad espiritual en su alma, pero se hace consciente de esta experiencia en su mente. El intelecto es el armonizador siempre presente que condiciona y cualifica la suma total de la experiencia mortal. Tanto las cosas-energía como los valores espirituales están teñidos por la interpretación que realizan los medios mentales de la conciencia.

\par
%\textsuperscript{(1136.2)}
\textsuperscript{103:6.7} La dificultad que tenéis para conseguir una coordinación más armoniosa entre la ciencia y la religión se debe a vuestra ignorancia total sobre el ámbito intermedio del mundo morontial de cosas y de seres. El universo local consta de tres grados, o estados, de manifestación de la realidad: la materia, la morontia y el espíritu. El ángulo de aproximación morontial borra toda divergencia entre los hallazgos de las ciencias físicas y el funcionamiento del espíritu de la religión. La razón es la técnica de comprensión de las ciencias; la fe es la técnica de perspicacia de la religión; la mota es la técnica del nivel morontial. La mota es una sensibilidad supermaterial a la realidad, que empieza a compensar el crecimiento incompleto; tiene por sustancia el conocimiento-razón y por esencia la fe-perspicacia. La mota es una reconciliación superfilosófica de las percepciones divergentes de la realidad, y las personalidades materiales no la pueden alcanzar; está basada en parte en la experiencia de haber sobrevivido a la vida material en la carne. Pero muchos mortales han reconocido la conveniencia de poseer algún método que reconcilie la interacción entre los campos ampliamente separados de la ciencia y la religión; y la metafísica es el resultado del intento infructuoso del hombre por tender un puente sobre este abismo bien reconocido. Pero la metafísica humana ha resultado ser más desconcertante que iluminadora. La metafísica representa el esfuerzo bien intencionado, pero inútil, del hombre por compensar la ausencia de la mota morontial.

\par
%\textsuperscript{(1136.3)}
\textsuperscript{103:6.8} La metafísica ha resultado ser un fracaso; el hombre no puede percibir la mota. La revelación es la única técnica que puede compensar, en un mundo material, la ausencia de la sensibilidad de la mota a la verdad. La revelación clarifica con autoridad la confusión de la metafísica desarrollada por la razón en una esfera evolutiva.

\par
%\textsuperscript{(1136.4)}
\textsuperscript{103:6.9} La ciencia es el intento del hombre por estudiar su entorno físico, el mundo de la energía-materia; la religión es la experiencia del hombre con el cosmos de los valores espirituales; la filosofía ha sido desarrollada por el esfuerzo mental del hombre por organizar y correlacionar los hallazgos de estos conceptos ampliamente separados en algo semejante a una actitud razonable y unificada ante el cosmos. La filosofía, clarificada por la revelación, funciona aceptablemente en ausencia de la mota y en presencia del derrumbamiento y el fracaso de la metafísica, creada por la razón del hombre para sustituir a la mota.

\par
%\textsuperscript{(1136.5)}
\textsuperscript{103:6.10} El hombre primitivo no diferenciaba entre el nivel de la energía y el nivel del espíritu. La raza violeta y sus sucesores anditas fueron los primeros que intentaron separar lo matemático de lo volitivo. El hombre civilizado ha seguido cada vez más los pasos de los primeros griegos y de los sumerios, los cuales distinguían entre lo animado y lo inanimado. A medida que progrese la civilización, la filosofía tendrá que colmar los abismos cada vez más grandes entre el concepto del espíritu y el concepto de la energía. Pero, en el tiempo del espacio, estas divergencias están unificadas en el Supremo.

\par
%\textsuperscript{(1137.1)}
\textsuperscript{103:6.11} La ciencia debe basarse siempre en la razón, aunque la imaginación y las conjeturas ayudan a extender sus fronteras. La religión depende para siempre de la fe, aunque la razón es una influencia estabilizadora y una sirviente útil. Siempre ha habido y siempre habrá interpretaciones engañosas de los fenómenos del mundo natural y del mundo espiritual, las ciencias y las religiones llamadas así equivocadamente.

\par
%\textsuperscript{(1137.2)}
\textsuperscript{103:6.12} Basándose en su comprensión incompleta de la ciencia, en su débil dominio de la religión y en sus tentativas frustradas en metafísica, el hombre ha intentado construir sus formulaciones filosóficas. El hombre moderno construiría en verdad una filosofía valiosa y atractiva de sí mismo y de su universo si no fuera por la ruptura de su importantísima e indispensable conexión metafísica entre los mundos de la materia y del espíritu, ya que la metafísica no ha logrado tender un puente sobre el abismo morontial entre lo físico y lo espiritual. Al hombre mortal le falta el concepto de la mente y la materia morontiales, y la \textit{revelación} es la única técnica que existe para reparar esta carencia de datos conceptuales que el hombre necesita tan urgentemente para poder construir una filosofía lógica del universo y para llegar a comprender satisfactoriamente el lugar seguro y establecido que ocupa en este universo.

\par
%\textsuperscript{(1137.3)}
\textsuperscript{103:6.13} La revelación es la única esperanza que tiene el hombre evolutivo para tender un puente sobre el abismo morontial. La fe y la razón, sin la ayuda de la mota, no pueden concebir ni construir un universo lógico. Sin la perspicacia de la mota, el hombre mortal no puede discernir la bondad, el amor y la verdad en los fenómenos del mundo material.

\par
%\textsuperscript{(1137.4)}
\textsuperscript{103:6.14} Cuando la filosofía del hombre se inclina intensamente hacia el mundo de la materia, se vuelve racionalista o \textit{naturalista}. Cuando la filosofía se inclina especialmente hacia el nivel espiritual, se vuelve \textit{idealista} e incluso mística. Cuando la filosofía tiene el desacierto de apoyarse en la metafísica, se vuelve infaliblemente \textit{escéptica}, confusa. En las épocas pasadas, la mayor parte del conocimiento y de las evaluaciones intelectuales del hombre han caído en una de estas tres deformaciones de la percepción. La filosofía no se atreve a proyectar sus interpretaciones de la realidad de manera lineal como lo hace la lógica; nunca debe olvidar tener en cuenta la simetría elíptica de la realidad y la curvatura esencial de todos los conceptos de relación.

\par
%\textsuperscript{(1137.5)}
\textsuperscript{103:6.15} La filosofía más elevada que puede alcanzar el hombre mortal debe estar basada lógicamente en la razón de la ciencia, la fe de la religión y la perspicacia de la verdad que proporciona la revelación. Mediante esta unión, el hombre puede compensar un poco su fracaso en desarrollar una metafísica adecuada y su incapacidad para comprender la mota de la morontia.

\section*{7. La ciencia y la religión}
\par
%\textsuperscript{(1137.6)}
\textsuperscript{103:7.1} La ciencia está sostenida por la razón, y la religión por la fe. Aunque la fe no está basada en la razón, es razonable; aunque sea independiente de la lógica, sin embargo está estimulada por una lógica sana. La fe ni siquiera puede ser alimentada por una filosofía ideal; la fe es en verdad, junto con la ciencia, la fuente misma de dicha filosofía. La fe, la perspicacia religiosa humana, sólo puede ser dirigida de manera segura por la revelación, sólo puede ser elevada con seguridad por la experiencia personal de los mortales con la presencia espiritual, bajo la forma de Ajustador, del Dios que es espíritu\footnote{\textit{Dios que es espíritu}: Jn 4:24.}.

\par
%\textsuperscript{(1137.7)}
\textsuperscript{103:7.2} La verdadera salvación es la técnica de la evolución divina de la mente mortal, desde su identificación con la materia, pasando por los mundos de enlace morontial, hasta el estado universal superior de la correlación espiritual. De la misma manera que, en la evolución terrestre, el instinto intuitivo material precede a la aparición del conocimiento razonado, la manifestación de la perspicacia intuitiva espiritual presagia la aparición posterior de la razón y de la experiencia morontial y espiritual en el excelso programa de la evolución celestial, que consiste en transmutar los potenciales del hombre temporal en la realidad y la divinidad del hombre eterno, de un finalitario del Paraíso.

\par
%\textsuperscript{(1138.1)}
\textsuperscript{103:7.3} Pero a medida que el hombre ascendente se dirige hacia el interior y hacia el Paraíso para efectuar su experiencia con Dios, se dirigirá igualmente hacia fuera y hacia el espacio para comprender, en términos energéticos, el cosmos material. La progresión de la ciencia no está limitada a la vida terrestre del hombre; su experiencia de ascensión en el universo y en el superuniverso será en gran medida el estudio de la transmutación de la energía y de la metamorfosis de la materia. Dios es espíritu, pero la Deidad es unidad, y la unidad de la Deidad engloba no solamente los valores espirituales del Padre Universal y del Hijo Eterno, sino que conoce también los hechos energéticos del Controlador Universal y de la Isla del Paraíso, mientras que estas dos fases de la realidad universal están perfectamente correlacionadas en las relaciones mentales del Actor Conjunto y unificadas, en el nivel finito, en la Deidad emergente del Ser Supremo.

\par
%\textsuperscript{(1138.2)}
\textsuperscript{103:7.4} La unión de la actitud científica y de la perspicacia religiosa, por mediación de la filosofía experiencial, forma parte de la larga experiencia humana de ascensión al Paraíso. Las aproximaciones de las matemáticas y las certezas de la perspicacia necesitarán siempre la función armonizadora de la lógica mental en todos los niveles experienciales inferiores a la máxima consecución del Supremo.

\par
%\textsuperscript{(1138.3)}
\textsuperscript{103:7.5} Pero la lógica nunca podrá conseguir armonizar los hallazgos de la ciencia y las percepciones de la religión, a menos que los aspectos científicos y religiosos de una personalidad estén dominados por la verdad, estén sinceramente deseosos de seguir a la verdad dondequiera que los conduzca, sin tener en cuenta las conclusiones a las que los pueda llevar.

\par
%\textsuperscript{(1138.4)}
\textsuperscript{103:7.6} La lógica es la técnica de la filosofía, su método de expresión. Dentro del ámbito de la ciencia verdadera, la razón siempre es sensible a la lógica auténtica; dentro del ámbito de la verdadera religión, la fe siempre es lógica cuando es contemplada desde la base de un punto de vista interior, aunque esta fe pueda parecer totalmente sin fundamento desde el punto de vista del enfoque científico, que la contempla desde fuera hacia dentro. Mirando desde fuera hacia dentro, el universo puede parecer material; mirando desde dentro hacia fuera, el mismo universo parece ser totalmente espiritual. La razón surge de la conciencia material, la fe, de la conciencia espiritual, pero gracias a la mediación de una filosofía reforzada por la revelación, la lógica puede confirmar tanto el punto de vista interior como el exterior, estabilizando de este modo tanto a la ciencia como a la religión. Así, a través de un contacto común con la lógica de la filosofía, la ciencia y la religión pueden volverse cada vez más tolerantes la una con la otra, cada vez menos escépticas.

\par
%\textsuperscript{(1138.5)}
\textsuperscript{103:7.7} Lo que la ciencia y la religión en desarrollo necesitan es una autocrítica más penetrante y audaz, una mayor conciencia de la condición incompleta de sus estados evolutivos. Los instructores de la ciencia y de la religión están a menudo, en conjunto, demasiado seguros de sí mismos y son demasiado dogmáticos. La ciencia y la religión sólo pueden autocriticar sus propios \textit{hechos}. A partir del momento en que se apartan del marco de los hechos, la razón abdica o bien degenera rápidamente en un compañero de falsa lógica.

\par
%\textsuperscript{(1138.6)}
\textsuperscript{103:7.8} La verdad ---la comprensión de las relaciones cósmicas, los hechos universales y los valores espirituales--- puede conseguirse mejor a través del ministerio del Espíritu de la Verdad, y puede ser criticada mejor por la \textit{revelación}. Pero la revelación no da origen a una ciencia ni a una religión; su función consiste en coordinar la ciencia y la religión con la verdad de la realidad. En ausencia de revelación, o a falta de aceptarla o de comprenderla, el hombre mortal siempre ha recurrido a su inútil gesto hacia la metafísica, ya que ésta es la única sustituta humana de la revelación de la verdad o de la mota de la personalidad morontial.

\par
%\textsuperscript{(1139.1)}
\textsuperscript{103:7.9} La ciencia del mundo material permite al hombre controlar, y hasta cierto punto dominar, su entorno físico. La religión de la experiencia espiritual es la fuente del impulso hacia la fraternidad que permite a los hombres convivir en las complejidades de la civilización de una era científica. La metafísica, pero con más seguridad la revelación, proporciona un terreno de encuentro común para los descubrimientos de la ciencia y de la religión, y hace posible el intento humano por correlacionar lógicamente estas esferas del pensamiento, separadas pero interdependientes, en una filosofía bien equilibrada impregnada de estabilidad científica y de certidumbre religiosa.

\par
%\textsuperscript{(1139.2)}
\textsuperscript{103:7.10} En el estado mortal no hay nada que se pueda probar de manera absoluta; tanto la ciencia como la religión están basadas en suposiciones. En el nivel morontial, los postulados de la ciencia y de la religión se pueden probar parcialmente mediante la lógica de la mota. En el nivel espiritual representado por el estado máximo, la necesidad de una prueba finita se desvanece gradualmente ante la experiencia efectiva de, y con, la realidad; pero incluso entonces existen muchas cosas más allá de lo finito que permanecen sin poderse probar.

\par
%\textsuperscript{(1139.3)}
\textsuperscript{103:7.11} Todas las divisiones del pensamiento humano están basadas en ciertas suposiciones que, aunque no están probadas, son aceptadas por la sensibilidad constitutiva a la realidad de la dotación mental del hombre. La ciencia inicia su carrera de razonamiento tan alabada \textit{suponiendo} la realidad de tres cosas: la materia, el movimiento y la vida. La religión se pone en marcha con la suposición de la validez de tres cosas: la mente, el espíritu y el universo ---el Ser Supremo.

\par
%\textsuperscript{(1139.4)}
\textsuperscript{103:7.12} La ciencia se convierte en el campo de reflexión de las matemáticas, de la energía y la materia temporales en el espacio. La religión no sólo pretende ocuparse del espíritu finito y temporal, sino también del espíritu de la eternidad y de la supremacía. Estas dos maneras extremas de percibir el universo sólo pueden llegar a proporcionar unas interpretaciones análogas sobre los orígenes, las funciones, las relaciones, las realidades y los destinos a través de una larga experiencia con la mota. La divergencia entre la energía y el espíritu encuentra su máxima armonización en el circuito de los Siete Espíritus Maestros; la primera unificación de esta divergencia se produce en la Deidad del Supremo, y la unidad de su finalidad, en la infinidad de la Fuente-Centro Primera, el YO SOY\footnote{\textit{YO SOY}: Ex 3:13-14.}.

\par
%\textsuperscript{(1139.5)}
\textsuperscript{103:7.13} La \textit{razón} es el acto de reconocer las conclusiones de la conciencia en relación con la experiencia en, y con, el mundo físico de energía y de materia. La \textit{fe} es el acto de reconocer la validez de la conciencia espiritual ---algo que no se puede probar humanamente de otra manera. La \textit{lógica} es la progresión sintética, mediante la búsqueda de la verdad, de la unidad de la fe y la razón, y está basada en los dones mentales constitutivos de los seres mortales, el reconocimiento innato de las cosas, los significados y los valores.

\par
%\textsuperscript{(1139.6)}
\textsuperscript{103:7.14} La presencia del Ajustador del Pensamiento aporta una verdadera prueba de la realidad espiritual, pero la validez de esta presencia no es demostrable para el mundo exterior, sino solamente para aquel que experimenta así la existencia interior de Dios. La conciencia de tener un Ajustador está basada en la recepción intelectual de la verdad, en la percepción supermental de la bondad, y en la motivación de la personalidad para amar.

\par
%\textsuperscript{(1139.7)}
\textsuperscript{103:7.15} La ciencia descubre el mundo material, la religión lo evalúa, y la filosofía se esfuerza por interpretar sus significados a la vez que coordina el punto de vista científico material con el concepto religioso espiritual. Pero la historia es un terreno donde la ciencia y la religión quizás no se pongan nunca plenamente de acuerdo.

\section*{8. La filosofía y la religión}
\par
%\textsuperscript{(1140.1)}
\textsuperscript{103:8.1} Aunque la ciencia y la filosofía puedan suponer la probabilidad de Dios mediante su razón y su lógica, sólo la experiencia religiosa personal de un hombre conducido por el espíritu puede afirmar con certeza que esta Deidad suprema y personal existe. Mediante la técnica de encarnar así la verdad viviente, la hipótesis filosófica de la probabilidad de Dios se convierte en una realidad religiosa.

\par
%\textsuperscript{(1140.2)}
\textsuperscript{103:8.2} La confusión en torno a la experiencia de la certidumbre sobre Dios proviene de las interpretaciones y relaciones desiguales que las distintas personas y las diferentes razas de hombres tienen de esta experiencia. El experimentar a Dios puede ser totalmente válido, pero la disertación \textit{sobre} Dios, como es intelectual y filosófica, es divergente y a menudo desconcertantemente falaz.

\par
%\textsuperscript{(1140.3)}
\textsuperscript{103:8.3} Un hombre bueno y noble puede estar totalmente enamorado de su esposa, pero ser completamente incapaz de pasar satisfactoriamente un examen escrito sobre la psicología del amor conyugal. Otro hombre, que tenga poco o ningún amor por su esposa, podría pasar este examen de una manera muy aceptable. La idea imperfecta que se hace el enamorado sobre la verdadera naturaleza del ser amado no invalida en lo más mínimo la realidad o la sinceridad de su amor.

\par
%\textsuperscript{(1140.4)}
\textsuperscript{103:8.4} Si creéis realmente en Dios ---si lo conocéis y lo amáis por la fe--- no permitáis que la realidad de esta experiencia sea disminuida o empañada de ninguna manera por las insinuaciones dubitativas de la ciencia, los reparos de la lógica, los postulados de la filosofía, o las sugerencias ingeniosas de las almas bien intencionadas que quisieran crear una religión sin Dios.

\par
%\textsuperscript{(1140.5)}
\textsuperscript{103:8.5} La certidumbre de la persona religiosa que conoce a Dios no debería alterarse por la incertidumbre de los materialistas incrédulos; la fe profunda y la certeza inquebrantable del creyente experiencial son más bien las que deberían constituir un poderoso desafío para la incertidumbre del no creyente.

\par
%\textsuperscript{(1140.6)}
\textsuperscript{103:8.6} La filosofía, para poder prestar el mayor servicio tanto a la ciencia como a la religión, debería evitar los extremos del materialismo y del panteísmo. Sólo una filosofía que reconoce la realidad de la personalidad ---la permanencia en presencia del cambio--- puede tener un valor moral para el hombre, puede servir de enlace entre las teorías de la ciencia material y las de la religión espiritual. La revelación viene a compensar la fragilidad de la filosofía en evolución.

\section*{9. La esencia de la religión}
\par
%\textsuperscript{(1140.7)}
\textsuperscript{103:9.1} La teología se ocupa del contenido intelectual de la religión, y la metafísica (la revelación) trata de los aspectos filosóficos. La experiencia religiosa \textit{es} el contenido espiritual de la religión. A pesar de las extravagancias mitológicas y las ilusiones psicológicas del contenido intelectual de la religión, de las suposiciones metafísicas erróneas y las técnicas para engañarse a sí mismo, de las deformaciones políticas y las perversiones socioeconómicas del contenido filosófico de la religión, la experiencia espiritual de la religión personal sigue siendo auténtica y válida.

\par
%\textsuperscript{(1140.8)}
\textsuperscript{103:9.2} La religión tiene que ver con el sentimiento, la acción y la vida, y no simplemente con el pensamiento. El pensamiento está más estrechamente relacionado con la vida material y debería estar dominado en general, aunque no del todo, por la razón y los hechos de la ciencia y, en sus tendencias inmateriales hacia los mundos del espíritu, por la verdad. Por muy ilusoria y errónea que sea vuestra teología, vuestra religión puede ser totalmente auténtica y eternamente verdadera.

\par
%\textsuperscript{(1141.1)}
\textsuperscript{103:9.3} El budismo, en su forma original, es una de las mejores religiones sin Dios que han aparecido en toda la historia evolutiva de Urantia, aunque esta doctrina no permaneció atea en el transcurso de su desarrollo. Una religión sin fe es una contradicción; una religión sin Dios es una inconsecuencia filosófica y un absurdo intelectual.

\par
%\textsuperscript{(1141.2)}
\textsuperscript{103:9.4} El origen mágico y mitológico de la religión natural no invalida la realidad y la verdad de las religiones revelatorias posteriores ni el evangelio salvador consumado de la religión de Jesús. La vida y las enseñanzas de Jesús despojaron finalmente a la religión de las supersticiones de la magia, de las ilusiones de la mitología y de la esclavitud del dogmatismo tradicional. Pero esta magia y esta mitología primitivas habían preparado muy eficazmente el camino para una religión posterior y superior mediante la suposición de la existencia y la realidad de los valores y los seres supermateriales.

\par
%\textsuperscript{(1141.3)}
\textsuperscript{103:9.5} Aunque la experiencia religiosa es un fenómeno subjetivo puramente espiritual, esta experiencia engloba una actitud de fe positiva y viviente hacia los reinos más elevados de la realidad objetiva universal. El ideal de la filosofía religiosa es una fe-confianza capaz de conducir al hombre a depender sin reservas del amor absoluto del Padre infinito del universo de universos. Esta experiencia religiosa auténtica trasciende de lejos la objetivación filosófica de los deseos idealistas; da realmente por descontada la salvación y sólo se preocupa por saber y hacer la voluntad del Padre que está en el Paraíso. Las marcas distintivas de una religión así son: la fe en una Deidad suprema, la esperanza de una supervivencia eterna, y el amor, especialmente el amor a los semejantes.

\par
%\textsuperscript{(1141.4)}
\textsuperscript{103:9.6} Cuando la teología domina a la religión, la religión muere; se convierte en una doctrina en lugar de ser una vida. La misión de la teología consiste simplemente en facilitar la toma de conciencia de la experiencia espiritual personal. La teología constituye el esfuerzo religioso por definir, clarificar, exponer y justificar las afirmaciones experienciales de la religión que, a fin de cuentas, sólo pueden ser validadas por una fe viviente. En la filosofía superior del universo, la sabiduría, al igual que la razón, se alía con la fe. La razón, la sabiduría y la fe son los logros más elevados del hombre. La razón introduce al hombre en el mundo de los hechos, de las cosas; la sabiduría lo introduce en el mundo de la verdad, de las relaciones; la fe lo hace entrar en el mundo de la divinidad, de la experiencia espiritual.

\par
%\textsuperscript{(1141.5)}
\textsuperscript{103:9.7} La fe arrastra con mucho gusto a la razón hasta donde la razón puede llegar; luego la fe continúa con la sabiduría hasta el máximo límite filosófico; y después se atreve a lanzarse a un viaje sin límites y sin fin por el universo en compañía únicamente de la verdad.

\par
%\textsuperscript{(1141.6)}
\textsuperscript{103:9.8} La ciencia (el conocimiento) está basada en la suposición inherente (ocasionada por el espíritu ayudante) de que la razón es válida, de que el universo puede ser comprendido. La filosofía (la comprensión coordinada) está basada en la suposición inherente (ocasionada por el espíritu de la sabiduría) de que la sabiduría es válida, de que el universo material puede ser coordinado con el espiritual. La religión (la verdad de la experiencia espiritual personal) está basada en la suposición inherente (ocasionada por el Ajustador del Pensamiento) de que la fe es válida, de que Dios puede ser conocido y alcanzado.

\par
%\textsuperscript{(1141.7)}
\textsuperscript{103:9.9} La comprensión completa de la realidad de la vida mortal consiste en un consentimiento progresivo a creer en estas suposiciones de la razón, la sabiduría y la fe. Una vida así está motivada por la verdad y dominada por el amor; estos son los ideales de la realidad cósmica objetiva, cuya existencia no se puede demostrar materialmente.

\par
%\textsuperscript{(1142.1)}
\textsuperscript{103:9.10} Una vez que la razón reconoce lo verdadero y lo falso, da muestras de sabiduría; cuando la sabiduría escoge entre lo verdadero y lo falso, entre la verdad y el error, demuestra la guía del espíritu. Así es como las funciones de la mente, el alma y el espíritu están siempre estrechamente unidas y funcionalmente interasociadas. La razón se ocupa del conocimiento basado en los hechos; la sabiduría se ocupa de la filosofía y la revelación; la fe se ocupa de la experiencia espiritual viviente. El hombre alcanza la belleza a través de la verdad, y por medio del amor espiritual asciende hacia la bondad.

\par
%\textsuperscript{(1142.2)}
\textsuperscript{103:9.11} La fe conduce a conocer a Dios, y no simplemente a un sentimiento místico de la presencia divina. La fe no debe estar influida excesivamente por sus consecuencias emotivas. La verdadera religión es una experiencia de creencia y de conocimiento, así como una satisfacción de los sentimientos.

\par
%\textsuperscript{(1142.3)}
\textsuperscript{103:9.12} Existe una realidad, en la experiencia religiosa, que es proporcional a su contenido espiritual, y esta realidad trasciende la razón, la ciencia, la filosofía, la sabiduría y todos los demás logros humanos. Las convicciones de esta experiencia son inatacables; la lógica de la vida religiosa es indiscutible; la certidumbre de este conocimiento es superhumana; las satisfacciones son magníficamente divinas, la valentía es indomable, las dedicaciones son incondicionales, las lealtades son supremas y los destinos son finales ---eternos, últimos y universales.

\par
%\textsuperscript{(1142.4)}
\textsuperscript{103:9.13} [Presentado por un Melquisedek de Nebadon.]


\chapter{Documento 104. El crecimiento del concepto de la Trinidad}
\par
%\textsuperscript{(1143.1)}
\textsuperscript{104:0.1} EL CONCEPTO de la Trinidad de la religión revelada no se debe confundir con las creencias en las tríadas de las religiones evolutivas. Las ideas de las tríadas surgieron de muchas relaciones sugerentes, pero principalmente porque los dedos tenían tres articulaciones, porque se necesitaba un mínimo de tres patas para estabilizar un taburete, porque tres puntos de apoyo podían sostener una tienda; además, el hombre primitivo no supo contar durante mucho tiempo más allá de tres.

\par
%\textsuperscript{(1143.2)}
\textsuperscript{104:0.2} Aparte de ciertos pareados naturales tales como el pasado y el presente, el día y la noche, el calor y el frío, lo masculino y lo femenino, el hombre tiende generalmente a pensar en tríadas: ayer, hoy y mañana; amanecer, mediodía y atardecer; padre, madre e hijo. Se dan tres vítores al vencedor. Los muertos son enterrados al tercer día, y se apacigua al fantasma mediante tres abluciones de agua.

\par
%\textsuperscript{(1143.3)}
\textsuperscript{104:0.3} La tríada hizo su aparición en la religión como consecuencia de estas asociaciones naturales en la experiencia humana, y esto sucedió mucho antes de que la Trinidad de las Deidades del Paraíso, o incluso algunos de sus representantes, fueran revelados a la humanidad. Más tarde, los persas, hindúes, griegos, egipcios, babilonios, romanos y escandinavos, todos tuvieron dioses que formaban tríadas, pero éstas no eran todavía verdaderas trinidades. Todas las deidades en tríadas tuvieron un origen natural y aparecieron en un momento u otro en la mayoría de los pueblos inteligentes de Urantia. A veces el concepto de una tríada evolutiva se ha mezclado con el de la Trinidad revelada; en estos casos, a menudo es imposible distinguir la una de la otra.

\section*{1. Los conceptos urantianos de la Trinidad}
\par
%\textsuperscript{(1143.4)}
\textsuperscript{104:1.1} La primera revelación urantiana que condujo a la comprensión de la Trinidad del Paraíso fue efectuada por el estado mayor del Príncipe Caligastia hace aproximadamente medio millón de años. Este primer concepto de la Trinidad se perdió para el mundo durante los tiempos agitados que siguieron a la rebelión planetaria.

\par
%\textsuperscript{(1143.5)}
\textsuperscript{104:1.2} La segunda presentación de la Trinidad fue realizada por Adán y Eva en el primero y segundo jardín. Estas enseñanzas no se habían perdido por completo ni siquiera en los tiempos de Maquiventa Melquisedek, cerca de treinta y cinco mil años más tarde, pues el concepto de los setitas sobre la Trinidad sobrevivió tanto en Mesopotamia como en Egipto, pero más especialmente en la India, donde fue perpetuado durante mucho tiempo en Agni, el dios védico tricéfalo del fuego.

\par
%\textsuperscript{(1143.6)}
\textsuperscript{104:1.3} La tercera presentación de la Trinidad fue efectuada por Maquiventa Melquisedek, y esta doctrina estaba simbolizada por los tres círculos concéntricos que el sabio de Salem llevaba en su pecho. Pero a Maquiventa le resultó muy difícil enseñarle cosas a los beduinos palestinos sobre el Padre Universal, el Hijo Eterno y el Espíritu Infinito. La mayoría de sus discípulos pensaban que la Trinidad consistía en los tres Altísimos de Norlatiadek; unos pocos concibieron que la Trinidad estaba compuesta por el Soberano del Sistema, el Padre de la Constelación y la Deidad Creadora del universo local; y aún menos discípulos todavía captaron remotamente la idea de la asociación paradisiaca del Padre, el Hijo y el Espíritu.

\par
%\textsuperscript{(1144.1)}
\textsuperscript{104:1.4} Las enseñanzas de Melquisedek sobre la Trinidad se difundieron gradualmente por una gran parte de Eurasia y el norte de África gracias a las actividades de los misioneros de Salem. A menudo es difícil distinguir entre las tríadas y las trinidades en la época más tardía de los anditas y en los tiempos posteriores a Melquisedek, cuando ambos conceptos se entremezclaron y fundieron hasta cierto punto.

\par
%\textsuperscript{(1144.2)}
\textsuperscript{104:1.5} Entre los hindúes, el concepto trinitario se arraigó bajo la forma de Ser, Inteligencia y Alegría. (Un concepto indio posterior fue el de Brahma, Siva y Vichnú.) Aunque las primeras descripciones de la Trinidad fueron llevadas hasta la India por los sacerdotes setitas, las ideas más recientes sobre la Trinidad fueron importadas por los misioneros de Salem y desarrolladas por los intelectos nativos de la India mediante una combinación de estas doctrinas con los conceptos evolutivos de la tríada.

\par
%\textsuperscript{(1144.3)}
\textsuperscript{104:1.6} La fe budista desarrolló dos doctrinas de naturaleza trinitaria: la primera fue Maestro, Ley y Fraternidad. Ésta fue la presentación realizada por Siddharta Gautama. La idea posterior, que se desarrolló en la rama septentrional de los seguidores de Buda, englobaba al Señor Supremo, al Espíritu Santo y al Salvador Encarnado.

\par
%\textsuperscript{(1144.4)}
\textsuperscript{104:1.7} Estas ideas de los hindúes y los budistas eran unos postulados realmente trinitarios, es decir, la idea de la triple manifestación de un Dios monoteísta. Un concepto verdaderamente trinitario no consiste simplemente en agrupar a tres dioses separados.

\par
%\textsuperscript{(1144.5)}
\textsuperscript{104:1.8} Los hebreos conocían el concepto de la Trinidad por medio de las tradiciones kenitas de los tiempos de Melquisedek, pero su ardor monoteísta por el Dios único Yahvé había eclipsado de tal manera todas estas enseñanzas, que en el momento de la aparición de Jesús la doctrina de los Elohim había sido prácticamente erradicada de la teología judía. La mente hebrea no podía conciliar el concepto trinitario con la creencia monoteísta en el Señor Único, el Dios de Israel.

\par
%\textsuperscript{(1144.6)}
\textsuperscript{104:1.9} Los seguidores de la fe islámica tampoco lograron captar la idea de la Trinidad. A un monoteísmo emergente siempre le resulta difícil tolerar el trinitarismo cuando se enfrenta con el politeísmo. La idea de la trinidad se afianza mejor en aquellas religiones que poseen una firme tradición monoteísta unida a una flexibilidad doctrinal. Los grandes monoteístas, los hebreos y los mahometanos, encontraron difícil distinguir entre la adoración de tres dioses (el politeísmo) y el trinitarismo, la adoración de una sola Deidad que existe bajo una manifestación trina de divinidad y de personalidad.

\par
%\textsuperscript{(1144.7)}
\textsuperscript{104:1.10} Jesús enseñó a sus apóstoles la verdad sobre las personas de la Trinidad del Paraíso, pero pensaron que les hablaba de manera figurada y simbólica. Como habían sido educados en el monoteísmo hebreo, les resultó difícil albergar cualquier creencia que pareciera estar en conflicto con su concepto dominante de Yahvé. Los primeros cristianos heredaron el prejuicio hebreo contra el concepto de la Trinidad.

\par
%\textsuperscript{(1144.8)}
\textsuperscript{104:1.11} La primera Trinidad del cristianismo fue proclamada en Antioquía y estaba compuesta por Dios, su Verbo y su Sabiduría\footnote{\textit{Visión original de la Trinidad de Pablo}: 1 Co 12:4-6.}. Pablo conocía la Trinidad paradisiaca del Padre, el Hijo y el Espíritu, pero raramente predicó sobre ella y sólo la mencionó en algunas de sus epístolas a las iglesias que se estaban formando. Incluso así, tal como les sucedió a sus compañeros apóstoles, Pablo confundió a Jesús, el Hijo Creador del universo local, con la Segunda Persona de la Deidad, el Hijo Eterno del Paraíso.

\par
%\textsuperscript{(1144.9)}
\textsuperscript{104:1.12} El concepto cristiano de la Trinidad, que empezó a conseguir reconocimiento hacia finales del siglo primero después de Cristo, incluía al Padre Universal, el Hijo Creador de Nebadon y la Divina Ministra de Salvington ---el Espíritu Madre del universo local y la consorte creativa del Hijo Creador\footnote{\textit{Concepto posterior de la Trinidad}: Mt 28:19; Hch 2:32-33; 2 Co 13:14; 1 Jn 5:7.}.

\par
%\textsuperscript{(1145.1)}
\textsuperscript{104:1.13} Desde los tiempos de Jesús, la verdadera identidad de la Trinidad del Paraíso no se ha conocido en Urantia (exceptuando a algunas personas a quienes les fue especialmente revelada) hasta la publicación de estas revelaciones. Pero aunque el concepto cristiano de la Trinidad estaba equivocado de hecho, era prácticamente verdadero en lo que se refiere a las relaciones espirituales. Este concepto sólo estaba confundido en sus implicaciones filosóficas y en sus consecuencias cosmológicas: A muchas personas con una mentalidad cósmica les ha resultado difícil creer que la Segunda Persona de la Deidad, el segundo miembro de una Trinidad infinita, residiera una vez en Urantia; y aunque esto sea cierto en espíritu, no es un hecho en la realidad. Los Migueles Creadores personifican plenamente la divinidad del Hijo Eterno, pero no son la personalidad absoluta.

\section*{2. La unidad de la Trinidad y la pluralidad de la Deidad}
\par
%\textsuperscript{(1145.2)}
\textsuperscript{104:2.1} El monoteísmo surgió como una protesta filosófica contra la inconsistencia del politeísmo. Primero se desarrolló a través de unas organizaciones de tipo panteón con una división departamental de las actividades sobrenaturales, luego a través de la exaltación henoteísta de un solo dios por encima de otros muchos, y finalmente excluyendo a todos los dioses excepto al Dios Único de valor final.

\par
%\textsuperscript{(1145.3)}
\textsuperscript{104:2.2} El trinitarismo tiene su origen en la protesta experiencial contra la imposibilidad de concebir la unicidad de una Deidad solitaria desprovista de antropomorfismo y de conexión con los significados universales. Con el tiempo suficiente, la filosofía tiende a hacer caso omiso de las cualidades personales contenidas en el concepto sobre la Deidad del puro monoteísmo, reduciendo así esta idea de un Dios inconexo al estado de un Absoluto panteísta. Siempre ha sido difícil comprender la naturaleza personal de un Dios que no tiene relaciones personales, en un pie de igualdad, con otros seres personales coordinados. La personalidad, en la Deidad, exige que dicha Deidad exista en relación con otra Deidad personal e igual.

\par
%\textsuperscript{(1145.4)}
\textsuperscript{104:2.3} Por medio del reconocimiento del concepto de la Trinidad, la mente del hombre puede esperar captar alguna cosa de las relaciones recíprocas entre el amor y la ley en las creaciones del espacio-tiempo. Por medio de la fe espiritual, el hombre consigue hacerse una idea del amor de Dios, pero pronto descubre que esta fe espiritual no tiene ninguna influencia sobre las leyes ordenadas del universo material. Independientemente de que el hombre crea con firmeza que Dios es su Padre Paradisiaco, los horizontes cósmicos en expansión exigen que reconozca también la realidad de que la Deidad del Paraíso es la ley universal, que reconozca la soberanía de la Trinidad, la cual se extiende desde el Paraíso hacia fuera y eclipsa incluso los universos locales evolutivos de los Hijos Creadores y de las Hijas Creativas de las tres personas eternas, cuya unión en deidad \textit{es} el hecho, la realidad y la indivisibilidad eterna de la Trinidad del Paraíso.

\par
%\textsuperscript{(1145.5)}
\textsuperscript{104:2.4} Esta misma Trinidad del Paraíso es una entidad real ---no es una personalidad, pero sin embargo es una realidad verdadera y absoluta; no es una personalidad, pero sin embargo es compatible con las personalidades coexistentes ---las personalidades del Padre, el Hijo y el Espíritu. La Trinidad es una realidad de la Deidad que supera la suma de sus componentes, y que surge de la conjunción de las tres Deidades del Paraíso. Las cualidades, características y funciones de la Trinidad no son la simple suma de los atributos de las tres Deidades del Paraíso; las funciones de la Trinidad son algo único, original y no del todo previsibles mediante el análisis de los atributos del Padre, el Hijo y el Espíritu.

\par
%\textsuperscript{(1146.1)}
\textsuperscript{104:2.5} Por ejemplo, cuando el Maestro estaba en la Tierra, advirtió a sus seguidores que la justicia nunca es un acto \textit{personal}; siempre es una función \textit{colectiva}. Los Dioses, como personas, tampoco administran la justicia, pero ejercen esta misma función como un todo colectivo, como la Trinidad del Paraíso.

\par
%\textsuperscript{(1146.2)}
\textsuperscript{104:2.6} La comprensión conceptual de la asociación trinitaria del Padre, el Hijo y el Espíritu prepara la mente humana para la presentación ulterior de otras ciertas relaciones triples. La razón teológica puede satisfacerse plenamente con el concepto de la Trinidad del Paraíso, pero la razón filosófica y cosmológica exige el reconocimiento de las otras asociaciones trinas de la Fuente-Centro Primera, de aquellas triunidades en las que el Infinito funciona en diversas capacidades no paternales de manifestación universal ---las relaciones del Dios de la fuerza, la energía, el poder, la causalidad, la reacción, la potencialidad, la actualidad, la gravedad, la tensión, el arquetipo, el principio y la unidad.

\section*{3. Las Trinidades y las triunidades}
\par
%\textsuperscript{(1146.3)}
\textsuperscript{104:3.1} Aunque a veces la humanidad ha intentado comprender la Trinidad de las tres personas de la Deidad, la coherencia exige que el intelecto humano perciba que existen ciertas relaciones entre los siete Absolutos. Pero todo aquello que es cierto respecto a la Trinidad del Paraíso, no lo es necesariamente respecto a una \textit{triunidad}, pues una triunidad es algo distinto a una trinidad. En algunos aspectos funcionales, una triunidad puede ser análoga a una trinidad, pero su naturaleza nunca es homóloga a la de una trinidad.

\par
%\textsuperscript{(1146.4)}
\textsuperscript{104:3.2} El hombre mortal está pasando en Urantia por una gran era de expansión de los horizontes y de ampliación de los conceptos, y la evolución de su filosofía cósmica debe acelerarse para mantenerse al mismo ritmo que la expansión del campo intelectual del pensamiento humano. A medida que se amplía la conciencia cósmica del hombre mortal, éste percibe la estrecha vinculación existente entre todo lo que encuentra en su ciencia material, su filosofía intelectual y su perspicacia espiritual. Sin embargo, junto con toda esta creencia en la unidad del cosmos, el hombre se percata de la diversidad de todo lo que existe. A pesar de todos los conceptos relacionados con la inmutabilidad de la Deidad, el hombre se da cuenta de que vive en un universo en constante cambio y en crecimiento experiencial. A pesar de que el hombre comprende que los valores espirituales sobrevivirán, siempre tiene que contar con las matemáticas y las prematemáticas de la fuerza, la energía y la potencia.

\par
%\textsuperscript{(1146.5)}
\textsuperscript{104:3.3} La eterna plenitud de la infinidad debe ser conciliada de alguna manera con el crecimiento temporal de los universos en evolución y con el estado incompleto de sus habitantes experienciales. El concepto de la infinitud total debe ser en cierto modo segmentado y limitado de tal manera, que el intelecto mortal y el alma morontial puedan captar este concepto de valor final y de significado espiritualizador.

\par
%\textsuperscript{(1146.6)}
\textsuperscript{104:3.4} Aunque la razón exige una unidad monoteísta de la realidad cósmica, la experiencia finita necesita el postulado de una pluralidad de Absolutos y de su coordinación en las relaciones cósmicas. La diversidad de las relaciones absolutas no tiene ninguna posibilidad de aparecer sin unas existencias coordinadas, y los factores diferenciales, variables, modificadores, atenuadores, limitadores o reductores no tienen ninguna probabilidad de funcionar.

\par
%\textsuperscript{(1146.7)}
\textsuperscript{104:3.5} La realidad total (la infinidad) ha sido presentada en estos documentos tal como existe en los siete Absolutos:

\par
%\textsuperscript{(1146.8)}
\textsuperscript{104:3.6} 1. El Padre Universal.

\par
%\textsuperscript{(1146.9)}
\textsuperscript{104:3.7} 2. El Hijo Eterno.

\par
%\textsuperscript{(1146.10)}
\textsuperscript{104:3.8} 3. El Espíritu Infinito.

\par
%\textsuperscript{(1147.1)}
\textsuperscript{104:3.9} 4. La Isla del Paraíso.

\par
%\textsuperscript{(1147.2)}
\textsuperscript{104:3.10} 5. El Absoluto de la Deidad.

\par
%\textsuperscript{(1147.3)}
\textsuperscript{104:3.11} 6. El Absoluto Universal.

\par
%\textsuperscript{(1147.4)}
\textsuperscript{104:3.12} 7. El Absoluto Incalificado.

\par
%\textsuperscript{(1147.5)}
\textsuperscript{104:3.13} La Fuente-Centro Primera, que es Padre para el Hijo Eterno, es también Arquetipo para la Isla del Paraíso. Es personalidad incalificada en el Hijo, pero personalidad potencial en el Absoluto de la Deidad. El Padre es energía revelada en el Paraíso-Havona y al mismo tiempo energía oculta en el Absoluto Incalificado. El Infinito se revela siempre en los actos incesantes del Actor Conjunto, mientras que ejerce eternamente sus funciones en las actividades compensadoras, pero disimuladas, del Absoluto Universal. Así pues, el Padre está relacionado con los seis Absolutos coordinados, y el conjunto de los siete abarca así el círculo de la infinidad a lo largo de todos los ciclos interminables de la eternidad.

\par
%\textsuperscript{(1147.6)}
\textsuperscript{104:3.14} Parece ser que las relaciones absolutas conducen inevitablemente a una triunidad. Las personalidades tratan de asociarse con otras personalidades tanto en los niveles absolutos como en todos los otros niveles. Y la asociación de las tres personalidades paradisiacas eterniza la primera triunidad, la unión entre las personalidades del Padre, el Hijo y el Espíritu. Pues cuando estas tres personas se unen, \textit{como personas}, para actuar de manera unida, constituyen de este modo una triunidad de unidad funcional; no es una trinidad ---una entidad orgánica--- pero no obstante sí es una triunidad, una triple unanimidad colectiva funcional.

\par
%\textsuperscript{(1147.7)}
\textsuperscript{104:3.15} La Trinidad del Paraíso no es una triunidad; no es una unanimidad funcional; es más bien una Deidad indivisa e indivisible. El Padre, el Hijo y el Espíritu (como personas) pueden mantener relaciones con la Trinidad del Paraíso, porque la Trinidad \textit{es} su Deidad indivisa. El Padre, el Hijo y el Espíritu no mantienen este tipo de relaciones personales con la primera triunidad, porque ésta \textit{es} su unión funcional como tres personas. Sólo como Trinidad ---como una Deidad indivisa--- mantienen colectivamente una relación externa con la triunidad de su unión personal.

\par
%\textsuperscript{(1147.8)}
\textsuperscript{104:3.16} Así es como la Trinidad del Paraíso es única entre todas las relaciones absolutas; hay varias triunidades existenciales, pero sólo una Trinidad existencial. Una triunidad \textit{no} es una entidad. Es más bien funcional que orgánica. Sus miembros son asociados más bien que corporativos. Los componentes de las triunidades pueden ser entidades, pero la triunidad misma es una asociación.

\par
%\textsuperscript{(1147.9)}
\textsuperscript{104:3.17} Existe sin embargo un punto de comparación entre una trinidad y una triunidad: las dos terminan en funciones que son otra cosa distinta a la suma discernible de los atributos de los miembros que las componen. Pero aunque se puedan comparar así desde un punto de vista funcional, no manifiestan por lo demás ninguna relación categórica. Están más o menos relacionadas como la relación que existe entre la función y la estructura. Pero la función de una asociación triunitaria no es la función de una estructura o entidad trinitaria.

\par
%\textsuperscript{(1147.10)}
\textsuperscript{104:3.18} Sin embargo, las triunidades son reales; son muy reales. En ellas, la realidad total está funcionalizada, y a través de ellas, el Padre Universal ejerce un control inmediato y personal sobre las actividades principales de la infinidad.

\section*{4. Las siete triunidades}
\par
%\textsuperscript{(1147.11)}
\textsuperscript{104:4.1} Al intentar describir las siete triunidades, dirigimos la atención hacia el hecho de que el Padre Universal es el miembro fundamental de cada una de ellas. Él es, era y siempre será el Primer Padre-Fuente Universal, el Centro Absoluto, la Causa Primordial, el Controlador Universal, el Activador Ilimitado, la Unidad Original, el Sostén Incalificado, la Primera Persona de la Deidad, el Arquetipo Cósmico Primordial y la Esencia de la Infinidad. El Padre Universal es la causa personal de los Absolutos; él es el absoluto de los Absolutos.

\par
%\textsuperscript{(1148.1)}
\textsuperscript{104:4.2} La naturaleza y el significado de las siete triunidades se pueden indicar como sigue:

\par
%\textsuperscript{(1148.2)}
\textsuperscript{104:4.3} \textit{La Primera Triunidad} ---\textit{la triunidad personal e intencional}. Es la agrupación de las tres personalidades de la Deidad:

\par
%\textsuperscript{(1148.3)}
\textsuperscript{104:4.4} 1. El Padre Universal.

\par
%\textsuperscript{(1148.4)}
\textsuperscript{104:4.5} 2. El Hijo Eterno.

\par
%\textsuperscript{(1148.5)}
\textsuperscript{104:4.6} 3. El Espíritu Infinito.

\par
%\textsuperscript{(1148.6)}
\textsuperscript{104:4.7} Es la triple unión del amor, la misericordia y el ministerio ---la asociación intencional y personal de las tres personalidades eternas del Paraíso. Es la asociación divinamente fraternal, que ama a las criaturas, actúa paternalmente y fomenta la ascensión. Las personalidades divinas de esta primera triunidad son los Dioses que transmiten la personalidad, conceden el espíritu y donan la mente.

\par
%\textsuperscript{(1148.7)}
\textsuperscript{104:4.8} Es la triunidad de la volición infinita; actúa a lo largo del eterno presente y en todo el transcurso pasado-presente-futuro del tiempo. Esta asociación produce la infinidad volitiva y proporciona los mecanismos a través de los cuales la Deidad personal puede revelarse a las criaturas del cosmos evolutivo.

\par
%\textsuperscript{(1148.8)}
\textsuperscript{104:4.9} \textit{La Segunda Triunidad} ---\textit{la triunidad de la potencia y el arquetipo}. Ya se trate de un minúsculo ultimatón, de una estrella resplandeciente o de una nebulosa en rotación, e incluso del universo central o de los superuniversos, desde las organizaciones materiales más pequeñas hasta las más grandes, el arquetipo físico ---la configuración cósmica--- procede siempre de la actividad de esta triunidad. Esta asociación consta de:

\par
%\textsuperscript{(1148.9)}
\textsuperscript{104:4.10} 1. El Padre-Hijo.

\par
%\textsuperscript{(1148.10)}
\textsuperscript{104:4.11} 2. La Isla del Paraíso.

\par
%\textsuperscript{(1148.11)}
\textsuperscript{104:4.12} 3. El Actor Conjunto.

\par
%\textsuperscript{(1148.12)}
\textsuperscript{104:4.13} La energía es organizada por los agentes cósmicos de la Fuente-Centro Tercera; la energía es moldeada según el arquetipo del Paraíso, que es la materialización absoluta; pero detrás de toda esta manipulación incesante se encuentra la presencia del Padre-Hijo, cuya unión activó por primera vez el arquetipo del Paraíso provocando la aparición de Havona que estuvo acompañada por el nacimiento del Espíritu Infinito, el Actor Conjunto.

\par
%\textsuperscript{(1148.13)}
\textsuperscript{104:4.14} En la experiencia religiosa, las criaturas se ponen en contacto con el Dios que es amor, pero esta perspicacia espiritual nunca debe eclipsar el reconocimiento inteligente del hecho universal de que el Paraíso es un arquetipo. Las personalidades del Paraíso consiguen la adoración voluntaria de todas las criaturas mediante el poder irresistible del amor divino, y conducen a todas estas personalidades nacidas del espíritu a las delicias celestiales del servicio interminable de los hijos finalitarios de Dios. La segunda triunidad es el arquitecto del escenario espacial donde se desarrollan estas actividades; ella determina los arquetipos de la configuración cósmica.

\par
%\textsuperscript{(1148.14)}
\textsuperscript{104:4.15} El amor puede caracterizar a la divinidad de la primera triunidad, pero el arquetipo es la manifestación galáctica de la segunda triunidad. La primera triunidad es para las personalidades evolutivas lo que la segunda es para los universos en evolución. El arquetipo y la personalidad son dos de las grandes manifestaciones de las actividades de la Fuente-Centro Primera; y por muy difícil que sea de comprender, sin embargo es cierto que la potencia-arquetipo y la persona amorosa son una sola y misma realidad universal; la Isla del Paraíso y el Hijo Eterno son revelaciones coordinadas, pero antípodas, de la naturaleza insondable del Padre-Fuerza Universal.

\par
%\textsuperscript{(1149.1)}
\textsuperscript{104:4.16} \textit{La Tercera Triunidad} ---\textit{la triunidad que hace evolucionar el espíritu}. La totalidad de la manifestación espiritual tiene su principio y su final en esta asociación, que está compuesta de:

\par
%\textsuperscript{(1149.2)}
\textsuperscript{104:4.17} 1. El Padre Universal.

\par
%\textsuperscript{(1149.3)}
\textsuperscript{104:4.18} 2. El Hijo-Espíritu.

\par
%\textsuperscript{(1149.4)}
\textsuperscript{104:4.19} 3. El Absoluto de la Deidad.

\par
%\textsuperscript{(1149.5)}
\textsuperscript{104:4.20} Desde la potencia espiritual hasta el espíritu paradisiaco, todo espíritu encuentra la expresión de su realidad en esta asociación trina entre la pura esencia espiritual del Padre, los valores espirituales activos del Hijo-Espíritu, y los potenciales espirituales ilimitados del Absoluto de la Deidad. Los valores existenciales del espíritu tienen su génesis primordial, su manifestación completa y su destino final en esta triunidad.

\par
%\textsuperscript{(1149.6)}
\textsuperscript{104:4.21} El Padre existe antes que el espíritu; el Hijo-Espíritu actúa como espíritu creador activo; el Absoluto de la Deidad existe como espíritu que lo engloba todo, incluyendo lo que está más allá del espíritu.

\par
%\textsuperscript{(1149.7)}
\textsuperscript{104:4.22} \textit{La Cuarta Triunidad} ---\textit{la triunidad de la infinidad energética}. Dentro de esta triunidad se eternizan los principios y los finales de toda realidad energética, desde la potencia espacial hasta la monota. Esta agrupación contiene los miembros siguientes:

\par
%\textsuperscript{(1149.8)}
\textsuperscript{104:4.23} 1. El Padre-Espíritu.

\par
%\textsuperscript{(1149.9)}
\textsuperscript{104:4.24} 2. La Isla del Paraíso.

\par
%\textsuperscript{(1149.10)}
\textsuperscript{104:4.25} 3. El Absoluto Incalificado.

\par
%\textsuperscript{(1149.11)}
\textsuperscript{104:4.26} El Paraíso es el centro que activa, mediante la energía-fuerza, el cosmos ---el emplazamiento universal de la Fuente-Centro Primera, el punto focal cósmico del Absoluto Incalificado, y la fuente de toda energía. El potencial energético del cosmos infinito se encuentra existencialmente presente en esta triunidad; el gran universo y el universo maestro sólo son manifestaciones parciales de dicho potencial.

\par
%\textsuperscript{(1149.12)}
\textsuperscript{104:4.27} La cuarta triunidad controla absolutamente las unidades fundamentales de la energía cósmica, y las libera del control del Absoluto Incalificado de manera directamente proporcional a la aparición, en las Deidades experienciales, de la capacidad subabsoluta para controlar y estabilizar el cosmos en metamorfosis.

\par
%\textsuperscript{(1149.13)}
\textsuperscript{104:4.28} Esta triunidad \textit{es} la fuerza y la energía. Las posibilidades ilimitadas del Absoluto Incalificado están centradas alrededor del absolutum de la Isla del Paraíso, de donde emanan unas agitaciones inimaginables procedentes de la quietud, por otra parte estática, del Incalificado. Las palpitaciones interminables del Paraíso, corazón material del cosmos infinito, laten en armonía con el arquetipo insondable y el plan impenetrable del Activador Infinito, la Fuente-Centro Primera.

\par
%\textsuperscript{(1149.14)}
\textsuperscript{104:4.29} \textit{La Quinta Triunidad} ---\textit{la triunidad de la infinidad reactiva}. Esta asociación consta de:

\par
%\textsuperscript{(1149.15)}
\textsuperscript{104:4.30} 1. El Padre Universal.

\par
%\textsuperscript{(1149.16)}
\textsuperscript{104:4.31} 2. El Absoluto Universal.

\par
%\textsuperscript{(1149.17)}
\textsuperscript{104:4.32} 3. El Absoluto Incalificado.

\par
%\textsuperscript{(1149.18)}
\textsuperscript{104:4.33} Esta agrupación eterniza la realización funcional, en la infinidad, de todo lo que es manifestable dentro del ámbito de la realidad no deificada. Esta triunidad manifiesta una capacidad de reacción ilimitada a las acciones y presencias volitivas, causativas, tensionales y arquetípicas de las otras triunidades.

\par
%\textsuperscript{(1150.1)}
\textsuperscript{104:4.34} \textit{La Sexta Triunidad} ---\textit{la triunidad de la Deidad en asociación cósmica}. Este grupo está compuesto por:

\par
%\textsuperscript{(1150.2)}
\textsuperscript{104:4.35} 1. El Padre Universal.

\par
%\textsuperscript{(1150.3)}
\textsuperscript{104:4.36} 2. El Absoluto de la Deidad.

\par
%\textsuperscript{(1150.4)}
\textsuperscript{104:4.37} 3. El Absoluto Universal.

\par
%\textsuperscript{(1150.5)}
\textsuperscript{104:4.38} Ésta es la asociación de la Deidad-en-el-cosmos, la inmanencia de la Deidad en conjunción con la trascendencia de la Deidad. Es la última extensión de la divinidad, en los niveles de la infinidad, hacia aquellas realidades que se encuentran fuera del ámbito de la realidad deificada.

\par
%\textsuperscript{(1150.6)}
\textsuperscript{104:4.39} \textit{La Séptima Triunidad} ---\textit{la triunidad de la unidad infinita}. Ésta es la unidad de la infinidad, manifiesta funcionalmente en el tiempo y en la eternidad, la unificación coordinada de los actuales y los potenciales. Este grupo consta de:

\par
%\textsuperscript{(1150.7)}
\textsuperscript{104:4.40} 1. El Padre Universal.

\par
%\textsuperscript{(1150.8)}
\textsuperscript{104:4.41} 2. El Actor Conjunto.

\par
%\textsuperscript{(1150.9)}
\textsuperscript{104:4.42} 3. El Absoluto Universal.

\par
%\textsuperscript{(1150.10)}
\textsuperscript{104:4.43} El Actor Conjunto integra universalmente los aspectos funcionales variables de toda la realidad efectiva en todos los niveles de manifestación, desde los finitos y los trascendentales hasta los absolutos. El Absoluto Universal compensa perfectamente los diferenciales inherentes a los aspectos variables de toda la realidad incompleta, desde las potencialidades ilimitadas de la realidad activo-volitiva y causativa de la Deidad, hasta las posibilidades sin límites de la realidad estática, reactiva y no deificada en los ámbitos incomprensibles del Absoluto Incalificado.

\par
%\textsuperscript{(1150.11)}
\textsuperscript{104:4.44} Tal como actúan en esta triunidad, el Actor Conjunto y el Absoluto Universal son igualmente sensibles tanto a la presencia de la Deidad como a la de la no deidad, al igual que lo es también la Fuente-Centro Primera, la cual, en esta relación, es prácticamente imposible de distinguir conceptualmente del YO SOY.

\par
%\textsuperscript{(1150.12)}
\textsuperscript{104:4.45} Estas aproximaciones son suficientes para dilucidar el concepto de las triunidades. Como no conocéis el nivel último de las triunidades, no podéis comprender plenamente los siete primeros. Aunque no estimamos que sea acertado intentar ofrecer una explicación adicional, podemos indicar que existen quince asociaciones trinas de la Fuente-Centro Primera, ocho de las cuales no se han revelado en estos documentos. Estas asociaciones no reveladas se ocupan de unas realidades, manifestaciones y potencialidades que se encuentran más allá del nivel experiencial de la supremacía.

\par
%\textsuperscript{(1150.13)}
\textsuperscript{104:4.46} Las triunidades son el timón funcional de la infinidad, la unificación de la unicidad de los Siete Absolutos de la Infinidad. La presencia existencial de las triunidades es la que permite al Padre-YO SOY experimentar la unidad funcional de la infinidad, a pesar de la diversificación de la infinidad en siete Absolutos. La Fuente-Centro Primera es el miembro unificador de todas las triunidades; en él todas las cosas tienen su comienzo incalificado, su existencia eterna y su destino infinito ---<<en él, todas las cosas consisten>>\footnote{\textit{En él, todas las cosas consisten}: Ro 11:36; Col 1:17; Heb 2:10.}.

\par
%\textsuperscript{(1150.14)}
\textsuperscript{104:4.47} Aunque estas asociaciones no puedan aumentar la infinidad del Padre-YO SOY, parece que hacen posible las manifestaciones subinfinitas y subabsolutas de su realidad. Las siete triunidades multiplican la diversidad de talentos, eternizan nuevas profundidades, deifican nuevos valores, desvelan nuevas potencialidades, revelan nuevos significados. Todas estas manifestaciones diversificadas en el tiempo y el espacio, y en el cosmos eterno, tienen su existencia en la estasis hipotética de la infinidad original del YO SOY.

\section*{5. Las triodidades}
\par
%\textsuperscript{(1151.1)}
\textsuperscript{104:5.1} Existen algunas otras relaciones trinas que no contienen al Padre en su constitución, pero no son verdaderas triunidades, y están siempre diferenciadas de las triunidades del Padre. Se les llama de manera diversa triunidades asociadas, triunidades coordinadas y \textit{triodidades}. Son una consecuencia de la existencia de las triunidades. Dos de estas asociaciones están constituidas como sigue:

\par
%\textsuperscript{(1151.2)}
\textsuperscript{104:5.2} \textit{La Triodidad de lo Manifestado}. Esta triodidad consiste en las relaciones recíprocas entre los tres actuales absolutos:

\par
%\textsuperscript{(1151.3)}
\textsuperscript{104:5.3} 1. El Hijo Eterno.

\par
%\textsuperscript{(1151.4)}
\textsuperscript{104:5.4} 2. La Isla del Paraíso.

\par
%\textsuperscript{(1151.5)}
\textsuperscript{104:5.5} 3. El Actor Conjunto.

\par
%\textsuperscript{(1151.6)}
\textsuperscript{104:5.6} El Hijo Eterno es el absoluto de la realidad espiritual, la personalidad absoluta. La Isla del Paraíso es el absoluto de la realidad cósmica, el arquetipo absoluto. El Actor Conjunto es el absoluto de la realidad mental, el coordinado de la realidad espiritual absoluta y la síntesis existencial, bajo la forma de Deidad, de la personalidad y el poder. Esta asociación trina produce la coordinación de la suma total de la realidad efectiva ---espiritual, cósmica o mental. Su estado de manifestación es incalificado.

\par
%\textsuperscript{(1151.7)}
\textsuperscript{104:5.7} \textit{La Triodidad de Potencialidad}. Esta triodidad consiste en la asociación de los tres Absolutos de potencialidad:

\par
%\textsuperscript{(1151.8)}
\textsuperscript{104:5.8} 1. El Absoluto de la Deidad.

\par
%\textsuperscript{(1151.9)}
\textsuperscript{104:5.9} 2. El Absoluto Universal.

\par
%\textsuperscript{(1151.10)}
\textsuperscript{104:5.10} 3. El Absoluto Incalificado.

\par
%\textsuperscript{(1151.11)}
\textsuperscript{104:5.11} Los depósitos infinitos de toda la realidad energética latente ---espiritual, mental o cósmica--- se encuentran interasociados de esta manera. Esta asociación produce la integración de toda la realidad energética latente. Su potencial es infinito.

\par
%\textsuperscript{(1151.12)}
\textsuperscript{104:5.12} Al igual que las triunidades se ocupan sobre todo de unificar funcionalmente la infinidad, las triodidades están implicadas en la aparición cósmica de las Deidades experienciales. Las triunidades se ocupan indirectamente de las Deidades experienciales ---Suprema, Última y Absoluta--- pero las triodidades se ocupan directamente de ellas. Aparecen en la síntesis emergente compuesta por el poder y la personalidad del Ser Supremo. Para las criaturas temporales del espacio, el Ser Supremo es una revelación de la unidad del YO SOY.

\par
%\textsuperscript{(1151.13)}
\textsuperscript{104:5.13} [Presentado por un Melquisedek de Nebadon.]


\chapter{Documento 105. La Deidad y la realidad}
\par
%\textsuperscript{(1152.1)}
\textsuperscript{105:0.1} INCLUSO para las órdenes elevadas de inteligencias del universo, la infinidad sólo es parcialmente comprensible y la finalidad de la realidad sólo es relativamente inteligible. Cuando la mente humana trata de penetrar en el misterio y la eternidad del origen y el destino de todo lo que llamamos \textit{real}, puede resultarle útil abordar el problema imaginando la eternidad y la infinidad como una elipse casi ilimitada producida por una sola causa absoluta, que ejerce su actividad en todo este círculo universal de diversificación interminable persiguiendo siempre algún potencial de destino absoluto e infinito.

\par
%\textsuperscript{(1152.2)}
\textsuperscript{105:0.2} Cuando el intelecto mortal intenta captar el concepto de la totalidad de la realidad, esa mente finita se encuentra cara a cara con la realidad de la infinidad. La totalidad de la realidad \textit{es} la infinidad, y por eso nunca puede ser plenamente comprendida por una mente que posea una capacidad conceptual subinfinita.

\par
%\textsuperscript{(1152.3)}
\textsuperscript{105:0.3} La mente humana no se puede formar un concepto muy adecuado de las existencias eternas, y a falta de esta comprensión, nos resulta imposible describir nuestros propios conceptos sobre la totalidad de la realidad. Sin embargo, podemos intentar presentarlos, aunque somos plenamente conscientes de que nuestros conceptos deberán sufrir una profunda deformación en el proceso de traducción y modificación para ponerlos al nivel de comprensión de la mente mortal.

\section*{1. El concepto filosófico del YO SOY}
\par
%\textsuperscript{(1152.4)}
\textsuperscript{105:1.1} Los filósofos del universo atribuyen la causalidad original absoluta en la infinidad al Padre Universal, actuando como el YO SOY infinito\footnote{\textit{YO SOY}: Ex 3:13-14.}, eterno y absoluto.

\par
%\textsuperscript{(1152.5)}
\textsuperscript{105:1.2} Presentar al intelecto mortal esta idea de un YO SOY infinito comporta muchos factores de peligro, ya que este concepto está tan alejado de la comprensión experiencial humana que ocasiona una grave deformación de los significados y una idea falsa de los valores. Sin embargo, el concepto filosófico del YO SOY proporciona a los seres finitos una base para intentar acceder a la comprensión parcial de los orígenes absolutos y de los destinos infinitos. Pero en todos nuestros esfuerzos por dilucidar la génesis y la fructificación de la realidad debemos indicar claramente que, en todo lo referente a los significados y valores de la personalidad, este concepto del YO SOY es sinónimo de la Primera Persona de la Deidad, el Padre Universal de todas las personalidades. Sin embargo, este postulado del YO SOY no es tan fácil de identificar en los ámbitos no deificados de la realidad universal.

\par
%\textsuperscript{(1152.6)}
\textsuperscript{105:1.3} \textit{El YO SOY es el Infinito; el YO SOY es también la infinidad}. Desde el punto de vista temporal o secuencial, toda la realidad tiene su origen en el infinito YO SOY, cuya existencia solitaria en la infinita eternidad del pasado ha de ser el primer postulado filosófico de las criaturas finitas. El concepto del YO SOY implica \textit{una infinidad incalificada}, la realidad no diferenciada de todo lo que podría existir alguna vez en toda una eternidad infinita.

\par
%\textsuperscript{(1153.1)}
\textsuperscript{105:1.4} Como concepto existencial, el YO SOY no es ni deificado ni no deificado, ni actual ni potencial, ni personal ni impersonal, ni estático ni dinámico. No se puede aplicar ningún calificativo al Infinito, salvo afirmar que el YO SOY \textit{es}. El postulado filosófico del YO SOY es un concepto universal algo más difícil de comprender que el del Absoluto Incalificado.

\par
%\textsuperscript{(1153.2)}
\textsuperscript{105:1.5} Para la mente finita debe haber simplemente un principio, y aunque la realidad nunca ha tenido realmente un principio, sin embargo la realidad manifiesta ciertas relaciones de origen con la infinidad. La situación primordial en la eternidad, anterior a la realidad, se puede imaginar más o menos como sigue: En un momento hipotético e infinitamente lejano de la eternidad pasada, se podría concebir al YO SOY como cosa y no cosa, como causa y efecto, como volición y reacción. En ese momento hipotético de la eternidad no existe ninguna diferenciación en toda la infinidad. La infinidad está colmada por el Infinito; el Infinito envuelve a la infinidad. Es el momento estático hipotético de la eternidad; los actuales están todavía contenidos en sus potenciales, y los potenciales aún no han aparecido dentro de la infinidad del YO SOY. Pero incluso en esa situación hipotética, debemos suponer que existe la posibilidad de la voluntad autónoma.

\par
%\textsuperscript{(1153.3)}
\textsuperscript{105:1.6} Recordad siempre que la comprensión humana del Padre Universal es una experiencia personal. Dios, como vuestro Padre espiritual, puede ser comprendido por vosotros y por todos los demás mortales. Pero \textit{vuestroconcepto cultual experiencial del Padre Universal siempre será menor quevuestro postulado filosófico de la infinidad de la Fuente-Centro Primera, elYO SOY}. Cuando hablamos del Padre, nos referimos a Dios tal como puede ser comprendido por sus criaturas tanto humildes como elevadas, pero la Deidad contiene muchas más cosas que son incomprensibles para las criaturas del universo. Dios, vuestro Padre y mi Padre, es esa fase del Infinito que percibimos en nuestra personalidad como una realidad experiencial efectiva, pero el YO SOY sigue siendo como nuestra hipótesis de todo lo que sentimos que es incognoscible en la Fuente-Centro Primera. E incluso esta hipótesis se encuentra probablemente muy por debajo de la infinidad insondable de la realidad original.

\par
%\textsuperscript{(1153.4)}
\textsuperscript{105:1.7} El universo de universos, con su innumerable multitud de personalidades que lo habitan, es un organismo inmenso y complejo, pero la Fuente-Centro Primera es infinitamente más compleja que los universos y las personalidades que han surgido a la realidad en respuesta a sus mandatos deliberados. Cuando contempláis con temor la magnitud del universo maestro, deteneos a considerar que incluso esta creación inconcebible no puede ser más que una revelación parcial del Infinito.

\par
%\textsuperscript{(1153.5)}
\textsuperscript{105:1.8} La infinidad está en verdad muy lejos del nivel experiencial de la comprensión mortal, pero incluso en esta época de Urantia vuestros conceptos sobre la infinidad están creciendo, y continuarán creciendo durante toda vuestra carrera sin fin que se extiende hacia adelante en la eternidad futura. La infinidad incalificada carece de sentido para las criaturas finitas, pero la infinidad es capaz de autolimitarse y es susceptible de expresar la realidad en todos los niveles de las existencias universales. Y el rostro que muestra el Infinito a todas las personalidades del universo es el rostro de un Padre, el Padre Universal del amor.

\section*{2. El YO SOY como trino y séptuple}
\par
%\textsuperscript{(1153.6)}
\textsuperscript{105:2.1} Al examinar la génesis de la realidad, tened siempre presente que toda la realidad absoluta procede de la eternidad y que su existencia no tiene principio. Cuando decimos realidad absoluta, nos referimos a las tres personas existenciales de la Deidad, a la Isla del Paraíso y a los tres Absolutos. Estas siete realidades son eternas de una manera coordinada, a pesar de que recurrimos al lenguaje del espacio-tiempo para presentar sus orígenes secuenciales a los seres humanos.

\par
%\textsuperscript{(1154.1)}
\textsuperscript{105:2.2} Al seguir la descripción cronológica de los orígenes de la realidad, debe existir un supuesto momento teórico en el que se produce la <<primera>> expresión volitiva y la <<primera>> reacción repercusiva dentro del YO SOY. En nuestro intento por describir la génesis y la generación de la realidad, esta etapa se puede concebir como aquella en la que \textit{El Uno Infinito} se diferencia de \textit{LaInfinitud}, pero el postulado de esta relación doble debe siempre ampliarse hasta un concepto trino mediante el reconocimiento del continuum eterno de \textit{LaInfinidad}, del YO SOY.

\par
%\textsuperscript{(1154.2)}
\textsuperscript{105:2.3} Esta autometamorfosis del YO SOY culmina en la múltiple diferenciación de la realidad deificada y de la realidad no deificada, de la realidad potencial y actual, y de algunas otras realidades que apenas pueden clasificarse de esta manera. Estas diferenciaciones del YO SOY teórico y monista están eternamente integradas gracias a las relaciones simultáneas que surgen dentro del mismo YO SOY ---la prerrealidad prepotencial, preactual, prepersonal y de un solo elemento que, aún siendo infinita, se revela como absoluta en la presencia de la Fuente-Centro Primera, y como personalidad en el amor ilimitado del Padre Universal.

\par
%\textsuperscript{(1154.3)}
\textsuperscript{105:2.4} Por medio de estas metamorfosis internas, el YO SOY establece las bases para una relación séptuple consigo mismo. Ahora, el concepto filosófico (temporal) del YO SOY solitario y el concepto transitorio (temporal) del YO SOY como trino se pueden ampliar hasta abarcar al YO SOY como séptuple. Esta naturaleza séptuple ---o de siete fases--- se puede sugerir mejor relacionándola con los Siete Absolutos de la Infinidad:

\par
%\textsuperscript{(1154.4)}
\textsuperscript{105:2.5} 1. \textit{El Padre Universal}. YO SOY el padre del Hijo Eterno. Ésta es la relación original de personalidad entre las realidades. La personalidad absoluta del Hijo hace absoluto el hecho de la paternidad de Dios y establece la filiación potencial de todas las personalidades. Esta relación demuestra la personalidad del Infinito y culmina su revelación espiritual en la personalidad del Hijo Original. Incluso los mortales que viven todavía en la carne pueden experimentar parcialmente, en los niveles espirituales, esta fase del YO SOY, puesto que pueden adorar a nuestro Padre.

\par
%\textsuperscript{(1154.5)}
\textsuperscript{105:2.6} 2. \textit{El Controlador Universal}. YO SOY la causa del Paraíso eterno. Ésta es la relación primordial impersonal entre las realidades, la asociación original no espiritual. El Padre Universal es Dios-como-amor; el Controlador Universal es Dios-como-arquetipo. Esta relación establece el potencial de las formas ---de las configuraciones--- y determina el arquetipo maestro de las relaciones impersonales y no espirituales ---el arquetipo maestro que sirve para crear todas las copias.

\par
%\textsuperscript{(1154.6)}
\textsuperscript{105:2.7} 3. \textit{El Creador Universal}. YO SOY uno con el Hijo Eterno. Esta unión del Padre y del Hijo (en presencia del Paraíso) pone en marcha el ciclo creativo, el cual culmina en la aparición de la personalidad conjunta y del universo eterno. Desde el punto de vista de los mortales finitos, la realidad tiene sus verdaderos comienzos con la aparición, en la eternidad, de la creación de Havona. Este acto creativo de la Deidad lo efectúa y se produce a través del Dios de Acción, que es en esencia la unidad del Padre y del Hijo, manifestada en y para todos los niveles de lo actual. Por consiguiente, la creatividad divina está caracterizada infaliblemente por la unidad, y esta unidad es el reflejo exterior de la unicidad absoluta de la dualidad Padre-Hijo y de la Trinidad Padre-Hijo-Espíritu.

\par
%\textsuperscript{(1155.1)}
\textsuperscript{105:2.8} 4. \textit{El Sostén Infinito}. YO SOY autoasociable. Ésta es la asociación primordial de los aspectos estáticos y potenciales de la realidad. En esta relación, todos los factores calificados e incalificados están compensados. Esta fase del YO SOY se comprende mejor como Absoluto Universal ---el unificador del Absoluto de la Deidad y del Absoluto Incalificado.

\par
%\textsuperscript{(1155.2)}
\textsuperscript{105:2.9} 5. \textit{El Potencial Infinito}. YO SOY autocalificado. Éste es el punto de referencia de la infinidad que atestigua eternamente que el YO SOY se ha limitado voluntariamente, en virtud de lo cual ha conseguido expresarse y revelarse de forma triple. Esta fase del YO SOY se comprende generalmente como Absoluto de la Deidad.

\par
%\textsuperscript{(1155.3)}
\textsuperscript{105:2.10} 6. \textit{La Capacidad Infinita}. YO SOY estático-reactivo. Ésta es la matriz sin fin, la posibilidad de todas las expansiones cósmicas futuras. La mejor manera de concebir esta fase del YO SOY es quizás la presencia supergravitatoria del Absoluto Incalificado.

\par
%\textsuperscript{(1155.4)}
\textsuperscript{105:2.11} 7. \textit{El Uno Universal de la Infinidad}. El YO SOY como YO SOY. Ésta es la estasis o relación de la Infinidad consigo misma, el hecho eterno de la realidad de la infinidad y la verdad universal de la infinidad de la realidad. En la medida en que esta relación es discernible como personalidad, es revelada a los universos en el Padre divino de toda personalidad ---incluso de la personalidad absoluta. En la medida en que es posible expresar esta relación de manera impersonal, el universo contacta con ella bajo la forma de la coherencia absoluta de la energía pura y del puro espíritu en presencia del Padre Universal. En la medida en que esta relación es concebible como un absoluto, es revelada en la primacía de la Fuente-Centro Primera; en ella todos vivimos, nos movemos y tenemos nuestra existencia\footnote{\textit{Vivimos, nos movemos y tenemos nuestra existencia}: Job 12:10; Hch 17:28.}, desde las criaturas del espacio hasta los ciudadanos del Paraíso. Y esto es tan cierto para el universo maestro como para el ultimatón infinitesimal, tan cierto para lo que va a ser como para lo que es y para lo que ha sido.

\section*{3. Los siete Absolutos de la Infinidad}
\par
%\textsuperscript{(1155.5)}
\textsuperscript{105:3.1} Las siete relaciones primordiales dentro del YO SOY se eternizan bajo la forma de los Siete Absolutos de la Infinidad. Pero, aunque podemos describir los orígenes de la realidad y la diferenciación de la infinidad mediante una narración secuencial, de hecho los siete Absolutos son eternos de una manera incalificada y coordinada. La mente mortal quizás necesite concebir que han tenido un principio, pero este concepto debería estar siempre eclipsado por la comprensión de que los siete Absolutos no han tenido un principio; son eternos y, como tales, han existido siempre. Los siete Absolutos son la premisa de la realidad, y han sido descritos en estos documentos como sigue:

\par
%\textsuperscript{(1155.6)}
\textsuperscript{105:3.2} 1. \textit{La Fuente-Centro Primera}. La Primera Persona de la Deidad y el arquetipo principal de lo que no es deidad, Dios, el Padre Universal, creador, controlador y sostén; el amor universal, el espíritu eterno y la energía infinita; el potencial de todos los potenciales y la fuente de todos los actuales; la estabilidad de todo lo estático y el dinamismo de todos los cambios; la fuente del arquetipo y el Padre de las personas. Los siete Absolutos equivalen colectivamente a la infinidad, pero el mismo Padre Universal es realmente infinito.

\par
%\textsuperscript{(1155.7)}
\textsuperscript{105:3.3} 2. \textit{La Fuente-Centro Segunda}. La Segunda Persona de la Deidad, el Hijo Eterno y Original; las realidades personales absolutas del YO SOY y la base para la realización y la revelación del <<YO SOY personalidad>>. Ninguna personalidad puede esperar alcanzar al Padre Universal si no es a través de su Hijo Eterno; la personalidad tampoco puede alcanzar los niveles de existencia espirituales sin la acción y la ayuda de este arquetipo absoluto de todas las personalidades. En la Fuente-Centro Segunda el espíritu es incalificado mientras que la personalidad es absoluta.

\par
%\textsuperscript{(1156.1)}
\textsuperscript{105:3.4} 3. \textit{El Paraíso como Fuente-Centro}. Segundo arquetipo de lo que no es deidad, la Isla eterna del Paraíso; la base para la revelación y la realización del <<YO SOY fuerza>> y el fundamento para establecer el control gravitatorio en todos los universos. El Paraíso es el absoluto de los arquetipos con respecto a toda la realidad manifestada, no espiritual, impersonal y no volitiva. Al igual que la energía espiritual está relacionada con el Padre Universal a través de la personalidad absoluta del Hijo-Madre, toda la energía cósmica está sujeta al control gravitatorio de la Fuente-Centro Primera a través del arquetipo absoluto de la Isla del Paraíso. El Paraíso no está en el espacio; el espacio existe en relación con el Paraíso, y la cronicidad del movimiento está determinada por medio de su relación con el Paraíso. La Isla eterna está totalmente en reposo; todas las demás energías organizadas, o en vías de organizarse, están en eterno movimiento. La presencia del Absoluto Incalificado es la única que permanece inmóvil en todo el espacio, y el Incalificado está coordinado con el Paraíso. El Paraíso existe en el centro del espacio, el Incalificado lo impregna y toda existencia relativa tiene su ser dentro de este ámbito.

\par
%\textsuperscript{(1156.2)}
\textsuperscript{105:3.5} 4. \textit{La Fuente-Centro Tercera}. La Tercera Persona de la Deidad, el Actor Conjunto; el integrador infinito de las energías cósmicas del Paraíso con las energías espirituales del Hijo Eterno; el coordinador perfecto de los móviles de la voluntad y de los mecanismos de la fuerza; el unificador de toda la realidad actual o en vías de actualizarse. El Espíritu Infinito revela la misericordia del Hijo Eterno mediante el servicio de sus múltiples hijos, y actúa al mismo tiempo como manipulador infinito, tejiendo para siempre el arquetipo del Paraíso en las energías del espacio. Este mismo Actor Conjunto, este Dios de Acción, es la expresión perfecta de los planes y de los propósitos ilimitados del Padre y del Hijo, actuando a la vez como fuente de la mente y donador del intelecto a las criaturas de un extenso cosmos.

\par
%\textsuperscript{(1156.3)}
\textsuperscript{105:3.6} 5. \textit{El Absoluto de la Deidad}. Las posibilidades causativas potencialmente personales de la realidad universal, la totalidad de todo el potencial de la Deidad. El Absoluto de la Deidad es el que atenúa deliberadamente las realidades incalificadas, absolutas y no divinas. El Absoluto de la Deidad es el que atenúa lo absoluto y hace absoluto lo restringido ---es el iniciador del destino.

\par
%\textsuperscript{(1156.4)}
\textsuperscript{105:3.7} 6. \textit{El Absoluto Incalificado}. Estático, reactivo y en reposo; la infinidad cósmica no revelada del YO SOY; la totalidad de la realidad no deificada y la finalidad de todo el potencial no personal. El espacio limita la actividad del Incalificado, pero la presencia del Incalificado no tiene límites, es infinita. Existe un concepto de periferia para el universo maestro, pero la presencia del Incalificado no tiene límites; ni siquiera la eternidad puede agotar la quietud ilimitada de este Absoluto que no es deidad.

\par
%\textsuperscript{(1156.5)}
\textsuperscript{105:3.8} 7. \textit{El Absoluto Universal}. Unificador de lo deificado y de lo no deificado; correlaciona lo absoluto y lo relativo. El Absoluto Universal (al ser estático, potencial y asociativo) compensa la tensión entre lo que existe desde siempre y lo inacabado.

\par
%\textsuperscript{(1156.6)}
\textsuperscript{105:3.9} Los Siete Absolutos de la Infinidad constituyen los comienzos de la realidad. Desde la perspectiva de la mente mortal, la Fuente-Centro Primera parecería ser anterior a todos los absolutos. Pero aunque este postulado sea útil, está invalidado por la coexistencia en la eternidad del Hijo, del Espíritu, de los tres Absolutos y de la Isla del Paraíso.

\par
%\textsuperscript{(1157.1)}
\textsuperscript{105:3.10} Es una \textit{verdad} que los Absolutos son manifestaciones del YO SOY-Fuente-Centro Primera; es un \textit{hecho} que estos Absolutos nunca han tenido un principio, sino que son los eternos coordinados de la Fuente-Centro Primera. Las relaciones entre los Absolutos en la eternidad no siempre se pueden presentar sin que ocasionen paradojas en el lenguaje del tiempo y en los modelos conceptuales del espacio. Pero independientemente de cualquier confusión relacionada con el origen de los Siete Absolutos de la Infinidad, es a la vez un hecho y una verdad que toda la realidad está basada en sus existencias en la eternidad y en sus relaciones en la infinidad.

\section*{4. Unidad, dualidad y triunidad}
\par
%\textsuperscript{(1157.2)}
\textsuperscript{105:4.1} Los filósofos del universo dan por sentado que la existencia del YO SOY en la eternidad es la fuente primordial de toda la realidad. Junto con esto admiten el postulado de que el YO SOY se segmenta en unas relaciones primarias consigo mismo ---las siete fases de la infinidad. Y simultáneamente con esta suposición efectúan el tercer postulado ---la aparición en la eternidad de los Siete Absolutos de la Infinidad, y la eternización de la asociación de dualidad entre las siete fases del YO SOY y estos siete Absolutos.

\par
%\textsuperscript{(1157.3)}
\textsuperscript{105:4.2} La autorrevelación del YO SOY empieza así por su yo estático, pasa por su autosegmentación y las relaciones consigo mismo, y culmina en las relaciones absolutas, en las relaciones con unos Absolutos derivados de sí mismo. La dualidad surge así a la existencia mediante la asociación eterna entre los Siete Absolutos de la Infinidad y la séptuple infinidad de las fases autosegmentadas del YO SOY que se autorrevela. Estas relaciones duales, que para los universos se eternizan bajo la forma de los siete Absolutos, hacen eternas las bases fundamentales de toda la realidad universal.

\par
%\textsuperscript{(1157.4)}
\textsuperscript{105:4.3} A veces se ha afirmado que la unidad engendra la dualidad, que ésta produce la triunidad, y que la triunidad es el eterno antepasado de todas las cosas. Existen en verdad tres grandes clases de relaciones primordiales, que son las siguientes:

\par
%\textsuperscript{(1157.5)}
\textsuperscript{105:4.4} 1. \textit{Las relaciones de unidad}. Las relaciones que existen dentro del YO SOY, cuando esta unidad se concibe como una diferenciación trina, y después séptuple, de sí mismo.

\par
%\textsuperscript{(1157.6)}
\textsuperscript{105:4.5} 2. \textit{Las relaciones de dualidad}. Las relaciones que existen entre el YO SOY como séptuple y los Siete Absolutos de la Infinidad.

\par
%\textsuperscript{(1157.7)}
\textsuperscript{105:4.6} 3. \textit{Las relaciones de triunidad}. Son las asociaciones funcionales de los Siete Absolutos de la Infinidad.

\par
%\textsuperscript{(1157.8)}
\textsuperscript{105:4.7} Las relaciones de triunidad surgen sobre unos fundamentos de dualidad porque la interasociación entre los Absolutos es inevitable. Estas asociaciones triunitarias eternizan el potencial de toda la realidad; abarcan a la vez la realidad deificada y la no deificada.

\par
%\textsuperscript{(1157.9)}
\textsuperscript{105:4.8} El YO SOY es la infinidad incalificada bajo la forma de \textit{unidad}. Las dualidades eternizan los \textit{fundamentos} de la realidad. Las triunidades existencian la realización de la infinidad como una \textit{función} universal.

\par
%\textsuperscript{(1157.10)}
\textsuperscript{105:4.9} Los preexistenciales se vuelven existenciales en los siete Absolutos, y los existenciales se vuelven funcionales en las triunidades, que son las asociaciones fundamentales de los Absolutos. Al mismo tiempo que se eternizan las triunidades, el escenario universal está preparado ---los potenciales existen y los actuales están presentes--- y la plenitud de la eternidad contempla la diversificación de la energía cósmica, el despliegue del espíritu del Paraíso y la atribución de la mente junto con la concesión de la personalidad, en virtud de la cual todos estos derivados de la Deidad y del Paraíso están unificados experiencialmente en el nivel de las criaturas, y también lo están mediante otras técnicas en el nivel por encima de las criaturas.

\section*{5. La promulgación de la realidad finita}
\par
%\textsuperscript{(1158.1)}
\textsuperscript{105:5.1} Al igual que la diversificación original del YO SOY debe atribuirse a una volición inherente y autónoma, la promulgación de la realidad finita debe imputarse a los actos volitivos de la Deidad del Paraíso y a los ajustes repercusivos de las triunidades funcionales.

\par
%\textsuperscript{(1158.2)}
\textsuperscript{105:5.2} Antes de dotar a lo finito de una deidad, parecería que toda la diversificación de la realidad tuvo lugar en los niveles absolutos; pero el acto volitivo de promulgar la realidad finita conlleva una atenuación de la absolutidad e implica la aparición de las relatividades.

\par
%\textsuperscript{(1158.3)}
\textsuperscript{105:5.3} Aunque presentamos esta narración de manera secuencial y describimos la aparición histórica de lo finito como un derivado directo de lo absoluto, se debe tener en cuenta que los trascendentales son al mismo tiempo anteriores y posteriores a todo lo finito. Los trascendentales últimos son, en relación con lo finito, tanto la causa como la culminación.

\par
%\textsuperscript{(1158.4)}
\textsuperscript{105:5.4} La posibilidad de lo finito es inherente al Infinito, pero la transformación de la posibilidad en probabilidad y en inevitabilidad debe atribuirse al libre albedrío existente por sí mismo de la Fuente-Centro Primera, que activa todas las asociaciones triunitarias. Únicamente la infinidad de la voluntad del Padre podía atenuar de tal manera el nivel de existencia absoluto como para existenciar un nivel último o crear un nivel finito.

\par
%\textsuperscript{(1158.5)}
\textsuperscript{105:5.5} Con la aparición de la realidad relativa y atenuada surge a la existencia un nuevo ciclo de la realidad ---el ciclo del crecimiento--- un majestuoso descenso desde las alturas de la infinidad hasta el ámbito de lo finito, que oscila eternamente hacia el Paraíso y la Deidad, buscando siempre unos destinos superiores proporcionados a una fuente infinita.

\par
%\textsuperscript{(1158.6)}
\textsuperscript{105:5.6} Estas operaciones inconcebibles señalan el principio de la historia del universo, indican el nacimiento del tiempo mismo. Para una criatura, el comienzo de lo finito \textit{es} la génesis de la realidad; tal como lo percibe la mente de la criatura, no existe ninguna realidad imaginable que sea anterior a la finita. Esta realidad finita recién aparecida existe en dos fases originales:

\par
%\textsuperscript{(1158.7)}
\textsuperscript{105:5.7} 1. \textit{Los máximos primarios}, la realidad supremamente perfecta, el tipo de universo y de criaturas de Havona.

\par
%\textsuperscript{(1158.8)}
\textsuperscript{105:5.8} 2. \textit{Los máximos secundarios}, la realidad supremamente perfeccionada, el tipo de creación y de criaturas superuniversales.

\par
%\textsuperscript{(1158.9)}
\textsuperscript{105:5.9} Éstas son pues las dos manifestaciones originales: la perfecta por constitución y la perfeccionada por evolución. Las dos están coordinadas en las relaciones de la eternidad, pero dentro de los límites del tiempo son aparentemente diferentes. El factor tiempo significa crecimiento para aquello que crece; los finitos secundarios crecen; por eso aquellos que crecen deben aparecer como incompletos en el tiempo. Pero estas diferencias, que son tan importantes en este lado del Paraíso, no existen en la eternidad.

\par
%\textsuperscript{(1158.10)}
\textsuperscript{105:5.10} Hablamos de lo perfecto y de lo perfeccionado como máximos primarios y secundarios, pero existe además otro tipo de máximo: Las relaciones trinitizadoras y de otros tipos entre los primarios y los secundarios producen la aparición de \textit{los máximos terciarios} ---las cosas, los significados y los valores que no son ni perfectos ni perfeccionados, pero que sin embargo están coordinados con estos dos factores ancestrales.

\section*{6. Las repercusiones de la realidad finita}
\par
%\textsuperscript{(1159.1)}
\textsuperscript{105:6.1} Toda la promulgación de las existencias finitas representa un trasvase desde los potenciales hasta los actuales en el interior de las asociaciones absolutas de la infinidad funcional. Entre las numerosas repercusiones que produjo la manifestación creativa de lo finito, se pueden citar las siguientes:

\par
%\textsuperscript{(1159.2)}
\textsuperscript{105:6.2} 1. \textit{La reacción de la deidad}, la aparición de los tres niveles de la supremacía experiencial: la realidad de la supremacía del espíritu personal en Havona, el potencial para la supremacía del poder personal en el gran universo en proyecto, y la capacidad de la mente experiencial para efectuar una actividad desconocida en un nivel de supremacía del futuro universo maestro.

\par
%\textsuperscript{(1159.3)}
\textsuperscript{105:6.3} 2. \textit{La reacción en el universo} implicaba una activación de los planes arquitectónicos para el nivel espacial superuniversal, y esta evolución continúa todavía en toda la organización física de los siete superuniversos.

\par
%\textsuperscript{(1159.4)}
\textsuperscript{105:6.4} 3. \textit{La repercusión con respecto a las criaturas} de la promulgación de la realidad finita tuvo como resultado la aparición de los seres perfectos del orden de los habitantes eternos de Havona, y de los ascendentes evolutivos perfeccionados procedentes de los siete superuniversos. Pero la experiencia evolutiva (creativa en el tiempo) de alcanzar la perfección implica tener como punto de partida algo distinto a la perfección. Así es como aparece la imperfección en las creaciones evolutivas. Y éste es el origen del mal potencial. Los defectos de adaptación, la falta de armonía y los conflictos, todas estas cosas son inherentes al crecimiento evolutivo, desde los universos físicos hasta las criaturas personales.

\par
%\textsuperscript{(1159.5)}
\textsuperscript{105:6.5} 4. \textit{La reacción de la divinidad} ante la imperfección inherente al retraso temporal de la evolución se revela en la presencia compensadora de Dios Séptuple, cuyas actividades integran aquello que está perfeccionándose con lo perfecto y con lo perfeccionado. Este retraso temporal es inseparable de la evolución, que es la creatividad en el tiempo. A causa de esto, y también por otras razones, el poder todopoderoso del Supremo está basado en los éxitos divinos de Dios Séptuple. Este retraso temporal hace posible la participación de las criaturas en la creación divina, permitiendo que las personalidades creadas se asocien con la Deidad para alcanzar el máximo desarrollo. Incluso la mente material de la criatura mortal se asocia así con el Ajustador divino para dualizar el alma inmortal. Dios Séptuple proporciona también las técnicas que compensan las limitaciones experienciales de la perfección inherente, compensando además las limitaciones preascensionales de la imperfección.

\section*{7. La existenciación de los trascendentales}
\par
%\textsuperscript{(1159.6)}
\textsuperscript{105:7.1} Los trascendentales son subinfinitos y subabsolutos, pero son superiores a los finitos y a las criaturas. Los trascendentales se existencian como un nivel integrador que correlaciona los supervalores de los absolutos con los valores máximos de los finitos. Desde el punto de vista de las criaturas, aquello que es trascendental parecería haberse existenciado como una consecuencia de lo finito, y desde el punto de vista de la eternidad, como una anticipación de lo finito; y existen aquellos que lo han considerado como una <<prerresonancia>> de lo finito.

\par
%\textsuperscript{(1159.7)}
\textsuperscript{105:7.2} Lo trascendental no es necesariamente algo que no se desarrolla, pero es superevolutivo en el sentido finito; tampoco es no experiencial, pero sí es una superexperiencia en la medida en que esta palabra tiene un significado para las criaturas. El mejor ejemplo de esta paradoja es quizás el universo central de perfección: Havona no es del todo absoluto ---únicamente la Isla del Paraíso es realmente absoluta en el sentido <<materializado>>. Tampoco es una creación evolutiva finita como los siete superuniversos. Havona es eterno, pero no invariable en el sentido de ser un universo donde el crecimiento no existe. Está habitado por unas criaturas (los nativos de Havona) que nunca han sido realmente creadas, ya que existen desde toda la eternidad. Havona es así un ejemplo de algo que no es exactamente finito ni tampoco absoluto. Havona actúa además como amortiguador entre el Paraíso absoluto y las creaciones finitas, ilustrando así nuevamente la función de los trascendentales. Pero Havona mismo no es un trascendental ---es solamente Havona.

\par
%\textsuperscript{(1160.1)}
\textsuperscript{105:7.3} Al igual que el Supremo está asociado con los finitos, el
Último está identificado con los trascendentales. Pero aunque comparamos así al Supremo con el Último, son diferentes por algo más que el grado; su diferencia es también una cuestión de calidad. El Último es algo más que un super-Supremo proyectado en el nivel trascendental. El Último es todo eso, pero también más: el Último es la existenciación de nuevas realidades de la Deidad, la atenuación de nuevas fases de lo que hasta entonces era incalificado.

\par
%\textsuperscript{(1160.2)}
\textsuperscript{105:7.4} Entre las realidades que están asociadas con el nivel trascendental, se encuentran las siguientes:

\par
%\textsuperscript{(1160.3)}
\textsuperscript{105:7.5} 1. La presencia de la Deidad del Último.

\par
%\textsuperscript{(1160.4)}
\textsuperscript{105:7.6} 2. El concepto del universo maestro.

\par
%\textsuperscript{(1160.5)}
\textsuperscript{105:7.7} 3. Los Arquitectos del Universo Maestro.

\par
%\textsuperscript{(1160.6)}
\textsuperscript{105:7.8} 4. Los dos grupos de organizadores de fuerza del Paraíso.

\par
%\textsuperscript{(1160.7)}
\textsuperscript{105:7.9} 5. Ciertas modificaciones en la potencia espacial.

\par
%\textsuperscript{(1160.8)}
\textsuperscript{105:7.10} 6. Ciertos valores del espíritu.

\par
%\textsuperscript{(1160.9)}
\textsuperscript{105:7.11} 7. Ciertos significados de la mente.

\par
%\textsuperscript{(1160.10)}
\textsuperscript{105:7.12} 8. Las cualidades y las realidades absonitas.

\par
%\textsuperscript{(1160.11)}
\textsuperscript{105:7.13} 9. La omnipotencia, la omnisciencia y la omnipresencia.

\par
%\textsuperscript{(1160.12)}
\textsuperscript{105:7.14} 10. El espacio.

\par
%\textsuperscript{(1160.13)}
\textsuperscript{105:7.15} Podemos imaginar que el universo donde vivimos ahora existe en los niveles finito, trascendental y absoluto. Es el escenario cósmico donde se representa el drama interminable de las actividades de la personalidad y de las metamorfosis de la energía.

\par
%\textsuperscript{(1160.14)}
\textsuperscript{105:7.16} Todas estas múltiples realidades están unificadas \textit{de manera absoluta} por las diversas triunidades, \textit{de manera funcional} por los Arquitectos del Universo Maestro, y \textit{de manera relativa} por los Siete Espíritus Maestros, los coordinadores subsupremos de la divinidad de Dios Séptuple.

\par
%\textsuperscript{(1160.15)}
\textsuperscript{105:7.17} Dios Séptuple representa la revelación de la personalidad y de la divinidad del Padre Universal a las criaturas que se encuentran en el estado máximo y submáximo, pero la Fuente-Centro Primera mantiene otras relaciones séptuples que no están relacionadas con la manifestación del divino ministerio espiritual del Dios que es espíritu.

\par
%\textsuperscript{(1160.16)}
\textsuperscript{105:7.18} En la eternidad del pasado, las fuerzas de los Absolutos, los espíritus de las Deidades y las personalidades de los Dioses se pusieron en movimiento en respuesta a la voluntad autónoma primordial de la voluntad autónoma existente por sí misma. En esta era del universo, todos estamos presenciando las prodigiosas repercusiones del inmenso panorama cósmico de las manifestaciones subabsolutas de los potenciales ilimitados de todas esas realidades. Es enteramente posible que la diversificación continua de la realidad original de la Fuente-Centro Primera siga aumentando y exteriorizándose a lo largo de las épocas, cada vez más, hasta las extensiones lejanas e inconcebibles de la infinidad absoluta.

\par
%\textsuperscript{(1161.1)}
\textsuperscript{105:7.19} [Presentado por un Melquisedek de Nebadon.]


\chapter{Documento 106. Los niveles de realidad del universo}
\par
%\textsuperscript{(1162.1)}
\textsuperscript{106:0.1} No es suficiente con que los mortales ascendentes conozcan algo sobre las relaciones de la Deidad con la génesis y las manifestaciones de la realidad cósmica; también deberían comprender algo acerca de las relaciones que existen entre ellos mismos y los numerosos niveles de realidades existenciales y experienciales, de realidades potenciales y actuales. La orientación del hombre en la Tierra, su perspicacia cósmica y la dirección de su conducta espiritual se vuelven más elevadas gracias a una mejor comprensión de las realidades del universo y de sus técnicas de interasociación, integración y unificación.

\par
%\textsuperscript{(1162.2)}
\textsuperscript{106:0.2} El gran universo de la época actual y el universo maestro emergente están compuestos por numerosas formas y fases de la realidad, que existen a su vez en diversos niveles de actividad funcional. Estas múltiples formas y fases existentes y latentes han sido indicadas anteriormente en estos documentos, y ahora las agrupamos para facilitar su concepción en las categorías siguientes:

\par
%\textsuperscript{(1162.3)}
\textsuperscript{106:0.3} 1. \textit{Finitos incompletos}. Éste es el estado presente de las criaturas ascendentes del gran universo, el estado presente de los mortales de Urantia. Este nivel abarca la existencia de las criaturas desde los humanos planetarios hasta, pero no incluídos, aquellos que han alcanzado su destino. Caracteriza a los universos desde sus primeros comienzos físicos hasta, pero no incluido, su establecimiento en la luz y la vida. Este nivel constituye la periferia actual de la actividad creativa en el tiempo y el espacio. Parece que se desplaza desde el Paraíso hacia el exterior, porque cuando termine la presente era del universo, que contemplará cómo el gran universo alcanza el estado de luz y vida, presenciará seguramente también cómo aparece algún nuevo tipo de desarrollo y de crecimiento en el primer nivel del espacio exterior.

\par
%\textsuperscript{(1162.4)}
\textsuperscript{106:0.4} 2. \textit{Finitos máximos}. Éste es el estado presente de todas las criaturas experienciales que han alcanzado su destino ---tal como este destino ha sido revelado dentro del marco de la presente era del universo. Incluso los universos pueden conseguir su estado máximo, tanto espiritual como físicamente. Pero la palabra <<máximo>> es en sí misma un término relativo ---¿máximo con respecto a qué? Lo que es máximo y aparentemente final en la presente era del universo, puede no ser más que un verdadero principio desde el punto de vista de las eras por venir. Algunas fases de Havona parecen hallarse en el orden máximo.

\par
%\textsuperscript{(1162.5)}
\textsuperscript{106:0.5} 3. \textit{Trascendentales}. Este nivel superfinito sigue al del progreso finito (precediéndolo). Dicho nivel implica la génesis prefinita de los comienzos finitos, y el significado postfinito de todas las terminaciones o destinos aparentemente finitos. Muchos elementos del Paraíso y Havona parecen pertenecer al orden trascendental.

\par
%\textsuperscript{(1162.6)}
\textsuperscript{106:0.6} 4. \textit{Últimos}. Este nivel abarca aquello que tiene un significado para el universo maestro y establece contacto con el nivel de destino del universo maestro acabado. El Paraíso-Havona (y sobre todo el circuito de los mundos del Padre) tiene en muchos aspectos un significado último.

\par
%\textsuperscript{(1163.1)}
\textsuperscript{106:0.7} 5. \textit{Coabsolutos}. Este nivel supone la proyección de los experienciales en un campo de expresión creativa que sobrepasa el universo maestro.

\par
%\textsuperscript{(1163.2)}
\textsuperscript{106:0.8} 6. \textit{Absolutos}. Este nivel implica la presencia en la eternidad de los siete Absolutos existenciales. También puede suponer cierto grado de realización experiencial asociada, pero si es así, no comprendemos cómo, quizás a través del potencial de contacto de la personalidad.

\par
%\textsuperscript{(1163.3)}
\textsuperscript{106:0.9} 7. \textit{Infinidad}. Este nivel es preexistencial y postexperiencial. La unidad incalificada de la infinidad es una realidad hipotética anterior a todos los comienzos y posterior a todos los destinos.

\par
%\textsuperscript{(1163.4)}
\textsuperscript{106:0.10} Estos niveles de realidad son unos símbolos prácticos aceptables sobre la presente era del universo y para la perspectiva de los mortales. Existen otras maneras de contemplar la realidad desde una perspectiva distinta a la de los mortales y desde el punto de vista de otras eras universales. Se debería reconocer así que los conceptos presentados aquí son totalmente relativos, en el sentido de que están condicionados y limitados por:

\par
%\textsuperscript{(1163.5)}
\textsuperscript{106:0.11} 1. Las limitaciones del lenguaje humano.

\par
%\textsuperscript{(1163.6)}
\textsuperscript{106:0.12} 2. Las limitaciones de la mente humana.

\par
%\textsuperscript{(1163.7)}
\textsuperscript{106:0.13} 3. El desarrollo limitado de los siete superuniversos.

\par
%\textsuperscript{(1163.8)}
\textsuperscript{106:0.14} 4. Vuestra ignorancia sobre los seis objetivos primordiales del desarrollo superuniversal, que no están relacionados con la ascensión de los mortales al Paraíso.

\par
%\textsuperscript{(1163.9)}
\textsuperscript{106:0.15} 5. Vuestra incapacidad para captar un punto de vista, aunque sea parcial, de la eternidad.

\par
%\textsuperscript{(1163.10)}
\textsuperscript{106:0.16} 6. La imposibilidad de describir la evolución y el destino cósmicos en relación con todas las eras universales, y no simplemente con respecto a la presente era del desarrollo evolutivo de los siete superuniversos.

\par
%\textsuperscript{(1163.11)}
\textsuperscript{106:0.17} 7. La incapacidad de todas las criaturas para captar el significado real de lo preexistencial y de lo postexperiencial ---de aquello que está situado antes de los comienzos y después de los destinos.

\par
%\textsuperscript{(1163.12)}
\textsuperscript{106:0.18} El crecimiento de la realidad está condicionado por las circunstancias de las eras sucesivas del universo. El universo central no experimentó ningún cambio evolutivo durante la era de Havona, pero en las épocas actuales de la era superuniversal está experimentando ciertos cambios progresivos inducidos por su coordinación con los superuniversos evolutivos. Los siete superuniversos que evolucionan en la actualidad alcanzarán algún día el estado permanente de luz y vida, conseguirán el límite del crecimiento establecido para la presente era del universo. Pero no hay duda de que la era siguiente, la era del primer nivel del espacio exterior, liberará a los superuniversos de aquello que limita su destino en la era actual. La repleción se superpone continuamente a la terminación.

\par
%\textsuperscript{(1163.13)}
\textsuperscript{106:0.19} Éstas son algunas de las limitaciones que encontramos al intentar presentar un concepto unificado del crecimiento cósmico de las cosas, los significados y los valores, y de su síntesis en unos niveles de realidad siempre ascendentes.

\section*{1. La asociación primaria de los funcionales finitos}
\par
%\textsuperscript{(1163.14)}
\textsuperscript{106:1.1} Las fases primarias, o de origen espiritual, de la realidad finita encuentran su expresión inmediata en los niveles de las criaturas bajo la forma de las personalidades perfectas, y en los niveles del universo bajo la forma de la perfecta creación de Havona. Incluso la Deidad experiencial está expresada de esta manera en la persona espiritual de Dios Supremo en Havona. Pero las fases secundarias de lo finito, evolutivas y condicionadas por el tiempo y la materia, sólo se integran cósmicamente como resultado del crecimiento y de los logros. Todos los finitos secundarios, o en vías de perfeccionarse, han de alcanzar finalmente un nivel equivalente al de la perfección primaria, pero este destino está sujeto a una demora temporal, una restricción constitutiva que se encuentra en los superuniversos pero que no se encuentra de manera innata en la creación central. (Sabemos que existen los finitos terciarios, pero la técnica para su integración no se ha revelado todavía.)

\par
%\textsuperscript{(1164.1)}
\textsuperscript{106:1.2} Este retraso temporal que se encuentra en los superuniversos, este obstáculo para alcanzar la perfección, asegura la participación de las criaturas en el crecimiento evolutivo. Esto hace posible que las criaturas puedan asociarse con el Creador para evolucionar ellas mismas. Y durante este período de crecimiento expansivo, lo inacabado está en correlación con lo perfecto a través del ministerio de Dios Séptuple.

\par
%\textsuperscript{(1164.2)}
\textsuperscript{106:1.3} Dios Séptuple significa que la Deidad del Paraíso reconoce las barreras del tiempo en los universos evolutivos del espacio. Por muy lejos que se halle del Paraíso el origen de una personalidad material superviviente, por muy profundamente que esté en el espacio, Dios Séptuple se encontrará allí presente y dedicado a su afectuoso y misericordioso ministerio de verdad, belleza y bondad para esa criatura inacabada, combativa y evolutiva. El ministerio de la divinidad que ejerce el Séptuple se extiende hacia el interior a través del Hijo Eterno hasta el Padre Paradisiaco, y hacia el exterior a través de los Ancianos de los Días hasta los Padres de los universos ---los Hijos Creadores.

\par
%\textsuperscript{(1164.3)}
\textsuperscript{106:1.4} El hombre, como es personal y se eleva mediante el progreso espiritual, encuentra la divinidad personal y espiritual de la Deidad Séptuple; pero existen otras fases del Séptuple que no están implicadas en el progreso de la personalidad. Los aspectos de divinidad de esta agrupación de la Deidad están actualmente integrados en la coordinación existente entre los Siete Espíritus Maestros y el Actor Conjunto, pero están destinados a unificarse eternamente en la personalidad emergente del Ser Supremo. Las otras fases de la Deidad Séptuple están diversamente integradas en la presente era del universo, pero todas están igualmente destinadas a unificarse en el Supremo. El Séptuple es, en todas las fases, la fuente de la unidad relativa de la realidad funcional del gran universo actual.

\section*{2. La integración secundaria suprema de lo finito}
\par
%\textsuperscript{(1164.4)}
\textsuperscript{106:2.1} Al igual que Dios Séptuple coordina funcionalmente la evolución finita, el Ser Supremo sintetiza finalmente la consecución del destino. El Ser Supremo es la culminación, bajo la forma de deidad, de la evolución del gran universo ---una evolución física alrededor de un núcleo espiritual, y el predominio final del núcleo espiritual sobre las esferas de la evolución física que lo envuelven y giran a su alrededor. Todo esto tiene lugar de acuerdo con los mandatos de la personalidad: la personalidad paradisiaca en el sentido más elevado, la personalidad del Creador en el sentido del universo, la personalidad mortal en el sentido humano, y la personalidad Suprema en el sentido culminante o totalizador de la experiencia.

\par
%\textsuperscript{(1164.5)}
\textsuperscript{106:2.2} El concepto del Supremo debe servir para reconocer la diferencia entre la persona espiritual, el poder evolutivo y la síntesis del poder y la personalidad ---la unificación del poder evolutivo con la personalidad espiritual, y el predominio de ésta sobre aquel.

\par
%\textsuperscript{(1164.6)}
\textsuperscript{106:2.3} A fin de cuentas, el espíritu viene del Paraíso a través de Havona. La energía-materia parece evolucionar en las profundidades del espacio, y es organizada bajo la forma de poder por los hijos del Espíritu Infinito en colaboración con los Hijos Creadores de Dios. Todo esto es experiencial; es una operación que se efectúa en el tiempo y el espacio e implica a una amplia gama de seres vivientes, incluyendo a las divinidades Creadoras y a las criaturas evolutivas. El dominio del poder por parte de las divinidades Creadoras se extiende lentamente por el gran universo hasta que abarque el establecimiento y la estabilización evolutiva de las creaciones espacio-temporales, y así se producirá el florecimiento del poder experiencial de Dios Séptuple. Este poder abarca toda la gama de las realizaciones de la divinidad en el tiempo y el espacio, desde la donación de los Ajustadores por parte del Padre Universal hasta la donación de la vida por parte de los Hijos Paradisiacos. Se trata de un poder ganado, de un poder demostrado, de un poder experiencial, que contrasta con el poder de la eternidad, con el poder insondable, con el poder existencial de las Deidades del Paraíso.

\par
%\textsuperscript{(1165.1)}
\textsuperscript{106:2.4} Este poder experiencial, que procede de los logros como divinidad del mismo Dios Séptuple, manifiesta las cualidades cohesivas de la divinidad al sintetizarse ---al totalizarse--- bajo la forma del poder todopoderoso del dominio experiencial adquirido sobre las creaciones evolutivas. Este poder todopoderoso encuentra a su vez la cohesión entre la personalidad y el espíritu en la esfera piloto del cinturón exterior de los mundos de Havona, uniéndose con la personalidad espiritual, presente en Havona, de Dios Supremo. La Deidad experiencial lleva así a su culminación la larga lucha evolutiva, confiriendo al producto del poder del tiempo y del espacio la presencia espiritual y la personalidad divina que residen en la creación central.

\par
%\textsuperscript{(1165.2)}
\textsuperscript{106:2.5} Así es como el Ser Supremo consigue englobar finalmente todo lo que evoluciona en el tiempo y el espacio, confiriéndole a esas cualidades una personalidad espiritual. Puesto que las criaturas, incluidas las mortales, participan como personalidades en esta majestuosa operación, es indudable que conseguirán la capacidad de conocer al Supremo y de percibirlo como verdaderos hijos de esta Deidad evolutiva.

\par
%\textsuperscript{(1165.3)}
\textsuperscript{106:2.6} Miguel de Nebadon es semejante al Padre Paradisiaco porque comparte su perfección paradisiaca\footnote{\textit{Perfección de Miguel}: Jn 14:9-10.}; los mortales evolutivos conseguirán algún día emparentarse así con el Supremo experiencial, porque compartirán realmente su perfección evolutiva\footnote{\textit{Sed perfectos}: Gn 17:1; 1 Re 8:61; Lv 19:2; Dt 18:13; Mt 5:48; 2 Co 13:11; Stg 1:4; 1 P 1:16.}.

\par
%\textsuperscript{(1165.4)}
\textsuperscript{106:2.7} Dios Supremo es experiencial; por consiguiente, es completamente experimentable. Las realidades existenciales de los siete Absolutos no son perceptibles mediante la técnica de la experiencia; la personalidad de la criatura finita sólo puede captar \textit{las realidades de la personalidad} del Padre, del Hijo y del Espíritu mediante la actitud de la oración y la adoración.

\par
%\textsuperscript{(1165.5)}
\textsuperscript{106:2.8} Cuando la síntesis del poder y la personalidad del Ser Supremo haya terminado, dentro de dicha síntesis estará asociada toda la absolutidad de las diversas triodidades que pueda asociarse así, y esta majestuosa personalidad de la evolución será alcanzable y comprensible experiencialmente por todas las personalidades finitas. Cuando los ascendentes alcancen el supuesto séptimo estado de existencia espiritual, experimentarán en él el desarrollo de un nuevo valor o significado de la absolutidad y de la infinidad de las triodidades, tal como esto se encuentra revelado en los niveles subabsolutos en el Ser Supremo, el cual es experimentable. Pero para alcanzar estas etapas de desarrollo máximo, habrá que esperar probablemente a que todo el gran universo esté establecido de manera coordinada en la luz y la vida.

\section*{3. La asociación trascendental terciaria de la realidad}
\par
%\textsuperscript{(1165.6)}
\textsuperscript{106:3.1} Los arquitectos absonitos establecen el proyecto; los Creadores Supremos lo traen a la existencia; el Ser Supremo lo llevará a su plenitud tal como fue creado en el tiempo por los Creadores Supremos, y tal como fue previsto en el espacio por los Arquitectos Maestros.

\par
%\textsuperscript{(1165.7)}
\textsuperscript{106:3.2} Durante la presente era del universo, los Arquitectos del Universo Maestro se ocupan de coordinar administrativamente el universo maestro. Pero la aparición del Todopoderoso Supremo al final de la presente era del universo significará que lo finito evolutivo ha alcanzado la primera etapa del destino experiencial. Este acontecimiento conducirá indudablemente al funcionamiento total de la primera Trinidad experiencial ---la unión de los Creadores Supremos, el Ser Supremo y los Arquitectos del Universo Maestro. Esta Trinidad está destinada a llevar a cabo la integración evolutiva ulterior de la creación maestra.

\par
%\textsuperscript{(1166.1)}
\textsuperscript{106:3.3} La Trinidad del Paraíso es realmente la Trinidad de la infinidad, y una Trinidad no puede ser de ninguna manera infinita si no incluye a esta Trinidad original. Pero la Trinidad original es una eventualidad de la asociación exclusiva de las Deidades absolutas; los seres subabsolutos no tuvieron nada que ver con esta asociación primordial. Las Trinidades experienciales que aparecieron posteriormente engloban incluso las aportaciones de las personalidades creadas. Esto es cierto sin duda en lo que concierne a la Trinidad Última, donde la presencia misma de los Hijos Creadores Maestros entre sus miembros Creadores Supremos revela la presencia concomitante de la experiencia real y auténtica de las criaturas \textit{dentro} de esta asociación de la Trinidad.

\par
%\textsuperscript{(1166.2)}
\textsuperscript{106:3.4} La primera Trinidad experiencial asegura el logro colectivo de las eventualidades últimas. Las asociaciones colectivas permiten anticiparse a, e incluso trascender, las capacidades individuales; y esto es así incluso más allá del nivel finito. En las eras venideras, después de que los siete superuniversos estén establecidos en la luz y la vida, el Cuerpo de la Finalidad difundirá sin duda los objetivos de las Deidades del Paraíso tal como sean dictados por la Trinidad Última, y tal como estén unificados bajo la forma del poder y la personalidad en el Ser Supremo.

\par
%\textsuperscript{(1166.3)}
\textsuperscript{106:3.5} Detectamos la expansión de los elementos comprensibles del Padre Universal a través de todos los gigantescos desarrollos universales de la eternidad pasada y futura. Consideramos como un postulado filosófico que el Padre, como YO SOY\footnote{\textit{YO SOY}: Ex 3:13-14.}, impregna toda la infinidad, pero ninguna criatura es capaz de abarcar este postulado por experiencia. A medida que se expanden los universos, a medida que la gravedad y el amor se extienden por el espacio que se organiza en el tiempo, somos capaces de comprender cada vez más cosas de la Fuente-Centro Primera. Observamos que la acción de la gravedad penetra la presencia espacial del Absoluto Incalificado, y detectamos que las criaturas espirituales evolucionan y se desarrollan dentro de la presencia de divinidad del Absoluto de la Deidad, mientras que la evolución tanto cósmica como espiritual se está unificando, por medio de la mente y de la experiencia, en los niveles finitos de la deidad bajo la forma del Ser Supremo, y se está coordinando en los niveles trascendentales como Trinidad Última.

\section*{4. La integración última o de cuarta fase}
\par
%\textsuperscript{(1166.4)}
\textsuperscript{106:4.1} La Trinidad del Paraíso coordina indudablemente en el sentido último, pero desempeña su actividad en este aspecto como un absoluto que se ha atenuado a sí mismo; la Trinidad Última experiencial, como trascendental que es, coordina lo trascendental. Cuando aumente su unidad en el eterno futuro, esta Trinidad experiencial activará aún más la presencia en vías de existenciarse de la Deidad
Última.

\par
%\textsuperscript{(1166.5)}
\textsuperscript{106:4.2} Aunque la Trinidad Última está destinada a coordinar la creación maestra, Dios Último es la personalización trascendental del poder que determina la meta hacia la que se dirige todo el universo maestro. La existenciación total del Último significará que la creación maestra ha llegado a su culminación, y traerá consigo la plena emergencia de esta Deidad trascendental.

\par
%\textsuperscript{(1166.6)}
\textsuperscript{106:4.3} No conocemos los cambios que se producirán cuando el Último emerja plenamente. Pero al igual que el Supremo está ahora personal y espiritualmente presente en Havona, el Último también lo está pero en el sentido absonito y superpersonal. Y habéis sido informados de la existencia de los Vicegerentes Calificados del Último, aunque no se os ha indicado cuál es su paradero o su función actual.

\par
%\textsuperscript{(1167.1)}
\textsuperscript{106:4.4} Pero sin tener en cuenta las repercusiones administrativas que acompañarán a la aparición de la Deidad Última, los valores personales de su divinidad trascendental serán experimentables por todas las personalidades que hayan participado en la manifestación de este nivel de la Deidad. La trascendencia de lo finito sólo puede conducir a alcanzar lo último. Dios Último existe en la trascendencia del tiempo y del espacio, pero sin embargo es subabsoluto, a pesar de su capacidad inherente para asociarse funcionalmente con los absolutos.

\section*{5. La asociación coabsoluta o de quinta fase}
\par
%\textsuperscript{(1167.2)}
\textsuperscript{106:5.1} El Último es la cima de la realidad trascendental, al igual que el Supremo es la coronación de la realidad evolutivo-experiencial. La aparición efectiva de estas dos Deidades experienciales coloca los fundamentos para la segunda Trinidad experiencial. Se trata de la Trinidad Absoluta, la unión de Dios Supremo, Dios Último y el Consumador no revelado del Destino del Universo. Esta Trinidad tiene la capacidad teórica de activar los Absolutos de potencialidad ---los Absolutos de la Deidad, Universal e Incalificado. Pero esta Trinidad Absoluta no puede formarse por completo hasta que concluya la evolución de todo el universo maestro, desde Havona hasta el cuarto nivel más alejado del espacio exterior.

\par
%\textsuperscript{(1167.3)}
\textsuperscript{106:5.2} Debemos indicar claramente que estas Trinidades experienciales relacionan entre sí no solamente las cualidades de personalidad de la Divinidad experiencial, sino también todas las cualidades distintas a las personales que caracterizan a la unidad de Deidad que han alcanzado. Aunque esta exposición trata principalmente de las fases personales de la unificación del cosmos, no es menos cierto que los aspectos impersonales del universo de universos están igualmente destinados a experimentar la unificación, tal como lo ilustra la síntesis del poder y la personalidad que se está produciendo actualmente en conexión con la evolución del Ser Supremo. Las cualidades personales y espirituales del Supremo son inseparables de las prerrogativas de poder del Todopoderoso, y las dos son complementadas por el potencial desconocido de la mente Suprema. Dios Último, como persona, tampoco puede ser examinado separadamente de los aspectos distintos a los personales de la Deidad Última. Y en el nivel absoluto, los Absolutos de la Deidad e Incalificado son inseparables e indistinguibles en presencia del Absoluto Universal.

\par
%\textsuperscript{(1167.4)}
\textsuperscript{106:5.3} Las Trinidades, en sí mismas y por sí mismas, no son personales, pero tampoco están en contra de la personalidad. Más bien la engloban y la correlacionan, en un sentido colectivo, con las funciones impersonales. Así pues, las Trinidades son siempre una realidad de la \textit{deidad}, pero nunca una realidad de la \textit{personalidad}. Los aspectos de una trinidad relacionados con la personalidad son inherentes a sus miembros individuales, y como personas individuales \textit{no} son esa trinidad. Sólo son una trinidad como grupo; esa colectividad \textit{es} una trinidad. Pero la trinidad siempre incluye a toda la deidad que engloba; la trinidad es la unidad de la deidad.

\par
%\textsuperscript{(1167.5)}
\textsuperscript{106:5.4} Los tres Absolutos ---de la Deidad, Universal e Incalificado--- no son una trinidad, porque no todos son deidades. Sólo lo que está deificado puede volverse una trinidad; todas las demás asociaciones son triunidades o triodidades.

\section*{6. La integración absoluta o de sexta fase}
\par
%\textsuperscript{(1167.6)}
\textsuperscript{106:6.1} El potencial actual del universo maestro no es del todo absoluto, aunque pueda muy bien estar cerca del último, y creemos que es imposible conseguir revelar plenamente los valores y significados absolutos dentro del marco de un cosmos subabsoluto. Nos encontramos pues con unas dificultades considerables cuando intentamos concebir una expresión total de las posibilidades ilimitadas de los tres Absolutos, e incluso cuando tratamos de visualizar la personalización experiencial de Dios Absoluto en el nivel, actualmente impersonal, del Absoluto de la Deidad.

\par
%\textsuperscript{(1168.1)}
\textsuperscript{106:6.2} El escenario espacial del universo maestro parece ser adecuado para la realización del Ser Supremo, para la formación y el pleno funcionamiento de la Trinidad Última, para la existenciación de Dios Último e incluso para el comienzo de la Trinidad Absoluta. Pero nuestros conceptos sobre el pleno funcionamiento de esta segunda Trinidad experiencial parecen implicar unos factores que se encuentra más allá incluso del universo maestro en vías de expansión.

\par
%\textsuperscript{(1168.2)}
\textsuperscript{106:6.3} Si suponemos la existencia de un cosmos infinito ---de una especie de cosmos ilimitado más allá del universo maestro--- y si concebimos que los desarrollos finales de la Trinidad Absoluta tendrán lugar en ese campo de acción superúltimo, entonces es posible conjeturar que la función total de esta Trinidad conseguirá expresarse de manera final en las creaciones de la infinidad, y completará la manifestación absoluta de \textit{todos} los potenciales. La integración y la asociación de los segmentos cada vez más amplios de la realidad se acercarán al estado absoluto en proporción a la inclusión de toda la realidad dentro de los segmentos así asociados.

\par
%\textsuperscript{(1168.3)}
\textsuperscript{106:6.4} Dicho de otra manera: la función total de la Trinidad Absoluta, tal como lo indica su nombre, es realmente absoluta. No sabemos cómo una función absoluta puede conseguir expresarse de manera total sobre una base atenuada, limitada o restringida de otras maneras. Por eso debemos suponer que cualquier función de totalidad de este tipo será incondicionada (en potencia). También podría parecer que lo incondicionado sería asimismo ilimitado, al menos desde un punto de vista cualitativo, aunque no estamos tan seguros en lo que se refiere a las relaciones cuantitativas.

\par
%\textsuperscript{(1168.4)}
\textsuperscript{106:6.5} Sin embargo, estamos seguros de una cosa: la Trinidad existencial del Paraíso es infinita y la Trinidad experiencial Última es subinfinita, pero la Trinidad Absoluta no es tan fácil de clasificar. Aunque su génesis y su constitución sean experienciales, se pone claramente en contacto con los Absolutos existenciales de potencialidad.

\par
%\textsuperscript{(1168.5)}
\textsuperscript{106:6.6} Aunque es poco provechoso para la mente humana intentar captar estos conceptos lejanos y superhumanos, sugerimos la idea de que la acción de la Trinidad Absoluta, en la eternidad, culmina en algún tipo de experiencialización de los Absolutos de potencialidad. Ésta parecería ser una conclusión razonable en lo que respecta al Absoluto Universal, y posiblemente también al Absoluto Incalificado; al menos sabemos que el Absoluto Universal no es solamente estático y potencial, sino también asociativo en el sentido en que estas palabras conciernen a la Deidad total. Pero en cuanto a los valores concebibles de la divinidad y de la personalidad, estos supuestos acontecimientos implican la personalización del Absoluto de la Deidad y la aparición de aquellos valores superpersonales y de aquellos significados ultrapersonales inherentes al acabamiento de la personalidad de Dios Absoluto ---la tercera y última Deidad experiencial.

\section*{7. La finalidad del destino}
\par
%\textsuperscript{(1168.6)}
\textsuperscript{106:7.1} Algunas dificultades que existen para formarse un concepto de la integración de la realidad infinita son inherentes al hecho de que todas estas ideas contienen alguna cosa de la finalidad del desarrollo universal, una especie de realización experiencial de todo lo que podría existir algún día. Y es inconcebible que la finalidad de la infinidad cuantitativa pueda realizarse nunca por completo. En los tres Absolutos potenciales deben quedar siempre unas posibilidades sin explorar que ninguna cantidad de desarrollo experiencial podrá nunca agotar. La eternidad misma, aunque es absoluta, no es más que absoluta.

\par
%\textsuperscript{(1169.1)}
\textsuperscript{106:7.2} Incluso un concepto provisional de integración final es inseparable de las fructificaciones de la eternidad incalificada y, por consiguiente, este concepto es prácticamente irrealizable en cualquier época futura que se pueda concebir.

\par
%\textsuperscript{(1169.2)}
\textsuperscript{106:7.3} El acto volitivo de las Deidades que componen la Trinidad del Paraíso es el que establece el destino; el destino está establecido en la inmensidad de los tres grandes potenciales, cuya absolutidad engloba las posibilidades de todo desarrollo futuro; el acto del Consumador del Destino del Universo es probablemente el que consuma el destino, y es probable que en este acto estén implicados el Supremo y el Último, que forman parte de la Trinidad Absoluta. Las criaturas que experimentan pueden comprender, al menos parcialmente, cualquier destino experiencial; pero un destino que roza los existenciales infinitos es difícilmente comprensible. El destino en la finalidad es una realización existencial-experiencial que parece implicar al Absoluto de la Deidad. Pero el Absoluto de la Deidad mantiene relaciones de eternidad con el Absoluto Incalificado debido al Absoluto Universal. Y estos tres Absolutos, que tienen la posibilidad de volverse experienciales, son realmente existenciales y mucho más, ya que no tienen límites, ni tiempo, ni espacio, ni confines, ni medidas ---son verdaderamente infinitos.

\par
%\textsuperscript{(1169.3)}
\textsuperscript{106:7.4} La improbabilidad de que se alcance la meta no impide sin embargo teorizar filosóficamente sobre estos destinos hipotéticos. La manifestación del Absoluto de la Deidad, como un Dios absoluto que se pueda alcanzar, quizás sea imposible de realizar en la práctica; sin embargo, esta fructificación de la finalidad sigue siendo una posibilidad teórica. La participación del Absoluto Incalificado en un tipo de cosmos infinito inconcebible puede estar inconmensurablemente lejana en el futuro de la eternidad sin fin, pero sin embargo se trata de una hipótesis válida. Los mortales, los morontiales, los espíritus, los finalitarios, los trascendentales y otros, así como los universos mismos y todas las demás fases de la realidad, tienen ciertamente \textit{un destino potencialmente final cuyo valor esabsoluto}; pero dudamos de que algún ser o universo pueda alcanzar nunca por completo todos los aspectos de un destino semejante.

\par
%\textsuperscript{(1169.4)}
\textsuperscript{106:7.5} Por mucho que pueda crecer vuestra comprensión del Padre, vuestra mente se tambaleará siempre ante la infinidad no revelada del Padre-YO SOY, una infinidad cuya inmensidad sin explorar permanecerá siempre insondable e incomprensible durante todos los ciclos de la eternidad. Por mucha parte de Dios que podáis alcanzar, siempre habrá una parte mucho más grande de él que ni siquiera sospecharéis que existía. Y creemos que esto es tan cierto en los niveles trascendentales como en el ámbito de la existencia finita. ¡La búsqueda de Dios no tiene fin!

\par
%\textsuperscript{(1169.5)}
\textsuperscript{106:7.6} Esta incapacidad para alcanzar a Dios en el sentido final no debería desanimar de ninguna manera a las criaturas del universo; es verdad que podéis alcanzar, y alcanzáis de hecho, los niveles de Deidad del Séptuple, del Supremo y del Último, los cuales significan para vosotros lo mismo que significa la comprensión infinita de Dios Padre para el Hijo Eterno y el Actor Conjunto en sus estados absolutos de existencia en la eternidad. La infinidad de Dios, en lugar de abrumar a las criaturas, debería ser la seguridad suprema de que a lo largo de todo el interminable futuro, toda personalidad ascendente tendrá delante de sí unas posibilidades para desarrollar su personalidad y para asociarse con la Deidad que ni siquiera la eternidad podrá agotar o ponerle término.

\par
%\textsuperscript{(1169.6)}
\textsuperscript{106:7.7} Para las criaturas finitas del gran universo, el concepto del universo maestro parece ser casi infinito, pero no hay duda de que sus arquitectos absonitos perciben su relación con los desarrollos futuros e inimaginables dentro del YO SOY sin fin. Incluso el espacio mismo no es más que un estado último, un estado atenuado \textit{dentro} de la absolutidad relativa de las zonas tranquilas de espacio intermedio.

\par
%\textsuperscript{(1170.1)}
\textsuperscript{106:7.8} En un momento inconcebiblemente lejano de la eternidad futura, cuando todo el universo maestro esté finalmente acabado, no hay duda de que todos contemplaremos retrospectivamente su historia completa como un simple comienzo, como la simple creación de ciertos fundamentos finitos y trascendentales con vistas a unas metamorfosis mucho más grandes y más cautivadoras en la infinidad sin explorar. En ese momento futuro de la eternidad, el universo maestro parecerá todavía joven; en verdad, siempre será joven ante las posibilidades ilimitadas de la eternidad interminable.

\par
%\textsuperscript{(1170.2)}
\textsuperscript{106:7.9} Es improbable que se alcance un destino infinito, pero eso no impide en lo más mínimo albergar ideas sobre ese destino, y no dudamos en afirmar que si los tres potenciales absolutos pudieran alguna vez manifestarse por completo, sería posible concebir la integración final de la realidad total. Esta realización, producto del desarrollo, está basada en la manifestación total de los Absolutos Incalificado, Universal y de la Deidad, las tres potencialidades cuya unión constituye el estado latente del YO SOY, las realidades en suspenso de la eternidad, las posibilidades en reposo de todo el futuro, y mucho más.

\par
%\textsuperscript{(1170.3)}
\textsuperscript{106:7.10} Lo menos que podemos decir es que estas eventualidades están más bien lejanas; sin embargo, en los mecanismos, las personalidades y las asociaciones de las tres Trinidades creemos detectar la posibilidad teórica de la reunión de las siete fases absolutas del Padre-YO SOY. Esto nos sitúa cara a cara con el concepto de la triple Trinidad, que engloba a la Trinidad del Paraíso, cuyo estado es existencial, y a las dos Trinidades que aparecen posteriormente, cuya naturaleza y origen es experiencial.

\section*{8. La Trinidad de Trinidades}
\par
%\textsuperscript{(1170.4)}
\textsuperscript{106:8.1} Es difícil describir a la mente humana la naturaleza de la Trinidad de Trinidades; es la suma real de la totalidad de la infinidad experiencial, tal como ésta se manifiesta en una infinidad teórica de realización en la eternidad. En la Trinidad de Trinidades, lo infinito experiencial logra identificarse con lo infinito existencial, y los dos forman uno solo en el YO SOY preexperiencial y preexistencial. La Trinidad de Trinidades es la expresión final de todo lo que contienen las quince triunidades y las triodidades asociadas. Las finalidades son difíciles de comprender para los seres relativos, ya sean éstas existenciales o experienciales; por eso siempre han de ser presentadas bajo la forma de relatividades.

\par
%\textsuperscript{(1170.5)}
\textsuperscript{106:8.2} La Trinidad de Trinidades existe en diversas fases. Contiene posibilidades, probabilidades e inevitabilidades que desconciertan la imaginación de los seres situados muy por encima del nivel humano. Contiene repercusiones probablemente insospechadas por los filósofos celestiales, pues estas repercusiones se encuentran en las triunidades, y las triunidades son, a fin de cuentas, insondables.

\par
%\textsuperscript{(1170.6)}
\textsuperscript{106:8.3} Se puede describir de diversas maneras la Trinidad de Trinidades. Escogemos presentar este concepto en tres niveles, que son los siguientes:

\par
%\textsuperscript{(1170.7)}
\textsuperscript{106:8.4} 1. El nivel de las tres Trinidades.

\par
%\textsuperscript{(1170.8)}
\textsuperscript{106:8.5} 2. El nivel de la Deidad experiencial.

\par
%\textsuperscript{(1170.9)}
\textsuperscript{106:8.6} 3. El nivel del YO SOY.

\par
%\textsuperscript{(1170.10)}
\textsuperscript{106:8.7} Se trata de unos niveles que reflejan una unificación creciente. En realidad, la Trinidad de Trinidades es el primer nivel, mientras que el segundo y el tercero son derivados y unificaciones del primero.

\par
%\textsuperscript{(1171.1)}
\textsuperscript{106:8.8} EL PRIMER NIVEL: Se cree que en este nivel de asociación inicial, las tres Trinidades funcionan como agrupaciones perfectamente sincronizadas, aunque distintas, de personalidades de la Deidad.

\par
%\textsuperscript{(1171.2)}
\textsuperscript{106:8.9} 1. \textit{La Trinidad del Paraíso}, la asociación de las tres Deidades del Paraíso ---el Padre, el Hijo y el Espíritu. Hay que recordar que la Trinidad del Paraíso posee una triple función ---una función absoluta, una función trascendental (la Trinidad de Ultimacía) y una función finita (la Trinidad de Supremacía). La Trinidad del Paraíso es cualquiera de estas funciones y todas a la vez, en cualquier momento y en todo momento.

\par
%\textsuperscript{(1171.3)}
\textsuperscript{106:8.10} 2. \textit{La Trinidad Última}. Es la asociación de deidades compuesta por los Creadores Supremos, Dios Supremo y los Arquitectos del Universo Maestro. Aunque ésta es una presentación adecuada de los aspectos de la divinidad de esta Trinidad, hay que indicar que esta Trinidad posee otras fases que parecen sin embargo coordinarse perfectamente con los aspectos de la divinidad.

\par
%\textsuperscript{(1171.4)}
\textsuperscript{106:8.11} 3. \textit{La Trinidad Absoluta}. Es la agrupación de Dios Supremo, Dios Último y el Consumador del Destino del Universo con respecto a todos los valores de la divinidad. Algunas otras fases de esta agrupación trina tienen relación con los valores que reflejan algo distinto a la divinidad en el cosmos en expansión. Pero estos valores se están unificando con las fases de la divinidad, al igual que los aspectos del poder y de la personalidad de las Deidades experienciales están ahora en proceso de síntesis experiencial.

\par
%\textsuperscript{(1171.5)}
\textsuperscript{106:8.12} La asociación de estas tres Trinidades en la Trinidad de Trinidades proporciona la posibilidad de una integración ilimitada de la realidad. Esta agrupación contiene las causas, los estados intermedios y los efectos finales; los iniciadores, los realizadores y los consumadores; los comienzos, las existencias y los destinos. La asociación del Padre y el Hijo se ha convertido en la asociación del Hijo y el Espíritu, luego en la del Espíritu y el Supremo, después en la del Supremo y el Último, más tarde en la del Último y el Absoluto, y finalmente en la del Absoluto y el Padre-Infinito ---la culminación del ciclo de la realidad. Del mismo modo, pero en otras fases que no están tan directamente relacionadas con la divinidad y la personalidad, la Gran Fuente-Centro Primera realiza en sí misma la no limitación de la realidad en torno al círculo de la eternidad, desde la absolutidad de la autoexistencia, pasando por la perpetuidad de la autorrevelación, hasta la finalidad de la autorrealización ---desde el absoluto de los existenciales hasta la finalidad de los experienciales.

\par
%\textsuperscript{(1171.6)}
\textsuperscript{106:8.13} EL SEGUNDO NIVEL: La coordinación de las tres Trinidades supone inevitablemente la unión asociativa de las Deidades experienciales que están genéticamente asociadas con estas Trinidades. La naturaleza de este segundo nivel ha sido presentada a veces como sigue:

\par
%\textsuperscript{(1171.7)}
\textsuperscript{106:8.14} 1. \textit{El Supremo}. Es la consecuencia en forma de deidad de la unidad de la Trinidad del Paraíso en conexión experiencial con los Hijos Creadores y las Hijas Creativas de las Deidades del Paraíso. El Supremo es la personificación, en forma de deidad, de la finalización de la primera etapa de la evolución finita.

\par
%\textsuperscript{(1171.8)}
\textsuperscript{106:8.15} 2. \textit{El Último}. Es la consecuencia en forma de deidad de la unidad existenciada de la segunda Trinidad, la personificación trascendental y absonita de la divinidad. El Último consiste en una unidad, variablemente considerada, de numerosas cualidades, y el concepto humano del mismo haría bien en incluir al menos aquellas fases de la ultimacía que dirigen el control, que son experimentables personalmente y que unifican mediante tensiones, pero la Deidad existenciada contiene otros muchos aspectos no revelados. Aunque el Último y el Supremo son comparables, no son idénticos, y el Último no es tampoco una simple amplificación del Supremo.

\par
%\textsuperscript{(1172.1)}
\textsuperscript{106:8.16} 3. \textit{El Absoluto}. Existen muchas teorías sobre el carácter del tercer miembro del segundo nivel de la Trinidad de Trinidades. Dios Absoluto está sin duda implicado en esta asociación como consecuencia, bajo la forma de personalidad, de la función final de la Trinidad Absoluta, y sin embargo el Absoluto de la Deidad es una realidad existencial que pertenece por su estado a la eternidad.

\par
%\textsuperscript{(1172.2)}
\textsuperscript{106:8.17} La dificultad para concebir este tercer miembro es inherente al hecho de que presuponer su presencia como miembro significa realmente que no hay más que un solo Absoluto. Teóricamente, si un acontecimiento así pudiera ocurrir, contemplaríamos la unificación \textit{experiencial} de los tres Absolutos en uno solo. Y nos enseñan que, en la infinidad y \textit{existencialmente}, hay un solo Absoluto. Aunque la identidad de este tercer miembro está muy poco clara, a menudo se supone que puede consistir en alguna forma de conexión inimaginable y de manifestación cósmica de los Absolutos de la Deidad, Universal e Incalificado. Es cierto que la Trinidad de Trinidades difícilmente podría conseguir ejercer su completa actividad sin la unificación total de los tres Absolutos, y los tres Absolutos difícilmente se pueden unificar sin que todos los potenciales infinitos se hayan realizado por completo.

\par
%\textsuperscript{(1172.3)}
\textsuperscript{106:8.18} Si se concibe al Absoluto Universal como el tercer miembro de la Trinidad de Trinidades, esto representará probablemente una mínima deformación de la verdad, con tal que este concepto imagine al Universal no solamente como estático y potencial, sino también como asociativo. Pero no percibimos todavía cómo está relacionado con los aspectos creativos y evolutivos de la función de la Deidad total.

\par
%\textsuperscript{(1172.4)}
\textsuperscript{106:8.19} Aunque es difícil formarse un concepto completo de la Trinidad de Trinidades, no es tan difícil hacerse una idea limitada. Si concebimos el segundo nivel de la Trinidad de Trinidades como esencialmente personal, es completamente posible suponer que la unión de Dios Supremo, Dios Último y Dios Absoluto es la repercusión personal de la unión de las Trinidades personales que son ancestrales a estas Deidades experienciales. Aventuramos la opinión de que estas tres Deidades experienciales se unificarán seguramente en el segundo nivel como consecuencia directa de la unidad creciente de sus Trinidades ancestrales y causativas, las cuales componen el primer nivel.

\par
%\textsuperscript{(1172.5)}
\textsuperscript{106:8.20} El primer nivel está compuesto de tres Trinidades; el segundo nivel existe como la asociación de personalidad que engloba a las personalidades experiencial-evolucionadas, experiencial-existenciadas y experiencial-existenciales de la Deidad. Independientemente de cualquier dificultad conceptual para comprender a la Trinidad de Trinidades en su totalidad, la asociación personal de estas tres Deidades en el segundo nivel se ha manifestado en nuestra propia época universal en el fenómeno de convertir a Majeston en una deidad, el cual se hizo real en este segundo nivel gracias al Absoluto de la Deidad, que actuó a través del Último y en respuesta al mandato creativo inicial del Ser Supremo.

\par
%\textsuperscript{(1172.6)}
\textsuperscript{106:8.21} EL TERCER NIVEL. La relación recíproca entre todas las fases de todos los tipos de realidad que existen, han existido o pudieran existir en la totalidad de la infinidad, está incluida en la hipótesis incalificada del segundo nivel de la Trinidad de Trinidades. El Ser Supremo no sólo es espíritu, sino también mente, poder y experiencia. El Último es todo esto y mucho más, mientras que en el concepto conjunto de la unicidad de los Absolutos de la Deidad, Universal e Incalificado, dicho concepto incluye la finalidad absoluta de toda la realización de la realidad.

\par
%\textsuperscript{(1172.7)}
\textsuperscript{106:8.22} En la unión que forman el Supremo, el Último y el Absoluto concluído, podría producirse la reunión funcional de aquellos aspectos de la infinidad que al principio fueron segmentados por el YO SOY y que ocasionaron la aparición de los Siete Absolutos de la Infinidad. Aunque los filósofos del universo estiman que se trata de una probabilidad sumamente lejana, sin embargo a menudo nos hacemos la pregunta siguiente: Si el segundo nivel de la Trinidad de Trinidades pudiera alcanzar alguna vez una unidad trinitaria, ¿qué sucedería entonces como consecuencia de esta unidad de deidad? No lo sabemos, pero estamos convencidos de que conduciría directamente a reconocer que el YO SOY podría ser alcanzado por experiencia. Desde el punto de vista de los seres personales, esto podría significar que el incognoscible YO SOY se ha vuelto accesible a la experiencia como Padre-Infinito. Lo que estos destinos absolutos puedan significar desde un punto de vista no personal es otra cuestión que sólo la eternidad podrá posiblemente clarificar. Pero cuando consideramos estas eventualidades lejanas como criaturas personales, deducimos que el destino final de todas las personalidades es conocer de manera final al Padre Universal de esas mismas personalidades.

\par
%\textsuperscript{(1173.1)}
\textsuperscript{106:8.23} El YO SOY, tal como lo concebimos filosóficamente en la eternidad pasada, está solo, no hay nadie más que él. Cuando miramos hacia la eternidad futura, no vemos la posibilidad de que el YO SOY, como existencial, pueda cambiar, pero nos inclinamos a pronosticar una enorme diferencia experiencial. Este concepto del YO SOY implica la completa realización de sí mismo ---abarca al conjunto ilimitado de personalidades que habrán participado volitivamente en la autorrevelación del YO SOY, y que permanecerán eternamente como partes volitivas absolutas de la totalidad de la infinidad, los hijos finales del Padre absoluto.

\section*{9. La unificación existencial infinita}
\par
%\textsuperscript{(1173.2)}
\textsuperscript{106:9.1} En el concepto de la Trinidad de Trinidades, admitimos la posible unificación experiencial de la realidad ilimitada, y a veces teorizamos que todo esto podría suceder en la inmensa lejanía de la distante eternidad. Pero existe no obstante una unificación presente y real de la infinidad en esta misma era, como en todas las eras pasadas y futuras del universo; esta unificación es existencial en la Trinidad del Paraíso. La unificación de la infinidad como realidad experiencial está inconcebiblemente lejana, pero una unidad incalificada de la infinidad domina ahora el momento presente de la existencia universal, y une las divergencias de toda la realidad con una majestad existencial \textit{absoluta}.

\par
%\textsuperscript{(1173.3)}
\textsuperscript{106:9.2} Cuando las criaturas finitas intentan concebir la unificación infinita en los niveles de finalidad de la eternidad consumada, se encuentran cara a cara con las limitaciones intelectuales inherentes a sus existencias finitas. El tiempo, el espacio y la experiencia constituyen unas barreras para la comprensión de las criaturas; y sin embargo, sin el tiempo, aparte del espacio y a excepción de la experiencia, ninguna criatura podría conseguir siquiera una comprensión limitada de la realidad universal. Sin la sensibilidad al tiempo, ninguna criatura evolutiva podría percibir de ninguna manera las relaciones secuenciales. Sin la percepción espacial, ninguna criatura podría comprender las relaciones de simultaneidad. Sin la experiencia, ninguna criatura evolutiva podría existir siquiera; sólo los Siete Absolutos de la Infinidad trascienden realmente la experiencia, e incluso ellos mismos pueden ser experienciales en algunas fases.

\par
%\textsuperscript{(1173.4)}
\textsuperscript{106:9.3} El tiempo, el espacio y la experiencia son los mayores auxiliares del hombre para percibir, de manera relativa, la realidad, y son sin embargo sus obstáculos más formidables para percibir, de manera completa, la realidad. Los mortales, y otras muchas criaturas del universo, necesitan pensar en los potenciales como que se hacen reales en el espacio y evolucionan hasta su fructificación en el tiempo, pero todo este proceso es un fenómeno espacio-temporal que no ocurre realmente en el Paraíso ni en la eternidad. En el nivel absoluto no existe ni el tiempo ni el espacio; todos los potenciales se pueden percibir allí como actuales.

\par
%\textsuperscript{(1173.5)}
\textsuperscript{106:9.4} El concepto de la unificación de toda la realidad, ya se produzca en esta era o en cualquier otra era del universo, es básicamente doble: existencial y experiencial. Esta unidad está en proceso de realizarse experiencialmente en la Trinidad de Trinidades, pero el grado de manifestación aparente de esta triple Trinidad es directamente proporcional a la desaparición de las atenuaciones e imperfecciones de la realidad en el cosmos. Sin embargo, la integración total de la realidad está presente de manera incalificada, eterna y existencial en la Trinidad del Paraíso, dentro de la cual la realidad infinita está absolutamente unificada en este mismo momento del universo.

\par
%\textsuperscript{(1174.1)}
\textsuperscript{106:9.5} Los puntos de vista experiencial y existencial crean una paradoja inevitable que está basada en parte en el hecho de que la Trinidad del Paraíso y la Trinidad de Trinidades son, cada una de ellas, un conjunto de relaciones que ha existido desde la eternidad, y que los mortales sólo pueden percibir como una relatividad espacio-temporal. El concepto humano sobre la manifestación experiencial gradual de la Trinidad de Trinidades ---el punto de vista temporal--- debe ser completado con el postulado adicional de que esto \textit{es} ya una realidad factual ---el punto de vista de la eternidad. Pero, ¿cómo se pueden conciliar estos dos puntos de vista? Sugerimos a los mortales finitos que acepten la verdad de que la Trinidad del Paraíso es la unificación existencial de la infinidad, y que la incapacidad para detectar la presencia efectiva y la manifestación completa de la Trinidad de Trinidades experiencial, se debe en parte a las deformaciones recíprocas causadas por:

\par
%\textsuperscript{(1174.2)}
\textsuperscript{106:9.6} 1. El limitado punto de vista humano, la incapacidad para captar el concepto de la eternidad incalificada.

\par
%\textsuperscript{(1174.3)}
\textsuperscript{106:9.7} 2. El estado imperfecto humano, la lejanía de los experienciales respecto al nivel absoluto.

\par
%\textsuperscript{(1174.4)}
\textsuperscript{106:9.8} 3. El propósito de la existencia humana, el hecho de que la humanidad está diseñada para evolucionar mediante la técnica de la experiencia y, por consiguiente, tiene que depender de la experiencia de manera inherente y constitutiva. Sólo un Absoluto puede ser a la vez existencial y experiencial.

\par
%\textsuperscript{(1174.5)}
\textsuperscript{106:9.9} El Padre Universal, en la Trinidad del Paraíso, es el YO SOY de la Trinidad de Trinidades, y las limitaciones finitas son las que impiden experimentar al Padre como infinito. El concepto del YO SOY \textit{existencial}, solitario, pretrinitario e inaccesible, y el postulado del YO SOY \textit{experiencial}, accesible y posterior a la Trinidad de Trinidades, no son más que una sola y misma hipótesis; ningún cambio real se ha producido en el Infinito; todos los desarrollos aparentes se deben a las capacidades crecientes para abarcar la realidad y para comprender el cosmos.

\par
%\textsuperscript{(1174.6)}
\textsuperscript{106:9.10} A fin de cuentas, el YO SOY debe existir \textit{antes} que todos los existenciales y \textit{después} de todos los experienciales. Aunque estas ideas no puedan clarificar en la mente humana las paradojas de la eternidad y de la infinidad, al menos deberían estimular a los intelectos finitos a intentar resolver de nuevo estos problemas sin fin, unos problemas que continuarán intrigándoos en Salvington y más tarde como finalitarios, y después durante todo el futuro interminable de vuestra carrera eterna en los universos en vías de expansión.

\par
%\textsuperscript{(1174.7)}
\textsuperscript{106:9.11} Tarde o temprano todas las personalidades del universo empiezan a darse cuenta de que la búsqueda final de la eternidad es la exploración sin fin de la infinidad, el viaje interminable de descubrimiento dentro de la absolutidad de la Fuente-Centro Primera. Tarde o temprano todos nos volvemos conscientes de que todo crecimiento de las criaturas es proporcional a su identificación con el Padre. Llegamos a comprender que vivir la voluntad de Dios es el pasaporte eterno para las posibilidades sin fin de la misma infinidad. Los mortales se darán cuenta algún día de que el éxito en la búsqueda del Infinito es directamente proporcional a la semejanza que se alcance con el Padre, y que durante esta era del universo, las realidades del Padre están reveladas en las cualidades de la divinidad. Y las criaturas del universo se apoderan personalmente de estas cualidades de la divinidad mediante la experiencia de vivir divinamente, y vivir divinamente significa vivir realmente la voluntad de Dios.

\par
%\textsuperscript{(1175.1)}
\textsuperscript{106:9.12} Para las criaturas materiales, evolutivas y finitas, una vida basada en vivir la voluntad del Padre conduce directamente a alcanzar la supremacía espiritual en el ámbito de la personalidad, y lleva a estas criaturas a avanzar un paso más en la comprensión del Padre-Infinito. Una vida centrada así en el Padre está basada en la verdad, es sensible a la belleza y está dominada por la bondad. La persona que conoce así a Dios está interiormente iluminada por la adoración, y exteriormente consagrada de todo corazón al servicio de la fraternidad universal de todas las personalidades, un ministerio de servicio lleno de misericordia y motivado por el amor, mientras que todas estas cualidades de vida están unificadas en la personalidad evolutiva en unos niveles siempre ascendentes de sabiduría cósmica, de autorrealización, de descubrimiento de Dios y de adoración del Padre.

\par
%\textsuperscript{(1175.2)}
\textsuperscript{106:9.13} [Presentado por un Melquisedek de Nebadon].


\chapter{Documento 107. El origen y la naturaleza de los Ajustadores del Pensamiento}
\par
%\textsuperscript{(1176.1)}
\textsuperscript{107:0.1} AUNQUE el Padre Universal reside personalmente en el Paraíso, en el centro mismo de los universos, también está realmente presente en los mundos del espacio en la mente de sus innumerables hijos del tiempo, ya que vive dentro de ellos bajo la forma de los Monitores de Misterio. El Padre eterno es el que se encuentra más lejos de sus hijos mortales planetarios, y es al mismo tiempo el que está más íntimamente asociado con ellos.

\par
%\textsuperscript{(1176.2)}
\textsuperscript{107:0.2} Los Ajustadores son la realidad del amor del Padre, encarnado en el alma de los hombres; son la verdadera promesa de la carrera eterna del hombre, encarcelada dentro de la mente mortal; son la esencia de la personalidad finalitaria perfeccionada del hombre, que éste puede saborear de antemano en el tiempo a medida que domina progresivamente la técnica divina de lograr vivir la voluntad del Padre, paso a paso, a través de la ascensión de un universo tras otro, hasta que alcanza realmente la presencia divina de su Padre Paradisiaco.

\par
%\textsuperscript{(1176.3)}
\textsuperscript{107:0.3} Dios, después de ordenarle al hombre que sea perfecto como Él mismo es perfecto\footnote{\textit{Sed perfectos}: Gn 17:1; 1 Re 8:61; Lv 19:2; Dt 18:13; Mt 5:48; 2 Co 13:11; Stg 1:4; 1 P 1:16.}, ha descendido en forma de Ajustador para convertirse en el asociado experiencial del hombre a fin de lograr el destino celestial que ha sido así ordenado. El fragmento de Dios que reside en la mente del hombre es la garantía absoluta e incalificada de que el hombre puede encontrar al Padre Universal en asociación con este Ajustador divino, el cual salió de Dios para encontrar al hombre y hacer de él su hijo incluso durante sus días en la carne.

\par
%\textsuperscript{(1176.4)}
\textsuperscript{107:0.4} Cualquier mortal que ha visto a un Hijo Creador ha visto al Padre Universal\footnote{\textit{El que ha visto al Hijo ha visto al Padre}: Jn 12:45; 14:7-9.}, y aquel que está habitado por un Ajustador divino está habitado por el Padre Paradisiaco. Todo mortal que sigue, consciente o inconscientemente, las directrices de su Ajustador interior, vive de acuerdo con la voluntad de Dios. La conciencia de la presencia del Ajustador es la conciencia de la presencia de Dios. La fusión eterna del Ajustador con el alma evolutiva del hombre es la experiencia objetiva de la unión eterna con Dios en calidad de asociado universal de la Deidad.

\par
%\textsuperscript{(1176.5)}
\textsuperscript{107:0.5} El Ajustador es el que crea, dentro del hombre, ese anhelo insaciable y ese ansia incesante de ser semejante a Dios, de alcanzar el Paraíso, y allí, delante de la persona real de la Deidad, de adorar a la fuente infinita de este don divino. El Ajustador es la presencia viviente que conecta realmente al hijo mortal con su Padre Paradisiaco y le acerca cada vez más al Padre\footnote{\textit{Gravedad espiritual}: Jer 31:3; Jn 6:44; 12:32.}. El Ajustador es para nosotros aquello que nivela de manera compensatoria la enorme tensión universal creada por la distancia que separa al hombre de Dios, y por el grado de parcialidad del hombre en contraste con la universalidad del Padre eterno.

\par
%\textsuperscript{(1176.6)}
\textsuperscript{107:0.6} El Ajustador es una esencia absoluta de un ser infinito, encarcelada en la mente de una criatura finita que, dependiendo de la elección de dicho mortal, puede consumar finalmente esta unión temporal entre Dios y el hombre y hacer verdaderamente real una nueva clase de ser para un servicio universal sin fin. El Ajustador es la realidad universal divina que convierte en un hecho la verdad de que Dios es el Padre del hombre. El Ajustador es la brújula cósmica infalible del hombre, que orienta siempre e infaliblemente el alma hacia Dios.

\par
%\textsuperscript{(1177.1)}
\textsuperscript{107:0.7} En los mundos evolutivos, las criaturas volitivas atraviesan tres etapas generales de desarrollo del ser: desde la llegada del Ajustador hasta el pleno desarrollo relativo, alrededor de los veinte años de edad en Urantia, los Monitores se denominan a veces Cambiadores del Pensamiento. Desde este momento hasta que se alcanza la edad del juicio, hacia los cuarenta años, los Monitores de Misterio se llaman Ajustadores del Pensamiento. Desde que se alcanza el juicio hasta la liberación de la carne, a menudo se les califica de Controladores del Pensamiento. Estas tres fases de la vida mortal no tienen ninguna relación con las tres etapas del progreso de los Ajustadores en la duplicación de la mente y en la evolución del alma.

\section*{1. El origen de los Ajustadores del Pensamiento}
\par
%\textsuperscript{(1177.2)}
\textsuperscript{107:1.1} Puesto que los Ajustadores del Pensamiento forman parte de la esencia de la Deidad original, nadie puede atreverse a disertar con autoridad sobre su naturaleza y origen; yo sólo puedo comunicar las tradiciones de Salvington y las creencias de Uversa; sólo puedo explicar cómo consideramos a estos Monitores de Misterio y a sus entidades asociadas en todo el gran universo.

\par
%\textsuperscript{(1177.3)}
\textsuperscript{107:1.2} Aunque circulan opiniones diversas sobre la manera en que se conceden los Ajustadores del Pensamiento, no existen tales diferencias en lo que se refiere a su origen; todos están de acuerdo en que proceden directamente del Padre Universal, la Fuente-Centro Primera. No son seres creados; son entidades fragmentadas que representan la presencia de hecho del Dios infinito. Al igual que sus numerosos asociados no revelados, los Ajustadores son de una divinidad pura y sin mezcla, partes incalificadas y no atenuadas de la Deidad; son de Dios y, en la medida en que podemos discernirlo, \textit{son Dios}.

\par
%\textsuperscript{(1177.4)}
\textsuperscript{107:1.3} En cuanto al momento en que empiezan su existencia separada fuera de la absolutidad de la Fuente-Centro Primera, no lo sabemos; y tampoco conocemos su número. Sabemos muy poco sobre sus carreras hasta que llegan a los planetas del tiempo para residir en las mentes humanas, pero desde ese momento en adelante, estamos más o menos familiarizados con su progresión cósmica hasta, e incluyendo, la culminación de su destino trino, es decir: la obtención de la personalidad mediante la fusión con un ascendente mortal, la obtención de la personalidad por mandato del Padre Universal, o la liberación de los Ajustadores del Pensamiento de sus tareas conocidas.

\par
%\textsuperscript{(1177.5)}
\textsuperscript{107:1.4} Aunque no lo sabemos, suponemos que los Ajustadores son continuamente individualizados a medida que se amplía el universo y que crece el número de candidatos destinados a fusionar con un Ajustador. Pero también es igualmente posible que cometamos un error al intentar atribuir una magnitud numérica a los Ajustadores; al igual que Dios mismo, estos fragmentos de su naturaleza insondable pueden ser existencialmente infinitos.

\par
%\textsuperscript{(1177.6)}
\textsuperscript{107:1.5} La técnica del origen de los Ajustadores del Pensamiento es una de las funciones no reveladas del Padre Universal. Tenemos todas las razones para creer que ninguno de los otros asociados absolutos de la Fuente-Centro Primera tiene nada que ver con la producción de los fragmentos del Padre. Los Ajustadores son simple y eternamente unos dones divinos; son de Dios, proceden de Dios y son semejantes a Dios.

\par
%\textsuperscript{(1177.7)}
\textsuperscript{107:1.6} En sus relaciones con las criaturas con las que fusionan, revelan un amor celestial y un ministerio espiritual que confirman profundamente la declaración de que Dios es espíritu. Pero además de este ministerio trascendente, hay muchas cosas que tienen lugar y que nunca se han revelado a los mortales de Urantia. Tampoco comprendemos por completo qué sucede realmente cuando el Padre Universal da algo de sí mismo para que forme parte de la personalidad de una criatura temporal. La progresión ascendente de los finalitarios del Paraíso tampoco ha revelado todavía todas las posibilidades inherentes a esta asociación celestial entre el hombre y Dios. A fin de cuentas, los fragmentos del Padre deben ser un don del Dios absoluto a aquellas criaturas cuyo destino abarca la posibilidad de alcanzar a Dios como absoluto.

\par
%\textsuperscript{(1178.1)}
\textsuperscript{107:1.7} Al igual que el Padre Universal fragmenta su Deidad prepersonal, el Espíritu Infinito individualiza porciones de su espíritu premental para que residan y fusionen realmente con las almas evolutivas de los mortales supervivientes de la serie que fusiona con el espíritu. Pero la naturaleza del Hijo Eterno no se puede fragmentar así; el espíritu del Hijo Original es o bien difuso o diferenciadamente personal. Las criaturas fusionadas con el Hijo están unidas a los dones individualizados del espíritu de los Hijos Creadores del Hijo Eterno.

\section*{2. Clasificación de los Ajustadores}
\par
%\textsuperscript{(1178.2)}
\textsuperscript{107:2.1} Los Ajustadores son individualizados como entidades vírgenes, y todos están destinados a convertirse en Monitores liberados, fusionados o Personalizados. Tenemos entendido que existen siete órdenes de Ajustadores del Pensamiento, aunque no comprendemos del todo estas divisiones. A menudo nos referimos a estas diferentes órdenes como sigue:

\par
%\textsuperscript{(1178.3)}
\textsuperscript{107:2.2} 1. \textit{Los Ajustadores vírgenes}, aquellos que sirven durante su misión inicial en la mente de los candidatos evolutivos a la supervivencia eterna. La naturaleza divina de los Monitores de Misterio es eternamente uniforme. Su naturaleza experiencial también es uniforme cuando salen por primera vez de Divinington; su diferenciación experiencial posterior es el resultado de su experiencia efectiva en el ministerio universal.

\par
%\textsuperscript{(1178.4)}
\textsuperscript{107:2.3} 2. \textit{Los Ajustadores avanzados}, aquellos que han servido durante uno o más períodos con las criaturas volitivas en los mundos donde la fusión final tiene lugar entre la identidad de la criatura temporal y una porción individualizada del espíritu de la manifestación en el universo local de la Fuente-Centro Tercera.

\par
%\textsuperscript{(1178.5)}
\textsuperscript{107:2.4} 3. \textit{Los Ajustadores supremos}, aquellos Monitores que han servido en la aventura del tiempo en los mundos evolutivos, pero cuyos asociados humanos han rechazado por alguna razón la supervivencia eterna, y aquellos Monitores que han sido destinados posteriormente a otras aventuras en otros mortales y en otros mundos evolutivos. Un Ajustador supremo no es más divino que un Monitor virgen, pero ha tenido más experiencia y puede hacer cosas en la mente humana que un Ajustador menos experimentado no podría hacer.

\par
%\textsuperscript{(1178.6)}
\textsuperscript{107:2.5} 4. \textit{Los Ajustadores desaparecidos}. Aquí se produce una laguna en nuestros esfuerzos por seguir la carrera de los Monitores de Misterio. Existe una cuarta fase de servicio sobre la que no estamos seguros. Los Melquisedeks enseñan que los Ajustadores de la cuarta fase están realizando misiones independientes, deambulando por el universo de universos. Los Mensajeros Solitarios tienden a creer que están unidos a la Fuente-Centro Primera, disfrutando de un período de asociación reconfortante con el Padre mismo. Y es totalmente posible que un Ajustador pueda estar deambulando por el universo maestro, y estar simultáneamente unido al Padre omnipresente.

\par
%\textsuperscript{(1178.7)}
\textsuperscript{107:2.6} 5. \textit{Los Ajustadores liberados}, los Monitores de Misterio que han sido liberados eternamente del servicio temporal para con los mortales de las esferas en evolución. No sabemos el tipo de actividad que desempeñan.

\par
%\textsuperscript{(1179.1)}
\textsuperscript{107:2.7} 6. \textit{Los Ajustadores fusionados} ---los finalitarios--- aquellos que se han vuelto una sola cosa con las criaturas ascendentes de los superuniversos, los asociados para la eternidad de los ascendentes temporales del Cuerpo Paradisiaco de la Finalidad. Los Ajustadores del Pensamiento fusionan generalmente con los mortales ascendentes del tiempo, y son registrados a su entrada y a su salida de Ascendington con estos mortales supervivientes; siguen el camino de los seres ascendentes. Después de fusionar con un alma ascendente evolutiva, parece que el Ajustador se traslada del nivel existencial absoluto del universo al nivel experiencial finito de la asociación funcional con una personalidad ascendente. Aunque conserva todo el carácter de la naturaleza existencial divina, un Ajustador fusionado se une indisolublemente a la carrera ascendente de un mortal superviviente.

\par
%\textsuperscript{(1179.2)}
\textsuperscript{107:2.8} 7. \textit{Los Ajustadores personalizados}, aquellos que han servido con los Hijos Paradisiacos encarnados, así como muchos otros que han conseguido distinguirse de manera excepcional durante su estancia en un mortal, pero cuyos sujetos han rechazado la supervivencia. Tenemos razones para creer que estos Ajustadores son personalizados por recomendación de los Ancianos de los Días del superuniverso donde han estado asignados.

\par
%\textsuperscript{(1179.3)}
\textsuperscript{107:2.9} Estos misteriosos fragmentos de Dios pueden ser clasificados de muchas maneras: según su tarea en el universo, según el grado de su éxito residiendo en un mortal individual, o incluso según la ascendencia racial del candidato mortal a la fusión.

\section*{3. El hogar de los Ajustadores en Divinington}
\par
%\textsuperscript{(1179.4)}
\textsuperscript{107:3.1} Todas las actividades universales relacionadas con el envío, la gestión, la dirección y el regreso de los Monitores de Misterio en servicio en los siete superuniversos parecen estar centradas en la esfera sagrada de Divinington. Que yo sepa, nadie, salvo los Ajustadores y otras entidades del Padre, ha estado en esta esfera. Parece probable que numerosas entidades prepersonales no reveladas compartan Divinington con los Ajustadores como esfera de origen. Conjeturamos que estas entidades semejantes pueden estar asociadas de alguna manera con el ministerio presente y futuro de los Monitores de Misterio. Pero en realidad no lo sabemos.

\par
%\textsuperscript{(1179.5)}
\textsuperscript{107:3.2} Cuando los Ajustadores del Pensamiento regresan al Padre, vuelven a Divinington, al mundo de su supuesto origen; y existe probablemente, como parte de esta experiencia, un contacto real con la personalidad paradisiaca del Padre, así como con la manifestación especializada de la divinidad del Padre que dicen que está situada en esta esfera secreta.

\par
%\textsuperscript{(1179.6)}
\textsuperscript{107:3.3} Aunque sabemos algo sobre las siete esferas secretas del Paraíso, sabemos menos de Divinington que de las demás. Los seres de las órdenes espirituales elevadas sólo reciben tres mandatos divinos, que son los siguientes:

\par
%\textsuperscript{(1179.7)}
\textsuperscript{107:3.4} 1. Mostrar siempre un respeto adecuado por la experiencia y los dones de sus mayores y superiores.

\par
%\textsuperscript{(1179.8)}
\textsuperscript{107:3.5} 2. Mostrar siempre consideración por las limitaciones y la inexperiencia de los más jóvenes y subordinados.

\par
%\textsuperscript{(1179.9)}
\textsuperscript{107:3.6} 3. No intentar nunca aterrizar en las orillas de Divinington.

\par
%\textsuperscript{(1179.10)}
\textsuperscript{107:3.7} A menudo he pensado que sería totalmente inútil para mí ir a Divinington; probablemente sería incapaz de ver a ninguno de sus residentes, excepto a los seres tales como los Ajustadores Personalizados, que ya he visto en otras partes. Estoy muy seguro de que no hay nada en Divinington que posea un verdadero valor o beneficio para mí, nada esencial para mi crecimiento y desarrollo, pues si no, no me habrían prohibido ir allí.

\par
%\textsuperscript{(1180.1)}
\textsuperscript{107:3.8} Puesto que Divinington nos permite aprender poco o nada sobre la naturaleza y el origen de los Ajustadores, nos vemos obligados a recoger información de mil y una fuentes diferentes, y es necesario reunir, asociar y correlacionar estos datos acumulados para que dicho conocimiento pueda ser informativo.

\par
%\textsuperscript{(1180.2)}
\textsuperscript{107:3.9} El valor y la sabiduría que manifiestan los Ajustadores del Pensamiento sugieren que han sufrido un entrenamiento de una amplitud y de una variedad extraordinarias. Puesto que no son personalidades, esta preparación debe ser impartida en las instituciones educativas de Divinington. Los excepcionales Ajustadores Personalizados constituyen sin duda el personal de las escuelas de formación para Ajustadores de Divinington. Y sabemos que este cuerpo central y supervisor está presidido por el Ajustador, actualmente Personalizado, del primer Hijo Paradisiaco de la Orden de los Migueles que completó su séptuple donación sobre las razas y pueblos de los mundos de su universo.

\par
%\textsuperscript{(1180.3)}
\textsuperscript{107:3.10} Sabemos en realidad muy poca cosa sobre los Ajustadores no personalizados; sólo nos ponemos en contacto y nos comunicamos con las órdenes personalizadas. A estos Ajustadores se les pone un nombre en Divinington, y siempre son conocidos por su nombre y no por su número. Los Ajustadores Personalizados tienen su domicilio permanente en Divinington; esta esfera sagrada es su hogar. Sólo salen de esta residencia por voluntad del Padre Universal. Se encuentran muy pocos en las esferas de los universos locales, pero están presentes en gran número en el universo central.

\section*{4. La naturaleza y la presencia de los Ajustadores}
\par
%\textsuperscript{(1180.4)}
\textsuperscript{107:4.1} Decir que un Ajustador del Pensamiento es divino es reconocer simplemente la naturaleza de su origen. Es muy probable que esta pureza de su divinidad abarque la esencia del potencial de todos los atributos de la Deidad que pueden estar contenidos dentro de un fragmento así de la esencia absoluta de la presencia universal del Padre Paradisiaco eterno e infinito.

\par
%\textsuperscript{(1180.5)}
\textsuperscript{107:4.2} La fuente real del Ajustador debe ser infinita, y antes de fusionar con el alma inmortal de un mortal evolutivo, la realidad del Ajustador debe lindar con la absolutidad. Los Ajustadores no son absolutos en el sentido universal, en el sentido de la Deidad, pero probablemente son verdaderos absolutos dentro de las potencialidades de sus naturalezas fragmentadas. Están restringidos en cuanto a su universalidad, pero no en cuanto a su naturaleza. Son limitados en extensión, pero en intensidad de significado, de valor y de hecho \textit{son absolutos}. Por esta razón, a estos dones divinos a veces les llamamos los fragmentos restringidos absolutos del Padre.

\par
%\textsuperscript{(1180.6)}
\textsuperscript{107:4.3} Ningún Ajustador ha sido nunca desleal hacia el Padre Paradisiaco; las órdenes más humildes de criaturas personales a veces tienen que luchar con compañeros desleales, pero nunca con los Ajustadores; son supremos e infalibles en su esfera celestial de ministerio hacia las criaturas y de función en el universo.

\par
%\textsuperscript{(1180.7)}
\textsuperscript{107:4.4} Los Ajustadores no personalizados sólo son visibles para los Ajustadores Personalizados. Mi orden, la de los Mensajeros Solitarios, así como los Espíritus Inspirados de la Trinidad, pueden detectar la presencia de los Ajustadores por medio de unos fenómenos de reacción espiritual; incluso los serafines pueden a veces discernir la luminosidad espiritual supuestamente asociada a la presencia de los Monitores en la mente material de los hombres; pero ninguno de nosotros es capaz de discernir verdaderamente la presencia real de los Ajustadores, a menos que hayan sido personalizados, aunque sus naturalezas son perceptibles en unión con las personalidades fusionadas de los mortales ascendentes de los mundos evolutivos. La invisibilidad universal de los Ajustadores sugiere poderosamente que su origen y su naturaleza son elevados y exclusivamente divinos.

\par
%\textsuperscript{(1181.1)}
\textsuperscript{107:4.5} Existe una luz característica, una luminosidad espiritual\footnote{\textit{Luminosidad espiritual}: Esd 6:38-44; Is 9:2; Mt 4:16; 5:14-16; Lc 1:79; 2:32; Jn 1:4-9; 3:19-21; 8:12; 9:5; 12:46; 1 Jn 1:5; 2:8.}, que acompaña a esta presencia divina, y que ha sido generalmente asociada con los Ajustadores del Pensamiento. En el universo de Nebadon, esta luminosidad paradisiaca es ampliamente conocida como la <<luz piloto>>\footnote{\textit{Luz piloto}: Lc 1:79.; Ro 2:19.}; en Uversa se le llama la <<luz de la vida>>\footnote{\textit{Luz de la vida}: Jn 8:12.}. En Urantia, a veces se ha hecho referencia a este fenómeno como <<la verdadera luz que ilumina a todo hombre que llega al mundo>>\footnote{\textit{Verdadera luz que ilumina al mundo}: Jn 1:9.}.

\par
%\textsuperscript{(1181.2)}
\textsuperscript{107:4.6} Los Ajustadores del Pensamiento Personalizados son visibles para todos los seres que han alcanzado al Padre Universal. Los Ajustadores de todas las etapas, así como todos los demás seres, entidades, espíritus, personalidades y manifestaciones espirituales, son siempre discernibles por las Personalidades Creadoras Supremas que tienen su origen en las Deidades del Paraíso, y que presiden los gobiernos principales del gran universo.

\par
%\textsuperscript{(1181.3)}
\textsuperscript{107:4.7} ¿Podéis daros cuenta realmente del verdadero significado que tiene la presencia interior del Ajustador?\footnote{\textit{El Ajustador, espíritu interior}: Job 32:8,18; Is 63:10-11; Ez 37:14; Mt 10:20; Lc 17:21; Jn 17:21-23; Ro 8:9-11; 1 Co 3:16-17; 1 Co 6:19; 2 Co 6:16; Gl 2:20; 1 Jn 3:24; 1 Jn 4:12-15; Ap 21:3.} ¿Podéis comprender realmente lo que significa tener un fragmento absoluto de la Deidad absoluta e infinita, el Padre Universal, que reside en vosotros y que fusiona con vuestra naturaleza mortal finita? Cuando el hombre mortal fusiona con un fragmento real de la Causa existencial del cosmos total, ya no se puede poner ningún límite al destino de esta asociación inimaginable y sin precedentes. El hombre descubrirá en la eternidad no solamente la infinidad de la Deidad objetiva, sino también la potencialidad sin fin del fragmento subjetivo de este mismo Dios. El Ajustador estará revelando siempre a la personalidad mortal la maravilla de Dios, y esta revelación celestial nunca podrá tener fin, porque el Ajustador viene de Dios y es como Dios para el hombre mortal.

\section*{5. La dotación mental de los Ajustadores}
\par
%\textsuperscript{(1181.4)}
\textsuperscript{107:5.1} Los mortales evolutivos tienden a considerar que la mente es como una mediación cósmica entre el espíritu y la materia, ya que éste es en verdad el ministerio principal de la mente tal como vosotros podéis discernirlo. Por eso a los humanos les resulta muy difícil percibir que los Ajustadores del Pensamiento tengan una mente, pues los Ajustadores son fragmentaciones de Dios en un nivel absoluto de realidad que no solamente es prepersonal, sino también anterior a toda divergencia entre la energía y el espíritu. En un nivel monista anterior a la diferenciación entre la energía y el espíritu no podría haber ninguna función mediadora de la mente, puesto que no existen divergencias para tener que mediar entre ellas.

\par
%\textsuperscript{(1181.5)}
\textsuperscript{107:5.2} Puesto que los Ajustadores pueden planificar, trabajar y amar, deben tener unos poderes en su individualidad proporcionales a la mente. Todos los tipos de Monitores que se encuentran por encima del primer grupo, o grupo virgen, poseen la capacidad ilimitada de comunicarse entre sí. En cuanto a la naturaleza y el contenido de sus intercomunicaciones, podemos revelar muy poco, porque no lo sabemos. Sabemos además que deben estar dotados de alguna forma de mente, porque si no, nunca podrían ser personalizados.

\par
%\textsuperscript{(1181.6)}
\textsuperscript{107:5.3} La dotación mental del Ajustador del Pensamiento es semejante a la \textit{dotación mental} del Padre Universal y del Hijo Eterno ---que son ancestrales a las \textit{mentes} que han surgido del Actor Conjunto\footnote{\textit{Mente espiritual}: Ro 8:27; 11:34; 1 Co 2:16; Ef 4:23; Flp 2:5.}.

\par
%\textsuperscript{(1181.7)}
\textsuperscript{107:5.4} El tipo de mente que se da por sentado en un Ajustador debe ser similar a la dotación mental de otras numerosas órdenes de entidades prepersonales que probablemente se originan de la misma manera en la Fuente-Centro Primera. Aunque muchas de estas órdenes no han sido reveladas en Urantia, todas muestran cualidades mentales. A estas individualizaciones de la Deidad original también les resulta posible unificarse con numerosos tipos evolutivos de seres no mortales, e incluso con un número limitado de seres no evolutivos que han desarrollado la capacidad de fusionarse con estos fragmentos de la Deidad.

\par
%\textsuperscript{(1182.1)}
\textsuperscript{107:5.5} Cuando un Ajustador del Pensamiento ha fusionado con el alma morontial inmortal en evolución del humano superviviente, la mente del Ajustador sólo se puede identificar como separada de la mente de la criatura hasta que el mortal ascendente alcanza los niveles espirituales de la progresión universal.

\par
%\textsuperscript{(1182.2)}
\textsuperscript{107:5.6} Cuando estos espíritus del sexto grado alcanzan los niveles finalitarios de la experiencia ascendente, parecen transmutar un factor mental que representa la unión de ciertas fases de la mente del mortal y de la mente del Ajustador que habían funcionado anteriormente como vínculo entre las fases humana y divina de estas personalidades ascendentes. Esta cualidad mental experiencial probablemente se <<suprematiza>>, y acrecienta posteriormente la dotación experiencial de la Deidad evolutiva ---del Ser Supremo.

\section*{6. Los Ajustadores como puros espíritus}
\par
%\textsuperscript{(1182.3)}
\textsuperscript{107:6.1} Los Ajustadores del Pensamiento, tal como se pueden encontrar en la experiencia de las criaturas, revelan la presencia y la guía de una influencia espiritual. El Ajustador es en verdad un espíritu, un espíritu puro, pero más que un espíritu. Nunca hemos sido capaces de clasificar satisfactoriamente a los Monitores de Misterio; todo lo que se puede decir con certeza de ellos es que son verdaderamente semejantes a Dios.

\par
%\textsuperscript{(1182.4)}
\textsuperscript{107:6.2} El Ajustador es la posibilidad que tiene el hombre de lograr la eternidad; el hombre es la posibilidad que tiene el Ajustador de lograr la personalidad. Vuestro Ajustador individual trabaja para espiritualizaros con la esperanza de eternizar vuestra identidad temporal. Los Ajustadores están saturados del hermoso amor del Padre de los espíritus, un amor que se dona por sí mismo. Os aman de manera real y divina; son los prisioneros de una esperanza espiritual, confinados en la mente de los hombres. Desean ardientemente que vuestra mente mortal alcance la divinidad para que pueda terminar su soledad, para poder ser liberados con vosotros de las limitaciones de la investidura material y del ropaje del tiempo.

\par
%\textsuperscript{(1182.5)}
\textsuperscript{107:6.3} Vuestro camino hacia el Paraíso es el camino del logro espiritual, y la naturaleza del Ajustador os descubrirá fielmente la revelación de la naturaleza espiritual del Padre Universal. Más allá de la ascensión al Paraíso y en las etapas postfinalitarias de la carrera eterna, es posible que el Ajustador se ponga en contacto con su antiguo compañero humano para llevar a cabo un ministerio distinto al espiritual; pero la ascensión al Paraíso y la carrera finalitaria representan la asociación entre el mortal que conoce a Dios y se espiritualiza, y el ministerio espiritual del Ajustador que revela a Dios.

\par
%\textsuperscript{(1182.6)}
\textsuperscript{107:6.4} Sabemos que los Ajustadores del Pensamiento son espíritus, espíritus puros, probablemente espíritus absolutos. Pero el Ajustador debe ser también algo más que una realidad espiritual exclusiva. Además de la presumible dotación mental, también están presentes los factores de energía pura. Si recordáis que Dios es la fuente de la energía pura y del puro espíritu, no será tan difícil percibir que sus fragmentos puedan ser ambas cosas. Es un hecho que los Ajustadores atraviesan el espacio por los circuitos de gravedad instantáneos y universales de la Isla del Paraíso.

\par
%\textsuperscript{(1182.7)}
\textsuperscript{107:6.5} El hecho de que los Monitores de Misterio estén así asociados con los circuitos materiales del universo de universos es en verdad un enigma. Pero sigue siendo un hecho que atraviesan como un relámpago todo el gran universo por los circuitos de la gravedad material. Es perfectamente posible que puedan incluso penetrar en los niveles del espacio exterior; seguramente podrían seguir la presencia gravitatoria del Paraíso en estas regiones, y aunque mi orden de personalidades puede atravesar también los circuitos mentales del Actor Conjunto más allá de los confines del gran universo, nunca hemos estado seguros de detectar la presencia de los Ajustadores en las regiones inexploradas del espacio exterior.

\par
%\textsuperscript{(1183.1)}
\textsuperscript{107:6.6} Y sin embargo, aunque los Ajustadores utilizan los circuitos de la gravedad material, no están sujetos a ella como lo está la creación material. Los Ajustadores son fragmentos del predecesor de la gravedad, no consecuencias de la gravedad; se han segmentado en un nivel universal de existencia que es hipotéticamente anterior a la aparición de la gravedad.

\par
%\textsuperscript{(1183.2)}
\textsuperscript{107:6.7} Los Ajustadores del Pensamiento no disfrutan de ningún descanso desde el momento de su donación hasta el día en que son liberados y pueden partir hacia Divinington después de la muerte natural de su sujeto mortal. Y aquellos Ajustadores cuyos sujetos no pasan por las puertas de la muerte natural, ni siquiera experimentan este respiro temporal. Los Ajustadores del Pensamiento no necesitan consumir energía; ellos son energía, una energía del tipo más elevado y más divino.

\section*{7. Los Ajustadores y la personalidad}
\par
%\textsuperscript{(1183.3)}
\textsuperscript{107:7.1} Los Ajustadores del Pensamiento no son personalidades, pero son entidades reales; están verdadera y perfectamente individualizados, aunque nunca están realmente personalizados mientras residen en los mortales. Los Ajustadores del Pensamiento no son verdaderas personalidades; son \textit{verdaderas realidades}, unas realidades del tipo más puro que se conoce en el universo de universos ---son la presencia divina. Aunque no son personales, a estos maravillosos fragmentos del Padre se les califica generalmente de seres, y a veces, en vista de las fases espirituales de su presente ministerio hacia los mortales, de entidades espirituales.

\par
%\textsuperscript{(1183.4)}
\textsuperscript{107:7.2} Si los Ajustadores del Pensamiento no son unas personalidades que posean las prerrogativas de la voluntad y de los poderes de elección, ¿cómo pueden entonces elegir a sus sujetos mortales y ofrecerse para residir en estas criaturas de los mundos evolutivos? Es una pregunta fácil de hacer, pero probablemente ningún ser en el universo de universos ha encontrado nunca la respuesta precisa. Incluso mi orden de personalidades, los Mensajeros Solitarios, no comprende plenamente la dotación de voluntad, de elección y de amor en unas entidades que no son personales.

\par
%\textsuperscript{(1183.5)}
\textsuperscript{107:7.3} A menudo hemos especulado que los Ajustadores del Pensamiento deben tener una volición en todos los niveles \textit{prepersonales} de elección. Se ofrecen como voluntarios para habitar en los seres humanos, hacen planes para la carrera eterna del hombre, los adaptan, modifican y sustituyen de acuerdo con las circunstancias, y estas actividades implican una volición auténtica. Sienten afecto por los mortales, desempeñan su actividad en las crisis del universo, siempre están preparados para actuar de manera decisiva de acuerdo con la elección humana, y todas estas reacciones son extremadamente volitivas. En todas las situaciones no relacionadas con el ámbito de la voluntad humana, manifiestan indiscutiblemente una conducta que revela el ejercicio de unos poderes que equivalen en todos los sentidos a la voluntad, al máximo de decisión.

\par
%\textsuperscript{(1183.6)}
\textsuperscript{107:7.4} Si los Ajustadores del Pensamiento poseen una volición, ¿por qué están sometidos entonces a la voluntad de los mortales? Creemos que esto se debe a que la naturaleza de la volición del Ajustador es absoluta, pero su manifestación es prepersonal. La voluntad humana ejerce su actividad en el nivel de personalidad de la realidad universal y, en todo el cosmos, lo impersonal ---lo no personal, lo subpersonal y lo prepersonal--- siempre es sensible a la voluntad y a los actos de la personalidad existente.

\par
%\textsuperscript{(1183.7)}
\textsuperscript{107:7.5} En todo el universo de los seres creados y de las energías no personales, no observamos que la voluntad, la volición, la elección y el amor se manifiesten con independencia de la personalidad. No vemos que estos atributos de la personalidad funcionen en asociación con las realidades impersonales, salvo en los Ajustadores y en otras entidades similares. No sería correcto indicar que un Ajustador es subpersonal, ni tampoco sería apropiado aludir a esta entidad como superpersonal, pero sería totalmente lícito calificar a este ser de prepersonal.

\par
%\textsuperscript{(1184.1)}
\textsuperscript{107:7.6} Para nuestras órdenes de seres, estos fragmentos de la Deidad son conocidos como dones divinos. Reconocemos que los Ajustadores tienen un origen divino, y que constituyen la prueba y la demostración probables de que el Padre Universal se ha reservado la posibilidad de comunicarse de manera directa e ilimitada con todas y cada una de las criaturas materiales de todos sus reinos prácticamente infinitos, y todo esto independientemente por completo de su presencia en las personalidades de sus Hijos Paradisiacos o de su ministerio indirecto a través de las personalidades del Espíritu Infinito.

\par
%\textsuperscript{(1184.2)}
\textsuperscript{107:7.7} No existen seres creados que no estén encantados de ser los anfitriones de los Monitores de Misterio, pero ninguna orden de seres está así habitada, salvo las criaturas volitivas y evolutivas con un destino finalitario.

\par
%\textsuperscript{(1184.3)}
\textsuperscript{107:7.8} [Presentado por un Mensajero Solitario de Orvonton.]


\chapter{Documento 108. La misión y el ministerio de los Ajustadores del Pensamiento}
\par
%\textsuperscript{(1185.1)}
\textsuperscript{108:0.1} LA MISIÓN de los Ajustadores del Pensamiento a favor de las razas humanas consiste en representar, en ser, el Padre Universal para las criaturas mortales del tiempo y del espacio; éste es el trabajo fundamental de los dones divinos. Su misión consiste también en elevar la mente mortal y en trasladar el alma inmortal de los hombres a las alturas divinas y a los niveles espirituales de la perfección del Paraíso. En la experiencia de transformar así la naturaleza humana de las criaturas temporales en la naturaleza divina de los finalitarios eternos, los Ajustadores dan nacimiento a un tipo único de seres, a unos seres compuestos por la unión eterna entre el Ajustador perfecto y la criatura perfeccionada, que sería imposible de reproducir por medio de cualquier otra técnica del universo.

\par
%\textsuperscript{(1185.2)}
\textsuperscript{108:0.2} En todo el universo no hay nada que pueda sustituir el hecho de la experiencia en los niveles no existenciales. El Dios infinito está, como siempre, repleto y completo, e incluye infinitamente a todas las cosas, excepto el mal y la experiencia de las criaturas. Dios no puede hacer el mal; es infalible. Dios no puede conocer experiencialmente lo que no ha experimentado nunca personalmente. El preconocimiento de Dios es existencial. Por eso el espíritu del Padre desciende del Paraíso para participar con los mortales finitos en cada experiencia de buena fe de la carrera ascendente; únicamente mediante este método es como el Dios existencial podía convertirse, en verdad y de hecho, en el Padre experiencial del hombre. La infinidad del Dios eterno abarca el potencial para la experiencia finita, el cual se vuelve real en verdad en el ministerio de los fragmentos Ajustadores, que comparten realmente las experiencias de las vicisitudes de la vida de los seres humanos.

\section*{1. Selección y asignación}
\par
%\textsuperscript{(1185.3)}
\textsuperscript{108:1.1} Cuando los Ajustadores son enviados desde Divinington para servir a los mortales, su dotación de divinidad existencial es idéntica, pero sus cualidades experienciales varían en proporción a sus contactos anteriores con las criaturas evolutivas y en ellas. No podemos explicar en qué se basan para asignar a los Ajustadores, pero suponemos que estos dones divinos son otorgados de acuerdo con algún tipo de política sabia y eficaz relacionada con la capacidad eterna de adaptación a la personalidad en la que residirán. Observamos que los Ajustadores más experimentados residen con frecuencia en los tipos de mentes humanas más elevados; la herencia humana debe ser por lo tanto un factor importante para determinar la selección y la asignación de los Ajustadores.

\par
%\textsuperscript{(1185.4)}
\textsuperscript{108:1.2} Aunque no lo sabemos con seguridad, creemos firmemente que todos los Ajustadores del Pensamiento son voluntarios. Pero antes de ofrecerse como voluntarios, poseen todos los datos relacionados con el candidato en el que residirán. Los bocetos seráficos sobre la ascendencia del candidato y los modelos proyectados sobre su conducta en la vida son trasmitidos, pasando por el Paraíso, hasta el cuerpo de reserva de los Ajustadores en Divinington mediante la técnica de la reflectividad, la cual se extiende hacia el interior desde las capitales de los universos locales hasta las sedes de los superuniversos. Este pronóstico abarca no solamente los antecedentes hereditarios del candidato mortal, sino también la estimación de sus dotes intelectuales y de su capacidad espiritual probables. Los Ajustadores se ofrecen así como voluntarios para residir en unas mentes cuyas naturalezas íntimas conocen por completo.

\par
%\textsuperscript{(1186.1)}
\textsuperscript{108:1.3} El Ajustador voluntario está interesado particularmente en tres aptitudes del candidato humano:

\par
%\textsuperscript{(1186.2)}
\textsuperscript{108:1.4} 1. \textit{La capacidad intelectual}. ¿La mente es normal? ¿Cuál es el potencial intelectual, la capacidad de la inteligencia? ¿Podrá convertirse el individuo en una criatura volitiva de buena fe? ¿Tendrá la sabiduría la posibilidad de manifestarse?

\par
%\textsuperscript{(1186.3)}
\textsuperscript{108:1.5} 2. \textit{La percepción espiritual}. Las perspectivas de desarrollo de la veneración, el nacimiento y el crecimiento de la naturaleza religiosa. ¿Cuál es el potencial del alma, su capacidad de receptividad espiritual probable?

\par
%\textsuperscript{(1186.4)}
\textsuperscript{108:1.6} 3. \textit{Los poderes intelectuales y espirituales combinados}. El grado en que estas dos dotaciones quizás puedan asociarse, combinarse, como para producir un fuerte carácter humano y contribuir a la evolución segura de un alma inmortal con valor de supervivencia.

\par
%\textsuperscript{(1186.5)}
\textsuperscript{108:1.7} Creemos que, con estos hechos ante ellos, los Monitores se ofrecen libremente como voluntarios para la misión. Existe probablemente más de un Ajustador que ofrece sus servicios; quizás las órdenes personalizadas supervisoras escogen, en este grupo de Ajustadores voluntarios, al más indicado para la tarea de espiritualizar y eternizar la personalidad del candidato mortal. (Para la asignación y el servicio de los Ajustadores, el sexo de la criatura no se tiene en cuenta.)

\par
%\textsuperscript{(1186.6)}
\textsuperscript{108:1.8} El corto período de tiempo que transcurre entre su ofrecimiento como voluntario y el envío real del Ajustador se emplea probablemente en las escuelas de los Monitores Personalizados en Divinington, donde un modelo de trabajo de la mente mortal en espera se utiliza para enseñar al Ajustador asignado los planes más eficaces que puede utilizar para abordar la personalidad y espiritualizar la mente. Este modelo de mente se puede formular gracias a una combinación de datos suministrados por el servicio de reflectividad del superuniverso. Esto es al menos lo que comprendemos, y tenemos esta creencia debido a que los Mensajeros Solitarios, en el transcurso de su larga carrera universal, han reunido toda esta información por medio de sus contactos con muchos Ajustadores Personalizados.

\par
%\textsuperscript{(1186.7)}
\textsuperscript{108:1.9} Una vez que los Ajustadores son enviados efectivamente desde Divinington, no transcurre prácticamente ningún tiempo entre ese momento y el de su aparición en la mente de sus sujetos escogidos. La duración media del tránsito de un Ajustador entre Divinington y Urantia es de 117 horas, 42 minutos y 7 segundos. Todo este tiempo se emplea prácticamente en el registro en Uversa.

\section*{2. Condiciones previas para que residan los Ajustadores}
\par
%\textsuperscript{(1186.8)}
\textsuperscript{108:2.1} Aunque los Ajustadores se ofrecen como voluntarios para el servicio tan pronto como los pronósticos sobre una personalidad han sido transmitidos a Divinington, no son asignados realmente hasta que el sujeto humano ha efectuado su primera decisión moral como personalidad. La primera elección moral de un niño humano es indicada de manera automática en el séptimo ayudante de la mente y se registra instantáneamente, a través del Espíritu Creativo del universo local, en el circuito universal de la gravedad mental del Actor Conjunto y en presencia del Espíritu Maestro que posee la jurisdicción sobre el superuniverso interesado, quien envía inmediatamente esta información a Divinington. Por término medio, los Ajustadores llegan a sus sujetos humanos en Urantia justo antes de que cumplan los seis años. En la presente generación están llegando a los cinco años, diez meses y cuatro días, es decir, a los 2.134 días de la vida terrestre del niño.

\par
%\textsuperscript{(1187.1)}
\textsuperscript{108:2.2} Los Ajustadores no pueden invadir la mente mortal hasta que ésta no ha sido debidamente preparada por el ministerio interior de los espíritus ayudantes de la mente, e incorporada en el circuito del Espíritu Santo. El funcionamiento coordinado de los siete ayudantes es necesario para capacitar así a la mente humana a fin de recibir un Ajustador. La mente de la criatura debe manifestar la tendencia a la adoración e indicar el funcionamiento de la sabiduría, mostrando su aptitud para escoger entre los valores emergentes del bien y el mal ---la elección moral.

\par
%\textsuperscript{(1187.2)}
\textsuperscript{108:2.3} Así es como el escenario de la mente humana está preparado para recibir a los Ajustadores, pero por regla general, éstos no aparecen inmediatamente para residir en dichas mentes, salvo en aquellos mundos donde el Espíritu de la Verdad ejerce su función como coordinador espiritual de estos diferentes ministerios espirituales. Si este espíritu de los Hijos donadores está presente, los Ajustadores llegan infaliblemente en el momento en que el séptimo espíritu ayudante de la mente empieza a funcionar y señala al Espíritu Madre del Universo que ha logrado coordinar en potencia a los seis ayudantes asociados que ejercían anteriormente su ministerio en este intelecto mortal. Por lo tanto, desde el día de Pentecostés, los Ajustadores divinos han sido otorgados universalmente en Urantia a todas las mentes normales que poseen una condición moral\footnote{\textit{Ajustadores después de Pentecostés}: Jl 2:28; Hch 1:8; 2:1-21.}.

\par
%\textsuperscript{(1187.3)}
\textsuperscript{108:2.4} Incluso en una mente dotada del Espíritu de la Verdad, el Ajustador no puede invadir arbitrariamente el intelecto mortal antes de la aparición de una decisión moral. Pero cuando se ha efectuado esta decisión moral, este asistente espiritual asume su jurisdicción directamente desde Divinington. No existen intermediarios ni otras autoridades o poderes intermedios que actúen entre los Ajustadores divinos y sus sujetos humanos; Dios y el hombre están relacionados directamente.

\par
%\textsuperscript{(1187.4)}
\textsuperscript{108:2.5} Antes de la época en que el Espíritu de la Verdad es derramado sobre los habitantes de un mundo evolutivo, parece ser que la donación de los Ajustadores está determinada por numerosas influencias espirituales y actitudes de la personalidad. No comprendemos plenamente las leyes que gobiernan estas donaciones; no entendemos con exactitud qué es lo que determina la salida de los Ajustadores que se han ofrecido como voluntarios para residir en dichas mentes en evolución. Pero sí observamos numerosas influencias y condiciones que parecen estar asociadas con la llegada de los Ajustadores a estas mentes antes de la donación del Espíritu de la Verdad, y son las siguientes:

\par
%\textsuperscript{(1187.5)}
\textsuperscript{108:2.6} 1. La asignación de guardianes seráficos personales. Si un mortal no ha sido previamente habitado por un Ajustador, la asignación de un guardián personal hace que el Ajustador llegue enseguida. Existe una relación muy precisa, pero desconocida, entre el ministerio de los Ajustadores y el ministerio de los guardianes seráficos personales.

\par
%\textsuperscript{(1187.6)}
\textsuperscript{108:2.7} 2. El hecho de alcanzar el tercer círculo de consecución intelectual y de realización espiritual. He observado que los Ajustadores llegan a la mente mortal en el momento de la conquista del tercer círculo, antes incluso de que este logro haya sido señalado a las personalidades del universo local encargadas de estos asuntos.

\par
%\textsuperscript{(1187.7)}
\textsuperscript{108:2.8} 3. En el momento de tomar una decisión suprema de importancia espiritual excepcional. Un comportamiento humano semejante, durante una crisis planetaria en la que se ve implicada la persona, va acompañado generalmente de la llegada inmediata del Ajustador en espera.

\par
%\textsuperscript{(1187.8)}
\textsuperscript{108:2.9} 4. El espíritu de fraternidad. Independientemente de la conquista de los círculos psíquicos y de la asignación de unos guardianes personales ---en ausencia de algo que se parezca a la decisión tomada en una crisis--- cuando un mortal en evolución empieza a estar dominado por el amor a sus semejantes y se consagra a un ministerio desinteresado hacia sus hermanos en la carne, el Ajustador que espera desciende invariablemente para residir en la mente de ese ministro mortal.

\par
%\textsuperscript{(1188.1)}
\textsuperscript{108:2.10} 5. La declaración de la intención de hacer la voluntad de Dios. Observamos que muchos mortales de los mundos del espacio pueden estar aparentemente preparados para recibir Ajustadores, y sin embargo los Monitores no aparecen. Continuamos observando a dichas criaturas en su vida diaria, y poco después llegan de manera tranquila y casi inconsciente a la decisión de empezar a intentar hacer la voluntad del Padre que está en los cielos. Entonces observamos el envío inmediato de los Ajustadores del Pensamiento.

\par
%\textsuperscript{(1188.2)}
\textsuperscript{108:2.11} 6. La influencia del Ser Supremo. En los mundos donde los Ajustadores no fusionan con las almas evolutivas de los habitantes mortales, observamos que a veces se conceden Ajustadores en respuesta a unas influencias que están totalmente más allá de nuestra comprensión. Suponemos que estas donaciones están determinadas por alguna acción refleja cósmica que tiene su origen en el Ser Supremo. En cuanto a las razones por las cuales estos Ajustadores no pueden fusionar, o no fusionan, con estos tipos particulares de mentes mortales evolutivas, no las sabemos. Estas operaciones nunca nos han sido reveladas.

\section*{3. Organización y administración}
\par
%\textsuperscript{(1188.3)}
\textsuperscript{108:3.1} Por lo que sabemos, los Ajustadores están organizados como una unidad independiente de trabajo en el universo de universos, y están aparentemente bajo la administración directa de Divinington. Son uniformes en los siete superuniversos, y todos los universos locales disfrutan del servicio de unos tipos idénticos de Monitores de Misterio. Sabemos, por lo que hemos observado, que existen numerosas series de Ajustadores que suponen una organización consecutiva que se extiende a través de las razas, por encima de las dispensaciones, y para los mundos, los sistemas y los universos. Sin embargo, es extremadamente difícil seguirle la pista a estos dones divinos, puesto que funcionan de manera intercambiable en todo el gran universo.

\par
%\textsuperscript{(1188.4)}
\textsuperscript{108:3.2} La lista completa de los Ajustadores sólo existe (fuera de Divinington) en las sedes de los siete superuniversos. El número y la orden de cada Ajustador que reside en cada criatura ascendente son indicados por las autoridades del Paraíso a la sede del superuniverso, y desde allí se comunican a la sede del universo local interesado, trasmitiéndose después al planeta particular correspondiente. Pero los archivos del universo local no revelan el número completo de los Ajustadores del Pensamiento; los archivos de Nebadon sólo contienen el número de su asignación al universo local, tal como así ha sido indicado por los representantes de los Ancianos de los Días. El significado real del número completo de un Ajustador sólo se conoce en Divinington.

\par
%\textsuperscript{(1188.5)}
\textsuperscript{108:3.3} A los sujetos humanos se les conoce a menudo por el número de su Ajustador; los mortales no reciben su verdadero nombre universal hasta después de fusionar con el Ajustador, una unión que queda señalada cuando el guardián del destino confiere un nuevo nombre a la nueva criatura\footnote{\textit{Nuevo nombre}: Ap 2:17; 3:12.}.

\par
%\textsuperscript{(1188.6)}
\textsuperscript{108:3.4} Aunque conocemos los archivos de los Ajustadores del Pensamiento en Orvonton, y aunque no tenemos ninguna autoridad en absoluto sobre ellos y ninguna conexión administrativa con ellos, creemos firmemente que existe una conexión administrativa muy estrecha entre los mundos individuales de los universos locales y la morada central de los dones divinos en Divinington. Sabemos que después de la aparición de un Hijo donador del Paraíso en un mundo evolutivo, un Ajustador Personalizado es asignado a ese mundo como supervisor planetario de los Ajustadores.

\par
%\textsuperscript{(1189.1)}
\textsuperscript{108:3.5} Es interesante observar que cuando los inspectores del universo local efectúan el examen de un planeta, siempre se dirigen al jefe planetario de los Ajustadores del Pensamiento, al igual que entregan sus encargos a los jefes de los serafines y a los dirigentes de otras órdenes de seres vinculados a la administración de un mundo en evolución. No hace mucho tiempo, Urantia sufrió una inspección periódica de este tipo por parte de Tabamantia, el supervisor soberano de todos los planetas que experimentan con la vida en el universo de Nebadon. Y los archivos revelan que además de expresar sus amonestaciones y críticas a los diversos jefes de las personalidades superhumanas, también expresó el siguiente reconocimiento al jefe de los Ajustadores, el cual podía hallarse en el planeta, en Salvington, en Uversa o en Divinington, no lo sabemos con seguridad, pero Tabamantia dijo:

\par
%\textsuperscript{(1189.2)}
\textsuperscript{108:3.6} <<Ahora me presento ante vosotros, superiores que estáis muy por encima de mí, como alguien que ha recibido una autoridad temporal sobre la serie de planetas experimentales; y vengo a expresar mi admiración y mi profundo respeto por este grupo magnífico de ministros celestiales, los Monitores de Misterio, que se han ofrecido como voluntarios para servir en esta esfera irregular. Por muy difíciles que sean las crisis, nunca vaciláis. Nunca se ha presentado, ni en los registros de Nebadon ni ante las comisiones de Orvonton, una acusación contra un Ajustador divino. Habéis sido leales a vuestras obligaciones; habéis sido divinamente fieles. Habéis ayudado a rectificar los errores y a compensar los defectos de todos los que trabajan en este confuso planeta. Sois unos seres maravillosos, los guardianes del bien en las almas de este mundo atrasado. Os presento mis respetos aunque estéis aparentemente bajo mi jurisdicción como ministros voluntarios. Me inclino ante vosotros en humilde reconocimiento de vuestro desinterés exquisito, de vuestro ministerio comprensivo y de vuestra devoción imparcial. Merecéis el nombre de servidores divinos de los habitantes mortales de este mundo destrozado por los conflictos, acongojado, y afligido por las enfermedades. ¡Os rindo homenaje! ¡Casi os adoro!>>

\par
%\textsuperscript{(1189.3)}
\textsuperscript{108:3.7} Como consecuencia de numerosos indicios que lo indican, creemos que los Ajustadores están perfectamente organizados, que existe una administración profundamente inteligente y eficaz que dirige a estos dones divinos desde alguna fuente central muy lejana, probablemente Divinington. Sabemos que vienen desde Divinington a los mundos, y vuelven indudablemente allí después de la muerte de sus sujetos.

\par
%\textsuperscript{(1189.4)}
\textsuperscript{108:3.8} Es extremadamente difícil descubrir los mecanismos administrativos de las órdenes superiores de espíritus. Aunque las personalidades de mi orden nos dedicamos a cumplir nuestros deberes específicos, participamos sin duda de manera inconsciente con otros numerosos grupos personales e impersonales, situados por debajo de la Deidad, que actúan de forma unida para poner en correlación el inmenso universo. Sospechamos que servimos así porque somos el único grupo de criaturas personalizadas (aparte de los Ajustadores Personalizados) que es uniformemente consciente de la presencia de numerosas órdenes de entidades prepersonales.

\par
%\textsuperscript{(1189.5)}
\textsuperscript{108:3.9} Somos conscientes de la presencia de los Ajustadores, que son los fragmentos de la Deidad prepersonal de la Fuente-Centro Primera. Sentimos la presencia de los Espíritus Inspirados de la Trinidad, que son las expresiones superpersonales de la Trinidad del Paraíso. También detectamos infaliblemente la presencia espiritual de ciertas órdenes no reveladas que descienden del Hijo Eterno y del Espíritu Infinito. Y no somos totalmente insensibles a otras entidades más que no os han sido reveladas.

\par
%\textsuperscript{(1190.1)}
\textsuperscript{108:3.10} Los Melquisedeks de Nebadon enseñan que los Mensajeros Solitarios son los coordinadores, como personalidades, de estas diversas influencias a medida que se registran en la Deidad en expansión del Ser Supremo evolutivo. Es muy posible que estemos participando en la unificación experiencial de muchos fenómenos inexplicados del tiempo, pero no tenemos conscientemente la certeza de actuar de esta manera.

\section*{4. Relación con otras influencias espirituales}
\par
%\textsuperscript{(1190.2)}
\textsuperscript{108:4.1} Aparte de su posible coordinación con otros fragmentos de la Deidad, los Ajustadores están totalmente solos en su esfera de actividad en la mente de los mortales. Aunque el Padre haya renunciado aparentemente a ejercer todo poder y autoridad personales y directos en todo el gran universo, a pesar de este acto de abnegación a favor de los Creadores Supremos, los hijos de las Deidades del Paraíso, los Monitores de Misterio demuestran elocuentemente el hecho de que el Padre se ha reservado sin duda para sí mismo el derecho indiscutible de estar presente en la mente y el alma de sus criaturas evolutivas, a fin de actuar de tal manera que pueda atraer hacia él a todas las criaturas de la creación, en coordinación con la gravedad espiritual de los Hijos Paradisiacos. Cuando vuestro Hijo donador Paradisiaco estaba todavía en Urantia, dijo: <<Si soy elevado, atraeré a todos los hombres>>\footnote{\textit{Atracción espiritual}: Jer 31:3; Jn 6:44; 12:32.}. Reconocemos y comprendemos este poder de atracción espiritual de los Hijos Paradisiacos y de sus asociadas creativas, pero no comprendemos tan plenamente los métodos del Padre infinitamente sabio cuando ejerce su actividad en, y a través de, estos Monitores de Misterio que viven y trabajan con tanta valentía dentro de la mente humana.

\par
%\textsuperscript{(1190.3)}
\textsuperscript{108:4.2} Aunque estas misteriosas presencias no estén subordinadas, coordinadas ni aparentemente relacionadas con el trabajo del universo de universos, aunque actúen independientemente en la mente de los hijos de los hombres, incitan sin cesar a las criaturas en las que habitan hacia los ideales divinos, atrayéndolas constantemente hacia arriba en dirección a los objetivos y las metas de una vida futura y mejor. Estos Monitores de Misterio ayudan continuamente a establecer el dominio espiritual de Miguel en todo el universo de Nebadon, contribuyendo misteriosamente a estabilizar la soberanía de los Ancianos de los Días en Orvonton. Los Ajustadores \textit{son} la voluntad de Dios, y puesto que los Creadores Supremos, los hijos de Dios, encarnan personalmente también esa misma voluntad, es inevitable que las actividades de los Ajustadores y la soberanía de los gobernantes del universo sean mutuamente interdependientes. Aunque no estén aparentemente conectadas, la presencia del Padre a través de los Ajustadores y la soberanía del Padre a través de Miguel de Nebadon deben ser manifestaciones diferentes de la misma divinidad.

\par
%\textsuperscript{(1190.4)}
\textsuperscript{108:4.3} Los Ajustadores del Pensamiento parecen ir y venir de forma totalmente independiente a cualquier otra presencia espiritual; parecen actuar de acuerdo con unas leyes universales completamente distintas a las que gobiernan y controlan las actividades de todas las demás influencias espirituales. Pero a pesar de esta independencia aparente, las observaciones a largo plazo revelan indiscutiblemente que los Ajustadores ejercen su actividad en la mente humana en perfecto sincronismo y coordinación con todos los demás ministerios espirituales, incluídos los espíritus ayudantes de la mente, el Espíritu Santo, el Espíritu de la Verdad y otras influencias.

\par
%\textsuperscript{(1190.5)}
\textsuperscript{108:4.4} Cuando un mundo es aislado a causa de la rebelión, cuando a un planeta se le corta de todos los circuitos de comunicación con el exterior, como le sucedió a Urantia después del levantamiento de Caligastia, sólo queda, aparte de los mensajeros personales, una sola posibilidad de comunicarse directamente con los planetas o con el universo, y es a través de la conexión con los Ajustadores de las esferas. Suceda lo que suceda en un mundo o en un universo, a los Ajustadores nunca les afecta directamente. El aislamiento de un planeta no afecta de ninguna manera a los Ajustadores ni a su capacidad para comunicarse con cualquier parte del universo local, del superuniverso o del universo central. Ésta es la razón por la que se establece contacto con tanta frecuencia, en los mundos en cuarentena, con los Ajustadores supremos y autónomos del cuerpo de reserva del destino. Se recurre a esta técnica como medio de eludir los obstáculos del aislamiento planetario. El circuito de los arcángeles ha funcionado en Urantia en los últimos años, pero este medio de comunicación está limitado principalmente a las actividades del propio cuerpo de los arcángeles.

\par
%\textsuperscript{(1191.1)}
\textsuperscript{108:4.5} Conocemos muchos fenómenos espirituales que tienen lugar en el vasto universo y que no sabemos cómo comprender plenamente. Todavía no dominamos todo lo que sucede a nuestro alrededor; y creo que una gran parte de este trabajo inescrutable es efectuado por los Mensajeros de Gravedad y por ciertos tipos de Monitores de Misterio. No creo que los Ajustadores se dediquen exclusivamente a rehacer la mente de los mortales. Estoy persuadido de que los Monitores Personalizados y otras órdenes de espíritus prepersonales no revelados representan el contacto directo e inexplicado del Padre Universal con las criaturas de los mundos.

\section*{5. La misión de los Ajustadores}
\par
%\textsuperscript{(1191.2)}
\textsuperscript{108:5.1} Los Ajustadores aceptan un trabajo difícil cuando se ofrecen como voluntarios para residir en unos seres compuestos como los que viven en Urantia. Pero han asumido la tarea de existir en vuestra mente, de recibir allí las recomendaciones de las inteligencias espirituales de los reinos, y luego intentar dictar o traducir estos mensajes espirituales a la mente material; son indispensables para la ascensión al Paraíso.

\par
%\textsuperscript{(1191.3)}
\textsuperscript{108:5.2} Aquello que el Ajustador del Pensamiento no puede utilizar en vuestra vida actual, aquellas verdades que no puede transmitir con éxito al hombre de sus esponsales, las conservará fielmente para utilizarlas en vuestro próximo estado de existencia, al igual que ahora transfiere de círculo en círculo aquellos detalles que no logra registrar en la experiencia de su sujeto humano, debido a la incapacidad o al fracaso de la criatura en ofrecer un grado suficiente de cooperación.

\par
%\textsuperscript{(1191.4)}
\textsuperscript{108:5.3} Podéis contar con una cosa: los Ajustadores nunca perderán nada de lo que ha sido confiado a su cuidado; nunca hemos escuchado que estos ayudantes espirituales hayan fallado. Los ángeles y otros tipos elevados de seres espirituales, sin exceptuar a los tipos de Hijos del universo local, pueden abrazar ocasionalmente el mal, pueden desviarse a veces del camino divino, pero los Ajustadores no titubean jamás. Son absolutamente fiables, y esto es igualmente cierto para cada uno de los siete grupos\footnote{\textit{Sin fallo}: Jn 6:39.}.

\par
%\textsuperscript{(1191.5)}
\textsuperscript{108:5.4} Vuestro Ajustador es el potencial de vuestra nueva y próxima orden de existencia, el don por adelantado de vuestra filiación eterna con Dios. Por medio del consentimiento de vuestra voluntad, y con él, el Ajustador tiene el poder de someter las tendencias de la mente material de la criatura a las acciones transformadoras de las motivaciones y los objetivos del alma morontial emergente.

\par
%\textsuperscript{(1191.6)}
\textsuperscript{108:5.5} Los Monitores de Misterio no son ayudantes del pensamiento; son ajustadores del pensamiento. Trabajan con la mente material a fin de construir, mediante ajuste y espiritualización, una nueva mente para vuestra carrera futura en los nuevos mundos y con un nuevo nombre\footnote{\textit{Nuevo nombre}: Ap 2:17; 3:12.}. Su misión está relacionada principalmente con la vida futura, no con esta vida. Se les llama ayudantes celestiales, no ayudantes terrenales. No están interesados en hacer fácil la carrera mortal; se ocupan más bien de hacer vuestra vida razonablemente difícil y dura a fin de estimular y multiplicar vuestras decisiones. La presencia de un gran Ajustador del Pensamiento no proporciona una vida fácil ni os libera de tener que pensar intensamente, pero este don divino os conferirá una sublime paz mental y una magnífica tranquilidad de espíritu.

\par
%\textsuperscript{(1192.1)}
\textsuperscript{108:5.6} Vuestras emociones pasajeras y siempre cambiantes de alegría y de tristeza son generalmente reacciones puramente humanas y materiales a vuestro estado psíquico interior y a vuestro entorno material exterior. No contéis pues con el Ajustador para recibir consuelos egoístas y comodidades humanas. La tarea del Ajustador consiste en prepararos para la aventura eterna, asegurar vuestra supervivencia. El Monitor de Misterio no tiene la misión de suavizar vuestros sentimientos agitados o de socorrer vuestro orgullo herido; la preparación de vuestra alma para la larga carrera ascendente es lo que retiene la atención y ocupa el tiempo del Ajustador.

\par
%\textsuperscript{(1192.2)}
\textsuperscript{108:5.7} Dudo de ser capaz de explicaros exactamente qué es lo que hacen los Ajustadores en vuestra mente y por vuestra alma. No sé si conozco por completo qué es lo que ocurre realmente en la asociación cósmica entre un Monitor divino y una mente humana. Todo esto es en cierto modo un misterio para nosotros, no en cuanto al plan y la finalidad, sino en cuanto a la manera real de llevarlo a cabo. Ésta es precisamente la razón por la que nos enfrentamos con la dificultad de encontrar un nombre apropiado para estos dones celestiales otorgados a los hombres mortales.

\par
%\textsuperscript{(1192.3)}
\textsuperscript{108:5.8} A los Ajustadores del Pensamiento les gustaría cambiar vuestros sentimientos de temor en convicciones de amor y confianza; pero no pueden hacer estas cosas de manera mecánica y arbitraria; esa es tarea vuestra. Cuando efectuáis aquellas decisiones que os liberan de las cadenas del miedo, suministráis literalmente el punto de apoyo psíquico sobre el que el Ajustador podrá aplicar posteriormente la palanca espiritual de una iluminación elevada y progresiva.

\par
%\textsuperscript{(1192.4)}
\textsuperscript{108:5.9} Cuando se trata de conflictos agudos y bien definidos entre las tendencias superiores e inferiores de las razas, entre lo que \textit{es realmente} bueno o malo (y no simplemente entre aquello que podéis llamar bueno y malo), podéis confiar en que el Ajustador participará siempre de alguna manera clara y activa en dichas experiencias. El hecho de que el compañero humano pueda ser inconsciente de esta actividad del Ajustador no disminuye en lo más mínimo su valor y su realidad.

\par
%\textsuperscript{(1192.5)}
\textsuperscript{108:5.10} Si tenéis un guardián personal del destino y no lográis sobrevivir, ese ángel guardián deberá ser juzgado con objeto de recibir la justificación de la ejecución fiel de su deber. Pero a los Ajustadores del Pensamiento no se les somete así a una investigación cuando sus sujetos no logran sobrevivir. Todos sabemos que un ángel quizás no puede cumplir con perfección su ministerio, pero los Ajustadores del Pensamiento trabajan a la manera de la perfección del Paraíso; su ministerio está caracterizado por una técnica sin defectos que está más allá de la posibilidad de recibir las críticas de cualquier ser fuera de Divinington. Tenéis unos guías perfectos; por consiguiente, la meta de la perfección es ciertamente alcanzable.

\section*{6. Dios en el hombre}
\par
%\textsuperscript{(1192.6)}
\textsuperscript{108:6.1} Es en verdad una maravilla de condescendencia divina que los sublimes y perfectos Ajustadores se ofrezcan para existir efectivamente en la mente de las criaturas materiales, tales como los mortales de Urantia, para consumar realmente una unión probatoria con los seres terrestres de origen animal.

\par
%\textsuperscript{(1193.1)}
\textsuperscript{108:6.2} Cualquiera que sea el estado anterior de los habitantes de un mundo, después de la donación de un Hijo divino y después de la donación del Espíritu de la Verdad a todos los humanos, los Ajustadores acuden en masa a dicho mundo para residir en la mente de todas las criaturas volitivas normales. Después de finalizar la misión de un Hijo donador Paradisiaco, estos Monitores se convierten verdaderamente en el <<reino de los cielos dentro de vosotros>>\footnote{\textit{Reino de los cielos dentro de vosotros}: Lc 17:21; Ro 8:9-11; 1 Co 3:16-17; 1 Co 6:19; 2 Co 6:16; 2 Ti 1:14; 1 Jn 4:12-15; Ap 21:3.}. A través de la donación de los dones divinos, el Padre se acerca tanto como le es posible al mal y al pecado, pues es literalmente cierto que el Ajustador ha de coexistir en la mente mortal en medio mismo de la iniquidad humana. Los pensamientos puramente sórdidos y egoístas atormentan particularmente a los Ajustadores interiores; se sienten afligidos por la falta de respeto hacia aquello que es hermoso y divino, y casi frustrados en su trabajo debido a los muchos e insensatos miedos animales y ansiedades infantiles del hombre.

\par
%\textsuperscript{(1193.2)}
\textsuperscript{108:6.3} Los Monitores de Misterio son indudablemente el don del Padre Universal, el reflejo de la imagen de Dios en el universo. Un gran educador exhortó en otro tiempo a los hombres a que se renovaran en el espíritu de su mente\footnote{\textit{Renovaran en el espíritu de su mente}: Ef 4:23-24.}; a que se convirtieran en hombres nuevos, semejantes a Dios, creados en la rectitud y en la consumación de la verdad. El Ajustador es la marca de la divinidad, la presencia de Dios. La <<imagen de Dios>>\footnote{\textit{Imagen de Dios}: Gn 1:26-27; 9:6.} no se refiere al parecido físico ni a las limitaciones circunscritas de los atributos de la criatura material, sino más bien al regalo de la presencia espiritual del Padre Universal en la donación celestial de los Ajustadores del Pensamiento a las humildes criaturas de los universos.

\par
%\textsuperscript{(1193.3)}
\textsuperscript{108:6.4} El Ajustador es la fuente, dentro de vosotros, del logro espiritual y la esperanza de adquirir un carácter divino. Es el poder, el privilegio y la posibilidad de la supervivencia, que os distingue por completo y para siempre de las criaturas simplemente animales. Es el estímulo espiritual del pensamiento, verdaderamente interno y superior, en contraste con los estímulos físicos y externos que llegan hasta la mente a través del mecanismo de la energía nerviosa del cuerpo material.

\par
%\textsuperscript{(1193.4)}
\textsuperscript{108:6.5} Estos fieles guardianes de la carrera futura hacen infaliblemente una copia de cada creación mental en un duplicado espiritual; así os van recreando de manera lenta y segura tal como sois realmente (sólo en espíritu) para la resurrección en los mundos de supervivencia\footnote{\textit{Metamorfosis}: 1 Co 15:42-54.}. Todas estas exquisitas recreaciones espirituales se conservan en la realidad emergente de vuestra alma evolutiva e inmortal, de vuestro yo morontial. Estas realidades están efectivamente ahí, a pesar de que el Ajustador raras veces puede ensalzar lo suficiente estas creaciones duplicadas como para mostrarlas a la luz de la conciencia.

\par
%\textsuperscript{(1193.5)}
\textsuperscript{108:6.6} Al igual que vosotros sois los padres humanos, el Ajustador es el padre divino de vuestro verdadero yo, vuestro yo superior y progresivo, vuestro mejor yo morontial y vuestro futuro yo espiritual. Este alma morontial evolutiva es la que disciernen los jueces y los censores cuando decretan vuestra supervivencia y os elevan a los nuevos mundos y a una existencia sin fin en unión eterna con vuestro fiel asociado ---Dios, el Ajustador.

\par
%\textsuperscript{(1193.6)}
\textsuperscript{108:6.7} Los Ajustadores son los progenitores eternos, los originales divinos, de vuestra alma inmortal en evolución; son el impulso incesante que conduce al hombre a intentar dominar la existencia material actual a la luz de la futura carrera espiritual. Los Monitores son los prisioneros de una esperanza imperecedera, las fuentes de una progresión perpetua. ¡Y cuánto disfrutan comunicándose con sus sujetos a través de unos canales más o menos directos! ¡Cuánto se regocijan cuando pueden prescindir de los símbolos y de otros métodos indirectos, y transmitir sus mensajes directamente al intelecto de sus asociados humanos!

\par
%\textsuperscript{(1194.1)}
\textsuperscript{108:6.8} Vosotros, los humanos, habéis empezado el despliegue interminable de un panorama casi infinito, una expansión ilimitada en unas esferas de oportunidades sin fin en constante aumento, donde llevar a cabo un servicio estimulante, aventuras incomparables, incertidumbres sublimes y logros sin límites. Cuando las nubes se acumulan sobre vuestras cabezas, vuestra fe debería aceptar el hecho de la presencia del Ajustador interior, y así deberíais ser capaces de mirar más allá de las brumas de las incertidumbres mortales, hacia el claro resplandor del sol de la rectitud eterna que ilumina las alturas atrayentes de los mundos de las mansiones de Satania.

\par
%\textsuperscript{(1194.2)}
\textsuperscript{108:6.9} [Presentado por un Mensajero Solitario de Orvonton.]


\chapter{Documento 109. Relación de los Ajustadores con las criaturas del universo}
\par
%\textsuperscript{(1195.1)}
\textsuperscript{109:0.1} LOS Ajustadores del Pensamiento son los hijos de la carrera universal, y en verdad, los Ajustadores vírgenes deben adquirir experiencia mientras las criaturas mortales crecen y se desarrollan. Al igual que la personalidad del niño humano se desarrolla para las luchas de la existencia evolutiva, el Ajustador crece durante los ensayos que efectúa para la próxima etapa de la vida ascendente. Así como el niño adquiere una flexibilidad de adaptación para sus actividades como adulto a través de la vida social y de juego de su primera infancia, el Ajustador interior adquiere destreza para la siguiente etapa de la vida cósmica mediante la planificación y el ensayo preparatorios, con los mortales, de aquellas actividades que están relacionadas con la carrera morontial. La existencia humana constituye un período de prácticas que el Ajustador utiliza eficazmente como preparación para las responsabilidades crecientes y las oportunidades más importantes de una vida futura. Pero los esfuerzos del Ajustador, mientras vive dentro de vosotros, no están muy relacionados con los asuntos de la vida temporal y de la existencia planetaria. Por decirlo así, los Ajustadores del Pensamiento están ensayando hoy las realidades de la carrera universal en la mente evolutiva de los seres humanos.

\section*{1. Desarrollo de los Ajustadores}
\par
%\textsuperscript{(1195.2)}
\textsuperscript{109:1.1} Debe existir un plan detallado y de gran amplitud para preparar y desarrollar a los Ajustadores vírgenes antes de ser enviados desde Divinington, pero en realidad no sabemos gran cosa sobre este asunto. Existe también indudablemente un amplio sistema para volver a entrenar a los Ajustadores que han tenido la experiencia de residir en un mortal, antes de embarcarse en una nueva misión para asociarse con otro mortal, pero de nuevo no lo sabemos realmente.

\par
%\textsuperscript{(1195.3)}
\textsuperscript{109:1.2} Los Ajustadores Personalizados me han indicado que cada vez que un mortal habitado por un Monitor no logra sobrevivir, el Ajustador es sometido a un amplio curso de entrenamiento cuando regresa a Divinington. Esta formación adicional resulta posible debido a la experiencia de haber residido en un ser humano, y siempre se imparte antes de enviar de nuevo al Ajustador a los mundos evolutivos del tiempo.

\par
%\textsuperscript{(1195.4)}
\textsuperscript{109:1.3} La experiencia viviente real no tiene ningún sustituto cósmico. La perfección de la divinidad de un Ajustador del Pensamiento recién formado no dota de ninguna manera a ese Monitor de Misterio de la capacidad para llevar a cabo un experto ministerio. La experiencia es inseparable de la existencia viviente; es la única cosa que ninguna cantidad de dotación divina puede dispensaros de la necesidad de conseguir mediante la \textit{vida real}. Por consiguiente, al igual que todos los seres que viven y ejercen su actividad dentro del ámbito actual del Supremo, los Ajustadores del Pensamiento deben adquirir experiencia; deben evolucionar desde los grupos inferiores e inexpertos hasta los grupos superiores y más experimentados.

\par
%\textsuperscript{(1196.1)}
\textsuperscript{109:1.4} Los Ajustadores pasan por una carrera concreta de desarrollo en la mente mortal; alcanzan una realidad de consecución que les pertenece de manera eterna. Adquieren progresivamente su capacidad y su destreza como Ajustadores a consecuencia de cada uno y de todos sus contactos con las razas materiales, independientemente de la supervivencia o no de sus sujetos mortales particulares. También están asociados en términos de igualdad con la mente humana para fomentar la evolución del alma inmortal con capacidad de supervivencia.

\par
%\textsuperscript{(1196.2)}
\textsuperscript{109:1.5} El Ajustador alcanza su primer grado de evolución cuando fusiona con el alma sobreviviente de un ser mortal. Así, mientras vosotros evolucionáis por naturaleza hacia dentro y hacia arriba, desde el hombre hasta Dios, los Ajustadores evolucionan por naturaleza hacia fuera y hacia abajo, desde Dios hasta el hombre; y así, el producto final de esta unión de la divinidad y de la humanidad será eternamente el hijo del hombre y el hijo de Dios.

\section*{2. Los Ajustadores autónomos}
\par
%\textsuperscript{(1196.3)}
\textsuperscript{109:2.1} Habéis sido informados sobre la clasificación de los Ajustadores según su experiencia ---vírgenes, avanzados y supremos. Deberíais reconocer también cierta clasificación funcional--- los Ajustadores autónomos. Un Ajustador autónomo es aquel que:

\par
%\textsuperscript{(1196.4)}
\textsuperscript{109:2.2} 1. Ha tenido cierta experiencia necesaria en la vida evolutiva de una criatura volitiva, ya sea como habitante temporal en un tipo de mundo donde los Ajustadores sólo son prestados a los sujetos mortales, o en un planeta donde se fusiona realmente, pero cuyo ser humano no ha logrado sobrevivir. Ese Monitor es un Ajustador avanzado o un Ajustador supremo.

\par
%\textsuperscript{(1196.5)}
\textsuperscript{109:2.3} 2. Ha adquirido el equilibrio del poder espiritual en un humano que ha alcanzado el tercer círculo psíquico y al cual se le ha asignado un guardián seráfico personal.

\par
%\textsuperscript{(1196.6)}
\textsuperscript{109:2.4} 3. Tiene un sujeto que ha tomado la decisión suprema, que ha contraído un compromiso sincero y solemne con el Ajustador. El Ajustador contempla por adelantado el momento de la fusión real y considera la unión como un hecho.

\par
%\textsuperscript{(1196.7)}
\textsuperscript{109:2.5} 4. Tiene un sujeto que ha sido enrolado en uno de los cuerpos de reserva del destino, en un mundo evolutivo de ascensión humana.

\par
%\textsuperscript{(1196.8)}
\textsuperscript{109:2.6} 5. En un momento dado, durante el sueño humano, se ha separado temporalmente de la mente del mortal donde estaba encarcelado, para llevar a cabo alguna proeza de conexión, contacto, reinscripción u otro servicio extrahumano relacionado con la administración espiritual del mundo donde está destinado.

\par
%\textsuperscript{(1196.9)}
\textsuperscript{109:2.7} 6. Ha servido, durante un período de crisis, en la experiencia de algún ser humano que era el complemento material de una personalidad espiritual encargada de realizar alguna proeza cósmica esencial para la economía espiritual del planeta.

\par
%\textsuperscript{(1196.10)}
\textsuperscript{109:2.8} Los Ajustadores autónomos parecen poseer un notable grado de voluntad en todos los asuntos que no conciernen a las personalidades humanas en las que habitan directamente, tal como lo indican sus numerosas proezas tanto dentro como fuera de los sujetos mortales a los que están vinculados. Estos Ajustadores participan en numerosas actividades del planeta, pero actúan con más frecuencia como habitantes desapercibidos de los tabernáculos terrestres que ellos mismos han elegido.

\par
%\textsuperscript{(1196.11)}
\textsuperscript{109:2.9} Estos tipos de Ajustadores más elevados y más experimentados pueden comunicarse indudablemente con aquellos que se encuentran en otros mundos. Pero aunque los Ajustadores autónomos se comunican así entre ellos, sólo lo hacen en los niveles de su trabajo mutuo y con la finalidad de conservar los datos entregados a su custodia, esenciales para que los Ajustadores efectúen su ministerio en los mundos donde residen, aunque se sabe que en ciertas ocasiones han actuado en asuntos interplanetarios durante las épocas de crisis.

\par
%\textsuperscript{(1197.1)}
\textsuperscript{109:2.10} Los Ajustadores supremos y autónomos pueden dejar el cuerpo humano a voluntad. Estos habitantes no son una parte orgánica o biológica de la vida mortal; están superpuestos divinamente a la vida. Los Ajustadores estaban previstos en los planes originales de vida, pero no son indispensables para la existencia material. Sin embargo, debemos indicar que muy raras veces dejan, ni siquiera temporalmente, sus tabernáculos mortales una vez que han establecido allí su residencia.

\par
%\textsuperscript{(1197.2)}
\textsuperscript{109:2.11} Los Ajustadores que actúan de manera superior son aquellos que han ejecutado triunfalmente las tareas que les fueron encomendadas, y sólo esperan la disolución del vehículo de la vida material o el traslado del alma inmortal.

\section*{3. Relación de los Ajustadores con los tipos de mortales}
\par
%\textsuperscript{(1197.3)}
\textsuperscript{109:3.1} Las características del trabajo detallado de los Monitores de Misterio varían de acuerdo con la naturaleza de su misión, según sean Ajustadores de \textit{enlace} o Ajustadores de \textit{fusión}. Algunos Ajustadores son simplemente prestados durante la vida temporal de sus sujetos; otros son otorgados como candidatos a la personalidad, con el permiso de fusionar perpetuamente si sus sujetos sobreviven. Su trabajo comporta también una ligera variación entre los distintos tipos planetarios así como en los diferentes sistemas y universos. Pero su labor es en general extraordinariamente uniforme, más uniforme que los deberes de cualquier otra orden creada de seres celestiales.

\par
%\textsuperscript{(1197.4)}
\textsuperscript{109:3.2} En ciertos mundos primitivos (el grupo de la primera serie), el Ajustador reside en la mente de la criatura como entrenamiento experiencial, principalmente para cultivarse y desarrollarse progresivamente. Los Ajustadores vírgenes se envían habitualmente a esos mundos durante los períodos iniciales en que los hombres primitivos llegan al valle de las decisiones, pero cuando relativamente pocos de ellos escogen ascender a las alturas morales que sobrepasan las colinas del dominio de sí mismo y de la adquisición del carácter, para alcanzar los niveles superiores de la espiritualidad emergente. (Sin embargo, muchos humanos que no logran fusionar con su Ajustador sobreviven como ascendentes fusionados con el Espíritu). Los Ajustadores reciben un entrenamiento valioso y adquieren una experiencia maravillosa durante su asociación transitoria con las mentes primitivas, y posteriormente son capaces de utilizar esta experiencia en beneficio de los seres superiores de otros mundos. \textit{En todo el extenso universo,nunca se pierde nada que tenga un valor de supervivencia}.

\par
%\textsuperscript{(1197.5)}
\textsuperscript{109:3.3} En otro tipo de mundos (el grupo de la segunda serie), los Ajustadores son simplemente prestados a los seres mortales. Aquí, los Monitores nunca pueden alcanzar la personalidad por medio de la fusión residiendo así en estos mortales, pero sí proporcionan una gran ayuda a sus sujetos humanos durante la vida mortal, mucho más de la que son capaces de dar a los mortales de Urantia. Los Ajustadores son prestados aquí a las criaturas mortales durante una sola vida como modelos para sus logros espirituales superiores, unos ayudantes temporales en la tarea fascinante de perfeccionar un carácter de supervivencia. Los Ajustadores no regresan después de la muerte natural; estos mortales sobrevivientes alcanzan la vida eterna mediante la fusión con el Espíritu.

\par
%\textsuperscript{(1197.6)}
\textsuperscript{109:3.4} En los mundos tales como Urantia (el grupo de la tercera serie), existen unos verdaderos esponsales con los dones divinos, un compromiso para la vida y la muerte. Si sobrevivís, se producirá una unión eterna, una fusión perpetua, la transformación del hombre y del Ajustador en un solo ser.

\par
%\textsuperscript{(1197.7)}
\textsuperscript{109:3.5} En los mortales tricerebrales de esta serie de mundos, los Ajustadores son capaces de establecer un contacto mucho más real con sus sujetos durante la vida temporal que en los tipos con uno o dos cerebros. Pero después de la muerte, el tipo tricerebral continúa su carrera exactamente igual que el tipo con un cerebro y los pueblos con dos cerebros ---las razas de Urantia.

\par
%\textsuperscript{(1198.1)}
\textsuperscript{109:3.6} En los mundos donde los humanos tienen dos cerebros, y después de la estancia de un Hijo donador Paradisiaco, los Ajustadores vírgenes son asignados raramente a las personas que tienen una capacidad indiscutible para sobrevivir. Creemos que en esos mundos, prácticamente todos los Ajustadores que residen en los hombres y las mujeres inteligentes con capacidad de supervivencia pertenecen al tipo avanzado o al tipo supremo.

\par
%\textsuperscript{(1198.2)}
\textsuperscript{109:3.7} En muchas razas evolutivas primitivas de Urantia había tres grupos de seres. Existían aquellos que estaban tan animalizados que carecían por completo de la capacidad de recibir un Ajustador. Estaban aquellos que mostraban una capacidad indudable para recibir a los Ajustadores, y los recibían de inmediato en cuanto alcanzaban la edad de la responsabilidad moral. Había una tercera clase que ocupaba una posición fronteriza; tenían capacidad para recibir un Ajustador, pero los Monitores sólo podían residir en sus mentes a petición personal de cada individuo.

\par
%\textsuperscript{(1198.3)}
\textsuperscript{109:3.8} Muchos Ajustadores vírgenes han adquirido una valiosa experiencia preliminar poniéndose en contacto con la mente evolutiva de unos seres prácticamente incapacitados para sobrevivir debido a las taras hereditarias de unos antepasados incapaces e inferiores; estos Ajustadores se han vuelto así más competentes para ser asignados posteriormente a unas mentes de tipo superior en algún otro mundo.

\section*{4. Los Ajustadores y la personalidad humana}
\par
%\textsuperscript{(1198.4)}
\textsuperscript{109:4.1} Los Ajustadores interiores facilitan enormemente las formas superiores de intercomunicación inteligente entre los seres humanos. Los animales tienen sentimientos de compañerismo, pero no se comunican conceptos entre sí; pueden expresar emociones, pero no ideas ni ideales. Los hombres de origen animal tampoco experimentan un intercambio intelectual de tipo superior ni una comunión espiritual con sus semejantes hasta que no se les conceden los Ajustadores del Pensamiento; sin embargo, cuando estas criaturas evolutivas desarrollan el habla, están en buen camino para recibir los Ajustadores.

\par
%\textsuperscript{(1198.5)}
\textsuperscript{109:4.2} Los animales se comunican entre sí de manera rudimentaria, pero hay poca o ninguna \textit{personalidad} en estos contactos primitivos. Los Ajustadores no son la personalidad; son seres prepersonales. Pero proceden de la fuente de la personalidad, y su presencia aumenta las manifestaciones cualitativas de la personalidad humana; esto es especialmente cierto si el Ajustador ha tenido una experiencia previa.

\par
%\textsuperscript{(1198.6)}
\textsuperscript{109:4.3} El tipo de Ajustador tiene mucho que ver con el potencial de expresión de la personalidad humana. A lo largo de todas las épocas, muchos grandes dirigentes intelectuales y espirituales de Urantia han ejercido su influencia principalmente debido a la superioridad y a la experiencia previa de sus Ajustadores interiores.

\par
%\textsuperscript{(1198.7)}
\textsuperscript{109:4.4} Los Ajustadores interiores han cooperado en gran medida con otras influencias espirituales para transformar y humanizar a los descendientes de los hombres primitivos de los tiempos antiguos. Si los Ajustadores que residen en la mente de los habitantes de Urantia fueran retirados, el mundo volvería lentamente a muchos actos y prácticas de los hombres de las épocas primitivas; los Monitores divinos son uno de los verdaderos potenciales de la civilización progresiva.

\par
%\textsuperscript{(1198.8)}
\textsuperscript{109:4.5} He observado a un Ajustador del Pensamiento que reside en una mente de Urantia que, según los archivos de Uversa, ha habitado anteriormente en quince mentes de Orvonton. No sabemos si este Monitor ha tenido experiencias similares en otros superuniversos, pero lo supongo. Se trata de un Ajustador maravilloso y es una de las fuerzas más útiles y poderosas que se encuentran en Urantia durante la época actual. Aquello que otros han perdido por haberse negado a sobrevivir, este ser humano (y todo vuestro mundo) lo gana ahora. A aquel que no posee cualidades de supervivencia se le quitará incluso el Ajustador experimentado que posee ahora, mientras que a aquel que tiene posibilidades de supervivencia se le dará incluso el Ajustador con experiencia previa de un desertor indolente\footnote{\textit{A quien tiene se le dará, pero al que no tiene}: Mt 13:12; 25:29; Mc 4:25; Lc 8:18; 19:26.}.

\par
%\textsuperscript{(1199.1)}
\textsuperscript{109:4.6} En cierto sentido, los Ajustadores pueden fomentar cierto grado de fecundación cruzada a nivel planetario en los ámbitos de la verdad, la belleza y la bondad. Pero en pocas ocasiones se les concede la experiencia de residir dos veces en el mismo planeta; ningún Ajustador que sirve actualmente en Urantia ha estado previamente en este mundo. Sé de lo que hablo, pues tenemos sus números y sus datos en los archivos de Uversa.

\section*{5. Obstáculos materiales para la estancia de los Ajustadores}
\par
%\textsuperscript{(1199.2)}
\textsuperscript{109:5.1} A menudo, los Ajustadores supremos y autónomos son capaces de aportar factores de importancia espiritual a la mente humana cuando éstos fluyen libremente en los canales liberados, pero controlados, de la imaginación creativa. En esos momentos, y a veces durante el sueño, el Ajustador puede detener las corrientes mentales, frenar el flujo, y luego desviar la procesión de las ideas; todo esto está destinado a efectuar profundas transformaciones espirituales en las partes recónditas superiores de la superconciencia. Las fuerzas y las energías de la mente están así más plenamente ajustadas a la clave de los tonos de contacto del nivel espiritual del presente y del futuro.

\par
%\textsuperscript{(1199.3)}
\textsuperscript{109:5.2} A veces es posible que se ilumine la mente\footnote{\textit{Iluminación}: 1 Co 2:6-16.}, que se escuche la voz divina que habla continuamente dentro de vosotros, de manera que podéis volveros parcialmente conscientes de la sabiduría, la verdad, la bondad y la belleza de la personalidad potencial que reside constantemente dentro de vosotros.

\par
%\textsuperscript{(1199.4)}
\textsuperscript{109:5.3} Pero vuestras actitudes mentales inestables y rápidamente cambiantes conducen con frecuencia a desbaratar los planes y a interrumpir el trabajo de los Ajustadores. La naturaleza innata de las razas mortales no sólo interfiere su tarea, sino que vuestras propias opiniones preconcebidas, ideas fijas y prejuicios de muchos años retrasan también enormemente este ministerio. Debido a estos obstáculos, muchas veces sus creaciones inacabadas son las únicas que emergen a la conciencia, y la confusión de los conceptos es inevitable. Por consiguiente, al examinar a fondo las situaciones mentales, la seguridad sólo reside en el rápido reconocimiento de cada pensamiento y de cada experiencia justo por lo que real y fundamentalmente es, despreocupándose por completo de lo que podría haber sido.

\par
%\textsuperscript{(1199.5)}
\textsuperscript{109:5.4} El gran problema de la vida consiste en ajustar las tendencias ancestrales de la vida a las exigencias de los impulsos espirituales iniciados por la presencia divina del Monitor de Misterio. Aunque en las carreras del universo y del superuniverso ningún hombre puede servir a dos señores a la vez, en la vida que ahora vivís en Urantia cada hombre debe servir forzosamente a dos señores. Debe volverse experto en el arte de practicar un compromiso humano continuo y temporal, concediendo al mismo tiempo su lealtad espiritual a un solo señor\footnote{\textit{No se puede servir a dos señores}: Mt 6:24; Lc 16:13.}; esta es la razón por la que tantas personas titubean y fracasan, se cansan y sucumben ante la tensión de la lucha evolutiva.

\par
%\textsuperscript{(1199.6)}
\textsuperscript{109:5.5} Aunque el legado hereditario de la dotación cerebral y el del supercontrol electroquímico actúan para delimitar la esfera de actividad eficaz del Ajustador, ninguna desventaja hereditaria impide nunca (en las mentes normales) el logro espiritual final. La herencia puede interferir en la velocidad de conquista de la personalidad, pero no impide la consumación final de la aventura ascendente. Si queréis cooperar con vuestro Ajustador, el don divino hará que tarde o temprano se desarrolle el alma morontial inmortal y, después de fusionar con ella, presentará a la nueva criatura ante el Hijo Maestro soberano del universo local y, a fin de cuentas, ante el Padre de los Ajustadores en el Paraíso.

\section*{6. La permanencia de los verdaderos valores}
\par
%\textsuperscript{(1200.1)}
\textsuperscript{109:6.1} Los Ajustadores no fallan nunca; nunca se pierde nada que sea digno de sobrevivir; todo valor significativo de toda criatura volitiva sobrevivirá con toda seguridad, sin tener en cuenta la supervivencia o no de la personalidad que ha descubierto o evaluado dicho significado. Así pues, una criatura mortal puede rechazar la supervivencia; sin embargo, la experiencia de su vida no se pierde; el Ajustador eterno se lleva las características valiosas de esa vida aparentemente fracasada a algún otro mundo, y esos significados y valores sobrevivientes los confiere allí a un tipo más elevado de mente mortal, a una mente con capacidad para sobrevivir. Ninguna experiencia valiosa sucede nunca en vano; ningún significado verdadero o ningún valor real perece jamás.

\par
%\textsuperscript{(1200.2)}
\textsuperscript{109:6.2} En lo que se refiere a los candidatos a la fusión, si un Monitor de Misterio es abandonado por su asociado mortal, si el compañero humano se niega a continuar la carrera ascendente, cuando el Ajustador es liberado debido a la muerte natural (o antes de ella), se lleva todo lo que posee un valor de supervivencia que haya evolucionado en la mente de esa criatura no sobreviviente. Si un Ajustador no logra conseguir repetidas veces la personalidad por medio de la fusión a causa de la no supervivencia de sus sujetos humanos sucesivos, y si ese Monitor fuera personalizado posteriormente, toda la experiencia adquirida por haber habitado y dominado todas estas mentes mortales pasaría a ser la posesión real de ese Ajustador recién Personalizado, una dotación que disfrutará y podrá utilizar en todas las épocas futuras. Un Ajustador Personalizado de esta orden es un conjunto compuesto de todas las características supervivientes de todas las criaturas anteriores que fueron anfitrionas suyas.

\par
%\textsuperscript{(1200.3)}
\textsuperscript{109:6.3} Cuando los Ajustadores con una larga experiencia universal se ofrecen como voluntarios para habitar en los Hijos divinos en misión de donación, saben muy bien que nunca podrán conseguir la personalidad a través de este servicio. Pero el Padre de los espíritus concede a menudo la personalidad a estos voluntarios y los establece como directores de su misma especie. Éstas son las personalidades honradas con autoridad en Divinington. Sus naturalezas singulares incorporan la humanidad variopinta de sus múltiples experiencias como residentes en los mortales, y también la transcripción espiritual de la divinidad humana del Hijo donador Paradisiaco en el que han residido como experiencia final.

\par
%\textsuperscript{(1200.4)}
\textsuperscript{109:6.4} Las actividades de los Ajustadores en vuestro universo local están dirigidas por el Ajustador Personalizado de Miguel de Nebadon, el mismo Monitor que lo guió paso a paso cuando vivió su vida humana en la carne de Josué ben José. Este Ajustador extraordinario fue fiel a su deber, este valiente Monitor dirigió sabiamente la naturaleza humana del Hijo Paradisiaco, guiando constantemente su mente mortal para que eligiera el camino de la voluntad perfecta del Padre. Este Ajustador había servido anteriormente en Maquiventa Melquisedek en los tiempos de Abraham, y había llevado a cabo proezas extraordinarias tanto antes de esta estancia como entre estas experiencias de donación.

\par
%\textsuperscript{(1200.5)}
\textsuperscript{109:6.5} Este Ajustador triunfó realmente en la mente humana de Jesús ---en aquella mente que, en cada una de las situaciones recurrentes de la vida, mantuvo una dedicación consagrada a la voluntad del Padre\footnote{\textit{Jesús hizo la voluntad de Dios}: Jn 4:34; 5:30; 6:38-40; 15:10; 17:4.}, diciendo: <<Que no se haga mi voluntad, sino la tuya>>\footnote{\textit{Que no se haga mi voluntad, sino la tuya}: Mt 26:39,42,44; Mc 14:36,39; Lc 22:42.}. Esta consagración decisiva constituye el verdadero pasaporte que conduce desde las limitaciones de la naturaleza humana hasta la finalidad donde se alcanza la divinidad.

\par
%\textsuperscript{(1200.6)}
\textsuperscript{109:6.6} Este mismo Ajustador refleja ahora, en la naturaleza inescrutable de su poderosa personalidad, la humanidad anterior al bautismo de Josué ben José, la transcripción eterna y viviente de los valores eternos y vivientes que el más grande de todos los urantianos hizo surgir de las humildes circunstancias de una vida corriente, tal como fue vivida hasta el agotamiento total de los valores espirituales alcanzables en la experiencia de un mortal.

\par
%\textsuperscript{(1201.1)}
\textsuperscript{109:6.7} Todo aquello que tiene un valor permanente y que ha sido confiado a un Ajustador tiene asegurada la supervivencia eterna. En ciertos casos, el Monitor conserva estas posesiones para regalarlas en el futuro a la mente mortal donde residirá; en otros casos, y después de ser personalizadas, estas realidades sobrevivientes y conservadas las guardan en depósito para utilizarlas en el futuro al servicio de los Arquitectos del Universo Maestro.

\section*{7. El destino de los Ajustadores Personalizados}
\par
%\textsuperscript{(1201.2)}
\textsuperscript{109:7.1} No podemos afirmar si los fragmentos no Ajustadores del Padre son personalizables o no, pero se os ha informado que la personalidad es la donación soberana del libre albedrío del Padre Universal. Por lo que sabemos, los fragmentos del Padre del tipo Ajustador sólo consiguen la personalidad adquiriendo los atributos personales a través de un ministerio de servicio hacia un ser personal. Estos Ajustadores Personalizados tienen su hogar en Divinington, donde enseñan y dirigen a sus asociados prepersonales.

\par
%\textsuperscript{(1201.3)}
\textsuperscript{109:7.2} Los Ajustadores del Pensamiento Personalizados, libres de trabas y de destino, son los estabilizadores y compensadores soberanos del inmenso universo de universos. Combinan la experiencia del Creador y de las criaturas ---lo existencial y lo experiencial. Son seres conjuntos del tiempo y de la eternidad. Asocian lo prepersonal y lo personal en la administración del universo.

\par
%\textsuperscript{(1201.4)}
\textsuperscript{109:7.3} Los Ajustadores Personalizados son los poderosos ejecutivos infinitamente sabios de los Arquitectos del Universo Maestro. Son los agentes personales del ministerio completo del Padre Universal ---personal, prepersonal y superpersonal. Son los ministros personales de lo extraordinario, lo inhabitual y lo inesperado en todos los ámbitos de las esferas trascendentales absonitas de los dominios de Dios
Último, e incluso en los niveles de Dios Absoluto.

\par
%\textsuperscript{(1201.5)}
\textsuperscript{109:7.4} Son los únicos seres de los universos que contienen dentro de sí mismos todas las relaciones conocidas de la personalidad; son omnipersonales ---son anteriores a la personalidad, son la personalidad y son posteriores a la personalidad. Al igual que en el pasado eterno, ministran la personalidad del Padre Universal en el presente eterno y en el futuro eterno.

\par
%\textsuperscript{(1201.6)}
\textsuperscript{109:7.5} El Padre concedió al Hijo Eterno una personalidad existencial de un orden infinito y absoluto, pero escogió reservarse para su propio ministerio la personalidad experiencial del tipo de los Ajustadores Personalizados, la cual la otorga a los Ajustadores existenciales prepersonales; las dos están así destinadas a la futura superpersonalidad eterna del ministerio trascendental en los reinos absonitos del
Último, del Supremo-Último, e incluso hasta los niveles del Último-Absoluto.

\par
%\textsuperscript{(1201.7)}
\textsuperscript{109:7.6} En general, a los Ajustadores Personalizados raras veces se les ve en los universos. De vez en cuando consultan con los Ancianos de los Días, y los Ajustadores Personalizados de los Hijos Creadores séptuples vienen a veces a los mundos centrales de las constelaciones para conferenciar con los gobernantes Vorondadeks.

\par
%\textsuperscript{(1201.8)}
\textsuperscript{109:7.7} Cuando el observador Vorondadek planetario de Urantia ---el Altísimo custodio que no hace mucho tiempo asumió urgentemente la regencia de vuestro mundo--- afirmó su autoridad en presencia del gobernador general residente, empezó su administración urgente de Urantia con todo el personal de su propia elección. A todos sus asociados y asistentes les asignó inmediatamente sus deberes planetarios. Pero no eligió a los tres Ajustadores Personalizados que aparecieron ante él en el momento de asumir la regencia. Ni siquiera sabía que aparecerían de esta manera, pues no habían manifestado su presencia divina durante la época de una regencia anterior. El Altísimo regente no asignó ningún servicio ni encargó ningún deber a estos Ajustadores Personalizados voluntarios. Sin embargo, estos tres seres omnipersonales figuraban entre los más activos de las numerosas órdenes de seres celestiales que por entonces servían en Urantia.

\par
%\textsuperscript{(1202.1)}
\textsuperscript{109:7.8} Los Ajustadores Personalizados realizan una amplia gama de servicios para numerosas órdenes de personalidades del universo, pero no nos está permitido hablar de estos ministerios a las criaturas evolutivas habitadas por un Ajustador. Estas extraordinarias divinidades humanas se encuentran entre las personalidades más notables de todo el gran universo, y nadie se atreve a predecir cuáles podrán ser sus misiones futuras.

\par
%\textsuperscript{(1202.2)}
\textsuperscript{109:7.9} [Presentado por un Mensajero Solitario de Orvonton.]


\chapter{Documento 110. Relación de los Ajustadores con los mortales individuales}
\par
%\textsuperscript{(1203.1)}
\textsuperscript{110:0.1} DOTAR de libertad a unos seres imperfectos implica tragedias inevitables, y es propio de la perfecta Deidad ancestral compartir de forma universal y afectuosa esos sufrimientos en amoroso compañerismo\footnote{\textit{Compartir sufrimientos}: Is 63:9.}.

\par
%\textsuperscript{(1203.2)}
\textsuperscript{110:0.2} En la medida en que estoy familiarizado con los asuntos de un universo, el amor y la devoción de un Ajustador del Pensamiento los considero como el afecto más verdaderamente divino de toda la creación. El amor que manifiestan los Hijos en su ministerio hacia las razas es magnífico, pero la devoción de un Ajustador hacia el individuo es conmovedoramente sublime, divinamente semejante a la del Padre. El Padre Paradisiaco se ha reservado aparentemente esta forma de contacto personal con sus criaturas individuales como una prerrogativa exclusiva de Creador. En todo el universo de universos no hay nada exactamente comparable al maravilloso ministerio de estas entidades impersonales que residen de una manera tan fascinante en los hijos de los planetas evolutivos.

\section*{1. La estancia en la mente de los mortales}
\par
%\textsuperscript{(1203.3)}
\textsuperscript{110:1.1} No se debe pensar que los Ajustadores viven en el cerebro material de los seres humanos. No son una parte orgánica de las criaturas físicas de los mundos. Se puede concebir de manera más apropiada que el Ajustador del Pensamiento reside en la mente mortal del hombre, en lugar de existir dentro de los confines de un órgano físico determinado. El Ajustador se comunica constantemente, de forma indirecta y sin ser reconocido, con el sujeto humano, especialmente durante las experiencias sublimes en las que la mente se pone en contacto de adoración con el espíritu en la superconciencia.

\par
%\textsuperscript{(1203.4)}
\textsuperscript{110:1.2} Desearía que me fuera posible ayudar a los mortales evolutivos a conseguir comprender mejor y a alcanzar una apreciación más completa del trabajo desinteresado y magnífico de los Ajustadores que viven dentro de ellos, y que son tan devotamente fieles a la tarea de fomentar el bienestar espiritual del hombre. Estos Monitores aportan su ministerio eficaz a las fases superiores de la mente de los hombres; manipulan con sabiduría y experiencia el potencial espiritual del intelecto humano. Estos ayudantes celestiales están dedicados a la prodigiosa tarea de guiaros con seguridad hacia dentro y hacia arriba hasta el refugio celestial de la felicidad. Estos trabajadores incansables están consagrados a la personificación futura del triunfo de la verdad divina en vuestra vida eterna. Son los obreros vigilantes que pilotan la mente humana consciente de Dios, alejándola de los escollos del mal mientras guían hábilmente el alma evolutiva del hombre hacia los puertos divinos de la perfección en las costas eternas y lejanas. Los Ajustadores son unos conductores amorosos, vuestros guías seguros y dignos de confianza a través de los laberintos oscuros e inciertos de vuestra breve carrera terrestre; son los pacientes educadores que impulsan constantemente a sus sujetos a avanzar por los caminos de la perfección progresiva. Son los guardianes cuidadosos de los valores sublimes del carácter de las criaturas. Desearía que pudierais amarlos más, cooperar más ampliamente con ellos y quererlos con más afecto.

\par
%\textsuperscript{(1204.1)}
\textsuperscript{110:1.3} Aunque los habitantes divinos se preocupan principalmente de vuestra preparación espiritual para la próxima etapa de la existencia sin fin, también se interesan profundamente por vuestro bienestar temporal y por vuestros logros reales en la Tierra. Les encanta contribuir a vuestra salud, felicidad y verdadera prosperidad. No son indiferentes a vuestro éxito en todos los asuntos relacionados con vuestro avance planetario que no sean contrarios a vuestra vida futura de progreso eterno.

\par
%\textsuperscript{(1204.2)}
\textsuperscript{110:1.4} A los Ajustadores les interesan y les preocupan vuestras actividades diarias y los múltiples detalles de vuestra vida en la medida exacta en que éstos influyen en la determinación de vuestras elecciones temporales significativas y de vuestras decisiones espirituales vitales y que son, en consecuencia, unos factores en la solución del problema de la supervivencia y del progreso eterno de vuestra alma. Aunque el Ajustador es pasivo en lo que se refiere a vuestro bienestar puramente temporal, es divinamente activo en todos los asuntos relacionados con vuestro futuro eterno.

\par
%\textsuperscript{(1204.3)}
\textsuperscript{110:1.5} El Ajustador permanece con vosotros en todos los desastres y durante todas las enfermedades que no destruyen por completo las funciones mentales. Pero cuán cruel es manchar a sabiendas o contaminar deliberadamente de otras maneras el cuerpo físico que debe servir de tabernáculo terrestre\footnote{\textit{El cuerpo es el tabernáculo}: Lc 17:21; Ro 8:9-11; 1 Co 3:16-17; 1 Co 6:19; 2 Co 6:16; 2 Ti 1:14; 1 Jn 4:12-15; Ap 21:3.} a este don maravilloso de Dios. Todos los venenos físicos retrasan considerablemente los esfuerzos del Ajustador por elevar la mente material, mientras que los venenos mentales del miedo, la cólera, la envidia, los celos, la desconfianza y la intolerancia obstaculizan también enormemente el progreso espiritual del alma evolutiva.

\par
%\textsuperscript{(1204.4)}
\textsuperscript{110:1.6} Actualmente estáis atravesando el período en que vuestro Ajustador os corteja; y si os limitáis a mostraros fieles a la confianza depositada en vosotros por el espíritu divino que busca vuestra mente y vuestra alma para una unión eterna, finalmente se producirá esa unidad morontial, esa armonía celestial, esa coordinación cósmica, esa sintonización divina, esa fusión celestial, esa mezcla interminable de identidad, esa unidad de existencia que será tan perfecta y final, que ni siquiera las personalidades más experimentadas podrán nunca separar o reconocer a los dos asociados fusionados ---el hombre mortal y el Ajustador divino--- como identidades separadas\footnote{\textit{Unidad de espíritu}: 1 Co 6:17.}.

\section*{2. Los Ajustadores y la voluntad humana}
\par
%\textsuperscript{(1204.5)}
\textsuperscript{110:2.1} Cuando los Ajustadores del Pensamiento habitan en la mente humana, traen consigo las carreras modelo, las vidas ideales que han sido determinadas y preordenadas por ellos mismos y por los Ajustadores Personalizados de Divinington, y certificadas por el Ajustador Personalizado de Urantia. Empiezan pues a trabajar con un plan definido y predeterminado para el desarrollo intelectual y espiritual de sus sujetos humanos, pero ningún ser humano está obligado a aceptar este plan. Todos sois sujetos predestinados, pero no está ordenado de antemano que tengáis que aceptar esta predestinación divina; tenéis plena libertad para rechazar cualquier parte o todo el programa de los Ajustadores del Pensamiento. Su misión es efectuar los cambios mentales y los ajustes espirituales que autoricéis de manera voluntaria e inteligente, a fin de conseguir más influencia sobre la orientación de vuestra personalidad; pero estos Monitores divinos no se aprovechan de vosotros en ninguna circunstancia ni influyen arbitrariamente de ninguna manera en vuestras elecciones y decisiones. Los Ajustadores respetan la soberanía de vuestra personalidad; \textit{siempre están subordinados a vuestra voluntad}.

\par
%\textsuperscript{(1204.6)}
\textsuperscript{110:2.2} Son perseverantes, ingeniosos y perfectos en sus métodos de trabajo, pero no violentan nunca la individualidad volitiva de sus anfitriones. Ningún ser humano será nunca espiritualizado en contra de su voluntad por un Monitor divino; la supervivencia es un don\footnote{\textit{La supervivencia es un don}: Ro 5:15-18; 6:23; 2 Co 9:15; Ef 2:8.} de los Dioses que ha de ser deseado por las criaturas del tiempo. A fin de cuentas, cualquier cosa que el Ajustador haya logrado hacer por vosotros, los archivos mostrarán que esa transformación ha sido realizada con vuestro consentimiento cooperativo\footnote{\textit{Consentimiento cooperativo}: Jn 3:15-16,36; Hch 8:37; 16:31; Ap 22:17.}; habréis sido un asociado voluntario del Ajustador para alcanzar cada etapa de la enorme transformación de la carrera ascendente.

\par
%\textsuperscript{(1205.1)}
\textsuperscript{110:2.3} El Ajustador no trata de controlar vuestro pensamiento como tal, sino más bien de espiritualizarlo, de eternizarlo. Ni los ángeles ni los Ajustadores se dedican directamente a influir sobre el pensamiento humano; ésta es una prerrogativa exclusiva de vuestra personalidad. Los Ajustadores se dedican a mejorar, modificar, ajustar y coordinar vuestros procesos mentales; pero se consagran más especial y específicamente a la tarea de construir las contrapartidas espirituales de vuestra carrera, las transcripciones morontiales de vuestro verdadero yo en progreso, a fin de hacerlo sobrevivir.

\par
%\textsuperscript{(1205.2)}
\textsuperscript{110:2.4} Los Ajustadores trabajan en las esferas de los niveles superiores de la mente humana, tratando sin cesar de producir los duplicados morontiales de cada concepto del intelecto mortal. Existen pues dos realidades que inciden y están centradas en los circuitos de la mente humana: una es un yo mortal surgido por evolución de los planes originales de los Portadores de Vida, y la otra es una entidad inmortal procedente de las altas esferas de Divinington, un don interior de Dios. Pero el yo mortal es también un yo personal; tiene una personalidad.

\par
%\textsuperscript{(1205.3)}
\textsuperscript{110:2.5} Vosotros, como criaturas personales, tenéis una mente y una voluntad. El Ajustador, como criatura prepersonal, tiene una premente y una prevoluntad. Si os ajustáis tan plenamente con la mente del Ajustador como para ver con los mismos ojos, entonces vuestras mentes se volverán una sola\footnote{\textit{Las mentes se volverán una sola}: Ro 7:23-25; 8:27; 12:2; Ef 4:23; Flp 2:5; Heb 8:10.}, y recibiréis el refuerzo de la mente del Ajustador. Posteriormente, si vuestra voluntad ordena e impone la ejecución de las decisiones de esta mente nueva o combinada, la voluntad prepersonal del Ajustador conseguirá expresarse como personalidad a través de vuestra decisión, y en la medida en que afecta a este proyecto particular, vosotros y el Ajustador seréis una sola cosa. Vuestra mente habrá alcanzado la sintonización con la divinidad, y la voluntad del Ajustador habrá logrado expresarse como personalidad.

\par
%\textsuperscript{(1205.4)}
\textsuperscript{110:2.6} En la medida en que se realiza esta identidad, os acercáis mentalmente al tipo de existencia morontial. El término mente morontial significa la sustancia y la suma total de unas mentes de naturaleza diversamente material y espiritual en cooperación. El intelecto morontial implica por lo tanto, en el universo local, una mente doble dominada por una sola voluntad. Para los mortales se trata de una voluntad de origen humano que se vuelve divina a medida que el hombre identifica su mente humana con la dotación mental de Dios.

\section*{3. La cooperación con el Ajustador}
\par
%\textsuperscript{(1205.5)}
\textsuperscript{110:3.1} Los Ajustadores juegan el juego magnífico y sagrado de todos los tiempos; están metidos en una de las aventuras supremas del tiempo en el espacio. Y qué felices se sienten cuando vuestra cooperación les permite prestaros ayuda en vuestras breves luchas temporales, mientras continúan llevando a cabo sus tareas eternas más amplias. Pero cuando vuestro Ajustador intenta comunicarse con vosotros, su mensaje se pierde generalmente en las corrientes materiales de los flujos de energía de la mente humana; sólo de vez en cuando captáis un eco, un eco débil y distante, de la voz divina.

\par
%\textsuperscript{(1205.6)}
\textsuperscript{110:3.2} El éxito de vuestro Ajustador en la empresa de guiaros a través de la vida mortal y de conseguir vuestra supervivencia no depende tanto de las teorías de vuestras creencias como de vuestras decisiones, determinaciones, y de vuestra \textit{fe} inquebrantable. Todos estos movimientos del crecimiento de la personalidad se convierten en unas poderosas influencias que contribuyen a vuestro progreso porque os ayudan a cooperar con el Ajustador; os ayudan a dejar de oponerle resistencia. Los Ajustadores del Pensamiento tienen éxito o fracasan en apariencia en sus empresas terrestres en la medida exacta en que los mortales logran o no logran cooperar con el programa destinado a hacerlos avanzar a lo largo del camino ascendente que lleva a alcanzar la perfección. El secreto de la supervivencia está envuelto en el supremo deseo humano de ser semejante a Dios, y en la buena voluntad asociada de hacer y de ser todas las cosas que son esenciales para alcanzar finalmente ese deseo dominante.

\par
%\textsuperscript{(1206.1)}
\textsuperscript{110:3.3} Cuando hablamos del éxito o del fracaso de un Ajustador, hablamos desde el punto de vista de la supervivencia humana. \textit{Los Ajustadores no fracasan nunca}; son de esencia divina y siempre salen triunfantes de cada una de sus empresas.

\par
%\textsuperscript{(1206.2)}
\textsuperscript{110:3.4} No puedo sino observar que muchos de vosotros empleáis mucho tiempo y esfuerzos mentales en las cosas insignificantes de la vida, mientras que pasáis por alto casi por completo las realidades más esenciales de importancia eterna, aquellos logros que están precisamente relacionados con el desarrollo de un acuerdo de trabajo más armonioso entre vosotros y vuestro Ajustador. La gran meta de la existencia humana consiste en sintonizarse con la divinidad del Ajustador interior; el gran logro de la vida mortal consiste en alcanzar una verdadera consagración comprensiva a los objetivos eternos del espíritu divino que espera y trabaja dentro de vuestra mente. Pero un esfuerzo ferviente y determinado por hacer realidad el destino eterno es enteramente compatible con una vida despreocupada y alegre, y con una carrera lograda y honorable en la Tierra. La cooperación con el Ajustador del Pensamiento no implica que haya que torturarse, fingir piedad o autodegradarse de manera hipócrita y ostentosa; la vida ideal consiste en servir con amor, en lugar de llevar una existencia de aprensión temerosa.

\par
%\textsuperscript{(1206.3)}
\textsuperscript{110:3.5} La confusión, el sentirse desconcertado e incluso a veces desalentado y perturbado, no significa necesariamente resistencia a las directrices del Ajustador interior. Estas actitudes implican a veces una falta de cooperación activa con el Monitor divino y, por lo tanto, pueden retrasar un poco el progreso espiritual, pero estas dificultades emotivas intelectuales no obstaculizan en lo más mínimo la supervivencia segura del alma que conoce a Dios. La ignorancia por sí sola nunca puede impedir la supervivencia, así como tampoco las dudas confusas o la incertidumbre temerosa. Sólo la resistencia consciente a la guía del Ajustador puede impedir la supervivencia del alma inmortal en evolución.

\par
%\textsuperscript{(1206.4)}
\textsuperscript{110:3.6} No debéis considerar que la cooperación con vuestro Ajustador es un proceso particularmente consciente, porque no lo es; pero vuestros móviles y decisiones, vuestras fieles determinaciones y vuestros deseos supremos constituyen de hecho una cooperación real y eficaz. Podéis acrecentar conscientemente la armonía con el Ajustador:

\par
%\textsuperscript{(1206.5)}
\textsuperscript{110:3.7} 1. Escogiendo responder a la guía divina\footnote{\textit{Escogiendo la guía divina}: Is 55:1-3; Mt 5:6; 6:33; Lc 12:31.}; basando sinceramente vuestra vida humana en vuestra conciencia más elevada sobre la verdad, la belleza y la bondad, y luego coordinar estas cualidades de la divinidad mediante la sabiduría, la adoración, la fe y el amor.

\par
%\textsuperscript{(1206.6)}
\textsuperscript{110:3.8} 2. Amando a Dios\footnote{\textit{Amando a Dios}: Dt 6:4-5; 10:12; 11:1,13,22; 13:3; 19:9; 30:6,16,20; Mt 22:37; Mc 12:30; Lc 10:27; Ro 8:28; Jos 22:5; 23:11.} y deseando pareceros a él ---el auténtico reconocimiento de la paternidad divina y la adoración amorosa del Padre celestial.

\par
%\textsuperscript{(1206.7)}
\textsuperscript{110:3.9} 3. Amando a los hombres\footnote{\textit{Amando a los hombres}: Lv 19:18,34; Mt 5:43-44; 19:19b; 22:39; Mc 12:31,33; Lc 10:27; Ro 13:9b; Gl 5:13-14; 1 Ts 4:9; Stg 2:8; 1 P 1:22; 1 Jn 3:11,23; 4:7.11-12,21; 2 Jn 1:5.} y deseando sinceramente servirles ---el reconocimiento sincero de la fraternidad de los hombres, unido a un afecto inteligente y sabio por cada uno de vuestros semejantes mortales\footnote{\textit{Amando a los semejantes como hizo Jesús}: Jn 13:34-35; 15:12,17.}.

\par
%\textsuperscript{(1206.8)}
\textsuperscript{110:3.10} 4. Aceptando alegremente la ciudadanía cósmica ---el reconocimiento honrado de vuestras obligaciones progresivas hacia el Ser Supremo, la conciencia de la interdependencia del hombre evolutivo y de la Deidad en evolución. Es el nacimiento de la moralidad cósmica y la comprensión naciente del deber universal.

\section*{4. El trabajo del Ajustador en la mente}
\par
%\textsuperscript{(1207.1)}
\textsuperscript{110:4.1} Los Ajustadores son capaces de recibir la corriente continua de inteligencia cósmica que llega por los principales circuitos del tiempo y del espacio; están plenamente en contacto con la inteligencia y la energía espirituales de los universos. Pero estos poderosos habitantes interiores son incapaces de transmitir una gran parte de esta riqueza de sabiduría y de verdad a la mente de sus sujetos mortales, debido a la falta de naturaleza común y a la ausencia de reconocimiento sensible.

\par
%\textsuperscript{(1207.2)}
\textsuperscript{110:4.2} El Ajustador del Pensamiento está ocupado en un esfuerzo constante por espiritualizar vuestra mente de tal manera que pueda hacer evolucionar vuestra alma morontial; pero vosotros mismos sois generalmente inconscientes de este ministerio interior. Sois totalmente incapaces de distinguir entre el producto de vuestro propio intelecto material y el de las actividades conjuntas de vuestra alma y el Ajustador.

\par
%\textsuperscript{(1207.3)}
\textsuperscript{110:4.3} Algunas presentaciones súbitas de pensamientos, conclusiones y otras imágenes mentales son a veces la obra directa o indirecta del Ajustador; pero se trata, mucho más a menudo, de la aparición repentina en la conciencia de unas ideas que se han agrupado por sí solas en los niveles mentales subconscientes, los sucesos naturales y cotidianos de la función psíquica normal y ordinaria inherente a los circuitos de la mente animal en evolución. (A diferencia de estas emanaciones subconscientes, las revelaciones del Ajustador aparecen a través del ámbito de la superconciencia.)

\par
%\textsuperscript{(1207.4)}
\textsuperscript{110:4.4} Confiad a la custodia de los Ajustadores todos los asuntos mentales que sobrepasan el nivel adormecido de la conciencia de sí. A su debido tiempo os darán buena cuenta de su gestión, si no en este mundo pues entonces en los mundos de las mansiones, y harán aparecer finalmente aquellos significados y valores que fueron confiados a su cargo y cuidado. Si sobrevivís, resucitarán cada tesoro valioso de vuestra mente mortal.

\par
%\textsuperscript{(1207.5)}
\textsuperscript{110:4.5} Existe un inmenso abismo entre lo humano y lo divino, entre el hombre y Dios. Las razas de Urantia están tan ampliamente controladas eléctrica y químicamente, su comportamiento común se parece tanto al de los animales, sus reacciones habituales son tan emotivas, que a los Monitores les resulta extremadamente difícil guiarlas y dirigirlas. Estáis tan desprovistos de decisiones valientes y de una cooperación consagrada, que a vuestros Ajustadores interiores les resulta casi imposible comunicarse directamente con la mente humana. Incluso cuando les es posible transmitir un destello de verdad nueva al alma mortal evolutiva, a menudo esta revelación espiritual ciega tanto a la criatura que provoca una conmoción de fanatismo o desencadena algún otro trastorno intelectual que resulta desastroso. Muchas religiones nuevas y extraños <<ismos>> han nacido como consecuencia de las comunicaciones abortadas, imperfectas, mal comprendidas y confusas de los Ajustadores del Pensamiento.

\par
%\textsuperscript{(1207.6)}
\textsuperscript{110:4.6} Durante muchos miles de años, y así lo muestran los archivos de Jerusem, en cada generación han vivido cada vez menos seres que podían trabajar sin peligro con los Ajustadores autónomos. Esto es un cuadro alarmante, y las personalidades supervisoras de Satania consideran favorablemente las propuestas de algunos de vuestros supervisores planetarios más inmediatos que recomiendan la introducción de medidas destinadas a fomentar y conservar los tipos espirituales más elevados de las razas de Urantia.

\section*{5. Conceptos erróneos sobre la guía de los Ajustadores}
\par
%\textsuperscript{(1207.7)}
\textsuperscript{110:5.1} No confundáis ni mezcléis la misión y la influencia del Ajustador con lo que se llama habitualmente la conciencia moral; no están directamente relacionadas. La conciencia moral es una reacción humana y puramente psíquica. No hay que menospreciarla, pero difícilmente es la voz de Dios para el alma, como lo sería en verdad la voz del Ajustador si pudiera ser escuchada. La conciencia moral os exhorta, con razón, a obrar bien; pero el Ajustador se esfuerza además por deciros cuál es realmente el bien; es decir, en el momento y la medida en que sois capaces de percibir la guía del Monitor.

\par
%\textsuperscript{(1208.1)}
\textsuperscript{110:5.2} Las experiencias oníricas del hombre, ese desfile desordenado y desconectado de la mente dormida no coordinada, ofrecen una prueba adecuada del fracaso de los Ajustadores en armonizar y asociar los factores divergentes de la mente del hombre. Los Ajustadores simplemente no pueden, en una sola vida, coordinar y sincronizar arbitrariamente dos tipos de pensamiento tan distintos y diferentes como el humano y el divino. Cuando lo hacen, como a veces lo han hecho, dichas almas son transferidas directamente a los mundos de las mansiones sin necesidad de pasar por la experiencia de la muerte.

\par
%\textsuperscript{(1208.2)}
\textsuperscript{110:5.3} Durante los períodos del sueño, el Ajustador sólo trata de llevar a cabo aquello que la voluntad de la personalidad habitada ha aprobado previamente por completo mediante las decisiones y las elecciones efectuadas durante los momentos en que la conciencia estaba plenamente despierta, unas decisiones y elecciones que se han alojado por ello en las zonas supermentales, el campo de conexión de las relaciones recíprocas entre lo humano y lo divino.

\par
%\textsuperscript{(1208.3)}
\textsuperscript{110:5.4} Mientras sus anfitriones mortales duermen, los Ajustadores tratan de registrar sus creaciones en los niveles superiores de la mente material, y algunos de vuestros sueños grotescos indican que los Ajustadores no han logrado establecer un contacto eficaz. Los absurdos de la vida onírica no demuestran solamente la presión de las emociones no expresadas, sino que también atestiguan que las representaciones de los conceptos espirituales presentados por los Ajustadores son horriblemente deformadas. Vuestras propias pasiones, impulsos y otras tendencias innatas se trasladan a la imagen mental, y sus deseos inexpresados sustituyen a los mensajes divinos que los habitantes interiores se esfuerzan por introducir en los registros psíquicos durante el sueño inconsciente.

\par
%\textsuperscript{(1208.4)}
\textsuperscript{110:5.5} Es extremadamente peligroso hacer suposiciones sobre lo que, en la vida onírica, procede de los Ajustadores. Los Ajustadores trabajan de hecho durante el sueño, pero vuestras experiencias oníricas ordinarias son unos fenómenos puramente fisiológicos y psicológicos. Asimismo, es arriesgado intentar diferenciar entre el registro de los conceptos del Ajustador y la recepción más o menos continua y consciente de los dictados de la conciencia moral humana. Éstos son unos problemas que deberán resolverse mediante el discernimiento individual y las decisiones personales. Pero un ser humano haría mejor en equivocarse, rechazando la expresión de un Ajustador por creer que se trata de una experiencia puramente humana, que cometer el error de elevar una reacción de la mente mortal a la esfera de dignidad divina. Recordad que la influencia de un Ajustador del Pensamiento es en su mayor parte, aunque no del todo, una experiencia superconsciente.

\par
%\textsuperscript{(1208.5)}
\textsuperscript{110:5.6} Vosotros os comunicáis con vuestro Ajustador en grados variables y de forma creciente a medida que ascendéis los círculos psíquicos, a veces directamente, pero más a menudo de manera indirecta. Pero es peligroso albergar la idea de que cada nuevo concepto que se origina en la mente humana es dictado por el Ajustador. Con más frecuencia, y en los seres de vuestra orden, aquello que aceptáis como la voz del Ajustador es en realidad la emanación de vuestro propio intelecto. El terreno es peligroso, y cada ser humano debe resolver estos problemas por sí mismo de acuerdo con su sabiduría humana natural y su perspicacia superhumana.

\par
%\textsuperscript{(1208.6)}
\textsuperscript{110:5.7} El Ajustador del ser humano a través del cual se transmite esta comunicación disfruta de un campo de acción tan amplio debido principalmente a que este humano manifiesta una indiferencia casi completa por toda manifestación exterior de la presencia interior del Ajustador; es en verdad una suerte que permanezca de forma consciente completamente indiferente a todo el proceso. Posee uno de los Ajustadores más experimentados de su época y de su generación, y sin embargo el guardián del destino declara que su reacción pasiva y su falta de preocupación por los fenómenos asociados a la presencia en su mente de este polifacético Ajustador es una reacción rara y fortuita. Todo esto constituye una favorable coordinación de influencias, favorable tanto para el Ajustador en su esfera superior de acción como para el asociado humano desde el punto de vista de la salud, la eficacia y la tranquilidad.

\section*{6. Los siete círculos psíquicos}
\par
%\textsuperscript{(1209.1)}
\textsuperscript{110:6.1} La suma total de la realización de la personalidad en un mundo material está contenida en la conquista sucesiva de los siete círculos psíquicos de potencialidad mortal. La entrada en el séptimo círculo señala el comienzo del verdadero funcionamiento de la personalidad humana. La culminación del primer círculo indica la madurez relativa del ser mortal. Aunque atravesar los siete círculos de crecimiento cósmico no equivale a la fusión con el Ajustador, el dominio de estos círculos revela que se han alcanzado las etapas preliminares para fusionar con el Ajustador.

\par
%\textsuperscript{(1209.2)}
\textsuperscript{110:6.2} El Ajustador es vuestro asociado en un plano de igualdad para alcanzar los siete círculos ---para lograr una madurez humana relativa. El Ajustador asciende los círculos con vosotros desde el séptimo hasta el primero, pero progresa hacia el estado de actividad autónoma y de supremacía de forma totalmente independiente a la cooperación activa de la mente mortal.

\par
%\textsuperscript{(1209.3)}
\textsuperscript{110:6.3} Los círculos psíquicos no son exclusivamente intelectuales ni totalmente morontiales; tienen que ver con el estado de la personalidad, los logros de la mente, el crecimiento del alma y la sintonización con el Ajustador. La travesía con éxito de estos niveles requiere el funcionamiento armónico de \textit{toda lapersonalidad}, y no simplemente de algunas de sus fases. El crecimiento de las partes no equivale a la verdadera maduración del todo; las partes crecen realmente en proporción a la expansión del yo completo ---de todo el yo--- material, intelectual y espiritual.

\par
%\textsuperscript{(1209.4)}
\textsuperscript{110:6.4} Cuando el desarrollo de la naturaleza intelectual avanza más deprisa que el de la espiritual, esta situación hace que la comunicación con el Ajustador del Pensamiento resulte difícil y peligrosa. Asimismo, un desarrollo espiritual excesivo tiende a ocasionar una interpretación fanática y desnaturalizada de las directrices espirituales del habitante divino. La falta de capacidad espiritual hace muy difícil transmitir a un intelecto material las verdades espirituales situadas en la superconciencia más elevada. A una mente perfectamente equilibrada, alojada en un cuerpo de costumbres sanas, de energías nerviosas estabilizadas y de funciones químicas equilibradas ---cuando los poderes físicos, mentales y espirituales se desarrollan en armonía trina--- es a la que se le puede comunicar un máximo de luz y de verdad con un mínimo de peligro o de riesgo temporales para el bienestar real de dicho ser. El hombre asciende los círculos de la progresión planetaria uno tras otro, desde el séptimo hasta el primero, gracias a este crecimiento equilibrado.

\par
%\textsuperscript{(1209.5)}
\textsuperscript{110:6.5} Los Ajustadores siempre están cerca de vosotros y en vosotros, pero es raro que os puedan hablar directamente como lo haría otro ser. Círculo tras círculo, vuestras decisiones intelectuales, elecciones morales y desarrollo espiritual aumentan la capacidad del Ajustador para funcionar en vuestra mente; círculo tras círculo os eleváis así desde los estados inferiores de asociación y de sintonización mental con el Ajustador, de manera que éste se encuentra cada vez más capacitado para registrar sus imágenes del destino, con una intensidad y una convicción crecientes, en la conciencia evolutiva de esta mente-alma que busca a Dios.

\par
%\textsuperscript{(1210.1)}
\textsuperscript{110:6.6} Cada decisión que tomáis impide o facilita la función del Ajustador; esas mismas decisiones determinan igualmente vuestro avance en los círculos de la consecución humana. Es cierto que la supremacía de una decisión, su relación con una crisis, tiene mucho que ver con su influencia para franquear los círculos; sin embargo, el número de decisiones, las repeticiones frecuentes, las repeticiones persistentes, son esenciales también para tener la certeza de que tales reacciones se convertirán en hábitos.

\par
%\textsuperscript{(1210.2)}
\textsuperscript{110:6.7} Es difícil definir con precisión los siete niveles de la progresión humana, ya que estos niveles son personales; varían para cada individuo y están aparentemente determinados por la capacidad de crecimiento de cada ser humano. La conquista de estos niveles de evolución cósmica se refleja de tres maneras:

\par
%\textsuperscript{(1210.3)}
\textsuperscript{110:6.8} 1. \textit{La sintonización con el Ajustador}. La mente que se espiritualiza se acerca a la presencia del Ajustador de manera proporcional a la conquista de los círculos.

\par
%\textsuperscript{(1210.4)}
\textsuperscript{110:6.9} 2. \textit{La evolución del alma}. La aparición del alma morontial indica la extensión y la profundidad del dominio de los círculos.

\par
%\textsuperscript{(1210.5)}
\textsuperscript{110:6.10} 3. \textit{La realidad de la personalidad}. La conquista de los círculos determina directamente el grado de realidad de la individualidad. Las personas se vuelven más reales a medida que se elevan desde el séptimo hasta el primer nivel de la existencia mortal.

\par
%\textsuperscript{(1210.6)}
\textsuperscript{110:6.11} A medida que el niño nacido de la evolución material atraviesa los círculos, se convierte en un humano maduro con una potencialidad inmortal. La realidad indistinta de la naturaleza embrionaria de un hombre que se encuentra en el séptimo círculo da paso a la manifestación más clara de la naturaleza morontial emergente de un ciudadano del universo local.

\par
%\textsuperscript{(1210.7)}
\textsuperscript{110:6.12} Aunque es imposible definir con precisión los siete niveles, o círculos psíquicos, del crecimiento humano, podemos sugerir los límites mínimos y máximos de estas etapas de realización de la madurez:

\par
%\textsuperscript{(1210.8)}
\textsuperscript{110:6.13} \textit{El séptimo círculo}. Los seres humanos entran en este nivel cuando desarrollan los poderes de la elección personal, la decisión individual, la responsabilidad moral y la capacidad para alcanzar la individualidad espiritual. Esto indica el funcionamiento unido de los siete espíritus ayudantes de la mente bajo la dirección del espíritu de la sabiduría, la inclusión de la criatura mortal en los circuitos de influencia del Espíritu Santo y, en Urantia, el funcionamiento inicial del Espíritu de la Verdad, junto con la recepción de un Ajustador del Pensamiento por parte de la mente mortal. La entrada en el séptimo círculo convierte a una criatura mortal en un verdadero ciudadano potencial del universo local.

\par
%\textsuperscript{(1210.9)}
\textsuperscript{110:6.14} \textit{El tercer círculo}. El trabajo del Ajustador es mucho más eficaz después de que el ascendente humano alcanza el tercer círculo y recibe un guardián seráfico personal del destino. Aunque en apariencia no existen unos esfuerzos concertados entre el Ajustador y el guardián seráfico, sin embargo se puede observar una mejora evidente en todas las fases de consecución cósmica y de desarrollo espiritual después de la asignación del asistente seráfico personal. Cuando se alcanza el tercer círculo, el Ajustador se esfuerza por morontializar la mente del hombre durante el resto de su vida como mortal, por conquistar los círculos restantes y por alcanzar la etapa final de la asociación humano-divina antes de que la muerte natural disuelva esta asociación excepcional.

\par
%\textsuperscript{(1210.10)}
\textsuperscript{110:6.15} \textit{El primer círculo}. Habitualmente, el Ajustador no puede hablar de manera directa e inmediata con vosotros hasta que alcanzáis el círculo primero y final de consecución mortal progresiva. Este nivel representa el máximo desarrollo posible al que pueden llegar las relaciones entre la mente y el Ajustador durante la experiencia humana, antes de que el alma morontial en evolución sea liberada de las vestiduras del cuerpo material. En lo que se refiere a la mente, las emociones y la perspicacia cósmica, alcanzar el primer círculo psíquico representa el acercamiento más grande posible entre la mente material y el Ajustador espiritual en la experiencia humana.

\par
%\textsuperscript{(1211.1)}
\textsuperscript{110:6.16} Estos círculos psíquicos de progresión mortal quizás deberían denominarse mejor \textit{niveles cósmicos} ---unos niveles donde se captan realmente los significados y se comprenden los valores del acercamiento progresivo a la conciencia morontial de la relación inicial entre el alma evolutiva y el Ser Supremo emergente. Esta misma relación es la que hace imposible para siempre explicar plenamente el significado de los círculos cósmicos a la mente material. La conquista de estos círculos sólo tiene una relación relativa con la conciencia de Dios. Una persona que se encuentra en el séptimo o sexto círculo puede conocer a Dios ---ser consciente de su filiación--- casi tan bien como aquella que esté en el segundo o el primer círculo, pero estos seres de los círculos inferiores son mucho menos conscientes de su relación experiencial con el Ser Supremo, de su ciudadanía universal. La conquista de estos círculos cósmicos formará parte de la experiencia de los ascendentes en los mundos de las mansiones, si no han logrado alcanzarlos antes de la muerte natural.

\par
%\textsuperscript{(1211.2)}
\textsuperscript{110:6.17} La motivación de la fe convierte en experiencial la realización completa de la filiación del hombre con Dios, pero \textit{la acción}, la consumación de las decisiones, es esencial para alcanzar por evolución la conciencia del parentesco progresivo con la \textit{realidad cósmica} del Ser Supremo. En el mundo espiritual, la fe transmuta los potenciales en actuales, pero los potenciales sólo se vuelven actuales, en los reinos finitos del Supremo, llevando a cabo la experiencia de la elección y a través de ella. Escoger hacer la voluntad de Dios une la fe espiritual con las decisiones materiales en los actos de la personalidad, proporcionando así un punto de apoyo divino y espiritual para que la palanca humana y material del hambre de Dios funcione con más eficacia. Esta sabia coordinación de las fuerzas materiales y espirituales acrecienta considerablemente tanto el entendimiento cósmico del Supremo como la comprensión morontial de las Deidades del Paraíso.

\par
%\textsuperscript{(1211.3)}
\textsuperscript{110:6.18} El dominio de los círculos cósmicos está relacionado con el crecimiento cuantitativo del alma morontial, la comprensión de los significados supremos. Pero el estado cualitativo de este alma inmortal depende \textit{totalmente} de que la fe viviente capte el valor del hecho, potencial y paradisiaco, de que el hombre mortal es un hijo del Dios eterno. Por eso aquellas personas que están en el séptimo círculo van a los mundos de las mansiones para alcanzar una realización cuantitativa adicional en su crecimiento cósmico, exactamente como las que se encuentran en el segundo o incluso en el primer círculo.

\par
%\textsuperscript{(1211.4)}
\textsuperscript{110:6.19} Sólo existe una relación indirecta entre la conquista de los círculos cósmicos y la experiencia religiosa espiritual real; estos logros son recíprocos y por lo tanto mutuamente beneficiosos. El desarrollo puramente espiritual puede tener muy poca relación con la prosperidad material planetaria, pero la conquista de los círculos acrecienta siempre el potencial del éxito humano y de los logros mortales.

\par
%\textsuperscript{(1211.5)}
\textsuperscript{110:6.20} Desde el séptimo hasta el tercer círculo, los siete espíritus ayudantes de la mente ejercen una acción creciente y unificada para liberar a la mente mortal de su dependencia de las realidades de los mecanismos de la vida material, con miras a introducirla cada vez más en los niveles morontiales de experiencia. Desde el tercer círculo en adelante, la influencia de los ayudantes disminuye progresivamente.

\par
%\textsuperscript{(1211.6)}
\textsuperscript{110:6.21} Los siete círculos abarcan la experiencia mortal que se extiende desde el nivel puramente animal más elevado hasta el nivel de conciencia morontial de contacto real más bajo como experiencia de la personalidad. El dominio del primer círculo cósmico señala que se ha alcanzado la madurez mortal premorontial, e indica la finalización del ministerio conjunto de los espíritus ayudantes de la mente como influencia exclusiva de acción mental en la personalidad humana. Más allá del primer círculo, la mente se vuelve cada vez más semejante a la inteligencia del estado morontial de evolución, el ministerio conjunto de la mente cósmica y de la dotación superayudante del Espíritu Creativo de un universo local.

\par
%\textsuperscript{(1212.1)}
\textsuperscript{110:6.22} Los grandes días en la carrera individual de los Ajustadores son los siguientes: primero, cuando el sujeto humano irrumpe en el tercer círculo psíquico, lo cual asegura la actividad autónoma y una gama creciente de funciones del Monitor (siempre que el habitante interior no fuera ya autónomo); luego, cuando el compañero humano alcanza el primer círculo psíquico, lo cual les permite comunicarse entre ellos, al menos hasta cierto punto; y finalmente, cuando fusionan de manera eterna y definitiva.

\section*{7. La consecución de la inmortalidad}
\par
%\textsuperscript{(1212.2)}
\textsuperscript{110:7.1} La conquista de los siete círculos cósmicos no equivale a la fusión con el Ajustador. Hay muchos mortales que viven en Urantia que han alcanzado sus círculos; pero la fusión depende además de otros logros espirituales más grandes y más sublimes, del hecho de conseguir una sintonización final y completa entre la voluntad mortal y la voluntad de Dios, tal como ésta reside en el Ajustador del Pensamiento.

\par
%\textsuperscript{(1212.3)}
\textsuperscript{110:7.2} Cuando un ser humano ha terminado los círculos de consecución cósmica, y además, cuando la elección final de la voluntad mortal permite al Ajustador completar la asociación entre la identidad humana y el alma morontial durante la vida física evolutiva, entonces estos enlaces consumados del alma y del Ajustador pasan independientemente a los mundos de las mansiones, y desde Uversa se emite el mandato que asegura la fusión inmediata del Ajustador y del alma morontial. Esta fusión durante la vida física consume instantáneamente el cuerpo material; los seres humanos que pudieran presenciar este espectáculo sólo observarían que el mortal en vías de ser transferido desaparece <<en carros de fuego>>\footnote{\textit{En carros de fuego}: 2 Re 2:11; 2 Re 6:17.}.

\par
%\textsuperscript{(1212.4)}
\textsuperscript{110:7.3} La mayor parte de los Ajustadores que han trasladado a sus sujetos fuera de Urantia eran muy experimentados y hay constancia de que habían residido anteriormente en numerosos mortales de otras esferas. Recordad que los Ajustadores adquieren una valiosa experiencia como habitantes interiores en los planetas donde sólo son prestados; de esto no hay que deducir que los Ajustadores sólo adquieren experiencia, para realizar un trabajo avanzado, en aquellos sujetos mortales que no logran sobrevivir.

\par
%\textsuperscript{(1212.5)}
\textsuperscript{110:7.4} Después de fusionar con vosotros los mortales, los Ajustadores comparten vuestro destino y vuestra experiencia; \textit{ellos son vosotros}. Después de la fusión del alma morontial inmortal con el Ajustador asociado, toda la experiencia y todos los valores de uno se vuelven finalmente propiedad del otro, de manera que los dos forman realmente una sola entidad. En cierto sentido, este nuevo ser pertenece al pasado eterno y existe para el eterno futuro. Todo lo que una vez fue humano en el alma sobreviviente, y todo lo que es experiencialmente divino en el Ajustador, se convierten ahora en la propiedad real de la nueva personalidad universal siempre ascendente. Pero en cada nivel del universo, el Ajustador sólo puede dotar a la nueva criatura de aquellos atributos que tienen un significado y un valor en ese nivel. La \textit{unidad} absoluta con el Monitor divino, el agotamiento completo de la dotación de un Ajustador, sólo se puede lograr en la eternidad después de haber alcanzado finalmente al Padre Universal, el Padre de los espíritus, la fuente permanente de estos dones divinos.

\par
%\textsuperscript{(1212.6)}
\textsuperscript{110:7.5} Cuando el alma evolutiva y el Ajustador divino han fusionados de manera eterna y final, cada uno de ellos adquiere todas las cualidades experimentables del otro. Esta personalidad coordinada posee toda la memoria experiencial de la supervivencia, conservada en otro tiempo por la mente mortal ancestral, y luego existente en el alma morontial; además de esto, este finalitario potencial contiene toda la memoria experiencial que el Ajustador ha acumulado a lo largo de sus estancias en los mortales durante todos los tiempos. Pero el Ajustador necesitará la eternidad del futuro para dotar plenamente esta asociación de personalidad con los significados y los valores que el Monitor divino aporta desde la eternidad del pasado.

\par
%\textsuperscript{(1213.1)}
\textsuperscript{110:7.6} Pero en la gran mayoría de los urantianos, el Ajustador debe esperar pacientemente la llegada de la liberación por medio de la muerte; debe esperar que el alma emergente se libere de la dominación casi completa de los modelos energéticos y de las fuerzas químicas inherentes a vuestra orden material de existencia. La principal dificultad que experimentáis para poneros en contacto con vuestro Ajustador consiste en esta misma naturaleza material inherente. Hay tan pocos mortales que sean verdaderos pensadores; no desarrolláis ni disciplináis espiritualmente vuestra mente hasta el punto de establecer una conexión favorable con los Ajustadores divinos. La mente humana hace casi oídos sordos a las súplicas espirituales que el Ajustador traduce de los múltiples mensajes de las transmisiones universales de amor procedentes del Padre de las misericordias. Al Ajustador le resulta casi imposible registrar estas directrices espirituales inspiradoras en una mente animal tan completamente dominada por las fuerzas químicas y eléctricas inherentes a vuestra naturaleza física.

\par
%\textsuperscript{(1213.2)}
\textsuperscript{110:7.7} Los Ajustadores se regocijan de ponerse en contacto con la mente mortal; pero deben tener paciencia a través de los largos años de estancia silenciosa durante los cuales son incapaces de romper la resistencia animal y de comunicarse directamente con vosotros. Cuanto más se elevan los Ajustadores del Pensamiento en la escala del servicio, más eficaces se vuelven. Pero mientras permanecéis en la carne, nunca pueden saludaros con el mismo afecto pleno, comprensivo y lleno de expresión con que lo harán cuando los discernáis, de mente a mente, en los mundos de las mansiones.

\par
%\textsuperscript{(1213.3)}
\textsuperscript{110:7.8} Durante la vida mortal, el cuerpo y la mente materiales os separan de vuestro Ajustador e impiden la libre comunicación con él; después de la muerte y de la fusión eterna, vosotros y el Ajustador seréis una sola cosa ---no se os podrá distinguir como seres separados--- y ya no existirá ninguna necesidad de comunicación tal como vosotros la entendéis.

\par
%\textsuperscript{(1213.4)}
\textsuperscript{110:7.9} Aunque la voz del Ajustador está siempre dentro de vosotros, la mayoría de vosotros la escuchará raramente durante la vida. Los seres humanos que se encuentran por debajo del tercero y del segundo círculos de consecución escuchan raras veces la voz directa del Ajustador, excepto en los momentos de un deseo supremo, en una situación suprema, o a consecuencia de una decisión suprema.

\par
%\textsuperscript{(1213.5)}
\textsuperscript{110:7.10} Durante el período en que se establece y se rompe el contacto entre la mente humana de un reservista del destino y los supervisores planetarios, el Ajustador interior se encuentra situado a veces de tal manera que le es posible transmitir un mensaje a su asociado mortal. No hace mucho tiempo, en Urantia, un Ajustador autónomo trasmitió un mensaje de este tipo a su asociado humano, miembro del cuerpo de reserva del destino. El mensaje empezaba con estas palabras: <<Ahora, sin perjuicio ni peligro para el sujeto de mi devoción solícita, y sin ninguna intención de desanimarlo o de castigarlo con exceso, registrad por mí esta súplica que le dirijo>>. Luego seguía una exhortación hermosamente conmovedora y atractiva. El Ajustador pedía, entre otras cosas, <<que me conceda más fielmente su cooperación sincera, soporte más alegremente las tareas de mi posición, lleve a cabo más fielmente el programa planeado por mí, pase más pacientemente por las pruebas que he escogido, camine de manera más perseverante y alegre por el sendero que he elegido, reciba más humildemente el crédito que pueda derivarse como consecuencia de mis esfuerzos incesantes ---trasmitid así mi exhortación al hombre en el que habito. Le obsequio con la devoción y el afecto supremos de un espíritu divino. Y decidle además a mi amado sujeto que actuaré con sabiduría y poder hasta el final, hasta que el último esfuerzo terrestre haya terminado; seré fiel a la personalidad que me ha sido confiada. Y le exhorto a sobrevivir, a que no me decepcione, a que no me prive de la recompensa de mi lucha paciente e intensa. Dependemos de la voluntad humana para conseguir la personalidad. Círculo tras círculo he elevado pacientemente esta mente humana, y tengo el testimonio de que el jefe de mi orden me concede su aprobación. Círculo tras círculo paso por un juicio. Espero con placer y sin aprensión el llamamiento nominal del destino; estoy preparado para someterlo todo a los tribunales de los Ancianos de los Días>>.

\par
%\textsuperscript{(1214.1)}
\textsuperscript{110:7.11} [Presentado por un Mensajero Solitario de Orvonton.]


\chapter{Documento 111. El Ajustador y el alma}
\par
%\textsuperscript{(1215.1)}
\textsuperscript{111:0.1} LA PRESENCIA del Ajustador divino en la mente humana hace que a la ciencia o a la filosofía les resulte eternamente imposible alcanzar una comprensión satisfactoria del alma evolutiva de la personalidad humana. El alma morontial es hija del universo y sólo se la puede conocer realmente a través de la perspicacia cósmica y del descubrimiento espiritual.

\par
%\textsuperscript{(1215.2)}
\textsuperscript{111:0.2} El concepto de un alma y de un espíritu interior no es nuevo en Urantia; ha aparecido con frecuencia en los diversos sistemas de creencias planetarias. Muchas religiones orientales, así como algunas doctrinas occidentales, han percibido que el hombre posee una herencia divina al igual que tiene una herencia humana. El sentimiento de la presencia interior, además de la omnipresencia exterior de la Deidad, ha formado parte largo tiempo de muchas religiones urantianas. Los hombres han creído durante mucho tiempo que hay algo que crece dentro de la naturaleza humana, algo vital destinado a perdurar más allá del corto espacio de una vida temporal.

\par
%\textsuperscript{(1215.3)}
\textsuperscript{111:0.3} Antes de que el hombre se diera cuenta de que su alma evolutiva era engendrada por un espíritu divino, se creía que ésta residía en diversos órganos físicos ---el ojo, el hígado, el riñón, el corazón y, más tarde, el cerebro. El salvaje asociaba el alma con la sangre, la respiración, las sombras y con su propia imagen reflejada en el agua.

\par
%\textsuperscript{(1215.4)}
\textsuperscript{111:0.4} En su concepto del \textit{atman}, los educadores hindúes se acercaron realmente a una apreciación de la naturaleza y de la presencia del Ajustador, pero no lograron distinguir la presencia concomitante del alma evolutiva potencialmente inmortal. Los chinos reconocieron sin embargo dos aspectos del ser humano, el \textit{yang} y el \textit{yin}, el alma y el espíritu. Los egipcios y muchas tribus africanas también creían en dos factores, el \textit{ka} y el \textit{ba}; generalmente no se creía que el alma fuera preexistente, sino sólo el espíritu.

\par
%\textsuperscript{(1215.5)}
\textsuperscript{111:0.5} Los habitantes del valle del Nilo creían que a todo individuo favorecido le concedían en el momento de su nacimiento, o poco después, un espíritu protector al que llamaban el ka. Enseñaban que este espíritu guardián permanecía con el sujeto mortal durante toda la vida y pasaba antes que él al estado futuro. En los muros de un templo de Luxor, donde se describe el nacimiento de Amenjótep III, el pequeño príncipe está representado en los brazos del dios del Nilo, y cerca de él se encuentra otro niño, de apariencia idéntica al príncipe, que simboliza esa entidad que los egipcios llamaban el ka. Esta escultura fue acabada en el siglo quince antes de Cristo.

\par
%\textsuperscript{(1215.6)}
\textsuperscript{111:0.6} Se creía que el ka era un genio espiritual superior que deseaba guiar al alma mortal asociada hacia los mejores caminos de la vida temporal, pero sobre todo influir sobre la suerte del sujeto humano en el más allá. Cuando un egipcio de este período moría, se contaba con que su ka lo estaría esperando al otro lado del Gran Río. Al principio, se suponía que sólo los reyes poseían un ka, pero poco después se creyó que todos los hombres justos tenían uno. Un gobernante egipcio, al hablar del ka interior de su corazón, dijo: <<No hice caso omiso de sus palabras; temía transgredir su guía. Por eso prosperé enormemente; triunfé así en virtud de lo que me indujo que hiciera; fui distinguido por su guía>>. Muchos creían que el ka era un <<oráculo de Dios que residía en toda la gente>>. Muchos creían que iban a <<pasar la eternidad con el corazón alegre en el favor del Dios que está en vosotros>>.

\par
%\textsuperscript{(1216.1)}
\textsuperscript{111:0.7} Cada raza de mortales evolutivos de Urantia tiene una palabra que equivale al concepto del alma. Muchos pueblos primitivos creían que el alma observaba el mundo a través de los ojos humanos; por eso temían tan cobardemente la malevolencia del mal de ojo. Durante mucho tiempo creyeron que <<el espíritu del hombre es la lámpara del Señor>>. El Rig Veda dice: <<Mi mente habla a mi corazón>>.

\section*{1. El campo mental de la elección}
\par
%\textsuperscript{(1216.2)}
\textsuperscript{111:1.1} Aunque el trabajo de los Ajustadores es de naturaleza espiritual, deben efectuar forzosamente toda su tarea sobre una base intelectual. La mente es el terreno humano a partir del cual el Monitor espiritual debe hacer evolucionar el alma morontial, con la cooperación de la personalidad en la que habita.

\par
%\textsuperscript{(1216.3)}
\textsuperscript{111:1.2} Existe una unidad cósmica en los diversos niveles mentales del universo de universos. Los yoes intelectuales tienen su origen en la mente cósmica de manera muy parecida a como las nebulosas tienen su origen en las energías cósmicas del espacio universal. En el nivel humano (así pues personal) de los yoes intelectuales, el potencial de evolución espiritual se vuelve dominante, con el consentimiento de la mente mortal, debido a las dotaciones espirituales de la personalidad humana, junto con la presencia creativa de un objeto-entidad de valor absoluto en esos yoes humanos. Pero este dominio del espíritu sobre la mente material está condicionado por dos experiencias: esta mente debe haber evolucionado gracias al ministerio de los siete espíritus ayudantes de la mente, y el yo material (personal) debe escoger cooperar con el Ajustador interior para crear y fomentar el yo morontial, el alma evolutiva potencialmente inmortal.

\par
%\textsuperscript{(1216.4)}
\textsuperscript{111:1.3} La mente material es el ámbito en el que viven las personalidades humanas, son conscientes de sí mismas, toman sus decisiones, escogen o abandonan a Dios, se eternizan o se destruyen a sí mismas.

\par
%\textsuperscript{(1216.5)}
\textsuperscript{111:1.4} La evolución material os ha proporcionado una máquina viviente, vuestro cuerpo; el Padre mismo os ha dotado de la realidad espiritual más pura que se conoce en el universo, vuestro Ajustador del Pensamiento. Pero la mente ha sido puesta en vuestras manos, sometida a vuestras propias decisiones, y es a través de la mente como vivís o morís. Con esta mente y dentro de esta mente es donde tomáis las decisiones morales que os permiten volveros semejantes al Ajustador, es decir semejantes a Dios.

\par
%\textsuperscript{(1216.6)}
\textsuperscript{111:1.5} La mente mortal es un sistema intelectual temporal prestado a los seres humanos para ser utilizado durante una vida material, y según la manera en que utilicen esta mente, estarán aceptando o rechazando el potencial de la existencia eterna. La mente es casi todo lo que poseéis de la realidad universal que está sometido a vuestra voluntad, y el alma ---el yo morontial--- describirá fielmente la cosecha de decisiones temporales que habrá tomado el yo mortal. La conciencia humana descansa suavemente sobre el mecanismo electroquímico situado debajo, y toca delicadamente el sistema energético morontial-espiritual situado encima. El ser humano nunca es completamente consciente de ninguno de estos dos sistemas durante su vida mortal; por eso tiene que trabajar en la mente, de la cual sí es consciente. Lo que asegura la supervivencia no es tanto lo que la mente comprende como lo que la mente desea comprender; lo que constituye la identificación con el espíritu no es tanto cómo es la mente sino cómo la mente se esfuerza por ser. Lo que conduce a la ascensión por el universo no es tanto que el hombre sea consciente de Dios como que el hombre anhele a Dios. Lo que sois hoy no es tan importante como lo que vais siendo día tras día y en la eternidad.

\par
%\textsuperscript{(1217.1)}
\textsuperscript{111:1.6} La mente es el instrumento cósmico donde la voluntad humana puede tocar las disonancias de la destrucción, o en el cual esta misma voluntad puede producir las exquisitas melodías de la identificación con Dios y de la consiguiente supervivencia eterna. A fin de cuentas, el Ajustador otorgado al hombre es impermeable al mal e incapaz de pecar, pero las maquinaciones pecaminosas de una voluntad humana perversa y egoísta pueden realmente deformar, desvirtuar y volver malvada y fea la mente mortal. Del mismo modo, esta mente puede volverse noble, hermosa, verdadera y buena ---realmente grande--- en conformidad con la voluntad iluminada por el espíritu de un ser humano que conoce a Dios.

\par
%\textsuperscript{(1217.2)}
\textsuperscript{111:1.7} La mente evolutiva sólo es plenamente estable y fiable cuando se manifiesta en los dos extremos de la intelectualidad cósmica ---totalmente mecanizada o enteramente espiritualizada. Entre los extremos intelectuales del puro control mecánico y de la verdadera naturaleza espiritual, se encuentra ese enorme grupo de mentes que evolucionan y ascienden, cuya estabilidad y tranquilidad dependen de las elecciones de su personalidad y de su identificación con el espíritu.

\par
%\textsuperscript{(1217.3)}
\textsuperscript{111:1.8} Pero el hombre no abandona su voluntad al Ajustador de una manera pasiva y servil. Elige más bien seguir de forma activa, positiva y cooperativa la guía del Ajustador cuando, y en la medida en que, esta guía difiere conscientemente de los deseos e impulsos de la mente mortal natural. Los Ajustadores manipulan la mente del hombre, pero nunca la dominan en contra de su voluntad; para los Ajustadores, la voluntad humana es suprema. La consideran y la respetan así mientras se esfuerzan por alcanzar las metas espirituales de ajuste del pensamiento y de transformación del carácter en el campo casi ilimitado del intelecto humano en evolución.

\par
%\textsuperscript{(1217.4)}
\textsuperscript{111:1.9} La mente es vuestro buque, el Ajustador es vuestro piloto, la voluntad humana es el capitán. El dueño del navío mortal debería tener la sabiduría de confiar en el piloto divino para que guíe su alma ascendente hacia los puertos morontiales de la supervivencia eterna. La voluntad del hombre sólo puede rechazar la guía de un piloto tan amoroso por egoísmo, pereza y maldad, y hacer naufragar finalmente su carrera como mortal en los nefastos bancos de arena del rechazo de la misericordia y en los arrecifes del abrazo del pecado. Con vuestro consentimiento, este piloto fiel os llevará de manera segura a través de las barreras del tiempo y de los obstáculos del espacio, hasta la fuente misma de la mente divina e incluso más allá, hasta el Padre Paradisiaco de los Ajustadores.

\section*{2. La naturaleza del alma}
\par
%\textsuperscript{(1217.5)}
\textsuperscript{111:2.1} En todas las funciones mentales de la inteligencia cósmica, la totalidad de la mente domina las fracciones de la función intelectual. La mente, en su esencia, es una unidad funcional; por eso la mente nunca deja de manifestar esta unidad constitutiva, incluso cuando se encuentra obstaculizada y entorpecida por las elecciones y los actos insensatos de un yo descaminado. Esta unidad de la mente busca invariablemente la coordinación con el espíritu en todos los niveles en que está asociada con unos yoes con dignidad volitiva y prerrogativas de ascensión.

\par
%\textsuperscript{(1217.6)}
\textsuperscript{111:2.2} La mente material del hombre mortal es el telar cósmico que contiene los tejidos morontiales sobre los cuales el Ajustador del Pensamiento interior entreteje las formas espirituales de un carácter universal compuesto de valores duraderos y de significados divinos ---un alma sobreviviente con un destino último y una carrera sin fin, un finalitario potencial.

\par
%\textsuperscript{(1218.1)}
\textsuperscript{111:2.3} La personalidad humana se identifica con la mente y el espíritu, unidos por la vida en una relación funcional en un cuerpo material. Esta relación funcional entre la mente y el espíritu no da como resultado una combinación de las cualidades o atributos de la mente y del espíritu, sino más bien un valor universal enteramente nuevo, original y único, con una duración potencialmente eterna: el \textit{alma}.

\par
%\textsuperscript{(1218.2)}
\textsuperscript{111:2.4} Existen tres factores, y no dos, en la creación evolutiva de este alma inmortal. Estos tres antecedentes del alma morontial humana son los siguientes:

\par
%\textsuperscript{(1218.3)}
\textsuperscript{111:2.5} 1. \textit{La mente humana} y todas las influencias cósmicas que la preceden e inciden sobre ella.

\par
%\textsuperscript{(1218.4)}
\textsuperscript{111:2.6} 2. \textit{El espíritu divino} que reside en esta mente humana, y todos los potenciales inherentes a este fragmento de espiritualidad absoluta, junto con todas las influencias y factores espirituales asociados en la vida humana.

\par
%\textsuperscript{(1218.5)}
\textsuperscript{111:2.7} 3. \textit{La relación entre la mente material y el espíritu divino}, que conlleva un valor y comporta un significado que no se encuentran en ninguno de los factores que contribuyen a esta asociación. La realidad de esta relación singular no es ni material ni espiritual, sino morontial. Es el alma.

\par
%\textsuperscript{(1218.6)}
\textsuperscript{111:2.8} Hace mucho tiempo que las criaturas intermedias han denominado mente intermedia a este alma evolutiva del hombre, para distinguirla de la mente inferior o material y de la mente superior o cósmica. Esta mente intermedia es en realidad un fenómeno morontial, ya que existe en la zona que se encuentra entre lo material y lo espiritual. El potencial de esta evolución morontial es inherente a los dos impulsos universales de la mente: el impulso de la mente finita de la criatura por conocer a Dios y alcanzar la divinidad del Creador, y el impulso de la mente infinita del Creador por conocer al hombre y llevar a cabo la \textit{experiencia} de la criatura.

\par
%\textsuperscript{(1218.7)}
\textsuperscript{111:2.9} Esta operación celestial de desarrollar por evolución el alma inmortal es posible porque la mente mortal es en primer lugar personal, y en segundo lugar porque está en contacto con unas realidades superanimales; posee una dotación supermaterial de ministerio cósmico que asegura la evolución de una naturaleza moral capaz de tomar decisiones morales, llevando a cabo así un auténtico contacto creativo con los ministerios espirituales asociados y con el Ajustador del Pensamiento interior.

\par
%\textsuperscript{(1218.8)}
\textsuperscript{111:2.10} El resultado inevitable de esta espiritualización, por contacto, de la mente humana es el nacimiento gradual de un alma\footnote{\textit{Nacimiento del alma}: Jn 3:3-10; 1 P 1:22-23.}, la progenitura conjunta de una mente ayudante dominada por una voluntad humana que anhela conocer a Dios, y que trabaja en unión con las fuerzas espirituales del universo que están bajo el supercontrol de un fragmento real del Dios mismo de toda la creación ---el Monitor de Misterio. Y así, la realidad material y mortal del yo trasciende las limitaciones temporales de la máquina de la vida física, y alcanza una nueva expresión y una nueva identificación en el vehículo evolutivo que deberá asegurar la continuidad de la individualidad: el alma morontial e inmortal.

\section*{3. El alma en evolución}
\par
%\textsuperscript{(1218.9)}
\textsuperscript{111:3.1} Los errores de la mente mortal y las equivocaciones de la conducta humana pueden retrasar notablemente la evolución del alma, aunque no pueden inhibir este fenómeno morontial una vez que ha sido iniciado por el Ajustador interior con el consentimiento de la voluntad de la criatura. Pero en cualquier momento anterior a la muerte física, esta misma voluntad material y humana tiene el poder de anular dicha elección y de rechazar la supervivencia. Incluso después de haber sobrevivido, el mortal ascendente conserva todavía esta prerrogativa de escoger rechazar la vida eterna; en cualquier momento antes de la fusión con el Ajustador, la criatura evolutiva y ascendente puede decidir abandonar la voluntad del Padre Paradisiaco. La fusión con el Ajustador señala el hecho de que el mortal ascendente ha elegido de manera eterna y sin reservas hacer la voluntad del Padre.

\par
%\textsuperscript{(1219.1)}
\textsuperscript{111:3.2} Durante la vida en la carne, el alma en evolución tiene la capacidad de reforzar las decisiones supermateriales de la mente mortal. Como es supermaterial, el alma no funciona por sí misma en el nivel material de la experiencia humana. Sin la colaboración de un espíritu de la Deidad, como el Ajustador, este alma subespiritual tampoco puede funcionar por encima del nivel morontial. El alma tampoco toma decisiones finales hasta que la muerte o el traslado la separan de su asociación material con la mente mortal, a menos que esta mente material delegue libre y voluntariamente dicha autoridad a su alma morontial con quien funciona de manera asociada. Durante la vida, la voluntad mortal, el poder de decisión y de elección de la personalidad, reside en los circuitos materiales de la mente; a medida que avanza el crecimiento del mortal en la Tierra, este yo, con sus inestimables poderes de elección, se identifica cada vez más con la entidad emergente del alma morontial; después de la muerte y de la resurrección en el mundo de las mansiones, la personalidad humana está completamente identificada con el yo morontial. El alma es así el embrión del futuro vehículo morontial de la identidad de la personalidad.

\par
%\textsuperscript{(1219.2)}
\textsuperscript{111:3.3} Este alma inmortal tiene al principio una naturaleza totalmente morontial, pero posee tal capacidad de desarrollo, que se eleva invariablemente hasta los verdaderos niveles espirituales que poseen un valor de fusión con los espíritus de la Deidad, habitualmente con el mismo espíritu del Padre Universal que desencadenó este fenómeno creativo en la mente de la criatura.

\par
%\textsuperscript{(1219.3)}
\textsuperscript{111:3.4} Tanto la mente humana como el Ajustador divino son conscientes de la presencia y de la naturaleza diferencial del alma en evolución ---el Ajustador lo es plenamente, y la mente parcialmente. El alma se vuelve cada vez más consciente de la mente y del Ajustador, como identidades asociadas, de manera proporcional a su propio crecimiento evolutivo. El alma comparte las cualidades de la mente humana y del espíritu divino, pero evoluciona constantemente hacia un acrecentamiento del control espiritual y del predominio divino mediante el fomento de una función mental cuyos significados tratan de coordinarse con los verdaderos valores espirituales.

\par
%\textsuperscript{(1219.4)}
\textsuperscript{111:3.5} La carrera mortal, la evolución del alma, es no tanto un período de prueba como un período de educación. La fe en la supervivencia de los valores supremos es el corazón de la religión; la experiencia religiosa auténtica consiste en unir los valores supremos y los significados cósmicos como una comprensión de la realidad universal.

\par
%\textsuperscript{(1219.5)}
\textsuperscript{111:3.6} La mente conoce la cantidad, la realidad, los significados. Pero la calidad ---los valores--- se \textit{siente}. Aquello que siente es la creación conjunta de la mente que conoce y del espíritu asociado que lo convierte en una realidad.

\par
%\textsuperscript{(1219.6)}
\textsuperscript{111:3.7} En la medida en que el alma morontial evolutiva del hombre se impregna de verdad, de belleza y de bondad como realización del valor de la conciencia de Dios, el ser resultante se vuelve indestructible. Si no existe ninguna supervivencia de los valores eternos en el alma evolutiva del hombre, entonces la existencia mortal no tiene sentido, y la vida misma es una trágica ilusión. Pero es eternamente cierto que aquello que empezáis en el tiempo, lo terminaréis ciertamente en la eternidad ---si vale la pena terminarlo.

\section*{4. La vida interior}
\par
%\textsuperscript{(1219.7)}
\textsuperscript{111:4.1} El reconocimiento es un proceso intelectual que consiste en encajar las impresiones sensoriales recibidas del mundo exterior en las configuraciones de la memoria del individuo. La comprensión implica que esas impresiones sensoriales reconocidas, y sus configuraciones de memoria asociadas, han sido integradas u organizadas en una red dinámica de principios.

\par
%\textsuperscript{(1220.1)}
\textsuperscript{111:4.2} Los significados proceden de la combinación del reconocimiento y de la comprensión. Los significados no existen en un mundo totalmente sensorial o material. Los significados y los valores sólo se perciben en las esferas interiores o supermateriales de la experiencia humana.

\par
%\textsuperscript{(1220.2)}
\textsuperscript{111:4.3} Todos los progresos de la verdadera civilización nacen en este mundo interior de la humanidad. Sólo la vida interior es realmente creativa. La civilización difícilmente puede progresar cuando la mayoría de la juventud de una generación cualquiera consagra sus intereses y sus energías a la persecución materialista del mundo sensorial o exterior.

\par
%\textsuperscript{(1220.3)}
\textsuperscript{111:4.4} El mundo interior y el mundo exterior tienen un conjunto de valores diferentes. Cualquier civilización está en peligro cuando las tres cuartas partes de su juventud se meten en profesiones materialistas y se dedican a buscar las actividades sensoriales del mundo exterior. La civilización está en peligro cuando la juventud deja de interesarse por la ética, la sociología, la eugenesia, la filosofía, las bellas artes, la religión y la cosmología.

\par
%\textsuperscript{(1220.4)}
\textsuperscript{111:4.5} Únicamente en los niveles superiores de la mente superconsciente, a medida que ésta incide en el ámbito espiritual de la experiencia humana, podréis encontrar aquellos conceptos superiores asociados a los modelos maestros eficaces que contribuirán a construir una civilización mejor y más duradera. La personalidad es intrínsecamente creativa, pero sólo funciona de esta manera en la vida interior del individuo.

\par
%\textsuperscript{(1220.5)}
\textsuperscript{111:4.6} Los cristales de nieve siempre tienen una forma hexagonal, pero nunca hay dos que sean iguales. Los niños se ajustan a los tipos, pero nunca hay dos que sean exactamente iguales, ni siquiera en el caso de los gemelos. La personalidad sigue unos tipos, pero siempre es única.

\par
%\textsuperscript{(1220.6)}
\textsuperscript{111:4.7} La felicidad y la alegría tienen su origen en la vida interior. No podéis experimentar una verdadera alegría completamente solos. Una vida solitaria es fatal para la felicidad. Incluso las familias y las naciones disfrutarán más de la vida si la comparten con las demás.

\par
%\textsuperscript{(1220.7)}
\textsuperscript{111:4.8} No podéis controlar por completo el mundo exterior ---el entorno. La creatividad del mundo interior es la que está más sujeta a vuestra dirección, porque vuestra personalidad se encuentra allí ampliamente liberada de las trabas de las leyes de la causalidad precedente. La personalidad lleva asociada una soberanía volitiva limitada.

\par
%\textsuperscript{(1220.8)}
\textsuperscript{111:4.9} Puesto que esta vida interior del hombre es verdaderamente creativa, cada persona tiene la responsabilidad de elegir si esta creatividad será espontánea y totalmente fortuita, o si estará controlada, dirigida y será constructiva. Una imaginación creativa, ¿cómo puede producir resultados valiosos, si el escenario sobre el que actúa ya está ocupado por los prejuicios, el odio, los miedos, los resentimientos, la venganza y los fanatismos?

\par
%\textsuperscript{(1220.9)}
\textsuperscript{111:4.10} Las ideas pueden tener su origen en los estímulos del mundo exterior, pero los ideales sólo nacen en los reinos creativos del mundo interior. Las naciones del mundo están dirigidas actualmente por hombres que tienen una superabundancia de ideas, pero que carecen de ideales. Ésta es la explicación de la pobreza, los divorcios, las guerras y los odios raciales.

\par
%\textsuperscript{(1220.10)}
\textsuperscript{111:4.11} El problema es el siguiente: si el hombre con libre albedrío está dotado en su fuero interno de los poderes de la creatividad, entonces tenemos que reconocer que la libre creatividad contiene el potencial de la libre destructividad. Y cuando la creatividad se orienta hacia la destructividad, os encontráis cara a cara con las devastaciones del mal y del pecado ---opresiones, guerras y destrucciones. El mal es una creatividad parcial que tiende hacia la desintegración y la destrucción final. Todo conflicto es malo en el sentido de que inhibe la función creativa de la vida interior ---es una especie de guerra civil en la personalidad.

\par
%\textsuperscript{(1221.1)}
\textsuperscript{111:4.12} La creatividad interior contribuye a ennoblecer el carácter mediante la integración de la personalidad y la unificación de la individualidad. Es eternamente cierto que el pasado es incambiable; sólo el futuro puede ser modificado mediante el ministerio de la creatividad del yo interior en el momento presente.

\section*{5. La consagración de la elección}
\par
%\textsuperscript{(1221.2)}
\textsuperscript{111:5.1} Hacer la voluntad de Dios es ni más ni menos que una manifestación de la buena voluntad de la criatura por compartir su vida interior con Dios ---con el mismo Dios que ha hecho posible la vida de esa criatura con sus valores y significados interiores. Compartir es parecerse a Dios ---es divino. Dios lo comparte todo con el Hijo Eterno y el Espíritu Infinito, y éstos a su vez comparten todas las cosas con los Hijos divinos y las Hijas espirituales de los universos.

\par
%\textsuperscript{(1221.3)}
\textsuperscript{111:5.2} Imitar a Dios es la clave de la perfección; hacer su voluntad es el secreto de la supervivencia y de la perfección en la supervivencia.

\par
%\textsuperscript{(1221.4)}
\textsuperscript{111:5.3} Los mortales viven en Dios, y por eso Dios ha querido vivir en los mortales. Al igual que los hombres confían en él, él ha confiado ---el primero--- una parte de sí mismo para que esté con los hombres; ha consentido en vivir en los hombres y en habitar en los hombres, sometido a la voluntad humana.

\par
%\textsuperscript{(1221.5)}
\textsuperscript{111:5.4} La paz en esta vida, la supervivencia en la muerte, la perfección en la próxima vida, el servicio en la eternidad ---todo esto se logra \textit{desde ahora} (en espíritu) cuando la personalidad de la criatura consiente--- elige ---someter su voluntad a la voluntad del Padre. El Padre ya ha elegido someter un fragmento de sí mismo a la voluntad de la personalidad de la criatura.

\par
%\textsuperscript{(1221.6)}
\textsuperscript{111:5.5} Esta elección de la criatura no supone un abandono de la voluntad. Es una consagración de la voluntad, una expansión de la voluntad, una glorificación de la voluntad, un perfeccionamiento de la voluntad; una elección así eleva la voluntad de la criatura desde el nivel de los significados temporales hasta ese estado superior en el que la personalidad del hijo creado comulga con la personalidad del Padre espíritu.

\par
%\textsuperscript{(1221.7)}
\textsuperscript{111:5.6} Este hecho de elegir la voluntad del Padre es el descubrimiento espiritual del Padre espíritu por parte del hombre mortal, aunque tenga que transcurrir una era antes de que el hijo creado pueda estar verdaderamente delante de la presencia real de Dios en el Paraíso. Esta elección no consiste tanto en la negación de la voluntad de la criatura ---<<Que no se haga mi voluntad sino la tuya>>\footnote{\textit{Que no se haga mi voluntad sino la de Dios}: Sal 143:10; Eclo 15:11-20; Mt 6:10; 7:21; 12:50; 26:39,42,44; Mc 3:35; 14:36.39; Lc 8:21; 11:2; 22:42; Jn 4:34; 5:30; 6:38-40; 7:16-17; 9:31; 14:21-24; 15:10,14-16; 17:4.}--- sino más bien en la afirmación categórica de la criatura: <<Es \textit{mi} voluntad que se haga \textit{tu} voluntad>>. Si hace esta elección, el hijo que ha escogido a Dios encontrará tarde o temprano la unión interior (la fusión) con el fragmento de Dios que vive en él, mientras que este mismo hijo que se perfecciona encontrará la satisfacción suprema de la personalidad en la comunión adoradora entre la personalidad del hombre y la personalidad de su Hacedor, dos personalidades cuyos atributos creativos se han unido eternamente en una reciprocidad de expresión deseada ---el nacimiento de una asociación eterna más entre la voluntad del hombre y la voluntad de Dios.

\section*{6. La paradoja humana}
\par
%\textsuperscript{(1221.8)}
\textsuperscript{111:6.1} Muchas dificultades temporales del hombre mortal proceden de su doble relación con el cosmos. El hombre es una parte de la naturaleza ---existe en la naturaleza--- y sin embargo es capaz de trascenderla. El hombre es finito, pero está habitado por una chispa de la infinidad. Esta situación dual no solamente proporciona el potencial para el mal, sino que engendra también numerosas situaciones sociales y morales cargadas de muchas incertidumbres y de no pocas inquietudes.

\par
%\textsuperscript{(1222.1)}
\textsuperscript{111:6.2} La valentía que se necesita para llevar a cabo la conquista de la naturaleza y para trascenderse a sí mismo es una valentía que puede sucumbir a las tentaciones del orgullo. El mortal que puede trascender su yo podría ceder a la tentación de deificar su propia conciencia de sí mismo. El dilema mortal consiste en el doble hecho de que el hombre está esclavizado a la naturaleza, mientras que al mismo tiempo posee una libertad única ---la libertad de elegir y de actuar espiritualmente. En los niveles materiales, el hombre se encuentra subordinado a la naturaleza, mientras que en los niveles espirituales triunfa sobre la naturaleza y sobre todas las cosas temporales y finitas. Esta paradoja es inseparable de las tentaciones, del mal potencial, de los errores de decisión, y cuando el yo se vuelve orgulloso y arrogante, el pecado puede aparecer.

\par
%\textsuperscript{(1222.2)}
\textsuperscript{111:6.3} El problema del pecado no existe por sí mismo en el mundo finito. El hecho de ser finito no es malo ni pecaminoso. El mundo finito ha sido hecho por un Creador infinito ---es la obra de sus Hijos divinos--- y por lo tanto debe ser \textit{bueno}\footnote{\textit{Bueno}: Gn 1:31; Sal 19:1.}. Lo que da origen al mal y al pecado es el mal uso, la deformación y la desnaturalización de lo finito.

\par
%\textsuperscript{(1222.3)}
\textsuperscript{111:6.4} El espíritu puede dominar la mente; del mismo modo, la mente puede controlar la energía. Pero la mente sólo puede controlar la energía mediante su propia manipulación inteligente de los potenciales metamórficos inherentes al nivel matemático de las causas y los efectos de los dominios físicos. La mente de la criatura no controla de manera inherente la energía; esto es una prerrogativa de la Deidad. Pero la mente de la criatura puede manipular la energía, y lo hace de hecho, en la medida exacta en que ha llegado a dominar los secretos energéticos del mundo físico.

\par
%\textsuperscript{(1222.4)}
\textsuperscript{111:6.5} Cuando el hombre desea modificar la realidad física, ya se trate de él mismo o de su entorno, lo consigue en la medida en que ha descubierto los caminos y los medios de controlar la materia y de dirigir la energía. La mente sin ayuda es impotente para influir sobre algo material, salvo sobre su propio mecanismo físico, con el que se encuentra inevitablemente vinculada. Pero mediante el empleo inteligente del mecanismo corporal, la mente puede crear otros mecanismos, e incluso relaciones energéticas y relaciones vivientes, y al utilizarlos, esta mente puede controlar cada vez más, e incluso dominar, su nivel físico en el universo.

\par
%\textsuperscript{(1222.5)}
\textsuperscript{111:6.6} La ciencia es la fuente de los hechos, y la mente no puede trabajar sin los hechos. En la construcción de la sabiduría, los hechos son los ladrillos que están colocados con el cemento de la experiencia de la vida. El hombre puede encontrar el amor de Dios sin los hechos, y el hombre puede descubrir las leyes de Dios sin el amor, pero el hombre nunca puede empezar a apreciar la simetría infinita, la armonía celestial, la exquisita plenitud de la naturaleza inclusiva de la Fuente-Centro Primera, hasta que no ha encontrado la ley divina y el amor divino y los ha unificado experiencialmente en su propia filosofía cósmica en evolución.

\par
%\textsuperscript{(1222.6)}
\textsuperscript{111:6.7} La expansión de los conocimientos materiales permite una mayor apreciación intelectual de los significados de las ideas y de los valores de los ideales. Un ser humano puede encontrar la verdad en su experiencia interior, pero necesita un claro conocimiento de los hechos para aplicar su descubrimiento personal de la verdad a las exigencias implacablemente prácticas de la vida diaria.

\par
%\textsuperscript{(1222.7)}
\textsuperscript{111:6.8} Es muy natural que el hombre mortal se sienta acosado por sentimientos de inseguridad cuando se ve inextricablemente atado a la naturaleza, mientras que posee unos poderes espirituales que trascienden por completo todas las cosas temporales y finitas. Sólo la confianza religiosa ---la fe viviente--- puede sostener al hombre en medio de estos problemas difíciles y desconcertantes.

\par
%\textsuperscript{(1223.1)}
\textsuperscript{111:6.9} De todos los peligros que acechan a la naturaleza mortal del hombre y ponen en peligro su integridad espiritual, el orgullo es el peor. La intrepidez es valerosa, pero el egotismo es vanaglorioso y suicida. Una confianza razonable en sí mismo no es deplorable. La capacidad del hombre para trascenderse es la única cosa que lo distingue del reino animal.

\par
%\textsuperscript{(1223.2)}
\textsuperscript{111:6.10} El orgullo es engañoso, embriagador, y engendra el pecado, ya sea en un individuo, un grupo, una raza o una nación. Es literalmente cierto que <<el orgullo precede a la caída>>\footnote{\textit{El orgullo precede a la caída}: Pr 16:18.}.

\section*{7. El problema del Ajustador}
\par
%\textsuperscript{(1223.3)}
\textsuperscript{111:7.1} La incertidumbre en la seguridad\footnote{\textit{Incertidumbre en la seguridad}: Mc 9:24; Ro 4:20; 11:20-33.} es la esencia de la aventura hacia el Paraíso ---incertidumbre en el tiempo y en la mente, incertidumbre en cuanto a los acontecimientos del desarrollo de la ascensión hacia el Paraíso; seguridad en espíritu y en la eternidad, seguridad en la confianza sin reserva del hijo creado en la compasión divina y en el amor infinito del Padre Universal; incertidumbre como ciudadano inexperto del universo; seguridad como hijo ascendente en las mansiones universales de un Padre infinitamente poderoso, sabio y amoroso.

\par
%\textsuperscript{(1223.4)}
\textsuperscript{111:7.2} ¿Puedo exhortaros a que prestéis atención al eco lejano de la llamada fiel que el Ajustador hace a vuestra alma? El Ajustador interior no puede detener ni tampoco cambiar materialmente las luchas de vuestra carrera en el tiempo; el Ajustador no puede disminuir las dificultades de la vida mientras viajáis a través de este mundo de trabajo penoso. El habitante divino sólo puede abstenerse pacientemente mientras libráis la batalla de la vida tal como ésta se vive en vuestro planeta; pero a medida que trabajáis y os preocupáis, lucháis y os afanáis, podríais permitir ---si tan sólo quisierais--- que el valiente Ajustador luchara con vosotros y por vosotros. Podríais sentiros tan reconfortados e inspirados, tan cautivados e intrigados, si tan sólo permitierais que el Ajustador os presentara constantemente las imágenes del verdadero motivo, de la meta final y del objetivo eterno de esta lucha difícil y penosa con los problemas corrientes de vuestro mundo material actual.

\par
%\textsuperscript{(1223.5)}
\textsuperscript{111:7.3} ¿Por qué no ayudáis al Ajustador en la tarea de mostraros la contrapartida espiritual de todos estos intensos esfuerzos materiales? ¿Por qué no permitís que el Ajustador os fortalezca con las verdades espirituales del poder cósmico, mientras lucháis contra las dificultades temporales de la existencia de las criaturas? ¿Por qué no incitáis al ayudante celestial a que os reconforte con la clara visión del panorama eterno de la vida universal, mientras contempláis con perplejidad los problemas del momento que pasa? ¿Por qué os negáis a ser iluminados e inspirados por el punto de vista del universo, mientras os afanáis en medio de los obstáculos del tiempo y camináis con dificultad por el laberinto de las incertidumbres que asaltan vuestro viaje por la vida mortal? ¿Por qué no permitís que el Ajustador espiritualice vuestros pensamientos, aunque vuestros pies tengan que caminar por los senderos materiales de los esfuerzos terrestres?

\par
%\textsuperscript{(1223.6)}
\textsuperscript{111:7.4} Las razas humanas superiores de Urantia están mezcladas de manera compleja; son una combinación de numerosas razas y linajes de orígenes diferentes. Esta naturaleza compuesta hace que a los Monitores les resulte extremadamente difícil trabajar con eficacia durante la vida, y aumenta claramente los problemas del Ajustador y del serafín guardián después de la muerte. No hace mucho tiempo me hallaba en Salvington, y escuché a un guardián del destino presentar una declaración formal para excusar las dificultades que había encontrado mientras servía a su sujeto mortal. Este serafín decía:

\par
%\textsuperscript{(1223.7)}
\textsuperscript{111:7.5} <<Una gran parte de mis dificultades se debían al conflicto interminable entre las dos naturalezas de mi sujeto: la indolencia animal oponiéndose al impulso de la ambición; los ideales de un pueblo superior contrariados por los instintos de una raza inferior; los objetivos elevados de una gran mente neutralizados por el impulso de una herencia primitiva; la visión a largo plazo de un Monitor previsor contrarrestada por la miopía de una criatura del tiempo; los planes progresivos de un ser ascendente modificados por los deseos y los anhelos de una naturaleza material; los destellos de la inteligencia universal anulados por los mandatos energético-químicos de la raza en evolución; las emociones de un animal oponiéndose al impulso de los ángeles; el entrenamiento de un intelecto anulado por las tendencias del instinto; las tendencias acumuladas de la raza oponiéndose a la experiencia del individuo; las metas de lo mejor eclipsadas por los objetivos de lo peor; el vuelo de la genialidad neutralizado por la gravedad de la mediocridad; el progreso de lo bueno retrasado por la inercia de lo malo; el arte de lo hermoso manchado por la presencia del mal; el empuje de la salud neutralizado por la debilidad de la enfermedad; la fuente de la fe contaminada por los venenos del miedo; el manantial de la alegría envenenado por las aguas de la tristeza; la felicidad de la anticipación desilusionada por la amargura de la realización; las alegrías de la vida siempre amenazadas por las tristezas de la muerte. ¡Qué vida y en qué planeta! Y sin embargo, debido a la ayuda y al impulso siempre presentes del Ajustador del Pensamiento, este alma ha alcanzado un buen grado de felicidad y de éxito, y ya ha ascendido a las salas de juicio de mansonia>>.

\par
%\textsuperscript{(1224.1)}
\textsuperscript{111:7.6} [Presentado por un Mensajero Solitario de Orvonton.]


\chapter{Documento 112. La supervivencia de la personalidad}
\par
%\textsuperscript{(1225.1)}
\textsuperscript{112:0.1} LOS PLANETAS evolutivos son las esferas de origen de los hombres, los mundos iniciales de la carrera humana ascendente. Urantia es vuestro punto de partida; aquí es donde os juntáis con vuestro divino Ajustador del Pensamiento en una unión temporal. Habéis sido dotados de un guía perfecto; así pues, si queréis participar sinceramente en la carrera del tiempo y alcanzar la meta final de la fe, la recompensa de los siglos será vuestra; os uniréis eternamente con vuestro Ajustador interior. Entonces empezará vuestra vida real, la vida ascendente, de la cual vuestro presente estado mortal no es más que el preludio. Entonces comenzará vuestra misión sublime y progresiva como finalitarios en la eternidad que se despliega ante vosotros. Durante todas estas épocas y etapas sucesivas de crecimiento evolutivo, una parte de vosotros permanece absolutamente inalterable, y es la personalidad ---la permanencia en presencia del cambio.

\par
%\textsuperscript{(1225.2)}
\textsuperscript{112:0.2} Aunque sería presuntuoso intentar definir la personalidad, puede resultar útil recordar algunas cosas que se conocen sobre ella:

\par
%\textsuperscript{(1225.3)}
\textsuperscript{112:0.3} 1. La personalidad es esa cualidad, dentro de la realidad, que es otorgada por el mismo Padre Universal, o por el Actor Conjunto actuando en nombre del Padre.

\par
%\textsuperscript{(1225.4)}
\textsuperscript{112:0.4} 2. Puede ser atribuida a cualquier sistema energético viviente que contenga la mente o el espíritu.

\par
%\textsuperscript{(1225.5)}
\textsuperscript{112:0.5} 3. No está totalmente sometida a las trabas de la causalidad antecedente. Es relativamente creativa o cocreativa.

\par
%\textsuperscript{(1225.6)}
\textsuperscript{112:0.6} 4. Cuando se concede a las criaturas materiales evolutivas, hace que el espíritu se esfuerce por dominar la energía-materia por mediación de la mente.

\par
%\textsuperscript{(1225.7)}
\textsuperscript{112:0.7} 5. Aunque está desprovista de identidad, la personalidad puede unificar la identidad de cualquier sistema energético viviente.

\par
%\textsuperscript{(1225.8)}
\textsuperscript{112:0.8} 6. Su reacción al circuito de la personalidad sólo es cualitativa, en contraste con las tres energías que muestran una reacción cualitativa y cuantitativa a la gravedad.

\par
%\textsuperscript{(1225.9)}
\textsuperscript{112:0.9} 7. La personalidad es invariable en presencia del cambio.

\par
%\textsuperscript{(1225.10)}
\textsuperscript{112:0.10} 8. Puede hacer un regalo a Dios ---dedicar su libre albedrío a hacer la voluntad de Dios.

\par
%\textsuperscript{(1225.11)}
\textsuperscript{112:0.11} 9. Está caracterizada por la moralidad ---la conciencia de la relatividad de las relaciones con otras personas. Discierne los niveles de comportamiento, y discrimina selectivamente entre ellos.

\par
%\textsuperscript{(1225.12)}
\textsuperscript{112:0.12} 10. La personalidad es única, absolutamente única: es única en el tiempo y en el espacio; es única en la eternidad y en el Paraíso; es única cuando es otorgada ---no existen copias de ella; es única durante todos los momentos de la existencia; es única con respecto a Dios ---que no hace acepción de personas\footnote{\textit{Dios no hace acepción de personas}: 2 Cr 19:7; Job 34:19; Eclo 35:12; Hch 10:24; Ro 2:11; Gl 2:6; 3:28; Ef 6:9; Col 3:11.}, pero que tampoco las suma, porque no son adicionables ---son asociables, pero no totalizables.

\par
%\textsuperscript{(1226.1)}
\textsuperscript{112:0.13} 11. La personalidad reacciona directamente a la presencia de otra personalidad.

\par
%\textsuperscript{(1226.2)}
\textsuperscript{112:0.14} 12. Es un elemento que puede ser añadido al espíritu, ilustrando así la primacía del Padre con respecto al Hijo. (No es necesario añadir la mente al espíritu).

\par
%\textsuperscript{(1226.3)}
\textsuperscript{112:0.15} 13. La personalidad puede sobrevivir a la muerte física con la identidad que se encuentra en el alma sobreviviente. El Ajustador y la personalidad son invariables; la relación entre ambos (en el alma) no es más que cambio, evolución continua; y si este cambio (el crecimiento) cesara, el alma dejaría de existir.

\par
%\textsuperscript{(1226.4)}
\textsuperscript{112:0.16} 14. La personalidad tiene una conciencia única del tiempo, que es diferente a la percepción que la mente o el espíritu tienen del mismo.

\section*{1. La personalidad y la realidad}
\par
%\textsuperscript{(1226.5)}
\textsuperscript{112:1.1} El Padre Universal confiere la personalidad a sus criaturas como un don potencialmente eterno. Este don divino está diseñado para funcionar en numerosos niveles y en situaciones universales sucesivas que se extienden desde el finito más humilde hasta el absonito más elevado, e incluso hasta las fronteras del absoluto. La personalidad actúa así en tres planos cósmicos o en tres fases del universo:

\par
%\textsuperscript{(1226.6)}
\textsuperscript{112:1.2} 1. \textit{El estado de situación}. La personalidad ejerce su actividad con la misma eficacia en el universo local, en el superuniverso y en el universo central.

\par
%\textsuperscript{(1226.7)}
\textsuperscript{112:1.3} 2. \textit{El estado de significado}. La personalidad actúa eficazmente en los niveles de lo finito, lo absonito e incluso incide en lo absoluto.

\par
%\textsuperscript{(1226.8)}
\textsuperscript{112:1.4} 3. \textit{El estado de valor}. La personalidad se puede realizar experiencialmente en los reinos progresivos de lo material, lo morontial y lo espiritual.

\par
%\textsuperscript{(1226.9)}
\textsuperscript{112:1.5} La personalidad posee un campo de acción perfeccionado de dimensiones cósmicas. La personalidad finita tiene tres dimensiones que funcionan más o menos como sigue:

\par
%\textsuperscript{(1226.10)}
\textsuperscript{112:1.6} 1. \textit{La longitud} representa la dirección y la naturaleza del progreso ---el movimiento a través del espacio y de acuerdo con el tiempo--- la evolución.

\par
%\textsuperscript{(1226.11)}
\textsuperscript{112:1.7} 2. La profundidad \textit{vertical} abarca los impulsos y las actitudes del organismo, los niveles variables de autorrealización y el fenómeno general de reacción al entorno.

\par
%\textsuperscript{(1226.12)}
\textsuperscript{112:1.8} 3. \textit{La anchura} abarca el ámbito de la coordinación, la asociación y la organización de la individualidad.

\par
%\textsuperscript{(1226.13)}
\textsuperscript{112:1.9} El tipo de personalidad otorgado a los mortales de Urantia posee un potencial de siete dimensiones de expresión del yo, o de realización de la persona. Estos fenómenos dimensionales son comprensibles a razón de tres en el nivel finito, tres en el nivel absonito y uno en el nivel absoluto. En los niveles subabsolutos, esta séptima dimensión, o dimensión de totalidad, puede ser experimentada como el \textit{hecho} de la personalidad. Esta dimensión suprema es un absoluto asociable y, aunque no es infinita, posee un potencial dimensional que permite la penetración subinfinita de lo absoluto.

\par
%\textsuperscript{(1226.14)}
\textsuperscript{112:1.10} Las dimensiones finitas de la personalidad están relacionadas con la longitud, la profundidad y la anchura cósmicas. La longitud indica el significado; la profundidad señala el valor; y la anchura abarca la perspicacia ---la capacidad de experimentar una conciencia indiscutible de la realidad cósmica.

\par
%\textsuperscript{(1227.1)}
\textsuperscript{112:1.11} En el nivel morontial, todas estas dimensiones finitas del nivel material se encuentran muy realzadas, y se pueden realizar ciertos nuevos valores dimensionales. Todas estas experiencias dimensionales ampliadas del nivel morontial están maravillosamente articuladas con la dimensión suprema, o dimensión de la personalidad, gracias a la influencia de la mota y también a causa de la contribución de las matemáticas morontiales.

\par
%\textsuperscript{(1227.2)}
\textsuperscript{112:1.12} Muchas dificultades que experimentan los mortales en su estudio de la personalidad humana se podrían evitar si la criatura finita recordara que los niveles dimensionales y los niveles espirituales no están coordinados en la realización experiencial de la personalidad.

\par
%\textsuperscript{(1227.3)}
\textsuperscript{112:1.13} La vida es en realidad un proceso que tiene lugar entre el organismo (la individualidad) y su entorno. La personalidad comunica un valor de identidad y unos significados de continuidad a esta asociación entre un organismo y su entorno. Se reconocerá así que el fenómeno de la reacción a los estímulos no es un simple proceso mecánico, puesto que la personalidad actúa como factor en la situación total. Es permanentemente cierto que los mecanismos son pasivos de forma innata, y los organismos inherentemente activos.

\par
%\textsuperscript{(1227.4)}
\textsuperscript{112:1.14} La vida física es un proceso que tiene lugar, no tanto dentro del organismo, como \textit{entre} el organismo y el entorno. Todo proceso de este tipo tiende a crear y a establecer unos modelos de reacción del organismo a ese entorno. Todos estos \textit{modelos directivos} ejercen una gran influencia en la elección de la meta.

\par
%\textsuperscript{(1227.5)}
\textsuperscript{112:1.15} El yo y el entorno establecen un contacto significativo por mediación de la mente. La capacidad y la buena disposición del organismo para efectuar estos contactos significativos con el entorno (para reaccionar a los estímulos) representa la \textit{actitud} de toda la personalidad.

\par
%\textsuperscript{(1227.6)}
\textsuperscript{112:1.16} La personalidad no puede trabajar muy bien cuando está aislada. El hombre es de manera innata una criatura sociable; está dominado por el ardiente deseo de la pertenencia. Es literalmente cierto que <<ningún hombre vive para sí mismo>>\footnote{\textit{Ningún hombre vive para sí mismo}: Ro 14:7.}.

\par
%\textsuperscript{(1227.7)}
\textsuperscript{112:1.17} Pero el concepto de la personalidad, en el sentido de la totalidad de la criatura que vive y actúa, significa mucho más que la integración de unas relaciones; significa la \textit{unificación} de todos los factores de la realidad, así como la coordinación de las relaciones. Entre dos objetos existen relaciones, pero tres objetos o más producen un \textit{sistema}, y este sistema representa mucho más que una relación ampliada o compleja. Esta distinción es fundamental, porque en un sistema cósmico los miembros individuales no están conectados entre sí salvo en relación con el todo, y gracias a la individualidad de ese todo.

\par
%\textsuperscript{(1227.8)}
\textsuperscript{112:1.18} En el organismo humano, la suma de las partes constituye el yo ---la individualidad--- pero este proceso no tiene absolutamente nada que ver con la personalidad, que unifica todos estos factores en sus relaciones con las realidades cósmicas.

\par
%\textsuperscript{(1227.9)}
\textsuperscript{112:1.19} En los conjuntos, las partes están sumadas; en los sistemas, las partes \textit{estánpuestas en orden}. Los sistemas son significativos debido a su organización ---a sus valores de posición. En un buen sistema, todos los factores están en posición cósmica. En un mal sistema, hay algo que falta o que está desplazado ---desordenado. En el sistema humano, la personalidad es la que unifica todas las actividades y comunica a la vez las cualidades de identidad y de creatividad.

\section*{2. El yo}
\par
%\textsuperscript{(1227.10)}
\textsuperscript{112:2.1} En el estudio de la individualidad, sería útil recordar:

\par
%\textsuperscript{(1227.11)}
\textsuperscript{112:2.2} 1. Que los sistema físicos están subordinados.

\par
%\textsuperscript{(1227.12)}
\textsuperscript{112:2.3} 2. Que los sistemas intelectuales están coordinados.

\par
%\textsuperscript{(1227.13)}
\textsuperscript{112:2.4} 3. Que la personalidad está superordenada.

\par
%\textsuperscript{(1227.14)}
\textsuperscript{112:2.5} 4. Que la fuerza espiritual interior es potencialmente directiva.

\par
%\textsuperscript{(1228.1)}
\textsuperscript{112:2.6} En todos los conceptos sobre la individualidad se debería reconocer que el hecho de la vida viene en primer lugar, y que su evaluación o interpretación viene después. El niño humano primero \textit{vive}, y posteriormente \textit{reflexiona} sobre su vida. En la economía cósmica, la perspicacia precede a la previsión.

\par
%\textsuperscript{(1228.2)}
\textsuperscript{112:2.7} El hecho universal de Dios volviéndose hombre ha cambiado para siempre todos los significados y ha alterado todos los valores de la personalidad humana. En el verdadero sentido de la palabra, el amor implica una estima mutua entre personalidades completas, ya sean humanas o divinas, o humanas \textit{y} divinas. Las partes componentes del yo pueden funcionar de numerosas maneras ---pensando, sintiendo, deseando--- pero sólo los atributos coordinados de la personalidad completa están enfocados hacia una acción inteligente; y todos estos poderes están asociados con la dotación espiritual de la mente mortal cuando un ser humano ama sincera y desinteresadamente a otro ser, ya sea humano o divino.

\par
%\textsuperscript{(1228.3)}
\textsuperscript{112:2.8} Todos los conceptos humanos sobre la realidad están basados en la suposición de que la personalidad humana es real; todos los conceptos sobre las realidades superhumanas están basados en la experiencia de la personalidad humana con, y en, las realidades cósmicas de ciertas entidades espirituales y personalidades divinas asociadas. Todo lo que no es espiritual en la experiencia humana, salvo la personalidad, es un medio para conseguir un fin. Toda verdadera relación del hombre mortal con otras personas ---humanas o divinas--- es un fin en sí misma. Y una comunión así con la personalidad de la Deidad es la meta eterna de la ascensión por el universo.

\par
%\textsuperscript{(1228.4)}
\textsuperscript{112:2.9} La posesión de una personalidad identifica al hombre como un ser espiritual, puesto que la unidad de la individualidad y la conciencia de tener una personalidad son dones del mundo supermaterial. El hecho mismo de que un mortal materialista pueda negar la existencia de las realidades supermateriales demuestra, en sí mismo y por sí mismo, que la síntesis espiritual y la conciencia cósmica están presentes y funcionando en su mente humana.

\par
%\textsuperscript{(1228.5)}
\textsuperscript{112:2.10} Existe un gran abismo cósmico entre la materia y el pensamiento, y este abismo es inconmensurablemente mayor entre la mente material y el amor espiritual. La conciencia, y mucho menos la conciencia de sí mismo, no puede ser explicada por ninguna teoría de asociación electrónica mecánica ni por ningún fenómeno de energía materialista.

\par
%\textsuperscript{(1228.6)}
\textsuperscript{112:2.11} A medida que la mente persigue la realidad hasta su análisis final, la materia desaparece para los sentidos materiales, pero puede seguir siendo real para la mente. Cuando la perspicacia espiritual persigue esta realidad que permanece después de desaparecer la materia, y la persigue hasta su análisis final, esta realidad desaparece para la mente, pero la perspicacia del espíritu puede percibir todavía unas realidades cósmicas y unos valores supremos de naturaleza espiritual. Por consiguiente, la ciencia cede el paso a la filosofía, mientras que la filosofía debe rendirse ante las conclusiones inherentes a la experiencia espiritual auténtica. El pensamiento se rinde ante la sabiduría, y la sabiduría se pierde en una adoración iluminada y reflexiva.

\par
%\textsuperscript{(1228.7)}
\textsuperscript{112:2.12} En la ciencia, el yo humano observa el mundo material; la filosofía es la observación de esta observación del mundo material; la religión, la verdadera experiencia espiritual, es la comprensión experiencial de la realidad cósmica de la observación de la observación de toda esta síntesis relativa de los elementos energéticos del tiempo y del espacio. Construir una filosofía sobre el universo basada exclusivamente en el materialismo es ignorar el hecho de que todas las cosas materiales son concebidas inicialmente como reales en la experiencia de la conciencia humana. El observador no puede ser la cosa observada; la evaluación necesita que se trascienda un poco a la cosa evaluada.

\par
%\textsuperscript{(1228.8)}
\textsuperscript{112:2.13} En el tiempo, el pensamiento conduce a la sabiduría y la sabiduría conduce a la adoración; en la eternidad, la adoración conduce a la sabiduría, y la sabiduría conduce a la finalidad del pensamiento.

\par
%\textsuperscript{(1229.1)}
\textsuperscript{112:2.14} La posibilidad de unificar el yo en evolución es inherente a las cualidades de sus factores constitutivos, que son: las energías básicas, los tejidos principales, el supercontrol químico fundamental, las ideas supremas, los móviles supremos, las metas supremas y el espíritu divino otorgado desde el Paraíso ---el secreto de la conciencia de la naturaleza espiritual del hombre.

\par
%\textsuperscript{(1229.2)}
\textsuperscript{112:2.15} La finalidad de la evolución cósmica consiste en alcanzar la unidad de la personalidad mediante el dominio creciente del espíritu, una reacción volitiva a las enseñanzas y directrices del Ajustador del Pensamiento. La personalidad, tanto humana como superhumana, está caracterizada por una cualidad cósmica inherente que podríamos llamar <<la evolución del dominio>>, la expansión del control sobre sí mismo y sobre el entorno.

\par
%\textsuperscript{(1229.3)}
\textsuperscript{112:2.16} Una personalidad ascendente, en otro tiempo humana, pasa por dos grandes fases de dominio volitivo creciente sobre el yo y en el universo:

\par
%\textsuperscript{(1229.4)}
\textsuperscript{112:2.17} 1. La experiencia prefinalitaria, o de la búsqueda de Dios, consistente en acrecentar la autorrealización mediante una técnica de expansión y de manifestación de la identidad, junto con la solución de los problemas cósmicos y el consiguiente dominio del universo.

\par
%\textsuperscript{(1229.5)}
\textsuperscript{112:2.18} 2. La experiencia postfinalitaria, o para revelar a Dios, en la que la autorrealización experimenta una expansión creativa mediante la revelación del Ser Supremo experiencial a las inteligencias que buscan a Dios, pero que aún no han alcanzado los niveles divinos en que son semejantes a Dios.

\par
%\textsuperscript{(1229.6)}
\textsuperscript{112:2.19} Las personalidades descendentes pasan por experiencias análogas durante sus diversas aventuras en el universo a medida que tratan de aumentar su capacidad para averiguar y ejecutar las voluntades divinas de las Deidades Suprema, Última y Absoluta.

\par
%\textsuperscript{(1229.7)}
\textsuperscript{112:2.20} Durante la vida física, el yo material, la entidad-ego de la identidad humana, depende del funcionamiento continuo del vehículo vital material, de la existencia continua del equilibrio inestable entre las energías y el intelecto, a lo que se le ha dado el nombre de \textit{vida} en Urantia. Pero la individualidad con valor de supervivencia, la individualidad que puede trascender la experiencia de la muerte, sólo evoluciona efectuando un traslado potencial de la sede de la identidad de la personalidad evolutiva desde el vehículo transitorio de la vida ---el cuerpo material--- hasta el alma morontial de naturaleza más duradera e inmortal, y luego más allá, hasta aquellos niveles en que el alma se impregna de la realidad espiritual y alcanza finalmente el estado de una realidad espiritual. Este traslado efectivo desde una asociación material hasta una identificación morontial se lleva a cabo mediante la sinceridad, la perseverancia y la firmeza de las decisiones de la criatura humana que busca a Dios.

\section*{3. El fenómeno de la muerte}
\par
%\textsuperscript{(1229.8)}
\textsuperscript{112:3.1} Los urantianos sólo reconocen en general un tipo de muerte, el cese físico de las energías vitales; pero en lo que se refiere a la supervivencia de la personalidad, existen en realidad tres tipos de muerte:

\par
%\textsuperscript{(1229.9)}
\textsuperscript{112:3.2} 1. \textit{La muerte espiritual (del alma)}. Si el hombre mortal rechaza la supervivencia, y cuando la ha rechazado definitivamente, cuando ha sido declarado espiritualmente insolvente, morontialmente en quiebra, según la opinión conjunta del Ajustador y del serafín de la supervivencia, cuando este informe coordinado ha sido registrado en Uversa, y después de que los Censores y sus asociados reflectantes han verificado estas conclusiones, los gobernantes de Orvonton ordenan la liberación inmediata del Monitor interior. Pero esta puesta en libertad del Ajustador no afecta de ninguna manera a los deberes del serafín personal o colectivo que se ocupa de ese individuo abandonado por el Ajustador. Este tipo de muerte tiene un significado definitivo, independientemente de la continuación temporal de las energías vivientes de los mecanismos físicos y mentales. Desde el punto de vista cósmico, el interesado ya está muerto; la continuación de su vida indica simplemente la persistencia del impulso material de las energías cósmicas.

\par
%\textsuperscript{(1230.1)}
\textsuperscript{112:3.3} 2. \textit{La muerte intelectual (de la mente)}. Cuando los circuitos vitales del ministerio ayudante superior se rompen debido a las aberraciones del intelecto o a causa de la destrucción parcial del mecanismo cerebral, y si estas condiciones sobrepasan cierto punto crítico irreparable, el Ajustador interior es liberado inmediatamente y parte hacia Divinington. En los archivos del universo se considera que una personalidad mortal ha encontrado la muerte cuando los circuitos mentales esenciales de la acción volitiva humana se han destruido. Esto también es la muerte, independientemente de que el mecanismo viviente del cuerpo físico continúe funcionando. El cuerpo menos la mente volitiva ya no es humano, pero el alma de dicho individuo puede sobrevivir de acuerdo con la elección anterior de su voluntad humana.

\par
%\textsuperscript{(1230.2)}
\textsuperscript{112:3.4} 3. \textit{La muerte física (del cuerpo y de la mente)}. Cuando la muerte sorprende a un ser humano, el Ajustador permanece en la ciudadela de la mente hasta que ésta deja de funcionar como mecanismo inteligente, aproximadamente en el momento en que las energías medibles del cerebro detienen sus pulsaciones rítmicas vitales. Después de esta disolución, el Ajustador se despide de la mente en vías de desaparición con tan poca ceremonia como había entrado en ella años atrás, y se dirige a Divinington pasando por Uversa.

\par
%\textsuperscript{(1230.3)}
\textsuperscript{112:3.5} Después de la muerte, el cuerpo material regresa al mundo elemental del cual provenía\footnote{\textit{Polvo al polvo}: Gn 3:19; Job 17:14-16; 34:15; Ec 3:20.}, pero dos factores no materiales de la personalidad sobreviviente permanecen: el Ajustador del Pensamiento preexistente, con la transcripción de la memoria de la carrera mortal, que se dirige a Divinington; y también subsiste el alma morontial inmortal del humano fallecido, que permanece bajo la custodia del guardián del destino. Estas fases y aspectos del alma, estas fórmulas de la identidad anteriormente cinéticas y ahora estáticas, son esenciales para la repersonalización en los mundos morontiales; la reunión del Ajustador y del alma es lo que reensambla la personalidad sobreviviente, lo que os devuelve la conciencia en el momento del despertar morontial.

\par
%\textsuperscript{(1230.4)}
\textsuperscript{112:3.6} Para aquellos que no tienen guardianes seráficos personales, los conservadores colectivos efectúan fiel y eficazmente el mismo servicio de custodia de la identidad y de resurrección de la personalidad. Los serafines son indispensables para reensamblar la personalidad.

\par
%\textsuperscript{(1230.5)}
\textsuperscript{112:3.7} En el momento de la muerte, el Ajustador del Pensamiento pierde temporalmente la personalidad, pero no la identidad; el sujeto humano pierde temporalmente la identidad, pero no la personalidad; en los mundos de las mansiones, los dos se reúnen en una manifestación eterna. Un Ajustador del Pensamiento que se ha ido no regresa nunca a la Tierra como si fuera el ser donde había residido anteriormente; la personalidad nunca se manifiesta sin la voluntad humana; y un ser humano separado de su Ajustador después de la muerte jamás manifiesta una identidad activa ni establece ningún tipo de comunicación con los seres que viven en la Tierra. Estas almas separadas de su Ajustador están total y absolutamente inconscientes durante el largo o corto sueño de la muerte. No puede haber ningún tipo de manifestación de la personalidad ni puede existir ninguna capacidad para ponerse en comunicación con otras personalidades hasta después de haberse consumado la supervivencia. A aquellos que van a los mundos de las mansiones no se les permite enviar mensajes de vuelta a sus seres queridos. En todos los universos existe la política de prohibir este tipo de comunicaciones durante el período de la dispensación en curso.

\section*{4. Los Ajustadores después de la muerte}
\par
%\textsuperscript{(1231.1)}
\textsuperscript{112:4.1} Cuando se produce la muerte, ya sea de naturaleza material, intelectual o espiritual, el Ajustador se despide de su anfitrión mortal y parte hacia Divinington. Desde las sedes del universo local y del superuniverso se establece un contacto reflectante con los supervisores de ambos gobiernos, y el Monitor es quitado de los registros con el mismo número que se le asignó cuando entró en los dominios del tiempo.

\par
%\textsuperscript{(1231.2)}
\textsuperscript{112:4.2} De alguna manera que no comprendemos plenamente, los Censores Universales son capaces de apoderarse del resumen de la vida humana que se encuentra incorporado en la transcripción duplicada, efectuada por el Ajustador, de los valores espirituales y de los significados morontiales de la mente en la que residió. Los Censores pueden apoderarse de la versión del Ajustador sobre el carácter de supervivencia y las cualidades espirituales del humano fallecido, y todos estos datos, junto con los archivos seráficos, están disponibles para ser presentados en el momento del juicio del individuo interesado. Esta información también se utiliza para confirmar las órdenes superuniversales que hacen posible que ciertos ascendentes puedan empezar inmediatamente su carrera morontial, después de su disolución mortal, y dirigirse a los mundos de las mansiones antes de terminar oficialmente la dispensación planetaria.

\par
%\textsuperscript{(1231.3)}
\textsuperscript{112:4.3} Después de la muerte física, y salvo para los individuos trasladados de entre los vivos, el Ajustador liberado se dirige inmediatamente a su esfera natal de Divinington. Los detalles de lo que sucede en ese mundo durante el período en que espera la reaparición efectiva del mortal sobreviviente depende principalmente de si el ser humano asciende a los mundos de las mansiones por su propio derecho individual, o aguarda el llamamiento dispensacional de los supervivientes dormidos de una era planetaria.

\par
%\textsuperscript{(1231.4)}
\textsuperscript{112:4.4} Si el asociado mortal pertenece a un grupo que será repersonalizado al final de una dispensación, el Ajustador no regresará de inmediato al mundo de las mansiones del antiguo sistema donde sirvió, sino que, según su elección, emprenderá una de las siguientes tareas temporales:

\par
%\textsuperscript{(1231.5)}
\textsuperscript{112:4.5} 1. Alistarse en las filas de los Monitores desaparecidos para llevar a cabo unos servicios no revelados.

\par
%\textsuperscript{(1231.6)}
\textsuperscript{112:4.6} 2. Ser destinado durante un tiempo a la observación del régimen del Paraíso.

\par
%\textsuperscript{(1231.7)}
\textsuperscript{112:4.7} 3. Inscribirse en una de las numerosas escuelas de formación de Divinington.

\par
%\textsuperscript{(1231.8)}
\textsuperscript{112:4.8} 4. Colocarse durante un tiempo como observador estudiantil en una de las otras seis esferas sagradas que constituyen el circuito de los mundos paradisiacos del Padre.

\par
%\textsuperscript{(1231.9)}
\textsuperscript{112:4.9} 5. Ser destinado al servicio de mensajeros de los Ajustadores Personalizados.

\par
%\textsuperscript{(1231.10)}
\textsuperscript{112:4.10} 6. Convertirse en instructor adjunto en las escuelas de Divinington dedicadas a la formación de los Monitores que pertenecen al grupo virgen.

\par
%\textsuperscript{(1231.11)}
\textsuperscript{112:4.11} 7. Ser designado para seleccionar un grupo de mundos posibles donde poder servir en caso de que existieran motivos razonables para creer que su asociado humano podría haber rechazado la supervivencia.

\par
%\textsuperscript{(1231.12)}
\textsuperscript{112:4.12} Si en el momento en que la muerte os sorprende habéis alcanzado el tercer círculo o un nivel superior y, por lo tanto, os han asignado un guardián personal del destino; si la transcripción final del resumen de vuestro carácter de supervivencia presentado por el Ajustador es certificada incondicionalmente por el guardián del destino ---si tanto el serafín como el Ajustador están esencialmente de acuerdo en cada detalle de sus informes y recomendaciones sobre vuestra vida--- ; si los Censores Universales y sus asociados reflectantes en Uversa confirman estos datos y lo hacen sin ambig\"uedad ni reservas, en ese caso, los Ancianos de los Días transmiten la orden de avanzar de posición por los circuitos de comunicación que van a Salvington; hecho esto, los tribunales del Soberano de Nebadon decretarán el paso inmediato del alma sobreviviente a las salas de resurrección de los mundos de las mansiones.

\par
%\textsuperscript{(1232.1)}
\textsuperscript{112:4.13} Se me ha informado que si el individuo humano sobrevive sin demora, el Ajustador se inscribe en Divinington, se dirige hacia la presencia paradisiaca del Padre Universal, regresa inmediatamente para ser abrazado por los Ajustadores Personalizados del superuniverso y del universo local donde está asignado, recibe el reconocimiento del jefe de los Monitores Personalizados de Divinington, y luego pasa inmediatamente a la <<realización de la transición de la identidad>>; desde allí es convocado para que al tercer período, y en el mundo de las mansiones, habite la forma real de la personalidad preparada para recibir el alma sobreviviente del mortal terrestre, tal como esta forma ha sido proyectada por el guardián del destino.

\section*{5. La supervivencia del yo humano}
\par
%\textsuperscript{(1232.2)}
\textsuperscript{112:5.1} La individualidad es una realidad cósmica, ya sea material, morontial o espiritual. La realidad del estado \textit{personal} es un don del Padre Universal que actúa en Sí mismo y por Sí mismo o a través de sus múltiples agentes universales. Decir que un ser es personal es reconocer la individuación relativa de ese ser dentro del organismo cósmico. El cosmos viviente es un agregado casi infinitamente integrado de unidades reales, y todas ellas están relativamente sujetas al destino del conjunto. Pero las unidades personales han sido dotadas de la facultad real de elegir entre aceptar o rechazar su destino.

\par
%\textsuperscript{(1232.3)}
\textsuperscript{112:5.2} Aquello que procede del Padre es eterno como el Padre\footnote{\textit{Venimos del Padre Eterno}: Jn 14:9-12,19-20.}, y esto es tan cierto en lo que concierne a la personalidad, que Dios concede por su propio libre albedrío, como en lo que se refiere al divino Ajustador del Pensamiento, un fragmento real de Dios. La personalidad del hombre es eterna, pero en cuanto a su identidad, es una realidad eterna condicionada. Después de aparecer en respuesta a la voluntad del Padre, la personalidad alcanzará su destino que es la Deidad, pero el hombre debe elegir si estará o no presente en el momento de alcanzar ese destino. En ausencia de esta elección, la personalidad alcanzará directamente la Deidad experiencial, volviéndose una parte del Ser Supremo. El ciclo está preordenado, pero la participación del hombre en dicho ciclo es opcional, personal y experiencial.

\par
%\textsuperscript{(1232.4)}
\textsuperscript{112:5.3} La identidad mortal es una condición transitoria de la vida temporal en el universo; sólo es real en la medida en que la personalidad elige volverse un fenómeno continuo en el universo. Ésta es la diferencia esencial entre el hombre y un sistema energético: el sistema energético ha de continuar, no tiene elección; pero el hombre tiene mucho que ver con la determinación de su propio destino. El Ajustador es verdaderamente el camino hacia el Paraíso, pero el hombre mismo debe seguir ese camino por su propia decisión, por la elección de su libre albedrío.

\par
%\textsuperscript{(1232.5)}
\textsuperscript{112:5.4} Los seres humanos sólo poseen la identidad en el sentido material. La mente material expresa estas cualidades del yo a medida que funciona en el sistema energético del intelecto. Cuando se dice que el hombre tiene una identidad, se reconoce que posee un circuito mental que ha sido subordinado a los actos y las elecciones de la voluntad de la personalidad humana. Pero esto es una manifestación material y puramente temporal, al igual que el embrión humano es una etapa parasitaria transitoria de la vida humana. Desde una perspectiva cósmica, los seres humanos nacen, viven y mueren relativamente en un instante; no son duraderos. Pero la personalidad mortal, por su propia elección, posee el poder de trasladar la sede de su identidad desde el sistema pasajero intelectual material al sistema superior del alma morontial, el cual, en asociación con el Ajustador del Pensamiento, es creado como nuevo vehículo para la manifestación de la personalidad.

\par
%\textsuperscript{(1233.1)}
\textsuperscript{112:5.5} Este mismo poder de elección, esta insignia universal de las criaturas con libre albedrío, es lo que constituye la oportunidad más grande del hombre y su responsabilidad cósmica suprema. El destino eterno del futuro finalitario depende de la integridad de la volición humana; el Ajustador divino depende de la sinceridad del libre albedrío humano para adquirir la personalidad eterna; el Padre Universal depende de la fidelidad de la elección humana para hacer realidad un nuevo hijo ascendente; el Ser Supremo depende de la constancia y de la sabiduría de las acciones y decisiones para llevar a cabo la evolución experiencial.

\par
%\textsuperscript{(1233.2)}
\textsuperscript{112:5.6} Los círculos cósmicos de crecimiento de la personalidad deben ser alcanzados finalmente, pero si los accidentes del tiempo y los obstáculos de la existencia material os impiden dominar, sin que haya culpa por vuestra parte, estos niveles en vuestro planeta natal, si vuestras intenciones y deseos tienen un valor de supervivencia, se promulgarán unos decretos para prolongar vuestro período de prueba. Se os concederá un tiempo adicional para que demostréis vuestra valía.

\par
%\textsuperscript{(1233.3)}
\textsuperscript{112:5.7} Si existen dudas en algún momento sobre la conveniencia de hacer avanzar una identidad humana a los mundos de las mansiones, los gobiernos del universo deciden invariablemente a favor de los intereses personales de ese individuo; elevan sin vacilar ese alma al estado de ser transicional, mientras continúan sus observaciones sobre sus intenciones morontiales y sus propósitos espirituales emergentes. Así, la justicia divina se cumple con certeza, y la misericordia divina tiene una nueva oportunidad para extender su ministerio.

\par
%\textsuperscript{(1233.4)}
\textsuperscript{112:5.8} Los gobiernos de Orvonton y de Nebadon no pretenden haber alcanzado una perfección absoluta en el funcionamiento detallado del plan universal de repersonalización de los mortales, pero sí pretenden manifestar paciencia, tolerancia, comprensión y una compasión misericordiosa, y lo hacen realmente. Preferimos asumir el riesgo de una rebelión en un sistema antes que correr el peligro de privar a un solo mortal, que lucha en cualquier mundo evolutivo, de la alegría eterna de continuar la carrera ascendente.

\par
%\textsuperscript{(1233.5)}
\textsuperscript{112:5.9} Esto no significa en absoluto que los seres humanos tengan que disfrutar de una segunda oportunidad después de haber rechazado la primera. Pero sí significa que todas las criaturas volitivas han de tener una verdadera oportunidad para efectuar una elección indudable, consciente y definitiva. Los Jueces soberanos de los universos no privarán del estado de personalidad a ningún ser que no haya hecho su elección eterna de manera plena y definitiva; el alma del hombre debe recibir, y recibirá, una plena y amplia oportunidad para revelar su verdadera intención y su propósito real.

\par
%\textsuperscript{(1233.6)}
\textsuperscript{112:5.10} Cuando los mortales cósmica y espiritualmente más avanzados mueren, pasan inmediatamente a los mundos de las mansiones; esta disposición funciona generalmente para aquellos que han tenido asignado un guardián seráfico personal. Otros mortales pueden ser detenidos hasta el momento en que el juicio de sus asuntos ha terminado, después de lo cual pueden pasar a los mundos de las mansiones, o ser destinados a las filas de los supervivientes dormidos que serán repersonalizados en masa al final de la dispensación planetaria en curso.

\par
%\textsuperscript{(1233.7)}
\textsuperscript{112:5.11} Hay dos dificultades que obstaculizan mis esfuerzos para explicar qué le sucede exactamente al \textit{yo} en la muerte, al \textit{yo} sobreviviente que es distinto al Ajustador que se va. Una de ellas consiste en la imposibilidad de transmitir a vuestro nivel de comprensión una descripción adecuada sobre una operación que tiene lugar en la frontera de los reinos físico y morontial. La otra se debe a las restricciones aplicadas por las autoridades celestiales que gobiernan Urantia sobre mi misión como revelador de la verdad. Hay muchos detalles interesantes que se podrían presentar, pero los omito por consejo de vuestros supervisores planetarios inmediatos. Pero dentro de los límites de lo que me está permitido, puedo decir lo siguiente:

\par
%\textsuperscript{(1234.1)}
\textsuperscript{112:5.12} Hay algo real, algo procedente de la evolución humana, algo adicional al Monitor de Misterio, que sobrevive a la muerte. Esta entidad recién aparecida es el alma\footnote{\textit{Nacimiento del alma}: Jn 3:3.}, y sobrevive a la muerte de vuestro cuerpo físico y de vuestra mente material. Esta entidad es la hija conjunta de la vida y de los esfuerzos combinados del yo humano en unión con el yo divino, el Ajustador. Esta hija de ascendencia humana y divina constituye el elemento sobreviviente de origen terrestre; es el yo morontial, el alma inmortal.

\par
%\textsuperscript{(1234.2)}
\textsuperscript{112:5.13} Esta hija, con un significado que perdura y con un valor de supervivencia, está totalmente inconsciente durante el período que transcurre entre la muerte y la repersonalización, y permanece bajo la custodia del guardián seráfico del destino durante todo este período de espera. Después de la muerte, no actuaréis como ser consciente hasta que hayáis conseguido la nueva conciencia morontial en los mundos de las mansiones de Satania.

\par
%\textsuperscript{(1234.3)}
\textsuperscript{112:5.14} En el momento de la muerte, la identidad funcional asociada a la personalidad humana se desbarata debido al cese del movimiento vital. Aunque la personalidad humana trasciende sus partes constituyentes, depende de ellas para su identidad funcional. La detención de la vida destruye las estructuras cerebrales físicas necesarias para la dotación mental, y el deterioro de la mente pone fin a la conciencia mortal. La conciencia de esa criatura no puede volver a aparecer posteriormente hasta que se haya preparado una situación cósmica que permita a esa misma personalidad humana ejercer de nuevo su actividad en relación con la energía viviente.

\par
%\textsuperscript{(1234.4)}
\textsuperscript{112:5.15} Durante la transición de los mortales sobrevivientes entre su mundo de origen y los mundos de las mansiones, ya sea que experimenten el reensamblaje de su personalidad al tercer período o que asciendan en el momento de una resurrección colectiva, el registro de la constitución de la personalidad es conservado fielmente por los arcángeles en sus mundos de actividades especiales. Estos seres no son los custodios de la personalidad (como los serafines guardianes lo son del alma), pero no es menos cierto que cada factor identificable de la personalidad está salvaguardado eficazmente bajo la custodia de estos fiables depositarios de la supervivencia mortal. En cuanto al paradero exacto de la personalidad mortal durante el período intermedio entre la muerte y la supervivencia, no lo sabemos.

\par
%\textsuperscript{(1234.5)}
\textsuperscript{112:5.16} La situación que hace posible la repersonalización tiene lugar en las salas de resurrección de los planetas receptores morontiales de un universo local. Aquí, en estas cámaras de ensamblaje de la vida, las autoridades supervisoras proporcionan esa relación de energía universal ---morontial, mental y espiritual --- que permite devolver la conciencia al sobreviviente dormido. La reunión de las partes constituyentes de una personalidad en otro tiempo material implica:

\par
%\textsuperscript{(1234.6)}
\textsuperscript{112:5.17} 1. La fabricación de una forma adecuada, de un modelo energético morontial, con el que el nuevo sobreviviente puede ponerse en contacto con la realidad no espiritual, y dentro del cual se puede poner en circuito la variante morontial de la mente cósmica.

\par
%\textsuperscript{(1234.7)}
\textsuperscript{112:5.18} 2. El regreso del Ajustador a la criatura morontial en espera. El Ajustador es el conservador eterno de vuestra identidad ascendente; vuestro Monitor representa la seguridad absoluta de que seréis vosotros mismos, y no otra persona, los que ocuparéis la forma morontial creada para el despertar de vuestra personalidad. Y el Ajustador estará presente en el reensamblaje de vuestra personalidad para asumir de nuevo el papel de guía paradisíaco de vuestro yo sobreviviente.

\par
%\textsuperscript{(1235.1)}
\textsuperscript{112:5.19} 3. Cuando estas condiciones previas para la repersonalización se han reunido, el conservador seráfico de las potencialidades del alma inmortal dormida, con la asistencia de numerosas personalidades cósmicas, confiere esta entidad morontial a la forma corporal y mental morontial que está esperando, mientras confía esta hija evolutiva del Supremo a la asociación eterna con el Ajustador que espera. Y esto completa la repersonalización, el reensamblaje de la memoria, de la perspicacia y de la conciencia ---la identidad.

\par
%\textsuperscript{(1235.2)}
\textsuperscript{112:5.20} El hecho de la repersonalización consiste en que el yo humano que se despierta se apodera de la fase morontial de la mente cósmica recién separada e incorporada en los circuitos. El fenómeno de la personalidad depende de la continuidad de la identidad de reacción de la individualidad al entorno universal; y esto sólo se puede llevar a cabo por medio de la mente. La individualidad se conserva a pesar de un cambio continuo en todos los factores que componen el yo; en la vida física, el cambio es gradual; después de la muerte y de la repersonalización, el cambio es repentino. La verdadera realidad de toda individualidad (personalidad) es capaz de actuar con sensibilidad a las condiciones del universo debido a los cambios incesantes de sus partes constituyentes; el estancamiento acaba inevitablemente en la muerte. La vida humana es un cambio sin fin de los factores de la vida, unificados por la estabilidad de la personalidad invariable.

\par
%\textsuperscript{(1235.3)}
\textsuperscript{112:5.21} Cuando os despertéis así en los mundos de las mansiones de Jerusem, estaréis tan cambiados, vuestra transformación espiritual será tan grande que, si no fuera por vuestro Ajustador del Pensamiento y el guardián del destino, que conectarán tan plenamente vuestra nueva vida en los nuevos mundos con vuestra antigua vida en el primer mundo, al principio tendríais dificultades para relacionar vuestra nueva conciencia morontial con la memoria restablecida de vuestra identidad anterior. A pesar de la continuidad de la individualidad personal, una gran parte de vuestra vida mortal parecerá al principio un sueño vago y nebuloso. Pero el tiempo clarificará muchas asociaciones humanas.

\par
%\textsuperscript{(1235.4)}
\textsuperscript{112:5.22} El Ajustador del Pensamiento sólo os recordará y enumerará aquellos recuerdos y experiencias que forman parte de, y son esenciales para, vuestra carrera universal. Si el Ajustador ha participado como asociado en la evolución de alguna cosa en la mente humana, estas experiencias valiosas sobrevivirán en la conciencia eterna del Ajustador. Pero una gran parte de vuestra vida pasada y de sus recuerdos, que no tienen un significado espiritual ni un valor morontial, perecerán con el cerebro material; muchas experiencias materiales desaparecerán como antiguos andamiajes que os sirvieron de puente para pasar al nivel morontial, pero que ya no tienen utilidad en el universo. Pero la personalidad y las relaciones entre personalidades nunca son andamiajes; la memoria mortal de las relaciones entre personalidades tiene un valor cósmico y sobrevivirá. En los mundos de las mansiones conoceréis y seréis conocidos\footnote{\textit{Conoceréis y seréis conocidos}: 1 Co 13:12.}, y aún más, recordaréis a, y seréis recordados por, vuestros antiguos asociados en la corta pero misteriosa vida en Urantia.

\section*{6. El yo morontial}
\par
%\textsuperscript{(1235.5)}
\textsuperscript{112:6.1} Al igual que una mariposa emerge del estado de oruga, la verdadera personalidad de los seres humanos emergerá en los mundos de las mansiones, manifestándose por primera vez separada de su antigua envoltura de carne material. La carrera morontial en el universo local está relacionada con la elevación continua del mecanismo de la personalidad, desde el nivel morontial inicial de existencia del alma hasta el nivel morontial final de espiritualidad progresiva.

\par
%\textsuperscript{(1235.6)}
\textsuperscript{112:6.2} Es difícil informaros acerca de las formas morontiales de vuestra personalidad para la carrera en el universo local. Seréis provistos de unas formas morontiales capaces de manifestar la personalidad, y se trata de unas investiduras que, a fin de cuentas, están más allá de vuestra comprensión. Estas formas, aunque son totalmente reales, no son unas configuraciones energéticas del tipo material que comprendéis ahora. Sin embargo, tienen la misma finalidad en los mundos del universo local que vuestros cuerpos materiales en los planetas donde nacen los humanos.

\par
%\textsuperscript{(1236.1)}
\textsuperscript{112:6.3} La apariencia de la forma del cuerpo material es sensible, hasta cierto punto, al carácter de la identidad de la personalidad; el cuerpo físico refleja algo de la naturaleza inherente de la personalidad, pero de una forma limitada. La forma morontial la refleja aún más. En la vida física, los mortales pueden ser hermosos por fuera pero desagradables por dentro; en la vida morontial, y de manera creciente en sus niveles superiores, la forma de la personalidad variará directamente de acuerdo con la naturaleza de la persona interior. En el nivel espiritual, la forma exterior y la naturaleza interior empiezan a acercarse a una identificación completa, que se perfecciona cada vez más en los niveles espirituales cada vez más elevados.

\par
%\textsuperscript{(1236.2)}
\textsuperscript{112:6.4} En el estado morontial, el mortal ascendente es dotado de la modificación nebadónica del don de la mente cósmica del Espíritu Maestro de Orvonton. El intelecto mortal, como tal, ha perecido, ha dejado de existir como entidad universal focalizada separada de los circuitos mentales indiferenciados del Espíritu Creativo. Pero los significados y valores de la mente mortal no han perecido. Ciertas fases de la mente subsisten en el alma sobreviviente; el Ajustador conserva ciertos valores experienciales de la antigua mente humana; y la historia de la vida humana, tal como fue vivida en la carne, se conserva en el universo local junto con ciertos registros vivientes en los numerosos seres que se ocupan de la evaluación final del mortal ascendente, unos seres que se extienden desde los serafines hasta los Censores Universales, y probablemente más allá hasta llegar al Supremo.

\par
%\textsuperscript{(1236.3)}
\textsuperscript{112:6.5} La volición de una criatura no puede existir sin la mente, pero subsiste a pesar de la pérdida del intelecto material. Durante los tiempos inmediatamente siguientes a la supervivencia, la personalidad ascendente se rige en gran medida por los patrones de carácter heredados de su vida humana, y por la acción recién aparecida de la mota morontial. Estas pautas de conducta, en mansonia, funcionan aceptablemente en las primeras etapas de la vida morontial y antes de que aparezca la voluntad morontial como expresión volitiva plenamente desarrollada de la personalidad ascendente.

\par
%\textsuperscript{(1236.4)}
\textsuperscript{112:6.6} En la carrera del universo local no existen influencias comparables a los siete espíritus ayudantes de la mente de la existencia humana. La mente morontial ha de evolucionar por contacto directo con la mente cósmica, tal como esta mente cósmica ha sido modificada y traducida por la fuente creativa del intelecto del universo local ---la Ministra Divina.

\par
%\textsuperscript{(1236.5)}
\textsuperscript{112:6.7} Antes de la muerte, la mente mortal tiene conciencia de ser independiente de la presencia del Ajustador; para poder funcionar, la mente que está bajo la influencia de los ayudantes sólo necesita la configuración energético-material que está asociada con ella. Pero el alma morontial, como está por encima de la influencia de los ayudantes, no retiene la conciencia de sí misma sin el Ajustador cuando es privada del mecanismo de la mente material. Este alma evolutiva posee sin embargo un carácter continuado procedente de las decisiones de su antigua mente asociada que estaba bajo la influencia de los ayudantes, y este carácter se convierte en una memoria activa cuando sus configuraciones son estimuladas por el Ajustador que regresa.

\par
%\textsuperscript{(1236.6)}
\textsuperscript{112:6.8} La persistencia de la memoria es una prueba de que la identidad de la individualidad original se conserva; es esencial para tener la plena conciencia de la continuidad y de la expansión de la personalidad. Aquellos mortales que ascienden sin Ajustador dependen de la enseñanza de sus asociados seráficos para reconstruir su memoria humana; las almas morontiales de los mortales fusionados con el Espíritu no tienen más limitaciones que ésta. La configuración de la memoria subsiste en el alma, pero esta configuración necesita la presencia del antiguo Ajustador para hacerse \textit{inmediatamente} reconocible como memoria continuada. Sin el Ajustador, el sobreviviente mortal necesita un tiempo considerable para volver a explorar y a aprender, para recuperar la memoria consciente de los significados y los valores de una existencia anterior.

\par
%\textsuperscript{(1237.1)}
\textsuperscript{112:6.9} El alma con valor de supervivencia refleja fielmente las acciones y las motivaciones tanto cualitativas como cuantitativas del intelecto material, sede anterior de la identidad de la individualidad. Al escoger la verdad, la belleza y la bondad, la mente mortal entra en su carrera universal premorontial bajo la tutela de los siete espíritus ayudantes de la mente, unificados bajo la dirección del espíritu de la sabiduría. Posteriormente, después de completarse los siete círculos de consecución premorontial, el don de la mente morontial se superpone a la mente que está bajo la influencia de los ayudantes, lo que inicia la carrera preespiritual o morontial de progresión en el universo local.

\par
%\textsuperscript{(1237.2)}
\textsuperscript{112:6.10} Cuando una criatura deja su planeta natal, deja tras ella el ministerio de los ayudantes y ya sólo depende del intelecto morontial. Cuando un ascendente deja el universo local, ha alcanzado el nivel espiritual de existencia, puesto que ha sobrepasado el nivel morontial. Esta entidad espiritual recién aparecida se sintoniza entonces con el ministerio directo de la mente cósmica de Orvonton.

\section*{7. La fusión con el Ajustador}
\par
%\textsuperscript{(1237.3)}
\textsuperscript{112:7.1} La fusión con el Ajustador del Pensamiento concede a la personalidad unas realidades eternas que anteriormente sólo eran potenciales. Entre estas nuevas dotaciones se pueden citar: la fijación de la cualidad de divinidad, la experiencia y la memoria de la eternidad pasada, la inmortalidad, y una fase de absolutidad potencial limitada.

\par
%\textsuperscript{(1237.4)}
\textsuperscript{112:7.2} Cuando hayáis corrido la carrera terrestre\footnote{\textit{Cuando hayáis acabado la carrera terrestre}: 2 Ti 4:7.} en vuestra forma temporal, os despertaréis en las orillas de un mundo mejor\footnote{\textit{Orillas de un mundo mejor}: Heb 11:16.}, y seréis unidos finalmente a vuestro fiel Ajustador en un abrazo eterno. Esta fusión constituye el misterio de hacer de Dios y del hombre un solo ser, el misterio de la evolución de la criatura finita, pero esto es eternamente cierto. La fusión es el secreto de la esfera sagrada de Ascendington, y ninguna criatura, salvo las que han experimentado la fusión con el espíritu de la Deidad, puede comprender el verdadero significado de los valores reales que se asocian cuando la identidad de una criatura del tiempo se une eternamente con el espíritu de la Deidad del Paraíso.

\par
%\textsuperscript{(1237.5)}
\textsuperscript{112:7.3} La fusión con el Ajustador se efectúa habitualmente mientras el ascendente reside en su sistema local. Puede producirse en su planeta natal como trascendencia de la muerte natural; puede tener lugar en cualquiera de los mundos de las mansiones o en la sede del sistema; se puede retrasar incluso hasta el momento de la estancia en la constelación; o, en casos especiales, puede no llegar a consumarse hasta que el ascendente se encuentra en la capital del universo local.

\par
%\textsuperscript{(1237.6)}
\textsuperscript{112:7.4} Cuando se ha llevado a cabo la fusión con el Ajustador, la carrera eterna de esa personalidad ya no corre ningún peligro futuro. Los seres celestiales pasan por una larga experiencia para ser puestos a prueba, pero los mortales pasan por unas pruebas relativamente cortas e intensas en los mundos evolutivos y morontiales.

\par
%\textsuperscript{(1237.7)}
\textsuperscript{112:7.5} La fusión con el Ajustador no se produce nunca hasta que los mandatos del superuniverso han declarado que la naturaleza humana ha efectuado una elección definitiva e irrevocable a favor de la carrera eterna. Es la autorización para la unión que, una vez emitida, constituye el permiso competente para que la personalidad fusionada deje finalmente los confines del universo local para dirigirse en su momento a la sede del superuniverso; desde allí, y en un futuro lejano, un seconafín envolverá al peregrino del tiempo para el largo vuelo hacia el universo central de Havona y la aventura de la Deidad.

\par
%\textsuperscript{(1238.1)}
\textsuperscript{112:7.6} En los mundos evolutivos, la individualidad es material; es una cosa en el universo y, como tal, está sometida a las leyes de la existencia material. Es un hecho en el tiempo y es sensible a las vicisitudes del mismo. \textit{Las decisiones sobrela supervivencia han de ser expresadas aquí}. En el estado morontial, el yo se ha convertido en una realidad universal nueva y más duradera, y su crecimiento continuo está basado en una sintonización creciente con los circuitos mentales y espirituales de los universos. \textit{Las decisiones sobre la supervivencia debenconfirmarse ahora}. Cuando el yo alcanza el nivel espiritual, se ha vuelto un valor seguro en el universo, y este nuevo valor está basado en el hecho de que \textit{lasdecisiones sobre la supervivencia se han tomado}, un hecho que está atestiguado por la fusión eterna con el Ajustador del Pensamiento. Después de haber alcanzado el estado de un verdadero valor en el universo, la criatura se vuelve potencialmente libre de buscar el valor universal más elevado ---Dios.

\par
%\textsuperscript{(1238.2)}
\textsuperscript{112:7.7} Las reacciones universales de estos seres fusionados son dobles: Son unos individuos morontiales distintos, no del todo diferentes a los serafines, y son también unos seres que pertenecen potencialmente a la orden de los finalitarios del Paraíso.

\par
%\textsuperscript{(1238.3)}
\textsuperscript{112:7.8} Pero el individuo fusionado es en realidad una sola personalidad, un solo ser, cuya unidad desafía todos los intentos de análisis por parte de cualquier inteligencia de los universos. Y así, después de haber pasado ante los tribunales del universo local, desde los más modestos hasta los más elevados, sin que ninguno de ellos haya sido capaz de identificar por separado al hombre o al Ajustador, seréis conducidos finalmente ante el Soberano de Nebadon, el Padre de vuestro universo local. Allí, de las manos mismas del ser cuya paternidad creativa en este universo temporal ha hecho posible el hecho de vuestra vida, recibiréis las credenciales que os darán derecho a continuar finalmente vuestra carrera en el superuniverso en busca del Padre Universal.

\par
%\textsuperscript{(1238.4)}
\textsuperscript{112:7.9} El Ajustador victorioso, ¿ha conseguido la personalidad gracias a su magnífico servicio a la humanidad, o es el valiente humano el que ha alcanzado la inmortalidad mediante sus sinceros esfuerzos por lograr parecerse al Ajustador?. No es ni lo uno ni lo otro, sino que los dos juntos han llevado a cabo la evolución de un miembro de uno de los tipos excepcionales de personalidades ascendentes del Supremo, un ser que siempre hallaréis servicial, fiel y eficaz, un candidato a un crecimiento y a un desarrollo adicionales siempre dirigidos hacia arriba, sin detenerse nunca en su ascensión celestial hasta haber atravesado los siete circuitos de Havona, y el alma de antiguo origen terrestre permanezca en adoración reconociendo la personalidad real del Padre en el Paraíso.

\par
%\textsuperscript{(1238.5)}
\textsuperscript{112:7.10} Durante toda esta magnífica ascensión, el Ajustador del Pensamiento es la garantía divina de la estabilización espiritual futura y completa del mortal ascendente. Entretanto, la presencia del libre albedrío humano proporciona al Ajustador un canal eterno para liberar la naturaleza divina e infinita. Estas dos identidades se han vuelto ahora una sola; ningún acontecimiento del tiempo o de la eternidad puede ya separar al hombre y al Ajustador; son inseparables, han fusionado para la eternidad.

\par
%\textsuperscript{(1238.6)}
\textsuperscript{112:7.11} En los mundos donde se fusiona con el Ajustador, el destino del Monitor de Misterio es idéntico al del mortal ascendente ---el Cuerpo Paradisiaco de la Finalidad. Ni el Ajustador ni el mortal pueden alcanzar esta meta única sin la plena cooperación y la ayuda fiel del otro. Esta asociación extraordinaria es uno de los fenómenos cósmicos más fascinantes y asombrosos de la presente era del universo.

\par
%\textsuperscript{(1239.1)}
\textsuperscript{112:7.12} Desde el momento de la fusión con el Ajustador, la condición del ascendente es la de una criatura evolutiva. El miembro humano fue el primero en disfrutar de la personalidad y, por consiguiente, es superior al Ajustador en todas las cuestiones relacionadas con el reconocimiento de la personalidad. La sede paradisiaca de este ser fusionado es Ascendington, y no Divinington; esta combinación única de Dios y de hombre se considera como un mortal ascendente durante todo el camino hasta llegar al Cuerpo de la Finalidad.

\par
%\textsuperscript{(1239.2)}
\textsuperscript{112:7.13} Una vez que un Ajustador fusiona con un mortal ascendente, el número de ese Ajustador es borrado de los archivos del superuniverso. En cuanto a lo que sucede con los archivos de Divinington, no lo sé, pero supongo que el registro de ese Ajustador es trasladado a los círculos secretos de las cortes interiores de Grandfanda, el director en funciones del Cuerpo de la Finalidad.

\par
%\textsuperscript{(1239.3)}
\textsuperscript{112:7.14} Con la fusión del Ajustador, el Padre Universal ha cumplido su promesa de darse a sí mismo a sus criaturas materiales; ha cumplido la promesa y ha consumado el plan de la donación eterna de la divinidad a la humanidad. Ahora empieza la tentativa humana por comprender y llevar a cabo las posibilidades ilimitadas inherentes a la asociación celestial con Dios, una asociación que se ha convertido así en un hecho.

\par
%\textsuperscript{(1239.4)}
\textsuperscript{112:7.15} El destino actualmente conocido de los mortales sobrevivientes es el Cuerpo Paradisiaco de la Finalidad; ésta es también la meta final para todos los Ajustadores del Pensamiento que se han unido de manera eterna con sus compañeros mortales. Los finalitarios del Paraíso trabajan actualmente en numerosas tareas en todo el gran universo, pero todos suponemos que tendrán otras tareas más celestiales que realizar en el lejano futuro, después de que los siete superuniversos se hayan establecido en la luz y la vida, y el Dios finito haya surgido finalmente del misterio que ahora rodea a esta Deidad Suprema.

\par
%\textsuperscript{(1239.5)}
\textsuperscript{112:7.16} Se os ha informado hasta cierto punto acerca de la organización y del personal del universo central, los superuniversos y los universos locales; se os han contado algunas cosas sobre el carácter y el origen de algunas de las diversas personalidades que gobiernan actualmente estas extensas creaciones. También se os ha informado que unas inmensas galaxias de universos están en proceso de organización mucho más allá de la periferia del gran universo, en el primer nivel del espacio exterior. En el transcurso de estas narraciones también se os ha indicado que el Ser Supremo desvelará su actividad terciaria no revelada en estas regiones actualmente inexploradas del espacio exterior; y también se os ha dicho que los finalitarios del cuerpo paradisiaco son los hijos experienciales del Supremo.

\par
%\textsuperscript{(1239.6)}
\textsuperscript{112:7.17} Creemos que los mortales fusionados con su Ajustador, así como sus asociados finalitarios, están destinados a ejercer su actividad de alguna manera en la administración de los universos del primer nivel del espacio exterior. No tenemos la menor duda de que, a su debido tiempo, estas enormes galaxias se convertirán en universos habitados. Y estamos igualmente convencidos de que entre sus administradores se encontrarán los finalitarios paradisiacos, cuyas naturalezas son la consecuencia cósmica de la mezcla de la criatura y del Creador.

\par
%\textsuperscript{(1239.7)}
\textsuperscript{112:7.18} ¡Qué aventura! ¡Qué gesta! Una creación gigantesca que será administrada por los hijos del Supremo, esos Ajustadores personalizados y humanizados, esos mortales eternizados y unidos a sus Ajustadores, esas combinaciones misteriosas y esas asociaciones eternas entre la manifestación más elevada que se conoce de la esencia de la Fuente-Centro Primera, y la forma más humilde de vida inteligente capaz de comprender y de alcanzar al Padre Universal. Pensamos que estos seres amalgamados, estas asociaciones entre el Creador y la criatura, se convertirán en unos gobernantes magníficos, unos administradores incomparables y unos directores comprensivos y compasivos para todas y cada una de las formas de vida inteligente que puedan llegar a existir en todos esos futuros universos del primer nivel del espacio exterior.

\par
%\textsuperscript{(1240.1)}
\textsuperscript{112:7.19} Es verdad que vosotros, los mortales, sois de origen terrestre, de origen animal; vuestro cuerpo es ciertamente de polvo\footnote{\textit{Nuestro cuerpo es de polvo}: Gn 2:7; 3:19; Sal 103:14; Ec 3:20.}. Pero si queréis realmente, si verdaderamente lo deseáis, es seguro que la herencia de los siglos será vuestra, y que algún día serviréis en todos los universos en vuestra verdadera condición ---la de hijos del Dios Supremo de la experiencia e hijos divinos del Padre Paradisiaco de todas las personalidades.

\par
%\textsuperscript{(1240.2)}
\textsuperscript{112:7.20} [Presentado por un Mensajero Solitario de Orvonton.]


\chapter{Documento 113. Los guardianes seráficos del destino}
\par
%\textsuperscript{(1241.1)}
\textsuperscript{113:0.1} DESPUÉS de haber presentado las narraciones sobre los Espíritus Ministrantes del Tiempo y las Huestes de Mensajeros del Espacio, llegamos al estudio de los ángeles guardianes, los serafines dedicados al ministerio de los mortales individuales, para cuya elevación y perfección se ha preparado todo el inmenso sistema de la supervivencia y de la progresión espiritual. Durante las épocas pasadas en Urantia, estos guardianes del destino eran casi el único grupo conocido de ángeles. Los serafines planetarios son en verdad los espíritus ministrantes enviados para servir a aquellas personas que sobrevivirán. Estos serafines acompañantes han desempeñado sus funciones como asistentes espirituales del hombre mortal en todos los grandes acontecimientos del pasado y del presente. En muchas revelaciones, <<la palabra fue pronunciada por los ángeles>>\footnote{\textit{La palabra fue pronunciada por los ángeles}: Heb 2:2.}; muchos mandatos del cielo han sido <<recibidos por el ministerio de los ángeles>>\footnote{\textit{Recibidos por el ministerio de los ángeles}: Hch 7:53.}.

\par
%\textsuperscript{(1241.2)}
\textsuperscript{113:0.2} Los serafines son los ángeles tradicionales del cielo; son los espíritus ministrantes que viven tan cerca de vosotros y hacen tanto por vosotros. Han servido en Urantia desde los primeros tiempos de la inteligencia humana.

\section*{1. Los ángeles guardianes}
\par
%\textsuperscript{(1241.3)}
\textsuperscript{113:1.1} La enseñanza sobre los ángeles guardianes\footnote{\textit{Ángeles guardianes}: Bar 6:7.} no es un mito; algunos grupos de seres humanos tienen realmente ángeles personales. En reconocimiento de este hecho, Jesús, cuando habló de los niños del reino celestial, dijo: <<Tened cuidado de no menospreciar a ninguno de estos pequeños, pues os digo que sus ángeles perciben continuamente la presencia del espíritu de mi Padre>>\footnote{\textit{Cuidado de no menospreciar a estos pequeños}: Mt 18:10.}.

\par
%\textsuperscript{(1241.4)}
\textsuperscript{113:1.2} En un principio, los serafines fueron asignados claramente a las distintas razas de Urantia. Pero desde la donación de Miguel son asignados con arreglo a la inteligencia, la espiritualidad y el destino humanos. Intelectualmente, la humanidad está dividida en tres clases:

\par
%\textsuperscript{(1241.5)}
\textsuperscript{113:1.3} 1. Los humanos con una mente subnormal --- aquellos que no ejercen un poder normal de voluntad; aquellos que no toman decisiones ordinarias. Esta clase abarca a los que no pueden comprender a Dios; les falta capacidad para adorar inteligentemente a la Deidad. Los seres subnormales de Urantia tienen asignado un cuerpo de serafines, una compañía, con un batallón de querubines, encargados de servirlos y de vigilar que se les manifieste justicia y misericordia en las luchas por la vida en la esfera.

\par
%\textsuperscript{(1241.6)}
\textsuperscript{113:1.4} 2. El tipo medio o normal de mente humana. Desde el punto de vista del ministerio seráfico, la mayor parte de los hombres y de las mujeres están agrupados en siete clases de acuerdo con el estado que han conseguido superando los círculos del progreso humano y del desarrollo espiritual.

\par
%\textsuperscript{(1241.7)}
\textsuperscript{113:1.5} 3. Los humanos con una mente supernormal ---aquellas personas con un gran poder de decisión y con un potencial indudable de logros espirituales; los hombres y las mujeres que disfrutan de un mayor o menor contacto con su Ajustador interior; los miembros de los diversos cuerpos de reserva del destino. Cualquiera que sea el círculo en el que se encuentre un ser humano, si ese individuo es alistado en cualquiera de los diversos cuerpos de reserva del destino, se le asigna inmediatamente un serafín personal, y desde ese momento hasta que termine su carrera terrestre, ese mortal disfrutará del ministerio continuo y de los cuidados incesantes de un ángel guardián. También, cuando un ser humano toma \textit{la} decisión suprema, cuando establece un verdadero compromiso con el Ajustador, un guardián personal se asigna inmediatamente a ese alma.

\par
%\textsuperscript{(1242.1)}
\textsuperscript{113:1.6} En el ministerio hacia los llamados seres normales, las asignaciones seráficas se efectúan de acuerdo con los círculos de intelectualidad y de espiritualidad que los seres humanos han alcanzado. Os ponéis en camino investidos de vuestra mente mortal en el séptimo círculo y viajáis hacia el interior en la tarea de comprenderos, conquistaros y dominaros a vosotros mismos; avanzáis círculo tras círculo (si la muerte natural no termina con vuestra carrera, transfiriendo vuestras luchas a los mundos de las mansiones) hasta que alcanzáis el primer círculo, o círculo interno de contacto y de comunión relativos con el Ajustador interior.

\par
%\textsuperscript{(1242.2)}
\textsuperscript{113:1.7} En el círculo inicial, o séptimo círculo, los seres humanos tienen un ángel guardián con una compañía de querubines auxiliares encargados del cuidado y de la custodia de mil mortales. En el sexto círculo, una pareja seráfica con una compañía de querubines está destinada a guiar a estos mortales ascendentes en grupos de quinientos. Cuando se alcanza el quinto círculo, los seres humanos son agrupados en compañías de unas cien personas, y una pareja de serafines guardianes con un grupo de querubines se encargan de ellas. Cuando alcanzan el cuarto círculo, los seres mortales son reunidos en grupos de diez, y una pareja de serafines, asistida por una compañía de querubines, se encarga nuevamente de ellos.

\par
%\textsuperscript{(1242.3)}
\textsuperscript{113:1.8} Cuando una mente mortal rompe la inercia de la herencia animal y alcanza el tercer círculo de intelectualidad humana y de espiritualidad adquirida, desde ese momento en adelante un ángel personal (en realidad dos) se dedicará total y exclusivamente a ese mortal ascendente. Además de los Ajustadores del Pensamiento interiores siempre presentes y cada vez más eficaces, estas almas humanas reciben así la ayuda indivisa de estos guardianes personales del destino en todos sus esfuerzos por terminar el tercer círculo, atravesar el segundo y alcanzar el primero.

\section*{2. Los guardianes del destino}
\par
%\textsuperscript{(1242.4)}
\textsuperscript{113:2.1} A los serafines no se les conoce como guardianes del destino hasta el momento en que son nombrados para asociarse a un alma humana que ha realizado uno o más de estos tres logros: ha tomado la decisión suprema de volverse semejante a Dios, ha entrado en el tercer círculo, o ha sido enrolada en uno de los cuerpos de reserva del destino.

\par
%\textsuperscript{(1242.5)}
\textsuperscript{113:2.2} En la evolución de las razas, un guardián del destino es asignado al primer ser humano que alcanza el círculo de conquista requerido. En Urantia, el primer mortal que consiguió un guardián personal fue Rantowoc, un sabio de la raza roja de hace mucho tiempo.

\par
%\textsuperscript{(1242.6)}
\textsuperscript{113:2.3} Todas las asignaciones angélicas se llevan a cabo en un grupo de serafines voluntarios, y estos nombramientos siempre están de acuerdo con las necesidades humanas y con relación al estado de la pareja angélica ---a la luz de la experiencia, la habilidad y la sabiduría seráficas. Únicamente los serafines que han servido durante mucho tiempo, los tipos más experimentados y probados, son asignados como guardianes del destino. Muchos guardianes han conseguido una gran experiencia valiosa en los mundos pertenecientes a la serie donde no se fusiona con el Ajustador. Al igual que lo hacen los Ajustadores, los serafines acompañan a estos seres durante una sola vida, y luego son liberados para realizar una nueva misión. Muchos guardianes de Urantia han tenido esta experiencia práctica previa en otros mundos.

\par
%\textsuperscript{(1243.1)}
\textsuperscript{113:2.4} Cuando los seres humanos no logran sobrevivir, sus guardianes personales o colectivos pueden servir repetidas veces en calidad similar en el mismo planeta. Los serafines desarrollan una estima sentimental por los mundos individuales y albergan un afecto especial por ciertas razas y tipos de criaturas mortales con las que han estado tan estrecha e íntimamente asociados.

\par
%\textsuperscript{(1243.2)}
\textsuperscript{113:2.5} Los ángeles desarrollan un afecto duradero por sus asociados humanos; y si pudierais visualizar a los serafines, desarrollaríais también un cálido afecto por ellos. Despojados de vuestros cuerpos materiales y provistos de formas espirituales, estaríais muy cerca de los ángeles en muchos atributos de la personalidad. Comparten la mayoría de vuestras emociones y experimentan algunas más. La única emoción que os impulsa y que es para ellos un poco difícil de comprender es la herencia del miedo animal que ocupa un lugar tan importante en la vida mental del habitante medio de Urantia. A los ángeles les resulta verdaderamente difícil de comprender por qué permitís de manera tan insistente que vuestros poderes intelectuales superiores, e incluso vuestra fe religiosa, estén tan dominados por el miedo, tan completamente desmoralizados por el pánico irreflexivo del temor y la ansiedad.

\par
%\textsuperscript{(1243.3)}
\textsuperscript{113:2.6} Todos los serafines tienen sus nombres individuales, pero en los registros de asignación al servicio de un mundo, se les designa con frecuencia por sus números planetarios. En la sede del universo están registrados con su nombre y su número. El guardián del destino del sujeto humano utilizado en esta comunicación de contacto es el número 3 del grupo 17, de la compañía 126, del batallón 4, de la unidad 384, de la legión 6, de la hueste 37, del ejército seráfico 182.314 de Nebadon. El número actual de asignación planetaria de este serafín en Urantia, y para este sujeto humano, es el 3.641.852.

\par
%\textsuperscript{(1243.4)}
\textsuperscript{113:2.7} En el ministerio de la tutela personal, en la asignación de los ángeles como guardianes del destino, los serafines siempre ofrecen voluntariamente sus servicios. En la ciudad donde efectuamos esta visita, cierto mortal fue admitido recientemente en el cuerpo de reserva del destino, y puesto que los ángeles guardianes acompañan personalmente a este tipo de humanos, más de cien serafines cualificados se ofrecieron para la misión. El director planetario seleccionó a doce entre los individuos más experimentados, y posteriormente nombró al serafín que ellos escogieron como el mejor adaptado para guiar a este ser humano durante su viaje por la vida. Es decir, escogieron a cierta pareja de serafines igualmente cualificados; uno de los miembros de esta pareja seráfica estará siempre de servicio.

\par
%\textsuperscript{(1243.5)}
\textsuperscript{113:2.8} Las tareas seráficas pueden ser incesantes, pero uno de los miembros de la pareja angélica puede desprenderse de todas las responsabilidades del ministerio. Al igual que los querubines, los serafines sirven generalmente en parejas, pero a diferencia de sus asociados menos avanzados, los serafines trabajan a veces solos. Pueden ejercer su actividad como individuos en prácticamente todos sus contactos con los seres humanos. Los dos ángeles sólo se necesitan para la comunicación y el servicio en los circuitos superiores de los universos.

\par
%\textsuperscript{(1243.6)}
\textsuperscript{113:2.9} Cuando una pareja seráfica acepta la misión de guardianes, sirven así durante el resto de la vida de ese ser humano. El complemento del ser (uno de los dos ángeles) se convierte en el registrador de la empresa. Estos serafines complementarios son los ángeles registradores de los mortales de los mundos evolutivos. Los registros son conservados por la pareja de querubines (un querubín y un sanobín) que están siempre asociados a los guardianes seráficos, pero estos registros siempre están patrocinados por uno de los serafines.

\par
%\textsuperscript{(1244.1)}
\textsuperscript{113:2.10} El guardián es reemplazado periódicamente por su complemento con el objeto de descansar y de recargarse con la energía vital de los circuitos del universo, y durante su ausencia, el querubín asociado actúa como registrador, tal como es también el caso cuando el serafín complementario se encuentra igualmente ausente.

\section*{3. Relación con otras influencias espirituales}
\par
%\textsuperscript{(1244.2)}
\textsuperscript{113:3.1} Una de las cosas más importantes que un guardián del destino hace por su sujeto mortal es efectuar una coordinación personal de las numerosas influencias espirituales impersonales que habitan, rodean e inciden en la mente y en el alma de la criatura material en evolución. Los seres humanos son personalidades, y a los espíritus no personales y a las entidades prepersonales les resulta extremadamente difícil ponerse en contacto directo con unas mentes tan sumamente materiales y tan diferenciadamente personales. El ministerio del ángel guardián unifica más o menos todas estas influencias y las hace más fácilmente apreciables por la naturaleza moral en expansión de la personalidad humana evolutiva.

\par
%\textsuperscript{(1244.3)}
\textsuperscript{113:3.2} El guardián seráfico puede correlacionar más especialmente los numerosos agentes e influencias del Espíritu Infinito que se extienden desde los dominios de los controladores físicos y de los espíritus ayudantes de la mente, hasta el Espíritu Santo de la Ministra Divina y hasta la presencia del Espíritu Omnipresente de la Fuente-Centro Tercera del Paraíso. Una vez que ha unificado así y ha hecho más personales estos amplios ministerios del Espíritu Infinito, el serafín se encarga entonces de correlacionar esta influencia integrada del Actor Conjunto con las presencias espirituales del Padre y del Hijo.

\par
%\textsuperscript{(1244.4)}
\textsuperscript{113:3.3} El Ajustador es la presencia del Padre; el Espíritu de la Verdad es la presencia de los Hijos. El ministerio de los serafines guardianes unifica y coordina estos dones divinos en los niveles inferiores de la experiencia espiritual humana. Los servidores angélicos tienen el don de combinar el amor del Padre y la misericordia del Hijo en su ministerio para con las criaturas mortales.

\par
%\textsuperscript{(1244.5)}
\textsuperscript{113:3.4} En esto se revela la razón por la que el guardián seráfico se vuelve finalmente el conservador personal de los modelos mentales, de las fórmulas de la memoria y de las realidades del alma del superviviente mortal durante el intervalo entre la muerte física y la resurrección morontial. Nadie, salvo los hijos ministrantes del Espíritu Infinito, podría actuar así a favor de la criatura humana durante esta fase de transición entre un nivel del universo y otro nivel más elevado. Incluso cuando emprendéis vuestro sueño de transición final, cuando pasáis del tiempo a la eternidad, un alto supernafín comparte igualmente el tránsito con vosotros como custodio de vuestra identidad de criatura y como garantía de vuestra integridad personal.

\par
%\textsuperscript{(1244.6)}
\textsuperscript{113:3.5} En el nivel espiritual, los serafines convierten en personales muchos ministerios del universo por otra parte impersonales y prepersonales; son coordinadores. En el nivel intelectual, ponen en correlación la mente y la morontia; son intérpretes. Y en el nivel físico, manipulan el entorno terrestre gracias a su conexión con los Controladores Físicos Maestros y a través del ministerio cooperativo de las criaturas intermedias.

\par
%\textsuperscript{(1244.7)}
\textsuperscript{113:3.6} Esto es un relato de las funciones múltiples y complicadas de un serafín acompañante; pero este tipo de personalidad angélica subordinada, creada tan sólo un poco por encima del nivel universal de la humanidad, ¿cómo puede hacer estas cosas tan difíciles y complejas? En realidad no lo sabemos, pero suponemos que este ministerio extraordinario es facilitado de alguna manera no desvelada por el trabajo no reconocido y no revelado del Ser Supremo, la Deidad en vías de manifestación de los universos evolutivos del tiempo y del espacio. A lo largo de todo el ámbito de la supervivencia progresiva, dentro y a través del Ser Supremo, los serafines son una parte esencial del progreso continuo de los mortales.

\section*{4. Los campos de acción seráficos}
\par
%\textsuperscript{(1245.1)}
\textsuperscript{113:4.1} Los serafines guardianes no son la mente, aunque proceden del Espíritu Creativo, la misma fuente que da origen también a la mente mortal. Los serafines son estimuladores de la mente; intentan continuamente provocar en la mente humana las decisiones que conducen a superar los círculos. No lo hacen como los Ajustadores, que actúan desde el interior y a través del alma, sino más bien desde el exterior hacia el interior, trabajando a través del entorno social, ético y moral de los seres humanos. Los serafines no son la atracción divina bajo la forma del Ajustador del Padre Universal, pero ejercen su actividad como agentes personales del ministerio del Espíritu Infinito.

\par
%\textsuperscript{(1245.2)}
\textsuperscript{113:4.2} El hombre mortal, sujeto a las directrices del Ajustador, es también sensible a la guía seráfica. El Ajustador es la esencia de la naturaleza eterna del hombre; el serafín es el educador de la naturaleza evolutiva del hombre ---de la mente mortal en esta vida, y del alma morontial en la siguiente. En los mundos de las mansiones seréis conscientes y tendréis conocimiento de los instructores seráficos, pero en la primera vida los hombres no son generalmente conscientes de ellos.

\par
%\textsuperscript{(1245.3)}
\textsuperscript{113:4.3} Los serafines actúan como educadores de los hombres, guiando los pasos de la personalidad humana por los caminos de las experiencias nuevas y progresivas. Aceptar la guía de un serafín raras veces significa disfrutar de una vida cómoda. Si seguís esta guía, encontraréis con toda seguridad las escarpadas colinas de la elección moral y del progreso espiritual, y si tenéis valentía, las atravesaréis.

\par
%\textsuperscript{(1245.4)}
\textsuperscript{113:4.4} El impulso a la adoración se origina principalmente en las incitaciones espirituales de los ayudantes superiores de la mente, reforzados por las directrices del Ajustador. Pero el impulso a la oración que experimentan tan a menudo los mortales conscientes de Dios surge con mucha frecuencia como resultado de la influencia seráfica. El serafín guardián manipula continuamente el entorno humano con objeto de aumentar la perspicacia cósmica del ascendente humano, a fin de que este candidato a la supervivencia pueda adquirir una conciencia acrecentada de la presencia del Ajustador interior y sea capaz de ofrecer así una mayor cooperación con la misión espiritual de la presencia divina.

\par
%\textsuperscript{(1245.5)}
\textsuperscript{113:4.5} Aunque no existe en apariencia ninguna comunicación entre los Ajustadores interiores y los serafines que rodean al hombre, siempre parecen trabajar en perfecta armonía y exquisito acuerdo. Los guardianes son más activos en los momentos en que los Ajustadores lo son menos, pero el ministerio de los dos está de alguna manera extrañamente correlacionado. Una cooperación tan magnífica difícilmente podría ser accidental o fortuita.

\par
%\textsuperscript{(1245.6)}
\textsuperscript{113:4.6} La personalidad ministrante del serafín guardián, la presencia de Dios bajo la forma del Ajustador interior, la acción en circuito del Espíritu Santo, y la conciencia del Hijo bajo la forma del Espíritu de la Verdad están todas divinamente correlacionadas en una unidad significativa de ministerio espiritual en la personalidad mortal y para la misma. Aunque proceden de orígenes diferentes y de niveles diferentes, todas estas influencias celestiales están integradas en la presencia envolvente y evolutiva del Ser Supremo.

\section*{5. El ministerio seráfico hacia los mortales}
\par
%\textsuperscript{(1245.7)}
\textsuperscript{113:5.1} Los ángeles no invaden la santidad de la mente humana; no manipulan la voluntad de los mortales; tampoco se ponen en contacto directo con los Ajustadores interiores. El guardián del destino os influye de todas las maneras posibles que estén de acuerdo con la dignidad de vuestra personalidad; estos ángeles no interfieren bajo ninguna circunstancia en la acción libre de la voluntad humana. Ni los ángeles ni ninguna otra orden de personalidad del universo tienen poder o autoridad para reducir o limitar las prerrogativas de la elección humana.

\par
%\textsuperscript{(1246.1)}
\textsuperscript{113:5.2} Los ángeles están tan cerca de vosotros y os cuidan con tanta ternura que de manera figurada <<lloran a causa de vuestra intolerancia y testarudez obstinadas>>. Los serafines no derraman lágrimas físicas; no tienen cuerpos físicos, y tampoco poseen alas. Pero sí tienen emociones espirituales, y experimentan sensaciones y sentimientos de naturaleza espiritual que son en cierto modo comparables a las emociones humanas.

\par
%\textsuperscript{(1246.2)}
\textsuperscript{113:5.3} Los serafines actúan a vuestro favor independientemente por completo de vuestras peticiones directas; ejecutan las órdenes de sus superiores y ejercen así su actividad sin tener en cuenta vuestros caprichos pasajeros o vuestro humor cambiante. Esto no implica que no podáis hacer sus tareas más fáciles o más difíciles, sino más bien que los ángeles no se ocupan directamente de vuestras peticiones ni de vuestras oraciones.

\par
%\textsuperscript{(1246.3)}
\textsuperscript{113:5.4} En la vida en la carne, la inteligencia de los ángeles no está a la disposición directa de los hombres mortales. No son ni jefes supremos ni directores; son simplemente guardianes. Los serafines os \textit{protegen}; no tratan de influiros directamente; debéis trazar vuestros propios derroteros, y estos ángeles actúan entonces para hacer el mejor uso posible del camino que habéis elegido. No intervienen (generalmente) de manera arbitraria en los asuntos rutinarios de la vida humana. Pero cuando reciben instrucciones de sus superiores para ejecutar alguna proeza inhabitual, podéis estar seguros de que estos guardianes encontrarán alguna manera de llevar a cabo esos mandatos. Por consiguiente, no se entrometen en la representación del drama humano excepto en casos de urgencia, y entonces lo hacen generalmente por orden directa de sus superiores. Son los seres que os van a seguir durante muchas épocas, y están recibiendo así una introducción a su trabajo futuro y a su asociación de personalidad.

\par
%\textsuperscript{(1246.4)}
\textsuperscript{113:5.5} En ciertas circunstancias, los serafines pueden ejercer sus funciones como ministros materiales para los seres humanos, pero su actividad en esta calidad es muy rara. Con la ayuda de las criaturas intermedias y de los controladores físicos, pueden ejercer una gran variedad de actividades a favor de los seres humanos, e incluso ponerse en contacto real con la humanidad, pero estos acontecimientos son muy poco frecuentes. En la mayoría de los casos, las circunstancias del reino material se desarrollan sin ser alteradas por la acción seráfica, aunque han surgido ocasiones en las que los eslabones vitales de la cadena de la evolución humana corrían peligro, y entonces los guardianes seráficos han actuado, y adecuadamente, por su propia iniciativa.

\section*{6. Los ángeles guardianes después de la muerte}
\par
%\textsuperscript{(1246.5)}
\textsuperscript{113:6.1} Después de haberos dicho algo sobre el ministerio de los serafines durante la vida física, intentaré informaros acerca de la conducta de los guardianes del destino en el momento de la disolución mortal de sus asociados humanos. Después de vuestra muerte, vuestros registros, vuestras especificaciones de identidad y la entidad morontial del alma humana ---desarrollada conjuntamente por el ministerio de la mente mortal y del Ajustador divino--- son fielmente conservados por el guardián del destino, junto con todos los otros valores relacionados con vuestra existencia futura, todo lo que constituye vuestro yo, vuestro yo real, excepto la identidad de la existencia continua, representada por el Ajustador que se va, y la realidad de la personalidad.

\par
%\textsuperscript{(1246.6)}
\textsuperscript{113:6.2} En cuanto desaparece la luz piloto en la mente humana, la luminosidad espiritual que los serafines asocian a la presencia del Ajustador, el ángel acompañante se presenta en persona a los ángeles que están al mando sucesivamente del grupo, la compañía, el batallón, la unidad, la legión y la hueste; y después de haber sido debidamente inscrito para la aventura final del tiempo y del espacio, dicho ángel recibe un certificado del jefe planetario de los serafines para presentarlo ante la Estrella Vespertina (u otro lugarteniente de Gabriel) que manda el ejército seráfico de ese candidato a la ascensión del universo. Cuando el comandante de esta suprema unidad organizada le concede el permiso, ese guardián del destino se dirige al primer mundo de las mansiones y espera allí a que se restablezca la conciencia de su antiguo pupilo en la carne.

\par
%\textsuperscript{(1247.1)}
\textsuperscript{113:6.3} En el caso de que el alma humana no logre sobrevivir después de haber recibido la asignación de un ángel personal, el serafín acompañante debe dirigirse a la sede del universo local para atestiguar sobre la exactitud de los datos completos que su complemento ha presentado anteriormente. A continuación se presenta ante los tribunales de los arcángeles para ser absuelto de culpa por el fracaso de su sujeto en el asunto de la supervivencia; y luego regresa a los mundos para ser asignado de nuevo a otro mortal con potencial de ascensión o a alguna otra división del ministerio seráfico.

\par
%\textsuperscript{(1247.2)}
\textsuperscript{113:6.4} Pero los ángeles sirven a las criaturas evolutivas de muchas maneras, además de los servicios de la tutela personal y colectiva. Los guardianes personales cuyos sujetos no van de inmediato a los mundos de las mansiones, no permanecen allí en la ociosidad esperando el llamamiento nominal dispensacional del juicio; son destinados de nuevo a numerosas misiones ministrantes por todo el universo.

\par
%\textsuperscript{(1247.3)}
\textsuperscript{113:6.5} El serafín guardián es el fideicomisario que custodia los valores de supervivencia del alma dormida del hombre mortal, al igual que el Ajustador ausente \textit{es} la identidad de ese ser inmortal del universo. Cuando los dos colaboran en las salas de resurrección de mansonia conjuntamente con la forma morontial recién fabricada, se produce la reunión de los factores constituyentes de la personalidad del ascendente mortal.

\par
%\textsuperscript{(1247.4)}
\textsuperscript{113:6.6} El Ajustador os identificará; el serafín guardián os repersonalizará y luego os presentará de nuevo al fiel Monitor de vuestros días terrestres.

\par
%\textsuperscript{(1247.5)}
\textsuperscript{113:6.7} Y así, cuando termina una época planetaria, cuando se reúne a aquellos que se encuentran en los círculos inferiores de realización humana, sus guardianes colectivos son los que los reensamblan en las salas de resurrección de las esferas de las mansiones, tal como lo dicen vuestras escrituras: <<Y él enviará a sus ángeles con una voz poderosa y reunirá a sus escogidos desde un extremo al otro del reino>>\footnote{\textit{Enviará a sus ángeles y reunirá a sus escogidos}: Mt 24:31; Mc 13:27.}.

\par
%\textsuperscript{(1247.6)}
\textsuperscript{113:6.8} La técnica de la justicia exige que los guardianes personales o colectivos respondan al llamamiento nominal dispensacional en nombre de todas las personalidades no sobrevivientes. Los Ajustadores de esos no sobrevivientes no regresan, y cuando se pasa lista, los serafines responden, pero los Ajustadores no contestan. Esto constituye la <<resurrección de los injustos>>\footnote{\textit{Resurrección de los injustos}: Hch 24:15.}, en realidad el reconocimiento oficial del cese de la existencia de la criatura. Este llamamiento nominal de la justicia siempre tiene lugar inmediatamente después del llamamiento nominal de la misericordia, la resurrección de los supervivientes dormidos. Pero estos asuntos no incumben a nadie más que a los Jueces supremos y omniscientes de los valores de supervivencia. Estos problemas de decisiones judiciales no nos conciernen realmente.

\par
%\textsuperscript{(1247.7)}
\textsuperscript{113:6.9} Los guardianes colectivos pueden servir en un planeta durante una época tras otra, y convertirse finalmente en los conservadores de las almas dormidas de miles y miles de supervivientes dormidos. Pueden servir así en muchos mundos diferentes de un sistema determinado, puesto que la respuesta de la resurrección tiene lugar en los mundos de las mansiones.

\par
%\textsuperscript{(1247.8)}
\textsuperscript{113:6.10} Todos los guardianes personales y colectivos del sistema de Satania que se extraviaron durante la rebelión de Lucifer han de permanecer detenidos en Jerusem hasta el juicio final de la rebelión, a pesar de que muchos se arrepintieron sinceramente de su locura. Los Censores Universales ya han quitado arbitrariamente a estos guardianes desobedientes e infieles todos los aspectos de sus fideicomisos de almas, y han depositado la protección de estas realidades morontiales bajo la custodia de los seconafines voluntarios.

\section*{7. Los serafines y la carrera ascendente}
\par
%\textsuperscript{(1248.1)}
\textsuperscript{113:7.1} Este primer despertar en las orillas del mundo de las mansiones constituye en verdad un momento inolvidable en la carrera de un mortal ascendente; ver allí realmente por primera vez a vuestros compañeros angélicos, tanto tiempo amados y siempre presentes, de vuestros días en la Tierra; haceros también allí verdaderamente conscientes de la identidad y de la presencia del Monitor divino que durante tanto tiempo residió en vuestra mente en la Tierra. Una experiencia así constituye un despertar glorioso, una verdadera resurrección.

\par
%\textsuperscript{(1248.2)}
\textsuperscript{113:7.2} En las esferas morontiales, los serafines acompañantes (hay dos de ellos) son abiertamente vuestros compañeros. Estos ángeles no solamente se asocian con vosotros a medida que progresáis en la carrera de los mundos de transición, ayudándoos de todas las maneras posibles a adquirir el estado morontial y espiritual, sino que también aprovechan la ocasión para avanzar ellos mismos por medio del estudio en las escuelas de divulgación para serafines evolutivos que existen en los mundos de las mansiones.

\par
%\textsuperscript{(1248.3)}
\textsuperscript{113:7.3} La raza humana fue creada apenas un poco por debajo de los tipos más sencillos de órdenes angélicas\footnote{\textit{Creados por debajo de los ángeles}: Sal 8:4-5; Heb 2:6-7.}. Por eso, en el momento en que alcancéis la conciencia de la personalidad después de haber sido liberados de los vínculos de la carne, vuestra primera tarea en la vida morontial consistirá en ayudar a los serafines en el trabajo inmediato que espera.

\par
%\textsuperscript{(1248.4)}
\textsuperscript{113:7.4} Antes de dejar los mundos de las mansiones, todos los mortales tendrán unos asociados o guardianes seráficos permanentes. Y a medida que ascendáis las esferas morontiales, los guardianes seráficos serán finalmente los que atestiguarán y certificarán los decretos de vuestra unión eterna con el Ajustador del Pensamiento. Juntos han establecido la identidad de vuestra personalidad como hijo de la carne procedente de los mundos del tiempo. Luego, cuando alcancéis la madurez del estado morontial, os acompañarán a través de Jerusem y de los mundos asociados de progreso y de cultura del sistema. Después de esto, irán con vosotros a Edentia y a sus setenta esferas de vida social avanzada, y posteriormente os guiarán hasta los Melquisedeks y os seguirán a lo largo de la magnífica carrera en los mundos sede del universo. Cuando hayáis aprendido la sabiduría y la cultura de los Melquisedeks, os llevarán a Salvington, donde os encontraréis cara a cara con el Soberano de todo Nebadon. Estos guías seráficos os seguirán además a través del sector menor y de los sectores mayores del superuniverso, y continuarán hasta los mundos receptores de Uversa, permaneciendo con vosotros hasta que un seconafín os envuelva finalmente para el largo viaje a Havona.

\par
%\textsuperscript{(1248.5)}
\textsuperscript{113:7.5} Algunos guardianes del destino vinculados a los peregrinos ascendentes durante la carrera humana siguen el recorrido de éstos a través de Havona. Los demás se despiden temporalmente de sus asociados humanos de largo tiempo, y luego, mientras estos mortales atraviesan los círculos del universo central, sus guardianes del destino superan los círculos de Serafington. Y estarán esperando en las orillas del Paraíso cuando sus asociados mortales se despierten del último sueño temporal de tránsito a las nuevas experiencias de la eternidad. Estos serafines ascendentes emprenden posteriormente diferentes servicios en el cuerpo finalitario y en el Cuerpo Seráfico de la Finalización.

\par
%\textsuperscript{(1248.6)}
\textsuperscript{113:7.6} El hombre y el ángel pueden estar o no reunidos en el servicio eterno, pero dondequiera que sus misiones seráficas puedan llevarlos, los serafines siempre están en comunicación con sus antiguos pupilos de los mundos evolutivos, los mortales ascendentes del tiempo. Las asociaciones íntimas y los vínculos afectuosos de los mundos de origen humano no se olvidan nunca ni tampoco se rompen por completo. En las épocas eternas, los hombres y los ángeles cooperarán en el servicio divino tal como lo hicieron en la carrera del tiempo.

\par
%\textsuperscript{(1249.1)}
\textsuperscript{113:7.7} Para los serafines, la manera más segura de llegar hasta las Deidades del Paraíso consiste en guiar con éxito a un alma de origen evolutivo hasta las puertas del Paraíso. Por eso la misión como guardián del destino es la función seráfica más apreciada.

\par
%\textsuperscript{(1249.2)}
\textsuperscript{113:7.8} Sólo los guardianes del destino son enrolados en el Cuerpo primario, o mortal, de la Finalidad, y estas parejas han emprendido la aventura suprema de unificar sus identidades; los dos seres han conseguido la biunificación espiritual en Serafington antes de ser admitidos en el cuerpo finalitario. En esta experiencia, las dos naturalezas angélicas, tan complementarias en todas sus funciones universales, consiguen la unidad espiritual última de ser dos en uno, lo cual repercute en una nueva capacidad para recibir un fragmento no Ajustador del Padre Paradisiaco y fusionar con él. Y así, algunos de vuestros amorosos asociados seráficos en el tiempo se convierten también en vuestros asociados finalitarios en la eternidad, hijos del Supremo e hijos perfeccionados del Padre Paradisiaco.

\par
%\textsuperscript{(1249.3)}
\textsuperscript{113:7.9} [Presentado por el Jefe de los Serafines estacionados en Urantia.]


\chapter{Documento 114. El gobierno planetario de los serafines}
\par
%\textsuperscript{(1250.1)}
\textsuperscript{114:0.1} LOS ALTÍSIMOS gobiernan en los reinos de los hombres\footnote{\textit{Los Altísimos gobiernan}: Dn 4:17,25,32; 5:21.} por medio de muchas fuerzas y agentes celestiales, pero principalmente a través del ministerio de los serafines.

\par
%\textsuperscript{(1250.2)}
\textsuperscript{114:0.2} Hoy al mediodía, la lista nominal de ángeles planetarios, guardianes y otros en Urantia contenía 501.234.619 parejas de serafines. Estaban destinadas a mi mando doscientas huestes seráficas ---597.196.800 parejas de serafines o 1.194.393.600 ángeles individuales. El registro muestra sin embargo a 1.002.469.238 individuos; de ello se deduce por tanto que 191.294.362 ángeles estaban ausentes de este mundo en servicios relacionados con el transporte, los mensajes o la muerte. (En Urantia hay aproximadamente el mismo número de querubines que de serafines, y están organizados de manera similar.)

\par
%\textsuperscript{(1250.3)}
\textsuperscript{114:0.3} Los serafines y sus querubines asociados tienen mucho que ver con los detalles del gobierno superhumano de un planeta, especialmente en los mundos que han sido aislados por la rebelión. Los ángeles, ayudados hábilmente por los intermedios, ejercen su actividad en Urantia como verdaderos ministros supermateriales que ejecutan las órdenes del gobernador general residente y de todos sus asociados y subordinados. Los serafines, como clase, se ocupan de muchas tareas distintas a las de la custodia personal o colectiva.

\par
%\textsuperscript{(1250.4)}
\textsuperscript{114:0.4} Urantia no carece de una supervisión apropiada y eficaz por parte de los gobernantes de su sistema, su constelación y su universo. Pero su gobierno planetario es diferente al de cualquier otro mundo del sistema de Satania, e incluso de todo Nebadon. La singularidad de vuestro plan de supervisión se debe a una serie de circunstancias poco comunes:

\par
%\textsuperscript{(1250.5)}
\textsuperscript{114:0.5} 1. El estado de Urantia, donde la vida ha sido modificada.

\par
%\textsuperscript{(1250.6)}
\textsuperscript{114:0.6} 2. Las exigencias de la rebelión de Lucifer.

\par
%\textsuperscript{(1250.7)}
\textsuperscript{114:0.7} 3. Los trastornos ocasionados por la falta adámica.

\par
%\textsuperscript{(1250.8)}
\textsuperscript{114:0.8} 4. Las irregularidades derivadas del hecho de que Urantia ha sido uno de los mundos de donación del Soberano del Universo. Miguel de Nebadon es el Príncipe Planetario de Urantia.

\par
%\textsuperscript{(1250.9)}
\textsuperscript{114:0.9} 5. La función especial de los veinticuatro directores planetarios.

\par
%\textsuperscript{(1250.10)}
\textsuperscript{114:0.10} 6. El emplazamiento en el planeta de un circuito de arcángeles.

\par
%\textsuperscript{(1250.11)}
\textsuperscript{114:0.11} 7. El nombramiento más reciente de Maquiventa Melquisedek, en otro tiempo encarnado en Urantia, como Príncipe Planetario vicegerente.

\section*{1. La soberanía de Urantia}
\par
%\textsuperscript{(1250.12)}
\textsuperscript{114:1.1} La soberanía original de Urantia estaba en manos del soberano del sistema de Satania. Éste la delegó en primer lugar a una comisión mixta de Melquisedeks y de Portadores de Vida, y este grupo funcionó en Urantia hasta la llegada de un Príncipe Planetario debidamente nombrado. Después de la caída del Príncipe Caligastia, en la época de la rebelión de Lucifer, Urantia no tuvo unas relaciones seguras y estables con el universo local y sus divisiones administrativas hasta que Miguel no finalizó su donación en la carne, cuando el Unión de los Días lo proclamó Príncipe Planetario de Urantia. Esta proclamación fijó para siempre, en principio y con seguridad, el estado de vuestro mundo, pero el Hijo Creador Soberano no hizo ningún gesto en la práctica para administrar personalmente el planeta, aparte de establecer en Jerusem una comisión de veinticuatro antiguos urantianos\footnote{\textit{La comisión de Jerusem}: Ap 4:4; 11:16.} con autoridad para representarlo en el gobierno de Urantia y de todos los demás planetas en cuarentena del sistema. Un miembro de este consejo reside ahora permanentemente en Urantia como gobernador general residente\footnote{\textit{Gobierno de Urantia}: Ap 4:4,10; 5:8,14; 7:11; 11:16; 14:3; 19:4.}.

\par
%\textsuperscript{(1251.1)}
\textsuperscript{114:1.2} La autoridad como vicegerente para actuar en nombre de Miguel como Príncipe Planetario se ha conferido recientemente a Maquiventa Melquisedek, pero este Hijo del universo local no ha tomado la más pequeña medida para modificar el régimen planetario actual de las administraciones sucesivas de los gobernadores generales residentes.

\par
%\textsuperscript{(1251.2)}
\textsuperscript{114:1.3} Existen pocas probabilidades de que se lleve a cabo un cambio notable en el gobierno de Urantia durante la presente dispensación, a menos que el Príncipe Planetario vicegerente llegue para asumir las responsabilidades de su título. Algunos de nuestros asociados piensan que, en algún momento del cercano futuro, el plan de enviar a uno de los veinticuatro consejeros a Urantia para actuar como gobernador general será reemplazado por la llegada oficial de Maquiventa Melquisedek con el mandato de vicegerente de la soberanía de Urantia. Como Príncipe Planetario en funciones, continuará indudablemente a cargo del planeta hasta la sentencia final de la rebelión de Lucifer, y probablemente más allá hasta la época lejana del establecimiento del planeta en la luz y la vida.

\par
%\textsuperscript{(1251.3)}
\textsuperscript{114:1.4} Algunos creen que Maquiventa no vendrá a hacerse cargo de la dirección personal de los asuntos de Urantia hasta el final de la dispensación en curso. Otros sostienen que el Príncipe vicegerente no puede venir, como tal, hasta que Miguel regrese algún día a Urantia tal como lo prometió cuando vivía todavía en la carne. Otros aún, incluyendo a este narrador, esperan que Melquisedek aparezca en cualquier momento.

\section*{2. La junta de supervisores planetarios}
\par
%\textsuperscript{(1251.4)}
\textsuperscript{114:2.1} Desde la época de la donación de Miguel en vuestro mundo, la administración general de Urantia fue confiada a un grupo especial de veinticuatro antiguos urantianos en Jerusem. Los requisitos para ser miembro de esta comisión no los conocemos, pero hemos observado que todos aquellos que han sido nombrados así han contribuido a ampliar la soberanía del Supremo en el sistema de Satania. Todos eran, por naturaleza, auténticos dirigentes cuando ejercían su actividad en Urantia, y (a excepción de Maquiventa Melquisedek) estas dotes de mando se han acrecentado aún más mediante la experiencia en los mundos de las mansiones, y se han completado con el entrenamiento de la ciudadanía de Jerusem. Los miembros son designados para la junta de los veinticuatro por el gabinete de Lanaforge, apoyados por los Altísimos de Edentia, aprobados por el Centinela Designado de Jerusem, y nombrados por Gabriel de Salvington de acuerdo con los mandatos de Miguel. Las personas designadas con carácter temporal ejercen sus funciones de la misma manera plena que los miembros permanentes de esta comisión de supervisores especiales.

\par
%\textsuperscript{(1251.5)}
\textsuperscript{114:2.2} Esta junta de directores planetarios se ocupa especialmente de supervisar las actividades de este mundo derivadas del hecho de que Miguel experimentó aquí su donación final. Se mantienen en contacto estrecho e inmediato con Miguel mediante las actividades de enlace de cierta Brillante Estrella Vespertina, el mismo ser que acompañó a Jesús durante toda su donación como mortal.

\par
%\textsuperscript{(1252.1)}
\textsuperscript{114:2.3} En el momento actual, un tal Juan, conocido por vosotros como <<el Bautista>>, preside este consejo cuando celebra sus sesiones en Jerusem. Pero el jefe de oficio de este consejo es el Centinela Designado de Satania, el representante directo y personal del Inspector Asociado de Salvington y del Ejecutivo Supremo de Orvonton.

\par
%\textsuperscript{(1252.2)}
\textsuperscript{114:2.4} Los miembros de esta misma comisión de antiguos urantianos también actúan como supervisores consultivos de los otros treinta y seis mundos del sistema aislados por la rebelión; efectúan un servicio muy valioso manteniendo a Lanaforge, el Soberano del Sistema, en contacto estrecho y compasivo con los asuntos de estos planetas que permanecen todavía más o menos bajo el supercontrol de los Padres de la Constelación de Norlatiadek. Estos veinticuatro consejeros viajan con frecuencia de forma individual a cada uno de los planetas en cuarentena, especialmente a Urantia.

\par
%\textsuperscript{(1252.3)}
\textsuperscript{114:2.5} Cada uno de los otros mundos aislados está asesorado por unas comisiones similares de tamaño variable compuestas por sus antiguos habitantes, pero estas otras comisiones están subordinadas al grupo urantiano de los veinticuatro. Aunque los miembros de esta última comisión están activamente interesados así en todas las fases del progreso humano de cada mundo en cuarentena de Satania, se preocupan de manera especial y particular por el bienestar y el progreso de las razas mortales de Urantia, pues no supervisan inmediata y directamente los asuntos de ninguno de los otros planetas, exceptuando a Urantia, e incluso aquí su autoridad no es completa, salvo en algunas cuestiones relacionadas con la supervivencia de los mortales.

\par
%\textsuperscript{(1252.4)}
\textsuperscript{114:2.6} Nadie sabe cuánto tiempo seguirán estos veinticuatro consejeros de Urantia en su estado actual, separados del programa regular de actividades universales. Continuarán sirviendo sin duda en su calidad actual hasta que se produzca algún cambio en la situación planetaria, tal como el final de una dispensación, la toma de posesión de toda la autoridad por parte de Maquiventa Melquisedek, la sentencia final de la rebelión de Lucifer o la reaparición de Miguel en el mundo de su donación final. El actual gobernador general residente de Urantia parece inclinado a pensar que todos, salvo Maquiventa, podrían ser liberados para ascender hacia el Paraíso en el momento en que el sistema de Satania sea restablecido en los circuitos de la constelación. Pero existen también otras opiniones.

\section*{3. El gobernador general residente}
\par
%\textsuperscript{(1252.5)}
\textsuperscript{114:3.1} El cuerpo de los veinticuatro supervisores planetarios de Jerusem designa cada cien años del tiempo de Urantia a uno de sus miembros para que resida en vuestro mundo y actúe como su representante ejecutivo, como gobernador general residente. Este director ejecutivo fue cambiado durante la época en que se preparaban estas narraciones, y el vigésimo gobernador en asegurar este servicio reemplazó al décimo noveno. No os indicamos el nombre del supervisor planetario actual porque el hombre mortal es muy propenso a venerar, e incluso a deificar, a sus compatriotas extraordinarios y a sus superiores superhumanos.

\par
%\textsuperscript{(1252.6)}
\textsuperscript{114:3.2} El gobernador general residente no tiene ninguna autoridad personal real para dirigir los asuntos del mundo, salvo como representante de los veinticuatro consejeros de Jerusem. Actúa como coordinador de la administración superhumana y es el jefe respetado y el dirigente universalmente reconocido de los seres celestiales que ejercen sus funciones en Urantia. Todas las órdenes de huestes angélicas lo consideran como su director coordinador, mientras que los intermedios unidos, desde la partida de 1-2-3 el primero para convertirse en uno de los veinticuatro consejeros, consideran realmente a los gobernadores generales sucesivos como sus padres planetarios.

\par
%\textsuperscript{(1253.1)}
\textsuperscript{114:3.3} Aunque el gobernador general no posee una autoridad real y personal sobre el planeta, emite cada día decenas de fallos y decisiones que son aceptados como finales por todas las personalidades interesadas. Es mucho más un consejero paternal que un jefe técnico. En ciertos aspectos ejerce sus funciones como lo haría un Príncipe Planetario, pero su administración se parece mucho más a la de los Hijos Materiales.

\par
%\textsuperscript{(1253.2)}
\textsuperscript{114:3.4} El gobierno de Urantia está representado en los consejos de Jerusem con arreglo a un convenio mediante el cual el gobernador general que regresa participa como miembro temporal en el gabinete de los Príncipes Planetarios del Soberano del Sistema. Cuando Maquiventa fue nombrado Príncipe vicegerente, se esperaba que ocuparía inmediatamente su lugar en el consejo de los Príncipes Planetarios de Satania, pero hasta ahora no ha hecho ningún gesto en este sentido.

\par
%\textsuperscript{(1253.3)}
\textsuperscript{114:3.5} El gobierno supermaterial de Urantia no mantiene una relación orgánica muy estrecha con las unidades superiores del universo local. En cierto modo, el gobernador general residente representa a Salvington así como a Jerusem, puesto que actúa en nombre de los veinticuatro consejeros que representan directamente a Miguel y Gabriel. Y como es un ciudadano de Jerusem, el gobernador planetario puede ejercer su actividad como portavoz del Soberano del Sistema. Las autoridades de la constelación están representadas directamente por un Hijo Vorondadek, el observador de Edentia.

\section*{4. El Altísimo observador}
\par
%\textsuperscript{(1253.4)}
\textsuperscript{114:4.1} La soberanía de Urantia está complicada además por el hecho de que, poco después de la rebelión planetaria, el gobierno de Norlatiadek se incautó arbitrariamente en el pasado de la autoridad planetaria. Un Hijo Vorondadek reside todavía en Urantia como observador de los Altísimos de Edentia y, en ausencia de una acción directa por parte de Miguel, como fideicomisario de la soberanía planetaria. El observador Altísimo actual (y antiguo regente) es el vigesimotercero que sirve así en Urantia.

\par
%\textsuperscript{(1253.5)}
\textsuperscript{114:4.2} Ciertos grupos de problemas planetarios permanecen todavía bajo el control de los Altísimos de Edentia, pues la jurisdicción sobre ellos se empezó a ejercer en la época de la rebelión de Lucifer. Un Hijo Vorondadek, el observador de Norlatiadek, ejerce la autoridad sobre estos asuntos y mantiene relaciones consultivas muy estrechas con los supervisores planetarios. Los comisionados raciales son muy activos en Urantia, y sus diversos jefes de grupo están oficiosamente sujetos al observador Vorondadek residente, que actúa como su director consultivo.

\par
%\textsuperscript{(1253.6)}
\textsuperscript{114:4.3} En caso de crisis, el jefe real y soberano del gobierno, excepto en algunos asuntos puramente espirituales, sería este Hijo Vorondadek de Edentia actualmente de servicio como observador. (En estos problemas exclusivamente espirituales y en ciertos asuntos puramente personales, la autoridad suprema parece corresponder al arcángel comandante vinculado al cuartel general divisionario de esta orden, recientemente establecido en Urantia.)

\par
%\textsuperscript{(1253.7)}
\textsuperscript{114:4.4} Un observador Altísimo está facultado para hacerse cargo, a su juicio, del gobierno planetario en tiempos de grave crisis planetaria, y los archivos indican que esto ha sucedido treinta y tres veces en la historia de Urantia. En tales momentos, el observador Altísimo desempeña las funciones de regente Altísimo, ejerciendo una autoridad indiscutida sobre todos los ministros y administradores que residen en el planeta, exceptuando solamente a la organización divisionaria de los arcángeles.

\par
%\textsuperscript{(1253.8)}
\textsuperscript{114:4.5} Las regencias de los Vorondadeks no son típicas de los planetas aislados por la rebelión, ya que los Altísimos pueden intervenir en cualquier momento en los asuntos de los mundos habitados, interponiendo la sabiduría superior de los gobernantes de la constelación en los asuntos de los reinos de los hombres.

\section*{5. El gobierno planetario}
\par
%\textsuperscript{(1254.1)}
\textsuperscript{114:5.1} La administración actual de Urantia es realmente difícil de describir. No existe un gobierno oficial a la manera de la organización del universo, con sus departamentos legislativo, ejecutivo y judicial separados. Los veinticuatro consejeros es lo que más se parece a la rama legislativa del gobierno planetario. El gobernador general es un jefe ejecutivo provisional y consultivo, pero el derecho al veto reside en el observador Altísimo. No hay ningún poder judicial con una autoridad absoluta que funcione en el planeta ---sólo existen las comisiones de conciliación.

\par
%\textsuperscript{(1254.2)}
\textsuperscript{114:5.2} La mayoría de los problemas que surgen entre los serafines y los intermedios son resueltos, por consentimiento mutuo, por el gobernador general. Pero todas las decisiones de éste último, excepto cuando expresan los mandatos de los veinticuatro consejeros, están sujetas a apelación ante las comisiones de conciliación, ante las autoridades locales constituidas para el funcionamiento planetario, o incluso ante el Soberano del Sistema de Satania.

\par
%\textsuperscript{(1254.3)}
\textsuperscript{114:5.3} La ausencia del estado mayor corpóreo de un Príncipe Planetario y del régimen material de un Hijo y una Hija Adámicos está compensada parcialmente por el ministerio especial de los serafines y por los servicios excepcionales de las criaturas intermedias. La ausencia del Príncipe Planetario está eficazmente compensada por la presencia trina de los arcángeles, el observador Altísimo y el gobernador general.

\par
%\textsuperscript{(1254.4)}
\textsuperscript{114:5.4} Este gobierno planetario, organizado de una manera más bien imprecisa y administrado de una forma en cierto modo personal, es más eficaz de lo que se esperaba a causa del ahorro de tiempo que supone la ayuda de los arcángeles y su circuito siempre disponible, el cual se utiliza con mucha frecuencia en caso de emergencia planetaria o de dificultades administrativas. Técnicamente, el planeta está todavía espiritualmente aislado de los circuitos de Norlatiadek, pero en caso de emergencia, este obstáculo se puede ahora evitar utilizando el circuito de los arcángeles. El aislamiento planetario afecta poco, por supuesto, a los mortales individuales desde que el Espíritu de la Verdad fue derramado sobre todo el género humano hace mil novecientos años.

\par
%\textsuperscript{(1254.5)}
\textsuperscript{114:5.5} Cada jornada administrativa en Urantia empieza con una conferencia consultiva a la que asisten el gobernador general, el jefe planetario de los arcángeles, el observador Altísimo, el supernafín supervisor, el jefe de los Portadores de Vida residentes, y los huéspedes invitados escogidos entre los Hijos elevados del universo o algunos de los visitantes estudiantiles que pueden estar residiendo por casualidad en el planeta.

\par
%\textsuperscript{(1254.6)}
\textsuperscript{114:5.6} El gabinete administrativo directo del gobernador general está compuesto por doce serafines, los jefes en funciones de los doce grupos de ángeles especiales que ejercen su actividad como directores superhumanos inmediatos del progreso y de la estabilidad planetarios.

\section*{6. Los serafines maestros de la supervisión planetaria}
\par
%\textsuperscript{(1254.7)}
\textsuperscript{114:6.1} Cuando el primer gobernador general llegó a Urantia, coincidiendo con la efusión del Espíritu de la Verdad, venía acompañado de doce cuerpos de serafines especiales, graduados de Serafington, que fueron asignados inmediatamente a ciertos servicios planetarios especiales. Estos ángeles elevados son conocidos con el nombre de serafines maestros de la supervisión planetaria y, aparte del supercontrol del Altísimo observador planetario, se encuentran bajo la dirección inmediata del gobernador general residente.

\par
%\textsuperscript{(1255.1)}
\textsuperscript{114:6.2} Estos doce grupos de ángeles, aunque desempeñan su actividad bajo la supervisión general del gobernador general residente, están dirigidos directamente por el consejo seráfico de los doce, por los jefes en funciones de cada grupo. Este consejo sirve también como gabinete voluntario del gobernador general residente.

\par
%\textsuperscript{(1255.2)}
\textsuperscript{114:6.3} Presido este consejo de jefes seráficos como jefe planetario de los serafines, y soy un supernafín voluntario de la orden primaria, que sirve en Urantia como sucesor del antiguo jefe de las huestes angélicas del planeta que se rebeló en la época de la secesión de Caligastia.

\par
%\textsuperscript{(1255.3)}
\textsuperscript{114:6.4} Los doce cuerpos de serafines maestros de la supervisión planetaria funcionan en Urantia como sigue:

\par
%\textsuperscript{(1255.4)}
\textsuperscript{114:6.5} 1. \textit{Los ángeles de la época}. Son los ángeles de la época en curso, el grupo dispensacional. Estos ministros celestiales están encargados de vigilar y dirigir los asuntos de cada generación tal como están destinados a adaptarse al mosaico de la época en la que se producen. El cuerpo actual de ángeles de la época que sirve en Urantia es el tercer grupo asignado al planeta durante la dispensación en curso.

\par
%\textsuperscript{(1255.5)}
\textsuperscript{114:6.6} 2. \textit{Los ángeles del progreso}. Estos serafines tienen encomendada la tarea de iniciar el progreso evolutivo de las épocas sociales sucesivas. Fomentan el desarrollo de la tendencia progresiva inherente a las criaturas evolutivas; trabajan sin cesar para hacer que las cosas sean como debieran ser. El grupo que está ahora de servicio es el segundo que ha sido asignado al planeta.

\par
%\textsuperscript{(1255.6)}
\textsuperscript{114:6.7} 3. \textit{Los guardianes de la religión}. Son los <<ángeles de las iglesias>>\footnote{\textit{Ángeles de las iglesias}: Ap 1:20.}, los ardientes luchadores por lo que es y por lo que ha sido. Se esfuerzan por mantener los ideales de lo que ha sobrevivido, para que los valores morales puedan pasar con seguridad de una época a la siguiente. Son los jaque y mate de los ángeles del progreso, e intentan transferir constantemente, de una generación a la siguiente, los valores imperecederos de las formas antiguas y pasajeras a los modelos de pensamiento y de conducta nuevos y, por consiguiente, menos estabilizados. Estos ángeles luchan por las formas espirituales, pero no son la fuente del sectarismo excesivo ni de las polémicas divisiones sin sentido de las personas supuestamente religiosas. El cuerpo que trabaja ahora en Urantia es el quinto que sirve así\footnote{\textit{Ángel de la Iglesia}: Ap 2:1,8,12,18; Ap 3:1,7,14.}.

\par
%\textsuperscript{(1255.7)}
\textsuperscript{114:6.8} 4. \textit{Los ángeles de la vida nacional}. Son los <<ángeles de las trompetas>>\footnote{\textit{Ángeles de las trompetas}: Ap 8:2,6.}, los directores de las realizaciones políticas de la vida nacional en Urantia. El grupo que asegura actualmente el supercontrol de las relaciones internacionales es el cuarto cuerpo que sirve en el planeta. El ministerio de esta división seráfica es el que hace particularmente posible que <<los Altísimos gobiernen en los reinos de los hombres>>\footnote{\textit{Los Altísimos gobiernan}: Dn 4:17,25,32; 5:21.}.

\par
%\textsuperscript{(1255.8)}
\textsuperscript{114:6.9} 5. \textit{Los ángeles de las razas}. Son aquellos que trabajan para conservar las razas evolutivas del tiempo, sin tener en cuenta sus enredos políticos ni sus agrupaciones religiosas. En Urantia existen restos de nueve razas humanas que se han mezclado y combinado para formar los pueblos de los tiempos modernos. Estos serafines están estrechamente asociados al ministerio de los comisionados raciales, y el grupo que sirve actualmente en Urantia es el cuerpo original asignado al planeta poco después del día de Pentecostés.

\par
%\textsuperscript{(1255.9)}
\textsuperscript{114:6.10} 6. \textit{Los ángeles del futuro}. Son los ángeles de los proyectos, que pronostican una época futura y hacen planes para que se realicen las mejores cosas de una dispensación nueva y progresiva; son los arquitectos de las eras sucesivas. El grupo que se encuentra actualmente en el planeta ha funcionado así desde el comienzo de la dispensación en curso.

\par
%\textsuperscript{(1256.1)}
\textsuperscript{114:6.11} 7. \textit{Los ángeles de la iluminación}. Urantia recibe actualmente la ayuda del tercer cuerpo de serafines dedicados a fomentar la educación planetaria. Estos ángeles se ocupan de la formación mental y moral relacionada con los individuos, las familias, los grupos, las escuelas, las comunidades, las naciones y las razas enteras.

\par
%\textsuperscript{(1256.2)}
\textsuperscript{114:6.12} 8. \textit{Los ángeles de la salud}. Son los ministros seráficos destinados a ayudar a aquellos agentes humanos que están consagrados a promover la salud y a prevenir las enfermedades. El cuerpo actual es el sexto grupo que sirve durante esta dispensación.

\par
%\textsuperscript{(1256.3)}
\textsuperscript{114:6.13} 9. \textit{Los serafines del hogar}. Urantia disfruta actualmente de los servicios del quinto grupo de ministros angélicos dedicados a preservar y a hacer progresar el hogar, la institución fundamental de la civilización humana.

\par
%\textsuperscript{(1256.4)}
\textsuperscript{114:6.14} 10. \textit{Los ángeles de la industria}. Este grupo seráfico se ocupa de fomentar el desarrollo industrial y de mejorar las condiciones económicas entre los pueblos de Urantia. Este cuerpo ha sido reemplazado siete veces desde la donación de Miguel.

\par
%\textsuperscript{(1256.5)}
\textsuperscript{114:6.15} 11. \textit{Los ángeles de la diversión}. Son los serafines que fomentan los valores del entretenimiento, el humor y el descanso. Intentan elevar continuamente las diversiones recreativas del hombre y promover así la utilización más provechosa del tiempo libre humano. El cuerpo actual es el tercero de esta orden que ejerce su ministerio en Urantia.

\par
%\textsuperscript{(1256.6)}
\textsuperscript{114:6.16} 12. \textit{Los ángeles del ministerio superhumano}. Son los ángeles de los ángeles, los serafines que están destinados al ministerio de todas las otras vidas superhumanas que residen de manera temporal o permanente en el planeta. Este cuerpo ha servido desde el comienzo de la dispensación actual.

\par
%\textsuperscript{(1256.7)}
\textsuperscript{114:6.17} Cuando estos grupos de serafines maestros no están de acuerdo en materia de política o de procedimiento planetarios, el gobernador general resuelve habitualmente sus diferencias, pero todas las decisiones de este último están sujetas a apelación, según sea la naturaleza y la gravedad de los asuntos implicados en el desacuerdo.

\par
%\textsuperscript{(1256.8)}
\textsuperscript{114:6.18} Ninguno de estos grupos angélicos ejerce un control directo o arbitrario sobre el ámbito de su asignación. No pueden controlar totalmente los asuntos de sus campos de acción respectivos, pero pueden manipular las condiciones planetarias y asociar las circunstancias de tal manera, y de hecho lo hacen, que pueden influir favorablemente sobre las esferas de la actividad humana a las que están vinculados.

\par
%\textsuperscript{(1256.9)}
\textsuperscript{114:6.19} Los serafines maestros de la supervisión planetaria utilizan numerosos agentes para cumplir sus misiones. Actúan como cámaras de compensación para las ideas, como focalizadores de la mente y como promotores de proyectos. Son incapaces de introducir conceptos nuevos y más elevados en la mente humana, pero actúan con frecuencia para intensificar algún ideal superior que ya ha aparecido en un intelecto humano.

\par
%\textsuperscript{(1256.10)}
\textsuperscript{114:6.20} Pero aparte de estas numerosas formas de acción positiva, los serafines maestros aseguran el progreso planetario contra los peligros vitales mediante la movilización, la preparación y el mantenimiento del cuerpo de reserva del destino. La función principal de estos reservistas consiste en proteger el progreso evolutivo contra una interrupción; ellos representan las precauciones que las fuerzas celestiales han tomado contra las sorpresas; son una garantía contra los desastres.

\section*{7. El cuerpo de reserva del destino}
\par
%\textsuperscript{(1257.1)}
\textsuperscript{114:7.1} El cuerpo de reserva del destino está compuesto por hombres y mujeres que viven y que han sido admitidos al servicio especial de la administración superhumana de los asuntos del mundo. Este cuerpo se compone de los hombres y las mujeres de cada generación que son escogidos por los directores espirituales del planeta para ayudar a conducir el ministerio de misericordia y de sabiduría hasta los hijos del tiempo en los mundos evolutivos. En la dirección de los asuntos relacionados con los planes de ascensión, la costumbre general es de empezar a utilizar este enlace de criaturas volitivas mortales en cuanto son competentes y dignas de confianza para asumir estas responsabilidades. Por consiguiente, tan pronto como los hombres y las mujeres aparecen en el escenario de la acción temporal con una capacidad mental suficiente, un estado moral adecuado y la espiritualidad requerida, son rápidamente asignados como enlaces humanos, como ayudantes mortales, al grupo celestial apropiado de personalidades planetarias.

\par
%\textsuperscript{(1257.2)}
\textsuperscript{114:7.2} Cuando los seres humanos son elegidos como protectores del destino planetario, cuando se convierten en individuos esenciales en los planes que llevan a cabo los administradores del mundo, en ese momento el jefe planetario de los serafines confirma su vinculación temporal al cuerpo seráfico, y designa a unos guardianes personales del destino para que sirvan con estos reservistas mortales. Todos los reservistas tienen Ajustadores conscientes de sí mismos, y la mayoría de ellos ejercen su actividad en los círculos cósmicos superiores de consecución intelectual y de conquista espiritual.

\par
%\textsuperscript{(1257.3)}
\textsuperscript{114:7.3} Los mortales del planeta son escogidos para servir en el cuerpo de reserva del destino de los mundos habitados por las razones siguientes:

\par
%\textsuperscript{(1257.4)}
\textsuperscript{114:7.4} 1. Una capacidad especial para ser preparados en secreto para numerosas posibles misiones de emergencia en la dirección de las diversas actividades de los asuntos del mundo.

\par
%\textsuperscript{(1257.5)}
\textsuperscript{114:7.5} 2. Una dedicación incondicional a alguna causa especial social, económica, política, espiritual u otra, unida a la buena voluntad de servir sin esperar reconocimiento ni recompensas humanas.

\par
%\textsuperscript{(1257.6)}
\textsuperscript{114:7.6} 3. Poseer un Ajustador del Pensamiento con una extraordinaria variedad de talentos y con una probable experiencia preurantiana para enfrentarse a las dificultades planetarias y luchar contra situaciones inminentes de emergencia mundial.

\par
%\textsuperscript{(1257.7)}
\textsuperscript{114:7.7} Cada división del servicio celestial planetario tiene derecho a un cuerpo de enlace compuesto por estos mortales del destino. Un mundo habitado de tipo medio emplea setenta cuerpos del destino diferentes, que están íntimamente conectados con la dirección superhumana en curso de los asuntos de ese mundo. En Urantia hay doce cuerpos de reserva del destino, uno para cada uno de los grupos planetarios de supervisión seráfica.

\par
%\textsuperscript{(1257.8)}
\textsuperscript{114:7.8} Los doce grupos de reservistas urantianos del destino están compuestos por habitantes mortales de la esfera, que han sido formados para ocupar numerosas posiciones cruciales en la Tierra y se mantienen preparados para actuar en las posibles emergencias planetarias. Este cuerpo combinado consta ahora de 962 personas. El cuerpo más pequeño asciende a 41, y el más grande a 172. A excepción de menos de una veintena de personalidades de contacto, los miembros de este grupo único no tienen ninguna conciencia de estar preparados para una posible actuación en ciertas crisis planetarias. Estos reservistas mortales son elegidos por el cuerpo al que están respectivamente vinculados, y son entrenados y preparados de la misma manera en su mente profunda mediante la técnica combinada del ministerio del Ajustador del Pensamiento así como del guardián seráfico. Muchas veces, otras numerosas personalidades celestiales participan en este entrenamiento inconsciente, y en toda esta preparación especial los intermedios prestan unos servicios valiosos e indispensables.

\par
%\textsuperscript{(1258.1)}
\textsuperscript{114:7.9} En muchos mundos, las criaturas intermedias secundarias mejor adaptadas son capaces de establecer diversos grados de contacto con los Ajustadores del Pensamiento de ciertos mortales favorablemente constituidos, penetrando hábilmente en la mente donde reside el Ajustador. (Estas revelaciones fueron materializadas en la lengua inglesa de Urantia debido precisamente a este tipo de combinación fortuita de ajustes cósmicos.) Estos mortales con potencial de contacto de los mundos evolutivos son movilizados en los numerosos cuerpos de reserva y, hasta cierto punto, la civilización espiritual avanza y los Altísimos pueden gobernar en los reinos de los hombres gracias a estos pequeños grupos de personalidades con visión de futuro. Los hombres y las mujeres de estos cuerpos de reserva del destino tienen así diversos grados de contacto con sus Ajustadores a través del ministerio intermedio de las criaturas intermedias; pero estos mismos mortales son poco conocidos por sus semejantes, salvo en aquellas raras emergencias sociales y urgencias espirituales en las que estas personalidades de reserva actúan para impedir la interrupción de la cultura evolutiva o la extinción de la luz de la verdad viviente. En Urantia, estos reservistas del destino raramente han sido ensalzados en las páginas de la historia humana.

\par
%\textsuperscript{(1258.2)}
\textsuperscript{114:7.10} Los reservistas actúan inconscientemente como conservadores de los conocimientos planetarios esenciales. Muchas veces, en el momento de la muerte de un reservista se efectúa un trasvase de ciertos datos vitales, desde la mente del reservista moribundo hasta un sucesor más joven, por medio de una conexión entre sus dos Ajustadores del Pensamiento. Los Ajustadores ejercen sin duda su actividad con estos cuerpos de reserva de otras muchas maneras desconocidas para nosotros.

\par
%\textsuperscript{(1258.3)}
\textsuperscript{114:7.11} Aunque el cuerpo de reserva del destino no tiene un jefe permanente en Urantia, tiene sus propios consejos permanentes que constituyen su organización gubernamental. Éstos abarcan el consejo judicial, el consejo de la historicidad, el consejo de la soberanía política y otros muchos. De vez en cuando, y de acuerdo con la organización del cuerpo, estos consejos permanentes han nombrado a unos jefes titulares (mortales) de todo el cuerpo de reserva para una función específica. La ocupación de estos jefes reservistas es un asunto que dura generalmente pocas horas, estando limitada a la realización de alguna tarea específica e inmediata.

\par
%\textsuperscript{(1258.4)}
\textsuperscript{114:7.12} El cuerpo de reserva de Urantia tuvo su mayor número de miembros en los tiempos de los adamitas y los anditas, disminuyendo constantemente con la dilución de la sangre violeta, y alcanzando su punto más bajo hacia la época de Pentecostés; desde entonces, los miembros del cuerpo de reserva han aumentado constantemente.

\par
%\textsuperscript{(1258.5)}
\textsuperscript{114:7.13} (El cuerpo de reserva cósmico de ciudadanos conscientes del universo en Urantia asciende actualmente a más de mil mortales, cuya perspicacia de la ciudadanía cósmica trasciende de lejos la esfera de su residencia terrestre, pero me está prohibido revelar la verdadera naturaleza de la función de este grupo excepcional de seres humanos vivientes.)

\par
%\textsuperscript{(1258.6)}
\textsuperscript{114:7.14} Los mortales de Urantia no deberían permitir que el aislamiento espiritual relativo de su mundo respecto a ciertos circuitos del universo local les produzca un sentimiento de abandono cósmico o de orfandad planetaria. En el planeta se encuentra operativa una supervisión superhumana muy definida y eficaz de los asuntos del mundo y de los destinos humanos.

\par
%\textsuperscript{(1258.7)}
\textsuperscript{114:7.15} Pero es cierto que, en el mejor de los casos, sólo podéis tener una idea insuficiente de un gobierno planetario ideal. Desde los primeros tiempos del Príncipe Planetario, Urantia ha sufrido el aborto del plan divino para el crecimiento del mundo y el desarrollo racial. Los mundos habitados leales de Satania no están gobernados como Urantia. Sin embargo, en comparación con los otros mundos aislados, vuestros gobiernos planetarios no han sido tan inferiores; se puede decir que sólo en uno o dos mundos son peores, y que en unos pocos pueden ser ligeramente mejores, pero la mayoría se encuentran en un nivel de igualdad con vosotros.

\par
%\textsuperscript{(1259.1)}
\textsuperscript{114:7.16} Nadie parece saber, en el universo local, cuándo terminará el estado inestable de la administración planetaria. Los Melquisedeks de Nebadon tienden a opinar que se producirán pocos cambios en el gobierno y la administración del planeta hasta la segunda venida personal de Miguel a Urantia. Es indudable que en ese momento, si no antes, se realizarán unos cambios radicales en la gestión del planeta. Pero en cuanto a la naturaleza de estas modificaciones en la administración del mundo, nadie parece ser capaz de hacer ni siquiera una conjetura. No existe ningún precedente de un episodio así en toda la historia de los mundos habitados del universo de Nebadon. Entre las numerosas cosas difíciles de comprender acerca del futuro gobierno de Urantia, una de las más sobresalientes es la instalación en el planeta de un circuito y de un cuartel general divisionario de arcángeles.

\par
%\textsuperscript{(1259.2)}
\textsuperscript{114:7.17} Vuestro mundo aislado no está olvidado en los consejos del universo. Urantia no es una huérfana cósmica estigmatizada por el pecado y excluida, por la rebelión, de los vigilantes cuidados divinos. Desde Uversa hasta Salvington y continuando hacia abajo hasta Jerusem, e incluso en Havona y en el Paraíso, todos saben que estamos aquí; y vosotros los mortales que vivís actualmente en Urantia, sois amados con el mismo afecto y cuidados, con la misma fidelidad, e incluso más, que si esta esfera no hubiera sido nunca traicionada por un Príncipe Planetario desleal. Es eternamente cierto que <<el Padre mismo os ama>>\footnote{\textit{El Padre mismo os ama}: Jn 16:27.}.

\par
%\textsuperscript{(1259.3)}
\textsuperscript{114:7.18} [Presentado por el Jefe de los Serafines estacionados en Urantia.]


\chapter{Documento 115. El Ser Supremo}
\par
%\textsuperscript{(1260.1)}
\textsuperscript{115:0.1} CON Dios Padre, la gran relación que existe es la filiación. Con Dios Supremo, la realización es el requisito previo para conseguir una posición ---uno tiene que hacer algo, así como ser algo.

\section*{1. Relatividad de los marcos conceptuales}
\par
%\textsuperscript{(1260.2)}
\textsuperscript{115:1.1} Los intelectos parciales, incompletos y evolutivos se encontrarían impotentes en el universo maestro, serían incapaces de formar el más mínimo modelo de pensamiento racional si no fuera porque todas las mentes, superiores o inferiores, tienen la capacidad innata de construir un \textit{marco universal} dentro del cual poder pensar. Si la mente no puede sacar conclusiones, si no puede penetrar hasta los verdaderos orígenes, entonces dicha mente dará infaliblemente por sentadas las conclusiones y se inventará los orígenes a fin de poder tener un medio de pensamiento lógico dentro del marco de esos postulados creados por la mente. Aunque estos marcos universales para el pensamiento de las criaturas son indispensables para las operaciones intelectuales racionales, todos son erróneos en mayor o menor grado, sin ninguna excepción.

\par
%\textsuperscript{(1260.3)}
\textsuperscript{115:1.2} Los marcos conceptuales del universo sólo son relativamente verdaderos; son unos andamios útiles que al final deben ceder el paso a la expansión de una comprensión cósmica más amplia. Las maneras de comprender la verdad, la belleza y la bondad, la moral, la ética, el deber, el amor, la divinidad, el origen, la existencia, la finalidad, el destino, el tiempo, el espacio, e incluso la Deidad, sólo son relativamente exactas. Dios es mucho, mucho más que un Padre, pero el Padre es el concepto humano más elevado de Dios; no obstante, la descripción de las relaciones entre el Creador y la criatura, como las que existen entre el Padre y el Hijo, se acrecentará gracias a los conceptos supermortales de la Deidad que se alcanzarán en Orvonton, en Havona y en el Paraíso. El hombre está obligado a pensar dentro de un marco universal humano, pero esto no significa que no pueda imaginar otros marcos más elevados dentro de los cuales pueda tener lugar el pensamiento.

\par
%\textsuperscript{(1260.4)}
\textsuperscript{115:1.3} Con el objeto de facilitar la comprensión humana del universo de universos, los diversos niveles de la realidad cósmica han sido denominados finito, absonito y absoluto. De todos ellos, sólo el nivel absoluto es incondicionalmente eterno, realmente existencial. Los absonitos y los finitos son derivados, modificaciones, limitaciones y atenuaciones de la realidad absoluta, original y primordial, de la infinidad.

\par
%\textsuperscript{(1260.5)}
\textsuperscript{115:1.4} Los reinos de lo finito existen en virtud del propósito eterno de Dios. Las criaturas finitas, superiores e inferiores, pueden proponer teorías, y así lo han hecho, sobre la necesidad de lo finito en la economía cósmica, pero a fin de cuentas lo finito existe porque Dios lo ha querido así. El universo no tiene explicación, y una criatura finita tampoco puede ofrecer un motivo racional para su propia existencia individual sin recurrir a los actos anteriores y a la volición preexistente de unos seres ancestrales, Creadores o procreadores.

\section*{2. La base absoluta para la supremacía}
\par
%\textsuperscript{(1261.1)}
\textsuperscript{115:2.1} Desde el punto de vista existencial, nada nuevo puede suceder en ninguna de las galaxias, pues la perfección de la infinidad inherente al YO SOY está eternamente presente en los siete Absolutos, funcionalmente asociada en las triunidades y asociada de manera transmisible en las triodidades. Pero el hecho de que la infinidad esté así existencialmente presente en estas asociaciones absolutas no impide de ninguna manera dar nacimiento a nuevos seres experienciales cósmicos. Desde el punto de vista de las criaturas finitas, la infinidad contiene muchas cosas que son potenciales, muchas cosas que pertenecen a las posibilidades futuras, en lugar de ser unas realidades presentes.

\par
%\textsuperscript{(1261.2)}
\textsuperscript{115:2.2} El valor es un elemento único en la realidad universal. No comprendemos cómo el valor de algo que es infinito y divino tendría la posibilidad de crecer. Pero descubrimos que \textit{los significados} se pueden modificar, si no acrecentar, incluso en las relaciones de la Deidad infinita. Para los universos experienciales, incluso los valores divinos crecen en forma de manifestaciones gracias a una mayor comprensión de los significados de la realidad.

\par
%\textsuperscript{(1261.3)}
\textsuperscript{115:2.3} Todo el proyecto de la creación y de la evolución universales, en todos los niveles experienciales, es aparentemente una cuestión de conversión de las potencialidades en manifestaciones; y esta transmutación concierne por igual a los dominios de la potencia espacial, de la potencia mental y de la potencia espiritual.

\par
%\textsuperscript{(1261.4)}
\textsuperscript{115:2.4} El método aparente por medio del cual las posibilidades del cosmos surgen a la existencia real varía de nivel en nivel; en el finito, se trata de la evolución experiencial, y en el absonito, de la existenciación experiencial. La infinidad existencial lo incluye verdaderamente todo sin restricción, y esta misma omni-inclusividad debe abarcar forzosamente incluso la posibilidad de efectuar experiencias evolutivas finitas. La posibilidad de este crecimiento experiencial se convierte en una realidad universal gracias a las relaciones de triodidad que inciden en el Supremo.

\section*{3. Lo original, lo manifestado y lo potencial}
\par
%\textsuperscript{(1261.5)}
\textsuperscript{115:3.1} Conceptualmente hablando, el cosmos absoluto no tiene límites; definir la extensión y la naturaleza de esta realidad primordial es ponerle limitaciones a la infinidad y atenuar el puro concepto de la eternidad. La idea de lo infinito eterno, de lo eterno infinito, es incalificada en extensión y absoluta de hecho. No existe un lenguaje en Urantia pasado, presente o futuro que sea adecuado para expresar la realidad de la infinidad o la infinidad de la realidad. El hombre, una criatura finita dentro de un cosmos infinito, tiene que contentarse con reflejos distorsionados y conceptos atenuados de esa existencia sin límites, sin trabas, sin principio ni fin, que sobrepasa realmente su capacidad de comprensión.

\par
%\textsuperscript{(1261.6)}
\textsuperscript{115:3.2} La mente no puede nunca esperar captar el concepto de un Absoluto sin intentar primero fragmentar la unidad de esa realidad. La mente unifica todas las divergencias, pero en ausencia total de tales divergencias, la mente no encuentra ninguna base para intentar formular conceptos comprensibles.

\par
%\textsuperscript{(1261.7)}
\textsuperscript{115:3.3} La estasis primordial de la infinidad necesita ser segmentada antes de que el ser humano intente comprenderla. La infinidad posee una unidad que en estos documentos ha sido denominada el YO SOY ---el primer postulado de la mente de las criaturas. Pero una criatura nunca podrá comprender cómo puede ser que esta unidad se convierta en una dualidad, una triunidad y una diversidad, y continúe siendo al mismo tiempo una unidad incalificada. El hombre se encuentra con un problema similar cuando se detiene a contemplar la Deidad indivisa de la Trinidad al lado de la personalización múltiple de Dios.

\par
%\textsuperscript{(1262.1)}
\textsuperscript{115:3.4} La distancia que separa al hombre de la infinidad es la única que ocasiona que este concepto sea expresado en una sola palabra. Aunque la infinidad es por una parte una UNIDAD, por otra es una DIVERSIDAD sin fin ni límites. La infinidad, tal como es observada por las inteligencias finitas, es la máxima paradoja de la filosofía de las criaturas y de la metafísica finita. Aunque la naturaleza espiritual del hombre se eleva, en la experiencia de la adoración, hacia el Padre que es infinito, la capacidad de comprensión intelectual del hombre queda agotada ante el concepto máximo del Ser Supremo. Más allá del Supremo, los conceptos se convierten cada vez más en simples nombres; cada vez definen con menos veracidad la realidad, y se transforman cada vez más en la proyección de la comprensión finita de las criaturas hacia lo superfinito.

\par
%\textsuperscript{(1262.2)}
\textsuperscript{115:3.5} Una concepción básica del nivel absoluto implica un postulado de tres fases:

\par
%\textsuperscript{(1262.3)}
\textsuperscript{115:3.6} 1. \textit{Lo Original}. El concepto incalificado de la Fuente-Centro Primera, esa manifestación original del YO SOY de la que surge toda la realidad.

\par
%\textsuperscript{(1262.4)}
\textsuperscript{115:3.7} 2. \textit{Lo Manifestado}. La unión de los tres Absolutos manifestados, los Orígenes-Centros Segundo, Tercero y Paradisíaco. Esta triodidad compuesta por el Hijo Eterno, el Espíritu Infinito y la Isla del Paraíso constituye la revelación manifestada de la originalidad de la Fuente-Centro Primera.

\par
%\textsuperscript{(1262.5)}
\textsuperscript{115:3.8} 3. \textit{Lo Potencial}. La unión de los tres Absolutos de potencialidad, los Absolutos de la Deidad, Incalificado y Universal. Esta triodidad de potencialidad existencial constituye la revelación potencial de la originalidad de la Fuente-Centro Primera.

\par
%\textsuperscript{(1262.6)}
\textsuperscript{115:3.9} La interasociación de lo Original, lo Manifestado y lo Potencial produce las tensiones, dentro de la infinidad, que dan como resultado la posibilidad de todo crecimiento universal; y el crecimiento es la naturaleza del Séptuple, del Supremo y del Último.

\par
%\textsuperscript{(1262.7)}
\textsuperscript{115:3.10} En la asociación de los Absolutos de la Deidad, Universal e Incalificado, la potencialidad es absoluta mientras que la manifestación es emergente; en la asociación de los Orígenes-Centros Segundo, Tercero y Paradisíaco, la manifestación es absoluta mientras que la potencialidad es emergente; en la originalidad de la Fuente-Centro Primera, no podemos decir si la manifestación o la potencialidad son existentes o emergentes ---\textit{el Padre es}.

\par
%\textsuperscript{(1262.8)}
\textsuperscript{115:3.11} Desde el punto de vista temporal, lo Manifestado es lo que era y lo que es; lo Potencial es lo que está surgiendo y lo que será; lo Original es lo que es. Desde el punto de vista de la eternidad, las diferencias entre lo Original, lo Manifestado y lo Potencial no son tan evidentes. Estas cualidades trinas no se distinguen así en los niveles de eternidad del Paraíso. En la eternidad, todo es ---sólo que todo aún no ha sido revelado en el tiempo y el espacio.

\par
%\textsuperscript{(1262.9)}
\textsuperscript{115:3.12} Desde el punto de vista de las criaturas, lo manifestado es la sustancia y la potencialidad es la capacidad. Lo manifestado existe en el centro mismo y desde allí se expande hacia la infinidad periférica; la potencialidad viene desde la periferia de la infinidad hacia el interior y converge en el centro de todas las cosas. La originalidad es aquello que primero causa y luego equilibra los dobles movimientos del ciclo de la metamorfosis de la realidad, transformando los potenciales en manifestaciones y convirtiendo en potencialidades las manifestaciones existentes.

\par
%\textsuperscript{(1262.10)}
\textsuperscript{115:3.13} Los tres Absolutos de potencialidad actúan en el nivel puramente eterno del cosmos, y por lo tanto nunca ejercen su actividad como tales en los niveles subabsolutos. En los niveles descendentes de la realidad, la triodidad de potencialidad se manifiesta con el Último y después del Supremo. Lo potencial quizás no logre manifestarse en el tiempo con respecto a una parte en algún nivel subabsoluto, pero nunca sucede así en el conjunto. La voluntad de Dios prevalece al final, no siempre en lo que concierne al individuo, pero invariablemente en lo que se refiere a la totalidad.

\par
%\textsuperscript{(1263.1)}
\textsuperscript{115:3.14} Todo lo que existe en el cosmos tiene su centro en la triodidad de lo manifestado; ya se trate del espíritu, de la mente o de la energía, todos están centrados en esta asociación compuesta por el Hijo, el Espíritu y el Paraíso. La personalidad del Hijo espiritual es el arquetipo maestro para todas las personalidades en todos los universos. La sustancia de la Isla del Paraíso es el arquetipo maestro del que Havona es una revelación perfecta, y los superuniversos una revelación en vías de perfeccionarse. El Actor Conjunto es al mismo tiempo el activador mental de la energía cósmica, el que transforma en conceptos las intenciones espirituales, y el que integra las causas y los efectos matemáticos de los niveles materiales con las intenciones y los móviles volitivos del nivel espiritual. En y para un universo finito, el Hijo, el Espíritu y el Paraíso ejercen su función en y sobre el Último, tal como éste se encuentra condicionado y atenuado en el Supremo.

\par
%\textsuperscript{(1263.2)}
\textsuperscript{115:3.15} La manifestación (de la Deidad) es lo que el hombre busca en su ascensión al Paraíso. La potencialidad (de la divinidad humana) es lo que el hombre desarrolla en esa búsqueda. Lo Original es lo que hace posible la coexistencia y la integración del hombre manifestado, del hombre potencial y del hombre eterno.

\par
%\textsuperscript{(1263.3)}
\textsuperscript{115:3.16} La dinámica final del cosmos consiste en trasvasar continuamente la realidad desde el estado potencial al estado manifestado. En teoría, esta metamorfosis debería tener un final, pero de hecho eso es imposible, porque tanto lo Potencial como lo Manifestado forman parte del circuito de lo Original (del YO SOY), y esta identificación impide para siempre ponerle límites al desarrollo progresivo del universo. Todo lo que está identificado con el YO SOY no puede dejar de progresar nunca, porque la manifestación de los potenciales del YO SOY es absoluta, y la potencialidad de las manifestaciones también lo es. Las manifestaciones siempre estarán abriendo nuevos caminos para que los potenciales, hasta entonces imposibles, se conviertan en realidades ---cada decisión humana no sólo hace que se manifieste una nueva realidad en la experiencia humana, sino que desarrolla también una nueva capacidad para el crecimiento humano. En cada niño vive un hombre, y en el hombre maduro que conoce a Dios reside el ascendente morontial.

\par
%\textsuperscript{(1263.4)}
\textsuperscript{115:3.17} La estática en el crecimiento nunca puede aparecer en la totalidad del cosmos, porque la base para el crecimiento ---las manifestaciones absolutas--- es incalificada, y porque las posibilidades para el crecimiento ---los potenciales absolutos--- son ilimitadas. Desde un punto de vista práctico, los filósofos del universo han llegado a la conclusión de que no existe nada que se pueda considerar como un \textit{final}.

\par
%\textsuperscript{(1263.5)}
\textsuperscript{115:3.18} Desde una visión circunscrita, existen en realidad muchas finalizaciones, muchas terminaciones de actividad, pero desde el punto de vista más amplio de un nivel superior del universo, no hay nada que termine, sino simplemente transiciones entre una fase de desarrollo y la siguiente. La cronicidad principal del universo maestro concierne a las diversas épocas del universo, las eras de Havona, de los superuniversos y de los universos exteriores. Pero incluso estas divisiones básicas de las relaciones secuenciales no pueden ser más que balizas relativas en la autovía interminable de la eternidad.

\par
%\textsuperscript{(1263.6)}
\textsuperscript{115:3.19} Para la criatura que progresa, la penetración final de la verdad, la belleza y la bondad del Ser Supremo sólo puede revelar aquellas cualidades absonitas de la divinidad última que están situadas más allá de los niveles conceptuales de la verdad, la belleza y la bondad.

\section*{4. Los orígenes de la realidad Suprema}
\par
%\textsuperscript{(1263.7)}
\textsuperscript{115:4.1} Cualquier análisis de los \textit{orígenes} de Dios Supremo debe empezar por la Trinidad del Paraíso, porque la Trinidad es la Deidad original, mientras que el Supremo es una Deidad derivada. Cualquier estudio sobre el \textit{crecimiento} del Supremo debe tomar en consideración a las triodidades existenciales, porque éstas abarcan todo lo manifestado absoluto y toda la potencialidad infinita (en conjunción con la Fuente-Centro Primera). El Supremo evolutivo es el foco culminante y personalmente volitivo de la transmutación ---la transformación--- de los potenciales en manifestaciones en y sobre el nivel de existencia finito. Las dos triodidades, la manifestada y la potencial, abarcan la totalidad de las relaciones recíprocas del crecimiento en los universos.

\par
%\textsuperscript{(1264.1)}
\textsuperscript{115:4.2} La fuente del Supremo se encuentra en la Trinidad del Paraíso ---en la Deidad eterna, manifestada e indivisa. El Supremo es ante todo una persona espiritual, y esta persona espiritual se deriva de la Trinidad. Pero el Supremo es en segundo lugar una Deidad de crecimiento ---de crecimiento evolutivo--- y este crecimiento procede de las dos triodidades, la manifestada y la potencial.

\par
%\textsuperscript{(1264.2)}
\textsuperscript{115:4.3} Si es difícil comprender que las triodidades infinitas pueden ejercer su actividad en el nivel finito, deteneos a considerar que esta misma infinidad debe contener en sí misma la potencialidad de lo finito; la infinidad abarca todas las cosas que se extienden desde la existencia finita más humilde y limitada hasta las realidades incondicionalmente absolutas más elevadas.

\par
%\textsuperscript{(1264.3)}
\textsuperscript{115:4.4} No es tan difícil comprender que lo infinito contiene de hecho a lo finito, sino entender exactamente de qué manera ese infinito se manifiesta realmente a lo finito. Pero los Ajustadores del Pensamiento que residen en los hombres mortales son una de las pruebas eternas de que incluso el Dios absoluto (como absoluto) puede ponerse en contacto directo, y así lo hace, incluso con las criaturas volitivas más humildes e insignificantes de todo el universo.

\par
%\textsuperscript{(1264.4)}
\textsuperscript{115:4.5} Las triodidades que abarcan colectivamente lo manifestado y lo potencial se manifiestan en el nivel finito en conjunción con el Ser Supremo. La técnica que emplean para manifestarse así es a la vez directa e indirecta: es directa en la medida en que las relaciones trioditarias repercuten directamente en el Supremo, e indirecta en la medida en que se derivan del nivel existenciado de lo absonito.

\par
%\textsuperscript{(1264.5)}
\textsuperscript{115:4.6} La realidad Suprema, que es la realidad finita total, está en proceso de crecimiento dinámico entre los potenciales incalificados del espacio exterior y las manifestaciones incalificadas que se encuentran en el centro de todas las cosas. El dominio finito se convierte así en un hecho gracias a la cooperación de los agentes absonitos del Paraíso y las Personalidades Creadoras Supremas del tiempo. El acto de hacer madurar las posibilidades restringidas de los tres grandes Absolutos potenciales es la ocupación absonita de los Arquitectos del Universo Maestro y de sus asociados trascendentales. Cuando estas eventualidades han alcanzado cierto grado de madurez, las Personalidades Creadoras Supremas salen del Paraíso para emprender la tarea secular de traer a la existencia real a los universos evolutivos.

\par
%\textsuperscript{(1264.6)}
\textsuperscript{115:4.7} El crecimiento de la Supremacía se deriva de las triodidades, y la persona espiritual del Supremo, de la Trinidad; pero las prerrogativas de poder del Todopoderoso están basadas en los logros divinos de Dios Séptuple, mientras que la unión de las prerrogativas de poder del Todopoderoso Supremo y la persona espiritual de Dios Supremo tiene lugar en virtud del ministerio del Actor Conjunto, que donó la mente del Supremo como factor de unión en esta Deidad evolutiva.

\section*{5. Relación del Supremo con la Trinidad del Paraíso}
\par
%\textsuperscript{(1264.7)}
\textsuperscript{115:5.1} El Ser Supremo depende de manera absoluta de la existencia y de los actos de la Trinidad del Paraíso para que su naturaleza personal y espiritual sean reales. Aunque el crecimiento del Supremo es una cuestión de relación con las triodidades, la personalidad espiritual de Dios Supremo depende, y se deriva, de la Trinidad del Paraíso, que siempre seguirá siendo la fuente-centro absoluta de la estabilidad perfecta e infinita alrededor de la cual se desarrolla progresivamente el crecimiento evolutivo del Supremo.

\par
%\textsuperscript{(1265.1)}
\textsuperscript{115:5.2} La actividad de la Trinidad está relacionada con la actividad del Supremo, porque la Trinidad actúa en todos los niveles (en la totalidad de ellos), incluido el nivel de actividad de la Supremacía. Pero al igual que la era de Havona cede el paso a la era de los superuniversos, la acción discernible de la Trinidad, como creadora inmediata, cede el paso a los actos creativos de los hijos de las Deidades del Paraíso.

\section*{6. Relación del Supremo con las triodidades}
\par
%\textsuperscript{(1265.2)}
\textsuperscript{115:6.1} La triodidad de lo manifestado continúa actuando directamente en las épocas posteriores a Havona; la gravedad del Paraíso sujeta las unidades básicas de la existencia material, la gravedad espiritual del Hijo Eterno actúa directamente sobre los valores fundamentales de la existencia espiritual, y la gravedad mental del Actor Conjunto aferra infaliblemente todos los significados vitales de la existencia intelectual.

\par
%\textsuperscript{(1265.3)}
\textsuperscript{115:6.2} Pero a medida que cada etapa de la actividad creativa avanza en el espacio inexplorado, dicha actividad existe y se ejerce cada vez más lejos de la acción directa de las fuerzas creativas y de las personalidades divinas del emplazamiento central ---la Isla absoluta del Paraíso y las Deidades infinitas que residen allí. Estos niveles sucesivos de existencia cósmica dependen por lo tanto cada vez más de los desarrollos que se produzcan dentro de las tres potencialidades absolutas de la infinidad.

\par
%\textsuperscript{(1265.4)}
\textsuperscript{115:6.3} El Ser Supremo contiene unas posibilidades para el ministerio cósmico que no están aparentemente manifestadas en el Hijo Eterno, en el Espíritu Infinito, o en las realidades no personales de la Isla del Paraíso. Hacemos esta afirmación con la debida consideración por la absolutidad de estas tres manifestaciones fundamentales, pero el crecimiento del Supremo no está basado solamente en estas manifestaciones de la Deidad y del Paraíso, sino que también está implicado en los desarrollos internos de los Absolutos de la Deidad, Universal e Incalificado.

\par
%\textsuperscript{(1265.5)}
\textsuperscript{115:6.4} El Supremo no crece solamente a medida que los Creadores y las criaturas de los universos evolutivos logran parecerse a Dios, sino que esta Deidad finita también experimenta el crecimiento como resultado del dominio que los Creadores y las criaturas han conseguido sobre las posibilidades finitas del gran universo. El movimiento del Supremo es doble: hacia el interior, es decir, hacia el Paraíso y la Deidad, y hacia el exterior, es decir, hacia lo ilimitado de los Absolutos de lo potencial.

\par
%\textsuperscript{(1265.6)}
\textsuperscript{115:6.5} En la era actual del universo, este doble movimiento se revela en las personalidades descendentes y ascendentes del gran universo. Las Personalidades Creadoras Supremas y todos sus asociados divinos reflejan el movimiento hacia el exterior y divergente del Supremo, mientras que los peregrinos ascendentes de los siete superuniversos indican la tendencia hacia el interior y convergente de la Supremacía.

\par
%\textsuperscript{(1265.7)}
\textsuperscript{115:6.6} La Deidad finita busca siempre una doble correlación: hacia el interior, es decir, hacia el Paraíso y sus Deidades, y hacia el exterior, es decir, hacia la infinidad y los Absolutos que se hallan en ella. La poderosa erupción de la divinidad creativa del Paraíso, que se personaliza en los Hijos Creadores y manifiesta su poder en los controladores de poder, indica la enorme oleada de Supremacía hacia los dominios de la potencialidad, mientras que la interminable procesión de las criaturas ascendentes del gran universo atestigua la poderosa oleada de Supremacía hacia la unidad con la Deidad del Paraíso.

\par
%\textsuperscript{(1265.8)}
\textsuperscript{115:6.7} Los seres humanos han aprendido que a veces se puede discernir el movimiento de lo invisible observando sus efectos sobre lo visible; y nosotros hace tiempo que hemos aprendido a detectar en los universos los movimientos y las tendencias de la Supremacía, observando las repercusiones de esas evoluciones en las personalidades y los modelos del gran universo.

\par
%\textsuperscript{(1266.1)}
\textsuperscript{115:6.8} Aunque no estamos seguros, creemos que el Supremo, como reflejo finito de la Deidad del Paraíso, ha emprendido un progreso eterno en el espacio exterior; pero como atenuación de los tres Absolutos potenciales del espacio exterior, este Ser Supremo busca constantemente la coherencia con el Paraíso. Este doble movimiento parece explicar la mayor parte de las actividades fundamentales que tienen lugar en los universos actualmente organizados.

\section*{7. La naturaleza del Supremo}
\par
%\textsuperscript{(1266.2)}
\textsuperscript{115:7.1} En la Deidad del Supremo, el Padre-YO SOY ha conseguido una liberación relativamente completa de las limitaciones inherentes al estado infinito, a la existencia eterna y a la naturaleza absoluta. Pero Dios Supremo sólo se ha liberado de todas las limitaciones existenciales sometiéndose a las restricciones experienciales de una función universal. Al conseguir la capacidad para la experiencia, el Dios finito se somete también a la necesidad de adquirirla; al lograr liberarse de la eternidad, el Todopoderoso se encuentra con las barreras del tiempo; y el Supremo sólo podía conocer el crecimiento y el desarrollo como consecuencia de una existencia parcial y de una naturaleza incompleta, las de un ser no absoluto.

\par
%\textsuperscript{(1266.3)}
\textsuperscript{115:7.2} Todo esto debe ser conforme con el plan del Padre, que ha basado el progreso finito en el esfuerzo, los logros de la criatura en la perseverancia, y el desarrollo de la personalidad en la fe. Al ordenar así la evolución experiencial del Supremo, el Padre ha hecho posible que las criaturas finitas puedan existir en los universos y que algún día consigan alcanzar la divinidad de la Supremacía por medio del progreso experiencial.

\par
%\textsuperscript{(1266.4)}
\textsuperscript{115:7.3} Toda la realidad es relativa, incluyendo al Supremo e incluso al Último, a excepción de los valores incalificados de los siete Absolutos. El hecho de la Supremacía está basado en el poder del Paraíso, en la personalidad del Hijo y en la acción del Conjunto, pero el crecimiento del Supremo está incluido en el Absoluto de la Deidad, el Absoluto Incalificado y el Absoluto Universal. Esta Deidad sintetizadora y unificadora ---Dios Supremo--- es la personificación de la sombra finita proyectada a través del gran universo por la unidad infinita de la naturaleza insondable del Padre Paradisiaco, la Fuente-Centro Primera.

\par
%\textsuperscript{(1266.5)}
\textsuperscript{115:7.4} En la medida en que las triodidades funcionan directamente en el nivel finito, entran en contacto con el Supremo, que es la focalización bajo la forma de Deidad y la suma cósmica total de las atenuaciones finitas de las naturalezas de lo Manifestado Absoluto y de lo Potencial Absoluto.

\par
%\textsuperscript{(1266.6)}
\textsuperscript{115:7.5} Se considera que la Trinidad del Paraíso es la inevitabilidad absoluta; los Siete Espíritus Maestros son aparentemente las inevitabilidades de la Trinidad; la manifestación del poder, la mente, el espíritu y la personalidad del Supremo debe ser la inevitabilidad evolutiva.

\par
%\textsuperscript{(1266.7)}
\textsuperscript{115:7.6} Dios Supremo no parece haber sido inevitable en la infinidad incalificada, pero parece serlo en todos los niveles de la relatividad. El Supremo es indispensable para focalizar, resumir y englobar la experiencia evolutiva, unificando eficazmente en su naturaleza de Deidad los resultados de esta manera de percibir la realidad. Y parece llevar a cabo todo esto con el fin de contribuir a la aparición de la \textit{existenciación inevitable}, la manifestación superexperiencial y superfinita de Dios
Último.

\par
%\textsuperscript{(1267.1)}
\textsuperscript{115:7.7} No se puede comprender plenamente al Ser Supremo sin tomar en consideración su fuente, su función y su destino: sus relaciones con la Trinidad que le dio origen, el universo donde ejerce su actividad y la Trinidad Última como destino inmediato.

\par
%\textsuperscript{(1267.2)}
\textsuperscript{115:7.8} Mediante el proceso de totalizar la experiencia evolutiva, el Supremo conecta lo finito con lo absonito, de la misma manera que la mente del Actor Conjunto integra la espiritualidad divina del Hijo personal con las energías inmutables del arquetipo Paradisíaco, y de la misma forma que la presencia del Absoluto Universal unifica la activación del Absoluto de la Deidad con la reactividad del Incalificado. Esta unidad debe ser una revelación del trabajo no detectado de la unidad original de la Primera Causa-Padre y Primer Arquetipo-Fuente de todas las cosas y de todos los seres.

\par
%\textsuperscript{(1267.3)}
\textsuperscript{115:7.9} [Patrocinado por un Poderoso Mensajero que reside temporalmente en Urantia].


\chapter{Documento 116. El Todopoderoso Supremo}
\par
%\textsuperscript{(1268.1)}
\textsuperscript{116:0.1} SI EL HOMBRE reconociera que sus Creadores ---sus supervisores inmediatos--- aunque sean divinos son también finitos, y que el Dios del tiempo y del espacio es una Deidad evolutiva y no absoluta, las contradicciones de las desigualdades temporales dejarían de ser profundas paradojas religiosas. La fe religiosa ya no se prostituiría fomentando la presunción social de los afortunados, y sirviendo sólo para estimular una resignación estoica entre las víctimas desafortunadas de las privaciones sociales.

\par
%\textsuperscript{(1268.2)}
\textsuperscript{116:0.2} Cuando contemplamos las esferas exquisitamente perfectas de Havona, es a la vez razonable y lógico creer que fueron hechas por un Creador perfecto, infinito y absoluto. Pero cuando cualquier persona honrada observa la confusión, las imperfecciones y las injusticias de Urantia, este mismo razonamiento y esta misma lógica la obligará a llegar a la conclusión de que vuestro mundo ha sido hecho y está dirigido por unos Creadores subabsolutos, preinfinitos y no necesariamente perfectos.

\par
%\textsuperscript{(1268.3)}
\textsuperscript{116:0.3} El crecimiento experiencial implica una asociación entre la criatura y el Creador ---Dios y el hombre asociados. El crecimiento es la marca distintiva de la Deidad experiencial: Havona no ha crecido; Havona existe y ha existido siempre; es existencial como los Dioses eternos que son su fuente. Por el contrario, el crecimiento caracteriza al gran universo.

\par
%\textsuperscript{(1268.4)}
\textsuperscript{116:0.4} El Todopoderoso Supremo es una Deidad viviente y evolutiva con poder y personalidad. Su campo de acción actual, el gran universo, es también un dominio que va creciendo en poder y en personalidad. El destino del Todopoderoso es la perfección, pero su experiencia actual abarca los elementos que crecen y que se encuentran en un estado incompleto.

\par
%\textsuperscript{(1268.5)}
\textsuperscript{116:0.5} El Ser Supremo ejerce sus funciones primarias en el universo central como una personalidad espiritual, y sus funciones secundarias en el gran universo como Dios Todopoderoso, una personalidad con poder. La función terciaria del Supremo en el universo maestro está ahora latente, y sólo existe como un potencial mental desconocido. Nadie sabe con exactitud qué es lo que revelará este tercer desarrollo del Ser Supremo. Algunos creen que cuando los superuniversos se establezcan en la luz y la vida, el Supremo ejercerá sus funciones desde Uversa como soberano todopoderoso y experiencial del gran universo, a la vez que ampliará su poder como super-omnipotente de los universos exteriores. Otros especulan que el tercer estado de la Supremacía consistirá en el tercer nivel de manifestación de la Deidad. Pero ninguno de nosotros lo sabe realmente.

\section*{1. La mente Suprema}
\par
%\textsuperscript{(1268.6)}
\textsuperscript{116:1.1} La experiencia de la personalidad de cada criatura evolutiva es una fase de la experiencia del Todopoderoso Supremo. El sometimiento inteligente de cada segmento físico de los superuniversos es una parte del control creciente del Todopoderoso Supremo. La síntesis creativa del poder y de la personalidad es una parte del impulso creador de la Mente Suprema, y constituye la esencia misma del crecimiento evolutivo de la unidad en el Ser Supremo.

\par
%\textsuperscript{(1269.1)}
\textsuperscript{116:1.2} La Mente Suprema tiene la función de unir los atributos del poder y de la personalidad de la Supremacía; el resultado de la evolución total del Todopoderoso Supremo será una Deidad unificada y personal ---y no una asociación de atributos divinos vagamente coordinada. Desde una perspectiva más amplia, no habrá ningún Todopoderoso aparte del Supremo, y ningún Supremo aparte del Todopoderoso.

\par
%\textsuperscript{(1269.2)}
\textsuperscript{116:1.3} Durante todas las épocas evolutivas, el potencial físico del poder del Supremo está depositado en los Siete Directores Supremos de Poder, y su potencial mental descansa en los Siete Espíritus Maestros. La Mente Infinita es la función del Espíritu Infinito; la mente cósmica es el ministerio de los Siete Espíritus Maestros; la mente Suprema está en proceso de manifestarse en la coordinación del gran universo y en asociación funcional con la revelación y los logros de Dios Séptuple.

\par
%\textsuperscript{(1269.3)}
\textsuperscript{116:1.4} La mente espacio-temporal, la mente cósmica, funciona de manera diferente en los siete superuniversos, pero está coordinada en el Ser Supremo mediante una técnica asociativa desconocida. El supercontrol del Todopoderoso sobre el gran universo no es exclusivamente físico y espiritual. En los siete superuniversos es principalmente material y espiritual, pero también están presentes otros fenómenos del Supremo que son tanto intelectuales como espirituales.

\par
%\textsuperscript{(1269.4)}
\textsuperscript{116:1.5} Sabemos menos en realidad sobre la mente de la Supremacía que sobre cualquier otro aspecto de esta Deidad evolutiva. Su mente está indiscutiblemente activa en todo el gran universo, y se cree que posee un destino potencial que abarcará extensas funciones en el universo maestro. Pero sí sabemos lo siguiente: Mientras que lo físico puede alcanzar un crecimiento completo y el espíritu puede conseguir la perfección de su desarrollo, la mente no deja nunca de progresar ---es la técnica experiencial del progreso sin fin. El Supremo es una Deidad experiencial y, por consiguiente, nunca logrará completar su perfeccionamiento mental.

\section*{2. El Todopoderoso y Dios Séptuple}
\par
%\textsuperscript{(1269.5)}
\textsuperscript{116:2.1} La aparición de la presencia del poder universal del Todopoderoso coincide con la aparición, en el escenario de la acción cósmica, de los elevados creadores y controladores de los superuniversos evolutivos.

\par
%\textsuperscript{(1269.6)}
\textsuperscript{116:2.2} Dios Supremo obtiene los atributos de su espíritu y de su personalidad de la Trinidad del Paraíso, pero está haciendo realidad su poder a través de las actividades de los Hijos Creadores, los Ancianos de los Días y los Espíritus Maestros, cuyos actos colectivos son la fuente de su creciente poder como soberano todopoderoso para los siete superuniversos y en ellos.

\par
%\textsuperscript{(1269.7)}
\textsuperscript{116:2.3} La Deidad Incalificada del Paraíso es incomprensible para las criaturas evolutivas del tiempo y del espacio. La eternidad y la infinidad conllevan un nivel de realidad de la deidad que las criaturas espacio-temporales no pueden comprender. La infinidad de la deidad y la soberanía absoluta son inherentes a la Trinidad del Paraíso, y la Trinidad es una realidad que está situada un poco más allá de la comprensión del hombre mortal. Las criaturas del espacio-tiempo necesitan orígenes, relatividades y destinos para captar las relaciones universales y comprender los valores significativos de la divinidad. Por eso la Deidad del Paraíso atenúa y limita de otras maneras las personalizaciones extraparadisíacas de la divinidad, trayendo así a la existencia a los Creadores Supremos y a sus asociados, que llevan continuamente la luz de la vida cada vez más lejos de su fuente Paradisíaca hasta que ésta encuentra su expresión más hermosa y lejana en la vida terrestre de los Hijos donadores en los mundos evolutivos.

\par
%\textsuperscript{(1270.1)}
\textsuperscript{116:2.4} Éste es el origen de Dios Séptuple\footnote{\textit{Siete espíritus de Dios}: Ap 1:4; 3:1; 4:5; 5:6.}, cuyos niveles sucesivos los va encontrando el hombre mortal en el orden siguiente:

\par
%\textsuperscript{(1270.2)}
\textsuperscript{116:2.5} 1. Los Hijos Creadores (y los Espíritus Creativos).

\par
%\textsuperscript{(1270.3)}
\textsuperscript{116:2.6} 2. Los Ancianos de los Días.

\par
%\textsuperscript{(1270.4)}
\textsuperscript{116:2.7} 3. Los Siete Espíritus Maestros.

\par
%\textsuperscript{(1270.5)}
\textsuperscript{116:2.8} 4. El Ser Supremo.

\par
%\textsuperscript{(1270.6)}
\textsuperscript{116:2.9} 5. El Actor Conjunto.

\par
%\textsuperscript{(1270.7)}
\textsuperscript{116:2.10} 6. El Hijo Eterno.

\par
%\textsuperscript{(1270.8)}
\textsuperscript{116:2.11} 7. El Padre Universal.

\par
%\textsuperscript{(1270.9)}
\textsuperscript{116:2.12} Los tres primeros niveles son los Creadores Supremos, y los tres últimos las Deidades del Paraíso. El Supremo interviene siempre como la personalización espiritual experiencial de la Trinidad del Paraíso, y como el foco experiencial del omnipotente poder evolutivo de los hijos creadores de las Deidades del Paraíso. En la presente era del universo, el Ser Supremo es la máxima revelación de la Deidad para los siete superuniversos.

\par
%\textsuperscript{(1270.10)}
\textsuperscript{116:2.13} Mediante la técnica de la lógica humana se podría deducir que la reunificación experiencial de los actos colectivos de los tres primeros niveles de Dios Séptuple equivaldría al nivel de la Deidad del Paraíso, pero esto no es así. La Deidad del Paraíso es una Deidad \textit{existencial}. Los Creadores Supremos, en su unidad divina de poder y de personalidad, constituyen y expresan un nuevo potencial de poder de la Deidad \textit{experiencial}. Este potencial de poder, de origen experiencial, se encuentra ineludible e inevitablemente unido a la Deidad experiencial que tiene su origen en la Trinidad ---el Ser Supremo.

\par
%\textsuperscript{(1270.11)}
\textsuperscript{116:2.14} Dios Supremo no es la Trinidad del Paraíso, ni tampoco es uno de los Creadores superuniversales o el conjunto de ellos, cuyas actividades funcionales sintetizan realmente su poder todopoderoso en evolución. Aunque Dios Supremo tiene su origen en la Trinidad, sólo se manifiesta a las criaturas evolutivas como una personalidad de poder a través de las funciones coordinadas de los tres primeros niveles de Dios Séptuple. El Todopoderoso Supremo se está convirtiendo ahora en un hecho, en el tiempo y el espacio, gracias a las actividades de las Personalidades Creadoras Supremas, al igual que en la eternidad el Actor Conjunto surgió instantáneamente a la existencia por voluntad del Padre Universal y del Hijo Eterno. Estos seres de los tres primeros niveles de Dios Séptuple constituyen la naturaleza y la fuente mismas del poder del Todopoderoso Supremo; por eso deben siempre acompañar y sostener sus actos administrativos.

\section*{3. El Todopoderoso y la Deidad del Paraíso}
\par
%\textsuperscript{(1270.12)}
\textsuperscript{116:3.1} Las Deidades del Paraíso no sólo actúan directamente en sus circuitos de gravedad por todo el gran universo, sino que también ejercen su actividad a través de sus diversos agentes y de otras manifestaciones tales como:

\par
%\textsuperscript{(1270.13)}
\textsuperscript{116:3.2} 1. \textit{Las focalizaciones mentales de la Fuente-Centro Tercera}. Los dominios finitos de la energía y del espíritu se mantienen literalmente unidos gracias a las presencias mentales del Actor Conjunto. Esto es así desde el Espíritu Creativo en un universo local, pasando por los Espíritus Reflectantes de un superuniverso, hasta los Espíritus Maestros en el gran universo. Los circuitos mentales que emanan de estos diversos centros de inteligencia representan el marco cósmico donde las criaturas efectúan sus elecciones. La mente es esa realidad flexible que las criaturas y los Creadores pueden manejar con tanta facilidad; es el eslabón vital que conecta la materia y el espíritu. La donación mental de la Fuente-Centro Tercera unifica la persona espiritual de Dios Supremo con el poder experiencial del Todopoderoso evolutivo.

\par
%\textsuperscript{(1271.1)}
\textsuperscript{116:3.3} 2. \textit{Las revelaciones como personalidad de la Fuente-Centro Segunda}. Las presencias mentales del Actor Conjunto unifican el espíritu de la divinidad con el arquetipo de la energía. Las encarnaciones donadoras del Hijo Eterno y de sus Hijos Paradisíacos unifican, fusionan realmente, la naturaleza divina de un Creador con la naturaleza evolutiva de una criatura. El Supremo es a la vez criatura y creador, y la posibilidad de ser ambas cosas se revela en los actos donadores del Hijo Eterno y de sus Hijos coordinados y subordinados. Las órdenes de filiación que se donan, los Migueles y los Avonales, acrecientan realmente su naturaleza divina con la auténtica naturaleza de las criaturas, la cual se vuelve suya viviendo la vida real de las criaturas en los mundos evolutivos. Cuando la divinidad se vuelve semejante a la humanidad, esta relación contiene la posibilidad inherente de que la humanidad pueda volverse divina.

\par
%\textsuperscript{(1271.2)}
\textsuperscript{116:3.4} 3. \textit{Las presencias internas de la Fuente-Centro Primera}. La mente unifica las causalidades espirituales con las reacciones energéticas; el ministerio donador unifica los descensos de la divinidad con la ascensión de las criaturas; y los fragmentos internos del Padre Universal unifican realmente a las criaturas evolutivas con Dios en el Paraíso. Existen muchas presencias parecidas del Padre que habitan en numerosas órdenes de personalidades, y en el hombre mortal, estos fragmentos divinos de Dios son los Ajustadores del Pensamiento. Los Monitores de Misterio son para los seres humanos lo que la Trinidad del Paraíso es para el Ser Supremo. Los Ajustadores son unos cimientos absolutos, y sobre estos cimientos absolutos las elecciones del libre albedrío pueden hacer que evolucione la realidad divina de una naturaleza que se eterniza, una naturaleza finalitaria en el caso del hombre, y una naturaleza de Deidad en Dios Supremo.

\par
%\textsuperscript{(1271.3)}
\textsuperscript{116:3.5} Las donaciones como criaturas de las órdenes paradisiacas de filiación permiten a estos Hijos divinos enriquecer su personalidad adquiriendo la naturaleza real de las criaturas del universo, mientras que estas donaciones revelan infaliblemente a las criaturas mismas el camino paradisiaco para alcanzar la divinidad. Las donaciones del Padre Universal, bajo la forma de Ajustadores, le permiten atraer hacia él a las personalidades de las criaturas volitivas. En todas estas relaciones que se producen en los universos finitos, el Actor Conjunto es la fuente siempre presente del ministerio mental que hace posible estas actividades.

\par
%\textsuperscript{(1271.4)}
\textsuperscript{116:3.6} Las Deidades del Paraíso participan de ésta y de otras muchas maneras en las evoluciones del tiempo a medida que se despliegan en los planetas que giran en el espacio, y a medida que culminan en la aparición de la personalidad del Supremo, consecuencia de toda la evolución.

\section*{4. El Todopoderoso y los Creadores Supremos}
\par
%\textsuperscript{(1271.5)}
\textsuperscript{116:4.1} La unidad del Todo Supremo depende de la unificación progresiva de las partes finitas; la manifestación del Supremo es el resultado y la causa de estas mismas unificaciones de los factores de la supremacía ---los creadores, criaturas, inteligencias y energías de los universos.

\par
%\textsuperscript{(1272.1)}
\textsuperscript{116:4.2} Durante las épocas en que la soberanía de la Supremacía está experimentando su desarrollo en el tiempo, el poder todopoderoso del Supremo depende de los actos de divinidad de Dios Séptuple, mientras que parece existir una relación particularmente estrecha entre el Ser Supremo y el Actor Conjunto, al igual que con sus personalidades primarias, los Siete Espíritus Maestros. El Espíritu Infinito, como Actor Conjunto, ejerce su actividad de muchas maneras que compensan el estado incompleto de la Deidad evolutiva, y mantiene relaciones muy estrechas con el Supremo. Los Siete Espíritus Maestros comparten en cierto modo la intimidad de esta relación, pero especialmente el Espíritu Maestro Número Siete, que habla en nombre del Supremo. Este Espíritu Maestro conoce al Supremo ---está en contacto personal con él.

\par
%\textsuperscript{(1272.2)}
\textsuperscript{116:4.3} Cuando empezó a concebirse el proyecto de la creación superuniversal, los Espíritus Maestros se unieron con la Trinidad ancestral para cocrear los cuarenta y nueve Espíritus Reflectantes, y al mismo tiempo el Ser Supremo actuó creativamente para llevar a su culminación los actos conjuntos de la Trinidad del Paraíso y de los hijos creativos de la Deidad del Paraíso. Majeston apareció, y desde entonces ha focalizado la presencia cósmica de la Mente Suprema, mientras que los Espíritus Maestros continúan siendo los orígenes y centros del extenso ministerio de la mente cósmica.

\par
%\textsuperscript{(1272.3)}
\textsuperscript{116:4.4} Pero los Espíritus Maestros continúan supervisando a los Espíritus Reflectantes. El Séptimo Espíritu Maestro (en su supervisión global de Orvonton desde el universo central) está en contacto personal con los siete Espíritus Reflectantes situados en Uversa (y tiene el supercontrol de los mismos). En su administración y control dentro de su superuniverso y entre los superuniversos, está en contacto reflectante con los Espíritus Reflectantes de su propio tipo situados en cada una de las capitales superuniversales.

\par
%\textsuperscript{(1272.4)}
\textsuperscript{116:4.5} Estos Espíritus Maestros no solamente apoyan y acrecientan la soberanía de la Supremacía, sino que son afectados a su vez por los propósitos creativos del Supremo. Las creaciones colectivas de los Espíritus Maestros son generalmente de tipo casi material (directores de poder, etc.), mientras que sus creaciones individuales son de tipo espiritual (supernafines, etc.). Pero cuando los Espíritus Maestros engendraron \textit{colectivamente} a los Siete Espíritus de los Circuitos en respuesta a la voluntad y al proyecto del Ser Supremo, hay que señalar que los frutos de este acto creativo fueron espirituales, y no materiales o casi materiales.

\par
%\textsuperscript{(1272.5)}
\textsuperscript{116:4.6} Lo mismo que sucede con los Espíritus Maestros de los superuniversos, también sucede con los gobernantes trinos de estas supercreaciones ---los Ancianos de los Días. Estas personificaciones del juicio y la justicia de la Trinidad, en el tiempo y el espacio, son los puntos de apoyo sobre el terreno destinados a movilizar el poder todopoderoso del Supremo, sirviendo como puntos focales séptuples para la evolución de la soberanía trinitaria en los dominios del tiempo y del espacio. Desde el lugar que ocupan, a medio camino entre el Paraíso y los mundos evolutivos, estos soberanos de origen Trinitario ven, conocen y coordinan los dos caminos.

\par
%\textsuperscript{(1272.6)}
\textsuperscript{116:4.7} Pero los universos locales son los verdaderos laboratorios en los que se elaboran los experimentos mentales, las aventuras galácticas, los despliegues de la divinidad y los progresos de la personalidad; la totalidad cósmica de estos factores constituye la base real sobre la que el Supremo está llevando a cabo, en y por experiencia, su evolución como deidad.

\par
%\textsuperscript{(1272.7)}
\textsuperscript{116:4.8} En los universos locales, los Creadores también evolucionan: la presencia del Actor Conjunto evoluciona desde un centro viviente de poder hasta el estado de la divina personalidad de un Espíritu Madre del Universo; el Hijo Creador evoluciona desde la naturaleza de una divinidad paradisíaca existencial hasta la naturaleza experiencial de la soberanía suprema. Los universos locales son los puntos de partida de la verdadera evolución, los semilleros de las personalidades imperfectas de buena fe dotadas de la libre elección de volverse cocreadoras de sí mismas tal como deseen llegar a ser.

\par
%\textsuperscript{(1273.1)}
\textsuperscript{116:4.9} En sus donaciones sobre los mundos evolutivos, los Hijos Magistrales adquieren finalmente una naturaleza que expresa la divinidad del Paraíso en unión experiencial con los valores espirituales más elevados de la naturaleza material humana. Mediante éstas y otras donaciones, los Migueles Creadores adquieren igualmente la naturaleza y el punto de vista cósmico de sus propios hijos del universo local. Estos Hijos Creadores Maestros se acercan a la culminación de la experiencia subsuprema, y cuando la soberanía sobre su universo local se amplía hasta englobar a los Espíritus Creativos asociados, se puede decir que se aproximan a los límites de la supremacía dentro de los potenciales actuales del gran universo en evolución.

\par
%\textsuperscript{(1273.2)}
\textsuperscript{116:4.10} Cuando los Hijos donadores revelan los nuevos caminos para que los hombres encuentren a Dios, no crean estos senderos que permiten alcanzar la divinidad; iluminan más bien las autovías eternas de progreso que conducen, a través de la presencia del Supremo, hasta la persona del Padre Paradisiaco.

\par
%\textsuperscript{(1273.3)}
\textsuperscript{116:4.11} El universo local es el punto de partida para aquellas personalidades que se encuentran más lejos de Dios, y que pueden experimentar así el mayor grado de ascensión espiritual en el universo, pueden conseguir la máxima participación experiencial en la cocreación de sí mismas. Estos mismos universos locales proporcionan también la profundidad experiencial más grande posible para las personalidades descendentes, las cuales consiguen así algo que para ellas es tan significativo como la ascensión al Paraíso lo es para una criatura evolutiva.

\par
%\textsuperscript{(1273.4)}
\textsuperscript{116:4.12} El hombre mortal parece ser necesario para el pleno funcionamiento de Dios Séptuple, tal como esta agrupación de divinidad culmina en el Supremo en vías de manifestarse. Existen otras muchas órdenes de personalidades universales que son igualmente necesarias para la evolución del poder todopoderoso del Supremo, pero esta descripción la presentamos para la edificación de los seres humanos, y por eso está en gran parte limitada a aquellos factores que actúan en la evolución de Dios Séptuple y que están relacionados con el hombre mortal.

\section*{5. El Todopoderoso y los Controladores Séptuples}
\par
%\textsuperscript{(1273.5)}
\textsuperscript{116:5.1} Habéis sido informados sobre las relaciones de Dios Séptuple con el Ser Supremo, y ahora deberíais reconocer que el Séptuple abarca a los controladores así como a los creadores del gran universo. Los controladores séptuples del gran universo son los siguientes:

\par
%\textsuperscript{(1273.6)}
\textsuperscript{116:5.2} 1. Los Controladores Físicos Maestros.

\par
%\textsuperscript{(1273.7)}
\textsuperscript{116:5.3} 2. Los Centros Supremos de Poder.

\par
%\textsuperscript{(1273.8)}
\textsuperscript{116:5.4} 3. Los Directores Supremos de Poder.

\par
%\textsuperscript{(1273.9)}
\textsuperscript{116:5.5} 4. El Todopoderoso Supremo.

\par
%\textsuperscript{(1273.10)}
\textsuperscript{116:5.6} 5. El Dios de Acción ---el Espíritu Infinito.

\par
%\textsuperscript{(1273.11)}
\textsuperscript{116:5.7} 6. La Isla del Paraíso.

\par
%\textsuperscript{(1273.12)}
\textsuperscript{116:5.8} 7. La Fuente del Paraíso ---el Padre Universal.

\par
%\textsuperscript{(1273.13)}
\textsuperscript{116:5.9} Estos siete grupos son funcionalmente inseparables de Dios Séptuple, y componen el nivel del control físico de esta asociación de Deidad.

\par
%\textsuperscript{(1273.14)}
\textsuperscript{116:5.10} La bifurcación de la energía y el espíritu (que provienen de la presencia conjunta del Hijo Eterno y de la Isla del Paraíso), quedó simbolizada en sentido superuniversal cuando los Siete Espíritus Maestros emprendieron juntos su primer acto de creación colectiva. Este episodio fue testigo de la aparición de los Siete Directores Supremos de Poder. Simultáneamente, los circuitos espirituales de los Espíritus Maestros se diferenciaron, por contraste, de las actividades físicas de supervisión de los directores de poder, y la mente cósmica apareció inmediatamente como un nuevo factor que coordinaba la materia y el espíritu.

\par
%\textsuperscript{(1274.1)}
\textsuperscript{116:5.11} El Todopoderoso Supremo evoluciona como supercontrolador del poder físico del gran universo. En la era actual del universo, este potencial de poder físico parece estar centrado en los Siete Directores Supremos de Poder, que funcionan a través de los emplazamientos fijos de los centros de poder y por medio de las presencias móviles de los controladores físicos.

\par
%\textsuperscript{(1274.2)}
\textsuperscript{116:5.12} Los universos temporales no son perfectos; ése es su destino. La lucha por la perfección no solamente es propia de los niveles intelectuales y espirituales, sino también del nivel físico de la energía y la masa. El establecimiento de los siete superuniversos en la luz y la vida presupone que han alcanzado la estabilidad física. Y se supone que cuando se consiga finalmente el equilibrio material, la evolución del control físico del Todopoderoso habrá concluido.

\par
%\textsuperscript{(1274.3)}
\textsuperscript{116:5.13} En los primeros tiempos de la construcción de un universo, incluso los Creadores Paradisíacos se preocupan principalmente del equilibrio material. La constitución de un universo local no sólo va tomando forma como resultado de las actividades de los centros de poder, sino también a causa de la presencia espacial del Espíritu Creativo. Durante todas estas épocas iniciales de la construcción de un universo local, el Hijo Creador manifiesta un atributo de control material poco comprendido, y no deja su planeta capital hasta que se ha establecido el equilibrio total del universo local.

\par
%\textsuperscript{(1274.4)}
\textsuperscript{116:5.14} A fin de cuentas, toda la energía reacciona a la mente, y los controladores físicos son los hijos del Dios de la mente, que es el activador del arquetipo del Paraíso. Los directores de poder dedican sin cesar su inteligencia a la tarea de conseguir el control material. Su lucha por dominar físicamente las relaciones de la energía y los movimientos de la masa no termina nunca hasta que consiguen la victoria finita sobre las energías y las masas que constituyen sus esferas perpetuas de actividad.

\par
%\textsuperscript{(1274.5)}
\textsuperscript{116:5.15} Las luchas espirituales del tiempo y del espacio tienen que ver con la evolución del dominio del espíritu sobre la materia por mediación de la mente (personal); la evolución física (no personal) de los universos tiene que ver con poner la energía cósmica en armonía con los conceptos mentales equilibrados sometidos al supercontrol del espíritu. La evolución total de todo el gran universo es un asunto de unificación, por medio de la personalidad, de la mente que controla la energía con el intelecto coordinado con el espíritu; esta unificación se revelará en la plena aparición del poder todopoderoso del Supremo.

\par
%\textsuperscript{(1274.6)}
\textsuperscript{116:5.16} La dificultad para lograr un estado de equilibrio dinámico es inherente al hecho del crecimiento del cosmos. Los circuitos establecidos de la creación física están continuamente en peligro debido a la aparición de nuevas energías y de nuevas masas. Un universo que crece es un universo inestable; por eso, ninguna parte del conjunto cósmico puede conseguir una estabilidad real hasta que la plenitud de los tiempos sea testigo de la terminación material de los siete superuniversos.

\par
%\textsuperscript{(1274.7)}
\textsuperscript{116:5.17} En los universos establecidos en la luz y la vida no se producen acontecimientos físicos inesperados de mayor importancia. Se ha conseguido un control relativamente completo sobre la creación material; sin embargo, los problemas de las relaciones entre los universos estabilizados y los universos en evolución continúan desafiando la habilidad de los Directores Universales de Poder. Pero estos problemas desaparecerán gradualmente cuando disminuyan las actividades creativas nuevas, a medida que el gran universo se acerque a la culminación de su expresión evolutiva.

\section*{6. La dominación del espíritu}
\par
%\textsuperscript{(1275.1)}
\textsuperscript{116:6.1} La energía-materia domina en los superuniversos evolutivos, salvo en la personalidad, donde el espíritu lucha, por mediación de la mente, para conseguir la superioridad. La meta de los universos evolutivos es someter la energía-materia a la acción de la mente, coordinar la mente con el espíritu, y conseguir todo ello en virtud de la presencia creativa y unificadora de la personalidad. Así pues, en relación con la personalidad, los sistemas físicos se vuelven subordinados, los sistemas mentales, coordinados, y los sistemas espirituales, directivos.

\par
%\textsuperscript{(1275.2)}
\textsuperscript{116:6.2} En los niveles de la deidad, esta unión del poder y de la personalidad se expresa en, y bajo la forma de, el Supremo. Pero la verdadera evolución de la dominación del espíritu es un crecimiento que está basado en los actos voluntarios de los Creadores y de las criaturas del gran universo.

\par
%\textsuperscript{(1275.3)}
\textsuperscript{116:6.3} En los niveles absolutos, la energía y el espíritu son una sola cosa. Pero en cuanto nos apartamos de estos niveles absolutos, aparecen las diferencias, y a medida que la energía y el espíritu se desplazan desde el Paraíso hacia el espacio, aumenta el abismo entre ellos hasta que, en los universos locales, se han vuelto totalmente divergentes. Han dejado de ser idénticos, tampoco son semejantes, y la mente tiene que intervenir para relacionarlos entre sí.

\par
%\textsuperscript{(1275.4)}
\textsuperscript{116:6.4} El hecho de que la energía pueda ser dirigida por la acción de la personalidad de los controladores revela que la energía es sensible a la acción de la mente. Que la masa pueda ser estabilizada gracias a la actividad de estas mismas entidades controladoras indica que la masa es sensible a la presencia generadora de orden de la mente. Y que el espíritu mismo, en una personalidad volitiva, pueda esforzarse por dominar la energía-materia a través de la mente, revela la unidad potencial de toda la creación finita.

\par
%\textsuperscript{(1275.5)}
\textsuperscript{116:6.5} En todo el universo de universos existe una interdependencia entre todas las fuerzas y personalidades. Los Hijos Creadores y los Espíritus Creativos dependen de la actividad cooperativa de los centros de poder y de los controladores físicos para organizar los universos; los Directores Supremos de Poder están incompletos sin el supercontrol de los Espíritus Maestros. En un ser humano, el mecanismo de la vida física es sensible en parte a los mandatos de la mente (personal). Esta misma mente puede estar dominada a su vez por las directrices de un espíritu resuelto, y el resultado de un desarrollo evolutivo semejante es la producción de un nuevo hijo del Supremo, una nueva unificación personal de los diversos tipos de realidades cósmicas.

\par
%\textsuperscript{(1275.6)}
\textsuperscript{116:6.6} Lo mismo que sucede con las partes, sucede con el todo; la persona espiritual de la Supremacía necesita el poder evolutivo del Todopoderoso para lograr completar su Deidad y alcanzar su destino de asociación con la Trinidad. El esfuerzo lo realizan las personalidades del tiempo y del espacio, pero la culminación y la consumación de este esfuerzo es tarea del Todopoderoso Supremo. Puesto que el crecimiento del todo es así la suma del crecimiento colectivo de las partes, de ello se deriva igualmente que la evolución de las partes es un reflejo segmentado del crecimiento intencional del todo.

\par
%\textsuperscript{(1275.7)}
\textsuperscript{116:6.7} En el Paraíso, la monota y el espíritu forman una sola cosa ---sólo se pueden distinguir por el nombre. En Havona, aunque la materia y el espíritu son notablemente diferentes, poseen al mismo tiempo una armonía innata. Sin embargo, en los siete superuniversos existe una gran divergencia; existe un gran abismo entre la energía cósmica y el espíritu divino; hay por lo tanto un mayor potencial experiencial para que la actividad de la mente armonice y unifique finalmente la forma física con los objetivos espirituales. En los universos del espacio que evolucionan en el tiempo, la divinidad está más atenuada, los problemas por resolver son más difíciles, y su solución proporciona mayores ocasiones para adquirir experiencia. Toda esta situación superuniversal crea un campo más amplio, en la existencia evolutiva, en el que la posibilidad de las experiencias cósmicas se encuentra disponible por igual para la criatura y el Creador ---e incluso para la Deidad Suprema.

\par
%\textsuperscript{(1276.1)}
\textsuperscript{116:6.8} La dominación del espíritu, que es existencial en los niveles absolutos, se convierte en una experiencia evolutiva en los niveles finitos y en los siete superuniversos. Y esta experiencia la comparten todos del mismo modo, desde el hombre mortal hasta el Ser Supremo. Todos se esfuerzan, se esfuerzan personalmente, por perfeccionarse; todos participan, participan personalmente, en el destino.

\section*{7. El organismo viviente del gran universo}
\par
%\textsuperscript{(1276.2)}
\textsuperscript{116:7.1} El gran universo no es solamente una creación material de grandiosidad física, de sublimidad espiritual y de magnitud intelectual, sino que es también un organismo viviente magnífico y sensible. Existe una vida real que palpita en todo el mecanismo de la inmensa creación del vibrante cosmos. La realidad física de los universos simboliza la realidad perceptible del Todopoderoso Supremo; este organismo material y viviente está penetrado por circuitos de inteligencia, al igual que el cuerpo humano está atravesado por una red de conductos nerviosos sensibles. El universo físico está impregnado de canales de energía que activan eficazmente la creación material, al igual que el cuerpo humano es alimentado y vigorizado por la distribución circulatoria de los productos energéticos asimilables de la comida. El inmenso universo no está desprovisto de aquellos centros coordinadores que efectúan un magnífico supercontrol, y que pueden compararse con el delicado sistema de control químico del mecanismo humano. Si tan sólo supierais algo sobre la constitución de un centro de poder, podríamos contaros, por analogía, muchas más cosas sobre el universo físico.

\par
%\textsuperscript{(1276.3)}
\textsuperscript{116:7.2} Al igual que los mortales cuentan con la energía solar para mantenerse con vida, el gran universo depende de las energías inagotables que emanan del bajo Paraíso para sostener las actividades materiales y los movimientos cósmicos del espacio.

\par
%\textsuperscript{(1276.4)}
\textsuperscript{116:7.3} La mente ha sido concedida a los mortales para que con ella puedan volverse conscientes de la identidad y de la personalidad; una mente ---e incluso una Mente Suprema--- ha sido otorgada a la totalidad de lo finito, por medio de la cual el espíritu de esta personalidad emergente del cosmos se esfuerza siempre por dominar la energía-materia.

\par
%\textsuperscript{(1276.5)}
\textsuperscript{116:7.4} El hombre mortal es sensible a la guía del espíritu, al igual que el gran universo reacciona a la extensa atracción de la gravedad espiritual del Hijo Eterno, la cohesión supermaterial universal de los valores espirituales eternos de todas las creaciones que componen el cosmos finito del tiempo y del espacio.

\par
%\textsuperscript{(1276.6)}
\textsuperscript{116:7.5} Los seres humanos son capaces de identificarse para siempre con la realidad total e indestructible del universo ---fusionar con el Ajustador del Pensamiento interior. Del mismo modo, el Supremo depende eternamente de la estabilidad absoluta de la Deidad Original, la Trinidad del Paraíso.

\par
%\textsuperscript{(1276.7)}
\textsuperscript{116:7.6} El vivo deseo que siente el hombre por la perfección del Paraíso, sus esfuerzos por alcanzar a Dios, crean en el cosmos viviente una verdadera tensión de divinidad que sólo puede resolverse mediante la evolución de un alma inmortal; esto es lo que sucede en la experiencia de una criatura humana individual. Pero cuando todas las criaturas y todos los Creadores se esfuerzan del mismo modo en el gran universo por alcanzar a Dios y la perfección divina, se establece una profunda tensión cósmica que sólo encuentra su resolución en la síntesis sublime del poder todopoderoso con la persona espiritual del Dios evolutivo de todas las criaturas, el Ser Supremo.

\par
%\textsuperscript{(1277.1)}
\textsuperscript{116:7.7} (Patrocinado por un Poderoso Mensajero que reside temporalmente en Urantia.)


\chapter{Documento 117. Dios Supremo}
\par
%\textsuperscript{(1278.1)}
\textsuperscript{117:0.1} EN LA medida en que hacemos la voluntad de Dios en cualquier lugar del universo donde podamos tener nuestra existencia, el potencial todopoderoso del Supremo se manifiesta un paso más. La voluntad de Dios es el propósito de la Fuente-Centro Primera tal como se ha potencializado en los tres Absolutos, personalizado en el Hijo Eterno, unido para la actividad universal en el Espíritu Infinito, y eternizado en los arquetipos perpetuos del Paraíso. Y Dios Supremo se está convirtiendo en la manifestación finita más elevada de la voluntad total de Dios\footnote{\textit{Hacer la voluntad de Dios}: Sal 143:10; Eclo 15:11-20; Mt 6:10; 7:21; 12:50; 26:39,42,44; Mc 3:35; 14:36,39; Lc 8:21; 11:2; 22:42; Jn 4:34; 5:30; 6:38-40; 7:16-17; 9:31; 14:21-24; 15:10,14,16; 17:4.}.

\par
%\textsuperscript{(1278.2)}
\textsuperscript{117:0.2} Si todos los habitantes del gran universo consiguieran relativamente alguna vez vivir plenamente la voluntad de Dios, entonces las creaciones del espacio-tiempo se establecerían en la luz y la vida, y el Todopoderoso, el potencial bajo la forma de deidad de la Supremacía, se volvería entonces un hecho mediante la aparición de la personalidad divina de Dios Supremo.

\par
%\textsuperscript{(1278.3)}
\textsuperscript{117:0.3} Cuando una mente en evolución se sintoniza con los circuitos de la mente cósmica, cuando un universo en evolución se estabiliza a la manera del modelo del universo central, cuando un espíritu que progresa se pone en contacto con el ministerio unificado de los Espíritus Maestros, cuando la personalidad de un mortal ascendente se sintoniza finalmente con las directrices divinas de su Ajustador interior, entonces la manifestación del Supremo se vuelve un grado más real en los universos; la divinidad de la Supremacía ha avanzado entonces un paso más hacia su realización cósmica.

\par
%\textsuperscript{(1278.4)}
\textsuperscript{117:0.4} Las partes y los individuos del gran universo evolucionan como un reflejo de la evolución total del Supremo, mientras que el Supremo es a su vez la totalidad acumulativa sintética de toda la evolución del gran universo. Desde el punto de vista de los mortales, las dos cosas son fenómenos evolutivos y experienciales recíprocos.

\section*{1. La naturaleza del Ser Supremo}
\par
%\textsuperscript{(1278.5)}
\textsuperscript{117:1.1} El Supremo es la belleza de la armonía física, la verdad de los significados intelectuales y la bondad de los valores espirituales. Es el dulzor del éxito verdadero y la alegría de los logros perpetuos. Es la superalma del gran universo, la conciencia del cosmos finito, la culminación de la realidad finita, y la personificación de la experiencia del Creador y la criatura. A lo largo de toda la eternidad futura, Dios Supremo expresará la realidad de la experiencia volitiva en las relaciones trinitarias de la Deidad.

\par
%\textsuperscript{(1278.6)}
\textsuperscript{117:1.2} En las personas de los Creadores Supremos, los Dioses han descendido del Paraíso a los dominios del tiempo y del espacio para crear y hacer evolucionar allí a unas criaturas capaces de alcanzar el Paraíso y de ascender hasta allí en busca del Padre. Esta procesión universal de Creadores descendentes que revelan a Dios y de criaturas ascendentes que buscan a Dios revela la evolución, bajo la forma de Deidad, del Supremo, en quien tanto los descendentes como los ascendentes consiguen comprenderse mutuamente, descubren la fraternidad eterna y universal. El Ser Supremo se convierte así en la síntesis finita de la experiencia que reúne la causa producida por el Creador perfecto y la reacción que tienen las criaturas que se perfeccionan.

\par
%\textsuperscript{(1279.1)}
\textsuperscript{117:1.3} El gran universo contiene la posibilidad de unificarse por completo, siendo algo que busca constantemente, y esto se deriva del hecho de que esta existencia cósmica es una consecuencia de los actos creativos y de los mandatos de poder de la Trinidad del Paraíso, que es la unidad incalificada. Esta misma unidad trinitaria se expresa en el cosmos finito en el Supremo, cuya realidad se vuelve cada vez más evidente a medida que los universos alcanzan el máximo nivel de identificación con la Trinidad.

\par
%\textsuperscript{(1279.2)}
\textsuperscript{117:1.4} La voluntad del Creador y la voluntad de la criatura son cualitativamente diferentes, pero son también experiencialmente semejantes, pues el Creador y la criatura pueden colaborar para conseguir la perfección universal. El hombre puede trabajar en unión con Dios y así crear juntos un finalitario eterno. Dios puede trabajar incluso a la manera humana mediante las encarnaciones de sus Hijos, que consiguen así la supremacía de la experiencia de las criaturas.

\par
%\textsuperscript{(1279.3)}
\textsuperscript{117:1.5} El Creador y la criatura están unidos, en el Ser Supremo, en una sola Deidad cuya voluntad es la expresión de una sola personalidad divina. Esta voluntad del Supremo es algo más que la voluntad de la criatura o del Creador, al igual que la voluntad soberana del Hijo Maestro de Nebadon es actualmente algo más que una combinación de las voluntades de la divinidad y de la humanidad. La unión de la perfección del Paraíso y de la experiencia espacio-temporal produce un nuevo valor significativo en los niveles de deidad de la realidad.

\par
%\textsuperscript{(1279.4)}
\textsuperscript{117:1.6} La divina naturaleza evolutiva del Supremo se está convirtiendo en una fiel descripción de la experiencia incomparable de todas las criaturas y de todos los Creadores en el gran universo. En el Supremo, las naturalezas del creador y de la criatura están de acuerdo; están unidas para siempre por la experiencia nacida de las vicisitudes que acompañan a la solución de los numerosos problemas que acosan a toda la creación finita, a medida que ésta recorre el camino eterno buscando la perfección y la liberación de las trabas del estado incompleto.

\par
%\textsuperscript{(1279.5)}
\textsuperscript{117:1.7} La verdad, la belleza y la bondad están correlacionadas en el ministerio del Espíritu, la grandiosidad del Paraíso, la misericordia del Hijo y la experiencia del Supremo. Dios Supremo \textit{es} la verdad, la belleza y la bondad, ya que estos conceptos de la divinidad representan lo máximo que los seres finitos pueden concebir por experiencia. Los orígenes eternos de estas cualidades trinas de la divinidad están situados en unos niveles superfinitos, y una criatura sólo podría concebir estos orígenes como superverdad, superbelleza y superbondad.

\par
%\textsuperscript{(1279.6)}
\textsuperscript{117:1.8} Miguel, que es un creador, reveló el amor divino\footnote{\textit{Amor divino}: Jn 3:16; 15:9-13,17; 17:22-23; Ro 5:8; Tit 3:4; 1 Jn 4:7-19.} del Padre Creador por sus hijos terrestres. Una vez que han descubierto y recibido este afecto divino, los hombres pueden aspirar a revelar este amor a sus hermanos en la carne. Este afecto de las criaturas es un verdadero reflejo del amor del Supremo.

\par
%\textsuperscript{(1279.7)}
\textsuperscript{117:1.9} El Supremo es simétricamente inclusivo. La Fuente-Centro Primera es potencial en los tres grandes Absolutos, y está manifestada en el Paraíso, en el Hijo y en el Espíritu; pero el Supremo es a la vez manifestado y potencial, es un ser con una supremacía personal y un poder todopoderoso, sensible por igual al esfuerzo de la criatura y al propósito del Creador; actúa por sí mismo sobre el universo y reacciona en sí mismo a la suma total del universo; es al mismo tiempo el creador supremo y la criatura suprema. La Deidad de Supremacía expresa así la suma total de todo lo finito.

\section*{2. La fuente del crecimiento evolutivo}
\par
%\textsuperscript{(1280.1)}
\textsuperscript{117:2.1} El Supremo es Dios en el tiempo; suyo es el secreto del crecimiento de las criaturas en el tiempo; suya es también la conquista del presente incompleto y la consumación del futuro que se está perfeccionando. Y he aquí el fruto final de todo el crecimiento finito: el poder estará controlado por el espíritu a través de la mente, debido a la presencia unificadora y creativa de la personalidad. La consecuencia culminante de todo este crecimiento es el Ser Supremo.

\par
%\textsuperscript{(1280.2)}
\textsuperscript{117:2.2} Para el hombre mortal, existir equivale a crecer. Y parece ser que esto es así incluso en el sentido más amplio del universo, porque la existencia dirigida por el espíritu parece tener como resultado el crecimiento experiencial ---una elevación del estado. Sin embargo, hemos considerado durante mucho tiempo que el crecimiento actual que caracteriza a la existencia de las criaturas en la presente era del universo es una función del Supremo. Sostenemos igualmente que este tipo de crecimiento es propio de la era del crecimiento del Supremo, y que terminará cuando concluya el crecimiento del Supremo.

\par
%\textsuperscript{(1280.3)}
\textsuperscript{117:2.3} Considerad el estado de los hijos trinitizados por las criaturas: Han nacido y viven en la presente era del universo; poseen una personalidad así como unas dotaciones mentales y espirituales. Tienen experiencias y las recuerdan, pero no \textit{crecen} como los ascendentes. Creemos e interpretamos que estos hijos trinitizados por las criaturas, aunque se encuentran \textit{en} la presente era del universo, pertenecen en realidad \textit{a} la siguiente era universal ---la era que seguirá a la finalización del crecimiento del Supremo. Por eso no están \textit{en} el Supremo, cuyo estado actual es incompleto y en consecuencia está creciendo. Así pues, no participan en el crecimiento experiencial de la presente era del universo, y se mantienen en reserva para la próxima era universal.

\par
%\textsuperscript{(1280.4)}
\textsuperscript{117:2.4} Los Poderosos Mensajeros de mi propia orden, como han sido abrazados por la Trinidad, no participan en el crecimiento de la era actual del universo. En cierto sentido, nuestro estado pertenece a la era anterior del universo, como sucede de hecho con los Hijos Estacionarios de la Trinidad. Una cosa es segura: nuestro estado es fijo debido al abrazo de la Trinidad, y nuestra experiencia ya no se traduce en crecimiento.

\par
%\textsuperscript{(1280.5)}
\textsuperscript{117:2.5} Esto no sucede con los finalitarios ni con ninguna de las otras órdenes evolutivas y experienciales que participan en el proceso de desarrollo del Supremo. Vosotros, los mortales que vivís actualmente en Urantia y que podéis aspirar a alcanzar el Paraíso y el estado de finalitarios, deberíais comprender que ese destino sólo se puede conseguir porque estáis en el Supremo, formáis parte de él, y por lo tanto estáis participando en el ciclo del crecimiento del Supremo.

\par
%\textsuperscript{(1280.6)}
\textsuperscript{117:2.6} Algún día llegará el final del desarrollo del Supremo; su estado alcanzará su culminación (en el sentido espiritual y energético). La terminación de la evolución del Supremo presenciará también el final de la evolución de las criaturas como partes de la Supremacía. No sabemos qué tipo de desarrollo caracterizará a los universos del espacio exterior. Pero estamos muy seguros de que se tratará de algo muy diferente a todo lo que se ha visto en la presente era de la evolución de los siete superuniversos. Los ciudadanos evolutivos del gran universo tendrán sin duda la ocupación de compensar a los habitantes del espacio exterior por esta privación del crecimiento de la Supremacía.

\par
%\textsuperscript{(1280.7)}
\textsuperscript{117:2.7} El Ser Supremo, tal como exista cuando culmine la presente era universal, ejercerá su actividad como soberano experiencial en el gran universo. Los habitantes del espacio exterior ---los ciudadanos de la siguiente era universal--- tendrán un potencial de crecimiento postsuperuniversal, una capacidad para la consecución evolutiva que presupondrá la soberanía del Todopoderoso Supremo, excluyendo por lo tanto la participación de tales criaturas en la síntesis del poder y la personalidad de la presente era del universo.

\par
%\textsuperscript{(1281.1)}
\textsuperscript{117:2.8} Así pues, el estado incompleto del Supremo puede ser considerado como una ventaja, puesto que hace posible el crecimiento evolutivo de las criaturas creadas de los universos actuales. El vacío tiene en verdad sus ventajas, pues puede ser llenado con la experiencia.

\par
%\textsuperscript{(1281.2)}
\textsuperscript{117:2.9} Una de las preguntas más fascinantes de la filosofía finita es la siguiente: ¿El Ser Supremo se hace manifiesto en respuesta a la evolución del gran universo, o bien este cosmos finito evoluciona progresivamente en respuesta a la manifestación gradual del Supremo? ¿O es posible que sean mutuamente interdependientes para desarrollarse, que sean recíprocos evolutivos, poniendo en marcha cada cual el crecimiento del otro? Estamos seguros de una cosa: las criaturas y los universos, de todas las clases, están evolucionando dentro del Supremo, y a medida que evolucionan, está apareciendo la suma unificada de toda la actividad finita de esta era del universo. Y ésta es la aparición del Ser Supremo, que para todas las personalidades es la evolución del poder todopoderoso de Dios Supremo.

\section*{3. Significado del Supremo para las criaturas del universo}
\par
%\textsuperscript{(1281.3)}
\textsuperscript{117:3.1} La realidad cósmica que denominamos de manera diversa el Ser Supremo, Dios Supremo y el Todopoderoso Supremo, es la síntesis compleja y universal de las fases emergentes de todas las realidades finitas. La extensa diversificación de la energía eterna, el espíritu divino y la mente universal alcanza su culminación finita en la evolución del Supremo, que es la suma total de todo el crecimiento finito, llevado a cabo en los niveles de deidad del máximo acabamiento finito.

\par
%\textsuperscript{(1281.4)}
\textsuperscript{117:3.2} El Supremo es el canal divino por el que fluye la infinidad creativa de las triodidades, que se cristaliza en el panorama galáctico del espacio, donde tiene lugar el magnífico drama de las personalidades del tiempo: la conquista espiritual de la energía-materia por mediación de la mente.

\par
%\textsuperscript{(1281.5)}
\textsuperscript{117:3.3} Jesús dijo: <<Yo soy el camino viviente>>\footnote{\textit{Yo soy el camino viviente}: Jn 14:6; Heb 10:20.}, y él es en verdad el camino viviente que conduce desde el nivel material de la conciencia de sí hasta el nivel espiritual de la conciencia de Dios. Al igual que Jesús es este camino viviente que asciende desde el yo hasta Dios, el Supremo es el camino viviente que conduce desde la conciencia finita hasta la trascendencia de la conciencia, e incluso hasta la perspicacia de la absonitidad.

\par
%\textsuperscript{(1281.6)}
\textsuperscript{117:3.4} Vuestro Hijo Creador puede ser realmente este canal viviente entre la humanidad y la divinidad, puesto que ha experimentado personalmente la plenitud de la travesía de este camino universal de progreso, desde la verdadera humanidad de Josué ben José, el Hijo del Hombre, hasta la divinidad paradisiaca de Miguel de Nebadon, el Hijo del Dios infinito. Del mismo modo, el Ser Supremo puede ejercer su actividad como camino de acceso universal para trascender las limitaciones finitas, porque es la expresión efectiva y el resumen personal de toda la evolución, del progreso y de la espiritualización de las criaturas. Incluso las experiencias que efectúan las personalidades descendentes del Paraíso en el gran universo forman esa parte de la experiencia del Supremo que se complementa con la suma de las experiencias ascendentes de los peregrinos del tiempo.

\par
%\textsuperscript{(1281.7)}
\textsuperscript{117:3.5} El hombre mortal está hecho a imagen de Dios\footnote{\textit{El hombre hecho a imagen de Dios}: Gn 1:26-27; 9:6.} de una forma más que figurada. Desde un punto de vista físico, esta afirmación no es del todo cierta, pero en lo que se refiere a ciertas potencialidades universales, es un hecho real. En la raza humana se está desarrollando una parte del mismo drama de consecución evolutiva que está teniendo lugar, en una escala enormemente más grande, en el universo de universos. El hombre, una personalidad volitiva, se vuelve creativo en unión con un Ajustador, una entidad impersonal, en presencia de las potencialidades finitas del Supremo, y el resultado es el florecimiento de un alma inmortal. En los universos, las personalidades Creadoras del tiempo y del espacio trabajan en unión con el espíritu impersonal de la Trinidad del Paraíso, y se vuelven así creadoras de un nuevo potencial de poder de la realidad de la Deidad.

\par
%\textsuperscript{(1282.1)}
\textsuperscript{117:3.6} El hombre mortal, como es una criatura, no es exactamente semejante al Ser Supremo, que es una deidad, pero la evolución del hombre se parece en algunos aspectos al crecimiento del Supremo. El hombre crece conscientemente desde lo material hacia lo espiritual mediante la fuerza, el poder y la perseverancia de sus propias decisiones; también crece a medida que su Ajustador del Pensamiento desarrolla nuevas técnicas para descender desde los niveles espirituales hasta los niveles morontiales del alma; y una vez que el alma ha nacido, empieza a crecer en sí misma y por sí misma.

\par
%\textsuperscript{(1282.2)}
\textsuperscript{117:3.7} Esto se parece un poco a la forma en que se desarrolla el Ser Supremo. Su soberanía crece y se deriva de los actos y las realizaciones de las Personalidades Creadoras Supremas; es la evolución de la majestad de su poder como gobernante del gran universo. Su naturaleza como deidad depende igualmente de la unidad preexistente de la Trinidad del Paraíso. Pero la evolución de Dios Supremo presenta además otro aspecto: no sólo evoluciona gracias a los Creadores y se deriva de la Trinidad, sino que también evoluciona por sí mismo y se deriva de sí mismo. Dios Supremo mismo participa de manera volitiva y creativa en la realización de su propia deidad. El alma morontial humana es igualmente una asociada volitiva y cocreativa de su propia inmortalización.

\par
%\textsuperscript{(1282.3)}
\textsuperscript{117:3.8} El Padre colabora con el Actor Conjunto para manipular las energías del Paraíso y hacerlas sensibles al Supremo. El Padre colabora con el Hijo Eterno para engendrar las personalidades Creadoras cuyas actividades culminarán algún día en la soberanía del Supremo. El Padre colabora con el Hijo y el Espíritu para crear las personalidades trinitarias destinadas a ejercer su actividad como gobernantes del gran universo hasta el momento en que la evolución completa del Supremo lo capacite para asumir esta soberanía. El Padre coopera de éstas y de otras muchas maneras con sus coordinados, ya sean Deidades o no Deidades, para favorecer la evolución de la Supremacía, pero también actúa a solas en estos asuntos. Su labor solitaria se revela probablemente mejor en el ministerio de los Ajustadores del Pensamiento y de sus entidades asociadas.

\par
%\textsuperscript{(1282.4)}
\textsuperscript{117:3.9} La Deidad es una unidad, existencial en la Trinidad, experiencial en el Supremo y, en los mortales, las criaturas consiguen dicha unidad fusionando con el Ajustador. La presencia de los Ajustadores del Pensamiento en los hombres mortales revela la unidad esencial del universo, ya que el hombre, el tipo más humilde posible de personalidad universal, contiene dentro de sí un fragmento real de la realidad eterna más elevada, el Padre original de todas las personalidades.

\par
%\textsuperscript{(1282.5)}
\textsuperscript{117:3.10} El Ser Supremo evoluciona en virtud de su conexión con la Trinidad del Paraíso y a consecuencia de los éxitos de la divinidad de los hijos creadores y administradores de esta Trinidad. El alma inmortal del hombre desarrolla su propio destino eterno asociándose con la presencia divina del Padre Paradisiaco y de acuerdo con las decisiones que la mente humana lleva a cabo como personalidad. La Trinidad significa para Dios Supremo lo mismo que el Ajustador para el hombre en evolución.

\par
%\textsuperscript{(1282.6)}
\textsuperscript{117:3.11} Durante la presente era del universo, el Ser Supremo es en apariencia incapaz de actuar directamente como creador, salvo en aquellos casos en que los agentes creativos del tiempo y del espacio han agotado las posibilidades de acción finitas. Hasta ahora, esto sólo ha sucedido una vez en la historia del universo; cuando se agotaron las posibilidades de acción finita en materia de reflectividad universal, el Supremo actuó como culminador creativo de todas las acciones creadoras anteriores. Y creemos que ejercerá de nuevo su actividad como culminador en las épocas futuras cuando el conjunto de los creadores anteriores haya completado un ciclo apropiado de actividad creativa.

\par
%\textsuperscript{(1283.1)}
\textsuperscript{117:3.12} El Ser Supremo no ha creado al hombre, pero el hombre fue creado literalmente a partir de la potencialidad del Supremo, y su misma vida deriva de esta potencialidad. El Supremo tampoco hace evolucionar al hombre, y sin embargo el Supremo es la esencia misma de la evolución. Desde el punto de vista finito, vivimos, nos movemos y tenemos realmente nuestra existencia\footnote{\textit{Vivimos, nos movemos y existimos}: Job 12:10; Hch 17:28.} dentro de la inmanencia del Supremo.

\par
%\textsuperscript{(1283.2)}
\textsuperscript{117:3.13} El Supremo no puede iniciar aparentemente una causalidad original, pero parece ser el catalizador de todo el crecimiento universal y parece estar destinado a culminar por completo el destino de todos los seres evolutivos y experienciales. El Padre da origen al concepto de un cosmos finito; los Hijos Creadores convierten en un hecho esta idea en el tiempo y el espacio con el consentimiento y la cooperación de los Espíritus Creativos; el Supremo lleva a su culminación la totalidad finita y establece las relaciones de esta totalidad con el destino de lo absonito.

\section*{4. El Dios finito}
\par
%\textsuperscript{(1283.3)}
\textsuperscript{117:4.1} Cuando vemos las luchas incesantes de las criaturas de toda la creación por alcanzar el estado perfecto y la divinidad de existencia, no podemos más que creer que estos esfuerzos interminables demuestran la lucha continua del Supremo por lograr su propia realización divina. Dios Supremo es la Deidad finita, y tiene que hacer frente a los problemas de lo finito en el sentido total de esta palabra. Nuestras luchas contra las vicisitudes del tiempo en las evoluciones del espacio son un reflejo de sus esfuerzos por conseguir la realidad de su yo y la culminación de su soberanía, dentro de la esfera de acción que su naturaleza evolutiva está ampliando hasta los límites extremos de lo posible.

\par
%\textsuperscript{(1283.4)}
\textsuperscript{117:4.2} El Supremo lucha por expresarse en todo el gran universo. Su evolución divina está basada en cierta medida en las acciones y la sabiduría de cada personalidad que existe. Cuando un ser humano escoge la supervivencia eterna, está cocreando su destino; y el Dios finito encuentra, en la vida de ese mortal ascendente, un aumento de la autorrealización de su personalidad y una ampliación de su soberanía experiencial. Pero si una criatura rechaza la carrera eterna, aquella parte del Supremo que dependía de la elección de dicha criatura experimenta un retraso inevitable, una privación que ha de ser compensada con una experiencia sustitutiva o colateral. En cuanto a la personalidad del no sobreviviente, es absorbida en la superalma de la creación, volviéndose una parte de la Deidad del Supremo.

\par
%\textsuperscript{(1283.5)}
\textsuperscript{117:4.3} Dios es tan confiado, tan amoroso, que pone una parte de su naturaleza divina en las manos de los seres humanos para que la custodien y se autorrealicen. La naturaleza del Padre, la presencia del Ajustador, es indestructible, sin tener en cuenta la elección del ser mortal. El hijo del Supremo, el yo en evolución, puede ser destruido, aunque la personalidad potencialmente unificadora de ese yo descaminado subsista como un factor de la Deidad de Supremacía.

\par
%\textsuperscript{(1283.6)}
\textsuperscript{117:4.4} La personalidad humana puede destruir realmente la individualidad de su condición como criatura, y aunque subsista todo aquello que vale la pena en la vida de esa suicida cósmica, \textit{esas cualidades no sobrevivirán como una criaturaindividual}. El Supremo encontrará una nueva expresión entre las criaturas del universo, pero nunca más bajo la forma de aquella persona particular; la personalidad única de un no ascendente regresa al Supremo como una gota de agua vuelve al mar.

\par
%\textsuperscript{(1284.1)}
\textsuperscript{117:4.5} Cualquier acción aislada de las partes personales de lo finito tiene relativamente poca importancia para la aparición final del Todo Supremo, pero el todo depende no obstante de la totalidad de los actos de sus múltiples partes. La personalidad de un mortal individual es insignificante en presencia del total de la Supremacía, pero la personalidad de cada ser humano representa un significado y un valor irreemplazables en lo finito; una vez que la personalidad ha sido expresada, nunca más hallará una expresión idéntica salvo en la existencia continua de esa personalidad viviente.

\par
%\textsuperscript{(1284.2)}
\textsuperscript{117:4.6} Y así, a medida que nos esforzamos por expresar nuestro yo, el Supremo se esfuerza en nosotros y con nosotros por expresar la deidad. Al igual que nosotros encontramos al Padre, el Supremo encuentra de nuevo al Creador Paradisiaco de todas las cosas. A medida que dominamos los problemas de nuestra autorrealización, el Dios de la experiencia consigue la supremacía todopoderosa en los universos del tiempo y del espacio.

\par
%\textsuperscript{(1284.3)}
\textsuperscript{117:4.7} La humanidad no asciende sin esfuerzo en el universo, y el Supremo tampoco evoluciona sin una actividad decidida e inteligente. Las criaturas no alcanzan la perfección mediante la simple pasividad, y el espíritu de la Supremacía no puede convertir en un hecho el poder del Todopoderoso sin un continuo ministerio de servicio hacia la creación finita.

\par
%\textsuperscript{(1284.4)}
\textsuperscript{117:4.8} La relación temporal entre el hombre y el Supremo es la base de la moralidad cósmica, la sensibilidad universal al \textit{deber}, y su aceptación. Se trata de una moralidad que trasciende el sentido temporal del bien y del mal relativos; es una moralidad basada directamente en la apreciación por parte de la criatura consciente de sí misma de una obligación experiencial hacia la Deidad experiencial. El hombre mortal y todas las demás criaturas finitas son creados a partir del potencial viviente de energía, de mente y de espíritu que existe en el Supremo. El ascendente mortal provisto de un Ajustador extrae del Supremo los recursos para crear el carácter inmortal y divino de un finalitario. El Ajustador teje en la realidad misma del Supremo, con el consentimiento de la voluntad humana, los modelos de la naturaleza eterna de un hijo ascendente de Dios.

\par
%\textsuperscript{(1284.5)}
\textsuperscript{117:4.9} La evolución de los progresos que realiza el Ajustador para espiritualizar y eternizar a una personalidad humana producen directamente un aumento de la soberanía del Supremo. Estos logros de la evolución humana son al mismo tiempo unos logros para la manifestación evolutiva del Supremo. Aunque es cierto que las criaturas no podrían evolucionar sin el Supremo, es probable que también sea cierto que la evolución del Supremo nunca podrá alcanzar su plenitud sin que todas las criaturas finalicen su evolución. La gran responsabilidad cósmica de las personalidades conscientes de sí mismas radica en que la Deidad Suprema depende en cierto sentido de la elección de la voluntad humana. Y los mecanismos inescrutables de la reflectividad universal indican de manera fiel y completa a los Ancianos de los Días el progreso recíproco de la evolución de las criaturas y de la evolución del Supremo.

\par
%\textsuperscript{(1284.6)}
\textsuperscript{117:4.10} El gran desafío que ha sido lanzado a los hombres mortales es el siguiente: ¿Decidiréis personalizar en vuestra propia individualidad evolutiva los significados válidos y experimentables del cosmos? O al rechazar la supervivencia, ¿permitiréis que estos secretos de la Supremacía permanezcan inactivos, en espera de la acción de otra criatura que en alguna otra época intente contribuir a \textit{su} manera, como criatura, a la evolución del Dios finito?. Pero entonces se tratará de su contribución al Supremo, no de la vuestra.

\par
%\textsuperscript{(1284.7)}
\textsuperscript{117:4.11} La gran lucha de esta era del universo tiene lugar entre lo potencial y lo manifestado ---todo lo que hasta ahora no se ha expresado, trata de manifestarse. Cuando el hombre mortal avanza en la aventura del Paraíso, sigue los movimientos del tiempo que fluyen como corrientes en el río de la eternidad; cuando el hombre mortal rechaza la carrera eterna, se mueve en contra de la corriente de los acontecimientos de los universos finitos. La creación mecánica se mueve inexorablemente de acuerdo con el objetivo en vías de revelarse del Padre Paradisiaco, pero la creación volitiva tiene la elección de aceptar o rechazar el papel de la participación de la personalidad en la aventura de la eternidad. El hombre mortal no puede destruir los valores supremos de la existencia humana, pero puede impedir muy claramente la evolución de esos valores en su propia experiencia personal. En la medida en que el yo humano rehúsa así participar en la ascensión al Paraíso, en esa misma medida el Supremo sufre un retraso en conseguir expresar su divinidad en el gran universo.

\par
%\textsuperscript{(1285.1)}
\textsuperscript{117:4.12} El hombre mortal ha recibido a su cuidado no solamente la presencia del Padre Paradisiaco bajo la forma de Ajustador, sino también el control sobre el destino de una fracción infinitesimal del futuro del Supremo. Porque al igual que el hombre alcanza su destino humano, el Supremo consigue su destino en los niveles de la deidad.

\par
%\textsuperscript{(1285.2)}
\textsuperscript{117:4.13} Así pues, la decisión os espera a cada uno de vosotros, como en otro tiempo nos esperó a cada uno de nosotros: ¿Le fallaréis al Dios del tiempo, que depende tanto de las decisiones de la mente finita? ¿Le fallaréis a la personalidad Suprema de los universos mediante la pereza de la regresión animal? ¿Le fallaréis al gran hermano de todas las criaturas, que tanto depende de cada criatura? ¿Podéis permitiros pasar al reino de lo irrealizado, cuando se encuentra delante de vosotros la perspectiva encantadora de la carrera universal ---el divino descubrimiento del Padre Paradisiaco y la divina participación en la búsqueda y la evolución del Dios de la Supremacía?

\par
%\textsuperscript{(1285.3)}
\textsuperscript{117:4.14} Los dones de Dios ---sus donaciones de la realidad--- no son separaciones de sí mismo; él no aparta a la creación de sí mismo, pero ha establecido tensiones en las creaciones que rodean al Paraíso. Dios es el primero que ama al hombre y le confiere el potencial de la inmortalidad ---la realidad eterna. A medida que el hombre ama a Dios, el hombre se vuelve eterno en manifestación. Y he aquí un misterio: cuanto más estrechamente se acerca el hombre a Dios a través del amor, mayor es la realidad ---la manifestación--- de ese hombre. Cuanto más se aleja el hombre de Dios, más cerca se aproxima a la no realidad ---al cese de la existencia. Cuando el hombre consagra su voluntad a hacer la voluntad del Padre, cuando el hombre da a Dios todo lo que \textit{tiene}, entonces Dios hace que ese hombre sea más de lo que es\footnote{\textit{Plan de Dios para nosotros}: Jn 3:16; 15:16; 1 Jn 4:10,19.}.

\section*{5. La superalma de la creación}
\par
%\textsuperscript{(1285.4)}
\textsuperscript{117:5.1} El gran Supremo es la superalma cósmica del gran universo. Las cualidades y cantidades del cosmos encuentran en él su reflejo de deidad; su naturaleza de deidad es un mosaico compuesto por la inmensa totalidad de todas las naturalezas de los Creadores y las criaturas de todos los universos en evolución. Y el Supremo es también una Deidad en vías de manifestación, que incorpora una voluntad creativa que abarca un objetivo universal en evolución.

\par
%\textsuperscript{(1285.5)}
\textsuperscript{117:5.2} Los yoes intelectuales, potencialmente personales, de lo finito emergen de la Fuente-Centro Tercera y logran su síntesis finita espacio-temporal como Deidad en el Supremo. Cuando la criatura se somete a la voluntad del Creador, no sumerge ni renuncia a su personalidad; las personalidades individuales que participan en el proceso de la manifestación del Dios finito no pierden su individualidad volitiva por actuar así. Estas personalidades crecen más bien progresivamente mediante su participación en esta gran aventura de la Deidad; mediante esta unión con la divinidad, el hombre eleva, enriquece, espiritualiza y unifica su yo en evolución hasta el umbral mismo de la supremacía.

\par
%\textsuperscript{(1286.1)}
\textsuperscript{117:5.3} El alma inmortal y evolutiva del hombre, creación conjunta de la mente material y del Ajustador, asciende como tal hasta el Paraíso, y cuando es enrolada posteriormente en el Cuerpo de la Finalidad, se conecta de alguna nueva manera con el circuito de la gravedad espiritual del Hijo Eterno mediante una técnica experiencial conocida con el nombre de \textit{trascendenciafinalitaria}. Estos finalitarios se convierten entonces en candidatos aceptables para ser reconocidos experiencialmente como personalidades de Dios Supremo. Cuando estos intelectos mortales alcancen la séptima etapa de la existencia espiritual en las misiones futuras no reveladas del Cuerpo de la Finalidad, sus mentes duales se volverán trinas. Estas dos mentes sintonizadas, la humana y la divina, serán glorificadas en unión con la mente experiencial del Ser Supremo, que para entonces ya estará manifestado.

\par
%\textsuperscript{(1286.2)}
\textsuperscript{117:5.4} En el eterno futuro, Dios Supremo estará manifestado ---creativamente expresado y espiritualmente descrito--- en la mente espiritualizada, en el alma inmortal, del hombre ascendente, al igual que el Padre Universal fue revelado así en la vida terrestre de Jesús.

\par
%\textsuperscript{(1286.3)}
\textsuperscript{117:5.5} El hombre no se une con el Supremo y sumerge su identidad personal, pero las repercusiones universales de la experiencia de todos los hombres forman una parte de la experimentación divina del Supremo. <<El acto es nuestro, pero sus consecuencias pertenecen a Dios>>\footnote{\textit{El acto es nuestro, pero sus consecuencias pertenecen a Dios}: 1 Co 3:6-7.}.

\par
%\textsuperscript{(1286.4)}
\textsuperscript{117:5.6} La personalidad en progreso deja un rastro de realidad manifestada a medida que atraviesa los niveles ascendentes de los universos. Las creaciones crecientes del tiempo y del espacio, ya sean mentales, espirituales o energéticas, son modificadas por el progreso de la personalidad a través de sus dominios. Cuando el hombre actúa, el Supremo reacciona, y esta operación constituye el hecho del progreso.

\par
%\textsuperscript{(1286.5)}
\textsuperscript{117:5.7} Los grandes circuitos de la energía, la mente y el espíritu no son nunca una propiedad permanente de la personalidad ascendente; estos ministerios son siempre una parte de la Supremacía. En la experiencia mortal, el intelecto humano reside en las pulsaciones rítmicas de los espíritus ayudantes de la mente, y efectúa sus decisiones dentro del campo causado por su inclusión en el circuito de este ministerio. Después de la muerte física, el yo humano es separado para siempre del circuito de los ayudantes. Parece ser que estos ayudantes nunca transmiten la experiencia de una personalidad a otra, pero las repercusiones impersonales de las acciones y decisiones pueden transmitirlas, y de hecho lo hacen, hasta Dios Supremo a través de Dios Séptuple. (Al menos esto es así en lo que concierne a los ayudantes de la adoración y de la sabiduría).

\par
%\textsuperscript{(1286.6)}
\textsuperscript{117:5.8} Lo mismo sucede con los circuitos espirituales: el hombre los utiliza durante su ascensión por los universos, pero nunca llega a poseerlos como parte de su personalidad eterna. Estos circuitos del ministerio espiritual, ya sea el Espíritu de la Verdad, el Espíritu Santo o las presencias espirituales superuniversales, son receptivos y reactivos a los valores emergentes de la personalidad ascendente, y estos valores son transmitidos fielmente al Supremo a través del Séptuple.

\par
%\textsuperscript{(1286.7)}
\textsuperscript{117:5.9} Aunque estas influencias espirituales, como el Espíritu Santo y el Espíritu de la Verdad, sean unos ministerios de los universos locales, su guía no está confinada totalmente a los límites geográficos de una creación local determinada. A medida que el mortal ascendente pasa más allá de las fronteras de su universo local de origen, no se encuentra privado por completo del ministerio del Espíritu de la Verdad que lo ha guiado y enseñado constantemente a través de los laberintos filosóficos de los mundos materiales y morontiales, dirigiendo infaliblemente al peregrino del Paraíso en cada crisis de la ascensión, diciéndole siempre: <<Éste es el camino>>\footnote{\textit{Éste es el camino}: Is 30:21.}. Cuando dejéis los dominios del universo local, el espíritu directivo reconfortante de los Hijos Paradisiacos de Dios que se donan continuará guiando vuestra ascensión hacia el Paraíso por medio del ministerio del espíritu del Ser Supremo emergente y mediante las disposiciones de la reflectividad superuniversal.

\par
%\textsuperscript{(1287.1)}
\textsuperscript{117:5.10} Estos múltiples circuitos del ministerio cósmico, ¿cómo registran en el Supremo los significados, los valores y los hechos de la experiencia evolutiva? No estamos totalmente seguros, pero creemos que este registro tiene lugar a través de las personas de los Creadores Supremos de origen Paradisiaco, que son los donadores directos de estos circuitos del tiempo y del espacio. La experiencia mental acumulada de los siete espíritus ayudantes de la mente durante su ministerio en el nivel físico del intelecto es una parte de la experiencia de la Ministra Divina en el universo local, y a través de este Espíritu Creativo llega probablemente a registrarse en la mente de la Supremacía. Las experiencias de los mortales con el Espíritu de la Verdad y el Espíritu Santo también se registran probablemente mediante técnicas similares en la persona de la Supremacía.

\par
%\textsuperscript{(1287.2)}
\textsuperscript{117:5.11} Incluso la experiencia del hombre y del Ajustador debe encontrar su resonancia en la divinidad de Dios Supremo, pues los Ajustadores se parecen al Supremo en la manera de obtener su experiencia, y el alma evolutiva del hombre mortal es creada a partir de la posibilidad preexistente dentro del Supremo para llevar a cabo esta experiencia.

\par
%\textsuperscript{(1287.3)}
\textsuperscript{117:5.12} De esta manera, las múltiples experiencias de toda la creación se vuelven una parte de la evolución de la Supremacía. Las criaturas se limitan a utilizar las cualidades y cantidades de lo finito a medida que ascienden hacia el Padre; las consecuencias impersonales de esta utilización forman parte para siempre del cosmos viviente, de la persona Suprema.

\par
%\textsuperscript{(1287.4)}
\textsuperscript{117:5.13} Aquello que el hombre se lleva consigo como posesión de su personalidad son las consecuencias sobre su carácter de la experiencia de haber utilizado los circuitos mentales y espirituales del gran universo durante su ascensión al Paraíso. Cuando el hombre toma una decisión, y consuma esta decisión en una acción, el hombre efectúa una experiencia; los significados y valores de esta experiencia forman parte para siempre de su carácter eterno en todos los niveles, desde el finito hasta el final. Un carácter cósmicamente moral y divinamente espiritual representa la acumulación capital de las decisiones personales de la criatura, unas decisiones que han sido iluminadas por la adoración sincera, glorificadas por el amor inteligente, y consumadas en el servicio fraternal.

\par
%\textsuperscript{(1287.5)}
\textsuperscript{117:5.14} El Supremo en evolución compensará finalmente a las criaturas finitas por su incapacidad para conseguir algo más que un contacto experiencial limitado con el universo de universos. Las criaturas pueden alcanzar al Padre Paradisiaco, pero como sus mentes evolutivas son finitas, son incapaces de comprender realmente al Padre infinito y absoluto. Pero, puesto que todas las experiencias de las criaturas se registran en el Supremo y forman parte de él, cuando todas las criaturas alcancen el nivel final de la existencia finita, y después de que el desarrollo total del universo les permita alcanzar a Dios Supremo como presencia manifestada de la divinidad, entonces, el hecho de este contacto llevará implícito el ponerse en contacto con la totalidad de la experiencia. Lo finito del tiempo contiene en sí mismo las semillas de la eternidad; y nos han enseñado que cuando la plenitud de la evolución agote la capacidad para el crecimiento cósmico, la totalidad de lo finito se embarcará en las fases absonitas de la carrera eterna en busca del Padre como Último.

\section*{6. La búsqueda del Supremo}
\par
%\textsuperscript{(1287.6)}
\textsuperscript{117:6.1} Buscamos al Supremo en los universos, pero no lo encontramos. <<Él es el interior y el exterior de todas las cosas y de todos los seres, en movimiento y en reposo. Irreconocible en su misterio, está próximo aunque lejano>>\footnote{\textit{Él es el interior y el exterior}: Lc 11:40.}. El Todopoderoso Supremo es <<la forma de lo que aún no se ha formado, el arquetipo de lo que aún no se ha creado>>. El Supremo es vuestro hogar universal, y cuando lo encontréis, será como regresar al hogar. Es vuestro padre experiencial, y al igual que en la experiencia de los seres humanos, el Supremo ha crecido en la experiencia de la paternidad divina. Os conoce porque se parece a una criatura así como a un creador.

\par
%\textsuperscript{(1288.1)}
\textsuperscript{117:6.2} Si deseáis de verdad encontrar a Dios, no podréis evitar que nazca en vuestra mente la conciencia del Supremo. Al igual que Dios es vuestro Padre divino, el Supremo es vuestra Madre divina, de quien os alimentáis durante toda vuestra vida como criaturas del universo. <<¡Cuán universal es el Supremo ---está en todas partes! Las criaturas ilimitadas de la creación dependen de su presencia para vivir, y a ninguna se les rehúsa>>.

\par
%\textsuperscript{(1288.2)}
\textsuperscript{117:6.3} El Supremo es para el cosmos finito lo mismo que Miguel para Nebadon; su Deidad es la gran avenida por la que fluye exteriormente el amor del Padre hacia toda la creación, y él es la gran avenida por la que las criaturas finitas pasan hacia el interior en busca del Padre, que es amor. Incluso los Ajustadores del Pensamiento están relacionados con el Supremo; en su naturaleza y divinidad originales se parecen al Padre, pero cuando experimentan las operaciones del tiempo en los universos del espacio, se vuelven semejantes al Supremo.

\par
%\textsuperscript{(1288.3)}
\textsuperscript{117:6.4} El acto de la criatura consistente en escoger hacer la voluntad del Creador es un valor cósmico y posee un significado universal ante los cuales reacciona inmediatamente una fuerza de coordinación no revelada, pero omnipresente, que es probablemente la actividad cada vez más extensa del Ser Supremo.

\par
%\textsuperscript{(1288.4)}
\textsuperscript{117:6.5} El alma morontial de un mortal evolutivo es realmente la hija de la acción del Padre Universal a través del Ajustador, y la hija de la reacción cósmica del Ser Supremo, la Madre Universal. La influencia materna domina la personalidad humana durante toda la infancia, en el universo local, del alma en crecimiento. La influencia de los padres Divinos se hace más equivalente después de fusionar con el Ajustador y durante la carrera en el superuniverso, pero cuando las criaturas del tiempo empiezan la travesía del eterno universo central, la naturaleza Paterna se pone de manifiesto cada vez más, alcanzando el apogeo de su manifestación finita después de reconocer al Padre Universal y de ser admitidas en el Cuerpo de la Finalidad.

\par
%\textsuperscript{(1288.5)}
\textsuperscript{117:6.6} Durante la experiencia de alcanzar el estado finalitario, el contacto y la inyección de la presencia espiritual del Hijo Eterno y de la presencia mental del Espíritu Infinito afectan enormemente a las cualidades maternas experienciales del yo ascendente. Luego aparece, en todos los campos de actividad finalitaria en el gran universo, un nuevo despertar del potencial materno latente del Supremo, una nueva comprensión de los significados experienciales, y una nueva síntesis de los valores experienciales de toda la carrera ascendente. Parece ser que esta realización del yo continuará durante la carrera universal de los finalitarios de la sexta fase hasta que la herencia materna del Supremo consiga una sincronía finita con la herencia del Padre, representada por el Ajustador. Este período de actividad fascinante en el gran universo representa la continuación de la carrera adulta del mortal ascendente perfeccionado.

\par
%\textsuperscript{(1288.6)}
\textsuperscript{117:6.7} Cuando se culmine la sexta fase de la existencia y se entre en la séptima y última fase del estado espiritual, empezarán probablemente las épocas progresivas durante las cuales la experiencia se enriquecerá, la sabiduría madurará y la divinidad se hará más comprensible. Esto equivaldrá probablemente, en la naturaleza del finalitario, a la finalización total de la lucha mental por autorrealizarse espiritualmente, a la coordinación definitiva entre la naturaleza humana ascendente y la naturaleza divina del Ajustador, dentro de los límites de las posibilidades finitas. Este magnífico yo universal se vuelve así el hijo finalitario eterno del Padre Paradisiaco así como el hijo universal eterno del Supremo Madre, un yo universal capacitado para representar tanto al Padre como a la Madre de los universos y de las personalidades en cualquier actividad o empresa relacionada con la administración finita de las cosas y los seres creados, creadores o evolutivos.

\par
%\textsuperscript{(1289.1)}
\textsuperscript{117:6.8} Todos los humanos cuyas almas evolucionan son literalmente los hijos evolutivos de Dios Padre y de Dios Madre, el Ser Supremo. Pero hasta el momento en que el hombre mortal se vuelve consciente en su alma de su herencia divina, esta seguridad de su parentesco con la Deidad debe obtenerla por medio de la fe. La experiencia de la vida humana es el capullo cósmico donde los dones universales del Ser Supremo y la presencia universal del Padre Universal (unos dones y una presencia que no son personalidades), hacen evolucionar el alma morontial del tiempo y el carácter finalitario humano-divino que tienen un destino universal y un servicio eterno.

\par
%\textsuperscript{(1289.2)}
\textsuperscript{117:6.9} Los hombres olvidan demasiado a menudo que Dios es la experiencia más grande de la existencia humana. Las otras experiencias están limitadas en su naturaleza y en su contenido, pero la experiencia de Dios no tiene límites, salvo los de la capacidad de comprensión de las criaturas, y esta experiencia misma amplía por sí misma dicha capacidad. Cuando los hombres buscan a Dios, lo están buscando todo. Cuando encuentran a Dios, lo han encontrado todo. La búsqueda de Dios es la donación ilimitada de amor que viene acompañada del asombroso descubrimiento de un nuevo amor más grande que otorgar.

\par
%\textsuperscript{(1289.3)}
\textsuperscript{117:6.10} Todo amor verdadero procede de Dios\footnote{\textit{Todo amor verdadero procede de Dios}: 1 Jn 4:7.}, y el hombre recibe el afecto divino a medida que ofrece este amor a sus semejantes. El amor es dinámico. Nunca puede ser apresado; es vivo, libre, emocionante y está siempre en movimiento. El hombre nunca puede coger el amor del Padre y encarcelarlo dentro de su corazón. El amor del Padre sólo puede volverse real para el hombre mortal cuando pasa a través de la personalidad de ese hombre a medida que otorga a su vez este amor a sus semejantes. El gran circuito del amor procede del Padre, pasa de los hijos a los hermanos, y de ahí se dirige al Supremo. El amor del Padre aparece en la personalidad del mortal mediante el ministerio del Ajustador interior. Este hijo que conoce a Dios revela este amor a sus hermanos del universo, y este afecto fraternal es la esencia del amor del Supremo.

\par
%\textsuperscript{(1289.4)}
\textsuperscript{117:6.11} La única forma de acercarse al Supremo es a través de la experiencia, y en las épocas actuales de la creación sólo existen tres caminos para que las criaturas se aproximen a la Supremacía:

\par
%\textsuperscript{(1289.5)}
\textsuperscript{117:6.12} 1. Los Ciudadanos del Paraíso descienden de la Isla eterna a través de Havona, donde adquieren la capacidad de comprender la Supremacía observando el diferencial de realidad entre el Paraíso y Havona, y descubriendo por exploración las múltiples actividades de las Personalidades Creadoras Supremas que van desde los Espíritus Maestros hasta los Hijos Creadores.

\par
%\textsuperscript{(1289.6)}
\textsuperscript{117:6.13} 2. Los ascendentes espacio-temporales que suben de los universos evolutivos de los Creadores Supremos se acercan mucho al Supremo durante la travesía de Havona, como paso preliminar hacia una apreciación creciente de la unidad de la Trinidad del Paraíso.

\par
%\textsuperscript{(1289.7)}
\textsuperscript{117:6.14} 3. Los nativos de Havona adquieren una comprensión del Supremo a través de los contactos con los peregrinos descendentes del Paraíso y con los peregrinos ascendentes de los siete superuniversos. Los nativos de Havona se encuentran de manera inherente en la posición de armonizar los puntos de vista esencialmente diferentes de los ciudadanos de la Isla eterna y de los ciudadanos de los universos evolutivos.

\par
%\textsuperscript{(1290.1)}
\textsuperscript{117:6.15} Las criaturas evolutivas disponen de siete grandes maneras de acercarse al Padre Universal, y cada una de estas vías de ascensión al Paraíso pasa por la divinidad de uno de los Siete Espíritus Maestros; la criatura puede realizar cada uno de estos acercamientos porque ha servido en el superuniverso que refleja la naturaleza de ese Espíritu Maestro, y ha conseguido una ampliación de su receptividad experiencial. La suma total de estas siete experiencias constituye el límite actualmente conocido que puede tener la conciencia de una criatura sobre la realidad y la manifestación de Dios Supremo.

\par
%\textsuperscript{(1290.2)}
\textsuperscript{117:6.16} Las limitaciones propias del hombre no son las únicas que le impiden encontrar al Dios finito; es también el estado incompleto del universo; incluso el estado incompleto de todas las criaturas --- pasadas, presentes y futuras--- hace que el Supremo sea inaccesible. Cualquier persona que ha alcanzado el nivel divino de parecerse a Dios puede encontrar a Dios Padre, pero ninguna criatura \textit{individual} podrá descubrir nunca personalmente a Dios Supremo hasta el momento lejano en que \textit{todas} las criaturas lo encontrarán simultáneamente después de haberse alcanzado la perfección universal.

\par
%\textsuperscript{(1290.3)}
\textsuperscript{117:6.17} A pesar del hecho de que en esta era del universo no podéis encontrar personalmente al Supremo como podéis y encontraréis al Padre, al Hijo y al Espíritu, sin embargo la ascensión al Paraíso y la carrera universal posterior crearán gradualmente en vuestra conciencia el reconocimiento de la presencia universal y de la actividad cósmica del Dios de toda la experiencia. Los frutos del espíritu son la sustancia del Supremo tal como éste es comprensible en la experiencia humana.

\par
%\textsuperscript{(1290.4)}
\textsuperscript{117:6.18} El hecho de que el hombre alcance algún día al Supremo es una consecuencia de su fusión con el espíritu de la Deidad del Paraíso. Para los urantianos, este espíritu es la presencia del Ajustador del Padre Universal; y aunque el Monitor de Misterio procede del Padre y es como el Padre, dudamos de que incluso este don divino pueda conseguir la tarea imposible de revelar la naturaleza del Dios infinito a una criatura finita. Sospechamos que lo que los Ajustadores revelarán a los futuros finalitarios de la séptima fase será la divinidad y la naturaleza de Dios Supremo. Y esta revelación representará para una criatura finita lo mismo que la revelación del Infinito para un ser absoluto.

\par
%\textsuperscript{(1290.5)}
\textsuperscript{117:6.19} El Supremo no es infinito, pero abarca probablemente toda aquella parte de la infinidad que una criatura finita pueda llegar a entender nunca realmente. ¡Comprender más que el Supremo es ser más que finito!

\par
%\textsuperscript{(1290.6)}
\textsuperscript{117:6.20} Todas las creaciones experienciales dependen unas de otras para hacer realidad su destino. Sólo la realidad existencial está contenida en sí misma y existe por sí misma. Havona y los siete superuniversos se necesitan mutuamente para alcanzar el máximo de consecución finita; y algún día dependerán también de los universos futuros del espacio exterior para trascender lo finito.

\par
%\textsuperscript{(1290.7)}
\textsuperscript{117:6.21} Un ascendente humano puede encontrar al Padre; Dios es existencial y por lo tanto real, sin tener en cuenta el estado de la experiencia en el universo total. Pero ningún ascendente individual encontrará nunca al Supremo hasta que todos los ascendentes hayan alcanzado la máxima madurez universal que los capacite para participar simultáneamente en este descubrimiento.

\par
%\textsuperscript{(1290.8)}
\textsuperscript{117:6.22} El Padre no hace acepción de personas\footnote{\textit{No hace acepción de personas}: 2 Cr 19:7; Job 34:19; Eclo 35:12; Hch 10:34; Ro 2:11; Gl 2:6; 3:28; Ef 6:9; Col 3:11.}; trata a cada uno de sus hijos ascendentes como individuos cósmicos. El Supremo tampoco hace acepción de personas; trata a sus hijos experienciales como una sola totalidad cósmica.

\par
%\textsuperscript{(1290.9)}
\textsuperscript{117:6.23} El hombre puede descubrir al Padre en su corazón, pero tendrá que buscar al Supremo en el corazón de todos los demás hombres; y cuando todas las criaturas revelen perfectamente el amor del Supremo, éste se convertirá entonces en una realidad universal para todas las criaturas. Esto es simplemente otra manera de decir que los universos se habrán establecido en la luz y la vida.

\par
%\textsuperscript{(1291.1)}
\textsuperscript{117:6.24} El hecho de alcanzar una autorrealización perfeccionada por parte de todas las personalidades, más el logro del equilibrio perfeccionado en todos los universos, equivale a alcanzar al Supremo y atestigua que toda la realidad finita se ha liberado de las limitaciones de la existencia incompleta. Este agotamiento de todos los potenciales finitos permite alcanzar completamente al Supremo, y se puede definir de otra manera como la completa manifestación evolutiva del Ser Supremo mismo.

\par
%\textsuperscript{(1291.2)}
\textsuperscript{117:6.25} Los hombres no encuentran al Supremo de una manera espectacular y repentina como un terremoto que abre abismos entre las rocas, sino que lo encuentran lenta y pacientemente como un río que desgasta suavemente el lecho subyacente.

\par
%\textsuperscript{(1291.3)}
\textsuperscript{117:6.26} Cuando encontréis al Padre, habréis encontrado la gran causa de vuestra ascensión espiritual por los universos; cuando encontréis al Supremo, descubriréis el gran resultado de vuestra carrera de progreso hacia el Paraíso.

\par
%\textsuperscript{(1291.4)}
\textsuperscript{117:6.27} Pero ningún mortal que conoce a Dios estará nunca solo en su viaje por el cosmos, porque sabe que el Padre camina a su lado en cada etapa del camino, mientras que el camino mismo que atraviesa es la presencia del Supremo\footnote{\textit{Dios está con nosotros}: Sal 23:4; Is 43:2; Hch 18:10.}.

\section*{7. El futuro del Supremo}
\par
%\textsuperscript{(1291.5)}
\textsuperscript{117:7.1} La realización completa de todos los potenciales finitos equivale a la culminación de la realización de toda la experiencia evolutiva. Esto sugiere la aparición final del Supremo como presencia todopoderosa de la Deidad en los universos. Creemos que el Supremo, en este estado de su desarrollo, estará tan diferenciadamente personalizado como lo está el Hijo Eterno, tan concretamente dotado de poder como lo está la Isla del Paraíso, tan completamente unificado como lo está el Actor Conjunto, y todo ello dentro de los límites de las posibilidades finitas de la Supremacía en el momento de culminar la presente era del universo.

\par
%\textsuperscript{(1291.6)}
\textsuperscript{117:7.2} Aunque esto representa un concepto totalmente adecuado del futuro del Supremo, desearíamos llamar la atención sobre ciertos problemas inherentes a este concepto:

\par
%\textsuperscript{(1291.7)}
\textsuperscript{117:7.3} 1. Los Supervisores Incalificados del Supremo difícilmente podrían ser dotados de deidad en una fase anterior a la evolución consumada del Supremo, y sin embargo estos mismos supervisores ejercen actualmente la soberanía de la supremacía, de manera limitada, en los universos establecidos en la luz y la vida.

\par
%\textsuperscript{(1291.8)}
\textsuperscript{117:7.4} 2. El Supremo difícilmente podría ejercer sus funciones en la Trinidad Última hasta que no haya alcanzado la manifestación completa de su estado universal, y sin embargo la Trinidad Última es actualmente una realidad limitada, y habéis sido informados de la existencia de los Vicegerentes Calificados del Último.

\par
%\textsuperscript{(1291.9)}
\textsuperscript{117:7.5} 3. El Supremo no es completamente real para las criaturas del universo, pero existen numerosas razones para deducir que es totalmente real para la Deidad Séptuple, que abarca desde el Padre Universal en el Paraíso hasta los Hijos Creadores y los Espíritus Creativos de los universos locales.

\par
%\textsuperscript{(1291.10)}
\textsuperscript{117:7.6} Puede ser que en los límites superiores de lo finito, donde el tiempo se une con el tiempo trascendido, exista una especie de difuminación y de mezcla de las secuencias. Puede ser que el Supremo sea capaz de proyectar su presencia universal en esos niveles supertemporales, y luego anticiparse en un grado limitado a su evolución futura, reflejando esta previsión futura hacia atrás sobre los niveles creados como Inmanencia del Incompleto Proyectado. Estos fenómenos se pueden observar dondequiera que lo finito se pone en contacto con lo superfinito, como sucede en las experiencias de los seres humanos que están habitados por los Ajustadores del Pensamiento, los cuales son verdaderas predicciones de los futuros logros universales del hombre durante toda la eternidad.

\par
%\textsuperscript{(1292.1)}
\textsuperscript{117:7.7} Cuando los ascendentes mortales son admitidos en el cuerpo finalitario del Paraíso, prestan juramento a la Trinidad del Paraíso, y al prestar este juramento de lealtad, están prometiendo con ello fidelidad eterna a Dios Supremo, que \textit{es} la Trinidad tal como la pueden comprender todas las personalidades creadas finitas. Posteriormente, cuando las compañías de finalitarios ejercen su actividad en todos los universos en evolución, sólo están sometidas a las órdenes procedentes del Paraíso hasta la época memorable en que los universos locales se establecen en la luz y la vida. A medida que las nuevas organizaciones gubernamentales de estas creaciones perfeccionadas empiezan a reflejar la soberanía emergente del Supremo, observamos que las compañías dispersas de finalitarios reconocen entonces la autoridad jurisdiccional de estos nuevos gobiernos. Parece ser que Dios Supremo evoluciona como unificador del Cuerpo evolutivo de la Finalidad, pero es muy probable que el destino eterno de estos siete cuerpos esté dirigido por el Supremo como miembro que es de la Trinidad Última.

\par
%\textsuperscript{(1292.2)}
\textsuperscript{117:7.8} El Ser Supremo contiene tres posibilidades superfinitas de manifestación en el universo:

\par
%\textsuperscript{(1292.3)}
\textsuperscript{117:7.9} 1. La colaboración absonita en la primera Trinidad experiencial.

\par
%\textsuperscript{(1292.4)}
\textsuperscript{117:7.10} 2. La relación coabsoluta en la segunda Trinidad experiencial.

\par
%\textsuperscript{(1292.5)}
\textsuperscript{117:7.11} 3. La participación coinfinita en la Trinidad de Trinidades, pero no tenemos ningún concepto satisfactorio de lo que esto significa realmente.

\par
%\textsuperscript{(1292.6)}
\textsuperscript{117:7.12} Ésta es una de las hipótesis generalmente aceptadas sobre el futuro del Supremo, pero existen también muchas especulaciones sobre sus relaciones con el gran universo actual, después de que éste haya alcanzado el estado de luz y de vida.

\par
%\textsuperscript{(1292.7)}
\textsuperscript{117:7.13} La meta actual de los superuniversos, tal como son y dentro del límite de sus potenciales, es volverse perfectos como Havona. Esta perfección es propia de la consecución física y espiritual, e incluso del desarrollo de la administración, del gobierno y de la fraternidad. Se cree que en las eras por venir, las posibilidades de que exista falta de armonía, desajustes e inadaptaciones desaparecerán finalmente de los superuniversos. Los circuitos energéticos estarán perfectamente equilibrados y sometidos por completo a la mente, mientras que el espíritu, en presencia de la personalidad, habrá conseguido dominar la mente.

\par
%\textsuperscript{(1292.8)}
\textsuperscript{117:7.14} Se supone que en esa época tan lejana, la persona espiritual del Supremo y el poder que habrá alcanzado el Todopoderoso habrán logrado un desarrollo coordinado, y que los dos, unificados en y por la Mente Suprema, se volverán un hecho como Ser Supremo, una realidad consumada en los universos ---una realidad que será observable por todas las inteligencias de las criaturas, ante la cual reaccionarán todas las energías creadas, estará coordinada en todas las entidades espirituales, y será experimentada por todas las personalidades del universo.

\par
%\textsuperscript{(1292.9)}
\textsuperscript{117:7.15} Este concepto implica la soberanía efectiva del Supremo en el gran universo. Es muy probable que los administradores actuales de la Trinidad continúen como vicegerentes del Supremo, pero creemos que las demarcaciones actuales entre los siete superuniversos desaparecerán gradualmente, y que todo el gran universo funcionará como un conjunto perfeccionado.

\par
%\textsuperscript{(1292.10)}
\textsuperscript{117:7.16} Es posible que el Supremo resida entonces personalmente en Uversa, la sede central de Orvonton, desde donde dirigirá la administración de las creaciones temporales, pero en realidad esto no es más que una suposición. Sin embargo, es cierto que se podrá contactar claramente con la personalidad del Ser Supremo en un lugar concreto, aunque la ubiquidad de su presencia como Deidad continuará impregnando probablemente el universo de universos. No sabemos qué tipo de relación existirá entre los ciudadanos superuniversales de esa era y el Supremo, pero podría tratarse de algo parecido a las relaciones actuales entre los nativos de Havona y la Trinidad del Paraíso.

\par
%\textsuperscript{(1293.1)}
\textsuperscript{117:7.17} El gran universo perfeccionado de esas épocas del futuro será enormemente diferente a lo que es en la actualidad. Habrán terminado las aventuras emocionantes de la organización de las galaxias del espacio, de la implantación de la vida en los mundos inciertos del tiempo, y de la evolución de la armonía a partir del caos, de la belleza a partir de los potenciales, de la verdad a partir de los significados y de la bondad a partir de los valores. ¡Los universos del tiempo habrán logrado realizar su destino finito! Y quizás habrá descanso durante un espacio de tiempo, una disminución de la lucha secular por conseguir la perfección evolutiva. ¡Pero no será por mucho tiempo! El enigma de la Deidad emergente de Dios Último desafiará de manera cierta, segura e inexorable a estos ciudadanos perfeccionados de los universos estabilizados, al igual que la búsqueda de Dios Supremo desafió en otro tiempo a sus antepasados luchadores y evolutivos. La cortina del destino cósmico se descorrerá para revelar la grandeza trascendente de la atractiva búsqueda absonita para alcanzar al Padre Universal en los niveles nuevos y superiores donde se revela el aspecto último de la experiencia de las criaturas.

\par
%\textsuperscript{(1293.2)}
\textsuperscript{117:7.18} [Patrocinado por un Poderoso Mensajero que reside temporalmente en Urantia].


\chapter{Documento 118. El Supremo y el Último ---el tiempo y el espacio}
\par
%\textsuperscript{(1294.1)}
\textsuperscript{118:0.1} EN RELACIÓN con las diversas naturalezas de la Deidad, se puede decir que:

\par
%\textsuperscript{(1294.2)}
\textsuperscript{118:0.2} 1. El Padre es un yo que existe por sí mismo.

\par
%\textsuperscript{(1294.3)}
\textsuperscript{118:0.3} 2. El Hijo es un yo coexistente.

\par
%\textsuperscript{(1294.4)}
\textsuperscript{118:0.4} 3. El Espíritu es un yo que existe conjuntamente.

\par
%\textsuperscript{(1294.5)}
\textsuperscript{118:0.5} 4. El Supremo es un yo evolutivo-experiencial.

\par
%\textsuperscript{(1294.6)}
\textsuperscript{118:0.6} 5. El Séptuple es una divinidad autodistributiva.

\par
%\textsuperscript{(1294.7)}
\textsuperscript{118:0.7} 6. El Último es un yo trascendental-experiencial.

\par
%\textsuperscript{(1294.8)}
\textsuperscript{118:0.8} 7. El Absoluto es un yo existencial-experiencial.

\par
%\textsuperscript{(1294.9)}
\textsuperscript{118:0.9} Aunque Dios Séptuple es indispensable para alcanzar evolutivamente al Supremo, el Supremo es también indispensable para la aparición final del Último. Y la doble presencia del Supremo y del Último constituye la asociación básica de la Deidad subabsoluta y derivada, porque los dos son interdependientemente complementarios para alcanzar el destino. Juntos constituyen el puente experiencial que conecta los comienzos y las terminaciones de todo crecimiento creativo en el universo maestro.

\par
%\textsuperscript{(1294.10)}
\textsuperscript{118:0.10} El crecimiento creativo es interminable pero siempre satisfactorio, inacabable en extensión pero siempre puntualizado por aquellos momentos, satisfactorios para la personalidad, en que se alcanza una meta transitoria y que sirven tan eficazmente como preludios para la movilización hacia nuevas aventuras de crecimiento cósmico, de exploración del universo y de alcance de la Deidad.

\par
%\textsuperscript{(1294.11)}
\textsuperscript{118:0.11} Aunque el ámbito de las matemáticas está lleno de limitaciones cualitativas, proporciona a la mente finita una base conceptual para examinar la infinidad. Los números no tienen ninguna limitación cuantitativa, ni siquiera en la comprensión de la mente finita. Por muy grande que sea el número que se ha concebido, siempre podéis imaginar uno más a añadir. Podéis comprender también que esto es menor que la infinidad, pues por muchas veces que repitáis esta adición, siempre se podrá añadir un número más.

\par
%\textsuperscript{(1294.12)}
\textsuperscript{118:0.12} Al mismo tiempo, la serie infinita se puede totalizar en un punto dado cualquiera, y este total (o más bien este subtotal) proporciona a una persona determinada, en un momento dado y en un estado determinado, la plenitud del dulzor de haber alcanzado una meta. Pero tarde o temprano esta misma persona empieza a tener hambre y anhelo de metas nuevas y más grandes, y estas aventuras de crecimiento aparecerán constantemente en la plenitud de los tiempos y en los ciclos de la eternidad.

\par
%\textsuperscript{(1294.13)}
\textsuperscript{118:0.13} Cada época universal sucesiva es la antecámara de la era siguiente de crecimiento cósmico, y cada época del universo proporciona un destino inmediato para todas las etapas anteriores. Havona es, en sí misma y por sí misma, una creación perfecta, pero limitada por su perfección; la perfección de Havona, que se extiende hacia los superuniversos evolutivos, no solamente encuentra un destino cósmico sino también la liberación de las limitaciones de la existencia preevolutiva.

\section*{1. El tiempo y la eternidad}
\par
%\textsuperscript{(1295.1)}
\textsuperscript{118:1.1} Al hombre le es útil conseguir, para su orientación cósmica, la máxima comprensión posible de la relación de la Deidad con el cosmos. Aunque la naturaleza de la Deidad absoluta es eterna, los Dioses están relacionados con el tiempo como una experiencia en la eternidad. En los universos evolutivos, la eternidad es la perpetuidad temporal ---el eterno \textit{ahora}.

\par
%\textsuperscript{(1295.2)}
\textsuperscript{118:1.2} La personalidad de la criatura mortal puede eternizarse mediante su identificación con el espíritu interior por medio de la técnica de escoger hacer la voluntad del Padre. Esta consagración de la voluntad equivale a llevar a cabo una intención real y eterna. Esto significa que la intención de la criatura se ha vuelto invariable en relación con la sucesión de los momentos; dicho de otra manera, que la sucesión de los momentos no presenciará ningún cambio en la intención de la criatura. Un millón o mil millones de momentos no supondrán ninguna diferencia. Los números han dejado de tener significado en lo que se refiere a la intención de la criatura. Y así, la elección de la criatura más la elección de Dios se traducen en las realidades eternas de la unión interminable entre el espíritu de Dios y la naturaleza del hombre para el servicio perpetuo de los hijos de Dios y de su Padre Paradisiaco.

\par
%\textsuperscript{(1295.3)}
\textsuperscript{118:1.3} Existe una relación directa entre la madurez y la unidad de la conciencia del tiempo que tiene cualquier intelecto dado. La unidad de tiempo puede ser un día, un año o un período más largo, pero es inevitablemente el criterio mediante el cual el yo consciente evalúa las circunstancias de la vida, y mediante el cual el intelecto que concibe mide y evalúa los hechos de la existencia temporal.

\par
%\textsuperscript{(1295.4)}
\textsuperscript{118:1.4} La experiencia, la sabiduría y el juicio son los fenómenos que acompañan a la prolongación de la unidad de tiempo en la experiencia de los mortales. A medida que la mente humana retrocede en el pasado, evalúa la experiencia pasada con objeto de hacer que influya sobre una situación presente. Cuando la mente se introduce en el futuro, intenta evaluar el significado futuro de una posible acción. Una vez que ha tenido en cuenta así tanto la experiencia como la sabiduría, la voluntad humana ejerce su juicio y su decisión en el presente, y el plan de acción nacido así del pasado y del futuro surge a la existencia.

\par
%\textsuperscript{(1295.5)}
\textsuperscript{118:1.5} En la madurez del yo en desarrollo, el pasado y el futuro se reúnen para iluminar el verdadero significado del presente. A medida que el yo madura, se aleja cada vez más en el pasado en busca de experiencia, mientras que sus previsiones de sabiduría tratan de penetrar cada vez más profundamente en el futuro desconocido. Y a medida que el yo que concibe extiende su alcance cada vez más lejos tanto en el pasado como en el futuro, su juicio depende cada vez menos del presente pasajero. Las acciones y decisiones empiezan de esta manera a liberarse de las trabas del presente en movimiento, mientras que se comienza a aceptar los aspectos de importancia pasado-futura.

\par
%\textsuperscript{(1295.6)}
\textsuperscript{118:1.6} Aquellos mortales cuyas unidades de tiempo son cortas practican la paciencia; la verdadera madurez trasciende la paciencia mediante una tolerancia nacida de una verdadera comprensión.

\par
%\textsuperscript{(1295.7)}
\textsuperscript{118:1.7} Madurar significa vivir más intensamente en el presente, eludiendo al mismo tiempo las limitaciones del presente. Los planes de la madurez, basados en la experiencia pasada, nacen en el presente de tal manera que realzan los valores del futuro.

\par
%\textsuperscript{(1295.8)}
\textsuperscript{118:1.8} La unidad de tiempo de la inmadurez concentra los significados y los valores en el momento presente de tal manera, que separa el presente de su verdadera relación con el no presente ---con el pasado-futuro. La unidad de tiempo de la madurez está proporcionada para revelar la relación coordinada del pasado-presente-futuro de tal forma que el yo empieza a hacerse una idea de la totalidad de los acontecimientos, empieza a ver el paisaje del tiempo desde la perspectiva panorámica de unos horizontes más amplios, empieza quizás a sospechar la existencia del continuo eterno sin comienzo ni fin, cuyos fragmentos se llaman tiempo.

\par
%\textsuperscript{(1296.1)}
\textsuperscript{118:1.9} En los niveles de lo infinito y de lo absoluto, el momento presente contiene todo el pasado así como todo el futuro. YO SOY\footnote{\textit{YO SOY}: Ex 3:14.} significa también YO ERA y YO SERÉ. Y esto representa nuestro mejor concepto de la eternidad y de lo eterno.

\par
%\textsuperscript{(1296.2)}
\textsuperscript{118:1.10} En el nivel absoluto y eterno, la realidad potencial es tan significativa como la realidad manifestada. Sólo en el nivel finito, y para las criaturas atadas al tiempo, parece existir una diferencia tan enorme. Para Dios, como absoluto, un mortal ascendente que ha tomado la decisión eterna es ya un finalitario del Paraíso. Pero el Padre Universal, gracias a los Ajustadores del Pensamiento interiores, no está limitado así en su conocimiento, sino que también puede conocer y participar en todas las luchas temporales con los problemas de la ascensión de las criaturas, desde los niveles de existencia en que se parecen a los animales hasta los niveles de existencia en que se parecen a Dios.

\section*{2. La omnipresencia y la ubiquidad}
\par
%\textsuperscript{(1296.3)}
\textsuperscript{118:2.1} La ubiquidad de la Deidad no se debe confundir con la ultimidad de la omnipresencia divina. Es voluntad del Padre Universal que el Supremo, el Último y el Absoluto compensen, coordinen y unifiquen su ubiquidad espacio-temporal y su omnipresencia en el espacio-tiempo trascendido con su presencia universal y absoluta sin tiempo y sin espacio. Y deberíais recordar que aunque la ubiquidad de la Deidad pueda estar asociada con tanta frecuencia al espacio, no está necesariamente condicionada por el tiempo.

\par
%\textsuperscript{(1296.4)}
\textsuperscript{118:2.2} Como ascendentes mortales y morontiales, discernís progresivamente a Dios a través del ministerio de Dios Séptuple. A Dios Supremo lo descubrís a través de Havona. En el Paraíso lo encontráis como persona y luego, como finalitarios, pronto intentaréis conocerlo como Último. Siendo finalitarios, parece ser que después de haber alcanzado al Último sólo habría un camino a seguir, y sería empezar la búsqueda del Absoluto. Ningún finalitario se sentirá perturbado por las incertidumbres que le asaltarán para alcanzar al Absoluto de la Deidad, puesto que al final de las ascensiones suprema y última había encontrado a Dios Padre. Estos finalitarios creerán sin duda que, aunque consiguieran encontrar a Dios Absoluto, sólo estarían descubriendo al mismo Dios, al Padre Paradisiaco manifestándose en unos niveles más cercanos a lo infinito y a lo universal. El hecho de alcanzar a Dios en lo absoluto revelaría sin duda al Antepasado Primordial de los universos así como al Padre Final de las personalidades.

\par
%\textsuperscript{(1296.5)}
\textsuperscript{118:2.3} Dios Supremo puede no ser una demostración de la omnipresencia espacio-temporal de la Deidad, pero es literalmente una manifestación de la ubiquidad divina. Entre la presencia espiritual del Creador y las manifestaciones materiales de la creación se encuentra el inmenso dominio del \textit{devenir} ubicuo ---la aparición universal de la Deidad evolutiva.

\par
%\textsuperscript{(1296.6)}
\textsuperscript{118:2.4} Si Dios Supremo asume alguna vez el control directo de los universos del tiempo y del espacio, estamos convencidos de que esta administración de la Deidad funcionará bajo el supercontrol del Último. En tal caso, Dios Último empezaría a volverse manifiesto para los universos del tiempo como Todopoderoso trascendental (el Omnipotente), ejerciendo el supercontrol del supertiempo y del espacio trascendido sobre las funciones administrativas del Todopoderoso Supremo.

\par
%\textsuperscript{(1297.1)}
\textsuperscript{118:2.5} La mente mortal se puede preguntar, al igual que lo hacemos nosotros: Si la evolución de Dios Supremo hacia la autoridad administrativa en el gran universo viene acompañada de manifestaciones crecientes de Dios Último, la aparición correspondiente de Dios Último en los presupuestos universos del espacio exterior, ¿vendrá acompañada de revelaciones similares y crecientes de Dios Absoluto? En realidad no lo sabemos.

\section*{3. Las relaciones entre el tiempo y el espacio}
\par
%\textsuperscript{(1297.2)}
\textsuperscript{118:3.1} La Deidad sólo podía unificar sus manifestaciones espacio-temporales para la concepción finita por medio de la ubiquidad, ya que el tiempo es una sucesión de instantes, mientras que el espacio es un sistema de puntos asociados. Después de todo, vosotros percibís el tiempo por análisis y el espacio por síntesis. Coordináis y asociáis estas dos concepciones desiguales mediante la perspicacia integradora de la personalidad. De todo el mundo animal, sólo el hombre posee esta manera de percibir el espacio-tiempo. Para un animal, el movimiento tiene un significado, pero el movimiento sólo representa un valor para una criatura con categoría de personalidad.

\par
%\textsuperscript{(1297.3)}
\textsuperscript{118:3.2} Las cosas están condicionadas por el tiempo, pero la verdad está fuera del tiempo. Cuanta más verdad conocéis, más verdad \textit{sois}, más cosas podéis entender del pasado y comprender del futuro.

\par
%\textsuperscript{(1297.4)}
\textsuperscript{118:3.3} La verdad es inamovible ---está eternamente exenta de todas las vicisitudes transitorias, aunque nunca está muerta ni es formalista, sino siempre vibrante y adaptable ---radiantemente viva. Pero cuando la verdad se une a los hechos, entonces el tiempo y el espacio condicionan sus significados y correlacionan sus valores. Estas realidades de la verdad, enlazadas con los hechos, se vuelven conceptos y son relegadas en consecuencia al ámbito de las realidades cósmicas relativas.

\par
%\textsuperscript{(1297.5)}
\textsuperscript{118:3.4} La unión de la verdad absoluta y eterna del Creador con la experiencia objetiva de la criatura finita y temporal produce un nuevo valor emergente del Supremo. El concepto del Supremo es esencial para coordinar el mundo superior divino e invariable con el mundo inferior finito y en constante cambio.

\par
%\textsuperscript{(1297.6)}
\textsuperscript{118:3.5} De todas las cosas no absolutas, el espacio es el que está más cerca de ser absoluto. El espacio es en apariencia absolutamente último. La verdadera dificultad que tenemos para comprender el espacio en el nivel material se debe al hecho de que, aunque los cuerpos materiales existen en el espacio, el espacio también existe en esos mismos cuerpos materiales. Aunque hay muchas cosas relacionadas con el espacio que son absolutas, eso no quiere decir que el espacio sea absoluto.

\par
%\textsuperscript{(1297.7)}
\textsuperscript{118:3.6} Para comprender las relaciones espaciales, puede ser útil suponer que, hablando en términos relativos, el espacio es, después de todo, una propiedad de todos los cuerpos materiales. Por eso cuando un cuerpo se mueve por el espacio, también lleva consigo todas sus propiedades, incluido el espacio que está dentro de ese cuerpo en movimiento y forma parte de él.

\par
%\textsuperscript{(1297.8)}
\textsuperscript{118:3.7} Todas las formas de la realidad ocupan espacio en los niveles materiales, pero las formas espirituales sólo existen en relación con el espacio; no ocupan ni desplazan espacio, y tampoco lo contienen. Pero para nosotros, el enigma principal del espacio está relacionado con la forma de una idea. Cuando penetramos en el ámbito de la mente, nos encontramos con muchos rompecabezas. La forma ---la realidad--- de una idea, ¿ocupa espacio? En realidad no lo sabemos, aunque estamos seguros de que la forma de una idea no contiene espacio. Pero no sería muy prudente dar por sentado que lo inmaterial es siempre no espacial.

\section*{4. La causalidad primaria y secundaria}
\par
%\textsuperscript{(1298.1)}
\textsuperscript{118:4.1} Muchas dificultades teológicas y dilemas metafísicos del hombre mortal se deben al hecho de que el hombre no sitúa bien la personalidad de la Deidad y, en consecuencia, asigna atributos infinitos y absolutos a la Divinidad subordinada y a la Deidad evolutiva. No debéis olvidar que, aunque existe realmente una verdadera Causa Primera, hay también una multitud de causas coordinadas y subordinadas, unas causas tanto asociadas como secundarias.

\par
%\textsuperscript{(1298.2)}
\textsuperscript{118:4.2} La distinción vital entre las causas primeras y las causas segundas reside en el hecho de que las causas primeras producen unos efectos originales que están libres de la herencia de cualquier factor derivado de toda causalidad anterior. Las causas secundarias producen unos efectos que muestran invariablemente la herencia de otra causalidad precedente.

\par
%\textsuperscript{(1298.3)}
\textsuperscript{118:4.3} Los potenciales puramente estáticos inherentes al Absoluto Incalificado reaccionan a aquellas causalidades del Absoluto de la Deidad que son producidas por las acciones de la Trinidad del Paraíso. En presencia del Absoluto Universal, estos potenciales estáticos, impregnados de causalidad, se vuelven inmediatamente activos y sensibles a la influencia de ciertos agentes trascendentales cuyas acciones dan como resultado la transmutación de estos potenciales activados al estado de verdaderas posibilidades universales para el desarrollo, de unas capacidades efectivas para el crecimiento. Y sobre estos potenciales maduros, los creadores y los controladores del gran universo representan el drama interminable de la evolución cósmica.

\par
%\textsuperscript{(1298.4)}
\textsuperscript{118:4.4} Sin tener en cuenta a los existenciales, la causalidad tiene una constitución básicamente triple. Tal como funciona en esta era del universo y en lo que se refiere al nivel finito de los siete superuniversos, se la puede concebir como sigue:

\par
%\textsuperscript{(1298.5)}
\textsuperscript{118:4.5} 1. \textit{La activación de los potenciales estáticos}. Es el establecimiento del destino en el Absoluto Universal mediante las acciones del Absoluto de la Deidad, el cual funciona en el Absoluto Incalificado, y sobre él, como consecuencia de los mandatos volitivos de la Trinidad del Paraíso.

\par
%\textsuperscript{(1298.6)}
\textsuperscript{118:4.6} 2. \textit{La existenciación de las capacidades universales}. Esto implica la transformación de los potenciales no diferenciados en unos planes separados y definidos. Es el acto de la Ultimidad de la Deidad y de los múltiples agentes del nivel trascendental. Estos actos se anticipan perfectamente a las necesidades futuras de todo el universo maestro. En conexión con la separación de los potenciales, los Arquitectos del Universo Maestro existen como verdaderas personificaciones del concepto que se tiene de la Deidad en los universos. Sus planes parecen estar, de manera última, espacialmente limitados en extensión por la periferia conceptual del universo maestro, pero, \textit{como planes}, no están condicionados de otra manera por el tiempo o el espacio.

\par
%\textsuperscript{(1298.7)}
\textsuperscript{118:4.7} 3. \textit{La creación y la evolución de las manifestaciones universales}. Los Creadores Supremos actúan sobre un cosmos impregnado por la presencia productora de capacidad de la Ultimidad de la Deidad, para llevar a cabo las transmutaciones temporales de los potenciales maduros en manifestaciones experienciales. Dentro del universo maestro, toda manifestación de la realidad potencial está limitada por la capacidad última para el desarrollo, y está condicionada espacio-temporalmente en las etapas finales de su emergencia. Los Hijos Creadores que salen del Paraíso son, en realidad, creadores \textit{transformadores} en el sentido cósmico. Pero esto no invalida de ninguna manera el concepto que el hombre tiene de ellos como creadores; desde el punto de vista finito, por supuesto que pueden crear, y de hecho lo hacen.

\section*{5. La omnipotencia y la compatibilidad}
\par
%\textsuperscript{(1299.1)}
\textsuperscript{118:5.1} La omnipotencia de la Deidad no implica el poder de hacer lo que no es factible. Dentro del marco del espacio-tiempo, y desde el punto de referencia intelectual de la comprensión humana, incluso el Dios infinito no puede crear círculos cuadrados ni producir un mal que sea inherentemente bueno. Dios no puede hacer cosas no divinas. Esta contradicción de términos filosóficos equivale a una no entidad e implica que nada es creado así. Un rasgo de la personalidad no puede ser al mismo tiempo semejante a Dios y no semejante a Dios. La compatibilidad es innata en el poder divino. Y todo esto se deriva del hecho de que la omnipotencia no sólo crea cosas con una naturaleza, sino que también da origen a la naturaleza de todas las cosas y de todos los seres.

\par
%\textsuperscript{(1299.2)}
\textsuperscript{118:5.2} Al principio, el Padre lo hace todo, pero a medida que se despliega el panorama de la eternidad en respuesta a la voluntad y a los mandatos del Infinito, se hace cada vez más evidente que las criaturas, e incluso los hombres, han de convertirse en los asociados de Dios para llevar a cabo la finalidad del destino. Y esto es cierto incluso en la vida en la carne; cuando el hombre y Dios forman una asociación, no se puede poner ninguna limitación a las posibilidades futuras de esa asociación. Cuando el hombre se da cuenta de que el Padre Universal es su asociado en la progresión eterna, cuando fusiona con la presencia interior del Padre, ha roto en espíritu las cadenas del tiempo y ya ha entrado en las progresiones de la eternidad en busca del Padre Universal.

\par
%\textsuperscript{(1299.3)}
\textsuperscript{118:5.3} La conciencia humana pasa de los hechos a los significados, y luego a los valores. La conciencia del Creador parte del valor que aparece en el pensamiento, pasa por el significado que se manifiesta en la palabra, y llega al hecho de la acción. Dios siempre tiene que actuar para romper el punto muerto de la unidad incalificada inherente a la infinidad existencial. La Deidad tiene siempre que proporcionar el universo modelo, las personalidades perfectas, la verdad, la belleza y la bondad originales que todas las creaciones subdivinas se esfuerzan por conseguir. Dios debe siempre encontrar primero al hombre, para que el hombre pueda más tarde encontrar a Dios. Siempre debe haber un Padre Universal antes de que pueda existir una filiación universal y la fraternidad universal resultante\footnote{\textit{Dios nos ama primero}: 1 Jn 4:10,19.}.

\section*{6. La omnipotencia y la omnifaciencia}
\par
%\textsuperscript{(1299.4)}
\textsuperscript{118:6.1} Dios es realmente omnipotente, pero no es omnifaciente, ---no hace personalmente todo lo que se hace. La omnipotencia abarca el potencial de poder del Todopoderoso Supremo y del Ser Supremo, pero los actos volitivos de Dios Supremo no son las acciones personales del Dios Infinito.

\par
%\textsuperscript{(1299.5)}
\textsuperscript{118:6.2} Defender la omnifaciencia de la Deidad primordial equivaldría a quitarle sus derechos a casi un millón de Hijos Creadores Paradisiacos, sin mencionar las innumerables huestes de otras diversas órdenes de ayudantes creativos simultáneos. Sólo hay una Causa sin causa en todo el universo. Todas las demás causas se derivan de esta única Gran Fuente-Centro Primera. Y nada en esta filosofía va en contra del libre albedrío de los innumerables hijos de la Deidad diseminados por un inmenso universo.

\par
%\textsuperscript{(1299.6)}
\textsuperscript{118:6.3} Dentro de un marco local, la volición puede parecer que funciona como una causa sin causa, pero manifiesta infaliblemente unos factores hereditarios que establecen su relación con las Primeras Causas únicas, originales y absolutas.

\par
%\textsuperscript{(1299.7)}
\textsuperscript{118:6.4} Toda volición es relativa. En el sentido original, sólo el Padre-YO SOY posee la finalidad de la volición; en el sentido absoluto, sólo el Padre, el Hijo y el Espíritu muestran las prerrogativas de una volición no condicionada por el tiempo ni limitada por el espacio. El hombre mortal está dotado de libre albedrío, del poder de elegir, y aunque esta elección no sea absoluta, sin embargo es relativamente final en el nivel finito y en lo que concierne al destino de la personalidad que elige.

\par
%\textsuperscript{(1300.1)}
\textsuperscript{118:6.5} La volición en cualquier nivel, excepto en el absoluto, encuentra unas limitaciones que forman parte constituyente de la personalidad misma que ejerce el poder de elección. El hombre no puede elegir más allá de la gama de lo que es elegible. Por ejemplo, no puede escoger ser otra cosa que un ser humano, salvo que puede elegir llegar a ser más que un hombre; puede escoger embarcarse en el viaje de la ascensión del universo, pero esto se debe a que se da la circunstancia de que la elección humana y la voluntad divina coinciden en este punto. Y aquello que un hijo desea y el Padre quiere, sucederá con toda seguridad.

\par
%\textsuperscript{(1300.2)}
\textsuperscript{118:6.6} En la vida humana se abren y se cierran continuamente líneas de conducta diferenciales, y durante el tiempo en que la elección es posible, la personalidad humana decide constantemente entre esas numerosas líneas de acción. La volición temporal está ligada al tiempo, y debe esperar el paso del tiempo para encontrar la oportunidad de expresarse. La volición espiritual ha empezado a saborear la liberación de las cadenas del tiempo, pues ha logrado evadirse parcialmente de la secuencia del tiempo, y esto se debe a que la volición espiritual se va identificando con la voluntad de Dios.

\par
%\textsuperscript{(1300.3)}
\textsuperscript{118:6.7} La volición, el acto de escoger, ha de ejercerse dentro del marco universal que se ha hecho realidad en respuesta a una elección anterior y más elevada. Todo el campo de la voluntad humana está estrictamente limitado a lo finito, salvo en un detalle particular: cuando el hombre escoge encontrar a Dios y parecerse a él, esta elección es superfinita; sólo la eternidad podrá revelar si esta elección es también superabsonita.

\par
%\textsuperscript{(1300.4)}
\textsuperscript{118:6.8} Reconocer la omnipotencia de la Deidad es gozar de la seguridad en vuestra experiencia de la ciudadanía cósmica, es poseer la certeza de la seguridad en el largo viaje hacia el Paraíso. Pero aceptar la falacia de la omnifaciencia es abrazar el error colosal del panteísmo.

\section*{7. La omnisciencia y la predestinación}
\par
%\textsuperscript{(1300.5)}
\textsuperscript{118:7.1} En el gran universo, la función de la voluntad del Creador y de la voluntad de la criatura se ejerce dentro de los límites establecidos por los Arquitectos Maestros, y de acuerdo con las posibilidades determinadas por ellos. Sin embargo, la predeterminación de estos límites máximos no reduce en lo más mínimo la soberanía de la voluntad de la criatura dentro de esas fronteras. El preconocimiento último ---la plena tolerancia hacia todas las elecciones finitas--- tampoco constituye una abrogación de la volición finita. Un ser humano maduro y perspicaz podría ser capaz de prever con mucha exactitud la decisión de un asociado más joven, pero este preconocimiento no le quita ninguna libertad ni autenticidad a la decisión misma. Los Dioses han limitado sabiamente el campo de acción de la voluntad inmadura, pero sin embargo, dentro de esos límites definidos, es una verdadera voluntad.

\par
%\textsuperscript{(1300.6)}
\textsuperscript{118:7.2} Incluso la correlación suprema de todas las elecciones pasadas, presentes y futuras no invalida la autenticidad de dichas elecciones. Indica más bien la tendencia predeterminada del cosmos, y sugiere el preconocimiento de aquellos seres volitivos que pueden o no escoger convertirse en partes contribuyentes de la manifestación experiencial de toda la realidad.

\par
%\textsuperscript{(1300.7)}
\textsuperscript{118:7.3} El error en la elección finita está ligado al tiempo y limitado por éste. Sólo puede existir en el tiempo y \textit{dentro} de la presencia evolutiva del Ser Supremo. Esta elección errónea es posible en el tiempo e indica (además del estado incompleto del Supremo) esa cierta gama de elección con la que deben estar dotadas las criaturas inmaduras a fin de disfrutar de la progresión en el universo poniéndose voluntariamente en contacto con la realidad.

\par
%\textsuperscript{(1301.1)}
\textsuperscript{118:7.4} El pecado, en el espacio condicionado por el tiempo, prueba claramente la libertad temporal ---e incluso el libertinaje--- de la voluntad finita. El pecado representa la inmadurez deslumbrada por la libertad de la voluntad relativamente soberana de la personalidad, que al mismo tiempo no logra percibir las obligaciones y los deberes supremos de la ciudadanía cósmica.

\par
%\textsuperscript{(1301.2)}
\textsuperscript{118:7.5} La iniquidad, en los dominios finitos, revela la realidad transitoria de toda individualidad no identificada con Dios. Una criatura sólo se vuelve verdaderamente real en los universos a medida que se identifica con Dios. La personalidad finita no se crea a sí misma, pero en el campo superuniversal de la elección, ella misma determina su destino.

\par
%\textsuperscript{(1301.3)}
\textsuperscript{118:7.6} La concesión de la vida hace que los sistemas energético-materiales sean capaces de perpetuarse, de propagarse y de adaptarse. La concesión de la personalidad confiere a los organismos vivientes las prerrogativas adicionales de la autodeterminación, la evolución y la identificación de sí mismos con un espíritu de la Deidad capaz de fusionar con ellos.

\par
%\textsuperscript{(1301.4)}
\textsuperscript{118:7.7} Los seres vivos subpersonales indican que una mente activa la energía-materia, primero bajo la forma de controladores físicos y luego como espíritus ayudantes de la mente. El don de la personalidad procede del Padre y confiere al sistema viviente unas prerrogativas únicas de elección. Pero si la personalidad tiene la prerrogativa de ejercer la elección volitiva de identificarse con la realidad, y si esta elección es sincera y libre, entonces la personalidad evolutiva ha de tener también la posible elección de confundirse, de trastornarse y de destruirse. La posibilidad de destruirse cósmicamente no se puede evitar si la personalidad en evolución ha de ser verdaderamente libre en el ejercicio de su voluntad finita.

\par
%\textsuperscript{(1301.5)}
\textsuperscript{118:7.8} Por eso existe una seguridad creciente cuando se reducen los límites de la elección de la personalidad en todos los niveles inferiores de existencia. La elección se libera cada vez más a medida que se asciende en los universos; la elección se acerca finalmente a la libertad divina cuando la personalidad ascendente alcanza el estado de divinidad, la supremacía de la consagración a los objetivos del universo, la consecución total de la sabiduría cósmica, y la identificación final de la criatura con la voluntad y el camino de Dios.

\section*{8. El control y el supercontrol}
\par
%\textsuperscript{(1301.6)}
\textsuperscript{118:8.1} En las creaciones del espacio-tiempo, el libre albedrío está rodeado de restricciones, de limitaciones. La evolución de la vida material es al principio maquinal, luego es activada por la mente y (después de la concesión de la personalidad) puede dejarse dirigir por el espíritu. En los mundos habitados, los potenciales de las implantaciones originales de vida física de los Portadores de Vida limitan físicamente la evolución orgánica.

\par
%\textsuperscript{(1301.7)}
\textsuperscript{118:8.2} El hombre mortal es una máquina, un mecanismo viviente; sus raíces se encuentran realmente en el mundo físico de la energía. Muchas reacciones humanas son de naturaleza maquinal; una gran parte de la vida se parece a una máquina. Pero el hombre, un mecanismo, es mucho más que una máquina; está dotado de una mente y habitado por un espíritu; y aunque durante toda su vida material no pueda librarse nunca del mecanismo electroquímico de su existencia, puede aprender a subordinar cada vez más esta máquina de vida física a la sabiduría directriz de la experiencia, mediante el proceso de consagrar la mente humana a ejecutar los impulsos espirituales del Ajustador del Pensamiento interior.

\par
%\textsuperscript{(1301.8)}
\textsuperscript{118:8.3} El espíritu libera el funcionamiento de la voluntad, y el mecanismo lo limita. La elección imperfecta, no controlada por el mecanismo ni identificada con el espíritu, es peligrosa e inestable. El predominio mecánico asegura la estabilidad a expensas del progreso; la alianza con el espíritu libera a la elección del nivel físico y asegura al mismo tiempo la estabilidad divina producida por una perspicacia universal acrecentada y una mayor comprensión cósmica.

\par
%\textsuperscript{(1302.1)}
\textsuperscript{118:8.4} El gran peligro que acecha a la criatura, cuando consigue liberarse de las cadenas del mecanismo de la vida, es que no logre compensar esta pérdida de estabilidad llevando a cabo una armoniosa unión de trabajo con el espíritu. Cuando la elección de la criatura se libera relativamente de la estabilidad maquinal, puede intentar liberarse aún más con independencia de una mayor identificación con el espíritu.

\par
%\textsuperscript{(1302.2)}
\textsuperscript{118:8.5} Todo el principio de la evolución biológica hace imposible que el hombre primitivo aparezca en los mundos habitados provisto de un gran dominio de sí mismo. Por esta razón, el mismo diseño creativo que planeó la evolución provee igualmente aquellas restricciones externas de tiempo y de espacio, de hambre y de miedo, que circunscriben eficazmente el campo de las elecciones subespirituales de estas criaturas poco cultas. A medida que la mente del hombre sobrepasa con éxito unas barreras cada vez más difíciles, este mismo diseño creativo también ha previsto la lenta acumulación de la herencia racial de una sabiduría experiencial penosamente adquirida ---en otras palabras, el mantenimiento de un equilibrio entre las restricciones externas que disminuyen y las restricciones internas que aumentan.

\par
%\textsuperscript{(1302.3)}
\textsuperscript{118:8.6} La lentitud de la evolución, del progreso cultural humano, demuestra la eficacia de ese freno ---la inercia material--- que actúa con tanta eficiencia para retrasar las velocidades peligrosas del progreso. El tiempo mismo amortigua y distribuye así los resultados, por otra parte mortales, del hecho de librarse prematuramente de las barreras que rodean de cerca la actividad humana. Pues cuando la cultura avanza demasiado deprisa, cuando los logros materiales van más rápidos que la evolución de la sabiduría y la adoración, la civilización contiene en sí misma las semillas del retroceso; y a menos que esa civilización sea reforzada por un rápido aumento de la sabiduría experiencial, esas sociedades humanas descenderán desde los niveles elevados, pero prematuros, que han alcanzado, y las <<edades de las tinieblas>> del interregno de la sabiduría presenciarán el restablecimiento inexorable del desequilibrio entre la libertad del yo y el dominio de sí mismo.

\par
%\textsuperscript{(1302.4)}
\textsuperscript{118:8.7} La iniquidad de Caligastia consistió en desviar el regulador temporal de la liberación humana progresiva ---la destrucción gratuita de las barreras restrictivas, unas barreras que las mentes de los mortales de aquellos tiempos no habían sobrepasado por experiencia.

\par
%\textsuperscript{(1302.5)}
\textsuperscript{118:8.8} La mente que puede llevar a cabo una reducción parcial del tiempo y del espacio prueba, mediante este acto mismo, que posee en sí misma las semillas de sabiduría que pueden servir eficazmente en lugar de la barrera restrictiva que ha trascendido.

\par
%\textsuperscript{(1302.6)}
\textsuperscript{118:8.9} Lucifer intentó destruir del mismo modo el regulador temporal que frenaba la obtención prematura de ciertas libertades en el sistema local. Un sistema local asentado en la luz y la vida ha conseguido experiencialmente los puntos de vista y la perspicacia que hacen posible el funcionamiento de numerosas técnicas que serían perjudiciales y destructivas durante las eras anteriores al asentamiento de ese mismo reino.

\par
%\textsuperscript{(1302.7)}
\textsuperscript{118:8.10} A medida que el hombre se deshace de las trabas del miedo, a medida que recorre los continentes y los océanos con sus máquinas, y las generaciones y los siglos con sus escritos, debe sustituir cada restricción trascendida por una restricción nueva voluntariamente asumida de acuerdo con los dictados morales de la sabiduría humana en expansión. Estas restricciones autoimpuestas son a la vez los más poderosos y los más sutiles de todos los factores de la civilización humana ---los conceptos de la justicia y los ideales de la fraternidad. El hombre se capacita incluso para llevar las vestimentas restrictivas de la misericordia cuando se atreve a amar a sus semejantes, mientras que alcanza los principios de la fraternidad espiritual cuando escoge tratarlos como le gustaría ser tratado, e incluso tratarlos como imagina que Dios los trataría.

\par
%\textsuperscript{(1303.1)}
\textsuperscript{118:8.11} Una reacción universal automática es estable y, de alguna forma, tiene una continuidad en el cosmos. Una personalidad que conoce a Dios y que desea hacer su voluntad, que tiene perspicacia espiritual, es divinamente estable y existe eternamente. La gran aventura universal del hombre consiste en la transición de su mente mortal entre la estabilidad de la estática mecánica y la divinidad de la dinámica espiritual, y esta transformación la consigue mediante la fuerza y la constancia de las decisiones de su propia personalidad, declarando en cada situación de la vida: <<Es mi voluntad que se haga tu voluntad>>\footnote{\textit{Es mi voluntad que se haga tu voluntad}: Sal 143:10; Eclo 15:11-20; Mt 6:10; 7:21; 12:50; 26:39,42,44; Mc 3:35; 14:36,39; Lc 8:21; 11:2; 22:42; Jn 4:34; 5:30; 6:38-40; 7:16-17; 9:31; 14:21-24; 15:10,14; 17:4.}.

\section*{9. Los mecanismos del universo}
\par
%\textsuperscript{(1303.2)}
\textsuperscript{118:9.1} El tiempo y el espacio son un mecanismo conjunto del universo maestro. Son los dispositivos que permiten a las criaturas finitas coexistir con el Infinito en el cosmos. Las criaturas finitas están eficazmente aisladas de los niveles absolutos por el tiempo y el espacio. Pero estos medios de aislamiento, sin los cuales ningún mortal podría existir, funcionan directamente para limitar el campo de la acción finita. Sin ellos ninguna criatura podría actuar, pero a causa de ellos, los actos de cada criatura están claramente limitados.

\par
%\textsuperscript{(1303.3)}
\textsuperscript{118:9.2} Los mecanismos creados por las mentes superiores funcionan para liberar sus fuentes creativas pero, hasta cierto punto, limitan invariablemente la acción de todas las inteligencias subordinadas. Para las criaturas de los universos, esta limitación se hace evidente bajo la forma del mecanismo de los universos. El hombre no posee un libre albedrío sin trabas; el alcance de su elección tiene unos límites, pero dentro del radio de esta elección, su voluntad es relativamente soberana.

\par
%\textsuperscript{(1303.4)}
\textsuperscript{118:9.3} El mecanismo vital de la personalidad mortal, el cuerpo humano, es el producto de un diseño creativo supermortal; por eso nunca puede ser perfectamente controlado por el hombre mismo. Sólo cuando el hombre ascendente, en unión con el Ajustador fusionado, cree por sí mismo el mecanismo destinado a expresar su personalidad, conseguirá controlarlo a la perfección.

\par
%\textsuperscript{(1303.5)}
\textsuperscript{118:9.4} El gran universo es un mecanismo así como un organismo, mecánico y viviente ---un mecanismo viviente activado por una Mente Suprema, que se coordina con un Espíritu Supremo, y que encuentra su expresión en los máximos niveles de unificación del poder con la personalidad bajo la forma de Ser Supremo. Pero negar el mecanismo de la creación finita es negar un hecho y no hacer caso de la realidad.

\par
%\textsuperscript{(1303.6)}
\textsuperscript{118:9.5} Los mecanismos son el producto de una mente, de una mente creativa que actúa sobre los potenciales cósmicos y en ellos. Los mecanismos son las cristalizaciones fijas del pensamiento del Creador, y siempre funcionan de conformidad con el concepto volitivo que les dio origen. Pero la finalidad de cualquier mecanismo se encuentra en su origen, no en su función.

\par
%\textsuperscript{(1303.7)}
\textsuperscript{118:9.6} No se debería pensar que estos mecanismos limitan la acción de la Deidad; la verdad es más bien que mediante estos mismos mecanismos la Deidad ha llevado a cabo una fase de expresión eterna. Los mecanismos básicos del universo han surgido a la existencia en respuesta a la voluntad absoluta de la Fuente-Centro Primera y, en consecuencia, funcionarán de manera eterna en perfecta armonía con el plan del Infinito; son en verdad los arquetipos no volitivos de este mismo plan.

\par
%\textsuperscript{(1303.8)}
\textsuperscript{118:9.7} Comprendemos un poco la manera en que el mecanismo del Paraíso está correlacionado con la personalidad del Hijo Eterno; ésta es la función del Actor Conjunto. Y tenemos teorías sobre las operaciones del Absoluto Universal con respecto a los mecanismos teóricos del Incalificado y a la persona potencial del Absoluto de la Deidad. Pero observamos que, en las Deidades evolutivas del Supremo y del Último, ciertas fases impersonales se están uniendo realmente con sus contrapartidas volitivas, y se está desarrollando así una nueva relación entre el arquetipo y la persona.

\par
%\textsuperscript{(1304.1)}
\textsuperscript{118:9.8} En la eternidad del pasado, el Padre y el Hijo encontraron su unión en la unidad de expresión del Espíritu Infinito. Si en la eternidad del futuro los Hijos Creadores y los Espíritus Creativos de los universos locales del tiempo y del espacio alcanzan una unión creativa en los reinos del espacio exterior, ¿qué es lo que crearía esta unidad como expresión combinada de sus naturalezas divinas? Puede ser muy bien que presenciemos una manifestación hasta ahora no revelada de la Deidad Última, un superadministrador de un nuevo tipo. Estos seres poseerían unas prerrogativas de personalidad excepcionales, ya que serían la unión del Creador personal, del Espíritu Creativo impersonal, de la experiencia como criatura mortal y de la personalización progresiva de la Ministra Divina. Estos seres podrían ser últimos, en el sentido de que englobarían la realidad personal e impersonal, mientras que combinarían las experiencias del Creador y de la criatura. Cualesquiera que sean los atributos de estas terceras personas que formarán parte de estas supuestas trinidades funcionales de las creaciones del espacio exterior, mantendrán con sus Padres Creadores y sus Madres Creativas una relación un poco semejante a la que el Espíritu Infinito mantiene con el Padre Universal y el Hijo Eterno.

\par
%\textsuperscript{(1304.2)}
\textsuperscript{118:9.9} Dios Supremo es la personalización de toda la experiencia universal, la focalización de toda la evolución finita, el punto máximo de toda la realidad de las criaturas, la consumación de la sabiduría cósmica, la personificación de la belleza armoniosa de las galaxias del tiempo, la verdad de los significados de la mente cósmica y la bondad de los valores espirituales supremos. Y Dios Supremo sintetizará, en el eterno futuro, estas múltiples diversidades finitas en un conjunto experiencialmente significativo, tal como se encuentran ahora existencialmente unidas en los niveles absolutos en la Trinidad del Paraíso.

\section*{10. Las funciones de la Providencia}
\par
%\textsuperscript{(1304.3)}
\textsuperscript{118:10.1} La providencia no significa que Dios ha decidido todas las cosas para nosotros y por adelantado. Dios nos ama demasiado como para hacer esto, pues esto no sería más que una tiranía cósmica. El hombre posee unos poderes de elección relativos. Y el amor divino tampoco es ese afecto miope que mimaría y consentiría a los hijos de los hombres.

\par
%\textsuperscript{(1304.4)}
\textsuperscript{118:10.2} El Padre, el Hijo y el Espíritu ---como Trinidad\footnote{\textit{La Trinidad}: Mt 28:19; Hch 2:32-33; 2 Co 13:14; 1 Jn 5:7.}--- no son el Todopoderoso Supremo, pero la supremacía del Todopoderoso nunca puede manifestarse sin ellos. El \textit{crecimiento} del Todopoderoso está centrado en los Absolutos de manifestación y basado en los Absolutos de potencialidad. Pero las \textit{funciones} del Todopoderoso Supremo están relacionadas con las funciones de la Trinidad del Paraíso\footnote{\textit{La Trinidad (Primitiva visión de Pablo)}: 1 Co 12:4-6.}.

\par
%\textsuperscript{(1304.5)}
\textsuperscript{118:10.3} Parece ser que la personalidad de esta Deidad experiencial está reuniendo parcialmente todas las fases de la actividad universal en el Ser Supremo. Por consiguiente, cuando deseamos ver a la Trinidad como un solo Dios, y si limitamos este concepto al gran universo actual conocido y organizado, descubrimos que el Ser Supremo en evolución es la descripción parcial de la Trinidad del Paraíso. Y comprobamos además que esta Deidad Suprema está evolucionando, en forma de personalidad, como la síntesis de la materia, la mente y el espíritu finitos en el gran universo.

\par
%\textsuperscript{(1304.6)}
\textsuperscript{118:10.4} Los Dioses tienen atributos, pero la Trinidad tiene funciones y, al igual que la Trinidad, la providencia \textit{es} una función, el compuesto del supercontrol, distinto al personal, del universo de universos, que se extiende desde los niveles evolutivos del Séptuple, los cuales se sintetizan en el poder del Todopoderoso, y se eleva a través de los reinos trascendentales de la Ultimidad de la Deidad.

\par
%\textsuperscript{(1304.7)}
\textsuperscript{118:10.5} Dios ama a cada criatura como a un hijo\footnote{\textit{Dios ama a cada criatura como a un hijo}: Jn 3:16; 15:9-13; 17:22-23; Ro 5:8; Tit 3:4; 1 Jn 4:9-11,19.}, y este amor cubre con su sombra a cada criatura a través de todos los tiempos y de toda la eternidad. La providencia funciona con respecto a la totalidad y se ocupa de la función de cualquier criatura en la medida en que esa función está relacionada con la totalidad. La intervención providencial con respecto a un ser determinado indica la importancia de la \textit{función} de ese ser en lo que concierne al crecimiento evolutivo de alguna totalidad; dicha totalidad puede ser la raza total, la nación total, el planeta total o incluso un total superior. La importancia de la función de la criatura es la que provoca la intervención providencial, y no la importancia de la criatura como persona.

\par
%\textsuperscript{(1305.1)}
\textsuperscript{118:10.6} Sin embargo el Padre, como persona, puede interponer en cualquier momento una mano paternal en la corriente de los acontecimientos cósmicos de acuerdo totalmente con la voluntad de Dios, en consonancia con la sabiduría de Dios, y motivada por el amor de Dios.

\par
%\textsuperscript{(1305.2)}
\textsuperscript{118:10.7} Pero lo que el hombre llama providencia es con demasiada frecuencia el producto de su propia imaginación, la yuxtaposición fortuita de las circunstancias del azar. Existe, sin embargo, una providencia real y emergente en el reino finito de la existencia universal, una verdadera correlación, en vías de manifestarse, de las energías del espacio, los movimientos del tiempo, los pensamientos del intelecto, los ideales del carácter, los deseos de las naturalezas espirituales y los actos volitivos deliberados de las personalidades evolutivas. Las circunstancias de las creaciones materiales encuentran su integración finita final en las presencias entrelazadas del Supremo y del Último.

\par
%\textsuperscript{(1305.3)}
\textsuperscript{118:10.8} La providencia se vuelve cada vez más discernible a medida que los mecanismos del gran universo se perfeccionan hasta un punto de precisión final mediante el supercontrol de la mente, a medida que la mente de las criaturas se eleva a la perfección de haber alcanzado la divinidad mediante una integración perfeccionada con el espíritu y, por consiguiente, a medida que el Supremo emerge como unificador \textit{efectivo} de todos estos fenómenos del universo.

\par
%\textsuperscript{(1305.4)}
\textsuperscript{118:10.9} Algunas condiciones asombrosamente fortuitas que prevalecen ocasionalmente en los mundos evolutivos pueden deberse a la presencia gradualmente emergente del Supremo, a la anticipación de sus actividades universales futuras. La mayor parte de las cosas que un mortal llamaría providenciales, no lo son; su juicio en estos asuntos está muy obstaculizado por la falta de una visión perspicaz de los verdaderos significados de las circunstancias de la vida. Muchas cosas que un mortal llamaría buena suerte, pueden ser en realidad mala suerte; la sonrisa de la fortuna, que proporciona un tiempo libre no ganado y una riqueza inmerecida, puede ser la mayor de las aflicciones humanas; la crueldad aparente de un destino perverso que acumula tribulaciones sobre un mortal sufriente, puede ser en realidad el fuego templador que está transmutando el hierro dulce de la personalidad inmadura en el acero templado de un verdadero carácter.

\par
%\textsuperscript{(1305.5)}
\textsuperscript{118:10.10} Existe una providencia en los universos evolutivos, y las criaturas pueden descubrirla en la medida exacta en que han alcanzado la capacidad de percibir la finalidad de los universos en evolución. La capacidad total para discernir los objetivos del universo equivale a la culminación evolutiva de la criatura, y se puede expresar de otra manera diciendo que ha alcanzado al Supremo dentro de los límites del estado actual de los universos incompletos.

\par
%\textsuperscript{(1305.6)}
\textsuperscript{118:10.11} El amor del Padre actúa directamente en el corazón del individuo, independientemente de las acciones o reacciones de todos los demás individuos; la relación es personal ---el hombre y Dios. La presencia impersonal de la Deidad (el Todopoderoso Supremo y la Trinidad del Paraíso) manifiesta su consideración por el todo, no por la parte. La providencia del supercontrol de la Supremacía se vuelve cada vez más evidente a medida que las partes sucesivas del universo progresan en la conquista de sus destinos finitos. A medida que los sistemas, las constelaciones, los universos y los superuniversos se establecen en la luz y la vida, el Supremo emerge cada vez más como correlacionador significativo de todo lo que sucede, mientras que el Último emerge gradualmente como unificador trascendental de todas las cosas.

\par
%\textsuperscript{(1306.1)}
\textsuperscript{118:10.12} En los comienzos de un mundo evolutivo, los sucesos naturales de tipo material y los deseos personales de los seres humanos parecen ser con frecuencia antagónicos. Muchas cosas que suceden en un mundo en evolución son más bien difíciles de comprender para el hombre mortal ---la ley natural es muy a menudo aparentemente cruel, despiadada e indiferente hacia todo lo que es verdadero, bello y bueno para la comprensión humana. Pero a medida que la humanidad progresa en su desarrollo planetario, observamos que este punto de vista es modificado por los siguientes factores:

\par
%\textsuperscript{(1306.2)}
\textsuperscript{118:10.13} 1. \textit{La visión acrecentada del hombre} ---su comprensión creciente del mundo en el que vive; su capacidad más amplia para comprender los hechos materiales del tiempo, las ideas significativas del pensamiento, y los ideales valiosos de la perspicacia espiritual. Mientras los hombres se limiten a medir con la vara de las cosas de la naturaleza física, nunca pueden esperar encontrar la unidad en el tiempo y el espacio.

\par
%\textsuperscript{(1306.3)}
\textsuperscript{118:10.14} 2. \textit{El control creciente del hombre} ---la acumulación gradual del conocimiento de las leyes del mundo material, los objetivos de la existencia espiritual y las posibilidades de coordinar filosóficamente estas dos realidades. El hombre salvaje se encontraba desamparado ante los violentos ataques de las fuerzas naturales, era un esclavo del dominio cruel de sus propios miedos internos. El hombre semicivilizado empieza a abrir el almacén de los secretos de los reinos naturales, y su ciencia destruye de manera lenta pero eficaz sus supersticiones, mientras que al mismo tiempo le proporciona una base objetiva nueva y más amplia para comprender los significados de la filosofía y los valores de la verdadera experiencia espiritual. El hombre civilizado alcanzará algún día el dominio relativo de las fuerzas físicas de su planeta; el amor de Dios que reside en su corazón se derramará eficazmente bajo la forma de amor por sus semejantes, mientras que los valores de la existencia humana se acercarán a los límites de la capacidad mortal.

\par
%\textsuperscript{(1306.4)}
\textsuperscript{118:10.15} 3. \textit{La integración del hombre en el universo} ---el acrecentamiento de la perspicacia humana más el incremento de los logros experienciales humanos llevan al hombre hacia una armonía más estrecha con las presencias unificadoras de la Supremacía ---la Trinidad del Paraíso y el Ser Supremo. Y esto es lo que establece la soberanía del Supremo en los mundos asentados desde hace mucho tiempo en la luz y la vida. Estos planetas avanzados son en verdad unos poemas de armonía, unas imágenes de la belleza de la bondad alcanzada, conseguida a base de buscar la verdad cósmica. Si estas cosas pueden suceder en un planeta, entonces otras mucho más grandes pueden suceder en un sistema y en las unidades más amplias del gran universo, a medida que consigan también una estabilidad indicativa de que los potenciales para el crecimiento finito se han agotado.

\par
%\textsuperscript{(1306.5)}
\textsuperscript{118:10.16} En un planeta de este tipo avanzado, la providencia se ha vuelto una realidad, las circunstancias de la vida están correlacionadas, pero esto no sólo se debe a que el hombre ha llegado a dominar los problemas materiales de su mundo; se debe también a que ha empezado a vivir de acuerdo con la tendencia de los universos; sigue el camino de la Supremacía que le conduce a alcanzar al Padre Universal.

\par
%\textsuperscript{(1306.6)}
\textsuperscript{118:10.17} El reino de Dios está en el corazón de los hombres\footnote{\textit{El reino de Dios está en el corazón}: Lc 17:21.}; y cuando este reino se convierte en una realidad en el corazón de cada individuo de un mundo, entonces el reinado de Dios se ha vuelto real en ese planeta; y ésta es la soberanía conseguida del Ser Supremo.

\par
%\textsuperscript{(1306.7)}
\textsuperscript{118:10.18} Para hacer realidad la providencia en el tiempo, el hombre debe llevar a cabo la tarea de conseguir la perfección. Pero el hombre puede incluso ahora conocer de antemano esta providencia en sus significados eternos cuando reflexiona sobre el hecho universal de que todas las cosas, ya sean buenas o malas, trabajan unidas para el progreso de los mortales que conocen a Dios, en su búsqueda del Padre de todos\footnote{\textit{Todas las cosas trabajan para el bien}: Ro 8:28.}.

\par
%\textsuperscript{(1306.8)}
\textsuperscript{118:10.19} La providencia se discierne cada vez más a medida que los hombres se elevan de lo material a lo espiritual. Alcanzar una completa perspicacia espiritual permite a la personalidad ascendente detectar armonía donde hasta entonces sólo había caos. Incluso la mota morontial representa un progreso real en esta dirección.

\par
%\textsuperscript{(1307.1)}
\textsuperscript{118:10.20} La providencia es en parte el supercontrol del Supremo incompleto, manifestado en los universos incompletos, y por lo tanto siempre deberá ser:

\par
%\textsuperscript{(1307.2)}
\textsuperscript{118:10.21} 1. \textit{Parcial} ---debido a que la manifestación del Ser Supremo se encuentra en un estado incompleto, e

\par
%\textsuperscript{(1307.3)}
\textsuperscript{118:10.22} 2. \textit{Imprevisible} ---debido a las fluctuaciones de la actitud de las criaturas, que siempre varía de nivel en nivel, causando así una reacción recíproca aparentemente variable en el Supremo.

\par
%\textsuperscript{(1307.4)}
\textsuperscript{118:10.23} Cuando los hombres ruegan para que se produzca una intervención providencial en las circunstancias de la vida, muchas veces la respuesta a sus oraciones es su propio cambio de actitud hacia la vida. Pero la providencia no es caprichosa, y tampoco es fantástica ni mágica. Es la aparición lenta y segura del poderoso soberano de los universos finitos, cuya presencia majestuosa es detectada ocasionalmente por las criaturas evolutivas en su progreso universal. La providencia es la marcha cierta y segura de las galaxias del espacio y de las personalidades del tiempo hacia las metas de la eternidad, primero en el Supremo, luego en el Último, y quizás en el Absoluto. Creemos que esta misma providencia existe en la infinidad, y que se trata de la voluntad, las acciones y el propósito de la Trinidad del Paraíso, que motiva así el panorama cósmico de unos universos tras otros.

\par
%\textsuperscript{(1307.5)}
\textsuperscript{118:10.24} [Patrocinado por un Poderoso Mensajero que reside temporalmente en Urantia.]


\newpage
\pagestyle{empty}

\par {\huge Abreviaturas}
\bigbreak
\bigbreak
\begin{multicols}{2}
	\par LU \textit{(El Libro de Urantia)}
	\bigbreak
	\par Libros bíblicos:
	\bigbreak
	\par Abd \textit{(Abdías)}
	\par Am \textit{(Amós)}
	\par Ap \textit{(Apocalipsis)}
	\par Bar \textit{(Baruc)}
	\par Co \textit{(Epístola a los Corintios)}
	\par Cnt \textit{(El Cantar de los Cantares)}
	\par Col \textit{(Epístola a los Colosenses)}
	\par Cr \textit{(Crónicas)}
	\par Dn \textit{(Daniel)}
	\par Dt \textit{(Deuteronomio)}
	\par Ec \textit{(Eclesiastés)}
	\par Eclo \textit{(Ecclesiástico)}
	\par Ef \textit{(Epístola a los Efesios)}
	\par Esd \textit{(Esdras)}
	\par Est \textit{(Ester)}
	\par Ex \textit{(Éxodo)}
	\par Ez \textit{(Ezequiel)} 
	\par Flm \textit{(Epístola a Filemón)}
	\par Flp \textit{(Epístola a los Filipenses)}
	\par Gl \textit{(Epítosla a los Gálatas)}
	\par Gn \textit{(Génesis)}
	\par Hab \textit{(Habacuc)} 
	\par Hag \textit{(Ageo)}
	\par Hch \textit{(Hechos de los Apóstoles)}
	\par Heb \textit{(Epístola a los Hebreos)}
	\par Is \textit{(Isaías)}
	\par Jer \textit{(Jeremías)}
	\par Jl \textit{(Joel)}
	\par Jn \textit{(Juan, evangelio y epístolas)}
	\par Job \textit{(Job)}
	\par Jon \textit{(Jonás)}
	\par Jos \textit{(Josué)}
	\par Jud \textit{(Epístola de Judas)}
	\par Jue \textit{(Jueces)}
	\par Lc \textit{(Lucas)}
	\par Lm \textit{(Lamentaciones)}
	\par Lv \textit{(Levítico)}
	\par Mac \textit{(Macabeos)}
	\par Mal \textit{(Malaquías)}
	\par Mc \textit{(Marcos)}
	\par Miq \textit{(Miqueas)} 
	\par Mt \textit{(Mateo)}
	\par Nah \textit{(Nahúm)}
	\par Neh \textit{(Nehemías)} 
	\par Nm \textit{(Números)}
	\par Os \textit{(Oseas)}
	\par P \textit{(Epístola de Pedro)}
	\par Pr \textit{(Proverbios)}
	\par Re \textit{(Reyes)}
	\par Ro \textit{(Epístola a los Romanos)}
	\par Rt \textit{(Rut)}
	\par Sab \textit{(Sabiduría)}
	\par Sal \textit{(Salmos)}
	\par Sam \textit{(Samuel)}
	\par Sof \textit{(Sofonías)}
	\par Stg \textit{(Epístola a Santiago)}
	\par Ti \textit{(Epístola a Timoteo)}
	\par Tit \textit{(Epítosla a Tito)}
	\par Ts \textit{(Epístola a los Tesalonicenses)}
	\par Zac \textit{(Zacarías)}
	\bigbreak
	\par Libros bíblicos apócrifos:
	\bigbreak 
	\par AsMo \textit{(Asunción de Moisés)}
	\par Bel \textit{(Bel y el Dragón)} 
	\par Hen \textit{(Enoc)} 
	\par Man \textit{(Oración de Manasés)} 
	\par Tb \textit{(Tobit)}
	\bigbreak
	\par Libros de otras religiones: 
	\bigbreak
	\par XXX \textit{(YYYY)}
	
	
\end{multicols}

\end{document}

