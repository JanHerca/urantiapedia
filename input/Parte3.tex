% Author of this conversion to LaTeX format: Jan Herca, 2017
\documentclass[twoside, 11pt]{book}
\usepackage[T1]{fontenc} % indica al procesador cómo imprimir los caracteres
\usepackage{fontspec} % permite definir fuentes a partir de las instaladas en el SO
\usepackage{geometry}
\usepackage{graphicx}
\usepackage{float}
\usepackage{tocloft}
\usepackage{titleps}
\usepackage{emptypage}
\usepackage[spanish]{babel}
\usepackage{multicol}
% Text styles
\geometry{paperwidth=16cm, paperheight=24cm, top=2.5cm, bottom=1.7cm, inner=2.5cm, outer=1.2cm}

\makeatletter
\def\@makechapterhead#1{%
	\vspace*{50\p@}%
	{\parindent \z@ \raggedright \normalfont
		\interlinepenalty\@M
		\huge \bfseries #1\par\nobreak
		\vskip 40\p@
}}
\def\@makeschapterhead#1{%
	\vspace*{50\p@}%
	{\parindent \z@ \raggedright
		\normalfont
		\interlinepenalty\@M
		\huge \bfseries  #1\par\nobreak
		\vskip 40\p@
}}
\makeatother

\renewcommand{\cftchapleader}{\cftdotfill{\cftdotsep}}
\renewcommand{\thechapter}{}
\renewcommand{\cftchapfont}{\large}
\cftsetpnumwidth{3em}
\renewcommand{\cftchappagefont}{\large}


\title{La Quinta Revelación \newline Tercer Volumen \newline La historia de nuestra planeta, Urantia}
\date{}
\begin{document}
	
\begin{titlepage}
	\centering
	{\Huge\bfseries El Libro de Urantia\par}
	{\huge\bfseries La Quinta Revelación\par}
	\vspace{1cm}
	{\huge\bfseries Tercer Volumen\par}
	\vspace{1cm}
	{\huge\bfseries La historia de nuestra planeta, Urantia\par}
	\vfill
	{\scshape\Large URANTIA FOUNDATION\par}
	{\scshape\Large CHICAGO ILLINOIS\par}
	{\Large 2009 Traducción al español Europea\par}
\end{titlepage}
	
	
\par {\textcopyright} 2019 Jan Herca, de la edición
\par {\textcopyright} 2009 Urantia Foundation, de la traducción
\par {\textcopyright} 1993 Urantia Foundation, de otros materiales
\bigbreak
\par Jan Herca
\par Correo electrónico: janherca@gmail.com
\bigbreak
\par Urantia Foundation
\par 533 West Diversey Parkway
\par Chicago, IL 60614 EE.UU.A
\par Oficina: 1+(773) 525-3319
\par Fax: 1 +(773) 525-7739
\par Website: http://www.urantia.org
\par Correo electrónico: urantia@urantia.org
\bigbreak
\par Todos los derechos reservados, incluyendo el de traducción en los Estados Unidos de América, Canadá y en los demás países de la Unión Internacional de copyright. Todos los derechos reservados en los paises firmantes de la Union Panamericana de la Union internacional de copyright.
\par No todo el libro ni parte de él pueden ser copiados, reproducidos o traducidos en forma alguna, ya sea por medio electrónico, mecánico u otra forma, como fotocopia, grabación o archivo computerizado sin autorización por escrito del editor.
\par URANTIA,'' ``URANTIAN,'' ``EL LIBRO DE URANTIA'' y son marcas registradas de Urantia Foundation y su uso está sujeto a licencia.
\bigbreak
\par La Quinta Revelación es una reedición de El Libro de Urantia (Edición Europea). Está dividido en siete volúmenes para hacerlo más manejable y dispone de contenido adicional en forma de ayudas a la lectura integradas en el texto. El Libro de Urantia (Edición Europea) es una traducción de The Urantia Book realizada por la Fundación Urantia en 2009. 
\newpage

\begin{center}
	{\huge\bfseries Las partes del libro\par}
	\vspace{1cm}
	{\scshape\large PRIMER VOLUMEN\par}
	{\scshape\Large DIOS, EL UNIVERSO CENTRAL Y LOS SUPERUNIVERSOS\par}
	\vspace{1cm}
	
	{\scshape\large SEGUNDO VOLUMEN \par}
	{\scshape\Large EL UNIVERSO LOCAL\par}
	\vspace{1cm}
	
	{\scshape\large TERCER VOLUMEN \par}
	{\scshape\Large LA HISTORIA DE NUESTRO PLANETA, URANTIA\par}
	\vspace{1cm}
	
	{\scshape\large CUARTO VOLUMEN \par}
	{\scshape\Large LA EVOLUCIÓN DE LA CIVILIZACIÓN HUMANA\par}
	\vspace{1cm}
	
	{\scshape\large QUINTO VOLUMEN \par}
	{\scshape\Large LA RELIGIÓN, LA SOBREVIVENCIA A LA MUERTE Y LA DEIDAD EXPERIENCIAL\par}
	\vspace{1cm}
	
	{\scshape\large SEXTO VOLUMEN \par}
	{\scshape\Large LA VIDA Y LAS ENSEÑANZAS DE JESÚS - I\par}
	\vspace{1cm}
	
	{\scshape\large SÉPTIMO VOLUMEN \par}
	{\scshape\Large LA VIDA Y LAS ENSEÑANZAS DE JESÚS - II\par}
\end{center}
	
\newpage
\begin{center}
	{\small \textit {Intencionadamente en blanco}\par}
\end{center}
\newpage

\pagestyle{empty}


\tableofcontents

\newpagestyle{main}{
	%\setheadrule{1pt}% Header rule
	%\setfootrule{.4pt}% Footer rule
	\sethead[\small \thepage]% odd-left
	[]% odd-center
	[\begin{minipage}{0.9\textwidth}\begin{flushright}\scriptsize \MakeUppercase{\chaptertitle}\end{flushright}\end{minipage}]% odd-right
	{\begin{minipage}{0.9\textwidth}\scriptsize \MakeUppercase{\chaptertitle}\end{minipage}}% even-left
	{}% even-center
	{\small \thepage}% even-right
	\setfoot[]% odd-left
	[]% odd-center
	[]% odd-right
	{}% even-left
	{}% even-center
	{}% even-right
}

\pagestyle{main}
\renewcommand{\makeheadrule}{\rule[-.6\baselineskip]{\linewidth}{.4pt}}



\chapter{Documento 57. El origen de Urantia}
\par
%\textsuperscript{(651.1)}
\textsuperscript{57:0.1} AL PRESENTAR estos extractos de los archivos de Jerusem para los anales de Urantia, relacionados con sus antecedentes y su historia primitiva, nos han ordenado que calculemos el tiempo según el uso corriente ---el actual calendario bisiesto de 365{\textonequarter} días por año. Por regla general, no haremos ningún intento por indicar los años exactos, aunque estén registrados. Utilizaremos los números enteros más aproximados, pues es el mejor método para presentar estos hechos históricos.

\par
%\textsuperscript{(651.2)}
\textsuperscript{57:0.2} Cuando hagamos referencia a un acontecimiento que tuvo lugar hace uno o dos millones de años, tenemos la intención de remontarnos ese número de años hasta ese suceso, partiendo de las primeras décadas del siglo veinte de la era cristiana. Describiremos así esos acontecimientos lejanos como si hubieran ocurrido en períodos exactos de miles, millones o miles de millones de años.

\section*{1. La nebulosa de Andronover}
\par
%\textsuperscript{(651.3)}
\textsuperscript{57:1.1} Urantia tiene su origen en vuestro Sol, y vuestro Sol es uno de los múltiples frutos de la nebulosa de Andronover, que en otro tiempo fue organizada como parte componente del poder físico y de la sustancia material del universo local de Nebadon. Y esta misma gran nebulosa tuvo su origen en la carga de fuerza universal del espacio, en el superuniverso de Orvonton, hace muchísimo tiempo.

\par
%\textsuperscript{(651.4)}
\textsuperscript{57:1.2} En la época en que comienza esta narración, los Organizadores Maestros Primarios de Fuerza del Paraíso habían mantenido durante mucho tiempo el control completo de las energías espaciales que más tarde se organizarían bajo la forma de la nebulosa de Andronover.

\par
%\textsuperscript{(651.5)}
\textsuperscript{57:1.3} Hace \textit{987.000.000.000} de años, el organizador de fuerza asociado, en aquel entonces inspector en funciones número 811.307 de la serie de Orvonton, que viajaba fuera de Uversa, informó a los Ancianos de los Días que las condiciones espaciales eran favorables para iniciar los fenómenos de materialización en cierto sector del segmento, entonces oriental, de Orvonton.

\par
%\textsuperscript{(651.6)}
\textsuperscript{57:1.4} Hace \textit{900.000.000.000} de años, los archivos de Uversa revelan que se registró un permiso emitido por el Consejo del Equilibrio de Uversa para el gobierno del superuniverso, autorizando el envío de un organizador de fuerza y de su personal a la región anteriormente señalada por el inspector número 811.307. Las autoridades de Orvonton encargaron al primer explorador de este universo potencial que ejecutara el mandato de los Ancianos de los Días, el cual pedía que se organizara una nueva creación material.

\par
%\textsuperscript{(652.1)}
\textsuperscript{57:1.5} El registro de este permiso significa que el organizador de fuerza y su personal ya habían partido de Uversa para el largo viaje hacia ese sector oriental del espacio donde posteriormente emprenderían aquellas prolongadas actividades que culminarían en la aparición de una nueva creación física en Orvonton.

\par
%\textsuperscript{(652.2)}
\textsuperscript{57:1.6} Hace \textit{875.000.000.000} de años, la enorme nebulosa de Andronover, número 876.926, fue debidamente iniciada. Sólo se necesitaba la presencia del organizador de fuerza y su personal de enlace para inaugurar el torbellino de energía que se convertiría finalmente en este inmenso ciclón del espacio. Después de iniciar estas rotaciones nebulares, los organizadores de fuerza vivientes simplemente se retiran en ángulo recto respecto al plano del disco en rotación, y desde ese momento en adelante, las cualidades inherentes a la energía aseguran la evolución progresiva y ordenada de este nuevo sistema físico.

\par
%\textsuperscript{(652.3)}
\textsuperscript{57:1.7} Hacia esta época, la narración pasa a ocuparse de las actividades de las personalidades del superuniverso. En realidad, la historia comienza propiamente en este punto ---aproximadamente en el momento en que los organizadores de fuerza del Paraiso se disponen a retirarse, después de dejar preparadas las condiciones energéticas y espaciales para la acción de los directores de poder y los controladores físicos del superuniverso de Orvonton.

\section*{2. La etapa nebular primaria}
\par
%\textsuperscript{(652.4)}
\textsuperscript{57:2.1} Todas las creaciones materiales evolutivas nacen de nebulosas circulares y gaseosas, y todas estas nebulosas primarias son circulares durante la primera parte de su existencia gaseosa. A medida que envejecen se vuelven generalmente espirales, y cuando su función como formadoras de soles ha llegado a su fin, a menudo terminan como enjambres de estrellas o como soles enormes rodeados por un número variable de planetas, satélites y grupos más pequeños de materia, que en muchos aspectos se parecen a vuestro propio diminuto sistema solar.

\par
%\textsuperscript{(652.5)}
\textsuperscript{57:2.2} Hace \textit{800.000.000.000} de años, la creación de Andronover estaba bien establecida como una de las magníficas nebulosas primarias de Orvonton. Cuando los astrónomos de los universos cercanos contemplaban este fenómeno del espacio, observaban muy poca cosa que atrajera su atención. Los cálculos aproximados de la gravedad, realizados en las creaciones adyacentes, indicaban que se estaban produciendo materializaciones espaciales en las regiones de Andronover, pero eso era todo.

\par
%\textsuperscript{(652.6)}
\textsuperscript{57:2.3} Hace \textit{700.000.000.000} de años, el sistema de Andronover estaba alcanzando unas proporciones gigantescas, y se enviaron controladores físicos adicionales a nueve creaciones materiales circundantes para dar su apoyo y aportar su cooperación a los centros de poder de este nuevo sistema material que evolucionaba con tanta rapidez. En esta época lejana, todo el material legado a las creaciones posteriores estaba contenido dentro de los confines de esta gigantesca rueda espacial, que continuaba girando, y que después de haber alcanzado el máximo de su diámetro, giraba cada vez más deprisa a medida que continuaba condensándose y contrayéndose.

\par
%\textsuperscript{(652.7)}
\textsuperscript{57:2.4} Hace \textit{600.000.000.000} de años se alcanzó el punto culminante del período de movilización energética de Andronover; la nebulosa había adquirido el máximo de su masa. En aquel momento era una gigantesca nube circular de gas, con una forma un poco parecida a la de un esferoide aplanado. Éste fue el período inicial de la formación diferencial de la masa y de la variación en la velocidad de rotación. La gravedad y otras influencias estaban a punto de empezar su labor, convirtiendo los gases del espacio en materia organizada.

\section*{3. La etapa nebular secundaria}
\par
%\textsuperscript{(653.1)}
\textsuperscript{57:3.1} La enorme nebulosa empezó entonces a adoptar gradualmente la forma espiral y a volverse claramente visible incluso para los astrónomos de los universos lejanos. Ésta es la historia natural de la mayoría de las nebulosas; antes de empezar a arrojar soles y a emprender la tarea de construir un universo, estas nebulosas espaciales secundarias suelen observarse como \textit{fenómenos espirales}.

\par
%\textsuperscript{(653.2)}
\textsuperscript{57:3.2} Cuando los investigadores de estrellas de aquella época lejana, que vivían en las proximidades, observaron esta metamorfosis de la nebulosa de Andronover, vieron exactamente lo que ven los astrónomos del siglo veinte cuando dirigen sus telescopios hacia el espacio y examinan las nebulosas espirales actuales del espacio exterior adyacente.

\par
%\textsuperscript{(653.3)}
\textsuperscript{57:3.3} Hacia la época en que se alcanzó el máximo de masa, el control gravitatorio del contenido gaseoso empezó a debilitarse, lo cual fue seguido por el período de escape de gas. El gas salía a chorros como dos brazos gigantescos y distintos que tenían su origen en los lados opuestos de la masa materna. Las rápidas rotaciones de este enorme núcleo central pronto confirieron un aspecto espiral a estos dos chorros de gas lanzados por la nebulosa. El enfriamiento y la condensación posterior de algunas porciones de estos brazos sobresalientes produjeron finalmente su apariencia nudosa. Estas porciones más densas eran enormes sistemas y subsistemas de materia física que giraban rápidamente en el espacio en medio de la nube gaseosa de la nebulosa, permaneciendo firmemente sujetos al control gravitatorio de la rueda madre.

\par
%\textsuperscript{(653.4)}
\textsuperscript{57:3.4} Pero la nebulosa había empezado a contraerse, y el aumento de su velocidad de rotación redujo aún más el control de la gravedad; en poco tiempo, las regiones gaseosas exteriores empezaron a escaparse realmente del abrazo inmediato del núcleo nebular, saliendo al espacio en circuitos de contorno irregular, regresando a las regiones nucleares para completar sus circuitos, y así sucesivamente. Pero esto no era más que una etapa temporal de la evolución nebular. La velocidad de rotación cada vez mayor pronto iba a arrojar al espacio unos soles enormes en circuitos independientes.

\par
%\textsuperscript{(653.5)}
\textsuperscript{57:3.5} Y esto fue lo que sucedió en Andronover hace muchos millones de años. La rueda de energía creció y creció hasta que llegó a su máxima expansión, y entonces, cuando empezó la contracción, continuó girando cada vez más deprisa hasta que alcanzó finalmente la etapa centrífuga crítica y empezó la gran desintegración.

\par
%\textsuperscript{(653.6)}
\textsuperscript{57:3.6} Hace \textit{500.000.000.000} de años nació el primer sol de Andronover. Este haz resplandeciente se escapó del control de la gravedad materna y salió disparado al espacio hacia una aventura independiente en el cosmos de la creación. Su órbita quedó determinada por su trayectoria de escape. Estos soles tan jóvenes se vuelven rápidamente esféricos y empiezan su larga y extraordinaria carrera como estrellas del espacio. A excepción de los núcleos nebulares terminales, la inmensa mayoría de los soles de Orvonton han tenido un nacimiento semejante. Estos soles escapados pasan por diversos períodos de evolución y de servicio universal posterior.

\par
%\textsuperscript{(653.7)}
\textsuperscript{57:3.7} Hace \textit{400.000.000.000} de años empezó el período de recaptación de la nebulosa de Andronover. Muchos de los soles más cercanos y pequeños fueron capturados de nuevo a consecuencia de la ampliación gradual y de la condensación ulterior del núcleo materno. Muy pronto se inauguró la fase terminal de la condensación nebular, el período que precede siempre a la segregación final de estos inmensos agregados espaciales de energía y de materia.

\par
%\textsuperscript{(654.1)}
\textsuperscript{57:3.8} Apenas un millón de años después de esta época, Miguel de Nebadon, un Hijo Creador Paradisiaco, escogió esta nebulosa en desintegración como escenario para su aventura de construir un universo. Casi inmediatamente se empezaron a edificar los mundos arquitectónicos de Salvington y los cien grupos de planetas que forman las sedes centrales de las constelaciones. Se necesitó casi un millón de años para terminar estas agrupaciones de mundos especialmente creados. Los planetas sede de los sistemas locales se construyeron durante un período que se extendió desde esta época hasta hace unos cinco mil millones de años\footnote{\textit{La creación de Miguel}: Sal 33:6; 102:25; Is 45:12,18; Jn 1:1-3; Ef 2:10; 3:9; Col 1:16; Heb 1:2,10; Ap 4:11.}.

\par
%\textsuperscript{(654.2)}
\textsuperscript{57:3.9} Hace \textit{300.000.000.000} de años, los circuitos solares de Andronover estaban bien establecidos, y el sistema nebular estaba pasando por un período transitorio de relativa estabilidad física. Aproximadamente por esta época, el estado mayor de Miguel llegó a Salvington, y el gobierno de Orvonton en Uversa reconoció la existencia física del universo local de Nebadon.

\par
%\textsuperscript{(654.3)}
\textsuperscript{57:3.10} Hace \textit{200.000.000.000} de años se pudo presenciar el avance de la contracción y la condensación de Andronover, con una enorme generación de calor en su cúmulo central o masa nuclear. El espacio relativo apareció incluso en las regiones cercanas a la rueda madre solar central. Las regiones exteriores se volvían más estables y mejor organizadas; algunos planetas que giraban alrededor de los soles recién nacidos se habían enfriado lo suficiente como para ser idóneos para la implantación de la vida. Los planetas habitados más antiguos de Nebadon datan de estos tiempos.

\par
%\textsuperscript{(654.4)}
\textsuperscript{57:3.11} Ahora empieza a funcionar por primera vez el mecanismo universal terminado de Nebadon, y la creación de Miguel es registrada en Uversa como un universo para la habitación y la ascensión progresiva de los mortales.

\par
%\textsuperscript{(654.5)}
\textsuperscript{57:3.12} Hace \textit{100.000.000.000} de años, la tensión de la condensación nebular llegó a su apogeo; se había alcanzado el punto máximo de tensión calorífica. Esta etapa crítica de la lucha entre la gravedad y el calor a veces dura épocas enteras, pero tarde o temprano el calor gana la batalla contra la gravedad, y empieza el período espectacular de la dispersión de los soles. Esto señala el final de la carrera secundaria de una nebulosa del espacio.

\section*{4. Las etapas terciaria y cuaternaria}
\par
%\textsuperscript{(654.6)}
\textsuperscript{57:4.1} La etapa primaria de una nebulosa es circular; la secundaria, espiral; la etapa terciaria es la de la primera dispersión de los soles, mientras que la cuaternaria abarca el segundo y último ciclo de la dispersión solar, finalizando el núcleo madre como un cúmulo globular o como un sol solitario que funciona como centro de un sistema solar terminal.

\par
%\textsuperscript{(654.7)}
\textsuperscript{57:4.2} Hace \textit{75.000.000.000} de años, esta nebulosa había alcanzado el punto culminante de su etapa de familia solar. Éste fue el apogeo del primer período de pérdidas de soles. Desde entonces, la mayoría de estos soles se han apoderado de extensos sistemas de planetas, satélites, islas oscuras, cometas, meteoros y nubes de polvo cósmico.

\par
%\textsuperscript{(654.8)}
\textsuperscript{57:4.3} Hace \textit{50.000.000.000} de años, este primer período de dispersión de soles había concluido; la nebulosa terminaba rápidamente su ciclo terciario de existencia, durante el cual dio nacimiento a 876.926 sistemas solares.

\par
%\textsuperscript{(654.9)}
\textsuperscript{57:4.4} Hace \textit{25.000.000.000} de años se pudo contemplar la finalización del ciclo terciario de la vida nebular, lo que produjo la organización y la estabilización relativa de los extensos sistemas estelares derivados de esta nebulosa madre. Pero el proceso de contracción física y de creciente producción de calor continuó en la masa central del remanente nebular.

\par
%\textsuperscript{(655.1)}
\textsuperscript{57:4.5} Hace \textit{10.000.000.000} de años empezó el ciclo cuaternario de Andronover. La masa nuclear había alcanzado el máximo de temperatura; se acercaba el punto crítico de condensación. El núcleo madre original se convulsionaba bajo la presión combinada de la tensión de la condensación de su propio calor interno y la creciente atracción gravitatoria mareomotriz del enjambre de sistemas solares liberados que lo rodeaban. Las erupciones nucleares que iban a inaugurar el segundo ciclo nebular de dispersión solar eran inminentes. El ciclo cuaternario de existencia nebular estaba a punto de empezar.

\par
%\textsuperscript{(655.2)}
\textsuperscript{57:4.6} Hace \textit{8.000.000.000} de años comenzó la enorme erupción terminal. Sólo los sistemas exteriores están a salvo en el momento de un cataclismo cósmico semejante. Éste fue el principio del fin de la nebulosa. La descarga final de soles se prolongó durante un período de casi dos mil millones de años.

\par
%\textsuperscript{(655.3)}
\textsuperscript{57:4.7} Hace \textit{7.000.000.000} de años se pudo presenciar el punto culminante de la desintegración final de Andronover. Éste fue el período en que nacieron los soles terminales más grandes y el apogeo de las perturbaciones físicas locales.

\par
%\textsuperscript{(655.4)}
\textsuperscript{57:4.8} La época de hace \textit{6.000.000.000} de años señala el final de la desintegración terminal y el nacimiento de vuestro Sol, el quincuagésimo sexto antes del último de la segunda familia solar de Andronover. Esta erupción final del núcleo nebular dio origen a 136.702 soles, la mayoría de ellos esferas solitarias. El número total de soles y de sistemas solares que tuvieron su origen en la nebulosa de Andronover fue de 1.013.628. El Sol del sistema solar es el número 1.013.572.

\par
%\textsuperscript{(655.5)}
\textsuperscript{57:4.9} Ahora, la gran nebulosa de Andronover ya no existe, pero continúa viviendo en los numerosos soles y sus familias planetarias que se originaron en esta nube madre del espacio. El último resto nuclear de esta magnífica nebulosa arde todavía con un resplandor rojizo, y continúa emitiendo una luz y un calor moderados a su familia planetaria residual de ciento sesenta y cinco mundos, que giran ahora en torno a esta venerable madre de dos poderosas generaciones de monarcas de luz.

\section*{5. El origen de Monmatia ---el sistema solar de Urantia}
\par
%\textsuperscript{(655.6)}
\textsuperscript{57:5.1} Hace \textit{5.000.000.000} de años, vuestro Sol\footnote{\textit{El sistema solar}: Gn 1:3.} era una esfera llameante comparativamente aislada, que había atraído hacia sí la mayor parte de la materia cercana que circulaba por el espacio, los residuos del reciente cataclismo que había acompañado a su propio nacimiento.

\par
%\textsuperscript{(655.7)}
\textsuperscript{57:5.2} Vuestro Sol ha alcanzado hoy una estabilidad relativa, pero los ciclos de once años y medio de las manchas solares demuestran que era, en su juventud, una estrella variable. Durante los primeros tiempos de vuestro Sol, la contracción continua y el consiguiente aumento gradual de la temperatura iniciaron unas enormes convulsiones en su superficie. Estos levantamientos titánicos necesitaban tres días y medio para completar un ciclo de resplandor variable. Este estado variable, esta pulsación periódica, hicieron a vuestro Sol sumamente sensible a ciertas influencias externas que pronto iba a encontrar.

\par
%\textsuperscript{(655.8)}
\textsuperscript{57:5.3} El escenario del espacio local estaba así preparado para el origen excepcional de \textit{Monmatia}, nombre de la familia planetaria de vuestro Sol, el sistema solar al que pertenece vuestro mundo. Menos del uno por ciento de los sistemas planetarios de Orvonton han tenido un origen semejante.

\par
%\textsuperscript{(655.9)}
\textsuperscript{57:5.4} Hace \textit{4.500.000.000} de años, el enorme sistema de Angona empezó a aproximarse a los alrededores de este Sol solitario. El centro de este gran sistema era un gigante oscuro del espacio, sólido, muy cargado y con una enorme atracción gravitatoria.

\par
%\textsuperscript{(656.1)}
\textsuperscript{57:5.5} A medida que Angona se acercaba más al Sol, y en los momentos de la máxima expansión de las pulsaciones solares, unos chorros de material gaseoso salían lanzados hacia el espacio como gigantescas lenguas solares. Al principio, estas lenguas de gas llameantes volvían a caer invariablemente en el Sol, pero a medida que Angona se aproximaba cada vez más, la atracción gravitatoria del gigantesco visitante se hizo tan fuerte, que estas lenguas de gas se rompieron en algunos puntos; las raíces volvían a caer en el Sol mientras que las partes exteriores se separaban para formar cuerpos de materia independientes, meteoritos solares, que inmediatamente empezaban a girar alrededor del Sol en sus propias órbitas elípticas.

\par
%\textsuperscript{(656.2)}
\textsuperscript{57:5.6} A medida que el sistema de Angona se acercaba, las expulsiones solares se volvieron cada vez más grandes; una creciente cantidad de materia fue extraída del Sol para luego convertirse en cuerpos independientes que circulaban por el espacio circundante. Esta situación se desarrolló durante quinientos mil años, hasta que Angona alcanzó su punto más cercano al Sol; después de lo cual, y en conjunción con una de sus convulsiones periódicas internas, el Sol experimentó una ruptura parcial; enormes volúmenes de materia fueron arrojados simultáneamente por sus lados opuestos. Una inmensa columna de gases solares fue atraída hacia el lado de Angona; tenía los dos extremos más bien puntiagudos y el centro notablemente abultado, y se separó definitivamente del control gravitatorio inmediato del Sol.

\par
%\textsuperscript{(656.3)}
\textsuperscript{57:5.7} Esta gran columna de gases solares, que fue así separada del Sol, evolucionó posteriormente hasta convertirse en los doce planetas del sistema solar. Los gases expulsados por repercusión por el lado opuesto del Sol, en resonancia mareomotriz con la expulsión de este gigantesco antepasado del sistema solar, se han condensado desde entonces para formar los meteoros y el polvo espacial del sistema solar, aunque una gran cantidad de esta materia fue capturada de nuevo posteriormente por la gravedad solar a medida que el sistema de Angona se alejaba hacia el espacio distante.

\par
%\textsuperscript{(656.4)}
\textsuperscript{57:5.8} Aunque Angona consiguió extraer el material ancestral de los planetas del sistema solar y el enorme volumen de materia que ahora circula alrededor del Sol bajo la forma de asteroides y meteoros, no obtuvo para sí ninguna cantidad de esta materia solar. El sistema visitante no se acercó lo bastante como para robarle realmente alguna sustancia al Sol, pero sí pasó lo suficientemente cerca como para atraer hacia el espacio intermedio todo el material que compone el sistema solar actual.

\par
%\textsuperscript{(656.5)}
\textsuperscript{57:5.9} Los cinco planetas interiores y los cinco exteriores pronto se formaron en miniatura a partir de los núcleos que se iban enfriando y condensando en los extremos afilados y menos masivos de la gigantesca protuberancia gravitatoria que Angona había logrado separar del Sol, mientras que Saturno y Júpiter se formaron a partir de las porciones centrales más masivas y abultadas. La poderosa atracción gravitatoria de Júpiter y de Saturno pronto capturó la mayor parte del material robado a Angona, como lo atestigua el movimiento retrógrado de algunos de sus satélites.

\par
%\textsuperscript{(656.6)}
\textsuperscript{57:5.10} Como Júpiter y Saturno habían tenido su origen en el centro mismo de la enorme columna de gases solares sobrecalentados, contenían tanto material solar a alta temperatura que brillaban con una luz resplandeciente y emitían enormes cantidades de calor; durante un corto período de tiempo, después de su formación como cuerpos espaciales separados, fueron en realidad unos soles secundarios. Estos dos planetas, los más grandes del sistema solar, han continuado siendo ampliamente gaseosos hasta el día de hoy, pues aún no se han enfriado todavía hasta el punto de condensarse o de solidificarse por completo.

\par
%\textsuperscript{(656.7)}
\textsuperscript{57:5.11} Los núcleos gaseosos en contracción de los otros diez planetas pronto alcanzaron la etapa de la solidificación, y empezaron así a atraer hacia ellos cantidades crecientes de la materia meteórica que circulaba por el espacio cercano. Los mundos del sistema solar tuvieron pues un doble origen: fueron unos núcleos de condensación gaseosa, que más tarde aumentaron gracias a la captura de enormes cantidades de meteoros. De hecho, todavía continúan capturando meteoros, pero en cantidades mucho menores.

\par
%\textsuperscript{(657.1)}
\textsuperscript{57:5.12} Los planetas no dan vueltas alrededor del Sol en el plano ecuatorial de su madre solar, cosa que harían si hubieran sido arrojados por la rotación solar. Circulan más bien en el plano de la expulsión solar causada por Angona, plano que formaba un ángulo considerable con el del ecuador solar.

\par
%\textsuperscript{(657.2)}
\textsuperscript{57:5.13} Aunque Angona fue incapaz de capturar una mínima parte de la masa solar, vuestro Sol sí añadió a su familia planetaria en metamorfosis algunos materiales del sistema visitante que circulaban por el espacio. Debido al intenso campo gravitatorio de Angona, su familia planetaria tributaria describía sus órbitas a una distancia considerable del gigante oscuro. Poco después de la expulsión de la masa ancestral del sistema solar, y mientras Angona se encontraba todavía en las proximidades del Sol, tres de los planetas mayores del sistema de Angona pasaron tan cerca de este masivo antepasado del sistema solar, que su atracción gravitatoria, aumentada con la del Sol, fue suficiente para desequilibrar el control gravitatorio de Angona y separar definitivamente a estos tres tributarios del vagabundo celeste.

\par
%\textsuperscript{(657.3)}
\textsuperscript{57:5.14} Todo el material del sistema solar procedente del Sol estaba dotado originalmente de una órbita con una dirección homogénea, y si no hubiera sido por la intrusión de estos tres cuerpos espaciales extraños, todo el material del sistema solar continuaría manteniendo la misma dirección en su movimiento orbital. Sin embargo, el impacto de los tres tributarios de Angona inyectó unas fuerzas direccionales nuevas y extrañas en el sistema solar emergente, con la aparición resultante del \textit{movimiento retrógrado}. En cualquier sistema astronómico, el movimiento retrógrado siempre es accidental y aparece siempre a consecuencia del impacto debido a la colisión de cuerpos espaciales extraños. Estas colisiones no siempre producen un movimiento retrógrado, pero nunca aparece un movimiento retrógrado como no sea en un sistema que contenga unas masas de orígenes diversos.

\section*{6. La etapa del sistema solar ---La era de la formación de los planetas}
\par
%\textsuperscript{(657.4)}
\textsuperscript{57:6.1} El nacimiento del sistema solar fue seguido por un período de disminución de las descargas solares. Durante otros quinientos mil años, y de manera decreciente, el Sol continuó arrojando volúmenes de materia cada vez menores al espacio circundante. Pero durante estos tiempos primitivos de las órbitas erráticas, cuando los cuerpos circundantes se encontraban en su perihelio, la madre solar conseguía capturar de nuevo una gran parte de este material meteórico.

\par
%\textsuperscript{(657.5)}
\textsuperscript{57:6.2} Los planetas más cercanos al Sol fueron los primeros que aminoraron su rotación debido a la fricción mareomotriz. Estas influencias gravitatorias contribuyen también a la estabilización de las órbitas planetarias, ya que actúan como un freno sobre la velocidad de rotación axial del planeta; esto hace que un planeta gire cada vez más lentamente hasta que se detiene su rotación axial, quedando un hemisferio del planeta siempre vuelto hacia el Sol o el cuerpo más grande, tal como lo demuestran el planeta Mercurio y la Luna, la cual siempre presenta la misma cara a Urantia.

\par
%\textsuperscript{(657.6)}
\textsuperscript{57:6.3} Cuando las fricciones mareomotrices de la Luna y la Tierra se igualen, la Tierra siempre presentará el mismo hemisferio a la Luna, y el día y el mes serán análogos ---con una duración de unos cuarenta y siete días. Cuando se alcance esta estabilización de las órbitas, las fricciones mareomotrices actuarán en sentido contrario, dejando de impulsar a la Luna lejos de la Tierra, y atrayendo gradualmente al satélite hacia el planeta. Entonces, cuando en ese futuro muy distante la Luna se acerque a unos dieciocho mil kilómetros de la Tierra, la acción gravitatoria de ésta última hará que la Luna estalle, y esta explosión ocasionada por la gravedad mareomotriz la hará añicos, convirtiéndola en pequeñas partículas que podrán reunirse alrededor del mundo como anillos de materia parecidos a los de Saturno, o ser atraídas gradualmente hacia la Tierra en forma de meteoros.

\par
%\textsuperscript{(658.1)}
\textsuperscript{57:6.4} Si el tamaño y la densidad de los cuerpos espaciales son similares, pueden producirse colisiones. Pero si dos cuerpos espaciales de densidad semejante tienen un tamaño relativamente desigual, y el más pequeño se acerca progresivamente al mayor, entonces el más pequeño se desintegrará cuando el radio de su órbita se vuelva inferior a dos veces y media al radio del cuerpo mayor. Las colisiones entre los gigantes del espacio son realmente raras, pero estas explosiones de los cuerpos menores debidas a la gravedad mareomotriz son muy frecuentes.

\par
%\textsuperscript{(658.2)}
\textsuperscript{57:6.5} Las estrellas fugaces se encuentran en enjambres porque son los fragmentos de cuerpos materiales más grandes, que han estallado a causa de la gravedad mareomotriz ejercida por cuerpos espaciales cercanos y mucho más grandes. Los anillos de Saturno son los fragmentos de un satélite que reventó. Una de las lunas de Júpiter se está acercando ahora peligrosamente a la zona crítica de desintegración mareomotriz, y dentro de algunos millones de años o bien será reclamada por el planeta, o sufrirá la desintegración causada por la gravedad mareomotriz. Hace muchísimo tiempo, el quinto planeta del sistema solar recorrió una órbita irregular, acercándose periódicamente cada vez más a Júpiter, hasta que entró en la zona crítica de desintegración gravitatoria mareomotriz; entonces se fragmentó rápidamente y se convirtió en el enjambre actual de asteroides.

\par
%\textsuperscript{(658.3)}
\textsuperscript{57:6.6} Hace \textit{4.000.000.000} de años se pudo presenciar la organización de los sistemas de Júpiter y Saturno con una forma muy semejante a la que tienen hoy, a excepción de sus lunas, que continuaron aumentando de tamaño durante varios miles de millones de años. De hecho, todos los planetas y satélites del sistema solar siguen creciendo a consecuencia de las continuas capturas de meteoros.

\par
%\textsuperscript{(658.4)}
\textsuperscript{57:6.7} Hace \textit{3.500.000.000} de años, los núcleos de condensación de los otros diez planetas estaban bien formados, y el centro de la mayoría de las lunas estaba intacto, aunque algunos satélites más pequeños se unieron posteriormente para formar las lunas actuales más grandes. Esta época se puede considerar como la era de la formación planetaria.

\par
%\textsuperscript{(658.5)}
\textsuperscript{57:6.8} Hace \textit{3.000.000.000} de años, el sistema solar funcionaba de manera muy parecida a la de hoy. El tamaño de sus integrantes continuaba creciendo a medida que los meteoros del espacio seguían cayendo sobre los planetas y sus satélites a un ritmo prodigioso.

\par
%\textsuperscript{(658.6)}
\textsuperscript{57:6.9} Hacia esta época, vuestro sistema solar fue inscrito en el registro físico de Nebadon y se le dio el nombre de Monmatia.

\par
%\textsuperscript{(658.7)}
\textsuperscript{57:6.10} Hace \textit{2.500.000.000} de años, el tamaño de los planetas había aumentado inmensamente. Urantia era una esfera bien desarrollada; tenía aproximadamente una décima parte de su masa actual y continuaba aumentando rápidamente por acreción meteórica.

\par
%\textsuperscript{(658.8)}
\textsuperscript{57:6.11} Toda esta enorme actividad forma parte normalmente de la construcción de un mundo evolutivo del tipo de Urantia, y constituye los preliminares astronómicos que preparan el terreno para el comienzo de la evolución física de estos mundos del espacio, como parte de los preparativos para las aventuras de la vida en el tiempo.

\section*{7. La era meteórica --- La época volcánica --- La atmósfera planetaria primitiva}
\par
%\textsuperscript{(658.9)}
\textsuperscript{57:7.1} Durante todos estos tiempos primitivos, las regiones espaciales del sistema solar estaban plagadas de pequeños cuerpos formados por fragmentación y condensación, y a falta de una atmósfera protectora que los quemara, estos cuerpos espaciales se estrellaban directamente en la superficie de Urantia. Estos impactos constantes mantenían la superficie del planeta más o menos caliente, y esta circunstancia, unida a la creciente actividad de la gravedad a medida que la esfera se agrandaba, empezó a poner en funcionamiento aquellas influencias que provocaron gradualmente que los elementos más pesados, como el hierro, se asentaran cada vez más en el centro del planeta.

\par
%\textsuperscript{(659.1)}
\textsuperscript{57:7.2} Hace \textit{2.000.000.000} de años, la Tierra empezó a ganarle terreno decididamente a la Luna. El planeta siempre había sido más grande que su satélite, pero no había habido mucha diferencia de tamaño hasta esta época, durante la cual la Tierra capturó enormes cuerpos espaciales. Urantia tenía entonces aproximadamente una quinta parte de su tamaño actual y se había vuelto lo bastante grande como para retener la atmósfera primitiva que había empezado a aparecer a consecuencia de la lucha interna elemental entre el interior caliente y la corteza que se enfriaba.

\par
%\textsuperscript{(659.2)}
\textsuperscript{57:7.3} La actividad volcánica en firme data de estos tiempos. El calor interno de la Tierra continuaba aumentando debido al enterramiento cada vez más profundo de los elementos radiactivos, o más pesados, traídos del espacio por los meteoros. El estudio de estos elementos radiactivos revelará que la superficie de Urantia tiene más de mil millones de años. La datación por medio del radio es vuestro cronómetro más fiable para calcular científicamente la edad del planeta, pero todas estas estimaciones se quedan demasiado cortas, porque todos los materiales radiactivos disponibles para vuestro examen proceden de la superficie terrestre y representan por tanto unas adquisiciones de estos elementos, por parte de Urantia, relativamente recientes.

\par
%\textsuperscript{(659.3)}
\textsuperscript{57:7.4} Hace \textit{1.500.000.000} de años, la Tierra tenía dos tercios de su tamaño actual, mientras que la Luna se acercaba a su masa de hoy. El hecho de que la Tierra adelantara en tamaño rápidamente a la Luna, le permitió empezar a robarle lentamente a su satélite la poca atmósfera que tenía al principio.

\par
%\textsuperscript{(659.4)}
\textsuperscript{57:7.5} La actividad volcánica está ahora en su apogeo. Toda la Tierra es un verdadero infierno de fuego; su superficie se parece a la de su primitivo estado fundido antes de que los metales más pesados gravitaran hacia el centro. \textit{Es la era de los volcanes}. Sin embargo, una corteza compuesta principalmente de granito relativamente más ligero se está formando gradualmente. El escenario se está preparando en un planeta que algún día podrá mantener la vida.

\par
%\textsuperscript{(659.5)}
\textsuperscript{57:7.6} La atmósfera planetaria primitiva va evolucionando lentamente; en este momento contiene un poco de vapor de agua, monóxido de carbono, dióxido de carbono y cloruro de hidrógeno, pero hay poco o ningún nitrógeno libre u oxígeno libre. La atmósfera de un mundo en la era volcánica ofrece un espectáculo extraño. Además de los gases enumerados, está sobrecargada de numerosos gases volcánicos, y a medida que se forma el cinturón atmosférico, hay que añadir los productos de la combustión de las abundantes lluvias meteóricas que se precipitan constantemente sobre la superficie del planeta. Esta combustión meteórica mantiene el oxígeno atmosférico muy cerca del agotamiento, y el ritmo del bombardeo meteórico continúa siendo enorme.

\par
%\textsuperscript{(659.6)}
\textsuperscript{57:7.7} La atmósfera pronto se volvió más estable y se enfrió lo suficiente como para provocar precipitaciones de lluvia\footnote{\textit{Agua}: Gn 1:6-10; 2:6.} sobre la superficie rocosa caliente del planeta. Durante miles de años, Urantia estuvo envuelta en un continuo inmenso manto de vapor. Y durante estas épocas, el Sol no brilló nunca sobre la superficie de la Tierra.

\par
%\textsuperscript{(659.7)}
\textsuperscript{57:7.8} Una gran parte del carbono de la atmósfera fue extraído para formar los carbonatos de los diversos metales que abundaban en las capas superficiales del planeta. Más adelante, la prolífica vida vegetal primitiva consumió unas cantidades mucho mayores de estos gases carbónicos.

\par
%\textsuperscript{(660.1)}
\textsuperscript{57:7.9} Incluso en los períodos posteriores, las continuas corrientes de lava y las caídas de meteoros agotaron casi por completo el oxígeno del aire. Incluso los primeros depósitos del océano primitivo que pronto aparecería no contenían ni piedras coloreadas ni esquistos. Durante mucho tiempo después de que este océano apareciera, casi no hubo oxígeno libre en la atmósfera, y no apareció en cantidades significativas hasta que fue generado posteriormente por las algas marinas y otras formas de vida vegetal.

\par
%\textsuperscript{(660.2)}
\textsuperscript{57:7.10} La atmósfera planetaria primitiva de la era volcánica ofrece poca protección contra los impactos y colisiones de los enjambres meteóricos. Millones y millones de meteoros pueden penetrar en esta capa de aire para venir a estrellarse contra la corteza planetaria como cuerpos sólidos. Pero a medida que pasa el tiempo, hay cada vez menos meteoros que resulten lo bastante grandes para soportar el escudo de fricción, cada día más resistente, de la atmósfera enriquecida en oxígeno de las eras más tardías.

\section*{8. La estabilización de la corteza --- La época de los terremotos --- El océano mundial y el primer continente}
\par
%\textsuperscript{(660.3)}
\textsuperscript{57:8.1} Hace \textit{1.000.000.000} de años comienza realmente la historia de Urantia. El planeta había alcanzado aproximadamente su tamaño actual. Por esta época fue inscrito en los registros físicos de Nebadon y se le dio el nombre de \textit{Urantia}.

\par
%\textsuperscript{(660.4)}
\textsuperscript{57:8.2} La atmósfera, así como las constantes precipitaciones de humedad, facilitaron el enfriamiento de la corteza terrestre\footnote{\textit{Estabilización de la corteza}: Gn 1:9-10.}. La actividad volcánica igualó en poco tiempo la presión calorífica interna y la contracción de la corteza; y mientras los volcanes disminuían rápidamente, los terremotos hicieron su aparición a medida que avanzaba esta época de enfriamiento y de ajuste de la corteza.

\par
%\textsuperscript{(660.5)}
\textsuperscript{57:8.3} La verdadera historia geológica de Urantia comienza cuando la corteza terrestre se enfrió lo suficiente para provocar la formación del primer océano. La condensación del vapor de agua sobre la superficie de la Tierra que se enfriaba, una vez iniciada, continuó hasta que estuvo prácticamente concluida. Hacia el final de este período, el océano ocupaba el mundo entero, cubriendo todo el planeta con una profundidad media de casi dos kilómetros. Las mareas funcionaban de manera muy similar a la de hoy, pero este océano primitivo no era salado; era prácticamente una envoltura de agua dulce que cubría el mundo. En aquellos tiempos, la mayor parte del cloro estaba combinado con diversos metales, pero había suficiente cloro unido al hidrógeno para hacer que este agua fuera ligeramente ácida.

\par
%\textsuperscript{(660.6)}
\textsuperscript{57:8.4} Al comienzo de esta era lejana, Urantia podría considerarse como un planeta rodeado de agua. Más adelante, unas corrientes de lava más profundas, y por lo tanto más densas, brotaron en el fondo del actual Océano Pacífico, y esta parte de la superficie cubierta de agua se hundió considerablemente. La primera masa de suelo continental surgió del océano mundial para ajustar y compensar el equilibrio de la corteza terrestre que se volvía gradualmente más espesa.

\par
%\textsuperscript{(660.7)}
\textsuperscript{57:8.5} Hace \textit{950.000.000} de años, Urantia ofrece la imagen de un solo gran continente y una sola gran extensión de agua, el Océano Pacífico. Los volcanes están todavía esparcidos por todas partes y los terremotos son a la vez frecuentes e intensos. Los meteoros continúan bombardeando la Tierra, pero van disminuyendo tanto en frecuencia como en tamaño. La atmósfera se va aclarando, pero la cantidad de dióxido de carbono sigue siendo elevada. La corteza terrestre se va estabilizando poco a poco.

\par
%\textsuperscript{(660.8)}
\textsuperscript{57:8.6} Aproximadamente por esta época, Urantia fue asignada al sistema de Satania para su administración planetaria, y fue inscrita en el registro de vida de Norlatiadek. Entonces empezó el reconocimiento administrativo de la pequeña e insignificante esfera que estaba destinada a convertirse en el planeta donde Miguel acometería posteriormente la formidable empresa de donación como mortal, y participaría en aquellas experiencias que han hecho que, desde entonces, Urantia sea conocida localmente como <<el mundo de la cruz>>.

\par
%\textsuperscript{(661.1)}
\textsuperscript{57:8.7} Hace \textit{900.000.000} de años, se pudo presenciar la llegada a Urantia del primer grupo explorador de Satania, enviado desde Jerusem para examinar el planeta y hacer un informe sobre su adaptación como centro experimental de vida. Esta comisión constaba de veinticuatro miembros e incluía Portadores de Vida, Hijos Lanonandeks, Melquisedeks, serafines y otras órdenes de vida celestial que están relacionadas con la organización y la administración planetarias de los primeros tiempos.

\par
%\textsuperscript{(661.2)}
\textsuperscript{57:8.8} Después de haber realizado una cuidadosa inspección del planeta, esta comisión regresó a Jerusem e informó favorablemente al Soberano del Sistema, recomendando que Urantia fuera inscrita en el registro de experimentación con la vida. En consecuencia, vuestro mundo quedó inscrito en Jerusem como planeta decimal, y se notificó a los Portadores de Vida que se les concedería un permiso para establecer nuevos modelos de movilización mecánica, química y eléctrica en el momento de su llegada posterior con el mandato de transplantar e implantar la vida.

\par
%\textsuperscript{(661.3)}
\textsuperscript{57:8.9} A su debido tiempo, la comisión mixta de los doce en Jerusem finalizó los preparativos para la ocupación del planeta, los cuales fueron aprobados por la comisión planetaria de los setenta en Edentia. Estos planes, propuestos por los consejeros consultivos de los Portadores de Vida, fueron finalmente aceptados en Salvington. Poco tiempo después, las transmisiones de Nebadon difundieron la declaración de que Urantia se convertiría en el escenario donde los Portadores de Vida ejecutarían, en Satania, su sexagésimo experimento destinado a ampliar y mejorar el tipo sataniano de los modelos de vida de Nebadon.

\par
%\textsuperscript{(661.4)}
\textsuperscript{57:8.10} Poco después de que las transmisiones universales hubieran reconocido a Urantia por primera vez ante todo Nebadon, se le concedió la plena pertenencia a este universo. Poco después de esto, fue inscrita en los registros de los planetas sede del sector menor y del sector mayor del superuniverso; y antes del final de esta época, Urantia había sido asentada en el registro de la vida planetaria de Uversa.

\par
%\textsuperscript{(661.5)}
\textsuperscript{57:8.11} Toda esta época estuvo caracterizada por tormentas frecuentes y violentas. La corteza terrestre primitiva estaba en un estado de cambio continuo. El enfriamiento de la superficie alternaba con inmensas corrientes de lava. En ninguna parte de la superficie del mundo se puede encontrar un vestigio de su corteza planetaria original. Todo se ha mezclado demasiadas veces con las lavas expulsadas desde sus profundos orígenes y entremezclado con los depósitos posteriores del océano mundial primitivo.

\par
%\textsuperscript{(661.6)}
\textsuperscript{57:8.12} En ninguna parte de la superficie del mundo se podrán encontrar más restos modificados de estas antiguas rocas preoceánicas que en el nordeste de Canadá, alrededor de la Bahía de Hudson. Esta extensa elevación de granito está compuesta de una roca que pertenece a los tiempos preoceánicos. Estas capas rocosas han sido calentadas, curvadas, torcidas, aplastadas y han pasado muchas veces por estas experiencias metamórficas deformadoras.

\par
%\textsuperscript{(661.7)}
\textsuperscript{57:8.13} A lo largo de todas las épocas oceánicas, enormes capas de roca estratificada desprovista de fósiles se depositaron en el fondo de este antiguo océano. (La piedra caliza puede formarse a consecuencia de una precipitación química; no toda la antigua piedra caliza fue producida por los depósitos de la vida marina.) En ninguna de estas antiguas formaciones rocosas se encontrarán indicios de vida; no contienen fósiles, a menos que los depósitos posteriores de las épocas acuáticas se hayan mezclado por casualidad con estas capas más antiguas anteriores a la vida.

\par
%\textsuperscript{(662.1)}
\textsuperscript{57:8.14} La corteza terrestre primitiva era muy inestable, pero las montañas no estaban en proceso de formación. A medida que se formaba, el planeta se contraía bajo la presión de la gravedad. Las montañas no son el resultado del hundimiento de la corteza en vías de enfriamiento de una esfera en contracción, sino que aparecen más tarde a consecuencia de la acción de la lluvia, la gravedad y la erosión.

\par
%\textsuperscript{(662.2)}
\textsuperscript{57:8.15} La masa terrestre continental de esta era aumentó hasta cubrir casi un diez por ciento de la superficie de la Tierra. Los intensos terremotos no empezaron hasta que la masa continental no se elevó a un buen nivel por encima del agua. Una vez que empezaron, fueron aumentando en frecuencia y en intensidad durante épocas enteras. Los terremotos van disminuyendo desde hace muchos millones de años, pero Urantia aún sufre una media de quince por día.

\par
%\textsuperscript{(662.3)}
\textsuperscript{57:8.16} Hace \textit{850.000.000} de años que empezó realmente la primera época de la estabilización de la corteza terrestre. La mayoría de los metales más pesados se habían asentado en el centro del globo; la corteza en vías de enfriamiento había dejado de hundirse en unas proporciones tan extensas como en las épocas anteriores. Se había establecido un mejor equilibrio entre las extrusiones de tierra y el fondo más denso del océano. Debajo de la corteza, el flujo de la capa de lava se extendió casi por el mundo entero, lo que compensó y estabilizó las fluctuaciones debidas al enfriamiento, la contracción y los desplazamientos superficiales.

\par
%\textsuperscript{(662.4)}
\textsuperscript{57:8.17} Las erupciones volcánicas y los terremotos continuaron disminuyendo en frecuencia y en intensidad. La atmósfera se depuraba de los gases volcánicos y del vapor de agua, pero el porcentaje de dióxido de carbono continuaba siendo alto.

\par
%\textsuperscript{(662.5)}
\textsuperscript{57:8.18} Las perturbaciones eléctricas iban decreciendo también en el aire y en la tierra. Las corrientes de lava habían traído a la superficie una mezcla de elementos que diversificaron la corteza y aislaron mejor al planeta de ciertas energías espaciales. Todo esto contribuyó mucho a facilitar el control de la energía terrestre y a regular su circulación, como lo revela el funcionamiento de los polos magnéticos.

\par
%\textsuperscript{(662.6)}
\textsuperscript{57:8.19} Hace \textit{800.000.000} de años se pudo presenciar la inauguración de la primera gran época terrestre, el período de una creciente elevación continental.

\par
%\textsuperscript{(662.7)}
\textsuperscript{57:8.20} Desde la condensación de la hidrosfera terrestre, primero como océano mundial y posteriormente como Océano Pacífico, pensad que esta última masa de agua cubría entonces las nueve décimas partes de la superficie de la Tierra. Los meteoros que caían al mar se acumulaban en el fondo del océano, y los meteoros están compuestos generalmente de materiales pesados. Los que caían en la tierra se oxidaban considerablemente, luego eran desgastados por la erosión y llevados hacia las cuencas oceánicas. Así pues, el fondo del océano se volvió cada vez más pesado, y a esto había que añadir el peso de una masa de agua que en algunas partes tenía una profundidad de dieciséis kilómetros.

\par
%\textsuperscript{(662.8)}
\textsuperscript{57:8.21} El creciente empuje hacia abajo del Océano Pacífico actuó para empujar ulteriormente hacia arriba la masa continental. Europa y África empezaron a elevarse de las profundidades del Pacífico, junto con las masas que ahora se llaman Australia, América del Norte y del Sur y el continente de la Antártida, mientras que el fondo del Océano Pacífico emprendió un ajuste compensatorio hundiéndose aún más. Hacia el final de este período, casi un tercio de la superficie del planeta se componía de tierra, toda en un solo bloque continental.

\par
%\textsuperscript{(662.9)}
\textsuperscript{57:8.22} Las primeras diferencias climáticas del planeta aparecieron con este aumento de la elevación de las tierras. La elevación del suelo, las nubes cósmicas y las influencias oceánicas son los factores principales de las fluctuaciones climáticas. En el momento de la máxima emergencia de las tierras, la espina dorsal de la masa terrestre asiática alcanzó una altura de casi quince kilómetros. Si hubiera habido mucha humedad en el aire que se cernía sobre estas regiones tan elevadas, se habrían formado enormes capas de hielo, y la época glacial hubiera llegado mucho antes. Transcurrieron varios cientos de millones de años antes de que volvieran a aparecer tantas tierras por encima del agua.

\par
%\textsuperscript{(663.1)}
\textsuperscript{57:8.23} Hace \textit{750.000.000} años empezaron a aparecer las primeras fracturas en la masa continental, como por ejemplo la gran grieta norte-sur, que más tarde dejó entrar las aguas del océano y preparó el camino para la deriva hacia el oeste de los continentes de América del Norte y del Sur, incluyendo a Groenlandia. La larga hendidura este-oeste separó a África de Europa y apartó del continente asiático a las masas terrestres de Australia, las Islas del Pacífico y la Antártida.

\par
%\textsuperscript{(663.2)}
\textsuperscript{57:8.24} Hace \textit{700.000.000} de años, Urantia se estaba acercando a las condiciones de madurez adecuadas para mantener la vida. La deriva continental continuaba; el océano penetraba cada vez más en la tierra en forma de largos brazos de mar, proporcionando las aguas poco profundas y las bahías protegidas tan apropiadas para el hábitat de la vida marina.

\par
%\textsuperscript{(663.3)}
\textsuperscript{57:8.25} Hace \textit{650.000.000} de años se pudo presenciar una nueva separación de las masas terrestres y, en consecuencia, una nueva expansión de los mares continentales. Y estas aguas estaban alcanzando rápidamente el grado de salinidad imprescindible para la vida en Urantia.

\par
%\textsuperscript{(663.4)}
\textsuperscript{57:8.26} Estos mares y sus sucesores fueron los que establecieron los archivos vivientes de Urantia, tal como se descubrieron posteriormente en las páginas de piedra bien conservadas, volumen tras volumen, a medida que una era sucedía a la otra y una época daba nacimiento a la siguiente. Estos mares interiores de los tiempos antiguos fueron verdaderamente la cuna de la evolución.

\par
%\textsuperscript{(663.5)}
\textsuperscript{57:8.27} [Presentado por un Portador de Vida, miembro del Cuerpo original de Urantia y actualmente observador residente.]


\chapter{Documento 58. El establecimiento de la vida en Urantia}
\par
%\textsuperscript{(664.1)}
\textsuperscript{58:0.1} EN TODO Satania sólo existen sesenta y un mundos similares a Urantia, planetas donde se ha modificado la vida. La mayoría de los mundos habitados están poblados de acuerdo con unas técnicas establecidas; en dichas esferas, los Portadores de Vida tienen poca libertad para planear la implantación de la vida. Pero uno de cada diez mundos aproximadamente es designado como \textit{planeta decimal} y se le inscribe en el registro especial de los Portadores de Vida; en esos planetas se nos permite emprender ciertos experimentos con la vida para intentar modificar, o quizás mejorar, los tipos normales de seres vivos del universo.

\section*{1. Las condiciones previas para la vida física}
\par
%\textsuperscript{(664.2)}
\textsuperscript{58:1.1} Hace \textit{600.000.000} de años, la comisión de Portadores de Vida enviada desde Jerusem llegó a Urantia y empezó a estudiar las condiciones físicas preparatorias para desencadenar la vida en el mundo número 606 del sistema de Satania. Ésta iba a ser nuestra experiencia número seiscientos seis en la iniciación de los modelos de vida de Nebadon, en Satania, y nuestra sexagésima oportunidad para efectuar cambios y establecer modificaciones en los modelos de vida básicos y normales del universo local.

\par
%\textsuperscript{(664.3)}
\textsuperscript{58:1.2} Conviene aclarar que los Portadores de Vida no pueden iniciar la vida hasta que una esfera no se encuentra madura para la inauguración del ciclo evolutivo. Tampoco podemos prever un desarrollo de la vida más rápido del que puede sustentar y acomodar el progreso físico del planeta.

\par
%\textsuperscript{(664.4)}
\textsuperscript{58:1.3} Los Portadores de Vida de Satania habían proyectado un modelo de vida basado en el cloruro de sodio; por lo tanto, no se podía tomar ninguna medida para plantarlo hasta que las aguas del océano se hubieran vuelto suficientemente salobres. El tipo de protoplasma de Urantia sólo puede funcionar en una solución salina adecuada. Toda la vida ancestral ---vegetal y animal--- ha evolucionado en un hábitat de solución salina. Incluso los animales terrestres extremadamente organizados no podrían continuar viviendo si esta misma solución salina esencial no circulara por todo su cuerpo en la corriente sanguínea que baña abundantemente cada minúscula célula viviente, sumergiéndola literalmente en este <<océano>>.

\par
%\textsuperscript{(664.5)}
\textsuperscript{58:1.4} Vuestros antepasados primitivos circulaban libremente por el océano salado; hoy, esta misma solución salina, semejante a la del océano, circula libremente por vuestro cuerpo, bañando cada célula individual en un líquido químico comparable, en todos los aspectos fundamentales, al agua salada que estimuló las primeras reacciones protoplásmicas de las primeras células vivientes que funcionaron en el planeta.

\par
%\textsuperscript{(664.6)}
\textsuperscript{58:1.5} Pero al comienzo de esta era, Urantia evoluciona en todos los sentidos hacia un estado favorable para el mantenimiento de las formas iniciales de la vida marina. Poco a poco, pero de manera segura, los acontecimientos físicos en la Tierra y en las regiones adyacentes del espacio van preparando el escenario para los intentos posteriores destinados a establecer esas formas de vida que habíamos decidido que se adaptarían mejor al entorno físico ---tanto terrestre como espacial--- en vías de desarrollo.

\par
%\textsuperscript{(665.1)}
\textsuperscript{58:1.6} Posteriormente, la comisión de Portadores de Vida de Satania regresó a Jerusem, prefiriendo esperar a que se separara ulteriormente la masa terrestre continental, lo que proporcionaría aún más mares interiores y bahías abrigadas, antes de empezar realmente la implantación de la vida.

\par
%\textsuperscript{(665.2)}
\textsuperscript{58:1.7} En un planeta donde la vida tiene un origen marino, las condiciones ideales para la implantación de la vida son suministradas por un gran número de mares interiores, por un extenso litoral de aguas poco profundas y de bahías abrigadas; y precisamente las aguas de la Tierra se estaban distribuyendo rápidamente de esta manera. Estos antiguos mares interiores tenían raramente más de ciento cincuenta o ciento ochenta metros de profundidad, y la luz del Sol puede penetrar en el agua del océano hasta más de ciento ochenta metros.

\par
%\textsuperscript{(665.3)}
\textsuperscript{58:1.8} Y desde estos litorales, pero en los climas templados y regulares de una época más tardía, la vida vegetal primitiva consiguió llegar hasta la tierra. Allí, el alto grado de carbono en la atmósfera proporcionó a las nuevas variedades de vida terrestre la oportunidad de crecer con rapidez y exuberancia. Aunque esta atmósfera era entonces ideal para el crecimiento de las plantas, contenía tanta cantidad de dióxido de carbono que ningún animal, y mucho menos el hombre, podría haber vivido en la superficie de la Tierra.

\section*{2. La atmósfera de Urantia}
\par
%\textsuperscript{(665.4)}
\textsuperscript{58:2.1} La atmósfera planetaria filtra hasta la tierra aproximadamente una dos mil millonésima parte de la emanación luminosa total del Sol. Si la luz que cae sobre América del Norte se pagara a razón de dos centavos por kilovatio hora, la factura anual de la electricidad sobrepasaría los 800 mil billones de dólares. La factura de la luz solar de Chicago ascendería a una cantidad considerablemente superior a los 100 millones de dólares diarios. Y debéis recordar que recibís del Sol otras formas de energía ---la luz no es la única contribución solar que llega hasta vuestra atmósfera. Numerosas energías solares entran a raudales en Urantia, abarcando unas longitudes de onda que se extienden tanto por encima como por debajo del alcance de la visión humana.

\par
%\textsuperscript{(665.5)}
\textsuperscript{58:2.2} La atmósfera de la Tierra es casi opaca para una gran parte de la radiación solar del extremo ultravioleta del espectro. La mayoría de estas longitudes de onda corta son absorbidas por una capa de ozono que existe por todo un nivel situado a unos dieciséis kilómetros por encima de la superficie de la Tierra, y que se extiende hacia el espacio otros dieciséis kilómetros más. Si el ozono que impregna esta región se encontrara en las condiciones que prevalecen en la superficie de la Tierra, formaría una capa de sólo dos milímetros y medio de espesor; sin embargo, esta cantidad de ozono relativamente pequeña y aparentemente insignificante protege a los habitantes de Urantia del exceso de estas radiaciones ultravioletas, peligrosas y destructivas, que están presentes en la luz del Sol. Pero si esta capa de ozono fuera un poquito más espesa, estaríais privados de esos rayos ultravioletas extremadamente importantes y saludables que llegan actualmente hasta la superficie de la Tierra, y que son primordiales para la formación de una de vuestras vitaminas más necesarias.

\par
%\textsuperscript{(665.6)}
\textsuperscript{58:2.3} No obstante, algunos de vuestros mecanicistas humanos menos imaginativos persisten en considerar la creación material y la evolución humana como un accidente. Los intermedios de Urantia han reunido más de cincuenta mil hechos físicos y químicos que estiman que son incompatibles con las leyes del azar y que, según afirman, demuestran de manera inequívoca la presencia de un propósito inteligente en la creación material. Y todo esto no tiene en cuenta su catálogo de más de cien mil hallazgos ajenos al campo de la física y la química que, según mantienen, prueban la presencia de una mente en la planificación, la creación y el mantenimiento del cosmos material.

\par
%\textsuperscript{(666.1)}
\textsuperscript{58:2.4} Vuestro Sol derrama un verdadero diluvio de rayos mortíferos, y vuestra agradable vida en Urantia se debe a la influencia <<fortuita>> de más de cuarenta actividades protectoras, aparentemente casuales, similares a la acción de esta capa de ozono única.

\par
%\textsuperscript{(666.2)}
\textsuperscript{58:2.5} Si no fuera por el efecto <<invernadero>> de la atmósfera durante la noche, el calor se perdería por radiación con tanta rapidez que sería imposible mantener la vida sin disposiciones artificiales.

\par
%\textsuperscript{(666.3)}
\textsuperscript{58:2.6} Los ocho o diez primeros kilómetros de la atmósfera terrestre constituyen la troposfera; esta es la región de los vientos y de las corrientes de aire que producen los fenómenos meteorológicos. Por encima de esta región se encuentra la ionosfera interior e inmediatamente por encima de ésta, la estratosfera. Subiendo desde la superficie de la Tierra, la temperatura disminuye continuamente durante diez o trece kilómetros, y a esta altura se registran alrededor de 57 grados C. bajo cero. Esta gama de temperaturas entre
54 y 57 grados C. bajo cero permanece invariable a medida que se suben sesenta y cuatro kilómetros más; esta zona de temperatura constante es la estratosfera. A una altura de setenta y dos u ochenta kilómetros, la temperatura empieza a elevarse, y este aumento continúa hasta el nivel en que se despliegan las auroras, donde se alcanza una temperatura de 650 grados C., y este calor intenso es el que ioniza el oxígeno. Pero la temperatura en una atmósfera tan enrarecida apenas se puede comparar con la estimación del calor en la superficie de la Tierra. Tened presente que la mitad de toda vuestra atmósfera se encuentra en los primeros cinco kilómetros. Las fajas de luz más altas de las auroras ---a unos seiscientos cuarenta kilómetros--- indican el punto culminante de la atmósfera de la Tierra.

\par
%\textsuperscript{(666.4)}
\textsuperscript{58:2.7} Los fenómenos de las auroras están directamente relacionados con las manchas del Sol, esos ciclones solares que giran en direcciones opuestas por encima y por debajo del ecuador del Sol, al igual que lo hacen los huracanes tropicales terrestres. Estas perturbaciones atmosféricas giran en sentidos contrarios según se produzcan por encima o por debajo del ecuador.

\par
%\textsuperscript{(666.5)}
\textsuperscript{58:2.8} El poder que poseen las manchas solares para alterar las frecuencias de la luz demuestra que estos centros de tormentas solares funcionan como enormes imanes. Estos campos magnéticos son capaces de lanzar las partículas cargadas desde los cráteres de las manchas solares, y a través del espacio, hasta la atmósfera exterior de la Tierra, donde su influencia ionizadora produce las espectaculares manifestaciones de las auroras. Por esta razón, los fenómenos de las auroras más espléndidas se producen cuando las manchas solares están en su apogeo ---o poco tiempo después---, en aquellos momentos en que las manchas están situadas generalmente más cerca del ecuador.

\par
%\textsuperscript{(666.6)}
\textsuperscript{58:2.9} Incluso la aguja de la brújula es sensible a esta influencia solar, ya que se inclina ligeramente hacia el este cuando sale el Sol, y un poco hacia el oeste cuando el Sol está a punto de ponerse. Esto sucede todos los días, pero durante el apogeo de los ciclos de las manchas solares, esta variación de la brújula es dos veces mayor. Estas desviaciones diurnas de la brújula se producen en respuesta a la creciente ionización de la atmósfera superior, producida por la luz solar.

\par
%\textsuperscript{(666.7)}
\textsuperscript{58:2.10} La presencia de dos niveles diferentes de regiones conductoras electrizadas en la superestratosfera es la que explica la transmisión a larga distancia de vuestras emisiones de radio en onda corta y larga. Las terribles tormentas que rugen de vez en cuando en las zonas de estas ionosferas exteriores perturban algunas veces vuestras transmisiones.

\section*{3. El entorno espacial}
\par
%\textsuperscript{(666.8)}
\textsuperscript{58:3.1} Durante los primeros tiempos de la materialización de un universo, las regiones del espacio están salpicadas de inmensas nubes de hidrógeno muy semejantes a los cúmulos astronómicos de polvo que caracterizan actualmente muchas regiones de todo el espacio lejano. Una gran parte de la materia organizada que los soles llameantes descomponen y dispersan en forma de energía radiante, se fabricaba originalmente en estas nubes espaciales de hidrógeno. En ciertas condiciones poco frecuentes, la desintegración de los átomos también se produce en el núcleo de las masas de hidrógeno más grandes. Y tal como sucede en las nebulosas extremadamente calientes, todos estos fenómenos de formación y de disolución atómica van seguidos de la aparición de una oleada torrencial de rayos espaciales cortos de energía radiante. Estas diversas radiaciones van acompañadas de una forma de energía espacial desconocida en Urantia.

\par
%\textsuperscript{(667.1)}
\textsuperscript{58:3.2} Esta carga energética de rayos cortos del espacio universal es cuatrocientas veces mayor que todas las demás formas de energía radiante que existen en los dominios organizados del espacio. La producción de rayos espaciales cortos, ya procedan de las nebulosas llameantes, de los campos eléctricos de alta tensión, del espacio exterior o de las inmensas nubes de polvo compuestas de hidrógeno, es modificada cualitativa y cuantitativamente por las fluctuaciones y los cambios repentinos de tensión en la temperatura, la gravedad y las presiones electrónicas.

\par
%\textsuperscript{(667.2)}
\textsuperscript{58:3.3} Estas eventualidades en el origen de los rayos espaciales están determinadas por muchos sucesos cósmicos así como por las órbitas de la materia circulante, cuyas formas varían desde los círculos modificados hasta las elipses extremadamente alargadas. Las condiciones físicas también pueden estar enormemente alteradas debido a que los electrones giran a veces en sentido contrario al del comportamiento de la materia más densa, incluso en la misma zona física.

\par
%\textsuperscript{(667.3)}
\textsuperscript{58:3.4} Las inmensas nubes de hidrógeno son verdaderos laboratorios químicos del cosmos, y albergan todas las fases de la energía en evolución y de la materia en metamorfosis. También se producen grandes actividades energéticas en los gases marginales de las grandes estrellas binarias, los cuales se superponen con mucha frecuencia y, por lo tanto, se mezclan ampliamente. Pero ninguna de estas enormes y extensas actividades energéticas del espacio ejerce la menor influencia sobre los fenómenos de la vida organizada ---el plasma germinal de las criaturas y de los seres vivos. Estas condiciones energéticas del espacio guardan relación con el entorno necesario para el establecimiento de la vida, pero no tienen efecto sobre las modificaciones posteriores de los factores hereditarios del plasma germinal, como sí lo tienen algunos rayos más largos de la energía radiante. La vida implantada por los Portadores de Vida resiste plenamente todo este asombroso torrente de rayos espaciales cortos de la energía universal.

\par
%\textsuperscript{(667.4)}
\textsuperscript{58:3.5} Todas estas condiciones cósmicas esenciales tenían que evolucionar hacia un estado favorable antes de que los Portadores de Vida pudieran empezar realmente a establecer la vida en Urantia.

\section*{4. La era de los albores de la vida}
\par
%\textsuperscript{(667.5)}
\textsuperscript{58:4.1} El hecho de que nos llamemos Portadores de Vida no debe confundiros. Podemos llevar la vida hasta los planetas y lo hacemos, pero no trajimos ninguna vida hasta Urantia. La vida de Urantia es única, y tiene su origen en este planeta. Esta esfera es un mundo de modificación de la vida; toda la vida que ha aparecido sobre ella la formulamos aquí mismo en el planeta; y no hay ningún otro mundo en todo Satania, ni siquiera en todo Nebadon, donde la vida exista de una manera exactamente igual a la de Urantia\footnote{\textit{La era de los albores de la vida}: Gn 1:11-13,20-27.}.

\par
%\textsuperscript{(667.6)}
\textsuperscript{58:4.2} Hace \textit{550.000.000} de años, el cuerpo de Portadores de Vida regresó a Urantia. En cooperación con los poderes espirituales y las fuerzas superfísicas, organizamos e iniciamos los modelos originales de vida de este mundo, y los plantamos en las aguas hospitalarias del planeta\footnote{\textit{La vida comenzó en el agua}: Gn 1:20-21.}. Toda la vida planetaria (a excepción de las personalidades extraplanetarias) que existió hasta los tiempos de Caligastia, el Príncipe Planetario, tuvo su origen en nuestras tres implantaciones de vida marina, originales, idénticas y simultáneas. Estas tres implantaciones de vida han sido denominadas como sigue: la \textit{central} o eurasiático-africana, la \textit{oriental} o australasiática, y la \textit{occidental}, que incluía a Groenlandia y las Américas.

\par
%\textsuperscript{(668.1)}
\textsuperscript{58:4.3} Hace \textit{500.000.000} de años, la vida vegetal marina primitiva estaba bien establecida en Urantia. Groenlandia y la masa de tierra ártica, así como América del Norte y del Sur, empezaban su larga y lenta deriva hacia el oeste. África se desplazaba ligeramente hacia el sur, creando una depresión este-oeste, la cuenca del Mediterráneo, entre ella misma y el continente madre. La Antártida, Australia y la tierra indicada por las islas del Pacífico se separaron por el sur y el este, y desde entonces se han alejado considerablemente.

\par
%\textsuperscript{(668.2)}
\textsuperscript{58:4.4} Habíamos plantado la forma primitiva de la vida marina en las bahías tropicales abrigadas de los mares centrales de la hendidura este-oeste de la masa continental en vías de romperse. Al hacer las tres implantaciones de vida marina, nuestro objetivo era asegurarnos de que cada una de estas grandes masas de tierra se llevaría consigo esta vida, en sus cálidas aguas marinas, cuando las tierras se separaran posteriormente. Preveíamos que durante la era siguiente, cuando apareciera la vida terrestre, los grandes océanos separarían estas masas continentales a la deriva.

\section*{5. La deriva continental}
\par
%\textsuperscript{(668.3)}
\textsuperscript{58:5.1} La deriva continental continuaba. El núcleo de la Tierra se había vuelto tan denso y rígido como el acero, pues estaba sometido a una presión de unas 3.600 toneladas por centímetro cuadrado, y debido a la enorme presión de la gravedad, estaba y continúa estando muy caliente en las profundidades. La temperatura aumenta desde la superficie hacia abajo, hasta que en el centro es ligeramente superior a la temperatura superficial del Sol.

\par
%\textsuperscript{(668.4)}
\textsuperscript{58:5.2} Los mil seiscientos kilómetros exteriores de la masa terrestre están compuestos principalmente de diferentes clases de roca. Debajo se encuentran los elementos metálicos más densos y pesados. A lo largo de las épocas preatmosféricas primitivas, el mundo estaba tan cerca de ser fluido en su estado fundido y extremadamente caliente, que los metales más pesados se hundieron profundamente en el interior. Aquellos que hoy se encuentran cerca de la superficie representan el exudado de antiguos volcanes, de las grandes corrientes posteriores de lava y de los depósitos meteóricos más recientes.

\par
%\textsuperscript{(668.5)}
\textsuperscript{58:5.3} La corteza exterior tenía un espesor de unos sesenta y cinco kilómetros. Este caparazón exterior estaba sostenido por un mar de basalto fundido de un espesor variable, y descansaba directamente sobre él. Esta capa móvil de lava fundida se mantenía a alta presión, pero siempre tendía a fluir por aquí y por allá para equilibrar las presiones planetarias cambiantes, tendiendo así a estabilizar la corteza terrestre.

\par
%\textsuperscript{(668.6)}
\textsuperscript{58:5.4} Incluso hoy en día, los continentes continúan flotando sobre el cojín no cristalizado de este mar de basalto fundido. Si no existiera esta circunstancia protectora, los terremotos más fuertes sacudirían literalmente al mundo hasta hacerlo pedazos. El deslizamiento y los desplazamientos de la corteza sólida exterior son los que producen los terremotos, y no los volcanes.

\par
%\textsuperscript{(668.7)}
\textsuperscript{58:5.5} Las capas de lava de la corteza terrestre, una vez enfriadas, forman el granito. La densidad media de Urantia es un poco superior a cinco veces y media la del agua; la densidad del granito es casi tres veces superior a la del agua, y el núcleo de la Tierra es doce veces más denso que el agua.

\par
%\textsuperscript{(668.8)}
\textsuperscript{58:5.6} Los fondos marinos son más densos que las masas terrestres, y esto es lo que mantiene a los continentes por encima del agua. Cuando los fondos marinos son empujados por encima del nivel del mar, se descubre que están compuestos en su mayor parte de basalto, una forma de lava considerablemente más densa que el granito de las masas terrestres. Así pues, si los continentes no fueran más ligeros que el fondo de los océanos, la gravedad subiría el borde de los océanos por encima de la tierra, pero no se observa que ocurra este fenómeno.

\par
%\textsuperscript{(668.9)}
\textsuperscript{58:5.7} El peso de los océanos es también un factor que contribuye a aumentar la presión sobre el fondo de los mares. Los fondos oceánicos más bajos pero comparativamente más pesados, más el peso del agua que los cubre, tienen un peso que se aproxima al de los continentes, que son más altos pero mucho más ligeros. No obstante, todos los continentes tienden a deslizarse dentro de los océanos. La presión continental al nivel del fondo del océano es alrededor de 1.300 kilogramos por centímetro cuadrado. Es decir, ésta sería la presión de una masa continental que se elevara a 5.000 metros por encima del fondo del océano. La presión del agua en el fondo oceánico sólo es de unos 350 kilogramos por centímetro cuadrado. Estas presiones diferenciales tienden a hacer que los continentes se deslicen hacia el fondo de los océanos.

\par
%\textsuperscript{(669.1)}
\textsuperscript{58:5.8} El hundimiento del fondo del océano durante las épocas anteriores a la vida había elevado una masa continental solitaria hasta tal altura, que la presión lateral tendió a hacer que los bordes orientales, occidentales y meridionales se deslizaran cuesta abajo sobre los lechos subyacentes semiviscosos de lava, hasta las aguas circundantes del Océano Pacífico. Esto compensó tan plenamente la presión continental que no se produjo una amplia ruptura en la orilla oriental de este antiguo continente asiático, pero desde entonces, este litoral oriental se quedó suspendido sobre el precipicio de las profundidades oceánicas contiguas, amenazando con deslizarse hacia una tumba marina.

\section*{6. El período de transición}
\par
%\textsuperscript{(669.2)}
\textsuperscript{58:6.1} Hace \textit{450.000.000} de años se produjo la \textit{transición de la vida vegetal a la vida animal}. Esta metamorfosis tuvo lugar en las aguas poco profundas de las bahías y las lagunas tropicales abrigadas, situadas en los extensos litorales de los continentes que se estaban separando. Esta evolución, enteramente inherente a los modelos originales de vida, se produjo paulatinamente. Hubo muchas etapas de transición entre las formas primitivas iniciales de la vida vegetal y los organismos animales posteriores bien definidos\footnote{\textit{De la flora a la fauna}: Gn 1:9-12,20-25.}. Hoy sobreviven todavía los mohos de limo de la transición, y difícilmente se les puede clasificar como plantas o como animales.

\par
%\textsuperscript{(669.3)}
\textsuperscript{58:6.2} Se puede seguir la pista de la evolución de la vida vegetal a la vida animal, y se han encontrado series escalonadas de plantas y de animales que conducen progresivamente desde los organismos más simples hasta los más complejos y avanzados. Pero no podréis encontrar estos eslabones entre las grandes divisiones del reino animal, ni entre los tipos superiores de animales prehumanos y los hombres de los albores de las razas humanas. Estos supuestos <<eslabones perdidos>> continuarán perdidos para siempre, por la sencilla razón de que nunca han existido.

\par
%\textsuperscript{(669.4)}
\textsuperscript{58:6.3} Especies radicalmente nuevas de vida animal surgen de una era a otra. No evolucionan a consecuencia de la acumulación gradual de pequeñas variaciones, sino que aparecen como tipos de vida nuevos y desarrollados, y aparecen \textit{repentinamente}.

\par
%\textsuperscript{(669.5)}
\textsuperscript{58:6.4} La aparición \textit{repentina} de especies nuevas y de órdenes diversificadas de organismos vivientes es un fenómeno enteramente biológico, estrictamente natural. Estas mutaciones genéticas no tienen nada de sobrenatural.

\par
%\textsuperscript{(669.6)}
\textsuperscript{58:6.5} La vida animal evolucionó cuando los océanos alcanzaron el grado apropiado de salinidad, y fue relativamente sencillo hacer que las aguas salobres circularan por el cuerpo de los animales marinos. Pero cuando los océanos se contrajeron y aumentó considerablemente el porcentaje de sal, estos mismos animales desarrollaron la capacidad de reducir la salinidad de sus fluidos corporales, al igual que los organismos que aprendieron a vivir en el agua dulce adquirieron la capacidad de mantener el grado adecuado de cloruro sódico en sus fluidos corporales mediante técnicas ingeniosas para conservar la sal.

\par
%\textsuperscript{(669.7)}
\textsuperscript{58:6.6} El estudio de los fósiles de la vida marina, incrustados en la roca, revela las primeras luchas de estos organismos primitivos por adaptarse. Las plantas y los animales nunca dejan de hacer estas experiencias de adaptación. El entorno cambia continuamente, y los organismos vivientes siempre están procurando acomodarse a estas fluctuaciones interminables.

\par
%\textsuperscript{(670.1)}
\textsuperscript{58:6.7} El equipamiento fisiológico y la estructura anatómica de todos los nuevos tipos de vida existen como respuesta al funcionamiento de las leyes físicas, pero la dotación posterior de la mente es un don de los espíritus ayudantes de la mente de acuerdo con la capacidad innata del cerebro. Aunque la mente no proviene de la evolución física, depende por completo de la capacidad cerebral proporcionada por los desarrollos puramente físicos y evolutivos.

\par
%\textsuperscript{(670.2)}
\textsuperscript{58:6.8} A través de unos ciclos casi interminables de ganancias y pérdidas, de adaptaciones y readaptaciones, todos los organismos vivientes oscilan hacia adelante y hacia atrás de época en época. Los que alcanzan la unidad cósmica perduran, mientras que los que no consiguen esta meta dejan de existir.

\section*{7. El libro de la historia geológica}
\par
%\textsuperscript{(670.3)}
\textsuperscript{58:7.1} El amplio grupo de sistemas rocosos que constituían la corteza exterior del mundo durante los albores de la vida, o era proterozoica, actualmente no aparece en muchos puntos de la superficie terrestre. Y cuando emerge de la parte inferior de todas las acumulaciones de las épocas posteriores, sólo se encuentran los restos fósiles de la vida vegetal y de la vida animal primitiva inicial. Algunas de estas rocas más antiguas, depositadas por el agua, están mezcladas con estratos posteriores, y a veces revelan restos fósiles de algunas de las formas más iniciales de la vida vegetal, mientras que en los estratos superiores se pueden encontrar ocasionalmente algunas de las formas más primitivas de los primeros organismos animales marinos. En muchos lugares, estas capas rocosas estratificadas muy antiguas, que contienen los fósiles de la vida marina primitiva, tanto animal como vegetal, se pueden encontrar directamente encima de la roca indiferenciada más antigua.

\par
%\textsuperscript{(670.4)}
\textsuperscript{58:7.2} Los fósiles de esta era contienen algas, plantas semejantes al coral, protozoarios primitivos y organismos de transición parecidos a las esponjas. Pero la ausencia de estos fósiles en los estratos rocosos primitivos no prueba necesariamente que los organismos vivientes no existieran en otras partes en el momento en que aquellos se depositaron. La vida estuvo esparcida a lo largo de todos estos tiempos primitivos, y sólo lentamente se fue abriendo camino sobre la faz de la Tierra.

\par
%\textsuperscript{(670.5)}
\textsuperscript{58:7.3} Las rocas de esta antigua época se encuentran ahora en la superficie de la Tierra, o muy cerca de ella, sobre una octava parte aproximadamente de la superficie terrestre actual. El espesor medio de esta piedra de transición, las capas de roca estratificada más antiguas, es aproximadamente de dos kilómetros y medio. En algunos puntos, el espesor de estos antiguos sistemas rocosos alcanza seis kilómetros y medio, pero muchos estratos que se han atribuido a esta era pertenecen a períodos posteriores.

\par
%\textsuperscript{(670.6)}
\textsuperscript{58:7.4} En América del Norte, este estrato antiguo y primitivo de rocas fosilíferas aflora en las regiones orientales, centrales y septentrionales del Canadá. Existe también, de este a oeste, una cresta intermitente de esta roca que se extiende desde Pensilvania y las antiguas Montañas Adirondacks, hacia el oeste a través de Michigan, Wisconsin y Minesota. Otras cordilleras van desde Terranova hasta Alabama y desde Alaska hasta Méjico.

\par
%\textsuperscript{(670.7)}
\textsuperscript{58:7.5} Las rocas de esta era están al descubierto aquí y allá por todo el mundo, pero ninguna es tan fácil de interpretar como las de los alrededores del Lago Superior y del Gran Cañón del Río Colorado, donde estas rocas fosilíferas primitivas, existentes en diversos estratos, dan testimonio de los levantamientos y las fluctuaciones superficiales de aquellos tiempos lejanos.

\par
%\textsuperscript{(670.8)}
\textsuperscript{58:7.6} Esta capa de piedra, el estrato fosilífero más antiguo de la corteza terrestre, ha sido arrugada, plegada y retorcida grotescamente debido a los levantamientos causados por los terremotos y los primeros volcanes. Las corrientes de lava de esta época hicieron subir mucho hierro, cobre y plomo cerca de la superficie planetaria.

\par
%\textsuperscript{(670.9)}
\textsuperscript{58:7.7} Existen pocos lugares en la Tierra donde estas actividades se muestren de una manera más gráfica que en el Valle de Saint Croix, en Wisconsin. En esta región se produjeron ciento veintisiete inundaciones sucesivas de lava sobre una tierra que posteriormente fue sumergida en el agua, con el consiguiente depósito de rocas. Aunque una gran parte de la sedimentación rocosa superior y de los flujos intermitentes de lava están ausentes hoy en día, y aunque el fondo de este sistema está profundamente sepultado en la tierra, sin embargo alrededor de sesenta y cinco o setenta de estos archivos estratificados de las épocas pasadas están ahora expuestos a la vista.

\par
%\textsuperscript{(671.1)}
\textsuperscript{58:7.8} En estas épocas primitivas en las que una gran parte de la tierra estaba cerca del nivel del mar, se produjeron muchas sumersiones y surgimientos sucesivos. La corteza terrestre estaba entrando en su último período de estabilización relativa. Las ondulaciones, levantamientos y depresiones de la deriva continental anterior contribuyeron a la frecuencia de la inmersión periódica de las grandes masas terrestres.

\par
%\textsuperscript{(671.2)}
\textsuperscript{58:7.9} Durante estos tiempos de la vida marina primitiva, grandes superficies de las costas continentales se hundieron en los mares entre unos pocos metros y ochocientos metros de profundidad. Una gran parte de la arenisca y de los conglomerados más viejos representan las acumulaciones sedimentarias de estas riberas antiguas. Las rocas sedimentarias pertenecientes a esta estratificación primitiva descansan directamente sobre unos estratos que datan de mucho antes del origen de la vida, remontándose al principio de la aparición del océano mundial.

\par
%\textsuperscript{(671.3)}
\textsuperscript{58:7.10} Algunos estratos superiores de estos depósitos rocosos de transición contienen pequeñas cantidades de esquistos o pizarras de colores oscuros, que indican la presencia de carbono orgánico y atestiguan la existencia de los antepasados de aquellas formas de vida vegetal que invadieron la tierra durante la era siguiente, la era Carbonífera o del carbón. Una gran parte del cobre de estos estratos rocosos ha sido depositada por las aguas. Alguno se encuentra en las grietas de las rocas más antiguas y proviene de la concentración de las aguas pantanosas estancadas de algún antiguo litoral abrigado. Las minas de hierro de América del Norte y Europa están situadas en los depósitos y extrusiones que reposan en parte en las rocas no estratificadas más antiguas, y en parte en las rocas estratificadas posteriores de los períodos de transición de formación de la vida.

\par
%\textsuperscript{(671.4)}
\textsuperscript{58:7.11} Esta era es testigo de la propagación de la vida por todas las aguas del mundo; la vida marina ha quedado bien establecida en Urantia. Los fondos de los mares interiores, poco profundos y extensos, están siendo invadidos paulatinamente por un crecimiento de la vegetación profuso y exuberante, mientras que en las aguas de los litorales abundan las formas simples de la vida animal.

\par
%\textsuperscript{(671.5)}
\textsuperscript{58:7.12} Toda esta historia está contada de forma gráfica en las páginas fósiles del inmenso <<libro de piedra>> de los anales del mundo. Y las páginas de este gigantesco archivo biogeológico os dirán infaliblemente la verdad con que sólo adquiráis la habilidad de interpretarlas. Muchos de estos antiguos fondos marinos se encuentran ahora muy por encima del nivel de la tierra, y sus depósitos de una era tras otra cuentan la historia de las luchas por la vida durante aquellos tiempos primitivos. Como dijo vuestro poeta, es literalmente cierto que <<El polvo que pisamos estuvo vivo en otro tiempo.>>

\par
%\textsuperscript{(671.6)}
\textsuperscript{58:7.13} [Presentado por un miembro del Cuerpo de Portadores de Vida de Urantia, que reside actualmente en el planeta.]


\chapter{Documento 59. La era de la vida marina en Urantia}
\par
%\textsuperscript{(672.1)}
\textsuperscript{59:0.1} CONSIDERAMOS que la historia de Urantia empezó hace unos mil millones de años y que se extiende a lo largo de cinco eras principales:

\par
%\textsuperscript{(672.2)}
\textsuperscript{59:0.2} 1. \textit{La era anterior a la vida} se extiende sobre los primeros cuatrocientos cincuenta millones de años, desde casi el momento en que el planeta alcanzó su tamaño actual hasta el momento del establecimiento de la vida. Vuestros estudiosos han dado el nombre de \textit{Arqueozoico} a este período.

\par
%\textsuperscript{(672.3)}
\textsuperscript{59:0.3} 2. \textit{La era de los albores de la vida} se extiende sobre los ciento cincuenta millones de años siguientes. Esta época transcurre entre la era precedente anterior a la vida, o era de los cataclismos, y el período siguiente de la vida marina muy desarrollada. Vuestros investigadores conocen esta era con el nombre de \textit{Proterozoica}.

\par
%\textsuperscript{(672.4)}
\textsuperscript{59:0.4} 3. \textit{La era de la vida marina} abarca los doscientos cincuenta millones de años siguientes, y la conocéis mejor con el nombre de \textit{Paleozoica}.

\par
%\textsuperscript{(672.5)}
\textsuperscript{59:0.5} 4. \textit{La era de la vida terrestre primitiva} se extiende sobre los cien millones de años siguientes y se la conoce con el nombre de \textit{Mesozoica}.

\par
%\textsuperscript{(672.6)}
\textsuperscript{59:0.6} 5. \textit{La era de los mamíferos} ocupa los últimos cincuenta millones de años. Esta era de los tiempos recientes es conocida con el nombre de \textit{Cenozoica}.

\par
%\textsuperscript{(672.7)}
\textsuperscript{59:0.7} La era de la vida marina abarca pues alrededor de una cuarta parte de la historia de vuestro planeta. Se la puede subdividir en seis largos períodos, cada uno de ellos caracterizado por ciertos desarrollos bien definidos tanto en el ámbito geológico como en el terreno biológico.

\par
%\textsuperscript{(672.8)}
\textsuperscript{59:0.8} Cuando comienza esta era, los fondos marinos, las grandes plataformas continentales y las numerosas cuencas poco profundas cerca de las costas están cubiertos de una vegetación prolífica. Las formas más simples y primitivas de la vida animal ya se han desarrollado a partir de los organismos vegetales anteriores, y los primeros organismos animales se han abierto camino gradualmente a lo largo de los extensos litorales de las diversas masas terrestres hasta que los numerosos mares interiores están abarrotados de vida marina primitiva. Como muy pocos de estos organismos primitivos tenían conchas, se han conservado muy pocos como fósiles. Sin embargo, la escena está preparada para los primeros capítulos del gran <<libro de piedra>> dedicado a la conservación de los anales de la vida, que las épocas siguientes fueron guardando de manera tan metódica.

\par
%\textsuperscript{(672.9)}
\textsuperscript{59:0.9} El continente de América del Norte posee una riqueza asombrosa en depósitos fosilíferos que abarcan toda la era de la vida marina. Las primeras capas más antiguas están separadas de los estratos más recientes del período anterior por grandes depósitos causados por la erosión, que dividen claramente estas dos etapas del desarrollo planetario.

\section*{1. La vida marina primitiva en los mares poco profundos --- La época de los trilobites}
\par
%\textsuperscript{(673.1)}
\textsuperscript{59:1.1} Al principio de este período de tranquilidad relativa en la superficie de la Tierra, la vida está confinada a los diversos mares interiores y al litoral oceánico; hasta este momento no ha evolucionado ninguna forma de organismo terrestre. Los animales marinos primitivos están bien establecidos y preparados para el próximo desarrollo evolutivo. Las amebas, que habían aparecido hacia el final del período de transición anterior, son las supervivientes simbólicas de esta etapa inicial de la vida animal.

\par
%\textsuperscript{(673.2)}
\textsuperscript{59:1.2} Hace \textit{400.000.000} de años, la vida marina tanto vegetal como animal está bastante bien repartida por el mundo entero. El clima mundial se calienta ligeramente y se vuelve más uniforme. Se produce una inundación general de las costas de los diversos continentes, en particular de América del Norte y del Sur. Aparecen nuevos océanos, y las masas de agua más antiguas se agrandan considerablemente.

\par
%\textsuperscript{(673.3)}
\textsuperscript{59:1.3} La vegetación empieza ahora a trepar por primera vez sobre la tierra firme y no tarda en hacer progresos considerables en su adaptación a un hábitat no marino.

\par
%\textsuperscript{(673.4)}
\textsuperscript{59:1.4} \textit{De repente}, los primeros animales multicelulares hacen su aparición sin que sus antepasados sufrieran cambios graduales. Los trilobites han sido producidos por evolución y dominan los mares durante épocas enteras. Desde el punto de vista de la vida marina, ésta es la era de los trilobites.

\par
%\textsuperscript{(673.5)}
\textsuperscript{59:1.5} Hacia el final de este período de tiempo, una gran parte de América del Norte y de Europa emergió del mar. La corteza terrestre estaba temporalmente estabilizada; las montañas, o más bien unas altas elevaciones de tierra, surgieron a lo largo de las costas del Atlántico y del Pacífico, en las Antillas y en el sur de Europa. Toda la región del Caribe estaba sumamente elevada.

\par
%\textsuperscript{(673.6)}
\textsuperscript{59:1.6} Hace \textit{390.000.000} de años, la tierras continuaban estando elevadas. En algunas partes del este y del oeste de América y de Europa occidental se pueden encontrar los estratos de piedra que se depositaron durante estos tiempos; se trata de las rocas más antiguas que contienen fósiles de trilobites. Estas rocas fosilíferas se depositaron en los numerosos y largos brazos de mar que se adentraban en las masas continentales.

\par
%\textsuperscript{(673.7)}
\textsuperscript{59:1.7} Unos millones de años después, el Océano Pacífico empezó a invadir los continentes americanos. El hundimiento de las tierras se debió principalmente a un ajuste de la corteza, aunque la expansión lateral de las tierras, o deslizamiento continental, fue también una de las causas.

\par
%\textsuperscript{(673.8)}
\textsuperscript{59:1.8} Hace \textit{380.000.000} de años, Asia se estaba sumergiendo y todos los demás continentes experimentaban un surgimiento de corta duración. Pero a medida que avanzaba esta época, el Océano Atlántico recién aparecido hizo grandes incursiones en todos los litorales adyacentes. El Atlántico Norte, o mares árticos, estaba entonces comunicado con las aguas del Golfo meridional. Cuando este mar del sur penetró en la depresión apalache, sus olas se rompieron en el este contra unas montañas tan altas como los Alpes, pero en general los continentes estaban formados de tierras bajas sin interés, totalmente desprovistas de belleza natural.

\par
%\textsuperscript{(673.9)}
\textsuperscript{59:1.9} Los depósitos sedimentarios de estas épocas son de cuatro clases:

\par
%\textsuperscript{(673.10)}
\textsuperscript{59:1.10} 1. Conglomerados ---materiales depositados cerca de los litorales.

\par
%\textsuperscript{(673.11)}
\textsuperscript{59:1.11} 2. Areniscas ---depósitos formados en las aguas poco profundas pero donde había suficientes olas para impedir que se asentara el lodo.

\par
%\textsuperscript{(673.12)}
\textsuperscript{59:1.12} 3. Esquistos ---depósitos formados en unas aguas más profundas y más tranquilas.

\par
%\textsuperscript{(673.13)}
\textsuperscript{59:1.13} 4. Calizas ---incluyen los depósitos de conchas de los trilobites en aguas profundas.

\par
%\textsuperscript{(673.14)}
\textsuperscript{59:1.14} Los fósiles de trilobites de esta época presentan ciertas uniformidades fundamentales unidas a ciertas variaciones bien marcadas. Los animales primitivos que se desarrollaron a partir de las tres implantaciones originales de vida eran característicos; los que aparecieron en el hemisferio occidental eran ligeramente diferentes a los del grupo eurasiático y a los del tipo australasiático o australantártico.

\par
%\textsuperscript{(674.1)}
\textsuperscript{59:1.15} Hace \textit{370.000.000} de años se produjo la gran inmersión casi total de América del Norte y del Sur, seguida por el hundimiento de África y Australia. Sólo algunas partes de América del Norte permanecieron por encima de estos mares cámbricos poco profundos. Cinco millones de años más tarde, los mares se retiraron ante las tierras que se iban elevando. Todos estos fenómenos de hundimientos y levantamientos de tierras estaban exentos de dramatismo, pues se producían lentamente a lo largo de millones de años.

\par
%\textsuperscript{(674.2)}
\textsuperscript{59:1.16} Los estratos fosilíferos de trilobites de esta época afloran aquí y allá por todos los continentes, salvo en Asia central. Estas rocas son horizontales en muchas regiones, pero en las montañas están inclinadas y deformadas a causa de la presión y del plegamiento. En muchos lugares, esta presión ha cambiado el carácter original de estos depósitos. La arenisca se ha transformado en cuarzo, el esquisto ha sido cambiado en pizarra y la caliza se ha convertido en mármol.

\par
%\textsuperscript{(674.3)}
\textsuperscript{59:1.17} Hace \textit{360.000.000} de años, las tierras continuaban levantándose. América del Norte y del Sur se encontraban bien elevadas. Europa occidental y las Islas Británicas estaban emergiendo, a excepción de algunas partes del País de Gales, que se hallaban profundamente sumergidas. Durante estas épocas no había grandes capas de hielo. Los supuestos depósitos glaciales que aparecen relacionados con estos estratos en Europa, África, China y Australia, se deben a los glaciares de montaña aislados o al desplazamiento de detritos glaciales de origen más reciente. El clima mundial era oceánico, no continental. Los mares del sur eran entonces más cálidos que hoy, y se extendían hacia el norte por encima de Norteamérica hasta las regiones polares. La Corriente del Golfo pasaba por la parte central de América del Norte y se desviaba hacia el este para bañar y calentar las costas de Groenlandia, convirtiendo este continente, ahora cubierto por un manto de hielo, en un verdadero paraíso tropical.

\par
%\textsuperscript{(674.4)}
\textsuperscript{59:1.18} La vida marina era muy semejante en todo el mundo y consistía en algas marinas, organismos unicelulares, esponjas simples, trilobites y otros crustáceos ---camarones, cangrejos y langostas. Tres mil variedades de braquiópodos aparecieron al final de este período, de las cuales sólo han sobrevivido doscientas. Estos animales representan una variedad de la vida primitiva que ha llegado hasta la época actual prácticamente sin cambios.

\par
%\textsuperscript{(674.5)}
\textsuperscript{59:1.19} Pero los trilobites eran las criaturas vivientes dominantes. Eran animales sexuados y existían en muchas formas; como eran malos nadadores, flotaban perezosamente en el agua o se arrastraban por los fondos marinos, y se enroscaban para protegerse contra los ataques de sus enemigos que aparecieron más tarde. Alcanzaban una longitud entre cinco y treinta centímetros y se desarrollaron en cuatro grupos distintos: carnívoros, herbívoros, omnívoros y <<comedores de lodo>>. La capacidad de este último grupo para alimentarse ampliamente de materia inorgánica ---fueron los últimos animales multicelulares que pudieron hacerlo--- explica su gran multiplicación y su larga supervivencia.

\par
%\textsuperscript{(674.6)}
\textsuperscript{59:1.20} Éste era el cuadro biogeológico de Urantia al final de aquel largo período de la historia del mundo, que abarcó cincuenta millones de años, y que vuestros geólogos han denominado \textit{Cámbrico}.

\section*{2. La etapa de la primera inundación continental --- La época de los animales invertebrados}
\par
%\textsuperscript{(674.7)}
\textsuperscript{59:2.1} Los fenómenos periódicos de elevación y hundimiento de las tierras, característicos de estos tiempos, se producían todos de manera paulatina y sin ninguna espectacularidad, pues iban acompañados de poca o de ninguna actividad volcánica. Durante todas estas elevaciones y depresiones terrestres sucesivas, el continente asiático madre no compartió por completo la historia de las otras masas de tierra. Experimentó muchas inundaciones, sumergiéndose primero por un lado y luego por el otro, sobre todo durante su historia primitiva, pero no presenta los depósitos rocosos uniformes que se pueden descubrir en los otros continentes. En las épocas recientes, Asia ha sido la más estable de todas las masas terrestres.

\par
%\textsuperscript{(675.1)}
\textsuperscript{59:2.2} Hace \textit{350.000.000} de años se pudo observar el principio del período de las grandes inundaciones de todos los continentes, salvo Asia central. Las masas terrestres quedaron cubiertas repetidas veces por el agua; sólo las tierras altas de la costa permanecieron por encima de estos mares interiores oscilantes poco profundos pero extendidos. Este período estuvo caracterizado por tres inundaciones de gran importancia, pero antes de que terminara, los continentes subieron de nuevo, y el total de las tierras emergidas llegó a ser un quince por ciento mayor que en la actualidad. La región del Caribe estaba muy elevada. Este período no se distingue bien en Europa porque las fluctuaciones terrestres fueron menores, mientras que la actividad volcánica fue más continua.

\par
%\textsuperscript{(675.2)}
\textsuperscript{59:2.3} Hace \textit{340.000.000} de años se produjo otro extenso hundimiento terrestre, excepto en Asia y Australia. Las aguas de los océanos del mundo estaban mezcladas en general. Ésta fue la gran época de la piedra caliza; una gran parte de esta piedra fue depositada por las algas secretoras de cal.

\par
%\textsuperscript{(675.3)}
\textsuperscript{59:2.4} Algunos millones de años más tarde, grandes zonas de los continentes americanos y de Europa empezaron a emerger de las aguas. En el hemisferio occidental, sólo un brazo del Océano Pacífico permanecía sobre Méjico y las regiones actuales de las Montañas Rocosas, pero hacia el final de esta época, las costas del Atlántico y del Pacífico empezaron de nuevo a sumergirse.

\par
%\textsuperscript{(675.4)}
\textsuperscript{59:2.5} Hace \textit{330.000.000} de años se observa el comienzo de un período de tranquilidad relativa en todo el mundo, con muchas tierras de nuevo por encima del agua. La única excepción que hubo durante este reinado de tranquilidad terrestre fue la erupción del gran volcán norteamericano al este de Kentucky, una de las actividades volcánicas aisladas más grandes que el mundo haya conocido jamás. Las cenizas de este volcán cubrieron mil trescientos kilómetros cuadrados, con una profundidad entre cinco y seis metros.

\par
%\textsuperscript{(675.5)}
\textsuperscript{59:2.6} Hace \textit{320.000.000} de años se produjo la tercera inundación de gran importancia de este período. Las aguas de esta inundación cubrieron todas las tierras sumergidas por el diluvio anterior, y se extendieron además en muchas direcciones por todas las Américas y Europa. El este de Norteamérica y Europa occidental se encontraron entre 3.000 y 4.500 metros por debajo del agua.

\par
%\textsuperscript{(675.6)}
\textsuperscript{59:2.7} Hace \textit{310.000.000} de años, las masas terrestres del mundo se hallaban de nuevo bien elevadas, a excepción de las partes meridionales de América del Norte. Méjico emergió, creando así el Mar del Golfo, que desde entonces ha conservado siempre su identidad.

\par
%\textsuperscript{(675.7)}
\textsuperscript{59:2.8} La vida continúa evolucionando durante este período. Una vez más, el mundo está tranquilo y relativamente apacible; el clima sigue siendo templado y uniforme; las plantas terrestres van emigrando cada vez más lejos de los litorales. Los modelos de vida están bien desarrollados, aunque pocos fósiles vegetales de estos tiempos se puedan encontrar.

\par
%\textsuperscript{(675.8)}
\textsuperscript{59:2.9} Ésta fue la gran época de la evolución de los organismos animales individuales, aunque muchos cambios fundamentales, tales como la transición de la planta al animal, se habían producido anteriormente. La fauna marina se desarrolló hasta el punto de que todos los tipos de vida inferiores a los vertebrados estuvieron representados en los fósiles de las rocas que se depositaron durante estos tiempos. Pero todos estos animales eran organismos marinos. Ningún animal terrestre había aparecido todavía, excepto algunos tipos de gusanos que excavaban la tierra a lo largo de las costas, y las plantas terrestres aún no se habían extendido sobre los continentes; había todavía demasiado dióxido de carbono en el aire como para permitir la existencia de los respiradores de aire. Principalmente, todos los animales, excepto algunos de los más primitivos, dependen directa o indirectamente de la vida vegetal para existir.

\par
%\textsuperscript{(676.1)}
\textsuperscript{59:2.10} Los trilobites seguían predominando. Estos pequeños animales existían en decenas de miles de especies, y fueron los predecesores de los crustáceos modernos. Algunos trilobites tenían entre veinticinco y cuatro mil ojos minúsculos, y otros tenían ojos malogrados. Al final de este período, los trilobites compartían el dominio de los mares con otras diversas formas de la vida invertebrada, pero perecieron por completo al principio del período siguiente.

\par
%\textsuperscript{(676.2)}
\textsuperscript{59:2.11} Las algas que secretaban cal estaban muy extendidas. Existían miles de especies de los antepasados primitivos de los corales. Abundaban los gusanos de mar y había muchas variedades de medusas que se han extinguido desde entonces. Evolucionaron los corales y los tipos más recientes de esponjas. Los cefalópodos estaban bien desarrollados y han sobrevivido en los nautilos, los pulpos, las jibias y los calamares de los tiempos modernos.

\par
%\textsuperscript{(676.3)}
\textsuperscript{59:2.12} Había muchas variedades de animales con conchas, pero entonces no las necesitaban tanto para defenderse como en las épocas siguientes. Los gasterópodos estaban presentes en las aguas de los mares antiguos, e incluían a los perforadores de una sola concha, los bígaros y los caracoles. Los gasterópodos bivalvos han atravesado los millones de años intermedios hasta llegar a nuestros días casi como existían entonces, y engloban a los mejillones, las almejas, las ostras y las veneras. Los organismos con concha de valva evolucionaron también, y estos braquiópodos vivieron en aquellas aguas antiguas poco más o menos como existen hoy; sus valvas estaban provistas incluso de charnelas, de muescas y de otros tipos de dispositivos protectores.

\par
%\textsuperscript{(676.4)}
\textsuperscript{59:2.13} Así termina la historia evolutiva del segundo gran período de la vida marina, que vuestros geólogos conocen con el nombre de \textit{Ordovícico}.

\section*{3. La etapa de la segunda gran inundación --- El período del coral --- La época de los braquiópodos}
\par
%\textsuperscript{(676.5)}
\textsuperscript{59:3.1} Hace \textit{300.000.000} de años empezó otro gran período de inmersión de las tierras. El avance gradual de los antiguos mares silúricos hacia el norte y el sur los preparó para sumergir la mayor parte de Europa y América del Norte. Las tierras no estaban muy elevadas por encima del nivel del mar, de manera que no se produjeron muchos depósitos cerca de los litorales. Los mares rebosaban de vida con conchas calizas, y la caída de estas conchas hasta el fondo del mar fue formando gradualmente unas capas calcáreas muy espesas. Éste fue el primer depósito calcáreo ampliamente extendido, y cubre prácticamente toda Europa y América del Norte, pero sólo aparece en algunas partes de la superficie terrestre. El espesor medio de esta antigua capa rocosa es aproximadamente de trescientos metros, pero una gran parte de estos depósitos ha sido enormemente deformada desde entonces por las inclinaciones, los levantamientos y las fallas, y muchos se han transformado en cuarzo, en esquisto y en mármol.

\par
%\textsuperscript{(676.6)}
\textsuperscript{59:3.2} No se encuentran ni rocas ígneas ni lavas en las capas rocosas de este período, salvo las de los grandes volcanes del sur de Europa y del este de Maine, y los flujos de lava de Quebec. La actividad volcánica prácticamente había terminado. Éste fue el apogeo de los grandes depósitos marinos; se formaron pocas o ninguna cadena montañosa.

\par
%\textsuperscript{(676.7)}
\textsuperscript{59:3.3} Hace \textit{290.000.000} de años, el mar se había retirado ampliamente de los continentes, y los fondos de los océanos circundantes se estaban hundiendo. Las masas terrestres habían cambiado poco hasta que se sumergieron de nuevo. Los primeros movimientos montañosos estaban empezando en todos los continentes, y los levantamientos más importantes de la corteza fueron los Himalayas en Asia y las grandes Montañas de Caledonia, que se extienden desde Irlanda hasta Spitzbergen, pasando por Escocia.

\par
%\textsuperscript{(677.1)}
\textsuperscript{59:3.4} Una gran parte del gas, el petróleo, el zinc y el plomo se encuentran en los depósitos de esta época; el gas y el petróleo proceden de las enormes acumulaciones de materia vegetal y animal que se depositaron durante la inmersión terrestre anterior, mientras que los depósitos minerales representan la sedimentación de masas de agua en calma. Muchos depósitos de sal gema corresponden a este período.

\par
%\textsuperscript{(677.2)}
\textsuperscript{59:3.5} Los trilobites declinaron rápidamente y los moluscos más grandes, o cefalópodos, pasaron a ocupar el primer plano. Estos animales alcanzaban un tamaño de cinco metros de largo por treinta centímetros de diámetro, y se convirtieron en los dueños de los mares. Esta especie animal apareció \textit{repentinamente} y se hizo con el dominio de la vida marina.

\par
%\textsuperscript{(677.3)}
\textsuperscript{59:3.6} La gran actividad volcánica de esta época tuvo lugar en la zona europea. Desde hacía millones y millones de años no se habían producido unas erupciones volcánicas tan violentas y extensas como las que sucedieron ahora alrededor de la depresión del Mediterráneo, sobre todo en las cercanías de las Islas Británicas. Este flujo de lava sobre la región de las Islas Británicas aparece actualmente bajo la forma de capas alternas de lava y de roca con un espesor de unos 8.000 metros. Estas rocas fueron depositadas por las corrientes intermitentes de lava que se esparcieron sobre un lecho marino poco profundo, entremezclando así los depósitos de roca, y todo esto se elevó posteriormente a una gran altura sobre el nivel del mar. En el norte de Europa se produjeron violentos terremotos, particularmente en Escocia.

\par
%\textsuperscript{(677.4)}
\textsuperscript{59:3.7} El clima oceánico seguía siendo suave y uniforme, y los mares calientes bañaban las costas de las tierras polares. Los fósiles de los braquiópodos y de otras formas de vida marina se pueden encontrar en estos depósitos hasta en el mismo Polo Norte. Los gasterópodos, braquiópodos, esponjas y corales formadores de arrecifes continuaron aumentando.

\par
%\textsuperscript{(677.5)}
\textsuperscript{59:3.8} El final de esta época es testigo del segundo avance de los mares silúricos y de una nueva mezcla de las aguas oceánicas del norte y del sur. Los cefalópodos dominan la vida marina, mientras que las formas de vida asociadas se desarrollan y se diferencian progresivamente.

\par
%\textsuperscript{(677.6)}
\textsuperscript{59:3.9} Hace \textit{280.000.000} de años, los continentes habían emergido en gran parte de la segunda inundación silúrica. Los depósitos rocosos de esta inmersión se conocen en América del Norte con el nombre de calizas del Niágara, porque las Cataratas del Niágara fluyen actualmente sobre el estrato de esta roca. Esta capa rocosa se extiende desde las montañas del este hasta la región del valle del Misisipí, pero no hacia el oeste de esta región sino hacia el sur. Varias capas se extienden sobre Canadá, zonas de América del Sur, Australia y la mayor parte de Europa; el espesor medio de esta serie de capas del Niágara es de unos doscientos metros. En muchas regiones se pueden encontrar, inmediatamente por encima de estos depósitos de tipo Niágara, un conjunto de conglomerados, esquistos y sal gema. Se trata de la acumulación de asentamientos secundarios. Esta sal se asentó en grandes lagunas que estuvieron abiertas alternativamente hacia el mar, y luego fueron separadas de él, de manera que la evaporación produjo los depósitos de sal junto con otras materias que estaban disueltas en el agua. En algunas regiones, estos lechos de sal gema tienen un espesor de veinte metros.

\par
%\textsuperscript{(677.7)}
\textsuperscript{59:3.10} El clima es suave y moderado, y los fósiles marinos se depositan en las regiones árticas. Pero al final de esta época, los mares están tan extremadamente salados que poca vida puede sobrevivir.

\par
%\textsuperscript{(677.8)}
\textsuperscript{59:3.11} Hacia el final de la última inmersión silúrica, los equinodermos ---los lirios de mar--- aumentan considerablemente, tal como lo demuestran los depósitos calcáreos crinoideos. Los trilobites casi han desaparecido, y los moluscos continúan siendo los reyes de los mares; la formación de arrecifes de coral se incrementa enormemente. Durante esta época, los escorpiones acuáticos primitivos evolucionan por primera vez en los lugares más favorables. Poco después, los auténticos escorpiones ---los verdaderos respiradores de aire--- hacen su aparición \textit{repentinamente}.

\par
%\textsuperscript{(678.1)}
\textsuperscript{59:3.12} Estos progresos ponen fin al tercer período de la vida marina, que abarca veinticinco millones de años y que vuestros investigadores conocen con el nombre de \textit{Silúrico}.

\section*{4. La etapa del gran surgimiento de las tierras --- El período de la vida terrestre vegetal --- La época de los peces }
\par
%\textsuperscript{(678.2)}
\textsuperscript{59:4.1} En el transcurso de la lucha secular entre la tierra y el agua, los mares han ganado relativamente la batalla durante largos períodos, pero la hora de la victoria de la tierra está a punto de llegar. Las derivas continentales no han avanzado tanto y, a veces, prácticamente todas las tierras del mundo están conectadas por medio de delgados istmos y de estrechos puentes terrestres.

\par
%\textsuperscript{(678.3)}
\textsuperscript{59:4.2} Cuando las tierras emergen de la última inundación silúrica, un importante período del desarrollo del mundo y de la evolución de la vida llega a su fin. Es el principio de una nueva época en la Tierra. El paisaje desnudo y sin atractivo de los tiempos pasados empieza a vestirse con un verdor exuberante, y los primeros bosques espléndidos están a punto de aparecer.

\par
%\textsuperscript{(678.4)}
\textsuperscript{59:4.3} La vida marina de esta época era muy variada debido a la separación de las primeras especies, pero más adelante todos estos diversos tipos se mezclaron y se asociaron libremente. Los braquiópodos alcanzaron pronto su apogeo, luego les sucedieron los artrópodos, y los percebes aparecieron por primera vez. Pero el acontecimiento más grande de todos fue la aparición repentina de la familia de los peces. Esta época se convirtió en la era de los peces, ese período de la historia del mundo caracterizado por los tipos de animales \textit{vertebrados}.

\par
%\textsuperscript{(678.5)}
\textsuperscript{59:4.4} Hace \textit{270.000.000} de años, todos los continentes estaban por encima del agua. Desde hacía millones y millones de años, nunca había habido tantas tierras por encima del agua al mismo tiempo; fue una de las épocas más grandes de emergencia de tierras en toda la historia del mundo.

\par
%\textsuperscript{(678.6)}
\textsuperscript{59:4.5} Cinco millones de años después, las superficies de América del Norte y del Sur, Europa, África, el norte de Asia y Australia se inundaron durante corto tiempo; en uno u otro momento, la inmersión de América del Norte fue casi completa, y las capas calcáreas resultantes tienen un espesor que varía entre 150 y 1.500 metros. Estos diversos mares devonianos se extendieron primero en una dirección, y luego en otra, de manera que el inmenso mar interior ártico de América del Norte encontró una salida hacia el Océano Pacífico a través del norte de California.

\par
%\textsuperscript{(678.7)}
\textsuperscript{59:4.6} Hace \textit{260.000.000} de años, hacia el final de esta época de depresión terrestre, América del Norte estaba parcialmente cubierta por unos mares que se comunicaban simultáneamente con las aguas del Pacífico, del Atlántico, del Ártico y del Golfo. Los depósitos de estas etapas más recientes de la primera inundación devoniana tienen un espesor medio de unos trescientos metros. Los arrecifes de coral que caracterizan esta época indican que los mares interiores eran transparentes y poco profundos. Estos depósitos de coral están puestos al descubierto en las orillas del río Ohio, cerca de Louisville
(Kentucky), y tienen aproximadamente treinta metros de espesor, abarcando más de doscientas variedades. Estas formaciones coralinas se extienden a través del Canadá y el norte de Europa hasta las regiones árticas.

\par
%\textsuperscript{(678.8)}
\textsuperscript{59:4.7} Después de estas inmersiones, una gran parte de los litorales se elevó considerablemente, de manera que los depósitos primitivos fueron cubiertos de lodo o esquisto. También existe un estrato de arenisca roja que caracteriza una de las sedimentaciones devonianas, y esta capa roja se extiende por una gran parte de la superficie de la Tierra, encontrándose en América del Norte y del Sur, Europa, Rusia, China, África y Australia. Estos depósitos rojos evocan unas condiciones áridas o semiáridas, pero el clima de esta época continuó siendo templado y uniforme.

\par
%\textsuperscript{(679.1)}
\textsuperscript{59:4.8} A lo largo de todo este período, las tierras situadas al sudeste de la Isla de Cincinnati permanecieron completamente por encima del agua. Pero una gran parte de Europa occidental, incluyendo a las Islas Británicas, estaba sumergida. En el País de Gales, Alemania y otras partes de Europa, las rocas devonianas tienen un espesor de 6.000 metros.

\par
%\textsuperscript{(679.2)}
\textsuperscript{59:4.9} Hace \textit{250.000.000} de años se pudo presenciar la aparición de la familia de los peces, los vertebrados059:04.09 \footnote{\textit{Los peces vertebrados}: Gn 1:21.}, una de las etapas más importantes de toda la evolución prehumana.

\par
%\textsuperscript{(679.3)}
\textsuperscript{59:4.10} Los artrópodos, o crustáceos, fueron los antecesores de los primeros vertebrados. Los precursores de la familia de los peces fueron dos ascendientes artrópodos modificados; uno tenía un cuerpo largo que unía la cabeza y la cola, mientras que el otro era un pre-pez sin espina dorsal ni mandíbulas. Pero estos tipos preliminares fueron rápidamente aniquilados cuando los peces, los primeros vertebrados del mundo animal, aparecieron \textit{repentinamente} procedentes del norte.

\par
%\textsuperscript{(679.4)}
\textsuperscript{59:4.11} Muchos de los peces auténticos más grandes pertenecen a esta época, y algunas variedades provistas de dientes tenían entre ocho y diez metros de largo; los tiburones de hoy en día son los supervivientes de estos peces antiguos. Los peces con pulmón y coraza alcanzaron la cumbre de su evolución, y antes de que hubiera terminado esta época, los peces se habían adaptado tanto al agua dulce como a la salada.

\par
%\textsuperscript{(679.5)}
\textsuperscript{59:4.12} Se pueden encontrar verdaderos lechos óseos de dientes y esqueletos de peces en los depósitos acumulados hacia el final de este período, y existen unos lechos ricos en fósiles que están situados a lo largo de la costa de California, puesto que muchas bahías abrigadas del Océano Pacífico penetraban en las tierras de esta región.

\par
%\textsuperscript{(679.6)}
\textsuperscript{59:4.13} Las nuevas clases de vegetación terrestre estaban invadiendo la Tierra rápidamente. Hasta ahora crecían pocas plantas en la tierra, salvo en los bordes del agua. Entonces, la prolífica \textit{familia de los helechos} apareció \textit{repentinamente} y se extendió muy deprisa por la superficie de las tierras que se elevaban con rapidez en todas las partes del mundo. Pronto se desarrollaron unos tipos de árboles de sesenta centímetros de grueso y doce metros de altura; más tarde evolucionaron las hojas, pero estas variedades primitivas sólo poseían un follaje rudimentario. Existían muchas plantas más pequeñas, pero sus fósiles no se pueden encontrar puesto que las bacterias, que habían aparecido anteriormente, solían destruirlas.

\par
%\textsuperscript{(679.7)}
\textsuperscript{59:4.14} Cuando las tierras se elevaron, América del Norte quedó unida a Europa por medio de unos puentes terrestres que se extendían hasta Groenlandia. Y en la actualidad, Groenlandia conserva los restos de estas plantas terrestres primitivas bajo su manto de hielo.

\par
%\textsuperscript{(679.8)}
\textsuperscript{59:4.15} Hace \textit{240.000.000} de años, algunas partes de Europa y de América del Norte y del Sur empezaron a hundirse. Este hundimiento marcó la aparición de la última, y menos extensa, de todas las inundaciones devonianas. Los mares árticos se desplazaron de nuevo hacia el sur sobre una gran parte de Norteamérica; el Atlántico inundó gran parte de Europa y de Asia occidental, mientras que el Pacífico meridional cubría la mayoría de la India. Esta inundación fue tan lenta en aparecer como en retirarse. Las Montañas Catskill, situadas a lo largo del margen occidental del río Hudson, son uno de los mayores monumentos geológicos de esta época que se pueden encontrar en la superficie de América del Norte.

\par
%\textsuperscript{(679.9)}
\textsuperscript{59:4.16} Hace \textit{230.000.000} de años, los mares continuaban retirándose. Una gran parte de América del Norte estaba por encima del agua, y en la región del San Lorenzo se produjo una importante actividad volcánica. El Monte Real, en Montreal, es la chimenea erosionada de uno de estos volcanes. Los depósitos de toda esta época están bien visibles en los Montes Apalaches de América del Norte, allí donde el río Susquehanna ha tallado un valle que pone al descubierto estas capas sucesivas que alcanzaron más de 4.000 metros de espesor.

\par
%\textsuperscript{(680.1)}
\textsuperscript{59:4.17} Los continentes continuaban elevándose y la atmósfera se iba enriqueciendo en oxígeno. La Tierra estaba cubierta de inmensos bosques de helechos de treinta metros de alto, y de los árboles característicos de aquellos tiempos, unos bosques silenciosos donde no se escuchaba el menor ruido, ni siquiera el susurro de una hoja, pues aquellos árboles carecían de hojas.

\par
%\textsuperscript{(680.2)}
\textsuperscript{59:4.18} Y así llegó a su fin uno de los períodos más largos de la evolución de la vida marina, \textit{la época de los peces}. Este período de la historia del mundo duró casi cincuenta millones de años; vuestros investigadores lo conocen con el nombre de \textit{Devónico}.

\section*{5. La etapa de la deriva de la corteza --- El período carbonífero de los bosques de helechos --- La época de las ranas }
\par
%\textsuperscript{(680.3)}
\textsuperscript{59:5.1} La aparición de los peces durante el período anterior señala el punto culminante de la evolución de la vida marina. A partir de este momento, la evolución de la vida terrestre se vuelve cada vez más importante. Este período se inicia en unas condiciones casi ideales para la aparición de los primeros animales terrestres.

\par
%\textsuperscript{(680.4)}
\textsuperscript{59:5.2} Hace \textit{220.000.000} de años, muchas zonas continentales, incluyendo la mayor parte de América del Norte, se encontraban por encima del agua. La Tierra estaba invadida por una vegetación exuberante; fue realmente la \textit{época de los helechos}. El dióxido de carbono continuaba presente en la atmósfera, pero en menor grado.

\par
%\textsuperscript{(680.5)}
\textsuperscript{59:5.3} Poco tiempo después se inundó la porción central de América del Norte, creando dos grandes mares interiores. Las regiones montañosas de las costas del Atlántico y del Pacífico estaban situadas un poco más allá de los litorales actuales. Estos dos mares se unieron pronto, mezclando sus diversas formas de vida, y la unión de esta fauna marina marcó el comienzo del rápido declive mundial de la vida marina, y el principio del período siguiente de la vida terrestre.

\par
%\textsuperscript{(680.6)}
\textsuperscript{59:5.4} Hace \textit{210.000.000} de años, las cálidas aguas de los mares árticos cubrían la mayor parte de América del Norte y Europa. Las aguas polares del sur inundaban Sudamérica y Australia, mientras que África y Asia estaban muy elevadas.

\par
%\textsuperscript{(680.7)}
\textsuperscript{59:5.5} Cuando los mares alcanzaron su máximo nivel, un nuevo desarrollo evolutivo se produjo \textit{repentinamente}. Los primeros animales terrestres aparecieron bruscamente. Numerosas especies de estos animales podían vivir tanto en la tierra como en el agua. Estos anfibios que respiraban aire se desarrollaron a partir de los artrópodos, cuyas vejigas natatorias se habían transformado en pulmones.

\par
%\textsuperscript{(680.8)}
\textsuperscript{59:5.6} Los caracoles, los escorpiones y las ranas salieron de las aguas salobres de los mares y avanzaron por la tierra. Actualmente, las ranas continúan poniendo sus huevos en el agua, y sus crías comienzan su existencia como pececillos, los renacuajos. Este período podría conocerse muy bien como la \textit{época de las ranas}.

\par
%\textsuperscript{(680.9)}
\textsuperscript{59:5.7} Muy poco tiempo después aparecieron los insectos por primera vez, y pronto se extendieron por los continentes del mundo junto con las arañas, escorpiones, cucarachas, grillos y langostas. Las libélulas medían más de setenta y cinco centímetros de envergadura. Se desarrollaron mil especies de cucarachas, y algunas llegaron a medir diez centímetros de largo.

\par
%\textsuperscript{(680.10)}
\textsuperscript{59:5.8} Dos grupos de equinodermos se desarrollaron particularmente bien y son en realidad los fósiles guías de esta época. Los grandes tiburones que se alimentaban de animales con conchas también habían evolucionado mucho, y dominaron los océanos durante más de cinco millones de años. El clima era todavía templado y uniforme; la vida marina había cambiado poco. Los peces de agua dulce iban aumentando y los trilobites se acercaban a su extinción. Los corales eran escasos, y una gran parte de la caliza era elaborada por los crinoideos. Las calizas más finas para la construcción se depositaron durante esta época.

\par
%\textsuperscript{(681.1)}
\textsuperscript{59:5.9} Las aguas de muchos mares interiores estaban tan cargadas de cal y de otros minerales que dificultaron enormemente el progreso y el desarrollo de muchas especies marinas. Los mares se limpiaron finalmente a consecuencia de un extenso depósito de piedra que en algunas partes contenía zinc y plomo.

\par
%\textsuperscript{(681.2)}
\textsuperscript{59:5.10} Los depósitos de esta época carbonífera primitiva tienen entre 150 y 600 metros de espesor, y se componen de arenisca, esquisto y caliza. Los estratos más antiguos contienen fósiles de animales y plantas tanto terrestres como marinos, con mucha grava y sedimentos de las cuencas. Poco carbón explotable se encuentra en estos antiguos estratos. Los depósitos de este tipo, en toda Europa, son muy similares a los que se asentaron en América del Norte.

\par
%\textsuperscript{(681.3)}
\textsuperscript{59:5.11} Hacia el final de esta época, las tierras de América del Norte empezaron a elevarse. Hubo una breve interrupción, y el mar volvió a cubrir casi la mitad de sus lechos anteriores. Esta inundación fue de corta duración, y la mayor parte de las tierras se hallaron pronto muy por encima del agua. América del Sur estaba todavía conectada con Europa por medio de África.

\par
%\textsuperscript{(681.4)}
\textsuperscript{59:5.12} Esta época fue testigo del comienzo de la formación de los Vosgos, la Selva Negra y los Montes Urales. Las bases de otras montañas más antiguas se encuentran por toda Gran Bretaña y Europa.

\par
%\textsuperscript{(681.5)}
\textsuperscript{59:5.13} Hace \textit{200.000.000} de años empezaron las etapas realmente activas del período carbonífero. Los primeros depósitos de carbón se fueron asentando durante los veinte millones de años anteriores a esta época, pero ahora estaban en curso unas actividades más extensas para formar el carbón. La duración de la época efectiva de los depósitos de carbón fue un poco superior a los veinticinco millones de años.

\par
%\textsuperscript{(681.6)}
\textsuperscript{59:5.14} Las tierras subían y bajaban periódicamente debido a las variaciones del nivel del mar, provocadas por las actividades en los fondos oceánicos. Esta inestabilidad de la corteza ---el hundimiento y la elevación de las tierras--- en unión con la prolífica vegetación de los pantanos costeros, contribuyó a la formación de los inmensos depósitos de carbón, lo que ha motivado que este período se conozca con el nombre de \textit{Carbonífero}. El clima continuaba siendo templado en todo el mundo.

\par
%\textsuperscript{(681.7)}
\textsuperscript{59:5.15} Las capas de carbón alternaban con el esquisto, la piedra y el conglomerado. El espesor de estos yacimientos de carbón, en el centro y el este de los Estados Unidos, varía entre doce y quince metros. Pero muchos de estos depósitos fueron derrubiados durante las elevaciones terrestres posteriores. En algunas partes de América del Norte y Europa, los estratos carboníferos tienen 5.500 metros de espesor.

\par
%\textsuperscript{(681.8)}
\textsuperscript{59:5.16} La presencia de las raíces de los árboles que crecían en la arcilla que está debajo de los actuales yacimientos de hulla demuestra que el carbón se formó exactamente en el lugar donde se encuentra ahora. El carbón está constituido por los restos, conservados por el agua y modificados por la presión, de la vegetación exuberante que crecía en las ciénagas y en las orillas de los pantanos de esta época lejana. Los estratos de carbón contienen a menudo gas y petróleo a la vez. Los yacimientos de turba, restos de una antigua vegetación, se convertirían en un tipo de carbón si fueran sometidos a una presión y a una temperatura adecuadas. La antracita ha estado sometida a más presión y temperatura que otros tipos de carbón.

\par
%\textsuperscript{(681.9)}
\textsuperscript{59:5.17} En América del Norte, el número de las capas carboníferas de los distintos yacimientos, que indican la cantidad de veces que la tierra se hundió y se elevó, varía entre diez en Illinois, veinte en Pensilvania, treinta y cinco en Alabama, y setenta y cinco en Canadá. En los yacimientos de carbón se encuentran fósiles tanto de agua dulce como de agua salada.

\par
%\textsuperscript{(682.1)}
\textsuperscript{59:5.18} A lo largo de toda esta época, las montañas de América del Norte y del Sur estuvieron activas, elevándose tanto los Andes como las Montañas Rocosas ancestrales del sur. Las grandes regiones elevadas de las costas del Atlántico y del Pacífico empezaron a hundirse, volviéndose con el tiempo tan erosionadas y sumergidas que los litorales de los dos océanos se retiraron aproximadamente hasta sus posiciones actuales. Los depósitos de esta inundación tienen por término medio unos trescientos metros de espesor.

\par
%\textsuperscript{(682.2)}
\textsuperscript{59:5.19} Hace \textit{190.000.000} de años, el mar carbonífero de América del Norte se extendió hacia el oeste sobre la región actual de las Montañas Rocosas, desaguando en el Océano Pacífico a través del norte de California. El carbón continuó asentándose en todas las Américas y Europa, capa tras capa, a medida que las regiones costeras se elevaban y descendían durante estas épocas de oscilación de los litorales.

\par
%\textsuperscript{(682.3)}
\textsuperscript{59:5.20} Hace \textit{180.000.000} de años se terminó el período carbonífero, durante el cual el carbón se había formado en todo el mundo ---en Europa, la India, China, África del norte y las Américas. Al final de este período de formación del carbón, el este del valle del Misisipí, en América del Norte, se elevó, y la mayor parte de esta región ha permanecido desde entonces por encima del nivel del mar. Este período de elevación terrestre señala el comienzo de las montañas modernas de América del Norte, tanto en la región de los Apalaches como en el oeste. Los volcanes estaban activos en Alaska y California, así como en las regiones de Europa y Asia donde se estaban formando montañas. El este de América y el oeste de Europa estaban conectados por el continente de Groenlandia.

\par
%\textsuperscript{(682.4)}
\textsuperscript{59:5.21} La elevación de las tierras empezó a modificar el clima oceánico de las épocas anteriores, y a sustituirlo por los inicios del clima continental, menos benigno y más variable.

\par
%\textsuperscript{(682.5)}
\textsuperscript{59:5.22} Las plantas de estos tiempos eran esporíferas, y el viento podía diseminarlas en todas las direcciones. El tronco de los árboles carboníferos tenía generalmente dos metros de diámetro y a menudo treinta y ocho metros de altura. Los helechos modernos son verdaderas reliquias de estas épocas pasadas.

\par
%\textsuperscript{(682.6)}
\textsuperscript{59:5.23} Éstas fueron, por lo general, las épocas en que se desarrollaron los organismos de agua dulce; la vida marina anterior sufrió pocos cambios. Pero la característica importante de este período fue la aparición \textit{repentina} de las ranas y de sus múltiples primos. Las características de la vida de la época carbonífera fueron los \textit{helechos} y las \textit{ranas}.

\section*{6. La etapa de transición climática --- El período de las plantas con semillas --- La época de las tribulaciones biológicas}
\par
%\textsuperscript{(682.7)}
\textsuperscript{59:6.1} Este período señala el final del desarrollo evolutivo fundamental de la vida marina y el principio del período de transición que condujo a las épocas posteriores de los animales terrestres.

\par
%\textsuperscript{(682.8)}
\textsuperscript{59:6.2} Ésta fue una época de gran empobrecimiento de la vida. Miles de especies marinas perecieron, y la vida apenas estaba todavía bien establecida en la tierra. Fue un período de tribulaciones biológicas, una época en la que la vida casi desapareció de la faz de la Tierra y de las profundidades de los océanos. Hacia el final de la larga era de la vida marina, más de cien mil especies de criaturas vivientes existían en la Tierra. Al final de este período de transición, menos de quinientas habían sobrevivido.

\par
%\textsuperscript{(682.9)}
\textsuperscript{59:6.3} Las particularidades de este nuevo período no se debieron tanto al enfriamiento de la corteza terrestre o a la larga ausencia de la actividad volcánica como a una combinación inhabitual de influencias vulgares y preexistentes ---el estrechamiento de los mares y la creciente elevación de enormes masas terrestres. El templado clima oceánico de los tiempos pasados estaba desapareciendo, y el tipo de clima continental más severo se extendía rápidamente.

\par
%\textsuperscript{(683.1)}
\textsuperscript{59:6.4} Hace \textit{170.000.000} de años tuvieron lugar unas grandes adaptaciones y cambios evolutivos en toda la superficie de la Tierra. Los continentes se estaban elevando en todo el mundo a medida que los fondos oceánicos se hundían. Aparecieron cadenas montañosas aisladas. La parte oriental de América del Norte estaba muy por encima del mar; el oeste se elevaba lentamente. Los continentes estaban cubiertos de lagos salados grandes y pequeños, y de numerosos mares interiores que se comunicaban con los océanos por medio de angostos estrechos. Los estratos de este período de transición varían entre 300 y 2.100 metros de espesor.

\par
%\textsuperscript{(683.2)}
\textsuperscript{59:6.5} La corteza terrestre se plegó extensamente durante estas elevaciones de tierras. Fue una época de elevación continental, pero desaparecieron algunos puentes terrestres, incluyendo a los continentes que habían conectado durante tanto tiempo a América del Sur con África y a América del Norte con Europa.

\par
%\textsuperscript{(683.3)}
\textsuperscript{59:6.6} Los lagos y los mares interiores se iban secando gradualmente en todo el mundo. Empezaron a aparecer montañas aisladas y glaciares regionales, especialmente en el hemisferio sur, y el depósito glacial de estas formaciones de hielo locales se puede encontrar, en muchas regiones, incluso entre algunas capas superiores de los últimos depósitos de carbón. Aparecieron dos nuevos factores climáticos ---la glaciación y la aridez. Muchas de las regiones más elevadas de la Tierra se habían vuelto áridas y estériles.

\par
%\textsuperscript{(683.4)}
\textsuperscript{59:6.7} A lo largo de todos estos tiempos de cambios climáticos se produjeron también grandes variaciones en las plantas terrestres. Las \textit{plantas con semillas} aparecieron por primera vez y proporcionaron una mejor provisión de alimentos para la vida animal terrestre que se multiplicaría posteriormente. Los insectos sufrieron un cambio radical. Sus \textit{períodos de reposo} evolucionaron para hacer frente a las exigencias de la suspensión de las funciones vitales durante el invierno y las sequías.

\par
%\textsuperscript{(683.5)}
\textsuperscript{59:6.8} Entre los animales terrestres, las ranas alcanzaron su punto culminante en la época anterior, y declinaron rápidamente, pero sobrevivieron porque podían vivir mucho tiempo incluso en las charcas y los estanques en vías de secarse de aquellos tiempos lejanos extremadamente duros. Durante esta época de decadencia de las ranas, el primer paso de su evolución hacia los reptiles se produjo en África. Como las masas continentales aún estaban conectadas entre sí, estas criaturas pre-reptiles que respiraban aire se diseminaron por todo el mundo. La atmósfera había cambiado tanto en esta época que servía admirablemente para mantener la respiración animal. Poco tiempo después de la llegada de estas ranas pre-reptiles, América del Norte se quedó temporalmente aislada, separada de Europa, Asia y América del Sur.

\par
%\textsuperscript{(683.6)}
\textsuperscript{59:6.9} El enfriamiento paulatino de las aguas oceánicas contribuyó mucho a la destrucción de la vida en los mares. Los animales marinos de aquellos tiempos se refugiaron temporalmente en tres lugares favorables: la región actual del Golfo de Méjico, la Bahía del Ganges en la India y la Bahía de Sicilia en la cuenca mediterránea. Desde estas tres regiones, las nuevas especies marinas, nacidas para afrontar la adversidad, salieron más tarde para repoblar los mares.

\par
%\textsuperscript{(683.7)}
\textsuperscript{59:6.10} Hace \textit{160.000.000} de años, la Tierra estaba ampliamente cubierta de una vegetación adaptada al mantenimiento de la vida animal terrestre, y la atmósfera se había vuelto ideal para la respiración animal. Así terminan el período de reducción de la vida marina y los difíciles tiempos de adversidad biológica que eliminaron todas las formas de vida, salvo las que tenían un valor de supervivencia; por lo tanto, estas últimas merecieron ser los antepasados de la vida muy bien diferenciada que se desarrollaría con más rapidez durante las épocas siguientes de la evolución planetaria.

\par
%\textsuperscript{(684.1)}
\textsuperscript{59:6.11} El final de este período de tribulaciones biológicas, que vuestros estudiosos conocen con el nombre de \textit{Pérmico}, señala igualmente el final de la larga era \textit{Paleozoica}, que abarca una cuarta parte de la historia planetaria, o sea doscientos cincuenta millones de años.

\par
%\textsuperscript{(684.2)}
\textsuperscript{59:6.12} El inmenso criadero de vida que fueron los océanos de Urantia ha cumplido su objetivo. Durante las largas épocas en que las tierras eran inadecuadas para sostener la vida, antes de que la atmósfera contuviera el suficiente oxígeno para mantener a los animales terrestres superiores, el mar dio a luz a la vida primitiva del planeta y la alimentó. Ahora, la importancia biológica del mar disminuye progresivamente a medida que la segunda etapa de la evolución empieza a desarrollarse en la tierra firme.

\par
%\textsuperscript{(684.3)}
\textsuperscript{59:6.13} [Presentado por un Portador de Vida de Nebadon, miembro del cuerpo original asignado a Urantia.]


\chapter{Documento 60. Urantia durante la era de la vida terrestre primitiva}
\par
%\textsuperscript{(685.1)}
\textsuperscript{60:0.1} LA ERA de la vida exclusivamente marina ha terminado. La elevación de las tierras, el enfriamiento de la corteza y de los océanos, el estrechamiento de los mares y, como consecuencia de esto, el hacerse cada vez más profundos, así como el gran aumento de las tierras en las latitudes septentrionales, contribuyeron todos enormemente a cambiar el clima del mundo en todas las regiones alejadas de la zona ecuatorial.

\par
%\textsuperscript{(685.2)}
\textsuperscript{60:0.2} Las épocas finales de la era anterior fueron en verdad la era de las ranas, pero estos antepasados de los vertebrados terrestres ya no eran dominantes pues habían sobrevivido en cantidades muy reducidas. Muy pocos tipos salieron con vida de las rigurosas pruebas del período anterior de tribulaciones biológicas. Incluso las plantas esporíferas estuvieron a punto de extinguirse.

\section*{1. La época primitiva de los reptiles}
\par
%\textsuperscript{(685.3)}
\textsuperscript{60:1.1} Los depósitos de erosión de este período eran principalmente conglomerados, esquisto y arenisca. Tanto en América como en Europa, el yeso y las capas rojas de todas estas sedimentaciones indican que el clima de estos continentes era árido. Estas regiones áridas estuvieron sometidas a una gran erosión causada por los aguaceros periódicos y violentos que caían en las altas tierras circundantes.

\par
%\textsuperscript{(685.4)}
\textsuperscript{60:1.2} En estas capas se encuentran pocos fósiles, pero en la arenisca se pueden observar numerosas huellas de los reptiles terrestres. En muchas regiones, los depósitos de arenisca roja de trescientos metros de espesor, correspondientes a este período, no contienen ningún fósil. Los animales terrestres sólo vivieron de manera continuada en algunas partes de África.

\par
%\textsuperscript{(685.5)}
\textsuperscript{60:1.3} El espesor de estos depósitos varía entre 900 y 3.000 metros, y alcanza incluso 5.500 metros en la costa del Pacífico. Más tarde, la lava se introdujo por la fuerza entre muchas de estas capas. Los Acantilados del Río Hudson fueron formados por la extrusión de lavas basálticas entre estos estratos triásicos. La actividad volcánica era extensa en diversas partes del mundo.

\par
%\textsuperscript{(685.6)}
\textsuperscript{60:1.4} Los depósitos de este período se pueden encontrar en Europa, especialmente en Alemania y Rusia. La nueva arenisca roja de Inglaterra pertenece a esta época. La caliza se depositó en los Alpes meridionales a consecuencia de una invasión del mar, y ahora se puede observar bajo la forma peculiar de los muros, picos y pilares de caliza dolomítica de esas regiones. Esta capa se encuentra en toda África y Australia. El mármol de Carrara procede de esta caliza modificada. No se encontrará nada de este período en las regiones meridionales de América del Sur, pues aquella parte del continente permaneció sumergida y, por lo tanto, sólo presenta un depósito acuático o marino sin interrupción entre las épocas anteriores y posteriores.

\par
%\textsuperscript{(686.1)}
\textsuperscript{60:1.5} Hace \textit{150.000.000} de años comenzaron los primeros períodos de la vida terrestre en la historia del mundo. A la vida no le iba bien en general, pero le iba mejor que durante la etapa final, ardua y hostil, de la era de la vida marina.

\par
%\textsuperscript{(686.2)}
\textsuperscript{60:1.6} Al empezar esta era, las partes orientales y centrales de América del Norte, la mitad norte de América del Sur, la mayor parte de Europa y toda Asia están completamente por encima del agua. América del Norte se encuentra geográficamente aislada por primera vez, pero no por mucho tiempo, ya que el puente terrestre del Estrecho de Bering emerge pronto de nuevo, uniendo al continente con Asia.

\par
%\textsuperscript{(686.3)}
\textsuperscript{60:1.7} En América del Norte se formaron grandes depresiones paralelas a las costas del Atlántico y del Pacífico. En Connecticut apareció la gran falla oriental, y uno de sus lados se hundió con el tiempo más de tres kilómetros. Muchas de estas depresiones norteamericanas y muchas cuencas lacustres de agua dulce y salada de las regiones montañosas se llenaron posteriormente con depósitos de erosión. Más tarde, estas depresiones terrestres rellenas fueron elevadas considerablemente debido a las corrientes de lava que se produjeron bajo tierra. Los bosques petrificados de muchas regiones corresponden a esta época.

\par
%\textsuperscript{(686.4)}
\textsuperscript{60:1.8} La costa del Pacífico, que habitualmente permaneció por encima del agua durante las inmersiones continentales, se hundió, a excepción de la parte sur de California y de una gran isla que entonces existía en lo que hoy es el Océano Pacífico. Este antiguo mar de California era rico en vida marina y se extendía hacia el este hasta unirse con la vieja cuenca marítima de la región del mediooeste norteamericano.

\par
%\textsuperscript{(686.5)}
\textsuperscript{60:1.9} Hace \textit{140.000.000} de años, y con el único indicio de los dos antepasados pre-reptiles que se habían desarrollado en África durante la época anterior, los reptiles aparecieron \textit{repentinamente} con todos sus atributos\footnote{\textit{Los reptiles}: Gn 1:24.}. Se desarrollaron con rapidez, y pronto dieron nacimiento a los cocodrilos, a los reptiles con escamas y finalmente a las serpientes marinas y a los reptiles voladores. Sus antepasados de transición desaparecieron rápidamente.

\par
%\textsuperscript{(686.6)}
\textsuperscript{60:1.10} Estos dinosaurios reptiles que evolucionaban con rapidez se convirtieron pronto en los reyes de esta época. Ponían huevos y se distinguían de todos los demás animales por tener un cerebro pequeño, que pesaba menos de medio kilo y tenía que controlar un cuerpo que más adelante llegó a pesar cuarenta toneladas. Pero los primeros reptiles eran más pequeños, carnívoros, y caminaban sobre sus patas traseras igual que los canguros. Tenían los huesos huecos como las aves y posteriormente sólo desarrollaron tres dedos en sus patas traseras, por lo que muchas de sus huellas fosilizadas se han confundido con las de aves gigantes. Los dinosaurios herbívoros evolucionaron más tarde. Caminaban sobre las cuatro patas y una rama de este grupo desarrolló una coraza protectora.

\par
%\textsuperscript{(686.7)}
\textsuperscript{60:1.11} Los primeros mamíferos aparecieron varios millones de años después. No tenían placenta y rápidamente resultaron ser un fracaso; ninguno de ellos sobrevivió. Se trató de un esfuerzo experimental por mejorar los tipos de mamíferos, pero no tuvo éxito en Urantia.

\par
%\textsuperscript{(686.8)}
\textsuperscript{60:1.12} La vida marina de este período era escasa, pero mejoró rápidamente gracias a la nueva invasión de los mares, que produjo otra vez extensos litorales de aguas poco profundas. Como la cantidad de aguas poco profundas era mayor alrededor de Europa y Asia, los yacimientos más ricos en fósiles se encuentran cerca de estos continentes. Si hoy queréis estudiar la vida de esta época, examinad las regiones del Himalaya, Siberia y el Mediterráneo, así como la India y las islas de la cuenca del Pacífico Sur. Una característica destacada de la vida marina era la presencia de grandes cantidades de hermosos amonites, cuyos restos fósiles se encuentran por todo el mundo.

\par
%\textsuperscript{(686.9)}
\textsuperscript{60:1.13} Hace \textit{130.000.000} de años, los mares habían cambiado muy poco. Siberia y América del Norte estaban unidas por el puente terrestre del Estrecho de Bering. Una vida marina abundante y excepcional apareció en la costa californiana del Pacífico, donde más de mil especies de amonites se desarrollaron a partir de los tipos superiores de cefalópodos. Durante este período, los cambios en la vida fueron realmente revolucionarios, a pesar de ser transitorios y graduales.

\par
%\textsuperscript{(687.1)}
\textsuperscript{60:1.14} Este período se prolongó durante veinticinco millones de años, y se le conoce con el nombre de \textit{Triásico}.

\section*{2. La época posterior de los reptiles}
\par
%\textsuperscript{(687.2)}
\textsuperscript{60:2.1} Hace \textit{120.000.000} de años empezó una nueva fase de la época de los reptiles. El gran acontecimiento de este período fue la evolución y la decadencia de los dinosaurios. La vida animal terrestre alcanzó su máximo desarrollo en lo que se refiere al tamaño, y prácticamente había desaparecido de la faz de la Tierra al finalizar esta época. Evolucionaron dinosaurios de todos los tamaños, desde una especie que medía menos de sesenta centímetros hasta los enormes dinosaurios no carnívoros de casi veintitrés metros de longitud, cuya corpulencia no ha sido igualada nunca más por ninguna criatura viviente.

\par
%\textsuperscript{(687.3)}
\textsuperscript{60:2.2} Los dinosaurios más grandes tuvieron su origen en el oeste de América del Norte. Estos monstruosos reptiles están enterrados en todas las regiones de las Montañas Rocosas, a lo largo de toda la costa atlántica de América del Norte, en Europa occidental, África del Sur y la India, pero no en Australia.

\par
%\textsuperscript{(687.4)}
\textsuperscript{60:2.3} Estas criaturas macizas se volvieron menos activas y fuertes a medida que aumentaron de tamaño; pero necesitaban una cantidad de comida tan enorme y la Tierra estaba tan atestada de ellos, que se murieron literalmente de hambre y se extinguieron ---les faltó la inteligencia necesaria para enfrentarse con la situación.

\par
%\textsuperscript{(687.5)}
\textsuperscript{60:2.4} En esta época, la mayor parte del este de América del Norte, que había estado mucho tiempo elevada, había sido rebajada de nivel y arrastrada hacia el Océano Atlántico, de tal manera que la costa se extendía varios cientos de kilómetros más allá que en la actualidad. La parte occidental del continente aún estaba elevada, pero estas mismas regiones fueron invadidas más tarde tanto por el mar del norte como por el Pacífico, que se extendió hacia el este hasta la región de Black Hills, en Dakota.

\par
%\textsuperscript{(687.6)}
\textsuperscript{60:2.5} Ésta fue una época de agua dulce caracterizada por numerosos lagos interiores, tal como lo demuestran los abundantes fósiles de agua dulce de los llamados yacimientos <<Morrison>> de Colorado, Montana y Wyoming. El espesor de estos depósitos combinados de agua dulce y salada varía entre 600 y 1.500 metros; pero muy poca caliza está presente en estas capas.

\par
%\textsuperscript{(687.7)}
\textsuperscript{60:2.6} El mismo mar polar que se extendió tan lejos hacia el sur sobre América del Norte, cubrió igualmente toda Sudamérica, a excepción de la cordillera de los Andes que acababa de aparecer. La mayor parte de China y Rusia estaba inundada, pero la invasión de las aguas fue más importante en Europa. Durante esta inmersión se sedimentó la hermosa piedra litográfica de Alemania del sur, unos estratos en los que se han conservado, como si se hubieran depositado ayer mismo, unos fósiles tales como las alas más delicadas de los antiguos insectos.

\par
%\textsuperscript{(687.8)}
\textsuperscript{60:2.7} La flora de esta época era muy similar a la de la anterior. Los helechos persistían, mientras que las coníferas y los pinos se parecían cada vez más a las variedades de hoy en día. Aún se estaba formando un poco de carbón a lo largo de las costas septentrionales del Mediterráneo.

\par
%\textsuperscript{(687.9)}
\textsuperscript{60:2.8} El regreso de los mares mejoró el clima. Los corales se extendieron por las aguas europeas, lo que demuestra que el clima era todavía templado y uniforme, pero nunca volvieron a aparecer en los mares polares que se enfriaban lentamente. La vida marina de estos tiempos mejoró y se desarrolló considerablemente, sobre todo en las aguas europeas. Tanto los corales como los crinoideos aparecieron temporalmente en mayores cantidades que antes, pero los amonites dominaban la vida invertebrada de los océanos; su tamaño medio oscilaba entre siete y diez centímetros, aunque una especie alcanzó un diámetro de dos metros y medio. Las esponjas estaban por todas partes, y tanto las jibias como las ostras continuaron evolucionando.

\par
%\textsuperscript{(688.1)}
\textsuperscript{60:2.9} Hace \textit{110.000.000} de años, los potenciales de la vida marina continuaban desarrollándose. El erizo de mar fue una de las mutaciones sobresalientes de esta época. Los cangrejos, las langostas y otros tipos de crustáceos modernos se desarrollaron plenamente. Se produjeron cambios destacados en la familia de los peces, apareciendo por primera vez un tipo de esturión, pero las feroces serpientes de mar, descendientes de los reptiles terrestres, infestaban aún todos los mares y amenazaban con destruir la familia entera de los peces.

\par
%\textsuperscript{(688.2)}
\textsuperscript{60:2.10} Ésta continuaba siendo por excelencia la época de los dinosaurios. Invadieron la Tierra hasta tal punto que, durante el período anterior de invasión del mar, dos especies se habían adaptado al agua para subsistir. Estas serpientes de mar representan un paso atrás en la evolución. Mientras que algunas especies nuevas van progresando, ciertas cepas permanecen estacionarias y otras tienden a retroceder, volviendo a un estado anterior. Y esto es lo que sucedió cuando estos dos tipos de reptiles abandonaron la tierra firme.

\par
%\textsuperscript{(688.3)}
\textsuperscript{60:2.11} A medida que pasaba el tiempo, las serpientes de mar alcanzaron tales dimensiones que se volvieron muy lentas, y al final perecieron porque no tenían un cerebro lo bastante grande como para proteger sus inmensos cuerpos. Su cerebro pesaba menos de sesenta gramos, a pesar del hecho de que estos enormes ictiosaurios alcanzaban a veces quince metros de longitud, y la mayoría sobrepasaba los diez metros. Los cocodriloideos marinos fueron también una regresión del tipo de reptil terrestre, pero a diferencia de las serpientes marinas, estos animales siempre volvían a la tierra para poner sus huevos.

\par
%\textsuperscript{(688.4)}
\textsuperscript{60:2.12} Poco después de que dos especies de dinosaurios emigraran al agua en un intento vano por preservarse, otros dos tipos se vieron forzados a vivir en el aire debido a la lucha encarnizada por la vida en la tierra. Pero estos pterosaurios voladores no fueron los antepasados de las auténticas aves de las épocas posteriores; evolucionaron a partir de los dinosaurios saltadores de huesos huecos, y sus alas se parecían a las de los murciélagos, con una envergadura de seis a ocho metros. Estos antiguos reptiles voladores se desarrollaban hasta alcanzar tres metros de largo, y tenían unas mandíbulas separables muy parecidas a las de las serpientes modernas. Durante algún tiempo, estos reptiles voladores parecieron ser un éxito, pero no lograron evolucionar de manera que pudieran sobrevivir como navegantes aéreos. Representan las cepas extinguidas de los precursores de las aves.

\par
%\textsuperscript{(688.5)}
\textsuperscript{60:2.13} Las tortugas se multiplicaron durante este período, apareciendo por primera vez en América del Norte. Sus antepasados habían venido de Asia por el puente terrestre del norte.

\par
%\textsuperscript{(688.6)}
\textsuperscript{60:2.14} Hace cien millones de años, la época de los reptiles se acercaba a su fin. Los dinosaurios, a pesar de su enorme masa, eran unos animales casi sin cerebro, y carecían de la inteligencia suficiente para conseguir la comida necesaria a fin de alimentar unos cuerpos tan colosales. Por ese motivo, estos perezosos reptiles terrestres perecieron en cantidades cada vez mayores. De ahora en adelante, la evolución perseguirá el crecimiento del cerebro, y no la masa física; y el desarrollo del cerebro caracterizará cada época sucesiva de la evolución animal y del progreso planetario.

\par
%\textsuperscript{(688.7)}
\textsuperscript{60:2.15} Este período, que abarca el apogeo de los reptiles y el principio de su decadencia, duró casi veinticinco millones de años y se conoce con el nombre de \textit{Jurásico}.

\section*{3. La etapa cretácea --- El período de las plantas floríferas --- La época de las aves}
\par
%\textsuperscript{(688.8)}
\textsuperscript{60:3.1} El gran período cretáceo deriva su nombre del predominio en los mares de los prolíficos foraminíferos productores de creta. Este período conduce a Urantia cerca del final del largo dominio de los reptiles, y es testigo de la aparición en la Tierra de las plantas floríferas y las aves. Es también la época en que termina la deriva de los continentes hacia el oeste y el sur, acompañada de enormes deformaciones de la corteza junto con flujos de lava generalizados y grandes actividades volcánicas.

\par
%\textsuperscript{(689.1)}
\textsuperscript{60:3.2} Cerca del final del período geológico anterior, una gran parte de las tierras continentales estaban por encima de las aguas, aunque hasta ahora no había picos montañosos. Pero a medida que continuaba la deriva continental, ésta se encontró con el primer gran obstáculo en el fondo profundo del Pacífico. Esta contienda entre las fuerzas geológicas impulsó la formación de toda la enorme cordillera que se extiende en dirección norte-sur desde Alaska hasta el Cabo de Hornos, pasando por Méjico.

\par
%\textsuperscript{(689.2)}
\textsuperscript{60:3.3} En la historia geológica, este período se convierte así en la \textit{etapa de formación de las montañas modernas}. Antes de esta época existían pocos picos montañosos, sólo había lomas elevadas de gran anchura. En aquel entonces, la cordillera costera del Pacífico empezaba a elevarse, pero estaba situada a 1.100 kilómetros al oeste del litoral actual. Las Sierras estaban comenzando a formarse, y sus estratos de cuarzo auríferos son el resultado de las corrientes de lava de esta época. En la parte este de América del Norte, la presión de las aguas del Atlántico actuaba también para provocar una elevación de las tierras.

\par
%\textsuperscript{(689.3)}
\textsuperscript{60:3.4} Hace \textit{100.000.000} de años, el continente norteamericano y una parte de Europa estaban completamente por encima del agua. La deformación de los continentes americanos continuaba, produciendo la metamorfosis de los Andes sudamericanos y la elevación gradual de las llanuras occidentales de América del Norte. La mayor parte de Méjico se hundió bajo el mar, y el Atlántico meridional invadió la costa oriental de América del Sur, alcanzando finalmente el litoral actual. Los océanos Atlántico e
Índico eran entonces más o menos como hoy.

\par
%\textsuperscript{(689.4)}
\textsuperscript{60:3.5} Hace \textit{95.000.000} de años, las masas terrestres de América y Europa empezaron a hundirse de nuevo. Los mares del sur comenzaron a invadir América del Norte y se extendieron paulatinamente hacia el norte hasta comunicarse con el Océano Ártico, lo que constituyó la segunda gran inmersión del continente. Cuando este mar se retiró finalmente, dejó el continente casi como es en la actualidad. Antes de que empezara esta gran inmersión, las tierras altas del este de los Apalaches se habían desgastado casi por completo hasta el nivel del mar. Las capas policromas de arcilla pura que se utilizan ahora para fabricar objetos de barro se depositaron en las regiones costeras del Atlántico durante esta época, y tienen un espesor medio de unos 600 metros.

\par
%\textsuperscript{(689.5)}
\textsuperscript{60:3.6} Se produjeron grandes actividades volcánicas al sur de los Alpes y a lo largo de la cordillera costera actual de California. En Méjico tuvieron lugar las mayores deformaciones de la corteza que se habían observado durante millones y millones de años. También ocurrieron grandes cambios en Europa, Rusia, Japón y en la parte meridional de América del Sur. El clima se volvió cada vez más variado.

\par
%\textsuperscript{(689.6)}
\textsuperscript{60:3.7} Hace \textit{90.000.000} de años, las angiospermas emergieron de estos mares cretáceos primitivos y pronto invadieron los continentes. Estas plantas terrestres aparecieron \textit{repentinamente} junto con las higueras, las magnolias y los tulipaneros. Poco tiempo después, las higueras, los árboles del pan y las palmeras se extendieron sobre Europa y las llanuras occidentales de América del Norte. No apareció ningún nuevo animal terrestre.

\par
%\textsuperscript{(689.7)}
\textsuperscript{60:3.8} Hace \textit{85.000.000} de años se cerró el Estrecho de Bering, aislando a las aguas de los mares nórdicos en vías de enfriarse. Hasta entonces, la vida marina de las aguas del Golfo y del Atlántico había diferido enormemente de la del Océano Pacífico debido a las variaciones de temperatura de estas dos masas de agua, que ahora se volvieron uniformes.

\par
%\textsuperscript{(689.8)}
\textsuperscript{60:3.9} Los depósitos de creta y de marga de arenisca verde dan su nombre a este período. Las sedimentaciones de esta época son abigarradas, y consisten en creta, esquisto, arenisca y pequeñas cantidades de caliza, junto con carbón de calidad inferior o lignito, y en muchas regiones contienen petróleo. El espesor de estas capas varía entre 60 metros en algunos lugares hasta 3.000 metros en el oeste de América del Norte y en muchas localidades de Europa. Estos depósitos se pueden observar en las estribaciones inclinadas de los bordes orientales de las Montañas Rocosas.

\par
%\textsuperscript{(690.1)}
\textsuperscript{60:3.10} Estos estratos están impregnados de creta en todo el mundo, y estas capas de semirroca porosa recogen el agua en los afloramientos inclinados y la transportan hacia abajo para proporcionar suministro de agua a una gran parte de las regiones actualmente áridas de la Tierra.

\par
%\textsuperscript{(690.2)}
\textsuperscript{60:3.11} Hace \textit{80.000.000} de años se produjeron grandes perturbaciones en la corteza terrestre. El avance de la deriva continental hacia el oeste se estaba deteniendo, y la enorme energía de la pesada inercia de la masa continental interior desplomó el litoral Pacífico de las dos Américas, iniciándose como repercusión unos cambios profundos a lo largo de las costas asiáticas del Pacífico. Esta elevación de tierras alrededor del Pacífico, que culminó en las cadenas de montañas actuales, tiene más de cuarenta mil kilómetros de longitud. Los levantamientos que acompañaron su nacimiento fueron las mayores deformaciones de la superficie que han tenido lugar desde que la vida apareció en Urantia. Las corrientes de lava, tanto por encima como por debajo de la tierra, fueron extensas y generalizadas.

\par
%\textsuperscript{(690.3)}
\textsuperscript{60:3.12} La época de hace \textit{75.000.000} de años señala el final de la deriva continental. Desde Alaska hasta el Cabo de Hornos, las largas cadenas de montañas de la costa del Pacífico estaban concluidas, pero aún había pocos picos.

\par
%\textsuperscript{(690.4)}
\textsuperscript{60:3.13} El deslizamiento hacia atrás causado por la detención de la deriva continental continuó elevando las llanuras occidentales de América del Norte, mientras que en el este, los desgastados Montes Apalaches de la región costera del Atlántico fueron proyectados directamente hacia arriba, con poca o ninguna inclinación.

\par
%\textsuperscript{(690.5)}
\textsuperscript{60:3.14} Hace \textit{70.000.000} de años tuvieron lugar las deformaciones de la corteza relacionadas con la máxima elevación de la región de las Montañas Rocosas. Un gran segmento de roca fue empujado veinticuatro kilómetros sobre la superficie de la Columbia Británica; en este lugar las rocas cámbricas están tendidas oblicuamente sobre las capas cretáceas. Otro corrimiento espectacular se produjo en la vertiente oriental de las Montañas Rocosas, cerca de la frontera canadiense; aquí se pueden encontrar las capas de piedra anteriores a la vida colocadas encima de los depósitos cretáceos entonces recientes.

\par
%\textsuperscript{(690.6)}
\textsuperscript{60:3.15} Ésta fue una época de actividad volcánica en todo el mundo, que dio origen a numerosos pequeños conos volcánicos aislados. Unos volcanes submarinos estallaron en la región sumergida del Himalaya. Una gran parte del resto de Asia, incluyendo a Siberia, aún estaba también por debajo del agua.

\par
%\textsuperscript{(690.7)}
\textsuperscript{60:3.16} Hace \textit{65.000.000} de años se produjo una de las mayores erupciones de lava de todos los tiempos. Las capas depositadas por estas erupciones de lava y otras anteriores se pueden encontrar en todas las Américas, África del norte y del sur, Australia y algunas partes de Europa.

\par
%\textsuperscript{(690.8)}
\textsuperscript{60:3.17} Los animales terrestres habían cambiado poco, pero se multiplicaron rápidamente debido a una mayor emergencia continental, sobre todo en América del Norte. Como la mayor parte de Europa estaba sumergida, América del Norte fue el gran campo donde evolucionaron los animales terrestres de aquellos tiempos.

\par
%\textsuperscript{(690.9)}
\textsuperscript{60:3.18} El clima continuaba siendo cálido y uniforme. Las regiones árticas disfrutaban de un tiempo muy parecido al del clima actual del centro y el sur de América del Norte.

\par
%\textsuperscript{(690.10)}
\textsuperscript{60:3.19} Una gran evolución se estaba produciendo en la vida vegetal. Las angiospermas predominaban entre las plantas terrestres y muchos árboles actuales aparecieron por primera vez, incluyendo a las hayas, abedules, robles, nogales, sicomoros, arces y palmeras modernas. Abundaban las frutas, las hierbas y los cereales, y estas hierbas y árboles semillíferos significaron para el mundo vegetal lo que los antepasados del hombre para el mundo animal ---su importancia evolutiva sólo fue superada por la aparición del hombre mismo. \textit{Repentinamente} y sin una gradación previa, la gran familia de las plantas floríferas apareció por mutación. Esta nueva flora se extendió pronto por el mundo entero.

\par
%\textsuperscript{(691.1)}
\textsuperscript{60:3.20} Hace \textit{60.000.000} de años, aunque los reptiles terrestres estaban en decadencia, los dinosaurios continuaban siendo los reyes de la Tierra, y ahora pasaron a ocupar el primer lugar los tipos más ágiles y activos de dinosaurios carnívoros, pertenecientes a las variedades saltadoras más pequeñas, similares a los canguros. Pero algún tiempo antes habían aparecido unos nuevos tipos de dinosaurios herbívoros, que se multiplicaron rápidamente debido a la aparición de las plantas terrestres de la familia de las herbáceas. Uno de estos nuevos dinosaurios herbívoros era un verdadero cuadrúpedo, provisto de dos cuernos y un reborde parecido a una capa sobre la paletilla. Apareció el tipo de tortuga terrestre de seis metros de ancho, así como los cocodrilos modernos y las auténticas serpientes del tipo actual. También se estaban produciendo grandes cambios entre los peces y otras formas de vida marina.

\par
%\textsuperscript{(691.2)}
\textsuperscript{60:3.21} Las pre-aves zancudas y nadadoras de las épocas anteriores no habían prosperado en el aire, ni tampoco los dinosaurios voladores. Fueron unas especies efímeras que se extinguieron pronto. Sufrieron también el mismo destino que los dinosaurios, la destrucción, pues tenían muy poca sustancia cerebral en comparación con el tamaño de su cuerpo. Esta segunda tentativa por producir unos animales que pudieran navegar en la atmósfera fracasó, al igual que el intento frustrado por producir los mamíferos durante esta época y una época anterior.

\par
%\textsuperscript{(691.3)}
\textsuperscript{60:3.22} Hace \textit{55.000.000} de años, la marcha de la evolución estuvo marcada por la aparición \textit{repentina} de la primera \textit{auténtica ave},\footnote{\textit{Las aves}: Gn 1:20-22.} una pequeña criatura parecida a la paloma, que fue la antecesora de todas las aves. Era el tercer tipo de criatura voladora que aparecía en la Tierra; surgió directamente del grupo de los reptiles, y no de los dinosaurios voladores contemporáneos ni de los tipos anteriores de aves terrestres dentadas. Por eso a este período se le conoce como la \textit{época de las aves} así como la época de la decadencia de los reptiles.

\section*{4. El final del período cretáceo}
\par
%\textsuperscript{(691.4)}
\textsuperscript{60:4.1} El gran período cretáceo se acercaba a su fin, y su terminación señala el final de las grandes invasiones marinas de los continentes. Esto es particularmente cierto en lo que se refiere a América del Norte, donde había habido exactamente veinticuatro grandes inundaciones. Aunque posteriormente se produjeron inmersiones de menor importancia, ninguna de ellas se puede comparar con las extensas y prolongadas invasiones marinas de esta época y de otras anteriores. Estos períodos en los que la tierra y el mar predominaban alternativamente se produjeron durante ciclos de millones de años. La elevación y el hundimiento de los fondos oceánicos y de los niveles de las tierras continentales se efectuaron siguiendo un ritmo secular. Estos mismos movimientos rítmicos de la corteza continuarán produciéndose durante toda la historia de la Tierra, pero con menos frecuencia y en menor grado.

\par
%\textsuperscript{(691.5)}
\textsuperscript{60:4.2} Este período presencia también el final de la deriva continental y la formación de las montañas modernas de Urantia. Pero la presión de las masas continentales y el impulso transversal de su deriva secular no son los únicos factores que influyen en la formación de las montañas. El factor principal y subyacente que determina el emplazamiento de una cordillera es la existencia previa de una tierra baja, o depresión, que se ha rellenado con los depósitos relativamente más ligeros de la erosión terrestre y con los terrenos de acarreo marinos de las épocas anteriores. Estas zonas de tierra más ligeras tienen a veces un espesor de 4.500 a 6.000 metros; por consiguiente, cuando la corteza es sometida a una presión de cualquier origen, estas zonas más ligeras son las primeras en desplomarse, plegarse y levantarse para equilibrar y compensar las fuerzas y presiones en conflicto y contrapuestas que actúan en la corteza terrestre o por debajo de ella. Estos levantamientos de tierras se producen a veces sin plegamientos. Pero en relación con la elevación de las Montañas Rocosas, se produjeron unos grandes plegamientos e inclinaciones, junto con enormes deslizamientos de las distintas capas, tanto superficiales como subterráneas.

\par
%\textsuperscript{(692.1)}
\textsuperscript{60:4.3} Las montañas más antiguas del mundo están situadas en Asia, Groenlandia y Europa septentrional, en medio de las de los antiguos sistemas este-oeste. Las montañas con una edad media se encuentran en el grupo que rodea al Pacífico y en el segundo sistema este-oeste europeo, que nació aproximadamente al mismo tiempo. Este gigantesco levantamiento tiene casi dieciséis mil kilómetros de largo, y se extiende desde Europa hasta las elevaciones terrestres de las Antillas. Las montañas más jóvenes se encuentran en el sistema de las Montañas Rocosas donde, durante épocas enteras, las elevaciones de tierras sólo se produjeron para ser cubiertas sucesivamente por el mar, aunque algunas de las tierras más altas permanecieron como islas. Después de formarse las montañas de edad media, se elevaron unas tierras altas realmente montañosas, y posteriormente estuvieron destinadas a ser esculpidas por el arte combinado de los elementos de la naturaleza, hasta convertirse en las Montañas Rocosas actuales.

\par
%\textsuperscript{(692.2)}
\textsuperscript{60:4.4} La región actual de las Montañas Rocosas de América del Norte no es la elevación terrestre original; aquella elevación había sido nivelada por la erosión desde hacía mucho tiempo, y luego fue elevada de nuevo. La actual cadena de montañas de la parte delantera es todo lo que queda de los restos de la cadena original que volvió a elevarse. Los picos Pikes y Longs son unos ejemplos destacados de esta actividad montañosa, que se extendió durante dos o más generaciones de la vida de las montañas. Estos dos picos conservaron sus cimas por encima del agua durante varias inundaciones anteriores.

\par
%\textsuperscript{(692.3)}
\textsuperscript{60:4.5} Tanto biológica como geológicamente, ésta fue una época memorable y activa en la tierra y bajo el agua. Los erizos de mar aumentaron, mientras que los corales y los crinoideos disminuyeron. Los amonites, que habían tenido una influencia predominante durante una época anterior, también declinaron rápidamente. En la tierra, los pinos y otros árboles modernos, incluyendo a las gigantescas secuoyas, reemplazaron en gran parte a los bosques de helechos. Hacia el final de este período, aunque los mamíferos placentarios no han evolucionado todavía, el escenario biológico está totalmente preparado para la aparición, en una época posterior, de los primeros antepasados de los futuros tipos de mamíferos.

\par
%\textsuperscript{(692.4)}
\textsuperscript{60:4.6} Así finaliza una larga era de la evolución mundial, que se extiende desde la primera aparición de la vida terrestre hasta los tiempos más recientes de los antepasados inmediatos de la especie humana y sus ramas colaterales. Esta época, llamada \textit{Cretácea}, abarca cincuenta millones de años y pone fin a la era premamífera de la vida terrestre, que se prolonga durante un período de cien millones de años y se conoce con el nombre de \textit{Mesozoica}.

\par
%\textsuperscript{(692.5)}
\textsuperscript{60:4.7} [Presentado por un Portador de Vida de Nebadon asignado a Satania, y que ahora ejerce su actividad en Urantia.]


\chapter{Documento 61. La era de los mamíferos en Urantia}
\par
%\textsuperscript{(693.1)}
\textsuperscript{61:0.1} LA ERA de los mamíferos se extiende desde la época de los primeros mamíferos placentarios hasta el final del período glacial, abarcando un poco menos de cincuenta millones de años.

\par
%\textsuperscript{(693.2)}
\textsuperscript{61:0.2} Durante esta época cenozoica, el paisaje del mundo ofrecía un aspecto atractivo ---colinas onduladas, amplios valles, anchos ríos y grandes bosques. Durante este período de tiempo, el istmo de Panamá se elevó y se hundió dos veces, y el puente terrestre del Estrecho de Bering hizo tres veces lo mismo. Los tipos de animales eran muchos y variados a la vez. Los árboles rebosaban de pájaros y el mundo entero era un paraíso para los animales, a pesar de la lucha constante por la supremacía de las especies animales en evolución.

\par
%\textsuperscript{(693.3)}
\textsuperscript{61:0.3} Los depósitos acumulados durante los cinco períodos de esta era de cincuenta millones de años contienen los anales fosilizados de las dinastías sucesivas de mamíferos, y conducen directamente hasta los tiempos de la aparición misma del hombre.

\section*{1. La nueva etapa de las tierras continentales --- La época de los primeros mamíferos}
\par
%\textsuperscript{(693.4)}
\textsuperscript{61:1.1} Hace \textit{50.000.000} de años, las zonas terrestres del mundo se encontraban en general por encima del agua o sólo ligeramente sumergidas. Las formaciones y los depósitos de este período son terrestres y marinos a la vez, pero principalmente terrestres. Durante un tiempo considerable, las tierras se elevaron de manera gradual pero fueron erosionadas simultáneamente por las aguas hasta los niveles más bajos, y llevadas hacia los mares.

\par
%\textsuperscript{(693.5)}
\textsuperscript{61:1.2} Al principio de este período, los mamíferos del tipo placentario aparecieron \textit{repentinamente} en América del Norte, constituyendo el desarrollo evolutivo más importante acaecido hasta ese momento. Anteriormente habían existido grupos de mamíferos no placentarios, pero este nuevo tipo surgió directa y \textit{repentinamente} del antepasado reptil preexistente cuyos descendientes habían sobrevivido durante los tiempos de la decadencia de los dinosaurios. El padre de los mamíferos placentarios fue un dinosaurio pequeño muy activo, carnívoro, del tipo saltador.

\par
%\textsuperscript{(693.6)}
\textsuperscript{61:1.3} Los instintos fundamentales de los mamíferos empezaron a manifestarse en estos tipos primitivos. Los mamíferos poseen, sobre todas las demás formas de vida animal, una inmensa ventaja para sobrevivir, por el hecho de que pueden:

\par
%\textsuperscript{(693.7)}
\textsuperscript{61:1.4} 1. Dar nacimiento a unas crías relativamente maduras y bien desarrolladas.

\par
%\textsuperscript{(693.8)}
\textsuperscript{61:1.5} 2. Alimentar, enseñar y proteger a sus crías con una atención afectuosa.

\par
%\textsuperscript{(693.9)}
\textsuperscript{61:1.6} 3. Emplear su capacidad cerebral superior para perpetuarse.

\par
%\textsuperscript{(693.10)}
\textsuperscript{61:1.7} 4. Utilizar su mayor agilidad para escapar de sus enemigos.

\par
%\textsuperscript{(693.11)}
\textsuperscript{61:1.8} 5. Aplicar su inteligencia superior para ajustarse y adaptarse al medio.

\par
%\textsuperscript{(694.1)}
\textsuperscript{61:1.9} Hace \textit{45.000.000} de años, las espinas dorsales de los continentes se elevaron, al mismo tiempo que se produjo un hundimiento generalizado de las regiones costeras. Los mamíferos evolucionaban con rapidez. Prosperó un pequeño tipo de mamífero reptil que ponía huevos, y los antepasados de los futuros canguros vagaban por Australia. Pronto hubo pequeños caballos, rinocerontes veloces, tapires con trompa, cerdos primitivos, ardillas, lémures, zarig\"ueyas y varias tribus de animales simiescos. Todos eran pequeños, primitivos y mejor adaptados para vivir en los bosques de las regiones montañosas. Unas grandes aves terrestres parecidas al avestruz se desarrollaron hasta alcanzar tres metros de altura y ponían huevos de veintitrés por treinta y tres centímetros. Fueron las antepasadas de las gigantescas aves de pasajeros más tardías, que eran tan extremadamente inteligentes y transportaban antiguamente a los seres humanos por los aires.

\par
%\textsuperscript{(694.2)}
\textsuperscript{61:1.10} Los mamíferos del principio de la era cenozoica vivían en la tierra, bajo el agua, en el aire y en las copas de los árboles. Tenían entre uno y once pares de glándulas mamarias y todos estaban cubiertos de abundante pelo. Al igual que los grupos que aparecerían más tarde, desarrollaban dos dentaduras sucesivas y poseían un gran cerebro en comparación con el tamaño de su cuerpo. Pero ninguna de las especies modernas figuraba entre ellos.

\par
%\textsuperscript{(694.3)}
\textsuperscript{61:1.11} Hace \textit{40.000.000} de años, las regiones terrestres del hemisferio norte empezaron a elevarse, lo que produjo nuevos y extensos sedimentos y otras actividades terrestres, incluyendo corrientes de lava, deformaciones, formaciones lacustres y erosiones.

\par
%\textsuperscript{(694.4)}
\textsuperscript{61:1.12} La mayor parte de Europa estuvo sumergida al final de esta época. Después de una ligera elevación de las tierras, el continente se cubrió de lagos y bahías. El Océano
Ártico se deslizó hacia el sur a través de la depresión de los Urales para comunicarse con el Mar Mediterráneo, que entonces se extendía hacia el norte, y las tierras altas de los Alpes, Cárpatos, Apeninos y Pirineos permanecieron por encima del agua como islas en medio del mar. El istmo de Panamá estaba emergido; los océanos Atlántico y Pacífico se encontraban separados. América del Norte estaba conectada con Asia por el puente terrestre del Estrecho de Bering, y con Europa a través de Groenlandia e Islandia. El circuito terrestre continental de las latitudes nórdicas sólo estaba cortado en los Estrechos de los Urales, que unían los mares árticos con un Mediterráneo más extenso.

\par
%\textsuperscript{(694.5)}
\textsuperscript{61:1.13} En las aguas europeas se depositaron grandes cantidades de caliza foraminífera. Actualmente, esta misma piedra se halla a una altura de 3.000 metros en los Alpes, a 4.900 metros en el Himalaya y a 6.000 metros en el Tíbet. Los depósitos de creta de este período se encuentran a lo largo de las costas de África y Australia, en la costa oeste de América del Sur y alrededor de las Antillas.

\par
%\textsuperscript{(694.6)}
\textsuperscript{61:1.14} A lo largo de todo este período llamado \textit{Eoceno}, la evolución de los mamíferos y otras formas de vida emparentadas continuó con poca o ninguna interrupción. América del Norte estaba entonces comunicada por tierra con todos los continentes, excepto con Australia, y el mundo se llenaba paulatinamente de una fauna de diversos tipos de mamíferos primitivos.

\section*{2. La etapa reciente de las inundaciones --- La época de los mamíferos avanzados}
\par
%\textsuperscript{(694.7)}
\textsuperscript{61:2.1} Este período estuvo caracterizado por una nueva y rápida evolución de los mamíferos placentarios, ya que las formas más progresivas de mamíferos se desarrollaron durante estos tiempos.

\par
%\textsuperscript{(694.8)}
\textsuperscript{61:2.2} Aunque los primeros mamíferos placentarios procedían de antepasados carnívoros, muy pronto se desarrollaron las ramificaciones herbívoras, y en poco tiempo surgieron también familias de mamíferos omnívoros. Las angiospermas constituían el alimento principal de los mamíferos que aumentaban con rapidez, pues la flora terrestre moderna, incluyendo a la mayoría de las plantas y de los árboles actuales, había aparecido durante los períodos anteriores.

\par
%\textsuperscript{(695.1)}
\textsuperscript{61:2.3} Hace \textit{35.000.000} de años que empezó la época del dominio mundial de los mamíferos placentarios. El puente terrestre meridional era espacioso y conectaba de nuevo al inmenso continente antártico con América del Sur, Sudáfrica y Australia. A pesar de que las tierras estaban concentradas en las altas latitudes, el clima mundial continuaba siendo relativamente suave, porque el tamaño de los mares tropicales se había acrecentado enormemente y las tierras no se habían elevado lo suficiente como para producir glaciares. Grandes torrentes de lava tuvieron lugar en Groenlandia e Islandia, y cierta cantidad de carbón se depositó entre estas capas.

\par
%\textsuperscript{(695.2)}
\textsuperscript{61:2.4} En la fauna del planeta estaban ocurriendo cambios importantes. La vida marina sufría grandes modificaciones; la mayor parte de las especies actuales de animales marinos existía ya, y los foraminíferos continuaban desempeñando un papel importante. Los insectos se parecían mucho a los de la era anterior. Los yacimientos fósiles de Florissant, en Colorado, pertenecen a los últimos años de estos tiempos lejanos. La mayoría de las familias de insectos que viven en la actualidad se remontan a este período, pero muchas de las que existían entonces están ahora extinguidas, aunque permanecen sus fósiles.

\par
%\textsuperscript{(695.3)}
\textsuperscript{61:2.5} En la tierra firme, esta época fue por excelencia la de la renovación y expansión de los mamíferos. Entre los primeros mamíferos más primitivos, más de cien especies se habían extinguido antes de que finalizara este período. Incluso los mamíferos de gran tamaño y de cerebro pequeño perecieron pronto. El cerebro y la agilidad habían reemplazado a las corazas y al tamaño en el progreso de la supervivencia animal. Como la familia de los dinosaurios estaba en decadencia, los mamíferos asumieron poco a poco el dominio de la Tierra, destruyendo rápidamente y por completo al resto de sus antepasados reptiles.

\par
%\textsuperscript{(695.4)}
\textsuperscript{61:2.6} Junto con la desaparición de los dinosaurios, otros cambios importantes se produjeron en las diversas ramas de la familia de los saurios. Los miembros supervivientes de las primeras familias reptiles son las tortugas, las serpientes y los cocodrilos, así como las venerables ranas, el único grupo representativo que queda de los antepasados más lejanos del hombre.

\par
%\textsuperscript{(695.5)}
\textsuperscript{61:2.7} Varios grupos de mamíferos tuvieron su origen en un animal único, hoy extinto. Esta criatura carnívora era una especie de cruce entre el gato y la foca; podía vivir en la tierra o en el agua y era extremadamente inteligente y muy activa. En Europa apareció por evolución el predecesor de la familia canina, y pronto dio origen a numerosas especies de perros pequeños. Alrededor de la misma época aparecieron los roedores, incluyendo a los castores, ardillas, ardillas terrestres, ratones y conejos, y pronto se convirtieron en una forma de vida importante; muy pocos cambios se han producido después en esta familia. Los últimos depósitos de este período contienen los restos fósiles de perros, gatos, mapaches y comadrejas en su forma ancestral.

\par
%\textsuperscript{(695.6)}
\textsuperscript{61:2.8} Hace \textit{30.000.000} de años empezaron a hacer su aparición los tipos de mamíferos modernos. La mayoría de los mamíferos había vivido anteriormente en los montes, pues eran del tipo montaraz; \textit{repentinamente} empezó la evolución del tipo ungulado o de las llanuras, las especies que pastan, diferenciándose de los carnívoros con garras. Estos animales que pastaban descendían de un antepasado no diferenciado que tenía cinco dedos en las patas y cuarenta y cuatro dientes, el cual desapareció antes del final de esta época. A lo largo de todo este período, la evolución de los ungulados no progresó más allá de la etapa de los tres dedos.

\par
%\textsuperscript{(695.7)}
\textsuperscript{61:2.9} El caballo, un ejemplo sobresaliente de la evolución, vivió durante estos tiempos tanto en América del Norte como en Europa, pero su desarrollo no concluyó por completo hasta la época glacial posterior. Aunque la familia de los rinocerontes apareció al final de este período, su mayor expansión la experimentó posteriormente. Una pequeña criatura porcina se desarrolló igualmente, y se convirtió en el antepasado de las numerosas especies de cerdos, pecaríes e hipopótamos. Los camellos y las llamas tuvieron su origen en América del Norte hacia mediados de este período e invadieron las planicies del oeste. Más tarde, las llamas emigraron a Sudamérica, los camellos a Europa, y las dos especies se extinguieron pronto en América del Norte, aunque algunos camellos sobrevivieron hasta la era glacial.

\par
%\textsuperscript{(696.1)}
\textsuperscript{61:2.10} Alrededor de esta época se produjo un hecho importante en el oeste de Norteamérica: Los antepasados primitivos de los antiguos lémures aparecieron por primera vez. Aunque a esta familia no se la puede considerar como verdaderos lémures, su aparición marcó el establecimiento de la línea de la que surgirían posteriormente los verdaderos lémures.

\par
%\textsuperscript{(696.2)}
\textsuperscript{61:2.11} Así como las serpientes terrestres de una época anterior se habían adaptado a los mares, una tribu completa de mamíferos placentarios abandonó ahora la tierra para establecer su residencia en los océanos. Y desde entonces han permanecido en el mar, dando origen a las ballenas, delfines, marsopas, focas y leones marinos modernos.

\par
%\textsuperscript{(696.3)}
\textsuperscript{61:2.12} Las aves continuaron desarrollándose en el planeta, pero con pocos cambios evolutivos importantes. La mayoría de las aves modernas existía ya, incluyendo a las gaviotas, garzas, flamencos, buitres, halcones, águilas, buhos, codornices y avestruces.

\par
%\textsuperscript{(696.4)}
\textsuperscript{61:2.13} Hacia el final de este período \textit{Oligoceno}, que abarca diez millones de años, la vida vegetal, al igual que la vida marina y los animales terrestres, había evolucionado mucho y se encontraba presente en la Tierra casi como lo está en la actualidad. Posteriormente ha aparecido una especialización considerable, pero las formas ancestrales de la mayoría de los seres vivos ya existían entonces.

\section*{3. La etapa de las montañas modernas --- La época del elefante y del caballo}
\par
%\textsuperscript{(696.5)}
\textsuperscript{61:3.1} La elevación de las tierras y la separación de los mares estaban cambiando lentamente la meteorología del mundo; el tiempo se enfriaba progresivamente, pero el clima era todavía templado. Las secuoyas y las magnolias crecían en Groenlandia, pero las plantas subtropicales empezaban a emigrar hacia el sur. Hacia el final de este período, estas plantas y estos árboles de los climas calurosos habían desaparecido ampliamente de las latitudes septentrionales, siendo reemplazados por plantas más resistentes y por los árboles de hoja caduca.

\par
%\textsuperscript{(696.6)}
\textsuperscript{61:3.2} Las variedades de hierbas aumentaron enormemente, y los dientes de muchas especies de mamíferos se modificaron de manera gradual para ajustarse a los del tipo actual de animales herbívoros.

\par
%\textsuperscript{(696.7)}
\textsuperscript{61:3.3} Hace \textit{25.000.000} de años que se produjo una ligera inmersión terrestre después de una larga época de elevación continental. La región de las Montañas Rocosas permaneció muy elevada, de manera que los materiales de erosión continuaron depositándose en todas las tierras bajas del este. Las Sierras volvieron a levantarse mucho; de hecho, han continuado elevándose desde entonces. La gran falla vertical de seis kilómetros y medio de la región de California data de estos tiempos.

\par
%\textsuperscript{(696.8)}
\textsuperscript{61:3.4} La época de hace \textit{20.000.000} de años fue en verdad la edad de oro de los mamíferos. El puente terrestre del Estrecho de Bering se hallaba por encima del agua, y muchos grupos de animales emigraron desde Asia hasta América del Norte, incluyendo a los mastodontes con cuatro colmillos, los rinocerontes de patas cortas y muchas variedades de la familia de los felinos.

\par
%\textsuperscript{(696.9)}
\textsuperscript{61:3.5} Los primeros ciervos aparecieron, y en poco tiempo América del Norte se llenó de rumiantes ---ciervos, bueyes, camellos, bisontes y diversas especies de rinocerontes--- pero los cerdos gigantes, que medían dos metros de alto, se extinguieron.

\par
%\textsuperscript{(697.1)}
\textsuperscript{61:3.6} Los enormes elefantes de este período y de los siguientes tenían un gran cerebro así como un gran cuerpo, y pronto invadieron el mundo entero, a excepción de Australia. Por una vez el mundo estaba dominado por un animal enorme con un cerebro lo suficientemente grande como para permitirle seguir adelante. Comparado con la vida sumamente inteligente de aquellos tiempos, ningún animal del tamaño de un elefante podría haber sobrevivido a menos que poseyera un cerebro de gran tamaño y de calidad superior. En lo que se refiere a la inteligencia y a la facultad de adaptación, el caballo es el único que se acerca al elefante, el cual sólo es superado por el hombre mismo. Aun así, de las cincuenta especies de elefantes que existían al principio de este período, sólo han sobrevivido dos.

\par
%\textsuperscript{(697.2)}
\textsuperscript{61:3.7} Hace \textit{15.000.000} de años, las regiones montañosas de Eurasia se estaban elevando, y había cierta actividad volcánica en todas estas regiones, pero no se podía comparar con los ríos de lava del hemisferio occidental. Estas condiciones inestables prevalecían en el mundo entero.

\par
%\textsuperscript{(697.3)}
\textsuperscript{61:3.8} El Estrecho de Gibraltar se cerró, y España quedó conectada con África por el viejo puente terrestre, pero el Mediterráneo desembocaba en el Atlántico a través de un estrecho canal que cruzaba toda Francia, y los picos montañosos y las tierras altas aparecían como si fueran islas por encima de este mar antiguo. Más tarde, estos mares europeos empezaron a retirarse. Más tarde aún, el Mediterráneo se unió con el Océano Índico, mientras que al final de este período la región de Suez se elevó de tal manera que el Mediterráneo se convirtió por un tiempo en un mar interior de agua salada.

\par
%\textsuperscript{(697.4)}
\textsuperscript{61:3.9} El puente terrestre de Islandia se sumergió, y las aguas árticas se mezclaron con las del Océano Atlántico. La costa atlántica de América del Norte se enfrió rápidamente, pero la costa del Pacífico seguía estando más caliente que en la actualidad. Las grandes corrientes oceánicas estaban en funcionamiento y afectaban al clima de una manera muy parecida a la de hoy.

\par
%\textsuperscript{(697.5)}
\textsuperscript{61:3.10} La vida de los mamíferos continuó evolucionando. Enormes manadas de caballos se juntaron con los camellos en las planicies occidentales de América del Norte; ésta fue, en verdad, la época de los caballos así como la de los elefantes. En calidad animal, el cerebro del caballo es el más cercano al del elefante, pero es indudablemente inferior en un aspecto: el caballo nunca ha vencido por completo su propensión profundamente arraigada a huir cuando está asustado. El caballo carece del control emocional del elefante, mientras que el elefante tiene la gran desventaja de su tamaño y de su falta de agilidad. Durante este período evolucionó un animal que se parecía un poco tanto al caballo como al elefante, pero pronto fue destruido por la familia de los felinos que se multiplicaba con rapidez.

\par
%\textsuperscript{(697.6)}
\textsuperscript{61:3.11} A medida que Urantia entra en la llamada <<época sin caballos>>, deberíais hacer una pausa para considerar lo que este animal significó para vuestros antepasados. Al principio, los hombres utilizaron el caballo para alimentarse, luego para viajar y más tarde para la agricultura y la guerra. El caballo ha servido a la humanidad durante mucho tiempo y ha jugado un papel importante en el desarrollo de la civilización humana.

\par
%\textsuperscript{(697.7)}
\textsuperscript{61:3.12} Los desarrollos biológicos de este período contribuyeron mucho a preparar el terreno para la aparición posterior del hombre. En Asia central, los verdaderos tipos de monos primitivos así como de gorilas evolucionaron a partir de un antecesor común ya extinto. Pero ninguna de estas especies está relacionada con la línea de los seres vivos que habrían de convertirse, posteriormente, en los antepasados de la raza humana.

\par
%\textsuperscript{(697.8)}
\textsuperscript{61:3.13} La familia canina estaba representada por diversos grupos, principalmente por los lobos y los zorros; la tribu felina, por las panteras y los grandes tigres con dientes de sable; estos últimos aparecieron por primera vez en América del Norte. Las familias felina y canina modernas aumentaron en el mundo entero. Las comadrejas, martas, nutrias y mapaches prosperaron y se desarrollaron en todas las latitudes septentrionales.

\par
%\textsuperscript{(698.1)}
\textsuperscript{61:3.14} Las aves continuaron evolucionando, aunque se produjeron pocos cambios apreciables. Los reptiles eran similares a los tipos modernos ---serpientes, cocodrilos y tortugas.

\par
%\textsuperscript{(698.2)}
\textsuperscript{61:3.15} Y así llegó a su fin un período memorable y muy interesante de la historia del mundo. Esta época del elefante y del caballo se conoce con el nombre de \textit{Mioceno}.

\section*{4. La etapa reciente de la elevación continental --- La última gran emigración de los mamíferos}
\par
%\textsuperscript{(698.3)}
\textsuperscript{61:4.1} Este período es el de la elevación preglacial de las tierras en América del Norte, Europa y Asia. La topografía de la Tierra se modificó profundamente. Nacieron cadenas de montañas, los ríos cambiaron su curso y los volcanes aislados estallaron en el mundo entero.

\par
%\textsuperscript{(698.4)}
\textsuperscript{61:4.2} Hace \textit{10.000.000} de años que empezó una época de depósitos terrestres locales diseminados por las tierras bajas de los continentes, pero la mayoría de estas sedimentaciones se desplazó posteriormente. En aquel momento, una gran parte de Europa estaba aún bajo el agua, incluyendo algunas zonas de Inglaterra, Bélgica y Francia, y el Mar Mediterráneo cubría una gran parte del norte de África. En América del Norte, unos extensos depósitos se acumularon al pie de las montañas, en los lagos y en las grandes cuencas terrestres. Estos depósitos sólo tienen un espesor medio de unos sesenta metros, están más o menos coloreados y contienen pocos fósiles. Dos grandes lagos de agua dulce existían en el oeste de Norteamérica. Las Sierras se estaban elevando y los Montes Shasta, Hood y Rainier estaban empezando su carrera. Pero el deslizamiento de América del Norte hacia la depresión atlántica no empezó hasta la época glacial posterior.

\par
%\textsuperscript{(698.5)}
\textsuperscript{61:4.3} Durante un corto período de tiempo, todas las tierras del mundo estuvieron unidas de nuevo a excepción de Australia, y entonces se produjo la última gran emigración animal a escala mundial. América del Norte estaba conectada con Sudamérica y Asia a la vez, y la vida animal procedió a intercambiarse libremente. Los perezosos, armadillos, antílopes y osos de Asia penetraron en América del Norte, mientras que los camellos norteamericanos se fueron a China. Los rinocerontes emigraron por el mundo entero a excepción de Australia y América del Sur, pero al final de este período se habían extinguido en el hemisferio occidental.

\par
%\textsuperscript{(698.6)}
\textsuperscript{61:4.4} En general, la vida del período anterior continuó evolucionando y extendiéndose. La familia felina dominaba la vida animal, y la vida marina se encontraba casi estancada. Muchos caballos tenían todavía tres dedos, pero los tipos modernos estaban a punto de llegar; las llamas y los camellos parecidos a las jirafas se mezclaban con los caballos en los pastizales de las llanuras. La jirafa apareció en
África con un cuello tan largo como el de hoy. En América del Sur evolucionaron los perezosos, los armadillos, los osos hormigueros y los tipos sudamericanos de monos primitivos. Antes de que los continentes se quedaran definitivamente aislados, los mastodontes, aquellos animales macizos, emigraron a todas partes excepto a Australia.

\par
%\textsuperscript{(698.7)}
\textsuperscript{61:4.5} Hace \textit{5.000.000} de años, el caballo alcanzó su estado de evolución actual y emigró desde América del Norte hacia el mundo entero. Pero el caballo se había extinguido en su continente de origen mucho antes de que llegara el hombre rojo.

\par
%\textsuperscript{(698.8)}
\textsuperscript{61:4.6} El clima se iba enfriando paulatinamente, y las plantas terrestres se desplazaban lentamente hacia el sur. Al principio, el creciente frío en el norte fue el que detuvo las emigraciones animales por los istmos nórdicos; estos puentes terrestres norteamericanos se hundieron posteriormente. Poco después, el lazo terrestre entre África y América del Sur se sumergió definitivamente, y el hemisferio occidental se quedó aislado de manera muy similar a como se encuentra hoy. A partir de este momento empezaron a desarrollarse unos tipos de vida distintos en el hemisferio oriental y en el hemisferio occidental.

\par
%\textsuperscript{(699.1)}
\textsuperscript{61:4.7} Y así se cerró este período de casi diez millones de años, sin que el antepasado del hombre hubiera aparecido todavía. A esta época se le conoce generalmente con el nombre de \textit{Plioceno}.

\section*{5. El principio de la época glacial}
\par
%\textsuperscript{(699.2)}
\textsuperscript{61:5.1} Al final del período anterior, las tierras de la parte nordeste de América del Norte y de Europa septentrional estaban sumamente elevadas en una gran proporción; amplias zonas de Norteamérica alcanzaban una altitud de 9.000 metros y más. En estas regiones nórdicas habían prevalecido anteriormente unos climas templados, y todas las aguas árticas estuvieron expuestas a la evaporación; estas aguas continuaron estando libres de hielo casi hasta el final del período glacial.

\par
%\textsuperscript{(699.3)}
\textsuperscript{61:5.2} Las corrientes oceánicas se desplazaron al mismo tiempo que se producían estas elevaciones terrestres, y los vientos estacionales cambiaron de dirección. A consecuencia de los movimientos de la atmósfera fuertemente saturada, estas condiciones produjeron finalmente una precipitación casi constante de humedad sobre las tierras altas septentrionales. La nieve empezó a caer sobre estas regiones elevadas, y por tanto frías, y continuó cayendo hasta alcanzar un espesor de 6.000 metros. Las zonas donde la nieve era más espesa, unido a la altitud, determinaron los puntos centrales de los flujos que se produjeron posteriormente debido a la presión glacial. El período glacial persistió mientras esta precipitación excesiva continuó cubriendo las tierras altas del norte con este enorme manto de nieve, que pronto se transformó en hielo compacto pero móvil.

\par
%\textsuperscript{(699.4)}
\textsuperscript{61:5.3} Todas las grandes capas de hielo de este período estaban situadas en las tierras altas, no en las regiones montañosas donde se encuentran hoy. La mitad del hielo glacial se encontraba en América del Norte, una cuarta parte en Eurasia y otra cuarta parte en otros lugares, principalmente en la Antártida. África se hallaba poco afectada por los hielos, pero Australia estaba casi totalmente cubierta por el manto de hielo antártico.

\par
%\textsuperscript{(699.5)}
\textsuperscript{61:5.4} Las regiones nórdicas de este mundo han sufrido seis invasiones glaciales distintas y separadas, aunque hubo decenas de avances y de retrocesos en unión con la actividad de cada capa de hielo individual. Los hielos de América del Norte se acumularon en dos centros, y más tarde en tres. Groenlandia estaba cubierta de hielo e Islandia completamente sepultada bajo un flujo helado. En Europa, el hielo cubrió en diversas ocasiones las Islas Británicas, a excepción de la costa meridional de Inglaterra, y se extendió por Europa occidental hasta Francia.

\par
%\textsuperscript{(699.6)}
\textsuperscript{61:5.5} Hace \textit{2.000.000} de años, el primer glaciar norteamericano empezó a avanzar hacia el sur. La edad de hielo estaba ahora en gestación, y este glaciar empleó casi un millón de años en avanzar desde los centros nórdicos de presión y en retirarse de nuevo hacia ellos. La capa central de hielo se extendía hacia el sur hasta Kansas; los centros glaciares del este y del oeste no eran entonces tan extensos.

\par
%\textsuperscript{(699.7)}
\textsuperscript{61:5.6} Hace \textit{1.500.000} años, el primer gran glaciar se estaba retirando hacia el norte. Mientras tanto, enormes cantidades de nieve habían caído sobre Groenlandia y la parte nordeste de América del Norte, y poco tiempo después esta masa oriental de hielo empezó a deslizarse hacia el sur. Ésta fue la segunda invasión glacial.

\par
%\textsuperscript{(699.8)}
\textsuperscript{61:5.7} Estas dos primeras invasiones de hielo no fueron muy extensas en Eurasia. Durante estas épocas primitivas del período glacial, América del Norte estaba plagada de mastodontes, mamuts lanudos, caballos, camellos, ciervos, bueyes almizcleros, bisontes, perezosos terrestres, castores gigantes, tigres con dientes de sable, perezosos tan grandes como elefantes y muchos grupos de las familias felina y canina. Pero a partir de esta época se fueron reduciendo rápidamente a consecuencia del frío creciente del período glacial. Hacia el final de la edad de hielo, la mayoría de estas especies animales se habían extinguido en Norteamérica.

\par
%\textsuperscript{(700.1)}
\textsuperscript{61:5.8} La vida terrestre y acuática que se encontraba alejada del hielo había cambiado poco en el mundo. Entre las invasiones glaciales, el clima era casi tan templado como en la actualidad, quizás un poco más caluroso. Después de todo, los glaciares eran fenómenos locales, aunque se extendieron hasta cubrir inmensas superficies. El clima costero varió enormemente entre los períodos de inactividad glacial y los períodos en que los enormes icebergs se deslizaban lejos de la costa de Maine hacia el Atlántico, o salían por Puget Sound hacia el Pacífico, o bien se desplomaban con estruendo en los fiordos noruegos camino del Mar del Norte.

\section*{6. El hombre primitivo en la época glacial}
\par
%\textsuperscript{(700.2)}
\textsuperscript{61:6.1} El gran acontecimiento de este período glacial fue la aparición por evolución del hombre primitivo\footnote{\textit{Hombre primitivo}: Gn 1:26-27; 2:7.}. Un poco hacia el oeste de la India, en una tierra ahora sumergida y entre los descendientes de los antiguos tipos de lémures norteamericanos que emigraron a Asia, los mamíferos precursores del hombre aparecieron \textit{repentinamente}. Estos pequeños animales caminaban principalmente sobre sus patas traseras; poseían un cerebro grande en proporción a su tamaño y en comparación con el cerebro de otros animales. En la septuagésima generación de esta orden de vida, un nuevo grupo de animales superiores se diferenció \textit{repentinamente}. Estos nuevos mamíferos intermedios ---que eran casi el doble de grandes que sus predecesores y poseían proporcionalmente una mayor capacidad cerebral--- apenas acababan de establecerse bien cuando los primates, la tercera mutación vital, aparecieron \textit{repentinamente}. (Al mismo tiempo, un desarrollo retrógrado dentro de la familia de los mamíferos intermedios dio origen a los antepasados de los simios; desde aquel día hasta la fecha, la rama humana ha progresado mediante una evolución paulatina, mientras que las tribus simias han permanecido estacionarias o han retrocedido realmente.)

\par
%\textsuperscript{(700.3)}
\textsuperscript{61:6.2} Hace \textit{1.000.000} de años, Urantia fue registrada como \textit{mundo habitado}. Una mutación dentro de la familia de los primates que progresaban produjo \textit{repentinamente} dos seres humanos primitivos, los verdaderos antepasados de la humanidad.

\par
%\textsuperscript{(700.4)}
\textsuperscript{61:6.3} Este acontecimiento sucedió casi en la época en que empezó el tercer avance glacial; se puede observar así que vuestros primeros antepasados nacieron y se criaron en un entorno estimulante, vigorizante y difícil. Los únicos supervivientes de estos aborígenes de Urantia, los esquimales, prefieren vivir todavía hoy en los climas nórdicos muy fríos.

\par
%\textsuperscript{(700.5)}
\textsuperscript{61:6.4} Los seres humanos no habitaron en el hemisferio occidental hasta cerca del final de la era glacial. Pero durante las épocas interglaciares pasaron hacia el oeste rodeando el Mediterráneo y pronto invadieron el continente europeo. En las cuevas de Europa occidental se pueden encontrar huesos humanos mezclados con los restos de animales árticos y tropicales, lo que demuestra que el hombre vivió en estas regiones durante las últimas épocas del avance y del retroceso de los glaciares.

\section*{7. La continuación de la época glacial}
\par
%\textsuperscript{(700.6)}
\textsuperscript{61:7.1} A lo largo de todo el período glacial continuaron desarrollándose otras actividades, pero la acción de los hielos eclipsa todos los demás fenómenos en las latitudes nórdicas. Ninguna otra actividad terrestre deja unas pruebas tan características sobre la topografía. Los cantos rodados distintivos y las hendiduras superficiales tales como las marmitas de gigante, los lagos, las piedras desplazadas y las rocas pulverizadas, no están relacionados con ningún otro fenómeno de la naturaleza. El hielo es también responsable de esos abultamientos suaves, u ondulaciones del terreno, conocidos con el nombre de drumlins. A medida que avanza un glaciar, desplaza los ríos y modifica por completo la faz de la Tierra. Únicamente los glaciares dejan tras ellos unos derrubios reveladores ---las morrenas básicas, laterales y terminales. Estos derrubios, sobre todo las morrenas básicas, se extienden en Norteamérica desde la costa oriental hacia el norte y el oeste, y también se encuentran en Europa y Siberia.

\par
%\textsuperscript{(701.1)}
\textsuperscript{61:7.2} Hace \textit{750.000} años, la cuarta capa glacial formada por la unión de los campos de hielo del centro y del este de América del Norte estaba camino del sur; en su punto culminante alcanzó el sur de Illinois y desplazó el río Misisipí 80 kilómetros hacia el oeste, mientras que la parte oriental de la capa se extendió hacia el sur hasta el río Ohio y el centro de Pensilvania.

\par
%\textsuperscript{(701.2)}
\textsuperscript{61:7.3} En Asia, la capa de hielo siberiana llevó a cabo su invasión más meridional, mientras que el hielo que avanzaba en Europa se detuvo justamente delante de la barrera montañosa de los Alpes.

\par
%\textsuperscript{(701.3)}
\textsuperscript{61:7.4} Hace \textit{500.000} años, durante el quinto avance de los hielos, un nuevo acontecimiento aceleró el curso de la evolución humana. \textit{Repentinamente}, y en una sola generación, las seis razas de color aparecieron por mutación a partir de la familia humana aborigen. Esta fecha tiene una doble importancia puesto que señala también la llegada del Príncipe Planetario.

\par
%\textsuperscript{(701.4)}
\textsuperscript{61:7.5} En América del Norte, el quinto glaciar que avanzaba consistía en una invasión combinada de los tres centros de hielo. Sin embargo, el lóbulo oriental sólo se extendió a corta distancia por debajo del valle del San Lorenzo, y la capa de hielo occidental avanzó muy poco hacia el sur. Pero el lóbulo central alcanzó el sur hasta cubrir la mayor parte del estado de Iowa. En Europa, esta invasión de hielo no fue tan extensa como la anterior.

\par
%\textsuperscript{(701.5)}
\textsuperscript{61:7.6} Hace \textit{250.000} años que empezó la sexta y última glaciación. A pesar del hecho de que las tierras altas del norte habían empezado a hundirse ligeramente, durante este período se acumularon los mayores depósitos de nieve en los campos helados septentrionales.

\par
%\textsuperscript{(701.6)}
\textsuperscript{61:7.7} En el transcurso de esta invasión, las tres grandes capas glaciares se unieron en una sola inmensa masa de hielo, y todas las montañas del oeste participaron en esta actividad glacial. De todas las invasiones glaciares, ésta fue la mayor que se produjo en Norteamérica; el hielo se desplazó hacia el sur hasta una distancia de más de dos mil cuatrocientos kilómetros de sus centros de presión, y América del Norte sufrió sus temperaturas más bajas.

\par
%\textsuperscript{(701.7)}
\textsuperscript{61:7.8} Hace \textit{200.000} años, durante el avance del último glaciar, sucedió un episodio que tuvo mucho que ver con la marcha de los acontecimientos en Urantia ---la rebelión de Lucifer.

\par
%\textsuperscript{(701.8)}
\textsuperscript{61:7.9} Hace \textit{150.000} años, el sexto y último glaciar alcanzó los puntos más lejanos en su avance hacia el sur; la capa de hielo occidental atravesaba justo la frontera canadiense, la central llegaba hasta Kansas, Missouri e Illinois, y la capa oriental que avanzaba hacia el sur cubría la mayor parte de Pensilvania y Ohio.

\par
%\textsuperscript{(701.9)}
\textsuperscript{61:7.10} Éste es el glaciar que dejó las numerosas lenguas, o lóbulos de hielo, que esculpieron los lagos actuales, grandes y pequeños. El sistema norteamericano de los Grandes Lagos se produjo durante su retroceso. Los geólogos de Urantia han deducido con mucha exactitud las diversas etapas de esta evolución y han conjeturado correctamente que estas masas de agua desembocaron, en épocas diferentes, primero en el valle del Misisipí, luego hacia el este en el valle del Hudson, y finalmente, a través de una ruta septentrional, en el San Lorenzo. Hace treinta y siete mil años que el sistema comunicante de los Grandes Lagos empezó a verter sus aguas en la vía actual del Niágara.

\par
%\textsuperscript{(702.1)}
\textsuperscript{61:7.11} Hace \textit{100.000} años, las inmensas capas de hielo polares empezaron a formarse durante el retroceso del último glaciar, y el centro de las acumulaciones de hielo se desplazó considerablemente hacia el norte. Mientras las regiones polares continúen cubiertas de hielo, es muy difícil que se produzca otra época glacial, independientemente de las elevaciones terrestres o de las modificaciones de las corrientes oceánicas que tengan lugar en el futuro.

\par
%\textsuperscript{(702.2)}
\textsuperscript{61:7.12} Este último glaciar estuvo avanzando durante cien mil años, y necesitó la misma cantidad de tiempo para completar su retroceso hacia el norte. Las regiones templadas han estado libres de los hielos desde hace poco más de cincuenta mil años.

\par
%\textsuperscript{(702.3)}
\textsuperscript{61:7.13} Los rigores del período glacial destruyeron numerosas especies y cambiaron radicalmente muchas otras. Muchas especies fueron profundamente cribadas durante las emigraciones de un lado para otro que el avance y el retroceso de los hielos hicieron necesarias. Los animales que siguieron a los glaciares de acá para allá sobre la Tierra fueron el oso, el bisonte, el reno, el buey almizclero, el mamut y el mastodonte.

\par
%\textsuperscript{(702.4)}
\textsuperscript{61:7.14} El mamut buscaba las praderas abiertas, pero el mastodonte prefería los márgenes abrigados de las regiones boscosas. Hasta una fecha reciente, el mamut estuvo vagando desde Méjico hasta Canadá; la variedad siberiana se cubrió de lana. El mastodonte permaneció en América del Norte hasta que fue exterminado por el hombre rojo de manera muy similar a como el hombre blanco destruyó más tarde al bisonte.

\par
%\textsuperscript{(702.5)}
\textsuperscript{61:7.15} Durante la última glaciación, el caballo, el tapir, la llama y el tigre con dientes de sable se extinguieron en América del Norte. Fueron reemplazados por los perezosos, los armadillos y los cerdos de agua que subieron desde América del Sur.

\par
%\textsuperscript{(702.6)}
\textsuperscript{61:7.16} Las emigraciones forzosas de la vida ante el avance de los hielos condujeron a una mezcla extraordinaria de plantas y de animales. Después del retroceso de la última invasión glacial, muchas especies árticas, tanto animales como vegetales, quedaron atrapadas en lo alto de algunos picos montañosos donde se habían refugiado para escapar de la destrucción por el glaciar. Por eso, estas plantas y estos animales desplazados se pueden encontrar hoy en lo alto de los Alpes en Europa e incluso en los Montes Apalaches de América del Norte.

\par
%\textsuperscript{(702.7)}
\textsuperscript{61:7.17} La época glacial es el último período geológico completo, el llamado \textit{Pleistoceno}, y tuvo una duración de más de dos millones de años.

\par
%\textsuperscript{(702.8)}
\textsuperscript{61:7.18} Hace \textit{35.000} años que terminó la gran época glacial, excepto en las regiones polares del planeta. Esta fecha también es significativa porque se aproxima mucho a la de la llegada de un Hijo y una Hija Materiales y al principio de la dispensación adámica, que coincide aproximadamente con el principio del período Holoceno o postglacial.

\par
%\textsuperscript{(702.9)}
\textsuperscript{61:7.19} Esta narración se extiende desde el nacimiento de los mamíferos hasta el retroceso de los hielos y los tiempos históricos, abarcando un período de casi cincuenta millones de años. Es el último período geológico ---el actualmente vigente--- y vuestros investigadores lo conocen con el nombre de \textit{Cenozoico} o era de los tiempos recientes.

\par
%\textsuperscript{(702.10)}
\textsuperscript{61:7.20} [Patrocinado por un Portador de Vida residente.]


\chapter{Documento 62. Las razas en los albores del hombre primitivo}
\par
%\textsuperscript{(703.1)}
\textsuperscript{62:0.1} HACE casi un millón de años, los antepasados inmediatos del género humano hicieron su aparición mediante tres mutaciones repentinas y sucesivas en el tronco primitivo del tipo lémur de mamíferos placentarios. Los factores dominantes de estos lémures primitivos procedían del plasma vital evolutivo del grupo americano occidental o más reciente. Pero antes de establecer la línea directa del linaje humano, esta raza fue reforzada por las aportaciones de la implantación central de vida que había evolucionado en
África. El grupo oriental de vida contribuyó poco o nada a la producción efectiva de la especie humana.

\section*{1. Los tipos primitivos de lémures}
\par
%\textsuperscript{(703.2)}
\textsuperscript{62:1.1} Los lémures primitivos implicados en la ascendencia de la especie humana no estaban directamente emparentados con las tribus preexistentes de gibones y monos que vivían entonces en Eurasia y África del norte, y cuya progenie ha sobrevivido hasta la actualidad. Tampoco eran los descendientes del tipo moderno de lémur, aunque los dos procedían de un antepasado común que se había extinguido hacía mucho tiempo.

\par
%\textsuperscript{(703.3)}
\textsuperscript{62:1.2} Mientras estos lémures primitivos evolucionaban en el hemisferio occidental, los mamíferos antepasados directos de la humanidad se establecían en el suroeste de Asia, en la zona original de la implantación central de vida, pero en las fronteras de las regiones orientales. Hacía varios millones de años que los lémures del tipo norteamericano habían emigrado hacia el oeste por el puente terrestre de Bering, y habían avanzando lentamente hacia el suroeste a lo largo de la costa asiática. Estas tribus migratorias alcanzaron finalmente la región salubre situada entre el Mar Mediterráneo, entonces mucho más extenso, y las regiones montañosas en vías de elevarse de la península índica. En estas tierras situadas al oeste de la India se unieron con otras cepas favorables, y establecieron así la ascendencia de la raza humana.

\par
%\textsuperscript{(703.4)}
\textsuperscript{62:1.3} Con el paso del tiempo, el litoral de la India situado al suroeste de las montañas se sumergió progresivamente, y la vida de esta región quedó completamente aislada. Esta península mesopotámica o pérsica no tenía ninguna vía de acceso o de huida, salvo por el norte, y ésta fue cortada repetidas veces por las invasiones glaciares que se dirigían hacia el sur. Fue en esta zona, por aquel entonces casi paradisiaca, y a partir de los descendientes superiores de este tipo de mamíferos lémures, donde surgieron dos grandes grupos, las tribus simias de los tiempos modernos y la especie humana actual.

\section*{2. Los mamíferos precursores}
\par
%\textsuperscript{(703.5)}
\textsuperscript{62:2.1} Hace poco más de un millón de años que aparecieron \textit{repentinamente} los mamíferos precursores mesopotámicos, los descendientes directos del tipo de lémur norteamericano de mamíferos placentarios. Eran unas criaturas pequeñas y activas, que medían casi un metro de altura; y aunque no caminaban habitualmente sobre las patas traseras, podían mantenerse fácilmente de pie. Eran peludas y ágiles y chillaban a la manera de los monos, pero al contrario que las tribus simias, eran carnívoras. Tenían un pulgar oponible primitivo, así como un dedo gordo prensil en el pie extremadamente útil. A partir de este momento, las especies prehumanas desarrollaron sucesivamente el pulgar oponible y fueron perdiendo de manera progresiva el poder prensor del dedo gordo del pie. Las tribus posteriores de monos conservaron el dedo gordo prensil del pie, pero nunca desarrollaron el tipo de pulgar humano.

\par
%\textsuperscript{(704.1)}
\textsuperscript{62:2.2} Estos mamíferos precursores alcanzaban su pleno desarrollo a los tres o cuatro años de edad, y la duración potencial de su vida era por término medio de unos veinte años. Por regla general tenían una sola cría a la vez, aunque a veces nacían gemelos.

\par
%\textsuperscript{(704.2)}
\textsuperscript{62:2.3} Los miembros de esta nueva especie tenían un cerebro más grande, en comparación con su tamaño, que cualquier otro animal que hubiera vivido hasta entonces en la Tierra. Experimentaban una gran parte de las emociones y compartían un buen número de los instintos que caracterizarían más tarde al hombre primitivo; eran extremadamente curiosos y manifestaban una gran alegría cuando tenían éxito en cualquier empresa. El apetito por la comida y el deseo sexual estaban bien desarrollados, y manifestaban una selección sexual definida mediante una forma tosca de cortejo y elección de la pareja. Eran capaces de luchar ferozmente para defender a los suyos; eran bastante tiernos en sus relaciones familiares, y poseían un sentido de la autodegradación que rayaba en la verg\"uenza y el remordimiento. Eran muy afectuosos y de una fidelidad conmovedora hacia su pareja, pero si las circunstancias los separaban, escogían una nueva compañía.

\par
%\textsuperscript{(704.3)}
\textsuperscript{62:2.4} Como eran de pequeña estatura y tenían una mente aguda para darse cuenta de los peligros de su hábitat boscoso, desarrollaron un temor extraordinario que les condujo a tomar las prudentes medidas de precaución que tanto contribuyeron a su supervivencia, entre ellas la construcción de toscos refugios en lo alto de los árboles, lo cual eliminaba muchos peligros de la vida en el suelo. El origen de las tendencias al miedo que tiene la humanidad data más específicamente de estos tiempos.

\par
%\textsuperscript{(704.4)}
\textsuperscript{62:2.5} Estos mamíferos precursores desarrollaron un espíritu tribal que nunca se había manifestado anteriormente. Eran en verdad muy gregarios, pero sin embargo se mostraban extremadamente belicosos cuando eran molestados de alguna manera en las ocupaciones corrientes de su vida rutinaria; y ponían de manifiesto un temperamento fogoso cuando se despertaba toda su cólera. Sin embargo, su naturaleza belicosa sirvió para una finalidad favorable; los grupos superiores no dudaban en hacer la guerra a sus vecinos inferiores, y de esta manera la especie mejoró paulatinamente mediante la supervivencia selectiva. Muy pronto dominaron la vida de las criaturas más pequeñas de esta región, y muy pocas de las antiguas tribus simiescas no carnívoras lograron sobrevivir.

\par
%\textsuperscript{(704.5)}
\textsuperscript{62:2.6} Estos pequeños animales agresivos se multiplicaron y se diseminaron por la península mesopotámica durante más de mil años, mejorando constantemente el tipo físico y la inteligencia general. Exactamente setenta generaciones después de que esta nueva tribu se hubiera originado en el tipo superior de antecesores lémures, se produjo el siguiente acontecimiento que hizo época ---la \textit{repentina} diferenciación de los predecesores de la siguiente etapa vital en la evolución de los seres humanos en Urantia.

\section*{3. Los mamíferos intermedios}
\par
%\textsuperscript{(704.6)}
\textsuperscript{62:3.1} Al principio de la carrera de los mamíferos precursores, dos gemelos, un macho y una hembra, nacieron en la copa de un árbol en la morada de una pareja superior de estas ágiles criaturas. Comparadas con sus antepasados, eran unas pequeñas criaturas realmente hermosas. Tenían poco pelo en el cuerpo, pero esto no era ninguna desventaja puesto que vivían en un clima cálido y uniforme.

\par
%\textsuperscript{(705.1)}
\textsuperscript{62:3.2} Estas crías llegaron a medir poco más de un metro veinte de altura. Eran en todos los aspectos más grandes que sus progenitores, con piernas más largas y brazos más cortos. Tenían unos pulgares oponibles casi perfectos, que se adaptaban más o menos igual de bien a los trabajos más diversos que el pulgar de los humanos actuales. Caminaban erguidos, pues tenían unos pies casi tan adecuados para andar como los de las razas humanas posteriores.

\par
%\textsuperscript{(705.2)}
\textsuperscript{62:3.3} Su cerebro era inferior al de los seres humanos, y más pequeño, pero muy superior al de sus antepasados y relativamente mucho más grande. Los gemelos mostraron muy pronto una inteligencia superior y al poco tiempo fueron reconocidos como jefes de toda la tribu de los mamíferos precursores, instituyendo realmente una forma primitiva de organización social y una tosca división económica del trabajo. Este hermano y su hermana se aparearon y pronto disfrutaron de la compañía de veintiún hijos muy parecidos a ellos mismos, todos con más de un metro veinte de altura y superiores en todos los aspectos a la especie ancestral. Este nuevo grupo formó el núcleo de los mamíferos intermedios.

\par
%\textsuperscript{(705.3)}
\textsuperscript{62:3.4} Cuando aumentó el número de miembros de este grupo nuevo y superior, estalló la guerra, una guerra implacable; y cuando la terrible contienda terminó, no quedó vivo ni un solo individuo de la raza ancestral preexistente de mamíferos precursores. Los vástagos de la especie, menos numerosos pero más poderosos e inteligentes, habían sobrevivido a expensas de sus antepasados.

\par
%\textsuperscript{(705.4)}
\textsuperscript{62:3.5} Estas criaturas se convirtieron entonces en el terror de esta parte del mundo durante cerca de quince mil años (seiscientas generaciones). Todos los grandes animales feroces de los tiempos pasados habían perecido. Las grandes bestias originarias de estas regiones no eran carnívoras, y las especies más grandes de la familia felina, los leones y los tigres, aún no habían invadido este rincón particularmente protegido de la superficie de la Tierra. Por consiguiente, estos mamíferos intermedios se envalentonaron y subyugaron toda su parcela de la creación.

\par
%\textsuperscript{(705.5)}
\textsuperscript{62:3.6} Comparados con la especie ancestral, los mamíferos intermedios representaban una mejora en todos los sentidos. Incluso la duración potencial de su vida era más larga, siendo de unos veinticinco años. En esta nueva especie aparecieron algunas características humanas rudimentarias. Además de las propensiones innatas que mostraron sus antepasados, estos mamíferos intermedios eran capaces de manifestar repugnancia en ciertas situaciones repulsivas. Poseían también un instinto de atesoramiento bien definido; escondían la comida para utilizarla posteriormente y eran muy dados a coleccionar guijarros lisos y redondos y ciertos tipos de piedras redondas que les servían como munición defensiva y ofensiva.

\par
%\textsuperscript{(705.6)}
\textsuperscript{62:3.7} Estos mamíferos intermedios fueron los primeros que manifestaron una clara propensión a la construcción, tal como lo demuestra la rivalidad que tenían edificando casas en las copas de los árboles así como refugios subterráneos llenos de túneles; fueron la primera especie de mamíferos que buscó la seguridad tanto en los refugios arbóreos como subterráneos. Abandonaron en gran parte los árboles como lugar de residencia, viviendo en el suelo durante el día y durmiendo por la noche en las copas de los árboles.

\par
%\textsuperscript{(705.7)}
\textsuperscript{62:3.8} A medida que el tiempo pasaba, el aumento natural del número de miembros terminó por ocasionar una grave competición por la comida y una gran rivalidad sexual, lo que culminó en una serie de batallas de aniquilación mutua que destruyó casi toda la especie. Estas luchas continuaron hasta que sólo quedó vivo un grupo de menos de cien individuos. La paz reinó una vez más, y esta tribu solitaria superviviente volvió a construir sus dormitorios en las copas de los árboles y reanudó de nuevo una existencia normal y semipacífica.

\par
%\textsuperscript{(705.8)}
\textsuperscript{62:3.9} Apenas podéis imaginar cuán cerca estuvieron de la extinción una y otra vez vuestros antepasados prehumanos. Si la rana ancestral de toda la humanidad hubiera saltado en cierta ocasión cinco centímetros menos, todo el curso de la evolución hubiera cambiado notablemente. La madre directa, parecida a los lémures, de la especie de los mamíferos precursores, se libró por los pelos de la muerte al menos cinco veces antes de dar a luz al padre del nuevo orden de mamíferos superiores. Pero el mayor peligro de todos se produjo cuando un rayo cayó sobre el árbol donde dormía la futura madre de los gemelos primates. Los dos padres mamíferos intermedios sufrieron una fuerte conmoción y graves quemaduras, y tres de sus siete hijos murieron fulminados por este rayo caído del cielo. Estos animales en evolución eran casi supersticiosos. Esta pareja, cuyo refugio en la copa del árbol había sido golpeado por el rayo, era en realidad la pareja dirigente del grupo más progresivo de la especie de los mamíferos intermedios. Siguiendo su ejemplo, más de la mitad de la tribu, que incluía a las familias más inteligentes, se alejó a unos tres kilómetros de este lugar y empezó a construir sus nuevos domicilios en la copa de los árboles y nuevos refugios subterráneos ---sus guaridas transitorias en caso de peligro repentino.

\par
%\textsuperscript{(706.1)}
\textsuperscript{62:3.10} Poco después de terminar su casa, esta pareja veterana de tantas batallas se convirtió en los padres orgullosos de unos gemelos, los animales más interesantes e importantes que habían nacido en el mundo hasta ese momento, pues eran los primeros representantes de la nueva especie de los \textit{Primates}, y constituían la siguiente etapa vital de la evolución prehumana.

\par
%\textsuperscript{(706.2)}
\textsuperscript{62:3.11} En la misma época en que nacieron estos gemelos primates, otra pareja ---un macho y una hembra particularmente retrasados de la tribu de los mamíferos intermedios, una pareja mental y físicamente inferior--- también dio a luz a unos gemelos. Estos gemelos, un macho y una hembra, eran indiferentes a las conquistas; sólo se ocupaban de conseguir comida, y como no comían carne, pronto perdieron todo interés por buscar presas. Estos gemelos retrasados fueron los fundadores de las tribus simias modernas. Sus descendientes buscaron las regiones meridionales más cálidas, con sus climas templados y su abundancia en frutas tropicales, donde han continuado viviendo de manera muy parecida a la de aquella época, a excepción de las ramas que se aparearon con los tipos anteriores de gibones y monos, y que se deterioraron enormemente a consecuencia de ello.

\par
%\textsuperscript{(706.3)}
\textsuperscript{62:3.12} Así se puede ver fácilmente que el único parentesco entre el hombre y el mono reside en el hecho de que los dos descienden de los mamíferos intermedios, una tribu en la que se produjo el nacimiento contemporáneo y la separación posterior de dos parejas de gemelos: la pareja inferior destinada a engendrar los tipos modernos de monos, babuinos, chimpancés y gorilas, y la pareja superior destinada a continuar la línea ascendente que produjo por evolución al hombre mismo.

\par
%\textsuperscript{(706.4)}
\textsuperscript{62:3.13} El hombre moderno y los simios surgieron de la misma tribu y de la misma especie, pero no de los mismos padres. Los antepasados del hombre descendían de la cepa superior del resto seleccionado de esta tribu de mamíferos intermedios, mientras que los simios modernos (excepto algunos tipos preexistentes de lémures, gibones, monos y otras criaturas similares) son los descendientes de la pareja más inferior de este grupo de mamíferos intermedios, una pareja que sólo sobrevivió porque, en el transcurso de la última batalla encarnizada de su tribu, se ocultaron durante más de dos semanas en un refugio subterráneo donde almacenaban los alimentos, y no salieron hasta mucho después de que hubieran cesado las hostilidades.

\section*{4. Los primates}
\par
%\textsuperscript{(706.5)}
\textsuperscript{62:4.1} Regresemos al nacimiento de los gemelos superiores, un macho y una hembra, los dos miembros destacados de la tribu de los mamíferos intermedios. Estas crías eran de una clase excepcional; tenían aún menos pelo en el cuerpo que sus padres y desde muy pequeños insistieron en caminar erguidos. Sus antepasados siempre habían aprendido a caminar sobre sus patas traseras, pero estos gemelos primates estuvieron erguidos desde el principio. Alcanzaron una altura de más de un metro y medio, y sus cabezas eran más grandes en comparación con las de otros miembros de la tribu. Aprendieron muy pronto a comunicarse el uno con el otro por medio de señas y sonidos, pero nunca lograron que su pueblo comprendiera estos nuevos símbolos.

\par
%\textsuperscript{(707.1)}
\textsuperscript{62:4.2} Cuando tenían aproximadamente catorce años, huyeron de la tribu, dirigiéndose hacia el oeste para criar a su familia y fundar la nueva especie de los primates. A estas nuevas criaturas se les denomina muy adecuadamente \textit{Primates}, puesto que fueron los antepasados animales directos e inmediatos de la familia humana misma.

\par
%\textsuperscript{(707.2)}
\textsuperscript{62:4.3} Así es como los primates llegaron a ocupar una región en la costa oeste de la península mesopotámica, que en aquella época se adentraba en el mar del sur, mientras que las tribus menos inteligentes y estrechamente emparentadas vivían en la punta de la península a lo largo de su costa oriental.

\par
%\textsuperscript{(707.3)}
\textsuperscript{62:4.4} Los primates eran más humanos y menos animales que los mamíferos intermedios que los precedieron. Las proporciones del esqueleto de esta nueva especie eran muy similares a las de las razas humanas primitivas. El tipo de mano y de pie humanos se había desarrollado plenamente, y estas criaturas podían caminar e incluso correr tan bien como cualquiera de sus descendientes humanos posteriores. Abandonaron casi por completo la vida en los árboles, aunque continuaron recurriendo a las copas de los árboles como medida de seguridad durante la noche, pues al igual que sus antepasados anteriores, estaban extremadamente dominadas por el miedo. La creciente utilización de sus manos contribuyó mucho al desarrollo de la capacidad inherente de su cerebro, pero aún no poseían una mente que se pudiera calificar realmente de humana.

\par
%\textsuperscript{(707.4)}
\textsuperscript{62:4.5} Aunque la naturaleza emocional de los primates difería poco de la de sus antepasados, mostraban una tendencia más humana en todas sus inclinaciones. Eran en verdad unos animales espléndidos y superiores; alcanzaban la madurez hacia los diez años de edad y la duración de su vida natural era de unos cuarenta años. Esto significa que podrían haber vivido cuarenta años si hubieran muerto de muerte natural, pero en aquellos tiempos primitivos muy pocos animales morían de muerte natural; la lucha por la existencia era demasiado fuerte.

\par
%\textsuperscript{(707.5)}
\textsuperscript{62:4.6} A continuación, después de casi novecientas generaciones de desarrollo, que abarcaron cerca de veintiún mil años desde la aparición de los mamíferos precursores, los primates dieron a luz \textit{repentinamente} a dos asombrosas criaturas, los primeros seres verdaderamente humanos.

\par
%\textsuperscript{(707.6)}
\textsuperscript{62:4.7} Así es como los mamíferos precursores, que habían surgido del tipo norteamericano de lémures, dieron origen a los mamíferos intermedios, y estos últimos produjeron a su vez los primates superiores, que fueron los antepasados directos de la raza humana primitiva. Las tribus primates fueron el último eslabón vital en la evolución del hombre, pero en menos de cinco mil años no quedó ni un solo individuo de estas tribus extraordinarias.

\section*{5. Los primeros seres humanos}
\par
%\textsuperscript{(707.7)}
\textsuperscript{62:5.1} El nacimiento de los dos primeros seres humanos se produjo exactamente 993.419 años antes del año 1934 de la era cristiana\footnote{\textit{Primeros seres humanos}: Gn 1:26-27; 2:7.}.

\par
%\textsuperscript{(707.8)}
\textsuperscript{62:5.2} Estas dos criaturas extraordinarias eran unos seres verdaderamente humanos. Poseían un pulgar humano perfecto, como muchos de sus antepasados, y tenían unos pies tan perfectos como las razas humanas actuales. Estos seres caminaban y corrían, pero no trepaban; la función prensil del dedo gordo del pie ya no existía, había desaparecido por completo. Cuando el peligro los empujaba hacia las copas de los árboles, subían tal como lo harían los humanos de hoy. Subían por el tronco de los árboles como los osos y no como los chimpancés o los gorilas, balanceándose de rama en rama.

\par
%\textsuperscript{(708.1)}
\textsuperscript{62:5.3} Estos primeros seres humanos (y sus descendientes) alcanzaban la plena madurez a los doce años y la duración potencial de su vida era de unos setenta y cinco años.

\par
%\textsuperscript{(708.2)}
\textsuperscript{62:5.4} Pronto aparecieron muchas emociones nuevas en estos gemelos humanos. Sentían admiración tanto por los objetos como por los otros seres y daban muestras de una considerable vanidad. Pero el progreso más extraordinario en su desarrollo emocional fue la aparición repentina de un nuevo grupo de sentimientos realmente humanos, los sentimientos de adoración, que abarcaban el temor, la veneración, la humildad e incluso una forma primitiva de gratitud. El miedo, unido a la ignorancia de los fenómenos naturales, está a punto de dar nacimiento a la religión primitiva.

\par
%\textsuperscript{(708.3)}
\textsuperscript{62:5.5} En estos seres primitivos no sólo se manifestaban estos sentimientos humanos, sino que también estaban presentes, de manera rudimentaria, muchos sentimientos sumamente evolucionados. Conocían ligeramente la compasión, la verg\"uenza y el reproche, y tenían una aguda conciencia del amor, del odio y de la venganza; también eran propensos a experimentar unos celos muy acusados.

\par
%\textsuperscript{(708.4)}
\textsuperscript{62:5.6} Estos dos primeros humanos ---los gemelos--- fueron un gran tormento para sus padres primates. Eran tan curiosos y aventureros que estuvieron a punto de perder la vida en numerosas ocasiones antes de cumplir los ocho años. Sea como fuere, tenían bastantes cicatrices en el momento de cumplir los doce años.

\par
%\textsuperscript{(708.5)}
\textsuperscript{62:5.7} Aprendieron muy pronto a comunicarse verbalmente; a la edad de diez años habían elaborado un lenguaje perfeccionado de signos y palabras de casi cincuenta ideas, y habían mejorado y ampliado enormemente la técnica rudimentaria de comunicación de sus antepasados. Pero por mucho que se esforzaron, sólo lograron enseñar a sus padres algunos de sus signos y símbolos nuevos.

\par
%\textsuperscript{(708.6)}
\textsuperscript{62:5.8} Cuando tenían unos nueve años de edad, se alejaron un claro día río abajo y mantuvieron una conversación de gran importancia. Todas las inteligencias celestiales estacionadas en Urantia, incluido yo mismo, estaban presentes y observaban el desarrollo de esta cita al mediodía. Este día memorable llegaron al acuerdo de vivir el uno con el otro y el uno para el otro, y éste fue el primero de una serie de compromisos que culminaron finalmente en la decisión de huir de sus compañeros animales inferiores, y de partir hacia el norte, sin saber que de esta manera iban a fundar la raza humana.

\par
%\textsuperscript{(708.7)}
\textsuperscript{62:5.9} Aunque todos estábamos muy preocupados por los planes de estos dos pequeños salvajes, no teníamos poder para controlar el funcionamiento de sus mentes; no influimos arbitrariamente en sus decisiones ---no podíamos hacerlo. Pero dentro de los límites permisibles de nuestras funciones planetarias, nosotros, los Portadores de Vida, junto con nuestros asociados, nos confabulamos para inducir a los gemelos humanos a que se dirigieran hacia el norte, lejos de sus parientes peludos que vivían parcialmente en los árboles. Y así, en virtud de su propia elección inteligente, los gemelos \textit{emigraron}, y a causa de nuestra supervisión, emigraron \textit{hacia el norte}, hacia una región aislada donde escaparon a la posibilidad de degradarse biológicamente mezclándose con sus parientes inferiores de las tribus de los primates.

\par
%\textsuperscript{(708.8)}
\textsuperscript{62:5.10} Poco antes de partir de su bosque natal, perdieron a su madre durante un ataque por sorpresa de los gibones. Aunque ella no poseía la misma inteligencia que ellos, como mamífero tenía por sus hijos un noble afecto de orden superior; y dio su vida valientemente intentando salvar a la pareja maravillosa. Su sacrificio no fue en vano, pues contuvo al enemigo hasta que el padre llegó con refuerzos y puso en fuga a los invasores.

\par
%\textsuperscript{(709.1)}
\textsuperscript{62:5.11} Poco después de que esta joven pareja abandonara a sus compañeros para fundar la raza humana, su padre primate se quedó desconsolado ---tenía el corazón destrozado. Se negó a comer, incluso cuando sus otros hijos le llevaban la comida. Como había perdido a sus brillantes vástagos, la vida no le parecía digna de ser vivida al lado de sus mediocres semejantes; se alejó pues vagando por el bosque, fue atacado por unos gibones hostiles y éstos lo mataron a golpes.

\section*{6. La evolución de la mente humana}
\par
%\textsuperscript{(709.2)}
\textsuperscript{62:6.1} Nosotros, los Portadores de Vida que estábamos en Urantia, habíamos pasado por la larga vigilia de una espera vigilante desde el día en que plantamos por primera vez el plasma de vida en las aguas del planeta, y la aparición de los primeros seres realmente inteligentes y volitivos nos causó naturalmente una gran alegría y una satisfacción suprema.

\par
%\textsuperscript{(709.3)}
\textsuperscript{62:6.2} Habíamos estado observando el desarrollo mental de los gemelos mediante el funcionamiento de los siete espíritus ayudantes de la mente, asignados a Urantia en el momento de nuestra llegada al planeta. A lo largo de todo el desarrollo evolutivo de la vida planetaria, estos ministros incansables de la mente siempre habían registrado su creciente habilidad para ponerse en contacto con las capacidades cerebrales de los animales, las cuales se ampliaban sucesivamente a medida que las criaturas animales progresaban hacia niveles superiores.

\par
%\textsuperscript{(709.4)}
\textsuperscript{62:6.3} Al principio, únicamente el \textit{espíritu de la intuición} pudo actuar sobre el comportamiento instintivo y reflejo de la vida animal primigenia. Cuando los tipos superiores se diferenciaron, el \textit{espíritu de la comprensión} pudo dotar a estas criaturas con el don de asociar espontáneamente las ideas. Más tarde observamos que el \textit{espíritu de la valentía} estaba en funcionamiento; los animales en evolución desarrollaron realmente una forma rudimentaria de conciencia protectora de sí mismos. Después de la aparición de los grupos de mamíferos, contemplamos que el \textit{espíritu del conocimiento} se manifestaba cada vez más. La evolución de los mamíferos superiores permitió el funcionamiento del \textit{espíritu de consejo}, con el consiguiente incremento del instinto gregario y los comienzos de un desarrollo social primitivo.

\par
%\textsuperscript{(709.5)}
\textsuperscript{62:6.4} El servicio creciente de los cinco primeros ayudantes lo habíamos observado cada vez más durante los tiempos de los mamíferos precursores, los mamíferos intermedios y los primates. Pero los dos últimos ayudantes, los ministros superiores de la mente, nunca habían podido funcionar en el tipo de mente evolutiva de Urantia.

\par
%\textsuperscript{(709.6)}
\textsuperscript{62:6.5} Imaginad nuestra alegría cuando un día ---los gemelos tenían unos diez años--- el \textit{espíritu de adoración} se puso en contacto por primera vez con la mente de la gemela, y poco después con la del gemelo. Sabíamos que algo muy semejante a la mente humana se acercaba a su culminación; cerca de un año después, cuando resolvieron finalmente, debido a unos pensamientos meditados y a una decisión deliberada, huir del hogar y viajar hacia el norte, entonces el \textit{espíritu de la sabiduría} empezó a funcionar en Urantia y en estas dos mentes humanas, ahora reconocidas como tales.

\par
%\textsuperscript{(709.7)}
\textsuperscript{62:6.6} Un nuevo tipo de movilización se produjo inmediatamente en los siete espíritus ayudantes de la mente. Estábamos llenos de expectación; nos dábamos cuenta de que se acercaba el momento tanto tiempo esperado; sabíamos que estábamos a las puertas de hacer realidad nuestro prolongado esfuerzo por producir mediante la evolución unas criaturas volitivas en Urantia.

\section*{7. El reconocimiento como mundo habitado}
\par
%\textsuperscript{(709.8)}
\textsuperscript{62:7.1} No tuvimos que esperar mucho tiempo. Al día siguiente de la huida de los gemelos, el primer destello de prueba de las señales del circuito universal se produjo al mediodía en el centro receptor planetario de Urantia. Todos estábamos, por supuesto, muy emocionados, pues nos dábamos cuenta de que un gran acontecimiento era inminente; pero como este mundo era una estación experimental de vida, no teníamos la menor idea de la manera exacta en que seríamos informados de que la vida inteligente había sido reconocida en el planeta. Pero no permanecimos mucho tiempo en la incertidumbre. Al tercer día de la fuga de los gemelos, y antes de que partiera el cuerpo de los Portadores de Vida, llegó el arcángel de Nebadon que estaba encargado de establecer los circuitos planetarios iniciales.

\par
%\textsuperscript{(710.1)}
\textsuperscript{62:7.2} Fue un día memorable en Urantia cuando nuestro pequeño grupo se reunió alrededor del polo planetario de las comunicaciones espaciales, y recibió el primer mensaje de Salvington en el circuito mental recién instalado en el planeta. Este primer mensaje, dictado por el jefe del cuerpo de los arcángeles, decía:

\par
%\textsuperscript{(710.2)}
\textsuperscript{62:7.3} <<A los Portadores de Vida que están en Urantia ---¡Saludos! Transmitimos la certeza de que se ha experimentado un gran placer en Salvington, Edentia y Jerusem cuando en la sede central de Nebadon se registró la señal de que una mente con dignidad volitiva existía en Urantia. Se ha tomado nota de que los gemelos han decidido deliberadamente huir hacia el norte y apartar a sus descendientes de sus antepasados inferiores. Ésta es la primera decisión que toma una mente --- una mente de tipo humano--- en Urantia, y establece automáticamente el circuito de comunicación por el que este mensaje inicial de reconocimiento se está transmitiendo.>>

\par
%\textsuperscript{(710.3)}
\textsuperscript{62:7.4} Luego llegaron los saludos, por este nuevo circuito, de los Altísimos de Edentia, que contenían instrucciones para los Portadores de Vida residentes, prohibiéndonos interferir en el modelo de vida que habíamos establecido. Se nos ordenó que no interviniéramos en los asuntos del progreso humano. No se debe deducir que los Portadores de Vida interfieren de manera arbitraria y mecánica en el proceso natural de los planes evolutivos de un planeta, porque no lo hacemos. Pero hasta ese momento se nos había permitido manipular el entorno y proteger el plasma vital de una manera especial; y esta supervisión extraordinaria, pero completamente natural, es la que tenía que terminar.

\par
%\textsuperscript{(710.4)}
\textsuperscript{62:7.5} Apenas habían dejado de hablar los Altísimos cuando el hermoso mensaje de Lucifer, entonces soberano del sistema de Satania, empezó a escucharse en el planeta. Los Portadores de Vida escucharon las palabras de bienvenida de su propio jefe y recibieron su permiso para regresar a Jerusem. Este mensaje de Lucifer contenía la aceptación oficial del trabajo de los Portadores de Vida en Urantia, y nos absolvía de toda crítica futura contra cualquiera de nuestros esfuerzos por mejorar los modelos de vida de Nebadon, tal como estaban establecidos en el sistema de Satania.

\par
%\textsuperscript{(710.5)}
\textsuperscript{62:7.6} Estos mensajes de Salvington, Edentia y Jerusem señalaron oficialmente el final de la supervisión secular del planeta por los Portadores de Vida. Habíamos estado de servicio durante épocas enteras, asistidos solamente por los siete espíritus ayudantes de la mente y los Controladores Físicos Maestros. Y ahora que la voluntad, la facultad para elegir la adoración y la ascensión, había aparecido en las criaturas evolutivas del planeta, comprendimos que nuestro trabajo había terminado, y nuestro grupo se preparó para partir. Como Urantia era un mundo de modificación de la vida, se nos concedió el permiso de dejar atrás a dos Portadores de Vida más antiguos con doce asistentes; fui escogido como miembro de este grupo, y desde entonces he permanecido en Urantia.

\par
%\textsuperscript{(710.6)}
\textsuperscript{62:7.7} Hace exactamente 993.408 años (antes del año 1934 d. de J.C.) que Urantia fue reconocida oficialmente como planeta para la habitación humana en el universo de Nebadon. La evolución biológica había logrado una vez más los niveles humanos de dignidad volitiva; el hombre había aparecido en el planeta 606 de Satania.

\par
%\textsuperscript{(710.7)}
\textsuperscript{62:7.8} [Patrocinado por un Portador de Vida de Nebadon, residente en Urantia.]


\chapter{Documento 63. La primera familia humana}
\par
%\textsuperscript{(711.1)}
\textsuperscript{63:0.1} URANTIA fue registrada como mundo habitado cuando los dos primeros seres humanos ---los gemelos--- tenían once años, y antes de que se convirtieran en los padres del primogénito de la segunda generación de auténticos seres humanos. El mensaje arcangélico enviado desde Salvington en esta ocasión de reconocimiento oficial planetario terminaba con estas palabras:

\par
%\textsuperscript{(711.2)}
\textsuperscript{63:0.2} <<La mente humana ha aparecido en el 606 de Satania, y los padres de esta nueva raza se llamarán \textit{Andón} y \textit{Fonta}. Todos los arcángeles ruegan para que estas criaturas puedan ser dotadas rápidamente con la presencia personal del don del espíritu del Padre Universal.>>

\par
%\textsuperscript{(711.3)}
\textsuperscript{63:0.3} Andón es el nombre nebadónico que significa <<la primera criatura semejante al Padre que muestra una sed humana de perfección>>. Fonta significa <<la primera criatura semejante al Hijo que muestra una sed humana de perfección>>. Andón y Fonta nunca conocieron estos nombres hasta que les fueron atribuidos en el momento de fusionar con sus Ajustadores del Pensamiento. Durante toda su estancia como mortales en Urantia se llamaron el uno al otro Sonta-an y Sonta-en; Sonta-an significaba <<amado por la madre>> y Sonta-en <<amado por el padre>>. Estos nombres se los pusieron ellos mismos y su significado expresa muy bien la consideración y el afecto mutuo que se tenían.

\section*{1. Andón y Fonta}
\par
%\textsuperscript{(711.4)}
\textsuperscript{63:1.1} Andón y Fonta fueron en muchos aspectos la pareja de seres humanos más extraordinaria que jamás ha vivido sobre la faz de la Tierra. Estos dos seres maravillosos, los verdaderos padres de toda la humanidad, fueron superiores en todos los sentidos a muchos de sus descendientes inmediatos, y radicalmente diferentes a todos sus antepasados tanto cercanos como lejanos.

\par
%\textsuperscript{(711.5)}
\textsuperscript{63:1.2} Los padres de esta primera pareja humana eran aparentemente poco diferentes del promedio de su tribu, aunque figuraban entre sus miembros más inteligentes, el primer grupo que aprendió a lanzar piedras y a emplear palos en los combates. También utilizaban puntas afiladas de piedra, de sílex y de hueso.

\par
%\textsuperscript{(711.6)}
\textsuperscript{63:1.3} Mientras vivía todavía con sus padres, Andón había amarrado un trozo afilado de sílex en la punta de un palo, utilizando para ello los tendones de un animal, y al menos en doce ocasiones utilizó bien este arma para salvar su propia vida y la de su hermana, que era tan curiosa y aventurera como él, y lo acompañaba indefectiblemente en todas sus excursiones exploratorias.

\par
%\textsuperscript{(711.7)}
\textsuperscript{63:1.4} La decisión de Andón y Fonta de huir de la tribu de los primates implica una calidad de mente que estaba muy por encima de la inteligencia más inferior que caracterizó a tantos descendientes posteriores suyos, los cuales se rebajaron hasta aparearse con sus primos retrasados de las tribus simias. Pero el sentimiento vago de ser algo más que unos simples animales era debido a que poseían una personalidad, y estaba acrecentado por la presencia interior de sus Ajustadores del Pensamiento.

\section*{2. La huida de los gemelos}
\par
%\textsuperscript{(712.1)}
\textsuperscript{63:2.1} Después de que Andón y Fonta hubieron decidido huir hacia el norte, sucumbieron a sus miedos durante algún tiempo, principalmente al miedo de disgustar a su padre y a su familia inmediata. Imaginaron que podrían ser atacados por sus parientes hostiles y reconocieron así la posibilidad de encontrar la muerte a manos de los miembros de su tribu ya celosos de ellos. Cuando eran más pequeños, los gemelos habían pasado la mayor parte del tiempo en compañía el uno del otro, y por esta razón nunca habían sido demasiado populares entre sus primos animales de la tribu de los primates. El hecho de haber construido en los árboles un refugio separado y muy superior al de los demás tampoco había mejorado su posición en la tribu.

\par
%\textsuperscript{(712.2)}
\textsuperscript{63:2.2} En este nuevo hogar entre las copas de los árboles fue donde, después de haber sido despertados una noche por una violenta tormenta y mientras permanecían temerosa y cariñosamente abrazados, decidieron de manera firme y definitiva huir de su hábitat tribal y de su hogar arborícola.

\par
%\textsuperscript{(712.3)}
\textsuperscript{63:2.3} Ya habían preparado un tosco refugio en la copa de un árbol a casi media jornada de camino hacia el norte. Era su escondite seguro y secreto para el primer día que pasarían fuera de su bosque natal. Aunque los gemelos compartían con los primates el mismo miedo mortal a permanecer en el suelo durante la noche, se pusieron en camino hacia el norte poco antes del anochecer. Necesitaron un valor excepcional para emprender este viaje nocturno, incluso con Luna llena, pero dedujeron acertadamente que así era menos probable que los echaran de menos y que los persiguieran sus parientes y los miembros de su tribu. Y poco después de la medianoche llegaron sanos y salvos al lugar preparado de antemano.

\par
%\textsuperscript{(712.4)}
\textsuperscript{63:2.4} Mientras viajaban hacia el norte descubrieron un depósito de pedernal a cielo abierto, y como encontraron muchas piedras con formas adecuadas para diversos usos, cogieron una provisión para el futuro. Cuando Andón intentó tallar estos pedernales a fin de adaptarlos mejor para ciertas necesidades, descubrió sus propiedades chispeantes y concibió la idea de hacer fuego. Pero este pensamiento no se apoderó firmemente de él en aquel momento, pues el clima era todavía salubre y había poca necesidad de fuego.

\par
%\textsuperscript{(712.5)}
\textsuperscript{63:2.5} Pero el Sol del otoño bajaba continuamente en el cielo, y las noches se volvían cada vez más frías a medida que viajaban hacia el norte. Ya se habían visto obligados a servirse de las pieles de los animales para calentarse. Antes de llevar una luna fuera de su tierra natal, Andón indicó a su compañera que creía que podía hacer fuego con el pedernal. Durante dos meses intentaron utilizar la chispa del pedernal para encender un fuego, pero no lo consiguieron. Cada día, esta pareja golpeaba los pedernales y se esforzaba por prenderle fuego a la madera. Por fin una tarde, hacia la hora de ponerse el Sol, el secreto de la técnica se aclaró cuando a Fonta se le ocurrió subirse a un árbol cercano para coger el nido abandonado de un pájaro. El nido estaba seco y era muy inflamable, por lo que se encendió con una llamarada en cuanto la chispa cayó sobre él. Se quedaron tan sorprendidos y asustados de su éxito que estuvieron a punto de perder el fuego, pero lo salvaron añadiendo el combustible apropiado, y fue entonces cuando empezó la primera búsqueda de leña por parte de los padres de toda la humanidad.

\par
%\textsuperscript{(712.6)}
\textsuperscript{63:2.6} Éste fue uno de los momentos más felices de su corta pero agitada vida. Se quedaron levantados toda la noche viendo arder su fuego, comprendiendo vagamente que habían hecho un descubrimiento que les permitiría desafiar el clima y así ser independientes para siempre de sus parientes animales de las tierras del sur. Después de pasar tres días descansando y disfrutando del fuego, continuaron su viaje.

\par
%\textsuperscript{(712.7)}
\textsuperscript{63:2.7} Los antepasados primates de Andón habían conservado a menudo los fuegos que los rayos encendían, pero las criaturas de la Tierra nunca antes habían poseído un método para conseguir fuego a voluntad. Pero pasó mucho tiempo antes de que los gemelos aprendieran que el musgo seco y otros materiales servían igual de bien que los nidos de los pájaros para encender fuego.

\section*{3. La familia de Andón}
\par
%\textsuperscript{(713.1)}
\textsuperscript{63:3.1} Habían transcurrido casi dos años, desde la noche en que los gemelos partieron de su hogar, cuando nació su primer hijo. Le llamaron Sontad; y Sontad fue la primera criatura nacida en Urantia que fue envuelta en una ropa protectora en el momento de nacer. La raza humana había empezado, y con esta nueva evolución apareció el instinto de cuidar adecuadamente a los niños cada vez más frágiles, un instinto que caracterizaría el desarrollo progresivo de la mente de tipo intelectual, en contraste con el tipo simplemente animal.

\par
%\textsuperscript{(713.2)}
\textsuperscript{63:3.2} Andón y Fonta tuvieron en total diecinueve hijos, y vivieron para disfrutar de la compañía de casi cincuenta nietos y media docena de biznietos. La familia residía en cuatro refugios rocosos contiguos, o semicavernas, de las cuales tres se comunicaban mediante galerías que habían sido excavadas en la caliza blanda con herramientas de sílex inventadas por los hijos de Andón.

\par
%\textsuperscript{(713.3)}
\textsuperscript{63:3.3} Estos primeros andonitas mostraban un espíritu de clan muy acusado; cazaban en grupo y nunca se alejaban demasiado de su lugar de residencia. Parecían darse cuenta de que formaban un grupo aislado y excepcional de seres vivos, y que por lo tanto debían evitar separarse. Este sentimiento de parentesco íntimo se debía sin duda a una intensificación del ministerio mental de los espíritus ayudantes.

\par
%\textsuperscript{(713.4)}
\textsuperscript{63:3.4} Andón y Fonta trabajaron sin cesar para alimentar y edificar su clan. Vivieron hasta la edad de cuarenta y dos años, y los dos murieron durante un terremoto a causa de la caída de una roca en voladizo. Cinco hijos suyos y once nietos perecieron con ellos, y casi veinte de sus descendientes sufrieron heridas graves.

\par
%\textsuperscript{(713.5)}
\textsuperscript{63:3.5} A la muerte de sus padres, Sontad, a pesar de un pie gravemente herido, asumió inmediatamente la dirección del clan con la hábil ayuda de su mujer, la mayor de sus hermanas. Su primera tarea consistió en subir rodando unas piedras para sepultar adecuadamente a sus padres, hermanos, hermanas e hijos muertos. No se debe conceder un significado indebido a este acto de enterramiento. Sus ideas sobre la supervivencia después de la muerte eran muy vagas e indefinidas, pues procedían en gran parte de sus sueños fantásticos y variados.

\par
%\textsuperscript{(713.6)}
\textsuperscript{63:3.6} Esta familia de Andón y Fonta permaneció unida hasta la vigésima generación, cuando la lucha por la comida y las fricciones sociales se combinaron para provocar el principio de la dispersión.

\section*{4. Los clanes andónicos}
\par
%\textsuperscript{(713.7)}
\textsuperscript{63:4.1} Los hombres primitivos ---los andonitas--- tenían los ojos negros y la tez morena, algo así como un cruce entre la raza amarilla y la roja. La melanina es una sustancia colorante que se encuentra en la piel de todos los seres humanos. Es el pigmento original de la piel andónica. Por el aspecto general y el color de la piel, estos primeros andonitas se parecían más a los esquimales de hoy que a ningún otro tipo de seres humanos vivientes. Fueron las primeras criaturas que emplearon la piel de los animales para protegerse del frío; no tenían mucho más pelo en el cuerpo que los humanos de hoy.

\par
%\textsuperscript{(713.8)}
\textsuperscript{63:4.2} La vida tribal de los antepasados animales de estos primeros hombres había presagiado los principios de numerosos convencionalismos sociales. El desarrollo de las emociones y el aumento de la capacidad cerebral de estos seres produjeron un desarrollo inmediato de la organización social y una nueva división del trabajo en el clan. Eran sumamente imitativos, pero su instinto de juego apenas estaba desarrollado y su sentido del humor estaba casi totalmente ausente. El hombre primitivo sonreía alguna que otra vez, pero nunca se entregaba a una risa cordial. El humor fue un legado posterior de la raza adámica. Estos primeros seres humanos no eran tan sensibles al dolor ni tan reactivos a las situaciones desagradables como muchos de los mortales evolutivos posteriores. El parto no fue una prueba dolorosa o angustiosa para Fonta ni para su progenie inmediata.

\par
%\textsuperscript{(714.1)}
\textsuperscript{63:4.3} Formaban una tribu maravillosa. Los varones solían luchar heroicamente por la seguridad de sus compañeras y de su progenitura; las mujeres se consagraban cariñosamente a sus hijos. Pero su patriotismo se limitaba estrictamente a su clan inmediato. Eran muy leales a sus familias; estaban dispuestos a morir sin dudarlo para defender a sus hijos, pero no eran capaces de captar la idea de intentar hacer un mundo mejor para sus nietos. El altruismo no había nacido todavía en el corazón humano, aunque todas las emociones esenciales para el nacimiento de la religión se encontraban ya presentes en estos aborígenes de Urantia.

\par
%\textsuperscript{(714.2)}
\textsuperscript{63:4.4} Estos primeros hombres poseían un afecto conmovedor por sus camaradas y tenían ciertamente una idea real, aunque rudimentaria, de la amistad. En épocas posteriores fue muy común contemplar, durante las batallas que se repetían sin cesar contra las tribus inferiores, a uno de estos hombres primitivos luchar valientemente con una mano mientras continuaba esforzándose por proteger y salvar a un compañero de combate herido. Muchas de las características humanas más nobles y elevadas que se desarrollaron en el transcurso de la evolución posterior, se presagiaban de manera conmovedora en estos pueblos primitivos.

\par
%\textsuperscript{(714.3)}
\textsuperscript{63:4.5} El clan andónico original mantuvo una línea ininterrumpida de jefes hasta la vigésimo séptima generación, durante la cual, al no aparecer ningún vástago varón entre los descendientes directos de Sontad, dos miembros rivales del clan que aspiraban a la jefatura empezaron a luchar por la supremacía.

\par
%\textsuperscript{(714.4)}
\textsuperscript{63:4.6} Antes de la gran dispersión de los clanes andónicos, un lenguaje bien desarrollado había evolucionado a partir de los primeros esfuerzos por comunicarse entre ellos. Este lenguaje continuó enriqueciéndose y recibió aportaciones casi diarias debido a los nuevos inventos y a las adaptaciones al entorno que este pueblo activo, inquieto y curioso realizaba. Y este lenguaje se convirtió en la voz de Urantia, en la lengua de la familia humana primitiva, hasta la aparición posterior de las razas de color.

\par
%\textsuperscript{(714.5)}
\textsuperscript{63:4.7} A medida que el tiempo pasaba, los clanes andónicos aumentaron y el contacto entre estas familias en expansión empezó a producir fricciones y malentendidos. Sólo había dos cosas que llegaron a ocupar la mente de estos pueblos: cazar para obtener comida y combatir para vengarse de alguna injusticia o de algún insulto, real o supuesto, cometido por las tribus vecinas.

\par
%\textsuperscript{(714.6)}
\textsuperscript{63:4.8} Las disensiones familiares aumentaron, estallaron las guerras entre las tribus, y los mejores elementos de los grupos más capaces y avanzados sufrieron graves pérdidas. Algunas de estas pérdidas fueron irreparables; algunos de los elementos más valiosos en cuanto a capacidad e inteligencia se perdieron para siempre en el mundo. Estas guerras continuas entre los clanes amenazaron con extinguir a esta primera raza y a su civilización primitiva.

\par
%\textsuperscript{(714.7)}
\textsuperscript{63:4.9} Es imposible inducir a unos seres tan primitivos a que vivan juntos mucho tiempo en paz. El hombre desciende de animales combativos, y cuando la gente inculta está estrechamente asociada, se irritan y se ofenden mutuamente. Los Portadores de Vida conocen esta tendencia de las criaturas evolutivas y, por consiguiente, aseguran la separación final de los seres humanos en vías de desarrollo al menos en tres razas distintas y separadas, y más a menudo en seis.

\section*{5. La dispersión de los andonitas}
\par
%\textsuperscript{(715.1)}
\textsuperscript{63:5.1} Las primeras razas andonitas no penetraron mucho en el interior de Asia, y al principio no entraron en África. La geografía de aquellos tiempos las orientó hacia el norte, y estos pueblos viajaron cada vez más hacia el norte hasta que el hielo del tercer glaciar, que avanzaba lentamente, se lo impidió.

\par
%\textsuperscript{(715.2)}
\textsuperscript{63:5.2} Antes de que esta extensa capa de hielo llegara hasta Francia y las Islas Británicas, los descendientes de Andón y Fonta habían avanzado hacia el oeste por Europa, y habían establecido más de mil poblados separados a lo largo de los grandes ríos que desembocaban en el Mar del Norte, cuyas aguas eran cálidas en aquel entonces.

\par
%\textsuperscript{(715.3)}
\textsuperscript{63:5.3} Estas tribus andónicas fueron los primeros habitantes de las riberas de Francia; vivieron a lo largo del río Somme durante decenas de miles de años. El Somme es el único río que los glaciares no cambiaron, y en aquellos tiempos corría hacia el mar poco más o menos como en la actualidad. Esto explica por qué se encuentran tantos indicios de los descendientes andónicos a lo largo del valle de este río.

\par
%\textsuperscript{(715.4)}
\textsuperscript{63:5.4} Estos aborígenes de Urantia no vivían en los árboles, aunque en caso de necesidad aún se subían a las copas. Residían normalmente al abrigo de los precipicios que sobresalían por encima de los ríos y en las grutas de las laderas, que les proporcionaban una buena vista sobre las vías de acceso y los protegían de los elementos. Así podían disfrutar de la comodidad de sus fogatas sin que el humo les incomodara demasiado. Tampoco eran verdaderos trogloditas, aunque en épocas posteriores las últimas capas de hielo que avanzaron hacia el sur obligaron a sus descendientes a refugiarse en las cavernas. Preferían acampar cerca de los límites de un bosque y al lado de un riachuelo.

\par
%\textsuperscript{(715.5)}
\textsuperscript{63:5.5} Pronto se volvieron extraordinariamente hábiles en camuflar sus moradas parcialmente abrigadas, y demostraron una gran destreza en la construcción de cabañas de piedra en forma de cúpula, que utilizaban como habitación para dormir, en las cuales entraban a gatas por la noche. La entrada de estas cabañas se cerraba rodando una piedra delante de ella, una piedra grande que se había colocado en el interior para este fin antes de poner en su sitio las últimas piedras del techo.

\par
%\textsuperscript{(715.6)}
\textsuperscript{63:5.6} Los andonitas eran unos cazadores audaces y afortunados; a excepción de las bayas silvestres y de ciertas frutas de los árboles, se alimentaban exclusivamente de carne. Así como Andón había inventado el hacha de piedra, sus descendientes no tardaron en descubrir la lanza y el arpón, y los utilizaron de manera eficaz. Por fin una mente capaz de crear herramientas funcionaba en conjunción con una mano capaz de utilizarlas, y estos primeros humanos se volvieron muy diestros en la fabricación de herramientas de sílex. Viajaban por todas partes buscando sílex, de manera muy similar a como los humanos de hoy viajan hasta los confines de la Tierra en busca de oro, platino y diamantes.

\par
%\textsuperscript{(715.7)}
\textsuperscript{63:5.7} Estas tribus andónicas manifestaron, en otros muchos aspectos, un grado de inteligencia que sus descendientes retrógrados no alcanzaron en medio millón de años, aunque volvieran a descubrir una y otra vez diversos métodos para encender el fuego.

\section*{6. Onagar --- el primer instructor de la verdad}
\par
%\textsuperscript{(715.8)}
\textsuperscript{63:6.1} A medida que se extendía la dispersión andónica, el nivel cultural y espiritual de los clanes fue degenerando durante cerca de diez mil años hasta los tiempos de Onagar, el cual asumió la dirección de estas tribus, trajo la paz entre ellas y las condujo a todas, por primera vez, a la adoración de <<Aquel que da el Aliento a los hombres y a los animales>>\footnote{\textit{Aquel que da Aliento}: Gn 2:7; Is 42:5.}.

\par
%\textsuperscript{(716.1)}
\textsuperscript{63:6.2} La filosofía de Andón había sido muy confusa; le faltó muy poco para convertirse en un adorador del fuego a causa de la gran comodidad que le procuró su descubrimiento accidental. Sin embargo, la razón lo desvió de su propio descubrimiento y lo orientó hacia el Sol como fuente superior e imponente de luz y de calor; pero esta fuente estaba demasiado lejana, y Andón no se convirtió en un adorador del Sol.

\par
%\textsuperscript{(716.2)}
\textsuperscript{63:6.3} Los andonitas no tardaron en descubrir el miedo que les producían los elementos ---trueno, relámpago, lluvia, nieve, granizo e hielo. Pero el hambre era el estímulo que reaparecía constantemente en aquellos tiempos primitivos, y como se alimentaban en gran parte de los animales, desarrollaron con el tiempo una especie de adoración a los animales. Para Andón, los animales comestibles más grandes eran símbolos de fuerza creativa y de poder sustentador. De vez en cuando, tenían la costumbre de designar a alguno de estos animales más grandes como objeto de adoración. Cuando estaba en boga un animal determinado, dibujaban toscamente sus contornos en las paredes de las cavernas, y más tarde, a medida que las artes continuaron progresando, este dios animal era grabado en diversos ornamentos.

\par
%\textsuperscript{(716.3)}
\textsuperscript{63:6.4} Muy pronto, los pueblos andónicos adquirieron la costumbre de abstenerse de comer la carne del animal que se veneraba en su tribu. Luego, para causar una impresión más adecuada en la mente de los jóvenes, desarrollaron una ceremonia de veneración que realizaban alrededor del cuerpo de uno de aquellos animales reverenciados; y más tarde aún, esta celebración primitiva se transformó en las ceremonias sacrificatorias más complicadas que practicaron sus descendientes. Éste es el origen de los sacrificios como parte del culto. Esta idea fue elaborada por Moisés en el ritual hebreo, y conservada en su esencia por el apóstol Pablo como la doctrina de la expiación\footnote{\textit{Doctrina de la expiación}: Heb 9:22.} de los pecados mediante el <<derramamiento de sangre>>\footnote{\textit{Sacrificios de sangre}: Ex 20:24; Lv 17:11.}.

\par
%\textsuperscript{(716.4)}
\textsuperscript{63:6.5} La comida era la cosa más importante en la vida de estos seres humanos primitivos, tal como lo demuestra la oración que Onagar, su gran instructor, enseñó a esta gente sencilla. Esta oración decía así:

\par
%\textsuperscript{(716.5)}
\textsuperscript{63:6.6} <<Oh Aliento de la Vida, danos hoy nuestro alimento de cada día, líbranos de la maldición del hielo, sálvanos de nuestros enemigos del bosque, y recíbenos con misericordia en el Gran Más Allá.>>

\par
%\textsuperscript{(716.6)}
\textsuperscript{63:6.7} Onagar mantuvo su cuartel general en una población llamada Obán, situada en las orillas septentrionales del antiguo Mediterráneo, en la región actual del Mar Caspio. Esta población era un lugar de detención enclavado en el punto donde la ruta que conducía desde la Mesopotamia meridional hacia el norte, daba la vuelta hacia el oeste. Desde Obán, Onagar envió educadores a las poblaciones lejanas para difundir sus nuevas doctrinas sobre una sola Deidad y su concepto de la vida futura, que él llamaba el Gran Más Allá. Estos emisarios de Onagar fueron los primeros misioneros del mundo; fueron también los primeros seres humanos que asaron la carne, los primeros que utilizaron el fuego con regularidad para preparar la comida. Asaban la carne en la punta de unos palos y también sobre las piedras calientes; más tarde asaron grandes trozos al fuego, pero sus descendientes retrocedieron casi por completo al consumo de la carne cruda.

\par
%\textsuperscript{(716.7)}
\textsuperscript{63:6.8} Onagar nació 983.323 años antes del año 1934 de la era cristiana y vivió hasta los sesenta y nueve años de edad. La historia de las realizaciones de este maestro pensador y dirigente espiritual de los tiempos anteriores al Príncipe Planetario constituye un relato emocionante sobre la organización de estos pueblos primitivos en una verdadera sociedad. Instituyó un gobierno tribal eficaz que las generaciones sucesivas no lograron igualar en muchos milenios. Hasta la llegada del Príncipe Planetario, nunca más volvió a existir en la Tierra una civilización espiritual tan elevada. Esta gente sencilla tenía una verdadera religión, aunque fuera primitiva, pero sus descendientes en vías de degeneración la perdieron posteriormente.

\par
%\textsuperscript{(717.1)}
\textsuperscript{63:6.9} Aunque Andón y Fonta habían recibido Ajustadores del Pensamiento, así como muchos de sus descendientes, los Ajustadores y los serafines guardianes no llegaron en gran número a Urantia hasta los tiempos de Onagar. Esta época fue, en verdad, la edad de oro del hombre primitivo.

\section*{7. La supervivencia de Andón y Fonta}
\par
%\textsuperscript{(717.2)}
\textsuperscript{63:7.1} Andón y Fonta, los espléndidos fundadores de la raza humana, recibieron su reconocimiento en el momento del juicio de Urantia, cuando llegó el Príncipe Planetario, y terminaron el régimen de los mundos de las mansiones a su debido tiempo con la categoría de ciudadanos de Jerusem. Aunque nunca han recibido autorización para regresar a Urantia, están al corriente de la historia de la raza que fundaron. Se afligieron por la traición de Caligastia, se entristecieron con el fracaso de Adán, pero se regocijaron extremadamente cuando se recibió la noticia de que Miguel había escogido su mundo como escenario para su última donación.

\par
%\textsuperscript{(717.3)}
\textsuperscript{63:7.2} Andón y Fonta fusionaron en Jerusem con sus Ajustadores del Pensamiento, tal como lo hicieron varios hijos suyos, entre ellos Sontad; pero la mayoría de sus descendientes, incluso inmediatos, sólo lograron fusionar con el Espíritu.

\par
%\textsuperscript{(717.4)}
\textsuperscript{63:7.3} Poco después de llegar a Jerusem, Andón y Fonta recibieron permiso del Soberano del Sistema para regresar al primer mundo de las mansiones, a fin de servir con las personalidades morontiales que acogen a los peregrinos del tiempo que llegan de Urantia a las esferas celestiales. Y han sido asignados a esta tarea por un tiempo indeterminado. Intentaron enviar sus saludos a Urantia en el momento de estas revelaciones, pero su petición fue sabiamente denegada.

\par
%\textsuperscript{(717.5)}
\textsuperscript{63:7.4} Y ésta es la narración del capítulo más heroico y fascinante de toda la historia de Urantia, el relato de la evolución, la lucha por la vida, la muerte y la supervivencia eterna de los padres excepcionales de toda la humanidad.

\par
%\textsuperscript{(717.6)}
\textsuperscript{63:7.5} [Presentado por un Portador de Vida residente en Urantia.]


\chapter{Documento 64. Las razas evolutivas de color}
\par
%\textsuperscript{(718.1)}
\textsuperscript{64:0.1} ÉSTA es la historia de las razas evolutivas de Urantia desde los tiempos de Andón y Fonta, hace casi un millón de años, pasando por la época del Príncipe Planetario, hasta el final del período glacial.

\par
%\textsuperscript{(718.2)}
\textsuperscript{64:0.2} La raza humana tiene casi un millón de años de edad. La primera mitad de su historia corresponde aproximadamente a los tiempos anteriores al Príncipe Planetario de Urantia. La segunda mitad de la historia de la humanidad comienza en el momento de la llegada del Príncipe Planetario y de la aparición de las seis razas de color, y corresponde más o menos al período considerado generalmente como la antigua edad de piedra.

\section*{1. Los aborígenes andónicos}
\par
%\textsuperscript{(718.3)}
\textsuperscript{64:1.1} El hombre primitivo hizo su aparición evolutiva en la Tierra hace poco menos de un millón de años, y tuvo una dura experiencia. Trató instintivamente de evitar el peligro de mezclarse con las tribus simias inferiores. Pero no pudo emigrar hacia el este debido a las altas tierras áridas del Tíbet, con sus 9.000 metros por encima del nivel del mar; tampoco pudo ir hacia el sur o el oeste, porque el Mar Mediterráneo era mucho más grande que hoy, extendiéndose entonces hacia el este hasta el Océano Índico; y cuando se dirigió hacia el norte, se encontró con el hielo que venía avanzando. Pero incluso cuando el hielo bloqueó su emigración ulterior, y aunque las tribus que se dispersaban se volvían cada vez más hostiles, los grupos más inteligentes nunca albergaron la idea de dirigirse hacia el sur para vivir entre sus primos arborícolas peludos con un intelecto inferior.

\par
%\textsuperscript{(718.4)}
\textsuperscript{64:1.2} Muchas de las emociones religiosas más antiguas del hombre nacieron de su sensación de impotencia ante el entorno cerrado de esta situación geográfica ---montañas a la derecha, agua a la izquierda y el hielo al frente. Sin embargo, estos andonitas progresivos no querían volver atrás con sus parientes inferiores del sur que vivían en los árboles.

\par
%\textsuperscript{(718.5)}
\textsuperscript{64:1.3} Estos andonitas evitaban los bosques, en contraste con las costumbres de sus parientes no humanos. El hombre siempre ha degenerado en los bosques; la evolución humana sólo ha progresado en los espacios abiertos y en las latitudes más elevadas. El frío y el hambre que reinan en las tierras al descubierto estimulan la actividad, la invención y el ingenio. Mientras estas tribus andónicas producían a los pioneros de la raza humana actual en medio de las dificultades y privaciones de estos rigurosos climas nórdicos, sus primos atrasados disfrutaban en los bosques tropicales meridionales del país de su origen primitivo común.

\par
%\textsuperscript{(718.6)}
\textsuperscript{64:1.4} Estos acontecimientos se produjeron durante la época del tercer glaciar, el primero según el cálculo de los geólogos. Los dos primeros glaciares fueron poco extensos en Europa septentrional.

\par
%\textsuperscript{(718.7)}
\textsuperscript{64:1.5} Durante la mayor parte del período glacial, Inglaterra estuvo comunicada por tierra con Francia, mientras que más tarde África estuvo unida a Europa mediante el puente terrestre de Sicilia. En la época de las emigraciones andónicas, un camino terrestre continuo, que pasaba por Europa y Asia, conectaba a Inglaterra en el oeste con Java en el este; pero Australia estaba de nuevo aislada, lo que acentuó aún más el desarrollo de su propia fauna peculiar.

\par
%\textsuperscript{(719.1)}
\textsuperscript{64:1.6} Hace \textit{950.000} años, los descendientes de Andón y Fonta habían emigrado muy lejos hacia el este y el oeste. En el oeste, cruzaron por Europa y llegaron hasta Francia e Inglaterra. En épocas posteriores penetraron hacia el este hasta llegar a Java, donde recientemente se han encontrado sus huesos ---el llamado hombre de Java--- y luego continuaron su viaje hasta Tasmania.

\par
%\textsuperscript{(719.2)}
\textsuperscript{64:1.7} Los grupos que fueron hacia el oeste se contaminaron menos con las cepas atrasadas de origen ancestral común que los que se dirigieron hacia el este, los cuales se mezclaron muy ampliamente con sus primos animales retrasados. Estos individuos no progresivos se encaminaron hacia el sur y se aparearon enseguida con las tribus inferiores. Más tarde, un número creciente de mestizos regresaron al norte y se emparejaron con los pueblos andónicos que se extendían con rapidez; estas uniones desafortunadas deterioraron infaliblemente la raza superior. Cada vez menos poblados primitivos conservaron la adoración de Aquél que da el Aliento. Esta civilización en sus albores estuvo amenazada de extinción.

\par
%\textsuperscript{(719.3)}
\textsuperscript{64:1.8} Siempre ha sido así en Urantia. Unas civilizaciones muy prometedoras se han deteriorado sucesivamente y han terminado por extinguirse debido a la locura de permitir que los individuos superiores procreen libremente con los inferiores.

\section*{2. Los pueblos de Foxhall}
\par
%\textsuperscript{(719.4)}
\textsuperscript{64:2.1} Hace \textit{900.000} años, las artes de Andón y Fonta y la cultura de Onagar estaban desapareciendo de la faz de la Tierra; la cultura, la religión e incluso el trabajo del sílex se encontraban en su punto más bajo.

\par
%\textsuperscript{(719.5)}
\textsuperscript{64:2.2} Fue en estos tiempos cuando grandes grupos de mestizos inferiores, procedentes del sur de Francia, llegaron a Inglaterra. Estas tribus estaban tan cruzadas con las criaturas simiescas de los bosques que apenas eran humanas. No tenían ninguna religión, pero trabajaban el sílex de manera rudimentaria y poseían la suficiente inteligencia para encender el fuego.

\par
%\textsuperscript{(719.6)}
\textsuperscript{64:2.3} Estas tribus fueron seguidas, en Europa, por un pueblo prolífico y un poco superior, cuyos descendientes se diseminaron pronto por todo el continente, desde los hielos del norte hasta los Alpes y el Mediterráneo en el sur. Estas tribus formaban la llamada \textit{raza de Heidelberg}.

\par
%\textsuperscript{(719.7)}
\textsuperscript{64:2.4} Durante este largo período de decadencia cultural, los pueblos de Foxhall en Inglaterra y las tribus de Badonán en el noroeste de la India continuaron manteniendo algunas tradiciones de Andón y ciertos restos de la cultura de Onagar.

\par
%\textsuperscript{(719.8)}
\textsuperscript{64:2.5} Los pueblos de Foxhall eran los más occidentales y lograron conservar una gran parte de la cultura andónica; también preservaron sus conocimientos sobre el trabajo del sílex y los trasmitieron a sus descendientes, los antiguos antepasados de los esquimales.

\par
%\textsuperscript{(719.9)}
\textsuperscript{64:2.6} Aunque los vestigios de los pueblos de Foxhall han sido los últimos que se han descubierto en Inglaterra, estos andonitas fueron en realidad los primeros seres humanos que vivieron en estas regiones. En aquella época, el puente terrestre unía todavía a Francia con Inglaterra; y como la mayoría de las primeras colonias de los descendientes de Andón estaban situadas a lo largo de los ríos y las costas de aquellos tiempos antiguos, actualmente se encuentran bajo las aguas del Canal de la Mancha y del Mar del Norte, pero unas tres o cuatro siguen todavía por encima del agua en la costa inglesa.

\par
%\textsuperscript{(720.1)}
\textsuperscript{64:2.7} Una gran parte de los pueblos de Foxhall más inteligentes y espirituales mantuvieron su superioridad racial y perpetuaron sus costumbres religiosas primitivas. Este pueblo se mezcló ulteriormente con razas más recientes, partió de Inglaterra hacia el oeste después de una invasión glaciar posterior, y ha sobrevivido como los esquimales actuales.

\section*{3. Las tribus de Badonán}
\par
%\textsuperscript{(720.2)}
\textsuperscript{64:3.1} Además de los pueblos de Foxhall en el oeste, otro centro combativo de cultura sobrevivió en el este. Este grupo estaba situado en las estribaciones de las tierras altas del noroeste de la India, entre las tribus de Badonán, un tataranieto de Andón. Estos pobladores fueron los únicos descendientes de Andón que nunca practicaron los sacrificios humanos.

\par
%\textsuperscript{(720.3)}
\textsuperscript{64:3.2} Estos badonitas de las tierras altas ocupaban una extensa meseta rodeada de bosques, atravesada por arroyos y provista de abundante caza. Al igual que algunos de sus primos del Tíbet, vivían en toscas cabañas de piedra, en grutas situadas en las laderas y en pasajes semisubterráneos.

\par
%\textsuperscript{(720.4)}
\textsuperscript{64:3.3} Mientras las tribus del norte tenían cada vez más miedo del hielo, las que vivían cerca de su tierra de origen sentían pánico del agua. Habían observado que la península mesopotámica se hundía paulatinamente en el océano, y aunque ésta emergió varias veces, las tradiciones de estas razas primitivas se forjaron alrededor de los peligros del mar y del miedo a un hundimiento periódico. Este miedo, unido a su experiencia con las inundaciones fluviales, explica por qué buscaron las tierras altas como lugar seguro para vivir.

\par
%\textsuperscript{(720.5)}
\textsuperscript{64:3.4} Al este de los pueblos de Badonán, en las colinas Siwalik del norte de la India, se pueden encontrar los fósiles que se acercan, más que ningún otro en la Tierra, a los tipos de transición entre el hombre y los diversos grupos prehumanos.

\par
%\textsuperscript{(720.6)}
\textsuperscript{64:3.5} Hace \textit{850.000} años, las tribus superiores de Badonán empezaron una guerra de exterminio contra sus vecinos inferiores parecidos a los animales. En menos de mil años, la mayoría de los grupos animales de las fronteras de estas regiones habían sido destruídos o forzados a retroceder hasta los bosques del sur. Esta campaña para exterminar a los seres inferiores provocó un ligero mejoramiento de las tribus montañesas de aquella época. Los descendientes mezclados de este linaje badonita mejorado aparecieron en escena como un pueblo aparentemente nuevo, la \textit{raza de Neandertal}.

\section*{4. Las razas de Neandertal}
\par
%\textsuperscript{(720.7)}
\textsuperscript{64:4.1} Los hombres de Neandertal eran excelentes luchadores y viajaron enormemente. Partiendo de las tierras altas del noroeste de la India, se diseminaron gradualmente hasta Francia en el oeste, China en el este, y descendieron incluso hasta el norte de África. Dominaron el mundo durante casi medio millón de años, hasta la época de la emigración de las razas evolutivas de color.

\par
%\textsuperscript{(720.8)}
\textsuperscript{64:4.2} Hace \textit{800.000} años, la caza era abundante; muchas especies de ciervos, así como los elefantes y los hipopótamos, vagaban por Europa. Había gran cantidad de ganado; los caballos y los lobos estaban por todas partes. Los hombres de Neandertal eran grandes cazadores, y las tribus de Francia fueron las primeras que adoptaron la costumbre de conceder a los mejores cazadores el privilegio de elegir a las mujeres que deseaban como esposas.

\par
%\textsuperscript{(721.1)}
\textsuperscript{64:4.3} El reno fue extremadamente útil para estos pueblos neandertales, sirviéndoles de alimento, de vestido y para hacer herramientas, pues los cuernos y los huesos los empleaban para usos diversos. Tenían poca cultura, pero mejoraron tanto el trabajo del sílex que casi llegó a alcanzar los niveles de la época de Andón. Empezaron a utilizarse de nuevo los grandes sílex atados a unos mangos de madera que servían como hachas y piquetas.

\par
%\textsuperscript{(721.2)}
\textsuperscript{64:4.4} Hace \textit{750.000} años, la cuarta capa de hielo había avanzado mucho hacia el sur. Con sus herramientas mejoradas, los neandertales hacían agujeros en el hielo que cubría los ríos nórdicos, y así podían arponear los peces que subían hasta estas aberturas. Estas tribus retrocedieron constantemente ante el hielo que avanzaba, el cual efectuaba en aquella época su invasión más extensa en Europa.

\par
%\textsuperscript{(721.3)}
\textsuperscript{64:4.5} En aquellos tiempos, el glaciar siberiano estaba realizando su máximo avance hacia el sur, obligando al hombre primitivo a retroceder en la misma dirección hacia su tierra de origen. Pero la especie humana se había diferenciado tanto, que el peligro de mezclarse de nuevo con sus parientes simios, incapaces de progresar, había disminuido enormemente.

\par
%\textsuperscript{(721.4)}
\textsuperscript{64:4.6} Hace \textit{700.000} años que el cuarto glaciar, el más grande de todos en Europa, estaba retrocediendo; los hombres y los animales regresaban hacia el norte. El clima era fresco y húmedo, y el hombre primitivo prosperó de nuevo en Europa y Asia occidental. Los bosques se extendieron gradualmente hacia el norte sobre las tierras que el glaciar había cubierto tan recientemente.

\par
%\textsuperscript{(721.5)}
\textsuperscript{64:4.7} El gran glaciar había cambiado poco la vida de los mamíferos. Estos animales sobrevivieron en la estrecha franja de tierra situada entre el hielo y los Alpes, y cuando el glaciar retrocedió, volvieron a extenderse rápidamente por toda Europa. Los elefantes de colmillos rectos, los rinocerontes de hocico ancho, las hienas y los leones africanos llegaron de África por el puente terrestre de Sicilia; estos nuevos animales exterminaron prácticamente a los tigres con dientes de sable y a los hipopótamos.

\par
%\textsuperscript{(721.6)}
\textsuperscript{64:4.8} Hace \textit{650.000} años el clima continuaba siendo templado. Hacia mediados del período interglacial se había vuelto tan cálido que los Alpes casi se despojaron del hielo y la nieve.

\par
%\textsuperscript{(721.7)}
\textsuperscript{64:4.9} Hace \textit{600.000} años, el hielo había alcanzado entonces su máximo punto de retroceso hacia el norte, y después de una pausa de pocos miles de años, partió de nuevo en su quinto viaje hacia el sur. Pero el clima se modificó poco durante cincuenta mil años. Los hombres y los animales de Europa cambiaron muy poco. Disminuyó la ligera aridez del período anterior y los glaciares alpinos descendieron mucho hacia los valles de los ríos.

\par
%\textsuperscript{(721.8)}
\textsuperscript{64:4.10} Hace \textit{550.000} años, el avance del glaciar empujó de nuevo a los hombres y a los animales hacia el sur. Pero en esta ocasión los hombres dispusieron de mucho espacio dentro de la ancha franja de tierra que se extendía hacia el nordeste de Asia, y que estaba situada entre la capa de hielo y el Mar Negro, una prolongación entonces muy dilatada del Mediterráneo.

\par
%\textsuperscript{(721.9)}
\textsuperscript{64:4.11} Esta época de los glaciares cuarto y quinto contempló una nueva propagación de la cultura rudimentaria de las razas neandertales. Pero los progresos eran tan pequeños, que parecía en verdad que la tentativa de producir un tipo nuevo y modificado de vida inteligente en Urantia estaba a punto de fracasar. Durante cerca de un cuarto de millón de años, estos pueblos primitivos fueron a la deriva, cazando y luchando, mejorando esporádicamente en algunos aspectos, pero en general, degenerando continuamente en comparación con sus antepasados andónicos superiores.

\par
%\textsuperscript{(721.10)}
\textsuperscript{64:4.12} Durante estos tiempos de tinieblas espirituales, la humanidad supersticiosa alcanzó sus niveles culturales más bajos. En realidad, la religión de los neandertales no iba más allá de una vergonzosa superstición. Tenían un miedo mortal de las nubes, y principalmente de las brumas y las nieblas. Se desarrolló gradualmente una religión primitiva basada en el miedo a las fuerzas naturales, mientras que la adoración de los animales declinó a medida que el mejoramiento de las herramientas y la abundancia de la caza permitieron que estos pueblos vivieran con menos ansiedad por la comida; las recompensas sexuales concedidas a los mejores cazadores contribuyeron a mejorar enormemente las técnicas de la caza. Esta nueva religión del miedo condujo a las tentativas por aplacar las fuerzas invisibles que estaban ocultas detrás de los elementos naturales, y más tarde culminó en los sacrificios humanos a fin de apaciguar estas fuerzas físicas invisibles y desconocidas. Esta práctica terrible de los sacrificios humanos se ha perpetuado entre los pueblos más atrasados de Urantia hasta el mismo siglo veinte.

\par
%\textsuperscript{(722.1)}
\textsuperscript{64:4.13} Estos primeros hombres de Neandertal difícilmente pueden ser calificados de adoradores del Sol. Vivían más bien con el temor a la oscuridad; tenían un terror mortal del anochecer. Mientras la Luna brillaba un poco, se las arreglaban para seguir adelante; pero cuando ésta se oscurecía, se llenaban de pánico y empezaban a sacrificar a sus mejores especímenes de hombres y mujeres en un esfuerzo por incitar a la Luna a que brillara de nuevo. Pronto aprendieron que el Sol reaparecía con regularidad, pero conjeturaban que la Luna sólo volvía porque sacrificaban a los miembros de su tribu. A medida que la raza progresaba, el objeto y la meta de los sacrificios cambiaron gradualmente, pero la ofrenda de sacrificios humanos como parte del ceremonial religioso perduró durante mucho tiempo.

\section*{5. El origen de las razas de color}
\par
%\textsuperscript{(722.2)}
\textsuperscript{64:5.1} Hace \textit{500.000} años, las tribus de Badonán de las tierras altas del noroeste de la India se enredaron en otra gran lucha racial. Esta guerra implacable hizo estragos durante más de cien años, y cuando la larga lucha terminó, sólo quedaban unas cien familias. Pero estos supervivientes eran los más inteligentes y deseables de todos los descendientes de Andón y Fonta que vivían entonces.

\par
%\textsuperscript{(722.3)}
\textsuperscript{64:5.2} Un acontecimiento nuevo y extraño se produjo entonces entre estos badonitas de las tierras altas. Un hombre y una mujer que vivían en la parte nordeste de la región de las tierras altas entonces habitadas, empezaron a producir \textit{repentinamente} una familia de hijos excepcionalmente inteligentes. Fue la \textit{familia sangik}, los antepasados de las seis razas de color de Urantia.

\par
%\textsuperscript{(722.4)}
\textsuperscript{64:5.3} Estos hijos sangiks, diecinueve en total, no sólo eran más inteligentes que sus semejantes, sino que su piel manifestaba una tendencia sin igual a ponerse de colores diferentes cuando permanecía expuesta a la luz del Sol. De estos diecinueve hijos, cinco eran rojos, dos anaranjados, cuatro amarillos, dos verdes, cuatro azules y dos índigos. Estos colores se volvieron más pronunciados a medida que los niños crecieron, y cuando estos jóvenes se casaron más tarde con otros miembros de su tribu, todos sus descendientes tendieron a coger el color de la piel de su progenitor sangik.

\par
%\textsuperscript{(722.5)}
\textsuperscript{64:5.4} Interrumpo ahora esta narración cronológica, después de llamar vuestra atención sobre la llegada del Príncipe Planetario alrededor de esta época, para examinar por separado las seis razas sangiks de Urantia.

\section*{6. Las seis razas Sangik de Urantia}
\par
%\textsuperscript{(722.6)}
\textsuperscript{64:6.1} En un planeta evolutivo medio, las seis razas evolutivas de color aparecen de una en una; el hombre rojo es el primero que evoluciona, y vaga por el mundo durante épocas enteras antes de que aparezcan las siguientes razas de color. La aparición simultánea de las seis razas en Urantia, \textit{y dentro de una sola familia}, fue totalmente excepcional.

\par
%\textsuperscript{(723.1)}
\textsuperscript{64:6.2} La temprana aparición de los andonitas en Urantia fue también algo nuevo en Satania. En ningún otro mundo del sistema local se ha desarrollado una raza así de criaturas volitivas con antelación a las razas evolutivas de color.

\par
%\textsuperscript{(723.2)}
\textsuperscript{64:6.3} 1. \textit{El hombre rojo}. Estos pueblos fueron unos especímenes extraordinarios de la raza humana, superiores en muchos aspectos a Andón y Fonta. Formaron un grupo sumamente inteligente y fueron los primeros hijos sangiks que desarrollaron una civilización y un gobierno tribales. Siempre fueron monógamos, e incluso sus descendientes mezclados practicaron rara vez la poligamia.

\par
%\textsuperscript{(723.3)}
\textsuperscript{64:6.4} En tiempos posteriores tuvieron dificultades graves y prolongadas con sus hermanos amarillos en Asia. Les sirvió de ayuda el hecho de haber inventado pronto el arco y la flecha, pero desgraciadamente habían heredado una gran parte de la tendencia de sus antepasados a luchar entre ellos, y esto los debilitó de tal manera que las tribus amarillas pudieron expulsarlos del continente asiático.

\par
%\textsuperscript{(723.4)}
\textsuperscript{64:6.5} Hace aproximadamente ochenta y cinco mil años, los supervivientes relativamente puros de la raza roja pasaron en masa a América del Norte, y poco después el istmo terrestre de Bering se hundió, lo cual los aisló por completo. Ningún hombre rojo volvió nunca a Asia. Pero por toda Siberia, China, Asia central, la India y Europa, dejaron tras ellos a muchos descendientes suyos mezclados con las otras razas de color.

\par
%\textsuperscript{(723.5)}
\textsuperscript{64:6.6} Cuando el hombre rojo pasó a América, se llevó consigo muchas enseñanzas y tradiciones de su origen primero. Sus antepasados inmediatos habían estado en contacto con las últimas actividades de la sede mundial del Príncipe Planetario. Pero poco tiempo después de haber llegado a las Américas, el hombre rojo empezó a perder de vista estas enseñanzas y su cultura intelectual y espiritual sufrió una gran decadencia. Estos pueblos empezaron muy pronto a pelearse de nuevo entre ellos con tanta violencia, que pareció que estas guerras tribales ocasionarían la rápida extinción de este resto relativamente puro de la raza roja.

\par
%\textsuperscript{(723.6)}
\textsuperscript{64:6.7} Los hombres rojos parecían estar sentenciados a causa de este gran retroceso, cuando hace unos sesenta y cinco mil años apareció Onamonalontón como jefe y libertador espiritual. Trajo una paz temporal entre los hombres rojos americanos y restableció la adoración del <<Gran Espíritu>>\footnote{\textit{Gran Espíritu}: Jn 4:24.}. Onamonalontón vivió hasta los noventa y seis años de edad, y mantuvo su cuartel general entre las grandes secoyas de California. Muchos de sus descendientes posteriores han llegado hasta los tiempos modernos entre los indios Pies Negros.

\par
%\textsuperscript{(723.7)}
\textsuperscript{64:6.8} A medida que el tiempo pasaba, las enseñanzas de Onamonalontón se convirtieron en tradiciones muy vagas. Las guerras de aniquilación mutua empezaron de nuevo, y después de la época de este gran educador, ningún otro jefe ha logrado nunca establecer una paz universal entre ellos. Los linajes más inteligentes perecieron cada vez más en estas luchas tribales; de lo contrario, estos hombres rojos capaces e inteligentes hubieran construido una gran civilización en el continente norteamericano.

\par
%\textsuperscript{(723.8)}
\textsuperscript{64:6.9} Después de pasar desde China a América, el hombre rojo del norte nunca más volvió a entrar en contacto con otras influencias mundiales (a excepción de los esquimales) hasta que fue descubierto más tarde por el hombre blanco. Es muy lamentable que el hombre rojo perdiera casi por completo la oportunidad de mejorar su raza mezclándose con los descendientes posteriores de Adán. Tal como estaban las cosas, el hombre rojo no podía dominar al hombre blanco, y no quería servirlo voluntariamente. En tales circunstancias, si las dos razas no se mezclan, una u otra está condenada.

\par
%\textsuperscript{(723.9)}
\textsuperscript{64:6.10} 2. \textit{El hombre anaranjado}. La característica más destacada de esta raza fue su peculiar impulso de construir, de construir cualquier cosa, aunque sólo fuera apilar enormes montículos de piedra únicamente para ver qué tribu podía construir el montículo más grande. Aunque no fueron un pueblo progresivo, se beneficiaron mucho de las escuelas del Príncipe y enviaron allí a sus delegados para que se instruyeran.

\par
%\textsuperscript{(724.1)}
\textsuperscript{64:6.11} La raza anaranjada fue la primera que bajó por la costa hacia el sur en dirección a África a medida que el Mediterráneo se retiraba hacia el oeste. Pero nunca consiguieron establecerse en África y fueron aniquilados por la raza verde que llegó más tarde.

\par
%\textsuperscript{(724.2)}
\textsuperscript{64:6.12} Antes de que llegara su fin, este pueblo perdió una gran parte de sus fundamentos culturales y espirituales. Pero alcanzaron un gran renacimiento y una forma de vida superior a consecuencia de la sabia dirección de Porshunta, el cerebro principal de esta raza desafortunada, el cual les aportó su ministerio cuando tenían su cuartel general en Armagedón, hace unos trescientos mil años.

\par
%\textsuperscript{(724.3)}
\textsuperscript{64:6.13} La última gran batalla entre los hombres anaranjados y los verdes tuvo lugar en la región del bajo valle del Nilo, en Egipto. Esta guerra interminable se libró durante cerca de cien años, y cuando finalizó, muy pocos miembros de la raza anaranjada quedaban con vida. Los restos dispersos de este pueblo fueron absorbidos por los hombres verdes, y luego por los índigos que llegaron más tarde. Pero el hombre anaranjado dejó de existir como raza hace aproximadamente cien mil años.

\par
%\textsuperscript{(724.4)}
\textsuperscript{64:6.14} 3. \textit{El hombre amarillo}. Las tribus amarillas primitivas fueron las primeras que abandonaron la caza, establecieron comunidades estables y desarrollaron una vida hogareña basada en la agricultura. Intelectualmente eran un poco inferiores al hombre rojo, pero social y colectivamente se mostraron superiores a todos los pueblos sangiks en cuanto al fomento de la civilización racial. Como las diversas tribus desarrollaron un espíritu fraternal y aprendieron a convivir en una paz relativa, fueron capaces de empujar a la raza roja por delante de ellas a medida que se extendieron por Asia.

\par
%\textsuperscript{(724.5)}
\textsuperscript{64:6.15} Se alejaron mucho de las influencias del centro espiritual del mundo y cayeron en una gran oscuridad después de la apostasía de Caligastia; pero este pueblo conoció una época brillante hace alrededor de cien mil años, cuando Singlangtón asumió la dirección de estas tribus y proclamó la adoración de la <<Verdad Única>>\footnote{\textit{Verdad Única}: Jn 14:6.}.

\par
%\textsuperscript{(724.6)}
\textsuperscript{64:6.16} El número relativamente importante de supervivientes de la raza amarilla se debe a la paz que reinaba entre sus tribus. Desde la época de Singlangtón hasta los tiempos de la China moderna, la raza amarilla ha figurado entre las naciones más pacíficas de Urantia. Esta raza recibió un legado pequeño, pero poderoso, del linaje adámico importado posteriormente.

\par
%\textsuperscript{(724.7)}
\textsuperscript{64:6.17} 4. \textit{El hombre verde}. La raza verde fue uno de los grupos menos capaces de hombres primitivos y se debilitaron enormemente a causa de sus grandes emigraciones en diferentes direcciones. Antes de dispersarse, estas tribus experimentaron un gran renacimiento cultural bajo la dirección de Fantad, hace unos trescientos cincuenta mil años.

\par
%\textsuperscript{(724.8)}
\textsuperscript{64:6.18} La raza verde se fraccionó en tres divisiones mayores: Las tribus del norte fueron vencidas, esclavizadas y absorbidas por las razas amarilla y azul. El grupo oriental se amalgamó con los pueblos de la India de aquellos tiempos, y aún subsisten algunos restos entre ellos. La nación meridional penetró en África, donde destruyeron a sus primos anaranjados casi tan inferiores como ellos.

\par
%\textsuperscript{(724.9)}
\textsuperscript{64:6.19} En muchos aspectos, los dos grupos se enfrentaron de manera equitativa en esta lucha, puesto que cada uno poseía descendientes del tipo gigante: muchos de sus jefes medían entre dos metros cuarenta y dos metros setenta de altura. Estas familias gigantes del hombre verde estuvieron limitadas principalmente a esta nación meridional o egipcia.

\par
%\textsuperscript{(725.1)}
\textsuperscript{64:6.20} Los supervivientes victoriosos de los hombres verdes fueron absorbidos posteriormente por la raza índiga, el último de los pueblos de color que se desarrolló y emigró desde el centro original de dispersión racial de los sangiks.

\par
%\textsuperscript{(725.2)}
\textsuperscript{64:6.21} 5. \textit{El hombre azul}. Los hombres azules fueron un gran pueblo. Inventaron muy pronto la lanza y posteriormente elaboraron los rudimentos de muchas artes de la civilización moderna. El hombre azul tenía la capacidad cerebral del hombre rojo junto con el alma y los sentimientos del hombre amarillo. Los descendientes adámicos los prefirieron a todas las demás razas de color que subsistieron ulteriormente.

\par
%\textsuperscript{(725.3)}
\textsuperscript{64:6.22} Los primeros hombres azules fueron sensibles a las persuasiones de los instructores del estado mayor del Príncipe Caligastia, y cayeron en una gran confusión cuando estos jefes traidores desvirtuaron posteriormente sus propias enseñanzas. Al igual que otras razas primitivas, nunca se recuperaron por completo del trastorno provocado por la traición de Caligastia, y tampoco superaron nunca totalmente su tendencia a luchar entre ellos.

\par
%\textsuperscript{(725.4)}
\textsuperscript{64:6.23} Unos quinientos años después de la caída de Caligastia, se produjo un renacimiento generalizado del conocimiento y de la religión de tipo primitivo ---aunque no por ello menos real y beneficioso. Orlandof se convirtió en un gran instructor de la raza azul y volvió a llevar a muchas tribus a la adoración del verdadero Dios bajo el nombre de <<el Jefe Supremo>>. Éste fue el progreso más grande del hombre azul hasta las épocas más tardías en que su raza mejoró considerablemente gracias a la mezcla con la estirpe adámica.

\par
%\textsuperscript{(725.5)}
\textsuperscript{64:6.24} Las investigaciones y exploraciones europeas sobre la antigua edad de piedra han consistido ampliamente en la exhumación de herramientas, huesos y objetos de arte de estos antiguos hombres azules, puesto que permanecieron en Europa hasta una fecha reciente. Las llamadas \textit{razas blancas} de Urantia son los descendientes de estos hombres azules, que primero fueron modificados por una ligera mezcla con los amarillos y los rojos, y más tarde mejoraron enormemente debido a la asimilación de la mayor parte de la raza violeta.

\par
%\textsuperscript{(725.6)}
\textsuperscript{64:6.25} 6. \textit{La raza índiga}. Así como los hombres rojos fueron los más avanzados de todos los pueblos sangiks, los hombres negros fueron los menos progresivos. Fueron los últimos que emigraron de sus hogares de las tierras altas. Viajaron hasta África, tomaron posesión del continente y han permanecido allí desde entonces, excepto cuando han sido sacados a la fuerza, de siglo en siglo, para convertirlos en esclavos.

\par
%\textsuperscript{(725.7)}
\textsuperscript{64:6.26} Aislados en África, los pueblos índigos, al igual que los hombres rojos, recibieron poca o ninguna de la elevación racial que podrían haber obtenido de la inyección de la sangre adámica. Sola en África, la raza índiga hizo pocos progresos hasta los tiempos de Orvonón, durante los cuales experimentó un gran despertar espiritual. Más tarde olvidaron casi por completo al <<Dios de los Dioses>>\footnote{\textit{Dios de los Dioses}: 1 Ti 6:15.} proclamado por Orvonón, pero no perdieron del todo el deseo de adorar al Desconocido; al menos mantuvieron una forma de culto hasta hace pocos miles de años.

\par
%\textsuperscript{(725.8)}
\textsuperscript{64:6.27} A pesar de su atraso, estos pueblos índigos tienen exactamente la misma posición ante los poderes celestiales que cualquier otra raza de la Tierra.

\par
%\textsuperscript{(725.9)}
\textsuperscript{64:6.28} Fueron épocas de intensos combates entre las diversas razas, pero cerca de la sede central del Príncipe Planetario, los grupos más cultos y que habían sido instruidos en fechas más recientes convivieron en una armonía relativa; las razas del mundo aún no habían conseguido ninguna gran conquista cultural cuando este régimen quedó gravemente trastornado por el estallido de la rebelión de Lucifer.

\par
%\textsuperscript{(726.1)}
\textsuperscript{64:6.29} Todos estos diferentes pueblos experimentaron, de vez en cuando, renacimientos culturales y espirituales. Mansant fue un gran instructor de la época posterior al Príncipe Planetario. Pero sólo mencionamos a los dirigentes e instructores destacados que influyeron e inspiraron de manera notable a una raza entera. Con el paso del tiempo, numerosos educadores menos importantes aparecieron en distintas regiones; en conjunto, todos contribuyeron mucho a la suma total de influencias salvadoras que impidieron el hundimiento completo de la civilización cultural, sobre todo durante el largo período de oscurantismo entre la rebelión de Caligastia y la llegada de Adán.

\par
%\textsuperscript{(726.2)}
\textsuperscript{64:6.30} Existen muchas razones, buenas y suficientes, para llevar a cabo el proyecto de producir por evolución tres o seis razas de color en los mundos del espacio. Aunque los mortales de Urantia quizás no se encuentren en condiciones de apreciar plenamente todas estas razones, quisiéramos llamar la atención sobre los puntos siguientes:

\par
%\textsuperscript{(726.3)}
\textsuperscript{64:6.31} 1. La variedad es indispensable para permitir el amplio funcionamiento de la selección natural, la supervivencia diferencial de las cepas superiores.

\par
%\textsuperscript{(726.4)}
\textsuperscript{64:6.32} 2. Se obtienen razas mejores y más fuertes mediante el cruce entre los diversos pueblos, cuando esas razas diferentes son portadoras de factores hereditarios superiores. Las razas de Urantia se hubieran beneficiado pronto de una fusión semejante, si un pueblo así de amalgamado hubiera podido después ser mejorado eficazmente mezclándose por completo con la raza adámica superior. En las condiciones raciales actuales, cualquier intento por llevar a cabo un experimento de este tipo en Urantia sería extremadamente desastroso.

\par
%\textsuperscript{(726.5)}
\textsuperscript{64:6.33} 3. La diversificación de las razas incita a una sana competición.

\par
%\textsuperscript{(726.6)}
\textsuperscript{64:6.34} 4. Las diferencias de categoría entre las razas, y entre los grupos dentro de cada raza, son esenciales para el desarrollo de la tolerancia y del altruismo humanos.

\par
%\textsuperscript{(726.7)}
\textsuperscript{64:6.35} 5. La homogeneidad de la raza humana no es deseable hasta que los pueblos de un mundo evolutivo no alcanzan unos niveles relativamente elevados de desarrollo espiritual.

\section*{7. La dispersión de las razas de color}
\par
%\textsuperscript{(726.8)}
\textsuperscript{64:7.1} Cuando los descendientes de color de la familia sangik empezaron a multiplicarse y a buscar la posibilidad de expandirse por los territorios vecinos, el quinto glaciar, el tercero según el cálculo de los geólogos, ya había avanzado mucho en su camino hacia el sur sobre Europa y Asia. Estas primeras razas de color sufrieron una prueba extraordinaria debido a los rigores y dificultades del período glaciar en el cual se originaron. Este glaciar era tan extenso en Asia, que la emigración hacia el este de Asia estuvo cortada durante miles de años. Y no les fue posible llegar a África hasta que el Mar Mediterráneo retrocedió posteriormente a consecuencia de la elevación de Arabia.

\par
%\textsuperscript{(726.9)}
\textsuperscript{64:7.2} Por este motivo, durante cerca de cien mil años, los pueblos sangiks se diseminaron alrededor de sus colinas y se mezclaron más o menos entre ellos, a pesar de las antipatías particulares, pero naturales, que se manifestaron desde el principio entre las diferentes razas.

\par
%\textsuperscript{(726.10)}
\textsuperscript{64:7.3} Entre la época del Príncipe Planetario y la de Adán, la India se convirtió en el hogar de la población más cosmopolita que se haya visto nunca sobre la faz de la Tierra. Pero es muy lamentable que esta mezcla contuviera tanta proporción de las razas verde, anaranjada e índiga. Estos pueblos sangiks secundarios encontraban la existencia más fácil y agradable en las tierras del sur, y muchos emigraron posteriormente a África. Los pueblos sangiks primarios, las razas superiores, evitaron los trópicos; el hombre rojo se dirigió hacia el nordeste hasta llegar a Asia, seguido de cerca por el hombre amarillo, mientras que la raza azul partió hacia el noroeste hasta entrar en Europa.

\par
%\textsuperscript{(727.1)}
\textsuperscript{64:7.4} Los hombres rojos empezaron pronto a emigrar hacia el nordeste, pisándole los talones a los hielos que retrocedían, rodearon las tierras altas de la India y ocuparon todo el nordeste de Asia. Fueron seguidos de cerca por las tribus amarillas, las cuales los echaron posteriormente de Asia hacia América del Norte.

\par
%\textsuperscript{(727.2)}
\textsuperscript{64:7.5} Cuando los restos relativamente puros de la raza roja abandonaron Asia, formaban once tribus y sumaban poco más de siete mil hombres, mujeres y niños. Estas tribus iban acompañadas de tres pequeños grupos de ascendencia mixta, y el más grande de ellos era una combinación de las razas anaranjada y azul. Estos tres grupos nunca fraternizaron por completo con los hombres rojos y pronto se dirigieron hacia el sur hasta Méjico y América Central, donde más tarde se unió a ellos un pequeño grupo de amarillos y rojos mezclados. Todos estos pueblos se casaron entre sí y fundaron una nueva raza amalgamada mucho menos belicosa que los hombres rojos de raza pura. En el espacio de cinco mil años, esta raza amalgamada se dividió en tres grupos, los cuales establecieron las civilizaciones respectivas de Méjico, América Central y América del Sur. La ramificación sudamericana recibió un ligero toque de la sangre de Adán.

\par
%\textsuperscript{(727.3)}
\textsuperscript{64:7.6} Los primeros hombres rojos y amarillos se mezclaron en Asia hasta cierto punto, y los descendientes de esta unión se dirigieron hacia el este y a lo largo de la costa meridional; con el tiempo, la raza amarilla que se multiplicaba con rapidez los empujó hacia las penínsulas y las islas cercanas. Son los hombres cobrizos de la actualidad.

\par
%\textsuperscript{(727.4)}
\textsuperscript{64:7.7} La raza amarilla ha continuado ocupando las regiones centrales de Asia oriental. De las seis razas de color, ésta es la que ha sobrevivido en mayor número. Aunque los hombres amarillos se enfrascaron de vez en cuando en guerras raciales, no mantuvieron las guerras de exterminio constantes e implacables que sostuvieron los hombres rojos, verdes y anaranjados. Estas tres razas se destruyeron prácticamente a sí mismas antes de ser finalmente casi aniquiladas por sus enemigos de las otras razas.

\par
%\textsuperscript{(727.5)}
\textsuperscript{64:7.8} Puesto que el quinto glaciar no se extendió mucho hacia el sur de Europa, estos pueblos sangiks tuvieron el camino parcialmente abierto para emigrar hacia el noroeste; cuando el hielo se retiró, los hombres azules, junto con otros grupos raciales pequeños, emigraron hacia el oeste siguiendo las antiguas pistas de las tribus de Andón. Invadieron Europa en oleadas sucesivas y ocuparon la mayor parte del continente.

\par
%\textsuperscript{(727.6)}
\textsuperscript{64:7.9} Pronto se encontraron en Europa con los descendientes neandertales de su antepasado primitivo común, Andón. Estos neandertales europeos más antiguos habían sido empujados hacia el sur y el este por el glaciar, y se hallaban así en condiciones de encontrar y absorber rápidamente a sus primos invasores de las tribus sangiks.

\par
%\textsuperscript{(727.7)}
\textsuperscript{64:7.10} Para empezar, las tribus sangiks eran en general más inteligentes que los descendientes degenerados de los primeros hombres andónicos de las llanuras, y muy superiores a ellos en casi todos los aspectos; la unión de estas tribus sangiks con los pueblos neandertales mejoró inmediatamente a la raza más antigua. Esta inyección de sangre sangik, principalmente la del hombre azul, fue la que produjo en los pueblos neandertales la mejora apreciable que se manifestó en las oleadas sucesivas de las tribus cada vez más inteligentes que se extendieron por Europa viniendo del este.

\par
%\textsuperscript{(727.8)}
\textsuperscript{64:7.11} Durante el período interglacial siguiente, esta nueva raza neandertal se extendió desde Inglaterra hasta la India. El resto de la raza azul que había permanecido en la antigua península pérsica se amalgamó más tarde con algunos otros, principalmente amarillos; la mezcla resultante, que posteriormente fue un poco mejorada por la raza violeta de Adán, ha sobrevivido bajo la forma de las tribus nómadas morenas de los árabes modernos.

\par
%\textsuperscript{(728.1)}
\textsuperscript{64:7.12} Todos los esfuerzos por identificar a los antepasados sangiks de los pueblos modernos han de tener en cuenta la mejora ulterior que los linajes raciales obtuvieron al mezclarse posteriormente con la sangre adámica.

\par
%\textsuperscript{(728.2)}
\textsuperscript{64:7.13} Las razas superiores buscaron los climas nórdicos o templados, mientras que las razas anaranjada, verde e índiga tendieron a dirigirse sucesivamente hacia África por el puente terrestre recién emergido que separaba al Mediterráneo, que se retiraba hacia el oeste, del Océano Índico.

\par
%\textsuperscript{(728.3)}
\textsuperscript{64:7.14} El hombre índigo fue el último pueblo sangik que emigró desde el centro de origen de las razas. Aproximadamente en la época en que el hombre verde exterminaba a la raza anaranjada en Egipto, debilitándose mucho él mismo al hacerlo, el gran éxodo negro se puso en camino hacia el sur a lo largo de la costa de Palestina. Más tarde, cuando estos pueblos índigos con un gran vigor físico invadieron Egipto, borraron de la existencia al hombre verde con la sola fuerza de su número. Estas razas índigas absorbieron los restos del hombre anaranjado y una gran parte de la raza del hombre verde, y algunas tribus índigas mejoraron considerablemente gracias a esta amalgamación racial.

\par
%\textsuperscript{(728.4)}
\textsuperscript{64:7.15} Se puede observar así que Egipto estuvo dominado en primer lugar por el hombre anaranjado, luego por el verde, seguido por el hombre índigo (negro), y más tarde aún por una raza mestiza de índigos, azules y hombres verdes modificados. Pero mucho antes de la llegada de Adán, los hombres azules de Europa y las razas mezcladas de Arabia habían arrojado a la raza índiga fuera de Egipto muy lejos hacia el sur del continente africano.

\par
%\textsuperscript{(728.5)}
\textsuperscript{64:7.16} A medida que las emigraciones sangiks se acercan a su fin, las razas verde y anaranjada ya no existen, el hombre rojo ocupa América del Norte, el hombre amarillo Asia oriental, el hombre azul Europa, y la raza índiga se ha dirigido a África. La India alberga una mezcla de las razas sangiks secundarias, y el hombre cobrizo, una mezcla del rojo y el amarillo, posee las islas que se encuentran a la altura de la costa asiática. Una raza amalgamada dotada de un potencial más bien superior ocupa las tierras altas de América del Sur. Los andonitas más puros viven en las regiones nórdicas extremas de Europa, en Islandia, Groenlandia y el nordeste de América del Norte.

\par
%\textsuperscript{(728.6)}
\textsuperscript{64:7.17} Durante los períodos de máximo avance glaciar, las tribus andonitas más occidentales estuvieron a punto de ser arrojadas al mar. Vivieron durante años en una estrecha franja de tierra al sur de la actual isla de Inglaterra. La tradición de estos repetidos avances glaciares fue la que los impulsó a hacerse a la mar cuando finalmente apareció el sexto y último glaciar. Fueron los primeros aventureros del mar. Construyeron unos barcos y partieron a la búsqueda de nuevas tierras con la esperanza de que estuvieran libres de las espantosas invasiones de hielo. Algunos llegaron a Islandia, otros a Groenlandia, pero la gran mayoría pereció de hambre y de sed en pleno mar.

\par
%\textsuperscript{(728.7)}
\textsuperscript{64:7.18} Hace poco más de ochenta mil años, poco después de que el hombre rojo penetrara por el noroeste en América del Norte, la congelación de los mares del norte y el avance de los campos de hielo locales en Groenlandia obligaron a estos descendientes esquimales de los aborígenes de Urantia a buscar una tierra mejor, un nuevo hogar. Y lo consiguieron, cruzando sanos y salvos los angostos estrechos que separaban entonces a Groenlandia de las masas terrestres del nordeste de Norteamérica. Alcanzaron el continente unos dos mil cien años después de que el hombre rojo llegara a Alaska. Posteriormente, algunos descendientes mestizos del hombre azul viajaron hacia el oeste y se amalgamaron con los esquimales más recientes, y esta unión fue ligeramente beneficiosa para las tribus esquimales.

\par
%\textsuperscript{(728.8)}
\textsuperscript{64:7.19} Hace unos cinco mil años, una tribu india y un grupo esquimal aislado se encontraron por casualidad en la costa sudeste de la Bahía de Hudson. Estas dos tribus tuvieron dificultades para comunicarse entre sí, pero muy pronto se casaron entre ellos con el resultado de que estos esquimales fueron absorbidos finalmente por los hombres rojos más numerosos. Éste es el único contacto que tuvo el hombre rojo norteamericano con otra raza humana hasta hace aproximadamente mil años, cuando el hombre blanco desembarcó casualmente por primera vez en la costa atlántica.

\par
%\textsuperscript{(729.1)}
\textsuperscript{64:7.20} Las luchas de estas épocas primitivas estuvieron caracterizadas por el coraje, la valentía e incluso el heroísmo. Todos lamentamos que tantos de aquellos rasgos robustos y excelentes de vuestros primeros antepasados se hayan perdido para las razas más recientes. Aunque apreciamos el valor de muchos refinamientos de la civilización que progresa, echamos de menos la magnífica obstinación y la espléndida dedicación de vuestros primeros antepasados, las cuales rayaban a veces en la grandeza y la sublimidad.

\par
%\textsuperscript{(729.2)}
\textsuperscript{64:7.21} [Presentado por un Portador de Vida, residente en Urantia.]


\chapter{Documento 65. El supercontrol de la evolución}
\par
%\textsuperscript{(730.1)}
\textsuperscript{65:0.1} LA VIDA material evolutiva de base ---la vida anterior a la mente--- es formulada por los Controladores Físicos Maestros y conferida por el ministerio de los Siete Espíritus Maestros en asociación con los servicios activos de los Portadores de Vida encargados de ello. Debido al funcionamiento coordinado de esta triple actividad creadora, se desarrolla en el organismo una capacidad física para alojar a la mente ---unos mecanismos materiales destinados a reaccionar de manera inteligente a los estímulos ambientales externos y, más tarde, a los estímulos internos, a esas influencias que se originan en la mente misma del organismo.

\par
%\textsuperscript{(730.2)}
\textsuperscript{65:0.2} Existen, pues, tres niveles distintos de producción y de evolución de la vida:

\par
%\textsuperscript{(730.3)}
\textsuperscript{65:0.3} 1. El ámbito físico-energético ---la producción de la capacidad mental.

\par
%\textsuperscript{(730.4)}
\textsuperscript{65:0.4} 2. El ministerio mental de los espíritus ayudantes ---que incide en la capacidad espiritual.

\par
%\textsuperscript{(730.5)}
\textsuperscript{65:0.5} 3. La dotación espiritual de la mente mortal ---que culmina en el otorgamiento de los Ajustadores del Pensamiento.

\par
%\textsuperscript{(730.6)}
\textsuperscript{65:0.6} Los niveles maquinales y no enseñables de reacción al entorno que poseen los organismos pertenecen al ámbito de los controladores físicos. Los espíritus ayudantes de la mente activan y regulan los tipos de mentes adaptables o no maquinales y enseñables ---esos mecanismos reactivos de los organismos que son capaces de aprender por experiencia. De la misma manera que los espíritus ayudantes manipulan así los potenciales de la mente, los Portadores de Vida ejercen un considerable control discrecional sobre los aspectos ambientales de los procesos evolutivos, hasta el momento en que aparece la voluntad humana ---la capacidad para conocer a Dios y el poder de elegir adorarlo.

\par
%\textsuperscript{(730.7)}
\textsuperscript{65:0.7} El funcionamiento integrado de los Portadores de Vida, los controladores físicos y los espíritus ayudantes es el que condiciona el curso de la evolución orgánica en los mundos habitados. Por eso la evolución ---en Urantia o en otro lugar--- siempre es intencional y nunca accidental.

\section*{1. Las funciones de los Portadores de Vida}
\par
%\textsuperscript{(730.8)}
\textsuperscript{65:1.1} Los Portadores de Vida están dotados de unos potenciales de metamorfosis de la personalidad que muy pocas clases de criaturas poseen. Estos Hijos del universo local son capaces de ejercer su actividad en tres fases diferentes de existencia. Normalmente desempeñan sus tareas como Hijos de la fase media, siendo éste su estado original. Pero un Portador de Vida en ese estado de existencia no podría actuar de ninguna manera en el ámbito electroquímico como transformador de las energías físicas y de las partículas materiales en unidades de existencia viviente.

\par
%\textsuperscript{(730.9)}
\textsuperscript{65:1.2} Los Portadores de Vida son capaces de actuar, y actúan de hecho, en los tres niveles siguientes:

\par
%\textsuperscript{(730.10)}
\textsuperscript{65:1.3} 1. El nivel físico de la electroquímica.

\par
%\textsuperscript{(730.11)}
\textsuperscript{65:1.4} 2. La fase media habitual de existencia casi morontial.

\par
%\textsuperscript{(730.12)}
\textsuperscript{65:1.5} 3. El nivel semiespiritual avanzado.

\par
%\textsuperscript{(731.1)}
\textsuperscript{65:1.6} Cuando los Portadores de Vida se preparan para emprender una implantación de vida, y después de haber escogido los emplazamientos para tal empresa, convocan a la comisión arcangélica para la transmutación de los Portadores de Vida. Este grupo está compuesto de diez órdenes de personalidades diversas, incluyendo a los controladores físicos y sus asociados, y lo preside el jefe de los arcángeles, que actúa con esta autoridad por mandato de Gabriel y con el permiso de los Ancianos de los Días. Cuando estos seres están situados en circuito de manera adecuada, pueden efectuar sobre los Portadores de Vida las modificaciones que les permitirán funcionar inmediatamente en los niveles físicos de la electroquímica.

\par
%\textsuperscript{(731.2)}
\textsuperscript{65:1.7} Después de que los modelos de vida se han formulado y las organizaciones materiales se han concluido debidamente, las fuerzas supermateriales implicadas en la propagación de la vida se activan enseguida, y la vida existe. Entonces, los Portadores de Vida son devueltos inmediatamente a su fase media normal de existencia de la personalidad, en cuyo estado pueden manipular las unidades vivientes y manejar los organismos en evolución, aunque están despojados de toda capacidad para organizar ---para crear--- nuevos modelos de materia viviente.

\par
%\textsuperscript{(731.3)}
\textsuperscript{65:1.8} Después de que la evolución orgánica ha alcanzado cierto nivel y el libre albedrío de tipo humano ha aparecido en los organismos evolutivos más elevados, los Portadores de Vida deben abandonar el planeta o bien hacer una promesa solemne de renuncia; es decir, que deben comprometerse a abstenerse de todo intento por influir posteriormente en el curso de la evolución orgánica. Cuando esta promesa es pronunciada voluntariamente por los Portadores de Vida que eligen permanecer en el planeta para aconsejar en el futuro a los que estarán encargados de favorecer a las criaturas volitivas recién aparecidas por evolución, se convoca una comisión de doce miembros, presidida por el jefe de las Estrellas Vespertinas, que actúa por autorización del Soberano del Sistema y con el permiso de Gabriel; y estos Portadores de Vida son transmutados inmediatamente a la tercera fase de existencia de la personalidad ---al nivel semiespiritual de existencia. Y he trabajado en Urantia, en esta tercera fase de existencia, desde los tiempos de Andón y Fonta.

\par
%\textsuperscript{(731.4)}
\textsuperscript{65:1.9} Esperamos con ansia la época en que el universo estará establecido en la luz y la vida, y logremos un posible cuarto estado de existencia en el cual seremos totalmente espirituales; pero nunca se nos ha revelado la técnica por la cual podremos alcanzar ese estado deseable y avanzado.

\section*{2. El panorama de la evolución}
\par
%\textsuperscript{(731.5)}
\textsuperscript{65:2.1} La historia de la ascensión del hombre desde las algas marinas hasta el dominio de la creación terrestre es, en verdad, una aventura de luchas biológicas y de supervivencia mental. Los antepasados primordiales del hombre fueron literalmente el limo y el cieno del fondo oceánico, depositados en las bahías y lagunas de aguas cálidas y tranquilas de los extensos litorales de los antiguos mares interiores, las mismas aguas en las que los Portadores de Vida establecieron las tres implantaciones independientes de vida en Urantia.

\par
%\textsuperscript{(731.6)}
\textsuperscript{65:2.2} Existen en la actualidad muy pocas especies de los primeros tipos de vegetales marinos que participaron en los cambios históricos que dieron como resultado los organismos situados en la frontera de la vida animal. Las esponjas son las supervivientes de uno de estos tipos intermedios primitivos, de esos organismos a través de los cuales se produjo la transición \textit{gradual} del vegetal al animal. Estas primeras formas transitorias no eran idénticas a las esponjas modernas, pero sí muy similares a ellas; fueron unos organismos verdaderamente limítrofes ---ni vegetales ni animales--- pero condujeron finalmente al desarrollo de las verdaderas formas de vida animal.

\par
%\textsuperscript{(732.1)}
\textsuperscript{65:2.3} Las bacterias, unos simples organismos vegetales de naturaleza muy primitiva, han cambiado muy poco desde los primeros albores de la vida; incluso muestran cierto grado de retroceso en su comportamiento parasitario. Muchos hongos representan también un movimiento retrógrado en la evolución, pues se trata de plantas que han perdido su capacidad para fabricar clorofila y se han vuelto más o menos parasitarias. La mayoría de las bacterias que producen las enfermedades, y sus cuerpos auxiliares los virus, pertenecen en realidad a este grupo de hongos parasitarios renegados. Durante las épocas intermedias, todo el inmenso reino de la vida vegetal evolucionó a partir de unos antepasados de los que descienden también las bacterias.

\par
%\textsuperscript{(732.2)}
\textsuperscript{65:2.4} Pronto apareció, y apareció \textit{repentinamente}, el tipo protozoario más elevado de la vida animal. La ameba, el típico organismo animal unicelular, ha llegado desde aquellos tiempos tan lejanos hasta nuestros días con pocas modificaciones. Hoy retoza de manera muy parecida a como lo hacía cuando era el último logro más importante de la evolución de la vida. Esta criatura diminuta y sus primos protozoarios son, para la creación animal, lo mismo que las bacterias para el reino vegetal; representan la supervivencia de las primeras etapas evolutivas en la diferenciación de la vida, así como un \textit{fracaso en su desarrollo posterior}.

\par
%\textsuperscript{(732.3)}
\textsuperscript{65:2.5} Los primeros tipos de animales unicelulares no tardaron en asociarse en comunidades, al principio siguiendo la disposición del volvox, y luego a la manera de la hidra y la medusa. Más tarde aún aparecieron por evolución la estrella de mar, los crinoideos, erizos de mar, holoturias, ciempiés, insectos, arañas, crustáceos y los grupos estrechamente emparentados de los gusanos y las sanguijuelas, seguidos de cerca por los moluscos ---la ostra, el pulpo y el caracol. Cientos y cientos de especies aparecieron y perecieron; sólo mencionamos a aquellas que sobrevivieron a la interminable lucha. Estos especímenes no progresivos, así como la familia de los peces que apareció más tarde, representan en la actualidad los tipos estacionarios de animales primitivos e inferiores, las ramas del árbol de la vida que no lograron progresar.

\par
%\textsuperscript{(732.4)}
\textsuperscript{65:2.6} El escenario estaba así preparado para la aparición de los primeros animales vertebrados, los peces. De esta familia de los peces surgieron dos modificaciones excepcionales: la rana y la salamandra. Y fue la rana la que empezó, dentro de la vida animal, la serie de diferenciaciones progresivas que culminaron finalmente en el hombre mismo.

\par
%\textsuperscript{(732.5)}
\textsuperscript{65:2.7} La rana es uno de los antepasados supervivientes más primitivos de la raza humana, pero tampoco logró progresar, y su aspecto de hoy se parece mucho al de aquellos tiempos lejanos. La rana es la única especie ancestral de los albores de las razas que vive hoy en día sobre la faz de la Tierra. La raza humana no posee ningún antepasado que haya sobrevivido entre la rana y el esquimal.

\par
%\textsuperscript{(732.6)}
\textsuperscript{65:2.8} Las ranas dieron nacimiento a los reptiles, una gran familia animal prácticamente extinguida, pero que antes de desaparecer dio origen a toda la familia de las aves y a las numerosas clases de mamíferos.

\par
%\textsuperscript{(732.7)}
\textsuperscript{65:2.9} El salto aislado más grande de toda la evolución prehumana se llevó a cabo probablemente cuando el reptil se convirtió en un ave. Los tipos de aves actuales ---águilas, patos, palomas y avestruces--- descienden todos de los enormes reptiles de los tiempos prehistóricos.

\par
%\textsuperscript{(732.8)}
\textsuperscript{65:2.10} El reino de los reptiles, descendiente de la familia de las ranas, está representado actualmente por cuatro divisiones supervivientes: dos no progresivas, las serpientes y los lagartos, junto con sus primos los cocodrilos y las tortugas; una parcialmente progresiva, la familia de las aves; y la cuarta representa a los antepasados de los mamíferos y a la línea que desciende directamente hasta la especie humana. Aunque los reptiles del pasado desaparecieron hace mucho tiempo, su aspecto macizo encontró resonancia en el elefante y el mastodonte, mientras que sus formas particulares se perpetuaron en los canguros saltadores.

\par
%\textsuperscript{(733.1)}
\textsuperscript{65:2.11} En Urantia sólo han aparecido catorce phyla, siendo los peces el último de ellas, y no se ha desarrollado ninguna clase nueva después de las aves y los mamíferos.

\par
%\textsuperscript{(733.2)}
\textsuperscript{65:2.12} Los mamíferos placentarios surgieron \textit{repentinamente} de un ágil y pequeño dinosaurio reptil de hábitos carnívoros, pero provisto de un cerebro relativamente grande. Estos mamíferos se desarrollaron rápidamente y de muchas maneras diferentes, dando nacimiento no solamente a las variedades comunes modernas, sino que evolucionaron también hacia los tipos marinos tales como las ballenas y las focas, y hacia los navegantes aéreos como la familia de los murciélagos.

\par
%\textsuperscript{(733.3)}
\textsuperscript{65:2.13} El hombre se desarrolló pues a partir de los mamíferos superiores procedentes principalmente de la \textit{implantación occidental} de vida que se había efectuado en los antiguos mares abrigados situados entre el este y el oeste. El \textit{grupo oriental} y \textit{el grupo central} de organismos vivientes pronto progresaron favorablemente hacia la conquista de niveles prehumanos de existencia animal. Pero a medida que pasaban las épocas, el foco oriental de vida no logró alcanzar un nivel satisfactorio de inteligencia prehumana, pues había sufrido tales pérdidas repetidas e irreparables en sus tipos superiores de plasma germinal, que quedó privado para siempre de la capacidad de rehabilitar sus potencialidades humanas.

\par
%\textsuperscript{(733.4)}
\textsuperscript{65:2.14} Como la calidad de la capacidad mental para desarrollarse, en este grupo oriental, era tan claramente inferior a la de los otros dos grupos, los Portadores de Vida, con la aprobación de sus superiores, manipularon el entorno de tal manera que circunscribieron aún más estas cepas prehumanas inferiores de la vida evolutiva. Según las apariencias exteriores, la eliminación de estos grupos inferiores de criaturas fue accidental, pero en realidad fue enteramente intencional.

\par
%\textsuperscript{(733.5)}
\textsuperscript{65:2.15} En una fecha posterior del desarrollo evolutivo de la inteligencia, los antepasados lémures de la especie humana estaban mucho más avanzados en Norteamérica que en otras regiones; por eso fueron inducidos a emigrar desde el área de implantación de vida occidental, pasando por el puente terrestre de Bering y a lo largo de la costa, hasta el sudoeste de Asia, donde continuaron evolucionando y se beneficiaron de la adición de ciertas cepas del grupo central de vida. El hombre evolucionó así a partir de ciertas cepas vitales del centro-oeste, pero en las regiones centrales y próximo-orientales.

\par
%\textsuperscript{(733.6)}
\textsuperscript{65:2.16} La vida que se había plantado en Urantia evolucionó de esta manera hasta el período glaciar, época en que el hombre mismo apareció por primera vez y empezó su agitada carrera planetaria. Esta aparición del hombre primitivo en la Tierra durante el período glaciar no fue precisamente un accidente; fue intencional. Los rigores y la severidad climática de la era glaciar se adaptaban en todos los aspectos a la finalidad de fomentar la producción de un tipo robusto de ser humano, dotado de una formidable capacidad para sobrevivir.

\section*{3. El fomento de la evolución}
\par
%\textsuperscript{(733.7)}
\textsuperscript{65:3.1} Será muy difícil explicarle a la mente humana actual muchos sucesos extraños y aparentemente grotescos del progreso evolutivo inicial. A lo largo de todas estas evoluciones aparentemente extrañas de seres vivientes estaba funcionando un plan intencional, pero no nos está permitido intervenir arbitrariamente en el desarrollo de los modelos de vida una vez que se han activado.

\par
%\textsuperscript{(733.8)}
\textsuperscript{65:3.2} Los Portadores de Vida pueden emplear todos los recursos naturales posibles y utilizar todas y cada una de las circunstancias fortuitas que mejoren el progreso y el desarrollo del experimento de la vida, pero no nos está permitido intervenir mecánicamente en la evolución vegetal o animal, ni manipular arbitrariamente su conducta o su rumbo.

\par
%\textsuperscript{(733.9)}
\textsuperscript{65:3.3} Habéis sido informados de que los mortales de Urantia se desarrollaron pasando por la evolución de una rana primitiva, y que esta cepa ascendiente, contenida en potencia dentro de una sola rana, por poco se destruye en cierta ocasión. Pero no se debe deducir que la evolución de la humanidad hubiera terminado debido a un accidente en esta coyuntura. En aquel mismo momento estábamos observando y fomentando no menos de mil cepas de vida mutantes, diferentes y alejadas entre sí, que podían haber sido dirigidas hacia diversos modelos de desarrollo prehumano. Esta rana ancestral particular representaba nuestra tercera selección, pues las dos cepas de vida anteriores habían perecido a pesar de todos nuestros esfuerzos por conservarlas.

\par
%\textsuperscript{(734.1)}
\textsuperscript{65:3.4} Incluso la pérdida de Andón y Fonta antes de que tuvieran descendencia no hubiera impedido la evolución humana, aunque la habría retrasado. Después de la aparición de Andón y Fonta, y antes de que se agotaran los potenciales humanos en mutación de la vida animal, evolucionaron no menos de siete mil cepas favorables que podrían haber alcanzado alguna clase de desarrollo de tipo humano. Muchas de estas mejores cepas fueron asimiladas posteriormente por las diversas ramas de la especie humana en expansión.

\par
%\textsuperscript{(734.2)}
\textsuperscript{65:3.5} Mucho antes de que el Hijo y la Hija Materiales, los mejoradores biológicos, lleguen a un planeta, los potenciales humanos de las especies animales en evolución ya se han agotado. Este estado biológico de la vida animal es revelado a los Portadores de Vida mediante el fenómeno de la tercera fase de movilización de los espíritus ayudantes, que se produce automáticamente en el mismo momento en que toda la vida animal ha agotado su capacidad para dar nacimiento a los potenciales mutantes de los individuos prehumanos.

\par
%\textsuperscript{(734.3)}
\textsuperscript{65:3.6} La humanidad de Urantia debe resolver sus problemas de desarrollo mortal con los linajes humanos que posee ---ninguna nueva raza volverá a aparecer en el futuro a partir de fuentes prehumanas. Pero este hecho no impide la posibilidad de alcanzar unos niveles muy superiores de desarrollo humano mediante el fomento inteligente de los potenciales evolutivos que residen todavía en las razas mortales. Aquello que nosotros, los Portadores de Vida, hacemos para fomentar y conservar las cepas de vida antes de que aparezca la voluntad humana, el hombre debe hacerlo por sí mismo después de ese acontecimiento, cuando ya nos hemos retirado de toda participación activa en la evolución. El destino evolutivo del hombre se encuentra de manera general en sus propias manos, y tarde o temprano la inteligencia científica debe reemplazar el funcionamiento aleatorio de una selección natural no controlada y de una supervivencia sometida a la casualidad.

\par
%\textsuperscript{(734.4)}
\textsuperscript{65:3.7} Y hablando de fomento de la evolución, no sería inoportuno indicar que si en el lejano futuro que tenéis por delante alguna vez os vinculáis a un cuerpo de Portadores de Vida, dispondréis de amplias y abundantes ocasiones para ofrecer vuestras sugerencias y aportar todas las mejoras posibles a los planes y técnicas de gestión y trasplante de la vida. ¡Tened paciencia! Si tenéis buenas ideas, si vuestra imaginación es fértil en mejores métodos de administración para cualquier parte de los dominios universales, tendréis ciertamente la oportunidad de presentarlos a vuestros asociados y compañeros administradores en las épocas venideras.

\section*{4. La aventura urantiana}
\par
%\textsuperscript{(734.5)}
\textsuperscript{65:4.1} No olvidéis el hecho de que Urantia nos fue asignada como mundo para experimentar con la vida. En este planeta efectuamos nuestro sexagésimo intento para modificar y mejorar, si fuera posible, la adaptación sataniana de los diseños de vida de Nebadon, y consta en los registros que realizamos numerosas modificaciones beneficiosas en los modelos de vida normales. Para ser precisos, en Urantia elaboramos e hicimos la demostración satisfactoria de no menos de veintiocho características de modificación de la vida, que serán útiles para todo Nebadon en todas las épocas venideras.

\par
%\textsuperscript{(735.1)}
\textsuperscript{65:4.2} Pero el establecimiento de la vida nunca es experimental en ningún mundo, en el sentido de intentar algo desconocido y que no se ha probado. La evolución de la vida es una técnica siempre progresiva, diferencial y variable, pero nunca fortuita, incontrolada ni totalmente experimental en el sentido accidental.

\par
%\textsuperscript{(735.2)}
\textsuperscript{65:4.3} Muchas características de la vida humana proporcionan abundantes pruebas de que el fenómeno de la existencia mortal fue planeado de manera inteligente, que la evolución orgánica no es un simple accidente cósmico. Cuando una célula viviente es lesionada, posee la capacidad de elaborar ciertas sustancias químicas que tienen la facultad de estimular y activar las células normales vecinas, de tal manera que éstas empiezan inmediatamente a secretar ciertas sustancias que facilitan los procesos curativos de la herida. Al mismo tiempo, estas células normales no lesionadas empiezan a proliferar ---se ponen a trabajar realmente para crear nuevas células que reemplacen a todas las células semejantes que puedan haber sido destruidas por el accidente.

\par
%\textsuperscript{(735.3)}
\textsuperscript{65:4.4} Esta acción y esta reacción químicas implicadas en la curación de las heridas y en la reproducción de las células representan la elección, efectuada por los Portadores de Vida, de una fórmula que abarca más de cien mil fases y características de reacciones químicas y de repercusiones biológicas posibles. Los Portadores de Vida realizaron en sus laboratorios más de medio millón de experimentos específicos antes de decidirse finalmente por esta fórmula para experimentar con la vida en Urantia.

\par
%\textsuperscript{(735.4)}
\textsuperscript{65:4.5} Cuando los científicos de Urantia conozcan mejor estas sustancias químicas curativas, serán más eficaces en el tratamiento de las heridas, e indirectamente sabrán controlar mejor ciertas enfermedades graves.

\par
%\textsuperscript{(735.5)}
\textsuperscript{65:4.6} Desde que la vida se estableció en Urantia, los Portadores de Vida han mejorado esta técnica curativa introduciéndola en otro mundo de Satania, donde proporciona más alivio al dolor y ejerce un mejor control sobre la capacidad de proliferación de las células normales asociadas.

\par
%\textsuperscript{(735.6)}
\textsuperscript{65:4.7} Hubo muchas características excepcionales en el experimento con la vida en Urantia, pero los dos episodios más sobresalientes fueron la aparición de la raza andónica antes de la evolución de los seis pueblos de color y, más tarde, la aparición simultánea de los mutantes sangiks en una sola familia. Urantia es el primer mundo de Satania donde las seis razas de color nacieron de la misma familia humana. Normalmente suelen surgir, en linajes diversos, a partir de mutaciones independientes dentro de la cepa animal prehumana, y generalmente aparecen en el mundo de una en una y de manera sucesiva a lo largo de grandes períodos de tiempo, empezando por el hombre rojo y pasando por todos los colores hasta llegar al índigo.

\par
%\textsuperscript{(735.7)}
\textsuperscript{65:4.8} Otra variación sobresaliente de procedimiento fue la llegada tardía del Príncipe Planetario. Por regla general, el príncipe aparece en un planeta aproximadamente en el momento en que se desarrolla la voluntad; si este plan se hubiera seguido, Caligastia podría haber llegado a Urantia incluso durante la vida de Andón y Fonta, en lugar de hacerlo casi quinientos mil años después, simultáneamente con la aparición de las seis razas sangiks.

\par
%\textsuperscript{(735.8)}
\textsuperscript{65:4.9} En un mundo habitado normal, un Príncipe Planetario habría sido concedido a petición de los Portadores de Vida en el momento de la aparición de Andón y Fonta, o poco tiempo después. Pero como Urantia había sido designada como planeta de modificación de la vida, los observadores Melquisedeks, doce en total, fueron enviados por acuerdo previo como consejeros de los Portadores de Vida y como supervisores del planeta hasta la llegada posterior del Príncipe Planetario. Estos Melquisedeks llegaron en el momento en que Andón y Fonta tomaron las decisiones que permitieron a los Ajustadores del Pensamiento venir a residir en su mente mortal.

\par
%\textsuperscript{(736.1)}
\textsuperscript{65:4.10} Los esfuerzos realizados en Urantia por los Portadores de Vida para mejorar los modelos de vida de Satania tuvieron como resultado necesario la producción de numerosas formas de vida transitorias, aparentemente inútiles. Pero los beneficios ya acumulados son suficientes para justificar las modificaciones urantianas efectuadas en los diseños de vida normales.

\par
%\textsuperscript{(736.2)}
\textsuperscript{65:4.11} Teníamos la intención de producir una temprana manifestación de la voluntad en la vida evolutiva de Urantia, y lo conseguimos. La voluntad no surge habitualmente hasta mucho tiempo después del nacimiento de las razas de color, y generalmente aparece por primera vez entre los tipos superiores del hombre rojo. Vuestro mundo es el único planeta de Satania donde el tipo humano de voluntad ha aparecido en una raza anterior a las de color.

\par
%\textsuperscript{(736.3)}
\textsuperscript{65:4.12} Pero en nuestro esfuerzo por asegurar esta combinación y asociación de factores hereditarios que finalmente dieron origen a los antepasados mamíferos de la raza humana, nos enfrentamos con la necesidad de permitir que se produjeran cientos de miles de otras combinaciones y asociaciones de factores hereditarios relativamente inútiles. Cuando investiguéis el pasado del planeta, vuestra mirada se encontrará seguramente con muchos de estos subproductos, aparentemente extraños, de nuestros esfuerzos, y puedo comprender muy bien cuán enigmáticas deben ser algunas de estas cosas para el punto de vista limitado de los hombres.

\section*{5. Las vicisitudes de la evolución de la vida}
\par
%\textsuperscript{(736.4)}
\textsuperscript{65:5.1} Para los Portadores de Vida supuso una gran pena que nuestros esfuerzos especiales por modificar la vida inteligente en Urantia encontraran tantos obstáculos debido a unas trágicas perversiones que estaban más allá de nuestro control: la traición de Caligastia y la falta de Adán.

\par
%\textsuperscript{(736.5)}
\textsuperscript{65:5.2} Pero durante toda esta aventura biológica, nuestra mayor decepción fue el retroceso de ciertas plantas primitivas hasta los niveles preclorofílicos de las bacterias parasitarias, y que se produjera a una escala tan grande e inesperada. Esta eventualidad en la evolución de la vida de las plantas ha causado muchas enfermedades desoladoras en los mamíferos superiores, principalmente en la especie humana más vulnerable. Cuando nos enfrentamos con esta complicada situación, disminuimos un poco las dificultades implícitas porque sabíamos que la dosis posterior del plasma vital adámico reforzaría de tal manera la capacidad de resistencia de la raza mezclada resultante, que la inmunizaría prácticamente contra todas las enfermedades producidas por este tipo de organismo vegetal. Pero nuestras esperanzas estaban condenadas a sufrir una decepción debido a la desgracia de la falta adámica.

\par
%\textsuperscript{(736.6)}
\textsuperscript{65:5.3} El universo de universos, incluido este pequeño mundo llamado Urantia, no está gobernado simplemente para recibir nuestra aprobación ni para adaptarse a nuestra conveniencia, y mucho menos para agradar nuestros caprichos y satisfacer nuestra curiosidad. Los seres sabios y todopoderosos que tienen la responsabilidad de administrar el universo saben, sin ninguna duda, exactamente lo que tienen que hacer. Por eso conviene a los Portadores de Vida e incumbe a la mente mortal alistarse, mediante una espera paciente y una cooperación sincera, con la regla de la sabiduría, el reino del poder y la marcha del progreso.

\par
%\textsuperscript{(736.7)}
\textsuperscript{65:5.4} Existen, por supuesto, ciertas compensaciones por las tribulaciones, tales como la donación de Miguel en Urantia. Pero independientemente de todas estas consideraciones, los supervisores celestiales más recientes de este planeta expresan su total confianza en el triunfo evolutivo último de la raza humana y en la justificación final de nuestros planes y modelos de vida originales.

\section*{6. Las técnicas evolutivas de la vida}
\par
%\textsuperscript{(737.1)}
\textsuperscript{65:6.1} Es imposible determinar con precisión, y de manera simultánea, la posición exacta y la velocidad de un objeto en movimiento; cualquier intento por medir una de ellas implica inevitablemente una modificación de la otra. El hombre mortal se enfrenta con el mismo tipo de paradoja cuando emprende el análisis químico del protoplasma. El químico puede dilucidar la composición química del protoplasma \textit{muerto}, pero no puede percibir la organización física ni el comportamiento dinámico del protoplasma \textit{vivo}. El científico se acercará siempre cada vez más a los secretos de la vida, pero nunca los descubrirá por la sencilla razón de que debe matar al protoplasma para poder analizarlo. El protoplasma muerto pesa lo mismo que el protoplasma vivo, pero no es el mismo.

\par
%\textsuperscript{(737.2)}
\textsuperscript{65:6.2} Existe un don original de adaptación en las criaturas y los seres vivos. En cada célula \textit{viviente} animal o vegetal, en cada organismo \textit{vivo} ---material o espiritual--- existe un deseo insaciable por alcanzar una perfección cada vez mayor de ajuste al entorno, de adaptación del organismo, y de conseguir una vida mejor. Estos esfuerzos interminables de todas las criaturas vivientes demuestran que dentro de ellas existe una lucha innata por la perfección.

\par
%\textsuperscript{(737.3)}
\textsuperscript{65:6.3} La etapa más importante de la evolución vegetal fue el desarrollo de la capacidad para fabricar la clorofila, y el segundo avance en importancia fue la transformación evolutiva de la espora en una semilla compleja. La espora es extremadamente eficaz como agente reproductor, pero carece de los potenciales de variedad y versatilidad inherentes a la semilla.

\par
%\textsuperscript{(737.4)}
\textsuperscript{65:6.4} Uno de los episodios más útiles y complejos de la evolución de los tipos superiores de animales consistió en el desarrollo de la capacidad del hierro, dentro de los glóbulos que circulan en la sangre, para efectuar la doble tarea de transportar el oxígeno y eliminar el dióxido de carbono. Y esta labor de los glóbulos rojos ilustra la manera en que los organismos en evolución son capaces de adaptar sus funciones a un entorno variable o cambiante. Los animales superiores, incluído el hombre, oxigenan sus tejidos gracias a la acción del hierro contenido en los glóbulos rojos de la sangre, el cual transporta el oxígeno hasta las células vivas y, con la misma eficacia, elimina el dióxido de carbono. Sin embargo, se pueden utilizar otros metales para conseguir el mismo fin. La jibia emplea el cobre para esta función, y la ascidia utiliza el vanadio.

\par
%\textsuperscript{(737.5)}
\textsuperscript{65:6.5} La continuidad de estos ajustes biológicos queda ilustrada en la evolución de los dientes de los mamíferos superiores de Urantia. Los antepasados lejanos del hombre tuvieron hasta treinta y seis dientes, y luego empezó un reajuste adaptativo hacia los treinta y dos dientes del hombre primitivo y sus parientes cercanos. En la actualidad, la especie humana tiende lentamente a tener veintiocho dientes. El proceso de la evolución continúa progresando activamente y adaptándose a las circunstancias de este planeta.

\par
%\textsuperscript{(737.6)}
\textsuperscript{65:6.6} Pero muchos ajustes aparentemente misteriosos de los organismos vivientes son puramente químicos, totalmente físicos. En cualquier momento existe la posibilidad de que ocurran, en la corriente sanguínea de cualquier ser humano, más de 15.000.000 de reacciones químicas entre la producción hormonal de una docena de glándulas endocrinas.

\par
%\textsuperscript{(737.7)}
\textsuperscript{65:6.7} Las formas inferiores de la vida vegetal son totalmente sensibles al entorno físico, químico y eléctrico. Pero a medida que se asciende en la escala de la vida, los servicios mentales de los siete espíritus ayudantes entran en acción uno tras otro, y la mente tiende a ajustar, crear, coordinar y dominar cada vez más. La capacidad de los animales para adaptarse al aire, al agua y a la tierra no es un don sobrenatural, sino un ajuste superfísico.

\par
%\textsuperscript{(738.1)}
\textsuperscript{65:6.8} La física y la química solas no pueden explicar cómo surgió el ser humano por evolución a partir del protoplasma primitivo de los primeros mares. La capacidad para aprender, la memoria y la reacción diferencial al entorno, es un atributo de la mente. Las leyes de la física no son sensibles a la enseñanza; son inmutables e invariables. Las reacciones de la química no son modificadas por la educación; son uniformes y fiables. Aparte de la presencia del Absoluto Incalificado, las reacciones eléctricas y químicas son previsibles. Pero la mente puede beneficiarse de la experiencia, puede aprender de los hábitos reactivos del comportamiento en respuesta a la repetición de los estímulos.

\par
%\textsuperscript{(738.2)}
\textsuperscript{65:6.9} Los organismos preinteligentes reaccionan a los estímulos del entorno, pero los organismos que reaccionan al ministerio de la mente pueden ajustar y manipular el entorno mismo.

\par
%\textsuperscript{(738.3)}
\textsuperscript{65:6.10} El cerebro físico con su sistema nervioso asociado posee una capacidad innata para responder al ministerio de la mente, tal como la mente en desarrollo de una personalidad posee cierta capacidad innata para la receptividad espiritual, y contiene por tanto los potenciales para el progreso y la consecución espirituales. La evolución intelectual, social, moral y espiritual depende del ministerio mental de los siete espíritus ayudantes y sus asociados superfísicos.

\section*{7. Los niveles evolutivos de la mente}
\par
%\textsuperscript{(738.4)}
\textsuperscript{65:7.1} Los siete espíritus ayudantes de la mente son los polifacéticos ministros mentales para los seres inteligentes inferiores de un universo local. Este tipo de mente es administrada desde la sede del universo local o desde algún mundo conectado con ella, pero las capitales de los sistemas ejercen una dirección influyente sobre la función mental inferior.

\par
%\textsuperscript{(738.5)}
\textsuperscript{65:7.2} En un mundo evolutivo hay muchísimas cosas que dependen de la labor de estos siete ayudantes. Pero son ministros de la mente, y no se ocupan de la evolución física, que es el terreno de los Portadores de Vida. Sin embargo, la integración perfecta de estos dones del espíritu con el procedimiento natural y ordenado del régimen inherente, y en proceso de desarrollo, de los Portadores de Vida, es responsable de la incapacidad que tienen los mortales para discernir, en el fenómeno de la mente, otra cosa que la mano de la naturaleza y el trabajo de los procesos naturales, aunque a veces os sentís un poco confusos para poder explicar todo lo que está relacionado con las reacciones naturales de la mente cuando está asociada con la materia. Y si Urantia funcionara más en consonancia con los planes originales, observaríais aún menos cosas que atraerían vuestra atención sobre el fenómeno de la mente.

\par
%\textsuperscript{(738.6)}
\textsuperscript{65:7.3} Los siete espíritus ayudantes se parecen más a unos circuitos que a unas entidades, y en los mundos normales están conectados con otras funciones de ayuda que se efectúan en todo el universo local. Sin embargo, en los planetas donde se experimenta con la vida, están relativamente aislados. Y en Urantia, dada la naturaleza excepcional de los modelos de vida, los ayudantes inferiores tuvieron muchas más dificultades para ponerse en contacto con los organismos evolutivos que las que hubieran tenido con un tipo de dotación vital más normalizado.

\par
%\textsuperscript{(738.7)}
\textsuperscript{65:7.4} Por otra parte, en un mundo evolutivo medio, los siete espíritus ayudantes están mucho mejor sincronizados con las etapas progresivas del desarrollo animal de lo que lo estuvieron en Urantia. Para ponerse en contacto con la mente evolutiva de los organismos de Urantia, los ayudantes experimentaron las dificultades más grandes que han tenido nunca, con una sola excepción, en toda su actividad en todo el universo de Nebadon. En este mundo se desarrollaron muchas formas de fenómenos límites ---de combinaciones confusas de reacciones orgánicas de tipo maquinal no enseñable y de tipo no maquinal enseñable.

\par
%\textsuperscript{(739.1)}
\textsuperscript{65:7.5} Los siete espíritus ayudantes no se ponen en contacto con los tipos de organismos que reaccionan al entorno de manera puramente maquinal. Esas reacciones preinteligentes de los organismos vivientes pertenecen exclusivamente a los dominios energéticos de los centros de poder, de los controladores físicos y de sus asociados.

\par
%\textsuperscript{(739.2)}
\textsuperscript{65:7.6} La adquisición del potencial de la capacidad para \textit{aprender} por experiencia señala el comienzo del funcionamiento de los espíritus ayudantes, una actividad que ejercen desde las mentes más inferiores de los seres primitivos e invisibles, hasta los tipos superiores en la escala evolutiva de los seres humanos. Los ayudantes son la fuente y el modelo del comportamiento y de las rápidas reacciones que tiene la mente hacia el entorno material, un comportamiento por lo demás más o menos misterioso, y unas reacciones no comprendidas por completo. Estas influencias fieles y siempre seguras tienen que aportar largo tiempo su ministerio preliminar antes de que la mente animal alcance los niveles humanos de receptividad espiritual.

\par
%\textsuperscript{(739.3)}
\textsuperscript{65:7.7} Los ayudantes actúan exclusivamente en la evolución de la mente experiencial hasta el nivel de la sexta fase, el espíritu de adoración. En este nivel se produce una superposición inevitable de ministerios ---el fenómeno en el que lo superior desciende para coordinarse con lo inferior, esperando alcanzar posteriormente unos niveles avanzados de desarrollo. Un ministerio espiritual todavía adicional acompaña la actividad del séptimo y último ayudante, el espíritu de la sabiduría. A lo largo de todo el ministerio del mundo del espíritu, el individuo nunca experimenta transiciones bruscas en la cooperación espiritual; estos cambios son siempre graduales y recíprocos.

\par
%\textsuperscript{(739.4)}
\textsuperscript{65:7.8} Los ámbitos de las reacciones físicas (electroquímicas) y mentales a los estímulos del entorno deberían ser siempre diferenciados, y todos deben reconocerse a su vez como fenómenos separados de las actividades espirituales. Los ámbitos de la gravedad física, mental y espiritual son distintos reinos de la realidad cósmica, a pesar de sus correlaciones íntimas.

\section*{8. La evolución en el tiempo y el espacio}
\par
%\textsuperscript{(739.5)}
\textsuperscript{65:8.1} El tiempo y el espacio están indisolublemente enlazados; es una asociación innata. Los retrasos del tiempo son inevitables en presencia de ciertas condiciones del espacio.

\par
%\textsuperscript{(739.6)}
\textsuperscript{65:8.2} Si emplear tanto tiempo en efectuar los cambios evolutivos del desarrollo de la vida os produce perplejidad, os puedo decir que no podemos conseguir que los procesos de la vida se desarrollen más deprisa de lo que lo permiten las metamorfosis físicas de un planeta. Tenemos que esperar el desarrollo físico natural de un planeta; no tenemos absolutamente ningún control sobre la evolución geológica. Si las condiciones físicas lo permitieran, podríamos tomar medidas para que la evolución completa de la vida se efectuara en mucho menos de un millón de años. Pero todos estamos bajo la jurisdicción de los Gobernantes Supremos del Paraíso, y el tiempo no existe en el Paraíso.

\par
%\textsuperscript{(739.7)}
\textsuperscript{65:8.3} El patrón que utiliza una persona para medir el tiempo es la duración de su vida. Todas las criaturas están así condicionadas por el tiempo, y por eso consideran que la evolución es un proceso interminable. Para aquellos de nosotros cuya vida no está limitada por una existencia temporal, la evolución no parece ser una operación tan prolongada. En el Paraíso, donde el tiempo no existe, todas estas cosas están \textit{presentes} en la mente de la Infinidad y en los actos de la Eternidad.

\par
%\textsuperscript{(739.8)}
\textsuperscript{65:8.4} De la misma manera que la evolución de la mente depende del lento desarrollo de las condiciones físicas, el cual la retrasa, el progreso espiritual depende de la expansión mental, y el retraso intelectual lo demora infaliblemente. Pero esto no significa que la evolución espiritual dependa de la educación, la cultura o la sabiduría. El alma puede evolucionar independientemente de la cultura mental, pero no en ausencia de la capacidad mental y del deseo ---la elección de la supervivencia y la decisión de alcanzar una perfección siempre mayor--- de hacer la voluntad del Padre que está en los cielos. Aunque la supervivencia pueda no depender de la posesión del conocimiento y la sabiduría, el progreso depende de ellos con toda seguridad.

\par
%\textsuperscript{(740.1)}
\textsuperscript{65:8.5} En los laboratorios evolutivos cósmicos la mente siempre domina a la materia, y el espíritu siempre está en correlación con la mente. Si estos diversos dones no logran sincronizarse y coordinarse, se pueden producir retrasos en el tiempo; pero si el individuo conoce realmente a Dios y desea encontrarlo y parecerse a él, entonces su supervivencia está asegurada, a pesar de los obstáculos del tiempo. El estado físico puede obstaculizar a la mente, y la perversidad mental puede retrasar la consecución espiritual, pero ninguno de estos obstáculos puede vencer la elección que la voluntad ha hecho con toda su alma.

\par
%\textsuperscript{(740.2)}
\textsuperscript{65:8.6} Cuando las condiciones físicas están maduras, se pueden producir evoluciones mentales \textit{repentinas}; cuando el estado de la mente es propicio, pueden ocurrir transformaciones espirituales \textit{repentinas}; cuando los valores espirituales reciben el reconocimiento adecuado, entonces los significados cósmicos se vuelven discernibles, y la personalidad se libera cada vez más de los obstáculos del tiempo y de las limitaciones del espacio.

\par
%\textsuperscript{(740.3)}
\textsuperscript{65:8.7} [Patrocinado por un Portador de Vida de Nebadon residente en Urantia.]


\chapter{Documento 66. El Príncipe Planetario de Urantia}
\par
%\textsuperscript{(741.1)}
\textsuperscript{66:0.1} LA LLEGADA de un Hijo Lanonandek a un mundo normal significa que la voluntad, la capacidad para elegir el camino de la supervivencia eterna, se ha desarrollado en la mente del hombre primitivo. Pero el Príncipe Planetario llegó a Urantia casi medio millón de años después de la aparición de la voluntad humana.

\par
%\textsuperscript{(741.2)}
\textsuperscript{66:0.2} Caligastia, el Príncipe Planetario, llegó a Urantia hace unos quinientos mil años, coincidiendo con la aparición de las seis razas de color o razas sangiks. En el momento de llegar el Príncipe había en la Tierra cerca de quinientos millones de seres humanos primitivos, muy dispersos por Europa, Asia y África. La sede del Príncipe, que se estableció en Mesopotamia, estaba aproximadamente en el centro del mundo habitado.

\section*{1. El Príncipe Caligastia}
\par
%\textsuperscript{(741.3)}
\textsuperscript{66:1.1} Caligastia\footnote{\textit{El Príncipe Caligastia}: Jn 12:31; 14:30; 16:11; Ef 2:2; 6:12.} era un Hijo Lanonandek, el número 9.344 de la orden secundaria. Tenía experiencia en la administración de los asuntos del universo local en general, y durante las épocas más recientes, en la gestión del sistema local de Satania en particular.

\par
%\textsuperscript{(741.4)}
\textsuperscript{66:1.2} Antes del reinado de Lucifer en Satania, Caligastia había estado destinado en el consejo de asesores de los Portadores de Vida en Jerusem. Lucifer ascendió a Caligastia a un puesto en su estado mayor personal, y cumplió adecuadamente cinco misiones sucesivas de honor y de confianza.

\par
%\textsuperscript{(741.5)}
\textsuperscript{66:1.3} Caligastia intentó conseguir muy pronto un nombramiento como Príncipe Planetario pero, en diversas ocasiones, cada vez que su petición había sido sometida a la aprobación de los consejos de la constelación, no había logrado recibir el consentimiento de los Padres de la Constelación. Caligastia parecía particularmente deseoso de ser enviado como gobernante planetario a un mundo decimal o de modificación de la vida. Después de haberse denegado su demanda varias veces, fue asignado finalmente a Urantia.

\par
%\textsuperscript{(741.6)}
\textsuperscript{66:1.4} Caligastia salió de Jerusem, para hacerse cargo del gobierno de un mundo, con un historial envidiable de lealtad y de dedicación al bienestar de su universo de origen y de residencia, a pesar de cierta impaciencia característica unida a su tendencia a discrepar, en ciertos asuntos menores, con el orden establecido.

\par
%\textsuperscript{(741.7)}
\textsuperscript{66:1.5} Yo estaba presente en Jerusem cuando el brillante Caligastia partió de la capital del sistema. Ningún príncipe planetario había emprendido nunca una carrera de gobernante mundial con una experiencia preparatoria más rica ni con unas perspectivas mejores que las de Caligastia en aquel día memorable de hace medio millón de años. Una cosa es segura: Mientras efectuaba mi tarea de difundir la narración de aquel acontecimiento en las transmisiones del universo local, en ningún momento se me ocurrió la idea de que este noble Lanonandek traicionaría tan pronto su sagrado deber como custodio planetario, y mancharía de manera tan horrible el hermoso nombre de su elevada orden de filiación del universo. Yo consideraba realmente que Urantia era uno de los cinco o seis planetas más afortunados de toda Satania porque iba a tener, al timón de sus asuntos mundiales, a un cerebro tan original, brillante y experimentado. No comprendía entonces que Caligastia se estaba enamorando insidiosamente de sí mismo; no entendía entonces plenamente las sutilezas del orgullo de la personalidad.

\section*{2. El estado mayor del Príncipe}
\par
%\textsuperscript{(742.1)}
\textsuperscript{66:2.1} El Príncipe Planetario de Urantia no fue enviado solo a su misión, sino que le acompañó el cuerpo habitual de asistentes y de auxiliares en administración.

\par
%\textsuperscript{(742.2)}
\textsuperscript{66:2.2} A la cabeza de este grupo se encontraba Daligastia, el asistente asociado del Príncipe Planetario. Daligastia era también un Hijo Lanonandek secundario, el número 319.407 de esta orden. Tenía rango de asistente en el momento de ser asignado como asociado de Caligastia.

\par
%\textsuperscript{(742.3)}
\textsuperscript{66:2.3} El estado mayor planetario incluía una gran cantidad de cooperadores angélicos y una multitud de otros seres celestiales encargados de hacer progresar los intereses y de promover el bienestar de las razas humanas. Pero desde vuestro punto de vista, el grupo más interesante de todos era el de los miembros corpóreos del estado mayor del Príncipe ---que a veces se mencionan como \textit{loscien de Caligastia}.

\par
%\textsuperscript{(742.4)}
\textsuperscript{66:2.4} Caligastia escogió a estos cien miembros rematerializados del estado mayor del Príncipe entre más de 785.000 ciudadanos ascendentes de Jerusem que se ofrecieron voluntarios para embarcarse en la aventura de Urantia. Cada uno de los cien elegidos provenía de un planeta diferente, y ninguno de ellos era de Urantia.

\par
%\textsuperscript{(742.5)}
\textsuperscript{66:2.5} Estos voluntarios jerusemitas fueron traídos por transporte seráfico directamente desde la capital del sistema hasta Urantia. Después de su llegada, permanecieron enserafinados hasta que se les pudo proporcionar unas formas personales con la doble naturaleza del servicio planetario especial, unos verdaderos cuerpos de carne y hueso que también estaban adaptados a los circuitos vitales del sistema.

\par
%\textsuperscript{(742.6)}
\textsuperscript{66:2.6} Algún tiempo antes de la llegada de estos cien ciudadanos de Jerusem, los dos Portadores de Vida supervisores que residían en Urantia y que habían perfeccionado previamente sus planes, pidieron permiso a Jerusem y Edentia para trasplantar el plasma vital de cien supervivientes seleccionados del linaje de Andón y Fonta en los cuerpos materiales que estaban en proyecto para los miembros corpóreos del estado mayor del Príncipe. La petición fue concedida en Jerusem y aprobada en Edentia.

\par
%\textsuperscript{(742.7)}
\textsuperscript{66:2.7} En consecuencia, los Portadores de Vida escogieron a cincuenta hombres y cincuenta mujeres entre los descendientes de Andón y Fonta, que representaban la supervivencia de los mejores linajes de esta raza única. A excepción de uno o dos, estos andonitas que contribuyeron al progreso de la raza no se conocían entre sí. Procedían de lugares muy alejados y fueron reunidos en el umbral de la sede del Príncipe gracias a la dirección de los Ajustadores del Pensamiento en coordinación con la guía seráfica. Aquí, los cien sujetos humanos fueron puestos en manos de la comisión sumamente experta de voluntarios procedentes de Avalon, que dirigió la extracción material de una porción del plasma vital de estos descendientes de Andón. Este material viviente se transfirió después a los cuerpos materiales que se construyeron para los cien miembros jerusemitas del estado mayor del Príncipe. Mientras tanto, estos ciudadanos recién llegados de la capital del sistema permanecieron en el sueño del transporte seráfico.

\par
%\textsuperscript{(742.8)}
\textsuperscript{66:2.8} Estas operaciones, así como la creación literal de unos cuerpos especiales para los cien de Caligastia, dieron origen a numerosas leyendas, muchas de las cuales se confundieron posteriormente con las tradiciones más tardías acerca de la instalación planetaria de Adán y Eva.

\par
%\textsuperscript{(743.1)}
\textsuperscript{66:2.9} Toda la operación de la repersonalización, desde el momento de la llegada de los transportes seráficos que traían a los cien voluntarios de Jerusem, hasta que recuperaron la conciencia como seres triples del reino, duró exactamente diez días.

\section*{3. Dalamatia ---la ciudad del Príncipe}
\par
%\textsuperscript{(743.2)}
\textsuperscript{66:3.1} La sede del Príncipe Planetario estaba situada en la región del Golfo Pérsico de aquellos tiempos, en la zona correspondiente a la Mesopotamia posterior.

\par
%\textsuperscript{(743.3)}
\textsuperscript{66:3.2} El clima y el paisaje de la Mesopotamia de aquellos tiempos eran favorables, en todos los aspectos, para las empresas del estado mayor del Príncipe y sus asistentes, y muy diferentes de las condiciones que a veces han prevalecido desde entonces. Era necesario disponer de un clima tan favorable como parte del entorno natural destinado a incitar a los urantianos primitivos a que realizaran algunos progresos iniciales en la cultura y la civilización. La primera gran tarea de aquellas épocas consistía en transformar a aquellos cazadores en pastores, con la esperanza de que más tarde se convertirían en agricultores pacíficos y hogareños.

\par
%\textsuperscript{(743.4)}
\textsuperscript{66:3.3} La sede del Príncipe Planetario en Urantia era un ejemplo típico de este tipo de estaciones en una joven esfera en vías de desarrollo. El núcleo de la colonia del Príncipe era una ciudad muy sencilla pero muy hermosa, rodeada por una muralla de doce metros de altura. A este centro mundial de cultura se le llamó Dalamatia en honor a Daligastia.

\par
%\textsuperscript{(743.5)}
\textsuperscript{66:3.4} La ciudad estaba construida en diez subdivisiones, con los edificios de las sedes centrales de los diez consejos del estado mayor corpóreo situados en el centro de estas subdivisiones. En medio de la ciudad se encontraba el templo del Padre invisible. La sede administrativa del Príncipe y de sus asociados estaba repartida en doce salas agrupadas directamente alrededor del templo mismo.

\par
%\textsuperscript{(743.6)}
\textsuperscript{66:3.5} Todos los edificios de Dalamatia tenían un solo piso, a excepción de las sedes de los consejos, que tenían dos pisos, y el templo central del Padre de todos, que era pequeño pero tenía tres pisos.

\par
%\textsuperscript{(743.7)}
\textsuperscript{66:3.6} La ciudad se había construido con el mejor material de construcción de aquellos tiempos primitivos ---el ladrillo. Se empleó muy poca piedra o madera. El ejemplo de Dalamatia mejoró considerablemente la construcción de las viviendas y la arquitectura de las aldeas de los habitantes de los alrededores.

\par
%\textsuperscript{(743.8)}
\textsuperscript{66:3.7} Cerca de la sede del Príncipe vivían seres humanos de todos los colores y estratos sociales. Los primeros estudiantes de las escuelas del Príncipe se reclutaron entre estas tribus vecinas. Aunque estas primeras escuelas de Dalamatia eran rudimentarias, proporcionaban todo lo que se podía hacer por los hombres y las mujeres de aquella época primitiva.

\par
%\textsuperscript{(743.9)}
\textsuperscript{66:3.8} El estado mayor corpóreo del Príncipe reunía continuamente a su alrededor a los individuos superiores de las tribus circundantes, y después de haber preparado e inspirado a estos estudiantes, los enviaban de vuelta como instructores y dirigentes de sus pueblos respectivos.

\section*{4. Los primeros días de los cien}
\par
%\textsuperscript{(743.10)}
\textsuperscript{66:4.1} La llegada del estado mayor del Príncipe produjo una profunda impresión. Aunque se necesitaron casi mil años para que las noticias se difundieran por todas partes, las enseñanzas y la conducta de los cien nuevos habitantes de Urantia influyeron enormemente en estas tribus próximas a la sede mesopotámica. Una gran parte de vuestra mitología posterior tuvo su origen en las leyendas confusas sobre aquellos primeros días en que estos miembros del estado mayor del Príncipe fueron repersonalizados como superhombres en Urantia.

\par
%\textsuperscript{(744.1)}
\textsuperscript{66:4.2} La tendencia de los mortales a considerar a estos maestros extraplanetarios como si fueran dioses obstaculiza gravemente su buena influencia; pero aparte de la técnica de su aparición en la Tierra, los cien de Caligastia ---cincuenta hombres y cincuenta mujeres--- no recurrieron ni a métodos sobrenaturales ni a manipulaciones sobrehumanas.

\par
%\textsuperscript{(744.2)}
\textsuperscript{66:4.3} Pero sin embargo, el estado mayor corpóreo era superhumano. Empezaron su misión en Urantia como unos seres extraordinarios de naturaleza triple:

\par
%\textsuperscript{(744.3)}
\textsuperscript{66:4.4} 1. Eran materiales y relativamente humanos, pues tenían incorporado el verdadero plasma vital de una de las razas humanas, el plasma vital andónico de Urantia.

\par
%\textsuperscript{(744.4)}
\textsuperscript{66:4.5} Estos cien miembros del estado mayor del Príncipe estaban divididos por igual en cuanto al sexo, y con arreglo a su estado mortal anterior. Cada persona de este grupo era capaz de convertirse en el co-progenitor de algún nuevo tipo de seres físicos, pero se les había ordenado cuidadosamente que no recurrieran a la procreación salvo en ciertas condiciones. El estado mayor corpóreo de un Príncipe Planetario tiene la costumbre de procrear a sus sucesores algún tiempo antes de retirarse del servicio planetario especial. Esto sucede habitualmente en el momento de la llegada del Adán y la Eva Planetarios, o poco tiempo después.

\par
%\textsuperscript{(744.5)}
\textsuperscript{66:4.6} Por consiguiente, estos seres especiales tenían poca o ninguna idea del tipo de criatura material que podría nacer de su unión sexual. Y nunca lo supieron, porque antes de llegar a esta etapa de su obra mundial, la rebelión había trastornado todo el régimen, y aquellos que desempeñaron más tarde el papel de progenitores habían sido aislados de las corrientes vitales del sistema.

\par
%\textsuperscript{(744.6)}
\textsuperscript{66:4.7} Estos miembros materializados del estado mayor de Caligastia tenían el color de la piel y el idioma de la raza andónica. Se alimentaban como los mortales del reino, con la diferencia de que los cuerpos recreados de este grupo se satisfacían plenamente con una dieta sin carne. Ésta fue una de las razones que condujeron a que residieran en una región cálida donde abundaban las frutas y las nueces. La práctica de alimentarse mediante un régimen no carnívoro data de los tiempos de los cien de Caligastia, pues esta costumbre se extendió por todas partes y afectó los hábitos alimenticios de muchas tribus circundantes, unos grupos que descendían de las razas evolutivas que en otro tiempo habían sido exclusivamente carnívoras.

\par
%\textsuperscript{(744.7)}
\textsuperscript{66:4.8} 2. Los cien eran seres materiales pero superhumanos, y habían sido reconstituidos en Urantia como hombres y mujeres únicos de un orden especial y elevado.

\par
%\textsuperscript{(744.8)}
\textsuperscript{66:4.9} Aunque este grupo disfrutaba de la ciudadanía provisional de Jerusem, sus miembros aún no habían fusionado con sus Ajustadores del Pensamiento; cuando se ofrecieron como voluntarios y fueron aceptados para el servicio planetario en unión con las órdenes descendentes de filiación, sus Ajustadores se separaron de ellos. Pero estos jerusemitas eran seres superhumanos ---tenían un alma de crecimiento ascendente. Durante la vida como mortal en la carne, el alma está en estado embrionario; nace (resucita) en la vida morontial y experimenta su crecimiento a través de los mundos morontiales sucesivos. Y las almas de los cien de Caligastia se habían desarrollado de esta manera mediante las experiencias progresivas de los siete mundos de las mansiones, hasta alcanzar el estado de ciudadanos de Jerusem.

\par
%\textsuperscript{(744.9)}
\textsuperscript{66:4.10} Siguiendo las instrucciones que habían recibido, el estado mayor no procedió a la reproducción sexual, pero estudiaron con esmero su constitución personal y exploraron cuidadosamente todas las fases imaginables de unión intelectual (de la mente) y morontial (del alma). Durante el trigésimo tercer año de su estancia en Dalamatia, mucho antes de que se terminara la muralla, el número dos y el número siete del grupo danita descubrieron por casualidad un fenómeno que acompañaba la unión (supuestamente no sexual y no material) de sus yoes morontiales, y la consecuencia de esta aventura resultó ser la primera de las criaturas intermedias primarias. Este nuevo ser era totalmente visible para el estado mayor planetario y sus asociados celestiales, pero era invisible para los hombres y las mujeres de las diversas tribus humanas. Con la autorización del Príncipe Planetario, todo el estado mayor corpóreo emprendió la procreación de seres similares, y todos lo lograron siguiendo las instrucciones de la pareja pionera danita. Así es como el estado mayor del Príncipe trajo finalmente a la existencia al cuerpo original de 50.000 intermedios primarios.

\par
%\textsuperscript{(745.1)}
\textsuperscript{66:4.11} Estas criaturas de tipo intermedio prestaban un gran servicio llevando adelante los asuntos de la sede mundial. Eran invisibles para los seres humanos, pero a los residentes primitivos de Dalamatia se les enseñó la existencia de estos semiespíritus invisibles, y durante siglos constituyeron la totalidad del mundo espiritual para estos mortales en evolución.

\par
%\textsuperscript{(745.2)}
\textsuperscript{66:4.12} 3. Los cien de Caligastia eran personalmente inmortales, o imperecederos. Los complementos alexifármacos de las corrientes de vida del sistema circulaban por sus formas materiales, y si no hubieran perdido el contacto con los circuitos de vida a causa de la rebelión, habrían continuado viviendo indefinidamente hasta la llegada posterior de un Hijo de Dios, o hasta que hubieran sido liberados más tarde para reanudar el viaje interrumpido hacia Havona y el Paraíso.

\par
%\textsuperscript{(745.3)}
\textsuperscript{66:4.13} Los complementos alexifármacos de las corrientes de vida de Satania procedían del fruto del árbol de la vida, un arbusto de Edentia que los Altísimos de Norlatiadek habían enviado a Urantia en el momento de la llegada de Caligastia. En la época de Dalamatia, este árbol crecía en el patio central del templo del Padre invisible, y el fruto del árbol de la vida es el que permitía que los seres materiales, por otra parte mortales, del estado mayor del Príncipe continuaran viviendo indefinidamente mientras tuvieran acceso a él.

\par
%\textsuperscript{(745.4)}
\textsuperscript{66:4.14} Aunque no tenía ningún valor para las razas evolutivas, este superalimento era más que suficiente para conferir una vida continua a los cien de Caligastia y también a los cien andonitas modificados que estaban asociados con ellos.

\par
%\textsuperscript{(745.5)}
\textsuperscript{66:4.15} Conviene explicar a este respecto que cuando los cien andonitas aportaron su plasma germinativo humano a los miembros del estado mayor del Príncipe, los Portadores de Vida introdujeron en sus cuerpos mortales el complemento de los circuitos del sistema, y esto les permitió continuar viviendo simultáneamente con el estado mayor, siglo tras siglo, desafiando a la muerte física.

\par
%\textsuperscript{(745.6)}
\textsuperscript{66:4.16} A los cien andonitas se les informó finalmente acerca de su contribución a las nuevas formas de sus superiores, y estos mismos cien hijos de las tribus de Andón permanecieron en la sede como asistentes personales del estado mayor corpóreo del Príncipe.

\section*{5. La organización de los cien}
\par
%\textsuperscript{(745.7)}
\textsuperscript{66:5.1} Los cien estaban organizados para el servicio en diez consejos autónomos de diez miembros cada uno. Cuando dos o más consejos de estos diez se reunían en sesión conjunta, estas asambleas de enlace eran presididas por Daligastia. Estos diez grupos estaban constituidos como sigue:

\par
%\textsuperscript{(745.8)}
\textsuperscript{66:5.2} 1. \textit{El consejo de la alimentación y el bienestar material}. Ang presidía este grupo. Este cuerpo capaz fomentaba las cuestiones relacionadas con la alimentación, el agua, la ropa y el progreso material de la especie humana. Enseñaron la excavación de los pozos, el control de los manantiales y el riego. A los que venían de las altitudes más elevadas y de las zonas nórdicas les enseñaron mejores métodos para tratar las pieles destinadas a servir de vestidos, y los profesores de las artes y las ciencias introdujeron más tarde la tejeduría.

\par
%\textsuperscript{(746.1)}
\textsuperscript{66:5.3} Se realizaron grandes progresos en los métodos para almacenar los alimentos. La comida se conservó mediante la cocción, la desecación y el ahumado, convirtiéndose así en la primera forma de propiedad. Al hombre se le enseñó a prever los peligros de la escasez que diezmaba periódicamente al mundo.

\par
%\textsuperscript{(746.2)}
\textsuperscript{66:5.4} 2. \textit{El consejo de la domesticación y utilización de los animales}. Este consejo estaba dedicado a la tarea de seleccionar y criar a aquellos animales que estaban mejor adaptados para ayudar a los seres humanos a llevar las cargas y trasportarlos a ellos mismos, para servir de alimento, y más adelante para utilizarlos en el cultivo de la tierra. Este cuerpo competente estaba dirigido por Bon.

\par
%\textsuperscript{(746.3)}
\textsuperscript{66:5.5} Se domesticaron diversos tipos de animales útiles ya extintos, así como otros que han continuado siendo animales domésticos hasta nuestros días. El hombre llevaba mucho tiempo viviendo en compañía del perro, y el hombre azul ya había logrado domar al elefante. La vaca había mejorado tanto gracias a una cría esmerada que se convirtió en una valiosa fuente de alimentación; la mantequilla y el queso se volvieron artículos corrientes en el régimen alimenticio humano. Los hombres aprendieron a emplear los bueyes para llevar las cargas, pero el caballo no fue domesticado hasta una fecha posterior. Los miembros de este cuerpo fueron los primeros que enseñaron a los hombres a utilizar la rueda para facilitar la tracción.

\par
%\textsuperscript{(746.4)}
\textsuperscript{66:5.6} Fue en esta época cuando se utilizaron por primera vez las palomas mensajeras; se llevaban en los viajes largos para enviar mensajes o pedir ayuda. El grupo de Bon consiguió amaestrar a los grandes fándores como aves de pasajeros, pero éstos se extinguieron hace más de treinta mil años.

\par
%\textsuperscript{(746.5)}
\textsuperscript{66:5.7} 3. \textit{Los consejeros encargados de vencer a los animales de rapiña}. No era suficiente que el hombre primitivo intentara domesticar a ciertos animales, sino que también tenía que aprender a protegerse de la destrucción que podía causar el resto del mundo animal hostil. Este grupo estaba capitaneado por Dan.

\par
%\textsuperscript{(746.6)}
\textsuperscript{66:5.8} Las murallas de las ciudades antiguas tenían la finalidad de proteger contra las bestias feroces así como impedir los ataques por sorpresa de los humanos hostiles. Los que vivían fuera de las murallas y en el bosque dependían de los refugios en los árboles, de las cabañas de piedra y de las fogatas que alimentaban durante toda la noche. Por eso era muy natural que estos educadores consagraran mucho tiempo instruyendo a sus alumnos sobre cómo mejorar las viviendas humanas. Se realizaron grandes progresos en el sometimiento de los animales gracias al empleo de mejores técnicas y a la utilización de las trampas.

\par
%\textsuperscript{(746.7)}
\textsuperscript{66:5.9} 4. \textit{El cuerpo docente encargado de difundir y conservar el conocimiento}. Este grupo organizó y dirigió los esfuerzos puramente educativos de aquellos tiempos primitivos. Estaba presidido por Fad. Los métodos educativos de Fad consistían en supervisar el trabajo al mismo tiempo que enseñaba mejores métodos para realizarlo. Fad formuló el primer alfabeto e introdujo un sistema de escritura. Este alfabeto contenía veinticinco caracteres. Estos pueblos primitivos utilizaban como material para escribir la corteza de los árboles, las tablillas de arcilla, las losas de piedra, un tipo de pergamino hecho de pieles machacadas y una especie de papel sin refinar que hacían con los nidos de las avispas. La biblioteca de Dalamatia, destruida poco después de la deslealtad de Caligastia, contenía más de dos millones de documentos distintos y era conocida como <<la casa de Fad>>.

\par
%\textsuperscript{(746.8)}
\textsuperscript{66:5.10} El hombre azul tenía predilección por la escritura alfabética e hizo los mayores progresos en esta dirección. El hombre rojo prefería la escritura pictórica, mientras que las razas amarillas tendieron a utilizar símbolos para las palabras y las ideas, muy semejantes a los que emplean en la actualidad. Pero el alfabeto y otras muchas cosas se perdieron posteriormente para el mundo durante la confusión que acompañó a la rebelión. La deserción de Caligastia destruyó la esperanza mundial de tener un idioma universal, al menos durante incalculables milenios.

\par
%\textsuperscript{(747.1)}
\textsuperscript{66:5.11} 5. \textit{La comisión de la industria y el comercio}. Este consejo estaba encargado de fomentar la industria dentro de las tribus y de promover el intercambio comercial entre los diversos grupos pacíficos. Su director era Nod. Este cuerpo estimuló todas las formas de manufactura primitiva. Contribuyeron directamente a elevar el nivel de vida proporcionando muchos productos nuevos para atraer la curiosidad de los hombres primitivos. Extendieron enormemente el comercio de una sal mejorada producida por el consejo de las ciencias y las artes.

\par
%\textsuperscript{(747.2)}
\textsuperscript{66:5.12} El crédito comercial se practicó por primera vez entre estos grupos instruidos, educados en las escuelas de Dalamatia. Adquirían unas fichas en una bolsa central de crédito que eran aceptadas en lugar de los objetos reales de trueque. El mundo no mejoró estos métodos comerciales hasta cientos de miles de años después.

\par
%\textsuperscript{(747.3)}
\textsuperscript{66:5.13} 6. \textit{La escuela de la religión revelada}. Este cuerpo funcionó con lentitud. La civilización de Urantia se forjó literalmente entre el yunque de la necesidad y los martillos del miedo. Sin embargo, este grupo había hecho unos progresos considerables en sus esfuerzos por sustituir el temor a las criaturas (el culto de los fantasmas) por el temor al Creador, antes de que sus trabajos se vieran interrumpidos por la confusión posterior que acompañó al levantamiento separatista. El presidente de este consejo era Hap.

\par
%\textsuperscript{(747.4)}
\textsuperscript{66:5.14} Ningún miembro del estado mayor del Príncipe quiso ofrecer unas revelaciones que complicaran la evolución; sólo expusieron sus revelaciones como punto culminante cuando ya habían agotado las fuerzas de la evolución. Pero Hap cedió al deseo de los habitantes de la ciudad de que se estableciera una forma de servicio religioso. Su grupo proporcionó a los dalamatianos los siete cánticos del culto y también les dio la frase de alabanza diaria; luego les enseñó finalmente <<la oración del Padre>>, que decía:

\par
%\textsuperscript{(747.5)}
\textsuperscript{66:5.15} <<Padre de todos, cuyo Hijo honramos, míranos con favor. Líbranos del temor a todo, salvo a ti mismo. Haz que seamos una satisfacción para nuestros divinos maestros y pon siempre la verdad en nuestros labios. Líbranos de la violencia y de la ira; danos respeto por nuestros ancianos y por lo que pertenece a nuestro prójimo. Danos en esta época verdes pastos y rebaños abundantes para alegrarnos el corazón. Rogamos para que llegue pronto el mejorador prometido, y queremos hacer tu voluntad en este mundo al igual que otros la hacen en los mundos lejanos.>>

\par
%\textsuperscript{(747.6)}
\textsuperscript{66:5.16} Aunque el estado mayor del Príncipe permaneció limitado a los medios naturales y a los métodos corrientes para mejorar las razas, les ofreció la promesa del don adámico de una nueva raza como meta del crecimiento evolutivo posterior cuando se alcanzara la cúspide del desarrollo biológico.

\par
%\textsuperscript{(747.7)}
\textsuperscript{66:5.17} 7. \textit{Los guardianes de la salud y la vida}. Este consejo estaba encargado de introducir la sanidad y de promover una higiene primitiva; estaba dirigido por Lut.

\par
%\textsuperscript{(747.8)}
\textsuperscript{66:5.18} Sus miembros enseñaron muchas cosas que se perdieron durante la confusión de las épocas posteriores, y que nunca volvieron a descubrirse hasta el siglo veinte. Enseñaron a la humanidad que cocer, hervir y asar los alimentos eran medios de evitar las enfermedades; y también enseñaron que cocinar reducía enormemente la mortalidad infantil y facilitaba un pronto destete.

\par
%\textsuperscript{(747.9)}
\textsuperscript{66:5.19} Una gran parte de las primeras enseñanzas de los guardianes de la salud del grupo de Lut sobrevivieron entre las tribus de la Tierra hasta la época de Moisés, aunque de manera muy confusa y enormemente modificadas.

\par
%\textsuperscript{(748.1)}
\textsuperscript{66:5.20} El obstáculo principal para la promoción de la higiene entre estos pueblos ignorantes consistía en el hecho de que las verdaderas causas de muchas enfermedades eran demasiado pequeñas para poder verlas a simple vista, y también porque todos tenían un respeto supersticioso por el fuego. Se necesitaron miles de años para persuadirlos de que quemaran la basura. Mientras tanto se les insistió para que enterraran los desperdicios en descomposición. El gran progreso sanitario de esta época provino de la difusión del conocimiento relacionado con las propiedades saludables y curativas de la luz solar.

\par
%\textsuperscript{(748.2)}
\textsuperscript{66:5.21} Antes de la llegada del Príncipe, los baños habían sido un ceremonial exclusivamente religioso. Fue en verdad muy difícil persuadir a los hombres primitivos para que se lavaran el cuerpo como práctica de salud. Lut convenció finalmente a los educadores religiosos para que incluyeran las abluciones en las ceremonias de purificación que se practicaban una vez por semana durante las devociones del mediodía destinadas a la adoración del Padre de todos.

\par
%\textsuperscript{(748.3)}
\textsuperscript{66:5.22} Estos guardianes de la salud intentaron también introducir el apretón de manos para sustituir el intercambio de saliva o el beber la sangre como sello de amistad personal y símbolo de lealtad al grupo. Pero cuando se encontraron libres de la presión apremiante de las enseñanzas de sus jefes superiores, estos pueblos primitivos no tardaron en retroceder a sus antiguas prácticas ignorantes y supersticiosas que destruían la salud y multiplicaban las enfermedades.

\par
%\textsuperscript{(748.4)}
\textsuperscript{66:5.23} 8. \textit{El consejo planetario de las artes y las ciencias}. Este cuerpo contribuyó mucho a mejorar las técnicas industriales del hombre primitivo y a elevar sus conceptos de la belleza. Su director se llamaba Mek.

\par
%\textsuperscript{(748.5)}
\textsuperscript{66:5.24} Las artes y las ciencias se encontraban en un nivel muy bajo en todo el mundo, pero a los dalamatianos se les enseñó los rudimentos de la física y la química. La alfarería avanzó, todas las artes decorativas mejoraron, y los ideales de la belleza humana aumentaron considerablemente. Pero la música progresó muy poco hasta después de la llegada de la raza violeta.

\par
%\textsuperscript{(748.6)}
\textsuperscript{66:5.25} A pesar de las reiteradas exhortaciones de sus educadores, estos hombres primitivos no consintieron en experimentar con la energía del vapor; nunca pudieron superar su enorme temor al poder explosivo del vapor confinado. Sin embargo, al final se dejaron persuadir para trabajar con los metales y el fuego, aunque un pedazo de metal al rojo era un objeto aterrador para el hombre primitivo.

\par
%\textsuperscript{(748.7)}
\textsuperscript{66:5.26} Mek contribuyó mucho a elevar la cultura de los andonitas y a mejorar las artes del hombre azul. Una mezcla de los hombres azules con el linaje de Andón produjo unos tipos de hombres dotados de talentos artísticos, y muchos de ellos se convirtieron en unos escultores maestros. No trabajaban ni la piedra ni el mármol, pero sus obras de arcilla, endurecidas por cocción, adornaban los jardines de Dalamatia.

\par
%\textsuperscript{(748.8)}
\textsuperscript{66:5.27} Las artes domésticas hicieron grandes progresos, pero la mayor parte se perdió durante las largas épocas sombrías de la rebelión, y nunca se volvieron a descubrir hasta los tiempos modernos.

\par
%\textsuperscript{(748.9)}
\textsuperscript{66:5.28} 9. \textit{Los gobernadores de las relaciones tribales avanzadas}. Éste era el grupo encargado de la tarea de elevar la sociedad humana hasta el nivel de Estado. Su jefe era Tut.

\par
%\textsuperscript{(748.10)}
\textsuperscript{66:5.29} Estos dirigentes contribuyeron mucho a que se produjeran casamientos entre las diferentes tribus. Fomentaron el cortejo y el matrimonio después de haberlo pensado bien y de haber tenido amplias ocasiones para conocerse. Las danzas puramente guerreras fueron refinadas y puestas al servicio de valiosos fines sociales. Se introdujeron muchos juegos competitivos, pero estos pueblos antiguos eran serios; el humor no era una característica que adornara a estas tribus primitivas. Muy pocas de estas costumbres sobrevivieron a la desintegración posterior causada por la insurrección planetaria.

\par
%\textsuperscript{(749.1)}
\textsuperscript{66:5.30} Tut y sus compañeros se esforzaron por promover las asociaciones colectivas de naturaleza pacífica, por reglamentar y humanizar la guerra, por coordinar las relaciones intertribales y por mejorar los gobiernos tribales. En las cercanías de Dalamatia se desarrolló una cultura más avanzada, y estas relaciones sociales mejores tuvieron una influencia muy beneficiosa sobre las tribus más lejanas. Pero el modelo de civilización que prevalecía en la sede del Príncipe era muy diferente al de la sociedad bárbara que evolucionaba en otras partes, al igual que la sociedad del siglo veinte de la Ciudad del Cabo, en Sudáfrica, es totalmente distinta a la cultura rudimentaria de los pequeños bosquimanos del norte.

\par
%\textsuperscript{(749.2)}
\textsuperscript{66:5.31} 10. \textit{El tribunal supremo de coordinación tribal y de cooperación racial}. Este consejo supremo estaba dirigido por Van y servía como tribunal de apelación para las otras nueve comisiones especiales encargadas de supervisar los asuntos humanos. Este consejo tenía funciones muy amplias, pues se le habían confiado todos los asuntos terrestres que no dependían específicamente de los otros grupos. Este cuerpo selecto había sido aprobado por los Padres de la Constelación de Edentia antes de ser autorizado a asumir las funciones de tribunal supremo de Urantia.

\section*{6. El reinado del Príncipe}
\par
%\textsuperscript{(749.3)}
\textsuperscript{66:6.1} El grado de cultura de un mundo se mide por la herencia social de sus nativos, y la velocidad de la expansión cultural está totalmente determinada por la capacidad de sus habitantes para comprender las ideas nuevas y avanzadas.

\par
%\textsuperscript{(749.4)}
\textsuperscript{66:6.2} La esclavitud a la tradición produce la estabilidad y la cooperación enlazando sentimentalmente el pasado con el presente, pero al mismo tiempo ahoga la iniciativa y esclaviza los poderes creativos de la personalidad. El mundo entero estaba atrapado en el estancamiento de las costumbres atadas a la tradición cuando llegaron los cien de Caligastia y empezaron a proclamar el nuevo evangelio de la iniciativa individual dentro de los grupos sociales de aquellos tiempos. Pero este reinado benéfico se interrumpió tan pronto, que las razas nunca se han liberado por completo de la esclavitud a las costumbres; las maneras establecidas continúan dominando indebidamente en Urantia.

\par
%\textsuperscript{(749.5)}
\textsuperscript{66:6.3} Los cien de Caligastia ---diplomados de los mundos de las mansiones de Satania--- conocían muy bien las artes y la cultura de Jerusem, pero estos conocimientos casi no tienen valor en un planeta bárbaro poblado por unos humanos primitivos. Estos seres sabios sabían que no debían emprender la transformación \textit{repentina}, o la elevación en masa, de las razas primitivas de aquella época. Comprendían muy bien la lenta evolución de la especie humana, y se abstuvieron prudentemente de cualquier intento radical por modificar la manera de vivir de los hombres en la Tierra.

\par
%\textsuperscript{(749.6)}
\textsuperscript{66:6.4} Cada una de las diez comisiones planetarias se dedicó a hacer avanzar, de manera \textit{lenta} y natural, los intereses que se les habían confiado. Su plan consistió en atraer a las mejores inteligencias de las tribus circundantes, y después de haberlos enseñado, enviarlos de vuelta a sus pueblos respectivos como emisarios del progreso social.

\par
%\textsuperscript{(749.7)}
\textsuperscript{66:6.5} Nunca se enviaron emisarios extranjeros a una raza, a menos que el pueblo en cuestión lo solicitara expresamente. Aquellos que trabajaron por la elevación y el progreso de una tribu o de una raza determinada siempre fueron nativos de esa tribu o de esa raza. Los cien no trataron de imponer a una tribu los hábitos y las costumbres de otra raza, aunque fuera superior. Siempre trabajaron pacientemente para elevar y hacer avanzar las costumbres probadas por el tiempo de cada raza. Los pueblos sencillos de Urantia trajeron sus costumbres sociales a Dalamatia, no para cambiarlas por unas prácticas nuevas y mejores, sino para mejorarlas mediante el contacto con una cultura más elevada y en asociación con unas inteligencias superiores. El proceso fue lento pero muy eficaz.

\par
%\textsuperscript{(750.1)}
\textsuperscript{66:6.6} Los instructores de Dalamatia trataron de añadir una selección social consciente a la selección puramente natural de la evolución biológica. No trastornaron la sociedad humana, pero sí aceleraron notablemente su evolución normal y natural. Su móvil era la progresión a través de la evolución, y no la revolución por medio de la revelación. La raza humana había necesitado miles de años para adquirir el poco de religión y de moralidad que poseía, y estos superhombres se guardaron de robarle a la humanidad estos pequeños progresos, sumiéndola en la confusión y la consternación que siempre se producen cuando unos seres superiores e instruídos emprenden la elevación de las razas atrasadas, enseñándolas e iluminándolas con exceso.

\par
%\textsuperscript{(750.2)}
\textsuperscript{66:6.7} Cuando los misioneros cristianos van hasta el corazón de
África, donde se supone que los hijos y las hijas deben permanecer bajo el control y la dirección de sus padres mientras éstos vivan, sólo provocan la confusión y la ruptura de toda autoridad cuando intentan reemplazar esta práctica, en una sola generación, enseñando que los hijos deben liberarse de toda sujeción paternal después de cumplir los veintiún años.

\section*{7. La vida en Dalamatia}
\par
%\textsuperscript{(750.3)}
\textsuperscript{66:7.1} La sede del Príncipe, aunque era exquisitamente hermosa y estaba concebida para atemorizar a los hombres primitivos de aquella época, era en conjunto modesta. Los edificios no eran particularmente grandes, ya que estos instructores importados tenían la intención de estimular con el tiempo el desarrollo de la agricultura mediante la introducción de la ganadería. Las reservas de tierra dentro de las murallas de la ciudad eran suficientes para que los pastos y la horticultura pudieran mantener a una población de casi veinte mil habitantes.

\par
%\textsuperscript{(750.4)}
\textsuperscript{66:7.2} Los interiores del templo central de adoración y de las diez mansiones de los consejos de los grupos supervisores de superhombres eran en verdad hermosas obras de arte. Los edificios residenciales eran modelos de pulcritud y de limpieza, pero todo era muy sencillo y totalmente primitivo en comparación con los desarrollos posteriores. En esta sede de la cultura no se empleó ningún método que no perteneciera de manera natural a Urantia.

\par
%\textsuperscript{(750.5)}
\textsuperscript{66:7.3} El estado mayor corpóreo del Príncipe residía en viviendas sencillas y ejemplares, que cuidaban como hogares destinados a inspirar e impresionar favorablemente a los estudiantes observadores que residían temporalmente en el centro social y sede educativa del mundo.

\par
%\textsuperscript{(750.6)}
\textsuperscript{66:7.4} El orden definido de la vida familiar y la costumbre de vivir una sola familia en una sola vivienda en un lugar relativamente estable, data de estos tiempos de Dalamatia y se debe principalmente al ejemplo y las enseñanzas de los cien y sus alumnos. El hogar como unidad social nunca tuvo éxito hasta que los superhombres y las supermujeres de Dalamatia enseñaron a la humanidad a amar a sus nietos y a los hijos de sus nietos, y a hacer planes para ellos. El hombre salvaje ama a sus hijos, pero el hombre civilizado ama también a sus nietos\footnote{\textit{Amor por los nietos}: Pr 17:6.}.

\par
%\textsuperscript{(750.7)}
\textsuperscript{66:7.5} Los miembros del estado mayor del Príncipe vivían en parejas como padres y madres. Es cierto que no tenían hijos propios, pero los cincuenta hogares modelo de Dalamatia nunca albergaron menos de quinientos niños adoptados, escogidos entre las familias superiores de las razas andónicas y sangiks; muchos de estos niños eran huérfanos. Se beneficiaban de la disciplina y la educación de estos superpadres, y luego, después de tres años en las escuelas del Príncipe (entraban entre los trece y los quince años), eran adecuados para el matrimonio y estaban preparados para recibir su nombramiento como emisarios del Príncipe ante las tribus necesitadas de sus razas respectivas.

\par
%\textsuperscript{(751.1)}
\textsuperscript{66:7.6} Fad patrocinó el plan de enseñanza de Dalamatia, que se llevó a cabo mediante una escuela industrial en la que los alumnos aprendían a través de la práctica y se abrían camino realizando diariamente tareas útiles. Este plan educativo no pasaba por alto el lugar que ocupa el pensamiento y los sentimientos en el desarrollo del carácter, pero daba prioridad a la formación manual. La enseñanza era individual y colectiva. A los alumnos los enseñaban tanto los hombres como las mujeres, y los dos trabajando conjuntamente. La mitad de esta instrucción colectiva se impartía por sexos, y la otra mitad era enseñanza mixta. A los estudiantes se les enseñaba individualmente la destreza manual y se les reunía en grupos o clases para socializar. Se les educaba para que fraternizaran con los grupos más jóvenes, con los grupos de más edad y con los adultos, así como a trabajar en equipo con los de su misma edad. También se les familiarizaba con las asociaciones tales como los grupos familiares, los equipos de juego y las clases escolares.

\par
%\textsuperscript{(751.2)}
\textsuperscript{66:7.7} Entre los últimos estudiantes que se formaron en Mesopotamia para trabajar con sus razas respectivas se encontraban los andonitas de las tierras altas de la India occidental y algunos representantes de los hombres rojos y de los hombres azules; más tarde aún también se admitió a un pequeño número de la raza amarilla.

\par
%\textsuperscript{(751.3)}
\textsuperscript{66:7.8} Hap ofreció a las razas primitivas una ley moral. Este código era conocido como <<el Camino del Padre>> y consistía en los siete mandamientos siguientes:

\par
%\textsuperscript{(751.4)}
\textsuperscript{66:7.9} 1. No temerás ni servirás a ningún Dios, salvo al Padre de todos\footnote{\textit{Servirás sólo a Dios Padre}: Ex 20:3; Dt 5:7.}.

\par
%\textsuperscript{(751.5)}
\textsuperscript{66:7.10} 2. No desobedecerás al Hijo del Padre, el soberano del mundo, ni mostrarás falta de respeto por sus asociados superhumanos\footnote{\textit{Respetarás/obedecerás a tus dioses}: Ex 20:6; Dt 5:11.}.

\par
%\textsuperscript{(751.6)}
\textsuperscript{66:7.11} 3. No mentirás cuando seas convocado ante los jueces del pueblo\footnote{\textit{No mentirás ante los jueces}: Ex 20:16; Dt 5:20.}.

\par
%\textsuperscript{(751.7)}
\textsuperscript{66:7.12} 4. No matarás a hombres, mujeres o niños\footnote{\textit{No matarás}: Ex 20:13; Dt 5:17.}.

\par
%\textsuperscript{(751.8)}
\textsuperscript{66:7.13} 5. No robarás los bienes ni el ganado de tu prójimo\footnote{\textit{No robarás}: Ex 20:15; Dt 5:19.}.

\par
%\textsuperscript{(751.9)}
\textsuperscript{66:7.14} 6. No tocarás a la esposa de tu amigo\footnote{\textit{No tocarás a la esposa de tu amigo}: Ex 20:17; Dt 5:21.}.

\par
%\textsuperscript{(751.10)}
\textsuperscript{66:7.15} 7. No mostrarás falta de respeto por tus padres ni por los ancianos de la tribu\footnote{\textit{Respetarás a tus padres y ancianos}: Ex 20:12; Dt 5:16.}.

\par
%\textsuperscript{(751.11)}
\textsuperscript{66:7.16} Ésta fue la ley de Dalamatia durante cerca de trescientos mil años. Muchas de las piedras donde se inscribió esta ley yacen actualmente bajo las aguas a la altura de las costas de Mesopotamia y Persia. Se convirtió en una costumbre retener en la memoria uno de estos mandamientos por cada día de la semana, empleándose como saludo y como acción de gracias a la hora de comer.

\par
%\textsuperscript{(751.12)}
\textsuperscript{66:7.17} En esta época, el tiempo se medía por meses lunares, y este período se consideraba de veintiocho días. A excepción del día y de la noche, ésta era la única medida de tiempo que conocían estos pueblos primitivos. Los instructores de Dalamatia introdujeron la semana de siete días, que tuvo su origen en el hecho de que el número siete es la cuarta parte de veintiocho. El significado del número siete en el superuniverso les proporcionó sin duda alguna la oportunidad de introducir un recordatorio espiritual en el cálculo habitual del tiempo. Pero el período semanal no tiene un origen natural.

\par
%\textsuperscript{(751.13)}
\textsuperscript{66:7.18} El campo estaba muy bien colonizado en un radio de ciento sesenta kilómetros alrededor de la ciudad. En las inmediaciones de la ciudad, cientos de diplomados de las escuelas del Príncipe practicaban la ganadería o llevaban a cabo de otras maneras la enseñanza que habían recibido de su estado mayor y de sus numerosos colaboradores humanos. Unos cuantos se dedicaron a la agricultura y la horticultura.

\par
%\textsuperscript{(751.14)}
\textsuperscript{66:7.19} La humanidad no fue destinada al duro trabajo de la agricultura como castigo por un supuesto pecado. <<Comerás el fruto de los campos con el sudor de tu frente>>\footnote{\textit{Con el sudor de tu frente}: Gn 3:19.} no fue un castigo pronunciado contra el hombre por haber participado en las locuras de la rebelión de Lucifer bajo la dirección del traidor Caligastia. El cultivo de la tierra es inherente al establecimiento de una civilización progresiva en los mundos evolutivos, y este mandato fue el centro de toda la enseñanza del Príncipe Planetario y de su estado mayor durante los trescientos mil años que transcurrieron entre su llegada a Urantia y los días trágicos en que Caligastia compartió su suerte con la del rebelde Lucifer. El trabajo de la tierra no es una maldición; es más bien la bendición más elevada para todos aquellos que pueden disfrutar así de la más humana de todas las actividades humanas.

\par
%\textsuperscript{(752.1)}
\textsuperscript{66:7.20} Cuando estalló la rebelión, Dalamatia tenía una población permanente de casi seis mil habitantes. Esta cifra incluye a los estudiantes asiduos, pero no engloba a los visitantes ni a los observadores, que siempre ascendían a más de mil. Pero difícilmente os podéis hacer una idea de los progresos maravillosos de aquellos tiempos tan lejanos; la terrible confusión y las abyectas tinieblas espirituales que siguieron a la catástrofe de engaño y sedición de Caligastia destruyeron prácticamente todos los asombrosos logros humanos de aquella época.

\section*{8. Las desgracias de Caligastia}
\par
%\textsuperscript{(752.2)}
\textsuperscript{66:8.1} Cuando reflexionamos sobre la larga carrera de Caligastia, sólo encontramos una característica sobresaliente en su conducta que podría haber llamado la atención: era extremadamente individualista. Tenía la tendencia de ponerse de parte de casi todos los grupos que protestaban y simpatizaba generalmente con aquellos que expresaban con moderación sus críticas implícitas. Detectamos la aparición temprana de esta tendencia a impacientarse ante la autoridad, a ofenderse ligeramente ante todo tipo de supervisión. Aunque estuviera algo resentido por los consejos de sus mayores y fuera un poco reacio a la autoridad de sus superiores, sin embargo, cada vez que había sido sometido a una prueba, siempre se había mostrado leal a los gobernantes del universo y obediente a los mandatos de los Padres de la Constelación. Nunca se había encontrado ninguna verdadera falta en él hasta el momento de su vergonzosa traición en Urantia.

\par
%\textsuperscript{(752.3)}
\textsuperscript{66:8.2} Es preciso señalar que tanto a Lucifer como a Caligastia se les había informado con paciencia, y advertido con amor, acerca de sus tendencias a la crítica y del desarrollo sutil de su orgullo personal, con la correspondiente exageración del sentido de la vanidad. Pero todos estos intentos por ayudarlos habían sido malinterpretados como críticas infundadas e injerencias injustificadas en sus libertades personales. Tanto Caligastia como Lucifer estimaron que sus bondadosos consejeros actuaban con los mismos móviles reprensibles que empezaban a dominar sus propios pensamientos retorcidos y sus propios planes descaminados. Juzgaron a sus generosos consejeros según la evolución de su propio egoísmo.

\par
%\textsuperscript{(752.4)}
\textsuperscript{66:8.3} Desde la llegada del Príncipe Caligastia, la civilización planetaria progresó de manera bastante normal durante cerca de trescientos mil años. Aparte de ser una esfera de modificación de la vida, y por tanto sujeta a numerosas irregularidades y a episodios insólitos de fluctuaciones evolutivas, Urantia progresó de forma muy satisfactoria en su carrera planetaria hasta el momento de la rebelión de Lucifer y la traición simultánea de Caligastia. Este desatino catastrófico, así como el fracaso posterior de Adán y Eva en la realización de su misión planetaria, modificaron definitivamente toda la historia ulterior del planeta.

\par
%\textsuperscript{(752.5)}
\textsuperscript{66:8.4} El Príncipe de Urantia cayó en las tinieblas en el momento de la rebelión de Lucifer, precipitando así al planeta en una larga confusión. Posteriormente fue privado de su autoridad soberana mediante la acción coordinada de los gobernantes de la constelación y otras autoridades del universo. Compartió las vicisitudes inevitables del aislamiento de Urantia hasta la época de la estancia de Adán en el planeta, y contribuyó en parte al aborto del plan destinado a elevar las razas mortales mediante la inyección de la sangre vital de la nueva raza violeta ---los descendientes de Adán y Eva.

\par
%\textsuperscript{(753.1)}
\textsuperscript{66:8.5} La encarnación como mortal de Maquiventa Melquisedek, en la época de Abraham, redujo enormemente el poder que tenía el Príncipe caído para perturbar los asuntos humanos. Y posteriormente, durante la vida de Miguel en la carne, este Príncipe traidor fue finalmente despojado de toda autoridad en Urantia.

\par
%\textsuperscript{(753.2)}
\textsuperscript{66:8.6} Aunque la doctrina de un demonio personal en Urantia tenía algún fundamento debido a la presencia planetaria del traidor e inicuo Caligastia, sin embargo es totalmente ficticia cuando enseña que tal <<demonio>> puede influir en la mente humana normal en contra de su libre elección natural. Incluso antes de la donación de Miguel en Urantia, ni Caligastia ni Daligastia fueron nunca capaces de oprimir a los mortales o de coaccionar a un individuo normal a que realizara algún acto en contra de su voluntad humana. El libre albedrío del hombre es supremo en los asuntos morales; incluso el Ajustador del Pensamiento interior se niega a obligar al hombre a que tenga un solo pensamiento o realice un solo acto en contra de la elección de su propia voluntad.

\par
%\textsuperscript{(753.3)}
\textsuperscript{66:8.7} Y ahora, este rebelde del reino, despojado de todo poder para perjudicar a sus antiguos súbditos, aguarda la sentencia final de los Ancianos de los Días de Uversa para todos los que participaron en la rebelión de Lucifer.

\par
%\textsuperscript{(753.4)}
\textsuperscript{66:8.8} [Presentado por un Melquisedek de Nebadon.]


\chapter{Documento 67. La rebelión planetaria}
\par
%\textsuperscript{(754.1)}
\textsuperscript{67:0.1} ES IMPOSIBLE comprender los problemas relacionados con la existencia humana en Urantia si no se tiene conocimiento de ciertas grandes épocas del pasado, principalmente del acontecimiento y las consecuencias de la rebelión planetaria. Aunque esta sublevación no dificultó gravemente el progreso de la evolución orgánica, modificó de manera notable el curso de la evolución social y del desarrollo espiritual. Esta calamidad devastadora influyó profundamente en toda la historia superfísica del planeta.

\section*{1. La traición de Caligastia}
\par
%\textsuperscript{(754.2)}
\textsuperscript{67:1.1} Caligastia llevaba trescientos mil años encargado de Urantia cuando Satanás, el asistente de Lucifer, hizo una de sus visitas periódicas de inspección. Cuando Satanás llegó al planeta, su aspecto no se parecía en nada a vuestras caricaturas de su infame majestad. Era, y sigue siendo, un Hijo Lanonandek de gran esplendor. <<Y no os maravilléis, porque el mismo Satanás es una brillante criatura de luz>>\footnote{\textit{Satanás, brillante criatura}: 2 Co 11:14.}.

\par
%\textsuperscript{(754.3)}
\textsuperscript{67:1.2} En el transcurso de esta inspección, Satanás informó a Caligastia acerca de la <<Declaración de Libertad>> que Lucifer tenía entonces la intención de hacer, y tal como sabemos ahora, el Príncipe aceptó traicionar al planeta en cuanto se anunciara la rebelión. Las personalidades leales del universo consideran con un desdén particular al Príncipe Caligastia por esta traición premeditada de la confianza. El Hijo Creador expresó este desprecio cuando dijo: <<Te pareces a tu jefe Lucifer, y has perpetuado pecaminosamente su iniquidad. Fue un falsificador desde que empezó a exaltarse a sí mismo, porque no permanecía en la verdad>>\footnote{\textit{Es como Lucifer, un falsificador}: Jn 8:44.}.

\par
%\textsuperscript{(754.4)}
\textsuperscript{67:1.3} En todo el trabajo administrativo de un universo local, ningún cargo elevado se considera más sagrado que el que se confía a un Príncipe Planetario que asume la responsabilidad del bienestar y de la dirección de los mortales evolutivos de un mundo recién habitado. De todas las formas del mal, ninguna tiene un efecto más destructivo sobre la condición de la personalidad que la traición al deber y la deslealtad hacia unos amigos confiados. Al cometer este pecado deliberado, Caligastia deformó tanto su personalidad que su mente nunca más ha sido capaz de recuperar plenamente el equilibrio.

\par
%\textsuperscript{(754.5)}
\textsuperscript{67:1.4} Hay muchas maneras de considerar el pecado, pero desde el punto de vista filosófico del universo, el pecado es la actitud de una personalidad que se opone deliberadamente a la realidad cósmica. El error se puede considerar como una idea falsa o una deformación de la realidad. El mal es una comprensión parcial de las realidades del universo, o una inadaptación a ellas. Pero el pecado es una resistencia intencional a la realidad divina ---una elección consciente de oponerse al progreso espiritual--- mientras que la iniquidad consiste en desafiar de manera abierta y persistente la realidad reconocida, y representa tal grado de desintegración de la personalidad que raya en la locura cósmica.

\par
%\textsuperscript{(755.1)}
\textsuperscript{67:1.5} El error indica una falta de agudeza intelectual; el mal, una deficiencia de sabiduría; el pecado, una pobreza espiritual abyecta; pero la iniquidad indica que el control de la personalidad está desapareciendo.

\par
%\textsuperscript{(755.2)}
\textsuperscript{67:1.6} Cuando el pecado se ha elegido tantas veces y se ha repetido tan a menudo, puede convertirse en un hábito. Los pecadores empedernidos pueden volverse fácilmente inicuos, convertirse en unos rebeldes incondicionales contra el universo y todas sus realidades divinas. Aunque se pueden perdonar todas las clases de pecados, dudamos que el inicuo arraigado pueda experimentar nunca una aflicción sincera por sus fechorías o aceptar el perdón de sus pecados.

\section*{2. El comienzo de la rebelión}
\par
%\textsuperscript{(755.3)}
\textsuperscript{67:2.1} Poco después de la inspección de Satanás, cuando la administración planetaria estaba en vísperas de realizar grandes cosas en Urantia, un día a mediados del invierno de los continentes septentrionales Caligastia mantuvo una larga conversación con su asociado Daligastia, después de la cual este último convocó a los diez consejos de Urantia en sesión extraordinaria. Esta asamblea se inició con la declaración de que el Príncipe Caligastia estaba a punto de proclamarse soberano absoluto de Urantia, y exigía que todos los grupos administrativos abdicaran y pusieran todas sus funciones y poderes en manos de Daligastia, designado como fideicomisario hasta que se reorganizara el gobierno planetario y se redistribuyeran posteriormente estos cargos de autoridad administrativa.

\par
%\textsuperscript{(755.4)}
\textsuperscript{67:2.2} La presentación de esta asombrosa exigencia fue seguida por el llamamiento magistral de Van, presidente del consejo supremo de coordinación. Este administrador eminente y experto jurista tildó la vía que proponía Caligastia como un acto que rayaba en la rebelión planetaria, y rogó a sus compañeros que se abstuvieran de toda participación hasta que se pudiera presentar una apelación ante Lucifer, el Soberano del Sistema de Satania; y Van consiguió el apoyo de todo el estado mayor. En consecuencia, se interpuso una apelación a Jerusem y llegaron inmediatamente las órdenes designando a Caligastia como soberano supremo de Urantia y ordenando una lealtad absoluta e incondicional a sus mandatos. En respuesta a este asombroso mensaje, el noble Van contestó con su memorable discurso de siete horas en el cual acusó oficialmente a Daligastia, Caligastia y Lucifer de despreciar la soberanía del universo de Nebadon; y apeló a los Altísimos de Edentia para recibir su apoyo y su confirmación.

\par
%\textsuperscript{(755.5)}
\textsuperscript{67:2.3} Entretanto, los circuitos del sistema habían sido cortados; Urantia estaba aislada. Todos los grupos de vida celestial presentes en el planeta se encontraron repentinamente aislados sin ser advertidos, totalmente privados de todo consejo y asesoramiento exterior.

\par
%\textsuperscript{(755.6)}
\textsuperscript{67:2.4} Daligastia proclamó oficialmente a Caligastia <<Dios de Urantia y supremo por encima de todos>>\footnote{\textit{Dios de este mundo}: 2 Co 4:4.}. Ante esta proclamación, la alternativa estaba clara, y cada grupo se retiró para empezar sus deliberaciones, unas discusiones destinadas a determinar finalmente la suerte de todas las personalidades superhumanas que estaban en el planeta.

\par
%\textsuperscript{(755.7)}
\textsuperscript{67:2.5} Los serafines, los querubines y otros seres celestiales estuvieron implicados en las decisiones de esta lucha encarnizada, de este largo y pecaminoso conflicto. Muchos grupos superhumanos que se encontraban por casualidad en Urantia en el momento de ser aislada fueron retenidos aquí, y al igual que los serafines y sus asociados, se vieron obligados a elegir entre el pecado y la rectitud ---entre el camino de Lucifer y la voluntad del Padre invisible.

\par
%\textsuperscript{(756.1)}
\textsuperscript{67:2.6} Esta lucha continuó durante más de siete años. Las autoridades de Edentia no quisieron interferir, y no intervinieron, hasta que todas las personalidades involucradas hubieron tomado una decisión final. Fue en ese momento cuando Van y sus leales asociados recibieron la justificación y la liberación de su prolongada ansiedad y de su intolerable incertidumbre.

\section*{3. Los siete años decisivos}
\par
%\textsuperscript{(756.2)}
\textsuperscript{67:3.1} La noticia de que la rebelión había estallado en Jerusem, la capital de Satania, fue transmitida por el consejo de los Melquisedeks. Los Melquisedeks de emergencia fueron enviados inmediatamente a Jerusem, y Gabriel se ofreció voluntariamente para actuar como representante del Hijo Creador, cuya autoridad se había desafiado. El sistema fue puesto en cuarentena, quedó aislado de sus sistemas hermanos al mismo tiempo que se anunciaba el estado de rebelión en Satania. Había <<guerra en el cielo>>\footnote{\textit{Guerra en el cielo}: Ap 12:7.}, en la sede central de Satania, y esta guerra se extendió a todos los planetas del sistema local.

\par
%\textsuperscript{(756.3)}
\textsuperscript{67:3.2} En Urantia, cuarenta miembros del estado mayor corpóreo de los cien (Van incluido) rehusaron unirse a la insurrección. Muchos asistentes humanos (modificados y otros) del estado mayor eran también unos valientes y nobles defensores de Miguel y del gobierno de su universo. Hubo una terrible pérdida de personalidades entre los serafines y los querubines. Cerca de la mitad de los serafines administradores y de los serafines de transición asignados al planeta se unieron a su jefe y a Daligastia apoyando la causa de Lucifer. Cuarenta mil ciento diecinueve criaturas intermedias primarias se asociaron con Caligastia, pero el resto de estos seres permaneció fiel a su deber.

\par
%\textsuperscript{(756.4)}
\textsuperscript{67:3.3} El Príncipe traidor reunió a las criaturas intermedias desleales y a otros grupos de personalidades rebeldes y los organizó para que ejecutaran sus órdenes, mientras que Van congregó a los intermedios leales y a otros grupos fieles, y emprendió la gran batalla para salvar al estado mayor planetario y a las otras personalidades celestiales aisladas.

\par
%\textsuperscript{(756.5)}
\textsuperscript{67:3.4} Durante todo el tiempo de esta lucha, los leales residieron en una colonia mal protegida y sin murallas situada a unos kilómetros al este de Dalamatia, pero sus viviendas estaban custodiadas de día y de noche por las criaturas intermedias leales siempre alertas y vigilantes, y tenían en su poder el inestimable árbol de la vida.

\par
%\textsuperscript{(756.6)}
\textsuperscript{67:3.5} Cuando estalló la rebelión, unos querubines y serafines leales, con la ayuda de tres intermedios fieles, asumieron la custodia del árbol de la vida, y sólo permitieron que los cuarenta leales del estado mayor y sus asociados humanos modificados comieran del fruto y de las hojas de esta planta energética. Cincuenta y seis de estos asociados andonitas modificados estaban con Van, ya que dieciséis asistentes andonitas del estado mayor desleal se habían negado a seguir a sus jefes en la rebelión.

\par
%\textsuperscript{(756.7)}
\textsuperscript{67:3.6} A lo largo de los siete años decisivos de la rebelión de Caligastia, Van se consagró por completo a la tarea de atender a su ejército leal de hombres, intermedios y ángeles. La perspicacia espiritual y la constancia moral que permitieron a Van conservar esta actitud inquebrantable de lealtad al gobierno del universo fueron el resultado de un pensamiento claro, un razonamiento acertado, un juicio lógico, una motivación sincera, una intención desinteresada, una lealtad inteligente, una memoria experiencial, un carácter disciplinado y la consagración incondicional de su personalidad a hacer la voluntad del Padre que está en el Paraíso.

\par
%\textsuperscript{(756.8)}
\textsuperscript{67:3.7} Estos siete años de espera fueron un período de examen de conciencia y de disciplina del alma. Este tipo de crisis en los asuntos de un universo demuestran la enorme influencia de la mente como factor en la elección espiritual. La educación, la formación y la experiencia son factores que intervienen en la mayoría de las decisiones vitales de todas las criaturas morales evolutivas. Pero al espíritu interior le es totalmente posible ponerse en contacto directo con los poderes que determinan las decisiones de la personalidad humana, y facultar así a la voluntad plenamente consagrada de la criatura para que lleve a cabo unos actos asombrosos de devoción leal a la voluntad y al camino del Padre que está en el Paraíso. Y esto es precisamente lo que sucedió en la experiencia de Amadón, el asociado humano modificado de Van.

\par
%\textsuperscript{(757.1)}
\textsuperscript{67:3.8} Amadón es el héroe humano más destacado de la rebelión de Lucifer. Este descendiente varón de Andón y Fonta fue uno de los cien que aportaron su plasma vital al estado mayor del Príncipe, y desde aquel acontecimiento siempre había estado vinculado a Van en calidad de asociado y asistente humano. Amadón eligió permanecer con su jefe durante toda esta lucha prolongada y difícil. Fue un espectáculo inspirador contemplar a este hijo de las razas evolutivas permanecer impasible ante las sofisterías de Daligastia, mientras que durante los siete años de la lucha, él y sus compañeros leales resistieron con una inquebrantable entereza a todas las enseñanzas engañosas del brillante Caligastia.

\par
%\textsuperscript{(757.2)}
\textsuperscript{67:3.9} Caligastia, con un máximo de inteligencia y una inmensa experiencia en los asuntos del universo, se descarrió ---abrazó el pecado. Amadón, con un mínimo de inteligencia y totalmente desprovisto de experiencia universal, permaneció firme al servicio del universo y leal a su asociado. Van empleó tanto la mente como el espíritu en una magnífica y eficaz combinación de resolución intelectual y de perspicacia espiritual, logrando así un nivel experiencial de desarrollo de la personalidad del tipo más elevado que se pueda conseguir. Cuando la mente y el espíritu están plenamente unidos, poseen el potencial de crear valores superhumanos, e incluso realidades morontiales.

\par
%\textsuperscript{(757.3)}
\textsuperscript{67:3.10} La narración de los acontecimientos conmovedores de aquellos trágicos días sería interminable. Pero por fin la última personalidad que quedaba tomó su decisión final y entonces, sólo entonces, fue cuando llegó un Altísimo de Edentia con los Melquisedeks de emergencia para asumir la autoridad en Urantia. Los archivos panorámicos del reinado de Caligastia fueron borrados en Jerusem, y empezó la época probatoria de la rehabilitación planetaria.

\section*{4. Los cien de Caligastia después de la rebelión}
\par
%\textsuperscript{(757.4)}
\textsuperscript{67:4.1} Cuando finalmente se pasó lista, se descubrió que los miembros corpóreos del estado mayor del Príncipe se habían alineado como sigue: Van y todo su tribunal de coordinación habían permanecido leales. Ang y tres miembros del consejo de la alimentación habían sobrevivido. Todo el consejo de la ganadería se había unido a la rebelión así como todos los consejeros encargados de vencer a los animales. Fad y cinco miembros del cuerpo docente se habían salvado. Nod y toda la comisión de la industria y el comercio se habían unido a Caligastia. Hap y toda la escuela de la religión revelada permanecían leales a Van y a su noble grupo. Lut y todo el consejo de la salud se habían perdido. El consejo de las artes y las ciencias permanecía leal en su totalidad, pero Tut y toda la comisión encargada de los gobiernos tribales se habían descarriado. Así pues, de los cien se salvaron cuarenta, y más tarde fueron trasladados a Jerusem, donde reanudaron su carrera hacia el Paraíso.

\par
%\textsuperscript{(757.5)}
\textsuperscript{67:4.2} Los sesenta miembros del estado mayor planetario que entraron en la rebelión eligieron a Nod como jefe. Trabajaron con entusiasmo para el Príncipe rebelde, pero pronto descubrieron que estaban privados del alimento de los circuitos vitales del sistema. Se dieron cuenta del hecho de que habían sido degradados al estado de los seres mortales. Eran en verdad superhumanos, pero al mismo tiempo materiales y mortales. En un intento por acrecentar su número, Daligastia ordenó que recurrieran inmediatamente a la reproducción sexual, sabiendo muy bien que los sesenta originales y sus cuarenta y cuatro asociados andonitas modificados estaban condenados a morir tarde o temprano. Después de la caída de Dalamatia, el estado mayor desleal emigró hacia el norte y el este. Sus descendientes fueron conocidos durante mucho tiempo como los noditas y el lugar donde vivían como <<la tierra de Nod>>\footnote{\textit{La tierra de Nod}: Gn 4:16.}.

\par
%\textsuperscript{(758.1)}
\textsuperscript{67:4.3} La presencia de estos superhombres y supermujeres extraordinarios, abandonados a su suerte debido a la rebelión y que luego se unieron con los hijos y las hijas de la Tierra, dio fácilmente nacimiento a los relatos tradicionales de los dioses que descendían del cielo para casarse con los mortales. Éste fue el origen de las mil y una leyendas de naturaleza mítica, pero basadas en los hechos de los tiempos posteriores a la rebelión, que se incorporaron más adelante en los cuentos y las tradiciones folclóricas de diversos pueblos, cuyos antepasados habían participado en estos contactos con los noditas y sus descendientes.

\par
%\textsuperscript{(758.2)}
\textsuperscript{67:4.4} Privados del alimento espiritual, los rebeldes del estado mayor murieron finalmente de muerte natural. Una gran parte de la idolatría posterior de las razas humanas tuvo su origen en el deseo de perpetuar la memoria de estos seres sumamente respetados de la época de Caligastia.

\par
%\textsuperscript{(758.3)}
\textsuperscript{67:4.5} Cuando vinieron a Urantia, los cien del estado mayor habían sido separados temporalmente de sus Ajustadores del Pensamiento. Inmediatamente después de la llegada de los síndicos Melquisedeks, las personalidades leales (a excepción de Van) fueron devueltas a Jerusem y reunidas con sus Ajustadores que los esperaban. No conocemos el destino de los sesenta rebeldes del estado mayor; sus Ajustadores permanecen todavía en Jerusem. Las cosas continuarán sin duda tal como están ahora hasta que se juzgue finalmente toda la rebelión de Lucifer y se decrete el destino de todos los participantes.

\par
%\textsuperscript{(758.4)}
\textsuperscript{67:4.6} A unos seres como los ángeles y los intermedios les resultaba muy difícil concebir que unos brillantes dirigentes de confianza como Caligastia y Daligastia pudieran extraviarse ---cometieran un pecado de traición. Aquellos seres que cayeron en el pecado ---que no se sumaron a la rebelión de manera deliberada o premeditada--- fueron inducidos a error por sus superiores, engañados por sus jefes en quienes confiaban. También fue fácil conseguir el apoyo de los mortales evolutivos con mentalidad primitiva.

\par
%\textsuperscript{(758.5)}
\textsuperscript{67:4.7} La inmensa mayoría de los seres humanos y superhumanos que fueron víctimas de la rebelión de Lucifer en Jerusem y en los diversos planetas descarriados, hace mucho tiempo que se arrepintieron sinceramente de su locura. Y creemos de verdad que todos estos penitentes sinceros serán rehabilitados de alguna manera y reintegrados en cualquier fase del servicio del universo cuando los Ancianos de los Días terminen finalmente de juzgar los asuntos de la rebelión de Satania, cosa que han emprendido recientemente.

\section*{5. Los resultados inmediatos de la rebelión}
\par
%\textsuperscript{(758.6)}
\textsuperscript{67:5.1} Una gran confusión reinó en Dalamatia y en sus inmediaciones durante cerca de cincuenta años después de la instigación a la rebelión. Se intentó realizar una reorganización completa y radical del mundo entero; la revolución sustituyó a la evolución como política de progreso cultural y de mejoramiento racial. Apareció un progreso repentino en el nivel cultural de los alumnos superiores parcialmente educados que residían en Dalamatia y sus alrededores; pero cuando estos métodos nuevos y radicales se intentaron aplicar a los pueblos alejados, el resultado inmediato fue una confusión indescriptible y un pandemónium racial. La libertad fue transformada rápidamente en libertinaje por los hombres primitivos medio evolucionados de aquella época.

\par
%\textsuperscript{(758.7)}
\textsuperscript{67:5.2} Poco después de la rebelión, todo el estado mayor de la sedición estaba defendiendo enérgicamente la ciudad contra las hordas de semisalvajes que asediaban sus murallas a consecuencia de las doctrinas de libertad que se les habían enseñado prematuramente. Unos años antes de que la hermosa sede se sumergiera bajo las aguas del sur, las tribus equivocadas y mal instruidas de las tierras interiores de Dalamatia ya se habían precipitado en un asalto semisalvaje sobre la espléndida ciudad, arrojando hacia el norte al estado mayor secesionista y sus asociados.

\par
%\textsuperscript{(759.1)}
\textsuperscript{67:5.3} El proyecto de Caligastia de reconstruir inmediatamente la sociedad humana de acuerdo con sus ideas sobre las libertades individuales y colectivas resultó ser un fracaso inmediato y más o menos total. La sociedad volvió a hundirse rápidamente en su antiguo nivel biológico, y la lucha por el progreso empezó en todas partes partiendo de un punto no mucho más avanzado del que se encontraba al principio del régimen de Caligastia, ya que este levantamiento había dejado al mundo en la peor de las confusiones.

\par
%\textsuperscript{(759.2)}
\textsuperscript{67:5.4} Ciento sesenta y dos años después de la rebelión, una marejada barrió a Dalamatia y la sede planetaria se hundió bajo las aguas del mar; esta tierra no volvió a emerger hasta que casi todos los vestigios de la noble cultura de aquellas épocas espléndidas habían desaparecido.

\par
%\textsuperscript{(759.3)}
\textsuperscript{67:5.5} Cuando la primera capital del mundo se sumergió, sólo albergaba a los tipos más inferiores de las razas sangiks de Urantia, unos renegados que ya habían convertido el templo del Padre en un santuario dedicado a Nog, el falso dios de la luz y el fuego.

\section*{6. Van ---el inquebrantable}
\par
%\textsuperscript{(759.4)}
\textsuperscript{67:6.1} Los partidarios de Van se retiraron muy pronto a las tierras altas del oeste de la India, donde estuvieron a salvo de los ataques de las razas confundidas de las tierras bajas; desde este lugar apartado proyectaron la rehabilitación del mundo, al igual que sus antiguos predecesores badonitas habían trabajado involuntariamente en otra época por el bienestar de la humanidad, justo antes de que nacieran las tribus sangiks.

\par
%\textsuperscript{(759.5)}
\textsuperscript{67:6.2} Antes de la llegada de los síndicos Melquisedeks, Van puso la administración de los asuntos humanos en las manos de diez comisiones de cuatro miembros cada una, unos grupos idénticos a los del régimen del Príncipe. Los Portadores de Vida residentes más antiguos asumieron la dirección temporal de este consejo de cuarenta miembros, que funcionó durante los siete años de espera. Unos grupos similares de amadonitas asumieron estas responsabilidades cuando los treinta y nueve miembros leales del estado mayor regresaron a Jerusem.

\par
%\textsuperscript{(759.6)}
\textsuperscript{67:6.3} Estos \textit{amadonitas} procedían del grupo de 144 andonitas leales al que pertenecía Amadón, y a los cuales había dado su nombre. Este grupo constaba de treinta y nueve hombres y ciento cinco mujeres. De todos ellos, cincuenta y seis tenían el estado de inmortalidad, y todos fueron trasladados (a excepción de Amadón) en compañía de los miembros leales del estado mayor. El resto de este noble grupo continuó en la Tierra hasta el final de sus días como mortales bajo la dirección de Van y Amadón. Fueron la levadura biológica que se multiplicó y continuó asegurando la dirección del mundo durante las largas épocas tenebrosas de la era posterior a la rebelión.

\par
%\textsuperscript{(759.7)}
\textsuperscript{67:6.4} Van fue dejado en Urantia hasta la época de Adán, permaneciendo como jefe titular de todas las personalidades superhumanas que ejercían sus funciones en el planeta. Él y Amadón se sustentaron durante más de ciento cincuenta mil años mediante la técnica del árbol de la vida en unión con el ministerio vital especializado de los Melquisedeks.

\par
%\textsuperscript{(759.8)}
\textsuperscript{67:6.5} Los asuntos de Urantia fueron administrados durante mucho tiempo por un consejo de síndicos planetarios, doce Melquisedeks confirmados por orden del gobernante decano de la constelación, el Altísimo Padre de Norlatiadek. Un consejo asesor estaba asociado con los síndicos Melquisedeks, y se componía de: uno de los asistentes leales del Príncipe caído, los dos Portadores de Vida residentes, un Hijo Trinitizado en fase de aprendizaje, un Hijo Instructor voluntario, una Brillante Estrella Vespertina de Avalon (que venía periódicamente), los jefes de los serafines y los querubines, unos consejeros procedentes de dos planetas vecinos, el director general de la vida angélica subordinada y Van, el comandante en jefe de las criaturas intermedias. Urantia fue gobernada y administrada de esta manera hasta la llegada de Adán. No es de extrañar que al valiente y leal Van se le asignara una plaza en el consejo de los síndicos planetarios que administraron durante tanto tiempo los asuntos de Urantia.

\par
%\textsuperscript{(760.1)}
\textsuperscript{67:6.6} Los doce síndicos Melquisedeks de Urantia realizaron una labor heroica. Preservaron los restos de la civilización y su política planetaria fue ejecutada fielmente por Van. Cerca de mil años después de la rebelión, Van había dispersado más de trescientos cincuenta grupos avanzados por el mundo. Estos puestos avanzados de la civilización estaban compuestos en gran parte por los descendientes de los andonitas leales ligeramente mezclados con las razas sangiks, sobre todo con los hombres azules, y con los noditas.

\par
%\textsuperscript{(760.2)}
\textsuperscript{67:6.7} A pesar del terrible retroceso provocado por la rebelión, había muchos buenos linajes biológicamente prometedores en la Tierra. Bajo la supervisión de los síndicos Melquisedeks, Van y Amadón continuaron la tarea de fomentar la evolución natural de la raza humana, haciendo progresar la evolución física del hombre hasta que ésta alcanzó el punto culminante que justificó el envío de un Hijo y una Hija Materiales a Urantia.

\par
%\textsuperscript{(760.3)}
\textsuperscript{67:6.8} Van y Amadón permanecieron en la Tierra hasta poco después de la llegada de Adán y Eva. Algunos años más tarde fueron trasladados a Jerusem, donde Van se reunió con su Ajustador que lo esperaba. Van trabaja ahora al servicio de Urantia mientras espera la orden de continuar el larguísimo camino hacia la perfección del Paraíso y hacia el destino no revelado del Cuerpo de la Finalidad de los Mortales que está en proceso de formación.

\par
%\textsuperscript{(760.4)}
\textsuperscript{67:6.9} Debemos indicar que cuando Van apeló a los Altísimos de Edentia, después de que Lucifer apoyara a Caligastia en Urantia, los Padres de la Constelación enviaron inmediatamente una resolución apoyando a Van en todos los puntos en litigio. Este veredicto no logró llegar hasta Van porque los circuitos planetarios de comunicación fueron cortados mientras se estaba transmitiendo. Hace poco tiempo que se descubrió que esta orden efectiva se encontraba alojada en un transmisor repetidor de energía, donde había quedado bloqueada desde el aislamiento de Urantia. Sin este descubrimiento, realizado gracias a las investigaciones de los intermedios de Urantia, la comunicación de esta decisión hubiera tenido que esperar a que Urantia fuera restablecida en los circuitos de la constelación. Este accidente aparente en las comunicaciones interplanetarias se produjo porque los transmisores de energía pueden recibir y transmitir la información, pero no pueden iniciar las comunicaciones.

\par
%\textsuperscript{(760.5)}
\textsuperscript{67:6.10} El estado legal de Van en los archivos jurídicos de Satania no se pudo clarificar, de manera efectiva y definitiva, hasta que esta orden de los Padres de Edentia fue registrada en Jerusem.

\section*{7. Las repercusiones lejanas del pecado}
\par
%\textsuperscript{(760.6)}
\textsuperscript{67:7.1} Las consecuencias personales (centrípetas) del rechazo voluntario y persistente de la luz por parte de una criatura, son a la vez inevitables e individuales, y sólo incumben a la Deidad y a la criatura personal en cuestión. Esta cosecha de iniquidad, que destruye el alma, es la siega interior de la criatura volitiva inicua.

\par
%\textsuperscript{(761.1)}
\textsuperscript{67:7.2} Pero no sucede lo mismo con las repercusiones externas del pecado: Las consecuencias impersonales (centrífugas) por haber abrazado el pecado son a la vez inevitables y colectivas, y atañen a todas las criaturas que ejercen su actividad dentro de la zona afectada por esos acontecimientos.

\par
%\textsuperscript{(761.2)}
\textsuperscript{67:7.3} Cincuenta mil años después del derrumbamiento de la administración planetaria, los asuntos terrenales estaban tan desorganizados y atrasados que la raza humana había ganado muy poco con respecto a la situación evolutiva general que existía en la época de la llegada de Caligastia, trescientos cincuenta mil años antes. Se habían hecho progresos en ciertos aspectos, y se había perdido mucho terreno en otras direcciones.

\par
%\textsuperscript{(761.3)}
\textsuperscript{67:7.4} Los efectos del pecado no son nunca puramente locales. Los sectores administrativos de los universos son como un organismo; la condición de una personalidad debe ser compartida, hasta cierto punto, por todos. Como el pecado es una actitud de la persona con respecto a la realidad, está destinado a manifestar su cosecha negativa inherente en todos y cada uno de los niveles relacionados de valores universales. Pero las plenas consecuencias del pensamiento erróneo, de la maldad o de los proyectos pecaminosos, sólo se experimentan en el nivel de la acción misma. La transgresión de la ley universal puede ser fatal en el ámbito físico, sin implicar gravemente a la mente ni deteriorar la experiencia espiritual. El pecado sólo está cargado de consecuencias fatales para la supervivencia de la personalidad cuando representa la actitud de todo el ser, cuando significa la elección de la mente y la voluntad del alma.

\par
%\textsuperscript{(761.4)}
\textsuperscript{67:7.5} El mal y el pecado infligen sus consecuencias en los ámbitos materiales y sociales, e incluso a veces pueden retrasar el progreso espiritual en ciertos niveles de la realidad universal, pero el pecado de un ser determinado jamás le roba a otro ser la realización del derecho divino a la supervivencia de la personalidad. Las decisiones de la mente y la elección del alma del individuo mismo son las únicas que pueden poner en peligro la supervivencia eterna.

\par
%\textsuperscript{(761.5)}
\textsuperscript{67:7.6} El pecado cometido en Urantia retrasó muy poco la evolución biológica, pero tuvo el efecto de privar a las razas mortales del beneficio completo de la herencia adámica. El pecado retrasa enormemente el desarrollo intelectual, el crecimiento moral, el progreso social y la consecución espiritual de las masas. Pero no impide que cualquier persona que escoja conocer a Dios y hacer sinceramente su voluntad divina consiga el logro espiritual más elevado.

\par
%\textsuperscript{(761.6)}
\textsuperscript{67:7.7} Caligastia se rebeló, Adán y Eva incumplieron su deber, pero ningún mortal que ha nacido posteriormente en Urantia ha sufrido en su experiencia espiritual personal a consecuencia de estos desatinos. Todos los mortales que han nacido en Urantia después de la rebelión de Caligastia han sido perjudicados de alguna manera en el tiempo, pero el bienestar futuro de sus almas jamás ha corrido el menor peligro en la eternidad. A ninguna persona se le obliga nunca a sufrir una privación espiritual esencial a causa del pecado de otra. El pecado es totalmente personal en lo que se refiere a la culpabilidad moral o a las consecuencias espirituales, a pesar de sus extensas repercusiones en los ámbitos administrativos, intelectuales y sociales.

\par
%\textsuperscript{(761.7)}
\textsuperscript{67:7.8} Aunque no podemos comprender la sabiduría que permite estas catástrofes, siempre podemos discernir los efectos benéficos de estos desórdenes locales a medida que se reflejan en el universo en general.

\section*{8. El héroe humano de la rebelión}
\par
%\textsuperscript{(761.8)}
\textsuperscript{67:8.1} Muchos seres valientes se opusieron a la rebelión de Lucifer en los diversos mundos de Satania; pero los archivos de Salvington describen a Amadón como el personaje más sobresaliente de todo el sistema por su glorioso rechazo a los torrentes de sedición y por su devoción inquebrantable a Van ---los dos permanecieron inamovibles en su lealtad a la supremacía del Padre invisible y a la de su Hijo Miguel.

\par
%\textsuperscript{(762.1)}
\textsuperscript{67:8.2} En la época de estos importantes acontecimientos yo estaba destinado en Edentia, y todavía tengo conciencia de la alegría que experimenté cuando examiné las transmisiones de Salvington que contaban, día tras día, la increíble firmeza, la devoción trascendente y la exquisita lealtad de este antiguo semisalvaje surgido del linaje original y experimental de la raza andónica.

\par
%\textsuperscript{(762.2)}
\textsuperscript{67:8.3} Desde Edentia hasta Uversa, pasando por Salvington, y durante siete largos años, la primera pregunta de todos los seres celestiales subordinados con respecto a la rebelión de Satania era una y otra vez: <<¿Qué sucede con Amadón de Urantia, continúa inamovible?>>

\par
%\textsuperscript{(762.3)}
\textsuperscript{67:8.4} Si la rebelión de Lucifer ha perjudicado al sistema local y a sus mundos caídos, si la pérdida de este Hijo y de sus asociados descarriados ha obstaculizado temporalmente el progreso de la constelación de Norlatiadek, considerad por el contrario el efecto que tuvo la extensa exposición de la actuación inspiradora de este hijo único de la naturaleza y de su grupo resuelto de 143 camaradas, que abogaron inquebrantablemente por los conceptos más elevados de la gestión y la administración del universo, a pesar de la formidable presión adversa que ejercían sus superiores desleales. Permitidme aseguraros que esto ya ha hecho mucho más bien en el universo de Nebadon y el superuniverso de Orvonton, que lo que pueda pesar la suma total de todo el mal y la aflicción de la rebelión de Lucifer.

\par
%\textsuperscript{(762.4)}
\textsuperscript{67:8.5} Todo lo anterior ilustra de manera exquisitamente conmovedora y extraordinariamente magnífica la sabiduría del plan universal del Padre consistente en movilizar el Cuerpo de la Finalidad de los Mortales en el Paraíso, y en reclutar gran parte de este inmenso grupo de servidores misteriosos del futuro en la arcilla corriente de los mortales en progreso ascendente ---precisamente en unos mortales como el inquebrantable Amadón.

\par
%\textsuperscript{(762.5)}
\textsuperscript{67:8.6} [Presentado por un Melquisedek de Nebadon.]


\chapter{Documento 73. El Jardín del Edén}
\par
%\textsuperscript{(821.1)}
\textsuperscript{73:0.1} LA decadencia cultural y la pobreza espiritual que se derivaron de la caída de Caligastia y de la consiguiente confusión social, tuvieron poco efecto sobre el estado físico o biológico de los pueblos de Urantia. La evolución orgánica continuó a paso acelerado, sin tener en cuenta para nada la regresión cultural y moral que siguió tan rápidamente a la deslealtad de Caligastia y Daligastia. Hace casi cuarenta mil años, hubo un momento en la historia planetaria en que los Portadores de Vida de servicio observaron que, desde un punto de vista puramente biológico, el progreso del desarrollo de las razas de Urantia se acercaba a su culminación. Los síndicos Melquisedeks coincidieron con esta opinión y aceptaron unirse enseguida a los Portadores de Vida para hacer una petición a los Altísimos de Edentia solicitándoles que Urantia fuera inspeccionada con vistas a que se autorizara el envío de los mejoradores biológicos, un Hijo y una Hija Materiales.

\par
%\textsuperscript{(821.2)}
\textsuperscript{73:0.2} Esta petición se dirigió a los Altísimos de Edentia porque habían ejercido una jurisdicción directa sobre muchos asuntos de Urantia desde la caída de Caligastia y la ausencia temporal de autoridad en Jerusem.

\par
%\textsuperscript{(821.3)}
\textsuperscript{73:0.3} Tabamantia, el supervisor soberano de la serie de mundos decimales o experimentales, vino a inspeccionar el planeta, y después de examinar el progreso racial, recomendó debidamente que se concedieran unos Hijos Materiales a Urantia. Poco menos de cien años después de esta inspección, Adán y Eva, un Hijo y una Hija Materiales del sistema local, llegaron y emprendieron la difícil tarea de intentar desenredar los asuntos confusos de un planeta atrasado por la rebelión y que permanecía proscrito por el aislamiento espiritual.

\section*{1. Los noditas y los amadonitas}
\par
%\textsuperscript{(821.4)}
\textsuperscript{73:1.1} En un planeta normal, la llegada del Hijo Material anuncia generalmente la proximidad de una gran era de invención, de progreso material y de iluminación intelectual. En la mayoría de los mundos, la era postadámica es la gran época científica, pero no fue así en Urantia. Aunque el planeta estaba poblado de razas físicamente capacitadas, las tribus languidecían en el abismo del salvajismo y del estancamiento moral.

\par
%\textsuperscript{(821.5)}
\textsuperscript{73:1.2} Diez mil años después de la rebelión, todos los beneficios de la administración del Príncipe habían prácticamente desaparecido; las razas del mundo estaban poco mejor que si este Hijo descaminado no hubiera venido nunca a Urantia. Las tradiciones de Dalamatia y la cultura del Príncipe Planetario sólo perduraron entre los noditas y los amadonitas.

\par
%\textsuperscript{(821.6)}
\textsuperscript{73:1.3} \textit{Los noditas} eran los descendientes de los miembros rebeldes del estado mayor del Príncipe, y su nombre provenía de su primer jefe, Nod, el antiguo presidente de la comisión de la industria y el comercio de Dalamatia. \textit{Losamadonitas} eran los descendientes de aquellos andonitas que escogieron permanecer leales con Van y Amadón. <<Amadonita>> es más bien una denominación cultural y religiosa que un término racial; desde el punto de vista racial, los amadonitas eran esencialmente \textit{andonitas}. <<Nodita>> es un término tanto cultural como racial, ya que los mismos noditas constituyeron la octava raza de Urantia.

\par
%\textsuperscript{(822.1)}
\textsuperscript{73:1.4} Existía una enemistad tradicional entre los noditas y los amadonitas. Este odio hereditario afloraba constantemente cada vez que los descendientes de estos dos grupos intentaban participar en alguna empresa común. Incluso más tarde, les resultó extremadamente difícil trabajar juntos en paz en los asuntos del Edén.

\par
%\textsuperscript{(822.2)}
\textsuperscript{73:1.5} Poco después de la destrucción de Dalamatia, los seguidores de Nod se dividieron en tres grupos principales. El grupo central permaneció en las inmediaciones de su tierra natal, cerca de la cabecera del Golfo Pérsico. El grupo oriental emigró hacia las regiones de las tierras altas de Elam, justo al este del valle del Éufrates. El grupo occidental estaba situado en las costas sirias del nordeste del Mediterráneo y en el territorio adyacente.

\par
%\textsuperscript{(822.3)}
\textsuperscript{73:1.6} Estos noditas se habían casado frecuentemente con las razas sangiks y habían dejado tras ellos una progenitura capaz. Algunos descendientes de los rebeldes dalamatianos se unieron posteriormente a Van y a sus leales seguidores en las tierras situadas al norte de Mesopotamia. Aquí, en las proximidades del Lago Van y en la región sur del Mar Caspio, los noditas se unieron y se mezclaron con los amadonitas, y fueron contados entre los <<poderosos hombres de la antig\"uedad>>\footnote{\textit{Poderosos hombres de la antig\"uedad}: Gn 6:4.}.

\par
%\textsuperscript{(822.4)}
\textsuperscript{73:1.7} Antes de la llegada de Adán y Eva, estos grupos ---los noditas y los amadonitas--- eran las razas más avanzadas y cultas de la Tierra.

\section*{2. Los proyectos para el Jardín}
\par
%\textsuperscript{(822.5)}
\textsuperscript{73:2.1} Durante cerca de cien años antes de la inspección de Tabamantia, Van y sus asociados habían predicado, desde su sede de ética y de cultura mundial situada en las tierras altas, la venida de un Hijo prometido de Dios, mejorador de la raza, instructor de la verdad y digno sucesor del traidor Caligastia. La mayoría de los habitantes del mundo, en aquellos tiempos, mostró poco o ningún interés por estas predicciones, pero aquellos que estaban en contacto inmediato con Van y Amadón se tomaron en serio estas enseñanzas y empezaron a hacer planes para recibir adecuadamente al Hijo prometido.

\par
%\textsuperscript{(822.6)}
\textsuperscript{73:2.2} Van contó a sus asociados más allegados la historia de los Hijos Materiales de Jerusem, lo que había conocido de ellos antes de venir a Urantia. Sabía muy bien que estos Hijos Adámicos vivían siempre en hogares sencillos pero encantadores rodeados de jardines. Ochenta y tres años antes de la llegada de Adán y Eva, propuso que se dedicaran a proclamar la venida de estos Hijos Materiales y a preparar un hogar jardín para recibirlos.

\par
%\textsuperscript{(822.7)}
\textsuperscript{73:2.3} Desde su cuartel general en las tierras altas y desde sesenta y una colonias muy dispersas, Van y Amadón reclutaron un cuerpo de más de tres mil trabajadores dispuestos y entusiastas; en una asamblea solemne, se comprometieron para esta misión de preparar la llegada del Hijo prometido ---o al menos esperado.

\par
%\textsuperscript{(822.8)}
\textsuperscript{73:2.4} Van dividió a sus voluntarios en cien compañías, con un capitán al mando de cada una de ellas y un asociado que servía en su estado mayor personal como oficial de enlace, reteniendo a Amadón como asociado personal. Todas estas delegaciones empezaron en serio su trabajo preliminar, y la comisión encargada del emplazamiento del Jardín salió a buscar el lugar ideal.

\par
%\textsuperscript{(822.9)}
\textsuperscript{73:2.5} Aunque Caligastia y Daligastia habían sido despojados de una gran parte de su poder para hacer el mal, hicieron todo lo posible por impedir y obstaculizar el trabajo de preparar el Jardín. Pero sus maquinaciones perversas fueron compensadas ampliamente con las fieles actividades de casi diez mil criaturas intermedias leales, que trabajaron infatigablemente para que progresara la empresa.

\section*{3. El emplazamiento del Jardín}
\par
%\textsuperscript{(823.1)}
\textsuperscript{73:3.1} La comisión encargada del emplazamiento estuvo ausente durante cerca de tres años. Realizó un informe favorable sobre tres emplazamientos posibles: El primero era una isla del Golfo Pérsico; el segundo era un emplazamiento fluvial que fue ocupado más tarde por el segundo jardín; y el tercero era una península larga y estrecha ---casi una isla--- que sobresalía hacia el oeste desde las costas orientales del Mar Mediterráneo.

\par
%\textsuperscript{(823.2)}
\textsuperscript{73:3.2} La comisión apoyó casi por unanimidad la tercera solución. Se escogió este lugar, y se tardaron dos años en trasladar la sede cultural del mundo, incluyendo el árbol de la vida, a esta península mediterránea. Todos los habitantes de la península, a excepción de un solo grupo, se marcharon pacíficamente cuando llegaron Van y sus compañeros.

\par
%\textsuperscript{(823.3)}
\textsuperscript{73:3.3} Esta península mediterránea tenía un clima salubre y una temperatura uniforme; este tiempo estable se debía a las montañas que la rodeaban y al hecho de que esta zona era casi una isla en un mar interior. Llovía abundantemente en las tierras altas circundantes, pero rara vez en el propio Edén. Pero cada noche <<se levantaba una niebla>>\footnote{\textit{Se levantaba una niebla}: Gn 2:6.}, procedente de la extensa red de canales artificiales de riego, que refrescaba la vegetación del Jardín.

\par
%\textsuperscript{(823.4)}
\textsuperscript{73:3.4} El litoral de esta masa de tierra estaba considerablemente elevado, y el istmo que la unía al continente sólo tenía cuarenta y tres kilómetros de ancho en el punto más estrecho. El gran río que regaba el Jardín descendía de las tierras más altas de la península, corría hacia el este por el istmo peninsular hasta llegar al continente, y desde allí atravesaba las tierras bajas de Mesopotamia hasta el lejano mar. Estaba alimentado por cuatro afluentes que se originaban en las colinas costeras de la península edénica, y éstas eran las <<cuatro cabeceras>>\footnote{\textit{Cabeceras de los ríos de Edén}: Gn 2:10.} del río que <<salía del Edén>>, y que más tarde se confundieron con los brazos de los ríos que rodeaban al segundo jardín.

\par
%\textsuperscript{(823.5)}
\textsuperscript{73:3.5} Las piedras preciosas y los metales abundaban en las montañas que rodeaban al Jardín, aunque les prestaron muy poca atención. La idea predominante debía ser la glorificación de la horticultura y la exaltación de la agricultura.

\par
%\textsuperscript{(823.6)}
\textsuperscript{73:3.6} El lugar que se escogió para el Jardín era probablemente el paraje más hermoso de este tipo que había en el mundo entero, y el clima era entonces ideal. En ninguna otra parte había un lugar que se pudiera prestar de manera tan perfecta para convertirse en un paraíso semejante de expresión botánica. La flor y nata de la civilización de Urantia se estaba congregando en este lugar de reunión. Fuera de allí y aún más lejos, el mundo vivía en las tinieblas, la ignorancia y el salvajismo. Edén era el único punto luminoso de Urantia; era por naturaleza un sueño de belleza, y pronto se convirtió en un poema donde la gloria de los paisajes era exquisita y perfecta.

\section*{4. El establecimiento del Jardín}
\par
%\textsuperscript{(823.7)}
\textsuperscript{73:4.1} Cuando los Hijos Materiales, los mejoradores biológicos, empiezan su estancia temporal en un mundo evolutivo, su lugar de residencia se llama con frecuencia el Jardín del Edén, porque está caracterizado por la belleza floral y el esplendor botánico de Edentia, la capital de la constelación. Van conocía bien estas costumbres y dispuso en consecuencia que toda la península se consagrara al Jardín. Se hicieron proyectos para el pastoreo y la cría de ganado en las tierras contiguas del continente. En el parque sólo se encontraban, del reino animal, los pájaros y las diversas especies de animales domesticados. Van había ordenado que el Edén debía ser un jardín y sólo un jardín. Nunca se mató a ningún animal dentro de su recinto. Toda la carne que comieron los trabajadores del Jardín durante todos los años que duró su construcción procedía de los rebaños que se custodiaban en el continente.

\par
%\textsuperscript{(824.1)}
\textsuperscript{73:4.2} La primera tarea consistió en construir una muralla de ladrillo a través del istmo de la península. Una vez que se terminó, pudieron emprender sin estorbos el trabajo real de embellecer el paisaje y construir las viviendas.

\par
%\textsuperscript{(824.2)}
\textsuperscript{73:4.3} Se creó un jardín zoológico construyendo una muralla más pequeña justo más allá de la muralla principal; el espacio intermedio, ocupado por todo tipo de bestias salvajes, servía de protección adicional contra los ataques hostiles. Esta casa de fieras estaba organizada en doce grandes divisiones, con caminos amurallados que conducían entre estos grupos hasta las doce puertas del Jardín; el río y sus pastos adyacentes ocupaban la zona central.

\par
%\textsuperscript{(824.3)}
\textsuperscript{73:4.4} Sólo se emplearon trabajadores voluntarios para preparar el Jardín; nunca se contrató a ningún asalariado. Cultivaban el Jardín y cuidaban sus rebaños para poder vivir; también recibían aportaciones de alimentos de los creyentes cercanos. Y esta gran empresa se llevó a buen fin a pesar de las dificultades que la acompañaron debido al estado confuso del mundo durante estos tiempos turbulentos.

\par
%\textsuperscript{(824.4)}
\textsuperscript{73:4.5} Como no sabía cuánto tiempo tardarían en venir el Hijo y la Hija esperados, Van causó una gran desilusión cuando sugirió que también se adiestrara a la joven generación en el trabajo de continuar con la empresa, por si acaso se retrasaba la llegada de estos Hijos. Esta sugerencia pareció una confesión de falta de fe por parte de Van, lo que provocó una inquietud considerable, produciéndose numerosas deserciones; pero Van siguió adelante con su plan de preparación, mientras cubría los puestos de los desertores con otros voluntarios más jóvenes.

\section*{5. El hogar del Jardín}
\par
%\textsuperscript{(824.5)}
\textsuperscript{73:5.1} En el centro de la península edénica se encontraba el exquisito templo de piedra del Padre Universal, el santuario sagrado del Jardín. La sede administrativa se estableció en el norte; las casas para los obreros y sus familias se construyeron en el sur; en el oeste se reservó una parcela de terreno para las escuelas en proyecto del sistema educativo del Hijo esperado, mientras que al <<este del Edén>>\footnote{\textit{Este del Edén}: Gn 2:8; 3:24.} se construyeron las viviendas destinadas al Hijo prometido y a su descendencia inmediata. Los planes arquitectónicos del Edén preveían viviendas y tierras abundantes para un millón de seres humanos.

\par
%\textsuperscript{(824.6)}
\textsuperscript{73:5.2} En el momento de la llegada de Adán sólo se había terminado una cuarta parte del Jardín, pero ya había miles de kilómetros de canales de riego y cerca de veinte mil kilómetros de caminos y carreteras pavimentados. Había un poco más de cinco mil edificios de ladrillo en los diversos sectores, y los árboles y las plantas eran casi innumerables. Cualquier grupo de viviendas del parque no podía contener más de siete casas. Y aunque las estructuras del Jardín eran sencillas, eran muy artísticas. Las carreteras y los caminos estaban bien construidos, y el paisaje era exquisito.

\par
%\textsuperscript{(824.7)}
\textsuperscript{73:5.3} Las disposiciones sanitarias del Jardín eran muy avanzadas con respecto a todo lo que se había intentado hasta entonces en Urantia. En el Edén, el agua para beber se mantenía potable gracias al estricto cumplimiento de los reglamentos sanitarios destinados a conservar su pureza. Durante estos tiempos primitivos, el incumplimiento de estas reglas ocasionaba muchos problemas, pero Van inculcó gradualmente a sus compañeros la importancia de no permitir que cayera nada en el suministro de agua del Jardín.

\par
%\textsuperscript{(825.1)}
\textsuperscript{73:5.4} Antes de la instalación posterior de un sistema de depuración de las aguas residuales, los edenitas practicaron el entierro escrupuloso de todos los residuos o materiales en descomposición. Los inspectores de Amadón hacían su ronda diaria en busca de posibles causas de enfermedades. Los urantianos no han vuelto a tener conciencia de la importancia de la lucha preventiva contra las enfermedades humanas hasta finales del siglo diecinueve y en el siglo veinte. Antes de la desorganización del régimen adámico, se había construido un alcantarillado cubierto de ladrillos que pasaba por debajo de los muros y desembocaba en el río del Edén, aproximadamente un kilómetro y medio más allá del muro exterior o menor del Jardín.

\par
%\textsuperscript{(825.2)}
\textsuperscript{73:5.5} En la época de la llegada de Adán, la mayor parte de las plantas de esta región del mundo crecían en el Edén. Muchos frutos, cereales y nueces ya habían sido mejorados notablemente. Aquí se cultivaron por primera vez muchas legumbres y cereales modernos; pero decenas de variedades de plantas nutritivas se perdieron posteriormente para el mundo.

\par
%\textsuperscript{(825.3)}
\textsuperscript{73:5.6} Aproximadamente el cinco por ciento del Jardín estaba sometido a un cultivo artificial intensivo, el quince por ciento estaba parcialmente cultivado, y el resto se dejó en un estado más o menos natural hasta que llegara Adán, pues se consideraba que era mejor terminar el parque de acuerdo con sus ideas.

\par
%\textsuperscript{(825.4)}
\textsuperscript{73:5.7} Así es como se preparó el Jardín del Edén para recibir al Adán prometido y a su esposa. Este Jardín habría hecho honor a un mundo que hubiera estado bajo una administración perfeccionada y un control normal. Adán y Eva quedaron muy complacidos con el diseño general del Edén, aunque hicieron muchos cambios en el mobiliario de su residencia personal.

\par
%\textsuperscript{(825.5)}
\textsuperscript{73:5.8} Aunque el trabajo de embellecimiento no estaba terminado del todo en el momento de la llegada de Adán, el lugar ya era una joya de belleza botánica; y durante los primeros días de su estancia en el Edén, todo el Jardín tomó una nueva forma y asumió nuevas proporciones de belleza y de esplendor. Urantia no ha albergado nunca, ni antes ni después de esta época, una exposición de horticultura y agricultura tan hermosa y tan completa.

\section*{6. El árbol de la vida}
\par
%\textsuperscript{(825.6)}
\textsuperscript{73:6.1} En el centro del templo del Jardín, Van plantó el árbol de la vida\footnote{\textit{Árbol de la vida}: Gn 2:9; 3:22,24; Ap 2:7; 22:2,14.} que había guardado durante tanto tiempo, cuyas hojas servían para <<curar a las naciones>>\footnote{\textit{Hojas para curar a las naciones}: Ap 22:2.}, y cuyos frutos lo habían sustentado durante tanto tiempo en la Tierra. Van sabía muy bien que Adán y Eva dependerían también de este regalo de Edentia para mantenerse con vida una vez que hubieran aparecido en Urantia con una forma material.

\par
%\textsuperscript{(825.7)}
\textsuperscript{73:6.2} En las capitales de los sistemas, los Hijos Materiales no necesitan el árbol de la vida para subsistir. Sólo dependen de este complemento, para ser físicamente inmortales, cuando se repersonalizan en los planetas.

\par
%\textsuperscript{(825.8)}
\textsuperscript{73:6.3} El <<árbol del conocimiento del bien y del mal>>\footnote{\textit{Árbol del conocimiento del bien}: Gn 2:9,17.} puede ser una figura retórica, una descripción simbólica que abarca una multitud de experiencias humanas, pero el <<árbol de la vida>> no era un mito; era real y estuvo presente durante mucho tiempo en Urantia. Cuando los Altísimos de Edentia aprobaron el nombramiento de Caligastia como Príncipe Planetario de Urantia y el de los cien ciudadanos de Jerusem como su estado mayor administrativo, enviaron al planeta un arbusto de Edentia por medio de los Melquisedeks, y esta planta creció en Urantia hasta convertirse en el árbol de la vida. Esta forma de vida no inteligente es originaria de las esferas sede de las constelaciones y también se encuentra en los mundos sede de los universos locales y de los superuniversos, así como en las esferas de Havona, pero no en las capitales de los sistemas.

\par
%\textsuperscript{(826.1)}
\textsuperscript{73:6.4} Esta superplanta almacenaba ciertas energías espaciales que servían de antídoto contra los elementos que producen la vejez en la existencia animal. El fruto del árbol de la vida se parecía a una batería de acumuladores superquímicos que, cuando se comía, liberaba misteriosamente la fuerza del universo que prolonga la vida. Esta forma de sustento era completamente ineficaz para los seres evolutivos normales de Urantia, pero sí era expresamente útil para los cien miembros materializados del estado mayor de Caligastia y para los cien andonitas modificados que habían contribuído con su plasma vital al estado mayor del Príncipe, y que habían recibido a cambio un complemento de vida que les permitía utilizar el fruto del árbol de la vida para prolongar indefinidamente su existencia que, de otra manera, hubiera sido mortal.

\par
%\textsuperscript{(826.2)}
\textsuperscript{73:6.5} Durante la época del gobierno del Príncipe, el árbol crecía en la tierra del patio circular central del templo del Padre. Cuando estalló la rebelión, Van y sus asociados lo hicieron crecer de nuevo, a partir de su núcleo central, en su campamento provisional. Este arbusto de Edentia fue trasladado posteriormente a su refugio en las tierras altas, donde sirvió a Van y Amadón durante más de ciento cincuenta mil años.

\par
%\textsuperscript{(826.3)}
\textsuperscript{73:6.6} Cuando Van y sus asociados prepararon el Jardín para Adán y Eva, trasplantaron el árbol de Edentia al Jardín del Edén, donde creció una vez más en el patio circular central de otro templo del Padre. Adán y Eva comían periódicamente su fruto para mantener su forma dual de vida física.

\par
%\textsuperscript{(826.4)}
\textsuperscript{73:6.7} Cuando los planes del Hijo Material se desviaron del camino recto, Adán y su familia no obtuvieron la autorización de llevarse del Jardín el núcleo del árbol. Cuando los noditas invadieron el Edén, les contaron que se volverían como <<dioses si comían el fruto del árbol>>\footnote{\textit{Si comían el fruto se volverían dioses}: Gn 3:1-5.}. Para gran sorpresa suya, lo encontraron sin protección. Durante años comieron abundantemente su fruto, pero no les produjo ningún efecto; todos eran mortales materiales del reino; carecían del factor que actuaba como complemento del fruto del árbol. Su incapacidad para beneficiarse del árbol de la vida los enfureció, y durante una de sus guerras internas, tanto el templo como el árbol quedaron destruidos por el fuego; sólo permaneció de pie la muralla de piedra, hasta que posteriormente se sumergió el Jardín. Éste fue el segundo templo del Padre que se destruyó.

\par
%\textsuperscript{(826.5)}
\textsuperscript{73:6.8} Y ahora, todos los seres de Urantia han de seguir el curso natural de la vida y la muerte. Adán, Eva, sus hijos y los hijos de sus hijos, así como sus asociados, todos murieron con el transcurso del tiempo, quedando así sometidos al plan de ascensión del universo local, en el que la resurrección en los mundos de las mansiones sigue a la muerte física.

\section*{7. El destino del Edén}
\par
%\textsuperscript{(826.6)}
\textsuperscript{73:7.1} Después de que Adán se marchara del primer jardín, éste fue ocupado de manera diversa por los noditas, cutitas y suntitas. Más tarde se convirtió en el lugar de residencia de los noditas del norte, que se oponían a cooperar con los adamitas. Después de que Adán dejara el Jardín, estos noditas inferiores ocuparon la península durante cerca de cuatro mil años; entonces, en combinación con una violenta actividad de los volcanes circundantes y la sumersión del puente terrestre que unía Sicilia con África, el fondo oriental del Mar Mediterráneo se hundió, arrastrando bajo las aguas a toda la península edénica. Al mismo tiempo que se producía esta extensa sumersión, la costa oriental del Mediterráneo se elevó considerablemente. Y éste fue el final de la creación natural más hermosa que Urantia haya albergado jamás. El hundimiento no fue repentino, sino que se necesitaron varios cientos de años para que toda la península se sumergiera por completo.

\par
%\textsuperscript{(827.1)}
\textsuperscript{73:7.2} No podemos considerar de ninguna manera esta desaparición del Jardín como una consecuencia del aborto de los planes divinos, o como resultado de los errores de Adán y Eva. Consideramos que la sumersión del Edén no fue más que un acontecimiento natural, pero nos parece que el hundimiento del Jardín fue calculado para que se produjera aproximadamente en el momento en que la acumulación de las reservas de la raza violeta eran suficientes para emprender la tarea de rehabilitar los pueblos del mundo.

\par
%\textsuperscript{(827.2)}
\textsuperscript{73:7.3} Los Melquisedeks aconsejaron a Adán que no iniciara el programa de mejoramiento y mezcla de las razas hasta que su propia familia no contara con medio millón de miembros. Nunca se tuvo la intención de que el Jardín fuera el hogar permanente de los adamitas. Tenían que convertirse en los emisarios de una nueva vida para el mundo entero; tenían que movilizarse para llevar a cabo una donación desinteresada a las razas necesitadas de la Tierra.

\par
%\textsuperscript{(827.3)}
\textsuperscript{73:7.4} Las instrucciones que los Melquisedeks dieron a Adán implicaban que debería establecer unos centros raciales, continentales y divisionarios que estarían a cargo de sus hijos e hijas inmediatos, mientras que él y Eva tendrían que repartir su tiempo entre estas diversas capitales del mundo como consejeros y coordinadores del ministerio mundial para el mejoramiento biológico, el progreso intelectual y la rehabilitación moral.

\par
%\textsuperscript{(827.4)}
\textsuperscript{73:7.5} [Presentado por Solonia, la <<voz seráfica en el Jardín>>\footnote{\textit{Voz del Jardín}: Gn 3:8-13.}.]


\chapter{Documento 74. Adán y Eva}
\par
%\textsuperscript{(828.1)}
\textsuperscript{74:0.1} ADÁN y Eva llegaron a Urantia 37.848 años antes del año 1934 de la era cristiana. Llegaron a mediados de la temporada en la que el Jardín estaba en plena floración. A las doce en punto del mediodía, y sin ser anunciados, los dos transportes seráficos, acompañados del personal de Jerusem encargado de trasladar a los mejoradores biológicos hasta Urantia, se posaron suavemente en la superficie del planeta en rotación en las proximidades del templo del Padre Universal. Todo el trabajo de rematerialización de los cuerpos de Adán y Eva se llevó a cabo dentro del recinto de este santuario recién creado. Desde el momento de su llegada, transcurrieron diez días antes de que fueran recreados con una forma humana dual, para ser presentados como los nuevos dirigentes del mundo. Recuperaron la conciencia de manera simultánea. Los Hijos e Hijas Materiales siempre sirven juntos. En todo tiempo y lugar, la esencia de su servicio consiste en no estar nunca separados. Están destinados a trabajar en parejas; rara vez ejercen su actividad a solas\footnote{\textit{Creación del hombre}: Gn 1:26-27; 2:7,21-24; 5:1-2; 6:7; Is 45:12; Dt 4:32; Mal 2:10.}.

\section*{1. Adán y Eva en Jerusem}
\par
%\textsuperscript{(828.2)}
\textsuperscript{74:1.1} El Adán y la Eva planetarios de Urantia eran miembros del cuerpo decano de Hijos Materiales de Jerusem; y figuraban inscritos conjuntamente con el número 14.311. Pertenecían a la tercera serie física y medían unos dos metros y medio de altura.

\par
%\textsuperscript{(828.3)}
\textsuperscript{74:1.2} En la época en que fue escogido para para venir a Urantia, Adán estaba trabajando con su cónyuge en los laboratorios de pruebas y ensayos físicos de Jerusem. Llevaban más de quince mil años como directores del departamento de energía experimental aplicada a la modificación de las formas vivientes. Mucho tiempo antes de esto, habían sido instructores en las escuelas de ciudadanía para los recién llegados a Jerusem. Todo esto debe tenerse presente en la memoria en relación con la narración de su conducta posterior en Urantia.

\par
%\textsuperscript{(828.4)}
\textsuperscript{74:1.3} Cuando se emitió la proclamación que pedía voluntarios para la misión de la aventura adámica en Urantia, todo el cuerpo decano de Hijos e Hijas Materiales se ofreció como voluntario. Los examinadores Melquisedeks, con la aprobación de Lanaforge y los Altísimos de Edentia, eligieron finalmente al Adán y la Eva que posteriormente vinieron a ejercer sus funciones como mejoradores biológicos en Urantia.

\par
%\textsuperscript{(828.5)}
\textsuperscript{74:1.4} Adán y Eva habían permanecido leales a Miguel durante la rebelión de Lucifer; sin embargo, la pareja fue convocada ante el Soberano del Sistema y todo su gabinete para ser examinada y recibir instrucciones. Les dieron a conocer en detalle todos los asuntos de Urantia; les informaron minuciosamente de los planes que debían seguir al aceptar la responsabilidad de gobernar un mundo tan desgarrado por los conflictos. Prestaron un juramento conjunto de lealtad a los Altísimos de Edentia y a Miguel de Salvington. Se les advirtió debidamente que se consideraran sometidos al cuerpo de los síndicos Melquisedeks de Urantia, hasta que este órgano gobernante estimara oportuno renunciar al mando del mundo donde habían sido asignados.

\par
%\textsuperscript{(829.1)}
\textsuperscript{74:1.5} Esta pareja de Jerusem dejó tras ella, en la capital de Satania y en otras partes, a cien descendientes ---cincuenta hijos y cincuenta hijas---, unas criaturas magníficas que habían evitado los escollos de la evolución y que estaban todas en servicio activo como fieles administradores de confianza del universo en el momento en que sus padres partieron para Urantia. Todos estaban presentes en el hermoso templo de los Hijos Materiales para asistir a los actos de despedida asociados con las últimas ceremonias de aceptación de la donación. Estos hijos acompañaron a sus padres a la sede de desmaterialización de su orden, y fueron los últimos en despedirse de ellos y en desearles un éxito divino, mientras se quedaban dormidos durante la pérdida de conciencia de la personalidad que precede a la preparación para el transporte seráfico. Los hijos pasaron algún tiempo juntos en reunión familiar, regocijándose de que sus padres fueran a convertirse pronto en los jefes visibles, en realidad en los únicos gobernantes, del planeta 606 del sistema de Satania.

\par
%\textsuperscript{(829.2)}
\textsuperscript{74:1.6} Así es como Adán y Eva dejaron Jerusem en medio de las aclamaciones y los buenos deseos de sus ciudadanos. Partieron hacia sus nuevas responsabilidades debidamente equipados y plenamente instruidos de todos los deberes y peligros que encontrarían en Urantia.

\section*{2. La llegada de Adán y Eva}
\par
%\textsuperscript{(829.3)}
\textsuperscript{74:2.1} Adán y Eva se quedaron dormidos en Jerusem y cuando despertaron en el templo del Padre, en Urantia, en presencia de la gran multitud reunida para darles la bienvenida, se encontraron delante de dos seres de los que habían oído hablar mucho: Van y su fiel asociado Amadón. Estos dos héroes de la secesión de Caligastia fueron los primeros en darles la bienvenida a su nuevo hogar jardín.

\par
%\textsuperscript{(829.4)}
\textsuperscript{74:2.2} El idioma del Edén era el dialecto andónico que hablaba Amadón. Van y Amadón habían mejorado notablemente esta lengua creando un nuevo alfabeto de veinticuatro letras, y esperaban que se convertiría en el idioma de Urantia a medida que la cultura del Edén se extendiera por el mundo. Adán y Eva habían adquirido el pleno dominio de este dialecto humano antes de salir de Jerusem, de manera que este hijo de Andón oyó al eminente gobernante de su mundo dirigirse a él en su propia lengua.

\par
%\textsuperscript{(829.5)}
\textsuperscript{74:2.3} Aquel día hubo una gran animación y alegría en todo el Edén, mientras que los corredores se apresuraban en llegar al lugar donde se encontraban las palomas mensajeras reunidas de todas partes, exclamando: <<Soltad las palomas; que lleven la noticia de que el Hijo prometido ha llegado.>> Año tras año, cientos de colonias de creyentes habían mantenido fielmente la cantidad necesaria de palomas criadas en sus hogares precísamente para esta ocasión.

\par
%\textsuperscript{(829.6)}
\textsuperscript{74:2.4} A medida que la noticia de la llegada de Adán se difundía por todas partes, miles de miembros de las tribus cercanas aceptaron las enseñanzas de Van y Amadón, y durante muchos meses, los peregrinos continuaron llegando en masa al Edén para dar la bienvenida a Adán y Eva y rendir homenaje a su Padre invisible.

\par
%\textsuperscript{(829.7)}
\textsuperscript{74:2.5} Poco después de despertarse, Adán y Eva fueron escoltados hasta la recepción oficial en el gran montículo situado al norte del templo. Esta colina natural había sido ampliada y preparada para la instalación de los nuevos dirigentes del mundo. Es aquí donde, a mediodía, el comité de recepción de Urantia dio la bienvenida a este Hijo y a esta Hija del sistema de Satania. Amadón era el presidente de este comité, que estaba compuesto por los doce miembros siguientes: un representante de cada una de las seis razas sangiks; el jefe en ejercicio de los intermedios; Annán, una hija leal y portavoz de los noditas; Noé, el hijo del arquitecto y constructor del Jardín, y ejecutor de los proyectos de su padre fallecido; y los dos Portadores de Vida residentes.

\par
%\textsuperscript{(830.1)}
\textsuperscript{74:2.6} Durante el acto siguiente, el Melquisedek decano, jefe del consejo de los síndicos de Urantia, entregó la responsabilidad de la custodia del planeta a Adán y Eva. El Hijo y la Hija Materiales prestaron juramento de fidelidad a los Altísimos de Norlatiadek y a Miguel de Nebadon, y Van los proclamó gobernadores de Urantia, renunciando así a la autoridad nominal que había tenido durante más de ciento cincuenta mil años en virtud de una decisión de los síndicos Melquisedeks.

\par
%\textsuperscript{(830.2)}
\textsuperscript{74:2.7} Adán y Eva fueron revestidos con túnicas reales en esta ocasión, la de su instalación oficial como gobernadores del planeta. No todas las artes de Dalamatia se habían perdido en el mundo; la tejeduría aún se practicaba en la época del Edén.

\par
%\textsuperscript{(830.3)}
\textsuperscript{74:2.8} Entonces se escuchó la proclamación de los arcángeles, y la voz transmitida de Gabriel ordenó que se pasara lista para el segundo juicio de Urantia y la resurrección de los supervivientes dormidos de la segunda dispensación de gracia y misericordia del planeta 606 de Satania. La dispensación del Príncipe ha pasado; la era de Adán, la tercera época planetaria, se inicia en medio de unas escenas de sencilla grandiosidad; y los nuevos dirigentes de Urantia empiezan su reinado en unas condiciones aparentemente favorables, a pesar de la confusión mundial ocasionada por la falta de cooperación de su predecesor en autoridad en el planeta.

\section*{3. Adán y Eva se informan sobre el planeta}
\par
%\textsuperscript{(830.4)}
\textsuperscript{74:3.1} Ahora, después de su instalación oficial, Adán y Eva se dieron terriblemente cuenta de su aislamiento planetario. Las transmisiones que les eran familiares estaban silenciosas, y todos los circuitos de comunicación extraplanetaria estaban ausentes. Sus compañeros de Jerusem habían ido a unos planetas donde todo marchaba bien, con un Príncipe Planetario bien establecido y un estado mayor experimentado dispuesto a recibirlos y calificado para cooperar con ellos durante su experiencia inicial en esos mundos. Pero en Urantia la rebelión lo había cambiado todo. Aquí, la presencia del Príncipe Planetario se notaba demasiado, y aunque estaba privado de la mayor parte de su poder para hacer el mal, continuaba siendo capaz de dificultar la tarea de Adán y Eva, y de hacerla hasta cierto punto arriesgada. Aquella noche, mientras se paseaban por el Jardín bajo la luz de la Luna llena, hablando de los planes para el día siguiente, el Hijo y la Hija de Jerusem estaban serios y desilusionados.

\par
%\textsuperscript{(830.5)}
\textsuperscript{74:3.2} Así es como terminó el primer día de Adán y Eva en la aislada Urantia, el planeta confundido por la traición de Caligastia; pasearon y conversaron hasta muy avanzada la noche, su primera noche en la Tierra ---y se sintieron muy solos.

\par
%\textsuperscript{(830.6)}
\textsuperscript{74:3.3} Adán pasó su segundo día en la Tierra reunido con los síndicos planetarios y el consejo consultivo. Los Melquisedeks y sus asociados enseñaron a Adán y Eva más detalles acerca de la rebelión de Caligastia y el efecto de esta sublevación sobre el progreso del mundo. Este largo relato sobre la mala administración de los asuntos del planeta fue, en conjunto, una historia desalentadora. Se enteraron de todos los hechos relacionados con el derrumbamiento total de los planes de Caligastia para acelerar el proceso de la evolución social. También llegaron a darse cuenta plenamente de que es una locura intentar conseguir el avance planetario independientemente del plan divino de la evolución. Y así es como terminó un día triste pero instructivo ---su segundo día en Urantia.

\par
%\textsuperscript{(831.1)}
\textsuperscript{74:3.4} El tercer día lo dedicaron a inspeccionar el Jardín. Desde las grandes aves de pasajeros ---los fándores--- Adán y Eva contemplaron las inmensas extensiones del Jardín mientras surcaban los aires por encima del paraje más hermoso de la Tierra. Este día de inspección terminó con un enorme banquete en honor de todos los que habían trabajado para crear este jardín de una belleza y una grandiosidad edénicas. Y una vez más, el Hijo y su compañera se pasearon por el Jardín hasta horas avanzadas de la noche de su tercer día, y hablaron de la inmensidad de sus problemas.

\par
%\textsuperscript{(831.2)}
\textsuperscript{74:3.5} El cuarto día, Adán y Eva pronunciaron un discurso ante la asamblea del Jardín. Desde el montículo inaugural, hablaron al pueblo acerca de sus planes para rehabilitar el mundo y esbozaron los métodos que emplearían para tratar de rescatar la cultura social de Urantia de los bajos niveles en los que había caído a consecuencia del pecado y la rebelión. Fue un gran día, y concluyó con un banquete para el consejo de los hombres y las mujeres que habían sido seleccionados para asumir sus responsabilidades en la nueva administración de los asuntos del mundo. ¡Prestad atención! En este grupo había tanto mujeres como hombres, y era la primera vez que ocurría una cosa así en la Tierra desde los tiempos de Dalamatia. Fue una innovación asombrosa contemplar a Eva, una mujer, compartir con un hombre los honores y las responsabilidades de los asuntos del mundo. Así es como terminó el cuarto día en la Tierra.

\par
%\textsuperscript{(831.3)}
\textsuperscript{74:3.6} El quinto día se ocuparon de la organización del gobierno provisional, la administración que debería funcionar hasta que los síndicos Melquisedeks se marcharan de Urantia.

\par
%\textsuperscript{(831.4)}
\textsuperscript{74:3.7} El sexto día lo dedicaron a inspeccionar los numerosos tipos de hombres y de animales. Adán y Eva fueron acompañados todo el día a lo largo de las murallas orientales del Edén, observando la vida animal del planeta y llegando a comprender mejor lo que había que hacer para poner orden en la confusión de un mundo habitado por tal variedad de criaturas vivientes.

\par
%\textsuperscript{(831.5)}
\textsuperscript{74:3.8} Los que lo acompañaban en esta excursión se quedaron enormemente sorprendidos al observar que Adán comprendía plenamente la naturaleza y la función de los miles y miles de animales que le mostraban\footnote{\textit{Inspección de animales}: Gn 2:19-20.}. En cuanto echaba una ojeada a un animal, indicaba su naturaleza y su comportamiento. Adán podía, a primera vista, ponerles nombres que describían su origen, su naturaleza y su función a todas las criaturas materiales que veía. Aquellos que lo conducían en esta visita de inspección no sabían que el nuevo gobernante del mundo era uno de los anatomistas más expertos de toda Satania; y Eva era igual de versada. Adán asombró a sus asociados cuando les describió una multitud de seres vivientes demasiado pequeños para ser vistos por los ojos humanos.

\par
%\textsuperscript{(831.6)}
\textsuperscript{74:3.9} Cuando el sexto día de su estancia en la Tierra concluyó, Adán y Eva descansaron por primera vez\footnote{\textit{Día de descanso}: Gn 2:2.} en su nuevo hogar <<al este del Edén>>\footnote{\textit{Este del Edén}: Gn 2:8; 3:23-24.}. Los primeros seis días de la aventura de Urantia habían sido muy atareados, y estaban deseando con gran placer pasar un día entero desprovisto de toda actividad.

\par
%\textsuperscript{(831.7)}
\textsuperscript{74:3.10} Pero las circunstancias dispusieron las cosas de otra manera. La experiencia del día anterior en la que Adán había analizado con tanta inteligencia y minuciosidad la vida animal de Urantia, unida a su magistral discurso inaugural y a sus modales encantadores, habían conquistado el corazón y subyugado el intelecto de los habitantes del Jardín de tal manera, que no sólo estaban sinceramente decididos a aceptar como gobernantes al Hijo y a la Hija recién llegados de Jerusem, sino que la mayoría estaba casi dispuesta a postrarse y adorarlos como si fueran dioses.

\section*{4. El primer disturbio}
\par
%\textsuperscript{(832.1)}
\textsuperscript{74:4.1} Aquella noche, la noche que siguió al sexto día, mientras Adán y Eva dormían, se estaban produciendo cosas extrañas en las proximidades del templo del Padre, en el sector central del Edén. Allí, bajo la suave luz de la Luna, cientos de hombres y mujeres entusiastas y excitados escucharon durante horas los alegatos apasionados de sus dirigentes. Tenían buenas intenciones, pero simplemente no podían comprender la sencillez de los modales fraternales y democráticos de sus nuevos gobernantes. Mucho antes del amanecer, los nuevos administradores provisionales de los asuntos del mundo llegaron a la conclusión casi unánime de que Adán y su compañera eran demasiado modestos y recatados. Determinaron que la Divinidad había descendido a la Tierra en forma corporal, que Adán y Eva eran dioses en realidad, o estaban tan cerca de serlo, que eran dignos de una adoración reverente.

\par
%\textsuperscript{(832.2)}
\textsuperscript{74:4.2} Los asombrosos acontecimientos de los seis primeros días de Adán y Eva en la Tierra sobrepasaban por completo las mentes no preparadas de los hombres del mundo, incluso de los mejores. La cabeza les daba vueltas; estaban entusiasmados con la proposición de llevar al mediodía a la noble pareja hasta el templo del Padre, para que todos pudieran inclinarse en respetuosa adoración y postrarse en humilde sumisión. Y los habitantes del Jardín eran realmente sinceros al hacer todo esto.

\par
%\textsuperscript{(832.3)}
\textsuperscript{74:4.3} Van protestó. Amadón se encontraba ausente, pues estaba encargado de la guardia de honor que había permanecido con Adán y Eva durante toda la noche. Pero la protesta de Van fue rechazada. Le dijeron que él era también demasiado modesto, demasiado recatado; que él mismo no estaba lejos de ser un dios, o si no, ¿cómo había vivido tanto tiempo en la Tierra, y cómo había llevado a cabo un acontecimiento tan importante como la venida de Adán? Cuando los excitados edenitas estaban a punto de cogerlo y subirlo al montículo para adorarlo, Van se alejó abriéndose paso entre la multitud, y como podía comunicarse con los intermedios, envió a su jefe a toda prisa para que fuera a ver a Adán.

\par
%\textsuperscript{(832.4)}
\textsuperscript{74:4.4} Se acercaba el amanecer de su séptimo día en la Tierra cuando Adán y Eva escucharon la sorprendente noticia de la proposición de aquellos mortales bienintencionados, pero descaminados. Entonces, mientras las aves de pasajeros se acercaban velozmente para llevarlos al templo, los intermedios, que son capaces de hacer estas cosas, transportaron a Adán y Eva hasta el templo del Padre. Este séptimo día por la mañana temprano, desde el montículo donde habían sido recibidos tan recientemente, Adán ofreció una explicación de las órdenes de filiación divina e indicó claramente a estas mentes terrenales que sólo se debe adorar al Padre y a aquellos que él designe. Adán manifestó con claridad que aceptaría cualquier honor y recibiría todo tipo de respetos, pero que nunca consentiría la adoración.

\par
%\textsuperscript{(832.5)}
\textsuperscript{74:4.5} Fue un día de gran importancia. Poco antes del mediodía, casi en el momento en que llegaba un mensajero seráfico trayendo de Jerusem el reconocimiento de la instalación de los gobernantes del mundo, Adán y Eva se apartaron de la multitud, señalaron el templo del Padre, y dijeron: <<Id ahora hacia el símbolo material de la presencia invisible del Padre, e inclinaos para adorar a Aquel que nos ha creado a todos y nos mantiene con vida. Que este acto sea la promesa sincera de que nunca más tendréis la tentación de adorar a otro que no sea Dios.>> Todos hicieron lo que Adán les había ordenado. El Hijo y la Hija Materiales permanecieron solos en el montículo, con la cabeza inclinada, mientras que el pueblo se postraba alrededor del templo.

\par
%\textsuperscript{(832.6)}
\textsuperscript{74:4.6} Así es como se originó la tradición del día del sábado\footnote{\textit{Tradición del sábado}: Gn 2:3; Ex 20:8-11; Dt 5:12-15.}. El séptimo día siempre se dedicó, en el Edén, a la asamblea del mediodía en el templo; la costumbre de consagrar este día a la cultura personal subsistió durante mucho tiempo. La mañana se dedicaba al mejoramiento físico, el mediodía al culto espiritual, la tarde a la cultura de la mente, mientras que el anochecer se pasaba en celebraciones sociales. Esto nunca fue una ley en el Edén, pero tuvieron la costumbre de hacerlo mientras la administración adámica gobernó en la Tierra.

\section*{5. La administración de Adán}
\par
%\textsuperscript{(833.1)}
\textsuperscript{74:5.1} Los síndicos Melquisedeks permanecieron de servicio durante cerca de siete años después de la llegada de Adán, pero finalmente llegó el momento en que entregaron la administración de los asuntos del mundo a Adán y regresaron a Jerusem.

\par
%\textsuperscript{(833.2)}
\textsuperscript{74:5.2} La despedida de los síndicos ocupó un día entero; durante el anochecer, cada Melquisedek dio a Adán y Eva sus consejos de despedida y les expresó sus mejores deseos. Adán había pedido varias veces a sus consejeros que permanecieran con él en la Tierra, pero estas peticiones siempre fueron denegadas. Había llegado el momento en que los Hijos Materiales tenían que asumir la plena responsabilidad de la conducta de los asuntos del mundo. Así pues, los transportes seráficos de Satania partieron del planeta a medianoche con catorce seres hacia Jerusem, ya que el traslado de Van y Amadón se produjo al mismo tiempo que la partida de los doce Melquisedeks.

\par
%\textsuperscript{(833.3)}
\textsuperscript{74:5.3} Todo marchó bastante bien en Urantia durante algún tiempo, y parecía que Adán podría desarrollar finalmente algún plan para promover la expansión gradual de la civilización edénica. Siguiendo los consejos de los Melquisedeks, empezó fomentando las artes de la manufactura con la idea de desarrollar las relaciones comerciales con el mundo exterior. Cuando el Edén se desorganizó, más de cien instalaciones manufactureras primitivas estaban en funcionamiento, y se habían establecido amplias relaciones comerciales con las tribus cercanas.

\par
%\textsuperscript{(833.4)}
\textsuperscript{74:5.4} Durante miles de años, a Adán y Eva les habían enseñado la técnica de mejorar un mundo y de prepararlo para recibir sus contribuciones especializadas para el avance de la civilización evolutiva. Pero ahora tenían que hacer frente a unos problemas apremiantes, tales como el establecimiento del orden público en un mundo de salvajes, bárbaros y seres humanos semicivilizados. Aparte de la flor y nata de la población de la Tierra congregada en el Jardín, sólo unos pocos grupos dispersos estaban algo preparados para recibir la cultura adámica.

\par
%\textsuperscript{(833.5)}
\textsuperscript{74:5.5} Adán realizó un esfuerzo heróico y decidido para establecer un gobierno mundial, pero se encontró a cada paso con una resistencia obstinada. Adán ya había puesto en funcionamiento un sistema de control colectivo en todo el Edén, y había federado todos estos grupos en una liga edénica. Pero cuando salió del Jardín y trató de aplicar estas ideas a las tribus exteriores, se produjeron problemas, unos problemas muy graves. En cuanto los asociados de Adán empezaron a trabajar fuera del Jardín, se encontraron con la resistencia directa y bien organizada de Caligastia y Daligastia. El Príncipe caído había sido depuesto como gobernante del mundo, pero no había sido retirado del planeta. Continuaba estando presente en la Tierra y con el poder de oponerse, al menos hasta cierto punto, a todos los planes de Adán para rehabilitar la sociedad humana. Adán intentó prevenir a las razas contra Caligastia, pero la tarea era muy difícil porque su enemigo acérrimo era invisible para los ojos de los mortales.

\par
%\textsuperscript{(833.6)}
\textsuperscript{74:5.6} Incluso entre los edenitas había mentes confusas que se inclinaban hacia la enseñanza de Caligastia sobre la libertad personal desenfrenada, y causaron a Adán unos problemas sin fin; siempre estaban desbaratando los planes mejor preparados para un progreso ordenado y un desarrollo sustancial. Finalmente, Adán se vio obligado a renunciar a su programa destinado a la socialización inmediata, y volvió al método de organización de Van, dividiendo a los edenitas en compañías de cien miembros, con un capitán para cada una de ellas y un teniente encargado de cada grupo de diez.

\par
%\textsuperscript{(834.1)}
\textsuperscript{74:5.7} Adán y Eva habían venido para establecer un gobierno representativo en lugar de un gobierno monárquico, pero no encontraron ningún gobierno digno de este nombre en toda la faz de la Tierra. Por el momento, Adán abandonó todo esfuerzo por establecer un gobierno representativo, y antes del derrumbamiento del régimen edénico, logró establecer cerca de un centenar de centros comerciales y sociales alejados, donde unos representantes enérgicos gobernaban en su nombre. La mayoría de estos centros habían sido organizados anteriormente por Van y Amadón.

\par
%\textsuperscript{(834.2)}
\textsuperscript{74:5.8} El envío de embajadores de una tribu a otra data de los tiempos de Adán. Fue un gran paso hacia adelante en la evolución del gobierno.

\section*{6. La vida familiar de Adán y Eva}
\par
%\textsuperscript{(834.3)}
\textsuperscript{74:6.1} Las tierras de la familia adámica abarcaban poco más de mil trescientas hectáreas. En los alrededores inmediatos de este domicilio familiar se habían tomado disposiciones para cuidar de más de trescientos mil descendientes en línea directa. Pero sólo se construyó la primera unidad de los edifícios en proyecto. Antes de que la familia adámica hubiera crecido más allá de estas previsiones, todo el plan edénico se había desbaratado y el Jardín había sido desocupado.

\par
%\textsuperscript{(834.4)}
\textsuperscript{74:6.2} Adanson fue el primogénito de la raza violeta de Urantia, seguido de una hermana y luego de Evason, el segundo hijo de Adán y Eva. Antes de que se marcharan los Melquisedeks, Eva era madre de cinco hijos ---tres niños y dos niñas. Los dos siguientes fueron gemelos. Antes de la falta, había tenido sesenta y tres hijos, treinta y dos hembras y treinta y un varones. Cuando Adán y Eva dejaron el Jardín, su familia constaba de cuatro generaciones que ascendían a 1.647 descendientes en línea directa. Tuvieron cuarenta y dos hijos después de abandonar el Jardín, además de los dos descendientes de linaje conjunto con la estirpe mortal de la Tierra. Estas cifras no incluyen la descendencia adámica entre los noditas y las razas evolutivas.

\par
%\textsuperscript{(834.5)}
\textsuperscript{74:6.3} Los hijos de Adán no tomaban leche animal cuando dejaban de alimentarse con el pecho de su madre a la edad de un año. Eva tenía acceso a la leche de una gran variedad de nueces y a los jugos de numerosas frutas, y como conocía perfectamente la química y la energía de estos alimentos, los combinaba adecuadamente para alimentar a sus hijos hasta la aparición de los dientes.

\par
%\textsuperscript{(834.6)}
\textsuperscript{74:6.4} Aunque la cocción se empleaba de manera universal fuera del sector adámico cercano al Edén, en el hogar de Adán no se cocinaba nada. Encontraban sus alimentos ya preparados ---frutas, nueces y cereales--- a medida que maduraban. Comían una vez al día, poco después del mediodía. Adán y Eva también absorbían directamente <<luz y energía>> de ciertas emanaciones espaciales en conjunción con el ministerio del árbol de la vida.

\par
%\textsuperscript{(834.7)}
\textsuperscript{74:6.5} Los cuerpos de Adán y Eva despedían una luz tenue, pero siempre se vestían de acuerdo con la costumbre de sus asociados. Aunque llevaban poca ropa durante el día, al anochecer se ponían unas mantas. El origen de la aureola tradicional que rodea la cabeza de los hombres supuestamente piadosos y santos data de los tiempos de Adán y Eva. Puesto que los vestidos ocultaban una gran parte de las emanaciones luminosas de sus cuerpos, sólo se percibía el resplandor que irradiaban sus cabezas. Los descendientes de Adanson siempre describieron de esta manera su concepto de las personas que se creía que tenían un desarrollo espiritual extraordinario.

\par
%\textsuperscript{(834.8)}
\textsuperscript{74:6.6} Adán y Eva podían comunicarse el uno con el otro, y con sus hijos directos, hasta una distancia de unos ochenta kilómetros. Este intercambio de pensamientos se efectuaba mediante las delicadas cavidades de gas situadas muy cerca de sus estructuras cerebrales. Por medio de este mecanismo podían enviar y recibir las vibraciones del pensamiento. Pero este poder se interrumpió instantáneamente en cuanto abandonaron su mente a la discordia y a los trastornos del mal.

\par
%\textsuperscript{(835.1)}
\textsuperscript{74:6.7} Los hijos de Adán asistían a sus propias escuelas hasta que cumplían los dieciséis años, y los mayores enseñaban a los más jóvenes. Los pequeños cambiaban de actividad cada treinta minutos, y los más grandes cada hora. Fue sin duda un espectáculo nuevo en Urantia observar cómo jugaban estos hijos de Adán y Eva, realizando unas actividades alegres y estimulantes por la pura diversión de hacerlas. Los juegos y el humor de las razas actuales proceden en gran parte de la estirpe adámica. Todos los adamitas apreciaban mucho la música y tenían también un agudo sentido del humor.

\par
%\textsuperscript{(835.2)}
\textsuperscript{74:6.8} La edad media para prometerse en matrimonio era a los dieciocho años, y estos jóvenes empezaban entonces un curso de formación de dos años que los preparaba para asumir las responsabilidades matrimoniales. A los veinte años tenían derecho a casarse, y después de hacerlo empezaban el trabajo de su vida o iniciaban una preparación especial para el mismo.

\par
%\textsuperscript{(835.3)}
\textsuperscript{74:6.9} La costumbre que tuvieron algunas naciones posteriores de permitir que en las familias reales, supuestamente descendientes de los dioses, los hermanos se casaran con las hermanas, data de las tradiciones de los hijos de Adán ---que no tenían más remedio que casarse entre ellos. Adán y Eva siempre celebraron las ceremonias matrimoniales de la primera y segunda generación del Jardín.

\section*{7. La vida en el Jardín}
\par
%\textsuperscript{(835.4)}
\textsuperscript{74:7.1} Los hijos de Adán vivían y trabajaban <<al este del Edén>>\footnote{\textit{Este del Edén}: Gn 2:8; 3:23-24.}, excepto durante los cuatro años que asistían a las escuelas del oeste. Recibían una formación intelectual según los métodos de las escuelas de Jerusem hasta que tenían dieciséis años. Desde los dieciséis hasta los veinte se instruían en las escuelas de Urantia al otro extremo del Jardín, donde también ejercían como profesores en los cursos inferiores.

\par
%\textsuperscript{(835.5)}
\textsuperscript{74:7.2} La \textit{adaptación a la sociedad} era el único objetivo que tenía el sistema escolar del oeste del Jardín. Los períodos de recreo matinales se dedicaban a la horticultura y la agricultura prácticas, y los de la tarde a los juegos competitivos. El anochecer se empleaba para las relaciones sociales y el cultivo de las amistades personales. La educación religiosa y sexual se consideraba que incumbía al hogar, que era un deber de los padres.

\par
%\textsuperscript{(835.6)}
\textsuperscript{74:7.3} La enseñanza en estas escuelas incluía una formación acerca de:

\par
%\textsuperscript{(835.7)}
\textsuperscript{74:7.4} 1. La salud y el cuidado del cuerpo.

\par
%\textsuperscript{(835.8)}
\textsuperscript{74:7.5} 2. La regla de oro, la norma para las relaciones sociales.

\par
%\textsuperscript{(835.9)}
\textsuperscript{74:7.6} 3. La relación de los derechos individuales con los derechos colectivos y las obligaciones comunitarias.

\par
%\textsuperscript{(835.10)}
\textsuperscript{74:7.7} 4. La historia y la cultura de las diversas razas de la Tierra.

\par
%\textsuperscript{(835.11)}
\textsuperscript{74:7.8} 5. Los métodos para hacer progresar y mejorar el comercio mundial.

\par
%\textsuperscript{(835.12)}
\textsuperscript{74:7.9} 6. La coordinación de los deberes y las emociones en conflicto.

\par
%\textsuperscript{(835.13)}
\textsuperscript{74:7.10} 7. El cultivo de los juegos, el humor y los sustitutos competitivos de las luchas físicas.

\par
%\textsuperscript{(835.14)}
\textsuperscript{74:7.11} Las escuelas, y de hecho todas las actividades del Jardín, siempre estaban abiertas para los visitantes. Los observadores sin armas eran admitidos libremente en el Edén durante cortas visitas. Para residir en el Jardín, cualquier urantiano tenía que ser <<adoptado>>. Recibía información sobre el proyecto y la finalidad de la donación adámica, expresaba su intención de unirse a esta misión, y luego hacía una declaración de lealtad a las reglas sociales de Adán y a la soberanía espiritual del Padre Universal.

\par
%\textsuperscript{(836.1)}
\textsuperscript{74:7.12} Las leyes del Jardín estaban basadas en los antiguos códigos de Dalamatia y se promulgaron en siete títulos:

\par
%\textsuperscript{(836.2)}
\textsuperscript{74:7.13} 1. Las leyes de la salud y la higiene.

\par
%\textsuperscript{(836.3)}
\textsuperscript{74:7.14} 2. Las reglas sociales del Jardín.

\par
%\textsuperscript{(836.4)}
\textsuperscript{74:7.15} 3. El código del intercambio y el comercio.

\par
%\textsuperscript{(836.5)}
\textsuperscript{74:7.16} 4. Las leyes del juego limpio y la competición.

\par
%\textsuperscript{(836.6)}
\textsuperscript{74:7.17} 5. Las leyes de la vida familiar.

\par
%\textsuperscript{(836.7)}
\textsuperscript{74:7.18} 6. Los códigos civiles de la regla de oro.

\par
%\textsuperscript{(836.8)}
\textsuperscript{74:7.19} 7. Los siete mandamientos de la regla moral suprema.

\par
%\textsuperscript{(836.9)}
\textsuperscript{74:7.20} La ley moral del Edén difería poco de los siete mandamientos de Dalamatia, pero los adamitas enseñaban numerosas razones adicionales para justificarlos; por ejemplo, en lo que se refiere al mandato contra el homicidio, la presencia interior del Ajustador del Pensamiento se ofrecía como motivo adicional para no destruir la vida humana. Enseñaban que \guillemotleft quienquiera que derrama la sangre del hombre, su sangre será derramada por el hombre\footnote{\textit{Quien derrama sangre, su sangre será derramada}: Gn 9:6.}, porque Dios hizo al hombre a su imagen\guillemotright\footnote{\textit{Hombre hecho a imagen de Dios}: Gn 1:26-27.}.

\par
%\textsuperscript{(836.10)}
\textsuperscript{74:7.21} El culto público en el Edén tenía lugar a mediodía, y el culto familiar se realizaba a la puesta del Sol. Adán hizo todo lo que pudo por evitar el empleo de oraciones estereotipadas, enseñando que una oración eficaz debe ser totalmente personal, que debe representar <<el deseo del alma>>\footnote{\textit{El deseo del alma}: Sal 63:1; Sal 84:2; Is 26:8.}; pero los edenitas continuaron empleando las oraciones y los modelos establecidos, transmitidos desde la época de Dalamatia. Adán también se esforzó por sustituir los sacrificios sangrientos de las ceremonias religiosas por las ofrendas de los frutos de la tierra, pero había hecho pocos progresos en este sentido antes de la desorganización del Jardín.

\par
%\textsuperscript{(836.11)}
\textsuperscript{74:7.22} Adán intentó enseñar a las razas la igualdad de los sexos. La manera en que Eva trabajaba al lado de su marido causó una profunda impresión en todos los habitantes del Jardín. Adán les enseñó claramente que la mujer aporta, de igual modo que el hombre, los factores de la vida que se unen para formar un nuevo ser. La humanidad había supuesto, hasta ese momento, que toda la procreación residía en las <<costillas del padre>>\footnote{\textit{Costillas del padre}: Gn 35:11; 46:26; Heb 7:5,10.}. Habían considerado a la madre como un simple recurso para nutrir al nonato y amamantar al recién nacido.

\par
%\textsuperscript{(836.12)}
\textsuperscript{74:7.23} Adán enseñó a sus contemporáneos todo lo que podían comprender, pero comparativamente hablando, no fue gran cosa. Sin embargo, las razas más inteligentes de la Tierra esperaban con impaciencia el momento en que se les permitiría casarse con los hijos y las hijas superiores de la raza violeta. ¡Qué mundo tan diferente hubiera sido Urantia si se hubiera llevado a cabo este gran proyecto para mejorar las razas! Aún así, la pequeña cantidad de sangre que los pueblos evolutivos obtuvieron fortuitamente de esta raza importada ha producido unos beneficios extraordinarios.

\par
%\textsuperscript{(836.13)}
\textsuperscript{74:7.24} Así es como Adán trabajó por el bienestar y la elevación del mundo donde residió. Pero conducir a estos pueblos mezclados y mestizos por el mejor camino era una tarea muy difícil.

\section*{8. La leyenda de la creación}
\par
%\textsuperscript{(836.14)}
\textsuperscript{74:8.1} La historia de la creación de Urantia\footnote{\textit{Tradición de la creación}: Gn 1:1-31; Ex 20:11.} en seis días estaba basada en la tradición de que Adán y Eva habían pasado precísamente seis días inspeccionando inicialmente el Jardín. Esta circunstancia dió una justificación casi sagrada al período de tiempo de la semana\footnote{\textit{Tradición del sábado}: Gn 2:2-3; Ex 20:9-11.}, que había sido introducida en un principio por los dalamatianos. El hecho de que Adán pasara seis días inspeccionando el Jardín y formulando los planes preliminares para su organización no fue preparado de antemano; fue elaborado día a día. La elección del séptimo día para el culto fue algo totalmente casual según los hechos que acabamos de narrar.

\par
%\textsuperscript{(837.1)}
\textsuperscript{74:8.2} La leyenda de la creación del mundo en seis días fue una idea posterior que, de hecho, surgió más de treinta mil años después. Una característica de esta narración, la aparición repentina del Sol y la Luna\footnote{\textit{El sol y la luna}: Gn 1:3-5,14-18.}, puede haber tenido su origen en las tradiciones que contaban que, en el pasado, el mundo había surgido repentinamente de una densa nube espacial compuesta de materia diminuta, que había ocultado durante mucho tiempo tanto al Sol como a la Luna.

\par
%\textsuperscript{(837.2)}
\textsuperscript{74:8.3} La historia de la creación de Eva a partir de una costilla de Adán\footnote{\textit{Eva de una costilla de Adán}: Gn 2:21-23.} es un resumen confuso de la llegada de Adán y de la cirugía celestial efectuada durante el intercambio de sustancias vivientes que tuvo lugar cuando vino el estado mayor corpóreo del Príncipe Planetario, más de cuatrocientos cincuenta mil años antes.

\par
%\textsuperscript{(837.3)}
\textsuperscript{74:8.4} La mayoría de los pueblos del mundo ha sido influida por la tradición de que Adán y Eva poseían unas formas físicas que habían sido creadas para ellos en el momento de llegar a Urantia\footnote{\textit{Cuerpos de Adán y Eva}: Gn 2:7,21-22; 3:19,23.}. La creencia de que el hombre había sido creado del barro era casi universal en el hemisferio oriental; esta tradición se puede encontrar en todas partes, desde las Islas Filipinas hasta África. Muchos grupos aceptaron esta historia de que el hombre había surgido del barro mediante alguna forma de creación especial, en lugar de sus creencias anteriores en la creación progresiva ---en la evolución.

\par
%\textsuperscript{(837.4)}
\textsuperscript{74:8.5} Lejos de las influencias de Dalamatia y del Edén, la humanidad tendía a creer en la ascensión gradual de la raza humana. El hecho de la evolución no es un descubrimiento moderno; los antiguos comprendían el lento carácter evolutivo del progreso humano. Los primeros griegos tenían unas ideas claras sobre esto, a pesar de su proximidad con Mesopotamia. Aunque las diversas razas de la Tierra se confundieron lamentablemente en sus teorías sobre la evolución, sin embargo muchas tribus primitivas creían y enseñaban que eran los descendientes de diversos animales. Los pueblos primitivos tenían la costumbre de elegir como <<tótem>> a los animales que suponían habían tenido por ascendientes. Algunas tribus de indios norteamericanos creían que se habían originado en los castores y los coyotes. Ciertas tribus africanas enseñan que descienden de la hiena, una tribu malaya del lémur y un grupo de Nueva Guinea del loro.

\par
%\textsuperscript{(837.5)}
\textsuperscript{74:8.6} A causa de su contacto directo con los restos de la civilización de los adamitas, los babilonios ampliaron y embellecieron la historia de la creación del hombre, y enseñaron que el hombre había descendido directamente de los dioses. Se aferraron al origen aristocrático de la raza, lo cual era incompatible incluso con la doctrina de la creación a partir del barro.

\par
%\textsuperscript{(837.6)}
\textsuperscript{74:8.7} El relato de la creación en el Antiguo Testamento data de mucho tiempo después de la época de Moisés; éste nunca enseñó a los hebreos una historia tan deformada. Pero sí presentó a los israelitas un relato sencillo y condensado de la creación, esperando realzar así su llamamiento a la adoración del Creador, el Padre Universal, a quien él llamaba el Señor Dios de Israel.

\par
%\textsuperscript{(837.7)}
\textsuperscript{74:8.8} En sus primeras enseñanzas, Moisés no intentó, con mucho juicio, remontarse más atrás de la época de Adán, y puesto que Moisés era el instructor supremo de los hebreos, las historias de Adán se asociaron íntimamente con las de la creación. Las tradiciones más antiguas reconocían una civilización preadámica, lo que está claramente demostrado en el hecho de que los redactores posteriores, cuando intentaron eliminar toda referencia a los asuntos humanos anteriores a la época de Adán, olvidaron suprimir la referencia reveladora de la emigración de Caín a la <<tierra de Nod>>\footnote{\textit{Tierra de Nod}: Gn 4:16.}, donde se casó.

\par
%\textsuperscript{(838.1)}
\textsuperscript{74:8.9} Los hebreos no tuvieron ningún lenguaje escrito de uso común durante mucho tiempo después de llegar a Palestina. Aprendieron a utilizar el alfabeto gracias a sus vecinos los filisteos, que eran refugiados políticos de la civilización superior de Creta. Los hebreos escribieron poco hasta cerca del año 900 a. de J.C.; como no dispusieron de un lenguaje escrito hasta esta fecha tan tardía, diversas historias de la creación circularon entre ellos, pero después de la cautividad en Babilonia tendieron más a aceptar una versión mesopotámica modificada.

\par
%\textsuperscript{(838.2)}
\textsuperscript{74:8.10} La tradición judía se cristalizó alrededor de Moisés, y como éste se había esforzado en hacer remontar el linaje de Abraham\footnote{\textit{Linaje judío}: Gn 4:1-2,17-26; 5:3-32; 6:9-10; 10:1-32; 11:10-26.} hasta Adán, los judíos supusieron que Adán era el primer hombre de toda la humanidad. Yahvé era el creador, y como se creía que Adán era el primer hombre, Yahvé tenía que haber creado el mundo poco antes de hacer a Adán. Luego, la tradición de los seis días de Adán se entrelazó en la historia, con el resultado de que cerca de mil años después de la estancia de Moisés en la Tierra, la tradición de la creación en seis días se puso por escrito y posteriormente se le atribuyó a Moisés.

\par
%\textsuperscript{(838.3)}
\textsuperscript{74:8.11} Cuando los sacerdotes judíos regresaron a Jerusalén, ya habían terminado de escribir su relato sobre el comienzo de las cosas. Pronto afirmaron que esta narración era una historia de la creación escrita por Moisés y descubierta recientemente\footnote{\textit{El ``descubrimiento'' de la Ley}: 2 Cr 34:13-19.}. Pero los hebreos contemporáneos de los alrededores del año 500 a. de J.C. no consideraban que estas escrituras fueran revelaciones divinas; las contemplaban poco más o menos como los pueblos posteriores consideran los relatos mitológicos.

\par
%\textsuperscript{(838.4)}
\textsuperscript{74:8.12} Este documento apócrifo, que tenía fama de ser las enseñanzas de Moisés, atrajo la atención de Ptolomeo, el rey griego de Egipto, que lo mandó traducir al griego por una comisión de setenta eruditos para su nueva biblioteca de Alejandría. Este relato encontró así un lugar entre los escritos que más tarde formaron parte de las colecciones posteriores de <<escrituras sagradas>> de las religiones hebrea y cristiana. Debido a su identificación con estos sistemas teológicos, estos conceptos influyeron profundamente durante mucho tiempo en la filosofía de numerosos pueblos occidentales.

\par
%\textsuperscript{(838.5)}
\textsuperscript{74:8.13} Los instructores cristianos perpetuaron la creencia de que la raza humana había sido creada por decreto, y todo ello condujo directamente a formar la hipótesis de que en otro tiempo había existido una edad de oro de felicidad utópica, y a la teoría de la caída del hombre o del superhombre, la cual explicaba la condición nada utópica de la sociedad. Estos puntos de vista sobre la vida y el lugar del hombre en el universo eran, en el mejor de los casos, desalentadores, puesto que estaban basados en una creencia en la regresión más bien que en la progresión, y además implicaban una Deidad vengativa que había descargado su ira contra la raza humana como justo castigo por los errores de algunos antiguos administradores planetarios.

\par
%\textsuperscript{(838.6)}
\textsuperscript{74:8.14} La <<edad de oro>> es un mito, pero el Edén fue un hecho, y la civilización del Jardín se derrumbó realmente. Adán y Eva continuaron en el Jardín durante ciento diecisiete años, y entonces, a causa de la impaciencia de Eva y de los errores de juicio de Adán, se atrevieron a desviarse del camino ordenado, y atrajeron rápidamente un desastre sobre sí mismos y un retraso ruinoso sobre el desarrollo progresivo de toda Urantia.

\par
%\textsuperscript{(838.7)}
\textsuperscript{74:8.15} [Narrado por Solonia, la <<voz seráfica en el Jardín>>\footnote{\textit{Voz del Jardín}: Gn 3:8-10.}.]


\chapter{Documento 75. La falta de Adán y Eva}
\par
%\textsuperscript{(839.1)}
\textsuperscript{75:0.1} DESPUÉS de más de cien años de esfuerzos en Urantia, Adán podía observar muy pocos progresos fuera del Jardín; el mundo en general no parecía mejorar mucho. La realización de la mejora de las razas parecía estar muy lejana, y la situación daba la impresión de ser tan desesperada como para necesitar algún tipo de ayuda no contemplada en los planes originales. Al menos esto es lo que pasaba a menudo por la mente de Adán, y así se lo expresó muchas veces a Eva. Adán y su pareja eran leales, pero estaban aislados de los de su misma orden, y profundamente afligidos por la triste situación de su mundo.

\section*{1. El problema de Urantia}
\par
%\textsuperscript{(839.2)}
\textsuperscript{75:1.1} La misión adámica en Urantia, un planeta experimental, marcado por la rebelión y aislado, era una tarea monumental. El Hijo y la Hija Materiales no tardaron en darse cuenta de la dificultad y la complejidad de su misión planetaria. Sin embargo, emprendieron valientemente la tarea de resolver sus múltiples problemas. Pero cuando se dispusieron a realizar el trabajo tan importante de eliminar a los anormales y degenerados de los linajes humanos, se quedaron totalmente consternados. No lograban encontrar ninguna salida al dilema, y tampoco podían consultar a sus superiores de Jerusem ni de Edentia. Aquí estaban pues, aislados y teniendo que afrontar cada día algún enredo nuevo y complicado, algún problema que parecía insoluble.

\par
%\textsuperscript{(839.3)}
\textsuperscript{75:1.2} En condiciones normales, la primera tarea de un Adán y una Eva Planetarios hubiera sido la coordinación y la mezcla de las razas. Pero en Urantia este proyecto parecía casi irrealizable, pues aunque las razas estaban biológicamente preparadas, nunca habían sido depuradas de sus linajes atrasados y defectuosos.

\par
%\textsuperscript{(839.4)}
\textsuperscript{75:1.3} Adán y Eva se encontraban en una esfera que no estaba de ninguna manera preparada para la proclamación de la fraternidad de los hombres, en un mundo que andaba a tientas en una oscuridad espiritual abyecta, y afligido por una confusión que era aún más grave debido al fracaso de la misión de la administración anterior. La mente y la moralidad se encontraban en un nivel bajo, y en lugar de emprender la tarea de llevar a cabo la unidad religiosa, tenían que empezar de nuevo todo el trabajo de convertir a los habitantes a las formas más simples de creencias religiosas. En lugar de encontrarse con un idioma ya preparado para ser adoptado, tenían que enfrentarse con la confusión mundial de cientos y cientos de dialectos locales. Ningún Adán del servicio planetario había sido depositado jamás en un mundo más difícil; los obstáculos parecían insuperables y los problemas insolubles para una criatura.

\par
%\textsuperscript{(839.5)}
\textsuperscript{75:1.4} Estaban aislados, y el enorme sentimiento de soledad que pesaba sobre ellos se acrecentó aún más con la partida prematura de los síndicos Melquisedeks. Sólo a través de las órdenes angélicas podían comunicarse indirectamente con cualquier ser que estuviera fuera del planeta. Poco a poco su valentía se debilitaba, sus ánimos decaían, y a veces su fe casi vacilaba.

\par
%\textsuperscript{(840.1)}
\textsuperscript{75:1.5} Ésta es la verdadera imagen de la consternación que sentían estas dos nobles almas mientras reflexionaban sobre las tareas con las que se enfrentaban. Los dos eran profundamente conscientes de la enorme empresa que implicaba la ejecución de su misión planetaria.

\par
%\textsuperscript{(840.2)}
\textsuperscript{75:1.6} Es probable que ninguno de los Hijos Materiales de Nebadon tuvo que enfrentarse nunca con una tarea tan difícil, y aparentemente tan desesperada, como la que tenían Adán y Eva ante la triste situación de Urantia. Pero algún día hubieran conseguido el éxito si hubieran sido más perspicaces y \textit{pacientes}. Los dos, y sobre todo Eva, eran demasiado impacientes; no estaban dispuestos a acomodarse a la larguísima prueba de resistencia. Querían ver algunos resultados inmediatos, y los vieron, pero los resultados que consiguieron así fueron sumamente desastrosos tanto para ellos como para su mundo.

\section*{2. La conspiración de Caligastia}
\par
%\textsuperscript{(840.3)}
\textsuperscript{75:2.1} Caligastia visitó con frecuencia el Jardín y tuvo muchas conversaciones con Adán y Eva, pero éstos se mostraron inflexibles ante todas sus sugerencias de compromisos y de atajos aventureros. Tenían ante ellos bastantes resultados de la rebelión como para estar inmunizados de manera eficaz contra todas estas proposiciones insinuantes. Incluso las propuestas de Daligastia ejercían poca influencia sobre los jóvenes descendientes de Adán. Y por supuesto, ni Caligastia ni su asociado tenían poder para influir sobre un individuo cualquiera en contra de su voluntad, y mucho menos para persuadir a los hijos de Adán a que obraran mal.

\par
%\textsuperscript{(840.4)}
\textsuperscript{75:2.2} Conviene recordar que Caligastia era todavía el Príncipe Planetario titular de Urantia, un Hijo descaminado, pero a pesar de todo un Hijo elevado, del universo local. No fue depuesto finalmente hasta la época en que Cristo Miguel estuvo en Urantia.

\par
%\textsuperscript{(840.5)}
\textsuperscript{75:2.3} Pero el Príncipe caído era perseverante y decidido. Pronto renunció a convencer a Adán, y decidió intentar un astuto ataque indirecto contra Eva. El maligno llegó a la conclusión de que la única esperanza de tener éxito residía en la hábil utilización de las personas adecuadas que pertenecían a los estratos superiores del grupo nodita, los descendientes de sus antiguos asociados del estado mayor corpóreo. Y preparó sus planes en consecuencia para coger en una trampa a la madre de la raza violeta.

\par
%\textsuperscript{(840.6)}
\textsuperscript{75:2.4} Eva nunca tuvo la menor intención de hacer nada que estuviera en contra de los planes de Adán o que pusiera en peligro su deber planetario. Como conocían la tendencia de la mujer a buscar resultados inmediatos en lugar de hacer planes con visión de futuro y con efectos más lejanos, los Melquisedeks, antes de partir, habían advertido especialmente a Eva de los peligros específicos que amenazaban su situación aislada en el planeta, y le habían aconsejado en particular que nunca se apartara del lado de su marido, es decir, que no intentara métodos personales o secretos para fomentar sus empresas comunes. Eva había seguido escrupulosamente estas instrucciones durante más de cien años, y no se le ocurrió que hubiera ningún peligro en las conversaciones cada vez más privadas y confidenciales que disfrutaba con cierto jefe nodita llamado Serapatatia. Todo el asunto se desarrolló de manera tan gradual y natural que a Eva la cogió desprevenida.

\par
%\textsuperscript{(840.7)}
\textsuperscript{75:2.5} Los habitantes del Jardín habían estado en contacto con los noditas desde los primeros días del Edén. Habían recibido una ayuda valiosa y mucha cooperación de estos descendientes mixtos de los miembros rebeldes del estado mayor de Caligastia, y ahora el régimen edénico iba a encontrar a través de ellos su completa ruina y su destrucción final.

\section*{3. La tentación de Eva}
\par
%\textsuperscript{(841.1)}
\textsuperscript{75:3.1} Adán acababa de terminar sus primeros cien años en la Tierra cuando Serapatatia, a la muerte de su padre, asumió el mando de la confederación occidental o siria de las tribus noditas. Serapatatia era un hombre de piel morena, un brillante descendiente del antiguo jefe de la comisión sanitaria de Dalamatia, el cual se había casado con una de las mentes femeninas superiores de la raza azul de aquellos tiempos lejanos. Esta familia había ostentado la autoridad a lo largo de los siglos y había ejercido una gran influencia entre las tribus noditas del oeste.

\par
%\textsuperscript{(841.2)}
\textsuperscript{75:3.2} Serapatatia había visitado varias veces el Jardín y le había impresionado profundamente la rectitud de la causa de Adán. Poco después de asumir el mando de los noditas sirios, anunció su intención de establecer una relación muy estrecha con el trabajo de Adán y Eva en el Jardín. La mayoría de su pueblo se unió a él en este programa, y Adán se regocijó con la noticia de que la más poderosa y la más inteligente de todas las tribus vecinas había decidido casi en masa apoyar el programa para mejorar el mundo; era indudablemente alentador. Poco después de este gran acontecimiento, Adán y Eva recibieron a Serapatatia y a su nuevo estado mayor en su propia casa.

\par
%\textsuperscript{(841.3)}
\textsuperscript{75:3.3} Serapatatia se convirtió en uno de los lugartenientes de Adán más capaces y eficaces. Era totalmente honrado y completamente sincero en todas sus actividades; nunca fue consciente, ni siquiera posteriormente, de que el astuto Caligastia lo estaba utilizando como instrumento accesorio\footnote{\textit{El ardiz de Caligastia}: Gn 3:1a.}.

\par
%\textsuperscript{(841.4)}
\textsuperscript{75:3.4} Serapatatia se convirtió pronto en el presidente asociado de la comisión edénica para las relaciones tribales, y se prepararon numerosos planes para continuar más enérgicamente la tarea de conseguir que las tribus lejanas se interesaran por la causa del Jardín.

\par
%\textsuperscript{(841.5)}
\textsuperscript{75:3.5} Mantuvo muchas entrevistas con Adán y Eva ---sobre todo con Eva--- y hablaron de muchos proyectos para mejorar sus métodos. Un día, durante una conversación con Eva, a Serapatatia se le ocurrió que mientras esperaban el reclutamiento de una gran cantidad de representantes de la raza violeta, sería muy beneficioso que entretanto se pudiera hacer algo por el progreso inmediato de las tribus necesitadas que aguardaban. Serapatatia afirmó que si los noditas, en calidad de la raza más progresiva y cooperativa, pudieran tener un jefe que naciera entre ellos con una parte de sangre violeta, esto constituiría un vínculo poderoso que uniría más estrechamente a estos pueblos con el Jardín. Se consideró sensata y honestamente que todo esto sería beneficioso para el mundo, ya que este niño, que sería criado y educado en el Jardín, ejercería una gran influencia benéfica sobre el pueblo de su padre.

\par
%\textsuperscript{(841.6)}
\textsuperscript{75:3.6} Conviene recalcar de nuevo que Serapatatia era completamente honesto y totalmente sincero en todas sus proposiciones. Nunca sospechó que estaba haciendo el juego de Caligastia y Daligastia\footnote{\textit{El ardiz de Caligastia}: Gn 3:1a.}. Serapatatia era totalmente leal al proyecto de acumular una gran reserva de la raza violeta antes de intentar el mejoramiento mundial de los pueblos desorientados de Urantia. Pero esto último necesitaría cientos de años para llevarse a cabo, y él era impaciente; quería ver algunos resultados inmediatos ---algo que se produjera durante su propia vida. Indicó claramente a Eva que Adán estaba a menudo desanimado por lo poco que se había logrado para mejorar el mundo.

\par
%\textsuperscript{(841.7)}
\textsuperscript{75:3.7} Estos planes se maduraron en secreto durante más de cinco años. Al final se desarrollaron hasta tal punto que Eva consintió en tener una entrevista secreta con Cano, la mente más brillante y el jefe más activo de la colonia cercana de noditas amistosos. Cano simpatizaba mucho con el régimen adámico; de hecho era el guía espiritual sincero de los noditas vecinos que apoyaban las relaciones amistosas con el Jardín.

\par
%\textsuperscript{(842.1)}
\textsuperscript{75:3.8} La reunión fatídica se produjo durante las horas del crepúsculo de una tarde de otoño, cerca de la casa de Adán. Eva nunca se había encontrado antes con el hermoso y entusiasta Cano ---que era un magnífico ejemplar sobreviviente de la constitución física superior y del intelecto sobresaliente de sus lejanos progenitores del estado mayor del Príncipe. Cano creía también plenamente en la rectitud del proyecto de Serapatatia. (La poligamia se practicaba de manera habitual fuera del Jardín.)

\par
%\textsuperscript{(842.2)}
\textsuperscript{75:3.9} Influida por los halagos, el entusiasmo y una gran persuasión personal, Eva accedió enseguida a embarcarse en la empresa tan discutida, a añadir su propio pequeño proyecto de salvación del mundo al plan divino más amplio y de más largo alcance. Antes de darse plenamente cuenta de lo que sucedía, el paso fatal se había dado. Ya estaba hecho.

\section*{4. La toma de conciencia de la falta}
\par
%\textsuperscript{(842.3)}
\textsuperscript{75:4.1} La vida celestial del planeta estaba en efervescencia. Adán reconoció que algo iba mal y le pidió a Eva que fuera con él a un lado del Jardín. Adán escuchó entonces, por primera vez, toda la historia del plan madurado durante largo tiempo para acelerar el progreso del mundo, actuando simultáneamente en dos direcciones: la continuación del plan divino junto con la ejecución del proyecto de Serapatatia.

\par
%\textsuperscript{(842.4)}
\textsuperscript{75:4.2} Mientras el Hijo y la Hija Materiales conversaban así en el Jardín iluminado por la Luna, <<la voz en el Jardín>> les reprochó su desobediencia\footnote{\textit{Notificación de la falta}: Gn 3:8-13.}. Aquella voz no era otra que mi propio anuncio a la pareja edénica de que habían transgredido el pacto del Jardín, que habían desobedecido las instrucciones de los Melquisedeks, que habían fracasado en la ejecución del juramento de confianza que habían prestado al soberano del universo.

\par
%\textsuperscript{(842.5)}
\textsuperscript{75:4.3} Eva había consentido en participar en la práctica del bien y del mal. El bien es la realización de los planes divinos; el pecado es una transgresión deliberada de la voluntad divina; el mal es la inadaptación de los planes y el desajuste de las técnicas que acaban provocando la falta de armonía en el universo y la confusión planetaria.

\par
%\textsuperscript{(842.6)}
\textsuperscript{75:4.4} Cada vez que la pareja del Jardín había comido del fruto del árbol de la vida, el arcángel guardián les había advertido que se abstuvieran de ceder a las sugerencias de Caligastia tendentes a combinar el bien y el mal. Habían sido prevenidos en los términos siguientes: <<El día que mezcléis el bien y el mal, os volveréis sin duda como los mortales del mundo; moriréis con toda seguridad>>\footnote{\textit{Al mezclar bien con mal moriréis}: Gn 2:17.}.

\par
%\textsuperscript{(842.7)}
\textsuperscript{75:4.5} En el momento fatídico de su encuentro secreto, Eva le había contado a Cano esta advertencia\footnote{\textit{La advertencia}: Gn 3:1-5.} tantas veces repetida, pero Cano, que no conocía ni la importancia ni el significado de estos avisos, le había asegurado que los hombres y las mujeres con móviles buenos e intenciones sinceras no podían obrar mal, que ella seguramente no moriría, sino que más bien viviría de nuevo en la persona del hijo de los dos, el cual crecería para bendecir y estabilizar el mundo.

\par
%\textsuperscript{(842.8)}
\textsuperscript{75:4.6} Aunque este proyecto para modificar el plan divino se había concebido y ejecutado con toda sinceridad y únicamente con los móviles más elevados para el bienestar del mundo, constituía un mal porque representaba la manera equivocada de conseguir unos fines justos\footnote{\textit{El fin no justifica los medios}: Pr 14:12.}, porque se apartaba del camino recto, del plan divino.

\par
%\textsuperscript{(843.1)}
\textsuperscript{75:4.7} Es verdad que Eva había encontrado atractivo a Cano\footnote{\textit{Cano atractivo a los ojos de Eva}: Gn 3:6a.}, y experimentó todo lo que le prometía su seductor, pasando por <<un conocimiento nuevo y mayor de los asuntos humanos y una comprensión más viva de la naturaleza humana como complemento de la comprensión de la naturaleza adámica>>\footnote{\textit{Una comprensión más viva}: Gn 3:6b.}.

\par
%\textsuperscript{(843.2)}
\textsuperscript{75:4.8} Aquella noche estuve hablando en el Jardín con el padre y la madre de la raza violeta, como era mi deber en aquellas tristes circunstancias. Escuché el relato completo de todo lo que había conducido a la Madre Eva a cometer la falta, y les di a los dos asesoramiento y consejos respecto a la situación inmediata. Algunos de estos consejos los siguieron, y otros los pasaron por alto. Esta entrevista aparece en vuestros anales como <<el Señor Dios llamó a Adán y Eva en el Jardín y les preguntó:
`¿Dónde estáis?'>>\footnote{\textit{Registrado como ``Dios llamó''}: Gn 3:8-13.}. Las generaciones posteriores tenían la costumbre de atribuir todo lo que era insólito y extraordinario, ya fuera físico o espiritual, a la intervención personal directa de los Dioses.

\section*{5. Las repercusiones de la falta}
\par
%\textsuperscript{(843.3)}
\textsuperscript{75:5.1} La desilusión de Eva fue realmente patética. Adán percibió toda la difícil situación, y aunque tenía el corazón destrozado y estaba abatido, sólo albergaba compasión y simpatía por su compañera equivocada.

\par
%\textsuperscript{(843.4)}
\textsuperscript{75:5.2} Al día siguiente del tropiezo de Eva, desesperado por su conciencia del fracaso, Adán buscó a Laotta, la brillante nodita que dirigía las escuelas occidentales del Jardín, y cometió con premeditación la misma locura que Eva. Pero no os equivoquéis. Adán no fue seducido; sabía exactamente lo que hacía; escogió deliberadamente compartir el mismo destino que Eva. Amaba a su compañera con un afecto sobrehumano, y la idea de la posibilidad de una vigilia solitaria sin ella en Urantia sobrepasaba lo que podía soportar.

\par
%\textsuperscript{(843.5)}
\textsuperscript{75:5.3} Cuando se enteraron de lo que le había sucedido a Eva, los habitantes enfurecidos del Jardín se volvieron inmanejables; declararon la guerra a la colonia nodita vecina. Salieron rápidamente por las puertas del Edén y cayeron sobre esta población desprevenida, destruyéndola por completo ---no se salvó ni un solo hombre, mujer o niño. Cano, el padre de Caín aún por nacer, también pereció.

\par
%\textsuperscript{(843.6)}
\textsuperscript{75:5.4} Cuando se dio cuenta de lo que había sucedido, Serapatatia se hundió en la consternación; el miedo y los remordimientos lo pusieron fuera de sí, y al día siguiente se ahogó en el gran río.

\par
%\textsuperscript{(843.7)}
\textsuperscript{75:5.5} Los hijos de Adán trataron de consolar a su madre aturdida, mientras su padre vagaba en la soledad durante treinta días. Al final de este período se impuso el juicio; Adán regresó a su hogar y empezó a hacer planes para su futura línea de conducta.

\par
%\textsuperscript{(843.8)}
\textsuperscript{75:5.6} Las consecuencias de las locuras de unos padres descaminados son compartidas con mucha frecuencia por sus hijos inocentes. Los nobles y honrados hijos e hijas de Adán y Eva estaban abrumados por la inexplicable tristeza de la tragedia increíble que tan repentina y despiadadamente se había precipitado sobre ellos. Los hijos mayores tardaron más de cincuenta años en recuperarse del dolor y la tristeza de aquellos días trágicos, sobre todo del terror de aquel período de treinta días durante los cuales su padre estuvo ausente del hogar, mientras su madre aturdida ignoraba por completo cuál era su paradero o la suerte que había corrido.

\par
%\textsuperscript{(843.9)}
\textsuperscript{75:5.7} Estos mismos treinta días fueron para Eva como largos años de dolor y sufrimiento. Esta noble alma nunca se recuperó plenamente de los efectos de aquel período insoportable de sufrimiento mental y de tristeza espiritual. Ningún aspecto de sus privaciones y dificultades materiales posteriores pudo compararse nunca, en la memoria de Eva, con aquellos días terribles y aquellas noches espantosas de soledad y de incertidumbre insoportable. Se enteró del acto irreflexivo de Serapatatia y no sabía si su marido se había suicidado de dolor o había sido sacado del planeta como castigo por la falta de ella. Cuando Adán regresó, Eva sintió la satisfacción de una alegría y una gratitud que nunca se borró durante su larga y difícil vida conyugal de duro servicio.

\par
%\textsuperscript{(844.1)}
\textsuperscript{75:5.8} El tiempo pasaba, pero Adán no estuvo seguro de la naturaleza de su infracción hasta setenta días después de la falta de Eva, cuando los síndicos Melquisedeks regresaron a Urantia y asumieron la jurisdicción sobre los asuntos del mundo. Entonces supo que habían fracasado.

\par
%\textsuperscript{(844.2)}
\textsuperscript{75:5.9} Pero aún se estaban preparando más dificultades: La noticia de la aniquilación de la colonia nodita cercana al Edén no tardó en llegar hasta las tribus de origen de Serapatatia situadas en el norte, y pronto se congregó un gran ejército para dirigirse hacia el Jardín. Éste fue el principio de una larga guerra encarnizada entre los adamitas y los noditas, ya que estas hostilidades continuaron hasta mucho tiempo después de que Adán y sus seguidores emigraran al segundo jardín en el valle del Éufrates. Hubo una <<enemistad intensa y duradera entre aquel hombre y la mujer, entre la descendencia de él y la descendencia de ella>>\footnote{\textit{Enemistad intensa y duradera}: Gn 3:15.}.

\section*{6. Adán y Eva abandonan el Jardín}
\par
%\textsuperscript{(844.3)}
\textsuperscript{75:6.1} Cuando Adán se enteró de que los noditas estaban en marcha, buscó el asesoramiento de los Melquisedeks, pero éstos se negaron a aconsejarle; sólo le dijeron que hiciera lo que estimara más conveniente, y le prometieron su cooperación amistosa, en la medida de lo posible, en la línea de conducta que decidiera. A los Melquisedeks se les había prohibido que se entrometieran en los planes personales de Adán y Eva.

\par
%\textsuperscript{(844.4)}
\textsuperscript{75:6.2} Adán sabía que él y Eva habían fracasado; la presencia de los síndicos Melquisedeks se lo indicaba, aunque aún no sabía nada sobre su situación personal y su destino futuro. Mantuvo una reunión durante toda la noche con unos mil doscientos seguidores leales que se habían comprometido a seguir a su jefe, y al día siguiente al mediodía, estos peregrinos salieron del Edén en busca de un nuevo hogar\footnote{\textit{Abandono de Edén}: Gn 3:23-24.}. A Adán no le agradaba la guerra, y eligió en consecuencia dejar el primer jardín a los noditas sin oponer resistencia.

\par
%\textsuperscript{(844.5)}
\textsuperscript{75:6.3} Al tercer día de salir del Jardín, la caravana edénica fue detenida por la llegada de los transportes seráficos de Jerusem. A Adán y Eva se les informó por primera vez sobre cuál sería el destino de sus hijos. Mientras los transportes permanecían preparados, aquellos hijos que habían llegado a la edad de elegir (veinte años) recibieron la opción de permanecer en Urantia con sus padres, o de convertirse en los pupilos de los Altísimos de Norlatiadek. Dos tercios escogieron ir a Edentia, y casi un tercio eligió permanecer con sus padres. Todos los hijos menores de veinte años fueron llevados a Edentia. Nadie podría haber contemplado la dolorosa separación entre este Hijo y esta Hija Materiales y sus hijos, sin darse cuenta de que el camino del transgresor es duro\footnote{\textit{El camino del transgresor es duro}: Pr 13:15.}. Estos descendientes de Adán y Eva se encuentran ahora en Edentia; no sabemos cómo se dispondrá de ellos.

\par
%\textsuperscript{(844.6)}
\textsuperscript{75:6.4} Fue una caravana muy triste la que se preparó para continuar su viaje. ¡Nada podía haber sido más trágico! ¡Haber venido a un mundo con tan grandes esperanzas, haber sido recibidos tan favorablemente, y luego salir con oprobio del Edén, para perder además a más de las tres cuartas partes de sus hijos incluso antes de haber encontrado un nuevo lugar donde residir!

\section*{7. La degradación de Adán y Eva}
\par
%\textsuperscript{(845.1)}
\textsuperscript{75:7.1} Mientras la caravana edénica estaba detenida, a Adán y Eva se les informó sobre la naturaleza de sus transgresiones y se les comunicó el destino que les esperaba. Gabriel apareció para pronunciar la sentencia, y éste fue el veredicto: El Adán y la Eva Planetarios de Urantia son declarados en falta; han violado el pacto de su cargo de confianza como dirigentes de este mundo habitado.

\par
%\textsuperscript{(845.2)}
\textsuperscript{75:7.2} Aunque estaban abatidos por el sentimiento de culpabilidad, a Adán y Eva les animó enormemente el anuncio de que sus jueces de Salvington los habían absuelto de todos los cargos de <<desacato al gobierno del universo>>. No habían sido declarados culpables de rebelión.

\par
%\textsuperscript{(845.3)}
\textsuperscript{75:7.3} A la pareja edénica se le informó que ellos mismos se habían degradado al estado de los mortales del planeta, y que de ahora en adelante deberían comportarse\footnote{\textit{Futura conducta}: Gn 3:16-19.} como un hombre y una mujer de Urantia, considerando el futuro de las razas del mundo como el suyo propio.

\par
%\textsuperscript{(845.4)}
\textsuperscript{75:7.4} Mucho antes de que Adán y Eva salieran de Jerusem, sus instructores les habían explicado minuciosamente las consecuencias de cualquier desviación fundamental de los planes divinos. Yo les había advertido personalmente en muchas ocasiones, tanto antes como después de que llegaran a Urantia, que la degradación al estado mortal sería el resultado indudable, el castigo seguro, que acompañaría infaliblemente a cualquier negligencia en la ejecución de su misión planetaria. Pero es esencial comprender el estado de inmortalidad de la orden material de filiación para entender con claridad las consecuencias que acompañaron a la falta de Adán y Eva.

\par
%\textsuperscript{(845.5)}
\textsuperscript{75:7.5} 1. Adán y Eva, al igual que sus semejantes de Jerusem, mantenían su estado inmortal mediante una asociación intelectual con el circuito de gravedad mental del Espíritu. Cuando este sostén vital se rompe debido a una separación mental, entonces, sin tener en cuenta el nivel espiritual de existencia de la criatura, el estado de inmortalidad se pierde. El estado mortal, seguido de la disolución física, era la consecuencia inevitable de la falta intelectual de Adán y Eva.

\par
%\textsuperscript{(845.6)}
\textsuperscript{75:7.6} 2. El Hijo y la Hija Materiales de Urantia también estaban personalizados en la similitud de la carne mortal de este mundo, y dependían además del mantenimiento de un sistema circulatorio doble, el primer sistema derivado de sus naturalezas físicas, y el segundo de la superenergía acumulada en el fruto del árbol de la vida. El arcángel guardián siempre había advertido a Adán y Eva que un incumplimiento del deber culminaría en la degradación de su condición\footnote{\textit{Sentencia de muerte}: Gn 3:22-24.}, y después de la falta se les negó el acceso a esta fuente de energía.

\par
%\textsuperscript{(845.7)}
\textsuperscript{75:7.7} Caligastia logró hacer caer en la trampa a Adán y Eva, pero no consiguió su objetivo de conducirlos a una rebelión abierta contra el gobierno del universo. Lo que habían hecho estaba realmente mal, pero nunca fueron culpables de despreciar la verdad, ni tampoco se rebelaron deliberadamente contra el justo gobierno del Padre Universal y su Hijo Creador.

\section*{8. La supuesta caída del hombre}
\par
%\textsuperscript{(845.8)}
\textsuperscript{75:8.1} Adán y Eva cayeron de su estado superior de filiación material hasta la humilde condición de los hombres mortales. Pero ésta no fue la caída del hombre. La raza humana ha sido mejorada a pesar de las consecuencias inmediatas de la falta adámica. Aunque el plan divino consistente en otorgar la raza violeta a los pueblos de Urantia fracasó, las razas mortales se han beneficiado enormemente de la contribución limitada que Adán y sus descendientes aportaron a las razas de Urantia.

\par
%\textsuperscript{(846.1)}
\textsuperscript{75:8.2} No ha habido ninguna <<caída del hombre>>. La historia de la raza humana es una historia de evolución progresiva, y la donación adámica dejó a los pueblos del mundo enormemente mejorados en relación con su condición biológica anterior. Los linajes superiores de Urantia contienen ahora unos factores hereditarios que proceden como mínimo de cuatro fuentes diferentes: andonita, sangik, nodita y adámica.

\par
%\textsuperscript{(846.2)}
\textsuperscript{75:8.3} Adán no debería ser considerado como la causa de la maldición de la raza humana. Aunque fracasó en llevar adelante el plan divino, aunque transgredió su pacto con la Deidad, aunque él y su compañera fueron degradados con toda seguridad en su categoría como criaturas, a pesar de todo esto, su contribución a la raza humana hizo progresar mucho la civilización en Urantia.

\par
%\textsuperscript{(846.3)}
\textsuperscript{75:8.4} En el momento de estimar los resultados de la misión adámica en vuestro mundo, la justicia exige que se reconozca la condición del planeta. Adán se enfrentó con una tarea casi desesperada cuando fue transportado, con su hermosa compañera, desde Jerusem hasta este planeta sombrío y confuso. Pero si hubieran seguido los consejos de los Melquisedeks y sus asociados, si \textit{hubieran sido más pacientes}, habrían triunfado con el tiempo. Pero Eva escuchó la propaganda insidiosa a favor de la libertad personal y de la independencia de acción en el planeta. Fue inducida a experimentar con el plasma vital de la orden material de filiación, en el sentido de que permitió que este depósito de vida se mezclara prematuramente con el del tipo entonces ya mixto del proyecto original de los Portadores de Vida, que anteriormente se había combinado con el de los seres reproductores ligados en otro tiempo al estado mayor del Príncipe Planetario\footnote{\textit{El ``pecado'' de Adán y Eva}: Gn 3:6.}.

\par
%\textsuperscript{(846.4)}
\textsuperscript{75:8.5} En toda vuestra ascensión hacia el Paraíso, nunca ganaréis nada intentando sortear impacientemente el plan divino establecido por medio de atajos, invenciones personales u otras estratagemas para mejorar el camino de la perfección, hacia la perfección y para la perfección eterna.

\par
%\textsuperscript{(846.5)}
\textsuperscript{75:8.6} Considerándolo todo, es probable que nunca haya habido un error de sabiduría más descorazonador en ningún planeta de todo Nebadon. Pero no es de sorprender que estos pasos en falso se produzcan en los asuntos de los universos evolutivos. Formamos parte de una creación gigantesca, y no es de extrañar que todo no funcione a la perfección. Nuestro universo no fue creado perfecto; la perfección es nuestra meta eterna, no nuestro origen.

\par
%\textsuperscript{(846.6)}
\textsuperscript{75:8.7} Si éste fuera un universo mecánico, si la Gran Fuente-Centro Primera sólo fuera una fuerza y no tambíén una personalidad, si toda la creación fuera un inmenso conjunto de materia física dominado por leyes precisas caracterizadas por actividades energéticas invariables, entonces podría prevalecer la perfección, a pesar incluso del estado incompleto del universo. No habría ningún desacuerdo; no habría ninguna fricción. Pero en nuestro universo evolutivo de perfección e imperfección relativas, nos alegramos de que los desacuerdos y los malentendidos sean posibles, porque aportan la prueba del hecho y de la actividad de la personalidad en el universo. Y si nuestra creación es una existencia dominada por la personalidad, entonces podéis estar seguros de que la supervivencia, el progreso y la consecución de la personalidad son posibles; podemos confiar en el crecimiento, la experiencia y la aventura de la personalidad. ¡Qué universo tan magnífico, porque es personal y progresivo, y no simplemente mecánico o incluso pasivamente perfecto!

\par
%\textsuperscript{(846.7)}
\textsuperscript{75:8.8} [Presentado por Solonia, la <<voz seráfica en el Jardín>>.]


\chapter{Documento 76. El segundo Jardín}
\par
%\textsuperscript{(847.1)}
\textsuperscript{76:0.1} CUANDO Adán eligió dejar el primer jardín a los noditas sin oponer resistencia, no podía ir con sus seguidores hacia el oeste, porque los edenitas no disponían de barcos adecuados para esa aventura marina. No podían ir hacia el norte, pues los noditas del norte ya estaban en marcha hacia el Edén. Temían dirigirse hacia el sur, porque las colinas de aquella región estaban infestadas de tribus hostiles. La única vía abierta era hacia el este, y por eso viajaron hacia el este y las regiones entonces agradables situadas entre los ríos Tigris y Éufrates. Muchos de los que se habían quedado atrás viajaron más tarde hacia el este para unirse con los adamitas en su nueva residencia del valle\footnote{\textit{Viaje al este}: Gn 3:23-24.}.

\par
%\textsuperscript{(847.2)}
\textsuperscript{76:0.2} Caín y Sansa nacieron antes de que la caravana adámica hubiera alcanzado su destino entre los dos ríos de Mesopotamia. Laotta, la madre de Sansa, murió al nacer su hija; Eva sufrió mucho, pero sobrevivió debido a su fortaleza superior. Eva amamantó a Sansa, la hija de Laotta, y la crió con Caín. Sansa creció y llegó a ser una mujer de grandes aptitudes. Se convirtió en la esposa de Sargán, el jefe de las razas azules del norte, y contribuyó al progreso de los hombres azules de aquellos tiempos.

\section*{1. Los edenitas entran en Mesopotamia}
\par
%\textsuperscript{(847.3)}
\textsuperscript{76:1.1} La caravana de Adán necesitó casi un año entero para llegar al río Éufrates. Como lo encontraron crecido, permanecieron acampados casi seis semanas en las llanuras del oeste del río antes de atravesarlo para entrar en las tierras situadas entre los dos ríos, las cuales iban a convertirse en el segundo jardín.

\par
%\textsuperscript{(847.4)}
\textsuperscript{76:1.2} Cuando los habitantes del territorio del segundo jardín recibieron la noticia de que el rey y sumo sacerdote del Jardín del Edén marchaba hacia ellos, huyeron precipitadamente a las montañas del este. Cuando Adán llegó, encontró que todo el territorio que deseaba estaba desocupado. Aquí, en este nuevo lugar, Adán y sus colaboradores se pusieron a trabajar para construir sus nuevos hogares y establecer un nuevo centro de cultura y de religión.

\par
%\textsuperscript{(847.5)}
\textsuperscript{76:1.3} Adán sabía que este sitio era uno de los tres primeros lugares elegidos por la comisión encargada de escoger los posibles emplazamientos para el Jardín que Van y Amadón habían propuesto. Los dos ríos mismos formaban una buena defensa natural en aquellos tiempos; a poca distancia hacia el norte del segundo jardín, el Éufrates y el Tigris se acercaban mucho, de manera que se podía construir una muralla defensiva de noventa kilómetros para proteger el territorio hacia el sur y entre los mismos ríos.

\par
%\textsuperscript{(847.6)}
\textsuperscript{76:1.4} Después de instalarse en el nuevo Edén, se vieron en la necesidad de adoptar métodos de vida rudimentarios; parecía totalmente cierto que la tierra estuviera maldita. La naturaleza seguía de nuevo su curso. Ahora los adamitas se vieron obligados a arrebatarle a una tierra no preparada lo suficiente para vivir, y a enfrentarse con las realidades de la vida en medio de las hostilidades e incompatibilidades naturales de la existencia humana. Habían encontrado el primer jardín parcialmente preparado para ellos, pero el segundo tenía que ser creado con el trabajo de sus propias manos y con el <<sudor de su frente>>\footnote{\textit{Sudor de su frente}: Gn 3:19.}.

\section*{2. Caín y Abel}
\par
%\textsuperscript{(848.1)}
\textsuperscript{76:2.1} Abel\footnote{\textit{Nacimiento de Abel}: Gn 4:2.} nació menos de dos años después que Caín\footnote{\textit{Nacimiento de Caín}: Gn 4:1.}, y fue el primer hijo de Adán y Eva que nació en el segundo jardín. Cuando Abel cumplió los doce años, eligió convertirse en pastor; Caín había escogido dedicarse a la agricultura.

\par
%\textsuperscript{(848.2)}
\textsuperscript{76:2.2} Ahora bien, en aquellos tiempos existía la costumbre de hacer ofrendas\footnote{\textit{Ofrendas}: Gn 4:3-5.} al clero de las cosas que se tenían a mano. Los pastores ofrecían los animales de sus rebaños, y los campesinos los frutos de los campos; de conformidad con esta costumbre, Caín y Abel hacían también ofrendas periódicas a los sacerdotes. Los dos muchachos habían discutido muchas veces sobre los méritos relativos de sus profesiones, y Abel no tardó en señalar que se mostraba preferencia por sus sacrificios de animales. Caín recurría en vano a las tradiciones del primer Edén, a la antigua preferencia por los frutos del campo. Abel no lo admitía, y se mofaba del desconcierto de su hermano mayor.

\par
%\textsuperscript{(848.3)}
\textsuperscript{76:2.3} En los tiempos del primer Edén, Adán había procurado efectivamente no fomentar las ofrendas de animales sacrificados, de manera que Caín tenía un precedente que justificaba sus argumentos. Sin embargo, era difícil organizar la vida religiosa del segundo Edén. Adán estaba agobiado con mil y un detalles relacionados con los trabajos de la construcción, la defensa y la agricultura. Como estaba muy deprimido espiritualmente, confió la organización del culto y de la educación a los colaboradores de origen nodita que habían desempeñado estas funciones en el primer jardín; incluso en un plazo de tiempo tan corto, los sacerdotes noditas oficiantes empezaron a volver a las normas y reglas de los tiempos preadámicos.

\par
%\textsuperscript{(848.4)}
\textsuperscript{76:2.4} Los dos muchachos nunca se llevaron bien, y este asunto de los sacrificios contribuyó además a acrecentar el odio entre ellos. Abel sabía que era hijo de Adán y Eva, y nunca dejó de recalcarle a Caín que Adán no era su padre. Caín no era de pura raza violeta, puesto que su padre pertenecía a la raza nodita, que más tarde se había mezclado con los hombres azules y rojos y con la estirpe andónica aborigen. Todo esto, unido a la herencia belicosa natural de Caín, le indujo a alimentar un odio creciente hacia su hermano menor.

\par
%\textsuperscript{(848.5)}
\textsuperscript{76:2.5} Los muchachos tenían dieciocho y veinte años respectivamente cuando la tensión entre ellos se resolvió de manera definitiva; un día, las burlas de Abel enfurecieron tanto a su belicoso hermano, que Caín se revolvió airado contra él y lo mató\footnote{\textit{Caín mata a Abel}: Gn 4:8.}.

\par
%\textsuperscript{(848.6)}
\textsuperscript{76:2.6} El análisis de la conducta de Abel demuestra el valor del entorno y de la educación como factores en el desarrollo del carácter. Abel tenía una herencia ideal, y la herencia yace en el fondo de todo carácter; pero la influencia de un ambiente inferior neutralizó prácticamente esta herencia magnífica. Abel estuvo profundamente influído por su medio ambiente desfavorable, sobre todo durante sus primeros años. Se habría convertido en una persona totalmente diferente si hubiera vivido hasta los veinticinco o los treinta años; su herencia excelente se habría manifestado entonces. Aunque un buen entorno no puede contribuir mucho a vencer realmente las desventajas que una herencia inferior tiene para el carácter, un ambiente malo puede estropear de manera muy eficaz una herencia excelente, al menos durante los primeros años de la vida. Un buen entorno social y una educación adecuada constituyen el terreno y la atmósfera indispensables para sacar el mayor partido de una buena herencia.

\par
%\textsuperscript{(849.1)}
\textsuperscript{76:2.7} Los padres de Abel supieron que había muerto cuando sus perros llevaron los rebaños a la casa sin su dueño. Para Adán y Eva, Caín se iba convirtiendo rápidamente en el siniestro recuerdo de la locura que habían cometido, y lo animaron en su decisión de abandonar el jardín.

\par
%\textsuperscript{(849.2)}
\textsuperscript{76:2.8} La vida de Caín en Mesopotamia no había sido precisamente feliz, ya que era de manera tan peculiar el símbolo de la falta. No es que sus compañeros fueran poco amables con él, sino que él no ignoraba el resentimiento subconsciente que causaba su presencia. Pero Caín no tenía ninguna marca tribal\footnote{\textit{Ninguna marca tribal}: Gn 4:13-15.}, y sabía que lo matarían los primeros hombres de las tribus vecinas que se encontraran con él por casualidad. El miedo y cierto remordimiento le indujeron a arrepentirse. Caín nunca había tenido un Ajustador; siempre había desafiado la disciplina familiar y despreciado la religión de su padre. Pero ahora fue a ver a Eva, su madre, para pedirle ayuda y orientación espiritual, y en cuanto buscó sinceramente la asistencia divina, un Ajustador vino a residir dentro de él. Este Ajustador, que residía en el interior y miraba hacia el exterior, confirió a Caín una clara ventaja de superioridad que lo relacionó con la muy temida tribu de Adán\footnote{\textit{Marca de la tribu de Adán}: Gn 4:15.}.

\par
%\textsuperscript{(849.3)}
\textsuperscript{76:2.9} Caín partió pues hacia la tierra de Nod\footnote{\textit{Caín viaja a Nod}: Gn 4:16.}, al este del segundo Edén. Se convirtió en un gran jefe de uno de los grupos del pueblo de su padre, y realizó hasta cierto punto las predicciones de Serapatatia, pues durante toda su vida fomentó la paz entre esta división de los noditas y los adamitas. Caín se casó con Remona, su prima lejana, y su primer hijo, Enoc\footnote{\textit{Caín engendra a Enoc}: Gn 4:17.}, se convirtió en el jefe de los noditas elamitas. Los elamitas y los adamitas continuaron viviendo en paz durante cientos de años.

\section*{3. La vida en Mesopotamia}
\par
%\textsuperscript{(849.4)}
\textsuperscript{76:3.1} A medida que pasaba el tiempo en el segundo jardín, las consecuencias de la falta se volvían cada vez más evidentes. Adán y Eva echaban mucho de menos su antiguo hogar de belleza y tranquilidad, así como a sus hijos que habían sido deportados a Edentia. Resultaba realmente patético observar a esta pareja magnífica reducida a la condición de la naturaleza humana corriente del planeta; pero soportaron su estado disminuido con gracia y entereza.

\par
%\textsuperscript{(849.5)}
\textsuperscript{76:3.2} Adán pasaba juiciosamente la mayor parte del tiempo enseñando a sus hijos y a sus asociados la administración pública, los métodos educativos y las devociones religiosas. Si no hubiera sido por esta previsión, en el momento de su muerte se habría desencadenado un pandemónium. Tal como fueron las cosas, la muerte de Adán modificó muy poco la conducta de los asuntos de su pueblo. Pero mucho antes de fallecer, Adán y Eva reconocieron que sus hijos y seguidores habían aprendido gradualmente a olvidar sus días de gloria en el Edén. Para la mayoría de sus seguidores era mejor que olvidaran la grandiosidad del Edén, pues así no era probable que experimentaran un descontento excesivo hacia su entorno menos afortunado.

\par
%\textsuperscript{(849.6)}
\textsuperscript{76:3.3} Los gobernantes civiles de los adamitas descendían hereditariamente de los hijos del primer jardín. El primer hijo de Adán, Adanson (Adán ben Adán), fundó un centro secundario de la raza violeta al norte del segundo Edén. El segundo hijo de Adán, Evason, se convirtió en un dirigente y administrador magistral; fue el gran asistente de su padre. Evason no vivió tanto tiempo como Adán, y su hijo mayor, Jansad, se volvió el sucesor de Adán como jefe de las tribus adamitas.

\par
%\textsuperscript{(849.7)}
\textsuperscript{76:3.4} Los dirigentes religiosos, o sacerdotes, surgieron con Set\footnote{\textit{Nacimiento de Set}: Gn 5:3.}, el hijo mayor sobreviviente de Adán y Eva nacido en el segundo jardín. Nació ciento veintinueve años después de la llegada de Adán a Urantia. Set se centró en la tarea de mejorar el estado espiritual del pueblo de su padre, convirtiéndose en el jefe de los nuevos sacerdotes del segundo jardín. Su hijo, Enós\footnote{\textit{Enós}: Gn 5:6.}, fundó la nueva orden de culto, y su nieto, Cainán\footnote{\textit{Cainán}: Gn 5:9.}, instituyó el servicio exterior de misioneros para las tribus circundantes, cercanas y lejanas.

\par
%\textsuperscript{(850.1)}
\textsuperscript{76:3.5} El clero setita fue una empresa triple que abarcaba la religión, la salud y la educación. A los sacerdotes de esta orden se les enseñaba a oficiar en las ceremonias religiosas, a ejercer como médicos e inspectores sanitarios, y a trabajar como profesores en las escuelas del jardín.

\par
%\textsuperscript{(850.2)}
\textsuperscript{76:3.6} La caravana de Adán había transportado con ella las semillas y los bulbos de cientos de plantas y cereales del primer jardín hasta la tierra situada entre los dos ríos; también habían llevado consigo grandes rebaños y algunos ejemplares de todos los animales domesticados. Esto les proporcionaba grandes ventajas sobre las tribus que los rodeaban. Disfrutaban de muchos beneficios de la cultura anterior del Jardín original.

\par
%\textsuperscript{(850.3)}
\textsuperscript{76:3.7} Hasta el momento de abandonar el primer jardín, Adán y su familia siempre se habían alimentado de frutas, cereales y nueces. Camino de Mesopotamia habían comido por primera vez legumbres y verduras. El consumo de carne se introdujo pronto en el segundo jardín, pero Adán y Eva nunca comieron carne como parte de su dieta habitual. Adanson, Evason, y los demás hijos de la primera generación del primer jardín tampoco se volvieron carnívoros.

\par
%\textsuperscript{(850.4)}
\textsuperscript{76:3.8} Los adamitas superaban enormemente a los pueblos circundantes en realizaciones culturales y en desarrollo intelectual. Elaboraron el tercer alfabeto, y además sentaron las bases precursoras de una gran parte del arte, la ciencia y la literatura modernas. Aquí, en las tierras situadas entre el Tigris y el Éufrates, conservaron las artes de la escritura, el trabajo de los metales, la alfarería y la tejeduría, y realizaron un tipo de arquitectura que no fue superado durante miles de años.

\par
%\textsuperscript{(850.5)}
\textsuperscript{76:3.9} La vida familiar de los pueblos violetas era ideal para aquellos tiempos y aquella época. Los niños estaban sometidos a cursos de formación en agricultura, artesanía y ganadería, o bien se les educaba para desempeñar las triples obligaciones de los setitas: ser sacerdote, médico e instructor.

\par
%\textsuperscript{(850.6)}
\textsuperscript{76:3.10} Cuando penséis en los sacerdotes setitas, no confundáis a aquellos nobles y altruístas instructores de la salud y la religión, a aquellos verdaderos educadores, con los cleros envilecidos y comerciantes de las tribus posteriores y de las naciones circundantes. Sus conceptos religiosos de la Deidad y del universo eran avanzados y más o menos exactos, sus medidas de prevención sanitarias eran excelentes para su época, y sus métodos educativos jamás han sido superados desde entonces.

\section*{4. La raza violeta}
\par
%\textsuperscript{(850.7)}
\textsuperscript{76:4.1} Adán y Eva fueron los fundadores de la raza de hombres violetas, la novena raza humana que apareció en Urantia. Adán y sus descendientes tenían los ojos azules, y los pueblos violetas se caracterizaban por tener la tez clara y el cabello rubio ---amarillo, rojo y castaño.

\par
%\textsuperscript{(850.8)}
\textsuperscript{76:4.2} Eva no sufría dolores de parto, y tampoco los padecían las razas evolutivas primitivas. Sólo las razas mezcladas, surgidas de la unión de los hombres evolutivos con los noditas y más tarde con los adamitas, sufrieron los intensos dolores del parto.

\par
%\textsuperscript{(851.1)}
\textsuperscript{76:4.3} Adán y Eva, al igual que sus hermanos de Jerusem, obtenían su energía de una doble nutrición, manteniéndose a base de alimentos y de luz a la vez, con el complemento de ciertas energías superfísicas no reveladas en Urantia. Sus descendientes de Urantia no heredaron de sus padres el don de la absorción de la energía y de circulación de la luz. Poseían una sola circulación, el tipo humano de alimentación sanguínea. Eran deliberadamente mortales pero vivían mucho tiempo, aunque su longevidad tendía hacia las normas humanas con cada generación sucesiva.

\par
%\textsuperscript{(851.2)}
\textsuperscript{76:4.4} Adán y Eva y sus hijos de la primera generación no utilizaban la carne de los animales para alimentarse. Se mantenían totalmente a base de <<los frutos de los árboles>>\footnote{\textit{Frutos de los árboles}: Gn 1:29; 2:16; 3:2.}. Después de la primera generación, todos los descendientes de Adán empezaron a tomar productos lácteos, pero muchos de ellos continuaron con un régimen no carnívoro. Muchas tribus del sur con las que se unieron posteriormente tampoco eran carnívoras. Más tarde, la mayoría de estas tribus vegetarianas emigraron hacia el este y sobrevivieron en los pueblos actualmente mezclados de la India.

\par
%\textsuperscript{(851.3)}
\textsuperscript{76:4.5} Tanto la visión física como la visión espiritual de Adán y Eva eran muy superiores a la de los pueblos de hoy. Sus sentidos especiales eran mucho más agudos; eran capaces de ver a los intermedios y a las huestes angélicas, a los Melquisedeks y a Caligastia, el Príncipe caído que vino varias veces a conferenciar con su noble sucesor. Conservaron la capacidad de ver a estos seres celestiales durante más de cien años después de la falta. Estos sentidos especiales estaban menos aguzados en sus hijos y tendieron a disminuir con cada generación sucesiva.

\par
%\textsuperscript{(851.4)}
\textsuperscript{76:4.6} Los hijos adámicos tenían generalmente un Ajustador interior, puesto que todos poseían una capacidad indudable de supervivencia. Estos descendientes superiores no estaban tan sometidos al miedo como los hijos de la evolución. Las razas actuales de Urantia continúan teniendo tanto miedo porque vuestros antepasados recibieron muy poco plasma vital de Adán, debido al fracaso prematuro de los planes destinados al mejoramiento físico de las razas.

\par
%\textsuperscript{(851.5)}
\textsuperscript{76:4.7} Las células del cuerpo de los Hijos Materiales y de su progenie son mucho más resistentes a las enfermedades que las de los seres evolutivos originarios del planeta. Las células del cuerpo de las razas nativas son similares a los organismos vivientes microscópicos y ultramicroscópicos del planeta que producen las enfermedades. Estos hechos explican por qué los pueblos de Urantia tienen que hacer tantos esfuerzos en el campo científico para resistir tantos desórdenes físicos. Seríais mucho más resistentes a las enfermedades si vuestras razas llevaran más sangre adámica.

\par
%\textsuperscript{(851.6)}
\textsuperscript{76:4.8} Después de haberse establecido en el segundo jardín junto al
Éufrates, Adán decidió dejar tras él la mayor cantidad posible de su plasma vital para que el mundo se beneficiara después de su muerte. En consecuencia, Eva fue nombrada a la cabeza de una comisión de doce miembros para la mejora de la raza, y antes de la muerte de Adán, esta comisión había elegido a 1.682 mujeres del tipo más elevado de Urantia, y todas fueron fecundadas con el plasma vital adámico. Todos sus hijos llegaron hasta la madurez, a excepción de 112, de manera que el mundo se benefició así de la adición de 1.570 hombres y mujeres superiores. Aunque estas madres candidatas fueron elegidas entre todas las tribus circundantes y representaban a la mayor parte de las razas de la Tierra, la mayoría fue escogida entre los linajes superiores de los noditas, y formaron los orígenes iniciales de la poderosa raza andita. Estos niños nacieron y se criaron en el entorno tribal de sus madres respectivas\footnote{\textit{Elevación racial}: Gn 6:2.}.

\section*{5. La muerte de Adán y Eva}
\par
%\textsuperscript{(851.7)}
\textsuperscript{76:5.1} Poco tiempo después del establecimiento del segundo Edén, a Adán y Eva se les informó debidamente que su arrepentimiento era aceptable, y que, aunque estaban condenados a sufrir el destino de los mortales de su mundo, serían admitidos indudablemente en las filas de los supervivientes dormidos de Urantia. Creyeron plenamente en este evangelio de resurrección y rehabilitación que los Melquisedeks les proclamaron de manera tan conmovedora. Su transgresión había sido un error de juicio, y no el pecado de una rebelión consciente y deliberada.

\par
%\textsuperscript{(852.1)}
\textsuperscript{76:5.2} Cuando eran ciudadanos de Jerusem, Adán y Eva no tenían Ajustadores del Pensamiento, y tampoco estuvieron habitados por un Ajustador en Urantia cuando trabajaron en el primer jardín. Pero poco después de su degradación al estado mortal, se volvieron conscientes de una nueva presencia dentro de ellos, y cayeron en la cuenta de que el estado humano, acompañado de un arrepentimiento sincero, habían hecho posible que los Ajustadores vinieran a residir dentro de ellos. El hecho de saber que estaban habitados por un Ajustador animó enormemente a Adán y Eva durante el resto de sus vidas; sabían que habían fracasado como Hijos Materiales de Satania, pero también sabían que la carrera hacia el Paraíso permanecía abierta para ellos como hijos ascendentes del universo.

\par
%\textsuperscript{(852.2)}
\textsuperscript{76:5.3} Adán conocía la resurrección dispensacional que se había producido en el momento de su llegada al planeta, y creía que él y su compañera serían repersonalizados probablemente en conexión con la venida de la siguiente orden de filiación. No sabía que Miguel, el soberano de este universo, iba a aparecer tan pronto en Urantia; suponía que el siguiente Hijo que llegaría sería de la orden de los Avonales. Aún así, para Adán y Eva siempre fue un consuelo meditar sobre el único mensaje personal que recibieron de Miguel, aunque para ellos fuera un poco difícil de comprender. Este mensaje, entre otras expresiones de amistad y de aliento, decía: <<He tomado en consideración las circunstancias de vuestra falta; he recordado el deseo de vuestro corazón de ser siempre leales a la voluntad de mi Padre, y seréis llamados del abrazo del sueño mortal cuando yo llegue a Urantia, si los Hijos subordinados de mi universo no os envían a buscar antes de ese momento.>>

\par
%\textsuperscript{(852.3)}
\textsuperscript{76:5.4} Fue un gran misterio para Adán y Eva. En este mensaje podían comprender la promesa velada de una posible resurrección especial, y esta posibilidad les animó enormemente, pero no podían captar el significado de la indicación de que podrían descansar hasta el momento de una resurrección relacionada con la aparición personal de Miguel en Urantia. Así pues, la pareja edénica siempre proclamó que algún día vendría un Hijo de Dios, y a sus seres queridos comunicaron la creencia, o al menos la ardiente esperanza, de que el mundo de sus graves errores y de sus penas quizás se convertiría en la esfera donde el soberano de este universo decidiera actuar como Hijo donador del Paraíso. Parecía demasiado hermoso para ser verdad, pero Adán albergaba la idea de que Urantia, desgarrada por los conflictos, podría llegar a ser después de todo el mundo más afortunado del sistema de Satania, el planeta más envidiado de todo Nebadon.

\par
%\textsuperscript{(852.4)}
\textsuperscript{76:5.5} Adán vivió 530 años; murió de lo que se podría llamar vejez\footnote{\textit{Muerte de Adán}: Gn 5:5.}. Su mecanismo físico simplemente se desgastó; el proceso de desintegración le ganó terreno progresivamente al proceso de reparación, y el final inevitable llegó. Eva había muerto diecinueve años antes de una insuficiencia cardíaca. Los dos fueron enterrados en el centro del templo del servicio divino, que se había construido de acuerdo con sus planes poco después de haberse terminado la muralla de la colonia. Éste fue el origen de la costumbre de enterrar a los hombres y mujeres notables y piadosos bajo el suelo de los lugares de culto.

\par
%\textsuperscript{(852.5)}
\textsuperscript{76:5.6} El gobierno supermaterial de Urantia continuó bajo la dirección de los Melquisedeks, pero el contacto físico directo con las razas evolutivas se había roto. Los representantes físicos del gobierno del universo habían estado destacados en el planeta desde los tiempos lejanos de la llegada del estado mayor corpóreo del Príncipe Planetario, pasando por la época de Van y Amadón, hasta la llegada de Adán y Eva. Pero este régimen llegó a su fin con la falta adámica, después de haberse prolongado durante un período de más de cuatrocientos cincuenta mil años. En el ámbito espiritual, los ayudantes angélicos continuaron luchando en unión con los Ajustadores del Pensamiento, trabajando los dos heróicamente para salvar al individuo; pero ningún plan global para el bienestar a largo plazo del mundo se promulgó a los mortales de la Tierra hasta la llegada de Maquiventa Melquisedek en la época de Abraham. Con el poder, la paciencia y la autoridad de un Hijo de Dios, Maquiventa sentó las bases para la elevación ulterior y la rehabilitación espiritual de la desdichada Urantia.

\par
%\textsuperscript{(853.1)}
\textsuperscript{76:5.7} Sin embargo, la desgracia no ha sido el único destino de Urantia; este planeta ha sido también el más afortunado del universo local de Nebadon. Los urantianos deberían considerar como un beneficio que los desatinos de sus antepasados y los errores de los primeros gobernantes de este mundo sumieran al planeta en un estado de confusión tan desesperada, intensificada además por el mal y el pecado, que este mismo trasfondo de tinieblas atrajo tanto la atención de Miguel de Nebadon que escogió este mundo como escenario para revelar la personalidad amorosa del Padre que está en los cielos. No se trata de que Urantia necesitara a un Hijo Creador para poner en orden sus asuntos enredados, sino que el mal y el pecado en Urantia proporcionaron al Hijo Creador un trasfondo más llamativo para revelar el amor, la misericordia y la paciencia incomparables del Padre Paradisiaco.

\section*{6. La supervivencia de Adán y Eva}
\par
%\textsuperscript{(853.2)}
\textsuperscript{76:6.1} Adán y Eva se sumieron en su descanso mortal con una sólida fe en las promesas que les habían hecho los Melquisedeks de que algún día se despertarían del sueño de la muerte para volver a la vida en los mundos de las mansiones, unos mundos tan familiares para ellos en los tiempos anteriores a su misión en la carne física de la raza violeta de Urantia.

\par
%\textsuperscript{(853.3)}
\textsuperscript{76:6.2} No permanecieron mucho tiempo en el olvido del sueño inconsciente de los mortales del reino. Al tercer día de la muerte de Adán, dos días después de su respetuoso entierro, Lanaforge ordenó que se pasara una lista especial para los supervivientes notables de la falta adámica en Urantia. Sus órdenes, apoyadas por el Altísimo de Edentia en funciones y ratificadas por el Unión de los Días de Salvington, que actuaba en nombre de Miguel, fueron entregadas a Gabriel. De conformidad con este mandato de resurrección especial, el número veintiséis de la serie de Urantia, Adán y Eva fueron repersonalizados y reconstruídos en las salas de resurrección de los mundos de las mansiones de Satania junto con 1.316 asociados suyos de la experiencia del primer jardín. Muchas otras almas leales ya habían sido trasladadas en el momento de la llegada de Adán, que estuvo acompañada de un juicio dispensacional de los supervivientes dormidos y de los ascendentes vivientes cualificados.

\par
%\textsuperscript{(853.4)}
\textsuperscript{76:6.3} Adán y Eva pasaron rápidamente por los mundos de ascensión progresiva hasta que alcanzaron la ciudadanía de Jerusem, convirtiéndose una vez más en residentes de su planeta de origen, pero esta vez como miembros de una orden diferente de personalidades del universo. Habían partido de Jerusem como ciudadanos permanentes ---como Hijos de Dios, y volvieron como ciudadanos ascendentes--- como hijos del hombre. Fueron destinados inmediatamente al servicio de Urantia en la capital del sistema, y más tarde pasaron a ser miembros del consejo de los veinticuatro que funciona actualmente como órgano de control consultivo de Urantia.

\par
%\textsuperscript{(854.1)}
\textsuperscript{76:6.4} Así termina la historia del Adán y la Eva Planetarios de Urantia, una historia de pruebas, tragedias y triunfos, al menos de triunfo personal para vuestro Hijo y vuestra Hija Materiales bienintencionados pero engañados; y al final será sin duda una historia de triunfo último para su mundo y sus habitantes sacudidos por la rebelión y acosados por el mal. En resumidas cuentas, Adán y Eva contribuyeron poderosamente a favorecer la civilización y a acelerar el progreso biológico de la raza humana. Dejaron una gran cultura en la Tierra, pero esta civilización tan avanzada no pudo sobrevivir en presencia de la dilución prematura y la sumersión final de la herencia adámica. Son los pueblos los que hacen las civilizaciones; las civilizaciones no hacen a los pueblos.

\par
%\textsuperscript{(854.2)}
\textsuperscript{76:6.5} [Presentado por Solonia, la <<voz seráfica en el Jardín>>.]


\chapter{Documento 77. Las criaturas intermedias}
\par
%\textsuperscript{(855.1)}
\textsuperscript{77:0.1} LA MAYORÍA de los mundos habitados de Nebadon albergan uno o más grupos de seres singulares que existen en un nivel de actividad de los seres vivientes situado aproximadamente a medio camino entre el nivel de los mortales de los planetas y el de las órdenes angélicas, y por eso los llamamos criaturas \textit{intermedias}. Parecen ser un accidente del tiempo, pero se encuentran tan extendidos y son unos colaboradores tan valiosos, que todos los hemos aceptado desde hace mucho tiempo como uno de los grupos esenciales de nuestro servicio planetario combinado.

\par
%\textsuperscript{(855.2)}
\textsuperscript{77:0.2} En Urantia funcionan dos órdenes distintas de intermedios: el cuerpo primario o más antiguo, que nació en los tiempos de Dalamatia, y el grupo secundario o más joven, cuyo origen se remonta a la época de Adán.

\section*{1. Los intermedios primarios}
\par
%\textsuperscript{(855.3)}
\textsuperscript{77:1.1} Los intermedios primarios de Urantia tienen su génesis en una asociación singular entre lo material y lo espiritual. Sabemos que existen criaturas similares en otros mundos y en otros sistemas, pero se han originado mediante técnicas diferentes.

\par
%\textsuperscript{(855.4)}
\textsuperscript{77:1.2} Es conveniente tener siempre presente que las donaciones sucesivas de los Hijos de Dios en un planeta evolutivo producen unos cambios notables en la economía espiritual de ese mundo, y a veces modifican tanto el funcionamiento de la asociación entre los agentes espirituales y materiales de un planeta, que se crean situaciones realmente difíciles de comprender. El estatus de los cien miembros corpóreos del estado mayor del Príncipe Caligastia ilustra precisamente una interasociación singular de este tipo: Como ciudadanos morontiales ascendentes de Jerusem, eran criaturas supermateriales sin prerrogativas reproductoras. Como servidores planetarios descendentes en Urantia, eran criaturas materiales sexuadas capaces de procrear una descendencia material
(tal como algunos de ellos hicieron más tarde). Lo que no podemos explicar de una manera satisfactoria es cómo estos cien miembros pudieron desempeñar la función de padres en un nivel supermaterial, pero esto es exactamente lo que sucedió. La unión supermaterial (no sexual) de un hombre y una mujer del estado mayor corpóreo tuvo como resultado la aparición del primogénito de los intermedios primarios.

\par
%\textsuperscript{(855.5)}
\textsuperscript{77:1.3} Inmediatamente se descubrió que una criatura de esta índole, a medio camino entre el nivel humano y el nivel angélico, sería de una gran utilidad para llevar adelante los asuntos de la sede del Príncipe; en consecuencia, cada pareja del estado mayor corpóreo recibió la autorización de engendrar un ser similar. Este esfuerzo tuvo como resultado el primer grupo de cincuenta criaturas intermedias.

\par
%\textsuperscript{(855.6)}
\textsuperscript{77:1.4} Después de observar durante un año el trabajo de este grupo singular, el Príncipe Planetario autorizó la reproducción sin restricción de los intermedios. Este plan se llevó a cabo mientras duró la facultad de crear, y así es como surgió el cuerpo original de 50.000 intermedios.

\par
%\textsuperscript{(856.1)}
\textsuperscript{77:1.5} Entre el nacimiento de cada intermedio transcurría un período de medio año, y cuando cada pareja hubo engendrado mil seres de este tipo, ya no nació ninguno más. No existe ninguna explicación válida que nos indique por qué se agotó este poder cuando apareció el milésimo descendiente. Todos los intentos posteriores resultaron un fracaso.

\par
%\textsuperscript{(856.2)}
\textsuperscript{77:1.6} Estas criaturas constituyeron el cuerpo que recogía la información para la administración del Príncipe. Se diseminaron por todas partes, estudiando y observando a las razas del mundo, y prestando otros servicios inestimables al Príncipe y a su estado mayor en la tarea de influir sobre la sociedad humana que se encontraba alejada de la sede planetaria.

\par
%\textsuperscript{(856.3)}
\textsuperscript{77:1.7} Este régimen continuó hasta los trágicos días de la rebelión planetaria, que cogió en la trampa a un poco más de las cuatro quintas partes de los intermedios primarios. El cuerpo leal se puso al servicio de los síndicos Melquisedeks y funcionó bajo la dirección titular de Van hasta la época de Adán.

\section*{2. La raza nodita}
\par
%\textsuperscript{(856.4)}
\textsuperscript{77:2.1} Aunque ésta es la narración del origen, la naturaleza y las funciones de las criaturas intermedias de Urantia, el parentesco entre las dos órdenes ---la primaria y la secundaria--- hace necesario interrumpir en este punto la historia de los intermedios primarios para poder seguir el linaje descendente de los miembros rebeldes del estado mayor corpóreo del Príncipe Caligastia, desde los tiempos de la rebelión planetaria hasta la época de Adán. Esta línea hereditaria fue la que proporcionó, durante los primeros tiempos del segundo jardín, la mitad de los antepasados de la orden secundaria de criaturas intermedias.

\par
%\textsuperscript{(856.5)}
\textsuperscript{77:2.2} Los miembros corpóreos del estado mayor del Príncipe habían sido materializados como criaturas sexuadas para que pudieran participar en el proyecto de procrear una descendencia que incorporara las cualidades combinadas de su orden especial unidas a las de los linajes seleccionados de las tribus andónicas, y todo ello con miras a la aparición posterior de Adán. Los Portadores de Vida habían proyectado un nuevo tipo de mortales que englobarían la unión de los descendientes conjuntos del estado mayor del Príncipe con los hijos de Adán y Eva de la primera generación. Habían diseñado así un proyecto que contemplaba un nuevo tipo de criaturas planetarias, y esperaban que se convertirían en los dirigentes e instructores de la sociedad humana. Estos seres estaban destinados a la soberanía social, no a la soberanía civil. Pero como este proyecto fracasó casi por completo, nunca sabremos la clase de aristocracia de dirigentes benéficos y el tipo de cultura incomparable que se perdió así en Urantia. Porque cuando los miembros del estado mayor corpóreo se reprodujeron más tarde, lo hicieron después de la rebelión y tras haber sido privados de su conexión con las corrientes vitales del sistema.

\par
%\textsuperscript{(856.6)}
\textsuperscript{77:2.3} La era posterior a la rebelión en Urantia fue testigo de muchos sucesos inhabituales. Una gran civilización ---la cultura de Dalamatia--- se desmoronaba. <<Los nefilim (los noditas) estaban en la Tierra en aquellos días, y cuando estos hijos de los dioses fueron hasta las hijas de los hombres y tuvieron relaciones con ellas, sus hijos fueron `los poderosos hombres de la antig\"uedad', `los varones de renombre'>>\footnote{\textit{Hijos de los dioses, hijas de los hombres}: Gn 6:4.}. Aunque no eran del todo <<hijos de los dioses>>, el estado mayor y sus primeros descendientes fueron considerados como tales por los mortales evolutivos de aquellos tiempos lejanos; incluso su estatura fue exagerada por la tradición\footnote{\textit{Tierra de gigantes}: Dt 2:20; 3:13; Jos 12:4.}. Éste es, pues, el origen del relato folclórico casi universal de los dioses que descendieron a la Tierra y engendraron allí, con las hijas de los hombres, una antigua raza de héroes. Toda esta leyenda se volvió aún más confusa con las mezclas raciales de los adamitas que nacieron posteriormente en el segundo jardín.

\par
%\textsuperscript{(857.1)}
\textsuperscript{77:2.4} Puesto que los cien miembros corpóreos del estado mayor del Príncipe tenían el plasma germinal de los linajes humanos andónicos, si emprendían la reproducción sexual se podía esperar de manera natural que sus descendientes se parecieran por completo a los hijos de los otros padres andonitas. Pero cuando los sesenta rebeldes del estado mayor, los seguidores de Nod, emprendieron de hecho la reproducción sexual, sus hijos resultaron ser muy superiores en casi todos los aspectos tanto a los pueblos andonitas como a los pueblos sangiks. Esta superioridad inesperada no solamente se refería a sus cualidades físicas e intelectuales, sino también a sus capacidades espirituales.

\par
%\textsuperscript{(857.2)}
\textsuperscript{77:2.5} Estas características mutantes que aparecieron en la primera generación nodita se debían a ciertos cambios que se habían producido en la configuración y en los componentes químicos de los factores hereditarios del plasma germinal andónico. Estos cambios habían sido causados por la presencia, en el cuerpo de los miembros del estado mayor, de los poderosos circuitos de conservación de la vida del sistema de Satania. Estos circuitos vitales hicieron que los cromosomas del modelo especializado de Urantia se reorganizaran más a la manera de los modelos de la especialización normalizada en Satania de las manifestaciones vitales decretadas para Nebadon. La técnica de esta metamorfosis del plasma germinal, producida por la acción de las corrientes vitales del sistema, se parece a los procedimientos que emplean los científicos de Urantia para modificar el plasma germinal de las plantas y los animales mediante la utilización de los rayos X.

\par
%\textsuperscript{(857.3)}
\textsuperscript{77:2.6} Los pueblos noditas\footnote{\textit{Pueblos noditas}: Gn 4:16.} surgieron así de ciertas modificaciones particulares e inesperadas que se produjeron en el plasma vital que los cirujanos de Avalon habían trasladado desde el cuerpo de los cooperadores andonitas hasta el de los miembros del estado mayor corpóreo.

\par
%\textsuperscript{(857.4)}
\textsuperscript{77:2.7} Se debe recordar que los cien andonitas que contribuyeron con su plasma germinal recibieron a su vez el complemento orgánico del árbol de la vida, de manera que las corrientes vitales de Satania se extendieron igualmente por sus cuerpos. Los cuarenta y cuatro andonitas modificados que siguieron al estado mayor en la rebelión también se casaron entre ellos e hicieron una gran contribución a los mejores linajes del pueblo nodita.

\par
%\textsuperscript{(857.5)}
\textsuperscript{77:2.8} Estos dos grupos, que comprendían 104 individuos portadores del plasma germinal andonita modificado, fueron los antepasados de los noditas, la octava raza que apareció en Urantia. Esta nueva característica de la vida humana en Urantia representa otra fase del proceso del plan original consistente en utilizar este planeta como mundo de modificación de la vida, salvo que en esta ocasión se trató de un acontecimiento no previsto.

\par
%\textsuperscript{(857.6)}
\textsuperscript{77:2.9} Los noditas\footnote{\textit{Pueblos noditas}: Gn 4:16.} de pura cepa eran una raza magnífica, pero se mezclaron gradualmente con los pueblos evolutivos de la Tierra, y al poco tiempo se había producido una gran degeneración. Diez mil años después de la rebelión habían perdido tanto terreno que la duración media de su vida sólo era un poco superior a la de las razas evolutivas.

\par
%\textsuperscript{(857.7)}
\textsuperscript{77:2.10} Cuando los arqueólogos desentierran los registros en tablillas de arcilla de los últimos descendientes sumerios de los noditas, descubren unas listas de reyes sumerios que se remontan a varios miles de años; a medida que estos anales se internan en el pasado, el reinado de cada rey se prolonga desde unos veinticinco o treinta años hasta ciento cincuenta años o más. Esta prolongación del reinado de estos reyes antiguos significa que algunos de los primeros jefes noditas (los descendientes inmediatos del estado mayor del Príncipe) vivieron más tiempo que sus sucesores más recientes, y también indica un esfuerzo por remontar sus dinastías hasta la época de Dalamatia.

\par
%\textsuperscript{(857.8)}
\textsuperscript{77:2.11} Los datos sobre estos personajes tan longevos se deben también a la confusión entre los meses y los años como períodos de tiempo\footnote{\textit{Longitud de la vida}: Gn 5:5,8,11,14; 5:17,20,23,27; 5:31; 9:29; 11:10-26.}. Este hecho también se puede observar en la genealogía bíblica de Abraham y en los archivos primitivos de los chinos. La confusión entre el mes, o período de veintiocho días, y el año de más de trescientos cincuenta días que se introdujo más tarde, es responsable de la tradición de estas vidas humanas tan largas. Existen relatos de un hombre que vivió más de novecientos <<años>>. Este período no representa en realidad más de setenta años, pero estas vidas fueron consideradas durante siglos como muy largas, y más adelante se las denominó como <<sesenta años más diez>>\footnote{\textit{Sesenta años más diez}: Sal 90:10.}.

\par
%\textsuperscript{(858.1)}
\textsuperscript{77:2.12} El cálculo del tiempo por meses de veintiocho días sobrevivió mucho tiempo después de la época de Adán. Pero cuando los egipcios emprendieron la reforma del calendario, hace aproximadamente siete mil años, lo hicieron con una gran precisión, introduciendo el año de 365 días.

\section*{3. La torre de Babel}
\par
%\textsuperscript{(858.2)}
\textsuperscript{77:3.1} Después de la sumersión de Dalamatia, los noditas se dirigieron hacia el norte y el este y fundaron enseguida la nueva ciudad de Dilmun como su centro racial y cultural. Cerca de cincuenta mil años después de la muerte de Nod, los descendientes del estado mayor del Príncipe se habían vuelto demasiado numerosos como para poder subsistir en las tierras que rodeaban directamente su nueva ciudad de Dilmun. Después de extenderse hacia el exterior para casarse con las tribus andonitas y sangiks contiguas a sus fronteras, a sus dirigentes se les ocurrió que había que hacer algo para preservar su unidad racial. Por consiguiente, se convocó un consejo de tribus, y después de muchas deliberaciones, se aceptó el plan de Bablot, un descendiente de Nod.

\par
%\textsuperscript{(858.3)}
\textsuperscript{77:3.2} Bablot proponía erigir un templo pretencioso de glorificación racial en el centro del territorio que ocupaban en aquel entonces. Este templo debía tener una torre como el mundo no hubiera visto nunca otra igual. Tenía que ser un enorme monumento conmemorativo a su grandeza pasada. Muchos de ellos deseaban que este monumento se erigiera en Dilmun, pero otros afirmaban, recordando las tradiciones del hundimiento de Dalamatia, su primera capital, que una estructura tan grande debería colocarse a una distancia prudencial de los peligros del mar.

\par
%\textsuperscript{(858.4)}
\textsuperscript{77:3.3} Bablot tenía pensado que los nuevos edificios se convertirían en el núcleo del futuro centro de la cultura y la civilización noditas. Su opinión terminó por prevalecer, y se empezó a construir de acuerdo con sus planes. La nueva ciudad se llamaría \textit{Bablot} en honor al arquitecto y constructor de la torre. Este lugar se conoció más adelante con el nombre de Bablod, y finalmente como Babel\footnote{\textit{Torre de Babel}: Gn 11:1-9.}.

\par
%\textsuperscript{(858.5)}
\textsuperscript{77:3.4} Pero la opinión de los noditas continuaba estando un poco dividida en cuanto a los planes y la finalidad de esta empresa. Sus dirigentes tampoco estaban totalmente de acuerdo en cuanto a los planos de la construcción y la utilización de los edificios una vez construidos. Después de cuatro años y medio de trabajos, se originó una gran discusión sobre el objeto y el motivo de la construcción de la torre. La controversia se puso tan enconada que se detuvo todo el trabajo. Los portadores de alimentos difundieron la noticia de la disensión, y un gran número de tribus empezaron a reunirse en el lugar de las obras. Se proponían tres puntos de vista diferentes sobre la finalidad de la construcción de la torre.

\par
%\textsuperscript{(858.6)}
\textsuperscript{77:3.5} 1. El grupo más grande, aproximadamente la mitad, deseaba que la torre se construyera como un monumento conmemorativo a la historia y la superioridad racial de los noditas. Pensaban que debía ser una estructura grande e imponente que provocara la admiración de todas las generaciones futuras.

\par
%\textsuperscript{(858.7)}
\textsuperscript{77:3.6} 2. La siguiente facción en orden de importancia quería que la torre se destinara a conmemorar la cultura de Dilmun. Preveían que Bablot se convertiría en un gran centro de comercio, arte y manufactura.

\par
%\textsuperscript{(859.1)}
\textsuperscript{77:3.7} 3. El contingente más pequeño y minoritario sostenía que la construcción de la torre ofrecía una oportunidad para expiar la locura de sus progenitores que habían participado en la rebelión de Caligastia. Opinaban que la torre debería consagrarse a la adoración del Padre de todos, que toda la finalidad de la nueva ciudad debería consistir en sustituir a Dalamatia ---en funcionar como un centro cultural y religioso para los bárbaros de los alrededores.

\par
%\textsuperscript{(859.2)}
\textsuperscript{77:3.8} El grupo religioso fue rápidamente derrotado por votación. La mayoría rechazó la doctrina de que sus antepasados habían sido culpables de rebelión; les indignaba este estigma racial. Habiéndose librado de uno de los tres puntos de vista de la discusión, y no logrando arreglar los otros dos por medio del debate, recurrieron a la guerra. Los seguidores de la religión, los no combatientes, huyeron a sus casas del sur, mientras que sus compañeros lucharon hasta destruirse casi por completo.

\par
%\textsuperscript{(859.3)}
\textsuperscript{77:3.9} Hace unos doce mil años se efectuó un segundo intento por construir la torre de Babel. Las razas mezcladas de los anditas (noditas y adamitas) se propusieron levantar un nuevo templo sobre las ruinas del primer edificio, pero la empresa no recibió el apoyo suficiente; sucumbió bajo el peso de su propia pretensión. Esta región se conoció durante mucho tiempo como la tierra de Babel.

\section*{4. Los centros de civilización noditas}
\par
%\textsuperscript{(859.4)}
\textsuperscript{77:4.1} La consecuencia inmediata del conflicto de aniquilación recíproca debido a la torre de Babel fue la dispersión de los noditas. Esta guerra interna redujo considerablemente el número de los noditas más puros, y fue responsable en muchos aspectos de que no lograran establecer una gran civilización preadámica. A partir de este momento, la cultura nodita declinó durante más de ciento veinte mil años, hasta que fue elevada por la inyección adámica. Pero incluso en los tiempos de Adán, los noditas continuaban siendo un pueblo capaz. Muchos de sus descendientes mixtos figuraron entre los constructores del Jardín, y varios capitanes de los grupos de Van eran noditas. Algunos de los cerebros más competentes que prestaron sus servicios en el estado mayor de Adán pertenecían a esta raza.

\par
%\textsuperscript{(859.5)}
\textsuperscript{77:4.2} Inmediatamente después del conflicto de Bablot se establecieron tres de los cuatro grandes centros noditas:

\par
%\textsuperscript{(859.6)}
\textsuperscript{77:4.3} 1. \textit{Los noditas occidentales o sirios}. Los restos del grupo nacionalista, o partidarios del monumento racial, se dirigieron hacia el norte donde se unieron con los andonitas y fundaron los centros noditas ulteriores del noroeste de Mesopotamia. Éste fue el grupo más numeroso de noditas en dispersión, y contribuyeron mucho a la aparición de la estirpe asiria posterior.

\par
%\textsuperscript{(859.7)}
\textsuperscript{77:4.4} 2. \textit{Los noditas orientales o elamitas}. Los defensores de la cultura y del comercio emigraron en grandes cantidades hacia Elam en el este y allí se unieron con las tribus sangiks mestizas. Los elamitas de hace treinta o cuarenta mil años se habían vuelto ampliamente de carácter sangik, aunque continuaron manteniendo una civilización superior a la de los bárbaros circundantes.

\par
%\textsuperscript{(859.8)}
\textsuperscript{77:4.5} Después del establecimiento del segundo jardín, era habitual referirse a esta colonia nodita cercana como <<la tierra de Nod>>\footnote{\textit{La tierra de Nod}: Gn 4:16.}. Durante el largo período de paz relativa entre este grupo de noditas y los adamitas, las dos razas se mezclaron ampliamente, porque los Hijos de Dios (los adamitas) cogieron cada vez más la costumbre de casarse con las hijas de los hombres (los noditas)\footnote{\textit{Los adamitas se casan con los noditas}: Gn 6:2.}.

\par
%\textsuperscript{(860.1)}
\textsuperscript{77:4.6} 3. \textit{Los noditas centrales o presumerios}. En la desembocadura de los ríos Tigris y Éufrates hubo un pequeño grupo que conservó mejor su integridad racial. Sobrevivieron durante miles de años y proporcionaron con el tiempo los antepasados noditas que se mezclaron con los adamitas para fundar los pueblos sumerios de los tiempos históricos.

\par
%\textsuperscript{(860.2)}
\textsuperscript{77:4.7} Todo esto explica la manera en que los sumerios aparecieron tan repentina y misteriosamente en la esfera de acción de Mesopotamia. Los investigadores nunca podrán descubrir el rastro de estas tribus y seguirlo hasta el principio de los sumerios, que tuvieron su origen hace doscientos mil años después de la sumersión de Dalamatia. Sin un rastro de su origen en otras partes del mundo, estas tribus antiguas aparecieron repentinamente sobre el horizonte de la civilización con una cultura superior y plenamente desarrollada, que incluía templos, trabajo de los metales, agricultura, ganadería, alfarería, tejeduría, derecho mercantil, códigos civiles, un ceremonial religioso y un antiguo sistema de escritura. Al principio de la era histórica, hacía mucho tiempo que habían perdido el alfabeto de Dalamatia, y habían adoptado el sistema de escritura particular originario de Dilmun. El idioma sumerio, aunque prácticamente perdido para el mundo, no era semítico; tenía muchas cosas en común con las llamadas lenguas arias.

\par
%\textsuperscript{(860.3)}
\textsuperscript{77:4.8} Los escritos detallados que dejaron los sumerios describen el emplazamiento de una colonia extraordinaria situada en el Golfo Pérsico cerca de la antigua ciudad de Dilmun. Los egipcios llamaban Dilmat a esta ciudad de antigua gloria, mientras que los sumerios adamizados posteriores confundieron la primera y la segunda ciudad noditas con Dalamatia, y llamaron Dilmun a las tres. Los arqueólogos ya han encontrado estas antiguas tablillas sumerias de arcilla que hablan de este paraíso terrenal <<donde los dioses bendijeron por primera vez a la humanidad con el ejemplo de una vida civilizada y culta>>. Estas tablillas que describen a Dilmun, el paraíso de los hombres y de Dios, descansan ahora en el silencio de las estanterías polvorientas de muchos museos.

\par
%\textsuperscript{(860.4)}
\textsuperscript{77:4.9} Los sumerios conocían muy bien el primero y el segundo Edén, pero a pesar del gran número de matrimonios mixtos que tuvieron con los adamitas, continuaron considerando a los habitantes del jardín que vivían en el norte como una raza extraña. El orgullo que sentían los sumerios de la cultura nodita más antigua les indujo a no hacer caso de estas nuevas perspectivas de gloria, inclinándose a favor de la grandeza y las tradiciones paradisiacas de la ciudad de Dilmun.

\par
%\textsuperscript{(860.5)}
\textsuperscript{77:4.10} 4. \textit{Los noditas y amadonitas del norte ---los vanitas}. Este grupo surgió antes del conflicto de Bablot. Estos noditas más septentrionales descendían de aquellos que se habían separado de la dirección de Nod y sus sucesores para unirse a Van y Amadón.

\par
%\textsuperscript{(860.6)}
\textsuperscript{77:4.11} Algunos de los primeros asociados de Van se instalaron posteriormente cerca de las orillas del lago que aún lleva su nombre, y sus tradiciones nacieron alrededor de este lugar. El Ararat se convirtió en su montaña sagrada, que para los vanitas más recientes tuvo casi el mismo significado que el Monte Sinaí para los hebreos. Hace diez mil años, los antepasados vanitas de los asirios enseñaban que su ley moral de siete mandamientos había sido entregada a Van por los Dioses en el Monte Ararat. Creían firmemente que Van y su asociado Amadón habían sido sacados vivos del planeta mientras estaban en lo alto de la montaña dedicados a la adoración.

\par
%\textsuperscript{(860.7)}
\textsuperscript{77:4.12} El Monte Ararat era la montaña sagrada del norte de Mesopotamia, y como una gran parte de vuestras tradiciones sobre aquellos tiempos antiguos fue tomada en conexión con la historia babilónica del diluvio, no es de extrañar que el Monte Ararat y su región se entrelazaran posteriormente en la historia judía de Noé y el diluvio universal.

\par
%\textsuperscript{(860.8)}
\textsuperscript{77:4.13} Hacia el año 35.000 a. de J.C., Adanson visitó una de las antiguas colonias vanitas más orientales para fundar allí su centro de civilización.

\section*{5. Adanson y Ratta}
\par
%\textsuperscript{(861.1)}
\textsuperscript{77:5.1} Después de describir los antecedentes noditas del linaje de los intermedios secundarios, esta narración va a tratar ahora de la mitad adámica de dichos antepasados, porque los intermedios secundarios son también nietos de Adanson, el primogénito de la raza violeta de Urantia.

\par
%\textsuperscript{(861.2)}
\textsuperscript{77:5.2} Adanson formaba parte de aquel grupo de hijos de Adán y Eva que escogieron permanecer en la Tierra con su padre y su madre. Pues bien, este hijo mayor de Adán había escuchado a menudo a Van y Amadón contar la historia de su hogar en las tierras altas del norte, y algún tiempo después del establecimiento del segundo jardín decidió ir en busca de esta tierra de sus sueños juveniles.

\par
%\textsuperscript{(861.3)}
\textsuperscript{77:5.3} Adanson tenía entonces 120 años y había sido padre de treinta y dos hijos de pura sangre violeta en el primer jardín. Quería quedarse con sus padres y ayudarlos a preparar el segundo jardín, pero estaba profundamente perturbado por la pérdida de su compañera y de sus hijos, que habían elegido todos ir a Edentia con los otros hijos adámicos que escogieron convertirse en los pupilos de los Altísimos.

\par
%\textsuperscript{(861.4)}
\textsuperscript{77:5.4} Adanson no quería abandonar a sus padres en Urantia, estaba poco dispuesto a huir de las dificultades y los peligros, pero opinaba que las relaciones en el segundo jardín eran muy poco satisfactorias. Se esforzó mucho por promover las actividades iniciales de defensa y construcción, pero decidió marcharse hacia el norte en la primera ocasión. Aunque la despedida fue muy agradable, Adán y Eva estaban muy apenados por la pérdida de su hijo mayor, porque se aventurara en un mundo extraño y hostil de donde temían que no regresara nunca.

\par
%\textsuperscript{(861.5)}
\textsuperscript{77:5.5} Un grupo de veintisiete compañeros siguió a Adanson en su viaje hacia el norte en busca de los pueblos de sus fantasías infantiles. En poco más de tres años, el grupo encontró realmente el objetivo de su aventura, y Adanson descubrió entre aquella gente a una hermosa y maravillosa mujer de veinte años que afirmaba ser la última descendiente de pura cepa del estado mayor del Príncipe. Esta mujer, llamada Ratta, decía que todos sus antepasados descendían de dos miembros apóstatas del estado mayor del Príncipe. Ella era la última de su raza, pues no tenía hermanos ni hermanas vivos. Casi había decidido no casarse, casi había resuelto morir sin descendencia, pero se enamoró del majestuoso Adanson. Cuando oyó la historia del Edén y la manera en que las predicciones de Van y Amadón se habían hecho realidad, cuando escuchó el relato de la falta del Jardín, un solo pensamiento ocupó su mente ---el de casarse con este hijo y heredero de Adán. La idea maduró rápidamente dentro de Adanson, y en poco más de tres meses se casaron.

\par
%\textsuperscript{(861.6)}
\textsuperscript{77:5.6} Adanson y Ratta tuvieron una familia de sesenta y siete hijos. Dieron origen a un gran linaje de dirigentes del mundo, pero hicieron algo más. Conviene recordar que estos dos seres eran realmente superhumanos. Cada cuarto hijo que nacía era de una clase única: a menudo se volvía invisible. Nunca había ocurrido una cosa así en la historia del mundo. Ratta estaba profundamente perturbada ---e incluso se volvió supersticiosa--- pero Adanson conocía bien la existencia de los intermedios primarios, y llegó a la conclusión de que algo similar se estaba produciendo delante de sus ojos. Cuando nació el segundo hijo con este comportamiento extraño, decidió casarlos, pues uno era varón y el otro hembra, y éste es el origen de la orden de los intermedios secundarios. En menos de cien años, y antes de que cesara este fenómeno, habían nacido casi dos mil de ellos.

\par
%\textsuperscript{(862.1)}
\textsuperscript{77:5.7} Adanson vivió 396 años. Volvió muchas veces a visitar a su padre y a su madre. Cada siete años viajaba con Ratta hacia el sur para ir al segundo jardín, y entretanto los intermedios lo mantenían informado sobre el bienestar de su pueblo. Durante la vida de Adanson prestaron un gran servicio en la construcción de un nuevo centro mundial independiente a favor de la verdad y la rectitud.

\par
%\textsuperscript{(862.2)}
\textsuperscript{77:5.8} Adanson y Ratta tuvieron así a su disposición este cuerpo de asistentes maravillosos que trabajó con ellos durante sus largas vidas, ayudándoles a propagar una verdad avanzada y a difundir unos criterios superiores de vida espiritual, intelectual y física. Los resultados de este esfuerzo por mejorar el mundo nunca fueron completamente eclipsados por los retrocesos posteriores.

\par
%\textsuperscript{(862.3)}
\textsuperscript{77:5.9} Los adansonitas mantuvieron una cultura elevada durante cerca de siete mil años a partir de la época de Adanson y Ratta. Más tarde se mezclaron con los noditas y andonitas vecinos, y fueron también incluídos entre los <<poderosos hombres de la antig\"uedad>>\footnote{\textit{Poderosos hombres de la antig\"uedad}: Gn 6:4.}. Algunos progresos de aquella época sobrevivieron y se volvieron una parte latente del potencial cultural que más tarde se convirtió en la civilización europea.

\par
%\textsuperscript{(862.4)}
\textsuperscript{77:5.10} Este centro de civilización estaba situado en la región que se encuentra al este del extremo meridional del Mar Caspio, cerca del Kopet Dagh. Los vestigios de lo que en otro tiempo fue la sede adansonita de la raza violeta se encuentran a poca altura de las estribaciones del Turquestán. En estos parajes de las tierras altas, situados en un antiguo y estrecho cinturón fértil emplazado en las estribaciones más bajas de la cordillera del Kopet, surgieron sucesivamente en diversos períodos cuatro culturas distintas, fomentadas respectivamente por cuatro grupos diferentes de descendientes de Adanson. El segundo de estos grupos fue el que emigró hacia el oeste hasta Grecia y las islas del Mediterráneo. El resto de los descendientes de Adanson emigraron hacia el norte y el oeste, entrando en Europa con el linaje mixto de la última oleada andita que salió de Mesopotamia, y también figuraron entre los invasores andita-arios de la India.

\section*{6. Los intermedios secundarios}
\par
%\textsuperscript{(862.5)}
\textsuperscript{77:6.1} Aunque los intermedios primarios tuvieron un origen casi superhumano, la orden secundaria es la progenie de la raza adámica pura unida con una descendiente humanizada de unos antepasados comunes a los progenitores del cuerpo más antiguo.

\par
%\textsuperscript{(862.6)}
\textsuperscript{77:6.2} Entre los hijos de Adanson, los progenitores peculiares de los intermedios secundarios fueron exactamente dieciséis. Estos hijos singulares estaban divididos por igual entre los dos sexos, y cada pareja era capaz de engendrar un intermedio secundario cada setenta días mediante una técnica combinada de unión sexual y no sexual. Este fenómeno nunca había sido posible en la Tierra antes de esta época, ni ha vuelto a producirse desde entonces.

\par
%\textsuperscript{(862.7)}
\textsuperscript{77:6.3} Estos dieciséis hijos vivieron y murieron como los mortales del planeta (a excepción de sus características especiales), pero sus descendientes, cuya fuente de energía es la electricidad, viven de manera indefinida, sin estar sometidos a las limitaciones de la carne mortal.

\par
%\textsuperscript{(862.8)}
\textsuperscript{77:6.4} Cada una de las ocho parejas engendró finalmente 248 intermedios, surgiendo así a la existencia el cuerpo secundario original de 1.984 miembros. Existen ocho subgrupos de intermedios secundarios. Se les denomina a-b-c el primero, el segundo, el tercero, y así sucesivamente. Y luego están d-e-f el primero, el segundo, y así sucesivamente.

\par
%\textsuperscript{(862.9)}
\textsuperscript{77:6.5} Después de la falta de Adán, los intermedios primarios regresaron al servicio de los síndicos Melquisedeks; el grupo secundario permaneció ligado al centro de Adanson hasta la muerte de éste. Treinta y tres de estos intermedios secundarios, los jefes de su organización cuando murió Adanson, intentaron dar un giro a la orden entera para ponerla al servicio de los Melquisedeks y unirse así al cuerpo primario. Pero como no lograron realizar este proyecto, abandonaron a sus compañeros y pasaron en masa al servicio de los síndicos planetarios.

\par
%\textsuperscript{(863.1)}
\textsuperscript{77:6.6} Después de la muerte de Adanson, el resto de los intermedios secundarios ejerció una extraña influencia desorganizada e independiente en Urantia. Desde aquel momento, y hasta la época de Maquiventa Melquisedek, llevaron una existencia irregular y desorganizada. Este Melquisedek los puso parcialmente bajo control, pero continuaron produciendo muchos perjuicios hasta los tiempos de Cristo Miguel. Durante su estancia en la Tierra, todos tomaron sus decisiones definitivas en cuanto a su destino futuro, y la mayoría leal se puso entonces bajo el mando de los intermedios primarios.

\section*{7. Los intermedios rebeldes}
\par
%\textsuperscript{(863.2)}
\textsuperscript{77:7.1} La mayoría de los intermedios primarios cayeron en el pecado en la época de la rebelión de Lucifer. Cuando se hizo el cálculo de la devastación de la rebelión planetaria se descubrió, entre otras pérdidas, que 40.119 intermedios primarios, de los 50.000 originales, se habían unido a la secesión de Caligastia.

\par
%\textsuperscript{(863.3)}
\textsuperscript{77:7.2} El número inicial de intermedios secundarios era de 1.984; 873 de ellos no se alinearon con el gobierno de Miguel y fueron debidamente internados en el momento del juicio planetario de Urantia el día de Pentecostés. Nadie puede pronosticar el futuro de estas criaturas caídas.

\par
%\textsuperscript{(863.4)}
\textsuperscript{77:7.3} Los dos grupos de intermedios rebeldes están ahora detenidos en espera del juicio final de los asuntos de la rebelión sistémica. Pero realizaron muchas cosas extrañas en la Tierra antes de iniciarse la dispensación planetaria actual.

\par
%\textsuperscript{(863.5)}
\textsuperscript{77:7.4} Estos intermedios desleales eran capaces de manifestarse a los ojos de los mortales en ciertas circunstancias, y era especialmente el caso de los asociados de Belcebú\footnote{\textit{Belcebú}: Mt 10:25; 12:24,27; Mc 3:22; Lc 11:15,18-19.}, el jefe de los intermedios secundarios apóstatas. Pero estas criaturas singulares no se deben confundir con algunos querubines y serafines rebeldes que estuvieron también en la Tierra hasta la época de la muerte y resurrección de Cristo. Algunos de los escritores más antiguos designaron a estas criaturas intermedias rebeldes con el nombre de espíritus malignos y demonios, y a los serafines apóstatas con el de ángeles malos.

\par
%\textsuperscript{(863.6)}
\textsuperscript{77:7.5} Los espíritus malignos no pueden poseer la mente de un mortal, en ningún mundo, después de que un Hijo donador Paradisiaco ha vivido allí. Pero antes de la estancia de Cristo Miguel en Urantia ---antes de la llegada universal de los Ajustadores del Pensamiento y del derramamiento del espíritu del Maestro sobre toda la humanidad--- estos intermedios rebeldes eran capaces de influir realmente sobre la mente de ciertos mortales inferiores y controlar un poco sus actos. Todo esto lo realizaban de manera muy similar a como lo hacen las criaturas intermedias leales cuando prestan sus servicios como eficaces guardianes de contacto de las mentes humanas que pertenecen al cuerpo urantiano de reserva del destino, en aquellas ocasiones en que el Ajustador está separado realmente de la personalidad durante un período de contacto con las inteligencias superhumanas.

\par
%\textsuperscript{(863.7)}
\textsuperscript{77:7.6} No es una simple figura retórica aquello que indican los escritos: <<Y le trajeron todo tipo de enfermos, los que estaban poseídos por los demonios y los que eran lunáticos>>\footnote{\textit{Le trajeron enfermos}: Mt 4:24.}. Jesús sabía y reconocía la diferencia entre la demencia y la posesión demoníaca, aunque la mente de aquellos que vivieron en su época y generación confundía mucho estos estados.

\par
%\textsuperscript{(863.8)}
\textsuperscript{77:7.7} Incluso antes de Pentecostés, ningún espíritu rebelde podía dominar una mente humana normal, y desde aquel día, las débiles mentes de los mortales inferiores también están libres de esta posibilidad. Desde la llegada del Espíritu de la Verdad, los supuestos exorcismos contra los demonios han consistido en confundir una creencia en la posesión demoníaca con la histeria, la locura y la debilidad mental. La donación de Miguel ha liberado para siempre a todas las mentes humanas de Urantia de la posibilidad de la posesión demoníaca, pero no imaginéis que este riesgo no era real en los tiempos pasados.

\par
%\textsuperscript{(864.1)}
\textsuperscript{77:7.8} Todo el grupo de intermedios rebeldes está actualmente encarcelado por orden de los Altísimos de Edentia. Ya no vagan por este mundo abrigando malas intenciones. Independientemente de la presencia de los Ajustadores del Pensamiento, el derramamiento del Espíritu de la Verdad sobre todo el género humano impide para siempre que los espíritus desleales de cualquier tipo o clase puedan invadir de nuevo ni siquiera la mente humana más débil. Desde el día de Pentecostés, una cosa como la posesión demoníaca nunca podrá volver a suceder.

\section*{8. Los intermedios unidos}
\par
%\textsuperscript{(864.2)}
\textsuperscript{77:8.1} Durante el último juicio de este mundo, cuando Miguel trasladó a los supervivientes dormidos del tiempo, las criaturas intermedias fueron dejadas atrás para que ayudaran en el trabajo espiritual y semiespiritual del planeta. Ahora actúan como un solo cuerpo que engloba a las dos órdenes y asciende a 10.992 miembros. En la actualidad, el miembro más antiguo de cada orden gobierna alternativamente a \textit{Los Intermedios Unidos de Urantia}. Este régimen ha prevalecido desde su fusión en un solo grupo poco después de Pentecostés.

\par
%\textsuperscript{(864.3)}
\textsuperscript{77:8.2} Los miembros de la orden más antigua, o primaria, se conocen generalmente por números; a menudo se les dan nombres tales como 1-2-3 el primero, 4-5-6 el primero, y así sucesivamente. A los intermedios adámicos se les denomina alfabéticamente en Urantia con objeto de distinguirlos de la denominación numérica de los intermedios primarios.

\par
%\textsuperscript{(864.4)}
\textsuperscript{77:8.3} Los seres de las dos órdenes son inmateriales en lo que se refiere a la nutrición y la absorción de la energía, pero comparten muchas características humanas y pueden disfrutar y practicar vuestro humor así como vuestra adoración. Cuando están vinculados a los mortales, entran en el espíritu del trabajo, el descanso y el entretenimiento humanos. Pero los intermedios no duermen ni poseen la facultad de procrearse. En cierto sentido, los miembros del grupo secundario se diferencian según las características masculinas y femeninas, y a menudo se habla de ellos como <<él>> o <<ella>>. Trabajan juntos con frecuencia en parejas de este tipo.

\par
%\textsuperscript{(864.5)}
\textsuperscript{77:8.4} Los intermedios no son hombres y tampoco son ángeles, pero los intermedios secundarios se encuentran por naturaleza más cerca de los hombres que de los ángeles; pertenecen en cierto modo a vuestras razas y por eso son tan comprensivos y compasivos en sus contactos con los seres humanos; son inestimables para los serafines en el trabajo que éstos realizan para las diversas razas de la humanidad y con ellas, y las dos órdenes son imprescindibles para los serafines que ejercen como guardianes personales de los mortales.

\par
%\textsuperscript{(864.6)}
\textsuperscript{77:8.5} Los Intermedios Unidos de Urantia están organizados para servir con los serafines planetarios, según sus dones innatos y su habilidad adquirida, en los cuatro grupos siguientes:

\par
%\textsuperscript{(864.7)}
\textsuperscript{77:8.6} 1. \textit{Los mensajeros intermedios}. Los miembros de este grupo tienen nombres; forman un cuerpo pequeño y son de una gran ayuda, en un mundo evolutivo, en el servicio de las comunicaciones personales rápidas y seguras.

\par
%\textsuperscript{(864.8)}
\textsuperscript{77:8.7} 2. \textit{Los centinelas planetarios}. Los intermedios son los guardianes, los centinelas, de los mundos del espacio. Efectúan la importante función de observadores de los numerosos fenómenos y tipos de comunicaciones que tienen importancia para los seres sobrenaturales de la esfera. Son los que patrullan el ámbito espiritual invisible del planeta.

\par
%\textsuperscript{(865.1)}
\textsuperscript{77:8.8} 3. \textit{Las personalidades de contacto}. Las criaturas intermedias siempre se emplean para establecer contacto con los seres mortales de los mundos materiales, tales como los que se efectuaron con el sujeto a través del cual se transmitieron estas comunicaciones. Son un factor esencial en estas conexiones entre el nivel espiritual y el nivel material.

\par
%\textsuperscript{(865.2)}
\textsuperscript{77:8.9} 4. \textit{Los ayudantes del progreso}. Éstas son las criaturas intermedias más espirituales, y están repartidas como asistentes entre las diversas órdenes de serafines que ejercen su actividad en grupos especiales en el planeta.

\par
%\textsuperscript{(865.3)}
\textsuperscript{77:8.10} Los intermedios varían considerablemente en sus aptitudes para establecer contacto con los serafines por encima de ellos y con sus primos humanos por debajo de ellos. Por ejemplo, a los intermedios primarios les resulta extremadamente difícil ponerse en contacto directo con los organismos materiales. Están mucho más cerca de los seres de tipo angélico y por eso son asignados habitualmente a trabajar con las fuerzas espirituales residentes en el planeta y a aportarles su ayuda. Actúan como compañeros y guías de los visitantes celestiales y de los estudiantes temporales, mientras que las criaturas secundarias están ligadas casi exclusivamente al ministerio de los seres materiales del planeta.

\par
%\textsuperscript{(865.4)}
\textsuperscript{77:8.11} Los 1.111 intermedios secundarios leales están ocupados en importantes misiones en la Tierra. Comparados con sus asociados primarios, son indudablemente materiales. Existen un poco más allá del campo de la visión humana y poseen una libertad de adaptación suficiente como para establecer contacto físico a voluntad con lo que los seres humanos llaman <<cosas materiales>>. Estas criaturas únicas tienen ciertos poderes determinados sobre las cosas del tiempo y del espacio, sin excluir a los animales del planeta.

\par
%\textsuperscript{(865.5)}
\textsuperscript{77:8.12} Una gran parte de los fenómenos más tangibles que se atribuyen a los ángeles han sido ejecutados por las criaturas intermedias secundarias. Cuando los primeros instructores del evangelio de Jesús fueron encarcelados por los jefes religiosos ignorantes de aquella época, un verdadero <<ángel del Señor>> <<abrió por la noche las puertas de la cárcel y los sacó>>\footnote{\textit{Un ángel abre las puertas de la prisión}: Hch 5:19.}. Pero en el caso de la liberación de Pedro\footnote{\textit{Liberación de Pedro}: Hch 12:7-10.}, después de la muerte de Santiago por orden de Herodes, fue un intermedio secundario el que llevó a cabo el trabajo que se atribuyó a un ángel.

\par
%\textsuperscript{(865.6)}
\textsuperscript{77:8.13} La tarea principal que realizan actualmente consiste en ser los asociados desapercibidos de enlace personal de los hombres y las mujeres que componen el cuerpo de reserva planetario del destino. La labor de este grupo secundario, hábilmente apoyada por algunos miembros del cuerpo primario, fue la que produjo en Urantia la coordinación de las personalidades y de las circunstancias que indujeron finalmente a los supervisores celestiales del planeta a tomar la iniciativa de unas peticiones que condujeron a la concesión de las autorizaciones que hicieron posible la serie de revelaciones de las que esta presentación forma parte. Pero debemos indicar claramente que las criaturas intermedias no están implicadas en los sórdidos espectáculos que tienen lugar bajo la denominación general de <<espiritismo>>. Todos los intermedios que residen actualmente en Urantia tienen una reputación honorable, y no están relacionados con los fenómenos de la llamada <<mediumnidad>>; habitualmente no permiten que los humanos sean testigos de sus actividades físicas a veces necesarias, o de sus otros contactos con el mundo material, tal como los sentidos humanos los perciben.

\section*{9. Los ciudadanos permanentes de Urantia}
\par
%\textsuperscript{(865.7)}
\textsuperscript{77:9.1} Los intermedios se pueden considerar como el primer grupo de habitantes permanentes que se encuentran en los diversos tipos de mundos de los universos, en contraste con los ascendentes evolutivos tales como las criaturas mortales y las huestes angélicas. Estos ciudadanos permanentes se encuentran en diversos puntos de la ascensión hacia el Paraíso.

\par
%\textsuperscript{(866.1)}
\textsuperscript{77:9.2} A diferencia de las diversas órdenes de seres celestiales que están destinadas a \textit{servir} en un planeta, los intermedios \textit{viven} en un mundo habitado. Los serafines van y vienen, pero las criaturas intermedias se quedan y se quedarán, y el hecho de haber nacido en el planeta no les impide servir en él como ministros; ellos aseguran el único régimen continuo que armoniza y enlaza las administraciones cambiantes de las huestes seráficas.

\par
%\textsuperscript{(866.2)}
\textsuperscript{77:9.3} Como verdaderos ciudadanos de Urantia, los intermedios tienen un interés de familia por el destino de esta esfera. Forman una asociación decidida que trabaja continuamente por el progreso de su planeta natal. El lema de su orden evoca la determinación que poseen: <<Aquello que los Intermedios Unidos emprenden, los Intermedios Unidos lo realizan>>.

\par
%\textsuperscript{(866.3)}
\textsuperscript{77:9.4} Aunque la capacidad que tienen para atravesar los circuitos energéticos hace posible que cualquier intermedio pueda marcharse del planeta, se han comprometido individualmente a no dejar el planeta hasta que las autoridades del universo los liberen algún día de sus obligaciones. Los intermedios están anclados en un planeta hasta las épocas estabilizadas de luz y de vida. A excepción de 1-2-3 el primero, ninguna criatura intermedia leal ha partido nunca de Urantia.

\par
%\textsuperscript{(866.4)}
\textsuperscript{77:9.5} 1-2-3 el primero, el decano de la orden primaria, fue liberado de sus deberes planetarios inmediatos poco después de Pentecostés. Este noble intermedio se mantuvo inquebrantable con Van y Amadón durante los trágicos días de la rebelión planetaria, y su intrépido liderazgo contribuyó a reducir las bajas en su orden. Actualmente presta sus servicios en Jerusem como miembro del consejo de los veinticuatro\footnote{\textit{Consejo de los veinticuatro}: Ap 4:4,10; 5:8,14; 7:11; 11:16; 14:3; 19:4.}, y desde Pentecostés ya ha desempeñado una vez la función de gobernador general de Urantia.

\par
%\textsuperscript{(866.5)}
\textsuperscript{77:9.6} Los intermedios están atados al planeta, pero de la misma manera que los mortales hablan con los viajeros que vienen de lejos y se informan así sobre los lugares lejanos del planeta, los intermedios conversan también con los viajeros celestiales para informarse sobre los lugares alejados del universo. Así se familiarizan con este sistema y este universo local, e incluso con Orvonton y sus creaciones hermanas, y de esta forma se preparan para la ciudadanía en los niveles superiores de existencia de las criaturas.

\par
%\textsuperscript{(866.6)}
\textsuperscript{77:9.7} Aunque los intermedios fueron traídos a la existencia plenamente desarrollados ---sin experimentar ningún período de crecimiento o de desarrollo desde la inmadurez--- nunca dejan de crecer en sabiduría y experiencia. Al igual que los mortales, son criaturas evolutivas y poseen una cultura que es una auténtica consecución evolutiva. Hay muchas grandes inteligencias y espíritus poderosos en el cuerpo de intermedios de Urantia.

\par
%\textsuperscript{(866.7)}
\textsuperscript{77:9.8} Desde un punto de vista más amplio, la civilización de Urantia es el producto conjunto de los mortales y los intermedios de este planeta, y esto es así a pesar de la diferencia actual entre los dos niveles de cultura, una diferencia que no se compensará antes de las épocas de luz y de vida.

\par
%\textsuperscript{(866.8)}
\textsuperscript{77:9.9} Como la cultura de los intermedios es el producto de unos ciudadanos planetarios inmortales, es relativamente inmune a las vicisitudes temporales que acosan a la civilización humana. Las generaciones de los hombres olvidan; el cuerpo de los intermedios recuerda, y esta memoria es la mina de oro de las tradiciones de vuestro mundo habitado. La cultura de un planeta permanece así siempre presente en ese planeta, y en las circunstancias adecuadas, estos recuerdos atesorados de los acontecimientos pasados vuelven a estar disponibles; así es como los intermedios de Urantia dieron a sus primos carnales la historia de la vida y las enseñanzas de Jesús.

\par
%\textsuperscript{(867.1)}
\textsuperscript{77:9.10} Los intermedios son los expertos ministros que compensan la laguna que apareció después de la muerte de Adán y Eva entre los asuntos materiales y los asuntos espirituales de Urantia. Son también vuestros hermanos mayores, vuestros compañeros en la larga lucha por alcanzar un estado permanente de luz y de vida en Urantia. Los Intermedios Unidos son un cuerpo que ha sido sometido a la prueba de la rebelión, y cumplirán fielmente su función en la evolución planetaria hasta que este mundo alcance la meta de todos los tiempos, hasta ese lejano día en que la paz reine de hecho en la Tierra y haya de verdad buena voluntad en el corazón de los hombres.

\par
%\textsuperscript{(867.2)}
\textsuperscript{77:9.11} Debido al valioso trabajo realizado por estos intermedios, hemos llegado a la conclusión de que forman una parte realmente esencial de la organización espiritual de los mundos. Allí donde la rebelión no ha echado a perder los asuntos de un planeta, son de una ayuda mucho mayor para los serafines.

\par
%\textsuperscript{(867.3)}
\textsuperscript{77:9.12} Toda la organización de los espíritus superiores, las huestes angélicas y los compañeros intermedios se dedica con entusiasmo a fomentar el plan del Paraíso para la ascensión progresiva y la conquista de la perfección de los mortales evolutivos, una de las ocupaciones supremas del universo ---el grandioso plan de la supervivencia consistente en hacer bajar a Dios hasta los hombres y luego, mediante una especie de asociación sublime, hacer subir a los hombres hasta Dios y hacia una eternidad de servicio y la consecución de la divinidad--- tanto para los mortales como para los intermedios.

\par
%\textsuperscript{(867.4)}
\textsuperscript{77:9.13} [Presentado por un Arcángel de Nebadon.]


\chapter{Documento 78. La raza violeta después de la época de Adán}
\par
%\textsuperscript{(868.1)}
\textsuperscript{78:0.1} EL SEGUNDO Edén fue la cuna de la civilización durante cerca de treinta mil años. Los pueblos adámicos se mantuvieron allí en Mesopotamia, y enviaron a su progenie hasta los confines de la Tierra; más tarde se amalgamaron con las tribus noditas y sangiks y fueron conocidos con el nombre de anditas. De esta región salieron los hombres y las mujeres que iniciaron las actividades de los tiempos históricos y que aceleraron tan enormemente el progreso cultural de Urantia.

\par
%\textsuperscript{(868.2)}
\textsuperscript{78:0.2} Este documento describe la historia planetaria de la raza violeta, partiendo desde poco después de la falta de Adán, cerca de 35.000 años a. de J.C., pasando por su fusión con las razas nodita y sangiks hacia el año 15.000 a. de J.C. para formar los pueblos anditas, y continuando hasta su desaparición final de las tierras natales de Mesopotamia, aproximadamente 2.000 años a. de J.C.

\section*{1. La distribución racial y cultural}
\par
%\textsuperscript{(868.3)}
\textsuperscript{78:1.1} Aunque la vida mental y la moralidad de las razas estaban en un bajo nivel en el momento de la llegada de Adán, la evolución física había continuado sin verse afectada en absoluto por la crisis de la rebelión de Caligastia. La contribución que Adán hizo a la condición biológica de las razas, a pesar del fracaso parcial de la empresa, mejoró enormemente a los pueblos de Urantia.

\par
%\textsuperscript{(868.4)}
\textsuperscript{78:1.2} Adán y Eva también aportaron muchas cosas valiosas al progreso social, moral e intelectual de la humanidad; la presencia de sus descendientes aceleró enormemente la civilización. Pero hace treinta y cinco mil años, el mundo en general poseía poca cultura. Algunos centros de civilización existían aquí y allá, pero la mayor parte de Urantia languidecía en un estado salvaje. La distribución racial y cultural era la siguiente:

\par
%\textsuperscript{(868.5)}
\textsuperscript{78:1.3} 1. \textit{La raza violeta ---los adamitas y los adansonitas}. El centro principal de la cultura adamita se encontraba en el segundo jardín, ubicado en el triángulo de los ríos Tigris y Éufrates; ésta fue realmente la cuna de las civilizaciones occidental e india. El centro secundario o septentrional de la raza violeta era la sede adansonita, situada al este de la costa meridional del Mar Caspio, cerca de los montes Kopet. La cultura y el plasma vital que vivificaron inmediatamente a todas las razas se extendieron desde estos dos centros hacia los países circundantes.

\par
%\textsuperscript{(868.6)}
\textsuperscript{78:1.4} 2. \textit{Los presumerios y otros noditas}. En Mesopotamia también estaban presentes, cerca de la desembocadura de los ríos, los restos de la antigua cultura de la época de Dalamatia. A medida que los milenios pasaron, este grupo se mezcló por completo con los adamitas del norte, pero nunca perdió totalmente sus tradiciones noditas. Otros diversos grupos de noditas que se habían asentado en el Levante fueron absorbidos en general por la raza violeta cuando ésta se expandió posteriormente.

\par
%\textsuperscript{(869.1)}
\textsuperscript{78:1.5} 3. \textit{Los andonitas} mantenían cinco o seis colonias bastante representativas al norte y al este de la sede de Adanson. También estaban diseminados por todo el Turquestán, y algunos grupos aislados sobrevivieron en toda Eurasia, sobre todo en las regiones montañosas. Estos aborígenes continuaban ocupando las tierras nórdicas del continente eurasiático así como Islandia y Groenlandia, pero hacía mucho tiempo que habían sido expulsados de las llanuras de Europa por los hombres azules, y de los valles fluviales de la lejana Asia por la raza amarilla en expansión.

\par
%\textsuperscript{(869.2)}
\textsuperscript{78:1.6} 4. \textit{Los hombres rojos} ocupaban las Américas después de haber sido expulsados de Asia más de cincuenta mil años antes de la llegada de Adán.

\par
%\textsuperscript{(869.3)}
\textsuperscript{78:1.7} 5. \textit{La raza amarilla}. Los pueblos chinos controlaban muy bien todo el este de Asia. Sus colonias más avanzadas estaban situadas al noroeste de la China moderna, en las regiones limítrofes con el Tíbet.

\par
%\textsuperscript{(869.4)}
\textsuperscript{78:1.8} 6. \textit{La raza azul}. Los hombres azules estaban diseminados por toda Europa, pero sus mejores centros de cultura estaban situados en los valles entonces fértiles de la cuenca mediterránea y en el noroeste de Europa. La absorción de los neandertales había retrasado enormemente la cultura de los hombres azules, pero aparte de esto eran los más dinámicos, aventureros y exploradores de todos los pueblos evolutivos de Eurasia.

\par
%\textsuperscript{(869.5)}
\textsuperscript{78:1.9} 7. \textit{La India pre-dravidiana}. La mezcla compleja de las razas de la India ---que englobaba a todas las razas de la Tierra, pero sobre todo a la verde, la anaranjada y la negra--- mantenía una cultura ligeramente superior a la de las regiones exteriores.

\par
%\textsuperscript{(869.6)}
\textsuperscript{78:1.10} 8. \textit{La civilización sahariana}. Los elementos superiores de la raza índiga tenían sus colonias más progresivas en lo que hoy es el gran desierto del Sahara. Este grupo índigo-negro contenía numerosos linajes de las razas anaranjada y verde sumergidas.

\par
%\textsuperscript{(869.7)}
\textsuperscript{78:1.11} 9. \textit{La cuenca del Mediterráneo}. La raza más mezclada fuera de la India ocupaba lo que actualmente es la cuenca mediterránea. Los hombres azules del norte y los saharianos del sur se encontraron y se mezclaron aquí con los noditas y los adamitas del este.

\par
%\textsuperscript{(869.8)}
\textsuperscript{78:1.12} Ésta era la imagen del mundo antes de que empezaran las grandes expansiones de la raza violeta, hace aproximadamente veinticinco mil años. La esperanza de una civilización futura se encontraba en el segundo jardín, entre los ríos de Mesopotamia. Aquí, en el suroeste de Asia, existía el potencial de una gran civilización, la posibilidad de difundir por el mundo las ideas y los ideales que se habían salvado desde los tiempos de Dalamatia y la época del Edén.

\par
%\textsuperscript{(869.9)}
\textsuperscript{78:1.13} Adán y Eva habían dejado detrás una progenie limitada pero poderosa, y los observadores celestiales que estaban en Urantia esperaban ansiosamente descubrir cómo se desenvolverían estos descendientes del Hijo y la Hija Materiales desviados.

\section*{2. Los adamitas en el segundo Jardín}
\par
%\textsuperscript{(869.10)}
\textsuperscript{78:2.1} Los hijos de Adán trabajaron durante miles de años a lo largo de los ríos de Mesopotamia, resolviendo sus problemas de riego y de control de las inundaciones en el sur, perfeccionando sus defensas en el norte, e intentando preservar sus tradiciones de la gloria del primer Edén.

\par
%\textsuperscript{(869.11)}
\textsuperscript{78:2.2} El heroísmo que mostraron en la dirección del segundo jardín constituye una de las epopeyas asombrosas e inspiradoras de la historia de Urantia. Estas almas espléndidas nunca perdieron de vista por completo el objetivo de la misión adámica, y por eso rechazaron valientemente las influencias de las tribus circundantes e inferiores, mientras que enviaron voluntariamente a sus hijos e hijas más escogidos en una oleada ininterrumpida como emisarios entre las razas de la Tierra. Esta expansión agotaba a veces su cultura natal, pero estos pueblos superiores siempre lograron recobrarse.

\par
%\textsuperscript{(870.1)}
\textsuperscript{78:2.3} La civilización, la sociedad y la condición cultural de los adamitas estaban muy por encima del nivel general de las razas evolutivas de Urantia. Sólo había una civilización comparable a ella en todos los aspectos, y se encontraba entre las antiguas colonias de Van y Amadón y entre los adansonitas. Pero la civilización del segundo Edén era una estructura artificial ---no había sido producida por la evolución--- y por esta razón estaba condenada a deteriorarse hasta alcanzar un nivel evolutivo natural.

\par
%\textsuperscript{(870.2)}
\textsuperscript{78:2.4} Adán dejó tras él una gran cultura intelectual y espiritual, pero no era avanzada en dispositivos mecánicos ya que toda civilización está limitada por los recursos naturales disponibles, el genio inherente y el tiempo libre suficiente para asegurar la realización de los inventos. La civilización de la raza violeta estaba basada en la presencia de Adán y en las tradiciones del primer Edén. Después de la muerte de Adán y a medida que estas tradiciones se difuminaban con el paso de los milenios, el nivel cultural de los adamitas se deterioró continuamente hasta que alcanzó un estado de equilibrio recíproco entre la condición de los pueblos circundantes y las capacidades culturales de la raza violeta que evolucionaban de manera natural.

\par
%\textsuperscript{(870.3)}
\textsuperscript{78:2.5} Sin embargo, hacia el año 19.000 a. de J.C., los adamitas formaban una verdadera nación que ascendía a cuatro millones y medio de habitantes, y ya habían derramado a millones de sus descendientes entre los pueblos de los alrededores.

\section*{3. Las primeras expansiones de los adamitas}
\par
%\textsuperscript{(870.4)}
\textsuperscript{78:3.1} La raza violeta conservó las tradiciones pacíficas del Edén durante muchos milenios, lo que explica el gran retraso en llevar a cabo conquistas territoriales. Cuando sufrían la tensión de la superpoblación, en lugar de hacer la guerra para conseguir más territorios, enviaban el excedente de sus habitantes como instructores a las otras razas. El efecto cultural de estas primeras emigraciones no fue duradero, pero la absorción de los educadores, comerciantes y exploradores adamitas fortaleció biológicamente a los pueblos circundantes.

\par
%\textsuperscript{(870.5)}
\textsuperscript{78:3.2} Algunos adamitas viajaron pronto hacia el oeste hasta el valle del Nilo; otros se dirigieron hacia el este y penetraron en Asia, pero éstos fueron una minoría. El movimiento en masa de las épocas más tardías se dirigió ampliamente hacia el norte y desde allí hacia el oeste. Se trató, en general, de un avance gradual pero continuo hacia el norte; la mayoría se dirigió hacia el norte, y luego dio la vuelta hacia el oeste alrededor del Mar Caspio hasta penetrar en Europa.

\par
%\textsuperscript{(870.6)}
\textsuperscript{78:3.3} Hace aproximadamente veinticinco mil años, un gran número de los elementos adamitas más puros estaban de camino en su largo viaje hacia el norte. A medida que avanzaban en esta dirección se volvieron cada vez menos adámicos, y en la época en que ocuparon el Turquestán, se habían mezclado por completo con las otras razas, principalmente con los noditas. Muy pocos pueblos violetas de pura cepa penetraron profundamente en Europa o Asia.

\par
%\textsuperscript{(870.7)}
\textsuperscript{78:3.4} Desde cerca del año 30.000 hasta el 10.000 a. de J.C., en todo el suroeste de Asia se produjeron unas mezclas raciales que hicieron época. Los habitantes de las tierras altas del Turquestán eran un pueblo viril y vigoroso. Una gran parte de la cultura de los tiempos de Van sobrevivía en el noroeste de la India. Más al norte de estas colonias se había conservado lo mejor de los andonitas primitivos. Y estas dos razas, con una cultura y un carácter superiores, fueron absorbidas por los adamitas que se desplazaban hacia el norte. Esta fusión condujo a la adopción de muchas ideas nuevas; facilitó el progreso de la civilización e hizo avanzar considerablemente todas las fases del arte, las ciencias y la cultura social.

\par
%\textsuperscript{(871.1)}
\textsuperscript{78:3.5} Cuando el período de las primeras emigraciones adámicas terminó hacia el año 15.000 a. de J.C., ya había más descendientes de Adán en Europa y Asia central que en cualquier otra parte del mundo, incluida Mesopotamia. Las razas azules europeas habían sido ampliamente impregnadas. Todas las regiones meridionales de los países que ahora se llaman Rusia y Turquestán estaban ocupadas por una gran reserva de adamitas mezclados con noditas, andonitas y sangiks rojos y amarillos. Europa del sur y la franja del Mediterráneo estaban ocupadas por una raza mixta de pueblos andonitas y sangiks ---anaranjados, verdes e índigos--- con una pequeña parte del linaje adamita. Asia Menor y los países de Europa central y oriental estaban habitados por tribus predominantemente andonitas.

\par
%\textsuperscript{(871.2)}
\textsuperscript{78:3.6} Una raza mixta de color, enormemente reforzada hacia esta época por la gente que llegaba de Mesopotamia, se había establecido en Egipto y se preparaba para tomar posesión de la cultura en vías de desaparición del valle del Éufrates. Los pueblos negros se adentraban cada vez más en el sur de África y, al igual que la raza roja, estaban prácticamente aislados.

\par
%\textsuperscript{(871.3)}
\textsuperscript{78:3.7} La civilización sahariana se había desorganizado a causa de las sequías, y la de la cuenca del Mediterráneo debido a las inundaciones. Las razas azules no habían conseguido desarrollar hasta ese momento una cultura avanzada. Los andonitas continuaban diseminados por las regiones árticas y las de Asia central. Las razas verde y anaranjada habían sido exterminadas como tales. La raza índiga se dirigía hacia el sur de África para empezar allí su lenta degeneración racial que continuó durante mucho tiempo.

\par
%\textsuperscript{(871.4)}
\textsuperscript{78:3.8} Los pueblos de la India permanecían estancados, con una civilización que no progresaba; los hombres amarillos consolidaban sus posesiones en Asia central; los hombres cobrizos aún no habían iniciado su civilización en las islas cercanas del Pacífico.

\par
%\textsuperscript{(871.5)}
\textsuperscript{78:3.9} Estas distribuciones raciales, unidas a los extensos cambios climáticos, prepararon el escenario del mundo para la inauguración de la era andita de la civilización urantiana. Estas primeras emigraciones abarcaron un período de diez mil años, desde el año 25.000 hasta el 15.000 a. de J.C. Las emigraciones posteriores o anditas se extendieron desde cerca del año 15.000 hasta el 6000 a. de J.C.

\par
%\textsuperscript{(871.6)}
\textsuperscript{78:3.10} Las primeras oleadas de adamitas tardaron tanto tiempo en atravesar Eurasia, que una gran parte de su cultura se perdió por el camino. Sólo los anditas más tardíos se desplazaron con la rapidez suficiente como para conservar la cultura edénica a grandes distancias de Mesopotamia.

\section*{4. Los anditas}
\par
%\textsuperscript{(871.7)}
\textsuperscript{78:4.1} Las razas anditas constituían las mezclas primitivas entre la pura raza violeta y los noditas, más los pueblos evolutivos. Se puede considerar que los anditas contenían en general un porcentaje de sangre adámica mucho mayor que las razas modernas. El término andita se utiliza generalmente para designar a aquellos pueblos cuya herencia racial era entre una sexta y una octava parte violeta. Los urantianos modernos, incluso los de las razas blancas del norte, contienen un porcentaje mucho menor de la sangre de Adán.

\par
%\textsuperscript{(871.8)}
\textsuperscript{78:4.2} Los primeros pueblos anditas tuvieron su origen en las regiones colindantes con Mesopotamia hace más de veinticinco mil años, y consistieron en una mezcla de adamitas y noditas. El segundo jardín estaba rodeado de zonas concéntricas donde los habitantes poseían cada vez menos sangre violeta, y la raza andita nació precisamente en la periferia de este crisol racial. Más adelante, cuando los adamitas y los noditas en plena emigración entraron en las regiones entonces fértiles del Turquestán, se mezclaron rápidamente con sus habitantes superiores, y la mezcla racial resultante extendió el tipo andita hacia el norte.

\par
%\textsuperscript{(872.1)}
\textsuperscript{78:4.3} Los anditas eran, en todos los campos, la mejor raza humana que había aparecido en Urantia desde los tiempos de los pueblos de puro linaje violeta. Contenían la mayor parte de los tipos superiores de los restos sobrevivientes de las razas adamita y nodita y, más tarde, algunos de los mejores linajes de los hombres amarillos, azules y verdes.

\par
%\textsuperscript{(872.2)}
\textsuperscript{78:4.4} Estos primeros anditas no eran arios, sino prearios. No eran blancos, sino preblancos. No eran un pueblo occidental ni un pueblo oriental. Pero la herencia andita es la que confiere a la mezcla políglota de las llamadas razas blancas esa homogeneidad generalizada que ha sido denominada caucasoide.

\par
%\textsuperscript{(872.3)}
\textsuperscript{78:4.5} Los descendientes más puros de la raza violeta habían conservado la tradición adámica de buscar la paz, lo que explica por qué los primeros desplazamientos raciales habían tenido más bien el carácter de emigraciones pacíficas. Pero a medida que los adamitas se unieron con los linajes noditas, que ya eran entonces una raza belicosa, sus descendientes anditas se convirtieron, para su época, en los militaristas más hábiles y sagaces que hayan vivido jamás en Urantia. A partir de entonces, los desplazamientos de los mesopotámicos fueron teniendo un carácter cada vez más militar, y se asemejaron más a auténticas conquistas.

\par
%\textsuperscript{(872.4)}
\textsuperscript{78:4.6} Estos anditas eran aventureros; tenían inclinaciones errantes. Un aumento de sangre sangik o andonita tendió a estabilizarlos. Pero incluso así, sus descendientes más tardíos no se detuvieron hasta haber circunnavegado el globo y descubierto el último continente lejano.

\section*{5. Las emigraciones anditas}
\par
%\textsuperscript{(872.5)}
\textsuperscript{78:5.1} La cultura del segundo jardín sobrevivió durante veinte mil años, pero sufrió un declive continuo hasta cerca del año 15.000 a. de J.C., cuando la regeneración del clero setita y la jefatura de Amosad inauguraron una era brillante. Las oleadas masivas de civilización que se extendieron más tarde por Eurasia siguieron de cerca al gran renacimiento del Jardín, que fue una consecuencia de las numerosas uniones de los adamitas con los noditas mixtos circundantes, lo cual dio origen a los anditas.

\par
%\textsuperscript{(872.6)}
\textsuperscript{78:5.2} Estos anditas introdujeron nuevos progresos en toda Eurasia y
África del norte. La cultura andita dominaba desde Mesopotamia hasta el Sinkiang, y las emigraciones constantes hacia Europa eran continuamente compensadas con la nueva gente que llegaba de Mesopotamia. Pero no es muy exacto hablar de los anditas como de una raza en la propia Mesopotamia hasta cerca del comienzo de las emigraciones finales de los descendientes mixtos de Adán. Para entonces, las razas mismas del segundo jardín se habían mezclado de tal manera que ya no se podían considerar como adamitas.

\par
%\textsuperscript{(872.7)}
\textsuperscript{78:5.3} La civilización del Turquestán se avivaba y renovaba constantemente gracias a la gente que llegaba de Mesopotamia, y principalmente a los jinetes anditas posteriores. La llamada lengua madre aria estaba en proceso de formación en las tierras altas del Turquestán; era una mezcla del dialecto andónico de aquella región con el idioma de los adansonitas y los anditas posteriores. Muchas lenguas modernas se derivan de este lenguaje primitivo de las tribus de Asia central que conquistaron Europa, la India y las regiones superiores de las llanuras de Mesopotamia. Este antiguo idioma dio a las lenguas occidentales esa semejanza que se designa con el apelativo de aria.

\par
%\textsuperscript{(872.8)}
\textsuperscript{78:5.4} Hacia el año 12.000 a. de J.C., tres cuartas partes de los descendientes anditas del mundo residían en el norte y el este de Europa, y cuando más tarde se produjo el éxodo final desde Mesopotamia, el sesenta y cinco por ciento de estas últimas oleadas migratorias penetraron en Europa.

\par
%\textsuperscript{(873.1)}
\textsuperscript{78:5.5} Los anditas no solamente emigraron hacia Europa sino también hacia el norte de China y la India, mientras que muchos grupos se desplazaron hasta los confines de la Tierra como misioneros, educadores y comerciantes. Efectuaron una aportación considerable a los grupos de pueblos sangiks del norte del Sahara. Pero sólo unos pocos instructores y comerciantes penetraron en África más al sur de la cabecera del Nilo. Más tarde, los anditas mestizos y los egipcios descendieron por las costas orientales y occidentales de África muy por debajo del ecuador, pero no llegaron hasta Madagascar.

\par
%\textsuperscript{(873.2)}
\textsuperscript{78:5.6} Estos anditas fueron los conquistadores llamados dravidianos, y más tarde arios, de la India, y su presencia en Asia central mejoró considerablemente a los antepasados de los turanianos. Muchos miembros de esta raza viajaron hasta China tanto por el Sinkiang como por el Tíbet, y añadieron cualidades deseables a los linajes chinos posteriores. De vez en cuando, pequeños grupos se dirigieron hacia el Japón, Formosa, las Indias Orientales y el sur de China, aunque muy pocos entraron en el sur de China por la ruta costera.

\par
%\textsuperscript{(873.3)}
\textsuperscript{78:5.7} Ciento treinta y dos miembros de esta raza se embarcaron en una flotilla de barcos pequeños en el Japón y llegaron finalmente hasta América del Sur; por medio de matrimonios mixtos con los nativos de los Andes, dieron nacimiento a los antepasados de los soberanos posteriores de los Incas. Atravesaron el Pacífico en pequeñas etapas, deteniéndose en las numerosas islas que encontraron por el camino. Las islas de Polinesia eran entonces más numerosas y más grandes que en la actualidad, y estos marineros anditas, junto con otros que los siguieron, modificaron biológicamente a su paso a los grupos indígenas. Como consecuencia de la penetración andita, muchos centros florecientes de civilización se desarrollaron en estas tierras ahora sumergidas. La Isla de Pascua fue durante mucho tiempo el centro religioso y administrativo de uno de estos grupos desaparecidos. Pero de todos los anditas que navegaron por el Pacífico en aquellos tiempos lejanos, los ciento treinta y dos mencionados fueron los únicos que llegaron al continente de las Américas.

\par
%\textsuperscript{(873.4)}
\textsuperscript{78:5.8} Las conquistas migratorias de los anditas continuaron hasta sus últimas dispersiones entre los años 8000 y 6000 a. de J.C. A medida que salían en masa de Mesopotamia, agotaban continuamente las reservas biológicas de sus tierras natales, al mismo tiempo que fortalecían notablemente a los pueblos circundantes. A todas las naciones donde llegaron aportaron el humor, el arte, la aventura, la música y la manufactura. Eran unos hábiles domesticadores de animales y unos agricultores expertos. Al menos en esta época, su presencia mejoraba generalmente las creencias religiosas y las prácticas morales de las razas más antiguas. Así es como la cultura de Mesopotamia se difundió tranquilamente por Europa, la India, China, África del norte y las Islas del Pacífico.

\section*{6. Las últimas dispersiones anditas}
\par
%\textsuperscript{(873.5)}
\textsuperscript{78:6.1} Las tres últimas oleadas de anditas salieron en masa de Mesopotamia entre los años 8000 y 6000 a. de J.C. Estas tres grandes oleadas culturales fueron forzadas a salir de Mesopotamia a causa de la presión de las tribus de las colinas del este y al hostigamiento de los hombres de las llanuras del oeste. Los habitantes del valle del Éufrates y de los territorios adyacentes emprendieron su éxodo final en diversas direcciones:

\par
%\textsuperscript{(873.6)}
\textsuperscript{78:6.2} El sesenta y cinco por ciento entró en Europa por la ruta del Mar Caspio para conquistar a las razas blancas que acababan de aparecer ---la mezcla de los hombres azules con los primeros anditas--- y fusionarse con ellas.

\par
%\textsuperscript{(873.7)}
\textsuperscript{78:6.3} El diez por ciento, incluyendo un amplio grupo de sacerdotes setitas, se dirigió hacia el este a través de las tierras altas elamitas hasta la meseta iraní y el Turquestán. Posteriormente, muchos de sus descendientes fueron expulsados con sus hermanos arios desde las regiones del norte hacia la India.

\par
%\textsuperscript{(874.1)}
\textsuperscript{78:6.4} El diez por ciento de los mesopotámicos que viajaban hacia el norte se desviaron hacia el este para entrar en el Sinkiang, donde se fusionaron con sus habitantes anditas y amarillos mezclados. La mayoría de los hábiles descendientes de esta unión racial penetró posteriormente en China y contribuyó mucho al mejoramiento inmediato de la rama nórdica de la raza amarilla.

\par
%\textsuperscript{(874.2)}
\textsuperscript{78:6.5} El diez por ciento de estos anditas que huían atravesaron Arabia y entraron en Egipto.

\par
%\textsuperscript{(874.3)}
\textsuperscript{78:6.6} El cinco por ciento de los anditas, que poseía la cultura más superior del territorio costero cercano a la desembocadura de los ríos Tigris y Éufrates, había evitado mezclarse con los miembros inferiores de las tribus vecinas, y se negaron a abandonar sus hogares. Este grupo representaba la supervivencia de numerosos linajes noditas y adamitas superiores.

\par
%\textsuperscript{(874.4)}
\textsuperscript{78:6.7} Los anditas habían evacuado casi por completo esta región hacia el año 6000 a. de J.C., aunque sus descendientes, ampliamente mezclados con las razas sangiks circundantes y los andonitas de Asia Menor, permanecieron allí para presentar batalla a los invasores del norte y del este en una fecha mucho más tardía.

\par
%\textsuperscript{(874.5)}
\textsuperscript{78:6.8} La infiltración creciente de los linajes inferiores circundantes puso fin a la época cultural del segundo jardín. La civilización se desplazó hacia el oeste hasta el Nilo y las islas del Mediterráneo, donde continuó prosperando y progresando mucho tiempo después de que su fuente se hubiera deteriorado en Mesopotamia. Esta afluencia sin obstáculos de los pueblos inferiores preparó el camino para la conquista posterior de toda Mesopotamia por los bárbaros del norte, los cuales expulsaron a los linajes capacitados que quedaban. Incluso años después, a los elementos cultos restantes les seguía molestando la presencia de estos invasores ignorantes y toscos.

\section*{7. Las inundaciones en Mesopotamia}
\par
%\textsuperscript{(874.6)}
\textsuperscript{78:7.1} Los habitantes ribereños estaban acostumbrados a que los ríos se desbordaran en ciertas estaciones; estas inundaciones periódicas eran un acontecimiento anual en sus vidas. Pero nuevos peligros amenazaron al valle de Mesopotamia a consecuencia de unos cambios geológicos progresivos que se habían producido en el norte.

\par
%\textsuperscript{(874.7)}
\textsuperscript{78:7.2} Durante miles de años después del hundimiento del primer Edén, las montañas cercanas a la costa oriental del Mediterráneo y las del noroeste y nordeste de Mesopotamia continuaron elevándose. Esta elevación de las tierras altas se aceleró enormemente hacia el año 5000 a. de J.C., y este factor, unido a unas nevadas mucho más abundantes en las montañas del norte, produjo cada primavera unas inundaciones sin precedentes en todo el valle del Éufrates. Estas inundaciones primaverales empeoraron cada vez más, de manera que los habitantes de las regiones fluviales fueron empujados con el tiempo hacia las tierras altas del este. Durante cerca de mil años, decenas de ciudades se quedaron prácticamente abandonadas a causa de estos grandes diluvios.

\par
%\textsuperscript{(874.8)}
\textsuperscript{78:7.3} Cerca de cinco mil años más tarde, cuando los sacerdotes hebreos cautivos en Babilonia trataron de hacer remontar el origen del pueblo judío\footnote{\textit{Genealogías judías}: Gn 4:1-2,17-26; 5:3-32; 6:9-10; 10:1-32; 11:10-26.} hasta los tiempos de Adán, encontraron muchas dificultades para juntar las partes de la historia; entonces a uno de ellos se le ocurrió renunciar al esfuerzo, dejar que el mundo entero se ahogara en su perversidad en la época del diluvio de Noé, y encontrarse así en mejores condiciones para hacer remontar el origen de Abraham directamente hasta uno de los tres hijos sobrevivientes de Noé\footnote{\textit{La destrucción del mundo}: Gn 6:5-10.}.

\par
%\textsuperscript{(875.1)}
\textsuperscript{78:7.4} Las tradiciones que hablan de una época en que las aguas cubrían toda la superficie de la Tierra son universales. Muchas razas conservan la historia de un diluvio mundial que tuvo lugar en algún momento de las épocas pasadas. La historia bíblica de Noé, el arca y el diluvio es un invento del clero hebreo durante su cautividad en Babilonia. Nunca ha habido un diluvio universal\footnote{\textit{Nunca un diluvio universal}: Gn 7:10-24.} desde que la vida se estableció en Urantia. La única vez que la superficie de la Tierra estuvo completamente cubierta de agua fue durante las épocas arqueozoicas, antes de que la tierra firme empezara a aparecer.

\par
%\textsuperscript{(875.2)}
\textsuperscript{78:7.5} Pero Noé vivió\footnote{\textit{Noé vivió}: Gn 6:9-10.} realmente; era un viticultor de Aram, una colonia ribereña cerca de Erec. Año tras año conservaba sus anotaciones escritas sobre los períodos de las crecidas del río. Fue objeto de una gran irrisión mientras recorría el valle del río de arriba abajo recomendando que todas las casas se construyeran de madera, en forma de barco, y que los animales de la familia se subieran a bordo todas las noches cuando se acercara la estación de las inundaciones. Cada año se desplazaba hasta las colonias ribereñas vecinas y les avisaba de la fecha en que se producirían las inundaciones. Finalmente llegó un año en que las inundaciones anuales aumentaron considerablemente debido a fuertes aguaceros poco habituales, de manera que la crecida repentina de las aguas destruyó todo el pueblo; sólo Noé y su familia directa se salvaron en su casa flotante\footnote{\textit{La inundación real}: Gn 7:10-24.}.

\par
%\textsuperscript{(875.3)}
\textsuperscript{78:7.6} Estas inundaciones terminaron de disgregar la civilización andita. Al final de este período de diluvios, el segundo jardín había dejado de existir. Sólo subsistió algún rastro de su antigua gloria en el sur y entre los sumerios.

\par
%\textsuperscript{(875.4)}
\textsuperscript{78:7.7} Los restos de esta civilización, una de las más antiguas, se pueden encontrar en estas regiones de Mesopotamia así como al nordeste y al noroeste de ellas. Pero los vestigios aún más antiguos de la época de Dalamatia existen bajo las aguas del Golfo Pérsico, y el primer Edén yace sumergido bajo el extremo oriental del Mar Mediterráneo.

\section*{8. Los sumerios ---los últimos anditas}
\par
%\textsuperscript{(875.5)}
\textsuperscript{78:8.1} Cuando la última dispersión de los anditas rompió la espina dorsal biológica de la civilización mesopotámica, una pequeña minoría de esta raza superior permaneció en su tierra natal cerca de la desembocadura de los ríos. Eran los sumerios; hacia el año 6000 a. de J.C., su linaje se había vuelto en gran parte andita, aunque el carácter de su cultura era más exactamente nodita, y se aferraban a las antiguas tradiciones de Dalamatia. Sin embargo, estos sumerios de las regiones costeras eran los últimos anditas de Mesopotamia. Pero en esta fecha tardía las razas de Mesopotamia ya estaban completamente mezcladas, tal como lo demuestran los tipos de cráneos encontrados en las tumbas de esta época.

\par
%\textsuperscript{(875.6)}
\textsuperscript{78:8.2} Susa prosperó enormemente durante los tiempos de las inundaciones. La primera ciudad, la más baja, se inundó, de manera que la segunda ciudad, o más alta, sucedió a la primera como centro de las artesanías particulares de aquella época. Cuando estas inundaciones disminuyeron posteriormente, Ur se convirtió en el centro de la industria alfarera. Hace unos siete mil años, Ur se encontraba en el Golfo Pérsico; desde entonces los depósitos de aluvión han elevado las tierras hasta sus límites actuales. Estas colonias sufrieron menos los efectos de las inundaciones debido a sus obras de protección más adecuadas y al ensanchamiento de la desembocadura de los ríos.

\par
%\textsuperscript{(875.7)}
\textsuperscript{78:8.3} Los pacíficos cultivadores de cereales de los valles del Tigris y el Éufrates habían sido acosados durante mucho tiempo por las correrías de los bárbaros del Turquestán y de la meseta iraní. Pero en aquella época, la creciente sequía de los pastos de las tierras altas provocó una invasión concertada del valle del Éufrates. Esta invasión fue aún más grave porque estos pastores y cazadores de los alrededores poseían una gran cantidad de caballos domados. La posesión de los caballos les dio una enorme superioridad militar sobre sus ricos vecinos del sur. En poco tiempo invadieron toda Mesopotamia y expulsaron a las últimas oleadas de cultura, que se esparcieron por toda Europa, Asia occidental y África del norte.

\par
%\textsuperscript{(876.1)}
\textsuperscript{78:8.4} Estos conquistadores de Mesopotamia llevaban entre sus filas a un gran número de los mejores descendientes anditas de las razas mixtas nórdicas del Turquestán, incluyendo a algunos linajes adansonitas. Estas tribus del norte, menos avanzadas pero más vigorosas, asimilaron rápida y voluntariamente los restos de la civilización mesopotámica, y pronto se convirtieron en los pueblos mixtos que se encontraban en el valle del Éufrates al principio de los tiempos históricos. Restablecieron rápidamente muchas fases de la civilización moribunda de Mesopotamia, adoptando las artes de las tribus del valle y una gran parte de la cultura de los sumerios. Trataron incluso de construir una tercera torre de Babel, y más tarde adoptaron este nombre para designar a su nación.

\par
%\textsuperscript{(876.2)}
\textsuperscript{78:8.5} Cuando estos jinetes bárbaros procedentes del nordeste invadieron todo el valle del Éufrates, no lograron conquistar a los supervivientes anditas que vivían cerca de la desembocadura del río en el Golfo Pérsico. Estos sumerios fueron capaces de defenderse gracias a su inteligencia superior, a sus mejores armas y al extenso sistema de canales militares que habían añadido a sus métodos de riego por estanques comunicantes. Formaban un pueblo unido porque tenían una religión colectiva uniforme. De esta manera pudieron mantener su integridad racial y nacional hasta mucho tiempo después de que sus vecinos del noroeste se dividieran en ciudades-Estado aisladas. Ninguno de estos grupos urbanos fue capaz de vencer a los sumerios unidos.

\par
%\textsuperscript{(876.3)}
\textsuperscript{78:8.6} Los invasores del norte aprendieron pronto a confiar en estos sumerios amantes de la paz y a apreciar sus aptitudes como educadores y administradores. Fueron muy respetados y solicitados como instructores de las artes y la industria, como directores comerciales y como gobernantes civiles por todos los pueblos del norte, y desde Egipto en el oeste hasta la India en el este.

\par
%\textsuperscript{(876.4)}
\textsuperscript{78:8.7} Después de la desintegración de la primera confederación sumeria, las ciudades-Estado posteriores fueron gobernadas por los descendientes apóstatas de los sacerdotes setitas. Estos sacerdotes sólo se dieron el nombre de reyes cuando conquistaron las ciudades vecinas. Los reyes posteriores de estas ciudades no lograron formar unas confederaciones poderosas antes de la época de Sargón porque eran celosos de sus deidades. Cada ciudad creía que su dios municipal era superior a todos los demás dioses, y por tanto se negaban a someterse a un jefe común.

\par
%\textsuperscript{(876.5)}
\textsuperscript{78:8.8} Sargón\footnote{\textit{Sargón}: Is 20:1.}, el sacerdote de Kish, terminó con este largo período de gobiernos débiles de los sacerdotes urbanos; se proclamó rey y emprendió la conquista de toda Mesopotamia y de los países limítrofes. Esto puso fin, por el momento, a las ciudades-Estado gobernadas y tiranizadas por los sacerdotes, donde cada ciudad tenía su propio dios municipal y sus prácticas ceremoniales particulares.

\par
%\textsuperscript{(876.6)}
\textsuperscript{78:8.9} A la desintegración de esta confederación de Kish le siguió un largo período de continuas guerras por la supremacía entre estas ciudades del valle. La soberanía alternó de manera diversa entre Sumer, Accad, Kish, Erec, Ur y Susa.

\par
%\textsuperscript{(876.7)}
\textsuperscript{78:8.10} Cerca del año 2500 a. de J.C., los sumerios sufrieron graves derrotas a manos de los suitas y los guitas del norte. Lagash, la capital sumeria construida sobre montículos aluviales, cayó. Erec resistió durante treinta años después de la caída de Accad. En la época del establecimiento del reinado de Hamurabi, los sumerios habían sido absorbidos en la masa de los semitas del norte, y los anditas de Mesopotamia desaparecieron de las páginas de la historia.

\par
%\textsuperscript{(877.1)}
\textsuperscript{78:8.11} Entre los años 2500 y 2000 a. de J.C., los nómadas anduvieron destrozándolo todo a su paso desde el Atlántico hasta el Pacífico. Los neritas constituyeron la emanación final del grupo caspio de los descendientes mesopotámicos de las razas andonitas y anditas mezcladas. Los cambios climáticos posteriores consiguieron realizar aquello que los bárbaros no lograron hacer para llevar a cabo la ruina de Mesopotamia.

\par
%\textsuperscript{(877.2)}
\textsuperscript{78:8.12} Y ésta es la historia de la raza violeta después de la época de Adán, y del destino de su tierra natal entre el Tigris y el Éufrates. Su antigua civilización cayó finalmente debido a la emigración de los pueblos superiores y a la inmigración de sus vecinos inferiores. Pero mucho antes de que los jinetes bárbaros conquistaran el valle, una gran parte de la cultura del jardín se había extendido por Asia, África y Europa, para producir allí los fermentos que dieron como resultado la civilización urantiana del siglo veinte.

\par
%\textsuperscript{(877.3)}
\textsuperscript{78:8.13} [Presentado por un Arcángel de Nebadon.]


\chapter{Documento 79. La expansión andita en Oriente}
\par
%\textsuperscript{(878.1)}
\textsuperscript{79:0.1} ASIA es la cuna de la raza humana. Andón y Fonta nacieron precisamente en una península del sur de este continente, y en las regiones montañosas de lo que hoy es Afganistán, su descendiente Badonán fundó un centro primitivo de cultura que sobrevivió durante más de medio millón de años. Aquí, en este centro oriental de la raza humana, los pueblos sangiks se diferenciaron del linaje andonita, y Asia fue su primer hogar, su primer territorio de caza, su primer campo de batalla. El suroeste de Asia fue testigo de las civilizaciones sucesivas de dalamatianos, noditas, adamitas y anditas, y los potenciales de la civilización moderna se extendieron desde estas regiones hacia todo el mundo.

\section*{1. Los anditas del Turquestán}
\par
%\textsuperscript{(878.2)}
\textsuperscript{79:1.1} Durante más de veinticinco mil años, hasta cerca del año
2000 a. de J.C., el corazón de Eurasia fue predominantemente andita, aunque esta influencia fue disminuyendo. En las tierras bajas del Turquestán, los anditas se desviaron hacia el oeste alrededor de los lagos interiores para entrar en Europa, mientras que desde las tierras altas de esta región se infiltraron hacia el este. El Turquestán oriental
(Sinkiang), y en menor grado el Tíbet, fueron las antiguas puertas por las que estos pueblos de Mesopotamia penetraron en las montañas que conducían hacia las tierras nórdicas de los hombres amarillos. La infiltración andita en la India partió de las regiones montañosas del Turquestán hasta entrar en el Punjab, y de los pastos iraníes a través del Baluchistán. Estas emigraciones primitivas no tuvieron en ningún sentido el carácter de conquistas; se trataron más bien del desplazamiento continuo de las tribus anditas hacia el oeste de la India y China.

\par
%\textsuperscript{(878.3)}
\textsuperscript{79:1.2} Los centros de la cultura mixta andita sobrevivieron durante cerca de quince mil años en la cuenca del río Tarim en el Sinkiang, y hacia el sur en las regiones montañosas del Tíbet, donde los anditas y los andonitas se habían mezclado ampliamente. El valle del Tarim era el puesto oriental más avanzado de la verdadera cultura andita. Aquí establecieron sus colonias y empezaron a tener relaciones comerciales con los chinos progresivos hacia el este y con los andonitas hacia el norte. En aquella época, la región del Tarim poseía tierras fértiles y las lluvias eran abundantes. Hacia el este, el Gobi era una extensa pradera donde los pastores se iban transformando gradualmente en agricultores. Esta civilización pereció cuando los vientos de las lluvias cambiaron hacia el sudeste, pero en su momento rivalizó con la misma Mesopotamia.

\par
%\textsuperscript{(878.4)}
\textsuperscript{79:1.3} Hacia el año 8000 a. de J.C., la aridez lentamente creciente de las regiones montañosas de Asia central empezó a arrojar a los anditas hacia el fondo de los valles y las costas marítimas. Esta sequía cada vez mayor no solamente los empujó hacia los valles del Nilo, del Éufrates, del Indo y del Río Amarillo, sino que produjo un nuevo desarrollo en la civilización andita. Una nueva clase de hombres, los comerciantes, empezó a aparecer en grandes cantidades.

\par
%\textsuperscript{(879.1)}
\textsuperscript{79:1.4} Cuando las condiciones climáticas hicieron que la caza fuera poco provechosa para los anditas en plena emigración, éstos no siguieron la trayectoria evolutiva de las razas más antiguas convirtiéndose en pastores. El comercio y la vida urbana hicieron su aparición. Desde Egipto, Mesopotamia y el Turquestán hasta los ríos de China y la India, las tribus más civilizadas empezaron a congregarse en ciudades dedicadas a la manufactura y el comercio. Adonia, situada cerca de la ciudad actual de Ashjabad, se convirtió en la metrópolis comercial de Asia central. El comercio de las piedras, los metales, la madera y la alfarería se desarrolló rápidamente tanto por vía terrestre como por vía fluvial.

\par
%\textsuperscript{(879.2)}
\textsuperscript{79:1.5} Pero la creciente sequía provocó gradualmente el gran éxodo andita desde las tierras situadas al sur y al este del Mar Caspio. El flujo migratorio hacia el norte empezó a dirigirse hacia el sur, y la caballería de Babilonia empezó a entrar en Mesopotamia.

\par
%\textsuperscript{(879.3)}
\textsuperscript{79:1.6} La aridez creciente en Asia central contribuyó además a reducir la población y a hacer que estos pueblos fueran menos belicosos; y cuando las lluvias cada vez más escasas en el norte forzaron a los andonitas nómadas a dirigirse hacia el sur, se produjo un enorme éxodo de anditas desde el Turquestán. Ésta fue la penetración final de los pueblos llamados arios en el Levante y la India. Marcó el punto culminante de la larga dispersión de los descendientes mixtos de Adán, durante la cual estas razas superiores mejoraron hasta cierto punto a todos los pueblos asiáticos y a la mayoría de los pueblos insulares del Pacífico.

\par
%\textsuperscript{(879.4)}
\textsuperscript{79:1.7} Así, mientras se dispersaban por el hemisferio oriental, los anditas fueron desposeídos de sus tierras natales de Mesopotamia y del Turquestán, ya que este inmenso desplazamiento de los andonitas hacia el sur fue el que diluyó a los anditas en Asia central hasta el punto de casi hacerlos desaparecer.

\par
%\textsuperscript{(879.5)}
\textsuperscript{79:1.8} Pero incluso en el siglo veinte después de Cristo, aún quedan restos de sangre andita entre los pueblos turanianos y tibetanos, tal como se puede observar en los tipos rubios que se encuentran de vez en cuando en estas regiones. Los anales chinos primitivos describen la presencia de nómadas pelirrojos al norte de las pacíficas colonias del Río Amarillo, y aún se conservan pinturas que representan fielmente la presencia tanto del tipo rubio andita como del moreno mongol en la cuenca del Tarim de otros tiempos.

\par
%\textsuperscript{(879.6)}
\textsuperscript{79:1.9} La última gran manifestación del genio militar latente de los anditas de Asia central se produjo en el año 1200 d. de J.C. cuando los mongoles, bajo el mando de Gengis Kan, empezaron la conquista de la mayor parte del continente asiático. Y al igual que los antiguos anditas, estos guerreros proclamaron la existencia de <<un solo Dios en el cielo>>\footnote{\textit{Un solo Dios en el cielo}: Gn 24:3,7; 1 Re 8:23; 2 Cr 20:6; 36:23; Esd 1:2; 5:11-12; Neh 1:4-5; Sal 80:14; 136:26; Ec 5:2; Dn 2:18-19,28; Dt 4:39; Jos 2:11.}. La desintegración prematura de su imperio retrasó durante mucho tiempo el intercambio cultural entre Oriente y Occidente, y obstaculizó enormemente el crecimiento de un concepto monoteísta en Asia.

\section*{2. La conquista andita de la India}
\par
%\textsuperscript{(879.7)}
\textsuperscript{79:2.1} La India es el único lugar donde todas las razas de Urantia estaban mezcladas, y la invasión andita añadió el último linaje. Las razas sangiks surgieron a la existencia en las regiones montañosas del noroeste de la India, y en sus comienzos, los miembros de cada raza penetraron sin excepción en el subcontinente de la India, dejando tras ellos la mezcla de razas más heterogénea que jamás haya existido en Urantia. La India antigua fue como un territorio sin salida para las razas que emigraban. La base de la península era antiguamente un poco más angosta que ahora, pues una gran parte de los deltas del Indo y del Ganges se ha formado en los últimos cincuenta mil años.

\par
%\textsuperscript{(879.8)}
\textsuperscript{79:2.2} Las primeras mezclas raciales en la India consistieron en una fusión de las razas migratorias roja y amarilla con los aborígenes andonitas. Este grupo se debilitó más tarde debido a la absorción de la mayor parte de los pueblos verdes orientales ahora extintos, así como de una gran cantidad de individuos de la raza anaranjada; mejoró ligeramente gracias a una mezcla limitada con el hombre azul, pero se deterioró extremadamente al asimilar un gran número de miembros de la raza índiga. Pero los llamados aborígenes de la India apenas son representativos de estos pueblos primitivos; forman más bien la franja más inferior del sur y del este, que nunca fue completamente absorbida por los primeros anditas ni por sus primos arios que aparecieron más tarde.

\par
%\textsuperscript{(880.1)}
\textsuperscript{79:2.3} Hacia el año 20.000 a. de J.C., la población del oeste de la India ya se había impregnado de sangre adámica, y ningún otro pueblo, en toda la historia de Urantia, combinó nunca tantas razas diferentes. Pero es lamentable que predominaran los linajes sangiks secundarios, y fue una auténtica calamidad que los hombres rojos y azules estuvieran tan poco representados en este crisol racial del pasado lejano. Una mayor cantidad de linajes sangiks primarios hubiera contribuido mucho a realzar una civilización que podría haber sido mucho más importante. Tal como se desarrollaron las cosas, los hombres rojos se destruían en las Américas, los hombres azules retozaban en Europa, y los primeros hijos de Adán (así como la mayoría de sus descendientes) mostraban pocos deseos de mezclarse con los pueblos de color más oscuro, ya fuera en la India, en
África o en otras partes.

\par
%\textsuperscript{(880.2)}
\textsuperscript{79:2.4} Hacia el año 15.000 a. de J.C., la presión creciente de la población en todo el Turquestán e Irán produjo la primera emigración realmente importante de los anditas hacia la India. Durante más de quince siglos, estos pueblos superiores entraron en masa a través de las regiones montañosas del Baluchistán, diseminándose por los valles del Indo y del Ganges y desplazándose lentamente hacia el sur dentro del Decán. Esta presión andita procedente del noroeste expulsó a muchos pueblos inferiores del sur y del este hacia Birmania y el sur de China, pero no lo suficiente como para salvar a los invasores de la extinción racial.

\par
%\textsuperscript{(880.3)}
\textsuperscript{79:2.5} La India no consiguió su hegemonía sobre Eurasia debido principalmente a un problema de topografía. La presión de los pueblos que venían del norte se limitó a empujar a la mayoría de la gente hacia el sur, hacia el territorio cada vez más pequeño del Decán, rodeado por el mar por todas partes. Si hubiera habido tierras adyacentes para la emigración, entonces los pueblos inferiores se hubieran diseminado en todas direcciones, y los linajes superiores habrían establecido una civilización más elevada.

\par
%\textsuperscript{(880.4)}
\textsuperscript{79:2.6} Tal como se desarrollaron las cosas, estos conquistadores anditas primitivos hicieron un esfuerzo desesperado por conservar su identidad y detener la marea de la sumersión racial, estableciendo restricciones rígidas para los matrimonios mixtos. A pesar de todo, hacia el año 10.000 a. de J.C., los anditas habían sido absorbidos, pero toda la masa de la población había mejorado notablemente gracias a esta absorción.

\par
%\textsuperscript{(880.5)}
\textsuperscript{79:2.7} Las mezclas raciales siempre son ventajosas, ya que favorecen una cultura polifacética y contribuyen al progreso de la civilización, pero si predominan los elementos inferiores de los linajes raciales, estos logros serán de corta duración. Una cultura políglota sólo se puede conservar si los linajes superiores se reproducen con un margen de seguridad sobre los inferiores. La multiplicación incontrolada de los inferiores, unida a la reproducción decreciente de los superiores, conduce infaliblemente al suicidio de la civilización cultural.

\par
%\textsuperscript{(880.6)}
\textsuperscript{79:2.8} Si los conquistadores anditas hubieran sido tres veces más numerosos de lo que lo fueron, o si hubieran expulsado o destruido a la tercera parte menos deseable de los habitantes anaranjados, verdes e índigos mezclados, entonces la India se hubiera convertido en uno de los principales centros mundiales de la civilización cultural, y hubiera atraído indudablemente a una mayor cantidad de las oleadas posteriores de mesopotámicos que inundaron el Turquestán y desde allí se dirigieron hacia el norte hasta llegar a Europa.

\section*{3. La India dravidiana}
\par
%\textsuperscript{(881.1)}
\textsuperscript{79:3.1} La mezcla de los conquistadores anditas de la India con el linaje nativo se tradujo finalmente en la aparición de los pueblos mixtos que han sido llamados dravidianos. Los primeros dravidianos más puros poseían una gran capacidad para los logros culturales, que se debilitó continuamente a medida que su herencia andita se atenuó de manera progresiva. Y esto fue lo que condenó al fracaso a la civilización en ciernes de la India hace cerca de doce mil años. Pero incluso la inyección de esta pequeña cantidad de sangre de Adán produjo una aceleración apreciable del desarrollo social. Este linaje compuesto dio inmediatamente nacimiento a la civilización más polifacética que existía entonces en la Tierra.

\par
%\textsuperscript{(881.2)}
\textsuperscript{79:3.2} Poco tiempo después de conquistar la India, los anditas dravidianos perdieron su contacto racial y cultural con Mesopotamia, pero estas relaciones se restablecieron gracias a la apertura posterior de las líneas marítimas y de las rutas de las caravanas. En los últimos diez mil años, la India no ha estado en ningún momento totalmente desconectada de Mesopotamia en el oeste y de China en el este, aunque las barreras montañosas favorecían enormemente el intercambio con el oeste.

\par
%\textsuperscript{(881.3)}
\textsuperscript{79:3.3} La cultura superior y las tendencias religiosas de los pueblos de la India datan de los primeros tiempos de la dominación dravidiana y se deben, en parte, al hecho de que un gran número de sacerdotes setitas entró en la India tanto con las primeras invasiones anditas como con las invasiones arias posteriores. El hilo conductor de monoteísmo que atraviesa la historia religiosa de la India proviene así de las enseñanzas de los adamitas en el segundo jardín.

\par
%\textsuperscript{(881.4)}
\textsuperscript{79:3.4} En una fecha tan temprana como el año 16.000 a. de J.C., un grupo de cien sacerdotes setitas penetró en la India y estuvo a punto de conquistar religiosamente la mitad occidental de este pueblo políglota, pero su religión no sobrevivió. En el espacio de cinco mil años, sus doctrinas sobre la Trinidad del Paraíso\footnote{\textit{Trinidad del Paraíso}: Job 5:7; Mt 28:19; Hch 2:32-33; 2 Co 13:14. \textit{Antigua visión de la Trinidad del Paraíso}: 1 Co 12:4-6.} habían degenerado en el símbolo trino del dios del fuego.

\par
%\textsuperscript{(881.5)}
\textsuperscript{79:3.5} Pero durante más de siete mil años y hasta el final de las emigraciones anditas, el nivel religioso de los habitantes de la India fue muy superior al del resto del mundo. Durante aquellos tiempos, la India prometía dar nacimiento a la civilización cultural, religiosa, filosófica y comercial más avanzada del mundo. Si los anditas no hubieran sido completamente absorbidos por los pueblos del sur, este destino probablemente se hubiera realizado.

\par
%\textsuperscript{(881.6)}
\textsuperscript{79:3.6} Los centros culturales dravidianos estaban situados en los valles de los ríos, principalmente del Indo y del Ganges, y en el Decán a lo largo de los tres grandes ríos que fluyen a través de los Ghates orientales hacia el mar. Las colonias a lo largo de la costa de los Ghates occidentales debieron su importancia a las relaciones marítimas con Sumeria.

\par
%\textsuperscript{(881.7)}
\textsuperscript{79:3.7} Los dravidianos figuran entre los primeros pueblos que construyeron ciudades y que se dedicaron a un extenso comercio de importaciones y exportaciones, tanto por tierra como por mar. Hacia el año 7000 a. de J.C., las caravanas de camellos viajaban regularmente hasta la lejana Mesopotamia. Los barcos dravidianos navegaban a lo largo de la costa a través del mar de Arabia hasta las ciudades sumerias del Golfo Pérsico, y se aventuraban en las aguas del Golfo de Bengala hasta las Indias Orientales. Estos navegantes y mercaderes importaron de Sumeria un alfabeto así como el arte de la escritura.

\par
%\textsuperscript{(881.8)}
\textsuperscript{79:3.8} Estas relaciones comerciales contribuyeron enormemente a diversificar aún más una cultura ya cosmopolita, provocando la rápida aparición de una gran parte de los refinamientos, e incluso de los lujos, de la vida urbana. Cuando los arios que llegaron más tarde entraron en la India, no reconocieron en los dravidianos a sus primos anditas ya absorbidos por las razas sangiks, pero sí encontraron una civilización bien desarrollada. A pesar de sus limitaciones biológicas, los dravidianos habían fundado una civilización superior que se había difundido por toda la India y que ha sobrevivido en el Decán hasta los tiempos modernos.

\section*{4. La invasión aria de la India}
\par
%\textsuperscript{(882.1)}
\textsuperscript{79:4.1} La segunda penetración andita en la India fue la invasión aria que tuvo lugar durante un período de casi quinientos años a mediados del tercer milenio a. de J.C. Esta emigración marcó el éxodo final de los anditas desde sus tierras natales del Turquestán.

\par
%\textsuperscript{(882.2)}
\textsuperscript{79:4.2} Los primeros centros arios estaban diseminados por la mitad norte de la India, sobre todo en el noroeste. Estos invasores no completaron nunca la conquista del país, y esta negligencia causó posteriormente su ruina porque su inferioridad numérica los hizo vulnerables a la absorción por los dravidianos del sur, que invadieron más tarde toda la península, a excepción de las provincias del Himalaya.

\par
%\textsuperscript{(882.3)}
\textsuperscript{79:4.3} Los arios dejaron muy poca huella racial en la India, salvo en las provincias del norte. Su influencia en el Decán fue cultural y religiosa más bien que racial. La permanencia más prolongada de la llamada sangre aria en el norte de la India no se debe solamente a su presencia más numerosa en estas regiones, sino también al hecho de que fueron reforzados por los conquistadores, comerciantes y misioneros posteriores. Hasta el primer siglo antes de Cristo hubo una continua infiltración de sangre aria en el Punjab, y la última afluencia se produjo en el momento de las campañas de los pueblos helénicos.

\par
%\textsuperscript{(882.4)}
\textsuperscript{79:4.4} Los arios y los dravidianos se mezclaron finalmente en las llanuras del Ganges y dieron nacimiento a una cultura elevada; este centro fue reforzado más tarde con las aportaciones del nordeste procedentes de China.

\par
%\textsuperscript{(882.5)}
\textsuperscript{79:4.5} En la India florecieron de vez en cuando muchos tipos de organizaciones sociales, desde los sistemas semidemocráticos de los arios hasta las formas de gobierno despóticas y monárquicas. Pero el rasgo más característico de la sociedad fue la persistencia de las grandes castas sociales instituidas por los arios en un esfuerzo por perpetuar su identidad racial. Este elaborado sistema de castas se ha conservado hasta la época actual.

\par
%\textsuperscript{(882.6)}
\textsuperscript{79:4.6} De las cuatro grandes castas existentes, todas, a excepción de la primera, fueron establecidas con la inútil finalidad de impedir la fusión racial de los conquistadores arios con sus súbditos inferiores. Pero la casta principal, la de los sacerdotes-instructores, proviene de los setitas. Los brahmanes del siglo veinte después de Cristo son los descendientes culturales en línea directa de los sacerdotes del segundo jardín, aunque sus enseñanzas difieren enormemente de las de sus ilustres predecesores.

\par
%\textsuperscript{(882.7)}
\textsuperscript{79:4.7} Cuando los arios penetraron en la India, llevaban consigo sus conceptos de la Deidad tal como éstos se habían conservado en las tradiciones sobrevivientes de la religión del segundo jardín. Pero los sacerdotes brahmanes nunca fueron capaces de oponerse al ímpetu pagano fortalecido por el contacto repentino con las religiones inferiores del Decán después de la desaparición racial de los arios. La gran mayoría de la población cayó así en el cautiverio de las supersticiones esclavizantes de las religiones inferiores; y así es como la India no logró producir la civilización elevada que se había presagiado en épocas anteriores.

\par
%\textsuperscript{(882.8)}
\textsuperscript{79:4.8} El despertar espiritual del siglo sexto antes de Cristo no sobrevivió en la India, e incluso había desaparecido antes de la invasión mahometana. Pero algún día es posible que surja un Gautama aún más grande que conduzca a toda la India a la búsqueda del Dios viviente, y entonces el mundo podrá observar la realización de los potenciales culturales de un pueblo multifacético que ha permanecido tanto tiempo en coma bajo la influencia paralizante de una visión espiritual no progresiva.

\par
%\textsuperscript{(883.1)}
\textsuperscript{79:4.9} La cultura descansa sobre una base biológica, pero las castas por sí solas no podían perpetuar la cultura aria, porque la religión, la verdadera religión, es la fuente indispensable de esa energía más elevada que impulsa a los hombres a establecer una civilización superior basada en la fraternidad humana.

\section*{5. Los hombres rojos y los hombres amarillos}
\par
%\textsuperscript{(883.2)}
\textsuperscript{79:5.1} Mientras que la historia de la India es la historia de la conquista de los anditas y de su absorción final por los pueblos evolutivos más antiguos, la historia de Asia oriental es más bien la historia de los sangiks primarios, en particular de los hombres rojos y amarillos. Estas dos razas evitaron en gran parte mezclarse con el linaje degradado de Neandertal que tanto retrasó a los hombres azules en Europa, conservando así el potencial superior del tipo sangik primario.

\par
%\textsuperscript{(883.3)}
\textsuperscript{79:5.2} Los primeros hombres de Neandertal se habían extendido a todo lo ancho de Eurasia, pero la rama oriental era la que estaba más contaminada con las cepas animales degradadas. Estos tipos subhumanos fueron empujados hacia el sur por el quinto glaciar, por la misma capa de hielo que bloqueó durante tanto tiempo la emigración sangik hacia el este de Asia. Cuando el hombre rojo se dirigió hacia el nordeste bordeando las regiones montañosas de la India, encontró que el nordeste de Asia estaba libre de estos tipos subhumanos. Las razas rojas se organizaron en tribus más pronto que todos los demás pueblos, y fueron las primeras que emigraron del centro sangik de Asia central. Los linajes inferiores de Neandertal fueron destruidos o expulsados del continente por las tribus amarillas que emigraron más tarde. Pero el hombre rojo había reinado de manera suprema en el este de Asia durante cerca de cien mil años antes de que llegaran las tribus amarillas.

\par
%\textsuperscript{(883.4)}
\textsuperscript{79:5.3} Hace más de trescientos mil años, la masa principal de la raza amarilla entró en China bajo la forma de emigrantes que subían por la costa desde el sur. Cada milenio penetraron más hacia el interior, pero no entablaron contacto con sus hermanos tibetanos migratorios hasta una época relativamente reciente.

\par
%\textsuperscript{(883.5)}
\textsuperscript{79:5.4} La presión creciente de la población hizo que la raza amarilla que se desplazaba hacia el norte empezara a penetrar en los territorios de caza del hombre rojo. Esta intrusión, unida a un antagonismo racial natural, culminó en hostilidades crecientes, y así empezó la lucha decisiva por las tierras fértiles del Asia lejana.

\par
%\textsuperscript{(883.6)}
\textsuperscript{79:5.5} El relato de esta contienda secular entre las razas roja y amarilla es una epopeya de la historia de Urantia. Durante más de doscientos mil años, estas dos razas superiores libraron una guerra encarnizada e incesante. Los hombres rojos vencieron generalmente en las primeras batallas y sus incursiones hicieron estragos entre las colonias amarillas. Pero los hombres amarillos eran unos buenos alumnos en el arte de la guerra, y pronto manifestaron una destacada capacidad para vivir en paz con sus compatriotas. Los chinos fueron los primeros en aprender que la unión hace la fuerza. Las tribus rojas continuaron con sus conflictos de aniquilación mutua, y pronto empezaron a sufrir repetidas derrotas a manos de los agresivos e implacables chinos, que continuaban su marcha inexorable hacia el norte.

\par
%\textsuperscript{(883.7)}
\textsuperscript{79:5.6} Hace cien mil años, las tribus diezmadas de la raza roja se encontraban luchando de espaldas a los hielos del último glaciar en retroceso, y cuando el pasaje terrestre hacia el este por el istmo de Bering se hizo transitable, estas tribus no tardaron en abandonar las costas inhóspitas del continente asiático. Hace ahora ochenta y cinco mil años que los últimos hombres rojos de raza pura partieron de Asia, pero la larga lucha dejó su huella genética sobre la raza amarilla victoriosa. Los pueblos chinos del norte, junto con los siberianos andonitas, asimilaron una gran parte del linaje rojo y obtuvieron con ello un beneficio considerable.

\par
%\textsuperscript{(884.1)}
\textsuperscript{79:5.7} Los indios norteamericanos nunca se pusieron en contacto ni siquiera con los descendientes anditas de Adán y Eva, ya que habían sido desposeídos de sus tierras natales de Asia unos cincuenta mil años antes de la llegada de Adán. Durante la época de las emigraciones anditas, los linajes rojos puros se estaban diseminando por América del Norte como tribus nómadas, como cazadores que practicaban la agricultura en pequeña medida. Estas razas y grupos culturales permanecieron casi completamente aislados del resto del mundo desde su llegada a las Américas hasta el final del primer milenio de la era cristiana, cuando fueron descubiertos por las razas blancas de Europa. Hasta ese momento, los esquimales eran lo más parecido a un hombre blanco que las tribus nórdicas de hombres rojos hubieran visto nunca.

\par
%\textsuperscript{(884.2)}
\textsuperscript{79:5.8} Las razas roja y amarilla son las únicas razas humanas que alcanzaron un alto grado de civilización fuera de la influencia de los anditas. El centro cultural amerindio más antiguo fue el de Onamonalontón, en California, pero en el año 35.000 a. de J.C. hacía mucho tiempo que había desaparecido. En Méjico, en América Central y en las montañas de América del Sur, las civilizaciones posteriores y más duraderas fueron fundadas por una raza predominantemente roja, pero que contenía una mezcla considerable de componentes amarillos, anaranjados y azules.

\par
%\textsuperscript{(884.3)}
\textsuperscript{79:5.9} Estas civilizaciones fueron un producto evolutivo de los sangiks, aunque una pequeña cantidad de sangre andita llegó hasta el Perú. A excepción de los esquimales en América del Norte y de algunos anditas polinesios en América del Sur, los pueblos del hemisferio occidental no tuvieron ningún contacto con el resto del mundo hasta el final del primer milenio después de Cristo. En el plan original de los Melquisedeks para mejorar las razas de Urantia se había establecido que un millón de descendientes en línea directa de Adán irían hasta las Américas para elevar a los hombres rojos.

\section*{6. Los albores de la civilización china}
\par
%\textsuperscript{(884.4)}
\textsuperscript{79:6.1} Algún tiempo después de haber expulsado a los hombres rojos hacia América del Norte, los chinos en expansión echaron a los andonitas de los valles fluviales del este de Asia, empujándolos hacia Siberia en el norte y hacia el Turquestán en el oeste, donde pronto se pondrían en contacto con la cultura superior de los anditas.

\par
%\textsuperscript{(884.5)}
\textsuperscript{79:6.2} Las culturas de la India y de China se unieron y se mezclaron en Birmania y en la península de Indochina para dar nacimiento a las civilizaciones sucesivas de estas regiones. Aquí, la raza verde desaparecida ha subsistido en mayor proporción que en cualquier otra parte del mundo.

\par
%\textsuperscript{(884.6)}
\textsuperscript{79:6.3} Muchas razas diferentes ocuparon las islas del Pacífico. En general, las islas del sur, que eran entonces más grandes, estaban habitadas por pueblos que tenían un alto porcentaje de sangre verde e índiga. Las islas del norte estaban dominadas por los andonitas, y más tarde por razas que contenían una gran proporción de los linajes rojos y amarillos. Los antepasados del pueblo japonés no fueron arrojados del continente hasta el año 12.000 a. de J.C., momento en que fueron expulsados debido a la poderosa presión de las tribus chinas nórdicas que se dirigían hacia el sur a lo largo de la costa. Su éxodo final no se debió tanto a la presión de la población como a la iniciativa de un cacique a quien llegaron a considerar como un personaje divino.

\par
%\textsuperscript{(885.1)}
\textsuperscript{79:6.4} Al igual que los pueblos de la India y del Levante, las tribus victoriosas de los hombres amarillos establecieron sus primeros centros a lo largo de la costa y remontando el curso de los ríos. A las colonias costeras les fue mal en los años posteriores a medida que las inundaciones crecientes y el curso cambiante de los ríos hicieron insostenible la vida en las ciudades de las tierras bajas.

\par
%\textsuperscript{(885.2)}
\textsuperscript{79:6.5} Hace veinte mil años, los antepasados de los chinos habían construido una docena de poderosos centros de cultura y enseñanza primitivas, especialmente a lo largo del Río Amarillo y del Yang-tsé. Estos centros empezaron luego a reforzarse con la llegada de una corriente continua de pueblos mixtos superiores procedentes del Sinkiang y del Tíbet. La emigración desde el Tíbet hacia el valle del Yang-tsé no fue tan grande como en el norte, y los centros tibetanos tampoco eran tan avanzados como los de la cuenca del Tarim. Pero los dos movimientos migratorios llevaron cierta cantidad de sangre andita hacia las colonias ribereñas del este.

\par
%\textsuperscript{(885.3)}
\textsuperscript{79:6.6} La superioridad de la antigua raza amarilla se debía a cuatro grandes factores:

\par
%\textsuperscript{(885.4)}
\textsuperscript{79:6.7} 1. \textit{El factor genético}. A diferencia de sus primos azules de Europa, tanto la raza roja como la amarilla se habían librado ampliamente de mezclarse con los linajes humanos degradados. Los chinos del norte, ya reforzados con pequeñas cantidades de los linajes rojos y andonitas superiores, iban a beneficiarse pronto de una afluencia considerable de sangre andita. A los chinos del sur no les fue tan bien en este sentido; ya habían sufrido durante mucho tiempo las consecuencias de la absorción de la raza verde, y más tarde se debilitaron aún más debido a la infiltración de una multitud de pueblos inferiores que fueron expulsados de la India por la invasión andito-dravidiana. Hoy día existe en China una clara diferencia entre las razas del norte y las del sur.

\par
%\textsuperscript{(885.5)}
\textsuperscript{79:6.8} 2. \textit{El factor social}. La raza amarilla aprendió muy pronto el valor de vivir en paz entre ellos. Su pacifismo interno contribuyó de tal manera a aumentar la población, que aseguró la diseminación de su civilización entre millones de personas. Desde el año 25.000 hasta el 5000 a. de J.C., la mayor cantidad de hombres civilizados de Urantia se encontraba en el centro y norte de China. El hombre amarillo fue el primero que logró una solidaridad racial ---el primero que alcanzó una civilización cultural, social y política a gran escala.

\par
%\textsuperscript{(885.6)}
\textsuperscript{79:6.9} Los chinos del año 15.000 a. de J.C. eran unos militaristas enérgicos; no se habían debilitado a causa de un respeto excesivo por el pasado, y como eran menos de doce millones, formaban una masa compacta que hablaba un idioma común. Durante esta época construyeron una verdadera nación, mucho más unida y homogénea que sus uniones políticas de los tiempos históricos.

\par
%\textsuperscript{(885.7)}
\textsuperscript{79:6.10} 3. \textit{El factor espiritual}. Durante la era de las emigraciones anditas, los chinos se encontraban entre los pueblos más espirituales de la Tierra. Su prolongada adhesión al culto de la Verdad Única proclamada por Singlangtón los mantuvo por delante de la mayoría de las otras razas. El estímulo de una religión avanzada y progresiva es a menudo un factor decisivo en el desarrollo cultural. Mientras la India languidecía, China hacía grandes progresos bajo el estímulo vigorizador de una religión en la que la verdad se conservaba como si fuera la Deidad suprema.

\par
%\textsuperscript{(885.8)}
\textsuperscript{79:6.11} Esta adoración de la verdad estimulaba la investigación y la exploración intrépida de las leyes de la naturaleza y los potenciales de la humanidad. Incluso los chinos de hace seis mil años continuaban siendo unos estudiantes agudos y dinámicos en su búsqueda de la verdad.

\par
%\textsuperscript{(885.9)}
\textsuperscript{79:6.12} 4. \textit{El factor geográfico}. China está protegida al oeste por las montañas y al este por el Pacífico. La única vía abierta para los ataques se encuentra en el norte, y desde los tiempos de los hombres rojos hasta la llegada de los descendientes posteriores de los anditas, el norte nunca estuvo ocupado por una raza agresiva.

\par
%\textsuperscript{(886.1)}
\textsuperscript{79:6.13} Si no hubiera sido por las barreras montañosas y la decadencia posterior de su cultura espiritual, la raza amarilla habría atraído sin duda hacia ella la mayor parte de la emigración andita del Turquestán e, indiscutiblemente, hubiera dominado rápidamente la civilización del mundo.

\section*{7. Los anditas entran en China}
\par
%\textsuperscript{(886.2)}
\textsuperscript{79:7.1} Hace unos quince mil años, los anditas atravesaron en grandes cantidades el desfiladero de Ti Tao y se diseminaron por el valle superior del Río Amarillo entre las colonias chinas de Kansu. Luego penetraron hacia el este hasta llegar a Honan, donde se encontraban las colonias más progresivas. Esta infiltración procedente del oeste fue casi mitad andonita y mitad andita.

\par
%\textsuperscript{(886.3)}
\textsuperscript{79:7.2} Los centros culturales del norte, situados a lo largo del Río Amarillo, siempre habían sido más progresivos que las colonias meridionales del Yang-tsé. Pocos miles de años después de la llegada de estos mortales superiores, aunque fueran poco numerosos, las colonias del Río Amarillo habían adelantado a los pueblos del Yang-tsé y habían alcanzado una posición avanzada sobre sus hermanos del sur, que han conservado desde entonces.

\par
%\textsuperscript{(886.4)}
\textsuperscript{79:7.3} Los anditas no fueron muy numerosos y su cultura no era tan superior, pero la fusión con ellos produjo un linaje más polifacético. Los chinos del norte recibieron la suficiente sangre andita como para estimular ligeramente la capacidad innata de sus mentes, pero no la suficiente como para encender la inquieta curiosidad exploratoria tan característica de las razas blancas del norte. Esta inyección más limitada de herencia andita fue menos perturbadora para la estabilidad innata del tipo sangik.

\par
%\textsuperscript{(886.5)}
\textsuperscript{79:7.4} Las oleadas posteriores de anditas trajeron consigo algunos progresos culturales de Mesopotamia; esto es particularmente cierto en lo que se refiere a las últimas oleadas migratorias procedentes del oeste. Éstas mejoraron enormemente las prácticas económicas y educativas de los chinos del norte, y aunque su influencia sobre la cultura religiosa de la raza amarilla fue efímera, sus descendientes posteriores contribuyeron mucho a que se produjera un despertar espiritual ulterior. Pero las tradiciones anditas de la belleza del Edén y Dalamatia influyeron en las tradiciones chinas. Las primeras leyendas chinas sitúan <<la tierra de los dioses>> en el oeste.

\par
%\textsuperscript{(886.6)}
\textsuperscript{79:7.5} El pueblo chino no empezó a construir ciudades y a dedicarse a la manufactura hasta después del año 10.000 a. de J.C., con posterioridad a los cambios climáticos en el Turquestán y a la llegada de los últimos inmigrantes anditas. La inyección de esta sangre nueva no añadió gran cosa a la civilización de los hombres amarillos, pero sí estimuló un nuevo y rápido desarrollo de las tendencias latentes de los linajes superiores chinos. Desde Honan hasta Shensi, los potenciales de una civilización avanzada empezaron a manifestarse. El trabajo de los metales y todas las artes de la manufactura datan de esta época.

\par
%\textsuperscript{(886.7)}
\textsuperscript{79:7.6} Las similitudes entre algunos métodos de los chinos y mesopotámicos primitivos para el cálculo del tiempo, la astronomía y la administración gubernamental se debían a las relaciones comerciales entre estos dos centros tan alejados entre sí. Incluso en los tiempos de los sumerios, los mercaderes chinos recorrían las rutas terrestres que atravesaban el Turquestán hasta llegar a Mesopotamia. Este intercambio no fue unilateral ---el valle del Éufrates se benefició considerablemente de él así como los pueblos de la llanura del Ganges. Pero los cambios climáticos y las invasiones nómadas del tercer milenio antes de Cristo redujeron enormemente el volumen del comercio que pasaba por las pistas de las caravanas de Asia central.

\section*{8. La civilización china posterior}
\par
%\textsuperscript{(887.1)}
\textsuperscript{79:8.1} Mientras que los hombres rojos sufrieron las consecuencias de haber tenido demasiadas guerras, no es del todo incorrecto decir que la minuciosa conquista de Asia retrasó el desarrollo del Estado entre los chinos. Tenían un gran potencial de solidaridad racial que no llegó a desarrollarse adecuadamente porque les faltó el continuo estímulo impulsor del peligro siempre presente de una agresión procedente del exterior.

\par
%\textsuperscript{(887.2)}
\textsuperscript{79:8.2} El antiguo Estado militar se desintegró gradualmente cuando finalizó la conquista de Asia oriental ---las guerras del pasado fueron olvidadas. De las luchas épicas contra la raza roja sólo subsistió la vaga tradición de un antiguo enfrentamiento con los pueblos de los arqueros. Los chinos se orientaron pronto hacia los trabajos agrícolas, lo cual acrecentó sus tendencias pacíficas, y el hecho de que la proporción entre los hombres y las tierras fuera muy baja para una población agrícola contribuyó aún más a que la vida fuera cada vez más sosegada en el país.

\par
%\textsuperscript{(887.3)}
\textsuperscript{79:8.3} La conciencia de los éxitos del pasado (un poco atenuada en la actualidad), el conservadurismo de un pueblo en su inmensa mayoría agrícola y una vida familiar bien desarrollada dieron nacimiento a la veneración de los antepasados, que culminó en la costumbre de honrar a los hombres del pasado hasta el punto de rayar en la adoración. Una actitud muy similar prevaleció entre las razas blancas de Europa durante cerca de quinientos años después de la desintegración de la civilización grecorromana.

\par
%\textsuperscript{(887.4)}
\textsuperscript{79:8.4} La creencia y la adoración de la <<Verdad
Única>>\footnote{\textit{Verdad
Única}: Jn 14:6.}, tal como la había enseñado Singlangtón, nunca desapareció por completo; pero a medida que el tiempo pasaba, la tendencia creciente a venerar lo que ya estaba establecido eclipsó la búsqueda de una verdad nueva y más elevada. El genio de la raza amarilla se desvió lentamente de la búsqueda de lo desconocido hacia la conservación de lo conocido. Y ésta es la razón del estancamiento de lo que había sido la civilización que había progresado más rápidamente en el mundo.

\par
%\textsuperscript{(887.5)}
\textsuperscript{79:8.5} La reunificación política de la raza amarilla se consumó entre los años 4000 y 500 a. de J.C., pero la unión cultural entre los centros del Yang-tsé y del Río Amarillo ya se había efectuado. Esta reunificación política de los últimos grupos tribales no se llevó a cabo sin conflictos, pero la sociedad tenía una mala opinión de la guerra. El culto de los antepasados, el aumento de los dialectos y la ausencia de llamamientos para las acciones militares durante miles y miles de años habían vuelto a este pueblo ultrapacífico.

\par
%\textsuperscript{(887.6)}
\textsuperscript{79:8.6} A pesar de que no logró cumplir la promesa de desarrollar rápidamente un Estado avanzado, la raza amarilla avanzó progresivamente en la realización de las artes de la civilización, especialmente en los campos de la agricultura y la horticultura. Los problemas hidráulicos con los que se enfrentaban los agricultores de Shensi y Honan necesitaban una cooperación colectiva para poder solucionarlos. Estas dificultades relacionadas con el riego y la conservación del suelo contribuyeron en gran parte al desarrollo de la interdependencia, con el consiguiente fomento de la paz entre los grupos agrícolas.

\par
%\textsuperscript{(887.7)}
\textsuperscript{79:8.7} El rápido desarrollo de la escritura, junto con la creación de escuelas, contribuyeron a diseminar el conocimiento a una escala desconocida hasta entonces. Pero la naturaleza engorrosa del sistema de escritura ideográfica limitó el número de las clases cultas, a pesar de la aparición temprana de la imprenta. El proceso de uniformación social y la dogmatización religioso-filosófica continuó rápidamente por encima de todo lo demás. El desarrollo religioso de la veneración de los antepasados se complicó aún más debido a un torrente de supersticiones que incluían la adoración de la naturaleza, pero los vestigios sobrevivientes de un verdadero concepto de Dios permanecieron conservados en la adoración imperial de Shang-ti.

\par
%\textsuperscript{(888.1)}
\textsuperscript{79:8.8} La gran debilidad de la veneración de los antepasados consiste en que fomenta una filosofía centrada en el pasado. Por muy acertado que sea cosechar la sabiduría del pasado, es una locura considerar que el pasado es la fuente exclusiva de la verdad. La verdad es relativa y expansiva; \textit{vive} siempre en el presente, alcanzando nuevas expresiones en cada generación de hombres ---e incluso en cada vida humana.

\par
%\textsuperscript{(888.2)}
\textsuperscript{79:8.9} La gran fuerza de la veneración de los antepasados es el valor que esta actitud atribuye a la familia. La estabilidad y la persistencia asombrosas de la cultura china son una consecuencia de la posición suprema en que sitúan a la familia, porque la civilización depende directamente del funcionamiento eficaz de la familia. La familia alcanzó en China una importancia social, e incluso un significado religioso, que muy pocos pueblos han sabido alcanzar.

\par
%\textsuperscript{(888.3)}
\textsuperscript{79:8.10} La devoción filial y la lealtad familiar que exigía el culto creciente de la adoración de los antepasados aseguró el establecimiento de unas relaciones familiares superiores y de unos grupos familiares duraderos, todo lo cual facilitó los siguientes factores protectores de la civilización:

\par
%\textsuperscript{(888.4)}
\textsuperscript{79:8.11} 1. La conservación de los bienes y de la riqueza.

\par
%\textsuperscript{(888.5)}
\textsuperscript{79:8.12} 2. La puesta en común de la experiencia de diversas generaciones.

\par
%\textsuperscript{(888.6)}
\textsuperscript{79:8.13} 3. La educación eficaz de los niños en las artes y las ciencias del pasado.

\par
%\textsuperscript{(888.7)}
\textsuperscript{79:8.14} 4. El desarrollo de un fuerte sentido del deber, la elevación de la moralidad y el aumento de la sensibilidad ética.

\par
%\textsuperscript{(888.8)}
\textsuperscript{79:8.15} El período formativo de la civilización china, que empieza con la llegada de los anditas, continúa hasta el gran despertar ético, moral y semirreligioso del siglo sexto antes de Cristo. Y la tradición china conserva la información nebulosa del pasado evolutivo; la transición de la familia matriarcal a la familia patriarcal, el establecimiento de la agricultura, el desarrollo de la arquitectura, el comienzo de la industria ---todo esto se narra de manera sucesiva. Esta historia presenta, con mayor precisión que cualquier otro relato similar, la imagen de la magnífica ascensión de un pueblo superior a partir de los niveles de la barbarie. Durante este período, los chinos pasaron de una sociedad agrícola primitiva a una organización social más elevada que abarcaba la construcción de ciudades, la manufactura, el trabajo de los metales, el intercambio comercial, un gobierno, la escritura, las matemáticas, el arte, la ciencia y la imprenta.

\par
%\textsuperscript{(888.9)}
\textsuperscript{79:8.16} Así es como la antigua civilización de la raza amarilla ha perdurado a través de los siglos. Hace cerca de cuarenta mil años que se produjeron los primeros progresos importantes en la cultura china, y aunque ha habido muchos retrocesos, la civilización de los hijos de Han es la que presenta, mejor que cualquier otra, una imagen ininterrumpida de progreso continuo que llega hasta la época del siglo veinte. Los desarrollos religiosos y mecánicos de las razas blancas han sido de un orden elevado, pero nunca han superado a los chinos en lealtad familiar, en ética colectiva o en moralidad personal.

\par
%\textsuperscript{(888.10)}
\textsuperscript{79:8.17} Esta antigua cultura ha contribuido mucho a la felicidad humana; millones de seres humanos han vivido y han muerto bendecidos por sus logros. Esta gran civilización ha reposado durante siglos sobre los laureles del pasado, pero en este momento se está despertando de nuevo para visualizar otra vez las metas trascendentes de la existencia mortal, para reanudar una vez más la lucha incesante por el progreso sin fin.

\par
%\textsuperscript{(888.11)}
\textsuperscript{79:8.18} [Presentado por un Arcángel de Nebadon.]


\chapter{Documento 80. La expansión andita en Occidente}
\par
%\textsuperscript{(889.1)}
\textsuperscript{80:0.1} AUNQUE el hombre azul europeo no alcanzó por sí mismo una gran civilización cultural, suministró una base biológica impregnada de linajes adamizados; cuando éstos se mezclaron con los invasores anditas posteriores, produjeron una de las razas más poderosas capaces de conseguir una civilización dinámica como no había aparecido otra en Urantia desde los tiempos de la raza violeta y de sus sucesores anditas.

\par
%\textsuperscript{(889.2)}
\textsuperscript{80:0.2} Los pueblos blancos modernos contienen los linajes sobrevivientes de la estirpe adámica que se mezclaron con las razas sangiks, es decir con algunos hombres rojos y amarillos, pero sobre todo con los hombres azules. Todas las razas blancas contienen un porcentaje considerable del linaje andonita original y aún mucho más de las primeras estirpes noditas.

\section*{1. Los adamitas entran en Europa}
\par
%\textsuperscript{(889.3)}
\textsuperscript{80:1.1} Antes de que los últimos anditas fueran expulsados del valle del
Éufrates, muchos hermanos suyos habían penetrado en Europa como aventureros, educadores, comerciantes y guerreros. Durante los primeros tiempos de la raza violeta, la depresión mediterránea estaba protegida por el istmo de Gibraltar y el puente terrestre de Sicilia. Una parte del comercio marítimo inicial del hombre se estableció en estos lagos interiores, donde los hombres azules del norte y los saharianos del sur se encontraron con los noditas y los adamitas del este.

\par
%\textsuperscript{(889.4)}
\textsuperscript{80:1.2} Los noditas habían establecido uno de sus centros culturales más extensos en la depresión oriental del Mediterráneo, y desde allí habían penetrado un poco en el sur de Europa pero principalmente en el norte de África. Los sirios nodito-andonitas de cabeza ancha introdujeron muy pronto la alfarería y la agricultura en sus colonias del delta del Nilo, el cual se elevaba lentamente. Importaron también ovejas, cabras, ganado y otros animales domésticos, e introdujeron métodos muy perfeccionados para trabajar los metales, ya que Siria era entonces el centro de esta industria.

\par
%\textsuperscript{(889.5)}
\textsuperscript{80:1.3} Egipto recibió durante más de treinta mil años una oleada continua de mesopotámicos que trajeron su arte y su cultura para enriquecer la del valle del Nilo. Pero la entrada de una gran cantidad de pueblos del Sahara deterioró enormemente la antigua civilización que existía a lo largo del Nilo, de manera que Egipto alcanzó su nivel cultural más bajo hace unos quince mil años.

\par
%\textsuperscript{(889.6)}
\textsuperscript{80:1.4} Pero en tiempos anteriores, los adamitas habían encontrado pocos obstáculos que impidieran su emigración hacia el oeste. El Sahara era un pastizal abierto sembrado de pastores y agricultores. Estos saharianos nunca se dedicaron a la manufactura, ni tampoco fueron constructores de ciudades. Formaban un grupo índigo-negro que poseía abundantes linajes de las razas verde y anaranjada ya extintas. Pero recibieron una cantidad muy limitada de la herencia violeta antes de que el levantamiento de las tierras y el cambio de los vientos cargados de humedad dispersaran los restos de esta civilización próspera y pacífica.

\par
%\textsuperscript{(890.1)}
\textsuperscript{80:1.5} La sangre de Adán ha sido compartida por la mayoría de las razas humanas, pero algunas han recibido más que otras. Las razas mezcladas de la India y los pueblos más oscuros de África no eran atractivos para los adamitas. Se hubieran mezclado libremente con los hombres rojos si éstos no hubieran estado tan alejados en las Américas, y estaban favorablemente dispuestos hacia los hombres amarillos, pero también era difícil acceder a ellos en la lejana Asia. Por consiguiente, cuando los adamitas se sentían impulsados por la aventura o el altruismo, o cuando fueron expulsados del valle del Éufrates, escogieron unirse de manera muy natural con las razas azules de Europa.

\par
%\textsuperscript{(890.2)}
\textsuperscript{80:1.6} Los hombres azules, que entonces dominaban en Europa, no tenían unas prácticas religiosas que repelieran a los primeros emigrantes adamitas, y existía una gran atracción sexual entre la raza violeta y la raza azul. Los mejores hombres azules consideraban como un gran honor que se les permitiera casarse con las adamitas. Todo hombre azul abrigaba la ambición de volverse lo bastante hábil y artístico como para ganar el afecto de una mujer adamita, y la mayor aspiración de una mujer azul superior era recibir las atenciones de un adamita.

\par
%\textsuperscript{(890.3)}
\textsuperscript{80:1.7} Estos hijos migratorios del Edén se unieron lentamente con los tipos superiores de la raza azul, estimulando sus prácticas culturales mientras que exterminaban implacablemente los linajes retrasados de la raza neandertal. Esta técnica para mezclar las razas, combinada con la eliminación de los linajes inferiores, produjo una docena o más de grupos viriles y progresivos de hombres azules superiores, uno de los cuales habéis denominado Cro-Magnon.

\par
%\textsuperscript{(890.4)}
\textsuperscript{80:1.8} Por estas y otras razones, y no era la menos importante que se trataba de las rutas más favorables para la emigración, las primeras oleadas de cultura mesopotámica se dirigieron casi exclusivamente hacia Europa. Estas circunstancias fueron las que determinaron los antecedentes de la civilización europea moderna.

\section*{2. Los cambios climáticos y geológicos}
\par
%\textsuperscript{(890.5)}
\textsuperscript{80:2.1} La expansión inicial de la raza violeta por Europa fue interrumpida bruscamente por ciertos cambios climáticos y geológicos más bien repentinos. Con el retroceso de los campos de hielo septentrionales, los vientos que traían las lluvias del oeste cambiaron hacia el norte, convirtiendo gradualmente las grandes regiones de pastos abiertos del Sahara en un desierto estéril. Esta sequía dispersó a los habitantes morenos de pequeña estatura, ojos negros y cabezas alargadas, que vivían en la gran meseta del Sahara.

\par
%\textsuperscript{(890.6)}
\textsuperscript{80:2.2} Los elementos índigos más puros se dirigieron hacia los bosques de África central en el sur, donde han permanecido desde entonces. Los grupos más mezclados se dispersaron en tres direcciones: las tribus superiores del oeste emigraron a España y desde allí a las regiones adyacentes de Europa, formando el núcleo de las razas mediterráneas posteriores de cabeza alargada y color moreno. La rama menos progresiva del este de la meseta del Sahara emigró a Arabia y desde allí, a través del norte de Mesopotamia y la India, hasta la lejana Ceilán. El grupo central se dirigió hacia el norte y el este hasta el valle del Nilo y penetró en Palestina.

\par
%\textsuperscript{(890.7)}
\textsuperscript{80:2.3} Este sustrato sangik secundario es el que sugiere cierto grado de parentesco entre los pueblos modernos esparcidos desde el Decán, pasando por Irán y Mesopotamia, hasta las dos orillas del mar Mediterráneo.

\par
%\textsuperscript{(890.8)}
\textsuperscript{80:2.4} Hacia la época de estos cambios climáticos en África, Inglaterra se separó del continente y Dinamarca surgió del mar, mientras que el istmo de Gibraltar, que protegía la cuenca occidental del Mediterráneo, se hundió a consecuencia de un terremoto, elevando rápidamente este lago interior hasta el nivel del Océano Atlántico. Poco después se hundió el puente terrestre de Sicilia, creando así un solo Mar Mediterráneo y conectándolo con el Océano Atlántico. Este cataclismo de la naturaleza inundó decenas de poblaciones humanas y causó la mayor pérdida de vidas por inundación de toda la historia del mundo.

\par
%\textsuperscript{(891.1)}
\textsuperscript{80:2.5} Este hundimiento de la cuenca mediterránea redujo inmediatamente los desplazamientos de los adamitas hacia el oeste, mientras que la gran afluencia de saharianos los indujo a buscar salidas para su creciente población hacia el norte y el este del Edén. A medida que los descendientes de Adán dejaban los valles del Tigris y el Éufrates y viajaban hacia el norte, se encontraron con las barreras montañosas y el Mar Caspio, que era entonces más extenso. Durante muchas generaciones, los adamitas cazaron, cuidaron sus rebaños y cultivaron la tierra alrededor de sus colonias desparramadas por todo el Turquestán. Este pueblo magnífico amplió lentamente su territorio hacia Europa. Pero ahora, los adamitas entran en Europa por el este y encuentran que la cultura del hombre azul está miles de años más atrasada que la de Asia, puesto que esta región casi no ha tenido ningún contacto con Mesopotamia.

\section*{3. El hombre azul de Cro-Magnon}
\par
%\textsuperscript{(891.2)}
\textsuperscript{80:3.1} Los antiguos centros de cultura de los hombres azules estaban situados a lo largo de todos los ríos de Europa, pero el Somme es el único que fluye todavía por el mismo cauce que tenía en la época preglacial.

\par
%\textsuperscript{(891.3)}
\textsuperscript{80:3.2} Aunque decimos que el hombre azul ocupaba el continente europeo, había decenas de tipos raciales. Hace incluso treinta y cinco mil años, las razas azules europeas ya eran un pueblo muy mezclado que contenía linajes tanto rojos como amarillos, mientras que en las costas atlánticas y en las regiones de la Rusia actual habían absorbido una cantidad considerable de sangre andonita, y hacia el sur estaban en contacto con los pueblos saharianos. Pero sería inútil intentar enumerar estos diversos grupos raciales.

\par
%\textsuperscript{(891.4)}
\textsuperscript{80:3.3} La civilización europea de este período postadámico inicial era una mezcla única del vigor y el arte de los hombres azules con la imaginación creativa de los adamitas. Los hombres azules eran una raza de gran vigor, pero deterioraron enormemente el estado cultural y espiritual de los adamitas. A estos últimos les resultaba muy difícil inculcar su religión a los cro-mañones, porque muchos de éstos tenían la tendencia de engañar y pervertir a las muchachas. La religión en Europa se mantuvo en el punto más bajo durante diez mil años en comparación con su desarrollo en la India y Egipto.

\par
%\textsuperscript{(891.5)}
\textsuperscript{80:3.4} Los hombres azules eran completamente honrados en todas sus transacciones y estaban totalmente libres de los vicios sexuales de los adamitas mezclados. Respetaban la virginidad y sólo practicaban la poligamia cuando la guerra causaba una falta de hombres.

\par
%\textsuperscript{(891.6)}
\textsuperscript{80:3.5} Los pueblos de Cro-Magnon eran una raza valiente y previsora. Poseían un eficaz sistema de educación para los niños. Los dos padres participaban en estas tareas, y se utilizaba plenamente la ayuda de los hijos mayores. A todos los niños se les enseñaba cuidadosamente a ocuparse de las cavernas, a practicar las artes y a trabajar el sílex. Desde una edad temprana, las mujeres eran muy versadas en las artes domésticas y en una agricultura rudimentaria, mientras que los hombres eran hábiles cazadores y guerreros intrépidos.

\par
%\textsuperscript{(891.7)}
\textsuperscript{80:3.6} Los hombres azules eran cazadores, pescadores, colectores de alimento y expertos constructores de barcos. Fabricaban hachas de piedra, cortaban árboles y construían cabañas de troncos parcialmente subterráneas y con techos de pieles. Existen pueblos en Siberia que todavía construyen cabañas similares. Los cro-mañones del sur vivían generalmente en cavernas y grutas.

\par
%\textsuperscript{(892.1)}
\textsuperscript{80:3.7} Durante los rigores del invierno, no era raro que sus centinelas murieran congelados mientras permanecían de vigilancia nocturna a la entrada de las cavernas. Eran valientes, pero por encima de todo eran artistas; la mezcla con la sangre de Adán aceleró repentinamente su imaginación creativa. El arte del hombre azul tuvo su punto culminante hace unos quince mil años, antes de la época en que las razas de piel más oscura subieran de África hacia el norte a través de España.

\par
%\textsuperscript{(892.2)}
\textsuperscript{80:3.8} Hace unos quince mil años, los bosques alpinos se estaban extendiendo ampliamente. Los cazadores europeos eran empujados hacia los valles fluviales y las orillas del mar por las mismas coacciones climáticas que habían transformado los territorios de caza paradisíacos del mundo en desiertos secos y estériles. A medida que los vientos que traían las lluvias cambiaban hacia el norte, las grandes tierras abiertas de pastoreo de Europa se cubrieron de bosques. Estas grandes modificaciones climáticas, relativamente repentinas, forzaron a las razas de Europa que practicaban la caza en los espacios abiertos a convertirse en pastores y, hasta cierto punto, en pescadores y labradores.

\par
%\textsuperscript{(892.3)}
\textsuperscript{80:3.9} Aunque estos cambios ocasionaron progresos culturales, produjeron ciertas degeneraciones biológicas. Durante la era anterior de la caza, las tribus superiores se habían casado con los prisioneros de guerra de tipo superior y habían destruido invariablemente a los que consideraban inferiores. Pero a medida que empezaron a establecer poblados y a dedicarse a la agricultura y el comercio, comenzaron a conservar a muchos cautivos mediocres como esclavos. La progenie de estos esclavos fue la que tanto deterioró posteriormente todo el tipo Cro-Magnon. La cultura continuó degenerando hasta que recibió un nuevo impulso procedente del este cuando la masiva invasión final de mesopotámicos se extendió por Europa, absorbiendo rápidamente la cultura y el tipo Cro-Magnon e iniciando la civilización de las razas blancas.

\section*{4. Las invasiones anditas de Europa}
\par
%\textsuperscript{(892.4)}
\textsuperscript{80:4.1} Aunque los anditas afluyeron a Europa en una corriente continua, se produjeron siete invasiones principales, y los últimos en llegar vinieron a caballo en tres grandes oleadas. Algunos entraron en Europa por las islas del mar Egeo y remontando el valle del Danubio, pero la mayoría de los primeros linajes más puros emigraron al noroeste de Europa por la ruta del norte a través de las tierras de pastoreo del Volga y el Don.

\par
%\textsuperscript{(892.5)}
\textsuperscript{80:4.2} Entre la tercera y la cuarta invasión, una horda de andonitas penetró en Europa por el norte después de venir desde Siberia por los ríos rusos y el Báltico. Fueron asimilados inmediatamente por las tribus anditas del norte.

\par
%\textsuperscript{(892.6)}
\textsuperscript{80:4.3} Las expansiones iniciales de la raza violeta más pura fueron mucho más pacíficas que las de sus descendientes anditas posteriores, que eran semimilitares y amantes de las conquistas. Los adamitas eran pacíficos, y los noditas, belicosos. La unión de estos dos linajes, tal como se mezclaron más adelante con las razas sangiks, dio nacimiento a los hábiles y agresivos anditas que llevaron a cabo auténticas conquistas militares.

\par
%\textsuperscript{(892.7)}
\textsuperscript{80:4.4} El caballo fue el factor evolutivo que determinó el dominio de los anditas en occidente. El caballo proporcionó a los anditas en plena dispersión la ventaja hasta entonces inexistente de la movilidad, permitiendo a los últimos grupos de jinetes anditas avanzar rápidamente alrededor del Mar Caspio para invadir toda Europa. Todas las oleadas anteriores de anditas se habían desplazado tan lentamente que tenían tendencia a disgregarse cuando se alejaban mucho de Mesopotamia. Pero estas oleadas posteriores avanzaron tan rápidamente que llegaron a Europa en grupos coherentes, conservando en cierta medida su cultura superior.

\par
%\textsuperscript{(893.1)}
\textsuperscript{80:4.5} Desde hacía diez mil años, todo el mundo habitado, aparte de China y la región del Éufrates, había hecho progresos culturales muy limitados cuando los duros jinetes anditas hicieron su aparición en el séptimo y sexto milenio antes de Cristo. A medida que se desplazaban hacia el oeste a través de las llanuras rusas, absorbiendo lo mejor de los hombres azules y exterminando lo peor, se mezclaron hasta formar un solo pueblo. Fueron los ascendientes de las llamadas razas nórdicas, los antepasados de los pueblos escandinavos, germánicos y anglosajones.

\par
%\textsuperscript{(893.2)}
\textsuperscript{80:4.6} No pasó mucho tiempo antes de que los linajes azules superiores fueran totalmente absorbidos por los anditas en todo el norte de Europa. Sólo en Laponia (y hasta cierto punto en Bretaña) los antiguos andonitas conservaron una apariencia de identidad racial.

\section*{5. La conquista andita de Europa septentrional}
\par
%\textsuperscript{(893.3)}
\textsuperscript{80:5.1} Las tribus del norte de Europa eran continuamente reforzadas y mejoradas por la oleada constante de mesopotámicos que emigraban a través de las regiones del Turquestán y el sur de Rusia. Cuando las últimas oleadas de la caballería andita se extendieron por Europa, ya había en esta región más hombres con herencia andita que en cualquier otra parte del mundo.

\par
%\textsuperscript{(893.4)}
\textsuperscript{80:5.2} El cuartel general militar de los anditas del norte estuvo situado en Dinamarca durante tres mil años. Las oleadas sucesivas de conquista partieron desde este punto central, pero fueron perdiendo paulatinamente su carácter andita y con el paso de los siglos se volvieron cada vez más blancas a medida que se producía la mezcla final de los conquistadores mesopotámicos con los pueblos conquistados.

\par
%\textsuperscript{(893.5)}
\textsuperscript{80:5.3} Aunque los hombres azules habían sido absorbidos en el norte y habían sucumbido finalmente ante la caballería de los invasores blancos que penetraban en el sur, las tribus invasoras de la raza blanca mezclada se encontraron con la resistencia obstinada y prolongada de los cro-mañones; pero la inteligencia superior de la raza blanca y sus reservas biológicas en constante aumento le permitieron destruir por completo a la raza más antigua.

\par
%\textsuperscript{(893.6)}
\textsuperscript{80:5.4} Las batallas decisivas entre el hombre blanco y el hombre azul se libraron en el valle del Somme. Aquí, la flor y nata de la raza azul luchó encarnizadamente contra los anditas que avanzaban hacia el sur, y estos cro-mañones defendieron con éxito sus territorios durante más de quinientos años antes de sucumbir ante la estrategia militar superior de los invasores blancos. Thor, el jefe victorioso de los ejércitos del norte en la batalla final del Somme, se convirtió en el héroe de las tribus blancas septentrionales, y más tarde fue venerado como un dios por algunas de ellas.

\par
%\textsuperscript{(893.7)}
\textsuperscript{80:5.5} Las plazas fuertes de los hombres azules que resistieron más tiempo se encontraban en el sur de Francia, pero la última gran resistencia militar fue vencida a lo largo del Somme. La conquista posterior se efectuó mediante la penetración comercial, la presión de la población a lo largo de los ríos y los casamientos continuos con los elementos superiores, unido a la exterminación implacable de los inferiores.

\par
%\textsuperscript{(893.8)}
\textsuperscript{80:5.6} Cuando el consejo tribal andita de los ancianos declaraba inepto a un cautivo inferior, lo entregaba a los sacerdotes chamanes durante una ceremonia complicada, y éstos lo escoltaban hasta el río donde le administraban los ritos de iniciación hacia los <<territorios de caza paradisíacos>> ---el ahogamiento. Los invasores blancos de Europa exterminaron de esta manera a todos los pueblos que encontraron y que no fueron rápidamente absorbidos en sus propias filas; así es como los hombres azules llegaron a su fin ---y lo hicieron rápidamente.

\par
%\textsuperscript{(893.9)}
\textsuperscript{80:5.7} El hombre azul de Cro-Magnon constituyó la base biológica de las razas europeas modernas, pero sólo sobrevivió en la medida en que fue absorbido por los enérgicos conquistadores posteriores de sus tierras natales. El linaje azul aportó muchas características robustas y mucho vigor físico a las razas blancas de Europa, pero el humor y la imaginación de los pueblos mezclados europeos procedían de los anditas. Esta unión entre los anditas y los hombres azules, que tuvo como resultado las razas blancas nórdicas, produjo una caída inmediata de la civilización andita, un retraso de naturaleza transitoria. Al final, la superioridad latente de estos bárbaros nórdicos se manifestó y culminó en la civilización europea actual.

\par
%\textsuperscript{(894.1)}
\textsuperscript{80:5.8} Hacia el año 5000 a. de J.C., las razas blancas en evolución dominaban toda Europa septentrional, incluyendo el norte de Alemania, el norte de Francia y las Islas Británicas. Europa central estuvo controlada durante cierto tiempo por el hombre azul y los andonitas de cabeza redonda. Estos últimos estaban situados principalmente en el valle del Danubio y nunca fueron completamente desplazados por los anditas.

\section*{6. Los anditas a lo largo del Nilo}
\par
%\textsuperscript{(894.2)}
\textsuperscript{80:6.1} La cultura declinó en el valle del Éufrates desde la época de las emigraciones anditas finales, y el centro inmediato de la civilización se trasladó al valle del Nilo. Egipto se convirtió en el sucesor de Mesopotamia como centro del grupo más avanzado de la Tierra.

\par
%\textsuperscript{(894.3)}
\textsuperscript{80:6.2} El valle del Nilo empezó a sufrir inundaciones poco antes que los valles de Mesopotamia, pero le fue mucho mejor. Este contratiempo inicial estuvo más que compensado por la oleada continua de inmigrantes anditas, de manera que la cultura de Egipto, aunque provenía en realidad de la región del
Éufrates, parecía hacer grandes progresos. Pero en el año 5000 a. de J.C., durante el período de las inundaciones en Mesopotamia, había siete grupos distintos de seres humanos en Egipto, y todos salvo uno procedían de Mesopotamia.

\par
%\textsuperscript{(894.4)}
\textsuperscript{80:6.3} Cuando se produjo el último éxodo del valle del Éufrates, Egipto tuvo la fortuna de recibir un gran número de los artistas y artesanos más hábiles. Estos artesanos anditas se encontraron como en su casa ya que estaban completamente familiarizados con la vida fluvial, sus inundaciones, el riego y las épocas de sequía. Disfrutaban de la situación protegida del valle del Nilo, donde estaban mucho menos expuestos a los ataques y las incursiones hostiles que en las riberas del Éufrates. Acrecentaron enormemente la habilidad de los egipcios en el trabajo de los metales. Aquí trabajaron los minerales de hierro procedentes del monte Sinaí en lugar de los de las regiones del Mar Negro.

\par
%\textsuperscript{(894.5)}
\textsuperscript{80:6.4} Los egipcios reunieron muy pronto a sus deidades locales en un complicado sistema nacional de dioses. Desarrollaron una extensa teología y tuvieron un clero igualmente extenso pero gravoso. Varios jefes diferentes trataron de resucitar los restos de las primeras enseñanzas religiosas de los setitas, pero estos esfuerzos fueron efímeros. Los anditas construyeron las primeras estructuras de piedra en Egipto. La primera pirámide de piedra, y la más exquisita, fue levantada por Imhotep, un genio arquitectónico andita, mientras ejercía como primer ministro. Los edificios anteriores habían sido construidos de ladrillo, y aunque se habían levantado muchas estructuras de piedra en diferentes partes del mundo, ésta fue la primera en Egipto. Pero el arte de la construcción declinó sin cesar después de los tiempos de este gran arquitecto.

\par
%\textsuperscript{(894.6)}
\textsuperscript{80:6.5} Esta brillante época de cultura se interrumpió bruscamente debido a las guerras internas a lo largo del Nilo, y el país fue pronto invadido, como lo había sido Mesopotamia, por las tribus inferiores de la inhóspita Arabia y por los negros del sur. Como consecuencia de ello, el progreso social declinó constantemente durante más de quinientos años.

\section*{7. Los anditas de las islas del Mediterráneo}
\par
%\textsuperscript{(895.1)}
\textsuperscript{80:7.1} Durante la decadencia de la cultura en Mesopotamia, una civilización superior subsistió durante algún tiempo en las islas del Mediterráneo oriental.

\par
%\textsuperscript{(895.2)}
\textsuperscript{80:7.2} Hacia el año 12.000 a. de J.C., una brillante tribu de anditas emigró a Creta. Ésta fue la única isla colonizada tan pronto por un grupo tan superior, y transcurrieron casi dos mil años antes de que los descendientes de estos navegantes se diseminaran por las islas vecinas. Este grupo estaba compuesto por los anditas de cabeza estrecha y estatura pequeña que se habían casado con la rama vanita de los noditas del norte. Todos medían menos de un metro ochenta de altura y habían sido literalmente expulsados del continente por sus compañeros más altos pero inferiores. Estos emigrantes que fueron a Creta eran muy hábiles en la tejeduría, los metales, la alfarería, la instalación de cañerías y el empleo de la piedra como material de construcción. Utilizaban la escritura y vivían del pastoreo y la agricultura.

\par
%\textsuperscript{(895.3)}
\textsuperscript{80:7.3} Cerca de dos mil años después de la colonización de Creta, un grupo de descendientes de Adanson, de alta estatura, se dirigió por las islas del norte hasta Grecia, viniendo casi directamente desde su hogar en las tierras altas del norte de Mesopotamia. Estos antepasados de los griegos fueron conducidos hacia el oeste por Sato, un descendiente directo de Adanson y Ratta.

\par
%\textsuperscript{(895.4)}
\textsuperscript{80:7.4} El grupo que se estableció finalmente en Grecia estaba compuesto por trescientas setenta y cinco personas escogidas y superiores que formaban parte del resto de la segunda civilización de los adansonitas. Estos hijos más recientes de Adanson poseían los linajes entonces más valiosos de las razas blancas emergentes. Tenían un nivel intelectual superior y eran, desde el punto de vista físico, los hombres más hermosos que habían existido desde la época del primer Edén.

\par
%\textsuperscript{(895.5)}
\textsuperscript{80:7.5} Grecia y las islas del mar Egeo sucedieron enseguida a Mesopotamia y Egipto como centro occidental del comercio, el arte y la cultura. Pero tal como había ocurrido en Egipto, prácticamente todo el arte y la ciencia del mundo egeo procedían una vez más de Mesopotamia, excepto la cultura de los precursores adansonitas de los griegos. Todo el arte y la genialidad de este último pueblo son un legado directo de la posteridad de Adanson, el primer hijo de Adán y Eva, y de su extraordinaria segunda esposa, una hija descendiente en línea ininterrumpida del puro estado mayor nodita del Príncipe Caligastia. No es de extrañar que los griegos tuvieran las tradiciones mitológicas de que descendían directamente de los dioses y de seres superhumanos.

\par
%\textsuperscript{(895.6)}
\textsuperscript{80:7.6} La región egea pasó por cinco etapas culturales diferentes, cada una de ellas menos espiritual que la anterior. Antes de mucho tiempo, la última época de gloria artística pereció bajo el peso de los descendientes mediocres, que se multiplicaban rápidamente, de los esclavos del Danubio que habían sido importados por las generaciones posteriores de griegos.

\par
%\textsuperscript{(895.7)}
\textsuperscript{80:7.7} El \textit{culto a la madre} de los descendientes de Caín alcanzó su apogeo en Creta durante esta época. Este culto glorificaba a Eva en la adoración de la <<gran madre>>\footnote{\textit{Culto a Eva}: Gn 3:20.}. Había imágenes de Eva por todas partes. Se erigieron miles de santuarios públicos por toda Creta y Asia Menor. Este culto a la madre perduró hasta los tiempos de Cristo, y más tarde fue incorporado en la religión cristiana primitiva bajo la forma de la glorificación y la adoración de María, la madre terrestre de Jesús.

\par
%\textsuperscript{(895.8)}
\textsuperscript{80:7.8} Hacia el año 6500 a. de J.C. se había producido una gran decadencia en la herencia espiritual de los anditas. Los descendientes de Adán estaban extremadamente dispersos y habían sido prácticamente absorbidos por las razas humanas más antiguas y numerosas. Esta decadencia de la civilización andita, unida a la desaparición de sus normas religiosas, dejó a las razas espiritualmente empobrecidas del mundo en un estado deplorable.

\par
%\textsuperscript{(896.1)}
\textsuperscript{80:7.9} Hacia el año 5000 a. de J.C., los tres linajes más puros de los descendientes de Adán se encontraban en Sumeria, el norte de Europa y Grecia. Toda Mesopotamia se deterioraba lentamente debido al torrente de razas mezcladas y más oscuras que se infiltraba desde Arabia. La llegada de estos pueblos inferiores contribuyó aún más a la dispersión del residuo biológico y cultural de los anditas. Los pueblos más aventureros salieron en masa de todo el fértil creciente hacia las islas del oeste. Estos emigrantes cultivaban los cereales y las legumbres, y trajeron consigo a sus animales domésticos.

\par
%\textsuperscript{(896.2)}
\textsuperscript{80:7.10} Hacia el año 5000 a. de J.C., una inmensa multitud de mesopotámicos progresivos salió del valle del Éufrates y se instaló en la isla de Chipre. Esta civilización fue aniquilada unos dos mil años después por las hordas bárbaras del norte.

\par
%\textsuperscript{(896.3)}
\textsuperscript{80:7.11} Otra gran colonia se estableció en el Mediterráneo cerca del emplazamiento posterior de Cartago. Partiendo del norte de África, un gran número de anditas entró en España y más tarde se mezcló en Suiza con sus hermanos que habían salido anteriormente de las islas egeas para instalarse en Italia.

\par
%\textsuperscript{(896.4)}
\textsuperscript{80:7.12} Cuando Egipto siguió a Mesopotamia en su decadencia cultural, muchas familias de las más capaces y avanzadas se refugiaron en Creta, aumentando así considerablemente esta civilización ya avanzada. Cuando la llegada de los grupos inferiores procedentes de Egipto amenazó posteriormente la civilización de Creta, las familias más cultas partieron hacia Grecia en el oeste.

\par
%\textsuperscript{(896.5)}
\textsuperscript{80:7.13} Los griegos no fueron solamente unos grandes educadores y artistas, sino que fueron también los comerciantes y colonizadores más grandes del mundo. Antes de sucumbir ante la avalancha de inferioridad que sepultó finalmente su arte y su comercio, lograron establecer en el oeste tantos puestos avanzados de cultura, que una gran parte de los progresos de la civilización griega primitiva sobrevivió en los pueblos posteriores del sur de Europa, y muchos descendientes mixtos de estos adansonitas fueron incorporados en las tribus de las tierras continentales adyacentes.

\section*{8. Los andonitas del Danubio}
\par
%\textsuperscript{(896.6)}
\textsuperscript{80:8.1} Los pueblos anditas del valle del Éufrates emigraron hacia el norte hasta Europa para mezclarse con los hombres azules, y hacia el oeste hasta las regiones mediterráneas para unirse con los restos de los saharianos mezclados y los hombres azules del sur. Estas dos ramas de la raza blanca estaban, y continúan estando, ampliamente separadas por los supervivientes montañeses de cabeza ancha de las primeras tribus andonitas que habían vivido durante mucho tiempo en estas regiones centrales.

\par
%\textsuperscript{(896.7)}
\textsuperscript{80:8.2} Estos descendientes de Andón estaban dispersos por la mayoría de las regiones montañosas del centro y sudeste de Europa. Fueron reforzados a menudo por aquellos que llegaban de Asia Menor, una región que ocupaban en gran número. Los antiguos hititas provenían directamente de la estirpe andonita; su piel pálida y su cabeza ancha eran típicas de esta raza. Los antepasados de Abraham contenían este linaje, el cual contribuyó mucho al aspecto facial característico de sus descendientes judíos posteriores; éstos tenían una cultura y una religión derivadas de los anditas, pero hablaban una lengua muy diferente. Su idioma era claramente andonita.

\par
%\textsuperscript{(897.1)}
\textsuperscript{80:8.3} Las tribus que vivían en casas construidas sobre pilotes o pilares de troncos en los lagos de Italia, Suiza y Europa meridional pertenecían a la periferia en expansión de las emigraciones africanas, egeas y sobre todo danubianas.

\par
%\textsuperscript{(897.2)}
\textsuperscript{80:8.4} Los danubianos eran andonitas, eran los agricultores y pastores que habían entrado en Europa por la península balcánica y que se habían desplazado lentamente hacia el norte por la ruta del valle del Danubio. Eran alfareros, cultivaban la tierra y preferían vivir en los valles. La colonia más septentrional de los danubianos se encontraba en Lieja, en Bélgica. Estas tribus degeneraron rápidamente a medida que se alejaron del centro y fuente de su cultura. La mejor cerámica que fabricaron es el producto de las colonias más primitivas.

\par
%\textsuperscript{(897.3)}
\textsuperscript{80:8.5} Los danubianos se convirtieron en adoradores de la madre a consecuencia de la labor de los misioneros de Creta. Estas tribus se fusionaron más tarde con grupos de marineros andonitas que vinieron por barco desde la costa de Asia Menor, y que también eran adoradores de la madre. Una gran parte de Europa central fue así colonizada inicialmente por estos tipos mixtos de razas blancas de cabeza ancha que practicaban el culto a la madre y el rito religioso de incinerar a los muertos, ya que los practicantes del culto a la madre tenían la costumbre de quemar a sus muertos en cabañas de piedra.

\section*{9. Las tres razas blancas}
\par
%\textsuperscript{(897.4)}
\textsuperscript{80:9.1} Hacia el final de las emigraciones anditas, las mezclas raciales en Europa se habían generalizado en las tres razas blancas siguientes:

\par
%\textsuperscript{(897.5)}
\textsuperscript{80:9.2} 1. \textit{La raza blanca del norte}. Esta raza llamada nórdica estaba compuesta principalmente por los hombres azules más los anditas, pero también contenía una cantidad considerable de sangre andonita, así como cantidades más pequeñas de sangre sangik roja y amarilla. La raza blanca del norte englobaba así los cuatro linajes humanos más deseables, pero su herencia más importante provenía del hombre azul. El nórdico típico primitivo tenía la cabeza alargada, era alto y rubio. Pero hace mucho tiempo que esta raza se mezcló por completo con todas las ramas de los pueblos blancos.

\par
%\textsuperscript{(897.6)}
\textsuperscript{80:9.3} La cultura primitiva que los invasores nórdicos encontraron en Europa era la de los danubianos en retroceso, mezclados con el hombre azul. La cultura nórdico-danesa y la cultura danubiano-andonita se encontraron y se mezclaron en el Rin, tal como lo atestigua la existencia de dos grupos raciales en la Alemania de hoy.

\par
%\textsuperscript{(897.7)}
\textsuperscript{80:9.4} Los nórdicos continuaron con el comercio del ámbar desde la costa báltica, estableciendo un gran intercambio, a través del Paso del Brenner, con los habitantes de cabeza ancha del valle del Danubio. Este amplio contacto con los danubianos condujo a estos habitantes del norte al culto a la madre, y la incineración de los muertos fue casi universal en toda Escandinavia durante varios miles de años. Esto explica por qué no se pueden encontrar los restos de las razas blancas primitivas, aunque están enterrados por toda Europa ---sólo se encuentran sus cenizas en urnas de piedra o de arcilla. Estos hombres blancos también construían viviendas; nunca vivieron en cavernas. Y esto explica también por qué hay tan pocas pruebas de la cultura primitiva del hombre blanco, a pesar de que el tipo Cro-Magnon que lo precedió se encuentra bien conservado allí donde sus restos quedaron bien protegidos en cavernas y grutas. Tal como fueron las cosas, un día encontramos en el norte de Europa una cultura primitiva de danubianos en retroceso y de hombres azules, y al día siguiente hallamos la de unos hombres blancos que aparecen repentinamente y son inmensamente superiores.

\par
%\textsuperscript{(897.8)}
\textsuperscript{80:9.5} 2. \textit{La raza blanca central}. Aunque este grupo contiene linajes azules, amarillos y anditas, es predominantemente andonita. Estos pueblos son de cabeza ancha, morenos y rechonchos. Están introducidos como una cuña entre la raza nórdica y las razas mediterráneas, con su extensa base apoyada en Asia y el vértice penetrando en el este de Francia.

\par
%\textsuperscript{(898.1)}
\textsuperscript{80:9.6} Durante cerca de veinte mil años, los anditas habían empujado a los andonitas cada vez más lejos hacia el norte de Asia central. Hacia el año 3000 a. de J.C., la aridez creciente hizo retroceder a estos andonitas hacia el Turquestán. Este empuje andonita hacia el sur continuó durante más de mil años, se dividió alrededor del Mar Caspio y del Mar Negro, y penetró en Europa tanto por los Balcanes como por Ucrania. Esta invasión incluía a los grupos restantes de descendientes de Adanson, y durante la segunda mitad del período de invasión, trajo con ella a un gran número de anditas iraníes así como a muchos descendientes de los sacerdotes setitas.

\par
%\textsuperscript{(898.2)}
\textsuperscript{80:9.7} Hacia el año 2500 a. de J.C., el empuje que efectuaban los andonitas hacia el oeste llegó hasta Europa. Esta invasión de toda Mesopotamia, Asia Menor y la cuenca del Danubio por parte de los bárbaros de las colinas del Turquestán constituyó la regresión cultural más grave y duradera de todas las sucedidas hasta entonces. Estos invasores andonizaron claramente el carácter de las razas centroeuropeas, que desde entonces han continuado siendo característicamente alpinas.

\par
%\textsuperscript{(898.3)}
\textsuperscript{80:9.8} 3. \textit{La raza blanca del sur}. Esta raza morena mediterránea estaba compuesta por una mezcla de anditas y de hombres azules, con un linaje andonita menos importante que en el norte. Este grupo absorbió también, a través de los saharianos, una cantidad considerable de sangre sangik secundaria. En tiempos posteriores, unos poderosos elementos anditas procedentes del Mediterráneo oriental se fusionaron con esta rama meridional de la raza blanca.

\par
%\textsuperscript{(898.4)}
\textsuperscript{80:9.9} Sin embargo, las regiones costeras del Mediterráneo no se poblaron de anditas hasta la época de las grandes invasiones nómadas del año 2500 a. de J.C.. El transporte y el comercio terrestre permanecieron prácticamente interrumpidos durante estos siglos en que los nómadas invadieron las regiones orientales del Mediterráneo. Esta obstrucción de los viajes por tierra provocó la gran expansión del transporte y el comercio por mar; el comercio marítimo por el Mediterráneo estaba en pleno apogeo hace aproximadamente cuatro mil quinientos años. Este desarrollo del tráfico marítimo condujo a la expansión repentina de los descendientes de los anditas por todo el territorio costero de la cuenca mediterránea.

\par
%\textsuperscript{(898.5)}
\textsuperscript{80:9.10} Estas mezclas raciales establecieron los fundamentos de la raza europea del sur, la más mezclada de todas. Desde aquella época, esta raza ha sufrido además otras mezclas, principalmente con los pueblos azules-amarillos-anditas de Arabia. Esta raza mediterránea está de hecho tan mezclada con los pueblos circundantes que es prácticamente indiscernible como tipo aparte, pero sus miembros son en general bajos, de cabeza alargada y morenos.

\par
%\textsuperscript{(898.6)}
\textsuperscript{80:9.11} En el norte, los anditas eliminaron a los hombres azules por medio de la guerra y los matrimonios, pero los hombres azules sobrevivieron en gran número en el sur. Los vascos y los bereberes representan la supervivencia de dos ramas de esta raza, pero incluso estos pueblos se han mezclado por completo con los saharianos.

\par
%\textsuperscript{(898.7)}
\textsuperscript{80:9.12} Ésta es la imagen que ofrecía la mezcla de razas en Europa central hacia el año 3000 a. de J.C. A pesar de la falta parcial de Adán, los tipos superiores se habían mezclado.

\par
%\textsuperscript{(898.8)}
\textsuperscript{80:9.13} Eran los tiempos del Neolítico, que coincidían en parte con la Edad del Bronce que se aproximaba. En Escandinavia se estaba viviendo la Edad del Bronce asociada con el culto a la madre. El sur de Francia y España se hallaban en el Neolítico asociado con el culto al Sol. Fue la época en que se construyeron los templos circulares y sin techo dedicados al Sol. Los miembros de las razas blancas europeas eran unos constructores activos, y les encantaba colocar grandes piedras como símbolos del Sol, tal como lo hicieron sus descendientes posteriores en Stonehenge. La moda de la adoración del Sol indica que éste fue un gran período de agricultura en Europa del sur.

\par
%\textsuperscript{(899.1)}
\textsuperscript{80:9.14} Las supersticiones de esta era relativamente reciente de adoración del Sol continúan existiendo hoy en día en las costumbres de Bretaña. Aunque fueron cristianizados hace más de mil quinientos años, los bretones conservan todavía los amuletos del Neolítico para evitar el mal de ojo. Siguen guardando las piedras del trueno en sus chimeneas para protegerse contra el rayo. Los bretones nunca se mezclaron con los nórdicos de Escandinavia. Son los supervivientes de los habitantes andonitas originales de Europa occidental, mezclados con el linaje mediterráneo.

\par
%\textsuperscript{(899.2)}
\textsuperscript{80:9.15} Es un error pretender clasificar a los pueblos blancos en nórdicos, alpinos y mediterráneos. Ha habido, en conjunto, demasiadas mezclas como para permitir este agrupamiento. En cierto momento la raza blanca estaba dividida de manera bastante bien definida en estas clases, pero se han producido desde entonces unas mezclas muy extensas, y ya no es posible identificar estas distinciones con claridad. Incluso en el año 3000 a. de J.C., los antiguos grupos sociales ya no formaban parte de una sola raza, al igual que sucede con los habitantes actuales de América del Norte.

\par
%\textsuperscript{(899.3)}
\textsuperscript{80:9.16} Esta cultura europea continuó creciendo, y hasta cierto punto entremezclándose, durante cinco mil años. Pero la barrera del idioma impidió la plena reciprocidad entre las diversas naciones occidentales. Durante el siglo pasado, esta cultura experimentó la mejor oportunidad que tenía para mezclarse en la población cosmopolita de América del Norte; y el futuro de este continente estará determinado por la calidad de los factores raciales que se permita que entren en su población presente y futura, así como por el nivel de cultura social que se mantenga.

\par
%\textsuperscript{(899.4)}
\textsuperscript{80:9.17} [Presentado por un Arcángel de Nebadon.]


\chapter{Documento 93. Maquiventa Melquisedek}
\par
%\textsuperscript{(1014.1)}
\textsuperscript{93:0.1} LOS Melquisedeks son muy conocidos como Hijos de emergencia, porque se dedican a una asombrosa gama de actividades en los mundos de un universo local. Cuando surge algún problema extraordinario o cuando hay que intentar algo fuera de lo normal, es un Melquisedek el que acepta muy a menudo la misión. La capacidad de los Hijos Melquisedeks para actuar en los casos de urgencia y en niveles muy divergentes del universo, incluso en el nivel físico de manifestación de la personalidad, es típica de esta orden. Sólo los Portadores de Vida comparten hasta cierto punto esta gama metamórfica de actividades de la personalidad.

\par
%\textsuperscript{(1014.2)}
\textsuperscript{93:0.2} La orden Melquisedek de filiación del universo ha sido extremadamente activa en Urantia. Un cuerpo de doce miembros sirvió conjuntamente con los Portadores de Vida. Otro cuerpo posterior de doce se convirtió en los síndicos de vuestro mundo poco después de la secesión de Caligastia, y continuó al mando hasta la época de Adán y Eva. Estos doce Melquisedeks volvieron a Urantia después de la falta de Adán y Eva, y luego continuaron como síndicos planetarios hasta el día en que Jesús de Nazaret se convirtió, como Hijo del Hombre, en el Príncipe Planetario titular de Urantia.

\section*{1. La encarnación de Maquiventa}
\par
%\textsuperscript{(1014.3)}
\textsuperscript{93:1.1} La verdad revelada estuvo amenazada de desaparición durante los milenios que siguieron al fracaso de la misión adámica en Urantia. Aunque las razas humanas hacían progresos intelectuales, perdían lentamente terreno en el campo espiritual. Hacia el año 3000 a. de J. C., el concepto de Dios se había vuelto muy vago en la mente de los hombres.

\par
%\textsuperscript{(1014.4)}
\textsuperscript{93:1.2} Los doce síndicos Melquisedeks conocían la donación inminente de Miguel en el planeta, pero no sabían cuándo se produciría; por consiguiente, se reunieron en consejo solemne y pidieron a los Altísimos de Edentia que se tomara alguna disposición para mantener la luz de la verdad en Urantia. Esta petición fue desestimada con el mandato de que <<la conducta de los asuntos en la 606 de Satania está plenamente entre las manos de los custodios Melquisedeks>>. Los síndicos recurrieron entonces a la ayuda del Padre Melquisedek, pero sólo recibieron el mensaje de que debían continuar sosteniendo la verdad de la manera que ellos mismos escogieran <<hasta la llegada de un Hijo donador>> que <<salvaría los títulos planetarios de la pérdida y la incertidumbre>>.

\par
%\textsuperscript{(1014.5)}
\textsuperscript{93:1.3} A consecuencia de tener que valerse tan completamente por sí mismos, Maquiventa Melquisedek, uno de los doce síndicos planetarios, se ofreció como voluntario para hacer lo que sólo se había efectuado seis veces en toda la historia de Nebadon: personalizarse en la Tierra como un hombre temporal del planeta, donarse como Hijo de emergencia para ayudar al mundo. Las autoridades de Salvington concedieron el permiso para esta aventura, y la encarnación efectiva de Maquiventa Melquisedek se consumó cerca del lugar que llegaría a convertirse en la ciudad de Salem, en Palestina. Toda la operación de la materialización de este Hijo Melquisedek fue completada por los síndicos planetarios con la cooperación de los Portadores de Vida, de algunos Controladores Físicos Maestros y de otras personalidades celestiales residentes en Urantia.

\section*{2. El sabio de Salem}
\par
%\textsuperscript{(1015.1)}
\textsuperscript{93:2.1} Maquiventa se donó a las razas humanas de Urantia 1.973 años antes del nacimiento de Jesús. Su llegada no fue espectacular; su materialización no fue contemplada por los ojos humanos. La primera vez que un hombre mortal lo observó fue el día memorable en que entró en la tienda de Amdón, un pastor caldeo de origen sumerio. Y la proclamación de su misión estuvo sintetizada en la simple declaración que le hizo a este pastor: <<Soy Melquisedek, sacerdote de El Elyón, el Altísimo, el solo y único Dios>>\footnote{\textit{Melquisedek}: Heb 7:3. \textit{Melquisedek, sacerdote del Altísimo}: Gn 14:18-20.}.

\par
%\textsuperscript{(1015.2)}
\textsuperscript{93:2.2} Cuando el pastor se hubo recobrado de su sorpresa, y después de acosar a este desconocido con muchas preguntas, le pidió a Melquisedek que cenara con él. Ésta fue la primera vez, en su larga carrera universal, que Maquiventa consumió comida material, el alimento que habría de sustentarlo durante los noventa y cuatro años de su vida como ser material.

\par
%\textsuperscript{(1015.3)}
\textsuperscript{93:2.3} Aquella noche, mientras conversaban fuera bajo las estrellas, Melquisedek empezó su misión de revelar la verdad de la realidad de Dios cuando, con un amplio movimiento de su brazo, se volvió hacia Amdón y le dijo: <<El Elyón, el Altísimo, es el divino creador de las estrellas del firmamento e incluso de esta misma Tierra donde vivimos, y es también el Dios supremo del cielo>>\footnote{\textit{El Altísimo}: Sal 7:17; 9:2; 46:4; 78:17,35,56; 82:6; 91:1,9; 92:1,8; Is 14:14; Lm 3:35,38; Nm 24:16; Dn 3:26; 4:2,17,24-25,32; 4:32; 5:18,21; 7:18,22,25,27; Os 7:16; 11:7; Dt 32:8; Mc 5:7; Lc 8:28; Hch 7:48; 16:17; Heb 7:1; Man 1:7; 2 Sam 22:14. \textit{El Altísimo (El Elyón)}: Gn 14:18-20,22.}.

\par
%\textsuperscript{(1015.4)}
\textsuperscript{93:2.4} En pocos años, Melquisedek había reunido a su alrededor a un grupo de alumnos, discípulos y creyentes que formaron el núcleo de la comunidad posterior de Salem. Pronto fue conocido en toda Palestina como el sacerdote de El Elyón, el Altísimo, y como el sabio de Salem. En algunas tribus circundantes, a menudo se referían a él como el jeque, o el rey, de Salem\footnote{\textit{Otros nombres de Melquisedek}: Gn 14:18; Sal 110:4; Heb 5:6; 7:1-3.}. Salem era el lugar que, después de la desaparición de Melquisedek, se convirtió en la ciudad de Jebús, y más tarde fue llamada Jerusalén.

\par
%\textsuperscript{(1015.5)}
\textsuperscript{93:2.5} Melquisedek se parecía, en su apariencia personal, a los pueblos noditas y sumerios entonces mezclados; medía casi un metro ochenta de alto y tenía una presencia imponente. Hablaba el caldeo y media docena de otras lenguas. Se vestía poco más o menos como los sacerdotes cananeos, salvo que llevaba en su pecho un emblema de tres círculos concéntricos, el símbolo de la Trinidad del Paraíso vigente en Satania. En el transcurso de su ministerio, sus seguidores llegaron a considerar tan sagrada esta insignia de los tres círculos concéntricos, que nunca se atrevieron a utilizarla, y con el paso de algunas generaciones fue pronto olvidada.

\par
%\textsuperscript{(1015.6)}
\textsuperscript{93:2.6} Aunque Maquiventa vivió a la manera de los hombres del planeta, nunca se casó, ni podría haber dejado descendencia en la Tierra. Su cuerpo físico se parecía al de un varón humano, pero pertenecía en realidad al tipo de cuerpos especialmente construídos que habían utilizado los cien miembros materializados del estado mayor del Príncipe Caligastia, salvo que no contenía el plasma vital de ninguna raza humana\footnote{\textit{El cuerpo de Melquisedek}: Heb 7:3.}. El árbol de la vida tampoco estaba disponible en Urantia. Si Maquiventa hubiera permanecido un largo período de tiempo en la Tierra, su mecanismo físico se habría deteriorado paulatinamente; tal como sucedieron las cosas, terminó su misión de donación en noventa y cuatro años, mucho antes de que su cuerpo material empezara a desintegrarse.

\par
%\textsuperscript{(1016.1)}
\textsuperscript{93:2.7} Este Melquisedek encarnado recibió un Ajustador del Pensamiento que residió en su personalidad superhumana como monitor del tiempo y mentor de la carne, consiguiendo así aquella experiencia e introducción práctica a los problemas de Urantia y a la técnica de residir en un Hijo encarnado que permitió a este espíritu del Padre ejercer su actividad tan valientemente en la mente humana de Miguel, el Hijo de Dios que apareció más tarde en la Tierra en la similitud de la carne mortal. Éste es el único Ajustador del Pensamiento que ha trabajado en dos mentes en Urantia, pero las dos mentes eran divinas a la vez que humanas.

\par
%\textsuperscript{(1016.2)}
\textsuperscript{93:2.8} Maquiventa permaneció durante su encarnación en completo contacto con sus once compañeros del cuerpo de guardianes planetarios, pero no podía comunicarse con otras órdenes de personalidades celestiales. Aparte de los síndicos Melquisedeks, no tenía más contacto con las inteligencias superhumanas que un ser humano.

\section*{3. Las enseñanzas de Melquisedek}
\par
%\textsuperscript{(1016.3)}
\textsuperscript{93:3.1} Después de pasar una década, Melquisedek organizó sus escuelas en Salem según el modelo del antiguo sistema que había sido desarrollado por los primeros sacerdotes setitas del segundo Edén. Incluso la idea de un sistema de diezmo\footnote{\textit{Sistema del diezmo}: Gn 14:20; 28:22; 2 Cr 31:5-6,12; Lv 27:30-32; Mal 3:8-10.}, que fue introducido por Abraham, su converso posterior, también provenía de las tradiciones supervivientes de los métodos de los antiguos setitas.

\par
%\textsuperscript{(1016.4)}
\textsuperscript{93:3.2} Melquisedek enseñó el concepto de un solo Dios, de una Deidad universal, pero permitió que la gente asociara esta enseñanza con el Padre de la Constelación de Norlatiadek, a quien llamaba El Elyón\footnote{\textit{El Elyón, el Altísimo}: Gn 14:18-20; Heb 7:1.} ---el Altísimo\footnote{\textit{El Altísimo}: Gn 14:18-20,22; Sal 7:17; 9:2; 46:4; 78:17,35,56; 82:6; 91:1,9; 92:1,8; Is 14:14; Lm 3:35,38; Nm 24:16; Dn 3:26; 4:2,17,24-25,32; 4:32; 5:18,21; 7:18,22,25,27; Os 7:16; 11:7; Dt 32:8; Mc 5:7; Lc 8:28; Hch 7:48; 16:17; Heb 7:1; Man 1:7; 2 Sam 22:14.}. Melquisedek casi no dijo nada sobre la situación de Lucifer y el estado de los asuntos de Jerusem. Lanaforge, el Soberano del Sistema, tuvo que ocuparse poco de Urantia hasta después de que Miguel terminara su donación. Para la mayoría de los estudiantes de Salem, Edentia era el cielo y el Altísimo, Dios.

\par
%\textsuperscript{(1016.5)}
\textsuperscript{93:3.3} El símbolo de los tres círculos concéntricos, que Melquisedek adoptó como insignia de su donación, fue interpretado por la mayoría de la gente como que representaba tres reinos, el reino de los hombres, de los ángeles y de Dios. Se les permitió que continuaran con esta creencia; muy pocos de sus seguidores supieron nunca que estos tres círculos eran el símbolo de la infinidad, la eternidad y la universalidad de la Trinidad del Paraíso que lo mantiene y lo dirige todo de manera divina; incluso Abraham consideraba que este símbolo representaba más bien a los tres Altísimos de Edentia, pues se le había enseñado que los tres Altísimos actuaban como uno solo. Melquisedek enseñó el concepto de la Trinidad, simbolizado en su insignia, hasta el punto de que lo asociaba generalmente con los tres gobernantes Vorondadeks de la constelación de Norlatiadek.

\par
%\textsuperscript{(1016.6)}
\textsuperscript{93:3.4} Para la masa de sus seguidores, no hizo ningún esfuerzo por presentarles unas enseñanzas que sobrepasaran la realidad del gobierno de los Altísimos de Edentia\footnote{\textit{Los Altísimos reinan en los reinos de los hombres}: Dn 4:17,25,32; 5:21.} ---los Dioses de Urantia. Pero Melquisedek enseñó a algunos una verdad superior que abarcaba la conducta y la organización del universo local, mientras que a su brillante discípulo Nordán el Kenita y a su grupo de estudiantes aplicados les enseñó las verdades del superuniverso e incluso de Havona.

\par
%\textsuperscript{(1016.7)}
\textsuperscript{93:3.5} Los miembros de la familia de Katro, con quien Melquisedek vivió más de treinta años, conocían muchas de estas verdades superiores y las perpetuaron durante mucho tiempo en su familia, incluso hasta la época de su ilustre descendiente Moisés; éste contó así con una convincente tradición de los tiempos de Melquisedek que le había sido transmitida por esta rama, la de su padre, así como por otras fuentes pertenecientes al linaje de su madre.

\par
%\textsuperscript{(1016.8)}
\textsuperscript{93:3.6} Melquisedek enseñó a sus seguidores todo lo que fueron capaces de recibir y asimilar. Incluso muchas ideas religiosas modernas sobre el cielo y la Tierra, el hombre, Dios y los ángeles no están muy alejadas de estas enseñanzas de Melquisedek. Pero este gran maestro lo subordinó todo a la doctrina de un solo Dios, una Deidad universal, un Creador celestial, un Padre divino. Hizo hincapié en esta enseñanza con el fin de atraer la adoración del hombre y de preparar el camino para la aparición posterior de Miguel como Hijo de este mismo Padre Universal.

\par
%\textsuperscript{(1017.1)}
\textsuperscript{93:3.7} Melquisedek enseñó que en algún momento del futuro otro Hijo de Dios vendría a encarnarse como él, pero que nacería de una mujer; por esta razón numerosos educadores posteriores sostuvieron que Jesús era un sacerdote, o un ministro, <<para siempre a la manera de Melquisedek>>\footnote{\textit{Sacerdote para siempre, a la manera de Melquisedek}: Sal 110:4; Heb 5:6,10; 6:20; 7:11,17,21.}.

\par
%\textsuperscript{(1017.2)}
\textsuperscript{93:3.8} Melquisedek preparó así el camino y organizó el terreno monoteísta de la tendencia del mundo para la donación de un verdadero Hijo Paradisiaco del Dios único que él describía tan gráficamente como el Padre de todos, y que presentó a Abraham como un Dios que acepta al hombre con la simple condición de la fe personal. Y cuando Miguel apareció en la Tierra, confirmó todo lo que Melquisedek había enseñado sobre el Padre Paradisiaco.

\section*{4. La religión de Salem}
\par
%\textsuperscript{(1017.3)}
\textsuperscript{93:4.1} Las ceremonias del culto de Salem eran muy sencillas. Toda persona que firmaba o ponía una marca en las listas de las tablillas de arcilla de la iglesia de Melquisedek aprendía de memoria, y suscribía, la siguiente creencia:

\par
%\textsuperscript{(1017.4)}
\textsuperscript{93:4.2} 1. Creo en El Elyón, el Dios Altísimo, el único Padre Universal y Creador de todas las cosas.

\par
%\textsuperscript{(1017.5)}
\textsuperscript{93:4.3} 2. Acepto la alianza de Melquisedek con el Altísimo, la cual me otorga el favor de Dios por mi fe, y no por los sacrificios ni los holocaustos.

\par
%\textsuperscript{(1017.6)}
\textsuperscript{93:4.4} 3. Prometo obedecer los siete mandamientos de Melquisedek y divulgar a todos los hombres la buena nueva de esta alianza con el Altísimo.

\par
%\textsuperscript{(1017.7)}
\textsuperscript{93:4.5} Éste era todo el credo de la colonia de Salem. Pero incluso una declaración de fe tan simple y tan corta era totalmente excesiva y demasiado avanzada para los hombres de aquella época. Simplemente no podían captar la idea de conseguir el favor divino a cambio de nada ---sólo por la fe. Tenían demasiado arraigada la creencia de que el hombre había nacido con los derechos perdidos ante los dioses. Habían ofrecido sacrificios y habían hecho regalos a los sacerdotes durante demasiado tiempo y con demasiada seriedad como para ser capaces de comprender la buena nueva de que la salvación, el favor divino, era un regalo gratuito para todos los que quisieran creer en la alianza de Melquisedek. Pero Abraham creyó aunque con poco entusiasmo, e incluso esto le fue <<contado en justicia>>\footnote{\textit{Contado en justicia}: Gn 15:6; Sal 106:31; Ro 4:3,5; Gl 3:6; Stg 2:23.}.

\par
%\textsuperscript{(1017.8)}
\textsuperscript{93:4.6} Los siete mandamientos promulgados por Melquisedek estaban modelados según las ideas de la antigua ley suprema de Dalamatia, y se parecían mucho a los siete mandamientos que habían sido enseñados en el primero y segundo Edén. Estos mandamientos de la religión de Salem eran los siguientes:

\par
%\textsuperscript{(1017.9)}
\textsuperscript{93:4.7} 1. No servirás a ningún Dios salvo al Creador Altísimo del cielo y de la Tierra.

\par
%\textsuperscript{(1017.10)}
\textsuperscript{93:4.8} 2. No dudarás de que la fe es el único requisito para la salvación eterna.

\par
%\textsuperscript{(1017.11)}
\textsuperscript{93:4.9} 3. No levantarás falsos testimonios.

\par
%\textsuperscript{(1017.12)}
\textsuperscript{93:4.10} 4. No matarás.

\par
%\textsuperscript{(1017.13)}
\textsuperscript{93:4.11} 5. No robarás.

\par
%\textsuperscript{(1018.1)}
\textsuperscript{93:4.12} 6. No cometerás adulterio.

\par
%\textsuperscript{(1018.2)}
\textsuperscript{93:4.13} 7. No mostrarás falta de respeto por tus padres y tus mayores.

\par
%\textsuperscript{(1018.3)}
\textsuperscript{93:4.14} Aunque no se permitía ningún sacrificio dentro de la colonia, Melquisedek sabía muy bien lo difícil que es eliminar repentinamente unas costumbres establecidas durante mucho tiempo y, en consecuencia, ofreció sabiamente a este pueblo sustituir el antiguo sacrificio de carne y sangre por un sacramento de pan y vino\footnote{\textit{La creencia en los sacrificios}: Gn 15:9-10; 22:2-13; 31:54.}. Está escrito que <<Melquisedek, rey de Salem, trajo pan y vino>>\footnote{\textit{Melquisedek, rey de Salem}: Gn 14:18ff.}. Pero incluso esta prudente innovación no tuvo un éxito completo; todas las diversas tribus mantenían unos centros auxiliares en las afueras de Salem donde ofrecían sacrificios y holocaustos. El mismo Abraham recurrió a esta práctica bárbara después de su victoria sobre Kedorlaomer; sencillamente no se sentía tranquilo del todo hasta haber ofrecido un sacrificio convencional. Melquisedek nunca consiguió erradicar plenamente esta tendencia a los sacrificios de las prácticas religiosas de sus seguidores, ni siquiera de Abraham.

\par
%\textsuperscript{(1018.4)}
\textsuperscript{93:4.15} Al igual que Jesús, Melquisedek se ocupó estrictamente de cumplir la misión de su donación. No intentó reformar las costumbres, cambiar los hábitos del mundo, ni promulgar siquiera unas prácticas higiénicas avanzadas o unas verdades científicas. Vino para realizar dos tareas: Mantener viva en la Tierra la verdad del Dios único, y preparar el camino para la donación humana posterior de un Hijo Paradisiaco de ese Padre Universal.

\par
%\textsuperscript{(1018.5)}
\textsuperscript{93:4.16} Melquisedek enseñó en Salem una verdad revelada elemental a lo largo de noventa y cuatro años, y durante este tiempo Abraham asistió a la escuela de Salem en tres ocasiones diferentes. Finalmente se convirtió a las enseñanzas de Salem, volviéndose uno de los alumnos más brillantes y uno de los partidarios principales de Melquisedek.

\section*{5. La elección de Abraham}
\par
%\textsuperscript{(1018.6)}
\textsuperscript{93:5.1} Aunque pueda ser un error hablar de <<pueblo elegido>>\footnote{\textit{Pueblo elegido}: 1 Re 3:8; 1 Cr 17:21-22; Sal 33:12; 105:6,43; 135:4; Is 41:8-9; 43:20-21; 44:1; Dt 7:6; 14:2.}, no es una equivocación referirse a Abraham como un individuo elegido. Melquisedek confió a Abraham la responsabilidad de mantener viva la verdad de un Dios único, distinguiéndolo de la creencia predominante en unas deidades múltiples.

\par
%\textsuperscript{(1018.7)}
\textsuperscript{93:5.2} La elección de Palestina como sede de las actividades de Maquiventa estuvo basada en parte en el deseo de establecer contacto con una familia humana que llevara incorporados los potenciales de mando. En la época de la encarnación de Melquisedek, muchas familias de la Tierra estaban tan bien preparadas como la de Abraham para recibir la doctrina de Salem. Había familias igualmente dotadas entre los hombres rojos, los hombres amarillos y los descendientes de los anditas del oeste y del norte. Pero, una vez más, ninguno de estos lugares estaba tan favorablemente situado como la costa oriental del Mar Mediterráneo para la aparición posterior de Miguel en la Tierra. La misión de Melquisedek en Palestina y la aparición ulterior de Miguel en el pueblo hebreo estuvieron determinadas en gran parte por la geografía, por el hecho de que Palestina ocupaba un emplazamiento central con relación al comercio, los viajes y la civilización existentes en el mundo de entonces.

\par
%\textsuperscript{(1018.8)}
\textsuperscript{93:5.3} Los síndicos Melquisedeks habían estado observando durante algún tiempo a los antepasados de Abraham, y estaban convencidos de que en alguna generación nacería un descendiente que estaría caracterizado por la inteligencia, la iniciativa, la sagacidad y la sinceridad. Los hijos de Téraj, el padre de Abraham, respondían en todos los aspectos a estas expectativas. La posibilidad de ponerse en contacto con estos hijos polifacéticos de Téraj fue la que tuvo tanto que ver con la aparición de Maquiventa en Salem y no en Egipto, China, la India o en las tribus del norte.

\par
%\textsuperscript{(1019.1)}
\textsuperscript{93:5.4} Téraj y toda su familia creían a medias en la religión de Salem, que se había predicado en Caldea; habían oído hablar de Melquisedek a través de los sermones de Ovidio, un educador fenicio que proclamó en Ur las doctrinas de Salem. Salieron de Ur con la intención de ir directamente a Salem, pero Najor, el hermano de Abraham, que no había visto a Melquisedek, era poco entusiasta y los persuadió para que se quedaran en Jarán. Después de su llegada a Palestina, pasó mucho tiempo antes de que estuvieran dispuestos a destruir \textit{todos} los dioses lares que habían traído con ellos; fueron lentos en renunciar a los numerosos dioses de Mesopotamia en favor del Dios único de Salem.

\par
%\textsuperscript{(1019.2)}
\textsuperscript{93:5.5} Pocas semanas después de la muerte de Téraj\footnote{\textit{Muerte de Téraj}: Gn 11:32.}, el padre de Abraham, Melquisedek envió a uno de sus estudiantes, Yaram el Hitita, para que llevara a Abraham y a Najor la siguiente invitación: <<Venid a Salem, donde escucharéis nuestras enseñanzas sobre la verdad del Creador eterno, y el mundo entero será bendecido en vuestra progenie iluminada, la de los dos hermanos>>\footnote{\textit{Venid a Salem, escuchad la verdad}: Gn 12:1-2.}. Pero Najor no había aceptado por completo el evangelio de Melquisedek; se quedó atrás y construyó una poderosa ciudad-Estado que llevó su nombre; pero Lot\footnote{\textit{Lot y Abraham en Salem}: Gn 12:4-5.}, el sobrino de Abraham, decidió acompañar a su tío hasta Salem.

\par
%\textsuperscript{(1019.3)}
\textsuperscript{93:5.6} Cuando llegaron a Salem\footnote{\textit{Llegada de Abraham}: Gn 12:8.}, Abraham y Lot escogieron una fortaleza en las colinas, cerca de la ciudad, donde podían defenderse de los numerosos ataques por sorpresa de los ladrones del norte. En esta época, los hititas, asirios, filisteos y otros grupos asaltaban constantemente las tribus del centro y el sur de Palestina. Desde su plaza fuerte en las colinas, Abraham y Lot hicieron frecuentes peregrinajes a Salem.

\par
%\textsuperscript{(1019.4)}
\textsuperscript{93:5.7} Poco después de haberse establecido cerca de Salem, Abraham y Lot viajaron al valle del Nilo para conseguir víveres, pues en aquel momento había una sequía en Palestina. Durante su breve estancia en Egipto\footnote{\textit{Viaje a Egipto}: Gn 12:10.}, Abraham encontró a un pariente lejano en el trono egipcio, y sirvió como comandante de dos expediciones militares con mucho éxito para este rey. Durante la última parte de su estancia al borde del Nilo, Abraham y su esposa Sara vivieron en la corte, y cuando se marchó de Egipto, recibió una parte del botín de sus campañas militares\footnote{\textit{Buenas relaciones}: Gn 12:16; 13:1-2.}.

\par
%\textsuperscript{(1019.5)}
\textsuperscript{93:5.8} Abraham necesitó una gran resolución para renunciar a los honores de la corte egipcia y volver al trabajo más espiritual patrocinado por Maquiventa. Pero Melquisedek era respetado incluso en Egipto, y cuando informaron de toda la historia al faraón, éste incitó firmemente a Abraham a que regresara para cumplir sus promesas a favor de la causa de Salem.

\par
%\textsuperscript{(1019.6)}
\textsuperscript{93:5.9} Abraham ambicionaba ser rey, y en el camino de vuelta de Egipto, expuso a Lot su plan de someter a todo Canaán y poner a su gente bajo el dominio de Salem. Lot sentía más inclinación por los negocios, de manera que, después de un desacuerdo posterior, se dirigió a Sodoma\footnote{\textit{Lot marcha a Sodoma}: Gn 13:5-12.} para dedicarse al comercio y a la ganadería. A Lot no le gustaba ni la vida militar ni la vida de pastor.

\par
%\textsuperscript{(1019.7)}
\textsuperscript{93:5.10} Después de regresar con su familia a Salem, Abraham empezó a madurar sus proyectos militares. Pronto fue reconocido como gobernante civil del territorio de Salem y había confederado bajo su mando a siete tribus cercanas. Melquisedek tuvo en verdad grandes dificultades para frenar a Abraham, que estaba inflamado con el ardor de salir y reunir a las tribus vecinas con la espada, para que así pudieran conocer más rápidamente las verdades de Salem.

\par
%\textsuperscript{(1019.8)}
\textsuperscript{93:5.11} Melquisedek mantenía relaciones pacíficas con todas las tribus circundantes; no era militarista y nunca fue atacado por ninguno de los ejércitos en sus movimientos de avance o retroceso. Estaba totalmente dispuesto a que Abraham formulara una política defensiva para Salem, tal como la que se puso en práctica posteriormente, pero no aprobaba los ambiciosos proyectos de conquista de su alumno; se produjo pues una ruptura amistosa de relaciones, y Abraham se trasladó a Hebrón\footnote{\textit{Abraham en Hebrón}: Gn 13:18.} para establecer su capital militar.

\par
%\textsuperscript{(1020.1)}
\textsuperscript{93:5.12} Debido a su estrecha relación con el ilustre Melquisedek, Abraham poseía una gran ventaja sobre los reyezuelos de los alrededores; todos respetaban a Melquisedek y temían indebidamente a Abraham. Abraham conocía este miedo y sólo esperaba una ocasión favorable para atacar a sus vecinos; el pretexto se presentó cuando algunos de estos soberanos se atrevieron a asaltar las propiedades de su sobrino Lot\footnote{\textit{Los reyes capturan a Lot}: Gn 14:8-12.}, que residía en Sodoma. Al enterarse de esto, Abraham, a la cabeza de sus siete tribus confederadas, avanzó sobre el enemigo. Su propia escolta de 318 hombres dirigió el ejército de más de 4.000 soldados que atacaron en esta ocasión\footnote{\textit{Abraham contra los reyes}: Gn 14:13-16.}.

\par
%\textsuperscript{(1020.2)}
\textsuperscript{93:5.13} Cuando Melquisedek se enteró de que Abraham había declarado la guerra, salió para disuadirlo, pero sólo lo alcanzó cuando su antiguo discípulo volvía victorioso de la batalla. Abraham se empeñó en que el Dios de Salem le había dado la victoria sobre sus enemigos, e insistió en entregar una décima parte de su botín al tesoro de Salem\footnote{\textit{Abraham entrega el diezmo a Melquisedek}: Gn 14:18-20; Heb 7:2.}. El noventa por ciento restante lo trasladó a su capital en Hebrón.

\par
%\textsuperscript{(1020.3)}
\textsuperscript{93:5.14} Después de esta batalla de Siddim, Abraham se convirtió en el jefe de una segunda confederación de once tribus, y no solamente pagaba el diezmo a Melquisedek, sino que se aseguró de que todos los demás de aquella región hicieran lo mismo. Sus relaciones diplomáticas con el rey de Sodoma, junto con el temor que generalmente le tenían, tuvieron como resultado que el rey de Sodoma y otros se unieran a la confederación militar de Hebrón; Abraham estaba realmente en vías de establecer un poderoso Estado en Palestina.

\section*{6. La alianza de Melquisedek con Abraham}
\par
%\textsuperscript{(1020.4)}
\textsuperscript{93:6.1} Abraham tenía la intención de conquistar todo Canaán. Su determinación sólo estaba debilitada por el hecho de que Melquisedek no quería aprobar la empresa. Pero Abraham casi había decidido embarcarse en el proyecto cuando empezó a preocuparle la idea de que no tenía un hijo para sucederle como soberano de este futuro reino\footnote{\textit{Preocupación por la descendencia}: Gn 15:1-3.}. Preparó otra conferencia con Melquisedek; y en el transcurso de esta entrevista fue cuando el sacerdote de Salem, el Hijo visible de Dios, persuadió a Abraham para que abandonara su proyecto de conquistas materiales y de reinado temporal a favor del concepto espiritual del reino de los cielos.

\par
%\textsuperscript{(1020.5)}
\textsuperscript{93:6.2} Melquisedek explicó a Abraham la inutilidad de luchar contra la confederación amorita, pero también le indicó con claridad que estos clanes atrasados estaban suicidándose indudablemente a causa de sus prácticas insensatas, de manera que en pocas generaciones estarían tan debilitados que los descendientes de Abraham, que habrían aumentado considerablemente mientras tanto, podrían vencerlos fácilmente.

\par
%\textsuperscript{(1020.6)}
\textsuperscript{93:6.3} Melquisedek hizo una alianza formal con Abraham en Salem. Le dijo a Abraham: <<Mira ahora los cielos y cuenta las estrellas si puedes; tu descendencia será tan numerosa como ellas>>\footnote{\textit{Alianza (``mira las estrellas'')}: Gn 13:14-17; 15:4-5; 17:1-9.}. Abraham creyó a Melquisedek, <<y esto le fue contado en justicia>>\footnote{\textit{La fe le fue contado en justicia}: Gn 15:6.}. Melquisedek le contó entonces a Abraham la historia de la futura ocupación de Canaán por sus descendientes después de su estancia en Egipto\footnote{\textit{Anticipación de la ocupación de Canaán}: Gn 15:15-16,18-21. \textit{Anticipación de la salida de Egipto}: Gn 15:12-14.}.

\par
%\textsuperscript{(1020.7)}
\textsuperscript{93:6.4} Esta alianza de Melquisedek con Abraham representa el gran acuerdo urantiano entre la divinidad y la humanidad, según el cual Dios acepta hacerlo \textit{todo}, y el hombre sólo acepta \textit{creer} en las promesas de Dios y seguir sus instrucciones. Hasta ese momento se había creído que la salvación sólo se podía conseguir por medio de las obras ---los sacrificios y las ofrendas; ahora, Melquisedek traía de nuevo a Urantia la buena nueva de que la salvación, el favor de Dios, se puede obtener por la \textit{fe}. Pero este evangelio de la simple fe en Dios era demasiado avanzado; los hombres de las tribus semíticas prefirieron volver posteriormente a los antiguos sacrificios y a la expiación de los pecados mediante el derramamiento de sangre.

\par
%\textsuperscript{(1021.1)}
\textsuperscript{93:6.5} No mucho tiempo después del establecimiento de esta alianza fue cuando nació Isaac\footnote{\textit{Nacimiento de Isaac}: Gn 15:2-4; 17:4-7,16-21; 18:10-14; 21:1-8.}, el hijo de Abraham, de acuerdo con la promesa de Melquisedek. Después del nacimiento de Isaac, Abraham adoptó una actitud muy seria hacia su alianza con Melquisedek, y se desplazó hasta Salem para consignarla por escrito. Durante esta aceptación pública y oficial de la alianza\footnote{\textit{Alianza formal}: Gn 17:10-12,23-27.} fue cuando cambió su nombre de Abram por el de Abraham\footnote{\textit{Nombre cambiado a Abraham}: Gn 17:5.}.

\par
%\textsuperscript{(1021.2)}
\textsuperscript{93:6.6} La mayor parte de los creyentes de Salem habían practicado la circuncisión\footnote{\textit{Circuncisión}: Gn 17:10-13.}, aunque Melquisedek nunca la había hecho obligatoria. Pues bien, Abraham se había opuesto siempre tanto a la circuncisión que en esta ocasión decidió celebrar el acontecimiento aceptando solemnemente este rito como prueba de la ratificación de la alianza de Salem.

\par
%\textsuperscript{(1021.3)}
\textsuperscript{93:6.7} A consecuencia de esta renuncia pública y real a sus ambiciones personales en favor de los planes más amplios de Melquisedek, los tres seres celestiales\footnote{\textit{Los tres seres celestiales}: Gn 18:1-16.} se aparecieron a Abraham en las llanuras de Mambré. Esta aparición fue una realidad\footnote{\textit{Realidad y mitos}: Gn 18:16-33; 19:1-29.}, a pesar de haberse asociado posteriormente con las narraciones inventadas relacionadas con la destrucción natural de Sodoma y Gomorra. Estas leyendas de los acontecimientos de aquellos tiempos indican lo retrasadas que estaban la moral y la ética en una época tan relativamente reciente.

\par
%\textsuperscript{(1021.4)}
\textsuperscript{93:6.8} Con la consumación de esta alianza solemne, la reconciliación entre Abraham y Melquisedek fue completa. Abraham asumió de nuevo la jefatura civil y militar de la colonia de Salem, y las listas de la fraternidad de Melquisedek contaban en su apogeo con más de cien mil contribuyentes regulares que pagaban el diezmo. Abraham mejoró enormemente el templo de Salem y suministró nuevas tiendas para toda la escuela. No sólo amplió el sistema del diezmo, sino que también instituyó numerosos métodos más perfeccionados para dirigir los asuntos de la escuela, además de contribuir considerablemente a gobernar mejor el departamento de propaganda misionera. También contribuyó mucho a mejorar los rebaños y a reorganizar los proyectos de la industria lechera de Salem. Abraham era un hombre de negocios sagaz y eficaz, un hombre rico para su época; no era demasiado piadoso, pero era totalmente sincero y creía realmente en Maquiventa Melquisedek.

\section*{7. Los misioneros de Melquisedek}
\par
%\textsuperscript{(1021.5)}
\textsuperscript{93:7.1} Melquisedek continuó durante algunos años enseñando a sus estudiantes y preparando a los misioneros de Salem, que penetraron en todas las tribus de los alrededores, especialmente en Egipto, Mesopotamia y Asia Menor. A medida que pasaban las décadas, estos educadores se alejaron cada vez más de Salem, llevando con ellos el evangelio de Maquiventa sobre la creencia y la fe en Dios.

\par
%\textsuperscript{(1021.6)}
\textsuperscript{93:7.2} Los descendientes de Adanson, agrupados alrededor de las orillas del lago Van, escucharon de buena gana a los educadores hititas del culto de Salem. Desde este antiguo centro andita se enviaron instructores a las regiones lejanas de Europa y Asia. Los misioneros de Salem penetraron en toda Europa, incluidas las Islas Británicas. Un grupo fue por el camino de las Islas Feroe hasta los andonitas de Islandia, mientras que otro grupo atravesó China y llegó hasta los japoneses de las islas orientales. La vida y las experiencias de los hombres y mujeres que se arriesgaron a salir de Salem, Mesopotamia y el lago Van para iluminar a las tribus del hemisferio oriental representan un capítulo heroico en los anales de la raza humana.

\par
%\textsuperscript{(1022.1)}
\textsuperscript{93:7.3} Pero la tarea era tan grande y las tribus estaban tan atrasadas que los resultados fueron vagos e imprecisos. El evangelio de Salem fue acogido aquí y allá de generación en generación pero, a excepción de Palestina, la idea de un solo Dios nunca fue capaz de conseguir la lealtad continuada de una tribu o de una raza enteras. Mucho antes de la llegada de Jesús, las enseñanzas de los primeros misioneros de Salem se habían sumergido generalmente en las supersticiones y creencias más antiguas y universales. El evangelio original de Melquisedek había sido absorbido casi enteramente por las creencias en la Gran Madre, el Sol y otros cultos antiguos.

\par
%\textsuperscript{(1022.2)}
\textsuperscript{93:7.4} Vosotros que hoy disfrutáis de las ventajas del arte de la imprenta, no podéis comprender muy bien lo difícil que era perpetuar la verdad durante estos tiempos antiguos, y lo fácil que resultaba perder de vista una nueva doctrina de una generación a la siguiente. La nueva doctrina siempre tenía tendencia a ser absorbida por el conjunto más antiguo de enseñanzas religiosas y de prácticas mágicas. Una nueva revelación siempre se contamina con las creencias evolutivas más antiguas.

\section*{8. La partida de Melquisedek}
\par
%\textsuperscript{(1022.3)}
\textsuperscript{93:8.1} Poco después de la destrucción de Sodoma y Gomorra, Maquiventa decidió poner fin a su donación de emergencia en Urantia. La decisión de Melquisedek de terminar su estancia en la carne estuvo influida por numerosas circunstancias, siendo la principal la tendencia creciente de las tribus circundantes, e incluso de sus asociados inmediatos, a considerarlo como un semidiós, a mirarlo como un ser sobrenatural, cosa que era en realidad; pero habían empezado a venerarlo indebidamente y con un temor extremadamente supersticioso. Además de estas razones, Melquisedek deseaba abandonar el escenario de sus actividades terrestres lo suficientemente antes de la muerte de Abraham como para asegurarse de que la verdad de un solo y único Dios se establecería firmemente en la mente de sus seguidores. En consecuencia, Maquiventa se retiró una noche a su tienda de Salem, después de haber deseado las buenas noches a sus compañeros humanos, y cuando éstos fueron a llamarlo por la mañana, ya no estaba allí, pues sus semejantes se lo habían llevado.

\section*{9. Después de la partida de Melquisedek}
\par
%\textsuperscript{(1022.4)}
\textsuperscript{93:9.1} La desaparición tan repentina de Melquisedek fue una gran prueba para Abraham. Aunque Maquiventa había advertido plenamente a sus seguidores de que algún día tendría que irse como había llegado, éstos no se habían resignado a perder a su maravilloso jefe. La gran organización que se había establecido en Salem casi desapareció, aunque Moisés se basó en las tradiciones de esta época para conducir a los esclavos hebreos fuera de Egipto.

\par
%\textsuperscript{(1022.5)}
\textsuperscript{93:9.2} La pérdida de Melquisedek produjo una tristeza en el corazón de Abraham de la que nunca se repuso por completo. Había abandonado Hebrón cuando renunció a la ambición de construir un reino material; y ahora, después de perder a su asociado en la edificación del reino espiritual, partió de Salem y se dirigió hacia el sur\footnote{\textit{Viaje al sur}: Gn 20:1.} para vivir cerca de sus intereses en Guerar.

\par
%\textsuperscript{(1022.6)}
\textsuperscript{93:9.3} Inmediatamente después de la desaparición de Melquisedek, Abraham se volvió temeroso y asustadizo\footnote{\textit{Abraham muestra cobardía}: Gn 12:11-20; 20:2-14.}. Ocultó su identidad cuando llegó a Guerar, de manera que Abimélek se apropió de su esposa\footnote{\textit{Sara tomada por Abimélek}: Gn 20:2.}. (Poco después de casarse con Sara, Abraham había sorprendido cierta noche una conspiración para asesinarlo y quitarle su brillante esposa. Este temor se convirtió en terror para este dirigente por otra parte valiente y atrevido; toda su vida temió que alguien lo matara en secreto para llevarse a Sara. Esto explica por qué, en tres ocasiones diferentes, este hombre valeroso dio muestras de una auténtica cobardía).

\par
%\textsuperscript{(1023.1)}
\textsuperscript{93:9.4} Pero Abraham no iba a permanecer mucho tiempo desanimado en su misión como sucesor de Melquisedek. Pronto hizo conversiones entre los filisteos y el pueblo de Abimélek, firmó un tratado con ellos\footnote{\textit{Tratado}: Gn 21:22-32.}, y se contaminó a su vez con muchas de sus supersticiones\footnote{\textit{Supersticiones}: Gn 22:1-2.}, en particular con su práctica de sacrificar a los hijos primogénitos. Abraham se convirtió así otra vez en un gran dirigente en Palestina. Todos los grupos lo respetaban y todos los reyes lo honraban. Era el jefe espiritual de todas las tribus circundantes, y su influencia perduró algún tiempo después de su muerte. Durante los últimos años de su vida volvió una vez más a Hebrón\footnote{\textit{Regreso a Hebrón}: Gn 23:2,17-20.}, el escenario de sus primeras actividades y el lugar donde había trabajado en asociación con Melquisedek. El último acto de Abraham consistió en enviar a unos criados leales a la ciudad de su hermano Najor, en la frontera de Mesopotamia, para conseguir una mujer de su propio pueblo como esposa para su hijo Isaac\footnote{\textit{Conseguir mujer para Isaac}: Gn 24:2-4,9; 24:36-38,51.}. El pueblo de Abraham había tenido durante mucho tiempo la costumbre de casarse entre primos. Y Abraham murió confiando en la fe en Dios que había aprendido de Melquisedek en las escuelas desaparecidas de Salem\footnote{\textit{Muerte de Abraham}: Gn 25:8.}.

\par
%\textsuperscript{(1023.2)}
\textsuperscript{93:9.5} La generación siguiente tuvo dificultades para comprender la historia de Melquisedek; en menos de quinientos años, muchos consideraron todo el relato como un mito. Isaac conservó bastante bien las enseñanzas de su padre y fomentó el evangelio de la colonia de Salem, pero a Jacob le resultó más difícil captar el significado de estas tradiciones. José creía firmemente en Melquisedek y, debido principalmente a esto, sus hermanos lo consideraban como un soñador\footnote{\textit{Soñador}: Gn 37:19.}. Los honores que le concedieron a José en Egipto se debieron principalmente a la memoria de su bisabuelo Abraham. A José le ofrecieron el mando militar de los ejércitos egipcios, pero como era un creyente tan firme en las tradiciones de Melquisedek y en las enseñanzas posteriores de Abraham e Isaac, eligió servir como administrador civil\footnote{\textit{Administrador civil}: Gn 41:41.}, creyendo que así podría trabajar mejor por el progreso del reino de los cielos.

\par
%\textsuperscript{(1023.3)}
\textsuperscript{93:9.6} La enseñanza de Melquisedek fue plena y completa, pero los anales de estos tiempos parecieron imposibles y fantásticos a los sacerdotes hebreos posteriores, aunque muchos de ellos comprendieron en parte estas memorias, al menos hasta la época en que los documentos del Antiguo Testamento fueron redactados en masa en Babilonia.

\par
%\textsuperscript{(1023.4)}
\textsuperscript{93:9.7} Lo que los escritos del Antiguo Testamento describen como conversaciones entre Abraham y Dios, eran en realidad entrevistas entre Abraham y Melquisedek\footnote{\textit{Conversaciones con Melquisedek}: Gn 12:7; 13:14; 15:1,4,7; 16:9-13; 17:1-22; 18:1,9-10,13; 18:17,20,23.}. Los escribas posteriores consideraron que el término Melquisedek era sinónimo de Dios. El relato de los múltiples contactos de Abraham y Sara con <<el ángel del Señor>>\footnote{\textit{Ángel del Señor}: Gn 16:7,9-11; 22:11,15.} se refieren a sus numerosas conversaciones con Melquisedek.

\par
%\textsuperscript{(1023.5)}
\textsuperscript{93:9.8} Las narraciones hebreas sobre Isaac, Jacob y José son mucho más fiables que las que se refieren a Abraham, aunque también contienen muchas desviaciones de los hechos, unas alteraciones que los sacerdotes hebreos hicieron tanto intencionalmente como sin intención en la época de la compilación de estas historias durante la cautividad en Babilonia. Queturá\footnote{\textit{Queturá no era esposa}: Gn 25:1; 1 Cr 1:32.} no fue una esposa de Abraham; fue simplemente una concubina como Agar. Todas las propiedades de Abraham fueron heredadas por Isaac\footnote{\textit{Herencia de Isaac}: Gn 25:5.}, el hijo de Sara, la esposa legal. Abraham no era tan viejo\footnote{\textit{Edad de Abraham}: Gn 17:1,17; 18:11; 21:2,5; 24:1; 25:7-8.} como lo indican los relatos, y su esposa era mucho más joven\footnote{\textit{Edad de Sara}: Gn 17:17; 18:11; 23:1.}. Sus edades fueron cambiadas deliberadamente a fin de asegurar el supuesto nacimiento milagroso posterior de Isaac\footnote{\textit{Nacimiento milagroso de Isaac}: Gn 17:16-17; 18:10-12; 21:1-8.}.

\par
%\textsuperscript{(1023.6)}
\textsuperscript{93:9.9} El ego nacional de los judíos estaba enormemente deprimido debido a la cautividad en Babilonia. En su reacción contra su inferioridad nacional oscilaron hacia el otro extremo del egotismo nacional y racial, desvirtuando y desnaturalizando sus tradiciones con el objeto de elevarse por encima de todas las razas como pueblo elegido de Dios; por lo tanto corrigieron cuidadosamente todos sus documentos para elevar a Abraham y a sus otros jefes nacionales muy por encima de todas las demás personas, sin exceptuar al mismo Melquisedek. Los escribas hebreos destruyeron pues todos los archivos que pudieron encontrar sobre aquellos tiempos trascendentales, conservando únicamente el relato del encuentro de Abraham con Melquisedek después de la batalla de Siddim, que según ellos hacía recaer un gran honor sobre Abraham\footnote{\textit{Omisiones en los registros}: Gn 14:18-20; Heb 7:1-2.}.

\par
%\textsuperscript{(1024.1)}
\textsuperscript{93:9.10} Y así, al perder de vista a Melquisedek, también perdieron de vista la enseñanza de este Hijo de emergencia en lo que se refiere a la misión espiritual del Hijo donador prometido; perdieron de vista la naturaleza de esta misión de una manera tan plena y completa, que muy pocos de sus descendientes pudieron o quisieron reconocer y recibir a Miguel cuando éste apareció encarnado en la Tierra tal como Maquiventa lo había predicho.

\par
%\textsuperscript{(1024.2)}
\textsuperscript{93:9.11} Pero uno de los escritores del Libro de los Hebreos comprendió la misión de Melquisedek, pues está escrito: <<Este Melquisedek, sacerdote del Altísimo, era también un rey de paz; sin padre, ni madre, ni genealogía, sin comienzo de días ni fin de vida, asemejado al Hijo de Dios, permanece sacerdote para siempre>>\footnote{\textit{Rey de paz, sin padres}: Heb 7:1-3.}. Este escritor identificó a Melquisedek como un modelo de la donación posterior de Miguel, afirmando que Jesús era <<un sacerdote para siempre, a semejanza de Melquisedek>>\footnote{\textit{Sacerdote para siempre}: Sal 110:4; Heb 5:6,20; 6:20; 7:17,21. \textit{De la orden de Melquisedek}: Heb 5:10; 7:11.}. Aunque esta comparación no es del todo afortunada, es literalmente cierto que Cristo recibió el título provisional de Urantia <<a petición de los doce síndicos Melquisedeks>> de servicio en la época de su donación en este mundo.

\section*{10. El estado actual de Maquiventa Melquisedek}
\par
%\textsuperscript{(1024.3)}
\textsuperscript{93:10.1} Durante los años de la encarnación de Maquiventa, los síndicos Melquisedeks de Urantia ejercieron su actividad en número de once. Cuando Maquiventa consideró que su misión como Hijo de emergencia había terminado, señaló este hecho a sus once asociados y éstos prepararon inmediatamente la técnica por la cual sería liberado de la carne y restablecido a salvo en su estado original como Melquisedek. Al tercer día después de su desaparición de Salem, apareció entre sus once compañeros de misión en Urantia y reanudó su carrera interrumpida como uno de los síndicos planetarios de la 606 de Satania.

\par
%\textsuperscript{(1024.4)}
\textsuperscript{93:10.2} Maquiventa terminó su donación como criatura de carne y hueso de una manera tan brusca y repentina como la había empezado. Tanto su aparición como su partida no estuvieron acompañadas de ningún anuncio o demostración fuera de lo común; su aparición en Urantia no estuvo marcada por un llamamiento a la resurrección ni por el final de una dispensación planetaria; la suya fue una donación de urgencia. Pero Maquiventa no puso fin a su estancia en la similitud de los seres humanos hasta que no fue debidamente liberado por el Padre Melquisedek, e informado de que su donación de emergencia había recibido la aprobación de Gabriel de Salvington, el jefe ejecutivo de Nebadon.

\par
%\textsuperscript{(1024.5)}
\textsuperscript{93:10.3} Maquiventa Melquisedek continuó tomándose un gran interés por los asuntos de los descendientes de los hombres que habían creído en sus enseñanzas mientras vivía en la carne. Pero los descendientes de Abraham a través de Isaac, que se casaron con los kenitas, fueron el único linaje que continuó manteniendo durante mucho tiempo un concepto claro de las enseñanzas de Salem.

\par
%\textsuperscript{(1024.6)}
\textsuperscript{93:10.4} Este mismo Melquisedek siguió colaborando durante los diecinueve siglos siguientes con numerosos profetas y videntes, esforzándose así por mantener vivas las verdades de Salem hasta que Miguel apareciera a su debido tiempo en la Tierra.

\par
%\textsuperscript{(1025.1)}
\textsuperscript{93:10.5} Maquiventa continuó como síndico planetario hasta la época del triunfo de Miguel en Urantia. Posteriormente se le destinó al servicio de Urantia en Jerusem como uno de los veinticuatro directores, y recientemente acaba de ser elevado a la categoría de embajador personal del Hijo Creador en Jerusem, con el título de Príncipe Planetario Vicegerente de Urantia. Creemos que, mientras Urantia siga siendo un planeta habitado, Maquiventa Melquisedek no volverá a ejercer plenamente los deberes de su orden de filiación, sino que seguirá siendo siempre, hablando en términos temporales, un ministro planetario representante de Cristo Miguel.

\par
%\textsuperscript{(1025.2)}
\textsuperscript{93:10.6} Como su misión en Urantia fue una donación de emergencia, los archivos no indican cuál podrá ser el futuro de Maquiventa. Puede suceder que el cuerpo de los Melquisedeks de Nebadon haya sufrido la pérdida permanente de uno de sus miembros. Unas resoluciones recientes, transmitidas por los Altísimos de Edentia y confirmadas después por los Ancianos de los Días de Uversa, sugieren enormemente que este Melquisedek donador está destinado a sustituir a Caligastia, el Príncipe Planetario caído. Si nuestras conjeturas a este respecto son correctas, es totalmente posible que Maquiventa Melquisedek reaparezca en persona en Urantia y, de alguna manera modificada, reasuma las funciones del Príncipe Planetario destronado; o bien aparezca en la Tierra para ejercer su actividad como Príncipe Planetario vicegerente, representando a Cristo Miguel, que actualmente posee el título de Príncipe Planetario de Urantia. Aunque no está nada claro para nosotros cuál podrá ser el destino de Maquiventa, sin embargo, unos acontecimientos que han tenido lugar muy recientemente sugieren poderosamente que las conjeturas anteriormente mencionadas no están probablemente muy lejos de la verdad.

\par
%\textsuperscript{(1025.3)}
\textsuperscript{93:10.7} Comprendemos muy bien la manera en que, debido a su triunfo en Urantia, Miguel se volvió el sucesor de Caligastia y de Adán; la manera en que se convirtió en el Príncipe planetario de la Paz\footnote{\textit{Príncipe de la Paz}: Is 9:6.} y en el segundo Adán\footnote{\textit{Segundo Adán}: 1 Co 15:45-47.}. Y ahora observamos que a este Melquisedek se le confiere el título de Príncipe Planetario Vicegerente de Urantia. ¿Será nombrado también Hijo Material Vicegerente de Urantia? ¿O existe la posibilidad de que se produzca un acontecimiento inesperado y sin precedentes, como el regreso en algún momento al planeta de Adán y Eva o de algunos de sus descendientes, como representantes de Miguel y con los títulos de vicegerentes del segundo Adán de Urantia?

\par
%\textsuperscript{(1025.4)}
\textsuperscript{93:10.8} Todas estas especulaciones, unidas a la certidumbre de que tanto los Hijos Magistrales como los Hijos Instructores Trinitarios aparecerán en el futuro, conjuntamente con la promesa explícita del Hijo Creador de regresar algún día, convierten a Urantia en un planeta de incierto futuro y hacen que resulte una de las esferas más interesantes y fascinantes de todo el universo de Nebadon. Es totalmente posible que en alguna época futura, cuando Urantia se aproxime a la era de luz y de vida, después de que se hayan juzgado finalmente los asuntos de la rebelión de Lucifer y de la secesión de Caligastia, podamos contemplar la presencia simultánea en Urantia de Maquiventa, Adán, Eva y Cristo Miguel, así como de un Hijo Magistral o incluso de los Hijos Instructores Trinitarios.

\par
%\textsuperscript{(1025.5)}
\textsuperscript{93:10.9} Nuestra orden ha tenido mucho tiempo la opinión de que la presencia de Maquiventa en el cuerpo de los veinticuatro consejeros, los directores de Urantia en Jerusem, es una prueba suficiente para justificar la creencia de que Maquiventa está destinado a seguir a los mortales de Urantia a través de todo el programa universal de progresión y de ascensión, incluso hasta el Cuerpo Paradisiaco de la Finalidad. Sabemos que Adán y Eva están destinados a acompañar así a sus compañeros terrestres en la aventura hacia el Paraíso cuando Urantia se haya establecido en la luz y la vida.

\par
%\textsuperscript{(1025.6)}
\textsuperscript{93:10.10} Hace menos de mil años, este mismo Maquiventa Melquisedek, el antiguo sabio de Salem, estuvo presente de manera invisible en Urantia durante un período de cien años, desempeñando sus funciones como gobernador general residente del planeta; y si el sistema que se emplea actualmente para dirigir los asuntos planetarios continúa, deberá regresar para ocupar el mismo cargo en poco más de mil años.

\par
%\textsuperscript{(1026.1)}
\textsuperscript{93:10.11} Ésta es la historia de Maquiventa Melquisedek, uno de los personajes más extraordinarios que hayan estado jamás relacionados con la historia de Urantia, y una personalidad que puede estar destinada a jugar un papel importante en la experiencia futura de vuestro mundo irregular y poco común.

\par
%\textsuperscript{(1026.2)}
\textsuperscript{93:10.12} [Presentado por un Melquisedek de Nebadon.]


\chapter{Documento 94. Las enseñanzas de Melquisedek en Oriente}
\par
%\textsuperscript{(1027.1)}
\textsuperscript{94:0.1} LOS primeros educadores de la religión de Salem penetraron hasta las tribus más apartadas de África y Eurasia, predicando constantemente el evangelio enseñado por Maquiventa de la fe y la confianza del hombre en un solo Dios universal como único precio a pagar para obtener el favor divino. La alianza de Melquisedek con Abraham sirvió de modelo para toda la propaganda inicial que salió de Salem y de otros centros. Urantia nunca ha tenido, en ninguna religión, unos misioneros más entusiastas y dinámicos que estos nobles hombres y mujeres que llevaron las enseñanzas de Melquisedek por todo el hemisferio oriental. Estos misioneros fueron reclutados entre numerosos pueblos y razas, y difundieron sus enseñanzas principalmente por medio de los indígenas convertidos. Establecieron centros de educación en diferentes partes del mundo, donde enseñaron a los nativos la religión de Salem, y luego encargaron a estos alumnos que ejercieran como educadores en sus propios pueblos.

\section*{1. Las enseñanzas de Salem en la India védica}
\par
%\textsuperscript{(1027.2)}
\textsuperscript{94:1.1} En los tiempos de Melquisedek, la India era un país cosmopolita que había caído recientemente bajo el dominio político y religioso de los invasores ario-anditas procedentes del norte y del oeste. En esta época, sólo las partes nórdica y occidental de la península habían sido ampliamente impregnadas por los arios. Estos recién llegados védicos habían traído con ellos sus numerosas deidades tribales. Las formas religiosas de su culto seguían de cerca las prácticas ceremoniales de sus antiguos antepasados anditas, ya que el padre seguía actuando como sacerdote y la madre como sacerdotisa, y el fogón familiar se utilizaba todavía como altar.

\par
%\textsuperscript{(1027.3)}
\textsuperscript{94:1.2} El culto védico estaba entonces en proceso de crecimiento y metamorfosis bajo la dirección de la casta brahmánica de sacerdotes-educadores, los cuales asumían gradualmente el control del ritual de adoración en vías de desarrollo. La fusión de las antiguas treinta y tres deidades arias estaba muy avanzada cuando los misioneros de Salem penetraron en el norte de la India.

\par
%\textsuperscript{(1027.4)}
\textsuperscript{94:1.3} El politeísmo de estos arios representaba una degeneración de su monoteísmo anterior, causada por su separación en unidades tribales, donde cada tribu veneraba a su propio dios. Esta degeneración del monoteísmo y del trinitarismo originales de la Mesopotamia andita estaba pasando por un nuevo proceso de síntesis en los primeros siglos del segundo milenio antes de Cristo. Los numerosos dioses estaban organizados en un panteón bajo la dirección trina de Dyaus pitar, el señor de los cielos, de Indra, el tempestuoso señor de la atmósfera, y de Agni, el dios tricéfalo del fuego, señor de la Tierra y símbolo rudimentario de un concepto más antiguo de la Trinidad.

\par
%\textsuperscript{(1027.5)}
\textsuperscript{94:1.4} Unos desarrollos claramente henoteístas estaban preparando el camino para un monoteísmo evolucionado. Agni, la deidad más antigua, era ensalzada a menudo como padre-jefe de todo el panteón. El principio de la deidad-padre, a veces llamado Prajapati y otras veces denominado Brahma, quedó sumergido en la batalla teológica que los sacerdotes brahmánicos libraron más tarde contra los educadores de Salem. El principio de energía-divinidad que activaba todo el panteón védico era concebido como \textit{El Brahmán}.

\par
%\textsuperscript{(1028.1)}
\textsuperscript{94:1.5} Los misioneros de Salem predicaban el Dios único de Melquisedek, el Altísimo que está en el cielo. Esta descripción no era del todo discordante con el concepto emergente del Brahma-Padre como fuente de todos los dioses, pero la doctrina de Salem no era ritualista y por lo tanto se oponía directamente a los dogmas, tradiciones y enseñanzas del clero brahmánico. Los sacerdotes brahmánicos no quisieron aceptar nunca la enseñanza de Salem sobre la salvación a través de la fe, el favor de Dios sin prácticas ritualistas ni ceremoniales sacrificatorios.

\par
%\textsuperscript{(1028.2)}
\textsuperscript{94:1.6} El rechazo del evangelio de la confianza en Dios y de la salvación por medio de la fe, predicado por Melquisedek, marcó un hito capital para la India. Los misioneros de Salem habían contribuido mucho a que se perdiera la fe en todos los antiguos dioses védicos, pero los dirigentes, los sacerdotes del vedismo, se negaron a aceptar la enseñanza de Melquisedek sobre un solo Dios y una sola y sencilla fe.

\par
%\textsuperscript{(1028.3)}
\textsuperscript{94:1.7} Los brahmanes seleccionaron los escritos sagrados de su época en un esfuerzo por combatir a los educadores de Salem, y esta compilación, tal como fue revisada más tarde, ha llegado hasta los tiempos modernos bajo la forma del Rig-Veda, uno de los libros sagrados más antiguos. El segundo, tercero y cuarto Vedas vinieron después a medida que los brahmanes intentaron cristalizar, formalizar y fijar sus rituales de adoración y de sacrificios para la gente de aquellos tiempos. En aquello que poseen de mejor, estos escritos son equivalentes a cualquier otra obra de carácter similar en lo que se refiere a la belleza de los conceptos y al discernimiento de la verdad. Pero a medida que esta religión superior se contaminó con los millares de supersticiones, cultos y rituales de la India meridional, se transformó progresivamente en el sistema teológico más abigarrado que el hombre mortal haya desarrollado jamás. Un examen de los Vedas revelará algunos de los conceptos más elevados sobre la Deidad, y otros entre los más degradados, que se hayan concebido jamás.

\section*{2. El brahmanismo}
\par
%\textsuperscript{(1028.4)}
\textsuperscript{94:2.1} A medida que los misioneros de Salem penetraron hacia el sur en el Decán dravidiano, se encontraron con un sistema de castas cada vez mayor, el proyecto de los arios para impedir que se perdiera su identidad racial ante una marea creciente de pueblos sangiks secundarios. Puesto que la casta sacerdotal brahmánica era la esencia misma de este sistema, este orden social retrasó enormemente el progreso de los instructores de Salem. Este sistema de castas no consiguió salvar a la raza aria, pero sí logró perpetuar a los brahmanes, los cuales, a su vez, han mantenido su hegemonía religiosa en la India hasta la época actual.

\par
%\textsuperscript{(1028.5)}
\textsuperscript{94:2.2} Luego, con el debilitamiento del vedismo debido al rechazo de una verdad superior, el culto de los arios estuvo sometido a crecientes incursiones procedentes del Decán. En un esfuerzo desesperado por detener la marea de la extinción racial y la destrucción religiosa, la casta brahmánica trató de elevarse por encima de todo lo demás. Enseñaron que el sacrificio a la deidad era en sí mismo totalmente eficaz, que su fuerza era completamente irresistible. Proclamaron que, de los dos principios divinos esenciales del universo, uno era la deidad Brahmán y el otro el clero brahmánico. Los sacerdotes no se han atrevido, en ningún otro pueblo de Urantia, a elevarse por encima incluso de sus dioses, a atribuirse los honores debidos a sus dioses. Pero llegaron tan absurdamente lejos en estas afirmaciones presuntuosas, que todo este sistema precario se derrumbó ante los cultos degradantes que entraban a raudales procedentes de las civilizaciones circundantes menos avanzadas. El inmenso clero védico mismo tropezó y se hundió en la tenebrosa inundación de inercia y pesimismo que su propia presunción egoísta e insensata había provocado en toda la India.

\par
%\textsuperscript{(1029.1)}
\textsuperscript{94:2.3} La concentración excesiva en el yo condujo inevitablemente a temer la perpetuación no evolutiva del yo en un círculo sin fin de encarnaciones sucesivas como hombre, animal o hierba. De todas las creencias contaminantes que podían haberse adherido a lo que podría haber sido un monoteísmo emergente, ninguna fue tan embrutecedora como esta creencia en la transmigración ---la doctrina de la reencarnación de las almas--- que procedía del Decán dravidiano. Esta creencia en una serie monótona y agotadora de transmigraciones repetidas quitó a los mortales combativos su esperanza largamente acariciada de encontrar en la muerte la liberación y el avance espiritual que habían formado parte de la fe védica anterior.

\par
%\textsuperscript{(1029.2)}
\textsuperscript{94:2.4} A esta enseñanza filosóficamente debilitadora pronto le siguió la invención de la doctrina de que uno puede librarse eternamente de su yo sumergiéndose en el descanso y la paz universales de la unión absoluta con Brahmán, la superalma de toda la creación. Los deseos de los mortales y las ambiciones humanas fueron eficazmente eliminados y prácticamente destruidos. Durante más de dos mil años, los mejores cerebros de la India han intentado evitar todo deseo, y la puerta estaba así totalmente abierta para la entrada de los cultos y las enseñanzas posteriores que han atado prácticamente el alma de muchos pueblos hindúes a las cadenas de la desesperación espiritual. De todas las civilizaciones, la védico-aria fue la que pagó el precio más terrible por haber rechazado el evangelio de Salem.

\par
%\textsuperscript{(1029.3)}
\textsuperscript{94:2.5} Las castas por sí solas no podían perpetuar el sistema religioso-cultural ario, y a medida que las religiones inferiores del Decán penetraban en el norte, se desarrolló una era de desconsuelo y desesperación. El culto de no quitarle la vida a ninguna criatura surgió durante esta época sombría, y ha sobrevivido desde entonces. Muchos de estos nuevos cultos eran francamente ateos, y afirmaban que toda salvación que se pudiera alcanzar sólo podía provenir de los propios esfuerzos del hombre sin ayuda exterior. Sin embargo, a lo largo de una gran parte de toda esta filosofía desafortunada, se pueden encontrar los vestigios deformados de las enseñanzas de Melquisedek e incluso de Adán.

\par
%\textsuperscript{(1029.4)}
\textsuperscript{94:2.6} Ésta fue la época de la compilación de las escrituras más recientes de la fe hindú, los Brahmanas y los Upanishads. Después de haber rechazado las enseñanzas de la religión personal consistente en la experiencia de la fe personal con el Dios único, y después de haberse contaminado con la inundación de los cultos y credos degradantes y debilitantes del Decán, con sus antropomorfismos y reencarnaciones, el clero brahmánico experimentó una violenta reacción contra estas creencias corruptoras; existió un esfuerzo preciso por buscar y encontrar la \textit{verdadera realidad}. Los brahmanes empezaron a desantropomorfizar el concepto indio de la deidad, pero al hacerlo cometieron el grave error de despersonalizar el concepto de Dios, y salieron de esta situación, no con un ideal elevado y espiritual del Padre Paradisiaco, sino con la idea distante y metafísica de un Absoluto que lo abarca todo.

\par
%\textsuperscript{(1029.5)}
\textsuperscript{94:2.7} En sus esfuerzos por protegerse, los brahmanes habían rechazado al Dios único de Melquisedek, y ahora se encontraban con la hipótesis del Brahmán, ese yo filosófico impreciso e ilusorio, ese \textit{algo} impersonal e impotente, que ha dejado desamparada y postrada la vida espiritual de la India desde aquella época desdichada hasta el siglo veinte.

\par
%\textsuperscript{(1029.6)}
\textsuperscript{94:2.8} El budismo apareció en la India durante los tiempos en que se escribieron los Upanishads. Pero a pesar de sus mil años de éxito, no pudo competir con el hinduismo posterior; a pesar de su moralidad superior, su descripción inicial de Dios estaba incluso menos bien definida que la del hinduismo, el cual disponía de deidades menores y personales. El budismo cedió finalmente, en el norte de la India, ante los ataques violentos de un islam militante con su concepto bien definido de Alá como Dios supremo del universo.

\section*{3. La filosofía brahmánica}
\par
%\textsuperscript{(1030.1)}
\textsuperscript{94:3.1} Aunque la fase más elevada del brahmanismo apenas era una religión, constituyó realmente uno de los intentos más nobles de la mente mortal por alcanzar los dominios de la filosofía y la metafísica. Después de ponerse en camino para descubrir la realidad final, la mente india no se detuvo hasta haber especulado sobre casi todas las fases de la teología, a excepción del doble concepto esencial de la religión: la existencia del Padre Universal de todas las criaturas del universo, y el hecho de la experiencia ascendente en el universo de estas mismas criaturas mientras tratan de alcanzar al Padre eterno, el cual les ha ordenado que sean perfectas como él mismo es perfecto.

\par
%\textsuperscript{(1030.2)}
\textsuperscript{94:3.2} En el concepto del Brahmán, la mente de aquella época captaba realmente la idea de algún Absoluto que lo impregnaba todo, ya que a este postulado se le identificaba al mismo tiempo como energía creativa y reacción cósmica. Se pensaba que el Brahmán estaba más allá de toda definición, que sólo se podía comprender mediante la negación sucesiva de todas las cualidades finitas. Se trataba claramente de una creencia en un ser absoluto e incluso infinito, pero este concepto estaba ampliamente desprovisto de los atributos de la personalidad y, por lo tanto, no era experimentable por las personas religiosas individuales.

\par
%\textsuperscript{(1030.3)}
\textsuperscript{94:3.3} Al Brahmán-Narayana se le concibió como el Absoluto, el infinito ELLO ES, la fuerza creativa primordial del cosmos potencial, el Yo Universal que existe en el estado estático y potencial a lo largo de toda la eternidad. Si los filósofos de aquellos tiempos hubieran sido capaces de dar el siguiente paso en la concepción de la deidad, si hubieran sido capaces de concebir al Brahmán como asociativo y creador, como una personalidad alcanzable por los seres creados y evolutivos, entonces esta enseñanza podría haberse convertido en la descripción más avanzada de la Deidad en Urantia, puesto que habría abarcado los cinco primeros niveles de la función total de la deidad, y quizás hubiera imaginado los dos restantes.

\par
%\textsuperscript{(1030.4)}
\textsuperscript{94:3.4} En algunas fases, el concepto de la Única Superalma Universal como totalidad de la suma de la existencia de todas las criaturas, condujo a los filósofos indios muy cerca de la verdad del Ser Supremo, pero esta verdad no les sirvió de nada porque no lograron desarrollar una vía de acceso personal, razonable o racional, para poder alcanzar su meta monoteísta teórica del Brahmán-Narayana.

\par
%\textsuperscript{(1030.5)}
\textsuperscript{94:3.5} El principio kármico de la continuidad causal se encuentra también muy cerca de la verdad de que todas las acciones espacio-temporales repercuten, en forma de síntesis, en la presencia de la Deidad del Supremo; pero este postulado nunca aseguró que, paralelamente a todo lo anterior, el practicante individual de la religión pudiera alcanzar personalmente la Deidad, asegurando tan sólo la sumersión última de todas las personalidades en la Superalma Universal.

\par
%\textsuperscript{(1030.6)}
\textsuperscript{94:3.6} La filosofía del brahmanismo también se acercó mucho al descubrimiento de que los Ajustadores del Pensamiento residen en los hombres, pero este concepto se desvirtuó a causa de una idea falsa de la verdad. La enseñanza de que el alma es la morada del Brahmán hubiera preparado el camino para una religión avanzada, si este concepto no se hubiera contaminado por completo con la creencia de que no existe ninguna individualidad humana fuera de esta presencia interna del Uno Universal.

\par
%\textsuperscript{(1030.7)}
\textsuperscript{94:3.7} En la doctrina de que el alma individual se funde con la Superalma, los teólogos de la India no lograron prever la supervivencia de algo humano, de algo nuevo y único, de algo nacido de la unión de la voluntad del hombre y la voluntad de Dios. La enseñanza sobre el regreso del alma al Brahmán es estrechamente paralela a la verdad del regreso del Ajustador al seno del Padre Universal, pero hay algo distinto al Ajustador que sobrevive también, el duplicado morontial de la personalidad mortal. Este concepto vital estaba desgraciadamente ausente en la filosofía brahmánica.

\par
%\textsuperscript{(1031.1)}
\textsuperscript{94:3.8} La filosofía brahmánica se ha aproximado a muchos hechos del universo y se ha acercado a numerosas verdades cósmicas, pero con demasiada frecuencia ha caído víctima del error de no conseguir diferenciar entre los diversos niveles de la realidad, tales como el absoluto, el trascendental y el finito. No ha logrado tener en cuenta que aquello que puede ser finito e ilusorio en el nivel absoluto, puede ser absolutamente real en el nivel finito. Tampoco ha tenido en cuenta la personalidad esencial del Padre Universal, con quien se puede contactar personalmente en todos los niveles, desde el de la experiencia limitada de la criatura evolutiva con Dios, hasta el de la experiencia ilimitada del Hijo Eterno con el Padre Paradisiaco.

\section*{4. La religión hindú}
\par
%\textsuperscript{(1031.2)}
\textsuperscript{94:4.1} Con el paso de los siglos, el pueblo de la India volvió hasta cierto punto a los antiguos rituales de los Vedas, tal como éstos habían sido modificados por las enseñanzas de los misioneros de Melquisedek y cristalizados por el clero brahmánico posterior. Esta religión, la más antigua y la más cosmopolita del mundo, ha sufrido cambios adicionales en respuesta al budismo, al jainismo, y a las influencias del mahometismo y el cristianismo que aparecieron después. Pero cuando llegaron las enseñanzas de Jesús, ya estaban tan occidentalizadas que sólo eran una <<religión del hombre blanco>>, por lo tanto extrañas y ajenas para la mente hindú.

\par
%\textsuperscript{(1031.3)}
\textsuperscript{94:4.2} En la actualidad, la teología hindú describe cuatro niveles descendentes de la deidad y la divinidad:

\par
%\textsuperscript{(1031.4)}
\textsuperscript{94:4.3} 1. \textit{El Brahmán}, el Absoluto, el Uno Infinito, el ELLO ES.

\par
%\textsuperscript{(1031.5)}
\textsuperscript{94:4.4} 2. \textit{La Trimurti}, la trinidad suprema del hinduismo. Se piensa que el primer miembro de esta asociación, \textit{Brahma}, se ha creado a sí mismo a partir del Brahmán ---de la infinidad. Si no fuera por su estrecha identificación con el Uno Infinito panteísta, Brahma podría constituir el fundamento de un concepto del Padre Universal. A Brahma también se le identifica con el destino.

\par
%\textsuperscript{(1031.6)}
\textsuperscript{94:4.5} La adoración de Siva y Vichnú, el segundo y tercer miembros, surgió en el primer milenio después de Cristo. \textit{Siva} es el señor de la vida y la muerte, el dios de la fertilidad y el amo de la destrucción. \textit{Vichnú} es extremadamente popular debido a la creencia de que se encarna periódicamente en forma humana. De esta manera, Vichnú se vuelve real y viviente en la imaginación de los indios. Algunos consideran que Siva y Vichnú son supremos por encima de todo.

\par
%\textsuperscript{(1031.7)}
\textsuperscript{94:4.6} 3. \textit{Las deidades védicas y postvédicas}. Muchos dioses antiguos de los arios, tales como Agni, Indra y Soma, han sobrevivido como dioses de menor importancia que los tres miembros de la Trimurti. Numerosos dioses adicionales han surgido desde los primeros tiempos de la India védica, y éstos también han sido incorporados en el panteón hindú.

\par
%\textsuperscript{(1031.8)}
\textsuperscript{94:4.7} 4. \textit{Los semidioses:} superhombres, semidioses, héroes, demonios, fantasmas, espíritus malignos, hadas, monstruos, duendes, y santos de los cultos más recientes.

\par
%\textsuperscript{(1031.9)}
\textsuperscript{94:4.8} Aunque el hinduismo no ha logrado vivificar al pueblo indio durante mucho tiempo, ha sido generalmente a la vez una religión tolerante. Su gran fuerza reside en el hecho de que ha demostrado ser la religión más adaptable y amorfa que ha aparecido en Urantia. Es capaz de cambiar de una manera casi ilimitada y posee un nivel inhabitual de adaptación flexible, desde las especulaciones elevadas y semimonoteístas de los brahmanes intelectuales, hasta el fetichismo redomado y las prácticas cultuales primitivas de las clases degradadas y deprimidas de creyentes ignorantes.

\par
%\textsuperscript{(1032.1)}
\textsuperscript{94:4.9} El hinduismo ha sobrevivido porque es esencialmente una parte integrante del tejido social básico de la India. No posee una importante jerarquía que pueda ser perturbada o destruida; está entremezclado en la forma de vida del pueblo. Posee una adaptabilidad a las condiciones cambiantes que supera a todos los demás cultos, y muestra una actitud tolerante de adopción hacia otras muchas religiones, pretendiendo que Gautama Buda e incluso el mismo Cristo fueron encarnaciones de Vichnú.

\par
%\textsuperscript{(1032.2)}
\textsuperscript{94:4.10} Hoy, la India tiene la gran necesidad de una presentación del evangelio de Jesús ---la Paternidad de Dios y la filiación de todos los hombres, con la fraternidad consiguiente, que se lleva a cabo personalmente mediante el ministerio amoroso y el servicio social. En la India, el armazón filosófico existe, la estructura del culto está presente; lo único que se necesita es la chispa vivificante del amor dinámico descrito en el evangelio original del Hijo del Hombre, despojado de los dogmas y las doctrinas occidentales que han tendido a hacer de la donación vital de Miguel una religión del hombre blanco.

\section*{5. La lucha por la verdad en China}
\par
%\textsuperscript{(1032.3)}
\textsuperscript{94:5.1} A medida que los misioneros de Salem pasaron por Asia, divulgando la doctrina del Dios Altísimo y la salvación por medio de la fe, absorbieron una gran parte de la filosofía y el pensamiento religioso de los diversos países que atravesaron. Pero los educadores enviados por Melquisedek y sus sucesores no dejaron de cumplir con su deber; penetraron en todos los pueblos del continente eurasiático, y a mediados del segundo milenio antes de Cristo fue cuando llegaron a China. Los salemitas mantuvieron su sede en Si Fuch durante más de cien años, y allí instruyeron a los educadores chinos que enseñaron en todos los territorios de la raza amarilla.

\par
%\textsuperscript{(1032.4)}
\textsuperscript{94:5.2} La primera forma de taoísmo apareció en China a consecuencia directa de esta enseñanza, pero se trataba de una religión enormemente diferente a la que lleva este nombre hoy en día. El taoísmo primitivo, o prototaoísmo, estaba compuesto de los siguientes factores:

\par
%\textsuperscript{(1032.5)}
\textsuperscript{94:5.3} 1. Las enseñanzas sobrevivientes de Singlangtón, que subsistieron en el concepto de Shang-ti, el Dios del Cielo. En los tiempos de Singlangtón, el pueblo chino se volvió prácticamente monoteísta; concentraron su adoración en la Verdad Única, conocida más tarde como el Espíritu del Cielo, el soberano del universo. La raza amarilla nunca perdió por completo este concepto inicial de la Deidad, aunque en siglos posteriores muchos dioses y espíritus subordinados se deslizaron insidiosamente dentro de su religión.

\par
%\textsuperscript{(1032.6)}
\textsuperscript{94:5.4} 2. La religión salemita de una Altísima Deidad Creadora que otorgaría su favor a la humanidad en respuesta a la fe del hombre. Pero es demasiado cierto que, en la época en que los misioneros de Melquisedek penetraron en las tierras de la raza amarilla, su mensaje original se había apartado considerablemente de las simples doctrinas de Salem de los tiempos de Maquiventa.

\par
%\textsuperscript{(1032.7)}
\textsuperscript{94:5.5} 3. El concepto del Brahmán-Absoluto de los filósofos indios, unido al deseo de escapar a todo mal. En la diseminación hacia el este de la religión de Salem, la influencia externa más importante la ejercieron quizás los instructores indios de la fe védica, que introdujeron su concepto del Brahmán ---el Absoluto--- en el pensamiento salvacionista de los salemitas.

\par
%\textsuperscript{(1033.1)}
\textsuperscript{94:5.6} Esta creencia compuesta se difundió por los países de las razas amarilla y cobriza como una influencia subyacente en el pensamiento filosófico-religioso. En el Japón, este prototaoísmo fue conocido con el nombre de sintoísmo, y los pueblos de este país, muy alejado de Salem en Palestina, se enteraron de la encarnación de Maquiventa Melquisedek, que vivió en la Tierra para que la humanidad no olvidara el nombre de Dios.

\par
%\textsuperscript{(1033.2)}
\textsuperscript{94:5.7} En China, todas estas creencias se confundieron y se mezclaron más tarde con el culto en constante crecimiento de la adoración a los antepasados. Pero desde los tiempos de Singlangtón, los chinos nunca llegaron a ser unos esclavos desamparados del clericalismo. La raza amarilla fue la primera que emergió de la esclavitud barbárica y que entró en una civilización ordenada, porque fue la primera que se liberó en cierta medida del miedo abyecto a los dioses, y ni siquiera llegó a temer a los fantasmas de los muertos como les sucedió a las otras razas. China fracasó porque no logró progresar más allá de su emancipación inicial de los sacerdotes, porque cayó en un error casi igual de calamitoso: el del culto a los antepasados.

\par
%\textsuperscript{(1033.3)}
\textsuperscript{94:5.8} Pero los salemitas no trabajaron en vano. Sobre los fundamentos de su evangelio, los grandes filósofos de la China del siglo sexto a. de J.C. construyeron sus enseñanzas. La atmósfera moral y los sentimientos espirituales de los tiempos de Lao-Tse y Confucio tuvieron su origen en las enseñanzas que los misioneros de Salem habían predicado en una época anterior.

\section*{6. Lao-Tse y Confucio}
\par
%\textsuperscript{(1033.4)}
\textsuperscript{94:6.1} Unos seiscientos años antes de la llegada de Miguel, Melquisedek, que se había ido de este mundo hacía mucho tiempo, tuvo la impresión de que la pureza de su enseñanza en la Tierra se encontraba indebidamente en peligro a causa de su absorción general por las creencias más antiguas de Urantia. Durante un tiempo pareció que su misión como precursor de Miguel podía estar en peligro de fracaso. Y en el siglo sexto antes de Cristo, gracias a una coordinación excepcional de influencias espirituales, que ni siquiera los supervisores planetarios llegan a comprender plenamente, Urantia fue testigo de una presentación sumamente inhabitual de una verdad religiosa variada. Por mediación de diversos educadores humanos, el evangelio de Salem fue expuesto de nuevo y revitalizado, y una gran parte de lo que se presentó entonces ha sobrevivido hasta la época del presente escrito.

\par
%\textsuperscript{(1033.5)}
\textsuperscript{94:6.2} Este siglo incomparable de progreso espiritual estuvo caracterizado por la aparición de grandes instructores religiosos, morales y filosóficos en todo el mundo civilizado. En China, los dos maestros sobresalientes fueron Lao-Tse y Confucio.

\par
%\textsuperscript{(1033.6)}
\textsuperscript{94:6.3} \textit{Lao-Tse} construyó directamente sobre los conceptos de las tradiciones de Salem cuando declaró que el Tao era la Única Causa Primera de toda la creación. Lao era un hombre de una gran visión espiritual. Enseñó que <<el destino eterno del hombre era la unión perpetua con el Tao, Dios Supremo y Rey Universal>>. Su comprensión de la causalidad última era muy perspicaz, ya que escribió: <<La Unidad se origina en el Tao Absoluto, de la Unidad aparece la Dualidad cósmica, de esta Dualidad brota a la existencia la Trinidad, y la Trinidad es la fuente primordial de toda la realidad>>. <<Toda la realidad está siempre en equilibrio entre los potenciales y los actuales del cosmos, y éstos son eternamente armonizados por el espíritu de la divinidad>>.

\par
%\textsuperscript{(1033.7)}
\textsuperscript{94:6.4} Lao-Tse fue también uno de los primeros que presentó la doctrina de devolver bien por mal: <<La bondad engendra la bondad, pero para aquel que es verdaderamente bueno, el mal engendra también la bondad>>.

\par
%\textsuperscript{(1033.8)}
\textsuperscript{94:6.5} Enseñó el regreso de la criatura hacia el Creador y describió la vida como el surgimiento de una personalidad a partir de los potenciales cósmicos, mientras que la muerte se parecía al regreso al hogar de la personalidad de esa criatura. Su concepto de la verdadera fe era poco común, y él también lo comparó a la <<actitud de un niño>>.

\par
%\textsuperscript{(1034.1)}
\textsuperscript{94:6.6} Su comprensión del propósito eterno de Dios era clara, ya que dijo: <<La Deidad Absoluta no lucha, pero siempre vence; no coacciona a la humanidad, pero siempre está dispuesta a responder a sus deseos sinceros; la voluntad de Dios tiene una paciencia eterna y la inevitabilidad de su expresión es eterna>>. Al expresar la verdad de que es más bienaventurado dar que recibir, Lao-Tse dijo acerca de la persona auténticamente religiosa: <<El hombre bueno no trata de retener la verdad para sí mismo, sino que intenta más bien regalar estas riquezas a sus semejantes, ya que esto es hacer realidad la verdad. La voluntad del Dios Absoluto siempre beneficia, y nunca destruye; la intención del verdadero creyente es actuar siempre, y no coaccionar nunca>>.

\par
%\textsuperscript{(1034.2)}
\textsuperscript{94:6.7} La enseñanza de Lao sobre la no resistencia, y la distinción que hizo entre la \textit{acción} y la \textit{coacción}, fueron desvirtuadas más tarde en las creencias de <<no ver, no hacer y no pensar nada>>. Pero Lao nunca enseñó este error, aunque su presentación de la no resistencia ha sido un factor en el desarrollo ulterior de la predilección de los pueblos chinos por la paz.

\par
%\textsuperscript{(1034.3)}
\textsuperscript{94:6.8} Pero el taoísmo popular de la Urantia del siglo veinte tiene muy poco en común con los sentimientos elevados y los conceptos cósmicos del viejo filósofo, que enseñó la verdad tal como la percibía, es decir, que la fe en el Dios Absoluto es la fuente de la energía divina que reconstruirá el mundo, y por medio de la cual el hombre asciende hacia la unión espiritual con el Tao, la Deidad Eterna y el Creador Absoluto de los universos.

\par
%\textsuperscript{(1034.4)}
\textsuperscript{94:6.9} \textit{Confucio} (Kung-Fu-Tze) era un joven contemporáneo de Lao en la China del siglo sexto a. de J.C. Confucio basó sus doctrinas en las mejores tradiciones morales de la larga historia de la raza amarilla, y sufrió también un poco la influencia de las tradiciones sobrevivientes de los misioneros de Salem. Su trabajo principal consistió en compilar los sabios refranes de los filósofos antiguos. Fue rechazado como instructor durante su vida, pero sus escritos y enseñanzas han ejercido desde entonces una gran influencia en China y en Japón. Confucio marcó una nueva pauta para los chamanes, ya que colocó a la moralidad en el lugar de la magia. Pero construyó demasiado bien; hizo del \textit{orden} un nuevo fetiche y estableció un respeto por la conducta de los antepasados que los chinos veneran todavía en el momento del presente escrito.

\par
%\textsuperscript{(1034.5)}
\textsuperscript{94:6.10} Confucio predicaba la moralidad basándose en la teoría de que el camino terrenal es la sombra deformada del camino celestial, de que el verdadero modelo de la civilización temporal es el reflejo del orden eterno del cielo. El concepto potencial de Dios, en el confucianismo, estaba subordinado casi por completo al énfasis que puso en el Camino del Cielo, el arquetipo del cosmos.

\par
%\textsuperscript{(1034.6)}
\textsuperscript{94:6.11} Las enseñanzas de Lao se han perdido para todos, salvo para una minoría de Oriente, pero los escritos de Confucio han constituido desde entonces la base del tejido moral de la cultura de casi un tercio de los urantianos. Estos preceptos de Confucio, aunque perpetuaban lo mejor del pasado, iban un poco en contra del mismo espíritu de investigación chino que había conseguido unos logros tan venerados. La influencia de estas doctrinas fue combatida sin éxito por los esfuerzos imperiales de Tsin-Chi-Hoang-Ti y por las enseñanzas de Mo-Ti, el cual proclamó una fraternidad basada en el amor a Dios y no en el deber ético. Trató de reanimar la antigua búsqueda de las verdades nuevas, pero sus enseñanzas fracasaron ante la vigorosa oposición de los discípulos de Confucio.

\par
%\textsuperscript{(1034.7)}
\textsuperscript{94:6.12} Al igual que otros muchos educadores espirituales y morales, Confucio y Lao-Tse fueron finalmente deificados por sus seguidores durante las edades de tinieblas espirituales que envolvieron a China entre la decadencia y la perversión de la fe taoísta, y la llegada de los misioneros budistas procedentes de la India. Durante estos siglos de decadencia espiritual, la religión de la raza amarilla degeneró en una teología lamentable donde pululaban los diablos, los dragones y los espíritus malignos, denotando todos ellos el regreso de los miedos de la mente humana poco ilustrada. Y China, en otro tiempo a la cabeza de la sociedad humana gracias a su religión avanzada, se quedó entonces atrás por su incapacidad temporal para progresar en el verdadero camino del desarrollo de esa conciencia de Dios que es indispensable para el verdadero progreso, no solamente de los mortales individuales, sino también de las civilizaciones intrincadas y complejas que caracterizan el avance de la cultura y de la sociedad en un planeta evolutivo del tiempo y el espacio.

\section*{7. Siddharta Gautama}
\par
%\textsuperscript{(1035.1)}
\textsuperscript{94:7.1} Contemporáneo de Lao-Tse y de Confucio en China, otro gran instructor de la verdad surgió en la India. Siddharta Gautama nació en el siglo sexto antes de Cristo en la provincia del Nepal, al norte de la India. Sus seguidores lo presentaron más tarde como el hijo de un gobernante fabulosamente rico, pero era en verdad el heredero forzoso al trono de un cacique sin importancia que gobernaba por consentimiento tácito un pequeño valle montañoso aislado al sur del Himalaya.

\par
%\textsuperscript{(1035.2)}
\textsuperscript{94:7.2} Después de practicar inútilmente el yoga durante seis años, Gautama formuló las teorías que se convirtieron en la filosofía del budismo. Siddharta libró una batalla decidida pero infructuosa contra el sistema creciente de las castas. Este joven príncipe profeta poseía una gran sinceridad y una generosidad extraordinaria que atraían enormemente a los hombres de aquella época. Le restó valor a la práctica de buscar la salvación individual por medio de la aflicción física y del sufrimiento personal, y exhortó a sus seguidores a que llevaran su evangelio por todo el mundo.

\par
%\textsuperscript{(1035.3)}
\textsuperscript{94:7.3} Las enseñanzas más sensatas y más moderadas de Gautama llegaron como un alivio refrescante en medio de la confusión y las prácticas cultuales extremas de la India. Denunció a los dioses, a los sacerdotes y a sus sacrificios, pero él tampoco logró percibir la \textit{personalidad} del Uno Universal. Puesto que no creía en la existencia de las almas humanas individuales, Gautama luchó valientemente, por supuesto, contra la creencia consagrada por la tradición en la transmigración de las almas. Hizo un noble esfuerzo por liberar a los hombres del miedo, por hacer que se sintieran cómodos y a gusto en el gran universo, pero no logró mostrarles el camino hacia el auténtico hogar celestial de los mortales ascendentes ---el Paraíso--- y hacia el servicio creciente de la existencia eterna.

\par
%\textsuperscript{(1035.4)}
\textsuperscript{94:7.4} Gautama era un verdadero profeta, y si hubiera hecho caso de la enseñanza del ermitaño Godad, podría haber despertado a toda la India gracias a la inspiración que hubiera aportado el restablecimiento del evangelio de Salem consistente en la salvación por medio de la fe. Godad descendía de una familia que nunca había perdido las tradiciones de los misioneros de Melquisedek.

\par
%\textsuperscript{(1035.5)}
\textsuperscript{94:7.5} Gautama fundó su escuela en Benarés, y durante su segundo año de funcionamiento, un alumno llamado Baután comunicó a su maestro las tradiciones de los misioneros de Salem acerca de la alianza de Melquisedek con Abraham. Aunque Siddharta no tenía un concepto muy claro del Padre Universal, adoptó una actitud avanzada en lo referente a la salvación por medio de la fe ---de la simple creencia. Así lo declaró ante sus seguidores, y empezó a enviar a sus discípulos en grupos de sesenta para que proclamaran a los habitantes de la India <<la buena nueva de la salvación gratuita; que todos los hombres, de todas las clases, pueden alcanzar la felicidad gracias a la fe en la rectitud y la justicia>>.

\par
%\textsuperscript{(1035.6)}
\textsuperscript{94:7.6} La esposa de Gautama creía en el evangelio de su marido y fue la fundadora de una orden de monjas. Su hijo se convirtió en su sucesor y difundió mucho el culto; captó la nueva idea de la salvación por la fe, pero en sus últimos años vaciló ante el evangelio de Salem que prometía el favor divino a cambio únicamente de la fe, y en su vejez, las últimas palabras que pronunció antes de morir fueron: <<Elaborad vuestra propia salvación>>.

\par
%\textsuperscript{(1036.1)}
\textsuperscript{94:7.7} Cuando fue proclamado en su mejor momento, el evangelio de la salvación universal enseñado por Gautama, exento de sacrificios, torturas, rituales y sacerdotes, fue una doctrina revolucionaria y asombrosa para su tiempo. Estuvo sorprendentemente cerca de convertirse en un renacimiento del evangelio de Salem. Ayudó a millones de almas desesperadas, y a pesar de la grotesca desnaturalización que sufrió durante los siglos posteriores, sigue siendo todavía la esperanza de millones de seres humanos.

\par
%\textsuperscript{(1036.2)}
\textsuperscript{94:7.8} Siddharta enseñó muchas más verdades de las que han sobrevivido en los cultos modernos que llevan su nombre. El budismo moderno no refleja las enseñanzas de Siddharta Gautama mucho más de lo que el cristianismo lo hace con las enseñanzas de Jesús de Nazaret.

\section*{8. La fe budista}
\par
%\textsuperscript{(1036.3)}
\textsuperscript{94:8.1} Para hacerse budista, uno simplemente hacía una profesión pública de fe recitando el Refugio: <<Me refugio en el Buda; me refugio en la Doctrina; me refugio en la Fraternidad>>.

\par
%\textsuperscript{(1036.4)}
\textsuperscript{94:8.2} El budismo tuvo su origen en una personalidad histórica, no en un mito. Los seguidores de Gautama lo llamaban Sasta, que significaba maestro o instructor. Aunque no manifestó ninguna pretensión superhumana ni para él mismo ni para sus enseñanzas, sus discípulos empezaron pronto a llamarle \textit{el iluminado}, el Buda, y más tarde Sakya-Muni Buda.

\par
%\textsuperscript{(1036.5)}
\textsuperscript{94:8.3} El evangelio original de Gautama estaba basado en cuatro nobles verdades:

\par
%\textsuperscript{(1036.6)}
\textsuperscript{94:8.4} 1. Las nobles verdades del sufrimiento.

\par
%\textsuperscript{(1036.7)}
\textsuperscript{94:8.5} 2. Los orígenes del sufrimiento.

\par
%\textsuperscript{(1036.8)}
\textsuperscript{94:8.6} 3. La destrucción del sufrimiento.

\par
%\textsuperscript{(1036.9)}
\textsuperscript{94:8.7} 4. El camino para destruir el sufrimiento.

\par
%\textsuperscript{(1036.10)}
\textsuperscript{94:8.8} La filosofía del Sendero Óctuple estaba estrechamente vinculada a la doctrina del sufrimiento y a la manera de eludirlo: opiniones justas, aspiraciones justas, palabras justas, conducta justa, sustento justo, esfuerzo justo, atención justa y contemplación justa. Gautama no tenía la intención de intentar destruir todo esfuerzo, deseo y afecto mediante el acto de eludir el sufrimiento; su enseñanza estaba destinada más bien a describir al hombre mortal la futilidad de poner todas sus esperanzas y aspiraciones en las metas temporales y los objetivos materiales. No se trataba tanto de evitar amar a sus semejantes como de que el verdadero creyente debía mirar también más allá de las asociaciones de este mundo material, hacia las realidades del futuro eterno.

\par
%\textsuperscript{(1036.11)}
\textsuperscript{94:8.9} Los mandamientos morales de los sermones de Gautama eran cinco:

\par
%\textsuperscript{(1036.12)}
\textsuperscript{94:8.10} 1. No matarás.

\par
%\textsuperscript{(1036.13)}
\textsuperscript{94:8.11} 2. No robarás.

\par
%\textsuperscript{(1036.14)}
\textsuperscript{94:8.12} 3. No serás impúdico.

\par
%\textsuperscript{(1036.15)}
\textsuperscript{94:8.13} 4. No mentirás.

\par
%\textsuperscript{(1036.16)}
\textsuperscript{94:8.14} 5. No beberás bebidas embriagadoras.

\par
%\textsuperscript{(1036.17)}
\textsuperscript{94:8.15} Había diversos mandamientos adicionales o secundarios cuyo cumplimiento era facultativo para los creyentes.

\par
%\textsuperscript{(1036.18)}
\textsuperscript{94:8.16} Siddharta apenas creía en la inmortalidad de la personalidad humana; su filosofía sólo preveía una especie de continuidad funcional. Nunca definió claramente qué es lo que se proponía incluir en la doctrina del Nirvana. El hecho de que se pudiera experimentar teóricamente durante la existencia mortal indicaría que el nirvana no era considerado como un estado de aniquilación completa. Implicaba un estado de iluminación suprema y de felicidad celestial, en el que todas las cadenas que ataban al hombre al mundo material se habían roto; uno se sentía libre de los deseos de la vida mortal y liberado de todo peligro de tener que experimentar una nueva encarnación.

\par
%\textsuperscript{(1037.1)}
\textsuperscript{94:8.17} Según las enseñanzas originales de Gautama, la salvación se consigue con el esfuerzo humano, independientemente de la ayuda divina; no hay lugar ni para la fe salvadora ni para las oraciones a los poderes superhumanos. En su intento por minimizar las supersticiones de la India, Gautama se esforzó por desviar a los hombres de las llamativas afirmaciones de una salvación milagrosa. Pero al hacer este esfuerzo, dejó la puerta totalmente abierta para que sus sucesores malinterpretaran su enseñanza y proclamaran que todos los esfuerzos humanos por conseguir algo son desagradables y dolorosos. Sus seguidores pasaron por alto el hecho de que la felicidad suprema está unida a la persecución inteligente y entusiasta de unas metas nobles, y que estos logros constituyen un verdadero progreso en la autorrealización cósmica.

\par
%\textsuperscript{(1037.2)}
\textsuperscript{94:8.18} La gran verdad de la enseñanza de Siddharta fue su proclamación de un universo de justicia absoluta. Enseñó la mejor filosofía atea que un hombre mortal haya inventado jamás; era el humanismo ideal, y eliminó muy eficazmente todas las razones para las supersticiones, los rituales mágicos y el miedo a los fantasmas o los demonios.

\par
%\textsuperscript{(1037.3)}
\textsuperscript{94:8.19} La gran debilidad del evangelio original del budismo consistió en que no engendró una religión de servicio social desinteresado. La fraternidad budista no fue, durante mucho tiempo, una hermandad de creyentes, sino más bien una comunidad de instructores estudiosos. Gautama les prohibió que recibieran dinero y de esta manera intentó impedir el desarrollo de tendencias jerárquicas. Gautama mismo era extremadamente sociable; su vida fue en verdad mucho más grande que su predicación.

\section*{9. La difusión del budismo}
\par
%\textsuperscript{(1037.4)}
\textsuperscript{94:9.1} El budismo prosperó porque ofrecía la salvación a través de la creencia en Buda, el iluminado. Era más representativo de las verdades de Melquisedek que cualquier otro sistema religioso que se pudiera encontrar en toda Asia oriental. Pero el budismo no se difundió mucho como religión hasta que un monarca de baja casta, Asoka, lo adoptó para protegerse a sí mismo; después de Akenatón en Egipto, Asoka fue uno de los gobernantes civiles más notables entre la época de Melquisedek y la de Miguel. Asoka construyó un gran imperio indio gracias a la propaganda de sus misioneros budistas. Durante un período de veinticinco años educó a más de diecisiete mil misioneros y los envió hasta las fronteras más alejadas de todo el mundo conocido. En una sola generación hizo del budismo la religión dominante de la mitad del mundo. Ésta se asentó pronto en el Tíbet, Cachemira, Ceilán, Birmania, Java, Siam, Corea, China y Japón. En términos generales, era una religión enormemente superior a aquellas que sustituyó o mejoró.

\par
%\textsuperscript{(1037.5)}
\textsuperscript{94:9.2} La difusión del budismo desde su tierra natal en la India hacia toda Asia es una de las historias más emocionantes de la devoción espiritual y la perseverancia misionera de unas personas religiosas sinceras. Los instructores del evangelio de Gautama no solamente desafiaron los riesgos de las rutas de las caravanas por tierra, sino que se enfrentaron a los peligros de los mares de China mientras proseguían su misión en el continente asiático, llevando a todos los pueblos el mensaje de su fe. Pero este budismo ya no era la simple doctrina de Gautama; era un evangelio lleno de milagros que hacía de Siddharta un dios. Y a medida que el budismo se alejaba más de su hogar en las tierras altas de la India, más distinto se volvía de las enseñanzas de Gautama, y más se parecía a las religiones que reemplazaba.

\par
%\textsuperscript{(1038.1)}
\textsuperscript{94:9.3} Más tarde, el taoísmo en China, el sintoísmo en Japón y el cristianismo en el Tíbet afectaron mucho al budismo. En la India, después de mil años, el budismo simplemente se marchitó y expiró. Se brahmanizó y más tarde se rindió servilmente ante el islam, mientras que en una gran parte del resto de oriente degeneró en un ritual que Siddharta Gautama no hubiera reconocido nunca.

\par
%\textsuperscript{(1038.2)}
\textsuperscript{94:9.4} En el sur, el estereotipo fundamentalista de las enseñanzas de Siddharta sobrevivió en Ceilán, Birmania y en la península de Indochina. Ésta es la rama hinayana del budismo, que se aferra a la doctrina primitiva o asocial.

\par
%\textsuperscript{(1038.3)}
\textsuperscript{94:9.5} Pero incluso antes de su derrumbamiento en la India, los grupos de seguidores de Gautama del norte de la India y de China habían empezado a desarrollar la enseñanza mahayana del <<Camino Mayor>> hacia la salvación, en contraste con los puristas del sur que se aferraban al hinayana o <<Camino Menor>>. Estos mahayanistas se liberaron de las limitaciones sociales inherentes a la doctrina budista, y esta rama septentrional del budismo ha continuado evolucionando desde entonces en China y en Japón.

\par
%\textsuperscript{(1038.4)}
\textsuperscript{94:9.6} El budismo es hoy una religión viviente y creciente porque consigue conservar una gran parte de los valores morales más elevados de sus adeptos. Fomenta la calma y el dominio de sí mismo, aumenta la serenidad y la felicidad, y contribuye mucho a impedir la tristeza y la aflicción. Aquellos que creen en esta filosofía viven una vida mejor que muchos de los que no creen en ella.

\section*{10. La religión en el Tíbet}
\par
%\textsuperscript{(1038.5)}
\textsuperscript{94:10.1} En el Tíbet se puede encontrar la asociación más extraña de las enseñanzas de Melquisedek combinadas con el budismo, el hinduismo, el taoísmo y el cristianismo. Cuando los misioneros budistas entraron en el Tíbet, encontraron un estado de salvajismo primitivo muy similar a aquel que hallaron los primeros misioneros cristianos en las tribus nórdicas de Europa.

\par
%\textsuperscript{(1038.6)}
\textsuperscript{94:10.2} Estos tibetanos sencillos no querían renunciar íntegramente a su antigua magia ni a sus amuletos. El examen de las ceremonias religiosas de los rituales tibetanos de hoy en día revela la existencia de una cofradía excesivamente numerosa de sacerdotes con la cabeza rapada, que practican un ritual detallado que abarca campanas, cantos, incienso, procesiones, rosarios, imágenes, amuletos, pinturas, agua bendita, vestiduras magníficas y coros primorosos. Poseen dogmas rígidos y credos cristalizados, ritos místicos y ayunos especiales. Su jerarquía contiene monjes, monjas, abades y el Gran Lama. Rezan a los ángeles, a los santos, a una Madre Sagrada y a los dioses. Practican la confesión y creen en el purgatorio. Sus monasterios son enormes y sus catedrales magníficas. Mantienen una repetición interminable de rituales sagrados y creen que estas ceremonias confieren la salvación. Clavan sus oraciones en una rueda, y creen que cuando ésta gira sus súplicas se vuelven eficaces. En ningún otro pueblo de los tiempos modernos se puede encontrar la observancia de tantas cosas provenientes de tantas religiones; y es inevitable que esta liturgia acumulada se vuelva excesivamente incómoda e intolerablemente pesada.

\par
%\textsuperscript{(1038.7)}
\textsuperscript{94:10.3} Los tibetanos poseen alguna cosa de todas las religiones principales del mundo, excepto las simples enseñanzas del evangelio de Jesús: la filiación con Dios, la fraternidad entre los hombres y la ciudadanía siempre ascendente en el universo eterno.

\section*{11. La filosofía budista}
\par
%\textsuperscript{(1038.8)}
\textsuperscript{94:11.1} El budismo penetró en China en el primer milenio después de Cristo, y se adaptó bien a las costumbres religiosas de la raza amarilla. En su culto a los antepasados, habían dirigido sus oraciones durante mucho tiempo a los muertos; ahora también podían rezar por ellos. El budismo pronto se fusionó con las prácticas ritualistas sobrevivientes del taoísmo en desintegración. Esta nueva religión sintética, con sus templos para la adoración y un ceremonial religioso definido, pronto se convirtió en el culto generalmente aceptado por los pueblos de China, Corea y Japón.

\par
%\textsuperscript{(1039.1)}
\textsuperscript{94:11.2} En algunos aspectos, es lamentable que el budismo no fuera enseñado al mundo hasta después de que los seguidores de Gautama hubieron desvirtuado tanto las tradiciones y las enseñanzas del culto, que habían hecho de Siddharta un ser divino. Sin embargo este mito de su vida humana, embellecido como lo fue por una multitud de milagros, resultó muy atractivo para los oyentes del evangelio nórdico, o mahayana, del budismo.

\par
%\textsuperscript{(1039.2)}
\textsuperscript{94:11.3} Algunos de sus seguidores posteriores enseñaron que el espíritu de Sakya-Muni Buda regresaba periódicamente a la Tierra como Buda viviente, abriendo así el camino a una perpetuación indefinida de imágenes de Buda, templos, rituales y falsos <<Budas vivientes>>. Así es como la religión del gran protestatario indio se encontró finalmente encadenada a las mismas prácticas ceremoniales y conjuros ritualistas contra los que había luchado tan audazmente y que tan valientemente había denunciado.

\par
%\textsuperscript{(1039.3)}
\textsuperscript{94:11.4} El gran progreso que aportó la filosofía budista consistió en comprender que toda verdad es relativa. A través del mecanismo de esta hipótesis, los budistas han sido capaces de conciliar y correlacionar las divergencias internas de sus propias escrituras religiosas, así como las diferencias entre las suyas y muchas otras. Se enseñaba que las verdades pequeñas eran para las mentes limitadas, y las grandes verdades para las mentes sobresalientes.

\par
%\textsuperscript{(1039.4)}
\textsuperscript{94:11.5} Esta filosofía sostenía también que la naturaleza (divina) de Buda residía en todos los hombres; que el hombre, por medio de sus propios esfuerzos, podía alcanzar la comprensión de esta divinidad interior. Esta enseñanza es una de las presentaciones más claras de la verdad acerca de los Ajustadores internos que ninguna otra religión de Urantia haya realizado jamás.

\par
%\textsuperscript{(1039.5)}
\textsuperscript{94:11.6} Pero el evangelio original de Siddharta, tal como lo interpretaban sus seguidores, comportaba una gran limitación, ya que intentaba liberar completamente al yo humano de todas las limitaciones de la naturaleza mortal a través de la técnica de aislar al yo de la realidad objetiva. La auténtica autorrealización cósmica se obtiene como resultado de la identificación del yo con la realidad cósmica y con el cosmos finito de energía, mente y espíritu, limitado por el espacio y condicionado por el tiempo.

\par
%\textsuperscript{(1039.6)}
\textsuperscript{94:11.7} Aunque las ceremonias y las prácticas exteriores del budismo se contaminaron terriblemente con las de los países por los que viajaron, esta degeneración no tuvo plenamente lugar en la vida filosófica de los grandes pensadores que, de vez en cuando, abrazaron este sistema de pensamiento y creencia. Durante más de dos mil años, muchos de los mejores cerebros de Asia se han concentrado en el problema de averiguar la verdad absoluta y la verdad del Absoluto.

\par
%\textsuperscript{(1039.7)}
\textsuperscript{94:11.8} La evolución de un concepto elevado del Absoluto se consiguió a través de muchos canales de pensamiento y por medio de tortuosos caminos de razonamiento. El proceso ascendente de esta doctrina de la infinidad no estaba tan claramente definido como la evolución del concepto de Dios en la teología hebrea. Sin embargo, las inteligencias budistas alcanzaron ciertos niveles extensos, se detuvieron en ellos, y los atravesaron en su camino hacia la concepción de la Fuente Primordial de los universos:

\par
%\textsuperscript{(1039.8)}
\textsuperscript{94:11.9} 1. \textit{La leyenda de Gautama}. En la base del concepto se encontraba el hecho histórico de la vida y las enseñanzas de Siddharta, el príncipe profeta de la India. Esta leyenda se convirtió en mito a medida que viajó a través de los siglos y por los extensos países de Asia, hasta que sobrepasó el nivel de la idea de Gautama como iluminado y empezó a recibir atributos adicionales.

\par
%\textsuperscript{(1040.1)}
\textsuperscript{94:11.10} 2. \textit{Los numerosos Budas}. Se razonaba que, si Gautama había venido a los pueblos de la India, entonces las razas de la humanidad habían sido bendecidas en el lejano pasado con otros instructores de la verdad, y lo serían de nuevo indudablemente en el lejano futuro. Esto dio origen a la enseñanza de que había muchos Budas, un número ilimitado e infinito, e incluso que cualquiera podía aspirar a ser uno de ellos ---a alcanzar la divinidad de un Buda.

\par
%\textsuperscript{(1040.2)}
\textsuperscript{94:11.11} 3. \textit{El Buda Absoluto}. En el momento en que se creyó que el número de Budas se acercaba a la infinidad, las mentes de aquella época tuvieron necesidad de reunificar este concepto difícil de manejar. Por consiguiente, se empezó a enseñar que todos los Budas no eran más que la manifestación de alguna esencia superior, de algún Uno Eterno con una existencia infinita e incondicional, de alguna Fuente Absoluta de toda la realidad. A partir de entonces el concepto budista de la Deidad, en su forma más elevada, quedó separado de la persona humana de Siddharta Gautama, y se liberó de las limitaciones antropomórficas que lo habían mantenido atado. Esta concepción final del Buda Eterno se puede identificar muy bien con el Absoluto, y a veces incluso con el infinito YO SOY.

\par
%\textsuperscript{(1040.3)}
\textsuperscript{94:11.12} Aunque esta idea de la Deidad Absoluta nunca encontró un gran favor popular entre los pueblos de Asia, permitió que los intelectuales de estos países unificaran su filosofía y armonizaran su cosmología. El concepto del Buda Absoluto es a veces casi personal, a veces totalmente impersonal ---e incluso una fuerza creadora infinita. Aunque estos conceptos son útiles para la filosofía, no son vitales para el desarrollo religioso. Incluso un Yahvé antropomórfico tiene un valor religioso mucho mayor que el Absoluto infinitamente lejano del budismo o del brahmanismo.

\par
%\textsuperscript{(1040.4)}
\textsuperscript{94:11.13} A veces se llegó incluso a pensar que el Absoluto estaba contenido dentro del infinito YO SOY. Pero estas especulaciones aportaban un frío consuelo a las multitudes hambrientas que anhelaban escuchar palabras de promesa, escuchar el simple evangelio de Salem anunciando que la fe en Dios aseguraba el favor divino y la supervivencia eterna.

\section*{12. El concepto de Dios en el budismo}
\par
%\textsuperscript{(1040.5)}
\textsuperscript{94:12.1} La gran debilidad de la cosmología del budismo era doble: se había contaminado con numerosas supersticiones de la India y China, y había sublimado a Gautama, primero como iluminado y luego como Buda Eterno. De la misma manera que el cristianismo ha padecido la absorción de mucha filosofía humana errónea, el budismo lleva también su marca de nacimiento humana. Pero las enseñanzas de Gautama han continuado evolucionando durante los últimos dos mil quinientos años. Para un budista instruido, el concepto de Buda ya no es lo mismo que la personalidad humana de Gautama, al igual que para un cristiano instruido el concepto de Jehová tampoco es idéntico al espíritu demoníaco del Horeb. La escasez de terminología, unida a la conservación sentimental de una nomenclatura antigua, a menudo impide comprender el verdadero significado de la evolución de los conceptos religiosos.

\par
%\textsuperscript{(1040.6)}
\textsuperscript{94:12.2} El concepto de Dios, en contraste con el del Absoluto, empezó a aparecer gradualmente en el budismo. Sus orígenes se remontan a los primeros tiempos en que los seguidores del Camino Menor se diferenciaron de los del Camino Mayor. En esta última rama del budismo fue donde la doble concepción de Dios y del Absoluto terminó por madurar. El concepto de Dios ha evolucionado paso a paso y siglo tras siglo hasta que gracias a las enseñanzas de Ryonin, Honen Shonin y Shinran en el Japón, este concepto fructificó finalmente en la creencia en Amida Buda.

\par
%\textsuperscript{(1041.1)}
\textsuperscript{94:12.3} Entre estos creyentes se enseña que el alma, después de pasar por la muerte, puede elegir disfrutar de una estancia en el Paraíso antes de entrar en el Nirvana, la existencia definitiva. Proclaman que esta nueva salvación se consigue por la fe en las misericordias divinas y en los cuidados amorosos de Amida, el Dios del Paraíso en occidente. En su filosofía, los amidistas creen en una Realidad Infinita que está más allá de toda comprensión mortal finita; en su religión, se aferran a la fe en un Amida totalmente misericordioso que ama tanto al mundo, que no puede tolerar que un solo mortal que invoque su nombre con una fe sincera y un corazón puro, deje de conseguir la felicidad celestial del Paraíso.

\par
%\textsuperscript{(1041.2)}
\textsuperscript{94:12.4} La gran fuerza del budismo reside en que sus adeptos son libres de escoger la verdad en todas las religiones; esta libertad de elección ha caracterizado raras veces a una doctrina urantiana. A este respecto, la secta Shin del Japón se ha convertido en uno de los grupos religiosos más progresivos del mundo; ha reanimado el antiguo espíritu misionero de los seguidores de Gautama, y ha empezado a enviar educadores a otros pueblos. Esta buena disposición a apropiarse de la verdad, cualquiera que sea la fuente de donde proceda, es una tendencia realmente recomendable que aparece entre los creyentes religiosos de la primera mitad del siglo veinte después de Cristo.

\par
%\textsuperscript{(1041.3)}
\textsuperscript{94:12.5} El budismo mismo está experimentando un renacimiento en el siglo veinte. Debido a su contacto con el cristianismo, los aspectos sociales del budismo han mejorado enormemente. El deseo de aprender se ha vuelto a encender en el corazón de los monjes sacerdotes de la hermandad, y la difusión de la educación en toda esta comunidad doctrinal provocará indudablemente nuevos progresos en la evolución religiosa.

\par
%\textsuperscript{(1041.4)}
\textsuperscript{94:12.6} En el momento en que escribo estas líneas, una gran parte de Asia tiene puestas sus esperanzas en el budismo. Esta noble fe, que ha atravesado tan valientemente las edades de las tinieblas del pasado, ¿sabrá recibir de nuevo la verdad de unas realidades cósmicas más amplias, tal como los discípulos del gran instructor de la India escucharon en otro tiempo su proclamación de una verdad nueva? Esta antigua fe, ¿responderá una vez más al estímulo vigorizante de la presentación de unos nuevos conceptos de Dios y del Absoluto que ha buscado durante tanto tiempo?

\par
%\textsuperscript{(1041.5)}
\textsuperscript{94:12.7} Toda Urantia está esperando la proclamación del mensaje ennoblecedor de Miguel, sin las trabas de las doctrinas y los dogmas acumulados durante diecinueve siglos de contacto con las religiones de origen evolutivo. Ha llegado la hora de presentar al budismo, al cristianismo, al hinduismo, e incluso a los pueblos de todas las religiones, no el evangelio acerca de Jesús, sino la realidad viviente y espiritual del evangelio de Jesús.

\par
%\textsuperscript{(1041.6)}
\textsuperscript{94:12.8} [Presentado por un Melquisedek de Nebadon.]


\chapter{Documento 95. Las enseñanzas de Melquisedek en el Levante}
\par
%\textsuperscript{(1042.1)}
\textsuperscript{95:0.1} AL IGUAL que la India dio origen a muchas religiones y filosofías de Asia oriental, el Levante fue la cuna de las creencias del mundo occidental. Los misioneros de Salem se desparramaron por todo el suroeste de Asia, a través de Palestina, Mesopotamia, Egipto, Irán y Arabia, proclamando por todas partes la buena nueva del evangelio de Maquiventa Melquisedek. En algunos de estos países sus enseñanzas dieron frutos; en otros tuvieron un éxito variable. Sus fracasos se debieron a veces a una falta de sabiduría, y otras veces a circunstancias que estaban más allá de su control.

\section*{1. La religión de Salem en Mesopotamia}
\par
%\textsuperscript{(1042.2)}
\textsuperscript{95:1.1} Hacia el año 2000 a. de J. C., las religiones de Mesopotamia casi habían perdido las enseñanzas de los setitas, y se encontraban ampliamente bajo la influencia de las creencias primitivas de dos grupos de invasores: los beduinos semitas que se habían infiltrado desde el desierto occidental, y los jinetes bárbaros que habían descendido desde el norte.

\par
%\textsuperscript{(1042.3)}
\textsuperscript{95:1.2} Pero la costumbre que tenían los primeros pueblos adamitas de honrar el séptimo día de la semana nuna desapareció por completo en Mesopotamia. Sólo que, durante la era de Melquisedek, el séptimo día era considerado como el de mayor mala suerte. Estaba dominado por los tabúes; durante este nefasto séptimo día era ilegal partir de viaje, cocinar alimentos o hacer fuego. Los judíos trajeron de vuelta a Palestina un gran número de tabúes mesopotámicos que habían encontrado en Babilonia y que estaban basados en la observancia del séptimo día, el sabatum \footnote{\textit{Costumbre de guardar el sábado}: Ex 16:23-26; 20:8-11; 31:14-17; 35:2-3; Lv 23:3; Dt 5:12-15.}.

\par
%\textsuperscript{(1042.4)}
\textsuperscript{95:1.3} Aunque los educadores de Salem contribuyeron mucho a refinar y elevar las religiones de Mesopotamia, no consiguieron que los diversos pueblos reconocieran de manera permanente a un Dios único. Estas enseñanzas conservaron la supremacía durante más de ciento cincuenta años, y luego cedieron el paso gradualmente a la creencia más antigua en una multiplicidad de deidades.

\par
%\textsuperscript{(1042.5)}
\textsuperscript{95:1.4} Los educadores de Salem redujeron enormemente el número de dioses de Mesopotamia, y en cierto momento limitaron las principales deidades a siete: Belo\footnote{\textit{Contra Bel (Belo)}: Jer 51:44; Bel 1:1-22.}, Samas, Nabu\footnote{\textit{Contra Bel y Nabu (Nebo)}: Is 46:1.}, Anu, Ea, Marduc\footnote{\textit{Contra Belo y Marduk}: Jer 50:2.} y Sin. En el apogeo de la nueva enseñanza ensalzaron a tres de estos dioses por encima de todos los demás, la tríada babilónica compuesta por Belo, Ea y Anu, los dioses de la tierra, del mar y del cielo. Otras tríadas surgieron también en diferentes localidades; todas ellas evocaban las enseñanzas trinitarias de los anditas y los sumerios, y estaban basadas en la creencia de los salemitas en la insignia de los tres círculos de Melquisedek.

\par
%\textsuperscript{(1042.6)}
\textsuperscript{95:1.5} Los educadores de Salem nunca vencieron totalmente la popularidad de Istar, madre de los dioses y espíritu de la fertilidad sexual. Hicieron mucho por refinar la adoración de esta diosa, pero los babilonios y sus vecinos nunca habían perdido por completo sus formas disfrazadas de adoración del sexo. En toda Mesopotamia se había establecido la práctica universal de que todas las mujeres se sometieran, al menos una vez en su juventud, al abrazo de un desconocido; se pensaba que esto era una devoción exigida por Istar, y se creía que la fertilidad dependía en gran parte de este sacrificio sexual.

\par
%\textsuperscript{(1043.1)}
\textsuperscript{95:1.6} Los primeros progresos de la enseñanza de Melquisedek fueron muy satisfactorios hasta que Nabodad, el jefe de la escuela de Kish, decidió lanzar un ataque concertado contra las prácticas predominantes de la prostitución en los templos. Pero los misioneros de Salem fracasaron en su esfuerzo por llevar a cabo esta reforma social, y todas sus enseñanzas espirituales y filosóficas más importantes sucumbieron en este naufragio.

\par
%\textsuperscript{(1043.2)}
\textsuperscript{95:1.7} Este fracaso del evangelio de Salem fue seguido inmediatamente por un gran incremento del culto a Istar, un ritual que ya había invadido Palestina con el nombre de Astaroth\footnote{\textit{Astaroth}: 1 Re 11:5,33; 2 Re 23:13; Jue 2:13; 10:6; 1 Sam 7:3-4; 12:10. \textit{Contra Baal y Ashtaroth}: Jue 2:13; 10:6; 1 Sam 12:10; 7:3-4.}, Egipto con el de Isis, Grecia con el de Afrodita y las tribus del norte con el de Astarté. En conexión con este renacimiento de la adoración de Istar, los sacerdotes babilónicos volvieron otra vez a la observación de las estrellas; la astrología experimentó su último gran renacimiento en Mesopotamia, los adivinos se pusieron de moda, y el clero degeneró durante siglos cada vez más.

\par
%\textsuperscript{(1043.3)}
\textsuperscript{95:1.8} Melquisedek había advertido a sus seguidores que enseñaran la doctrina de un solo Dios, el Padre y Creador de todos, y que se limitaran a predicar el evangelio de la obtención del favor divino a través de la fe sola. Pero los instructores de una nueva verdad han cometido a menudo el error de intentar abarcar demasiado, de intentar sustituir la lenta evolución por la revolución repentina. Los misioneros de Melquisedek en Mesopotamia propusieron un nivel moral demasiado elevado para el pueblo; intentaron abarcar demasiado, y su noble causa terminó en el fracaso. Les habían encargado que predicaran un evangelio concreto, que proclamaran la verdad de la realidad del Padre Universal, pero se enredaron en la causa aparentemente meritoria de reformar las costumbres, y su gran misión fue así dejada de lado, perdiéndose prácticamente en la frustración y el olvido.

\par
%\textsuperscript{(1043.4)}
\textsuperscript{95:1.9} La sede central de Salem en Kish llegó a su fin en una sola generación, y la propaganda a favor de la creencia en un solo Dios dejó prácticamente de existir en toda Mesopotamia. Sin embargo, los vestigios de las escuelas de Salem sobrevivieron. Pequeños grupos dispersos aquí y allá continuaron creyendo en un solo Creador y lucharon contra la idolatría y la inmoralidad de los sacerdotes mesopotámicos.

\par
%\textsuperscript{(1043.5)}
\textsuperscript{95:1.10} Los misioneros salemitas del período siguiente al rechazo de sus enseñanzas fueron los que escribieron un gran número de salmos del Antiguo Testamento, grabándolos en las piedras, donde los sacerdotes hebreos posteriores los encontraron durante la cautividad y los incorporaron más tarde en la colección de himnos atribuídos a autores judíos. Estos hermosos salmos de Babilonia no fueron escritos en los templos de Belo-Marduc; fueron obra de los descendientes de los primeros misioneros salemitas, y ofrecen un contraste notable con las colecciones mágicas de los sacerdotes babilónicos. El libro de Job es un reflejo bastante bueno de las enseñanzas de la escuela salemita de Kish y de toda Mesopotamia.

\par
%\textsuperscript{(1043.6)}
\textsuperscript{95:1.11} Una gran parte de la cultura religiosa mesopotámica consiguió entrar en la literatura y la liturgia hebreas pasando por Egipto y gracias al trabajo de Amenemope y Akenatón. Los egipcios conservaron extraordinariamente bien las enseñanzas sobre las obligaciones sociales procedentes de los primeros mesopotámicos anditas, unas enseñanzas que los babilonios posteriores que ocuparon el valle del Éufrates habían perdido ampliamente.

\section*{2. La religión egipcia primitiva}
\par
%\textsuperscript{(1043.7)}
\textsuperscript{95:2.1} Las enseñanzas originales de Melquisedek echaron realmente sus raíces más profundas en Egipto, y desde allí se extendieron posteriormente hacia Europa. La religión evolutiva del valle del Nilo creció periódicamente debido a la llegada de linajes superiores de noditas, adamitas y de pueblos anditas más tardíos procedentes del valle del Éufrates. Muchos administradores civiles egipcios fueron de vez en cuando sumerios. Al igual que la India de aquellos tiempos albergaba la mayor mezcla de razas del mundo, Egipto favoreció el tipo de filosofía religiosa más completamente mezclado que se haya podido encontrar en Urantia, y desde el valle del Nilo se extendió hacia numerosas partes del mundo. Los judíos recibieron de los babilonios una gran parte de sus ideas sobre la creación del mundo, pero el concepto de la Providencia divina lo obtuvieron de los egipcios.

\par
%\textsuperscript{(1044.1)}
\textsuperscript{95:2.2} Las tendencias políticas y morales, en lugar de las inclinaciones filosóficas o religiosas, fueron las que hicieron que Egipto resultara más favorable que Mesopotamia para las enseñanzas de Salem. Cada jefe tribal de Egipto, después de luchar para conseguir el trono, trataba de perpetuar su dinastía proclamando que su dios tribal era la deidad original y el creador de todos los demás dioses. De esta manera, los egipcios se acostumbraron gradualmente a la idea de un superdios, que sirvió de trampolín para la doctrina posterior de una Deidad creadora universal. La idea del monoteísmo se tambaleó de acá para allá en Egipto durante muchos siglos; la creencia en un solo Dios siempre ganó terreno, pero nunca dominó por completo los conceptos evolutivos del politeísmo.

\par
%\textsuperscript{(1044.2)}
\textsuperscript{95:2.3} Los pueblos egipcios se habían dedicado durante miles de años a la adoración de los dioses de la naturaleza; cada una de las cuarenta tribus diferentes tenía más específicamente un dios especial para su grupo: una adoraba al toro, otra al león, una tercera al carnero, y así sucesivamente. Anteriormente habían sido unas tribus con tótemes, muy semejantes a los amerindios.

\par
%\textsuperscript{(1044.3)}
\textsuperscript{95:2.4} Los egipcios observaron con el tiempo que los cadáveres colocados en las tumbas sin ladrillos permanecían conservados ---embalsamados--- por la acción de la arena impregnada de sosa, mientras que los que estaban enterrados en bóvedas de ladrillos se descomponían. Estas observaciones condujeron a los experimentos que dieron como resultado la práctica posterior de embalsamar a los muertos. Los egipcios creían que la conservación del cuerpo facilitaba la travesía de la vida futura. Para que el individuo pudiera ser adecuadamente identificado en el futuro lejano después de la descomposición del cuerpo, colocaban una estatua fúnebre en la tumba al lado del cadáver, y esculpían un retrato en el ataúd. La confección de estas estatuas fúnebres condujo a una gran mejora del arte egipcio.

\par
%\textsuperscript{(1044.4)}
\textsuperscript{95:2.5} Durante siglos, los egipcios pusieron su confianza en las tumbas para salvaguardar los cuerpos y la consiguiente supervivencia agradable después de la muerte. La evolución posterior de las prácticas mágicas, aunque fueron incómodas para la vida desde la cuna hasta la tumba, los liberó eficazmente de la religión de las tumbas. Los sacerdotes solían escribir en los ataúdes unos textos mágicos que se creía que protegían al hombre contra el peligro de que <<le quitaran el corazón en el otro mundo>>. Poco después se coleccionó y se conservó un variado surtido de estos textos mágicos con el nombre de El Libro de los Muertos. Pero, en el valle del Nilo, el ritual mágico se mezcló muy pronto con el ámbito de la conciencia y del carácter hasta un grado pocas veces alcanzado por los rituales de aquella época. Posteriormente se confió más, para la salvación, en estos ideales éticos y morales que en las tumbas tan elaboradas.

\par
%\textsuperscript{(1044.5)}
\textsuperscript{95:2.6} Las supersticiones de estos tiempos se encuentran bien ilustradas en la creencia general en la eficacia del escupitajo como agente curativo\footnote{\textit{Escupitajo como agente curativo}: Mc 8:23; Jn 9:6.}, una idea que tenía su origen en Egipto y que se había difundido desde allí hasta Arabia y Mesopotamia. En la legendaria batalla entre Horus y Set, el joven dios perdió un ojo, pero después de la derrota de Set, el ojo fue restablecido por el sabio dios Thot, que escupió sobre la herida y la curó.

\par
%\textsuperscript{(1044.6)}
\textsuperscript{95:2.7} Los egipcios creyeron durante mucho tiempo que las estrellas que centelleaban en el cielo nocturno representaban la supervivencia de las almas de los muertos virtuosos; pensaban que los otros supervivientes eran absorbidos por el Sol. Durante cierto período, la veneración solar se convirtió en una especie de culto a los antepasados. El pasadizo de entrada inclinado de la gran pirámide señalaba directamente hacia la estrella polar para que el alma del rey, cuando surgiera de la tumba, pudiera ir en línea recta a las constelaciones estacionarias y establecidas de las estrellas fijas, la supuesta morada de los reyes.

\par
%\textsuperscript{(1045.1)}
\textsuperscript{95:2.8} Cuando se observaba que los rayos oblicuos del Sol llegaban hasta la Tierra a través de una abertura en las nubes, se creía que anunciaban el descenso de una escalera celestial por la que el rey y otras almas justas podían ascender. <<El rey Pepi ha puesto su resplandor como una escalera debajo de sus pies para ascender hasta su madre>>.

\par
%\textsuperscript{(1045.2)}
\textsuperscript{95:2.9} Cuando Melquisedek apareció en persona, los egipcios tenían una religión muy superior a la de los pueblos circundantes. Creían que un alma separada del cuerpo, si estaba armada adecuadamente de fórmulas mágicas, podía evitar a los espíritus malignos intermedios y abrirse camino hasta la sala de juicios de Osiris, donde sería admitida en los reinos de la felicidad si era inocente de <<asesinato, robo, falsedad, adulterio, hurto y egoísmo>>. Si este alma era pesada en las balanzas y se la encontraba deficiente, era enviada al infierno, a la Devoradora. Éste era un concepto relativamente avanzado de la vida futura, en comparación con las creencias de muchos pueblos circundantes.

\par
%\textsuperscript{(1045.3)}
\textsuperscript{95:2.10} El concepto de un juicio en el más allá por los pecados cometidos en la vida carnal en la Tierra fue introducido en la teología hebrea procedente de Egipto. La palabra juicio no aparece más que una vez en todo el Libro hebreo de los Salmos, y este salmo concreto fue escrito por un egipcio.

\section*{3. La evolución de los conceptos morales}
\par
%\textsuperscript{(1045.4)}
\textsuperscript{95:3.1} Aunque la cultura y la religión de Egipto procedían principalmente de la Mesopotamia andita y fueron transmitidas ampliamente a las civilizaciones posteriores a través de los hebreos y los griegos, una parte muy importante del idealismo social y ético de los egipcios surgió en el valle del Nilo como un desarrollo puramente evolutivo. A pesar de la importación de una gran parte de la verdad y de la cultura de origen andita, en Egipto se desarrolló, como un progreso puramente humano, más cultura moral de la que apareció mediante técnicas naturales similares en cualquier otra zona circunscrita antes de la donación de Miguel.

\par
%\textsuperscript{(1045.5)}
\textsuperscript{95:3.2} La evolución moral no depende totalmente de la revelación. La propia experiencia del hombre puede dar nacimiento a unos conceptos morales elevados. El hombre puede incluso desarrollar los valores espirituales y obtener la perspicacia cósmica partiendo de su vida personal experiencial, porque un espíritu divino reside en su interior. Estos desarrollos naturales de la conciencia y del carácter fueron acrecentados también por la llegada periódica de instructores de la verdad procedentes, en los tiempos antiguos, del segundo Edén, y más tarde de la sede central de Melquisedek en Salem.

\par
%\textsuperscript{(1045.6)}
\textsuperscript{95:3.3} Miles de años antes de que el evangelio de Salem penetrara en Egipto, sus dirigentes morales enseñaban la justicia, la equidad y que había que evitar la avaricia. Tres mil años antes de que se redactaran las escrituras hebreas, los egipcios tenían el lema: <<Sólido es el hombre cuya regla es la rectitud, y que camina según esta línea de conducta>>. Enseñaban la amabilidad, la moderación y la discreción. Uno de los grandes instructores de esta época dejó este mensaje: <<Actuad con rectitud y tratad a todos con justicia>>. La tríada egipcia de estos tiempos era la Verdad, la Justicia y la Rectitud. De todas las religiones puramente humanas de Urantia, ninguna ha superado nunca los ideales sociales y la grandeza moral de este antiguo humanismo del valle del Nilo.

\par
%\textsuperscript{(1045.7)}
\textsuperscript{95:3.4} Las doctrinas supervivientes de la religión de Salem florecieron en el terreno de estas ideas éticas y de estos ideales morales en evolución. Los conceptos del bien y del mal encontraron una rápida respuesta en el corazón de un pueblo que creía que <<la vida se concede a los pacíficos, y la muerte a los culpables>>. <<El pacífico es aquel que hace lo que es agradable; el culpable es aquel que hace lo que es detestable>>. Los habitantes del valle del Nilo habían vivido durante siglos de acuerdo con estas normas éticas y sociales emergentes antes de albergar los conceptos posteriores de lo justo y lo injusto ---del bien y del mal.

\par
%\textsuperscript{(1046.1)}
\textsuperscript{95:3.5} Egipto era un país intelectual y moral, pero no excesivamente espiritual. En seis mil años sólo surgieron cuatro grandes profetas entre los egipcios. A Amenemope lo siguieron durante una temporada; a Okhbán lo asesinaron; aceptaron a Akenatón, aunque sin entusiasmo, durante una corta generación, y rechazaron a Moisés. Una vez más, las circunstancias políticas, más bien que las religiosas, fueron las que hicieron que a Abraham, y más tarde a José, les resultara fácil ejercer una gran influencia en todo Egipto a favor de las enseñanzas salemitas sobre un solo Dios. Pero cuando los misioneros de Salem entraron por primera vez en Egipto, encontraron que esta cultura evolutiva altamente ética estaba mezclada con las normas morales modificadas de los inmigrantes mesopotámicos. Estos educadores iniciales del valle del Nilo fueron los primeros que proclamaron que la conciencia era el mandamiento de Dios, la voz de la Deidad.

\section*{4. Las enseñanzas de Amenemope}
\par
%\textsuperscript{(1046.2)}
\textsuperscript{95:4.1} A su debido tiempo surgió en Egipto un instructor que muchos llamaron el <<hijo del hombre>>, y otros Amenemope. Este vidente ensalzó la conciencia hasta convertirla en el árbitro supremo entre el bien y el mal, enseñó que los pecados serían castigados, y proclamó que la salvación se obtenía recurriendo a la deidad solar.

\par
%\textsuperscript{(1046.3)}
\textsuperscript{95:4.2} Amenemope enseñó que las riquezas y la fortuna eran dones de Dios\footnote{\textit{La riqueza como dones de Dios}: Sal 112:1-3; 115:13; Pr 22:4; Ec 8:12-13.}, y este concepto influyó profundamente en la filosofía hebrea que apareció más tarde. Este noble instructor creía que la conciencia de Dios era el factor determinante de toda conducta; que había que vivir cada momento siendo consciente de la presencia de Dios y de nuestra responsabilidad hacia él. Las enseñanzas de este sabio fueron traducidas posteriormente al hebreo y se convirtieron en el libro sagrado de este pueblo mucho antes de que el Antiguo Testamento fuera consignado por escrito. El sermón principal de este hombre de bien consistió en instruir a su hijo\footnote{\textit{Enseñanzas para su hijo}: Pr 1:8-16; 2:1-2; 3:1-2; 4:10-20; 5:2; 6:20-21; 7:1-3.} sobre la rectitud y la honradez en los puestos de confianza gubernamentales, y estos nobles sentimientos de hace mucho tiempo honrarían a cualquier estadista moderno.

\par
%\textsuperscript{(1046.4)}
\textsuperscript{95:4.3} Este sabio del Nilo enseñó que <<las riquezas cogen alas y emprenden el vuelo>>\footnote{\textit{Las riquezas cogen alas y emprenden el vuelo}: Pr 23:5.} ---que todas las cosas terrestres son efímeras. Su oración principal era <<líbrame del temor>>\footnote{\textit{Líbrame del temor}: Job 21:9; Sal 23:4; 27:3; Pr 1:33; Is 41:10,13-14; 44:8; 54:4,14; Jer 30:10.}. Exhortó a todos a que se apartaran de las <<palabras de los hombres>> y se volvieran hacia <<los actos de Dios>>\footnote{\textit{Apartarse de las palabras y actuar}: Mt 3:2; 4:17.}. Enseñó en esencia que el hombre propone, pero que Dios dispone. Sus enseñanzas, traducidas al hebreo, determinaron la filosofía del Libro de los Proverbios del Antiguo Testamento. Traducidas al griego, influyeron en toda la filosofía religiosa helénica posterior. Filón, el filósofo ulterior de Alejandría, poseía un ejemplar del Libro de la Sabiduría.

\par
%\textsuperscript{(1046.5)}
\textsuperscript{95:4.4} Amenemope ejerció su actividad para conservar la ética de la evolución y la moral de la revelación, y en sus escritos las transmitió tanto a los hebreos como a los griegos. No fue el instructor religioso más grande de esta época, pero fue el más influyente en el sentido de que dejó su huella en el pensamiento posterior de dos eslabones vitales para el crecimiento de la civilización occidental ---los hebreos, entre los cuales se produjo el apogeo de la fe religiosa occidental, y los griegos, que desarrollaron el pensamiento filosófico puro hasta sus niveles europeos más elevados.

\par
%\textsuperscript{(1046.6)}
\textsuperscript{95:4.5} En el Libro de los Proverbios hebreos, los capítulos quince, diecisiete, veinte, y desde el capítulo veintidós versículo diecisiete hasta el capítulo veinticuatro versículo veintidós, fueron cogidos casi literalmente del Libro de la Sabiduría de Amenemope\footnote{\textit{Del Libro de la Sabiduría}: Pr 15:all; 17; 20; 22:17-29; 23; 24:1-22.}. El salmo primero del Libro hebreo de los Salmos fue escrito por Amenemope\footnote{\textit{Amenemope escribió el primer salmo}: Sal 1.} y es la esencia de las enseñanzas de Akenatón.

\section*{5. El extraordinario Akenatón}
\par
%\textsuperscript{(1047.1)}
\textsuperscript{95:5.1} Las enseñanzas de Amenemope perdían lentamente su dominio sobre la mente egipcia cuando, gracias a la influencia de un médico salemita egipcio, una mujer de la familia real abrazó las enseñanzas de Melquisedek. Esta mujer convenció a su hijo Akenatón, faraón de Egipto, para que aceptara estas doctrinas de Un Solo Dios.

\par
%\textsuperscript{(1047.2)}
\textsuperscript{95:5.2} Desde la desaparición física de Melquisedek, ningún ser humano había poseído hasta ese momento un concepto tan asombrosamente claro de la religión revelada de Salem como Akenatón. En algunos aspectos, este joven rey egipcio es una de las personas más extraordinarias de la historia humana. Durante esta época de creciente depresión espiritual en Mesopotamia, Akenatón conservó viva en Egipto la doctrina de El Elyón, el Dios Único, manteniendo así abierto el canal filosófico monoteísta que fue fundamental para el trasfondo religioso de la entonces futura donación de Miguel. Y fue en reconocimiento de esta proeza, entre otras razones, por lo que el niño Jesús fue llevado a Egipto\footnote{\textit{Jesús en Egipto}: Mt 2:14.}, donde algunos sucesores espirituales de Akenatón le vieron, y comprendieron hasta cierto punto algunas fases de su misión divina en Urantia.

\par
%\textsuperscript{(1047.3)}
\textsuperscript{95:5.3} Moisés, el personaje más importante aparecido entre Melquisedek y Jesús, fue el regalo conjunto que dieron al mundo la raza hebrea y la familia real egipcia. Si Akenatón hubiera poseído la diversidad de talentos y la capacidad de Moisés, si hubiera manifestado una genialidad política comparable a su sorprendente autoridad religiosa, Egipto se habría convertido entonces en la gran nación monoteísta de esta época; y si esto hubiera sucedido, es muy posible que Jesús hubiera vivido la mayor parte de su vida mortal en Egipto.

\par
%\textsuperscript{(1047.4)}
\textsuperscript{95:5.4} Ningún rey procedió nunca en toda la historia a hacer que una nación entera cambiara tan metódicamente del politeísmo al monoteísmo como lo hizo este extraordinario Akenatón. Con la más asombrosa determinación, este joven soberano rompió con el pasado, cambió su nombre, abandonó su capital, construyó una ciudad totalmente nueva, y creó una literatura y un arte nuevos para todo un pueblo. Pero fue demasiado deprisa; construyó demasiado, más de lo que podía perdurar después de su partida. Además, no logró asegurar la estabilidad y la prosperidad material de sus súbditos, los cuales reaccionaron desfavorablemente contra sus enseñanzas religiosas cuando las aguas posteriores de la adversidad y la opresión asolaron a los egipcios.

\par
%\textsuperscript{(1047.5)}
\textsuperscript{95:5.5} Si este hombre con una perspicacia asombrosamente clara y una resolución extraordinaria hubiera tenido la sagacidad política de Moisés, habría cambiado toda la historia de la evolución de la religión y de la revelación de la verdad en el mundo occidental. Durante su vida fue capaz de refrenar las actividades de los sacerdotes, a los cuales desacreditó en general, pero éstos mantuvieron sus cultos en secreto y se lanzaron a la acción en cuanto el joven rey desapareció del poder; y no tardaron en relacionar todas las dificultades posteriores de Egipto con el establecimiento del monoteísmo durante su reinado.

\par
%\textsuperscript{(1047.6)}
\textsuperscript{95:5.6} Akenatón trató muy sabiamente de establecer el monoteísmo bajo la apariencia del dios Sol. Esta decisión de enfocar la adoración del Padre Universal absorbiendo a todos los dioses en la adoración del Sol se debió al consejo del médico salemita. Akenatón cogió las doctrinas generalizadas de la religión entonces existente de Atón sobre la paternidad y la maternidad de la Deidad, y creó una religión que reconocía una relación íntima de adoración entre el hombre y Dios.

\par
%\textsuperscript{(1048.1)}
\textsuperscript{95:5.7} Akenatón fue lo bastante sabio como para mantener la adoración exterior de Atón, el dios Sol, mientras que condujo a sus asociados a la adoración disfrazada del Dios único, el creador de Atón y el Padre supremo de todos. Este joven rey-instructor fue un escritor prolífico, siendo el autor de la exposición titulada <<El Dios Único>>, un libro de treinta y un capítulos que los sacerdotes destruyeron por completo cuando recuperaron el poder. Akenatón escribió también ciento treinta y siete himnos, doce de los cuales se conservan actualmente en el Libro de los Salmos del Antiguo Testamento, atribuídos a autores hebreos.

\par
%\textsuperscript{(1048.2)}
\textsuperscript{95:5.8} La palabra suprema de la religión de Akenatón en la vida diaria era <<rectitud>>, y amplió rápidamente el concepto de la acción correcta hasta abarcar tanto la ética internacional como la nacional. Ésta fue una generación de una piedad personal asombrosa y estuvo caracterizada por la sincera aspiración, entre los hombres y las mujeres más inteligentes, de encontrar a Dios y conocerlo. En aquella época, la posición social o la riqueza no concedía a ningún egipcio ninguna ventaja a los ojos de la ley. La vida familiar de Egipto contribuyó mucho a conservar y aumentar la cultura moral, y sirvió posteriormente de inspiración para la magnífica vida familiar de los judíos en Palestina.

\par
%\textsuperscript{(1048.3)}
\textsuperscript{95:5.9} La debilidad fatídica del evangelio de Akenatón consistió en su verdad más grande, la enseñanza de que Atón no sólo era el creador de Egipto, sino también del <<mundo entero, de los hombres y los animales, y de todos los países extranjeros, incluídos Siria y Cush, además de esta tierra de Egipto. A todos los coloca en su lugar y satisface sus necesidades>>\footnote{\textit{Dios como creador de todo}: Gn 1:1; Gn 2:4-23; Gn 5:1-2; Ex 20:11; Ex 31:17; 2 Re 19:15; 2 Cr 2:12; Neh 9:6; Sal 115:15; Sal 121:2; Sal 124:8; Sal 134:3; Sal 146:6; Eclo 1:1-4; Eclo 33:10; Is 37:16; Is 40:26,28; Is 42:5; Is 45:12,18; Jer 10:11-12; Jer 32:17; Jer 51:15; Bar 3:32-36; Am 4:13; Mal 2:10; Mc 13:19; Jn 1:1-3; Hch 4:24; Hch 14:15; Ef 3:9; Col 1:16; Heb 1:2; 1 P 4:19; Ap 4:11; Ap 10:6; Ap 14:7.}. Estos conceptos de la Deidad eran elevados y sublimes, pero no eran nacionalistas. Estos sentimientos internacionalistas en materia religiosa no lograban aumentar la moral del ejército egipcio en el campo de batalla, mientras que proporcionaban a los sacerdotes unas armas eficaces que podían utilizar en contra del joven rey y de su nueva religión. Tenía un concepto de la Deidad muy por encima del de los hebreos posteriores, pero era demasiado avanzado para servir los objetivos del constructor de una nación.

\par
%\textsuperscript{(1048.4)}
\textsuperscript{95:5.10} Aunque el ideal monoteísta sufrió con la desaparición de Akenatón, la idea de un solo Dios sobrevivió en la mente de muchos grupos. El yerno de Akenatón estuvo de acuerdo con los sacerdotes, volvió a la adoración de los antiguos dioses y cambió su nombre por el de Tut-Ank-Ammon. La capital regresó a Tebas y los sacerdotes se enriquecieron con las tierras, llegando finalmente a poseer una séptima parte de todo Egipto; poco después, un miembro de esta misma orden de sacerdotes se atrevió a apoderarse del trono.

\par
%\textsuperscript{(1048.5)}
\textsuperscript{95:5.11} Pero los sacerdotes no pudieron vencer por completo la oleada monoteísta. Se vieron obligados a reunir y fusionar progresivamente a sus dioses; la familia de dioses se contrajo cada vez más. Akenatón había asociado el disco llameante de los cielos con el Dios creador, y esta idea continuó ardiendo en el corazón de los hombres, incluso de los sacerdotes, mucho tiempo después de la muerte del joven reformador. El concepto del monoteísmo no desapareció nunca del corazón de los hombres de Egipto ni del mundo. Sobrevivió incluso hasta la llegada del Hijo Creador de este mismo Padre divino, el Dios único que Akenatón había proclamado con tanto entusiasmo para que todo Egipto lo adorara.

\par
%\textsuperscript{(1048.6)}
\textsuperscript{95:5.12} La debilidad de la doctrina de Akenatón residía en el hecho de que proponía una religión tan avanzada, que sólo los egipcios instruidos podían comprender plenamente sus enseñanzas. La masa de los obreros agrícolas nunca captó realmente su evangelio, y por lo tanto se encontraba preparada para volver, con los sacerdotes, a la antigua adoración de Isis y de su consorte Osiris, el cual se suponía que había sido resucitado milagrosamente de una muerte cruel a manos de Set, el dios de las tinieblas y del mal.

\par
%\textsuperscript{(1049.1)}
\textsuperscript{95:5.13} La enseñanza de que todos los hombres podían alcanzar la inmortalidad era demasiado avanzada para los egipcios. Sólo se prometía la resurrección a los reyes y a los ricos; por esta razón, embalsamaban y conservaban tan cuidadosamente sus cuerpos en las tumbas para el día del juicio. Pero la democracia de la salvación y la resurrección, tal como la enseñó Akenatón, terminó por prevalecer, incluso hasta el punto de que los egipcios creyeron posteriormente en la supervivencia de los animales.

\par
%\textsuperscript{(1049.2)}
\textsuperscript{95:5.14} Aunque el esfuerzo de este soberano egipcio por imponer a su pueblo la adoración de un solo Dios pareció fracasar, debemos indicar que las repercusiones de su obra sobrevivieron durante siglos tanto en Palestina como en Grecia, y que Egipto se convirtió así en el agente que transmitió la cultura evolutiva combinada del Nilo y la religión revelada del Éufrates a todos los pueblos occidentales posteriores.

\par
%\textsuperscript{(1049.3)}
\textsuperscript{95:5.15} La gloria de esta gran era de desarrollo moral y de crecimiento espiritual en el valle del Nilo fue desapareciendo rápidamente hacia la época en que empezó la vida nacional de los hebreos; como resultado de su estancia en Egipto, estos beduinos se llevaron una gran parte de estas enseñanzas y perpetuaron numerosas doctrinas de Akenatón en su religión racial.

\section*{6. Las doctrinas de Salem en Irán}
\par
%\textsuperscript{(1049.4)}
\textsuperscript{95:6.1} Desde Palestina, algunos misioneros de Melquisedek atravesaron Mesopotamia y llegaron hasta la gran meseta iraní. Durante más de quinientos años, los educadores de Salem hicieron progresos en Irán, y toda la nación estaba oscilando hacia la religión de Melquisedek cuando un cambio de gobernantes precipitó una implacable persecución que puso prácticamente fin a las enseñanzas monoteístas del culto de Salem. La doctrina de la alianza con Abraham estaba a punto de extinguirse en Persia cuando, en el siglo sexto antes de Cristo, aquel gran siglo de renacimiento moral, Zoroastro apareció para reanimar las ascuas ardientes del evangelio de Salem.

\par
%\textsuperscript{(1049.5)}
\textsuperscript{95:6.2} Este fundador de una nueva religión era un joven enérgico y aventurero que, en su primera peregrinación a Ur en Mesopotamia, había oído hablar de las tradiciones de Caligastia y de la rebelión de Lucifer ---junto con otras muchas tradiciones--- todo lo cual había impresionado poderosamente su naturaleza religiosa. Por consiguiente, a consecuencia de un sueño que tuvo en Ur, estableció el programa de regresar a su hogar en el norte para emprender la reforma de la religión de su pueblo. Se había impregnado de la idea hebrea de un Dios de justicia, el concepto mosaico de la divinidad. La idea de un Dios supremo estaba clara en su mente y consideró a todos los otros dioses como diablos, los relegó a la categoría de demonios, sobre los cuales había oído hablar en Mesopotamia. Se había enterado de la historia de los Siete Espíritus Maestros cuya tradición subsistía en Ur y, en consecuencia, creó una constelación de siete dioses supremos con Ahura-Mazda a la cabeza. Estos dioses subordinados los asoció con la idealización de la Ley Justa, el Buen Pensamiento, el Gobierno Noble, el Carácter Santo, la Salud y la Inmortalidad.

\par
%\textsuperscript{(1049.6)}
\textsuperscript{95:6.3} Esta nueva religión era una religión de acción ---de trabajo--- no de oraciones ni rituales. Su Dios era un ser supremamente sabio y el protector de la civilización; era una filosofía religiosa militante que se atrevió a combatir el mal, la inactividad y el atraso.

\par
%\textsuperscript{(1049.7)}
\textsuperscript{95:6.4} Zoroastro no enseñó la adoración del fuego, sino que trató de utilizar la llama como símbolo del Espíritu puro y sabio que predomina de manera suprema y universal. (Es desgraciadamente cierto que sus seguidores posteriores veneraron y adoraron este fuego simbólico). Finalmente, después de la conversión de un príncipe iraní, esta nueva religión fue difundida por la espada. Y Zoroastro murió luchando heroicamente por lo que creía que era la <<verdad del Señor de la luz>>.

\par
%\textsuperscript{(1050.1)}
\textsuperscript{95:6.5} El zoroastrismo es el único credo urantiano que perpetúa las enseñanzas edénicas y dalamatianas sobre los Siete Espíritus Maestros. Aunque no logró desarrollar el concepto de la Trinidad, se acercó en cierto modo al de Dios Séptuple. El zoroastrismo original no era un puro dualismo; aunque las enseñanzas iniciales describían al mal como un coordinado temporal de la bondad, en la eternidad estaba claramente sumergido en la realidad última del bien. La creencia de que el bien y el mal luchaban en igualdad de condiciones sólo mereció crédito en tiempos posteriores.

\par
%\textsuperscript{(1050.2)}
\textsuperscript{95:6.6} Las tradiciones judías sobre el cielo y el infierno y la doctrina sobre los demonios, tal como están registradas en las escrituras hebreas, aunque estaban basadas en las tradiciones sobrevivientes de Lucifer y Caligastia, procedían principalmente de los zoroastrianos durante la época en que los judíos estuvieron bajo el dominio político y cultural de los persas. Al igual que los egipcios, Zoroastro enseñó el <<día del juicio>>, pero este acontecimiento lo relacionó con el fin del mundo.

\par
%\textsuperscript{(1050.3)}
\textsuperscript{95:6.7} Incluso la religión que sucedió en Persia al zoroastrismo recibió una notable influencia de éste. Cuando los sacerdotes iraníes trataron de destruir las enseñanzas de Zoroastro, resucitaron el antiguo culto de Mitra. Y el mitracismo se difundió por todas las regiones del Levante y del Mediterráneo, siendo algún tiempo contemporáneo tanto del judaísmo como del cristianismo. Las enseñanzas de Zoroastro dejaron así su huella sucesivamente en tres grandes religiones: el judaísmo, el cristianismo y, a través de ellos, el mahometismo.

\par
%\textsuperscript{(1050.4)}
\textsuperscript{95:6.8} Pero existe un gran abismo entre las enseñanzas sublimes y los nobles salmos de Zoroastro, y las tergiversaciones modernas de su evangelio llevadas a cabo por los parsis, con su gran temor a los muertos, unido al mantenimiento de la creencia en unos sofismas que Zoroastro nunca se rebajó a aceptar.

\par
%\textsuperscript{(1050.5)}
\textsuperscript{95:6.9} Este gran hombre formó parte de aquel grupo incomparable que surgió en el siglo sexto antes de Cristo para evitar que finalmente se extinguiera por completo la luz de Salem que brillaba tan débilmente para mostrar a los hombres, en su mundo ensombrecido, el camino luminoso que conduce a la vida eterna.

\section*{7. Las enseñanzas de Salem en Arabia}
\par
%\textsuperscript{(1050.6)}
\textsuperscript{95:7.1} Las enseñanzas de Melquisedek sobre un solo Dios se establecieron en el desierto de Arabia en una fecha relativamente reciente. Al igual que les sucedió en Grecia, los misioneros de Salem fracasaron en Arabia debido a que habían comprendido mal las instrucciones de Maquiventa relacionadas con el exceso de organización. Pero no les entorpeció la interpretación que hicieron de su advertencia en contra de todo esfuerzo por extender el evangelio mediante la fuerza militar o la coacción civil.

\par
%\textsuperscript{(1050.7)}
\textsuperscript{95:7.2} Las enseñanzas de Melquisedek no fracasaron ni siquiera en China o en Roma de una manera más completa que en esta región desértica tan cercana a la misma Salem. Mucho tiempo después de que la mayoría de los pueblos orientales y occidentales se hubieran vuelto budistas y cristianos respectivamente, los del desierto de Arabia continuaban viviendo como hacía miles de años. Cada tribu adoraba a su antiguo fetiche, y muchas familias tenían sus propios dioses lares particulares. La lucha continuó durante mucho tiempo entre la Istar babilónica, el Yahvé hebreo, el Ahura iraní y el Padre cristiano del Señor Jesucristo. Ninguno de estos conceptos fue nunca capaz de desplazar completamente a los otros.

\par
%\textsuperscript{(1051.1)}
\textsuperscript{95:7.3} En toda Arabia había familias y clanes aquí y allá que se aferraban a la vaga idea de un solo Dios. Estos grupos guardaban como un tesoro las tradiciones de Melquisedek, Abraham, Moisés y Zoroastro. Había numerosos centros que podían haber respondido al evangelio de Jesús, pero los misioneros cristianos de los países desérticos formaban un grupo austero e inflexible, en contraste con los misioneros innovadores y dispuestos a hacer compromisos que ejercieron su actividad en los países mediterráneos. Si los seguidores de Jesús se hubieran tomado más en serio su mandato de <<ir por todo el mundo para predicar el evangelio>>\footnote{\textit{El gran mandato}: Mt 24:14; 28:19-20a; Mc 13:10; 16:15; Lc 24:47; Jn 17:18; Hch 1:8b.}, y si hubieran sido más amables en esta predicación, menos estrictos en las exigencias sociales colaterales inventadas por ellos mismos, entonces muchos países hubieran recibido con agrado el simple evangelio del hijo del carpintero\footnote{\textit{El hijo del carpintero}: Mt 13:55; Mc 6:3; Lc 3:23; 4:22; Jn 1:45; 6:42.}, entre ellos Arabia.

\par
%\textsuperscript{(1051.2)}
\textsuperscript{95:7.4} A pesar del hecho de que los grandes monoteísmos levantinos no lograron arraigar en Arabia, esta tierra desértica fue capaz de dar nacimiento a una religión que, aunque era menos exigente en sus requisitos sociales, sin embargo era monoteísta.

\par
%\textsuperscript{(1051.3)}
\textsuperscript{95:7.5} Las creencias primitivas y desorganizadas del desierto sólo tenían un factor de naturaleza tribal, racial o nacional, y era el respeto especial y general que casi todas las tribus árabes estaban dispuestas a manifestar a cierta piedra negra fetiche situada en cierto templo de la Meca. Este punto de contacto y de veneración comunes condujo posteriormente al establecimiento de la religión islámica. La piedra de la Caaba se volvió para los árabes lo que Yahvé, el espíritu del volcán, era para sus primos los judíos semitas.

\par
%\textsuperscript{(1051.4)}
\textsuperscript{95:7.6} La fuerza del islam ha residido en su presentación clara y bien definida de Alá como la sola y única Deidad; su debilidad ha consistido en utilizar la fuerza militar para promulgar su religión, junto con la degradación de las mujeres. Pero el islam se ha mantenido inquebrantablemente fiel a su presentación de la Única Deidad Universal de todos, <<que conoce lo invisible y lo visible.
Él es el misericordioso y el compasivo>>. <<En verdad, Dios concede su bondad en abundancia a todos los hombres>>. <<Y cuando estoy enfermo, él es el que me cura>>. <<Porque cada vez que tres personas se reúnen para hablar, Dios está presente como una cuarta>>, porque ¿acaso no es <<el primero y el último, y también el visible y el oculto>>?

\par
%\textsuperscript{(1051.5)}
\textsuperscript{95:7.7} [Presentado por un Melquisedek de Nebadon.]


\chapter{Documento 96. Yahvé ---el Dios de los hebreos}
\par
%\textsuperscript{(1052.1)}
\textsuperscript{96:0.1} AL HACERSE un concepto de la Deidad, el hombre empieza por incluir a todos los dioses, luego subordina todos los dioses extranjeros a su deidad tribal, y finalmente los excluye a todos salvo al Dios único de valor final y supremo. Los judíos sintetizaron a todos los dioses en su concepto más sublime del Señor Dios de Israel. Los hindúes fusionaron igualmente a sus múltiples deidades en <<la espiritualidad única de los dioses>> descrita en el Rig Veda, mientras que los mesopotámicos redujeron a sus dioses al concepto más centralizado de Belo-Marduc. Estas ideas monoteístas maduraron en el mundo entero poco después de la aparición de Maquiventa Melquisedek en Salem, en Palestina. Pero el concepto de la Deidad predicado por Melquisedek era diferente al de la filosofía evolutiva de inclusión, subordinación y exclusión; estaba basado exclusivamente en el \textit{poder creador}, y muy pronto influyó sobre los conceptos más elevados de la deidad que existían en Mesopotamia, la India y Egipto.

\par
%\textsuperscript{(1052.2)}
\textsuperscript{96:0.2} La religión de Salem fue venerada como una tradición por los kenitas y otras diversas tribus cananeas. Uno de los objetivos de la encarnación de Melquisedek fue fomentar una religión de un solo Dios de tal manera que preparara el camino para la donación en la Tierra de un Hijo de este Dios único. Miguel difícilmente podía venir a Urantia antes de que existiera un pueblo que creyera en el Padre Universal, en medio del cual poder aparecer.

\par
%\textsuperscript{(1052.3)}
\textsuperscript{96:0.3} La religión de Salem sobrevivió como credo de los kenitas de Palestina, y esta religión, tal como los hebreos la adoptaron más tarde\footnote{\textit{Religión hebrea}: Gn 17:2-9.}, fue influida primero por las enseñanzas morales de los egipcios, más adelante por el pensamiento teológico babilónico, y finalmente por los conceptos iraníes sobre el bien y el mal. Objetivamente, la religión hebrea está basada en la alianza entre Abraham y Maquiventa Melquisedek; evolutivamente, es la consecuencia de muchas circunstancias debidas a situaciones extraordinarias, pero culturalmente se ha apropiado libremente de la religión, la moralidad y la filosofía de todo el Levante. Una gran parte de la moralidad y del pensamiento religioso de Egipto, Mesopotamia e Irán fue transmitida a los pueblos occidentales a través de la religión hebrea.

\section*{1. Los conceptos de la Deidad entre los semitas}
\par
%\textsuperscript{(1052.4)}
\textsuperscript{96:1.1} Los primeros semitas consideraban que todas las cosas estaban habitadas por un espíritu. Tenían los espíritus del mundo animal y del mundo vegetal; los espíritus de las estaciones del año, el señor de la progenie; los espíritus del fuego, el agua y el aire; un verdadero panteón de espíritus para temer y adorar. Las enseñanzas de Melquisedek referentes a un Creador Universal nunca destruyeron por completo la creencia en estos espíritus subordinados o dioses de la naturaleza.

\par
%\textsuperscript{(1052.5)}
\textsuperscript{96:1.2} El progreso que hicieron los hebreos desde el politeísmo hasta el monoteísmo, pasando por el henoteísmo, no fue un desarrollo conceptual continuo e ininterrumpido. Sufrieron muchos retrocesos en la evolución de sus conceptos sobre la Deidad, mientras que en una época cualquiera existieron ideas variables sobre Dios entre los diferentes grupos de creyentes semitas. De vez en cuando aplicaron numerosos términos a sus conceptos de Dios, y con el fin de impedir la confusión, definiremos estas diversas denominaciones de la Deidad tal como están relacionadas con la evolución de la teología judía:

\par
%\textsuperscript{(1053.1)}
\textsuperscript{96:1.3} 1. \textit{Yahvé}\footnote{\textit{Yahvé}: Gn 22:14; Ex 6:3.} era el dios de las tribus palestinas del sur, que asociaron este concepto de la deidad con el Monte Horeb, el volcán del Sinaí. Yahvé era simplemente uno de los cientos de miles de dioses de la naturaleza que retenían la atención y reclamaban la adoración de las tribus y los pueblos semitas.

\par
%\textsuperscript{(1053.2)}
\textsuperscript{96:1.4} 2. \textit{El Elyón}\footnote{\textit{El Elyón}: Gn 14:18-22; Heb 7:1.}. Después de la estancia de Melquisedek en Salem, su doctrina de la Deidad sobrevivió durante siglos en diversas versiones, pero generalmente connotaban el término de El Elyón, el Dios Altísimo del cielo. Muchos semitas, incluyendo a los descendientes inmediatos de Abraham, adoraron en distintas épocas a Yahvé y a El Elyón al mismo tiempo.

\par
%\textsuperscript{(1053.3)}
\textsuperscript{96:1.5} 3. \textit{El Shaddai}\footnote{\textit{El Shaddai}: Gn 17:1; 28:3; Ex 6:3.}. Es difícil explicar lo que representaba El Shaddai. Esta idea de Dios era un compuesto procedente de las enseñanzas del Libro de la Sabiduría de Amenemope, modificadas por la doctrina de Atón enseñada por Akenatón, e influidas además por las enseñanzas de Melquisedek que estaban incorporadas en el concepto de El Elyón. Pero a medida que el concepto de El Shaddai impregnó el pensamiento hebreo, sufrió la profunda influencia de las creencias que había en el desierto sobre Yahvé.

\par
%\textsuperscript{(1053.4)}
\textsuperscript{96:1.6} Una de las ideas predominantes de la religión de esta era fue el concepto egipcio de la Providencia divina, la enseñanza de que la prosperidad material era una recompensa por haber servido a El Shaddai.

\par
%\textsuperscript{(1053.5)}
\textsuperscript{96:1.7} 4. \textit{El}\footnote{\textit{El}: Gn 31:13; Dt 4:24.}. En medio de toda esta confusión de terminología y de vaguedad de conceptos, muchos creyentes devotos se esforzaron sinceramente por adorar todas estas ideas evolutivas de la divinidad, y se estableció la costumbre de referirse a esta Deidad compuesta como El. Este término incluía además otros dioses de la naturaleza adorados por los beduinos.

\par
%\textsuperscript{(1053.6)}
\textsuperscript{96:1.8} 5. \textit{Elohim}\footnote{\textit{Elohim}: Gn 1:1; Ex 6:2.}. En Kish y en Ur subsistieron durante mucho tiempo unos grupos sumerio-caldeos que enseñaron un concepto de Dios de tres en uno basado en las tradiciones de los tiempos de Adán y de Melquisedek. Esta doctrina fue llevada a Egipto, donde se adoró a esta Trinidad con el nombre de Elohim, o Eloah en singular. Los círculos filosóficos de Egipto y los educadores alejandrinos posteriores de origen hebreo enseñaron esta unidad de dioses plurales. En la época del éxodo, muchos consejeros de Moisés creían en esta Trinidad. Pero el concepto del Elohim trinitario nunca formó realmente parte de la teología hebrea hasta después de sufrir la influencia política de los babilonios.

\par
%\textsuperscript{(1053.7)}
\textsuperscript{96:1.9} 6. \textit{Nombres diversos}. A los semitas no les gustaba pronunciar el nombre de su Deidad, por lo que de vez en cuando recurrieron a numerosas denominaciones tales como: el Espíritu de Dios, el Señor, el Ángel del Señor, el Todopoderoso, el Santo, el Altísimo, Adonai, el Anciano de los Días, el Señor Dios de Israel, el Creador del Cielo y de la Tierra, Kyrios, Jah, el Señor de los Ejércitos y el Padre que está en los Cielos\footnote{\textit{El Espíritu de Dios}: Gn 1:2. \textit{El Señor}: Gn 18:27. \textit{Ángel del Señor}: Gn 16:7. \textit{Todopoderoso}: Gn 49:25. \textit{El Santo}: 2 Re 19:22; Job 6:10; Sal 71:22; Is 1:4. \textit{El Altísimo}: Nm 24:16. \textit{Adonai}: Jos 3:11,13. \textit{Anciano de los Días}: Dn 7:9,13,22. \textit{Señor Dios de Israel}: Ex 5:1. \textit{Creador del Cielo y de la Tierra}: Gn 1:1; 2:4. \textit{Kyrios}: Hch 19:20. \textit{Jah}: Sal 68:4. \textit{Señor de los Ejércitos}: 1 Sam 1:3. \textit{Padre que está en los Cielos}: Mt 5:16,45; Lc 11:2.}.

\par
%\textsuperscript{(1053.8)}
\textsuperscript{96:1.10} \textit{Jehová}\footnote{\textit{Jehová}: Gn 22:14; Ex 6:3; Sal 83:18; Is 12:2; 26:4.} es un término que se ha empleado en tiempos recientes para designar el concepto definitivo de Yahvé que apareció finalmente por evolución en la larga experiencia de los hebreos. Pero el nombre de Jehová no se empezó a utilizar hasta mil quinientos años después de la época de Jesús.

\par
%\textsuperscript{(1054.1)}
\textsuperscript{96:1.11} Hasta cerca del año 2000 a. de J. C., el Monte Sinaí fue un volcán intermitentemente activo donde se produjeron erupciones ocasionales hasta la época de la estancia de los israelitas en esta región. El fuego y el humo, junto con las detonaciones estruendosas que acompañaban a las erupciones de esta montaña volcánica, impresionaban y atemorizaban a los beduinos de las regiones circundantes, provocándoles un gran temor de Yahvé. Este espíritu del Monte Horeb se convirtió más tarde en el dios de los semitas hebreos, los cuales terminaron por creer que era supremo por encima de todos los demás dioses.

\par
%\textsuperscript{(1054.2)}
\textsuperscript{96:1.12} Los cananeos habían venerado durante mucho tiempo a Yahvé, y aunque muchos kenitas creían más o menos en El Elyón, el superdios de la religión de Salem, la mayoría de los cananeos se mantenía vagamente fiel a la adoración de las antiguas deidades tribales. Estaban poco dispuestos a abandonar a sus deidades nacionales a favor de un Dios internacional, por no decir interplanetario. No se sentían inclinados hacia una deidad universal, y por eso estas tribus continuaron adorando a sus deidades tribales, incluyendo a Yahvé y a los becerros de plata y de oro que simbolizaban el concepto que tenían los pastores beduinos del espíritu del volcán del Sinaí.

\par
%\textsuperscript{(1054.3)}
\textsuperscript{96:1.13} Aunque los sirios adoraban a sus dioses, también creían en el Yahvé de los hebreos, porque sus profetas le habían dicho al rey de Siria: <<Sus dioses son dioses de las colinas; por eso fueron más fuertes que nosotros; pero luchemos contra ellos en la llanura, y seguramente seremos más fuertes que ellos>>\footnote{\textit{Sus dioses son dioses de las colinas}: 1 Re 20:23.}.

\par
%\textsuperscript{(1054.4)}
\textsuperscript{96:1.14} A medida que el hombre posee más cultura, los dioses menores quedan subordinados a una deidad suprema; el gran Júpiter sólo sobrevive como una exclamación. Los monoteístas conservan a sus dioses subordinados como espíritus, demonios, Parcas, Nereidas, hadas, duendes, enanos, hadas malignas y el mal de ojo. Los hebreos pasaron por el henoteísmo y creyeron durante mucho tiempo en la existencia de otros dioses diferentes a Yahvé, pero consideraron cada vez más que estas deidades extranjeras estaban subordinadas a Yahvé. Admitían la existencia de Quemos, el dios de los amoritas, pero sostenían que estaba subordinado a Yahvé.

\par
%\textsuperscript{(1054.5)}
\textsuperscript{96:1.15} De todas las teorías humanas sobre Dios, la idea de Yahvé es la que ha sufrido el desarrollo más extenso. Su evolución progresiva sólo se puede comparar con la metamorfosis del concepto de Buda en Asia, que al final condujo al concepto del Absoluto Universal, al igual que el concepto de Yahvé condujo finalmente a la idea del Padre Universal. Pero se debe comprender como un hecho histórico que, aunque los judíos cambiaron así sus ideas sobre la Deidad desde el dios tribal del Monte Horeb hasta el Padre Creador amante y misericordioso de los tiempos posteriores, no cambiaron su nombre; a este concepto evolutivo de la Deidad continuaron llamándole siempre Yahvé.

\section*{2. Los pueblos semitas}
\par
%\textsuperscript{(1054.6)}
\textsuperscript{96:2.1} Los semitas del este eran unos jinetes bien organizados y bien dirigidos que invadieron las regiones orientales de la medialuna fértil y allí se unieron con los babilonios. Los caldeos cercanos a Ur figuraban entre los semitas orientales más avanzados. Los fenicios eran un grupo superior y bien organizado de semitas mezclados que ocupaban la región occidental de Palestina, a lo largo de la costa mediterránea. Desde el punto de vista racial, los semitas se encontraban entre los pueblos más mezclados de Urantia, pues contenían factores hereditarios de casi todas las nueve razas del mundo.

\par
%\textsuperscript{(1054.7)}
\textsuperscript{96:2.2} Los semitas árabes penetraron combatiendo una y otra vez en el norte de la Tierra Prometida, la tierra que <<abundaba en leche y miel>>\footnote{\textit{Abundaba en leche y miel}: Ex 3:8,17.}, pero todas las veces fueron expulsados por los semitas y los hititas del norte mejor organizados y mucho más civilizados. Más tarde, durante una hambruna excepcionalmente grave, estos beduinos errantes entraron en gran número en Egipto como obreros contratados para los trabajos públicos egipcios, y terminaron padeciendo la amarga experiencia de la esclavitud en el duro trabajo diario de los obreros corrientes y oprimidos del valle del Nilo.

\par
%\textsuperscript{(1055.1)}
\textsuperscript{96:2.3} Únicamente después de la época de Maquiventa Melquisedek y Abraham fue cuando algunas tribus de semitas, debido a sus creencias religiosas particulares, fueron llamadas hijos de Israel\footnote{\textit{Hijos de Israel}: Gn 32:32; 45:21.} y, más tarde aún, hebreos, judíos y el <<pueblo elegido>>\footnote{\textit{Pueblo elegido}: 1 Re 3:8; 1 Cr 17:21-22; Sal 33:12; 105:6,43; 135:4,; Is 41:8-9; 43:20-21; 44:1; Dt 7:6; 14:2.}. Abraham no era el padre racial de todos los hebreos\footnote{\textit{Hebreos}: Gn 40:15.}; no fue siquiera ni el antepasado de todos los beduinos semitas que fueron retenidos cautivos en Egipto. Es verdad que cuando sus descendientes salieron de Egipto, formaron el núcleo del pueblo judío\footnote{\textit{Judíos}: 2 Re 16:6.} posterior, pero la inmensa mayoría de los hombres y mujeres que se unieron a los clanes de Israel no habían vivido nunca en Egipto. Se trataba simplemente de nómadas como ellos que escogieron seguir el liderazgo de Moisés cuando los hijos de Abraham y sus compañeros semitas de Egipto viajaban por el norte de Arabia.

\par
%\textsuperscript{(1055.2)}
\textsuperscript{96:2.4} La enseñanza de Melquisedek sobre El Elyón, el Altísimo, y la alianza del favor divino a través de la fe, se habían olvidado ampliamente en la época en que los egipcios esclavizaron a los pueblos semitas que pronto iban a formar la nación hebrea. Pero durante todo este período de cautividad, estos nómadas árabes conservaron una creencia tradicional sobreviviente en Yahvé, su deidad racial.

\par
%\textsuperscript{(1055.3)}
\textsuperscript{96:2.5} Más de cien tribus árabes diferentes adoraban a Yahvé, y a excepción del matiz existente en el concepto de El Elyón enseñado por Melquisedek, un concepto que sobrevivió entre las clases más instruidas de Egipto, incluyendo a los linajes hebreos y egipcios mezclados, la religión de la masa de esclavos hebreos cautivos era una versión modificada del antiguo ritual de magia y de sacrificios de Yahvé.

\section*{3. El incomparable Moisés}
\par
%\textsuperscript{(1055.4)}
\textsuperscript{96:3.1} El comienzo de la evolución de los conceptos y de los ideales hebreos acerca de un Creador Supremo data de la salida de Egipto de los semitas bajo la dirección de ese gran jefe, instructor y organizador llamado Moisés. Su madre pertenecía a la familia real de Egipto; su padre era un oficial de enlace semita entre el gobierno y los beduinos cautivos\footnote{\textit{Padres de Moisés}: Ex 2:1-10.}. Moisés poseía así unas cualidades procedentes de unos orígenes raciales superiores; su linaje estaba tan extremadamente mezclado que es imposible clasificarlo en un grupo racial determinado. Si no hubiera pertenecido a este tipo mixto, nunca hubiera demostrado la variedad de talentos y la adaptabilidad poco comunes que le permitieron dirigir a la horda diversificada que terminó por unirse a los beduinos semitas que huían de Egipto bajo su mando hacia el desierto de Arabia.

\par
%\textsuperscript{(1055.5)}
\textsuperscript{96:3.2} A pesar de los atractivos de la cultura del reino del Nilo, Moisés escogió compartir la suerte del pueblo de su padre. En la época en que este gran organizador estaba formulando sus planes para la liberación final del pueblo de su padre, los beduinos cautivos apenas tenían una religión digna de este nombre; carecían prácticamente de un verdadero concepto de Dios y no tenían esperanzas en el mundo.

\par
%\textsuperscript{(1055.6)}
\textsuperscript{96:3.3} Ningún jefe emprendió nunca la reforma y la elevación de un grupo de seres humanos más desesperados, abatidos, descorazonados e ignorantes. Pero estos esclavos contenían unas posibilidades latentes de desarrollo en sus linajes hereditarios, y Moisés había entrenado a un número suficiente de dirigentes instruidos como parte de los preparativos para que el día de la sublevación y del ataque por la libertad formaran un cuerpo de organizadores eficaces. Estos hombres superiores habían sido empleados como supervisores indígenas de su pueblo, y habían recibido cierta educación debido a la influencia de Moisés entre los dirigentes egipcios.

\par
%\textsuperscript{(1056.1)}
\textsuperscript{96:3.4} Moisés se esforzó por negociar diplomáticamente la libertad de sus compañeros semitas. Él y su hermano hicieron un pacto con el rey de Egipto por el cual se les concedía la autorización de abandonar pacíficamente el valle del Nilo para dirigirse al desierto de Arabia. Iban a recibir un modesto pago en dinero y mercancías como muestra de su largo servicio en Egipto. Los hebreos por su parte hicieron el acuerdo de mantener relaciones amistosas con los faraones y de no formar parte de ninguna alianza contra Egipto. Pero más tarde, el rey estimó conveniente rechazar este tratado, ofreciendo como razón la excusa de que sus espías habían descubierto que los esclavos beduinos eran desleales. Alegó que buscaban la libertad con la intención de dirigirse al desierto para organizar a los nómadas en contra de Egipto.

\par
%\textsuperscript{(1056.2)}
\textsuperscript{96:3.5} Pero Moisés no se desanimó; esperó su momento oportuno y, en menos de un año, cuando las fuerzas militares egipcias estaban totalmente ocupadas resistiendo los violentos ataques simultáneos de una fuerte ofensiva libia por el sur y de una invasión naval griega por el norte, este intrépido organizador condujo a sus compatriotas fuera de Egipto en una fuga nocturna espectacular\footnote{\textit{El éxodo}: Ex 14:8-29.}. Esta huida hacia la libertad fue planeada cuidadosamente y ejecutada con habilidad. Y tuvieron éxito, a pesar de que fueron seguidos de cerca por el faraón y un pequeño grupo de egipcios, los cuales cayeron todos ante las defensas de los fugitivos, dejándoles mucho botín, el cual aumentó debido al saqueo de la multitud de esclavos que avanzaban huyendo hacia su hogar ancestral en el desierto.

\section*{4. La proclamación de Yahvé}
\par
%\textsuperscript{(1056.3)}
\textsuperscript{96:4.1} La evolución y la elevación de la enseñanza de Moisés han influido sobre casi la mitad del mundo, y aún continúan influyendo incluso en el siglo veinte. Aunque Moisés comprendía la filosofía religiosa egipcia más avanzada, los esclavos beduinos sabían poco de estas enseñanzas, pero nunca habían olvidado por completo al dios del Monte Horeb, a quien sus antepasados habían llamado Yahvé.

\par
%\textsuperscript{(1056.4)}
\textsuperscript{96:4.2} Moisés había oído hablar de las enseñanzas de Maquiventa Melquisedek tanto por su padre como por su madre, y esta creencia religiosa común explica la unión insólita entre una mujer de sangre real y un hombre de una raza cautiva. El suegro de Moisés era un kenita adorador de El Elyón, pero los padres del emancipador creían en El Shaddai. Moisés fue educado pues como un el shaddaísta, pero debido a la influencia de su suegro se convirtió en un el elyonísta; y cuando los hebreos acamparon cerca del Monte Sinaí después de la huida de Egipto, había formulado un nuevo concepto ampliado de la Deidad (derivado de todas sus creencias anteriores), que decidió sabiamente proclamar a su pueblo como un concepto más desarrollado de Yahvé, su antiguo dios tribal.

\par
%\textsuperscript{(1056.5)}
\textsuperscript{96:4.3} Moisés se había esforzado por enseñar a estos beduinos la idea de El Elyón, pero antes de dejar Egipto se había convencido de que nunca comprenderían plenamente esta doctrina. Por esta razón, optó deliberadamente por el compromiso de adoptar a su dios tribal del desierto como el solo y único dios de sus seguidores. Moisés no enseñó específicamente que otros pueblos y naciones no pudieran tener otros dioses, pero mantuvo resueltamente, especialmente para los hebreos, que Yahvé estaba por encima de todos. Pero siempre se sintió atormentado por la difícil situación de tener que presentar a aquellos esclavos ignorantes su idea nueva y superior de la Deidad bajo la apariencia de la antigua denominación de Yahvé, el cual siempre había estado simbolizado por el becerro de oro de las tribus beduinas.

\par
%\textsuperscript{(1056.6)}
\textsuperscript{96:4.4} El hecho de que Yahvé fuera el Dios de los hebreos que huían explica por qué permanecieron tanto tiempo delante de la montaña sagrada del Sinaí, y por qué recibieron allí los Diez Mandamientos que Moisés promulgó en nombre de Yahvé, el dios del Horeb. Durante esta prolongada estancia delante del Sinaí, los ceremoniales religiosos del culto hebreo recién nacido fueron perfeccionados aún más.

\par
%\textsuperscript{(1057.1)}
\textsuperscript{96:4.5} No parece que Moisés hubiera logrado nunca establecer su culto ceremonial un tanto avanzado, ni mantener intactos a sus seguidores durante un cuarto de siglo, si no hubiera sido por la violenta erupción del Horeb durante la tercera semana de su estancia de adoración en la base del monte. <<La montaña de Yahvé se consumía en el fuego, y el humo subía como el humo de un horno, y toda la montaña temblaba enormemente>>\footnote{\textit{La montaña de Yahvé se consumía en el fuego}: Ex 19:18.}. En vista de este cataclismo, no es de sorprender que Moisés pudiera inculcar a sus hermanos la enseñanza de que su Dios era <<poderoso, terrible, un fuego devorador, temible y todopoderoso>>\footnote{\textit{Poderoso}: Sal 29:4. \textit{Poderoso, terrible}: Neh 9:32; Jer 20:11; Dt 7:21; Dt 10:17. \textit{Terrible}: Dt 28:58. \textit{Fuego devorador}: Ex 24:17; Is 29:6; Is 30:27,30.}.

\par
%\textsuperscript{(1057.2)}
\textsuperscript{96:4.6} Moisés proclamó que Yahvé era el Señor Dios de Israel, que había escogido a los hebreos como su pueblo elegido; estaba construyendo una nueva nación, y nacionalizó sabiamente sus enseñanzas religiosas diciendo a sus seguidores que Yahvé era muy estricto y exigente, un <<Dios celoso>>\footnote{\textit{Dios celoso}: Ex 20:5; Dt 6:15. \textit{Celo}: Ez 39:25; Jl 2:18; Zac 1:14; 8:2. \textit{Celoso de ...}: Ex 34:14; Nah 1:2; Dt 4:24; 5:9; Jos 24:19.}. Pero a pesar de todo, intentó ampliar su concepto de la divinidad cuando les enseñó que Yahvé era el <<Dios de los espíritus de todo el género humano>>\footnote{\textit{Dios de los espíritus de todo el género humano}: Nm 16:22; 27:16.}, y cuando dijo <<El Dios eterno es tu refugio, y por debajo de ti están los brazos eternos>>\footnote{\textit{El Dios eterno es mi refugio}: Dt 33:27. \textit{El Dios de Jacob es mi refugio}: Sal 46:7,11.}. Moisés enseñó que Yahvé era un Dios que mantenía su alianza; que <<no os abandonará, ni os destruirá, ni olvidará la alianza de vuestros padres, porque el Señor os ama y no olvidará el juramento que hizo a vuestros padres>>\footnote{\textit{Dios no os abandonará ni os destruirá}: Dt 4:31. \textit{No olvida la alianza}: Dt 7:8.}.

\par
%\textsuperscript{(1057.3)}
\textsuperscript{96:4.7} Moisés hizo un esfuerzo heroico por elevar a Yahvé a la dignidad de una Deidad suprema cuando lo presentó como el <<Dios de la verdad, sin iniquidad, justo y equitativo en toda su conducta>>\footnote{\textit{Dios de la verdad y sin iniquidad}: Dt 32:4.}. Y sin embargo, a pesar de esta enseñanza elevada, la comprensión limitada de sus seguidores hizo necesario que hablara de Dios a imagen y semejanza del hombre, como si estuviera sujeto a ataques de ira, cólera y severidad, e incluso que era vengativo y fácilmente influenciable por la conducta del hombre.

\par
%\textsuperscript{(1057.4)}
\textsuperscript{96:4.8} Gracias a las enseñanzas de Moisés, Yahvé, este dios tribal de la naturaleza, se convirtió en el Señor Dios de Israel, que siguió a los hebreos en el desierto e incluso en el exilio, donde pronto fue concebido como el Dios de todos los pueblos. La cautividad posterior que esclavizó a los judíos en Babilonia liberó finalmente el concepto evolutivo de Yahvé hasta asumir el papel monoteísta de Dios de todas las naciones.

\par
%\textsuperscript{(1057.5)}
\textsuperscript{96:4.9} La característica más singular y asombrosa de la historia religiosa de los hebreos es esta evolución continua del concepto de la Deidad, que empezó con el dios primitivo del Monte Horeb, avanzó gracias a las enseñanzas de sus dirigentes espirituales sucesivos, y llegó hasta el alto grado de desarrollo descrito en las doctrinas de los dos Isaías sobre la Deidad, los cuales proclamaron el magnífico concepto del Padre Creador amante y misericordioso.

\section*{5. Las enseñanzas de Moisés}
\par
%\textsuperscript{(1057.6)}
\textsuperscript{96:5.1} Moisés era una mezcla extraordinaria de jefe militar, organizador social y educador religioso. Fue el instructor y el jefe individual más importante del mundo entre la época de Maquiventa y la de Jesús. Moisés intentó introducir muchas reformas en Israel de las que no queda ningún registro escrito. En el espacio de una sola vida humana, sacó de la esclavitud y de un vagabundeo incivilizado a la horda políglota de los llamados hebreos, y sentó las bases para el nacimiento posterior de una nación y la perpetuación de una raza.

\par
%\textsuperscript{(1057.7)}
\textsuperscript{96:5.2} Existen muy pocos datos sobre la gran obra de Moisés porque los hebreos no tenían un lenguaje escrito en la época del éxodo. Los relatos de los tiempos y de las actividades de Moisés tuvieron su origen en las tradiciones que existían más de mil años después de la muerte de este gran dirigente.

\par
%\textsuperscript{(1058.1)}
\textsuperscript{96:5.3} Una gran parte de los progresos que Moisés aportó por encima de la religión de los egipcios y de las tribus levantinas circundantes se debieron a las tradiciones kenitas de la época de Melquisedek. Sin la enseñanza de Maquiventa a Abraham y a sus contemporáneos, los hebreos hubieran salido de Egipto en una ignorancia desesperante. Moisés y su suegro Jetro reunieron los restos de las tradiciones de los tiempos de Melquisedek, y estas enseñanzas, unidas a la erudición de los egipcios, guiaron a Moisés en la creación de la religión y el ritual más perfeccionados de los israelitas. Moisés era un organizador; seleccionó lo mejor que poseían la religión y las costumbres de Egipto y Palestina, asoció estas prácticas con las tradiciones de las enseñanzas de Melquisedek, y organizó el sistema ceremonial de adoración hebreo.

\par
%\textsuperscript{(1058.2)}
\textsuperscript{96:5.4} Moisés creía en la Providencia; estaba totalmente contaminado por las doctrinas egipcias sobre el control sobrenatural del Nilo y de los otros elementos de la naturaleza. Tenía una gran visión de Dios, pero era totalmente sincero cuando enseñó a los hebreos que si obedecían a Dios, <<os amará, os bendecirá y os multiplicará. Multiplicará el fruto de vuestro vientre y el fruto de vuestra tierra ---el trigo, el vino, el aceite y vuestros rebaños. Vuestra prosperidad será superior a la de todos los pueblos, y el Señor vuestro Dios apartará de vosotros toda enfermedad y no os impondrá ninguna de las plagas malignas de Egipto>>\footnote{\textit{Os amará, os bendecirá}: Dt 7:13-15.}. Moisés dijo incluso: <<Recordad al Señor vuestro Dios, porque él es el que os da el poder de conseguir las riquezas>>\footnote{\textit{Recordad al Señor vuestro Dios}: Dt 8:18.}. <<Prestaréis a muchas naciones, pero no pediréis prestado. Reinaréis sobre muchas naciones, pero ellas no reinarán sobre vosotros>>\footnote{\textit{Prestaréis a muchas naciones}: Dt 15:6.}.

\par
%\textsuperscript{(1058.3)}
\textsuperscript{96:5.5} Pero era realmente lastimoso observar a Moisés, este gran pensador, intentando adaptar su concepto sublime de El Elyón, el Altísimo, a la comprensión de los hebreos ignorantes y analfabetos. A sus dirigentes reunidos les decía con estruendo: <<El Señor vuestro Dios es un Dios único; no hay ninguno aparte de él>>\footnote{\textit{El Señor vuestro Dios es un Dios único}: Dt 6:4. \textit{No hay ninguno aparte de él}: Dt 4:35,39.}, mientras que a la multitud variopinta le preguntaba: <<¿Quién es igual a vuestro Dios entre todos los dioses?>>\footnote{\textit{Quién es como nuestro Dios entre todos}: Ex 15:11.} Moisés se alzó de una manera valiente y con un éxito parcial en contra de los fetiches y la idolatría, declarando: <<No visteis ninguna imagen el día que vuestro Dios os habló en el Horeb en medio del fuego>>\footnote{\textit{No visteis ninguna imagen el día}: Dt 4:15.}. También prohibió la realización de imágenes de todo tipo\footnote{\textit{El Señor es un único Dios}: 2 Re 19:19; 1 Cr 17:20; Neh 9:6; Sal 86:10; Eclo 36:5; Is 37:16; 44:6,8; 45:5-6,21; Mc 12:29,32; Jn 17:3; Ro 3:30; 1 Co 8:4-6; Gl 3:20; Ef 4:6; 1 Ti 2:5; Stg 2:19; 1 Sam 2:2; 2 Sam 7:22.}.

\par
%\textsuperscript{(1058.4)}
\textsuperscript{96:5.6} Moisés temía proclamar la misericordia de Yahvé, y prefirió atemorizar a su pueblo con el miedo a la justicia de Dios, diciendo: <<El Señor vuestro Dios es el Dios de los Dioses, el Señor de los Señores, un gran Dios, un Dios poderoso y terrible que no tiene consideración con los hombres>>\footnote{\textit{Dios de los Dioses, el Señor de los Señores}: Dt 10:17.}. Además, intentó controlar a los clanes turbulentos cuando afirmó que <<vuestro Dios mata cuando le desobedecéis; cura y da la vida cuando le obedecéis>>\footnote{\textit{Dios mata cuando desobedecéis}: Dt 28; 32:39.}. Pero Moisés enseñó a estas tribus que sólo se convertirían en el pueblo elegido de Dios\footnote{\textit{Pueblo elegido de Dios}: 1 Re 3:8; 1 Cr 17:21-22; Sal 33:12; 105:6,43; 135:4; Is 41:8-9; 43:20-21; 44:1; Dt 7:6; 14:2.} a condición de que <<guardaran todos sus mandamientos y obedecieran todos sus decretos>>\footnote{\textit{Guardar todos sus mandamientos}: Dt 7:11; 10:12-13; 12:25,28,32; 13:18.}.

\par
%\textsuperscript{(1058.5)}
\textsuperscript{96:5.7} Durante estos primeros tiempos, a los hebreos se les enseñó poco acerca de la misericordia de Dios. Se enteraron de que Dios era <<el Todopoderoso; el Señor es un guerrero, el Dios de las batallas, con un poder glorioso, que hace pedazos a sus enemigos>>\footnote{\textit{El Todopoderoso}: Gn 35:11. \textit{El Señor es un guerrero}: Ex 15:3. \textit{El Dios de las batallas}: Sal 24:8. \textit{Con un poder glorioso}: Ex 15:6. \textit{Hace pedazos a sus enemigos}: Ex 15:6.}. <<El Señor vuestro Dios camina en medio del campamento para liberaros>>\footnote{\textit{Camina en medio del campamento}: Dt 23:14.}. Los israelitas pensaban que su Dios era alguien que les amaba, pero que también había <<endurecido el corazón del faraón>>\footnote{\textit{Endurecido el corazón del faraón}: Ex 7:13.} y <<maldecido a sus enemigos>>\footnote{\textit{Maldecido a sus enemigos}: Dt 30:7.}.

\par
%\textsuperscript{(1058.6)}
\textsuperscript{96:5.8} Aunque Moisés presentó a los hijos de Israel un vislumbre fugaz de una Deidad universal y benéfica, su concepto cotidiano de Yahvé sólo era, en general, el de un Dios un poco mejor que los dioses tribales de los pueblos circundantes. Su concepto de Dios era primitivo, burdo y antropomórfico; cuando Moisés falleció, estas tribus beduinas volvieron rápidamente a las ideas semibárbaras de sus antiguos dioses del Horeb y del desierto. La visión ampliada y más sublime de Dios que Moisés presentaba de vez en cuando a sus dirigentes fue pronto perdida de vista, mientras que la mayoría de la gente volvió a la adoración de sus becerros de oro fetiches, el símbolo de Yahvé para los pastores palestinos.

\par
%\textsuperscript{(1059.1)}
\textsuperscript{96:5.9} Cuando Moisés entregó el mando de los hebreos a Josué, ya había reunido a miles de descendientes colaterales de Abraham, Najor, Lot y otras tribus emparentadas, y los había fustigado a convertirse en una nación de guerreros pastoriles capaces de sustentarse y de reglamentarse parcialmente.

\section*{6. El concepto de Dios después de la muerte de Moisés}
\par
%\textsuperscript{(1059.2)}
\textsuperscript{96:6.1} Después de la muerte de Moisés, su elevado concepto de Yahvé degeneró rápidamente. Josué y los dirigentes de Israel siguieron conservando las tradiciones mosaicas del Dios infinitamente sabio, benéfico y todopoderoso, pero la gente común volvió rápidamente a la antigua idea de Yahvé que tenían en el desierto. Este movimiento hacia atrás del concepto de la Deidad continuó aumentando bajo el gobierno sucesivo de los diversos jeques tribales, los llamados Jueces.

\par
%\textsuperscript{(1059.3)}
\textsuperscript{96:6.2} El hechizo de la personalidad extraordinaria de Moisés había mantenido viva en el corazón de sus seguidores la inspiración de un concepto cada vez más amplio de Dios; pero una vez que llegaron a las tierras fértiles de Palestina, estos pastores nómadas se convirtieron rápidamente en agricultores establecidos y en cierto modo tranquilos. Esta evolución de las costumbres de vida y este cambio de punto de vista religioso exigieron una transformación más o menos completa del carácter de la idea que tenían sobre la naturaleza de su Dios Yahvé. Durante la época en que empezó la transmutación del dios del desierto del Sinaí, austero, burdo, exigente y estruendoso, en el concepto que apareció más tarde de un Dios de amor, justicia y misericordia, los hebreos casi perdieron de vista las elevadas enseñanzas de Moisés. Estuvieron a punto de perder todo concepto de monoteísmo; casi perdieron la oportunidad de convertirse en el pueblo que serviría de eslabón fundamental para la evolución espiritual de Urantia, en el grupo que conservaría la enseñanza de Melquisedek sobre un solo Dios hasta la época de la encarnación de un Hijo donador de este Padre de todos.

\par
%\textsuperscript{(1059.4)}
\textsuperscript{96:6.3} Josué trató desesperadamente de mantener en la mente de los hombres de las tribus el concepto de un Yahvé supremo, que inducía a que se proclamara: <<Al igual que estuve con Moisés, estaré con vosotros; no os defraudaré ni os abandonaré>>\footnote{\textit{Al igual que estuve con Moisés}: Jos 1:5.}. Josué estimó necesario predicar un evangelio severo a su pueblo incrédulo, un pueblo demasiado dispuesto a creer en su antigua religión indígena, pero poco deseoso de avanzar en la religión de la fe y la rectitud. La idea central de la enseñanza de Josué fue: <<Yahvé es un Dios santo; es un Dios celoso; no perdonará vuestras transgresiones ni vuestros pecados>>\footnote{\textit{Yahvé es santo, celoso, Dios}: Jos 24:19.}. El concepto más elevado de esta época describía a Yahvé como un <<Dios de poder, de juicio y de justicia>>\footnote{\textit{Dios de poder}: Ex 9:15; Esd 8:22; Dt 9:29. \textit{Dios de poder y juicio}: Job 37:23. \textit{Dios del juicio}: Sal 33:5; Dt 1:17; 32:4; 2 Sam 22:23. \textit{Dios de justicia}: Sal 89:14; Dt 33:21.}.

\par
%\textsuperscript{(1059.5)}
\textsuperscript{96:6.4} Pero incluso en esta época sombría, un instructor solitario aparecía de vez en cuando para proclamar el concepto mosaico de la divinidad: <<Vosotros, hijos de la perversidad, no podéis servir al Señor, porque él es un Dios santo>>\footnote{\textit{Los perversos no pueden servir a Dios}: Jos 24:19.}. <<¿Será el hombre mortal más justo que Dios? ¿Será un hombre más puro que su Creador?>>\footnote{\textit{¿Es el hombre más que Dios?}: Job 4:17.}. <<¿Podéis encontrar a Dios, buscándolo? ¿Podéis descubrir al Todopoderoso en su perfección? Mirad, Dios es grande y no lo conocemos. Aunque toquemos al Todopoderoso, no podemos descubrirlo>>\footnote{\textit{¿Podéis encontrar a Dios?}: Job 11:7. \textit{Dios es grande}: Job 36:26. \textit{Aunque toquemos al Todopoderoso}: Job 37:23.}.

\section*{7. Los salmos y el Libro de Job}
\par
%\textsuperscript{(1060.1)}
\textsuperscript{96:7.1} Bajo la dirección de sus jeques y sacerdotes, los hebreos se establecieron de forma dispersa por Palestina. Pero pronto se dejaron llevar por las creencias ignorantes del desierto y se contaminaron con las prácticas religiosas menos avanzadas de los cananeos. Se volvieron idólatras y licenciosos, y su idea de la Deidad cayó muy por debajo de los conceptos egipcios y mesopotámicos sobre Dios que mantenían ciertos grupos salemitas supervivientes, y que están registrados en algunos salmos y en el llamado Libro de Job.

\par
%\textsuperscript{(1060.2)}
\textsuperscript{96:7.2} Los salmos son la obra de una veintena o más de autores; muchos de ellos fueron escritos por educadores egipcios y mesopotámicos. Durante estos tiempos en que el Levante adoraba a los dioses de la naturaleza, seguía existiendo un gran número de personas que creían en la supremacía de El Elyón, el Altísimo.

\par
%\textsuperscript{(1060.3)}
\textsuperscript{96:7.3} Ninguna colección de escritos religiosos expresa una riqueza de devoción y de ideas inspiradoras sobre Dios como el Libro de los Salmos. Al leer atentamente esta maravillosa colección de literatura piadosa, sería muy útil tomar en consideración la fuente y la cronología de cada himno aislado de alabanza y de adoración, teniendo en cuenta que ninguna otra colección individual abarca un período tan largo de tiempo. Este Libro de los Salmos es el registro de los conceptos variables sobre Dios que albergaban los creyentes de la religión de Salem en todo el Levante, y abarca todo el período existente entre Amenemope e Isaías. En los salmos se representa a Dios en todas las fases de concepción, desde la idea rudimentaria de una deidad tribal hasta el ideal sumamente desarrollado de los hebreos más tardíos, donde se describe a Yahvé como un soberano amoroso y un Padre misericordioso.

\par
%\textsuperscript{(1060.4)}
\textsuperscript{96:7.4} Considerados de esta manera, este grupo de salmos constituye la gama más valiosa y útil de sentimientos piadosos que los hombres hayan reunido jamás hasta la época del siglo veinte. El espíritu de adoración de esta colección de himnos trasciende al de todos los otros libros sagrados del mundo.

\par
%\textsuperscript{(1060.5)}
\textsuperscript{96:7.5} La imagen variada de la Deidad que se presenta en el Libro de Job es el producto de más de veinte educadores religiosos de Mesopotamia a lo largo de un período de casi trescientos años. Cuando leáis los conceptos elevados de la divinidad que se encuentran en esta compilación de creencias mesopotámicas, reconoceréis que en las cercanías de Ur, en Caldea, fue donde la idea de un Dios real se conservó mejor durante la edad de las tinieblas en Palestina.

\par
%\textsuperscript{(1060.6)}
\textsuperscript{96:7.6} Los palestinos captaron a menudo la sabiduría y la omnipresencia de Dios, pero raras veces su amor y su misericordia. El Yahvé de estos tiempos <<envía a los espíritus malignos para que dominen el alma de sus enemigos>>\footnote{\textit{Envía a los espíritus malignos}: Jue 9:23; 1 Sam 16:14-16.}; favorece a sus propios hijos obedientes, mientras que maldice e inflige terribles castigos a todos los demás. <<Frustra los proyectos de los astutos; coge a los hábiles en sus propios engaños>>\footnote{\textit{Frustra los proyectos de los astutos}: Job 5:12-13.}.

\par
%\textsuperscript{(1060.7)}
\textsuperscript{96:7.7} Solamente en Ur se elevó una voz para pregonar la misericordia de Dios, diciendo: <<Orará a Dios y encontrará su favor y verá su rostro con alegría, porque Dios concederá al hombre la rectitud divina>>\footnote{\textit{Orará a Dios}: Job 33:26.}. La salvación, el favor divino, por la fe, se predica así desde Ur: <<Es misericordioso con el que se arrepiente, y dice: `Líbralo de bajar al infierno, porque he encontrado una redención'. Si alguien dice: `He pecado y he pervertido lo que era justo, y no me ha beneficiado', Dios impedirá que su alma vaya al infierno, y verá la luz>>\footnote{\textit{Es misericordioso con el que se arrepiente}: Job 33:24. \textit{Si alguien dice: `He pecado'}: Job 33:27-28.}. Desde los tiempos de Melquisedek, el mundo levantino no había oído un mensaje tan sonoro y esperanzador de salvación humana como esta enseñanza extraordinaria de Eliju\footnote{\textit{Eliju, profeta de Ur y sacerdote}: Job 32:2ff.}, profeta de Ur y sacerdote de los creyentes salemitas, es decir, de los restos de la antigua colonia de Melquisedek en Mesopotamia.

\par
%\textsuperscript{(1061.1)}
\textsuperscript{96:7.8} Así es como los misioneros de Salem que quedaban en Mesopotamia mantuvieron la luz de la verdad durante el período de la desorganización de los pueblos hebreos, hasta que apareció el primero de la larga serie de instructores de Israel, que nunca se detuvieron en su construcción, concepto tras concepto, hasta que consiguieron hacer realidad el ideal del Padre Universal y Creador de todos, la cumbre de la evolución del concepto de Yahvé.

\par
%\textsuperscript{(1061.2)}
\textsuperscript{96:7.9} [Presentado por un Melquisedek de Nebadon.]


\chapter{Documento 97. La evolución del concepto de Dios entre los hebreos}
\par
%\textsuperscript{(1062.1)}
\textsuperscript{97:0.1} LOS dirigentes espirituales de los hebreos llevaron a cabo lo que nadie había logrado nunca realizar antes que ellos ---desantropomorfizar su concepto de Dios, sin convertirlo en una abstracción de la Deidad comprensible únicamente por los filósofos. Incluso la gente corriente era capaz de considerar el concepto maduro de Yahvé como un Padre, si no del individuo, al menos de la raza.

\par
%\textsuperscript{(1062.2)}
\textsuperscript{97:0.2} Aunque el concepto de la personalidad de Dios había sido enseñado claramente en Salem en la época de Melquisedek, era vago e impreciso en el momento de la huida de Egipto, y sólo evolucionó gradualmente en la mente hebrea, de generación en generación, en respuesta a las enseñanzas de los dirigentes espirituales. La percepción de la personalidad de Yahvé siguió una evolución progresiva mucho más continua que la de cualquier otro atributo de la Deidad. Desde Moisés hasta Malaquías, en la mente hebrea se produjo un crecimiento casi ininterrumpido de las ideas sobre la personalidad de Dios, y este concepto fue finalmente realzado y glorificado por las enseñanzas de Jesús sobre el Padre que está en los cielos.

\section*{1. Samuel ---el primer profeta hebreo}
\par
%\textsuperscript{(1062.3)}
\textsuperscript{97:1.1} La presión hostil de los pueblos que rodeaban a Palestina enseñó muy pronto a los jeques hebreos que no podían esperar sobrevivir a menos que confederaran sus organizaciones tribales en un gobierno centralizado. Y esta centralización de la autoridad administrativa proporcionó a Samuel\footnote{\textit{Samuel, el primer profeta hebreo}: Hch 3:24.} una mejor ocasión para ejercer como instructor y reformador.

\par
%\textsuperscript{(1062.4)}
\textsuperscript{97:1.2} Samuel surgió de una larga serie de educadores salemitas que habían continuado manteniendo las verdades de Melquisedek como una parte de sus formas de culto. Este instructor era un hombre enérgico y resuelto. Únicamente su gran devoción, unida a su extraordinaria determinación, le permitieron resistir la oposición casi universal que encontró cuando empezó a llevar de nuevo a todo Israel a la adoración del Yahvé supremo de la época de Moisés. E incluso entonces sólo tuvo un éxito parcial; sólo recuperó para el servicio del concepto superior de Yahvé a la mitad más inteligente de los hebreos; la otra mitad continuó adorando a los dioses tribales del país y manteniendo sus conceptos inferiores de Yahvé.

\par
%\textsuperscript{(1062.5)}
\textsuperscript{97:1.3} Samuel era un tipo de hombre tosco\footnote{\textit{Tosco y de acción}: 1 Sam 7:3-4.}, un reformador práctico capaz de salir un día con sus compañeros y derribar una veintena de lugares reservados a Baal. Los progresos que consiguió se debieron a la pura fuerza de la coacción; predicó poco, enseñó aún menos, pero sí actuó. Un día se burlaba del sacerdote de Baal, y al día siguiente despedazaba a un rey cautivo\footnote{\textit{Despedazar a un rey cautivo}: 1 Sam 15:32-33.}. Creía con devoción en el Dios único, y tenía un concepto claro de ese Dios único como creador del cielo y de la Tierra: <<Las columnas de la Tierra pertenecen al Señor, y ha puesto al mundo sobre ellas>>\footnote{\textit{Las columnas de la Tierra pertenecen al Señor}: 1 Sam 2:8.}.

\par
%\textsuperscript{(1063.1)}
\textsuperscript{97:1.4} Pero la gran contribución que Samuel hizo al desarrollo del concepto de la Deidad fue su declaración resonante de que Yahvé era \textit{invariable}, de que personificaba constantemente la misma perfección y divinidad infalibles. En aquella época se concebía a Yahvé como un Dios caprichoso lleno de antojos envidiosos, lamentándose siempre de haber hecho esto o aquello. Pero ahora, por primera vez desde que habían salido de Egipto, los hebreos escuchaban estas palabras sorprendentes: <<La Fuerza de Israel no miente ni se arrepiente, porque no es un hombre que tenga que arrepentirse>>\footnote{\textit{La Fuerza de Israel no miente ni se arrepiente}: 1 Sam 15:29.}. La estabilidad en las relaciones con la Divinidad se había proclamado. Samuel reiteró la alianza de Melquisedek con Abraham y afirmó que el Señor Dios de Israel era la fuente de toda verdad, estabilidad y constancia. Los hebreos siempre habían considerado a su Dios como un hombre, un superhombre, un espíritu elevado de origen desconocido; pero ahora escuchaban cómo el antiguo espíritu del Horeb era ensalzado como un Dios inmutable en su perfección creadora. Samuel ayudó a que el concepto evolutivo de Dios se elevara muy por encima del estado cambiante de la mente de los hombres y de las vicisitudes de la existencia mortal. Gracias a su enseñanza, el Dios de los hebreos empezó a ascender desde una idea parecida a la de los dioses tribales hasta el ideal del Creador y \textit{Supervisor} todopoderoso e invariable de toda la creación.

\par
%\textsuperscript{(1063.2)}
\textsuperscript{97:1.5} Predicó de nuevo el concepto de la sinceridad de Dios, de su fiabilidad en el mantenimiento de sus alianzas. Samuel dijo: <<El Señor no abandonará a su pueblo>>\footnote{\textit{El Señor no abandonará a su pueblo}: 1 Sam 12:22.}. <<Ha hecho con nosotros una alianza perpetua, ordenada y segura en todas las cosas>>\footnote{\textit{Ha hecho con nosotros una alianza}: 2 Sam 23:5.}. Así es como resonaba en toda Palestina la llamada para volver a adorar al Yahvé supremo. Este enérgico educador proclamaba constantemente: <<Eres grande, oh Señor Dios, pues no hay nadie como tú, ni tampoco hay ningún Dios aparte de ti>>\footnote{\textit{Eres grande, oh Señor Dios}: 2 Sam 7:22.}.

\par
%\textsuperscript{(1063.3)}
\textsuperscript{97:1.6} Hasta ese momento, los hebreos habían considerado el favor de Yahvé principalmente en términos de prosperidad material. Cuando Samuel se atrevió a hacer la proclamación siguiente, produjo una gran conmoción en Israel, y casi le cuesta la vida: <<El Señor enriquece y empobrece; humilla y eleva. Levanta del polvo a los pobres y eleva a los mendigos para colocarlos entre los príncipes y hacerles heredar el trono de la gloria>>\footnote{\textit{El Señor enriquece y empobrece}: 1 Sam 2:7-8.}. Unas promesas tan alentadoras para los humildes y los menos afortunados no se habían proclamado desde los tiempos de Moisés, y miles de desesperados, entre los pobres, empezaron a tener la esperanza de que podían mejorar su estado espiritual.

\par
%\textsuperscript{(1063.4)}
\textsuperscript{97:1.7} Pero Samuel no progresó mucho más allá del concepto de un dios tribal. Proclamó a un Yahvé que había creado a todos los hombres, pero que se ocupaba principalmente de los hebreos, su pueblo elegido. Incluso así, al igual que en los tiempos de Moisés, el concepto de Dios describía una vez más a una Deidad santa y justa. <<No hay nadie tan santo como el Señor. ¿Quién puede ser comparado con este santo Señor Dios?>>\footnote{\textit{No hay nadie tan santo como el Señor}: 1 Sam 2:2. \textit{¿Quién puede ser comparado?}: Ex 15:11; Sal 89:6.}

\par
%\textsuperscript{(1063.5)}
\textsuperscript{97:1.8} A medida que pasaban los años, el viejo dirigente entrecano progresó en su comprensión de Dios, pues declaró: <<El Señor es un Dios de conocimiento, y él es el que pesa las acciones. El Señor juzgará los confines de la Tierra, mostrando misericordia a los misericordiosos, y también será justo con el hombre justo>>\footnote{\textit{Dios de conocimiento}: 1 Sam 2:3. \textit{El Señor juzgará la Tierra}: 1 Sam 2:10. \textit{Mostrando misericordia a los misericordiosos}: 2 Sam 22:26.}. Aquí se encuentran ya los albores de la misericordia, aunque limitada a aquellos que son misericordiosos. Posteriormente avanzó un paso más cuando exhortó a su pueblo en la adversidad: <<Pongámonos ahora en manos del Señor, porque su compasión es grande>>\footnote{\textit{Pongámonos en manos del Señor}: 2 Sam 24:14.}. <<El Señor no tiene ninguna limitación para salvar a muchos o a pocos>>\footnote{\textit{No tiene limitación para salvar a muchos o a pocos}: 1 Sam 14:6.}.

\par
%\textsuperscript{(1063.6)}
\textsuperscript{97:1.9} Este desarrollo gradual del concepto del carácter de Yahvé continuó bajo el ministerio de los sucesores de Samuel. Intentaron presentar a Yahvé como un Dios que cumplía sus alianzas, pero apenas mantuvieron el ritmo marcado por Samuel; no lograron desarrollar la idea de la misericordia de Dios tal como Samuel la había concebido en sus últimos años. Se produjo un retroceso continuo hacia el reconocimiento de otros dioses, a pesar de mantener que Yahvé estaba por encima de todos. <<Tuyo es el reino, oh Señor, y eres ensalzado como jefe por encima de todos>>\footnote{\textit{Tuyo es el reino, oh Señor}: 1 Cr 29:11b.}.

\par
%\textsuperscript{(1064.1)}
\textsuperscript{97:1.10} La idea central de esta época era el poder divino; los profetas de estos tiempos predicaban una religión destinada a favorecer al rey que estaba en el trono hebreo. <<Tuya es, oh Señor, la grandeza, el poder, la gloria, la victoria y la majestad. En tu mano se encuentra el poder y la fuerza, y tú puedes engrandecer y fortalecer a todos>>\footnote{\textit{Tuya es la grandeza}: 1 Cr 29:11a. \textit{En tu mano se encuentra el poder}: 1 Cr 29:12.}. Éste era el estado del concepto de Dios durante la época de Samuel y de sus sucesores inmediatos.

\section*{2. Elías y Eliseo}
\par
%\textsuperscript{(1064.2)}
\textsuperscript{97:2.1} En el siglo décimo antes de Cristo, la nación hebrea se dividió en dos reinos. En estas dos divisiones políticas, muchos instructores de la verdad se esforzaron por detener la marea reaccionaria de decadencia espiritual que había empezado a subir, y que continuó desastrosamente después de la guerra de separación. Pero estos esfuerzos por hacer progresar la religión hebrea no prosperaron hasta que Elías\footnote{\textit{Elías comienza sus enseñanzas}: 1 Re 17:1.}, el guerrero resuelto y audaz de la rectitud, empezó sus enseñanzas. Elías restableció en el reino del norte un concepto de Dios comparable al que había existido en los tiempos de Samuel. Elías dispuso de pocas ocasiones para presentar un concepto avanzado de Dios; al igual que Samuel antes que él, estaba muy ocupado derribando los altares de Baal\footnote{\textit{Enemigo de Baal}: 1 Re 18:40.} y destruyendo los ídolos de los falsos dioses. Y llevó adelante sus reformas a pesar de la oposición de un monarca idólatra\footnote{\textit{Monarca idólatra}: 1 Re 16:30-33; 1 Re 21:25-26.}; su tarea fue aún más gigantesca y difícil que la que Samuel había afrontado.

\par
%\textsuperscript{(1064.3)}
\textsuperscript{97:2.2} Cuando Elías fue llamado a otro lugar\footnote{\textit{La marcha de Elías}: 2 Re 2:1-15.}, Eliseo\footnote{\textit{Eliseo}: 1 Re 19:16,19-20.}, su fiel compañero, se encargó de su obra, y con la ayuda inestimable de Miqueas\footnote{\textit{Miqueas}: 1 Re 22:7-28; 2 Cr 18:6-8.}, un profeta poco conocido, mantuvo viva la luz de la verdad en Palestina.

\par
%\textsuperscript{(1064.4)}
\textsuperscript{97:2.3} Pero ésta no fue una época de progreso en el concepto de la Deidad. Los hebreos ni siquiera se habían elevado todavía a la altura del ideal de Moisés. La era de Elías y Eliseo se cerró con el regreso de las mejores clases de hebreos a la adoración del Yahvé supremo, y presenció cómo se restablecía la idea del Creador Universal en el punto aproximado en que Samuel la había dejado.

\section*{3. Yahvé y Baal}
\par
%\textsuperscript{(1064.5)}
\textsuperscript{97:3.1} La controversia interminable entre los creyentes en Yahvé y los seguidores de Baal era un conflicto socioeconómico de ideologías, más bien que una diferencia de creencias religiosas.

\par
%\textsuperscript{(1064.6)}
\textsuperscript{97:3.2} Los habitantes de Palestina tenían actitudes diferentes en cuanto a la propiedad privada de la tierra. Las tribus meridionales o errantes de Arabia (los yahveítas) consideraban la tierra como algo inalienable ---como un don de la Deidad al clan. Estimaban que la tierra no se podía vender ni hipotecar. <<Yahvé habló y dijo: `La tierra no se venderá, porque la tierra me pertenece'>>\footnote{\textit{Yahvé ordena no vender la tierra}: Lv 25:23.}.

\par
%\textsuperscript{(1064.7)}
\textsuperscript{97:3.3} Los cananeos del norte, más establecidos, (los baalitas) compraban, vendían e hipotecaban libremente sus tierras. La palabra Baal significa propietario. El culto de Baal estaba basado en dos doctrinas principales: primero, la validación del intercambio, los contratos y los pactos sobre la propiedad ---el derecho a comprar y vender las tierras; y segundo, se suponía que Baal enviaba la lluvia--- era el dios de la fertilidad del suelo. Las buenas cosechas dependían del favor de Baal. El culto estaba ampliamente relacionado con la \textit{tierra}, su posesión y su fertilidad.

\par
%\textsuperscript{(1065.1)}
\textsuperscript{97:3.4} Los baalitas poseían generalmente casas, tierras y esclavos. Eran los propietarios aristócratas y vivían en las ciudades. Cada Baal tenía su lugar sagrado, su clero y sus <<santas mujeres>>, las prostitutas rituales.

\par
%\textsuperscript{(1065.2)}
\textsuperscript{97:3.5} Los profundos antagonismos en las actitudes sociales, económicas, morales y religiosas que manifestaban los cananeos y los hebreos se produjeron a causa de esta diferencia fundamental relacionada con la tierra. Esta controversia socioeconómica no se convirtió en un asunto claramente religioso hasta la época de Elías. A partir de los tiempos de este dinámico profeta, el asunto se resolvió luchando en un campo más estrictamente religioso ---Yahvé contra Baal\footnote{\textit{Yahvé contra Baal}: 1 Re 18:17-40.}--- y terminó con la victoria de Yahvé y el impulso posterior hacia el monoteísmo.

\par
%\textsuperscript{(1065.3)}
\textsuperscript{97:3.6} Elías trasladó la controversia entre Yahvé y Baal desde la cuestión de las tierras al aspecto religioso de las ideologías hebrea y cananea. Cuando Ajab asesinó a los Nabot\footnote{\textit{Asesinato de los Nabot}: 1 Re 21:1-16.} en el transcurso de la intriga para conseguir sus tierras, Elías convirtió las antiguas costumbres sobre las tierras en un problema moral y lanzó su vigorosa campaña contra los baalitas. Fue también una lucha de la gente del campo contra la dominación que ejercían las ciudades. Yahvé se convirtió en Elohim principalmente bajo la influencia de Elías. El profeta empezó como reformador agrario y terminó realzando a la Deidad. Había muchos Baales, pero Yahvé era \textit{uno solo} ---el monoteísmo triunfó sobre el politeísmo.

\section*{4. Amós y Oseas}
\par
%\textsuperscript{(1065.4)}
\textsuperscript{97:4.1} Amós franqueó una etapa importante en la transición entre el dios tribal ---el dios al que habían servido durante tanto tiempo mediante sacrificios y ceremonias, el Yahvé de los primeros hebreos--- y un Dios que castigaría el crimen y la inmoralidad incluso de su propio pueblo. Amós apareció procedente de las colinas del sur para denunciar la criminalidad, la embriaguez, la opresión y la inmoralidad de las tribus del norte. Desde los tiempos de Moisés no se habían proclamado unas verdades tan resonantes en Palestina.

\par
%\textsuperscript{(1065.5)}
\textsuperscript{97:4.2} Amós no se limitó simplemente a restaurar o a reformar; descubrió también unos nuevos conceptos de la Deidad. Proclamó muchas cosas sobre Dios que habían sido anunciadas por sus predecesores, y atacó valientemente la creencia en un Ser Divino que aprobara el pecado de su propio pueblo llamado elegido. Por primera vez desde la época de Melquisedek, los oídos humanos escucharon la denuncia del doble criterio de la justicia y la moralidad nacionales. Los oídos hebreos escucharon por primera vez en su historia que su propio Dios, Yahvé, ya no toleraría el crimen y el pecado en sus vidas, como tampoco lo toleraría en cualquier otro pueblo. Amós imaginó al Dios severo y justo de Samuel y Elías, pero también vio a un Dios que no consideraba a los hebreos de manera diferente a cualquier otra nación cuando se trataba de castigar la maldad. Era un ataque directo contra la doctrina egoísta del <<pueblo elegido>>\footnote{\textit{Pueblo elegido}: 1 Re 3:8; 1 Cr 17:21-22; Sal 33:12; 105:6,43; 135:4; Is 41:8-9; 43:20-21; 44:1; Dt 7:6; 14:2.}, y muchos hebreos de aquella época se sintieron enormemente ofendidos.

\par
%\textsuperscript{(1065.6)}
\textsuperscript{97:4.3} Amós dijo: <<Buscad al que ha formado las montañas y ha creado el viento, al que ha formado las siete estrellas y Orión, que transforma la sombra de la muerte en un amanecer, y pone el día tan oscuro como la noche>>\footnote{\textit{El que ha formado las montañas}: Am 4:13. \textit{El que creado los cielos}: Am 5:8.}. Al denunciar a sus contemporáneos semirreligiosos, oportunistas y a veces inmorales, intentó describir la justicia inexorable de un Yahvé invariable cuando dijo de los malhechores: <<Aunque se hundan en el infierno, allí los cogeré; aunque suban trepando a los cielos, los haré bajar de allí>>\footnote{\textit{Aunque se hundan en el infierno}: Am 9:2.}. <<Y aunque vayan al cautiverio delante de sus enemigos, allí dirigiré la espada de la justicia, y ella los matará>>\footnote{\textit{Y aunque vayan al cautiverio}: Am 9:4.}. Amós asustó aún más a sus oyentes cuando los señaló con un dedo acusador y reprobatorio, y declaró en nombre de Yahvé: <<Estad seguros de que nunca olvidaré ninguna de vuestras obras>>\footnote{\textit{No olvidaré vuestras obras}: Am 8:7.}. <<Y pasaré por la criba a la casa de Israel entre todas las naciones, como el trigo se criba en un tamiz>>\footnote{\textit{Pasaré por la criba a la casa de Israel}: Am 9:9.}.

\par
%\textsuperscript{(1066.1)}
\textsuperscript{97:4.4} Amós proclamó que Yahvé era el <<Dios de todas las naciones>> y advirtió a los israelitas que el ritual no debía sustituir a la rectitud\footnote{\textit{Ritual contra rectitud}: Am 5:21-24.}. Antes de que este valiente educador fuera lapidado, había difundido suficiente levadura de la verdad como para salvar la doctrina del Yahvé supremo; había asegurado la evolución ulterior de la revelación de Melquisedek.

\par
%\textsuperscript{(1066.2)}
\textsuperscript{97:4.5} Oseas siguió a Amós y a su doctrina de un Dios universal de justicia resucitando el concepto mosaico de un Dios de amor. Oseas predicó el perdón a través del arrepentimiento, y no por medio del sacrificio\footnote{\textit{Arrepentimiento, no sacrificio}: Os 6:6.}. Proclamó un evangelio de bondad y de misericordia divina, diciendo: <<Os desposaré conmigo para siempre; sí, os desposaré conmigo en rectitud y en juicio, en bondad y en misericordia. Incluso os desposaré conmigo en fidelidad>>\footnote{\textit{Os desposaré conmigo para siempre}: Os 2:19-20.}. <<Los amaré abundantemente, pues mi cólera se ha desviado>>\footnote{\textit{Los amaré abundantemente, sin cólera}: Os 14:4.}.

\par
%\textsuperscript{(1066.3)}
\textsuperscript{97:4.6} Oseas continuó fielmente las advertencias morales de Amós, diciendo de Dios: <<Los castigaré cuando lo desee>>\footnote{\textit{Los castigaré cuando lo desee}: Os 10:10.}. Pero los israelitas consideraron como una crueldad que rayaba en la traición las palabras que dijo: <<Diré a aquellos que no eran mi pueblo: `Vosotros sois mi pueblo', y ellos dirán: `Tú eres nuestro Dios'>>\footnote{\textit{Diré a aquellos que no son mi pueblo}: Os 2:23.}. Continuó predicando el arrepentimiento y el perdón, diciendo: <<Yo curaré su apostasía; los amaré abundantemente, pues mi cólera se ha desviado>>\footnote{\textit{Yo curaré su apostasía}: Os 14:4.}. Oseas proclamó constantemente la esperanza y el perdón. La idea central de su mensaje fue siempre: <<Tendré misericordia de mi pueblo. No conocerán a ningún Dios salvo a mí, porque no hay ningún salvador aparte de mí>>\footnote{\textit{Tendré misericordia}: Os 1:7. \textit{No conocerán a ningún Dios salvo a mí}: Os 13:4.}.

\par
%\textsuperscript{(1066.4)}
\textsuperscript{97:4.7} Amós estimuló la conciencia nacional de los hebreos para que reconocieran que Yahvé no perdonaría ni el crimen ni el pecado entre ellos porque fueran supuestamente el pueblo elegido, mientras que Oseas hizo sonar las notas de apertura en los acordes misericordiosos posteriores de la compasión y la bondad divinas, que fueron cantados de manera tan exquisita por Isaías y sus compañeros.

\section*{5. El primer Isaías}
\par
%\textsuperscript{(1066.5)}
\textsuperscript{97:5.1} Ésta fue una época en que algunos proclamaban amenazas de castigo para los pecados personales y los crímenes nacionales de los clanes del norte, mientras que otros predecían calamidades como castigo por las transgresiones del reino del sur. Después de este despertar de la conciencia y del conocimiento en las naciones hebreas, el primer Isaías hizo su aparición.

\par
%\textsuperscript{(1066.6)}
\textsuperscript{97:5.2} Isaías continuó predicando la naturaleza eterna de Dios, su sabiduría infinita, la fiabilidad de su perfección invariable. Representó al Dios de Israel, diciendo: <<El juicio lo pondré también como vara de medir, y la rectitud como plomada>>\footnote{\textit{El juicio lo pondré como vara de medir}: Is 28:17.}. <<El Señor os hará descansar de vuestras penas, de vuestros miedos, y de la dura servidumbre en la que el hombre ha sido puesto>>\footnote{\textit{El Señor os hará descansar}: Is 14:3.}. <<Vuestros oídos escucharán una palabra detrás de vosotros, diciendo: `éste es el camino, seguidlo'>>\footnote{\textit{Vuestros oídos escucharán `éste es el camino'}: Is 30:21.}. <<Mirad, Dios es mi salvación; confiaré y no tendré miedo, porque el Señor es mi fuerza y mi canción>>\footnote{\textit{Mirad, Dios es mi salvación}: Is 12:2.}. <<`Venid ahora y razonemos juntos', dice el Señor: si vuestros pecados son como la escarlata, se volverán tan blancos como la nieve; si son rojos como el carmesí, se volverán como la lana'>>\footnote{\textit{Venid ahora y razonemos juntos}: Is 1:18.}.

\par
%\textsuperscript{(1066.7)}
\textsuperscript{97:5.3} Hablándole a las almas hambrientas de los hebreos dominados por el miedo, este profeta dijo: <<Levantaos y resplandeced, porque vuestra luz ha llegado, y la gloria del Señor se ha alzado sobre vosotros>>\footnote{\textit{Levantaos y resplandeced, la luz ha llegado}: Is 60:1.}. <<El espíritu del Señor está en mí porque me ha ungido para que predique la buena nueva a los mansos; me ha enviado para vendar a los que tienen el corazón destrozado, para proclamar la libertad a los cautivos y la apertura de las prisiones a los que están atados>>\footnote{\textit{El espíritu del Señor está en mí}: Is 61:1.}. <<Me regocijaré profundamente en el Señor, mi alma estará contenta en mi Dios, porque me ha vestido con las ropas de la salvación y me ha cubierto con su manto de rectitud>>\footnote{\textit{Me regocijaré profundamente en el Señor}: Is 61:10.}. <<En todas sus aflicciones, él estaba afligido, y el ángel de su presencia los salvó. Con su amor y su compasión los ha redimido>>\footnote{\textit{El comparte las aflicciones}: Is 63:9.}.

\par
%\textsuperscript{(1067.1)}
\textsuperscript{97:5.4} Este Isaías fue seguido de Miqueas y Abdías, que confirmaron y embellecieron su evangelio que satisfacía el alma. Estos dos valientes mensajeros denunciaron audazmente el ritual de los hebreos, dominado por los sacerdotes, y atacaron intrépidamente todo el sistema sacrificatorio.

\par
%\textsuperscript{(1067.2)}
\textsuperscript{97:5.5} Miqueas criticó a <<los jefes que juzgan por una recompensa, los sacerdotes que enseñan por un salario y los profetas que adivinan por dinero>>\footnote{\textit{Los jefes que juzgan por una recompensa}: Miq 3:11.}. Enseñó la llegada de un día en que se estaría libre de las supersticiones y del clericalismo, diciendo: <<Cada hombre se sentará debajo de su propia vid, y nadie le infundirá temor, porque cada cual vivirá de acuerdo con su comprensión de Dios>>\footnote{\textit{Cada hombre se sentará debajo de su propia vid}: Miq 4:4-5.}.

\par
%\textsuperscript{(1067.3)}
\textsuperscript{97:5.6} La idea central del mensaje de Miqueas fue siempre: <<¿Me presentaré ante Dios con holocaustos? ¿Le agradarán al Señor mil carneros o diez mil ríos de aceite? ¿Entregaré a mi primogénito por mi transgresión, el fruto de mi cuerpo por el pecado de mi alma? Él me ha mostrado, oh hombre, lo que es bueno; y qué exige el Señor de vosotros sino que actuéis con justicia, que améis la misericordia y que caminéis humildemente con vuestro Dios>>\footnote{\textit{¿Me presentaré ante Dios con holocaustos?}: Miq 6:6-8.}. Fue una gran época; fueron en verdad unos tiempos de grandes cambios durante los cuales los hombres mortales escucharon, y algunos incluso creyeron, estos mensajes emancipadores hace más de dos milenios y medio. Y si no hubiera sido por la resistencia obstinada de los sacerdotes, estos educadores habrían eliminado todo el ceremonial sangriento del ritual de adoración de los hebreos.

\section*{6. Jeremías el intrépido}
\par
%\textsuperscript{(1067.4)}
\textsuperscript{97:6.1} Aunque diversos instructores continuaron exponiendo el evangelio de Isaías, le perteneció a Jeremías dar el siguiente paso audaz en la internacionalización de Yahvé, Dios de los hebreos.

\par
%\textsuperscript{(1067.5)}
\textsuperscript{97:6.2} Jeremías declaró intrépidamente que Yahvé no estaba del lado de los hebreos en sus contiendas militares con otras naciones\footnote{\textit{Dios no estaba con los hebreos en la guerra}: Jer 21:3-7.}. Afirmó que Yahvé era el Dios de toda la Tierra, de todas las naciones y de todos los pueblos\footnote{\textit{Dios de todas las naciones}: Jer 10:6-7. \textit{Dios de todos los pueblos}: Jer 32:27.}. La enseñanza de Jeremías representó el crescendo del movimiento ascendente hacia la internacionalización del Dios de Israel; este intrépido predicador proclamó de una vez por todas que Yahvé era el Dios de todas las naciones, y que no existía ni Osiris para los egipcios, ni Belo para los babilonios, ni Asur para los asirios, ni Dagón para los filisteos. La religión de los hebreos participó así en el renacimiento del monoteísmo que tuvo lugar en todo el mundo alrededor de esta época y después de ella; por fin, el concepto de Yahvé se había elevado a un nivel de Deidad de dignidad planetaria e incluso cósmica. Pero muchos compañeros de Jeremías encontraron difícil concebir a Yahvé separado de la nación hebrea.

\par
%\textsuperscript{(1067.6)}
\textsuperscript{97:6.3} Jeremías predicó también sobre el Dios justo y amoroso descrito por Isaías, declarando: <<Sí, os he amado con un amor eterno; por eso os he atraído con mi bondad>>\footnote{\textit{Amados con un amor eterno}: Jer 31:3.}. <<Pues él no aflige voluntariamente a los hijos de los hombres>>\footnote{\textit{Él no aflige voluntariamente}: Lm 3:33.}.

\par
%\textsuperscript{(1067.7)}
\textsuperscript{97:6.4} Este intrépido profeta dijo: <<Nuestro Señor es justo, grande en sus consejos y poderoso en sus obras. Sus ojos están abiertos a todas las conductas de todos los hijos de los hombres, para darle a cada uno según su conducta y de acuerdo con el fruto de sus acciones>>\footnote{\textit{Nuestro Señor es justo}: Jer 12:1. \textit{Grande en sus consejos}: Jer 32:19.}. Pero se consideró como una traición blasfema cuando dijo, durante el asedio de Jerusalén: <<Y ahora he puesto estas tierras en manos de Nabucodonosor, rey de Babilonia, mi servidor>>\footnote{\textit{Ahora he puesto estas tierras en manos}: Jer 27:6.}. Cuando Jeremías aconsejó que se rindiera la ciudad\footnote{\textit{Jeremías aconsejó la rendición}: Jer 38:2-3.}, los sacerdotes y los gobernantes civiles lo arrojaron al hoyo cenagoso de una lúgubre mazmorra\footnote{\textit{Jeremías en la mazmorra}: Jer 38:6.}.

\section*{7. El segundo Isaías}
\par
%\textsuperscript{(1068.1)}
\textsuperscript{97:7.1} La destrucción de la nación hebrea y su cautividad en Mesopotamia habrían resultado de gran provecho para su teología en expansión si no hubiera sido por la acción decidida de sus sacerdotes. La nación hebrea había caído ante los ejércitos de Babilonia, y su Yahvé nacionalista había padecido los sermones internacionalistas de los dirigentes espirituales. El resentimiento por la pérdida de su dios nacional fue lo que condujo a los sacerdotes judíos a inventar tantas fábulas y a multiplicar tantos acontecimientos de apariencia milagrosa en la historia hebrea, en un esfuerzo por restablecer a los judíos como el pueblo elegido de incluso la idea nueva y ampliada de un Dios internacional de todas las naciones.

\par
%\textsuperscript{(1068.2)}
\textsuperscript{97:7.2} Las tradiciones y leyendas babilónicas influyeron mucho sobre los judíos durante su cautividad, aunque debe tenerse en cuenta que mejoraron constantemente el carácter moral y el significado espiritual de las historias caldeas que adoptaron, a pesar de que deformaron invariablemente estas leyendas para hacer recaer el honor y la gloria sobre la ascendencia y la historia de Israel.

\par
%\textsuperscript{(1068.3)}
\textsuperscript{97:7.3} Estos sacerdotes y escribas hebreos tenían una sola idea en su mente: la rehabilitación de la nación judía, la glorificación de las tradiciones hebreas y la exaltación de su historia racial. Si se tiene resentimiento por el hecho de que estos sacerdotes imprimieran sus ideas erróneas en una parte tan amplia del mundo occidental, debe recordarse que no lo hicieron intencionalmente; no pretendieron escribir por inspiración; no hicieron ninguna declaración de estar escribiendo un libro sagrado. Estaban simplemente preparando un libro de texto destinado a reforzar el ánimo decreciente de sus compañeros de cautiverio. Tenían el propósito concreto de mejorar el espíritu y el estado de ánimo nacional de sus compatriotas. Los hombres de una época posterior fueron los que reunieron estos y otros escritos en un libro guía cuyas enseñanzas eran supuestamente infalibles.

\par
%\textsuperscript{(1068.4)}
\textsuperscript{97:7.4} Los sacerdotes judíos utilizaron libremente estos escritos después de la cautividad, pero su influencia sobre sus compañeros cautivos fue considerablemente obstaculizada por la presencia de un profeta joven e indomable, el segundo Isaías, que se había convertido plenamente al Dios de justicia, amor, rectitud y misericordia del Isaías anterior. Creía también, junto con Jeremías, que Yahvé se había convertido en el Dios de todas las naciones. Predicó estas teorías sobre la naturaleza de Dios con un efecto tan contundente, que hizo conversos por igual entre los judíos y sus captores. Este joven predicador dejó sus enseñanzas por escrito, pero los sacerdotes hostiles e implacables intentaron separarlas de toda conexión con él, aunque el puro respeto por su belleza y su grandeza condujo a su incorporación entre los escritos del primer Isaías. Y así, los escritos de este segundo Isaías se pueden encontrar en el libro que lleva este nombre, abarcando desde el capítulo cuarenta hasta el capítulo cincuenta y cinco, ambos inclusive.

\par
%\textsuperscript{(1068.5)}
\textsuperscript{97:7.5} Desde Maquiventa hasta la época de Jesús, ningún profeta o educador religioso alcanzó el alto concepto de Dios que el segundo Isaías proclamó durante este período de cautiverio. El Dios que proclamó este dirigente espiritual no era ningún Dios pequeño, antropomorfo o fabricado por el hombre. <<Mirad, levanta las islas como si fueran diminutas>>\footnote{\textit{Mirad, levanta las islas}: Is 40:15.}. <<Al igual que los cielos son más elevados que la Tierra, mis caminos son más elevados que los vuestros, y mis pensamientos más elevados que vuestros pensamientos>>\footnote{\textit{Como los cielos son más elevados que la Tierra}: Is 55:9.}.

\par
%\textsuperscript{(1069.1)}
\textsuperscript{97:7.6} Maquiventa Melquisedek podía por fin contemplar a unos educadores humanos que proclamaban un verdadero Dios a los hombres mortales. Al igual que el primer Isaías, este dirigente predicaba un Dios que creaba y sostenía el universo. <<He creado la Tierra y he puesto al hombre sobre ella. No la he creado en vano; la he formado para que sea habitada>>\footnote{\textit{He creado la Tierra y el hombre}: Is 45:12. \textit{No la he creado en vano}: Is 45:18.}. <<Yo soy el primero y el último; no hay ningún Dios aparte de mí>>\footnote{\textit{Yo soy el primero y el último}: Is 41:4; 44:6; 48:12; Ap 1:8,11,17; 2:8; 21:6; 22:13.}. Hablando en nombre del Señor Dios de Israel, este nuevo profeta dijo: <<Los cielos pueden desaparecer y la Tierra envejecer, pero mi rectitud perdurará siempre y mi salvación se extenderá de generación en generación>>\footnote{\textit{Los cielos pueden desaparecer}: Is 51:6. \textit{Mi salvación se extenderá generaciones}: Is 51:8.}. <<No temáis, porque estoy con vosotros; no os desalentéis, porque yo soy vuestro Dios>>\footnote{\textit{No temáis, porque estoy con vosotros}: Is 41:10.}. <<No hay ningún Dios aparte de mí ---un Dios justo y un Salvador>>\footnote{\textit{No hay Dios aparte de mí, un Dios justo}: Is 45:21.}.

\par
%\textsuperscript{(1069.2)}
\textsuperscript{97:7.7} A los cautivos judíos les confortó, como ha confortado a miles y miles de personas desde entonces, el escuchar unas palabras tales como: <<Así dice el Señor: `Yo os he creado, os he redimido, os he llamado por vuestro nombre; sois míos'>>\footnote{\textit{Dice el Señor: `Yo os he creado'}: Is 43:1.}. <<Cuando paséis por las dificultades, yo estaré con vosotros, puesto que sois inapreciables a mis ojos>>\footnote{\textit{Atravesar las aguas}: Is 43:2. \textit{Sois inapreciables}: Is 43:4.}. <<¿Puede una mujer olvidar a su hijo lactante y no tener compasión por su hijo? Sí, ella puede olvidar, pero yo no olvidaré a mis hijos, porque mirad, los he grabado en la palma de mis manos; los he cubierto incluso con la sombra de mis manos>>\footnote{\textit{¿Puede una mujer olvidar?}: Is 49:15-16. \textit{Los he cubierto con la sombra}: Is 51:16.}. <<Que el perverso abandone sus caminos y el hombre inicuo sus pensamientos; que vuelvan al Señor, y él tendrá misericordia de ellos; que regresen a nuestro Dios, pues él perdonará abundantemente>>\footnote{\textit{Que el perverso abandone sus caminos}: Is 55:7.}.

\par
%\textsuperscript{(1069.3)}
\textsuperscript{97:7.8} Escuchad de nuevo el evangelio de esta nueva revelación del Dios de Salem: <<Apacentará a su rebaño como un pastor; cogerá a los corderos en sus brazos y los llevará en su seno. Da energía a los débiles y acrecienta el vigor de los que no tienen fuerzas. Aquellos que esperan en el Señor renovarán su vigor; se elevarán con alas como las águilas; correrán y no se cansarán; caminarán y no se fatigarán>>\footnote{\textit{Apacentará a su rebaño como un pastor}: Is 40:11. \textit{Da energía a los débiles}: Is 40:29. \textit{Aquellos que esperan en el Señor}: Is 40:31.}.

\par
%\textsuperscript{(1069.4)}
\textsuperscript{97:7.9} Este Isaías dirigió una extensa propaganda evangélica del concepto ampliado de un Yahvé supremo. Rivalizó con Moisés en la elocuencia con que describió al Señor Dios de Israel como Creador Universal. Su descripción de los atributos infinitos del Padre Universal fue poética. Nunca se han vuelto a efectuar unas declaraciones más hermosas sobre el Padre celestial. Los escritos de Isaías, al igual que los Salmos, figuran entre las presentaciones más sublimes y verdaderas del concepto espiritual de Dios que hayan escuchado nunca los oídos de los hombres mortales antes de la llegada de Miguel a Urantia. Escuchad su descripción de la Deidad: <<Yo soy el elevado y el sublime que habita la eternidad>>\footnote{\textit{Soy el elevado y el sublime}: Esd 8:20; Is 57:15.}. <<Yo soy el primero y el último, y aparte de mí no existe ningún otro Dios>>\footnote{\textit{Yo soy el primero y el último, único Dios}: Is 44:6.}. <<La mano del Señor no es tan corta que no pueda salvar, ni su oído tan duro que no pueda escuchar>>\footnote{\textit{La mano del Señor no es tan corta}: Is 59:1.}. Para el pueblo judío fue una doctrina nueva que este profeta benigno, pero con autoridad, insistiera en predicar la constancia divina, la fidelidad de Dios. Declaró que <<Dios no olvidará, no abandonará>>\footnote{\textit{Dios no olvidará}: Is 49:15. \textit{Dios no abandonará}: Is 41:17.}.

\par
%\textsuperscript{(1069.5)}
\textsuperscript{97:7.10} Este instructor atrevido proclamó que el hombre estaba estrechamente relacionado con Dios, diciendo: <<Todos aquellos que son llamados por mi nombre, los he creado para mi gloria, y ellos proclamarán mi alabanza. Yo, soy yo el que borra sus trasgresiones por mi propia satisfacción, y no me acordaré de sus pecados>>\footnote{\textit{He creado al hombre para mi gloria}: Is 43:7. \textit{Ellos proclamarán mi alabanza}: Is 43:21. \textit{Soy yo el que borra sus trasgresiones}: Is 43:25.}.

\par
%\textsuperscript{(1069.6)}
\textsuperscript{97:7.11} Escuchad cómo este gran hebreo echa por tierra el concepto de un Dios nacional, mientras que proclama gloriosamente la divinidad del Padre Universal, del cual dice: <<Los cielos son mi trono, y la Tierra es mi escabel>>\footnote{\textit{Los cielos son mi trono, y la Tierra es mi escabel}: Is 66:1.}. Y el Dios de Isaías era sin embargo santo, majestuoso, justo e inescrutable. El concepto del Yahvé encolerizado, vengativo y celoso de los beduinos del desierto casi se ha desvanecido. Un nuevo concepto del Yahvé supremo y universal ha aparecido en la mente del hombre mortal, para no ser perdido de vista nunca más por la humanidad. La comprensión de la justicia divina ha empezado a destruir la magia primitiva y el miedo biológico. Por fin se le presenta al hombre un universo de ley y de orden, y un Dios universal con unos atributos fiables y finales.

\par
%\textsuperscript{(1070.1)}
\textsuperscript{97:7.12} Este predicador de un Dios celestial nunca dejó de proclamar este \textit{Dios deamor}. <<Vivo en el lugar alto y santo, y también con aquel que tiene un espíritu humilde y contrito>>\footnote{\textit{Vivo en el lugar alto y santo}: Is 57:15.}. Este gran instructor dijo también nuevas palabras de consuelo a sus contemporáneos: <<El Señor os guiará continuamente y satisfará vuestra alma. Seréis como un jardín regado y como un manantial donde no faltan las aguas. Y si el enemigo llega como una inundación, el espíritu del Señor levantará una defensa contra él>>\footnote{\textit{El Señor os guiará continuamente}: Is 58:11. \textit{El espíritu del Señor os defenderá}: Is 59:19.}. El evangelio de Melquisedek, destructor del miedo, y la religión de Salem, que engendraba la confianza, brillaron una vez más para bendición de la humanidad.

\par
%\textsuperscript{(1070.2)}
\textsuperscript{97:7.13} El valiente y perspicaz Isaías eclipsó eficazmente al Yahvé nacionalista mediante su descripción sublime de la majestad y la omnipotencia universal del Yahvé supremo, Dios de amor, soberano del universo y Padre afectuoso de toda la humanidad. Desde aquellos días memorables, el concepto más elevado de Dios en occidente ha englobado siempre la justicia universal, la misericordia divina y la rectitud eterna. En un lenguaje magnífico y con una elegancia incomparable, este gran instructor describió al Creador todopoderoso como un Padre infinitamente amoroso.

\par
%\textsuperscript{(1070.3)}
\textsuperscript{97:7.14} Este profeta de la cautividad predicó a su pueblo y a la gente de muchas naciones que le escuchaban cerca del río en Babilonia. Este segundo Isaías contribuyó mucho a contrarrestar los numerosos conceptos erróneos y racialmente egoístas sobre la misión del Mesías prometido. Pero sus esfuerzos no tuvieron un éxito completo. Si los sacerdotes no se hubieran dedicado a la tarea de construir un nacionalismo mal entendido, las enseñanzas de los dos Isaías hubieran preparado el terreno para el reconocimiento y el recibimiento del Mesías prometido.

\section*{8. Historia sagrada e historia profana}
\par
%\textsuperscript{(1070.4)}
\textsuperscript{97:8.1} La costumbre de considerar el relato de las experiencias de los hebreos como historia sagrada, y las actividades del resto del mundo como historia profana, es responsable de una gran parte de la confusión que existe en la mente humana en cuanto a la interpretación de la historia. Esta dificultad aparece porque no existe una historia laica de los judíos. Durante el exilio en Babilonia, los sacerdotes prepararon su nuevo relato sobre las relaciones supuestamente milagrosas entre Dios y los hebreos, la historia sagrada de Israel tal como figura en el Antiguo Testamento. Luego destruyeron de manera cuidadosa y por completo los archivos existentes de los asuntos hebreos ---los libros tales como <<Los Actos de los reyes de Israel>> y <<Los Actos de los reyes de Judá>>, así como otros diversos documentos más o menos precisos de la historia hebrea.

\par
%\textsuperscript{(1070.5)}
\textsuperscript{97:8.2} La presión devastadora y la coacción inevitable de la historia laica aterrorizaban tanto a los judíos cautivos y gobernados por los extranjeros, que intentaron reescribir y refundir completamente su historia. Para poder comprender esto, deberíamos examinar brevemente el relato de su complicada experiencia nacional. Se debe recordar que los judíos no lograron desarrollar una adecuada filosofía no teológica de la vida. Lucharon contra su concepto egipcio original de las recompensas divinas por la rectitud, unidas a los terribles castigos por el pecado. La historia dramática de Job fue en cierto modo una protesta contra esta filosofía errónea. El pesimismo manifiesto del Eclesiastés fue una sabia reacción mundana contra estas creencias excesivamente optimistas en la Providencia\footnote{\textit{El Eclesiastés es pesimista}: Ec 1:1-18; 2:12-17.}.

\par
%\textsuperscript{(1071.1)}
\textsuperscript{97:8.3} Pero quinientos años de soberanía por parte de unos gobernantes extranjeros eran demasiados incluso para los pacientes y resignados judíos. Los profetas y los sacerdotes empezaron a exclamar: <<¿Hasta cuándo, oh Señor, hasta cuándo?>>\footnote{\textit{¿Hasta cuándo, oh Señor, hasta cuándo?}: Is 6:11.} Cuando los judíos honrados indagaban en las Escrituras, su confusión se volvía aún más profunda. Un antiguo vidente había prometido que Dios protegería y liberaría a su <<pueblo elegido>>\footnote{\textit{La promesa de Dios al pueblo elegido}: Dt 7:6-15.}. Amós había amenazado con que Dios abandonaría a Israel a menos que restablecieran sus criterios de rectitud nacional\footnote{\textit{La amenaza y advertencia de Amós}: Am 5:1-27.}. El escriba del Deuteronomio había descrito la Gran Elección ---entre el bien y el mal, entre la bendición y la maldición\footnote{\textit{La Gran Elección}: Dt 11:26-28.}. El primer Isaías había predicado sobre un rey liberador benéfico\footnote{\textit{Isaías y el rey liberador}: Is 32:1.}. Jeremías había proclamado una era de rectitud interior ---la alianza escrita en las tablillas del corazón\footnote{\textit{Jeremías y la rectitud interior}: Jer 24:7.}. El segundo Isaías había hablado de la salvación por medio del sacrificio y la redención\footnote{\textit{El segundo Isaías y el sacrificio y la rendención}: Is 43:22-26.}. Ezequiel había proclamado la liberación a través del servicio consagrado\footnote{\textit{Ezequiel y el servicio y la devoción}: Ez 14:1-23.}, y Esdras había prometido la prosperidad mediante la observancia de la ley\footnote{\textit{Esdras y la observancia de la ley}: Esd 7:10.}. Pero a pesar de todo esto, continuaban siendo esclavos y la liberación se retrasaba. Daniel presentó entonces el drama de la <<crisis>> inminente ---la destrucción de la gran estatua y el establecimiento inmediato del reinado perpetuo de la rectitud, el reino mesiánico\footnote{\textit{Daniel y el reino mesiánico}: Dn 2:31-45.}.

\par
%\textsuperscript{(1071.2)}
\textsuperscript{97:8.4} Todas estas falsas esperanzas condujeron a tal grado de decepción y de frustración raciales, que los dirigentes de los judíos se sintieron confundidos hasta el punto de no lograr reconocer ni aceptar la misión y el ministerio de un Hijo divino del Paraíso cuando éste vino poco después hacia ellos en la similitud de la carne mortal ---encarnado como Hijo del Hombre.

\par
%\textsuperscript{(1071.3)}
\textsuperscript{97:8.5} Todas las religiones modernas han cometido un grave error cuando han intentado dar una interpretación milagrosa a ciertas épocas de la historia humana. Aunque es cierto que Dios ha tendido muchas veces una mano paternal interviniendo providencialmente en la corriente de los asuntos humanos, es un error considerar a los dogmas teológicos y a las supersticiones religiosas como una sedimentación sobrenatural que aparece mediante una intervención milagrosa en esta corriente de la historia humana. El hecho de que <<los Altísimos gobiernen en los reinos de los hombres>>\footnote{\textit{Los Altísimos gobiernen en los reinos de los hombres}: Dn 4:17,25,32; 5:21.} no convierte la historia laica en una historia supuestamente sagrada.

\par
%\textsuperscript{(1071.4)}
\textsuperscript{97:8.6} Los autores del Nuevo Testamento y los escritores cristianos posteriores complicaron aún más la deformación de la historia hebrea mediante sus intentos bien intencionados por presentar a los profetas judíos como trascendentes. La historia hebrea ha sido así explotada desastrosamente por los escritores judíos y cristianos a la vez. La historia laica de los hebreos ha sido completamente dogmatizada. Ha sido convertida en una ficción de historia sagrada y ha sido inextricablemente relacionada con los conceptos morales y las enseñanzas religiosas de las naciones llamadas cristianas.

\par
%\textsuperscript{(1071.5)}
\textsuperscript{97:8.7} Una breve exposición de los puntos sobresalientes de la historia hebrea ilustrará la manera en que los hechos que figuraban en los archivos fueron tan alterados por los sacerdotes judíos en Babilonia, que la historia laica cotidiana de su pueblo la transformaron en una historia sagrada ficticia.

\section*{9. La historia de los hebreos}
\par
%\textsuperscript{(1071.6)}
\textsuperscript{97:9.1} Nunca existieron doce tribus de israelitas ---sólo tres o cuatro tribus se establecieron en Palestina. La nación hebrea apareció como resultado de la unión de los llamados israelitas con los cananeos \footnote{\textit{Israelitas y cananeos}: Jue 3:5-6.}. <<Y los hijos de Israel habitaron entre los cananeos. Y tomaron a sus hijas por esposas y dieron a sus hijas a los hijos de los cananeos>>. Los hebreos nunca expulsaron a los cananeos de Palestina\footnote{\textit{Los cananeos nunca fueron destruidos}: Nm 21:1-3; Jos 17:18.}, a pesar de que el relato efectuado por los sacerdotes sobre estos hechos afirmaba sin vacilar que lo hicieron.

\par
%\textsuperscript{(1071.7)}
\textsuperscript{97:9.2} La conciencia israelita tuvo su origen en la región montañosa de Efraín; la conciencia judía posterior se originó en el clan meridional de Judá. Los judíos (los judaítas) siempre intentaron difamar y denigrar la historia de los israelitas del norte (los efraimitas).

\par
%\textsuperscript{(1072.1)}
\textsuperscript{97:9.3} La pretenciosa historia de los hebreos empieza con Saúl cuando reunió a los clanes del norte para resistir un ataque de los ammonitas\footnote{\textit{Saúl derrota a los ammonitas}: 1 Sam 11:1-11.} contra los miembros de una tribu hermana ---los galaaditas--- al este del Jordán. Con un ejército de poco más de tres mil hombres derrotó al enemigo, y esta hazaña fue la que condujo a las tribus de las colinas a hacerlo rey\footnote{\textit{Saúl hecho rey por las tropas}: 1 Sam 11:15.}. Cuando los sacerdotes exiliados reescribieron esta historia, aumentaron el ejército de Saúl a 330.000 soldados\footnote{\textit{Números inflados}: 1 Sam 11:8.}, y añadieron <<Judá>> a la lista de las tribus que habían participado en la batalla.

\par
%\textsuperscript{(1072.2)}
\textsuperscript{97:9.4} Inmediatamente después de la derrota de los ammonitas, Saúl se convirtió en rey por elección popular de sus tropas. Ningún sacerdote o profeta participó en este asunto. Pero más tarde, los sacerdotes consignaron en las crónicas que Saúl había sido coronado rey por el profeta Samuel siguiendo las instrucciones divinas\footnote{\textit{Hechos modificados}: 1 Sam 11:14-15.}. Actuaron de esta manera a fin de establecer una <<línea divina de descendencia>> para la monarquía judaíta de David.

\par
%\textsuperscript{(1072.3)}
\textsuperscript{97:9.5} De todas las deformaciones de la historia judía, la mayor de ellas estuvo relacionada con David. Después de la victoria de Saúl sobre los ammonitas (que él atribuyó a Yahvé), los filisteos se alarmaron y empezaron a atacar a los clanes del norte. David y Saúl no lograron nunca ponerse de acuerdo. David estableció una alianza con los filisteos\footnote{\textit{Alianza con los filisteos}: 1 Sam 27:2-3.} y subió por la costa con seiscientos hombres hasta Esdraelón. En Gat, los filisteos le ordenaron a David que dejara el campo de batalla; temían que pudiera aliarse con Saúl. David se retiró\footnote{\textit{David se retira}: 1 Sam 29:1-11.}; los filisteos atacaron y derrotaron a Saúl\footnote{\textit{Trágica derrota de Saúl}: 1 Cr 10:1-9; 1 Sam 31:1-9; 2 Sam 1:6-11.}. No habrían podido conseguirlo si David hubiera permanecido leal a Israel. El ejército de David\footnote{\textit{El ejército de David}: 1 Sam 22:1-2.} era un conjunto políglota de descontentos, compuesto en su mayor parte de inadaptados sociales y fugitivos de la justicia.

\par
%\textsuperscript{(1072.4)}
\textsuperscript{97:9.6} La trágica derrota de Saúl en Gilboa a manos de los filisteos disminuyó la importancia que tenía Yahvé entre los dioses a los ojos de los cananeos vecinos. Normalmente, la derrota de Saúl habría sido imputada a una apostasía de Yahvé, pero en esta ocasión los redactores judaítas la atribuyeron a errores de ritual\footnote{\textit{Errores de ritual}: 1 Sam 28:18.}. Necesitaban la tradición de Saúl y Samuel como trasfondo para el reinado de David.

\par
%\textsuperscript{(1072.5)}
\textsuperscript{97:9.7} David estableció su cuartel general con su pequeño ejército en la ciudad no hebrea de Hebrón\footnote{\textit{David en Hebrón}: 2 Sam 2:1-4.}. Sus compatriotas no tardaron en proclamarlo rey del nuevo reino de Judá. Judá estaba compuesto principalmente por elementos no hebreos ---kenitas, calebitas, jebuseos y otros cananeos. Eran nómadas ---pastores--- y por lo tanto partidarios de la idea hebrea sobre la propiedad de la tierra. Conservaban las ideologías de los clanes del desierto.

\par
%\textsuperscript{(1072.6)}
\textsuperscript{97:9.8} La diferencia entre la historia sagrada y la historia profana está bien ilustrada en los dos relatos diferentes acerca de la coronación de David como rey, que figuran en el Antiguo Testamento. Los sacerdotes dejaron por inadvertencia en los archivos una parte de la historia profana sobre la manera en que los seguidores inmediatos de David (su ejército) lo hicieron rey\footnote{\textit{Coronado rey de Judá por el ejército}: 2 Sam 2:4.}, y posteriormente prepararon el largo y prosaico relato de la historia sagrada, en el que se describe cómo el profeta Samuel, por instrucción divina, escogió a David entre sus hermanos y procedió a ungirlo oficialmente, por medio de ceremonias solemnes y elaboradas, como rey de los hebreos, y luego lo proclamó sucesor de Saúl\footnote{\textit{Versión de los sacerdotes de la coronación}: 1 Sam 16:1-13.}.

\par
%\textsuperscript{(1072.7)}
\textsuperscript{97:9.9} Después de preparar sus relatos ficticios sobre las relaciones milagrosas entre Dios e Israel, los sacerdotes olvidaron muchas veces suprimir por completo las afirmaciones claras y realistas que ya existían en dichos documentos.

\par
%\textsuperscript{(1072.8)}
\textsuperscript{97:9.10} David intentó mejorar su posición política casándose primero con la hija de Saúl\footnote{\textit{David se casa con Mical}: 1 Sam 18:20-27.}, luego con la viuda de Nabal, el rico edomita\footnote{\textit{David se casa con Abigail}: 1 Sam 25:42.}, y después con la hija de Talmai\footnote{\textit{David se casa con Maaca}: 2 Sam 3:3.}, el rey de Geshur. Tomó seis esposas entre las mujeres de Jebus, sin mencionar a Betsabé\footnote{\textit{Betsabé}: 2 Sam 11:26-27.}, la esposa del hitita.

\par
%\textsuperscript{(1073.1)}
\textsuperscript{97:9.11} Por medio de estos métodos y de estas personas fue como David construyó la ficción de un reino divino de Judá, que era el sucesor de la herencia y las tradiciones del reino septentrional del Israel efraimita en vías de desaparición. La tribu cosmopolita de David, llamada Judá, estaba compuesta por más gentiles que judíos; sin embargo, los ancianos oprimidos de Efraín bajaron de sus montañas y <<le ungieron como rey de Israel>>\footnote{\textit{Le ungieron como rey de Israel}: 2 Sam 5:3.}. Después de una amenaza militar, David hizo entonces un pacto con los jebuseos y estableció su capital del reino unido en Jebus (Jerusalén)\footnote{\textit{Jerusalén}: 2 Sam 5:6-9.}, que era una ciudad fuertemente amurallada a medio camino entre Judá e Israel. Los filisteos se sublevaron y no tardaron en atacar a David\footnote{\textit{Ataque de los filisteos}: 2 Sam 5:17-25.}. Después de una violenta batalla fueron derrotados\footnote{\textit{Victoria de David}: 2 Sam 5:6-9.}, y Yahvé fue establecido una vez más como <<el Señor Dios de los Ejércitos>>\footnote{\textit{Señor Dios de los Ejércitos}: 2 Sam 5:10.}.

\par
%\textsuperscript{(1073.2)}
\textsuperscript{97:9.12} Pero Yahvé tenía que compartir forzosamente una parte de esta gloria con los dioses cananeos, pues la mayor parte del ejército de David no era hebrea. Por eso aparece en vuestras escrituras esta declaración reveladora (que los redactores judaítas pasaron por alto): <<Yahvé ha derribado a mis enemigos delante de mí. Por eso le ha puesto a aquel lugar el nombre de Baal-Perazim>>\footnote{\textit{Yahvé ha derribado a mis enemigos}: 2 Sam 5:20.}. Actuaron así porque el ochenta por ciento de los soldados de David eran baalitas.

\par
%\textsuperscript{(1073.3)}
\textsuperscript{97:9.13} David explicó la derrota de Saúl en Gilboa haciendo observar que Saúl había atacado la ciudad cananea de Gibeón, cuya población tenía un tratado de paz con los efraimitas\footnote{\textit{Saúl había roto el tratado}: 2 Sam 21:1-2.}. A causa de esto, Yahvé lo había abandonado. Incluso en los tiempos de Saúl, David había defendido la ciudad cananea de Keila contra los filisteos\footnote{\textit{David defiende Keila}: 1 Sam 23:1-5.}, y luego estableció su capital en una ciudad cananea. Siguiendo su política de compromiso con los cananeos, David entregó siete descendientes de Saúl a los gibeonitas para que los ahorcaran\footnote{\textit{Descendientes de Saúl ahorcados}: 2 Sam 21:3-9.}.

\par
%\textsuperscript{(1073.4)}
\textsuperscript{97:9.14} Después de la derrota de los filisteos, David tomó posesión del <<arca de Yahvé>>\footnote{\textit{Arca de Yahvé}: 1 Cr 15:25-29; 2 Sam 6:1-17.}, la llevó a Jerusalén e instaló oficialmente el culto a Yahvé en su reino. Luego impuso fuertes tributos a las tribus vecinas ---edomitas, moabitas, ammonitas y sirios\footnote{\textit{Fuertes tributos a los vecinos}: 2 Sam 8:11-12.}.

\par
%\textsuperscript{(1073.5)}
\textsuperscript{97:9.15} La maquinaria política corrupta de David empezó a apoderarse personalmente de las tierras del norte, violando las costumbres hebreas, y poco después logró controlar los aranceles de las caravanas, anteriormente recaudados por los filisteos. Luego se produjo una serie de atrocidades que culminaron en el asesinato de Urías\footnote{\textit{Asesinato de Urías}: 2 Sam 11:14-17.}. Todas las apelaciones judiciales se juzgaban en Jerusalén; <<los ancianos>> ya no podían administrar la justicia. No es de extrañar que estallara la rebelión. Hoy se calificaría a Absalón de demagogo\footnote{\textit{Absalón aspira al trono}: 2 Sam 15:2-6.}; su madre era cananea. Había media docena de aspirantes al trono además de Salomón, el hijo de Betsabé.

\par
%\textsuperscript{(1073.6)}
\textsuperscript{97:9.16} Después de la muerte de David, Salomón purgó la maquinaria política de todas las influencias nórdicas, pero continuó con toda la tiranía y el sistema tributario del régimen de su padre\footnote{\textit{Tributos a los edificios}: 1 Re 9:15.}. Salomón arruinó la nación con los lujos de su corte y su detallado programa de construcciones, entre ellas la casa del Líbano, el palacio de la hija del faraón\footnote{\textit{Salomón y la hija del faraón}: 1 Re 7:8.}, el templo de Yahvé, el palacio del rey y la restauración de las murallas de muchas ciudades. Salomón creó una enorme flota hebrea\footnote{\textit{La armada hebrea}: 1 Re 9:26-27.}, dirigida por marineros sirios, que comerciaba con el mundo entero. Su harén estaba compuesto por cerca de mil mujeres\footnote{\textit{El harén de Salomón}: 1 Re 11:3.}.

\par
%\textsuperscript{(1073.7)}
\textsuperscript{97:9.17} El templo de Yahvé en Silo cayó en descrédito hacia esta época, y todo el culto de la nación fue centralizado en la espléndida capilla real de Jebus. El reino del norte volvió más a la adoración de Elohim. Disfrutaban del favor de los faraones, que más tarde esclavizaron a Judá\footnote{\textit{La esclavización por parte de Egipto}: Esd 1:25-30.}, sometiendo al reino del sur a pagar tributo.

\par
%\textsuperscript{(1073.8)}
\textsuperscript{97:9.18} Hubo altibajos ---guerras entre Israel y Judá. Después de cuatro años de guerra civil y de tres dinastías, Israel cayó bajo el dominio de los déspotas de la ciudad, que empezaron a comerciar con las tierras\footnote{\textit{Intento de comerciar con tierras}: 1 Re 16:23-24.}. Incluso el rey Omri intentó comprar las propiedades de Semer. Pero el fin se acercó rápidamente cuando Salmanasar III decidió controlar la costa mediterránea. Ajab, el rey de Efraín, reunió a otros diez grupos y resistió en Karkar; la batalla terminó en un empate. Detuvieron a los asirios, pero los aliados quedaron diezmados. Esta gran batalla ni siquiera se menciona en el Antiguo Testamento.

\par
%\textsuperscript{(1074.1)}
\textsuperscript{97:9.19} Surgieron nuevos problemas cuando el rey Ajab intentó comprar las tierras de Nabot. Su esposa fenicia falsificó la firma de Ajab en los documentos que ordenaban la confiscación de las tierras de Nabot, acusado de haber blasfemado contra los nombres de <<Elohim y del rey>>. Él y sus hijos fueron rápidamente ejecutados. El enérgico Elías apareció en escena denunciando a Ajab por el asesinato de los Nabot\footnote{\textit{Elías denuncia a Ajab}: 1 Re 21:17-24. \textit{Asesinato de los Nabot}: 1 Re 21:1-16.}. Así es como Elías, uno de los profetas más grandes, empezó su enseñanza como defensor de las antiguas costumbres sobre la tierra y en contra de la actitud de los baalitas de vender las tierras, contra la tentativa de las ciudades por dominar el campo. Pero la reforma no tuvo éxito hasta que el terrateniente Jehú\footnote{\textit{Las reformas de Jehú}: 2 Re 10:15-28.} unió sus fuerzas a las del cacique gitano Yonadab para destruir a los profetas (agentes inmobiliarios) de Baal en Samaria.

\par
%\textsuperscript{(1074.2)}
\textsuperscript{97:9.20} Una nueva vida apareció cuando Joás\footnote{\textit{Joás}: 2 Re 12:1-2.} y su hijo Jeroboam liberaron a Israel de sus enemigos. Pero en esta época gobernaba en Samaria una nobleza de bandidos cuyas depredaciones rivalizaban con las de la dinastía de David de los tiempos antiguos. El Estado y la iglesia estaban de común acuerdo. El intento por suprimir la libertad de expresión condujo a Elías, Amós y Oseas a empezar a escribir en secreto, y éste fue el auténtico comienzo de las Biblias judía y cristiana.

\par
%\textsuperscript{(1074.3)}
\textsuperscript{97:9.21} Pero el reino del norte no desapareció de la historia hasta que el rey de Israel conspiró con el rey de Egipto y se negó a continuar pagando tributo a Asiria. Entonces empezó un asedio de tres años, seguido por la dispersión total del reino del norte. Efraín (Israel) desapareció de esta manera\footnote{\textit{Fin del reino}: 2 Re 17:4-6.}. Judá ---los judíos, <<el resto de Israel>>\footnote{\textit{El resto de Israel}: 2 Cr 34:9; Is 10:20; Jer 6:9; Ez 11:13; Miq 2:12; Sof 3:13.} ---había empezado a concentrar las tierras entre las manos de unos pocos, tal como dijo Isaías: <<Acumulando una casa tras otra y un campo tras otro>>\footnote{\textit{Acumulando una casa tras otra}: Is 5:8.}. Pronto hubo en Jerusalén un templo de Baal al lado del templo de Yahvé. Este reinado de terror terminó en una sublevación monoteísta dirigida por el rey niño Joás\footnote{\textit{Revuelta del rey niño Joás}: 2 Re 11:12,17-19; 2 Cr 23:11,16-20.}, que hizo una cruzada a favor de Yahvé durante treinta y cinco años.

\par
%\textsuperscript{(1074.4)}
\textsuperscript{97:9.22} El rey siguiente, Amasías\footnote{\textit{Amasías derrotado}: 2 Cr 25:27.}, tuvo dificultades con los contribuyentes edomitas rebeldes y con sus vecinos. Después de una victoria notable, se dirigió a atacar a sus vecinos del norte y sufrió una derrota igualmente notable. Luego se rebelaron los campesinos; asesinaron al rey y pusieron en el trono a su hijo Azarías, de dieciséis años, llamado Ozías por Isaías\footnote{\textit{Ozías hecho rey}: 2 Cr 26:1.}. Después de Ozías, las cosas fueron de mal en peor, y Judá vivió durante cien años pagando tributo a los reyes de Asiria. El primer Isaías les dijo que como Jerusalén era la ciudad de Yahvé, no caería nunca\footnote{\textit{Isaías y que Jerusalén no caería}: Is 31:5.}. Pero Jeremías no dudó en proclamar su caída\footnote{\textit{Jeremías y la caída de Jerusalén}: Jer 6:1-6; 15:5-6.}.

\par
%\textsuperscript{(1074.5)}
\textsuperscript{97:9.23} La verdadera ruina de Judá fue llevada a cabo por una banda de ricos políticos corruptos que actuaba bajo el gobierno del rey niño Manasés\footnote{\textit{El rey niño Manasés}: 2 Re 21:1-2; 2 Cr 33:1-3.}. La economía cambiante favoreció la vuelta a la adoración de Baal, cuyas transacciones privadas con las tierras estaban en contra de la ideología de Yahvé. La caída de Asiria y la ascensión de Egipto trajeron la liberación de Judá durante un tiempo, y los campesinos tomaron el poder. Bajo Josías, destruyeron la banda de políticos corruptos de Jerusalén.

\par
%\textsuperscript{(1074.6)}
\textsuperscript{97:9.24} Pero esta era llegó a su fin trágicamente cuando Josías\footnote{\textit{El atrevimiento de Josías}: 2 Re 23:25-30; 2 Cr 35:20-24; Esd 1:25-38.} se atrevió a salir para interceptar al poderoso ejército de Nekó que subía por la costa desde Egipto para ayudar a Asiria contra Babilonia. Josías fue arrasado, y Judá tuvo que pagar tributo a Egipto. El partido político de Baal volvió al poder en Jerusalén, y así es como empezó la \textit{verdadera} esclavitud hacia Egipto. Luego siguió un período durante el cual los políticos de Baal controlaron tanto los tribunales como el clero. El culto a Baal era un sistema económico y social que se ocupaba de los derechos de propiedad y también tenía que ver con la fertilidad del suelo.

\par
%\textsuperscript{(1075.1)}
\textsuperscript{97:9.25} Con la derrota de Nekó a manos de Nabucodonosor\footnote{\textit{Derrota de Nekó}: Jer 46:2.}, Judá cayó bajo el dominio de Babilonia y se le concedieron diez años de gracia, pero pronto se rebeló. Cuando Nabucodonosor vino contra ellos, los judaítas pusieron en marcha algunas reformas sociales, tales como la liberación de los esclavos, para influir sobre Yahvé. El ejército babilonio se retiró temporalmente, y los hebreos se regocijaron porque las virtudes de sus reformas los habían salvado. Durante este período fue cuando Jeremías les anunció la ruina inminente\footnote{\textit{Anuncio de la ruina}: Jer 38:2-3.} que les esperaba, y poco después volvió Nabucodonosor\footnote{\textit{El regreso de Nabucodonosor}: 2 Re 25:1; Jer 39:1.}.

\par
%\textsuperscript{(1075.2)}
\textsuperscript{97:9.26} El fin de Judá sobrevino así repentinamente. La ciudad fue destruida y la población llevada a Babilonia\footnote{\textit{La destrucción de la ciudad}: 2 Re 25:2-17; Esd 1:45-58; Jer 38:2-9.}. La lucha entre Yahvé y Baal terminó en la cautividad. Y la conmoción de la cautividad llevó al resto de Israel al monoteísmo.

\par
%\textsuperscript{(1075.3)}
\textsuperscript{97:9.27} En Babilonia, los judíos llegaron a la conclusión de que no podían existir en Palestina como un pequeño grupo, con sus propias costumbres sociales y económicas particulares, y que si sus ideologías habían de prevalecer, tenían que convertir a los gentiles. Así es como se originó su nuevo concepto del destino ---la idea de que los judíos debían convertirse en los servidores elegidos de Yahvé. La religión judía del Antiguo Testamento evolucionó realmente durante la cautividad en Babilonia.

\par
%\textsuperscript{(1075.4)}
\textsuperscript{97:9.28} La doctrina de la inmortalidad también tomó forma en Babilonia. Los judíos habían creído que la idea de la vida futura reducía la importancia de su evangelio de justicia social. Ahora, por primera vez, la teología desplazaba a la sociología y a la economía. La religión estaba tomando forma como sistema de pensamiento y de conducta humanos, separándose cada vez más de la política, la sociología y la economía.

\par
%\textsuperscript{(1075.5)}
\textsuperscript{97:9.29} Y así, la verdad sobre el pueblo judío revela que muchas cosas que han sido consideradas como historia sagrada no son mucho más que la crónica de una historia profana común y corriente. El judaísmo fue el terreno donde creció el cristianismo, pero los judíos no eran un pueblo milagroso.

\section*{10. La religión hebrea}
\par
%\textsuperscript{(1075.6)}
\textsuperscript{97:10.1} Sus dirigentes habían enseñado a los israelitas que eran un pueblo elegido\footnote{\textit{Pueblo elegido}: 1 Re 3:8; 1 Cr 17:21-22; Sal 33:12; 105:6,43; 135:4; Is 41:8-9; 43:20-21; 44:1; Dt 7:6; 14:2.}, no por una complacencia y un monopolio especiales del favor divino, sino para el servicio especial de llevar la verdad del Dios único y supremo a todas las naciones. Y habían prometido a los judíos que, si cumplían con este destino, se convertirían en los dirigentes espirituales de todos los pueblos, y que el Mesías venidero reinaría sobre ellos y sobre el mundo entero como Príncipe de la Paz.

\par
%\textsuperscript{(1075.7)}
\textsuperscript{97:10.2} Cuando los judíos fueron liberados por los persas, sólo regresaron a Palestina para caer en la esclavitud de su propio código de leyes, sacrificios y rituales dominado por los sacerdotes. Y al igual que los clanes hebreos rechazaron la maravillosa historia de Dios presentada en el discurso de despedida de Moisés\footnote{\textit{Discurso de despedida de Moisés}: Dt 32:1-43.} a favor de los rituales de sacrificio y de penitencia, estos restos de la nación hebrea rechazaron también el magnífico concepto del segundo Isaías a favor de las reglas, las reglamentaciones y los rituales de su clero en crecimiento.

\par
%\textsuperscript{(1075.8)}
\textsuperscript{97:10.3} El egotismo nacional, la falsa confianza en un Mesías prometido y mal comprendido, así como la esclavitud y la tiranía crecientes de los sacerdotes, silenciaron para siempre las voces de los dirigentes espirituales (exceptuando a Daniel, Ezequiel, Ageo y Malaquías); y desde aquel tiempo hasta la época de Juan el Bautista, todo Israel experimentó un retroceso espiritual cada vez mayor. Pero los judíos no perdieron nunca el concepto del Padre Universal; han continuado manteniendo este concepto de la Deidad incluso hasta el siglo veinte después de Cristo.

\par
%\textsuperscript{(1076.1)}
\textsuperscript{97:10.4} Desde Moisés hasta Juan el Bautista existió una línea ininterrumpida de fieles educadores que pasaron la antorcha de la luz monoteísta de una generación a la siguiente, al mismo tiempo que reprendían sin cesar a los gobernantes sin escrúpulos, denunciaban a los sacerdotes mercantilistas y exhortaban siempre al pueblo a que cumplieran con la adoración del Yahvé supremo, el Señor Dios de Israel.

\par
%\textsuperscript{(1076.2)}
\textsuperscript{97:10.5} Los judíos terminaron por perder su identidad política como nación, pero la religión hebrea de la creencia sincera en el Dios único y universal continúa viviendo\footnote{\textit{Supervivencia de la religión hebrea}: Bar 2:9-5:9.} en el corazón de los exiliados dispersos. Esta religión sobrevive porque ha desempeñado eficazmente su función de conservar los valores más elevados de sus seguidores. La religión judía logró preservar los ideales de un pueblo, pero no consiguió fomentar el progreso ni estimular el descubrimiento filosófico creativo en los ámbitos de la verdad. La religión judía tenía muchos defectos ---era deficiente en filosofía y estaba casi desprovista de cualidades estéticas--- pero sí conservó los valores morales; por eso sobrevivió. Comparado con otros conceptos de la Deidad, el concepto del Yahvé supremo era claro, intenso, personal y moral.

\par
%\textsuperscript{(1076.3)}
\textsuperscript{97:10.6} Los judíos amaban la justicia, la sabiduría, la verdad y la rectitud como pocos pueblos lo han hecho, pero contribuyeron menos que todos los demás pueblos a la comprensión intelectual y al entendimiento espiritual de estas cualidades divinas. Aunque la teología hebrea se negó a crecer, jugó un papel importante en el desarrollo de otras dos religiones mundiales: el cristianismo y el mahometismo.

\par
%\textsuperscript{(1076.4)}
\textsuperscript{97:10.7} La religión judía sobrevivió también a causa de sus instituciones. Es difícil que la religión sobreviva cuando sólo es la práctica privada de unos individuos aislados. Los dirigentes religiosos siempre han cometido el siguiente error: Al observar los males de la religión institucionalizada, tratan de destruir la técnica de las actividades en grupo. En lugar de destruir todo el ritual, harían mejor en reformarlo. A este respecto, Ezequiel fue más sabio que sus contemporáneos; aunque se unió a ellos para insistir en la responsabilidad moral personal\footnote{\textit{Ezequiel y la responsabilidad moral}: Ez 18:1-9.}, también se dedicó a establecer el fiel cumplimiento de un ritual superior y purificado\footnote{\textit{Ritual superior y purificado}: Ez 43:15-46:24.}.

\par
%\textsuperscript{(1076.5)}
\textsuperscript{97:10.8} Así es como los educadores sucesivos de Israel llevaron a cabo, en la evolución de la religión, la hazaña más grande que se haya realizado jamás en Urantia: la transformación gradual pero continua del concepto bárbaro del demonio salvaje Yahvé, el dios espíritu celoso y cruel del volcán fulminante del Sinaí, en el concepto posterior sublime y celestial de un Yahvé supremo, creador de todas las cosas y Padre amante y misericordioso de toda la humanidad. Este concepto hebreo de Dios fue la imagen humana más elevada que se tuvo del Padre Universal hasta el momento en que fue aún más ampliada y exquisitamente desarrollada mediante las enseñanzas personales y el ejemplo de la vida de su Hijo, Miguel de Nebadon.

\par
%\textsuperscript{(1076.6)}
\textsuperscript{97:10.9} [Presentado por un Melquisedek de Nebadon.]


\chapter{Documento 98. Las enseñanzas de Melquisedek en Occidente}
\par
%\textsuperscript{(1077.1)}
\textsuperscript{98:0.1} LAS enseñanzas de Melquisedek penetraron en Europa por muchos caminos, pero llegaron principalmente a través de Egipto y fueron incorporadas en la filosofía occidental después de haber sido completamente helenizadas y más tarde cristianizadas. Los ideales del mundo occidental eran esencialmente socráticos, y su filosofía religiosa posterior llegó a ser la de Jesús, pero con las modificaciones y los compromisos debidos al contacto con la filosofía y la religión occidentales en evolución, culminando todo ello en la iglesia cristiana.

\par
%\textsuperscript{(1077.2)}
\textsuperscript{98:0.2} Los misioneros de Salem continuaron sus actividades durante mucho tiempo en Europa, y fueron absorbidos gradualmente por los numerosos cultos y grupos rituales que surgían periódicamente. Entre aquellos que mantuvieron las enseñanzas de Salem en su forma más pura se debe mencionar a los cínicos. Estos predicadores de la fe y la confianza en Dios ejercían todavía su actividad en la Europa romana del siglo primero después de Cristo, y más tarde fueron incorporados en la religión cristiana que estaba empezando a formarse.

\par
%\textsuperscript{(1077.3)}
\textsuperscript{98:0.3} Una gran parte de la doctrina de Salem fue difundida en Europa por los soldados mercenarios judíos que participaron en tantos combates militares en Occidente. En los tiempos antiguos, los judíos eran famosos tanto por su valor militar como por sus peculiaridades teológicas.

\par
%\textsuperscript{(1077.4)}
\textsuperscript{98:0.4} Las doctrinas fundamentales de la filosofía griega, de la teología judía y de la ética cristiana fueron esencialmente repercusiones de las enseñanzas anteriores de Melquisedek.

\section*{1. La religión de Salem entre los griegos}
\par
%\textsuperscript{(1077.5)}
\textsuperscript{98:1.1} Los misioneros de Salem podrían haber construido una gran estructura religiosa entre los griegos si no hubieran interpretado tan estrictamente su juramento de ordenación, un compromiso impuesto por Maquiventa que prohibía organizar congregaciones exclusivas para el culto, y que exigía la promesa de cada educador de no ejercer nunca como sacerdote, de no recibir nunca honorarios por sus servicios religiosos, sino únicamente alimentos, vestidos y un techo. Cuando los instructores de Melquisedek penetraron en la Grecia prehelénica, encontraron a un pueblo que fomentaba todavía las tradiciones de Adanson y de los tiempos de los anditas, pero estas enseñanzas habían sido enormemente adulteradas por los conceptos y las creencias de las hordas de esclavos inferiores que habían sido traídos en cantidades crecientes hasta las costas griegas. Esta adulteración produjo un retorno a un animismo burdo con ritos sangrientos, donde las clases inferiores llegaban incluso a convertir en una ceremonia la ejecución de los criminales condenados.

\par
%\textsuperscript{(1077.6)}
\textsuperscript{98:1.2} La influencia inicial de los educadores de Salem fue casi destruida por la invasión llamada aria procedente de Europa meridional y de Oriente. Estos invasores helénicos trajeron con ellos unos conceptos antropomórficos de Dios similares a los que sus hermanos arios habían llevado hasta la India. Esta importación inauguró la evolución de la familia griega de dioses y diosas. Esta nueva religión estaba basada en parte en los cultos de los bárbaros helénicos recién llegados, pero también compartía los mitos de los antiguos habitantes de Grecia.

\par
%\textsuperscript{(1078.1)}
\textsuperscript{98:1.3} Los griegos helenos encontraron el mundo mediterráneo ampliamente dominado por el culto a la madre, e impusieron a estos pueblos su dios-hombre Dyaus-Zeus, que ya se había convertido, al igual que Yahvé entre los semitas henoteístas, en el jefe de todo el panteón griego de dioses subordinados. Los griegos habrían llegado finalmente a un verdadero monoteísmo con el concepto de Zeus si no hubieran conservado la idea de que la Suerte lo controlaba todo. Un Dios de valor final debe ser él mismo el árbitro de la suerte y el creador del destino.

\par
%\textsuperscript{(1078.2)}
\textsuperscript{98:1.4} Como consecuencia de estos factores en la evolución religiosa, pronto se desarrolló la creencia popular en los dioses despreocupados del Monte Olimpo, unos dioses más humanos que divinos, unos dioses que los griegos inteligentes nunca se tomaron muy en serio. Ni amaban ni temían mucho a estas divinidades que ellos mismos habían creado. Tenían un sentimiento patriótico y racial hacia Zeus y su familia de semihombres y semidioses, pero apenas los veneraban ni los adoraban.

\par
%\textsuperscript{(1078.3)}
\textsuperscript{98:1.5} Los helenos se impregnaron tanto de las doctrinas anticlericales de los primeros educadores de Salem, que en Grecia nunca surgió ningún clero de importancia. Incluso la fabricación de imágenes de los dioses se convirtió más en un trabajo artístico que en una materia de culto.

\par
%\textsuperscript{(1078.4)}
\textsuperscript{98:1.6} Los dioses olímpicos ilustran el antropomorfismo típico del hombre. Pero la mitología griega era más estética que ética. La religión griega era útil en el sentido de que describía un universo gobernado por un grupo de deidades. Pero la moral, la ética y la filosofía griegas avanzaron enseguida mucho más allá del concepto teísta, y este desequilibrio entre el crecimiento intelectual y el desarrollo espiritual fue tan peligroso para Grecia como lo había sido para la India.

\section*{2. El pensamiento filosófico griego}
\par
%\textsuperscript{(1078.5)}
\textsuperscript{98:2.1} Una religión superficial y considerada a la ligera no puede perdurar, principalmente cuando no posee ningún clero que fomente sus formas y llene de temor y respeto el corazón de sus adeptos. La religión del Olimpo no prometía la salvación ni aplacaba la sed espiritual de sus creyentes; por eso estaba condenada a perecer. Menos de un milenio después de su nacimiento casi había desaparecido, y los griegos se quedaron sin una religión nacional, ya que los dioses del Olimpo habían perdido su influencia sobre los mejores pensadores.

\par
%\textsuperscript{(1078.6)}
\textsuperscript{98:2.2} Ésta era la situación cuando en el siglo sexto antes de Cristo, Oriente y el Levante experimentaron un renacimiento de la conciencia espiritual y un nuevo despertar al reconocimiento del monoteísmo. Pero Occidente no tomó parte en este nuevo desarrollo; ni Europa ni el norte de África participaron ampliamente en este renacimiento religioso. Sin embargo, los griegos emprendieron un magnífico progreso intelectual. Habían empezado a dominar el miedo y ya no buscaban la religión como antídoto del mismo, pero no percibían que la verdadera religión cura el hambre del alma, la inquietud espiritual y la desesperación moral. Buscaban el consuelo del alma en el pensamiento profundo ---en la filosofía y la metafísica. Se apartaron de la contemplación de la preservación de sí mismo ---la salvación--- y se volvieron hacia la autorrealización y el conocimiento de sí mismo.

\par
%\textsuperscript{(1078.7)}
\textsuperscript{98:2.3} Por medio de una reflexión rigurosa, los griegos intentaron alcanzar la conciencia de una seguridad que pudiera sustituir a la creencia en la supervivencia, pero fracasaron por completo. Sólo las personas más inteligentes de las clases superiores de los pueblos helénicos pudieron captar esta nueva enseñanza; la masa de los descendientes de los esclavos de las generaciones anteriores no tenía ninguna capacidad para recibir este nuevo sustituto de la religión.

\par
%\textsuperscript{(1079.1)}
\textsuperscript{98:2.4} Los filósofos desdeñaban todas las formas de culto, a pesar de que prácticamente todos ellos se mantenían vagamente fieles al trasfondo de una creencia en la doctrina de Salem sobre la <<Inteligencia del universo>>, <<la idea de Dios>> y <<la Gran Fuente>>. En la medida en que los filósofos griegos reconocían lo divino y lo superfinito, eran claramente monoteístas; daban un escaso reconocimiento a toda la constelación de dioses y diosas del Olimpo.

\par
%\textsuperscript{(1079.2)}
\textsuperscript{98:2.5} Los poetas griegos de los siglos sexto y quinto antes de Cristo, principalmente Píndaro, intentaron reformar la religión griega. Elevaron los ideales de esta última, pero eran más artistas que personas religiosas. No lograron desarrollar una técnica para fomentar y conservar los valores supremos.

\par
%\textsuperscript{(1079.3)}
\textsuperscript{98:2.6} Jenófanes enseñó la doctrina de un Dios único, pero su concepto de la deidad era demasiado panteísta como para poder ser un Padre personal para el hombre mortal. Anaxágoras era un mecanicista, excepto que reconocía una Causa Primera, una Mente Inicial. Sócrates y sus sucesores, Platón y Aristóteles, enseñaron que la virtud es el conocimiento, que la bondad es la salud del alma, que es mejor sufrir la injusticia que ser culpable de ella, que es un error devolver mal por mal, y que los dioses son sabios y buenos. Sus virtudes cardinales eran la sabiduría, el valor, la moderación y la justicia.

\par
%\textsuperscript{(1079.4)}
\textsuperscript{98:2.7} La evolución de la filosofía religiosa en los pueblos helénicos y hebreos proporciona un ejemplo contrastante de la función de la iglesia como institución en el desarrollo del progreso cultural. En Palestina, el pensamiento humano estaba tan controlado por los sacerdotes y tan dirigido por las escrituras, que la filosofía y la estética estaban totalmente sumergidas en la religión y la moralidad. En Grecia, la ausencia casi total de sacerdotes y de <<escrituras sagradas>> dejó libre y sin trabas a la mente humana, produciéndose un desarrollo sorprendente en la profundidad de pensamiento. Pero la religión, como experiencia personal, no logró seguir el mismo ritmo que la investigación intelectual de la naturaleza y de la realidad del cosmos.

\par
%\textsuperscript{(1079.5)}
\textsuperscript{98:2.8} En Grecia, la creencia estaba subordinada al pensamiento; en Palestina, el pensamiento se mantenía sometido a la creencia. Una gran parte de la fuerza del cristianismo se debe a que ha tomado prestadas muchas cosas tanto de la moralidad hebrea como del pensamiento griego.

\par
%\textsuperscript{(1079.6)}
\textsuperscript{98:2.9} En Palestina, el dogma religioso se cristalizó tanto que puso en peligro el crecimiento ulterior; en Grecia, el pensamiento humano se volvió tan abstracto que el concepto de Dios se disipó en un vapor nebuloso de especulaciones panteístas, no muy diferentes a la Infinidad impersonal de los filósofos brahmánicos.

\par
%\textsuperscript{(1079.7)}
\textsuperscript{98:2.10} Pero los hombres corrientes de aquellos tiempos no podían captar, ni tampoco les interesaba mucho, la filosofía griega de la autorrealización y de una Deidad abstracta; anhelaban más bien promesas de salvación, unidas a un Dios personal que pudiera escuchar sus oraciones. Exiliaron a los filósofos, persiguieron a los adeptos que quedaban del culto de Salem, ya que las dos doctrinas se habían mezclado mucho, y se prepararon para la terrible inmersión orgiástica en los desatinos de los cultos de misterio que entonces estaban extendiéndose por los países mediterráneos. Los misterios eleusinos crecieron dentro del panteón olímpico, y eran una versión griega del culto a la fertilidad; floreció el culto dionisíaco a la naturaleza; el mejor culto de todos era la fraternidad órfica, cuyos sermones morales y promesas de salvación ofrecían un gran atractivo para muchas personas.

\par
%\textsuperscript{(1080.1)}
\textsuperscript{98:2.11} Toda Grecia se dedicó a estos nuevos métodos de conseguir la salvación, a estos ceremoniales ardientes y emotivos. Ninguna nación alcanzó nunca unas cotas tan altas de filosofía artística en un tiempo tan corto; ninguna creó nunca un sistema ético tan avanzado, prácticamente sin una Deidad y totalmente desprovisto de promesas de salvación humana. Ninguna nación se hundió nunca tan rápida, profunda y violentamente en un abismo semejante de estancamiento intelectual, depravación moral y pobreza espiritual como estos mismos pueblos griegos cuando se arrojaron al torbellino insensato de los cultos de misterio.

\par
%\textsuperscript{(1080.2)}
\textsuperscript{98:2.12} Las religiones han podido durar mucho tiempo sin apoyo filosófico, pero pocas filosofías han sobrevivido mucho, como tales, sin identificarse de alguna manera con una religión. La filosofía es a la religión lo que la idea es a la acción. Pero el estado ideal humano es aquél en el que la filosofía, la religión y la ciencia están soldadas en una unidad significativa gracias a la acción conjunta de la sabiduría, la fe y la experiencia.

\section*{3. Las enseñanzas de Melquisedek en Roma}
\par
%\textsuperscript{(1080.3)}
\textsuperscript{98:3.1} Después de tener su origen en las primitivas formas religiosas de adoración de los dioses familiares, y de pasar por la veneración tribal de Marte, el dios de la guerra, era natural que la religión posterior de los latinos fuera mucho más una observancia política que los sistemas intelectuales de los griegos y de los brahmanes, o que las religiones más espirituales de otros diversos pueblos.

\par
%\textsuperscript{(1080.4)}
\textsuperscript{98:3.2} Durante el gran renacimiento monoteísta del evangelio de Melquisedek que se produjo en el siglo sexto antes de Cristo, muy pocos misioneros de Salem penetraron en Italia, y aquellos que lo hicieron fueron incapaces de vencer la influencia del clero etrusco en rápida expansión, con su nueva constelación de dioses y templos, los cuales quedaron todos integrados en la religión estatal romana. Esta religión de las tribus latinas no era banal y corrupta como la de los griegos, ni tampoco austera y tiránica como la de los hebreos; consistía principalmente en la simple observancia de las formas, los votos y los tabúes.

\par
%\textsuperscript{(1080.5)}
\textsuperscript{98:3.3} La religión romana sufrió la profunda influencia de las abundantes importaciones culturales procedentes de Grecia. La mayor parte de los dioses olímpicos fueron finalmente trasplantados e incorporados en el panteón latino. Los griegos adoraron durante mucho tiempo la lumbre del fuego familiar ---Hestia era la diosa virgen del fuego familiar; Vesta era la diosa romana del hogar. Zeus se convirtió en Júpiter, Afrodita se transformó en Venus, y así sucesivamente con las numerosas deidades del Olimpo.

\par
%\textsuperscript{(1080.6)}
\textsuperscript{98:3.4} La iniciación religiosa de los jóvenes romanos era la ocasión en que se consagraban solemnemente al servicio del Estado. Los juramentos y el reconocimiento como ciudadanos eran en realidad ceremonias religiosas. Los pueblos latinos mantenían templos, altares y santuarios y, en caso de crisis, solían consultar a los oráculos. Conservaban los huesos de los héroes y, más tarde, los de los santos cristianos.

\par
%\textsuperscript{(1080.7)}
\textsuperscript{98:3.5} Esta forma oficial y poco emotiva de patriotismo seudorreligioso estaba condenada a derrumbarse, al igual que la adoración extremadamente intelectual y artística de los griegos había sucumbido ante la adoración ferviente y profundamente emotiva de los cultos de misterio. El más importante de estos cultos devastadores era la religión de misterio de la secta de la Madre de Dios, que en aquellos tiempos tenía su sede en el lugar exacto de la actual iglesia de San Pedro, en Roma.

\par
%\textsuperscript{(1080.8)}
\textsuperscript{98:3.6} El Estado romano emergente fue políticamente conquistador, pero fue conquistado a su vez por los cultos, rituales, misterios y conceptos sobre dios de Egipto, Grecia y el Levante. Estos cultos importados continuaron floreciendo en todo el Estado romano hasta la época de Augusto, quien por razones puramente políticas y cívicas hizo un esfuerzo heroico, y en cierto modo con éxito, por destruir los misterios y restablecer la antigua religión política.

\par
%\textsuperscript{(1081.1)}
\textsuperscript{98:3.7} Uno de los sacerdotes de la religión estatal le contó a Augusto las tentativas anteriores de los educadores de Salem por diseminar la doctrina de un solo Dios, de una Deidad final que gobernaba a todos los seres sobrenaturales; esta idea se apoderó tan firmemente del emperador que construyó numerosos templos, los abasteció abundantemente con hermosas imágenes, reorganizó el clero del Estado, restableció la religión estatal, se nombró a sí mismo sumo sacerdote en ejercicio de todos y, como emperador, no dudó en proclamarse dios supremo.

\par
%\textsuperscript{(1081.2)}
\textsuperscript{98:3.8} Esta nueva religión del culto a Augusto floreció y fue respetada en todo el imperio durante su vida, excepto en Palestina, la patria de los judíos. Esta época de dioses humanos continuó hasta que el culto oficial romano contuvo una lista de más de cuarenta deidades humanas que se habían encumbrado a sí mismas, alegando todas ellas nacimientos milagrosos y otros atributos sobrehumanos.

\par
%\textsuperscript{(1081.3)}
\textsuperscript{98:3.9} Un ferviente grupo de predicadores, los cínicos, opuso la última resistencia que presentó la agrupación decreciente de creyentes salemitas; exhortaron a los romanos a que abandonaran sus rituales religiosos salvajes e insensatos y a que volvieran a una forma de culto que incluyera el evangelio de Melquisedek, tal como éste se había modificado y contaminado a causa de su contacto con la filosofía de los griegos. Pero el pueblo en general rechazó a los cínicos; prefirieron sumergirse en los rituales de los misterios, que no solamente ofrecían esperanzas de salvación personal, sino que también satisfacían el deseo de diversión, de emociones y de distracción.

\section*{4. Los cultos de misterio}
\par
%\textsuperscript{(1081.4)}
\textsuperscript{98:4.1} Como la mayoría de los habitantes del mundo grecorromano habían perdido sus religiones primitivas familiares y estatales, y como eran incapaces o no deseaban captar el significado de la filosofía griega, desviaron su atención hacia los cultos de misterio espectaculares y emotivos de Egipto y del Levante. La gente común y corriente deseaba ardientemente promesas de salvación ---un consuelo religioso para hoy y las seguridades de una esperanza de inmortalidad para después de la muerte.

\par
%\textsuperscript{(1081.5)}
\textsuperscript{98:4.2} Los tres cultos de misterio que se volvieron más populares fueron:

\par
%\textsuperscript{(1081.6)}
\textsuperscript{98:4.3} 1. El culto frigio de Cibeles y su hijo Atis.

\par
%\textsuperscript{(1081.7)}
\textsuperscript{98:4.4} 2. El culto egipcio de Osiris y su madre Isis.

\par
%\textsuperscript{(1081.8)}
\textsuperscript{98:4.5} 3. El culto iraní de la adoración de Mitra como salvador y redentor de la humanidad pecadora.

\par
%\textsuperscript{(1081.9)}
\textsuperscript{98:4.6} Los misterios frigio y egipcio enseñaban que el hijo divino (Atis y Osiris respectivamente) había pasado por la muerte y había sido resucitado por el poder divino, y que además todos los que eran iniciados adecuadamente en el misterio y celebraran respetuosamente el aniversario de la muerte y la resurrección del dios, compartirían de este modo su naturaleza divina y su inmortalidad.

\par
%\textsuperscript{(1081.10)}
\textsuperscript{98:4.7} Las ceremonias frigias eran impresionantes pero degradantes; sus fiestas sangrientas indican hasta qué punto se degradaron y se volvieron primitivos estos misterios levantinos. El día más sagrado era el Viernes Negro, el <<día de la sangre>>, que conmemoraba la muerte voluntaria de Atis. Después de celebrar durante tres días el sacrificio y la muerte de Atis, la fiesta se convertía en un regocijo en honor de su resurrección.

\par
%\textsuperscript{(1082.1)}
\textsuperscript{98:4.8} Los ritos del culto de Isis y Osiris eran más refinados e impresionantes que los del culto frigio. Este rito egipcio estaba construido alrededor de la leyenda del antiguo dios del Nilo, un dios que murió y fue resucitado; este concepto provenía de la observación de que el crecimiento de la vegetación se detiene periódicamente cada año, y es seguido por el restablecimiento de todas las plantas vivientes durante la primavera. La observancia frenética de estos cultos de misterio y las orgías de sus ceremonias, que conducían supuestamente al <<entusiasmo>> de la comprensión de la divinidad, eran a veces sumamente repugnantes.

\section*{5. El culto de Mitra}
\par
%\textsuperscript{(1082.2)}
\textsuperscript{98:5.1} Los misterios frigios y egipcios desaparecieron finalmente ante el culto de misterio más importante de todos, la adoración de Mitra. El culto mitríaco resultaba atractivo para una amplia gama de temperamentos humanos y sustituyó gradualmente a sus dos predecesores. El mitracismo se extendió por el imperio romano gracias a la propaganda de las legiones romanas reclutadas en el Levante, donde esta religión estaba de moda, pues los soldados llevaban esta creencia por dondequiera que iban. Este nuevo rito religioso supuso un gran progreso sobre los cultos de misterio anteriores.

\par
%\textsuperscript{(1082.3)}
\textsuperscript{98:5.2} El culto de Mitra surgió en Irán y sobrevivió durante mucho tiempo en su tierra natal a pesar de la oposición militante de los seguidores de Zoroastro. Pero en la época en que el mitracismo llegó a Roma, había mejorado considerablemente debido a la absorción de numerosas enseñanzas de Zoroastro. La religión de Zoroastro ejerció su influencia sobre el cristianismo que apareció más tarde principalmente a través del culto mitríaco.

\par
%\textsuperscript{(1082.4)}
\textsuperscript{98:5.3} El culto mitríaco describía a un dios belicoso que había tenido su origen en una gran roca, que realizaba valientes hazañas, y que hacía brotar agua de una roca golpeándola con sus flechas. Había un diluvio del que se había salvado un hombre en un barco especialmente construido, y una última cena que Mitra celebraba con el dios Sol antes de ascender al cielo. Este dios Sol, o Sol Invictus, era una degeneración de Ahura-Mazda, el concepto de la deidad en el zoroastrismo. A Mitra se le concebía como el campeón sobreviviente del dios Sol en su lucha contra el dios de las tinieblas. En reconocimiento por haber matado al toro mítico sagrado, Mitra fue hecho inmortal, siendo elevado a la posición de intercesor por la raza humana ante los dioses del cielo.

\par
%\textsuperscript{(1082.5)}
\textsuperscript{98:5.4} Los adeptos de este culto lo practicaban en cuevas y en otros lugares secretos, donde cantaban himnos, murmuraban palabras mágicas, comían la carne de los animales sacrificados y bebían su sangre. Adoraban tres veces al día, con ceremonias semanales especiales el día del dios Sol, y la celebración más esmerada de todas tenía lugar durante la fiesta anual de Mitra, el veinticinco de diciembre. Se creía que compartir el sacramento aseguraba la vida eterna, el paso inmediato, después de la muerte, al seno de Mitra, donde se permanecía en la dicha hasta el día del juicio. Ese día, las llaves mitríacas del cielo abrirían las puertas del Paraíso para recibir a los fieles; entonces, todos los no bautizados entre los vivos y los muertos serían aniquilados en el momento del regreso de Mitra a la Tierra. Se enseñaba que cuando un hombre moría iba a la presencia de Mitra para ser juzgado, y que al final del mundo, Mitra llamaría a todos los muertos de sus tumbas para que afrontaran el juicio final. Los malvados serían destruidos por el fuego, y los justos reinarían con Mitra para siempre.

\par
%\textsuperscript{(1082.6)}
\textsuperscript{98:5.5} Al principio sólo era una religión para hombres, y los creyentes podían iniciarse sucesivamente en siete órdenes diferentes. Más tarde, las esposas y las hijas de los creyentes fueron admitidas en los templos de la Gran Madre, que estaban contiguos a los templos mitríacos. El culto de las mujeres era una mezcla del ritual mitríaco y de las ceremonias del culto frigio de Cibeles, la madre de Atis.

\section*{6. El mitracismo y el cristianismo}
\par
%\textsuperscript{(1083.1)}
\textsuperscript{98:6.1} Antes de la llegada de los cultos de misterio y del cristianismo, la religión personal apenas se había desarrollado como institución independiente en los países civilizados de África del norte y de Europa; era más bien un asunto de familia, de ciudad-Estado, de política y de imperio. Los griegos helénicos no desarrollaron nunca un sistema de culto centralizado; el ritual era local; no tenían ni clero ni <<libro sagrado>>. Casi al igual que los romanos, sus instituciones religiosas carecían de un poderoso agente motor que sirviera para preservar los valores morales y espirituales más elevados. Aunque es cierto que la institucionalización de la religión ha reducido generalmente su calidad espiritual, es también un hecho que ninguna religión ha logrado sobrevivir hasta ahora sin la ayuda de algún tipo de organización institucional, más grande o más pequeña.

\par
%\textsuperscript{(1083.2)}
\textsuperscript{98:6.2} La religión occidental languideció así hasta la época de los escépticos, los cínicos, los epicúreos y los estoicos, pero muy en particular hasta los tiempos de la gran controversia entre el mitracismo y la nueva religión cristiana de Pablo.

\par
%\textsuperscript{(1083.3)}
\textsuperscript{98:6.3} Durante el siglo tercero después de Cristo, las iglesias mitríaca y cristiana eran muy similares tanto en su apariencia como en el carácter de sus rituales. La mayoría de sus lugares de culto eran subterráneos, y las dos contenían altares cuyos trasfondos representaban de manera variada los sufrimientos del salvador que había traído la salvación a una raza humana maldita por el pecado.

\par
%\textsuperscript{(1083.4)}
\textsuperscript{98:6.4} Los adoradores de Mitra siempre habían tenido la costumbre de mojar sus dedos en agua bendita al entrar en el templo. Y como en algunos barrios había personas que pertenecían al mismo tiempo a las dos religiones, introdujeron esta costumbre en la mayoría de las iglesias cristianas cercanas a Roma. La dos religiones empleaban el bautismo y compartían el sacramento del pan y del vino. La única gran diferencia entre el mitracismo y el cristianismo, aparte del carácter de Mitra y de Jesús, consistía en que el primero estimulaba el militarismo mientras que el segundo era ultrapacífico. La tolerancia del mitracismo hacia otras religiones (excepto hacia el cristianismo posterior) le condujo a su ruina final. Pero el factor decisivo en la lucha entre los dos fue la admisión de las mujeres como miembros de pleno derecho en la comunidad de la fe cristiana.

\par
%\textsuperscript{(1083.5)}
\textsuperscript{98:6.5} La fe cristiana nominal terminó por dominar en Occidente. La filosofía griega suministró los conceptos de valor ético, el mitracismo aportó el ritual de la observancia del culto, y el cristianismo como tal proporcionó la técnica para conservar los valores morales y sociales.

\section*{7. La religión cristiana}
\par
%\textsuperscript{(1083.6)}
\textsuperscript{98:7.1} Un Hijo Creador no se encarnó en la similitud de la carne mortal ni se donó a la humanidad de Urantia para reconciliarla con un Dios enojado, sino más bien para conseguir que todos los hombres reconocieran el amor del Padre y fueran conscientes de su filiación con Dios. Después de todo, incluso el gran defensor de la doctrina de la expiación comprendió una parte de esta verdad, pues declaró que <<Dios estaba, en Cristo, reconciliando el mundo consigo mismo>>\footnote{\textit{Reconciliación de Cristo}: Ro 5:10; 2 Co 5:19.}.

\par
%\textsuperscript{(1084.1)}
\textsuperscript{98:7.2} No es incumbencia de este documento tratar sobre el origen y la difusión de la religión cristiana. Es suficiente con decir que está construida alrededor de la persona de Jesús de Nazaret, el Hijo Miguel de Nebadon encarnado como ser humano, conocido en Urantia como el Cristo, el ungido. El cristianismo fue difundido por todo el Levante y Occidente por los seguidores de este galileo, y su entusiasmo misionero igualó al de sus ilustres predecesores, los setitas y los salemitas, así como al de sus fervientes contemporáneos asiáticos, los educadores budistas.

\par
%\textsuperscript{(1084.2)}
\textsuperscript{98:7.3} La religión cristiana, como sistema de creencia urantiano, surgió de la combinación de las enseñanzas, influencias, creencias, cultos y actitudes individuales personales siguientes:

\par
%\textsuperscript{(1084.3)}
\textsuperscript{98:7.4} 1. Las enseñanzas de Melquisedek, que son un factor fundamental en todas las religiones que han surgido en Oriente y Occidente durante los últimos cuatro mil años.

\par
%\textsuperscript{(1084.4)}
\textsuperscript{98:7.5} 2. El sistema hebreo de moralidad, ética, teología y creencia tanto en la Providencia como en el Yahvé supremo.

\par
%\textsuperscript{(1084.5)}
\textsuperscript{98:7.6} 3. El concepto zoroastriano de la lucha entre el bien y el mal cósmicos, que ya había dejado su huella tanto en el judaísmo como en el mitracismo. Debido al contacto prolongado que acompañó a las luchas entre el mitracismo y el cristianismo, las doctrinas del profeta iraní fueron un factor poderoso en la determinación de la apariencia y la estructura teológicas y filosóficas de los dogmas, los principios y la cosmología de las versiones helenizada y latinizada de las enseñanzas de Jesús.

\par
%\textsuperscript{(1084.6)}
\textsuperscript{98:7.7} 4. Los cultos de misterio, especialmente el mitracismo, pero también la adoración de la Gran Madre en el culto frigio. Incluso las leyendas sobre el nacimiento de Jesús en Urantia fueron contaminadas por la versión romana del nacimiento milagroso de Mitra, el héroe-salvador iraní, cuya venida a la Tierra sólo había sido supuestamente presenciada por un puñado de pastores cargados de regalos que habían sido informados de este acontecimiento inminente por los ángeles.

\par
%\textsuperscript{(1084.7)}
\textsuperscript{98:7.8} 5. El hecho histórico de la vida humana de Josué ben José, la realidad de Jesús de Nazaret como Cristo glorificado, el Hijo de Dios.

\par
%\textsuperscript{(1084.8)}
\textsuperscript{98:7.9} 6. El punto de vista personal de Pablo de Tarso. Y hay que señalar que el mitracismo era la religión dominante en Tarso durante su adolescencia. Pablo poco podía imaginar que sus cartas bienintencionadas a sus conversos serían algún día consideradas por los cristianos posteriores como la <<palabra de Dios>>. Los educadores bienintencionados como Pablo no deben ser considerados responsables del uso que sus sucesores más tardíos han hecho de sus escritos.

\par
%\textsuperscript{(1084.9)}
\textsuperscript{98:7.10} 7. El pensamiento filosófico de los pueblos helenos, desde Alejandría y Antioquía, pasando por Grecia, hasta Siracusa y Roma. La filosofía de los griegos estaba más en armonía con la versión paulina del cristianismo que con cualquier otro sistema religioso en curso, y llegó a ser un factor importante en el éxito del cristianismo en Occidente. La filosofía griega, unida a la teología de Pablo, forma todavía la base de la ética europea.

\par
%\textsuperscript{(1084.10)}
\textsuperscript{98:7.11} A medida que las enseñanzas originales de Jesús penetraron en Occidente, fueron occidentalizadas, y a medida que fueron occidentalizadas, empezaron a perder su atracción potencialmente universal para todas las razas y tipos de hombres. El cristianismo de hoy se ha convertido en una religión bien adaptada a las costumbres sociales, económicas y políticas de las razas blancas. Hace tiempo que dejó de ser la religión de Jesús, aunque todavía presenta valientemente una hermosa religión acerca de Jesús a aquellas personas que intentan seguir sinceramente el camino de sus enseñanzas. El cristianismo ha glorificado a Jesús como Cristo, el ungido mesiánico de Dios, pero ha olvidado ampliamente el evangelio personal del Maestro: la Paternidad de Dios y la fraternidad universal de todos los hombres.

\par
%\textsuperscript{(1085.1)}
\textsuperscript{98:7.12} Ésta es la larga historia de las enseñanzas de Maquiventa Melquisedek en Urantia. Hace cerca de cuatro mil años que este Hijo de emergencia de Nebadon se donó en Urantia, y durante este tiempo las enseñanzas del <<sacerdote de El Elyón, el Dios Altísimo>>\footnote{\textit{Sacerdote de El Elyón, el Dios Altísimo}: Gn 14:18; Heb 7:1.}, han penetrado en todas las razas y pueblos. Y Maquiventa consiguió el objetivo de su donación excepcional: cuando Miguel se preparó para aparecer en Urantia, el concepto de Dios estaba presente en el corazón de los hombres y las mujeres, el mismo concepto de Dios que vuelve a brillar otra vez en la experiencia espiritual viviente de los numerosos hijos del Padre Universal, a medida que viven sus enigmáticas vidas temporales en los planetas que giran en el espacio.

\par
%\textsuperscript{(1085.2)}
\textsuperscript{98:7.13} [Presentado por un Melquisedek de Nebadon.]


\newpage
\pagestyle{empty}

\par {\huge Abreviaturas}
\bigbreak
\bigbreak
\begin{multicols}{2}
	\par LU \textit{(El Libro de Urantia)}
	\bigbreak
	\par Libros bíblicos:
	\bigbreak
	\par Abd \textit{(Abdías)}
	\par Am \textit{(Amós)}
	\par Ap \textit{(Apocalipsis)}
	\par Bar \textit{(Baruc)}
	\par Co \textit{(Epístola a los Corintios)}
	\par Cnt \textit{(El Cantar de los Cantares)}
	\par Col \textit{(Epístola a los Colosenses)}
	\par Cr \textit{(Crónicas)}
	\par Dn \textit{(Daniel)}
	\par Dt \textit{(Deuteronomio)}
	\par Ec \textit{(Eclesiastés)}
	\par Eclo \textit{(Ecclesiástico)}
	\par Ef \textit{(Epístola a los Efesios)}
	\par Esd \textit{(Esdras)}
	\par Est \textit{(Ester)}
	\par Ex \textit{(Éxodo)}
	\par Ez \textit{(Ezequiel)} 
	\par Flm \textit{(Epístola a Filemón)}
	\par Flp \textit{(Epístola a los Filipenses)}
	\par Gl \textit{(Epítosla a los Gálatas)}
	\par Gn \textit{(Génesis)}
	\par Hab \textit{(Habacuc)} 
	\par Hag \textit{(Ageo)}
	\par Hch \textit{(Hechos de los Apóstoles)}
	\par Heb \textit{(Epístola a los Hebreos)}
	\par Is \textit{(Isaías)}
	\par Jer \textit{(Jeremías)}
	\par Jl \textit{(Joel)}
	\par Jn \textit{(Juan, evangelio y epístolas)}
	\par Job \textit{(Job)}
	\par Jon \textit{(Jonás)}
	\par Jos \textit{(Josué)}
	\par Jud \textit{(Epístola de Judas)}
	\par Jue \textit{(Jueces)}
	\par Lc \textit{(Lucas)}
	\par Lm \textit{(Lamentaciones)}
	\par Lv \textit{(Levítico)}
	\par Mac \textit{(Macabeos)}
	\par Mal \textit{(Malaquías)}
	\par Mc \textit{(Marcos)}
	\par Miq \textit{(Miqueas)} 
	\par Mt \textit{(Mateo)}
	\par Nah \textit{(Nahúm)}
	\par Neh \textit{(Nehemías)} 
	\par Nm \textit{(Números)}
	\par Os \textit{(Oseas)}
	\par P \textit{(Epístola de Pedro)}
	\par Pr \textit{(Proverbios)}
	\par Re \textit{(Reyes)}
	\par Ro \textit{(Epístola a los Romanos)}
	\par Rt \textit{(Rut)}
	\par Sab \textit{(Sabiduría)}
	\par Sal \textit{(Salmos)}
	\par Sam \textit{(Samuel)}
	\par Sof \textit{(Sofonías)}
	\par Stg \textit{(Epístola a Santiago)}
	\par Ti \textit{(Epístola a Timoteo)}
	\par Tit \textit{(Epítosla a Tito)}
	\par Ts \textit{(Epístola a los Tesalonicenses)}
	\par Zac \textit{(Zacarías)}
	\bigbreak
	\par Libros bíblicos apócrifos:
	\bigbreak 
	\par AsMo \textit{(Asunción de Moisés)}
	\par Bel \textit{(Bel y el Dragón)} 
	\par Hen \textit{(Enoc)} 
	\par Man \textit{(Oración de Manasés)} 
	\par Tb \textit{(Tobit)}
	\bigbreak
	\par Libros de otras religiones: 
	\bigbreak
	\par XXX \textit{(YYYY)}
	
	
\end{multicols}

\end{document}

