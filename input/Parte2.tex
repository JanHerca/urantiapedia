% Author of this conversion to LaTeX format: Jan Herca, 2017
\documentclass[twoside, 11pt]{book}
\usepackage[T1]{fontenc} % indica al procesador cómo imprimir los caracteres
\usepackage{fontspec} % permite definir fuentes a partir de las instaladas en el SO
\usepackage{geometry}
\usepackage{graphicx}
\usepackage{float}
\usepackage{tocloft}
\usepackage{titleps}
\usepackage{emptypage}
\usepackage[spanish]{babel}
\usepackage{multicol}
% Text styles
\geometry{paperwidth=16cm, paperheight=24cm, top=2.5cm, bottom=1.7cm, inner=2.5cm, outer=1.2cm}

\makeatletter
\def\@makechapterhead#1{%
	\vspace*{50\p@}%
	{\parindent \z@ \raggedright \normalfont
		\interlinepenalty\@M
		\huge \bfseries #1\par\nobreak
		\vskip 40\p@
}}
\def\@makeschapterhead#1{%
	\vspace*{50\p@}%
	{\parindent \z@ \raggedright
		\normalfont
		\interlinepenalty\@M
		\huge \bfseries  #1\par\nobreak
		\vskip 40\p@
}}
\makeatother

\renewcommand{\cftchapleader}{\cftdotfill{\cftdotsep}}
\renewcommand{\thechapter}{}
\renewcommand{\cftchapfont}{\large}
\cftsetpnumwidth{3em}
\renewcommand{\cftchappagefont}{\large}


\title{La Quinta Revelación \newline Segundo Volumen \newline El Universo Local}
\date{}
\begin{document}
	
\begin{titlepage}
	\centering
	{\Huge\bfseries El Libro de Urantia\par}
	{\huge\bfseries La Quinta Revelación\par}
	\vspace{1cm}
	{\huge\bfseries Segundo Volumen\par}
	\vspace{1cm}
	{\huge\bfseries El Universo Local\par}
	\vfill
	{\scshape\Large URANTIA FOUNDATION\par}
	{\scshape\Large CHICAGO ILLINOIS\par}
	{\Large 2009 Traducción al español Europea\par}
\end{titlepage}
	
	
\par {\textcopyright} 2019 Jan Herca, de la edición
\par {\textcopyright} 2009 Urantia Foundation, de la traducción
\par {\textcopyright} 1993 Urantia Foundation, de otros materiales
\bigbreak
\par Jan Herca
\par Correo electrónico: janherca@gmail.com
\bigbreak
\par Urantia Foundation
\par 533 West Diversey Parkway
\par Chicago, IL 60614 EE.UU.A
\par Oficina: 1+(773) 525-3319
\par Fax: 1 +(773) 525-7739
\par Website: http://www.urantia.org
\par Correo electrónico: urantia@urantia.org
\bigbreak
\par Todos los derechos reservados, incluyendo el de traducción en los Estados Unidos de América, Canadá y en los demás países de la Unión Internacional de copyright. Todos los derechos reservados en los paises firmantes de la Union Panamericana de la Union internacional de copyright.
\par No todo el libro ni parte de él pueden ser copiados, reproducidos o traducidos en forma alguna, ya sea por medio electrónico, mecánico u otra forma, como fotocopia, grabación o archivo computerizado sin autorización por escrito del editor.
\par URANTIA,'' ``URANTIAN,'' ``EL LIBRO DE URANTIA'' y son marcas registradas de Urantia Foundation y su uso está sujeto a licencia.
\bigbreak
\par La Quinta Revelación es una reedición de El Libro de Urantia (Edición Europea). Está dividido en siete volúmenes para hacerlo más manejable y dispone de contenido adicional en forma de ayudas a la lectura integradas en el texto. El Libro de Urantia (Edición Europea) es una traducción de The Urantia Book realizada por la Fundación Urantia en 2009. 
\newpage

\begin{center}
	{\huge\bfseries Las partes del libro\par}
	\vspace{1cm}
	{\scshape\large PRIMER VOLUMEN\par}
	{\scshape\Large DIOS, EL UNIVERSO CENTRAL Y LOS SUPERUNIVERSOS\par}
	\vspace{1cm}
	
	{\scshape\large SEGUNDO VOLUMEN \par}
	{\scshape\Large EL UNIVERSO LOCAL\par}
	\vspace{1cm}
	
	{\scshape\large TERCER VOLUMEN \par}
	{\scshape\Large LA HISTORIA DE NUESTRO PLANETA, URANTIA\par}
	\vspace{1cm}
	
	{\scshape\large CUARTO VOLUMEN \par}
	{\scshape\Large LA EVOLUCIÓN DE LA CIVILIZACIÓN HUMANA\par}
	\vspace{1cm}
	
	{\scshape\large QUINTO VOLUMEN \par}
	{\scshape\Large LA RELIGIÓN, LA SOBREVIVENCIA A LA MUERTE Y LA DEIDAD EXPERIENCIAL\par}
	\vspace{1cm}
	
	{\scshape\large SEXTO VOLUMEN \par}
	{\scshape\Large LA VIDA Y LAS ENSEÑANZAS DE JESÚS - I\par}
	\vspace{1cm}
	
	{\scshape\large SÉPTIMO VOLUMEN \par}
	{\scshape\Large LA VIDA Y LAS ENSEÑANZAS DE JESÚS - II\par}
\end{center}
	
\newpage
\begin{center}
	{\small \textit {Intencionadamente en blanco}\par}
\end{center}
\newpage

\pagestyle{empty}


\tableofcontents

\newpagestyle{main}{
	%\setheadrule{.4pt}% Header rule
	%\setfootrule{.4pt}% Footer rule
	\sethead[\small \thepage]% odd-left
	[]% odd-center
	[\begin{minipage}{0.9\textwidth}\begin{flushright}\scriptsize \MakeUppercase{\chaptertitle}\end{flushright}\end{minipage}]% odd-right
	{\begin{minipage}{0.9\textwidth}\scriptsize \MakeUppercase{\chaptertitle}\end{minipage}}% even-left
	{}% even-center
	{\small \thepage}% even-right
	\setfoot[]% odd-left
	[]% odd-center
	[]% odd-right
	{}% even-left
	{}% even-center
	{}% even-right
}
\pagestyle{main}
\renewcommand{\makeheadrule}{\rule[-.6\baselineskip]{\linewidth}{.4pt}}


\chapter{Documento 32. La evolución de los universos locales}
\par
%\textsuperscript{(357.1)}
\textsuperscript{32:0.1} UN UNIVERSO local es la obra de un Hijo Creador de la orden paradisiaca de los Migueles. Consta de cien constelaciones, y cada una de ellas abarca cien sistemas de mundos habitados. Cada sistema contendrá finalmente unas mil esferas habitadas.

\par
%\textsuperscript{(357.2)}
\textsuperscript{32:0.2} Estos universos del tiempo y del espacio son todos evolutivos. El plan creativo de los Migueles del Paraíso sigue siempre el curso de la evolución gradual y del desarrollo progresivo de las naturalezas y de las capacidades físicas, intelectuales y espirituales de las múltiples criaturas que habitan los diversos tipos de esferas que componen ese universo local.

\par
%\textsuperscript{(357.3)}
\textsuperscript{32:0.3} Urantia pertenece a un universo local cuyo soberano es el Dios-hombre de Nebadon, Jesús de Nazaret y Miguel de Salvington\footnote{\textit{Miguel de Salvington}: Jn 1:1-3.}. Todos los planes de Miguel para este universo local fueron plenamente aprobados por la Trinidad del Paraíso antes de que Miguel emprendiera la aventura suprema del espacio.

\par
%\textsuperscript{(357.4)}
\textsuperscript{32:0.4} Los Hijos de Dios pueden elegir los reinos de sus actividades creadoras, pero los Arquitectos Paradisiacos del Universo Maestro son los que proyectan y planifican originariamente estas creaciones materiales.

\section*{1. Aparición física de los universos}
\par
%\textsuperscript{(357.5)}
\textsuperscript{32:1.1} Las manipulaciones preuniversales de la fuerza espacial y de las energías primordiales son obra de los Organizadores de la Fuerza Maestros del Paraíso; pero en los dominios superuniversales, cuando la energía emergente se vuelve sensible a la gravedad local o lineal, los Organizadores de la Fuerza se retiran a favor de los directores del poder del superuniverso interesado.

\par
%\textsuperscript{(357.6)}
\textsuperscript{32:1.2} Estos directores del poder actúan solos en las fases de la creación de un universo local anteriores a la materia y posteriores a la fuerza. Un Hijo Creador no tiene ninguna posibilidad de empezar la organización de su universo hasta que los directores del poder no han efectuado la suficiente movilización de las energías espaciales como para proporcionar una base material ---soles tangibles y esferas materiales--- al universo emergente.

\par
%\textsuperscript{(357.7)}
\textsuperscript{32:1.3} Todos los universos locales tienen aproximadamente el mismo potencial energético, aunque difieren enormemente en sus dimensiones físicas y puedan variar de vez en cuando en su contenido de materia visible. La carga de poder y la dotación de materia potencial de un universo local están determinadas por las manipulaciones de los directores del poder y sus predecesores, así como por las actividades del Hijo Creador y por la dotación sobre el control físico inherente que posee su asociada creativa.

\par
%\textsuperscript{(358.1)}
\textsuperscript{32:1.4} La carga energética de un universo local es la cienmilésima parte aproximadamente de la dotación de fuerza de su superuniverso. En el caso de Nebadon, vuestro universo local, la materialización de la masa es un poco menor. Hablando en sentido físico, Nebadon posee toda la dotación física de energía y de materia que se puede encontrar en cualquier creación local de Orvonton. La única limitación física a la expansión del desarrollo del universo de Nebadon consiste en la carga cuantitativa de energía espacial mantenida cautiva por el control gravitatorio de los poderes y de las personalidades asociados del mecanismo universal combinado.

\par
%\textsuperscript{(358.2)}
\textsuperscript{32:1.5} Cuando la energía-materia ha alcanzado cierto grado de materialización de la masa, un Hijo Creador Paradisiaco aparece en escena, acompañado de una Hija Creativa del Espíritu Infinito. Al mismo tiempo que llega el Hijo Creador se empieza el trabajo de construir la esfera arquitectónica que llegará a convertirse en el mundo sede del universo local en proyecto. Esta creación local evoluciona durante largas eras, los soles se estabilizan, los planetas se forman y giran en sus órbitas, mientras continúa el trabajo de creación de los mundos arquitectónicos que van a servir como sedes de las constelaciones y como capitales de los sistemas.

\section*{2. Organización de los universos}
\par
%\textsuperscript{(358.3)}
\textsuperscript{32:2.1} A los Hijos Creadores los preceden, en la organización de sus universos, los directores del poder y otros seres que tienen su origen en la Fuente-Centro Tercera. A partir de las energías del espacio, organizadas previamente de esta manera, Miguel, vuestro Hijo Creador, estableció los reinos habitados del universo de Nebadon y desde entonces se ha dedicado cuidadosamente a administrarlos. A partir de la energía preexistente, estos Hijos divinos materializan la materia visible, proyectan las criaturas vivientes y, con la cooperación de la presencia en sus universos del Espíritu Infinito, crean un variado séquito de personalidades espirituales.

\par
%\textsuperscript{(358.4)}
\textsuperscript{32:2.2} Estos directores del poder y estos controladores de la energía que precedieron con tanta antelación al Hijo Creador en el trabajo físico preliminar de organizar su universo, sirven posteriormente en magnífica coordinación con este Hijo del Universo, conservando para siempre el control asociado de aquellas energías que al principio organizaron e incorporaron en sus circuitos. En Salvington ejercen actualmente su actividad los mismos cien centros del poder que cooperaron con vuestro Hijo Creador para formar inicialmente este universo local.

\par
%\textsuperscript{(358.5)}
\textsuperscript{32:2.3} El primer acto de creación física que se efectuó en Nebadon consistió en organizar el mundo sede, la esfera arquitectónica de Salvington, con sus satélites. Desde el momento de las acciones iniciales de los centros del poder y de los controladores físicos hasta la llegada del personal viviente a las esferas terminadas de Salvington, transcurrió un poco más de mil millones de años de vuestro tiempo actual planetario. A la construcción de Salvington le siguió de inmediato la creación de los cien mundos sede de las constelaciones en proyecto, y de las diez mil esferas sede de los sistemas locales en proyecto destinadas al control y a la administración planetarios, junto con sus satélites arquitectónicos. Estos mundos arquitectónicos están diseñados para alojar a las personalidades físicas y a las espirituales, así como a los estados intermedios de existencia morontiales o de transición.

\par
%\textsuperscript{(359.1)}
\textsuperscript{32:2.4} Salvington, la sede central de Nebadon, está situada en el centro exacto de energía-masa del universo local. Pero vuestro universo local no es un sistema astronómico simple, aunque existe un sistema de gran tamaño en su centro físico.

\par
%\textsuperscript{(359.2)}
\textsuperscript{32:2.5} Salvington es la sede personal de Miguel de Nebadon, pero éste no siempre se encuentra allí. Aunque el funcionamiento armonioso de vuestro universo local ya no necesita la presencia permanente del Hijo Creador en la esfera capital, esto no era así en las épocas iniciales de la organización física. Un Hijo Creador no puede dejar su mundo sede hasta el momento en que se ha efectuado la estabilización gravitatoria del reino mediante la materialización de una energía suficiente como para permitir que los diversos circuitos y sistemas se equilibren entre sí mediante una atracción material mutua.

\par
%\textsuperscript{(359.3)}
\textsuperscript{32:2.6} Poco después termina el proyecto físico de un universo y el Hijo Creador, en asociación con el Espíritu Creativo, diseña su plan para crear la vida; después de lo cual, esta representante del Espíritu Infinito empieza su actividad universal como personalidad creativa distinta. Cuando se formula y se ejecuta este primer acto creador, surge a la existencia la Radiante Estrella Matutina, la personificación de este concepto creativo inicial de identidad e ideal de divinidad. Éste es el jefe ejecutivo del universo, el asociado personal del Hijo Creador, un ser semejante a él en todos los aspectos del carácter, aunque notablemente limitado en sus atributos de divinidad.

\par
%\textsuperscript{(359.4)}
\textsuperscript{32:2.7} Y ahora que el brazo derecho y jefe ejecutivo del Hijo Creador ha aparecido, a esto le sigue la venida a la existencia de una inmensa y maravillosa serie de criaturas diversas. Los hijos y las hijas del universo local aparecen, y poco después se le proporciona un gobierno a esta creación, un gobierno que se extiende desde los consejos supremos del universo hasta los padres de las constelaciones y los soberanos de los sistemas locales ---los conjuntos de mundos que están destinados a convertirse posteriormente en las moradas de las diversas razas mortales de criaturas volitivas; y cada uno de estos mundos será presidido por un Príncipe Planetario.

\par
%\textsuperscript{(359.5)}
\textsuperscript{32:2.8} Luego, cuando ese universo ha sido completamente organizado y plenamente equipado de personal, el Hijo Creador emprende el proyecto del Padre consistente en crear al hombre mortal a su divina imagen\footnote{\textit{Hombre a imagen de Dios}: Gn 1:26-27; 9:6.}.

\par
%\textsuperscript{(359.6)}
\textsuperscript{32:2.9} La organización de las moradas planetarias continúa desarrollándose en Nebadon, pues este universo es en verdad un grupo joven en los reinos estelares y planetarios de Orvonton. En el momento del último registro había en Nebadon 3.840.101 planetas habitados, y Satania, el sistema local de vuestro mundo, es bastante típico en relación con los otros sistemas.

\par
%\textsuperscript{(359.7)}
\textsuperscript{32:2.10} Satania no es un sistema físico uniforme, una unidad u organización astronómica simple. Sus 619 mundos habitados están situados en más de quinientos sistemas físicos diferentes. Sólo cinco tienen más de dos mundos habitados, y de estos cinco uno solo tiene cuatro planetas poblados, mientras que hay cuarenta y seis que tienen dos mundos habitados.

\par
%\textsuperscript{(359.8)}
\textsuperscript{32:2.11} El sistema de mundos habitados de Satania está muy alejado de Uversa y del gran grupo de soles que funciona como centro físico o astronómico del séptimo superuniverso. Desde Jerusem, la sede central de Satania, hay más de doscientos mil años luz hasta el centro físico del superuniverso de Orvonton, situado lejos, muy lejos en el denso diámetro de la Vía Láctea. Satania se encuentra en la periferia del universo local, y Nebadon se halla ahora muy afuera hacia el borde de Orvonton. Desde el sistema más alejado de mundos habitados hasta el centro del superuniverso hay un poco menos de doscientos cincuenta mil años luz.

\par
%\textsuperscript{(360.1)}
\textsuperscript{32:2.12} El universo de Nebadon gira ahora lejos en el sureste del circuito superuniversal de Orvonton. Los universos vecinos más cercanos son: Avalon, Henselon, Sanselon, Portalon, Wolvering, Fanoving y Alvoring.

\par
%\textsuperscript{(360.2)}
\textsuperscript{32:2.13} Pero la evolución de un universo local es una larga historia. Los documentos que tratan del superuniverso presentan este tema; los de esta sección, que tratan de las creaciones locales, lo continúan, mientras que los documentos siguientes, que se refieren a la historia y al destino de Urantia, completan el relato. Pero sólo podéis comprender adecuadamente el destino de los mortales de una creación local como ésta, estudiando la narración de la vida y las enseñanzas de vuestro Hijo Creador tal como vivió en otra época la vida del hombre, en la similitud de la carne mortal, en vuestro propio mundo evolutivo.

\section*{3. La idea evolutiva}
\par
%\textsuperscript{(360.3)}
\textsuperscript{32:3.1} La única creación que está perfectamente estabilizada es Havona, el universo central, que fue creada directamente por el pensamiento del Padre Universal y la palabra del Hijo Eterno. Havona es un universo existencial, perfecto y repleto, que rodea la morada de las Deidades eternas, el centro de todas las cosas. Las creaciones de los siete superuniversos son finitas, evolutivas y, en consecuencia, progresivas.

\par
%\textsuperscript{(360.4)}
\textsuperscript{32:3.2} Todos los sistemas físicos del tiempo y del espacio tienen un origen evolutivo. Ni siquiera están estabilizados físicamente hasta que no son incorporados en los circuitos establecidos de sus superuniversos. Un universo local tampoco está establecido en la luz y la vida hasta que no se han agotado sus posibilidades físicas de expansión y de desarrollo, y hasta que el estado espiritual de todos sus mundos habitados no se ha establecido y estabilizado para siempre.

\par
%\textsuperscript{(360.5)}
\textsuperscript{32:3.3} La perfección es una consecución progresiva, excepto en el universo central. La creación central nos sirve como modelo de perfección, pero todos los demás reinos deben alcanzar esa perfección mediante los métodos establecidos para el progreso de esos mundos o universos particulares. Y los planes de los Hijos Creadores para organizar, hacer evolucionar, disciplinar y estabilizar sus universos locales respectivos están caracterizados por una variedad casi infinita.

\par
%\textsuperscript{(360.6)}
\textsuperscript{32:3.4} A excepción de la presencia de deidad del Padre, cada universo local es, en cierto sentido, una reproducción de la organización administrativa de la creación central o modelo. Aunque el Padre Universal está personalmente presente en el universo residencial, no habita en la mente de los seres que tienen su origen en ese universo, tal como sí habita literalmente en el alma de los mortales del tiempo y del espacio. Parece haber una compensación infinitamente sabia en el ajuste y la reglamentación de los asuntos espirituales de la extensa creación. En el universo central, el Padre está personalmente presente como tal, pero está ausente de la mente de los hijos de esa creación perfecta; en los universos del espacio, el Padre está ausente en persona, estando representado por sus Hijos Soberanos, mientras que se encuentra íntimamente presente en la mente de sus hijos mortales, estando espiritualmente representado por la presencia prepersonal de los Monitores de Misterio que residen en la mente de estas criaturas volitivas.

\par
%\textsuperscript{(360.7)}
\textsuperscript{32:3.5} En la sede de un universo local residen todas las personalidades creadoras y creativas que representan una autoridad independiente y una autonomía administrativa, excepto la presencia personal del Padre Universal. En el universo local se puede encontrar a casi todas las clases de seres inteligentes que existen en el universo central, salvo al Padre Universal. Aunque el Padre Universal no está personalmente presente en un universo local, está representado personalmente por su Hijo Creador, al principio vicegerente de Dios y posteriormente gobernante supremo y soberano por su propio derecho.

\par
%\textsuperscript{(361.1)}
\textsuperscript{32:3.6} Cuanto más descendemos la escala de la vida, más difícil es localizar, con los ojos de la fe, al Padre invisible. A las criaturas inferiores ---y a veces incluso a las personalidades superiores--- siempre les resulta difícil ver al Padre Universal en sus Hijos Creadores. Así pues, hasta el momento de su exaltación espiritual en que la perfección de su desarrollo les permitirá ver a Dios en persona, las criaturas se cansan en su progresión, albergan dudas espirituales, tropiezan en la confusión y se aíslan así de las metas espirituales progresivas de su época y de su universo. De esta manera pierden la capacidad de ver al Padre cuando contemplan al Hijo Creador. Durante la larga lucha por alcanzar al Padre, durante el período en que las condiciones inherentes hacen que esta consecución resulte imposible, la salvaguardia más segura para la criatura consiste en aferrarse tenazmente al hecho-verdad de la presencia del Padre en sus Hijos. Literal y figurativamente, espiritual y personalmente, el Padre y los Hijos son uno solo\footnote{\textit{El Padre y el Hijo son uno}: Jn 1:1; 5:17-18; 10:30,38; 14:7-11,20; 17:11,21-22.}. Es un hecho: aquel que ha visto a un Hijo Creador ha visto al Padre\footnote{\textit{Quien ha visto al Hijo, ha visto al Padre}: Jn 12:45; 14:7-9.}.

\par
%\textsuperscript{(361.2)}
\textsuperscript{32:3.7} Las personalidades de un universo dado sólo son estables y fiables, al principio, de acuerdo con su grado de parecido con la Deidad. Cuando el origen de las criaturas se aparta bastante de las Fuentes originales y divinas, ya se trate de los Hijos de Dios o de las criaturas ministrantes pertenecientes al Espíritu Infinito, existe la posibilidad de que aumente la falta de armonía, la confusión y a veces la rebelión ---el pecado.

\par
%\textsuperscript{(361.3)}
\textsuperscript{32:3.8} A excepción de los seres perfectos que tienen su origen en la Deidad, todas las criaturas volitivas de los superuniversos son de naturaleza evolutiva; empiezan en un estado humilde y se elevan siempre hacia arriba, en realidad hacia el interior. Incluso las personalidades sumamente espirituales continúan ascendiendo la escala de la vida mediante traslados progresivos de vida en vida y de esfera en esfera. Y en el caso de aquellos que reciben Monitores de Misterio, las alturas posibles de su ascensión espiritual y de sus logros universales no tienen en verdad ningún límite.

\par
%\textsuperscript{(361.4)}
\textsuperscript{32:3.9} Cuando las criaturas del tiempo alcanzan finalmente la perfección, ésta es enteramente una adquisición, una auténtica posesión de la personalidad. Aunque los elementos de la gracia estén abundantemente mezclados, los logros de las criaturas son sin embargo el resultado de sus esfuerzos individuales y de sus vivencias reales, de la reacción de su personalidad al entorno existente.

\par
%\textsuperscript{(361.5)}
\textsuperscript{32:3.10} A los ojos del universo, el hecho de tener un origen evolutivo animal no supone un estigma para ninguna personalidad, puesto que éste es el método exclusivo de engendrar uno de los dos tipos fundamentales de criaturas volitivas inteligentes finitas. Cuando las alturas de la perfección y de la eternidad se han alcanzado, tanto más honor para aquellos que empezaron desde abajo y ascendieron alegremente la escala de la vida, peldaño tras peldaño y que, cuando lleguen a las alturas de la gloria, habrán adquirido una experiencia personal que abarcará un conocimiento real de cada fase de la vida desde abajo hasta arriba.

\par
%\textsuperscript{(361.6)}
\textsuperscript{32:3.11} La sabiduría de los Creadores se manifiesta en todo esto. Al Padre Universal le resultaría igual de fácil hacer que todos los mortales fueran seres perfectos, comunicarles la perfección mediante su palabra divina. Pero esto los privaría de la maravillosa experiencia de la aventura y de la formación asociadas a la larga ascensión gradual hacia el interior, una experiencia que sólo pueden poseer aquellos que son tan afortunados como para empezar en el punto más bajo de la existencia viviente.

\par
%\textsuperscript{(362.1)}
\textsuperscript{32:3.12} Los universos que rodean a Havona sólo están provistos del número suficiente de criaturas perfectas que puedan satisfacer la necesidad de guías instructores modelos para aquellos que están ascendiendo la escala evolutiva de la vida. La naturaleza experiencial del tipo evolutivo de personalidad es el complemento cósmico natural de la naturaleza siempre perfecta de las criaturas del Paraíso-Havona. En realidad, tanto las criaturas perfectas como las criaturas perfeccionadas son incompletas con respecto a la totalidad finita. Pero en la asociación complementaria entre las criaturas existencialmente perfectas del sistema Paraíso-Havona y los finalitarios experiencialmente perfeccionados que ascienden de los universos evolutivos, los dos tipos encuentran la liberación de sus limitaciones inherentes y pueden intentar así alcanzar de manera conjunta las alturas sublimes del estado último de las criaturas.

\par
%\textsuperscript{(362.2)}
\textsuperscript{32:3.13} Estas actividades de las criaturas son las repercusiones universales de acciones y reacciones en el interior de la Deidad Séptuple, en la que la divinidad eterna de la Trinidad del Paraíso se une con la divinidad evolutiva de los Creadores Supremos de los universos espacio-temporales en, por medio de, y a través de, la Deidad del Ser Supremo cuyo poder está en vías de manifestarse.

\par
%\textsuperscript{(362.3)}
\textsuperscript{32:3.14} La criatura divinamente perfecta y la criatura evolutiva perfeccionada tienen el mismo grado de potencial de divinidad, pero son de una especie diferente. Cada una tiene que depender de la otra para alcanzar la supremacía del servicio. Los superuniversos evolutivos dependen del perfecto Havona para que proporcione la formación final a sus ciudadanos ascendentes, pero el perfecto universo central también necesita la existencia de los superuniversos que se perfeccionan para que proporcionen el pleno desarrollo a sus habitantes descendentes.

\par
%\textsuperscript{(362.4)}
\textsuperscript{32:3.15} Las dos manifestaciones primordiales de la realidad finita, la perfección innata y la perfección adquirida por evolución, ya se trate de personalidades o de universos, son dependientes y están coordinadas e integradas. Cada una necesita a la otra para conseguir que sus funciones, su servicio y su destino sean completos.

\section*{4. Las relaciones de Dios con un universo local}
\par
%\textsuperscript{(362.5)}
\textsuperscript{32:4.1} No alberguéis la idea de que, puesto que el Padre Universal ha delegado en otros una parte tan grande de sí mismo y de su poder, es un miembro silencioso o inactivo de la asociación de las Deidades. Aparte de los dominios de la personalidad y de la concesión de los Ajustadores, es en apariencia la menos activa de las Deidades del Paraíso, ya que permite que sus coordinados en Deidad, sus Hijos, y numerosas inteligencias creadas, realicen tantas cosas con el fin de llevar a cabo su propósito eterno. Pero sólo es el miembro silencioso del trío creativo en el sentido de que nunca hace nada que cualquiera de sus asociados coordinados o subordinados puedan hacer.

\par
%\textsuperscript{(362.6)}
\textsuperscript{32:4.2} Dios comprende plenamente la necesidad que tiene cada criatura inteligente de actuar y de experimentar y, por lo tanto, en todas las situaciones, ya se trate del destino de un universo o del bienestar de la más humilde de sus criaturas, Dios se retira de la actividad a favor de la galaxia de personalidades creadas y Creadoras que intervienen de manera inherente entre él mismo y cualquier situación universal o acontecimiento creativo dados. Pero a pesar de este retiro, de esta manifestación de coordinación infinita, hay por parte de Dios una participación real, literal y personal en estos acontecimientos por medio de, y a través de, dichos agentes y personalidades ordenados. El Padre trabaja en todos estos canales, y a través de ellos, por el bienestar de toda su extensa creación.

\par
%\textsuperscript{(363.1)}
\textsuperscript{32:4.3} En lo que se refiere a la política, la conducta y la administración de un universo local, el Padre Universal actúa a través de la persona de su Hijo Creador. En las relaciones entre los Hijos de Dios, en las asociaciones colectivas de las personalidades que tienen su origen en la Fuente-Centro Tercera, o en las relaciones entre otras criaturas tales como los seres humanos ---en lo que concierne a estas asociaciones, el Padre Universal no interviene nunca. La ley del Hijo Creador, el gobierno de los Padres de las Constelaciones, de los Soberanos de los Sistemas y de los Príncipes Planetarios ---la política y los procedimientos ordenados para ese universo--- prevalecen siempre. No hay ninguna división de autoridad; nunca hay oposición entre el poder y el propósito divinos. Las Deidades actúan con unanimidad perfecta y eterna.

\par
%\textsuperscript{(363.2)}
\textsuperscript{32:4.4} El Hijo Creador gobierna de manera suprema en todas las cuestiones relacionadas con las asociaciones éticas, las relaciones entre cualquier agrupación de criaturas y cualquier otra clase de criaturas, o entre dos o más individuos dentro de un grupo dado; pero este plan no significa que el Padre Universal no pueda intervenir a su propia manera, y hacer lo que le agrada a la mente divina con cualquier \textit{criatura individual} en toda la creación, en lo referente al estado actual o a las perspectivas futuras de ese individuo, y conforme al plan eterno y al propósito infinito del Padre.

\par
%\textsuperscript{(363.3)}
\textsuperscript{32:4.5} En las criaturas mortales volitivas, el Padre está realmente presente mediante el Ajustador interior, un fragmento de su espíritu prepersonal; y el Padre es también la fuente de la personalidad de dichas criaturas mortales volitivas.

\par
%\textsuperscript{(363.4)}
\textsuperscript{32:4.6} Estos Ajustadores del Pensamiento, donados por el Padre Universal, están relativamente aislados; habitan la mente humana pero no tienen ninguna conexión perceptible con las cuestiones éticas de una creación local. No están directamente coordinados con el servicio seráfico ni con la administración de los sistemas, las constelaciones o un universo local, y ni siquiera con el gobierno de un Hijo Creador, cuya voluntad es la ley suprema de su universo.

\par
%\textsuperscript{(363.5)}
\textsuperscript{32:4.7} Los Ajustadores interiores son uno de los modos de contacto particulares, pero unificados, de Dios con las criaturas de su creación casi infinita. El que es invisible para el hombre mortal manifiesta así su presencia\footnote{\textit{El Dios invisible manifestado}: Col 1:15-16; 1 Ti 1:17; Heb 11:27.} y, si pudiera hacerlo, se mostraría a nosotros además de otras maneras, pero una revelación adicional así no es divinamente posible.

\par
%\textsuperscript{(363.6)}
\textsuperscript{32:4.8} Podemos ver y comprender el mecanismo por el cual los Hijos disfrutan de un conocimiento íntimo y completo de los universos que están bajo su jurisdicción; pero no podemos comprender plenamente los métodos por los cuales Dios está tan plena y tan personalmente familiarizado con los detalles del universo de universos, aunque al menos podemos reconocer la vía por la cual el Padre Universal puede recibir información acerca de los seres de su inmensa creación, y manifestarles su presencia. A través de su circuito de personalidad, el Padre conoce ---tiene un conocimiento personal--- de todos los pensamientos y todos los actos de todos los seres de todos los sistemas de todos los universos de toda la creación. Aunque no podemos captar plenamente esta técnica de la comunión de Dios con sus hijos, podemos sentirnos fortalecidos en la seguridad de que <<el Señor conoce a sus hijos>>\footnote{\textit{El Señor conoce a sus hijos}: 1 Re 8:39; 2 Cr 6:30; 2 Ti 2:19.}, y de que <<toma nota del lugar donde hemos nacido>> cada uno de nosotros.

\par
%\textsuperscript{(363.7)}
\textsuperscript{32:4.9} Espiritualmente hablando, el Padre Universal está presente, en vuestro universo y en vuestro corazón, por medio de uno de los Siete Espíritus Maestros de la morada central y, específicamente, mediante el Ajustador divino que vive, trabaja y espera en las profundidades de la mente mortal.

\par
%\textsuperscript{(363.8)}
\textsuperscript{32:4.10} Dios no es una personalidad egocéntrica; el Padre se distribuye generosamente a su creación y a sus criaturas. Vive y actúa no sólo en las Deidades, sino también en sus Hijos, a quienes les confía la realización de todo aquello que les es divinamente posible realizar. El Padre Universal se ha despojado realmente de toda función que puede ser realizada por otro ser. Y esto es tan cierto en lo que concierne al hombre mortal como al Hijo Creador que gobierna en lugar de Dios en la sede de un universo local. Así es como contemplamos la manifestación del amor ideal e infinito del Padre Universal.

\par
%\textsuperscript{(364.1)}
\textsuperscript{32:4.11} En esta donación universal de sí mismo tenemos una prueba abundante de la magnitud y de la magnanimidad de la naturaleza divina del Padre. Si Dios ha retenido algo para sí mismo de la creación universal, entonces de ese residuo está confiriendo, con una profusa generosidad, los Ajustadores del Pensamiento a los mortales de los reinos, los Monitores de Misterio del tiempo que con tanta paciencia habitan en los candidatos mortales a la vida eterna.

\par
%\textsuperscript{(364.2)}
\textsuperscript{32:4.12} El Padre Universal se ha derramado, por decirlo así, para que toda la creación se enriquezca con la posesión de la personalidad y el potencial de la consecución espiritual. Dios se ha dado a nosotros para que podamos parecernos a él, y sólo se ha reservado el poder y la gloria necesarios para mantener aquellas cosas por cuyo amor se ha despojado así de todo lo demás.

\section*{5. El propósito eterno y divino}
\par
%\textsuperscript{(364.3)}
\textsuperscript{32:5.1} Existe un propósito grande y glorioso en la marcha de los universos a través del espacio. Todas vuestras luchas mortales no tienen lugar en vano. Todos formamos parte de un plan inmenso, de una empresa gigantesca, y la enormidad de la empresa es la que hace que sea imposible ver una gran parte de ella en un momento dado y durante una vida determinada. Todos formamos parte de un proyecto eterno que los Dioses supervisan y están llevando a cabo. Todo el maravilloso mecanismo universal se mueve majestuosamente a través del espacio al compás de la música del pensamiento infinito y del propósito eterno de la Gran Fuente-Centro Primera.

\par
%\textsuperscript{(364.4)}
\textsuperscript{32:5.2} El propósito eterno del Dios eterno es un ideal espiritual elevado. Los acontecimientos del tiempo y las luchas de la existencia material no son más que el andamiaje transitorio que tiende un puente hacia el otro lado, hacia la tierra prometida de la realidad espiritual y de la existencia celestial. Por supuesto que a vosotros los mortales os resulta difícil captar la idea de un propósito eterno; sois prácticamente incapaces de comprender la idea de la eternidad, de algo que nunca empieza y que nunca termina. Todo lo que os es familiar tiene un final.

\par
%\textsuperscript{(364.5)}
\textsuperscript{32:5.3} En lo que se refiere a una vida individual, a la duración de un reino o a la cronología de una serie conectada de acontecimientos, parecería que estamos tratando con un intervalo aislado de tiempo; todo parece tener un comienzo y un final. Y podría parecer que cuando una serie de estas experiencias, vidas, eras o épocas está enlazada de manera sucesiva, forma un camino recto, un acontecimiento aislado del tiempo, que pasa momentáneamente como un relámpago por delante del rostro infinito de la eternidad. Pero cuando contemplamos todo esto desde detrás del escenario, una visión más comprensiva y un entendimiento más completo sugieren que dicha explicación está desconectada, es inadecuada y totalmente inapropiada para explicar convenientemente las transacciones del tiempo, y correlacionarlas además con los propósitos subyacentes y las reacciones fundamentales de la eternidad.

\par
%\textsuperscript{(364.6)}
\textsuperscript{32:5.4} A fin de poder explicarlo a la mente de los mortales, a mí me parece más adecuado concebir la eternidad como un ciclo, y el propósito eterno como un círculo sin fin, un ciclo de eternidad sincronizado de alguna manera con los ciclos transitorios materiales del tiempo. En lo que se refiere a los sectores del tiempo conectados con el ciclo de la eternidad, del cual forman parte, nos vemos obligados a reconocer que estas épocas temporales nacen, viven y mueren exactamente como nacen, viven y mueren los seres transitorios del tiempo. La mayoría de los seres humanos mueren porque no han logrado alcanzar el nivel espiritual de fusión con el Ajustador, y la metamorfosis de la muerte constituye el único procedimiento posible por el que pueden escapar de las cadenas del tiempo y de las trabas de la creación material, lo que les permite adoptar el paso espiritual de la procesión progresiva de la eternidad. Después de sobrevivir a la vida de prueba del tiempo y de la existencia material, os será posible continuar en contacto con la eternidad, e incluso como una parte de ella, girando para siempre con los mundos del espacio alrededor del círculo de las eras eternas.

\par
%\textsuperscript{(365.1)}
\textsuperscript{32:5.5} Los sectores del tiempo se parecen a los destellos de la personalidad en su forma temporal; aparecen durante una temporada, y luego los ojos humanos los pierden de vista, para reaparecer después como actores nuevos y factores continuos en la vida superior del movimiento sin fin alrededor del círculo eterno. La eternidad difícilmente se puede concebir como un camino en línea recta, en vista de nuestra creencia en un universo delimitado que se mueve en un enorme círculo alargado alrededor de la morada central del Padre Universal.

\par
%\textsuperscript{(365.2)}
\textsuperscript{32:5.6} Con toda sinceridad, la eternidad es incomprensible para la mente finita del tiempo. Simplemente no la podéis abarcar; no podéis comprenderla. Yo no la visualizo por completo, y aunque lo hiciera me resultaría imposible transmitir mi concepto a la mente humana. Sin embargo, he hecho todo lo posible por describir una parte de nuestro punto de vista, por contaros un poco nuestra comprensión de las cosas eternas. Me esfuerzo por ayudaros a cristalizar vuestros pensamientos sobre estos valores que son de naturaleza infinita y de importancia eterna.

\par
%\textsuperscript{(365.3)}
\textsuperscript{32:5.7} En la mente de Dios hay un plan que incluye a todas las criaturas de todos sus inmensos dominios, y este plan consiste en un propósito eterno de oportunidades sin límites, de progreso ilimitado y de vida sin fin. ¡Y los tesoros infinitos de esta carrera incomparable serán vuestros con tal que os esforcéis por alcanzarlos!

\par
%\textsuperscript{(365.4)}
\textsuperscript{32:5.8} ¡La meta de la eternidad está hacia adelante! ¡La aventura para alcanzar la divinidad se extiende delante de vosotros! ¡La carrera hacia la perfección está en marcha! Quienquiera que lo desee puede participar\footnote{\textit{Quien lo quiera puede participar}: Sal 50:15; Jl 2:32; Zac 13:9; Mt 7:24; 10:32-33; 12:50; 16:24-25; Mc 3:35; 8:34-35; Lc 6:47; 9:23-24; 12:8; Jn 3:15-16; 4:13-14; 11:25-26; 12:46; Hch 2:21; 10:42-43; 13:26; Ro 9:33; 10:13; 1 Jn 2:23; 4:15; 5:1; Ap 22:17b.}, y una victoria segura\footnote{\textit{Victoria segura}: 2 Ti 4:6-8.} coronará los esfuerzos de todo ser humano que corra la carrera de la fe y de la confianza, dependiendo a cada paso del camino de las directrices del Ajustador interior y de la guía de ese buen espíritu del Hijo del Universo\footnote{\textit{Espíritu de la Verdad}: Ez 11:19; 18:31; 36:26-27; Jl 2:28-29; Lc 24:49; Jn 7:39; 14:16-18,23,26; 15:4,26; 16:7-8,13-14; 17:21-23; Hch 1:5,8a; 2:1-4,16-18; 2:33; 2 Co 13:5; Gl 2:20; 4:6; Ef 1:13; 4:30; 1 Jn 4:12-15.} que ha sido derramado tan generosamente sobre toda carne.

\par
%\textsuperscript{(365.5)}
\textsuperscript{32:5.9} [Presentado por un Mensajero Poderoso vinculado temporalmente al Consejo Supremo de Nebadon y asignado a esta misión por Gabriel de Salvington.]


\chapter{Documento 33. La administración del universo local}
\par
%\textsuperscript{(366.1)}
\textsuperscript{33:0.1} AUNQUE el Padre Universal gobierna con toda seguridad sobre su inmensa creación, en la administración de un universo local actúa a través de la persona del Hijo Creador. El Padre no actúa personalmente de otra manera en los asuntos administrativos de un universo local. Estas materias las ha confiado al Hijo Creador, al Espíritu Madre del universo local, y a los múltiples hijos de ambos. Los planes, la política y los actos administrativos del universo local son concebidos y ejecutados por este Hijo, el cual, conjuntamente con su Espíritu asociado, delega el poder ejecutivo en Gabriel, y la autoridad jurisdiccional en los Padres de las Constelaciones, los Soberanos de los Sistemas y los Príncipes Planetarios.

\section*{1. Miguel de Nebadon}
\par
%\textsuperscript{(366.2)}
\textsuperscript{33:1.1} Nuestro Hijo Creador es la personificación del concepto original
611.121 de identidad infinita que tuvo origen simultáneamente en el Padre Universal y el Hijo Eterno. El Miguel de Nebadon es el <<Hijo unigénito>>\footnote{\textit{Hijo unigénito}: Sal 2:7; Jn 1:14,18; Jn 3:16,18; Hch 13:33; Heb 1:5; Heb 5:5; 1 Jn 4:9.} que personaliza este 611.121{\textordmasculine} concepto universal de divinidad y de infinidad. Su sede se encuentra en la triple mansión de luz en Salvington. Y esta morada está dispuesta así porque Miguel ha experimentado el modo de vivir de las tres fases de la existencia de las criaturas inteligentes: la espiritual, la morontial y la material. Debido al nombre asociado a su séptima y última donación en Urantia, a veces se le llama Cristo Miguel.

\par
%\textsuperscript{(366.3)}
\textsuperscript{33:1.2} Nuestro Hijo Creador no es el Hijo Eterno, el asociado existencial paradisiaco del Padre Universal y del Espíritu Infinito. Miguel de Nebadon no es un miembro de la Trinidad del Paraíso. Sin embargo, nuestro Hijo Maestro posee en su reino todos los atributos y poderes divinos que el mismo Hijo Eterno manifestaría si estuviera efectivamente presente en Salvington y ejerciera su actividad en Nebadon. Miguel posee incluso un poder y una autoridad adicionales, porque no sólo personifica al Hijo Eterno, sino que también representa plenamente y expresa efectivamente la presencia de personalidad del Padre Universal para este universo local, y en él. Representa incluso al Padre-Hijo. Estas relaciones hacen de un Hijo Creador el más poderoso, polifacético e influyente de todos los seres divinos capaces de administrar directamente los universos evolutivos y de ponerse en contacto personal con las criaturas inmaduras.

\par
%\textsuperscript{(366.4)}
\textsuperscript{33:1.3} Desde la sede del universo local, nuestro Hijo Creador ejerce el mismo poder de atracción espiritual, la misma gravedad espiritual, que el Hijo Eterno del Paraíso ejercería si estuviera personalmente presente en Salvington, e incluso \textit{más aún}; este Hijo del Universo es también la personificación del Padre Universal para el universo de Nebadon. Los Hijos Creadores son los centros de personalidad para las fuerzas espirituales del Padre-Hijo Paradisiacos. Los Hijos Creadores son las focalizaciones finales del poder y de la personalidad de los poderosos atributos espacio-temporales de Dios Séptuple.

\par
%\textsuperscript{(367.1)}
\textsuperscript{33:1.4} El Hijo Creador personaliza la vicegerencia del Padre Universal, es el coordinado en divinidad del Hijo Eterno, y el asociado creativo del Espíritu Infinito. A todos los efectos prácticos, el Hijo Soberano es Dios\footnote{\textit{El Padre y el Hijo son uno}: Jn 1:1; 5:17-18; 10:30,38; 14:7-11,20; 17:11,21-22.} para nuestro universo y todos sus mundos habitados. Personifica todo lo que los mortales evolutivos pueden comprender con discernimiento de las Deidades del Paraíso\footnote{\textit{Ver al hijo es ver al Padre}: Jn 12:45; 14:7-9.}. Este Hijo y su Espíritu asociado \textit{son} vuestros padres creadores. Para vosotros, Miguel, el Hijo Creador, es la personalidad suprema; para vosotros, el Hijo Eterno es supersupremo ---una personalidad infinita de la Deidad.

\par
%\textsuperscript{(367.2)}
\textsuperscript{33:1.5} En la persona del Hijo Creador tenemos a un gobernante y a un padre divino que es exactamente tan poderoso, eficaz y benefactor como lo serían el Padre Universal y el Hijo Eterno si los dos estuvieran presentes en Salvington y se ocuparan de la administración de los asuntos del universo de Nebadon.

\section*{2. El Soberano de Nebadon}
\par
%\textsuperscript{(367.3)}
\textsuperscript{33:2.1} El observar a los Hijos Creadores revela que algunos se parecen más al Padre, otros al Hijo, mientras que otros son una mezcla de sus dos padres infinitos. Nuestro Hijo Creador manifiesta muy claramente unas características y unos atributos que se parecen más a los del Hijo Eterno.

\par
%\textsuperscript{(367.4)}
\textsuperscript{33:2.2} Miguel eligió organizar este universo local, y ahora reina aquí de manera suprema. Su poder personal está limitado por los circuitos gravitatorios preexistentes centrados en el Paraíso, y por el hecho de que los Ancianos de los Días del gobierno superuniversal se reservan todos los juicios ejecutivos finales relacionados con la extinción de la personalidad. La personalidad es el don exclusivo del Padre, pero los Hijos Creadores, con la aprobación del Hijo Eterno, inician nuevos proyectos de criaturas, y con la cooperación de trabajo de sus Espíritus asociados pueden intentar nuevas transformaciones de la energía-materia.

\par
%\textsuperscript{(367.5)}
\textsuperscript{33:2.3} Miguel es la personificación del Padre-Hijo Paradisiacos para, y en, el universo local de Nebadon; por consiguiente, cuando el Espíritu Madre Creativo, que representa al Espíritu Infinito en el universo local, se subordinó a Cristo Miguel cuando éste regresó de su donación final en Urantia, el Hijo Maestro adquirió con ello la jurisdicción sobre <<todos los poderes en el cielo y en la Tierra>>\footnote{\textit{Todo poder en el cielo y en la Tierra}: Mt 28:18.}.

\par
%\textsuperscript{(367.6)}
\textsuperscript{33:2.4} Esta subordinación de las Ministras Divinas a los Hijos Creadores de los universos locales convierte a estos Hijos Maestros en los depositarios personales de la divinidad, manifestable de manera finita, del Padre, del Hijo y del Espíritu, mientras que las experiencias donadoras de los Migueles bajo la forma de sus criaturas los cualifican para representar la divinidad experiencial del Ser Supremo. No existen otros seres en los universos que hayan agotado personalmente así los potenciales de la experiencia finita actual, y no hay otros seres en los universos que posean unas aptitudes semejantes para ejercer la soberanía solitaria.

\par
%\textsuperscript{(367.7)}
\textsuperscript{33:2.5} Aunque la sede central de Miguel está situada oficialmente en Salvington, la capital de Nebadon, pasa una gran parte de su tiempo visitando las sedes de las constelaciones y de los sistemas, e incluso los planetas individuales. Viaja periódicamente al Paraíso y con frecuencia a Uversa, donde mantiene sesiones con los Ancianos de los Días. Cuando está fuera de Salvington, Gabriel ocupa su lugar y actúa entonces como regente del universo de Nebadon.

\section*{3. El Hijo y el Espíritu del universo}
\par
%\textsuperscript{(368.1)}
\textsuperscript{33:3.1} Aunque el Espíritu Infinito impregna todos los universos del tiempo y del espacio, actúa desde la sede de cada universo local como una focalización especializada que adquiere todas las cualidades de la personalidad mediante la técnica de la cooperación creativa con el Hijo Creador. En lo que se refiere a un universo local, la autoridad administrativa de un Hijo Creador es suprema; el Espíritu Infinito, bajo la forma de Ministra Divina, es totalmente cooperativo aunque está perfectamente coordinado.

\par
%\textsuperscript{(368.2)}
\textsuperscript{33:3.2} El Espíritu Madre Universal de Salvington, la asociada de Miguel en el control y la administración de Nebadon, pertenece al sexto grupo de los Espíritus Supremos y lleva el número 611.121 de esta orden. Se ofreció como voluntaria para acompañar a Miguel cuando éste fue liberado de sus obligaciones paradisiacas, y desde entonces siempre ha trabajado con él para crear y gobernar su universo.

\par
%\textsuperscript{(368.3)}
\textsuperscript{33:3.3} El Hijo Maestro Creador es el soberano personal de su universo, pero en todos los detalles de la administración, el Espíritu del Universo es codirector con el Hijo. Aunque el Espíritu siempre reconoce al Hijo como soberano y gobernante, el Hijo siempre le concede al Espíritu una posición coordinada y una autoridad igual a la suya en todos los asuntos del reino. En todo su trabajo de amor y de donación de la vida, el Hijo Creador está siempre y para siempre perfectamente apoyado y hábilmente asistido por el Espíritu del Universo omnisapiente y siempre fiel, y por todo su séquito diversificado de personalidades angélicas. Esta Ministra Divina es en realidad la madre de los espíritus y de las personalidades espirituales, la consejera omnisapiente y siempre presente del Hijo Creador, una manifestación fiel y verdadera del Espíritu Infinito del Paraíso.

\par
%\textsuperscript{(368.4)}
\textsuperscript{33:3.4} El Hijo actúa como un padre en su universo local. El Espíritu, tal como lo podrían comprender las criaturas mortales, representa el papel de una madre, ayudando siempre al Hijo y permaneciendo eternamente indispensable para la administración del universo. En presencia de una insurrección, sólo el Hijo y sus Hijos asociados pueden actuar como liberadores. El Espíritu nunca puede oponerse a una rebelión ni defender la autoridad, pero el Espíritu apoya siempre al Hijo en todo lo que éste necesite experimentar en sus esfuerzos por estabilizar el gobierno y mantener la autoridad en los mundos contaminados por el mal o dominados por el pecado. Sólo un Hijo puede rehabilitar la obra que han creado juntos, pero ningún Hijo podría esperar el éxito final sin la cooperación incesante de la Ministra Divina y de su inmenso conjunto de asistentes espirituales, las hijas de Dios, que luchan con tanta valentía y fidelidad por el bienestar de los hombres mortales y por la gloria de sus padres divinos.

\par
%\textsuperscript{(368.5)}
\textsuperscript{33:3.5} Cuando el Hijo Creador finaliza su séptima y última donación como criatura, las incertidumbres del aislamiento periódico terminan para la Ministra Divina, y la asistente universal del Hijo se instala para siempre en la seguridad y en el control. Durante la entronización del Hijo Creador como Hijo Maestro, en el jubileo de los jubileos, es cuando el Espíritu del Universo reconoce por primera vez pública y universalmente, ante las multitudes reunidas, su subordinación al Hijo, prometiéndole fidelidad y obediencia. Este acontecimiento se produjo en Nebadon cuando Miguel regresó a Salvington después de su donación en Urantia. Antes de este importante acontecimiento, el Espíritu del Universo nunca había reconocido su subordinación al Hijo del Universo, y hasta después de esta renuncia voluntaria al poder y a la autoridad por parte del Espíritu no se pudo proclamar en verdad que <<todos los poderes en el cielo y en la Tierra han sido puestos en sus manos>>\footnote{\textit{Todo poder en el cielo y en la Tierra}: Mt 28:18.}.

\par
%\textsuperscript{(369.1)}
\textsuperscript{33:3.6} Después de esta promesa de subordinación por parte del Espíritu Madre Creativo, Miguel de Nebadon reconoció noblemente su eterna dependencia de su Espíritu compañero, nombró al Espíritu cogobernante de los dominios de su universo, y pidió a todas sus criaturas que prometieran su lealtad al Espíritu como lo habían hecho con el Hijo; entonces se promulgó y se publicó la <<Proclamación final de Igualdad>>. Aunque era el soberano de este universo local, el Hijo proclamó a los mundos el hecho de que el Espíritu era igual a él en todos los dones de la personalidad y en todos los atributos del carácter divino. Y esto se convierte en el modelo trascendente para organizar y dirigir la familia, incluso entre las criaturas humildes de los mundos del espacio. Éste es, de hecho y en verdad, el elevado ideal de la familia y de la institución humana del matrimonio voluntario.

\par
%\textsuperscript{(369.2)}
\textsuperscript{33:3.7} El Hijo y el Espíritu presiden ahora el universo de manera muy similar a como un padre y una madre velan y cuidan a su familia de hijos e hijas. No está totalmente fuera de lugar referirse al Espíritu del Universo como la compañera creativa del Hijo Creador, y considerar a las criaturas de los reinos como sus hijos e hijas ---una familia grande y gloriosa, que exige responsabilidades incalculables y cuidados sin fin.

\par
%\textsuperscript{(369.3)}
\textsuperscript{33:3.8} El Hijo inicia la creación de ciertos hijos del universo, mientras que el Espíritu tiene la única responsabilidad de traer a la existencia a las numerosas órdenes de personalidades espirituales que ayudan y sirven bajo la dirección y la guía de este mismo Espíritu Madre. En la creación de otros tipos de personalidades universales, tanto el Hijo como el Espíritu actúan juntos, y en ningún acto creativo ninguno de ellos hace nada sin el consejo y la aprobación del otro.

\section*{4. Gabriel ---el Jefe Ejecutivo}
\par
%\textsuperscript{(369.4)}
\textsuperscript{33:4.1} La Radiante Estrella Matutina es la personalización del primer concepto de la identidad y del ideal de personalidad concebido por el Hijo Creador y por la manifestación del Espíritu Infinito en el universo local. Retrocediendo a los primeros tiempos del universo local, antes de la unión del Hijo Creador y del Espíritu Madre en los lazos de una asociación creativa, allá por las épocas anteriores al comienzo de la creación de su polifacética familia de hijos e hijas, el primer acto conjunto de la asociación inicial y libre de estas dos personas divinas dio como resultado la creación de la personalidad espiritual más elevada surgida del Hijo y del Espíritu, la Radiante Estrella Matutina.

\par
%\textsuperscript{(369.5)}
\textsuperscript{33:4.2} En cada universo local sólo nace un ser con esta sabiduría y esta majestad. El Padre Universal y el Hijo Eterno pueden crear un número ilimitado de Hijos iguales en divinidad a ellos mismos, y de hecho lo hacen; pero estos Hijos, en unión con las Hijas del Espíritu Infinito, sólo pueden crear en cada universo una Radiante Estrella Matutina, un ser semejante a ellos mismos que comparte abundantemente sus naturalezas combinadas, pero no sus prerrogativas creadoras. Gabriel de Salvington se parece al Hijo del Universo en divinidad de naturaleza, aunque está considerablemente limitado en atributos de Deidad.

\par
%\textsuperscript{(369.6)}
\textsuperscript{33:4.3} Este primogénito de los padres de un nuevo universo es una personalidad única que posee muchas características maravillosas que no están presentes de manera visible en ninguno de sus progenitores, un ser de una variedad de talentos sin precedentes y de una brillantez inimaginable. Esta personalidad celestial engloba la voluntad divina del Hijo combinada con la imaginación creativa del Espíritu. Los pensamientos y los actos de la Radiante Estrella Matutina representarán siempre plenamente tanto al Hijo Creador como al Espíritu Creativo. Este ser también es capaz de comprender ampliamente y de establecer un contacto compasivo tanto con las huestes seráficas espirituales como con las criaturas volitivas materiales evolutivas.

\par
%\textsuperscript{(370.1)}
\textsuperscript{33:4.4} La Radiante Estrella Matutina no es un creador, pero es un maravilloso administrador, es el representante administrativo personal del Hijo Creador. Aparte de la creación y de la concesión de la vida, el Hijo y el Espíritu nunca deliberan sobre importantes procedimientos universales sin la presencia de Gabriel.

\par
%\textsuperscript{(370.2)}
\textsuperscript{33:4.5} Gabriel de Salvington es el jefe ejecutivo del universo de Nebadon y el árbitro de todas las apelaciones ejecutivas relacionadas con su administración. Este ejecutivo del universo fue creado plenamente dotado para su trabajo, pero ha adquirido experiencia con el crecimiento y la evolución de nuestra creación local.

\par
%\textsuperscript{(370.3)}
\textsuperscript{33:4.6} Gabriel es el director en jefe que ejecuta los mandatos superuniversales relacionados con los asuntos no personales del universo local. La mayor parte de las materias relativas a los juicios en masa y a las resurrecciones dispensacionales, juzgadas por los Ancianos de los Días, son también delegadas en Gabriel y en su estado mayor para que las ejecuten. Gabriel es así el jefe ejecutivo combinado de los gobernantes del superuniverso y del universo local. Tiene a su mando a un cuerpo capaz de asistentes administrativos, creados para su trabajo especial, que no han sido revelados a los mortales evolutivos. Además de estos asistentes, Gabriel puede emplear todas las órdenes de seres celestiales que ejercen su actividad en Nebadon, y es también el comandante en jefe de <<los ejércitos del cielo>>\footnote{\textit{Los ejércitos celestiales}: Ap 19:14.} ---de las huestes celestiales.

\par
%\textsuperscript{(370.4)}
\textsuperscript{33:4.7} Gabriel y su estado mayor no son educadores; son administradores. Nunca se ha sabido que hayan dejado su trabajo habitual, salvo cuando Miguel se encarnaba para llevar a cabo una donación como criatura. Durante estas donaciones, Gabriel siempre tenía en cuenta la voluntad del Hijo encarnado, y con la colaboración del Unión de los Días, se convirtió en el director efectivo de los asuntos del universo durante las últimas donaciones. Gabriel ha estado estrechamente identificado con la historia y el desarrollo de Urantia desde la donación humana de Miguel.

\par
%\textsuperscript{(370.5)}
\textsuperscript{33:4.8} Aparte de encontrar a Gabriel en los mundos de donación y en las épocas de los llamamientos nominales durante las resurrecciones generales y especiales, los mortales raramente lo encontrarán mientras ascienden por el universo local hasta que no sean admitidos en el trabajo administrativo de la creación local. Como administradores de cualquier tipo o de cualquier grado, estaréis bajo la dirección de Gabriel.

\section*{5. Los Embajadores de la Trinidad}
\par
%\textsuperscript{(370.6)}
\textsuperscript{33:5.1} La administración de las personalidades con origen en la Trinidad termina en el gobierno de los superuniversos. Los universos locales están caracterizados por una supervisión doble, el comienzo del concepto padre-madre. El padre del universo es el Hijo Creador; la madre del universo es la Ministra Divina, el Espíritu Creativo del universo local. Sin embargo, cada universo local está bendecido con la presencia de ciertas personalidades del universo central y del Paraíso. A la cabeza de este grupo paradisiaco en Nebadon se encuentra el embajador de la Trinidad del Paraíso ---Emmanuel de Salvington--- el Unión de los Días asignado al universo local de Nebadon. En cierto sentido, este elevado Hijo de la Trinidad es también el representante personal del Padre Universal ante la corte del Hijo Creador; de ahí su nombre Emmanuel.

\par
%\textsuperscript{(370.7)}
\textsuperscript{33:5.2} Emmanuel de Salvington, número 611.121 de la sexta orden de Personalidades Trinitarias Supremas, es un ser de una dignidad sublime y de una condescendencia tan magnífica que rehúsa el culto y la adoración de todas las criaturas vivientes. Se distingue por ser la única personalidad en todo Nebadon que nunca ha reconocido su subordinación a su hermano Miguel. Actúa como asesor del Hijo Soberano, pero sólo ofrece sus consejos si se le solicitan. En ausencia del Hijo Creador puede presidir cualquier alto consejo del universo, pero no participa de otra manera en los asuntos ejecutivos del universo a menos que se le solicite.

\par
%\textsuperscript{(371.1)}
\textsuperscript{33:5.3} Este embajador del Paraíso en Nebadon no está sometido a la jurisdicción del gobierno del universo local. Tampoco ejerce una jurisdicción autorizada sobre los asuntos ejecutivos de un universo local en evolución, salvo en lo que se refiere a la supervisión de sus hermanos coordinados, los Fieles de los Días, que sirven en las sedes de las constelaciones.

\par
%\textsuperscript{(371.2)}
\textsuperscript{33:5.4} Al igual que el Unión de los Días, los Fieles de los Días nunca proponen su asesoramiento ni ofrecen su ayuda a los gobernantes de las constelaciones a menos que se les pida. Estos embajadores del Paraíso ante las constelaciones representan la presencia personal final de los Hijos Estacionarios de la Trinidad que ejercen sus funciones consultivas en los universos locales. Las constelaciones están relacionadas más estrechamente con la administración superuniversal que los sistemas locales, los cuales están administrados exclusivamente por personalidades nativas del universo local.

\section*{6. La administración general}
\par
%\textsuperscript{(371.3)}
\textsuperscript{33:6.1} Gabriel es el jefe ejecutivo y el administrador efectivo de Nebadon. El hecho de que Miguel se ausente de Salvington no interfiere de ninguna manera la conducta ordenada de los asuntos del universo. Durante la ausencia de Miguel, como lo hizo recientemente para reunirse con los Hijos Maestros de Orvonton en el Paraíso, Gabriel es el regente del universo. En esos momentos, Gabriel siempre busca el consejo de Emmanuel de Salvington para todos los problemas importantes.

\par
%\textsuperscript{(371.4)}
\textsuperscript{33:6.2} El Padre Melquisedek es el primer ayudante de Gabriel. Cuando la Radiante Estrella Matutina está ausente de Salvington, sus responsabilidades son asumidas por este Hijo Melquisedek original.

\par
%\textsuperscript{(371.5)}
\textsuperscript{33:6.3} Las diversas subadministraciones del universo tienen asignados ciertos ámbitos de responsabilidad especiales. Aunque el gobierno de un sistema se ocupa en general del bienestar de sus planetas, se preocupa más particularmente por el estado físico de los seres vivientes, por los problemas biológicos. Los gobernantes de la constelación prestan a su vez una atención especial a las condiciones sociales y gubernamentales que prevalecen en los diferentes planetas y sistemas. El gobierno de una constelación se preocupa principalmente de la unificación y la estabilización. Más arriba aún, los gobernantes del universo se ocupan más del estado espiritual de los reinos.

\par
%\textsuperscript{(371.6)}
\textsuperscript{33:6.4} Los embajadores son nombrados por decreto judicial y representan a los universos ante otros universos. Los cónsules representan a las constelaciones entre sí y ante la sede del universo; son nombrados por decreto legislativo y sólo ejercen sus funciones dentro de los confines del universo local. Los observadores son nombrados por un decreto ejecutivo del Soberano de un Sistema para representar a ese sistema ante otros sistemas y ante la capital de la constelación, y ellos también sólo desempeñan sus funciones dentro de los confines del universo local.

\par
%\textsuperscript{(371.7)}
\textsuperscript{33:6.5} Las transmisiones se emiten simultáneamente desde Salvington hacia las sedes de las constelaciones, las sedes de los sistemas y los planetas individuales. Todas las órdenes superiores de seres celestiales son capaces de utilizar este servicio para comunicarse con sus compañeros dispersos por todo el universo. La transmisión universal se extiende a todos los mundos habitados sin tener en cuenta su estado espiritual. La intercomunicación planetaria sólo se niega a aquellos mundos que están en cuarentena espiritual.

\par
%\textsuperscript{(372.1)}
\textsuperscript{33:6.6} Las transmisiones de las constelaciones se emiten periódicamente desde la sede de la constelación por el jefe de los Padres de la Constelación.

\par
%\textsuperscript{(372.2)}
\textsuperscript{33:6.7} Un grupo especial de seres que se encuentran en Salvington son los que cuentan, calculan y rectifican la cronología. El día oficial de Nebadon equivale a dieciocho días y seis horas del tiempo de Urantia, más dos minutos y medio. El año de Nebadon consiste en un segmento del tiempo del recorrido del universo en relación con el circuito de Uversa, y equivale a cien días del tiempo oficial del universo, unos cinco años del tiempo de Urantia.

\par
%\textsuperscript{(372.3)}
\textsuperscript{33:6.8} El tiempo de Nebadon, que se transmite desde Salvington, es el tiempo oficial para todas las constelaciones y todos los sistemas de este universo local. Cada constelación dirige sus asuntos según el tiempo de Nebadon, pero los sistemas mantienen su propia cronología, tal como lo hacen los planetas individuales.

\par
%\textsuperscript{(372.4)}
\textsuperscript{33:6.9} El día de Satania, tal como se calcula en Jerusem, es un poco menos (en 1 hora, 4 minutos y 15 segundos) de tres días del tiempo de Urantia. Estos tiempos se conocen generalmente como el tiempo de Salvington o universal, y el tiempo de Satania o del sistema. El tiempo oficial es el tiempo del universo.

\section*{7. Los tribunales de Nebadon}
\par
%\textsuperscript{(372.5)}
\textsuperscript{33:7.1} Miguel, el Hijo Maestro, sólo se preocupa de manera suprema de tres cosas: la creación, el sostenimiento y el ministerio. No participa personalmente en la tarea judicial del universo. Los Creadores nunca juzgan a sus criaturas; esta función pertenece exclusivamente a las criaturas que poseen una gran formación y una experiencia real como criaturas.

\par
%\textsuperscript{(372.6)}
\textsuperscript{33:7.2} Todo el mecanismo judicial de Nebadon se encuentra bajo la supervisión de Gabriel. Los tribunales supremos, situados en Salvington, se ocupan de los problemas que tienen una importancia general para el universo y de los casos de apelación que proceden de los tribunales de los sistemas. Estas cortes universales tienen setenta ramas y funcionan en siete divisiones de diez secciones cada una. En todos los asuntos a juzgar, la presidencia es ejercida por una doble magistratura compuesta por un juez con antecedentes perfectos y un magistrado con experiencia ascendente.

\par
%\textsuperscript{(372.7)}
\textsuperscript{33:7.3} En lo que se refiere a la jurisdicción, los tribunales del universo local están limitados en las materias siguientes:

\par
%\textsuperscript{(372.8)}
\textsuperscript{33:7.4} 1. La administración del universo local se ocupa de la creación, la evolución, el mantenimiento y el ministerio. Por consiguiente, a los tribunales del universo se les rehúsa el derecho de juzgar los casos en los que está implicada la cuestión de la vida y de la muerte eternas. Esto no se refiere a la muerte natural tal como ésta prevalece en Urantia, pero si la cuestión del derecho a la existencia continuada, a la vida eterna, ha de ser juzgada, tiene que remitirse a los tribunales de Orvonton, y si el fallo es desfavorable para el individuo, todas las sentencias de extinción se ejecutan bajo las órdenes, y a través de los agentes, de los dirigentes del supergobierno.

\par
%\textsuperscript{(372.9)}
\textsuperscript{33:7.5} 2. La negligencia o la deserción de cualquier Hijo de Dios del Universo Local, que ponga en peligro su estado y su autoridad como Hijo, nunca se juzga en los tribunales de un Hijo; una desavenencia de este tipo sería llevada inmediatamente ante los tribunales del superuniverso.

\par
%\textsuperscript{(372.10)}
\textsuperscript{33:7.6} 3. La cuestión de readmitir a cualquier parte constituyente de un universo local ---por ejemplo un sistema local--- en la comunidad del pleno estado espiritual de la creación local, después de haber estado aislada espiritualmente, debe acordarse en la alta asamblea del superuniverso.

\par
%\textsuperscript{(373.1)}
\textsuperscript{33:7.7} En todos los demás casos, los tribunales de Salvington son definitivos y supremos. Sus decisiones y decretos no se pueden apelar ni eludir.

\par
%\textsuperscript{(373.2)}
\textsuperscript{33:7.8} Por muy injustamente que las controversias humanas parezcan juzgarse a veces en Urantia, la justicia y la equidad divina prevalecen en el universo. Vivís en un universo bien ordenado, y podéis contar con que tarde o temprano seréis tratados con justicia, e incluso con misericordia.

\section*{8. Las funciones legislativas y ejecutivas}
\par
%\textsuperscript{(373.3)}
\textsuperscript{33:8.1} En Salvington, la sede de Nebadon, no existen cuerpos verdaderamente legislativos. Los mundos sede de los universos se ocupan ampliamente de los juicios. Las asambleas legislativas del universo local están situadas en las sedes de las cien constelaciones. Los sistemas se ocupan principalmente del trabajo ejecutivo y administrativo de las creaciones locales. Los Soberanos de los Sistemas y sus asociados hacen cumplir los mandatos legislativos de los gobernantes de las constelaciones y ejecutan los decretos judiciales de los tribunales supremos del universo.

\par
%\textsuperscript{(373.4)}
\textsuperscript{33:8.2} Aunque en la sede del universo no se decreta una verdadera legislación, en Salvington funciona una variedad de asambleas consultivas y de investigación compuestas y dirigidas de manera diversa de acuerdo con su alcance y su propósito. Algunas son permanentes y otras se disuelven después de conseguir sus objetivos.

\par
%\textsuperscript{(373.5)}
\textsuperscript{33:8.3} \textit{El consejo supremo} del universo local está compuesto de tres miembros de cada sistema y de siete representantes de cada constelación. Los sistemas aislados no tienen representación en esta asamblea, pero se les permite enviar a sus observadores, los cuales asisten a todas las deliberaciones y las estudian.

\par
%\textsuperscript{(373.6)}
\textsuperscript{33:8.4} \textit{Los cien consejos de sanciones supremas} también están situados en Salvington. Los presidentes de estos consejos componen el gabinete de trabajo directo de Gabriel.

\par
%\textsuperscript{(373.7)}
\textsuperscript{33:8.5} Todas las conclusiones de los altos consejos consultivos del universo se remiten, o bien a los cuerpos judiciales de Salvington, o a las asambleas legislativas de las constelaciones. Estos altos consejos no tienen autoridad ni poder para hacer cumplir sus recomendaciones. Si su informe está basado en las leyes fundamentales del universo, entonces los tribunales de Nebadon emitirán los mandatos de ejecución; pero si sus recomendaciones tienen que ver con las condiciones locales o de urgencia, han de enviarse a las asambleas legislativas de la constelación para su promulgación deliberativa, y luego a las autoridades del sistema para su ejecución. Estos altos consejos son en realidad las superlegislaturas del universo, pero funcionan sin la autoridad de decretar y sin el poder de ejecutar.

\par
%\textsuperscript{(373.8)}
\textsuperscript{33:8.6} Aunque hablamos de la administración del universo en términos de <<tribunales>> y de <<asambleas>>, debe comprenderse que estas actuaciones espirituales son muy diferentes a las actividades más primitivas y materiales que llevan estos mismos nombres en Urantia.

\par
%\textsuperscript{(373.9)}
\textsuperscript{33:8.7} [Presentado por el Jefe de los Arcángeles de Nebadon.]


\chapter{Documento 34. El Espíritu Madre del universo local}
\par
%\textsuperscript{(374.1)}
\textsuperscript{34:0.1} CUANDO un Hijo Creador es personalizado por el Padre Universal y el Hijo Eterno, el Espíritu Infinito individualiza entonces una representación nueva y única de sí mismo para que acompañe a ese Hijo Creador a los reinos del espacio, para ser allí su compañera, primero en la organización física, y luego en la creación y el ministerio hacia las criaturas del universo recién proyectado.

\par
%\textsuperscript{(374.2)}
\textsuperscript{34:0.2} Un Espíritu Creativo reacciona ante las realidades físicas y ante las realidades espirituales; y lo mismo le sucede a un Hijo Creador; y así están coordinados y asociados en la administración de un universo local del tiempo y del espacio.

\par
%\textsuperscript{(374.3)}
\textsuperscript{34:0.3} Estas Hijas Espirituales son de la esencia del Espíritu Infinito, pero no pueden ejercer simultáneamente sus funciones en el trabajo de la creación física y en el del ministerio espiritual. En la creación física, el Hijo del Universo proporciona el modelo, mientras que el Espíritu del Universo inicia la materialización de las realidades físicas. El Hijo trabaja en los proyectos del poder, pero el Espíritu transforma estas creaciones energéticas en sustancias físicas. Aunque es un poco difícil describir esta presencia universal inicial del Espíritu Infinito como una persona, sin embargo, para el Hijo Creador, el Espíritu asociado es personal y siempre ha actuado como un individuo distinto.

\section*{1. La personalización del Espíritu Creativo}
\par
%\textsuperscript{(374.4)}
\textsuperscript{34:1.1} Cuando la organización física de un enjambre estelar y planetario ha terminado y los centros del poder superuniversal han establecido los circuitos de la energía, después de este trabajo preliminar de creación por parte de los agentes del Espíritu Infinito que trabajan a través de su focalización creativa en el universo local, y bajo su dirección, el Hijo Miguel emite la proclamación de que la vida está a punto de proyectarse en el universo recién organizado. Tras el reconocimiento paradisiaco de esta declaración de intención, una reacción de aprobación tiene lugar en la Trinidad del Paraíso, que es seguida por la desaparición, en el resplandor espiritual de las Deidades, del Espíritu Maestro en cuyo superuniverso se está organizando esta nueva creación. Mientras tanto, los otros Espíritus Maestros se acercan a este alojamiento central de las Deidades del Paraíso y, posteriormente, cuando el Espíritu Maestro abrazado por la Deidad aparece y es reconocido por sus compañeros, se produce lo que se conoce como una <<erupción primaria>>. Se trata de un extraordinario relámpago espiritual, un fenómeno que se puede percibir claramente incluso en la lejana sede del superuniverso interesado; simultáneamente con esta manifestación poco comprendida de la Trinidad, un cambio notable tiene lugar en la naturaleza de la presencia y del poder del espíritu creativo del Espíritu Infinito que reside en el universo local interesado. En respuesta a estos fenómenos del Paraíso, y en la presencia misma del Hijo Creador, una nueva representación personal del Espíritu Infinito se personaliza de inmediato. Se trata de la Ministra Divina. El Espíritu Creativo individualizado, la colaboradora del Hijo Creador, se ha convertido en su asociada creativa personal, en el Espíritu Madre del universo local.

\par
%\textsuperscript{(375.1)}
\textsuperscript{34:1.2} De esta nueva segregación personal del Creador Conjunto, y a través de ella, proceden las corrientes establecidas y los circuitos ordenados del poder de espíritu y de la influencia espiritual destinados a impregnar todos los mundos y todos los seres de ese universo local. En realidad, esta presencia nueva y personal no es más que una transformación de la asociada preexistente y menos personal que tenía el Hijo durante su trabajo inicial de organización física del universo.

\par
%\textsuperscript{(375.2)}
\textsuperscript{34:1.3} Éste es el relato en pocas palabras de un drama prodigioso, pero representa casi todo lo que se puede decir sobre estos acontecimientos tan importantes. Son instantáneos, inescrutables e incomprensibles; el secreto de su técnica y de su procedimiento reside en el seno de la Trinidad del Paraíso. Sólo estamos seguros de una cosa: la presencia del Espíritu en el universo local durante la época de la creación o de la organización puramente física estaba incompletamente diferenciada del espíritu del Espíritu Infinito del Paraíso; pero, después de que el Espíritu Maestro supervisor reaparece del abrazo secreto de los Dioses, y después del destello de energía espiritual, la manifestación del Espíritu Infinito en el universo local se transforma repentinamente y por completo en la apariencia personal del Espíritu Maestro que estaba en unión transmutante con el Espíritu Infinito. El Espíritu Madre del universo local adquiere así una naturaleza personal impregnada de la del Espíritu Maestro del superuniverso que posee esa jurisdicción astronómica.

\par
%\textsuperscript{(375.3)}
\textsuperscript{34:1.4} Esta presencia personalizada del Espíritu Infinito, el Espíritu Madre Creativo del universo local, se conoce en Satania como la Ministra Divina. A todos los fines prácticos y para todos los propósitos espirituales, esta manifestación de la Deidad es un individuo divino, una persona espiritual. Y así la reconoce y la considera el Hijo Creador. A través de esta localización y personalización de la Fuente-Centro Tercera en nuestro universo local es como el Espíritu podía someterse posteriormente de una manera tan completa al Hijo Creador como para que pudiera decirse en verdad de este Hijo que <<Todos los poderes en el cielo y en la Tierra le han sido confiados>>\footnote{\textit{Todo poder en el cielo y en la Tierra}: Mt 28:18.}.

\section*{2. La naturaleza de la Ministra Divina}
\par
%\textsuperscript{(375.4)}
\textsuperscript{34:2.1} Después de experimentar una notable metamorfosis en su personalidad durante la época de la creación de la vida, la Ministra Divina actúa a continuación como una persona y coopera de una manera muy personal con el Hijo Creador en la planificación y la dirección de los extensos asuntos de su creación local. Para muchos tipos de seres del universo, incluso esta representación del Espíritu Infinito puede no parecer totalmente personal durante las eras anteriores a la donación final de Miguel; pero después de la elevación del Hijo Creador a la autoridad soberana de un Hijo Maestro, las cualidades personales del Espíritu Madre Creativo se acrecientan de tal manera que es reconocida personalmente por todos los individuos que contactan con ella.

\par
%\textsuperscript{(375.5)}
\textsuperscript{34:2.2} Desde su más temprana asociación con el Hijo Creador, el Espíritu del Universo posee todos los atributos del Espíritu Infinito relacionados con el control físico, incluyendo el pleno don de la antigravedad. Después de alcanzar el estado personal, el Espíritu del Universo ejerce en el universo local un control de la gravedad mental tan pleno y tan completo como lo ejercería el Espíritu Infinito si estuviera personalmente presente.

\par
%\textsuperscript{(375.6)}
\textsuperscript{34:2.3} En cada universo local, la Ministra Divina actúa de acuerdo con la naturaleza y las características inherentes del Espíritu Infinito, tal como éstas se encuentran personificadas en uno de los Siete Espíritus Maestros del Paraíso. Aunque existe una uniformidad básica de carácter en todos los Espíritus de los Universos, hay también una diversidad de funciones determinada por su origen, en el cual ha estado implicado uno de los Siete Espíritus Maestros. Esta diferencia de origen explica las diversas técnicas que emplean los Espíritus Madres de los universos locales para ejercer su actividad en los diferentes superuniversos. Pero en todos sus atributos espirituales esenciales, estos Espíritus son idénticos, igualmente espirituales y totalmente divinos, sin tener en cuenta su diferenciación superuniversal.

\par
%\textsuperscript{(376.1)}
\textsuperscript{34:2.4} El Espíritu Creativo comparte con el Hijo Creador la responsabilidad de engendrar a las criaturas de los mundos, y nunca le falla al Hijo en todos sus esfuerzos por sostener y conservar estas creaciones. La vida es proporcionada y mantenida por mediación del Espíritu Creativo. <<Envías a tu Espíritu, y son creados. Renuevas la faz de la Tierra>>\footnote{\textit{Envías tu Espíritu}: Sal 104:30.}.

\par
%\textsuperscript{(376.2)}
\textsuperscript{34:2.5} En la creación de un universo de criaturas inteligentes, el Espíritu Madre Creativo ejerce primero su actividad en la esfera de la perfección universal, colaborando con el Hijo para engendrar a la Radiante Estrella Matutina\footnote{\textit{Radiante Estrella Matutina}: Ap 22:16.}. Posteriormente, la descendencia del Espíritu se acerca cada vez más a la orden de los seres creados en los planetas, al igual que los Hijos se escalonan gradualmente desde los Melquisedeks hasta los Hijos Materiales que se ponen realmente en contacto con los mortales de los reinos. En la evolución posterior de las criaturas mortales, los Hijos Portadores de Vida proporcionan el cuerpo físico, fabricado con el material organizado existente del reino, mientras que el Espíritu del Universo aporta <<el soplo de vida>>\footnote{\textit{El soplo de vida}: Gn 2:7.}.

\par
%\textsuperscript{(376.3)}
\textsuperscript{34:2.6} Aunque el séptimo segmento del gran universo pueda ser lento en muchos aspectos de su desarrollo, aquellos que estudian cuidadosamente nuestros problemas esperan la evolución de una creación extraordinariamente bien equilibrada en las eras por venir. Predecimos este alto grado de simetría en Orvonton porque el Espíritu que preside este superuniverso es el jefe de los Espíritus Maestros que están en las alturas, y es una inteligencia espiritual que personifica la unión equilibrada y la perfecta coordinación de las características y del carácter de las tres Deidades eternas. Somos lentos y estamos atrasados en comparación con otros sectores, pero es indudable que nos espera un desarrollo trascendente y una consecución sin precedentes en algún momento de las eras eternas del futuro.

\section*{3. El Hijo y el Espíritu en el tiempo y el espacio}
\par
%\textsuperscript{(376.4)}
\textsuperscript{34:3.1} Ni el Hijo Eterno ni el Espíritu Infinito están limitados o condicionados por el tiempo o el espacio, pero la mayor parte de sus descendientes sí lo están.

\par
%\textsuperscript{(376.5)}
\textsuperscript{34:3.2} El Espíritu Infinito impregna todo el espacio y habita el círculo de la eternidad. Sin embargo, en su contacto personal con los hijos del tiempo, las personalidades del Espíritu Infinito deben contar a menudo con los elementos temporales, aunque no tanto con el espacio. Muchos ministerios de la mente ignoran el espacio, pero sufren un retraso de tiempo al efectuar la coordinación de los diversos niveles de la realidad universal. Un Mensajero Solitario es prácticamente independiente del espacio, salvo que necesita realmente tiempo para viajar de un lugar a otro; y existen entidades similares desconocidas para vosotros.

\par
%\textsuperscript{(376.6)}
\textsuperscript{34:3.3} En sus prerrogativas personales, un Espíritu Creativo es total y completamente independiente del espacio, pero no del tiempo. No existe una presencia personal especializada de ese Espíritu del Universo ni en las sedes de las constelaciones ni en las de los sistemas. Está igualmente presente de manera difusa en todo su universo local y, por lo tanto, está tan literal y tan personalmente presente en un mundo como en cualquier otro.

\par
%\textsuperscript{(376.7)}
\textsuperscript{34:3.4} En su ministerio universal, un Espíritu Creativo sólo está siempre limitado con respecto al elemento tiempo. Un Hijo Creador actúa instantáneamente en todo su universo; pero el Espíritu Creativo debe contar con el tiempo en el ministerio de la mente universal, salvo cuando se vale de manera consciente e intencional de las prerrogativas personales del Hijo del Universo. En las actividades de puro espíritu, el Espíritu Creativo también actúa con independencia del tiempo, al igual que cuando colabora con el misterioso funcionamiento de la reflectividad universal.

\par
%\textsuperscript{(377.1)}
\textsuperscript{34:3.5} Aunque el circuito de la gravedad espiritual del Hijo Eterno funciona con independencia del tiempo y del espacio, todas las actividades de los Hijos Creadores no están exentas de las limitaciones del espacio. Si exceptuamos sus actuaciones en los mundos evolutivos, estos Hijos Migueles parecen ser capaces de trabajar con relativa independencia del tiempo. Un Hijo Creador no sufre el obstáculo del tiempo, pero está condicionado por el espacio; no puede estar personalmente en dos lugares al mismo tiempo. Miguel de Nebadon actúa con independencia del tiempo dentro de su propio universo y, por medio de la reflectividad, actúa de la misma manera en el superuniverso. Se comunica directamente con el Hijo Eterno sin las limitaciones del tiempo.

\par
%\textsuperscript{(377.2)}
\textsuperscript{34:3.6} La Ministra Divina es la ayudante comprensiva del Hijo Creador, permitiéndole vencer y compensar sus limitaciones inherentes con respecto al espacio, pues cuando los dos trabajan en unión administrativa son prácticamente independientes del tiempo \textit{y} del espacio dentro de los confines de su creación local. Por lo tanto, tal como se pueden observar en la práctica en todo un universo local, el Hijo Creador y el Espíritu Creativo ejercen habitualmente su actividad con independencia tanto del tiempo como del espacio, puesto que cada uno de ellos puede siempre disponer de la liberación que el otro disfruta o bien del tiempo o bien del espacio.

\par
%\textsuperscript{(377.3)}
\textsuperscript{34:3.7} Sólo los seres absolutos son independientes del tiempo y del espacio en sentido absoluto. La mayoría de las personas subordinadas al Hijo Eterno y al Espíritu Infinito están sometidas tanto al tiempo como al espacio.

\par
%\textsuperscript{(377.4)}
\textsuperscript{34:3.8} Cuando un Espíritu Creativo se vuelve <<consciente del espacio>>, se está preparando para reconocer como suyo un <<territorio espacial>> circunscrito, un reino en el que estará libre del espacio, en contraste con todo el resto del espacio que la condicionaría. Uno sólo es libre de elegir y de actuar dentro del ámbito de su propia conciencia.

\section*{4. Los circuitos del universo local}
\par
%\textsuperscript{(377.5)}
\textsuperscript{34:4.1} En el universo local de Nebadon hay tres circuitos espirituales distintos:

\par
%\textsuperscript{(377.6)}
\textsuperscript{34:4.2} 1. El espíritu donador del Hijo Creador, el Consolador, el Espíritu de la Verdad\footnote{\textit{Espíritu de la Verdad}: Ez 11:19; 18:31; 36:26-27; Jl 2:28-29; Lc 24:49; Jn 7:39; 14:16-18,23,26; 15:4,26; 16:7-8,13-14; 17:21-23; Hch 1:5,8a; 2:1-4,16-18; 2:33; 2 Co 13:5; Gl 2:20; 4:6; Ef 1:13; 4:30; 1 Jn 4:12-15.}.

\par
%\textsuperscript{(377.7)}
\textsuperscript{34:4.3} 2. El circuito espiritual de la Ministra Divina, el Espíritu Santo\footnote{\textit{Espíritu Santo}: Gn 1:2; Ex 31:3; 35:31; Job 33:4; Sal 51:10-11; 139:7; Pr 1:23; Is 44:3; 59:21; 61:1; 63:10-11; Lc 4:1; 11:13; Jn 1:33; 3:5; 2 Ti 1:14.}.

\par
%\textsuperscript{(377.8)}
\textsuperscript{34:4.4} 3. El circuito del ministerio de la inteligencia, que incluye las actividades más o menos unificadas, pero que funcionan de manera diversa, de los siete espíritus ayudantes de la mente.

\par
%\textsuperscript{(377.9)}
\textsuperscript{34:4.5} Los Hijos Creadores están dotados de un espíritu que tiene una presencia universal análoga de muchas maneras a la de los Siete Espíritus Maestros del Paraíso. Se trata del Espíritu de la Verdad que un Hijo donador derrama sobre un mundo después de recibir el título espiritual sobre esa esfera. Este Consolador donado es la fuerza espiritual que siempre atrae\footnote{\textit{Atracción espiritual}: Jer 31:3; Jn 6:44; 12:32.} a todos los buscadores de la verdad hacia Aquel que personifica la verdad en el universo local. Este espíritu es un don inherente del Hijo Creador, y emerge de su naturaleza divina como los circuitos maestros del gran universo proceden de las presencias de personalidad de las Deidades del Paraíso.

\par
%\textsuperscript{(377.10)}
\textsuperscript{34:4.6} El Hijo Creador puede ir y venir; su presencia personal puede estar en el universo local o en otra parte; aún así, el Espíritu de la Verdad funciona sin perturbaciones puesto que esta presencia divina, aunque procede de la personalidad del Hijo Creador, está centrada funcionalmente en la persona de la Ministra Divina.

\par
%\textsuperscript{(378.1)}
\textsuperscript{34:4.7} Sin embargo, el Espíritu Madre del Universo no deja nunca el mundo sede del universo local. El espíritu del Hijo Creador puede funcionar, y de hecho lo hace, independientemente de la presencia personal del Hijo, pero no sucede lo mismo con el espíritu personal de ella. El Espíritu Santo de la Ministra Divina dejaría de funcionar si su presencia personal fuera retirada de Salvington. Su presencia espiritual parece estar fijada en el mundo sede del universo, y este mismo hecho es el que permite que el espíritu del Hijo Creador funcione con independencia del paradero del Hijo. El Espíritu Madre del Universo actúa como foco y centro universal del Espíritu de la Verdad así como del de su propia influencia personal, el Espíritu Santo.

\par
%\textsuperscript{(378.2)}
\textsuperscript{34:4.8} Tanto el Hijo-Padre Creador como el Espíritu Madre Creativo contribuyen de diversas maneras a la dotación mental de los hijos de su universo local. Pero el Espíritu Creativo no confiere la mente hasta que ella misma no es dotada de prerrogativas personales.

\par
%\textsuperscript{(378.3)}
\textsuperscript{34:4.9} Las órdenes superevolutivas de personalidad de un universo local están dotadas del modelo mental superuniversal adaptado a ese universo local. Las órdenes humanas y subhumanas de vida evolutiva están dotadas de los tipos de espíritus ayudantes del ministerio mental.

\par
%\textsuperscript{(378.4)}
\textsuperscript{34:4.10} Los siete espíritus ayudantes de la mente\footnote{\textit{Espíritus ayudantes de la mente}: Is 11:2-3.} son la creación de la Ministra Divina de un universo local. Estos espíritus de la mente tienen caracteres similares pero poderes diferentes, y todos comparten de la misma manera la naturaleza del Espíritu del Universo, aunque difícilmente son considerados como personalidades, salvo por su Madre Creadora. Los siete ayudantes han recibido los nombres siguientes: el espíritu de \textit{sabiduría}, el espíritu de \textit{adoración}, el espíritu de \textit{consejo}, el espíritu de \textit{conocimiento}, el espíritu de \textit{valentía}, el espíritu de \textit{comprensión} y el espíritu de \textit{intuición} ---de percepción rápida.

\par
%\textsuperscript{(378.5)}
\textsuperscript{34:4.11} Éstos son los <<siete espíritus de Dios>>, <<como lámparas encendidas delante del trono>>\footnote{\textit{Siete espíritus de Dios, como lámparas}: Ap 4:5.} que el profeta vio en los símbolos de su visión. Pero no vio los asientos de los veinticuatro centinelas\footnote{\textit{24 centinelas o ancianos}: Ap 4:4,10.} alrededor de estos siete espíritus ayudantes de la mente. Este relato representa la confusión de dos presentaciones, una referente a la sede del universo y la otra a la capital del sistema. Los asientos de los veinticuatro ancianos están en Jerusem, la sede de vuestro sistema local de mundos habitados.

\par
%\textsuperscript{(378.6)}
\textsuperscript{34:4.12} Pero es de Salvington de quien Juan escribió: <<Y del trono salían relámpagos, truenos y voces>>\footnote{\textit{Relámpagos saliendo del trono}: Ap 4:5-6a.} ---las transmisiones del universo hacia los sistemas locales. También contempló a las criaturas del universo local encargadas del control direccional, las brújulas vivientes del mundo sede. Las cuatro criaturas controladoras de Salvington mantienen este control direccional en Nebadon, actúan sobre las corrientes universales y reciben la hábil ayuda del espíritu de la mente que funciona primero, el ayudante de la intuición, el espíritu de la <<comprensión rápida>>\footnote{\textit{Comprensión rápida}: Is 11:3.}. Pero la descripción de estas cuatro criaturas ---llamadas bestias---\footnote{\textit{Cuatro ``bestias''}: Ap 4:6b-9; 5:8,14.} ha sido lamentablemente desfigurada. Tienen una belleza incomparable y una forma exquisita.

\par
%\textsuperscript{(378.7)}
\textsuperscript{34:4.13} Los cuatro puntos de la brújula son universales e inherentes a la vida de Nebadon. Todas las criaturas vivientes poseen unidades corporales que son sensibles y responden a estas corrientes direccionales. Estas facultades de las criaturas se reproducen en todo el universo hasta llegar a los planetas individuales y, conjuntamente con las fuerzas magnéticas de los mundos, activan de tal manera la multitud de cuerpos microscópicos del organismo animal que estas células direccionales indican siempre el norte y el sur. El sentido de la orientación está así fijado para siempre en los seres vivos del universo. La humanidad no carece por completo de la posesión consciente de este sentido. Estos cuerpos fueron observados por primera vez en Urantia hacia la época de esta narración.

\section*{5. El ministerio del Espíritu}
\par
%\textsuperscript{(379.1)}
\textsuperscript{34:5.1} La Ministra Divina coopera con el Hijo Creador para formular la vida y crear nuevas órdenes de seres hasta la época de su séptima donación y, posteriormente, después de su elevación a la plena soberanía del universo, continúa colaborando con el Hijo y con el espíritu donado por el Hijo en el trabajo ulterior del ministerio mundial y de la progresión planetaria.

\par
%\textsuperscript{(379.2)}
\textsuperscript{34:5.2} El Espíritu comienza el trabajo de la progresión evolutiva en los mundos habitados empezando por el material inanimado del reino, dotando en primer lugar a la vida vegetal, luego a los organismos animales y más tarde a las primeras órdenes de existencia humana; y cada concesión sucesiva contribuye al desarrollo adicional del potencial evolutivo de la vida planetaria, desde las etapas iniciales y primitivas hasta la aparición de las criaturas volitivas. El Espíritu efectúa esta labor en gran parte a través de los siete ayudantes, los espíritus de la promesa, el espíritu-mente unificador y coordinador de los planetas evolutivos, que conduce siempre y de manera unida a las razas de los hombres hacia las ideas superiores y los ideales espirituales.

\par
%\textsuperscript{(379.3)}
\textsuperscript{34:5.3} El hombre mortal experimenta por primera vez el ministerio del Espíritu en conjunción con la mente cuando la mente puramente animal de las criaturas evolutivas desarrolla la capacidad de recibir a los ayudantes de la adoración y de la sabiduría. Este ministerio del sexto y del séptimo ayudantes indica que la evolución de la mente ha cruzado el umbral del ministerio espiritual. Estas mentes capaces de obrar con adoración y sabiduría son incluidas de inmediato en los circuitos espirituales de la Ministra Divina.

\par
%\textsuperscript{(379.4)}
\textsuperscript{34:5.4} Cuando la mente está dotada así del ministerio del Espíritu Santo, posee la capacidad de elegir (consciente o inconscientemente) la presencia espiritual del Padre Universal ---el Ajustador del Pensamiento. Pero todas las mentes normales no están automáticamente preparadas para recibir a los Ajustadores del Pensamiento hasta que el Hijo donador no ha liberado el Espíritu de la Verdad para que dispense su ministerio planetario a todos los mortales. El Espíritu de la Verdad trabaja al unísono con la presencia del espíritu de la Ministra Divina. Esta unión espiritual doble se cierne sobre los mundos, tratando de enseñar la verdad y de iluminar espiritualmente la mente de los hombres, de inspirar el alma de las criaturas de las razas ascendentes, y de conducir siempre a los seres que viven en los planetas evolutivos hacia la meta paradisiaca de su destino divino.

\par
%\textsuperscript{(379.5)}
\textsuperscript{34:5.5} Aunque el Espíritu de la Verdad\footnote{\textit{Espíritu de la Verdad}: Ez 11:19; 18:31; 36:26-27; Jl 2:28-29; Lc 24:49; Jn 7:39; 14:16-18,23,26; 15:4,26; 16:7-8,13-14; 17:21-23; Hch 1:5,8a; 2:1-4,16-18; 2:33; 2 Co 13:5; Gl 2:20; 4:6; Ef 1:13; 4:30; 1 Jn 4:12-15.} se derrama sobre toda carne, la actividad y el poder de este espíritu del Hijo están casi totalmente limitados por la receptividad personal del hombre a aquello que constituye la suma y la sustancia de la misión del Hijo donador. El Espíritu Santo es en parte independiente de la actitud humana, y está parcialmente condicionado por las decisiones y la cooperación de la voluntad del hombre. No obstante, el ministerio del Espíritu Santo se vuelve cada vez más eficaz para santificar y espiritualizar la vida interior de aquellos mortales que \textit{obedecen} de la manera más completa las directrices divinas.

\par
%\textsuperscript{(379.6)}
\textsuperscript{34:5.6} Vosotros no poseéis personalmente, como individuos, una parte o entidad aislada del espíritu del Hijo-Padre Creador o del Espíritu Madre Creativo; estos ministerios no se ponen en contacto con los centros pensantes de la mente del individuo, ni habitan en ellos, como lo hacen los Monitores de Misterio. Los Ajustadores del Pensamiento son individualizaciones concretas de la realidad prepersonal del Padre Universal, que residen efectivamente en la mente mortal como parte integrante de esa mente, y siempre trabajan en perfecta armonía con los espíritus combinados del Hijo Creador y del Espíritu Creativo.

\par
%\textsuperscript{(380.1)}
\textsuperscript{34:5.7} La presencia del Espíritu Santo de la Hija Universal del Espíritu Infinito, del Espíritu de la Verdad del Hijo Universal del Hijo Eterno, y del espíritu-Ajustador del Padre Paradisiaco en, o con, un mortal evolutivo indica una simetría de dotación y de ministerio espirituales y capacita a ese mortal para comprender conscientemente el hecho basado en la fe de su filiación con Dios.

\section*{6. El espíritu en el hombre}
\par
%\textsuperscript{(380.2)}
\textsuperscript{34:6.1} Con la evolución progresiva de un planeta habitado y la espiritualización ulterior de sus habitantes, esas personalidades maduras pueden recibir influencias espirituales adicionales. A medida que los mortales progresan en control mental y en percepción espiritual, el funcionamiento de estos múltiples ministerios espirituales se vuelve cada vez más coordinado; se mezclan de manera creciente con el superministerio de la Trinidad del Paraíso.

\par
%\textsuperscript{(380.3)}
\textsuperscript{34:6.2} Aunque la manifestación de la Divinidad puede ser múltiple, en la experiencia humana la Deidad es única, siempre \textit{una sola}. El ministerio espiritual tampoco es múltiple en la experiencia humana. Sin tener en cuenta sus múltiples orígenes, todas las influencias espirituales funcionan como una sola. En verdad son una sola, pues se trata del ministerio espiritual de Dios Séptuple en y para las criaturas del gran universo; y a medida que crece la apreciación y la receptividad de las criaturas a este ministerio unificador del espíritu, en su experiencia se convierte en el ministerio de Dios Supremo.

\par
%\textsuperscript{(380.4)}
\textsuperscript{34:6.3} Mediante una larga serie de pasos, el Espíritu divino desciende desde las alturas de la gloria eterna para encontrarse con vosotros, tal como sois y allí donde estáis, para después, en la asociación de la fe, abrazar con amor el alma de origen mortal y emprender el regreso cierto y seguro sobre sus pasos condescendientes, sin detenerse nunca hasta que el alma evolutiva sea elevada con seguridad hasta las alturas mismas de felicidad de las que el Espíritu divino salió originalmente para llevar a cabo esta misión de misericordia y de ministerio.

\par
%\textsuperscript{(380.5)}
\textsuperscript{34:6.4} Las fuerzas espirituales buscan y alcanzan infaliblemente sus propios niveles originales. Como han salido del Eterno, regresarán a él con toda seguridad\footnote{\textit{Regreso del espíritu}: Ec 3:21; 12:7.}, llevando consigo a todos los hijos del tiempo y del espacio que han adoptado la guía y las enseñanzas del Ajustador interior, a aquellos que realmente han <<nacido del Espíritu>>\footnote{\textit{Nacido del Espíritu}: Jn 3:3-7.}, los hijos de Dios por la fe.

\par
%\textsuperscript{(380.6)}
\textsuperscript{34:6.5} El Espíritu divino es la fuente de un ministerio y de un estímulo continuos para los hijos de los hombres. Vuestro poder y vuestros logros serán <<conformes con su misericordia, a través de la renovación del Espíritu>>\footnote{\textit{Conformes con su misericordia}: Tit 3:5.}. La vida espiritual, al igual que la energía física, se consume. El esfuerzo espiritual conduce a un agotamiento espiritual relativo. Toda la experiencia ascendente es real así como espiritual; por eso está escrito en verdad: <<El Espíritu es el que vivifica>>\footnote{\textit{El Espíritu es el que vivifica}: Jn 6:63; Ro 8:11.}. <<El Espíritu da la vida>>\footnote{\textit{El Espíritu da la vida}: Job 33:4; Jn 6:63; 2 Co 3:6.}.

\par
%\textsuperscript{(380.7)}
\textsuperscript{34:6.6} La teoría muerta, incluso de las doctrinas religiosas más elevadas, no tiene poder para transformar el carácter humano o para controlar el comportamiento de los mortales. Lo que el mundo de hoy necesita es la verdad que vuestro instructor de antaño declaró: <<No solamente en palabras, sino también en poder y en el Espíritu Santo>>\footnote{\textit{No sólo en palabras, sino en poder}: Zac 4:6; 1 Ts 1:5.}. La semilla de la verdad teórica está muerta y los conceptos morales más elevados no tienen efecto a menos que, y hasta que, el Espíritu divino sople sobre las formas de la verdad y vivifique las fórmulas de la rectitud.

\par
%\textsuperscript{(381.1)}
\textsuperscript{34:6.7} Aquellos que han recibido y reconocido la presencia interior de Dios han nacido del Espíritu. <<Sois el templo de Dios, y el espíritu de Dios habita en vosotros>>\footnote{\textit{Sois el templo de Dios, y os habita el espíritu}: Lc 17:21; Ro 8:9-11; 1 Co 3:16-17; 1 Co 6:19; 2 Co 6:16; 2 Ti 1:14; 1 Jn 4:12-15; Ap 21:3. \textit{Espíritu de Dios en el hombre}: Job 32:8,18; Is 63:10-11; Ez 37:14; Mt 10:20; Lc 17::21; Jn 17:21-23; Ro 8:9-11; 1 Co 3:16-17; 1 Co 6:19; 2 Co 6:16; Gl 2:20; 1 Jn 3:24; 1 Jn 4:12-15; Ap 21:3.}. No es suficiente con que este espíritu se haya derramado sobre vosotros; el Espíritu divino debe dominar y controlar cada fase de la experiencia humana.

\par
%\textsuperscript{(381.2)}
\textsuperscript{34:6.8} La presencia del Espíritu divino, el agua de la vida, es la que impide la sed devoradora del descontento de los mortales y el hambre indescriptible de la mente humana no espiritualizada. Los seres motivados por el espíritu <<nunca tienen sed, pues este agua espiritual será en ellos una fuente de satisfacción que mana hasta la vida eterna>>\footnote{\textit{Nunca tener sed, agua espiritual}: Is 55:1; Jn 4:10,13-14; 7:37-38; Ap 21:6; 22:17.}. Estas almas divinamente regadas son casi independientes del entorno material en lo que se refiere a las alegrías de la vida y a las satisfacciones de la existencia terrenal. Están iluminadas y refrescadas espiritualmente, fortalecidas y dotadas moralmente.

\par
%\textsuperscript{(381.3)}
\textsuperscript{34:6.9} En todo mortal existe una naturaleza doble: la herencia de las tendencias animales y el impulso elevado del don espiritual. Durante la corta vida que vivís en Urantia, estos dos impulsos opuestos y diferentes rara vez se pueden conciliar plenamente; difícilmente se pueden armonizar y unificar; pero durante toda vuestra vida, el Espíritu combinado aporta siempre su ministerio para ayudaros a someter la carne cada vez más a la guía del Espíritu. Aunque tenéis que vivir vuestra vida material hasta el fin, aunque no podéis escapar del cuerpo ni de sus necesidades, sin embargo, en lo que se refiere a vuestros propósitos e ideales, tenéis la facultad de someter cada vez más la naturaleza animal al dominio del Espíritu. Existe en verdad dentro de vosotros una conspiración de fuerzas espirituales, una confederación de poderes divinos, cuyo propósito exclusivo consiste en liberaros definitivamente de la esclavitud material y de los obstáculos finitos.

\par
%\textsuperscript{(381.4)}
\textsuperscript{34:6.10} El propósito de todo este ministerio es <<que podáis sentiros poderosamente fortalecidos por medio de Su espíritu en el hombre interior>>\footnote{\textit{Fortalecidos por el poder}: Ef 3:16.}. Y todo esto no representa más que las etapas preliminares para el logro final de la perfección de la fe y del servicio, esa experiencia en la que estaréis <<llenos de toda la plenitud de Dios>>\footnote{\textit{Llenos de la plenitud de Dios}: Ef 1:23; 3:19.}, <<porque todos aquellos que son conducidos por el espíritu de Dios, son hijos de Dios>>\footnote{\textit{Todos conducidos por el espíritu son hijos de Dios}: Ro 8:14. \textit{Hijos de Dios}: 1 Cr 22:10; Sal 2:7; Is 56:5; Mt 5:9,16,45; Lc 20:36; Jn 1:12-13; 11:52; Hch 17:28-29; Ro 8:14-17,19,21; 9:26; 2 Co 6:18; Gl 3:26; 4:5-7; Ef 1:5; Flp 2:15; Heb 12:5:8; 1 Jn 3:1-2,10; 5:2; Ap 21:7; 2 Sam 7:14.}.

\par
%\textsuperscript{(381.5)}
\textsuperscript{34:6.11} El Espíritu nunca \textit{fuerza}, sólo guía. Si sois un estudiante de buena voluntad, si queréis alcanzar los niveles espirituales y llegar a las alturas divinas, si deseáis sinceramente alcanzar la meta eterna, entonces el Espíritu divino os guiará con suavidad y amor por el camino de la filiación y del progreso espiritual. Cada paso que deis deberéis efectuarlo mediante una cooperación voluntaria, inteligente y alegre. La dominación del Espíritu nunca está manchada de coerción ni comprometida por la coacción.

\par
%\textsuperscript{(381.6)}
\textsuperscript{34:6.12} Y cuando una vida guiada así por el espíritu es aceptada de manera libre e inteligente, dentro de la mente humana se desarrolla gradualmente la conciencia positiva de un contacto divino y la seguridad de comulgar con el espíritu; tarde o temprano <<el Espíritu atestigua con vuestro espíritu (el Ajustador) que sois un hijo de Dios>>. Vuestro propio Ajustador del Pensamiento ya os ha informado de vuestro parentesco con Dios, por eso las escrituras declaran que el Espíritu atestigua <<\textit{con} vuestro espíritu>>\footnote{\textit{El Espíritu atestigua}: Ro 8:16.}, no \textit{a} vuestro espíritu.

\par
%\textsuperscript{(381.7)}
\textsuperscript{34:6.13} La conciencia de la dominación de una vida humana por el espíritu pronto es acompañada por una manifestación creciente de las características del Espíritu en las reacciones vitales de ese mortal conducido por el espíritu, <<porque los frutos del espíritu son el amor, la alegría, la paz, la paciencia, la amabilidad, la bondad, la fe, la mansedumbre y la templanza>>\footnote{\textit{Frutos del espíritu}: Gl 5:22-23; Ef 5:9.}. Aunque estos mortales guiados por el espíritu y divinamente iluminados caminan todavía por los humildes senderos del trabajo agotador y cumplen con fidelidad humana los deberes de sus tareas terrenales, ya han empezado a discernir las luces de la vida eterna que brillan en las orillas lejanas de otro mundo; ya han empezado a comprender la realidad de esta verdad inspiradora y reconfortante: <<El reino de Dios no es comida ni bebida, sino rectitud, paz y alegría en el Espíritu Santo>>\footnote{\textit{El reino de Dios no es comida...}: Ro 14:17. \textit{El reino de Dios}: Mt 6:33; 12:28; 19:24; 21:31,43; Mc 1:14-15; 4:11,26,30; 9:1,47; 10:14-15,23-25; 12:34; 14:25; 15:43; Lc 4:43; 6:20; 7:28; 8:1,10; 9:2,11,27; 9:60,62; 10:9-11; 11:20; 12:31-32; 13:18,20,28,29; 14:15; 16:16; 17:20-21; 18:16-17,24-25; 19:11; 21:31; 22:16-18; 23:51; Jn 3:3,5; Ro 14:17; 1 Co 4:20; 6:9-10. \textit{Reino del cielo}: Mt 3:2; 4:17; 5:3,10,19-20; 7:21; 8:11; 10:7; 11:11-12; 13:11,24,31-52; 16:19; 18:1-4,23; 19:14,23; 20:1; 22:2; 23:13; 25:1,14. \textit{Reino}: Mt 4:23; 9:35; 24:14.}. A lo largo de cada prueba y en presencia de cada dificultad, las almas nacidas del espíritu están sostenidas por esa esperanza que trasciende todo temor, porque el amor de Dios se derrama en todos los corazones a través de la presencia del Espíritu divino.

\section*{7. El espíritu y la carne}
\par
%\textsuperscript{(382.1)}
\textsuperscript{34:7.1} La carne, la naturaleza inherente derivada de las razas de origen animal, no produce por naturaleza los frutos del Espíritu divino\footnote{\textit{Frutos del Espíritu}: Gl 5:22-23; Ef 5:9.}. Cuando la naturaleza mortal ha sido elevada mediante la adición de la naturaleza de los Hijos Materiales de Dios, como las razas de Urantia mejoraron en cierta medida gracias a la donación de Adán, entonces el camino está mejor preparado para que el Espíritu de la Verdad\footnote{\textit{Espíritu de la Verdad}: Ez 11:19; 18:31; 36:26-27; Jl 2:28-29; Lc 24:49; Jn 7:39; 14:16-18,23,26; 15:4,26; 16:7-8,13-14; 17:21-23; Hch 1:5,8a; 2:1-4,16-18; 2:33; 2 Co 13:5; Gl 2:20; 4:6; Ef 1:13; 4:30; 1 Jn 4:12-15.} coopere con el Ajustador interior a fin de producir la hermosa cosecha de los frutos del espíritu sobre el carácter. Si no rechazáis este espíritu, y aunque necesitéis la eternidad para llevar a cabo la misión, <<él os guiará hacia toda verdad>>\footnote{\textit{El Espíritu os guiará hacia la verdad}: Jn 16:13.}.

\par
%\textsuperscript{(382.2)}
\textsuperscript{34:7.2} Los mortales evolutivos que habitan en los mundos normales de progreso espiritual no experimentan los agudos conflictos entre el espíritu y la carne que caracterizan a las razas urantianas de la época actual. Pero incluso en los planetas más ideales, el hombre preadámico debe emplear sus esfuerzos positivos para ascender desde el plano de existencia puramente animal hasta los niveles sucesivos de significados intelectuales crecientes y de valores espirituales superiores.

\par
%\textsuperscript{(382.3)}
\textsuperscript{34:7.3} Los mortales de un mundo normal no experimentan una guerra constante entre su naturaleza física y su naturaleza espiritual. Tienen que enfrentarse a la necesidad de elevarse desde los niveles de existencia animal hasta los planos superiores de la vida espiritual, pero esta ascensión se parece más a un entrenamiento educativo cuando se la compara con los intensos conflictos que sufren los mortales de Urantia en este terreno de las naturalezas material y espiritual divergentes.

\par
%\textsuperscript{(382.4)}
\textsuperscript{34:7.4} Los pueblos de Urantia sufren las consecuencias de una doble privación de ayuda en esta tarea de consecución espiritual planetaria progresiva. La sublevación de Caligastia provocó una confusión mundial y les robó a todas las generaciones posteriores la asistencia moral que les hubiera proporcionado una sociedad bien ordenada. Pero la falta de Adán fue aun más desastrosa, ya que privó a las razas de ese tipo superior de naturaleza física que habría estado más de acuerdo con las aspiraciones espirituales.

\par
%\textsuperscript{(382.5)}
\textsuperscript{34:7.5} Los mortales de Urantia están obligados a sufrir esta lucha acentuada entre el espíritu y la carne porque sus lejanos antepasados no fueron más plenamente adamizados por la donación edénica. El plan divino preveía que las razas mortales de Urantia tuvieran una naturaleza física más sensible al espíritu de manera natural.

\par
%\textsuperscript{(382.6)}
\textsuperscript{34:7.6} A pesar de este doble desastre para la naturaleza del hombre y su entorno, los mortales de hoy en día experimentarían menos esta guerra aparente entre la carne y el espíritu si quisieran entrar en el reino del espíritu, donde los hijos de Dios por la fe disfrutan de una liberación relativa de la esclavitud de la carne mediante el servicio iluminado y liberador de la devoción sincera a hacer la voluntad del Padre que está en los cielos. Jesús mostró a la humanidad la nueva manera de vivir de los mortales mediante la cual los seres humanos pueden eludir en gran parte las terribles consecuencias de la rebelión de Caligastia y compensar muy eficazmente las privaciones resultantes de la falta de Adán. <<El espíritu de la vida de Cristo Jesús nos ha liberado de la ley de la vida animal y de las tentaciones del mal y del pecado>>\footnote{\textit{El espíritu de la vida de Jesús}: Ro 8:2.}. <<Ésta es la victoria que triunfa sobre la carne, vuestra fe misma>>\footnote{\textit{La fe triunfa sobre la carne}: 1 Jn 5:4.}.

\par
%\textsuperscript{(383.1)}
\textsuperscript{34:7.7} Los hombres y las mujeres que conocen a Dios y que han nacido del Espíritu ya no experimentan más conflictos con su naturaleza mortal que los habitantes de los mundos más normales, de los planetas que nunca han sido manchados por el pecado ni afectados por la rebelión. Los hijos de la fe trabajan en unos niveles intelectuales y viven en unos planos espirituales que están muy por encima de los conflictos producidos por unos deseos físicos desenfrenados o anormales. Los vivos deseos normales de los seres animales y los apetitos e impulsos naturales de la naturaleza física no están en conflicto con los logros espirituales incluso más elevados, excepto en la mente de las personas ignorantes, mal instruidas o lamentablemente demasiado escrupulosas.

\par
%\textsuperscript{(383.2)}
\textsuperscript{34:7.8} Después de emprender el camino de la vida eterna, después de aceptar vuestra tarea y de recibir vuestras órdenes para progresar, no temáis los peligros de la falta de memoria de los hombres ni la inconstancia de los mortales, no os inquietéis por el miedo al fracaso o por las confusiones que causan perplejidad, no vaciléis ni pongáis en duda vuestro estado ni vuestra posición, porque en todas las horas sombrías, en todas las encrucijadas de la lucha por el progreso, el Espíritu de la Verdad siempre hablará, diciendo: <<Éste es el camino>>\footnote{\textit{Éste es el camino}: Is 30:21.}.

\par
%\textsuperscript{(383.3)}
\textsuperscript{34:7.9} [Presentado por un Mensajero Poderoso, destinado temporalmente a servir en Urantia.]


\chapter{Documento 35. Los Hijos de Dios de los universos locales}
\par
%\textsuperscript{(384.1)}
\textsuperscript{35:0.1} LOS Hijos de Dios presentados anteriormente han tenido un origen paradisiaco. Son los descendientes de los Gobernantes divinos de los dominios universales. Los Hijos Creadores pertenecen a la primera orden de filiación paradisiaca, y en Nebadon sólo hay uno de ellos, Miguel, el padre y soberano del universo. Los Hijos Avonales o Magistrales pertenecen a la segunda orden de filiación del Paraíso, y Nebadon tiene su contingente al completo ---1.062 miembros. Estos <<Cristos menores>> son tan eficaces y tan todopoderosos en sus donaciones planetarias como el Hijo Creador y Maestro lo fue en Urantia. Como la tercera orden tiene su origen en la Trinidad, no está registrada en un universo local, pero calculo que hay en Nebadon entre quince y veinte mil Hijos Instructores Trinitarios, aparte de los 9.642 asistentes trinitizados por las criaturas que sí están registrados. Estos Daynales del Paraíso no son ni magistrados ni administradores; son supereducadores.

\par
%\textsuperscript{(384.2)}
\textsuperscript{35:0.2} Los tipos de Hijos que vamos a estudiar tienen su origen en el universo local; son los descendientes de un Hijo Creador Paradisiaco en asociación variada con el Espíritu Madre Universal complementario. En estas narraciones mencionaremos las siguientes ordenes de filiación del universo local:

\par
%\textsuperscript{(384.3)}
\textsuperscript{35:0.3} 1. Los Hijos Melquisedeks.

\par
%\textsuperscript{(384.4)}
\textsuperscript{35:0.4} 2. Los Hijos Vorondadeks.

\par
%\textsuperscript{(384.5)}
\textsuperscript{35:0.5} 3. Los Hijos Lanonandeks.

\par
%\textsuperscript{(384.6)}
\textsuperscript{35:0.6} 4. Los Hijos Portadores de Vida.

\par
%\textsuperscript{(384.7)}
\textsuperscript{35:0.7} La Deidad trina del Paraíso actúa para crear tres órdenes de filiación: los Migueles, los Avonales y los Daynales. La Deidad doble del universo local, el Hijo y el Espíritu, también actúa para crear tres órdenes elevadas de Hijos: los Melquisedeks, los Vorondadeks y los Lanonandeks; y después de conseguir esta expresión triple, colaboran con el siguiente nivel de Dios Séptuple para engendrar la orden polifacética de los Portadores de Vida. Estos seres están clasificados con los Hijos descendentes de Dios, pero constituyen una forma de vida única y original en el universo. Todo el documento siguiente lo dedicaremos a su estudio.

\section*{1. El Padre Melquisedek}
\par
%\textsuperscript{(384.8)}
\textsuperscript{35:1.1} Después de traer a la existencia a sus ayudantes personales, tales como la Radiante Estrella Matutina y otras personalidades administrativas, de acuerdo con el propósito divino y los planes creativos de un universo dado, una nueva forma de unión creativa se produce entre el Hijo Creador y el Espíritu Creativo, la Hija del Espíritu Infinito en el universo local. La personalidad resultante de esta asociación creativa es el Melquisedek original ---el Padre Melquisedek--- ese ser único que colabora posteriormente con el Hijo Creador y el Espíritu Creativo para traer a la existencia a todo el grupo que lleva este nombre.

\par
%\textsuperscript{(385.1)}
\textsuperscript{35:1.2} El Padre Melquisedek actúa en el universo de Nebadon como el primer asociado ejecutivo de la Radiante Estrella Matutina. Gabriel se ocupa más de la política del universo, y Melquisedek de los procedimientos prácticos. Gabriel preside los tribunales y consejos regularmente constituidos de Nebadon, y Melquisedek las comisiones y los cuerpos consultivos especiales, extraordinarios y de emergencia. Gabriel y el Padre Melquisedek nunca están fuera de Salvington al mismo tiempo, porque en ausencia de Gabriel, el Padre Melquisedek actúa como jefe ejecutivo de Nebadon.

\par
%\textsuperscript{(385.2)}
\textsuperscript{35:1.3} Todos los Melquisedeks de nuestro universo fueron creados en el transcurso de un solo milenio del tiempo oficial por el Hijo Creador y el Espíritu Creativo en unión con el Padre Melquisedek. Como se trata de una orden de filiación en la que uno de sus propios miembros actuó como creador coordinado, los Melquisedeks son en parte de origen autónomo en su constitución, y en consecuencia son candidatos a llevar a cabo un tipo celestial de gobierno autónomo. Eligen periódicamente a su propio jefe administrativo por un período de siete años del tiempo oficial, y funcionan por lo demás como una orden que se regula ella misma, aunque el Melquisedek original ejerce ciertas prerrogativas inherentes a su calidad de coprogenitor. Este Padre Melquisedek designa de vez en cuando a ciertos individuos de su orden para que actúen como Portadores de Vida especiales en los mundos midsonitos, un tipo de planeta habitado hasta ahora no revelado en Urantia.

\par
%\textsuperscript{(385.3)}
\textsuperscript{35:1.4} Los Melquisedeks no ejercen ampliamente su actividad fuera del universo local, salvo cuando son llamados como testigos para los asuntos que están pendientes ante los tribunales del superuniverso, y cuando son designados como embajadores especiales, como a veces les sucede, para representar a un universo ante otro dentro del mismo superuniverso. El Melquisedek original o primogénito de cada universo puede siempre viajar libremente a los universos vecinos o al Paraíso para misiones relacionadas con los intereses y las obligaciones de su orden.

\section*{2. Los Hijos Melquisedeks}
\par
%\textsuperscript{(385.4)}
\textsuperscript{35:2.1} Los Melquisedeks son la primera orden de Hijos divinos que se acercan lo suficiente a la vida de las criaturas inferiores como para poder actuar directamente en el ministerio de elevar a los mortales, de servir a las razas evolutivas sin necesidad de encarnarse. Estos Hijos se hallan por naturaleza en el punto medio de la gran escala descendente de personalidades, encontrándose por su origen casi exactamente a medio camino entre la Divinidad más elevada y las criaturas dotadas de voluntad más humildes. De este modo se convierten en los intermediarios naturales entre los niveles superiores y divinos de existencia viviente y las formas inferiores, e incluso materiales, de vida de los mundos evolutivos. A las órdenes seráficas, a los ángeles, les encanta trabajar con los Melquisedeks; de hecho, todas las formas de vida inteligente encuentran en estos Hijos a unos amigos comprensivos, unos instructores compasivos y unos consejeros sabios.

\par
%\textsuperscript{(385.5)}
\textsuperscript{35:2.2} Los Melquisedeks son una orden que se gobierna de forma autónoma. En este grupo excepcional encontramos el primer intento de autodeterminación por parte de unos seres del universo local, y observamos el tipo más elevado de un verdadero gobierno autónomo. Estos Hijos organizan su propio mecanismo para administrar su grupo y su planeta nativo, así como las seis esferas asociadas y sus mundos tributarios. Y debemos indicar que nunca han abusado de sus prerrogativas; en todo el superuniverso de Orvonton, estos Hijos Melquisedeks no han traicionado nunca, ni una sola vez, la confianza depositada en ellos. Son la esperanza de todos los grupos del universo que aspiran a la autonomía; son el modelo de la autonomía gubernamental y la enseñan en todas las esferas de Nebadon. Todas las órdenes de seres inteligentes, las superiores que están por encima y las subordinadas que están por debajo, elogian sinceramente el gobierno de los Melquisedeks.

\par
%\textsuperscript{(386.1)}
\textsuperscript{35:2.3} La orden de filiación de los Melquisedeks ocupa la posición, y asume la responsabilidad, del hijo mayor en una gran familia. La mayor parte de su trabajo es regular y un poco rutinario, pero una gran parte del mismo es voluntario y totalmente autoimpuesto. La mayoría de las asambleas especiales que se reúnen de vez en cuando en Salvington se convocan a petición de los Melquisedeks. Estos Hijos patrullan su universo nativo por su propia iniciativa. Mantienen una organización autónoma dedicada al servicio de información universal, y presentan informes periódicos al Hijo Creador que son independientes de toda la información que llega a la sede del universo a través de los agentes regulares relacionados con la administración rutinaria del reino. Son observadores imparciales por naturaleza; gozan de la plena confianza de todas las clases de seres inteligentes.

\par
%\textsuperscript{(386.2)}
\textsuperscript{35:2.4} Los Melquisedeks actúan como tribunales de revisión itinerantes y consultivos de los reinos; estos Hijos del universo van a los mundos en pequeños grupos para servir como comisiones consultivas, tomar declaraciones, recibir sugerencias y actuar como consejeros, ayudando así a serenar las dificultades importantes y a resolver las graves diferencias que surgen de vez en cuando en los asuntos de los dominios evolutivos.

\par
%\textsuperscript{(386.3)}
\textsuperscript{35:2.5} Estos Hijos mayores de un universo son los ayudantes principales de la Radiante Estrella Matutina en la tarea de ejecutar los mandatos del Hijo Creador. Cuando un Melquisedek va a un mundo lejano en nombre de Gabriel, puede desempeñar las funciones, con miras a esa misión particular, en nombre de aquel que le envía, y en ese caso aparecerá en el planeta de destino con la plena autoridad de la Radiante Estrella Matutina. Esto es especialmente cierto en aquellas esferas donde un Hijo más elevado aún no ha aparecido en la similitud de las criaturas del reino.

\par
%\textsuperscript{(386.4)}
\textsuperscript{35:2.6} Cuando un Hijo Creador emprende su carrera de donación en un mundo evolutivo, va solo; pero cuando uno de sus hermanos del Paraíso, un Hijo Avonal, emprende una donación, va acompañado por los Melquisedeks que lo apoyan, doce en total, que contribuyen tan eficazmente al éxito de la misión donadora. También apoyan a los Avonales del Paraíso en sus misiones magistrales en los mundos habitados y, durante estas misiones, los Melquisedeks son visibles para los ojos de los mortales si el Hijo Avonal también se manifiesta de esta manera.

\par
%\textsuperscript{(386.5)}
\textsuperscript{35:2.7} No existe ninguna fase de las necesidades espirituales planetarias a la que no aporten su ministerio. Son los educadores que con tanta frecuencia consiguen que mundos enteros de vida avanzada reconozcan de manera plena y definitiva al Hijo Creador y a su Padre Paradisiaco.

\par
%\textsuperscript{(386.6)}
\textsuperscript{35:2.8} La sabiduría de los Melquisedeks es casi perfecta, pero su juicio no es infalible. Cuando están solos y aislados en sus misiones planetarias, a veces se han equivocado en cuestiones menores, es decir, han elegido hacer ciertas cosas que sus supervisores no han aprobado posteriormente. Este error de juicio incapacita temporalmente a un Melquisedek hasta que va a Salvington y, en el transcurso de una audiencia con el Hijo Creador, recibe la enseñanza que lo purificará eficazmente de la falta de armonía que provocó el desacuerdo con sus compañeros; y luego, después del descanso correccional, se reincorporará al servicio al tercer día. Pero estas inadaptaciones menores en la actividad de los Melquisedeks se han producido raras veces en Nebadon.

\par
%\textsuperscript{(387.1)}
\textsuperscript{35:2.9} Estos Hijos no forman una orden que aumente; su número es fijo, aunque varía en cada universo local. El número de Melquisedeks registrados en su planeta sede de Nebadon supera los diez millones.

\section*{3. Los mundos de los Melquisedeks}
\par
%\textsuperscript{(387.2)}
\textsuperscript{35:3.1} Los Melquisedeks ocupan un mundo propio cerca de Salvington, la sede del universo. Esta esfera, llamada Melquisedek, es el mundo piloto del circuito de setenta esferas primarias de Salvington, cada una de las cuales está rodeada por seis esferas tributarias dedicadas a actividades especializadas. A estas esferas maravillosas ---setenta primarias y 420 tributarias--- se les llama a menudo la Universidad Melquisedek. Los mortales ascendentes de todas las constelaciones de Nebadon pasan por la formación de estos 490 mundos para adquirir el estado residencial en Salvington. Pero la educación de los ascendentes no es más que una fase de las múltiples actividades que tienen lugar en el grupo de esferas arquitectónicas de Salvington.

\par
%\textsuperscript{(387.3)}
\textsuperscript{35:3.2} Las 490 esferas del circuito de Salvington están divididas en diez grupos, y cada uno contiene siete esferas primarias y cuarenta y dos tributarias. Cada uno de estos grupos se encuentra bajo la supervisión general de una de las órdenes principales de la vida universal. El primer grupo, que abarca el mundo piloto y las seis esferas primarias siguientes en la procesión planetaria circundante, se encuentra bajo la supervisión de los Melquisedeks. Estos mundos Melquisedeks son los siguientes:

\par
%\textsuperscript{(387.4)}
\textsuperscript{35:3.3} 1. El mundo piloto ---el mundo nativo de los Hijos Melquisedeks.

\par
%\textsuperscript{(387.5)}
\textsuperscript{35:3.4} 2. El mundo de las escuelas de la vida física y de los laboratorios de las energías vivientes.

\par
%\textsuperscript{(387.6)}
\textsuperscript{35:3.5} 3. El mundo de la vida morontial.

\par
%\textsuperscript{(387.7)}
\textsuperscript{35:3.6} 4. La esfera de la vida espiritual inicial.

\par
%\textsuperscript{(387.8)}
\textsuperscript{35:3.7} 5. El mundo de la vida espiritual intermedia.

\par
%\textsuperscript{(387.9)}
\textsuperscript{35:3.8} 6. La esfera de la vida espiritual avanzada.

\par
%\textsuperscript{(387.10)}
\textsuperscript{35:3.9} 7. El dominio de la autorrealización coordinada y suprema.

\par
%\textsuperscript{(387.11)}
\textsuperscript{35:3.10} Los seis mundos tributarios de cada una de estas esferas Melquisedeks están dedicados a las actividades relacionadas con el trabajo de la esfera primaria asociada.

\par
%\textsuperscript{(387.12)}
\textsuperscript{35:3.11} El mundo piloto, la esfera \textit{Melquisedek}, es el punto de encuentro común para todos los seres que se ocupan de educar y de espiritualizar a los mortales ascendentes del tiempo y del espacio. Para un ascendente, este mundo es probablemente el lugar más interesante de todo Nebadon. Todos los mortales evolutivos que han terminado su formación en las constelaciones están destinados a aterrizar en Melquisedek, donde son iniciados en el régimen de las disciplinas y de la progresión espiritual del sistema educativo de Salvington. Nunca olvidaréis las reacciones de vuestro primer día de vida en este mundo único, ni siquiera después de que hayáis alcanzado vuestro destino en el Paraíso.

\par
%\textsuperscript{(387.13)}
\textsuperscript{35:3.12} Los mortales ascendentes residen en el mundo Melquisedek mientras continúan su formación en los seis planetas circundantes de educación especializada. Y este mismo método se aplica durante toda la estancia en los setenta mundos culturales, las esferas primarias del circuito de Salvington.

\par
%\textsuperscript{(387.14)}
\textsuperscript{35:3.13} Muchas actividades diversas ocupan el tiempo de los numerosos seres que residen en los seis mundos tributarios de la esfera Melquisedek, pero en lo que se refiere a los mortales ascendentes, estos satélites se dedican a las fases especiales de estudio siguientes:

\par
%\textsuperscript{(388.1)}
\textsuperscript{35:3.14} 1. La esfera número uno se ocupa de revisar la vida planetaria inicial de los mortales ascendentes. Este trabajo se efectúa en clases compuestas por aquellos que proceden de un mundo dado de origen mortal. Los que provienen de Urantia realizan juntos esta revisión experiencial.

\par
%\textsuperscript{(388.2)}
\textsuperscript{35:3.15} 2. El trabajo especial de la esfera número dos consiste en una revisión similar de las experiencias vividas en los mundos de las mansiones que rodean al primer satélite de la sede del sistema local.

\par
%\textsuperscript{(388.3)}
\textsuperscript{35:3.16} 3. Las revisiones de esta esfera están relacionadas con la estancia en la capital del sistema local y abarcan las actividades del resto de los mundos arquitectónicos del grupo que forma la sede del sistema.

\par
%\textsuperscript{(388.4)}
\textsuperscript{35:3.17} 4. La cuarta esfera se ocupa de revisar las experiencias de los setenta mundos tributarios de la constelación y de sus esferas asociadas.

\par
%\textsuperscript{(388.5)}
\textsuperscript{35:3.18} 5. En la quinta esfera se realiza la revisión de la estancia ascendente en el mundo sede de la constelación.

\par
%\textsuperscript{(388.6)}
\textsuperscript{35:3.19} 6. El tiempo se dedica, en la esfera número seis, a intentar correlacionar estas cinco épocas y lograr así una coordinación de la experiencia, como preparación para entrar en las escuelas primarias Melquisedeks de formación universal.

\par
%\textsuperscript{(388.7)}
\textsuperscript{35:3.20} Las escuelas de administración universal y de sabiduría espiritual están situadas en el mundo nativo de los Melquisedeks, donde también se encuentran las escuelas dedicadas a una sola línea de investigación, tales como la energía, la materia, la organización, la comunicación, los archivos, la ética y la existencia comparada de las criaturas.

\par
%\textsuperscript{(388.8)}
\textsuperscript{35:3.21} En la Facultad Melquisedek de Dotación Espiritual, todas las órdenes de Hijos de Dios, incluidas las del Paraíso, cooperan con los Melquisedeks y los educadores seráficos para formar a las multitudes de evángeles del destino que salen a proclamar la libertad espiritual y la filiación divina incluso hasta los mundos lejanos del universo. Esta facultad particular de la Universidad Melquisedek es una institución exclusiva del universo; los visitantes estudiantiles procedentes de otros reinos no son admitidos aquí.

\par
%\textsuperscript{(388.9)}
\textsuperscript{35:3.22} El curso de formación más elevado en administración universal es impartido por los Melquisedeks en su mundo nativo. Esta Facultad de Ética Superior está presidida por el Padre Melquisedek original. Es a estas facultades donde los diversos universos envían sus estudiantes de intercambio. Aunque el joven universo de Nebadon se encuentra en un bajo nivel en la escala de los universos en cuanto a los logros espirituales y a un desarrollo ético elevado, sin embargo, nuestros problemas administrativos han convertido de tal manera a todo el universo en una inmensa clínica para otras creaciones cercanas, que las facultades Melquisedeks están atestadas de visitantes estudiantiles y de observadores de otros reinos. Además del inmenso grupo de inscritos locales, siempre hay más de cien mil estudiantes extranjeros que asisten a las escuelas Melquisedeks, porque la orden de los Melquisedeks de Nebadon es famosa en todo Splandon.

\section*{4. El trabajo especial de los Melquisedeks}
\par
%\textsuperscript{(388.10)}
\textsuperscript{35:4.1} Una rama sumamente especializada de las actividades de los Melquisedeks está relacionada con la supervisión de la carrera morontial progresiva de los mortales ascendentes. Una gran parte de esta formación está dirigida por los pacientes y sabios ministros seráficos, ayudados por los mortales que han ascendido hasta unos niveles relativamente superiores de consecución universal, pero todo este trabajo educativo se encuentra bajo la supervisión general de los Melquisedeks en asociación con los Hijos Instructores Trinitarios.

\par
%\textsuperscript{(389.1)}
\textsuperscript{35:4.2} Aunque las órdenes de los Melquisedeks se dedican principalmente al extenso sistema educativo y al régimen de formación experiencial del universo local, también actúan en misiones excepcionales y en circunstancias poco habituales. En un universo evolutivo que terminará por contener aproximadamente diez millones de mundos habitados, muchas cosas fuera de lo normal están destinadas a suceder, y en estos casos de emergencia es cuando actúan los Melquisedeks. En Edentia, la sede de vuestra constelación, se les conoce como Hijos de emergencia. Siempre están preparados para servir en todas las situaciones de necesidad ---físicas, intelectuales o espirituales--- ya sea en un planeta, en un sistema, en una constelación o en el universo. En cualquier momento y lugar en que se necesite una ayuda especial, allí encontraréis a uno o más Hijos Melquisedeks.

\par
%\textsuperscript{(389.2)}
\textsuperscript{35:4.3} Cuando algún aspecto del plan del Hijo Creador está amenazado de fracaso, un Melquisedek irá inmediatamente a prestar ayuda. Pero raras veces se les pide que actúen en presencia de una rebelión pecaminosa, como la que se produjo en Satania.

\par
%\textsuperscript{(389.3)}
\textsuperscript{35:4.4} Los Melquisedeks son los primeros en actuar en todas las emergencias de cualquier naturaleza en todos los mundos donde viven las criaturas volitivas. A veces actúan como guardianes temporales de los planetas desobedientes, sirviendo como síndicos de un gobierno planetario rebelde. En una crisis planetaria, estos Hijos Melquisedeks sirven en muchas tareas excepcionales. A este tipo de Hijo le resulta fácil hacerse visible a los seres mortales, y a veces un miembro de esta orden se ha encarnado incluso en la similitud de la carne mortal. Siete veces en Nebadon ha servido un Melquisedek en un mundo evolutivo en la similitud de la carne mortal, y estos Hijos han aparecido en numerosas ocasiones en la similitud de otras órdenes de criaturas del universo. Son en verdad los ministros de urgencia polifacéticos y voluntarios para todas las órdenes de inteligencias del universo y para todos los mundos y sistemas de mundos.

\par
%\textsuperscript{(389.4)}
\textsuperscript{35:4.5} El Melquisedek que vivió en Urantia en los tiempos de Abraham fue conocido localmente como el Príncipe de Salem\footnote{\textit{Melquisedek, Príncipe de Salem}: Gn 14:18; Heb 7:1-3.}, porque presidía una pequeña colonia de buscadores de la verdad que residían en un lugar llamado Salem. Se ofreció como voluntario para encarnarse en la similitud de la carne mortal, y lo hizo con la aprobación de los síndicos Melquisedeks del planeta, los cuales temían que la luz de la vida podría extinguirse durante aquel período de oscuridad espiritual creciente. Fomentó la verdad de su época y la transmitió de manera segura a Abraham y a sus asociados.

\section*{5. Los Hijos Vorondadeks}
\par
%\textsuperscript{(389.5)}
\textsuperscript{35:5.1} Después de la creación de los ayudantes personales y del primer grupo de los polifacéticos Melquisedeks, el Hijo Creador y el Espíritu Creativo del universo local planificaron, y trajeron a la existencia, a la segunda gran orden variada de filiación universal: los Vorondadeks. Se les conoce de manera más general como los Padres de las Constelaciones porque un Hijo de esta orden se encuentra uniformemente a la cabeza del gobierno de cada constelación en todos los universos locales.

\par
%\textsuperscript{(389.6)}
\textsuperscript{35:5.2} El número de Vorondadeks varía en cada universo local, y el número de ellos registrado en Nebadon se eleva exactamente a un millón. Estos Hijos, al igual que sus coordinados los Melquisedeks, no poseen el poder de reproducirse. No existe ningún método conocido por el cual puedan incrementar su número.

\par
%\textsuperscript{(389.7)}
\textsuperscript{35:5.3} Estos Hijos forman, en muchos aspectos, un cuerpo autónomo; como individuos, como grupos, e incluso como totalidad, se autodeterminan en gran medida como lo hacen los Melquisedeks, pero los Vorondadeks no ejercen sus funciones en una variedad tan amplia de actividades. No tienen la misma brillante diversidad de talentos que sus hermanos Melquisedeks, pero como gobernantes y administradores previsores son incluso más fiables y eficaces. Administrativamente tampoco se parecen por completo a sus subordinados, los Lanonandeks Soberanos de los Sistemas, pero superan a todas las órdenes de filiación del universo en la estabilidad de sus propósitos y en la divinidad de sus juicios.

\par
%\textsuperscript{(390.1)}
\textsuperscript{35:5.4} Aunque los fallos y las decisiones de esta orden de Hijos están siempre de acuerdo con el espíritu de filiación divina y en armonía con la política del Hijo Creador, han sido citados a causa de sus errores ante el Hijo Creador, y sus decisiones relativas a detalles técnicos a veces han sido revocadas en la apelación a los tribunales superiores del universo. Pero estos Hijos raras veces caen en el error, y nunca han emprendido una rebelión; en toda la historia de Nebadon nunca se ha oído decir que un Vorondadek haya cometido desacato al gobierno del universo.

\par
%\textsuperscript{(390.2)}
\textsuperscript{35:5.5} El servicio de los Vorondadeks en los universos locales es amplio y variado. Sirven como embajadores ante otros universos y como cónsules representando a las constelaciones dentro de su universo nativo. De todas las órdenes de filiación de un universo local, es a ellos a quienes más a menudo se les confía la plena delegación de los poderes soberanos a ejercer en las situaciones críticas del universo.

\par
%\textsuperscript{(390.3)}
\textsuperscript{35:5.6} En aquellos mundos aislados en las tinieblas espirituales, en aquellas esferas que han sufrido el aislamiento planetario debido a la rebelión y a la negligencia, un observador Vorondadek está generalmente presente hasta el restablecimiento del estado normal. En ciertos casos de emergencia, este observador Altísimo podría ejercer una autoridad absoluta y arbitraria sobre todos los seres celestiales destinados en ese planeta. Los archivos de Salvington mencionan que los Vorondadeks han ejercido a veces esta autoridad como regentes Altísimos de tales planetas. Y esto también ha sucedido incluso en los mundos habitados que no han sido afectados por la rebelión.

\par
%\textsuperscript{(390.4)}
\textsuperscript{35:5.7} A menudo, un cuerpo de doce o más Hijos Vorondadeks forman un alto tribunal de revisión y de apelación con respecto a casos especiales que afectan al estado de un planeta o de un sistema. Pero su trabajo está principalmente relacionado con las funciones legislativas autóctonas de los gobiernos de las constelaciones. Como resultado de todos estos servicios, los Hijos Vorondadeks se han convertido en los historiadores de los universos locales; están familiarizados personalmente con todas las luchas políticas y todas las agitaciones sociales de los mundos habitados.

\section*{6. Los Padres de las Constelaciones}
\par
%\textsuperscript{(390.5)}
\textsuperscript{35:6.1} Al menos tres Vorondadeks están asignados al gobierno de cada una de las cien constelaciones de un universo local. Estos Hijos son elegidos por el Hijo Creador y son nombrados por Gabriel como Altísimos de las constelaciones para servir allí durante un decamilenio ---10.000 años oficiales, unos 50.000 años del tiempo de Urantia. El Altísimo reinante, el Padre de la Constelación, tiene dos asociados, uno más antiguo y otro más reciente. En cada cambio de administración, el asociado más antiguo se convierte en el jefe del gobierno, y el más reciente asume los deberes del más antiguo, mientras que los Vorondadeks sin tarea asignada que residen en los mundos de Salvington proponen a uno de sus miembros como candidato a ser elegido para asumir las responsabilidades del asociado más reciente. Así, de acuerdo con la política actual, cada uno de los gobernantes Altísimos tiene un período de servicio en la sede de una constelación de tres decamilenios, unos
150.000 años de Urantia.

\par
%\textsuperscript{(390.6)}
\textsuperscript{35:6.2} Los cien Padres de las Constelaciones, los jefes que presiden realmente los gobiernos de las constelaciones, componen el gabinete consultivo supremo del Hijo Creador. Este consejo celebra sesiones frecuentes en la sede del universo, y el alcance y la variedad de sus deliberaciones son ilimitados, pero se ocupa principalmente del bienestar de las constelaciones y de la unificación de la administración de todo el universo local.

\par
%\textsuperscript{(391.1)}
\textsuperscript{35:6.3} Cuando el Padre de una Constelación está ocupándose de sus obligaciones en la sede del universo, como sucede con frecuencia, el asociado más antiguo se convierte en el director interino de los asuntos de la constelación. La actividad normal del asociado más antiguo es la supervisión de los asuntos espirituales, mientras que el asociado más reciente se ocupa personalmente del bienestar físico de la constelación. Sin embargo, ninguna política importante se lleva nunca a cabo en una constelación a menos que los tres Altísimos estén de acuerdo sobre todos los detalles de su ejecución.

\par
%\textsuperscript{(391.2)}
\textsuperscript{35:6.4} Todo el mecanismo de la información espiritual y de los canales de comunicación está a la disposición de los Altísimos de las constelaciones. Se encuentran en contacto perfecto con sus superiores en Salvington y con sus subordinados directos, los soberanos de los sistemas locales. Con frecuencia se reúnen en consejo con estos Soberanos de los Sistemas para deliberar sobre el estado de la constelación.

\par
%\textsuperscript{(391.3)}
\textsuperscript{35:6.5} Los Altísimos se rodean de un cuerpo de consejeros, que varía de vez en cuando en cantidad y en personal con arreglo a la presencia de los diversos grupos en las sedes de las constelaciones, y también a medida que varían las necesidades locales. Durante los períodos difíciles pueden solicitar más Hijos de la orden Vorondadek para que los ayuden en el trabajo administrativo, y los reciben rápidamente. Norlatiadek, vuestra propia constelación, está administrada actualmente por doce Hijos Vorondadeks.

\section*{7. Los mundos Vorondadeks}
\par
%\textsuperscript{(391.4)}
\textsuperscript{35:7.1} El segundo grupo de siete mundos que se encuentra en el circuito de las setenta esferas primarias que rodean a Salvington contiene los planetas Vorondadeks. Cada una de estas esferas, con sus seis satélites circundantes, está dedicada a una fase especial de las actividades Vorondadeks. En estos cuarenta y nueve reinos, los mortales ascendentes alcanzan el apogeo de su educación sobre la legislación del universo.

\par
%\textsuperscript{(391.5)}
\textsuperscript{35:7.2} Los mortales ascendentes han observado el funcionamiento de las asambleas legislativas en los mundos sede de las constelaciones, pero aquí, en estos mundos Vorondadeks, participan en la promulgación de la legislación general real del universo local bajo la tutela de los Vorondadeks más antiguos. Estas promulgaciones están destinadas a coordinar las diversas declaraciones de las asambleas legislativas autónomas de las cien constelaciones. La enseñanza que se recibe en las escuelas Vorondadeks es insuperable incluso en Uversa. Esta formación es progresiva y se extiende desde la primera esfera, con trabajos adicionales en sus seis satélites, hasta las seis esferas primarias restantes y sus grupos de satélites asociados.

\par
%\textsuperscript{(391.6)}
\textsuperscript{35:7.3} Los peregrinos ascendentes iniciarán numerosas actividades nuevas en estos mundos de estudio y de trabajo práctico. No se nos prohíbe emprender la revelación de estas ocupaciones nuevas e inimaginables, pero desesperamos de poder describir estas tareas a la mente material de los seres mortales. No tenemos palabras para transmitir los significados de estas actividades celestiales, y no existen tareas humanas análogas que se puedan utilizar como ejemplos de estas nuevas ocupaciones de los mortales ascendentes que continúan sus estudios en estos cuarenta y nueve mundos. Y otras muchas actividades, que no forman parte del régimen ascendente, están centradas en estos mundos Vorondadeks del circuito de Salvington.

\section*{8. Los Hijos Lanonandeks}
\par
%\textsuperscript{(392.1)}
\textsuperscript{35:8.1} Después de la creación de los Vorondadeks, el Hijo Creador y el Espíritu Madre del Universo se unen con el objeto de traer a la existencia a la tercera orden de filiación del universo: los Lanonandeks. Aunque se ocupan de tareas diversas relacionadas con las administraciones de los sistemas, son mejor conocidos como Soberanos de los Sistemas, los gobernantes de los sistemas locales, y como Príncipes Planetarios, los jefes administrativos de los mundos habitados.

\par
%\textsuperscript{(392.2)}
\textsuperscript{35:8.2} Como forman una orden de filiación más tardía e inferior ---en lo que se refiere a los niveles de divinidad--- estos seres necesitaron pasar por ciertos cursos de formación en los mundos Melquisedeks a fin de prepararse para su servicio posterior. Fueron los primeros estudiantes de la Universidad Melquisedek y fueron clasificados y certificados por sus educadores y examinadores Melquisedeks de acuerdo con sus capacidades, su personalidad y sus logros.

\par
%\textsuperscript{(392.3)}
\textsuperscript{35:8.3} El universo de Nebadon empezó su existencia con doce millones exactos de Lanonandeks, y después de pasar por la esfera Melquisedek, en las pruebas finales fueron divididos en tres clases:

\par
%\textsuperscript{(392.4)}
\textsuperscript{35:8.4} 1. \textit{Los Lanonandeks primarios}. De la categoría más elevada había 709.841 miembros. Éstos son los Hijos designados como Soberanos de los Sistemas y asistentes de los consejos supremos de las constelaciones, y como consejeros en el trabajo administrativo superior del universo.

\par
%\textsuperscript{(392.5)}
\textsuperscript{35:8.5} 2. \textit{Los Lanonandeks secundarios}. Cuando esta orden salió de Melquisedek había 10.234.601 miembros. Son destinados como Príncipes Planetarios y a las reservas de esta orden.

\par
%\textsuperscript{(392.6)}
\textsuperscript{35:8.6} 3. \textit{Los Lanonandeks terciarios}. Este grupo contenía 1.055.558 miembros. Estos Hijos actúan como asistentes subordinados, mensajeros, custodios, comisionados, observadores, y llevan a cabo los diversos deberes de un sistema y de los mundos que lo componen.

\par
%\textsuperscript{(392.7)}
\textsuperscript{35:8.7} A estos Hijos no les resulta posible progresar de un grupo a otro como les sucede a los seres evolutivos. Después de estar sometidos a la formación de los Melquisedeks, una vez que han sido probados y clasificados, sirven continuamente en la categoría asignada. Estos Hijos tampoco pueden reproducirse; su número en el universo es fijo.

\par
%\textsuperscript{(392.8)}
\textsuperscript{35:8.8} En números redondos, la orden de los Hijos Lanonandeks está clasificada en Salvington como sigue:

\par
%\textsuperscript{(392.9)}
\textsuperscript{35:8.9} Coordinadores del Universo y Consejeros de las Constelaciones . .
100.000

\par
%\textsuperscript{(392.10)}
\textsuperscript{35:8.10} Soberanos de los Sistemas y Asistentes . . . . . . . . . . . . .
. . 600.000

\par
%\textsuperscript{(392.11)}
\textsuperscript{35:8.11} Príncipes Planetarios y Reservas . . . . . . . . . . . . . . .
. . . .10.000.000

\par
%\textsuperscript{(392.12)}
\textsuperscript{35:8.12} Cuerpo de Mensajeros . . . . . . . . . . . . . . . . . . . . . .
. . . . . 400.000

\par
%\textsuperscript{(392.13)}
\textsuperscript{35:8.13} Custodios y Archivistas. . . . . . . . . . . . . . . . . . . . .
. . . . . . 100.000

\par
%\textsuperscript{(392.14)}
\textsuperscript{35:8.14} Cuerpo de Reserva. . . . . . . . . . . . . . . . . . . . . . . .
. . . . . . 800.000

\par
%\textsuperscript{(392.15)}
\textsuperscript{35:8.15} Puesto que los Lanonandeks son una orden de filiación un poco inferior a las de los Melquisedeks y los Vorondadeks, prestan un servicio aún mayor en las unidades subordinadas del universo, puesto que son capaces de acercarse más a las humildes criaturas de las razas inteligentes. También corren un mayor peligro de descarriarse, de apartarse de la técnica adecuada del gobierno universal. Pero estos Lanonandeks, especialmente los de la orden primaria, son los más capaces y polifacéticos de todos los administradores de los universos locales. En capacidad ejecutiva sólo son superados por Gabriel y sus asociados no revelados.

\section*{9. Los gobernantes Lanonandeks}
\par
%\textsuperscript{(393.1)}
\textsuperscript{35:9.1} Los Lanonandeks son los gobernantes continuos de los planetas y los soberanos rotativos de los sistemas. Uno de estos Hijos gobierna ahora en Jerusem, la sede de vuestro sistema local de mundos habitados.

\par
%\textsuperscript{(393.2)}
\textsuperscript{35:9.2} Los Soberanos de los Sistemas gobiernan en comisiones de dos o tres miembros en la sede de cada sistema de mundos habitados. El Padre de la Constelación nombra a uno de estos Lanonandeks como jefe cada decamilenio. A veces no se efectúa ningún cambio de jefe en el trío, siendo el asunto totalmente optativo para los gobernantes de la constelación. El personal de los gobiernos de los sistemas no cambia repentinamente, a menos que se produzca una tragedia de algún tipo.

\par
%\textsuperscript{(393.3)}
\textsuperscript{35:9.3} Cuando los Soberanos de los Sistemas o los asistentes son retirados, el consejo supremo situado en la sede de la constelación ocupa dichos puestos mediante una selección efectuada entre las reservas de esta orden, un grupo que es más numeroso en Edentia que la media indicada.

\par
%\textsuperscript{(393.4)}
\textsuperscript{35:9.4} Los consejos supremos de los Lanonandeks están estacionados en las diversas sedes de las constelaciones. Este cuerpo está presidido por el Altísimo asociado más antiguo del Padre de la Constelación, mientras que el asociado más reciente supervisa las reservas de la orden secundaria.

\par
%\textsuperscript{(393.5)}
\textsuperscript{35:9.5} Los Soberanos de los Sistemas son fieles a sus nombres; son casi soberanos en los asuntos locales de los mundos habitados. Son casi paternales en su manera de dirigir a los Príncipes Planetarios, los Hijos Materiales y los espíritus ministrantes. El dominio personal del soberano es casi completo. Estos gobernantes no están supervisados por los observadores trinitarios procedentes del universo central. Forman la división ejecutiva del universo local; como custodios de que se cumplan los mandatos legislativos y como ejecutivos encargados de aplicar los veredictos judiciales, representan el único puesto en toda la administración del universo donde la deslealtad personal hacia la voluntad del Hijo Miguel podría afianzarse y tratar de imponerse con más facilidad y rapidez.

\par
%\textsuperscript{(393.6)}
\textsuperscript{35:9.6} Nuestro universo local ha sido desafortunado, ya que más de setecientos Hijos de la orden Lanonandek se han rebelado contra el gobierno del universo, precipitando así la confusión sobre diversos sistemas y numerosos planetas. De toda esta cantidad de fracasos, sólo tres eran Soberanos de Sistemas; prácticamente todos estos Hijos pertenecían a las órdenes segunda y tercera, las de los Príncipes Planetarios y los Lanonandeks terciarios.

\par
%\textsuperscript{(393.7)}
\textsuperscript{35:9.7} El gran número de estos Hijos que han faltado a su integridad no indica ningún defecto en sus creadores. Podían haber sido creados divinamente perfectos, pero fueron creados de tal manera que pudieran comprender mejor, y acercarse más, a las criaturas evolutivas que viven en los mundos del tiempo y del espacio.

\par
%\textsuperscript{(393.8)}
\textsuperscript{35:9.8} De todos los universos locales de Orvonton, a excepción de Henselon, nuestro universo es el que ha perdido el mayor número de esta orden de Hijos. En Uversa existe la opinión general de que hemos tenido tantos problemas administrativos en Nebadon porque nuestros Hijos de la orden Lanonandek fueron creados con un amplio grado de libertad personal para elegir y hacer planes. No hago este comentario como una crítica. El Creador de nuestro universo tiene pleno poder y autoridad para hacer esto. Nuestros elevados gobernantes opinan que, aunque estos Hijos con libertad de elección provocan excesivos conflictos en los primeros tiempos del universo, cuando las cosas se hayan cribado por completo y establecido definitivamente, los beneficios de una lealtad más elevada y de un servicio volitivo más completo por parte de estos Hijos totalmente probados, compensarán con creces la confusión y las tribulaciones de las épocas anteriores.

\par
%\textsuperscript{(394.1)}
\textsuperscript{35:9.9} En el caso de una rebelión en la sede de un sistema, normalmente se instala a un nuevo soberano dentro de un plazo relativamente corto, pero no sucede lo mismo en los planetas individuales. Éstos son las unidades que componen la creación material, y el libre albedrío de las criaturas es un factor a tener en cuenta en el juicio final de todos estos problemas. Se nombran Príncipes Planetarios sucesores para los mundos aislados, para los planetas cuyos príncipes con autoridad pueden haberse descarriado, pero no asumen el gobierno activo de dichos mundos hasta que los resultados de la insurrección no se hayan superado y eliminado parcialmente gracias a las medidas reparadoras adoptadas por los Melquisedeks y otras personalidades ministrantes. La rebelión de un Príncipe Planetario aísla instantáneamente a su planeta; los circuitos espirituales locales se cortan de inmediato. Sólo un Hijo donador puede restablecer las líneas de comunicación interplanetarias de ese mundo espiritualmente aislado.

\par
%\textsuperscript{(394.2)}
\textsuperscript{35:9.10} Existe un plan para salvar a estos Hijos desobedientes e imprudentes, y muchos de ellos han utilizado esta disposición misericordiosa; pero nunca más podrán ejercer su actividad en aquellos puestos donde fallaron. Después de su rehabilitación son asignados a las tareas de custodia y a los departamentos de la administración física.

\section*{10. Los mundos Lanonandeks}
\par
%\textsuperscript{(394.3)}
\textsuperscript{35:10.1} En el circuito de setenta planetas de Salvington, el tercer grupo de siete mundos con sus cuarenta y dos satélites respectivos constituyen el enjambre de esferas administrativas de los Lanonandeks. En estos reinos, los Lanonandeks experimentados que pertenecen al cuerpo de antiguos Soberanos Sistémicos ejercen sus funciones como instructores administrativos de los peregrinos ascendentes y de las huestes seráficas. Los mortales evolutivos observan el trabajo de los administradores del sistema en las capitales de los sistemas, pero aquí participan en la coordinación efectiva de las declaraciones administrativas de los diez mil sistemas locales.

\par
%\textsuperscript{(394.4)}
\textsuperscript{35:10.2} Estas escuelas administrativas del universo local están supervisadas por un cuerpo de Hijos Lanonandeks que han tenido una larga experiencia como Soberanos Sistémicos y como consejeros en las constelaciones. Estos colegios ejecutivos sólo son superados por las escuelas administrativas de Ensa.

\par
%\textsuperscript{(394.5)}
\textsuperscript{35:10.3} Aunque sirven como esferas de formación para los mortales ascendentes, los mundos Lanonandeks son los centros de extensas empresas relacionadas con las actividades administrativas normales y rutinarias del universo. Durante todo el camino hacia el Paraíso, los peregrinos ascendentes continúan sus estudios en las escuelas prácticas de conocimientos aplicados ---una verdadera formación que consiste en hacer realmente las cosas que les enseñan. El sistema educativo universal patrocinado por los Melquisedeks es práctico, progresivo, significativo y experiencial. Abarca la instrucción en las cosas materiales, intelectuales, morontiales y espirituales.

\par
%\textsuperscript{(394.6)}
\textsuperscript{35:10.4} En conexión con estas esferas administrativas de los Lanonandeks, la mayoría de los Hijos salvados de esta orden sirven como custodios y directores de los asuntos planetarios. Estos Príncipes Planetarios rebeldes, y sus asociados en la rebelión, que eligen aceptar la rehabilitación ofrecida, continuarán sirviendo en estas funciones rutinarias al menos hasta que el universo de Nebadon se establezca en la luz y la vida.

\par
%\textsuperscript{(395.1)}
\textsuperscript{35:10.5} Sin embargo, muchos Hijos Lanonandeks de los sistemas más antiguos han establecido maravillosos historiales de servicio, de administración y de logros espirituales. Forman un grupo noble, fiel y leal, a pesar de su tendencia a caer en el error debido a los sofismas de la libertad personal y a las ficciones de la autodeterminación.

\par
%\textsuperscript{(395.2)}
\textsuperscript{35:10.6} [Patrocinado por el Jefe de los Arcángeles, que actúa por autorización de Gabriel de Salvington.]


\chapter{Documento 36. Los Portadores de Vida}
\par
%\textsuperscript{(396.1)}
\textsuperscript{36:0.1} LA VIDA no se origina de manera espontánea. La vida se construye de acuerdo con los planes formulados por los Arquitectos del Ser (no revelados), y aparece en los planetas habitados o bien por importación directa o como resultado de las operaciones de los Portadores de Vida de los universos locales. Estos portadores de la vida se encuentran entre los miembros más interesantes y polifacéticos de la variada familia de Hijos del universo. Se les ha confiado diseñar y llevar la vida de las criaturas a las esferas planetarias. Después de plantar esta vida en esos nuevos mundos, permanecen allí durante largos períodos de tiempo para fomentar su desarrollo.

\section*{1. El origen y la naturaleza de los Portadores de Vida}
\par
%\textsuperscript{(396.2)}
\textsuperscript{36:1.1} Aunque los Portadores de Vida pertenecen a la familia de filiación divina, son un tipo particular y distinto de Hijos universales, pues constituyen el único grupo de vida inteligente de un universo local en cuya creación participan los gobernantes de un superuniverso. Los Portadores de Vida son los descendientes de tres personalidades pre-existentes: el Hijo Creador, el Espíritu Madre del Universo y, por designación, uno de los tres Ancianos de los Días que presiden los destinos del superuniverso interesado. Estos Ancianos de los Días, los únicos que pueden decretar la extinción de la vida inteligente, participan en la creación de los Portadores de Vida, que están encargados de establecer la vida física en los mundos evolutivos.

\par
%\textsuperscript{(396.3)}
\textsuperscript{36:1.2} En el universo de Nebadon tenemos constancia de la creación de cien millones de Portadores de Vida. Este cuerpo eficaz de propagadores de la vida no es un grupo realmente autónomo. Está dirigido por el trío determinante de la vida compuesto por Gabriel, el Padre Melquisedek y Nambia, el Portador de Vida primogénito y original de Nebadon. Pero son autónomos en todas las fases de su administración divisionaria.

\par
%\textsuperscript{(396.4)}
\textsuperscript{36:1.3} Los Portadores de Vida están clasificados en tres grandes divisiones: la primera división es la de los Portadores de Vida más antiguos; la segunda, la de los ayudantes, y la tercera, la de los custodios. La primera división está subdividida en doce grupos de especialistas en las diversas formas de manifestación de la vida. La separación de estas tres divisiones fue efectuada por los Melquisedeks, que realizaron pruebas con esta finalidad en la esfera sede de los Portadores de Vida. Los Melquisedeks han estado desde entonces estrechamente asociados a los Portadores de Vida, y siempre los acompañan cuando salen para establecer la vida en un nuevo planeta.

\par
%\textsuperscript{(396.5)}
\textsuperscript{36:1.4} Cuando un planeta evolutivo se establece finalmente en la luz y la vida, los Portadores de Vida se organizan en cuerpos deliberativos superiores con capacidad consultiva para ayudar a la administración y al desarrollo ulteriores del mundo y de sus seres glorificados. En las épocas estabilizadas y más tardías de un universo evolutivo, a estos Portadores de Vida se les confían muchas nuevas obligaciones.

\section*{2. Los mundos de los Portadores de Vida}
\par
%\textsuperscript{(397.1)}
\textsuperscript{36:2.1} Los Melquisedeks ejercen la supervisión general del cuarto grupo de siete esferas primarias del circuito de Salvington. Estos mundos de los Portadores de Vida se denominan como sigue:

\par
%\textsuperscript{(397.2)}
\textsuperscript{36:2.2} 1. La sede de los Portadores de Vida.

\par
%\textsuperscript{(397.3)}
\textsuperscript{36:2.3} 2. La esfera de planificación de la vida.

\par
%\textsuperscript{(397.4)}
\textsuperscript{36:2.4} 3. La esfera de conservación de la vida.

\par
%\textsuperscript{(397.5)}
\textsuperscript{36:2.5} 4. La esfera de la evolución de la vida.

\par
%\textsuperscript{(397.6)}
\textsuperscript{36:2.6} 5. La esfera de la vida asociada con la mente.

\par
%\textsuperscript{(397.7)}
\textsuperscript{36:2.7} 6. La esfera de la mente y del espíritu en los seres vivientes.

\par
%\textsuperscript{(397.8)}
\textsuperscript{36:2.8} 7. La esfera de la vida no revelada.

\par
%\textsuperscript{(397.9)}
\textsuperscript{36:2.9} Cada una de estas esferas primarias está rodeada de seis satélites, en los cuales están centradas las fases especiales de todas las actividades de los Portadores de Vida en el universo.

\par
%\textsuperscript{(397.10)}
\textsuperscript{36:2.10} \textit{El mundo número uno}, la esfera sede, junto con sus seis satélites tributarios, está dedicado al estudio de la vida universal, de la vida en todas sus fases conocidas de manifestación. Aquí está situado el colegio de planificación de la vida, donde ejercen su actividad los instructores y consejeros de Uversa, Havona e incluso del Paraíso. Tengo permiso para revelar que los siete emplazamientos centrales de los espíritus ayudantes de la mente están situados en este mundo de los Portadores de Vida.

\par
%\textsuperscript{(397.11)}
\textsuperscript{36:2.11} El número diez ---el sistema decimal--- es inherente al universo físico, pero no al espiritual. El ámbito de la vida está caracterizado por los números tres, siete y doce, o por múltiplos y combinaciones de estos números fundamentales. Existen tres planes de vida primordiales y esencialmente diferentes, según la orden de las tres Fuentes y Centros del Paraíso, y en el universo de Nebadon, estas tres formas básicas de vida están separadas en tres tipos diferentes de planetas. Originalmente había doce conceptos divinos y distintos de vida transmisible. Este número doce, con sus subdivisiones y múltiplos, aparece en todos los modelos básicos de vida de los siete superuniversos. Hay también siete tipos arquitectónicos de diseño de vida, las combinaciones fundamentales de las configuraciones reproductoras de la materia viviente. Los modelos de vida de Orvonton están configurados bajo la forma de doce portadores de la herencia. Las diferentes órdenes de criaturas volitivas están configuradas según los números 12, 24, 48, 96, 192, 384 y 768. En Urantia hay cuarenta y ocho unidades de control arquetípico ---de determinadores de las características--- en las células sexuales de la reproducción humana.

\par
%\textsuperscript{(397.12)}
\textsuperscript{36:2.12} \textit{El segundo mundo} es la esfera donde se diseña la vida; aquí se elaboran todos los nuevos modos de organizar la vida. Aunque los diseños originales de vida son proporcionados por el Hijo Creador, la elaboración real de estos planes se confía a los Portadores de Vida y a sus asociados. Cuando los planes generales de vida para un nuevo mundo se han formulado, se transmiten a la esfera sede, donde son examinados minuciosamente por el consejo supremo de los Portadores de Vida más antiguos en colaboración con un cuerpo de asesores Melquisedeks. Si los planes se desvían de las fórmulas previamente aceptadas, deben ser pasados al Hijo Creador y aprobados por él. El jefe de los Melquisedeks representa con frecuencia al Hijo Creador en estas deliberaciones.

\par
%\textsuperscript{(397.13)}
\textsuperscript{36:2.13} En consecuencia, aunque la vida planetaria es similar en algunos aspectos, difiere de muchas maneras en cada mundo evolutivo. Incluso en una serie de vida uniforme de una sola familia de mundos, la vida no es exactamente la misma en dos planetas dados; siempre existe un tipo planetario, ya que los Portadores de Vida se esfuerzan constantemente por mejorar las fórmulas vitales confiadas a su cuidado.

\par
%\textsuperscript{(398.1)}
\textsuperscript{36:2.14} Hay más de un millón de fórmulas químicas fundamentales o cósmicas que componen los modelos parentales y las numerosas variaciones funcionales básicas de las manifestaciones de la vida. El satélite número uno de la esfera donde se planifica la vida es el campo de actividad de los físicos y los electroquímicos del universo, que sirven como ayudantes técnicos de los Portadores de Vida en el trabajo de capturar, organizar y manipular las unidades esenciales de energía que se emplean para construir los vehículos materiales que transmiten la vida, el llamado plasma germinativo.

\par
%\textsuperscript{(398.2)}
\textsuperscript{36:2.15} Los laboratorios planetarios donde se planifica la vida están situados en el segundo satélite de este mundo número dos. En estos laboratorios, los Portadores de Vida y todos sus asociados colaboran con los Melquisedeks esforzándose por modificar, y quizás mejorar, la vida destinada a ser implantada en los \textit{planetasdecimales} de Nebadon. La vida que evoluciona actualmente en Urantia fue planeada y parcialmente elaborada en este mismo mundo, ya que Urantia es un planeta decimal, un mundo donde se experimenta con la vida. En un mundo de cada diez se permite que los diseños normales de vida varíen más que en los otros mundos (no experimentales).

\par
%\textsuperscript{(398.3)}
\textsuperscript{36:2.16} \textit{El mundo número tres} está dedicado a la conservación de la vida. Aquí, los ayudantes y los custodios del cuerpo de Portadores de Vida estudian y desarrollan diversas maneras de proteger y de conservar la vida. Los planes de vida para cada nuevo mundo siempre prevén que se establezca pronto la comisión para la conservación de la vida, compuesta por los custodios especialistas en la experta manipulación de los modelos básicos de vida. En Urantia había veinticuatro custodios comisionados de este tipo, dos por cada modelo fundamental o principal de la organización arquitectónica del material de vida. En los planetas como el vuestro, la forma más elevada de vida es reproducida por un haz portador de vida que posee veinticuatro unidades modelo. (Y puesto que la vida intelectual se deriva de la vida física, y está basada en ella, surgen a la existencia las veinticuatro órdenes básicas de organización psíquica).

\par
%\textsuperscript{(398.4)}
\textsuperscript{36:2.17} \textit{La esfera número cuatro} y sus satélites tributarios están dedicados al estudio de la evolución de la vida de las criaturas en general, y a los antecedentes evolutivos de cualquier nivel de vida en particular. El plasma original de vida de un mundo evolutivo debe contener todo el potencial de todas las variaciones de desarrollo futuras y de todos los cambios y modificaciones evolutivos posteriores. La provisión para estos proyectos de gran alcance de metamorfosis de la vida puede necesitar la aparición de muchas formas aparentemente inútiles de vida animal y vegetal. Estos subproductos de la evolución planetaria, previstos o imprevistos, sólo aparecen en el campo de acción para desaparecer, pero a través de todo este largo proceso corre el hilo de las formulaciones sabias e inteligentes de los diseñadores originales del plan de la vida planetaria y del proyecto de las especies. Todos los múltiples subproductos de la evolución biológica son esenciales para el funcionamiento pleno y final de las formas superiores de vida inteligente, a pesar de que una gran falta de armonía exterior pueda prevalecer de vez en cuando en la larga lucha ascendente de las criaturas superiores por dominar las formas inferiores de vida, muchas de las cuales son a veces tan hostiles para la paz y la comodidad de las criaturas volitivas evolutivas.

\par
%\textsuperscript{(398.5)}
\textsuperscript{36:2.18} \textit{El mundo número cinco} se ocupa enteramente de la vida asociada a la mente. Cada uno de sus satélites se dedica al estudio de una sola fase de la mente correlacionada con la vida de las criaturas. La mente, tal como el hombre la comprende, es un don de los siete espíritus ayudantes de la mente, que los agentes del Espíritu Infinito superponen a los niveles no enseñables o maquinales de la mente. Los modelos de vida responden de maneras diversas a estos ayudantes y a los diferentes ministerios espirituales que funcionan en todos los universos del tiempo y del espacio. La capacidad de las criaturas materiales para reaccionar de forma espiritual depende totalmente del don mental asociado que, a su vez, ha dirigido el curso de la evolución biológica de estas mismas criaturas mortales.

\par
%\textsuperscript{(399.1)}
\textsuperscript{36:2.19} \textit{El mundo número seis} se dedica a correlacionar la mente con el espíritu tal como están asociados con las formas y los organismos vivientes. Este mundo y sus seis tributarios contienen las escuelas de coordinación de las criaturas, donde los educadores procedentes tanto del universo central como del superuniverso colaboran con los instructores de Nebadon para presentar los niveles más elevados que las criaturas pueden alcanzar en el tiempo y el espacio.

\par
%\textsuperscript{(399.2)}
\textsuperscript{36:2.20} \textit{La séptima esfera} de los Portadores de Vida se dedica a los dominios no revelados de la vida evolutiva de las criaturas, tal como está relacionada con la filosofía cósmica de la manifestación creciente del Ser Supremo.

\section*{3. El transplante de la vida}
\par
%\textsuperscript{(399.3)}
\textsuperscript{36:3.1} La vida no aparece de forma espontánea en los universos; los Portadores de Vida deben iniciarla en los planetas estériles. Ellos son los portadores, los propagadores y los guardianes de la vida tal como ésta aparece en los mundos evolutivos del espacio. Toda vida de la clase y de las formas que se conocen en Urantia surge con estos Hijos, aunque no todas las formas de vida planetaria existen en Urantia.

\par
%\textsuperscript{(399.4)}
\textsuperscript{36:3.2} El cuerpo de Portadores de Vida encargado de plantar la vida en un nuevo mundo está compuesto normalmente de cien portadores más antiguos, cien ayudantes y mil custodios. Los Portadores de Vida llevan a menudo el plasma vital concreto a un nuevo mundo, pero no siempre. A veces organizan los modelos de la vida después de llegar al planeta asignado, de acuerdo con las fórmulas aprobadas previamente para la nueva aventura de establecer la vida. Éste fue el origen de la vida planetaria en Urantia.

\par
%\textsuperscript{(399.5)}
\textsuperscript{36:3.3} Cuando los modelos físicos conformes con las fórmulas aprobadas han sido suministrados, entonces los Portadores de Vida catalizan este material inanimado comunicándole a través de sus personas la chispa vital del espíritu, y los modelos inertes se convierten inmediatamente en materia viviente.

\par
%\textsuperscript{(399.6)}
\textsuperscript{36:3.4} La chispa vital ---el misterio de la vida---\footnote{\textit{Chispa vital, misterio de la vida}: Sal 119:93; Jn 5:21; 6:63; 1 Ti 6:13.} se confiere a través de los Portadores de Vida, pero no procede de ellos. Ellos supervisan en verdad estas operaciones, formulan el plasma vital mismo, pero es el Espíritu Madre del Universo el que proporciona el factor esencial del plasma viviente. De la Hija Creativa del Espíritu Infinito proviene esa chispa de energía que anima el cuerpo y presagia la mente.

\par
%\textsuperscript{(399.7)}
\textsuperscript{36:3.5} Los Portadores de Vida no transmiten nada de su naturaleza personal cuando conceden la vida, ni siquiera en aquellas esferas donde se proyectan nuevos tipos de vida. En tales ocasiones se limitan a iniciar y transmitir la chispa de la vida, a poner en marcha las rotaciones necesarias de la materia de acuerdo con las especificaciones físicas, químicas y eléctricas de los planes y modelos ordenados. Los Portadores de Vida son presencias catalíticas vivientes que agitan, organizan y vivifican los elementos, por otra parte inertes, del tipo de existencia material.

\par
%\textsuperscript{(400.1)}
\textsuperscript{36:3.6} A los Portadores de Vida de un cuerpo planetario les conceden cierto plazo de tiempo para establecer la vida en un nuevo mundo, aproximadamente medio millón de años del tiempo de ese planeta. Al final de este período, indicado por ciertos logros en el desarrollo de la vida planetaria, ponen fin a sus esfuerzos de implantación, y ya no pueden añadir posteriormente nada nuevo o suplementario a la vida de ese planeta.

\par
%\textsuperscript{(400.2)}
\textsuperscript{36:3.7} Durante las épocas intermedias entre el establecimiento de la vida y la aparición de las criaturas humanas con categoría moral, los Portadores de Vida tienen permiso para manipular el entorno de la vida y dirigir favorablemente de otras maneras el curso de la evolución biológica. Y así lo hacen durante largos períodos de tiempo.

\par
%\textsuperscript{(400.3)}
\textsuperscript{36:3.8} Cuando los Portadores de Vida que trabajan en un nuevo mundo han conseguido dar nacimiento una vez a un ser con voluntad, con el poder de decisión moral y de elección espiritual, en ese mismo instante finaliza su trabajo ---han terminado; ya no pueden manipular la vida en evolución. Desde ese momento en adelante, la evolución de los seres vivos debe continuar con arreglo a la dotación de la naturaleza y de las tendencias inherentes que ya han sido comunicadas a las fórmulas y modelos de la vida planetaria, y establecidas en ellos. A los Portadores de Vida no les permiten experimentar con la voluntad o interferir en ella; no tienen permiso para dominar o influir arbitrariamente sobre las criaturas morales.

\par
%\textsuperscript{(400.4)}
\textsuperscript{36:3.9} Cuando llega un Príncipe Planetario, se preparan para marcharse, aunque dos de los portadores más antiguos y doce custodios pueden ofrecerse como voluntarios, haciendo votos temporales de renuncia, para permanecer indefinidamente en el planeta como consejeros en la cuestión del desarrollo y la conservación ulteriores del plasma de vida. Dos de estos Hijos y sus doce asociados sirven actualmente en Urantia.

\section*{4. Los Portadores de Vida Melquisedeks}
\par
%\textsuperscript{(400.5)}
\textsuperscript{36:4.1} En cada sistema local de mundos habitados de todo Nebadon hay una sola esfera donde los Melquisedeks han actuado como portadores de vida. Estas moradas se conocen como los mundos \textit{midsonitos} del sistema, y en cada uno de ellos, un Hijo Melquisedek materialmente modificado se ha emparejado con una Hija seleccionada de la orden material de filiación. Las Madres Evas de estos mundos midsonitos son enviadas desde la sede del sistema que tiene la jurisdicción sobre ellos, habiendo sido elegidas por el portador de vida Melquisedek designado; son escogidas entre las numerosas voluntarias que responden al llamamiento dirigido por el Soberano del Sistema a las Hijas Materiales de su esfera.

\par
%\textsuperscript{(400.6)}
\textsuperscript{36:4.2} Los descendientes de un portador de vida Melquisedek y de una Hija Material se conocen con el nombre de \textit{midsonitarios}. El padre Melquisedek de esta raza de criaturas celestiales se marcha finalmente del planeta donde ha ejercido esta función vital excepcional, y la Madre Eva de esta orden especial de seres universales también se marcha cuando aparece la séptima generación de su descendencia planetaria. La dirección de un mundo así recae entonces sobre su hijo mayor.

\par
%\textsuperscript{(400.7)}
\textsuperscript{36:4.3} Las criaturas midsonitas viven y desempeñan sus funciones como seres reproductores en sus mundos magníficos hasta que cumplen mil años oficiales de edad, después de lo cual son trasladadas por transporte seráfico. Los midsonitarios ya no pueden reproducirse después, porque la técnica de la desmaterialización por la que pasan para ser transportados por los serafines los priva para siempre de sus prerrogativas reproductoras.

\par
%\textsuperscript{(400.8)}
\textsuperscript{36:4.4} El estado actual de estos seres difícilmente se puede considerar como mortal o inmortal, y tampoco se les puede clasificar categóricamente como humanos o divinos. Estas criaturas no están habitadas por Ajustadores, por lo que no son del todo inmortales. Pero tampoco parecen mortales; ningún midsonitario ha experimentado la muerte. Todos los midsonitarios nacidos en Nebadon siguen vivos en la actualidad, ejerciendo su actividad en sus mundos nativos, en alguna esfera intermedia, o en la esfera midsonita de Salvington situada en el grupo de mundos de los finalitarios.

\par
%\textsuperscript{(401.1)}
\textsuperscript{36:4.5} \textit{Los Mundos de los Finalitarios situados en Salvington}. Los portadores de vida Melquisedeks, al igual que las Madres Evas asociadas, van desde las esferas midsonitas del sistema a los mundos de los finalitarios del circuito de Salvington, donde sus descendientes también están destinados a reunirse.

\par
%\textsuperscript{(401.2)}
\textsuperscript{36:4.6} Debemos explicar a este respecto que el quinto grupo de siete mundos primarios del circuito de Salvington es el de los mundos de los finalitarios de Nebadon. Los hijos de los portadores de vida Melquisedeks y de las Hijas Materiales están domiciliados en el séptimo mundo de los finalitarios, la esfera midsonita de Salvington.

\par
%\textsuperscript{(401.3)}
\textsuperscript{36:4.7} Los satélites de los siete mundos primarios de los finalitarios son el punto de encuentro de las personalidades de los superuniversos y del universo central que pueden estar realizando misiones en Nebadon. Aunque los mortales ascendentes circulan libremente por todos los mundos culturales y las esferas educativas de los 490 mundos que componen la Universidad Melquisedek, hay ciertas escuelas especiales y numerosas zonas prohibidas a las que no se les permite entrar. Esto es especialmente cierto en lo que se refiere a las cuarenta y nueve esferas que están bajo la jurisdicción de los finalitarios.

\par
%\textsuperscript{(401.4)}
\textsuperscript{36:4.8} El destino de las criaturas midsonitas no se conoce en la actualidad, pero parece ser que estas personalidades se están reuniendo en el séptimo mundo finalitario como preparación para alguna eventualidad futura de la evolución del universo. Nuestras peticiones de información acerca de las razas midsonitas siempre son enviadas a los finalitarios, y los finalitarios siempre rehúsan hablar del destino de sus pupilos. A pesar de nuestra incertidumbre en cuanto al futuro de los midsonitarios, sabemos que cada universo local de Orvonton alberga un cuerpo creciente de estos seres misteriosos. Los portadores de vida Melquisedeks creen que sus hijos midsonitos serán dotados algún día por Dios Último del espíritu trascendental y eterno de la absonidad.

\section*{5. Los siete espíritus ayudantes de la mente}
\par
%\textsuperscript{(401.5)}
\textsuperscript{36:5.1} La presencia de los siete espíritus ayudantes de la mente\footnote{\textit{Espíritus ayudantes de la mente}: Is 11:2; Ap 1:4; 3:1; 4:5; 5:6.} en los mundos primitivos es la que condiciona el curso de la evolución orgánica; esto explica por qué la evolución es intencional y no accidental. Estos ayudantes representan el funcionamiento del ministerio mental del Espíritu Infinito, que se extiende hasta las órdenes inferiores de vida inteligente a través de las actividades del Espíritu Madre de un universo local. Los ayudantes son los hijos del Espíritu Madre del Universo y constituyen su ministerio personal hacia la mente material de los reinos. En cualquier momento y lugar en que se manifiesta este tipo de mente, estos espíritus están actuando de maneras diversas.

\par
%\textsuperscript{(401.6)}
\textsuperscript{36:5.2} Los siete espíritus ayudantes de la mente\footnote{\textit{Espíritus ayudantes de la mente}: Is 11:2; Ap 1:4; 3:1; 4:5; 5:6.} reciben nombres que equivalen a las designaciones siguientes: intuición, comprensión, valentía, conocimiento, consejo, adoración y sabiduría. Estos espíritus de la mente envían su influencia a todos los mundos habitados como un impulso diferencial, buscando cada uno de ellos la capacidad de recepción para manifestarse, independientemente por completo del grado de receptividad y de la oportunidad para funcionar que hayan conseguido sus compañeros.

\par
%\textsuperscript{(401.7)}
\textsuperscript{36:5.3} Los alojamientos centrales de los espíritus ayudantes, en el mundo sede de los Portadores de Vida, indican a los Portadores de Vida supervisores el alcance y la calidad del funcionamiento mental de los ayudantes en cualquier mundo y en cualquier organismo viviente dado que posea un intelecto. Estos emplazamientos de la mente unida a la vida son unos indicadores perfectos del funcionamiento mental viviente de los cinco primeros ayudantes. Pero en lo que se refiere a los espíritus ayudantes sexto y séptimo ---adoración y sabiduría--- estos alojamientos centrales sólo indican un funcionamiento cualitativo. La actividad cuantitativa del ayudante de la adoración y del ayudante de la sabiduría se registra en la presencia directa de la Ministra Divina en Salvington, y es una experiencia personal del Espíritu Madre del Universo.

\par
%\textsuperscript{(402.1)}
\textsuperscript{36:5.4} Los siete espíritus ayudantes de la mente acompañan siempre a los Portadores de Vida a un nuevo planeta, pero no deben ser considerados como entidades; se parecen más a unos circuitos. Los espíritus de los siete ayudantes del universo no funcionan como personalidades separadamente de la presencia universal de la Ministra Divina; son de hecho un nivel de conciencia de la Ministra Divina, y siempre están subordinados a la acción y a la presencia de su madre creadora.

\par
%\textsuperscript{(402.2)}
\textsuperscript{36:5.5} Carecemos de palabras para denominar adecuadamente a estos siete espíritus ayudantes de la mente. Son los ministros de los niveles inferiores de la mente experiencial y, en el orden de los logros evolutivos, se pueden describir como sigue:

\par
%\textsuperscript{(402.3)}
\textsuperscript{36:5.6} 1. \textit{El espíritu de intuición} ---de percepción rápida, los instintos reflejos físicos primitivos e inherentes, la dotación direccional y otros instintos de conservación que poseen todas las creaciones mentales; el único ayudante que funciona tan ampliamente en las órdenes inferiores de vida animal, y el único que establece un extenso contacto funcional con los niveles no enseñables de la mente maquinal.

\par
%\textsuperscript{(402.4)}
\textsuperscript{36:5.7} 2. \textit{El espíritu de comprensión}\footnote{\textit{Espíritu de comprensión}: Ex 31:3; 35:31; Job 32:8; Eclo 1:4; Is 11:2; Dn 5:11-12,14.} ---el impulso de coordinación, la asociación espontánea y aparentemente automática de las ideas. Es el don de coordinar el conocimiento adquirido, el fenómeno del razonamiento inmediato, del juicio rápido y de la decisión pronta.

\par
%\textsuperscript{(402.5)}
\textsuperscript{36:5.8} 3. \textit{El espíritu de valentía} ---el don de la fidelidad--- en los seres personales, la base para adquirir el carácter y la raíz intelectual del vigor moral y de la valentía espiritual. Cuando está iluminado por los hechos e inspirado por la verdad, se convierte en el secreto del impulso de la ascensión evolutiva a través de los canales de una dirección autónoma inteligente y concienzuda.

\par
%\textsuperscript{(402.6)}
\textsuperscript{36:5.9} 4. \textit{El espíritu de conocimiento}\footnote{\textit{Espíritu de conocimiento}: Ex 31:3; 35:31; Is 11:2; Ef 1:17.} ---la curiosidad como madre de la aventura y del descubrimiento, el espíritu científico; el guía y el fiel asociado de los espíritus de valentía y de consejo; el impulso de dirigir los dones de la valentía hacia caminos de crecimiento útiles y progresivos.

\par
%\textsuperscript{(402.7)}
\textsuperscript{36:5.10} 5. \textit{El espíritu de consejo}\footnote{\textit{Espíritu de consejo}: Is 11:2.} ---el impulso social, el don de la cooperación con la especie; la capacidad de las criaturas volitivas para armonizarse con sus compañeros, el origen del instinto gregario entre las criaturas más inferiores.

\par
%\textsuperscript{(402.8)}
\textsuperscript{36:5.11} 6. \textit{El espíritu de adoración}\footnote{\textit{Espíritu de adoración}: Is 11:2.} ---el impulso religioso, la primera pulsión diferencial que separa a las criaturas mentales en las dos clases fundamentales de la existencia mortal. El espíritu de adoración distingue para siempre al animal con el que está asociado de las criaturas sin alma dotadas de mente. La adoración es el distintivo de la candidatura a la ascensión espiritual.

\par
%\textsuperscript{(402.9)}
\textsuperscript{36:5.12} 7. \textit{El espíritu de sabiduría}\footnote{\textit{Espíritu de sabiduría}: Ex 31:3; 35:31; Eclo 1:4ff; Sab 1:6ff; Is 11:2; Lc 2:40; Ef 1:17.} ---la tendencia inherente de todas las criaturas morales hacia un avance evolutivo ordenado y progresivo. Éste es el ayudante más elevado, el espíritu que coordina y articula el trabajo de todos los demás. Este espíritu es el secreto de ese impulso innato de las criaturas mentales que inicia y mantiene el programa práctico y eficaz de la escala ascendente de la existencia; ese don de los seres vivientes que da cuenta de su inexplicable capacidad para sobrevivir y para utilizar, en la supervivencia, la coordinación de todas sus experiencias pasadas y de todas sus oportunidades presentes para adquirir la totalidad de lo que los otros seis ministros mentales pueden movilizar en la mente del organismo interesado. La sabiduría es la cumbre de la realización intelectual. La sabiduría es la meta de una existencia puramente mental y moral.

\par
%\textsuperscript{(403.1)}
\textsuperscript{36:5.13} Los espíritus ayudantes de la mente crecen en experiencia pero nunca se vuelven personales. Evolucionan en su funcionamiento, y el funcionamiento de los cinco primeros en las órdenes animales es hasta cierto punto esencial para que los siete puedan funcionar como intelecto humano. Esta relación con los animales hace que los ayudantes sean más eficaces en la práctica como mente humana; así pues, los animales son hasta cierto punto indispensables para la evolución intelectual del hombre así como para su evolución física.

\par
%\textsuperscript{(403.2)}
\textsuperscript{36:5.14} Estos ayudantes mentales del Espíritu Madre de un universo local están relacionados con la vida de las criaturas inteligentes poco más o menos como los centros del poder y los controladores físicos están relacionados con las fuerzas no vivientes del universo. Efectúan un servicio inapreciable en los circuitos mentales de los mundos habitados, y colaboran de manera eficaz con los Controladores Físicos Maestros, los cuales sirven también como controladores y directores de los niveles mentales preayudantes, los niveles de la mente no enseñable o maquinal.

\par
%\textsuperscript{(403.3)}
\textsuperscript{36:5.15} La mente viviente anterior a la aparición de la capacidad para aprender por experiencia pertenece al dominio de servicio de los Controladores Físicos Maestros. Antes de que la mente de las criaturas adquiera la capacidad para reconocer la divinidad y adorar la Deidad, pertenece al dominio exclusivo de los espíritus ayudantes\footnote{\textit{Espíritus ayudantes de la mente}: Sab 1:6-7.}. Con la aparición de la reacción espiritual del intelecto de las criaturas, estas mentes creadas se vuelven de inmediato supermentales, y son incorporadas instantáneamente en el circuito de los ciclos espirituales del Espíritu Madre del universo local.

\par
%\textsuperscript{(403.4)}
\textsuperscript{36:5.16} Los espíritus ayudantes de la mente no están relacionados directamente de ninguna manera con el funcionamiento diverso y sumamente espiritual del espíritu de la presencia personal de la Ministra Divina, el Espíritu Santo de los mundos habitados; pero son funcionalmente anteriores a la aparición de este mismo espíritu en el hombre evolutivo, y preparatorios para ella. Los ayudantes proporcionan al Espíritu Madre del Universo un contacto variado con las criaturas materiales vivientes de un universo local, y un control sobre ellas, pero no producen repercusiones en el Ser Supremo cuando actúan en los niveles de la prepersonalidad.

\par
%\textsuperscript{(403.5)}
\textsuperscript{36:5.17} La mente no espiritual es o una manifestación de la energía espiritual, o un fenómeno de la energía física. Incluso la mente humana, la mente personal, no posee cualidades de supervivencia si no está identificada con el espíritu. La mente es un don de la divinidad, pero no es inmortal cuando funciona sin la perspicacia espiritual, ni cuando está desprovista de la capacidad para adorar y anhelar la supervivencia.

\section*{6. Las fuerzas vivientes}
\par
%\textsuperscript{(403.6)}
\textsuperscript{36:6.1} La vida es a la vez mecánica y vitalista ---material y espiritual\footnote{\textit{La vida, tanto material como espiritual}: Sal 104:30; Ec 12:7; Jn 1:4; Hch 17:25; 1 Ti 6:13.}. Los físicos y los químicos de Urantia progresarán constantemente en su comprensión de las formas protoplásmicas de la vida vegetal y animal, pero nunca serán capaces de producir organismos vivientes. La vida es algo que difiere de todas las manifestaciones de la energía; incluso la vida material de las criaturas físicas no es inherente a la materia.

\par
%\textsuperscript{(403.7)}
\textsuperscript{36:6.2} Las cosas materiales pueden disfrutar de una existencia independiente, pero la vida sólo surge de la vida. La mente sólo puede proceder de una mente preexistente. El espíritu sólo tiene su origen en unos antecesores espirituales. La criatura puede producir las formas de la vida, pero sólo una personalidad creadora o una fuerza creativa puede proporcionar la chispa activadora viviente.

\par
%\textsuperscript{(404.1)}
\textsuperscript{36:6.3} Los Portadores de Vida pueden organizar las formas materiales o los modelos físicos de los seres vivientes, pero el Espíritu aporta la chispa inicial de vida y concede el don de la mente. Incluso las formas vivientes de vida experimental que los Portadores de Vida organizan en sus mundos de Salvington siempre están desprovistas de los poderes reproductores. Cuando las fórmulas de la vida y los modelos vitales están correctamente ensamblados y adecuadamente organizados, la presencia de un Portador de Vida es suficiente para iniciar la vida, pero todos estos organismos vivientes carecen de dos atributos esenciales ---el don de la mente y los poderes de reproducción. La mente animal y la mente humana son dones del Espíritu Madre del universo local actuando a través de los siete espíritus ayudantes de la mente, mientras que la capacidad de reproducción de las criaturas es la concesión específica y personal del Espíritu del Universo al plasma vital ancestral inaugurado por los Portadores de Vida.

\par
%\textsuperscript{(404.2)}
\textsuperscript{36:6.4} Cuando los Portadores de Vida han diseñado los modelos de vida, después de haber organizado los sistemas de energía, un fenómeno adicional debe producirse; el <<soplo de vida>>\footnote{\textit{Soplo de vida}: Gn 2:7; Gn 6:17; Gn 7:15,22.} ha de conferirse a esas formas sin vida. Los Hijos de Dios pueden construir las formas de vida, pero el Espíritu de Dios es el que aporta realmente la chispa vital. Y cuando la vida así conferida se extingue, el cuerpo material que queda se convierte una vez más en materia muerta. Cuando la vida otorgada se agota, el cuerpo regresa al seno del universo material de donde fue tomado por los Portadores de Vida para servir como vehículo transitorio para ese don de vida que transmitieron a esa asociación visible de energía-materia.

\par
%\textsuperscript{(404.3)}
\textsuperscript{36:6.5} La vida otorgada a las plantas y a los animales por los Portadores de Vida no regresa a los Portadores de Vida después de morir la planta o el animal. La vida que sale de esos seres vivientes no posee ni identidad ni personalidad; no sobrevive individualmente a la muerte. Durante su existencia y el tiempo de su estancia en el cuerpo material, ha sufrido un cambio; ha experimentado una evolución energética y sólo sobrevive como parte de las fuerzas cósmicas del universo; no sobrevive como vida individual. La supervivencia de las criaturas mortales está basada enteramente en la evolución de un alma inmortal dentro de la mente mortal.

\par
%\textsuperscript{(404.4)}
\textsuperscript{36:6.6} Hablamos de la vida como de una <<energía>> y como de una <<fuerza>>, pero no es en realidad ninguna de las dos. La energía-fuerza es sensible de diversas maneras a la gravedad; pero la vida no lo es. El modelo tampoco es sensible a la gravedad, pues es una configuración de energías que ya ha cumplido con todas sus obligaciones reactivas hacia la gravedad. La vida, como tal, representa la animación de un sistema de energía ---material, mental o espiritual--- configurado en un modelo o separado de otra manera.

\par
%\textsuperscript{(404.5)}
\textsuperscript{36:6.7} Hay algunas cosas relacionadas con la elaboración de la vida en los planetas evolutivos que no están del todo claras para nosotros. Comprendemos plenamente la organización física de las fórmulas electroquímicas de los Portadores de Vida, pero no entendemos por completo la naturaleza y la fuente de la \textit{chispa que activa la vida}. Sabemos que la vida proviene del Padre, pasa por el Hijo y fluye \textit{a través} del Espíritu. Es muy probable que los Espíritus Maestros sean el canal séptuple del río de vida que se derrama sobre toda la creación. Pero no comprendemos la técnica por medio de la cual el Espíritu Maestro supervisor participa en el episodio inicial de conferir la vida en un nuevo planeta. Estamos seguros de que los Ancianos de los Días también participan de alguna manera en esta inauguración de la vida en un nuevo mundo, pero ignoramos por completo la naturaleza de dicha participación. Sabemos que el Espíritu Madre del Universo vitaliza realmente los modelos sin vida y confiere a ese plasma activado las prerrogativas de la reproducción del organismo. Observamos que estas tres personalidades constituyen los niveles de Dios Séptuple, y son a veces denominadas los Creadores Supremos del tiempo y del espacio; pero por lo demás, sabemos poco más que los mortales de Urantia ---simplemente que el concepto es inherente al Padre, la expresión al Hijo y la realización de la vida al Espíritu.

\par
%\textsuperscript{(405.1)}
\textsuperscript{36:6.8} [Redactado por un Hijo Vorondadek estacionado en Urantia como observador, y que actúa en esta calidad a petición del Jefe Melquisedek del Cuerpo Revelador Supervisor.]


\chapter{Documento 37. Las personalidades del universo local}
\par
%\textsuperscript{(406.1)}
\textsuperscript{37:0.1} A LA cabeza de todas las personalidades de Nebadon se encuentra Miguel, el Hijo Creador y Maestro, el padre y soberano del universo. Su coordinada en divinidad y su complementaria en atributos creativos es el Espíritu Madre del universo local, la Ministra Divina de Salvington. Y estos creadores son, en un sentido muy literal, el Padre-Hijo y el Espíritu-Madre de todas las criaturas nativas de Nebadon.

\par
%\textsuperscript{(406.2)}
\textsuperscript{37:0.2} Los documentos anteriores han tratado de las órdenes creadas de filiación; las narraciones siguientes describirán a los espíritus ministrantes y a las órdenes ascendentes de filiación. Este documento se ocupa principalmente de un grupo intermedio, los Ayudantes del Universo, pero también examinará brevemente algunos de los espíritus más elevados que están estacionados en Nebadon y ciertas órdenes de ciudadanos permanentes del universo local.

\section*{1. Los Ayudantes del Universo}
\par
%\textsuperscript{(406.3)}
\textsuperscript{37:1.1} Muchas de las órdenes singulares agrupadas generalmente en esta categoría no han sido reveladas, pero los Ayudantes del Universo, tal como se presentan en estos documentos, incluyen a las siete órdenes siguientes:

\par
%\textsuperscript{(406.4)}
\textsuperscript{37:1.2} 1. Las Radiantes Estrellas Matutinas\footnote{\textit{Radiante Estrella Matutina}: Ap 22:16.}.

\par
%\textsuperscript{(406.5)}
\textsuperscript{37:1.3} 2. Las Brillantes Estrellas Vespertinas.

\par
%\textsuperscript{(406.6)}
\textsuperscript{37:1.4} 3. Los Arcángeles.

\par
%\textsuperscript{(406.7)}
\textsuperscript{37:1.5} 4. Los Asistentes Altísimos.

\par
%\textsuperscript{(406.8)}
\textsuperscript{37:1.6} 5. Los Altos Comisionados.

\par
%\textsuperscript{(406.9)}
\textsuperscript{37:1.7} 6. Los Supervisores Celestiales.

\par
%\textsuperscript{(406.10)}
\textsuperscript{37:1.8} 7. Los Educadores de los Mundos de las Mansiones.

\par
%\textsuperscript{(406.11)}
\textsuperscript{37:1.9} De la primera orden de Ayudantes del Universo, las Radiantes Estrellas Matutinas, sólo hay un representante en cada universo local, y es el primogénito de todas las criaturas nativas de ese universo local. A la Radiante Estrella Matutina de nuestro universo se le conoce con el nombre de Gabriel de Salvington. Es el jefe ejecutivo de todo Nebadon, y actúa como representante personal del Hijo Soberano y como portavoz de su consorte creativa.

\par
%\textsuperscript{(406.12)}
\textsuperscript{37:1.10} Durante los primeros tiempos de Nebadon, Gabriel trabajó totalmente solo con Miguel y el Espíritu Creativo. A medida que el universo creció y que los problemas administrativos se multiplicaron, se le proporcionó un estado mayor personal de asistentes no revelados, y este grupo aumentó con el tiempo mediante la creación del cuerpo de Estrellas Vespertinas de Nebadon.

\section*{2. Las Brillantes Estrellas Vespertinas}
\par
%\textsuperscript{(407.1)}
\textsuperscript{37:2.1} Los Melquisedeks proyectaron estas brillantes criaturas y luego fueron traídas a la existencia por el Hijo Creador y el Espíritu Creativo. Sirven en muchas ocupaciones, pero principalmente como agentes de enlace de Gabriel, el jefe ejecutivo del universo local. Uno o más de estos seres actúa como representante suyo en la capital de cada constelación y de cada sistema de Nebadon.

\par
%\textsuperscript{(407.2)}
\textsuperscript{37:2.2} Como jefe ejecutivo de Nebadon, Gabriel es el presidente de oficio de la mayoría de los cónclaves de Salvington, o asiste como observador a ellos, y sucede a menudo que mil de estos cónclaves celebran sus sesiones simultáneamente. Las Brillantes Estrellas Vespertinas representan a Gabriel en esas ocasiones; él no puede estar en dos lugares a la vez, y estos superángeles compensan esta limitación. Prestan un servicio análogo para el cuerpo de los Hijos Instructores Trinitarios.

\par
%\textsuperscript{(407.3)}
\textsuperscript{37:2.3} Aunque Gabriel está personalmente ocupado con sus deberes administrativos, se mantiene en contacto con todas las otras fases de la vida y de los asuntos del universo a través de las Brillantes Estrellas Vespertinas. Éstas siempre le acompañan en sus giras planetarias y van con frecuencia en misiones especiales a los planetas individuales como representantes personales suyos. Durante estas misiones, a veces se les ha conocido como <<el ángel del Señor>>\footnote{\textit{El ángel del Señor}: Gn 16:7-11; Ex 3:2; Mt 1:20,24; 2:13,19.}. Van a menudo a Uversa para representar a la Radiante Estrella Matutina ante los tribunales y las asambleas de los Ancianos de los Días, pero raras veces viajan más allá de los confines de Orvonton.

\par
%\textsuperscript{(407.4)}
\textsuperscript{37:2.4} Las Brillantes Estrellas Vespertinas forman una orden doble de carácter único, pues algunos de sus miembros lo son por dignidad creada y otros lo han conseguido mediante el servicio. En Nebadon, el cuerpo de estos superángeles asciende actualmente a 13.641 miembros. Hay 4.832 de dignidad creada, mientras que 8.809 son espíritus ascendentes que han alcanzado esta meta de servicio elevado. Muchas de estas Estrellas Vespertinas ascendentes empezaron su carrera universal como serafines; otras han ascendido desde los niveles no revelados de la vida de las criaturas. Como meta a alcanzar, este elevado cuerpo nunca está cerrado para los candidatos a la ascensión mientras un universo no se establezca en la luz y la vida.

\par
%\textsuperscript{(407.5)}
\textsuperscript{37:2.5} Los dos tipos de Brillantes Estrellas Vespertinas son fácilmente visibles para las personalidades morontiales y para ciertos tipos de seres materiales supermortales. Los seres creados de esta interesante y polifacética orden poseen una fuerza espiritual que se puede manifestar con independencia de su presencia personal.

\par
%\textsuperscript{(407.6)}
\textsuperscript{37:2.6} El jefe de estos superángeles es Gavalia, el primogénito de esta orden en Nebadon. Desde que Cristo Miguel regresó de su donación triunfal en Urantia, Gavalia ha estado asignado al ministerio de los mortales ascendentes, y durante los últimos mil novecientos años urantianos su asociado Galantia ha mantenido su sede en Jerusem, donde pasa casi la mitad de su tiempo. Galantia es el primer superángel ascendente que ha alcanzado esta elevada posición.

\par
%\textsuperscript{(407.7)}
\textsuperscript{37:2.7} Para las Brillantes Estrellas Vespertinas no existe ninguna agrupación u organización en compañías más que la de su asociación habitual en parejas a fin de realizar numerosas funciones. No se les destina a muchas misiones relacionadas con la carrera ascendente de los mortales, pero cuando se las encargan, nunca actúan solos. Siempre trabajan en parejas ---uno es un ser creado y el otro una Estrella Vespertina ascendente.

\par
%\textsuperscript{(407.8)}
\textsuperscript{37:2.8} Uno de los deberes elevados de las Estrellas Vespertinas consiste en acompañar a los Hijos donadores Avonales en sus misiones planetarias, tal como Gabriel acompañó a Miguel durante su donación en Urantia. Los dos superángeles acompañantes son las personalidades de mayor categoría de estas misiones, y sirven como comandantes conjuntos de los arcángeles y de todos los otros seres asignados a estas empresas. El comandante más antiguo de estos superángeles es el que, a la edad y en el momento oportunos, le dice al Hijo Avonal donador: <<Ocúpate de los asuntos de tu hermano>>.

\par
%\textsuperscript{(408.1)}
\textsuperscript{37:2.9} Unas parejas similares de estos superángeles son destinadas al cuerpo planetario de los Hijos Instructores Trinitarios que trabajan para establecer la era espiritual naciente, o posterior a la donación, en un mundo habitado. En estas misiones, las Estrellas Vespertinas sirven de enlace entre los mortales del reino y el cuerpo invisible de los Hijos Instructores.

\par
%\textsuperscript{(408.2)}
\textsuperscript{37:2.10} \textit{Los Mundos de las Estrellas Vespertinas}. El sexto grupo de siete mundos de Salvington y sus cuarenta y dos satélites tributarios están destinados a la administración de las Brillantes Estrellas Vespertinas. Las órdenes creadas de estos superángeles presiden los siete mundos primarios, mientras que los satélites tributarios están administrados por las Estrellas Vespertinas ascendentes.

\par
%\textsuperscript{(408.3)}
\textsuperscript{37:2.11} Los satélites de los tres primeros mundos están consagrados a las escuelas de los Hijos Instructores y de las Estrellas Vespertinas, dedicadas a las personalidades espirituales del universo local. Los tres grupos siguientes contienen escuelas conjuntas similares consagradas a la formación de los mortales ascendentes. Los satélites del séptimo mundo están reservados para las deliberaciones trinas de los Hijos Instructores, las Estrellas Vespertinas y los finalitarios. Durante los últimos tiempos, estos superángeles han estado estrechamente identificados con el trabajo del Cuerpo de la Finalidad en el universo local, y han estado asociados durante mucho tiempo con los Hijos Instructores. Existe una conexión de un poder y de una importancia extraordinarios entre las Estrellas Vespertinas y los Mensajeros de Gravedad vinculados a los grupos de trabajo finalitarios. El séptimo mundo primario mismo está reservado a aquellos asuntos no revelados que serán propios de las relaciones futuras que existirán entre los Hijos Instructores, los finalitarios y las Estrellas Vespertinas, después de que la manifestación superuniversal de la personalidad de Dios Supremo haya emergido por completo.

\section*{3. Los Arcángeles}
\par
%\textsuperscript{(408.4)}
\textsuperscript{37:3.1} Los arcángeles son la progenitura del Hijo Creador y del Espíritu Madre del Universo. Son el tipo más elevado de seres espirituales superiores engendrados en grandes cantidades en un universo local, y en el momento del último registro había cerca de ochocientos mil en Nebadon.

\par
%\textsuperscript{(408.5)}
\textsuperscript{37:3.2} Los Arcángeles son uno de los pocos grupos de personalidades del universo local que no están normalmente bajo la jurisdicción de Gabriel. No están relacionados de ninguna manera con la administración rutinaria del universo, estando dedicados a la tarea de la supervivencia de las criaturas y a fomentar la carrera ascendente de los mortales del tiempo y del espacio. Aunque habitualmente no están sujetos a la dirección de la Radiante Estrella Matutina, los arcángeles actúan a veces por autoridad suya. También colaboran con otros Ayudantes del Universo tales como las Estrellas Vespertinas, como queda ilustrado en ciertas actividades descritas en la narración sobre el transplante de la vida en vuestro mundo.

\par
%\textsuperscript{(408.6)}
\textsuperscript{37:3.3} El cuerpo de los arcángeles de Nebadon está dirigido por el primogénito de esta orden y, en tiempos más recientes, una sede divisionaria de arcángeles se ha mantenido en Urantia. Este hecho inhabitual es el que atrae rápidamente la atención de los visitantes estudiantiles procedentes del exterior de Nebadon. Entre las primeras cosas que observan en las operaciones intrauniversales se encuentra el descubrimiento de que muchas actividades ascendentes de las Brillantes Estrellas Vespertinas están dirigidas desde la capital de un sistema local, el de Satania. Al profundizar en su examen descubren que ciertas actividades arcangélicas están dirigidas desde un pequeño mundo habitado, aparentemente insignificante, llamado Urantia. Luego sigue la revelación de que Miguel se donó en Urantia, y estos visitantes se interesan de inmediato vivamente por vosotros y por vuestra humilde esfera.

\par
%\textsuperscript{(409.1)}
\textsuperscript{37:3.4} ¿Captáis la importancia del hecho de que vuestro humilde y confuso planeta se ha convertido en una sede divisionaria de la administración del universo y de la dirección de ciertas actividades arcangélicas relacionadas con el programa de la ascensión al Paraíso? Esto presagia indudablemente la futura concentración de otras actividades ascendentes en el mundo donde Miguel se donó, y confiere una importancia enorme y solemne a la promesa personal del Maestro: <<Regresaré>>\footnote{\textit{Regresaré de nuevo}: Mt 24:27,37,42; Mt 25:13; Mc 13:32-33; Jn 14:3,28.}.

\par
%\textsuperscript{(409.2)}
\textsuperscript{37:3.5} Los arcángeles están asignados en general al servicio y al ministerio de la orden de filiación Avonal, pero no lo hacen hasta después de haber pasado por una extensa formación preliminar en todas las fases del trabajo de los diversos espíritus ministrantes. Un cuerpo de cien arcángeles acompaña a cada Hijo Paradisiaco que se dona en un mundo habitado, y le están temporalmente asignados mientras dura esa donación. Si el Hijo Magistral se convirtiera en el gobernante temporal del planeta, estos arcángeles actuarían como jefes directores de toda la vida celestial de esa esfera.

\par
%\textsuperscript{(409.3)}
\textsuperscript{37:3.6} Dos arcángeles más antiguos siempre son asignados como ayudantes personales a un Avonal Paradisiaco en todas sus misiones planetarias, ya se trate de acciones judiciales, de misiones magistrales o de encarnaciones donadoras. Cuando este Hijo Paradisiaco ha terminado el juicio de un reino y se realiza el llamamiento de los muertos de acuerdo con los registros (la llamada resurrección), es literalmente cierto que los guardianes seráficos de las personalidades dormidas responden a <<la voz del arcángel>>\footnote{\textit{La voz del arcángel}: 1 Ts 4:16.}. Un arcángel acompañante es el que promulga el llamamiento nominal al final de una dispensación. Es el arcángel de la resurrección, llamado a veces el <<arcángel de Miguel>>\footnote{\textit{Arcángel de Miguel}: Mt 27:52-53; Jud 1:9.}.

\par
%\textsuperscript{(409.4)}
\textsuperscript{37:3.7} \textit{Los Mundos de los Arcángeles}. El séptimo grupo de mundos que rodea a Salvington, con sus satélites asociados, está asignado a los arcángeles. La esfera número uno y sus seis satélites tributarios están ocupados por los conservadores de los registros de la personalidad. Este inmenso cuerpo de registradores se ocupa de mantener en orden la historia de cada mortal del tiempo desde el momento de su nacimiento, pasando por su carrera universal, hasta que esa persona o bien deja Salvington para incorporarse al régimen superuniversal, o es <<tachada de la existencia registrada>>\footnote{\textit{Tachada de los registros}: Sal 69:28.} por mandato de los Ancianos de los Días.

\par
%\textsuperscript{(409.5)}
\textsuperscript{37:3.8} En estos mundos es donde los informes sobre la personalidad y las garantías de la identidad son clasificados, archivados y conservados durante ese período que media entre la muerte física y el momento de la repersonalización, la resurrección después de la muerte.

\section*{4. Los Asistentes Altísimos}
\par
%\textsuperscript{(409.6)}
\textsuperscript{37:4.1} Los Asistentes Altísimos son un grupo de seres voluntarios que tienen su origen fuera del universo local, y que son nombrados temporalmente como representantes u observadores del universo central y de los superuniversos ante las creaciones locales. Su número varía constantemente, pero siempre se eleva a muchos millones.

\par
%\textsuperscript{(409.7)}
\textsuperscript{37:4.2} De vez en cuando nos beneficiamos así del ministerio y de la ayuda de unos seres de origen paradisiaco tales como los Perfeccionadores de la Sabiduría, los Consejeros Divinos, los Censores Universales, los Espíritus Inspirados Trinitarios, los Hijos Trinitizados, los Mensajeros Solitarios, los supernafines, los seconafines, los terciafines y otros ministros misericordiosos que residen con nosotros con el objeto de ayudar a nuestras personalidades nativas en su esfuerzo por conducir a todo Nebadon hacia una armonía más plena con las ideas de Orvonton y los ideales del Paraíso.

\par
%\textsuperscript{(410.1)}
\textsuperscript{37:4.3} Cualquiera de estos seres puede estar sirviendo voluntariamente en Nebadon y sin embargo estar técnicamente fuera de nuestra jurisdicción, pero cuando actúan por haber sido nombrados para ello, estas personalidades de los superuniversos y del universo central no están totalmente exentas de las reglamentaciones del universo local donde residen, aunque continúan ejerciendo como representantes de los universos superiores y trabajando de acuerdo con las instrucciones que constituyen su misión en nuestro reino. Su sede general está situada en el sector del Unión de los Días en Salvington y trabajan en Nebadon sometidos a la supervisión suprema de este embajador de la Trinidad del Paraíso. Cuando sirven en grupos independientes, estas personalidades de los reinos superiores se gobiernan habitualmente de forma autónoma, pero cuando sirven a petición de los interesados, a menudo se colocan voluntariamente bajo la jurisdicción total de los directores que supervisan los reinos donde actúan por encargo.

\par
%\textsuperscript{(410.2)}
\textsuperscript{37:4.4} Los Asistentes Altísimos sirven en los universos locales y en las constelaciones, pero no están directamente vinculados a los gobiernos de los sistemas o de los planetas. Sin embargo pueden ejercer su actividad en cualquier parte del universo local y ser asignados a cualquier fase de la actividad de Nebadon ---administrativa, ejecutiva, educativa u otras.

\par
%\textsuperscript{(410.3)}
\textsuperscript{37:4.5} La mayor parte de este cuerpo se ha reclutado para ayudar a las personalidades paradisiacas de Nebadon ---el Unión de los Días, el Hijo Creador, los Fieles de los Días, los Hijos Magistrales y los Hijos Instructores Trinitarios. En el tratamiento de los asuntos de una creación local, de vez en cuando es sabio ocultar temporalmente ciertos detalles al conocimiento de casi todas las personalidades nativas de ese universo local. Ciertos planes avanzados y ciertas decisiones complejas son también mejor captados y más plenamente comprendidos por el cuerpo más maduro y previsor de los Asistentes Altísimos, y es en estas situaciones y en muchas otras en las que son tan extremadamente útiles para los gobernantes y los administradores del universo.

\section*{5. Los Altos Comisionados}
\par
%\textsuperscript{(410.4)}
\textsuperscript{37:5.1} Los Altos Comisionados son mortales ascendentes que han fusionado con el Espíritu; no están fusionados con el Ajustador. Comprendéis bastante bien la carrera de la ascensión universal de un candidato mortal a la fusión con el Ajustador, pues ése es el alto destino en perspectiva para todos los mortales de Urantia desde la donación de Cristo Miguel. Pero éste no es el destino exclusivo de todos los mortales de las épocas anteriores a la donación en los mundos como el vuestro, y existe otro tipo de mundo cuyos habitantes nunca están permanentemente habitados por Ajustadores del Pensamiento. Esos mortales nunca se unen de manera permanente con un Monitor de Misterio donado desde el Paraíso; sin embargo, los Ajustadores sí habitan en ellos transitoriamente, sirviendo como guías y modelos mientras dura la vida en la carne. Durante esa estancia temporal, favorecen la evolución de un alma inmortal exactamente igual que lo hacen en aquellos seres con quienes esperan fusionar, pero cuando la carrera mortal ha terminado, se despiden eternamente de las criaturas con quienes han estado temporalmente asociados.

\par
%\textsuperscript{(410.5)}
\textsuperscript{37:5.2} Las almas sobrevivientes de este tipo alcanzan la inmortalidad mediante la fusión eterna con un fragmento individualizado del espíritu del Espíritu Madre del universo local. No forman un grupo numeroso, al menos en Nebadon. En los mundos de las mansiones encontraréis a estos mortales fusionados con el Espíritu y fraternizaréis con ellos mientras ascienden con vosotros el camino del Paraíso hasta llegar a Salvington, donde se detienen. Algunos de ellos pueden ascender posteriormente hasta niveles universales superiores, pero la mayoría permanecerá para siempre al servicio del universo local; como clase, no están destinados a alcanzar el Paraíso.

\par
%\textsuperscript{(411.1)}
\textsuperscript{37:5.3} Como no están fusionados con un Ajustador, nunca llegarán a ser finalitarios, pero se integrarán finalmente en el Cuerpo de la Perfección del universo local. Habrán obedecido en espíritu al mandato del Padre: <<Sed perfectos>>\footnote{\textit{Sed perfectos}: Gn 17:1; 1 Re 8:61; Lv 19:2; Dt 18:13; Mt 5:48; 2 Co 13:11; Stg 1:4; 1 P 1:16.}.

\par
%\textsuperscript{(411.2)}
\textsuperscript{37:5.4} Después de alcanzar el Cuerpo de la Perfección de Nebadon, los ascendentes fusionados con el Espíritu pueden aceptar misiones como Ayudantes del Universo, siendo ésta una de las vías que tienen abiertas para continuar creciendo experiencialmente. Así se vuelven candidatos a ser nombrados para el elevado servicio de interpretar los puntos de vista de las criaturas evolutivas de los mundos materiales ante las autoridades celestiales del universo local.

\par
%\textsuperscript{(411.3)}
\textsuperscript{37:5.5} Los Altos Comisionados empiezan su servicio en los planetas como comisionados de las razas. En esta función interpretan los puntos de vista, y describen las necesidades, de las diversas razas humanas. Están dedicados de manera suprema al bienestar de las razas mortales, de las cuales son portavoces, tratando siempre de conseguir para ellas misericordia, justicia y un trato equitativo en todas sus relaciones con los otros pueblos. Los comisionados de las razas actúan en una serie interminable de crisis planetarias, y sirven como expresión articulada de grupos enteros de mortales que luchan.

\par
%\textsuperscript{(411.4)}
\textsuperscript{37:5.6} Después de una larga experiencia solucionando problemas en los mundos habitados, estos comisionados de las razas son ascendidos a niveles de funcionamiento superiores, alcanzando finalmente el estado de Altos Comisionados del universo local, y en él. El último registro indicaba que había poco más de mil millones y medio de estos Altos Comisionados en Nebadon. Estos seres no son finalitarios, pero son seres ascendentes con una larga experiencia y prestan un gran servicio a su universo nativo.

\par
%\textsuperscript{(411.5)}
\textsuperscript{37:5.7} A estos comisionados los encontramos invariablemente en todos los tribunales de justicia, desde los más humildes hasta los más elevados. No es que participen en los procesos de la justicia, sino que actúan como amigos de los tribunales, asesorando a los magistrados que presiden respecto a los antecedentes, el entorno y la naturaleza inherente de los implicados en el juicio.

\par
%\textsuperscript{(411.6)}
\textsuperscript{37:5.8} Los Altos Comisionados están vinculados a las diversas huestes de mensajeros del espacio, y siempre lo están a los espíritus ministrantes del tiempo. Se les encuentra en los programas de las diversas asambleas universales, y estos mismos comisionados con sabiduría humana siempre forman parte de las misiones de los Hijos de Dios en los mundos del espacio.

\par
%\textsuperscript{(411.7)}
\textsuperscript{37:5.9} Cada vez que la equidad y la justicia exigen que se comprenda cómo una política o un procedimiento previstos podría afectar a las razas evolutivas del tiempo, estos comisionados están disponibles para presentar sus recomendaciones; siempre están presentes para hablar en nombre de aquellos que no pueden estar presentes para expresarse por sí mismos.

\par
%\textsuperscript{(411.8)}
\textsuperscript{37:5.10} \textit{Los Mundos de los Mortales fusionados con el Espíritu}. El octavo grupo compuesto por siete mundos primarios y sus satélites tributarios, en el circuito de Salvington, es propiedad exclusiva de los mortales de Nebadon fusionados con el Espíritu. Los mortales ascendentes fusionados con el Ajustador no están relacionados con estos mundos, salvo para disfrutar de muchas estancias agradables y beneficiosas como huéspedes invitados de los residentes fusionados con el Espíritu.

\par
%\textsuperscript{(411.9)}
\textsuperscript{37:5.11} Estos mundos son la residencia permanente de los supervivientes fusionados con el Espíritu, salvo para aquellos pocos que alcanzan Uversa y el Paraíso. Esta limitación deliberada a la ascensión de los mortales resulta beneficiosa para los universos locales, pues asegura la retención de una población permanente evolucionada cuya experiencia creciente continuará aumentando la estabilización y la diversificación futuras de la administración del universo local. Puede ser que estos seres no alcancen el Paraíso, pero consiguen una sabiduría experiencial en el dominio de los problemas de Nebadon que sobrepasa por completo la que pueden alcanzar los ascendentes transitorios. Y estas almas sobrevivientes continúan como combinaciones únicas de lo humano y de lo divino, siendo cada vez más capaces de unir los puntos de vista de estos dos niveles ampliamente separados, y de presentar este doble punto de vista con una sabiduría cada vez mayor.

\section*{6. Los Supervisores Celestiales}
\par
%\textsuperscript{(412.1)}
\textsuperscript{37:6.1} El sistema educativo de Nebadon está administrado conjuntamente por los Hijos Instructores Trinitarios y el cuerpo de enseñantes Melquisedeks, pero los Supervisores Celestiales llevan a cabo una gran parte del trabajo destinado a mantenerlo y a fortalecerlo. Estos seres forman un cuerpo reclutado que abarca todos los tipos de individuos relacionados con el programa de educar y de instruir a los mortales ascendentes. Hay más de tres millones de ellos en Nebadon, y todos son voluntarios que se han cualificado por experiencia para servir como asesores educativos en todo el reino. Desde su sede en los mundos Melquisedeks de Salvington, estos supervisores recorren el universo local como inspectores de la técnica académica de Nebadon destinada a formar la mente y a educar el espíritu de las criaturas ascendentes.

\par
%\textsuperscript{(412.2)}
\textsuperscript{37:6.2} Esta formación de la mente y esta educación del espíritu se llevan a cabo desde los mundos de origen humano, pasando por los mundos de las mansiones del sistema y las otras esferas de progreso asociadas a Jerusem, hasta los setenta reinos de vida social vinculados a Edentia y las cuatrocientas noventa esferas de progreso espiritual que rodean a Salvington. En la misma sede del universo se encuentran las numerosas escuelas de los Melquisedeks, las facultades de los Hijos del Universo, las universidades seráficas y las escuelas de los Hijos Instructores y del Unión de los Días. Se toman todas las disposiciones posibles a fin de capacitar a las diversas personalidades del universo para que realicen un servicio más elevado y una actividad mejor. Todo el universo es una inmensa escuela.

\par
%\textsuperscript{(412.3)}
\textsuperscript{37:6.3} Los métodos que se emplean en muchas escuelas superiores sobrepasan el concepto humano sobre el arte de enseñar la verdad, pero he aquí la piedra angular de todo el sistema educativo: la adquisición del carácter mediante una experiencia iluminada. Los educadores aportan la iluminación; el lugar que se ocupa en el universo y el estatus del ascendente proporcionan la oportunidad de experimentar; la sabia utilización de estos dos factores acrecienta el carácter.

\par
%\textsuperscript{(412.4)}
\textsuperscript{37:6.4} El sistema educativo de Nebadon asegura fundamentalmente vuestra asignación a una tarea, y luego os proporciona la oportunidad de enseñaros el método ideal y divino de realizar mejor esa tarea. Se os encarga una tarea determinada a realizar, y al mismo tiempo se os proporcionan los educadores cualificados para enseñaros el mejor método de ejecutar vuestro trabajo. El plan de educación divino asegura la íntima asociación entre el trabajo y la enseñanza. Os enseñamos la mejor manera de ejecutar las cosas que os mandamos hacer.

\par
%\textsuperscript{(412.5)}
\textsuperscript{37:6.5} La finalidad de toda esta formación y de toda esta experiencia es la de prepararos para que seáis admitidos en las esferas educativas superiores y más espirituales del superuniverso. El progreso dentro de un reino determinado es individual, pero la transición de una fase a otra se efectúa generalmente por clases.

\par
%\textsuperscript{(412.6)}
\textsuperscript{37:6.6} La progresión de la eternidad no consiste únicamente en el desarrollo espiritual. La adquisición intelectual también forma parte de la educación universal. La experiencia mental se amplía paralelamente a la expansión del horizonte espiritual. La mente y el espíritu reciben oportunidades semejantes para formarse y avanzar. Pero durante toda esta magnífica preparación mental y espiritual, estáis liberados para siempre de los obstáculos de la carne mortal. Ya no tenéis que arbitrar constantemente las contiendas conflictivas entre vuestras naturalezas espiritual y material divergentes. Por fin estáis cualificados para disfrutar del impulso unificado de una mente glorificada, despojada desde hace mucho tiempo de sus tendencias primitivas animales hacia las cosas materiales.

\par
%\textsuperscript{(413.1)}
\textsuperscript{37:6.7} Antes de dejar el universo de Nebadon, la mayoría de los mortales de Urantia recibirán la oportunidad de servir durante un período más o menos largo como miembros del cuerpo de los Supervisores Celestiales de Nebadon.

\section*{7. Los educadores de los mundos de las mansiones}
\par
%\textsuperscript{(413.2)}
\textsuperscript{37:7.1} Los Educadores de los Mundos de las Mansiones son querubines reclutados y glorificados. Al igual que la mayoría de los otros instructores de Nebadon, son nombrados por los Melquisedeks. Ejercen su actividad en la mayoría de las empresas educativas de la vida morontial, y su numero sobrepasa por completo la comprensión de la mente humana.

\par
%\textsuperscript{(413.3)}
\textsuperscript{37:7.2} Como nivel de consecución de los querubines y de los sanobines, los Educadores de los Mundos de las Mansiones serán objeto de un estudio adicional en el documento siguiente, mientras que como educadores que juegan un papel importante en la vida morontial, hablaremos de ellos más extensamente en el documento que lleva ese nombre.

\section*{8. Las órdenes de espíritus superiores asignadas}
\par
%\textsuperscript{(413.4)}
\textsuperscript{37:8.1} Además de los centros del poder y de los controladores físicos, algunos seres espirituales de origen superior, pertenecientes a la familia del Espíritu Infinito, están asignados permanentemente al universo local. De las órdenes espirituales superiores de la familia del Espíritu Infinito, están asignadas así las que se indican a continuación:

\par
%\textsuperscript{(413.5)}
\textsuperscript{37:8.2} Los \textit{Mensajeros Solitarios}, cuando están vinculados funcionalmente a la administración del universo local, nos prestan un servicio inapreciable en nuestros esfuerzos por vencer los obstáculos del tiempo y del espacio. Cuando no están asignados de esta manera, nosotros los de los universos locales no tenemos ninguna autoridad en absoluto sobre ellos, pero incluso entonces estos seres únicos siempre están dispuestos a ayudarnos a resolver nuestros problemas y a cumplir nuestras misiones.

\par
%\textsuperscript{(413.6)}
\textsuperscript{37:8.3} Andovontia es el nombre del \textit{Supervisor} terciario \textit{de los Circuitos Universales} estacionado en nuestro universo local. Sólo se ocupa de los circuitos espirituales y morontiales, y no de aquellos que están bajo la jurisdicción de los directores del poder. Él es el que aisló a Urantia en la época en que Caligastia traicionó el planeta durante los difíciles momentos de la rebelión de Lucifer. Al enviar sus saludos a los mortales de Urantia, expresa de antemano su placer de que algún día seréis reintegrados en los circuitos universales que él supervisa.

\par
%\textsuperscript{(413.7)}
\textsuperscript{37:8.4} Salsatia, el \textit{Director del Censo} de Nebadon, mantiene su sede en Salvington dentro del sector de Gabriel. Conoce de manera automática el nacimiento y la muerte de la voluntad, y registra constantemente el número exacto de criaturas volitivas que ejercen su actividad en el universo local. Trabaja en estrecha asociación con los registradores de la personalidad domiciliados en los mundos de registro de los arcángeles.

\par
%\textsuperscript{(413.8)}
\textsuperscript{37:8.5} Un \textit{Inspector Asociado} reside en Salvington. Es el representante personal del Ejecutivo Supremo de Orvonton. Sus asociados, los \textit{Centinelas Asignados} a los sistemas locales, también representan al Ejecutivo Supremo de Orvonton.

\par
%\textsuperscript{(414.1)}
\textsuperscript{37:8.6} Los \textit{Conciliadores Universales} son los tribunales itinerantes de los universos del tiempo y del espacio, y ejercen su actividad desde los mundos evolutivos hasta cada una de las secciones del universo local, e incluso más allá. Estos árbitros están registrados en Uversa; el número exacto que trabaja en Nebadon no está anotado, pero estimo que en nuestro universo local hay cerca de cien millones de comisiones conciliadoras.

\par
%\textsuperscript{(414.2)}
\textsuperscript{37:8.7} De los \textit{Asesores Técnicos}, las mentes jurídicas del reino, tenemos nuestro cupo, aproximadamente quinientos millones. Estos seres son las bibliotecas legales experienciales, vivientes y circulantes, de todo el espacio.

\par
%\textsuperscript{(414.3)}
\textsuperscript{37:8.8} De los \textit{Registradores Celestiales}, los serafines ascendentes, tenemos setenta y cinco en Nebadon. Son los registradores supervisores o más antiguos. Los estudiantes avanzados de esta orden que se están formando ascienden a casi cuatro mil millones.

\par
%\textsuperscript{(414.4)}
\textsuperscript{37:8.9} El ministerio de los setenta mil millones de \textit{Compañeros Morontiales} en Nebadon se describe en las narraciones que tratan de los planetas de transición de los peregrinos del tiempo.

\par
%\textsuperscript{(414.5)}
\textsuperscript{37:8.10} Cada universo tiene su propio cuerpo angélico nativo; sin embargo, hay circunstancias en las que es muy útil tener la ayuda de los espíritus superiores que tienen su origen fuera de la creación local. Los supernafines prestan ciertos servicios excepcionales y poco frecuentes; el jefe actual de los serafines de Urantia es un supernafín primario del Paraíso. A los seconafines reflectantes se les encuentra en todos los lugares donde trabaja el personal del superuniverso, y un gran número de terciafines están temporalmente de servicio como Asistentes Altísimos.

\section*{9. Los ciudadanos permanentes del universo local}
\par
%\textsuperscript{(414.6)}
\textsuperscript{37:9.1} Al igual que los superuniversos y el universo central, el universo local tiene sus órdenes de ciudadanos permanentes. Estas órdenes incluyen los tipos creados siguientes:

\par
%\textsuperscript{(414.7)}
\textsuperscript{37:9.2} 1. Los Susatias.

\par
%\textsuperscript{(414.8)}
\textsuperscript{37:9.3} 2. Los Univitatias.

\par
%\textsuperscript{(414.9)}
\textsuperscript{37:9.4} 3. Los Hijos Materiales.

\par
%\textsuperscript{(414.10)}
\textsuperscript{37:9.5} 4. Las Criaturas Intermedias.

\par
%\textsuperscript{(414.11)}
\textsuperscript{37:9.6} Estos nativos de la creación local, junto con los ascendentes fusionados con el Espíritu y los espirongas (que están clasificados de otra manera), constituyen una ciudadanía relativamente permanente. Estas órdenes de seres no son, en general, ni ascendentes ni descendentes. Todas son criaturas experienciales, pero su experiencia creciente continúa estando disponible para el universo en su nivel de origen. Aunque esto no es totalmente cierto en lo que concierne a los Hijos Adámicos y a las criaturas intermedias, es relativamente cierto en lo que se refiere a estas órdenes.

\par
%\textsuperscript{(414.12)}
\textsuperscript{37:9.7} \textit{Los Susatias}. Estos seres maravillosos residen y trabajan como ciudadanos permanentes en Salvington, la sede de este universo local. Son los brillantes descendientes del Hijo Creador y del Espíritu Creativo, y están estrechamente asociados con los ciudadanos ascendentes del universo local, los mortales fusionados con el Espíritu integrados en el Cuerpo de la Perfección de Nebadon.

\par
%\textsuperscript{(414.13)}
\textsuperscript{37:9.8} \textit{Los Univitatias}. Cada uno de los grupos de esferas arquitectónicas que componen las sedes de las cien constelaciones disfruta del ministerio continuo de una orden residencial de seres conocidos con el nombre de univitatias. Estos hijos del Hijo Creador y del Espíritu Creativo constituyen la población permanente de los mundos sede de las constelaciones. Son seres que no se reproducen y que existen en un plano de vida situado casi a medio camino entre el estado semimaterial de los Hijos Materiales domiciliados en las sedes de los sistemas, y el plano más claramente espiritual de los mortales fusionados con el Espíritu y de los susatias de Salvington; pero los univitatias no son seres morontiales. Realizan por los mortales ascendentes, durante la travesía de las esferas de la constelación, lo que los nativos de Havona hacen por los espíritus peregrinos que pasan por la creación central.

\par
%\textsuperscript{(415.1)}
\textsuperscript{37:9.9} \textit{Los Hijos Materiales de Dios}. Cuando un enlace creativo entre el Hijo Creador y la representante universal del Espíritu Infinito, el Espíritu Madre del Universo, ha completado su ciclo, cuando ya no aparecen más descendientes de sus naturalezas combinadas, entonces el Hijo Creador personaliza de manera doble su último concepto del ser, confirmando así definitivamente su propio origen doble original. En sí mismo y de sí mismo crea entonces a los hermosos y magníficos Hijos e Hijas de la orden material de filiación universal. Éste es el origen del Adán y la Eva originales de cada sistema local de Nebadon. Son una orden de filiación que se reproduce, pues son creados masculinos y femeninos. Sus descendientes trabajan como ciudadanos relativamente permanentes de la capital de un sistema, aunque algunos de ellos reciben el nombramiento de Adanes Planetarios.

\par
%\textsuperscript{(415.2)}
\textsuperscript{37:9.10} Durante una misión planetaria, el Hijo y la Hija Materiales reciben el encargo de fundar la raza adámica de ese mundo, una raza destinada a amalgamarse finalmente con los habitantes mortales de esa esfera. Los Adanes Planetarios son Hijos descendentes así como ascendentes, pero habitualmente los clasificamos como ascendentes.

\par
%\textsuperscript{(415.3)}
\textsuperscript{37:9.11} \textit{Las Criaturas Intermedias}. En los primeros tiempos de la mayoría de los mundos habitados, algunos seres superhumanos pero materializados son destinados allí, pero generalmente se retiran cuando llegan los Adanes Planetarios. Las actividades de estos seres y los esfuerzos de los Hijos Materiales por mejorar las razas evolutivas tienen a menudo como resultado la aparición de un número limitado de criaturas que son difíciles de clasificar. Estos seres únicos se encuentran con frecuencia a medio camino entre los Hijos Materiales y las criaturas evolutivas; de ahí su denominación de criaturas intermedias. En un sentido comparativo, estos intermedios son los ciudadanos permanentes de los mundos evolutivos. Desde los primeros tiempos de la llegada de un Príncipe Planetario hasta la época lejana del establecimiento del planeta en la luz y la vida, son el único grupo de seres inteligentes que permanecen continuamente en la esfera. En Urantia, los ministros intermedios son en realidad los verdaderos guardianes del planeta; son prácticamente los ciudadanos de Urantia. Los mortales son en verdad los habitantes físicos y materiales de un mundo evolutivo, pero sois todos tan efímeros; permanecéis en vuestro planeta natal un tiempo tan corto. Nacéis, vivís, morís y pasáis a otros mundos de progresión evolutiva. Incluso los seres superhumanos que sirven en los planetas como ministros celestiales están destinados allí de manera transitoria; pocos de ellos están mucho tiempo vinculados a una esfera determinada. Sin embargo, las criaturas intermedias aseguran la continuidad de la administración planetaria a pesar de los ministerios celestiales siempre cambiantes y de los habitantes mortales que varían constantemente. Durante todos estos cambios y modificaciones incesantes, las criaturas intermedias permanecen en el planeta llevando adelante su trabajo sin interrupción.

\par
%\textsuperscript{(415.4)}
\textsuperscript{37:9.12} De la misma manera, todas las divisiones de la organización administrativa de los universos locales y de los superuniversos tienen sus poblaciones más o menos permanentes, sus habitantes con categoría de ciudadanos. Al igual que Urantia tiene sus intermedios, Jerusem, la capital de vuestro sistema, tiene a los Hijos y las Hijas Materiales; Edentia, la sede de vuestra constelación, tiene a los univitatias, mientras que los ciudadanos de Salvington son de dos tipos, los susatias creados y los mortales evolucionados fusionados con el Espíritu. Los mundos administrativos de los sectores menores y mayores de los superuniversos no tienen ciudadanos permanentes. Pero las esferas sede de Uversa están continuamente favorecidas con un asombroso grupo de seres conocidos con el nombre de \textit{abandontarios}, creados por los agentes no revelados de los Ancianos de los Días y los siete Espíritus Reflectantes residentes en la capital de Orvonton. Estos ciudadanos que residen en Uversa administran actualmente los asuntos rutinarios de su mundo bajo la supervisión directa del cuerpo de los mortales fusionados con el Hijo situado en Uversa. Incluso Havona tiene sus seres nativos, y la Isla central de Luz y de Vida es el hogar de los diversos grupos de Ciudadanos del Paraíso.

\section*{10. Otros grupos del universo local}
\par
%\textsuperscript{(416.1)}
\textsuperscript{37:10.1} Además de las órdenes seráficas y mortales, que serán examinadas en documentos posteriores, hay numerosos seres adicionales relacionados con el mantenimiento y el perfeccionamiento de una organización tan gigantesca como el universo de Nebadon, que ahora mismo ya tiene más de tres millones de mundos habitados, con diez millones en perspectiva. Los diversos tipos de vida de Nebadon son demasiado numerosos para ser catalogados en este documento, pero podemos mencionar dos órdenes excepcionales que ejercen ampliamente su actividad en las 647.591 esferas arquitectónicas del universo local.

\par
%\textsuperscript{(416.2)}
\textsuperscript{37:10.2} Los \textit{Espirongas} son los descendientes espirituales de la Radiante Estrella Matutina y el Padre Melquisedek. Están exentos de que se ponga fin a su personalidad, pero no son seres evolutivos ni ascendentes. Tampoco están implicados funcionalmente en el régimen de la ascensión evolutiva. Son los ayudantes espirituales del universo local, y realizan las tareas espirituales rutinarias de Nebadon.

\par
%\textsuperscript{(416.3)}
\textsuperscript{37:10.3} Los \textit{Espornagias}. Los mundos sede arquitectónicos del universo local son mundos reales ---creaciones físicas. Su conservación física requiere mucho trabajo, y para ello contamos con la ayuda de un grupo de criaturas físicas llamadas espornagias. Se dedican al cuidado y al cultivo de las fases materiales de estos mundos sede, desde Jerusem hasta Salvington. Los espornagias no son ni espíritus ni personas; son una orden animal de existencia, pero si pudierais verlos estaríais de acuerdo en que parecen animales perfectos.

\par
%\textsuperscript{(416.4)}
\textsuperscript{37:10.4} Las diversas \textit{colonias de cortesía} están domiciliadas en Salvington y en otros lugares. En las constelaciones nos beneficiamos especialmente del ministerio de los artesanos celestiales, y sacamos provecho de las actividades de los directores de la reversión que trabajan principalmente en las capitales de los sistemas locales.

\par
%\textsuperscript{(416.5)}
\textsuperscript{37:10.5} Un cuerpo de mortales ascendentes, incluyendo a las criaturas intermedias glorificadas, siempre está destinado al servicio del universo. Después de llegar a Salvington, estos ascendentes son empleados en una variedad casi infinita de actividades relacionadas con la dirección de los asuntos del universo. Desde cada nivel que han alcanzado, estos mortales que progresan retroceden y descienden para echar una mano a sus compañeros que los siguen en la ascensión. Estos mortales que residen temporalmente en Salvington son enviados, cuando son solicitados, a casi todos los cuerpos de personalidades celestiales como ayudantes, estudiantes, observadores y educadores.

\par
%\textsuperscript{(416.6)}
\textsuperscript{37:10.6} Existen además otros tipos de vida inteligente relacionados con la administración de un universo local, pero el plan de esta narración no prevé la revelación adicional de estas órdenes creadas. Aquí se describe lo suficiente sobre la vida y la administración de este universo como para proporcionarle a la mente mortal un vislumbre de la realidad y la grandiosidad de la existencia en la supervivencia. Las experiencias ulteriores de vuestra carrera progresiva os revelarán cada vez más estos seres interesantes y encantadores. Esta narración no puede ser más que un breve esbozo de la naturaleza y del trabajo de las múltiples personalidades que atestan los universos del espacio, administrando estas creaciones como enormes escuelas formativas, unas escuelas donde los peregrinos del tiempo avanzan de vida en vida y de mundo en mundo, hasta que son enviados con amor desde las fronteras de su universo de origen hacia el régimen educativo superior del superuniverso, y desde allí hacia los mundos de formación espiritual de Havona, y finalmente hacia el Paraíso y el elevado destino de los finalitarios ---la asignación eterna a misiones aún no reveladas a los universos del tiempo y del espacio.

\par
%\textsuperscript{(417.1)}
\textsuperscript{37:10.7} [Dictado por una Brillante Estrella Vespertina de Nebadon, Número 1.146 del Cuerpo Creado.]


\chapter{Documento 38. Los espíritus ministrantes del universo local}
\par
%\textsuperscript{(418.1)}
\textsuperscript{38:0.1} LAS personalidades del Espíritu Infinito se dividen en tres órdenes distintas. El impetuoso apóstol comprendió esto cuando escribió acerca de Jesús <<que ha subido al cielo, se encuentra a la diestra de Dios\footnote{\textit{Derecha de Dios}: Sal 110:1; Mt 22:43-44; Mc 12:36; 16:19; Lc 20:42; Hch 7:55-56; Ro 8:34; Col 3:1; Heb 1:3; 8:1; 10:12; 12:2; 1 P 3:22.}, y los ángeles, las autoridades y las potestades están sometidas a él>>. Los ángeles son los espíritus ministrantes del tiempo; las autoridades son las huestes de mensajeros del espacio, y las potestades son las personalidades superiores del Espíritu Infinito.

\par
%\textsuperscript{(418.2)}
\textsuperscript{38:0.2} Al igual que los supernafines en el universo central y los seconafines en un superuniverso, los serafines, con sus querubines y sanobines asociados, constituyen el cuerpo angélico de un universo local.

\par
%\textsuperscript{(418.3)}
\textsuperscript{38:0.3} El diseño de todos los serafines es bastante uniforme. De un universo a otro, a lo largo y ancho de los siete superuniversos, muestran un mínimo de variaciones; de todos los tipos espirituales de seres personales, son los que más se acercan a un tipo estándar. Sus diversas órdenes componen el cuerpo de ministros ordinarios y cualificados de las creaciones locales.

\section*{1. El origen de los serafines}
\par
%\textsuperscript{(418.4)}
\textsuperscript{38:1.1} Los serafines\footnote{\textit{Serafín}: Is 6:2,6.} son creados por el Espíritu Madre del Universo y fueron proyectados en formaciones unitarias ---41.472 a la vez--- desde la creación de los <<ángeles modelo>> y de ciertos arquetipos angélicos en los primeros tiempos de Nebadon. El Hijo Creador y la representante del Espíritu Infinito en el universo colaboran para crear un gran número de Hijos y de otras personalidades del universo. Después de culminar este esfuerzo unido, el Hijo emprende la creación de los Hijos Materiales, las primeras criaturas sexuadas, mientras que el Espíritu Madre del Universo acomete paralelamente su esfuerzo solitario inicial de reproducción espiritual. Así empieza la creación de las huestes seráficas de un universo local.

\par
%\textsuperscript{(418.5)}
\textsuperscript{38:1.2} Estas órdenes angélicas se proyectan en la época en que se hacen planes para la evolución de las criaturas volitivas mortales. La creación de los serafines data del momento en que el Espíritu Madre del Universo consiguió una personalidad relativa, no como coordinada posterior del Hijo Maestro, sino como asistente creativa inicial del Hijo Creador. Antes de este acontecimiento, los serafines que servían en Nebadon habían sido prestados temporalmente por un universo vecino.

\par
%\textsuperscript{(418.6)}
\textsuperscript{38:1.3} Periódicamente se siguen creando serafines; el universo de Nebadon está todavía en construcción. El Espíritu Madre del Universo nunca pone fin a su actividad creativa en un universo que está creciendo y perfeccionándose.

\section*{2. Las naturalezas angélicas}
\par
%\textsuperscript{(419.1)}
\textsuperscript{38:2.1} Los ángeles no tienen cuerpos materiales, pero son seres definidos y distintos; tienen una naturaleza y un origen espirituales. Aunque son invisibles para los mortales, os perciben tal como sois en la carne sin la ayuda de los transformadores o de los traductores; comprenden intelectualmente la manera de vivir de los mortales, y comparten todas las emociones y sentimientos no sensuales del hombre. Aprecian vuestros esfuerzos en el campo de la música, del arte y del verdadero humor, y disfrutan enormemente con ellos. Conocen plenamente vuestras luchas morales y vuestras dificultades espirituales. Aman a los seres humanos, y sólo puede resultar algo bueno de vuestros esfuerzos por comprenderlos y amarlos.

\par
%\textsuperscript{(419.2)}
\textsuperscript{38:2.2} Aunque los serafines son unos seres muy afectuosos y comprensivos, no son criaturas con emociones sexuales. Son en gran medida como vosotros seréis en los mundos de las mansiones, donde <<ni os casaréis ni seréis dados en matrimonio, sino que seréis como los ángeles del cielo>>\footnote{\textit{Los ángeles no se casan}: Mt 22:30; Mc 12:25.}. Porque todos los que <<sean considerados dignos de llegar a los mundos de las mansiones\footnote{\textit{Ángeles en los mundos mansiones}: Lc 20:35-36.}, ni se casan ni son dados en matrimonio; y ya no mueren más, pues son iguales a los ángeles>>. Sin embargo, cuando tratamos con criaturas sexuadas tenemos la costumbre de llamar hijos de Dios a los seres que descienden más directamente del Padre y del Hijo, e hijas de Dios cuando nos referimos a los hijos del Espíritu. Por consiguiente, en los planetas sexuados, a los ángeles los designamos normalmente con pronombres femeninos.

\par
%\textsuperscript{(419.3)}
\textsuperscript{38:2.3} Los serafines son creados de tal manera que pueden ejercer su actividad tanto en el nivel espiritual como en el nivel tangible. Existen pocas fases de la actividad morontial o espiritual que no estén abiertas a sus servicios. Aunque los ángeles no están muy alejados de los seres humanos en cuanto a su estado personal, los serafines los trascienden considerablemente en ciertas actividades funcionales. Poseen muchos poderes que se encuentran mucho más allá de la comprensión humana. Por ejemplo: se os ha dicho que <<los cabellos mismos de vuestra cabeza están contados>>\footnote{\textit{Los cabellos de la cabeza están contados}: Mt 10:30; Lc 12:7.}, y es verdad que lo están, pero un serafín no emplea su tiempo contándolos y manteniendo su número corregido al día. Los ángeles poseen poderes inherentes y automáticos (es decir, automáticos hasta donde podríais percibirlos) para saber estas cosas; vosotros consideraríais en verdad a un serafín como un prodigio matemático. Por eso numerosos deberes que serían enormes tareas para los mortales son realizados con suma facilidad por los serafines.

\par
%\textsuperscript{(419.4)}
\textsuperscript{38:2.4} El estado espiritual de los ángeles es superior al vuestro, pero no son vuestros jueces ni vuestros acusadores. Cualesquiera que sean vuestras faltas, <<los ángeles, aunque son más grandes en poder y en fuerza, no formulan ninguna acusación contra vosotros>>\footnote{\textit{Los ángeles no te acusan}: 2 P 2:11.}. Los ángeles no juzgan a la humanidad, y los mortales individuales tampoco deberían juzgar de antemano a sus semejantes.

\par
%\textsuperscript{(419.5)}
\textsuperscript{38:2.5} Hacéis bien en amarlos, pero no debéis adorarlos; los ángeles no son objetos de adoración. Cuando vuestro vidente <<se postró a los pies del ángel para adorarlo>>, el gran serafín Loyalatia le dijo: <<Procura no hacerlo; soy un servidor como tú y los de tus razas, y todos habéis recibido el mandato de adorar a Dios>>\footnote{\textit{No adoréis a los ángeles}: Ap 19:10; 22:8-9.}.

\par
%\textsuperscript{(419.6)}
\textsuperscript{38:2.6} En la escala de la existencia de las criaturas, los serafines sólo están un poquito por delante de las razas mortales en cuanto a naturaleza y a dotación de personalidad. En verdad, cuando sois liberados de la carne os volvéis muy parecidos a ellos. En los mundos de las mansiones empezaréis a apreciar a los serafines, en las esferas de la constelación a disfrutar de ellos, mientras que en Salvington compartirán con vosotros sus lugares de descanso y de adoración. Durante toda la ascensión morontial y la ascensión espiritual posterior, vuestra fraternidad con los serafines será ideal; vuestro compañerismo será magnífico.

\section*{3. Los ángeles no revelados}
\par
%\textsuperscript{(420.1)}
\textsuperscript{38:3.1} Hay numerosas órdenes de seres espirituales que ejercen su actividad en todos los dominios del universo local y que no son revelados a los mortales porque no están relacionados de ninguna manera con el plan evolutivo de ascensión al Paraíso. La palabra <<ángel>>, en este documento, se limita intencionalmente a designar a los descendientes seráficos y asociados del Espíritu Madre del Universo que se ocupan tan ampliamente de trabajar en los planes de la supervivencia de los mortales. En el universo local sirven otras seis órdenes de seres emparentados, los ángeles no revelados, que no están conectados de ninguna manera específica con las actividades universales relacionadas con la ascensión de los mortales evolutivos al Paraíso. A estos seis grupos de asociados angélicos nunca los llamamos serafines, y tampoco nos referimos a ellos como espíritus ministrantes. Estas personalidades se ocupan enteramente de las cuestiones administrativas y de otros asuntos de Nebadon, unas ocupaciones que no están relacionadas de ninguna manera con la carrera progresiva del hombre consistente en ascender espiritualmente y alcanzar la perfección.

\section*{4. Los mundos seráficos}
\par
%\textsuperscript{(420.2)}
\textsuperscript{38:4.1} El noveno grupo de siete esferas primarias del circuito de Salvington está formado por los mundos de los serafines. Cada uno de estos mundos tiene seis satélites tributarios donde se encuentran las escuelas especiales dedicadas a todas las fases de la formación seráfica. Aunque los serafines tienen acceso a los cuarenta y nueve mundos que componen este grupo de esferas de Salvington, sólo ocupan de manera exclusiva el primer grupo de siete. Los otros seis grupos están ocupados por las seis órdenes de asociados angélicos no revelados en Urantia; cada uno de estos grupos mantiene su sede en uno de estos seis mundos primarios y realiza actividades especializadas en los seis satélites tributarios. Cada orden angélica tiene libre acceso a todos los mundos de estos siete grupos distintos.

\par
%\textsuperscript{(420.3)}
\textsuperscript{38:4.2} Estos mundos sede se cuentan entre los reinos más magníficos de Nebadon; las residencias seráficas están caracterizadas tanto por su belleza como por su inmensidad. Aquí cada serafín tiene un verdadero hogar, y <<hogar>> significa el domicilio de dos serafines; viven en parejas.

\par
%\textsuperscript{(420.4)}
\textsuperscript{38:4.3} Aunque no son masculinos y femeninos como los Hijos Materiales y las razas mortales, los serafines son positivos y negativos. En la mayoría de las misiones se necesitan dos ángeles para realizar la tarea. Cuando no están situados en circuito pueden trabajar solos; y cuando están estacionarios tampoco necesitan a su complemento. Normalmente conservan a su complemento original, pero no necesariamente. Estas asociaciones se necesitan principalmente debido a las funciones que han de realizar; no están caracterizadas por las emociones sexuales, aunque son extremadamente personales y verdaderamente afectuosas.

\par
%\textsuperscript{(420.5)}
\textsuperscript{38:4.4} Además de sus hogares asignados, los serafines también tienen sus sedes de grupo, de compañías, de batallones y de unidades. Cada milenio se reúnen en asambleas y todos están presentes con arreglo a la época en que fueron creados. Si un serafín tiene responsabilidades que le impiden ausentarse de su deber, alterna con su complemento para asistir a la reunión, siendo reemplazado por un serafín nacido en otra fecha. Cada asociado seráfico está así presente al menos en una reunión de cada dos.

\section*{5. La formación seráfica}
\par
%\textsuperscript{(420.6)}
\textsuperscript{38:5.1} Los serafines pasan su primer milenio como observadores sin cometido en Salvington y en sus mundos-escuela asociados. El segundo milenio lo pasan en los mundos seráficos del circuito de Salvington. Su escuela central de formación está presidida actualmente por los primeros cien mil serafines de Nebadon, y a la cabeza se encuentra el ángel original o primogénito de este universo local. El primer grupo creado de serafines de Nebadon fue instruido por un cuerpo de mil serafines procedentes de Avalon; posteriormente, nuestros ángeles han sido enseñados por sus propios compañeros más antiguos. Los Melquisedeks juegan también un papel importante en la educación y la formación de todos los ángeles del universo local ---serafines, querubines y sanobines.

\par
%\textsuperscript{(421.1)}
\textsuperscript{38:5.2} Al final de este período de formación en los mundos seráficos de Salvington, los serafines son movilizados en los grupos y las unidades convencionales de la organización angélica, y son destinados a una de las constelaciones. Todavía no son nombrados como espíritus ministrantes, aunque ya han entrado en las fases de formación angélica previas al nombramiento.

\par
%\textsuperscript{(421.2)}
\textsuperscript{38:5.3} Los serafines se inician como espíritus ministrantes sirviendo como observadores en los mundos evolutivos más inferiores. Después de esta experiencia, regresan a los mundos asociados de la sede de la constelación donde están destinados para empezar sus estudios avanzados y prepararse con más precisión para servir en algún sistema local particular. Después de esta educación general, se les promueve a servir en uno de los sistemas locales. Nuestros serafines completan su formación en los mundos arquitectónicos asociados a la capital de algún sistema de Nebadon, y son nombrados como espíritus ministrantes del tiempo.

\par
%\textsuperscript{(421.3)}
\textsuperscript{38:5.4} Una vez que los serafines reciben su nombramiento, pueden recorrer todo Nebadon, e incluso Orvonton, cumpliendo misiones. Su trabajo en el universo no tiene trabas ni limitaciones; están estrechamente asociados con las criaturas materiales de los mundos, y siempre están al servicio de las órdenes inferiores de personalidades espirituales, poniendo en contacto a estos seres del mundo espiritual con los mortales de los reinos materiales.

\section*{6. La organización seráfica}
\par
%\textsuperscript{(421.4)}
\textsuperscript{38:6.1} Después del segundo milenio de estancia en la sede seráfica, los serafines se organizan bajo el mando de sus jefes en grupos de doce (12 parejas, 24 serafines), y doce grupos de éstos constituyen una compañía (144 parejas, 288 serafines), que es dirigida por un jefe. Doce compañías bajo las órdenes de un comandante constituyen un batallón (1.728 parejas o 3.456 serafines), y doce batallones bajo las órdenes de un director equivalen a una unidad seráfica (20.736 parejas o 41.472 individuos), mientras que doce unidades, sujetas al mando de un supervisor, constituyen una legión que suma 248.832 parejas o 497.664 individuos. Jesús aludió a este tipo de grupo de ángeles aquella noche en el jardín de Getsemaní, cuando dijo: <<Ahora mismo puedo pedírselo a mi Padre, y él me dará enseguida más de doce legiones de ángeles>>\footnote{\textit{Doce legiones de ángeles}: Mt 26:53.}.

\par
%\textsuperscript{(421.5)}
\textsuperscript{38:6.2} Doce legiones de ángeles componen una hueste que asciende a
2.985.984 parejas o 5.971.968 individuos, y doce huestes de éstas (35.831.808 parejas o 71.663.616 individuos) forman la organización operativa más grande de serafines, un ejército angélico. Una hueste seráfica está dirigida por un arcángel o por alguna otra personalidad con rango coordinado, mientras que los ejércitos angélicos están dirigidos por las Brillantes Estrellas Vespertinas o por otros lugartenientes directos de Gabriel. Y Gabriel es el <<comandante supremo de los ejércitos del cielo>>\footnote{\textit{Comandante supremo}: Ap 19:14.}, el jefe ejecutivo del Soberano de Nebadon, <<el Señor Dios de los ejércitos>>\footnote{\textit{Señor Dios de los ejércitos}: 1 Re 19:10,14; Sal 80:4,19; 2 Sam 5:10.}.

\par
%\textsuperscript{(421.6)}
\textsuperscript{38:6.3} Desde la donación de Miguel en Urantia, y aunque sirven bajo la supervisión directa del Espíritu Infinito tal como éste está personalizado en Salvington, los serafines y todas las demás órdenes del universo local han quedado sometidos a la soberanía del Hijo Maestro. Incluso cuando Miguel nació en la carne en Urantia, se emitió una transmisión superuniversal a todo Nebadon proclamando <<Que todos los ángeles lo adoren>>\footnote{\textit{Que todos los ángeles lo adoren}: Heb 1:6.}. Todas las categorías de ángeles están sujetas a su soberanía; forman parte del grupo que ha sido denominado <<sus ángeles poderosos>>\footnote{\textit{Sus ángeles poderosos}: 2 Ts 1:7; Ap 10:1; Ap 18:21.}.

\section*{7. Los querubines y los sanobines}
\par
%\textsuperscript{(422.1)}
\textsuperscript{38:7.1} Los querubines y los sanobines son similares a los serafines en todas sus dotaciones esenciales. Tienen el mismo origen, pero no siempre el mismo destino. Son asombrosamente inteligentes, maravillosamente eficaces, conmovedoramente afectuosos, y casi humanos. Forman la orden más inferior de ángeles, de ahí que sean los parientes más cercanos de los tipos más progresivos de seres humanos de los mundos evolutivos.

\par
%\textsuperscript{(422.2)}
\textsuperscript{38:7.2} Los querubines y los sanobines están inherentemente asociados, funcionalmente unidos. Uno es, en relación con la energía, una personalidad positiva y el otro una personalidad negativa. El deflector de la derecha, o ángel cargado positivamente, es el querubín ---la personalidad más antigua o controladora. El deflector de la izquierda, o ángel cargado negativamente, es el sanobín--- el complemento del ser. Las funciones solitarias de cada tipo de ángel son muy limitadas; por eso sirven habitualmente en parejas. Cuando sirven independientemente de sus directores seráficos dependen más que nunca del contacto mutuo, y siempre trabajan juntos.

\par
%\textsuperscript{(422.3)}
\textsuperscript{38:7.3} Los querubines y los sanobines son los ayudantes fieles y eficaces de los ministros seráficos, y las siete órdenes de serafines están provistas de estos asistentes subordinados. Los querubines y los sanobines sirven durante épocas enteras en estas funciones, pero no acompañan a los serafines en las misiones que realizan más allá de los confines del universo local.

\par
%\textsuperscript{(422.4)}
\textsuperscript{38:7.4} Los querubines y los sanobines son los trabajadores espirituales rutinarios de los mundos individuales de los sistemas. En una misión no personal y en un caso de urgencia, pueden servir en el lugar de una pareja seráfica, pero nunca ejercen su actividad, ni siquiera temporalmente, como ángeles acompañantes de los seres humanos; éste es un privilegio exclusivamente seráfico.

\par
%\textsuperscript{(422.5)}
\textsuperscript{38:7.5} Cuando son destinados a un planeta, los querubines ingresan en los cursos locales de formación, incluyendo el estudio de las costumbres y de los idiomas planetarios. Todos los espíritus ministrantes del tiempo son biling\"ues, pues hablan el idioma de su universo local de origen y el de su superuniverso nativo. Y adquieren otras lenguas adicionales estudiándolas en las escuelas de los reinos. Los querubines y los sanobines, al igual que los serafines y todas las demás órdenes de seres espirituales, se esfuerzan continuamente por mejorarse. Únicamente los seres subordinados que controlan el poder y la dirección de la energía son incapaces de progresar; todas las criaturas que poseen la volición manifestada o potencial de la personalidad buscan nuevos logros.

\par
%\textsuperscript{(422.6)}
\textsuperscript{38:7.6} Los querubines y los sanobines están por naturaleza muy cerca del nivel morontial de existencia, y demuestran ser sumamente eficaces en el trabajo fronterizo entre los dominios físico, morontial y espiritual. Estos hijos del Espíritu Madre del universo local están caracterizados por las <<cuartas criaturas>> de manera muy similar a los Servitales de Havona y a las comisiones conciliadoras. Cada cuarto querubín y cada cuarto sanobín son casi materiales, pareciéndose muy claramente al nivel morontial de existencia.

\par
%\textsuperscript{(422.7)}
\textsuperscript{38:7.7} Estas cuartas criaturas angélicas son de una gran ayuda para los serafines en las fases más tangibles de sus actividades universales y planetarias. Estos querubines morontiales también llevan a cabo numerosas tareas limítrofes indispensables en los mundos formativos morontiales, y son destinados en gran número al servicio de los Compañeros Morontiales. Representan, para las esferas morontiales, casi lo mismo que las criaturas intermedias para los planetas evolutivos. En los mundos habitados, estos querubines morontiales trabajan con frecuencia en unión con las criaturas intermedias. Los querubines y las criaturas intermedias son unas órdenes de seres claramente distintas; tienen orígenes diferentes, pero revelan una gran similitud de naturaleza y de funcionamiento.

\section*{8. La evolución de los querubines y los sanobines}
\par
%\textsuperscript{(423.1)}
\textsuperscript{38:8.1} Los querubines y los sanobines tienen abiertas numerosas vías de servicio progresivo que conducen a una elevación de su estado, el cual puede mejorar aún más gracias al abrazo de la Ministra Divina. En lo que se refiere al potencial evolutivo, existen tres grandes clases de querubines y de sanobines:

\par
%\textsuperscript{(423.2)}
\textsuperscript{38:8.2} 1. \textit{Los candidatos a la ascensión}. Estos seres son por naturaleza candidatos al estado seráfico. Los querubines y los sanobines de esta orden son brillantes, aunque por su dotación inherente no son iguales a los serafines; pero mediante la aplicación y la experiencia les resulta posible alcanzar la plena condición seráfica.

\par
%\textsuperscript{(423.3)}
\textsuperscript{38:8.3} 2. \textit{Los querubines de la fase media}. Todos los querubines y sanobines no poseen el mismo potencial de ascensión, y éstos son los seres inherentemente limitados de las creaciones angélicas. La mayor parte de ellos seguirán siendo querubines y sanobines, aunque los individuos más dotados pueden conseguir un servicio seráfico limitado.

\par
%\textsuperscript{(423.4)}
\textsuperscript{38:8.4} 3. \textit{Los querubines morontiales}. Estas <<cuartas criaturas>> de las órdenes angélicas conservan siempre sus características casi materiales. Continuarán siendo querubines y sanobines, junto con una mayoría de sus hermanos de la fase media, hasta la aparición completa del Ser Supremo.

\par
%\textsuperscript{(423.5)}
\textsuperscript{38:8.5} Aunque el segundo y el tercer grupo están un poco limitados en su potencial de crecimiento, los candidatos a la ascensión pueden alcanzar las alturas del servicio seráfico universal. Muchos querubines entre los más experimentados de este tipo son vinculados a los guardianes seráficos del destino y están situados así en línea directa para ascender al estado de Educadores de los Mundos de las Mansiones cuando sean abandonados por sus decanos seráficos. Los guardianes del destino no tienen querubines ni sanobines como ayudantes cuando sus pupilos mortales alcanzan la vida morontial. Y cuando a otros tipos de serafines evolutivos les conceden permiso para ir a Serafington y al Paraíso, tienen que separarse de sus antiguos subordinados cuando salen de los confines de Nebadon. Estos querubines y sanobines abandonados son abrazados generalmente por el Espíritu Madre del Universo, consiguiendo así un nivel equivalente al de un Educador de los Mundos de las Mansiones en el camino de alcanzar el estado seráfico.

\par
%\textsuperscript{(423.6)}
\textsuperscript{38:8.6} Cuando los querubines y los sanobines ya abrazados han servido durante mucho tiempo como Educadores de los Mundos de las Mansiones en las esferas morontiales, desde la más humilde hasta la más elevada, y cuando su grupo de Salvington contiene demasiados miembros, la Radiante Estrella Matutina convoca a estos fieles servidores de las criaturas del tiempo para que aparezcan en su presencia. Prestan el juramento de la transformación de la personalidad e, inmediatamente después, estos querubines y sanobines avanzados y decanos son abrazados de nuevo por el Espíritu Madre del Universo en grupos de siete mil. De este segundo abrazo surgen como serafines plenamente desarrollados. De ahora en adelante, la carrera plena y completa de un serafín, con todas sus posibilidades paradisiacas, está abierta para estos querubines y sanobines que han nacido de nuevo. Estos ángeles pueden ser nombrados como guardianes del destino de algún ser mortal, y si su pupilo mortal consigue la supervivencia, entonces tendrán derecho a avanzar hasta Serafington y los siete círculos de consecución seráfica, e incluso hasta el Paraíso y el Cuerpo de la Finalidad.

\section*{9. Las criaturas intermedias}
\par
%\textsuperscript{(424.1)}
\textsuperscript{38:9.1} Las criaturas intermedias tienen una clasificación triple: están adecuadamente clasificadas con los Hijos ascendentes de Dios; están agrupadas de hecho con las órdenes de ciudadanos permanentes, y son contadas funcionalmente entre los espíritus ministrantes del tiempo debido a su asociación íntima y eficaz con las huestes angélicas en el trabajo de servir al hombre mortal en los mundos individuales del espacio.

\par
%\textsuperscript{(424.2)}
\textsuperscript{38:9.2} Estas criaturas únicas aparecen en la mayoría de los mundos habitados, y siempre se las encuentra en los planetas decimales como Urantia donde se experimenta con la vida. Los intermedios son de dos tipos ---primarios y secundarios--- y aparecen por medio de las técnicas siguientes:

\par
%\textsuperscript{(424.3)}
\textsuperscript{38:9.3} 1. \textit{Los Intermedios Primarios}, el grupo más espiritual, son una orden de seres un poco tipificada que desciende uniformemente de los mortales ascendentes modificados pertenecientes al estado mayor de los Príncipes Planetarios. El número de criaturas intermedias primarias es siempre de cincuenta mil, y ningún planeta que disfruta de su ministerio posee un grupo más numeroso.

\par
%\textsuperscript{(424.4)}
\textsuperscript{38:9.4} 2. \textit{Los Intermedios Secundarios} es el grupo más material de estas criaturas, y su número varía considerablemente en los diferentes mundos, aunque el promedio es de unos cincuenta mil. Descienden de maneras diversas de los mejoradores biológicos planetarios, los Adanes y las Evas, o de su progenie directa. Existen no menos de veinticuatro técnicas distintas para dar nacimiento a estas criaturas intermedias secundarias en los mundos evolutivos del espacio. La manera en que este grupo se originó en Urantia fue inhabitual y extraordinaria.

\par
%\textsuperscript{(424.5)}
\textsuperscript{38:9.5} Ninguno de estos grupos es un accidente evolutivo; los dos constituyen unos elementos esenciales en los planes predeterminados de los arquitectos del universo, y su aparición en los mundos evolutivos en la coyuntura oportuna se produce con arreglo a los diseños originales y a los planes en desarrollo de los Portadores de Vida supervisores.

\par
%\textsuperscript{(424.6)}
\textsuperscript{38:9.6} Los intermedios primarios reciben su energía intelectual y espiritual por medio de la técnica angélica, y su nivel intelectual es uniforme. Los siete espíritus ayudantes de la mente no se ponen en contacto con ellos; sólo el sexto y el séptimo, el espíritu de adoración y el espíritu de sabiduría, son capaces de aportar su ministerio al grupo secundario.

\par
%\textsuperscript{(424.7)}
\textsuperscript{38:9.7} Los intermedios secundarios reciben su energía física mediante la técnica adámica, están espiritualmente situados en circuito mediante la técnica seráfica, y están dotados intelectualmente del tipo de mente morontial de transición. Están divididos en cuatro tipos físicos, en siete órdenes espirituales y en doce niveles de reacción intelectual al ministerio conjunto de los dos últimos espíritus ayudantes y de la mente morontial. Estas variantes determinan su diferencial de actividad y de asignaciones planetarias.

\par
%\textsuperscript{(424.8)}
\textsuperscript{38:9.8} Los intermedios primarios se parecen más a los ángeles que a los mortales; las órdenes secundarias se parecen mucho más a los seres humanos. Cada una de ellas aporta una ayuda inapreciable a la otra en la ejecución de sus múltiples tareas planetarias. Los ministros primarios pueden conseguir cooperar en enlace tanto con los controladores de la energía morontial y espiritual como con los que tienen que ver con el circuito mental. El grupo secundario sólo puede establecer relaciones de trabajo con los controladores físicos y los manipuladores de los circuitos materiales. Pero, puesto que cada orden de intermedios puede establecer un perfecto sincronismo de contacto con la otra, cada grupo es capaz de utilizar en la práctica toda la gama de energías que se extienden desde el poder físico bruto de los mundos materiales, pasando por las fases de transición de las energías universales, hasta las fuerzas superiores de la realidad espiritual de los reinos celestiales.

\par
%\textsuperscript{(425.1)}
\textsuperscript{38:9.9} La laguna entre los mundos materiales y espirituales está perfectamente colmada mediante la asociación en serie del hombre mortal, el intermedio secundario, el intermedio primario, el querubín morontial, el querubín de la fase media y el serafín. En la experiencia personal de un mortal individual, estos diversos niveles están indudablemente más o menos unificados y tienen un significado personal gracias a las actividades desapercibidas y misteriosas del Ajustador del Pensamiento divino.

\par
%\textsuperscript{(425.2)}
\textsuperscript{38:9.10} En los mundos normales, los intermedios primarios mantienen su servicio como cuerpo de información y como anfitriones celestiales en nombre del Príncipe Planetario, mientras que los ministros secundarios continúan cooperando con el régimen adámico para fomentar la causa de la civilización planetaria progresiva. En caso de deserción del Príncipe Planetario y de fallo del Hijo Material, como sucedió en Urantia, las criaturas intermedias se convierten en los pupilos del Soberano del Sistema y sirven bajo la dirección del custodio en funciones del planeta. Pero en Satania sólo hay otros tres mundos donde estos seres trabajan en un solo grupo bajo un mando unificado, como lo hacen los ministros intermedios unidos de Urantia.

\par
%\textsuperscript{(425.3)}
\textsuperscript{38:9.11} El trabajo planetario de los intermedios primarios y secundarios es variado y diverso en los numerosos mundos individuales de un universo, pero en los planetas normales y medios, sus actividades son muy diferentes a las obligaciones que ocupan su tiempo en las esferas aisladas como Urantia.

\par
%\textsuperscript{(425.4)}
\textsuperscript{38:9.12} Los intermedios primarios son los historiadores planetarios que, desde el momento de la llegada del Príncipe Planetario hasta la época del establecimiento de la luz y la vida, elaboran los espectáculos y diseñan las descripciones de la historia planetaria para las exposiciones de los planetas en los mundos sede de los sistemas.

\par
%\textsuperscript{(425.5)}
\textsuperscript{38:9.13} Los intermedios permanecen durante largos períodos en un mundo habitado, pero si son fieles a su deber, serán finalmente reconocidos con toda seguridad por su servicio secular para el mantenimiento de la soberanía del Hijo Creador; serán debidamente recompensados por su paciente ministerio hacia los mortales materiales en su mundo del tiempo y del espacio. Tarde o temprano, todas las criaturas intermedias acreditadas serán enroladas en las filas de los Hijos ascendentes de Dios, y serán debidamente introducidas en la larga aventura de la ascensión al Paraíso en compañía de los mismos mortales de origen animal, sus hermanos terrestres, a quienes protegieron tan celosamente y sirvieron con tanta eficacia durante su larga estancia planetaria.

\par
%\textsuperscript{(425.6)}
\textsuperscript{38:9.14} [Presentado por un Melquisedek que actúa a petición del Jefe de las Huestes Seráficas de Nebadon.]


\chapter{Documento 39. Las huestes seráficas}
\par
%\textsuperscript{(426.1)}
\textsuperscript{39:0.1} POR lo que sabemos, el Espíritu Infinito, tal como está personalizado en las sedes de los universos locales, tiene la intención de engendrar serafines uniformemente perfectos, pero por alguna razón desconocida estos descendientes seráficos son muy diversos. Esta diversidad puede ser el resultado de una interposición desconocida de la Deidad experiencial en evolución; si es así, no podemos probarlo. Pero sí observamos que cuando los serafines han sido sometidos a las pruebas educativas y a la disciplina formativa, se clasifican de manera infalible y bien determinada en los siete grupos siguientes:

\par
%\textsuperscript{(426.2)}
\textsuperscript{39:0.2} 1. Los Serafines Supremos.

\par
%\textsuperscript{(426.3)}
\textsuperscript{39:0.3} 2. Los Serafines Superiores.

\par
%\textsuperscript{(426.4)}
\textsuperscript{39:0.4} 3. Los Serafines Supervisores.

\par
%\textsuperscript{(426.5)}
\textsuperscript{39:0.5} 4. Los Serafines Administradores.

\par
%\textsuperscript{(426.6)}
\textsuperscript{39:0.6} 5. Los Ayudantes Planetarios.

\par
%\textsuperscript{(426.7)}
\textsuperscript{39:0.7} 6. Los Ministros de Transición.

\par
%\textsuperscript{(426.8)}
\textsuperscript{39:0.8} 7. Los Serafines del Futuro.

\par
%\textsuperscript{(426.9)}
\textsuperscript{39:0.9} Decir que un serafín cualquiera es inferior a un ángel de cualquier otro grupo no sería exactamente cierto. Sin embargo, el servicio de cada ángel está limitado, al principio, al grupo de su clasificación original e inherente. Manotia, mi asociado seráfico en la preparación de esta exposición, es un serafín supremo y anteriormente sólo ha ejercido su actividad como serafín supremo. Gracias a su aplicación y a su servicio dedicado, ha llevado a cabo uno tras otro los siete servicios seráficos, ha trabajado en casi todos los campos de actividad abiertos a un serafín, y actualmente tiene el cargo de jefe asociado de los serafines en Urantia.

\par
%\textsuperscript{(426.10)}
\textsuperscript{39:0.10} A los seres humanos a veces les resulta difícil comprender que una capacidad creada para realizar un ministerio de nivel superior no implica necesariamente la aptitud para trabajar en niveles de servicio relativamente inferiores. El hombre inicia su vida como un niño indefenso; por eso cada logro humano debe contener todos los requisitos previos experienciales; los serafines no tienen esta vida preadulta ---no tienen infancia. Sin embargo, son criaturas experienciales, y por medio de la experiencia y a través de una educación adicional pueden acrecentar la dotación de sus aptitudes divinas e inherentes, adquiriendo experiencialmente la habilidad funcional en uno o más servicios seráficos.

\par
%\textsuperscript{(426.11)}
\textsuperscript{39:0.11} Después de entrar en servicio, los serafines son destinados a las reservas de su grupo inherente. Aquellos que poseen la posición planetaria de administradores sirven a menudo durante largos períodos según su clasificación original, pero cuanto más elevado es el nivel inherente de actividad de los ministros angélicos, con más perseverancia buscan ser destinados a los tipos inferiores de servicio universal. Desean ser destinados especialmente a las reservas de los ayudantes planetarios y, si lo consiguen, se inscriben en las escuelas celestiales vinculadas a la sede del Príncipe Planetario de algún mundo evolutivo. Allí empiezan el estudio de los idiomas, la historia y las costumbres locales de las razas de la humanidad. Los serafines tienen que adquirir el conocimiento y conseguir la experiencia en gran medida como lo hacen los seres humanos. No están muy alejados de vosotros en ciertos atributos de la personalidad. Todos anhelan empezar desde el fondo, en el nivel de ministerio más bajo posible; así pueden esperar alcanzar el nivel más elevado posible de destino experiencial.

\section*{1. Los Serafines supremos}
\par
%\textsuperscript{(427.1)}
\textsuperscript{39:1.1} De las siete órdenes reveladas de ángeles del universo local, estos serafines son los más elevados. Desempeñan su actividad en siete grupos, cada uno de los cuales está estrechamente asociado con los ministros angélicos del Cuerpo Seráfico de la Finalización.

\par
%\textsuperscript{(427.2)}
\textsuperscript{39:1.2} 1. \textit{Los Ministros del Hijo-Espíritu}. El primer grupo de serafines supremos está asignado al servicio de los Hijos elevados y de los seres con origen en el Espíritu que residen y actúan en el universo local. Este grupo de ministros angélicos sirve también al Hijo del Universo y al Espíritu del Universo, y está estrechamente asociado con el cuerpo de información de la Radiante Estrella Matutina, el jefe ejecutivo universal de las voluntades unidas del Hijo Creador y del Espíritu Creativo.

\par
%\textsuperscript{(427.3)}
\textsuperscript{39:1.3} Puesto que están asignados a los Hijos y a los Espíritus elevados, estos serafines se encuentran asociados por naturaleza con los extensos servicios de los Avonales del Paraíso, los descendientes divinos del Hijo Eterno y del Espíritu Infinito. En todas sus misiones magistrales y donadoras, los Avonales del Paraíso siempre están asistidos por esta orden de serafines elevada y experimentada, que en tales ocasiones se dedican a organizar y a administrar el trabajo especial relacionado con la finalización de una dispensación planetaria y con la inauguración de una nueva era. Pero no se ocupan de la tarea de juzgar, que puede o no formar parte de estos cambios de dispensación.

\par
%\textsuperscript{(427.4)}
\textsuperscript{39:1.4} \textit{Los Asistentes de las Donaciones}. Cuando realizan una misión donadora, los Avonales del Paraíso, pero no los Hijos Creadores, siempre van acompañados de un cuerpo de 144 asistentes de la donación. Estos 144 ángeles son los jefes de todos los otros ministros procedentes del Hijo y del Espíritu que pueden estar asociados a una misión de donación. Puede haber legiones de ángeles sometidas al mando de un Hijo encarnado de Dios en una donación planetaria, pero todos estos serafines estarán organizados y dirigidos por los 144 asistentes de la donación. Las órdenes superiores de ángeles, los supernafines y los seconafines, también pueden formar parte de la hueste acompañante, y aunque sus misiones sean distintas a las de los serafines, todas estas actividades estarían coordinadas por los asistentes de la donación.

\par
%\textsuperscript{(427.5)}
\textsuperscript{39:1.5} Estos asistentes de las donaciones son serafines consumados; todos han atravesado los círculos de Serafington y han alcanzado el Cuerpo Seráfico de la Finalización. Y además han sido especialmente entrenados para hacer frente a las dificultades y para enfrentarse con las urgencias asociadas a las donaciones de los Hijos de Dios para el progreso de los hijos del tiempo. Todos estos serafines han alcanzado el Paraíso y el abrazo personal de la Fuente-Centro Segunda, el Hijo Eterno.

\par
%\textsuperscript{(427.6)}
\textsuperscript{39:1.6} Los serafines anhelan igualmente ser destinados a las misiones de los Hijos encarnados y estar vinculados como guardianes del destino a los mortales de los reinos; esta última tarea es el pasaporte seráfico más seguro para el Paraíso, mientras que los asistentes de las donaciones han realizado el servicio más elevado del universo local entre los serafines consumados que han alcanzado el Paraíso.

\par
%\textsuperscript{(428.1)}
\textsuperscript{39:1.7} 2. \textit{Los Asesores de los Tribunales}. Son los asesores y los ayudantes seráficos vinculados a todas las órdenes de seres relacionadas con los juicios, desde los conciliadores hasta los tribunales más elevados del reino. El propósito de estos tribunales no es determinar las sentencias punitivas, sino más bien juzgar las honradas diferencias de opinión y decretar la supervivencia eterna de los mortales ascendentes. El deber de los asesores de los tribunales consiste en asegurarse de que todos los cargos contra las criaturas mortales sean expuestos con justicia y juzgados con misericordia. En esta tarea están estrechamente asociados con los Altos Comisionados, los mortales ascendentes fusionados con el Espíritu que sirven en el universo local.

\par
%\textsuperscript{(428.2)}
\textsuperscript{39:1.8} Los asesores seráficos de los tribunales sirven ampliamente como defensores de los mortales. No es que exista ninguna predisposición a ser injustos con las humildes criaturas de los reinos, pero mientras que la justicia exige el juicio de todas las faltas durante la ascensión hacia la perfección divina, la misericordia requiere que cada paso en falso sea juzgado con equidad de acuerdo con la naturaleza de la criatura y con el propósito divino. Estos ángeles exponen y demuestran con el ejemplo el elemento de misericordia inherente a la justicia divina ---la equidad basada en el conocimiento de los hechos subyacentes en los móviles personales y en las tendencias raciales.

\par
%\textsuperscript{(428.3)}
\textsuperscript{39:1.9} Esta orden de ángeles sirve desde los consejos de los Príncipes Planetarios hasta los tribunales más elevados del universo local, mientras que sus asociados del Cuerpo Seráfico de la Finalización ejercen su actividad en los reinos superiores de Orvonton, e incluso en los tribunales de los Ancianos de los Días de Uversa.

\par
%\textsuperscript{(428.4)}
\textsuperscript{39:1.10} 3. \textit{Los Orientadores Universales}. Son los verdaderos amigos y consejeros de todas las criaturas ascendentes que ya se han graduado y que se detienen por última vez en Salvington, en su universo de origen, cuando están a punto de emprender la aventura espiritual que se extiende por delante de ellos en el inmenso superuniverso de Orvonton. En tales momentos muchos ascendentes tienen un sentimiento que los mortales sólo pueden comprender comparándolo con la emoción humana de la nostalgia. Detrás de ellos se encuentran los reinos que han alcanzado, los reinos que se han vuelto familiares mediante el largo servicio y la consecución morontial; delante de ellos se extiende el misterio desafiante de un universo aún más grande y más inmenso.

\par
%\textsuperscript{(428.5)}
\textsuperscript{39:1.11} Los orientadores universales tienen la tarea de facilitar el paso de los peregrinos ascendentes desde los niveles que han alcanzado hasta los niveles no alcanzados de servicio universal, de ayudar a estos peregrinos a efectuar, en la comprensión de los significados y los valores, los ajustes caleidoscópicos inherentes al hecho de saber que un ser espiritual de la primera fase no se encuentra al final y en el punto culminante de la ascensión morontial del universo local, sino más bien en el punto más bajo de la larga escalera de la ascensión espiritual hacia el Padre Universal en el Paraíso.

\par
%\textsuperscript{(428.6)}
\textsuperscript{39:1.12} Muchos graduados de Serafington, miembros del Cuerpo Seráfico de la Finalización que están asociados con estos serafines, se dedican intensamente a enseñar en ciertas escuelas de Salvington relacionadas con la preparación de las criaturas de Nebadon para las relaciones de la próxima era del universo.

\par
%\textsuperscript{(428.7)}
\textsuperscript{39:1.13} 4. \textit{Los Consejeros Docentes}. Estos ángeles son los ayudantes inapreciables del cuerpo docente espiritual del universo local. Los consejeros docentes son los secretarios de todas las órdenes de instructores, desde los Melquisedeks y los Hijos Instructores Trinitarios hasta los mortales morontiales que están destinados como ayudantes de aquellos de su misma especie que se encuentran justo detrás de ellos en la escala de la vida ascendente. \textit{Veréis} por primera vez a estos serafines docentes asociados en alguno de los siete mundos de las mansiones que rodean a Jerusem.

\par
%\textsuperscript{(428.8)}
\textsuperscript{39:1.14} Estos serafines se convierten en los asociados de los jefes de división de las numerosas instituciones educativas y formativas de los universos locales, y están destinados en gran número a las facultades de los siete mundos formativos de los sistemas locales y de las setenta esferas educativas de las constelaciones. Estos ministerios se extienden hacia abajo hasta los mundos individuales. Incluso los auténticos educadores consagrados del tiempo reciben la ayuda de estos consejeros de los serafines supremos, y a menudo están acompañados por ellos.

\par
%\textsuperscript{(429.1)}
\textsuperscript{39:1.15} La cuarta donación del Hijo Creador bajo la forma de una criatura tuvo lugar en la similitud de un consejero docente de los serafines supremos de Nebadon.

\par
%\textsuperscript{(429.2)}
\textsuperscript{39:1.16} 5. \textit{Los Directores de la Asignación}. Los ángeles que sirven en las esferas evolutivas y arquitectónicas habitadas por las criaturas eligen de vez en cuando a un cuerpo de 144 serafines supremos. Éste es el consejo angélico más elevado de cualquier esfera, y coordina las fases autónomas del servicio y de la asignación seráficos. Estos ángeles presiden todas las asambleas seráficas relacionadas con la línea del deber o el llamamiento a la adoración.

\par
%\textsuperscript{(429.3)}
\textsuperscript{39:1.17} 6. \textit{Los Registradores}. Son los registradores oficiales que trabajan para los serafines supremos. Muchos de estos ángeles elevados nacieron con sus dones plenamente desarrollados; otros se han capacitado para sus puestos de confianza y de responsabilidad aplicándose diligentemente al estudio y realizando fielmente deberes similares mientras estaban vinculados a órdenes más humildes o con menos responsabilidades.

\par
%\textsuperscript{(429.4)}
\textsuperscript{39:1.18} 7. \textit{Los Ministros Disponibles}. Una gran cantidad de serafines disponibles de la orden suprema sirven por su cuenta en las esferas arquitectónicas y en los planetas habitados. Estos ministros satisfacen voluntariamente el diferencial de demanda existente para conseguir los servicios de los serafines supremos, formando así la reserva general de esta orden.

\section*{2. Los Serafines Superiores}
\par
%\textsuperscript{(429.5)}
\textsuperscript{39:2.1} Los serafines superiores no reciben este nombre porque sean en algún sentido cualitativamente superiores a las otras órdenes de ángeles, sino porque están a cargo de las actividades superiores de un universo local. Muchos miembros de los dos primeros grupos de este cuerpo seráfico son serafines porque lo han conseguido, son ángeles que han servido en todas las fases formativas y que han regresado para realizar una tarea glorificada como directores de los seres de su misma especie en las esferas de sus actividades iniciales. Como Nebadon es un universo joven, no tiene muchos ángeles de esta orden.

\par
%\textsuperscript{(429.6)}
\textsuperscript{39:2.2} Los serafines superiores ejercen su actividad en los siete grupos siguientes:

\par
%\textsuperscript{(429.7)}
\textsuperscript{39:2.3} 1. \textit{El Cuerpo de Información}. Estos serafines pertenecen al estado mayor personal de Gabriel, la Radiante Estrella Matutina. Recorren el universo local reuniendo la información de los reinos para su buen gobierno en los consejos de Nebadon. Son el cuerpo de información de las poderosas huestes que Gabriel preside como vicegerente del Hijo Maestro. Estos serafines no están vinculados directamente ni a los sistemas ni a las constelaciones, y su información llega directamente a Salvington por un circuito continuo, directo e independiente.

\par
%\textsuperscript{(429.8)}
\textsuperscript{39:2.4} Los cuerpos de información de los diversos universos locales pueden intercomunicarse, y de hecho lo hacen, pero sólo dentro de un superuniverso dado. Existe un diferencial de energía que separa eficazmente los asuntos y las operaciones de los diversos supergobiernos. Generalmente, un superuniverso sólo se puede comunicar con otro superuniverso a través de las disposiciones y las instalaciones de la cámara distribuidora de información del Paraíso.

\par
%\textsuperscript{(430.1)}
\textsuperscript{39:2.5} 2. \textit{La Voz de la Misericordia}. La misericordia es la tónica del servicio seráfico y del ministerio angélico. Por eso es justo que exista un cuerpo de ángeles que describa la misericordia de una manera especial. Estos serafines son los verdaderos ministros de la misericordia en los universos locales. Son los guías inspirados que fomentan los impulsos superiores y las emociones más sagradas de los hombres y de los ángeles. Actualmente, los directores de estas legiones siempre son serafines consumados que son también los guardianes graduados del destino de los mortales; es decir, que cada pareja angélica ha guiado al menos a un alma de origen animal durante su vida en la carne, ha atravesado posteriormente los círculos de Serafington y ha sido enrolada en el Cuerpo Seráfico de la Finalización.

\par
%\textsuperscript{(430.2)}
\textsuperscript{39:2.6} 3. \textit{Los Coordinadores Espirituales}. El tercer grupo de serafines superiores tiene su base en Salvington, pero ejerce su actividad en el universo local en cualquier parte donde pueda prestar un servicio beneficioso. Aunque sus tareas son esencialmente espirituales y sobrepasan por tanto la comprensión real de la mente humana, quizás captéis una parte de su ministerio hacia los mortales si os explicamos que a estos ángeles se les ha confiado la tarea de preparar a los ascendentes que residen en Salvington para su última transición en el universo local ---desde el nivel morontial más elevado hasta el estado de seres espirituales recién nacidos. Al igual que los planificadores de la mente ayudan a las criaturas supervivientes en los mundos de las mansiones a adaptarse a los potenciales de la mente morontial y a utilizarlos eficazmente, estos serafines instruyen a los graduados morontiales en Salvington acerca de las capacidades recién adquiridas de la mente espiritual. Y sirven a los mortales ascendentes de otras muchas maneras.

\par
%\textsuperscript{(430.3)}
\textsuperscript{39:2.7} 4. \textit{Los Educadores Asistentes}. Los educadores asistentes son los ayudantes y asociados de sus compañeros serafines, los consejeros docentes. También están relacionados individualmente con las extensas empresas educativas del universo local, en especial con el séptuple programa de formación que está en vigor en los mundos de las mansiones de los sistemas locales. Un maravilloso cuerpo de esta orden de serafines ejerce su actividad en Urantia con el objeto de favorecer y fomentar la causa de la verdad y la rectitud.

\par
%\textsuperscript{(430.4)}
\textsuperscript{39:2.8} 5. \textit{Los Transportadores}. Todos los grupos de espíritus ministrantes tienen sus cuerpos de transporte, sus órdenes angélicas dedicadas al ministerio de transportar a aquellas personalidades que son incapaces de viajar por sí mismas de una esfera a otra. El quinto grupo de serafines superiores tiene su sede en Salvington y presta sus servicios atravesando el espacio desde, y hacia, la sede del universo local. Al igual que otras subdivisiones de los serafines superiores, algunos de estos ángeles fueron creados como tales mientras que otros se han elevado partiendo de los grupos inferiores o menos dotados.

\par
%\textsuperscript{(430.5)}
\textsuperscript{39:2.9} El <<alcance energético>> de los serafines es enteramente adecuado para las necesidades del universo local e incluso del superuniverso, pero nunca podrían resistir las exigencias energéticas implicadas en un viaje tan largo como el de Uversa hasta Havona. Un viaje tan agotador requiere los poderes especiales de un seconafín primario dotado para el transporte. Los transportadores se recargan de energía para volar mientras están en tránsito, y recuperan su fuerza personal al final del viaje.

\par
%\textsuperscript{(430.6)}
\textsuperscript{39:2.10} Los mortales ascendentes no poseen formas personales de tránsito ni siquiera en Salvington. Los ascendentes tienen que depender de los transportes seráficos para avanzar de un mundo a otro hasta después del último sueño de descanso en el círculo interior de Havona y del despertar eterno en el Paraíso. Después de esto ya no dependeréis de los ángeles para transportaros de un universo a otro.

\par
%\textsuperscript{(430.7)}
\textsuperscript{39:2.11} El proceso de estar enserafinado no es muy diferente a la experiencia de la muerte o del sueño, salvo que en el sueño de tránsito hay un elemento temporal automático. Estáis conscientemente inconscientes durante el descanso seráfico. Pero el Ajustador del Pensamiento está plena y totalmente consciente, de hecho es excepcionalmente eficaz, puesto que sois incapaces de oponeros, resistir o dificultar de otras maneras su trabajo creativo y transformador.

\par
%\textsuperscript{(431.1)}
\textsuperscript{39:2.12} Cuando sois enserafinados, os dormís durante un período concreto y os despertáis en el momento indicado. Durante el sueño de tránsito la duración del viaje es indiferente. No os dais directamente cuenta del paso del tiempo. Es como si os durmierais en un vehículo de transporte en una ciudad, y después de haber dormido tranquilamente toda la noche, os despertarais en otra metrópolis lejana. Habéis viajado mientras dormíais. Así pues, alzáis el vuelo por el espacio, enserafinados, mientras descansáis ---mientras dormís. El sueño de tránsito es provocado por la unión entre los Ajustadores y los transportadores seráficos.

\par
%\textsuperscript{(431.2)}
\textsuperscript{39:2.13} Los ángeles no pueden transportar los cuerpos combustibles ---de carne y hueso--- tales como los que tenéis ahora, pero pueden transportar todos los demás, desde las formas morontiales inferiores hasta las formas espirituales más elevadas. No actúan en caso de muerte natural. Cuando termináis vuestra carrera terrestre, vuestro cuerpo se queda en este planeta. Vuestro Ajustador del Pensamiento se dirige al seno del Padre, y estos ángeles no se ocupan directamente de la reconstitución posterior de vuestra personalidad en el mundo de identificación de las mansiones. Allí, vuestro nuevo cuerpo es una forma morontial, una forma que puede ser enserafinada. <<Sembráis un cuerpo mortal>> en la tumba, y <<cosecháis una forma morontial>>\footnote{\textit{Sembráis mortalidad, recogéis morontia}: 1 Co 15:44.} en los mundos de las mansiones.

\par
%\textsuperscript{(431.3)}
\textsuperscript{39:2.14} 6. \textit{Los Registradores}. Estas personalidades se ocupan especialmente de recibir, archivar y volver a enviar los registros de Salvington y de sus mundos asociados. Sirven también como registradores especiales para los grupos residentes de personalidades superiores y del superuniverso, y como actuarios de los tribunales de Salvington y secretarios de sus dirigentes.

\par
%\textsuperscript{(431.4)}
\textsuperscript{39:2.15} \textit{Los Transmisores} ---receptores y emisores--- son una subdivisión especializada de los registradores seráficos, y se ocupan de enviar los registros y de diseminar la información esencial. Su trabajo es de tipo elevado, pues manejan tal cantidad de circuitos que 144.000 mensajes pueden atravesar simultáneamente las mismas líneas de energía. Adaptan las técnicas ideográficas superiores de los jefes registradores superáficos, y con estos símbolos comunes mantienen un contacto recíproco tanto con los coordinadores de la información de los supernafines terciarios como con los coordinadores glorificados de la información del Cuerpo Seráfico de la Finalización.

\par
%\textsuperscript{(431.5)}
\textsuperscript{39:2.16} Los registradores seráficos de la orden superior efectúan así una estrecha unión con el cuerpo de información de su propia orden y con todos los registradores subordinados, mientras que las transmisiones les permiten mantener una comunicación constante con los registradores superiores del superuniverso y, a través de este canal, con los registradores de Havona y con los custodios del conocimiento situados en el Paraíso. Muchos miembros de la orden superior de los registradores son serafines ascendidos que habían realizado tareas similares en las secciones inferiores del universo.

\par
%\textsuperscript{(431.6)}
\textsuperscript{39:2.17} 7. \textit{Las Reservas}. En Salvington se mantienen numerosas reservas de todos los tipos de serafines superiores, disponibles instantáneamente para ser enviados hasta los mundos más alejados de Nebadon cuando son solicitados por los directores de las asignaciones o a petición de los administradores del universo. Las reservas de los serafines superiores también proporcionan ayudantes mensajeros a petición del jefe de las Brillantes Estrellas Vespertinas, el cual está encargado de custodiar y de enviar todas las comunicaciones personales. Un universo local está plenamente provisto de los medios de intercomunicación adecuados, pero siempre hay un residuo de mensajes que es preciso enviar por medio de mensajeros personales.

\par
%\textsuperscript{(432.1)}
\textsuperscript{39:2.18} En los mundos seráficos de Salvington se mantienen las reservas básicas para todo el universo local. Este cuerpo incluye a todos los tipos de todos los grupos de ángeles.

\section*{3. Los Serafines Supervisores}
\par
%\textsuperscript{(432.2)}
\textsuperscript{39:3.1} Esta polifacética orden de ángeles del universo está destinada al servicio exclusivo de las constelaciones. Estos hábiles ministros tienen sus sedes en las capitales de las constelaciones, pero ejercen su actividad en todo Nebadon en beneficio de los reinos que les están asignados.

\par
%\textsuperscript{(432.3)}
\textsuperscript{39:3.2} 1. \textit{Los Asistentes Supervisores}. La primera orden de serafines supervisores está destinada al trabajo colectivo de los Padres de las Constelaciones, y siempre ayudan de manera eficaz a los Altísimos. Estos serafines se ocupan principalmente de unificar y de estabilizar toda una constelación.

\par
%\textsuperscript{(432.4)}
\textsuperscript{39:3.3} 2. \textit{Los Pronosticadores de la Ley}. El fundamento intelectual de la justicia es la ley, y en un universo local la ley tiene su origen en las asambleas legislativas de las constelaciones. Estos cuerpos deliberativos codifican y promulgan oficialmente las leyes fundamentales de Nebadon, unas leyes destinadas a proporcionar el máximo de coordinación posible de toda una constelación de acuerdo con la política fija de no violar el libre albedrío moral de las criaturas personales. La segunda orden de serafines supervisores tiene la tarea de presentar ante los legisladores de la constelación un pronóstico sobre la manera en que un decreto propuesto afectaría a la vida de las criaturas dotadas de libre albedrío. Están bien cualificados para realizar este servicio en virtud de su larga experiencia en los sistemas locales y en los mundos habitados. Estos serafines no pretenden favorecer especialmente a un grupo o a otro, pero comparecen ante los legisladores celestiales para hablar en nombre de aquellos que no pueden estar presentes para hablar por sí mismos. Incluso el hombre mortal puede contribuir a la evolución de la ley universal, pues estos mismos serafines describen plena y fielmente, no necesariamente los deseos transitorios y conscientes del hombre, sino más bien los verdaderos anhelos del hombre interior, del alma morontial evolutiva del mortal material que reside en los mundos del espacio.

\par
%\textsuperscript{(432.5)}
\textsuperscript{39:3.4} 3. \textit{Los Arquitectos Sociales}. Estos serafines trabajan desde los planetas individuales hasta los mundos formativos morontiales para intensificar todos los contactos sociales sinceros y para fomentar la evolución social de las criaturas del universo. Son los ángeles que tratan de despojar a las asociaciones de seres inteligentes de toda artificialidad, esforzándose al mismo tiempo por facilitar la interasociación de las criaturas volitivas sobre la base de una verdadera comprensión de sí mismo y de un aprecio mutuo sincero.

\par
%\textsuperscript{(432.6)}
\textsuperscript{39:3.5} Los arquitectos sociales hacen todo lo que está dentro de su campo y de sus posibilidades para reunir a los individuos compatibles con el fin de que puedan formar grupos de trabajo eficaces y agradables en la Tierra; y a veces estos grupos se han asociado de nuevo en los mundos de las mansiones para continuar su fructífero servicio. Pero estos serafines no siempre consiguen sus objetivos; no siempre son capaces de reunir a aquellos que podrían formar el grupo más ideal para conseguir un objetivo dado o realizar una tarea determinada; en estas condiciones, tienen que utilizar el mejor material disponible.

\par
%\textsuperscript{(432.7)}
\textsuperscript{39:3.6} Estos ángeles continúan su ministerio en los mundos de las mansiones y en los mundos morontiales superiores. Se ocupan de todas las tareas que tienen que ver con el progreso en los mundos morontiales y que afectan a tres o más personas. Cuando dos seres trabajan juntos, se considera que lo hacen sobre la base del emparejamiento, la complementariedad o la asociación, pero cuando tres o más seres están agrupados para realizar un servicio, constituyen un problema social y, por consiguiente, caen dentro de la jurisdicción de los arquitectos sociales. En Edentia, estos eficaces serafines están organizados en setenta divisiones, y estas divisiones aportan su ministerio en los setenta mundos de progreso morontial que rodean a la esfera sede.

\par
%\textsuperscript{(433.1)}
\textsuperscript{39:3.7} 4. \textit{Los Sensibilizadores Éticos}. Estos serafines tienen la misión de fomentar y de promover en las criaturas el crecimiento de la apreciación de la moralidad de las relaciones interpersonales, pues éste es el origen y el secreto del crecimiento continuado e intencional de la sociedad y del gobierno, humano o superhumano. Estos acrecentadores de la apreciación ética actúan en cualquier lugar donde puedan prestar sus servicios como consejeros voluntarios de los gobernantes planetarios y como instructores de intercambio en los mundos formativos de los sistemas. Sin embargo, no caeréis bajo su completa dirección hasta que no alcancéis las escuelas de fraternidad de Edentia, donde estimularán vuestra apreciación por las mismas verdades sobre la fraternidad que en ese momento estaréis explorando con tanta aplicación mediante la experiencia real de vivir con los univitatias en los laboratorios sociales de Edentia, los setenta satélites de la capital de Norlatiadek.

\par
%\textsuperscript{(433.2)}
\textsuperscript{39:3.8} 5. \textit{Los Transportadores}. Los serafines supervisores del quinto grupo trabajan como transportadores de personalidades, trayendo y llevando a los seres a las sedes de las constelaciones. Cuando estos serafines transportadores vuelan de una esfera a otra, son plenamente conscientes de su velocidad, dirección y paradero astronómico. No atraviesan el espacio como lo haría un proyectil inanimado. Pueden pasar los unos cerca de los otros durante su vuelo espacial sin el menor peligro de colisión. Son totalmente capaces de variar la velocidad de su marcha y de alterar la dirección de su vuelo, e incluso de cambiar de destino si sus directores se lo ordenan así en cualquier cruce espacial de los circuitos universales de información.

\par
%\textsuperscript{(433.3)}
\textsuperscript{39:3.9} Estas personalidades de transporte están organizadas de tal manera que pueden utilizar simultáneamente las tres líneas de energía distribuidas por el universo, cada una de las cuales tiene una velocidad espacial neta de 299.790 kilómetros por segundo. Así pues, estos transportadores son capaces de superponer la velocidad de la energía a la velocidad del poder hasta alcanzar, en el transcurso de sus largos viajes, una velocidad media que varía entre 893.000 y casi 900.000 de vuestros kilómetros por segundo de vuestro tiempo. La velocidad es afectada por la masa y la proximidad de la materia vecina y por la intensidad y la dirección de los principales circuitos cercanos de poder universal. Hay numerosos tipos de seres similares a los serafines que pueden atravesar el espacio, y que también son capaces de transportar a otros seres que han sido debidamente preparados.

\par
%\textsuperscript{(433.4)}
\textsuperscript{39:3.10} 6. \textit{Los Registradores}. Los serafines supervisores se la sexta orden actúan como registradores especiales de los asuntos de las constelaciones. Un cuerpo numeroso y eficaz ejerce su actividad en Edentia, la sede de la constelación de Norlatiadek, a la que pertenecen vuestro sistema y vuestro planeta.

\par
%\textsuperscript{(433.5)}
\textsuperscript{39:3.11} 7. \textit{Las Reservas}. Las reservas generales de los serafines supervisores se mantienen en las sedes de las constelaciones. Estos reservistas angélicos no están inactivos en ningún sentido; muchos de ellos sirven como ayudantes mensajeros de los gobernantes de las constelaciones; otros están vinculados a las reservas de los Vorondadeks sin destino estacionados en Salvington; y otros aún pueden estar vinculados a los Hijos Vorondadeks encargados de una misión especial, tales como el observador Vorondadek, y a veces Altísimo regente, de Urantia.

\section*{4. Los Serafines Administradores}
\par
%\textsuperscript{(434.1)}
\textsuperscript{39:4.1} La cuarta orden de serafines está asignada a las tareas administrativas de los sistemas locales. Son nativos de las capitales de los sistemas pero están estacionados en gran número en las esferas de las mansiones y morontiales y en los mundos habitados. Los serafines de la cuarta orden están dotados por naturaleza de una capacidad administrativa excepcional. Son los hábiles ayudantes de los directores de las divisiones inferiores del gobierno universal de un Hijo Creador, y se ocupan principalmente de los asuntos de los sistemas locales y de los mundos que los componen. Están organizados para el servicio de la manera siguiente:

\par
%\textsuperscript{(434.2)}
\textsuperscript{39:4.2} 1. \textit{Los Asistentes Administrativos}. Estos hábiles serafines son los asistentes directos del Soberano de un Sistema, de un Hijo Lanonandek primario. Son unos ayudantes inapreciables para llevar a cabo los complicados detalles del trabajo ejecutivo de la sede de un sistema. Sirven también como agentes personales de los gobernantes de los sistemas, y viajan en gran número de un sitio para otro a los diversos mundos de transición y a los planetas habitados, cumpliendo numerosos cometidos por el bienestar del sistema y por los intereses físicos y biológicos de sus mundos habitados.

\par
%\textsuperscript{(434.3)}
\textsuperscript{39:4.3} Estos mismos administradores seráficos también están vinculados a los gobiernos de los soberanos de los mundos, los Príncipes Planetarios. La mayoría de los planetas de un universo dado se encuentran bajo la jurisdicción de un Hijo Lanonandek secundario, pero en ciertos mundos, como sucedió en Urantia, el plan divino ha fracasado. En caso de deserción de un Príncipe Planetario, estos serafines quedan vinculados a los síndicos Melquisedeks y a sus sucesores en la autoridad planetaria. El gobernante en funciones actual de Urantia tiene la ayuda de un cuerpo de mil miembros de esta polifacética orden de serafines.

\par
%\textsuperscript{(434.4)}
\textsuperscript{39:4.4} 2. \textit{Los Guías de la Justicia}. Son los ángeles que presentan el resumen de las pruebas relacionadas con el bienestar eterno de los hombres y de los ángeles cuando estos asuntos se someten a juicio en los tribunales de un sistema o de un planeta. Preparan las declaraciones para todas las audiencias preliminares donde está implicada la supervivencia de los mortales, unas declaraciones que se presentan posteriormente, con los informes de estos casos, ante los tribunales superiores del universo y del superuniverso. En todos los casos en que la supervivencia es dudosa, la defensa es preparada por estos serafines que poseen una comprensión perfecta de todos los detalles de cada característica de cada cargo que figura en las acusaciones presentadas por los administradores de la justicia universal.

\par
%\textsuperscript{(434.5)}
\textsuperscript{39:4.5} Estos ángeles no tienen la misión de vencer o de retrasar la justicia, sino más bien de asegurar que una justicia infalible llena de generosa misericordia se aplicará con equidad a todas las criaturas. Estos serafines ejercen a menudo sus funciones en los mundos locales, apareciendo con frecuencia ante los tríos arbitrales de las comisiones conciliadoras ---los tribunales que juzgan los malentendidos menores. Muchos ángeles que han servido en otro tiempo como guías de la justicia en los mundos inferiores, aparecen más tarde como Voces de la Misericordia en las esferas superiores y en Salvington.

\par
%\textsuperscript{(434.6)}
\textsuperscript{39:4.6} Muy pocos guías de la justicia se perdieron durante la rebelión de Lucifer en Satania, pero más de una cuarta parte de los otros serafines administradores y de las órdenes inferiores de ministros seráficos se engañaron y se descarriaron a causa de los sofismas de una libertad personal desenfrenada.

\par
%\textsuperscript{(434.7)}
\textsuperscript{39:4.7} 3. \textit{Los Intérpretes de la Ciudadanía Cósmica}. Cuando los mortales ascendentes han terminado su formación en los mundos de las mansiones, su primer aprendizaje como estudiantes en la carrera universal, se les permite disfrutar de las satisfacciones pasajeras de una madurez relativa ---de la ciudadanía en la capital del sistema. Aunque la conquista de cada meta ascendente es un logro objetivo, en un sentido más amplio estas metas no son más que hitos en el largo sendero ascendente hacia el Paraíso. Pero por muy relativos que sean estos éxitos, a ninguna criatura evolutiva se le niega nunca la satisfacción completa, aunque transitoria, de haber alcanzado una meta. De vez en cuando hay una pausa en la ascensión al Paraíso, un corto respiro, durante el cual los horizontes universales permanecen inmóviles, el estado de la criatura es estacionario, y la personalidad saborea el dulzor de haber alcanzado una meta.

\par
%\textsuperscript{(435.1)}
\textsuperscript{39:4.8} El primero de estos períodos en la carrera de un ascendente mortal tiene lugar en la capital de un sistema local. Durante esta pausa, y como ciudadanos de Jerusem, intentaréis expresar en vuestra vida como criaturas aquellas cosas que habréis adquirido durante las ocho experiencias de vida anteriores ---que abarcan Urantia y los siete mundos de las mansiones.

\par
%\textsuperscript{(435.2)}
\textsuperscript{39:4.9} Los intérpretes seráficos de la ciudadanía cósmica guían a los nuevos ciudadanos de las capitales de los sistemas y estimulan su apreciación de las responsabilidades del gobierno de un universo. Estos serafines también están estrechamente asociados con los Hijos Materiales en la administración de los sistemas, mientras describen la responsabilidad y la moralidad de la ciudadanía cósmica a los mortales materiales de los mundos habitados.

\par
%\textsuperscript{(435.3)}
\textsuperscript{39:4.10} 4. \textit{Los Estimuladores de la Moralidad}. En los mundos de las mansiones empezáis a aprender el dominio de vosotros mismos en beneficio de todos los interesados. Vuestra mente aprende a cooperar, aprende la manera de hacer planes con otros seres más sabios. En la sede del sistema, los educadores seráficos estimularán aún más vuestra apreciación de la moralidad cósmica ---de las interacciones entre la libertad y la lealtad.

\par
%\textsuperscript{(435.4)}
\textsuperscript{39:4.11} ¿Qué es la lealtad? Es el fruto de una apreciación inteligente de la fraternidad universal; uno no puede recibir mucho sin dar nada. A medida que ascendéis la escala de la personalidad, primero aprendéis a ser leales, luego a amar, después a ser filiales, y entonces podréis ser libres; pero hasta que no seáis finalitarios, hasta que no hayáis alcanzado la perfección de la lealtad, no podréis daros cuenta por vosotros mismos de la finalidad de la libertad.

\par
%\textsuperscript{(435.5)}
\textsuperscript{39:4.12} Estos serafines enseñan lo fructífera que es la paciencia; que el estancamiento es la muerte segura, pero que el crecimiento excesivamente rápido es igualmente suicida; que al igual que una gota de agua cae desde un nivel más alto hasta uno más bajo, y corriendo hacia adelante desciende continuamente a través de una sucesión de pequeñas caídas, así es siempre el progreso hacia arriba en los mundos morontiales y espirituales ---igual de lento y mediante las mismas etapas graduales.

\par
%\textsuperscript{(435.6)}
\textsuperscript{39:4.13} Los estimuladores de la moralidad describen la vida mortal a los mundos habitados como una cadena ininterrumpida de muchos eslabones. Vuestra corta estancia en Urantia, en esta esfera de infancia humana, sólo es un simple eslabón, el primero de la larga cadena que ha de extenderse a través de los universos y de las eras eternas. No se trata tanto de lo que aprendáis en esta primera vida; lo importante es la experiencia de vivir esta vida. Incluso el \textit{trabajo} en este mundo, por muy importante que sea, no es ni mucho menos tan importante como la \textit{manera} de hacerlo. No existe ninguna recompensa material para una vida recta, pero existe una profunda satisfacción ---una conciencia de haberlo logrado--- y ésta trasciende cualquier recompensa material imaginable.

\par
%\textsuperscript{(435.7)}
\textsuperscript{39:4.14} Las llaves del reino\footnote{\textit{Llaves del reino}: Mt 16:19.} de los cielos son la sinceridad, más sinceridad y aún más sinceridad. Todos los hombres poseen estas llaves. Los hombres las utilizan ---elevan su estado espiritual--- mediante sus decisiones, más decisiones y aún más decisiones. La elección moral más elevada consiste en elegir el valor más elevado posible, y ésta siempre consiste ---en cualquier esfera, y en todas ellas--- en elegir hacer la voluntad de Dios. Si el hombre elige hacerla, \textit{es} grande, aunque sea el ciudadano más humilde de Jerusem o incluso el mortal más insignificante de Urantia.

\par
%\textsuperscript{(436.1)}
\textsuperscript{39:4.15} 5. \textit{Los Transportadores}. Son los serafines de transporte que ejercen su actividad en los sistemas locales. En vuestro sistema de Satania llevan a los pasajeros desde Jerusem a un sitio y a otro, y sirven de otras maneras como transportadores interplanetarios. Es raro que pase un solo día sin que un serafín transportador de Satania no deposite en las orillas de Urantia a algún visitante estudiantil o a algún otro viajero de naturaleza espiritual o semiespiritual. Estos mismos ángeles que recorren el espacio os llevarán y traerán algún día entre los diversos mundos del grupo sede del sistema, y cuando hayáis terminado vuestra tarea en Jerusem, os llevarán hacia adelante hasta Edentia. Pero en ninguna circunstancia os llevarán hacia atrás al mundo de vuestro origen humano. Un mortal no regresa nunca a su planeta natal durante la dispensación de su existencia temporal, y si sucede que regresa durante una dispensación posterior, estaría acompañado por un serafín transportador del grupo perteneciente a la sede del universo.

\par
%\textsuperscript{(436.2)}
\textsuperscript{39:4.16} 6. \textit{Los Registradores}. Estos serafines son los guardianes de los archivos triples de los sistemas locales. El templo de los archivos situado en la capital de un sistema es una estructura única; un tercio es material y está construido con metales y cristales luminosos; un tercio es morontial y está fabricado con la unión de la energía espiritual y material pero que se salen del campo de la visión humana; y un tercio es espiritual. Los registradores de esta orden dirigen y mantienen este triple sistema de archivos. Los mortales ascendentes consultarán al principio los archivos materiales, los Hijos Materiales y los seres de transición más elevados consultan los de las salas morontiales, mientras que los serafines y las personalidades espirituales superiores del reino examinan los archivos de la sección espiritual.

\par
%\textsuperscript{(436.3)}
\textsuperscript{39:4.17} 7. \textit{Las Reservas}. El cuerpo de reserva de los serafines administradores situado en Jerusem pasa una gran parte de su tiempo de espera conversando, como compañeros espirituales, con los mortales ascendentes recién llegados de los diversos mundos del sistema ---los graduados acreditados de los mundos de las mansiones. Uno de los encantos de vuestra estancia en Jerusem consistirá en hablar y conversar, durante vuestros períodos de descanso, con estos serafines del cuerpo de reserva en espera, que han viajado mucho y son muy experimentados.

\par
%\textsuperscript{(436.4)}
\textsuperscript{39:4.18} Estas relaciones amistosas son precisamente las que hacen que los mortales ascendentes se encariñen tanto con la capital de un sistema. En Jerusem encontraréis entremezclados por primera vez a los Hijos Materiales, los ángeles y los peregrinos ascendentes. Aquí fraternizan los seres totalmente espirituales y semiespirituales con los individuos que acaban de salir de la existencia material. Las formas mortales están aquí tan modificadas y el campo humano de reacción a la luz tan ampliado, que todos son capaces de disfrutar del hecho de reconocerse mutuamente y de comprender con simpatía la personalidad del otro.

\section*{5. Los Ayudantes Planetarios}
\par
%\textsuperscript{(436.5)}
\textsuperscript{39:5.1} Estos serafines mantienen sus sedes en las capitales de los sistemas y, aunque están estrechamente asociados con los ciudadanos adámicos que residen allí, están asignados principalmente al servicio de los Adanes Planetarios, los mejoradores físicos o biológicos de las razas materiales de los mundos evolutivos. El trabajo ministrante de los ángeles se vuelve cada vez más interesante a medida que se acerca a los mundos habitados, a medida que se acerca a los problemas reales que afrontan los hombres y las mujeres del tiempo que se están preparando para intentar alcanzar la meta de la eternidad.

\par
%\textsuperscript{(437.1)}
\textsuperscript{39:5.2} La mayoría de los ayudantes planetarios fueron retirados de Urantia después del derrumbamiento del régimen adámico, y la supervisión seráfica de vuestro mundo recayó en gran parte sobre los administradores, los ministros de transición y los guardianes del destino. Pero estos ayudantes seráficos de vuestros Hijos Materiales negligentes sirven aún a Urantia en los grupos siguientes:

\par
%\textsuperscript{(437.2)}
\textsuperscript{39:5.3} 1. \textit{Las Voces del Jardín}. Cuando el curso planetario de la evolución humana alcanza su nivel biológico más elevado, los Hijos y las Hijas Materiales, los Adanes y las Evas, siempre aparecen para acrecentar la evolución ulterior de las razas mediante la contribución efectiva de su plasma vital superior. La sede planetaria de un Adán y una Eva se denomina generalmente Jardín del Edén, y a sus serafines personales se les conoce a menudo como las <<voces del Jardín>>\footnote{\textit{Voces del Jardín}: Gn 3:8.}. Estos serafines prestan un servicio inapreciable a los Adanes Planetarios en todos sus proyectos dirigidos a elevar física e intelectualmente a las razas evolutivas. Después de la falta adámica en Urantia, a algunos de estos serafines los dejaron en el planeta y fueron asignados a los sucesores de Adán en autoridad.

\par
%\textsuperscript{(437.3)}
\textsuperscript{39:5.4} 2. \textit{Los Espíritus de la Fraternidad}. Cuando un Adán y una Eva llegan a un mundo evolutivo, es evidente que la tarea de conseguir la armonía racial y la cooperación social entre sus diversas razas es de proporciones considerables. Estas razas de colores diferentes y de naturalezas variadas raras veces aceptan con gusto el plan de la fraternidad humana. Estos hombres primitivos sólo llegan a reconocer la sabiduría de la interasociación pacífica como resultado de una experiencia humana madura y gracias al ministerio fiel de los espíritus seráficos de la fraternidad. Sin el trabajo de estos serafines, los esfuerzos de los Hijos Materiales por armonizar y hacer avanzar a las razas de un mundo evolutivo se retrasarían enormemente. Y si vuestro Adán se hubiera adherido al plan original para el avance de Urantia, estos espíritus de la fraternidad ya habrían realizado a estas alturas unas transformaciones increíbles en la raza humana. En vista de la falta adámica, es realmente admirable que estas órdenes seráficas hayan sido capaces de fomentar y de llevar a cabo el grado de fraternidad que disfrutáis actualmente en Urantia.

\par
%\textsuperscript{(437.4)}
\textsuperscript{39:5.5} 3. \textit{Las Almas de la Paz}. Los primeros milenios de esfuerzos ascendentes de los hombres evolutivos están caracterizados por numerosas luchas. La paz no es el estado natural de los reinos materiales. Los mundos llevan a cabo por primera vez <<la paz en la Tierra y la buena voluntad entre los hombres>> gracias al ministerio de las almas seráficas de la paz. Aunque estos ángeles sufrieron muchas frustraciones en sus primeros esfuerzos en Urantia, Vevona, el jefe de las almas de la paz en la época de Adán, fue dejado en Urantia, y ahora está vinculado al estado mayor del gobernador general residente. Cuando Miguel nació, este mismo Vevona es el que, como jefe de las huestes angélicas, anunció a los mundos: <<Gloria a Dios en Havona\footnote{\textit{Gloria a Dios en Havona}: Lc 2:14.} y, en la Tierra, paz y buena voluntad entre los hombres>>\footnote{\textit{Paz y buena voluntad entre los hombres}: Lc 2:14.}.

\par
%\textsuperscript{(437.5)}
\textsuperscript{39:5.6} En las épocas más avanzadas de la evolución planetaria, estos serafines contribuyen a reemplazar, como filosofía de la supervivencia de los mortales, la idea de la expiación por el concepto de la sintonización con lo divino.

\par
%\textsuperscript{(437.6)}
\textsuperscript{39:5.7} 4. \textit{Los Espíritus de la Confianza}.\footnote{\textit{Espíritus de la Confianza}: Jue 9:15.} La desconfianza es la reacción inherente de los hombres primitivos; las luchas por la supervivencia durante los primeros tiempos no engendran de forma natural la confianza. La confianza es una nueva adquisición humana provocada por el ministerio de estos serafines planetarios del régimen adámico. La misión de estos ángeles consiste en inculcar la confianza en la mente de los hombres evolutivos. Los Dioses son muy confiados; el Padre Universal está dispuesto a confiar sin reservas ---bajo la forma de Ajustador--- en la asociación con el hombre.

\par
%\textsuperscript{(438.1)}
\textsuperscript{39:5.8} Todo este grupo de serafines fue transferido al nuevo régimen después de malograrse el plan adámico, y desde entonces han continuado trabajando en Urantia. Y no han fracasado del todo, puesto que actualmente se está desarrollando una civilización que incorpora una gran parte de sus ideales sobre la confianza y la fiabilidad.

\par
%\textsuperscript{(438.2)}
\textsuperscript{39:5.9} En las eras planetarias más avanzadas, estos serafines acrecientan la apreciación humana de la verdad de que la incertidumbre es el secreto de la continuidad satisfecha. Ayudan a los filósofos mortales a comprender que cuando la ignorancia es esencial para conseguir algo, sería un desatino colosal que la criatura conociera el futuro. Realzan el gusto del hombre por el dulzor de la incertidumbre, por el encanto y el atractivo de un futuro impreciso y desconocido.

\par
%\textsuperscript{(438.3)}
\textsuperscript{39:5.10} 5. \textit{Los Transportadores}. Los transportadores planetarios están al servicio de los mundos individuales. La mayoría de los seres enserafinados que llegan a este planeta están de paso; hacen simplemente una parada; están custodiados por sus propios transportadores seráficos especiales; pero hay un gran número de estos serafines estacionados en Urantia. Son las personalidades transportadoras que operan desde los planetas locales, como por ejemplo desde Urantia hasta Jerusem.

\par
%\textsuperscript{(438.4)}
\textsuperscript{39:5.11} Vuestra idea convencional sobre los ángeles ha nacido de la manera siguiente: en los momentos inmediatamente anteriores a la muerte física, a veces se produce un fenómeno reflectante en la mente humana, y esta conciencia en vías de apagarse parece visualizar algo de la forma del ángel acompañante; esto se interpreta de inmediato en los términos del concepto habitual que la mente de ese individuo tiene sobre los ángeles.

\par
%\textsuperscript{(438.5)}
\textsuperscript{39:5.12} La idea errónea de que los ángeles poseen alas\footnote{\textit{Alas de los ángeles}: Ex 25:20; Is 6:2; Ez 10:5.} no se debe íntegramente a las antiguas nociones de que debían tener alas para volar por los aires. A los seres humanos se les ha permitido a veces observar a los serafines que se estaban preparando para realizar un servicio de transporte, y las tradiciones acerca de estas experiencias han determinado ampliamente el concepto urantiano sobre los ángeles. Al observar a un serafín transportador que se está preparando para recibir a un pasajero para un viaje interplanetario, se puede ver lo que parece ser un doble dispositivo de alas que se extiende desde la cabeza hasta los pies del ángel. Estas alas son en realidad aisladores energéticos ---escudos contra la fricción\footnote{\textit{Escudos de fricción}: Ez 1:6,11,23; 10:21.}.

\par
%\textsuperscript{(438.6)}
\textsuperscript{39:5.13} Cuando los seres celestiales han de ser enserafinados para ser trasladados de un mundo a otro, se les lleva a la sede de la esfera y, después del debido registro, se les provoca el sueño de tránsito. Entretanto, el serafín transportador\footnote{\textit{Transportadores}: Ez 1:19-21; 10:17.} se coloca en posición horizontal inmediatamente por encima del polo energético universal del planeta. Mientras los escudos energéticos están totalmente abiertos, los asistentes seráficos de servicio depositan hábilmente a la personalidad dormida directamente encima del ángel transportador. Luego, los pares de escudos tanto superiores como inferiores se cierran y se ajustan cuidadosamente.

\par
%\textsuperscript{(438.7)}
\textsuperscript{39:5.14} Entonces, bajo la influencia de los transformadores y los transmisores, empieza una extraña metamorfosis a medida que se prepara al serafín para ser lanzado a las corrientes energéticas de los circuitos universales. Según la apariencia exterior, el serafín se vuelve puntiagudo en ambos extremos, y está tan envuelto en una extraña luz\footnote{\textit{Envuelto en luz}: Ez 1:4,13,14,22,27; 10:2,9.} de tonalidad ámbar que muy pronto es imposible distinguir a la personalidad enserafinada. Cuando todo está preparado para la partida, el jefe de los transportes inspecciona adecuadamente el vehículo de vida, efectúa el examen rutinario para comprobar si el ángel está o no adecuadamente conectado a los circuitos; luego anuncia que el viajero está debidamente enserafinado, que las energías están ajustadas, que el ángel está aislado, y que todo está preparado para el destello de la salida. Dos controladores maquinales ocupan entonces sus puestos. Para entonces, el serafín transportador se ha convertido en una silueta casi transparente, vibrante, con forma de torpedo y con una luminosidad resplandeciente. El expedidor de los transportes del reino convoca entonces a las baterías auxiliares de los transmisores vivientes de energía, que generalmente ascienden a mil; cuando anuncia el destino del transporte, se acerca y toca el punto más cercano del vehículo seráfico, el cual sale disparado a la velocidad de un relámpago, dejando una estela de luminosidad celestial hasta donde se prolonga la envoltura atmosférica planetaria. En menos de diez minutos, el maravilloso espectáculo se pierde de vista incluso para la visión reforzada de los serafines\footnote{\textit{Partida de un ángel}: Ez 1:12-14,19-24; 10:15-19.}.

\par
%\textsuperscript{(439.1)}
\textsuperscript{39:5.15} Aunque los informes espaciales planetarios se reciben a mediodía en el meridiano de la sede espiritual indicada, los transportadores son enviados a medianoche desde este mismo lugar. Es el momento más favorable para la partida y es la hora oficial, cuando no se indica lo contrario.

\par
%\textsuperscript{(439.2)}
\textsuperscript{39:5.16} 6. \textit{Los Registradores}. Son los custodios de los asuntos importantes del planeta tal como éste funciona como una parte del sistema y tal como está relacionado con, e implicado en, el gobierno del universo. Ejercen su actividad registrando los asuntos planetarios, pero no se ocupan de las cuestiones relacionadas con la vida y la existencia de los individuos.

\par
%\textsuperscript{(439.3)}
\textsuperscript{39:5.17} 7. \textit{Las Reservas}. El cuerpo de reserva de los serafines planetarios de Satania se mantiene en Jerusem en estrecha asociación con las reservas de los Hijos Materiales. Estas abundantes reservas aseguran plenamente cada fase de las múltiples actividades de esta orden seráfica. Estos ángeles son también los portadores de los mensajes personales de los sistemas locales. Sirven a los mortales de transición, a los ángeles y a los Hijos Materiales, así como a otros seres domiciliados en la sede del sistema. Aunque Urantia está excluida actualmente de los circuitos espirituales de Satania y de Norlatiadek, por lo demás estáis en contacto íntimo con los asuntos interplanetarios, pues estos mensajeros de Jerusem vienen con frecuencia a este mundo así como a todas las otras esferas del sistema.

\section*{6. Los Ministros de las Transiciones}
\par
%\textsuperscript{(439.4)}
\textsuperscript{39:6.1} Tal como lo sugiere su nombre, los serafines que realizan un ministerio de transición sirven donde pueden contribuir a la transición de las criaturas entre el estado material y el estado espiritual. Estos ángeles sirven desde los mundos habitados hasta las capitales de los sistemas, pero los de Satania dirigen actualmente sus mayores esfuerzos hacia la educación de los mortales supervivientes en los siete mundos de las mansiones. Este ministerio está diversificado de acuerdo con los siete tipos de destino siguientes:

\par
%\textsuperscript{(439.5)}
\textsuperscript{39:6.2} 1. Los Evángeles Seráficos.

\par
%\textsuperscript{(439.6)}
\textsuperscript{39:6.3} 2. Los Intérpretes Raciales.

\par
%\textsuperscript{(439.7)}
\textsuperscript{39:6.4} 3. Los Planificadores de la Mente.

\par
%\textsuperscript{(439.8)}
\textsuperscript{39:6.5} 4. Los Consejeros Morontiales.

\par
%\textsuperscript{(439.9)}
\textsuperscript{39:6.6} 5. Los Técnicos.

\par
%\textsuperscript{(439.10)}
\textsuperscript{39:6.7} 6. Los Instructores-Registradores.

\par
%\textsuperscript{(439.11)}
\textsuperscript{39:6.8} 7. Las Reservas Ministrantes.

\par
%\textsuperscript{(439.12)}
\textsuperscript{39:6.9} Aprenderéis más cosas sobre estos ministros seráficos de los ascendentes de transición en las narraciones que tratan sobre los mundos de las mansiones y la vida morontial.

\section*{7. Los Serafines del Futuro}
\par
%\textsuperscript{(440.1)}
\textsuperscript{39:7.1} Estos ángeles sólo ejercen ampliamente su ministerio en los reinos más antiguos y en los planetas más avanzados de Nebadon. Un gran número de ellos se mantienen de reserva en los mundos seráficos cercanos a Salvington, donde se dedican a las ocupaciones relacionadas con el nacimiento futuro de la era de luz y de vida en Nebadon. Estos serafines están relacionados de hecho en su actividad con la carrera mortal-ascendente, pero aportan su ministerio casi exclusivamente a aquellos mortales que sobreviven mediante alguno de los tipos modificados de ascensión.

\par
%\textsuperscript{(440.2)}
\textsuperscript{39:7.2} Puesto que estos ángeles no se ocupan ahora directamente ni de Urantia ni de los urantianos, consideramos que es mejor abstenernos de describir sus fascinantes actividades.

\section*{8. El destino de los Serafines}
\par
%\textsuperscript{(440.3)}
\textsuperscript{39:8.1} Los serafines tienen su origen en los universos locales, y algunos consiguen su destino de servicio en estos mismos reinos donde han nacido. Con la ayuda y los consejos de los arcángeles más antiguos, algunos serafines pueden ser elevados a las exaltadas funciones de las Brillantes Estrellas Vespertinas, mientras que otros consiguen el estado y el servicio de los coordinados no revelados de las Estrellas Vespertinas. Pueden intentar también otras aventuras conectadas con el destino del universo local, pero Serafington sigue siendo la meta eterna de todos los ángeles. Serafington es el umbral angélico para entrar en el Paraíso y alcanzar la Deidad, la esfera de transición entre el ministerio del tiempo y el servicio exaltado de la eternidad.

\par
%\textsuperscript{(440.4)}
\textsuperscript{39:8.2} Los serafines pueden alcanzar el Paraíso por decenas ---por centenares--- de caminos, pero los más importantes que se tratan en estas narraciones son los siguientes:

\par
%\textsuperscript{(440.5)}
\textsuperscript{39:8.3} 1. Ser admitido a título personal en la residencia seráfica del Paraíso, consiguiendo la perfección en un servicio especializado como artesano celestial, Asesor Técnico o Registrador Celestial. Convertirse en un Compañero Paradisiaco y, después de alcanzar así el centro de todas las cosas, transformarse entonces quizás en ministro y asesor eterno de las órdenes seráficas y de otras órdenes.

\par
%\textsuperscript{(440.6)}
\textsuperscript{39:8.4} 2. Ser citado para presentarse en Serafington. Bajo ciertas condiciones, los serafines son llamados a comparecer en las alturas; en otras circunstancias, los ángeles a veces alcanzan el Paraíso en un espacio de tiempo mucho más corto que los mortales. Pero por muy capacitada que esté una pareja seráfica, no puede iniciar su partida hacia Serafington ni hacia ninguna otra parte. Sólo los guardianes del destino que han tenido éxito pueden estar seguros de dirigirse hacia el Paraíso por un camino progresivo de ascensión evolutiva. Todos los demás deben esperar pacientemente la llegada de los mensajeros paradisiacos de los supernafines terciarios con una citación que les ordene presentarse en las alturas.

\par
%\textsuperscript{(440.7)}
\textsuperscript{39:8.5} 3. Alcanzar el Paraíso mediante la técnica evolutiva de los mortales. La elección suprema de los serafines, en la carrera del tiempo, es el puesto de ángel guardián a fin de poder alcanzar la carrera de la finalidad y cualificarse para ser destinados a las esferas eternas del servicio seráfico. Estos guías personales de los hijos del tiempo se llaman guardianes del destino, lo que significa que custodian a las criaturas mortales en el sendero del destino divino, y que al hacer esto están determinando su propio elevado destino.

\par
%\textsuperscript{(440.8)}
\textsuperscript{39:8.6} Los guardianes del destino son elegidos entre las filas de las personalidades angélicas más experimentadas de todas las órdenes de serafines que se han cualificado para este servicio. Todos los mortales sobrevivientes destinados a fusionar con su Ajustador tienen asignados guardianes temporales, y estos asociados pueden permanecer vinculados a ellos de manera permanente cuando los supervivientes mortales alcanzan el desarrollo intelectual y espiritual necesario. Antes de que los ascendentes mortales dejen los mundos de las mansiones, todos tienen asociados seráficos permanentes. Este grupo de espíritus ministrantes lo analizaremos en las narraciones relacionadas con Urantia.

\par
%\textsuperscript{(441.1)}
\textsuperscript{39:8.7} A los ángeles no les resulta posible alcanzar a Dios partiendo desde el nivel humano de origen, pues son creados un <<poco superiores a vosotros>>\footnote{\textit{Los ángeles creados un poco por encima}: Sal 8:5; Heb 2:7,9.}; pero aunque no pueden empezar de ninguna manera desde el punto más bajo, desde las tierras bajas espirituales de la existencia mortal, se ha dispuesto sabiamente que puedan descender hasta aquellos que sí empiezan en el fondo y guiar a estas criaturas paso a paso, mundo tras mundo, hasta las puertas de Havona. Cuando los ascendentes mortales dejan Uversa para empezar en los círculos de Havona, los guardianes que les fueron asignados después de la vida en la carne se despiden temporalmente de sus asociados peregrinos y viajan a Serafington, el destino de los ángeles del gran universo. Aquí, estos guardianes intentarán alcanzar los siete círculos de la luz seráfica, y lo conseguirán indudablemente.

\par
%\textsuperscript{(441.2)}
\textsuperscript{39:8.8} Muchos de estos serafines asignados como guardianes del destino durante la vida material, pero no todos, acompañan a sus asociados mortales por los círculos de Havona, y algunos otros serafines pasan por los circuitos del universo central de una manera enteramente diferente a la de la ascensión de los mortales. Pero cualquiera que sea el itinerario de la ascensión, todos los serafines evolutivos atraviesan Serafington, y la mayoría pasa por esta experiencia en lugar de pasar por los circuitos de Havona.

\par
%\textsuperscript{(441.3)}
\textsuperscript{39:8.9} Serafington es la esfera de destino de los ángeles, y el hecho de alcanzar este mundo es algo totalmente diferente a las experiencias de los peregrinos mortales en Ascendington. Los ángeles no están absolutamente seguros de su futuro eterno hasta que no han llegado a Serafington. Se sabe que ningún ángel que ha alcanzado Serafington se ha descarriado nunca; el pecado nunca encontrará respuesta en el corazón de un serafín consumado.

\par
%\textsuperscript{(441.4)}
\textsuperscript{39:8.10} Los graduados de Serafington reciben misiones diversas: los guardianes del destino con experiencia en los círculos de Havona entran generalmente en el Cuerpo de los Finalitarios Mortales. Otros guardianes, después de haber pasado por las pruebas de clasificación de Havona, se reúnen frecuentemente con sus asociados mortales en el Paraíso, y algunos se convierten en los asociados perpetuos de los finalitarios mortales, mientras que otros entran en los diversos cuerpos finalitarios no mortales, y muchos son enrolados en el Cuerpo de la Finalización Seráfica.

\section*{9. El Cuerpo de la Finalización Seráfica}
\par
%\textsuperscript{(441.5)}
\textsuperscript{39:9.1} Después de alcanzar al Padre de los espíritus y de ser admitidos en el servicio seráfico de la finalización, a los ángeles a veces se les destina al ministerio de los mundos establecidos en la luz y la vida. Consiguen vincularse a los elevados seres trinitizados de los universos y a los servicios exaltados del Paraíso y de Havona. Estos serafines de los universos locales han compensado experiencialmente el diferencial en potencial de divinidad que los separaba anteriormente de los espíritus ministrantes del universo central y de los superuniversos. Los ángeles del Cuerpo Seráfico de la Finalización sirven como asociados de los seconafines superuniversales y como asistentes de las elevadas órdenes de supernafines del Paraíso-Havona. Para estos ángeles la carrera del tiempo ha terminado; de ahora en adelante y para siempre, son los servidores de Dios, los asociados de las personalidades divinas y los iguales de los finalitarios del Paraíso.

\par
%\textsuperscript{(441.6)}
\textsuperscript{39:9.2} Un gran número de serafines consumados regresan a sus universos nativos para complementar allí el ministerio de la dotación divina con el ministerio de la perfección experiencial. Nebadon es, comparativamente hablando, uno de los universos más jóvenes, y por eso no posee tantos graduados que hayan regresado de Serafington como se pueden encontrar en otro reino más antiguo; sin embargo, nuestro universo local está adecuadamente provisto de serafines consumados, pues es significativo que los reinos evolutivos revelen una creciente necesidad de sus servicios a medida que se acercan al estado de luz y de vida. En la actualidad, los serafines consumados sirven más ampliamente con las órdenes supremas de serafines, pero algunos sirven con cada una de las otras órdenes angélicas. Incluso vuestro mundo disfruta del amplio ministerio de doce grupos especializados del Cuerpo Seráfico de la Finalización; estos serafines maestros de la supervisión planetaria acompañan a cada Príncipe Planetario recién nombrado a los mundos habitados.

\par
%\textsuperscript{(442.1)}
\textsuperscript{39:9.3} Muchas vías fascinantes están abiertas al ministerio de los serafines consumados, pero al igual que todos ellos anhelaban ser nombrados guardianes del destino antes de llegar al Paraíso, en su experiencia post-paradisiaca lo que más desean es servir como acompañantes durante la donación de los Hijos Paradisiacos encarnados. Permanecen dedicados de manera suprema al plan universal de poner en camino a las criaturas mortales de los mundos evolutivos en el largo y atractivo viaje hacia la meta paradisiaca de la divinidad y la eternidad. Durante toda la aventura humana de encontrar a Dios y de conseguir la perfección divina, estos ministros espirituales de la consumación seráfica, junto con los fieles espíritus ministrantes del tiempo, son siempre y para siempre vuestros verdaderos amigos y vuestros colaboradores indefectibles.

\par
%\textsuperscript{(442.2)}
\textsuperscript{39:9.4} [Presentado por un Melquisedek que actúa a petición del Jefe de las Huestes Seráficas de Nebadon.]


\chapter{Documento 40. Los Hijos ascendentes de Dios}
\par
%\textsuperscript{(443.1)}
\textsuperscript{40:0.1} COMO ha sucedido con muchos grupos principales de seres universales, se han revelado siete clases generales de Hijos Ascendentes de Dios:

\par
%\textsuperscript{(443.2)}
\textsuperscript{40:0.2} 1. Los Mortales fusionados con el Padre.

\par
%\textsuperscript{(443.3)}
\textsuperscript{40:0.3} 2. Los Mortales fusionados con el Hijo.

\par
%\textsuperscript{(443.4)}
\textsuperscript{40:0.4} 3. Los Mortales fusionados con el Espíritu.

\par
%\textsuperscript{(443.5)}
\textsuperscript{40:0.5} 4. Los Serafines evolutivos.

\par
%\textsuperscript{(443.6)}
\textsuperscript{40:0.6} 5. Los Hijos Materiales ascendentes.

\par
%\textsuperscript{(443.7)}
\textsuperscript{40:0.7} 6. Los Intermedios trasladados.

\par
%\textsuperscript{(443.8)}
\textsuperscript{40:0.8} 7. Los Ajustadores Personalizados.

\par
%\textsuperscript{(443.9)}
\textsuperscript{40:0.9} La historia de estos seres, desde los humildes mortales de origen animal de los mundos evolutivos hasta los Ajustadores Personalizados del Padre Universal, presenta un relato glorioso de la donación ilimitada de amor divino y de condescendencia bondadosa a través de todos los tiempos y en todos los universos de la extensa creación de las Deidades del Paraíso.

\par
%\textsuperscript{(443.10)}
\textsuperscript{40:0.10} Estas presentaciones empezaron con una descripción de las Deidades y, grupo tras grupo, la narración ha descendido la escala universal de los seres vivientes hasta llegar a la orden más humilde de vida dotada del potencial de la inmortalidad; ahora he sido enviado desde Salvington ---en otro tiempo fui un mortal originario de un mundo evolutivo del espacio--- para elaborar y continuar el relato del propósito eterno de los Dioses respecto a las órdenes ascendentes de filiación, y más particularmente con relación a las criaturas mortales del tiempo y del espacio.

\par
%\textsuperscript{(443.11)}
\textsuperscript{40:0.11} Puesto que la mayor parte de esta narración se dedicará a analizar las tres órdenes fundamentales de mortales ascendentes, examinaremos en primer lugar las órdenes de filiación ascendentes no mortales ---las de los serafines, los Adanes, los intermedios y los Ajustadores.

\section*{1. Los Serafines evolutivos}
\par
%\textsuperscript{(443.12)}
\textsuperscript{40:1.1} Las criaturas mortales de origen animal no son los únicos seres que gozan del privilegio de disfrutar de la filiación; las huestes angélicas también comparten la oportunidad celestial de alcanzar el Paraíso. Los serafines guardianes, a través de su experiencia y de su servicio con los mortales ascendentes del tiempo, también consiguen el estado de la filiación ascendente. Estos ángeles alcanzan el Paraíso a través de Serafington, y muchos de ellos son incluso enrolados en el Cuerpo de la Finalidad de los Mortales.

\par
%\textsuperscript{(443.13)}
\textsuperscript{40:1.2} Ascender hasta las alturas celestiales de la filiación finalitaria con Dios es una proeza magistral para un ángel, un logro que trasciende de lejos vuestra conquista de la supervivencia eterna a través del plan del Hijo Eterno y de la ayuda siempre presente del Ajustador interior; pero los serafines guardianes, y de vez en cuando otros serafines, efectúan realmente estas ascensiones.

\section*{2. Los Hijos Materiales ascendentes}
\par
%\textsuperscript{(444.1)}
\textsuperscript{40:2.1} Los Hijos Materiales de Dios son creados en el universo local junto con los Melquisedeks y sus asociados, estando todos clasificados como Hijos descendentes. Y es verdad que los Adanes Planetarios ---los Hijos y las Hijas Materiales de los mundos evolutivos--- son Hijos descendentes, pues descienden desde sus esferas de origen, las capitales de los sistemas locales, hasta los mundos habitados.

\par
%\textsuperscript{(444.2)}
\textsuperscript{40:2.2} Cuando un Adán y una Eva triunfan plenamente en su misión planetaria conjunta como mejoradores biológicos, comparten el destino de los habitantes de su mundo. Cuando ese mundo se establece en las etapas avanzadas de luz y de vida, a estos fieles Hijos e Hijas Materiales se les permite renunciar a todos sus deberes administrativos planetarios, y después de ser liberados así de la aventura descendente, se les permite registrarse en los archivos del universo local como Hijos Materiales perfeccionados. Del mismo modo, cuando su nombramiento para ir a un planeta se demora durante mucho tiempo, los Hijos Materiales que tienen un estado estacionario ---los ciudadanos de los sistemas locales--- pueden retirarse de las actividades de las esferas a las que pertenecen, y registrarse de manera similar como Hijos Materiales perfeccionados. Después de estas formalidades, estos Adanes y estas Evas liberados son acreditados como Hijos ascendentes de Dios, y pueden empezar inmediatamente el largo viaje hacia Havona y el Paraíso, partiendo desde el punto exacto de su estado presente y de sus logros espirituales conseguidos. Este viaje lo hacen en compañía de los mortales y de otros Hijos ascendentes, y lo continúan hasta que encuentran a Dios y alcanzan el Cuerpo de la Finalidad de los Mortales que está al servicio eterno de las Deidades del Paraíso.

\section*{3. Los Intermedios trasladados}
\par
%\textsuperscript{(444.3)}
\textsuperscript{40:3.1} Aunque estén privados de los beneficios inmediatos de las donaciones planetarias de los Hijos descendentes de Dios, aunque la ascensión hacia el Paraíso se demore durante mucho tiempo, sin embargo, poco después de que un planeta evolutivo ha alcanzado las épocas intermedias de luz y de vida (si no antes), los dos grupos de criaturas intermedias son liberados de sus deberes planetarios. A veces la mayoría de ellos son trasladados, junto con sus primos humanos, el día en que desciende el templo de luz y el Príncipe Planetario es elevado a la dignidad de Soberano Planetario. Después de ser liberadas de su servicio planetario, las dos órdenes se registran en el universo local como Hijos ascendentes de Dios y empiezan inmediatamente la larga ascensión hacia el Paraíso por las mismas vías ordenadas para la progresión de las razas mortales de los mundos materiales. El grupo primario es destinado a diversos cuerpos finalitarios, pero todos los intermedios secundarios o adámicos se dirigen a inscribirse en el Cuerpo de los Mortales de la Finalidad.

\section*{4. Los Ajustadores Personalizados}
\par
%\textsuperscript{(444.4)}
\textsuperscript{40:4.1} Cuando los mortales del tiempo no consiguen la supervivencia eterna de su alma en asociación planetaria con el don espiritual del Padre Universal, este fracaso nunca se debe de ninguna manera a una negligencia en el deber, el ministerio, el servicio o la devoción del Ajustador. En el momento de la muerte física, estos Monitores abandonados regresan a Divinington y, posteriormente, después del juicio del no sobreviviente, pueden ser destinados de nuevo a los mundos del tiempo y del espacio. A veces, después de repetidos servicios de este tipo o con posterioridad a alguna experiencia excepcional, como por ejemplo trabajar como Ajustador interior de un Hijo donador encarnado, estos eficaces Ajustadores son personalizados por el Padre Universal.

\par
%\textsuperscript{(445.1)}
\textsuperscript{40:4.2} Los Ajustadores Personalizados son seres de una orden única e insondable. En un principio su estado era prepersonal y existencial, pero se han vuelto experienciales participando en la vida y la carrera de los humildes mortales de los mundos materiales. Y puesto que la personalidad otorgada a estos Ajustadores del Pensamiento experimentados tiene su origen y su fuente en el ministerio personal y continuado del Padre Universal, que otorga la personalidad experiencial a las criaturas de su creación, estos Ajustadores Personalizados están clasificados como Hijos ascendentes de Dios, siendo la más elevada de todas estas órdenes de filiación.

\section*{5. Los Mortales del tiempo y del espacio}
\par
%\textsuperscript{(445.2)}
\textsuperscript{40:5.1} Los mortales representan el último eslabón de la cadena de seres llamados hijos de Dios. El sello personal del Hijo Original y Eterno se transmite a través de una serie de personalizaciones cada vez menos divinas y cada vez más humanas, hasta llegar a un ser que se parece mucho a vosotros, un ser que podéis ver, oír y tocar. Entonces os volvéis espiritualmente conscientes de la gran verdad que vuestra fe puede captar ---¡vuestra filiación con el Dios eterno!

\par
%\textsuperscript{(445.3)}
\textsuperscript{40:5.2} De la misma manera, el Espíritu Original e Infinito, por medio de una larga serie de órdenes cada vez menos divinas y cada vez más humanas, se acerca cada vez más a las criaturas que luchan en los reinos, alcanzando el límite de su expresión en los ángeles ---respecto a los cuales sólo habéis sido creados un poco inferiores\footnote{\textit{Somos un poco por debajo de los ángeles}: Sal 8:5; Heb 2:7.}--- que os custodian y os guían personalmente en el viaje por la vida de la carrera humana del tiempo.

\par
%\textsuperscript{(445.4)}
\textsuperscript{40:5.3} Dios Padre no desciende, no puede descender así, para establecer este contacto personal íntimo con el número casi ilimitado de criaturas ascendentes de todo el universo de universos. Pero el Padre no está privado de un contacto personal con sus humildes criaturas; no estáis privados de la presencia divina. Aunque Dios Padre no pueda estar con vosotros mediante una manifestación directa de su personalidad, está en vosotros y forma parte de vosotros mediante la identidad de los Ajustadores del Pensamiento interiores, los Monitores divinos. Así es como el Padre, que es el que está más lejos de vosotros en personalidad y en espíritu, es el que más se acerca a vosotros en el circuito de la personalidad y en el contacto espiritual de la comunión interior con el alma misma de sus hijos e hijas mortales.

\par
%\textsuperscript{(445.5)}
\textsuperscript{40:5.4} La identificación con el espíritu constituye el secreto de la supervivencia personal y determina el destino de la ascensión espiritual. Y puesto que los Ajustadores del Pensamiento son los únicos espíritus con un potencial de fusión que se pueden identificar con el hombre durante la vida en la carne, los mortales del tiempo y del espacio están clasificados principalmente de acuerdo con su relación con estos dones divinos, los Monitores de Misterio interiores. Esta clasificación es la siguiente:

\par
%\textsuperscript{(445.6)}
\textsuperscript{40:5.5} 1. Mortales en quienes la estancia del Ajustador es transitoria o experiencial.

\par
%\textsuperscript{(445.7)}
\textsuperscript{40:5.6} 2. Mortales de los tipos que no fusionan con el Ajustador.

\par
%\textsuperscript{(445.8)}
\textsuperscript{40:5.7} 3. Mortales que tienen el potencial de fusionar con el Ajustador.

\par
%\textsuperscript{(445.9)}
\textsuperscript{40:5.8} \textit{Primera serie ---los mortales en quienes la estancia del Ajustador estransitoria o experiencial}. La denominación de esta serie es temporal para todo planeta en evolución, y se utiliza durante las etapas primitivas de todos los mundos habitados, a excepción de aquellos de la segunda serie.

\par
%\textsuperscript{(445.10)}
\textsuperscript{40:5.9} Los mortales de la primera serie habitan los mundos del espacio durante las épocas iniciales de la evolución de la humanidad, y contienen los tipos más primitivos de mentes humanas. En muchos mundos como Urantia antes de Adán, un gran número de hombres primitivos de los tipos superiores y más avanzados adquieren la capacidad de sobrevivir, pero no consiguen fusionar con el Ajustador. Durante eras y eras, antes de que el hombre ascienda al nivel de la volición espiritual superior, los Ajustadores ocupan la mente de estas criaturas luchadoras durante sus cortas vidas en la carne, y en cuanto estas criaturas volitivas son habitadas por los Ajustadores, los ángeles guardianes colectivos empiezan a actuar. Aunque estos mortales de la primera serie no tienen guardianes personales, poseen custodios colectivos.

\par
%\textsuperscript{(446.1)}
\textsuperscript{40:5.10} Un Ajustador experiencial permanece con un ser humano primitivo durante toda su vida en la carne. Los Ajustadores contribuyen en gran medida al progreso de los hombres primitivos, pero son incapaces de formar uniones eternas con dichos mortales. Este ministerio transitorio de los Ajustadores logra dos cosas: primero, adquieren una experiencia valiosa y real de la naturaleza y del funcionamiento del intelecto evolutivo, una experiencia que será inapreciable cuando contacten posteriormente en otros mundos con seres de un desarrollo superior. Segundo, la estancia transitoria de los Ajustadores contribuye mucho a preparar a sus sujetos mortales para una posible fusión posterior con el Espíritu. Todas las almas de este tipo que buscan a Dios consiguen la vida eterna mediante el abrazo espiritual del Espíritu Madre del universo local, convirtiéndose así en mortales ascendentes sometidos al régimen del universo local. Muchas personas de la Urantia pre-adámica fueron elevadas así a los mundos de las mansiones de Satania.

\par
%\textsuperscript{(446.2)}
\textsuperscript{40:5.11} (446.2 Los Dioses que han ordenado que el hombre mortal se eleve a los niveles superiores de inteligencia espiritual a través de largas épocas de pruebas y de tribulaciones evolutivas, toman nota de su estado y de sus necesidades en cada fase de la ascensión; y siempre son divinamente equitativos y justos, e incluso encantadoramente misericordiosos, en sus juicios finales de estos mortales luchadores de los primeros tiempos de las razas en evolución.

\par
%\textsuperscript{(446.3)}
\textsuperscript{40:5.12} \textit{Segunda serie ---los mortales de los tipos que no fusionan con el Ajustador}. Se trata de tipos especializados de seres humanos que no son capaces de llevar a cabo una unión eterna con su Ajustador interior. El hecho de estar clasificado entre las razas que poseen uno, dos o tres cerebros no es un factor para la fusión con el Ajustador; todos estos mortales son semejantes, pero estos tipos que no fusionan con el Ajustador pertenecen a una orden enteramente diferente y notablemente modificada de criaturas volitivas. Muchos tipos de seres no respiradores pertenecen a esta serie, y existen otros numerosos grupos que no fusionan habitualmente con los Ajustadores.

\par
%\textsuperscript{(446.4)}
\textsuperscript{40:5.13} Al igual que en la serie número uno, cada miembro de este grupo disfruta del ministerio de un solo Ajustador durante su vida en la carne. Durante la vida temporal, estos Ajustadores hacen por los sujetos en los que residen temporalmente todo lo que se hace en los otros mundos donde los mortales tienen el potencial de fusionar. Los mortales de esta segunda serie están habitados con frecuencia por Ajustadores vírgenes, pero los tipos humanos superiores están a menudo en contacto con Monitores magistrales y experimentados.

\par
%\textsuperscript{(446.5)}
\textsuperscript{40:5.14} En el plan ascendente para elevar a las criaturas de origen animal, estos seres disfrutan del mismo servicio dedicado de los Hijos de Dios que se ofrece al tipo de mortales de Urantia. En los planetas donde no se fusiona, la cooperación seráfica con los Ajustadores está tan plenamente asegurada como en los mundos con potencial de fusión; los guardianes del destino ejercen su ministerio en estas esferas exactamente igual que lo hacen en Urantia, y actúan de forma similar en el momento de la supervivencia de los mortales, en el momento en que el alma sobreviviente fusiona con el Espíritu.

\par
%\textsuperscript{(446.6)}
\textsuperscript{40:5.15} Cuando encontréis a estos tipos de mortales modificados en los mundos de las mansiones, no tendréis ninguna dificultad para comunicaros con ellos. Allí hablan el mismo idioma del sistema pero mediante una técnica modificada. Estos seres son idénticos a vuestra orden de vida creada en las manifestaciones del espíritu y de la personalidad, y sólo se diferencian en ciertas características físicas y en el hecho de que no pueden fusionar con los Ajustadores del Pensamiento.

\par
%\textsuperscript{(447.1)}
\textsuperscript{40:5.16} En cuanto a la razón exacta por la cual este tipo de criaturas no pueden fusionar nunca con los Ajustadores del Padre Universal, soy incapaz de decírosla. Algunos de nosotros nos inclinamos a creer que los Portadores de Vida, en sus esfuerzos por crear unos seres capaces de mantener su existencia en un entorno planetario inhabitual, se enfrentan a la necesidad de hacer unas modificaciones tan radicales en el plan universal de las criaturas volitivas inteligentes, que resulta imposible por inherencia efectuar una unión permanente con los Ajustadores. A menudo nos hemos preguntado: ¿forma esto una parte intencional o involuntaria del plan de la ascensión? Pero no hemos encontrado la respuesta.

\par
%\textsuperscript{(447.2)}
\textsuperscript{40:5.17} \textit{Tercera serie ---los mortales que tienen el potencial de fusionar con elAjustador}. Todos los mortales fusionados con el Padre tienen un origen animal, exactamente igual que las razas de Urantia. Engloba a los mortales pertenecientes a los tipos de uno, dos y tres cerebros que tienen el potencial de fusionar con el Ajustador. Los urantianos pertenecen al tipo intermedio, o de dos cerebros, siendo humanamente superiores en muchos aspectos a los grupos de un cerebro, pero claramente limitados en comparación con las órdenes que poseen tres cerebros. La dotación físico-cerebral de estos tres tipos no es un factor que influya en la concesión de los Ajustadores, ni en el servicio seráfico, ni en cualquier otra fase del ministerio espiritual. El diferencial intelectual y espiritual entre los tres tipos cerebrales caracteriza a unos individuos que son por otra parte totalmente semejantes en su dotación mental y en su potencial espiritual; esta diferencia es mayor durante la vida temporal, y tiende a disminuir a medida que se atraviesan los mundos de las mansiones uno tras otro. A partir de la sede del sistema, la progresión de estos tres tipos es la misma, y su destino final en el Paraíso es idéntico.

\par
%\textsuperscript{(447.3)}
\textsuperscript{40:5.18} \textit{Las series sin numerar}. Estas narraciones no pueden abarcar de ninguna manera todas las fascinantes variaciones que existen en los mundos evolutivos. Sabéis que cada décimo mundo es un planeta decimal o experimental, pero no sabéis nada sobre las otras variables que salpican la procesión de las esferas evolutivas. Incluso entre las órdenes reveladas de criaturas vivientes, así como entre los planetas del mismo grupo, las diferencias son demasiado numerosas como para ser descritas, pero esta exposición indica claramente las diferencias esenciales en relación con la carrera de la ascensión. Y la carrera de la ascensión es el factor más importante en cualquier estudio sobre los mortales del tiempo y del espacio.

\par
%\textsuperscript{(447.4)}
\textsuperscript{40:5.19} En cuanto a las posibilidades de supervivencia de los mortales, que quede claro para siempre: todas las almas pertenecientes a cada fase posible de la existencia mortal sobrevivirán a condición de que manifiesten la buena voluntad de cooperar con su Ajustador interior y muestren el deseo de encontrar a Dios y de alcanzar la perfección divina, aunque estos deseos sólo sean los primeros débiles parpadeos de la comprensión primitiva de esa <<verdadera luz\footnote{\textit{Verdadera luz}: Is 9:2; 49:6; Mt 4:16; Lc 1:79; 2:32; Jn 1:4-9; 8:12; 9:5; 12:35-36,46; 1 Jn 2:8.} que ilumina a todo hombre que entra en el mundo>>\footnote{\textit{Verdadera luz que ilumina a todo}: Jn 1:9.}.

\section*{6. Los hijos de Dios por la fe}
\par
%\textsuperscript{(447.5)}
\textsuperscript{40:6.1} Las razas mortales figuran como representantes de la orden más humilde de la creación inteligente y personal. Vosotros, los mortales, sois divinamente amados, y cada uno de vosotros puede elegir aceptar el destino seguro de una experiencia gloriosa, pero todavía no pertenecéis por naturaleza a la orden divina; sois totalmente mortales. Seréis considerados como hijos ascendentes en el instante en que tenga lugar la fusión, pero antes del acontecimiento de la amalgamación final del alma mortal sobreviviente con algún tipo de espíritu eterno e inmortal, el estado de los mortales del tiempo y del espacio es el de hijos por la fe\footnote{\textit{Hijos de Dios por la fe}: 1 Cr 22:10; Sal 2:7; Is 56:5; Mt 5:9,16,45; Lc 20:36; Jn 1:12-13; 11:52; Hch 17:28-29; Ro 8:14-17,19,21; 9:26; 2 Co 6:18; Gl 3:26; 4:5-7; Ef 1:5; Flp 2:15; Heb 12:5-8; 1 Jn 3:1-2,10; 5:2; Ap 21:7; 2 Sam 7:14.}.

\par
%\textsuperscript{(448.1)}
\textsuperscript{40:6.2} Es un hecho solemne y celestial que unas criaturas tan humildes y materiales como los seres humanos de Urantia sean hijos de Dios, hijos del Altísimo por la fe. <<Mirad la clase de amor que el Padre nos ha otorgado para que seamos llamados hijos de Dios>>\footnote{\textit{Mirad la clase de amor}: 1 Jn 3:1.}. <<A todos los que lo han recibido les ha dado el poder de conocer que son hijos de Dios>>\footnote{\textit{Recibir a Jesús es ser hijos de Dios}: Jn 1:12.}. Aunque <<todavía no es evidente lo que llegaréis a ser>>\footnote{\textit{Todavía no sabéis qué seréis}: 1 Jn 3:2.} incluso ahora <<sois los hijos de Dios por la fe>>\footnote{\textit{Ahora sois hijos de Dios por la fe}: Gl 3:26.}; <<pues no habéis recibido el espíritu de la esclavitud para temer de nuevo, sino que habéis recibido el espíritu de la filiación por medio del cual exclamáis ``Padre nuestro''>>\footnote{\textit{Recibido el espíritu de la filiación}: Ro 8:15.}. El profeta de antaño dijo en nombre del Dios eterno: <<Incluso a ellos les daré un lugar en mi casa y un nombre mejor que el de hijos; les daré un nombre perpetuo, un nombre que nunca perecerá>>\footnote{\textit{Les daré un nombre perpetuo}: Is 56:5.}. <<Y puesto que sois hijos, Dios ha enviado el espíritu de su Hijo a vuestros corazones>>\footnote{\textit{Espíritu de Dios en los corazones}: Gl 4:6.}.

\par
%\textsuperscript{(448.2)}
\textsuperscript{40:6.3} Todos los mundos evolutivos habitados por los mortales albergan a estos hijos de Dios por la fe, hijos de la gracia y de la misericordia, seres humanos que pertenecen a la familia divina y que son llamados en consecuencia hijos de Dios. Los mortales de Urantia tienen derecho a considerarse como hijos de Dios porque:

\par
%\textsuperscript{(448.3)}
\textsuperscript{40:6.4} 1. Sois los hijos de una promesa espiritual, los hijos por la fe; habéis aceptado el estado de la filiación. Creéis en la realidad de vuestra filiación, y vuestra filiación con Dios se vuelve así eternamente real.

\par
%\textsuperscript{(448.4)}
\textsuperscript{40:6.5} 2. Un Hijo Creador surgido de Dios se volvió uno de vosotros; es de hecho vuestro hermano mayor; y si os convertís, en espíritu, en hermanos verdaderamente emparentados con Cristo, el victorioso Miguel, entonces también debéis ser, en espíritu, los hijos de ese Padre que tenéis en común, el mismo Padre Universal de todos.

\par
%\textsuperscript{(448.5)}
\textsuperscript{40:6.6} 3. Sois hijos porque el espíritu de un Hijo ha sido derramado sobre vosotros, ha sido conferido de manera gratuita y segura a todas las razas de Urantia. Este espíritu siempre os atrae hacia el Hijo divino, que es su fuente, y hacia el Padre Paradisiaco, que es la fuente de ese Hijo divino.

\par
%\textsuperscript{(448.6)}
\textsuperscript{40:6.7} 4. El Padre Universal os ha dado, por su libre albedrío divino, vuestra personalidad de criatura. Habéis sido dotados de una parte de esa divina espontaneidad de acción, basada en el libre albedrío, que Dios comparte con todos aquellos que pueden convertirse en sus hijos.

\par
%\textsuperscript{(448.7)}
\textsuperscript{40:6.8} 5. Dentro de vosotros reside un fragmento del Padre Universal, y estáis así directamente emparentados con el Padre divino de todos los Hijos de Dios.

\section*{7. Los mortales fusionados con el Padre}
\par
%\textsuperscript{(448.8)}
\textsuperscript{40:7.1} El envío de los Ajustadores, su presencia dentro de vosotros, es en verdad uno de los misterios insondables de Dios Padre. Estos fragmentos de la naturaleza divina del Padre Universal traen consigo el potencial de la inmortalidad de las criaturas. Los Ajustadores son espíritus inmortales, y la unión con ellos confiere la vida eterna al alma del mortal fusionado.

\par
%\textsuperscript{(448.9)}
\textsuperscript{40:7.2} Vuestras propias razas de mortales sobrevivientes pertenecen a este grupo de Hijos ascendentes de Dios. Ahora sois hijos planetarios, criaturas evolutivas derivadas de las implantaciones de los Portadores de Vida y modificadas por la inyección de vida adámica, pero apenas sois todavía hijos ascendentes; pero sois en verdad unos hijos dotados del potencial de la ascensión ---incluso hasta las alturas más elevadas de la gloria y de la consecución de la divinidad--- y este estado espiritual de filiación ascendente lo podéis alcanzar a través de la fe y de la cooperación voluntaria con las actividades espiritualizantes del Ajustador interior. Cuando hayáis fusionado finalmente y para siempre con vuestro Ajustador, cuando los dos seáis una sola cosa, como el Hijo de Dios y el Hijo del Hombre son una sola cosa en Cristo Miguel, entonces os habréis convertido de hecho en los hijos ascendentes de Dios.

\par
%\textsuperscript{(449.1)}
\textsuperscript{40:7.3} Los detalles de la carrera de los Ajustadores efectuando su ministerio dentro de los mortales en un planeta probatorio y evolutivo no forman parte de mi misión; la elaboración de esta gran verdad abarca toda vuestra carrera. Incluyo la mención de ciertas funciones de los Ajustadores con el fin de efectuar una exposición completa con respecto a los mortales fusionados con el Ajustador. Estos fragmentos interiores de Dios están con vuestra orden de seres desde los primeros tiempos de vuestra existencia física, luego durante toda la carrera ascendente en Nebadon y en Orvonton, y después a través de Havona hasta el Paraíso mismo. Más tarde, durante la aventura eterna, este mismo Ajustador será una sola cosa con vosotros y formará parte de vosotros.

\par
%\textsuperscript{(449.2)}
\textsuperscript{40:7.4} Éstos son los mortales que han recibido el mandato del Padre Universal: <<Sed perfectos como yo soy perfecto>>\footnote{\textit{Sed perfectos}: Gn 17:1; 1 Re 8:61; Lv 19:2; Dt 18:13; Mt 5:48; 2 Co 13:11; Stg 1:4; 1 P 1:16.}. El Padre se ha dado a vosotros, ha puesto su propio espíritu dentro de vosotros; \textit{por eso} exige una perfección última de vosotros. La narración de la ascensión humana desde las esferas del tiempo donde viven los mortales hasta los reinos divinos de la eternidad constituye un relato fascinante que no está incluido en mi misión, pero esta aventura celestial debería ser el estudio supremo del hombre mortal.

\par
%\textsuperscript{(449.3)}
\textsuperscript{40:7.5} La fusión con un fragmento del Padre Universal equivale a una validación divina de que finalmente se alcanzará el Paraíso, y todos estos mortales fusionados con el Ajustador son la única clase de seres humanos que atraviesan los circuitos de Havona y encuentran a Dios en el Paraíso. Para el mortal fusionado con el Ajustador, la carrera del servicio universal está totalmente abierta. ¡Qué destino tan digno y qué consecución tan gloriosa os espera a cada uno de vosotros! ¿Apreciáis plenamente lo que se ha hecho por vosotros? ¿Comprendéis la grandiosidad de las alturas de los logros eternos que se extienden ante vosotros ---incluso ante vosotros que ahora camináis con dificultad por el humilde sendero de la vida a través de vuestro llamado <<valle de lágrimas>>?

\section*{8. Los mortales fusionados con el Hijo}
\par
%\textsuperscript{(449.4)}
\textsuperscript{40:8.1} Aunque prácticamente todos los mortales sobrevivientes fusionan con su Ajustador en uno de los mundos de las mansiones o inmediatamente después de llegar a las esferas morontiales superiores, existen ciertos casos en que la fusión se retrasa, y algunos no experimentan la seguridad final de sobrevivir hasta que no alcanzan los últimos mundos educativos de la sede del universo; y una minoría de estos candidatos mortales a la vida sin fin no logran en absoluto fusionar su identidad con su fiel Ajustador.

\par
%\textsuperscript{(449.5)}
\textsuperscript{40:8.2} Estos mortales han sido considerados dignos de sobrevivir por las autoridades que juzgan, e incluso sus Ajustadores, por el hecho de regresar de Divinington, han estado de acuerdo en que debían ascender a los mundos de las mansiones. Estos seres han ascendido a través de un sistema, una constelación y los mundos educativos del circuito de Salvington; han disfrutado de las <<setenta veces siete>>\footnote{\textit{Setenta veces siete}: Mt 18:22; Lc 17:3-4.} oportunidades para fusionar, y sin embargo han sido incapaces de alcanzar la unidad con su Ajustador.

\par
%\textsuperscript{(449.6)}
\textsuperscript{40:8.3} Cuando se vuelve evidente que alguna dificultad de sincronización impide la fusión con el Padre, se convoca a los árbitros del Hijo Creador encargados de la supervivencia. Cuando este tribunal de investigación, autorizado por un representante personal de los Ancianos de los Días, determina finalmente que el mortal ascendente no es culpable de ninguna causa que se haya podido descubrir que impide la fusión, lo certifican así en los registros del universo local y trasmiten debidamente sus conclusiones a los Ancianos de los Días. Inmediatamente después, el Ajustador interior regresa enseguida a Divinington para recibir la confirmación de los Monitores Personalizados y, tras esta despedida, el mortal morontial es fusionado inmediatamente con un don individualizado del espíritu del Hijo Creador.

\par
%\textsuperscript{(450.1)}
\textsuperscript{40:8.4} Al igual que las esferas morontiales de Nebadon son compartidas con los mortales fusionados con el Espíritu, estas criaturas fusionadas con el Hijo comparten los servicios de Orvonton con sus hermanos fusionados con el Ajustador, los cuales viajan hacia el interior y la lejana Isla del Paraíso. Son verdaderamente vuestros hermanos, y disfrutaréis mucho de vuestra asociación con ellos cuando paséis por los mundos formativos del superuniverso.

\par
%\textsuperscript{(450.2)}
\textsuperscript{40:8.5} Los mortales fusionados con el Hijo no componen un grupo numeroso, pues hay menos de un millón en el superuniverso de Orvonton. Aparte del destino residencial en el Paraíso, son iguales en todos los sentidos a sus asociados fusionados con el Ajustador. Viajan con frecuencia al Paraíso para llevar a cabo misiones superuniversales, pero raras veces residen allí de manera permanente pues están limitados como clase al superuniverso donde han nacido.

\section*{9. Los mortales fusionados con el Espíritu}
\par
%\textsuperscript{(450.3)}
\textsuperscript{40:9.1} Los mortales ascendentes fusionados con el Espíritu no son personalidades de la Fuente Tercera; están incluidos en el circuito de personalidad del Padre, pero han fusionado con individualizaciones del espíritu premental de la Fuente-Centro Tercera. Esta fusión con el Espíritu nunca se produce en el transcurso de la vida física; sólo tiene lugar en el momento en que el mortal se despierta a la existencia morontial en los mundos de las mansiones. En la experiencia de la fusión no hay ninguna superposición. La criatura volitiva fusiona o bien con el Espíritu, o con el Hijo, o con el Padre. Aquellos que fusionan con el Ajustador, o sea con el Padre, no fusionan nunca con el Espíritu ni con el Hijo.

\par
%\textsuperscript{(450.4)}
\textsuperscript{40:9.2} El hecho de que estos tipos de criaturas mortales no sean candidatos a la fusión con el Ajustador no impide que los Ajustadores habiten en ellos durante la vida en la carne. Los Ajustadores trabajan en la mente de estos seres durante el período de la vida material, pero nunca se unen eternamente con el alma de sus pupilos. Durante esta estancia temporal, los Ajustadores construyen de hecho la misma contrapartida espiritual de la naturaleza mortal ---el alma--- que en los candidatos a la fusión con el Ajustador. Hasta el momento de la muerte física, el trabajo de los Ajustadores es totalmente semejante a su actividad en vuestras propias razas, pero tras la disolución de la muerte, los Ajustadores se despiden eternamente de estos candidatos a la fusión con el Espíritu, y luego se dirigen directamente a Divinington, la sede de todos los Monitores divinos, para esperar allí las nuevas misiones de su orden.

\par
%\textsuperscript{(450.5)}
\textsuperscript{40:9.3} Cuando estos supervivientes dormidos son repersonalizados en los mundos de las mansiones, el lugar de los Ajustadores que han partido es ocupado por una individualización del espíritu de la Ministra Divina, la representante del Espíritu Infinito en el universo local interesado. Esta fusión con el espíritu convierte a estas criaturas sobrevivientes en mortales fusionados con el Espíritu. Estos seres son en todos los sentidos iguales a vosotros en mente y en espíritu; son en verdad vuestros contemporáneos, compartiendo las esferas de las mansiones y las morontiales con vuestra orden de candidatos a la fusión y con aquellos que fusionarán con el Hijo.

\par
%\textsuperscript{(450.6)}
\textsuperscript{40:9.4} Hay sin embargo un detalle que diferencia a los mortales fusionados con el Espíritu de sus hermanos ascendentes: la memoria mortal de la experiencia humana vivida en los mundos materiales de origen sobrevive a la muerte en la carne porque el Ajustador interior ha adquirido una contrapartida, o transcripción, espiritual de aquellos acontecimientos de la vida humana que tuvieron un significado espiritual. Pero en los mortales fusionados con el Espíritu no existe ningún mecanismo de este tipo gracias al cual la memoria humana pueda continuar. Las transcripciones de la memoria realizadas por los Ajustadores están completas e intactas, pero estas adquisiciones son propiedad experiencial de los Ajustadores que han partido, y no están disponibles para las criaturas en las que habitaron anteriormente, las cuales se despiertan por tanto en las salas de resurrección de las esferas morontiales de Nebadon como si fueran unos seres recién creados, unas criaturas sin conciencia de haber tenido una existencia anterior.

\par
%\textsuperscript{(451.1)}
\textsuperscript{40:9.5} A estos hijos del universo local se les permite recuperar una gran parte de la experiencia de su antigua memoria humana haciendo que se la cuenten los serafines y querubines asociados y consultando los registros de su carrera como mortales, archivados por los ángeles registradores. Pueden hacer esto con una seguridad indudable porque, aunque el alma sobreviviente con origen experiencial en la vida material y mortal no tenga ningún recuerdo de los acontecimientos terrestres, posee una reacción residual de reconocimiento experiencial hacia esos acontecimientos olvidados de su experiencia pasada.

\par
%\textsuperscript{(451.2)}
\textsuperscript{40:9.6} Cuando a un mortal fusionado con el Espíritu le cuentan los acontecimientos de su experiencia pasada olvidada, se produce una reacción inmediata de reconocimiento experiencial dentro del alma (de la identidad) de ese sobreviviente, que le confiere instantáneamente al acontecimiento narrado el matiz emocional de la realidad y la calidad intelectual del hecho; esta doble reacción constituye la reconstrucción, el reconocimiento y la validación de una faceta olvidada de su experiencia como mortal.

\par
%\textsuperscript{(451.3)}
\textsuperscript{40:9.7} Incluso entre los candidatos a la fusión con el Ajustador, sólo aquellas experiencias humanas que tenían un valor espiritual son propiedad común del mortal sobreviviente y del Ajustador que ha regresado, y por eso son recordadas inmediatamente después de la supervivencia del mortal. En cuanto a aquellos sucesos que no tenían un significado espiritual, incluso estos fusionados con el Ajustador tienen que depender del atributo de la reacción de reconocimiento del alma sobreviviente. Y puesto que cualquier acontecimiento puede tener una connotación espiritual para un mortal pero no para otro, a un grupo de ascendentes contemporáneos procedentes del mismo planeta les resulta posible reunir su depósito de acontecimientos recordados por sus Ajustadores, y reconstruir así cualquier experiencia que hayan tenido en común y que tenía un valor espiritual en la vida de cualquiera de ellos.

\par
%\textsuperscript{(451.4)}
\textsuperscript{40:9.8} Aunque comprendemos bastante bien estas técnicas de reconstrucción de la memoria, no captamos la técnica para reconocer la personalidad. Las personalidades que en otro tiempo estuvieron asociadas reaccionan de manera mutua, independientemente por completo del funcionamiento de la memoria, aunque la memoria misma y las técnicas para su reconstrucción sean necesarias para conferirle a esta reacción mutua de las personalidades la plenitud del reconocimiento.

\par
%\textsuperscript{(451.5)}
\textsuperscript{40:9.9} Un sobreviviente fusionado con el Espíritu también es capaz de aprender muchas cosas sobre la vida que vivió en la carne volviendo a visitar el mundo donde nació después de la dispensación planetaria en la que vivió. A estos hijos fusionados con el Espíritu se les permite disfrutar de estas oportunidades para investigar su carrera humana, puesto que generalmente están limitados al servicio del universo local. No comparten vuestro elevado y sublime destino en el Cuerpo Paradisiaco de la Finalidad; sólo los mortales fusionados con el Ajustador, u otros seres ascendentes especialmente abrazados, son enrolados en las filas de aquellos que esperan la aventura eterna de la Deidad. Los mortales fusionados con el Espíritu son los ciudadanos permanentes de los universos locales; pueden aspirar al destino del Paraíso, pero no pueden estar seguros de ello. En Nebadon, su hogar universal es el octavo grupo de mundos que rodean a Salvington, un cielo de destino cuya naturaleza y ubicación se parecen mucho a las imaginadas por las tradiciones planetarias de Urantia.

\section*{10. Los destinos ascendentes}
\par
%\textsuperscript{(452.1)}
\textsuperscript{40:10.1} Los mortales fusionados con el Espíritu están generalmente limitados a un universo local. Los supervivientes fusionados con el Hijo están restringidos a un superuniverso; los mortales fusionados con un Ajustador están destinados a penetrar el universo de universos. Los espíritus que fusionan con los mortales siempre ascienden a su nivel de origen; estas entidades espirituales regresan infaliblemente a la esfera de su fuente original.

\par
%\textsuperscript{(452.2)}
\textsuperscript{40:10.2} Los mortales fusionados con el Espíritu pertenecen al universo local; generalmente no ascienden más allá de los confines de su reino nativo, más allá de las fronteras del alcance espacial del espíritu que los impregna. Los ascendentes fusionados con el Hijo se elevan igualmente hasta la fuente que los ha dotado del espíritu, pues al igual que el Espíritu de la Verdad de un Hijo Creador se focaliza en la Ministra Divina asociada, su <<espíritu de fusión>> lo ponen en ejecución los Espíritus Reflectantes de los universos superiores. Estas relaciones espirituales entre los niveles locales y superuniversales de Dios Séptuple pueden ser difíciles de explicar pero no de discernir, pues están reveladas inequívocamente en los hijos de los Espíritus Reflectantes ---las Voces secoráficas de los Hijos Creadores. Como el Ajustador del Pensamiento procede del Padre que está en el Paraíso, nunca se detiene hasta que el hijo mortal se halla delante del Dios eterno.

\par
%\textsuperscript{(452.3)}
\textsuperscript{40:10.3} En la técnica de la asociación, la variable misteriosa por la que un ser mortal no fusiona o no puede fusionar eternamente con el Ajustador del Pensamiento interior, puede parecer revelar un defecto en el programa de la ascensión; superficialmente, la fusión con el Hijo o con el Espíritu parecen ser compensaciones por los fallos inexplicados en algún detalle del plan para alcanzar el Paraíso; pero todas estas conclusiones son erróneas; se nos enseña que todos estos sucesos se desarrollan de conformidad con las leyes establecidas por los Gobernantes Supremos del Universo.

\par
%\textsuperscript{(452.4)}
\textsuperscript{40:10.4} Hemos analizado este problema y hemos llegado a la conclusión indudable de que el envío de todos los mortales hacia un destino último en el Paraíso sería injusto para los universos espacio-temporales, ya que las cortes de los Hijos Creadores y de los Ancianos de los Días dependerían entonces por completo de los servicios de aquellos que están de paso hacia otros reinos más elevados. Y parece ser perfectamente justo que los gobiernos locales y superuniversales estén provistos, cada uno de ellos, de un grupo permanente de ciudadanos ascendentes; que las actividades de estas administraciones se enriquezcan con los esfuerzos de ciertos grupos de mortales glorificados que tienen un estado permanente, los complementos evolutivos de los abandontarios y de los susatias. Ahora bien, es totalmente evidente que el programa actual de la ascensión proporciona eficazmente a las administraciones espacio-temporales estos grupos de criaturas ascendentes; y muchas veces nos hemos preguntado: ¿Representa todo esto una parte intencional de los planes que reflejan en todos los aspectos la sabiduría de los Arquitectos del Universo Maestro, destinados a suministrar a los Hijos Creadores y a los Ancianos de los Días una población ascendente permanente con órdenes evolucionadas de ciudadanos que serán cada vez más competentes para llevar adelante los asuntos de estos reinos en las eras universales por venir?

\par
%\textsuperscript{(452.5)}
\textsuperscript{40:10.5} El hecho de que el destino de los mortales varíe de esta forma no prueba de ninguna manera que uno de estos destinos sea necesariamente más grande o más pequeño que el otro, sino simplemente que son diferentes. Los ascendentes fusionados con el Ajustador tienen en verdad una magnífica y gloriosa carrera como finalitarios que se extiende ante ellos en el eterno futuro, pero esto no significa que sean preferidos a sus hermanos ascendentes. No existe ningún favoritismo, nada que sea arbitrario, en el funcionamiento selectivo del plan divino para la supervivencia de los mortales.

\par
%\textsuperscript{(453.1)}
\textsuperscript{40:10.6} Aunque los finalitarios fusionados con el Ajustador disfrutan evidentemente de la oportunidad de servir más grande de todas, el hecho de alcanzar esta meta los aparta automáticamente de la posibilidad de participar en la lucha secular de algún universo o superuniverso, desde las épocas más primitivas y menos estables hasta las eras posteriores y establecidas en que se ha alcanzado una perfección relativa. Los finalitarios adquieren una maravillosa y extensa experiencia de servicio transitorio en los siete segmentos del gran universo, pero generalmente no adquieren ese íntimo conocimiento de un universo concreto que incluso ahora ya caracteriza a los veteranos del Cuerpo de la Finalización de Nebadon fusionados con el Espíritu. Estos seres disfrutan de la oportunidad de presenciar la procesión ascendente de las eras planetarias a medida que se despliegan unas tras otras en diez millones de mundos habitados. Durante el fiel servicio de estos ciudadanos del universo local, las experiencias se superponen a las experiencias hasta que la plenitud de los tiempos hace madurar esa sabiduría de elevada calidad engendrada por la experiencia focalizada ---la sabiduría con \textit{autoridad}--- y esto en sí mismo es un factor vital para la estabilización de cualquier universo local.

\par
%\textsuperscript{(453.2)}
\textsuperscript{40:10.7} Aquello que sucede con los fusionados con el Espíritu sucede también con los mortales fusionados con el Hijo que han conseguido el estado residencial en Uversa. Algunos de estos seres proceden de las épocas más tempranas de Orvonton y representan un cuerpo que se acumula lentamente, con una sabiduría cada vez más profunda en perspicacia, que contribuye de forma creciente con su servicio al bienestar y a la estabilización final del séptimo superuniverso.

\par
%\textsuperscript{(453.3)}
\textsuperscript{40:10.8} No sabemos cuál será el destino final de estas órdenes estacionarias de ciudadanos de los universos locales y de los superuniversos, pero es muy posible que, cuando los finalitarios del Paraíso exploren las fronteras en expansión de la divinidad en los sistemas planetarios del primer nivel del espacio exterior, sus hermanos de la lucha evolutiva ascendente, fusionados con el Hijo o con el Espíritu, contribuirán de manera aceptable al mantenimiento del equilibrio experiencial de los superuniversos perfeccionados, mientras que se mantendrán preparados para dar la bienvenida a la oleada entrante de peregrinos en dirección al Paraíso que podrán, en esa época lejana, entrar a raudales en Orvonton y en sus creaciones hermanas como un inmenso torrente, en busca del espíritu, procedente de esas galaxias actualmente inexploradas y deshabitadas del espacio exterior.

\par
%\textsuperscript{(453.4)}
\textsuperscript{40:10.9} Aunque la mayoría de los fusionados con el Espíritu sirven de forma permanente como ciudadanos de los universos locales, no todos lo hacen. Si alguna fase de su ministerio universal requiriera su presencia personal en el superuniverso, entonces se efectuarían en estos ciudadanos esas transformaciones del ser que les permitirían ascender al universo superior; y tras la llegada de los Guardianes Celestiales con la orden de presentar a estos mortales fusionados con el Espíritu ante las cortes de los Ancianos de los Días, ascenderían así para no regresar jamás. Se convierten en los pupilos del superuniverso, y sirven de forma permanente como ayudantes de los Guardianes Celestiales, salvo aquellos pocos que son llamados a su vez al servicio del Paraíso y de Havona.

\par
%\textsuperscript{(453.5)}
\textsuperscript{40:10.10} Al igual que sus hermanos fusionados con el Espíritu, los fusionados con el Hijo ni atraviesan Havona ni alcanzan el Paraíso, a menos que hayan sufrido ciertas transformaciones modificadoras. Estos cambios se han efectuado, por buenas y suficientes razones, en algunos supervivientes fusionados con el Hijo, y a estos seres se les puede encontrar de vez en cuando en los siete circuitos del universo central. Así es como cierto número de mortales fusionados con el Hijo o con el Espíritu ascienden efectivamente hasta el Paraíso, alcanzan una meta equivalente en muchos aspectos a la que espera a los mortales fusionados con el Padre.

\par
%\textsuperscript{(453.6)}
\textsuperscript{40:10.11} Los mortales fusionados con el Padre son finalitarios en potencia; su destino es el Padre Universal, y de hecho llegan hasta él, pero dentro del ámbito de la presente era del universo, los finalitarios, como tales, no alcanzan su destino. Siguen siendo criaturas inacabadas ---espíritus de la sexta fase--- y por tanto inactivas en los dominios evolutivos cuyo estado es anterior al de la luz y la vida.

\par
%\textsuperscript{(454.1)}
\textsuperscript{40:10.12} Cuando un finalitario mortal es abrazado por la Trinidad ---cuando se convierte en un Hijo Trinitizado, como por ejemplo un Mensajero Poderoso--- entonces ese finalitario ha alcanzado su destino, al menos durante la presente era del universo. Los Mensajeros Poderosos y sus compañeros quizás no sean, en el sentido exacto, espíritus de la séptima fase, pero además de otras cosas, el abrazo de la Trinidad los dota de todo aquello que un finalitario conseguirá algún día como espíritu de la séptima fase. Después de ser trinitizados, los mortales fusionados con el Espíritu o con el Hijo pasan por la experiencia del Paraíso con los ascendentes fusionados con el Ajustador, siendo entonces idénticos a ellos en todas las cuestiones relacionadas con la administración superuniversal. Estos Hijos de la Elección o de la Consecución Trinitizados son, al menos por el momento, criaturas acabadas, en contraste con los finalitarios, que son en la actualidad criaturas inacabadas.

\par
%\textsuperscript{(454.2)}
\textsuperscript{40:10.13} Así pues, a fin de cuentas, no sería del todo adecuado utilizar las palabras <<más grande>> o <<menor>> al comparar los destinos de las órdenes ascendentes de filiación. Cada uno de estos hijos de Dios comparte la paternidad de Dios, y Dios ama a cada uno de sus hijos\footnote{\textit{Dios ama a sus hijos}: Jn 3:16; 1 Jn 3:1; 4:9-10.} creados de la misma manera; no hace más acepción\footnote{\textit{Dios no hace acepción de personas}: 2 Cr 19:7; Job 34:19; Eclo 35:12; Hch 10:34; Ro 2:11; Gl 2:6; 3:28; Ef 6:9; Col 3:11.} de los destinos ascendentes que de las criaturas que puedan alcanzar esos destinos. El Padre ama a \textit{cada uno} de sus hijos, y este afecto no es menos que verdadero, sagrado, divino, ilimitado, eterno y único ---un amor otorgado a \textit{este} hijo y a \textit{aquel} hijo, de manera individual, personal y exclusiva. Y este amor eclipsa por completo todos los demás hechos. La filiación es la relación suprema de la criatura con el Creador.

\par
%\textsuperscript{(454.3)}
\textsuperscript{40:10.14} Como mortales, ahora podéis reconocer vuestro lugar en la familia de la filiación divina y empezar a sentir la obligación de aprovecharos de las ventajas que ofrece tan abundantemente el plan paradisiaco para la supervivencia de los mortales, un plan que fue tan realzado e iluminado por la experiencia de la vida de un Hijo donador. Se han proporcionado todas las facilidades y todos los poderes para asegurar que alcanzaréis finalmente la meta paradisiaca de la perfección divina.

\par
%\textsuperscript{(454.4)}
\textsuperscript{40:10.15} [Presentado por un Mensajero Poderoso vinculado temporalmente al estado mayor de Gabriel de Salvington.]


\chapter{Documento 41. Aspectos físicos del universo local}
\par
%\textsuperscript{(455.1)}
\textsuperscript{41:0.1} EL FENÓMENO espacial característico que diferencia a cada creación local de todas las demás es la presencia del Espíritu Creativo. Todo Nebadon está ciertamente impregnado por la presencia espacial de la Ministra Divina de Salvington, y esta presencia termina igual de ciertamente en los bordes exteriores de nuestro universo local. Nebadon \textit{es} aquello que está impregnado por el Espíritu Madre de nuestro universo local; aquello que se extiende más allá de su presencia espacial está fuera de Nebadon, son las regiones espaciales del superuniverso de Orvonton exteriores a Nebadon ---otros universos locales.

\par
%\textsuperscript{(455.2)}
\textsuperscript{41:0.2} Aunque la organización administrativa del gran universo revela una división bien definida entre los gobiernos del universo central, los superuniversos y los universos locales, y aunque estas divisiones tienen su paralelismo astronómico en la separación espacial entre Havona y los siete superuniversos, no existen unas líneas tan claras de demarcación física que separen a las creaciones locales. Incluso los sectores mayores y menores de Orvonton son claramente distinguibles (para nosotros), pero no es tan fácil identificar los límites físicos de los universos locales. Esto se debe a que estas creaciones locales están organizadas administrativamente de acuerdo con ciertos principios \textit{creativos} que gobiernan la segmentación de la carga energética total de un superuniverso, mientras que sus componentes físicos, las esferas del espacio ---los soles, las islas oscuras, los planetas, etc.--- tienen su origen principalmente en las nebulosas, y éstas hacen su aparición astronómica de acuerdo con ciertos planes \textit{precreativos} (trascendentales) de los Arquitectos del Universo Maestro.

\par
%\textsuperscript{(455.3)}
\textsuperscript{41:0.3} Una o más de estas nebulosas ---e incluso muchas--- pueden estar incluidas dentro del dominio de un solo universo local, lo mismo que Nebadon se formó físicamente con la progenie estelar y planetaria de Andronover y de otras nebulosas. Las esferas de Nebadon tienen una ascendencia nebular diversa, pero todas tuvieron cierta frecuencia mínima de movimiento espacial que fue ajustada de tal manera por los esfuerzos inteligentes de los directores del poder que produjeron nuestro agregado actual de cuerpos espaciales, los cuales viajan juntos como una unidad contigua en las órbitas del superuniverso.

\par
%\textsuperscript{(455.4)}
\textsuperscript{41:0.4} Ésta es la constitución de la nube estelar local de Nebadon, que actualmente gira en una órbita cada vez más estable alrededor del centro, situado en Sagitario, del sector menor de Orvonton al cual pertenece nuestra creación local.

\section*{1. Los Centros de Poder de Nebadon}
\par
%\textsuperscript{(455.5)}
\textsuperscript{41:1.1} Las nebulosas espirales y de otros tipos, las ruedas madres de las esferas del espacio, son iniciadas por los organizadores de fuerza del Paraíso; después de la evolución de la reacción gravitatoria de la nebulosa, son reemplazados en su función superuniversal por los centros de poder y los controladores físicos, que asumen de inmediato la plena responsabilidad de dirigir la evolución física de las generaciones siguientes de descendientes estelares y planetarios. Tras la llegada de nuestro Hijo Creador, esta supervisión física del preuniverso de Nebadon fue coordinada inmediatamente con su plan para organizar el universo. Dentro de los dominios de este Hijo Paradisiaco de Dios, los Centros Supremos del Poder y los Controladores Físicos Maestros colaboraron con los Supervisores del Poder Morontial y con otras entidades, aparecidos más tarde, para dar nacimiento al inmenso complejo de líneas de comunicación, circuitos de energía y canales de poder que ligan firmemente los múltiples cuerpos espaciales de Nebadon en una sola unidad administrativa integrada.

\par
%\textsuperscript{(456.1)}
\textsuperscript{41:1.2} Cien Centros Supremos de Poder de la cuarta orden están asignados de manera permanente a nuestro universo local. Estos seres reciben las líneas entrantes de poder procedentes de los centros de la tercera orden de Uversa, y retransmiten los circuitos reducidos y modificados a los centros de poder de nuestras constelaciones y sistemas. Estos centros de poder actúan en asociación para producir el sistema viviente de control y de igualación que funciona para mantener el equilibrio y la distribución de las energías que, de otra manera, serían fluctuantes y variables. Sin embargo, los centros de poder no se ocupan de los trastornos energéticos transitorios y locales tales como las manchas solares y las perturbaciones eléctricas del sistema; la luz y la electricidad no son las energías fundamentales del espacio; son manifestaciones secundarias y subsidiarias.

\par
%\textsuperscript{(456.2)}
\textsuperscript{41:1.3} Los cien centros del universo local están estacionados en Salvington, donde ejercen su actividad en el centro energético exacto de esta esfera. Las esferas arquitectónicas tales como Salvington, Edentia y Jerusem están iluminadas, calentadas y alimentadas energéticamente mediante unos métodos que las hacen totalmente independientes de los soles del espacio. Los centros de poder y los controladores físicos construyeron ---hicieron a medida--- estas esferas, y fueron diseñadas para ejercer una poderosa influencia sobre la distribución de la energía. Basando sus actividades en estos puntos focales de control de la energía, los centros de poder orientan y canalizan las energías físicas del espacio por medio de sus presencias vivientes. Y estos circuitos energéticos son fundamentales para todos los fenómenos físico-materiales y morontio-espirituales.

\par
%\textsuperscript{(456.3)}
\textsuperscript{41:1.4} Diez Centros Supremos de Poder de la quinta orden están asignados a cada una de las subdivisiones primarias de Nebadon, a las cien constelaciones. En vuestra constelación, Norlatiadek, no están estacionados en la esfera sede, sino que están situados en el centro del enorme sistema estelar que constituye el núcleo físico de la constelación. En Edentia hay diez controladores maquinales asociados y diez frandalanks que están en conexión perfecta y constante con los centros de poder cercanos.

\par
%\textsuperscript{(456.4)}
\textsuperscript{41:1.5} Un Centro Supremo de Poder de la sexta orden está estacionado en el centro exacto de gravedad de cada sistema local. En el sistema de Satania, el centro de poder\footnote{\textit{Centro de poder}: Job 26:7.} destinado allí ocupa una isla oscura del espacio situada en el centro astronómico del sistema. Muchas de estas islas oscuras son inmensas dinamos que movilizan y orientan ciertas energías espaciales, y estas circunstancias naturales son utilizadas eficazmente por el Centro de Poder de Satania, cuya masa viviente funciona como punto de conexión con los centros superiores, dirigiendo las corrientes de poder más materializado hacia los Controladores Físicos Maestros estacionados en los planetas evolutivos del espacio.

\section*{2. Los Controladores Físicos de Satania}
\par
%\textsuperscript{(456.5)}
\textsuperscript{41:2.1} Aunque los Controladores Físicos Maestros sirven con los centros de poder en todo el gran universo, sus funciones en un sistema local como Satania son más fáciles de comprender. Satania es uno de los cien sistemas locales que componen la organización administrativa de la constelación de Norlatiadek, y tiene por vecinos inmediatos a los sistemas de Sandmatia, Assuntia, Porogia, Sortoria, Rantulia y Glantonia. Los sistemas de Norlatiadek difieren en muchos aspectos, pero todos son evolutivos y progresivos de manera muy semejante a Satania.

\par
%\textsuperscript{(457.1)}
\textsuperscript{41:2.2} Satania mismo está compuesto por más de siete mil grupos astronómicos o sistemas físicos, pocos de los cuales han tenido un origen similar al de vuestro sistema solar. El centro astronómico de Satania es una enorme isla oscura del espacio que, con sus esferas acompañantes, está situada no lejos de la sede del gobierno del sistema.

\par
%\textsuperscript{(457.2)}
\textsuperscript{41:2.3} A excepción de la presencia del centro de poder asignado, la supervisión de todo el sistema de energía física de Satania está centrada en Jerusem. Un Controlador Físico Maestro, estacionado en esta esfera sede, trabaja en coordinación con el centro de poder del sistema, sirviendo como jefe de enlace de los inspectores de poder domiciliados en Jerusem y que ejercen su actividad en todo el sistema local.

\par
%\textsuperscript{(457.3)}
\textsuperscript{41:2.4} La puesta en circuito y la canalización de la energía están supervisadas por los quinientos mil manipuladores vivientes e inteligentes de la energía dispersos por todo Satania. Gracias a la acción de estos controladores físicos, los centros de poder supervisores controlan de manera completa y perfecta la mayoría de las energías fundamentales del espacio, incluyendo las emanaciones de los orbes extremadamente calientes y de las esferas oscuras cargadas de energía. Este grupo de entidades vivientes puede movilizar, transformar, transmutar, manipular y transmitir casi todas las energías físicas del espacio organizado.

\par
%\textsuperscript{(457.4)}
\textsuperscript{41:2.5} La vida posee una capacidad inherente para movilizar y transmutar la energía universal. Estáis familiarizados con la acción de la vida vegetal que transforma la energía material de la luz en las manifestaciones variadas del reino vegetal. También conocéis una parte del método por el cual esta energía vegetativa se puede convertir en los fenómenos de las actividades animales, pero no sabéis prácticamente nada sobre la técnica de los directores de poder y de los controladores físicos, que están dotados de la capacidad de movilizar, transformar, orientar y concentrar las múltiples energías del espacio.

\par
%\textsuperscript{(457.5)}
\textsuperscript{41:2.6} Estos seres de los reinos energéticos no se ocupan directamente de la energía como factor componente de las criaturas vivientes, ni tampoco del ámbito de la química fisiológica. A veces se ocupan de los preliminares físicos de la vida, de elaborar los sistemas energéticos que pueden servir como vehículos físicos para las energías vivientes de los organismos materiales elementales. En cierto modo, los controladores físicos están relacionados con las manifestaciones previvientes de la energía material de la misma forma que los espíritus ayudantes de la mente se ocupan de las funciones preespirituales de la mente material.

\par
%\textsuperscript{(457.6)}
\textsuperscript{41:2.7} Estas criaturas inteligentes que controlan el poder y dirigen la energía deben ajustar su técnica en cada esfera de acuerdo con la constitución y la arquitectura físicas de ese planeta. Utilizan infaliblemente los cálculos y las deducciones de sus grupos respectivos de físicos y otros asesores técnicos sobre la influencia local de los soles extremadamente calientes y de otros tipos de estrellas supercargadas. También deben contar con los enormes gigantes fríos y oscuros del espacio y con las nubes rebosantes de polvo estelar; todos estos elementos materiales se tienen en cuenta en los problemas prácticos de la manipulación de la energía.

\par
%\textsuperscript{(457.7)}
\textsuperscript{41:2.8} Los Controladores Físicos Maestros tienen la responsabilidad de supervisar la energía-poder en los mundos evolutivos habitados, pero estos seres no son responsables de todos los desarreglos energéticos que tienen lugar en Urantia. Existen numerosas razones para que se produzcan estas perturbaciones, algunas de las cuales están más allá del ámbito y del control de los custodios físicos. Urantia se encuentra en la trayectoria de unas energías asombrosas, un pequeño planeta en un circuito de masas enormes, y los controladores locales a veces emplean un enorme número de miembros de su orden en un esfuerzo por igualar estas líneas de energía. Lo consiguen bastante bien con los circuitos físicos de Satania, pero tienen dificultades para aislar al planeta de las poderosas corrientes de Norlatiadek.

\section*{3. Nuestros asociados estelares}
\par
%\textsuperscript{(458.1)}
\textsuperscript{41:3.1} Hay más de dos mil soles\footnote{\textit{Nuestros asociados estelares}: Gn 1:14-15; Job 9:9; 38:31-33; Sal 147:4; Am 5:8.} brillantes que derraman su luz y su energía en Satania, y vuestro propio Sol es un globo resplandeciente de tipo medio. De los treinta soles más cercanos al vuestro, sólo tres son más brillantes. Los Directores del Poder Universal inician las corrientes especializadas de energía que actúan entre las estrellas individuales y sus sistemas respectivos. Estos hornos solares, junto con los gigantes oscuros del espacio, sirven de parada obligada a los centros de poder y a los controladores físicos para concentrar y orientar eficazmente los circuitos energéticos de las creaciones materiales.

\par
%\textsuperscript{(458.2)}
\textsuperscript{41:3.2} Los soles de Nebadon no son diferentes a los de otros universos. La composición material de todos los soles, islas oscuras, planetas y satélites, e incluso meteoros, es totalmente idéntica. Estos soles tienen un diámetro medio de casi un millón seiscientos mil kilómetros, pero el de vuestro propio globo solar es ligeramente menor. La estrella más grande del universo, la nube estelar de Antares, tiene cuatrocientas cincuenta veces el diámetro de vuestro Sol y sesenta millones de veces su volumen. Pero hay espacio abundante para alojar a todos estos soles enormes. Tienen, en comparación, tanto sitio en el espacio como una docena de naranjas circulando por el interior de Urantia si el planeta fuera un globo hueco.

\par
%\textsuperscript{(458.3)}
\textsuperscript{41:3.3} Cuando una rueda madre nebular expulsa soles demasiado grandes, éstos se rompen pronto o forman estrellas dobles. Todos los soles son al principio verdaderamente gaseosos, aunque más tarde pueden existir transitoriamente en estado semilíquido. Cuando vuestro Sol alcanzó este estado casi líquido de presión supergaseosa, no era lo suficientemente grande como para partirse por el ecuador, siendo éste un tipo de formación de las estrellas dobles.

\par
%\textsuperscript{(458.4)}
\textsuperscript{41:3.4} Cuando estas esferas llameantes tienen menos de una décima parte el tamaño de vuestro Sol, se contraen, se condensan y se enfrían rápidamente. Cuando tienen más de treinta veces el tamaño del Sol ---o más bien treinta veces su contenido bruto en materia real--- los soles se parten rápidamente en dos cuerpos separados y se convierten o bien en los centros de nuevos sistemas, o bien permanecen dentro de la atracción gravitatoria del otro sol, girando alrededor de un centro común como un tipo de estrella doble.

\par
%\textsuperscript{(458.5)}
\textsuperscript{41:3.5} Entre las mayores erupciones cósmicas de Orvonton, la más reciente fue la explosión extraordinaria de una estrella doble, cuya luz llegó a Urantia en el año 1572. Esta conflagración fue tan intensa que la explosión era claramente visible en pleno día.

\par
%\textsuperscript{(458.6)}
\textsuperscript{41:3.6} No todas las estrellas son sólidas, pero muchas de las más antiguas sí lo son. Algunas de las estrellas rojizas que brillan débilmente han adquirido en el centro de sus masas enormes una densidad que se podría expresar diciendo que si un centímetro cúbico de dicha estrella estuviera en Urantia pesaría ciento sesenta y seis kilos. La enorme presión, acompañada de la pérdida de calor y de la energía circulante, ha conducido a acercar cada vez más las órbitas de las unidades materiales básicas hasta que en este momento se aproximan mucho al estado de la condensación electrónica. Este proceso de enfriamiento y de contracción puede continuar hasta el punto límite y crítico de explosión de la condensación ultimatónica.

\par
%\textsuperscript{(459.1)}
\textsuperscript{41:3.7} La mayor parte de los soles gigantes son relativamente jóvenes; la mayoría de las estrellas enanas son viejas, pero no todas. Las enanas procedentes de colisiones pueden ser muy jóvenes y pueden brillar con una intensa luz blanca sin haber conocido nunca la etapa roja inicial del brillo de la juventud. Tanto los soles muy jóvenes como los muy viejos brillan generalmente con un color rojizo. El matiz amarillento indica una juventud moderada o la vejez que se acerca, pero la luz blanca brillante significa una vida adulta vigorosa y prolongada.

\par
%\textsuperscript{(459.2)}
\textsuperscript{41:3.8} Aunque los soles adolescentes no pasan todos, al menos visiblemente, por una etapa de pulsaciones, cuando miráis al espacio podéis observar muchas de estas estrellas más jóvenes cuyos gigantescos movimientos respiratorios necesitan de dos a siete días para completar un ciclo. Vuestro propio Sol lleva consigo todavía un legado decreciente de las poderosas hinchazones de sus tiempos más jóvenes, pero el periodo de tres días y medio de las antiguas pulsaciones se ha alargado hasta los ciclos actuales de once años y medio de las manchas solares.

\par
%\textsuperscript{(459.3)}
\textsuperscript{41:3.9} Las variables estelares tienen numerosos orígenes. En algunas estrellas dobles, las mareas causadas por los rápidos cambios de distancia mientras los dos cuerpos giran alrededor de sus órbitas también ocasionan fluctuaciones periódicas de la luz. Estas variaciones gravitatorias producen llamaradas regulares y recurrentes, de la misma manera que la captura de los meteoros, por el acrecentamiento de la materia energética en la superficie, tiene como resultado un destello de luz relativamente repentino que disminuye rápidamente hasta el brillo normal de ese sol. A veces un sol captura una corriente de meteoros en una línea de oposición gravitatoria menor, y las colisiones producen de vez en cuando llamaradas estelares, pero la mayoría de estos fenómenos se debe totalmente a las fluctuaciones internas.

\par
%\textsuperscript{(459.4)}
\textsuperscript{41:3.10} El período de fluctuación de la luz, en un grupo de estrellas variables, depende directamente de la luminosidad, y el conocimiento de este hecho permite a los astrónomos utilizar estos soles como faros universales, o puntos de medición precisos, para explorar ulteriormente los enjambres distantes de estrellas. Con esta técnica es posible medir las distancias estelares con mayor precisión hasta más allá de un millón de años luz de distancia. Algún día, los métodos mejores para medir el espacio y la técnica telescópica más perfeccionada revelarán más plenamente las diez grandes divisiones del superuniverso de Orvonton; al menos reconoceréis ocho de estos inmensos sectores como enormes enjambres de estrellas bastante simétricos.

\section*{4. La densidad del Sol}
\par
%\textsuperscript{(459.5)}
\textsuperscript{41:4.1} La masa de vuestro Sol es ligeramente mayor de lo que estiman vuestros físicos, que han calculado que tiene unos mil ochocientos cuatrillones (1,8 x 10\textsuperscript{27}) de toneladas. Actualmente se encuentra casi a medio camino entre las estrellas más densas y las más difusas, y tiene alrededor de una vez y media la densidad del agua. Pero vuestro Sol no es ni líquido ni sólido ---es gaseoso--- y esto es así a pesar de la dificultad de explicar cómo puede alcanzar la materia gaseosa esta densidad e incluso otras mucho mayores.

\par
%\textsuperscript{(459.6)}
\textsuperscript{41:4.2} Los estados sólidos, líquidos y gaseosos son cuestiones de relaciones atómico-moleculares, pero la densidad es una relación entre el espacio y la masa. La densidad varía directamente con la cantidad de masa en el espacio, e inversamente con la cantidad de espacio en la masa, del espacio que se encuentra entre los núcleos centrales de la materia y las partículas que giran alrededor de estos centros, así como del espacio que existe dentro de estas partículas materiales.

\par
%\textsuperscript{(459.7)}
\textsuperscript{41:4.3} Las estrellas que se enfrían pueden ser físicamente gaseosas y enormemente densas al mismo tiempo. No estáis familiarizados con los \textit{supergases} solares, pero estas formas de materia y otras formas poco usuales explican cómo incluso los soles no sólidos pueden alcanzar una densidad equivalente a la del hierro ---casi la misma que tiene Urantia--- y sin embargo encontrarse en un estado gaseoso extremadamente caliente y continuar funcionando como soles. En estos densos supergases, los átomos son excepcionalmente pequeños y contienen pocos electrones. Estos soles también han perdido en gran parte sus reservas energéticas de ultimatones libres.

\par
%\textsuperscript{(460.1)}
\textsuperscript{41:4.4} Uno de los soles cercanos a vosotros, que empezó su vida con casi la misma masa que el vuestro, se ha contraído ahora hasta tener casi el tamaño de Urantia, y se ha vuelto cuarenta mil veces más denso que vuestro Sol. El peso de este sólido-gaseoso caliente-frío es de unos cincuenta y cinco kilos por centímetro cúbico. Y este sol sigue brillando con un débil resplandor rojizo, la tenue luz senil de un monarca de luz moribundo.

\par
%\textsuperscript{(460.2)}
\textsuperscript{41:4.5} Sin embargo, la mayor parte de los soles no son tan densos. Uno de vuestros vecinos más cercanos posee una densidad exactamente igual a la de vuestra atmósfera a nivel del mar. Si estuvierais en el interior de este sol no podríais discernir nada. Y si la temperatura lo permitiera, podríais penetrar en la mayoría de los soles que parpadean en el cielo nocturno, pero no observaríais más materia que la que percibís en el aire de vuestras salas de estar terrestres.

\par
%\textsuperscript{(460.3)}
\textsuperscript{41:4.6} El masivo sol de Veluntia, uno de los más grandes de Orvonton, posee una densidad que sólo es una milésima parte la de la atmósfera de Urantia. Si su composición fuera similar a la de vuestra atmósfera y no estuviera supercaliente, habría tal vacío que los seres humanos se ahogarían rápidamente si estuvieran dentro de él.

\par
%\textsuperscript{(460.4)}
\textsuperscript{41:4.7} Otro de los gigantes de Orvonton tiene ahora una temperatura superficial de unos mil seiscientos grados (C). Su diámetro mide más de cuatrocientos ochenta millones de kilómetros ---hay espacio suficiente para alojar a vuestro Sol y a la órbita actual de la Tierra. Sin embargo, a pesar de este enorme tamaño, más de cuarenta millones de veces el de vuestro Sol, su masa sólo es unas treinta veces mayor. Estos soles enormes tienen una periferia tan extensa que casi alcanza a la de los otros.

\section*{5. La radiación solar}
\par
%\textsuperscript{(460.5)}
\textsuperscript{41:5.1} Los soles del espacio no son muy densos, y este hecho queda demostrado por las corrientes continuas de energías luminosas que se escapan de ellos. Una densidad demasiado grande retendría la luz por opacidad hasta que la presión de la energía luminosa alcanzara el punto de explosión. La enorme presión de la luz o del gas dentro de un sol es la que hace que emita tal corriente de energía como para penetrar el espacio durante millones y millones de kilómetros para energizar, iluminar y calentar los planetas lejanos. Cinco metros de superficie con la densidad de Urantia impedirían eficazmente el escape de todos los rayos X y de todas las energías luminosas de un sol, hasta que la presión interna creciente de las energías que se acumulan como resultado del desmembramiento atómico vencería la gravedad con una enorme explosión.

\par
%\textsuperscript{(460.6)}
\textsuperscript{41:5.2} En presencia de los gases propulsivos, la luz es extremadamente explosiva cuando está confinada a altas temperaturas por muros opacos de contención. La luz es real. Tal como valoráis la energía y el poder en vuestro mundo, la luz del Sol sería económica a dos millones de dólares el kilo.

\par
%\textsuperscript{(460.7)}
\textsuperscript{41:5.3} El interior de vuestro Sol es un enorme generador de rayos X. Los soles se sostienen desde el interior por medio del bombardeo incesante de estas poderosas emanaciones.

\par
%\textsuperscript{(460.8)}
\textsuperscript{41:5.4} Un electrón estimulado por los rayos X necesita más de medio millón de años para abrirse camino desde el centro mismo de un sol medio hasta la superficie solar, de donde parte hacia su aventura espacial quizás para calentar un planeta habitado, para ser capturado por un meteoro, para participar en el nacimiento de un átomo, para ser atraído por una isla oscura del espacio extremadamente cargada o para terminar su vuelo espacial cayendo finalmente en la superficie de un sol similar al que le dio origen.

\par
%\textsuperscript{(461.1)}
\textsuperscript{41:5.5} Los rayos X del interior de un sol cargan los electrones extremadamente calientes y agitados con una energía suficiente como para enviarlos a través del espacio, más allá de la multitud de influencias obstaculizantes de la materia intermedia, y a pesar de las atracciones gravitatorias divergentes, hasta las esferas distantes de los sistemas lejanos. La gran energía que se necesita para escapar de las garras de la gravedad de un sol es suficiente como para asegurar que el rayo de sol viajará a una velocidad constante hasta que encuentre considerables masas de materia; después de lo cual se transformará rápidamente en calor con la liberación de otras energías.

\par
%\textsuperscript{(461.2)}
\textsuperscript{41:5.6} Ya sea como luz o bajo otras formas, la energía se desplaza hacia adelante en línea recta en su vuelo por el espacio. Las partículas reales con existencia material atraviesan el espacio como una descarga de fusilería. Avanzan en línea o en procesión recta e ininterrumpida, salvo cuando son guiadas por fuerzas superiores, y salvo cuando obedecen a la atracción gravitatoria lineal inherente a las masas materiales y a la presencia gravitatoria circular de la Isla del Paraíso.

\par
%\textsuperscript{(461.3)}
\textsuperscript{41:5.7} La energía solar parece que se propulsa en ondas, pero esto se debe a la acción de diversas influencias coexistentes. Una forma dada de energía organizada no se desplaza en ondas sino en línea recta. La presencia de una segunda o de una tercera forma de energía-fuerza puede hacer que la corriente observada \textit{parezca} viajar en formación ondulada, al igual que durante una tormenta cegadora acompañada de fuertes vientos, el agua parece caer a veces en forma de cortina o descender en oleadas. Las gotas de lluvia caen en una procesión ininterrumpida de líneas rectas, pero la acción del viento es tal que produce la apariencia visible de cortinas de agua y de oleadas de gotas.

\par
%\textsuperscript{(461.4)}
\textsuperscript{41:5.8} La acción de ciertas energías secundarias y de otras energías no descubiertas, presentes en las regiones espaciales de vuestro universo local, es tal que las emanaciones de luz solar parecen ejecutar ciertos fenómenos ondulados, y además parecen estar cortadas en porciones infinitesimales de una longitud y de un peso determinados. Desde un punto de vista práctico, esto es exactamente lo que sucede. Apenas podéis esperar llegar a comprender mejor el comportamiento de la luz hasta el momento en que adquiráis un concepto más claro de la interacción y de la interrelación de las diversas fuerzas espaciales y energías solares que actúan en las regiones espaciales de Nebadon. Vuestra confusión actual se debe también a que captáis de manera incompleta este problema en el que están implicadas las actividades interasociadas del control personal y no personal del universo maestro ---las presencias, las actuaciones y la coordinación del Actor Conjunto y del Absoluto Incalificado.

\section*{6. El calcio ---el vagabundo del espacio}
\par
%\textsuperscript{(461.5)}
\textsuperscript{41:6.1} En el momento de descifrar los fenómenos espectrales se debe recordar que el espacio no está vacío; que la luz, cuando atraviesa el espacio, es a veces ligeramente modificada por las diversas formas de energía y de materia que circulan por todo el espacio organizado. Algunas líneas que indican una materia desconocida y que aparecen en el espectro de vuestro Sol se deben a las modificaciones de unos elementos bien conocidos que están flotando en todo el espacio de forma desintegrada, las víctimas atómicas de los violentos encuentros de las batallas elementales solares. El espacio está lleno de estos deshechos errantes, especialmente de sodio y de calcio.

\par
%\textsuperscript{(461.6)}
\textsuperscript{41:6.2} El calcio es de hecho el elemento principal que impregna de materia el espacio de todo Orvonton. Todo nuestro superuniverso está salpicado de piedra diminutamente pulverizada. La piedra es literalmente el material básico de construcción de los planetas y de las esferas del espacio. La nube cósmica, el gran manto espacial, está compuesto en su mayor parte de átomos modificados de calcio. El átomo de piedra es uno de los elementos más extendidos y persistentes. No sólo soporta la ionización solar ---la escisión--- sino que sobrevive en una identidad asociativa incluso después de haber sido azotado por los destructivos rayos X y destrozado por las altas temperaturas solares. El calcio posee una individualidad y una longevidad que superan a todas las formas más comunes de la materia.

\par
%\textsuperscript{(462.1)}
\textsuperscript{41:6.3} Tal como vuestros físicos lo han sospechado, estos restos mutilados de calcio solar cabalgan literalmente sobre los rayos de luz durante distancias variadas, lo que facilita enormemente su amplia diseminación por todo el espacio. El átomo de sodio, con ciertas modificaciones, también es capaz de locomoción mediante la luz y la energía. La proeza del calcio es mucho más notable puesto que la masa de este elemento es casi el doble que la del sodio. La impregnación del espacio local por el calcio se debe al hecho de que se escapa de la fotosfera solar, bajo una forma modificada, cabalgando literalmente sobre los rayos de sol que salen. De todos los elementos solares, el calcio, a pesar de su volumen relativo ---pues contiene veinte electrones giratorios--- es el que consigue escapar mejor del interior solar hacia los reinos del espacio. Esto explica por qué hay en el Sol una capa de calcio, una superficie gaseosa de piedra, que tiene casi diez mil kilómetros de espesor; y todo esto a pesar del hecho de que diecinueve elementos más ligeros, y numerosos elementos más pesados, se encuentran por debajo de ella.

\par
%\textsuperscript{(462.2)}
\textsuperscript{41:6.4} El calcio es un elemento activo y polifacético a las temperaturas solares. El átomo de piedra tiene dos ágiles electrones débilmente vinculados en los dos circuitos electrónicos exteriores, que están muy cerca el uno del otro. En la lucha atómica pierde pronto su electrón exterior, después de lo cual emprende el acto magistral de hacer malabarismos con el electrón diecinueve de acá para allá entre los circuitos diecinueve y veinte de la revolución electrónica. Al lanzar a este electrón diecinueve de acá para allá entre su propia órbita y la de su compañero perdido durante más de veinticinco mil veces por segundo, un átomo mutilado de piedra es capaz de desafiar parcialmente la gravedad y de cabalgar así con éxito sobre las corrientes emergentes de luz y de energía, los rayos de sol, hacia la libertad y la aventura. Este átomo de calcio se marcha hacia fuera mediante sacudidas alternas de propulsión hacia adelante, agarrando y soltando el rayo de sol unas veinticinco mil veces por segundo. Ésta es la razón por la cual la piedra es el componente principal de los mundos del espacio. El calcio es el más experto en escaparse de la prisión solar.

\par
%\textsuperscript{(462.3)}
\textsuperscript{41:6.5} La agilidad de este electrón acrobático del calcio se refleja en el hecho de que, cuando es lanzado por las fuerzas solares de la temperatura y de los rayos X al círculo de la órbita superior, sólo permanece en esta órbita una millonésima de segundo; pero antes de que el poder eléctrico-gravitatorio del núcleo atómico lo eche para atrás hacia su antigua órbita, es capaz de completar un millón de revoluciones alrededor del centro atómico.

\par
%\textsuperscript{(462.4)}
\textsuperscript{41:6.6} Vuestro Sol se ha separado de una enorme cantidad de su calcio, ha perdido cantidades extraordinarias durante los tiempos de sus erupciones convulsivas relacionadas con la formación del sistema solar. Una gran parte del calcio solar se encuentra ahora en la corteza exterior del Sol.

\par
%\textsuperscript{(462.5)}
\textsuperscript{41:6.7} Se debe recordar que los análisis espectrales sólo muestran las composiciones de la superficie del Sol. Por ejemplo: los espectros solares muestran muchas líneas correspondientes al hierro, pero el hierro no es el elemento principal del Sol. Este fenómeno se debe casi por completo a la temperatura actual de la superficie del Sol, que es un poco menos de 3.300 grados (C); esta temperatura es muy favorable para el registro del espectro del hierro.

\section*{7. Las fuentes de la energía solar}
\par
%\textsuperscript{(463.1)}
\textsuperscript{41:7.1} La temperatura interna de muchos soles, incluido el vuestro, es mucho más alta de lo que se cree generalmente. En el interior de un sol no existe prácticamente ningún átomo entero; todos están más o menos desintegrados por el intenso bombardeo de los rayos X, característico de estas altas temperaturas. Sin tener en cuenta los elementos materiales que puedan aparecer en las capas exteriores de un sol, aquellos que están en el interior se vuelven muy similares debido a la acción disociativa de los rayos X disruptivos. El rayo X es el gran nivelador de la existencia atómica.

\par
%\textsuperscript{(463.2)}
\textsuperscript{41:7.2} La temperatura superficial de vuestro Sol es de unos 3.300 grados (C), pero a medida que se penetra en el interior, aumenta rápidamente hasta que llega a alcanzar la cifra increíble de unos 19.400.000 grados (C) en las regiones centrales. (Todas estas temperaturas están expresadas en grados Celsius).

\par
%\textsuperscript{(463.3)}
\textsuperscript{41:7.3} Todos estos fenómenos indican un enorme gasto de energía, y las fuentes de la energía solar, citadas por orden de importancia, son:

\par
%\textsuperscript{(463.4)}
\textsuperscript{41:7.4} 1. La aniquilación de los átomos y, finalmente, de los electrones.

\par
%\textsuperscript{(463.5)}
\textsuperscript{41:7.5} 2. La transmutación de los elementos, incluido el grupo radioactivo de energías así liberadas.

\par
%\textsuperscript{(463.6)}
\textsuperscript{41:7.6} 3. La acumulación y la transmisión de ciertas energías espaciales universales.

\par
%\textsuperscript{(463.7)}
\textsuperscript{41:7.7} 4. La materia espacial y los meteoros que caen sin cesar en los soles resplandecientes.

\par
%\textsuperscript{(463.8)}
\textsuperscript{41:7.8} 5. La contracción solar; el enfriamiento y la contracción consiguiente de un sol producen una energía y un calor a veces mayores que los proporcionados por la materia espacial.

\par
%\textsuperscript{(463.9)}
\textsuperscript{41:7.9} 6. La acción de la gravedad a altas temperaturas transforma cierto poder, situado en circuito, en energías radiantes.

\par
%\textsuperscript{(463.10)}
\textsuperscript{41:7.10} 7. La luz y otras materias recaptadas que son atraídas de nuevo hacia el Sol después de haberlo abandonado, junto con otras energías que tienen un origen extrasolar.

\par
%\textsuperscript{(463.11)}
\textsuperscript{41:7.11} Existe un manto regulador de gases calientes (que a veces tiene millones de grados de temperatura) que envuelve a los soles y que actúa para estabilizar la pérdida de calor y para impedir de otras maneras las fluctuaciones peligrosas de la disipación del calor. Durante la vida activa de un sol, la temperatura interna de 19.400.000 grados (C) permanece casi sin cambios, independientemente por completo de la caída progresiva de la temperatura externa.

\par
%\textsuperscript{(463.12)}
\textsuperscript{41:7.12} Podríais intentar visualizar que 19.400.000 grados (C) de calor, en asociación con ciertas presiones gravitatorias, representan el punto de ebullición electrónica. Bajo esta presión y a esta temperatura, todos los átomos se degradan y se desintegran en sus componentes electrónicos y en otros componentes ancestrales; incluso los electrones y otras asociaciones de ultimatones pueden desintegrarse, pero los soles no son capaces de degradar a los ultimatones.

\par
%\textsuperscript{(463.13)}
\textsuperscript{41:7.13} Estas temperaturas solares actúan para acelerar enormemente los ultimatones y los electrones, al menos aquellos de estos últimos que continúan existiendo en estas condiciones. Os daréis cuenta de lo que significa una alta temperatura pasando por la aceleración de las actividades ultimatónicas y electrónicas si os detenéis a considerar que una gota de agua común contiene más de mil trillones de átomos. Es la energía de más de cien caballos de vapor ejercida de manera continua durante dos años. El calor total que el Sol del sistema solar emite ahora cada segundo es suficiente para hacer hervir toda el agua de todos los océanos de Urantia en un solo segundo de tiempo.

\par
%\textsuperscript{(464.1)}
\textsuperscript{41:7.14} Sólo los soles que funcionan en los canales directos de las corrientes principales de energía universal pueden brillar para siempre. Estos hornos solares arden indefinidamente, pues son capaces de reponer sus pérdidas materiales absorbiendo la fuerza espacial y las energías análogas circulantes. Pero las estrellas muy alejadas de estos canales principales de recarga están destinadas a sufrir el agotamiento de su energía ---a enfriarse gradualmente y al final apagarse.

\par
%\textsuperscript{(464.2)}
\textsuperscript{41:7.15} Estos soles muertos o moribundos pueden rejuvenecer mediante el impacto de una colisión, o pueden recargarse gracias a ciertas islas energéticas no luminosas del espacio, o robando por medio de la gravedad los soles o los sistemas cercanos más pequeños. La mayoría de los soles muertos serán revivificados por estos medios u otras técnicas evolutivas. Aquellos que con el tiempo no se recarguen así están destinados a deteriorarse por la explosión de su masa cuando la condensación gravitatoria alcance el nivel crítico de la condensación ultimatónica causada por la presión de la energía. Estos soles que desaparecen se convierten así en una de las formas más raras de energía, admirablemente adaptada para energizar otros soles situados más favorablemente.

\section*{8. Las reacciones de la energía solar}
\par
%\textsuperscript{(464.3)}
\textsuperscript{41:8.1} En aquellos soles que están integrados en los canales de la energía espacial, la energía solar se libera mediante diversas y complejas cadenas de reacción nuclear, y la más común de ellas es la reacción hidrógeno-carbono-helio. En esta metamorfosis, el carbono actúa como un catalizador de la energía, puesto que no sufre ningún tipo de cambio efectivo durante este proceso de convertirse el hidrógeno en helio. En ciertas condiciones de altas temperaturas, el hidrógeno penetra en los núcleos del carbono. Puesto que el carbono no puede contener más de cuatro de estos protones, cuando alcanza este estado de saturación empieza a emitir protones tan rápidamente como llegan los nuevos. En esta reacción, las partículas entrantes de hidrógeno salen como átomos de helio.

\par
%\textsuperscript{(464.4)}
\textsuperscript{41:8.2} La reducción del contenido de hidrógeno aumenta la luminosidad de un sol. En los soles destinados a apagarse, la máxima luminosidad se alcanza en el punto en que se agota el hidrógeno. Después de ese momento, el brillo se mantiene debido al proceso resultante de la contracción gravitatoria. Esta estrella se volverá con el tiempo lo que se llama una enana blanca, una esfera extremadamente condensada.

\par
%\textsuperscript{(464.5)}
\textsuperscript{41:8.3} En los soles grandes ---en las pequeñas nebulosas circulares---, cuando el hidrógeno está agotado y la contracción gravitatoria tiene lugar a continuación, si dicho cuerpo no es lo suficientemente opaco como para retener la presión interna que apoya las regiones gaseosas exteriores, entonces se produce un colapso repentino. Los cambios eléctrico-gravitatorios dan origen a inmensas cantidades de minúsculas partículas desprovistas de potencial eléctrico, y estas partículas se escapan rápidamente del interior solar, ocasionando así en pocos días el desmoronamiento de un sol gigantesco. Una emigración de estas <<partículas fugitivas>> fue la que provocó el desplome de la nova gigante de la nebulosa de Andrómeda hace unos cincuenta años. Este inmenso cuerpo estelar colapsó en cuarenta minutos del tiempo de Urantia.

\par
%\textsuperscript{(464.6)}
\textsuperscript{41:8.4} Por regla general, la enorme expulsión de materia continúa existiendo alrededor del sol residual que se enfría bajo la forma de extensas nubes de gases nebulares. Todo esto explica el origen de muchos tipos de nebulosas irregulares tales como la nebulosa del Cangrejo, que tuvo su origen hace unos novecientos años, y que todavía muestra a su esfera madre como una estrella solitaria cerca del centro de esta masa nebular irregular.

\section*{9. La estabilidad de los soles}
\par
%\textsuperscript{(465.1)}
\textsuperscript{41:9.1} Los soles más grandes mantienen tal control gravitatorio sobre sus electrones que la luz sólo se escapa con la ayuda de los poderosos rayos X. Estos rayos ayudantes penetran todo el espacio y están involucrados en el mantenimiento de las asociaciones ultimatónicas básicas de la energía. En los primeros tiempos de un sol, las grandes pérdidas de energía que se producen después de haber alcanzado su máxima temperatura ---más de 19.400.000 grados (C)--- no se deben tanto al escape de la luz como a las pérdidas de ultimatones. Durante las épocas adolescentes de los soles, estas energías ultimatónicas se escapan hacia el espacio como una verdadera explosión de energía, para emprender la aventura de la asociación electrónica y de la materialización de la energía.

\par
%\textsuperscript{(465.2)}
\textsuperscript{41:9.2} Los átomos y los electrones están sometidos a la gravedad. Los ultimatones \textit{no} están sometidos a la gravedad local, a la interacción de la atracción material, pero obedecen plenamente a la gravedad absoluta o gravedad del Paraíso, a la dirección, al recorrido del círculo universal y eterno del universo de universos. La energía ultimatónica no obedece a la atracción gravitatoria lineal o directa de las masas materiales cercanas o lejanas, pero siempre gira fielmente en el circuito de la gran elipse de la extensa creación.

\par
%\textsuperscript{(465.3)}
\textsuperscript{41:9.3} Vuestro propio centro solar irradia anualmente casi cien mil millones de toneladas de materia real, mientras que los soles gigantescos pierden su materia a un ritmo prodigioso durante su crecimiento inicial, durante sus primeros mil millones de años. La vida de un sol se estabiliza después de que alcanza el máximo de su temperatura interna y las energías subatómicas empiezan a ser liberadas. En este punto crítico es precisamente cuando los soles más grandes sufren pulsaciones convulsivas.

\par
%\textsuperscript{(465.4)}
\textsuperscript{41:9.4} La estabilidad de los soles depende enteramente del equilibrio de la contienda entre la gravedad y el calor ---unas presiones enormes contrapesadas por unas temperaturas inimaginables. La elasticidad del gas interior de los soles sostiene las capas de materiales diversos que los recubren, y cuando la gravedad y el calor están en equilibrio, el peso de los materiales exteriores es igual exactamente a la presión de la temperatura de los gases interiores subyacentes. En muchas estrellas de las más jóvenes, la continua condensación gravitatoria produce unas temperaturas internas en constante aumento, y a medida que crece el calor interno, la presión interior de los rayos X procedente de los vientos supergaseosos se vuelve tan fuerte que, en combinación con el movimiento centrífugo, un sol empieza a arrojar sus capas exteriores al espacio, restableciendo así el desequilibrio entre la gravedad y el calor.

\par
%\textsuperscript{(465.5)}
\textsuperscript{41:9.5} Hace mucho tiempo que vuestro propio Sol alcanzó un equilibrio relativo entre sus ciclos de expansión y de contracción, esas perturbaciones que producen las gigantescas pulsaciones de muchas estrellas más jóvenes. Vuestro Sol ha cumplido ahora sus seis mil millones de años. En el momento actual está funcionando en su período de mayor economía. Continuará brillando con la eficacia actual durante más de veinticinco mil millones de años. Es probable que experimente un período de decadencia, parcialmente eficaz, tan largo como los períodos combinados de su juventud y de su funcionamiento estabilizado.

\section*{10. El origen de los mundos habitados}
\par
%\textsuperscript{(465.6)}
\textsuperscript{41:10.1} Algunas estrellas variables que se encuentran en el estado de máxima pulsación, o se acercan a él, están dando origen a sistemas subsidiarios, muchos de los cuales terminarán por parecerse mucho a vuestro propio Sol y sus planetas rotatorios. Vuestro Sol se encontraba precisamente en este estado de poderosa pulsación cuando el masivo sistema de Angona se acercó considerablemente, y la superficie exterior del Sol empezó a arrojar verdaderas corrientes ---capas continuas--- de materia. Esto continuó con una violencia creciente hasta que se produjo la yuxtaposición más cercana, momento en que se alcanzaron los límites de la cohesión solar, y un inmenso pináculo de materia, el predecesor del sistema solar, fue expulsado. En circunstancias similares, la máxima aproximación del cuerpo atrayente extrae a veces planetas enteros e incluso una cuarta parte o un tercio de un sol. Estas expulsiones mayores forman ciertos tipos peculiares de mundos rodeados de nubes, de esferas muy parecidas a Júpiter y a Saturno.

\par
%\textsuperscript{(466.1)}
\textsuperscript{41:10.2} Sin embargo, la mayoría de los sistemas solares ha tenido un origen totalmente diferente al vuestro, y esto se aplica incluso a aquellos que nacieron mediante la técnica de las mareas gravitatorias. Pero cualquiera que sea la técnica que pueda prevalecer en la construcción de los mundos, la gravedad siempre produce un tipo de creación similar al del sistema solar, es decir, un sol central o una isla oscura con sus planetas, satélites, subsatélites y meteoros.

\par
%\textsuperscript{(466.2)}
\textsuperscript{41:10.3} Los aspectos físicos de los mundos individuales están ampliamente determinados por su manera de originarse, su situación astronómica y su entorno físico. La edad, el tamaño, la velocidad de rotación y la velocidad a través del espacio son también factores determinantes. Tanto los mundos que provienen de las contracciones gaseosas como los que proceden de los acrecentamientos sólidos están caracterizados por montañas y, durante su vida primitiva, si no son demasiado pequeños, por el agua y el aire. Los mundos surgidos de la división de un astro en fusión y los mundos resultantes de las colisiones a veces están desprovistos de extensas cadenas montañosas.

\par
%\textsuperscript{(466.3)}
\textsuperscript{41:10.4} Durante los primeros tiempos de todos estos nuevos mundos, los terremotos son frecuentes, y todos están caracterizados por grandes perturbaciones físicas; esto es especialmente así en las esferas surgidas de las contracciones gaseosas, los mundos nacidos de los inmensos anillos nebulares que son dejados atrás después de las primeras condensaciones y contracciones de ciertos soles individuales. Los planetas que tienen un origen doble como Urantia pasan por una carrera juvenil menos violenta y tempestuosa. Incluso así, vuestro mundo experimentó una fase primitiva de poderosas agitaciones, caracterizada por erupciones volcánicas, terremotos, inundaciones y tormentas terroríficas.

\par
%\textsuperscript{(466.4)}
\textsuperscript{41:10.5} Urantia está relativamente aislada en las afueras de Satania, pues vuestro sistema solar, con una sola excepción, es el que se encuentra más lejos de Jerusem, mientras que Satania misma está cerca del sistema más exterior de Norlatiadek, y esta constelación está atravesando ahora la periferia exterior de Nebadon. Figurabais realmente entre los más pequeños de toda la creación, hasta que la donación de Miguel elevó vuestro planeta a una posición de honor y de gran interés para el universo. A veces el último es el primero\footnote{\textit{El último se convierte en el primero}: Mt 19:30; 20:16; Mc 9:35; 10:31; Lc 13:30.}, mientras que el más pequeño se convierte realmente en el más grande\footnote{\textit{Quién será el más grande}: Mt 18:1-4; 20:26-27; 23:11-12; Mc 10:43-44; Lc 9:46-48; 22:26.}.

\par
%\textsuperscript{(466.5)}
\textsuperscript{41:10.6} [Presentado por un Arcángel en colaboración con el Jefe de los Centros de Poder de Nebadon.]


\chapter{Documento 42. La energía ---la mente y la materia}
\par
%\textsuperscript{(467.1)}
\textsuperscript{42:0.1} EL FUNDAMENTO del universo es material, en el sentido de que la energía es la base de toda existencia, y la energía pura está controlada por el Padre Universal. La fuerza, la energía, es la única cosa que se mantiene como un monumento perpetuo que demuestra y prueba la existencia y la presencia del Absoluto Universal. Esta inmensa corriente de energía procedente de las Presencias Paradisiacas nunca ha decaído, nunca ha fallado; nunca ha habido una interrupción en el sostén infinito.

\par
%\textsuperscript{(467.2)}
\textsuperscript{42:0.2} La manipulación de la energía universal se efectúa siempre de acuerdo con la voluntad personal y los mandatos omnisapientes del Padre Universal. Este control personal del poder manifestado y de la energía circulante es modificado por los actos y las decisiones coordinadas del Hijo Eterno, así como por los objetivos unidos del Hijo y del Padre ejecutados por el Actor Conjunto. Estos seres divinos actúan de manera personal y como individuos; también ejercen su actividad a través de las personas y de los poderes de un número casi ilimitado de subordinados, expresando cada uno de ellos de forma diversa el propósito eterno y divino en el universo de universos. Pero estas modificaciones o transmutaciones funcionales y provisionales del poder divino no disminuyen de ninguna manera la verdad de la afirmación de que toda la energía-fuerza se encuentra bajo el control último de un Dios personal que reside en el centro de todas las cosas.

\section*{1. Las fuerzas y las energías del Paraíso}
\par
%\textsuperscript{(467.3)}
\textsuperscript{42:1.1} El fundamento del universo es la materia, pero la esencia de la vida es el espíritu. El Padre de los espíritus es también el predecesor de los universos; el Padre eterno del Hijo Original es también la fuente en la eternidad del arquetipo original, la Isla del Paraíso.

\par
%\textsuperscript{(467.4)}
\textsuperscript{42:1.2} Como fenómeno universal, la materia ---la energía---, pues no son más que manifestaciones diversas de la misma realidad cósmica, es inherente al Padre Universal. <<Todas las cosas radican en él>>\footnote{\textit{Todas las cosas radican en él}: Hch 17:28; Col 1:17.}. La materia puede parecer manifestar una energía inherente y mostrar unos poderes autónomos, pero las líneas de gravedad incluidas en las energías implicadas en todos estos fenómenos físicos proceden y dependen del Paraíso. El ultimatón, la primera forma mensurable de energía, tiene por núcleo al Paraíso.

\par
%\textsuperscript{(467.5)}
\textsuperscript{42:1.3} Existe una forma de energía desconocida en Urantia que es innata en la materia y que está presente en el espacio universal. Cuando se efectúe finalmente este descubrimiento, los físicos tendrán entonces la impresión de que al menos casi habrán resuelto el misterio de la materia. Así se habrán acercado un paso más al Creador; así habrán dominado una fase más de la técnica divina; pero en ningún sentido habrán encontrado a Dios, ni tampoco habrán demostrado que la existencia de la materia o el funcionamiento de las leyes naturales son algo aparte de la técnica cósmica del Paraíso y del propósito motivador del Padre Universal.

\par
%\textsuperscript{(468.1)}
\textsuperscript{42:1.4} Después de que se realicen progresos aún más grandes y descubrimientos adicionales, después de que Urantia haya avanzado inconmensurablemente en comparación con el conocimiento actual, aunque consigáis controlar las rotaciones energéticas de las unidades eléctricas de la materia hasta el punto de modificar sus manifestaciones físicas ---incluso después de todos estos posibles progresos, los científicos serán siempre incapaces de crear un solo átomo de materia, o de producir un destello de energía, o de añadir nunca a la materia aquello que llamamos vida.

\par
%\textsuperscript{(468.2)}
\textsuperscript{42:1.5} La creación de la energía y la concesión de la vida son prerrogativas del Padre Universal y de sus personalidades Creadoras asociadas. El río de energía y de vida es una efusión continua de las Deidades, es la corriente universal y unida de la fuerza paradisiaca que sale hacia todo el espacio. Esta energía divina impregna toda la creación. Los organizadores de la fuerza inician los cambios y establecen las modificaciones de la fuerza espacial que se traducen en energía; los directores del poder transmutan la energía en materia; y así nacen los mundos materiales. Los Portadores de Vida inician en la materia muerta los procesos que llamamos vida, la vida material. Los Supervisores del Poder Morontial cumplen igualmente su misión en todos los reinos de transición entre los mundos materiales y los mundos espirituales. Los Creadores espirituales superiores inauguran procesos similares en las formas divinas de la energía, y se originan las formas espirituales superiores de la vida inteligente.

\par
%\textsuperscript{(468.3)}
\textsuperscript{42:1.6} La energía procede del Paraíso y está modelada al estilo divino. La energía ---la energía pura--- comparte la naturaleza de la organización divina; está modelada a semejanza de los tres Dioses unidos en uno solo, tal como ejercen su actividad en la sede del universo de universos. Toda fuerza es puesta en circuito en el Paraíso, proviene de las Presencias Paradisiacas y regresa a ellas, y es en esencia una manifestación de la Causa sin causa ---del Padre Universal; y sin el Padre, nada de lo que existe existiría.

\par
%\textsuperscript{(468.4)}
\textsuperscript{42:1.7} La fuerza que procede de la Deidad autoexistente existe perpetuamente por sí misma. La energía-fuerza es imperecedera, indestructible; estas manifestaciones del Infinito pueden estar sometidas a transmutaciones ilimitadas, a transformaciones sin fin y a metamorfosis eternas; pero en ningún sentido ni en ningún grado, ni siquiera en el más mínimo imaginable, pueden sufrir ni sufrirán nunca la extinción. Pero aunque la energía surge del Infinito, no se manifiesta de manera infinita; el universo maestro, tal como se concibe actualmente, tiene límites exteriores.

\par
%\textsuperscript{(468.5)}
\textsuperscript{42:1.8} La energía es eterna pero no infinita; siempre reacciona a la atracción global de la Infinidad. La fuerza y la energía duran para siempre; como han salido del Paraíso, deben regresar allí, aunque necesiten una era tras otra para completar el circuito ordenado. Aquello que tiene su origen en la Deidad del Paraíso sólo puede tener como destino el Paraíso o la Deidad.

\par
%\textsuperscript{(468.6)}
\textsuperscript{42:1.9} Todo esto confirma nuestra creencia en un universo de universos circular, un poco limitado, pero extenso y ordenado. Si esto no fuera así, entonces tarde o temprano aparecería en algún punto una prueba de la disminución de la energía. Todas las leyes, las organizaciones, la administración y el testimonio de los exploradores del universo ---todo indica la existencia de un Dios infinito, pero, hasta ahora, de un universo finito, de una forma circular de existencia sin fin, casi ilimitada, pero sin embargo finita, en contraste con la infinidad.

\section*{2. Los sistemas energéticos universales no espirituales (las energías físicas)}
\par
%\textsuperscript{(469.1)}
\textsuperscript{42:2.1} Es difícil en verdad encontrar en el idioma inglés [o español] las palabras adecuadas para designar y describir los diversos niveles de la fuerza y la energía ---físicas, mentales o espirituales. Estas narraciones no pueden adaptarse plenamente a las definiciones que tenéis aceptadas para la fuerza, la energía y el poder. La pobreza del lenguaje es tal que tenemos que emplear estos términos con múltiples significados. Por ejemplo, en este documento la palabra \textit{energía} se utiliza para designar todas las fases y formas del movimiento, la acción y el potencial fenoménicos, mientras que \textit{fuerza} se aplica a las fases de la energía anteriores a la gravedad, y \textit{poder} a las fases de la energía posteriores a la gravedad.

\par
%\textsuperscript{(469.2)}
\textsuperscript{42:2.2} Sin embargo, intentaré disminuir la confusión conceptual sugiriendo la conveniencia de adoptar la clasificación siguiente para la fuerza cósmica, la energía emergente y el poder universal ---la energía física:

\par
%\textsuperscript{(469.3)}
\textsuperscript{42:2.3} 1. \textit{La potencia espacial}. Es la presencia espacial libre e indiscutible del Absoluto Incalificado. La extensión de este concepto implica el potencial universal de la fuerza espacial inherente a la totalidad funcional del Absoluto Incalificado, mientras que la connotación de este concepto implica la totalidad de la realidad cósmica ---los universos--- que emanó en la eternidad de la Isla del Paraíso, la cual no tiene ni principio ni fin, ni movimiento ni cambio.

\par
%\textsuperscript{(469.4)}
\textsuperscript{42:2.4} Los fenómenos que nacen en la parte inferior del Paraíso abarcan probablemente tres zonas donde la presencia y la actuación de la fuerza son absolutas: la zona-punto de apoyo del Absoluto Incalificado, la zona de la Isla del Paraíso misma, y la zona intermedia de ciertos agentes o funciones igualadores y compensadores no identificados. Estas tres zonas concéntricas son el centro del ciclo paradisiaco de la realidad cósmica.

\par
%\textsuperscript{(469.5)}
\textsuperscript{42:2.5} La potencia espacial es una pre-realidad; es el ámbito del Absoluto Incalificado y sólo es sensible a la atracción personal del Padre Universal, a pesar de que es aparentemente modificable por la presencia de los Organizadores Maestros Primarios de la Fuerza.

\par
%\textsuperscript{(469.6)}
\textsuperscript{42:2.6} En Uversa, la potencia espacial se denomina absoluta.

\par
%\textsuperscript{(469.7)}
\textsuperscript{42:2.7} 2. \textit{La fuerza primordial}. Representa el primer cambio fundamental en la potencia espacial y puede tratarse de una de las funciones del Absoluto Incalificado en el bajo Paraíso. Sabemos que la presencia espacial que sale del bajo Paraíso es modificada de alguna manera por aquella que entra. Pero sin tener en cuenta estas posibles relaciones, la transmutación abiertamente reconocida de la potencia espacial en fuerza primordial es la función diferenciadora primaria de la presencia-tensión de los organizadores de la fuerza vivientes del Paraíso.

\par
%\textsuperscript{(469.8)}
\textsuperscript{42:2.8} La fuerza pasiva y potencial se vuelve activa y primordial en respuesta a la resistencia ofrecida por la presencia espacial de los Organizadores Maestros de la Fuerza Existenciados Primarios. La fuerza emerge entonces del dominio exclusivo del Absoluto Incalificado hacia los reinos de la reacción múltiple ---de la reacción a ciertos movimientos primordiales iniciados por el Dios de Acción y luego a ciertos movimientos compensatorios que proceden del Absoluto Universal. La fuerza primordial parece reaccionar a la causalidad trascendental en proporción a la absolutidad.

\par
%\textsuperscript{(469.9)}
\textsuperscript{42:2.9} La fuerza primordial se denomina a veces \textit{energía pura}; en Uversa nos referimos a ella con el nombre de segregata.

\par
%\textsuperscript{(470.1)}
\textsuperscript{42:2.10} 3. \textit{Las energías emergentes}. La presencia pasiva de los organizadores primarios de la fuerza es suficiente para transformar la potencia espacial en fuerza primordial, y sobre este campo espacial activado, estos mismos organizadores de la fuerza empiezan sus operaciones iniciales y activas. La fuerza primordial está destinada a pasar por dos fases distintas de transmutación en los reinos de la manifestación de la energía antes de aparecer como poder universal. Estos dos niveles de la energía emergente son:

\par
%\textsuperscript{(470.2)}
\textsuperscript{42:2.11} a. \textit{La energía potente}. Es la energía poderosamente orientable, movida por la masa, con una tensión muy fuerte y una reacción enérgica ---los gigantescos sistemas de energía puestos en movimiento por las actividades de los organizadores primarios de la fuerza. Esta energía primaria o potente no es al principio claramente sensible a la atracción gravitatoria del Paraíso, aunque la masa de su conjunto o su orientación espacial producen probablemente una reacción ante el grupo colectivo de influencias absolutas que operan en la parte inferior del Paraíso. Cuando la energía emerge hasta el nivel de reaccionar inicialmente a la atracción gravitatoria circular y absoluta del Paraíso, los organizadores primarios de la fuerza ceden el paso a la actividad de sus asociados secundarios.

\par
%\textsuperscript{(470.3)}
\textsuperscript{42:2.12} b. \textit{La energía gravitatoria}. La energía que aparece ahora y que reacciona a la gravedad contiene el potencial del poder universal y se convierte en la antecesora activa de toda la materia universal. Esta energía gravitatoria o secundaria es el producto de la elaboración energética derivada de la presencia de la presión y de las tendencias tensionales establecidas por los Organizadores Maestros de la Fuerza Trascendentales Asociados. En respuesta al trabajo de estos manipuladores de la fuerza, la energía espacial pasa rápidamente de la fase potente a la fase gravitatoria, volviéndose así directamente sensible a la atracción circular de la gravedad (absoluta) del Paraíso, y revelando a la vez cierto potencial de sensibilidad a la atracción de la gravedad lineal inherente a las masas materiales que pronto aparecerán como resultado de las etapas electrónicas y postelectrónicas de la energía y de la materia. Tras la aparición de la reacción a la gravedad, los Organizadores Maestros de la Fuerza Asociados pueden retirarse de los ciclones energéticos del espacio, siempre que los Directores del Poder Universal sean destinados a ese campo de acción.

\par
%\textsuperscript{(470.4)}
\textsuperscript{42:2.13} Estamos totalmente inseguros en cuanto a las causas exactas de las etapas iniciales de la evolución de la fuerza, pero reconocemos la acción inteligente del Último en los dos niveles de manifestación de la energía emergente. Cuando la energía potente y la energía gravitatoria son consideradas colectivamente, en Uversa las llamamos ultimata.

\par
%\textsuperscript{(470.5)}
\textsuperscript{42:2.14} 4. \textit{El poder universal}. La fuerza espacial ha sido cambiada en energía espacial y después en energía controlada por la gravedad. La energía física ha sido así preparada hasta el punto en que puede ser dirigida hacia los canales de poder y ser puesta al servicio de los múltiples propósitos de los Creadores del universo. Los polifacéticos directores, centros y controladores de la energía física continúan este trabajo en el gran universo ---en las creaciones organizadas y habitadas. Estos Directores del Poder Universal asumen el control más o menos completo de veintiuna de las treinta fases de la energía que componen el actual sistema energético de los siete superuniversos. Este ámbito del poder-energía-materia es el reino de las actividades inteligentes del Séptuple, que desempeña sus funciones bajo el supercontrol espacio-temporal del Supremo.

\par
%\textsuperscript{(470.6)}
\textsuperscript{42:2.15} En Uversa nos referimos al ámbito del poder universal con el nombre de gravita.

\par
%\textsuperscript{(470.7)}
\textsuperscript{42:2.16} 5. \textit{La energía de Havona}. Los conceptos de esta narración se han desplazado hacia el Paraíso a medida que seguíamos la transmutación de la fuerza espacial, nivel tras nivel, hasta el nivel de funcionamiento de la energía-poder de los universos del tiempo y del espacio. Continuando hacia el Paraíso se encuentra luego una fase preexistente de la energía que es característica del universo central. Aquí, el ciclo evolutivo parece retroceder sobre sí mismo; la energía-poder parece que ahora empieza a volver atrás hacia la fuerza, pero hacia una fuerza de una naturaleza muy distinta a la de la potencia espacial y a la de la fuerza primordial. Los sistemas energéticos de Havona no son dobles; son trinos. Éste es el ámbito energético existencial del Actor Conjunto, que ejerce su actividad en nombre de la Trinidad del Paraíso.

\par
%\textsuperscript{(471.1)}
\textsuperscript{42:2.17} En Uversa, estas energías de Havona se conocen con el nombre de triata.

\par
%\textsuperscript{(471.2)}
\textsuperscript{42:2.18} 6. \textit{La energía trascendental}. Este sistema energético funciona en y desde el nivel superior del Paraíso, y sólo en relación con las personas absonitas. En Uversa se le llama tranosta.

\par
%\textsuperscript{(471.3)}
\textsuperscript{42:2.19} 7. \textit{La monota}. La energía está estrechamente emparentada con la divinidad cuando es la energía del Paraíso. Nos inclinamos a creer que la monota es la energía viviente y no espiritual del Paraíso ---una contrapartida, desde la eternidad, de la energía viviente y espiritual del Hijo Original--- de ahí el sistema energético no espiritual del Padre Universal.

\par
%\textsuperscript{(471.4)}
\textsuperscript{42:2.20} No podemos diferenciar entre la \textit{naturaleza} del espíritu paradisiaco y la de la monota paradisiaca; son aparentemente semejantes. Tienen nombres diferentes, pero difícilmente se os pueden decir muchas cosas sobre una realidad cuyas manifestaciones espirituales y no espirituales sólo se pueden distinguir por el \textit{nombre}.

\par
%\textsuperscript{(471.5)}
\textsuperscript{42:2.21} Sabemos que las criaturas finitas pueden alcanzar la experiencia de adorar al Padre Universal a través del ministerio de Dios Séptuple y de los Ajustadores del Pensamiento, pero dudamos de que una sola personalidad subabsoluta, ni siquiera los directores del poder, pueda comprender la infinidad energética de la Gran Fuente-Centro Primera. Una cosa es segura: si los directores del poder conocen la técnica de la metamorfosis de la fuerza espacial, no nos revelan el secreto a los demás. Tengo la opinión de que no comprenden plenamente la actividad de los organizadores de la fuerza.

\par
%\textsuperscript{(471.6)}
\textsuperscript{42:2.22} Estos mismos directores del poder son catalizadores de la energía, es decir, mediante su presencia hacen que la energía se segmente, se organice o se reúna en formaciones unitarias. Todo esto implica que debe haber algo inherente a la energía que la hace funcionar así en presencia de estas entidades del poder. Hace mucho tiempo que al fenómeno de la transmutación de la fuerza cósmica en poder universal los Melquisedeks de Nebadon lo han denominado una de las siete <<infinidades de la divinidad>>. Y esto es todo lo que podréis avanzar en este punto durante vuestra ascensión por el universo local.

\par
%\textsuperscript{(471.7)}
\textsuperscript{42:2.23} A pesar de nuestra incapacidad para comprender plenamente el origen, la naturaleza y las transmutaciones de la fuerza cósmica, conocemos perfectamente todas las fases del comportamiento de la energía emergente desde el momento en que responde de manera directa e inequívoca a la acción de la gravedad del Paraíso ---aproximadamente desde el momento en que los directores del poder de los superuniversos empiezan su actividad.

\section*{3. Clasificación de la materia}
\par
%\textsuperscript{(471.8)}
\textsuperscript{42:3.1} La materia es idéntica en todos los universos, salvo en el universo central. Las propiedades físicas de la materia dependen de la velocidad de revolución de sus elementos componentes, del número y del tamaño de las partículas que giran, de su distancia al cuerpo nuclear o del contenido espacial de la materia, así como de la presencia de ciertas fuerzas que aún no se han descubierto en Urantia.

\par
%\textsuperscript{(471.9)}
\textsuperscript{42:3.2} Existen diez grandes divisiones de la materia en los diversos soles, planetas y cuerpos espaciales:

\par
%\textsuperscript{(472.1)}
\textsuperscript{42:3.3} 1. La materia ultimatónica ---las unidades físicas primordiales de la existencia material, las partículas de energía que van a componer los electrones.

\par
%\textsuperscript{(472.2)}
\textsuperscript{42:3.4} 2. La materia subelectrónica ---la etapa explosiva y repulsiva de los supergases solares.

\par
%\textsuperscript{(472.3)}
\textsuperscript{42:3.5} 3. La materia electrónica ---la etapa eléctrica de la diferenciación material--- los electrones, los protones y las otras diversas unidades que entran en la constitución variada de los grupos electrónicos.

\par
%\textsuperscript{(472.4)}
\textsuperscript{42:3.6} 4. La materia subatómica ---la materia que existe en grandes cantidades en el interior de los soles calientes.

\par
%\textsuperscript{(472.5)}
\textsuperscript{42:3.7} 5. Los átomos desintegrados ---que se encuentran en los soles que se enfrían y en todo el espacio.

\par
%\textsuperscript{(472.6)}
\textsuperscript{42:3.8} 6. La materia ionizada ---los átomos individuales despojados de sus electrones exteriores (químicamente activos) debido a las actividades eléctricas, térmicas, de los rayos X y a los disolventes.

\par
%\textsuperscript{(472.7)}
\textsuperscript{42:3.9} 7. La materia atómica ---la etapa química de la organización elemental, las unidades componentes de la materia molecular o visible.

\par
%\textsuperscript{(472.8)}
\textsuperscript{42:3.10} 8. La etapa molecular de la materia ---la materia tal como existe en Urantia en un estado de materialización relativamente estable en condiciones ordinarias.

\par
%\textsuperscript{(472.9)}
\textsuperscript{42:3.11} 9. La materia radioactiva ---la tendencia y la actividad desorganizadoras de los elementos más pesados en condiciones de calor moderado y de presión gravitatoria disminuida.

\par
%\textsuperscript{(472.10)}
\textsuperscript{42:3.12} 10. La materia colapsada ---la materia relativamente estacionaria que se encuentra en el interior de los soles fríos o muertos. Esta forma de materia no está realmente estacionaria; existe aún cierta actividad ultimatónica e incluso electrónica, pero estas unidades están muy cerca las unas de las otras, y sus velocidades de rotación han disminuido enormemente.

\par
%\textsuperscript{(472.11)}
\textsuperscript{42:3.13} La clasificación arriba indicada se refiere a la organización de la materia y no a las formas con las que aparece a los seres creados. Tampoco tiene en cuenta las etapas pre-emergentes de la energía ni las materializaciones eternas en el Paraíso y en el universo central.

\section*{4. Las transmutaciones de la energía y de la materia}
\par
%\textsuperscript{(472.12)}
\textsuperscript{42:4.1} La luz, el calor, la electricidad, el magnetismo, la química, la energía y la materia son ---en su origen, su naturaleza y su destino--- una sola y misma cosa, junto con otras realidades materiales aún no descubiertas en Urantia.

\par
%\textsuperscript{(472.13)}
\textsuperscript{42:4.2} No comprendemos plenamente los cambios casi infinitos que puede sufrir la energía física. En un universo aparece como luz, en otro como luz y calor, en otro como formas de energía desconocidas en Urantia; dentro de un número incalculable de millones de años puede reaparecer como alguna forma de energía eléctrica encrespada y agitada, o de poder magnético; más tarde aún puede aparecer de nuevo en un universo posterior como alguna forma de materia variable que pasa por una serie de metamorfosis, seguida después por su desaparición física exterior en algún gran cataclismo de los reinos. Y entonces, después de eras incontables y de un vagabundeo casi sin fin por innumerables universos, esta misma energía puede resurgir otra vez y cambiar muchas veces de forma y de potencial; y estas transformaciones continúan así durante las eras sucesivas y a través de incontables reinos. La materia sigue avanzando así, sufriendo las transmutaciones del tiempo pero girando siempre fielmente en el círculo de la eternidad; aunque durante mucho tiempo no pueda regresar a su fuente, siempre es sensible a ella, y siempre sigue el camino ordenado por la Personalidad Infinita que la envió.

\par
%\textsuperscript{(473.1)}
\textsuperscript{42:4.3} Los centros del poder y sus asociados se ocupan intensamente del trabajo de transmutar el ultimatón en los circuitos y revoluciones del electrón. Estos seres únicos controlan y combinan el poder manipulando hábilmente las unidades básicas de la energía materializada, los ultimatones. Son los amos de la energía que circula en este estado primitivo. En unión con los controladores físicos, son capaces de controlar y de dirigir eficazmente la energía incluso después de que ésta ha transmutado al nivel eléctrico, a la llamada etapa electrónica. Pero su campo de acción se reduce enormemente cuando la energía electrónicamente organizada entra en los torbellinos de los sistemas atómicos. Tras esta materialización, estas energías caen bajo el dominio completo del poder de atracción de la gravedad lineal.

\par
%\textsuperscript{(473.2)}
\textsuperscript{42:4.4} La gravedad actúa positivamente en las líneas de poder y en los canales de energía de los centros del poder y de los controladores físicos, pero estos seres sólo se relacionan de manera negativa con la gravedad ---ejerciendo sus facultades antigravitatorias.

\par
%\textsuperscript{(473.3)}
\textsuperscript{42:4.5} El frío y otras influencias trabajan en todo el espacio para organizar creativamente los ultimatones en electrones. El calor es la medida de la actividad electrónica, mientras que el frío significa simplemente ausencia de calor ---reposo relativo de la energía--- el estado de la carga-fuerza universal del espacio, con tal que ni la energía emergente ni la materia organizada estén presentes para responder a la gravedad.

\par
%\textsuperscript{(473.4)}
\textsuperscript{42:4.6} La presencia y la acción de la gravedad son las que impiden la aparición del cero teórico absoluto, pues el espacio interestelar no está a la temperatura del cero absoluto. En todo el espacio organizado hay corrientes de energía, circuitos de poder y actividades ultimatónicas, así como energías electrónicas organizadoras, que responden a la gravedad. Dicho de manera práctica, el espacio no está vacío. Incluso la atmósfera de Urantia se disipa cada vez más hasta unos cinco mil kilómetros de altura, donde empieza a desvanecerse en la materia espacial media de esta sección del universo. El espacio más vacío que se conoce en Nebadon contiene unos cien ultimatones ---el equivalente de un electrón--- por cada 16,4 cm\textsuperscript{3}. Esta escasez de materia se considera como espacio prácticamente vacío.

\par
%\textsuperscript{(473.5)}
\textsuperscript{42:4.7} La temperatura ---el frío y el calor--- sólo es secundaria con respecto a la gravedad en los reinos donde evolucionan la energía y la materia. Los ultimatones obedecen humildemente a las temperaturas extremas. Las bajas temperaturas favorecen ciertas formas de construcción electrónica y de agrupación atómica, mientras que las altas temperaturas facilitan todo tipo de dispersión atómica y de desintegración material.

\par
%\textsuperscript{(473.6)}
\textsuperscript{42:4.8} Cuando están sometidas al calor y a la presión de ciertos estados solares internos, todas las asociaciones de la materia, salvo las más primitivas, pueden desintegrarse. El calor puede vencer ampliamente así la estabilidad gravitatoria. Pero ningún calor o presión solar conocidos pueden convertir a los ultimatones en energía potente.

\par
%\textsuperscript{(473.7)}
\textsuperscript{42:4.9} Los soles resplandecientes pueden transformar la materia en diversas formas de energía, pero los mundos oscuros y todo el espacio exterior pueden reducir la actividad electrónica y ultimatónica hasta el punto de convertir estas energías en la materia de los reinos. Ciertas asociaciones electrónicas de naturaleza parecida, así como muchas asociaciones fundamentales de la materia nuclear, se forman en las temperaturas extremadamente bajas del espacio abierto, y se acrecientan posteriormente al asociarse con grandes adiciones de energía en proceso de materialización.

\par
%\textsuperscript{(473.8)}
\textsuperscript{42:4.10} Durante toda esta metamorfosis interminable de la energía y de la materia, debemos contar con la influencia de la presión gravitatoria y con el comportamiento antigravitatorio de las energías ultimatónicas que se encuentran en ciertas condiciones de temperatura, de velocidad y de revolución. La temperatura, las corrientes de energía, la distancia y la presencia de los organizadores vivientes de la fuerza y de los directores del poder también tienen su importancia sobre todos los fenómenos de transmutación de la energía y de la materia.

\par
%\textsuperscript{(474.1)}
\textsuperscript{42:4.11} El aumento de la masa en la materia es igual al aumento de la energía dividido por el cuadrado de la velocidad de la luz. En un sentido dinámico, el trabajo que puede realizar la materia en reposo es igual a la energía que ha gastado para reunir sus partes desde el Paraíso, menos la resistencia de las fuerzas a vencer durante el tránsito, y la atracción ejercida por las partes de la materia unas sobre otras.

\par
%\textsuperscript{(474.2)}
\textsuperscript{42:4.12} La existencia de las formas preelectrónicas de la materia es indicada por los dos pesos atómicos del plomo. El plomo de formación original pesa un poco más que el producido por la desintegración del uranio por medio de las emanaciones de radio; y esta diferencia de peso atómico representa la pérdida real de energía en la desintegración atómica.

\par
%\textsuperscript{(474.3)}
\textsuperscript{42:4.13} La integridad relativa de la materia está asegurada por el hecho de que la energía sólo puede ser absorbida o liberada en las cantidades exactas que los científicos de Urantia han llamado cuantos. Esta acertada disposición de los reinos materiales sirve para mantener los universos en funcionamiento.

\par
%\textsuperscript{(474.4)}
\textsuperscript{42:4.14} Cuando la posición de los electrones o de otros elementos cambia, la cantidad de energía absorbida o emitida es siempre un <<cuanto>> o un múltiplo del mismo, pero las dimensiones de las estructuras materiales correspondientes determinan totalmente el comportamiento vibratorio u ondulatorio de estas unidades de energía. Estos rizos ondulatorios de energía tienen 860 veces el diámetro de los ultimatones, electrones, átomos u otras unidades que actúan así. La confusión interminable que acompaña a la observación de la mecánica ondulatoria del comportamiento del cuanto se debe a la superposición de las ondas de energía: dos crestas se pueden combinar para formar una cresta de doble altura, mientras que una cresta y un seno se pueden combinar y producirse así una anulación mutua.

\section*{5. Las manifestaciones de la energía ondulatoria}
\par
%\textsuperscript{(474.5)}
\textsuperscript{42:5.1} En el superuniverso de Orvonton hay cien octavas de energía ondulatoria. De estos cien grupos de manifestaciones energéticas, sesenta y cuatro están reconocidas de manera total o parcial en Urantia. Los rayos del Sol representan cuatro octavas en la escala superuniversal, abarcando los rayos visibles una sola octava, la número cuarenta y seis de esta serie. El grupo ultravioleta viene a continuación, mientras que los rayos X se encuentran diez octavas más arriba, seguidos por los rayos gamma del radio. Treinta y dos octavas por encima de la luz visible del Sol están los rayos energéticos del espacio exterior, mezclados con tanta frecuencia con las minúsculas partículas de materia extremadamente activadas y asociadas a ellos. Inmediatamente por debajo de la luz visible del Sol aparecen los rayos infrarrojos, y treinta octavas más abajo se encuentra el grupo que sirve para trasmitir la radiodifusión.

\par
%\textsuperscript{(474.6)}
\textsuperscript{42:5.2} Desde el punto de vista del conocimiento científico del siglo veinte en Urantia, las manifestaciones de la energía ondulatoria se pueden clasificar en los diez grupos siguientes:

\par
%\textsuperscript{(474.7)}
\textsuperscript{42:5.3} 1. \textit{Los rayos infraultimatónicos} ---las rotaciones fronterizas de los ultimatones cuando empiezan a tomar una forma definida. Es la primera etapa de la energía emergente en la que se pueden detectar y medir los fenómenos ondulatorios.

\par
%\textsuperscript{(474.8)}
\textsuperscript{42:5.4} 2. \textit{Los rayos ultimatónicos} ---el ensamblaje de la energía en las diminutas esferas de los ultimatones ocasiona vibraciones discernibles y mensurables en el contenido del espacio. Mucho antes de que los físicos descubran el ultimatón, detectarán sin duda los fenómenos de estos rayos que llueven sobre Urantia. Estos rayos cortos y poderosos representan la actividad inicial de los ultimatones cuando reducen su velocidad hasta el punto de virar hacia la organización electrónica de la materia. A medida que los ultimatones se reúnen en electrones, se produce una condensación con el consiguiente almacenamiento de energía.

\par
%\textsuperscript{(475.1)}
\textsuperscript{42:5.5} 3. \textit{Los rayos espaciales cortos}. De todas las vibraciones puramente electrónicas, éstas son las más cortas, y representan la etapa preatómica de esta forma de materia. Para producir estos rayos se necesitan unas temperaturas extraordinariamente bajas o elevadas. Estos rayos espaciales son de dos tipos: uno que acompaña el nacimiento de los átomos y el otro que indica la desorganización atómica. Emanan en mayores cantidades del plano más denso del superuniverso, el de la Vía Láctea, que es también el plano más denso de los universos exteriores.

\par
%\textsuperscript{(475.2)}
\textsuperscript{42:5.6} 4. \textit{La etapa electrónica}. Esta etapa de la energía es la base de toda materialización en los siete superuniversos. Cuando los electrones pasan desde los niveles energéticos superiores de revolución orbital a los niveles inferiores, siempre se emiten cuantos. Los cambios orbitales de los electrones conducen a la expulsión o a la absorción de partículas mensurables de energía-luz muy determinadas y uniformes, mientras que los electrones individuales siempre abandonan una partícula de energía-luz cuando sufren una colisión. Las actividades de los cuerpos positivos y de los otros elementos de la etapa electrónica también van acompañadas de manifestaciones energéticas ondulatorias.

\par
%\textsuperscript{(475.3)}
\textsuperscript{42:5.7} 5. \textit{Los rayos gamma} ---las emanaciones que caracterizan la disociación espontánea de la materia atómica. El mejor ejemplo de esta forma de actividad electrónica se encuentra en los fenómenos asociados con la desintegración del radio.

\par
%\textsuperscript{(475.4)}
\textsuperscript{42:5.8} 6. \textit{El grupo de los rayos X}. El paso siguiente en la disminución de la velocidad del electrón produce las diversas formas de los rayos X solares junto con los rayos X generados artificialmente. La carga electrónica crea un campo eléctrico; el movimiento da nacimiento a una corriente eléctrica; la corriente produce un campo magnético. Cuando un electrón se detiene repentinamente, la conmoción electromagnética resultante produce el rayo X; el rayo X es \textit{esa} perturbación. Los rayos X solares son idénticos a los que se generan de forma mecánica para explorar el interior del cuerpo humano, salvo que son ligeramente más largos.

\par
%\textsuperscript{(475.5)}
\textsuperscript{42:5.9} 7. \textit{Los rayos ultravioletas} o químicos de la luz del Sol y sus diversas producciones mecánicas.

\par
%\textsuperscript{(475.6)}
\textsuperscript{42:5.10} 8. \textit{La luz blanca} ---toda la luz visible de los soles.

\par
%\textsuperscript{(475.7)}
\textsuperscript{42:5.11} 9. \textit{Los rayos infrarrojos} ---la reducción de la velocidad de la actividad electrónica que se acerca aún más a la etapa del calor apreciable.

\par
%\textsuperscript{(475.8)}
\textsuperscript{42:5.12} 10. \textit{Las ondas hertzianas} ---las energías que se utilizan en Urantia para la radiodifusión.

\par
%\textsuperscript{(475.9)}
\textsuperscript{42:5.13} De estas diez fases de la actividad energética ondulatoria, el ojo humano sólo puede reaccionar a una octava, a la de la totalidad de la luz solar ordinaria.

\par
%\textsuperscript{(475.10)}
\textsuperscript{42:5.14} El llamado éter es simplemente un nombre colectivo que se utiliza para designar un grupo de actividades de la fuerza y de la energía que tienen lugar en el espacio. Los ultimatones, los electrones y los otros agregados masivos de energía son partículas uniformes de materia, y en su tránsito por el espacio, avanzan realmente en línea recta. La luz y todas las otras formas de manifestaciones energéticas reconocibles consisten en una sucesión de partículas energéticas determinadas que avanzan en línea recta, salvo cuando son modificadas por la gravedad y por otras fuerzas que intervienen. Estas procesiones de partículas energéticas aparecen como fenómenos ondulatorios cuando se someten a ciertas observaciones, y esto se debe a la resistencia del manto de fuerza no diferenciado de todo el espacio, al éter hipotético, y a la tensión intergravitatoria de los agregados asociados de materia. El espaciamiento de los intervalos entre las partículas de materia, junto con la velocidad inicial de los rayos de energía, establece la apariencia ondulatoria de muchas formas de energía-materia.

\par
%\textsuperscript{(476.1)}
\textsuperscript{42:5.15} La excitación del contenido del espacio produce una reacción ondulatoria al paso de las partículas de materia en rápido movimiento, al igual que el paso de un barco por el agua da inicio a unas olas de amplitud y de intervalos variables.

\par
%\textsuperscript{(476.2)}
\textsuperscript{42:5.16} El comportamiento de la fuerza primordial da origen a unos fenómenos que son análogos en muchos aspectos a vuestro supuesto éter. El espacio no está vacío; las esferas de todo el espacio giran y se sumergen en un inmenso océano de energía-fuerza desplegada; el contenido espacial de un átomo tampoco está vacío. Sin embargo, el éter no existe, y la ausencia misma de este éter hipotético permite a los planetas habitados librarse de caer en el sol y a los electrones envolventes resistirse a caer en el núcleo.

\section*{6. Los ultimatones, los electrones y los átomos}
\par
%\textsuperscript{(476.3)}
\textsuperscript{42:6.1} Aunque la carga espacial de la fuerza universal es homogénea y no está diferenciada, la organización en materia de la energía evolucionada implica la concentración de la energía en distintas masas de dimensiones determinadas y de peso establecido ---implica una reacción gravitatoria precisa.

\par
%\textsuperscript{(476.4)}
\textsuperscript{42:6.2} La gravedad local o lineal entra plenamente en funcionamiento con la aparición de la organización atómica de la materia. La materia preatómica se vuelve ligeramente sensible a la gravedad cuando es activada por los rayos X y otras energías similares, pero la gravedad lineal no ejerce ninguna atracción mensurable sobre las partículas de energía electrónica libres, independientes y no cargadas, ni sobre los ultimatones no asociados.

\par
%\textsuperscript{(476.5)}
\textsuperscript{42:6.3} Los ultimatones funcionan por atracción mutua, y sólo responden a la atracción circular de la gravedad del Paraíso. Como no responden a la gravedad lineal, se mantienen así en la corriente universal del espacio. Los ultimatones son capaces de acelerar su velocidad de rotación hasta el punto de tener un comportamiento parcialmente antigravitatorio, pero sin la intervención de los organizadores de la fuerza o de los directores del poder, no pueden alcanzar la velocidad crítica de escape que les haría perder su individualidad y les haría regresar a la etapa de la energía potente. En la naturaleza, los ultimatones sólo se libran del estado de la existencia física cuando participan en la desorganización terminal de un sol enfriado y moribundo.

\par
%\textsuperscript{(476.6)}
\textsuperscript{42:6.4} Los ultimatones, desconocidos en Urantia, reducen su velocidad por medio de muchas fases de actividad física antes de alcanzar las condiciones energéticas y rotatorias esenciales para su organización electrónica. Los ultimatones poseen tres variedades de movimientos: su resistencia mutua a la fuerza cósmica, sus rotaciones individuales con potencial antigravitatorio, y las posiciones intraelectrónicas de los cien ultimatones mutuamente interasociados.

\par
%\textsuperscript{(476.7)}
\textsuperscript{42:6.5} La atracción mutua mantiene unidos a cien ultimatones en la formación de un electrón; y nunca hay ni más ni menos que cien ultimatones en un electrón típico. La pérdida de uno o de más ultimatones destruye la identidad electrónica típica, trayendo así a la existencia a una de las diez formas modificadas del electrón.

\par
%\textsuperscript{(476.8)}
\textsuperscript{42:6.6} Los ultimatones no describen órbitas ni giran en circuitos dentro de los electrones, pero se separan o se agrupan de acuerdo con sus velocidades de rotación axiales, determinando así las dimensiones electrónicas diferenciales. Esta misma velocidad ultimatónica de rotación axial también determina las reacciones positivas o negativas de los diversos tipos de unidades electrónicas. Toda la separación y el agrupamiento de la materia electrónica, junto con la diferenciación eléctrica de los cuerpos negativos y positivos de la energía-materia, son provocados por estas diversas funciones de las interasociaciones ultimatónicas componentes.

\par
%\textsuperscript{(477.1)}
\textsuperscript{42:6.7} Cada átomo tiene un diámetro ligeramente superior a 1/4.000.000 de milímetro, mientras que un electrón pesa un poco más que la 1/2.000 parte del átomo más pequeño, el hidrógeno. El protón positivo, característico del núcleo atómico, aunque puede no ser más grande que un electrón negativo, pesa casi dos mil veces más.

\par
%\textsuperscript{(477.2)}
\textsuperscript{42:6.8} Si la masa de la materia se pudiera aumentar hasta que la masa de un electrón equivaliera a una décima parte de una onza [2,8 gramos], y si su tamaño aumentara proporcionalmente, el volumen de dicho electrón sería tan grande como el de la Tierra. Si el volumen de un protón ---mil ochocientas veces más pesado que un electrón--- se pudiera aumentar hasta tener el tamaño de la cabeza de un alfiler, entonces, en comparación, la cabeza de un alfiler alcanzaría un diámetro igual al de la órbita de la Tierra alrededor del Sol.

\section*{7. La materia atómica}
\par
%\textsuperscript{(477.3)}
\textsuperscript{42:7.1} Toda la materia se forma de manera parecida a la del sistema solar. En el centro de cada diminuto universo de energía hay una porción nuclear de existencia material relativamente estable, comparativamente estacionaria. Esta unidad central está dotada de una triple posibilidad de manifestación. Alrededor de este centro energético giran en una profusión sin fin, pero en circuitos fluctuantes, las unidades de energía ligeramente comparables a los planetas que rodean al sol de un grupo estelar semejante a vuestro propio sistema solar.

\par
%\textsuperscript{(477.4)}
\textsuperscript{42:7.2} Dentro del átomo, los electrones giran alrededor del protón central con casi el mismo espacio comparativo que tienen los planetas que giran alrededor del Sol en el espacio del sistema solar. En comparación con su tamaño real, la distancia relativa existente entre el núcleo atómico y el circuito electrónico interior es la misma que existe entre el planeta interior Mercurio y vuestro Sol.

\par
%\textsuperscript{(477.5)}
\textsuperscript{42:7.3} Las rotaciones axiales de los electrones y sus velocidades orbitales alrededor del núcleo atómico se encuentran más allá de la imaginación humana, sin mencionar las velocidades de los ultimatones componentes. Las partículas positivas del radio salen hacia el espacio a razón de dieciséis mil kilómetros por segundo, mientras que las partículas negativas alcanzan una velocidad cercana a la de la luz.

\par
%\textsuperscript{(477.6)}
\textsuperscript{42:7.4} Los universos locales se construyen según el sistema decimal. Hay exactamente cien materializaciones atómicas distinguibles de la energía espacial en un universo doble; es la máxima organización posible de la materia en Nebadon. Estas cien formas de materia consisten en una serie regular en la que entre uno y cien electrones giran alrededor de un núcleo central relativamente compacto. Esta asociación fiable y ordenada de las diversas energías es la que compone la materia.

\par
%\textsuperscript{(477.7)}
\textsuperscript{42:7.5} No todos los mundos muestran en su superficie los cien elementos reconocibles, pero éstos están presentes en alguna parte, han estado presentes, o están en proceso de evolución. Las condiciones que rodean el origen y la evolución posterior de un planeta determinan el número de estos cien tipos atómicos que será observable. Los átomos más pesados no se encuentran en la superficie de muchos mundos. Incluso en Urantia, los elementos conocidos más pesados manifiestan la tendencia de hacerse pedazos, tal como lo ilustra el comportamiento del radio.

\par
%\textsuperscript{(477.8)}
\textsuperscript{42:7.6} La estabilidad del átomo depende del número de neutrones eléctricamente inactivos que se encuentran en el cuerpo central. El comportamiento químico depende enteramente de la actividad de los electrones que giran libremente.

\par
%\textsuperscript{(478.1)}
\textsuperscript{42:7.7} En Orvonton nunca ha sido posible reunir de forma natural más de cien electrones orbitales en un solo sistema atómico. Cuando ciento un electrones se han introducido artificialmente en un campo orbital, el resultado siempre ha sido la desorganización casi instantánea del protón central y la dispersión desordenada de los electrones y de otras energías liberadas.

\par
%\textsuperscript{(478.2)}
\textsuperscript{42:7.8} Aunque los átomos pueden contener de uno a cien electrones orbitales, sólo los diez electrones exteriores de los átomos más grandes giran alrededor del núcleo central como cuerpos distintos y bien determinados, dando vueltas de manera intacta y compacta alrededor de unas órbitas precisas y definidas. Los treinta electrones más cercanos al centro son difíciles de observar o de detectar como cuerpos separados y organizados. Esta misma proporción relativa del comportamiento electrónico en relación con su proximidad al núcleo prevalece en todos los átomos, sin tener en cuenta el número de electrones que contenga. Cuanto más cerca del núcleo, menos individualidad electrónica hay. La prolongación energética ondulatoria de un electrón puede ensancharse tanto que llega a ocupar la totalidad de las órbitas atómicas más pequeñas; esto es especialmente cierto en los electrones más cercanos al núcleo atómico.

\par
%\textsuperscript{(478.3)}
\textsuperscript{42:7.9} Los treinta electrones orbitales más interiores tienen una individualidad, pero sus sistemas energéticos tienden a entremezclarse, extendiéndose de un electrón a otro y casi de una órbita a otra. Los treinta electrones siguientes componen la segunda familia, o zona energética, y su individualidad es más pronunciada; son cuerpos de materia que ejercen un control más completo sobre sus sistemas energéticos concomitantes. Los treinta electrones siguientes, la tercera zona energética, están aún más individualizados y circulan en órbitas más determinadas y mejor definidas. Los últimos diez electrones, presentes solamente en los diez elementos más pesados, poseen la dignidad de la independencia, y son capaces por tanto de escapar más o menos libremente al control del núcleo madre. Con un mínimo de variación en la temperatura y en la presión, los miembros de este cuarto grupo más exterior de electrones se escaparán de la atracción del núcleo central, tal como lo ilustran la desorganización espontánea del uranio y de los elementos emparentados.

\par
%\textsuperscript{(478.4)}
\textsuperscript{42:7.10} Los primeros veintisiete átomos, aquellos que contienen de uno a veintisiete electrones orbitales, son más fáciles de comprender que los demás. Del veintiocho en adelante nos encontramos cada vez más con la imprevisibilidad de la supuesta presencia del Absoluto Incalificado. Pero una parte de esta imprevisibilidad electrónica se debe a las diferentes velocidades de rotación axial de los ultimatones y a su tendencia inexplicable a <<apiñarse>>. Otras influencias ---físicas, eléctricas, magnéticas y gravitatorias--- también actúan para producir un comportamiento electrónico variable. Los átomos son pues similares a las personas en cuanto a su previsibilidad. Los estadísticos pueden anunciar las leyes que gobiernan a un gran número de átomos o de personas, pero éstas no sirven para un solo átomo o una sola persona.

\section*{8. La cohesión atómica}
\par
%\textsuperscript{(478.5)}
\textsuperscript{42:8.1} Aunque la gravedad es uno de los diversos factores que se ocupan de mantener unido un minúsculo sistema atómico de energía, también está presente, dentro y entre estas unidades físicas básicas, una energía poderosa y desconocida, el secreto de su constitución básica y de su comportamiento fundamental, una fuerza que aún no se ha descubierto en Urantia. Esta influencia universal impregna todo el espacio comprendido en esta minúscula organización energética.

\par
%\textsuperscript{(478.6)}
\textsuperscript{42:8.2} El espacio interelectrónico de un átomo no está vacío. En todo el átomo, este espacio interelectrónico está activado por manifestaciones ondulatorias que están perfectamente sincronizadas con la velocidad electrónica y con las rotaciones ultimatónicas. Vuestras leyes reconocidas sobre la atracción positiva y negativa no dominan totalmente esta fuerza; por lo tanto, su comportamiento es a veces imprevisible. Esta influencia innominada parece ser una reacción del Absoluto Incalificado ante la fuerza espacial.

\par
%\textsuperscript{(479.1)}
\textsuperscript{42:8.3} Los protones cargados y los neutrones no cargados del núcleo del átomo se mantienen unidos gracias al funcionamiento alternativo del mesotrón, una partícula de materia 180 veces más pesada que el electrón. Sin esta disposición, la carga eléctrica transportada por los protones desorganizaría el núcleo atómico.

\par
%\textsuperscript{(479.2)}
\textsuperscript{42:8.4} Tal como los átomos están constituidos, ni las fuerzas eléctricas ni las gravitatorias podrían mantener unido el núcleo. La integridad del núcleo se mantiene gracias al funcionamiento cohesivo recíproco del mesotrón, que es capaz de mantener unidas las partículas cargadas y no cargadas debido al poder superior de su fuerza-masa y a su función adicional de hacer que los protones y los neutrones cambien constantemente de lugar. El mesotrón hace que la carga eléctrica de las partículas nucleares sea lanzada sin cesar de un sitio para otro entre los protones y los neutrones. Durante una fracción infinitesimal de segundo, una partícula nuclear dada es un protón cargado, y a la fracción siguiente es un neutrón no cargado. Estas alternancias del estado energético son tan increíblemente rápidas que la carga eléctrica no tiene la menor oportunidad de funcionar como influencia disruptiva. El mesotrón funciona así como una partícula <<portadora de energía>> que contribuye poderosamente a la estabilidad nuclear del átomo.

\par
%\textsuperscript{(479.3)}
\textsuperscript{42:8.5} La presencia y el funcionamiento del mesotrón explican también otro enigma atómico. Cuando los átomos actúan de forma radioactiva, emiten mucha más energía de la que se podría esperar. Este exceso de radiación procede de la desintegración del mesotrón <<portador de energía>>, que se convierte así en un simple electrón. La desintegración mesotrónica también va acompañada de la emisión de ciertas pequeñas partículas no cargadas.

\par
%\textsuperscript{(479.4)}
\textsuperscript{42:8.6} El mesotrón explica ciertas propiedades cohesivas del núcleo atómico, pero no da cuenta de la cohesión entre los protones ni de la adhesión entre los neutrones. La fuerza paradójica y poderosa que asegura la integridad cohesiva atómica es una forma de energía que aún no se ha descubierto en Urantia.

\par
%\textsuperscript{(479.5)}
\textsuperscript{42:8.7} Estos mesotrones se encuentran abundantemente en los rayos espaciales que chocan constantemente con vuestro planeta.

\section*{9. La filosofía natural}
\par
%\textsuperscript{(479.6)}
\textsuperscript{42:9.1} La religión no es la única en ser dogmática; la filosofía natural tiende igualmente a dogmatizar. Cuando un famoso educador religioso razonó que el número siete era fundamental en la naturaleza porque hay siete aberturas en la cabeza humana, si hubiera conocido mejor la química habría podido defender su creencia basándose en un fenómeno verdadero del mundo físico. En todos los universos físicos del tiempo y del espacio, y a pesar de que la constitución decimal de la energía se manifieste de manera universal, existe el recordatorio siempre presente de la realidad de que la premateria tiene una organización electrónica séptuple.

\par
%\textsuperscript{(479.7)}
\textsuperscript{42:9.2} El número siete es fundamental en el universo central y en el sistema espiritual de las transmisiones inherentes del carácter, pero el número diez, el sistema decimal, es inherente a la energía, a la materia y a la creación material. Sin embargo, el mundo atómico muestra cierta caracterización periódica que se repite en grupos de siete ---una marca de nacimiento que lleva este mundo material y que indica su lejano origen espiritual.

\par
%\textsuperscript{(480.1)}
\textsuperscript{42:9.3} Cuando los elementos básicos son organizados según sus pesos atómicos, esta persistencia séptuple de su constitución creativa se manifiesta en los dominios químicos bajo la forma de una reaparición de las propiedades físicas y químicas similares a lo largo de períodos separados de siete. Cuando los elementos químicos de Urantia se ordenan en fila de esta manera, cualquier cualidad o propiedad dada tiende a repetirse cada siete elementos. Este cambio periódico de siete en siete se repite de forma decreciente y con variaciones a lo largo de toda la tabla química, observándose más acusadamente en las agrupaciones atómicas iniciales o más ligeras. Partiendo de cualquier elemento, y después de haber observado una de sus propiedades, dicha cualidad cambiará durante los seis elementos consecutivos, pero al llegar al octavo, tiende a reaparecer, es decir, que el octavo elemento químicamente activo se parece al primero, el noveno al segundo, y así sucesivamente. Este hecho del mundo físico señala sin lugar a dudas la constitución séptuple de la energía ancestral, e indica la realidad fundamental de la diversidad séptuple de las creaciones del tiempo y del espacio. El hombre también debería tomar nota de que hay siete colores en el espectro natural.

\par
%\textsuperscript{(480.2)}
\textsuperscript{42:9.4} Pero no todas las suposiciones de la filosofía natural son válidas; por ejemplo, el éter hipotético representa un intento ingenioso del hombre por unificar su ignorancia acerca de los fenómenos espaciales. La filosofía del universo no se puede basar en las observaciones de la llamada ciencia. Un científico tendería a negar la posibilidad de que una mariposa se desarrolle a partir de una oruga si no pudiera ver dicha metamorfosis.

\par
%\textsuperscript{(480.3)}
\textsuperscript{42:9.5} La estabilidad física, asociada a la elasticidad biológica, sólo está presente en la naturaleza gracias a la sabiduría casi infinita que poseen los Arquitectos Maestros de la creación. Nada inferior a una sabiduría trascendental podría diseñar nunca unas unidades de materia que son al mismo tiempo tan estables y tan eficazmente flexibles.

\section*{10. Los sistemas energéticos universales no espirituales (los sistemas de le mente material)}
\par
%\textsuperscript{(480.4)}
\textsuperscript{42:10.1} El alcance sin fin de la realidad cósmica relativa, desde la absolutidad de la monota del Paraíso hasta la absolutidad de la potencia espacial, hace pensar en ciertas evoluciones de las relaciones dentro de las realidades no espirituales de la Fuente-Centro Primera ---de esas realidades que están ocultas en la potencia espacial, que se revelan en la monota, y que se desvelan provisionalmente en los niveles cósmicos intermedios. Este ciclo eterno de la energía, puesto que está incluido en el circuito del Padre de los universos, es absoluto, y como es absoluto, no se puede extender ni como un hecho ni como un valor; sin embargo, el Padre Primordial se está haciendo realidad en este mismo momento ---como siempre--- a partir de un campo en constante expansión de significados espacio-temporales y de significados espacio-temporales trascendidos, un campo de relaciones cambiantes donde la energía-materia está siendo sometida progresivamente al supercontrol del espíritu viviente y divino por medio del esfuerzo experiencial de la mente personal y viviente.

\par
%\textsuperscript{(480.5)}
\textsuperscript{42:10.2} Las energías universales no espirituales están reasociadas en los sistemas vivientes de las mentes no Creadoras en diversos niveles, algunos de los cuales se pueden describir como sigue:

\par
%\textsuperscript{(480.6)}
\textsuperscript{42:10.3} 1. \textit{La mente anterior a los espíritus ayudantes}. Este nivel mental no es experiencial y, en los mundos habitados, está atendido por los Controladores Físicos Maestros. Es la mente maquinal, el intelecto no enseñable de las formas más primitivas de la vida material, pero la mente no enseñable funciona en muchos niveles además del de la vida planetaria primitiva.

\par
%\textsuperscript{(481.1)}
\textsuperscript{42:10.4} 2. \textit{La mente asistida por los espíritus ayudantes}. Se trata del ministerio del Espíritu Madre de un universo local, que ejerce su actividad a través de sus siete espíritus ayudantes de la mente en el nivel enseñable (no maquinal) de la mente material. En este nivel, la mente material experimenta como intelecto subhumano (animal) en los cinco primeros ayudantes, como intelecto humano
(moral) en los siete ayudantes, y como intelecto superhumano (intermedio) en los dos últimos ayudantes.

\par
%\textsuperscript{(481.2)}
\textsuperscript{42:10.5} 3. \textit{La mente morontial en evolución} ---la conciencia en expansión de las personalidades evolutivas durante la carrera ascendente en el universo local. Es el don del Espíritu Madre del universo local en unión con el Hijo Creador. Este nivel mental implica la organización del tipo morontial de vehículo vital, una síntesis de lo material y de lo espiritual que es efectuada por los Supervisores del Poder Morontial del universo local. La mente morontial funciona de manera diferencial en respuesta a los 570 niveles de la vida morontial, revelando una creciente capacidad asociativa con la mente cósmica en los niveles superiores de consecución. Es el camino evolutivo de las criaturas mortales, pero el Hijo y el Espíritu de un universo también confieren la mente de tipo no morontial a los hijos no morontiales de las creaciones locales.

\par
%\textsuperscript{(481.3)}
\textsuperscript{42:10.6} \textit{La mente cósmica}. Es la séptuple mente diversificada del tiempo y del espacio, y cada uno de los Siete Espíritus Maestros aporta su ministerio a una fase de esta mente en uno de los siete superuniversos. La mente cósmica abarca todos los niveles de la mente finita y se coordina experiencialmente con los niveles de la deidad evolutiva de la Mente Suprema, coordinándose trascendentalmente con los niveles existenciales de la mente absoluta ---con los circuitos directos del Actor Conjunto.

\par
%\textsuperscript{(481.4)}
\textsuperscript{42:10.7} En el Paraíso, la mente es absoluta; en Havona es absonita; en Orvonton es finita. La mente siempre conlleva la actividad y la presencia de un ministerio viviente además de los diversos sistemas energéticos, y esto es así en todos los niveles y en todos los tipos de mente. Pero más allá de la mente cósmica, las relaciones de la mente con la energía no espiritual se vuelven cada vez más difíciles de describir. La mente havoniana es subabsoluta pero superevolutiva; como es existencial-experiencial, está más cerca de lo absonito que cualquier otro concepto que se haya revelado. La mente paradisiaca está más allá de la comprensión humana; es existencial, no espacial y no temporal. Sin embargo, todos estos niveles mentales están eclipsados por la presencia universal del Actor Conjunto ---por la atracción de la gravedad mental del Dios de la mente que se encuentra en el Paraíso.

\section*{11. Los mecanismos del universo}
\par
%\textsuperscript{(481.5)}
\textsuperscript{42:11.1} En la valoración y el reconocimiento de la mente, se debe recordar que el universo no es ni mecánico ni mágico; es una creación de la mente y un mecanismo con leyes. En la práctica, las leyes de la naturaleza funcionan en los reinos aparentemente dobles de lo físico y de lo espiritual, pero en realidad estos reinos son uno solo. La Fuente-Centro Primera es la causa original de todas las materializaciones, y es al mismo tiempo el Padre primero y final de todos los espíritus\footnote{\textit{Padre de todos los espíritus}: Heb 12:9.}. En los universos exteriores a Havona, el Padre Paradisiaco sólo aparece personalmente como energía pura y como puro espíritu ---bajo la forma de los Ajustadores del Pensamiento y otras fragmentaciones similares.

\par
%\textsuperscript{(481.6)}
\textsuperscript{42:11.2} Los mecanismos no dominan de manera absoluta toda la creación; el universo de universos \textit{en su totalidad} está planeado por la mente, construido por la mente y administrado por la mente. Pero el mecanismo divino del universo de universos es demasiado perfecto como para que los métodos científicos de la mente finita del hombre puedan discernir siquiera una huella de la dominación de la mente infinita. Pues esta mente creadora, controladora y sostenedora no es ni una mente material ni la mente de una criatura; es una mente espiritual que ejerce su actividad en, y desde, los niveles creadores de la realidad divina.

\par
%\textsuperscript{(482.1)}
\textsuperscript{42:11.3} La capacidad para discernir y descubrir la mente en los mecanismos del universo depende enteramente de la aptitud, el alcance y la capacidad de la mente investigadora dedicada a esa tarea de observación. Las mentes espacio-temporales, organizadas con las energías del tiempo y del espacio, están sometidas a los mecanismos del tiempo y del espacio.

\par
%\textsuperscript{(482.2)}
\textsuperscript{42:11.4} El movimiento y la gravitación universal son facetas gemelas del mecanismo impersonal espacio-temporal del universo de universos. Los niveles en los que el espíritu, la mente y la materia responden a la gravedad son totalmente independientes del tiempo, pero únicamente los verdaderos niveles espirituales de la realidad son independientes del espacio (son no espaciales). Los niveles mentales superiores del universo ---los niveles de la mente-espíritu--- también pueden ser no espaciales, pero los niveles de la mente material, tales como el de la mente humana, son sensibles a las interacciones de la gravitación universal, y sólo pierden esta sensibilidad en proporción a su identificación con el espíritu. Los niveles de la realidad espiritual se reconocen por su contenido espiritual, y la espiritualidad en el tiempo y el espacio se mide inversamente a su sensibilidad a la gravedad lineal.

\par
%\textsuperscript{(482.3)}
\textsuperscript{42:11.5} La sensibilidad a la gravedad lineal es una medida cuantitativa de la energía no espiritual. Todas las masas ---energías organizadas--- están sometidas a esta atracción, salvo en la medida en que el movimiento y la mente actúan sobre ellas. La gravedad lineal es la fuerza cohesiva de corto alcance del macrocosmos, en cierto modo como las fuerzas de cohesión intraatómica son las fuerzas de corto alcance del microcosmos. La energía física materializada, organizada bajo la forma de lo que llamamos materia, no puede atravesar el espacio sin afectar a la reacción a la gravedad lineal. Aunque esta reacción a la gravedad es directamente proporcional a la masa, está tan modificada por el espacio intermedio que el resultado final sólo puede ser ligeramente aproximado cuando se expresa de manera inversa al cuadrado de la distancia. El espacio conquista finalmente la gravitación lineal a causa de la presencia dentro de él de las influencias antigravitatorias de numerosas fuerzas supermateriales que actúan para neutralizar la acción de la gravedad y todas las reacciones a ella.

\par
%\textsuperscript{(482.4)}
\textsuperscript{42:11.6} Unos mecanismos cósmicos extremadamente complejos y que parecen ampliamente automáticos tienden siempre a ocultar la presencia de la mente interna originadora o creativa a todas y cada una de las inteligencias situadas muy por debajo de los niveles universales de la naturaleza y de la capacidad del mecanismo mismo. Por eso es inevitable que los mecanismos superiores del universo parezcan desprovistos de inteligencia a las órdenes inferiores de criaturas. La única excepción posible a esta conclusión sería la implicación de una mente en el asombroso fenómeno de un \textit{universo que se mantiene aparentemente por sí solo} ---pero esto es una cuestión de filosofía más bien que de experiencia real.

\par
%\textsuperscript{(482.5)}
\textsuperscript{42:11.7} Puesto que la mente coordina el universo, la fijeza de los mecanismos no existe. El fenómeno de la evolución progresiva, asociado con el automantenimiento cósmico, es universal. La capacidad evolutiva del universo es inagotable en la infinidad de la espontaneidad. El progreso hacia una unidad armoniosa, una síntesis experiencial creciente superpuesta a una complejidad de relaciones cada vez mayor, sólo podía efectuarla una mente intencional y dominante.

\par
%\textsuperscript{(482.6)}
\textsuperscript{42:11.8} Cuanto más elevada sea la mente universal asociada a cualquier fenómeno del universo, a los tipos inferiores de mente más difícil les resultará descubrirla. Puesto que la mente del mecanismo del universo es una mente-espíritu creativa (la mente misma del Infinito), nunca puede ser descubierta ni discernida por las mentes de los niveles inferiores del universo, y mucho menos por la mente \textit{más humilde} de todas, la mente humana. Aunque la mente animal evolutiva busca a Dios de manera natural, a solas y por sí misma no conoce inherentemente a Dios.

\section*{12. Los arquetipos y las formas ---la dominación de la mente}
\par
%\textsuperscript{(483.1)}
\textsuperscript{42:12.1} La evolución de los mecanismos implica e indica la presencia y la dominación ocultas de una mente creativa. La capacidad del intelecto mortal para concebir, diseñar y crear mecanismos automáticos demuestra las cualidades superiores, creativas e intencionales de la mente del hombre como influencia dominante en el planeta. La mente siempre tiende a:

\par
%\textsuperscript{(483.2)}
\textsuperscript{42:12.2} 1. Crear mecanismos materiales.

\par
%\textsuperscript{(483.3)}
\textsuperscript{42:12.3} 2. Descubrir misterios ocultos.

\par
%\textsuperscript{(483.4)}
\textsuperscript{42:12.4} 3. Explorar situaciones lejanas.

\par
%\textsuperscript{(483.5)}
\textsuperscript{42:12.5} 4. Formular sistemas mentales.

\par
%\textsuperscript{(483.6)}
\textsuperscript{42:12.6} 5. Alcanzar metas de sabiduría.

\par
%\textsuperscript{(483.7)}
\textsuperscript{42:12.7} 6. Lograr niveles espirituales.

\par
%\textsuperscript{(483.8)}
\textsuperscript{42:12.8} 7. Conseguir los destinos divinos ---supremo, último y absoluto.

\par
%\textsuperscript{(483.9)}
\textsuperscript{42:12.9} La mente siempre es creativa. La dotación mental individual de un animal, un mortal, un ser morontial, un ascendente espiritual o un ser que ha alcanzado la finalidad, siempre es capaz de producir un cuerpo adecuado y útil para la identidad de la criatura viviente. Pero el fenómeno de la presencia de una personalidad o el arquetipo de una identidad no son, como tales, una manifestación de la energía, ya sea física, mental o espiritual. La forma de la personalidad es el aspecto \textit{arquetípico} de un ser viviente; conlleva la \textit{organización} de unas energías, y esto, más la vida y el movimiento, es el \textit{mecanismo} de la existencia de las criaturas.

\par
%\textsuperscript{(483.10)}
\textsuperscript{42:12.10} Incluso los seres espirituales tienen una forma, y estas formas
(estos arquetipos) espirituales son reales. Incluso los tipos más elevados de personalidades espirituales tienen formas ---presencias de la personalidad análogas en todos los sentidos a los cuerpos mortales de Urantia. Casi todos los seres que se encuentran en los siete superuniversos poseen una forma. Pero hay algunas excepciones a esta regla general: los Ajustadores del Pensamiento parecen no tener una forma hasta después de fusionar con el alma sobreviviente de sus asociados mortales. Los Mensajeros Solitarios, los Espíritus Inspirados Trinitarios, los Ayudantes Personales del Espíritu Infinito, los Mensajeros de Gravedad, los Registradores Trascendentales y algunos otros tampoco tienen una forma que se pueda descubrir. Pero éstas son las pocas excepciones típicas; la gran mayoría posee una auténtica forma para su personalidad, una forma que caracteriza a cada individuo, y que es reconocible y personalmente distinguible.

\par
%\textsuperscript{(483.11)}
\textsuperscript{42:12.11} La unión entre la mente cósmica y el ministerio de los espíritus ayudantes de la mente da nacimiento a un tabernáculo físico adecuado para el ser humano en evolución. La mente morontial individualiza igualmente una forma morontial para todos los supervivientes mortales. Al igual que el cuerpo mortal es personal y característico para cada ser humano, la forma morontial será también sumamente individual y adecuadamente característica de la mente creativa que la domina. Dos formas morontiales no se parecen mucho más que dos cuerpos humanos cualquiera. Los Supervisores del Poder Morontial patrocinan, y los serafines asistentes proporcionan, el material morontial sin diferenciar con el que la vida morontial puede empezar a funcionar. Y después de la vida morontial se descubrirá que las formas espirituales son igualmente diversas, personales y características de sus habitantes mentales-espirituales respectivos.

\par
%\textsuperscript{(483.12)}
\textsuperscript{42:12.12} En un mundo material pensáis que un cuerpo tiene un espíritu, pero nosotros consideramos que el espíritu tiene un cuerpo. Los ojos materiales son en verdad las ventanas del alma nacida del espíritu. El espíritu es el arquitecto, la mente es el constructor, el cuerpo es el edificio material.

\par
%\textsuperscript{(484.1)}
\textsuperscript{42:12.13} Las energías físicas, espirituales y mentales, como tales y en estado puro, no interaccionan plenamente como realidades de los universos fenoménicos. En el Paraíso, las tres energías son semejantes, en Havona están coordinadas, mientras que en los niveles universales de las actividades finitas se pueden encontrar todas las gamas de la dominación material, mental y espiritual. La energía física parece predominar en las situaciones no personales del tiempo y del espacio, pero también parece ser que cuanto más se acerca la actividad mental-espiritual a la divinidad de propósito y a la supremacía de acción, la fase espiritual se vuelve más dominante; y que en el nivel último, la mente-espíritu puede volverse casi completamente dominante\footnote{\textit{Dominación de la mente}: 1 Co 2:16; Flp 2:5.}. En el nivel absoluto, el espíritu domina con toda seguridad. Partiendo de allí hacia los reinos del tiempo y del espacio, dondequiera que esté presente una realidad espiritual divina, cada vez que actúe una verdadera mente-espíritu, siempre tiende a producirse una contrapartida material o física de esa realidad espiritual.

\par
%\textsuperscript{(484.2)}
\textsuperscript{42:12.14} El espíritu es la realidad creadora; la contrapartida física es el reflejo espacio-temporal de la realidad espiritual, la repercusión física de la acción creadora de la mente-espíritu.

\par
%\textsuperscript{(484.3)}
\textsuperscript{42:12.15} La mente domina universalmente a la materia, al igual que es sensible a su vez al supercontrol último del espíritu. Y en el hombre mortal, sólo la mente que se somete libremente a la dirección del espíritu puede esperar sobrevivir a la existencia mortal espacio-temporal como un hijo inmortal del mundo espiritual eterno del Supremo, del Último y del Absoluto: del Infinito.

\par
%\textsuperscript{(484.4)}
\textsuperscript{42:12.16} [Presentado, a petición de Gabriel, por un Mensajero Poderoso de servicio en Nebadon.]


\chapter{Documento 43. Las constelaciones}
\par
%\textsuperscript{(485.1)}
\textsuperscript{43:0.1} A URANTIA se la conoce generalmente como la 606 de Satania en Norlatiadek de Nebadon, lo que significa que es el mundo habitado seiscientos seis del sistema local de Satania, el cual está situado en la constelación de Norlatiadek, una de las cien constelaciones del universo local de Nebadon. Como las constelaciones son las divisiones primarias de un universo local, sus gobernantes enlazan los sistemas locales de mundos habitados con la administración central del universo local en Salvington y, por reflectividad, con la superadministración de los Ancianos de los Días en Uversa.

\par
%\textsuperscript{(485.2)}
\textsuperscript{43:0.2} El gobierno de vuestra constelación está situado en un grupo de
771 esferas arquitectónicas, de las cuales la más grande y la más central es Edentia, la sede de la administración de los Padres de la Constelación, los Altísimos de Norlatiadek. Edentia misma es aproximadamente cien veces más grande que vuestro mundo. Las setenta esferas principales que rodean a Edentia tienen casi diez veces el tamaño de Urantia, mientras que los diez satélites que giran alrededor de cada uno de estos setenta mundos tienen casi el mismo tamaño que Urantia. El tamaño de estas 771 esferas arquitectónicas es totalmente comparable al de las otras constelaciones.

\par
%\textsuperscript{(485.3)}
\textsuperscript{43:0.3} El cálculo del tiempo y la medición de las distancias en Edentia son los mismos que en Salvington, y al igual que las esferas de la capital del universo, los mundos sede de las constelaciones están plenamente provistos de todas las órdenes de inteligencias celestiales. En general, estas personalidades no son muy diferentes de las que se han descrito en relación con la administración del universo.

\par
%\textsuperscript{(485.4)}
\textsuperscript{43:0.4} Los serafines supervisores, la tercera orden de ángeles del universo local, están destinados al servicio de las constelaciones. Establecen sus sedes en las esferas capitales y aportan ampliamente su ministerio a los mundos educativos morontiales que las rodean. En Norlatiadek, las setenta esferas principales, junto con sus setecientos satélites menores, están habitadas por los univitatias, los ciudadanos permanentes de la constelación. Todos estos mundos arquitectónicos están íntegramente administrados por los diversos grupos de vida nativa, en su mayor parte no revelados, pero que incluyen a los eficaces espirongas y a los hermosos espornagias. Como está situada en el punto medio del régimen educativo morontial, la vida morontial de las constelaciones es, como podéis imaginar, tanto típica como ideal.

\section*{1. La sede de la constelación}
\par
%\textsuperscript{(485.5)}
\textsuperscript{43:1.1} Edentia abunda en tierras altas fascinantes, en extensas elevaciones de materia física coronadas de vida morontial y cubiertas de gloria espiritual, pero no existen escarpadas cadenas montañosas como las que aparecen en Urantia. Hay decenas de miles de lagos centelleantes y miles y miles de arroyos que los conectan entre sí, pero no hay ni grandes océanos ni ríos torrenciales. Sólo las tierras altas están desprovistas de estos arroyos en su superficie.

\par
%\textsuperscript{(486.1)}
\textsuperscript{43:1.2} El agua de Edentia y de las esferas arquitectónicas similares no es diferente al agua de los planetas evolutivos. Los sistemas hidráulicos de estas esferas son tanto superficiales como subterráneos, y la humedad circula constantemente. Se puede navegar alrededor de Edentia por estas diversas rutas acuáticas, aunque la principal vía de transporte es la atmósfera. Los seres espirituales viajan de forma natural por encima de la superficie de la esfera, mientras que los seres morontiales y materiales utilizan medios materiales y semimateriales para salvar la travesía atmosférica.

\par
%\textsuperscript{(486.2)}
\textsuperscript{43:1.3} Edentia y sus mundos asociados tienen una verdadera atmósfera, la mezcla habitual de tres gases característica de estas creaciones arquitectónicas, y que contiene los dos elementos de la atmósfera urantiana más el gas morontial adecuado para la respiración de las criaturas morontiales. Pero aunque esta atmósfera es material así como morontial, no hay ni tormentas ni huracanes; y tampoco hay veranos ni inviernos. Esta ausencia de perturbaciones atmosféricas y de variaciones estacionales permite embellecer todas las partes exteriores de estos mundos especialmente creados.

\par
%\textsuperscript{(486.3)}
\textsuperscript{43:1.4} Las tierras altas de Edentia forman unos magníficos relieves físicos, y su belleza se acrecienta con la interminable profusión de vida que abunda a todo lo largo y ancho de la esfera. Aparte de algunas estructuras más bien aisladas, estas tierras altas no contienen ninguna obra realizada por las manos de las criaturas. Los adornos materiales y morontiales están limitados a las zonas habitadas. En las elevaciones menores se encuentran los emplazamientos de las residencias especiales, que están hermosamente embellecidas con obras de arte tanto biológicas como morontiales.

\par
%\textsuperscript{(486.4)}
\textsuperscript{43:1.5} Las salas de resurrección de Edentia están situadas en la cima de la séptima cadena de tierras altas, y allí se despiertan los mortales ascendentes de la orden secundaria modificada de ascensión. Estas cámaras de reensamblaje de las criaturas se encuentran bajo la supervisión de los Melquisedeks. La primera esfera receptora de Edentia (al igual que el planeta Melquisedek cerca de Salvington) también posee salas especiales de resurrección donde se reensambla a los mortales de las órdenes modificadas de ascensión.

\par
%\textsuperscript{(486.5)}
\textsuperscript{43:1.6} Los Melquisedeks también mantienen dos colegios especiales en Edentia. Uno, la escuela de urgencia, se consagra al estudio de los problemas derivados de la rebelión de Satania. Y el otro, la escuela de la donación, se dedica a dominar los nuevos problemas resultantes del hecho de que Miguel efectuó su donación final en uno de los mundos de Norlatiadek. Este último colegio se estableció hace casi cuarenta mil años, inmediatamente después de que Miguel anunciara que Urantia había sido elegida como mundo para su donación final.

\par
%\textsuperscript{(486.6)}
\textsuperscript{43:1.7} El mar de cristal\footnote{\textit{Mar de cristal}: Ap 4:6; 15:2.}, el área receptora de Edentia, está cerca del centro administrativo y se halla rodeado por el anfiteatro de la sede central. Alrededor de esta zona se encuentran los centros gubernativos de las setenta divisiones de los asuntos de la constelación. La mitad de Edentia está dividida en setenta secciones triangulares cuyos límites convergen en los edificios de la sede de sus sectores respectivos. El resto de esta esfera es un inmenso parque natural, los jardines de Dios.

\par
%\textsuperscript{(486.7)}
\textsuperscript{43:1.8} Durante vuestras visitas periódicas a Edentia, aunque todo el planeta está abierto a vuestro examen, la mayor parte de vuestro tiempo la pasaréis en el triángulo administrativo cuyo número corresponde al de vuestro mundo residencial habitual. Siempre seréis bienvenidos como observadores en las asambleas legislativas.

\par
%\textsuperscript{(486.8)}
\textsuperscript{43:1.9} El área morontial asignada a los mortales ascendentes que residen en Edentia está situada en la zona media del triángulo número treinta y cinco, contiguo a la sede de los finalitarios, la cual está ubicada en el triángulo treinta y seis. La sede general de los univitatias ocupa una zona enorme en la región media del triángulo treinta y cuatro, inmediatamente contiguo a la reserva residencial de los ciudadanos morontiales. Por estos arreglos se puede ver que se han tomado disposiciones para alojar al menos a setenta divisiones mayores de la vida celestial, y también que cada una de estas setenta zonas triangulares está correlacionada con alguna de las setenta esferas principales de educación morontial.

\par
%\textsuperscript{(487.1)}
\textsuperscript{43:1.10} El mar de cristal\footnote{\textit{Mar de cristal}: Ap 4:6; 15:2.} de Edentia es un enorme cristal circular de unos ciento sesenta kilómetros de circunferencia por unos cincuenta kilómetros de profundidad. Este magnífico cristal sirve como campo de recepción para todos los serafines transportadores y otros seres que llegan desde puntos exteriores a la esfera; este mar de cristal facilita enormemente el aterrizaje de los serafines transportadores.

\par
%\textsuperscript{(487.2)}
\textsuperscript{43:1.11} En casi todos los mundos arquitectónicos hay un campo de cristal de este tipo; aparte de su valor decorativo, sirve para muchos fines, siendo utilizado para describir la reflectividad superuniversal a los grupos reunidos, y como factor en la técnica de transformar la energía para modificar las corrientes del espacio y para adaptar otras corrientes entrantes de energía física.

\section*{2. El gobierno de la constelación}
\par
%\textsuperscript{(487.3)}
\textsuperscript{43:2.1} Las constelaciones son las unidades autónomas de un universo local, y cada constelación está administrada de acuerdo con sus propios decretos legislativos. Cuando los tribunales de Nebadon juzgan los asuntos del universo, todas las cuestiones internas son juzgadas según las leyes imperantes en la constelación interesada. Estos decretos judiciales de Salvington, junto con los estatutos legislativos de las constelaciones, son ejecutados por los administradores de los sistemas locales.

\par
%\textsuperscript{(487.4)}
\textsuperscript{43:2.2} Las constelaciones funcionan así como unidades legislativas o elaboradoras de las leyes, mientras que los sistemas locales sirven como unidades ejecutivas o aplicadoras de las leyes. El gobierno de Salvington es la autoridad judicial y coordinadora suprema.

\par
%\textsuperscript{(487.5)}
\textsuperscript{43:2.3} Aunque la función judicial suprema depende de la administración central de un universo local, hay dos tribunales subsidiarios pero importantes en la sede de cada constelación, el consejo Melquisedek y la corte del Altísimo.

\par
%\textsuperscript{(487.6)}
\textsuperscript{43:2.4} Todos los problemas judiciales son revisados primero por el consejo de los Melquisedeks. Doce miembros de esta orden, que han adquirido cierta experiencia necesaria en los planetas evolutivos y en los mundos sede de los sistemas, están facultados para examinar las pruebas, resumir los alegatos y formular los veredictos provisionales, los cuales son transmitidos a la corte del Altísimo, el Padre reinante de la Constelación. La división humana de este último tribunal está compuesta por siete jueces, todos ellos mortales ascendentes. Cuanto más ascendéis en el universo, más seguros estaréis de ser juzgados por aquellos de vuestra misma clase.

\par
%\textsuperscript{(487.7)}
\textsuperscript{43:2.5} El cuerpo legislativo de la constelación está dividido en tres grupos. El programa legislativo de una constelación tiene su origen en la cámara baja de los ascendentes, un grupo presidido por un finalitario y compuesto de mil mortales representativos. Cada sistema nombra a diez miembros para que ocupen su escaño en esta asamblea deliberativa. En Edentia, este cuerpo no está plenamente al completo en este momento.

\par
%\textsuperscript{(487.8)}
\textsuperscript{43:2.6} La cámara media de los legisladores está compuesta por las huestes seráficas y sus asociados, otros hijos del Espíritu Madre del universo local. Este grupo asciende a cien miembros y es nombrado por las personalidades supervisoras que presiden las diversas actividades de estos seres cuando ejercen sus funciones en la constelación.

\par
%\textsuperscript{(488.1)}
\textsuperscript{43:2.7} El cuerpo asesor o superior de los legisladores de la constelación es la cámara de los pares ---la cámara de los Hijos divinos. Este cuerpo es elegido por los Padres Altísimos y consta de diez miembros. Sólo los Hijos con una experiencia especial pueden servir en esta cámara superior. Es el grupo que averigua los hechos, ahorra tiempo y sirve de manera muy eficaz a las dos divisiones inferiores de la asamblea legislativa.

\par
%\textsuperscript{(488.2)}
\textsuperscript{43:2.8} El consejo combinado de legisladores consta de tres miembros procedentes de cada una de estas ramas diferentes de la asamblea deliberativa de la constelación, y está presidido por el Altísimo reinante más reciente. Este grupo aprueba la forma definitiva de todos los decretos y autoriza su promulgación a través de los transmisores. La aprobación de esta comisión suprema convierte a los decretos legislativos en la ley del reino; sus actos son definitivos. Los dictámenes legislativos de Edentia representan la ley fundamental de toda Norlatiadek.

\section*{3. Los Altísimos de Norlatiadek}
\par
%\textsuperscript{(488.3)}
\textsuperscript{43:3.1} Los gobernantes de las constelaciones pertenecen a la orden Vorondadek de filiación del universo local. Cuando son nombrados para servir activamente en el universo como gobernantes de las constelaciones o en otras funciones, a estos Hijos se les conoce con el nombre de \textit{Altísimos}\footnote{\textit{Altísimos}: Gn 14:18-20,22; Sal 7:17; 9:2; 46:4; 78:17,35,56; 82:6; 91:1,9; 92:1,8; Is 14:14; Lm 3:35,38; Nm 24:16; Dn 3:26; 4:2,17,24-25,32; 4:32; 5:18,21; 7:18,22,25,27; Os 7:16; 11:7; Dt 32:8; Mc 5:7; Lc 8:28; Hch 7:48; 16:17; Heb 7:1; Man 1:7; 2 Sam 22:14.} puesto que personifican la sabiduría administrativa más elevada, unida a la lealtad más perspicaz e inteligente, de todas las órdenes de Hijos de Dios del Universo Local. Su integridad personal y su lealtad como grupo nunca han sido puestas en duda; en Nebadon nunca se ha producido un descontento entre los Hijos Vorondadeks.

\par
%\textsuperscript{(488.4)}
\textsuperscript{43:3.2} Gabriel nombra como Altísimos de cada una de las constelaciones de Nebadon al menos a tres Hijos Vorondadeks. El miembro que preside este trío es conocido como el \textit{Padre de la Constelación} y sus dos asociados como el \textit{Altísimo más antiguo} y el \textit{Altísimo más reciente}. El Padre de una Constelación reina durante diez mil años oficiales (unos 50.000 años de Urantia), habiendo servido previamente como asociado más reciente y como asociado más antiguo durante períodos iguales.

\par
%\textsuperscript{(488.5)}
\textsuperscript{43:3.3} El salmista sabía que Edentia estaba gobernada por tres Padres de la Constelación y, en consecuencia, habló de su morada en plural: <<Hay un río cuyas aguas alegrarán la ciudad de Dios, el lugar más sagrado de los tabernáculos de los Altísimos>>\footnote{\textit{Hay un río}: Sal 46:4.}.

\par
%\textsuperscript{(488.6)}
\textsuperscript{43:3.4} A lo largo de los siglos ha habido una gran confusión en Urantia acerca de los diversos gobernantes del universo. Muchos educadores más tardíos confundieron sus vagas e indefinidas deidades tribales con los Padres Altísimos. Más tarde aún, los hebreos fusionaron todos estos gobernantes celestiales en una Deidad compuesta. Un educador comprendió que los Altísimos no eran los Gobernantes Supremos, pues dijo: <<Aquél que habita en el lugar secreto del Altísimo vivirá a la sombra del Todopoderoso>>\footnote{\textit{Aquel que habita en lugar secreto}: Sal 91:1.}. En las crónicas de Urantia, a veces es muy difícil saber a quien se refieren exactamente con el término <<Altísimo>>. Pero Daniel comprendió plenamente estas cuestiones, pues dijo: <<El Altísimo gobierna en el reino de los hombres y se lo da a quien quiere>>\footnote{\textit{El Altísimo gobierna}: Dn 4:17,25,32; 5:21.}.

\par
%\textsuperscript{(488.7)}
\textsuperscript{43:3.5} Los Padres de las Constelaciones se ocupan muy poco de los individuos de un planeta habitado, pero están estrechamente asociados a las funciones legislativas y de elaboración de las leyes de las constelaciones, que tanto afectan a cada \textit{raza} mortal y a cada \textit{grupo} nacional de los mundos habitados.

\par
%\textsuperscript{(489.1)}
\textsuperscript{43:3.6} Aunque el régimen de la constelación se halla entre vosotros y la administración del universo, como individuos os ocuparéis generalmente poco del gobierno de la constelación. Vuestro mayor interés se centrará normalmente en el sistema local de Satania; pero Urantia está temporalmente en estrecha relación con los gobernantes de la constelación debido a ciertas condiciones sistémicas y planetarias derivadas de la rebelión de Lucifer.

\par
%\textsuperscript{(489.2)}
\textsuperscript{43:3.7} Los Altísimos de Edentia se incautaron de ciertas fases de la autoridad planetaria en los mundos rebeldes en la época de la secesión de Lucifer. Han continuado ejerciendo este poder, y hace mucho tiempo que los Ancianos de los Días confirmaron que podían asumir el control de estos mundos desobedientes. No hay duda de que continuarán ejerciendo esta jurisdicción que han asumido mientras viva Lucifer. En un sistema leal, una gran parte de esta autoridad se conferiría normalmente al Soberano del Sistema.

\par
%\textsuperscript{(489.3)}
\textsuperscript{43:3.8} Pero existe otra razón por la que Urantia llegó a estar relacionada de manera particular con los Altísimos. Cuando Miguel, el Hijo Creador, estaba efectuando su misión final de donación, el sucesor de Lucifer no poseía una plena autoridad en el sistema local, y todos los asuntos de Urantia relacionados con la donación de Miguel estuvieron supervisados directamente por los Altísimos de Norlatiadek.

\section*{4. El monte de la asamblea ---El Fiel de los Días}
\par
%\textsuperscript{(489.4)}
\textsuperscript{43:4.1} El santísimo monte de la asamblea es el lugar donde reside el Fiel de los Días, el representante de la Trinidad del Paraíso que ejerce sus funciones en Edentia.

\par
%\textsuperscript{(489.5)}
\textsuperscript{43:4.2} Este Fiel de los Días es un Hijo de la Trinidad del Paraíso y ha estado presente en Edentia como representante personal de Emmanuel desde la creación de este mundo sede. El Fiel de los Días permanece siempre a la diestra de los Padres de la Constelación para asesorarlos, pero nunca ofrece su consejo a menos que se lo pidan. Los elevados Hijos Paradisiacos no participan nunca en la dirección de los asuntos de un universo local, salvo a petición de los gobernantes en funciones de esos dominios. Pero un Fiel de los Días es para los Altísimos de una constelación lo mismo que un Unión de los Días para un Hijo Creador.

\par
%\textsuperscript{(489.6)}
\textsuperscript{43:4.3} La residencia del Fiel de los Días en Edentia es el centro, para la constelación, del sistema paradisiaco de comunicación y de información exteriores al universo. Estos Hijos de la Trinidad, con sus estados mayores de personalidades de Havona y del Paraíso, en conexión con el Unión de los Días supervisor, están en comunicación directa y constante con los miembros de su orden en todos los universos, e incluso en Havona y el Paraíso.

\par
%\textsuperscript{(489.7)}
\textsuperscript{43:4.4} El santísimo monte es exquisitamente hermoso y está maravillosamente equipado, pero la residencia misma del Hijo Paradisiaco es modesta en comparación con la morada central de los Altísimos y las setenta estructuras que la rodean, las cuales componen la unidad residencial de los Hijos Vorondadeks. Estas instalaciones son exclusivamente residenciales; están totalmente separadas de los extensos edificios que constituyen la sede administrativa donde se tratan los asuntos de la constelación.

\par
%\textsuperscript{(489.8)}
\textsuperscript{43:4.5} La residencia del Fiel de los Días en Edentia está situada al norte de estas residencias de los Altísimos y se la conoce como <<el monte de la asamblea del Paraíso>>\footnote{\textit{Monte de la asamblea del Paraíso}: Is 14:13. \textit{Monte de la asamblea}: Heb 12:22-24.}. En estas tierras altas consagradas, los mortales ascendentes se reúnen periódicamente para oír hablar a este Hijo Paradisiaco del largo y fascinante viaje de los mortales progresivos por los mil millones de mundos de perfección de Havona y hacia las maravillas indescriptibles del Paraíso. En estas reuniones especiales en el Monte de la Asamblea es donde los mortales morontiales llegan a conocer mejor a los diversos grupos de personalidades originarias del universo central.

\par
%\textsuperscript{(490.1)}
\textsuperscript{43:4.6} Cuando el traidor Lucifer, antiguo soberano de Satania, anunció sus pretensiones a una jurisdicción más extensa, trató de desplazar a todas las órdenes superiores de filiación en el plan gubernamental del universo local. Se lo propuso en su corazón, diciendo: <<Exaltaré mi trono por encima de los Hijos de Dios; me sentaré en el Monte de la Asamblea en el norte; y seré como el Altísimo>>\footnote{\textit{Exaltaré mi trono por encima de los Hijos}: Is 14:13-14.}.

\par
%\textsuperscript{(490.2)}
\textsuperscript{43:4.7} Los cien Soberanos Sistémicos asisten periódicamente a los cónclaves de Edentia que deliberan sobre el bienestar de la constelación. Después de la rebelión de Satania, los archirrebeldes de Jerusem solían venir a estos consejos de Edentia tal como lo habían hecho en ocasiones anteriores. Y no se encontró ninguna manera de detener este descaro arrogante hasta después de que Miguel se donara en Urantia y asumiera posteriormente la soberanía ilimitada en todo Nebadon. Desde aquel día, a estos instigadores del pecado nunca se les ha permitido sentarse en los consejos de los Soberanos leales de los Sistemas en Edentia.

\par
%\textsuperscript{(490.3)}
\textsuperscript{43:4.8} Los educadores de antaño conocían estas cosas, tal como lo demuestra el escrito: <<Y hubo un día en que los Hijos de Dios vinieron a presentarse ante los Altísimos, y Satán vino también y se presentó ante ellos>>\footnote{\textit{El día que los Hijos de Dios vinieron}: Job 1:6; 2:1.}. Esto es una exposición de los hechos, independientemente de su conexión con el texto en el que aparece por casualidad.

\par
%\textsuperscript{(490.4)}
\textsuperscript{43:4.9} Desde el triunfo de Cristo, toda Norlatiadek está siendo purificada de pecado y de rebeldes. Poco antes de la muerte de Miguel en la carne, Satán, el asociado caído de Lucifer, intentó asistir a un cónclave en Edentia, pero la solidificación de los sentimientos contra los archirrebeldes había alcanzado el punto en que las puertas de la simpatía estaban tan casi universalmente cerradas que los adversarios de Satania no encontraron ningún sitio donde poder estar. Cuando no hay ninguna puerta abierta para recibir al mal, no existe ninguna oportunidad para albergar el pecado. Las puertas de los corazones de toda Edentia se cerraron para Satán; fue unánimemente rechazado por los Soberanos Sistémicos reunidos, y fue en ese momento cuando el Hijo del Hombre <<vio caer a Satán como un relámpago desde el cielo>>\footnote{\textit{Vio caer a Satán}: Lc 10:18. \textit{Derrota de Satán}: Ap 12:7-10.}.

\par
%\textsuperscript{(490.5)}
\textsuperscript{43:4.10} Desde la rebelión de Lucifer se ha construido una nueva estructura cerca de la residencia del Fiel de los Días. Este edificio temporal es la sede del enlace del Altísimo, el cual ejerce su actividad en estrecho contacto con el Hijo Paradisiaco como asesor para el gobierno de la constelación en todas las cuestiones relacionadas con la política y la actitud de la orden de los Días hacia el pecado y la rebelión.

\section*{5. Los Padres de Edentia desde la rebelión de Lucifer}
\par
%\textsuperscript{(490.6)}
\textsuperscript{43:5.1} La rotación de los Altísimos en Edentia se suspendió en la época de la rebelión de Lucifer. Actualmente tenemos los mismos gobernantes que estaban de servicio en aquellos tiempos. Deducimos que no se efectuará ningún cambio en estos gobernantes hasta que no se hayan deshecho finalmente de Lucifer y sus asociados.

\par
%\textsuperscript{(490.7)}
\textsuperscript{43:5.2} Sin embargo, el gobierno actual de la constelación ha sido ampliado hasta incluir a doce Hijos de la orden Vorondadek. Estos doce miembros son los siguientes:

\par
%\textsuperscript{(490.8)}
\textsuperscript{43:5.3} 1. El Padre de la Constelación. El Altísimo gobernante actual de Norlatiadek es el número 617.318 de la serie Vorondadek de Nebadon. Ha servido en muchas constelaciones de todo nuestro universo local antes de aceptar sus responsabilidades en Edentia.

\par
%\textsuperscript{(490.9)}
\textsuperscript{43:5.4} 2. El asociado Altísimo más antiguo.

\par
%\textsuperscript{(491.1)}
\textsuperscript{43:5.5} 3. El asociado Altísimo más reciente.

\par
%\textsuperscript{(491.2)}
\textsuperscript{43:5.6} 4. El asesor Altísimo, el representante personal de Miguel desde que éste alcanzó la condición de Hijo Maestro.

\par
%\textsuperscript{(491.3)}
\textsuperscript{43:5.7} 5. El ejecutivo Altísimo, el representante personal de Gabriel estacionado en Edentia desde la rebelión de Lucifer.

\par
%\textsuperscript{(491.4)}
\textsuperscript{43:5.8} 6. El jefe Altísimo de los observadores planetarios, el director de los observadores Vorondadeks estacionados en los mundos aislados de Satania.

\par
%\textsuperscript{(491.5)}
\textsuperscript{43:5.9} 7. El árbitro Altísimo, el Hijo Vorondadek encargado de la función de ajustar todas las dificultades resultantes de la rebelión dentro de la constelación.

\par
%\textsuperscript{(491.6)}
\textsuperscript{43:5.10} 8. El administrador de emergencia Altísimo, el Hijo Vorondadek encargado de la tarea de adaptar los decretos de emergencia de la legislatura de Norlatiadek a los mundos de Satania aislados por la rebelión.

\par
%\textsuperscript{(491.7)}
\textsuperscript{43:5.11} 9. El mediador Altísimo, el Hijo Vorondadek nombrado para armonizar los ajustes especiales de la donación en Urantia con la administración rutinaria de la constelación. La presencia de ciertas actividades arcangélicas y de otros numerosos ministerios irregulares en Urantia, junto con las actividades especiales de las Brillantes Estrellas Vespertinas en Jerusem, hacen necesaria la actividad de este Hijo.

\par
%\textsuperscript{(491.8)}
\textsuperscript{43:5.12} 10. El juez-abogado Altísimo, el jefe del tribunal de emergencia dedicado a ajustar los problemas especiales de Norlatiadek derivados de la confusión resultante de la rebelión en Satania.

\par
%\textsuperscript{(491.9)}
\textsuperscript{43:5.13} 11. El enlace Altísimo, el Hijo Vorondadek vinculado a los gobernantes de Edentia, pero nombrado como consejero especial del Fiel de los Días respecto al mejor camino a seguir en la gestión de los problemas relacionados con la rebelión y la deslealtad de las criaturas.

\par
%\textsuperscript{(491.10)}
\textsuperscript{43:5.14} 12. El director Altísimo, el presidente del consejo de emergencia de Edentia. Todas las personalidades asignadas a Norlatiadek a causa de la sublevación en Satania componen el consejo de emergencia, y la autoridad que lo preside es un Hijo Vorondadek con una experiencia extraordinaria.

\par
%\textsuperscript{(491.11)}
\textsuperscript{43:5.15} Todo esto no tiene en cuenta a los numerosos Vorondadeks, enviados de las constelaciones de Nebadon, y a otros que también residen en Edentia.

\par
%\textsuperscript{(491.12)}
\textsuperscript{43:5.16} Desde la rebelión de Lucifer, los Padres de Edentia han prestado una atención especial a Urantia y a los otros mundos aislados de Satania. Hace mucho tiempo que el profeta reconoció la mano controladora de los Padres de la Constelación en los asuntos de las naciones: <<Cuando el Altísimo dividió su herencia entre las naciones, cuando separó a los hijos de Adán, estableció los límites de los pueblos>>\footnote{\textit{Los Altísimos dividieron las naciones}: Dt 32:8.}.

\par
%\textsuperscript{(491.13)}
\textsuperscript{43:5.17} Cada mundo en cuarentena o aislado tiene a un Hijo Vorondadek que actúa como observador. No participa en la administración planetaria, salvo cuando el Padre de la Constelación le ordena que intervenga en los asuntos de las naciones. Este observador Altísimo es realmente el que <<gobierna en los reinos de los hombres>>\footnote{\textit{Gobierna en los reinos de los hombres}: Dn 4:17,25,32; Dn 5:21.}. Urantia es uno de los mundos aislados de Norlatiadek, y un observador Vorondadek ha estado estacionado en el planeta desde la traición de Caligastia. Cuando Maquiventa Melquisedek ejerció su ministerio bajo una forma semimaterial en Urantia, rindió un respetuoso homenaje al observador Altísimo entonces de servicio, tal como está escrito: <<Y Melquisedek, rey de Salem, era el sacerdote del Altísimo>>\footnote{\textit{Melquisedek, rey de Salem}: Gn 14:18ff; Sal 110:4; Heb 5:6,10; Heb 6:20; Heb 7:1-3,10,17,21; Heb 7:21.}. Melquisedek reveló las relaciones de este observador Altísimo con Abraham cuando dijo: <<Y bendito sea el Altísimo, que puso a tus enemigos en tus manos>>\footnote{\textit{Bendito sea el Altísimo}: Gn 14:20; Heb 7:1.}.

\section*{6. Los jardines de Dios}
\par
%\textsuperscript{(492.1)}
\textsuperscript{43:6.1} Las capitales de los sistemas están embellecidas principalmente con construcciones materiales y minerales, mientras que la sede del universo refleja más la gloria espiritual, pero las capitales de las constelaciones son el apogeo de las actividades morontiales y de los adornos vivientes. En los mundos sede de las constelaciones se utilizan generalmente más los adornos vivientes, y este predominio de la vida ---este arte botánico--- es el que hace que estos mundos sean llamados <<los jardines de Dios>>\footnote{\textit{Los jardines de Dios}: Is 51:3; Ez 28:13; 31:8-9.}.

\par
%\textsuperscript{(492.2)}
\textsuperscript{43:6.2} Casi la mitad de Edentia está dedicada a los exquisitos jardines de los Altísimos, y estos jardines figuran entre las creaciones morontiales más encantadoras del universo local. Esto explica por qué los lugares extraordinariamente hermosos de los mundos habitados de Norlatiadek se llamen tan a menudo <<jardines del Edén>>\footnote{\textit{Los jardines del Edén}: Gn 2:8-9; 3:23; Ez 36:35; Jl 2:3.}.

\par
%\textsuperscript{(492.3)}
\textsuperscript{43:6.3} El santuario de adoración de los Altísimos está situado en un lugar central de este magnífico jardín. El salmista debió saber algo de estas cosas, puesto que escribió: <<¿Quién subirá a la colina de los Altísimos?\footnote{\textit{¿Quién ascenderá la colina?}: Sal 24:3-4.} ¿Quién permanecerá en este lugar sagrado? Aquel que tenga las manos limpias y el corazón puro, aquel que no haya abandonado su alma a la vanidad ni jurado en falso>>. Cada décimo día de descanso, los Altísimos conducen a toda Edentia a la contemplación adoradora de Dios Supremo en este santuario.

\par
%\textsuperscript{(492.4)}
\textsuperscript{43:6.4} Los mundos arquitectónicos disfrutan de diez formas de vida de tipo material. En Urantia existe la vida vegetal y animal, pero en un mundo como Edentia, las clases materiales de vida existen en diez divisiones. Si pudierais ver estas diez divisiones de la vida de Edentia, calificaríais rápidamente a las tres primeras de vegetales y a las tres últimas de animales, pero seríais totalmente incapaces de comprender la naturaleza de los cuatro grupos intermedios de formas de vida prolíficas y fascinantes.

\par
%\textsuperscript{(492.5)}
\textsuperscript{43:6.5} Incluso la vida claramente animal es muy diferente a la de los mundos evolutivos, tan diferente que es totalmente imposible describirle a la mente mortal el carácter único y la naturaleza afectuosa de estas criaturas que no hablan. Hay miles y miles de criaturas vivientes que vuestra imaginación no podría figurarse de ninguna manera. Toda la creación animal es de una clase enteramente diferente a las burdas especies animales de los planetas evolutivos. Pero toda esta vida animal es sumamente inteligente y exquisitamente útil, y todas las diversas especies son asombrosamente mansas y conmovedoramente sociables. En estos mundos arquitectónicos no hay criaturas carnívoras; no hay nada en toda Edentia que pueda causarle temor a un ser viviente.

\par
%\textsuperscript{(492.6)}
\textsuperscript{43:6.6} La vida vegetal es también muy diferente a la de Urantia, estando compuesta de variedades tanto materiales como morontiales. Los brotes materiales tienen un colorido verde característico, pero los equivalentes morontiales de la vida vegetativa tienen un matiz orquidáceo o violeta, con tintes y reflejos variables. Esta vegetación morontial es un producto puramente energético; cuando se come no deja ningún residuo.

\par
%\textsuperscript{(492.7)}
\textsuperscript{43:6.7} Como están dotados de diez divisiones de vida física, sin mencionar las variantes morontiales, estos mundos arquitectónicos ofrecen inmensas posibilidades para embellecer biológicamente el paisaje y las estructuras materiales y morontiales. Los artesanos celestiales dirigen a los espornagias nativos en este extenso trabajo de decoración botánica y de adorno biológico. Mientras que vuestros artistas deben recurrir a la pintura inerte y al mármol sin vida para describir sus conceptos, los artesanos celestiales y los univitatias utilizan con más frecuencia los materiales vivientes para representar sus ideas y para captar sus ideales.

\par
%\textsuperscript{(493.1)}
\textsuperscript{43:6.8} Si disfrutáis con las flores, los arbustos y los árboles de Urantia, entonces os regalaréis la vista con la belleza botánica y la grandiosidad floral de los jardines celestiales de Edentia\footnote{\textit{Los jardines de Edentia}: Is 64:4; 1 Co 2:9.}. Pero tratar de transmitir a la mente mortal un concepto adecuado sobre estas bellezas de los mundos celestiales se encuentra más allá de mi poder de descripción. Los ojos no han visto, en verdad, unas glorias como las que os esperan a vuestra llegada a estos mundos relacionados con la aventura de la ascensión de los mortales.

\section*{7. Los univitatias}
\par
%\textsuperscript{(493.2)}
\textsuperscript{43:7.1} Los univitatias son los ciudadanos permanentes de Edentia y de sus mundos asociados, y los setecientos setenta mundos que rodean la sede de la constelación se encuentran bajo su supervisión. Estos hijos del Hijo Creador y del Espíritu Creativo son proyectados en un plano de existencia intermedio entre lo material y lo espiritual, pero no son criaturas morontiales. Los nativos de cada una de las setenta esferas principales de Edentia poseen unas formas visibles diferentes, y a los mortales morontiales les adaptan sus formas morontiales para que se correspondan con la escala ascendente de los univitatias cada vez que cambian de residencia de una esfera de Edentia a otra a medida que pasan sucesivamente del mundo número uno al mundo número setenta.

\par
%\textsuperscript{(493.3)}
\textsuperscript{43:7.2} Espiritualmente, los univitatias son semejantes; intelectualmente, varían como varían los mortales; en su forma se parecen mucho al estado morontial de existencia, y son creados para ejercer su actividad en setenta clases diferentes de personalidades. Cada una de estas clases de univitatias muestra diez variaciones principales de actividad intelectual, y cada uno de estos tipos intelectuales distintos preside las escuelas educativas y culturales especiales de adaptación progresiva, ocupacional o práctica a la vida social en uno de los diez satélites que giran alrededor de cada uno de los mundos principales de Edentia.

\par
%\textsuperscript{(493.4)}
\textsuperscript{43:7.3} Estos setecientos mundos menores son esferas técnicas de educación práctica en el funcionamiento de todo el universo local, y están abiertas a todas las clases de seres inteligentes. Estas escuelas donde se enseñan habilidades especiales y conocimientos técnicos no están organizadas exclusivamente para los mortales ascendentes, aunque los estudiantes morontiales constituyen con mucho el grupo más numeroso de todos los que asisten a estos cursos de formación. Cuando seáis recibidos en uno de los setenta mundos principales de cultura social, os darán inmediatamente permiso para visitar cada uno de los diez satélites que lo rodean.

\par
%\textsuperscript{(493.5)}
\textsuperscript{43:7.4} En las diversas colonias de cortesía, los mortales ascendentes morontiales predominan entre los directores de la reversión, pero los univitatias representan el grupo más importante asociado al cuerpo de los artesanos celestiales de Nebadon. En todo Orvonton, ningún ser exterior a Havona, a excepción de los abandontarios de Uversa, puede igualar a los univitatias en habilidad artística, adaptabilidad social e ingenio coordinador.

\par
%\textsuperscript{(493.6)}
\textsuperscript{43:7.5} Estos ciudadanos de la constelación no son realmente miembros del cuerpo de los artesanos, pero trabajan libremente con todos los grupos, y contribuyen mucho a hacer que los mundos de las constelaciones sean las esferas principales para desarrollar las magníficas posibilidades artísticas de la cultura de transición. No ejercen su actividad más allá de los confines de los mundos sede de las constelaciones.

\section*{8. Los mundos formativos de Edentia}
\par
%\textsuperscript{(493.7)}
\textsuperscript{43:8.1} La dotación física de Edentia y de las esferas que la rodean es casi perfecta; difícilmente podrían igualar la grandiosidad espiritual de las esferas de Salvington, pero superan de lejos las glorias de los mundos formativos de Jerusem. Todas estas esferas de Edentia reciben directamente la energía de las corrientes universales del espacio, y sus enormes sistemas de poder, tanto materiales como morontiales, son expertamente supervisados y distribuidos por los centros de la constelación, asistidos por un cuerpo competente de Controladores Físicos Maestros y de Supervisores del Poder Morontial.

\par
%\textsuperscript{(494.1)}
\textsuperscript{43:8.2} El tiempo que pasáis en los setenta mundos formativos de cultura morontial transicional, asociados a la era de la ascensión de los mortales en Edentia, representa el período más tranquilo de la carrera de un mortal ascendente hasta que éste alcanza el estado de finalitario; ésta es realmente la vida típica morontial. Aunque os vuelven a poner en sintonía cada vez que pasáis de un mundo cultural principal a otro, conserváis el mismo cuerpo morontial, y la personalidad no sufre ningún período de inconciencia.

\par
%\textsuperscript{(494.2)}
\textsuperscript{43:8.3} Vuestra estancia en Edentia y en sus esferas asociadas se dedicará principalmente a dominar la ética colectiva, el secreto de las relaciones agradables y beneficiosas entre las diversas órdenes universales y superuniversales de personalidades inteligentes.

\par
%\textsuperscript{(494.3)}
\textsuperscript{43:8.4} En los mundos de las mansiones terminasteis de unificar la personalidad humana en evolución; en la capital del sistema alcanzasteis la ciudadanía de Jerusem y consentisteis en someter vuestro yo a las disciplinas de las actividades colectivas y de las empresas coordinadas; pero ahora, en los mundos formativos de la constelación, tenéis que conseguir hacer realmente sociable vuestra personalidad morontial evolutiva. Esta adquisición cultural celestial consiste en aprender a:

\par
%\textsuperscript{(494.4)}
\textsuperscript{43:8.5} 1. Vivir con felicidad y trabajar eficazmente con diez compañeros morontiales diferentes, mientras que diez grupos de estos están asociados en compañías de cien, y luego federados en cuerpos de mil.

\par
%\textsuperscript{(494.5)}
\textsuperscript{43:8.6} 2. Residir con alegría y cooperar cordialmente con diez univitatias que, aunque sean intelectualmente similares a los seres morontiales, son muy diferentes en todos los demás aspectos. Y además tenéis que ejercer vuestra actividad con este grupo de diez que está coordinado con otras diez familias, las cuales a su vez están confederadas en un cuerpo de mil univitatias.

\par
%\textsuperscript{(494.6)}
\textsuperscript{43:8.7} 3. Lograr adaptaros simultáneamente tanto a vuestros compañeros morontiales como a estos univitatias anfitriones. Adquirir la capacidad de cooperar voluntaria y eficazmente con vuestra propia orden de seres, en estrecha asociación de trabajo con un grupo de criaturas inteligentes un poco diferentes.

\par
%\textsuperscript{(494.7)}
\textsuperscript{43:8.8} 4. Mientras trabajáis socialmente así con seres similares y diferentes a vosotros, conseguir una armonía intelectual y efectuar un ajuste práctico con los dos grupos de asociados.

\par
%\textsuperscript{(494.8)}
\textsuperscript{43:8.9} 5. Mientras conseguís hacer satisfactoriamente sociable vuestra personalidad en los niveles intelectuales y prácticos, perfeccionar aún más vuestra capacidad para vivir en contacto íntimo con seres similares y con seres ligeramente diferentes, experimentando cada vez menos irritabilidad y menos resentimientos. Los directores de la reversión contribuyen mucho a hacer realidad este último logro mediante sus actividades recreativas en grupo.

\par
%\textsuperscript{(494.9)}
\textsuperscript{43:8.10} 6. Ajustar todas estas diversas técnicas de adaptación a la vida social para fomentar la coordinación progresiva de la carrera de ascensión al Paraíso; aumentar vuestra perspicacia universal mediante el mejoramiento de vuestra capacidad para captar las metas y los significados eternos, ocultos en estas actividades espacio-temporales aparentemente insignificantes.

\par
%\textsuperscript{(494.10)}
\textsuperscript{43:8.11} 7. Y finalmente, llevar a su punto culminante todos estos múltiples procedimientos de adaptación a la vida social con el acrecentamiento simultáneo de la perspicacia espiritual, tal como están relacionados con el aumento de todas las fases de la dotación personal mediante la asociación espiritual y la coordinación morontial entre los grupos. En el aspecto intelectual, social y espiritual, cuando dos criaturas morales emplean la técnica de la asociación, no simplemente duplican sus potenciales personales de consecución universal, sino que casi cuadruplican sus posibilidades de consecución y de realización.

\par
%\textsuperscript{(495.1)}
\textsuperscript{43:8.12} Hemos descrito la adaptación a la vida social en Edentia como la asociación de un mortal morontial con un grupo familiar de univitatias compuesto por diez individuos intelectualmente diferentes, acompañada de una asociación similar con diez compañeros morontiales. Pero en los siete primeros mundos principales, un solo mortal ascendente vive con diez univitatias. En el segundo grupo de siete mundos principales, dos mortales residen con cada grupo nativo de diez, y así sucesivamente hasta que en el último grupo de siete esferas principales, diez seres morontiales están domiciliados con diez univitatias. A medida que aprendéis a establecer mejores relaciones sociales con los univitatias, practicaréis esta ética mejorada en vuestras relaciones con los compañeros morontiales que progresan con vosotros.

\par
%\textsuperscript{(495.2)}
\textsuperscript{43:8.13} Como mortales ascendentes, disfrutaréis de vuestra estancia en los mundos de progreso de Edentia, pero no experimentaréis esa sensación de satisfacción personal que caracteriza vuestro contacto inicial con los asuntos universales en la sede del sistema o vuestro toque de despedida de estas realidades en los mundos finales de la capital del universo.

\section*{9. La ciudadanía en Edentia}
\par
%\textsuperscript{(495.3)}
\textsuperscript{43:9.1} Después de graduarse en el mundo número setenta, los mortales ascendentes establecen su residencia en Edentia. Los ascendentes asisten ahora por primera vez a las <<asambleas del Paraíso>>\footnote{\textit{Asambleas del Paraíso}: Sal 89:7; Sal 111:1; Heb 12:22-23.}, y escuchan la historia de su extensa carrera descrita por el Fiel de los Días, la primera de las Personalidades Supremas con origen en la Trinidad que han conocido.

\par
%\textsuperscript{(495.4)}
\textsuperscript{43:9.2} Toda esta estancia en los mundos formativos de la constelación, que culmina en la ciudadanía de Edentia, es un período de verdadera felicidad celestial para los progresores morontiales. Durante toda vuestra estancia en los mundos del sistema, estuvisteis evolucionando desde una criatura casi animal a una criatura morontial; erais más materiales que espirituales. En las esferas de Salvington evolucionaréis desde un ser morontial al estado de un verdadero espíritu; seréis más espirituales que materiales. Pero en Edentia, los ascendentes se encuentran a medio camino entre su estado anterior y su estado futuro, a medio camino en su paso desde el animal evolutivo al espíritu ascendente. Durante toda vuestra estancia en Edentia y sus mundos sois <<como los ángeles>>\footnote{\textit{Sois como los ángeles}: Mt 22:30; Mc 12:25; Lc 20:36.}; progresáis constantemente, pero conserváis todo el tiempo un estado morontial general y típico.

\par
%\textsuperscript{(495.5)}
\textsuperscript{43:9.3} Esta estancia de un mortal ascendente en la constelación es la época más uniforme y estable de toda la carrera de la progresión morontial. Esta experiencia constituye la educación de los ascendentes en la adaptación pre-espiritual a la vida social. Es análoga a la experiencia espiritual prefinalitaria en Havona y a la formación preabsonita en el Paraíso.

\par
%\textsuperscript{(495.6)}
\textsuperscript{43:9.4} En Edentia, los mortales ascendentes se ocupan principalmente de sus tareas en los setenta mundos progresivos de los univitatias. También sirven en diversas ocupaciones en Edentia misma, principalmente en conjunción con el programa de la constelación que se ocupa del bienestar colectivo, racial, nacional y planetario. Los Altísimos se dedican relativamente poco a fomentar el progreso individual en los mundos habitados; gobiernan más bien en los reinos de los hombres que en el corazón de los individuos.

\par
%\textsuperscript{(495.7)}
\textsuperscript{43:9.5} El día que estéis preparados para dejar Edentia con vistas a la carrera en Salvington, haréis una pausa y recordaréis una de las épocas más hermosas y refrescantes de todos vuestros períodos de formación a este lado del Paraíso. Pero la gloria de todo esto aumentará a medida que ascendáis hacia el interior y consigáis una capacidad creciente para apreciar más ampliamente los significados divinos y los valores espirituales.

\par
%\textsuperscript{(496.1)}
\textsuperscript{43:9.6} [Patrocinado por Malavatia Melquisedek.]


\chapter{Documento 44. Los artesanos celestiales}
\par
%\textsuperscript{(497.1)}
\textsuperscript{44:0.1} ENTRE las colonias de cortesía de los diversos mundos sede divisionarios y universales se puede encontrar una orden única de personalidades compuestas denominada los artesanos celestiales. Estos seres son los artistas y los artesanos maestros de los reinos morontiales y de los reinos espirituales inferiores. Son los espíritus y semiespíritus que se ocupan de los adornos morontiales y de los embellecimientos espirituales. Estos artesanos están distribuidos por todo el gran universo ---en los mundos sede de los superuniversos, de los universos locales, de las constelaciones y de los sistemas, así como en todas las esferas establecidas en la luz y la vida; pero su campo de actividad principal se encuentra en las constelaciones y especialmente en los setecientos setenta mundos que rodean a cada esfera sede.

\par
%\textsuperscript{(497.2)}
\textsuperscript{44:0.2} Aunque su trabajo puede ser casi incomprensible para la mente material, se puede comprender que los mundos morontiales y espirituales no están desprovistos de artes superiores ni de culturas celestiales.

\par
%\textsuperscript{(497.3)}
\textsuperscript{44:0.3} Los artesanos celestiales no son creados como tales; son un cuerpo de seres seleccionados y reclutados, compuesto de ciertas personalidades educadoras nativas del universo central y de sus alumnos voluntarios elegidos entre los mortales ascendentes y otros numerosos grupos celestiales. El cuerpo docente original de estos artesanos fue nombrado en otro tiempo por el Espíritu Infinito en colaboración con los Siete Espíritus Maestros, y estaba compuesto por siete mil instructores de Havona, mil para cada una de las siete divisiones de artesanos. Con este núcleo para empezar, este brillante cuerpo de hábiles trabajadores en los asuntos espirituales y morontiales se ha desarrollado a través de las épocas.

\par
%\textsuperscript{(497.4)}
\textsuperscript{44:0.4} Cualquier personalidad morontial o entidad espiritual, es decir, cualquier ser que tenga un rango inferior al de la filiación divina inherente, tiene derecho a ser admitido en el cuerpo de los artesanos celestiales. Después de su llegada a los mundos morontiales, los hijos ascendentes de Dios procedentes de las esferas evolutivas pueden solicitar ser admitidos en el cuerpo de los artesanos y, si están suficientemente dotados, pueden elegir esta carrera durante un período más o menos largo. Pero nadie puede alistarse con los artesanos celestiales durante menos de un milenio, de mil años del tiempo superuniversal.

\par
%\textsuperscript{(497.5)}
\textsuperscript{44:0.5} Todos los artesanos celestiales están registrados en la sede del superuniverso, pero en las capitales de los universos locales son dirigidos por los supervisores morontiales. El cuerpo central de los supervisores morontiales, que ejerce su actividad en el mundo sede de cada universo local, los pone en servicio en las siete divisiones principales de actividad siguientes:

\par
%\textsuperscript{(497.6)}
\textsuperscript{44:0.6} 1. Los Músicos Celestiales\footnote{\textit{Músicos celestiales}: Sal 66:1-2.}.

\par
%\textsuperscript{(497.7)}
\textsuperscript{44:0.7} 2. Los Reproductores Celestiales.

\par
%\textsuperscript{(497.8)}
\textsuperscript{44:0.8} 3. Los Constructores Divinos.

\par
%\textsuperscript{(497.9)}
\textsuperscript{44:0.9} 4. Los Registradores del Pensamiento.

\par
%\textsuperscript{(498.1)}
\textsuperscript{44:0.10} 5. Los Manipuladores de la Energía.

\par
%\textsuperscript{(498.2)}
\textsuperscript{44:0.11} 6. Los Diseñadores y los Embellecedores.

\par
%\textsuperscript{(498.3)}
\textsuperscript{44:0.12} 7. Los Trabajadores de la Armonía.

\par
%\textsuperscript{(498.4)}
\textsuperscript{44:0.13} Todos los instructores originales de estos siete grupos procedían de los mundos perfectos de Havona, y Havona contiene los arquetipos, los estudios arquetípicos, de todas las fases y formas del arte espiritual. Aunque intentar trasladar estas artes de Havona a los mundos del espacio es una tarea gigantesca, la técnica y la ejecución de los artesanos celestiales han mejorado de era en era. Como sucede en todas las demás fases de la carrera ascendente, a aquellos que están más avanzados en cualquier empeño se les pide constantemente que impartan su conocimiento y su habilidad superiores a sus compañeros menos favorecidos.

\par
%\textsuperscript{(498.5)}
\textsuperscript{44:0.14} Estas artes trasplantadas de Havona las empezaréis a vislumbrar por primera vez en los mundos de las mansiones, y su belleza, y vuestra apreciación de su belleza, aumentarán y se harán más brillantes hasta que lleguéis a las salas espirituales de Salvington, donde contemplaréis las obras maestras inspiradoras de los artistas celestiales de los reinos espirituales.

\par
%\textsuperscript{(498.6)}
\textsuperscript{44:0.15} Todas estas actividades de los mundos morontiales y espirituales son reales. El mundo espiritual es una realidad para los seres espirituales. Para nosotros, el mundo material es el más irreal. Las formas superiores de los espíritus atraviesan libremente la materia ordinaria. Los espíritus elevados no reaccionan a nada material, salvo a ciertas energías fundamentales. Para los seres materiales, el mundo espiritual es más o menos irreal; para los seres espirituales, el mundo material es casi enteramente irreal, es simplemente una sombra de la sustancia de las realidades espirituales.

\par
%\textsuperscript{(498.7)}
\textsuperscript{44:0.16} Con la visión exclusivamente espiritual, no puedo percibir el edificio en el que se está traduciendo y registrando esta narración. Un Consejero Divino de Uversa que se encuentra a mi lado por casualidad percibe aún menos estas creaciones puramente materiales. El aspecto que tienen para vosotros estas estructuras materiales lo discernimos contemplando una contrapartida espiritual que es presentada a nuestra mente por uno de los transformadores de la energía que nos acompañan. Este edificio material no es exactamente real para mí, que soy un ser espiritual, pero por supuesto es muy real y muy útil para los mortales materiales.

\par
%\textsuperscript{(498.8)}
\textsuperscript{44:0.17} Hay ciertos tipos de seres que son capaces de discernir la realidad de las criaturas de los mundos espirituales y de los mundos materiales. A esta clase pertenecen las llamadas cuartas criaturas de los Servitales de Havona y las cuartas criaturas de los conciliadores. Los ángeles del tiempo y del espacio están dotados de la capacidad de discernir tanto a los seres espirituales como a los seres materiales, y los mortales ascendentes también poseen este don después de ser liberados de la vida en la carne. Después de alcanzar los niveles espirituales superiores, los ascendentes son capaces de reconocer las realidades materiales, morontiales y espirituales.

\par
%\textsuperscript{(498.9)}
\textsuperscript{44:0.18} Aquí también está conmigo un Mensajero Poderoso de Uversa, un ascendente fusionado con su Ajustador, en otro tiempo un ser mortal, que os percibe tal como sois, y al mismo tiempo puede ver al Mensajero Solitario, al supernafín y a los otros seres celestiales presentes. Durante vuestra larga ascensión nunca perderéis el poder de reconocer a vuestros asociados de las existencias anteriores. A medida que ascendáis hacia el interior en la escala de la vida, siempre conservaréis la capacidad de reconocer y de fraternizar con los compañeros de vuestros niveles de experiencia anteriores e inferiores. Cada nuevo traslado o resurrección añadirá un grupo más de seres espirituales a vuestro campo visual, sin privaros en lo más mínimo de la capacidad de reconocer a vuestros amigos y compañeros de los estados anteriores.

\par
%\textsuperscript{(498.10)}
\textsuperscript{44:0.19} Todo esto es posible en la experiencia de los mortales ascendentes gracias a la acción de los Ajustadores del Pensamiento interiores. Como conservan los duplicados de todas las experiencias de vuestras vidas, podéis estar seguros de que nunca perderéis ningún auténtico atributo que hayáis poseído alguna vez; y estos Ajustadores recorren todo el camino con vosotros, como una parte de vosotros, en realidad como \textit{vosotros mismos}.

\par
%\textsuperscript{(499.1)}
\textsuperscript{44:0.20} Pero casi pierdo la esperanza de poder transmitir a la mente material la naturaleza del trabajo de los artesanos celestiales. Me veo constantemente en la necesidad de desvirtuar el pensamiento y de deformar el lenguaje en un esfuerzo por exponer a la mente humana la realidad de estas actividades morontiales y de estos fenómenos casi espirituales. Vuestra comprensión es incapaz de captar, y vuestro lenguaje es inadecuado para transmitir, el significado, el valor y las relaciones de estas actividades semiespirituales. Continúo en este esfuerzo de iluminar a la mente humana en lo referente a estas realidades, comprendiendo plenamente que me es totalmente imposible tener mucho éxito en esta tarea.

\par
%\textsuperscript{(499.2)}
\textsuperscript{44:0.21} No puedo hacer otra cosa que intentar esbozar un paralelismo rudimentario entre las actividades materiales de los mortales y las múltiples funciones de los artesanos celestiales. Si las razas de Urantia estuvieran más avanzadas en el arte y en las otras realizaciones culturales, podría ir mucho más allá en mis esfuerzos por presentar a la mente humana las cosas morontiales, partiendo de las cosas materiales. Casi todo lo que puedo esperar conseguir es recalcar el hecho de que estas actividades de los mundos morontiales y espirituales son reales.

\section*{1. Los músicos celestiales}
\par
%\textsuperscript{(499.3)}
\textsuperscript{44:1.1} Con el alcance limitado del oído humano, difícilmente podéis concebir las melodías morontiales. Existe incluso una gama material de hermosos sonidos que el sentido humano del oído no reconoce, sin mencionar la amplitud inconcebible de la armonía morontial y espiritual. Las melodías espirituales no son ondas sonoras materiales, sino pulsaciones espirituales que reciben los espíritus de las personalidades celestiales. La inmensidad del alcance y el alma de la expresión, así como la grandiosidad de la ejecución asociadas a la melodía de las esferas, sobrepasan por completo la comprensión humana. He visto a millones de seres embelesados que permanecían en un éxtasis sublime mientras la melodía del reino sonaba sobre la energía espiritual de los circuitos celestiales. Estas maravillosas melodías se pueden transmitir hasta las zonas más alejadas de un universo.

\par
%\textsuperscript{(499.4)}
\textsuperscript{44:1.2} Los músicos celestiales se ocupan de producir las armonías celestiales manipulando las fuerzas espirituales siguientes:

\par
%\textsuperscript{(499.5)}
\textsuperscript{44:1.3} 1. \textit{Los sonidos espirituales} ---las interrupciones de la corriente espiritual.

\par
%\textsuperscript{(499.6)}
\textsuperscript{44:1.4} 2. \textit{La luz espiritual} ---el control y la intensificación de la luz de los reinos morontiales y espirituales.

\par
%\textsuperscript{(499.7)}
\textsuperscript{44:1.5} 3. \textit{Las incidencias energéticas} ---la melodía producida por la hábil dirección de las energías morontiales y espirituales.

\par
%\textsuperscript{(499.8)}
\textsuperscript{44:1.6} 4. \textit{Las sinfonías de color} ---la melodía de los tonos morontiales de color, que figura entre los logros más elevados de los músicos celestiales.

\par
%\textsuperscript{(499.9)}
\textsuperscript{44:1.7} 5. \textit{La armonía de los espíritus asociados} ---la colocación y la asociación mismas de diferentes órdenes de seres espirituales y morontiales producen unas melodías majestuosas.

\par
%\textsuperscript{(499.10)}
\textsuperscript{44:1.8} 6. \textit{La melodía del pensamiento} ---el hecho de tener pensamientos espirituales se puede perfeccionar hasta el punto de estallar en las melodías de Havona.

\par
%\textsuperscript{(499.11)}
\textsuperscript{44:1.9} 7. \textit{La música del espacio} ---las melodías de otras esferas se pueden captar, mediante una sintonización adecuada, en los circuitos de las transmisiones universales.

\par
%\textsuperscript{(500.1)}
\textsuperscript{44:1.10} Hay más de cien mil maneras diferentes de manipular el sonido, el color y la energía, y son técnicas análogas al empleo de los instrumentos musicales por parte de los humanos. Vuestros conjuntos de baile representan sin duda un intento rudimentario y grotesco de las criaturas materiales por acercarse a la armonía celestial de la colocación de los seres y de la disposición de las personalidades. El mecanismo sensorial de los cuerpos materiales no reconoce las otras cinco formas de melodías morontiales.

\par
%\textsuperscript{(500.2)}
\textsuperscript{44:1.11} La armonía, la música de los siete niveles de la asociación melódica, es el único código universal de comunicación espiritual. La música, tal como la comprenden los mortales de Urantia, alcanza su máxima expresión en las escuelas de Jerusem, la sede del sistema, donde los seres semimateriales aprenden las armonías del sonido. Los mortales no reaccionan ante otras formas de melodía morontial o de armonía celestial.

\par
%\textsuperscript{(500.3)}
\textsuperscript{44:1.12} En Urantia, la apreciación de la música es tanto física como espiritual; y vuestros músicos humanos han hecho mucho por elevar el gusto musical desde la monotonía bárbara de vuestros antepasados primitivos hasta los niveles superiores de la apreciación de los sonidos. La mayoría de los mortales de Urantia reaccionan ante la música principalmente con los músculos materiales, y muy poco con la mente y el espíritu; pero la apreciación musical ha mejorado constantemente durante más de treinta y cinco mil años.

\par
%\textsuperscript{(500.4)}
\textsuperscript{44:1.13} La síncopa melodiosa representa una transición entre la monotonía musical de los hombres primitivos y la armonía llena de expresión y las melodías significativas de vuestros músicos más recientes. Estos tipos de ritmos primitivos estimulan la reacción de los sentidos que aprecian la música, sin implicar el empleo de los poderes intelectuales superiores que aprecian la armonía, atrayendo generalmente más a los individuos inmaduros o espiritualmente indolentes.

\par
%\textsuperscript{(500.5)}
\textsuperscript{44:1.14} La mejor música de Urantia no es más que un eco efímero de los magníficos acordes que escuchan los asociados celestiales de vuestros músicos, los cuales sólo han dejado registrados fragmentos de estas armonías de las fuerzas morontiales bajo la forma de las melodías musicales de las armonías sonoras. La música morontio-espiritual emplea con frecuencia las siete formas de expresión y de reproducción, de manera que la mente humana tropieza con unos obstáculos enormes cuando trata de reducir estas melodías de las esferas superiores a las simples notas de los sonidos musicales. Un esfuerzo así se parecería en parte al hecho de intentar reproducir los acordes de una gran orquesta por medio de un solo instrumento musical.

\par
%\textsuperscript{(500.6)}
\textsuperscript{44:1.15} Aunque habéis reunido algunas hermosas melodías en Urantia, musicalmente no habéis progresado tanto como vuestros planetas vecinos de Satania. Si Adán y Eva tan sólo hubieran sobrevivido, entonces habríais tenido una verdadera música; pero el don de la armonía, tan desarrollado en sus naturalezas, ha sido tan diluido por los linajes con tendencias no musicales que una gran apreciación de la armonía sólo se produce una vez cada mil vidas mortales. Pero no os desaniméis; algún día puede aparecer en Urantia un verdadero músico, y pueblos enteros se sentirán cautivados por los magníficos acordes de sus melodías. Un ser humano así podría cambiar para siempre el curso de una nación entera, e incluso de todo el mundo civilizado. Es literalmente cierto que <<la melodía tiene el poder de transformar a un mundo entero>>. La música seguirá siendo para siempre el idioma universal de los hombres, los ángeles y los espíritus. La armonía es el lenguaje de Havona.

\section*{2. Los reproductores celestiales}
\par
%\textsuperscript{(500.7)}
\textsuperscript{44:2.1} El hombre mortal apenas puede esperar algo más que un concepto pobre y deformado sobre las actividades de los reproductores celestiales, un concepto que debo intentar ilustrar mediante el simbolismo burdo y limitado de vuestro lenguaje material. El mundo morontio-espiritual posee mil y una cosas que tienen un valor supremo, cosas dignas de ser reproducidas pero que son desconocidas en Urantia, experiencias que pertenecen a la categoría de las actividades que difícilmente han <<penetrado en la mente del hombre>>\footnote{\textit{Penetrado en la mente del hombre}: Is 64:4; 1 Co 2:9.}, esas realidades que Dios tiene en espera para aquellos que sobrevivan a la vida en la carne.

\par
%\textsuperscript{(501.1)}
\textsuperscript{44:2.2} Hay siete grupos de reproductores celestiales, y voy a intentar ilustrar su trabajo clasificándolos de la manera siguiente:

\par
%\textsuperscript{(501.2)}
\textsuperscript{44:2.3} 1. \textit{Los cantores} ---los armonistas que reiteran las armonías específicas del pasado e interpretan las melodías del presente. Pero todo esto se efectúa en el nivel morontial.

\par
%\textsuperscript{(501.3)}
\textsuperscript{44:2.4} 2. \textit{Los trabajadores del color} ---los artistas de la luz y la sombra que vosotros llamaríais dibujantes y pintores, los artistas que conservan las escenas pasajeras y los episodios transitorios para el disfrute morontial del futuro.

\par
%\textsuperscript{(501.4)}
\textsuperscript{44:2.5} 3. \textit{Los cineastas de la luz} ---los autores de la conservación de los verdaderos fenómenos semiespirituales, de la cual el cine sólo sería un ejemplo muy rudimentario.

\par
%\textsuperscript{(501.5)}
\textsuperscript{44:2.6} 4. \textit{Los realizadores de los espectáculos históricos} ---aquellos que reproducen mediante representaciones dramáticas los acontecimientos cruciales de los anales y de la historia del universo.

\par
%\textsuperscript{(501.6)}
\textsuperscript{44:2.7} 5. \textit{Los artistas proféticos} ---aquellos que proyectan los significados de la historia hacia el futuro.

\par
%\textsuperscript{(501.7)}
\textsuperscript{44:2.8} 6. \textit{Los narradores de biografías} ---aquellos que perpetúan el significado y la importancia de la experiencia de la vida. La proyección de las experiencias personales actuales hacia los valores que se alcanzarán en el futuro.

\par
%\textsuperscript{(501.8)}
\textsuperscript{44:2.9} 7. \textit{Los actores administrativos} ---aquellos que describen la importancia de la filosofía gubernamental y de la técnica administrativa, los dramaturgos celestiales de la soberanía.

\par
%\textsuperscript{(501.9)}
\textsuperscript{44:2.10} Los reproductores celestiales colaboran con mucha frecuencia y eficacia con los directores de la reversión para combinar la recapitulación de los recuerdos con ciertas formas de descanso mental y de diversión de la personalidad. Antes de los cónclaves morontiales y de las asambleas espirituales, estos reproductores a veces se asocian en enormes espectáculos dramáticos para representar la finalidad de dichas reuniones. Recientemente presencié una prodigiosa representación de este tipo en la que más de un millón de actores produjeron una sucesión de mil escenas.

\par
%\textsuperscript{(501.10)}
\textsuperscript{44:2.11} Los educadores intelectuales superiores y los ministros de transición utilizan de manera abundante y eficaz a estos diversos grupos de reproductores en sus actividades educativas morontiales. Pero todos sus esfuerzos no los dedican a los ejemplos transitorios; una gran parte, una grandísima parte de su trabajo es de carácter permanente, y quedará para siempre como legado para todos los tiempos futuros. Estos artesanos son tan polifacéticos que, cuando actúan en masa, son capaces de volver a representar una era y, en colaboración con los ministros seráficos, pueden describir realmente los valores eternos del mundo espiritual a los videntes mortales del tiempo.

\section*{3. Los constructores divinos}
\par
%\textsuperscript{(501.11)}
\textsuperscript{44:3.1} Hay ciudades <<cuyo constructor y hacedor es Dios>>\footnote{\textit{Cuyo constructor y hacedor es Dios}: 2 Co 5:1; Heb 11:10.}. Poseemos la contrapartida espiritual de todo aquello con que estáis familiarizados los mortales, e indeciblemente más. Tenemos hogares, comodidades espirituales y las cosas morontiales necesarias. Por cada satisfacción material que los humanos pueden disfrutar, tenemos miles de realidades espirituales que sirven para enriquecer y desarrollar nuestra existencia. Los constructores divinos ejercen su actividad en siete grupos:

\par
%\textsuperscript{(502.1)}
\textsuperscript{44:3.2} 1. \textit{Los diseñadores y constructores de hogares} ---aquellos que construyen y transforman las residencias asignadas a los individuos y a los grupos de trabajo. Estos domicilios morontiales y espirituales son reales. Serían invisibles para vuestra visión limitada, pero son muy reales y muy hermosos para nosotros. Hasta cierto punto, todos los seres espirituales pueden compartir con los constructores ciertos detalles sobre la planificación y la creación de sus moradas morontiales o espirituales. Estos hogares están equipados y adornados de acuerdo con las necesidades de las criaturas morontiales o espirituales que van a habitarlos. En todas estas construcciones, los individuos encuentran una variedad abundante y amplias oportunidades para poder expresarse.

\par
%\textsuperscript{(502.2)}
\textsuperscript{44:3.3} 2. \textit{Los constructores de edificios profesionales} ---aquellos que trabajan diseñando y ensamblando las moradas de los trabajadores regulares y rutinarios de los reinos espirituales y morontiales. Estos constructores son comparables a los que construyen los talleres y otras instalaciones industriales en Urantia. Los mundos de transición tienen una economía necesaria de ayuda mutua y de división especializada del trabajo. Cada uno de nosotros no lo hace todo; existe una diversidad de funciones entre los seres morontiales y los espíritus evolutivos, y estos constructores de edificios profesionales no sólo construyen talleres mejores, sino que también contribuyen a la elevación profesional de los trabajadores.

\par
%\textsuperscript{(502.3)}
\textsuperscript{44:3.4} 3. \textit{Los constructores de edificios recreativos} ---Hay enormes edificios que se utilizan durante los períodos de descanso, lo que los mortales llamarían esparcimiento y, en cierto sentido, diversión. Se prevé un escenario adecuado para los directores de la reversión, los humoristas de los mundos morontiales, esas esferas de transición donde tiene lugar la educación de los seres ascendentes que acaban de ser trasladados desde los planetas evolutivos. Incluso los espíritus superiores se dedican a cierta forma de humor reminiscente durante sus períodos de recarga espiritual.

\par
%\textsuperscript{(502.4)}
\textsuperscript{44:3.5} 4. \textit{Los constructores de edificios para la adoración} ---los arquitectos experimentados de los templos espirituales y morontiales. Todos los mundos por donde ascienden los mortales tienen templos para la adoración, y son las creaciones más exquisitas de los reinos morontiales y de las esferas espirituales.

\par
%\textsuperscript{(502.5)}
\textsuperscript{44:3.6} 5. \textit{Los constructores de edificios educativos} ---aquellos que construyen las sedes para la formación morontial y los estudios espirituales avanzados. El camino siempre está abierto para adquirir más conocimiento, para conseguir una información adicional sobre vuestro trabajo presente y futuro así como sobre el conocimiento cultural universal, una información destinada a hacer que los mortales ascendentes sean unos ciudadanos más inteligentes y eficaces en los mundos morontiales y espirituales.

\par
%\textsuperscript{(502.6)}
\textsuperscript{44:3.7} 6. \textit{Los planificadores morontiales} ---aquellos que construyen para las asociaciones coordinadas de todas las personalidades de todos los reinos, a medida que se encuentran presentes en cualquier momento en cualquier esfera. Estos planificadores colaboran con los Supervisores del Poder Morontial para enriquecer la coordinación de la vida morontial progresiva.

\par
%\textsuperscript{(502.7)}
\textsuperscript{44:3.8} 7. \textit{Los constructores de edificios públicos} ---los artesanos que planifican y construyen los lugares para las reuniones, distintos a los destinados a la adoración. Los lugares para las reuniones públicas son grandes y magníficos.

\par
%\textsuperscript{(502.8)}
\textsuperscript{44:3.9} Aunque ni estas estructuras ni sus adornos serían exactamente reales para la comprensión sensorial de los mortales materiales, son muy reales para nosotros. Seríais incapaces de ver estos templos aunque estuvierais allí en persona; sin embargo, todas estas creaciones supermateriales están realmente allí, y nosotros las discernimos claramente al igual que disfrutamos plenamente de ellas.

\section*{4. Los registradores del pensamiento}
\par
%\textsuperscript{(503.1)}
\textsuperscript{44:4.1} Estos artesanos se dedican a conservar y reproducir el pensamiento superior de los reinos, y ejercen su actividad en siete grupos:

\par
%\textsuperscript{(503.2)}
\textsuperscript{44:4.2} 1. \textit{Los conservadores del pensamiento}. Son los artesanos que se dedican a conservar el pensamiento superior de los reinos. En los mundos morontiales, atesoran realmente las joyas de la actividad intelectual. Antes de venir por primera vez a Urantia, vi los registros y escuché las transmisiones de la ideación de algunas grandes mentes de este planeta. Los registradores del pensamiento conservan estas nobles ideas en la lengua de Uversa.

\par
%\textsuperscript{(503.3)}
\textsuperscript{44:4.3} Cada superuniverso tiene su propio idioma, una lengua hablada por sus personalidades y que predomina en todos sus sectores. En nuestro superuniverso se la conoce como la lengua de Uversa. Cada universo local tiene también su propio idioma. Todas las órdenes superiores de Nebadon son biling\"ues, y hablan tanto el idioma de Nebadon como la lengua de Uversa. Cuando dos individuos de diferentes universos locales se encuentran, se comunican en la lengua de Uversa; sin embargo, si uno de ellos procede de otro superuniverso, tienen que recurrir a un traductor. En el universo central hay poca necesidad de un idioma; allí existe una comprensión perfecta y casi completa; los Dioses son los únicos que no son allí plenamente comprendidos. Nos enseñan que un encuentro casual en el Paraíso revela una comprensión mutua mayor que la que se podría comunicar mediante una lengua humana en mil años. Incluso en Salvington <<conocemos de igual forma que somos conocidos>>\footnote{\textit{Conocemos como somos conocidos}: Jn 10:15; 1 Co 13:12.}.

\par
%\textsuperscript{(503.4)}
\textsuperscript{44:4.4} La capacidad para traducir el pensamiento a un idioma en las esferas morontiales y espirituales sobrepasa la comprensión de los mortales. La velocidad a la que reducimos el pensamiento a un registro permanente puede ser acelerada por los expertos registradores de tal manera que en un minuto del tiempo de Urantia se puede registrar el equivalente de más de medio millón de palabras o símbolos de pensamiento. Estos idiomas universales son mucho más ricos que las lenguas de los mundos en evolución. Los símbolos conceptuales de Uversa abarcan más de mil millones de caracteres, aunque su alfabeto básico sólo contiene setenta símbolos. El idioma de Nebadon no es en absoluto tan elaborado, pues sus símbolos básicos, o alfabeto, sólo ascienden a cuarenta y ocho.

\par
%\textsuperscript{(503.5)}
\textsuperscript{44:4.5} 2. \textit{Los registradores de conceptos}. Este segundo grupo de registradores se ocupa de conservar las imágenes conceptuales, las configuraciones de las ideas. Es una forma de registro permanente desconocida en los reinos materiales; con este método yo podría adquirir, en una hora de vuestro tiempo, más conocimiento del que vosotros podríais conseguir leyendo atentamente vuestros escritos ordinarios durante cien años.

\par
%\textsuperscript{(503.6)}
\textsuperscript{44:4.6} 3. \textit{Los registradores ideográfícos}. Tenemos el equivalente de vuestro lenguaje tanto hablado como escrito, pero para conservar el pensamiento empleamos generalmente la ilustración de los conceptos y las técnicas ideográficas. Aquellos que conservan los ideogramas son capaces de mejorar mil veces el trabajo de los registradores de conceptos.

\par
%\textsuperscript{(503.7)}
\textsuperscript{44:4.7} 4. \textit{Los promotores de la oratoria}. Este grupo de registradores se ocupa de la tarea de conservar el pensamiento para reproducirlo mediante la oratoria. Pero en el idioma de Nebadon podríamos exponer, en una alocución de media hora, el tema de toda la vida de un mortal de Urantia. La única esperanza que tenéis de comprender estas operaciones consiste en hacer una pausa y examinar la técnica de vuestra vida onírica desorganizada y confusa ---la manera en que podéis atravesar en pocos segundos años de experiencia durante esas fantasías del período nocturno.

\par
%\textsuperscript{(503.8)}
\textsuperscript{44:4.8} La oratoria del mundo espiritual es uno de los placeres excepcionales que os esperan, a vosotros que sólo habéis escuchado los discursos imperfectos y titubeantes de Urantia. En los discursos de Salvington y de Edentia hay una armonía musical y una eufonía expresiva que nos inspiran más allá de lo que se puede describir. Estos conceptos ardientes son como joyas de belleza en diademas de gloria. ¡Pero no puedo lograrlo! ¡No puedo transmitir a la mente humana la amplitud y la profundidad de estas realidades de otro mundo!

\par
%\textsuperscript{(504.1)}
\textsuperscript{44:4.9} 5. \textit{Los directores de las transmisiones}. Las transmisiones del Paraíso, de los superuniversos y de los universos locales se encuentran bajo la supervisión general de este grupo de conservadores del pensamiento. Sirven como censores y redactores, así como coordinadores, del material a transmitir, efectuando una adaptación para los superuniversos de todas las transmisiones del Paraíso, y adaptando y traduciendo las transmisiones de los Ancianos de los Días a las lenguas individuales de los universos locales.

\par
%\textsuperscript{(504.2)}
\textsuperscript{44:4.10} Las transmisiones del universo local también se deben modificar para que los sistemas y los planetas individuales puedan recibirlas. La transmisión de estos informes espaciales se supervisa cuidadosamente, y siempre hay un registro de confirmación que asegura la recepción adecuada de cada informe en todos los mundos de un circuito dado. Estos directores de las transmisiones son unos expertos en la técnica de utilizar las corrientes del espacio para comunicar la información.

\par
%\textsuperscript{(504.3)}
\textsuperscript{44:4.11} 6. \textit{Los registradores de ritmos}. No hay duda de que los urantianos denominarían poetas a estos artesanos, aunque sus obras son muy diferentes a vuestras producciones poéticas y las trascienden de manera casi infinita. El ritmo es menos agotador para los seres morontiales y espirituales, y por eso se realiza con frecuencia un esfuerzo por acrecentar la eficacia, así como por aumentar el placer, efectuando numerosas actividades de manera rítmica. Sólo desearía que tuvierais el privilegio de escuchar algunas transmisiones poéticas de las asambleas de Edentia para disfrutar de la riqueza de colores y de sonidos de los genios de la constelación, los cuales son unos maestros en esta exquisita forma de expresión personal y de armonización social.

\par
%\textsuperscript{(504.4)}
\textsuperscript{44:4.12} 7. \textit{Los registradores morontiales}. No sé cómo describir a la mente material las funciones de este importante grupo de registradores del pensamiento asignados a la tarea de conservar las imágenes de conjunto de las diversas agrupaciones encargadas de los asuntos morontiales y de las operaciones espirituales; utilizando un ejemplo imperfecto, son los fotógrafos colectivos de los mundos de transición. Salvaguardan para el futuro las escenas y las asociaciones vitales de estas épocas progresivas, conservándolas en los archivos de las salas de registro morontiales.

\section*{5. Los manipuladores de la energía}
\par
%\textsuperscript{(504.5)}
\textsuperscript{44:5.1} Estos interesantes y eficaces artesanos se ocupan de todos los tipos de energía: física, mental y espiritual.

\par
%\textsuperscript{(504.6)}
\textsuperscript{44:5.2} 1. \textit{Los manipuladores de la energía física}. Los manipuladores de la energía física sirven durante largos períodos con los directores del poder, y son expertos en la manipulación y el control de muchas fases de la energía física. Están familiarizados con las tres corrientes fundamentales y con las treinta divisiones energéticas subsidiarias de los superuniversos. Estos seres son de una ayuda inestimable para los Supervisores del Poder Morontial de los mundos de transición. Son los estudiosos permanentes de las proyecciones cósmicas del Paraíso.

\par
%\textsuperscript{(504.7)}
\textsuperscript{44:5.3} 2. \textit{Los manipuladores de la energía mental}. Son los expertos en las comunicaciones entre los seres morontiales y otros tipos de seres inteligentes. Esta forma de comunicación entre los mortales no existe prácticamente en Urantia. Son los especialistas que promueven la capacidad de los seres morontiales ascendentes para comunicarse entre sí, y su trabajo abarca numerosas aventuras excepcionales de enlaces intelectuales que se encuentran mucho más allá de mi capacidad para describirlas a la mente material. Estos artesanos son los aplicados estudiosos de los circuitos mentales del Espíritu Infinito.

\par
%\textsuperscript{(505.1)}
\textsuperscript{44:5.4} 3. \textit{Los manipuladores de la energía espiritual}. Los manipuladores de la energía espiritual forman un grupo fascinante. La energía espiritual actúa de acuerdo con las leyes establecidas, tal como lo hace la energía física. Es decir, cuando se estudia la fuerza espiritual, ésta proporciona conclusiones fiables y puede ser tratada con precisión, igual que sucede con las energías físicas. Las leyes del mundo espiritual son tan seguras y fiables como las que existen en los reinos materiales. Durante los últimos pocos millones de años, estos estudiantes de las leyes fundamentales del Hijo Eterno, las cuales gobiernan la energía espiritual tal como ésta se aplica a las órdenes morontiales y a otras órdenes de seres celestiales en todos los universos, han efectuado muchas mejoras en las técnicas para absorber la energía espiritual.

\par
%\textsuperscript{(505.2)}
\textsuperscript{44:5.5} 4. \textit{Los manipuladores combinados}. Es el grupo aventurero de seres bien preparados que se dedican a la asociación funcional de las tres fases originales de la energía divina que se manifiestan en todos los universos como energía física, mental y espiritual. Son las personalidades aplicadas que están tratando de descubrir en realidad la presencia universal de Dios Supremo, ya que en esta personalidad de la Deidad deberá producirse la unificación experiencial de toda la divinidad del gran universo. Y, hasta cierto punto, estos artesanos han conseguido algunos éxitos en los últimos tiempos.

\par
%\textsuperscript{(505.3)}
\textsuperscript{44:5.6} 5. \textit{Los asesores de los transportes}. Este cuerpo de asesores técnicos para los serafines transportadores es sumamente hábil colaborando con los estudiosos de las estrellas para elaborar los itinerarios y ayudar de otras maneras a los jefes de los transportes en los mundos del espacio. Son los supervisores del tráfico de las esferas y están presentes en todos los planetas habitados. Un cuerpo de setenta asesores de los transportes está sirviendo en Urantia.

\par
%\textsuperscript{(505.4)}
\textsuperscript{44:5.7} 6. \textit{Los expertos en las comunicaciones}. Doce técnicos en comunicaciones interplanetarias e interuniversales están igualmente de servicio en Urantia. Estos seres tan experimentados son unos expertos en el conocimiento de las leyes que gobiernan las transmisiones y las interferencias tal como éstas se aplican a las comunicaciones de los reinos. Este cuerpo se ocupa de todas las formas de mensajes espaciales, salvo de aquellos de los Mensajeros de Gravedad y de los Mensajeros Solitarios. En Urantia, una gran parte de su trabajo ha de realizarse a través del circuito de los arcángeles.

\par
%\textsuperscript{(505.5)}
\textsuperscript{44:5.8} 7. \textit{Los profesores del descanso}. El descanso divino\footnote{\textit{Descanso divino y físico}: Gn 2:2-3; Ex 20:11; Lv 25:4-5.} está asociado a la técnica de la absorción de la energía espiritual. La energía morontial y espiritual ha de reponerse tan ciertamente como la energía física, pero no por las mismas razones. Me veo obligado a emplear forzosamente unos ejemplos rudimentarios en mis intentos por iluminaros; sin embargo, nosotros, los del mundo espiritual, debemos interrumpir periódicamente nuestras actividades regulares y trasladarnos a los lugares adecuados de reunión donde entramos en el descanso divino y recuperamos así nuestras energías agotadas.

\par
%\textsuperscript{(505.6)}
\textsuperscript{44:5.9} Recibiréis vuestras primeras lecciones en estas materias cuando lleguéis a los mundos de las mansiones después de haberos convertido en seres morontiales y de haber empezado a experimentar la técnica de los asuntos espirituales. Sabéis algunas cosas sobre el círculo más interior de Havona y que, después de que los peregrinos del espacio han atravesado los círculos precedentes, deben ser inducidos al largo descanso revivificante del Paraíso. Esto no es solamente un requisito técnico para pasar de la carrera del tiempo al servicio de la eternidad, sino que es también una necesidad, una forma de descanso necesaria para reponer las pérdidas energéticas inherentes a las etapas finales de la experiencia ascendente, y almacenar las reservas de poder espiritual para la fase siguiente de la carrera sin fin.

\par
%\textsuperscript{(506.1)}
\textsuperscript{44:5.10} Estos manipuladores de la energía ejercen también su actividad de centenares de otras maneras demasiado numerosas como para ser catalogadas, tales como aconsejar a los serafines, querubines y sanobines sobre las formas más eficaces de absorber la energía, y en lo relacionado con el mantenimiento del equilibrio más útil entre las fuerzas divergentes de los querubines activos y de los sanobines pasivos. Estos expertos prestan su ayuda de otras muchas maneras a las criaturas morontiales y espirituales en sus esfuerzos por comprender el descanso divino, que es tan esencial para utilizar eficazmente las energías fundamentales del espacio.

\section*{6. Los diseñadores y los embellecedores}
\par
%\textsuperscript{(506.2)}
\textsuperscript{44:6.1} ¡Cómo desearía saber la manera de describir el trabajo exquisito de estos artesanos únicos! Todo intento por mi parte por explicar el trabajo del embellecimiento espiritual\footnote{\textit{Embellecedores}: 1 Cr 29:2; Is 54:11-12.} sólo haría recordar a la mente material vuestros propios esfuerzos lamentables, pero meritorios, por llevar a cabo estas cosas en vuestro mundo de mente y de materia.

\par
%\textsuperscript{(506.3)}
\textsuperscript{44:6.2} Aunque este cuerpo abarca más de mil subdivisiones de actividad, está agrupado en las siete categorías principales siguientes:

\par
%\textsuperscript{(506.4)}
\textsuperscript{44:6.3} 1. \textit{Los artesanos del color}. Son ellos los que hacen que los diez mil tonos de color del reflejo espiritual repiquen sus exquisitos mensajes de belleza armoniosa. Aparte de la percepción de los colores, no hay nada en la experiencia humana con lo que estas actividades se puedan comparar.

\par
%\textsuperscript{(506.5)}
\textsuperscript{44:6.4} 2. \textit{Los diseñadores de los sonidos}. Estos diseñadores de lo que vosotros llamaríais sonidos describen las ondas espirituales de diversa identidad que se pueden apreciar morontialmente. Estos impulsos son en realidad los magníficos reflejos de las almas espirituales desnudas y gloriosas de las huestes celestiales.

\par
%\textsuperscript{(506.6)}
\textsuperscript{44:6.5} 3. \textit{Los diseñadores de las emociones}. Estos realzadores y conservadores de las sensaciones son los que guardan los sentimientos morontiales y las emociones de la divinidad para el estudio y la edificación de los hijos del tiempo, y para la inspiración y el embellecimiento de los progresores morontiales y de los espíritus que avanzan.

\par
%\textsuperscript{(506.7)}
\textsuperscript{44:6.6} 4. \textit{Los artistas del olor}. Esta comparación de las actividades celestiales espirituales con el reconocimiento físico de los olores químicos es realmente desacertada, pero los mortales de Urantia difícilmente podrían reconocer este ministerio si utilizamos cualquier otro nombre. Estos artesanos crean sus variadas sinfonías para la edificación y el deleite de los hijos de la luz que progresan. En la Tierra no tenéis nada que se pueda comparar, ni siquiera remotamente, con este tipo de grandiosidad espiritual.

\par
%\textsuperscript{(506.8)}
\textsuperscript{44:6.7} 5. \textit{Los embellecedores de las presencias}. Estos artesanos no se ocupan de las artes de adornarse ni de la técnica de embellecer a las criaturas. Están dedicados a la tarea de causar reacciones alegres y multitudinarias en las criaturas individuales morontiales y espirituales, representando escénicamente la importancia de las relaciones mediante los valores de las posiciones que asignan a las diferentes órdenes morontiales y espirituales en los conjuntos que componen con estos seres diversos. Estos artistas colocan a los seres supermateriales como vosotros lo haríais con las notas musicales, los olores y los paisajes vivientes, y luego los mezclan en himnos de gloria.

\par
%\textsuperscript{(506.9)}
\textsuperscript{44:6.8} 6. \textit{Los diseñadores del gusto}. ¡Y qué os puedo decir de estos artistas! Podría ligeramente sugerir que son los que mejoran el gusto morontial, y también se esfuerzan por acrecentar la apreciación de la belleza mediante la agudización de los sentidos espirituales en evolución.

\par
%\textsuperscript{(507.1)}
\textsuperscript{44:6.9} 7. \textit{Los sintetizadores morontiales}. Son los artesanos maestros que, cuando todos los demás han aportado sus contribuciones respectivas, añaden entonces los toques finales y culminantes al conjunto morontial, consiguiendo así una representación inspiradora de lo divinamente hermoso, una inspiración duradera para los seres espirituales y sus asociados morontiales. Pero tendréis que esperar a ser liberados del cuerpo animal antes de poder empezar a concebir las glorias artísticas y las bellezas estéticas de los mundos morontiales y espirituales.

\section*{7. Los trabajadores de la armonía}
\par
%\textsuperscript{(507.2)}
\textsuperscript{44:7.1} En contra de lo que podríais suponer, estos artistas no se ocupan de la música, ni de la pintura, ni de nada similar. Se ocupan de manipular y de organizar las fuerzas y las energías especializadas que están presentes en el mundo espiritual, pero que no son reconocidas por los mortales. Si tuviera la más mínima base para comparar, trataría de describir este campo excepcional de consecución espiritual, pero pierdo la esperanza de poder hacerlo ---no existe ninguna esperanza de transmitir a las mentes mortales esta esfera del arte celestial. Sin embargo, aquello que no se puede describir puede no obstante estar implícito:

\par
%\textsuperscript{(507.3)}
\textsuperscript{44:7.2} La belleza, el ritmo y la armonía están intelectualmente asociados y son espiritualmente semejantes. La verdad, los hechos y las relaciones son intelectualmente inseparables y están asociados con los conceptos filosóficos de la belleza. La bondad, la rectitud y la justicia están filosóficamente interrelacionados y espiritualmente unidos a la verdad viviente y a la belleza divina.

\par
%\textsuperscript{(507.4)}
\textsuperscript{44:7.3} Los conceptos cósmicos de la verdadera filosofía, la descripción del arte celestial o el intento de los mortales por describir el reconocimiento humano de la belleza divina nunca pueden ser verdaderamente satisfactorios si estas tentativas de progreso por parte de las criaturas no están unificadas. Estas expresiones del impulso divino dentro de la criatura en evolución pueden ser intelectualmente verdaderas, emocionalmente hermosas y espiritualmente buenas; pero la verdadera alma de la expresión estará ausente, a menos que estas realidades de la verdad, estos significados de la belleza y estos valores de la bondad estén unificados en la experiencia vital del artesano, del científico o del filósofo.

\par
%\textsuperscript{(507.5)}
\textsuperscript{44:7.4} Estas cualidades divinas están perfecta y absolutamente unificadas en Dios. Y todo hombre o ángel que conoce a Dios posee el potencial de expresarse sin límites en unos niveles progresivos de autorrealización unificada mediante la técnica de conseguir interminablemente parecerse a Dios ---la mezcla experiencial, en la experiencia evolutiva, de la verdad eterna, la belleza universal y la bondad divina.

\section*{8. Las aspiraciones humanas y los logros morontiales}
\par
%\textsuperscript{(507.6)}
\textsuperscript{44:8.1} Aunque los artesanos celestiales no trabajan personalmente en los planetas materiales tales como Urantia, de vez en cuando vienen desde la sede del sistema para ofrecer su ayuda a los individuos dotados por naturaleza de las razas mortales. Cuando tienen esta misión, estos artesanos trabajan temporalmente bajo la supervisión de los ángeles planetarios del progreso. Las huestes seráficas cooperan con estos artesanos para intentar ayudar a aquellos artistas mortales que poseen facultades inherentes, y que también poseen Ajustadores con una experiencia previa y especial.

\par
%\textsuperscript{(507.7)}
\textsuperscript{44:8.2} Las capacidades humanas especiales tienen tres orígenes posibles: En el fondo, \textit{siempre} hay una aptitud natural o inherente. Una capacidad especial nunca es un don arbitrario de los Dioses; en todo talento sobresaliente siempre hay una base ancestral. Además de esta capacidad natural, o más bien adicional a ella, el Ajustador del Pensamiento puede contribuir con sus directrices en aquellos individuos cuyos Ajustadores interiores pueden haber tenido, en ese ámbito, experiencias auténticas y reales en otros mundos y en otras criaturas mortales. En aquellos casos en que tanto la mente humana como el Ajustador interior son extraordinariamente hábiles, los artesanos espirituales pueden recibir el encargo de actuar como armonizadores de esos talentos y de ayudar e inspirar de otras maneras a esos mortales en su búsqueda de unos ideales cada vez más perfectos y en sus intentos por describirlos de forma elevada para la edificación del reino.

\par
%\textsuperscript{(508.1)}
\textsuperscript{44:8.3} En las filas de los artesanos espirituales no hay ninguna casta. Por muy humilde que sea vuestro origen, si tenéis la capacidad y el don de la expresión, conseguiréis un reconocimiento adecuado y recibiréis la debida apreciación a medida que ascendáis hacia arriba en la escala de la experiencia morontial y de la consecución espiritual. No puede haber ningún obstáculo debido a la herencia humana, ni ninguna privación causada por el entorno mortal, que la carrera morontial no compense plenamente y elimine por completo. Vuestros propios esfuerzos personales por avanzar de manera progresiva producirán todas estas satisfacciones de logros artísticos y de autorrealización expresiva. Por fin se podrán realizar las aspiraciones de la medianía evolutiva. Aunque los Dioses no conceden arbitrariamente talentos y capacidades a los hijos del tiempo, proporcionan los medios para que satisfagan todos sus nobles anhelos y para contentar todo apetito humano por expresarse de manera celestial.

\par
%\textsuperscript{(508.2)}
\textsuperscript{44:8.4} Pero todo ser humano debería recordar que muchas ambiciones por sobresalir, que atormentan a los mortales durante su vida en la carne, no subsistirán en la carrera morontial y espiritual de esos mismos mortales. Los morontiales ascendentes aprenden a hacer sociables sus antiguos anhelos puramente interesados y sus antiguas ambiciones egoístas. Sin embargo, aquellas cosas que tan ardientemente deseasteis hacer en la Tierra y que las circunstancias os negaron tan continuamente, si todavía deseáis hacerlas después de haber adquirido la verdadera perspicacia de la mota durante la carrera morontial, entonces se os concederán con toda seguridad todas las oportunidades de satisfacer plenamente vuestros deseos tanto tiempo acariciados.

\par
%\textsuperscript{(508.3)}
\textsuperscript{44:8.5} Antes de que los mortales ascendentes dejen el universo local para emprender su carrera espiritual, serán saciados en todos sus anhelos o verdaderas ambiciones intelectuales, artísticas y sociales que hayan podido caracterizar sus planos de existencia mortales o morontiales. Es conseguir la igualdad en lo que respecta a satisfacer la expresión y la realización de sí mismo, pero no es alcanzar un estado experiencial idéntico ni eliminar por completo la individualidad característica en las áreas de la habilidad, la técnica y la expresión. Pero el nuevo diferencial espiritual de consecución experiencial personal no llegará a nivelarse ni a equilibrarse así hasta después de que hayáis terminado en el último círculo de la carrera de Havona. Y los residentes del Paraíso se enfrentarán entonces a la necesidad de ajustarse a ese diferencial absonito de experiencia personal, que sólo se podrá nivelar alcanzando en grupo el estado último de las criaturas ---el destino de los finalitarios mortales como espíritus de la séptima fase.

\par
%\textsuperscript{(508.4)}
\textsuperscript{44:8.6} Ésta es pues la historia de los artesanos celestiales, ese cuerpo cosmopolita de trabajadores exquisitos que tanto contribuyen a glorificar las esferas arquitectónicas con las representaciones artísticas de la belleza divina de los Creadores Paradisiacos.

\par
%\textsuperscript{(508.5)}
\textsuperscript{44:8.7} [Redactado por un Arcángel de Nebadon.]


\chapter{Documento 45. La administración del sistema local}
\par
%\textsuperscript{(509.1)}
\textsuperscript{45:0.1} EL CENTRO administrativo de Satania está compuesto por un grupo de cincuenta y siete esferas arquitectónicas ---Jerusem misma, los siete satélites mayores y los cuarenta y nueve subsatélites. Jerusem, la capital del sistema, tiene casi cien veces el tamaño de Urantia, aunque su gravedad es un poco menor. Los satélites mayores de Jerusem son los siete mundos de transición, y cada uno de ellos es casi diez veces más grande que Urantia, mientras que los siete subsatélites de estas esferas de transición tienen casi exactamente el tamaño de Urantia.

\par
%\textsuperscript{(509.2)}
\textsuperscript{45:0.2} Los siete mundos de las mansiones son los siete subsatélites del mundo de transición número uno.

\par
%\textsuperscript{(509.3)}
\textsuperscript{45:0.3} Todo este sistema de cincuenta y siete mundos arquitectónicos está iluminado, calentado y abastecido de agua y de energía de forma independiente gracias a la coordinación del Centro de Poder de Satania y de los Controladores Físicos Maestros, de acuerdo con la técnica establecida para la organización y la disposición físicas de estas esferas especialmente creadas. Los espornagias nativos también las cuidan físicamente y se encargan de su mantenimiento de otras maneras.

\section*{1. Los mundos culturales de transición}
\par
%\textsuperscript{(509.4)}
\textsuperscript{45:1.1} A los siete mundos mayores que giran alrededor de Jerusem se les conoce generalmente como las esferas culturales de transición. Sus gobernantes son nombrados de vez en cuando por el consejo ejecutivo supremo de Jerusem. Estas esferas tienen los nombres y los números siguientes:

\par
%\textsuperscript{(509.5)}
\textsuperscript{45:1.2} \textit{Número 1. El mundo de los finalitarios}. Es la sede del cuerpo finalitario del sistema local y está rodeada por los mundos receptores, los siete mundos de las mansiones, tan plenamente dedicados al programa de la ascensión de los mortales. El mundo finalitario es accesible a los habitantes de los siete mundos de las mansiones. Los serafines transportadores llevan a las personalidades ascendentes de un sitio para otro durante estos peregrinajes que están destinados a cultivar su fe en el destino último de los mortales de transición. Aunque los finalitarios y sus edificios no son habitualmente perceptibles para la visión morontial, os sentiréis más que emocionados cuando los transformadores de la energía y los Supervisores del Poder Morontial os permitan vislumbrar momentáneamente, de vez en cuando, estas elevadas personalidades espirituales que han terminado realmente la ascensión al Paraíso, y que han regresado a los mundos mismos donde estáis empezando este largo viaje para garantizar la seguridad de que os es posible y podéis terminar esta formidable empresa. Todos los residentes de los mundos de las mansiones van a la esfera finalitaria al menos una vez al año para asistir a estas asambleas donde perciben a los finalitarios.

\par
%\textsuperscript{(510.1)}
\textsuperscript{45:1.3} \textit{Número 2. El mundo de la morontia}. Este planeta es la sede de los supervisores de la vida morontial y está rodeado por las siete esferas donde los jefes morontiales enseñan a sus asociados y ayudantes, que son tanto seres morontiales como mortales ascendentes.

\par
%\textsuperscript{(510.2)}
\textsuperscript{45:1.4} Cuando paséis por los siete mundos de las mansiones, también progresaréis por estas esferas culturales y sociales donde se efectúa un contacto creciente con la morontia. Cuando avancéis del primer mundo de las mansiones al segundo, tendréis derecho a un permiso para visitar la sede de transición número dos, el mundo de la morontia, y así sucesivamente. Y cuando estéis presentes en una de estas seis esferas culturales, podréis visitar y observar, por invitación, cualquiera de los siete mundos de actividades colectivas asociadas que la rodean.

\par
%\textsuperscript{(510.3)}
\textsuperscript{45:1.5} \textit{Número 3. El mundo de los ángeles}. Es la sede de todas las huestes seráficas que se dedican a las actividades del sistema, y está rodeada por los siete mundos donde se enseña y se instruye a los ángeles. Son las esferas sociales seráficas.

\par
%\textsuperscript{(510.4)}
\textsuperscript{45:1.6} \textit{Número 4. El mundo de los superángeles}. Esta esfera es, en Satania, el hogar de las Brillantes Estrellas Vespertinas y de una inmensa concurrencia de seres coordinados y casi coordinados. Los siete satélites de este mundo están asignados a los siete grupos principales de estos seres celestiales innominados.

\par
%\textsuperscript{(510.5)}
\textsuperscript{45:1.7} \textit{Número 5. El mundo de los Hijos}. Este planeta es la sede de los Hijos divinos de todas las órdenes, incluyendo a los hijos trinitizados por las criaturas. Los siete mundos que lo rodean están dedicados a ciertas agrupaciones individuales de estos hijos divinamente emparentados.

\par
%\textsuperscript{(510.6)}
\textsuperscript{45:1.8} \textit{Número 6. El mundo del Espíritu}. Esta esfera sirve como punto sistémico de encuentro para las personalidades elevadas del Espíritu Infinito. Los siete satélites que la rodean están asignados a los grupos individuales de estas diversas órdenes. Pero en el mundo de transición número seis no hay representación del Espíritu, y esta presencia tampoco se puede observar en las capitales de los sistemas; la Ministra Divina de Salvington se encuentra \textit{portodas partes} en Nebadon.

\par
%\textsuperscript{(510.7)}
\textsuperscript{45:1.9} \textit{Número 7. El mundo del Padre}. Es la esfera silenciosa del sistema. Ningún grupo de seres está domiciliado aquí. El gran templo de luz ocupa un lugar central, pero no se puede discernir a nadie en su interior. Todos los seres de todos los mundos del sistema son bienvenidos como adoradores.

\par
%\textsuperscript{(510.8)}
\textsuperscript{45:1.10} Los siete satélites que rodean al mundo del Padre se utilizan de diversas maneras en los diferentes sistemas. En Satania se emplean actualmente como esferas de detención para los grupos internados de la rebelión de Lucifer. Edentia, la capital de la constelación, no tiene mundos prisiones análogos; los pocos serafines y querubines que se unieron a los rebeldes durante la rebelión de Satania han sido confinados desde hace mucho tiempo en estos mundos de aislamiento de Jerusem.

\par
%\textsuperscript{(510.9)}
\textsuperscript{45:1.11} Como residentes del séptimo mundo de las mansiones, tendréis acceso al séptimo mundo de transición, la esfera del Padre Universal, y también tendréis permiso para visitar los mundos prisiones de Satania que rodean a este planeta, donde actualmente están confinados Lucifer y la mayoría de las personalidades que lo siguieron en su rebelión contra Miguel. Este triste espectáculo ha podido ser observado durante las eras recientes y continuará sirviendo como advertencia solemne para todo Nebadon hasta que los Ancianos de los Días juzguen el pecado de Lucifer y de sus asociados caídos que rechazaron la salvación ofrecida por Miguel, el Padre de su universo.

\section*{2. El Soberano del Sistema}
\par
%\textsuperscript{(511.1)}
\textsuperscript{45:2.1} El jefe ejecutivo de un sistema local de mundos habitados es un Hijo Lanonandek primario, el Soberano del Sistema. En nuestro universo local, a estos soberanos les confían grandes responsabilidades ejecutivas, unas prerrogativas personales excepcionales. Incluso en Orvonton, no todos los universos están organizados para permitir que los Soberanos de los Sistemas ejerzan estos poderes discrecionales personales tan extraordinariamente amplios en la dirección de los asuntos sistémicos. Pero en toda la historia de Nebadon, estos ejecutivos sin trabas sólo han mostrado su deslealtad en tres ocasiones. La rebelión de Lucifer en el sistema de Satania ha sido la última y la más extensa de todas.

\par
%\textsuperscript{(511.2)}
\textsuperscript{45:2.2} En Satania, incluso después de este levantamiento desastroso, la técnica administrativa del sistema no ha sufrido absolutamente ningún cambio. El Soberano actual del Sistema posee todo el poder y ejerce toda la autoridad que le habían sido conferidos a su indigno predecesor, salvo en ciertas materias que se encuentran actualmente bajo la supervisión de los Padres de la Constelación y que los Ancianos de los Días aún no han restituido plenamente a Lanaforge, el sucesor de Lucifer.

\par
%\textsuperscript{(511.3)}
\textsuperscript{45:2.3} El jefe actual de Satania es un gobernante brillante y bondadoso, un soberano a prueba de rebeliones. Cuando servía como asistente del Soberano de otro Sistema, Lanaforge fue fiel a Miguel durante un levantamiento anterior en el universo de Nebadon. Este poderoso y brillante Señor de Satania es un administrador probado y experimentado. En la época de la segunda rebelión sistémica en Nebadon, cuando el Soberano de aquel Sistema tropezó y cayó en las tinieblas, Lanaforge, entonces primer asistente de este jefe equivocado, tomó las riendas del gobierno y condujo de tal manera los asuntos del sistema que se perdieron relativamente pocas personalidades tanto en los mundos sede como en los planetas habitados de aquel sistema poco afortunado. Lanaforge tiene la distinción de ser el único Hijo Lanonandek primario de todo Nebadon que actuó así de manera leal al servicio de Miguel y en presencia misma del fallo de su hermano que poseía una autoridad superior y un rango precedente. Lanaforge no será probablemente retirado de Jerusem hasta que todos los resultados de la locura anterior hayan sido superados y los productos de la rebelión hayan sido eliminados de Satania.

\par
%\textsuperscript{(511.4)}
\textsuperscript{45:2.4} Aunque todos los asuntos de los mundos aislados de Satania no han sido puestos de nuevo bajo su jurisdicción, Lanaforge muestra un gran interés por el bienestar de tales planetas y visita con frecuencia Urantia. Tal como sucede en otros sistemas normales, el Soberano preside el consejo sistémico de los gobernantes de los mundos, los Príncipes Planetarios y los gobernadores generales residentes de los mundos aislados. Este consejo planetario se reúne de vez en cuando en la sede del sistema ---<<Cuando los Hijos de Dios se reúnen>>\footnote{\textit{Cuando los Hijos de Dios se reúnen}: Job 1:6; 2:1.}.

\par
%\textsuperscript{(511.5)}
\textsuperscript{45:2.5} Una vez por semana, cada diez días de Jerusem, el Soberano celebra un cónclave con algún grupo de las diversas órdenes de personalidades domiciliadas en el mundo sede. Son los momentos encantadoramente informales de Jerusem, unos acontecimientos inolvidables. En Jerusem reina la fraternidad más grande entre todas las diversas órdenes de seres, y entre cada uno de estos grupos y el Soberano del Sistema.

\par
%\textsuperscript{(511.6)}
\textsuperscript{45:2.6} Estas asambleas incomparables se celebran en el mar de cristal, el gran campo de reunión de la capital del sistema. Se trata de unos actos puramente sociales y espirituales; nunca se discute nada relacionado con la administración planetaria y ni siquiera con el plan de la ascensión. Los mortales ascendentes se reúnen en esos momentos simplemente para divertirse y encontrarse con sus compañeros jerusemitas. Los grupos que no son invitados a estos descansos semanales del Soberano se reúnen en sus propias sedes.

\section*{3. El gobierno del sistema}
\par
%\textsuperscript{(512.1)}
\textsuperscript{45:3.1} El jefe ejecutivo de un sistema local, el Soberano del Sistema, está siempre apoyado por dos o tres Hijos Lanonandeks que ejercen su actividad como primero y segundo asistentes. Pero en el momento actual, el sistema de Satania está administrado por un estado mayor de siete Lanonandeks:

\par
%\textsuperscript{(512.2)}
\textsuperscript{45:3.2} 1. \textit{El Soberano del Sistema} ---Lanaforge, número 2.709 de la orden primaria y sucesor del apóstata Lucifer.

\par
%\textsuperscript{(512.3)}
\textsuperscript{45:3.3} 2. \textit{El primer Soberano asistente} ---Mansurotia, número 17.841 de los Lanonandeks terciarios. Fue enviado a Satania junto con Lanaforge.

\par
%\textsuperscript{(512.4)}
\textsuperscript{45:3.4} 3. \textit{El segundo Soberano asistente} ---Sadib, número 271.402 de la orden terciaria. Sadib vino también a Satania con Lanaforge.

\par
%\textsuperscript{(512.5)}
\textsuperscript{45:3.5} 4. \textit{El guardián del sistema} ---Holdant, número 19 del cuerpo terciario, el vigilante y controlador de todos los espíritus internados que están por encima del tipo de existencia mortal. Holdant vino igualmente a Satania con Lanaforge.

\par
%\textsuperscript{(512.6)}
\textsuperscript{45:3.6} 5. \textit{El registrador sistémico} ---Vilton, secretario del ministerio Lanonandek de Satania, número 374 de la orden tercera. Vilton era miembro del grupo original de Lanaforge.

\par
%\textsuperscript{(512.7)}
\textsuperscript{45:3.7} 6. \textit{El director de la donación} ---Fortant, número 319.847 de las reservas de los Lanonandeks secundarios y director temporal de todas las actividades universales trasladadas a Jerusem desde la donación de Miguel en Urantia. Fortant ha formado parte del estado mayor de Lanaforge durante mil novecientos años del tiempo de Urantia.

\par
%\textsuperscript{(512.8)}
\textsuperscript{45:3.8} 7. \textit{El alto consejero} ---Hanavard, número
67 de los Hijos Lanonandeks primarios y miembro del cuerpo superior de consejeros y coordinadores universales. Actúa como presidente en funciones del consejo ejecutivo de Satania. Hanavard es el duodécimo de esta orden que sirve así en Jerusem desde la rebelión de Lucifer.

\par
%\textsuperscript{(512.9)}
\textsuperscript{45:3.9} Este grupo ejecutivo de siete Lanonandeks compone la administración de emergencia ampliada que se hizo necesaria debido a las exigencias de la rebelión de Lucifer. En Jerusem sólo hay tribunales menores, puesto que el sistema es la unidad administrativa, no judicial, pero la administración Lanonandek está apoyada por el consejo ejecutivo de Jerusem, el cuerpo asesor supremo de Satania. Este consejo está compuesto por doce miembros:

\par
%\textsuperscript{(512.10)}
\textsuperscript{45:3.10} 1. Hanavard, el presidente Lanonandek.

\par
%\textsuperscript{(512.11)}
\textsuperscript{45:3.11} 2. Lanaforge, el Soberano del Sistema.

\par
%\textsuperscript{(512.12)}
\textsuperscript{45:3.12} 3. Mansurotia, el primer Soberano asistente.

\par
%\textsuperscript{(512.13)}
\textsuperscript{45:3.13} 4. El jefe de los Melquisedeks de Satania.

\par
%\textsuperscript{(512.14)}
\textsuperscript{45:3.14} 5. El director en funciones de los Portadores de Vida de Satania.

\par
%\textsuperscript{(512.15)}
\textsuperscript{45:3.15} 6. El jefe de los finalitarios de Satania.

\par
%\textsuperscript{(512.16)}
\textsuperscript{45:3.16} 7. El Adán original de Satania, jefe supervisor de los Hijos Materiales.

\par
%\textsuperscript{(512.17)}
\textsuperscript{45:3.17} 8. El director de las huestes seráficas de Satania.

\par
%\textsuperscript{(512.18)}
\textsuperscript{45:3.18} 9. El jefe de los controladores físicos de Satania.

\par
%\textsuperscript{(512.19)}
\textsuperscript{45:3.19} 10. El director de los Supervisores del Poder Morontial del sistema.

\par
%\textsuperscript{(513.1)}
\textsuperscript{45:3.20} 11. El director en funciones de las criaturas intermedias del sistema.

\par
%\textsuperscript{(513.2)}
\textsuperscript{45:3.21} 12. El jefe en funciones del cuerpo de los mortales ascendentes.

\par
%\textsuperscript{(513.3)}
\textsuperscript{45:3.22} Este consejo elige periódicamente a tres miembros para que representen al sistema local en el consejo supremo de la sede del universo, pero esta representación se encuentra suspendida debido a la rebelión. Satania dispone ahora de un observador en la sede del universo local, pero desde la donación de Miguel, el sistema ha reanudado la elección de diez miembros para la legislatura de Edentia.

\section*{4. Los veinticuatro consejeros}
\par
%\textsuperscript{(513.4)}
\textsuperscript{45:4.1} En el centro de los siete círculos residenciales angélicos de Jerusem está situada la sede del consejo asesor de Urantia, los veinticuatro consejeros. Juan el Revelador los llamó los veinticuatro ancianos: <<Y alrededor del trono\footnote{\textit{Trono del juicio}: Ap 4:2.} había veinticuatro asientos, y en los asientos vi a veinticuatro ancianos sentados, cubiertos con vestidos blancos>>\footnote{\textit{Veinticuatro ancianos}: Ap 4:4.}. El trono situado en el centro de este grupo es el tribunal del arcángel que preside, el trono desde el que se efectúa el llamamiento resurreccional de la misericordia y la justicia para toda Satania. Este tribunal ha estado siempre en Jerusem, pero los veinticuatro asientos que lo rodean fueron colocados en su sitio hace sólo mil novecientos años, poco después de que Cristo Miguel fuera elevado a la plena soberanía de Nebadon. Estos veinticuatro consejeros son sus agentes personales en Jerusem, y tienen autoridad para representar al Hijo Maestro en todos los asuntos relacionados con los llamamientos nominales de Satania y en otras muchas fases del programa de la ascensión de los mortales en los mundos aislados del sistema. Son los agentes que han sido designados para ejecutar las peticiones especiales de Gabriel y los mandatos inhabituales de Miguel.

\par
%\textsuperscript{(513.5)}
\textsuperscript{45:4.2} Estos veinticuatro consejeros han sido reclutados entre las ocho razas de Urantia, y los últimos de este grupo fueron convocados en la época del llamamiento nominal a la resurrección efectuado por Miguel hace mil novecientos años. Este consejo asesor de Urantia está compuesto por los miembros siguientes:

\par
%\textsuperscript{(513.6)}
\textsuperscript{45:4.3} 1. \textit{Onagar}, el pensador más importante de la era anterior al Príncipe Planetario, que dirigió a sus semejantes hacia la adoración del <<Dador del Aliento>>\footnote{\textit{Dador del Aliento}: Gn 2:7; Hch 17:25.}.

\par
%\textsuperscript{(513.7)}
\textsuperscript{45:4.4} 2. \textit{Mansant}, el gran educador de la era posterior al Príncipe Planetario en Urantia, que orientó a sus semejantes hacia la veneración de la <<Gran Luz>>\footnote{\textit{La Gran Luz}: Is 9:2; Is 60:1-3; Jn 8:12; Jn 12:35-36,46; 1 Jn 1:5.}.

\par
%\textsuperscript{(513.8)}
\textsuperscript{45:4.5} 3. \textit{Onamonalontón}, un antiguo jefe de los hombres rojos, el que dirigió a esta raza desde la adoración de muchos dioses hasta la veneración del <<Gran Espíritu>>\footnote{\textit{El Gran Espíritu}: Jn 4:24; 2 Co 3:17.}.

\par
%\textsuperscript{(513.9)}
\textsuperscript{45:4.6} 4. \textit{Orlandof}, un príncipe de los hombres azules que los condujo a reconocer la divinidad del <<Jefe Supremo>>\footnote{\textit{El Jefe Supremo}: 1 P 5:4.}.

\par
%\textsuperscript{(513.10)}
\textsuperscript{45:4.7} 5. \textit{Porshunta}, el oráculo de la extinta raza anaranjada que guió a este pueblo hacia la adoración del <<Gran Educador>>\footnote{\textit{El Gran Educador}: Jn 3:2.}.

\par
%\textsuperscript{(513.11)}
\textsuperscript{45:4.8} 6. \textit{Singlangtón}, el primer hombre amarillo que enseñó y dirigió a su pueblo hacia la adoración de la <<Verdad Única>>\footnote{\textit{Verdad única}: Jn 14:6.} en lugar de múltiples verdades. Hace miles de años, los hombres amarillos ya conocían al Dios único\footnote{\textit{Dios único}: 2 Re 19:19; 1 Cr 17:20; Neh 9:6; Sal 86:10; Eclo 36:5; Is 37:16; 44:6,8; 45:5-6,21; Dt 4:35,39; 6:4; Mc 12:29,32; Jn 17:3; Ro 3:30; 1 Co 8:4-6; Gl 3:20; Ef 4:6; 1 Ti 2:5; Stg 2:19; 1 Sam 2:2; 2 Sam 7:22.}.

\par
%\textsuperscript{(513.12)}
\textsuperscript{45:4.9} 7. \textit{Fantad}, el que liberó a los hombres verdes de las tinieblas y los condujo a la adoración de la <<Única Fuente de la Vida>>\footnote{\textit{Única Fuente de la Vida}: Gn 1:1,30; 2:7; Jn 1:1-4.}.

\par
%\textsuperscript{(513.13)}
\textsuperscript{45:4.10} 8. \textit{Orvonón}, el que iluminó a las razas de color índigo y las dirigió hacia el antiguo servicio del <<Dios de los Dioses>>\footnote{\textit{Dios de los Dioses}: Sal 136:2; Dn 2:47; 11:36; Dt 10:17; Jos 22:22.}.

\par
%\textsuperscript{(514.1)}
\textsuperscript{45:4.11} 9. \textit{Adán}, el padre planetario de Urantia, desacreditado pero rehabilitado, un Hijo Material de Dios que fue degradado a la similitud de la carne mortal, pero que sobrevivió y fue elevado posteriormente a esta posición por decreto de Miguel.

\par
%\textsuperscript{(514.2)}
\textsuperscript{45:4.12} 10. \textit{Eva}, la madre de la raza violeta de Urantia, que sufrió el castigo de la falta con su compañero y que fue también rehabilitada con él y designada para servir con este grupo de supervivientes mortales.

\par
%\textsuperscript{(514.3)}
\textsuperscript{45:4.13} 11. \textit{Enoc}, el primer mortal de Urantia que fusionó con su Ajustador del Pensamiento durante su vida humana en la carne\footnote{\textit{El traslado de Enoc}: Gn 5:24; Heb 11:5.}.

\par
%\textsuperscript{(514.4)}
\textsuperscript{45:4.14} 12. \textit{Moisés}, el emancipador de un resto de la raza violeta sumergida y el que instigó el renacimiento de la adoración del Padre Universal bajo el nombre de <<el Dios de Israel>>\footnote{\textit{El Dios de Israel}: Ex 5:1.}.

\par
%\textsuperscript{(514.5)}
\textsuperscript{45:4.15} 13. \textit{Elías},\footnote{\textit{El traslado de Elías}: 2 Re 2:1,11.} un alma trasladada que alcanzó brillantes logros espirituales durante la era posterior al Hijo Material.

\par
%\textsuperscript{(514.6)}
\textsuperscript{45:4.16} 14. \textit{Maquiventa Melquisedek}\footnote{\textit{Maquiventa Melquisedek}: Gn 14:18ff; Sal 110:4; Heb 5:6,10; 6:20; 7:1-3,10,17,21; 7:21.}, el único hijo de esta orden que se ha donado a las razas de Urantia. Aunque figura todavía como un Melquisedek, se ha convertido <<para siempre en un ministro de los Altísimos>>, asumiendo eternamente la misión de servir como un ascendente mortal después de residir en Urantia en la similitud de la carne mortal, en Salem, en los tiempos de Abraham. Este Melquisedek ha sido proclamado recientemente Príncipe Planetario vicegerente de Urantia con sede en Jerusem y con autoridad para actuar en nombre de Miguel, que es realmente el Príncipe Planetario del mundo donde efectuó su donación final en forma humana. A pesar de todo esto, Urantia sigue estando supervisada por los gobernadores generales residentes sucesivos, miembros de los veinticuatro consejeros.

\par
%\textsuperscript{(514.7)}
\textsuperscript{45:4.17} 15. \textit{Juan el Bautista}\footnote{\textit{Juan el Bautista}: Jn 1:6-8.}, el precursor de la misión de Miguel en Urantia, y primo lejano del Hijo del Hombre en la carne.

\par
%\textsuperscript{(514.8)}
\textsuperscript{45:4.18} 16. \textit{1-2-3 el Primero}, el jefe de las criaturas intermedias leales al servicio de Gabriel en la época de la traición de Caligastia, elevado a esta posición por Miguel poco después de que éste obtuviera la soberanía incondicional.

\par
%\textsuperscript{(514.9)}
\textsuperscript{45:4.19} A petición de Gabriel, estas personalidades escogidas están exentas por ahora del régimen de la ascensión, y no tenemos ni idea de cuánto tiempo servirán en esta tarea.

\par
%\textsuperscript{(514.10)}
\textsuperscript{45:4.20} Los asientos número 17, 18, 19 y 20 no están ocupados de manera permanente. Están ocupados temporalmente por consentimiento unánime de los dieciséis miembros permanentes, conservándose vacantes para su asignación ulterior a los mortales ascendentes de la era actual, la era posterior al Hijo donador en Urantia.

\par
%\textsuperscript{(514.11)}
\textsuperscript{45:4.21} Los números 21, 22, 23 y 24 también están ocupados temporalmente, mientras se mantienen en reserva para los grandes educadores de otras eras posteriores que seguirán sin duda a la era actual. En Urantia se debe prever que llegarán las eras de los Hijos Magistrales, los Hijos Instructores y las eras de luz y de vida, independientemente de las visitas inesperadas de los Hijos divinos que puedan o no tener lugar.

\section*{5. Los Hijos Materiales}
\par
%\textsuperscript{(514.12)}
\textsuperscript{45:5.1} Las grandes divisiones de la vida celestial tienen sus sedes y sus inmensas reservas en Jerusem, incluyendo a las diversas órdenes de Hijos divinos, espíritus elevados, superángeles, ángeles y criaturas intermedias. La morada central de este maravilloso sector es el templo principal de los Hijos Materiales.

\par
%\textsuperscript{(515.1)}
\textsuperscript{45:5.2} La zona de los Adanes es el centro de atracción para todos los que llegan de nuevo a Jerusem. Es una región enorme compuesta de mil centros, aunque cada familia de Hijos e Hijas Materiales vive en una residencia propia hasta el momento en que sus miembros parten para servir en los mundos evolutivos del espacio, o hasta que emprenden la carrera de la ascensión hacia el Paraíso.

\par
%\textsuperscript{(515.2)}
\textsuperscript{45:5.3} Estos Hijos Materiales representan el tipo más elevado de seres que se reproducen sexualmente y que se encuentran en las esferas educativas de los universos en evolución. Y son realmente materiales; incluso los Adanes y las Evas Planetarios son claramente visibles para las razas mortales de los mundos habitados. Estos Hijos Materiales son el último eslabón físico de la cadena de personalidades que se extiende desde la divinidad y la perfección de arriba hasta la humanidad y la existencia material de abajo. Estos Hijos proporcionan a los mundos habitados un intermediario, con quien pueden contactar mutuamente, entre el Príncipe Planetario invisible y las criaturas materiales de los reinos.

\par
%\textsuperscript{(515.3)}
\textsuperscript{45:5.4} En el último registro milenario de Salvington había constancia en Nebadon de 161.432.840 Hijos e Hijas Materiales con categoría de ciudadanos en las capitales de los sistemas locales. El número de Hijos Materiales varía en los distintos sistemas, y su número crece constantemente por reproducción natural. En el ejercicio de sus funciones reproductoras, no se guían totalmente por los deseos personales de las personalidades que tienen estas relaciones, sino también por los cuerpos gobernantes y los consejos asesores superiores.

\par
%\textsuperscript{(515.4)}
\textsuperscript{45:5.5} Estos Hijos e Hijas Materiales son los habitantes permanentes de Jerusem y de sus mundos asociados. Ocupan inmensos conjuntos residenciales en Jerusem y participan ampliamente en la dirección local de la esfera capital, administrando prácticamente todos los asuntos rutinarios con la ayuda de los intermedios y de los ascendentes.

\par
%\textsuperscript{(515.5)}
\textsuperscript{45:5.6} En Jerusem, estos Hijos que se reproducen tienen permiso para experimentar con los ideales de un gobierno autónomo a la manera de los Melquisedeks, y están consiguiendo un tipo muy elevado de sociedad. Las órdenes superiores de filiación se reservan el derecho de veto en el reino, pero en casi todos los aspectos, los adamitas de Jerusem se gobiernan por sufragio universal y mediante un gobierno representativo. Esperan que algún día les concedan una autonomía prácticamente completa.

\par
%\textsuperscript{(515.6)}
\textsuperscript{45:5.7} El carácter del servicio de los Hijos Materiales está determinado en gran parte por la edad. Aunque no cumplen con los requisitos para ser admitidos en la Universidad Melquisedek de Salvington ---pues son materiales y están generalmente limitados a ciertos planetas--- sin embargo, los Melquisedeks mantienen grandes facultades de profesores en la sede de cada sistema para instruir a las generaciones más jóvenes de Hijos Materiales. El alcance, la técnica y la viabilidad de los sistemas de formación educativos y espirituales ofrecidos para el desarrollo de los Hijos y las Hijas Materiales más jóvenes representan el apogeo de la perfección.

\section*{6. La educación adámica de los ascendentes}
\par
%\textsuperscript{(515.7)}
\textsuperscript{45:6.1} Los Hijos y las Hijas Materiales, junto con sus hijos, presentan un espectáculo atractivo que nunca deja de despertar la curiosidad y de atraer la atención de todos los mortales ascendentes. Son tan similares a vuestras propias razas sexuadas materiales que los dos encontráis mucho interés común en compartir vuestros pensamientos y en ocupar vuestro tiempo en contactos fraternales.

\par
%\textsuperscript{(515.8)}
\textsuperscript{45:6.2} Los supervivientes mortales pasan una gran parte de su tiempo libre en la capital del sistema observando y estudiando los hábitos de vida y la conducta de estas criaturas sexuadas semifísicas superiores, pues estos ciudadanos de Jerusem son los padrinos y los mentores directos de los supervivientes mortales desde el momento en que consiguen la ciudadanía en el mundo sede hasta que se despiden para dirigirse a Edentia.

\par
%\textsuperscript{(516.1)}
\textsuperscript{45:6.3} En los siete mundos de las mansiones, a los mortales ascendentes se les proporcionan amplias oportunidades para compensar todas las privaciones experienciales sufridas en sus mundos de origen, ya sean debidas a la herencia, al entorno o a un desafortunado fin prematuro de su carrera en la carne. Esto es así en todos los sentidos, salvo en lo que se refiere a la vida sexual humana y a los ajustes que la acompañan. Miles de mortales llegan a los mundos de las mansiones sin haberse beneficiado particularmente de las disciplinas derivadas de unas relaciones sexuales comunes y corrientes en sus esferas nativas. La experiencia de los mundos de las mansiones puede proporcionar pocas oportunidades para compensar estas privaciones tan personales. La experiencia sexual, en el sentido físico, pertenece al pasado para estos ascendentes, pero en estrecha asociación con los Hijos y las Hijas Materiales, como individuos y como miembros de sus familias, estos mortales sexualmente deficientes pueden compensar los aspectos sociales, intelectuales, emocionales y espirituales de sus deficiencias. Así pues, a todos aquellos humanos a quienes las circunstancias o el juicio erróneo los privaron de los beneficios de una asociación sexual ventajosa en los mundos evolutivos, aquí en las capitales de los sistemas se les proporcionan todas las oportunidades para adquirir estas experiencias humanas esenciales en estrecha y afectuosa asociación con las criaturas sexuadas adámicas celestiales que residen de forma permanente en las capitales de los sistemas.

\par
%\textsuperscript{(516.2)}
\textsuperscript{45:6.4} Ningún mortal sobreviviente, ningún intermedio o serafín puede ascender al Paraíso, alcanzar al Padre y ser enrolado en el Cuerpo de la Finalidad sin haber pasado por la sublime experiencia de establecer una relación parental con un hijo evolutivo de los mundos, o haber pasado por alguna otra experiencia análoga y equivalente. La relación entre padres e hijos es fundamental para comprender el concepto esencial del Padre Universal y sus hijos del universo. Por eso esta experiencia es indispensable en la formación experiencial de todos los ascendentes.

\par
%\textsuperscript{(516.3)}
\textsuperscript{45:6.5} Las criaturas intermedias ascendentes y los serafines evolutivos deben pasar por esta experiencia parental en asociación con los Hijos y las Hijas Materiales de la sede del sistema. Estos ascendentes que no se reproducen adquieren así la experiencia parental ayudando a los Adanes y las Evas de Jerusem a criar y educar a su progenie.

\par
%\textsuperscript{(516.4)}
\textsuperscript{45:6.6} Todos los supervivientes mortales que no han experimentado la paternidad en los mundos evolutivos también deben adquirir esta formación necesaria mientras residen en los hogares de los Hijos Materiales de Jerusem como asociados parentales de estos magníficos padres y madres. Esto es así, salvo en la medida en que dichos mortales hayan sido capaces de compensar sus deficiencias en la guardería infantil del sistema, situada en el primer mundo de cultura de transición de Jerusem.

\par
%\textsuperscript{(516.5)}
\textsuperscript{45:6.7} Ciertas personalidades morontiales mantienen esta guardería infantil probatoria de Satania en el mundo de los finalitarios, donde una mitad del planeta está dedicada a esta tarea de criar a los niños. Aquí se reciben y se reensamblan ciertos hijos de los mortales supervivientes tales como aquellos descendientes que fallecieron en los mundos evolutivos antes de adquirir un estado espiritual como individuos. La ascensión de cualquiera de sus padres naturales asegura que a este hijo mortal de los reinos se le concederá la repersonalización en el planeta finalitario del sistema y allí se le permitirá demostrar, mediante su libre elección posterior, si escoge o no seguir el camino parental de la ascensión humana. Los niños aparecen aquí como en su mundo de nacimiento, salvo que la diferenciación sexual está ausente. Después de la experiencia de la vida en los mundos habitados, ya no existe la reproducción de tipo humana.

\par
%\textsuperscript{(517.1)}
\textsuperscript{45:6.8} Los estudiantes de los mundos de las mansiones que tienen uno o más hijos en la guardería probatoria del mundo finalitario y que tienen deficiencias en su experiencia parental esencial, pueden solicitar un permiso a los Melquisedeks para interrumpir las tareas de la ascensión en los mundos de las mansiones y trasladarse temporalmente al mundo finalitario donde se les concede la oportunidad de actuar como padres asociados de sus propios hijos y de otros niños. Este servicio en forma de ministerio parental puede ser reconocido más tarde en Jerusem, considerándose que estos ascendentes han efectuado la mitad del aprendizaje que necesitan realizar en las familias de los Hijos y las Hijas Materiales.

\par
%\textsuperscript{(517.2)}
\textsuperscript{45:6.9} La guardería probatoria misma está supervisada por mil parejas de Hijos e Hijas Materiales, voluntarios de la colonia de su orden en Jerusem. Reciben la ayuda directa de un número casi igual de grupos parentales midsonitos voluntarios que se detienen aquí para prestar este servicio en su camino desde el mundo midsonito de Satania hasta su destino no revelado en los mundos especiales reservados para ellos entre las esferas finalitarias de Salvington.

\section*{7. Las escuelas Melquisedeks}
\par
%\textsuperscript{(517.3)}
\textsuperscript{45:7.1} Los Melquisedeks son los directores de ese numeroso cuerpo de instructores ---criaturas volitivas y otras, parcialmente espiritualizadas--- que ejercen su actividad de manera tan aceptable en Jerusem y en sus mundos asociados, pero especialmente en los siete mundos de las mansiones. En estos planetas es donde se detienen aquellos mortales que no logran fusionar con su Ajustador interior durante la vida en la carne, y son reconstruídos aquí con una forma transitoria para recibir una ayuda adicional y disfrutar de amplias oportunidades para continuar sus esfuerzos por alcanzar sus objetivos espirituales, los mismos esfuerzos que fueron interrumpidos prematuramente por la muerte. O si por alguna otra razón de impedimento hereditario, de entorno desfavorable o de confabulación de circunstancias este logro del alma no se consiguió, cualquiera que sea la razón, todos los que tienen un propósito sincero y son dignos en espíritu se encontrarán presentes, tal como son, en los planetas de continuación, donde deberán aprender a dominar los elementos esenciales de la carrera eterna y a conseguir las características que no pudieron adquirir, o no adquirieron, durante su vida en la carne.

\par
%\textsuperscript{(517.4)}
\textsuperscript{45:7.2} Las Brillantes Estrellas Vespertinas (y sus coordinados innominados) sirven con frecuencia como instructores en las diversas empresas educativas del universo, incluyendo aquellas que están patrocinadas por los Melquisedeks. Los Hijos Instructores Trinitarios también colaboran, e imparten los toques de la perfección del Paraíso en estas escuelas de formación progresiva. Pero todas estas actividades no están dedicadas exclusivamente al progreso de los mortales ascendentes; muchas de ellas se ocupan igualmente de la formación progresiva de las personalidades espirituales nativas de Nebadon.

\par
%\textsuperscript{(517.5)}
\textsuperscript{45:7.3} Los Hijos Melquisedeks dirigen más de treinta centros educativos diferentes en Jerusem. Estas escuelas formativas empiezan con el colegio de la autoevaluación y terminan con las escuelas de la ciudadanía en Jerusem, donde los Hijos y las Hijas Materiales se unen a los Melquisedeks y a otros seres en su esfuerzo supremo por capacitar a los supervivientes mortales para que asuman las altas responsabilidades del gobierno representativo. Todo el universo está organizado y administrado en el plano \textit{representativo}. Entre los seres no perfectos, el gobierno representativo es el ideal divino del gobierno autónomo.

\par
%\textsuperscript{(517.6)}
\textsuperscript{45:7.4} Cada cien años del tiempo del universo, cada sistema elige a sus diez representantes para que ocupen sus escaños en la legislatura de la constelación. Son escogidos por el consejo de los mil de Jerusem, un cuerpo electoral encargado del deber de representar a los grupos del sistema en todas estas materias delegadas o que se cubren por nombramiento. Todos los representantes u otros delegados son elegidos por el consejo de los mil electores, y deben ser diplomados de la escuela superior del Colegio de Administración Melquisedek, como lo son también todos aquellos que componen este grupo de mil electores. Los Melquisedeks patrocinan esta escuela, ayudados últimamente por los finalitarios.

\par
%\textsuperscript{(518.1)}
\textsuperscript{45:7.5} Hay muchos cuerpos electivos en Jerusem, y de vez en cuando son elegidos para ejercer su autoridad por tres órdenes de ciudadanía ---los Hijos y las Hijas Materiales, los serafines y sus asociados, incluyendo a las criaturas intermedias, y los mortales ascendentes. Para recibir el honor de ser nombrado representante, un candidato debe haber conseguido el reconocimiento necesario en las escuelas de administración Melquisedek.

\par
%\textsuperscript{(518.2)}
\textsuperscript{45:7.6} El sufragio es universal en Jerusem entre estos tres grupos de ciudadanos, pero el voto se emite de forma diferencial de acuerdo con la posesión personal en mota ---en sabiduría morontial--- debidamente reconocida y registrada. El voto emitido por cualquier personalidad en una elección de Jerusem tiene un valor que va desde uno hasta mil. Los ciudadanos de Jerusem están pues clasificados según sus logros en mota.

\par
%\textsuperscript{(518.3)}
\textsuperscript{45:7.7} Los ciudadanos de Jerusem se presentan de vez en cuando ante los examinadores Melquisedeks, los cuales certifican sus logros en sabiduría morontial. Luego se presentan ante el cuerpo examinador de las Brillantes Estrellas Vespertinas o sus delegados, que comprueban su grado de perspicacia espiritual. A continuación aparecen en presencia de los veinticuatro consejeros y sus asociados, que juzgan el nivel de sus logros experienciales en vida social. Estos tres factores se llevan después a los registradores de ciudadanía del gobierno representativo, que calculan rápidamente el nivel de mota y asignan las aptitudes para el sufragio de acuerdo con dicho nivel.

\par
%\textsuperscript{(518.4)}
\textsuperscript{45:7.8} Bajo la supervisión de los Melquisedeks, los Hijos Materiales se encargan de los mortales ascendentes, especialmente de aquellos que son lentos en unificar su personalidad en los nuevos niveles morontiales, y les proporcionan una formación intensiva destinada a rectificar dichas deficiencias. Ningún mortal ascendente deja la sede del sistema para emprender la carrera más extensa y variada de adaptación a la vida social en la constelación hasta que estos Hijos Materiales no han certificado los logros conseguidos en mota por su personalidad ---una individualidad que combina la existencia humana consumada en asociación experiencial con la carrera morontial en ciernes, estando las dos debidamente armonizadas gracias al supercontrol espiritual del Ajustador del Pensamiento.

\par
%\textsuperscript{(518.5)}
\textsuperscript{45:7.9} [Presentado por un Melquisedek destinado temporalmente en Urantia.]


\chapter{Documento 46. La sede del sistema local}
\par
%\textsuperscript{(519.1)}
\textsuperscript{46:0.1} JERUSEM, la sede de Satania, es una capital de tipo medio de un sistema local, y aparte de las numerosas irregularidades ocasionadas por la rebelión de Lucifer y la donación de Miguel en Urantia, es una esfera típica como las otras similares. Vuestro sistema local ha pasado por algunas experiencias borrascosas, pero en la actualidad está administrado de manera muy eficaz, y a medida que transcurren las eras, los resultados de la falta de armonía se están erradicando de manera lenta pero segura. El orden y la buena voluntad se están restableciendo, y las condiciones en Jerusem se acercan cada vez más al estado celestial de vuestras tradiciones, pues la sede del sistema es en verdad el cielo que imagina la mayoría de los creyentes religiosos del siglo veinte.

\section*{1. Los aspectos físicos de Jerusem}
\par
%\textsuperscript{(519.2)}
\textsuperscript{46:1.1} Jerusem está dividida en mil sectores latitudinales y diez mil zonas longitudinales. La esfera tiene siete capitales mayores y setenta centros administrativos menores. Las siete capitales regionales se ocupan de diversas actividades, y el Soberano del Sistema visita cada una de ellas al menos una vez al año.

\par
%\textsuperscript{(519.3)}
\textsuperscript{46:1.2} El kilómetro estándar de Jerusem equivale aproximadamente a once kilómetros de Urantia. El peso estándar, el <<gradant>>, se ha elaborado mediante el sistema decimal partiendo del ultimatón maduro, y representa unos doscientos ochenta gramos de vuestro peso. El día de Satania equivale a tres días del tiempo de Urantia, menos una hora, cuatro minutos y quince segundos, siendo ésta la duración de la rotación axial de Jerusem. El año del sistema consta de cien días de Jerusem. La hora del sistema es transmitida por los maestros cronoldeks.

\par
%\textsuperscript{(519.4)}
\textsuperscript{46:1.3} La energía de Jerusem está magníficamente controlada y circula alrededor de la esfera por los canales longitudinales, los cuales están directamente alimentados por las cargas energéticas del espacio y expertamente administrados por los Controladores Físicos Maestros. La resistencia natural al paso de estas energías por los canales físicos de conducción proporciona el calor necesario para producir la temperatura uniforme de Jerusem. La temperatura a plena luz se mantiene alrededor de los veintiún grados centígrados, mientras que durante el período de recesión de la luz cae un poco por debajo de los diez grados.

\par
%\textsuperscript{(519.5)}
\textsuperscript{46:1.4} El sistema de iluminación de Jerusem\footnote{\textit{Sistema de iluminación de Jerusem}: Is 60:19-20; Ap 21:23; 22:5.} no debería ser tan difícil de comprender por vosotros. No hay ni días ni noches, ni períodos de calor ni de frío. Los transformadores del poder mantienen cien mil centros desde donde las energías enrarecidas son proyectadas hacia arriba a través de la atmósfera planetaria, sufriendo ciertos cambios, hasta que alcanzan el techo eléctrico atmosférico de la esfera; entonces estas energías son reflejadas hacia abajo bajo la forma de una luz suave, tamizada y uniforme, con una intensidad parecida a la de la luz solar cuando el Sol brilla en el cielo a las diez de la mañana en Urantia.

\par
%\textsuperscript{(520.1)}
\textsuperscript{46:1.5} En estas condiciones de iluminación, los rayos luminosos no parecen proceder de un solo sitio; sencillamente se filtran a través del cielo, emanando por igual desde todas las direcciones del espacio. Esta luz es muy similar a la luz natural del Sol, salvo que contiene mucho menos calor. Así pues se podrá admitir que estos mundos sede no son luminosos en el espacio; si Jerusem estuviera muy cerca de Urantia, no sería visible.

\par
%\textsuperscript{(520.2)}
\textsuperscript{46:1.6} Los gases que reflejan esta energía luminosa desde la ionosfera superior de Jerusem hacia el suelo son muy similares a los de las zonas atmosféricas superiores de Urantia que están relacionados con los fenómenos de vuestras llamadas auroras boreales, aunque éstas se producen por causas diferentes. En Urantia, este mismo escudo gaseoso es el que impide que se escapen las ondas terrestres de transmisión, reflejándolas hacia la Tierra cuando chocan contra este cinturón gaseoso en su vuelo directo hacia el exterior. Las transmisiones son retenidas de esta manera cerca de la superficie mientras viajan por el aire alrededor de vuestro mundo.

\par
%\textsuperscript{(520.3)}
\textsuperscript{46:1.7} Esta iluminación de la esfera se mantiene de manera uniforme durante el setenta y cinco por ciento del día de Jerusem, y luego se produce una recesión gradual hasta que, en las horas de mínima iluminación, la luz se parece a la de vuestra Luna llena en una noche clara. Es el momento de la quietud para todo Jerusem. Únicamente las estaciones receptoras de las transmisiones siguen funcionando durante este período de descanso y de recuperación.

\par
%\textsuperscript{(520.4)}
\textsuperscript{46:1.8} Jerusem recibe una pálida luz de diversos soles cercanos ---una especie de brillante luz estelar--- pero no depende de ellos; los mundos como Jerusem no están sometidos a las vicisitudes de las perturbaciones solares, ni tampoco se enfrentan con el problema de un sol en vías de enfriarse o de morir.

\par
%\textsuperscript{(520.5)}
\textsuperscript{46:1.9} Los siete mundos educativos de transición y sus cuarenta y nueve satélites están calentados, iluminados, energizados y abastecidos de agua con la técnica que se utiliza en Jerusem.

\section*{2. Las características físicas de Jerusem}
\par
%\textsuperscript{(520.6)}
\textsuperscript{46:2.1} En Jerusem echaréis de menos las escarpadas cadenas montañosas de Urantia y de otros mundos surgidos por evolución, puesto que no hay ni terremotos ni lluvias, pero disfrutaréis de las hermosas tierras altas y de otras variaciones incomparables de la topografía y del paisaje. Inmensas extensiones de Jerusem se conservan en <<estado natural>>, y la grandiosidad de estas regiones sobrepasa por completo la capacidad de la imaginación humana.

\par
%\textsuperscript{(520.7)}
\textsuperscript{46:2.2} Hay miles y miles de pequeños lagos, pero ni ríos turbulentos ni extensos océanos. No hay lluvias, ni tormentas, ni ventiscas en ninguno de los mundos arquitectónicos, pero la condensación de la humedad produce una precipitación diaria durante las horas de menor temperatura que acompañan a la recesión de la luz. (El grado de rocío es más elevado en un mundo con tres gases que en un planeta con dos gases como Urantia). La vida física vegetal y el mundo morontial de criaturas vivientes necesitan humedad, pero ésta es ampliamente proporcionada por el sistema de circulación subterráneo que se extiende por toda la esfera e incluso hasta las cumbres mismas de las tierras altas. Este sistema hidráulico no es enteramente subterráneo, pues hay muchos canales que conectan entre sí a los lagos centelleantes de Jerusem.

\par
%\textsuperscript{(520.8)}
\textsuperscript{46:2.3} La atmósfera de Jerusem es una mezcla de tres gases. Este aire es muy similar al de Urantia, con la adición de un gas adaptado a la respiración del tipo de vida morontial. Este tercer gas no hace de ninguna manera que el aire sea inadecuado para la respiración de los animales o las plantas de las órdenes materiales.

\par
%\textsuperscript{(521.1)}
\textsuperscript{46:2.4} El sistema de transporte está ligado a los torrentes circulatorios por donde se mueven las energías, y estas corrientes energéticas principales están situadas a intervalos de dieciséis kilómetros. Ajustando sus mecanismos físicos, los seres materiales del planeta pueden desplazarse a una velocidad que varía entre trescientos y ochocientos kilómetros por hora. Las aves transportadoras vuelan a unos ciento sesenta kilómetros por hora. Los mecanismos aéreos de los Hijos Materiales viajan a unos ochocientos kilómetros por hora. Los seres materiales y los seres morontiales iniciales deben emplear estos medios mecánicos de transporte, pero las personalidades espirituales se desplazan utilizando su conexión con las fuerzas superiores y las fuentes espirituales de energía.

\par
%\textsuperscript{(521.2)}
\textsuperscript{46:2.5} Jerusem y sus mundos asociados están dotados de las diez divisiones normales de vida física, características de las esferas arquitectónicas de Nebadon. Y puesto que la evolución orgánica no existe en Jerusem, no hay formas competitivas de vida, ni lucha por la existencia, ni supervivencia de los más capacitados. Existe más bien una adaptación creativa que presagia la belleza, la armonía y la perfección de los mundos eternos del universo central y divino. Toda esta perfección creativa contiene la mezcla más asombrosa de vida física y de vida morontial, cuyos contrastes son resaltados artísticamente por los artesanos celestiales y sus compañeros.

\par
%\textsuperscript{(521.3)}
\textsuperscript{46:2.6} Jerusem es en verdad una anticipación de la gloria y de la grandiosidad paradisiacas. Pero nunca podréis esperar haceros una idea adecuada de estos gloriosos mundos arquitectónicos por medio de tentativas de descripción. Hay tan pocas cosas que se puedan comparar con las cosas de vuestro mundo, y aunque se pudiera, las cosas de Jerusem trascienden tanto a las cosas de Urantia, que la comparación es casi grotesca. Hasta que no lleguéis realmente a Jerusem, difícilmente podréis albergar algo que se parezca a un verdadero concepto de los mundos celestiales, pero no está tan lejos ese momento del futuro en el que vuestra experiencia venidera en la capital del sistema se podrá comparar con vuestra llegada algún día a las esferas educativas más distantes del universo, del superuniverso y de Havona.

\par
%\textsuperscript{(521.4)}
\textsuperscript{46:2.7} El sector industrial o de los laboratorios de Jerusem ocupa una extensa superficie, que los urantianos difícilmente reconocerían puesto que no tiene chimeneas humeantes; sin embargo, estos mundos especiales llevan asociada una compleja economía material, y la perfección de sus técnicas mecánicas y de sus logros físicos asombraría, e incluso pasmaría, a vuestros químicos e inventores más experimentados. Haced un alto y considerad que este primer mundo donde os detenéis en vuestro viaje hacia el Paraíso es mucho más material que espiritual. Durante toda vuestra estancia en Jerusem y sus mundos de transición, estáis mucho más cerca de vuestra vida terrestre y sus cosas materiales que de vuestra vida posterior con su existencia espiritual progresiva.

\par
%\textsuperscript{(521.5)}
\textsuperscript{46:2.8} El Monte Serafín es la cima más elevada de Jerusem, tiene unos cuatro mil seiscientos metros de altura, y es el punto de partida para todos los serafines transportadores. Se utilizan numerosos desarrollos mecánicos para proporcionar la energía inicial necesaria para escapar de la gravedad planetaria y vencer la resistencia del aire. Un transporte seráfico parte cada tres segundos del tiempo de Urantia durante todo el período diurno y, a veces, hasta mucho después de la recesión de la luz. Los transportadores despegan a unos veinticinco kilómetros estándar por segundo del tiempo de Urantia, y no alcanzan su velocidad normal hasta que no se encuentran a más de dos mil kilómetros de Jerusem.

\par
%\textsuperscript{(521.6)}
\textsuperscript{46:2.9} Los transportes llegan al campo de vidrio, al llamado mar de cristal\footnote{\textit{Mar de cristal}: Ap 4:6; 15:2.}. Alrededor de esta zona se encuentran las estaciones receptoras para las diversas órdenes de seres que atraviesan el espacio mediante el transporte seráfico. Cerca de la estación polar receptora de cristal, destinada a los visitantes estudiantiles, podéis subir al observatorio nacarado y ver el inmenso mapa en relieve de todo el planeta sede.

\section*{3. Las transmisiones de Jerusem}
\par
%\textsuperscript{(522.1)}
\textsuperscript{46:3.1} Las transmisiones del superuniverso y del Paraíso-Havona se reciben en Jerusem en coordinación con Salvington y por medio de una técnica en la que está implicado el vidrio polar, el mar de cristal. Además de los recursos para recibir estas comunicaciones procedentes del exterior de Nebadon, hay tres grupos distintos de estaciones receptoras. Estos grupos de estaciones, diferentes pero tricirculares, están adaptados para recibir las transmisiones procedentes de los mundos locales, de la sede de la constelación y de la capital del universo local. Todas estas transmisiones se visualizan automáticamente para que sean perceptibles para todos los tipos de seres presentes en el anfiteatro central de las transmisiones; de todas las ocupaciones de un mortal ascendente en Jerusem, ninguna es más atractiva y absorbente que la de escuchar el torrente sin fin de informes espaciales del universo.

\par
%\textsuperscript{(522.2)}
\textsuperscript{46:3.2} Esta estación receptora de transmisiones de Jerusem está rodeada por un enorme anfiteatro construido con materiales centelleantes, en su mayor parte desconocidos en Urantia, y con asientos para más de cinco mil millones de seres ---materiales y morontiales--- además de alojar a innumerables personalidades espirituales. La diversión favorita de todo Jerusem consiste en pasar su tiempo libre en la estación transmisora para conocer el bienestar y el estado del universo. Es la única actividad planetaria que no disminuye durante la recesión de la luz.

\par
%\textsuperscript{(522.3)}
\textsuperscript{46:3.3} Los mensajes de Salvington llegan continuamente a este anfiteatro receptor de transmisiones\footnote{\textit{Receptores de Jerusem}: Ap 11:19.}. Cerca de allí, las palabras de los Altísimos Padres de la Constelación se reciben al menos una vez al día procedentes de Edentia. Las transmisiones regulares y especiales de Uversa se difunden periódicamente a través de Salvington; cuando se reciben los mensajes del Paraíso, toda la población se reúne alrededor del mar de cristal, y los amigos de Uversa añaden el fenómeno de la reflectividad a la técnica de las transmisiones del Paraíso, de manera que todo lo que se escucha se puede ver. A los supervivientes mortales se les proporcionan de esta forma anticipaciones continuas de la belleza y de la grandiosidad progresivas, a medida que viajan en la aventura eterna hacia el interior.

\par
%\textsuperscript{(522.4)}
\textsuperscript{46:3.4} La estación emisora\footnote{\textit{Emisora de Jerusem}: Ap 4:5.} de Jerusem está situada en el polo opuesto de la esfera. Todas las transmisiones destinadas a los mundos individuales son enviadas desde las capitales de los sistemas, salvo los mensajes de Miguel, que a veces van directamente a su destino por el circuito de los arcángeles.

\section*{4. Las zonas residenciales y administrativas}
\par
%\textsuperscript{(522.5)}
\textsuperscript{46:4.1} Grandes partes de Jerusem están destinadas a zonas residenciales, mientras que otras partes de la capital del sistema están dedicadas a las funciones administrativas necesarias que se ocupan de la supervisión de los asuntos de 619 esferas habitadas, 56 mundos de cultura de transición y la capital misma del sistema. En Jerusem y en Nebadon, estas disposiciones están diseñadas como sigue:

\par
%\textsuperscript{(522.6)}
\textsuperscript{46:4.2} 1. \textit{Los círculos} ---las zonas residenciales para los no nativos.

\par
%\textsuperscript{(522.7)}
\textsuperscript{46:4.3} 2. \textit{Los cuadrados} ---las zonas administrativo-ejecutivas del sistema.

\par
%\textsuperscript{(522.8)}
\textsuperscript{46:4.4} 3. \textit{Los rectángulos} ---el lugar de reunión de la vida nativa inferior.

\par
%\textsuperscript{(522.9)}
\textsuperscript{46:4.5} 4. \textit{Los triángulos} ---las zonas administrativas locales o de Jerusem.

\par
%\textsuperscript{(522.10)}
\textsuperscript{46:4.6} Esta organización de las actividades del sistema en círculos, cuadrados, rectángulos y triángulos es común para todas las capitales sistémicas de Nebadon. En otro universo puede predominar una organización enteramente diferente. Estas cuestiones son determinadas por los diversos planes de los Hijos Creadores.

\par
%\textsuperscript{(523.1)}
\textsuperscript{46:4.7} Nuestra narración acerca de estas zonas residenciales y administrativas no tiene en cuenta las inmensas y hermosas propiedades de los Hijos Materiales de Dios, los ciudadanos permanentes de Jerusem, ni tampoco mencionamos otras numerosas órdenes fascinantes de criaturas espirituales y casi espirituales. Por ejemplo: Jerusem disfruta de los servicios eficaces de los espirongas, diseñados para ejercer su actividad en el sistema. Estos seres se dedican a un ministerio espiritual a favor de los residentes y visitantes supermateriales. Forman un grupo maravilloso de seres inteligentes y hermosos que son los servidores de transición de las criaturas morontiales superiores y de los ayudantes morontiales que trabajan para conservar y embellecer todas las creaciones morontiales. Significan para Jerusem lo que las criaturas intermedias significan para Urantia, unos ayudantes intermedios que desempeñan su actividad entre lo material y lo espiritual.

\par
%\textsuperscript{(523.2)}
\textsuperscript{46:4.8} Las capitales de los sistemas son únicas, en el sentido de que son los únicos mundos que muestran de una manera casi perfecta las tres fases de la existencia universal: la material, la morontial y la espiritual. Ya seáis una personalidad material, morontial o espiritual, os sentiréis como en casa en Jerusem; así se sienten también los seres combinados tales como las criaturas intermedias y los Hijos Materiales.

\par
%\textsuperscript{(523.3)}
\textsuperscript{46:4.9} Jerusem posee grandes edificios de tipo tanto material como morontial, aunque el embellecimiento de las zonas puramente espirituales es no menos exquisito y completo\footnote{\textit{La belleza de Jerusem}: Ap 21:2,10-11; 21:18-21.}. ¡Si tan sólo tuviera palabras para contaros las contrapartidas morontiales del maravilloso equipamiento físico de Jerusem! ¡Si tan sólo pudiera seguir describiendo la grandiosidad sublime y la exquisita perfección de los detalles espirituales de este mundo sede! Vuestro concepto más imaginativo sobre la perfección de la belleza y la plenitud de los detalles difícilmente se acercaría a este esplendor. Y Jerusem sólo es el primer paso en el camino hacia la perfección celestial de la belleza del Paraíso.

\section*{5. Los círculos de Jerusem}
\par
%\textsuperscript{(523.4)}
\textsuperscript{46:5.1} Las reservas residenciales asignadas a los grupos principales de vida universal se denominan los círculos de Jerusem. Estos grupos de círculos que se mencionan en estas narraciones son los siguientes:

\par
%\textsuperscript{(523.5)}
\textsuperscript{46:5.2} 1. Los círculos de los Hijos de Dios.

\par
%\textsuperscript{(523.6)}
\textsuperscript{46:5.3} 2. Los círculos de los ángeles y de los espíritus superiores.

\par
%\textsuperscript{(523.7)}
\textsuperscript{46:5.4} 3. Los círculos de los Ayudantes Universales, incluyendo a los hijos trinitizados por las criaturas no asignados a los Hijos Instructores Trinitarios.

\par
%\textsuperscript{(523.8)}
\textsuperscript{46:5.5} 4. Los círculos de los Controladores Físicos Maestros.

\par
%\textsuperscript{(523.9)}
\textsuperscript{46:5.6} 5. Los círculos de los mortales ascendentes asignados, incluyendo a las criaturas intermedias.

\par
%\textsuperscript{(523.10)}
\textsuperscript{46:5.7} 6. Los círculos de las colonias de cortesía.

\par
%\textsuperscript{(523.11)}
\textsuperscript{46:5.8} 7. Los círculos del Cuerpo de la Finalidad.

\par
%\textsuperscript{(523.12)}
\textsuperscript{46:5.9} Cada uno de estos agrupamientos residenciales consiste en siete círculos concéntricos sucesivamente elevados. Todos están construidos según el mismo estilo, pero tienen tamaños diferentes y están fabricados con materiales distintos. Todos están rodeados por recintos de gran alcance que se elevan hasta formar extensos paseos que envuelven por completo a cada grupo de siete círculos concéntricos.

\par
%\textsuperscript{(524.1)}
\textsuperscript{46:5.10} 1. \textit{Los círculos de los Hijos de Dios}. Aunque los Hijos de Dios poseen un planeta social propio, uno de los mundos de cultura de transición, también ocupan estas extensas zonas en Jerusem. En su mundo de cultura de transición, los ascendentes mortales se mezclan libremente con todas las órdenes de filiación divina. Allí conoceréis personalmente y amaréis a estos Hijos, pero su vida social está en gran parte limitada a este mundo especial y a sus satélites. Sin embargo, en los círculos de Jerusem se puede observar cómo trabajan estos diversos grupos de filiación. Y puesto que la vista morontial tiene un enorme alcance, podréis caminar por los paseos de los Hijos y observar las actividades fascinantes de sus numerosas órdenes.

\par
%\textsuperscript{(524.2)}
\textsuperscript{46:5.11} Estos siete círculos de los Hijos son concéntricos y están sucesivamente elevados, de manera que cada uno de los círculos exteriores más grandes domina los círculos interiores más pequeños, estando cada uno de ellos rodeado por un muro que sirve de paseo público. Estos muros están construidos con gemas cristalinas de un brillo centelleante y son tan elevados como para dominar todos los círculos residenciales respectivos. Las numerosas puertas ---entre cincuenta y ciento cincuenta mil--- que atraviesan cada uno de estos muros están hechas de un solo cristal nacarado.

\par
%\textsuperscript{(524.3)}
\textsuperscript{46:5.12} El primer círculo de la zona de los Hijos está ocupado por los Hijos Magistrales y sus estados mayores personales. Aquí están centrados todos los planes y todas las actividades inmediatas relacionadas con los servicios donadores y judiciales de estos Hijos jurídicos. Los Avonales del sistema también se mantienen en contacto con el universo a través de este centro.

\par
%\textsuperscript{(524.4)}
\textsuperscript{46:5.13} El segundo círculo está ocupado por los Hijos Instructores Trinitarios. En esta zona sagrada, los Daynales y sus asociados llevan adelante el entrenamiento de los Hijos Instructores primarios recién llegados. En todo este trabajo reciben la hábil ayuda de una división de ciertos coordinados de las Brillantes Estrellas Vespertinas. Los hijos trinitizados por las criaturas ocupan un sector del círculo de los Daynales. Los Hijos Instructores Trinitarios son los que están más cerca de ser los representantes personales del Padre Universal en un sistema local; al menos se trata de seres que tienen su origen en la Trinidad. Este segundo círculo es una zona de extraordinario interés para toda la población de Jerusem.

\par
%\textsuperscript{(524.5)}
\textsuperscript{46:5.14} El tercer círculo está dedicado a los Melquisedeks. Aquí residen los jefes sistémicos que supervisan las actividades casi sin fin de estos polifacéticos Hijos. Desde el primer mundo de las mansiones y durante toda la carrera de los mortales ascendentes en Jerusem, los Melquisedeks son sus padres adoptivos y sus consejeros siempre presentes. No sería inoportuno decir que son la influencia dominante en Jerusem, aparte de las actividades en todas partes presentes de los Hijos y las Hijas Materiales.

\par
%\textsuperscript{(524.6)}
\textsuperscript{46:5.15} El cuarto círculo es el hogar de los Vorondadeks y de todas las otras órdenes de Hijos visitantes y observadores que no se alojan en otra parte. Los Altísimos Padres de la Constelación establecen su residencia en este círculo durante sus visitas de inspección al sistema local. Los Perfeccionadores de la Sabiduría, los Consejeros Divinos y los Censores Universales residen todos en este círculo cuando están de servicio en el sistema.

\par
%\textsuperscript{(524.7)}
\textsuperscript{46:5.16} El quinto círculo es la morada de los Lanonandeks, la orden de filiación de los Soberanos Sistémicos y de los Príncipes Planetarios. Los tres grupos se mezclan en uno solo cuando residen en esta zona. Las reservas del sistema se encuentran en este círculo, mientras que el Soberano del Sistema tiene un templo situado en el centro del grupo de edificios gubernamentales en la colina de la administración.

\par
%\textsuperscript{(524.8)}
\textsuperscript{46:5.17} El sexto círculo es el lugar donde viven los Portadores de Vida del sistema. Todas las órdenes de estos Hijos se reúnen aquí, y salen de aquí hacia sus misiones en los mundos.

\par
%\textsuperscript{(524.9)}
\textsuperscript{46:5.18} El séptimo círculo es el punto de reunión de los hijos ascendentes, de aquellos mortales asignados que pueden estar trabajando temporalmente en la sede del sistema, junto con sus consortes seráficos. Todos los antiguos mortales con categoría superior a la de ciudadanos de Jerusem e inferior a la de finalitarios se considera que pertenecen al grupo que tiene su sede en este círculo.

\par
%\textsuperscript{(525.1)}
\textsuperscript{46:5.19} Estas reservas circulares de los Hijos ocupan una superficie enorme, y hasta hace mil novecientos años había un gran espacio libre en su centro. Esta región central está ocupada ahora por el monumento conmemorativo a Miguel, el cual se terminó hace unos quinientos años. Cuando este templo se inauguró hace cuatrocientos noventa y cinco años, Miguel estuvo presente en persona, y todo Jerusem escuchó la conmovedora historia de la donación del Hijo Maestro en Urantia, el planeta menos importante de Satania. El monumento a Miguel es actualmente el centro de todas las actividades integradas en la dirección del sistema, la cual ha sido modificada a consecuencia de la donación de Miguel, incluyendo la mayor parte de las actividades recientemente trasladadas desde Salvington. El personal del monumento conmemorativo asciende a más de un millón de personalidades.

\par
%\textsuperscript{(525.2)}
\textsuperscript{46:5.20} 2. \textit{Los círculos de los ángeles}. Al igual que la zona residencial de los Hijos, estos círculos de los ángeles constan de siete círculos concéntricos sucesivamente elevados, y cada uno de ellos tiene vista a las zonas interiores.

\par
%\textsuperscript{(525.3)}
\textsuperscript{46:5.21} El primer círculo de los ángeles está ocupado por las Personalidades Superiores del Espíritu Infinito que pueden estar estacionadas en el mundo sede ---los Mensajeros Solitarios y sus asociados. El segundo círculo está dedicado a las huestes de mensajeros, Asesores Técnicos, compañeros, inspectores y registradores que puedan estar trabajando de vez en cuando en Jerusem. El tercer círculo pertenece a los espíritus ministrantes de las órdenes y las agrupaciones superiores.

\par
%\textsuperscript{(525.4)}
\textsuperscript{46:5.22} El cuarto círculo está ocupado por los serafines administradores, y los serafines que sirven en un sistema local como Satania forman una <<hueste innumerable de ángeles>>\footnote{\textit{Hueste innumerable de ángeles}: Heb 12:22.}. El quinto círculo está ocupado por los serafines planetarios, mientras que el sexto es el hogar de los ministros de transición. El séptimo círculo es la esfera donde residen ciertas órdenes no reveladas de serafines. Los registradores de todos estos grupos de ángeles no viven con sus compañeros, estando domiciliados en el templo de los archivos de Jerusem. Todos los registros se conservan por triplicado en esta triple sala de archivos. En la sede de un sistema, los registros se conservan siempre bajo forma material, morontial y espiritual.

\par
%\textsuperscript{(525.5)}
\textsuperscript{46:5.23} Estos siete círculos están rodeados por la exposición panorámica de Jerusem, que tiene cinco mil kilómetros estándar de circunferencia, y está dedicada a presentar el estado progresivo de los mundos habitados de Satania; sufre constantes revisiones a fin de que represente realmente las condiciones actualizadas de los planetas individuales. No dudo de que este inmenso paseo que domina los círculos de los ángeles será el primer lugar de interés de Jerusem que atraerá vuestra atención cuando os permitan tener mucho tiempo libre durante vuestras primeras visitas.

\par
%\textsuperscript{(525.6)}
\textsuperscript{46:5.24} Estas exposiciones están a cargo de los nativos de Jerusem, pero reciben la ayuda de los ascendentes de los diversos mundos de Satania que se detienen en Jerusem camino de Edentia. La representación de las condiciones planetarias y del progreso de los mundos se lleva a cabo utilizando muchos métodos, algunos de ellos conocidos por vosotros, pero principalmente utilizando técnicas desconocidas en Urantia. Estas exposiciones ocupan el borde exterior de este inmenso muro. El resto del paseo está casi totalmente vacío, pero embellecido de una forma extremadamente magnífica.

\par
%\textsuperscript{(525.7)}
\textsuperscript{46:5.25} 3. \textit{Los círculos de los Ayudantes Universales} tienen situada la sede de las Estrellas Vespertinas en el enorme espacio central. Aquí se encuentra la sede sistémica de Galantia, el jefe asociado de este poderoso grupo de superángeles y el primero en entrar en servicio de todas las Estrellas Vespertinas ascendentes. Aunque se trata de una de las construcciones más recientes, es uno de los sectores administrativos más magníficos de Jerusem. Este centro tiene ochenta kilómetros de diámetro. La sede de Galantia es un cristal fundido monolítico, totalmente transparente. Tanto los seres morontiales como los seres materiales aprecian enormemente estos cristales morontio-materiales. Las Estrellas Vespertinas creadas ejercen su influencia sobre todo Jerusem, pues poseen esos atributos adicionales en su personalidad. Todo este mundo se ha llenado de una fragancia espiritual desde que muchas actividades suyas fueron transferidas aquí desde Salvington.

\par
%\textsuperscript{(526.1)}
\textsuperscript{46:5.26} 4. \textit{Los círculos de los Controladores Físicos Maestros}. Las diversas órdenes de Controladores Físicos Maestros están organizadas concéntricamente alrededor del inmenso templo de poder donde ejerce como presidente el jefe de poder del sistema en asociación con el jefe de los Supervisores del Poder Morontial. Este templo de poder es uno de los dos sectores de Jerusem donde no se permite la presencia de los mortales ascendentes ni de las criaturas intermedias. El otro es el sector de las desmaterializaciones en la zona de los Hijos Materiales, una serie de laboratorios donde los serafines transportadores transforman a los seres materiales en un estado totalmente semejante al de la orden morontial de existencia.

\par
%\textsuperscript{(526.2)}
\textsuperscript{46:5.27} 5. \textit{Los círculos de los mortales ascendentes}. La zona central de los círculos de los mortales ascendentes está ocupada por un grupo de 619 monumentos planetarios que representan a los mundos habitados del sistema, y estas estructuras sufren periódicamente grandes cambios. Los mortales de cada mundo tienen el privilegio de decidir, de vez en cuando, ciertas modificaciones o adiciones a realizar en sus monumentos planetarios. En la actualidad se siguen efectuando muchos cambios en las estructuras que representan a Urantia. El centro de estos 619 templos está ocupado por una maqueta de trabajo de Edentia y de sus numerosos mundos de cultura ascendente. Esta maqueta tiene unos sesenta y cinco kilómetros de diámetro y es una verdadera reproducción del sistema de Edentia, fiel al original en todos sus detalles.

\par
%\textsuperscript{(526.3)}
\textsuperscript{46:5.28} Los ascendentes disfrutan sirviendo en Jerusem y se complacen observando las técnicas que utilizan otros grupos. Todo lo que se hace en estos diversos círculos está abierto a la plena observación de todo Jerusem.

\par
%\textsuperscript{(526.4)}
\textsuperscript{46:5.29} Las actividades de un mundo como éste son de tres tipos distintos: trabajo, progreso y entretenimiento. Dicho de otra manera: servicio, estudio y distracción. Las actividades compuestas consisten en relaciones sociales, diversiones colectivas y adoración divina. El hecho de mezclarse con grupos distintos de personalidades, con órdenes muy diferentes a la de uno mismo, tiene un gran valor educativo.

\par
%\textsuperscript{(526.5)}
\textsuperscript{46:5.30} 6. \textit{Los círculos de las colonias de cortesía}. Los siete círculos de las colonias de cortesía están adornados con tres estructuras enormes: el vasto observatorio astronómico de Jerusem, la gigantesca galería de arte de Satania y el inmenso salón de actos de los directores de la reversión, el teatro de las actividades morontiales dedicadas al descanso y a la diversión.

\par
%\textsuperscript{(526.6)}
\textsuperscript{46:5.31} Los artesanos celestiales dirigen a los espornagias y aportan la multitud de adornos creativos y de monumentos conmemorativos que abundan en cada lugar de reunión pública. Los talleres de estos artesanos figuran entre los más grandes y los más hermosos de todos los edificios incomparables de este mundo maravilloso. Las otras colonias de cortesía tienen unas sedes amplias y hermosas. Muchos de estos edificios están totalmente construidos con gemas cristalinas. Todos los mundos arquitectónicos abundan en cristales y en metales llamados preciosos.

\par
%\textsuperscript{(527.1)}
\textsuperscript{46:5.32} 7. \textit{Los círculos de los finalitarios} tienen una estructura única en su centro. Un templo vacío de este mismo tipo se encuentra en cada mundo sede de todos los sistemas de Nebadon. Este edificio situado en Jerusem lleva el sello de la insignia de Miguel y posee la siguiente inscripción: <<No dedicado a la séptima fase del espíritu ---a la misión eterna>>. Gabriel colocó el sello en este templo misterioso, y nadie salvo Miguel puede romper el sello de la soberanía puesto por la Radiante Estrella Matutina. Algún día contemplaréis este templo silencioso, aunque no podáis descubrir su misterio.

\par
%\textsuperscript{(527.2)}
\textsuperscript{46:5.33} \textit{Otros círculos de Jerusem:} Además de estos círculos residenciales, en Jerusem hay numerosas moradas designadas adicionales.

\section*{6. Los cuadrados ejecutivo-administrativos}
\par
%\textsuperscript{(527.3)}
\textsuperscript{46:6.1} Las divisiones ejecutivo-administrativas del sistema están situadas en los inmensos cuadrados departamentales, mil en total. Cada unidad administrativa está dividida en cien subdivisiones de diez subgrupos cada una. Estos mil cuadrados están agrupados en diez grandes divisiones, formando así los diez departamentos administrativos siguientes:

\par
%\textsuperscript{(527.4)}
\textsuperscript{46:6.2} 1. Mantenimiento físico y mejoramiento material, el ámbito del poder y de la energía físicos.

\par
%\textsuperscript{(527.5)}
\textsuperscript{46:6.3} 2. Arbitraje, ética y juicio administrativo.

\par
%\textsuperscript{(527.6)}
\textsuperscript{46:6.4} 3. Asuntos planetarios y locales.

\par
%\textsuperscript{(527.7)}
\textsuperscript{46:6.5} 4. Asuntos de la constelación y del universo.

\par
%\textsuperscript{(527.8)}
\textsuperscript{46:6.6} 5. Educación y otras actividades de los Melquisedeks.

\par
%\textsuperscript{(527.9)}
\textsuperscript{46:6.7} 6. Progreso físico planetario y sistémico, los campos científicos de las actividades de Satania.

\par
%\textsuperscript{(527.10)}
\textsuperscript{46:6.8} 7. Asuntos morontiales.

\par
%\textsuperscript{(527.11)}
\textsuperscript{46:6.9} 8. Actividades y ética puramente espirituales.

\par
%\textsuperscript{(527.12)}
\textsuperscript{46:6.10} 9. Ministerio ascendente.

\par
%\textsuperscript{(527.13)}
\textsuperscript{46:6.11} 10. Filosofía del gran universo.

\par
%\textsuperscript{(527.14)}
\textsuperscript{46:6.12} Estas estructuras son transparentes; por eso todas las actividades del sistema pueden ser observadas incluso por los visitantes estudiantiles.

\section*{7. Los rectángulos ---los espornagias}
\par
%\textsuperscript{(527.15)}
\textsuperscript{46:7.1} Los mil \textit{rectángulos} de Jerusem están ocupados por la vida nativa inferior del planeta sede, y en su centro se encuentra situada la inmensa sede circular de los espornagias.

\par
%\textsuperscript{(527.16)}
\textsuperscript{46:7.2} En Jerusem os quedaréis asombrados con los logros agrícolas de los maravillosos espornagias. Allí, la tierra se cultiva principalmente con fines estéticos y decorativos. Los espornagias son los jardineros paisajistas de los mundos sede, y el tratamiento que dan a los espacios abiertos de Jerusem es a la vez original y artístico. Utilizan animales y numerosos dispositivos mecánicos para cultivar el suelo. Son unos expertos en el empleo inteligente de los agentes de poder de sus reinos, así como en la utilización de las numerosas órdenes de hermanos menores suyos pertenecientes a las creaciones animales inferiores, muchos de los cuales les son proporcionados en estos mundos especiales. Esta orden de vida animal está ahora dirigida en gran parte por las criaturas intermedias ascendentes que proceden de las esferas evolutivas.

\par
%\textsuperscript{(528.1)}
\textsuperscript{46:7.3} Los espornagias no están habitados por Ajustadores. No poseen almas que sobrevivan, pero disfrutan de una larga vida, a veces hasta llegar a los cuarenta o cincuenta mil años oficiales. Su número es enorme, y aportan su ministerio físico a todas las órdenes de personalidades universales que necesiten un servicio material.

\par
%\textsuperscript{(528.2)}
\textsuperscript{46:7.4} Aunque los espornagias no poseen ni desarrollan un alma que sobreviva, aunque no tienen una personalidad, sin embargo desarrollan una individualidad que puede experimentar la reencarnación. Cuando los cuerpos físicos de estas criaturas únicas se deterioran con el paso del tiempo debido al uso y a la edad, sus creadores, en colaboración con los Portadores de Vida, les fabrican unos nuevos cuerpos en los cuales los viejos espornagias vuelven a establecer su residencia.

\par
%\textsuperscript{(528.3)}
\textsuperscript{46:7.5} Los espornagias son las únicas criaturas de todo el universo de Nebadon que experimentan este tipo o cualquier otro tipo de reencarnación. Sólo reaccionan a los primeros cinco espíritus ayudantes de la mente; no son sensibles a los espíritus de adoración y de sabiduría. Pero la mente con cinco ayudantes equivale a una totalidad, es decir, al sexto nivel de realidad, y este factor es el que sobrevive como identidad experiencial.

\par
%\textsuperscript{(528.4)}
\textsuperscript{46:7.6} Al tratar de describir estas criaturas útiles y poco comunes, carezco por completo de comparaciones, pues en los mundos evolutivos no existen animales que puedan compararse con ellas. No son seres evolutivos, pues fueron proyectados por los Portadores de Vida con su forma y su estado actuales. Son bisexuales y procrean a medida que se necesitan para hacer frente a las necesidades de una población creciente.

\par
%\textsuperscript{(528.5)}
\textsuperscript{46:7.7} A las mentes de Urantia quizás yo les pueda sugerir algo mejor acerca de la naturaleza de estas hermosas y útiles criaturas, diciendoles que engloban las características combinadas de un caballo fiel y de un perro afectuoso, y que manifiestan una inteligencia que sobrepasa la de los tipos superiores de chimpancés. Y evaluándolas según los criterios físicos de Urantia, son muy hermosas. Aprecian mucho las atenciones que les manifiestan los residentes materiales y semimateriales de estos mundos arquitectónicos. Tienen una vista que les permite reconocer ---además de los seres materiales--- las creaciones morontiales, las órdenes angélicas inferiores, las criaturas intermedias y algunas órdenes inferiores de personalidades espirituales. No comprenden la adoración del Infinito, ni tampoco captan la importancia del Eterno, pero, por el afecto que les tienen a sus dueños, participan en las devociones espirituales exteriores de sus reinos.

\par
%\textsuperscript{(528.6)}
\textsuperscript{46:7.8} Algunos creen que en una era futura del universo, estos fieles espornagias se liberarán de su nivel de existencia animal y alcanzarán un digno destino evolutivo de crecimiento intelectual progresivo e incluso de logros espirituales.

\section*{8. Los triángulos de Jerusem}
\par
%\textsuperscript{(528.7)}
\textsuperscript{46:8.1} Los asuntos puramente locales y rutinarios de Jerusem están dirigidos desde los cien \textit{triángulos}. Estas unidades están agrupadas alrededor de las diez maravillosas estructuras que albergan la administración local de Jerusem. Los triángulos están rodeados por una representación panorámica de la historia de la sede sistémica. En la actualidad se encuentran borrados más de dos kilómetros estándar de esta historia circular. Este sector será restaurado cuando Satania sea readmitida en la familia de la constelación. Los decretos de Miguel lo han previsto todo para este acontecimiento, pero el tribunal de los Ancianos de los Días aún no ha terminado de evaluar los asuntos de la rebelión de Lucifer. Satania no puede volver a la plena comunidad de Norlatiadek mientras albergue archirrebeldes, esos elevados seres creados que han caído desde la luz a las tinieblas.

\par
%\textsuperscript{(529.1)}
\textsuperscript{46:8.2} Cuando Satania pueda regresar al redil de la constelación, entonces se propondrá para su estudio la readmisión de los mundos aislados en la familia sistémica de planetas habitados, acompañada de su restablecimiento en la comunión espiritual de los reinos. Pero aunque Urantia fuera restablecida en los circuitos del sistema, seguiríais estando en una situación incómoda por el hecho de que todo vuestro sistema permanece en la cuarentena de Norlatiadek, que lo aísla parcialmente de todos los otros sistemas.

\par
%\textsuperscript{(529.2)}
\textsuperscript{46:8.3} Pero antes de que transcurra mucho tiempo, el juicio de Lucifer y de sus asociados restablecerá al sistema de Satania en la constelación de Norlatiadek y, posteriormente, Urantia y las otras esferas aisladas serán reintegradas en los circuitos de Satania, y estos mundos disfrutarán de nuevo de los privilegios de las comunicaciones interplanetarias y de la comunión intersistémica.

\par
%\textsuperscript{(529.3)}
\textsuperscript{46:8.4} Los rebeldes y la rebelión tendrán un final. Los Gobernantes Supremos son misericordiosos y pacientes, pero la ley relacionada con el mal deliberadamente alimentado se ejecuta de manera universal e infalible. <<El pecado se paga con la muerte>>\footnote{\textit{La paga del pecado es la muerte}: Gn 2:17; Ro 6:23; Stg 1:15; Ap 21:8.} ---con la aniquilación eterna.

\par
%\textsuperscript{(529.4)}
\textsuperscript{46:8.5} [Presentado por un Arcángel de Nebadon.]


\chapter{Documento 47. Los siete mundos de las mansiones}
\par
%\textsuperscript{(530.1)}
\textsuperscript{47:0.1} CUANDO el Hijo Creador estuvo en Urantia, habló de las <<numerosas mansiones en el universo del Padre>>\footnote{\textit{Numerosas mansiones en el universo}: Jn 14:2.}. En cierto sentido, los cincuenta y seis mundos que rodean a Jerusem están dedicados a la cultura de transición de los mortales ascendentes, pero los siete satélites del mundo número uno se conocen más expresamente como los mundos de las mansiones.

\par
%\textsuperscript{(530.2)}
\textsuperscript{47:0.2} El mismo mundo de transición número uno está dedicado de manera exclusiva y por completo a las actividades ascendentes, y es la sede del cuerpo finalitario destinado en Satania. Este mundo sirve actualmente de sede para más de cien mil compañías de finalitarios, y en cada uno de estos grupos hay mil seres glorificados.

\par
%\textsuperscript{(530.3)}
\textsuperscript{47:0.3} Cuando un sistema está establecido en la luz y la vida, a medida que los mundos de las mansiones dejan de servir unos tras otros como lugares para instruir a los mortales, son ocupados por la población finalitaria creciente que se acumula en estos sistemas más antiguos y mucho más perfeccionados.

\par
%\textsuperscript{(530.4)}
\textsuperscript{47:0.4} Los siete mundos de las mansiones están a cargo de los supervisores morontiales y de los Melquisedeks. En cada mundo hay un gobernador en funciones que es directamente responsable ante los gobernantes de Jerusem. Los conciliadores de Uversa mantienen una sede en cada mundo de las mansiones, mientras que el punto de reunión local de los Asesores Técnicos se encuentra contiguo a ella. Los directores de la reversión y los artesanos celestiales mantienen una sede colectiva en cada uno de estos mundos. Los espirongas ejercen su actividad desde el mundo de las mansiones número dos en adelante, mientras que los siete, así como los otros planetas de cultura de transición y el mundo sede, están abundantemente provistos de espornagias del tipo normal.

\section*{1. El mundo de los finalitarios}
\par
%\textsuperscript{(530.5)}
\textsuperscript{47:1.1} Aunque en el mundo de transición número uno sólo residen los finalitarios y ciertos grupos de hijos salvados, así como sus cuidadores, se han tomado disposiciones para albergar a todas las clases de seres espirituales, de mortales de transición y de visitantes estudiantiles. Los espornagias, que ejercen su actividad en todos estos mundos, son los hospitalarios anfitriones de todos los seres que pueden reconocer. Tienen una vaga sensación con respecto a los finalitarios, pero no pueden verlos. Deben considerarlos poco más o menos como vosotros consideráis a los ángeles en vuestro estado físico actual.

\par
%\textsuperscript{(530.6)}
\textsuperscript{47:1.2} Aunque el mundo de los finalitarios es una esfera con una belleza física exquisita y un embellecimiento morontial extraordinario, la gran morada espiritual situada en el centro de las actividades, el templo de los finalitarios, no es perceptible sin ayuda para la vista material ni para la vista morontial inicial. Pero los transformadores de la energía son capaces de hacer visibles muchas de estas realidades a los mortales ascendentes, y de vez en cuando así lo hacen, como en los casos de las asambleas por clases de los estudiantes de los mundos de las mansiones en esta esfera cultural.

\par
%\textsuperscript{(531.1)}
\textsuperscript{47:1.3} Durante toda vuestra experiencia en los mundos de las mansiones, seréis en cierto modo espiritualmente conscientes de la presencia de vuestros hermanos glorificados que han alcanzado el Paraíso, pero es muy reconfortante percibirlos realmente de vez en cuando mientras ejercen sus actividades en las moradas de su sede. No veréis espontáneamente a los finalitarios hasta que no hayáis adquirido la verdadera visión espiritual.

\par
%\textsuperscript{(531.2)}
\textsuperscript{47:1.4} En el primer mundo de las mansiones, todos los supervivientes deben cumplir los requisitos que exige la comisión parental de sus planetas nativos. La comisión actual de Urantia está compuesta por doce parejas parentales, llegadas recientemente, que han pasado por la experiencia humana de criar a tres o más hijos hasta la edad de la pubertad. El servicio en esta comisión es rotativo y sólo se presta generalmente durante diez años. Todos aquellos cuya experiencia parental no logra satisfacer a estos comisionados, deben capacitarse posteriormente sirviendo en los hogares de los Hijos Materiales de Jerusem, o sirviendo en parte en la guardería probatoria del mundo finalitario.

\par
%\textsuperscript{(531.3)}
\textsuperscript{47:1.5} Pero sin tener en cuenta su experiencia parental, los padres de los mundos de las mansiones que tienen hijos creciendo en la guardería probatoria reciben todo tipo de oportunidades para colaborar con los guardianes morontiales de dichos niños en lo relacionado con su instrucción y formación. A estos padres se les permite viajar allí para visitarlos hasta cuatro veces al año. Observar a los padres de los mundos de las mansiones abrazar a sus descendientes materiales durante las ocasiones de sus peregrinaciones periódicas al mundo finalitario es una de las escenas más conmovedoramente hermosas de toda la carrera ascendente. Aunque uno de los padres, o los dos, pueden marcharse del mundo de las mansiones antes que el hijo, muy a menudo son contemporáneos durante una temporada.

\par
%\textsuperscript{(531.4)}
\textsuperscript{47:1.6} Ningún mortal ascendente puede eludir la experiencia de criar hijos ---los suyos o los de otros--- ya sea en los mundos materiales, o bien posteriormente en el mundo finalitario o en Jerusem. Los padres deben pasar por esta experiencia esencial tan ciertamente como las madres. La idea que tienen los pueblos modernos de Urantia de que criar a los hijos es una tarea que incumbe principalmente a las madres es una idea errónea y desacertada. Los niños necesitan a su padre tanto como a su madre, y los padres necesitan esta experiencia parental tanto como las madres.

\section*{2. La guardería probatoria}
\par
%\textsuperscript{(531.5)}
\textsuperscript{47:2.1} Las escuelas receptoras infantiles de Satania están situadas en el mundo finalitario, la primera esfera cultural de transición de Jerusem. Estas escuelas que reciben a los niños son unas empresas dedicadas a criar y educar a los hijos del tiempo, incluyendo a aquellos que han muerto en los mundos evolutivos del espacio antes de haber adquirido su condición de individuos en los registros del universo. En el caso de que uno o los dos padres de ese niño sobrevivan, el guardián del destino delega a su querubín asociado como custodio de la identidad potencial del niño, encargando al querubín la responsabilidad de poner ese alma no desarrollada en las manos de los Educadores de los Mundos de las Mansiones en las guarderías probatorias de los mundos morontiales.

\par
%\textsuperscript{(531.6)}
\textsuperscript{47:2.2} Estos mismos querubines abandonados son los que, como Educadores de los Mundos de las Mansiones, y bajo la supervisión de los Melquisedeks, mantienen estas extensas instalaciones educativas para instruir a los pupilos probatorios de los finalitarios. Estos pupilos de los finalitarios, estos hijos de los mortales ascendentes, siempre son personalizados en el estado físico exacto que tenían en el momento de morir, salvo en lo que se refiere a su potencial de reproducción. Este despertar se produce en el momento preciso en que llega uno de sus progenitores al primer mundo de las mansiones. Estos niños reciben entonces, tal como son, todo tipo de oportunidades para elegir el camino celestial, exactamente tal como podrían haber hecho esta elección en los mundos donde la muerte puso fin tan prematuramente a su carrera.

\par
%\textsuperscript{(532.1)}
\textsuperscript{47:2.3} En el mundo de la guardería, las criaturas a prueba se encuentran agrupadas según posean o no un Ajustador, pues los Ajustadores vienen a residir en estos niños materiales exactamente igual que en los mundos del tiempo. Los niños que no tienen edad para poseer un Ajustador son cuidados en familias de cinco, desde la edad de un año o menos hasta aproximadamente cinco años, la edad en que llega el Ajustador.

\par
%\textsuperscript{(532.2)}
\textsuperscript{47:2.4} Todos los niños de los mundos evolutivos que tienen su Ajustador del Pensamiento, pero que antes de morir no habían hecho su elección sobre la carrera hacia el Paraíso, también son repersonalizados en el mundo finalitario del sistema, donde crecen igualmente dentro de las familias de los Hijos Materiales y sus asociados, como lo hacen aquellos pequeños que llegaron sin Ajustador pero que recibirán posteriormente su Monitor de Misterio después de llegar a la edad necesaria para la elección moral.

\par
%\textsuperscript{(532.3)}
\textsuperscript{47:2.5} Los niños y los jóvenes habitados por un Ajustador que viven en el mundo finalitario son criados también en familias de cinco, y sus edades varían entre seis y catorce años; estas familias están compuestas, aproximadamente, por niños que tienen seis, ocho, diez, doce y catorce años. En cualquier momento después de los dieciséis años, si han efectuado su elección final, se trasladan al primer mundo de las mansiones y empiezan su ascensión hacia el Paraíso. Algunos hacen su elección antes de esta edad y van a las esferas de ascensión, pero en los mundos de las mansiones encontraréis muy pocos niños por debajo de los dieciséis años, tal como se calcula la edad según los criterios de Urantia.

\par
%\textsuperscript{(532.4)}
\textsuperscript{47:2.6} Los serafines guardianes se ocupan de estos jóvenes en la guardería probatoria del mundo finalitario exactamente de la misma manera que aportan su ministerio espiritual a los mortales en los planetas evolutivos, mientras que los fieles espornagias atienden sus necesidades físicas. Y estos niños crecen así en el mundo de transición hasta el momento en que efectúan su elección final.

\par
%\textsuperscript{(532.5)}
\textsuperscript{47:2.7} Cuando la vida material ha terminado su curso, si no han elegido la vida ascendente, o si estos hijos del tiempo han decidido definitivamente estar en contra de la aventura de Havona, la muerte pone fin automáticamente a su carrera de prueba. Estos casos no necesitan juicio; no existe resurrección para esta segunda muerte. Simplemente se vuelven como si no hubieran existido.

\par
%\textsuperscript{(532.6)}
\textsuperscript{47:2.8} Pero si eligen el camino paradisiaco de la perfección, se les prepara inmediatamente para trasladarlos al primer mundo de las mansiones, donde muchos de ellos llegan a tiempo para reunirse con sus padres en la ascensión hacia Havona. Después de pasar por Havona y de llegar hasta las Deidades, estas almas salvadas de origen mortal componen la ciudadanía ascendente permanente del Paraíso. Estos niños que han sido privados de la valiosa y esencial experiencia evolutiva en los mundos donde nacen los mortales no son enrolados en el Cuerpo de la Finalidad.

\section*{3. El primer mundo de las mansiones}
\par
%\textsuperscript{(532.7)}
\textsuperscript{47:3.1} En los mundos de las mansiones, los supervivientes mortales resucitados reanudan su vida exactamente donde la dejaron cuando la muerte les sorprendió\footnote{\textit{Resurrección}: Jn 5:28-29; 6:39-40; 11:24-26.}. Cuando vayáis desde Urantia al primer mundo de las mansiones, notaréis un cambio considerable, pero si vinierais de una esfera del tiempo más normal y progresiva, apenas notaríais la diferencia salvo por el hecho de que poseéis un cuerpo diferente; el tabernáculo de carne y hueso ha sido dejado atrás en el mundo de nacimiento.

\par
%\textsuperscript{(532.8)}
\textsuperscript{47:3.2} El verdadero centro de todas las actividades del primer mundo de las mansiones es la sala de resurrección, el enorme templo donde se ensamblan las personalidades. Esta estructura gigantesca es el punto de reunión central de los guardianes seráficos del destino, los Ajustadores del Pensamiento y los arcángeles de la resurrección. Los Portadores de Vida también trabajan con estos seres celestiales para resucitar a los muertos.

\par
%\textsuperscript{(533.1)}
\textsuperscript{47:3.3} Las transcripciones de la mente mortal y las configuraciones activas de la memoria de la criatura, tal como han sido transformadas desde los niveles materiales a los niveles espirituales, son propiedad individual de los Ajustadores del Pensamiento separados; estos factores espiritualizados de la mente, la memoria y la personalidad de la criatura forman parte para siempre de esos Ajustadores. La matriz mental de la criatura y los potenciales pasivos de su identidad están presentes en el alma morontial confiada al cuidado de los guardianes seráficos del destino. La reunión del alma morontial confiada a los serafines y de la mente espiritual confiada al Ajustador es lo que reensambla la personalidad de la criatura y constituye la resurrección de un superviviente dormido\footnote{\textit{Cambios de la resurrección}: Mt 27:52-53; Lc 14:14; 20:35-36; 1 Co 15:42-55.}.

\par
%\textsuperscript{(533.2)}
\textsuperscript{47:3.4} Si una personalidad transitoria de origen mortal no fuera nunca reensamblada de esta manera, los elementos espirituales de la criatura mortal no sobreviviente continuarían para siempre formando parte integrante de la dotación experiencial individual de su antiguo Ajustador interior.

\par
%\textsuperscript{(533.3)}
\textsuperscript{47:3.5} Desde el Templo de la Vida Nueva se extienden siete alas radiales, las salas de resurrección de las razas mortales. Cada una de estas estructuras está dedicada a ensamblar a una de las siete razas del tiempo. Cada una de estas siete alas contiene cien mil cámaras personales de resurrección, las cuales terminan en las salas circulares de ensamblaje por clases, que sirven como cámaras para despertar a no menos de un millón de individuos. Estas salas están rodeadas por las cámaras donde se ensambla la personalidad de las razas mezcladas de los mundos postadámicos normales. Cualquiera que sea la técnica que se pueda emplear en los mundos individuales del tiempo en los momentos de las resurrecciones especiales o dispensacionales, el verdadero reensamblaje consciente de una personalidad real y completa tiene lugar en las salas de resurrección de la mansonia número uno. Durante toda la eternidad recordaréis las profundas impresiones que habrá causado en vuestra memoria el haber presenciado por primera vez estas mañanas de resurrección.

\par
%\textsuperscript{(533.4)}
\textsuperscript{47:3.6} Desde las salas de resurrección os trasladáis al sector Melquisedek, donde os asignan una residencia permanente. Luego disponéis de diez días de libertad personal. Sois libres de explorar los alrededores inmediatos de vuestro nuevo hogar y de familiarizaros con el programa inminente que os espera. También tendréis tiempo para satisfacer vuestro deseo de consultar el registro y de visitar a vuestros seres queridos y a otros amigos terrestres que puedan haberos precedido en estos mundos. Al final de este período de diez días de tiempo libre empezáis la segunda etapa del viaje hacia el Paraíso, pues los mundos de las mansiones son auténticas esferas de formación, y no simplemente unos planetas donde os detenéis.

\par
%\textsuperscript{(533.5)}
\textsuperscript{47:3.7} En el mundo de las mansiones número uno (o en otro, en caso de poseer un estado más avanzado) reanudaréis vuestra educación intelectual y vuestro desarrollo espiritual en el nivel exacto en que fueron interrumpidos por la muerte. Entre el momento de la muerte planetaria, o traslado, y la resurrección en el mundo de las mansiones, el hombre mortal no gana absolutamente nada, aparte de experimentar el hecho de la supervivencia. Allí empezáis exactamente donde lo dejasteis aquí.

\par
%\textsuperscript{(533.6)}
\textsuperscript{47:3.8} Casi toda la experiencia en el mundo de las mansiones número uno está relacionada con la corrección de las deficiencias. Los supervivientes que llegan a esta primera esfera de detención presentan tantos y tan variados defectos en su carácter como criaturas y tantas deficiencias en su experiencia humana, que las actividades principales del reino consisten en corregir y curar estos múltiples legados de la vida en la carne en los mundos evolutivos materiales del tiempo y del espacio.

\par
%\textsuperscript{(534.1)}
\textsuperscript{47:3.9} La estancia en el mundo de las mansiones número uno está destinada a desarrollar a los supervivientes mortales al menos hasta el nivel de la dispensación postadámica de los mundos evolutivos normales. Espiritualmente, los estudiantes del mundo de las mansiones están por supuesto muy por encima de ese nivel de simple desarrollo humano.

\par
%\textsuperscript{(534.2)}
\textsuperscript{47:3.10} Si no tenéis que permanecer en el mundo de las mansiones número uno, al cabo de diez días entraréis en el sueño de traslado y os dirigiréis al mundo número dos, y después avanzaréis así cada diez días hasta que lleguéis al mundo de vuestro destino.

\par
%\textsuperscript{(534.3)}
\textsuperscript{47:3.11} El centro de los siete círculos principales de la administración del primer mundo de las mansiones está ocupado por el templo de los Compañeros Morontiales, los guías personales asignados a los mortales ascendentes. Estos compañeros son la progenie del Espíritu Madre del universo local, y hay varios millones de ellos en los mundos morontiales de Satania. Aparte de aquellos que están asignados como compañeros de grupo, tendréis mucho que ver con los intérpretes y traductores, los guardianes de los edificios y los supervisores de las excursiones. Todos estos compañeros cooperan activamente con aquellos que tienen que ver con el desarrollo de los factores mentales y espirituales de vuestra personalidad dentro del cuerpo morontial.

\par
%\textsuperscript{(534.4)}
\textsuperscript{47:3.12} Cuando empezáis en el primer mundo de las mansiones, un Compañero Morontial es asignado a cada compañía de mil mortales ascendentes, pero encontraréis cantidades mayores a medida que progreséis por las siete esferas de las mansiones. Estos seres hermosos y polifacéticos son unos asociados sociables y unos guías encantadores. Son libres de acompañar a los individuos o a los grupos escogidos a cualquiera de las esferas culturales de transición, incluídos sus mundos satélites. Son los guías de las excursiones y los asociados recreativos de todos los mortales ascendentes. A menudo acompañan a los grupos supervivientes en sus visitas periódicas a Jerusem, y en cualquier momento de vuestra estancia allí, podéis ir al sector de los registros de la capital del sistema y encontraros con los mortales ascendentes de los siete mundos de las mansiones, puesto que éstos viajan libremente de aquí para allá entre sus moradas residenciales y la sede del sistema.

\section*{4. El segundo mundo de las mansiones}
\par
%\textsuperscript{(534.5)}
\textsuperscript{47:4.1} En esta esfera es donde os instaláis más plenamente en la vida de las mansonias. Las agrupaciones de la vida morontial empiezan a tomar forma; los grupos de trabajo y las organizaciones sociales empiezan a funcionar, las comunidades alcanzan sus proporciones normales, y los mortales que progresan dan origen a nuevas órdenes sociales y a nuevas disposiciones gubernamentales.

\par
%\textsuperscript{(534.6)}
\textsuperscript{47:4.2} Los supervivientes fusionados con el Espíritu ocupan los mundos de las mansiones junto con los mortales ascendentes fusionados con el Ajustador. Aunque las diversas órdenes de vida celestial son diferentes, todas son amistosas y fraternales. En ninguno de los mundos ascendentes encontraréis nada que se parezca a la intolerancia humana y a las discriminaciones de los sistemas desconsiderados de las castas.

\par
%\textsuperscript{(534.7)}
\textsuperscript{47:4.3} A medida que ascendáis los mundos de las mansiones uno tras otro, los encontraréis más abarrotados con las actividades morontiales de los supervivientes que progresan. A medida que avancéis reconoceréis que los mundos de las mansiones contienen cada vez más características de Jerusem. El mar de cristal\footnote{\textit{Mar de cristal}: Ap 4:6; 15:2.} hace su aparición en la segunda mansonia.

\par
%\textsuperscript{(534.8)}
\textsuperscript{47:4.4} Cada vez que avancéis de un mundo de las mansiones a otro, adquirís un cuerpo morontial recién desarrollado y adecuadamente adaptado. Os dormís para el transporte seráfico y os despertáis en las salas de resurrección con el nuevo cuerpo sin desarrollar, de manera muy parecida a cuando llegasteis por primera vez al mundo de las mansiones número uno, salvo que el Ajustador del Pensamiento no os deja durante estos sueños de tránsito entre los mundos de las mansiones. Una vez que habéis pasado desde los mundos evolutivos al mundo inicial de las mansiones, vuestra personalidad permanece intacta.

\par
%\textsuperscript{(535.1)}
\textsuperscript{47:4.5} A medida que ascendéis por la vida morontial, vuestra memoria custodiada por el Ajustador permanece totalmente intacta. Aquellas asociaciones mentales que eran puramente animales y totalmente materiales perecieron de manera natural con el cerebro físico, pero todas las cosas valiosas de vuestra vida mental que tenían un valor de supervivencia fueron duplicadas por el Ajustador y se conservan como parte de la memoria personal durante toda la carrera ascendente. Tendréis conciencia de todas vuestras experiencias valiosas a medida que avancéis de un mundo de las mansiones a otro y de una sección del universo a otra ---incluso hasta el Paraíso.

\par
%\textsuperscript{(535.2)}
\textsuperscript{47:4.6} Aunque tenéis un cuerpo morontial, continuáis comiendo, bebiendo y descansando a lo largo de todos estos siete mundos. Tomáis los alimentos de tipo morontial, un reino de energía viviente desconocido en los mundos materiales. El cuerpo morontial utiliza plenamente tanto la comida como el agua, pero no hay desechos residuales. Deteneos a pensar: la mansonia número uno es una esfera muy material que presenta los comienzos iniciales del régimen morontial. Sois todavía casi humanos y no estáis muy alejados de los puntos de vista limitados de la vida mortal, pero cada mundo revela un progreso definido. De esfera en esfera os volvéis menos materiales, más intelectuales y un poco más espirituales. De estos siete mundos progresivos, el progreso espiritual es mayor en los tres últimos.

\par
%\textsuperscript{(535.3)}
\textsuperscript{47:4.7} Las deficiencias biológicas fueron ampliamente compensadas en el primer mundo de las mansiones. Allí, los defectos de la experiencia planetaria relacionados con la vida sexual, la asociación familiar y la función parental fueron corregidos o bien se hicieron proyectos para su rectificación futura dentro de las familias de los Hijos Materiales en Jerusem.

\par
%\textsuperscript{(535.4)}
\textsuperscript{47:4.8} La mansonia número dos asegura más específicamente la eliminación de todas las fases de los conflictos intelectuales y la curación de la falta de armonía mental en todas sus variedades. El esfuerzo que empezó en el primer mundo de las mansiones por dominar el significado de la mota morontial continúa aquí con más intensidad. El desarrollo que se alcanza en la mansonia número dos es comparable con el nivel intelectual de la cultura posterior al Hijo Magistral en los mundos evolutivos ideales.

\section*{5. El tercer mundo de las mansiones}
\par
%\textsuperscript{(535.5)}
\textsuperscript{47:5.1} La tercera mansonia es la sede de los Educadores de los Mundos de las Mansiones. Aunque ejercen su actividad en las siete esferas de las mansiones, mantienen su sede colectiva en el centro de los círculos académicos del mundo número tres. Hay millones de estos instructores en los mundos de las mansiones y en los mundos morontiales superiores. Estos querubines avanzados y glorificados sirven como educadores morontiales a lo largo de todos los mundos de las mansiones hasta la última esfera de educación ascendente del universo local. Se encontrarán entre los últimos en deciros un afectuoso adiós cuando se acerque el momento de la despedida, el momento en que diréis adiós ---al menos durante algunas eras--- al universo de vuestro origen, cuando os enserafinéis para el traslado a los mundos receptores del sector menor del superuniverso.

\par
%\textsuperscript{(535.6)}
\textsuperscript{47:5.2} Durante vuestra estancia en el primer mundo de las mansiones, tendréis permiso para visitar el primer mundo de transición, la sede de los finalitarios y la guardería probatoria del sistema donde se cría a los niños evolutivos no desarrollados. Cuando lleguéis a la mansonia número dos, recibiréis permiso para visitar periódicamente el mundo de transición número dos, donde están situadas la sede de la supervisión morontial para toda Satania y las escuelas educativas para las diversas órdenes morontiales. Cuando lleguéis al mundo de las mansiones número tres, os concederán inmediatamente un permiso para visitar la tercera esfera de transición, sede de las órdenes angélicas y centro de sus diversas escuelas educativas en el sistema. Las visitas desde este mundo a Jerusem son cada vez más beneficiosas y tienen un interés creciente para los mortales que progresan.

\par
%\textsuperscript{(536.1)}
\textsuperscript{47:5.3} La tercera mansonia es un mundo de grandes logros personales y sociales para todos aquellos que no han experimentado el equivalente de estos círculos de cultura antes de ser liberados de la carne en sus mundos de nacimiento como mortales. En esta esfera empieza un trabajo educativo más positivo. La formación en los dos primeros mundos de las mansiones es principalmente de naturaleza negativa ---compensar deficiencias--- en el sentido de que consiste en completar la experiencia de la vida en la carne. En este tercer mundo de las mansiones, los supervivientes empiezan realmente su cultura morontial progresiva. El propósito principal de esta educación consiste en aumentar la comprensión de la correlación entre la mota morontial y la lógica de los mortales, la coordinación de la mota morontial con la filosofía humana. Ahora, los mortales supervivientes llegan a comprender bien, en la práctica, la verdadera metafísica. Es la auténtica introducción a la comprensión inteligente de los significados cósmicos y de las interrelaciones universales. La cultura del tercer mundo de las mansiones comparte la naturaleza de la época posterior a la donación de un Hijo en un planeta habitado normal.

\section*{6. El cuarto mundo de las mansiones}
\par
%\textsuperscript{(536.2)}
\textsuperscript{47:6.1} Cuando llegáis al cuarto mundo de las mansiones, ya estáis bien introducidos en la carrera morontial; habéis efectuado un largo camino de progreso desde vuestra existencia material inicial. Ahora se os concede permiso para visitar el mundo de transición número cuatro y os familiaricéis allí con la sede y las escuelas formativas de los superángeles, incluyendo a las Brillantes Estrellas Vespertinas. Gracias a los buenos oficios de estos superángeles del cuarto mundo de transición, los visitantes morontiales pueden acercarse mucho a las diversas órdenes de Hijos de Dios durante sus visitas periódicas a Jerusem, ya que a los mortales que progresan se les van abriendo gradualmente nuevos sectores de la capital del sistema a medida que visitan repetidamente el mundo sede. Nuevas grandiosidades se van desplegando progresivamente para las mentes en expansión de estos ascendentes.

\par
%\textsuperscript{(536.3)}
\textsuperscript{47:6.2} En la cuarta mansonia, el ascendente individual encuentra más apropiadamente su lugar en el trabajo de grupo y en las actividades de clase de la vida morontial. Los ascendentes desarrollan aquí una mayor apreciación de las transmisiones y de otras fases de la cultura y del progreso del universo local.

\par
%\textsuperscript{(536.4)}
\textsuperscript{47:6.3} Durante el período de formación en el mundo número cuatro es cuando los mortales ascendentes son iniciados realmente por primera vez en las exigencias y los encantos de la verdadera vida social de las criaturas morontiales. Para las criaturas evolutivas es en verdad una nueva experiencia participar en unas actividades sociales que no están basadas ni en el engrandecimiento personal ni en la conquista egoísta. Se os introduce en un nuevo orden social, un orden basado en la simpatía comprensiva del aprecio mutuo, el amor desinteresado de servirse mutuamente, y la motivación dominante de llevar a cabo un destino común y supremo ---la meta paradisiaca de la perfección adoradora y divina. Todos los ascendentes se vuelven conscientes de conocer a Dios, de revelar a Dios, de buscar a Dios y de encontrar a Dios.

\par
%\textsuperscript{(536.5)}
\textsuperscript{47:6.4} La cultura intelectual y social de este cuarto mundo de las mansiones se puede comparar con la vida mental y social de la época posterior al Hijo Instructor en los planetas que tienen una evolución normal. El nivel espiritual es mucho más avanzado que el de esa dispensación mortal.

\section*{7. El quinto mundo de las mansiones}
\par
%\textsuperscript{(537.1)}
\textsuperscript{47:7.1} El transporte al quinto mundo de las mansiones representa un enorme paso hacia adelante en la vida de un progresor morontial. La experiencia en este mundo es una verdadera anticipación de la vida en Jerusem. Aquí empezáis a daros cuenta del elevado destino de los mundos evolutivos leales, puesto que pueden progresar normalmente hasta este estado durante su desarrollo planetario natural. La cultura de este mundo de las mansiones corresponde en general a la de la era inicial de luz y de vida en los planetas cuyo progreso evolutivo es normal. Esto os permitirá comprender por qué está planeado que los tipos de seres sumamente cultos y progresivos, que a veces habitan en esos mundos evolutivos avanzados, estén exentos de pasar por una o más, o incluso por todas las esferas de las mansiones.

\par
%\textsuperscript{(537.2)}
\textsuperscript{47:7.2} Como habéis dominado el idioma del universo local antes de dejar el cuarto mundo de las mansiones, ahora dedicáis más tiempo a perfeccionar la lengua de Uversa con el objeto de que podáis ser unos expertos en los dos idiomas antes de llegar a Jerusem con la categoría de residentes. Todos los mortales ascendentes son biling\"ues desde la sede del sistema hasta Havona. Y allí sólo es necesario ampliar el vocabulario del superuniverso, necesitándose aún una ampliación adicional para residir en el Paraíso.

\par
%\textsuperscript{(537.3)}
\textsuperscript{47:7.3} Después de llegar a la mansonia número cinco, el peregrino recibe permiso para visitar el mundo de transición correspondiente a este número, la sede de los Hijos. Aquí, el mortal ascendente se familiariza personalmente con los diversos grupos de filiación divina. Ha oído hablar de estos seres magníficos y ya los ha encontrado en Jerusem, pero ahora llega a conocerlos realmente.

\par
%\textsuperscript{(537.4)}
\textsuperscript{47:7.4} En la quinta mansonia empezáis a aprender cosas sobre los mundos de estudio de la constelación. Aquí encontráis al primero de los instructores que empieza a prepararos para vuestra estancia posterior en la constelación. Esta preparación continúa en los mundos seis y siete, mientras que los toques finales se dan en el sector de los mortales ascendentes situado en Jerusem.

\par
%\textsuperscript{(537.5)}
\textsuperscript{47:7.5} En la mansonia número cinco se produce un verdadero nacimiento de la conciencia cósmica. Estáis llegando a tener una mentalidad universal. Éste es en verdad un período de expansión de los horizontes. La mente en expansión de los mortales ascendentes empieza a darse cuenta de que un destino prodigioso y magnífico, un destino celestial y divino, espera a todos aquellos que terminan la ascensión progresiva al Paraíso, la cual ha empezado tan laboriosamente pero de una manera tan alegre y favorable. Aproximadamente en este punto, el ascendente mortal de tipo medio empieza a manifestar un auténtico entusiasmo experiencial por la ascensión a Havona. El estudio se vuelve voluntario, el servicio desinteresado, natural, y la adoración, espontánea. Está brotando un verdadero carácter morontial; se está desarrollando una verdadera criatura morontial.

\section*{8. El sexto mundo de las mansiones}
\par
%\textsuperscript{(537.6)}
\textsuperscript{47:8.1} Los que residen en esta esfera tienen permiso para visitar el mundo de transición número seis, donde aprenden más cosas sobre los espíritus elevados del superuniverso, aunque no sean capaces de ver a muchos de estos seres celestiales. Aquí reciben también sus primeras lecciones relacionadas con la carrera espiritual futura que empieza inmediatamente después de graduarse en la educación morontial del universo local.

\par
%\textsuperscript{(537.7)}
\textsuperscript{47:8.2} El Soberano asistente del Sistema visita con frecuencia este mundo, y aquí empieza la instrucción inicial en la técnica de la administración del universo. Ahora se imparten las primeras lecciones que abarcan los asuntos de un universo entero.

\par
%\textsuperscript{(538.1)}
\textsuperscript{47:8.3} Es una era brillante para los mortales ascendentes, la cual presencia generalmente la fusión perfecta entre la mente humana y el Ajustador divino. Esta fusión puede haberse producido en potencia anteriormente, pero muchas veces la identidad válida real no se consigue hasta el momento en que se reside en el quinto mundo de las mansiones o incluso en el sexto.

\par
%\textsuperscript{(538.2)}
\textsuperscript{47:8.4} El llamamiento seráfico del superángel supervisor encargado de los supervivientes resucitados y del arcángel autorizado encargado de aquellos que van a juicio al tercer día señala la unión del alma inmortal evolutiva con el Ajustador eterno y divino; luego, en presencia de los asociados morontiales de dicho superviviente, estos mensajeros confirmatorios dicen: <<Éste es un hijo amado en quien me siento muy complacido>>\footnote{\textit{Éste es un hijo amado}: Mt 3:17; 17:5; Mc 1:11; Lc 3:22; 2 P 1:17.}. Esta sencilla ceremonia marca la entrada de un mortal ascendente en la carrera eterna del servicio paradisiaco.

\par
%\textsuperscript{(538.3)}
\textsuperscript{47:8.5} Inmediatamente después de confirmarse la fusión con el Ajustador, el nuevo ser morontial es presentado por primera vez a sus compañeros con su nuevo nombre\footnote{\textit{Nuevo nombre}: Is 62:2; Ap 2:17; 3:12.}, y se le conceden cuarenta días de retiro espiritual de todas las actividades rutinarias para comulgar consigo mismo, escoger una de las rutas optativas para dirigirse a Havona, y elegir entre las técnicas diferenciales existentes para alcanzar el Paraíso.

\par
%\textsuperscript{(538.4)}
\textsuperscript{47:8.6} Pero estos seres brillantes son todavía más o menos materiales; están lejos de ser verdaderos espíritus; espiritualmente hablando, se parecen más a unos seres supermortales, todavía un poco inferiores a los ángeles. Pero se están convirtiendo realmente en unas criaturas maravillosas.

\par
%\textsuperscript{(538.5)}
\textsuperscript{47:8.7} Durante la estancia en el mundo número seis, los estudiantes de este mundo de las mansiones consiguen un estado comparable al del elevado desarrollo que caracteriza a aquellos mundos evolutivos que han progresado normalmente más allá de la etapa inicial de luz y de vida. La organización de la sociedad en esta mansonia es de un orden elevado. La sombra de la naturaleza mortal disminuye cada vez más a medida que estos mundos se ascienden uno tras otro. Os volvéis cada vez más encantadores a medida que dejáis atrás los burdos vestigios de vuestro origen animal planetario. <<Ascender a base de grandes tribulaciones>>\footnote{\textit{A base de grandes tribulaciones}: Ap 7:14.} sirve para hacer que los mortales glorificados sean muy buenos y comprensivos, muy compasivos y tolerantes.

\section*{9. El séptimo mundo de las mansiones}
\par
%\textsuperscript{(538.6)}
\textsuperscript{47:9.1} La experiencia en esta esfera es el logro que corona la carrera que sigue de inmediato a la muerte. Durante vuestra estancia aquí recibiréis la enseñanza de muchos educadores, y todos cooperarán en la tarea de prepararos para residir en Jerusem. Cualquier diferencia discernible entre aquellos mortales procedentes de los mundos aislados y retrasados y aquellos supervivientes que provienen de las esferas más avanzadas e iluminadas es prácticamente eliminada durante la estancia en el séptimo mundo de las mansiones. Aquí seréis purificados de todos los restos de una herencia desafortunada, de un entorno malsano y de las tendencias planetarias no espirituales. Los últimos restos de la <<marca de la bestia>>\footnote{\textit{Eliminar la marca de la bestia}: Ap 13:15-17; 14:9-11; 16:2; 19:20; 20:4.} son erradicados aquí.

\par
%\textsuperscript{(538.7)}
\textsuperscript{47:9.2} Mientras se reside en la mansonia número siete, se concede permiso para visitar el mundo de transición número siete, el mundo del Padre Universal. Aquí empezáis una nueva adoración más espiritual del Padre invisible, una costumbre que practicaréis cada vez más durante toda vuestra larga carrera ascendente. En este mundo de cultura de transición encontráis el templo del Padre, pero no veis al Padre.

\par
%\textsuperscript{(538.8)}
\textsuperscript{47:9.3} Ahora empieza la formación de las clases con el fin de graduarse para residir en Jerusem. Habéis ido de mundo en mundo como individuos, pero ahora os preparáis para partir en grupo hacia Jerusem, aunque, dentro de ciertos límites, un ascendente puede elegir quedarse en el séptimo mundo de las mansiones con el fin de esperar la llegada de un miembro rezagado de su grupo de trabajo terrestre o mansoniano.

\par
%\textsuperscript{(539.1)}
\textsuperscript{47:9.4} El personal de la séptima mansonia se reúne en el mar de cristal para presenciar vuestra partida hacia Jerusem con la categoría de residentes. Podéis haber visitado Jerusem cientos o miles de veces, pero siempre como invitados; nunca antes os habíais dirigido hacia la capital del sistema en compañía de un grupo de compañeros vuestros que se despedían eternamente como mortales ascendentes de toda la carrera en las mansonias. Pronto seréis acogidos en el campo de recepción del mundo sede como ciudadanos de Jerusem.

\par
%\textsuperscript{(539.2)}
\textsuperscript{47:9.5} Disfrutaréis mucho progresando a través de los siete mundos desmaterializantes; son unas esferas donde os volvéis realmente menos mortales. En el primer mundo de las mansiones sois principalmente humanos, simplemente un ser mortal menos su cuerpo material, una mente humana alojada en una forma morontial ---un cuerpo material del mundo morontial, pero no un tabernáculo mortal de carne y hueso. Pasáis realmente del estado mortal al estado inmortal en el momento de fusionar con el Ajustador, y cuando hayáis terminado vuestra carrera en Jerusem, seréis unos morontianos plenamente desarrollados.

\section*{10. La ciudadanía de Jerusem}
\par
%\textsuperscript{(539.3)}
\textsuperscript{47:10.1} La recepción de una nueva clase de graduados de los mundos de las mansiones es la señal que espera todo Jerusem para reunirse como comité de bienvenida. Incluso los espornagias disfrutan con la llegada de estos ascendentes triunfantes de origen evolutivo, que han participado en la carrera planetaria y han terminado su progresión en los mundos de las mansiones. Únicamente los controladores físicos y los Supervisores del Poder Morontial están ausentes en estas ocasiones de regocijo.

\par
%\textsuperscript{(539.4)}
\textsuperscript{47:10.2} Juan el Revelador tuvo una visión de la llegada de una clase de mortales que avanzaban desde el séptimo mundo de las mansiones hasta su primer cielo, hasta las glorias de Jerusem. Dejó escrito: <<Y vi como un mar de cristal mezclado con fuego\footnote{\textit{Mar de cristal y fuego}: Ap 4:6; 15:2-3.}; y a aquellos que habían logrado vencer a la bestia que al principio estaba en ellos y en la imagen que subsistía a través de los mundos de las mansiones y finalmente en la última marca y huella, que se hallaban en el mar de cristal, con las arpas de Dios, y cantando la canción de la liberación del temor y de la muerte humanos>>. (A todos estos mundos llegan las comunicaciones perfeccionadas del espacio; y estas comunicaciones las podéis recibir en cualquier parte si lleváis el <<arpa de Dios>>\footnote{\textit{Arpa de Dios}: Ap 5:8; 14:2; 15:2.}, un aparato morontial que compensa la incapacidad para adaptar directamente el mecanismo sensorial morontial inmaduro a la recepción de las comunicaciones espaciales).

\par
%\textsuperscript{(539.5)}
\textsuperscript{47:10.3} Pablo también tuvo una visión del cuerpo de ciudadanos ascendentes de mortales en vías de perfeccionarse en Jerusem, pues escribió: <<Pero habéis llegado hasta el Monte Sión y hasta la ciudad del Dios vivo, la Jerusalén celestial, y hasta una innumerable compañía de ángeles, hasta la gran asamblea de Miguel, y hasta los espíritus de los hombres justos que se han hecho perfectos>>\footnote{\textit{Monte Sión, ciudad del Dios vivo}: Heb 12:22-23.}.

\par
%\textsuperscript{(539.6)}
\textsuperscript{47:10.4} Después de que los mortales han conseguido la residencia en la sede del sistema, ya no experimentarán más resurrecciones literales. La forma morontial que se os concede al dejar la carrera de los mundos de las mansiones es tal que os acompañará hasta el final de vuestra experiencia en el universo local. De vez en cuando se efectuarán cambios, pero conservaréis esta misma forma hasta que os despidáis de ella cuando emerjáis como espíritus de la primera fase antes de ser transportados a los mundos de cultura ascendente y de formación espiritual del superuniverso.

\par
%\textsuperscript{(540.1)}
\textsuperscript{47:10.5} Aquellos mortales que pasan por toda la carrera de las mansonias experimentan siete veces el sueño de ajuste y el despertar de la resurrección. Pero la última sala de resurrección, la cámara del despertar definitivo, fue dejada atrás en el séptimo mundo de las mansiones. Los cambios de forma ya no volverán a necesitar la pérdida de la conciencia o una interrupción en la continuidad de la memoria personal.

\par
%\textsuperscript{(540.2)}
\textsuperscript{47:10.6} La personalidad mortal que dio comienzo en los mundos evolutivos metida en un tabernáculo de carne ---habitada por un Monitor de Misterio e investida del Espíritu de la Verdad--- no se moviliza, realiza y unifica plenamente hasta el día en que este ciudadano de Jerusem recibe permiso para ir a Edentia y es proclamado como un verdadero miembro del cuerpo morontial de Nebadon ---un superviviente inmortal asociado con su Ajustador, un ascendente al Paraíso, una personalidad con categoría morontial y un verdadero hijo de los Altísimos.

\par
%\textsuperscript{(540.3)}
\textsuperscript{47:10.7} La muerte física es una técnica para escapar de la vida material en la carne; y la experiencia de la vida progresiva en las mansonias a través de siete mundos de formación correctora y de educación cultural representa la entrada de los supervivientes mortales en la carrera morontial, la vida de transición que media entre la existencia material evolutiva y los logros espirituales superiores de los ascendentes del tiempo que están destinados a alcanzar las puertas de la eternidad.

\par
%\textsuperscript{(540.4)}
\textsuperscript{47:10.8} [Patrocinado por una Brillante Estrella Vespertina.]


\chapter{Documento 48. La vida morontial}
\par
%\textsuperscript{(541.1)}
\textsuperscript{48:0.1} LOS Dioses no pueden transformar, mediante un acto misterioso de magia creativa, a una criatura de naturaleza animal ordinaria en un espíritu perfeccionado ---al menos no lo hacen. Cuando los Creadores desean dar nacimiento a unos seres perfectos, lo hacen mediante una creación directa y original, pero nunca emprenden el convertir en una sola etapa a las criaturas materiales de origen animal en unos seres de perfección.

\par
%\textsuperscript{(541.2)}
\textsuperscript{48:0.2} La vida morontial, que se extiende como lo hace a lo largo de la diversas fases de la carrera en el universo local, es el único acceso posible por el que los mortales materiales pueden alcanzar el umbral del mundo espiritual. ¿Qué tipo de magia podría tener la muerte, la disolución natural del cuerpo material, para que este simple paso transformara instantáneamente a la mente mortal y material en un espíritu inmortal y perfeccionado? Estas creencias no son más que supersticiones ignorantes y fábulas agradables.

\par
%\textsuperscript{(541.3)}
\textsuperscript{48:0.3} Esta transición morontial siempre media entre el estado mortal y el estado espiritual posterior de los seres humanos supervivientes. Este estado intermedio de progreso en el universo difiere notablemente en las diversas creaciones locales, pero todas son en la práctica bastante similares. La organización de los mundos de las mansiones y de los mundos morontiales superiores en Nebadon es bastante típica de los regímenes morontiales de transición de esta parte de Orvonton.

\section*{1. Los materiales morontiales}
\par
%\textsuperscript{(541.4)}
\textsuperscript{48:1.1} Los reinos morontiales son las esferas del universo local que enlazan los niveles materiales y los niveles espirituales de existencia de las criaturas. Esta vida morontial se ha conocido en Urantia desde los primeros tiempos del Príncipe Planetario. Este estado de transición se ha enseñado de vez en cuando a los mortales, y el concepto ha encontrado su sitio de manera desvirtuada en las religiones de hoy en día.

\par
%\textsuperscript{(541.5)}
\textsuperscript{48:1.2} Las esferas morontiales son las fases de transición de la ascensión de los mortales a través de los mundos de progreso del universo local. Los siete mundos que rodean a la esfera finalitaria de los sistemas locales son los únicos que se llaman mundos de las mansiones, pero las cincuenta y seis moradas sistémicas de transición, junto con las esferas superiores que están alrededor de las sedes de las constelaciones y del universo, se llaman mundos morontiales. Estas creaciones comparten la belleza física y la grandiosidad morontial de las esferas sede del universo local.

\par
%\textsuperscript{(541.6)}
\textsuperscript{48:1.3} Todos estos mundos son esferas arquitectónicas y tienen exactamente el doble de elementos que los planetas evolutivos. Estos mundos hechos por encargo no solamente abundan en metales pesados y en cristales, pues tienen cien elementos físicos, sino que también poseen exactamente cien formas de una organización energética única llamada \textit{materia morontial}. Los Controladores Físicos Maestros y los Supervisores del Poder Morontial son capaces de modificar la rotación de las unidades primarias de la materia y de transformar al mismo tiempo estas asociaciones energéticas de tal manera que pueden crear esta nueva sustancia.

\par
%\textsuperscript{(542.1)}
\textsuperscript{48:1.4} La vida morontial inicial en los sistemas locales se parece mucho a la de vuestro mundo material actual, volviéndose menos física y más verdaderamente morontial en los mundos de estudio de la constelación. Y cuando lleguéis a las esferas de Salvington, alcanzaréis unos niveles espirituales cada vez más elevados.

\par
%\textsuperscript{(542.2)}
\textsuperscript{48:1.5} Los Supervisores del Poder Morontial son capaces de efectuar una unión de las energías materiales y espirituales, organizando así una forma de materialización morontial que es receptiva a la superposición de un espíritu que la controle. Cuando atravesáis la vida morontial de Nebadon, estos mismos pacientes y hábiles Supervisores del Poder Morontial os proporcionarán sucesivamente 570 cuerpos morontiales, y cada uno de ellos representará una fase de vuestra transformación progresiva. Desde el momento en que dejáis los mundos materiales hasta que os convertís en espíritus de la primera fase en Salvington, pasaréis exactamente por 570 cambios morontiales distintos y ascendentes. Ocho de ellos se producen en el sistema, setenta y uno en la constelación y 491 durante la estancia en las esferas de Salvington.

\par
%\textsuperscript{(542.3)}
\textsuperscript{48:1.6} Durante los años que vivís en la carne mortal, el espíritu divino reside en vosotros casi como una cosa aparte ---en realidad, el espíritu otorgado por el Padre Universal invade al hombre. Pero en la vida morontial, el espíritu se convertirá en una parte real de vuestra personalidad, y a medida que paséis sucesivamente por las 570 transformaciones progresivas, ascenderéis desde el estado material al estado espiritual de vida de las criaturas.

\par
%\textsuperscript{(542.4)}
\textsuperscript{48:1.7} Pablo conocía la existencia de los mundos morontiales y la realidad de la materia morontial, pues escribió: <<Tienen en el cielo una sustancia mejor y más duradera>>\footnote{\textit{Una mejor sustancia en el cielo}: Mt 6:19-20; Heb 10:34.}. Y estos materiales morontiales son reales, tangibles, como en <<la ciudad que tiene cimientos, cuyo constructor y hacedor es Dios>>\footnote{\textit{Ciudad cuyo constructor es Dios}: 2 Co 5:1; Heb 11:10.}. Y cada una de estas esferas maravillosas es <<un país mejor, es decir, un país celestial>>\footnote{\textit{Un país mejor, celestial}: Heb 11:16.}.

\section*{2. Los supervisores del poder morontial}
\par
%\textsuperscript{(542.5)}
\textsuperscript{48:2.1} Estos seres únicos se ocupan exclusivamente de supervisar aquellas actividades que representan una combinación válida de las energías espirituales y físicas o semimateriales. Se dedican exclusivamente al ministerio de la progresión morontial. No es que ayuden mucho a los mortales durante la experiencia de transición, sino que más bien hacen posible el entorno de transición a las criaturas morontiales que progresan. Son los canales de poder morontial que sostienen y energizan las fases morontiales de los mundos de transición.

\par
%\textsuperscript{(542.6)}
\textsuperscript{48:2.2} Los Supervisores del Poder Morontial son la progenie del Espíritu Madre del universo local. Son diseñados de manera bastante uniforme, aunque su naturaleza difiere ligeramente en las diversas creaciones locales. Son creados para su tarea específica y no necesitan ninguna formación antes de asumir sus responsabilidades.

\par
%\textsuperscript{(542.7)}
\textsuperscript{48:2.3} En un universo local, la creación de los primeros Supervisores del Poder Morontial se efectúa al mismo tiempo que llega el primer superviviente mortal a las orillas de uno de los primeros mundos de las mansiones. Son creados en grupos de mil y están clasificados como sigue:

\par
%\textsuperscript{(542.8)}
\textsuperscript{48:2.4} 1. 400 Reguladores de Circuitos.

\par
%\textsuperscript{(542.9)}
\textsuperscript{48:2.5} 2. 200 Coordinadores de Sistemas.

\par
%\textsuperscript{(542.10)}
\textsuperscript{48:2.6} 3. 100 Guardianes Planetarios.

\par
%\textsuperscript{(543.1)}
\textsuperscript{48:2.7} 4. 100 Controladores Combinados.

\par
%\textsuperscript{(543.2)}
\textsuperscript{48:2.8} 5. 100 Estabilizadores de Enlaces.

\par
%\textsuperscript{(543.3)}
\textsuperscript{48:2.9} 6. 50 Clasificadores Selectivos.

\par
%\textsuperscript{(543.4)}
\textsuperscript{48:2.10} 7. 50 Registradores Asociados.

\par
%\textsuperscript{(543.5)}
\textsuperscript{48:2.11} Los supervisores del poder siempre sirven en su universo nativo. Son dirigidos exclusivamente por la actividad espiritual conjunta del Hijo del Universo y del Espíritu del Universo, pero forman por lo demás un grupo totalmente autónomo. Mantienen una sede en cada primer mundo de las mansiones de los sistemas locales, donde trabajan en estrecha asociación con los controladores físicos y los serafines, pero ejercen su actividad en un mundo propio cuando se trata de la manifestación de la energía y de la aplicación del espíritu.

\par
%\textsuperscript{(543.6)}
\textsuperscript{48:2.12} A veces trabajan también en los mundos evolutivos, en conexión con los fenómenos supermateriales, como ministros destinados allí temporalmente. Pero raras veces sirven en los planetas habitados; y tampoco trabajan en los mundos educativos superiores del superuniverso, estando dedicados principalmente al régimen de transición de la progresión morontial de un universo local.

\par
%\textsuperscript{(543.7)}
\textsuperscript{48:2.13} 1. \textit{Los Reguladores de los Circuitos}. Son los seres sin igual que coordinan la energía física y espiritual y regulan su flujo en los canales separados de las esferas morontiales, y estos circuitos son exclusivamente planetarios, estando limitados a un solo mundo. Los circuitos morontiales son distintos de los circuitos tanto físicos como espirituales de los mundos de transición, pero adicionales a ellos, y se necesitan millones de reguladores de este tipo para energizar incluso un sistema de mundos de las mansiones como el de Satania.

\par
%\textsuperscript{(543.8)}
\textsuperscript{48:2.14} Los reguladores de los circuitos introducen en las energías materiales aquellos cambios que las dejan sometidas al control y a la regulación de sus asociados. Estos seres son generadores morontiales de poder así como reguladores de circuitos. Al igual que una dinamo genera aparentemente electricidad de la atmósfera, estas dinamos morontiales vivientes parecen transformar las energías omnipresentes del espacio en aquellos materiales que los supervisores morontiales tejen en los cuerpos y en las actividades vitales de los mortales ascendentes.

\par
%\textsuperscript{(543.9)}
\textsuperscript{48:2.15} 2. \textit{Los Coordinadores de los Sistemas}. Puesto que cada mundo morontial posee un tipo distinto de energía morontial, a los humanos les resulta extremadamente difícil visualizar estas esferas. Pero en cada esfera sucesiva de transición, los mortales encontrarán que la vida vegetal y todo lo demás relacionado con la existencia morontial están progresivamente modificados para corresponderse con la espiritualización creciente de los supervivientes ascendentes. Y puesto que el sistema energético de cada mundo está individualizado de esta manera, estos coordinadores trabajan para armonizar y combinar estos diferentes sistemas de poder en una unidad de trabajo para las esferas asociadas de un grupo determinado.

\par
%\textsuperscript{(543.10)}
\textsuperscript{48:2.16} Los mortales ascendentes progresan gradualmente de lo físico a lo espiritual a medida que avanzan de un mundo morontial a otro; de ahí la necesidad de proporcionarles una escala ascendente de esferas morontiales y una escala ascendente de formas morontiales.

\par
%\textsuperscript{(543.11)}
\textsuperscript{48:2.17} Cuando los ascendentes de los mundos de las mansiones pasan de una esfera a otra, los serafines transportadores los entregan a los receptores de los coordinadores sistémicos en el mundo más avanzado. Aquí, en estos templos sin igual situados en el centro de las setenta alas radiantes donde se encuentran las cámaras de transición similares a las salas de resurrección del mundo inicial que recibe a los mortales de origen terrestre, los coordinadores sistémicos efectúan hábilmente los cambios necesarios en la forma de las criaturas. Se necesitan unos siete días del tiempo oficial para llevar a cabo estos cambios iniciales en la forma morontial.

\par
%\textsuperscript{(544.1)}
\textsuperscript{48:2.18} 3. \textit{Los Guardianes Planetarios}. Cada mundo morontial, desde las esferas de las mansiones hasta la sede del universo, está custodiado ---en lo que se refiere a los asuntos morontiales--- por setenta guardianes. Forman el consejo planetario local provisto de una autoridad morontial suprema. Este consejo concede el material para las formas morontiales de todas las criaturas ascendentes que aterrizan en las esferas, y autoriza los cambios en la forma de las criaturas que permiten a un ascendente pasar a la esfera siguiente. Después de haber atravesado los mundos de las mansiones, os trasladaréis de una fase de la vida morontial a otra sin tener que perder la conciencia. La inconciencia sólo acompaña a las primeras metamorfosis y a las transiciones posteriores de un universo a otro y de Havona al Paraíso.

\par
%\textsuperscript{(544.2)}
\textsuperscript{48:2.19} 4. \textit{Los Controladores Combinados}. En el centro de cada unidad administrativa de un mundo morontial siempre está estacionado uno de estos seres extremadamente maquinales. Un controlador combinado es sensible a las energías físicas, espirituales y morontiales, y funciona con ellas; y con este ser siempre están asociados dos coordinadores de sistemas, cuatro reguladores de circuitos, un guardián planetario, un estabilizador de enlaces y, o bien un registrador asociado o un clasificador selectivo.

\par
%\textsuperscript{(544.3)}
\textsuperscript{48:2.20} 5. \textit{Los Estabilizadores de Enlaces}. Son los reguladores de la energía morontial en asociación con las fuerzas físicas y espirituales del reino. Hacen posible la conversión de la energía morontial en materia morontial. Toda la organización morontial de la existencia depende de los estabilizadores. Disminuyen la rotación de las energías hasta el punto en que pueden volverse físicas. Pero no dispongo de términos con los que poder comparar o ilustrar el ministerio de estos seres. Sobrepasa por completo la imaginación humana.

\par
%\textsuperscript{(544.4)}
\textsuperscript{48:2.21} 6. \textit{Los Clasificadores Selectivos}. A medida que progresáis de una clase o fase de un mundo morontial a otro, tenéis que ser reafinados o sintonizados con vuestro avance, y los clasificadores selectivos tienen la tarea de manteneros en sincronización progresiva con la vida morontial.

\par
%\textsuperscript{(544.5)}
\textsuperscript{48:2.22} Aunque las formas básicas de la vida y de la materia morontiales son idénticas desde el primer mundo de las mansiones hasta la última esfera de transición del universo, existe una progresión funcional que se extiende gradualmente desde lo material hasta lo espiritual. Vuestra adaptación a esta creación básicamente uniforme, pero cada vez más avanzada y espiritualizada, se efectúa mediante esta resintonización selectiva. Este ajuste en el mecanismo de la personalidad equivale a una nueva creación, a pesar de que conserváis la misma forma morontial.

\par
%\textsuperscript{(544.6)}
\textsuperscript{48:2.23} Podéis someteros repetidas veces a las pruebas de estos examinadores, y en cuanto reflejéis un logro espiritual adecuado, certificarán con mucho gusto que podéis pasar a una posición más avanzada. Estos cambios progresivos tienen como resultado reacciones diferentes al entorno morontial, tales como modificaciones en las necesidades alimenticias y en otras numerosas prácticas personales.

\par
%\textsuperscript{(544.7)}
\textsuperscript{48:2.24} Los clasificadores selectivos realizan también un gran servicio agrupando a las personalidades morontiales a efectos de estudio, enseñanza y otros proyectos. Indican de forma natural cuáles son los seres que trabajarán mejor en asociación temporal.

\par
%\textsuperscript{(544.8)}
\textsuperscript{48:2.25} 7. \textit{Los Registradores Asociados}. El mundo morontial posee sus propios registradores, los cuales sirven en asociación con los registradores espirituales en la tarea de supervisar y custodiar los archivos y otros datos autóctonos de las creaciones morontiales. Los archivos morontiales están a la disposición de todas las órdenes de personalidades.

\par
%\textsuperscript{(545.1)}
\textsuperscript{48:2.26} Todos los reinos morontiales de transición son accesibles de la misma manera a los seres materiales y espirituales. Como progresores morontiales, permaneceréis en pleno contacto con el mundo material y con las personalidades materiales, mientras que discerniréis y fraternizaréis cada vez más con los seres espirituales; y en el momento de despediros del régimen morontial, habréis visto a todas las órdenes de espíritus, a excepción de algunos tipos superiores tales como los Mensajeros Solitarios.

\section*{3. Los compañeros morontiales}
\par
%\textsuperscript{(545.2)}
\textsuperscript{48:3.1} Estos anfitriones de los mundos de las mansiones y de los mundos morontiales son la progenie del Espíritu Madre de un universo local. Son creados de era en era en grupos de cien mil, y en Nebadon hay actualmente más de setenta mil millones de estos seres excepcionales.

\par
%\textsuperscript{(545.3)}
\textsuperscript{48:3.2} Los Compañeros Morontiales son entrenados para el servicio por los Melquisedeks en un planeta especial cerca de Salvington; no pasan por las escuelas centrales de los Melquisedeks. Su servicio se extiende desde los mundos de las mansiones más humildes de los sistemas hasta las esferas de estudio superiores de Salvington, pero raras veces se les encuentra en los mundos habitados. Sirven bajo la supervisión general de los Hijos de Dios y bajo la dirección inmediata de los Melquisedeks.

\par
%\textsuperscript{(545.4)}
\textsuperscript{48:3.3} Los Compañeros Morontiales mantienen diez mil sedes en un universo local ---en cada primer mundo de las mansiones de los sistemas locales. Son una orden casi enteramente autónoma y forman, en general, un grupo de seres inteligentes y leales; pero de vez en cuando, en conexión con ciertos disturbios celestiales desafortunados, se ha sabido que se han descarriado. Durante los tiempos de la rebelión de Lucifer en Satania se perdieron miles de estas útiles criaturas. Vuestro sistema local posee ahora su contingente completo de estos seres, pues las pérdidas debidas a la rebelión de Lucifer sólo se han compensado recientemente.

\par
%\textsuperscript{(545.5)}
\textsuperscript{48:3.4} Hay dos tipos distintos de Compañeros Morontiales; un tipo es dinámico y el otro reservado, pero por lo demás su estatus es equivalente. No son criaturas sexuadas, pero manifiestan un afecto conmovedoramente hermoso el uno por el otro. Aunque no llegan a cohabitar en el sentido material
(humano), son parientes muy cercanos de las razas humanas en la orden de existencia de las criaturas. Las criaturas intermedias de los mundos son vuestros parientes más cercanos; luego vienen los querubines morontiales y después de ellos los Compañeros Morontiales.

\par
%\textsuperscript{(545.6)}
\textsuperscript{48:3.5} Estos compañeros son unos seres conmovedoramente afectuosos y encantadoramente sociales. Poseen personalidades diferentes, y cuando los conozcáis en los mundos de las mansiones, después de aprender a reconocerlos como clase, pronto discerniréis su individualidad. Todos los mortales se parecen unos a otros; y al mismo tiempo, cada uno de vosotros posee una personalidad distinta y reconocible.

\par
%\textsuperscript{(545.7)}
\textsuperscript{48:3.6} Se puede obtener una idea de la naturaleza del trabajo de estos Compañeros Morontiales partiendo de la siguiente clasificación de sus actividades en un sistema local:

\par
%\textsuperscript{(545.8)}
\textsuperscript{48:3.7} 1. \textit{Los Guardianes de los Peregrinos} no tienen asignada una tarea específica en su asociación con los progresores morontiales. Estos compañeros son los responsables de toda la carrera morontial y son, por consiguiente, los que coordinan el trabajo de todos los otros ministros morontiales y de transición.

\par
%\textsuperscript{(546.1)}
\textsuperscript{48:3.8} 2. \textit{Los Receptores de los Peregrinos y los Asociadores Libres}. Son los compañeros sociales de los que acaban de llegar a los mundos de las mansiones. Uno de ellos estará ciertamente allí para daros la bienvenida cuando os despertéis del primer sueño de tránsito del tiempo en el mundo inicial de las mansiones, cuando experimentéis la resurrección a la vida morontial después de la muerte en la carne. Y desde el momento en que seáis debidamente recibidos así cuando os despertéis hasta el día en que dejéis el universo local como espíritus de la primera fase, estos Compañeros Morontiales estarán siempre con vosotros.

\par
%\textsuperscript{(546.2)}
\textsuperscript{48:3.9} Los compañeros no son asignados de forma permanente a los individuos. Un mortal ascendente que esté en uno de los mundos de las mansiones o en un mundo superior puede tener un compañero diferente en cada una de las diversas ocasiones sucesivas, y por otra parte puede pasar largos períodos de tiempo sin ninguno. Todo dependerá de las necesidades y también de la oferta de compañeros disponibles.

\par
%\textsuperscript{(546.3)}
\textsuperscript{48:3.10} 3. \textit{Los Anfitriones de los Visitantes Celestiales}. Estas amables criaturas se dedican a entretener a los grupos superhumanos de visitantes estudiantiles y a otros seres celestiales que pueden encontrarse en los mundos de transición. Tendréis amplias ocasiones de visitar cualquier reino que hayáis alcanzado por experiencia. Los visitantes estudiantiles tienen permiso para ir a todos los planetas habitados, incluidos aquellos que están aislados.

\par
%\textsuperscript{(546.4)}
\textsuperscript{48:3.11} 4. \textit{Los Coordinadores y los Directores de Enlace}. Estos compañeros se dedican a facilitar las relaciones morontiales y a impedir las confusiones. Son los instructores de la conducta social y del progreso morontial, patrocinando clases y otras actividades de grupo entre los mortales ascendentes. Mantienen amplias zonas donde reúnen a sus alumnos y, de vez en cuando, solicitan a los artesanos celestiales y a los directores de la reversión que embellezcan sus programas. A medida que progreséis entraréis en contacto íntimo con estos compañeros, y os encariñaréis profundamente con los dos grupos. Estaréis asociados al azar con un compañero o bien de tipo dinámico o bien de tipo reservado.

\par
%\textsuperscript{(546.5)}
\textsuperscript{48:3.12} 5. \textit{Los Intérpretes y los Traductores}. Durante vuestra carrera inicial en las mansonias, tendréis que recurrir con frecuencia a los intérpretes y traductores. Éstos conocen y hablan todas las lenguas de un universo local; son los ling\"uistas de los reinos.

\par
%\textsuperscript{(546.6)}
\textsuperscript{48:3.13} Los nuevos idiomas no los adquiriréis de manera automática; allí aprenderéis un idioma de forma muy similar a como lo hacéis aquí, y estos seres brillantes serán vuestros profesores de idiomas. El primer estudio en los mundos de las mansiones será la lengua de Satania y luego el idioma de Nebadon. Y mientras domináis estas nuevas lenguas, los Compañeros Morontiales serán vuestros intérpretes eficaces y vuestros pacientes traductores. En ninguno de estos mundos encontraréis nunca a un visitante a quien no pueda servir de intérprete algún Compañero Morontial.

\par
%\textsuperscript{(546.7)}
\textsuperscript{48:3.14} 6. \textit{Los Supervisores de las Excursiones y de la Reversión}. Estos compañeros os acompañarán durante los viajes más largos a la esfera sede y a los mundos de cultura de transición que la rodean. Planifican, dirigen y supervisan todas estas giras individuales y colectivas alrededor de los mundos formativos y culturales del sistema.

\par
%\textsuperscript{(546.8)}
\textsuperscript{48:3.15} 7. \textit{Los Guardianes de las Superficies y de los Edificios}. Incluso las estructuras materiales y morontiales crecen en perfección y en grandiosidad a medida que avanzáis en la carrera de las mansonias. Como individuos y como grupos, tenéis permiso para efectuar ciertos cambios en las moradas asignadas como domicilios para vuestra estancia en los diferentes mundos de las mansiones. Muchas actividades de estas esferas tienen lugar en los recintos abiertos de los círculos, cuadrados y triángulos diversamente indicados. La mayoría de las estructuras de los mundos de las mansiones no tienen techo, tratándose de unos recintos con una construcción magnífica y un embellecimiento exquisito. Las condiciones climáticas y las otras condiciones físicas que predominan en los mundos arquitectónicos hacen que los techos sean totalmente innecesarios.

\par
%\textsuperscript{(547.1)}
\textsuperscript{48:3.16} Estos guardianes de las fases de transición de la vida ascendente gestionan de forma suprema los asuntos morontiales. Fueron creados para este trabajo, y hasta que el Ser Supremo no se convierta en un hecho, seguirán siendo siempre Compañeros Morontiales; nunca realizan otras funciones.

\par
%\textsuperscript{(547.2)}
\textsuperscript{48:3.17} A medida que los sistemas y los universos se establecen en la luz y la vida, los mundos de las mansiones dejan gradualmente de funcionar como esferas de transición de formación morontial. Los finalitarios establecen cada vez más su nuevo régimen educativo, que parece estar diseñado para trasladar la conciencia cósmica desde el nivel actual del gran universo al de los futuros universos exteriores. Los Compañeros Morontiales están destinados a trabajar cada vez más en asociación con los finalitarios y en otros numerosos reinos no revelados actualmente en Urantia.

\par
%\textsuperscript{(547.3)}
\textsuperscript{48:3.18} Podéis prever que estos seres probablemente contribuirán mucho a que disfrutéis de los mundos de las mansiones, que vuestra estancia allí sea corta o larga. Y continuaréis disfrutando de ellos durante todo el camino hasta Salvington. No son técnicamente esenciales para ninguna parte de vuestra experiencia de supervivencia. Podríais alcanzar Salvington sin ellos, pero los echaríais mucho de menos. Constituyen un lujo para la personalidad en vuestra carrera ascendente en el universo local.

\section*{4. Los directores de la reversión}
\par
%\textsuperscript{(547.4)}
\textsuperscript{48:4.1} La risa alegre y el equivalente de la sonrisa son tan universales como la música. Existe un equivalente morontial y espiritual de la alegría y de la risa. La vida ascendente está dividida casi por igual entre el trabajo y la diversión ---la ausencia de obligaciones.

\par
%\textsuperscript{(547.5)}
\textsuperscript{48:4.2} Las distracciones celestiales y el humor superhumano son totalmente diferentes a sus análogos humanos, pero todos nos entregamos de hecho a una forma de los dos; en nuestro estado, hacen realmente por nosotros casi lo que el humor ideal es capaz de hacer por vosotros en Urantia. Los Compañeros Morontiales son unos hábiles patrocinadores de la diversión, y los directores de la reversión los apoyan con mucha habilidad.

\par
%\textsuperscript{(547.6)}
\textsuperscript{48:4.3} Tal vez comprenderíais mejor el trabajo de los directores de la reversión si los comparáramos con los tipos superiores de humoristas de Urantia, aunque ésta sería una manera extremadamente burda, y un poco desacertada, de intentar transmitiros una idea de la actividad de estos directores del cambio y de la distracción, de estos ministros del humor elevado de los reinos morontiales y espirituales.

\par
%\textsuperscript{(547.7)}
\textsuperscript{48:4.4} Al hablar del humor espiritual, dejadme deciros en primer lugar aquello que \textit{no} es. La broma espiritual nunca tiene el matiz de acentuar las desgracias de los débiles o de los equivocados. Nunca es tampoco una blasfemia contra la rectitud y la gloria de la divinidad. Nuestro humor abarca tres niveles generales de apreciación:

\par
%\textsuperscript{(547.8)}
\textsuperscript{48:4.5} 1. \textit{Las bromas reminiscentes}. Las ocurrencias derivadas de los recuerdos de los episodios pasados de nuestra experiencia llena de combates, de luchas, a veces de temores, y a menudo de ridículas ansiedades infantiles. Para nosotros, esta fase del humor procede de la capacidad profundamente arraigada y permanente de recurrir al pasado para buscar los recuerdos con los que sazonar de manera agradable las pesadas cargas del presente y aliviarlas de otras maneras.

\par
%\textsuperscript{(548.1)}
\textsuperscript{48:4.6} 2. \textit{El humor corriente}. La insensatez de muchas cosas que nos causan tan a menudo graves preocupaciones, la alegría de descubrir la insignificancia de una gran parte de nuestras graves ansiedades personales. Apreciamos mucho mejor esta fase del humor cuando somos más capaces de disminuir las ansiedades del presente en favor de las certezas del futuro.

\par
%\textsuperscript{(548.2)}
\textsuperscript{48:4.7} 3. \textit{La alegría profética}. A los mortales quizás les resulte difícil imaginar esta fase del humor, pero obtenemos una satisfacción particular de la seguridad de que <<todas las cosas trabajan juntas para el bien>>\footnote{\textit{Todas las cosas trabajan juntas}: Ro 8:28.} ---para los espíritus y los seres morontiales, así como para los mortales. Este aspecto del humor celestial surge de nuestra fe en los cuidados amorosos de nuestros superiores y en la estabilidad divina de nuestros Directores Supremos.

\par
%\textsuperscript{(548.3)}
\textsuperscript{48:4.8} Pero los directores de la reversión de los reinos no se ocupan exclusivamente de describir el humor elevado de las diversas órdenes de seres inteligentes; también se dedican a dirigir las diversiones, las distracciones espirituales y el entretenimiento morontial. En este terreno cuentan con la cooperación cordial de los artesanos celestiales.

\par
%\textsuperscript{(548.4)}
\textsuperscript{48:4.9} Los mismos directores de la reversión no son un grupo creado; son un cuerpo reclutado que engloba a unos seres que van desde los nativos de Havona, pasando por las huestes de mensajeros del espacio y los espíritus ministrantes del tiempo, hasta los progresores morontiales de los mundos evolutivos. Todos son voluntarios, y se dedican a la tarea de ayudar a sus compañeros a conseguir cambiar de pensamiento y descansar la mente, pues estas actitudes son muy útiles para recuperar las energías agotadas.

\par
%\textsuperscript{(548.5)}
\textsuperscript{48:4.10} Cuando se está parcialmente agotado por los esfuerzos para conseguir los objetivos, y mientras se espera recibir nuevas cargas de energía, existe un agradable placer en revivir los actos de otros tiempos y de otras eras. \textit{Esrelajante recordar las experiencias iniciales de la raza o de la orden}. Y ésta es exactamente la razón por la que estos artistas se llaman directores de la reversión ---ayudan a que la memoria regrese a un antiguo estado de desarrollo o a una condición en la que el ser tenía menos experiencia.

\par
%\textsuperscript{(548.6)}
\textsuperscript{48:4.11} Todos los seres disfrutan de este tipo de reversión salvo aquellos que son Creadores intrínsecos, de ahí que rejuvenezcan de forma automática, y ciertos tipos de criaturas sumamente especializadas tales como los centros del poder y los controladores físicos, cuyas reacciones son siempre y eternamente totalmente prácticas. Estos alivios periódicos de la tensión de los deberes funcionales forman parte habitual de la vida en todos los mundos de todo el universo de universos, pero no en la Isla del Paraíso. Los seres autóctonos de la morada central son incapaces de agotarse, y por tanto no tienen necesidad de recargarse de energía. Para estos seres dotados de la perfección eterna del Paraíso no puede haber este tipo de reversión a las experiencias evolutivas.

\par
%\textsuperscript{(548.7)}
\textsuperscript{48:4.12} La mayoría de nosotros nos hemos elevado desde los estados inferiores de existencia o a través de los niveles progresivos de nuestras órdenes, y recordar ciertos episodios de nuestra experiencia inicial es reconfortante y, en cierto modo, divertido. Es relajante contemplar aquello que pertenece al pasado de nuestra propia orden, y que subsiste como recuerdo en poder de la mente. El futuro significa lucha y progreso; representa trabajo, esfuerzos y logros; pero el pasado tiene el sabor de las cosas ya dominadas y conseguidas; la contemplación del pasado permite relajarse y analizarlo de manera tan despreocupada como para provocar la risa espiritual y un estado mental morontial que raya en la alegría.

\par
%\textsuperscript{(548.8)}
\textsuperscript{48:4.13} Incluso el humor humano se vuelve muy cordial cuando describe episodios que afectan a aquellos que están un poco por debajo de nuestro estado de desarrollo actual, o cuando presenta a nuestros supuestos superiores cayendo víctimas de las experiencias generalmente asociadas a los supuestos inferiores. Vosotros, los de Urantia, habéis permitido que muchas cosas que son al mismo tiempo crueles y vulgares se confundan con vuestro humor, pero en general, se os puede felicitar por vuestro sentido relativamente agudo del humor. Algunas razas vuestras poseen una rica vena de humor que las ayuda considerablemente en sus carreras terrenales. Al parecer, una gran parte del humor lo habéis recibido de vuestra herencia adámica, mucho más de lo que habéis obtenido tanto en música como en arte.

\par
%\textsuperscript{(549.1)}
\textsuperscript{48:4.14} Durante los períodos de entretenimiento, durante esos períodos en que los habitantes del sistema resucitan de manera refrescante los recuerdos de un estado inferior de existencia, toda Satania se edifica con el humor agradable de un cuerpo de directores de la reversión procedente de Urantia. El sentido del humor celestial nos acompaña siempre, incluso cuando estamos ocupados en la más difícil de las misiones. Ayuda a evitar que la noción de nuestra propia importancia se desarrolle con exceso. Pero no le damos rienda suelta libremente, no <<lo pasamos bien>>, como diríais vosotros, salvo cuando estamos apartados de las serias tareas de nuestras órdenes respectivas.

\par
%\textsuperscript{(549.2)}
\textsuperscript{48:4.15} Cuando sentimos la tentación de exagerar nuestra propia importancia, si nos detenemos a contemplar la infinidad de la grandeza y de la nobleza de nuestros Hacedores, nuestra propia glorificación se vuelve supremamente ridícula, rayando incluso en lo humorístico. Una de las funciones del humor es la de ayudarnos a todos a tomarnos menos en serio. \textit{El humor es el antídoto divino contra la exaltación del ego}\footnote{\textit{Rebajar el ego}: Ro 12:3; 2 Co 12:7; Gl 6:3.}.

\par
%\textsuperscript{(549.3)}
\textsuperscript{48:4.16} La necesidad de distraerse y de divertirse por medio del humor es mayor en aquellas órdenes de seres ascendentes que están sometidas a una tensión continua en sus luchas por elevarse. Los dos extremos de la vida tienen poca necesidad de diversiones humorísticas. Los hombres primitivos no tienen capacidad para ellas, y los seres perfectos del Paraíso no las necesitan. Las huestes de Havona son por naturaleza un conjunto alegre y animado de personalidades supremamente felices. En el Paraíso, la calidad de la adoración obvia la necesidad de las actividades de reversión. Pero para aquellos que empiezan su carrera muy por debajo de la meta de la perfección paradisiaca, hay mucho sitio para el ministerio de los directores de la reversión.

\par
%\textsuperscript{(549.4)}
\textsuperscript{48:4.17} Cuanto más elevada es la especie humana, mayor es la tensión y mayor es la capacidad para el humor, así como la necesidad de recurrir a él. En el mundo espiritual es cierto lo contrario: cuanto más ascendemos, menos necesitamos las diversiones de las experiencias de la reversión. Pero cuando se desciende la escala de la vida espiritual desde el Paraíso hasta las huestes seráficas, existe una necesidad creciente de la misión de la risa y del ministerio de la diversión. Los seres que más necesitan la acción refrescante de la reversión periódica al estado intelectual de sus experiencias anteriores son los tipos superiores de las especies humanas, los morontianos, los ángeles y los Hijos Materiales, junto con todos los tipos similares de personalidades.

\par
%\textsuperscript{(549.5)}
\textsuperscript{48:4.18} El humor debería funcionar como una válvula automática de seguridad para impedir la acumulación de las presiones excesivas debidas a la monotonía de la contemplación seria y continua de sí mismo, asociada a la intensa lucha por el progreso para desarrollarse y por alcanzar noblemente los objetivos. El humor también funciona para disminuir el choque del impacto inesperado de los hechos o de la verdad, de los hechos rígidos e inflexibles y de la verdad flexible y siempre viva. La personalidad mortal, que nunca está segura de lo próximo que se va a encontrar, capta rápidamente a través del humor ---ve la cuestión y consigue perspicacia--- la naturaleza inesperada de la situación, ya se trate de un hecho o de una verdad.

\par
%\textsuperscript{(549.6)}
\textsuperscript{48:4.19} Aunque el humor de Urantia es extremadamente rudimentario y muy poco artístico, cumple una valiosa finalidad como seguro de salud y como liberador de las presiones emocionales, impidiendo así las tensiones nerviosas perjudiciales y la contemplación demasiado seria de sí mismo. El humor y el entretenimiento ---la distracción--- nunca son las reacciones de un esfuerzo progresivo; siempre son los ecos de una mirada hacia atrás, una reminiscencia del pasado. Incluso tal como sois actualmente en Urantia, siempre encontráis rejuvenecedor el poder suspender durante un corto período de tiempo el empleo de los esfuerzos intelectuales nuevos y más intensos, y volver a las ocupaciones más simples de vuestros antepasados.

\par
%\textsuperscript{(550.1)}
\textsuperscript{48:4.20} Los principios de la vida recreativa urantiana son filosóficamente válidos y continúan aplicándose durante toda vuestra vida ascendente, a través de los circuitos de Havona hasta las orillas eternas del Paraíso. Como seres ascendentes, poseéis los recuerdos personales de todas vuestras existencias anteriores e inferiores, y sin estos recuerdos que vuestra identidad tiene del pasado no existiría ninguna base para el humor del presente, ya se trate de la risa de los mortales o de la alegría morontial. Este recuerdo de las experiencias pasadas es el que proporciona la base para la diversión y el regocijo del presente. Así pues, disfrutaréis de los equivalentes celestiales de vuestro humor terrestre durante todo el camino ascendente de vuestra carrera morontial, y luego de vuestra carrera cada vez más espiritual. Y esa parte de Dios (el Ajustador) que se convierte en una parte eterna de la personalidad de un mortal ascendente aporta las notas de la divinidad a las expresiones gozosas, e incluso a la risa espiritual, de las criaturas ascendentes del tiempo y del espacio.

\section*{5. Los educadores de los mundos de las mansiones}
\par
%\textsuperscript{(550.2)}
\textsuperscript{48:5.1} Los Educadores de los Mundos de las Mansiones son un cuerpo de querubines y de sanobines abandonados pero glorificados. Cuando un peregrino del tiempo avanza desde un mundo de prueba del espacio hasta los mundos de las mansiones y los mundos asociados de formación morontial, va acompañado de su serafín personal o colectivo, el guardián del destino. En los mundos de la existencia mortal, el serafín recibe la hábil ayuda de los querubines y los sanobines; pero cuando su pupilo mortal es liberado de las cadenas de la carne y emprende la carrera ascendente, cuando empieza la vida postmaterial o morontial, el serafín acompañante ya no tiene necesidad del servicio de sus antiguos lugartenientes, el querubín y el sanobín.

\par
%\textsuperscript{(550.3)}
\textsuperscript{48:5.2} Estos ayudantes abandonados de los serafines ministrantes son convocados con frecuencia a la sede del universo, donde pasan por el abrazo íntimo del Espíritu Madre del Universo, y luego salen hacia las esferas formativas del sistema como Educadores de los Mundos de las Mansiones. Estos instructores visitan a menudo los mundos materiales y ejercen su actividad desde los mundos de las mansiones más inferiores hasta las esferas educativas más superiores asociadas a la sede del universo. Pueden regresar por su propia iniciativa a su antiguo trabajo asociativo con los serafines ministrantes.

\par
%\textsuperscript{(550.4)}
\textsuperscript{48:5.3} Hay millones y millones de estos educadores en Satania, y su número aumenta constantemente porque, en la mayoría de los casos, cuando un serafín avanza hacia el interior con un mortal fusionado con el Ajustador, deja atrás a un querubín y a un sanobín.

\par
%\textsuperscript{(550.5)}
\textsuperscript{48:5.4} Los Educadores de los Mundos de las Mansiones, al igual que la mayoría de los otros instructores, son nombrados por los Melquisedeks. Están generalmente supervisados por los Compañeros Morontiales, pero como individuos y como educadores se encuentran bajo la supervisión de los jefes en funciones de las escuelas o esferas donde ejercen como instructores.

\par
%\textsuperscript{(550.6)}
\textsuperscript{48:5.5} Estos querubines ascendidos trabajan habitualmente en parejas, tal como lo hacían cuando estaban vinculados al serafín. Están por naturaleza muy cerca del tipo morontial de existencia, son los educadores inherentemente comprensivos de los mortales ascendentes y dirigen muy eficazmente el programa de los mundos de las mansiones y del sistema educativo morontial.

\par
%\textsuperscript{(551.1)}
\textsuperscript{48:5.6} En las escuelas de la vida morontial, estos educadores se ocupan de enseñar a los individuos, los grupos, las clases y las masas. En los mundos de las mansiones, estas escuelas están organizadas en tres grupos generales de cien divisiones cada uno: las escuelas de pensamiento, las escuelas de sentimiento y las escuelas de acción. Cuando llegáis a la constelación se añaden las escuelas de ética, las escuelas de administración y las escuelas de adaptación social. En los mundos sede del universo entraréis en las escuelas de filosofía, de divinidad y de espiritualidad pura.

\par
%\textsuperscript{(551.2)}
\textsuperscript{48:5.7} Aquellas cosas que podríais haber aprendido en la Tierra, pero que no lograsteis aprender, deben ser adquiridas bajo la tutela de estos fieles y pacientes educadores. No existen caminos reales, ni atajos ni senderos fáciles para alcanzar el Paraíso. Independientemente de las variaciones individuales de itinerario, domináis las lecciones de una esfera antes de pasar a otra; al menos esto es así una vez que habéis dejado vuestro mundo de nacimiento.

\par
%\textsuperscript{(551.3)}
\textsuperscript{48:5.8} Uno de los objetivos de la carrera morontial consiste en erradicar de manera permanente en los supervivientes mortales aquellas características rudimentarias animales tales como la postergación, la ambig\"uedad, la falta de sinceridad, el eludir los problemas, la injusticia y la búsqueda de la facilidad. La vida en las mansonias enseña muy pronto a los jóvenes alumnos morontiales que posponer no significa en ningún sentido evitar. Después de la vida en la carne, ya no se dispone del factor tiempo como técnica para esquivar las situaciones o para evitar las obligaciones desagradables.

\par
%\textsuperscript{(551.4)}
\textsuperscript{48:5.9} Los Educadores de los Mundos de las Mansiones empiezan a servir en las esferas de detención más inferiores, y luego avanzan, por medio de la experiencia, a través de las esferas educativas del sistema y de la constelación hasta los mundos formativos de Salvington. No están sometidos a ninguna disciplina especial ni antes ni después de ser abrazados por el Espíritu Madre del Universo. Ya han sido entrenados para su trabajo mientras servían como asociados seráficos en los mundos nativos de sus alumnos que ahora residen en los mundos de las mansiones. Han tenido una experiencia efectiva con estos mortales progresivos en los mundos habitados. Son unos educadores prácticos y compasivos, unos instructores sabios y comprensivos, unos guías capaces y eficaces. Están totalmente familiarizados con los planes ascendentes y poseen una gran experiencia en las fases iniciales de la carrera de progresión.

\par
%\textsuperscript{(551.5)}
\textsuperscript{48:5.10} Muchos de estos educadores más antiguos, aquellos que han servido durante mucho tiempo en los mundos del circuito de Salvington, son abrazados de nuevo por el Espíritu Madre del Universo, y estos querubines y sanobines surgen de este segundo abrazo con la categoría de serafines.

\section*{6. Los serafines de los mundos morontiales ---los ministros de transición}
\par
%\textsuperscript{(551.6)}
\textsuperscript{48:6.1} Aunque todas las órdenes de ángeles, desde los ayudantes planetarios hasta los serafines supremos, sirven en los mundos morontiales, los ministros de transición son los que están asignados con más exclusividad a estas actividades. Estos ángeles pertenecen a la sexta orden de servidores seráficos, y su ministerio está dedicado a facilitar el tránsito de las criaturas materiales y mortales entre la vida temporal en la carne y las primeras etapas de la existencia morontial en los siete mundos de las mansiones.

\par
%\textsuperscript{(551.7)}
\textsuperscript{48:6.2} Deberíais comprender que la vida morontial de un mortal ascendente empieza en realidad en los mundos habitados en el momento de concebirse el alma, en ese instante en que la mente de la criatura con estatus moral es habitada por el Ajustador espiritual. Desde ese momento en adelante, el alma mortal posee la capacidad potencial de actuar de manera supermortal, e incluso de ser reconocida en los niveles superiores de las esferas morontiales del universo local.

\par
%\textsuperscript{(552.1)}
\textsuperscript{48:6.3} Sin embargo, no seréis conscientes del ministerio de los serafines de transición hasta que no lleguéis a los mundos de las mansiones, donde trabajan incansablemente por el progreso de sus alumnos mortales, siendo destinados a servir en las siete divisiones siguientes:

\par
%\textsuperscript{(552.2)}
\textsuperscript{48:6.4} 1. \textit{Los Evángeles Seráficos}. En el momento en que recuperáis la conciencia en los mundos de las mansiones, sois clasificados en los registros del sistema como espíritus en evolución. Es verdad que todavía no sois verdaderos espíritus, pero ya no sois seres mortales o materiales; habéis emprendido la carrera preespiritual y habéis sido debidamente admitidos en la vida morontial.

\par
%\textsuperscript{(552.3)}
\textsuperscript{48:6.5} En los mundos de las mansiones, los evángeles seráficos os ayudarán a elegir sabiamente entre las rutas opcionales hacia Edentia, Salvington, Uversa y Havona. Si existen varias rutas igualmente aconsejables, os las mostrarán, y tendréis permiso para elegir la que más os atraiga. Estos serafines presentan luego sus sugerencias a los veinticuatro consejeros que están en Jerusem sobre el camino que sería más ventajoso para cada alma ascendente.

\par
%\textsuperscript{(552.4)}
\textsuperscript{48:6.6} No se os ofrece una elección sin restricciones en cuanto a vuestro futuro camino; pero podéis elegir dentro de los límites de lo que los ministros de transición y sus superiores determinan sabiamente como lo más adecuado para vuestra consecución espiritual futura. El mundo espiritual está gobernado por el principio de respetar la elección de vuestro libre albedrío, a condición de que el camino que escojáis no sea perjudicial para vosotros o nocivo para vuestros compañeros.

\par
%\textsuperscript{(552.5)}
\textsuperscript{48:6.7} Estos evángeles seráficos se dedican a proclamar el evangelio de la progresión eterna, el triunfo del logro de la perfección. En los mundos de las mansiones proclaman la gran ley de la conservación y del predominio de la bondad: ninguna buena acción se pierde nunca por completo; puede ser frustrada durante mucho tiempo, pero nunca es totalmente anulada, y es eternamente poderosa en proporción a la divinidad de su motivación.

\par
%\textsuperscript{(552.6)}
\textsuperscript{48:6.8} Incluso en Urantia, los evángeles aconsejan a los maestros humanos de la verdad y de la rectitud que se adhieran a la predicación de <<la bondad de Dios que conduce al arrepentimiento>>\footnote{\textit{La bondad de Dios que conduce al ...}: Ro 2:4.}, a proclamar <<el amor de Dios que expulsa todo temor>>\footnote{\textit{El amor de Dios que expulsa todo temor}: 1 Jn 4:18.}. Así es como estas verdades han sido declaradas en vuestro mundo\footnote{\textit{Salmo 23}: Sal 23:1-6.}:

\par
%\textsuperscript{(552.7)}
\textsuperscript{48:6.9} Los Dioses son mis guardianes; no me desviaré;

\par
%\textsuperscript{(552.8)}
\textsuperscript{48:6.10} Juntos me conducen por los hermosos senderos y el glorioso descanso de la vida eterna.

\par
%\textsuperscript{(552.9)}
\textsuperscript{48:6.11} En esta Divina Presencia no tendré necesidad de alimento ni sed de agua.

\par
%\textsuperscript{(552.10)}
\textsuperscript{48:6.12} Aunque descienda al valle de la incertidumbre o ascienda a los mundos de la duda,

\par
%\textsuperscript{(552.11)}
\textsuperscript{48:6.13} Aunque camine en soledad o con mis semejantes,

\par
%\textsuperscript{(552.12)}
\textsuperscript{48:6.14} Aunque triunfe en los coros de la luz o titubee en los lugares solitarios de las esferas,

\par
%\textsuperscript{(552.13)}
\textsuperscript{48:6.15} Tu buen espíritu me ayudará y tu ángel glorioso me confortará.

\par
%\textsuperscript{(552.14)}
\textsuperscript{48:6.16} Aunque descienda a los abismos de las tinieblas y de la misma muerte,

\par
%\textsuperscript{(552.15)}
\textsuperscript{48:6.17} No dudaré de ti ni te temeré,

\par
%\textsuperscript{(552.16)}
\textsuperscript{48:6.18} Porque sé que en la plenitud de los tiempos y en la gloria de tu nombre

\par
%\textsuperscript{(552.17)}
\textsuperscript{48:6.19} Me levantarás para sentarme contigo en las almenas de las alturas.

\par
%\textsuperscript{(553.1)}
\textsuperscript{48:6.20} Ésta es la historia que se susurró al pastorcillo durante la noche. No pudo retenerla palabra por palabra, pero basándose en sus mejores recuerdos la comunicó poco más o menos tal como se conserva hoy.

\par
%\textsuperscript{(553.2)}
\textsuperscript{48:6.21} Estos serafines son también los evángeles del evangelio del logro de la perfección para todo el sistema así como para el ascendente individual. Incluso ahora, en el joven sistema de Satania, sus enseñanzas y sus planes contienen disposiciones para las épocas futuras, cuando los mundos de las mansiones hayan dejado de servir a los ascendentes mortales como trampolines para las esferas de arriba.

\par
%\textsuperscript{(553.3)}
\textsuperscript{48:6.22} 2. \textit{Los Intérpretes Raciales}. Todas las razas de seres mortales no son iguales. Es verdad que existe un modelo planetario que se manifiesta en la naturaleza y las tendencias físicas, mentales y espirituales de las diversas razas de un mundo dado; pero existen también distintos tipos raciales, y la progenie de estos diferentes tipos básicos de seres humanos está caracterizada por unas tendencias sociales muy definidas. En los mundos del tiempo, los intérpretes raciales seráficos favorecen los esfuerzos de los comisionados raciales para armonizar los diversos puntos de vista de las razas, y continúan ejerciendo su actividad en los mundos de las mansiones, donde estas mismas diferencias tienden a persistir en cierta medida. En un planeta confuso como Urantia, estos seres brillantes apenas han tenido una oportunidad favorable para actuar, pero son los hábiles sociólogos y los sabios consejeros étnicos del primer cielo.

\par
%\textsuperscript{(553.4)}
\textsuperscript{48:6.23} Deberíais reflexionar sobre la declaración acerca de <<el cielo>> y <<el cielo de los cielos>>\footnote{\textit{El cielo y el cielo de los cielos}: 1 Re 8:27; 2 Cr 2:6; 2 Cr 6:18; Neh 9:6; Sal 148:4; Dt 10:14.}. El cielo concebido por la mayoría de vuestros profetas era el primer mundo de las mansiones del sistema local. Cuando el apóstol dijo que había sido <<arrebatado hasta el tercer cielo>>\footnote{\textit{Arrebatado hasta el tercer cielo}: 2 Co 12:2.}, se refería a aquella experiencia en la que su Ajustador se había separado durante el sueño y, en ese estado insólito, efectuó una proyección hasta el tercero de los siete mundos de las mansiones. Algunos de vuestros sabios han tenido la visión del cielo más grande, <<el cielo de los cielos>>, en el que la séptuple experiencia de los mundos de las mansiones sólo era el primer cielo; el segundo era Jerusem, el tercero Edentia y sus satélites, el cuarto Salvington y las esferas educativas que lo rodean, el quinto Uversa, el sexto Havona y el séptimo el Paraíso.

\par
%\textsuperscript{(553.5)}
\textsuperscript{48:6.24} 3. \textit{Los Planificadores de la Mente}. Estos serafines se dedican a agrupar eficazmente a los seres morontiales y a organizar su trabajo en equipo en los mundos de las mansiones. Son los psicólogos del primer cielo. La mayoría de esta división especial de ministros seráficos ha tenido una experiencia anterior como ángeles guardianes de los hijos del tiempo, pero por alguna razón sus pupilos no lograron personalizarse en los mundos de las mansiones, o sobrevivieron de otra manera mediante la técnica de la fusión con el Espíritu.

\par
%\textsuperscript{(553.6)}
\textsuperscript{48:6.25} La tarea de los planificadores de la mente consiste en estudiar la naturaleza, la experiencia y el estado de las almas provistas de Ajustador que transitan por los mundos de las mansiones, y facilitar su agrupamiento con vistas a las asignaciones y al avance. Pero estos planificadores de la mente no conspiran, ni manipulan, ni se aprovechan de otras maneras de la ignorancia o de otras limitaciones de los estudiantes de los mundos de las mansiones. Son totalmente equitativos y eminentemente justos. Respetan vuestra voluntad morontial recién nacida, os consideran como seres volitivos independientes, e intentan estimular vuestro desarrollo y vuestro avance rápidos. Aquí os encontráis cara a cara con unos verdaderos amigos y unos consejeros comprensivos, unos ángeles que son realmente capaces de ayudaros <<a veros como los demás os ven>> y <<a conoceros como los ángeles os conocen>>.

\par
%\textsuperscript{(553.7)}
\textsuperscript{48:6.26} Estos serafines enseñan, incluso en Urantia, esta verdad eterna: si vuestra propia mente no os sirve bien, podéis cambiarla por la mente de Jesús de Nazaret\footnote{\textit{La mente de Jesús}: 1 Co 2:16; Flp 2:5.}, que siempre os sirve bien.

\par
%\textsuperscript{(554.1)}
\textsuperscript{48:6.27} 4. \textit{Los Consejeros Morontiales}. Estos ministros se llaman así porque tienen la misión de enseñar, dirigir y aconsejar a los mortales sobrevivientes de los mundos de origen humano, las almas en tránsito hacia las escuelas superiores de la sede del sistema. Son los educadores de aquellos que tratan de discernir la unidad experiencial de los niveles de vida divergentes, aquellos que intentan integrar los significados y unificar los valores. Ésta es la función de la filosofía en la vida humana, y de la mota en las esferas morontiales.

\par
%\textsuperscript{(554.2)}
\textsuperscript{48:6.28} La mota es más que una filosofía superior; es con respecto a la filosofía lo que dos ojos lo son con respecto a uno solo; posee un efecto estereoscópico sobre los significados y los valores. El hombre material ve el universo, por así decirlo, con un solo ojo ---plano. Los estudiantes de los mundos de las mansiones consiguen la perspectiva cósmica ---la profundidad--- superponiendo las percepciones de la vida morontial a las percepciones de la vida física. Y son capaces de enfocar con exactitud estos puntos de vista materiales y morontiales gracias, en gran medida, al ministerio incansable de sus consejeros seráficos, que enseñan con tanta paciencia a los estudiantes de los mundos de las mansiones y a los progresores morontiales. Muchos consejeros instructores de la orden suprema de serafines empezaron su carrera como asesores de las almas recién liberadas de los mortales del tiempo.

\par
%\textsuperscript{(554.3)}
\textsuperscript{48:6.29} 5. \textit{Los Técnicos}. Son los serafines que ayudan a los nuevos ascendentes a adaptarse al entorno nuevo y relativamente extraño de las esferas morontiales. La vida en los mundos de transición implica un contacto real con las energías y los materiales de los niveles físicos y morontiales y, hasta cierto punto, con las realidades espirituales. Los ascendentes deben aclimatarse a cada nuevo nivel morontial, y los técnicos seráficos los ayudan enormemente en todo esto. Estos serafines actúan como enlaces con los Supervisores del Poder Morontial y con los Controladores Físicos Maestros, y ejercen ampliamente su actividad como instructores de los peregrinos ascendentes en lo relacionado con la naturaleza de las energías que se utilizan en las esferas de transición. Sirven atravesando el espacio en caso de urgencia, y efectúan otras numerosas tareas regulares y especiales.

\par
%\textsuperscript{(554.4)}
\textsuperscript{48:6.30} 6. \textit{Los Educadores-Registradores}. Estos serafines son los registradores de las actividades fronterizas entre lo espiritual y lo físico, de las relaciones entre los hombres y los ángeles, de las operaciones morontiales de los reinos inferiores del universo. Sirven también instruyendo sobre las técnicas eficaces y vigentes que se utilizan para registrar los hechos. La reunión y la coordinación inteligentes de los datos relacionados es un arte, y este arte se intensifica con la colaboración de los artesanos celestiales, e incluso los mortales ascendentes se asocian así con los serafines registradores.

\par
%\textsuperscript{(554.5)}
\textsuperscript{48:6.31} Los registradores de todas las órdenes seráficas dedican cierta cantidad de tiempo a educar y preparar a los progresores morontiales. Estos guardianes angélicos de los hechos del tiempo son los instructores ideales de todos los buscadores de hechos. Antes de que dejéis Jerusem estaréis totalmente familiarizados con la historia de Satania y de sus 619 mundos habitados, y una gran parte de esta historia será impartida por los registradores seráficos.

\par
%\textsuperscript{(554.6)}
\textsuperscript{48:6.32} Todos estos ángeles forman parte de la cadena de registradores que se extiende desde los guardianes más humildes hasta los guardianes más elevados de los hechos del tiempo y de las verdades de la eternidad. Algún día os enseñarán a buscar la verdad así como los hechos, a desarrollar vuestra alma así como vuestra mente. Incluso ahora deberíais aprender a regar el jardín de vuestro corazón así como a buscar las áridas arenas del conocimiento. Las formas no tienen valor cuando las lecciones se han aprendido. No se puede obtener un polluelo sin un cascarón, y ningún cascarón vale nada después de que ha salido el polluelo. Pero a veces el error es tan grande, que rectificarlo por medio de la revelación podría ser fatal para aquellas verdades que emergen lentamente y que son esenciales para destruir el error por medio de la experiencia. Cuando los niños tienen sus ideales, no los suprimáis; dejadlos crecer. Y mientras aprendéis a pensar como hombres, también deberíais aprender a rezar como niños.

\par
%\textsuperscript{(555.1)}
\textsuperscript{48:6.33} La ley es la vida misma, y no las reglas de su conducta. El mal es una transgresión de la ley, no una violación de las reglas de conducta relacionadas con la vida, que \textit{es} la ley. La falsedad no es una cuestión de técnica narrativa, sino algo premeditado para desnaturalizar la verdad. La creación de nuevas imágenes basadas en hechos antiguos, la repetición de la vida de los padres en la vida de los hijos ---éstos son los triunfos artísticos de la verdad. La sombra del desvío de un cabello, premeditado con una finalidad desleal, la más mínima deformación o perversión de aquello que es un principio ---estas cosas constituyen la falsedad. Pero el fetiche de la verdad convertida en un hecho, de la verdad fosilizada, la cadena de hierro de la llamada verdad invariable, os mantiene ciegamente en un círculo cerrado de hechos muertos. Uno puede llevar técnicamente razón en cuanto a los hechos, y estar eternamente equivocado en cuanto a la verdad.

\par
%\textsuperscript{(555.2)}
\textsuperscript{48:6.34} 7. \textit{Las Reservas Ministrantes}. En el primer mundo de las mansiones se mantiene un cuerpo importante de todas las órdenes de serafines de transición. De todas las órdenes de serafines, y después de los guardianes del destino, estos ministros de transición son los que más se acercan a los humanos, y muchos de vuestros momentos de ocio los pasaréis con ellos. Los ángeles se deleitan con el servicio, y cuando no tienen una misión, a menudo aportan su ministerio como voluntarios. El alma de muchos mortales ascendentes se ha inflamado por primera vez con el fuego divino de la voluntad de servir gracias a una amistad personal con los servidores voluntarios de las reservas seráficas.

\par
%\textsuperscript{(555.3)}
\textsuperscript{48:6.35} De ellos aprenderéis a dejar que la presión se desarrolle en estabilidad y certidumbre; a ser fieles y serios y, al mismo tiempo, alegres; a aceptar los desafíos sin quejaros y a enfrentaros con las dificultades y las incertidumbres sin temor. Ellos os preguntarán: si fracasáis, ¿os levantaréis indomablemente para intentarlo de nuevo? Si triunfáis, ¿mantendréis un aplomo bien equilibrado ---una actitud estabilizada y espiritualizada--- durante todos los esfuerzos de la larga lucha por romper las cadenas de la inercia material, por alcanzar la libertad de la existencia espiritual?

\par
%\textsuperscript{(555.4)}
\textsuperscript{48:6.36} Al igual que los mortales, estos ángeles también han sido autores de muchas decepciones, y ellos os indicarán que a veces vuestros desengaños más decepcionantes se han convertido en vuestras mayores bendiciones. Cuando se planta una semilla, a veces se necesita que muera, que mueran vuestras esperanzas más apreciadas, antes de que pueda renacer para producir los frutos de una vida nueva y de nuevas oportunidades. De ellos aprenderéis a sufrir menos penas y decepciones, primero haciendo menos planes personales respecto a otras personalidades, y luego aceptando vuestra suerte cuando habéis cumplido fielmente con vuestro deber.

\par
%\textsuperscript{(555.5)}
\textsuperscript{48:6.37} Aprenderéis que aumentáis vuestras cargas y disminuís la posibilidad del éxito tomándoos demasiado en serio. Nada puede tener prioridad sobre el trabajo de la esfera en la que estáis ---este mundo o el siguiente. El trabajo de preparación para la siguiente esfera más elevada es muy importante, pero nada es más importante que el trabajo para el mundo en el que estáis viviendo realmente. Pero aunque el \textit{trabajo} es importante, el \textit{yo} no lo es. Cuando os sentís importantes, perdéis vuestra energía deteriorando la dignidad de vuestro ego, de manera que queda poca energía para hacer el trabajo. El engreimiento, no la importancia del trabajo, agota a las criaturas inmaduras; el elemento yo es el que agota, y no el esfuerzo por alcanzar los objetivos. Podéis hacer un trabajo importante si no os volvéis engreídos; podéis hacer diversas cosas tan fácilmente como una sola si dejáis fuera a vuestro yo. La variedad es relajante; la monotonía es la que desgasta y agota. Día tras día es lo mismo ---o bien la vida, o la alternativa de la muerte.

\section*{7. La mota morontial}
\par
%\textsuperscript{(556.1)}
\textsuperscript{48:7.1} Los planos inferiores de la mota morontial se unen directamente con los niveles superiores de la filosofía humana. En el primer mundo de las mansiones se tiene la costumbre de enseñar a los estudiantes menos avanzados por medio de la técnica comparativa, es decir, en una columna se presentan los conceptos más sencillos de los significados en mota, y en la columna contraria se mencionan las afirmaciones análogas de la filosofía humana.

\par
%\textsuperscript{(556.2)}
\textsuperscript{48:7.2} No hace mucho tiempo, mientras realizaba una misión en el primer mundo de las mansiones de Satania, tuve la ocasión de observar este método de enseñanza; y aunque no puedo presentar el contenido en mota de la lección, tengo permiso para mencionar las veintiocho afirmaciones de filosofía humana que este instructor morontial estaba utilizando como material aclaratorio destinado a ayudar a estos nuevos residentes de los mundos de las mansiones en sus primeros esfuerzos por captar la importancia y el significado de la mota. Estos ejemplos de filosofía humana eran los siguientes:

\par
%\textsuperscript{(556.3)}
\textsuperscript{48:7.3} 1. Una demostración de habilidad especializada no significa que se posea capacidad espiritual. El ingenio no sustituye al verdadero carácter.

\par
%\textsuperscript{(556.4)}
\textsuperscript{48:7.4} 2. Pocas personas viven a la altura de la fe que poseen realmente. El miedo irracional es un fraude intelectual magistral ejercido sobre el alma mortal en evolución.

\par
%\textsuperscript{(556.5)}
\textsuperscript{48:7.5} 3. Las capacidades inherentes no se pueden sobrepasar; una botella de medio litro nunca podrá contener un litro. El concepto espiritual no puede ser forzado para que entre mecánicamente en el molde de la memoria material.

\par
%\textsuperscript{(556.6)}
\textsuperscript{48:7.6} 4. Pocos mortales se atreven nunca a extraer nada similar a la cantidad de créditos establecidos para su personalidad por los ministerios combinados de la naturaleza y de la gracia. La mayoría de las almas empobrecidas son realmente ricas, pero se niegan a creerlo.

\par
%\textsuperscript{(556.7)}
\textsuperscript{48:7.7} 5. Las dificultades pueden desafiar a la mediocridad y derrotar a los temerosos, pero no hacen más que estimular a los verdaderos hijos de los Altísimos.

\par
%\textsuperscript{(556.8)}
\textsuperscript{48:7.8} 6. Disfrutar de los privilegios sin abusar, emplear la libertad sin libertinaje, poseer el poder y negarse firmemente a utilizarlo para el engrandecimiento propio ---éstos son los signos de una civilización elevada.

\par
%\textsuperscript{(556.9)}
\textsuperscript{48:7.9} 7. En el cosmos no se producen accidentes ciegos e imprevistos. Y los seres celestiales tampoco ayudan a un ser inferior que se niega a actuar según las luces que posee sobre la verdad.

\par
%\textsuperscript{(556.10)}
\textsuperscript{48:7.10} 8. El esfuerzo no siempre produce alegría, pero no existe felicidad sin un esfuerzo inteligente.

\par
%\textsuperscript{(556.11)}
\textsuperscript{48:7.11} 9. La acción consigue la fuerza; la moderación se traduce en encanto.

\par
%\textsuperscript{(556.12)}
\textsuperscript{48:7.12} 10. La rectitud hace sonar los acordes armónicos de la verdad, y la melodía vibra en todo el cosmos, e incluso la reconoce el Infinito.

\par
%\textsuperscript{(556.13)}
\textsuperscript{48:7.13} 11. Los débiles se conforman con los propósitos, pero los fuertes actúan. La vida sólo es el trabajo de un día ---hacedlo bien. El acto es nuestro; las consecuencias pertenecen a Dios.

\par
%\textsuperscript{(556.14)}
\textsuperscript{48:7.14} 12. La mayor aflicción del cosmos consiste en no haber estado nunca afligido. Los mortales sólo aprenden la sabiduría experimentando tribulaciones.

\par
%\textsuperscript{(556.15)}
\textsuperscript{48:7.15} 13. Las estrellas se disciernen mejor en el aislamiento solitario de las profundidades experienciales, y no en las cimas iluminadas y extáticas de las montañas.

\par
%\textsuperscript{(556.16)}
\textsuperscript{48:7.16} 14. Estimulad el apetito de vuestros asociados por la verdad; ofreced vuestro consejo sólo cuando os lo pidan.

\par
%\textsuperscript{(557.1)}
\textsuperscript{48:7.17} 15. La afectación es el esfuerzo ridículo de los ignorantes por parecer sabios, el intento del alma estéril por parecer rica.

\par
%\textsuperscript{(557.2)}
\textsuperscript{48:7.18} 16. No podéis percibir la verdad espiritual hasta que no la experimentéis con sensibilidad, y muchas verdades no se sienten realmente salvo en la adversidad.

\par
%\textsuperscript{(557.3)}
\textsuperscript{48:7.19} 17. La ambición es peligrosa hasta que no se socializa plenamente. No habréis adquirido realmente una virtud hasta que vuestros actos no os hagan dignos de ella.

\par
%\textsuperscript{(557.4)}
\textsuperscript{48:7.20} 18. La impaciencia es un veneno del espíritu; la ira es como una piedra que se arroja en un nido de avispas.

\par
%\textsuperscript{(557.5)}
\textsuperscript{48:7.21} 19. Hay que abandonar la ansiedad. Las decepciones más difíciles de soportar son aquellas que no llegan nunca.

\par
%\textsuperscript{(557.6)}
\textsuperscript{48:7.22} 20. Sólo un poeta puede discernir la poesía en la prosa corriente de la existencia rutinaria.

\par
%\textsuperscript{(557.7)}
\textsuperscript{48:7.23} 21. La elevada misión de cualquier arte es anunciar, mediante sus ilusiones, una realidad universal superior, cristalizar las emociones del tiempo en el pensamiento de la eternidad.

\par
%\textsuperscript{(557.8)}
\textsuperscript{48:7.24} 22. El alma en evolución no se vuelve divina por lo que hace, sino por lo que se esfuerza en hacer.

\par
%\textsuperscript{(557.9)}
\textsuperscript{48:7.25} 23. La muerte no ha añadido nada a la posesión intelectual ni a la dotación espiritual, pero ha añadido al estado experiencial la conciencia de la \textit{supervivencia}.

\par
%\textsuperscript{(557.10)}
\textsuperscript{48:7.26} 24. El destino de la eternidad se determina de momento en momento mediante los logros de la vida diaria. Los actos de hoy forman el destino de mañana.

\par
%\textsuperscript{(557.11)}
\textsuperscript{48:7.27} 25. La grandeza no reside tanto en poseer la fuerza como en hacer un uso sabio y divino de dicha fuerza.

\par
%\textsuperscript{(557.12)}
\textsuperscript{48:7.28} 26. El conocimiento sólo se posee compartiéndolo; es salvaguardado por la sabiduría y se socializa por medio del amor.

\par
%\textsuperscript{(557.13)}
\textsuperscript{48:7.29} 27. El progreso exige el desarrollo de la individualidad; la mediocridad intenta perpetuarse en la uniformidad.

\par
%\textsuperscript{(557.14)}
\textsuperscript{48:7.30} 28. Los argumentos necesarios para defender cualquier proposición son inversamente proporcionales a la verdad que contiene dicha proposición.

\par
%\textsuperscript{(557.15)}
\textsuperscript{48:7.31} Éste es el trabajo de los principiantes en el primer mundo de las mansiones, mientras que los alumnos más avanzados de los mundos siguientes van dominando los niveles superiores de la perspicacia cósmica y de la mota morontial.

\section*{8. Los progresores morontiales}
\par
%\textsuperscript{(557.16)}
\textsuperscript{48:8.1} Desde el momento de graduarse en los mundos de las mansiones hasta que alcanzan el estado espiritual en la carrera superuniversal, los mortales ascendentes son denominados progresores morontiales. Vuestro paso por esta maravillosa vida fronteriza será una experiencia inolvidable, un recuerdo encantador. Es la puerta evolutiva hacia la vida espiritual y hacia la conquista final de la perfección de las criaturas, gracias a la cual los ascendentes alcanzan la meta del tiempo ---encontrar a Dios en el Paraíso.

\par
%\textsuperscript{(557.17)}
\textsuperscript{48:8.2} Existe un propósito determinado y divino en todo este programa morontial, y posteriormente espiritual, para la progresión de los mortales, en esta detallada escuela de formación universal para las criaturas ascendentes. Los Creadores tienen la intención de proporcionar a las criaturas del tiempo una oportunidad gradual para dominar los detalles del funcionamiento y de la administración del gran universo, y este largo ciclo de formación se lleva mejor adelante haciendo que los mortales sobrevivientes asciendan gradualmente, y permitiendo que participen realmente en cada etapa de la ascensión.

\par
%\textsuperscript{(558.1)}
\textsuperscript{48:8.3} El plan de supervivencia de los mortales tiene un objetivo práctico y útil; no sois los destinatarios de toda esta labor divina y de todo este esmerado entrenamiento sólo para que podáis sobrevivir y disfrutar de una felicidad sin fin y de un descanso eterno. Existe una meta de servicio trascendente oculta más allá del horizonte de la presente era del universo. Si los Dioses simplemente hubieran planeado llevaros a una larga excursión de alegría eterna, ciertamente no habrían transformado en tan gran medida todo el universo en una inmensa y compleja escuela de educación práctica, no habrían requisado una parte considerable de la creación celestial como maestros e instructores, y luego pasar eras y eras guiándoos, uno a uno, a través de esta gigantesca escuela universal de educación experiencial. Fomentar el programa de la progresión de los mortales parece ser una de las ocupaciones principales del actual universo organizado, y la mayoría de las innumerables órdenes de inteligencias creadas están ocupadas, directa o indirectamente, en hacer avanzar alguna fase de este plan progresivo de perfección.

\par
%\textsuperscript{(558.2)}
\textsuperscript{48:8.4} Al atravesar la escala ascendente de la existencia viviente desde el hombre mortal hasta el abrazo de la Deidad, vivís realmente la vida misma de todas las fases y etapas posibles de la existencia perfeccionada de las criaturas dentro de los límites de la presente era del universo. Aquello que hay desde el hombre mortal hasta el finalitario del Paraíso abarca todo lo que puede existir ahora ---engloba todo lo que es posible actualmente para las órdenes vivientes de criaturas finitas inteligentes y perfeccionadas. Si el destino futuro de los finalitarios del Paraíso es servir en los nuevos universos ahora en gestación, es seguro que esta nueva creación futura no contendrá órdenes creadas de seres experienciales cuyas vidas serán totalmente diferentes a las que los finalitarios mortales habrán vivido en algún mundo como parte de su formación ascendente, como una de las etapas de su progreso milenario desde el animal hasta el ángel, desde el ángel hasta el espíritu y desde el espíritu hasta Dios.

\par
%\textsuperscript{(558.3)}
\textsuperscript{48:8.5} [Presentado por un Arcángel de Nebadon.]


\chapter{Documento 49. Los mundos habitados}
\par
%\textsuperscript{(559.1)}
\textsuperscript{49:0.1} TODOS los mundos habitados por los mortales tienen un origen y una naturaleza evolutivos. Estas esferas son el semillero, la cuna evolutiva, de las razas mortales del tiempo y del espacio. Cada unidad de la vida ascendente es una verdadera escuela educativa para la fase de existencia inmediatamente siguiente, y esto es así durante todas las etapas de la ascensión progresiva del hombre hacia el Paraíso; es tan cierto para la experiencia mortal inicial en un planeta evolutivo como para la escuela final de los Melquisedeks en la sede del universo, una escuela a la que no asisten los mortales ascendentes hasta poco antes de ser trasladados al régimen del superuniverso y de alcanzar la primera fase de la existencia espiritual.

\par
%\textsuperscript{(559.2)}
\textsuperscript{49:0.2} Todos los mundos habitados están básicamente agrupados en sistemas locales para su administración celestial, y cada uno de estos sistemas locales está limitado a unos mil mundos evolutivos. Esta limitación ha sido decretada por los Ancianos de los Días, y se refiere a los planetas efectivamente evolutivos donde viven los mortales con posibilidades de sobrevivir. A este grupo no pertenecen ni los mundos definitivamente establecidos en la luz y la vida ni los planetas que se encuentran en la etapa prehumana de desarrollo de la vida.

\par
%\textsuperscript{(559.3)}
\textsuperscript{49:0.3} Satania misma es un sistema inacabado que sólo contiene 619 mundos habitados. Estos planetas están numerados de forma secuencial con arreglo a su inscripción como mundos habitados, como mundos habitados por criaturas volitivas. Así es como Urantia recibió el número \textit{606 de Satania}, lo que significa que es el 606{\textordmasculine} mundo de este sistema local donde el largo proceso evolutivo de la vida culminó con la aparición de seres humanos. Hay treinta y seis planetas no habitados que se están acercando a la etapa en que serán dotados de vida, y varios están siendo preparados ahora para los Portadores de Vida. Hay casi doscientas esferas que evolucionan de tal manera que estarán preparadas para la implantación de la vida dentro de los próximos millones de años.

\par
%\textsuperscript{(559.4)}
\textsuperscript{49:0.4} No todos los planetas son adecuados para albergar la vida de los mortales. Los planetas pequeños con una elevada velocidad de rotación axial son totalmente inadecuados como hábitats para la vida. En diversos sistemas físicos de Satania, los planetas que giran alrededor del sol central son demasiado grandes como para ser habitados, pues su gran masa produce una gravedad opresiva. Muchas de estas enormes esferas tienen satélites, a veces media docena o más, y estas lunas tienen a menudo un tamaño muy similar al de Urantia, por lo que son casi ideales para ser habitadas.

\par
%\textsuperscript{(559.5)}
\textsuperscript{49:0.5} El mundo habitado más antiguo de Satania, el mundo número uno, es Anova, uno de los cuarenta y cuatro satélites que giran alrededor de un enorme planeta oscuro, pero que está expuesto a la luz diferencial de tres soles vecinos. Anova se encuentra en un estado avanzado de civilización progresiva.

\section*{1. La vida planetaria}
\par
%\textsuperscript{(559.6)}
\textsuperscript{49:1.1} Los universos del tiempo y del espacio se desarrollan de forma gradual; la progresión de la vida ---terrestre o celestial--- no es ni arbitraria ni mágica. Puede ser que la evolución cósmica no sea siempre comprensible (previsible), pero es estrictamente no accidental.

\par
%\textsuperscript{(560.1)}
\textsuperscript{49:1.2} La unidad biológica de la vida material es la célula protoplásmica, la asociación colectiva de las energías químicas, eléctricas y otras energías básicas. Las fórmulas químicas difieren en cada sistema, y la técnica de la reproducción de las células vivientes es ligeramente diferente en cada universo local, pero los Portadores de Vida son siempre los catalizadores vivientes que inician las reacciones primordiales de la vida material; son los instigadores de los circuitos energéticos de la materia viviente.

\par
%\textsuperscript{(560.2)}
\textsuperscript{49:1.3} Todos los mundos de un sistema local revelan una similitud física inequívoca; sin embargo, cada planeta tiene su propia escala de vida, y no hay dos mundos que sean exactamente iguales en su dotación vegetal y animal. Estas variaciones planetarias de los tipos de vida del sistema son el resultado de las decisiones de los Portadores de Vida. Pero estos seres no son ni caprichosos ni antojadizos; los universos están dirigidos de acuerdo con la ley y el orden. Las leyes de Nebadon son los mandatos divinos de Salvington, y el tipo evolutivo de vida de Satania está en consonancia con el arquetipo evolutivo de Nebadon.

\par
%\textsuperscript{(560.3)}
\textsuperscript{49:1.4} La evolución es la regla del desarrollo humano, pero el proceso mismo varía enormemente en los diferentes mundos. A veces la vida es iniciada en un solo centro, a veces en tres, como fue el caso en Urantia. En los mundos atmosféricos tiene generalmente un origen marino, pero no siempre; depende mucho del estado físico de un planeta. Los Portadores de Vida tienen una gran libertad en su función de iniciar la vida.

\par
%\textsuperscript{(560.4)}
\textsuperscript{49:1.5} En el desarrollo de la vida planetaria, la forma vegetal siempre precede a la forma animal y ya está plenamente desarrollada antes de que se diferencien los modelos animales. Todos los tipos de animales se desarrollan a partir de los modelos básicos del anterior reino vegetal de seres vivientes; no están organizados por separado.

\par
%\textsuperscript{(560.5)}
\textsuperscript{49:1.6} Las etapas iniciales de la evolución de la vida no están totalmente en conformidad con vuestras ideas de hoy en día. \textit{El hombre mortal no es un accidente evolutivo}. Hay un sistema preciso, una ley universal, que determina el desarrollo del plan de la vida planetaria en las esferas del espacio. El tiempo y la producción de una gran cantidad de especies no son las influencias controladoras. Los ratones se reproducen mucho más rápidamente que los elefantes, sin embargo los elefantes evolucionan más rápidamente que los ratones.

\par
%\textsuperscript{(560.6)}
\textsuperscript{49:1.7} El proceso de la evolución planetaria es ordenado y está controlado. El desarrollo de organismos superiores a partir de agrupaciones de vida más inferiores no es accidental. A veces el progreso evolutivo se demora temporalmente debido a la destrucción de ciertas líneas favorables de plasma vital existentes en una especie seleccionada. A menudo se necesitan eras y eras para reparar el daño ocasionado por la pérdida de una sola cepa superior de herencia humana. Una vez que estas cepas seleccionadas y superiores de protoplasma viviente han hecho su aparición, deberían ser celosa e inteligentemente protegidas. En la mayor parte de los mundos habitados, estos potenciales superiores de vida son mucho más valorados que en Urantia.

\section*{2. Los tipos físicos planetarios}
\par
%\textsuperscript{(560.7)}
\textsuperscript{49:2.1} En cada sistema hay un modelo estándar y básico de vida vegetal y animal. Pero los Portadores de Vida se enfrentan a menudo con la necesidad de modificar estos modelos básicos para adaptarlos a las condiciones físicas variables que encuentran en numerosos mundos del espacio. Fomentan un tipo generalizado de criatura mortal en el sistema, pero hay siete tipos físicos distintos, así como miles y miles de variantes menores de estas siete diferenciaciones sobresalientes:

\par
%\textsuperscript{(561.1)}
\textsuperscript{49:2.2} 1. Los tipos atmosféricos.

\par
%\textsuperscript{(561.2)}
\textsuperscript{49:2.3} 2. Los tipos elementales.

\par
%\textsuperscript{(561.3)}
\textsuperscript{49:2.4} 3. Los tipos gravitatorios.

\par
%\textsuperscript{(561.4)}
\textsuperscript{49:2.5} 4. Los tipos térmicos.

\par
%\textsuperscript{(561.5)}
\textsuperscript{49:2.6} 5. Los tipos eléctricos.

\par
%\textsuperscript{(561.6)}
\textsuperscript{49:2.7} 6. Los tipos energizadores.

\par
%\textsuperscript{(561.7)}
\textsuperscript{49:2.8} 7. Los tipos innominados.

\par
%\textsuperscript{(561.8)}
\textsuperscript{49:2.9} El sistema de Satania contiene todos estos tipos y numerosos grupos intermedios, aunque algunos están muy pocos representados.

\par
%\textsuperscript{(561.9)}
\textsuperscript{49:2.10} 1. \textit{Los tipos atmosféricos}. Las diferencias físicas entre los mundos habitados por los mortales están principalmente determinadas por la naturaleza de la atmósfera; las otras influencias que contribuyen a la diferenciación planetaria de la vida son relativamente menores.

\par
%\textsuperscript{(561.10)}
\textsuperscript{49:2.11} Las condiciones atmosféricas actuales de Urantia son casi ideales para mantener al tipo de hombre respirador, pero el tipo humano se puede modificar de tal manera que puede vivir tanto en planetas superatmosféricos como subatmosféricos. Estas modificaciones también se extienden a la vida animal, la cual difiere enormemente en las diversas esferas habitadas. Las órdenes animales sufren unas modificaciones muy grandes tanto en los mundos subatmosféricos como en los superatmosféricos.

\par
%\textsuperscript{(561.11)}
\textsuperscript{49:2.12} De los tipos atmosféricos de Satania, cerca del dos y medio por ciento son subrespiradores, casi el cinco por ciento son superrespiradores, y más del noventa y uno por ciento son respiradores medios, representando en conjunto el noventa y ocho y medio por ciento de los mundos de Satania.

\par
%\textsuperscript{(561.12)}
\textsuperscript{49:2.13} Los seres tales como las razas de Urantia están clasificados como respiradores medios; representáis la orden respiradora media o típica de existencia mortal. Si existieran criaturas inteligentes en un planeta con una atmósfera similar a la de Venus, vuestro vecino más cercano, pertenecerían al grupo de los superrespiradores, mientras que los habitantes de un planeta con una atmósfera tan enrarecida como la de Marte, vuestro vecino exterior, serían denominados subrespiradores.

\par
%\textsuperscript{(561.13)}
\textsuperscript{49:2.14} Si los mortales vivieran en un planeta desprovisto de aire como vuestra Luna, pertenecerían a la orden particular de los no respiradores. Este tipo representa una adaptación radical o extrema al entorno planetario, y será examinado por separado. Los no respiradores suponen el uno y medio por ciento restante de los mundos de Satania.

\par
%\textsuperscript{(561.14)}
\textsuperscript{49:2.15} 2. \textit{Los tipos elementales}. Estas diferenciaciones tienen que ver con la relación de los mortales con el agua, el aire y la tierra, y existen cuatro especies distintas de vida inteligente según sea su relación con estos hábitats. Las razas de Urantia pertenecen a la orden terrestre.

\par
%\textsuperscript{(561.15)}
\textsuperscript{49:2.16} Es totalmente imposible que podáis imaginar el entorno que impera durante las primeras épocas de algunos mundos. Estas condiciones insólitas hacen necesario que la vida animal en evolución permanezca en su hábitat semillero marino durante unos períodos más largos que en aquellos planetas que ofrecen muy pronto un entorno terrestre y atmosférico hospitalario. Por el contrario, en algunos mundos de los superrespiradores, cuando el planeta no es demasiado grande, a veces es conveniente prever un tipo mortal que pueda franquear fácilmente el corredor atmosférico. Estos navegantes aéreos se encuentran a veces entre los grupos acuáticos y los grupos terrestres, y siempre viven en cierta medida en el suelo, llegando finalmente a residir sólo en la tierra. Pero en algunos mundos continúan volando durante eras enteras incluso después de haberse convertido en seres de tipo terrestre.

\par
%\textsuperscript{(562.1)}
\textsuperscript{49:2.17} Es asombroso y divertido a la vez observar la civilización inicial de una raza primitiva de seres humanos que va tomando forma, en unos casos en el aire y en las copas de los árboles, y en otros en medio de las aguas poco profundas de las cuencas tropicales abrigadas, así como en el fondo, en las orillas y en las costas de estos jardines marinos de las razas recién aparecidas en estas esferas extraordinarias. Incluso en Urantia hubo un largo período durante el cual el hombre primitivo se protegió e hizo progresar su civilización primitiva viviendo la mayor parte del tiempo en las copas de los árboles, tal como lo habían hecho sus antepasados arbóreos anteriores. Y en Urantia tenéis todavía un grupo de mamíferos diminutos (la familia de los murciélagos) que son navegantes aéreos, y vuestras focas y ballenas, cuyo hábitat es marino, también pertenecen a la orden de los mamíferos.

\par
%\textsuperscript{(562.2)}
\textsuperscript{49:2.18} Entre los tipos elementales de Satania, el siete por ciento son acuáticos, el diez por ciento aéreos, el setenta por ciento terrestres, y el trece por ciento son tipos terrestres y aéreos combinados. Pero estas modificaciones de las criaturas inteligentes primitivas no son ni peces humanos ni pájaros humanos. Pertenecen a los tipos humanos y prehumanos, y no son ni superpeces ni pájaros glorificados, sino claramente mortales.

\par
%\textsuperscript{(562.3)}
\textsuperscript{49:2.19} 3. \textit{Los tipos gravitatorios}. Mediante la modificación del diseño creativo, los seres inteligentes son estructurados de tal manera que pueden ejercer libremente su actividad en esferas más pequeñas o más grandes que Urantia, adaptándose así adecuadamente a la gravedad de aquellos planetas que no tienen un tamaño ni una densidad ideales.

\par
%\textsuperscript{(562.4)}
\textsuperscript{49:2.20} La altura de los diversos tipos planetarios de mortales es variable, y el término medio en Nebadon se encuentra un poco por encima de los dos metros. Algunos de los mundos más grandes están poblados por seres que sólo tienen una altura de unos setenta y cinco centímetros. La estatura de los mortales varía entre ésta última, pasando por las alturas medias en los planetas de tamaño medio, hasta alrededor de los tres metros en las esferas habitadas más pequeñas. En Satania sólo hay una raza que tiene menos de un metro veinte de altura. El veinte por ciento de los mundos habitados de Satania está poblado por mortales de los tipos gravitatorios modificados que ocupan los planetas más grandes y los más pequeños.

\par
%\textsuperscript{(562.5)}
\textsuperscript{49:2.21} 4. \textit{Los tipos térmicos}. Es posible crear seres vivientes que puedan resistir temperaturas mucho más altas o mucho más bajas que la gama vital de las razas de Urantia. Tal como están clasificados con relación a los mecanismos reguladores de la temperatura, existen cinco órdenes distintas de seres. Las razas de Urantia ocupan en esta escala el número tres. El treinta por ciento de los mundos de Satania están poblados por razas de los tipos térmicos modificados. En comparación con los urantianos, los cuales funcionan en el grupo de las temperaturas medias, el doce por ciento pertenecen a las gamas de temperatura más elevadas y el dieciocho por ciento a las más bajas.

\par
%\textsuperscript{(562.6)}
\textsuperscript{49:2.22} 5. \textit{Los tipos eléctricos}. El comportamiento eléctrico, magnético y electrónico de los mundos varía enormemente. Existen diez diseños de vida mortal adaptados de maneras diversas para resistir la energía diferencial de las esferas. Estas diez variedades también reaccionan de forma ligeramente diferente a los rayos químicos de la luz solar ordinaria. Pero estas pequeñas variaciones físicas no afectan de ninguna manera a la vida intelectual o espiritual.

\par
%\textsuperscript{(562.7)}
\textsuperscript{49:2.23} De las agrupaciones eléctricas de la vida mortal, casi el veintitrés por ciento pertenece a la clase número cuatro, el tipo de existencia urantiano. Estos tipos están distribuidos como sigue: clase número 1, uno por ciento; número 2, dos por ciento; número 3, cinco por ciento; número 4, veintitrés por ciento; número 5, veintisiete por ciento; número 6, veinticuatro por ciento; número 7, ocho por ciento; número 8, cinco por ciento; número 9, tres por ciento; número 10, dos por ciento ---en porcentajes totales.

\par
%\textsuperscript{(563.1)}
\textsuperscript{49:2.24} 6. \textit{Los tipos energizadores}. No todos los mundos son iguales en la manera de absorber la energía. No todos los mundos habitados tienen un océano atmosférico adecuado para el intercambio respiratorio de los gases, como el que está presente en Urantia. Durante las etapas iniciales y posteriores de muchos planetas, los seres de vuestra orden actual no podrían existir; cuando los factores respiratorios de un planeta son muy elevados o muy bajos, pero cuando todas las demás condiciones previas para la vida inteligente son adecuadas, los Portadores de Vida establecen a menudo en esos mundos una forma modificada de existencia mortal, unos seres que son capaces de efectuar directamente los intercambios de sus procesos vitales utilizando la energía luminosa y las transmutaciones directas del poder de los Controladores Físicos Maestros.

\par
%\textsuperscript{(563.2)}
\textsuperscript{49:2.25} Existen seis tipos diferentes de nutrición animal y humana: los subrespiradores emplean el primer tipo de nutrición, los habitantes marinos el segundo, los respiradores medios el tercero, como sucede en Urantia. Los superrespiradores emplean el cuarto tipo de absorción de la energía, mientras que los no respiradores utilizan la quinta orden de nutrición y de energía. La sexta técnica de energización está limitada a las criaturas intermedias.

\par
%\textsuperscript{(563.3)}
\textsuperscript{49:2.26} 7. \textit{Los tipos innominados}. Existen numerosas variaciones físicas adicionales en la vida planetaria, pero todas estas diferencias son enteramente cuestiones de modificaciones anatómicas, de diferenciaciones fisiológicas y de ajustes electroquímicos. Estas distinciones no afectan a la vida intelectual o espiritual.

\section*{3. Los mundos de los no respiradores}
\par
%\textsuperscript{(563.4)}
\textsuperscript{49:3.1} La mayoría de los planetas habitados están poblados por el tipo respirador de seres inteligentes. Pero existen también unas órdenes de mortales que son capaces de vivir en mundos que tienen poco o ningún aire. De los mundos habitados de Orvonton, este tipo asciende a menos del siete por ciento. En Nebadon este porcentaje es inferior al tres. En todo Satania sólo hay nueve mundos de este tipo.

\par
%\textsuperscript{(563.5)}
\textsuperscript{49:3.2} En Satania hay tan pocos mundos habitados del tipo no respirador porque esta sección de Norlatiadek, organizada más recientemente, abunda todavía en cuerpos espaciales meteóricos; y los mundos sin una atmósfera aislante protectora están sometidos al bombardeo incesante de estos vagabundos. Incluso algunos cometas están compuestos de enjambres de meteoros, pero por regla general se trata de cuerpos de materia más pequeños y desorganizados.

\par
%\textsuperscript{(563.6)}
\textsuperscript{49:3.3} Millones y millones de meteoritos penetran diariamente en la atmósfera de Urantia, entrando a una velocidad de casi trescientos veinte kilómetros por segundo. En los mundos donde no se respira, las razas avanzadas deben hacer muchas cosas para protegerse de los daños meteóricos, construyendo instalaciones eléctricas que se encargan de consumir o de desviar los meteoros. Se enfrentan a grandes peligros cuando se aventuran más allá de estas zonas protegidas. Estos mundos también están sometidos a unas tormentas eléctricas desastrosas de una naturaleza desconocida en Urantia. Durante esos períodos de enormes fluctuaciones energéticas, los habitantes deben refugiarse en sus estructuras especiales de aislamiento protector.

\par
%\textsuperscript{(563.7)}
\textsuperscript{49:3.4} La vida en los mundos de los no respiradores es radicalmente diferente a la que existe en Urantia. Los no respiradores no ingieren comida ni beben agua como lo hacen las razas de Urantia. Las reacciones del sistema nervioso, el mecanismo regulador de la temperatura y el metabolismo de estos pueblos especializados son radicalmente diferentes a estas mismas funciones en los mortales de Urantia. Aparte de la reproducción, casi todos los actos de la vida difieren, e incluso los métodos de procreación son un poco diferentes.

\par
%\textsuperscript{(564.1)}
\textsuperscript{49:3.5} En los mundos donde no se respira, las especies animales son radicalmente distintas a las que se encuentran en los planetas atmosféricos. El plan de la vida donde no se respira varía respecto a la técnica de la existencia en un mundo atmosférico; sus pueblos difieren incluso en la supervivencia, siendo candidatos a la fusión con el Espíritu. Sin embargo, estos seres disfrutan de la vida y llevan adelante las actividades del reino con las mismas dificultades y alegrías relativas que experimentan los mortales que viven en los mundos atmosféricos. En cuanto a la mente y al carácter, los no respiradores no difieren de los otros tipos de mortales.

\par
%\textsuperscript{(564.2)}
\textsuperscript{49:3.6} Estaríais más que interesados en la conducta planetaria de este tipo de mortales, porque una raza de seres de esta clase vive en una esfera muy cercana a Urantia.

\section*{4. Las criaturas volitivas evolutivas}
\par
%\textsuperscript{(564.3)}
\textsuperscript{49:4.1} Hay grandes diferencias entre los mortales de los distintos mundos, incluso entre aquellos que pertenecen a los mismos tipos intelectuales y físicos, pero todos los mortales con dignidad volitiva son animales erguidos, bípedos.

\par
%\textsuperscript{(564.4)}
\textsuperscript{49:4.2} Hay seis razas evolutivas básicas: tres primarias ---roja, amarilla y azul; y tres secundarias--- anaranjada, verde e índigo. La mayoría de los mundos habitados poseen todas estas razas, pero muchos planetas cuyas razas tienen tres cerebros sólo albergan los tres tipos primarios. Algunos sistemas locales sólo tienen también estas tres razas.

\par
%\textsuperscript{(564.5)}
\textsuperscript{49:4.3} Los seres humanos están dotados de una media de doce sentidos físicos especiales, aunque los sentidos especiales de los mortales con tres cerebros se prolongan un poco más allá de los de los tipos con uno y dos cerebros; pueden ver y oír considerablemente más que las razas de Urantia.

\par
%\textsuperscript{(564.6)}
\textsuperscript{49:4.4} Los jóvenes nacen generalmente de uno en uno, los nacimientos múltiples son una excepción, y la vida familiar es bastante uniforme en todos los tipos de planetas. La igualdad entre los sexos prevalece en todos los mundos avanzados; la dotación mental y el estado espiritual de los hombres y de las mujeres son iguales. No consideramos que un planeta ha salido de la barbarie mientras uno de los sexos trata de tiranizar al otro. Esta característica de la experiencia de las criaturas siempre mejora mucho después de la llegada de un Hijo y una Hija Materiales.

\par
%\textsuperscript{(564.7)}
\textsuperscript{49:4.5} Las variaciones de las estaciones y de las temperaturas se producen en todos los planetas iluminados y calentados por un sol. La agricultura es universal en todos los mundos atmosféricos; el cultivo de la tierra es la única ocupación común de las razas que progresan en todos estos planetas.

\par
%\textsuperscript{(564.8)}
\textsuperscript{49:4.6} En los primeros tiempos, todos los mortales tienen las mismas luchas generales contra sus enemigos microscópicos, tal como las que vosotros experimentáis actualmente en Urantia, aunque quizás no tan extensas. La duración de la vida varía en los diferentes planetas desde veinticinco años en los mundos primitivos hasta cerca de quinientos en las esferas más avanzadas y más antiguas.

\par
%\textsuperscript{(564.9)}
\textsuperscript{49:4.7} Todos los seres humanos son gregarios, tanto en sentido tribal como racial. Estas separaciones en grupos son inherentes a su origen y a su constitución. Estas tendencias sólo se pueden modificar con el avance de la civilización y una espiritualización gradual. Los problemas sociales, económicos y gubernamentales de los mundos habitados varían con arreglo a la edad de los planetas y al grado en que han sido influidos por las estancias sucesivas de los Hijos divinos.

\par
%\textsuperscript{(564.10)}
\textsuperscript{49:4.8} La mente es un don del Espíritu Infinito y funciona exactamente igual en los diversos entornos. La mente de los mortales es semejante, independientemente de ciertas diferencias estructurales y químicas que caracterizan la naturaleza física de las criaturas volitivas de los sistemas locales. Sin tener en cuenta las diferencias planetarias personales o físicas, la vida mental de todas estas diversas órdenes de mortales es muy similar, y sus carreras inmediatas después de la muerte son muy parecidas.

\par
%\textsuperscript{(565.1)}
\textsuperscript{49:4.9} Pero la mente mortal sin el espíritu inmortal no puede sobrevivir. La mente del hombre es mortal; sólo el espíritu otorgado es inmortal. La supervivencia depende de la espiritualización gracias al ministerio del Ajustador ---del nacimiento y de la evolución del alma inmortal; al menos no debe haberse desarrollado un antagonismo hacia la misión del Ajustador, la cual consiste en efectuar la transformación espiritual de la mente material.

\section*{5. Las series planetarias de mortales}
\par
%\textsuperscript{(565.2)}
\textsuperscript{49:5.1} Será un poco difícil hacer una descripción adecuada de las series planetarias de mortales, porque sabéis muy pocas cosas sobre ellos y porque hay demasiadas variaciones. Sin embargo, las criaturas mortales se pueden estudiar desde numerosos puntos de vista, entre los cuales figuran los siguientes:

\par
%\textsuperscript{(565.3)}
\textsuperscript{49:5.2} 1. La adaptación al entorno planetario.

\par
%\textsuperscript{(565.4)}
\textsuperscript{49:5.3} 2. Las series de los tipos cerebrales.

\par
%\textsuperscript{(565.5)}
\textsuperscript{49:5.4} 3. Las series receptoras al espíritu.

\par
%\textsuperscript{(565.6)}
\textsuperscript{49:5.5} 4. Las épocas planetarias de los mortales.

\par
%\textsuperscript{(565.7)}
\textsuperscript{49:5.6} 5. Las series de las criaturas emparentadas.

\par
%\textsuperscript{(565.8)}
\textsuperscript{49:5.7} 6. Las series de los que fusionan con el Ajustador.

\par
%\textsuperscript{(565.9)}
\textsuperscript{49:5.8} 7. Las técnicas para salir del planeta.

\par
%\textsuperscript{(565.10)}
\textsuperscript{49:5.9} Las esferas habitadas de los siete superuniversos están pobladas de mortales que se clasifican simultáneamente en una o más categorías de cada una de estas siete clases generalizadas de vida evolutiva de las criaturas. Pero ni siquiera en estas clasificaciones generales están previstos unos seres tales como los midsonitarios ni otras ciertas formas de vida inteligente. Los mundos habitados, tal como han sido presentados en estas narraciones, están poblados de criaturas mortales evolutivas, pero existen otras formas de vida.

\par
%\textsuperscript{(565.11)}
\textsuperscript{49:5.10} 1. \textit{La adaptación al entorno planetario}. Desde el punto de vista de la adaptación de la vida de las criaturas al entorno planetario, hay tres grupos generales de mundos habitados: el grupo de la adaptación normal, el grupo de la adaptación radical y el grupo experimental.

\par
%\textsuperscript{(565.12)}
\textsuperscript{49:5.11} Las adaptaciones normales a las condiciones planetarias siguen los modelos físicos generales anteriormente examinados. Los mundos de los no respiradores representan la adaptación radical o extrema, pero en este grupo también están incluídos otros tipos. Los mundos experimentales están idealmente adaptados en general a las formas típicas de vida, y en estos planetas decimales los Portadores de Vida intentan producir variaciones beneficiosas en los diseños estándar de vida. Puesto que vuestro mundo es un planeta experimental, difiere notablemente de sus esferas hermanas de Satania; en Urantia han aparecido muchas formas de vida que no se encuentran en otra parte; del mismo modo, muchas especies comunes están ausentes de vuestro planeta.

\par
%\textsuperscript{(565.13)}
\textsuperscript{49:5.12} En el universo de Nebadon, todos los mundos donde se ha modificado la vida están conectados en serie y constituyen un campo especial de los asuntos universales que recibe la atención de unos administradores designados; y todos estos mundos experimentales son inspeccionados periódicamente por un cuerpo de directores universales cuyo jefe es el veterano finalitario conocido en Satania con el nombre de Tabamantia.

\par
%\textsuperscript{(566.1)}
\textsuperscript{49:5.13} 2. \textit{Las series de los tipos cerebrales}. La única uniformidad física que tienen los mortales es el cerebro y el sistema nervioso; sin embargo, existen tres organizaciones básicas del mecanismo cerebral: los tipos con uno, dos o tres cerebros. Los urantianos pertenecen al tipo con dos cerebros, un poco más imaginativos, aventureros y filosóficos que los mortales con un solo cerebro, pero un poco menos espirituales, éticos y adoradores que las órdenes con tres cerebros. Estas diferencias cerebrales caracterizan incluso a las existencias animales prehumanas.

\par
%\textsuperscript{(566.2)}
\textsuperscript{49:5.14} Partiendo del tipo de corteza cerebral urantiana con dos hemisferios podéis comprender algo, por analogía, sobre el tipo con un solo cerebro. El tercer cerebro de las órdenes tricerebrales se puede concebir mejor como una evolución de vuestra forma de cerebro inferior o rudimentario, que se desarrolla hasta el punto de funcionar principalmente para controlar las actividades físicas, dejando libres a los dos cerebros superiores para tareas más elevadas: uno para las funciones intelectuales y el otro para las actividades de duplicación espiritual del Ajustador del Pensamiento.

\par
%\textsuperscript{(566.3)}
\textsuperscript{49:5.15} Mientras que los logros terrestres de las razas con un solo cerebro están ligeramente limitados en comparación con los de las órdenes bicerebrales, los planetas más antiguos del grupo con tres cerebros muestran unas civilizaciones que asombrarían a los urantianos, y que avergonzarían en cierto modo a las vuestras si se comparan con ellas. En desarrollo mecánico y en civilización material, e incluso en progreso intelectual, los mundos de los mortales con dos cerebros son capaces de igualar a las esferas de los que tienen tres cerebros. Pero en el control superior de la mente y en el desarrollo de la reciprocidad intelectual y espiritual, sois un poco inferiores.

\par
%\textsuperscript{(566.4)}
\textsuperscript{49:5.16} Todas estas estimaciones comparativas relacionadas con el progreso intelectual o los logros espirituales de cualquier mundo o grupo de mundos deberían reconocer, en justicia, la edad planetaria; muchísimas cosas dependen de la edad, de la ayuda de los mejoradores biológicos y de las misiones posteriores de las diversas órdenes de Hijos divinos.

\par
%\textsuperscript{(566.5)}
\textsuperscript{49:5.17} Aunque los pueblos con tres cerebros son capaces de alcanzar una evolución planetaria ligeramente superior a la de las órdenes con uno o dos cerebros, todos poseen el mismo tipo de plasma vital y ejercen sus actividades planetarias de una manera muy similar, poco más o menos como lo hacen los seres humanos en Urantia. Estos tres tipos de mortales están distribuidos por todos los mundos de los sistemas locales. En la mayoría de los casos, las condiciones planetarias tuvieron muy poco que ver con las decisiones de los Portadores de Vida de proyectar estas diversas órdenes de mortales en los diferentes mundos; los Portadores de Vida tienen la prerrogativa de planificar y de ejecutar sus planes de esta manera.

\par
%\textsuperscript{(566.6)}
\textsuperscript{49:5.18} Estas tres órdenes se hallan en un pie de igualdad en la carrera de la ascensión. Cada una debe atravesar la misma escala intelectual de desarrollo, y cada una debe dominar las mismas pruebas espirituales de progresión. La administración sistémica de estos diferentes mundos y el supercontrol de la constelación sobre ellos están uniformemente libres de discriminación; incluso los regímenes de los Príncipes Planetarios son idénticos.

\par
%\textsuperscript{(566.7)}
\textsuperscript{49:5.19} 3. \textit{Las series receptoras al espíritu}. Hay tres grupos de diseño mental en lo que respecta al contacto con los asuntos espirituales. Esta clasificación no se refiere a las órdenes de mortales con uno, dos o tres cerebros; se refiere principalmente a la química glandular, y más particularmente a la organización de ciertas glándulas comparables a los cuerpos pituitarios. En algunos mundos, las razas tienen una glándula, en otros dos, como los urantianos, mientras que en otras esferas las razas tienen tres de estos cuerpos extraordinarios. Esta dotación química diferencial influye claramente sobre la imaginación inherente y la receptividad espiritual.

\par
%\textsuperscript{(566.8)}
\textsuperscript{49:5.20} De los tipos receptores al espíritu, el sesenta y cinco por ciento pertenece al segundo grupo, como las razas de Urantia. El doce por ciento son del primer tipo, menos receptivos por naturaleza, mientras que el veintitrés por ciento tiene una mayor inclinación espiritual durante la vida terrestre. Pero estas distinciones no sobreviven a la muerte natural; todas estas diferencias raciales sólo se refieren a la vida en la carne.

\par
%\textsuperscript{(567.1)}
\textsuperscript{49:5.21} 4. \textit{Las épocas planetarias de los mortales}. Esta clasificación reconoce la sucesión de las dispensaciones temporales en la medida en que afectan el estatus terrestre del hombre y a su recepción del ministerio celestial.

\par
%\textsuperscript{(567.2)}
\textsuperscript{49:5.22} La vida es iniciada en los planetas por los Portadores de Vida, que vigilan su desarrollo hasta poco después de la aparición evolutiva del hombre mortal. Antes de dejar un planeta, los Portadores de Vida instalan debidamente a un Príncipe Planetario como gobernante del reino. Con este gobernante llega un contingente completo de auxiliares subordinados y de ayudantes ministrantes, y el primer juicio de los vivos y de los muertos tiene lugar simultáneamente con su llegada.

\par
%\textsuperscript{(567.3)}
\textsuperscript{49:5.23} Con la aparición de las agrupaciones humanas, este Príncipe Planetario llega para inaugurar la civilización humana y para enfocar la sociedad humana. Vuestro mundo confuso no es un criterio de los primeros tiempos del reino de los Príncipes Planetarios, porque casi al principio de esta administración en Urantia fue cuando Caligastia, vuestro Príncipe Planetario, unió su suerte a la rebelión de Lucifer, el Soberano del Sistema. Desde entonces vuestro planeta ha seguido una carrera borrascosa.

\par
%\textsuperscript{(567.4)}
\textsuperscript{49:5.24} En un mundo evolutivo normal, el progreso racial alcanza su apogeo biológico natural durante el régimen del Príncipe Planetario, y poco después el Soberano del Sistema envía a un Hijo y a una Hija Materiales a ese planeta. Estos seres importados prestan su servicio como mejoradores biológicos; su fallo en Urantia complicó aún más vuestra historia planetaria.

\par
%\textsuperscript{(567.5)}
\textsuperscript{49:5.25} Cuando el progreso intelectual y ético de una raza humana ha alcanzado los límites del desarrollo evolutivo, un Hijo Avonal del Paraíso llega en misión magistral; y más tarde aún, cuando el estado espiritual de ese mundo se acerca al límite de sus logros naturales, el planeta recibe la visita de un Hijo donador del Paraíso. La misión principal de un Hijo donador consiste en establecer el estatus planetario, liberar al Espíritu de la Verdad para que funcione en el planeta, y posibilitar así la llegada universal de los Ajustadores del Pensamiento.

\par
%\textsuperscript{(567.6)}
\textsuperscript{49:5.26} Aquí, una vez más, Urantia se desvía: nunca ha habido una misión magistral en vuestro mundo, y vuestro Hijo donador tampoco pertenecía a la orden de los Avonales; vuestro planeta disfrutó del notable honor de convertirse en el planeta natal humano del Hijo Soberano, Miguel de Nebadon.

\par
%\textsuperscript{(567.7)}
\textsuperscript{49:5.27} Como resultado del ministerio de todas las órdenes sucesivas de filiación divina, los mundos habitados y sus razas progresivas empiezan a acercarse a la cúspide de la evolución planetaria. Estos mundos están ahora maduros para la misión culminante, para la llegada de los Hijos Instructores Trinitarios. Esta época de los Hijos Instructores es el vestíbulo de la era planetaria final ---de la utopía evolutiva--- la era de luz y de vida.

\par
%\textsuperscript{(567.8)}
\textsuperscript{49:5.28} Esta clasificación de los seres humanos recibirá una atención especial en un documento posterior.

\par
%\textsuperscript{(567.9)}
\textsuperscript{49:5.29} 5. \textit{Las series de las criaturas emparentadas}. Los planetas no sólo están organizados verticalmente en sistemas, constelaciones y así sucesivamente, sino que la administración universal también mantiene agrupaciones horizontales de acuerdo con el tipo, la serie y otras relaciones. Esta administración lateral del universo está más particularmente relacionada con la coordinación de las actividades de naturaleza semejante que han sido fomentadas de forma independiente en esferas diferentes. Estas clases emparentadas de criaturas del universo son inspeccionadas periódicamente por ciertos cuerpos compuestos de elevadas personalidades, presididos por finalitarios con una larga experiencia.

\par
%\textsuperscript{(568.1)}
\textsuperscript{49:5.30} Estos factores de parentesco se manifiestan en todos los niveles, pues las series emparentadas existen entre las personalidades no humanas así como entre las criaturas mortales ---e incluso entre las órdenes humanas y superhumanas. Los seres inteligentes están emparentados verticalmente en doce grandes grupos de siete divisiones principales cada uno. Es probable que la coordinación de estos grupos de seres vivientes excepcionalmente emparentados se efectúe mediante una técnica del Ser Supremo que no comprendemos por completo.

\par
%\textsuperscript{(568.2)}
\textsuperscript{49:5.31} 6. \textit{Las series de los que fusionan con el Ajustador}. La clasificación o agrupación espiritual de todos los mortales durante su experiencia anterior a la fusión está enteramente determinada por la relación entre el estatus de la personalidad y el Monitor de Misterio interior. Casi el noventa por ciento de los mundos habitados de Nebadon está poblado por mortales que fusionan con su Ajustador, en contraste con un universo cercano donde apenas más de la mitad de los mundos alberga seres habitados por Ajustadores y candidatos a la fusión eterna.

\par
%\textsuperscript{(568.3)}
\textsuperscript{49:5.32} 7. \textit{Las técnicas para salir del planeta}. Existe fundamentalmente una sola manera en que la vida humana individual puede dar comienzo en los mundos habitados, y es mediante la procreación de las criaturas y el nacimiento natural; pero existen numerosas técnicas por medio de las cuales el hombre escapa a su estado terrestre y logra acceder a la corriente centrípeta de los que ascienden hacia el Paraíso.

\section*{6. La salida del planeta}
\par
%\textsuperscript{(568.4)}
\textsuperscript{49:6.1} Todos los diferentes tipos físicos y series planetarias de mortales disfrutan por igual del ministerio de los Ajustadores del Pensamiento, de los ángeles guardianes y de las diversas órdenes de las huestes de mensajeros del Espíritu Infinito. Todos son liberados\footnote{\textit{Liberación de la carne}: Jn 5:28-29; 6:39-40; 11:24-26.} por igual de las cadenas de la carne mediante la emancipación por la muerte natural, y todos van por igual desde allí a los mundos morontiales de evolución espiritual y de progreso mental.

\par
%\textsuperscript{(568.5)}
\textsuperscript{49:6.2} De vez en cuando, por iniciativa de las autoridades planetarias o de los gobernantes del sistema, se llevan a cabo resurrecciones especiales de los supervivientes dormidos. Estas resurrecciones se producen al menos cada milenio del tiempo planetario, cuando <<muchos de los que duermen en el polvo se despiertan>>\footnote{\textit{Muchos de los que duermen despiertan}: Is 26:19; Dn 12:2; Os 13:14.}, pero no todos. Estas resurrecciones especiales ofrecen la ocasión de movilizar grupos especiales de ascendentes para un servicio específico en el plan del universo local para la ascensión de los mortales. Existen razones prácticas así como asociaciones sentimentales que están conectadas con estas resurrecciones especiales.

\par
%\textsuperscript{(568.6)}
\textsuperscript{49:6.3} Durante las épocas primitivas de un mundo habitado, muchos humanos son llamados a las esferas de las mansiones en el momento de las resurrecciones especiales\footnote{\textit{Resurrecciones especiales}: Mt 27:52-53.} y milenarias, pero la mayoría de los supervivientes son repersonalizados en el momento de inaugurarse una nueva dispensación asociada a la venida de un Hijo divino que va a servir en el planeta.

\par
%\textsuperscript{(568.7)}
\textsuperscript{49:6.4} 1. \textit{Los mortales de la orden de supervivencia dispensacional o colectiva}. Cuando llega el primer Ajustador a un mundo habitado, los serafines guardianes también hacen su aparición; son indispensables para salir del planeta. Durante todo el período en que los supervivientes dormidos carecen de vida, los valores espirituales y las realidades eternas de sus almas recién desarrolladas e inmortales son conservados como un depósito sagrado por los serafines guardianes personales o colectivos.

\par
%\textsuperscript{(568.8)}
\textsuperscript{49:6.5} Los guardianes colectivos asignados a los supervivientes dormidos siempre ejercen su actividad con los Hijos judiciales cuando éstos vienen a los mundos. <<Enviará a sus ángeles, y éstos reunirán a sus elegidos procedentes de los cuatro vientos>>\footnote{\textit{Los ángeles reunirán a los elegidos}: Mt 24:31; Mc 13:27.}. El Ajustador que ha regresado trabaja con cada serafín asignado a la repersonalización de un mortal dormido; es el mismo fragmento inmortal del Padre que vivió en el ser humano durante su vida en la carne, y así es como se restablece la identidad y se resucita la personalidad. Durante el sueño de sus sujetos, estos Ajustadores en espera sirven en Divinington; nunca habitan en otra mente mortal durante este ínterin.

\par
%\textsuperscript{(569.1)}
\textsuperscript{49:6.6} Mientras los mundos más antiguos donde existen los mortales albergan aquellos tipos de seres humanos extremadamente desarrollados y exquisitamente espirituales que están prácticamente exentos de la vida morontial, las épocas iniciales de las razas de origen animal están caracterizadas por mortales primitivos que son tan inmaduros que es imposible la fusión con su Ajustador. Los serafines guardianes llevan a cabo el despertar de estos mortales en conjunción con una fracción individualizada del espíritu inmortal de la Fuente-Centro Tercera.

\par
%\textsuperscript{(569.2)}
\textsuperscript{49:6.7} Los supervivientes dormidos de una era planetaria son repersonalizados así en los llamamientos dispensacionales. Pero en cuanto a las personalidades no salvables de un reino, ningún espíritu inmortal se encuentra presente para actuar con los guardianes colectivos del destino, y esto representa el cese de la existencia de la criatura\footnote{\textit{No sobrevimiento}: Jer 51:39; Mc 3:29; Jn 5:28-29; Ro 13:1-2.}. Aunque algunos de vuestros relatos han descrito que estos acontecimientos tienen lugar en los planetas de la muerte física, todos se producen en realidad en los mundos de las mansiones.

\par
%\textsuperscript{(569.3)}
\textsuperscript{49:6.8} 2. \textit{Los mortales de las órdenes individuales de ascensión}. El progreso individual de los seres humanos se mide por la conquista y la travesía sucesivas (el dominio) de los siete círculos cósmicos. Estos círculos de progresión humana son unos niveles de valores intelectuales, sociales, espirituales y de perspicacia cósmica asociados. Empezando por el séptimo círculo, los mortales se esfuerzan por alcanzar el primero, y a todos los que han llegado al tercero se les asignan de inmediato unos guardianes personales del destino. Estos mortales pueden ser repersonalizados en la vida morontial, independientemente de los juicios dispensacionales o de otro tipo.

\par
%\textsuperscript{(569.4)}
\textsuperscript{49:6.9} Durante las épocas primitivas de un mundo evolutivo, pocos mortales van a juicio al tercer día. Pero a medida que pasan las eras, a los mortales que progresan se les asignan cada vez más guardianes personales del destino, y un número creciente de estas criaturas evolutivas son repersonalizadas así en el primer mundo de las mansiones al tercer día después de su muerte natural. En tales ocasiones, el regreso del Ajustador señala el despertar del alma humana, y esto supone la repersonalización de los muertos tan literalmente como cuando se pasa lista en masa al final de una dispensación en los mundos evolutivos.

\par
%\textsuperscript{(569.5)}
\textsuperscript{49:6.10} Hay tres grupos de ascendentes individuales: los menos avanzados aterrizan en el mundo inicial o primer mundo de las mansiones. El grupo más avanzado puede empezar la carrera morontial en cualquier mundo intermedio de las mansiones de acuerdo con su progresión planetaria anterior. Los más avanzados de estas órdenes empiezan realmente su experiencia morontial en el séptimo mundo de las mansiones.

\par
%\textsuperscript{(569.6)}
\textsuperscript{49:6.11} 3. \textit{Los mortales de las órdenes de ascensión que dependen de un período deprueba}. La llegada de un Ajustador establece la identidad a los ojos del universo, y todos los seres habitados por un Ajustador figuran en las listas nominales de la justicia. Pero la vida temporal en los mundos evolutivos es incierta, y muchos mueren jóvenes antes de escoger la carrera del Paraíso. Estos niños y jóvenes habitados por un Ajustador siguen a aquel de sus padres que tiene el estado espiritual más avanzado, yendo así al mundo finalitario del sistema (a la guardería probatoria) al tercer día, o en el momento de una resurrección especial, o al efectuarse los llamamientos nominales regulares milenarios y dispensacionales.

\par
%\textsuperscript{(570.1)}
\textsuperscript{49:6.12} Los niños que mueren demasiado jóvenes como para tener un Ajustador del Pensamiento son repersonalizados en el mundo finalitario de los sistemas locales en el momento de llegar uno de sus padres a los mundos de las mansiones. Un niño adquiere su identidad física en el momento de nacer como mortal, pero en materia de supervivencia, todos los niños sin Ajustador se considera que están vinculados todavía a sus padres.

\par
%\textsuperscript{(570.2)}
\textsuperscript{49:6.13} Los Ajustadores del Pensamiento vienen a residir a su debido tiempo en estos pequeños, mientras que el ministerio seráfico para los dos grupos de órdenes de supervivencia que dependen de un período de prueba es similar en general al del progenitor más avanzado, o es equivalente al del progenitor en el caso de que uno solo de ellos sobreviva. A aquellos que alcanzan el tercer círculo se les conceden guardianes personales, independientemente del nivel de sus padres.

\par
%\textsuperscript{(570.3)}
\textsuperscript{49:6.14} En las esferas finalitarias de la constelación y de la sede del universo se mantienen guarderías probatorias similares para los niños sin Ajustador de las órdenes primarias y secundarias modificadas de ascendentes.

\par
%\textsuperscript{(570.4)}
\textsuperscript{49:6.15} 4. \textit{Los mortales de las órdenes secundarias modificadas de ascensión}. Son los seres humanos progresivos de los mundos evolutivos intermedios. Por regla general no están inmunizados contra la muerte natural, pero están exentos de pasar por los siete mundos de las mansiones.

\par
%\textsuperscript{(570.5)}
\textsuperscript{49:6.16} El grupo menos perfeccionado se despierta en la sede de su sistema local, dejando sólo de lado los mundos de las mansiones. El grupo intermedio va a los mundos educativos de la constelación; dejan de lado todo el régimen morontial del sistema local. Más tarde aún, durante las épocas planetarias de los esfuerzos espirituales, muchos supervivientes se despiertan en la sede de la constelación y empiezan allí su ascensión hacia el Paraíso.

\par
%\textsuperscript{(570.6)}
\textsuperscript{49:6.17} Pero antes de que uno de estos grupos pueda seguir adelante, deben regresar como instructores a los mundos que se saltaron, adquiriendo como educadores muchas experiencias en aquellos reinos que dejaron de lado como estudiantes. Todos se dirigen posteriormente hacia el Paraíso por las rutas ordenadas de la progresión humana.

\par
%\textsuperscript{(570.7)}
\textsuperscript{49:6.18} 5. \textit{Los mortales de la orden primaria modificada de ascensión}. Estos mortales pertenecen al tipo de vida evolutiva que fusiona con el Ajustador, pero representan con mucha frecuencia las fases finales del desarrollo humano en un mundo en evolución. Estos seres glorificados están exentos de pasar por las puertas de la muerte; están sometidos a la atracción del Hijo; son trasladados de entre los vivos y aparecen inmediatamente en presencia del Hijo Soberano en la sede del universo local.

\par
%\textsuperscript{(570.8)}
\textsuperscript{49:6.19} Son los mortales que fusionan con su Ajustador durante la vida humana, y estas personalidades fusionadas con el Ajustador atraviesan el espacio libremente antes de ser vestidas con las formas morontiales. Estas almas fusionadas van por tránsito directo del Ajustador a las salas de resurrección de las esferas morontiales superiores, donde reciben su investidura morontial inicial exactamente igual que todos los demás mortales que llegan de los mundos evolutivos.

\par
%\textsuperscript{(570.9)}
\textsuperscript{49:6.20} Esta orden primaria modificada de ascensión humana puede aplicarse a los individuos de cualquier serie planetaria, desde los estados más bajos hasta los estados más elevados de los mundos donde se fusiona con el Ajustador, pero funciona con más frecuencia en las esferas más antiguas de este tipo después de que han recibido los beneficios de las numerosas estancias de los Hijos divinos.

\par
%\textsuperscript{(570.10)}
\textsuperscript{49:6.21} Con el establecimiento de la era planetaria de luz y de vida, muchos mortales van a los mundos morontiales del universo mediante el tipo primario modificado de traslado. Más tarde aún, durante las etapas avanzadas de la existencia establecida, cuando la mayoría de los mortales que dejan un reino están incluídos en esta clase, se considera que el planeta pertenece a esta serie. La muerte natural se vuelve cada vez menos frecuente en estas esferas establecidas durante mucho tiempo en la luz y la vida.

\par
%\textsuperscript{(571.1)}
\textsuperscript{49:6.22} [Presentado por un Melquisedek de la Escuela de Administración Planetaria de Jerusem.]


\chapter{Documento 50. Los Príncipes Planetarios}
\par
%\textsuperscript{(572.1)}
\textsuperscript{50:0.1} AUNQUE pertenecen a la orden de los Hijos Lanonandeks, los Príncipes Planetarios están tan especializados en su servicio que se les considera generalmente como un grupo distinto. Después de que los Melquisedeks han certificado que son Lanonandeks secundarios, estos Hijos del universo local son destinados a las reservas de su orden en la sede de la constelación. Desde allí, el Soberano del Sistema los destina a diversas tareas y los nombra finalmente como Príncipes Planetarios y los envía a gobernar los mundos habitados en evolución.

\par
%\textsuperscript{(572.2)}
\textsuperscript{50:0.2} La señal para que el Soberano de un Sistema actúe en el asunto de asignar un gobernante a un planeta dado se produce cuando recibe la solicitud de los Portadores de Vida para que envíe a un jefe administrativo que ejerza su actividad en ese planeta donde han establecido la vida y han desarrollado seres evolutivos inteligentes. Todos los planetas que están habitados por criaturas mortales evolutivas tienen asignado un gobernante planetario de esta orden de filiación.

\section*{1. La misión de los Príncipes}
\par
%\textsuperscript{(572.3)}
\textsuperscript{50:1.1} El Príncipe Planetario y sus hermanos asistentes representan el máximo acercamiento personalizado (aparte de la encarnación) que puede hacer el Hijo Eterno del Paraíso a las humildes criaturas del tiempo y del espacio. Es verdad que el Hijo Creador se acerca a las criaturas del reino a través de su espíritu, pero el Príncipe Planetario representa la última de las órdenes de Hijos personales que se extienden desde el Paraíso hasta los hijos de los hombres. El Espíritu Infinito se acerca mucho mediante las personas de los guardianes del destino y otros seres angélicos; el Padre Universal vive en el hombre mediante la presencia prepersonal de los Monitores de Misterio; pero el Príncipe Planetario representa el último esfuerzo del Hijo Eterno y de sus Hijos por acercarse a vosotros. En un mundo recién habitado, el Príncipe Planetario es el único representante de la divinidad completa, pues procede del Hijo Creador (descendiente del Padre Universal y del Hijo Eterno) y de la Ministra Divina (la Hija universal del Espíritu Infinito).

\par
%\textsuperscript{(572.4)}
\textsuperscript{50:1.2} El príncipe de un mundo recién habitado está rodeado de un cuerpo leal de ayudantes y de asistentes y de un gran número de espíritus ministrantes. Pero el cuerpo dirigente de estos nuevos mundos debe estar compuesto de las órdenes inferiores de administradores de un sistema, a fin de que puedan comprender y tener una simpatía innata por los problemas y las dificultades planetarios. Todo este esfuerzo por proporcionar a los mundos evolutivos un gobierno compasivo conlleva el inconveniente creciente de que estas personalidades casi humanas puedan descarriarse mediante la exaltación de su propia mente por encima de la voluntad de los Gobernantes Supremos.

\par
%\textsuperscript{(572.5)}
\textsuperscript{50:1.3} Como están totalmente solos como representantes de la divinidad en los planetas individuales, estos Hijos están sometidos a una dura prueba, y Nebadon ha sufrido la desgracia de varias rebeliones. En la creación de los Soberanos Sistémicos y de los Príncipes Planetarios se produce la personalización de un concepto que se ha alejado cada vez más del Padre Universal y del Hijo Eterno, y existe el peligro creciente de perder el sentido de las proporciones en cuanto a la propia importancia, y una mayor probabilidad de no lograr mantener una comprensión adecuada de los valores y de las relaciones entre las numerosas órdenes de seres divinos y sus jerarquías de autoridad. El hecho de que el Padre no esté personalmente presente en el universo local también impone cierta prueba de fe y de lealtad a todos estos Hijos.

\par
%\textsuperscript{(573.1)}
\textsuperscript{50:1.4} Pero estos príncipes de los mundos fracasan pocas veces en su misión de organizar y de administrar las esferas habitadas, y su éxito facilita enormemente las misiones posteriores de los Hijos Materiales, que vienen para injertar las formas superiores de vida de las criaturas en los hombres primitivos de los mundos. Su gobierno también contribuye mucho a preparar los planetas para los Hijos Paradisiacos de Dios, que vienen posteriormente para juzgar a los mundos e inaugurar las dispensaciones sucesivas.

\section*{2. La administración planetaria}
\par
%\textsuperscript{(573.2)}
\textsuperscript{50:2.1} Todos los Príncipes Planetarios se encuentran bajo la jurisdicción administrativa universal de Gabriel, el jefe ejecutivo de Miguel, aunque en lo que se refiere a la autoridad inmediata están sometidos a los mandatos ejecutivos de los Soberanos Sistémicos.

\par
%\textsuperscript{(573.3)}
\textsuperscript{50:2.2} Los Príncipes Planetarios pueden pedir en cualquier momento el consejo de los Melquisedeks, sus antiguos instructores y padrinos, pero no se les exige arbitrariamente que soliciten esta ayuda, y si no piden voluntariamente dicha ayuda, los Melquisedeks no interfieren en la administración planetaria. Estos gobernantes de los mundos también pueden utilizar el asesoramiento de los veinticuatro consejeros, reclutados entre los mundos de donación del sistema. En Satania, todos estos consejeros son actualmente nativos de Urantia. Y en la sede de la constelación existe un consejo análogo de setenta miembros elegidos también entre los seres evolutivos de los reinos.

\par
%\textsuperscript{(573.4)}
\textsuperscript{50:2.3} El gobierno de los planetas evolutivos durante sus carreras iniciales e inestables es principalmente autocrático. Los Príncipes Planetarios organizan sus grupos especializados de asistentes escogiéndolos entre su cuerpo de ayudantes planetarios. Generalmente se rodean de un consejo supremo de doce miembros, pero la elección y la constitución de este consejo varía en los diferentes mundos. Un Príncipe Planetario también puede tener como ayudante a un miembro o más de la tercera orden de su propio grupo de filiación y, a veces, en ciertos mundos, a un miembro de su propia orden, a un asociado Lanonandek secundario.

\par
%\textsuperscript{(573.5)}
\textsuperscript{50:2.4} Todo el estado mayor del gobernante de un mundo está compuesto de personalidades del Espíritu Infinito, de ciertos tipos de seres superiores evolucionados y de mortales ascendentes procedentes de otros mundos. Este estado mayor tiene por término medio unos mil seres, y a medida que el planeta progresa, este cuerpo de ayudantes puede aumentar hasta cien mil o más. En cualquier momento que sientan la necesidad de más ayudantes, los Príncipes Planetarios sólo tienen que solicitarlos a sus hermanos, los Soberanos de los Sistemas, y su petición se les concede enseguida.

\par
%\textsuperscript{(573.6)}
\textsuperscript{50:2.5} La naturaleza, la organización y la administración de los planetas varían enormemente, pero todos están provistos de tribunales de justicia. El sistema judicial de un universo local tiene sus orígenes en los tribunales de un Príncipe Planetario, que están presididos por un miembro de su estado mayor personal; los decretos de estos tribunales reflejan una actitud extremadamente paternal y discrecional. Todos los problemas que implican más cosas que la reglamentación de los habitantes planetarios están sujetos a apelación ante los tribunales superiores, pero los asuntos pertenecientes al ámbito de su mundo se resuelven principalmente de acuerdo con el juicio personal del príncipe.

\par
%\textsuperscript{(574.1)}
\textsuperscript{50:2.6} Las comisiones itinerantes de conciliadores sirven y complementan a los tribunales planetarios, y tanto los controladores espirituales como los físicos están sometidos a las conclusiones de estos conciliadores. Pero ninguna ejecución arbitraria se lleva nunca a cabo sin el consentimiento del Padre de la Constelación, porque los <<Altísimos gobiernan en los reinos de los hombres>>\footnote{\textit{Los Altísimos gobiernan en los reinos}: Dn 4:17,25,32; 5:21.}.

\par
%\textsuperscript{(574.2)}
\textsuperscript{50:2.7} Los controladores y los transformadores asignados al planeta también pueden colaborar con los ángeles y otras órdenes de seres celestiales para hacer visibles estas últimas personalidades a las criaturas mortales. En ocasiones especiales, los ayudantes seráficos e incluso los Melquisedeks pueden hacerse visibles a los habitantes de los mundos evolutivos, y de hecho lo hacen. La razón principal para traer a unos ascendentes mortales desde la capital del sistema, como parte del estado mayor del Príncipe Planetario, es facilitar la comunicación con los habitantes del reino.

\section*{3. El estado mayor corpóreo del Príncipe}
\par
%\textsuperscript{(574.3)}
\textsuperscript{50:3.1} Cuando va a un mundo joven, un Príncipe Planetario lleva generalmente consigo a un grupo de seres ascendentes voluntarios procedentes de la sede del sistema local. Estos ascendentes acompañan al Príncipe como consejeros y ayudantes en la tarea de mejorar inicialmente la raza. Este cuerpo de ayudantes materiales constituye el lazo de unión entre el Príncipe y las razas del mundo. Caligastia, el Príncipe de Urantia, disponía de un cuerpo de cien ayudantes de este tipo.

\par
%\textsuperscript{(574.4)}
\textsuperscript{50:3.2} Estos asistentes voluntarios son ciudadanos de la capital de un sistema, y ninguno de ellos ha fusionado con su Ajustador interior. El estatus de los Ajustadores de estos servidores voluntarios sigue siendo el de residentes de la sede del sistema mientras estos progresores morontiales regresan temporalmente a un estado material anterior.

\par
%\textsuperscript{(574.5)}
\textsuperscript{50:3.3} Los Portadores de Vida, arquitectos de la forma, proporcionan a estos voluntarios unos nuevos cuerpos físicos que ellos ocupan durante los períodos de su estancia planetaria. Estas formas de la personalidad, aunque están exentas de las enfermedades ordinarias de los reinos, están sometidas, al igual que los cuerpos morontiales iniciales, a ciertos accidentes de naturaleza mecánica.

\par
%\textsuperscript{(574.6)}
\textsuperscript{50:3.4} El estado mayor corpóreo del príncipe es retirado generalmente del planeta en conexión con el juicio siguiente que tiene lugar cuando llega un segundo Hijo a la esfera. Antes de marcharse, habitualmente asignan sus diversas tareas a sus descendientes comunes y a ciertos voluntarios nativos superiores. En aquellos mundos donde estos ayudantes del príncipe han tenido permiso para emparejarse con los grupos superiores de las razas nativas, estos descendientes los suceden generalmente.

\par
%\textsuperscript{(574.7)}
\textsuperscript{50:3.5} Estos asistentes del Príncipe Planetario raras veces se emparejan con las razas del mundo, pero siempre se emparejan entre ellos. Estas uniones producen dos clases de seres: el tipo primario de criaturas intermedias y ciertos tipos elevados de seres materiales que permanecen vinculados al estado mayor del príncipe después de que sus padres han sido retirados del planeta en el momento de la llegada de Adán y Eva. Estos hijos no se emparejan con las razas mortales, salvo en ciertas situaciones de emergencia, y entonces sólo lo hacen por mandato del Príncipe Planetario. En un caso así, sus hijos ---los nietos del estado mayor corpóreo--- tienen el mismo estatus que las razas superiores de su época y de su generación. Todos los descendientes de estos asistentes semimateriales del Príncipe Planetario están habitados por un Ajustador.

\par
%\textsuperscript{(575.1)}
\textsuperscript{50:3.6} Al final de la dispensación del príncipe, cuando llega el momento en que este <<estado mayor revertido>> ha de regresar a la sede del sistema para reanudar su carrera hacia el Paraíso, estos ascendentes se presentan ante los Portadores de Vida para entregar sus cuerpos materiales. Entran en el sueño de transición y se despiertan libres de su investidura mortal y vestidos con las formas morontiales, preparados para el transporte seráfico de vuelta a la capital del sistema, donde les esperan sus Ajustadores separados. Llevan un retraso de una dispensación entera con respecto a su clase de Jerusem, pero han adquirido una experiencia única y extraordinaria, un raro capítulo en la carrera de un mortal ascendente.

\section*{4. La sede y las escuelas planetarias}
\par
%\textsuperscript{(575.2)}
\textsuperscript{50:4.1} El estado mayor corpóreo del príncipe organiza pronto las escuelas planetarias de formación y de cultura, donde se instruye a la flor y nata de las razas evolutivas y luego se les envía para que enseñen estas mejores costumbres a sus pueblos. Estas escuelas del príncipe están situadas en la sede material del planeta.

\par
%\textsuperscript{(575.3)}
\textsuperscript{50:4.2} El estado mayor corpóreo realiza una gran parte del trabajo físico relacionado con el establecimiento de esta ciudad sede. Estas ciudades o colonias sede de los primeros tiempos del Príncipe Planetario son muy diferentes de lo que un mortal de Urantia podría imaginar. En comparación con las épocas posteriores, son sencillas y están caracterizadas por adornos minerales y por una construcción material relativamente avanzada. Todo esto contrasta con el régimen adámico, que está centrado alrededor de una sede ajardinada desde la cual efectúan su trabajo a favor de las razas durante la segunda dispensación de los Hijos del universo.

\par
%\textsuperscript{(575.4)}
\textsuperscript{50:4.3} En la colonia sede de vuestro mundo, cada morada humana estaba provista de abundantes tierras. Aunque las tribus lejanas continuaban cazando y buscando alimentos, todos los estudiantes y profesores de las escuelas del príncipe eran agricultores y horticultores. El tiempo estaba dividido casi por igual entre las ocupaciones siguientes:

\par
%\textsuperscript{(575.5)}
\textsuperscript{50:4.4} 1. \textit{Trabajo físico}. Cultivo del suelo, asociado con la construcción y el embellecimiento de las viviendas.

\par
%\textsuperscript{(575.6)}
\textsuperscript{50:4.5} 2. \textit{Actividades sociales}. Representaciones de obras y agrupaciones sociales culturales.

\par
%\textsuperscript{(575.7)}
\textsuperscript{50:4.6} 3. \textit{Aplicación educativa}. Instrucción individual en conexión con la enseñanza colectiva familiar, completada mediante una formación especializada por clases.

\par
%\textsuperscript{(575.8)}
\textsuperscript{50:4.7} 4. \textit{Formación profesional}. Escuelas para el matrimonio y las tareas del hogar, escuelas de artes y oficios, y las clases para la formación de los profesores ---laicos, culturales y religiosos.

\par
%\textsuperscript{(575.9)}
\textsuperscript{50:4.8} 5. \textit{Cultura espiritual}. La fraternidad de los profesores, la instrucción de los grupos infantiles y juveniles, y la formación de los niños nativos adoptados como misioneros para sus pueblos.

\par
%\textsuperscript{(575.10)}
\textsuperscript{50:4.9} Un Príncipe Planetario no es visible para los seres mortales; es una prueba de fe el creer en las descripciones que efectúan los seres semimateriales de su estado mayor. Pero estas escuelas de cultura y de formación están bien adaptadas a las necesidades de cada planeta, y pronto se desarrolla una intensa y elogiosa rivalidad entre las razas de hombres en sus esfuerzos por ser admitidos en estas diversas instituciones de estudio.

\par
%\textsuperscript{(575.11)}
\textsuperscript{50:4.10} Desde este centro mundial de cultura y de consecución irradia gradualmente hacia todos los pueblos una influencia edificante y civilizadora que transforma de manera lenta pero segura a las razas evolutivas. Mientras tanto, los niños instruidos y espiritualizados de los pueblos circundantes, que han sido adoptados y educados en las escuelas del príncipe, regresan a sus grupos nativos y, haciendo lo mejor que pueden, establecen allí nuevos y poderosos centros de estudio y de cultura que dirigen de acuerdo con el plan de las escuelas del príncipe.

\par
%\textsuperscript{(576.1)}
\textsuperscript{50:4.11} En Urantia, estos planes para el progreso planetario y el avance cultural estaban bien encaminados, desarrollándose de la manera más satisfactoria, cuando la adhesión de Caligastia a la rebelión de Lucifer llevó a toda la empresa a un fin más bien repentino y de lo más ignominioso.

\par
%\textsuperscript{(576.2)}
\textsuperscript{50:4.12} Para mí, uno de los episodios más profundamente chocantes de esta rebelión fue cuando me enteré de la perfidia cruel de Casligastia, un miembro de mi propia orden de filiación, que deliberadamente y con premeditación pervirtió sistemáticamente la instrucción y envenenó la enseñanza que se daba en todas las escuelas planetarias que funcionaban en aquel momento en Urantia. El hundimiento de estas escuelas fue rápido y completo.

\par
%\textsuperscript{(576.3)}
\textsuperscript{50:4.13} Una gran parte de la progenie de los ascendentes vinculados al estado mayor materializado del Príncipe permanecieron leales, desertando de las filas de Caligastia. Los síndicos Melquisedeks de Urantia alentaron a estos seres leales, y en tiempos posteriores sus descendientes contribuyeron mucho a mantener los conceptos planetarios sobre la verdad y la rectitud. El trabajo de estos evángeles leales ayudó a impedir la desaparición total de la verdad espiritual en Urantia. Estas almas valerosas y sus descendientes mantuvieron vivo cierto conocimiento sobre el gobierno del Padre, y conservaron para las razas del mundo el concepto de las dispensaciones planetarias sucesivas de las diversas órdenes de Hijos divinos.

\section*{5. La civilización progresiva}
\par
%\textsuperscript{(576.4)}
\textsuperscript{50:5.1} Los príncipes leales de los mundos habitados están vinculados de forma permanente a los planetas donde fueron destinados al principio. Los Hijos Paradisiacos y sus dispensaciones pueden ir y venir, pero un Príncipe Planetario que tiene éxito continúa siendo el gobernante de su reino. Su trabajo es totalmente independiente de las misiones de los Hijos superiores, pues está destinado a fomentar el desarrollo de la civilización planetaria.

\par
%\textsuperscript{(576.5)}
\textsuperscript{50:5.2} El progreso de la civilización apenas se parece en dos planetas cualquiera. Los detalles del desarrollo de la evolución humana son muy diferentes en los numerosos mundos desiguales. A pesar de estas múltiples variaciones en el desarrollo planetario de los aspectos físicos, intelectuales y sociales, todas las esferas evolutivas progresan en ciertas direcciones bien definidas.

\par
%\textsuperscript{(576.6)}
\textsuperscript{50:5.3} Bajo el gobierno favorable de un Príncipe Planetario, acrecentado por los Hijos Materiales y puntualizado por las misiones periódicas de los Hijos Paradisiacos, las razas mortales de un mundo medio del tiempo y del espacio pasarán sucesivamente por las siete épocas de desarrollo siguientes:

\par
%\textsuperscript{(576.7)}
\textsuperscript{50:5.4} 1. \textit{La época de la nutrición}. Las criaturas prehumanas y las primeras razas de hombres primitivos se preocupan principalmente por los problemas de la alimentación. Estos seres evolutivos pasan sus horas de vigilia buscando comida o bien luchando de forma ofensiva o defensiva. La búsqueda de alimento es lo más importante de todo en la mente de estos antepasados primitivos de la civilización posterior.

\par
%\textsuperscript{(576.8)}
\textsuperscript{50:5.5} 2. \textit{La era de la seguridad}. Tan pronto como el cazador primitivo puede ahorrar algo de tiempo en su búsqueda de alimentos, emplea este tiempo libre en aumentar su seguridad. Cada vez dedica más atención a la técnica de la guerra. Fortifica sus viviendas y los clanes se solidifican mediante el miedo mutuo y la inculcación del odio hacia los grupos exteriores. El instinto de supervivencia es una actividad que siempre sigue a la conservación de sí mismo.

\par
%\textsuperscript{(577.1)}
\textsuperscript{50:5.6} 3. \textit{La era de la comodidad material}. Después de haber resuelto parcialmente los problemas alimenticios y de haber alcanzado cierto grado de seguridad, el tiempo libre adicional se utiliza para favorecer la comodidad personal. El lujo rivaliza con la necesidad para ocupar el centro del escenario de las actividades humanas. Una era así está caracterizada con demasiada frecuencia por la tiranía, la intolerancia, la glotonería y la embriaguez. Los elementos más débiles de las razas tienden a los excesos y a la brutalidad. Estas personas débiles que buscan el placer son gradualmente sometidas por los elementos más fuertes de la civilización progresiva que aman la verdad.

\par
%\textsuperscript{(577.2)}
\textsuperscript{50:5.7} 4. \textit{La búsqueda del conocimiento y de la sabiduría}. El alimento, la seguridad, el placer y el tiempo libre proporcionan las bases para el desarrollo de la cultura y la propagación del conocimiento. El esfuerzo por poner en práctica el conocimiento conduce a la sabiduría, y cuando una cultura ha aprendido a beneficiarse y a mejorar por medio de la experiencia, la civilización ha llegado de verdad. El alimento, la seguridad y la comodidad material dominan todavía a la sociedad, pero muchos individuos con visión de futuro tienen hambre de conocimiento y sed de sabiduría. Todo niño recibe la oportunidad de aprender haciendo; la educación es la consigna de estas eras.

\par
%\textsuperscript{(577.3)}
\textsuperscript{50:5.8} 5. \textit{La época de la filosofía y de la fraternidad}. Cuando los mortales aprenden a pensar y empiezan a beneficiarse de la experiencia, se vuelven filosóficos ---empiezan a razonar dentro de sí mismos y a ejercer un juicio discriminatorio. La sociedad de esta época se vuelve ética, y los mortales de una era así se vuelven realmente seres morales. Unos seres morales sabios son capaces de establecer la fraternidad humana en ese mundo en progreso. Los seres éticos y morales pueden aprender a vivir de acuerdo con la regla de oro.

\par
%\textsuperscript{(577.4)}
\textsuperscript{50:5.9} 6. \textit{La era de los esfuerzos espirituales}. Cuando los mortales evolutivos han pasado por las etapas del desarrollo físico, intelectual y social, tarde o temprano alcanzan los niveles de perspicacia personal que los impulsan a buscar las satisfacciones espirituales y los conocimientos cósmicos. La religión va terminando de ascender desde los ámbitos emocionales del miedo y de la superstición hasta los niveles superiores de la sabiduría cósmica y de la experiencia espiritual personal. La educación aspira a alcanzar los significados, y la cultura capta las relaciones cósmicas y los valores verdaderos. Estos mortales evolutivos son auténticamente cultos, están realmente educados y conocen exquisitamente a Dios.

\par
%\textsuperscript{(577.5)}
\textsuperscript{50:5.10} 7. \textit{La era de luz y de vida}. Es el florecimiento de las eras sucesivas de seguridad física, de expansión intelectual, de cultura social y de consecución espiritual. Estos logros humanos están ahora mezclados, asociados y coordinados en una unidad cósmica y en un servicio desinteresado. Dentro de las limitaciones de la naturaleza finita y de los dones materiales, las posibilidades de los logros evolutivos de las generaciones progresivas que viven sucesivamente en estos mundos excelsos y establecidos del tiempo y del espacio no tienen límites.

\par
%\textsuperscript{(577.6)}
\textsuperscript{50:5.11} Después de servir a sus esferas durante las dispensaciones sucesivas de la historia del mundo y las épocas progresivas de avance planetario, los Príncipes Planetarios son elevados a la categoría de Soberanos Planetarios en el momento de inaugurarse la era de luz y de vida.

\section*{6. La cultura planetaria}
\par
%\textsuperscript{(578.1)}
\textsuperscript{50:6.1} El aislamiento de Urantia hace que resulte imposible intentar presentar muchos detalles sobre la vida y el entorno de vuestros vecinos de Satania. En estas presentaciones estamos limitados por la cuarentena planetaria y el aislamiento del sistema. En todos nuestros esfuerzos por iluminar a los mortales de Urantia, debemos guiarnos por estas restricciones, pero en la medida de lo permisible os hemos informado sobre el progreso de un mundo evolutivo medio, y podéis comparar la carrera de un mundo así con el estado actual de Urantia.

\par
%\textsuperscript{(578.2)}
\textsuperscript{50:6.2} El desarrollo de la civilización en Urantia no ha sido tan diferente al de los otros mundos que han soportado la desgracia del aislamiento espiritual. Pero cuando vuestro planeta es comparado con los mundos leales del universo, parece de lo más confuso y enormemente retrasado en todas las fases del progreso intelectual y de la consecución espiritual.

\par
%\textsuperscript{(578.3)}
\textsuperscript{50:6.3} Debido a vuestras desgracias planetarias, los urantianos no pueden comprender muchas cosas de la cultura de los mundos normales. Pero no deberíais imaginar que los mundos evolutivos, ni siquiera los más ideales, son unas esferas donde la vida es un lecho de rosas. La vida inicial de las razas mortales siempre va acompañada de luchas. El esfuerzo y la decisión son una parte esencial de la adquisición de los valores de supervivencia.

\par
%\textsuperscript{(578.4)}
\textsuperscript{50:6.4} La cultura presupone la calidad de mente; la cultura no puede mejorar a menos que se eleve la mente. Un intelecto superior buscará una cultura noble y encontrará alguna manera de alcanzar esa meta. Las mentes inferiores despreciarán la cultura más elevada, aunque se la presenten ya hecha. También depende mucho de las misiones sucesivas de los Hijos divinos y del grado de iluminación que reciben las épocas de sus dispensaciones respectivas.

\par
%\textsuperscript{(578.5)}
\textsuperscript{50:6.5} No deberíais olvidar que durante doscientos mil años, todos los mundos de Satania han permanecido bajo la prohibición espiritual de Norlatiadek a consecuencia de la rebelión de Lucifer. Y se necesitará una era tras otra para reparar los perjuicios resultantes del pecado y de la secesión. Vuestro mundo sigue todavía una carrera irregular y con altibajos como resultado de la doble tragedia de un Príncipe Planetario rebelde y de un Hijo Material negligente. Ni siquiera la donación de Cristo Miguel en Urantia apartó inmediatamente las consecuencias temporales de estos graves errores de la administración inicial del mundo.

\section*{7. Las recompensas del aislamiento}
\par
%\textsuperscript{(578.6)}
\textsuperscript{50:7.1} A primera vista, podría parecer que Urantia y los mundos aislados asociados son de lo más desafortunados por estar privados de la presencia y de la influencia benéficas de unas personalidades superhumanas tales como un Príncipe Planetario y un Hijo y una Hija Materiales. Pero el aislamiento de estas esferas ofrece a sus razas una oportunidad única para ejercitar su fe y para desarrollar una calidad de confianza especial en la fiabilidad cósmica que no dependen de la vista ni de ninguna otra consideración material. Al final puede resultar que las criaturas mortales procedentes de los mundos que están en cuarentena a consecuencia de la rebelión sean extremadamente afortunadas. Hemos descubierto que a estos ascendentes les confían muy pronto numerosas tareas especiales en empresas cósmicas donde una fe incuestionable y una confianza sublime son esenciales para triunfar.

\par
%\textsuperscript{(579.1)}
\textsuperscript{50:7.2} En Jerusem, los ascendentes de estos mundos aislados ocupan un sector residencial propio y se les conoce con el nombre de \textit{agondontarios}\footnote{\textit{Agondontarios}: Jn 20:29.}, lo que significa criaturas volitivas evolutivas que pueden creer sin ver, perseverar cuando están aisladas y vencer dificultades insuperables incluso estando solas. Esta agrupación funcional de los agondontarios persiste durante toda la ascensión del universo local y la travesía del superuniverso; desaparece durante la estancia en Havona, pero reaparece de inmediato cuando se alcanza el Paraíso, y subsiste definitivamente en el Cuerpo de la Finalidad de los Mortales. Tabamantia es un \textit{agondontario}, con estatus de finalitario, que sobrevivió a una de las esferas en cuarentena implicadas en la primera rebelión que tuvo lugar en los universos del tiempo y del espacio.

\par
%\textsuperscript{(579.2)}
\textsuperscript{50:7.3} A lo largo de toda la carrera hacia el Paraíso, la recompensa sigue al esfuerzo como consecuencia de las causas. Estas recompensas separan al individuo del término medio, proporcionan un diferencial en la experiencia de las criaturas, y contribuyen al carácter polifacético de las realizaciones últimas en el cuerpo colectivo de los finalitarios.

\par
%\textsuperscript{(579.3)}
\textsuperscript{50:7.4} [Presentado por un Hijo Lanonandek Secundario del Cuerpo de Reserva.]


\chapter{Documento 51. Los Adanes Planetarios}
\par
%\textsuperscript{(580.1)}
\textsuperscript{51:0.1} DURANTE la dispensación de un Príncipe Planetario, el hombre primitivo alcanza el límite del desarrollo evolutivo natural, y este logro biológico da la señal al Soberano del Sistema para el envío a ese mundo de la segunda orden de filiación, los mejoradores biológicos. A estos Hijos, pues son dos, ---el Hijo y la Hija Materiales--- se les conoce generalmente en un planeta como Adán y Eva. El Hijo Material original de Satania es Adán, y aquellos que van a los mundos del sistema como mejoradores biológicos siempre llevan el nombre de este primer Hijo original de su orden excepcional.

\par
%\textsuperscript{(580.2)}
\textsuperscript{51:0.2} Estos Hijos son el don material del Hijo Creador a los mundos habitados. Permanecen en el planeta donde han sido destinados, junto con el Príncipe Planetario, durante toda la trayectoria evolutiva de esa esfera. Una aventura así en un mundo que tiene un Príncipe Planetario dista mucho de ser un riesgo, pero en un planeta apóstata, en un reino sin un gobernante espiritual y privado de las comunicaciones interplanetarias, una misión así está llena de graves peligros.

\par
%\textsuperscript{(580.3)}
\textsuperscript{51:0.3} Aunque no podéis esperar saberlo todo sobre el trabajo de estos Hijos en todos los mundos de Satania y de otros sistemas, otros documentos describen más plenamente la vida y las experiencias de Adán y Eva, la interesante pareja del cuerpo de mejoradores biológicos de Jerusem que vino para elevar a las razas de Urantia. Aunque los planes ideales para mejorar vuestras razas nativas fracasaron, sin embargo la misión de Adán no tuvo lugar en vano; Urantia se ha beneficiado inconmensurablemente del don de Adán y Eva, y entre sus compañeros y en los consejos de las alturas su trabajo no es considerado como una pérdida total.

\section*{1. El origen y la naturaleza de los Hijos Materiales de Dios}
\par
%\textsuperscript{(580.4)}
\textsuperscript{51:1.1} Los Hijos y las Hijas materiales o sexuados son la progenie del Hijo Creador; el Espíritu Madre del Universo no participa en la creación de estos seres que están destinados a ejercer su actividad como mejoradores físicos en los mundos evolutivos.

\par
%\textsuperscript{(580.5)}
\textsuperscript{51:1.2} La orden material de filiación no es uniforme en todo el universo local. El Hijo Creador sólo engendra una pareja de estos seres en cada sistema local; la naturaleza de estas parejas originales es diversa, estando sintonizada con la configuración de vida de sus sistemas respectivos. Es una disposición necesaria puesto que, de otra manera, el potencial reproductor de los Adanes no funcionaría con el de los seres mortales evolutivos de los mundos de un sistema particular cualquiera. El Adán y la Eva que vinieron a Urantia descendían de la pareja original de Hijos Materiales de Satania.

\par
%\textsuperscript{(580.6)}
\textsuperscript{51:1.3} La estatura de los Hijos Materiales varía entre los dos metros y medio y los tres metros, y su cuerpo resplandece con el brillo de una luz radiante de tinte violeta. Aunque la sangre material circula por sus cuerpos materiales, también están sobrecargados de energía divina y saturados de luz celestial. Estos Hijos Materiales (los Adanes) y estas Hijas Materiales (las Evas) son iguales entre sí, y sólo difieren en su naturaleza reproductora y en ciertas dotaciones químicas. Son iguales pero diferenciales, masculino y femenino ---en consecuencia complementarios--- y están diseñados para servir en parejas en casi todas sus misiones.

\par
%\textsuperscript{(581.1)}
\textsuperscript{51:1.4} Los Hijos Materiales disfrutan de una nutrición doble; son realmente dobles en su naturaleza y en su constitución, consumiendo la energía materializada poco más o menos como lo hacen los seres físicos del reino, mientras que su existencia inmortal se mantiene plenamente mediante la absorción directa y automática de ciertas energías cósmicas sustentadoras. Si fracasan en alguna misión asignada o incluso si se rebelan de forma consciente y deliberada, los Hijos de esta orden son aislados, se les corta la conexión con la fuente universal de la luz y la vida. Inmediatamente después se vuelven prácticamente seres materiales, destinados a seguir el curso de la vida material en el mundo donde están asignados, y obligados a recurrir a los magistrados del universo para ser juzgados. La muerte material terminará finalmente con la carrera planetaria de esta Hija o de este Hijo Material desacertado e imprudente.

\par
%\textsuperscript{(581.2)}
\textsuperscript{51:1.5} Un Adán y una Eva originales o directamente creados son inmortales por don inherente, como lo son todas las otras órdenes de filiación del universo local, pero sus hijos e hijas están caracterizados por una disminución del potencial de inmortalidad. Esta pareja original no puede transmitir la inmortalidad incondicionada a los hijos e hijas que procrea. Para continuar viviendo, su progenie depende de un sincronismo intelectual ininterrumpido con el circuito de gravedad mental del Espíritu. Desde los comienzos del sistema de Satania, trece Adanes Planetarios se han perdido por rebelión y por faltas y 681.204 en puestos de confianza subordinados. La mayoría de estas deserciones se produjeron en la época de la rebelión de Lucifer.

\par
%\textsuperscript{(581.3)}
\textsuperscript{51:1.6} Mientras viven como ciudadanos permanentes en las capitales de los sistemas, e incluso cuando cumplen misiones descendentes en los planetas evolutivos, los Hijos Materiales no poseen Ajustador del Pensamiento, pero gracias a estos servicios mismos es como adquieren la capacidad experiencial para ser habitados por un Ajustador y para emprender la carrera de ascensión hacia el Paraíso. Estos seres únicos y maravillosamente útiles son el eslabón que conecta el mundo espiritual con el mundo físico. Están concentrados en las sedes de los sistemas, donde se reproducen y continúan viviendo como ciudadanos materiales del reino, y desde allí son enviados a los mundos evolutivos.

\par
%\textsuperscript{(581.4)}
\textsuperscript{51:1.7} A diferencia de los otros Hijos creados que sirven en los planetas, la orden material de filiación no es, por naturaleza, invisible para las criaturas materiales tales como los habitantes de Urantia. Estos Hijos de Dios pueden ser vistos y comprendidos por las criaturas del tiempo, y a su vez pueden mezclarse realmente con ellas, e incluso podrían procrear con ellas, aunque esta función de elevación biológica recae generalmente sobre la progenie de los Adanes Planetarios.

\par
%\textsuperscript{(581.5)}
\textsuperscript{51:1.8} En Jerusem, los hijos leales de un Adán y una Eva son inmortales, pero los descendientes procreados por un Hijo y una Hija Materiales después de haber llegado a un planeta evolutivo no están inmunizados así contra la muerte natural. Cuando estos Hijos son rematerializados para ejercer su función reproductora en un mundo evolutivo se produce un cambio en el mecanismo de trasmisión de la vida. Los Portadores de Vida privan adrede a los Adanes y las Evas Planetarios del poder de engendrar hijos e hijas que no mueren. Si no cometen una falta, un Adán y una Eva en misión planetaria pueden vivir indefinidamente, pero dentro de ciertos límites, sus hijos experimentan una longevidad decreciente en cada nueva generación.

\section*{2. El transporte de los Adanes Planetarios}
\par
%\textsuperscript{(582.1)}
\textsuperscript{51:2.1} Cuando recibe la noticia de que otro mundo habitado ha alcanzado el punto culminante de la evolución física, el Soberano del Sistema convoca al cuerpo de Hijos e Hijas Materiales en la capital del sistema; después de analizar las necesidades de ese mundo evolutivo, dos miembros del grupo de voluntarios ---un Adán y una Eva del cuerpo más antiguo de Hijos Materiales--- son elegidos para emprender la aventura, para someterse al sueño profundo antes de ser enserafinados y transportados desde el hogar donde efectúan su servicio asociado hasta el nuevo reino con sus nuevas oportunidades y sus nuevos peligros.

\par
%\textsuperscript{(582.2)}
\textsuperscript{51:2.2} Los Adanes y las Evas son criaturas semimateriales y, como tales, no pueden ser transportadas por los serafines. Deben someterse a la desmaterialización en la capital del sistema antes de poder ser enserafinadas para el transporte hasta el mundo de destino. Los serafines transportadores son capaces de efectuar en los Hijos Materiales y en otros seres semimateriales los cambios que les permitirán ser enserafinados y transportados así a través del espacio desde un mundo o un sistema a otro. Esta preparación para el transporte dura unos tres días del tiempo oficial, y se necesita la cooperación de un Portador de Vida para devolver a su existencia normal a esta criatura desmaterializada cuando llega al final de su viaje por transporte seráfico.

\par
%\textsuperscript{(582.3)}
\textsuperscript{51:2.3} Aunque existe esta técnica de desmaterialización para preparar a los Adanes a fin de ser transportados desde Jerusem hasta los mundos evolutivos, no existe un método equivalente para sacarlos de dichos mundos a menos que se vacíe todo el planeta, en cuyo caso se instala de urgencia la técnica de la desmaterialización para toda la población salvable. Si una catástrofe física pusiera en peligro la residencia planetaria de una raza en evolución, los Melquisedeks y los Portadores de Vida instalarían la técnica de la desmaterialización para todos los supervivientes, y estos seres serían llevados por transporte seráfico hasta el nuevo mundo preparado para continuar su existencia. Una vez que la evolución de una raza humana ha empezado en un mundo del espacio, debe continuar independientemente por completo de la supervivencia física de ese planeta, pero durante las épocas evolutivas, no está planeado de otra manera que un Adán o una Eva Planetarios dejen el mundo que han elegido.

\par
%\textsuperscript{(582.4)}
\textsuperscript{51:2.4} Cuando llegan a su destino planetario, el Hijo y la Hija Materiales son rematerializados bajo la dirección de los Portadores de Vida. El proceso completo dura entre diez y veintiocho días del tiempo de Urantia. La inconciencia del sueño seráfico continúa durante todo este período de reconstrucción. Cuando el reensamblaje del organismo físico ha terminado, estos Hijos e Hijas Materiales se encuentran en su nuevo hogar y en su nuevo mundo prácticamente tal como estaban antes de someterse al proceso de desmaterialización en Jerusem.

\section*{3. Las misiones adámicas}
\par
%\textsuperscript{(582.5)}
\textsuperscript{51:3.1} En los mundos habitados, los Hijos y las Hijas Materiales construyen sus propios hogares jardín, y pronto reciben la ayuda de sus propios hijos. El emplazamiento del jardín ha sido elegido generalmente por el Príncipe Planetario, y su estado mayor corpóreo efectúa una gran parte del trabajo preliminar de preparación con la ayuda de muchos individuos superiores de las razas nativas.

\par
%\textsuperscript{(583.1)}
\textsuperscript{51:3.2} Estos Jardines del Edén\footnote{\textit{Edén}: Gn 2:8-10.} se llaman así en homenaje a Edentia, la capital de la constelación, y porque están modelados según la grandiosidad botánica del mundo sede de los Padres Altísimos. Estos hogares jardín están habitualmente situados en una región apartada y en una zona cercana a los trópicos. En un mundo de tipo medio, son unas creaciones maravillosas. No podéis formaros ninguna opinión sobre estos hermosos centros de cultura por el relato fragmentario del desarrollo abortado de una empresa así en Urantia.

\par
%\textsuperscript{(583.2)}
\textsuperscript{51:3.3} Un Adán y una Eva Planetarios son, en potencia, el don completo de la gracia física para las razas mortales. La tarea principal de esta pareja importada consiste en multiplicarse y en mejorar a los hijos del tiempo. Pero no se produce un cruce inmediato entre la población del jardín y los pueblos del mundo. Durante muchas generaciones, Adán y Eva permanecen biológicamente separados de los mortales evolutivos, mientras construyen una fuerte raza de su orden. Éste es el origen de la raza violeta en los mundos habitados.

\par
%\textsuperscript{(583.3)}
\textsuperscript{51:3.4} Los planes para mejorar la raza son preparados por el Príncipe Planetario y su estado mayor, y ejecutados por Adán y Eva. Y aquí es donde vuestro Hijo Material y su compañera tuvieron una gran desventaja cuando llegaron a Urantia. Caligastia se opuso con astucia y eficacia a la misión adámica; y a pesar de que los síndicos Melquisedeks de Urantia habían advertido debidamente tanto a Adán como a Eva de los peligros planetarios inherentes a la presencia del Príncipe Planetario rebelde, este archirrebelde, mediante una astuta estratagema, se mostró más hábil que la pareja edénica y los hizo caer en la trampa de violar el pacto de su fideicomiso como gobernantes visibles de vuestro mundo. El Príncipe Planetario traidor logró comprometer a vuestro Adán y a vuestra Eva, pero fracasó en su esfuerzo por implicarlos en la rebelión de Lucifer.

\par
%\textsuperscript{(583.4)}
\textsuperscript{51:3.5} Los ángeles de la quinta orden, los ayudantes planetarios, están vinculados a la misión adámica, y siempre acompañan a los Adanes Planetarios en sus aventuras en los mundos. El cuerpo que se asigna inicialmente está compuesto por lo general de unos cien mil miembros. Cuando el Adán y la Eva de Urantia emprendieron su trabajo de manera prematura, cuando se apartaron del plan ordenado, una de las Voces seráficas del Jardín\footnote{\textit{Voces del jardín}: Gn 3:8-19.} fue la que los amonestó por su conducta reprensible. El relato que poseéis sobre este suceso ilustra bien la manera en que vuestras tradiciones planetarias han tendido a imputarle al Señor Dios todo lo que es sobrenatural. A causa de esto, los urantianos han llegado a confundirse a menudo sobre la naturaleza del Padre Universal, puesto que generalmente se le han atribuido las palabras y los actos de todos sus asociados y subordinados. En el caso de Adán y Eva, el ángel del Jardín no era otro que el jefe de los ayudantes planetarios entonces de servicio. Este serafín, llamado Solonia, proclamó el fracaso del plan divino y solicitó el regreso de los síndicos Melquisedeks a Urantia.

\par
%\textsuperscript{(583.5)}
\textsuperscript{51:3.6} Las criaturas intermedias secundarias forman parte de los descendientes autóctonos de las misiones adámicas. Al igual que sucede con el estado mayor corpóreo del Príncipe Planetario, los descendientes de los Hijos y las Hijas Materiales son de dos tipos: sus hijos físicos y la orden secundaria de criaturas intermedias. Estos ministros planetarios materiales, pero generalmente invisibles, contribuyen mucho al avance de la civilización e incluso al sometimiento de las minorías insubordinadas que pueden intentar socavar las bases del desarrollo social y del progreso espiritual.

\par
%\textsuperscript{(583.6)}
\textsuperscript{51:3.7} No se debe confundir a los intermedios secundarios con la orden primaria, que data de los tiempos cercanos a la llegada del Príncipe Planetario. En Urantia, la mayoría de estas criaturas intermedias iniciales se unieron a la rebelión con Caligastia, y han estado internadas desde Pentecostés. Muchos miembros del grupo adámico que no permanecieron leales a la administración planetaria están internados de la misma manera.

\par
%\textsuperscript{(584.1)}
\textsuperscript{51:3.8} El día de Pentecostés, los intermedios leales primarios y los secundarios llevaron a cabo una unión voluntaria, y desde entonces han actuado como una sola unidad en los asuntos del mundo. Sirven bajo el mando de los intermedios leales elegidos alternativamente en los dos grupos.

\par
%\textsuperscript{(584.2)}
\textsuperscript{51:3.9} Vuestro mundo ha sido visitado por cuatro órdenes de filiación: Caligastia, el Príncipe Planetario\footnote{\textit{Antiguo Príncipe Planetario}: Jn 12:31; 14:30; 16:11; 2 Co 4:4; Ef 2:2-3.}; Adán\footnote{\textit{Adán}: Gn 2:19-23.} y Eva\footnote{\textit{Eva}: Gn 3:20.}, de los Hijos Materiales de Dios; Maquiventa Melquisedek\footnote{\textit{Melquisedek}: Gn 14:18 ff; Sal 110:4; Heb 5:6,10; 6:20; 7:1-3,10,17; 7:21.}, el <<sabio de Salem>> en los tiempos de Abraham; y Cristo Miguel\footnote{\textit{Cristo Miguel}: Mt 2:1; Lc 2:4-7,11; Jn 1:14; 3:16.}, que vino como Hijo paradisiaco donador. ¡Cuánto más eficaz y hermoso hubiera sido si Miguel, el gobernante supremo del universo de Nebadon, hubiera sido acogido en vuestro mundo por un Príncipe Planetario leal y eficiente y por un Hijo Material dedicado y que ha tenido éxito, los dos que podrían haber hecho tanto por realzar la misión y el trabajo de la vida del Hijo donador! Pero no todos los mundos han sido tan desafortunados como Urantia, y las misiones de los Adanes Planetarios tampoco han sido siempre tan difíciles o tan peligrosas. Cuando tienen éxito, contribuyen al desarrollo de un gran pueblo, continuando como jefes visibles de los asuntos planetarios incluso mucho tiempo después de que ese mundo se ha establecido en la luz y la vida.

\section*{4. Las seis razas evolutivas}
\par
%\textsuperscript{(584.3)}
\textsuperscript{51:4.1} La raza que domina durante las primeras eras de los mundos habitados es la del hombre rojo, que es habitualmente la primera en alcanzar los niveles humanos de desarrollo. Pero aunque el hombre rojo es la raza más antigua de los planetas, los pueblos siguientes de color empiezan a hacer su aparición al principio de la era en que surgen los mortales.

\par
%\textsuperscript{(584.4)}
\textsuperscript{51:4.2} Las primeras razas son un poco superiores a las posteriores; el hombre rojo se halla muy por encima de la raza índiga ---negra. Los Portadores de Vida confieren el don completo de las energías vivientes a la raza roja o inicial, y cada manifestación evolutiva sucesiva de un grupo distinto de mortales representa una variación a expensas de la dotación original. Incluso la estatura de los mortales tiende a disminuir desde el hombre rojo hasta la raza índiga, aunque en Urantia aparecieron linajes inesperados de gigantismo entre los pueblos verde y anaranjado.

\par
%\textsuperscript{(584.5)}
\textsuperscript{51:4.3} En aquellos mundos que tienen las seis razas evolutivas, los pueblos superiores son la primera, la tercera y la quinta razas ---la roja, la amarilla y la azul. Las razas evolutivas alternan así en capacidad para el crecimiento intelectual y el desarrollo espiritual, estando la segunda, la cuarta y la sexta un poco menos dotadas. Estas razas secundarias son los pueblos que faltan en ciertos mundos; son los que han sido exterminados en otros muchos. Es una desgracia que en Urantia hayáis perdido tan ampliamente a vuestros hombres azules superiores, salvo en la medida en que subsisten en vuestra <<raza blanca>> amalgamada. La pérdida de vuestros linajes naranja y verde no es de un interés tan importante.

\par
%\textsuperscript{(584.6)}
\textsuperscript{51:4.4} La evolución de seis ---o de tres--- razas de color, aunque parezca deteriorar la dotación original del hombre rojo, proporciona ciertas variaciones muy deseables en los tipos mortales y permite una expresión, de otra manera inalcanzable, de los diversos potenciales humanos. Estas modificaciones son beneficiosas para el progreso de la humanidad en su totalidad, con tal que sean posteriormente mejoradas por la raza adámica o violeta importada. En Urantia, este plan normal de amalgamación no se llevó ampliamente a cabo, y este fracaso en la ejecución del plan para la evolución racial hace que os resulte imposible comprender muchas cosas sobre el estado de estos pueblos en un planeta habitado de tipo medio a través de la observación de los restos de estas primeras razas de vuestro mundo.

\par
%\textsuperscript{(585.1)}
\textsuperscript{51:4.5} En los primeros tiempos del desarrollo racial, los hombres rojos, amarillos y azules tienen una ligera tendencia a cruzarse; las razas anaranjada, verde e índiga tienen una tendencia similar a entremezclarse.

\par
%\textsuperscript{(585.2)}
\textsuperscript{51:4.6} Las razas más progresivas utilizan habitualmente como obreros a los humanos más atrasados. Esto explica el origen de la esclavitud en los planetas durante las épocas primitivas. Los hombres rojos normalmente someten a los anaranjados y los reducen a la condición de sirvientes ---a veces son exterminados. Los hombres amarillos y los rojos fraternizan a menudo, pero no siempre. La raza amarilla esclaviza habitualmente a la verde, mientras que el hombre azul somete al índigo. Para estas razas de hombres primitivos, el utilizar los servicios de sus compañeros atrasados en trabajos forzosos no supone más de lo que significa para los urantianos el hecho de comprar y vender caballos y ganado.

\par
%\textsuperscript{(585.3)}
\textsuperscript{51:4.7} En la mayoría de los mundos normales, la servidumbre involuntaria no sobrevive a la dispensación del Príncipe Planetario, aunque los deficientes mentales y los delincuentes sociales son a menudo todavía obligados a realizar trabajos involuntarios. Pero en todas las esferas normales, esta especie de esclavitud primitiva es abolida poco después de la llegada de la raza adámica o violeta importada.

\par
%\textsuperscript{(585.4)}
\textsuperscript{51:4.8} Estas seis razas evolutivas están destinadas a mezclarse y a ser realzadas mediante su amalgamación con la progenie de los mejoradores adámicos. Pero antes de que estos pueblos se mezclen, los inferiores y los incapaces son eliminados en su mayoría. El Príncipe Planetario y el Hijo Material, con otras autoridades planetarias adecuadas, se pronuncian sobre la aptitud de los linajes reproductores. La dificultad para ejecutar un programa radical como éste en Urantia consiste en la ausencia de jueces competentes para decidir sobre la aptitud o la incapacidad biológica de los individuos de las razas de vuestro mundo. A pesar de este obstáculo, parece ser que deberíais ser capaces de poneros de acuerdo sobre la exclusión biológica de vuestros linajes más acusadamente incapaces, deficientes, degenerados y antisociales.

\section*{5. La amalgamación racial ---la donación de la sangre adámica}
\par
%\textsuperscript{(585.5)}
\textsuperscript{51:5.1} Cuando un Adán y una Eva Planetarios llegan a un mundo habitado, sus superiores les han informado plenamente sobre la manera más conveniente de efectuar el mejoramiento de las razas existentes de seres inteligentes. El plan del procedimiento no es uniforme; una gran parte se deja al juicio de la pareja ministrante, y los errores no son raros, especialmente en los mundos desordenados e insurrectos tales como Urantia.

\par
%\textsuperscript{(585.6)}
\textsuperscript{51:5.2} Generalmente, los pueblos violetas no empiezan a amalgamarse con los nativos planetarios hasta que su propio grupo no asciende a más de un millón de miembros. Pero mientras tanto, el estado mayor del Príncipe Planetario proclama que los hijos de los Dioses han descendido para fundirse, por así decirlo, con las razas de los hombres; y la gente espera con impaciencia el día en que se anunciará que aquellos que han satisfecho los requisitos para pertenecer a los linajes raciales superiores pueden dirigirse hacia el Jardín del Edén para ser elegidos allí por los hijos y las hijas de Adán como padres y madres evolutivos del nuevo tipo mezclado de humanidad.

\par
%\textsuperscript{(585.7)}
\textsuperscript{51:5.3} En los mundos normales, el Adán y la Eva Planetarios no se emparejan nunca con las razas evolutivas. Este trabajo de mejoramiento biológico es una función de la progenie adámica. Pero estos adamitas no salen hacia las razas; el estado mayor del príncipe trae al Jardín del Edén a los hombres y mujeres superiores para que se emparejen voluntariamente con la descendencia adámica. Y en la mayoría de los mundos se considera que el honor más elevado es ser elegido como candidato para casarse con los hijos y las hijas del jardín.

\par
%\textsuperscript{(586.1)}
\textsuperscript{51:5.4} Las guerras raciales y otras luchas tribales disminuyen por primera vez, mientras que las razas del mundo se esfuerzan cada vez más por capacitarse para ser reconocidas y admitidas en el jardín. Vosotros sólo podéis tener, en el mejor de los casos, una idea muy pobre sobre la manera en que esta lucha competitiva llega a ocupar el centro de todas las actividades en un planeta normal. Todo este proyecto de mejora racial se hundió muy pronto en Urantia.

\par
%\textsuperscript{(586.2)}
\textsuperscript{51:5.5} La raza violeta es un pueblo monógamo, y todo hombre o mujer evolutivos que se une con los hijos y las hijas adámicos promete no tomar otros cónyuges y enseñar la monogamia a sus hijos e hijas. Los hijos de cada una de estas uniones son educados e instruidos en las escuelas del Príncipe Planetario, y luego se les permite ir hacia la raza de su progenitor evolutivo para casarse allí entre los grupos seleccionados de mortales superiores.

\par
%\textsuperscript{(586.3)}
\textsuperscript{51:5.6} Cuando este linaje de los Hijos Materiales se añade a las razas en evolución de los mundos, da comienzo una nueva era más grande de progreso evolutivo. Después de esta efusión procreadora de capacidades importadas y de características superevolutivas, se produce una sucesión de rápidos avances en la civilización y en el desarrollo racial; en cien mil años se hacen más progresos que en un millón de años de luchas anteriores. En vuestro mundo se han realizado grandes progresos, a pesar del fracaso de los planes ordenados, desde que el plasma vital de Adán fue donado a vuestros pueblos.

\par
%\textsuperscript{(586.4)}
\textsuperscript{51:5.7} Pero aunque los hijos de pura cepa de un Jardín del Edén planetario pueden donarse a los miembros superiores de las razas evolutivas y mejorar así el nivel biológico de la humanidad, a los linajes superiores de los mortales de Urantia no les resultaría beneficioso emparejarse con las razas inferiores; un proceder tan poco sabio como éste pondría en peligro toda la civilización en vuestro mundo. Como no se ha logrado llevar a cabo la armonización racial mediante la técnica adámica, ahora tenéis que resolver vuestro problema planetario de mejoramiento racial mediante otros métodos de adaptación y de control, principalmente humanos.

\section*{6. El régimen edénico}
\par
%\textsuperscript{(586.5)}
\textsuperscript{51:6.1} En la mayoría de los mundos habitados, los Jardines del Edén permanecen como magníficos centros culturales y continúan funcionando época tras época como modelos sociales de conducta y de costumbres planetarias. Incluso en los primeros tiempos, cuando los pueblos violetas están relativamente aislados, sus escuelas reciben a los candidatos apropiados procedentes de las razas del mundo, mientras que los desarrollos industriales del jardín abren nuevos canales de relaciones comerciales. Así es como los Adanes y las Evas y su progenie contribuyen a la expansión repentina de la cultura y al rápido mejoramiento de las razas evolutivas de sus mundos. La amalgamación de las razas evolutivas con los hijos de Adán acrecienta y sella todas estas relaciones, teniendo como resultado el mejoramiento inmediato del estado biológico, la estimulación del potencial intelectual y el aumento de la receptividad espiritual.

\par
%\textsuperscript{(586.6)}
\textsuperscript{51:6.2} En los mundos normales, la sede jardín de la raza violeta se convierte en el segundo centro de la cultura mundial y, junto con la ciudad sede del Príncipe Planetario, marca la pauta del desarrollo de la civilización. Las escuelas de la ciudad sede del Príncipe Planetario y las escuelas del jardín de Adán y Eva son contemporáneas durante siglos. Generalmente no están muy alejadas, y trabajan juntas en una cooperación armoniosa.

\par
%\textsuperscript{(587.1)}
\textsuperscript{51:6.3} Pensad en lo que significaría para vuestro mundo que en alguna parte del Levante hubiera un centro mundial de civilización, una gran universidad planetaria de cultura, que hubiera funcionado sin interrupción durante 37.000 años. Y además deteneos a considerar de qué manera estaría reforzada la autoridad moral de un centro tan antiguo como éste, si no muy lejos de allí estuviera situada otra sede aún más antigua de ministerio celestial cuyas tradiciones ejercieran una fuerza acumulada de 500.000 años de influencia evolutiva integrada. Es la costumbre la que difunde con el tiempo los ideales del Edén en un mundo entero.

\par
%\textsuperscript{(587.2)}
\textsuperscript{51:6.4} Las escuelas del Príncipe Planetario se ocupan principalmente de la filosofía, la religión, la moral y las realizaciones intelectuales y artísticas superiores. Las escuelas del jardín de Adán y Eva se dedican habitualmente a las artes prácticas, la formación intelectual básica, la cultura social, el desarrollo económico, las relaciones comerciales, la eficacia física y el gobierno civil. Estos centros mundiales se amalgaman finalmente, pero esta asociación efectiva a veces no se produce hasta la época del primer Hijo Magistral.

\par
%\textsuperscript{(587.3)}
\textsuperscript{51:6.5} La existencia continuada del Adán y de la Eva Planetarios, junto con el núcleo de linaje puro de la raza violeta, comunica a la cultura edénica esa estabilidad de crecimiento en virtud de la cual llega a actuar sobre la civilización de un mundo con la fuerza irresistible de la tradición. En estos Hijos e Hijas Materiales inmortales encontramos al último eslabón indispensable que conecta a Dios con el hombre, que colma el abismo casi infinito entre el Creador eterno y las personalidades finitas más humildes del tiempo. He aquí a un ser de alto origen que es físico, material, e incluso una criatura sexuada como los mortales de Urantia, que puede ver y comprender al Príncipe Planetario invisible y servirle de intérprete ante las criaturas mortales del reino, pues los Hijos y las Hijas Materiales son capaces de ver a todas las órdenes inferiores de seres espirituales; visualizan al Príncipe Planetario y a todo su estado mayor, visible e invisible.

\par
%\textsuperscript{(587.4)}
\textsuperscript{51:6.6} Con el paso de los siglos, y gracias a la amalgamación de su progenie con las razas de los hombres, este mismo Hijo y esta misma Hija Materiales son aceptados como antepasados comunes de la humanidad, como los padres comunes de los descendientes ahora mezclados de las razas evolutivas. Se tiene la intención de que los mortales que salen de un mundo habitado tengan la experiencia de reconocer a siete padres:

\par
%\textsuperscript{(587.5)}
\textsuperscript{51:6.7} 1. El padre biológico ---el padre carnal.

\par
%\textsuperscript{(587.6)}
\textsuperscript{51:6.8} 2. El padre del reino ---el Adán Planetario.

\par
%\textsuperscript{(587.7)}
\textsuperscript{51:6.9} 3. El padre de las esferas ---el Soberano del Sistema.

\par
%\textsuperscript{(587.8)}
\textsuperscript{51:6.10} 4. El Padre Altísimo ---el Padre de la Constelación.

\par
%\textsuperscript{(587.9)}
\textsuperscript{51:6.11} 5. El Padre del universo ---el Hijo Creador y gobernante supremo de las creaciones locales.

\par
%\textsuperscript{(587.10)}
\textsuperscript{51:6.12} 6. Los super-Padres ---los Ancianos de los Días que gobiernan el superuniverso.

\par
%\textsuperscript{(587.11)}
\textsuperscript{51:6.13} 7. El Padre espiritual o Padre de Havona ---el Padre Universal que reside en el Paraíso y que confiere su espíritu para que viva y trabaje en la mente de las humildes criaturas que habitan el universo de universos.

\section*{7. La administración unida}
\par
%\textsuperscript{(587.12)}
\textsuperscript{51:7.1} Los Hijos Avonales del Paraíso vienen de vez en cuando a los mundos habitados para llevar a cabo acciones judiciales, pero el primer Avonal que llega en misión magistral inaugura la cuarta dispensación de un mundo evolutivo del tiempo y del espacio. En algunos planetas donde este Hijo Magistral es aceptado de manera universal, permanece allí durante una era; y el planeta prospera así bajo el mando conjunto de tres Hijos: el Príncipe Planetario, el Hijo Material y el Hijo Magistral, siendo los dos últimos visibles para todos los habitantes del reino.

\par
%\textsuperscript{(588.1)}
\textsuperscript{51:7.2} Antes de que el primer Hijo Magistral concluya su misión en un mundo evolutivo normal, ya se ha efectuado la unión del trabajo educativo y administrativo del Príncipe Planetario y del Hijo Material. Esta amalgamación de la doble supervisión de un planeta trae a la existencia un tipo nuevo y eficaz de administración mundial. Cuando el Hijo Magistral se retira, el Adán Planetario asume la dirección exterior de la esfera. El Hijo y la Hija Materiales actúan conjuntamente así como administradores planetarios hasta que el mundo se establece en la era de luz y de vida; después de lo cual, el Príncipe Planetario es elevado a la posición de Soberano Planetario. Durante esta era de evolución avanzada, Adán y Eva se convierten en lo que se podría llamar primeros ministros conjuntos del reino glorificado.

\par
%\textsuperscript{(588.2)}
\textsuperscript{51:7.3} Tan pronto como la nueva capital consolidada del mundo evolutivo está bien instalada, y tan rápidamente como se puede instruir de manera adecuada a unos administradores subordinados competentes, se fundan subcapitales en los territorios lejanos y entre los diferentes pueblos. Antes de que llegue otro Hijo dispensacional se habrán organizado entre cincuenta y cien subcentros de este tipo.

\par
%\textsuperscript{(588.3)}
\textsuperscript{51:7.4} El Príncipe Planetario y su estado mayor siguen fomentando los campos de actividad espirituales y filosóficos. Adán y Eva prestan una atención particular al estado físico, científico y económico del reino. Los dos grupos dedican igualmente sus energías a promover las artes, las relaciones sociales y los logros intelectuales.

\par
%\textsuperscript{(588.4)}
\textsuperscript{51:7.5} En el momento de inaugurarse la quinta dispensación de los asuntos del mundo, se ha conseguido una magnífica administración de las actividades planetarias. La existencia mortal en una esfera tan bien gestionada es en verdad estimulante y beneficiosa. Si los urantianos tan sólo pudieran observar la vida en un planeta así, apreciarían de inmediato el valor de aquellas cosas que su mundo ha perdido por haber abrazado el mal y haber participado en la rebelión.

\par
%\textsuperscript{(588.5)}
\textsuperscript{51:7.6} [Presentado por un Hijo Lanonandek Secundario del Cuerpo de Reserva.]


\chapter{Documento 52. Las épocas planetarias de los mortales}
\par
%\textsuperscript{(589.1)}
\textsuperscript{52:0.1} DESDE el comienzo de la vida en un planeta evolutivo hasta el momento de su florecimiento final en la era de luz y de vida, en el escenario de la acción del mundo aparecen al menos siete épocas de vida humana. Estas eras sucesivas están determinadas por las misiones planetarias de los Hijos divinos, y en un mundo habitado de tipo medio, estas épocas aparecen en el orden siguiente:

\par
%\textsuperscript{(589.2)}
\textsuperscript{52:0.2} 1. El Hombre anterior al Príncipe Planetario.

\par
%\textsuperscript{(589.3)}
\textsuperscript{52:0.3} 2. El Hombre posterior al Príncipe Planetario.

\par
%\textsuperscript{(589.4)}
\textsuperscript{52:0.4} 3. El Hombre postadámico.

\par
%\textsuperscript{(589.5)}
\textsuperscript{52:0.5} 4. El Hombre posterior al Hijo Magistral.

\par
%\textsuperscript{(589.6)}
\textsuperscript{52:0.6} 5. El Hombre posterior al Hijo Donador.

\par
%\textsuperscript{(589.7)}
\textsuperscript{52:0.7} 6. El Hombre posterior al Hijo Instructor.

\par
%\textsuperscript{(589.8)}
\textsuperscript{52:0.8} 7. La Era de Luz y de Vida.

\par
%\textsuperscript{(589.9)}
\textsuperscript{52:0.9} Tan pronto como los mundos del espacio son físicamente adecuados para la vida, son inscritos en el registro de los Portadores de Vida y, a su debido tiempo, estos Hijos son enviados a esos planetas con el fin de iniciar la vida. Todo el período que transcurre desde el inicio de la vida hasta la aparición del hombre se denomina era prehumana y precede a las sucesivas épocas humanas que se examinan en esta narración.

\section*{1. El hombre primitivo}
\par
%\textsuperscript{(589.10)}
\textsuperscript{52:1.1} Desde el momento en que el hombre emerge del nivel animal ---cuando puede elegir adorar al Creador--- hasta la llegada del Príncipe Planetario, las criaturas volitivas mortales se denominan \textit{hombres primitivos}. Hay seis tipos básicos o razas de hombres primitivos, y estos pueblos iniciales aparecen sucesivamente en el orden de los colores del espectro, empezando por el rojo. La cantidad de tiempo que se consume en esta evolución primitiva de la vida varía enormemente en los diferentes mundos, oscilando entre ciento cincuenta mil y más de un millón de años del tiempo de Urantia.

\par
%\textsuperscript{(589.11)}
\textsuperscript{52:1.2} Las razas evolutivas de color ---roja, anaranjada, amarilla, verde, azul e índiga--- empiezan a aparecer hacia la época en que el hombre primitivo desarrolla un lenguaje sencillo y empieza a ejercer su imaginación creativa. Para entonces, el hombre está bien acostumbrado a permanecer erguido.

\par
%\textsuperscript{(589.12)}
\textsuperscript{52:1.3} Los hombres primitivos son unos cazadores extraordinarios y unos luchadores feroces. La ley de esta era es la supervivencia física de los más capacitados; el gobierno de estos tiempos es totalmente tribal. En muchos mundos, algunas razas evolutivas son eliminadas durante las luchas raciales primitivas, tal como sucedió en Urantia. Habitualmente, aquellos que sobreviven se mezclan posteriormente con la raza violeta importada más tarde, con los pueblos adámicos.

\par
%\textsuperscript{(589.13)}
\textsuperscript{52:1.4} A la luz de la civilización posterior, esta era del hombre primitivo es un largo capítulo sombrío y sangriento. La ley de la jungla y la moral de los bosques primitivos no están de acuerdo con los valores morales de las dispensaciones más tardías con su religión revelada y su desarrollo espiritual superior. En los mundos normales y no experimentales, esta época es muy diferente a la de las luchas prolongadas y extraordinariamente brutales que caracterizaron a esta era en Urantia. Cuando emerjáis de la experiencia de vuestro primer mundo, empezaréis a ver por qué esta larga y dolorosa lucha tiene lugar en los mundos evolutivos, y a medida que avancéis por el camino hacia el Paraíso, comprenderéis cada vez mejor la sabiduría de estos hechos aparentemente extraños. Pero a pesar de todas las vicisitudes de las primeras eras de la aparición humana, las realizaciones del hombre primitivo representan un capítulo espléndido, e incluso heroico, en los anales de un mundo evolutivo del tiempo y del espacio.

\par
%\textsuperscript{(590.1)}
\textsuperscript{52:1.5} El hombre evolutivo inicial no es una criatura pintoresca. Estos mortales primitivos viven generalmente en cuevas o residen en los acantilados. También construyen cabañas rudimentarias en los grandes árboles. Antes de que adquieran un elevado tipo de inteligencia, las clases más grandes de animales invaden a veces los planetas. Pero al principio de esta era los mortales aprenden a encender y a mantener el fuego, y con el aumento de la imaginación inventiva y el mejoramiento de las herramientas, el hombre en evolución vence pronto a los animales más grandes y más pesados. Las razas primitivas también utilizan ampliamente los animales voladores más grandes. Estas aves enormes son capaces de llevar a uno o dos hombres de tamaño medio durante un vuelo sin escalas de más de ochocientos kilómetros. En algunos planetas estas aves son de gran utilidad puesto que poseen un elevado tipo de inteligencia, y a menudo son capaces de decir muchas palabras de los idiomas del reino. Estas aves son sumamente inteligentes, muy obedientes e increíblemente afectuosas. Estas aves de pasajeros se extinguieron hace mucho tiempo en Urantia, pero vuestros antepasados primitivos disfrutaron de sus servicios.

\par
%\textsuperscript{(590.2)}
\textsuperscript{52:1.6} La adquisición por parte del hombre del juicio ético, de la voluntad moral, coincide generalmente con la aparición del lenguaje primitivo. Tras alcanzar el nivel humano después de esta aparición de la voluntad mortal, estos seres se vuelven receptivos a la estancia temporal de los Ajustadores divinos, y después de morir, muchos de ellos son debidamente elegidos como supervivientes y confirmados por los arcángeles\footnote{\textit{Confirmación de los arcángeles para la resurrección}: Ap 7:2-3.} para ser resucitados ulteriormente y fusionados con el Espíritu. Los arcángeles acompañan siempre a los Príncipes Planetarios, y al mismo tiempo que llega el príncipe tiene lugar un juicio dispensacional del reino.

\par
%\textsuperscript{(590.3)}
\textsuperscript{52:1.7} Todos los mortales que están habitados por un Ajustador del Pensamiento son adoradores potenciales; han sido <<iluminados por la verdadera luz>>\footnote{\textit{Iluminados por la verdadera luz}: Job 2:8; Is 9:2; 49:6; Mt 4:16; Lc 1:79; 2:32; Jn 1:4-9; 8:12; 9:5; 12:35-36,46.}, y poseen la capacidad de buscar un contacto recíproco con la divinidad. Sin embargo, la religión inicial o biológica del hombre primitivo es principalmente una persistencia del miedo animal unido al temor ignorante y a la superstición tribal. La supervivencia de la superstición en las razas de Urantia no es del todo halagadora para vuestro desarrollo evolutivo, ni tampoco es compatible con vuestros logros, por otra parte espléndidos, en el campo del progreso material. Pero esta religión primitiva del miedo cumple un objetivo muy valioso subyugando los temperamentos fogosos de estas criaturas primitivas. Es la precursora de la civilización y el terreno donde el Príncipe Planetario y sus ministros plantarán posteriormente la semilla de la religión revelada.

\par
%\textsuperscript{(590.4)}
\textsuperscript{52:1.8} El Príncipe Planetario llega generalmente cerca de cien mil años después del momento en que el hombre adquiere la postura erguida; el Soberano del Sistema lo envía cuando los Portadores de Vida le informan de que la voluntad funciona, aunque relativamente pocos individuos se hayan desarrollado así. Los mortales primitivos reciben generalmente bien al Príncipe Planetario y a su estado mayor visible; de hecho, a menudo los miran con temor y reverencia y, si no se les refrena, casi con adoración.

\section*{2. El hombre posterior al Príncipe Planetario}
\par
%\textsuperscript{(591.1)}
\textsuperscript{52:2.1} Con la llegada del Príncipe Planetario empieza una nueva dispensación. El gobierno aparece en la Tierra y se alcanza la época de progreso de las tribus. Durante algunos miles de años de este régimen se llevan a cabo grandes progresos sociales. En condiciones normales, los mortales alcanzan un alto grado de civilización durante esta época. No luchan en la barbarie durante tanto tiempo como lo hicieron las razas de Urantia. Pero la vida en un mundo habitado está tan cambiada por la rebelión que sólo podéis tener una pequeña o ninguna idea de cómo es un régimen así en un planeta normal.

\par
%\textsuperscript{(591.2)}
\textsuperscript{52:2.2} La duración media de esta dispensación es de unos quinientos mil años, a veces más y a veces menos. Durante esta era, el planeta se establece en los circuitos del sistema, y un contingente completo de serafines y de otros ayudantes es asignado a su administración. Los Ajustadores del Pensamiento vienen en cantidades crecientes, y los guardianes seráficos amplían su régimen de supervisión de los mortales.

\par
%\textsuperscript{(591.3)}
\textsuperscript{52:2.3} Cuando el Príncipe Planetario llega a un mundo primitivo, la religión evolutiva del miedo y de la ignorancia es la que prevalece. El príncipe y su estado mayor efectúan las primeras revelaciones sobre la verdad superior y la organización del universo. Estas presentaciones iniciales de la religión revelada son muy sencillas y habitualmente se refieren a los asuntos del sistema local. Antes de la llegada del Príncipe Planetario, la religión es enteramente un proceso evolutivo. Posteriormente, la religión progresa mediante revelaciones graduales así como por medio del crecimiento evolutivo. Cada dispensación, cada época humana, recibe una presentación más amplia de la verdad espiritual y de la ética religiosa. La evolución de la capacidad para la receptividad religiosa en los habitantes de un mundo determina en gran parte la velocidad de sus progresos espirituales y el alcance de la revelación religiosa.

\par
%\textsuperscript{(591.4)}
\textsuperscript{52:2.4} Esta dispensación presencia un amanecer espiritual, y las diferentes razas y sus diversas tribus tienden a desarrollar unos sistemas especializados de pensamiento religioso y filosófico. Dos tendencias atraviesan uniformemente todas estas religiones raciales: los miedos iniciales de los hombres primitivos y las revelaciones posteriores del Príncipe Planetario. En algunos aspectos, los urantianos no parecen haber salido por completo de esta etapa de evolución planetaria. A medida que continuéis este estudio, discerniréis con más claridad cuánto se aleja vuestro mundo del camino medio del progreso y del desarrollo evolutivos.

\par
%\textsuperscript{(591.5)}
\textsuperscript{52:2.5} Pero el Príncipe Planetario no es <<el Príncipe de la Paz>>\footnote{\textit{Príncipe de la Paz}: Is 9:6.}. Las luchas raciales y las guerras tribales continúan durante esta dispensación, pero con una frecuencia y un rigor cada vez menor. Es la gran era de la dispersión racial, y culmina en un período de intenso nacionalismo. El color es la base de las agrupaciones tribales y nacionales, y las diferentes razas desarrollan a menudo sus idiomas independientes. Cada grupo de mortales en expansión tiende a buscar el aislamiento. La existencia de muchos idiomas favorece esta separación. Antes de que las diversas razas se unifiquen, sus guerras implacables conducen a veces a la desaparición de pueblos enteros; los hombres anaranjados y los verdes están particularmente expuestos a esta extinción.

\par
%\textsuperscript{(591.6)}
\textsuperscript{52:2.6} En los mundos de tipo medio, durante la última parte del gobierno del príncipe, la vida nacional empieza a reemplazar a la organización tribal, o más bien a superponerse a las agrupaciones tribales existentes. Pero el gran logro social de la época del príncipe es la aparición de la vida familiar. Hasta ese momento, las relaciones humanas han sido principalmente tribales; ahora empieza a materializarse el hogar.

\par
%\textsuperscript{(591.7)}
\textsuperscript{52:2.7} Ésta es la dispensación en la que se lleva a cabo la igualdad entre los sexos. En algunos planetas el hombre domina a la mujer; en otros prevalece lo contrario. Durante esta época, los mundos normales establecen la plena igualdad entre los sexos, siendo éste el paso preliminar para hacer más plenamente realidad los ideales de la vida de familia. Es el amanecer de la era de oro del hogar. La idea del gobierno tribal cede gradualmente el paso al doble concepto de la vida nacional y de la vida familiar.

\par
%\textsuperscript{(592.1)}
\textsuperscript{52:2.8} Durante esta época la agricultura hace su aparición. El crecimiento de la idea de la familia es incompatible con la vida errante e inestable del cazador. Las costumbres de las moradas fijas y del cultivo de la tierra se establecen gradualmente. La domesticación de los animales y el desarrollo de las artes hogareñas avanzan rápidamente. Cuando se llega a la cumbre de la evolución biológica, se ha alcanzado un alto nivel de civilización, pero hay poco desarrollo de tipo mecánico; la invención es la característica de la era siguiente.

\par
%\textsuperscript{(592.2)}
\textsuperscript{52:2.9} Antes del final de esta era, las razas se purifican y alcanzan un alto estado de perfección física y de fuerza intelectual. El plan destinado a promover el aumento de los tipos superiores de mortales, con una reducción proporcional de los tipos inferiores, ayuda enormemente al desarrollo inicial de un mundo normal. La incapacidad de vuestros pueblos primitivos para discriminar así entre estos tipos es lo que explica la presencia de tantos individuos deficientes y degenerados entre las razas actuales de Urantia.

\par
%\textsuperscript{(592.3)}
\textsuperscript{52:2.10} Uno de los grandes logros de la era del príncipe es esta restricción a la multiplicación de los individuos mentalmente deficientes y socialmente incapaces. Mucho antes de la época de la llegada de los segundos Hijos, los Adanes, la mayoría de los mundos se dedican seriamente a la tarea de purificar la raza, cosa que los pueblos de Urantia ni siquiera han emprendido seriamente todavía.

\par
%\textsuperscript{(592.4)}
\textsuperscript{52:2.11} Este problema de mejorar la raza no es una empresa de tanta envergadura cuando se ataca en esta fecha temprana de la evolución humana. El período anterior de las luchas tribales y de la dura competición por la supervivencia racial ha eliminado la mayor parte de los linajes anormales y defectuosos. Un idiota no tiene muchas posibilidades de sobrevivir en una organización social tribal primitiva y guerrera. El falso sentimentalismo de vuestras civilizaciones parcialmente perfeccionadas es el que fomenta, protege y perpetúa los linajes irremediablemente defectuosos de las razas humanas evolutivas.

\par
%\textsuperscript{(592.5)}
\textsuperscript{52:2.12} No es ni ternura ni altruismo ofrecer una compasión inútil a unos seres humanos degenerados, a unos mortales anormales e inferiores insalvables. Incluso en el más normal de los mundos evolutivos, existen diferencias suficientes entre los individuos y entre los numerosos grupos sociales como para asegurar el pleno ejercicio de todas aquellas nobles características de los sentimientos altruistas y del ministerio humano desinteresado, sin perpetuar los linajes socialmente incapaces y moralmente degenerados de la humanidad en evolución. Existen abundantes oportunidades para el ejercicio de la tolerancia y el funcionamiento del altruismo en favor de aquellos individuos desafortunados y necesitados que no han perdido irremediablemente su herencia moral ni han destruido para siempre su derecho espiritual de nacimiento.

\section*{3. El hombre postadámico}
\par
%\textsuperscript{(592.6)}
\textsuperscript{52:3.1} Cuando el ímpetu original de la vida evolutiva ha terminado su carrera biológica, cuando el hombre ha alcanzado la cumbre del desarrollo animal, llega la segunda orden de filiación y se inaugura la segunda dispensación de gracia y de ministerio. Esto es así en todos los mundos evolutivos. Cuando se ha alcanzado el nivel de vida evolutiva más elevado posible, cuando el hombre primitivo ha ascendido tan alto como le ha sido posible en la escala biológica, un Hijo y una Hija Materiales siempre aparecen en el planeta, enviados por el Soberano del Sistema.

\par
%\textsuperscript{(593.1)}
\textsuperscript{52:3.2} Los Ajustadores del Pensamiento se conceden de forma creciente a los hombres postadámicos, y un número en constante aumento de estos mortales alcanza la capacidad de fusionar posteriormente con el Ajustador. Aunque ejercen su actividad como Hijos descendentes, los Adanes no poseen Ajustadores, pero sus descendientes planetarios ---directos y mezclados--- se convierten en candidatos legítimos para recibir a su debido tiempo los Monitores de Misterio. Antes de terminarse la era postadámica, el planeta está en posesión de su contingente completo de ministros celestiales; sólo los Ajustadores destinados a la fusión no se confieren todavía de forma universal.

\par
%\textsuperscript{(593.2)}
\textsuperscript{52:3.3} El propósito principal del régimen adámico es influir sobre el hombre evolutivo para que termine de pasar desde la etapa de civilización de los cazadores y de los pastores a la de los agricultores y los horticultores, que más tarde será completada con la aparición de los complementos urbanos e industriales de la civilización. Diez mil años de esta dispensación de los mejoradores biológicos son suficientes para llevar a cabo una transformación maravillosa. Veinticinco mil años de una administración así dotada de la sabiduría conjunta del Príncipe Planetario y de los Hijos Materiales prepara generalmente a la esfera para la venida de un Hijo Magistral.

\par
%\textsuperscript{(593.3)}
\textsuperscript{52:3.4} Esta época presencia generalmente el final de la eliminación de los incapaces y la purificación adicional de los linajes raciales; en los mundos normales, las tendencias bestiales defectuosas se eliminan casi por completo de las estirpes reproductoras del reino.

\par
%\textsuperscript{(593.4)}
\textsuperscript{52:3.5} La progenie adámica no se amalgama nunca con los linajes inferiores de las razas evolutivas. El plan divino tampoco contempla que el Adán o la Eva Planetarios se emparejen personalmente con los pueblos evolutivos. Este proyecto de mejoramiento racial es tarea de su progenie. Pero los descendientes del Hijo y de la Hija Materiales son movilizados durante generaciones antes de que se inaugure el ministerio de la amalgamación racial.

\par
%\textsuperscript{(593.5)}
\textsuperscript{52:3.6} La donación del plasma vital adámico a las razas mortales tiene como resultado una elevación inmediata de la capacidad intelectual y una aceleración del progreso espiritual. También hay habitualmente cierto mejoramiento físico. En un mundo de tipo medio, la dispensación postadámica es una época de grandes invenciones, de control de la energía y de desarrollo mecánico. Es la era en que aparecen las manufacturas multiformes y el control de las fuerzas naturales; es la edad de oro de la exploración y del sometimiento final del planeta. Una gran parte del progreso material de un mundo tiene lugar durante este período en que comienza el desarrollo de las ciencias físicas, precisamente la época que Urantia está experimentando ahora. Vuestro mundo lleva un retraso de una dispensación o más con respecto al programa planetario medio.

\par
%\textsuperscript{(593.6)}
\textsuperscript{52:3.7} Hacia el final de la dispensación adámica en un planeta normal, las razas están prácticamente mezcladas, de manera que se puede proclamar en verdad que <<Dios ha hecho a todas las naciones de una sola sangre>>\footnote{\textit{Todas las naciones de una sola sangre}: Hch 17:26.}, y que su Hijo <<ha hecho a todos los pueblos de un solo color>>. El color de esta raza amalgamada es una especie de matiz aceitunado del tinte violeta, el <<blanco>> racial de las esferas.

\par
%\textsuperscript{(593.7)}
\textsuperscript{52:3.8} El hombre primitivo es principalmente carnívoro; los Hijos y las Hijas Materiales no comen carne, pero al cabo de algunas generaciones su progenie tiende generalmente hacia el nivel omnívoro, aunque a veces grupos enteros de sus descendientes siguen sin comer carne. Este doble origen de las razas postadámicas explica por qué estas estirpes humanas mezcladas muestran unos vestigios anatómicos que pertenecen tanto a los grupos animales herbívoros como a los carnívoros.

\par
%\textsuperscript{(593.8)}
\textsuperscript{52:3.9} Al cabo de diez mil años de amalgamación racial, las estirpes resultantes muestran diversos grados de mezcla anatómica; algunos linajes llevan más signos de sus ascendientes no comedores de carne, y otros manifiestan más rasgos distinguibles y más características físicas de sus progenitores evolutivos carnívoros. La mayoría de estas razas del mundo pronto se vuelven omnívoras, sustentándose con una amplia gama de alimentos procedentes tanto del reino animal como del reino vegetal.

\par
%\textsuperscript{(594.1)}
\textsuperscript{52:3.10} La época postadámica es la dispensación del internacionalismo. Con la tarea de la mezcla racial a punto de concluir, el nacionalismo disminuye y la fraternidad entre los hombres empieza realmente a materializarse. El gobierno representativo comienza a sustituir a la forma de reinado monárquico o paternalista. El sistema educativo se vuelve mundial y los idiomas de las razas ceden gradualmente el paso a la lengua del pueblo violeta. La paz y la cooperación universales raramente se alcanzan hasta que las razas no están bastante bien mezcladas y hasta que no hablan un idioma común.

\par
%\textsuperscript{(594.2)}
\textsuperscript{52:3.11} Durante los siglos finales de la era postadámica se desarrolla un nuevo interés por el arte, la música y la literatura, y este despertar mundial es la señal para que aparezca un Hijo Magistral. El desarrollo que corona esta era es el interés universal por las realidades intelectuales, por la verdadera filosofía. La religión se vuelve menos nacionalista, se convierte cada vez más en un asunto planetario. Estos tiempos están caracterizados por nuevas revelaciones de la verdad, y los Altísimos de las constelaciones empiezan a gobernar en los asuntos de los hombres. La verdad es revelada hasta englobar la administración de las constelaciones.

\par
%\textsuperscript{(594.3)}
\textsuperscript{52:3.12} Un gran progreso ético caracteriza a esta era; la fraternidad entre los hombres es la meta de su sociedad. La paz mundial ---el cese de los conflictos raciales y de las animosidades nacionales--- es la indicadora de que el planeta está maduro para la venida de la tercera orden de filiación, el Hijo Magistral.

\section*{4. El hombre posterior al Hijo Magistral}
\par
%\textsuperscript{(594.4)}
\textsuperscript{52:4.1} En los planetas normales y leales, esta época se abre con las razas mortales mezcladas y biológicamente sanas. No hay problemas de razas ni de color; todas las naciones y todas las razas son literalmente de una sola sangre. La fraternidad entre los hombres florece y las naciones aprenden a vivir en el mundo en paz y tranquilidad. Un mundo así se encuentra en vísperas de un gran desarrollo intelectual culminante.

\par
%\textsuperscript{(594.5)}
\textsuperscript{52:4.2} Cuando un mundo evolutivo está así de maduro para la era magistral, un miembro de la elevada orden de los Hijos Avonales hace su aparición en misión magistral. El Príncipe Planetario y los Hijos Materiales tienen su origen en el universo local; el Hijo Magistral procede del Paraíso.

\par
%\textsuperscript{(594.6)}
\textsuperscript{52:4.3} Cuando los Avonales del Paraíso vienen a las esferas mortales para llevar a cabo actos judiciales, únicamente como jueces de una dispensación, nunca están encarnados. Pero cuando vienen para realizar misiones magistrales, siempre están encarnados, al menos durante la misión inicial, aunque no experimentan el nacimiento ni tampoco mueren como los habitantes del reino. En aquellos casos en que permanecen como gobernantes de ciertos planetas, pueden seguir viviendo durante generaciones. Cuando sus misiones han terminado, abandonan su vida planetaria y regresan a su estado anterior de filiación divina.

\par
%\textsuperscript{(594.7)}
\textsuperscript{52:4.4} Cada nueva dispensación amplía el horizonte de la religión revelada, y los Hijos Magistrales extienden la revelación de la verdad hasta describir los asuntos del universo local y de todos sus tributarios.

\par
%\textsuperscript{(594.8)}
\textsuperscript{52:4.5} Después de la visita inicial de un Hijo Magistral, las razas efectúan pronto su liberación económica. El trabajo diario que necesita hacer una persona para mantener su independencia representaría dos horas y media de vuestro tiempo. No supone ningún riesgo liberar a estos mortales éticos e inteligentes. Estos pueblos refinados saben muy bien cómo utilizar el tiempo libre para el mejoramiento personal y el avance planetario. Esta época presencia la purificación adicional de los linajes raciales mediante la restricción de la reproducción entre los individuos menos capacitados y mal dotados.

\par
%\textsuperscript{(595.1)}
\textsuperscript{52:4.6} El gobierno político y la administración social de las razas continúan mejorando, y el gobierno autónomo está bastante bien establecido hacia el final de esta era. Cuando decimos gobierno autónomo nos referimos al tipo más elevado de gobierno representativo. Estos mundos sólo promocionan y honran a aquellos dirigentes y gobernantes que están más capacitados para llevar las responsabilidades sociales y políticas.

\par
%\textsuperscript{(595.2)}
\textsuperscript{52:4.7} Durante esta época, la mayoría de los mortales del mundo están habitados por Ajustadores. Pero incluso entonces, la concesión de los Monitores divinos no siempre es universal. Los Ajustadores destinados a la fusión aún no se conceden a todos los mortales planetarios; todavía es necesario que las criaturas volitivas escojan recibir a los Monitores de Misterio.

\par
%\textsuperscript{(595.3)}
\textsuperscript{52:4.8} Durante los tiempos finales de esta dispensación, la sociedad empieza a volver a formas de vida más simplificadas. La naturaleza compleja de una civilización en progreso sigue su curso, y los mortales aprenden a vivir de una manera más natural y eficaz. Esta tendencia se acrecienta en cada época siguiente. Es la era del florecimiento del arte, de la música y del saber superior. Las ciencias físicas ya han alcanzado la cumbre de su desarrollo. En un mundo ideal, el final de esta época presencia la plenitud de un gran despertar religioso, de una iluminación espiritual mundial. Este amplio despertar de la naturaleza espiritual de las razas es la señal para que llegue el Hijo donador y para que se inaugure la quinta época de los mortales.

\par
%\textsuperscript{(595.4)}
\textsuperscript{52:4.9} En muchos mundos sucede que el planeta no está preparado para recibir a un Hijo donador después de una sola misión magistral; en ese caso habrá un segundo e incluso una sucesión de Hijos Magistrales, cada uno de los cuales hará avanzar a las razas de una dispensación a otra hasta que el planeta esté preparado para el don del Hijo donador. En la segunda misión y en las siguientes, los Hijos Magistrales pueden o no estar encarnados. Pero cualquiera que sea el número de Hijos Magistrales que aparezcan ---y también pueden venir como tales después del Hijo donador--- la llegada de cada uno de ellos señala el final de una dispensación y el comienzo de otra.

\par
%\textsuperscript{(595.5)}
\textsuperscript{52:4.10} Estas dispensaciones de los Hijos Magistrales abarcan en todas partes entre veinticinco mil y cincuenta mil años del tiempo de Urantia. A veces una época de este tipo es mucho más corta, y en raros casos incluso más larga. Pero en la plenitud de los tiempos, uno de estos mismos Hijos Magistrales nacerá como Hijo Paradisiaco donador.

\section*{5. El hombre posterior al Hijo donador}
\par
%\textsuperscript{(595.6)}
\textsuperscript{52:5.1} Cuando se alcanza cierto nivel de desarrollo intelectual y espiritual en un mundo habitado, siempre llega un Hijo Paradisiaco donador. En los mundos normales no aparece encarnado hasta que las razas no han alcanzado los niveles más elevados de desarrollo intelectual y de logros éticos. Pero en Urantia el Hijo donador, exactamente vuestro propio Hijo Creador, apareció al final de la dispensación adámica, pero éste no es el orden habitual de los acontecimientos en los mundos del espacio.

\par
%\textsuperscript{(595.7)}
\textsuperscript{52:5.2} Cuando los mundos están maduros para la espiritualización, llega el Hijo donador. Estos Hijos siempre pertenecen a la orden Magistral o Avonal salvo en el caso, que se produce una sola vez en cada universo local, en que el Hijo Creador se prepara para su donación final en un mundo evolutivo, tal como sucedió cuando Miguel de Nebadon apareció en Urantia para donarse a vuestras razas mortales. Únicamente un mundo, entre cerca de diez millones, puede disfrutar de un don así; todos los otros mundos avanzan espiritualmente gracias a la donación de un Hijo Paradisiaco de la orden Avonal.

\par
%\textsuperscript{(596.1)}
\textsuperscript{52:5.3} El Hijo donador llega a un mundo que posee una elevada cultura educativa y encuentra a una raza espiritualmente instruida y preparada para asimilar unas enseñanzas avanzadas y para apreciar la misión donadora. Es una época caracterizada por la búsqueda mundial de la cultura moral y de la verdad espiritual. La pasión de los mortales de esta dispensación es penetrar la realidad cósmica y comulgar con la realidad espiritual. Las revelaciones de la verdad se amplían hasta incluir al superuniverso. Se establecen sistemas de educación y de gobierno enteramente nuevos para sustituir a los regímenes rudimentarios de los tiempos anteriores. La alegría de vivir adquiere un nuevo color, y las reacciones de la vida se elevan hasta unas alturas de tono y de timbre celestiales.

\par
%\textsuperscript{(596.2)}
\textsuperscript{52:5.4} El Hijo donador vive y muere para elevar espiritualmente a las razas mortales de un mundo. Establece el <<nuevo camino viviente>>\footnote{\textit{Nuevo camino viviente}: Jn 14:6; Heb 10:20.}; su vida es una encarnación de la verdad del Paraíso en la carne mortal, de esa misma verdad ---el Espíritu mismo de la Verdad--- cuyo conocimiento hará libres a los hombres.

\par
%\textsuperscript{(596.3)}
\textsuperscript{52:5.5} En Urantia, el establecimiento de este <<nuevo camino viviente>>\footnote{\textit{Nuevo camino viviente}: Jn 14:6; Heb 10:20.} fue una cuestión de hecho así como de verdad. El aislamiento de Urantia debido a la rebelión de Lucifer había suspendido el procedimiento gracias al cual los mortales pueden pasar directamente, después de morir, a las orillas de los mundos de las mansiones. Antes de la época de Cristo Miguel en Urantia, todas las almas continuaban durmiendo hasta las resurrecciones dispensacionales o las milenarias especiales. Incluso a Moisés\footnote{\textit{Resurrección de Moisés}: Jud 1:9; AsMo all.} no se le permitió pasar al otro lado hasta el momento de una resurrección especial, pues Caligastia, el Príncipe Planetario caído, impugnaba esta liberación. Pero desde el día de Pentecostés, los mortales de Urantia pueden dirigirse de nuevo directamente a las esferas morontiales.

\par
%\textsuperscript{(596.4)}
\textsuperscript{52:5.6} Cuando se produce la resurrección de un Hijo donador, al tercer día después de abandonar su vida encarnada, asciende a la derecha del Padre Universal, recibe la seguridad de que su misión donadora es aceptada, y regresa hacia el Hijo Creador en la sede del universo local. Inmediatamente después, el Avonal donador y el Miguel Creador envían su espíritu conjunto, el Espíritu de la Verdad, al mundo de la donación. Es el momento en que <<el espíritu del Hijo triunfante es derramado sobre toda carne>>\footnote{\textit{Espíritu derramado sobre toda carne}: Job 4:12-15; Ez 11:19; 18:31; 36:26-27; Jl 2:28-29; Lc 24:49; Jn 7:39; 14:16-18,23,26; 15:4,26; 16:7-8,13-14; 17:21-23; Hch 1:8a; 2:1-4,16-18; 2:33; 2 Co 13:5; Gl 2:20; 4:6; Ef 1:13; 4:30.}. El Espíritu Madre del Universo también participa en esta donación del Espíritu de la Verdad y, concomitante con ello, se promulga el edicto para la concesión de los Ajustadores del Pensamiento. Después de esto, todas las criaturas volitivas con una mente normal de ese mundo recibirán un Ajustador en cuanto lleguen a la edad de la responsabilidad moral, de la elección espiritual.

\par
%\textsuperscript{(596.5)}
\textsuperscript{52:5.7} Si ese Avonal donador tuviera que regresar al mundo después de su misión de donación, no se encarnaría, sino que vendría <<cubierto de gloria con las huestes seráficas>>\footnote{\textit{Cubierto de gloria con los ángeles}: Mt 16:27; 24:30; 25:31; Mc 8:38; 13:26-27; Lc 9:26; 21:27.}.

\par
%\textsuperscript{(596.6)}
\textsuperscript{52:5.8} La era posterior al Hijo donador puede durar entre diez mil y cien mil años. No se asigna ningún tiempo arbitrario a ninguna de estas eras dispensacionales. Es un período de gran progreso ético y espiritual. Bajo la influencia espiritual de estas épocas, el carácter humano sufre unas transformaciones enormes y experimenta un desarrollo espectacular. Resulta posible poner en práctica la regla de oro. Las enseñanzas de Jesús son realmente aplicables en un mundo de mortales que han tenido la formación preliminar de los Hijos anteriores a la donación, con sus dispensaciones para ennoblecer el carácter y aumentar la cultura.

\par
%\textsuperscript{(596.7)}
\textsuperscript{52:5.9} Durante esta era se han resuelto prácticamente los problemas de las enfermedades y de la delincuencia. La reproducción selectiva ya ha eliminado ampliamente la degeneración. La enfermedad ha sido prácticamente vencida gracias a las cualidades extremadamente resistentes de los linajes adámicos y a la inteligente aplicación mundial de los descubrimientos de las ciencias físicas de las épocas precedentes. La duración media de la vida durante este período asciende muy por encima del equivalente de trescientos años del tiempo de Urantia.

\par
%\textsuperscript{(597.1)}
\textsuperscript{52:5.10} La supervisión gubernamental disminuye gradualmente a lo largo de esta época. El verdadero gobierno autónomo empieza a funcionar; cada vez se necesitan menos leyes restrictivas. Las ramas militares de la resistencia nacional van desapareciendo; la era de la armonía internacional está llegando realmente. Hay muchas naciones, determinadas principalmente por la distribución de las tierras, pero sólo hay una raza, un idioma y una religión. Los asuntos de los mortales casi se acercan a la utopía, aunque no del todo. ¡Es en verdad una era grande y gloriosa!

\section*{6. La era posterior a la donación en Urantia}
\par
%\textsuperscript{(597.2)}
\textsuperscript{52:6.1} El Hijo donador es el Príncipe de la Paz. Llega con el mensaje <<paz en la Tierra y buena voluntad entre los hombres>>\footnote{\textit{Paz en la Tierra y buena voluntad}: Lc 2:14.}. En los mundos normales, ésta es una dispensación de paz mundial; las naciones ya no aprenden a hacer la guerra. Pero estas influencias saludables no acompañaron la llegada de Cristo Miguel, vuestro Hijo donador. Urantia no camina según el orden normal. Vuestro mundo no sigue el paso de la procesión planetaria. Cuando vuestro Maestro estaba en la Tierra, advirtió a sus discípulos que su venida no traería el reino habitual de paz a Urantia\footnote{\textit{Príncipe de la Paz}: Is 9:6.}. Les dijo claramente que habría <<guerras y rumores de guerras>>\footnote{\textit{Guerras y rumores de guerras}: Mt 24:6-7; Mc 13:7-8; Lc 21:9-10.}, y que las naciones se levantarían contra las naciones. En otro momento dijo: <<No penséis que he venido a traer la paz a la Tierra>>\footnote{\textit{No penséis que he venido a traer la paz}: Mt 10:34; Lc 12:51.}.

\par
%\textsuperscript{(597.3)}
\textsuperscript{52:6.2} Incluso en los mundos evolutivos normales, la realización de la fraternidad mundial de los hombres no es una tarea fácil. En un planeta confuso y desordenado como Urantia, esta realización requiere un tiempo mucho más largo y necesita un esfuerzo mucho más grande. Una evolución social sin ayuda difícilmente puede conseguir estos felices resultados en una esfera espiritualmente aislada. La revelación religiosa es esencial para llevar a cabo la fraternidad en Urantia. Aunque Jesús ha mostrado el camino para alcanzar inmediatamente la fraternidad espiritual, la realización de la fraternidad social en vuestro mundo depende mucho de que se lleven a cabo las transformaciones personales y los ajustes planetarios siguientes:

\par
%\textsuperscript{(597.4)}
\textsuperscript{52:6.3} 1. \textit{La fraternidad social}. La multiplicación de los contactos sociales internacionales e interraciales, y de las asociaciones fraternales, a través de los viajes, el comercio y los juegos competitivos. El desarrollo de un idioma común y la multiplicación de los multiling\"uistas. El intercambio racial y nacional de estudiantes, profesores, industriales y filósofos religiosos.

\par
%\textsuperscript{(597.5)}
\textsuperscript{52:6.4} 2. \textit{La fecundación intelectual cruzada}. La fraternidad es imposible en un mundo cuyos habitantes son tan primitivos que no logran reconocer la locura del egoísmo sin freno. Debe producirse un intercambio de literatura nacional y racial. Cada raza debe familiarizarse con el pensamiento de todas las razas; cada nación debe conocer los sentimientos de todas las naciones. La ignorancia engendra la desconfianza, y la desconfianza es incompatible con la actitud esencial de simpatía y de amor.

\par
%\textsuperscript{(597.6)}
\textsuperscript{52:6.5} 3. \textit{El despertar ético}. Sólo una conciencia ética puede desenmascarar la inmoralidad de la intolerancia humana y lo pecaminoso de las luchas fratricidas. Sólo una conciencia moral puede condenar los males de la envidia nacional y de los celos raciales. Sólo unos seres morales buscarán siempre esa perspicacia espiritual que es esencial para vivir la regla de oro.

\par
%\textsuperscript{(598.1)}
\textsuperscript{52:6.6} 4. \textit{La sabiduría política}. La madurez emocional es esencial para el dominio de sí mismo. Sólo la madurez emocional puede asegurar que las técnicas internacionales del juicio civilizado sustituirán al arbitraje bárbaro de la guerra. Los estadistas sabios trabajarán algún día por el bienestar de la humanidad aunque sigan esforzándose por promover el interés de sus grupos nacionales o raciales. La sagacidad política egoísta es finalmente suicida ---perjudicial para todas aquellas cualidades duraderas que aseguran la supervivencia colectiva planetaria.

\par
%\textsuperscript{(598.2)}
\textsuperscript{52:6.7} 5. \textit{La perspicacia espiritual}. La fraternidad de los hombres está basada, después de todo, en el reconocimiento de la paternidad de Dios. La manera más rápida de llevar a cabo la fraternidad de los hombres en Urantia consiste en efectuar la transformación espiritual de la humanidad actual. La única técnica para acelerar la tendencia natural de la evolución social es la de aplicar una presión espiritual desde arriba, acrecentando así la perspicacia moral y elevando al mismo tiempo la capacidad del alma de cada mortal para comprender y amar a todos los demás mortales. La comprensión mutua y el amor fraternal son unos civilizadores trascendentes y unos factores poderosos en la realización mundial de la fraternidad de los hombres.

\par
%\textsuperscript{(598.3)}
\textsuperscript{52:6.8} Si pudierais ser transportados desde vuestro mundo atrasado y confuso hasta un planeta normal que se encuentre ahora en la era posterior al Hijo donador, pensaríais que habéis sido trasladados al cielo de vuestras tradiciones. Difícilmente podríais creer que estabais observando las actividades evolutivas normales de una esfera terrestre habitada por seres humanos. Estos mundos están incluídos en los circuitos espirituales de su reino, y disfrutan de todas las ventajas de las transmisiones universales y de los servicios de la reflectividad del superuniverso.

\section*{7. El hombre posterior a los Hijos Instructores}
\par
%\textsuperscript{(598.4)}
\textsuperscript{52:7.1} La siguiente orden de Hijos que llega a un mundo evolutivo medio es la de los Hijos Instructores Trinitarios, los Hijos Divinos de la Trinidad del Paraíso. Encontramos una vez más que Urantia no lleva el paso de sus esferas hermanas, en el sentido de que vuestro Jesús prometió regresar. Cumplirá ciertamente su promesa, pero nadie sabe si su segunda venida precederá o seguirá a la aparición del Hijo Magistral o de los Hijos Instructores en Urantia.

\par
%\textsuperscript{(598.5)}
\textsuperscript{52:7.2} Los Hijos Instructores vienen en grupo a los mundos que se espiritualizan. Un Hijo Instructor planetario recibe la ayuda y el apoyo de setenta Hijos primarios, doce Hijos secundarios y tres miembros de los más elevados y experimentados de la orden suprema de los Daynales. Este cuerpo permanece durante algún tiempo en el mundo, el suficiente para efectuar la transición entre las épocas evolutivas y la era de luz y de vida ---no menos de mil años del tiempo planetario y a menudo mucho más. Esta misión es una contribución de la Trinidad a los esfuerzos anteriores de todas las personalidades divinas que han aportado su ministerio a un mundo habitado.

\par
%\textsuperscript{(598.6)}
\textsuperscript{52:7.3} La revelación de la verdad se amplía ahora hasta el universo central y el Paraíso. Las razas se vuelven sumamente espirituales. Un gran pueblo ha evolucionado y se acerca una gran época. Los sistemas educativos, económicos y administrativos del planeta sufren unas transformaciones radicales. Se establecen nuevos valores y nuevas relaciones. El reino de los cielos aparece en el planeta, y la gloria de Dios se derrama por el mundo.

\par
%\textsuperscript{(598.7)}
\textsuperscript{52:7.4} Ésta es la dispensación durante la cual muchos mortales son trasladados de entre los vivos. A medida que progresa la era de los Hijos Instructores Trinitarios, la lealtad espiritual de los mortales del tiempo se hace cada vez más universal. La muerte natural se vuelve menos frecuente a medida que los Ajustadores fusionan de manera creciente con sus sujetos durante la vida en la carne. El planeta es clasificado finalmente dentro de la orden primaria modificada de ascensión de los mortales.

\par
%\textsuperscript{(599.1)}
\textsuperscript{52:7.5} La vida durante esta era es agradable y provechosa. La degeneración y los productos antisociales finales de la larga lucha evolutiva han sido prácticamente eliminados. La duración de la vida se acerca a los quinientos años de Urantia, y el índice reproductor del incremento racial está controlado de forma inteligente. Un tipo de sociedad enteramente nuevo ha llegado. Existen todavía grandes diferencias entre los mortales, pero el estado de la sociedad se acerca mucho más a los ideales de la fraternidad social y de la igualdad espiritual. El gobierno representativo está en vías de desaparecer y el mundo pasa a regirse por la regla del autocontrol individual. La función del gobierno se dirige principalmente a las tareas colectivas de la administración social y de la coordinación económica. La edad de oro llega con rapidez; la meta temporal de la larga e intensa lucha evolutiva planetaria está a la vista. La recompensa de los siglos pronto se hará realidad; la sabiduría de los Dioses está a punto de manifestarse.

\par
%\textsuperscript{(599.2)}
\textsuperscript{52:7.6} Durante esta época, la administración física de un mundo necesita alrededor de una hora diaria del tiempo de cada individuo adulto, es decir, el equivalente de una hora de Urantia. El planeta está en estrecho contacto con los asuntos del universo, y su gente escudriña las últimas transmisiones con el mismo vivo interés que vosotros mostráis ahora por las últimas ediciones de vuestros periódicos diarios. Estas razas se ocupan de mil cosas interesantes desconocidas en vuestro mundo.

\par
%\textsuperscript{(599.3)}
\textsuperscript{52:7.7} La verdadera lealtad planetaria hacia el Ser Supremo crece cada vez más. Generación tras generación, un número creciente de miembros de la raza sigue la conducta de aquellos que practican la justicia y viven la misericordia. El mundo va siendo ganado, lentamente pero con seguridad, para el servicio gozoso de los Hijos de Dios. Las dificultades físicas y los problemas materiales han sido resueltos en su mayoría; el planeta madura para una vida avanzada y una existencia más estable.

\par
%\textsuperscript{(599.4)}
\textsuperscript{52:7.8} A lo largo de su dispensación, los Hijos Instructores continúan llegando de vez en cuando a estos mundos pacíficos. No se marchan de un mundo hasta que no observan que el plan evolutivo que concierne a ese planeta funciona sin problemas. Un Hijo Magistral encargado de juzgar acompaña habitualmente a los Hijos Instructores en sus misiones sucesivas, mientras que otro Hijo de este tipo actúa cuando se marchan, y estos actos judiciales continúan de era en era mientras dura el régimen mortal del tiempo y del espacio.

\par
%\textsuperscript{(599.5)}
\textsuperscript{52:7.9} Cada misión periódica de los Hijos Instructores Trinitarios eleva sucesivamente a ese mundo excelso a unas alturas crecientes de sabiduría, de espiritualidad y de iluminación cósmica. Pero los nobles nativos de una esfera así siguen siendo finitos y mortales. Nada es perfecto; sin embargo, se va desarrollando una cualidad de casi perfección en el funcionamiento de un mundo imperfecto y en la vida de sus habitantes humanos.

\par
%\textsuperscript{(599.6)}
\textsuperscript{52:7.10} Los Hijos Instructores Trinitarios pueden volver muchas veces al mismo mundo. Pero tarde o temprano, en conexión con la finalización de una de sus misiones, el Príncipe Planetario es elevado a la posición de Soberano Planetario, y el Soberano del Sistema aparece para proclamar la entrada de ese mundo en la era de la luz y la vida.

\par
%\textsuperscript{(599.7)}
\textsuperscript{52:7.11} Juan escribió acerca de la terminación de la misión final de los Hijos Instructores (al menos ésta sería la cronología en un mundo normal): <<Y vi un nuevo cielo y una nueva Tierra, y la nueva Jerusalén que bajaba de Dios saliendo del cielo, preparada como una princesa adornada para su príncipe>>\footnote{\textit{Nuevo cielo y nueva tierra}: Is 65:17; 66:22; 2 P 3:13; Ap 21:1-2.}.

\par
%\textsuperscript{(600.1)}
\textsuperscript{52:7.12} Ésta es la misma Tierra renovada, el avanzado estado planetario, que el antiguo vidente imaginó cuando escribió: <<Porque igual que los nuevos cielos y la nueva Tierra que yo crearé perdurarán ante mí, así sobreviviréis vosotros y vuestros hijos; y sucederá que, desde una Luna nueva hasta la otra y desde un sábado hasta el otro, todo el género humano vendrá a postrarse en adoración ante mí', dice el Señor>>\footnote{\textit{Toda la carne adorará a Dios}: Is 66:22-23.}.

\par
%\textsuperscript{(600.2)}
\textsuperscript{52:7.13} Los mortales de esta era son los que están descritos como <<una generación elegida, un sacerdocio real, una nación santa, un pueblo elevado; y vosotros daréis a conocer las alabanzas de Aquél que os ha hecho salir de las tinieblas hacia esta maravillosa luz>>\footnote{\textit{Una generación elegida}: 1 P 2:9.}.

\par
%\textsuperscript{(600.3)}
\textsuperscript{52:7.14} Cualquiera que sea la historia natural especial de un planeta individual, indiferentemente de que el reino haya sido totalmente leal, haya estado contaminado por el mal o maldito por el pecado ---cualquiera que sean los antecedentes--- tarde o temprano la gracia de Dios y el ministerio de los ángeles anunciarán el día de la venida de los Hijos Instructores Trinitarios; y su partida, después de su misión final, inaugurará esta magnífica era de luz y de vida.

\par
%\textsuperscript{(600.4)}
\textsuperscript{52:7.15} Todos los mundos de Satania pueden unirse a la esperanza de aquél que escribió: <<Sin embargo, de acuerdo con Su promesa, nosotros esperamos un nuevo cielo y una nueva Tierra, donde reside la rectitud. Por lo cual, bienamados, en vista de que esperáis estas cosas, sed diligentes para que Él pueda encontraros en paz, sin mancha e irreprochables>>\footnote{\textit{Nueva tierra de rectitud}: 2 P 3:13-14.}.

\par
%\textsuperscript{(600.5)}
\textsuperscript{52:7.16} La partida del cuerpo de los Hijos Instructores al final de su primer reinado o de alguno posterior, anuncia los albores de la era de luz y de vida ---el umbral de la transición entre el tiempo y el vestíbulo de la eternidad. La realización planetaria de esta era de luz y de vida está mucho más allá de las expectativas más acariciadas por los mortales de Urantia, los cuales no han albergado otros conceptos clarividentes sobre la vida futura que aquellos incluídos en las creencias religiosas que describen el cielo como el destino inmediato y la morada final de los mortales sobrevivientes.

\par
%\textsuperscript{(600.6)}
\textsuperscript{52:7.17} [Patrocinado por un Mensajero Poderoso vinculado temporalmente al estado mayor de Gabriel.]


\chapter{Documento 53. La rebelión de Lucifer}
\par
%\textsuperscript{(601.1)}
\textsuperscript{53:0.1} LUCIFER era un brillante Hijo Lanonandek primario de Nebadon. Tenía la experiencia de haber servido en muchos sistemas, había sido un alto consejero de su grupo, y se había distinguido por su sabiduría, sagacidad y eficacia. Lucifer era el número 37 de su orden, y cuando fue nombrado por los Melquisedeks, fue designado como una de las cien personalidades más capaces y brillantes entre más de setecientas mil de su misma clase. Partiendo de unos comienzos tan magníficos, a través del mal y del error, abrazó el pecado y ahora figura como uno de los tres Soberanos Sistémicos de Nebadon que sucumbieron al impulso del yo y se entregaron a los sofismas de la falsa libertad personal ---rechazo a la lealtad universal y desprecio a las obligaciones fraternales, ceguera hacia las relaciones cósmicas.

\par
%\textsuperscript{(601.2)}
\textsuperscript{53:0.2} En el universo de Nebadon, dominio de Cristo Miguel, hay diez mil sistemas de mundos habitados. En toda la historia de los Hijos Lanonandeks, en todo su trabajo a lo largo de estos miles de sistemas y en la sede del universo, únicamente tres Soberanos Sistémicos han cometido desacato al gobierno del Hijo Creador.

\section*{1. Los jefes de la rebelión}
\par
%\textsuperscript{(601.3)}
\textsuperscript{53:1.1} Lucifer no era un ser ascendente; era un Hijo creado del universo local, y de él se había dicho: <<Eras perfecto en todos tus caminos desde el día en que fuiste creado hasta que la injusticia se halló en ti>>\footnote{\textit{Eras perfecto hasta que ...}: Ez 28:15.}. Se había reunido en consejo muchas veces con los Altísimos de Edentia. Y Lucifer reinaba <<sobre la montaña sagrada de Dios>>\footnote{\textit{Montaña sagrada de Dios}: Is 11:9; Ez 28:14.}, el monte administrativo de Jerusem, porque era el jefe ejecutivo de un gran sistema de 607 mundos habitados.

\par
%\textsuperscript{(601.4)}
\textsuperscript{53:1.2} Lucifer era un ser magnífico, una personalidad brillante; después de los Padres Altísimos de las constelaciones, era el siguiente en la línea directa de la autoridad universal. A pesar de la transgresión de Lucifer, las inteligencias subordinadas se abstuvieron de mostrarle falta de respeto y desdén antes de la donación de Miguel en Urantia. Incluso el arcángel de Miguel, en la época de la resurrección de Moisés, <<no emitió un juicio acusador contra él, sino que simplemente dijo: `el Juez te reprenda'>>\footnote{\textit{El juez te reprenda}: Jud 1:9; AsMo all.}. El juicio de estos asuntos pertenece a los Ancianos de los Días, los gobernantes del superuniverso.

\par
%\textsuperscript{(601.5)}
\textsuperscript{53:1.3} Lucifer es ahora el Soberano caído y depuesto de Satania. La contemplación de sí mismo es sumamente desastrosa, incluso para las altas personalidades del mundo celestial. De Lucifer se dijo: <<Tu corazón se elevó a causa de tu belleza; corrompiste tu sabiduría a causa de tu resplandor>>\footnote{\textit{Lucifer enorgullecido de su belleza}: Ez 28:17.}. Vuestro antiguo profeta vio su triste estado cuando escribió: <<¡Cómo has caído del cielo, oh Lucifer, hijo de la mañana! ¡Cómo has sido derribado, tú que te atreviste a confundir a los mundos!>>\footnote{\textit{Cómo has caído del cielo}: Is 14:12.}.

\par
%\textsuperscript{(602.1)}
\textsuperscript{53:1.4} En Urantia se había oído hablar muy poco de Lucifer debido al hecho de que nombró a Satanás, su primer lugarteniente, para que defendiera su causa en vuestro planeta. Satanás\footnote{\textit{Satanás}: Job 1:6-12; 2:1-7; Zac 3:1-2; Mt 4:10; Lc 10:18.} era miembro del mismo grupo primario de Lanonandeks, pero nunca había ejercido la función de Soberano Sistémico; entró de lleno en la insurrección de Lucifer. El <<demonio>>\footnote{\textit{El demonio}: Mt 4:1-11; 13:39; 25:41; Lc 4:2-6,13; 8:12,29; Hch 10:38; Jud 1:9; Ap 12:9; 20:2.} no es otro que Caligastia, el Príncipe Planetario depuesto\footnote{\textit{El demonio, antiguo Príncipe de este mundo}: Jn 12:31; 14:30; 16:11; Ef 2:2; 6:12.} de Urantia e Hijo de la orden secundaria de los Lanonandeks. En la época en que Miguel estaba encarnado en Urantia, Lucifer, Satanás y Caligastia se unieron para hacer abortar su misión de donación. Pero fracasaron rotundamente.

\par
%\textsuperscript{(602.2)}
\textsuperscript{53:1.5} Abaddon\footnote{\textit{Abaddon, jefe de los rebeldes}: Ap 9:11.} era el jefe del estado mayor de Caligastia. Siguió a su señor en la rebelión y desde entonces ha actuado como jefe ejecutivo de los rebeldes de Urantia. Belcebú\footnote{\textit{Belcebú: ser intermedio rebelde}: Mt 12:24,27; Mc 3:22; Lc 11:15-19.} era el cabecilla de las criaturas intermedias desleales que se aliaron con las fuerzas del traidor Caligastia.

\par
%\textsuperscript{(602.3)}
\textsuperscript{53:1.6} El dragón se convirtió finalmente en la representación simbólica de todos estos malvados personajes. Después del triunfo de Miguel, <<Gabriel descendió de Salvington y ató al dragón (a todos los jefes rebeldes) durante una era>>\footnote{\textit{Gabriel ató a los jefes rebeldes}: Ap 20:1-2.}. De los rebeldes seráficos de Jerusem se ha escrito: <<Y a los ángeles que no conservaron su estado primero, sino que abandonaron su propia morada, los ha reservado en las cadenas seguras de las tinieblas hasta el juicio del gran día>>\footnote{\textit{Ángeles que abandonaron su estado inicial}: Jud 1:6.}.

\section*{2. Las causas de la rebelión}
\par
%\textsuperscript{(602.4)}
\textsuperscript{53:2.1} Lucifer y su primer ayudante, Satanás, habían reinado en Jerusem durante más de quinientos mil años cuando empezaron a alinearse en su corazón contra el Padre Universal y su Hijo Miguel, vicegerente por aquel entonces.

\par
%\textsuperscript{(602.5)}
\textsuperscript{53:2.2} En el sistema de Satania no existían condiciones particulares o especiales que sugirieran o favorecieran una rebelión. Creemos que la idea tuvo su origen y tomó forma en la mente de Lucifer, y que pudo haber instigado una rebelión así en cualquier lugar donde hubiera estado situado. Lucifer anunció sus planes primero a Satanás, pero fueron necesarios varios meses para corromper la mente de su brillante e inteligente asociado. Sin embargo, una vez convertido a las teorías rebeldes, se volvió un defensor intrépido y entusiasta de la <<reafirmación personal y de la libertad>>\footnote{\textit{Reafirmación personal y de la libertad}: Is 14:12-14.}.

\par
%\textsuperscript{(602.6)}
\textsuperscript{53:2.3} Nadie le sugirió nunca a Lucifer que se rebelara. La idea de la reafirmación personal, en oposición a la voluntad de Miguel y a los planes del Padre Universal, tal como éstos están representados por Miguel, tuvo su origen en su propia mente. Sus relaciones con el Hijo Creador habían sido íntimas y siempre cordiales. En ningún momento anterior a la exaltación de su propia mente, Lucifer había expresado abiertamente su insatisfacción acerca de la administración del universo. A pesar de su silencio, y durante más de cien años del tiempo oficial, el Unión de los Días de Salvington había estado indicando por reflectividad a Uversa que no todo estaba en paz en la mente de Lucifer. Esta información también fue comunicada al Hijo Creador y a los Padres de la Constelación de Norlatiadek.

\par
%\textsuperscript{(602.7)}
\textsuperscript{53:2.4} Durante todo este período, Lucifer se puso a criticar cada vez más todo el plan de la administración universal, pero siempre expresó una lealtad sincera hacia los Gobernantes Supremos. Su primera deslealtad abierta la manifestó en el momento de una visita de Gabriel a Jerusem, justo pocos días antes de proclamar abiertamente su Declaración Luciferina de Libertad. Gabriel quedó tan profundamente impresionado con la certeza de una sublevación inminente, que se dirigió directamente a Edentia para consultar con los Padres de la Constelación acerca de las medidas a emplear en el caso de una rebelión abierta.

\par
%\textsuperscript{(603.1)}
\textsuperscript{53:2.5} Es muy difícil indicar la causa o las causas exactas que culminaron finalmente en la rebelión de Lucifer. Sólo estamos seguros de una cosa, y es que cualesquiera que fueran esos primeros comienzos, tuvieron su origen en la mente de Lucifer. Debe haber existido un orgullo del yo que se alimentó hasta el punto de engañarse a sí mismo, de tal manera que Lucifer se persuadió realmente durante algún tiempo de que su proyecto de rebelión era verdaderamente por el bien del sistema, si no del universo. Cuando sus planes se hubieron desarrollado hasta el punto de desilusionarlo\footnote{\textit{Orgullo y desilusión}: Ez 28:17.}, no hay duda de que había ido demasiado lejos como para que su orgullo original y dañino le permitiera detenerse. En algún momento de esta experiencia dejó de ser sincero, y el mal se transformó en pecado deliberado y voluntario. Que esto sucedió así está demostrado por la conducta posterior de este brillante ejecutivo. Durante mucho tiempo se le ofreció la oportunidad de arrepentirse, pero sólo algunos de sus subordinados aceptaron la misericordia ofrecida. A petición de los Padres de la Constelación, el Fiel de los Días de Edentia presentó en persona el plan de Miguel para salvar a estos rebeldes flagrantes, pero la misericordia del Hijo Creador siempre fue rechazada, y rechazada con un desprecio y un desdén cada vez mayores.

\section*{3. El manifiesto de Lucifer}
\par
%\textsuperscript{(603.2)}
\textsuperscript{53:3.1} Cualesquiera que hubieran sido los orígenes iniciales del problema en el corazón de Lucifer y de Satanás, la sublevación final tomó la forma de la Declaración Luciferina de Libertad\footnote{\textit{Manifiesto de Lucifer}: Is 14:12-14; Jud 1:4.}. La causa de los rebeldes fue dada a conocer en tres puntos:

\par
%\textsuperscript{(603.3)}
\textsuperscript{53:3.2} 1. \textit{La realidad del Padre Universal}. Lucifer denunció que el Padre Universal no existía realmente, que la gravedad física y la energía espacial eran inherentes al universo, y que el Padre era un mito inventado por los Hijos Paradisiacos para permitirles conservar el gobierno de los universos en nombre del Padre. Negó que la personalidad fuera un don del Padre Universal. Insinuó incluso que los finalitarios estaban de connivencia con los Hijos Paradisiacos para imponer el fraude a toda la creación, puesto que nunca traían una idea muy clara sobre la personalidad real del Padre tal como ésta se puede discernir en el Paraíso. Utilizó la veneración a su favor, calificándola de ignorancia. La acusación era violenta, terrible y blasfema. No hay duda de que este ataque velado contra los finalitarios fue el que influyó sobre los ciudadanos ascendentes que estaban entonces en Jerusem para que se mantuvieran firmes y permanecieran inquebrantables en su resistencia a todas las propuestas de los rebeldes.

\par
%\textsuperscript{(603.4)}
\textsuperscript{53:3.3} 2. \textit{El gobierno universal de Miguel, el Hijo Creador}. Lucifer afirmó que los sistemas locales debían ser autónomos. Protestó contra el derecho de Miguel, el Hijo Creador, a asumir la soberanía de Nebadon en nombre de un Padre Paradisiaco hipotético, y a exigirle a todas las personalidades que reconocieran su lealtad hacia este Padre invisible. Afirmó que todo el plan de la adoración era una ingeniosa estratagema para engrandecer a los Hijos Paradisiacos. Estaba dispuesto a reconocer a Miguel como su padre Creador, pero no como su Dios y su soberano legítimo.

\par
%\textsuperscript{(603.5)}
\textsuperscript{53:3.4} Atacó de la manera más encarnizada el derecho de los Ancianos de los Días ---<<potentados extranjeros>>--- a interferir en los asuntos de los sistemas y de los universos locales. Denunció a estos gobernantes como tiranos y usurpadores. Exhortó a sus seguidores a que creyeran que ninguno de estos gobernantes podía hacer nada por interferir en el funcionamiento de una autonomía completa si los hombres y los ángeles tan sólo tuvieran la valentía de afirmarse y de reclamar audazmente sus derechos.

\par
%\textsuperscript{(603.6)}
\textsuperscript{53:3.5} Afirmó que a los ejecutores de los Ancianos de los Días se les podía impedir que actuaran en los sistemas locales si los seres nativos se atrevían a afirmar su independencia. Sostuvo que la inmortalidad era inherente a las personalidades del sistema, que la resurrección era natural y automática, y que todos los seres vivirían eternamente si no fuera por los actos arbitrarios e injustos de los ejecutores de los Ancianos de los Días.

\par
%\textsuperscript{(604.1)}
\textsuperscript{53:3.6} 3. \textit{El ataque contra el plan universal para educar a los mortales ascendentes}. Lucifer sostenía que se empleaba demasiado tiempo y energía en el proyecto de instruir a fondo a los mortales ascendentes en los principios de la administración del universo, unos principios que calificaba de inmorales y de poco sólidos. Protestó contra el programa milenario consistente en preparar a los mortales del espacio para algún destino desconocido, e indicó que la presencia del cuerpo finalitario en Jerusem era una prueba de que estos mortales habían pasado eras enteras de preparación para un destino de pura ficción. Señaló con irrisión que los finalitarios no habían encontrado un destino más glorioso que el de ser devueltos a unas humildes esferas, similares a las de su origen. Insinuó que habían sido corrompidos por un exceso de disciplina y una formación demasiado prolongada, y que en realidad traicionaban a sus compañeros mortales puesto que ahora cooperaban en el proyecto de esclavizar a toda la creación a las ficciones de un mítico destino eterno para los mortales ascendentes. Defendió que los ascendentes debían disfrutar de la libertad de la autodeterminación individual. Puso en tela de juicio y condenó todo el plan para la ascensión de los mortales tal como está patrocinado por los Hijos Paradisiacos de Dios y apoyado por el Espíritu Infinito.

\par
%\textsuperscript{(604.2)}
\textsuperscript{53:3.7} Con esta Declaración de Libertad es con la que Lucifer emprendió su orgía de tinieblas y de muerte.

\section*{4. El comienzo de la rebelión}
\par
%\textsuperscript{(604.3)}
\textsuperscript{53:4.1} El manifiesto de Lucifer se publicó en el cónclave anual de Satania celebrado en el mar de cristal, en presencia de las huestes reunidas de Jerusem, el último día del año hace unos doscientos mil años del tiempo de Urantia. Satanás proclamó que se podía adorar a las fuerzas universales ---físicas, intelectuales y espirituales--- pero que sólo se podía tener lealtad a Lucifer, el gobernante efectivo y actual, el <<amigo de los hombres y de los ángeles>> y el <<Dios de la libertad>>.

\par
%\textsuperscript{(604.4)}
\textsuperscript{53:4.2} La reafirmación personal fue el grito de guerra de la rebelión\footnote{\textit{Rebelión}: Ap 12:7.} de Lucifer. Uno de sus argumentos principales fue que, si el gobierno autónomo era bueno y apropiado para los Melquisedeks y otros grupos, era igualmente bueno para todas las órdenes de inteligencias. Fue resuelto e insistente en su defensa de la <<igualdad de la mente>> y de <<la fraternidad de la inteligencia>>. Sostenía que todo gobierno debía limitarse a los planetas locales y a su confederación voluntaria en los sistemas locales. Rechazó toda otra supervisión. Prometió a los Príncipes Planetarios que gobernarían los mundos como ejecutivos supremos. Denunció que las actividades legislativas estuvieran situadas en la sede de la constelación y que los asuntos judiciales estuvieran dirigidos desde la capital del universo. Afirmó que todas estas funciones gubernamentales debían estar concentradas en las capitales de los sistemas, y procedió a establecer su propia asamblea legislativa y a organizar sus propios tribunales bajo la jurisdicción de Satanás. Y ordenó que los príncipes de los mundos apóstatas hicieran lo mismo.

\par
%\textsuperscript{(604.5)}
\textsuperscript{53:4.3} Todo el gabinete administrativo de Lucifer se pasó en masa a su campo, y sus miembros prestaron juramento públicamente como agentes de la administración del nuevo jefe de <<los mundos y de los sistemas liberados>>.

\par
%\textsuperscript{(605.1)}
\textsuperscript{53:4.4} Aunque había habido dos rebeliones anteriores en Nebadon, se habían producido en constelaciones lejanas. Lucifer consideraba que estas insurrecciones habían fracasado porque la mayoría de las inteligencias dejaron de seguir a sus jefes. Afirmaba que <<las mayorías gobiernan>>, que <<la mente es infalible>>. La libertad que le permitieron los gobernantes del universo sostuvo aparentemente muchas de sus opiniones infames. Desafió a todos sus superiores; sin embargo, éstos parecieron no tomar nota de sus acciones. Se le dejó el campo libre para que prosiguiera su plan seductor sin obstáculos ni trabas.

\par
%\textsuperscript{(605.2)}
\textsuperscript{53:4.5} Lucifer indicó que todos los aplazamientos misericordiosos de la justicia eran una prueba de que el gobierno de los Hijos Paradisiacos era incapaz de detener la rebelión. Solía desafiar abiertamente y retar con arrogancia a Miguel, a Emmanuel y a los Ancianos de los Días, y luego señalaba que el hecho de que no se produjera ninguna acción era una prueba evidente de la impotencia de los gobiernos del universo y del superuniverso.

\par
%\textsuperscript{(605.3)}
\textsuperscript{53:4.6} Gabriel estuvo personalmente presente durante toda esta cadena de acontecimientos desleales y sólo anunció que a su debido tiempo hablaría en nombre de Miguel, y que todos los seres efectuarían su elección de manera libre y sin ser molestados; que el <<gobierno de los Hijos en nombre del Padre sólo deseaba una lealtad y una devoción que fueran voluntarias, sinceras y a prueba de sofismas>>.

\par
%\textsuperscript{(605.4)}
\textsuperscript{53:4.7} A Lucifer se le permitió establecer plenamente y organizar por completo su gobierno rebelde antes de que Gabriel hiciera el menor esfuerzo por impugnar el derecho a la secesión o por contrarrestar la propaganda rebelde. Pero los Padres de la Constelación limitaron de inmediato la acción de estas personalidades desleales al sistema de Satania. Sin embargo, esta demora fue un período de grandes pruebas y sufrimientos para los seres leales de toda Satania. Durante algunos años todo fue un caos, y hubo una gran confusión en los mundos de las mansiones.

\section*{5. La naturaleza del conflicto}
\par
%\textsuperscript{(605.5)}
\textsuperscript{53:5.1} Cuando estalló la rebelión en Satania, Miguel pidió consejo a Emmanuel, su hermano paradisiaco. Después de esta conferencia tan importante, Miguel anunció que continuaría la misma política que había caracterizado su conducta ante unos disturbios similares en el pasado, una actitud de no intromisión.

\par
%\textsuperscript{(605.6)}
\textsuperscript{53:5.2} En la época de esta rebelión y de las dos que la precedieron, no existía ninguna autoridad soberana absoluta y personal en el universo de Nebadon. Miguel gobernaba por derecho divino como vicegerente del Padre Universal, pero no todavía por su propio derecho personal. No había terminado su carrera de donación; todavía no había sido investido de <<todos los poderes en el cielo y en la Tierra>>\footnote{\textit{Todos los poderes en el cielo y en la Tierra}: Mt 28:18.}.

\par
%\textsuperscript{(605.7)}
\textsuperscript{53:5.3} Desde el estallido de la rebelión hasta el día de su entronización como gobernante soberano de Nebadon, Miguel no se opuso nunca a las fuerzas rebeldes de Lucifer; se les permitió seguir su curso libremente durante cerca de doscientos mil años del tiempo de Urantia. Cristo Miguel posee ahora suficiente poder y autoridad para enfrentarse de inmediato, e incluso sumariamente, con estos estallidos de deslealtad, pero dudamos de que esta autoridad soberana le conduzca a actuar de manera diferente si se produjera otro levantamiento de este tipo.

\par
%\textsuperscript{(605.8)}
\textsuperscript{53:5.4} Puesto que Miguel eligió mantenerse apartado de la guerra misma de la rebelión de Lucifer, Gabriel convocó a su estado mayor personal en Edentia y, por consejo de los Altísimos, eligió asumir el mando de las huestes leales de Satania. Miguel permaneció en Salvington mientras Gabriel se dirigió a Jerusem; se estableció en la esfera dedicada al Padre ---al mismo Padre Universal cuya personalidad habían puesto en duda Lucifer y Satanás---, y en presencia de las huestes de personalidades leales reunidas, desplegó el estandarte de Miguel, el emblema material del gobierno trinitario de toda la creación, los tres círculos concéntricos de color azul celeste sobre fondo blanco.

\par
%\textsuperscript{(606.1)}
\textsuperscript{53:5.5} El emblema de Lucifer era un estandarte blanco con un círculo rojo, en cuyo centro aparecía un sólido círculo negro.

\par
%\textsuperscript{(606.2)}
\textsuperscript{53:5.6} <<Había guerra en el cielo; el comandante de Miguel y sus ángeles lucharon contra el dragón (Lucifer, Satanás y los príncipes apóstatas); y el dragón y sus ángeles rebeldes lucharon, pero no triunfaron>>\footnote{\textit{Guerra en el cielo}: Ap 12:7-8.}. Esta <<guerra en el cielo>> no fue una batalla física tal como un conflicto así se puede concebir en Urantia. En los primeros días de la lucha, Lucifer pronunció continuos discursos en el anfiteatro planetario. Gabriel dirigió una exposición incesante de los sofismas rebeldes desde su sede situada en las cercanías. Las diversas personalidades presentes en la esfera que tenían dudas sobre la actitud a tomar iban de acá para allá entre estas discusiones hasta que llegaron a una decisión final.

\par
%\textsuperscript{(606.3)}
\textsuperscript{53:5.7} Pero esta guerra en el cielo fue muy terrible y muy real. Aunque no mostraba ninguna de las barbaridades tan características de la guerra física en los mundos inmaduros, este conflicto era mucho más mortífero; la vida material está en peligro en los combates materiales, pero la guerra en el cielo se libraba en términos de vida eterna.

\section*{6. Un comandante seráfico leal}
\par
%\textsuperscript{(606.4)}
\textsuperscript{53:6.1} Muchos actos nobles e inspiradores de devoción y de lealtad fueron efectuados por numerosas personalidades durante el intervalo de tiempo que transcurrió entre el comienzo de las hostilidades y la llegada del nuevo gobernante del sistema con su estado mayor. Pero la más emocionante de todas estas atrevidas pruebas de devoción fue la valiente conducta de Manotia, el segundo comandante de los serafines de la sede de Satania.

\par
%\textsuperscript{(606.5)}
\textsuperscript{53:6.2} Cuando la rebelión estalló en Jerusem, el jefe de las huestes seráficas se unió a la causa de Lucifer. Esto explica sin duda por qué un número tan grande de serafines de la cuarta orden, los administradores sistémicos, se descarrió. El jefe seráfico quedó espiritualmente cegado por la brillante personalidad de Lucifer; sus maneras encantadoras fascinaban a las órdenes inferiores de seres celestiales. No podían simplemente comprender que una personalidad tan deslumbrante pudiera equivocarse de dirección.

\par
%\textsuperscript{(606.6)}
\textsuperscript{53:6.3} No hace mucho tiempo, al describir las experiencias asociadas con el comienzo de la rebelión de Lucifer, Manotia dijo: <<Pero el momento más estimulante para mí fue la emocionante aventura relacionada con la rebelión de Lucifer, cuando en mi calidad de segundo comandante seráfico me negué a participar en el proyecto de insultar a Miguel; y los poderosos rebeldes trataron de destruirme por medio de las fuerzas de enlace que habían organizado. Hubo una enorme agitación en Jerusem, pero ni un solo serafín leal sufrió daño alguno>>.

\par
%\textsuperscript{(606.7)}
\textsuperscript{53:6.4} <<Tras la falta de mi superior inmediato, recayó sobre mí el asumir el mando de las huestes angélicas de Jerusem como director titular de los confusos asuntos seráficos del sistema. Los Melquisedeks me apoyaron moralmente, una mayoría de Hijos Materiales me ayudó hábilmente, un enorme grupo de mi propia orden me abandonó, pero los mortales ascendentes de Jerusem me sostuvieron de una forma magnífica>>.

\par
%\textsuperscript{(606.8)}
\textsuperscript{53:6.5} <<Como nos habían expulsado automáticamente de los circuitos de la constelación debido a la secesión de Lucifer, dependíamos de la lealtad de nuestro cuerpo de información, el cual enviaba llamadas de socorro a Edentia desde el cercano sistema de Rantulia; y descubrimos que el reino del orden, la lealtad intelectual y el espíritu de la verdad triunfaban de manera inherente sobre la rebelión, la reafirmación personal y la supuesta libertad personal; fuimos capaces de seguir adelante hasta la llegada del nuevo Soberano del Sistema, del noble sucesor de Lucifer. Inmediatamente después fui destinado al cuerpo de los síndicos Melquisedeks de Urantia, y asumí la jurisdicción sobre las órdenes seráficas leales en el mundo del traidor Caligastia, el cual había proclamado a su esfera miembro del sistema recién proyectado de `mundos liberados y de personalidades emancipadas', propuesto en la infame Declaración de Libertad promulgada por Lucifer en su llamada a las `inteligencias amantes de la libertad, librepensadoras y progresistas de los mundos mal gobernados y mal administrados de Satania'>>.

\par
%\textsuperscript{(607.1)}
\textsuperscript{53:6.6} Este ángel está todavía de servicio en Urantia, donde ejerce su actividad como jefe asociado de los serafines.

\section*{7. La historia de la rebelión}
\par
%\textsuperscript{(607.2)}
\textsuperscript{53:7.1} La rebelión de Lucifer abarcó todo el sistema. Treinta y siete Príncipes Planetarios separatistas pusieron una gran parte de las administraciones de sus mundos del lado del archirrebelde. Sólo en Panoptia el Príncipe Planetario no logró arrastrar a sus pueblos con él. En este mundo, y bajo la dirección de los Melquisedeks, la gente se unió en apoyo de Miguel. Elanora, una joven de este reino de mortales, tomó el mando de las razas humanas y ni una sola alma de este mundo desgarrado por los conflictos se alistó bajo el estandarte de Lucifer. Desde aquel entonces, estos leales panoptianos han servido en el séptimo mundo de transición de Jerusem como vigilantes y constructores en la esfera del Padre y en los siete mundos de detención que la rodean. Los panoptianos no sólo actúan como guardianes literales de estos mundos, sino que también ejecutan las órdenes personales de Miguel destinadas a embellecer estas esferas para algún uso futuro desconocido. Efectúan este trabajo mientras se demoran en su camino hacia Edentia.

\par
%\textsuperscript{(607.3)}
\textsuperscript{53:7.2} Durante todo este período, Caligastia estuvo defendiendo la causa de Lucifer en Urantia. Los Melquisedeks se opusieron hábilmente al Príncipe Planetario apóstata, pero los sofismas de la libertad desenfrenada y las ilusiones de la reafirmación personal tenían todas las posibilidades de engañar a los pueblos primitivos de un mundo joven y no desarrollado.

\par
%\textsuperscript{(607.4)}
\textsuperscript{53:7.3} Toda la propaganda de la secesión tuvo que llevarse a cabo mediante esfuerzos personales, porque el servicio de las transmisiones y todos los otros medios de comunicación interplanetaria estaban suspendidos debido a la acción de los supervisores de los circuitos del sistema. En el momento de estallar realmente la insurrección, todo el sistema de Satania fue aislado tanto de los circuitos de la constelación como de los del universo. Durante este período, todos los mensajes que llegaban y salían eran enviados a través de los agentes seráficos y de los Mensajeros Solitarios. Los circuitos que llegaban hasta los mundos caídos también estaban cortados, de manera que Lucifer no pudo utilizar esta vía para fomentar su infame proyecto. Y estos circuitos no se restablecerán mientras el archirrebelde viva dentro de los confines de Satania.

\par
%\textsuperscript{(607.5)}
\textsuperscript{53:7.4} Fue una rebelión Lanonandek. Las órdenes superiores de filiación del universo local no se unieron a la secesión de Lucifer, aunque la rebelión de los príncipes desleales influyó un poco sobre algunos Portadores de Vida estacionados en los planetas rebeldes. Ninguno de los Hijos Trinitizados se descarrió. Los Melquisedeks, los arcángeles y las Brillantes Estrellas Vespertinas permanecieron todos leales a Miguel y, junto con Gabriel, lucharon valientemente por la voluntad del Padre y el gobierno del Hijo.

\par
%\textsuperscript{(608.1)}
\textsuperscript{53:7.5} Ningún ser originario del Paraíso estuvo implicado en la deslealtad. Junto con los Mensajeros Solitarios, establecieron su sede en el mundo del Espíritu y permanecieron bajo el mando del Fiel de los Días de Edentia. Ninguno de los conciliadores apostató, y ni uno solo de los Registradores Celestiales se descarrió. Pero hubo grandes pérdidas entre los Compañeros Morontiales y los Educadores de los Mundos de las Mansiones.

\par
%\textsuperscript{(608.2)}
\textsuperscript{53:7.6} No se perdió ni un solo ángel de la orden suprema de serafines, pero de la orden siguiente, la superior, un grupo considerable fue engañado y atrapado. También se descarriaron algunos miembros de la orden tercera, u orden supervisora, de ángeles. Pero el terrible desmoronamiento se produjo en el cuarto grupo, el de los ángeles administradores, los serafines que están normalmente asignados a las tareas de las capitales de los sistemas. Manotia salvó a casi dos tercios de ellos, pero un poco más de un tercio siguió a su jefe en las filas rebeldes. De todos los querubines de Jerusem vinculados a los ángeles administradores, un tercio se perdió con sus serafines desleales.

\par
%\textsuperscript{(608.3)}
\textsuperscript{53:7.7} De los ayudantes angélicos planetarios, de aquellos que están asignados a los Hijos Materiales, alrededor de un tercio fueron engañados, y casi el diez por ciento de los ministros de transición fueron atrapados. Juan vio todo esto simbólicamente cuando escribió del gran dragón rojo, diciendo: <<Y su cola atrajo a una tercera parte de las estrellas del cielo y las arrojó a las tinieblas>>\footnote{\textit{Una tercera parte de las estrellas}: Ap 12:3-4.}.

\par
%\textsuperscript{(608.4)}
\textsuperscript{53:7.8} Las pérdidas más grandes tuvieron lugar en las filas angélicas, pero la mayor parte de las órdenes inferiores de inteligencias estuvieron implicadas en la deslealtad. De los 681.217 Hijos Materiales que se perdieron en Satania, el noventa y cinco por ciento fueron víctimas de la rebelión de Lucifer. Un gran número de criaturas intermedias se perdió en aquellos planetas individuales cuyos Príncipes Planetarios se unieron a la causa de Lucifer.

\par
%\textsuperscript{(608.5)}
\textsuperscript{53:7.9} Esta rebelión fue, en muchos aspectos, la más extensa y desastrosa de todos los sucesos de este tipo acaecidos en Nebadon. En esta insurrección estuvieron implicadas más personalidades que en el conjunto de las otras dos. Y permanecerá en su eterno deshonor el hecho de que los emisarios de Lucifer y de Satanás no respetaran las escuelas de educación infantil del planeta cultural de los finalitarios, sino que más bien intentaron corromper a estas mentes en desarrollo salvadas por misericordia de los mundos evolutivos.

\par
%\textsuperscript{(608.6)}
\textsuperscript{53:7.10} Los mortales ascendentes eran vulnerables, pero resistieron mejor que los espíritus inferiores a los sofismas de la rebelión. Aunque cayeron muchos seres en los mundos de las mansiones más inferiores, los que no habían logrado fusionar finalmente con su Ajustador, está registrado para la gloria de la sabiduría del programa de la ascensión que ni un solo miembro de los ciudadanos ascendentes de Satania, residentes en Jerusem, participó en la rebelión de Lucifer.

\par
%\textsuperscript{(608.7)}
\textsuperscript{53:7.11} Hora tras hora y día tras día, las estaciones emisoras de todo Nebadon estaban atestadas de observadores ansiosos de todas las clases imaginables de inteligencias celestiales, que leían atentamente los boletines sobre la rebelión de Satania y se regocijaban a medida que los informes narraban continuamente la lealtad inquebrantable de los mortales ascendentes que, bajo la dirección de los Melquisedeks, resistían con éxito a los esfuerzos combinados y prolongados de todas las sutiles fuerzas del mal que se habían congregado con tanta rapidez alrededor de los estandartes de la secesión y del pecado.

\par
%\textsuperscript{(608.8)}
\textsuperscript{53:7.12} Desde el comienzo de la <<guerra en el cielo>> hasta la instalación del sucesor de Lucifer pasaron más de dos años\footnote{\textit{Después de dos años de guerra en el cielo}: Ap 12:7.} del tiempo del sistema. Pero el nuevo Soberano llegó por fin, aterrizando en el mar de cristal con su estado mayor. Yo me encontraba entre las reservas movilizadas por Gabriel en Edentia, y recuerdo muy bien el primer mensaje de Lanaforge al Padre de la Constelación de Norlatiadek. Decía: <<No se ha perdido ni un solo ciudadano de Jerusem\footnote{\textit{Ningún ciudadano de Jerusem perdido}: Ap 12:8.}. Todos los mortales ascendentes han sobrevivido a la prueba de fuego y han salido triunfantes y totalmente victoriosos de la prueba decisiva>>. Este mensaje llegó hasta Salvington, Uversa y el Paraíso asegurando que la experiencia sobreviviente de la ascensión de los mortales es la mayor garantía contra la rebelión y la más firme salvaguardia contra el pecado. Este noble grupo de Jerusem ascendía exactamente a 187.432.811 fieles mortales.

\par
%\textsuperscript{(609.1)}
\textsuperscript{53:7.13} Con la llegada de Lanaforge, los archirrebeldes fueron destronados y privados de todo poder gobernante, aunque se les permitió circular libremente por Jerusem, las esferas morontiales e incluso los mundos habitados individuales. Continuaron con sus esfuerzos engañosos y seductores para confundir y descarriar a las mentes de los hombres y de los ángeles. Pero en lo que se refiere a su trabajo en el monte administrativo de Jerusem, <<ya no hubo sitio para ellos>>\footnote{\textit{Su lugar ya no se encontró más}: Ap 12:8.}.

\par
%\textsuperscript{(609.2)}
\textsuperscript{53:7.14} Aunque Lucifer estaba privado de toda autoridad administrativa en Satania, no existía entonces ningún poder ni tribunal en el universo local que pudiera detener o destruir a este malvado rebelde; en aquella época, Miguel aún no era gobernante soberano. Los Ancianos de los Días apoyaron a los Padres de la Constelación en su incautación del gobierno del sistema, pero nunca han anunciado ninguna decisión posterior sobre los numerosos recursos todavía pendientes relacionados con el estado presente y la disposición futura que se hará de Lucifer, Satanás y sus asociados.

\par
%\textsuperscript{(609.3)}
\textsuperscript{53:7.15} A estos archirrebeldes se les permitió así que vagaran por todo el sistema tratando de extender aún más sus doctrinas de descontento y de reafirmación personal. Pero han sido incapaces de engañar a otro mundo desde hace casi doscientos mil años de Urantia. Ningún mundo de Satania se ha perdido desde la caída de los treinta y siete, ni siquiera los mundos más jóvenes que fueron poblados después de la época de la rebelión\footnote{\textit{Lucifer fracasa}: Ap 12:7-8.}.

\section*{8. El Hijo del Hombre en Urantia}
\par
%\textsuperscript{(609.4)}
\textsuperscript{53:8.1} Lucifer y Satanás vagaron libremente por el sistema de Satania hasta que Miguel finalizó su misión donadora en Urantia. Estuvieron juntos por última vez en vuestro mundo en el momento de su ataque combinado contra el Hijo del Hombre.

\par
%\textsuperscript{(609.5)}
\textsuperscript{53:8.2} Anteriormente, cuando los Príncipes Planetarios, los <<Hijos de Dios>>, se congregaban periódicamente\footnote{\textit{Cuando los Hijos de Dios se congregaban}: Job 1:6.}, <<Satanás también asistía>>\footnote{\textit{Satanás también asistía}: Job 2:1.}, afirmando que representaba a todos los mundos aislados de los Príncipes Planetarios caídos. Pero no se le ha concedido esta libertad en Jerusem desde la donación final de Miguel. Después de sus esfuerzos por corromper a Miguel durante su donación en la carne, toda simpatía por Lucifer y Satanás ha perecido en toda Satania, es decir, fuera de los mundos aislados por el pecado.

\par
%\textsuperscript{(609.6)}
\textsuperscript{53:8.3} La donación de Miguel puso fin a la rebelión\footnote{\textit{Fin de la rebelión}: Mt 4:1-11; Mc 1:13.} de Lucifer en toda Satania, salvo en los planetas de los Príncipes Planetarios apóstatas. Éste fue el significado de la experiencia personal de Jesús poco antes de su muerte en la carne, cuando cierto día exclamó a sus discípulos: <<Y vi caer a Satanás desde el cielo como un rayo>>\footnote{\textit{Vi caer a Satanás como un rayo}: Lc 10:18.}. Había venido con Lucifer a Urantia para librar la última batalla decisiva.

\par
%\textsuperscript{(609.7)}
\textsuperscript{53:8.4} El Hijo del Hombre tenía confianza en su éxito y sabía que su triunfo en vuestro mundo fijaría para siempre el estado de sus enemigos seculares, no solamente en Satania sino también en los otros dos sistemas donde había penetrado el pecado. La supervivencia de los mortales y la seguridad de los ángeles estuvo garantizada cuando vuestro Maestro, en respuesta a las propuestas de Lucifer, replicó tranquilamente y con una seguridad divina: <<Detrás de mí, Satanás>>\footnote{\textit{Ve detrás de mí, Satanás}: Mt 4:10; 16:23; Mc 8:33; Lc 4:8.}. Éste fue, en principio, el verdadero final de la rebelión de Lucifer. Es verdad que los tribunales de Uversa aún no han pronunciado la decisión ejecutiva relacionada con la apelación de Gabriel solicitando la destrucción de los rebeldes, pero no hay duda de que este decreto se recibirá en la plenitud de los tiempos puesto que ya se han dado los primeros pasos para la audiencia de este caso.

\par
%\textsuperscript{(610.1)}
\textsuperscript{53:8.5} El Hijo del Hombre reconoció que Caligastia era el Príncipe técnico de Urantia poco tiempo antes de su muerte. Jesús dijo: <<Ahora es el juicio de este mundo; ahora el príncipe de este mundo será derribado>>\footnote{\textit{Ahora es el juicio}: Jn 12:31. \textit{Príncipe de este mundo}: Jn 12:31; 14:30; 16:11; Ef 2:2; 6:12.}. Y más tarde aún, antes de finalizar la misión de su vida, anunció: <<El príncipe de este mundo es juzgado>>\footnote{\textit{El príncipe de este mundo es juzgado}: Jn 16:11.}. Este mismo Príncipe destronado y desacreditado es el que en otro tiempo fue llamado <<Dios de Urantia>>\footnote{\textit{Dios de Urantia}: 2 Co 4:4.}.

\par
%\textsuperscript{(610.2)}
\textsuperscript{53:8.6} El último acto de Miguel antes de dejar Urantia consistió en ofrecer misericordia a Caligastia y a Daligastia, pero éstos despreciaron su tierna oferta. Caligastia, vuestro Príncipe Planetario apóstata, sigue siendo libre de proseguir sus infames intenciones en Urantia, pero no tiene ningún poder en absoluto para penetrar en la mente de los hombres ni tampoco puede acercarse a sus almas para tentarlas o corromperlas, a menos que los hombres deseen realmente ser maldecidos por su malvada presencia.

\par
%\textsuperscript{(610.3)}
\textsuperscript{53:8.7} Antes de la donación de Miguel, estos gobernantes de las tinieblas trataron de mantener su autoridad en Urantia, y se resistieron con insistencia a las personalidades celestiales menores y subordinadas. Pero desde el día de Pentecostés, este traidor Caligastia y su igualmente despreciable asociado Daligastia son serviles ante la majestad divina de los Ajustadores del Pensamiento paradisiacos y del Espíritu de la Verdad protector, el espíritu de Miguel, que ha sido derramado sobre todo el género humano.

\par
%\textsuperscript{(610.4)}
\textsuperscript{53:8.8} Pero incluso así, ningún espíritu caído ha tenido nunca el poder de invadir la mente ni de atormentar el alma de los hijos de Dios. Ni Satanás ni Caligastia han podido nunca influir o acercarse a los hijos de Dios por la fe; la fe es una armadura eficaz contra el pecado\footnote{\textit{La fe, armadura contra el pecado}: Ef 6:16.} y la iniquidad. Es verdad que <<aquel que ha nacido de Dios se protege a sí mismo, y que el malvado no le influye>>\footnote{\textit{El nacido de Dios se protege a sí mismo}: 1 Jn 5:18.}.

\par
%\textsuperscript{(610.5)}
\textsuperscript{53:8.9} Cuando se supone, en general, que los mortales débiles y disolutos se encuentran bajo la influencia de los diablos y los demonios, están simplemente dominados por sus propias tendencias inherentes y degradadas, se dejan llevar por sus propias inclinaciones naturales. Al diablo se le ha atribuido una gran cantidad de méritos que no le pertenecen. Caligastia ha permanecido relativamente impotente desde la cruz de Cristo.

\section*{9. El estado actual de la rebelión}
\par
%\textsuperscript{(610.6)}
\textsuperscript{53:9.1} En los primeros tiempos de la rebelión de Lucifer, Miguel ofreció la salvación a todos los rebeldes. A todos los que mostraran un arrepentimiento sincero les ofreció el perdón y la reintegración en alguna forma de servicio universal en cuanto lograra la plena soberanía sobre su universo. Ninguno de los dirigentes aceptó esta oferta misericordiosa. Pero miles de ángeles y de seres celestiales de las órdenes inferiores, incluyendo a cientos de Hijos e Hijas Materiales, aceptaron la misericordia proclamada por los panoptianos y fueron rehabilitados en el momento de la resurrección de Jesús hace mil novecientos años. Desde entonces, estos seres han sido trasladados al mundo del Padre cercano a Jerusem, donde han de permanecer técnicamente retenidos hasta que los tribunales de Uversa anuncien su decisión sobre el asunto de Gabriel \textit{contra} Lucifer. Pero nadie duda de que estas personalidades arrepentidas y salvadas estarán excluidas del decreto de extinción cuando se pronuncie el veredicto de aniquilación. Estas almas a prueba trabajan ahora con los panoptianos en la tarea de cuidar del mundo del Padre\footnote{\textit{Estado presente de la rebelión}: 1 P 3:19-20.}.

\par
%\textsuperscript{(611.1)}
\textsuperscript{53:9.2} El archiembaucador no ha estado nunca en Urantia desde la época en que intentó desviar a Miguel del propósito de finalizar su donación y de establecerse de manera segura y definitiva como gobernante incondicional de Nebadon. Cuando Miguel se convirtió en el jefe establecido del universo de Nebadon, Lucifer fue detenido por los agentes de los Ancianos de los Días de Uversa, y desde entonces ha estado preso en el satélite número uno del grupo de esferas de transición del Padre que rodean a Jerusem\footnote{\textit{Estado de Lucifer}: Ap 12:7-11.}. Aquí, los gobernantes de otros mundos y de otros sistemas contemplan el final del infiel Soberano de Satania. Pablo conocía el estado de estos cabecillas rebeldes después de la donación de Miguel, pues describió a los jefes de Caligastia como <<una hueste espiritual de maldad en los lugares celestiales>>\footnote{\textit{Hueste espiritual de maldad}: Ef 6:12.}.

\par
%\textsuperscript{(611.2)}
\textsuperscript{53:9.3} Cuando Miguel asumió la soberanía suprema de Nebadon, solicitó a los Ancianos de los Días la autorización de internar a todas las personalidades implicadas en la rebelión de Lucifer hasta que se pronunciara el fallo de los tribunales superuniversales en el caso de Gabriel \textit{contra} Lucifer, inscrito en los registros del tribunal supremo de Uversa hace cerca de doscientos mil años tal como vosotros calculáis el tiempo\footnote{\textit{Estado de los rebeldes}: Ap 12:9; 20:1-13.}. En cuanto al grupo de la capital del sistema, los Ancianos de los Días concedieron la petición de Miguel pero con una sola excepción: a Satanás se le permitiría hacer visitas periódicas a los príncipes apóstatas de los mundos caídos hasta que otro Hijo de Dios fuera aceptado por esos mundos apóstatas, o hasta el momento en que los tribunales de Uversa empezaran a juzgar el caso de Gabriel \textit{contra} Lucifer.

\par
%\textsuperscript{(611.3)}
\textsuperscript{53:9.4} Satanás podía venir a Urantia porque no teníais ningún Hijo de alta categoría que residiera aquí ---ni un Príncipe Planetario ni un Hijo Material. Desde entonces, Maquiventa Melquisedek ha sido proclamado Príncipe Planetario vicegerente de Urantia, y la apertura del caso de Gabriel \textit{contra} Lucifer ha señalado el comienzo de unos regímenes planetarios temporales en todos los mundos aislados. Es verdad que Satanás visitó periódicamente a Caligastia y a otros príncipes caídos hasta el momento de la presentación de estas revelaciones, cuando ha tenido lugar la primera audiencia de la petición de Gabriel para la aniquilación de los archirrebeldes. Satanás está ahora detenido incondicionalmente en los mundos prisiones de Jerusem.

\par
%\textsuperscript{(611.4)}
\textsuperscript{53:9.5} Desde la donación final de Miguel, nadie, en todo Satania, ha deseado ir a los mundos prisiones para ayudar a los rebeldes internados. Y ningún otro ser se ha sentido atraído por la causa del embaucador. Durante mil novecientos años, la situación ha permanecido sin cambios\footnote{\textit{Estado de los rebeldes}: Ap 12:9.}.

\par
%\textsuperscript{(611.5)}
\textsuperscript{53:9.6} No esperamos que se supriman las actuales restricciones en Satania hasta que los Ancianos de los Días no hayan dispuesto definitivamente de los archirrebeldes. Los circuitos del sistema no serán restablecidos mientras viva Lucifer. Entretanto, éste último está totalmente inactivo\footnote{\textit{Estado de los rebeldes}: Ez 28:16-19.}.

\par
%\textsuperscript{(611.6)}
\textsuperscript{53:9.7} La rebelión ha finalizado en Jerusem. Y termina en los mundos caídos tan pronto como llegan los Hijos divinos. Creemos que todos los rebeldes que han querido aceptar la misericordia ya lo han hecho. Estamos a la espera de la transmisión centelleante que privará a estos traidores de la existencia de la personalidad. Prevemos que el veredicto de Uversa será anunciado mediante la transmisión ejecutoria que efectuará la aniquilación de estos rebeldes internados. Entonces buscaréis sus sitios\footnote{\textit{Sitios no encontrados más}: Ap 12:8.} pero no los encontraréis. <<Y aquellos mundos que os conocen se quedarán asombrados de vosotros; habéis sido un terror, pero nunca más volveréis a existir>>\footnote{\textit{Mundos asombrados}: Ez 28:19.}. Así es como todos estos indignos traidores <<se volverán como si no hubieran existido>>\footnote{\textit{Serán como si no hubieran existido}: Abd 1:16.}. Todos esperan el decreto de Uversa.

\par
%\textsuperscript{(611.7)}
\textsuperscript{53:9.8} Pero durante eras enteras, los siete mundos prisiones de tinieblas espirituales de Satania han constituido una advertencia solemne para todo Nebadon, proclamando de manera elocuente y eficaz la gran verdad de que <<el camino del transgresor es duro>>\footnote{\textit{El camino del transgresor es duro}: Pr 13:15.}; que <<dentro de cada pecado se oculta la semilla de su propia destrucción>>; que <<el salario del pecado es la muerte>>\footnote{\textit{El salario del pecado es la muerte}: Gn 2:17; Ro 6:23.}.

\par
%\textsuperscript{(612.1)}
\textsuperscript{53:9.9} [Presentado por Manovandet Melquisedek, en otro tiempo vinculado a los síndicos de Urantia.]


\chapter{Documento 54. Los problemas de la rebelión de Lucifer}
\par
%\textsuperscript{(613.1)}
\textsuperscript{54:0.1} AL HOMBRE evolutivo le resulta difícil comprender plenamente el significado y captar el sentido del mal, del error, del pecado y de la iniquidad. El hombre es lento en percibir que la perfección y la imperfección contrapuestas producen el mal potencial; que la verdad y la falsedad en conflicto crean el error desconcertante; que el don divino de poder elegir mediante el libre albedrío conduce a los reinos divergentes del pecado y de la rectitud; que la búsqueda perseverante de la divinidad conduce al reino de Dios, en contraste con su continuo rechazo, el cual conduce a los dominios de la iniquidad.

\par
%\textsuperscript{(613.2)}
\textsuperscript{54:0.2} Los Dioses no crean el mal ni permiten el pecado y la rebelión. El mal potencial existe en el tiempo en un universo que contiene niveles diferenciales de significados y de valores sobre la perfección. El pecado es potencial en todos los reinos donde los seres imperfectos están dotados de la capacidad de elegir entre el bien y el mal. La misma presencia contrapuesta de la verdad y de la mentira, del hecho y de la falsedad, constituye la potencialidad del error. La elección deliberada del mal constituye el pecado; el rechazo voluntario de la verdad es el error; la persecución insistente del pecado y del error es la iniquidad.

\section*{1. La verdadera y la falsa libertad}
\par
%\textsuperscript{(613.3)}
\textsuperscript{54:1.1} De todos los confusos problemas derivados de la rebelión de Lucifer, ninguno ha ocasionado más dificultades que la incapacidad de los mortales evolutivos inmaduros para distinguir entre la verdadera y la falsa libertad\footnote{\textit{La verdadera y la falsa libertad}: Jn 8:32,36; Gl 5:13; Stg 1:25; 1 P 2:15-16.}.

\par
%\textsuperscript{(613.4)}
\textsuperscript{54:1.2} La verdadera libertad es la búsqueda de los siglos y la recompensa del progreso evolutivo. La falsa libertad es el engaño sutil del error del tiempo y del mal del espacio. La libertad duradera está basada en la realidad de la justicia ---la inteligencia, la madurez, la fraternidad y la equidad.

\par
%\textsuperscript{(613.5)}
\textsuperscript{54:1.3} La libertad es una técnica autodestructora de la existencia cósmica cuando su motivación es poco inteligente, incondicional e incontrolada. La verdadera libertad está progresivamente relacionada con la realidad y siempre es respetuosa con la equidad social, la justicia cósmica, la fraternidad universal y las obligaciones divinas.

\par
%\textsuperscript{(613.6)}
\textsuperscript{54:1.4} La libertad es suicida cuando está divorciada de la justicia material, de la equidad intelectual, de la paciencia social, del deber moral y de los valores espirituales. La libertad no existe fuera de la realidad cósmica, y toda realidad de una personalidad es proporcional a sus relaciones con la divinidad.

\par
%\textsuperscript{(613.7)}
\textsuperscript{54:1.5} La voluntad personal sin frenos y la expresión desordenada del yo equivalen a un egoísmo total, al súmmum de la impiedad. La libertad, sin una conquista asociada y cada vez mayor del yo, es un producto de la imaginación humana egoísta. La libertad motivada por el yo es una ilusión conceptual, un cruel engaño. La licencia disfrazada con los vestidos de la libertad es la precursora de una esclavitud abyecta.

\par
%\textsuperscript{(614.1)}
\textsuperscript{54:1.6} La verdadera libertad es la asociada de la auténtica autoestima; la falsa libertad es la consorte de la admiración de sí mismo. La verdadera libertad es el fruto del autocontrol; la falsa libertad es la pretensión de la reafirmación personal. El autocontrol conduce al servicio altruista; la admiración de sí mismo tiende a explotar a los demás para el engrandecimiento egoísta del individuo equivocado que está dispuesto a sacrificar una justa consecución a fin de poseer un poder injusto sobre sus semejantes.

\par
%\textsuperscript{(614.2)}
\textsuperscript{54:1.7} Incluso la sabiduría sólo es divina y digna de confianza cuando tiene un alcance cósmico y una motivación espiritual.

\par
%\textsuperscript{(614.3)}
\textsuperscript{54:1.8} No existe un error más grande que esa especie de autoengaño que conduce a los seres inteligentes a anhelar ejercer el poder sobre otros seres con el objeto de privar a esas personas de sus libertades naturales. La regla de oro de la equidad humana clama contra todos estos fraudes, injusticias, egoísmos y faltas de rectitud. Sólo una libertad verdadera y auténtica es compatible con el reino del amor y el ministerio de la misericordia.

\par
%\textsuperscript{(614.4)}
\textsuperscript{54:1.9} ¡Cómo se atreve la criatura obstinada a usurpar los derechos de sus semejantes en nombre de la libertad personal, cuando los Gobernantes Supremos del universo se apartan con un respeto misericordioso ante estas prerrogativas de la voluntad y estos potenciales de la personalidad! En el ejercicio de su supuesta libertad personal, ningún ser tiene el derecho de privar a otro ser de aquellos privilegios de la existencia otorgados por los Creadores y debidamente respetados por todos sus asociados, subordinados y sujetos leales.

\par
%\textsuperscript{(614.5)}
\textsuperscript{54:1.10} El hombre evolutivo quizás tenga que luchar por sus libertades materiales contra los tiranos y los opresores en un mundo de pecado y de iniquidad, o durante los primeros tiempos de una esfera primitiva en evolución, pero esto no es así en los mundos morontiales ni en las esferas espirituales. La guerra es la herencia del hombre evolutivo primitivo, pero en los mundos donde la civilización progresa de manera normal, hace mucho tiempo que el combate físico, como técnica para ajustar los malentendidos raciales, ha caído en desprestigio.

\section*{2. El robo de la libertad}
\par
%\textsuperscript{(614.6)}
\textsuperscript{54:2.1} Dios proyectó el eterno Havona con el Hijo y en el Espíritu, y desde entonces ha prevalecido el arquetipo eterno de la participación coordinada en la creación ---el compartir. Este arquetipo del compartir es el diseño maestro para cada uno de los Hijos e Hijas de Dios que salen al espacio para emprender el intento de copiar en el tiempo el universo central de perfección eterna.

\par
%\textsuperscript{(614.7)}
\textsuperscript{54:2.2} Toda criatura de todo universo en evolución que aspira a hacer la voluntad del Padre está destinada a convertirse en la asociada de los Creadores espacio-temporales en esta magnífica aventura de alcanzar la perfección por experiencia. Si esto no fuera así, el Padre difícilmente habría dotado a tales criaturas del libre albedrío creativo, y tampoco habitaría en ellas, llegando a asociarse realmente con ellas por medio de su propio espíritu.

\par
%\textsuperscript{(614.8)}
\textsuperscript{54:2.3} La locura de Lucifer consistió en intentar hacer lo irrealizable: saltarse el tiempo en un universo experiencial. El crimen de Lucifer consistió en intentar privar a todas las personalidades de Satania de sus derechos creativos, de reducir sin reconocerlo la participación personal de las criaturas ---la libre participación voluntaria--- en la larga lucha evolutiva por alcanzar el estado de luz y de vida de manera tanto individual como colectiva. Al hacer esto, este antiguo Soberano de vuestro sistema colocó el proyecto temporal de su propia voluntad directamente en contra del proyecto eterno de la voluntad de Dios tal como está revelado en la concesión del libre albedrío a todas las criaturas personales. La rebelión de Lucifer amenazaba así con violar de manera suprema la elección del libre albedrío de los ascendentes y de los servidores del sistema de Satania ---la amenaza de privar para siempre jamás a cada uno de estos seres de la experiencia emocionante de contribuir con algo personal y único al monumento que se levanta lentamente a la sabiduría experiencial y que algún día existirá bajo la forma del sistema perfeccionado de Satania. Así pues, el manifiesto de Lucifer, disfrazado con los vestidos de la libertad, se presentaba a la clara luz de la razón como una amenaza monumental destinada a consumar el robo de la libertad personal, y realizarlo a una escala a la que sólo nos habíamos acercado dos veces en toda la historia de Nebadon.

\par
%\textsuperscript{(615.1)}
\textsuperscript{54:2.4} En resumen, Lucifer habría quitado a los hombres y a los ángeles aquello que Dios les había dado, es decir el privilegio divino de participar en la creación de sus propios destinos y del destino de este sistema local de mundos habitados.

\par
%\textsuperscript{(615.2)}
\textsuperscript{54:2.5} Ningún ser en todo el universo tiene la legítima libertad de privar a otro ser de la verdadera libertad, del derecho de amar y de ser amado, del privilegio de adorar a Dios y de servir a sus semejantes.

\section*{3. La demora de la justicia}
\par
%\textsuperscript{(615.3)}
\textsuperscript{54:3.1} Las criaturas volitivas morales de los mundos evolutivos siempre están preocupadas por la pregunta irreflexiva de saber por qué los Creadores omnisapientes permiten el mal y el pecado. No logran comprender que los dos son inevitables si la criatura ha de ser realmente libre. El libre albedrío de los hombres evolutivos o de los ángeles exquisitos no es un simple concepto filosófico, un ideal simbólico. La capacidad del hombre para elegir el bien o el mal es una realidad en el universo. Esta libertad de elegir por sí mismo es un don de los Gobernantes Supremos, y éstos no permitirán que ningún ser o grupo de seres prive a una sola personalidad del inmenso universo de esta libertad divinamente concedida ---ni siquiera para satisfacer a aquellos seres descaminados e ignorantes en el disfrute de esta mal llamada libertad personal.

\par
%\textsuperscript{(615.4)}
\textsuperscript{54:3.2} Aunque la identificación consciente e incondicional con el mal
(con el pecado) es equivalente a la no existencia (a la aniquilación), entre el momento de esta identificación personal con el pecado y la ejecución del castigo ---resultado automático por haber abrazado deliberadamente el mal--- siempre debe transcurrir un período de tiempo lo suficientemente largo como para permitir que el juicio del estado universal de dicho individuo resulte ser enteramente satisfactorio para todas las personalidades universales relacionadas con el caso, y que sea tan justo y equitativo como para conseguir la aprobación del pecador mismo.

\par
%\textsuperscript{(615.5)}
\textsuperscript{54:3.3} Pero si este rebelde del universo que está en contra de la realidad de la verdad y de la bondad se niega a aprobar el veredicto, y si el culpable reconoce en su corazón la justicia de su condena pero rehúsa confesarla, entonces la ejecución de la sentencia debe ser aplazada de acuerdo con el criterio de los Ancianos de los Días. Y los Ancianos de los Días se niegan a aniquilar a un ser hasta que todos los valores morales y todas las realidades espirituales no se hayan extinguido tanto en el malhechor como en todos sus partidarios relacionados y en sus posibles simpatizantes.

\section*{4. El intervalo de la misericordia}
\par
%\textsuperscript{(615.6)}
\textsuperscript{54:4.1} Otro problema un poco difícil de explicar en la constelación de Norlatiadek es el referente a las razones por las que se permitió que Lucifer, Satanás y los príncipes caídos sembraran la discordia durante tanto tiempo antes de ser detenidos, internados y juzgados.

\par
%\textsuperscript{(616.1)}
\textsuperscript{54:4.2} Los padres, aquellos que han tenido y criado hijos, son capaces de comprender mejor por qué Miguel, un Creador-padre, puede ser lento en condenar y destruir a sus propios Hijos. La historia del hijo pródigo\footnote{\textit{El hijo pródigo}: Lc 15:11-32.} narrada por Jesús ilustra muy bien la manera en que un padre amoroso puede esperar mucho tiempo el arrepentimiento de su hijo equivocado.

\par
%\textsuperscript{(616.2)}
\textsuperscript{54:4.3} El hecho mismo de que una criatura malvada pueda elegir realmente hacer el mal ---cometer el pecado--- establece el hecho del libre albedrío y justifica plenamente cualquier largo retraso en la ejecución de la justicia, con tal que la misericordia facilitada pueda conducir al arrepentimiento y a la rehabilitación.

\par
%\textsuperscript{(616.3)}
\textsuperscript{54:4.4} Lucifer ya poseía la mayor parte de las libertades que buscaba; y otras las iba a recibir en el futuro. Todos estos preciosos dones se perdieron por ceder el paso a la impaciencia y por entregarse al deseo de poseer lo que uno anhela ahora, y poseerlo despreciando toda obligación de respetar los derechos y las libertades de todos los demás seres que componen el universo de universos. Las obligaciones éticas son innatas, divinas y universales.

\par
%\textsuperscript{(616.4)}
\textsuperscript{54:4.5} Conocemos muchas razones por las cuales los Gobernantes Supremos no destruyeron o internaron de inmediato a los cabecillas de la rebelión de Lucifer. No hay duda de que aún existen otras razones posiblemente mejores que nosotros no conocemos. Miguel de Nebadon facilitó personalmente las características misericordiosas de esta demora en la ejecución de la justicia. Si no hubiera sido por el afecto de este Creador-padre por sus Hijos equivocados, la justicia suprema del superuniverso habría actuado. Si un episodio como el de la rebelión de Lucifer hubiera ocurrido en Nebadon mientras Miguel estaba encarnado en Urantia, los instigadores de un mal así podrían haber sido aniquilados de manera instantánea y absoluta.

\par
%\textsuperscript{(616.5)}
\textsuperscript{54:4.6} La justicia suprema puede actuar instantáneamente cuando no está refrenada por la misericordia divina. Pero el ministerio de la misericordia para con los hijos del tiempo y del espacio asegura siempre esta demora temporal, este intervalo salvador entre la siembra y la cosecha. Si la siembra es buena, este intervalo asegura la puesta a prueba y la construcción del carácter; si la siembra es mala, esta demora misericordiosa proporciona tiempo para el arrepentimiento y la rectificación. Este aplazamiento temporal del juicio y de la ejecución de los malhechores es inherente al ministerio de misericordia de los siete superuniversos. Este freno de la misericordia sobre la justicia prueba que Dios es amor\footnote{\textit{Dios es amor}: 1 Jn 4:8,16.}, y que este Dios de amor domina los universos y controla con misericordia el destino y el juicio de todas sus criaturas.

\par
%\textsuperscript{(616.6)}
\textsuperscript{54:4.7} Las demoras temporales de la misericordia se conceden por mandato del libre albedrío de los Creadores. El universo puede obtener un bien de esta técnica de paciencia que se utiliza con los rebeldes pecadores. Aunque es demasiado cierto que el bien no puede provenir del mal para aquel que proyecta y que realiza el mal, es igualmente cierto que todas las cosas (incluyendo el mal, potencial o manifestado) trabajan juntas para el bien\footnote{\textit{Todas las cosas trabajan para el bien}: Ro 8:28; Heb 12:5-11; Ap 3:19.} de todos los seres que conocen a Dios, aman hacer su voluntad y ascienden hacia el Paraíso de acuerdo con su plan eterno y su propósito divino.

\par
%\textsuperscript{(616.7)}
\textsuperscript{54:4.8} Pero estas demoras de la misericordia no son interminables. A pesar del largo retraso en juzgarse la rebelión de Lucifer (tal como se calcula el tiempo en Urantia), podemos indicar que durante el período de efectuar esta revelación se ha celebrado en Uversa la primera audiencia del caso pendiente de Gabriel \textit{contra} Lucifer, y poco después se ha promulgado un mandato de los Ancianos de los Días ordenando que Satanás sea confinado de ahora en adelante en el mundo prisión con Lucifer. Esto pone fin a la capacidad de Satanás para continuar haciendo visitas a cualquiera de los mundos caídos de Satania. En un universo dominado por la misericordia, la justicia puede ser lenta, pero es segura.

\section*{5. La sabiduría de la demora}
\par
%\textsuperscript{(617.1)}
\textsuperscript{54:5.1} Entre las muchas razones que conozco por las cuales Lucifer y sus cómplices no fueron internados ni juzgados más pronto, se me permite enumerar las siguientes:

\par
%\textsuperscript{(617.2)}
\textsuperscript{54:5.2} 1. La misericordia exige que todo malhechor tenga tiempo suficiente para formular una actitud deliberada y plenamente elegida en lo que se refiere a sus malos pensamientos y a sus actos pecaminosos.

\par
%\textsuperscript{(617.3)}
\textsuperscript{54:5.3} 2. La justicia suprema está dominada por el amor de un Padre; por eso la justicia nunca destruirá aquello que la misericordia puede salvar. A todo malhechor se le concede tiempo para que acepte la salvación.

\par
%\textsuperscript{(617.4)}
\textsuperscript{54:5.4} 3. Ningún padre afectuoso se apresura nunca a infligir un castigo a un miembro equivocado de su familia. La paciencia no puede funcionar con independencia del tiempo.

\par
%\textsuperscript{(617.5)}
\textsuperscript{54:5.5} 4. Aunque la maldad siempre es perjudicial para una familia, la sabiduría y el amor exhortan a los hijos honrados a tener paciencia con un hermano equivocado durante el tiempo concedido por el padre afectuoso para que el pecador pueda ver el error de su conducta y abrazar la salvación.

\par
%\textsuperscript{(617.6)}
\textsuperscript{54:5.6} 5. Sin tener en cuenta la actitud de Miguel hacia Lucifer, a pesar de ser el Creador-padre de Lucifer, al Hijo Creador no le incumbía ejercer una jurisdicción sumaria sobre el Soberano apóstata del Sistema porque en aquella época no había terminado su carrera donadora que le permitiría conseguir la soberanía incondicional sobre Nebadon.

\par
%\textsuperscript{(617.7)}
\textsuperscript{54:5.7} 6. Los Ancianos de los Días podían haber aniquilado inmediatamente a estos rebeldes, pero raras veces ejecutan a los malhechores sin haber escuchado plenamente su caso. En esta ocasión se negaron a anular las decisiones de Miguel.

\par
%\textsuperscript{(617.8)}
\textsuperscript{54:5.8} 7. Es evidente que Emmanuel aconsejó a Miguel que permaneciera apartado de los rebeldes y que permitiera que la rebelión siguiera su curso natural de autodestrucción. Y la sabiduría del Unión de los Días es el reflejo en el tiempo de la sabiduría unida de la Trinidad del Paraíso.

\par
%\textsuperscript{(617.9)}
\textsuperscript{54:5.9} 8. El Fiel de los Días que reside en Edentia aconsejó a los Padres de la Constelación que permitieran a los rebeldes tener el camino libre a fin de que toda simpatía por estos malhechores se desarraigara lo más pronto posible del corazón de todo ciudadano presente y futuro de Norlatiadek ---de toda criatura mortal, morontial o espiritual.

\par
%\textsuperscript{(617.10)}
\textsuperscript{54:5.10} 9. En Jerusem, el representante personal del Ejecutivo Supremo de Orvonton aconsejó a Gabriel que fomentara todo tipo de oportunidades para que cada criatura viviente madurara una decisión deliberada respecto a los asuntos incluidos en la Declaración de Libertad de Lucifer. Una vez planteadas las cuestiones de la rebelión, el consejero paradisiaco para situaciones de emergencia de Gabriel declaró que si esta oportunidad plena y libre no se daba a todas las criaturas de Norlatiadek, entonces la cuarentena del Paraíso contra todas estas criaturas posiblemente poco entusiastas y afectadas por las dudas se extendería, como medida de autoprotección, a toda la constelación. Para mantener abiertas las puertas de la ascensión hacia el Paraíso a los seres de Norlatiadek era necesario facilitar el desarrollo completo de la rebelión, y asegurar la plena definición de la actitud de todos los seres implicados de alguna manera en ella.

\par
%\textsuperscript{(617.11)}
\textsuperscript{54:5.11} 10. La Ministra Divina de Salvington emitió un mandato, su tercera proclamación independiente, ordenando que no se hiciera nada por curar a medias, suprimir cobardemente o esconder de otras maneras el horrible rostro de los rebeldes y de la rebelión. A las huestes angélicas se les indicó que trabajaran para que la expresión del pecado tuviera la oportunidad ilimitada de revelarse plenamente, siendo ésta la técnica más rápida para conseguir la curación perfecta y final de la plaga del mal y del pecado.

\par
%\textsuperscript{(618.1)}
\textsuperscript{54:5.12} 11. En Jerusem se organizó un consejo de emergencia de ex-mortales compuesto de Mensajeros Poderosos, mortales glorificados que habían tenido una experiencia personal en situaciones semejantes, junto con sus colegas. Informaron a Gabriel que si se intentaban métodos de represión arbitrarios o sumarios, al menos un número tres veces mayor de seres se descarriarían. Todo el cuerpo de consejeros de Uversa coincidió en aconsejar a Gabriel que permitiera que la rebelión siguiera plenamente su curso natural, aunque se necesitara un millón de años para acabar con las consecuencias.

\par
%\textsuperscript{(618.2)}
\textsuperscript{54:5.13} 12. El tiempo, incluso en un universo temporal, es relativo: si un mortal de Urantia con una vida de duración media cometiera un crimen que provocara un pandemonio mundial, y si fuera detenido, juzgado y ejecutado a los dos o tres días de haber perpetrado el crimen, ¿os parecería un tiempo muy largo? Y sin embargo, esta comparación es la más cercana teniendo en cuenta la duración de la vida de Lucifer, aunque su juicio, ya iniciado, no finalice hasta dentro de cien mil años de Urantia. Desde el punto de vista de Uversa, donde el litigio está pendiente, este período relativo de tiempo podría ser indicado diciendo que el crimen de Lucifer fue llevado a juicio a los dos segundos y medio de haberse cometido. Desde el punto de vista del Paraíso, el juicio es simultáneo con el acto.

\par
%\textsuperscript{(618.3)}
\textsuperscript{54:5.14} Vosotros comprenderíais parcialmente un número equivalente de razones para no detener arbitrariamente la rebelión de Lucifer, pero no me está permitido indicarlas. Puedo informaros que en Uversa enseñamos cuarenta y ocho razones para permitir que el mal siga plenamente el curso de su propia ruina moral y extinción espiritual. No dudo de que habrá otras tantas razones adicionales que no conozco.

\section*{6. El triunfo del amor}
\par
%\textsuperscript{(618.4)}
\textsuperscript{54:6.1} Cualesquiera que sean las dificultades que los mortales evolutivos puedan encontrar en sus esfuerzos por comprender la rebelión de Lucifer, debería estar claro para todos los pensadores reflexivos que la técnica utilizada para tratar a los rebeldes es una confirmación del amor divino. La misericordia amorosa concedida a los rebeldes parece haber metido a muchos seres inocentes en dificultades y tribulaciones, pero todas estas personalidades trastornadas pueden confiar con seguridad en que los Jueces omnisapientes juzgarán sus destinos con misericordia así como con justicia.

\par
%\textsuperscript{(618.5)}
\textsuperscript{54:6.2} En todas sus relaciones con los seres inteligentes, tanto el Hijo Creador como su Padre Paradisiaco están dominados por el amor. Es imposible comprender muchas fases de la actitud de los gobernantes del universo hacia los rebeldes y la rebelión ---hacia el pecado y los pecadores--- a menos que se recuerde que Dios como Padre tiene prioridad sobre todas las otras fases de la manifestación de la Deidad en todas las relaciones de la divinidad con la humanidad. También se debería recordar que todos los Hijos Creadores Paradisiacos están motivados por la misericordia.

\par
%\textsuperscript{(618.6)}
\textsuperscript{54:6.3} Si el padre afectuoso de una gran familia elige mostrar misericordia a uno de sus hijos culpable de un grave delito, puede suceder muy bien que la concesión de misericordia a ese hijo que se ha portado mal cause dificultades temporales a todos los otros hijos que se han portado bien. Estas eventualidades son inevitables; este riesgo es inseparable de la situación real de tener un padre amoroso y de ser miembro de un grupo familiar. Cada miembro de una familia se beneficia de la conducta honrada de todos los otros miembros; del mismo modo, cada miembro ha de sufrir las consecuencias temporales inmediatas de la mala conducta de cualquier otro miembro. Las familias, los grupos, las naciones, las razas, los mundos, los sistemas, las constelaciones y los universos son relaciones de asociación que poseen una individualidad; y por lo tanto, cada miembro de cualquier grupo, grande o pequeño, cosecha los beneficios y sufre las consecuencias del bien y del mal que hacen todos los otros miembros del grupo interesado\footnote{\textit{Impacto social del mal}: Ro 12:5; 1 Co 10:17; 1 Co 12:12-27; Ef 4:25.}.

\par
%\textsuperscript{(619.1)}
\textsuperscript{54:6.4} Pero hay una cosa que debe quedar clara: si llegáis a sufrir las consecuencias funestas del pecado de algún miembro de vuestra familia, de algún conciudadano o de algún compañero humano, e incluso de una rebelión en el sistema o en otra parte ---cualquiera que sea lo que tengáis que soportar debido a la maldad de vuestros asociados, compañeros o superiores--- podéis confiar en la certidumbre eterna de que esas tribulaciones son aflicciones transitorias. Ninguna de estas consecuencias fraternales de la mala conducta en el grupo puede poner nunca en peligro vuestras perspectivas eternas ni privaros en lo más mínimo de vuestro derecho divino a ascender al Paraíso y alcanzar a Dios.

\par
%\textsuperscript{(619.2)}
\textsuperscript{54:6.5} Existen compensaciones para estas pruebas, demoras y decepciones que acompañan invariablemente al pecado de rebelión. Entre las muchas repercusiones valiosas de la rebelión de Lucifer que se podrían mencionar, sólo llamaré vuestra atención sobre el mejoramiento de las carreras de aquellos ascendentes mortales, ciudadanos de Jerusem, que por resistirse a los sofismas del pecado se colocaron en la vía de convertirse en futuros Mensajeros Poderosos, en compañeros de mi propia orden. Todo ser que pasó la prueba de este episodio nefasto, elevó inmediatamente de ese modo su estatus administrativo y acrecentó su valía espiritual.

\par
%\textsuperscript{(619.3)}
\textsuperscript{54:6.6} Al principio, la sublevación de Lucifer pareció ser una calamidad absoluta para el sistema y para el universo. Gradualmente, los beneficios empezaron a acumularse. Con el paso de veinticinco mil años del tiempo del sistema (veinte mil años del tiempo de Urantia), los Melquisedeks empezaron a enseñar que el bien resultante de la locura de Lucifer había llegado a igualar el mal que se había sufrido. La suma del mal se había quedado en aquel momento casi inmóvil, sólo continuaba creciendo en ciertos mundos aislados, mientras que las repercusiones beneficiosas continuaban multiplicándose y extendiéndose por el universo y el superuniverso, e incluso hasta Havona. Los Melquisedeks enseñan ahora que el bien resultante de la rebelión de Satania equivale a más de mil veces la suma de todo el mal.

\par
%\textsuperscript{(619.4)}
\textsuperscript{54:6.7} Pero una cosecha tan extraordinaria y tan benéfica procedente de la maldad sólo se podía conseguir gracias a la actitud sabia, divina y misericordiosa de todos los superiores de Lucifer, desde los Padres de la Constelación en Edentia hasta el Padre Universal en el Paraíso. El paso del tiempo ha acrecentado el bien indirecto que se puede obtener de la locura de Lucifer; y puesto que el mal a castigar se había desarrollado por completo en un período de tiempo relativamente corto, es evidente que los gobernantes omnisapientes y clarividentes del universo prolongarían ciertamente el plazo de tiempo para cosechar unos resultados cada vez más beneficiosos. Sin tener en cuenta las numerosas razones adicionales para retrasar la detención y el juicio de los rebeldes de Satania, este beneficio por sí solo hubiera sido suficiente para explicar por qué estos pecadores no fueron internados antes y por qué no han sido juzgados y destruidos.

\par
%\textsuperscript{(619.5)}
\textsuperscript{54:6.8} La mente humana, corta de miras y atada al tiempo, debería ser lenta en criticar las demoras temporales concedidas por los administradores clarividentes y omnisapientes de los asuntos del universo.

\par
%\textsuperscript{(620.1)}
\textsuperscript{54:6.9} Uno de los errores del pensamiento humano con respecto a estos problemas consiste en la idea de que todos los mortales evolutivos de un planeta en evolución hubieran elegido emprender la carrera hacia el Paraíso si el pecado no hubiera maldecido su mundo. La capacidad para rechazar la supervivencia no data de los tiempos de la rebelión de Lucifer. El hombre mortal siempre ha poseído el don de la libre elección en cuanto a la carrera hacia el Paraíso.

\par
%\textsuperscript{(620.2)}
\textsuperscript{54:6.10} A medida que ascendáis en la experiencia de la supervivencia, ampliaréis vuestros conceptos sobre el universo y extenderéis vuestro horizonte de significados y de valores; y así seréis capaces de comprender mejor por qué se permite a unos seres como Lucifer y Satanás continuar con su rebelión. También comprenderéis mejor cómo se puede obtener un bien último (si no inmediato) de un mal limitado en el tiempo. Después de que alcancéis el Paraíso, os sentiréis realmente iluminados y confortados cuando escuchéis a los filósofos superáficos discutir y explicar estos profundos problemas de adaptación universal. Pero incluso entonces dudo de que estéis plenamente satisfechos en vuestra propia mente. Al menos yo no lo estuve, ni siquiera cuando hube alcanzado así la cima de la filosofía universal. No conseguí comprender plenamente estas complejidades hasta después de ser destinado a las funciones administrativas del superuniverso, donde adquirí por medio de la experiencia real la capacidad conceptual adecuada para comprender estos complejos problemas con equidad cósmica y con filosofía espiritual. A medida que ascendáis hacia el Paraíso, aprenderéis cada vez mejor que muchas características problemáticas de la administración universal sólo se pueden comprender después de adquirir una mayor capacidad experiencial y de conseguir una perspicacia espiritual elevada. La sabiduría cósmica es esencial para comprender las situaciones cósmicas.

\par
%\textsuperscript{(620.3)}
\textsuperscript{54:6.11} [Presentado por un Mensajero Poderoso que sobrevivió experiencialmente a la primera rebelión sistémica de los universos del tiempo, vinculado en la actualidad al gobierno superuniversal de Orvonton y que actúa en esta materia a petición de Gabriel de Salvington.]


\chapter{Documento 55. Las esferas de luz y de vida}
\par
%\textsuperscript{(621.1)}
\textsuperscript{55:0.1} LA ERA de luz y de vida es el logro evolutivo final de un mundo del tiempo y del espacio. Desde los primeros tiempos del hombre primitivo, ese mundo habitado ha pasado por las eras planetarias sucesivas ---la era anterior y posterior al Príncipe Planetario, la era postadámica, la era posterior al Hijo Magistral y la era posterior al Hijo donador. Luego ese mundo es preparado para el logro evolutivo culminante, para el estado permanente de luz y de vida, mediante el ministerio de las misiones planetarias sucesivas de los Hijos Instructores Trinitarios, con sus revelaciones crecientes sobre la verdad divina y la sabiduría cósmica. En estos esfuerzos por establecer la era planetaria final, los Hijos Instructores disfrutan siempre de la ayuda de las Brillantes Estrellas Vespertinas y a veces de los Melquisedeks.

\par
%\textsuperscript{(621.2)}
\textsuperscript{55:0.2} Esta era de luz y de vida, inaugurada por los Hijos Instructores al concluir su misión planetaria final, continúa indefinidamente en los mundos habitados. Las acciones judiciales de los Hijos Magistrales pueden dividir cada etapa progresiva de este estado asentado en una sucesión de dispensaciones; pero todas estas acciones judiciales son puramente técnicas y no modifican de ninguna manera el curso de los acontecimientos planetarios.

\par
%\textsuperscript{(621.3)}
\textsuperscript{55:0.3} Sólo aquellos planetas que consiguen existir en los circuitos principales del superuniverso tienen asegurada la supervivencia continua, pero por lo que sabemos, estos mundos establecidos en la luz y la vida están destinados a seguir su camino durante las eras eternas de todos los tiempos futuros.

\par
%\textsuperscript{(621.4)}
\textsuperscript{55:0.4} El desarrollo de la era de luz y de vida en un mundo evolutivo consta de siete etapas, y a este respecto se debe tener en cuenta que los mundos de los mortales que fusionan con el Espíritu evolucionan de idéntica manera a los de las series que fusionan con el Ajustador. Estas siete etapas de luz y de vida son las siguientes:

\par
%\textsuperscript{(621.5)}
\textsuperscript{55:0.5} 1. La primera etapa o etapa planetaria.

\par
%\textsuperscript{(621.6)}
\textsuperscript{55:0.6} 2. La segunda etapa o etapa del sistema.

\par
%\textsuperscript{(621.7)}
\textsuperscript{55:0.7} 3. La tercera etapa o etapa de la constelación.

\par
%\textsuperscript{(621.8)}
\textsuperscript{55:0.8} 4. La cuarta etapa o etapa del universo local.

\par
%\textsuperscript{(621.9)}
\textsuperscript{55:0.9} 5. La quinta etapa o etapa del sector menor.

\par
%\textsuperscript{(621.10)}
\textsuperscript{55:0.10} 6. La sexta etapa o etapa del sector mayor.

\par
%\textsuperscript{(621.11)}
\textsuperscript{55:0.11} 7. La séptima etapa o etapa del superuniverso.

\par
%\textsuperscript{(621.12)}
\textsuperscript{55:0.12} Al final de esta narración, estas etapas de desarrollo progresivo se describen según sea su relación con la organización del universo, pero cualquier mundo puede alcanzar los valores planetarios de cualquier etapa, independientemente por completo del desarrollo de otros mundos o de los niveles superplanetarios de la administración del universo.

\section*{1. El templo morontial}
\par
%\textsuperscript{(622.1)}
\textsuperscript{55:1.1} La presencia de un templo morontial en la capital de un mundo habitado es el certificado de la admisión de esa esfera en las épocas estables de luz y de vida. Antes de que los Hijos Instructores dejen un mundo después de concluir su misión terminal, inauguran esta época final de logros evolutivos; presiden el día en que <<el templo sagrado desciende sobre el mundo>>. Este acontecimiento, que señala los albores de la era de luz y de vida, siempre se ve honrado con la presencia personal del Hijo Paradisiaco donador de ese planeta, que viene a presenciar este gran día. Aquí, en este templo de una belleza incomparable, este Hijo donador del Paraíso proclama al que ha sido tanto tiempo Príncipe Planetario nuevo Soberano Planetario, y confiere a este fiel Hijo Lanonandek nuevos poderes y una mayor autoridad sobre los asuntos planetarios. El Soberano del Sistema también está presente y toma la palabra para confirmar estas declaraciones.

\par
%\textsuperscript{(622.2)}
\textsuperscript{55:1.2} Un templo morontial tiene tres partes: en el centro está el santuario del Hijo Paradisiaco donador. A la derecha se encuentra el asiento del antiguo Príncipe Planetario, ahora Soberano Planetario; y cuando este Hijo Lanonandek está presente en el templo, es visible para los individuos más espirituales del reino. A la izquierda se encuentra el asiento del jefe en funciones de los finalitarios vinculados al planeta.

\par
%\textsuperscript{(622.3)}
\textsuperscript{55:1.3} Aunque se ha dicho que los templos planetarios <<descienden del cielo>>, en realidad no se transporta ningún material concreto desde la sede del sistema. La arquitectura de cada uno de ellos se elabora en miniatura en la capital del sistema, y los Supervisores del Poder Morontial traen posteriormente estos planes aprobados al planeta. Aquí, en asociación con los Controladores Físicos Maestros, proceden a construir el templo morontial de acuerdo con las especificaciones.

\par
%\textsuperscript{(622.4)}
\textsuperscript{55:1.4} Un templo morontial de tipo medio tiene capacidad para unos trescientos mil espectadores. Estos edificios no se utilizan para la adoración, ni para el entretenimiento ni para recibir las transmisiones; están dedicados a las ceremonias especiales del planeta tales como: comunicaciones con el Soberano del Sistema o con los Altísimos, ceremonias especiales de visualización destinadas a revelar la presencia de la personalidad de los seres espirituales, y contemplación cósmica silenciosa. Las escuelas de filosofía cósmica dirigen aquí sus ejercicios de graduación, y los mortales del reino también reciben aquí el reconocimiento planetario por haber efectuado importantes servicios sociales y por otros logros sobresalientes.

\par
%\textsuperscript{(622.5)}
\textsuperscript{55:1.5} Un templo morontial de esta clase sirve también como lugar de reunión para presenciar el traslado de los mortales vivientes a la existencia morontial. El templo para los traslados está compuesto de materiales morontiales, y por eso no se destruye con la gloria resplandeciente del fuego arrollador que deshace por completo los cuerpos físicos de aquellos mortales que experimentan allí la fusión final con su Ajustador divino. En un mundo grande estas llamaradas de partida son casi continuas, y a medida que crece el número de traslados se habilitan santuarios auxiliares de vida morontial en diferentes zonas del planeta. No hace mucho tiempo residí en un mundo situado muy al norte donde funcionaban veinticinco santuarios morontiales.

\par
%\textsuperscript{(622.6)}
\textsuperscript{55:1.6} En los mundos aún no establecidos, en los planetas sin templos morontiales, estos destellos de la fusión se producen muchas veces en la atmósfera planetaria, donde el cuerpo material de un candidato al traslado es elevado por las criaturas intermedias y los controladores físicos.

\section*{2. La muerte y el traslado}
\par
%\textsuperscript{(623.1)}
\textsuperscript{55:2.1} La muerte física natural no es una inevitabilidad para los humanos. La mayoría de los seres evolutivos avanzados, los ciudadanos de los mundos que existen en la era final de luz y de vida, no mueren; son trasladados directamente de la vida en la carne a la existencia morontial.

\par
%\textsuperscript{(623.2)}
\textsuperscript{55:2.2} La frecuencia de esta experiencia de traslado de la vida material al estado morontial ---la fusión del alma inmortal con el Ajustador interior--- crece de manera proporcional al progreso evolutivo del planeta. Al principio, sólo algunos mortales de cada era alcanzan los niveles de progreso espiritual que permiten el traslado, pero con la llegada de las épocas sucesivas de los Hijos Instructores, se producen cada vez más fusiones con el Ajustador antes de finalizar la vida, cada vez más larga, de estos mortales que progresan; y en la época de la misión final de los Hijos Instructores, aproximadamente una cuarta parte de estos magníficos mortales está exenta de la muerte natural.

\par
%\textsuperscript{(623.3)}
\textsuperscript{55:2.3} Más adelante aún durante la era de luz y de vida, las criaturas intermedias o sus asociados perciben que se acerca el estado en que un alma puede probablemente unirse con su Ajustador y señalan este hecho a los guardianes del destino, los cuales comunican a su vez esta cuestión al grupo finalitario bajo cuya jurisdicción puede estar trabajando ese mortal; entonces el Soberano Planetario emite un llamamiento para que ese mortal renuncie a todas sus funciones planetarias, se despida de su mundo de origen y acuda al templo interior del Soberano Planetario para esperar allí el tránsito morontial, el destello del traslado, entre el ámbito material de evolución y el nivel morontial de progresión preespiritual.

\par
%\textsuperscript{(623.4)}
\textsuperscript{55:2.4} Cuando la familia, los amigos y el grupo de trabajo de ese candidato a la fusión se han congregado en el templo morontial, se distribuyen alrededor del escenario central donde descansan los candidatos a la fusión mientras conversan libremente con sus amigos reunidos. Un círculo intermedio de personalidades celestiales se forma para proteger a los mortales materiales de la acción de las energías que se manifiestan en el instante del <<destello de vida>>, el cual libera al candidato a la ascensión de las cadenas de la carne material y hace por ese mortal evolutivo todo lo que hace la muerte natural por aquellos que libera de la carne.

\par
%\textsuperscript{(623.5)}
\textsuperscript{55:2.5} Muchos candidatos a la fusión pueden estar reunidos al mismo tiempo en el amplio templo. ¡Qué hermoso acontecimiento cuando los mortales se reúnen así para presenciar la ascensión de sus seres queridos en las llamas espirituales, y qué contraste con las épocas anteriores en que los mortales tenían que entregar a sus muertos al abrazo de los elementos terrestres! Las escenas de llantos y de lamentos, características de las épocas primitivas de la evolución humana, son reemplazadas ahora por una alegría extática y por el entusiasmo más sublime cuando estos mortales que conocen a Dios se despiden temporalmente de sus seres queridos mientras son apartados de sus asociaciones materiales por los fuegos espirituales de una grandiosidad arrolladora y de una gloria ascendente. En los mundos establecidos en la luz y la vida, los <<funerales>> son ocasiones en que se experimenta una alegría suprema, una satisfacción profunda y una esperanza inexpresable.

\par
%\textsuperscript{(623.6)}
\textsuperscript{55:2.6} Las almas de estos mortales que progresan están cada vez más llenas de fe, de esperanza y de seguridad. El estado de ánimo que impregna a aquellos que se encuentran reunidos alrededor del santuario de traslado se parece al de unos amigos y parientes alegres que se hubieran reunido para celebrar la graduación de un miembro de su grupo, o que se hubieran congregado para presenciar la concesión de un gran honor a uno de los suyos. Y sería decididamente beneficioso que los mortales menos avanzados pudieran aprender a considerar la muerte natural con un poco de esta misma alegría y desenfado.

\par
%\textsuperscript{(624.1)}
\textsuperscript{55:2.7} Los observadores mortales no pueden ver nada de sus asociados trasladados después del destello de la fusión. Estas almas trasladadas se dirigen directamente, por tránsito de Ajustador, a la sala de resurrección del mundo apropiado de educación morontial. Estas operaciones relacionadas con el traslado de los seres humanos vivientes al mundo morontial están supervisadas por un arcángel que fue destinado a ese mundo el día que se estableció por primera vez en la luz y la vida.

\par
%\textsuperscript{(624.2)}
\textsuperscript{55:2.8} Cuando un mundo llega a la cuarta etapa de luz y de vida, más de la mitad de los mortales dejan el planeta por traslado de entre los vivos. Esta disminución de la muerte continúa sin cesar, pero no conozco ningún sistema cuyos mundos habitados, aunque lleven establecidos mucho tiempo en la vida, estén totalmente libres de la muerte natural como técnica para escapar de las cadenas de la carne. Hasta que este estado superior de evolución planetaria no se alcance de manera uniforme, los mundos de formación morontial del universo local deberán continuar sirviendo como esferas educativas y culturales para los progresores morontiales en evolución. La eliminación de la muerte es teóricamente posible, pero según mis observaciones, aún no se ha producido. Quizás se pueda alcanzar este estado durante los períodos lejanos de las épocas sucesivas de la séptima etapa de la vida planetaria establecida.

\par
%\textsuperscript{(624.3)}
\textsuperscript{55:2.9} Las almas trasladadas durante las épocas florecientes de las esferas establecidas no pasan por los mundos de las mansiones. Tampoco se detienen, como estudiantes, en los mundos morontiales del sistema o de la constelación. No pasan por ninguna de las fases iniciales de la vida morontial. Son los únicos mortales ascendentes que casi llegan a eludir la transición morontial entre la existencia material y el estado semiespiritual. La experiencia inicial en la carrera ascendente de estos mortales \textit{asidos por el Hijo} tiene lugar en los servicios de los mundos de progresión de la sede del universo. Y desde estos mundos de estudio de Salvington, regresan como instructores a los mismos mundos que dejaron de lado, dirigiéndose posteriormente hacia el interior y el Paraíso por el camino establecido para la ascensión de los mortales.

\par
%\textsuperscript{(624.4)}
\textsuperscript{55:2.10} Si tan sólo pudierais visitar un planeta en un estado avanzado de desarrollo, captaríais rápidamente las razones por las cuales se permite la recepción diferencial de unos mortales ascendentes en los mundos de las mansiones y en los mundos morontiales superiores. Comprenderíais fácilmente que unos seres que proceden de unas esferas tan sumamente evolucionadas están preparados para reanudar su ascensión hacia el Paraíso mucho antes que el mortal de tipo medio que llega de un mundo desordenado y atrasado como Urantia.

\par
%\textsuperscript{(624.5)}
\textsuperscript{55:2.11} Cualquiera que sea el nivel de logro planetario con el que los seres humanos puedan ascender a los mundos morontiales, las siete esferas de las mansiones les proporcionan amplias oportunidades para adquirir por experiencia, como alumnos-maestros, todo aquello que dejaron de lado debido al estado avanzado de sus planetas nativos.

\par
%\textsuperscript{(624.6)}
\textsuperscript{55:2.12} El universo es infalible en la aplicación de estas técnicas igualizadoras destinadas a asegurar que ningún ascendente sea privado de nada esencial para su experiencia de ascensión.

\section*{3. Las edades de oro}
\par
%\textsuperscript{(624.7)}
\textsuperscript{55:3.1} Durante esta era de luz y de vida, el mundo prospera cada vez más bajo el gobierno paternal del Soberano Planetario. En esa época los mundos progresan bajo el impulso de un solo idioma, de una sola religión y, en las esferas normales, de una sola raza. Pero esta era no es perfecta. Estos mundos poseen todavía hospitales bien equipados, clínicas para cuidar a los enfermos. Aún subsisten los problemas de atender las lesiones accidentales y las enfermedades inevitables que acompañan a la decrepitud de la vejez y a los trastornos de la senilidad. La enfermedad no ha sido vencida por completo y los animales terrestres tampoco han sido sometidos perfectamente; pero estos mundos son como el Paraíso en comparación con los primeros tiempos del hombre primitivo durante la era anterior al Príncipe Planetario. Si pudierais ser transportados repentinamente a un planeta con este estado de desarrollo, describiríais instintivamente a este reino como el cielo en la Tierra.

\par
%\textsuperscript{(625.1)}
\textsuperscript{55:3.2} Durante toda esta era de progreso y de perfección relativos, el gobierno humano continúa funcionando para dirigir los asuntos materiales. En un mundo que visité recientemente y que se encuentra en la primera etapa de luz y de vida, las actividades públicas estaban financiadas mediante la técnica del diezmo. Cada trabajador adulto ---y todos los ciudadanos sanos trabajaban en algo--- pagaba el diez por ciento de sus ingresos o de sus aumentos al tesoro público, y era desembolsado como sigue:

\par
%\textsuperscript{(625.2)}
\textsuperscript{55:3.3} 1. El tres por ciento se empleaba para promover la verdad ---la ciencia, la educación y la filosofía.

\par
%\textsuperscript{(625.3)}
\textsuperscript{55:3.4} 2. El tres por ciento se consagraba a la belleza ---el entretenimiento, el ocio social y el arte.

\par
%\textsuperscript{(625.4)}
\textsuperscript{55:3.5} 3. El tres por ciento se dedicaba a la bondad ---el servicio social, el altruismo y la religión.

\par
%\textsuperscript{(625.5)}
\textsuperscript{55:3.6} 4. El uno por ciento estaba destinado a las reservas del seguro contra el riesgo de incapacidad para el trabajo, resultante de los accidentes, las enfermedades, la vejez o los desastres inevitables.

\par
%\textsuperscript{(625.6)}
\textsuperscript{55:3.7} Los recursos naturales de este planeta eran administrados como posesiones sociales, como bienes de la comunidad.

\par
%\textsuperscript{(625.7)}
\textsuperscript{55:3.8} En este mundo, el honor más elevado que se confería a un ciudadano era la orden del <<servicio supremo>>, el único título de reconocimiento que se concedía en el templo morontial. Este reconocimiento se otorgaba a aquellos que se habían distinguido durante mucho tiempo en alguna fase del descubrimiento supermaterial o del servicio social planetario.

\par
%\textsuperscript{(625.8)}
\textsuperscript{55:3.9} La mayoría de los cargos sociales y administrativos estaban ocupados conjuntamente por hombres y mujeres. La mayor parte de la enseñanza también se impartía conjuntamente; asimismo, todas las tareas de confianza judiciales eran desempeñadas por parejas asociadas similares.

\par
%\textsuperscript{(625.9)}
\textsuperscript{55:3.10} En estos mundos magníficos, el período de maternidad no es muy prolongado. No es conveniente que haya demasiada diferencia de edad entre los hijos de una familia. Cuando sus edades están más próximas, los niños pueden contribuir mucho más a su educación mutua. Y en estos mundos son magníficamente educados mediante sistemas competitivos de intensos esfuerzos en los ámbitos y divisiones avanzados donde se consiguen diversos logros en el dominio de la verdad, la belleza y la bondad. Pero no temáis, que incluso estas esferas glorificadas presentan una cantidad suficiente de mal, real y potencial, como para estimular la elección entre la verdad y el error, el bien y el mal, el pecado y la rectitud.

\par
%\textsuperscript{(625.10)}
\textsuperscript{55:3.11} Sin embargo, existe cierto precio inevitable a pagar ligado a la existencia humana en esos planetas evolutivos avanzados. Cuando un mundo establecido progresa más allá de la tercera etapa de luz y de vida, todos los ascendentes están destinados a recibir, antes de llegar al sector menor, algún tipo de misión transitoria en un planeta que está pasando por las primeras etapas de la evolución.

\par
%\textsuperscript{(626.1)}
\textsuperscript{55:3.12} Cada una de estas épocas sucesivas representa unas realizaciones más avanzadas en todas las fases de los logros planetarios. En la era inicial de luz, la revelación de la verdad se amplía hasta incluir el funcionamiento del universo de universos, mientras que el estudio de la Deidad durante la segunda era es un intento por dominar el concepto proteico de la naturaleza, la misión, el ministerio, las asociaciones, el origen y el destino de los Hijos Creadores, el primer nivel de Dios Séptuple.

\par
%\textsuperscript{(626.2)}
\textsuperscript{55:3.13} Cuando un planeta del tamaño de Urantia está bastante bien establecido, suele tener unos cien centros subadministrativos. Estos centros subordinados suelen estar presididos por uno de los grupos de administradores cualificados siguientes:

\par
%\textsuperscript{(626.3)}
\textsuperscript{55:3.14} 1. Los jóvenes Hijos e Hijas Materiales traídos desde la sede del sistema para actuar como asistentes del Adán y de la Eva gobernantes.

\par
%\textsuperscript{(626.4)}
\textsuperscript{55:3.15} 2. La progenie del estado mayor semimortal del Príncipe Planetario que fue procreada en ciertos mundos para estas responsabilidades y otras similares.

\par
%\textsuperscript{(626.5)}
\textsuperscript{55:3.16} 3. La progenie planetaria directa de Adán y Eva.

\par
%\textsuperscript{(626.6)}
\textsuperscript{55:3.17} 4. Las criaturas intermedias materializadas y humanizadas.

\par
%\textsuperscript{(626.7)}
\textsuperscript{55:3.18} 5. Los mortales en condiciones de fusionar con su Ajustador que, a petición propia y por orden del Ajustador Personalizado que tiene la jefatura en el universo, están exentos temporalmente de ser trasladados para que puedan continuar en el planeta en ciertos puestos administrativos importantes.

\par
%\textsuperscript{(626.8)}
\textsuperscript{55:3.19} 6. Los mortales especialmente capacitados de las escuelas planetarias de administración que también han merecido la orden del servicio supremo del templo morontial.

\par
%\textsuperscript{(626.9)}
\textsuperscript{55:3.20} 7. Ciertas comisiones electivas de tres ciudadanos adecuadamente cualificados que a veces son elegidos por el conjunto de los ciudadanos por mandato del Soberano Planetario de acuerdo con su capacidad especial para realizar alguna tarea determinada que es necesaria en ese sector planetario particular.

\par
%\textsuperscript{(626.10)}
\textsuperscript{55:3.21} El gran obstáculo que se le presenta a Urantia en el asunto de alcanzar el elevado destino planetario de la luz y la vida se encuentra en los problemas de la enfermedad, la degeneración, la guerra, las razas multicolores y el multiling\"uismo.

\par
%\textsuperscript{(626.11)}
\textsuperscript{55:3.22} Ningún mundo evolutivo puede esperar progresar más allá de la primera etapa del establecimiento en la luz hasta que no haya alcanzado el objetivo de un solo idioma, una sola religión y una sola filosofía. El hecho de poseer una sola raza facilita enormemente esta consecución, pero la existencia de numerosos pueblos en Urantia no impide que se alcancen unos estados más elevados.

\section*{4. Los reajustes administrativos}
\par
%\textsuperscript{(626.12)}
\textsuperscript{55:4.1} Durante las etapas sucesivas de la existencia establecida, los mundos habitados efectúan un progreso maravilloso bajo la administración sabia y comprensiva del Cuerpo voluntario de la Finalidad, los ascendentes que han alcanzado el Paraíso y que han regresado para ayudar a sus hermanos en la carne. Estos finalitarios cooperan activamente con los Hijos Instructores Trinitarios, pero no empiezan a participar realmente en los asuntos mundiales hasta que el templo morontial no aparece en el mundo.

\par
%\textsuperscript{(626.13)}
\textsuperscript{55:4.2} Tras el inicio oficial del ministerio planetario del Cuerpo de la Finalidad, la mayoría de las huestes celestiales se retiran. Pero los guardianes seráficos del destino continúan su ministerio personal hacia los mortales que progresan en la luz; estos ángeles llegan en verdad en cantidades crecientes durante las eras establecidas, puesto que grupos cada vez más grandes de seres humanos alcanzan el tercer círculo cósmico de consecución humana coordinada durante el período de la vida planetaria.

\par
%\textsuperscript{(627.1)}
\textsuperscript{55:4.3} Éste es simplemente el primero de los ajustes administrativos sucesivos que acompañan al desarrollo de las épocas sucesivas de logros cada vez más brillantes en los mundos habitados que van pasando de la primera a la séptima etapa de existencia establecida.

\par
%\textsuperscript{(627.2)}
\textsuperscript{55:4.4} 1. \textit{La primera etapa de luz y de vida}. Un mundo en esta etapa establecida inicial está administrado por tres gobernantes:

\par
%\textsuperscript{(627.3)}
\textsuperscript{55:4.5} a. El Soberano Planetario, ahora aconsejado por un Hijo Instructor Trinitario que lo asesora, con toda probabilidad el jefe del último cuerpo de estos Hijos que ejerció su actividad en el planeta.

\par
%\textsuperscript{(627.4)}
\textsuperscript{55:4.6} b. El jefe del cuerpo planetario de los finalitarios.

\par
%\textsuperscript{(627.5)}
\textsuperscript{55:4.7} c. Adán y Eva, que desempeñan conjuntamente su actividad como unificadores de la doble jefatura del Príncipe-Soberano y del jefe de los finalitarios.

\par
%\textsuperscript{(627.6)}
\textsuperscript{55:4.8} Las criaturas intermedias exaltadas y liberadas actúan como intérpretes para los guardianes seráficos y los finalitarios. Uno de los últimos actos de los Hijos Instructores Trinitarios durante su misión final consiste en liberar a los intermedios del reino y promoverlos (o restablecerlos) a un estado planetario avanzado, asignándolos a puestos de responsabilidad en la nueva administración de la esfera establecida. En el campo de la visión humana ya se han efectuado los cambios necesarios para permitir que los mortales reconozcan a estos primos hasta ahora invisibles del régimen adámico inicial. Esto llega a ser posible gracias a los descubrimientos finales de la ciencia física en unión con las funciones planetarias más extensas de los Controladores Físicos Maestros.

\par
%\textsuperscript{(627.7)}
\textsuperscript{55:4.9} El Soberano del Sistema tiene autoridad para liberar a las criaturas intermedias en cualquier momento después de la primera etapa establecida, para que puedan humanizarse en el nivel morontial con la ayuda de los Portadores de Vida y de los controladores físicos y, después de recibir sus Ajustadores del Pensamiento, empezar su ascensión hacia el Paraíso.

\par
%\textsuperscript{(627.8)}
\textsuperscript{55:4.10} En la tercera etapa y en las siguientes, algunos intermedios siguen ejerciendo su actividad principalmente como personalidades de contacto para los finalitarios, pero a medida que se inicia cada etapa de luz y de vida, nuevas órdenes de ministros de enlace reemplazan en gran parte a los intermedios; muy pocos de ellos quedan nunca más allá de la cuarta etapa de luz. La séptima etapa presenciará la llegada de los primeros ministros absonitos procedentes del Paraíso para servir en los puestos de ciertas criaturas del universo.

\par
%\textsuperscript{(627.9)}
\textsuperscript{55:4.11} 2. \textit{La segunda etapa de luz y de vida}. Esta época está señalada en los mundos por la llegada de un Portador de Vida que se convierte en el consejero voluntario de los gobernantes planetarios en lo referente a los esfuerzos adicionales por purificar y estabilizar la raza mortal. Los Portadores de Vida participan activamente así en la evolución ulterior de la raza humana ---física, social y económicamente. Luego extienden su supervisión a la purificación adicional del linaje mortal mediante la drástica eliminación de los restos atrasados subsistentes dotados de un potencial inferior en su naturaleza intelectual, filosófica, cósmica y espiritual. Aquellos que diseñan y plantan la vida en un mundo habitado son plenamente competentes para aconsejar a los Hijos y las Hijas Materiales, los cuales poseen una autoridad plena e indiscutible para purificar a la raza en evolución de todas las influencias perjudiciales.

\par
%\textsuperscript{(627.10)}
\textsuperscript{55:4.12} Desde la segunda etapa y durante toda la carrera de un planeta establecido, los Hijos Instructores sirven como consejeros de los finalitarios. Durante estas misiones sirven como voluntarios, no por asignación; y prestan su servicio exclusivamente al cuerpo finalitario, salvo que, con el consentimiento del Soberano del Sistema, el Adán y la Eva Planetarios pueden tenerlos como asesores.

\par
%\textsuperscript{(628.1)}
\textsuperscript{55:4.13} 3. \textit{La tercera etapa de luz y de vida}. Durante esta época, los mundos habitados llegan a una nueva apreciación de los Ancianos de los Días, la segunda fase de Dios Séptuple, y los representantes de estos gobernantes superuniversales emprenden nuevas relaciones con la administración planetaria.

\par
%\textsuperscript{(628.2)}
\textsuperscript{55:4.14} En cada época siguiente de existencia establecida, los finalitarios ejercen su actividad en funciones cada vez más amplias. Existe una estrecha relación de trabajo entre los finalitarios, las Estrellas Vespertinas (los superángeles) y los Hijos Instructores Trinitarios.

\par
%\textsuperscript{(628.3)}
\textsuperscript{55:4.15} Durante esta era o la siguiente, un Hijo Instructor, ayudado por el cuarteto de espíritus ministrantes, es atribuido al jefe ejecutivo humano electivo, el cual se convierte ahora en el asociado del Soberano Planetario como administrador conjunto de los asuntos del mundo. Estos jefes ejecutivos humanos sirven durante veinticinco años del tiempo planetario, y este nuevo desarrollo es el que facilita que el Adán y la Eva Planetarios consigan liberarse, durante las épocas siguientes, del mundo donde han estado tanto tiempo destinados.

\par
%\textsuperscript{(628.4)}
\textsuperscript{55:4.16} Los cuartetos de espíritus ministrantes están compuestos de: el jefe seráfico de la esfera, el consejero secoráfico del superuniverso, el arcángel de los traslados y el omniafín que actúa como representante personal del Centinela Asignado situado en la sede del sistema. Pero estos asesores nunca ofrecen su consejo a menos que se les pida.

\par
%\textsuperscript{(628.5)}
\textsuperscript{55:4.17} 4. \textit{La cuarta etapa de luz y de vida}. Los Hijos Instructores Trinitarios aparecen en los mundos con nuevas funciones. Ayudados por los hijos trinitizados por las criaturas asociados desde hace tanto tiempo con su orden, ahora llegan a los mundos como consejeros y asesores voluntarios del Soberano Planetario y de sus asociados. Estas parejas ---los hijos trinitizados del Paraíso-Havona y los hijos trinitizados por los ascendentes ---representan puntos de vista universales diferentes y experiencias personales diversas que son sumamente útiles para los gobernantes planetarios.

\par
%\textsuperscript{(628.6)}
\textsuperscript{55:4.18} En cualquier momento después de esta época, el Adán y la Eva Planetarios pueden solicitar al Hijo Creador Soberano que los libere de sus deberes planetarios a fin de empezar su ascensión hacia el Paraíso; o pueden permanecer en el planeta como directores del tipo de sociedad recién aparecido y cada vez más espiritual, compuesta de mortales avanzados que se esfuerzan por comprender las enseñanzas filosóficas de los finalitarios, descritas por las Brillantes Estrellas Vespertinas que están ahora destinadas en estos mundos para colaborar en parejas con los seconafines procedentes de la sede del superuniverso.

\par
%\textsuperscript{(628.7)}
\textsuperscript{55:4.19} Los finalitarios se dedican principalmente a iniciar las nuevas actividades supermateriales de la sociedad ---sociales, culturales, filosóficas, cósmicas y espirituales. Por lo que podemos discernir, continuarán efectuando este ministerio hasta muy entrada la séptima época de estabilidad evolutiva, cuando es posible que vayan a servir al espacio exterior; con lo cual suponemos que sus puestos pueden ser ocupados por seres absonitos procedentes del Paraíso.

\par
%\textsuperscript{(628.8)}
\textsuperscript{55:4.20} 5. \textit{La quinta etapa de luz y de vida}. Los reajustes de esta etapa de existencia establecida se refieren casi enteramente a los dominios físicos y son la ocupación fundamental de los Controladores Físicos Maestros.

\par
%\textsuperscript{(628.9)}
\textsuperscript{55:4.21} 6. \textit{La sexta etapa de luz y de vida} presencia el desarrollo de nuevas funciones de los circuitos mentales del reino. La sabiduría cósmica parece volverse constitutiva en el ministerio universal de la mente.

\par
%\textsuperscript{(628.10)}
\textsuperscript{55:4.22} 7. \textit{La séptima etapa de luz y de vida}. Al principio de la séptima época, al Instructor Trinitario consejero del Soberano Planetario se le une un asesor voluntario enviado por los Ancianos de los Días, y más tarde se le añadirá un tercer consejero procedente del Ejecutivo Supremo del superuniverso.

\par
%\textsuperscript{(629.1)}
\textsuperscript{55:4.23} Durante esta época, si no ha sucedido antes, Adán y Eva siempre son liberados de sus deberes planetarios. Si en el cuerpo finalitario hay un Hijo Material, puede asociarse con el jefe ejecutivo humano, y a veces es un Melquisedek el que se ofrece como voluntario para ejercer esta función. Si hay un intermedio entre los finalitarios, todos los miembros de esta orden que permanecen en el planeta son liberados de inmediato.

\par
%\textsuperscript{(629.2)}
\textsuperscript{55:4.24} Tras conseguir liberarse de su misión milenaria, un Adán y una Eva Planetarios pueden elegir entre las carreras siguientes:

\par
%\textsuperscript{(629.3)}
\textsuperscript{55:4.25} 1. Pueden obtener su liberación planetaria e iniciar inmediatamente, desde la sede del universo, su carrera hacia el Paraíso, recibiendo los Ajustadores del Pensamiento al final de su experiencia morontial.

\par
%\textsuperscript{(629.4)}
\textsuperscript{55:4.26} 2. Muy a menudo, un Adán y una Eva Planetarios reciben sus Ajustadores mientras sirven todavía en un mundo establecido en la luz, y esto sucede en el momento de recibir sus Ajustadores algunos de sus hijos importados de linaje puro que se han ofrecido como voluntarios para un período de servicio planetario. Posteriormente todos pueden ir a la sede del universo y empezar allí la carrera hacia el Paraíso.

\par
%\textsuperscript{(629.5)}
\textsuperscript{55:4.27} 3. Un Adán y una Eva Planetarios pueden elegir ir directamente al mundo midsonito durante una breve temporada ---como lo hacen los Hijos y las Hijas Materiales de la capital del sistema--- para recibir allí sus Ajustadores.

\par
%\textsuperscript{(629.6)}
\textsuperscript{55:4.28} 4. Pueden decidir regresar a la sede del sistema, para ocupar allí sus asientos durante un tiempo en el tribunal supremo, y después de este servicio recibirán sus Ajustadores y empezarán la ascensión hacia el Paraíso.

\par
%\textsuperscript{(629.7)}
\textsuperscript{55:4.29} 5. Después de dejar sus funciones administrativas, pueden elegir regresar a su mundo nativo para servir como instructores durante una temporada, y ser habitados por un Ajustador en el momento de ser trasladados a la sede del universo.

\par
%\textsuperscript{(629.8)}
\textsuperscript{55:4.30} Durante todas estas épocas, los Hijos y las Hijas Materiales importados como ayudantes ejercen una enorme influencia sobre los grupos sociales y económicos en progreso. Son potencialmente inmortales, al menos hasta el momento en que eligen humanizarse, recibir sus Ajustadores y partir hacia el Paraíso.

\par
%\textsuperscript{(629.9)}
\textsuperscript{55:4.31} En los mundos evolutivos, un ser debe humanizarse para recibir un Ajustador del Pensamiento. Todos los miembros ascendentes del Cuerpo de los Mortales Finalitarios han estado habitados por un Ajustador y han fusionado con él, excepto los serafines, y éstos son habitados por otro tipo de espíritu del Padre en el momento de ser enrolados en este cuerpo.

\section*{5. El apogeo del desarrollo material}
\par
%\textsuperscript{(629.10)}
\textsuperscript{55:5.1} Las criaturas mortales que viven en un mundo aislado, afligido por el pecado, dominado por el mal y egoísta como Urantia, difícilmente pueden concebir la perfección física, los logros intelectuales y el desarrollo espiritual que caracterizan a estas épocas avanzadas de evolución en una esfera libre de pecado.

\par
%\textsuperscript{(629.11)}
\textsuperscript{55:5.2} Las etapas avanzadas de un mundo establecido en la luz y la vida representan la cima del desarrollo material evolutivo. En estos mundos cultos no queda nada de la ociosidad y las fricciones de las épocas primitivas anteriores. La pobreza y la desigualdad social casi se han desvanecido, la degeneración ha desaparecido y la delincuencia se observa raramente. La locura ha dejado prácticamente de existir y la debilidad mental es una rareza.

\par
%\textsuperscript{(629.12)}
\textsuperscript{55:5.3} El estado económico, social y administrativo de estos mundos es de un tipo elevado y perfeccionado. La ciencia, el arte y la industria florecen, y la sociedad es un mecanismo de elevados logros materiales, intelectuales y culturales que funciona sin problemas. La industria se ha desviado en gran parte hacia el servicio de los objetivos superiores de esta magnífica civilización. La vida económica de este mundo se ha vuelto ética.

\par
%\textsuperscript{(630.1)}
\textsuperscript{55:5.4} La guerra se ha convertido en una cuestión histórica, y ya no existen ni ejércitos ni fuerzas de policía. El gobierno desaparece gradualmente. El autocontrol hace lentamente que las leyes promulgadas por los humanos resulten obsoletas. En un estado intermedio de civilización progresiva, la importancia del gobierno civil y de la reglamentación legal es inversamente proporcional a la moral y a la espiritualidad de los ciudadanos.

\par
%\textsuperscript{(630.2)}
\textsuperscript{55:5.5} Las escuelas han mejorado considerablemente y están dedicadas a la educación de la mente y a la expansión del alma. Los centros artísticos son exquisitos y las organizaciones musicales magníficas. Los templos para la adoración, con sus escuelas asociadas de filosofía y de religión experiencial, son unas creaciones llenas de belleza y de grandiosidad. Las zonas al aire libre para las asambleas cultuales son igualmente sublimes en la simplicidad de sus detalles artísticos.

\par
%\textsuperscript{(630.3)}
\textsuperscript{55:5.6} Las instalaciones para los juegos competitivos, el humor y otras fases de las realizaciones personales y colectivas son amplias y apropiadas. Una característica especial de las actividades competitivas en un mundo tan sumamente culto se refiere a los esfuerzos de los individuos y de los grupos por sobresalir en las ciencias y las filosofías de la cosmología. La literatura y la oratoria florecen, y el idioma ha mejorado tanto que es capaz de simbolizar los conceptos así como de expresar las ideas. La vida es de una sencillez refrescante; el hombre ha coordinado por fin un elevado estado de desarrollo mecánico con unos logros intelectuales inspiradores, y ha eclipsado los dos con un logro espiritual exquisito. La búsqueda de la felicidad es una experiencia de alegría y de satisfacción.

\section*{6. El mortal individual}
\par
%\textsuperscript{(630.4)}
\textsuperscript{55:6.1} A medida que los mundos avanzan en el estado establecido de la luz y la vida, la sociedad se vuelve cada vez más pacífica. El individuo, aunque no es menos independiente ni está menos dedicado a su familia, se ha vuelto más altruista y fraternal.

\par
%\textsuperscript{(630.5)}
\textsuperscript{55:6.2} En Urantia, y tal como estáis, poco podéis apreciar el estado avanzado y la naturaleza progresiva de las razas iluminadas de estos mundos perfeccionados. Estos pueblos son el florecimiento de las razas evolutivas. Pero estos seres siguen siendo mortales; continúan respirando, comiendo, durmiendo y bebiendo. Esta gran evolución no es el cielo, pero es un presagio sublime de los mundos divinos que se encontrarán durante la ascensión hacia el Paraíso.

\par
%\textsuperscript{(630.6)}
\textsuperscript{55:6.3} En un mundo normal, hace mucho tiempo que la aptitud biológica de la raza mortal fue llevada a un nivel elevado durante las épocas postadámicas; y ahora, la evolución física del hombre continúa de época en época a lo largo de las eras establecidas. Tanto la vista como el oído se amplían. Ahora, la cifra de la población se ha vuelto estable. La reproducción está regulada con arreglo a las necesidades planetarias y a los dones hereditarios innatos: durante esta era, los mortales del planeta están divididos entre cinco y diez grupos, y a los grupos inferiores sólo se les permite procrear la mitad de hijos que a los grupos superiores. El mejoramiento continuo de una raza tan magnífica durante toda la era de luz y de vida es principalmente una cuestión de reproducción selectiva de aquellos linajes raciales que manifiestan unas cualidades superiores de naturaleza social, filosófica, cósmica y espiritual.

\par
%\textsuperscript{(630.7)}
\textsuperscript{55:6.4} Los Ajustadores continúan llegando como en las eras evolutivas anteriores, y a medida que pasan las épocas, estos mortales son cada vez más capaces de comulgar con el fragmento interior del Padre. Durante las etapas embrionarias y preespirituales de desarrollo, los espíritus ayudantes de la mente siguen funcionando. El Espíritu Santo y el ministerio de los ángeles son incluso más eficaces a medida que se experimentan las épocas sucesivas de vida establecida. En la cuarta etapa de luz y de vida, los mortales avanzados parecen experimentar un contacto consciente importante con la presencia espiritual del Espíritu Maestro que tiene la jurisdicción sobre ese superuniverso, mientras que la filosofía de ese mundo está centrada en el intento por comprender las nuevas revelaciones de Dios Supremo. Más de la mitad de los habitantes humanos de los planetas que han llegado a este nivel avanzado experimentan el traslado de entre los vivos al estado morontial. Así es como <<las antiguas cosas están desapareciendo; mirad, todas las cosas se vuelven nuevas>>.

\par
%\textsuperscript{(631.1)}
\textsuperscript{55:6.5} Pensamos que la evolución física habrá alcanzado su pleno desarrollo al final de la quinta época de la era de luz y de vida. Observamos que los límites superiores del desarrollo espiritual, asociado a la mente humana en evolución, están determinados por el nivel de fusión con el Ajustador de los valores morontiales y de los significados cósmicos conjuntos. Pero en lo que se refiere a la sabiduría, aunque no lo sabemos realmente, suponemos que nunca podrá haber un límite a la evolución intelectual y a la adquisición de la sabiduría. En un mundo en la séptima etapa, la sabiduría puede agotar los potenciales materiales, empezar a captar la mota, e incluso saborear finalmente la grandiosidad absonita.

\par
%\textsuperscript{(631.2)}
\textsuperscript{55:6.6} Observamos que en estos mundos extremadamente evolucionados que llevan mucho tiempo en la séptima etapa, los seres humanos aprenden por completo el idioma del universo local antes de ser trasladados; y he visitado algunos planetas muy antiguos donde los abandontarios enseñaban a los mortales más ancianos la lengua del superuniverso. Y he observado en estos mundos la técnica mediante la cual las personalidades absonitas revelan la presencia de los finalitarios en el templo morontial.

\par
%\textsuperscript{(631.3)}
\textsuperscript{55:6.7} Ésta es la historia de la magnífica meta de los esfuerzos humanos en los mundos evolutivos; y todo esto tiene lugar incluso antes de que los seres humanos emprendan su carrera morontial; todo este espléndido desarrollo es alcanzable por los mortales materiales en los mundos habitados, la primerísima etapa de esa carrera interminable e incomprensible para ascender al Paraíso y alcanzar la divinidad.

\par
%\textsuperscript{(631.4)}
\textsuperscript{55:6.8} Pero ¿os resulta posible imaginar la clase de mortales evolutivos que está ascendiendo ahora desde los mundos que existen hace mucho tiempo en la séptima época de luz y de vida establecidas? Son semejantes a los que avanzan hasta los mundos morontiales de la capital del universo local para empezar su carrera de ascensión.

\par
%\textsuperscript{(631.5)}
\textsuperscript{55:6.9} Si los mortales de la afligida Urantia tan sólo pudieran ver uno de estos mundos más avanzados y establecidos hace mucho tiempo en la luz y la vida, no volverían a poner en duda nunca más la sabiduría del plan evolutivo de la creación. Aunque no existiera ningún futuro de progresión eterna para las criaturas, los magníficos logros evolutivos de las razas mortales de estos mundos establecidos que han alcanzado sus metas por completo justificarían ampliamente la creación del hombre en los mundos del tiempo y del espacio.

\par
%\textsuperscript{(631.6)}
\textsuperscript{55:6.10} A menudo reflexionamos: Si el gran universo se estableciera en la luz y la vida, ¿los exquisitos mortales ascendentes continuarían siendo destinados al Cuerpo de la Finalidad? Pero no lo sabemos.

\section*{7. La primera etapa o etapa planetaria}
\par
%\textsuperscript{(631.7)}
\textsuperscript{55:7.1} Esta época se extiende desde la aparición del templo morontial en la nueva sede del planeta hasta el momento en que todo el sistema se establece en la luz y la vida. Los Hijos Instructores Trinitarios inauguran esta era al final de sus misiones mundiales sucesivas cuando el Príncipe Planetario es elevado a la categoría de Soberano Planetario por mandato y en la presencia personal del Hijo Paradisiaco donador de esa esfera. En concomitancia con esto, los finalitarios inauguran su participación activa en los asuntos planetarios.

\par
%\textsuperscript{(632.1)}
\textsuperscript{55:7.2} Según las apariencias exteriores y visibles, los gobernantes o directores reales de un mundo así establecido en la luz y la vida son el Hijo y la Hija Materiales, el Adán y la Eva Planetarios. Los finalitarios son invisibles, como también lo es el Príncipe-Soberano, salvo cuando está en el templo morontial. Los jefes reales y literales del régimen planetario son por tanto el Hijo y la Hija Materiales. El conocimiento de estas disposiciones es lo que le ha dado prestigio a la idea de los reyes y de las reinas en todos los planetas del universo. Los reyes y las reinas constituyen un gran éxito en estas circunstancias ideales, cuando un mundo puede disponer de estas elevadas personalidades para que actúen en nombre de unos gobernantes aún mas elevados pero invisibles.

\par
%\textsuperscript{(632.2)}
\textsuperscript{55:7.3} Cuando vuestro mundo alcance esta era, no hay duda de que Maquiventa Melquisedek, ahora Príncipe Planetario vicegerente de Urantia, ocupará el asiento del Soberano Planetario; y en Jerusem se ha supuesto desde hace mucho tiempo que estará acompañado por un hijo y una hija del Adán y la Eva de Urantia, hijos actualmente retenidos en Edentia como pupilos de los Altísimos de Norlatiadek. Estos hijos de Adán podrían servir así en Urantia en asociación con el Soberano Melquisedek, pues fueron privados de sus poderes procreadores hace cerca de 37.000 años cuando dejaron sus cuerpos materiales en Urantia como preparación para ser trasladados a Edentia.

\par
%\textsuperscript{(632.3)}
\textsuperscript{55:7.4} Esta era establecida continúa sin cesar hasta que todos los planetas habitados del sistema alcanzan la era de la estabilización; entonces, cuando el mundo más joven ---el último en alcanzar la luz y la vida--- ha experimentado esta estabilidad durante un milenio del tiempo del sistema, todo el sistema inicia el estado estabilizado, y los mundos individuales entran en la época sistémica de la era de luz y de vida.

\section*{8. La segunda etapa o etapa del sistema}
\par
%\textsuperscript{(632.4)}
\textsuperscript{55:8.1} Cuando un sistema entero se establece en la vida, se inaugura un nuevo tipo de gobierno. Los Soberanos Planetarios se convierten en miembros del cónclave del sistema, y este nuevo cuerpo administrativo, sujeto al veto de los Padres de la Constelación, tiene una autoridad suprema. Un sistema así de mundos habitados se vuelve prácticamente autónomo. La asamblea legislativa del sistema se constituye en el mundo sede, y cada planeta envía allí a sus diez representantes. Ahora los tribunales están establecidos en las capitales de los sistemas, y a la sede del universo sólo se envían las apelaciones.

\par
%\textsuperscript{(632.5)}
\textsuperscript{55:8.2} Con el establecimiento del sistema, el Centinela Asignado, representante del Ejecutivo Supremo del superuniverso, se convierte en el consejero voluntario del tribunal supremo del sistema y en el dignatario que preside realmente la nueva asamblea legislativa.

\par
%\textsuperscript{(632.6)}
\textsuperscript{55:8.3} Después de que un sistema entero se establece en la luz y la vida, los Soberanos Sistémicos dejan de ir y venir. Un soberano así permanece perpetuamente al frente de su sistema. Los soberanos asistentes continúan cambiando como en las épocas anteriores.

\par
%\textsuperscript{(632.7)}
\textsuperscript{55:8.4} Durante esta época de estabilización, los midsonitarios llegan por primera vez desde los mundos sede del universo donde residen para actuar como consejeros de las asambleas legislativas y como asesores de los tribunales judiciales. Estos midsonitarios realizan también ciertos esfuerzos por inculcar nuevos significados de mota, que tienen un valor supremo, en las empresas educativas que patrocinan en unión con los finalitarios. Aquello que los Hijos Materiales hicieron biológicamente por las razas mortales, las criaturas midsonitas lo hacen ahora por estos humanos unificados y glorificados en el terreno en constante progreso de la filosofía y del pensamiento espiritualizado.

\par
%\textsuperscript{(633.1)}
\textsuperscript{55:8.5} En los mundos habitados, los Hijos Instructores se convierten en los colaboradores voluntarios de los finalitarios, y estos mismos Hijos Instructores también acompañan a los finalitarios a los mundos de las mansiones cuando estas esferas dejan de utilizarse como mundos receptores diferenciales después de que todo el sistema está establecido en la luz y la vida; al menos esto es así en la época en que toda la constelación ha evolucionado de esta manera. Pero no existen grupos tan avanzados en Nebadon.

\par
%\textsuperscript{(633.2)}
\textsuperscript{55:8.6} No se nos permite revelar la naturaleza del trabajo de los finalitarios que supervisarán estos mundos de las mansiones dedicados a otras actividades. Sin embargo, se os ha informado que existen en todos los universos diversos tipos de criaturas inteligentes que no han sido descritas en estas narraciones.

\par
%\textsuperscript{(633.3)}
\textsuperscript{55:8.7} Y ahora, a medida que los sistemas se establecen uno tras otro en la luz en virtud del progreso de los mundos que los componen, llega el momento en que el último sistema de una constelación dada alcanza la estabilización, y los administradores del universo ---el Hijo Maestro, el Unión de los Días y la Radiante Estrella Matutina--- llegan a la capital de la constelación para proclamar a los Altísimos como gobernantes incondicionales de la familia recién perfeccionada de cien sistemas establecidos de mundos habitados.

\section*{9. La tercera etapa o etapa de la constelación}
\par
%\textsuperscript{(633.4)}
\textsuperscript{55:9.1} La unificación de toda una constelación de sistemas establecidos viene acompañada de nuevas distribuciones de la autoridad ejecutiva y de reajustes adicionales en la administración del universo. Esta época presencia unos logros avanzados en todos los mundos habitados, pero está caracterizada particularmente por los reajustes en la sede de la constelación, con una notable modificación de las relaciones tanto con la supervisión sistémica como con el gobierno del universo local. Durante esta era, muchas actividades de la constelación y del universo se transfieren a las capitales de los sistemas, y los representantes del superuniverso establecen unas relaciones nuevas y más profundas con los gobernantes de los planetas, de los sistemas y del universo. En concomitancia con estas nuevas asociaciones, algunos administradores superuniversales se establecen en las capitales de las constelaciones como consejeros voluntarios de los Padres Altísimos.

\par
%\textsuperscript{(633.5)}
\textsuperscript{55:9.2} Cuando una constelación se establece así en la luz, la función legislativa cesa, y la cámara de los Soberanos de los Sistemas, presidida por los Altísimos, funciona en su lugar. Ahora, y por primera vez, estos grupos administrativos tratan directamente con el gobierno del superuniverso los asuntos concernientes a las relaciones con Havona y el Paraíso. Por lo demás, la constelación sigue relacionada con el universo local como antes. Durante las etapas sucesivas de la vida establecida, los univitatias continúan administrando los mundos morontiales de la constelación.

\par
%\textsuperscript{(633.6)}
\textsuperscript{55:9.3} A medida que pasan las épocas, los Padres de la Constelación se hacen cargo cada vez más de las funciones administrativas detalladas o de supervisión que estaban centradas anteriormente en la sede del universo. Cuando se alcance la sexta etapa de estabilización, estas constelaciones unificadas habrán alcanzado la posición de una autonomía casi completa. El comienzo de la séptima etapa establecida presenciará sin duda la elevación de estos gobernantes a la verdadera dignidad que indican sus nombres, los Altísimos. A todos los efectos prácticos, las constelaciones tratarán entonces directamente con los gobernantes del superuniverso, mientras que el gobierno del universo local se ampliará hasta abarcar las responsabilidades de nuevas obligaciones hacia el gran universo.

\section*{10. La cuarta etapa o etapa del universo local}
\par
%\textsuperscript{(634.1)}
\textsuperscript{55:10.1} Cuando un universo se instala en la luz y la vida, pronto empieza a girar en los circuitos establecidos del superuniverso, y los Ancianos de los Días proclaman el establecimiento del \textit{consejo supremo de autoridad ilimitada}. Este nuevo cuerpo gobernante está compuesto por los cien Fieles de los Días, presididos por el Unión de los Días, y el primer acto de este consejo supremo consiste en reconocer la continuidad de la soberanía del Hijo Maestro Creador.

\par
%\textsuperscript{(634.2)}
\textsuperscript{55:10.2} La administración del universo, en lo que concierne a Gabriel y al Padre Melquisedek, permanece sin cambios. Este consejo de autoridad ilimitada se ocupa principalmente de los nuevos problemas y de las nuevas condiciones resultantes del estado avanzado de luz y de vida.

\par
%\textsuperscript{(634.3)}
\textsuperscript{55:10.3} El Inspector Asociado moviliza ahora a todos los Centinelas Asignados para formar el \textit{cuerpo de estabilización del universo local}, e invita al Padre Melquisedek a que comparta su supervisión con él. Ahora, un cuerpo de Espíritus Inspirados Trinitarios es destinado por primera vez al servicio del Unión de los Días.

\par
%\textsuperscript{(634.4)}
\textsuperscript{55:10.4} El establecimiento de todo un universo local en la luz y la vida inaugura profundos reajustes en todo el sistema administrativo, desde los mundos habitados individuales hasta la sede del universo. Se desarrollan nuevas relaciones con las constelaciones y los sistemas. El Espíritu Madre del universo local experimenta nuevas relaciones de enlace con el Espíritu Maestro del superuniverso, y Gabriel establece un contacto directo con los Ancianos de los Días que pueda ser operativo cuando el Hijo Maestro esté ausente del mundo sede.

\par
%\textsuperscript{(634.5)}
\textsuperscript{55:10.5} Durante esta era y las siguientes, los Hijos Magistrales continúan actuando como jueces dispensacionales, mientras que cien de estos Hijos Avonales del Paraíso componen el nuevo consejo superior de la Radiante Estrella Matutina en la capital del universo. Más tarde, y a petición de los Soberanos de los Sistemas, uno de estos Hijos Magistrales se convertirá en el consejero supremo situado en el mundo sede de cada sistema local hasta que se alcance la séptima etapa de unidad.

\par
%\textsuperscript{(634.6)}
\textsuperscript{55:10.6} Durante esta época, los Hijos Instructores Trinitarios no sólo actúan como consejeros voluntarios de los Soberanos Planetarios, sino que sirven de manera similar en grupos de tres a los Padres de las Constelaciones. Por fin estos Hijos encuentran su lugar en el universo local, pues durante este período se les retira de la jurisdicción de la creación local y se les destina al servicio del consejo supremo de autoridad ilimitada.

\par
%\textsuperscript{(634.7)}
\textsuperscript{55:10.7} El cuerpo finalitario reconoce ahora por primera vez la jurisdicción de una autoridad exterior al Paraíso: el consejo supremo. Hasta este momento los finalitarios no habían reconocido ninguna supervisión a este lado del Paraíso.

\par
%\textsuperscript{(634.8)}
\textsuperscript{55:10.8} Los Hijos Creadores de estos universos establecidos pasan una gran parte de su tiempo en el Paraíso y en sus mundos asociados, y aconsejando a los numerosos grupos finalitarios que sirven en toda la creación local. De esta manera, Miguel como hombre encontrará una fraternidad de asociación más completa con los mortales finalitarios glorificados.

\par
%\textsuperscript{(634.9)}
\textsuperscript{55:10.9} Es totalmente inútil especular sobre la función de estos Hijos Creadores en relación con los universos exteriores actualmente en proceso de formación preliminar. Pero todos nos dedicamos de vez en cuando a estas suposiciones. Cuando se alcanza esta cuarta etapa de desarrollo, el Hijo Creador se vuelve administrativamente libre; la Ministra Divina armoniza progresivamente su ministerio con el del Espíritu Maestro del superuniverso y con el Espíritu Infinito. Parece que se desarrolla una relación nueva y sublime entre el Hijo Creador, el Espíritu Creativo, las Estrellas Vespertinas, los Hijos Instructores y el cuerpo finalitario en constante aumento.

\par
%\textsuperscript{(635.1)}
\textsuperscript{55:10.10} Si Miguel tuviera que salir alguna vez de Nebadon, Gabriel se convertiría sin duda alguna en el administrador en jefe con el Padre Melquisedek como asociado. Al mismo tiempo se concedería una nueva categoría a todas las órdenes de ciudadanos permanentes tales como los Hijos Materiales, los univitatias, los midsonitarios, los susatias y los mortales fusionados con el Espíritu. Pero mientras continúe la evolución, los serafines y los arcángeles serán necesarios en la administración del universo.

\par
%\textsuperscript{(635.2)}
\textsuperscript{55:10.11} Sin embargo, estamos convencidos respecto a dos características de nuestras especulaciones: si los Hijos Creadores son destinados a los universos exteriores, las Ministras Divinas los acompañarán sin duda alguna. También estamos seguros de que los Melquisedeks permanecerán en sus universos de origen. Consideramos que los Melquisedeks están destinados a desempeñar un papel de responsabilidad creciente en el gobierno y la administración del universo local.

\section*{11. La etapa del sector menor y del sector mayor}
\par
%\textsuperscript{(635.3)}
\textsuperscript{55:11.1} Los sectores menores y mayores del superuniverso no figuran directamente en el plan de instalarse en la luz y la vida. Esta progresión evolutiva incumbe principalmente al universo local como unidad, y sólo concierne a los componentes de un universo local. Un superuniverso se establece en la luz y la vida cuando todos sus universos locales componentes se han perfeccionado de esta manera. Pero ninguno de los siete superuniversos ha alcanzado un nivel de progreso que se acerque siquiera a este estado.

\par
%\textsuperscript{(635.4)}
\textsuperscript{55:11.2} \textit{La era del sector menor}. Hasta donde nuestras observaciones pueden penetrar, la quinta etapa de estabilización, o etapa del sector menor, está exclusivamente relacionada con el estado físico y con la instalación coordinada de los cien universos locales asociados en los circuitos establecidos del superuniverso. Al parecer nadie, salvo los centros del poder y sus asociados, están implicados en estos reajustes de la creación material.

\par
%\textsuperscript{(635.5)}
\textsuperscript{55:11.3} \textit{La era del sector mayor}. En cuanto a la sexta etapa, la de la estabilización del sector mayor, sólo podemos hacer conjeturas puesto que ninguno de nosotros ha presenciado un acontecimiento así. Sin embargo, podemos dar por sentadas muchas cosas en lo que concierne a los reajustes administrativos y de otros tipos que probablemente acompañarían a este estado avanzado de los mundos habitados y de sus agrupaciones en el universo.

\par
%\textsuperscript{(635.6)}
\textsuperscript{55:11.4} Puesto que el estado del sector menor tiene que ver con el equilibrio físico coordinado, deducimos que la unificación del sector mayor estará relacionada con ciertos nuevos niveles de consecución intelectuales, posiblemente algunos logros avanzados en la realización suprema de la sabiduría cósmica.

\par
%\textsuperscript{(635.7)}
\textsuperscript{55:11.5} Llegamos a estas conclusiones sobre los reajustes que podrían acompañar a la conquista de unos niveles de progreso evolutivo aún no alcanzados observando los resultados de estos logros en los mundos individuales y en las experiencias de los mortales individuales que viven en estas esferas más antiguas y extremadamente desarrolladas.

\par
%\textsuperscript{(635.8)}
\textsuperscript{55:11.6} Que quede muy claro que los mecanismos administrativos y las técnicas gubernamentales de un universo o de un superuniverso no pueden limitar o retrasar de ninguna manera el desarrollo evolutivo o el progreso espiritual de un planeta individual habitado o de un mortal individual de esa esfera.

\par
%\textsuperscript{(635.9)}
\textsuperscript{55:11.7} En algunos de los universos más antiguos encontramos mundos establecidos en la quinta y en la sexta etapa de luz y de vida ---e incluso muy adentrados en la séptima época--- cuyos sistemas locales aún no están establecidos en la luz. Los planetas más jóvenes pueden retrasar la unificación de un sistema, pero esto no dificulta en lo más mínimo el progreso de un mundo más antiguo y avanzado. Las limitaciones del entorno, ni siquiera en un mundo aislado, tampoco pueden frustrar los logros personales del mortal individual; Jesús de Nazaret, como hombre entre los hombres, alcanzó personalmente el estado de luz y de vida en Urantia hace más de mil novecientos años.

\par
%\textsuperscript{(636.1)}
\textsuperscript{55:11.8} Observando lo que sucede en los mundos establecidos desde hace mucho tiempo es como llegamos a unas conclusiones bastante fiables sobre lo que ocurrirá cuando un superuniverso entero se establezca en la luz, aunque no podemos dar por sentado con seguridad el caso de la estabilización de los siete superuniversos.

\section*{12. La séptima etapa o etapa del superuniverso}
\par
%\textsuperscript{(636.2)}
\textsuperscript{55:12.1} No podemos prever de manera categórica lo que sucederá cuando un superuniverso se establezca en la luz porque un acontecimiento así no se ha producido nunca. Según las enseñanzas de los Melquisedeks, que nunca han sido contradichas, deducimos que se efectuarán unos cambios radicales en toda la organización y la administración de cada unidad de las creaciones del tiempo y del espacio, desde los mundos habitados hasta la sede del superuniverso.

\par
%\textsuperscript{(636.3)}
\textsuperscript{55:12.2} Se cree de forma general que un gran número de hijos trinitizados por las criaturas, por otra parte disponibles, serían agrupados en las sedes y en las capitales divisionarias de los superuniversos establecidos. Esto podría hacerse pensando en la llegada futura de los habitantes del espacio exterior en su camino interior hacia Havona y el Paraíso; pero en realidad no lo sabemos.

\par
%\textsuperscript{(636.4)}
\textsuperscript{55:12.3} Si un superuniverso se estableciera en la luz y la vida, creemos que cuando esto suceda los Supervisores Incalificados del Supremo, actualmente asesores suyos, se convertirían en el cuerpo administrativo superior del mundo sede del superuniverso. Éstas son las personalidades que pueden ponerse en contacto directo con los administradores absonitos, los cuales desempeñarían enseguida su actividad en el superuniverso establecido. Aunque estos Supervisores Incalificados han actuado durante mucho tiempo como consejeros y asesores en unidades evolutivas avanzadas de la creación, no asumirán responsabilidades administrativas hasta que la autoridad del Ser Supremo se haya vuelto soberana.

\par
%\textsuperscript{(636.5)}
\textsuperscript{55:12.4} Los Supervisores Incalificados del Supremo, que ejercen más ampliamente su actividad durante esta época, no son finitos, ni absonitos, ni últimos, ni infinitos; \textit{son} la supremacía y sólo representan a Dios Supremo. Son la personalización de la supremacía en el tiempo y el espacio y, por lo tanto, no desempeñan sus funciones en Havona. Sólo actúan como unificadores supremos. Quizás estén implicados en la técnica de la reflectividad universal, pero no estamos seguros.

\par
%\textsuperscript{(636.6)}
\textsuperscript{55:12.5} Ninguno de nosotros alberga un concepto satisfactorio sobre lo que sucederá cuando el gran universo (los siete superuniversos que dependen de Havona) se establezca totalmente en la luz y la vida. Ese acontecimiento representará sin duda el suceso más profundo de los anales de la eternidad desde la aparición del universo central. Están aquellos que sostienen que el Ser Supremo mismo saldrá del misterio de Havona que envuelve a su persona espiritual, y establecerá su residencia en la sede del séptimo superuniverso como soberano todopoderoso y experiencial de las creaciones perfeccionadas del tiempo y del espacio. Pero en realidad no lo sabemos.

\par
%\textsuperscript{(636.7)}
\textsuperscript{55:12.6} [Presentado por un Mensajero Poderoso destinado temporalmente en el Consejo de los Arcángeles en Urantia.]


\chapter{Documento 56. La unidad universal}
\par
%\textsuperscript{(637.1)}
\textsuperscript{56:0.1} DIOS es unidad. La Deidad está universalmente coordinada. El universo de universos es un inmenso mecanismo integrado que está absolutamente controlado por una sola mente infinita. Los ámbitos físicos, intelectuales y espirituales de la creación universal están divinamente correlacionados. Lo perfecto y lo imperfecto están realmente interrelacionados, y por eso las criaturas evolutivas finitas pueden ascender hasta el Paraíso en conformidad con el mandato del Padre Universal: <<Sed perfectos como yo soy perfecto>>\footnote{\textit{Sed perfectos como yo soy}: Gn 17:1; 1 Re 8:61; Lv 19:2; Dt 18:13; Mt 5:48; 2 Co 13:11; Stg 1:4; 1 P 1:16.}.

\par
%\textsuperscript{(637.2)}
\textsuperscript{56:0.2} Todos los diversos niveles de la creación están unificados en los planes y en la administración de los Arquitectos del Universo Maestro. Para la mente circunscrita de los mortales del espacio-tiempo, el universo puede presentar muchos problemas y situaciones que muestran aparentemente una falta de armonía y que indican la ausencia de una coordinación efectiva; pero aquellos de nosotros que son capaces de observar una gama más amplia de fenómenos universales, que tienen más experiencia en este arte de detectar la unidad fundamental que se oculta tras la diversidad creativa, y de descubrir la unidad divina que se extiende sobre todo este funcionamiento de la pluralidad, perciben mejor el propósito único y divino que muestran todas estas múltiples manifestaciones de la energía creativa universal.

\section*{1. La coordinación física}
\par
%\textsuperscript{(637.3)}
\textsuperscript{56:1.1} La creación física o material no es infinita, pero está perfectamente coordinada. Existen la fuerza, la energía y el poder, pero todas son una sola cosa en su origen. Los siete superuniversos parecen duales, y el universo central, trino; pero el Paraíso tiene una constitución singular. El Paraíso es la fuente efectiva de todos los universos materiales ---pasados, presentes y futuros. Pero esta derivación cósmica es un acontecimiento de la \textit{eternidad}; en ningún \textit{tiempo} ---pasado, presente o futuro--- el espacio o el cosmos material surgen de la Isla nuclear de Luz. Como fuente cósmica, el Paraíso funciona con anterioridad al espacio y antes del tiempo; de ahí que sus derivaciones parecerían estar huérfanas en el tiempo y en el espacio si no aparecieran a través del Absoluto Incalificado, su depositario último en el espacio y su revelador y regulador en el tiempo.

\par
%\textsuperscript{(637.4)}
\textsuperscript{56:1.2} El Absoluto Incalificado sostiene el universo físico, mientras que el Absoluto de la Deidad motiva el exquisito supercontrol de toda la realidad material; y los dos Absolutos están unificados funcionalmente por el Absoluto Universal. Todas las personalidades ---materiales, morontiales, absonitas o espirituales--- comprenden mejor esta correlación cohesiva del universo material observando la reacción gravitatoria de toda la auténtica realidad material a la gravedad centrada en el bajo Paraíso.

\par
%\textsuperscript{(638.1)}
\textsuperscript{56:1.3} La unificación por medio de la gravedad es universal e invariable; la reacción a la energía pura es igualmente universal e ineludible. La energía pura (la fuerza primordial) y el espíritu puro son totalmente pre-sensibles a la gravedad. Estas fuerzas fundamentales, inherentes a los Absolutos, están personalmente controladas por el Padre Universal; de ahí que toda la gravedad esté centrada en la presencia personal del Padre Paradisiaco de la energía pura y del puro espíritu, y en su morada supermaterial.

\par
%\textsuperscript{(638.2)}
\textsuperscript{56:1.4} La energía pura es la predecesora de todas las realidades relativas funcionales no espirituales, mientras que el espíritu puro es el potencial del supercontrol divino que dirige todos los sistemas energéticos fundamentales. Estas dos realidades, que se manifiestan en todo el espacio y se observan en los movimientos del tiempo de forma tan diversa, están centradas en la persona del Padre Paradisiaco. En él son una sola cosa ---deben estar unificadas--- porque Dios es uno. La personalidad del Padre está absolutamente unificada.

\par
%\textsuperscript{(638.3)}
\textsuperscript{56:1.5} En la naturaleza infinita de Dios Padre no podría existir de ninguna manera una dualidad de la realidad\footnote{\textit{Dualidad de la realidad}: Jn 3:6-7; Hch 17:28.}, como por ejemplo la física y la espiritual; pero en cuanto apartamos la vista de los niveles infinitos y de la realidad absoluta de los valores personales del Padre Paradisiaco, observamos la existencia de estas dos realidades y reconocemos que son plenamente sensibles a su presencia personal; en él radican todas las cosas\footnote{\textit{En él radican todas las cosas}: Col 1:17.}.

\par
%\textsuperscript{(638.4)}
\textsuperscript{56:1.6} En el momento en que uno se aparta del concepto incondicional de la personalidad infinita del Padre Paradisiaco, hay que presuponer que la MENTE es la técnica inevitable para unificar la divergencia creciente de estas manifestaciones universales duales de la personalidad original y de un solo elemento del Creador, la Fuente-Centro Primera ---el YO SOY.

\section*{2. La unidad intelectual}
\par
%\textsuperscript{(638.5)}
\textsuperscript{56:2.1} El Padre-Pensamiento hace realidad la expresión del espíritu en el Hijo-Verbo y consigue desarrollar la realidad, en los extensos universos materiales, a través del Paraíso. Las expresiones espirituales del Hijo Eterno están correlacionadas con los niveles materiales de la creación mediante las funciones del Espíritu Infinito; las realidades espirituales de la Deidad y las repercusiones materiales de la Deidad están correlacionadas entre sí gracias al ministerio mental sensible al espíritu del Espíritu Infinito y en sus actos mentales que dirigen lo físico.

\par
%\textsuperscript{(638.6)}
\textsuperscript{56:2.2} La mente es el atributo funcional del Espíritu Infinito, por lo que su potencial es infinito y su concesión es universal. El pensamiento primordial del Padre Universal se eterniza en una expresión doble: la Isla del Paraíso y el Hijo espiritual y Eterno, su igual en Deidad. Esta dualidad de la realidad eterna hace que el Dios mental, el Espíritu Infinito, resulte inevitable. La mente es el canal de comunicación indispensable entre las realidades espirituales y las realidades materiales. La criatura material evolutiva sólo puede concebir y comprender al espíritu interior mediante el ministerio de la mente.

\par
%\textsuperscript{(638.7)}
\textsuperscript{56:2.3} Esta mente infinita y universal ejerce su ministerio en los universos del tiempo y del espacio bajo la forma de la mente cósmica; y aunque abarca desde el ministerio primitivo de los espíritus ayudantes hasta la magnífica mente del jefe ejecutivo de un universo, incluso esta mente cósmica está adecuadamente unificada en la supervisión de los Siete Espíritus Maestros, que están a su vez coordinados con la Mente Suprema del tiempo y del espacio y perfectamente correlacionados con la mente global del Espíritu Infinito.

\section*{3. La unificación espiritual}
\par
%\textsuperscript{(639.1)}
\textsuperscript{56:3.1} Al igual que la gravedad mental universal está centrada en la presencia personal paradisiaca del Espíritu Infinito, la gravedad espiritual universal tiene su centro en la presencia personal paradisiaca del Hijo Eterno. El Padre Universal es uno, pero para el espacio-tiempo se revela en los fenómenos duales de la energía pura y del puro espíritu.

\par
%\textsuperscript{(639.2)}
\textsuperscript{56:3.2} Las realidades espirituales del Paraíso son igualmente una, pero en todas las situaciones y relaciones espacio-temporales este espíritu único se revela en los fenómenos duales de las personalidades y emanaciones espirituales del Hijo Eterno, y en las personalidades e influencias espirituales del Espíritu Infinito y sus creaciones asociadas; y aún existe un tercer fenómeno ---las fragmentaciones del espíritu puro---, la donación, por parte del Padre, de los Ajustadores del Pensamiento y de otras entidades espirituales prepersonales.

\par
%\textsuperscript{(639.3)}
\textsuperscript{56:3.3} Cualquiera que sea el nivel de las actividades universales donde podáis encontrar los fenómenos espirituales o contactar con los seres espirituales, podéis saber que todos ellos proceden del Dios que es espíritu\footnote{\textit{Dios es espíritu}: Jn 4:24.} a través del ministerio del Hijo Espiritual y del Espíritu Mental Infinito. Este extenso espíritu actúa como fenómeno en los mundos evolutivos del tiempo según las directrices procedentes de las sedes de los universos locales. Desde estas capitales de los Hijos Creadores, el Espíritu Santo y el Espíritu de la Verdad, junto con el ministerio de los espíritus ayudantes de la mente, descienden hasta los niveles evolutivos inferiores de las mentes materiales.

\par
%\textsuperscript{(639.4)}
\textsuperscript{56:3.4} Aunque la mente está más unificada en el nivel de los Espíritus Maestros en asociación con el Ser Supremo y como mente cósmica subordinada a la Mente Absoluta, el ministerio espiritual para los mundos en evolución está más directamente unificado en las personalidades que residen en las sedes de los universos locales y en las personas de las Ministras Divinas que los presiden, las cuales están correlacionadas a su vez de una forma casi perfecta con el circuito gravitatorio paradisiaco del Hijo Eterno, donde se produce la unificación final de todas las manifestaciones espirituales en el espacio-tiempo.

\par
%\textsuperscript{(639.5)}
\textsuperscript{56:3.5} La existencia como criatura perfeccionada se puede alcanzar, mantener y eternizar gracias a la fusión de la mente autoconsciente con un fragmento de la dotación espiritual pretrinitaria de una de las personas de la Trinidad del Paraíso. La mente mortal es la creación de los Hijos y de las Hijas del Hijo Eterno y del Espíritu Infinito, y cuando fusiona con el Ajustador del Pensamiento procedente del Padre, comparte la triple dotación espiritual de los reinos evolutivos. Pero estas tres expresiones espirituales se unifican perfectamente en los finalitarios tal como estaban unificadas así, en la eternidad, en el YO SOY Universal antes de que se convirtiera en el Padre Universal del Hijo Eterno y del Espíritu Infinito.

\par
%\textsuperscript{(639.6)}
\textsuperscript{56:3.6} En última instancia, el espíritu debe expresarse siempre de manera triple y su realización final debe estar unificada con la Trinidad. El espíritu tiene su origen en una sola fuente por medio de una expresión triple; y al final debe alcanzar, y alcanza, su plena realización en esa unificación divina que se experimenta cuando se encuentra a Dios en la eternidad ---la unidad con la divinidad--- y por medio del ministerio de la mente cósmica de la expresión infinita de la palabra eterna del pensamiento universal del Padre.

\section*{4. La unificación de la personalidad}
\par
%\textsuperscript{(639.7)}
\textsuperscript{56:4.1} El Padre Universal es una personalidad divinamente unificada; por eso todos sus hijos ascendentes que son llevados hasta el Paraíso por el impulso de rebote de los Ajustadores del Pensamiento que salieron del Paraíso para residir en los mortales materiales obedeciendo al mandato del Padre, serán igualmente unas personalidades plenamente unificadas antes de llegar a Havona.

\par
%\textsuperscript{(640.1)}
\textsuperscript{56:4.2} La personalidad intenta de forma inherente unificar todas las realidades que la constituyen. La personalidad infinita de la Fuente-Centro Primera, del Padre Universal, unifica a los siete Absolutos que constituyen la Infinidad; y puesto que la personalidad del hombre mortal es un don exclusivo y directo del Padre Universal, posee igualmente el potencial de unificar los factores constituyentes de la criatura mortal. Esta creatividad unificadora que posee toda personalidad de criatura es una marca de nacimiento de su elevada fuente exclusiva y es una prueba adicional de su contacto ininterrumpido con esa misma fuente a través del circuito de la personalidad, gracias al cual la personalidad de la criatura mantiene un contacto directo y sostenido con el Padre de todas las personalidades que reside en el Paraíso.

\par
%\textsuperscript{(640.2)}
\textsuperscript{56:4.3} A pesar de que Dios se manifiesta desde los dominios del Séptuple, pasando por la supremacía y la ultimidad, hasta Dios Absoluto, el circuito de la personalidad, que está centrado en el Paraíso y en la persona de Dios Padre, asegura la unificación completa y perfecta de todas estas expresiones diversas de la personalidad divina en lo que se refiere a todas las personalidades de las criaturas en todos los niveles de existencia inteligente y en todos los reinos de los universos perfectos, perfeccionados y en vías de perfeccionarse.

\par
%\textsuperscript{(640.3)}
\textsuperscript{56:4.4} Aunque Dios es para los universos, y en los universos, todo lo que hemos descrito, sin embargo, para vosotros y para todas las otras criaturas que conocen a Dios es uno solo\footnote{\textit{Dios es uno}: 2 Re 19:19; 1 Cr 17:20; Neh 9:6; Sal 86:10; Eclo 36:5; Is 37:16; 44:6,8; 45:5-6,21; Dt 4:35,39; 6:4; Mc 12:29,32; Jn 17:3; Ro 3:30; 1 Co 8:4-6; Gl 3:20; Ef 4:6; 1 Ti 2:5; Stg 2:19; 1 Sam 2:2; 2 Sam 7:22.}, vuestro Padre y su Padre. Para una personalidad Dios no puede ser múltiple. Dios es Padre para cada una de sus criaturas, y es literalmente imposible que un hijo pueda tener más de un padre.

\par
%\textsuperscript{(640.4)}
\textsuperscript{56:4.5} Filosóficamente, cósmicamente y con relación a los niveles y lugares diferenciales de manifestación, podéis y debéis forzosamente concebir el funcionamiento de unas Deidades múltiples y presuponer la existencia de unas Trinidades múltiples; pero en la experiencia adoradora del contacto personal de cada personalidad que adora en todo el universo maestro, Dios es uno; y esta Deidad unificada y personal es nuestro padre paradisiaco, Dios Padre, el donador, el conservador y el Padre de todas las personalidades, desde el hombre mortal en los mundos habitados hasta el Hijo Eterno en la Isla central de Luz.

\section*{5. La unidad de la Deidad}
\par
%\textsuperscript{(640.5)}
\textsuperscript{56:5.1} La unidad, la indivisibilidad, de la Deidad del Paraíso es existencial y absoluta. Hay tres personalizaciones eternas de la Deidad ---el Padre Universal, el Hijo Eterno y el Espíritu Infinito--- pero en la Trinidad del Paraíso\footnote{\textit{La Trinidad}: Mt 28:19; Hch 2:32-33; 2 Co 13:14; 1 Jn 5:7. \textit{La antigua visión de Pablo acerca de la Trinidad}: 1 Co 12:4-6.} son \textit{en realidad} una sola Deidad, indivisa e indivisible.

\par
%\textsuperscript{(640.6)}
\textsuperscript{56:5.2} Desde el nivel Paraíso-Havona original de la realidad existencial, se han diferenciado dos niveles subabsolutos, y sobre ellos el Padre, el Hijo y el Espíritu han empezado la creación de numerosos asociados y subordinados personales. Y aunque a este respecto no es apropiado emprender el análisis de la unificación absonita de la deidad en los niveles trascendentales de la ultimidad, sí es factible examinar algunas características de la función unificadora de las diversas personalizaciones de la Deidad en quienes la divinidad se manifiesta funcionalmente a los diversos sectores de la creación y a las diferentes clases de seres inteligentes.

\par
%\textsuperscript{(640.7)}
\textsuperscript{56:5.3} El funcionamiento actual de la divinidad en los superuniversos se manifiesta activamente en las obras de los Creadores Supremos ---los Hijos y los Espíritus Creadores de los universos locales, los Ancianos de los Días de los superuniversos y los Siete Espíritus Maestros del Paraíso. Estos seres constituyen los tres primeros niveles de Dios Séptuple que conducen interiormente hacia el Padre Universal, y todo este dominio de Dios Séptuple se está coordinando en el primer nivel de la deidad experiencial en el Ser Supremo en evolución.

\par
%\textsuperscript{(641.1)}
\textsuperscript{56:5.4} En el Paraíso y en el universo central, la unidad de la Deidad es un hecho de la existencia. En todos los universos evolutivos del tiempo y del espacio, la unidad de la Deidad es una consecución.

\section*{6. La unificación de la Deidad evolutiva}
\par
%\textsuperscript{(641.2)}
\textsuperscript{56:6.1} Cuando las tres personas eternas de la Deidad actúan como una Deidad indivisa en la Trinidad del Paraíso, consiguen una unidad perfecta; del mismo modo, cuando crean, ya sea en asociación o por separado, su progenie paradisíaca muestra la unidad característica de la divinidad. Y esta divinidad de propósito que manifiestan los Creadores y los Gobernantes Supremos de los dominios espacio-temporales se traduce en el potencial unificante de poder de la soberanía de la supremacía experiencial que, en presencia de la unidad energética impersonal del universo, establece una tensión de la realidad que sólo se puede resolver mediante una unificación adecuada con las realidades experienciales de personalidad de la Deidad experiencial.

\par
%\textsuperscript{(641.3)}
\textsuperscript{56:6.2} Las realidades de personalidad del Ser Supremo proceden de las Deidades del Paraíso, y en el mundo piloto del circuito exterior de Havona se unifican con las prerrogativas de poder del Todopoderoso Supremo que provienen de las divinidades Creadoras del gran universo. Dios Supremo, como persona, existía en Havona antes de la creación de los siete superuniversos, pero sólo ejercía su actividad en los niveles espirituales. La evolución del poder Todopoderoso de la Supremacía mediante la síntesis diversa de la divinidad en los universos evolutivos se tradujo en una nueva presencia de poder de la Deidad que se coordinó con la persona espiritual del Supremo en Havona por medio de la Mente Suprema, la cual se trasladó simultáneamente desde el potencial que residía en la mente infinita del Espíritu Infinito a la mente funcional activa del Ser Supremo.

\par
%\textsuperscript{(641.4)}
\textsuperscript{56:6.3} Las criaturas con mentalidad material de los mundos evolutivos de los siete superuniversos sólo pueden comprender la unidad de la Deidad tal como está evolucionando en esta síntesis del poder y de la personalidad del Ser Supremo. En cualquier nivel de existencia, Dios no puede sobrepasar la capacidad conceptual de los seres que viven en ese nivel. A través del reconocimiento de la verdad, de la apreciación de la belleza y de la adoración de la bondad, el hombre mortal debe desarrollar el reconocimiento de un Dios de amor y luego progresar por los niveles ascendentes de la deidad hasta la comprensión del Supremo. Cuando se ha comprendido así que la Deidad está unificada en poder, entonces puede ser personalizada en espíritu para que las criaturas puedan comprenderla y alcanzarla.

\par
%\textsuperscript{(641.5)}
\textsuperscript{56:6.4} Aunque los mortales ascendentes consiguen comprender el poder del Todopoderoso en las capitales de los superuniversos y logran comprender la personalidad del Supremo en los circuitos exteriores de Havona, en verdad no encuentran al Ser Supremo del mismo modo que están destinados a encontrar a las Deidades del Paraíso. Ni siquiera los finalitarios, que son espíritus de la sexta fase, han encontrado al Ser Supremo, ni lo podrán encontrar probablemente hasta que no hayan alcanzado el estado espiritual de la séptima fase, y hasta que el Supremo no desempeñe realmente sus funciones en las actividades de los futuros universos exteriores.

\par
%\textsuperscript{(641.6)}
\textsuperscript{56:6.5} Pero cuando los ascendentes encuentran al Padre Universal como séptimo nivel de Dios Séptuple, han alcanzado la personalidad de la Primera Persona de \textit{todos} los niveles de las relaciones personales de la deidad con las criaturas del universo.

\section*{7. Las repercusiones evolutivas universales}
\par
%\textsuperscript{(642.1)}
\textsuperscript{56:7.1} El progreso continuo de la evolución en los universos del espacio-tiempo va acompañado de revelaciones cada vez más amplias de la Deidad para todas las criaturas inteligentes. Cuando se alcanza la cima del progreso evolutivo en un mundo, en un sistema, una constelación, un universo, un superuniverso o en el gran universo, este hecho señala un aumento correspondiente de la función de la deidad en esas unidades progresivas de la creación, y para ellas. Y todo aumento local de la comprensión de la divinidad va acompañado de ciertas repercusiones bien definidas de la manifestación más amplia de la deidad para todos los otros sectores de la creación. Partiendo del Paraíso hacia el exterior, cada nuevo dominio de la evolución realizada y alcanzada constituye una revelación nueva y más amplia de la Deidad experiencial para el universo de universos.

\par
%\textsuperscript{(642.2)}
\textsuperscript{56:7.2} A medida que las partes componentes de un universo local se establecen progresivamente en la luz y la vida, Dios Séptuple se manifiesta cada vez más. La evolución espacio-temporal empieza en un planeta bajo el control de la primera expresión de Dios Séptuple ---la asociación del Hijo Creador y del Espíritu Creativo. Con el establecimiento de un sistema en la luz, esta unión del Hijo y del Espíritu alcanza la plenitud de su función; y cuando una constelación entera se establece de esta forma, la segunda fase de Dios Séptuple se vuelve más activa en todo ese reino. La completa evolución administrativa de un universo local va acompañada de unos servicios nuevos y más directos de los Espíritus Maestros superuniversales; y en este punto también comienzan esa revelación y ese entendimiento crecientes de Dios Supremo que culminan en la comprensión del Ser Supremo por parte de los ascendentes mientras pasan por los mundos del sexto circuito de Havona.

\par
%\textsuperscript{(642.3)}
\textsuperscript{56:7.3} El Padre Universal, el Hijo Eterno y el Espíritu Infinito son manifestaciones existenciales de la deidad para las criaturas inteligentes, y por esta razón no se amplían del mismo modo en las relaciones de personalidad con las criaturas mentales y espirituales de toda la creación.

\par
%\textsuperscript{(642.4)}
\textsuperscript{56:7.4} Se debería tener en cuenta que los mortales ascendentes pueden experimentar la presencia impersonal de los niveles sucesivos de la Deidad mucho antes de que se vuelvan suficientemente espirituales y adecuadamente educados como para lograr reconocer de manera personal y experiencial a estas Deidades y ponerse en contacto con ellas como seres personales.

\par
%\textsuperscript{(642.5)}
\textsuperscript{56:7.5} Cada nuevo logro evolutivo dentro de un sector de la creación, así como cada nueva invasión del espacio por parte de las manifestaciones de la divinidad, van acompañados de ampliaciones simultáneas de la revelación funcional de la Deidad dentro de las unidades entonces existentes y previamente organizadas de toda la creación. Esta nueva invasión del trabajo administrativo de los universos y de las unidades que los componen no siempre puede parecer que se ejecuta de acuerdo exactamente con la técnica esbozada aquí, porque es costumbre enviar por adelantado unos grupos de administradores para que preparen el camino de las eras posteriores y sucesivas del nuevo supercontrol administrativo. Incluso Dios Último presagia su supercontrol trascendental sobre los universos durante las etapas más tardías de un universo local establecido en la luz y la vida.

\par
%\textsuperscript{(642.6)}
\textsuperscript{56:7.6} Es un hecho que, a medida que las creaciones del tiempo y del espacio se establecen progresivamente en el estado evolutivo, se observa un funcionamiento nuevo y más completo de Dios Supremo en concomitancia con una retirada correspondiente de las tres primeras manifestaciones de Dios Séptuple. Si el gran universo se estableciera en la luz y la vida, cuando esto sucediera ¿cuál sería entonces la futura función de los Hijos Creadores y de las Hijas Creativas, manifestaciones de Dios Séptuple, si Dios Supremo asume el control directo de estas creaciones del tiempo y del espacio? Estos organizadores y pioneros de los universos espacio-temporales, ¿serán liberados para realizar actividades similares en el espacio exterior? No lo sabemos, pero hacemos muchas especulaciones sobre estas materias y otras relacionadas.

\par
%\textsuperscript{(643.1)}
\textsuperscript{56:7.7} A medida que las fronteras de la Deidad experiencial se extienden hacia los dominios del Absoluto Incalificado, visualizamos la actividad de Dios Séptuple durante las épocas evolutivas iniciales de estas creaciones del futuro. No todos estamos de acuerdo en cuanto al estado futuro de los Ancianos de los Días y de los Espíritus Maestros de los superuniversos. Tampoco sabemos si el Ser Supremo actuará o no allí como en los siete superuniversos. Pero todos suponemos que los Migueles, los Hijos Creadores, están destinados a ejercer su actividad en esos universos exteriores. Algunos sostienen que las eras futuras presenciarán una forma de unión más estrecha entre los Hijos Creadores y las Ministras Divinas asociados; incluso es posible que esta unión de creadores pueda traducirse en alguna nueva expresión de identidad asociativo-creativa de naturaleza última. Pero en realidad no sabemos nada sobre estas posibilidades del futuro no revelado.

\par
%\textsuperscript{(643.2)}
\textsuperscript{56:7.8} Sin embargo, sí sabemos que en los universos del tiempo y del espacio Dios Séptuple facilita un acercamiento progresivo al Padre Universal, y que este acercamiento evolutivo está unificado experiencialmente en Dios Supremo. Podríamos suponer que este plan debería prevalecer en los universos exteriores; por otra parte, las nuevas órdenes de seres que algún día puedan habitar esos universos podrían ser capaces de acercarse a la Deidad en los niveles últimos y mediante técnicas absonitas. En resumen, no tenemos ni la más remota idea sobre la técnica que se empleará para acercarse a la deidad en los futuros universos del espacio exterior.

\par
%\textsuperscript{(643.3)}
\textsuperscript{56:7.9} Creemos, no obstante, que los superuniversos perfeccionados se convertirán de alguna manera en una parte de la carrera de ascensión al Paraíso de aquellos seres que puedan habitar esas creaciones exteriores. Es totalmente posible que en esa era futura podamos ver a los habitantes del espacio exterior acercarse a Havona a través de los siete superuniversos, administrados por Dios Supremo con o sin la colaboración de los Siete Espíritus Maestros.

\section*{8. El Unificador Supremo}
\par
%\textsuperscript{(643.4)}
\textsuperscript{56:8.1} El Ser Supremo tiene una triple función en la experiencia del hombre mortal: En primer lugar, es el unificador de Dios Séptuple, la divinidad espacio-temporal; en segundo lugar, él es lo máximo que las criaturas finitas pueden comprender realmente sobre la Deidad; en tercer lugar, es el único camino que tiene el hombre mortal para acercarse a la experiencia trascendental de asociarse con la mente absonita, el espíritu eterno y la personalidad paradisiaca.

\par
%\textsuperscript{(643.5)}
\textsuperscript{56:8.2} Puesto que los finalitarios ascendentes han nacido en los universos locales, se han nutrido en los superuniversos y se han capacitado en el universo central, en sus experiencias personales contienen todo el potencial necesario para comprender la divinidad espacio-temporal de Dios Séptuple, que se unifica en el Supremo. Los finalitarios prestan sus servicios sucesivos en unos superuniversos diferentes a los de su nacimiento, superponiendo así una experiencia tras otra hasta que engloben la plenitud de la séptuple diversidad de las experiencias posibles de las criaturas. Los finalitarios tienen la posibilidad de \textit{encontrar} al Padre Universal gracias al ministerio de los Ajustadores interiores, pero es por medio de estas técnicas experienciales como estos finalitarios llegan a \textit{conocer} realmente al Ser Supremo, y están destinados a servir y a \textit{revelar} a esta Deidad Suprema en los futuros universos del espacio exterior, y a ellos.

\par
%\textsuperscript{(644.1)}
\textsuperscript{56:8.3} Recordad que todo lo que Dios Padre y sus Hijos Paradisiacos hacen por nosotros, nosotros a nuestra vez y en espíritu tenemos la oportunidad de hacerlo por el Ser Supremo emergente, y en él. La experiencia del amor, de la alegría y del servicio en el universo es mutua. Dios Padre no necesita que sus hijos le devuelvan todo lo que les da, pero éstos a su vez dan (o pueden dar) todo esto a sus semejantes y al Ser Supremo en evolución.

\par
%\textsuperscript{(644.2)}
\textsuperscript{56:8.4} Todos los fenómenos pertenecientes a la creación reflejan unas actividades espirituales creadoras antecedentes. Jesús dijo, y es literalmente cierto, que <<el Hijo sólo hace aquellas cosas que ve hacer a su Padre>>\footnote{\textit{El Hijo imita al Padre}: Jn 5:19; 8:38.}. En el tiempo, vosotros los mortales podréis empezar a revelar el Supremo a vuestros semejantes, y podréis acrecentar cada vez más esta revelación a medida que ascendáis hacia el Paraíso. En la eternidad, quizás se os permita hacer revelaciones crecientes de este Dios de las criaturas evolutivas en los niveles supremos ---e incluso últimos--- cuando seáis finalitarios del séptimo grado.

\section*{9. La unidad universal absoluta}
\par
%\textsuperscript{(644.3)}
\textsuperscript{56:9.1} El Absoluto Incalificado y el Absoluto de la Deidad están unificados en el Absoluto Universal. Los Absolutos están coordinados en el Último, condicionados en el Supremo y modificados espacio-temporalmente en Dios Séptuple. En los niveles subinfinitos hay \textit{tres} Absolutos, pero en la infinidad parecen ser \textit{uno solo}. En el Paraíso hay tres personalizaciones de la Deidad, pero en la Trinidad \textit{son} una sola.

\par
%\textsuperscript{(644.4)}
\textsuperscript{56:9.2} El problema filosófico principal del universo maestro es el siguiente: ¿Existía el Absoluto (los tres Absolutos bajo la forma de uno solo en la infinidad) antes que la Trinidad? ¿Es el Absoluto el antecesor de la Trinidad, o es la Trinidad la antecedente del Absoluto?

\par
%\textsuperscript{(644.5)}
\textsuperscript{56:9.3} ¿Es el Absoluto Incalificado una presencia de fuerza independiente de la Trinidad? La presencia del Absoluto de la Deidad, ¿conlleva el funcionamiento ilimitado de la Trinidad? Y el Absoluto Universal, ¿es la función final de la Trinidad, o incluso una Trinidad de Trinidades?

\par
%\textsuperscript{(644.6)}
\textsuperscript{56:9.4} A primera vista, el concepto del Absoluto como antepasado de todas las cosas ---incluso de la Trinidad--- parece proporcionar la satisfacción transitoria de una gratificación coherente y de una unificación filosófica, pero cualquier conclusión de este tipo está invalidada por el hecho de que la eternidad de la Trinidad del Paraíso es una realidad. Se nos enseña, y nosotros lo creemos, que la naturaleza y la existencia del Padre Universal y de sus asociados de la Trinidad son eternas. No hay entonces más que una conclusión filosófica coherente, y es la siguiente: El Absoluto es, para todas las inteligencias del universo, la reacción impersonal y coordinada de la Trinidad (de Trinidades) hacia todas las situaciones fundamentales y primarias del espacio, en el interior y en el exterior de los universos. Para todas las inteligencias con personalidad del gran universo, la Trinidad del Paraíso se mantiene para siempre en finalidad, eternidad, supremacía y ultimidad, y a todos los efectos prácticos de la comprensión personal y de la realización de la criatura, es absoluta.

\par
%\textsuperscript{(644.7)}
\textsuperscript{56:9.5} Tal como la mente de la criatura puede considerar este problema, llega al postulado final de que el YO SOY Universal es la causa primordial y la fuente incondicional tanto de la Trinidad como del Absoluto. Por tanto, cuando anhelamos albergar un concepto personal del Absoluto, volvemos a nuestras ideas e ideales sobre el Padre Paradisiaco. Cuando deseamos facilitar la comprensión o aumentar la conciencia de este Absoluto por otra parte impersonal, volvemos al hecho de que el Padre Universal es el Padre existencial con personalidad absoluta; el Hijo Eterno es la Persona Absoluta aunque no es, en el sentido experiencial, la personalización del Absoluto. Luego pasamos a visualizar que las Trinidades experienciales culminan en la personalización experiencial del Absoluto de la Deidad, mientras concebimos que el Absoluto Universal constituye los fenómenos universales y extrauniversales de la presencia manifiesta de las actividades impersonales de las asociaciones unificadas y coordinadas de supremacía, de ultimidad y de infinidad de la Deidad ---la Trinidad de Trinidades.

\par
%\textsuperscript{(645.1)}
\textsuperscript{56:9.6} Dios Padre es discernible en todos los niveles, desde el finito hasta el infinito, y aunque sus criaturas, desde las del Paraíso hasta las de los mundos evolutivos, lo han percibido de maneras diversas, sólo el Hijo Eterno y el Espíritu Infinito lo conocen como infinidad.

\par
%\textsuperscript{(645.2)}
\textsuperscript{56:9.7} La personalidad espiritual sólo es absoluta en el Paraíso, y el concepto del Absoluto sólo es incondicional en la infinidad. La presencia de la Deidad sólo es absoluta en el Paraíso, y la revelación de Dios siempre ha de ser parcial, relativa y progresiva hasta que su poder se vuelva experiencialmente infinito en la potencia espacial del Absoluto Incalificado, la manifestación de su personalidad se vuelva experiencialmente infinita en la presencia manifiesta del Absoluto de la Deidad, y estos dos potenciales de la infinidad se vuelvan unificados en una sola realidad en el Absoluto Universal.

\par
%\textsuperscript{(645.3)}
\textsuperscript{56:9.8} Pero más allá de los niveles subinfinitos, los tres Absolutos \textit{son} uno solo, y por eso la infinidad es comprendida por la Deidad, sin tener en cuenta que cualquiera otra orden de existencia pueda nunca tener conciencia de la infinidad.

\par
%\textsuperscript{(645.4)}
\textsuperscript{56:9.9} El estado existencial en la eternidad implica una auto-conciencia existencial de la infinidad, aunque haga falta otra eternidad para experimentar la comprensión de las potencialidades experienciales inherentes a una eternidad de infinidad ---a una infinidad eterna.

\par
%\textsuperscript{(645.5)}
\textsuperscript{56:9.10} Dios Padre es la fuente personal de todas las manifestaciones de la Deidad y de la realidad para todas las criaturas inteligentes y seres espirituales en todo el universo de universos. Como personalidades, ahora o en las experiencias universales sucesivas del eterno futuro, sin importar que logréis alcanzar a Dios Séptuple, comprender a Dios Supremo, encontrar a Dios Último o intentéis captar el concepto de Dios Absoluto, descubriréis para vuestra satisfacción eterna que al culminar cada aventura habréis vuelto a descubrir, en nuevos niveles experienciales, al Dios eterno ---al Padre Paradisíaco de todas las personalidades del universo.

\par
%\textsuperscript{(645.6)}
\textsuperscript{56:9.11} El Padre Universal es la explicación de la unidad universal tal como ésta debe ser comprendida de manera suprema, e incluso última, en la unidad post-última de los valores y significados absolutos ---la Realidad incondicional.

\par
%\textsuperscript{(645.7)}
\textsuperscript{56:9.12} Los Organizadores de la Fuerza Maestros salen al espacio y movilizan las energías espaciales para hacerlas gravitatoriamente sensibles a la atracción paradisiaca del Padre Universal; posteriormente llegan los Hijos Creadores, que organizan estas fuerzas sensibles a la gravedad en universos habitados, donde producen por evolución criaturas inteligentes que reciben dentro de sí mismas el espíritu del Padre Paradisiaco, y ascienden ulteriormente hacia el Padre para volverse como él en todos los atributos posibles de la divinidad.

\par
%\textsuperscript{(645.8)}
\textsuperscript{56:9.13} El avance incesante y creciente de las fuerzas creativas del Paraíso a través del espacio parece presagiar el ámbito en constante expansión de la atracción gravitatoria del Padre Universal y la multiplicación sin fin de los diversos tipos de criaturas inteligentes que son capaces de amar a Dios y de ser amadas por él, y que, al conocer así a Dios, pueden escoger parecerse a él, pueden elegir alcanzar el Paraíso y encontrar a Dios.

\par
%\textsuperscript{(646.1)}
\textsuperscript{56:9.14} El universo de universos está completamente unificado. Dios es uno en poder y en personalidad. Todos los niveles de la energía y todas las fases de la personalidad están coordinados. Filosófica y experiencialmente, en concepto y en la realidad, todas las cosas y todos los seres tienen su centro en el Padre Paradisiaco. Dios es todo y está en todo, y ninguna cosa y ningún ser existen sin él\footnote{\textit{Dios es todo y está en todo}: Hch 17:28; Ro 11:36; 1 Co 8:6; 12:6; 15:28; Ef 1:23; 4:6; Col 1:17; 3:11; Heb 2:10-11.}.

\section*{10. La verdad, la belleza y la bondad}
\par
%\textsuperscript{(646.2)}
\textsuperscript{56:10.1} A medida que los mundos establecidos en la luz y la vida progresan desde la etapa inicial hasta la séptima época, tratan sucesivamente de comprender la realidad de Dios Séptuple, extendiéndose desde la adoración del Hijo Creador hasta la veneración de su Padre Paradisiaco. Durante toda la séptima etapa de la historia de un mundo de este tipo, los mortales en constante progreso crecen en el conocimiento de Dios Supremo, mientras disciernen vagamente la realidad del ministerio eclipsante de Dios Último.

\par
%\textsuperscript{(646.3)}
\textsuperscript{56:10.2} Durante toda esta época gloriosa, la ocupación principal de los mortales que progresan es la búsqueda de una mejor comprensión y de una apreciación más completa de los elementos comprensibles de la Deidad ---la verdad, la belleza y la bondad. Esto representa el esfuerzo del hombre por discernir a Dios en la mente, la materia y el espíritu. Y a medida que los mortales continúan esta búsqueda, se encuentran cada vez más sumergidos en el estudio experiencial de la filosofía, la cosmología y la divinidad.

\par
%\textsuperscript{(646.4)}
\textsuperscript{56:10.3} Captáis un poco la filosofía, y comprendéis a la divinidad en la adoración, el servicio social y la experiencia espiritual personal, pero la búsqueda de la belleza ---la cosmología--- la limitáis con demasiada frecuencia al estudio de los rudimentarios esfuerzos artísticos del hombre. La belleza, el arte, es sobre todo una cuestión de unificación de contrastes. La variedad es esencial para el concepto de la belleza. La belleza suprema, la cima del arte finito, es el drama de la unificación de la inmensidad de los extremos cósmicos que son el Creador y la criatura. El hombre que encuentra a Dios y Dios que encuentra al hombre ---la criatura que se vuelve perfecta como lo es el Creador--- ésta es la realización celestial de lo supremamente hermoso, esto es alcanzar la cúspide del arte cósmico.

\par
%\textsuperscript{(646.5)}
\textsuperscript{56:10.4} Por eso el materialismo, el ateísmo, es el colmo de la fealdad, la cúspide de la antítesis finita de lo bello. La belleza más elevada consiste en el panorama de la unificación de las variaciones que han nacido de una realidad armoniosa preexistente.

\par
%\textsuperscript{(646.6)}
\textsuperscript{56:10.5} Alcanzar unos niveles cosmológicos de pensamiento incluye:

\par
%\textsuperscript{(646.7)}
\textsuperscript{56:10.6} 1. \textit{La curiosidad}. El hambre de armonía y la sed de belleza. Los intentos persistentes por descubrir nuevos niveles de relaciones cósmicas armoniosas.

\par
%\textsuperscript{(646.8)}
\textsuperscript{56:10.7} 2. \textit{La apreciación estética}. El amor de lo bello y la apreciación creciente del toque artístico que existe en todas las manifestaciones creativas en todos los niveles de la realidad.

\par
%\textsuperscript{(646.9)}
\textsuperscript{56:10.8} 3. \textit{La sensibilidad ética}. Mediante la comprensión de la verdad, la apreciación de la belleza conduce al sentido de la adecuación eterna de aquellas cosas que inciden en el reconocimiento de la bondad divina en las relaciones de la Deidad con todos los seres; de este modo, incluso la cosmología conduce a la búsqueda de los valores divinos de la realidad ---a la conciencia de Dios.

\par
%\textsuperscript{(646.10)}
\textsuperscript{56:10.9} Los mundos establecidos en la luz y la vida se interesan tanto por comprender la verdad, la belleza y la bondad porque estos valores cualitativos engloban la revelación de la Deidad a los reinos del tiempo y del espacio. Los significados de la verdad eterna ejercen una atracción combinada sobre las naturalezas intelectual y espiritual del hombre mortal. La belleza universal abarca las relaciones y los ritmos armoniosos de la creación cósmica; esto constituye más claramente la atracción intelectual y conduce a la comprensión unificada y sincrónica del universo material. La bondad divina representa la revelación de los valores infinitos a la mente finita, para que sean percibidos y elevados allí hasta el umbral mismo del nivel espiritual de la comprensión humana.

\par
%\textsuperscript{(647.1)}
\textsuperscript{56:10.10} La verdad es la base de la ciencia y de la filosofía, y representa el fundamento intelectual de la religión. La belleza patrocina el arte, la música y los ritmos significativos de toda experiencia humana. La bondad engloba el sentido de la ética, la moralidad y la religión ---el hambre de perfección experiencial.

\par
%\textsuperscript{(647.2)}
\textsuperscript{56:10.11} La existencia de la belleza implica la presencia de una mente de criatura que la aprecie, tan ciertamente como el hecho de que la evolución progresiva indica la dominación de la Mente Suprema. La belleza es el reconocimiento intelectual de la síntesis espacio-temporal armoniosa de la extensa diversificación de la realidad fenoménica, cuya totalidad es el resultado de una unidad preexistente y eterna.

\par
%\textsuperscript{(647.3)}
\textsuperscript{56:10.12} La bondad es el reconocimiento mental de los valores relativos de los diversos niveles de la perfección divina. El reconocimiento de la bondad implica una mente con categoría moral, una mente personal con la capacidad de discriminar entre el bien y el mal. Pero la posesión de la bondad, la grandeza, es la medida del verdadero logro de la divinidad.

\par
%\textsuperscript{(647.4)}
\textsuperscript{56:10.13} El reconocimiento de las \textit{verdaderas relaciones} implica una mente capaz de discriminar entre la verdad y el error. El Espíritu de la Verdad otorgado, que envuelve a las mentes humanas de Urantia, reacciona infaliblemente a la verdad ---la relación espiritual viviente entre todas las cosas y todos los seres tal como están coordinados en la ascensión eterna hacia Dios.

\par
%\textsuperscript{(647.5)}
\textsuperscript{56:10.14} Cada impulso de cada electrón, pensamiento o espíritu es una unidad que actúa en todo el universo. Sólo el pecado es una resistencia gravitatoria aislada y nociva en los niveles mentales y espirituales. El universo es un todo; ninguna cosa y ningún ser existe o vive en el aislamiento. La auto-realización es potencialmente mala si es antisocial. Es literalmente cierto que <<ningún hombre vive para sí mismo>>\footnote{\textit{Ningún hombre vive para sí mismo}: Ro 14:7.}. La adaptación a la sociedad cósmica constituye la forma más elevada de unificación de la personalidad. Jesús dijo: <<Aquél de vosotros que quiera ser el más grande, que sea el servidor de todos>>\footnote{\textit{Quien quiera ser el más grande, que sea el servidor}: Mt 20:26-27; 23:11-12; Mc 9:35; 10:43-44; Lc 22:26.}.

\par
%\textsuperscript{(647.6)}
\textsuperscript{56:10.15} Incluso la verdad, la belleza y la bondad ---el acercamiento intelectual del hombre al universo mental, material y espiritual--- deben estar combinadas en un concepto unificado de un \textit{ideal} divino y supremo. Al igual que la personalidad mortal unifica la experiencia humana con la materia, la mente y el espíritu, este ideal divino y supremo se unifica con el poder en la Supremacía y luego se personaliza como un Dios de amor paternal.

\par
%\textsuperscript{(647.7)}
\textsuperscript{56:10.16} Cualquier idea que se tenga sobre las relaciones entre las partes y un todo determinado necesita una captación comprensiva de la relación entre todas las partes y ese todo; en el universo esto significa la relación de las partes creadas con el Todo Creador. La Deidad se convierte así en la meta trascendental, e incluso infinita, de la consecución universal y eterna.

\par
%\textsuperscript{(647.8)}
\textsuperscript{56:10.17} La belleza universal es el reconocimiento del reflejo de la Isla del Paraíso en la creación material, mientras que la verdad eterna es el ministerio especial de los Hijos Paradisíacos que no sólo se donan a las razas mortales, sino que incluso derraman su Espíritu de la Verdad sobre todos los pueblos. La bondad divina se manifiesta más plenamente en el ministerio amoroso de las múltiples personalidades del Espíritu Infinito. Pero el amor, la suma total de estas tres cualidades, es la percepción que el hombre tiene de Dios como su Padre espiritual.

\par
%\textsuperscript{(648.1)}
\textsuperscript{56:10.18} La materia física es la sombra espacio-temporal del resplandor energético paradisiaco de las Deidades absolutas. Los significados de la verdad son las repercusiones en el intelecto humano de la palabra eterna de la Deidad ---la comprensión espacio-temporal de los conceptos supremos. Los valores de bondad de la divinidad son los ministerios misericordiosos de las personalidades espirituales del Universal, del Eterno y del Infinito para con las criaturas espacio-temporales finitas de las esferas evolutivas.

\par
%\textsuperscript{(648.2)}
\textsuperscript{56:10.19} Estos significativos valores de realidad de la divinidad están mezclados, bajo la forma de amor divino, en las relaciones del Padre con cada criatura personal. Están coordinados en el Hijo y en sus Hijos bajo la forma de misericordia divina. Manifiestan sus cualidades a través del Espíritu y de sus hijos espirituales bajo la forma del ministerio divino, la demostración de la misericordia amorosa hacia los hijos del tiempo. El Ser Supremo manifiesta principalmente estas tres divinidades bajo la forma de la síntesis del poder con la personalidad. Dios Séptuple las da a conocer de diversas maneras en siete asociaciones diferentes de significados y de valores divinos en siete niveles ascendentes.

\par
%\textsuperscript{(648.3)}
\textsuperscript{56:10.20} Para el hombre finito, la verdad, la belleza y la bondad abarcan la revelación completa de la realidad de la divinidad. A medida que esta comprensión de que la Deidad es amor\footnote{\textit{Dios es amor}: 1 Jn 4:8,16.} encuentra su expresión espiritual en la vida de los mortales que conocen a Dios, se producen los frutos de la divinidad: la paz intelectual, el progreso social, la satisfacción moral, la alegría espiritual y la sabiduría cósmica. Los mortales avanzados de un mundo en la séptima etapa de luz y de vida han aprendido que el amor es la cosa más grande del universo ---y saben que Dios es amor.

\par
%\textsuperscript{(648.4)}
\textsuperscript{56:10.21} El amor es el deseo de hacer el bien a los demás.

\par
%\textsuperscript{(648.5)}
\textsuperscript{56:10.22} [Presentado por un Mensajero Poderoso de visita en Urantia, a petición del Cuerpo Revelador de Nebadon y en colaboración con cierto Melquisedek, Príncipe Planetario vicegerente de Urantia.]

\par
%\textsuperscript{(648.6)}
\textsuperscript{56:10.23} Este documento sobre la Unidad Universal es el vigésimo quinto de una serie de presentaciones efectuadas por diversos autores y que han sido patrocinadas, como grupo, por una comisión de doce personalidades de Nebadon que han actuado bajo la dirección de Mantutia Melquisedek. Estas narraciones las redactamos y las tradujimos a la lengua inglesa, mediante una técnica autorizada por nuestros superiores, en el año 1934 del tiempo de Urantia.

\newpage
\pagestyle{empty}

\par {\huge Abreviaturas}
\bigbreak
\bigbreak
\begin{multicols}{2}
	\par LU \textit{(El Libro de Urantia)}
	\bigbreak
	\par Libros bíblicos:
	\bigbreak
	\par Abd \textit{(Abdías)}
	\par Am \textit{(Amós)}
	\par Ap \textit{(Apocalipsis)}
	\par Bar \textit{(Baruc)}
	\par Co \textit{(Epístola a los Corintios)}
	\par Cnt \textit{(El Cantar de los Cantares)}
	\par Col \textit{(Epístola a los Colosenses)}
	\par Cr \textit{(Crónicas)}
	\par Dn \textit{(Daniel)}
	\par Dt \textit{(Deuteronomio)}
	\par Ec \textit{(Eclesiastés)}
	\par Eclo \textit{(Ecclesiástico)}
	\par Ef \textit{(Epístola a los Efesios)}
	\par Esd \textit{(Esdras)}
	\par Est \textit{(Ester)}
	\par Ex \textit{(Éxodo)}
	\par Ez \textit{(Ezequiel)} 
	\par Flm \textit{(Epístola a Filemón)}
	\par Flp \textit{(Epístola a los Filipenses)}
	\par Gl \textit{(Epítosla a los Gálatas)}
	\par Gn \textit{(Génesis)}
	\par Hab \textit{(Habacuc)} 
	\par Hag \textit{(Ageo)}
	\par Hch \textit{(Hechos de los Apóstoles)}
	\par Heb \textit{(Epístola a los Hebreos)}
	\par Is \textit{(Isaías)}
	\par Jer \textit{(Jeremías)}
	\par Jl \textit{(Joel)}
	\par Jn \textit{(Juan, evangelio y epístolas)}
	\par Job \textit{(Job)}
	\par Jon \textit{(Jonás)}
	\par Jos \textit{(Josué)}
	\par Jud \textit{(Epístola de Judas)}
	\par Jue \textit{(Jueces)}
	\par Lc \textit{(Lucas)}
	\par Lm \textit{(Lamentaciones)}
	\par Lv \textit{(Levítico)}
	\par Mac \textit{(Macabeos)}
	\par Mal \textit{(Malaquías)}
	\par Mc \textit{(Marcos)}
	\par Miq \textit{(Miqueas)} 
	\par Mt \textit{(Mateo)}
	\par Nah \textit{(Nahúm)}
	\par Neh \textit{(Nehemías)} 
	\par Nm \textit{(Números)}
	\par Os \textit{(Oseas)}
	\par P \textit{(Epístola de Pedro)}
	\par Pr \textit{(Proverbios)}
	\par Re \textit{(Reyes)}
	\par Ro \textit{(Epístola a los Romanos)}
	\par Rt \textit{(Rut)}
	\par Sab \textit{(Sabiduría)}
	\par Sal \textit{(Salmos)}
	\par Sam \textit{(Samuel)}
	\par Sof \textit{(Sofonías)}
	\par Stg \textit{(Epístola a Santiago)}
	\par Ti \textit{(Epístola a Timoteo)}
	\par Tit \textit{(Epítosla a Tito)}
	\par Ts \textit{(Epístola a los Tesalonicenses)}
	\par Zac \textit{(Zacarías)}
	\bigbreak
	\par Libros bíblicos apócrifos:
	\bigbreak 
	\par AsMo \textit{(Asunción de Moisés)}
	\par Bel \textit{(Bel y el Dragón)} 
	\par Hen \textit{(Enoc)} 
	\par Man \textit{(Oración de Manasés)} 
	\par Tb \textit{(Tobit)}
	\bigbreak
	\par Libros de otras religiones: 
	\bigbreak
	\par XXX \textit{(YYYY)}
	
	
\end{multicols}


\end{document}
