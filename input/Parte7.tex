% Author of this conversion to LaTeX format: Jan Herca, 2017
\documentclass[twoside, 11pt]{book}
\usepackage[T1]{fontenc} % indica al procesador cómo imprimir los caracteres
\usepackage{fontspec} % permite definir fuentes a partir de las instaladas en el SO
\usepackage{geometry}
\usepackage{graphicx}
\usepackage{float}
\usepackage{tocloft}
\usepackage{titleps}
\usepackage{emptypage}
\usepackage[spanish]{babel}
\usepackage{multicol}
% Text styles
\geometry{paperwidth=16cm, paperheight=24cm, top=2.5cm, bottom=1.7cm, inner=2.5cm, outer=1.2cm}

\makeatletter
\def\@makechapterhead#1{%
	\vspace*{50\p@}%
	{\parindent \z@ \raggedright \normalfont
		\interlinepenalty\@M
		\huge \bfseries #1\par\nobreak
		\vskip 40\p@
}}
\def\@makeschapterhead#1{%
	\vspace*{50\p@}%
	{\parindent \z@ \raggedright
		\normalfont
		\interlinepenalty\@M
		\huge \bfseries  #1\par\nobreak
		\vskip 40\p@
}}
\makeatother

\renewcommand{\cftchapleader}{\cftdotfill{\cftdotsep}}
\renewcommand{\thechapter}{}
\renewcommand{\cftchapfont}{\large}
\cftsetpnumwidth{3em}
\renewcommand{\cftchappagefont}{\large}



\title{La Quinta Revelación \newline Séptimo Volumen \newline La Vida y las Enseñanzas de Jesús - II}
\date{}
\begin{document}
	
\begin{titlepage}
	\centering
	{\Huge\bfseries El Libro de Urantia\par}
	{\huge\bfseries La Quinta Revelación\par}
	\vspace{1cm}
	{\huge\bfseries Séptimo Volumen\par}
	\vspace{1cm}
	{\huge\bfseries La Vida y las Enseñanzas de Jesús\par}
	{\huge\bfseries II\par}
	\vfill
	{\scshape\Large URANTIA FOUNDATION\par}
	{\scshape\Large CHICAGO ILLINOIS\par}
	{\Large 2009 Traducción al español Europea\par}
\end{titlepage}
	
	
\par {\textcopyright} 2019 Jan Herca, de la edición
\par {\textcopyright} 2009 Urantia Foundation, de la traducción
\par {\textcopyright} 1993 Urantia Foundation, de otros materiales
\bigbreak
\par Jan Herca
\par Correo electrónico: janherca@gmail.com
\bigbreak
\par Urantia Foundation
\par 533 West Diversey Parkway
\par Chicago, IL 60614 EE.UU.A
\par Oficina: 1+(773) 525-3319
\par Fax: 1 +(773) 525-7739
\par Website: http://www.urantia.org
\par Correo electrónico: urantia@urantia.org
\bigbreak
\par Todos los derechos reservados, incluyendo el de traducción en los Estados Unidos de América, Canadá y en los demás países de la Unión Internacional de copyright. Todos los derechos reservados en los paises firmantes de la Union Panamericana de la Union internacional de copyright.
\par No todo el libro ni parte de él pueden ser copiados, reproducidos o traducidos en forma alguna, ya sea por medio electrónico, mecánico u otra forma, como fotocopia, grabación o archivo computerizado sin autorización por escrito del editor.
\par URANTIA,'' ``URANTIAN,'' ``EL LIBRO DE URANTIA'' y son marcas registradas de Urantia Foundation y su uso está sujeto a licencia.
\bigbreak
\par La Quinta Revelación es una reedición de El Libro de Urantia (Edición Europea). Está dividido en siete volúmenes para hacerlo más manejable y dispone de contenido adicional en forma de ayudas a la lectura integradas en el texto. El Libro de Urantia (Edición Europea) es una traducción de The Urantia Book realizada por la Fundación Urantia en 2009. 
\newpage

\begin{center}
	{\huge\bfseries Las partes del libro\par}
	\vspace{1cm}
	{\scshape\large PRIMER VOLUMEN\par}
	{\scshape\Large DIOS, EL UNIVERSO CENTRAL Y LOS SUPERUNIVERSOS\par}
	\vspace{1cm}
	
	{\scshape\large SEGUNDO VOLUMEN \par}
	{\scshape\Large EL UNIVERSO LOCAL\par}
	\vspace{1cm}
	
	{\scshape\large TERCER VOLUMEN \par}
	{\scshape\Large LA HISTORIA DE NUESTRO PLANETA, URANTIA\par}
	\vspace{1cm}
	
	{\scshape\large CUARTO VOLUMEN \par}
	{\scshape\Large LA EVOLUCIÓN DE LA CIVILIZACIÓN HUMANA\par}
	\vspace{1cm}
	
	{\scshape\large QUINTO VOLUMEN \par}
	{\scshape\Large LA RELIGIÓN, LA SOBREVIVENCIA A LA MUERTE Y LA DEIDAD EXPERIENCIAL\par}
	\vspace{1cm}
	
	{\scshape\large SEXTO VOLUMEN \par}
	{\scshape\Large LA VIDA Y LAS ENSEÑANZAS DE JESÚS - I\par}
	\vspace{1cm}
	
	{\scshape\large SÉPTIMO VOLUMEN \par}
	{\scshape\Large LA VIDA Y LAS ENSEÑANZAS DE JESÚS - II\par}
\end{center}
	
\newpage
\begin{center}
	{\small \textit {Intencionadamente en blanco}\par}
\end{center}
\newpage

\pagestyle{empty}

\tableofcontents


\newpagestyle{main}{
	%\setheadrule{.4pt}% Header rule
	%\setfootrule{.4pt}% Footer rule
	\sethead[\small \thepage]% odd-left
	[]% odd-center
	[\begin{minipage}{0.9\textwidth}\begin{flushright}\scriptsize \MakeUppercase{\chaptertitle}\end{flushright}\end{minipage}]% odd-right
	{\begin{minipage}{0.9\textwidth}\scriptsize \MakeUppercase{\chaptertitle}\end{minipage}}% even-left
	{}% even-center
	{\small \thepage}% even-right
	\setfoot[]% odd-left
	[]% odd-center
	[]% odd-right
	{}% even-left
	{}% even-center
	{}% even-right
}
\pagestyle{main}
\renewcommand{\makeheadrule}{\rule[-.6\baselineskip]{\linewidth}{.4pt}}


\chapter{Documento 153. La crisis en Cafarnaúm}
\par 
%\textsuperscript{(1707.1)}
\textsuperscript{153:0.1} EL VIERNES por la noche, día de su llegada a Betsaida, y el sábado por la mañana, los apóstoles observaron que Jesús estaba seriamente ocupado en algún problema de gran importancia; se daban cuenta de que el Maestro reflexionaba de manera poco habitual en algún asunto importante. No tomó su desayuno y comió poco al mediodía. Todo el sábado por la mañana y la noche anterior, los doce y sus compañeros se habían reunido en pequeños grupos alrededor de la casa, en el jardín y a lo largo de la playa. Pesaba sobre todos ellos la tensión de la incertidumbre y la ansiedad del temor. Jesús les había dicho poca cosa desde que salieron de Jerusalén.

\par 
%\textsuperscript{(1707.2)}
\textsuperscript{153:0.2} Hacía meses que no veían al Maestro tan preocupado y tan poco comunicativo. Incluso Simón Pedro estaba deprimido, si no abatido. Andrés no sabía qué hacer por sus asociados desanimados. Natanael dijo que estaban en medio de la <<calma antes de la tormenta>>. Tomás expresó la opinión de que <<algo fuera de lo común está a punto de suceder>>. Felipe aconsejó a David Zebedeo que <<se olvidara de los planes para alimentar y alojar a la multitud, hasta que sepamos en qué está pensando el Maestro>>. Mateo se ocupaba con renovado esfuerzo en reaprovisionar la tesorería. Santiago y Juan conversaban sobre el próximo sermón en la sinagoga y hacían muchas especulaciones sobre su probable naturaleza y alcance. Simón Celotes expresaba la creencia, en realidad la esperanza, de que <<el Padre que está en los cielos puede estar a punto de intervenir de manera inesperada para justificar y sostener a su Hijo>>, mientras que Judas Iscariote se atrevía a abrigar el pensamiento de que Jesús estaba posiblemente abrumado por los remordimientos, por <<no haber tenido el coraje y la osadía de permitir a los cinco mil que lo proclamaran rey de los judíos>>.

\par 
%\textsuperscript{(1707.3)}
\textsuperscript{153:0.3} Aquella hermosa tarde de sábado, Jesús salió de este grupo de seguidores deprimidos y apesadumbrados para predicar su memorable sermón en la sinagoga de Cafarnaúm. Las únicas palabras de saludo jovial o buenos deseos que recibió de sus discípulos inmediatos provinieron de uno de los confiados gemelos Alfeo, que, cuando Jesús salía de la casa camino de la sinagoga, lo saludó alegremente, diciendo: <<Oramos para que el Padre te ayude, y para que podamos tener unas multitudes más grandes que nunca>>.

\section*{1. La preparación del escenario}
\par 
%\textsuperscript{(1707.4)}
\textsuperscript{153:1.1} Una asamblea distinguida recibió a Jesús a las tres de la tarde de este precioso sábado en la nueva sinagoga de Cafarnaúm. Jairo presidía y entregó las Escrituras a Jesús para la lectura. El día anterior, cincuenta y tres fariseos y saduceos habían llegado de Jerusalén; también estaban presentes más de treinta jefes y dirigentes de las sinagogas vecinas. Estos jefes religiosos judíos actuaban directamente bajo las órdenes del sanedrín de Jerusalén, y constituían la vanguardia ortodoxa que había venido para iniciar una guerra abierta contra Jesús y sus discípulos. Al lado de estos dirigentes judíos, en los asientos de honor de la sinagoga, estaban sentados los observadores oficiales de Herodes Antipas, el cual les había ordenado que averiguaran la verdad sobre los inquietantes rumores de que el pueblo había intentado proclamar a Jesús rey de los judíos en los dominios de su hermano Felipe.

\par 
%\textsuperscript{(1708.1)}
\textsuperscript{153:1.2} Jesús comprendía que iba a enfrentarse con la declaración inmediata de una guerra manifiesta y abierta por parte de sus enemigos cada vez más numerosos, y eligió audazmente emprender la ofensiva. Cuando alimentó a los cinco mil, había desafiado sus ideas sobre el Mesías material; ahora, decidió de nuevo atacar abiertamente sus conceptos del libertador judío. Esta crisis, que comenzó con la alimentación de los cinco mil y terminó con el sermón de este sábado por la tarde, marcó el momento en que se redujo la corriente de la fama y de las aclamaciones populares. De ahora en adelante, el trabajo del reino iba a ocuparse cada vez más de la tarea más importante de ganar conversos espirituales duraderos para la fraternidad verdaderamente religiosa de la humanidad. Este sermón marcó la crisis de transición entre el período de discusión, controversia y decisión, y el de la guerra abierta, con la aceptación final o el rechazo definitivo.

\par 
%\textsuperscript{(1708.2)}
\textsuperscript{153:1.3} El Maestro sabía muy bien que muchos de sus seguidores estaban preparándose mentalmente, de manera lenta pero segura, para rechazarlo definitivamente. También sabía que muchos de sus discípulos estaban pasando, de manera lenta pero segura, por esa preparación de la mente y esa disciplina del alma que les permitiría triunfar sobre las dudas y afirmar valientemente su fe completa en el evangelio del reino. Jesús comprendía plenamente cómo se preparan los hombres para las decisiones de una crisis y para llevar a cabo acciones repentinas basadas en elecciones valientes, mediante el lento proceso de elegir reiteradamente entre el bien y el mal en las situaciones recurrentes. A sus mensajeros elegidos los sometió a repetidas desilusiones y les proporcionó frecuentes oportunidades de pruebas para que escogieran entre la buena y la mala manera de enfrentarse a las dificultades espirituales. Sabía que podía confiar en sus seguidores, que cuando se enfrentaran con la prueba final, tomarían sus decisiones esenciales de acuerdo con las actitudes mentales y las reacciones espirituales habituales adquiridas anteriormente.

\par 
%\textsuperscript{(1708.3)}
\textsuperscript{153:1.4} Esta crisis en la vida terrestre de Jesús empezó con la alimentación de los cinco mil y terminó con este sermón en la sinagoga; la crisis en la vida de los apóstoles empezó con este sermón en la sinagoga y continuó durante un año entero, terminando solamente con el juicio y la crucifixión del Maestro.

\par 
%\textsuperscript{(1708.4)}
\textsuperscript{153:1.5} Aquella tarde, mientras estaban sentados allí en la sinagoga, antes de que Jesús empezara a hablar, en la mente de todos sólo había un gran misterio, una pregunta suprema. Tanto sus amigos como sus enemigos tenían un solo pensamiento: <<¿Por qué él mismo hizo retroceder tan deliberada y eficazmente la corriente del entusiasmo popular?>> Fue inmediatamente antes y después de este sermón cuando las dudas y las decepciones de sus partidarios descontentos se convirtieron en una oposición inconsciente que finalmente se transformó en un verdadero odio. Fue después de este sermón en la sinagoga cuando Judas Iscariote pensó conscientemente por primera vez en desertar. Pero, por el momento, supo dominar eficazmente todas estas inclinaciones.

\par 
%\textsuperscript{(1708.5)}
\textsuperscript{153:1.6} Todos estaban perplejos. Jesús los había dejado confundidos y desconcertados. Recientemente había emprendido la mayor demostración de poder sobrenatural de toda su carrera. La alimentación de los cinco mil fue el único acontecimiento de su vida terrestre que más se acercó al concepto judío del Mesías esperado. Pero esta ventaja extraordinaria fue contrarrestada de manera inmediata e inexplicable por su negativa resuelta e inequívoca a ser proclamado rey.

\par 
%\textsuperscript{(1709.1)}
\textsuperscript{153:1.7} El viernes por la noche, y de nuevo el sábado por la mañana, los dirigentes de Jerusalén le habían insistido a Jairo larga y encarecidamente que impidiera que Jesús hablara en la sinagoga, pero fue en vano. La única respuesta de Jairo a todos sus argumentos fue: <<He concedido esta petición, y no faltaré a mi palabra>>.

\section*{2. El sermón memorable}
\par 
%\textsuperscript{(1709.2)}
\textsuperscript{153:2.1} Jesús dio comienzo a este sermón leyendo en la ley el pasaje que se encuentra en el Deuteronomio: <<Pero sucederá que, si este pueblo no escucha la voz de Dios, las maldiciones de la transgresión le alcanzarán con seguridad. El Señor hará que tus enemigos te golpeen; serás llevado por todos los reinos de la Tierra. El Señor te pondrá, junto con el rey que hayas establecido por encima de ti, en las manos de una nación extranjera. Te convertirás en un motivo de asombro, de proverbio y de burla entre todas las naciones. Tus hijos e hijas irán al cautiverio. Los extranjeros que viven contigo adquirirán una alta autoridad y tú descenderás muy bajo. Estas cosas te sucederán para siempre, a ti y a tu descendencia, porque no has querido escuchar la palabra del Señor. Por eso servirás a tus enemigos que vendrán contra ti. Sufrirás el hambre y la sed y llevarás este yugo extranjero de hierro. El Señor traerá contra ti a una nación venida de lejos, de los confines de la Tierra, una nación cuya lengua no comprenderás, una nación de aspecto feroz, una nación que tendrá pocas consideraciones contigo. Te asediará en todas tus ciudades hasta que los altos muros fortificados en los que has puesto tu confianza se vengan abajo; y todo el país caerá en sus manos. Y sucederá que te verás obligado a comer el fruto de tu propio cuerpo, la carne de tus hijos e hijas, durante ese tiempo de asedio, a causa de la penuria con que te oprimirán tus enemigos>>.

\par 
%\textsuperscript{(1709.3)}
\textsuperscript{153:2.2} Cuando Jesús hubo terminado esta lectura, pasó a los Profetas y leyó en Jeremías: <<`Si no queréis escuchar las palabras de mis servidores, los profetas que os he enviado, entonces pondré a esta casa como Silo, y haré de esta ciudad una maldición para todas las naciones de la Tierra.` Los sacerdotes y los educadores oyeron a Jeremías pronunciar estas palabras en la casa del Señor. Y sucedió que, cuando Jeremías terminó de decir todo lo que el Señor le había ordenado que proclamara a todo el pueblo, los sacerdotes y los educadores lo agarraron, diciendo: `Es seguro que morirás.' Y todo el pueblo se apiñó alrededor de Jeremías en la casa del Señor. Cuando los príncipes de Judá oyeron estas cosas, se sentaron para juzgar a Jeremías. Entonces, los sacerdotes y los educadores hablaron a los príncipes y a todo el pueblo, diciendo: `Este hombre merece la muerte porque ha profetizado en contra de nuestra ciudad, y lo habéis escuchado con vuestros propios oídos.' Entonces Jeremías dijo a todos los príncipes y a todo el pueblo: `El Señor me ha enviado a profetizar contra esta casa y contra esta ciudad todas las palabras que habéis oído. Corregid pues vuestra conducta y reformad vuestras acciones, y obedeced la voz del Señor vuestro Dios, para que podáis escapar de los males que se han pronunciado contra vosotros. En cuanto a mí, heme aquí en vuestras manos. Haced conmigo lo que a vuestro entender os parezca bueno y justo. Pero tened por seguro que, si me quitáis la vida, atraeréis una sangre inocente sobre vosotros y sobre este pueblo, porque en verdad el Señor me ha enviado para decir todas estas palabras en vuestros oídos.'>>

\par 
%\textsuperscript{(1710.1)}
\textsuperscript{153:2.3} <<Los sacerdotes y los educadores de aquella época intentaron matar a Jeremías, pero los jueces no lo consintieron; sin embargo, debido a sus palabras de advertencia, permitieron que lo bajaran con unas cuerdas a una mazmorra inmunda, donde se hundió en el lodo hasta las axilas. Esto es lo que este pueblo le hizo al profeta Jeremías cuando obedeció la orden del Señor de prevenir a sus hermanos sobre su inminente caída política. Hoy deseo preguntaros: ¿Qué harán los principales sacerdotes y los jefes religiosos de este pueblo con el hombre que se atreve a advertirles del día de su condena espiritual? ¿Trataréis también de quitarle la vida al instructor que se atreve a proclamar la palabra del Señor, y que no tiene miedo de indicar cómo os negáis a caminar en la senda de la luz que conduce a la entrada del reino de los cielos?>>

\par 
%\textsuperscript{(1710.2)}
\textsuperscript{153:2.4} <<¿Qué buscáis como prueba de mi misión en la Tierra? Os hemos dejado tranquilos en vuestras posiciones de influencia y de poder, mientras predicábamos la buena nueva a los pobres y a los proscritos. No hemos lanzado ningún ataque hostil contra aquello que veneráis, sino que hemos proclamado una nueva libertad para el alma del hombre dominada por el miedo. He venido al mundo para revelar a mi Padre y para establecer en la Tierra la fraternidad espiritual de los hijos de Dios, el reino de los cielos. Aunque os he recordado muchas veces que mi reino no es de este mundo, sin embargo mi Padre os ha otorgado muchas manifestaciones de prodigios materiales, además de las transformaciones y regeneraciones espirituales más evidentes>>.

\par 
%\textsuperscript{(1710.3)}
\textsuperscript{153:2.5} <<¿Qué nuevo signo esperáis de mí? Os aseguro que ya tenéis pruebas suficientes como para poder tomar vuestras decisiones. En verdad, en verdad les digo a muchos de los que hoy están sentados delante de mí, que os enfrentáis con la necesidad de escoger el camino que vais a seguir. A vosotros os digo, como Josué se lo dijo a vuestros antepasados: `escoged en este día a quién queréis servir.' Muchos de vosotros os encontráis hoy en el cruce de los caminos>>.

\par 
%\textsuperscript{(1710.4)}
\textsuperscript{153:2.6} <<Cuando no pudisteis encontrarme después del banquete de la multitud en la otra orilla, algunos de vosotros alquilasteis la flota pesquera de Tiberiades, que una semana antes se había refugiado en las cercanías durante una tormenta, para salir en mi persecución, y ¿para qué? ¡No para buscar la verdad y la rectitud, ni para aprender a servir y a ayudar mejor a vuestros semejantes! No, sino más bien para conseguir más pan sin haber trabajado para obtenerlo. No era para llenar vuestra alma con la palabra de la vida, sino solamente para llenaros la barriga con el pan de la facilidad. Os han enseñando desde hace mucho tiempo que cuando llegara el Mesías realizaría aquellos prodigios que harían la vida agradable y fácil para todo el pueblo elegido. Así pues, no es de extrañar que vosotros, que habéis recibido esta educación, deseéis con vehemencia los panes y los peces. Pero os afirmo que ésa no es la misión del Hijo del Hombre. He venido para proclamar la libertad espiritual, enseñar la verdad eterna y fomentar la fe viviente>>.

\par 
%\textsuperscript{(1710.5)}
\textsuperscript{153:2.7} <<Hermanos míos, no anheléis la comida perecedera, sino buscad más bien el alimento espiritual que nutre incluso en la vida eterna; éste es el pan de la vida que el Hijo da a todos los que quieran cogerlo y comerlo, porque el Padre ha dado esta vida al Hijo sin restricción. Cuando me habéis preguntado: `¿Qué debemos hacer para realizar las obras de Dios?', os he dicho claramente: `La obra de Dios consiste en creer en aquel que él ha enviado.'>>

\par 
%\textsuperscript{(1710.6)}
\textsuperscript{153:2.8} Luego, Jesús señaló el emblema de una vasija de maná adornada con racimos de uva, que decoraba el dintel de esta nueva sinagoga, y dijo: <<Habéis creído que vuestros antepasados comieron en el desierto el maná ---el pan del cielo--- pero yo os digo que aquello era el pan de la tierra. Aunque Moisés no dio a vuestros padres el pan procedente del cielo, mi Padre está ahora dispuesto a daros el verdadero pan de la vida. El pan del cielo es lo que desciende de Dios y da la vida eterna a los hombres del mundo. Cuando me digáis: Danos de ese pan viviente, yo contestaré: Yo soy ese pan de la vida. El que viene a mí no tendrá hambre, y el que cree en mí nunca tendrá sed. Me habéis visto, habéis vivido conmigo, habéis contemplado mis obras, y sin embargo no creéis que yo haya salido del Padre. Pero a aquellos que sí creen ---no temáis. Todos los que son dirigidos por el Padre vendrán a mí, y el que venga a mí de ninguna manera será rechazado>>.

\par 
%\textsuperscript{(1711.1)}
\textsuperscript{153:2.9} <<Ahora, dejad que os afirme de una vez por todas que he descendido a la Tierra no para hacer mi propia voluntad, sino la voluntad de Aquél que me ha enviado. Y la voluntad final de Aquél que me ha enviado es que yo no pierda ni uno solo de todos los que me ha dado. Y ésta es la voluntad del Padre: Que todo el que contemple al Hijo y crea en él, tenga la vida eterna. Ayer mismo os dí de comer pan para vuestro cuerpo; hoy os ofrezco el pan de la vida para vuestras almas hambrientas. ¿Queréis coger ahora el pan del espíritu, como entonces comisteis de tan buena gana el pan de este mundo?>>

\par 
%\textsuperscript{(1711.2)}
\textsuperscript{153:2.10} Mientras Jesús se detenía un momento para echar una mirada a la asamblea, uno de los educadores de Jerusalén (miembro del sanedrín) se levantó y preguntó: <<¿He comprendido bien cuando has dicho que eres el pan que ha bajado del cielo, y que el maná que Moisés dio a nuestros padres en el desierto no lo era?>> Jesús respondió al fariseo: <<Has comprendido bien>>. Entonces dijo el fariseo: <<Pero, ¿no eres Jesús de Nazaret, el hijo de José el carpintero? ¿Tu padre y tu madre, así como tus hermanos y hermanas, no son bien conocidos para muchos de nosotros? ¿Cómo puede ser entonces que aparezcas aquí en la casa de Dios afirmando que has descendido del cielo?>>

\par 
%\textsuperscript{(1711.3)}
\textsuperscript{153:2.11} En aquellos momentos había mucho murmullo en la sinagoga, y amenazaba con producirse tal alboroto, que Jesús se puso de pie y dijo: <<Seamos pacientes; la verdad nunca teme un examen honesto. Soy todo lo que dices y aun más. El Padre y yo somos uno; el Hijo hace solamente lo que el Padre le enseña, y todos aquellos que son dados al Hijo por el Padre, el Hijo los recibirá en sí mismo. Habéis leído lo que está escrito en los Profetas: `Todos seréis enseñados por Dios', y `Aquellos que son enseñados por el Padre también escucharán a su Hijo.' Cualquiera que se abandona a la enseñanza del espíritu interior del Padre acabará por venir a mí. Ningún hombre ha visto al Padre, pero el espíritu del Padre vive dentro del hombre. El Hijo que ha descendido del cielo ha visto ciertamente al Padre. Y aquellos que creen sinceramente en este Hijo, ya tienen la vida eterna>>.

\par 
%\textsuperscript{(1711.4)}
\textsuperscript{153:2.12} <<Yo soy el pan de la vida. Vuestros padres comieron el maná en el desierto y han muerto. En cuanto a este pan que desciende de Dios, si un hombre lo come, nunca morirá en espíritu. Repito que soy este pan viviente, y toda alma que consigue obtener esta naturaleza unida de Dios y hombre vivirá para siempre. Este pan de vida que doy a todos los que quieren recibirlo es mi propia naturaleza viviente y combinada. El Padre está en el Hijo y el Hijo es uno con el Padre ---ésta es mi revelación donadora de vida al mundo y mi don de salvación para todas las naciones>>.

\par 
%\textsuperscript{(1711.5)}
\textsuperscript{153:2.13} Cuando Jesús terminó de hablar, el jefe de la sinagoga disolvió la asamblea, pero no querían irse. Se agolparon alrededor de Jesús para hacerle más preguntas, mientras que otros murmuraban y discutían entre ellos. Este estado de cosas continuó durante más de tres horas. Finalmente, el auditorio se dispersó mucho después de las siete de la tarde.

\section*{3. Después de la reunión}
\par 
%\textsuperscript{(1712.1)}
\textsuperscript{153:3.1} A Jesús le hicieron muchas preguntas durante esta reunión después del sermón. Algunas fueron formuladas por sus discípulos perplejos, pero la mayoría la realizaron los incrédulos sofistas que sólo intentaban ponerlo en evidencia y hacerlo caer en una trampa.

\par 
%\textsuperscript{(1712.2)}
\textsuperscript{153:3.2} Uno de los fariseos visitantes se subió en un pedestal y gritó esta pregunta: <<Nos dices que eres el pan de la vida. ¿Cómo puedes darnos tu carne para comer o tu sangre para beber? ¿Para qué sirve tu enseñanza si no se puede llevar a cabo?>> Jesús contestó a esta pregunta diciendo: <<Yo no os he enseñado que mi carne sea el pan de la vida ni mi sangre el agua viva. Pero os he dicho que mi vida en la carne es una donación del pan del cielo. El hecho de la Palabra de Dios donada en la carne y el fenómeno del Hijo del Hombre sujeto a la voluntad de Dios, constituyen una realidad de experiencia que equivale al alimento divino. No podéis comer mi carne ni beber mi sangre, pero podéis volveros uno conmigo, en espíritu, como yo soy uno en espíritu con el Padre. Podéis ser alimentados con la palabra eterna de Dios, que es en verdad el pan de la vida, y que ha sido donada en la similitud de la carne mortal; y vuestra alma puede ser regada con el espíritu divino, que es verdaderamente el agua de la vida. El Padre me ha enviado al mundo para mostrar cómo desea habitar y dirigir a todos los hombres; y he vivido esta vida en la carne de tal manera que pueda inspirar también a todos los hombres para que intenten siempre conocer y hacer la voluntad del Padre celestial que reside en ellos>>.

\par 
%\textsuperscript{(1712.3)}
\textsuperscript{153:3.3} Entonces, uno de los espías de Jerusalén que había estado observando a Jesús y a sus apóstoles, dijo: <<Observamos que ni tú ni tus apóstoles os laváis las manos convenientemente antes de comer pan. Debéis saber muy bien que la práctica de comer con las manos sucias y sin lavar es una transgresión de la ley de los ancianos. Tampoco laváis correctamente vuestras copas para beber ni vuestros recipientes para comer. ¿Por qué mostráis tan poco respeto por las tradiciones de los padres y las leyes de nuestros ancianos?>> Después de haberlo escuchado, Jesús respondió: <<¿Por qué transgredís los mandamientos de Dios con las leyes de vuestra tradición? El mandamiento dice: `Honra a tu padre y a tu madre', y ordena que compartáis con ellos vuestros bienes si es necesario; pero promulgáis una ley basada en la tradición, que permite que los hijos desobedientes digan que el dinero que podría haber ayudado a los padres ha sido `entregado a Dios'. La ley de los ancianos libera así de sus responsabilidades a estos hijos astutos, aunque utilicen posteriormente todo ese dinero para su propio bienestar. ¿Cómo puede ser que anuléis el mandamiento de esta manera con vuestra propia tradición? Hipócritas, Isaías profetizó bien acerca de vosotros cuando dijo: `Este pueblo me honra con sus labios, pero su corazón está lejos de mí. Me adoran en vano, pues enseñan como doctrinas los preceptos de los hombres.'>>

\par 
%\textsuperscript{(1712.4)}
\textsuperscript{153:3.4} <<Podéis ver cómo abandonáis el mandamiento para aferraros a las tradiciones de los hombres. Estáis totalmente dispuestos a rechazar la palabra de Dios para mantener vuestras propias tradiciones. Y os atrevéis a ensalzar de otras muchas maneras vuestras propias enseñanzas por encima de la ley y los profetas>>.

\par 
%\textsuperscript{(1712.5)}
\textsuperscript{153:3.5} Jesús dirigió entonces sus comentarios a todos los presentes, diciendo: <<Oídme todos con atención. El hombre no se contamina espiritualmente con lo que entra en su boca, sino más bien con lo que sale de su boca y procede del corazón>>. Pero ni siquiera los apóstoles lograron captar plenamente el significado de sus palabras, porque Simón Pedro también le preguntó: <<Para que algunos de tus oyentes no se sientan ofendidos innecesariamente, ¿podrías explicarnos el significado de estas palabras?>> Entonces Jesús le dijo a Pedro: <<¿También tú eres duro de entendimiento? ¿No sabes que toda planta que mi Padre celestial no haya sembrado será arrancada? Vuelve ahora tu atención hacia aquellos que quisieran conocer la verdad. No puedes obligar a los hombres a amar la verdad. Muchos de estos educadores son guías ciegos. Y ya sabes que si un ciego conduce a otro ciego, los dos se caerán al precipicio. Pero, escucha con atención mientras te digo la verdad acerca de las cosas que manchan moralmente y que contaminan espiritualmente a los hombres. Declaro que lo que entra en el cuerpo por la boca o penetra en la mente a través de los ojos y los oídos, no es lo que mancha al hombre. El hombre sólo se mancha con el mal que se puede originar en su corazón, y que se expresa en las palabras y en los actos de esas personas impías. ¿No sabes que es del corazón de donde provienen los malos pensamientos, los proyectos perversos de asesinato, robo y adulterio, junto con la envidia, el orgullo, la ira, la venganza, las injurias y los falsos testimonios? Éstas son exactamente las cosas que manchan a los hombres, y no el hecho de comer pan con las manos ceremonialmente sucias>>.

\par 
%\textsuperscript{(1713.1)}
\textsuperscript{153:3.6} Los delegados fariseos del sanedrín de Jerusalén estaban ahora casi convencidos de que había que detener a Jesús bajo la acusación de blasfemia o por mofarse de la ley sagrada de los judíos; de ahí sus esfuerzos por implicarlo en una discusión sobre algunas tradiciones de los ancianos, las llamadas leyes orales de la nación, para que tuviera la posibilidad de atacarlas. Por mucha escasez que hubiera de agua, estos judíos esclavizados por la tradición nunca dejaban de ejecutar la ceremonia exigida de lavarse las manos antes de cada comida. Tenían la creencia de que <<es mejor morir que transgredir los mandamientos de los ancianos>>. Los espías habían hecho esta pregunta porque se decía que Jesús había afirmado: <<La salvación es una cuestión de corazón limpio, más bien que de manos limpias>>. Estas creencias son difíciles de eliminar una vez que se han vuelto parte de vuestra religión. Incluso muchos años después de esto, el apóstol Pedro continuaba siendo esclavo del miedo a muchas de estas tradiciones sobre las cosas puras e impuras, y sólo se liberó finalmente después de experimentar un sueño vívido y extraordinario. Todo esto se puede comprender mejor cuando se recuerda que estos judíos consideraban con los mismos ojos el comer sin lavarse las manos que el comerciar con una prostituta; ambas acciones eran igualmente castigables con la excomunión.

\par 
%\textsuperscript{(1713.2)}
\textsuperscript{153:3.7} Por eso el Maestro eligió debatir y exponer la insensatez de todo el sistema rabínico de reglas y reglamentos representado por la ley oral ---las tradiciones de los ancianos, pues todas eran consideradas como más sagradas y más obligatorias para los judíos que las mismas enseñanzas de las Escrituras. Y Jesús se expresó con menos reserva porque sabía que había llegado la hora en que no podía hacer nada más por impedir una ruptura de relaciones clara y abierta con estos dirigentes religiosos.

\section*{4. Las últimas palabras en la sinagoga}
\par 
%\textsuperscript{(1713.3)}
\textsuperscript{153:4.1} En medio de las discusiones de esta reunión después de los oficios, uno de los fariseos de Jerusalén trajo ante Jesús a un joven trastornado que estaba poseído por un espíritu indómito y rebelde. Al conducir a este muchacho demente delante Jesús, dijo: <<¿Qué puedes hacer por una aflicción como ésta? ¿Puedes echar fuera a los demonios?>> Cuando el Maestro contempló al joven, se sintió conmovido por la compasión y, haciéndole una señal al muchacho para que se acercara, lo cogió de la mano y dijo: <<Tú sabes quién soy; sal de él; y encargo a uno de tus compañeros leales que procure que no vuelvas>>. Inmediatamente, el joven se sintió normal y en su pleno juicio. Éste es el primer caso en el que Jesús echó realmente a un <<espíritu maligno>> fuera de un ser humano. Todos los casos anteriores habían sido solamente supuestas posesiones del diablo; pero éste era un auténtico caso de posesión demoníaca, como a veces se producían en aquella época hasta el día de Pentecostés, en que el espíritu del Maestro fue derramado sobre todo el género humano, haciendo imposible para siempre que estos pocos rebeldes celestiales se aprovecharan de ciertos tipos inestables de seres humanos.

\par 
%\textsuperscript{(1714.1)}
\textsuperscript{153:4.2} Como el pueblo se maravillaba, uno de los fariseos se levantó y acusó a Jesús de que podía hacer estas cosas porque estaba aliado con los demonios; que gracias al lenguaje que había empleado para echar fuera a este diablo, Jesús admitía que se conocían mutuamente; y continuó exponiendo que los educadores y los dirigentes religiosos de Jerusalén habían concluido que Jesús realizaba todos sus supuestos milagros por el poder de Belcebú, el príncipe de los demonios. El fariseo dijo: <<No tengáis nada en común con este hombre; está asociado con Satanás>>.

\par 
%\textsuperscript{(1714.2)}
\textsuperscript{153:4.3} Entonces Jesús dijo: <<¿Cómo puede Satanás echar fuera a Satanás? Un reino dividido contra sí mismo no puede subsistir; si una casa está dividida contra sí misma, pronto cae en la desolación. ¿Puede una ciudad resistir el asedio si está desunida? Si Satanás echa a Satanás, está dividido contra sí mismo; ¿cómo podrá entonces subsistir su reino? Pero deberíais saber que nadie puede entrar en la casa de un hombre fuerte y despojarlo de sus bienes, a menos que primero lo haya vencido y atado. Así pues, si echo fuera a los demonios por el poder de Belcebú, ¿por quién los echan vuestros hijos? Por eso ellos serán vuestros jueces. Pero si echo fuera a los demonios por el espíritu de Dios, entonces el reino de Dios ha venido realmente hasta vosotros. Si no estuvierais cegados por los prejuicios y descarriados por el miedo y el orgullo, percibiríais fácilmente que alguien más grande que los demonios está en medio de vosotros. Me obligáis a proclamar que el que no está conmigo está contra mí, y que el que no recoge conmigo desparrama en todas direcciones. ¡Dejad que os haga una advertencia solemne, a vosotros que, con los ojos abiertos y una malicia premeditada, os atrevéis a atribuir a sabiendas las obras de Dios a las acciones de los demonios! En verdad, en verdad os digo que todos vuestros pecados serán perdonados, e incluso todas vuestras blasfemias, pero cualquiera que blasfeme contra Dios de manera deliberada y con una intención perversa, nunca será perdonado. Puesto que esos autores permanentes de la iniquidad nunca buscarán ni recibirán el perdón, son culpables del pecado de rechazar eternamente el perdón divino>>.

\par 
%\textsuperscript{(1714.3)}
\textsuperscript{153:4.4} <<Muchos de vosotros habéis llegado hoy al cruce de los caminos; habéis llegado al punto en que tenéis que efectuar la elección inevitable entre la voluntad del Padre y los caminos de las tinieblas escogidos por vosotros mismos. Según lo que escojáis ahora, eso mismo llegaréis a ser con el tiempo. O bien tenéis que mejorar el árbol y su fruto, o de otro modo el árbol y su fruto se corromperán. Declaro que en el reino eterno de mi Padre, el árbol se conoce por sus frutos. Pero algunos de vosotros, que sois como víboras, ¿cómo podéis producir buenos frutos si ya habéis escogido el mal? Después de todo, vuestra boca expresa claramente la abundancia de mal que hay en vuestro corazón>>.

\par 
%\textsuperscript{(1714.4)}
\textsuperscript{153:4.5} Entonces se levantó otro fariseo, diciendo: <<Maestro, quisiéramos que nos dieras un signo predeterminado que nosotros aceptaríamos como demostración de tu autoridad y de tu derecho a enseñar. ¿Estás de acuerdo con este arreglo?>> Cuando Jesús escuchó esto, dijo: <<Esta generación sin fe y en busca de signos desea una señal, pero no se os dará más signo que el que ya tenéis, y aquel que veréis cuando el Hijo del Hombre se separe de vosotros>>.

\par 
%\textsuperscript{(1714.5)}
\textsuperscript{153:4.6} Cuando hubo terminado de hablar, sus apóstoles lo rodearon y lo condujeron fuera de la sinagoga. Recorrieron el trayecto con él, en silencio, hasta la casa de Betsaida. Todos estaban asombrados y un poco aterrorizados por el cambio repentino en la táctica de enseñanza del Maestro. No estaban acostumbrados en absoluto a verlo actuar de una manera tan militante.

\section*{5. El sábado por la tarde}
\par 
%\textsuperscript{(1715.1)}
\textsuperscript{153:5.1} Una y otra vez, Jesús había hecho añicos las esperanzas de sus apóstoles, y había destruido repetidas veces sus expectativas más acariciadas, pero nunca habían pasado por unos momentos de decepción ni por unos períodos de tristeza equivalentes a los que ahora estaban sufriendo. Además, un miedo real por su seguridad se mezclaba ahora con su depresión. Todos estaban sorprendidos y alarmados por la deserción tan repentina y completa del pueblo. También estaban un poco asustados y desconcertados por la audacia inesperada y la resolución afirmativa que mostraban los fariseos que habían venido de Jerusalén. Pero por encima de todo, estaban aturdidos a causa del repentino cambio de táctica de Jesús. En circunstancias normales, habrían acogido bien la aparición de esta actitud más militante, pero al producirse como se había producido, unida a tantas cosas inesperadas, esto les asustó.

\par 
%\textsuperscript{(1715.2)}
\textsuperscript{153:5.2} Y ahora, para colmo de todas estas inquietudes, cuando llegaron a casa, Jesús se negó a comer. Se aisló durante horas en una de las habitaciones de arriba. Era cerca de la medianoche cuando Joab, el jefe de los evangelistas, regresó con la noticia de que aproximadamente un tercio de sus asociados habían abandonado la causa. Durante toda la noche, los discípulos leales estuvieron yendo y viniendo para informar de que el cambio súbito de sentimientos hacia el Maestro era general en Cafarnaúm. Los dirigentes de Jerusalén se apresuraron a alimentar este sentimiento de desafecto y, de todas las maneras posibles, procuraron fomentar un movimiento para que la gente se alejara de Jesús y sus enseñanzas. Durante estas horas difíciles, las doce mujeres mantenían una reunión en la casa de Pedro. Estaban enormemente trastornadas, pero ninguna de ellas desertó.

\par 
%\textsuperscript{(1715.3)}
\textsuperscript{153:5.3} Poco después de la medianoche, Jesús bajó de la habitación de arriba y se mezcló con los doce y sus compañeros, unos treinta en total. Dijo: <<Reconozco que esta criba del reino os aflige, pero es inevitable. Sin embargo, después de toda la preparación que habéis recibido, ¿había alguna buena razón para que tropezarais con mis palabras? ¿Cómo puede ser que estéis llenos de miedo y de consternación cuando veis que el reino se está despojando de esas multitudes tibias y de esos discípulos indiferentes? ¿Por qué os afligís cuando está despuntando un nuevo día en el que las enseñanzas espirituales del reino de los cielos van a brillar con una nueva gloria? Si encontráis difícil soportar esta prueba, ¿qué haréis entonces cuando el Hijo del Hombre deba regresar al Padre? ¿Cuándo y cómo os prepararéis para el momento en que ascenderé al lugar de donde vine a este mundo?>>

\par 
%\textsuperscript{(1715.4)}
\textsuperscript{153:5.4} <<Amados míos, debéis recordar que es el espíritu el que vivifica; la carne y todo lo relacionado con ella es de poco provecho. Las palabras que os he dicho son espíritu y vida. ¡Tened buen ánimo! No os he abandonado. Mucha gente se ofenderá por la claridad de mis palabras de estos días. Ya habéis oído que muchos de mis discípulos se han vuelto atrás; ya no caminan conmigo. Sabía desde el principio que estos creyentes sin entusiasmo se quedarían por el camino. ¿No os escogí a vosotros doce y os aparté como embajadores del reino? Y ahora, en un momento como éste, ¿desertaréis vosotros también? Que cada uno de vosotros vele por su propia fe, porque uno de vosotros corre un grave peligro>>. Cuando Jesús hubo terminado de hablar, Simón Pedro dijo: <<Sí, Señor, estamos tristes y perplejos, pero nunca te abandonaremos. Tú nos has enseñado las palabras de la vida eterna. Hemos creído en ti y te hemos seguido todo este tiempo. No nos volveremos atrás, porque sabemos que has sido enviado por Dios>>. Cuando Pedro terminó de hablar, todos asintieron unánimemente con la cabeza, aprobando su promesa de lealtad.

\par 
%\textsuperscript{(1716.1)}
\textsuperscript{153:5.5} Entonces Jesús dijo: <<Id a descansar, porque se acercan momentos de mucho trabajo; los próximos días van a ser muy activos>>.


\chapter{Documento 154. Los últimos días en Cafarnaúm}
\par 
%\textsuperscript{(1717.1)}
\textsuperscript{154:0.1} DURANTE la noche memorable del sábado 30 de abril, mientras Jesús dirigía unas palabras de consuelo y de ánimo a sus discípulos abatidos y desconcertados, Herodes Antipas celebraba un consejo en Tiberiades con un grupo de delegados especiales que representaban al sanedrín de Jerusalén. Estos escribas y fariseos instaron a Herodes para que arrestara a Jesús; hicieron todo lo posible para convencerlo de que Jesús incitaba al pueblo a la disensión e incluso a la rebelión. Pero Herodes se negó a emprender una acción contra él como delincuente político. Los consejeros de Herodes le habían informado correctamente sobre el episodio sucedido al otro lado del lago, cuando la gente intentó proclamar rey a Jesús y cómo éste había rechazado la proposición.

\par 
%\textsuperscript{(1717.2)}
\textsuperscript{154:0.2} Un miembro de la familia oficial de Herodes, Chuza, cuya esposa pertenecía al cuerpo asistente de mujeres, le había informado que Jesús no se proponía entrometerse en los asuntos de la soberanía terrestre; que sólo estaba interesado en establecer la fraternidad espiritual de sus creyentes, una fraternidad que él llamaba el reino de los cielos. Herodes tenía confianza en los informes de Chuza, de tal manera que se negó a interferir en las actividades de Jesús. En esta época, la actitud de Herodes hacia Jesús también estaba influida por su miedo supersticioso a Juan el Bautista. Herodes era uno de esos judíos apóstatas que, aunque no creía en nada, tenía miedo de todo. Tenía cargo de conciencia por haber hecho morir a Juan, y no quería verse enredado en estas intrigas contra Jesús. Conocía muchos casos de enfermedades que habían sido curadas aparentemente por Jesús, y lo consideraba como un profeta o como un fanático religioso relativamente inofensivo.

\par 
%\textsuperscript{(1717.3)}
\textsuperscript{154:0.3} Cuando los judíos lo amenazaron con informar al César de que estaba amparando a un súbdito traidor, Herodes los expulsó de su cámara de consejos. Las cosas permanecieron así durante una semana, a lo largo de la cual Jesús preparó a sus seguidores para la dispersión inminente.

\section*{1. Una semana de deliberaciones}
\par 
%\textsuperscript{(1717.4)}
\textsuperscript{154:1.1} Del 1 al 7 de mayo, Jesús mantuvo deliberaciones íntimas con sus seguidores en la casa de Zebedeo. Sólo los discípulos probados y de confianza fueron admitidos a estas conferencias. En esta época sólo había unos cien discípulos que tenían la valentía moral de desafiar la oposición de los fariseos y de declarar abiertamente su adhesión a Jesús. Con este grupo mantuvo sesiones por la mañana, por la tarde y por la noche. Todas las tardes se congregaban pequeños conjuntos de investigadores al borde del mar, donde algunos evangelistas o apóstoles conversaban con ellos. Estos grupos raras veces contenían más de cincuenta personas.

\par 
%\textsuperscript{(1717.5)}
\textsuperscript{154:1.2} El viernes de esta semana, los dirigentes de la sinagoga de Cafarnaúm tomaron medidas oficiales para cerrar la casa de Dios a Jesús y a todos sus seguidores. Esta acción se emprendió a instigación de los fariseos de Jerusalén. Jairo dimitió como dirigente principal y se alineó abiertamente con Jesús.

\par 
%\textsuperscript{(1718.1)}
\textsuperscript{154:1.3} La última reunión al lado del mar tuvo lugar el sábado 7 de mayo por la tarde. Jesús se dirigió a menos de ciento cincuenta personas que se habían congregado en esta ocasión. Este sábado por la noche, la corriente de la estima popular por Jesús y sus enseñanzas se encontró en su punto más bajo. Desde entonces en adelante, los sentimientos favorables aumentaron lenta y constantemente, pero de una manera más sana y digna de confianza; se empezó a formar un nuevo grupo de partidarios que estaba mejor cimentado en la fe espiritual y en la verdadera experiencia religiosa. Ahora había terminado definitivamente la etapa de transición, más o menos mixta y de compromisos, entre los conceptos materialistas del reino que tenían los seguidores del Maestro, y los conceptos más idealistas y espirituales que Jesús enseñaba. De ahora en adelante, el evangelio del reino se proclamó más abiertamente en su más amplia extensión y con sus vastas implicaciones espirituales.

\section*{2. Una semana de descanso}
\par 
%\textsuperscript{(1718.2)}
\textsuperscript{154:2.1} El domingo 8 de mayo del año 29, el sanedrín aprobó un decreto en Jerusalén que cerraba todas las sinagogas de Palestina a Jesús y a sus seguidores. Fue una usurpación de autoridad, nueva y sin precedentes, por parte del sanedrín de Jerusalén. Hasta ese momento, cada sinagoga había existido y funcionado como una congregación independiente de fieles, bajo el mando y la dirección de su propio consejo rector. Sólo las sinagogas de Jerusalén se habían sometido a la autoridad del sanedrín. Cinco miembros del sanedrín dimitieron a consecuencia de esta acción sumaria. Se despacharon inmediatamente cien mensajeros para transmitir e imponer este decreto. En el corto espacio de dos semanas, todas las sinagogas de Palestina se habían inclinado ante esta proclamación del sanedrín, excepto la de Hebrón. Los dirigentes de la sinagoga de Hebrón se negaron a reconocer el derecho del sanedrín a ejercer esta jurisdicción sobre su asamblea. Esta negativa a aceptar el decreto de Jerusalén se basaba más en la discordia sobre la autonomía de su congregación, que en su simpatía por la causa de Jesús. Poco tiempo después, la sinagoga de Hebrón fue destruida por un incendio.

\par 
%\textsuperscript{(1718.3)}
\textsuperscript{154:2.2} Jesús decretó una semana de vacaciones este mismo domingo por la mañana, y estimuló a todos sus discípulos a que volvieran a sus hogares o con sus amigos para dar descanso a sus almas perturbadas y expresar palabras de aliento a sus seres queridos. Dijo: <<Id a vuestros lugares de residencia para distraeros o pescar, mientras rezáis por la expansión del reino>>.

\par 
%\textsuperscript{(1718.4)}
\textsuperscript{154:2.3} Esta semana de descanso permitió a Jesús visitar a muchas familias y grupos cerca de la costa. También fue a pescar en varias ocasiones con David Zebedeo; aunque circulaba solo la mayor parte del tiempo, siempre estaba vigilado de cerca por dos o tres de los mensajeros más fieles de David, que tenían órdenes precisas de su jefe con respecto a la seguridad de Jesús. No hubo ningún tipo de enseñanza pública durante esta semana de descanso.

\par 
%\textsuperscript{(1718.5)}
\textsuperscript{154:2.4} Ésta fue la semana en que Natanael y Santiago Zebedeo sufrieron una grave enfermedad. Durante tres días y tres noches padecieron la fase aguda de un doloroso desorden digestivo. A la tercera noche, Jesús envió a descansar a Salomé, la madre de Santiago, mientras él cuidaba de sus apóstoles que sufrían. Por supuesto, Jesús podía haber curado instantáneamente a estos dos hombres, pero éste no es el método que emplean el Hijo o el Padre para tratar estas dificultades y aflicciones corrientes de los hijos de los hombres en los mundos evolutivos del tiempo y del espacio. A lo largo de toda su vida extraordinaria en la carne, Jesús no utilizó ni una sola vez ningún tipo de ayuda sobrenatural para ningún miembro de su familia terrestre ni en beneficio de ninguno de sus seguidores inmediatos.

\par 
%\textsuperscript{(1719.1)}
\textsuperscript{154:2.5} Es necesario enfrentarse con las dificultades del universo y tropezar con los obstáculos planetarios, como parte de la educación experiencial proporcionada para el crecimiento y el desarrollo, para la perfección progresiva, del alma evolutiva de las criaturas mortales. La espiritualización del alma humana requiere una experiencia íntima con el proceso educativo de resolver una amplia gama de problemas universales reales. La naturaleza animal y las formas inferiores de criaturas volitivas no progresan favorablemente en un ambiente fácil. Las situaciones problemáticas, asociadas con los estímulos para ponerse en acción, se confabulan para producir esas actividades de la mente, del alma y del espíritu que contribuyen poderosamente a la obtención de los objetivos meritorios de la progresión mortal, y a la consecución de los niveles superiores de destino espiritual.

\section*{3. La segunda conferencia en Tiberiades}
\par 
%\textsuperscript{(1719.2)}
\textsuperscript{154:3.1} El 16 de mayo se convocó en Tiberiades la segunda conferencia entre las autoridades de Jerusalén y Herodes Antipas. Tanto los jefes religiosos como los dirigentes políticos de Jerusalén estaban presentes. Los líderes judíos pudieron informar a Herodes de que prácticamente todas las sinagogas de Galilea y de Judea habían cerrado sus puertas a las enseñanzas de Jesús. Hicieron nuevos esfuerzos por conseguir que Herodes arrestara a Jesús, pero él se negó a ceder a sus peticiones. Sin embargo, el 18 de mayo, Herodes aceptó el plan que permitía a las autoridades del sanedrín apresar a Jesús y llevarlo a Jerusalén para ser juzgado por infracciones religiosas, a condición de que el gobernador romano de Judea estuviera de acuerdo. Mientras tanto, los enemigos de Jesús difundieron activamente el rumor, por toda Galilea, de que Herodes se había vuelto hostil a Jesús, y que tenía la intención de exterminar a todos los que creían en sus enseñanzas.

\par 
%\textsuperscript{(1719.3)}
\textsuperscript{154:3.2} El sábado 21 de mayo por la noche llegó a Tiberiades la noticia de que las autoridades civiles de Jerusalén no ponían objeciones al acuerdo establecido entre Herodes y los fariseos de que Jesús fuera arrestado y llevado a Jerusalén para ser juzgado delante del sanedrín, acusado de burlarse de las leyes sagradas de la nación judía. En consecuencia, poco antes de la medianoche de este día, Herodes firmó el decreto que autorizaba a los oficiales del sanedrín a prender a Jesús dentro de los dominios de Herodes, y a llevarlo a la fuerza a Jerusalén para ser juzgado. Herodes sufrió fuertes presiones de muchos lados antes de que se decidiera a conceder este permiso, y sabía muy bien que Jesús no podía esperar un juicio justo de sus enemigos encarnizados de Jerusalén.

\section*{4. El sábado por la noche en Cafarnaúm}
\par 
%\textsuperscript{(1719.4)}
\textsuperscript{154:4.1} Este mismo sábado por la noche, un grupo de cincuenta ciudadanos importantes de Cafarnaúm se reunió en la sinagoga para debatir la importante cuestión: <<¿Qué vamos a hacer con Jesús?>> Hablaron y discutieron hasta después de la medianoche, pero no pudieron encontrar ningún terreno común para ponerse de acuerdo. Aparte de algunas personas que tendían a creer que Jesús podría ser el Mesías, o al menos un hombre santo, o quizás un profeta, la asamblea estaba dividida en cuatro grupos casi iguales, que sostenían respectivamente los puntos de vista siguientes sobre Jesús:

\par 
%\textsuperscript{(1719.5)}
\textsuperscript{154:4.2} 1. Que era un fanático religioso iluso e inofensivo.

\par 
%\textsuperscript{(1719.6)}
\textsuperscript{154:4.3} 2. Que era un agitador peligroso y astuto, capaz de incitar a la rebelión.

\par 
%\textsuperscript{(1720.1)}
\textsuperscript{154:4.4} 3. Que estaba aliado con los demonios, y que podía incluso ser un príncipe de los demonios.

\par 
%\textsuperscript{(1720.2)}
\textsuperscript{154:4.5} 4. Que estaba fuera de sí, que estaba loco, desequilibrado mentalmente.

\par 
%\textsuperscript{(1720.3)}
\textsuperscript{154:4.6} Se habló mucho sobre las doctrinas que Jesús predicaba y que trastornaban a la gente corriente; sus enemigos sostenían que sus enseñanzas eran impracticables, que todo saltaría en pedazos si todo el mundo hiciera un esfuerzo honrado por vivir de acuerdo con sus ideas. Los hombres de muchas generaciones posteriores han dicho las mismas cosas. Incluso en la época más iluminada de las presentes revelaciones, muchos hombres inteligentes y con buenas intenciones sostienen que la civilización moderna no podría haberse construido sobre las enseñanzas de Jesús ---y en parte tienen razón. Pero todos esos escépticos olvidan que se podría haber construido una civilización mucho mejor sobre sus enseñanzas, y que alguna vez se construirá. Este mundo nunca ha intentado seriamente poner en práctica, a gran escala, las enseñanzas de Jesús, aunque a menudo se han hecho intentos poco entusiastas por seguir las doctrinas del llamado cristianismo.

\section*{5. El memorable domingo por la mañana}
\par 
%\textsuperscript{(1720.4)}
\textsuperscript{154:5.1} El 22 de mayo fue un día memorable en la vida de Jesús. Este domingo por la mañana, antes del amanecer, uno de los mensajeros de David llegó apresuradamente de Tiberiades, trayendo la noticia de que Herodes había autorizado, o estaba a punto de autorizar, el arresto de Jesús por parte de los oficiales del sanedrín. Al recibir la noticia de este peligro inminente, David Zebedeo despertó a sus mensajeros y los envió a todos los grupos locales de discípulos para convocarlos a una reunión de emergencia a las siete de aquella misma mañana. Cuando la cuñada de Judá (hermano de Jesús) escuchó este informe alarmante, avisó rápidamente a todos los miembros de la familia de Jesús que vivían cerca, convocándolos a que se congregaran inmediatamente en la casa de Zebedeo. En respuesta a este llamamiento apresurado, María, Santiago, José, Judá y Rut se reunieron enseguida.

\par 
%\textsuperscript{(1720.5)}
\textsuperscript{154:5.2} En esta reunión por la mañana temprano, Jesús impartió sus instrucciones de despedida a los discípulos reunidos; es decir, se despidió de ellos por ahora, sabiendo muy bien que pronto serían expulsados de Cafarnaúm. Aconsejó a todos que buscaran la guía de Dios y que continuaran la obra del reino sin preocuparse por las consecuencias. Los evangelistas debían trabajar como estimaran conveniente hasta el momento en que se les pudiera llamar. Escogió a doce evangelistas para que lo acompañaran; ordenó a los doce apóstoles que permanecieran con él, pasara lo que pasara. Indicó a las doce mujeres que permanecieran en la casa de Zebedeo y en la de Pedro hasta que enviara a buscarlas.

\par 
%\textsuperscript{(1720.6)}
\textsuperscript{154:5.3} Jesús permitió que David Zebedeo continuara con su servicio de mensajeros por todo el país, y al despedirse luego del Maestro, David dijo: <<Ve a efectuar tu labor, Maestro. No te dejes atrapar por los fanáticos, y no dudes nunca de que los mensajeros te seguirán. Mis hombres nunca perderán el contacto contigo; gracias a ellos, sabrás cómo progresa el reino en otras regiones, y por medio de ellos todos tendremos noticias tuyas. Nada que pueda ocurrirme interrumpirá este servicio, porque he nombrado un primero y un segundo sustitutos, e incluso un tercero. No soy ni un instructor ni un predicador, pero mi corazón me exige que haga esto, y no hay nada que pueda detenerme>>.

\par 
%\textsuperscript{(1720.7)}
\textsuperscript{154:5.4} Aproximadamente a las siete y media de esta mañana, Jesús empezó su discurso de despedida a casi cien creyentes que se habían congregado en el interior de la casa para escucharlo. Fue un acontecimiento solemne para todos los presentes, pero Jesús parecía excepcionalmente alegre; una vez más volvía a ser el mismo de siempre. La seriedad de las últimas semanas había desaparecido, y los inspiró a todos con sus palabras de fe, de esperanza y de valentía.

\section*{6. Llega la familia de Jesús}
\par 
%\textsuperscript{(1721.1)}
\textsuperscript{154:6.1} Eran aproximadamente las ocho de la mañana de este domingo cuando cinco miembros de la familia terrestre de Jesús llegaron al lugar, en respuesta al llamamiento urgente de la cuñada de Judá. De toda su familia carnal, solamente Rut había creído constantemente y de todo corazón en la divinidad de su misión en la Tierra. Judá y Santiago, e incluso José, aún conservaban una gran parte de su fe en Jesús, pero habían permitido que el orgullo dificultara su mejor juicio y sus verdaderas inclinaciones espirituales. María estaba desgarrada por igual entre el amor y el temor, entre el amor maternal y el orgullo familiar. Aunque estaba abrumada por las dudas, nunca había podido olvidar por completo la visita de Gabriel antes del nacimiento de Jesús. Los fariseos se habían esforzado por persuadir a María de que Jesús estaba fuera de sí, de que estaba loco. Le insistieron para que fuera con sus hijos y tratara de disuadirlo de continuar con sus esfuerzos de enseñanza pública. Aseguraron a María que la salud de Jesús estaba a punto de quebrantarse, y que si se le permitía continuar, el único resultado sería que el deshonor y la ignominia caerían sobre toda la familia. Así pues, cuando recibieron la noticia de la cuñada de Judá, los cinco partieron inmediatamente hacia la casa de Zebedeo, pues se hallaban todos juntos en el hogar de María, donde se habían reunido con los fariseos la noche anterior. Habían conversado con los dirigentes de Jerusalén hasta muy entrada la noche, y todos estaban más o menos convencidos de que Jesús actuaba de una manera extraña, de que se había comportado de forma extravagante desde hacía algún tiempo. Aunque Rut no podía explicar todos los motivos de su conducta, insistió en que Jesús siempre había tratado equitativamente a su familia, y se negó a participar en el programa consistente en intentar disuadirlo de que continuara su obra.

\par 
%\textsuperscript{(1721.2)}
\textsuperscript{154:6.2} Por el camino hacia la casa de Zebedeo, discutieron sobre todas estas cosas y acordaron entre ellos tratar de persuadir a Jesús para que volviera a casa con ellos, porque, según decía María: <<Sé que podría influir en mi hijo si tan sólo quisiera venir a casa y escucharme>>. Santiago y Judá habían oído rumores sobre los planes para arrestar a Jesús y llevarlo a Jerusalén para ser juzgado. También tenían miedo por su propia seguridad. Mientras Jesús había sido una figura popular a los ojos de la gente, su familia había dejado que las cosas siguieran su curso, pero ahora que la población de Cafarnaúm y los dirigentes de Jerusalén se habían vuelto repentinamente contra él, empezaron a sentir en lo más vivo la presión de la supuesta desgracia de su embarazosa situación.

\par 
%\textsuperscript{(1721.3)}
\textsuperscript{154:6.3} Habían esperado encontrar a Jesús, cogerlo aparte, e instarlo a que volviera a casa con ellos. Habían pensado en asegurarle que se olvidarían de que los había descuidado ---que perdonarían y olvidarían--- con que sólo renunciara a la insensatez de intentar predicar una nueva religión que sólo le acarrearía problemas y traería el deshonor a su familia. Ante todos estos razonamientos, Rut se limitaba a decir: <<Le diré a mi hermano que pienso que es un hombre de Dios, y que espero que esté dispuesto a morir antes que permitir que esos malvados fariseos pongan fin a su predicación>>. José prometió mantener callada a Rut mientras los demás trataban de convencer a Jesús.

\par 
%\textsuperscript{(1721.4)}
\textsuperscript{154:6.4} Cuando llegaron a la casa de Zebedeo, Jesús estaba en plena exposición de su discurso de despedida a los discípulos. Trataron de entrar en la casa, pero estaba atestada a rebosar. Terminaron por instalarse en el pórtico de atrás e hicieron saber a Jesús, de persona en persona, la noticia de su llegada; finalmente, Simón Pedro se lo anunció en voz baja, interrumpiendo su discurso para decirle: <<Mira, tu madre y tus hermanos están fuera, y están muy impacientes por hablar contigo>>. Ahora bien, a su madre no se le había ocurrido pensar en la importancia que tenía dar este mensaje de despedida a sus seguidores, ni tampoco sabía que su discurso podía terminar probablemente en cualquier momento por la llegada de sus captores. Después de una separación aparente tan prolongada, y en vista del favor que ella y sus hermanos le hacían viniendo de hecho hasta él, María creía realmente que Jesús dejaría de hablar e iría a reunirse con ellos en cuanto se enterara de que lo estaban esperando.

\par 
%\textsuperscript{(1722.1)}
\textsuperscript{154:6.5} Éste fue otro de esos casos en los que su familia terrestre no podía comprender que Jesús tenía que ocuparse de los asuntos de su Padre. Así pues, María y sus hermanos se sintieron profundamente ofendidos cuando, a pesar de que interrumpió su discurso para recibir el mensaje, en lugar de salir precipitadamente para saludarlos, escucharon su voz melodiosa aumentar de tono para decir: <<Decid a mi madre y a mis hermanos que no teman nada por mí. El Padre que me ha enviado al mundo no me abandonará, y mi familia tampoco sufrirá ningún daño. Rogadles que tengan buen ánimo y que pongan su confianza en el Padre del reino. Pero, después de todo, ¿quién es mi madre y quiénes son mis hermanos?>> Y extendiendo las manos hacia todos sus discípulos congregados en la sala, dijo: <<No tengo madre; no tengo hermanos. ¡He aquí a mi madre y he aquí a mis hermanos! Porque cualquiera que hace la voluntad de mi Padre que está en los cielos, ése es mi madre, mi hermano y mi hermana>>.

\par 
%\textsuperscript{(1722.2)}
\textsuperscript{154:6.6} Cuando María escuchó estas palabras, se desmayó en los brazos de Judá. La llevaron al jardín para reanimarla, mientras Jesús pronunciaba las últimas palabras de su mensaje de despedida. Entonces hubiera salido para conversar con su madre y sus hermanos, pero un mensajero llegó apresuradamente de Tiberiades, trayendo la noticia de que los oficiales del sanedrín estaban de camino con autoridad para detener a Jesús y llevarlo a Jerusalén. Andrés recibió este mensaje, e interrumpió a Jesús para comunicarselo.

\par 
%\textsuperscript{(1722.3)}
\textsuperscript{154:6.7} Andrés no se acordaba de que David había apostado unos veinticinco centinelas alrededor de la casa de Zebedeo, de manera que nadie podía cogerlos por sorpresa; por eso preguntó a Jesús qué debían hacer. El Maestro permaneció allí de pie en silencio, mientras su madre se recuperaba en el jardín de la conmoción de haberle oído decir las palabras: <<Yo no tengo madre>>. En ese preciso momento, una mujer se levantó en la sala y exclamó: <<Benditas sean las entrañas que te engendraron y benditos sean los senos que te amamantaron>>. Jesús se desvió un momento de su conversación con Andrés para responder a esta mujer, diciendo: <<No, bendito es más bien aquel que escucha la palabra de Dios y se atreve a obedecerla>>.

\par 
%\textsuperscript{(1722.4)}
\textsuperscript{154:6.8} María y los hermanos de Jesús pensaban que Jesús no los comprendía, que había perdido su interés por ellos, sin darse cuenta de que eran ellos los que no lograban comprenderlo. Jesús comprendía plenamente lo difícil que es para los hombres romper con su pasado. Sabía hasta qué punto los seres humanos se dejan influir por la elocuencia de un predicador, y de qué modo la conciencia responde al llamamiento emocional, como la mente responde a la lógica y a la razón, pero también sabía que es muchísimo más difícil persuadir a los hombres para que \textit{renuncien al pasado}.

\par 
%\textsuperscript{(1722.5)}
\textsuperscript{154:6.9} Es eternamente cierto que todos los que puedan pensar que son incomprendidos o mal apreciados, tienen en Jesús a un amigo compasivo y a un consejero comprensivo. Había advertido a sus apóstoles que los enemigos de un hombre pueden ser los de su propia casa, pero difícilmente había imaginado que esta predicción se aplicaría tan de cerca a su propia experiencia. Jesús no abandonó a su familia terrestre para hacer la obra de su Padre ---fueron ellos los que lo abandonaron. Más tarde, después de la muerte y resurrección del Maestro, cuando su hermano Santiago se unió al movimiento cristiano primitivo, sufrió enormemente por no haber sabido disfrutar de esta asociación inicial con Jesús y sus discípulos.

\par 
%\textsuperscript{(1723.1)}
\textsuperscript{154:6.10} Para pasar por estos acontecimientos, Jesús escogió dejarse guiar por el conocimiento limitado de su mente humana. Deseaba sufrir la experiencia con sus compañeros como un simple hombre. En la mente humana de Jesús estaba la idea de ver a su familia antes de irse. No quería detenerse en medio de su discurso y transformar así en un espectáculo público este primer encuentro después de una separación tan larga. Había tenido la intención de terminar su alocución y luego charlar con ellos antes de partir, pero este plan se frustró debido a la confabulación de acontecimientos que se produjeron inmediatamente después.

\par 
%\textsuperscript{(1723.2)}
\textsuperscript{154:6.11} La llegada de un grupo de mensajeros de David a la puerta trasera de la casa de Zebedeo hizo que aumentara la precipitación por huir. La agitación que produjeron estos hombres asustó a los apóstoles, pues les hizo pensar que estos recién llegados podían ser sus captores; temiendo ser arrestados inmediatamente, se precipitaron por la puerta delantera hacia la barca que les estaba esperando. Todo esto explica por qué Jesús no vio a su familia que lo estaba esperando en el porche de atrás.

\par 
%\textsuperscript{(1723.3)}
\textsuperscript{154:6.12} Sin embargo, al subir a la barca en esta huida precipitada, le dijo a David Zebedeo: <<Di a mi madre y a mis hermanos que aprecio su venida, y que tenía la intención de verlos. Recomiéndales que no se ofendan por mi conducta, sino que traten más bien de conocer la voluntad de Dios y de tener la gracia y el coraje de hacer esa voluntad>>.

\section*{7. La huida precipitada}
\par 
%\textsuperscript{(1723.4)}
\textsuperscript{154:7.1} Y así, este domingo por la mañana 22 de mayo del año 29, Jesús, con sus doce apóstoles y los doce evangelistas, emprendió esta huida precipitada de los oficiales del sanedrín, que se dirigían a Betsaida con la autorización de Herodes Antipas para arrestarlo y llevarlo a Jerusalén, donde sería juzgado bajo la inculpación de blasfemia y de otras violaciones de las leyes sagradas de los judíos. Eran casi las ocho y media de esta hermosa mañana cuando este grupo de veinticinco personas se sentó a los remos para bogar hacia la costa oriental del Mar de Galilea.

\par 
%\textsuperscript{(1723.5)}
\textsuperscript{154:7.2} Una embarcación más pequeña iba detrás de la barca del Maestro, conteniendo a los seis mensajeros de David que tenían la orden de mantenerse en contacto con Jesús y sus compañeros, y procurar que la información sobre su paradero y su seguridad se transmitiera regularmente a la casa de Zebedeo en Betsaida, la cual había servido de cuartel general para la obra del reino durante algún tiempo. Pero la casa de Zebedeo no sería nunca más el hogar de Jesús. De ahora en adelante, y durante el resto de su vida en la Tierra, el Maestro verdaderamente <<no tuvo dónde reposar su cabeza>>. Nunca más llegó a tener algo que se pareciera a un domicilio fijo.

\par 
%\textsuperscript{(1723.6)}
\textsuperscript{154:7.3} Remaron hasta cerca del pueblo de Jeresa, encargaron la custodia de su barca a unos amigos, y empezaron las peregrinaciones de este último año memorable de la vida del Maestro en la Tierra. Permanecieron algún tiempo en los dominios de Felipe, yendo de Jeresa a Cesarea de Filipo, y desde allí se dirigieron hacia la costa de Fenicia.

\par 
%\textsuperscript{(1723.7)}
\textsuperscript{154:7.4} La multitud permaneció alrededor de la casa de Zebedeo, observando las dos embarcaciones que navegaban por el lago hacia la orilla oriental; ya estaban lejos cuando los oficiales de Jerusalén llegaron precipitadamente y empezaron a buscar a Jesús. Se negaban a creer que se les había escapado, y mientras Jesús y su grupo viajaban hacia el norte por Batanea, los fariseos y sus ayudantes se pasaron casi una semana entera buscándolo en vano por las inmediaciones de Cafarnaúm.

\par 
%\textsuperscript{(1724.1)}
\textsuperscript{154:7.5} La familia de Jesús volvió a sus hogares de Cafarnaúm, y pasaron casi una semana hablando, discutiendo y orando. Estaban llenos de confusión y de consternación. No disfrutaron de tranquilidad hasta el jueves por la tarde, cuando Rut volvió de una visita a la casa de Zebedeo, donde David le había informado que su hermano-padre estaba a salvo y con buena salud, y que se dirigía hacia la costa de Fenicia.


\chapter{Documento 155. La huida por el norte de Galilea}
\par 
%\textsuperscript{(1725.1)}
\textsuperscript{155:0.1} POCO después de desembarcar cerca de Jeresa este domingo memorable, Jesús y los veinticuatro caminaron un poco hacia el norte, donde pasaron la noche en un hermoso parque al sur de Betsaida-Julias. Conocían bien este lugar para acampar porque se habían detenido allí en el pasado. Antes de retirarse para pasar la noche, el Maestro convocó a sus seguidores alrededor de él y discutió con ellos los planes del viaje que tenían proyectado hasta la costa de Fenicia, pasando por Batanea y el norte de Galilea.

\section*{1. ¿Por qué están furiosos los paganos?}
\par 
%\textsuperscript{(1725.2)}
\textsuperscript{155:1.1} Jesús dijo: <<Todos deberíais recordar lo que el salmista indicó sobre estos tiempos, cuando dijo: `¿Por qué están furiosos los paganos y los pueblos conspiran en vano? Los reyes de la Tierra se establecen a sí mismos, y los dirigentes del pueblo se aconsejan entre ellos, en contra del Señor y en contra de su ungido, diciendo: Rompamos las ataduras de la misericordia y desechemos los lazos del amor.'>>

\par 
%\textsuperscript{(1725.3)}
\textsuperscript{155:1.2} <<Hoy veis que esto se cumple delante de vuestros ojos. Pero no veréis cumplirse el resto de la profecía del salmista, porque tenía ideas erróneas sobre el Hijo del Hombre y su misión en la Tierra. Mi reino está basado en el amor, es proclamado con misericordia y se establece mediante el servicio desinteresado. Mi Padre no está sentado en el cielo riéndose burlonamente de los paganos. No está colérico en su gran desagrado. Es verdad la promesa de que el Hijo recibirá por herencia a esos llamados paganos (en realidad, sus hermanos ignorantes y faltos de instrucción). Y recibiré a esos gentiles con los brazos abiertos, con misericordia y afecto. Se mostrará toda esta misericordia a los supuestos paganos, a pesar de la desacertada declaración de la escritura que insinúa que el Hijo triunfante `los quebrantará con una barra de hierro y los hará añicos como una vasija de alfarero.' El salmista os exhortaba a: `servir al Señor con temor' ---Yo os invito a que entréis, por la fe, en los elevados privilegios de la filiación divina; él os ordena que os regocijéis temblando. Yo os pido que os regocijéis con seguridad. Él dice: `Besad al Hijo, no sea que se irrite y perezcáis cuando se encienda su cólera.' Pero vosotros, que habéis vivido conmigo, sabéis muy bien que la ira y la cólera no forman parte del establecimiento del reino de los cielos en el corazón de los hombres. Sin embargo, el salmista vislumbró la verdadera luz cuando dijo al final de esta exhortación: `Benditos sean los que ponen su confianza en este Hijo.'>>

\par 
%\textsuperscript{(1725.4)}
\textsuperscript{155:1.3} Jesús continuó enseñando a los veinticuatro, diciendo: <<Los paganos no están faltos de razón al estar furiosos con nosotros. Como su concepto de la vida es limitado y estrecho, pueden concentrar sus energías con entusiasmo. Tienen una meta cercana y más o menos visible; por eso se esfuerzan con una destreza valiente y eficaz. Vosotros, que habéis confesado vuestra entrada en el reino de los cielos, sois en general demasiado vacilantes e imprecisos en vuestra manera de enseñar. Los paganos se dirigen directamente hacia sus objetivos; vosotros sois culpables de tener demasiados anhelos crónicos. Si deseáis entrar en el reino, ¿por qué no os apoderáis de él mediante un asalto espiritual, como los paganos se apoderan de una ciudad sitiada? Difícilmente sois dignos del reino cuando vuestro servicio consiste tan ampliamente en la actitud de lamentaros del pasado, quejaros del presente y tener una esperanza vana para el futuro. ¿Por qué están furiosos los paganos? Porque no conocen la verdad. ¿Por qué languidecéis en anhelos fútiles? Porque no \textit{obedecéis} a la verdad. Poned fin a vuestras ansias inútiles y salid a hacer valientemente lo que está relacionado con el establecimiento del reino>>.

\par 
%\textsuperscript{(1726.1)}
\textsuperscript{155:1.4} <<En todo lo que hagáis, no os volváis parciales y no os especialicéis con exceso. Los fariseos que buscan nuestra destrucción creen de verdad que están sirviendo a Dios. La tradición los ha limitado tanto, que están cegados por los prejuicios y endurecidos por el miedo. Contemplad a los griegos, que tienen una ciencia sin religión, mientras que los judíos tienen una religión desprovista de ciencia. Cuando los hombres se extravían de esta manera, aceptando una desintegración estrecha y confusa de la verdad, su única esperanza de salvación consiste en coordinarse con la verdad ---en convertirse>>.

\par 
%\textsuperscript{(1726.2)}
\textsuperscript{155:1.5} <<Dejadme expresar enérgicamente esta verdad eterna: Si gracias a vuestra coordinación con la verdad, aprendéis a manifestar en vuestra vida esta hermosa integridad de la rectitud, entonces vuestros semejantes os buscarán para conseguir lo que habéis adquirido así. La cantidad de buscadores de la verdad que se sentirán atraídos hacia vosotros representa la medida de vuestra dotación de la verdad, de vuestra rectitud. La cantidad de mensaje que tenéis que llevar a la gente es, en cierto modo, la medida de vuestro fracaso en vivir la vida plena o recta, la vida coordinada con la verdad>>.

\par 
%\textsuperscript{(1726.3)}
\textsuperscript{155:1.6} El Maestro enseñó otras muchas cosas a sus apóstoles y a los evangelistas antes de que le desearan las buenas noches y se retiraran a descansar.

\section*{2. Los evangelistas en Corazín}
\par 
%\textsuperscript{(1726.4)}
\textsuperscript{155:2.1} El lunes 23 de mayo por la mañana, Jesús ordenó a Pedro que fuera a Corazín con los doce evangelistas, mientras que él partía con los once hacia Cesarea de Filipo, dirigiéndose por la ruta del Jordán hasta la carretera de Damasco a Cafarnaúm; desde allí fueron por el nordeste hasta la unión con la carretera de Cesarea de Filipo, continuando luego hasta esta ciudad, donde se detuvieron y enseñaron durante dos semanas. Llegaron en el transcurso de la tarde del martes 24 de mayo.

\par 
%\textsuperscript{(1726.5)}
\textsuperscript{155:2.2} Pedro y los evangelistas permanecieron dos semanas en Corazín, predicando el evangelio del reino a un grupo de creyentes poco numeroso, pero serio. No pudieron conseguir muchos nuevos conversos. Ninguna otra ciudad de Galilea dio menos almas al reino que Corazín. Siguiendo las instrucciones de Pedro, los doce evangelistas hablaron menos sobre las curaciones ---las cosas físicas--- y se dedicaron a predicar y a enseñar, con un vigor acrecentado, las verdades espirituales del reino celestial. Estas dos semanas en Corazín constituyeron un verdadero bautismo de adversidad para los doce evangelistas, en el sentido de que éste fue el período más difícil e improductivo de su carrera hasta ese momento. Al sentirse privados así de la satisfacción de conseguir almas para el reino, cada uno de ellos hizo un inventario más serio y honrado de su propia alma y del progreso de ésta en los senderos espirituales de la nueva vida.

\par 
%\textsuperscript{(1726.6)}
\textsuperscript{155:2.3} Cuando se hizo evidente que no había más gente que estuviera dispuesta a intentar entrar en el reino, Pedro convocó a sus compañeros, el martes 7 de junio, y partió para reunirse con Jesús y los apóstoles en Cesarea de Filipo. Llegaron el miércoles alrededor del mediodía y pasaron toda la tarde narrando sus experiencias con los incrédulos de Corazín. Durante las discusiones de esta tarde, Jesús se refirió de nuevo a la parábola del sembrador y les enseñó muchas cosas sobre el significado de los fracasos aparentes en las empresas de la vida.

\section*{3. En Cesarea de Filipo}
\par 
%\textsuperscript{(1727.1)}
\textsuperscript{155:3.1} Aunque Jesús no efectuó ninguna labor pública durante esta estancia de dos semanas cerca de Cesarea de Filipo, los apóstoles celebraron por las tardes numerosas reuniones tranquilas en la ciudad; muchos creyentes fueron hasta el campamento para hablar con el Maestro, pero muy pocos de ellos fueron agregados al grupo de creyentes como resultado de esta visita. Jesús conversó diariamente con los apóstoles y éstos discernieron con más claridad que ahora estaba empezando una nueva fase de la tarea de predicar el reino de los cielos. Estaban empezando a comprender que el <<reino de los cielos no es comida y bebida, sino la realización de la alegría espiritual de aceptar la filiación divina>>.

\par 
%\textsuperscript{(1727.2)}
\textsuperscript{155:3.2} La estancia en Cesarea de Filipo fue una verdadera prueba para los once apóstoles; fueron dos semanas difíciles de pasar para todos ellos. Estaban casi deprimidos, y echaban de menos el estímulo periódico de la personalidad entusiasta de Pedro. En aquellos momentos, el hecho de creer en Jesús y de ponerse a seguirlo era realmente una gran aventura y una prueba. Hicieron pocas conversiones durante estas dos semanas, pero aprendieron muchas cosas de sus conferencias diarias con el Maestro, que fueron muy beneficiosas para ellos.

\par 
%\textsuperscript{(1727.3)}
\textsuperscript{155:3.3} Los apóstoles aprendieron que los judíos estaban espiritualmente estancados y moribundos porque habían cristalizado la verdad en un credo; que cuando se formula la verdad como una línea divisoria de exclusivismo presuntuoso, en lugar de servir como un poste indicador para la orientación y el progreso espiritual, dichas enseñanzas pierden su poder creativo y vivificante, y acaban por volverse simplemente conservadoras y fosilizantes.

\par 
%\textsuperscript{(1727.4)}
\textsuperscript{155:3.4} Aprendieron cada vez más de Jesús a considerar a las personalidades humanas en términos de sus posibilidades en el tiempo y en la eternidad. Aprendieron que a muchas almas se les puede inducir mejor a amar al Dios invisible, si primero se les enseña a amar a sus hermanos que pueden ver. En relación con estas lecciones, se atribuyó un nuevo significado a la declaración del Maestro sobre el servicio desinteresado a los semejantes: <<Puesto que lo habéis hecho por el más humilde de mis hermanos, lo habéis hecho por mí>>.

\par 
%\textsuperscript{(1727.5)}
\textsuperscript{155:3.5} Una de las grandes lecciones de esta estancia en Cesarea tuvo que ver con el origen de las tradiciones religiosas, con el grave peligro de permitir que se atribuya un carácter sagrado a las cosas no sagradas, a las ideas corrientes o a los acontecimientos cotidianos. De una de estas conferencias salieron con la enseñanza de que la verdadera religión es la lealtad que un hombre siente en el fondo de su corazón hacia sus convicciones más elevadas y más sinceras.

\par 
%\textsuperscript{(1727.6)}
\textsuperscript{155:3.6} Jesús advirtió a sus creyentes que, si sus anhelos religiosos eran únicamente materiales, el conocimiento creciente de la naturaleza acabaría por quitarles su fe en Dios, debido a la sustitución progresiva del origen supuestamente sobrenatural de las cosas. Pero si su religión era espiritual, el progreso de la ciencia física nunca podría perturbar su fe en las realidades eternas y en los valores divinos.

\par 
%\textsuperscript{(1727.7)}
\textsuperscript{155:3.7} Aprendieron que cuando la religión tiene unos móviles enteramente espirituales, hace que toda la vida valga más la pena, llenándola de objetivos elevados, dignificándola con valores transcendentales, inspirándola con móviles magníficos, y confortando todo el tiempo el alma humana con una esperanza sublime y vigorizante. La verdadera religión está destinada a disminuir las tensiones de la existencia; libera la fe y el coraje para la vida diaria y el servicio desinteresado. La fe fomenta la vitalidad espiritual y la fecundidad de la rectitud.

\par 
%\textsuperscript{(1727.8)}
\textsuperscript{155:3.8} Jesús enseñó repetidas veces a sus apóstoles que ninguna civilización puede sobrevivir mucho tiempo a la pérdida de las mejores cosas que posee su religión. Nunca se cansó de señalar a los doce el gran peligro que supone aceptar los símbolos y las ceremonias religiosos como sustitutos de la experiencia religiosa. Toda su vida terrestre estuvo firmemente consagrada a la misión de derretir las formas congeladas de la religión, para darles las libertades líquidas de una filiación iluminada.

\section*{4. En el camino de Fenicia}
\par 
%\textsuperscript{(1728.1)}
\textsuperscript{155:4.1} El jueves 9 de junio por la mañana, después de que los mensajeros de David trajeran de Betsaida las noticias relacionadas con el progreso del reino, este grupo de veinticinco instructores de la verdad abandonó Cesarea de Filipo para emprender su viaje hacia la costa de Fenicia. Rodearon la región pantanosa, pasando por Luz, hasta el empalme con el camino de Magdala hacia el Monte Líbano, y desde allí hasta el cruce con la carretera que conducía a Sidón, donde llegaron el viernes por la tarde.

\par 
%\textsuperscript{(1728.2)}
\textsuperscript{155:4.2} Mientras se detenían para almorzar a la sombra de una cornisa rocosa inclinada, cerca de Luz, Jesús pronunció uno de los discursos más notables que sus apóstoles hubieran escuchado nunca a lo largo de todos sus años de asociación con él. Apenas se habían sentado para partir el pan, Simón Pedro le preguntó a Jesús: <<Maestro, puesto que el Padre que está en los cielos lo sabe todo, y puesto que su espíritu es nuestro sostén para establecer el reino de los cielos en la Tierra, ¿cómo es que huimos de las amenazas de nuestros enemigos? ¿Por qué nos negamos a enfrentarnos con los enemigos de la verdad?>> Pero antes de que Jesús hubiera empezado a contestar la pregunta de Pedro, Tomás interrumpió para interrogar: <<Maestro, me gustaría saber realmente qué hay exactamente de erróneo en la religión de nuestros enemigos de Jerusalén. ¿Cuál es la diferencia real entre su religión y la nuestra? ¿Cómo puede ser que tengamos tanta diversidad de creencias si todos profesamos servir al mismo Dios?>> Cuando Tomás hubo terminado, Jesús dijo: <<No deseo ignorar la pregunta de Pedro, porque sé muy bien lo fácil que es malinterpretar mis razones para evitar un choque abierto con los jefes de los judíos en este preciso momento; pero sin embargo, será más útil para todos vosotros que elija contestar más bien la pregunta de Tomás. Y eso es lo que voy a hacer cuando hayáis terminado de almorzar>>.

\section*{5. El discurso sobre la verdadera religión}
\par 
%\textsuperscript{(1728.3)}
\textsuperscript{155:5.1} Este discurso memorable sobre la religión, resumido y expuesto de nuevo en un lenguaje moderno, expresó las verdades siguientes:

\par 
%\textsuperscript{(1728.4)}
\textsuperscript{155:5.2} Aunque las religiones del mundo tienen un origen doble ---natural y revelado--- en todo momento se pueden encontrar, en cualquier pueblo, tres formas distintas de devoción religiosa. Estas tres manifestaciones del impulso religioso son:

\par 
%\textsuperscript{(1728.5)}
\textsuperscript{155:5.3} 1. \textit{La religión primitiva}. La propensión seminatural e instintiva a tener miedo de las energías misteriosas y a adorar las fuerzas superiores; es principalmente una religión de la naturaleza física, la religión del miedo.

\par 
%\textsuperscript{(1728.6)}
\textsuperscript{155:5.4} 2. \textit{La religión de la civilización}. Los conceptos y las prácticas religiosos progresivos de las razas que se civilizan ---la religión de la mente--- la teología intelectual basada en la autoridad de la tradición religiosa establecida.

\par 
%\textsuperscript{(1728.7)}
\textsuperscript{155:5.5} 3. \textit{La verdadera religión} ---\textit{la religión de la revelación}. La revelación de los valores sobrenaturales, un atisbo parcial de las realidades eternas, un vislumbre de la bondad y la belleza del carácter infinito del Padre que está en los cielos ---la religión del espíritu tal como está demostrada en la experiencia humana.

\par 
%\textsuperscript{(1729.1)}
\textsuperscript{155:5.6} El Maestro se negó a menospreciar la religión de los sentidos físicos y de los temores supersticiosos del hombre común, aunque deploró el hecho de que sobrevivieran tantos elementos de esta forma primitiva de adoración en las prácticas religiosas de las razas más inteligentes de la humanidad. Jesús indicó claramente que la gran diferencia entre la religión de la mente y la religión del espíritu reside en que, mientras la primera está sostenida por la autoridad eclesiástica, la segunda está enteramente basada en la experiencia humana.

\par 
%\textsuperscript{(1729.2)}
\textsuperscript{155:5.7} Luego, durante su hora de enseñanza, el Maestro continuó aclarando las verdades siguientes:

\par 
%\textsuperscript{(1729.3)}
\textsuperscript{155:5.8} Hasta que las razas se vuelvan sumamente inteligentes y más completamente civilizadas, seguirán existiendo muchas de esas ceremonias infantiles y supersticiosas que son tan características de las prácticas religiosas evolutivas de los pueblos primitivos y atrasados. Hasta que la raza humana no alcance el nivel de un reconocimiento más elevado y más general de las realidades de la experiencia espiritual, un gran número de hombres y mujeres continuarán mostrando su preferencia personal por esas religiones de autoridad que sólo requieren un asentimiento intelectual, en contraste con la religión del espíritu, que implica una participación activa de la mente y del alma en la aventura de la fe consistente en luchar con las realidades rigurosas de la experiencia humana progresiva.

\par 
%\textsuperscript{(1729.4)}
\textsuperscript{155:5.9} La aceptación de las religiones tradicionales de autoridad representa la salida fácil para el impulso que siente el hombre de intentar satisfacer las ansias de su naturaleza espiritual. Las religiones de autoridad, asentadas, cristalizadas y establecidas, proporcionan un refugio disponible donde el alma trastornada y angustiada del hombre puede huir cuando se siente abrumada por el miedo y atormentada por la incertidumbre. Como precio a pagar por las satisfacciones y las seguridades que proporciona, una religión así sólo exige a sus devotos un asentimiento pasivo y puramente intelectual.

\par 
%\textsuperscript{(1729.5)}
\textsuperscript{155:5.10} En la Tierra vivirán durante mucho tiempo esos individuos tímidos, miedosos e indecisos que preferirán obtener de esta manera sus consuelos religiosos, aunque al ligar su suerte con las religiones de autoridad, comprometen la soberanía de su personalidad, degradan la dignidad de la autoestima, y renuncian por completo al derecho de participar en la más emocionante e inspiradora de todas las experiencias humanas posibles: la búsqueda personal de la verdad, el regocijo de afrontar los peligros del descubrimiento intelectual, la determinación de explorar las realidades de la experiencia religiosa personal, la satisfacción suprema de experimentar el triunfo personal de conseguir realmente la victoria de la fe espiritual sobre las dudas intelectuales, una victoria que se gana honradamente durante la aventura suprema de toda la existencia humana ---el hombre a la búsqueda de Dios, por sí mismo y como tal hombre, y que lo encuentra.

\par 
%\textsuperscript{(1729.6)}
\textsuperscript{155:5.11} La religión del espíritu significa esfuerzo, lucha, conflicto, fe, determinación, amor, lealtad y progreso. La religión de la mente ---la teología de la autoridad--- exige pocos o ninguno de estos esfuerzos a sus creyentes formales. La tradición es un refugio seguro y un sendero fácil para las almas temerosas y sin entusiasmo que rehuyen instintivamente las luchas espirituales y las incertidumbres mentales que acompañan a esos viajes, en la fe, de aventuras atrevidas por los altos mares de la verdad inexplorada, en búsqueda de las orillas muy lejanas de las realidades espirituales, tal como éstas pueden ser descubiertas por la mente humana progresiva, y experimentadas por el alma humana en evolución.

\par 
%\textsuperscript{(1729.7)}
\textsuperscript{155:5.12} Jesús continuó diciendo: <<En Jerusalén, los jefes religiosos han formulado un sistema establecido de creencias intelectuales, una religión de autoridad, con las diversas doctrinas de sus instructores tradicionales y de los profetas de antaño. Todo ese tipo de religiones recurre principalmente a la mente. Ahora estamos a punto de entrar en un conflicto implacable con ese tipo de religión, puesto que muy pronto vamos a empezar a proclamar audazmente una nueva religión ---una religión que no es una religión en el sentido que hoy se atribuye a esa palabra, una religión que apela principalmente al espíritu divino de mi Padre que reside en la mente del hombre; una religión que obtendrá su autoridad de los frutos de su aceptación, unos frutos que aparecerán con toda seguridad en la experiencia personal de todos los que se conviertan en creyentes reales y sinceros de las verdades de esta comunión espiritual superior>>.

\par 
%\textsuperscript{(1730.1)}
\textsuperscript{155:5.13} Señalando a cada uno de los veinticuatro y llamándolos por su nombre, Jesús dijo: <<Y ahora, ¿quién de vosotros preferiría coger ese sendero fácil del conformismo a una religión establecida y fosilizada, como la que defienden los fariseos de Jerusalén, en lugar de sufrir las dificultades y persecuciones que acompañarán la misión de proclamar un camino mejor de salvación para los hombres, mientras obtenéis la satisfacción de descubrir, por vosotros mismos, las bellezas de las realidades de una experiencia viviente y personal de las verdades eternas y de las grandezas supremas del reino de los cielos? ¿Sois miedosos, blandos y buscáis la facilidad? ¿Tenéis miedo de confiar vuestro futuro entre las manos del Dios de la verdad, de quien sois hijos? ¿Desconfiáis del Padre, de quien sois hijos? ¿Vais a retroceder al sendero fácil de la certidumbre y de la estabilidad intelectual de la religión de autoridad tradicional, o vais a ceñiros para avanzar conmigo en el futuro incierto y agitado en el que proclamaremos las verdades nuevas de la religión del espíritu, el reino de los cielos en el corazón de los hombres?>>

\par 
%\textsuperscript{(1730.2)}
\textsuperscript{155:5.14} Sus veinticuatro oyentes se pusieron todos de pie con la intención de anunciar su respuesta unánime y leal a este llamamiemto emotivo, uno de los pocos que Jesús les hizo nunca, pero él levantó la mano y los detuvo, diciendo: <<Separaos ahora; que cada uno se quede a solas con el Padre, y encuentre allí la respuesta no emotiva a mi pregunta. Una vez que hayáis descubierto la actitud verdadera y sincera de vuestra alma, expresad esa respuesta de manera franca y audaz a mi Padre y vuestro Padre, cuya vida infinita de amor es el espíritu mismo de la religión que proclamamos>>.

\par 
%\textsuperscript{(1730.3)}
\textsuperscript{155:5.15} Los evangelistas y los apóstoles se separaron cada uno por su lado durante un corto período de tiempo. Tenían el espíritu elevado, la mente inspirada y las emociones poderosamente agitadas por las palabras de Jesús. Sin embargo, cuando Andrés los reunió, el Maestro se limitó a decir: <<Reanudemos nuestro viaje. Vamos a Fenicia para quedarnos una temporada, y todos deberíais orar al Padre para que transforme vuestras emociones mentales y corporales en lealtades mentales superiores y en experiencias espirituales más satisfactorias>>.

\par 
%\textsuperscript{(1730.4)}
\textsuperscript{155:5.16} Los veinticuatro permanecieron silenciosos mientras bajaban por el camino, pero pronto empezaron a charlar entre ellos, y a las tres de la tarde ya no pudieron aguantar más. Se detuvieron, y Pedro se acercó a Jesús, diciendo: <<Maestro, nos has dirigido palabras de vida y de verdad. Quisiéramos escuchar más; te rogamos que continúes hablándonos de estas materias>>.

\section*{6. El segundo discurso sobre la religión}
\par 
%\textsuperscript{(1730.5)}
\textsuperscript{155:6.1} Y así, mientras hacían una pausa a la sombra de una ladera, Jesús continuó enseñándoles acerca de la religión del espíritu, diciendo en esencia:

\par 
%\textsuperscript{(1730.6)}
\textsuperscript{155:6.2} Habéis surgido de entre aquellos semejantes vuestros que han elegido permanecer satisfechos con una religión de la mente, que ansían la seguridad y prefieren el conformismo. Habéis elegido cambiar vuestros sentimientos de certidumbre autoritaria por las seguridades del espíritu de una fe aventurera y progresiva. Os habéis atrevido a protestar contra la esclavitud abrumadora de una religión institucional y a rechazar la autoridad de las tradiciones escritas actualmente consideradas como la palabra de Dios. Nuestro Padre habló en verdad a través de Moisés, Elías, Isaías, Amós y Oseas, pero no ha dejado de suministrar al mundo palabras de verdad cuando estos antiguos profetas terminaron sus proclamaciones. Mi Padre no hace acepción de razas ni de generaciones, en el sentido de que la palabra de la verdad sea otorgada a una época y ocultada a la siguiente. No cometáis la locura de llamar divino a lo que es puramente humano, y no dejéis de discernir las palabras de la verdad, aunque no provengan de los oráculos tradicionales de una supuesta inspiración.

\par 
%\textsuperscript{(1731.1)}
\textsuperscript{155:6.3} Os he llamado para que nazcáis de nuevo, para que nazcáis del espíritu. Os he llamado para que salgáis de las tinieblas de la autoridad y del letargo de la tradición, y entréis en la luz trascendente donde obtendréis la posibilidad de hacer por vosotros mismos el mayor descubrimiento posible que el alma humana puede hacer ---la experiencia celestial de encontrar a Dios por vosotros mismos, en vosotros mismos y para vosotros mismos, y efectuar todo esto como un hecho en vuestra propia experiencia personal. Así podréis pasar de la muerte a la vida, de la autoridad de la tradición a la experiencia de conocer a Dios; así pasaréis de las tinieblas a la luz, de una fe racial heredada a una fe personal conseguida mediante una experiencia real; de este modo progresaréis de una teología de la mente, transmitida por vuestros antepasados, a una verdadera religión del espíritu que será edificada en vuestra alma como una dotación eterna.

\par 
%\textsuperscript{(1731.2)}
\textsuperscript{155:6.4} Vuestra religión dejará de ser una simple creencia intelectual en una autoridad tradicional, para convertirse en la experiencia efectiva de esa fe viviente que es capaz de captar la realidad de Dios y todo lo relacionado con el espíritu divino del Padre. La religión de la mente os ata sin esperanzas al pasado; la religión del espíritu consiste en una revelación progresiva y os llama constantemente para que consigáis unos ideales espirituales y unas realidades eternas más elevados y más santos.

\par 
%\textsuperscript{(1731.3)}
\textsuperscript{155:6.5} Aunque la religión de autoridad puede conferir un sentimiento inmediato de seguridad estable, el precio que pagáis por esa satisfacción pasajera es la pérdida de vuestra independencia espiritual y de vuestra libertad religiosa. Como precio para entrar en el reino de los cielos, mi Padre no os exige que os forcéis a creer en cosas que son espiritualmente repugnantes, impías y falsas. No se os pide que ultrajéis vuestro propio sentido de la misericordia, de la justicia y de la verdad por medio de vuestro sometimiento a un sistema anticuado de formalidades y de ceremonias religiosas. La religión del espíritu os deja eternamente libres para seguir la verdad, dondequiera que os lleven las directrices del espíritu. ¿Y quién puede juzgar ---quizás este espíritu tenga algo que comunicar a esta generación, que las otras generaciones han rehusado escuchar?

\par 
%\textsuperscript{(1731.4)}
\textsuperscript{155:6.6} ¡Verg\"uenza deberían sentir esos falsos educadores religiosos, que quisieran arrastrar a las almas hambrientas al oscuro y lejano pasado, para luego abandonarlas allí! Esas personas desgraciadas están condenadas así a asustarse de todo nuevo descubrimiento, y a sentirse desconcertadas con cada nueva revelación de la verdad. El profeta que dijo: <<Aquel cuya mente descansa en Dios se mantendrá en una paz perfecta>> no era un simple creyente intelectual en una teología autoritaria. Este ser humano, que conocía la verdad, había descubierto a Dios; no se limitaba a hablar de Dios.

\par 
%\textsuperscript{(1731.5)}
\textsuperscript{155:6.7} Os recomiendo que abandonéis la costumbre de citar constantemente a los profetas del pasado y de alabar a los héroes de Israel; aspirad más bien a convertiros en profetas vivientes del Altísimo y en héroes espirituales del reino venidero. En verdad, quizás valga la pena honrar a los jefes del pasado que conocían a Dios, pero cuando lo hagáis, ¿por qué tenéis que sacrificar la experiencia suprema de la existencia humana: encontrar a Dios por vosotros mismos y conocerlo en vuestra propia alma?

\par 
%\textsuperscript{(1732.1)}
\textsuperscript{155:6.8} Cada raza de la humanidad tiene su propia perspectiva mental sobre la existencia humana; por consiguiente, la religión de la mente debe siempre armonizarse con estos diversos puntos de vista raciales. Las religiones de autoridad nunca podrán llegar a unificarse. La unidad humana y la fraternidad de los mortales sólo se pueden conseguir por medio, y a través de, la dotación superior de la religión del espíritu. Las mentes de las razas pueden ser diferentes, pero toda la humanidad está habitada por el mismo espíritu divino y eterno. La esperanza de la fraternidad humana sólo se puede realizar cuando, y a medida que, la religión unificante y ennoblecedora del espíritu ---la religión de la experiencia espiritual personal--- impregne y eclipse a las religiones de autoridad mentales y divergentes.

\par 
%\textsuperscript{(1732.2)}
\textsuperscript{155:6.9} Las religiones de autoridad sólo pueden dividir a los hombres y levantar unas conciencias contra otras; la religión del espíritu unirá progresivamente a los hombres y los inducirá a sentir una simpatía comprensiva los unos por los otros. Las religiones de autoridad exigen a los hombres una creencia uniforme, pero esto es imposible de realizar en el estado actual del mundo. La religión del espíritu sólo exige una unidad de experiencia ---un destino uniforme--- aceptando plenamente la diversidad de creencias. La religión del espíritu sólo pide la uniformidad de perspicacia, no la uniformidad de punto de vista ni de perspectiva. La religión del espíritu no exige la uniformidad de puntos de vista intelectuales, sino solamente la unidad de sentimientos espirituales. Las religiones de autoridad se cristalizan en credos sin vida; la religión del espíritu se desarrolla en la alegría y la libertad crecientes de las acciones ennoblecedoras del servicio amoroso y de la ayuda misericordiosa.

\par 
%\textsuperscript{(1732.3)}
\textsuperscript{155:6.10} Pero tened cuidado, no sea que alguno de vosotros considere con desdén a los hijos de Abraham porque les ha tocado vivir en estos malos tiempos de tradición estéril. Nuestros antepasados se dedicaron de lleno a la búsqueda insistente y apasionada de Dios, y lo descubrieron como ninguna otra raza total de hombres lo ha conocido nunca desde los tiempos de Adán, que sabía muchas de estas cosas, porque él mismo era un Hijo de Dios. Mi Padre no ha dejado de observar la larga e incansable lucha de Israel, desde la época de Moisés, por encontrar y conocer a Dios. Durante largas generaciones, los judíos no han dejado de afanarse, sudar, gemir, trabajar, soportar los sufrimientos y experimentar las tristezas de un pueblo incomprendido y despreciado, y todo ello para poder acercarse un poco más al descubrimiento de la verdad acerca de Dios. Desde Moisés hasta los tiempos de Amós y de Oseas, y a pesar de todos los fracasos y titubeos de Israel, nuestros padres revelaron progresivamente a todo el mundo una imagen cada vez más clara y más verdadera del Dios eterno. Así es como se preparó el camino para la revelación aún más grande del Padre, en la que habéis sido llamados a participar.

\par 
%\textsuperscript{(1732.4)}
\textsuperscript{155:6.11} No olvidéis nunca que sólo hay una aventura más satisfactoria y emocionante que la tentativa de descubrir la voluntad del Dios vivo, y es la experiencia suprema de intentar hacer honradamente esa voluntad divina. Y recordad siempre que la voluntad de Dios se puede hacer en cualquier ocupación terrestre. No hay profesiones santas y profesiones laicas. Todas las cosas son sagradas en la vida de aquellos que están dirigidos por el espíritu, es decir, subordinados a la verdad, ennoblecidos por el amor, dominados por la misericordia y refrenados por la equidad ---por la justicia. El espíritu que mi Padre y yo enviaremos al mundo no es solamente el Espíritu de la Verdad, sino también el espíritu de la belleza idealista.

\par 
%\textsuperscript{(1732.5)}
\textsuperscript{155:6.12} Tenéis que dejar de buscar la palabra de Dios únicamente en las páginas de los viejos escritos con autoridad teológica. Aquellos que han nacido del espíritu de Dios discernirán en lo sucesivo la palabra de Dios, independientemente del lugar de donde parezca originarse. No hay que desestimar la verdad divina porque se haya otorgado a través de un canal aparentemente humano. Muchos de vuestros hermanos aceptan mentalmente la teoría de Dios, pero no consiguen darse cuenta espiritualmente de la presencia de Dios. Ésta es precisamente la razón por la que os he enseñado tantas veces que la mejor manera de comprender el reino de los cielos es adquiriendo la actitud espiritual de un niño sincero. No os recomiendo la inmadurez mental de un niño, sino más bien la \textit{ingenuidad espiritual} de un pequeño que cree con facilidad y que confía plenamente. No es tan importante que conozcáis el hecho de Dios, como que desarrolléis cada vez más la habilidad de \textit{sentir la presencia de Dios}.

\par 
%\textsuperscript{(1733.1)}
\textsuperscript{155:6.13} Una vez que empecéis a descubrir a Dios en vuestra alma, no tardaréis en empezar a descubrirlo en el alma de los otros hombres, y finalmente en todas las criaturas y creaciones de un poderoso universo. Pero ¿qué posibilidades tiene el Padre de aparecer, como el Dios de las lealtades supremas y de los ideales divinos, en el alma de unos hombres que dedican poco o ningún tiempo a la contemplación reflexiva de estas realidades eternas? Aunque la mente no es la sede de la naturaleza espiritual, es en verdad la entrada que conduce a ella.

\par 
%\textsuperscript{(1733.2)}
\textsuperscript{155:6.14} Pero no cometáis el error de intentar probar a otros hombres que habéis encontrado a Dios; no podéis presentar conscientemente una prueba así de válida, aunque existen dos demostraciones positivas y poderosas del hecho de que conocéis a Dios, y son las siguientes:

\par 
%\textsuperscript{(1733.3)}
\textsuperscript{155:6.15} 1. La manifestación de los frutos del espíritu de Dios en vuestra vida diaria habitual.

\par 
%\textsuperscript{(1733.4)}
\textsuperscript{155:6.16} 2. El hecho de que todo el plan de vuestra vida proporciona una prueba positiva de que habéis arriesgado sin reserva todo lo que sois y poseéis en la aventura de la supervivencia después de la muerte, persiguiendo la esperanza de encontrar al Dios de la eternidad, cuya presencia habéis saboreado anticipadamente en el tiempo.

\par 
%\textsuperscript{(1733.5)}
\textsuperscript{155:6.17} Y ahora, no os equivoquéis, mi Padre responderá siempre a la más tenue llama vacilante de fe. Él toma nota de las emociones físicas y supersticiosas del hombre primitivo. Y con esas almas honradas pero temerosas, cuya fe es tan débil que no llega a ser mucho más que un conformismo intelectual a una actitud pasiva de asentimiento a las religiones de autoridad, el Padre siempre está alerta para honrar y fomentar incluso todas estas débiles tentativas por llegar hasta él. Pero se espera que vosotros, que habéis sido sacados de las tinieblas y traídos a la luz, creáis de todo corazón; vuestra fe dominará las actitudes combinadas del cuerpo, la mente y el espíritu.

\par 
%\textsuperscript{(1733.6)}
\textsuperscript{155:6.18} Vosotros sois mis apóstoles, y la religión no se convertirá para vosotros en un refugio teológico al que podréis huir cuando temáis enfrentaros con las duras realidades del progreso espiritual y de la aventura idealista. Vuestra religión se convertirá más bien en el hecho de una experiencia real que atestigua que Dios os ha encontrado, idealizado, ennoblecido y espiritualizado, y que os habéis alistado en la aventura eterna de encontrar al Dios que así os ha encontrado y os ha hecho hijos suyos.

\par 
%\textsuperscript{(1733.7)}
\textsuperscript{155:6.19} Cuando Jesús terminó de hablar, hizo una seña a Andrés, apuntó hacia el oeste en dirección a Fenicia, y dijo: <<Pongámonos en camino>>.


\chapter{Documento 156. La estancia en Tiro y Sidón}
\par 
%\textsuperscript{(1734.1)}
\textsuperscript{156:0.1} EL VIERNES 10 de junio por la tarde, Jesús y sus compañeros llegaron a las cercanías de Sidón, donde se detuvieron en la casa de una mujer rica que había sido paciente en el hospital de Betsaida durante la época en que Jesús se encontraba en la cumbre del favor popular. Los evangelistas y los apóstoles se alojaron con unos amigos de ella en las proximidades inmediatas, y descansaron el día del sábado en medio de estos paisajes vivificantes. Pasaron casi dos semanas y media en Sidón y sus cercanías antes de prepararse para visitar las ciudades costeras del norte.

\par 
%\textsuperscript{(1734.2)}
\textsuperscript{156:0.2} Este sábado de junio fue un día muy tranquilo. Los evangelistas y los apóstoles estaban totalmente absortos en sus meditaciones sobre los discursos del Maestro acerca de la religión, que habían escuchado en el camino hacia Sidón. Todos eran capaces de apreciar algo de lo que Jesús les había dicho, pero ninguno de ellos captaba plenamente la importancia de su enseñanza.

\section*{1. La mujer siria}
\par 
%\textsuperscript{(1734.3)}
\textsuperscript{156:1.1} Cerca de la casa de Karuska, donde se alojaba el Maestro, vivía una mujer siria que había oído hablar mucho de Jesús como gran sanador e instructor, y este sábado por la tarde vino a verlo con su hijita. La chica, que tenía unos doce años de edad, estaba afligida con un doloroso trastorno nervioso caracterizado por convulsiones y otras manifestaciones angustiosas.

\par 
%\textsuperscript{(1734.4)}
\textsuperscript{156:1.2} Jesús había encargado a sus asociados que no informaran a nadie de su presencia en la casa de Karuska, explicando que deseaba descansar. Aunque habían obedecido las instrucciones de su Maestro, la criada de Karuska había ido a la casa de esta mujer siria, llamada Norana, para informarle que Jesús estaba alojado en la casa de su ama, y había incitado a esta madre ansiosa a que llevara a su hija afligida para que la curara. Esta madre creía, por supuesto, que su hija estaba poseída por un demonio, por un espíritu impuro.

\par 
%\textsuperscript{(1734.5)}
\textsuperscript{156:1.3} Cuando Norana llegó con su hija, los gemelos Alfeo le explicaron, por medio de un intérprete, que el Maestro estaba descansando y que no se le podía molestar, a lo cual Norana replicó que se quedaría allí con la niña hasta que el Maestro hubiera terminado su descanso. Pedro también intentó razonar con ella y persuadirla para que volviera a su casa. Le explicó que Jesús estaba rendido de cansancio de tanto enseñar y curar, y que había venido a Fenicia para pasar un período de tranquilidad y descanso. Pero fue inútil. Norana no quiso irse. Ante las súplicas de Pedro, ella se limitó a responder: <<No me marcharé hasta que haya visto a tu Maestro. Sé que puede echar al demonio de mi niña, y no me iré hasta que el sanador haya visto a mi hija>>.

\par 
%\textsuperscript{(1734.6)}
\textsuperscript{156:1.4} Entonces, Tomás intentó despedir a la mujer, pero tampoco tuvo éxito. Ella le dijo: <<Tengo fe en que tu Maestro será capaz de echar a este demonio que atormenta a mi hija. He oído hablar de sus obras poderosas en Galilea, y creo en él. ¿Qué os ha sucedido a vosotros, sus discípulos, para que queráis despedir a los que vienen buscando la ayuda de vuestro Maestro?>> Cuando ella hubo dicho esto, Tomás se retiró.

\par 
%\textsuperscript{(1735.1)}
\textsuperscript{156:1.5} Luego se adelantó Simón Celotes para amonestar a Norana. Simón dijo: <<Mujer, eres una gentil que habla griego. No es justo que esperes que el Maestro coja el pan destinado a los hijos de la casa favorecida y se lo eche a los perros>>. Pero Norana rehusó ofenderse por el ataque de Simón. Se limitó a replicar: <<Sí, maestro, comprendo tus palabras. No soy más que un perro a los ojos de los judíos, pero en lo que respecta a tu Maestro, soy un perro creyente. Estoy decidida a que él vea a mi hija, porque estoy persuadida de que, con que sólo la mire, la curará. Y ni siquiera tú, buen hombre, te atreverías a privar a los perros del privilegio de conseguir las migajas que puedan caer de la mesa de los hijos>>.

\par 
%\textsuperscript{(1735.2)}
\textsuperscript{156:1.6} En ese preciso momento, la chiquilla sufrió una violenta convulsión delante de todos ellos, y la madre exclamó: <<Ahora podéis ver que mi hija está poseída por un espíritu maligno. Si nuestra miseria no os impresiona, sí conmoverá a vuestro Maestro, que me han dicho que ama a todos los hombres y que se atreve incluso a curar a los gentiles cuando estos creen. No sois dignos de ser sus discípulos. No me iré hasta que mi hija haya sido curada>>.

\par 
%\textsuperscript{(1735.3)}
\textsuperscript{156:1.7} Jesús, que había escuchado toda esta conversación por una ventana abierta, salió entonces, para gran sorpresa de todos, y dijo: <<Oh mujer, tu fe es grande, tan grande que no puedo rehusar lo que deseas; puedes irte en paz. Tu hija ya ha recuperado la salud>>. Y la chiquilla se sintió bien a partir de aquel momento. Cuando Norana y la niña iban a despedirse, Jesús les rogó que no le contaran a nadie este suceso; aunque sus compañeros sí cumplieron esta petición, la madre y la niña no dejaron de proclamar por toda la región, e incluso en Sidón, el hecho de que la chiquilla había sido curada, de tal manera que Jesús estimó conveniente cambiar de residencia pocos días después.

\par 
%\textsuperscript{(1735.4)}
\textsuperscript{156:1.8} Al día siguiente, mientras Jesús enseñaba a sus apóstoles, comentando la curación de la hija de la mujer siria, dijo: <<Siempre ha sido así desde el principio; ya veis por vosotros mismos que los gentiles son capaces de ejercer una fe salvadora en las enseñanzas del evangelio del reino de los cielos. En verdad, en verdad os digo que los gentiles se apoderarán del reino del Padre si los hijos de Abraham no están dispuestos a mostrar la fe suficiente para entrar en él>>.

\section*{2. La enseñanza en Sidón}
\par 
%\textsuperscript{(1735.5)}
\textsuperscript{156:2.1} Al entrar en Sidón, Jesús y sus asociados pasaron por un puente, el primero que muchos de ellos habían visto nunca. Mientras caminaban por él, entre otras cosas, Jesús dijo: <<Este mundo no es más que un puente; podéis atravesarlo, pero no deberíais pensar en construir una morada encima de él>>.

\par 
%\textsuperscript{(1735.6)}
\textsuperscript{156:2.2} Mientras los veinticuatro empezaron sus trabajos en Sidón, Jesús fue a quedarse en una casa situada exactamente al norte de la ciudad, en el hogar de Justa y de su madre Berenice. Todas las mañanas, Jesús enseñaba a los veinticuatro en la casa de Justa, y durante la tarde y la noche se marchaban a Sidón para enseñar y predicar.

\par 
%\textsuperscript{(1735.7)}
\textsuperscript{156:2.3} Los apóstoles y los evangelistas se sintieron muy animados por la manera en que los gentiles de Sidón recibieron su mensaje; durante su corta estancia, muchos de ellos se añadieron al reino. Este período de unas seis semanas en Fenicia fue un momento muy fructífero en la tarea de ganar almas, pero los escritores judíos que redactaron más tarde los evangelios cogieron la costumbre de pasar por alto alegremente la historia de esta cálida recepción que hicieron los gentiles a las enseñanzas de Jesús, en el preciso momento en que un número tan grande de su propia gente adoptaba una postura hostil contra él.

\par 
%\textsuperscript{(1736.1)}
\textsuperscript{156:2.4} En muchos aspectos, estos creyentes gentiles apreciaron las enseñanzas de Jesús de manera más completa que los judíos. Muchos de estos sirofenicios de habla griega no solamente llegaron a discernir que Jesús se parecía a Dios, sino también que Dios se parecía a Jesús. Estos supuestos paganos consiguieron comprender bien las enseñanzas del Maestro sobre la uniformidad de las leyes de este mundo y de todo el universo. Comprendieron la enseñanza de que Dios no hace acepción de personas, de razas o de naciones; que con el Padre Universal no existen los favoritismos; que el universo siempre obedece totalmente a las leyes y es infaliblemente digno de confianza. Estos gentiles no tenían miedo de Jesús; se atrevían a aceptar su mensaje. A lo largo de todos los siglos posteriores, los hombres no han sido incapaces de comprender a Jesús; han tenido miedo de comprenderlo.

\par 
%\textsuperscript{(1736.2)}
\textsuperscript{156:2.5} Jesús indicó claramente a los veinticuatro que no había huido de Galilea porque careciera de coraje para enfrentarse con sus enemigos. Comprendieron que aún no estaba preparado para un conflicto abierto con la religión establecida, y que no trataba de convertirse en un mártir. Durante una de estas conferencias en la casa de Justa, el Maestro dijo por primera vez a sus discípulos que <<aunque el cielo y la Tierra desaparezcan, mis palabras de verdad no desaparecerán>>.

\par 
%\textsuperscript{(1736.3)}
\textsuperscript{156:2.6} Durante la estancia en Sidón, el tema de las enseñanzas de Jesús fue el progreso espiritual. Dijo a sus discípulos que no podían detenerse; que tenían que progresar en rectitud o retroceder hacia el mal y el pecado. Les recomendó que <<se olvidaran de las cosas del pasado, mientras que avanzaban para abrazar las realidades más grandes del reino>>. Les rogó que no se contentaran con seguir siendo niños en el evangelio, sino que se esforzaran por alcanzar la plena estatura de la filiación divina en la comunión del espíritu y en la hermandad de los creyentes.

\par 
%\textsuperscript{(1736.4)}
\textsuperscript{156:2.7} Jesús dijo: <<Mis discípulos no solamente deben dejar de hacer el mal, sino aprender a hacer el bien; no sólo tenéis que purificaros de todo pecado consciente, sino que tenéis que negaros incluso a albergar sentimientos de culpa. Si confesáis vuestros pecados, están perdonados; por eso tenéis que mantener una conciencia desprovista de faltas>>.

\par 
%\textsuperscript{(1736.5)}
\textsuperscript{156:2.8} Jesús disfrutaba mucho con el agudo sentido del humor que mostraban estos gentiles. El sentido del humor manifestado por Norana, la mujer siria, así como su fe grande y perseverante, fueron las cosas que conmovieron tanto el corazón del Maestro y atrajeron su misericordia. Jesús lamentaba mucho que su gente ---los judíos--- estuvieran tan faltos de humor. Una vez le dijo a Tomás: <<Mi gente se toma demasiado en serio a sí misma; son casi incapaces de apreciar el humor. La religión aburrida de los fariseos nunca podría haberse originado en un pueblo con sentido del humor. También les falta coherencia; filtran los mosquitos y se tragan los camellos>>.

\section*{3. El viaje subiendo por la costa}
\par 
%\textsuperscript{(1736.6)}
\textsuperscript{156:3.1} El martes 28 de junio, el Maestro y sus compañeros salieron de Sidón y subieron por la costa hasta Porfireón y Heldua. Fueron bien recibidos por los gentiles, y muchos de éstos ingresaron en el reino durante esta semana de enseñanza y predicación. Los apóstoles predicaron en Porfireón y los evangelistas enseñaron en Heldua. Mientras los veinticuatro se ocupaban así de su trabajo, Jesús los dejó durante un período de tres o cuatro días para hacer una visita a la ciudad costera de Beirut, donde estuvo charlando con un sirio llamado Malac, que era creyente y había estado en Betsaida el año anterior.

\par 
%\textsuperscript{(1737.1)}
\textsuperscript{156:3.2} El miércoles 6 de julio, todos regresaron a Sidón y permanecieron en la casa de Justa hasta el domingo por la mañana; entonces partieron hacia Tiro, dirigiéndose por la costa hacia el sur, por el camino de Sarepta, y llegaron a Tiro el lunes 11 de julio. Por esta época, los apóstoles y los evangelistas se estaban acostumbrando a trabajar entre estos llamados gentiles, que en realidad descendían principalmente de las antiguas tribus cananeas que tenían un origen semítico aún más antiguo. Todos estos pueblos hablaban la lengua griega. Los apóstoles y los evangelistas se quedaron muy sorprendidos al observar la avidez con que estos gentiles escuchaban el evangelio, y al advertir la prontitud con que muchos de ellos creían.

\section*{4. En Tiro}
\par 
%\textsuperscript{(1737.2)}
\textsuperscript{156:4.1} Desde el 11 hasta el 24 de julio enseñaron en Tiro. Cada uno de los apóstoles se llevó consigo a uno de los evangelistas, y así enseñaron y predicaron de dos en dos en todos los rincones de Tiro y sus alrededores. La población políglota de este activo puerto marítimo los escuchaba con placer, y muchos de ellos fueron bautizados en la hermandad exterior del reino. Jesús estableció su cuartel general en la casa de un judío llamado José, un creyente que vivía a cinco o seis kilómetros al sur de Tiro, no lejos de la tumba de Hiram, que había sido rey de la ciudad-Estado de Tiro en la época de David y Salomón.

\par 
%\textsuperscript{(1737.3)}
\textsuperscript{156:4.2} Durante este período de dos semanas, los apóstoles y los evangelistas entraban diariamente en Tiro por el muelle de Alejandro para dirigir pequeñas reuniones, y la mayoría de ellos regresaba cada noche al campamento de la casa de José, al sur de la ciudad. Los creyentes salían cada día de la ciudad para conversar con Jesús en el lugar donde estaba descansando. El Maestro sólo habló en Tiro una vez, el 20 de julio por la tarde, y enseñó a los creyentes sobre el amor del Padre por toda la humanidad y acerca de la misión del Hijo de revelar el Padre a todas las razas humanas. Estos gentiles mostraban tal interés por el evangelio del reino que, en esta ocasión, abrieron a Jesús las puertas del templo de Melcart, y es interesante indicar que en años posteriores se construyó una iglesia cristiana en el mismo lugar donde estaba situado este antiguo templo.

\par 
%\textsuperscript{(1737.4)}
\textsuperscript{156:4.3} En esta región se fabricaba la púrpura de Tiro, el tinte que había hecho famosas a Tiro y a Sidón en el mundo entero, y que había contribuido tanto a su comercio mundial y a su consiguiente enriquecimiento; muchos dirigentes de esta industria creyeron en el reino. Poco tiempo después empezó a disminuir el abastecimiento de animales marinos que servían para extraer este colorante, y estos fabricantes de tinte se fueron en busca de nuevas regiones donde se encontraban dichos mariscos. Así emigraron hasta los confines de la Tierra, llevando con ellos el mensaje de la paternidad de Dios y de la fraternidad de los hombres ---el evangelio del reino.

\section*{5. La enseñanza de Jesús en Tiro}
\par 
%\textsuperscript{(1737.5)}
\textsuperscript{156:5.1} En el transcurso de su alocución de este miércoles por la tarde, Jesús empezó contando a sus seguidores la historia del lirio blanco que alza su cabeza pura y nevada hacia la luz del Sol, mientras que sus raíces están enterradas en el lodo y el estiércol del suelo tenebroso. <<De la misma manera>>, dijo, <<aunque el hombre mortal tiene las raíces de su origen y de su ser en el suelo animal de la naturaleza humana, mediante la fe puede elevar su naturaleza espiritual hacia la luz solar de la verdad celestial, y producir realmente los nobles frutos del espíritu>>.

\par 
%\textsuperscript{(1738.1)}
\textsuperscript{156:5.2} Fue durante este mismo sermón cuando Jesús utilizó la primera y única parábola relacionada con su propio oficio ---la carpintería. En el transcurso de su recomendación sobre <<construir bien los cimientos para el crecimiento de un carácter noble impregnado de dones espirituales>>, dijo: <<Para producir los frutos del espíritu, tenéis que haber nacido del espíritu. El espíritu es el que debe enseñaros y conduciros si queréis vivir una vida de plenitud espiritual entre vuestros semejantes. Pero no cometáis el error del carpintero necio que derrocha un tiempo precioso cuadrando, midiendo y cepillando una madera de construcción carcomida por los gusanos y podrida en su interior, para después de haber consagrado todo su esfuerzo a esa viga podrida, tiene que rechazarla porque es inadecuada para formar parte de los cimientos del edificio que quería construir, y que deberán resistir los ataques del tiempo y de las tempestades. Que todo hombre se asegure de que los cimientos intelectuales y morales de su carácter tengan tal solidez que sostengan adecuadamente la superestructura de su naturaleza espiritual que aumenta y se ennoblece, la cual transformará así la mente mortal para después, en asociación con esa mente recreada, conseguir desarrollar el alma cuyo destino es inmortal. Vuestra naturaleza espiritual ---el alma creada conjuntamente--- es un producto viviente, pero la mente y la moral del individuo son el terreno del que deben brotar esas manifestaciones superiores del desarrollo humano y del destino divino. El suelo del alma evolutiva es humano y material, pero el destino de esta criatura compuesta de mente y de espíritu es espiritual y divino>>.

\par 
%\textsuperscript{(1738.2)}
\textsuperscript{156:5.3} Este mismo día por la tarde, Natanael le preguntó a Jesús: <<Maestro, ¿por qué le pedimos a Dios que no nos induzca a la tentación, cuando sabemos muy bien, por tu revelación del Padre, que él nunca hace tales cosas?>> Jesús contestó a Natanael:

\par 
%\textsuperscript{(1738.3)}
\textsuperscript{156:5.4} <<No es de extrañar que hagas estas preguntas, puesto que estás empezando a conocer al Padre como yo lo conozco, y no como lo veían tan confusamente los antiguos profetas hebreos. Sabes bien que nuestros antepasados tenían la tendencia de ver a Dios en casi todas las cosas que sucedían. Buscaban la mano de Dios en todas los acontecimientos naturales y en cada episodio insólito de la experiencia humana. Asociaban a Dios tanto con el bien como con el mal. Pensaban que Dios había ablandado el corazón de Moisés y endurecido el del faraón. Cuando el hombre sentía un fuerte impulso de hacer algo, bueno o malo, tenía la costumbre de explicar estas emociones poco frecuentes diciendo: `El Señor me ha hablado para decirme: haz esto o haz aquello, ve aquí o ve allí.' En consecuencia, como los hombres caían tan a menudo y con tanta violencia en la tentación, nuestros antepasados cogieron la costumbre de creer que Dios les inducía a ella para probarlos, castigarlos o fortalecerlos. Pero tú, por supuesto, sabes ahora más cosas. Sabes que, con demasiada frecuencia, los hombres son inducidos a la tentación por el ímpetu de su propio egoísmo y los impulsos de su naturaleza animal. Cuando te sientas tentado de esta manera, te recomiendo que, al mismo tiempo que reconoces honrada y sinceramente la tentación exactamente por lo que es, reorientes de manera inteligente las energías espirituales, mentales y corporales que intentan expresarse hacia unos canales superiores y unas metas más idealistas. De esta manera podrás transformar tus tentaciones en los tipos más elevados de servicio humano edificante, y al mismo tiempo evitarás casi por completo los conflictos destructivos y debilitantes entre la naturaleza animal y la naturaleza espiritual>>.

\par 
%\textsuperscript{(1738.4)}
\textsuperscript{156:5.5} <<Pero déjame prevenirte contra la locura de intentar superar la tentación mediante el esfuerzo de reemplazar un deseo por otro deseo supuestamente superior, utilizando la simple fuerza de la voluntad humana. Si quieres triunfar realmente sobre las tentaciones de la naturaleza más baja e inferior, debes alcanzar esa posición de superioridad espiritual en la que habrás desarrollado, de manera real y sincera, un interés efectivo y un amor por esas formas de conducta superiores y más idealistas que tu mente desea sustituir por los hábitos de comportamiento inferiores y menos idealistas que reconoces como tentaciones. De esta manera podrás liberarte gracias a la transformación espiritual, en lugar de sentirte cada vez más sobrecargado por la represión engañosa de los deseos humanos. Lo antiguo y lo inferior serán olvidados en el amor por lo nuevo y lo superior. La belleza siempre triunfa sobre la fealdad en el corazón de todos los que están iluminados por el amor a la verdad. Existe un enorme poder en la energía expulsiva de un afecto espiritual nuevo y sincero. Te lo repito de nuevo, no te dejes vencer por el mal, sino más bien vence al mal con el bien>>.

\par 
%\textsuperscript{(1739.1)}
\textsuperscript{156:5.6} Los apóstoles y los evangelistas continuaron haciendo preguntas hasta muy entrada la noche, y de las numerosas respuestas de Jesús, desearíamos presentar los pensamientos siguientes, que exponemos en un lenguaje moderno:

\par 
%\textsuperscript{(1739.2)}
\textsuperscript{156:5.7} Una ambición enérgica, un juicio inteligente y una sabiduría madura son los factores esenciales para conseguir el éxito material. Las dotes de mando dependen de la aptitud natural, la discreción, el poder de la voluntad y la determinación. El destino espiritual depende de la fe, el amor y la devoción a la verdad ---el hambre y la sed de rectitud--- el deseo entusiasta de encontrar a Dios y parecerse a él.

\par 
%\textsuperscript{(1739.3)}
\textsuperscript{156:5.8} No os desaniméis por el descubrimiento de que sois humanos. La naturaleza humana puede tender hacia el mal, pero no es pecaminosa de manera inherente. No os sintáis abatidos por vuestra incapacidad para olvidar completamente algunas de vuestras experiencias más lamentables. Los errores que no consigáis olvidar en el tiempo, serán olvidados en la eternidad. Aligerad las cargas de vuestra alma mediante la rápida adquisición de una visión a largo plazo de vuestro destino, de una expansión de vuestra carrera en el universo.

\par 
%\textsuperscript{(1739.4)}
\textsuperscript{156:5.9} No cometáis el error de apreciar el valor del alma según las imperfecciones de la mente o los apetitos del cuerpo. No juzguéis el alma ni evaluéis su destino sobre la base de un solo episodio humano desafortunado. Vuestro destino espiritual sólo está condicionado por vuestros anhelos e intenciones espirituales.

\par 
%\textsuperscript{(1739.5)}
\textsuperscript{156:5.10} La religión es la experiencia exclusivamente espiritual del alma inmortal evolutiva del hombre que conoce a Dios, pero el poder moral y la energía espiritual son unas fuerzas poderosas que se pueden utilizar para tratar situaciones sociales difíciles y para resolver problemas económicos complicados. Estos dones morales y espirituales enriquecen más todos los niveles de la vida humana, y los hacen más significativos.

\par 
%\textsuperscript{(1739.6)}
\textsuperscript{156:5.11} Si aprendéis a amar solamente a aquellos que os aman, estáis destinados a vivir una vida limitada y mediocre. Es cierto que el amor humano puede ser recíproco, pero el amor divino es extrovertido en toda su búsqueda de la satisfacción. Cuanto menos amor hay en la naturaleza de una criatura, más grande es su necesidad de amor, y más intenta el amor divino satisfacer esa necesidad. El amor nunca es egoísta, y no puede ser dirigido hacia uno mismo. El amor divino no puede estar encerrado en sí mismo; necesita darse generosamente.

\par 
%\textsuperscript{(1739.7)}
\textsuperscript{156:5.12} Los creyentes en el reino deben poseer una fe implícita, una creencia con toda el alma, en el triunfo seguro de la rectitud. Los constructores del reino no deben dudar de que el evangelio de la salvación eterna es verdadero. Los creyentes deben aprender cada vez más a apartarse de las precipitaciones de la vida ---a huir de los agobios de la existencia material--- mientras que vivifican su alma, inspiran su mente y renuevan su espíritu por medio de la comunión en la adoración.

\par 
%\textsuperscript{(1739.8)}
\textsuperscript{156:5.13} Los individuos que conocen a Dios no se desaniman por las desgracias ni se dejan abatir por las decepciones. Los creyentes están inmunizados contra la depresión que sigue a los cataclismos puramente materiales; los que llevan una vida espiritual no se inquietan por los episodios del mundo material. Los candidatos a la vida eterna practican una técnica vigorizante y constructiva para hacer frente a todas las vicisitudes y agobios de la vida mortal. Un verdadero creyente, cada día que vive, encuentra \textit{más fácil} hacer lo que es justo.

\par 
%\textsuperscript{(1740.1)}
\textsuperscript{156:5.14} La vida espiritual acrecienta poderosamente la verdadera autoestima. Pero la autoestima no es la admiración de sí mismo. La autoestima siempre está coordinada con el amor y el servicio a los semejantes. No es posible estimarse más a sí mismo de lo que se ama al prójimo; lo uno mide la capacidad para hacer lo otro.

\par 
%\textsuperscript{(1740.2)}
\textsuperscript{156:5.15} A medida que pasan los días, todo verdadero creyente se vuelve más hábil en atraer a sus semejantes hacia el amor de la verdad eterna. ¿Sois hoy más ingeniosos que ayer en revelar la bondad a la humanidad? ¿Sabéis recomendar mejor la rectitud este año que el año pasado? ¿Os estáis volviendo cada vez más artistas en vuestra técnica para conducir a las almas hambrientas hacia el reino espiritual?

\par 
%\textsuperscript{(1740.3)}
\textsuperscript{156:5.16} ¿Son vuestros ideales lo suficientemente elevados como para garantizar vuestra salvación eterna, y vuestras ideas son al mismo tiempo tan prácticas como para convertiros en unos ciudadanos útiles que funcionan en la Tierra en asociación con sus compañeros mortales? En el espíritu, vuestra ciudadanía está en los cielos; en la carne, todavía sois ciudadanos de los reinos de la Tierra. Dad a los césares las cosas que son materiales, y a Dios las que son espirituales.

\par 
%\textsuperscript{(1740.4)}
\textsuperscript{156:5.17} La medida de la capacidad espiritual del alma evolutiva es vuestra fe en la verdad y vuestro amor por los hombres; pero la medida de vuestra fuerza de carácter humano es vuestra aptitud para resistir la influencia de los resentimientos y vuestra capacidad para soportar las cavilaciones en presencia de una pena profunda. La derrota es el verdadero espejo donde podéis contemplar honradamente vuestro yo real.

\par 
%\textsuperscript{(1740.5)}
\textsuperscript{156:5.18} A medida que tenéis más años y os volvéis más experimentados en los asuntos del reino, ¿empleáis más tacto en vuestras relaciones con los mortales inoportunos y más tolerancia en la convivencia con vuestros compañeros testarudos? El tacto es el punto de apoyo de la influencia social, y la tolerancia es el distintivo de un alma grande. Si poseéis estos dones raros y encantadores, a medida que pasan los días os volveréis más alertas y expertos en vuestros esfuerzos meritorios por evitar todos los malentendidos sociales inútiles. Estas almas sabias son capaces de evitar un buen número de dificultades que se abaten con seguridad sobre todos los que sufren una falta de adaptación emocional, los que se niegan a crecer, y los que no aceptan envejecer con elegancia.

\par 
%\textsuperscript{(1740.6)}
\textsuperscript{156:5.19} Evitad la falta de honradez y la injusticia en todos vuestros esfuerzos por predicar la verdad y proclamar el evangelio. No busquéis un reconocimiento no ganado y no anheléis una simpatía inmerecida. Recibid libremente el amor que os llegue tanto de fuentes divinas como humanas, independientemente de que lo merezcáis o no, y amad a cambio generosamente. Pero en todas las demás cosas relacionadas con el honor y la adulación, buscad sólo lo que os pertenezca honradamente.

\par 
%\textsuperscript{(1740.7)}
\textsuperscript{156:5.20} El mortal consciente de Dios está seguro de salvarse; no le teme a la vida; es honrado y consecuente. Sabe cómo soportar valientemente los sufrimientos inevitables; no se queja cuando se enfrenta con las penalidades ineludibles.

\par 
%\textsuperscript{(1740.8)}
\textsuperscript{156:5.21} El verdadero creyente no se cansa de hacer el bien simplemente porque se sienta frustrado. Las dificultades estimulan el ardor de los amantes de la verdad, mientras que los obstáculos sólo sirven para desafiar los esfuerzos de los intrépidos constructores del reino.

\par 
%\textsuperscript{(1740.9)}
\textsuperscript{156:5.22} Y Jesús les enseñó otras muchas cosas antes de que se prepararan para marcharse de Tiro.

\par 
%\textsuperscript{(1740.10)}
\textsuperscript{156:5.23} El día antes de salir de Tiro para regresar a la región del Mar de Galilea, Jesús reunió a sus asociados y ordenó a los doce evangelistas que volvieran por una ruta diferente a la que él y los doce apóstoles iban a utilizar. Después de separarse aquí de Jesús, los evangelistas nunca más volvieron a estar tan íntimamente asociados con él.

\section*{6. El regreso de Fenicia}
\par 
%\textsuperscript{(1741.1)}
\textsuperscript{156:6.1} El domingo 24 de julio hacia el mediodía, Jesús y los doce salieron de la casa de José, al sur de Tiro. Bajaron por la costa hasta Tolemaida, donde se detuvieron un día, y expresaron palabras de aliento al grupo de creyentes que residía allí. Pedro predicó para ellos el 25 de julio por la noche.

\par 
%\textsuperscript{(1741.2)}
\textsuperscript{156:6.2} El martes salieron de Tolemaida y se dirigieron tierra adentro hacia el este, por el camino de Tiberiades, hasta cerca de Jotapata. El miércoles se detuvieron en Jotapata y dieron a los creyentes una enseñanza adicional sobre las cosas del reino. El jueves salieron de Jotapata y se encaminaron hacia el norte por la ruta de Nazaret al Monte Líbano hasta llegar al pueblo de Zabulón, pasando por Ramá. El viernes mantuvieron reuniones en Ramá y se quedaron hasta el sábado. El domingo día 31 llegaron a Zabulón, celebraron una reunión aquella noche y partieron a la mañana siguiente.

\par 
%\textsuperscript{(1741.3)}
\textsuperscript{156:6.3} Cuando salieron de Zabulón, viajaron hasta el cruce con la carretera de Magdala a Sidón, cerca de Giscala, y desde allí se dirigieron a Genesaret por la costa occidental del lago de Galilea, al sur de Cafarnaúm, donde habían acordado reunirse con David Zebedeo, y donde tenían la intención de deliberar sobre el siguiente paso a dar en la tarea de predicar el evangelio del reino.

\par 
%\textsuperscript{(1741.4)}
\textsuperscript{156:6.4} Durante una breve conversación con David, se enteraron de que muchos dirigentes se encontraban reunidos en ese momento en la orilla opuesta del lago, cerca de Jeresa, y en consecuencia, aquella misma noche atravesaron el lago en una barca. Pasaron un día descansando tranquilamente en las colinas, y al día siguiente continuaron hasta el parque cercano donde el Maestro había alimentado anteriormente a los cinco mil. Descansaron allí durante tres días y celebraron diariamente unas conferencias a las que asistieron unos cincuenta hombres y mujeres, el resto del antiguo grupo numeroso de creyentes que residían en Cafarnaúm y sus alrededores.

\par 
%\textsuperscript{(1741.5)}
\textsuperscript{156:6.5} Mientras Jesús estaba ausente de Cafarnaúm y Galilea, durante el período de su estancia en Fenicia, sus enemigos consideraron que todo el movimiento había sido destruido; concluyeron que la prisa de Jesús por alejarse de allí indicaba que estaba tan asustado que probablemente no volvería nunca más a molestarlos. Toda la oposición activa a sus enseñanzas casi se había calmado. Los creyentes empezaban de nuevo a celebrar reuniones públicas, y los supervivientes probados y leales de la gran criba por la que acababan de pasar los creyentes en el evangelio se iban consolidando de manera gradual pero eficaz.

\par 
%\textsuperscript{(1741.6)}
\textsuperscript{156:6.6} Felipe, el hermano de Herodes, se había convertido en un creyente a medias en Jesús y envió un mensaje indicando que el Maestro tenía libertad para vivir y trabajar en sus dominios.

\par 
%\textsuperscript{(1741.7)}
\textsuperscript{156:6.7} La orden de cerrar las sinagogas de todo el mundo judío a las enseñanzas de Jesús y de todos sus seguidores se había vuelto en contra de los escribas y fariseos. En cuanto Jesús se retiró como objeto de controversia, se produjo una reacción en toda la población judía; hubo un resentimiento general contra los fariseos y los dirigentes del sanedrín de Jerusalén. Muchos jefes de las sinagogas empezaron a abrir subrepticiamente sus sinagogas a Abner y sus asociados, declarando que estos instructores eran seguidores de Juan, y no discípulos de Jesús.

\par 
%\textsuperscript{(1741.8)}
\textsuperscript{156:6.8} Incluso Herodes Antipas experimentó un cambio de sentimiento. Al enterarse de que Jesús estaba residiendo al otro lado del lago, en el territorio de su hermano Felipe, le envió el recado de que, aunque había firmado unas órdenes para que lo arrestaran en Galilea, no por ello había autorizado su captura en Perea, indicando de esta manera que Jesús no sería molestado si permanecía fuera de Galilea; y esta misma decisión se la comunicó a los judíos de Jerusalén.

\par 
%\textsuperscript{(1742.1)}
\textsuperscript{156:6.9} Ésta era la situación hacia primeros de agosto del año 29, cuando el Maestro regresó de su misión en Fenicia y empezó a reorganizar sus fuerzas dispersas, puestas a prueba y reducidas, con vistas a este último año memorable de su misión en la Tierra.

\par 
%\textsuperscript{(1742.2)}
\textsuperscript{156:6.10} Los resultados de la batalla están claramente delineados mientras el Maestro y sus compañeros se preparan para empezar la proclamación de una nueva religión, la religión del espíritu del Dios vivo que reside en la mente de los hombres.


\chapter{Documento 157. En Cesarea de Filipo}
\par 
%\textsuperscript{(1743.1)}
\textsuperscript{157:0.1} ANTES de llevarse a los doce para pasar unos días en las cercanías de Cesarea de Filipo, Jesús había planeado, por medio de los mensajeros de David, desplazarse hasta Cafarnaúm para reunirse con su familia el domingo 7 de agosto. Según habían arreglado de antemano, esta visita tendría lugar en el taller de barcas de Zebedeo. David Zebedeo había dispuesto con Judá, el hermano de Jesús, que toda la familia de Nazaret estaría presente ---María y todos los hermanos y hermanas de Jesús--- y Jesús se desplazó con Andrés y Pedro para cumplir con este compromiso. María y sus hijos tenían indudablemente la intención de acudir a esta cita, pero sucedió también que un grupo de fariseos, sabiendo que Jesús se encontraba al otro lado del lago en los dominios de Felipe, decidió visitar a María para averiguar lo que pudieran sobre su paradero. La llegada de estos emisarios de Jerusalén inquietó mucho a María, y cuando observaron la tensión y el nerviosismo de toda la familia, concluyeron que debían estar esperando una visita de Jesús. En consecuencia, se instalaron en la casa de María, y después de pedir refuerzos, esperaron pacientemente la llegada de Jesús. Por supuesto, esto impidió eficazmente que algún miembro de la familia tratara de acudir a la cita con Jesús. Durante todo el día, tanto Judá como Rut intentaron varias veces eludir la vigilancia de los fariseos para poder enviar un mensaje a Jesús, pero fue en vano.

\par 
%\textsuperscript{(1743.2)}
\textsuperscript{157:0.2} Al principio de la tarde, los mensajeros de David informaron a Jesús que los fariseos estaban acampados en el umbral de la casa de su madre; por lo tanto, Jesús no hizo ningún intento por visitar a su familia. Una vez más, y sin que ninguna de las dos partes tuviera la culpa, Jesús y su familia terrestre no lograron ponerse en contacto.

\section*{1. El recaudador de impuestos del templo}
\par 
%\textsuperscript{(1743.3)}
\textsuperscript{157:1.1} Mientras Jesús se demoraba con Andrés y Pedro al borde del lago, cerca del taller de barcas, un recaudador de impuestos del templo se encontró con ellos, reconoció a Jesús, y llamó a Pedro aparte para decirle: <<¿Tu Maestro no paga el impuesto del templo?>> Pedro se sintió tentado a mostrar su indignación ante la sugerencia de que Jesús debía contribuir al mantenimiento de las actividades religiosas de sus enemigos declarados; pero al observar una expresión particular en el rostro del recaudador de impuestos, supuso con razón que éste tenía la intención de atraparlos en el acto de negarse a pagar el medio siclo habitual para el sostén de los servicios del templo en Jerusalén. En consecuencia, Pedro contestó: <<Por supuesto, el Maestro paga el impuesto del templo. Espera en la puerta, que vuelvo enseguida con la contribución>>.

\par 
%\textsuperscript{(1743.4)}
\textsuperscript{157:1.2} Pero Pedro había hablado precipitadamente, porque Judas, que transportaba los fondos, estaba al otro lado del lago. Ni Pedro, ni su hermano ni Jesús habían traído dinero. Sabiendo que los fariseos los estaban buscando, no podían ir a Betsaida para conseguir dinero. Cuando Pedro le contó a Jesús lo del recaudador y que le había prometido el dinero, Jesús dijo: <<Si lo has prometido, debes pagar. Pero ¿con qué vas a cumplir tu promesa? ¿Volverás a ser pescador para poder cumplir con tu palabra? Sin embargo, Pedro, en estas circunstancias es conveniente que paguemos el impuesto. No le demos a estos hombres ningún motivo para que se ofendan con nuestra actitud. Esperaremos aquí mientras vas con la barca a coger los peces, y cuando los hayas vendido en ese mercado de ahí, págale al recaudador por nosotros tres>>.

\par 
%\textsuperscript{(1744.1)}
\textsuperscript{157:1.3} El mensajero secreto de David, que se hallaba cerca, alcanzó a oír toda esta conversación, y entonces le hizo señas a un asociado que estaba pescando cerca de la orilla para que viniera enseguida. Cuando Pedro estuvo preparado para salir a pescar en la barca, este mensajero y su amigo pescador le ofrecieron varias cestas grandes de peces y le ayudaron a llevarlas hasta el comerciante de pescado cercano, el cual compró la pesca y pagó lo suficiente como para, con lo que añadió el mensajero de David, poder saldar el impuesto del templo de los tres. El recaudador aceptó el tributo sin cobrarles la multa por el pago atrasado, ya que habían estado ausentes de Galilea durante algún tiempo.

\par 
%\textsuperscript{(1744.2)}
\textsuperscript{157:1.4} No es de extrañar que tengáis un relato donde se describe a Pedro capturando un pez con un siclo en la boca. En aquella época circulaban muchas narraciones sobre el descubrimiento de tesoros en la boca de los peces; estas historias casi milagrosas eran frecuentes. Por eso, cuando Pedro se iba para dirigirse hacia la barca, Jesús comentó medio en broma: <<Es raro que los hijos del rey tengan que pagar tributo; generalmente son los extranjeros los que pagan los impuestos para mantener la corte; pero es conveniente que no proporcionemos ningún escollo a las autoridades. ¡Vete pues! quizás atrapes el pez con el siclo en la boca>>. Como Jesús había dicho esto y Pedro había regresado tan rápidamente con el tributo para el templo, no es de sorprender que el episodio se exagerara más tarde hasta convertirse en el milagro que cuenta el escritor del evangelio según Mateo.

\par 
%\textsuperscript{(1744.3)}
\textsuperscript{157:1.5} Jesús esperó al lado de la playa, con Andrés y Pedro, hasta cerca de la puesta del Sol. Unos mensajeros le trajeron la noticia de que la casa de María continuaba estando vigilada; por consiguiente, una vez que oscureció, los tres hombres que aguardaban subieron a su barca y remaron lentamente hacia la costa oriental del Mar de Galilea.

\section*{2. En Betsaida-Julias}
\par 
%\textsuperscript{(1744.4)}
\textsuperscript{157:2.1} El lunes 8 de agosto, mientras Jesús y los doce apóstoles estaban acampados en el parque de Magadán, cerca de Betsaida-Julias, más de cien creyentes, los evangelistas, el cuerpo de mujeres y otras personas interesadas en el establecimiento del reino, vinieron desde Cafarnaúm para celebrar una conferencia. Al enterarse de que Jesús estaba allí, muchos fariseos vinieron también. Para entonces, algunos saduceos se habían unido a los fariseos en sus esfuerzos por coger a Jesús en una trampa. Antes de empezar la conferencia privada con los creyentes, Jesús celebró una reunión pública a la que asistieron los fariseos, los cuales importunaron al Maestro y trataron de perturbar la asamblea de otras maneras. El jefe de los alborotadores dijo: <<Maestro, nos gustaría que nos dieras un signo de la autoridad que tienes para enseñar, y entonces, cuando se produzca ese signo, todos los hombres sabrán que has sido enviado por Dios>>. Y Jesús les respondió: <<Cuando llega el atardecer, decís que hará buen tiempo porque el cielo está rojo. Por la mañana decís que hará mal tiempo porque el cielo está rojo y encapotado. Cuando veis que una nube se levanta por el oeste, decís que va a llover; cuando el viento sopla del sur, decís que va a hacer un calor abrasador. ¿Cómo puede ser que sepáis discernir tan bien el aspecto del cielo, y seáis totalmente incapaces de discernir los signos de los tiempos? A los que quieren conocer la verdad, ya se les ha dado un signo; pero no se dará ningún signo a una generación malintencionada e hipócrita>>.

\par 
%\textsuperscript{(1745.1)}
\textsuperscript{157:2.2} Después de haber hablado así, Jesús se retiró y se preparó para la conferencia nocturna con sus seguidores. En esta conferencia se decidió emprender una misión en común por todas las ciudades y pueblos de la Decápolis, en cuanto Jesús y los doce regresaran de la visita que tenían la intención de hacer a Cesarea de Filipo. El Maestro participó en la planificación de la misión en la Decápolis, y al disolver la reunión, dijo: <<Os lo digo, tened cuidado con la influencia de los fariseos y los saduceos. No os dejéis engañar por sus demostraciones de gran erudición y su profunda lealtad a las ceremonias de la religión. Preocupaos solamente por el espíritu de la verdad viviente y por el poder de la religión verdadera. El miedo a una religión muerta no es lo que os salvará, sino más bien vuestra fe en una experiencia viviente con las realidades espirituales del reino. No os dejéis cegar por los prejuicios ni paralizar por el miedo. No permitáis tampoco que el respeto por las tradiciones deforme tanto vuestra comprensión que vuestros ojos no vean y vuestros oídos no oigan. La finalidad de la religión verdadera no es simplemente aportar la paz, sino más bien asegurar el progreso. Y no puede haber paz en el corazón, ni progreso en la mente, si no os enamoráis de todo corazón de la verdad, de los ideales de las realidades eternas. Las consecuencias de la vida y de la muerte están delante de vosotros ---los placeres pecaminosos del tiempo contra las justas realidades de la eternidad. Incluso ahora, deberíais empezar a liberaros de la esclavitud del miedo y de la duda, a medida que comenzáis a vivir la nueva vida de la fe y la esperanza. Cuando los sentimientos del servicio por vuestros compañeros humanos aparezcan en vuestra alma, no los ahoguéis; cuando las emociones del amor por vuestro prójimo broten en vuestro corazón, manifestad esos impulsos afectivos atendiendo inteligentemente las necesidades reales de vuestros semejantes>>.

\section*{3. La confesión de Pedro}
\par 
%\textsuperscript{(1745.2)}
\textsuperscript{157:3.1} El martes por la mañana temprano, Jesús y los doce apóstoles salieron del parque de Magadán hacia Cesarea de Filipo, la capital de la soberanía del tetrarca Felipe. Esta ciudad estaba situada en una región de una belleza admirable, abrigada en un valle encantador entre colinas pintorescas, donde el Jordán surgía de una gruta subterránea. Hacia el norte se podían contemplar las cumbres del Monte Hermón, mientras que desde las colinas del sur se tenía una vista espléndida sobre el alto Jordán y el Mar de Galilea.

\par 
%\textsuperscript{(1745.3)}
\textsuperscript{157:3.2} Jesús había ido al Monte Hermón durante sus primeras experiencias en los asuntos del reino, y ahora que emprendía la fase final de su obra, deseaba regresar a esta montaña de prueba y de triunfo, donde esperaba que los apóstoles pudieran conseguir una nueva visión de sus responsabilidades, y adquirir nuevas fuerzas para los tiempos difíciles que se avecinaban. Mientras viajaban por el camino, cuando iban a pasar al sur de las Aguas de Merom, los apóstoles empezaron a charlar entre ellos sobre sus recientes experiencias en Fenicia y en otros lugares, mencionando cómo había sido recibido su mensaje y la manera en que las diferentes poblaciones consideraban al Maestro.

\par 
%\textsuperscript{(1745.4)}
\textsuperscript{157:3.3} Cuando se detuvieron para almorzar, Jesús planteó repentinamente a los doce la primera pregunta que les hubiera hecho nunca sobre sí mismo. Les hizo esta pregunta sorprendente: <<¿Quién dicen los hombres que soy?>>

\par 
%\textsuperscript{(1746.1)}
\textsuperscript{157:3.4} Jesús había pasado largos meses instruyendo a estos apóstoles sobre la naturaleza y el carácter del reino de los cielos, y sabía muy bien que había llegado la hora de empezar a enseñarles más cosas sobre su propia naturaleza y su relación personal con el reino. Ahora, mientras estaban sentados debajo de unas moreras, el Maestro se preparó para celebrar una de las sesiones más importantes de su larga asociación con los apóstoles escogidos.

\par 
%\textsuperscript{(1746.2)}
\textsuperscript{157:3.5} Más de la mitad de los apóstoles participaron en la respuesta a la pregunta de Jesús. Le dijeron que todos los que lo conocían lo consideraban como un profeta o un hombre extraordinario; que incluso sus enemigos le temían mucho, y que explicaban sus poderes mediante la acusación de que estaba aliado con el príncipe de los demonios. Le dijeron que algunas personas de Judea y Samaria, que no lo habían conocido personalmente, creían que era Juan el Bautista resucitado de entre los muertos. Pedro explicó que, en diversas ocasiones, distintas personas lo habían comparado con Moisés, Elías, Isaías y Jeremías. Después de haber escuchado estos comentarios, Jesús se puso de pie, miró a los doce sentados en semicírculo alrededor de él, y con un énfasis sorprendente los señaló con un movimiento expresivo de la mano, y les preguntó: <<Pero ¿quién decís vosotros que soy?>> Hubo un momento de tenso silencio, en el que los doce no despegaron sus ojos del Maestro. Luego, Simón Pedro se levantó de un salto, y exclamó: <<Tú eres el Libertador, el Hijo del Dios vivo>>. Y los once apóstoles que estaban sentados se levantaron al unísono, indicando así que Pedro había hablado por todos ellos.

\par 
%\textsuperscript{(1746.3)}
\textsuperscript{157:3.6} Jesús les señaló que se sentaran de nuevo, y mientras permanecía de pie delante de ellos, dijo: <<Esto os ha sido revelado por mi Padre. Ha llegado la hora de que conozcáis la verdad sobre mí. Pero, de momento, os encargo que no le contéis esto a nadie. Vámonos de aquí>>.

\par 
%\textsuperscript{(1746.4)}
\textsuperscript{157:3.7} Así pues, reanudaron su viaje hacia Cesarea de Filipo, donde llegaron tarde aquella noche, y se alojaron en la casa de Celsus, que los estaba esperando. Los apóstoles durmieron poco aquella noche; parecían sentir que un gran acontecimiento se había producido en sus vidas y en la obra del reino.

\section*{4. La conversación sobre el reino}
\par 
%\textsuperscript{(1746.5)}
\textsuperscript{157:4.1} Desde los sucesos del bautismo de Jesús por Juan y la transformación del agua en vino en Caná, los apóstoles lo habían aceptado virtualmente, en diversas ocasiones, como el Mesías. Durante cortos períodos, algunos de ellos habían creído realmente que era el Libertador esperado. Pero apenas nacían estas esperanzas en su corazón, el Maestro las hacía añicos con alguna palabra aplastante o con algún acto que los desilusionaba. Durante mucho tiempo habían estado agitados por el conflicto entre los conceptos del Mesías esperado, que conservaban en su mente, y la experiencia de su asociación extraordinaria con este hombre extraordinario, que conservaban en su corazón.

\par 
%\textsuperscript{(1746.6)}
\textsuperscript{157:4.2} Al final de la mañana de este miércoles, los apóstoles se congregaron en el jardín de Celsus para almorzar. Durante la mayor parte de la noche y desde que se habían levantado aquella mañana, Simón Pedro y Simón Celotes se habían esforzado ardientemente por convencer a todos sus hermanos de que aceptaran al Maestro de todo corazón, no solamente como Mesías, sino también como Hijo divino del Dios vivo. Los dos Simones estaban casi de acuerdo en su apreciación de Jesús, y trabajaron diligentemente para persuadir a sus hermanos de que aceptaran plenamente su punto de vista. Aunque Andrés continuaba siendo el director general del cuerpo apostólico, su hermano Simón Pedro se estaba convirtiendo cada vez más, por consentimiento general, en el portavoz de los doce.

\par 
%\textsuperscript{(1747.1)}
\textsuperscript{157:4.3} A eso del mediodía, todos estaban sentados en el jardín cuando apareció el Maestro. Tenían una expresión digna y solemne, y todos se levantaron al acercarse a ellos. Jesús suavizó la tensión con esa sonrisa amistosa y fraternal tan característica de él cada vez que sus seguidores se tomaban demasiado en serio a sí mismos o algún suceso relacionado con ellos. Con un gesto imperativo les indicó que se sentaran. Los doce nunca más recibieron a su Maestro poniéndose de pie al aproximarse a ellos. Se dieron cuenta de que no aprobaba esta muestra exterior de respeto.

\par 
%\textsuperscript{(1747.2)}
\textsuperscript{157:4.4} Después de haber compartido el almuerzo y de haberse puesto a discutir los planes de su próxima gira por la Decápolis, Jesús los miró repentinamente a la cara y dijo: <<Ahora que ha pasado un día entero desde que aprobasteis la declaración de Simón Pedro sobre la identidad del Hijo del Hombre, deseo preguntaros si continuáis manteniendo vuestra decisión>>. Al escuchar esto, los doce se pusieron de pie, y Simón Pedro avanzó unos pasos hacia Jesús, diciendo: <<Sí, Maestro, la mantenemos. Creemos que eres el Hijo del Dios vivo>>. Y Pedro volvió a sentarse con sus hermanos.

\par 
%\textsuperscript{(1747.3)}
\textsuperscript{157:4.5} Jesús, que permanecía de pie, dijo entonces a los doce: <<Sois mis embajadores escogidos, pero sé que, en estas circunstancias, no podríais tener esta creencia como resultado de un simple conocimiento humano. Ésta es una revelación del espíritu de mi Padre a lo más profundo de vuestra alma. Así pues, si hacéis esta confesión por la perspicacia del espíritu de mi Padre que reside en vosotros, me veo inducido a declarar que sobre este cimiento construiré la fraternidad del reino de los cielos. Sobre esta roca de realidad espiritual, construiré el templo viviente de la hermandad espiritual en las realidades eternas del reino de mi Padre. Todas las fuerzas del mal y los ejércitos del pecado no prevalecerán contra esta fraternidad humana del espíritu divino. Aunque el espíritu de mi Padre será siempre el guía y el mentor divino de todos los que se vinculen a esta hermandad espiritual, a vosotros y a vuestros sucesores entrego ahora las llaves del reino exterior ---la autoridad sobre las cosas temporales--- los aspectos sociales y económicos de esta asociación de hombres y mujeres, como miembros del reino>>. Y les encargó de nuevo que, por el momento, no le dijeran a nadie que era el Hijo de Dios.

\par 
%\textsuperscript{(1747.4)}
\textsuperscript{157:4.6} Jesús estaba empezando a tener fe en la lealtad y la integridad de sus apóstoles. El Maestro pensaba que una fe capaz de resistir lo que sus representantes escogidos habían pasado recientemente, podría soportar sin duda las duras pruebas que se avecinaban, y emerger del naufragio aparente de todas sus esperanzas hacia la nueva luz de una nueva dispensación, y así ser capaces de salir para iluminar a un mundo sumido en las tinieblas. Este día, el Maestro empezó a creer en la fe de sus apóstoles, salvo en uno.

\par 
%\textsuperscript{(1747.5)}
\textsuperscript{157:4.7} Desde aquel día, este mismo Jesús ha estado construyendo ese templo viviente sobre ese mismo cimiento eterno de su filiación divina; y aquellos que de ese modo se vuelven conscientes de ser hijos de Dios, son las piedras humanas que componen este templo viviente de filiación que se levanta hasta la gloria y el honor de la sabiduría y el amor del Padre eterno de los espíritus.

\par 
%\textsuperscript{(1747.6)}
\textsuperscript{157:4.8} Después de haber hablado así, Jesús ordenó a los doce que se retiraran a solas en las colinas, hasta la hora de la cena, para buscar la sabiduría, la fuerza y la guía espiritual. E hicieron lo que el Maestro les había recomendado.

\section*{5. El nuevo concepto}
\par 
%\textsuperscript{(1748.1)}
\textsuperscript{157:5.1} La característica nueva y esencial de la confesión de Pedro fue el reconocimiento bien claro de que Jesús era el Hijo de Dios, de su divinidad incuestionable. Desde su bautismo y las bodas de Caná, estos apóstoles lo habían considerado de diversas maneras como el Mesías, pero que éste tuviera que ser \textit{divino} no formaba parte del concepto judío del libertador nacional. Los judíos no habían enseñado que el Mesías tuviera que proceder de la divinidad; debía ser <<el ungido>>, pero difícilmente habían contemplado que tuviera que ser <<el Hijo de Dios>>. En la segunda confesión se puso más énfasis en la \textit{naturaleza} \textit{combinada} de Jesús, en el hecho excelso de que era el Hijo del Hombre \textit{y} el Hijo de Dios. Y Jesús declaró que construiría el reino de los cielos sobre esta gran verdad de la unión de la naturaleza humana con la naturaleza divina.

\par 
%\textsuperscript{(1748.2)}
\textsuperscript{157:5.2} Jesús había intentado vivir su vida en la Tierra y terminar su misión donadora como Hijo del Hombre. Sus seguidores estaban dispuestos a considerarlo como el Mesías esperado. Sabiendo que nunca podría colmar sus expectativas mesiánicas, se esforzó por modificar el concepto que tenían del Mesías de tal manera que le permitiera a él satisfacer parcialmente sus esperanzas. Pero ahora comprendió que este plan difícilmente podía llevarse a cabo con éxito. Por consiguiente, escogió audazmente revelar su tercer plan ---anunciar abiertamente su divinidad, reconocer la veracidad de la confesión de Pedro, y proclamar directamente a los doce que él era un Hijo de Dios.

\par 
%\textsuperscript{(1748.3)}
\textsuperscript{157:5.3} Durante tres años, Jesús había proclamado que era el <<Hijo del Hombre>>, mientras que durante estos mismos tres años, los apóstoles habían insistido cada vez más en que era el Mesías judío esperado. Ahora reveló que era el Hijo de Dios, y decidió construir el reino de los cielos sobre el concepto de la \textit{naturaleza combinada} del Hijo del Hombre y del Hijo de Dios. Había decidido abstenerse de hacer nuevos esfuerzos por convencerlos de que no era el Mesías. Ahora se propuso revelarles audazmente lo que él \textit{es}, y no hacer caso de la determinación de ellos de continuar considerándolo como el Mesías.

\section*{6. La tarde siguiente}
\par 
%\textsuperscript{(1748.4)}
\textsuperscript{157:6.1} Jesús y los apóstoles permanecieron un día más en la casa de Celsus, esperando que los mensajeros de David Zebedeo llegaran con el dinero. Después de haberse derrumbado la popularidad que Jesús tenía entre las masas, los ingresos habían disminuido considerablemente. Cuando llegaron a Cesarea de Filipo, la tesorería estaba vacía. Mateo era reacio a separarse de Jesús y sus hermanos en aquel momento, y no tenía fondos propios disponibles para entregarselos a Judas, como tantas veces había hecho anteriormente. Sin embargo, David Zebedeo había previsto esta probable disminución de los ingresos; en consecuencia, había indicado a sus mensajeros que mientras atravesaban Judea, Samaria y Galilea, debían actuar como recaudadores de dinero para enviarlo a los apóstoles desterrados y a su Maestro. Así es como este día por la noche, los mensajeros llegaron de Betsaida trayendo fondos suficientes como para sostener a los apóstoles hasta que volvieran para emprender la gira por la Decápolis. Mateo esperaba que, para entonces, ya tendría el dinero de la venta de su última propiedad de Cafarnaúm, y había dispuesto que estos fondos fueran entregados a Judas de manera anónima.

\par 
%\textsuperscript{(1749.1)}
\textsuperscript{157:6.2} Ni Pedro ni los demás apóstoles tenían un concepto muy adecuado de la divinidad de Jesús. Apenas se daban cuenta de que éste era el principio de una nueva época en la carrera terrestre de su Maestro, la época en que el instructor-sanador se convertiría en el Mesías según el nuevo concepto ---el Hijo de Dios. A partir de este momento, un nuevo tono apareció en el mensaje del Maestro. En lo sucesivo, su único ideal en la vida fue la revelación del Padre, y la idea única de su enseñanza fue la de presentar a su universo la personificación de esa sabiduría suprema que solamente se puede comprender viviéndola. Vino para que todos pudiéramos tener la vida, y tenerla de manera más abundante.

\par 
%\textsuperscript{(1749.2)}
\textsuperscript{157:6.3} Jesús empezaba ahora la cuarta y última etapa de su vida humana en la carne. La primera etapa fue la de su infancia, los años en que sólo tenía una conciencia nebulosa de su origen, naturaleza y destino como ser humano. La segunda etapa fue la de la conciencia creciente de los años de su juventud y su edad adulta progresiva, durante los cuales comprendió más claramente su naturaleza divina y su misión humana. Esta segunda etapa finalizó con las experiencias y revelaciones asociadas con su bautismo. La tercera etapa de la experiencia terrestre del Maestro se extendió desde su bautismo, a través de los años de su ministerio como educador y sanador, hasta el momento importante de la confesión de Pedro en Cesarea de Filipo. Este tercer período de su vida terrestre abarcó la época en que sus apóstoles y sus discípulos inmediatos lo conocieron como el Hijo del Hombre y lo consideraron como el Mesías. El cuarto y último período de su carrera terrestre comenzó aquí, en Cesarea de Filipo, y continuó hasta la crucifixión. Esta etapa de su ministerio estuvo caracterizada por el reconocimiento de su divinidad, y abarcó las obras de su último año en la carne. Durante este cuarto período, aunque la mayoría de sus discípulos seguía considerándolo como el Mesías, los apóstoles lo conocieron como el Hijo de Dios. La confesión de Pedro marcó el principio del nuevo período de una comprensión más completa de la verdad de su ministerio supremo como Hijo donador en Urantia y para todo un universo, y el reconocimiento de este hecho, al menos vagamente, por parte de sus embajadores escogidos.

\par 
%\textsuperscript{(1749.3)}
\textsuperscript{157:6.4} Jesús dio así ejemplo en su vida de lo que enseñaba en su religión: el crecimiento de la naturaleza espiritual mediante la técnica del progreso viviente. No hizo hincapié, como lo hicieron sus seguidores posteriores, en la lucha incesante entre el alma y el cuerpo. Enseñó más bien que el espíritu vencía fácilmente a los dos y reconciliaba de manera eficaz y provechosa un gran número de estas luchas intelectuales e instintivas.

\par 
%\textsuperscript{(1749.4)}
\textsuperscript{157:6.5} A partir de este momento, todas las enseñanzas de Jesús adquieren un nuevo significado. Antes de Cesarea de Filipo, se presentó como el instructor principal del evangelio del reino. Después de Cesarea de Filipo apareció no solamente como instructor, sino como representante divino del Padre eterno, que es el centro y la circunferencia de este reino espiritual; y era necesario que hiciera todo esto como un ser humano, como el Hijo del Hombre.

\par 
%\textsuperscript{(1749.5)}
\textsuperscript{157:6.6} Jesús se había esforzado sinceramente por conducir a sus seguidores hasta el reino espiritual, primero como instructor y luego como instructor-sanador, pero no hicieron caso. Sabía muy bien que su misión terrestre no podría colmar de ninguna manera las esperanzas mesiánicas del pueblo judío; los antiguos profetas habían descrito a un Mesías que él nunca podría ser. Intentó establecer el reino del Padre como Hijo del Hombre, pero sus discípulos no quisieron seguirlo en esta aventura. Al ver esto, Jesús escogió entonces ir al encuentro de sus creyentes hasta cierto punto, y al hacerlo, se preparó para asumir abiertamente el papel de Hijo donador de Dios.

\par 
%\textsuperscript{(1750.1)}
\textsuperscript{157:6.7} En consecuencia, los apóstoles aprendieron muchas cosas nuevas escuchando a Jesús este día en el jardín. Algunas de estas declaraciones les resultaron extrañas incluso a ellos. Entre otras afirmaciones sorprendentes, escucharon algunas como las siguientes:

\par 
%\textsuperscript{(1750.2)}
\textsuperscript{157:6.8} <<Desde ahora en adelante, si un hombre quiere asociarse con nosotros, que asuma las obligaciones de la filiación y que me siga. Cuando ya no esté con vosotros, no creáis que el mundo os va a tratar mejor de lo que trató a vuestro Maestro. Si me amáis, preparaos para poner a prueba ese afecto mediante vuestra buena disposición a hacer el sacrificio supremo>>.

\par 
%\textsuperscript{(1750.3)}
\textsuperscript{157:6.9} <<Retened bien mis palabras: No he venido para llamar a los justos, sino a los pecadores. El Hijo del Hombre no ha venido para ser servido, sino para servir y para donar su vida como un regalo para todos. Os aseguro que he venido para buscar y salvar a los que están perdidos>>.

\par 
%\textsuperscript{(1750.4)}
\textsuperscript{157:6.10} <<Ningún hombre de este mundo ve ahora al Padre, salvo el Hijo que ha venido del Padre. Pero si el Hijo es elevado, atraerá a todos los hombres hacia él, y cualquiera que crea en esta verdad de la naturaleza combinada del Hijo, será dotado de una vida más larga que la que dura una era>>.

\par 
%\textsuperscript{(1750.5)}
\textsuperscript{157:6.11} <<Todavía no podemos proclamar abiertamente que el Hijo del Hombre es el Hijo de Dios, pero esto ya os ha sido revelado; por eso os hablo audazmente de estos misterios. Aunque estoy delante de vosotros con esta presencia física, he venido de Dios Padre. Antes de que Abraham fuera, yo soy. He venido desde el Padre a este mundo tal como me habéis conocido, y os declaro que pronto tendré que dejar este mundo y regresar al trabajo de mi Padre>>.

\par 
%\textsuperscript{(1750.6)}
\textsuperscript{157:6.12} <<Y ahora, ¿puede comprender vuestra fe la verdad de estas declaraciones, ante mi advertencia de que el Hijo del Hombre no satisfará las esperanzas de vuestros padres, tal como ellos concebían al Mesías? Mi reino no es de este mundo. ¿Podéis creer la verdad sobre mí ante el hecho de que, aunque los zorros tienen guaridas y los pájaros del cielo tienen nidos, yo no tengo dónde reposar mi cabeza?>>

\par 
%\textsuperscript{(1750.7)}
\textsuperscript{157:6.13} <<Sin embargo, os hago saber que el Padre y yo somos uno. El que me ha visto a mí, ha visto al Padre. Mi Padre trabaja conmigo en todas estas cosas, y nunca me dejará solo en mi misión, como yo nunca os abandonaré cuando dentro de poco salgáis a proclamar este evangelio por todo el mundo>>.

\par 
%\textsuperscript{(1750.8)}
\textsuperscript{157:6.14} <<Ahora, os he traído aparte y a solas conmigo durante un corto período, para que podáis comprender la gloria y captar la grandeza de la vida a la que os he llamado: la aventura de establecer, por la fe, el reino de mi Padre en el corazón de los hombres, la construcción de mi hermandad de asociación viviente con las almas de todos los que creen en este evangelio>>.

\par 
%\textsuperscript{(1750.9)}
\textsuperscript{157:6.15} Los apóstoles escucharon en silencio estas declaraciones audaces y sorprendentes; estaban atónitos. Luego se dispersaron en pequeños grupos para discutir y examinar las palabras del Maestro. Habían confesado que Jesús era el Hijo de Dios, pero no podían captar el significado completo de lo que habían sido inducidos a hacer.

\section*{7. Las entrevistas de Andrés}
\par 
%\textsuperscript{(1750.10)}
\textsuperscript{157:7.1} Aquella noche, Andrés se encargó de tener una entrevista personal y escrutadora con cada uno de sus hermanos; tuvo unas charlas provechosas y alentadoras con todos sus compañeros, excepto con Judas Iscariote. Andrés nunca había tenido con Judas una asociación personal tan íntima como con los otros apóstoles; por esta razón, no le había dado importancia al hecho de que Judas nunca se hubiera relacionado de manera espontánea y confidencial con el jefe del cuerpo apostólico. Pero Andrés estaba ahora tan preocupado por la actitud de Judas que, más tarde aquella noche, después de que todos los apóstoles estuvieran profundamente dormidos, buscó a Jesús y le expuso la causa de su ansiedad. Jesús le dijo: <<No está de más, Andrés, que hayas venido a mí con este asunto, pero ya no podemos hacer nada más. Continúa concediéndole la máxima confianza a este apóstol. Y no digas nada a sus hermanos de esta conversación conmigo>>.

\par 
%\textsuperscript{(1751.1)}
\textsuperscript{157:7.2} Esto fue todo lo que Andrés pudo sonsacarle a Jesús. Siempre había habido algunas reservas entre este judeo y sus hermanos galileos. Judas se había sentido conmocionado por la muerte de Juan el Bautista, gravemente ofendido por las reprimendas del Maestro en diversas ocasiones, decepcionado cuando Jesús se negó a ser proclamado rey, humillado cuando huyó de los fariseos, disgustado cuando se negó a aceptar el desafío de los fariseos que le pedían un signo, desconcertado por la negativa de su Maestro a recurrir a manifestaciones de poder y, más recientemente, deprimido y a veces abatido porque la tesorería estaba vacía. Además, Judas echaba de menos el estímulo de las multitudes.

\par 
%\textsuperscript{(1751.2)}
\textsuperscript{157:7.3} Cada uno de los otros apóstoles estaba igualmente afectado, en mayor o menor grado, por estas mismas pruebas y tribulaciones, pero amaban a Jesús. Al menos deben haber amado al Maestro más que Judas, porque continuaron con él hasta el amargo final.

\par 
%\textsuperscript{(1751.3)}
\textsuperscript{157:7.4} Como era de Judea, Judas tomó como una ofensa personal la reciente advertencia de Jesús a los apóstoles: <<tened cuidado con la influencia de los fariseos>>; tendía a considerar esta declaración como una alusión velada a él mismo. Pero el gran error de Judas era el siguiente: una y otra vez, cuando Jesús enviaba a sus apóstoles a orar a solas, Judas se entregaba a pensamientos de temor humano, en lugar de buscar una comunión sincera con las fuerzas espirituales del universo; además, se empeñaba en mantener dudas sutiles acerca de la misión de Jesús, y se entregaba a su tendencia desafortunada a albergar sentimientos de revancha.

\par 
%\textsuperscript{(1751.4)}
\textsuperscript{157:7.5} Jesús quería ahora llevar consigo a sus apóstoles al Monte Hermón, donde había decidido inaugurar, como Hijo de Dios, la cuarta fase de su ministerio terrestre. Algunos de ellos habían estado presentes en su bautismo en el Jordán y habían presenciado el comienzo de su carrera como Hijo del Hombre, y deseaba que algunos de ellos estuvieran presentes también para escuchar la autoridad con que asumiría el nuevo papel público de Hijo de Dios. En consecuencia, el viernes 12 de agosto por la mañana, Jesús dijo a los doce: <<Comprad provisiones y preparaos para viajar a aquella montaña, donde el espíritu me pide que vaya para recibir los dones que me permitirán terminar mi obra en la Tierra. Y deseo llevar conmigo a mis hermanos para que también puedan ser fortalecidos con vistas a los tiempos difíciles que les esperan cuando atraviesen conmigo esta experiencia>>.


\chapter{Documento 158. El monte de la transfiguración}
\par 
%\textsuperscript{(1752.1)}
\textsuperscript{158:0.1} El viernes por la tarde 12 de agosto del año 29, el Sol iba a ponerse cuando Jesús y sus compañeros llegaron al pie del Monte Hermón, cerca del mismo lugar donde el joven Tiglat había aguardado en otro tiempo mientras el Maestro subía solo a la montaña para asegurar los destinos espirituales de Urantia y poner fin técnicamente a la rebelión de Lucifer. Permanecieron aquí durante dos días, preparándose espiritualmente para los acontecimientos que se iban a producir en breve.

\par 
%\textsuperscript{(1752.2)}
\textsuperscript{158:0.2} De una manera general, Jesús sabía de antemano lo que iba a suceder en la montaña, y deseaba vivamente que todos sus apóstoles pudieran compartir esta experiencia. Se detuvo con ellos al pie de la montaña con el fin de prepararlos para esta revelación de sí mismo. Pero no pudieron alcanzar los niveles espirituales que hubieran justificado el hecho de exponerlos a la experiencia completa de la visita de los seres celestiales que pronto iban a aparecer sobre la Tierra. Y como no podía llevar a todos sus compañeros con él, decidió llevarse únicamente a los tres que lo acompañaban habitualmente en estas vigilias especiales. En consecuencia, solamente Pedro, Santiago y Juan compartieron, aunque de forma parcial, esta experiencia única con el Maestro.

\section*{1. La transfiguración}
\par 
%\textsuperscript{(1752.3)}
\textsuperscript{158:1.1} El lunes 15 de agosto por la mañana temprano, seis días después de la memorable confesión de Pedro, realizada un mediodía al borde del camino debajo de las moreras, Jesús y los tres apóstoles empezaron la ascensión del Monte Hermón.

\par 
%\textsuperscript{(1752.4)}
\textsuperscript{158:1.2} Jesús había sido llamado para que subiera solo a la montaña con el fin de gestionar unos asuntos importantes que tenían que ver con el desarrollo de su donación en la carne, ya que esta experiencia estaba relacionada con el universo creado por él mismo. Es significativo que este acontecimiento extraordinario estuviera calculado para que ocurriera mientras Jesús y los apóstoles se encontraban en las tierras de los gentiles, y que se produjera efectivamente en una montaña de los gentiles.

\par 
%\textsuperscript{(1752.5)}
\textsuperscript{158:1.3} Llegaron a su destino, casi a mitad de camino de la cima, un poco antes del mediodía. Mientras almorzaban, Jesús contó a los tres apóstoles una parte de la experiencia que había tenido, poco después de su bautismo, en las colinas al este del Jordán, y también les dijo algo más sobre su experiencia en el Monte Hermón durante su visita anterior a este retiro solitario.

\par 
%\textsuperscript{(1752.6)}
\textsuperscript{158:1.4} Cuando era niño, Jesús tenía la costumbre de subir a la colina que estaba cerca de su casa, y soñar con las batallas que los ejércitos de los imperios habían librado en la planicie de Esdraelón; ahora, subía al Monte Hermón para recibir la dotación que lo prepararía para descender a las llanuras del Jordán y representar las escenas finales del drama de su donación en Urantia. Este día, en el Monte Hermón, el Maestro hubiera podido abandonar la lucha y volver a gobernar sus dominios universales, pero no solamente escogió satisfacer las exigencias de su orden de filiación divina, contenidas en el mandato del Hijo Eterno que está en el Paraíso, sino que también escogió satisfacer plenamente y hasta el fin la presente voluntad de su Padre Paradisiaco. Este día de agosto, tres de sus apóstoles vieron cómo rehusaba ser investido con la plena autoridad sobre su universo. Observaron aterrados la partida de los mensajeros celestiales, dejándolo solo para que terminara su vida terrestre como Hijo del Hombre e Hijo de Dios.

\par 
%\textsuperscript{(1753.1)}
\textsuperscript{158:1.5} La fe de los apóstoles alcanzó su punto culminante en el momento de la alimentación de los cinco mil, y luego cayó rápidamente casi hasta el punto cero. Ahora, debido a que el Maestro había admitido su divinidad, la fe rezagada de los doce se elevó hasta su apogeo en las pocas semanas que siguieron, para sufrir después un declive progresivo. El tercer resurgimiento de su fe no se produjo hasta después de la resurrección del Maestro.

\par 
%\textsuperscript{(1753.2)}
\textsuperscript{158:1.6} Hacia las tres de esta hermosa tarde, Jesús se despidió de los tres apóstoles, diciendo: <<Me voy solo durante un tiempo para comulgar con el Padre y sus mensajeros; os ruego que os quedéis aquí, y mientras esperáis mi regreso, orad para que se haga la voluntad del Padre en toda vuestra experiencia relacionada con el resto de la misión donadora del Hijo del Hombre>>. Después de haberles dicho esto, Jesús se retiró para celebrar una larga conferencia con Gabriel y el Padre Melquisedek, y no regresó hasta cerca de las seis. Cuando Jesús observó la ansiedad de sus apóstoles debido a su ausencia prolongada, dijo: <<¿Por qué teníais miedo? Sabéis muy bien que debo ocuparme de los asuntos de mi Padre; ¿por qué dudáis cuando no estoy con vosotros? Os declaro ahora que el Hijo del Hombre ha optado por pasar toda su vida en medio de vosotros y como uno de vosotros. Estad alegres; no os abandonaré hasta que haya terminado mi obra>>.

\par 
%\textsuperscript{(1753.3)}
\textsuperscript{158:1.7} Mientras compartían una cena frugal, Pedro le preguntó al Maestro: <<¿Cuánto tiempo vamos a permanecer en esta montaña, lejos de nuestros hermanos?>> Jesús contestó: <<Hasta que hayáis visto la gloria del Hijo del Hombre y sepáis que todo lo que os he declarado es verdad>>. Y hablaron de los asuntos de la rebelión de Lucifer, mientras estaban sentados cerca del rescoldo encendido de su fuego, hasta que la oscuridad los envolvió y los párpados de los apóstoles se hicieron pesados, pues habían emprendido su viaje muy temprano aquella mañana.

\par 
%\textsuperscript{(1753.4)}
\textsuperscript{158:1.8} Los tres dormían profundamente desde hacía una media hora, cuando fueron despertados repentinamente por un crujido cercano; al mirar a su alrededor, para su gran sorpresa y consternación, vieron a Jesús conversando íntimamente con dos seres brillantes vestidos con las vestiduras de luz del mundo celestial. El rostro y la silueta de Jesús brillaban con la luminosidad de una luz celestial. Los tres hablaban en un lenguaje extraño, pero por ciertas cosas dichas, Pedro supuso erróneamente que los seres que estaban con Jesús eran Moisés y Elías; en realidad se trataba de Gabriel y del Padre Melquisedek. A petición de Jesús, los controladores físicos habían dispuesto lo necesario para que los apóstoles presenciaran esta escena.

\par 
%\textsuperscript{(1753.5)}
\textsuperscript{158:1.9} Los tres apóstoles estaban tan enormemente asustados que tardaron en recuperarse; mientras la deslumbrante visión se desvanecía delante de ellos y observaban que Jesús se quedaba solo, Pedro, que fue el primero en recuperarse, dijo: <<Jesús, Maestro, es provechoso haber estado aquí. Nos alegramos de ver esta gloria. Nos disgusta tener que regresar al mundo ignominioso. Si te parece bien, quedémonos aquí, y levantaremos tres tiendas, una para ti, otra para Moisés y otra para Elías>>. Pedro dijo esto a causa de su confusión, y porque no se le ocurrió ninguna otra cosa en ese momento.

\par 
%\textsuperscript{(1753.6)}
\textsuperscript{158:1.10} Mientras Pedro aún estaba hablando, una nube plateada se les acercó y ensombreció a los cuatro. Ahora los apóstoles se asustaron mucho, y cuando caían de bruces para adorar, oyeron una voz, la misma que había hablado en el momento del bautismo de Jesús, que decía: <<Éste es mi Hijo amado; prestadle atención>>. Cuando la nube se desvaneció, Jesús estaba de nuevo solo con los tres; se inclinó y los tocó, diciendo: <<Levantaos y no temáis; veréis cosas más grandes que ésta>>. Pero los apóstoles estaban realmente aterrorizados; mientras se preparaban para bajar de la montaña, poco antes de la medianoche, formaban un trío silencioso y pensativo.

\section*{2. El descenso de la montaña}
\par 
%\textsuperscript{(1754.1)}
\textsuperscript{158:2.1} Durante cerca de la primera mitad del descenso de la montaña, no se dijo ni una palabra. Jesús empezó entonces la conversación, comentando: <<Aseguraos de que no le contáis a nadie, ni siquiera a vuestros hermanos, lo que habéis visto y oído en esta montaña, hasta que el Hijo del Hombre haya resucitado de entre los muertos>>. Los tres apóstoles se quedaron anonadados y desconcertados por las palabras del Maestro <<hasta que el Hijo del Hombre haya resucitado de entre los muertos>>. Habían reafirmado tan recientemente su fe en él como Libertador, el Hijo de Dios, y acababan de verlo transfigurado en gloria delante de sus propios ojos, ¡y ahora empezaba a hablar de <<resurrección de entre los muertos>>!

\par 
%\textsuperscript{(1754.2)}
\textsuperscript{158:2.2} Pedro se estremeció con el pensamiento de la muerte del Maestro ---era una idea demasiado desagradable de soportar--- y temiendo que Santiago o Juan pudieran hacer alguna pregunta relacionada con esta declaración, pensó que sería mejor iniciar una conversación sobre otro tema; al no saber de qué hablar, expresó el primer pensamiento que le pasó por la cabeza, diciendo: <<Maestro, ¿cómo es que los escribas dicen que Elías debe venir primero antes de que aparezca el Mesías?>> Sabiendo que Pedro intentaba evitar mencionar su muerte y resurrección, Jesús respondió: <<Es cierto que Elías viene primero para preparar el camino del Hijo del Hombre, el cual debe sufrir muchas cosas y al final ser rechazado. Pero te hago saber que Elías ya ha venido, y que no le recibieron, sino que le hicieron todo lo que quisieron>>. Los tres apóstoles se dieron cuenta entonces de que se refería a Juan el Bautista como si fuera Elías. Jesús sabía que, si insistían en considerarlo como el Mesías, entonces Juan debía ser el Elías de la profecía.

\par 
%\textsuperscript{(1754.3)}
\textsuperscript{158:2.3} Jesús les recomendó que guardaran silencio sobre la visión anticipada que habían tenido de la gloria que le esperaba después de su resurrección porque no quería fomentar la idea, ahora que era recibido como el Mesías, de que iba a cumplir en alguna medida sus conceptos erróneos de un libertador que realizaba prodigios. Aunque Pedro, Santiago y Juan reflexionaron sobre todas estas cosas, no hablaron de ellas a nadie hasta después de la resurrección del Maestro.

\par 
%\textsuperscript{(1754.4)}
\textsuperscript{158:2.4} Mientras continuaban descendiendo de la montaña, Jesús les dijo: <<No habéis querido recibirme como Hijo del Hombre; por eso he permitido que me recibáis de acuerdo con vuestra resolución establecida; pero no os equivoquéis, la voluntad de mi Padre debe prevalecer. Si escogéis seguir así la tendencia de vuestra propia voluntad, debéis prepararos para sufrir muchas desilusiones y experimentar muchas pruebas; pero el entrenamiento que os he dado debería bastar para que atraveséis triunfalmente estas penas que vosotros mismos habréis escogido>>.

\par 
%\textsuperscript{(1754.5)}
\textsuperscript{158:2.5} Jesús no se llevó a Pedro, Santiago y Juan a la montaña de la transfiguración porque estuvieran, de alguna manera, mejor preparados que los otros apóstoles para presenciar lo que sucedió, o porque fueran espiritualmente más capaces de disfrutar de este raro privilegio. De ninguna manera. Sabía muy bien que ninguno de los doce estaba cualificado espiritualmente para esta experiencia; por eso se llevó solamente a los tres apóstoles que estaban asignados para acompañarlo en los momentos en que deseaba estar solo para disfrutar de una comunión solitaria.

\section*{3. El significado de la transfiguración}
\par 
%\textsuperscript{(1755.1)}
\textsuperscript{158:3.1} Lo que Pedro, Santiago y Juan presenciaron en la montaña de la transfiguración fue un vislumbre fugaz del espectáculo celestial que tuvo lugar aquel día memorable en el Monte Hermón. La transfiguración fue un acto para:

\par 
%\textsuperscript{(1755.2)}
\textsuperscript{158:3.2} 1. La aceptación, por parte del Hijo-Madre Eterno del Paraíso, de la plenitud de la donación de la vida encarnada de Miguel en Urantia. Jesús había recibido ahora la seguridad de que había cumplido las exigencias del Hijo Eterno. Fue Gabriel quien le aportó a Jesús esta seguridad.

\par 
%\textsuperscript{(1755.3)}
\textsuperscript{158:3.3} 2. El testimonio de la satisfacción del Espíritu Infinito en cuanto a la plenitud de la donación en Urantia en la similitud de la carne mortal. La representante del Espíritu Infinito en el universo local, la asociada inmediata y la colaboradora siempre presente de Miguel en Salvington, habló en esta ocasión a través del Padre Melquisedek.

\par 
%\textsuperscript{(1755.4)}
\textsuperscript{158:3.4} Jesús recibió con agrado estos testimonios del éxito de su misión terrestre, presentados por los mensajeros del Hijo Eterno y del Espíritu Infinito, pero observó que su Padre no indicaba que la donación en Urantia hubiera terminado; la presencia invisible del Padre sólo dio testimonio a través del Ajustador Personalizado de Jesús, diciendo: <<Éste es mi hijo amado; prestadle atención>>. Y esto fue expresado en palabras para que los tres apóstoles también pudieran escucharlas.

\par 
%\textsuperscript{(1755.5)}
\textsuperscript{158:3.5} Después de esta visita celestial, Jesús intentó conocer la voluntad de su Padre y decidió continuar la donación mortal hasta su fin natural. Éste fue el significado de la transfiguración para Jesús. Para los tres apóstoles, se trató de un acontecimiento que marcó la entrada del Maestro en la fase final de su carrera terrestre como Hijo de Dios e Hijo del Hombre.

\par 
%\textsuperscript{(1755.6)}
\textsuperscript{158:3.6} Después de la visita oficial de Gabriel y del Padre Melquisedek, Jesús mantuvo una conversación familiar con estos Hijos ayudantes suyos, y habló con ellos sobre los asuntos del universo.

\section*{4. El muchacho epiléptico}
\par 
%\textsuperscript{(1755.7)}
\textsuperscript{158:4.1} Jesús y sus compañeros llegaron al campamento apostólico este martes por la mañana un poco antes de la hora del desayuno. A medida que se acercaban, observaron una multitud considerable reunida alrededor de los apóstoles, y pronto empezaron a oír las ruidosas discusiones y controversias de este grupo de unas cincuenta personas, que incluía a los nueve apóstoles y a una asamblea dividida por igual entre los escribas de Jerusalén y los discípulos creyentes, que habían seguido a Jesús y a sus asociados en su viaje desde Magadán.

\par 
%\textsuperscript{(1755.8)}
\textsuperscript{158:4.2} Aunque la muchedumbre sostenía discusiones diversas, la controversia principal estaba centrada en cierto ciudadano de Tiberiades que había llegado el día anterior en busca de Jesús. Este hombre, Santiago de Safed, tenía un hijo único de unos catorce años de edad, que estaba gravemente afligido de epilepsia. Además de esta enfermedad nerviosa, el muchacho había sido poseído por uno de esos intermedios errantes, malévolos y rebeldes, que entonces estaban presentes y sin control en la Tierra, de manera que el joven era epiléptico y a la vez estaba poseído por un demonio.

\par 
%\textsuperscript{(1755.9)}
\textsuperscript{158:4.3} Durante cerca de dos semanas, este padre ansioso, oficial subalterno de Herodes Antipas, había vagado por las fronteras occidentales de los dominios de Felipe buscando a Jesús para suplicarle que curara a su hijo afligido. Y no alcanzó al grupo apostólico hasta alrededor del mediodía de este día, mientras Jesús estaba arriba en la montaña con los tres apóstoles.

\par 
%\textsuperscript{(1756.1)}
\textsuperscript{158:4.4} Los nueve apóstoles se quedaron muy sorprendidos y bastante inquietos cuando este hombre, acompañado de casi cuarenta personas más que venían buscando a Jesús, se encontró repentinamente con ellos. En el momento de llegar este grupo, los nueve apóstoles, o al menos la mayoría de ellos, habían sucumbido a su antigua tentación ---la de discutir quién sería el más grande en el reino venidero; estaban atareados discurriendo sobre los puestos probables que serían asignados a cada apóstol. No podían simplemente liberarse por completo de la idea, tanto tiempo acariciada, de la misión material del Mesías. Ahora que el mismo Jesús había aceptado la confesión de los apóstoles de que era realmente el Libertador ---al menos había admitido el hecho de su divinidad--- qué cosa más natural que, durante este período en que estaban separados del Maestro, se pusieran a hablar de las esperanzas y ambiciones que predominaban en sus corazones. Estaban ocupados en estas discusiones cuando Santiago de Safed y sus compañeros, que buscaban a Jesús, dieron con ellos.

\par 
%\textsuperscript{(1756.2)}
\textsuperscript{158:4.5} Andrés se levantó para saludar a este padre y a su hijo, diciendo: <<¿A quién buscáis?>> Santiago dijo: <<Mi buen hombre, busco a tu Maestro. Busco la curación para mi hijo afligido. Quisiera que Jesús echara a ese diablo que posee a mi hijo>>. El padre se puso entonces a contar a los apóstoles que su hijo estaba tan afligido, que muchas veces casi había perdido la vida a consecuencia de estos ataques malignos.

\par 
%\textsuperscript{(1756.3)}
\textsuperscript{158:4.6} Mientras los apóstoles escuchaban, Simón Celotes y Judas Iscariote se acercaron al padre, diciendo: <<Nosotros podemos curarlo; no necesitas esperar a que regrese el Maestro. Somos los embajadores del reino, y ya no mantenemos estas cosas en secreto. Jesús es el Libertador, y nos han sido entregadas las llaves del reino>>. Para entonces, Andrés y Tomás estaban consultándose a un lado, mientras que Natanael y los demás observaban la escena, asombrados; todos estaban horrorizados por la súbita audacia, si no presunción, de Simón y de Judas. El padre dijo entonces: <<Si os ha sido dado el hacer estas obras, os ruego que pronunciéis las palabras que liberarán a mi hijo de esta esclavitud>>. Entonces Simón se adelantó, colocó su mano sobre la cabeza del niño, lo miró fijamente a los ojos y ordenó: <<Sal de él, espíritu impuro; en nombre de Jesús, obedéceme>>. Pero el muchacho tuvo simplemente un ataque más violento, mientras los escribas se mofaban de los apóstoles y los creyentes decepcionados sufrían las burlas de estos críticos hostiles.

\par 
%\textsuperscript{(1756.4)}
\textsuperscript{158:4.7} Andrés estaba profundamente disgustado por este esfuerzo descaminado y su lamentable fracaso. Reunió aparte a los apóstoles para conversar y orar. Después de este período de meditación, sintiendo la aguda punzada de la derrota y la humillación que caía sobre todos ellos, Andrés hizo una segunda tentativa por echar al demonio, pero el fracaso coronó de nuevo sus esfuerzos. Andrés confesó francamente su derrota y le rogó al padre que permaneciera con ellos durante la noche o hasta que Jesús regresara, diciendo: <<Quizás esta clase de demonios no se va, a menos que se lo ordene personalmente el Maestro>>.

\par 
%\textsuperscript{(1756.5)}
\textsuperscript{158:4.8} Y así, mientras Jesús descendía de la montaña con Pedro, Santiago y Juan, exuberantes y extasiados, sus nueve hermanos estaban también desvelados pero a causa de su confusión, abatimiento y humillación. Formaban un grupo desanimado y escarmentado. Pero Santiago de Safed no quiso darse por vencido. Aunque no podían darle una idea de cuándo volvería Jesús, decidió quedarse allí hasta que regresara el Maestro.

\section*{5. Jesús cura al muchacho}
\par 
%\textsuperscript{(1757.1)}
\textsuperscript{158:5.1} Mientras Jesús se acercaba, los nueve apóstoles se sintieron más que aliviados de recibirlo y muy animados al contemplar la alegría y el entusiasmo poco común que se reflejaba en los rostros de Pedro, Santiago y Juan. Todos se abalanzaron para saludar a Jesús y a sus tres hermanos. Mientras intercambiaban los saludos, el gentío se acercó, y Jesús preguntó: <<¿Sobre qué estabais discutiendo cuando nos acercábamos?>> Pero antes de que los apóstoles desconcertados y humillados pudieran contestar a la pregunta del Maestro, el ansioso padre del joven afligido se adelantó y, arrodillándose a los pies de Jesús, dijo: <<Maestro, tengo un hijo, un hijo único, que está poseído por un espíritu maligno. Cuando tiene un ataque, no solamente grita de terror, echa espuma por la boca y cae como muerto, sino que con mucha frecuencia este espíritu maligno que lo posee lo destroza con convulsiones y a veces lo ha arrojado al agua e incluso al fuego. Mi hijo se está consumiendo con un gran rechinar de dientes y a consecuencia de sus numerosas magulladuras. Su vida es peor que la muerte; su madre y yo tenemos el corazón triste y el espíritu destrozado. Ayer, hacia el mediodía, buscándote a ti encontré a tus discípulos, y mientras te esperábamos, tus apóstoles intentaron echar a este demonio, pero no pudieron hacerlo. Y ahora, Maestro, ¿harás esto por nosotros, curarás a mi hijo?>>

\par 
%\textsuperscript{(1757.2)}
\textsuperscript{158:5.2} Cuando Jesús escuchó este relato, tocó al padre arrodillado y le rogó que se levantara, mientras echaba una mirada penetrante a los apóstoles cercanos. Jesús dijo entonces a todos los que estaban delante de él: <<Oh generación incrédula y perversa, ¿cuánto tiempo seré indulgente con vosotros? ¿Cuánto tiempo estaré con vosotros? ¿Cuánto tiempo necesitaréis para aprender que las obras de la fe no aparecen a petición de la incredulidad escéptica?>> Luego, señalando al padre desconcertado, Jesús dijo: <<Trae aquí a tu hijo>>. Cuando Santiago hubo traído al muchacho, Jesús preguntó: <<¿Cuánto tiempo hace que el niño está afligido de esta manera?>> El padre respondió: <<Desde que era muy pequeño>>. Mientras hablaban, el joven sufrió un ataque violento y cayó en medio de ellos, rechinando los dientes y echando espuma por la boca. Después de una serie de convulsiones violentas, se quedó tendido como muerto delante de ellos. El padre se arrodilló de nuevo a los pies de Jesús mientras imploraba al Maestro, diciendo: <<Si puedes curarlo, te suplico que tengas compasión de nosotros y nos liberes de esta aflicción>>. Cuando Jesús escuchó estas palabras, bajó la mirada hacia el rostro ansioso del padre, y dijo: <<No pongas en duda el poder del amor de mi Padre, sino solamente la sinceridad y el alcance de tu fe. Todas las cosas son posibles para aquel que cree realmente>>. Entonces, Santiago de Safed pronunció aquellas palabras inolvidables mezcladas de fe y de duda: <<Señor, yo creo. Te ruego que me ayudes en mi incredulidad>>.

\par 
%\textsuperscript{(1757.3)}
\textsuperscript{158:5.3} Cuando Jesús escuchó estas palabras, se adelantó, cogió al niño de la mano y dijo: <<Voy a hacer esto de acuerdo con la voluntad de mi Padre y en honor de la fe viviente. Hijo mío, ¡levántate! Espíritu desobediente, sal de él y no vuelvas>>. Luego, Jesús puso la mano del joven en la de su padre, y dijo: <<Sigue tu camino. El Padre ha concedido el deseo de tu alma>>. Todos los que estaban presentes, incluídos los enemigos de Jesús, se quedaron asombrados por lo que habían visto.

\par 
%\textsuperscript{(1757.4)}
\textsuperscript{158:5.4} Para los tres apóstoles que habían disfrutado tan recientemente del éxtasis espiritual de las escenas y experiencias de la transfiguración, fue realmente una desilusión volver tan pronto a la escena de la derrota y la frustración de sus compañeros apóstoles. Pero siempre fue así con estos doce embajadores del reino. Alternaban constantemente entre la exaltación y la humillación en las experiencias de su vida.

\par 
%\textsuperscript{(1758.1)}
\textsuperscript{158:5.5} Ésta fue una verdadera curación de una doble aflicción, una dolencia física y una enfermedad de espíritu. La curación del muchacho fue permanente a partir de aquel momento. Cuando Santiago hubo partido con su hijo restablecido, Jesús dijo: <<Vamos ahora a Cesarea de Filipo; preparaos enseguida>>. Y formaban un grupo silencioso a medida que viajaban hacia el sur, mientras la multitud iba detrás.

\section*{6. En el jardín de Celsus}
\par 
%\textsuperscript{(1758.2)}
\textsuperscript{158:6.1} Pasaron la noche con Celsus, y aquella tarde en el jardín, después de que hubieran comido y descansado, los doce se reunieron alrededor de Jesús, y Tomás dijo: <<Maestro, como los que nos quedamos atrás continuamos ignorando todavía lo que sucedió arriba en la montaña, que en tan gran medida animó a nuestros hermanos que te acompañaban, deseamos ardientemente que nos hables de nuestra derrota y nos instruyas en estas cuestiones, puesto que las cosas que sucedieron en la montaña no se pueden revelar en este momento>>.

\par 
%\textsuperscript{(1758.3)}
\textsuperscript{158:6.2} Jesús le contestó a Tomás, diciendo: <<Todo lo que tus hermanos escucharon en la montaña os será revelado a su debido tiempo. Pero ahora quiero mostraros la causa de vuestra derrota en aquello que intentasteis tan imprudentemente. Ayer, mientras vuestro Maestro y sus compañeros, vuestros hermanos, subían a aquella montaña para buscar un conocimiento más amplio de la voluntad del Padre y pedir una dotación más rica de sabiduría para hacer eficazmente esa voluntad divina, vosotros que permanecíais aquí de vigilancia, con la instrucción de procurar adquirir una mente con perspicacia espiritual y de orar con nosotros para obtener una revelación más completa de la voluntad del Padre, en lugar de ejercer la fe que está a vuestra disposición, cedisteis a la tentación y caísteis en vuestras viejas tendencias nocivas de buscar para vosotros mismos unos puestos de preferencia en el reino de los cielos ---en ese reino material y temporal que persistís en imaginar. Y os aferráis a estos conceptos erróneos a pesar de la declaración reiterativa de que mi reino no es de este mundo>>.

\par 
%\textsuperscript{(1758.4)}
\textsuperscript{158:6.3} <<Apenas capta vuestra fe la identidad del Hijo del Hombre, vuestro deseo egoísta por los ascensos mundanos os arrastra de nuevo, y empezáis a discutir entre vosotros quién será el más grande en el reino de los cielos, un reino que no existe ni existirá nunca tal como os empeñáis en concebirlo. ¿No os he dicho que el que quiera ser el más grande en el reino de la fraternidad espiritual de mi Padre debe volverse pequeño a sus propios ojos, y convertirse así en el servidor de sus hermanos? La grandeza espiritual consiste en un amor comprensivo semejante al amor de Dios, y no en el placer de ejercer el poder material para la exaltación del yo. En aquello que intentasteis y fracasasteis de manera tan completa, vuestra intención no era pura. Vuestro móvil no era divino. Vuestro ideal no era espiritual. Vuestra ambición no era altruista. Vuestra manera de obrar no estaba basada en el amor, y la meta que queríais alcanzar no era la voluntad del Padre que está en los cielos>>.

\par 
%\textsuperscript{(1758.5)}
\textsuperscript{158:6.4} <<¿Cuánto tiempo os llevará aprender que no podéis abreviar el curso de los fenómenos naturales establecidos, salvo cuando estas cosas están de acuerdo con la voluntad del Padre? Tampoco podéis realizar una obra espiritual en ausencia de poder espiritual. Y no podéis hacer ninguna de estas cosas, aunque su potencial esté presente, sin la existencia de un tercer factor humano esencial, la experiencia personal de poseer una fe viviente. ¿Necesitáis siempre las manifestaciones materiales para sentiros atraídos hacia las realidades espirituales del reino? ¿No podéis captar el significado espiritual de mi misión sin la manifestación visible de obras excepcionales? ¿Cuándo se podrá contar con vosotros para que os adhiráis a las realidades espirituales superiores del reino, sin hacer caso de la apariencia exterior de todas las manifestaciones materiales?>>

\par 
%\textsuperscript{(1759.1)}
\textsuperscript{158:6.5} Después de haber hablado así a los doce, Jesús añadió: <<Ahora, id a descansar, porque mañana volveremos a Magadán y allí deliberaremos sobre nuestra misión en las ciudades y pueblos de la Decápolis. Como conclusión de la experiencia de este día, dejadme repetir a cada uno de vosotros lo que dije a vuestros hermanos en la montaña, y que estas palabras se graben profundamente en vuestro corazón: El Hijo del Hombre empieza ahora la última fase de su donación. Estamos a punto de comenzar los trabajos que luego conducirán a la gran prueba final de vuestra fe y devoción, cuando yo sea entregado entre las manos de los hombres que buscan mi destrucción. Y recordad lo que os digo: Al Hijo del Hombre le darán muerte, pero resucitará>>.

\par 
%\textsuperscript{(1759.2)}
\textsuperscript{158:6.6} Se retiraron para pasar la noche, llenos de tristeza. Estaban desconcertados; no podían comprender estas palabras. Aunque temían hacer alguna pregunta sobre lo que Jesús había dicho, recordaron todo esto después de su resurrección.

\section*{7. La protesta de Pedro}
\par 
%\textsuperscript{(1759.3)}
\textsuperscript{158:7.1} Aquel miércoles por la mañana temprano, Jesús y los doce salieron de Cesarea de Filipo hacia el parque de Magadán, cerca de Betsaida-Julias. Los apóstoles habían dormido muy poco aquella noche; así pues, se levantaron temprano y se prepararon para partir. Incluso a los imperturbables gemelos Alfeo les había conmocionado esta conversación sobre la muerte de Jesús. A medida que viajaban hacia el sur, un poco más allá de las Aguas de Merom llegaron a la carretera de Damasco, y como deseaban evitar a los escribas y a otras personas que Jesús sabía que pronto vendrían caminando detrás de ellos, ordenó continuar hasta Cafarnaúm por la carretera de Damasco que atraviesa Galilea. Hizo esto porque sabía que aquellos que lo seguían continuarían por la carretera al este del Jordán, pues suponían que Jesús y los apóstoles tendrían miedo de cruzar por el territorio de Herodes Antipas. Jesús intentaba eludir a sus críticos y a la multitud que lo seguía, para poder estar a solas con sus apóstoles ese día.

\par 
%\textsuperscript{(1759.4)}
\textsuperscript{158:7.2} Habían caminado a través de Galilea hasta bien pasada la hora del almuerzo, cuando se detuvieron a la sombra para descansar. Después de que hubieron compartido la comida, Andrés, dirigiéndose a Jesús, dijo: <<Maestro, mis hermanos no comprenden tus palabras profundas. Hemos llegado a creer plenamente que eres el Hijo de Dios, y ahora escuchamos esas extrañas palabras acerca de dejarnos, acerca de morir. No comprendemos tu enseñanza. ¿Es que nos hablas en parábolas? Te rogamos que nos hables claramente y de una manera no velada>>.

\par 
%\textsuperscript{(1759.5)}
\textsuperscript{158:7.3} En respuesta a la petición de Andrés, Jesús dijo: <<Hermanos míos, debido a que habéis confesado que soy el Hijo de Dios, me veo obligado a empezar a desvelaros la verdad sobre el final de la donación del Hijo del Hombre en la Tierra. Insistís en aferraros a la creencia de que soy el Mesías, y no queréis abandonar la idea de que el Mesías debe sentarse en un trono en Jerusalén; por eso insisto en deciros que el Hijo del Hombre deberá pronto ir a Jerusalén, sufrir muchas cosas, ser rechazado por los escribas, los ancianos y los principales sacerdotes, y después de todo eso, ser ejecutado y resucitar de entre los muertos. Y no os estoy diciendo una parábola. Os digo la verdad a fin de que estéis preparados para cuando esos acontecimientos caigan repentinamente sobre nosotros>>. Mientras estaba hablando todavía, Simón Pedro se precipitó impetuosamente hacia él, puso su mano en el hombro del Maestro y dijo: <<Maestro, está lejos de nuestra intención discutir contigo, pero declaro que estas cosas no te sucederán nunca>>.

\par 
%\textsuperscript{(1760.1)}
\textsuperscript{158:7.4} Pedro habló así porque amaba a Jesús; pero la naturaleza humana del Maestro reconoció en estas palabras de afecto bien intencionado la sugerencia sutil de una tentación, la de cambiar su política de continuar hasta el fin su donación terrestre de acuerdo con la voluntad de su Padre Paradisiaco. Precisamente porque detectó el peligro de permitir que las sugerencias de sus mismos amigos afectuosos y leales le disuadieran, Jesús se volvió hacia Pedro y los otros apóstoles, diciendo: <<Quédate detrás de mí. Hueles al espíritu del adversario, al tentador. Cuando habláis de esta manera, no estáis de mi parte, sino más bien de parte de nuestro enemigo. De esta forma vuestro amor por mí lo convertís en un obstáculo para yo hacer la voluntad del Padre. No prestéis atención a los caminos de los hombres, sino más bien a la voluntad de Dios>>.

\par 
%\textsuperscript{(1760.2)}
\textsuperscript{158:7.5} Cuando se hubieron recobrado del primer impacto de la punzante reprimenda de Jesús, y antes de reanudar su viaje, el Maestro dijo además: <<Si alguien quiere seguirme, que no haga caso de sí mismo, que se encargue diariamente de sus responsabilidades y que me siga. Porque el que quiera salvar su vida egoístamente, la perderá, pero el que pierda su vida por mi causa y por el evangelio, la salvará. ¿De qué le sirve a un hombre ganar el mundo entero si pierde su propia alma? ¿Qué podría dar un hombre a cambio de la vida eterna? No os avergoncéis de mí y de mis palabras en esta generación pecaminosa e hipócrita, como yo no me avergonzaré de reconoceros cuando aparezca con gloria delante de mi Padre en presencia de todas las huestes celestiales. Sin embargo, muchos de vosotros que estáis ahora delante de mí no experimentaréis la muerte hasta que hayáis visto llegar con poder este reino de Dios>>.

\par 
%\textsuperscript{(1760.3)}
\textsuperscript{158:7.6} Jesús indicó así claramente a los doce el camino doloroso y conflictivo que debían pisar si querían seguirlo. ¡Qué impacto causaron estas palabras en estos pescadores galileos que se empeñaban en soñar con un reino terrenal con puestos de honor para sí mismos! Pero sus corazones leales se conmovieron ante este llamamiento valiente, y ninguno de ellos sintió deseos de abandonarlo. Jesús no los enviaba solos al combate; él los conducía. Sólo les pedía que lo siguieran valientemente.

\par 
%\textsuperscript{(1760.4)}
\textsuperscript{158:7.7} Los doce captaban lentamente la idea de que Jesús les estaba diciendo algo sobre la posibilidad de su muerte. Sólo comprendían vagamente lo que les decía sobre su muerte, mientras que su declaración acerca de resucitar de entre los muertos no consiguió en absoluto grabarse en sus mentes. A medida que pasaban los días, y recordaban su experiencia en la montaña de la transfiguración, Pedro, Santiago y Juan llegaron a comprender mejor algunas de estas cuestiones.

\par 
%\textsuperscript{(1760.5)}
\textsuperscript{158:7.8} En toda la asociación de los doce con su Maestro, sólo unas pocas veces vieron la mirada centellante y escucharon las vivas palabras de reproche que Pedro y el resto de los apóstoles recibieron en esta ocasión. Jesús siempre había sido paciente con los defectos humanos de sus apóstoles, pero no fue así cuando se enfrentó a una amenaza inminente contra su programa de hacer implícitamente la voluntad de su Padre durante el resto de su carrera terrestre. Los apóstoles se quedaron literalmente anonadados; estaban asombrados y horrorizados. No encontraban palabras para expresar su tristeza. Empezaron a darse cuenta lentamente de lo que el Maestro tendría que soportar y de que deberían atravesar estas experiencias con él, pero no despertaron a la realidad de estos acontecimientos venideros hasta mucho tiempo después de estas primeras alusiones a la tragedia que amenazaba los últimos días de su vida.

\par 
%\textsuperscript{(1761.1)}
\textsuperscript{158:7.9} Jesús y los doce partieron en silencio hacia su campamento del parque de Magadán, pasando por Cafarnaúm. A medida que transcurría la tarde, aunque no conversaron con Jesús, hablaron mucho entre ellos mientras Andrés charlaba con el Maestro.

\section*{8. En la casa de Pedro}
\par 
%\textsuperscript{(1761.2)}
\textsuperscript{158:8.1} Entraron en Cafarnaúm al anochecer, pasaron por calles poco frecuentadas, y fueron directamente a la casa de Simón Pedro para cenar. Mientras David Zebedeo se preparaba para llevarlos al otro lado del lago, se demoraron en la casa de Simón, y entonces Jesús, mirando a Pedro y a los demás apóstoles, preguntó: <<Cuando caminabais juntos esta tarde, ¿de qué hablabais tan seriamente entre vosotros?>> Los apóstoles guardaron silencio, porque muchos de ellos habían continuado la discusión que empezaron en el Monte Hermón sobre los puestos que iban a tener en el reino venidero, sobre quién sería el más grande, y así sucesivamente. Conociendo las cosas que habían ocupado sus pensamientos durante aquel día, Jesús hizo señas a uno de los hijos pequeños de Pedro, sentó al niño entre ellos, y dijo: <<En verdad, en verdad os digo que a menos que cambiéis de opinión y os parezcáis más a este niño, poco progreso haréis en el reino de los cielos. Quienquiera que se humille y se vuelva como este pequeño, se convertirá en el más grande en el reino de los cielos. Quienquiera que recibe a un pequeño como éste, me recibe a mí. Y aquellos que me reciben, reciben también a Aquél que me ha enviado. Si queréis ser los primeros en el reino, procurad aportar estas buenas verdades a vuestros hermanos en la carne. Pero si alguien hace tropezar a uno de estos pequeños, sería mejor para él que le ataran una piedra de molino al cuello y lo arrojaran al mar. Si las cosas que hacéis con vuestras manos, o las cosas que veis con vuestros ojos, ofenden en el progreso del reino, sacrificad esos ídolos queridos, porque es mejor entrar en el reino desprovistos de muchas de las cosas que se aman en la vida, que aferrarse a esos ídolos y encontrarse excluido del reino. Pero por encima de todo, procurad no despreciar a uno solo de estos pequeños, porque sus ángeles están siempre contemplando el rostro de las huestes celestiales>>.

\par 
%\textsuperscript{(1761.3)}
\textsuperscript{158:8.2} Cuando Jesús hubo terminado de hablar, subieron a la barca y navegaron hacia el otro lado en dirección a Magadán.


\chapter{Documento 159. La gira por la Decápolis}
\par 
%\textsuperscript{(1762.1)}
\textsuperscript{159:0.1} CUANDO Jesús y los doce llegaron al parque de Magadán, encontraron que los estaba esperando un grupo de casi cien evangelistas y discípulos, incluyendo al cuerpo de mujeres, que ya estaban preparados para empezar inmediatamente la gira de enseñanza y predicación por las ciudades de la Decápolis.

\par 
%\textsuperscript{(1762.2)}
\textsuperscript{159:0.2} Este jueves 18 de agosto por la mañana, el Maestro reunió a sus seguidores y ordenó que cada uno de los apóstoles se asociara con uno de los doce evangelistas, y que junto con otros evangelistas, salieran en doce grupos para trabajar en las ciudades y pueblos de la Decápolis. Al cuerpo de mujeres y a los otros discípulos les ordenó que permanecieran con él. Jesús concedió a sus seguidores cuatro semanas para hacer esta gira, y les indicó que regresaran a Magadán como muy tarde el viernes 16 de septiembre. Prometió visitarlos a menudo durante este período. En el transcurso de este mes, los doce grupos trabajaron en Gerasa, Gamala, Hipos, Zafón, Gadara, Abila, Edrei, Filadelfia, Hesbón, Dium, Escitópolis y otras muchas ciudades. Durante toda esta gira, no se produjeron milagros de curación u otros acontecimientos extraordinarios.

\section*{1. El sermón sobre el perdón}
\par 
%\textsuperscript{(1762.3)}
\textsuperscript{159:1.1} Una tarde en Hipos, en respuesta a la pregunta de un discípulo, Jesús enseñó la lección sobre el perdón. El Maestro dijo:

\par 
%\textsuperscript{(1762.4)}
\textsuperscript{159:1.2} <<Si un hombre de buen corazón tiene cien ovejas y una de ellas se extravía, ¿no dejará inmediatamente a las noventa y nueve para salir en busca de la que se ha extraviado? Y si es un buen pastor, ¿no continuará buscando a la oveja perdida hasta que la haya encontrado? Entonces, cuando el pastor ha encontrado a su oveja perdida, se la echa al hombro y, mientras vuelve alegremente a su casa, llama a sus amigos y vecinos para decirles: `Regocijaos conmigo, porque he encontrado a mi oveja que estaba perdida.' Os aseguro que hay más alegría en el cielo por un pecador que se arrepiente, que por noventa y nueve justos que no necesitan arrepentirse. Sin embargo, no es la voluntad de mi Padre que está en los cielos que se extravíe uno de estos pequeños, y mucho menos que perezca. En vuestra religión, Dios puede recibir a los pecadores arrepentidos; en el evangelio del reino, el Padre sale a buscarlos antes incluso de que hayan pensado seriamente en arrepentirse>>.

\par 
%\textsuperscript{(1762.5)}
\textsuperscript{159:1.3} <<El Padre que está en los cielos ama a sus hijos, y por eso deberíais aprender a amaros los unos a los otros; el Padre que está en los cielos os perdona vuestros pecados; por eso deberíais aprender a perdonaros los unos a los otros. Si tu hermano peca contra ti, ve a verle y, con tacto y con paciencia, muestrale su falta. Y haz todo esto a solas con él. Si quiere escucharte, entonces habrás ganado a tu hermano. Pero si tu hermano no quiere escucharte, si persiste en su camino erróneo, ve a verle de nuevo, llevando contigo a uno o dos amigos comunes, para que así puedas tener dos o incluso tres testigos que confirmen tu testimonio y demuestren el hecho de que has tratado con justicia y misericordia al hermano que te ha ofendido. Pero si se niega a escuchar a tus hermanos, puedes contar toda la historia a la congregación, y si también se niega a escuchar a la fraternidad, que ésta tome la medida que estime más sabia; que ese miembro indisciplinado se vuelva un proscrito del reino. Aunque no podéis pretender juzgar el alma de vuestros semejantes, y aunque no podéis perdonar los pecados ni atreveros a usurpar de otra manera las prerrogativas de los supervisores de las huestes celestiales, sin embargo el mantenimiento del orden temporal en el reino de la Tierra ha sido depositado entre vuestras manos. Aunque no podéis entremeteros en los decretos divinos relacionados con la vida eterna, resolveréis los problemas de conducta en lo que respecta al bienestar temporal de la fraternidad en la Tierra. Así pues, en todas estas cuestiones relacionadas con la disciplina de la fraternidad, todo lo que decretéis en la Tierra será reconocido en el cielo. Aunque no podéis determinar el destino eterno del individuo, podéis legislar en lo que se refiere a la conducta del grupo, porque, cuando dos o tres de vosotros estéis de acuerdo sobre alguna de estas cosas y me lo pidáis a mí, se os concederá si vuestra petición no es incompatible con la voluntad de mi Padre que está en los cielos. Todo esto es perpetuamente cierto, porque allí donde dos o tres creyentes están reunidos, allí estoy yo en medio de ellos>>.

\par 
%\textsuperscript{(1763.1)}
\textsuperscript{159:1.4} Simón Pedro era el apóstol que estaba encargado de los que trabajaban en Hipos, y cuando escuchó hablar así a Jesús, preguntó: <<Señor, ¿cuántas veces tendré que perdonar a mi hermano que peca contra mí? ¿Hasta siete veces?>> Jesús le contestó a Pedro: <<No solamente siete veces, sino hasta setenta veces más siete. Por eso el reino de los cielos se puede comparar a cierto rey que ordenó un arreglo de cuentas con sus mayordomos. Cuando empezaron a realizar este examen de cuentas, trajeron ante él a uno de sus criados principales que confesó que le debía diez mil talentos a su rey. Este funcionario de la corte del rey alegó que había pasado por tiempos difíciles, y que no tenía con qué pagar sus obligaciones. El rey ordenó entonces que se confiscaran sus propiedades y que sus hijos fueran vendidos para pagar su deuda. Cuando el mayordomo principal escuchó este severo decreto, cayó de bruces ante el rey y le imploró que tuviera misericordia y le concediera más tiempo, diciendo: `Señor, ten un poco más de paciencia conmigo, y te lo pagaré todo.' Cuando el rey contempló a este servidor negligente y a su familia, se conmovió de compasión. Ordenó que lo liberaran y que se le perdonara completamente su deuda>>.

\par 
%\textsuperscript{(1763.2)}
\textsuperscript{159:1.5} <<Habiendo recibido así la misericordia y el perdón del rey, el mayordomo principal se fue a sus asuntos, y al encontrarse con uno de sus mayordomos subordinados que sólo le debía cien denarios, lo agarró, lo cogió por el cuello y le dijo: `Págame todo lo que me debes.' Entonces este mayordomo compañero suyo se postró delante del mayordomo principal y le suplicó diciendo: `Ten un poco de paciencia conmigo, y pronto podré pagarte.' Pero el mayordomo principal no quiso mostrarle misericordia a su colega, sino que lo arrojó a un calabozo hasta que pagara su deuda. Cuando sus compañeros de servicio vieron lo que había sucedido, se sintieron tan apenados que fueron a decírselo al rey, su señor y maestro. Cuando el rey se enteró del comportamiento de su mayordomo principal, llamó ante él a este hombre desagradecido e implacable y le dijo: `Eres un administrador perverso e indigno. Cuando buscaste compasión, te perdoné generosamente toda tu deuda. ¿Por qué no fuiste también misericordioso con tu compañero, como yo lo fui contigo?' El rey estaba tan sumamente enojado que entregó a su desagradecido mayordomo principal a los carceleros para que lo custodiaran hasta que pagara toda su deuda. De la misma manera, mi Padre celestial mostrará la más abundante misericordia a los que son profusamente misericordiosos con sus semejantes. ¿Cómo podéis acudir a Dios para pedirle que tenga consideración con vuestros defectos, si tenéis la costumbre de castigar a vuestros hermanos por ser culpables de esas mismas debilidades humanas? Os lo digo a todos: Habéis recibido generosamente las cosas buenas del reino; dad pues generosamente a vuestros compañeros de la Tierra>>.

\par 
%\textsuperscript{(1764.1)}
\textsuperscript{159:1.6} Jesús enseñó así los peligros e ilustró la injusticia de emitir un juicio personal sobre nuestros semejantes. La disciplina ha de ser mantenida y la justicia debe ser administrada, pero la sabiduría de la fraternidad debería prevalecer en todas estas cuestiones. Jesús confirió la autoridad legislativa y judicial al \textit{grupo}, y no al \textit{individuo}. Incluso esta autoridad que se concede al grupo no debe ser ejercida como una autoridad personal. Siempre existe el peligro de que el veredicto de un individuo pueda estar deformado por el prejuicio o distorsionado por la pasión. El juicio de la colectividad es más apropiado para alejar los peligros y eliminar la injusticia de las predisposiciones personales. Jesús siempre intentó reducir al mínimo los factores de injusticia, de represalias y de venganza.

\par 
%\textsuperscript{(1764.2)}
\textsuperscript{159:1.7} [La utilización del término setenta y siete, como ejemplo de la misericordia y la clemencia, fue extraído del pasaje de las Escrituras que alude al regocijo de Lamec ante las armas de metal de su hijo Tubal-Caín. Al comparar estos instrumentos superiores con los de sus enemigos, aquel exclamó: <<Si Caín, con ningún arma en la mano, fue vengado siete veces, yo seré vengado ahora setenta y siete veces>>.]

\section*{2. El predicador extranjero}
\par 
%\textsuperscript{(1764.3)}
\textsuperscript{159:2.1} Jesús fue a Gamala para visitar a Juan y a los que trabajaban con él en aquel lugar. Aquella noche, después de la sesión de preguntas y respuestas, Juan le dijo a Jesús: <<Maestro, ayer fui a Astarot para ver a un hombre que enseñaba en tu nombre y que incluso pretendía ser capaz de echar a los diablos. Pero este hombre nunca ha estado con nosotros, ni tampoco nos sigue; por consiguiente, le he prohibido hacer esas cosas>>. Jesús dijo entonces: <<No se lo prohíbas. ¿No percibes que este evangelio del reino pronto será proclamado en todo el mundo? ¿Cómo puedes esperar que todos los que crean en el evangelio van a estar sometidos a tu dirección? Regocíjate de que nuestras enseñanzas ya han empezado a manifestarse más allá de los límites de nuestra influencia personal. ¿No ves, Juan, que los que afirman hacer grandes obras en mi nombre acabarán por sostener nuestra causa? Sin duda no se darán prisa en hablar mal de mí. Hijo mío, en este tipo de cosas, sería mejor que consideraras que quien no está contra nosotros está a nuestro favor. En las generaciones por venir, muchos hombres no enteramente dignos harán muchas cosas extrañas en mi nombre, pero no se lo prohibiré. Te hago saber que, incluso cuando alguien da una simple copa de agua fría a un alma sedienta, los mensajeros del Padre siempre toman nota de ese servicio realizado por amor>>.

\par 
%\textsuperscript{(1764.4)}
\textsuperscript{159:2.2} Juan se quedó muy perplejo con esta enseñanza. ¿No había oído decir al Maestro que <<El que no está conmigo está contra mí?>> No percibía que, en aquel caso, Jesús se había referido a la relación personal del hombre con las enseñanzas espirituales del reino, mientras que en el caso presente, hacía referencia a las extensas relaciones sociales exteriores entre los creyentes respecto a las cuestiones del control administrativo y de la jurisdicción de un grupo de creyentes sobre el trabajo de otros grupos que acabarían por formar la fraternidad mundial venidera.

\par 
%\textsuperscript{(1765.1)}
\textsuperscript{159:2.3} Pero Juan refirió a menudo esta experiencia en conexión con sus trabajos posteriores a favor del reino. Sin embargo, los apóstoles se ofendieron muchas veces con aquellos que tenían la audacia de enseñar en nombre del Maestro. Siempre les pareció inadecuado que los que nunca se habían sentado a los pies de Jesús se atrevieran a enseñar en su nombre.

\par 
%\textsuperscript{(1765.2)}
\textsuperscript{159:2.4} El hombre a quien Juan le había prohibido enseñar y trabajar en nombre de Jesús no hizo caso de la orden del apóstol. Siguió adelante con sus esfuerzos y reunió en Canata a un grupo considerable de creyentes antes de proseguir hacia Mesopotamia. Este hombre, llamado Aden, había sido inducido a creer en Jesús gracias al testimonio del demente que Jesús había curado cerca de Jeresa, el cual creía con toda seguridad que los supuestos espíritus malignos que el Maestro había echado fuera de él habían entrado en la piara de cerdos y los habían despeñado por el acantilado hacia su destrucción.

\section*{3. Las instrucciones para los educadores y los creyentes}
\par 
%\textsuperscript{(1765.3)}
\textsuperscript{159:3.1} En Edrei, donde trabajaban Tomás y sus compañeros, Jesús pasó un día y una noche. En el transcurso de la discusión vespertina, expresó los principios que deberían guiar a los que predican la verdad e impulsar a todos los que enseñan el evangelio del reino. Resumido y expuesto de nuevo en un lenguaje moderno, he aquí lo que Jesús enseñó:

\par 
%\textsuperscript{(1765.4)}
\textsuperscript{159:3.2} Respetad siempre la personalidad del hombre. Una causa justa nunca se debe promover por la fuerza; las victorias espirituales sólo se pueden ganar por medio del poder espiritual. Esta orden en contra del empleo de las influencias materiales se refiere tanto a la fuerza psíquica como a la fuerza física. No se deben emplear los argumentos abrumadores ni la superioridad mental para coaccionar a los hombres y a las mujeres para que entren en el reino. La mente del hombre no debe ser aplastada con el solo peso de la lógica, ni intimidada con una elocuencia sagaz. Aunque la emoción, como factor en las decisiones humanas, no se puede eliminar por completo, los que quieran hacer progresar la causa del reino no deberían recurrir directamente a la emoción en sus enseñanzas. Apelad directamente al espíritu divino que reside en la mente de los hombres. No recurráis al miedo, a la lástima o al simple sentimiento. Cuando apeléis a los hombres, sed justos; ejerced el autocontrol y manifestad la debida compostura; mostrad un respeto adecuado por la personalidad de vuestros alumnos. Recordad que he dicho: <<Mirad, me detengo en la puerta y llamo, y si alguien quiere abrir, entraré>>.

\par 
%\textsuperscript{(1765.5)}
\textsuperscript{159:3.3} Cuando atraigáis a los hombres hacia el reino, no disminuyáis ni destruyáis su autoestima. Una autoestima excesiva puede destruir la humildad adecuada y terminar en orgullo, presunción y arrogancia, pero la pérdida de la autoestima acaba a menudo en la parálisis de la voluntad. Este evangelio tiene la finalidad de restablecer la autoestima en aquellos que la han perdido, y de refrenarla en los que la tienen. No cometáis el error de limitaros a condenar las equivocaciones que veáis en la vida de vuestros alumnos; recordad también que debéis reconocer generosamente las cosas más dignas de elogio que veáis en sus vidas. No olvidéis que no me detendré ante nada para restablecer la autoestima en aquellos que la han perdido, y que realmente desean recuperarla.

\par 
%\textsuperscript{(1765.6)}
\textsuperscript{159:3.4} Cuidad de no herir la autoestima de las almas tímidas y temerosas. No os permitáis ser sarcásticos a expensas de mis hermanos ingenuos. No seáis cínicos con mis hijos atormentados por el miedo. El desempleo destruye la autoestima; por lo tanto, recomendad a vuestros hermanos que se mantengan siempre ocupados en las tareas que han elegido, y que hagan todo tipo de esfuerzos por conseguirle un trabajo a aquellos que se encuentran sin empleo.

\par 
%\textsuperscript{(1766.1)}
\textsuperscript{159:3.5} No seáis nunca culpables de utilizar tácticas indignas como la de intentar asustar a los hombres y a las mujeres para que entren en el reino. Un padre amoroso no asusta a sus hijos para hacer que obedezcan sus justas exigencias.

\par 
%\textsuperscript{(1766.2)}
\textsuperscript{159:3.6} Los hijos del reino comprenderán alguna vez que las fuertes sensaciones emotivas no equivalen a las directrices del espíritu divino. Cuando una impresión fuerte y extraña os impulsa a hacer algo o a ir a cierto lugar, eso no significa necesariamente que tales impulsos sean las directrices del espíritu interior.

\par 
%\textsuperscript{(1766.3)}
\textsuperscript{159:3.7} Advertid a todos los creyentes acerca de la zona de conflicto que tendrán que atravesar todos aquellos que pasan de la vida que se vive en la carne a la vida superior que se vive en el espíritu. Para los que viven plenamente en uno de los dos reinos, existe poco conflicto o confusión, pero todos están destinados a experimentar un mayor o menor grado de incertidumbre durante el período de transición entre los dos niveles de vida. Cuando entráis en el reino, no podéis eludir sus responsabilidades ni evitar sus obligaciones, pero recordad que el yugo del evangelio es cómodo y que el peso de la verdad es ligero.

\par 
%\textsuperscript{(1766.4)}
\textsuperscript{159:3.8} El mundo está lleno de almas hambrientas que se mueren de hambre delante mismo del pan de la vida; los hombres se mueren buscando al mismo Dios que vive dentro de ellos. Los hombres buscan los tesoros del reino con un corazón anhelante y unos pasos cansados, cuando todos se encuentran al alcance inmediato de la fe viviente. La fe es para la religión lo que las velas para un barco; es un aumento de poder, no una carga adicional de la vida. Sólo hay una lucha que tienen que sostener los que entran en el reino, y es el buen combate de la fe. El creyente sólo tiene que librar una batalla, y es contra la duda ---contra la incredulidad.

\par 
%\textsuperscript{(1766.5)}
\textsuperscript{159:3.9} Cuando prediquéis el evangelio del reino, estaréis enseñando simplemente la amistad con Dios. Y esta comunión atraerá por igual a los hombres y a las mujeres, en el sentido de que ambos encontrarán en ella lo que satisface de manera más efectiva sus anhelos e ideales característicos. Decid a mis hijos que no solamente soy sensible a sus sentimientos y paciente con sus debilidades, sino que también soy despiadado con el pecado e intolerante con la iniquidad. En verdad, soy manso y humilde en presencia de mi Padre, pero también soy implacablemente inexorable cuando hay una acción malvada deliberada y una rebelión pecaminosa contra la voluntad de mi Padre que está en los cielos.

\par 
%\textsuperscript{(1766.6)}
\textsuperscript{159:3.10} No describáis a vuestro maestro como un hombre de tristezas. Las generaciones futuras deberán conocer también el esplendor de nuestra alegría, el optimismo de nuestra buena voluntad, y la inspiración de nuestro buen humor. Proclamamos un mensaje de buenas noticias, cuyo poder transformador es contagioso. Nuestra religión palpita con una nueva vida y unos nuevos significados. Los que aceptan esta enseñanza se llenan de alegría, y su corazón les obliga a regocijarse para siempre jamás. Todos los que están seguros acerca de Dios experimentan siempre una felicidad creciente.

\par 
%\textsuperscript{(1766.7)}
\textsuperscript{159:3.11} Enseñad a todos los creyentes que eviten apoyarse en los soportes inseguros de la falsa compasión. No podéis desarrollar un carácter fuerte si tenéis inclinación por la autocompasión; esforzaos honradamente por evitar la influencia engañosa de la simple comunión en la desdicha. Conceded vuestra simpatía a los valientes y a los intrépidos, sin ofrecer un exceso de compasión a aquellas almas cobardes que se limitan a levantarse sin entusiasmo ante las pruebas de la vida. No ofrezcáis vuestro consuelo a los que se tumban ante las dificultades, sin luchar. No simpaticéis con vuestros semejantes con la única finalidad de recibir a cambio su simpatía.

\par 
%\textsuperscript{(1766.8)}
\textsuperscript{159:3.12} Una vez que mis hijos se hagan conscientes de la certeza de la presencia divina, esa fe abrirá su mente, ennoblecerá su alma, fortalecerá su personalidad, aumentará su felicidad, intensificará su percepción espiritual y realzará su poder para amar y ser amados.

\par 
%\textsuperscript{(1767.1)}
\textsuperscript{159:3.13} Enseñad a todos los creyentes que el hecho de entrar en el reino no los inmuniza contra los accidentes del tiempo ni las catástrofes ordinarias de la naturaleza. La creencia en el evangelio no impedirá que tengáis dificultades, pero sí asegurará que \textit{no tendréis miedo} cuando se presenten las dificultades. Si os atrevéis a creer en mí y empezáis a seguirme de todo corazón, al hacerlo os meteréis con toda seguridad en el camino preciso que lleva a las dificultades. No os prometo liberaros de las aguas de la adversidad, pero lo que sí os prometo es atravesarlas todas con vosotros.

\par 
%\textsuperscript{(1767.2)}
\textsuperscript{159:3.14} Jesús enseñó muchas más cosas a este grupo de creyentes antes de que se prepararan para el descanso nocturno. Aquellos que habían escuchado estas palabras las atesoraron en su corazón y las repitieron a menudo para edificar a los apóstoles y discípulos que no estaban presentes cuando fueron pronunciadas.

\section*{4. La conversación con Natanael}
\par 
%\textsuperscript{(1767.3)}
\textsuperscript{159:4.1} Jesús se desplazó entonces a Abila, donde trabajaban Natanael y sus compañeros. Natanael estaba muy confundido por algunas declaraciones de Jesús que parecían disminuir la autoridad de las escrituras hebreas reconocidas. En consecuencia, aquella noche, después de la sesión habitual de preguntas y respuestas, Natanael apartó a Jesús de los demás y le preguntó: <<Maestro, ¿podrías confiar en mí como para hacerme saber la verdad sobre las Escrituras? Observo que nos enseñas solamente una parte de las escrituras sagradas ---la mejor en mi opinión--- y deduzco que rechazas las enseñanzas rabínicas que afirman que las palabras de la ley son las palabras mismas de Dios, que estaban con Dios en el cielo incluso antes de la época de Abraham y Moisés. ¿Cuál es la verdad sobre las Escrituras?>> Cuando Jesús escuchó la pregunta de su apóstol desconcertado, respondió:

\par 
%\textsuperscript{(1767.4)}
\textsuperscript{159:4.2} <<Natanael, has juzgado bien; yo no considero las Escrituras como lo hacen los rabinos. Hablaré contigo de este asunto a condición de que no comentes estas cosas con tus hermanos, porque no todos están preparados para recibir esta enseñanza. Las palabras de la ley de Moisés y las enseñanzas de las Escrituras no existían antes de Abraham. Las Escrituras han sido reunidas en una época reciente bajo la forma que las poseemos ahora. Aunque contienen lo mejor de las ideas y los anhelos más elevados del pueblo judío, también contienen muchas cosas que están lejos de representar el carácter y las enseñanzas del Padre que está en los cielos; por eso tengo que escoger, entre las mejores enseñanzas, aquellas verdades que han de ser extraídas para el evangelio del reino>>.

\par 
%\textsuperscript{(1767.5)}
\textsuperscript{159:4.3} <<Estos escritos son obras de los hombres, algunos de ellos santos y otros no tan santos. Las enseñanzas de estos libros representan los puntos de vista y el grado de iluminación de la época en que se originaron. Como revelación de la verdad, los últimos libros son más dignos de confianza que los primeros. Las Escrituras son defectuosas y su origen es enteramente humano, pero no te equivoques, pues constituyen la mejor recopilación de sabiduría religiosa y de verdad espiritual que se puede encontrar actualmente en el mundo entero>>.

\par 
%\textsuperscript{(1767.6)}
\textsuperscript{159:4.4} <<Muchos de estos libros no fueron escritos por las personas cuyos nombres figuran en ellos, pero eso no disminuye en nada el valor de las verdades que contienen. Aunque la historia de Jonás no fuera un hecho, e incluso si Jonás nunca hubiera existido, la profunda verdad de este relato ---el amor de Dios por Nínive y por los supuestos paganos--- no sería por ello menos preciosa a los ojos de todos aquellos que aman a sus semejantes. Las Escrituras son sagradas porque exponen los pensamientos y los actos de los hombres que buscaban a Dios, y que dejaron en estos escritos sus conceptos más elevados sobre la rectitud, la verdad y la santidad. Las Escrituras contienen muchas, muchísimas cosas que son verdaderas, pero a la luz de la enseñanza que estás recibiendo, sabes que estos escritos contienen también muchas cosas que desfiguran la imagen del Padre que está en los cielos, el Dios amoroso que he venido a revelar a todos los mundos>>.

\par 
%\textsuperscript{(1768.1)}
\textsuperscript{159:4.5} <<Natanael, nunca te permitas creer ni un instante en los relatos de las Escrituras que te dicen que el Dios del amor ordenó a tus antepasados que salieran a luchar para matar a todos sus enemigos ---hombres, mujeres y niños. Esos documentos son palabras de hombres, de hombres no muy santos, pero no son la palabra de Dios. Las Escrituras siempre han reflejado, y reflejarán siempre, el estado intelectual, moral y espiritual de sus autores. ¿No has observado que los conceptos de Yahvé crecen en belleza y en gloria a medida que los profetas elaboran sus escritos, desde Samuel hasta Isaías? Y deberías recordar que las Escrituras están destinadas a la instrucción religiosa y a la orientación espiritual. No son la obra de unos historiadores ni de unos filósofos>>.

\par 
%\textsuperscript{(1768.2)}
\textsuperscript{159:4.6} <<La cosa más deplorable no es simplemente esa idea errónea de que los relatos de las Escrituras son absolutamente perfectos y que sus enseñanzas son infalibles, sino más bien la mala interpretación confusa que los escribas y fariseos de Jerusalén, esclavizados por la tradición, hacen de estos escritos sagrados. Y ahora, en sus esfuerzos resueltos por contrarrestar las enseñanzas más modernas del evangelio del reino, van a emplear tanto la doctrina de que las Escrituras son inspiradas como las falsas interpretaciones que hacen de ellas. Natanael, no lo olvides nunca: el Padre no limita la revelación de la verdad a una generación concreta ni a un pueblo determinado. Muchos buscadores ardientes de la verdad se han sentido, y continuarán sintiéndose confundidos y desanimados debido a estas doctrinas de la perfección de las Escrituras>>.

\par 
%\textsuperscript{(1768.3)}
\textsuperscript{159:4.7} <<La autoridad de la verdad es el espíritu mismo que reside en sus manifestaciones vivientes, y no las palabras muertas de los hombres de otra generación, menos iluminados y supuestamente inspirados. Y aunque esos santos antiguos vivieran unas vidas inspiradas y repletas de espíritu, eso no significa que sus \textit{palabras} estuvieran igualmente inspiradas por el espíritu. Actualmente no ponemos por escrito las enseñanzas de este evangelio del reino, por temor a que después de mi partida, os dividáis rápidamente en varios grupos que compitan por la verdad a consecuencia de vuestras diversas interpretaciones de mis enseñanzas. Para esta generación, es mejor que \textit{vivamos} estas verdades, evitando ponerlas por escrito>>.

\par 
%\textsuperscript{(1768.4)}
\textsuperscript{159:4.8} <<Toma buena nota de mis palabras, Natanael: nada de lo que la naturaleza humana ha tocado puede ser considerado como infalible. Es cierto que la verdad divina puede brillar a través de la mente humana, pero siempre con una pureza relativa y una divinidad parcial. La criatura puede desear ardientemente la infalibilidad, pero sólo los Creadores la poseen>>.

\par 
%\textsuperscript{(1768.5)}
\textsuperscript{159:4.9} <<Pero el error más grande de la enseñanza acerca de las Escrituras consiste en la doctrina que las presenta como libros herméticos de misterio y de sabiduría, que sólo los sabios de la nación se atreven a interpretar. Las revelaciones de la verdad divina no están precintadas, salvo por la ignorancia humana, la beatería y la intolerancia mezquina. La luz de las Escrituras sólo está empañada por los prejuicios y oscurecida por la superstición. Un falso miedo a lo sagrado ha impedido que el sentido común salvaguarde la religión. El miedo a la autoridad de los escritos sagrados del pasado impide eficazmente que las almas honradas de hoy acepten la nueva luz del evangelio, una luz que anhelaron ver con tanta intensidad aquellos mismos hombres que conocieron a Dios en generaciones anteriores>>.

\par 
%\textsuperscript{(1769.1)}
\textsuperscript{159:4.10} <<Pero lo más triste de todo esto es el hecho de que algunos de los que enseñan la santidad de este tradicionalismo conocen esta misma verdad. Comprenden más o menos plenamente estas limitaciones de las Escrituras, pero son moralmente cobardes e intelectualmente deshonestos. Conocen la verdad acerca de los escritos sagrados, pero prefieren ocultarle al pueblo estos hechos perturbadores. Y así desnaturalizan y tergiversan las Escrituras, convirtiéndolas en una guía para los detalles serviles de la vida diaria, y en una autoridad para las cosas no espirituales, en lugar de recurrir a los escritos sagrados como depósito de la sabiduría moral, la inspiración religiosa y la enseñanza espiritual de los hombres que conocieron a Dios en las generaciones pasadas>>.

\par 
%\textsuperscript{(1769.2)}
\textsuperscript{159:4.11} Natanael se sintió iluminado, y conmocionado, por las declaraciones del Maestro. Reflexionó largamente, en las profundidades de su alma, sobre esta conversación, pero no le habló a nadie acerca de este diálogo hasta después de la ascensión de Jesús; e incluso entonces, temió dar a conocer la historia completa de la enseñanza del Maestro.

\section*{5. La naturaleza positiva de la religión de Jesús}
\par 
%\textsuperscript{(1769.3)}
\textsuperscript{159:5.1} En Filadelfia, donde Santiago estaba trabajando, Jesús enseñó a los discípulos acerca de la naturaleza positiva del evangelio del reino. En el transcurso de sus comentarios, insinuó que algunas partes de las Escrituras contenían más verdades que otras, y recomendó a sus oyentes que alimentaran su alma con el mejor alimento espiritual. Santiago interrumpió al Maestro para preguntarle: <<Maestro, ¿tendrías la bondad de sugerirnos cómo podemos escoger los mejores pasajes de las Escrituras para nuestra edificación personal?>> Y Jesús replicó: <<Sí, Santiago; cuando leáis las Escrituras, buscad las enseñanzas eternamente verdaderas y divinamente hermosas, tales como:>>

\par 
%\textsuperscript{(1769.4)}
\textsuperscript{159:5.2} <<Crea en mi, Oh Señor, un corazón limpio>>.

\par 
%\textsuperscript{(1769.5)}
\textsuperscript{159:5.3} <<El Señor es mi pastor; nada me faltará>>.

\par 
%\textsuperscript{(1769.6)}
\textsuperscript{159:5.4} <<Deberías amar a tu prójimo como a ti mismo>>.

\par 
%\textsuperscript{(1769.7)}
\textsuperscript{159:5.5} <<Porque yo, el Señor tu Dios, sostendré tu mano derecha, diciendo: no temas; yo te ayudaré>>.

\par 
%\textsuperscript{(1769.8)}
\textsuperscript{159:5.6} <<Las naciones ya no aprenderán a hacer la guerra>>.

\par 
%\textsuperscript{(1769.9)}
\textsuperscript{159:5.7} Esto ilustra la manera en que Jesús, día tras día, se apropiaba de lo mejor que tenían las Escrituras hebreas para instruir a sus discípulos y para incluirlo en las enseñanzas del nuevo evangelio del reino. Otras religiones habían sugerido la idea de que Dios estaba cerca del hombre, pero Jesús equiparó la preocupación de Dios por el hombre al afán de un padre amoroso por el bienestar de sus hijos que dependen de él, y luego convirtió esta enseñanza en la piedra angular de su religión. Y así la doctrina de la paternidad de Dios hizo imperativa la práctica de la fraternidad de los hombres. La adoración de Dios y el servicio del hombre se convirtieron en la suma y la sustancia de su religión. Jesús cogió lo mejor de la religión judía y lo transfirió al digno marco de las nuevas enseñanzas del evangelio del reino.

\par 
%\textsuperscript{(1769.10)}
\textsuperscript{159:5.8} Jesús introdujo el espíritu de la acción positiva en las doctrinas pasivas de la religión judía. En lugar de una obediencia negativa a las exigencias ceremoniales, Jesús prescribió la ejecución positiva de lo que su nueva religión exigía a los que la aceptaban. La religión de Jesús no consistía simplemente en \textit{creer}, sino en \textit{hacer} realmente las cosas que exigía el evangelio. No enseñó que la esencia de su religión consistiera en el servicio social, sino más bien que el servicio social era uno de los efectos seguros de la posesión del espíritu de la verdadera religión.

\par 
%\textsuperscript{(1770.1)}
\textsuperscript{159:5.9} Jesús no dudó en apropiarse de la mejor mitad de un pasaje de las Escrituras, rechazando la parte menos interesante. Su gran exhortación <<Ama a tu prójimo como a ti mismo>> la cogió del pasaje de las Escrituras que dice: <<No te vengarás de los hijos de tu pueblo, sino que amarás a tu prójimo como a ti mismo>>. Jesús se apropió de la parte positiva de este extracto, y rechazó la parte negativa. Incluso llegó a oponerse a la no resistencia negativa o puramente pasiva. Dijo: <<Si un enemigo te golpea en una mejilla, no te quedes allí mudo y pasivo, sino que adopta una actitud positiva y ofrécele la otra; es decir, haz activamente todo lo posible por sacar del mal camino a tu hermano equivocado, y llevarlo hacia los mejores senderos de una vida recta>>. Jesús pedía a sus seguidores que reaccionaran de una manera positiva y dinámica en todas las situaciones de la vida. El hecho de ofrecer la otra mejilla, o cualquier otro acto semejante, exige iniciativa y requiere una expresión vigorosa, activa y valiente de la personalidad del creyente.

\par 
%\textsuperscript{(1770.2)}
\textsuperscript{159:5.10} Jesús no defendía la práctica de someterse negativamente a los ultrajes de aquellos que intentan engañar adrede a los que practican la no resistencia ante el mal, sino más bien que sus seguidores fueran sabios y despiertos en sus reacciones rápidas y positivas a favor del bien y en contra del mal, a fin de que pudieran vencer eficazmente el mal por medio del bien. No olvidéis que el verdadero bien es invariablemente más poderoso que el mal más nocivo. El Maestro enseñó una norma positiva de rectitud: <<Si alguien desea ser mi discípulo, que no haga caso de sí mismo y que asuma diariamente la totalidad de sus responsabilidades para seguirme>>. Él mismo vivió de esta manera, en el sentido de que <<iba de un sitio para otro haciendo el bien>>. Este aspecto del evangelio estuvo bien ilustrado en las numerosas parábolas que más adelante contó a sus seguidores. Nunca exhortó a sus discípulos a que soportaran pacientemente sus obligaciones, sino más bien a que vivieran con energía y entusiasmo la totalidad de sus responsabilidades humanas y de sus privilegios divinos en el reino de Dios.

\par 
%\textsuperscript{(1770.3)}
\textsuperscript{159:5.11} Cuando Jesús enseñó a sus apóstoles que si alguien les quitaba injustamente el abrigo, ofrecieran su otro vestido, no se refería literalmente a un segundo abrigo, sino más bien a la idea de hacer algo \textit{positivo} para salvar al malhechor, en lugar de seguir el antiguo consejo de pagar con la misma moneda ---<<ojo por ojo>> y así sucesivamente. Jesús aborrecía la idea de las represalias y la de convertirse en un simple sufridor pasivo o en una víctima de la injusticia. En esta ocasión, les enseñó las tres maneras de luchar contra el mal y de oponerse a él:

\par 
%\textsuperscript{(1770.4)}
\textsuperscript{159:5.12} 1. Devolver el mal por el mal ---el método positivo pero injusto.

\par 
%\textsuperscript{(1770.5)}
\textsuperscript{159:5.13} 2. Soportar el mal sin quejarse ni resistirse ---el método puramente negativo.

\par 
%\textsuperscript{(1770.6)}
\textsuperscript{159:5.14} 3. Devolver el bien por el mal, afirmar la voluntad para volverse el dueño de la situación, vencer al mal con el bien ---el método positivo y justo.

\par 
%\textsuperscript{(1770.7)}
\textsuperscript{159:5.15} Uno de los apóstoles preguntó una vez: <<Maestro, ¿qué debería hacer si un extranjero me forzara a llevar su carga durante una milla?>> Jesús contestó: <<No te sientes y sueltes un suspiro de alivio, mientras reprendes al extranjero en voz baja. La rectitud no proviene de esas actitudes pasivas. Si no se te ocurre hacer nada más positivo y eficaz, al menos puedes llevar la carga una segunda milla. Es seguro que eso desafiará al extranjero injusto e impío>>.

\par 
%\textsuperscript{(1770.8)}
\textsuperscript{159:5.16} Los judíos habían oído hablar de un Dios que estaba dispuesto a perdonar a los pecadores arrepentidos y a intentar olvidar sus transgresiones, pero hasta que vino Jesús, los hombres nunca habían oído hablar de un Dios que fuera en busca de las ovejas perdidas, que tomara la iniciativa de buscar a los pecadores, y que se regocijara cuando los encontraba dispuestos a regresar a la casa del Padre. Jesús extendió esta nota positiva de la religión incluso a sus oraciones. Y convirtió la regla de oro negativa en una exhortación positiva de equidad humana.

\par 
%\textsuperscript{(1771.1)}
\textsuperscript{159:5.17} En toda su enseñanza, Jesús evitaba indefectiblemente los detalles que distraían la atención. Esquivaba el lenguaje florido y eludía las simples imágenes poéticas de los juegos de palabras. Habitualmente introducía grandes significados en expresiones sencillas. Jesús invertía, con fines ilustrativos, el significado corriente de muchos términos tales como sal, levadura, pesca y niños pequeños. Empleaba la antítesis de la manera más eficaz, comparando lo pequeño con lo infinito, y así sucesivamente. Sus descripciones eran sorprendentes, como por ejemplo <<el ciego que conduce al ciego>>. Pero la fuerza más grande de su enseñanza ilustrativa se encontraba en su naturalidad. Jesús trajo la filosofía de la religión desde el cielo a la Tierra. Describía las necesidades elementales del alma con una nueva perspicacia y una nueva donación de afecto.

\section*{6. El regreso a Magadán}
\par 
%\textsuperscript{(1771.2)}
\textsuperscript{159:6.1} La misión de cuatro semanas en la Decápolis tuvo un éxito moderado. Cientos de almas fueron recibidas en el reino, y los apóstoles y los evangelistas adquirieron una valiosa experiencia al tener que continuar su trabajo sin la inspiración de la presencia personal inmediata de Jesús.

\par 
%\textsuperscript{(1771.3)}
\textsuperscript{159:6.2} El viernes 16 de septiembre, todo el cuerpo de evangelizadores se congregó en el parque de Magadán tal como habían convenido de antemano. El día del sábado, más de cien creyentes celebraron un consejo en el que se consideraron a fondo los planes futuros para ampliar el trabajo del reino. Los mensajeros de David estuvieron presentes e informaron sobre el bienestar de los creyentes en Judea, Samaria, Galilea y las regiones adyacentes.

\par 
%\textsuperscript{(1771.4)}
\textsuperscript{159:6.3} En esta época, pocos seguidores de Jesús apreciaban plenamente el gran valor de los servicios que efectuaba el cuerpo de mensajeros. Los mensajeros no solamente mantenían en contacto a los creyentes, por toda Palestina, entre ellos y con Jesús y los apóstoles, sino que durante estos días sombríos, también servían como recaudadores de fondos, no sólo para el mantenimiento de Jesús y sus compañeros, sino también para ayudar a las familias de los doce apóstoles y de los doce evangelistas.

\par 
%\textsuperscript{(1771.5)}
\textsuperscript{159:6.4} Aproximadamente por esta época, Abner trasladó su centro de operaciones de Hebrón a Belén; esta última ciudad era también el cuartel general de los mensajeros de David en Judea. David mantenía un servicio de mensajeros de relevo durante la noche entre Jerusalén y Betsaida. Estos corredores salían de Jerusalén todas las tardes, se relevaban en Sicar y Escitópolis, y llegaban a Betsaida a la hora del desayuno de la mañana siguiente.

\par 
%\textsuperscript{(1771.6)}
\textsuperscript{159:6.5} Jesús y sus compañeros se dispusieron ahora a tomar una semana de descanso, antes de prepararse para iniciar la última época de sus trabajos a favor del reino. Éste fue su último período de descanso, porque la misión en Perea se convirtió en una campaña de predicación y de enseñanza que se prolongó hasta el momento de su llegada a Jerusalén y de la representación de los episodios finales de la carrera terrestre de Jesús.


\chapter{Documento 160. Rodán de Alejandría}
\par 
%\textsuperscript{(1772.1)}
\textsuperscript{160:0.1} EL DOMINGO 18 de septiembre por la mañana, Andrés anunció que no se planearía ningún trabajo para la semana siguiente. Todos los apóstoles, excepto Natanael y Tomás, fueron a sus casas para visitar a sus familias o permanecer con sus amigos. Esta semana, Jesús disfrutó de un período de descanso casi completo, pero Natanael y Tomás estuvieron muy ocupados discutiendo con cierto filósofo griego de Alejandría llamado Rodán. Este griego se había hecho recientemente discípulo de Jesús gracias a las enseñanzas de uno de los asociados de Abner, que había dirigido una misión en Alejandría. Rodán estaba ahora seriamente ocupado en la tarea de armonizar su filosofía de la vida con las nuevas enseñanzas religiosas de Jesús, y había venido a Magadán con la esperanza de que el Maestro discutiera estos problemas con él. También deseaba obtener una versión autorizada y de primera mano del evangelio, ya fuera de Jesús o de uno de sus apóstoles. Aunque el Maestro declinó participar en este tipo de conversaciones con Rodán, lo recibió amablemente y ordenó de inmediato que Natanael y Tomás escucharan todo lo que tenía que decir, y que a su vez le hablaran sobre el evangelio.

\section*{1. La filosofía griega de Rodán}
\par 
%\textsuperscript{(1772.2)}
\textsuperscript{160:1.1} El lunes por la mañana temprano, Rodán comenzó una serie de diez discursos para Natanael, Tomás y un grupo de unas dos docenas de creyentes que se encontraban casualmente en Magadán. Estas conversaciones, condensadas, combinadas y reexpuestas en un lenguaje moderno, ofrecen para su estudio los pensamientos siguientes:

\par 
%\textsuperscript{(1772.3)}
\textsuperscript{160:1.2} La vida humana consiste en tres grandes estímulos: los impulsos, los deseos y los alicientes. Un carácter fuerte, una personalidad con autoridad, sólo se puede adquirir convirtiendo el impulso natural de la vida en el arte social de vivir, transformando los deseos inmediatos en esos anhelos elevados que son capaces de logros duraderos, mientras que el aliciente común de la existencia debemos transferirlo desde las ideas personales, convencionales y establecidas, hasta los niveles más elevados de las ideas no exploradas y de los ideales por descubrir.

\par 
%\textsuperscript{(1772.4)}
\textsuperscript{160:1.3} Cuanto más compleja se vuelva la civilización, más difícil será el arte de vivir. Cuanto más rápidamente cambien los usos sociales, más complicada será la tarea de desarrollar el carácter. Para que el progreso pueda continuar, la humanidad tiene que aprender de nuevo el arte de vivir cada diez generaciones. Y si el hombre se vuelve tan ingenioso que aumenta con más rapidez las complejidades de la sociedad, el arte de vivir tendrá que ser dominado de nuevo en menos tiempo, quizás en cada generación. Si la evolución del arte de vivir no logra seguir el mismo ritmo que la técnica de la existencia, la humanidad retrocederá rápidamente al simple impulso de vivir ---a la satisfacción de los deseos inmediatos. De esta manera, la humanidad seguirá siendo inmadura; la sociedad no logrará desarrollarse hasta su plena madurez.

\par 
%\textsuperscript{(1773.1)}
\textsuperscript{160:1.4} La madurez social es equivalente al grado en que el hombre está dispuesto a renunciar a satisfacer sus meros deseos pasajeros e inmediatos, para mantener esos anhelos superiores cuya obtención, por medio del esfuerzo, proporciona las satisfacciones más abundantes del avance progresivo hacia objetivos permanentes. Pero el verdadero distintivo de la madurez social es la buena voluntad de un pueblo para renunciar al derecho de vivir satisfecho y en paz bajo las normas que promueven la facilidad, basadas en el aliciente de las creencias establecidas y de las ideas convencionales, para perseguir el aliciente inquietante, y que necesita energía, de las posibilidades inexploradas de alcanzar los objetivos no descubiertos de las realidades espirituales idealistas.

\par 
%\textsuperscript{(1773.2)}
\textsuperscript{160:1.5} Los animales reaccionan noblemente al impulso de la vida, pero sólo el hombre puede alcanzar el arte de vivir, aunque la mayoría de la humanidad sólo experimenta el impulso animal de vivir. Los animales no conocen más que este impulso ciego e instintivo; el hombre es capaz de trascender este impulso que le incita al funcionamiento natural. El hombre puede decidir vivir en el plano elevado del arte inteligente, e incluso en el plano de la alegría celestial y del éxtasis espiritual. Los animales no se preguntan por el propósito de la vida; por eso nunca se preocupan ni tampoco se suicidan. Entre los hombres, el suicidio demuestra que estos seres han sobrepasado el estado puramente animal de la existencia, y el hecho adicional de que los esfuerzos exploratorios de tales seres humanos no han logrado alcanzar los niveles en que la experiencia mortal se vuelve un arte. Los animales no conocen el significado de la vida; el hombre no sólo posee la capacidad de reconocer los valores y de comprender los significados, sino que también tiene conciencia del significado de los significados ---es consciente de su propia perspicacia.

\par 
%\textsuperscript{(1773.3)}
\textsuperscript{160:1.6} Cuando los hombres se atreven a abandonar una vida de intensos deseos naturales a favor de un arte de vivir arriesgado y de una lógica incierta, deben contar con soportar los riesgos correspondientes de los accidentes emocionales ---conflictos, infelicidad e incertidumbres--- al menos hasta el momento en que alcanzan cierto grado de madurez intelectual y emocional. El desaliento, la preocupación y la indolencia son una prueba evidente de la inmadurez moral. La sociedad humana se enfrenta con dos problemas: alcanzar la madurez por parte del individuo, y alcanzar la madurez por parte de la raza. El ser humano maduro empieza pronto a mirar a todos los demás mortales con sentimientos de ternura y con emociones de tolerancia. Los hombres maduros perciben a sus compañeros inmaduros con el amor y la consideración que los padres tienen por sus hijos.

\par 
%\textsuperscript{(1773.4)}
\textsuperscript{160:1.7} El éxito en la vida no es ni más ni menos que el arte de dominar las técnicas fiables para solucionar los problemas ordinarios. El primer paso para solucionar un problema cualquiera consiste en localizar la dificultad, aislar el problema y reconocer francamente su naturaleza y su gravedad. Cuando los problemas de la vida despiertan nuestros temores profundos, cometemos el gran error de negarnos a reconocerlos. Asimismo, cuando reconocer nuestras dificultades implica reducir nuestra vanidad largamente acariciada, admitir que somos envidiosos, o abandonar unos prejuicios profundamente arraigados, la persona de tipo medio prefiere aferrarse a sus viejas ilusiones de seguridad y a sus falsas sensaciones de estabilidad largo tiempo cultivadas. Sólo una persona valiente está dispuesta a admitir honradamente aquello que descubre una mente sincera y lógica, y a enfrentarse a ello sin temor.

\par 
%\textsuperscript{(1773.5)}
\textsuperscript{160:1.8} Para solucionar de manera sabia y eficaz cualquier problema, se necesita una mente libre de inclinaciones, de pasiones y de cualquier otro prejuicio puramente personal que pueda interferir con el análisis imparcial de los factores reales que juntos constituyen el problema que se ha presentado para ser resuelto. La solución de los problemas de la vida requiere valentía y sinceridad. Sólo las personas honradas y valientes son capaces de continuar valerosamente su camino a través del laberinto confuso y desconcertante de la vida al que pueda llevarles la lógica de una mente intrépida. Esta emancipación de la mente y del alma nunca puede producirse sin el poder impulsor de un entusiasmo inteligente que roza el fervor religioso. Se necesita el atractivo de un gran ideal para impulsar al hombre en pos de un objetivo rodeado de problemas materiales difíciles y de riesgos intelectuales múltiples.

\par 
%\textsuperscript{(1774.1)}
\textsuperscript{160:1.9} Aunque estéis eficazmente preparados para afrontar las situaciones difíciles de la vida, no podéis esperar mucho éxito a menos que estéis provistos de esa sabiduría de la mente y de ese encanto de la personalidad que os permita conseguir el apoyo y la cooperación sincera de vuestros semejantes. Tanto en el trabajo seglar como en el trabajo religioso, no podéis esperar mucho éxito a menos que aprendáis a persuadir a vuestros semejantes, a convencer a los hombres. Simplemente debéis de tener tacto y tolerancia.

\par 
%\textsuperscript{(1774.2)}
\textsuperscript{160:1.10} Pero el mejor de todos los métodos para solucionar los problemas lo he aprendido de Jesús, vuestro Maestro. Me refiero a lo que él practica con tanta perseverancia, y que tan fielmente os ha enseñado: la meditación adoradora en solitario. En esta costumbre que tiene Jesús de apartarse con tanta frecuencia para comulgar con el Padre que está en los cielos, se encuentra la técnica, no sólo para acumular las fuerzas y la sabiduría necesarias para los conflictos ordinarios de la vida, sino también para apropiarse de la energía necesaria para resolver los problemas más elevados de naturaleza moral y espiritual. Pero incluso los métodos correctos para solucionar los problemas no compensan los defectos inherentes a la personalidad, ni reparan la ausencia de hambre y de sed de verdadera rectitud.

\par 
%\textsuperscript{(1774.3)}
\textsuperscript{160:1.11} Me impresiona profundamente la costumbre de Jesús de retirarse a solas para emprender esos períodos de examen solitario de los problemas de la vida; para buscar nuevas reservas de sabiduría y de energía para poder enfrentarse a las múltiples exigencias del servicio social; para vivificar y hacer más profundo el propósito supremo de la vida, sometiendo realmente su personalidad total a la conciencia del contacto con la divinidad; para tratar de conseguir métodos nuevos y mejores para adaptarse a las situaciones siempre cambiantes de la existencia viviente; para efectuar esas reconstrucciones y reajustes vitales de las actitudes personales, que son tan esenciales para comprender mejor todo lo que es válido y real. Y hacer todo esto con miras a la sola gloria de Dios ---decir sinceramente la oración favorita de vuestro Maestro: <<Que se haga, no mi voluntad, sino la tuya>>.

\par 
%\textsuperscript{(1774.4)}
\textsuperscript{160:1.12} Esta práctica de adoración de vuestro Maestro aporta ese descanso que renueva la mente, esa iluminación que inspira el alma, ese valor que permite enfrentarse valientemente con los problemas de uno mismo, esa comprensión de sí mismo que elimina el temor debilitante, y esa conciencia de la unión con la divinidad que equipa al hombre con la seguridad que le permite atreverse a ser como Dios. El descanso de la adoración, o comunión espiritual, tal como la practica el Maestro, alivia la tensión, elimina los conflictos y aumenta poderosamente los recursos totales de la personalidad. Y toda esta filosofía, más el evangelio del reino, constituyen la nueva religión tal como yo la comprendo.

\par 
%\textsuperscript{(1774.5)}
\textsuperscript{160:1.13} Los prejuicios ciegan el alma impidiéndole reconocer la verdad, y los prejuicios sólo se pueden eliminar mediante la devoción sincera del alma a la adoración de una causa que abarque e incluya a todos nuestros semejantes humanos. Los prejuicios están inseparablemente vinculados con el egoísmo. Los prejuicios sólo se pueden suprimir abandonando el egocentrismo y reemplazándolo por la búsqueda de la satisfacción de servir a una causa que sea no sólo más grande que uno mismo, sino incluso más grande que toda la humanidad ---la búsqueda de Dios, la adquisición de la divinidad. La prueba de la madurez de la personalidad consiste en la transformación de los deseos humanos de tal manera que busquen constantemente la comprensión de los valores más elevados y más divinamente reales.

\par 
%\textsuperscript{(1774.6)}
\textsuperscript{160:1.14} En un mundo que cambia continuamente, en medio de un orden social en evolución, es imposible mantener unas metas de destino establecidas y asentadas. Sólo pueden experimentar la estabilidad de la personalidad aquellos que han descubierto y abrazado al Dios viviente como meta eterna de consecución infinita. Para transferir así la meta individual del tiempo a la eternidad, de la Tierra al Paraíso, de lo humano a lo divino, es necesario que el hombre se regenere, se convierta, nazca de nuevo, que se vuelva el hijo re-creado del espíritu divino, que logre su entrada en la fraternidad del reino de los cielos. Todas las filosofías y religiones que estén por debajo de estos ideales son inmaduras. La filosofía que yo enseño, unida al evangelio que vosotros predicáis, representa la nueva religión de la madurez, el ideal de todas las generaciones futuras. Y esto es verdad porque nuestro ideal es definitivo, infalible, eterno, universal, absoluto e infinito.

\par 
%\textsuperscript{(1775.1)}
\textsuperscript{160:1.15} Mi filosofía me ha impulsado a buscar las realidades de la consecución verdadera, la meta de la madurez. Pero mi impulso era impotente, mi búsqueda carecía de fuerza motriz, mi indagación sufría la falta de certidumbre de una orientación. Estas deficiencias han sido ampliamente colmadas con este nuevo evangelio de Jesús, con su aumento del discernimiento, su elevación de los ideales y su estabilidad de objetivos. Sin más dudas ni desconfianzas, ahora puedo emprender de todo corazón la aventura eterna.

\section*{2. El arte de vivir}
\par 
%\textsuperscript{(1775.2)}
\textsuperscript{160:2.1} Los mortales sólo tienen dos maneras de vivir juntos: la manera material o animal y la manera espiritual o humana. Por medio de signos y sonidos, los animales pueden comunicarse entre ellos en una medida limitada. Pero estas formas de comunicación no transmiten ni los significados, ni los valores ni las ideas. La única diferencia entre el hombre y el animal es que el hombre puede comunicarse con sus semejantes por medio de \textit{símbolos} que designan e identifican con precisión los significados, los valores, las ideas e incluso los ideales.

\par 
%\textsuperscript{(1775.3)}
\textsuperscript{160:2.2} Puesto que los animales no pueden comunicarse ideas entre sí, no pueden desarrollar una personalidad. El hombre desarrolla una personalidad porque puede comunicar a sus semejantes tanto las ideas como los ideales.

\par 
%\textsuperscript{(1775.4)}
\textsuperscript{160:2.3} Esta capacidad para comunicar y compartir los significados es lo que constituye la cultura humana y permite al hombre, a través de las asociaciones sociales, construir las civilizaciones. El conocimiento y la sabiduría se vuelven acumulativos debido a la capacidad del hombre para comunicar estas posesiones a las generaciones siguientes, surgiendo de esta manera las actividades culturales de la raza: el arte, la ciencia, la religión y la filosofía.

\par 
%\textsuperscript{(1775.5)}
\textsuperscript{160:2.4} La comunicación simbólica entre los seres humanos predetermina la aparición de los grupos sociales. El grupo social más eficaz de todos es la familia, y más concretamente los \textit{dos padres}. El afecto personal es el lazo espiritual que mantiene unidas estas asociaciones materiales. Una relación tan eficaz también es posible entre dos personas del mismo sexo, como lo ilustran tan abundantemente las devociones de las amistades auténticas.

\par 
%\textsuperscript{(1775.6)}
\textsuperscript{160:2.5} Estas asociaciones basadas en la amistad y en el afecto mutuos son socializadoras y ennoblecedoras porque fomentan y facilitan los siguientes factores esenciales de los niveles superiores del arte de vivir:

\par 
%\textsuperscript{(1775.7)}
\textsuperscript{160:2.6} 1. \textit{Expresarse y comprenderse mutuamente}. Muchos nobles impulsos humanos perecen porque no hay nadie que escuche su expresión. En verdad, no es bueno que el hombre esté solo. Cierto grado de reconocimiento y cierta cantidad de aprecio son esenciales para el desarrollo del carácter humano. Sin el amor auténtico del hogar, ningún niño puede alcanzar el pleno desarrollo de un carácter normal. El carácter es algo más que la mera mente y la mera moralidad. De todas las relaciones sociales pensadas para desarrollar el carácter, la más eficaz e ideal es la amistad afectuosa y comprensiva de un hombre y una mujer en el abrazo mutuo de una vida conyugal inteligente. El matrimonio, con sus múltiples relaciones, es el que está mejor destinado a hacer surgir esos preciosos impulsos y esos motivos elevados que son indispensables para el desarrollo de un carácter fuerte. No dudo en glorificar así la vida familiar, porque vuestro Maestro ha elegido sabiamente la relación de padre a hijo como la piedra angular misma de este nuevo evangelio del reino. Esta comunidad incomparable de relaciones, un hombre y una mujer en el abrazo afectuoso de los ideales superiores del tiempo, es una experiencia tan valiosa y satisfactoria que vale cualquier precio, cualquier sacrificio que sea necesario para poseerla.

\par 
%\textsuperscript{(1776.1)}
\textsuperscript{160:2.7} 2. \textit{La unión de las almas ---la movilización de la sabiduría}. Todo ser humano adquiere, tarde o temprano, cierto concepto de este mundo y cierta visión del siguiente. Ahora bien, es posible, mediante la asociación de las personalidades, unificar estos puntos de vista sobre la existencia temporal y las perspectivas eternas. Así, la mente de uno acrecienta sus valores espirituales adquiriendo una gran parte de la perspicacia del otro. De esta manera, los hombres enriquecen su alma poniendo en común sus posesiones espirituales respectivas. Y también de esta misma manera el hombre consigue evitar esa tendencia siempre presente a caer víctima de su visión distorsionada, de su punto de vista parcial y de su estrechez de juicio. El miedo, la envidia y la vanidad sólo se pueden impedir mediante el contacto íntimo con otras mentes. Llamo vuestra atención sobre el hecho de que el Maestro nunca os envía solos a trabajar para la expansión del reino; siempre os envía de dos en dos. Y puesto que la sabiduría es un superconocimiento, de esto se deduce que, al unir su sabiduría, el grupo social, grande o pequeño, comparte mutuamente todo el conocimiento.

\par 
%\textsuperscript{(1776.2)}
\textsuperscript{160:2.8} 3. \textit{El entusiasmo de vivir}. El aislamiento tiende a agotar la carga de energía del alma. La asociación con nuestros semejantes es esencial para renovar el entusiasmo por la vida, y es indispensable para conservar la valentía para librar esas batallas que siguen a la ascensión a unos niveles superiores de vida humana. La amistad aumenta las alegrías y glorifica los triunfos de la vida. Las asociaciones humanas afectuosas e íntimas tienden a quitarle al sufrimiento su tristeza, y a las dificultades mucha parte de su amargura. La presencia de un amigo realza toda belleza y exalta toda bondad. Por medio de símbolos inteligentes, el hombre es capaz de vivificar y de ampliar las capacidades apreciativas de sus amigos. Este poder y esta posibilidad de estimularse mutuamente la imaginación es una de las glorias supremas de la amistad humana. Existe un gran poder espiritual inherente a la conciencia de estar consagrado de todo corazón a una causa común, de ser mutuamente leales a una Deidad cósmica.

\par 
%\textsuperscript{(1776.3)}
\textsuperscript{160:2.9} 4. \textit{La defensa creciente contra todo mal}. La asociación entre personalidades y el afecto mutuo son un seguro eficaz contra el mal. Las dificultades, las tristezas, las decepciones y las derrotas son más dolorosas y desalentadoras cuando se soportan a solas. La asociación no transforma el mal en rectitud, pero ayuda mucho a disminuir las heridas. Vuestro Maestro ha dicho: <<Bienaventurados los que están de luto>> ---si hay un amigo cerca para consolarlos. Hay una fuerza positiva en el conocimiento de que vivís para el bienestar de los demás, y que los demás viven igualmente para vuestro bienestar y vuestro progreso. El hombre languidece en el aislamiento. Los seres humanos se desaniman infaliblemente cuando ven solamente las transacciones transitorias del tiempo. Cuando el presente está separado del pasado y del futuro, se vuelve de una trivialidad exasperante. Vislumbrar el círculo de la eternidad es lo único que puede inspirar al hombre para hacer lo mejor posible, y que puede desafiar lo mejor que hay en él para que haga lo máximo. Cuando el hombre se encuentra así en sus mejores disposiciones, vive de manera muy generosa para el bien de los demás, para sus semejantes que residen con él en el tiempo y en la eternidad.

\par 
%\textsuperscript{(1777.1)}
\textsuperscript{160:2.10} Repito que esta asociación inspiradora y ennoblecedora encuentra sus posibilidades ideales en las relaciones del matrimonio humano. Es verdad que se pueden conseguir muchas cosas fuera del matrimonio, y que muchísimos matrimonios no logran producir en absoluto estos frutos morales y espirituales. Demasiadas veces contraen matrimonio aquellos que buscan otros valores que son inferiores a estos acompañamientos superiores de la madurez humana. El matrimonio ideal debe estar fundamentado en algo más estable que las fluctuaciones del sentimiento y la inconstancia de la simple atracción sexual; debe estar basado en una devoción personal auténtica y mutua. Así pues, si se pueden construir estas pequeñas unidades dignas de confianza y eficaces de asociaciones humanas, cuando se reúnan en conjunto, el mundo contemplará una gran estructura social glorificada, la civilización de la madurez de los mortales. Una raza así podría empezar a realizar una parte del ideal de vuestro Maestro de <<paz en la Tierra y buena voluntad entre los hombres>>. Aunque una sociedad así no sería perfecta ni estaría completamente libre del mal, al menos se acercaría a la estabilización de la madurez.

\section*{3. Los atractivos de la madurez}
\par 
%\textsuperscript{(1777.2)}
\textsuperscript{160:3.1} El esfuerzo por conseguir la madurez necesita trabajo, y el trabajo requiere energía. ¿De dónde viene el poder para realizar todo esto?. Las cosas físicas se pueden dar por sentadas, pero el Maestro bien ha dicho que <<No sólo de pan vive el hombre>>. Una vez que se posee un cuerpo normal y una salud razonablemente buena, debemos buscar a continuación aquellos atractivos que actuarán como estímulo para hacer surgir las fuerzas espirituales dormidas del hombre. Jesús nos ha enseñado que Dios vive en el hombre; entonces, ¿cómo podemos inducir al hombre a que libere estos poderes de la divinidad y de la infinidad que están ligados en su alma? ¿Cómo induciremos a los hombres a que dejen paso a Dios y Éste pueda brotar para refrescar nuestras propias almas mientras transita hacia el exterior, y luego sirva al propósito de iluminar, elevar y bendecir a otras innumerables almas? ¿De qué manera puedo despertar mejor estos poderes latentes para el bien que yace dormido en vuestra alma? De una cosa estoy seguro: la excitación emocional no es el estímulo espiritual ideal. La excitación no aumenta la energía; más bien agota las fuerzas de la mente y del cuerpo. ¿De dónde viene pues la energía para hacer estas grandes cosas? Observad a vuestro Maestro. En este mismo momento se encuentra allá en las colinas, llenándose de fuerza, mientras nosotros estamos aquí gastando energía. El secreto de todo este problema está envuelto en la comunión espiritual, en la adoración. Desde el punto de vista humano, se trata de combinar la meditación y la relajación. La meditación pone en contacto a la mente con el espíritu; la relajación determina la capacidad para la receptividad espiritual. Este intercambio de la debilidad por la fuerza, del temor por el valor, de la mente del yo por la voluntad de Dios, constituye la adoración. Al menos, el filósofo lo ve de esta manera.

\par 
%\textsuperscript{(1777.3)}
\textsuperscript{160:3.2} Cuando estas experiencias se repiten con frecuencia, se cristalizan en hábitos, en unos hábitos de adoración que dan fuerzas, y estos hábitos se traducen con el tiempo en un carácter espiritual, y este carácter es reconocido finalmente por nuestros semejantes como \textit{una personalidad madura}. Al principio, estas prácticas son difíciles y llevan mucho tiempo, pero cuando se vuelven habituales, proporcionan descanso y ahorro de tiempo a la vez. Cuanto más compleja se vuelva la sociedad, cuanto más se multipliquen los atractivos de la civilización, más urgente será la necesidad, para los individuos que conocen a Dios, de adquirir estas prácticas habituales protectoras destinadas a conservar y aumentar sus energías espirituales.

\par 
%\textsuperscript{(1778.1)}
\textsuperscript{160:3.3} Otro requisito para alcanzar la madurez es la adaptación cooperativa de los grupos sociales a un entorno en constante cambio. El individuo inmaduro despierta el antagonismo de sus semejantes; el hombre maduro se gana la cooperación cordial de sus asociados, multiplicando así muchas veces los frutos de los esfuerzos de su vida.

\par 
%\textsuperscript{(1778.2)}
\textsuperscript{160:3.4} Mi filosofía me dice que hay momentos en que debo luchar, si hace falta, para defender mi concepto de la rectitud, pero no dudo de que el Maestro, con su tipo de personalidad más madura, conseguiría fácil y elegantemente una victoria equivalente mediante su técnica superior y encantadora de tacto y de tolerancia. Demasiado a menudo, cuando luchamos por una cosa justa, resulta que tanto el vencedor como el vencido sufren una derrota. Ayer mismo oí decir al Maestro que <<si un hombre sabio trata de entrar por una puerta cerrada, no destruye la puerta, sino que busca la llave para abrirla>>. Con mucha frecuencia nos ponemos a luchar sólo para convencernos de que no tenemos miedo.

\par 
%\textsuperscript{(1778.3)}
\textsuperscript{160:3.5} Este nuevo evangelio del reino presta un gran servicio al arte de vivir, en el sentido de que proporciona un incentivo nuevo y más rico para una vida más elevada. Presenta una meta de destino nueva y sublime, un propósito supremo para la vida. Estos nuevos conceptos de la meta eterna y divina de la existencia son en sí mismos unos estímulos trascendentes que suscitan la reacción de lo mejor que existe en la naturaleza superior del hombre. En toda cima del pensamiento intelectual se encuentra un descanso para la mente, una fuerza para el alma y una comunión para el espíritu. Desde esta posición de ventaja de la vida superior, el hombre es capaz de trascender las irritaciones materiales de los niveles inferiores de pensamiento ---las preocupaciones, los celos, la envidia, la venganza y el orgullo de la personalidad inmadura. Las almas que ascienden a estas alturas se liberan de una multitud de conflictos a contracorriente de las nimiedades de la vida, volviéndose así libres para alcanzar la conciencia de las corrientes superiores de los conceptos espirituales y de las comunicaciones celestiales. Pero el propósito de la vida debe ser celosamente protegido contra la tentación de buscar los logros fáciles y transitorios; asimismo, debe ser fomentado de tal manera que se vuelva inmune a las amenazas desastrosas del fanatismo.

\section*{4. El equilibrio de la madurez}
\par 
%\textsuperscript{(1778.4)}
\textsuperscript{160:4.1} Mientras tenéis la vista puesta en alcanzar las realidades eternas, debéis también atender las necesidades de la vida temporal. Aunque el espíritu sea nuestra meta, la carne es un hecho. Puede suceder que lo que necesitamos para vivir caiga en nuestras manos por casualidad, pero en general, tenemos que trabajar con inteligencia para conseguirlo. Los dos problemas principales de la vida son: ganarse la vida temporal y conseguir la supervivencia eterna. Incluso el problema de ganarse la vida necesita a la religión para solucionarse de manera ideal. Estos dos problemas son muy personales. De hecho, la verdadera religión no funciona separadamente del individuo.

\par 
%\textsuperscript{(1778.5)}
\textsuperscript{160:4.2} Los factores esenciales de la vida temporal, tal como yo los veo, son:

\par 
%\textsuperscript{(1778.6)}
\textsuperscript{160:4.3} 1. Una buena salud física.

\par 
%\textsuperscript{(1778.7)}
\textsuperscript{160:4.4} 2. Un pensamiento claro y limpio.

\par 
%\textsuperscript{(1778.8)}
\textsuperscript{160:4.5} 3. La capacidad y la habilidad.

\par 
%\textsuperscript{(1778.9)}
\textsuperscript{160:4.6} 4. La riqueza ---los bienes de la vida.

\par 
%\textsuperscript{(1778.10)}
\textsuperscript{160:4.7} 5. La capacidad para resistir la derrota.

\par 
%\textsuperscript{(1778.11)}
\textsuperscript{160:4.8} 6. La cultura ---la educación y la sabiduría.

\par 
%\textsuperscript{(1779.1)}
\textsuperscript{160:4.9} Incluso los problemas materiales de la salud y la eficacia físicas se resuelven mejor cuando se ven desde el punto de vista religioso de las enseñanzas de nuestro Maestro: el cuerpo y la mente del hombre son el lugar donde vive el don de los Dioses, el espíritu de Dios que se convierte en el espíritu del hombre. La mente del hombre se vuelve así la mediadora entre las cosas materiales y las realidades espirituales.

\par 
%\textsuperscript{(1779.2)}
\textsuperscript{160:4.10} Se necesita inteligencia para conseguir la parte que nos corresponde de las cosas deseables de la vida. Es totalmente erróneo suponer que hacer fielmente nuestro trabajo diario nos asegurará la recompensa de la riqueza. Exceptuando la adquisición ocasional y accidental de las riquezas, se descubre que las recompensas materiales de la vida temporal fluyen por ciertos canales bien organizados, y sólo aquellos que tienen acceso a esos canales pueden esperar ser bien recompensados por sus esfuerzos temporales. La pobreza será siempre el destino de todos los hombres que buscan la riqueza en canales aislados e individuales. Por consiguiente, una planificación sabia se convierte en la única cosa esencial para la prosperidad material. El éxito requiere no solamente vuestra devoción al trabajo, sino también que funcionéis como una parte de uno de los canales de la riqueza material. Si sois poco sabios, podéis otorgarle a vuestra generación una vida dedicada sin recompensa material; si os beneficiáis accidentalmente del flujo de la riqueza, podréis nadar en el lujo aunque no hayáis hecho nada útil por vuestros semejantes.

\par 
%\textsuperscript{(1779.3)}
\textsuperscript{160:4.11} La capacidad se hereda, mientras que la habilidad se adquiere. La vida es irreal para aquel que no sabe hacer alguna cosa bien, expertamente. La habilidad es una de las verdaderas fuentes de satisfacción en la vida. La capacidad implica el don de la previsión, de la visión de futuro. No os dejéis engañar por las recompensas tentadoras de los logros deshonestos; estad dispuestos a trabajar por las retribuciones posteriores inherentes a los esfuerzos honrados. El hombre sabio es capaz de distinguir entre los medios y los fines; por otra parte, un exceso de planes para el futuro a veces hace fracasar su propio propósito elevado. En cuanto a la búsqueda de los placeres, deberíais siempre aspirar a producirlos tanto como a consumirlos.

\par 
%\textsuperscript{(1779.4)}
\textsuperscript{160:4.12} Entrenad vuestra memoria para que conserve como un depósito sagrado los episodios fortalecedores y valiosos de la vida, a fin de poderlos recordar a voluntad para vuestro placer y edificación. Construid así para vosotros y dentro de vosotros galerías en reserva de belleza, de bondad y de grandeza artística. Pero los recuerdos más nobles de todos son las memorias atesoradas de los grandes momentos de una magnífica amistad. Todos estos tesoros de la memoria irradian su influencia más preciosa y sublime con el contacto liberador de la adoración espiritual.

\par 
%\textsuperscript{(1779.5)}
\textsuperscript{160:4.13} Pero la vida se convertirá en una carga de la existencia si no aprendéis a fracasar con elegancia. Aceptar las derrotas es un arte que las almas nobles siempre adquieren; debéis saber perder con alegría; debéis ser intrépidos ante las decepciones. No dudéis nunca en admitir un fracaso. No intentéis ocultar el fracaso con sonrisas engañosas y un optimismo radiante. Suena muy bien afirmar que siempre se tiene éxito, pero los resultados finales son espantosos. Esta técnica conduce directamente a la creación de un mundo irreal y a la caída inevitable en la desilusión final.

\par 
%\textsuperscript{(1779.6)}
\textsuperscript{160:4.14} El éxito puede generar la valentía y promover la confianza, pero la sabiduría sólo proviene de las experiencias de adaptación a los resultados de los fracasos personales. Los hombres que prefieren las ilusiones optimistas a la realidad, nunca podrán volverse sabios. Sólo aquellos que se enfrentan con los hechos y los adaptan a sus ideales pueden conseguir la sabiduría. La sabiduría engloba los hechos y los ideales, y por eso salva a sus adeptos de los dos extremos estériles de la filosofía ---el hombre cuyo idealismo excluye los hechos, y el materialista desprovisto de perspectiva espiritual. Las almas tímidas que sólo pueden mantener la lucha por la vida mediante la ayuda continua de las falsas ilusiones del éxito, están condenadas a sufrir fracasos y a experimentar derrotas cuando se despierten finalmente del mundo ilusorio de su propia imaginación.

\par 
%\textsuperscript{(1780.1)}
\textsuperscript{160:4.15} En esta cuestión de enfrentarse con el fracaso y de adaptarse a la derrota es donde la visión de gran alcance de la religión ejerce su influencia suprema. El fracaso es simplemente un episodio educativo ---una experiencia cultural para adquirir sabiduría--- en la experiencia del hombre que busca a Dios y que ha emprendido la aventura eterna de explorar un universo. Para este tipo de hombres, la derrota no es más que una nueva herramienta para alcanzar los niveles superiores de la realidad universal.

\par 
%\textsuperscript{(1780.2)}
\textsuperscript{160:4.16} La carrera de un hombre que busca a Dios puede resultar ser un gran éxito a la luz de la eternidad, aunque toda la empresa de su vida temporal pueda parecer un fracaso abrumador, con tal que cada fracaso de su vida haya producido el cultivo de la sabiduría y el logro espiritual. No cometáis el error de confundir el conocimiento, la cultura y la sabiduría. Están relacionados en la vida, pero representan valores espirituales extremadamente diferentes; la sabiduría domina siempre al conocimiento y glorifica siempre a la cultura.

\section*{5. La religión del Ideal}
\par 
%\textsuperscript{(1780.3)}
\textsuperscript{160:5.1} Me habéis dicho que vuestro Maestro considera que la auténtica religión humana es la experiencia del individuo con las realidades espirituales. Yo he considerado la religión como la experiencia del hombre que reacciona ante algo que le parece digno del homenaje y de la devoción de toda la humanidad. En este sentido, la religión simboliza nuestra devoción suprema a aquello que representa nuestro concepto más elevado de los ideales de la realidad, y el máximo alcance de nuestra mente hacia las posibilidades eternas de la consecución espiritual.

\par 
%\textsuperscript{(1780.4)}
\textsuperscript{160:5.2} Cuando los hombres reaccionan ante la religión en un sentido tribal, nacional o racial, es porque consideran que aquellos que no pertenecen a su grupo no son realmente humanos. Siempre consideramos que el objeto de nuestra lealtad religiosa es digno de ser venerado por todos los hombres. La religión nunca puede ser un asunto de simple creencia intelectual o de razonamiento filosófico; la religión es siempre y para siempre una manera de reaccionar ante las situaciones de la vida; es una especie de conducta. La religión abarca el pensamiento, el sentimiento y el actuar con reverencia hacia una realidad que consideramos digna de la adoración universal.

\par 
%\textsuperscript{(1780.5)}
\textsuperscript{160:5.3} Si algo se ha vuelto una religión en vuestra experiencia, es evidente que ya sois evangelistas activos de esa religión, puesto que consideráis que el concepto supremo de vuestra religión es digno de la adoración de toda la humanidad, de todas las inteligencias del universo. Si no sois unos evangelistas convencidos y misioneros de vuestra religión, os engañáis a vosotros mismos, en el sentido de que aquello que llamáis religión no es más que una creencia tradicional o un simple sistema de filosofía intelectual. Si vuestra religión es una experiencia espiritual, el objeto de vuestra adoración debe ser la realidad y el ideal espiritual universal de todos vuestros conceptos espiritualizados. Todas las religiones que se basan en el miedo, la emoción, la tradición y la filosofía, las califico de religiones intelectuales, mientras que aquellas que se basan en la verdadera experiencia espiritual las calificaría de religiones verdaderas. El objeto de la devoción religiosa puede ser material o espiritual, verdadero o falso, real o irreal, humano o divino. Las religiones pueden ser, por tanto, buenas o malas.

\par 
%\textsuperscript{(1780.6)}
\textsuperscript{160:5.4} La moralidad y la religión no son necesariamente la misma cosa. Si un sistema de moralidad se aferra a un objeto de adoración, puede volverse una religión. Cuando una religión pierde su llamamiento universal a la lealtad y a la devoción suprema, puede convertirse en un sistema de filosofía o en un código de moralidad. Esa cosa, ser, estado, orden de existencia o posibilidad de consecución que constituye el ideal supremo de la lealtad religiosa, y que es el receptor de la devoción religiosa de aquellos que adoran, es Dios. Sin tener en cuenta el nombre que se aplique a este ideal de la realidad espiritual, es Dios.

\par 
%\textsuperscript{(1781.1)}
\textsuperscript{160:5.5} La característica social de una verdadera religión consiste en el hecho de que ésta busca invariablemente convertir al individuo y transformar el mundo. La religión implica la existencia de ideales no descubiertos que trascienden de lejos las normas éticas y morales conocidas, incorporadas en los usos sociales, incluso más elevados, de las instituciones más maduras de la civilización. La religión trata de alcanzar ideales no descubiertos, realidades inexploradas, valores sobrehumanos, una sabiduría divina y un verdadero logro espiritual. La verdadera religión hace todo esto; todas las demás creencias no son dignas de este nombre. No podéis tener una religión espiritual auténtica sin el ideal supremo y celestial de un Dios eterno. Una religión sin este Dios es un invento del hombre, una institución humana de creencias intelectuales sin vida y de ceremonias emocionales sin sentido. Una religión puede pretender tener un gran ideal como objeto de su devoción. Pero estos ideales irreales son inaccesibles; un concepto así es ilusorio. Los únicos ideales susceptibles de ser alcanzados por los hombres son las realidades divinas de los valores infinitos que residen en el hecho espiritual del Dios eterno.

\par 
%\textsuperscript{(1781.2)}
\textsuperscript{160:5.6} La palabra Dios, la \textit{idea} de Dios en contraposición con el \textit{ideal} de Dios, puede volverse una parte de cualquier religión, por muy falsa o pueril que pueda ser esa religión. Y aquellos que conciben esta idea de Dios pueden hacer con ella cualquier cosa que quieran. Las religiones inferiores modelan sus ideas de Dios para satisfacer el estado natural del corazón humano; las religiones superiores exigen que el corazón humano cambie para satisfacer las demandas de los ideales de la verdadera religión.

\par 
%\textsuperscript{(1781.3)}
\textsuperscript{160:5.7} La religión de Jesús trasciende todos nuestros conceptos anteriores sobre la idea de la adoración, en el sentido de que no solamente describe a su Padre como el ideal de la realidad infinita, sino que declara categóricamente que esta fuente divina de los valores y el centro eterno del universo es verdadera y personalmente accesible para toda criatura mortal que elija entrar en el reino de los cielos en la Tierra, reconociendo así que acepta la filiación con Dios y la fraternidad con el hombre. Sugiero que éste es el concepto más elevado de la religión que el mundo haya conocido jamás, y declaro que nunca puede haber uno superior puesto que este evangelio abarca la infinidad de las realidades, la divinidad de los valores y la eternidad de los logros universales. Un concepto así constituye la realización de la experiencia del idealismo de lo supremo y de lo último.

\par 
%\textsuperscript{(1781.4)}
\textsuperscript{160:5.8} No solamente me intrigan los ideales consumados de esta religión de vuestro Maestro, sino que me siento poderosamente impulsado a confesar mi creencia en su declaración de que estos ideales de las realidades espirituales son accesibles; que vosotros y yo podemos emprender esta larga y eterna aventura, con su garantía de que al final llegaremos ciertamente a las puertas del Paraíso. Hermanos míos, soy un creyente, me he embarcado; estoy de camino con vosotros en esta aventura eterna. El Maestro dice que ha venido del Padre y que nos mostrará el camino. Estoy totalmente persuadido de que dice la verdad. Estoy definitivamente convencido de que fuera del Padre Universal y eterno no existen ideales de realidad ni valores de perfección que se puedan alcanzar.

\par 
%\textsuperscript{(1781.5)}
\textsuperscript{160:5.9} Vengo pues a adorar, no simplemente al Dios de las existencias, sino al Dios de la posibilidad de todas las existencias futuras. Por lo tanto, vuestra devoción a un ideal supremo, si este ideal es real, debe ser una devoción a este Dios de los universos pasados, presentes y futuros de cosas y de seres. Y no hay otro Dios, porque no puede haber de ninguna manera ningún otro Dios. Todos los demás dioses son invenciones de la imaginación, ilusiones de la mente mortal, distorsiones de la falsa lógica e ídolos engañosos de aquellos que los crean. Sí, podéis tener una religión sin este Dios, pero no significa nada. Si tratáis de sustituir la realidad de este ideal del Dios viviente por la palabra Dios, sólo os engañaréis a vosotros mismos poniendo una idea en el lugar de un ideal, de una realidad divina. Estas creencias son simplemente religiones de quimeras.

\par 
%\textsuperscript{(1782.1)}
\textsuperscript{160:5.10} En las enseñanzas de Jesús veo la religión en su mejor expresión. Este evangelio nos permite buscar al verdadero Dios y encontrarlo. Pero, ¿estamos dispuestos a pagar el precio de esta entrada en el reino de los cielos?. ¿Estamos dispuestos a nacer de nuevo, a ser rehechos?. ¿Estamos dispuestos a someternos a ese terrible proceso probatorio de la destrucción del yo y de la reconstrucción del alma?. ¿Acaso no ha dicho el Maestro: <<El que quiera salvar su vida ha de perderla. No creáis que he venido para traer la paz, sino más bien una lucha del alma>>?. Es verdad que después de pagar el precio de la dedicación a la voluntad del Padre experimentamos una gran paz, a condición de que continuemos caminando en los senderos espirituales de la vida consagrada.

\par 
%\textsuperscript{(1782.2)}
\textsuperscript{160:5.11} Ahora estamos abandonando de verdad los alicientes del orden de existencia conocido, mientras nos dedicamos sin reservas a buscar los encantos del orden de existencia desconocido e inexplorado de una vida futura de aventuras en los mundos espirituales del idealismo superior de la realidad divina. Y buscamos esos símbolos significativos con los que transmitir a nuestros semejantes estos conceptos de la realidad del idealismo de la religión de Jesús, y no dejaremos de rezar por ese día en que toda la humanidad se emocionará con la visión común de esta verdad suprema. En este momento, nuestro concepto focalizado del Padre, tal como lo tenemos en nuestro corazón, es que Dios es espíritu; tal como lo trasmitimos a nuestros semejantes, Dios es amor.

\par 
%\textsuperscript{(1782.3)}
\textsuperscript{160:5.12} La religión de Jesús exige una experiencia viviente y espiritual. Otras religiones pueden consistir en creencias tradicionales, sentimientos emotivos, conciencias filosóficas, y todo eso junto, pero la enseñanza del Maestro requiere que se alcancen los niveles reales del progreso espiritual verdadero.

\par 
%\textsuperscript{(1782.4)}
\textsuperscript{160:5.13} La conciencia del impulso a ser semejante a Dios no es la verdadera religión. Los sentimientos emotivos de adorar a Dios no son la verdadera religión. La convicción consciente de abandonar el yo y servir a Dios no es la verdadera religión. La sabiduría del razonamiento de que esta religión es la mejor de todas, no es la religión como experiencia personal y espiritual. La verdadera religión tiene relación con el destino y la realidad de lo que se logra, así como con la realidad y el idealismo de aquello que se acepta de todo corazón por la fe. Y todo esto debe hacerse personal para nosotros mediante la revelación del Espíritu de la Verdad.

\par 
%\textsuperscript{(1782.5)}
\textsuperscript{160:5.14} Así terminaron las disertaciones del filósofo griego, uno de los más grandes de su raza, que se había vuelto creyente en el evangelio de Jesús.


\chapter{Documento 161. Otras discusiones con Rodán}
\par 
%\textsuperscript{(1783.1)}
\textsuperscript{161:0.1} EL DOMINGO 25 de septiembre del año 29, los apóstoles y los evangelistas se congregaron en Magadán. Aquella tarde, después de una larga conferencia con sus asociados, Jesús los sorprendió a todos anunciando que, al día siguiente, partiría temprano hacia Jerusalén con los doce apóstoles para asistir a la fiesta de los tabernáculos. Ordenó a los evangelistas que visitaran a los creyentes en Galilea, y al cuerpo de mujeres que regresara durante un tiempo a Betsaida.

\par 
%\textsuperscript{(1783.2)}
\textsuperscript{161:0.2} Cuando llegó la hora de salir hacia Jerusalén, Natanael y Tomás estaban aún en medio de sus discusiones con Rodán de Alejandría, y consiguieron el permiso del Maestro para quedarse unos días en Magadán. Y así, mientras Jesús y los diez iban de camino hacia Jerusalén, Natanael y Tomás estaban ocupados en un serio debate con Rodán. La semana anterior, durante la cual Rodán había expuesto su filosofía, Tomás y Natanael se habían alternado para presentar el evangelio del reino al filósofo griego. Rodán descubrió que las enseñanzas de Jesús le habían sido bien expuestas por su instructor de Alejandría, uno de los antiguos apóstoles de Juan el Bautista.

\section*{1. La personalidad de Dios}
\par 
%\textsuperscript{(1783.3)}
\textsuperscript{161:1.1} Había una cuestión que Rodán y los dos apóstoles no percibían de la misma manera, y era la personalidad de Dios. Rodán aceptaba de buena gana todo lo que se le exponía sobre los atributos de Dios, pero sostenía que el Padre que está en los cielos no es, y no puede ser, una persona tal como el hombre concibe la personalidad. Aunque los apóstoles tenían dificultades para intentar probar que Dios es una persona, Rodán encontraba aún más difícil probar que no es una persona.

\par 
%\textsuperscript{(1783.4)}
\textsuperscript{161:1.2} Rodán sostenía que el hecho de la personalidad consiste en el hecho simultáneo de que unos seres semejantes que son capaces de entenderse con afinidad, se comunican plena y mutuamente entre ellos. Rodán dijo: <<Para que Dios sea una persona, debe utilizar unos símbolos de comunicación espiritual que le permitan ser plenamente comprendido por los que se ponen en contacto con él. Pero como Dios es infinito y eterno, y es el Creador de todos los demás seres, de esto se desprende que, en lo que concierne a los seres semejantes, Dios está solo en el universo. No hay nadie igual a él; no hay nadie con quien pueda comunicarse de igual a igual. Dios puede ser en verdad la fuente de toda personalidad, pero como tal trasciende la personalidad, de la misma manera que el Creador está por encima y más allá de la criatura>>.

\par 
%\textsuperscript{(1783.5)}
\textsuperscript{161:1.3} Este argumento había perturbado mucho a Tomás y Natanael, y habían pedido a Jesús que viniera a ayudarlos, pero el Maestro se negó a participar en sus discusiones. Sin embargo le dijo a Tomás: <<Poco importa la \textit{idea} que podáis tener del Padre, con tal que conozcáis espiritualmente el \textit{ideal} de su naturaleza infinita y eterna>>.

\par 
%\textsuperscript{(1784.1)}
\textsuperscript{161:1.4} Tomás sostenía que Dios se comunica con el hombre, y que por consiguiente el Padre es una persona, según incluso la definición de Rodán. El griego rechazó esto sobre la base de que Dios no se revela personalmente, de que continúa siendo un misterio. Entonces, Natanael recurrió a su propia experiencia personal con Dios, y Rodán la admitió afirmando que recientemente había tenido experiencias similares, pero sostenía que estas experiencias probaban solamente la \textit{realidad} de Dios, no su \textit{personalidad}.

\par 
%\textsuperscript{(1784.2)}
\textsuperscript{161:1.5} El lunes por la noche, Tomás se rindió. Pero el martes por la noche, Natanael había conseguido que Rodán creyera en la personalidad del Padre, y había producido este cambio de opinión en el griego mediante las etapas de razonamiento siguientes:

\par 
%\textsuperscript{(1784.3)}
\textsuperscript{161:1.6} 1. El Padre Paradisiaco goza de una igualdad de comunicación con al menos otros dos seres que son plenamente iguales y totalmente semejantes a él ---el Hijo Eterno y el Espíritu Infinito. En vista de la doctrina de la Trinidad, el griego estuvo obligado a admitir la posibilidad de que el Padre Universal tuviera una personalidad. (El examen posterior de estas discusiones fue lo que condujo a una ampliación del concepto de la Trinidad en la mente de los doce apóstoles. Por supuesto, la creencia general consideraba que Jesús era el Hijo Eterno).

\par 
%\textsuperscript{(1784.4)}
\textsuperscript{161:1.7} 2. Puesto que Jesús era igual al Padre, y puesto que este Hijo había conseguido manifestar su personalidad a sus hijos terrestres, este fenómeno constituía la prueba del hecho, y la demostración de la posibilidad, de que las tres Deidades poseían una personalidad, y zanjaba para siempre la cuestión respecto a la aptitud de Dios para comunicarse con el hombre y a la posibilidad del hombre de comunicarse con Dios.

\par 
%\textsuperscript{(1784.5)}
\textsuperscript{161:1.8} 3. Jesús estaba en términos de asociación mutua y de comunicación perfecta con el hombre; Jesús era el Hijo de Dios. La relación entre el Hijo y el Padre presupone una igualdad de comunicación y un entendimiento afín mutuo; Jesús y el Padre eran uno solo. Jesús mantenía igualmente y al mismo tiempo una comunicación comprensiva tanto con Dios como con el hombre; puesto que ambos, Dios y el hombre, comprendían el significado de los símbolos de la comunicación de Jesús, tanto Dios como el hombre poseían los atributos de la personalidad en lo referente a los requisitos para tener la aptitud de intercomunicarse. La personalidad de Jesús demostraba la personalidad de Dios, y al mismo tiempo probaba de manera concluyente la presencia de Dios en el hombre. Dos cosas que están relacionadas con una tercera, están relacionadas entre sí.

\par 
%\textsuperscript{(1784.6)}
\textsuperscript{161:1.9} 4. La personalidad representa el concepto más elevado que el hombre tiene de la realidad humana y de los valores divinos; Dios también representa el concepto más elevado que el hombre tiene de la realidad divina y de los valores infinitos; por consiguiente, Dios debe ser una personalidad divina e infinita, una personalidad de hecho, aunque trascienda de manera infinita y eterna el concepto y la definición humanos de la personalidad, pero sin embargo continúa siendo siempre y universalmente una personalidad.

\par 
%\textsuperscript{(1784.7)}
\textsuperscript{161:1.10} 5. Dios debe ser una personalidad, puesto que es el Creador de toda personalidad y el destino de toda personalidad. La enseñanza de Jesús <<Sed pues perfectos como vuestro Padre que está en los cielos es perfecto>>, había causado una enorme influencia sobre Rodán.

\par 
%\textsuperscript{(1784.8)}
\textsuperscript{161:1.11} Cuando Rodán escuchó estos argumentos, dijo: <<Estoy convencido. Reconoceré que Dios es una persona si me permitís modificar mi confesión de esta creencia atribuyendo al significado de personalidad un conjunto de valores más amplios, tales como sobrehumano, trascendental, supremo, infinito, eterno, final y universal. Ahora estoy convencido de que, aunque Dios debe ser infinitamente más que una personalidad, no puede ser nada menos. Estoy satisfecho de poner fin a la controversia y de aceptar a Jesús como la revelación personal del Padre y como la compensación de todas las lagunas de la lógica, la razón y la filosofía>>.

\section*{2. La naturaleza divina de Jesús}
\par 
%\textsuperscript{(1785.1)}
\textsuperscript{161:2.1} Natanael y Tomás habían aprobado plenamente los puntos de vista de Rodán sobre el evangelio del reino, y sólo quedaba un punto más por examinar: la enseñanza relacionada con la naturaleza divina de Jesús, una doctrina que se había anunciado públicamente muy recientemente. Natanael y Tomás presentaron conjuntamente sus puntos de vista sobre la naturaleza divina del Maestro, y el relato que sigue es una presentación abreviada, readaptada y reformulada de sus enseñanzas:

\par 
%\textsuperscript{(1785.2)}
\textsuperscript{161:2.2} 1. Jesús ha admitido su divinidad, y nosotros le creemos. Muchas cosas notables han sucedido en conexión con su ministerio, y sólo las podemos comprender si creemos que es el Hijo de Dios así como el Hijo del Hombre.

\par 
%\textsuperscript{(1785.3)}
\textsuperscript{161:2.3} 2. Su asociación cotidiana con nosotros ejemplifica el ideal de la amistad humana; sólo un ser divino podría ser tal vez un amigo humano de este tipo. Es la persona más sinceramente desinteresada que hemos conocido nunca. Es amigo incluso de los pecadores; se atreve a amar a sus enemigos. Es muy leal con nosotros. Aunque no duda en reprendernos, es evidente para todos que nos ama realmente. Cuanto más lo conoces, más lo amas. Te encantará su consagración inquebrantable. Durante todos estos años en que no hemos logrado comprender su misión, ha sido un amigo fiel. Aunque no emplea la adulación, nos trata a todos con la misma benevolencia; es invariablemente tierno y compasivo. Ha compartido con nosotros su vida y todas las demás cosas. Formamos una comunidad feliz; compartimos todas las cosas. No creemos que un simple ser humano pueda vivir una vida tan libre de culpa en unas circunstancias tan duras.

\par 
%\textsuperscript{(1785.4)}
\textsuperscript{161:2.4} 3. Pensamos que Jesús es divino porque nunca hace el mal; no comete errores. Su sabiduría es extraordinaria y su piedad, magnífica. Vive día tras día en perfecta armonía con la voluntad del Padre. Nunca se arrepiente de haber actuado mal porque no transgrede ninguna de las leyes del Padre. Ora por nosotros y con nosotros, pero nunca nos pide que oremos por él. Creemos que está constantemente libre de pecado. No creemos que alguien que sea únicamente humano haya pretendido nunca vivir una vida semejante. Afirma vivir una vida perfecta, y reconocemos que lo hace. Nuestra piedad procede del arrepentimiento, pero la suya proviene de la rectitud. Afirma incluso perdonar los pecados y cura de hecho las enfermedades. Ningún simple hombre en su sano juicio declararía que perdona los pecados; eso es una prerrogativa divina. Desde el momento de nuestro primer contacto con él, nos ha parecido así de perfecto en su rectitud. Nosotros crecemos en la gracia y en el conocimiento de la verdad, pero nuestro Maestro manifiesta la madurez de la rectitud desde el principio. Todos los hombres, buenos y malos, reconocen estos elementos de bondad en Jesús. Sin embargo, su piedad nunca es inoportuna ni ostentosa. Él es a la vez humilde e intrépido. Parece aprobar nuestra creencia en su divinidad. O bien él es lo que declara ser, o por el contrario es el hipócrita y el impostor más grande que el mundo ha conocido nunca. Estamos persuadidos de que es exactamente lo que declara ser.

\par 
%\textsuperscript{(1785.5)}
\textsuperscript{161:2.5} 4. Su carácter sin igual y la perfección de su control emotivo nos convencen de que es una combinación de humanidad y de divinidad. Reacciona infaliblemente ante el espectáculo de la miseria humana; el sufrimiento nunca deja de conmoverlo. Su compasión se despierta por igual ante el sufrimiento físico, la angustia mental o la pesadumbre espiritual. Reconoce rápidamente y admite con generosidad la presencia de la fe o de cualquier otra gracia en sus semejantes. Es tan justo y equitativo, y al mismo tiempo tan misericordioso y considerado. Se entristece por la obstinación espiritual de la gente, y se regocija cuando consienten en ver la luz de la verdad.

\par 
%\textsuperscript{(1786.1)}
\textsuperscript{161:2.6} 5. Parece conocer los pensamientos de la mente de los hombres y comprender los anhelos de su corazón. Siempre es compasivo con nuestros espíritus perturbados. Parece poseer todas nuestras emociones humanas, pero magníficamente glorificadas. Ama ardientemente la bondad y detesta el pecado con la misma intensidad. Posee una conciencia sobrehumana de la presencia de la Deidad. Reza como un hombre, pero actúa como un Dios. Parece conocer las cosas de antemano; incluso ahora, se atreve a hablar de su muerte, de una referencia mística a su futura glorificación. Aunque es amable, también es valiente e intrépido. Nunca vacila en el cumplimiento de su deber.

\par 
%\textsuperscript{(1786.2)}
\textsuperscript{161:2.7} 6. Estamos constantemente impresionados por el fenómeno de su conocimiento sobrehumano. Casi no pasa un solo día sin que nos enteremos de algo que revela que el Maestro sabe lo que sucede lejos de su presencia inmediata. También parece saber lo que piensan sus asociados. Está indudablemente en comunión con las personalidades celestiales; vive indiscutiblemente en un plano espiritual muy por encima del resto de nosotros. Todo parece estar abierto a su comprensión excepcional. Nos hace preguntas para estimularnos, no para conseguir información.

\par 
%\textsuperscript{(1786.3)}
\textsuperscript{161:2.8} 7. De un tiempo a esta parte, el Maestro no duda en afirmar su naturaleza sobrehumana. Desde el día de nuestra ordenación como apóstoles hasta una época reciente, nunca ha negado que venía del Padre del cielo. Habla con la autoridad de un instructor divino. El Maestro no vacila en refutar las enseñanzas religiosas de hoy en día, y en proclamar el nuevo evangelio con una autoridad positiva. Es asertivo, positivo y está lleno de autoridad. Incluso Juan el Bautista, cuando lo escuchó hablar, declaró que Jesús era el Hijo de Dios. Parece bastarse a sí mismo. No anhela el apoyo de las multitudes; es indiferente a la opinión de los hombres. Es valiente y sin embargo está libre de orgullo.

\par 
%\textsuperscript{(1786.4)}
\textsuperscript{161:2.9} 8. Habla constantemente de Dios como de un asociado siempre presente en todo lo que hace. Circula haciendo el bien, porque Dios parece estar en él. Hace las afirmaciones más asombrosas sobre sí mismo y su misión en la Tierra, unas declaraciones que serían absurdas si no fuera divino. Una vez declaró: <<Antes de que Abraham fuera, yo soy>>. Ha afirmado categóricamente su divinidad; declara estar en asociación con Dios. Agota prácticamente las posibilidades del lenguaje para reiterar sus afirmaciones de que está asociado íntimamente con el Padre celestial. Se atreve incluso a afirmar que él y el Padre son uno solo. Dice que cualquiera que lo ha visto, ha visto al Padre. Dice y hace todas estas cosas extraordinarias con la naturalidad de un niño. Alude a su asociación con el Padre de la misma manera con que se refiere a su asociación con nosotros. Parece estar tan seguro de Dios, y habla de estas relaciones de una manera muy natural.

\par 
%\textsuperscript{(1786.5)}
\textsuperscript{161:2.10} 9. En su vida de oración, parece comunicarse directamente con su Padre. Hemos oído pocas oraciones suyas, pero las pocas que hemos oído dan a entender que habla con Dios, por así decirlo, cara a cara. Parece conocer el futuro tan bien como el pasado. Simplemente no podría ser todo esto, y hacer todas estas cosas extraordinarias, si no fuera algo más que humano. Sabemos que es humano, estamos seguros de eso, pero estamos casi igualmente seguros de que también es divino. Creemos que es divino. Estamos convencidos de que es el Hijo del Hombre y el Hijo de Dios.

\par 
%\textsuperscript{(1787.1)}
\textsuperscript{161:2.11} Cuando Natanael y Tomás hubieron terminado sus conversaciones con Rodán, partieron de prisa para reunirse con sus compañeros apóstoles en Jerusalén, donde llegaron el viernes de aquella semana. Había sido una gran experiencia en la vida de estos tres creyentes, y los otros apóstoles aprendieron mucho cuando Natanael y Tomás les contaron estas experiencias.

\par 
%\textsuperscript{(1787.2)}
\textsuperscript{161:2.12} Rodán regresó a Alejandría, donde enseñó su filosofía durante mucho tiempo en la escuela de Meganta. Llegó a ser un hombre extraordinario en los asuntos posteriores del reino de los cielos; fue un creyente fiel hasta el final de sus días terrestres, y entregó su vida en Grecia con otros creyentes durante el apogeo de las persecuciones.

\section*{3. La mente humana y la mente divina de Jesús}
\par 
%\textsuperscript{(1787.3)}
\textsuperscript{161:3.1} La conciencia de la divinidad se desarrolló de manera gradual en la mente de Jesús hasta el momento de su bautismo. Después de volverse plenamente consciente de su naturaleza divina, de su existencia prehumana y de sus prerrogativas universales, parece ser que poseía el poder de limitar de diversas maneras la conciencia humana de su divinidad. A nosotros nos parece que, desde su bautismo hasta la crucifixión, Jesús dispuso plenamente de la opción de depender exclusivamente de su mente humana, o de utilizar a la vez el conocimiento de la mente humana y de la mente divina. A veces parecía valerse únicamente de la información que poseía su intelecto humano. En otras ocasiones, parecía actuar con tal plenitud de conocimiento y de sabiduría, que sólo la utilización del contenido sobrehumano de su conciencia divina podía proporcionárselo.

\par 
%\textsuperscript{(1787.4)}
\textsuperscript{161:3.2} Sólo podemos comprender sus actuaciones extraordinarias aceptando la teoría de que él mismo podía limitar a voluntad la conciencia de su divinidad. Sabemos plenamente que ocultaba con frecuencia a sus asociados su presciencia de los acontecimientos, y de que era consciente de la naturaleza de los pensamientos y proyectos de sus compañeros. Comprendemos que no deseara que sus seguidores supieran con demasiada certeza que era capaz de discernir sus pensamientos y de penetrar en sus planes. No deseaba trascender con exceso el concepto de lo humano que formaba parte de la mente de sus apóstoles y de sus discípulos.

\par 
%\textsuperscript{(1787.5)}
\textsuperscript{161:3.3} Somos totalmente incapaces de efectuar una diferencia entre su práctica de limitar su conciencia divina, y su técnica para ocultar a sus asociados humanos su preconocimiento y su discernimiento de los pensamientos. Estamos convencidos de que utilizaba ambas técnicas, pero no siempre somos capaces de especificar, en un caso concreto, el método que pudo haber empleado. Observábamos con frecuencia que sólo actuaba con el contenido humano de su conciencia; en otros momentos lo vimos conversar con los dirigentes de las huestes celestiales del universo, y discerníamos el funcionamiento indudable de su mente divina. Y luego, en multitud de ocasiones, presenciamos el funcionamiento de esta personalidad combinada de hombre y de Dios, activada por la unión aparentemente perfecta de su mente humana y de su mente divina. Éste es el límite de nuestro conocimiento sobre estos fenómenos; en realidad no sabemos de hecho toda la verdad sobre este misterio.


\chapter{Documento 162. En la fiesta de los tabernáculos}
\par 
%\textsuperscript{(1788.1)}
\textsuperscript{162:0.1} CUANDO Jesús partió hacia Jerusalén con los diez apóstoles, decidió pasar a través de Samaria porque era el camino más corto. En consecuencia, se dirigieron por la costa oriental del lago, y entraron en la frontera de Samaria a través de Escitópolis. Al anochecer, Jesús envió a Felipe y Mateo a un pueblo situado en las pendientes orientales del Monte Gilboa, para asegurar el alojamiento del grupo. Pero sucedió que aquellos aldeanos tenían grandes prejuicios contra los judíos, más grandes aún que la mayoría de los samaritanos, y estos sentimientos se encontraban exaltados en aquel preciso momento ya que muchos judíos se dirigían a la fiesta de los tabernáculos. Esta gente sabía muy pocas cosas sobre Jesús, y se negaron a alojarlo porque él y sus asociados eran judíos. Cuando Mateo y Felipe manifestaron su indignación e informaron a estos samaritanos de que estaban rechazando hospedar al Santo de Israel, los enfurecidos aldeanos los echaron a palos y pedradas de su pequeña ciudad.

\par 
%\textsuperscript{(1788.2)}
\textsuperscript{162:0.2} Felipe y Mateo regresaron con sus compañeros y les contaron cómo habían sido echados del pueblo; entonces, Santiago y Juan se acercaron a Jesús y le dijeron: <<Maestro, te rogamos que nos des permiso para pedir que caiga fuego del cielo y destruya a esos samaritanos insolentes e impenitentes>>. Cuando Jesús escuchó estas palabras de venganza, se volvió hacia los hijos de Zebedeo y les reprendió con severidad: <<No sabéis el tipo de actitud que estáis manifestando. La venganza no tiene cabida en el reino de los cielos. En lugar de discutir, vamos hacia el pueblecito que se encuentra cerca del vado del Jordán>>. Y así, a causa de sus prejuicios sectarios, estos samaritanos se privaron del honor de ofrecer su hospitalidad al Hijo Creador de un universo.

\par 
%\textsuperscript{(1788.3)}
\textsuperscript{162:0.3} Jesús y los diez se detuvieron para pasar la noche en el pueblo cercano al vado del Jordán. A primeras horas del día siguiente, atravesaron el río y continuaron su camino hacia Jerusalén por la carretera al este del Jordán, llegando a Betania al final de la tarde del miércoles. Tomás y Natanael, que se habían retrasado a causa de sus conversaciones con Rodán, llegaron el viernes.

\par 
%\textsuperscript{(1788.4)}
\textsuperscript{162:0.4} Jesús y los doce permanecieron en las cercanías de Jerusalén hasta el final del mes siguiente (octubre), aproximadamente cuatro semanas y media. El mismo Jesús sólo entró en la ciudad unas pocas veces, y estas breves visitas tuvieron lugar durante los días de la fiesta de los tabernáculos. Una gran parte del mes de octubre la pasó en Belén con Abner y sus asociados.

\section*{1. Los peligros de la visita a Jerusalén}
\par 
%\textsuperscript{(1788.5)}
\textsuperscript{162:1.1} Mucho antes de que huyeran de Galilea, los seguidores de Jesús le habían suplicado que fuera a Jerusalén para proclamar el evangelio del reino, a fin de que su mensaje tuviera el prestigio de haber sido predicado en el centro de la cultura y de la erudición judías; pero ahora que había venido de hecho a Jerusalén para enseñar, temían por su vida. Sabiendo que el sanedrín había intentado llevar a Jesús a Jerusalén para juzgarlo, y al recordar las recientes declaraciones reiteradas del Maestro de que debía someterse a la muerte, los apóstoles se habían quedado literalmente pasmados ante su repentina decisión de asistir a la fiesta de los tabernáculos. A todas sus súplicas anteriores para que fuera a Jerusalén, Jesús había contestado: <<Aún no ha llegado la hora>>. Ahora, ante sus protestas de temor, se limitaba a responder: <<Pero ya ha llegado la hora>>.

\par 
%\textsuperscript{(1789.1)}
\textsuperscript{162:1.2} Durante la fiesta de los tabernáculos, Jesús entró audazmente en Jerusalén en varias ocasiones y enseñó públicamente en el templo. Hizo esto a pesar de los esfuerzos de sus apóstoles por disuadirlo. Aunque le habían insistido durante mucho tiempo para que proclamara su mensaje en Jerusalén, ahora temían verlo entrar en la ciudad en estos momentos, porque sabían muy bien que los escribas y los fariseos estaban decididos a llevarlo a la muerte.

\par 
%\textsuperscript{(1789.2)}
\textsuperscript{162:1.3} La audaz aparición de Jesús en Jerusalén confundió más que nunca a sus seguidores. Muchos discípulos suyos, e incluso el apóstol Judas Iscariote, se habían atrevido a pensar que Jesús había huido precipitadamente a Fenicia porque tenía miedo de los dirigentes judíos y de Herodes Antipas. No comprendían el significado de los desplazamientos del Maestro. Su presencia en Jerusalén en la fiesta de los tabernáculos, incluso en contra de los consejos de sus seguidores, bastó para poner fin definitivamente a todos los rumores sobre su miedo y su cobardía.

\par 
%\textsuperscript{(1789.3)}
\textsuperscript{162:1.4} Durante la fiesta de los tabernáculos, miles de creyentes de todas las partes del imperio romano vieron a Jesús, le oyeron enseñar, y muchos de ellos fueron incluso hasta Betania para conversar con él sobre el progreso del reino en sus regiones nativas.

\par 
%\textsuperscript{(1789.4)}
\textsuperscript{162:1.5} Había muchas razones para que Jesús pudiera predicar públicamente en los patios del templo durante los días de la fiesta; la razón principal era el miedo que se había adueñado de los oficiales del sanedrín a consecuencia de la secreta división de sentimientos que se había producido en sus propias filas. Era un hecho de que muchos miembros del sanedrín creían secretamente en Jesús o bien estaban decididamente en contra de que se le arrestara durante la fiesta, cuando tantísimos visitantes estaban presentes en Jerusalén, muchos de los cuales creían en él o al menos simpatizaban con el movimiento espiritual que patrocinaba.

\par 
%\textsuperscript{(1789.5)}
\textsuperscript{162:1.6} Los esfuerzos de Abner y de sus compañeros a través de Judea también habían contribuido mucho a consolidar un sentimiento favorable hacia el reino, de tal manera que los enemigos de Jesús no se atrevían a manifestar demasiado abiertamente su oposición. Ésta fue una de las razones por las que Jesús pudo visitar públicamente Jerusalén y salir de allí con vida. Uno o dos meses antes, le hubieran dado muerte con toda seguridad.

\par 
%\textsuperscript{(1789.6)}
\textsuperscript{162:1.7} El atrevimiento audaz de Jesús de aparecer públicamente en Jerusalén intimidó a sus enemigos; no estaban preparados para un desafío tan atrevido. Durante este mes, el sanedrín hizo débiles tentativas por arrestar al Maestro en varias ocasiones, pero estos esfuerzos no condujeron a nada. Sus enemigos estaban tan sorprendidos por la inesperada aparición pública de Jesús en Jerusalén, que supusieron que las autoridades romanas le habían prometido su protección. Como sabían que Felipe (el hermano de Herodes Antipas) era casi un discípulo de Jesús, los miembros del sanedrín consideraron que Felipe había obtenido unas promesas para proteger a Jesús de sus enemigos. Antes de que se dieran cuenta de que se habían equivocado al creer que su aparición repentina y audaz en Jerusalén se debía a un acuerdo secreto con los funcionarios romanos, Jesús ya había salido del territorio de su jurisdicción.

\par 
%\textsuperscript{(1789.7)}
\textsuperscript{162:1.8} Sólo los doce apóstoles sabían que Jesús se proponía asistir a la fiesta de los tabernáculos cuando partieron de Magadán. Los otros seguidores del Maestro se quedaron muy asombrados cuando apareció en los patios del templo y empezó a enseñar públicamente, y las autoridades judías se llevaron una sorpresa indescriptible cuando les informaron que estaba enseñando en el templo.

\par 
%\textsuperscript{(1790.1)}
\textsuperscript{162:1.9} Aunque los discípulos de Jesús no esperaban que asistiera a la fiesta, la gran mayoría de los peregrinos que venían de lejos, y que habían oído hablar de él, albergaban la esperanza de poder verlo en Jerusalén. Y no quedaron decepcionados, porque enseñó en diversas ocasiones en el Pórtico de Salomón y en otras partes de los patios del templo. En realidad, estas enseñanzas fueron la proclamación oficial o solemne de la divinidad de Jesús al pueblo judío y al mundo entero.

\par 
%\textsuperscript{(1790.2)}
\textsuperscript{162:1.10} Las opiniones de las multitudes que escuchaban las enseñanzas del Maestro estaban divididas. Unos decían que era un buen hombre; otros, que era un profeta; otros, que era realmente el Mesías; otros decían que era un intrigante malicioso, que desviaba a la gente con sus doctrinas extrañas. Sus enemigos dudaban en acusarlo abiertamente por temor a los creyentes que estaban a su favor, mientras que sus amigos temían reconocerlo abiertamente por temor a los dirigentes judíos, sabiendo que el sanedrín estaba decidido a matarlo. Pero incluso sus enemigos se maravillaban de su enseñanza, pues sabían que no había sido instruido en las escuelas de los rabinos.

\par 
%\textsuperscript{(1790.3)}
\textsuperscript{162:1.11} Cada vez que Jesús iba a Jerusalén, sus apóstoles se llenaban de terror. Día tras día, se sentían más atemorizados cuando escuchaban sus declaraciones cada vez más audaces sobre la naturaleza de su misión en la Tierra. No estaban acostumbrados a escuchar a Jesús hacer unas proclamaciones tan rotundas y unas afirmaciones tan sorprendentes, ni siquiera cuando predicaba entre sus amigos.

\section*{2. El primer discurso en el templo}
\par 
%\textsuperscript{(1790.4)}
\textsuperscript{162:2.1} La primera tarde que Jesús enseñó en el templo, un número considerable de personas estaban sentadas y escuchaban sus palabras describiendo la libertad del nuevo evangelio y la alegría de los que creen en la buena nueva, cuando un oyente curioso le interrumpió para preguntar: <<Maestro, ¿cómo puede ser que puedas citar las Escrituras y enseñar a la gente con tanta facilidad, cuando me dicen que no has sido instruido en la ciencia de los rabinos?>> Jesús contestó: <<Ningún hombre me ha enseñado las verdades que os proclamo. Esta enseñanza no es mía, sino de Aquél que me ha enviado. Si algún hombre desea realmente hacer la voluntad de mi Padre, sabrá con certeza si mi enseñanza viene de Dios o si hablo por mi mismo. El que habla por sí mismo busca su propia gloria, pero cuando proclamo las palabras del Padre, busco así la gloria de aquél que me ha enviado. Pero antes de intentar entrar en la nueva luz, ¿no deberíais seguir más bien la luz que ya poseéis? Moisés os dio la ley, y sin embargo, ¿cuántos de vosotros intentan honradamente satisfacer sus exigencias? En esta ley, Moisés os ordena:
`No matarás'; y a pesar de este mandamiento, algunos de vosotros pretenden matar al Hijo del Hombre>>.

\par 
%\textsuperscript{(1790.5)}
\textsuperscript{162:2.2} Cuando la multitud escuchó estas palabras, empezaron a discutir entre ellos. Algunos decían que estaba loco; otros, que estaba poseído por un demonio. Otros decían que éste era en verdad el profeta de Galilea que los escribas y fariseos intentaban matar desde hacía tiempo. Algunos decían que las autoridades religiosas tenían miedo de molestarlo; otros pensaban que no le habían echado mano porque se habían convertido en creyentes suyos. Después de una discusión prolongada, un miembro de la muchedumbre se adelantó y le preguntó a Jesús: <<Por qué los dirigentes intentan matarte?>> Y él respondió: <<Los dirigentes pretenden matarme porque les indigna mi enseñanza sobre la buena nueva del reino, un evangelio que libera a los hombres de las pesadas tradiciones de una religión formalista de ceremonias, que esos educadores están decididos a mantener a toda costa. Practican la circuncisión, de acuerdo con la ley, el día del sábado, pero quieren matarme porque una vez, en un día de sábado, liberé a un hombre que era esclavo de una aflicción. Me siguen durante el sábado para espiarme, pero quieren matarme porque en otra ocasión escogí curar por completo, un día de sábado, a un hombre que estaba gravemente enfermo. Tratan de matarme porque saben muy bien que si creéis honradamente en mi enseñanza y os atrevéis a aceptarla, su sistema de religión tradicional será derrocado, destruido para siempre. Y así se quedarán privados de autoridad sobre aquello a lo que han consagrado su vida, puesto que se niegan firmemente a aceptar este evangelio nuevo y más glorioso del reino de Dios. Y ahora sí que os lo pido a cada uno de vosotros: No juzguéis por las apariencias exteriores, sino juzgad más bien por el verdadero espíritu de estas enseñanzas; juzgad con rectitud>>.

\par 
%\textsuperscript{(1791.1)}
\textsuperscript{162:2.3} Entonces, otro indagador dijo: <<Sí, Maestro, buscamos al Mesías, pero sabemos que cuando llegue, su aparición se producirá de manera misteriosa. Sabemos de dónde vienes. Has estado entre tus hermanos desde el principio. El libertador vendrá con poder para restaurar el trono del reino de David. ¿Pretendes realmente ser el Mesías?>> Jesús respondió: <<Pretendes conocerme y saber de dónde vengo. Desearía que tus afirmaciones fueran verdaderas, porque entonces sí que encontrarías una vida abundante en ese conocimiento. Pero os aseguro que no he venido hasta vosotros por mí mismo; he sido enviado por el Padre, y aquél que me ha enviado es verdadero y fiel. Cuando os negáis a escucharme, os negáis a recibir a Aquél que me envía. Si recibís este evangelio, llegaréis a conocer a Aquél que me ha enviado. Yo conozco al Padre, porque he venido del Padre para proclamarlo y revelarlo a vosotros>>.

\par 
%\textsuperscript{(1791.2)}
\textsuperscript{162:2.4} Los agentes de los escribas querían prenderlo, pero le tenían miedo a la multitud porque muchos creían en él. La obra de Jesús desde su bautismo era bien conocida en toda la sociedad judía, y cuando mucha de esta gente refería estas cosas, se decían entre ellos: <<Aunque este instructor sea de Galilea, y aunque no satisfaga todas nuestras expectativas del Mesías, nos preguntamos si cuando llegue el libertador hará realmente algo más maravilloso que lo que ya ha hecho este Jesús de Nazaret>>.

\par 
%\textsuperscript{(1791.3)}
\textsuperscript{162:2.5} Cuando los fariseos y sus agentes escucharon al pueblo hablar de esta manera, consultaron con sus dirigentes y decidieron que había que hacer algo inmediatamente para poner fin a estas apariciones públicas de Jesús en los patios del templo. En general, los dirigentes de los judíos estaban dispuestos a evitar un enfrentamiento con Jesús, pues creían que las autoridades romanas le habían prometido la inmunidad. No podían explicarse de otra manera la audacia que tenía de venir en esta época a Jerusalén; pero los funcionarios del sanedrín no creían totalmente en este rumor. Deducían que los gobernantes romanos no hubieran hecho una cosa así en secreto y sin que lo supieran las más altas autoridades de la nación judía.

\par 
%\textsuperscript{(1791.4)}
\textsuperscript{162:2.6} En consecuencia, Eber, el oficial apropiado del sanedrín, fue enviado con dos asistentes para arrestar a Jesús. Mientras Eber se abría paso hacia Jesús, el Maestro dijo: <<No tengas miedo de aproximarte a mí. Acércate mientras escuchas mi enseñanza. Sé que has sido enviado para capturarme, pero deberías comprender que al Hijo del Hombre no le sucederá nada hasta que llegue su hora. Tú no estás en contra mía; sólo vienes a ejecutar la orden de tus superiores, e incluso esos dirigentes de los judíos creen de verdad que están sirviendo a Dios cuando buscan en secreto mi destrucción>>.

\par 
%\textsuperscript{(1792.1)}
\textsuperscript{162:2.7} <<No os tengo aversión a ninguno de vosotros. El Padre os ama, y por eso deseo vuestra liberación de la esclavitud a los prejuicios y a las tinieblas de la tradición. Os ofrezco la libertad de la vida y la alegría de la salvación. Proclamo el nuevo camino viviente, la liberación del mal y la ruptura de la servidumbre del pecado. He venido para que podáis tener la vida, y la tengáis eternamente. Intentáis desembarazaros de mí y de mis enseñanzas inquietantes. ¡Si pudierais daros cuenta de que sólo estaré poco tiempo con vosotros! Dentro de poco volveré hacia Aquél que me ha enviado a este mundo. Entonces, muchos de vosotros me buscaréis con diligencia, pero no descubriréis mi presencia, porque no podéis venir adonde estoy a punto de ir. Pero todos los que traten sinceramente de encontrarme, alcanzarán alguna vez la vida que conduce a la presencia de mi Padre>>.

\par 
%\textsuperscript{(1792.2)}
\textsuperscript{162:2.8} Algunos de los que se mofaban se dijeron entre ellos: <<¿Adónde irá este hombre para que no podamos encontrarlo? ¿Se irá a vivir con los griegos? ¿Se quitará la vida? ¿Qué quiere decir cuando afirma que pronto nos dejará y que no podemos ir adonde él va?>>

\par 
%\textsuperscript{(1792.3)}
\textsuperscript{162:2.9} Eber y sus ayudantes se negaron a detener a Jesús, y regresaron sin él a su lugar de reunión. Por consiguiente, cuando los sacerdotes principales y los fariseos reprendieron a Eber y a sus ayudantes por no haber traído a Jesús, Eber se limitó a contestar: <<Hemos tenido miedo de arrestarlo en medio de la multitud, porque muchos de ellos creen en él. Además, nunca hemos oído a nadie hablar como ese hombre. Ese instructor tiene algo fuera de lo común. Todos haríais bien en ir a escucharlo>>. Cuando los jefes principales escucharon estas palabras, se quedaron sorprendidos y le dijeron sarcásticamente a Eber: <<¿También tú te has extraviado? ¿Estás a punto de creer en ese impostor? ¿Has oído decir que alguno de nuestros sabios o de nuestros dirigentes haya creído en él? ¿Algún escriba o fariseo ha sido engañado por sus hábiles enseñanzas? ¿Cómo puede ser que te dejes influir por la conducta de esa multitud ignorante que no conoce ni la ley ni los profetas? ¿No sabes que esos iletrados están malditos?>> Entonces, Eber contestó: <<Es verdad, señores, pero ese hombre dirige palabras de misericordia y de esperanza a la multitud. Anima a los abatidos, y sus palabras han confortado incluso nuestras almas. ¿Qué puede haber de malo en esas enseñanzas, aunque no sea el Mesías de las Escrituras? Y aún así, ¿es que nuestra ley no exige la equidad? ¿Condenamos a un hombre antes de escucharlo?>> El jefe del sanedrín se encolerizó con Eber y, volviéndose hacia él, le dijo: <<¿Te has vuelto loco? ¿Eres por casualidad también de Galilea? Busca en las Escrituras, y descubrirás que no surge ningún profeta de Galilea, y mucho menos el Mesías>>.

\par 
%\textsuperscript{(1792.4)}
\textsuperscript{162:2.10} El sanedrín se dispersó en la confusión, y Jesús se retiró a Betania para pasar la noche.

\section*{3. La mujer sorprendida en adulterio}
\par 
%\textsuperscript{(1792.5)}
\textsuperscript{162:3.1} Fue durante esta visita a Jerusalén cuando Jesús intervino en el caso de cierta mujer de mala reputación que los acusadores de ella y los enemigos del Maestro trajeron a su presencia. El relato tergiversado que poseéis de este episodio insinúa que los escribas y fariseos habían llevado a esta mujer ante Jesús, y que Jesús los trató de tal manera que daba a entender que estos jefes religiosos de los judíos podían haber sido ellos mismos culpables de inmoralidad. Jesús sabía muy bien que estos escribas y fariseos estaban espiritualmente ciegos y llenos de prejuicios intelectuales a causa de su lealtad a la tradición, pero que había que contarlos entre los hombres más completamente morales de aquella época y de aquella generación.

\par 
%\textsuperscript{(1793.1)}
\textsuperscript{162:3.2} He aquí lo que sucedió en realidad: A primeras horas de la tercera mañana de la fiesta, cuando Jesús se acercaba al templo, se encontró con un grupo de agentes pagados por el sanedrín que arrastraban con ellos a una mujer. Cuando se acercaron, el portavoz dijo: <<Maestro, esta mujer ha sido sorprendida in fraganti cometiendo adulterio. Pues bien, la ley de Moisés ordena que una mujer así debe ser lapidada. Según tú, ¿qué se debería hacer con ella?>>

\par 
%\textsuperscript{(1793.2)}
\textsuperscript{162:3.3} Los enemigos de Jesús planeaban lo siguiente: Si apoyaba la ley de Moisés, la cual exigía que la pecadora que confesaba su falta fuera apedreada, lo enredarían en dificultades con los dirigentes romanos, que habían negado a los judíos el derecho de infligir la pena de muerte sin la aprobación de un tribunal romano. Si prohibía apedrear a la mujer, lo acusarían ante el sanedrín de elevarse por encima de Moisés y de la ley judía. Si permanecía en silencio, lo acusarían de cobardía. Pero el Maestro manejó la situación de tal manera que toda la trama se hizo pedazos por el propio peso de su mezquindad.

\par 
%\textsuperscript{(1793.3)}
\textsuperscript{162:3.4} Esta mujer, en otra época bien parecida, era la esposa de un ciudadano inferior de Nazaret, un hombre que le había causado dificultades a Jesús durante toda su juventud. Después de haberse casado con esta mujer, la forzó de la manera más vergonzosa a ganarse la vida de los dos comerciando con su cuerpo. Había venido a la fiesta de Jerusalén para que su mujer pudiera prostituir así sus encantos físicos y obtener una ganancia monetaria. Había hecho un pacto con los mercenarios de los dirigentes judíos para traicionar así a su propia esposa en su vicio comercializado. Por eso venían con la mujer y su compañero de culpa, a fin de que Jesús cayera en una trampa al efectuar alguna declaración que pudiera ser utilizada contra él en el caso de que fuera arrestado.

\par 
%\textsuperscript{(1793.4)}
\textsuperscript{162:3.5} Jesús examinó superficialmente a la multitud, y vio al marido que se encontraba detrás de los demás. Sabía el tipo de hombre que era y percibió que era cómplice en esta transacción despreciable. Jesús empezó por caminar alrededor de la multitud para acercarse donde se encontraba este marido degenerado, y escribió unas palabras en la arena que le hicieron marcharse precipitadamente. Luego regresó ante la mujer y escribió de nuevo en el suelo en provecho de sus pretendidos acusadores; cuando leyeron sus palabras, también se fueron uno tras otro. Cuando el Maestro escribió por tercera vez en la arena, el compañero de infortunio de la mujer se alejó a su vez, de manera que cuando el Maestro se incorporó después de escribir, observó que la mujer estaba sola delante de él. Jesús dijo: <<Mujer, ¿dónde están tus acusadores? ¿Ya no queda nadie para lapidarte?>> La mujer levantó la mirada y respondió: <<Nadie, Señor>>. Jesús dijo entonces: <<Conozco tu caso, y yo tampoco te condeno. Puedes irte en paz>>. Y esta mujer, llamada Hildana, abandonó a su perverso marido y se unió a los discípulos del reino.

\section*{4. La fiesta de los tabernáculos}
\par 
%\textsuperscript{(1793.5)}
\textsuperscript{162:4.1} La presencia de una gente que venía de todos los rincones del mundo conocido, desde España hasta la India, hacía que la fiesta de los tabernáculos fuera una ocasión ideal para que Jesús proclamara públicamente, por primera vez en Jerusalén, la totalidad de su evangelio. Durante esta fiesta, la gente vivía mucho al aire libre, en cabañas de hojas. Era la fiesta de la cosecha y al tener lugar, como así era, en el frescor de los meses de otoño, los judíos del mundo asistían en mayor número que a la fiesta de la Pascua, al final del invierno, o a la de Pentecostés al principio del verano. Los apóstoles veían, por fin, a su Maestro proclamar audazmente su misión en la Tierra delante, por así decirlo, del mundo entero.

\par 
%\textsuperscript{(1794.1)}
\textsuperscript{162:4.2} Ésta era la fiesta de las fiestas, pues todo sacrificio que no se hubiera efectuado en las otras festividades se podía hacer en este momento. Ésta era la ocasión en que se recibían las ofrendas en el templo; era una combinación de los placeres de las vacaciones y de los ritos solemnes del culto religioso. Era un momento de regocijo racial, mezclado con los sacrificios, los cantos levíticos y los toques solemnes de las trompetas plateadas de los sacerdotes. Por la noche, el impresionante espectáculo del templo y sus multitudes de peregrinos estaba intensamente iluminado por los grandes candelabros que ardían con esplendor en el patio de las mujeres, así como por el resplandor de docenas de antorchas colocadas en los patios del templo. Toda la ciudad estaba decorada alegremente, excepto el castillo romano de Antonia, que dominaba con un contraste siniestro esta escena festiva y de culto. ¡Cuánto odiaban los judíos este recordatorio siempre presente del yugo romano!

\par 
%\textsuperscript{(1794.2)}
\textsuperscript{162:4.3} Durante la fiesta se sacrificaban setenta bueyes, el símbolo de las setenta naciones del mundo pagano. La ceremonia del derramamiento del agua simbolizaba la efusión del espíritu divino. Esta ceremonia del agua tenía lugar después de la procesión de los sacerdotes y levitas a la salida del Sol. Los fieles bajaban por los peldaños que conducían del patio de Israel al patio de las mujeres, mientras sonaba una sucesión de toques en las trompetas de plata. Luego, los fieles continuaban caminando hacia la hermosa puerta, que se abría hacia el patio de los gentiles. Allí se volvían para ponerse frente al oeste, repetir sus cantos y continuar su camino hacia el agua simbólica.

\par 
%\textsuperscript{(1794.3)}
\textsuperscript{162:4.4} Casi cuatrocientos cincuenta sacerdotes, con un número correspondiente de levitas, oficiaban el último día de la fiesta. Al amanecer se congregaban los peregrinos de todas las partes de la ciudad; cada uno llevaba en la mano derecha un haz de mirto, de sauce y de palmas, y en la mano izquierda una rama de manzana del paraíso ---la cidra o <<fruta prohibida>>. Estos peregrinos se dividían en tres grupos para esta ceremonia matutina. Un grupo permanecía en el templo para asistir a los sacrificios de la mañana. Otro grupo bajaba de Jerusalén hasta cerca de Maza para cortar las ramas de sauce destinadas a adornar el altar de los sacrificios. El tercer grupo formaba una procesión que caminaba detrás del sacerdote encargado del agua, que llevaba la jarra de oro destinada a contener el agua simbólica; este grupo salía del templo por Ofel y llegaba hasta cerca de Siloé, donde se encontraba la puerta de la fuente. Después de haber llenado la jarra de oro en el estanque de Siloé, la procesión regresaba al templo, entraba por la puerta del agua y se dirigía directamente al patio de los sacerdotes, donde el sacerdote que llevaba la jarra de agua se unía al sacerdote que llevaba el vino para la ofrenda de la bebida. Estos dos sacerdotes se encaminaban después a los embudos de plata que conducían a la base del altar, y vertían en ellos el contenido de las jarras. La ejecución de este rito de verter el vino y el agua era la señal que esperaban los peregrinos reunidos para empezar a cantar los salmos
113 al 118 inclusive, alternando con los levitas. A medida que repetían estos versos, ondeaban sus haces hacia el altar. Luego se realizaban los sacrificios del día, asociados con la repetición del salmo del día; el último día de la fiesta se repetía el salmo ochenta y dos a partir del quinto verso.

\section*{5. El sermón sobre la luz del mundo}
\par 
%\textsuperscript{(1794.4)}
\textsuperscript{162:5.1} Al anochecer del penúltimo día de la fiesta, cuando la escena se encontraba intensamente iluminada por las luces de los candelabros y de las antorchas, Jesús se levantó en medio de la multitud reunida y dijo:

\par 
%\textsuperscript{(1795.1)}
\textsuperscript{162:5.2} <<Yo soy la luz del mundo. El que me sigue no caminará en las tinieblas, sino que tendrá la luz de la vida. Como os atrevéis a enjuiciarme y asumís el papel de jueces, declaráis que si doy testimonio de mí mismo, mi testimonio no puede ser verdadero. Pero la criatura nunca puede juzgar al Creador. Aunque dé testimonio de mí mismo, mi testimonio es eternamente verdadero, porque sé de dónde he venido, quién soy y adónde voy. Vosotros, que queréis matar al Hijo del Hombre, no sabéis de dónde he venido, quién soy, ni adónde voy. Sólo juzgáis por las apariencias de la carne; no percibís las realidades del espíritu. Yo no juzgo a nadie, ni siquiera a mi mayor enemigo. Pero si decidiera juzgar, mi juicio sería exacto y recto, porque yo no juzgaría solo, sino en asociación con mi Padre que me ha enviado al mundo, y que es la fuente de todo juicio verdadero. Admitís incluso que se puede aceptar el testimonio de dos personas dignas de confianza ---pues bien, doy testimonio de esas verdades, y mi Padre que está en los cielos también lo da. Cuando ayer os dije esto mismo, me preguntasteis en vuestra ignorancia: `¿Dónde está tu Padre?' En verdad, no me conocéis ni a mí ni a mi Padre, porque si me hubierais conocido, habríais conocido también al Padre>>.

\par 
%\textsuperscript{(1795.2)}
\textsuperscript{162:5.3} <<Ya os he dicho que me marcho, y que me buscaréis pero que no me encontraréis, porque allí donde voy no podéis venir. Vosotros, que quisierais rechazar esta luz, sois de abajo; yo soy de arriba. Vosotros, que preferís permanecer en las tinieblas, sois de este mundo; yo no soy de este mundo, y vivo en la luz eterna del Padre de las luces. Todos habéis tenido numerosas oportunidades para saber quién soy, pero tendréis además otras pruebas que confirmarán la identidad del Hijo del Hombre. Yo soy la luz de la vida, y todo aquél que rechaza deliberadamente y a sabiendas esta luz salvadora, morirá en sus pecados. Tengo muchas cosas que deciros, pero sois incapaces de recibir mis palabras. Sin embargo, aquél que me ha enviado es verdadero y fiel; mi Padre ama incluso a sus hijos descarriados. Y todo lo que mi Padre ha dicho, yo también lo proclamo al mundo>>.

\par 
%\textsuperscript{(1795.3)}
\textsuperscript{162:5.4} <<Cuando el Hijo del Hombre sea elevado, entonces todos sabréis que yo soy él, y que no he hecho nada por mí mismo, sino tan sólo lo que el Padre me ha enseñado. Os dirijo estas palabras a vosotros y a vuestros hijos. Aquél que me ha enviado también está ahora conmigo; no me ha dejado solo, porque siempre hago lo que resulta agradable a sus ojos>>.

\par 
%\textsuperscript{(1795.4)}
\textsuperscript{162:5.5} Mientras Jesús enseñaba así a los peregrinos en los patios del templo, muchos creyeron. Y nadie se atrevió a apresarlo.

\section*{6. El discurso sobre el agua de la vida}
\par 
%\textsuperscript{(1795.5)}
\textsuperscript{162:6.1} El último día, el gran día de la fiesta, después de que la procesión del estanque de Siloé pasara por los patios del templo, e inmediatamente después de que los sacerdotes hubieran vertido el agua y el vino en el altar, Jesús, que se hallaba entre los peregrinos, dijo: <<Si alguien tiene sed, que acuda a mí y beba. Traigo a este mundo el agua de la vida que procede del Padre que está en lo alto. El que cree en mí se llenará con el espíritu que este agua representa, porque incluso las Escrituras han dicho: `De él manarán ríos de agua viva'. Cuando el Hijo del Hombre haya terminado su obra en la Tierra, el Espíritu viviente de la Verdad será derramado sobre todo el género humano. Los que reciban este espíritu no conocerán nunca la sed espiritual>>.

\par 
%\textsuperscript{(1795.6)}
\textsuperscript{162:6.2} Jesús no interrumpió el servicio para pronunciar estas palabras. Se dirigió a los fieles inmediatamente después del canto del Halel, la lectura correspondiente de los salmos que era acompañada por el ondear de las ramas delante del altar. Precisamente entonces se hacía una pausa mientras se preparaban los sacrificios, y fue en ese momento cuando los peregrinos escucharon la voz fascinante del Maestro proclamar que él era el dador del agua viva para todas las almas sedientas de espíritu.

\par 
%\textsuperscript{(1796.1)}
\textsuperscript{162:6.3} Al final de este oficio matutino, Jesús continuó enseñando a la multitud, diciendo: <<¿No habéis leído en las Escrituras: `Mirad, así como las aguas descienden sobre la tierra seca y cubren el suelo árido, así os daré el espíritu de santidad para que descienda sobre vuestros hijos y bendiga incluso a los hijos de vuestros hijos?' ¿Por qué tenéis sed del ministerio del espíritu, cuando tratáis de regar vuestra alma con las tradiciones de los hombres, que fluyen de las jarras rotas de los oficios ceremoniales? El espectáculo que veis en este templo es la manera en que vuestros padres intentaron simbolizar la donación del espíritu divino a los hijos de la fe, y habéis hecho bien en perpetuar estos símbolos hasta el día de hoy. Pero ahora, la revelación del Padre de los espíritus ha llegado hasta esta generación a través de la donación de su Hijo, y a todo esto le seguirá con seguridad la donación del espíritu del Padre y del Hijo a los hijos de los hombres. Para todo el que tiene fe, esta donación del espíritu se convertirá en el verdadero instructor del camino que conduce a la vida eterna, a las verdaderas aguas de la vida en el reino del cielo en la Tierra, y en el Paraíso del Padre en el más allá>>.

\par 
%\textsuperscript{(1796.2)}
\textsuperscript{162:6.4} Y Jesús continuó contestando a las preguntas de la multitud y de los fariseos. Algunos pensaban que era un profeta; otros creían que era el Mesías; otros decían que no podía ser el Cristo, ya que venía de Galilea, y que el Mesías debía restablecer el trono de David. Sin embargo, no se atrevieron a arrestarlo.

\section*{7. El discurso sobre la libertad espiritual}
\par 
%\textsuperscript{(1796.3)}
\textsuperscript{162:7.1} La tarde del último día de la fiesta, después de que los apóstoles hubieran fracasado en sus esfuerzos por persuadirlo para que huyera de Jerusalén, Jesús entró de nuevo en el templo para enseñar. Al encontrar un gran grupo de creyentes reunidos en el Pórtico de Salomón, les habló diciendo:

\par 
%\textsuperscript{(1796.4)}
\textsuperscript{162:7.2} <<Si mis palabras permanecen en vosotros y estáis dispuestos a hacer la voluntad de mi Padre, entonces sois realmente mis discípulos. Conoceréis la verdad, y la verdad os hará libres. Sé que vais a contestarme: Somos los hijos de Abraham, y no somos esclavos de nadie; ¿cómo vamos pues a ser liberados? Pero no os hablo de una servidumbre exterior a la autoridad de otro; me refiero a las libertades del alma. En verdad, en verdad os digo que todo aquel que comete pecado es esclavo del pecado. Y sabéis que no es probable que el esclavo resida para siempre en la casa del amo. También sabéis que el hijo permanece en la casa de su padre. Por consiguiente, si el Hijo os libera, y os convierte en hijos, seréis verdaderamente libres>>.

\par 
%\textsuperscript{(1796.5)}
\textsuperscript{162:7.3} <<Sé que sois la semilla de Abraham, y sin embargo vuestros dirigentes intentan matarme porque no han permitido que mi palabra ejerza su influencia transformadora en sus corazones. Sus almas están selladas por los prejuicios y cegadas por el orgullo de la venganza. Os declaro la verdad que me muestra el Padre eterno, mientras que esos educadores engañados sólo tratan de hacer las cosas que han aprendido de sus padres temporales. Cuando contestáis que Abraham es vuestro padre, entonces os digo que, si fuerais los hijos de Abraham, ejecutaríais las obras de Abraham. Algunos de vosotros creéis en mi enseñanza, pero otros tratáis de destruirme porque os he dicho la verdad que he recibido de Dios. Pero Abraham no trató así la verdad de Dios. Percibo que algunos de vosotros estáis decididos a realizar las obras del maligno. Si Dios fuera vuestro Padre, me conoceríais y amaríais la verdad que os revelo. ¿No queréis ver que vengo del Padre, que he sido enviado por Dios, que no estoy haciendo esta obra por mí mismo? ¿Por qué no comprendéis mis palabras? ¿Es porque habéis elegido convertiros en los hijos del mal? Si sois los hijos de las tinieblas, difícilmente podréis caminar en la luz de la verdad que os revelo. Los hijos del maligno sólo siguen los caminos de su padre, que era un farsante y no defendía la verdad, porque no llegó a haber ninguna verdad en él. Pero ahora viene el Hijo del Hombre, que dice y vive la verdad, y muchos de vosotros os negáis a creer>>.

\par 
%\textsuperscript{(1797.1)}
\textsuperscript{162:7.4} <<¿Quién de vosotros me condena por pecador? Si proclamo y vivo la verdad que me muestra el Padre, ¿por qué no creéis? El que es de Dios escucha con placer las palabras de Dios; por eso, muchos de vosotros no escucháis mis palabras, porque no sois de Dios. Vuestros instructores se han atrevido incluso a decir que realizo mis obras por el poder del príncipe de los demonios. Uno que está aquí cerca acaba de decir que poseo un demonio, que soy un hijo del diablo. Pero todos aquellos de vosotros que os comportáis honradamente con vuestra propia alma sabéis muy bien que no soy un diablo. Sabéis que honro al Padre, aunque vosotros quisierais deshonrarme. No busco mi propia gloria, sino únicamente la gloria de mi Padre Paradisiaco. Y no os juzgo, porque hay alguien que juzga por mí>>.

\par 
%\textsuperscript{(1797.2)}
\textsuperscript{162:7.5} <<En verdad, en verdad os digo a vosotros que creéis en el evangelio, que si un hombre conserva viva en su corazón esta palabra de verdad, nunca conocerá la muerte. Ahora, un escriba que está a mi lado dice que esta declaración es la prueba de que poseo un demonio, ya que Abraham está muerto y los profetas también. Y pregunta: `¿Eres mucho más grande que Abraham y los profetas como para atreverte a estar aquí y decir que el que conserva tu palabra no conocerá la muerte? ¿Quién pretendes ser para atreverte a decir tales blasfemias?' A todos los que piensan así les digo que, si me glorifico a mí mismo, mi gloria no vale nada. Pero es el Padre el que me glorificará, el mismo Padre que llamáis Dios. Pero no habéis conseguido conocer a este Dios, vuestro Dios y mi Padre, y he venido para uniros, para mostraros cómo llegar a ser de verdad los hijos de Dios. Aunque no conocéis al Padre, yo lo conozco realmente. Incluso Abraham se alegró de ver mi día, lo vio por la fe y se regocijó>>.

\par 
%\textsuperscript{(1797.3)}
\textsuperscript{162:7.6} Cuando los judíos incrédulos y los agentes del sanedrín que para entonces se habían congregado escucharon estas palabras, provocaron un alboroto, gritando: <<No tienes cincuenta años, y sin embargo hablas de haber visto a Abraham; ¡eres un hijo del diablo!>> Jesús no pudo continuar su discurso. Sólo dijo al partir: <<En verdad, en verdad os lo digo, antes de que Abraham fuera, yo soy>>. Muchos incrédulos corrieron en busca de piedras para arrojárselas, y los agentes del sanedrín trataron de arrestarlo, pero el Maestro se alejó rápidamente por los corredores del templo, y se escapó hacia un lugar de reunión secreto, cerca de Betania, donde lo esperaban Marta, María y Lázaro.

\section*{8. La charla con Marta y María}
\par 
%\textsuperscript{(1797.4)}
\textsuperscript{162:8.1} Se había acordado que Jesús se alojaría con Lázaro y sus hermanas en la casa de un amigo, mientras que los apóstoles se dispersarían aquí y allá en pequeños grupos; habían tomado estas precauciones porque las autoridades judías se atrevían de nuevo a ejecutar sus planes de arrestar al Maestro.

\par 
%\textsuperscript{(1797.5)}
\textsuperscript{162:8.2} Durante años, los tres jóvenes habían tenido la costumbre de dejarlo todo para escuchar la enseñanza de Jesús cada vez que éste tenía la oportunidad de visitarlos. Después de perder a sus padres, Marta había asumido la responsabilidad del hogar, y por eso en esta ocasión, mientras Lázaro y María estaban sentados a los pies de Jesús bebiendo sus enseñanzas vivificantes, Marta se dispuso a servir la cena. Es necesario explicar que Marta se distraía innecesariamente con numerosas tareas superfluas, y que se embrollaba con muchas inquietudes insignificantes; pero era su manera de ser.

\par 
%\textsuperscript{(1798.1)}
\textsuperscript{162:8.3} Mientras Marta estaba ocupada con todos estos supuestos deberes, se sentía inquieta porque María no hacía nada por ayudarla. Por eso se acercó a Jesús y le dijo: <<Maestro, ¿no te importa que mi hermana me haya dejado hacer sola todo el servicio? ¿No quisieras pedirle que venga a ayudarme?>> Jesús respondió: <<Marta, Marta, ¿por qué te inquietas siempre por tantas cosas, y te preocupas por tantas bagatelas? Sólo hay una cosa que vale realmente la pena, y puesto que María ha escogido esta parte buena y necesaria, no se la voy a quitar. Pero, ¿cuándo aprenderéis las dos a vivir como os he enseñado: a servir en cooperación y a refrescar vuestras almas al unísono? ¿No podéis aprender que hay un tiempo para cada cosa ---que las cuestiones secundarias de la vida deben dejar paso a las cosas más grandes del reino celestial?>>

\section*{9. En Belén con Abner}
\par 
%\textsuperscript{(1798.2)}
\textsuperscript{162:9.1} Durante toda la semana que siguió a la fiesta de los tabernáculos, decenas de creyentes se reunieron en Betania y fueron instruidos por los doce apóstoles. El sanedrín no hizo ningún esfuerzo por importunar estas reuniones, ya que Jesús no estaba presente; durante todo este período, estuvo trabajando en Belén con Abner y sus compañeros. Al día siguiente del final de la fiesta, Jesús había partido para Betania y no volvió a enseñar en el templo durante esta visita a Jerusalén.

\par 
%\textsuperscript{(1798.3)}
\textsuperscript{162:9.2} En esta época, Abner tenía su cuartel general en Belén, y desde aquel centro se habían enviado muchos discípulos a las ciudades de Judea y del sur de Samaria, e incluso a Alejandría. A los pocos días de su llegada, Jesús y Abner completaron los acuerdos para consolidar la obra de los dos grupos de apóstoles.

\par 
%\textsuperscript{(1798.4)}
\textsuperscript{162:9.3} Durante toda su visita a la fiesta de los tabernáculos, Jesús había dividido su tiempo casi por igual entre Betania y Belén. En Betania, pasó mucho tiempo con sus apóstoles; en Belén, impartió muchas enseñanzas a Abner y a los otros antiguos apóstoles de Juan. Este contacto íntimo fue lo que les llevó finalmente a creer en él. Estos antiguos apóstoles de Juan el Bautista se sintieron influidos por el coraje que Jesús había mostrado enseñando públicamente en Jerusalén, así como por la amable comprensión que experimentaron durante su enseñanza privada en Belén. Estas influencias conquistaron de manera plena y final a cada uno de los compañeros de Abner, y les llevaron a aceptar de todo corazón el reino y todo lo que implicaba un paso así.

\par 
%\textsuperscript{(1798.5)}
\textsuperscript{162:9.4} Antes de marcharse de Belén por última vez, el Maestro tomó medidas para que todos se asociaran con él en el esfuerzo unido que iba a preceder el final de su carrera terrenal en la carne. Acordaron que Abner y sus compañeros se reunirían pronto con Jesús y los doce en el Parque de Magadán.

\par 
%\textsuperscript{(1798.6)}
\textsuperscript{162:9.5} En conformidad con este acuerdo, a principios de noviembre Abner y sus once compañeros unieron su suerte a la de Jesús y los doce, y trabajaron con ellos como una sola organización hasta el día de la crucifixión.

\par 
%\textsuperscript{(1798.7)}
\textsuperscript{162:9.6} A finales de octubre, Jesús y los doce se alejaron de las proximidades inmediatas de Jerusalén. El domingo 30 de octubre, Jesús y sus asociados dejaron la ciudad de Efraín, donde el Maestro había descansado aislado durante unos días; tomaron la carretera al oeste del Jordán y se dirigieron directamente al Parque de Magadán, donde llegaron al final de la tarde del miércoles 2 de noviembre.

\par 
%\textsuperscript{(1799.1)}
\textsuperscript{162:9.7} Los apóstoles se sintieron muy aliviados por tener al Maestro de vuelta en una región amistosa; nunca más le insistieron para que fuera a Jerusalén a proclamar el evangelio del reino.


\chapter{Documento 163. La ordenación de los setenta en Magadán}
\par 
%\textsuperscript{(1800.1)}
\textsuperscript{163:0.1} POCOS días después de que Jesús y los doce regresaran de Jerusalén a Magadán, Abner y un grupo de unos cincuenta discípulos llegaron de Belén. En ese momento también se encontraban reunidos en el campamento de Magadán el cuerpo de los evangelistas, el cuerpo de mujeres y aproximadamente otros ciento cincuenta discípulos sinceros y probados de todas las regiones de Palestina. Después de consagrar unos días a los contactos personales y a la reorganización del campamento, Jesús y los doce emprendieron un curso de formación intensiva para este grupo especial de creyentes; de este conjunto de discípulos bien preparados y experimentados, el Maestro escogió posteriormente a los setenta educadores y los envió a proclamar el evangelio del reino. Esta instrucción regular empezó el viernes 4 de noviembre y continuó hasta el sábado 19 de noviembre.

\par 
%\textsuperscript{(1800.2)}
\textsuperscript{163:0.2} Jesús daba una charla, todas las mañanas, a este conjunto de personas. Pedro enseñaba los métodos de predicación pública. Natanael los instruía en el arte de enseñar; Tomás explicaba la manera de contestar a las preguntas, y Mateo dirigía la organización de sus finanzas colectivas. Los otros apóstoles participaron también en esta formación según su experiencia especial y sus talentos naturales.

\section*{1. La ordenación de los setenta}
\par 
%\textsuperscript{(1800.3)}
\textsuperscript{163:1.1} El sábado 19 de noviembre por la tarde, Jesús ordenó a los setenta en el campamento de Magadán, y Abner fue puesto al frente de estos predicadores e instructores del evangelio. Este cuerpo de setenta estaba compuesto por Abner y diez antiguos apóstoles de Juan, cincuenta y uno de los primeros evangelistas y otros ocho discípulos que se habían distinguido en el servicio del reino.

\par 
%\textsuperscript{(1800.4)}
\textsuperscript{163:1.2} Hacia las dos de la tarde de este sábado, en medio de chubascos, un grupo de creyentes, acrecentado por la llegada de David y de la mayoría de su cuerpo de mensajeros, en total más de cuatrocientas personas, se congregó en la orilla del lago de Galilea para presenciar la ordenación de los setenta.

\par 
%\textsuperscript{(1800.5)}
\textsuperscript{163:1.3} Antes de imponer sus manos sobre la cabeza de los setenta para diferenciarlos como mensajeros del evangelio, Jesús se dirigió a ellos diciendo: <<En verdad, la cosecha es abundante pero los trabajadores son pocos; por eso os exhorto a todos a que recéis para que el Señor de la cosecha envíe a más trabajadores a su cosecha. Estoy a punto de seleccionaros como mensajeros del reino; estoy a punto de enviaros hacia los judíos y los gentiles como corderos entre lobos. Cuando emprendáis vuestro camino de dos en dos, os recomiendo que no llevéis ni bolsa ni ropa adicional, porque esta primera misión será de corta duración. No saludéis a nadie por el camino, ocupaos únicamente de vuestro trabajo. Siempre que vayáis a quedaros en un hogar, empezad por decir: Que la paz sea en esta casa. Si los que viven allí aman la paz, residiréis allí; si no, entonces partiréis. Cuando hayáis escogido un hogar, quedaos en él durante toda vuestra estancia en esa ciudad, comiendo y bebiendo lo que os ofrezcan. Haréis esto porque el obrero merece su sustento. No os trasladéis de casa en casa porque os ofrezcan un alojamiento mejor. Recordad que al salir a proclamar la paz en la Tierra y la buena voluntad entre los hombres, tendréis que luchar contra unos enemigos encarnizados que se engañan a sí mismos; sed pues tan prudentes como las serpientes y tan inofensivos como las palomas>>.

\par 
%\textsuperscript{(1801.1)}
\textsuperscript{163:1.4} <<Dondequiera que vayáis, predicad diciendo: `El reino de los cielos está cerca', y ayudad a todos los que estén enfermos de la mente o del cuerpo. Habéis recibido gratuitamente las buenas cosas del reino; dad gratuitamente. Si la gente de una ciudad os recibe, encontrarán una entrada abundante en el reino del Padre; pero si la gente de una ciudad se niega a recibir este evangelio, aun así proclamaréis vuestro mensaje en el momento de marcharos de esa comunidad incrédula; a los que rechacen vuestra enseñanza, les diréis al partir: `Aunque rechazáis la verdad, sin embargo el reino de Dios se ha acercado a vosotros.' Quienquiera que os escuche, me escucha a mí. Y quienquiera que me escucha, escucha a Aquél que me ha enviado. El que rechace vuestro mensaje evangélico, me rechaza a mí. Y el que me rechaza a mí, rechaza a Aquél que me ha enviado>>.

\par 
%\textsuperscript{(1801.2)}
\textsuperscript{163:1.5} Después de que Jesús les hubiera hablado así, los setenta se arrodillaron en círculo a su alrededor, e impuso sus manos sobre la cabeza de cada uno de ellos, empezando por Abner.

\par 
%\textsuperscript{(1801.3)}
\textsuperscript{163:1.6} A primeras horas de la mañana siguiente, Abner envió a los setenta mensajeros a todas las ciudades de Galilea, Samaria y Judea. Estas treinta y cinco parejas salieron a predicar y a enseñar durante unas seis semanas, y el viernes 30 de diciembre todos regresaron al nuevo campamento cerca de Pella, en Perea.

\section*{2. El joven rico y otros casos}
\par 
%\textsuperscript{(1801.4)}
\textsuperscript{163:2.1} Más de cincuenta discípulos que deseaban la ordenación y el nombramiento como miembros de los setenta fueron rechazados por el comité que Jesús había designado para seleccionar a estos candidatos. Este comité estaba compuesto por Andrés, Abner y el jefe en activo del cuerpo evangélico. En todos los casos en que este comité de tres miembros no se ponía de acuerdo unánimemente, llevaban al candidato ante Jesús. El Maestro no rechazó a ninguna persona particular que deseara ardientemente la ordenación como mensajero del evangelio, pero después de haber hablado con Jesús, más de una docena de candidatos ya no desearon convertirse en mensajeros del evangelio.

\par 
%\textsuperscript{(1801.5)}
\textsuperscript{163:2.2} Un discípulo ferviente vino a ver a Jesús, diciendo: <<Maestro, quisiera ser uno de tus nuevos apóstoles, pero mi padre es muy anciano y está a punto de morir; ¿se me permitiría volver a mi casa para enterrarlo?>> Jesús le dijo a este hombre: <<Hijo mío, los zorros tienen guaridas y los pájaros del cielo tienen nidos, pero el Hijo del Hombre no tiene donde recostar su cabeza. Eres un discípulo fiel, y puedes continuar siéndolo mientras regresas a tu hogar para cuidar a tus seres queridos, pero no sucede así con los mensajeros de mi evangelio. Lo han abandonado todo para seguirme y proclamar el reino. Si quieres ser ordenado instructor, debes dejar que otros entierren a los muertos mientras sales a anunciar la buena nueva>>. Y este hombre se alejó, muy desilusionado.

\par 
%\textsuperscript{(1801.6)}
\textsuperscript{163:2.3} Otro discípulo vino a ver al Maestro y le dijo: <<Quisiera ser ordenado mensajero, pero me gustaría ir a mi casa durante un corto período de tiempo para confortar a mi familia>>. Jesús replicó: <<Si deseas ser ordenado, debes estar dispuesto a abandonarlo todo. Los mensajeros del evangelio no pueden tener su afecto dividido. Ningún hombre que ha puesto la mano en el arado, y se vuelve atrás, es digno de convertirse en un mensajero del reino>>.

\par 
%\textsuperscript{(1801.7)}
\textsuperscript{163:2.4} Andrés trajo entonces ante Jesús a cierto joven rico que era un fervoroso creyente y deseaba recibir la ordenación. Este joven, llamado Matadormo, era miembro del sanedrín de Jerusalén; había escuchado enseñar a Jesús y posteriormente había sido instruido en el evangelio del reino por Pedro y los otros apóstoles. Jesús habló con Matadormo sobre los requisitos de la ordenación y le pidió que demorara su decisión hasta que hubiera reflexionado más plenamente sobre el asunto. A primeras horas de la mañana siguiente, cuando Jesús salía a dar un paseo, este joven se acercó y le dijo: <<Maestro, quisiera conocer por ti las seguridades de la vida eterna. Puesto que he cumplido todos los mandamientos desde mi juventud, me gustaría saber qué más debo hacer para conseguir la vida eterna>>. En respuesta a esta pregunta, Jesús dijo: <<Si guardas todos los mandamientos ---no cometerás adulterio, no matarás, no robarás, no darás falso testimonio, no engañarás, honrarás a tus padres ---haces bien, pero la salvación es la recompensa de la fe, y no simplemente de las obras. ¿Crees en este evangelio del reino?>> Y Matadormo contestó: <<Sí, Maestro, creo todo lo que tú y tus apóstoles me habéis enseñado>>. Jesús dijo: <<Entonces, eres en verdad mi discípulo y un hijo del reino>>.

\par 
%\textsuperscript{(1802.1)}
\textsuperscript{163:2.5} El joven dijo entonces: <<Pero Maestro, no me conformo con ser tu discípulo; quisiera ser uno de tus nuevos mensajeros>>. Cuando Jesús escuchó esto, lo miró con un gran amor y dijo: <<Haré que seas uno de mis mensajeros si estás dispuesto a pagar el precio, si suples la única cosa que te falta>>. Matadormo respondió: <<Maestro, haré lo que sea para que se me permita seguirte>>. Jesús besó en la frente al joven arrodillado, y le dijo: <<Si quieres ser mi mensajero, ve a vender todo lo que posees; cuando hayas dado el producto a los pobres o a tus hermanos, ven y sígueme, y tendrás un tesoro en el reino de los cielos>>.

\par 
%\textsuperscript{(1802.2)}
\textsuperscript{163:2.6} Cuando Matadormo escuchó estas palabras, su semblante cambió. Se levantó y se alejó apenado, pues tenía grandes posesiones. Este joven fariseo rico había sido criado en la creencia de que la riqueza era el signo del favor de Dios. Jesús sabía que Matadormo no estaba liberado del amor de sí mismo y de sus riquezas. El Maestro quería liberarlo del \textit{amor} a la riqueza, no necesariamente de la riqueza. Aunque los discípulos de Jesús no se deshacían de todos sus bienes terrenales, los apóstoles y los setenta sí lo hacían. Matadormo deseaba ser uno de los setenta nuevos mensajeros, y por ese motivo Jesús le pidió que se deshiciera de todas sus posesiones temporales.

\par 
%\textsuperscript{(1802.3)}
\textsuperscript{163:2.7} Casi todo ser humano tiene alguna cosa a la que se aferra como a un mal favorito, y tiene que renunciar a ella como parte del precio de admisión en el reino de los cielos. Si Matadormo se hubiera deshecho de su riqueza, probablemente hubiera sido puesta de nuevo en sus manos para que la administrara como tesorero de los setenta. Porque más adelante, después del establecimiento de la iglesia en Jerusalén, Matadormo sí obedeció el mandato del Maestro, aunque ya era demasiado tarde para disfrutar de la asociación con los setenta, y se convirtió en el tesorero de la iglesia de Jerusalén, cuyo jefe era Santiago, el hermano carnal del Señor.

\par 
%\textsuperscript{(1802.4)}
\textsuperscript{163:2.8} Siempre ha sido así y siempre será así: Los hombres deben tomar sus propias decisiones. Los mortales pueden hacer uso de cierta gama de posibilidades dentro de la libertad de elección. Las fuerzas del mundo espiritual no desean coaccionar al hombre; le permiten seguir el camino que él mismo ha elegido.

\par 
%\textsuperscript{(1802.5)}
\textsuperscript{163:2.9} Jesús preveía que Matadormo, con sus riquezas, no podría ser de ninguna manera ordenado como compañero de unos hombres que lo habían abandonado todo por el evangelio; al mismo tiempo veía que, sin sus riquezas, se convertiría en el máximo dirigente de todos ellos. Pero, al igual que los mismos hermanos de Jesús, Matadormo nunca llegó a ser grande en el reino porque él mismo se privó de esa asociación íntima y personal con el Maestro que podría haber experimentado si hubiera estado dispuesto a hacer en ese momento lo que Jesús le pedía, cosa que hizo en efecto, pero varios años después.

\par 
%\textsuperscript{(1803.1)}
\textsuperscript{163:2.10} Las riquezas no tienen ninguna relación directa con la entrada en el reino de los cielos, pero el \textit{amor a la riqueza sí tiene que ver}. Las lealtades espirituales hacia el reino son incompatibles con la servidumbre a la codicia materialista. El hombre no puede compartir su lealtad suprema a un ideal espiritual con una devoción material.

\par 
%\textsuperscript{(1803.2)}
\textsuperscript{163:2.11} Jesús no enseñó nunca que fuera malo poseer riquezas. Sólo a los doce y a los setenta les pidió que dedicaran todas sus posesiones terrenales a la causa común. Incluso entonces, se encargó de que sus bienes se liquidaran de una manera ventajosa, como en el caso del apóstol Mateo. Jesús aconsejó muchas veces a sus discípulos acaudalados lo que le había enseñado al hombre rico de Roma. El Maestro consideraba que la inversión sabia de las ganancias sobrantes era una forma legítima de asegurarse contra la inevitable adversidad futura. Cuando la tesorería apostólica estaba desbordante, Judas ponía fondos en depósito para utilizarlos posteriormente en el caso de que los apóstoles sufrieran una gran disminución de los ingresos. Judas hacía esto después de consultarlo con Andrés. Jesús no se ocupó nunca personalmente de las finanzas apostólicas, excepto de los desembolsos destinados a las limosnas. Pero había un abuso económico que condenó muchas veces, y fue la explotación injusta de los hombres débiles, ignorantes y menos afortunados por parte de sus semejantes fuertes, agudos y más inteligentes. Jesús declaró que este tratamiento inhumano de hombres, mujeres y niños era incompatible con los ideales de la fraternidad del reino de los cielos.

\section*{3. La discusión sobre la riqueza}
\par 
%\textsuperscript{(1803.3)}
\textsuperscript{163:3.1} Mientras Jesús terminaba de hablar con Matadormo, Pedro y algunos apóstoles se habían reunido a su alrededor, y cuando el joven rico se hubo marchado, Jesús se volvió hacia los apóstoles y dijo: <<¡Ya veis lo difícil que es para los que tienen riquezas entrar plenamente en el reino de Dios! La adoración espiritual no se puede compartir con las devociones materiales; ningún hombre puede servir a dos señores. Tenéis un dicho que dice que `es más fácil que un camello pase por el ojo de una aguja, a que los paganos hereden la vida eterna.' Y yo declaro que es igual de fácil que ese camello pase por el ojo de la aguja, a que estos ricos satisfechos de sí mismos entren en el reino de los cielos>>.

\par 
%\textsuperscript{(1803.4)}
\textsuperscript{163:3.2} Cuando Pedro y los apóstoles escucharon estas palabras, se quedaron extremadamente sorprendidos, de tal manera que Pedro dijo: <<¿Entonces, Señor, quién puede salvarse? ¿Todos los que tienen riquezas se quedarán fuera del reino?>> Jesús respondió: <<No, Pedro, pero todos los que ponen su confianza en las riquezas, difícilmente entrarán en la vida espiritual que conduce al progreso eterno. Pero aún así, muchas cosas que son imposibles para el hombre, no están fuera del alcance del Padre que está en los cielos; deberíamos reconocer más bien que con Dios todas las cosas son posibles>>.

\par 
%\textsuperscript{(1803.5)}
\textsuperscript{163:3.3} Mientras cada uno se iba por su lado, a Jesús le entristeció que Matadormo no se quedara con ellos, porque lo amaba profundamente. Cuando bajaron al lago, se sentaron al lado del agua y Pedro, hablando en nombre de los doce (que estaban todos presentes en aquel momento), dijo: <<Estamos confundidos por tus palabras al joven rico. ¿Tenemos que exigir a los que quieran seguirte que renuncien a todas sus riquezas terrenales?>> Jesús dijo: <<No, Pedro, sólo a los que quieran convertirse en apóstoles, y deseen vivir conmigo como vosotros lo hacéis, y como una sola familia. Pero el Padre exige que el afecto de sus hijos sea puro e indiviso. Cualquier cosa o persona que se interponga entre vosotros y el amor a las verdades del reino, debe ser abandonada. Si la riqueza que uno posee no invade los recintos del alma, no tiene ninguna consecuencia sobre la vida espiritual de los que desean entrar en el reino>>.

\par 
%\textsuperscript{(1804.1)}
\textsuperscript{163:3.4} Pedro dijo entonces: <<Pero, Maestro, nosotros lo hemos abandonado todo para seguirte; ¿qué poseeremos entonces?>> Jesús se dirigió a la totalidad de los doce, diciendo: <<En verdad, en verdad os digo que no hay nadie que haya abandonado su riqueza, su hogar, a su esposa, a sus hermanos, a sus padres o a sus hijos, por amor por mí y por el reino de los cielos, que no reciba mucho más en este mundo ---quizás con algunas persecuciones--- y la vida eterna en el mundo venidero. Muchos que son los primeros serán los últimos, mientras que los últimos serán a menudo los primeros. El Padre trata a sus criaturas según sus necesidades y de acuerdo con sus justas leyes de consideración misericordiosa y amorosa por el bienestar de un universo>>.

\par 
%\textsuperscript{(1804.2)}
\textsuperscript{163:3.5} <<El reino de los cielos se parece a un propietario que empleaba a muchos hombres, y que salió por la mañana temprano a contratar a unos obreros para que trabajaran en su viña. Después de acordar con los trabajadores que les pagaría un denario por día, los envió a su viña. Luego salió a eso de las nueve, y al ver a otros parados en la plaza del mercado, les dijo: `Id también a trabajar en mi viña, y os pagaré lo que sea justo.' Y fueron inmediatamente a trabajar. El propietario salió de nuevo a eso de las doce y hacia las tres, e hizo lo mismo. Fue a la plaza del mercado alrededor de las cinco de la tarde, encontró a otros obreros parados, y les preguntó: `¿Por qué estáis aquí todo el día sin hacer nada?' Los hombres contestaron: `Porque nadie nos ha contratado.' El propietario dijo entonces: `Id vosotros también a trabajar en mi viña, y os pagaré lo que sea justo.'>>

\par 
%\textsuperscript{(1804.3)}
\textsuperscript{163:3.6} <<Cuando llegó la noche, el propietario de la viña dijo a su administrador: `Llama a los obreros y págales su salario, empezando por los últimos contratados y terminando por los primeros.' Cuando llegaron los que habían sido contratados a eso de las cinco, cada uno recibió un denario, y todos los demás trabajadores recibieron el mismo salario. Cuando los hombres que habían sido contratados al principio del día vieron lo que habían cobrado los últimos en llegar, esperaron recibir más de la cantidad acordada. Pero al igual que los demás, cada hombre sólo recibió un denario. Cuando todos hubieron recibido su paga, se quejaron al propietario, diciendo: `Los últimos hombres que contrataste sólo han trabajado una hora, y sin embargo les has pagado lo mismo que a nosotros, que hemos aguantado todo el día bajo el Sol abrasador.'>>

\par 
%\textsuperscript{(1804.4)}
\textsuperscript{163:3.7} <<El propietario contestó entonces: `Amigos míos, no soy injusto con vosotros. ¿No aceptasteis trabajar por un denario al día? Tomad ahora lo que es vuestro y seguid vuestro camino, porque es mi deseo dar a los últimos que llegaron lo mismo que os he dado a vosotros. ¿No me es lícito hacer lo que desee con lo que es mío? ¿O acaso os molesta mi generosidad, porque deseo ser bondadoso y mostrar misericordia?'>>

\section*{4. La despedida de los setenta}
\par 
%\textsuperscript{(1804.5)}
\textsuperscript{163:4.1} El día que los setenta salieron para efectuar su primera misión fue un momento emocionante en el campamento de Magadán. Aquella mañana temprano, en su última conversación con los setenta, Jesús hizo hincapié en los puntos siguientes:

\par 
%\textsuperscript{(1804.6)}
\textsuperscript{163:4.2} 1. El evangelio del reino debe ser proclamado en el mundo entero, tanto a los gentiles como a los judíos.

\par 
%\textsuperscript{(1804.7)}
\textsuperscript{163:4.3} 2. Cuando cuidéis a los enfermos, absteneos de enseñarles a esperar milagros.

\par 
%\textsuperscript{(1805.1)}
\textsuperscript{163:4.4} 3. Proclamad una fraternidad espiritual de los hijos de Dios, y no un reino exterior de poder mundano y de gloria material.

\par 
%\textsuperscript{(1805.2)}
\textsuperscript{163:4.5} 4. Evitad perder el tiempo mediante un exceso de visitas sociales y otras trivialidades, que podrían disminuir vuestra consagración entusiasta a la predicación del evangelio.

\par 
%\textsuperscript{(1805.3)}
\textsuperscript{163:4.6} 5. Si la primera casa que hayáis elegido como cuartel general resulta ser un hogar digno, permaneced allí durante toda vuestra estancia en esa ciudad.

\par 
%\textsuperscript{(1805.4)}
\textsuperscript{163:4.7} 6. Indicad claramente a todos los creyentes fieles que ha llegado la hora de romper abiertamente con los jefes religiosos de los judíos de Jerusalén.

\par 
%\textsuperscript{(1805.5)}
\textsuperscript{163:4.8} 7. Enseñad que todo el deber del hombre se encuentra resumido en este mandamiento único: Ama al Señor tu Dios con toda tu mente y con toda tu alma, y a tu prójimo como a ti mismo. (Debían enseñar que esto representaba todo el deber del hombre, en lugar de las 613 reglas de vida expuestas por los fariseos.)

\par 
%\textsuperscript{(1805.6)}
\textsuperscript{163:4.9} Después de que Jesús hubiera hablado así a los setenta en presencia de todos los apóstoles y discípulos, Simón Pedro se los llevó aparte y les predicó su sermón de ordenación; fue una ampliación de las instrucciones que les había dado el Maestro en el momento de imponerles las manos y de seleccionarlos como mensajeros del reino. Pedro exhortó a los setenta a que fomentaran, en su experiencia, las virtudes siguientes:

\par 
%\textsuperscript{(1805.7)}
\textsuperscript{163:4.10} 1. \textit{La devoción consagrada}. Orar siempre para que más obreros sean enviados a la cosecha del evangelio. Explicó que, cuando uno ora así, dirá más probablemente: <<Aquí estoy; envíame>>. Les exhortó a que no olvidaran su culto diario.

\par 
%\textsuperscript{(1805.8)}
\textsuperscript{163:4.11} 2. \textit{El coraje verdadero}. Les advirtió que se encontrarían con hostilidades y que estuvieran seguros de que sufrirían persecuciones. Pedro les dijo que su misión no era una empresa para cobardes, y aconsejó a los que tuvieran miedo que se retiraran antes de partir. Pero ninguno desistió.

\par 
%\textsuperscript{(1805.9)}
\textsuperscript{163:4.12} 3. \textit{La fe y la confianza}. Para esta corta misión, debían salir sin recursos ningunos; debían confiar en el Padre para la comida, el alojamiento y todas las demás necesidades.

\par 
%\textsuperscript{(1805.10)}
\textsuperscript{163:4.13} 4. \textit{El ardor y la iniciativa}. Debían estar dominados por un ardor y un entusiasmo inteligente; debían ocuparse estrictamente de los asuntos de su Maestro. El saludo oriental era una ceremonia bastante larga y elaborada; por eso Jesús les había indicado que <<no saludaran a nadie por el camino>>. Se trataba de una expresión corriente para exhortar a alguien a ocuparse de sus asuntos sin perder el tiempo. No tenía nada que ver con la cuestión del saludo amistoso.

\par 
%\textsuperscript{(1805.11)}
\textsuperscript{163:4.14} 5. \textit{La amabilidad y la cortesía}. El Maestro les había ordenado que evitaran perder el tiempo de manera innecesaria en ceremonias sociales, pero les recomendó la cortesía hacia todos aquellos con quienes se pusieran en contacto. Debían mostrar una gran amabilidad con los que los hospedaran en su hogar. Fueron estrictamente advertidos contra el hecho de dejar un hogar modesto para hospedarse en uno más cómodo o más influyente.

\par 
%\textsuperscript{(1805.12)}
\textsuperscript{163:4.15} 6. \textit{La asistencia a los enfermos}. Pedro encargó a los setenta que buscaran a los que estaban enfermos de la mente y del cuerpo, y que hicieran todo lo que pudieran por aliviar o curar sus enfermedades.

\par 
%\textsuperscript{(1805.13)}
\textsuperscript{163:4.16} Una vez que hubieron recibido sus órdenes y sus instrucciones, partieron de dos en dos para realizar su misión en Galilea, Samaria y Judea.

\par 
%\textsuperscript{(1806.1)}
\textsuperscript{163:4.17} Aunque los judíos tenían una estima particular por el número setenta, considerando a veces que las naciones paganas sumaban un total de setenta, y aunque estos setenta mensajeros debían llevar el evangelio a todos los pueblos, sin embargo, por lo que podemos discernir, el hecho de que este grupo comportara exactamente setenta miembros era una simple coincidencia. Es indudable de que Jesús hubiera aceptado a media docena más, pero no estaban dispuestos a pagar el precio de separarse de sus riquezas y de sus familias.

\section*{5. El traslado del campamento a Pella}
\par 
%\textsuperscript{(1806.2)}
\textsuperscript{163:5.1} Jesús y los doce se prepararon ahora para establecer su último cuartel general en Perea, cerca de Pella, donde el Maestro había sido bautizado en el Jordán. Los últimos diez días de noviembre los pasaron deliberando en Magadán, y el martes 6 de diciembre, el grupo entero compuesto por casi trescientas personas partió al amanecer con todos sus efectos para alojarse aquella noche cerca de Pella, al lado del río. Se trataba del mismo lugar, cerca del manantial, que Juan el Bautista había ocupado con su campamento varios años antes.

\par 
%\textsuperscript{(1806.3)}
\textsuperscript{163:5.2} Después de levantarse el campamento de Magadán, David Zebedeo regresó a Betsaida y empezó inmediatamente a reducir el servicio de mensajeros. El reino estaba entrando en una nueva fase. Los peregrinos llegaban diariamente de todas las partes de Palestina e incluso de las regiones remotas del imperio romano. A veces venían creyentes de Mesopotamia y de los territorios situados al este del Tigris. En consecuencia, el domingo 18 de diciembre, con la ayuda de su cuerpo de mensajeros, David cargó en los animales de carga el equipo del campamento, que estaba entonces almacenado en la casa de su padre; era el material con el que había organizado anteriormente el campamento de Betsaida, al lado del lago. Se despidió de Betsaida por un tiempo y descendió por la orilla del lago, y a lo largo del Jordán, hasta un punto situado aproximadamente a un kilómetro al norte del campamento apostólico; en menos de una semana estaba preparado para ofrecer su hospitalidad a cerca de mil quinientos peregrinos visitantes. El campamento apostólico podía alojar a unas quinientas personas. Como era la estación de las lluvias en Palestina, se necesitaban estos alojamientos para cuidar al creciente número de interesados, en su mayoría serios, que venían hasta Perea para ver a Jesús y escuchar su enseñanza.

\par 
%\textsuperscript{(1806.4)}
\textsuperscript{163:5.3} David hizo todo esto por su propia iniciativa, aunque había consultado con Felipe y Mateo en Magadán. La mayor parte de su antiguo cuerpo de mensajeros los empleó como asistentes para dirigir este campamento; ahora utilizaba menos de veinte hombres en el servicio regular de mensajeros. A finales de diciembre y antes de que regresaran los setenta, cerca de ochocientos visitantes estaban congregados alrededor del Maestro, y encontraron alojamiento en el campamento de David.

\section*{6. El regreso de los setenta}
\par 
%\textsuperscript{(1806.5)}
\textsuperscript{163:6.1} El viernes 30 de diciembre, mientras Jesús estaba ausente en las colinas cercanas con Pedro, Santiago y Juan, los setenta mensajeros fueron llegando de dos en dos al cuartel general de Pella, acompañados por numerosos creyentes. Hacia las cinco de la tarde, cuando Jesús regresó al campamento, los setenta estaban reunidos en el lugar dedicado a la enseñanza. La cena se retrasó más de una hora, mientras estos entusiastas del evangelio del reino contaban sus experiencias. Los mensajeros de David habían traído a los apóstoles muchas de estas noticias durante las semanas anteriores, pero fue realmente inspirador escuchar a estos instructores del evangelio, ordenados recientemente, contar en persona cómo los judíos y los gentiles ávidos habían recibido su mensaje. Por fin Jesús podía ver a unos hombres que salían a difundir la buena nueva fuera de su presencia personal. El Maestro sabía ahora que podía dejar este mundo sin obstaculizar seriamente el progreso del reino.

\par 
%\textsuperscript{(1807.1)}
\textsuperscript{163:6.2} Cuando los setenta contaron que <<hasta los demonios se sometían>> a ellos, se referían a las curas maravillosas que habían realizado en los casos de víctimas con trastornos nerviosos. Sin embargo, estos ministros habían aliviado algunos casos de verdadera posesión por los espíritus, y refiriéndose a ellos, Jesús dijo: <<No es de extrañar que esos espíritus menores desobedientes se sometan a vosotros, puesto que he visto a Satanás caer del cielo como un rayo. Pero no os regocijéis tanto por eso, porque os declaro que, en cuanto regrese al lado de mi Padre, enviaremos nuestros espíritus al interior de la mente misma de los hombres para que esos pocos espíritus perdidos ya no puedan penetrar en la mente de los mortales desafortunados. Me regocijo con vosotros de que tengáis influencia sobre los hombres, pero no os sintáis ensalzados por esta experiencia, sino regocijaos más bien porque vuestros nombres están inscritos en los archivos del cielo, y porque vais a avanzar así en una carrera sin fin de conquista espiritual>>.

\par 
%\textsuperscript{(1807.2)}
\textsuperscript{163:6.3} Fue en ese instante, poco antes de compartir la cena, cuando Jesús experimentó uno de esos raros momentos de éxtasis emocional que sus seguidores tuvieran ocasión de presenciar. Dijo: <<Te doy las gracias, Padre mío, Señor del cielo y de la Tierra, porque el espíritu ha revelado estas glorias espirituales a estos hijos del reino, mientras que este evangelio maravilloso era ocultado a los sabios y a los presuntuosos. Sí, Padre mío, debe haber sido agradable a tus ojos hacer esto, y me regocijo al saber que la buena nueva se difundirá por el mundo entero después de que yo haya vuelto a ti y al trabajo que me has encomendado. Estoy extremadamente emocionado cuando me doy cuenta de que estás a punto de poner en mis manos toda la autoridad, que sólo tú sabes realmente quién soy, y que sólo yo te conozco realmente, así como aquellos a quienes te he revelado. Cuando haya finalizado esta revelación a mis hermanos en la carne, la continuaré con tus criaturas del cielo>>.

\par 
%\textsuperscript{(1807.3)}
\textsuperscript{163:6.4} Después de haberle hablado así al Padre, Jesús se volvió para decirle a sus apóstoles y ministros: <<Benditos sean los ojos que ven y los oídos que oyen estas cosas. Dejadme deciros que muchos profetas y muchos grandes hombres de las épocas pasadas desearon contemplar lo que veis ahora, pero no les fue concedido. Y muchas generaciones venideras de hijos de la luz, cuando oigan estas cosas, os envidiarán porque vosotros las habéis visto y oído>>.

\par 
%\textsuperscript{(1807.4)}
\textsuperscript{163:6.5} Luego se dirigió a todos los discípulos, y dijo: <<Habéis oído cuántas ciudades y pueblos han recibido la buena nueva del reino, y cómo han sido recibidos mis ministros e instructores tanto por los judíos como por los gentiles. Benditas son en verdad esas comunidades que han elegido creer en el evangelio del reino. Pero, ¡ay de los habitantes que rechazan la luz en Corazín, Betsaida-Julias y Cafarnaúm, esas ciudades que no han recibido bien a estos mensajeros! Declaro que si las obras poderosas que se han hecho en esos lugares hubieran sido hechas en Tiro y en Sidón, los habitantes de esas ciudades llamadas paganas se habrían arrepentido hace mucho tiempo dándose golpes de pecho. En el día del juicio, el destino de Tiro y de Sidón será por cierto más llevadero>>.

\par 
%\textsuperscript{(1807.5)}
\textsuperscript{163:6.6} Como el día siguiente era sábado, Jesús se reunió aparte con los setenta y les dijo: <<En verdad, me he regocijado con vosotros cuando habéis regresado con las buenas noticias de que el evangelio del reino ha sido acogido por tanta gente diseminada por toda Galilea, Samaria y Judea. Pero, ¿por qué os sentíais tan sorprendentemente exaltados? ¿No esperabais que la comunicación de vuestro mensaje se manifestaría con poder? ¿Salisteis con tan poca fe en este evangelio como para regresar sorprendidos de su eficacia? Y ahora, aunque no quisiera apagar vuestro entusiasmo, deseo advertiros severamente contra las sutilezas del orgullo, del orgullo espiritual. Si pudierais comprender la caída de Lucifer, el inicuo, evitaríais solemnemente todas las formas de orgullo espiritual>>.

\par 
%\textsuperscript{(1808.1)}
\textsuperscript{163:6.7} <<Habéis emprendido la gran tarea de enseñar al hombre mortal que es un hijo de Dios. Os he mostrado el camino; salid a realizar vuestro deber y no os canséis de hacer el bien. A vosotros y a todos los que sigan vuestros pasos a lo largo de los siglos, dejad que os diga que siempre estoy cerca, y que mi convocatoria es, y será siempre: Venid a mí, todos los que os afanáis y lleváis una carga pesada, que yo os daré el descanso. Haced vuestro mi yugo y aprended de mí, pues soy sincero y leal, y encontraréis el descanso espiritual para vuestra alma>>.

\par 
%\textsuperscript{(1808.2)}
\textsuperscript{163:6.8} Cuando pusieron a prueba las promesas del Maestro, comprobaron que sus palabras eran ciertas. Y desde aquel día, un número incalculable de personas también han probado y comprobado la certeza de estas mismas promesas.

\section*{7. Los preparativos para la última misión}
\par 
%\textsuperscript{(1808.3)}
\textsuperscript{163:7.1} Los días siguientes estuvieron llenos de actividad en el campamento de Pella; los preparativos para la misión en Perea se estaban ultimando. Jesús y sus asociados estaban a punto de emprender su última misión, la gira de tres meses por toda Perea, que sólo llegó a su fin cuando el Maestro entró en Jerusalén para llevar a cabo sus últimos trabajos en la Tierra. Durante todo este período, el cuartel general de Jesús y los doce apóstoles se mantuvo aquí, en el campamento de Pella.

\par 
%\textsuperscript{(1808.4)}
\textsuperscript{163:7.2} Jesús ya no tenía necesidad de salir para enseñar a la gente. Ahora acudían a él en cantidades que aumentaban cada semana y procedentes de todas partes, no solamente de Palestina, sino de todo el mundo romano y del próximo oriente. Aunque el Maestro participó con los setenta en la gira por Perea, pasó una gran parte de su tiempo en el campamento de Pella, enseñando a la multitud e instruyendo a los doce. Durante todo este período de tres meses, al menos diez apóstoles permanecieron con Jesús.

\par 
%\textsuperscript{(1808.5)}
\textsuperscript{163:7.3} El cuerpo de mujeres también se preparó para salir de dos en dos, acompañando a los setenta, para trabajar en las ciudades más importantes de Perea. Este grupo original de doce mujeres había entrenado recientemente a un cuerpo más numeroso de cincuenta mujeres en la tarea de visitar los hogares y en el arte de cuidar a los enfermos y a los afligidos. Perpetua, la esposa de Simón Pedro, se unió a esta nueva división del cuerpo de mujeres y le confiaron la dirección de este trabajo femenino más amplio, bajo las órdenes de Abner. Después de Pentecostés, permaneció con su ilustre marido y lo acompañó en todas sus giras misioneras; el día que Pedro fue crucificado en Roma, ella sirvió de alimento a las bestias feroces en la arena. Este nuevo cuerpo de mujeres también contaba entre sus miembros a las esposas de Felipe y Mateo, y a la madre de Santiago y Juan.

\par 
%\textsuperscript{(1808.6)}
\textsuperscript{163:7.4} El trabajo del reino se preparaba ahora para entrar en su fase final bajo la dirección personal de Jesús. Esta fase estaba caracterizada por la profundidad espiritual, en contraste con aquella en que las multitudes, propensas a los milagros y buscadoras de prodigios, seguían al Maestro durante los primeros días de su popularidad en Galilea. Sin embargo, aún había cierto número de seguidores suyos que tenían tendencias materialistas, y que no lograban captar la verdad de que el reino de los cielos es la fraternidad espiritual de los hombres, basada en el hecho eterno de la paternidad universal de Dios.


\chapter{Documento 164. La fiesta de la consagración}
\par 
%\textsuperscript{(1809.1)}
\textsuperscript{164:0.1} MIENTRAS se instalaba el campamento de Pella, Jesús se llevó consigo a Natanael y a Tomás, y subió en secreto a Jerusalén para asistir a la fiesta de la consagración. Los dos apóstoles no se dieron cuenta de que su Maestro se dirigía a Jerusalén hasta que cruzaron el Jordán por el vado de Betania. Cuando percibieron que se proponía realmente estar presente en la fiesta de la consagración, le hicieron los reproches más serios e intentaron disuadirlo utilizando todo tipo de argumentos. Pero sus esfuerzos fueron en vano; Jesús estaba decidido a visitar Jerusalén. A todas sus súplicas y a todas sus advertencias recalcando la locura y el peligro de ponerse voluntariamente entre las manos del sanedrín, él se limitaba a responder: <<Quisiera dar otra oportunidad a esos educadores de Israel para que vean la luz, antes de que llegue mi hora>>.

\par 
%\textsuperscript{(1809.2)}
\textsuperscript{164:0.2} Prosiguieron su camino hacia Jerusalén, mientras los dos apóstoles continuaban expresando sus sentimientos de temor y manifestando sus dudas sobre la sabiduría de esta empresa aparentemente presuntuosa. Llegaron a Jericó hacia las cuatro y media y se prepararon para alojarse allí durante la noche.

\section*{1. La historia del buen samaritano}
\par 
%\textsuperscript{(1809.3)}
\textsuperscript{164:1.1} Aquella noche, un considerable número de personas se reunió alrededor de Jesús y de los dos apóstoles para hacer preguntas; los apóstoles contestaron muchas de ellas, mientras que el Maestro respondió a las demás. En el transcurso de la noche, cierto jurista trató de enredar a Jesús en una discusión comprometedora, diciendo: <<Instructor, me gustaría preguntarte qué debo hacer exactamente para heredar la vida eterna>>. Jesús contestó: <<¿Qué está escrito en la ley y los profetas?; ¿cómo interpretas las Escrituras?>> Conociendo las enseñanzas de Jesús así como las de los fariseos, el jurista respondió: <<Amar al Señor Dios con todo tu corazón, con toda tu alma, con toda tu mente y con todas tus fuerzas, y a tu prójimo como a ti mismo>>. Entonces dijo Jesús: <<Has contestado bien; si lo haces realmente, eso te conducirá a la vida eterna>>.

\par 
%\textsuperscript{(1809.4)}
\textsuperscript{164:1.2} Pero al hacer esta pregunta, el jurista no era totalmente sincero; deseando justificarse y esperando al mismo tiempo desconcertar a Jesús, se atrevió a hacer otra pregunta. Se acercó un poco más al Maestro, y dijo: <<Pero, Instructor, me gustaría que me dijeras quién es exactamente mi prójimo>>. El jurista hizo esta pregunta con la esperanza de que Jesús cayera en la trampa de hacer alguna declaración que infringiera la ley judía, la cual definía al prójimo como <<los hijos de su propio pueblo>>. Los judíos consideraban a todos los demás como <<perros gentiles>>. Este jurista estaba un poco familiarizado con las enseñanzas de Jesús y por eso sabía muy bien que el Maestro pensaba de manera diferente; así pues, esperaba inducirlo a decir algo que se pudiera interpretar como un ataque contra la ley sagrada.

\par 
%\textsuperscript{(1810.1)}
\textsuperscript{164:1.3} Pero Jesús discernía los móviles del jurista, y en lugar de caer en la trampa, procedió a contar a sus oyentes una historia, una historia que podía ser plenamente apreciada por cualquier audiencia de Jericó. Jesús dijo: <<Un hombre que bajaba de Jerusalén a Jericó cayó en manos de unos crueles bandidos que le robaron, lo despojaron, le golpearon y se fueron dejándolo medio muerto. Poco después, un sacerdote bajó por casualidad por aquel camino y llegó hasta donde se encontraba el herido; al ver su estado lastimoso, pasó de largo por el otro lado de la carretera. De la misma manera, cuando llegó un levita también y vio al hombre, pasó de largo por el otro lado. Luego, aproximadamente a esa hora, un samaritano que viajaba hasta Jericó se encontró con el herido, y cuando vio cómo le habían robado y golpeado, se llenó de compasión; se acercó a él, le vendó sus heridas poniéndoles aceite y vino, y colocando al hombre en su propia montura, lo trajo aquí al albergue y cuidó de él. A la mañana siguiente, sacó algún dinero y se lo dio al posadero, diciendo: `Cuida bien de mi amigo, y si los gastos son más elevados, te los pagaré a mi regreso.' Ahora, permíteme preguntarte: ¿Cuál de estos tres resultó ser el prójimo del hombre que cayó en manos de los ladrones?>> Cuando el jurista percibió que había caído en su propia trampa, respondió: <<El que fue misericordioso con él>>. Y Jesús dijo: <<Ve pues y haz lo mismo>>.

\par 
%\textsuperscript{(1810.2)}
\textsuperscript{164:1.4} El jurista respondió <<el que fue misericordioso>> para abstenerse incluso de tener que pronunciar la odiosa palabra de <<samaritano>>. A la pregunta <<¿Quién es mi prójimo?>>, el jurista se vio obligado a dar la respuesta misma que Jesús deseaba, mientras que si Jesús la hubiera contestado, eso lo hubiera implicado directamente en una acusación de herejía. Jesús no solamente confundió al jurista deshonesto, sino que contó a sus oyentes una historia que era, al mismo tiempo, una hermosa advertencia para todos sus seguidores y un impresionante reproche para todos los judíos por su actitud hacia los samaritanos. Y esta historia ha continuado estimulando el amor fraternal entre todos los que han creído posteriormente en el evangelio de Jesús.

\section*{2. En Jerusalén}
\par 
%\textsuperscript{(1810.3)}
\textsuperscript{164:2.1} Jesús había asistido a la fiesta de los tabernáculos para poder proclamar el evangelio a los peregrinos de todas las partes del imperio; ahora iba a la fiesta de la consagración con la única intención de ofrecer al sanedrín y a los dirigentes judíos otra oportunidad para que vieran la luz. El acontecimiento principal de estos pocos días en Jerusalén tuvo lugar el viernes por la noche en la casa de Nicodemo, donde se habían reunido unos veinticinco dirigentes judíos que creían en la enseñanza de Jesús. En este grupo se encontraban catorce hombres que eran entonces, o habían sido recientemente, miembros del sanedrín. Eber, Matadormo y José de Arimatea asistieron a esta reunión.

\par 
%\textsuperscript{(1810.4)}
\textsuperscript{164:2.2} En esta ocasión, todos los oyentes de Jesús eran hombres eruditos, y tanto ellos como los dos apóstoles se asombraron de la amplitud y de la profundidad de las observaciones que el Maestro hizo a este grupo distinguido. Desde la época en que había enseñado en Alejandría, en Roma y en las islas del Mediterráneo, Jesús no había mostrado tanta erudición ni había manifestado una comprensión semejante de los asuntos humanos, tanto laicos como religiosos.

\par 
%\textsuperscript{(1810.5)}
\textsuperscript{164:2.3} Cuando esta pequeña reunión se disolvió, todos se fueron desconcertados por la personalidad del Maestro, encantados con sus modales agradables y enamorados de este ser humano. Habían intentando aconsejar a Jesús en relación con su deseo de conquistar a los restantes miembros del sanedrín. El Maestro escuchó atentamente, pero en silencio, todas sus proposiciones. Sabía muy bien que no funcionaría ninguno de los planes de estas personas. Suponía que la mayoría de los dirigentes judíos nunca aceptaría el evangelio del reino; sin embargo, les proporcionó a todos esta nueva oportunidad para elegir. Pero cuando salió aquella noche con Natanael y Tomás para alojarse en el Monte de los Olivos, el Maestro aún no había decidido el método que iba adoptar para atraer una vez más, sobre su obra, la atención del sanedrín.

\par 
%\textsuperscript{(1811.1)}
\textsuperscript{164:2.4} Natanael y Tomás durmieron poco aquella noche; estaban demasiado impresionados por lo que habían escuchado en la casa de Nicodemo. Pensaron mucho en el comentario final de Jesús relacionado con la oferta de los miembros antiguos y actuales del sanedrín de acompañarlo ante los setenta. El Maestro dijo: <<No, hermanos míos, no serviría para nada. Multiplicaríais la cólera, que recaería sobre vuestras propias cabezas, pero no mitigaríais en lo más mínimo el odio que me tienen. Id cada cual a ocuparos de los asuntos del Padre según os conduzca el espíritu, mientras yo atraeré una vez más su atención sobre el reino de la manera que mi Padre me indique>>.

\section*{3. La curación del mendigo ciego}
\par 
%\textsuperscript{(1811.2)}
\textsuperscript{164:3.1} A la mañana siguiente, los tres fueron a desayunar a la casa de Marta en Betania, y luego se dirigieron inmediatamente a Jerusalén. Este sábado por la mañana, cuando Jesús y sus dos apóstoles se acercaban al templo, se encontraron con un mendigo muy conocido, un hombre que había nacido ciego, que estaba sentado en su lugar de costumbre. Aunque estos mendigos no pedían ni recibían limosnas el día del sábado, se les permitía que se sentaran en sus lugares habituales. Jesús se detuvo y miró al mendigo. Mientras contemplaba a este hombre que había nacido ciego, se le ocurrió una nueva manera de atraer la atención del sanedrín, y de los demás dirigentes judíos e instructores religiosos, sobre su misión en la Tierra.

\par 
%\textsuperscript{(1811.3)}
\textsuperscript{164:3.2} Mientras el Maestro permanecía allí delante del ciego, absorto en sus meditaciones, Natanael reflexionaba sobre la posible causa de la ceguera de este hombre, y preguntó: <<Maestro, para que este hombre naciera ciego, ¿quién pecó, él o sus padres?>>

\par 
%\textsuperscript{(1811.4)}
\textsuperscript{164:3.3} Los rabinos enseñaban que todos estos casos de ceguera de nacimiento estaban causados por el pecado. No sólo los niños eran concebidos y nacían en el pecado, sino que un niño podía nacer ciego como castigo por un pecado determinado cometido por su padre. Enseñaban incluso que el mismo niño podía pecar antes de nacer en el mundo. También enseñaban que estos defectos podían ser causados por algún pecado u otro vicio de la madre mientras estaba embarazada.

\par 
%\textsuperscript{(1811.5)}
\textsuperscript{164:3.4} En todas estas regiones existía una vaga creencia en la reencarnación. Los antiguos educadores judíos, así como Platón, Filón y muchos esenios, toleraban la teoría de que los hombres pueden cosechar en una encarnación lo que han sembrado en una existencia anterior; y así creían que en una vida expiaban los pecados cometidos en las vidas precedentes. El Maestro encontró difícil hacer creer a los hombres que sus almas no habían tenido una existencia anterior.

\par 
%\textsuperscript{(1811.6)}
\textsuperscript{164:3.5} Sin embargo, por muy inconsistente que parezca, aunque se suponía que este tipo de ceguera era el resultado del pecado, los judíos sostenían que era altamente meritorio dar limosnas a estos mendigos ciegos. Estos ciegos tenían la costumbre de cantar constantemente a los que pasaban: <<Oh tiernos de corazón, conseguid méritos ayudando a los ciegos>>.

\par 
%\textsuperscript{(1811.7)}
\textsuperscript{164:3.6} Jesús emprendió la discusión de este caso con Natanael y Tomás, no solamente porque ya había decidido utilizar a este ciego como medio de atraer otra vez aquel día, y de manera sobresaliente, la atención de los dirigentes judíos sobre su misión, sino también porque siempre estimulaba a sus apóstoles a que buscaran las verdaderas causas de todos los fenómenos naturales o espirituales. Les había advertido con frecuencia que evitaran la tendencia común de atribuir a los acontecimientos físicos corrientes unas causas espirituales.

\par 
%\textsuperscript{(1812.1)}
\textsuperscript{164:3.7} Jesús decidió utilizar a este mendigo en sus planes para la obra de aquel día, pero antes de hacer nada por el ciego, cuyo nombre era Josías, empezó por contestar a la pregunta de Natanael. El Maestro dijo: <<Ni este hombre ni sus padres han pecado, para que las obras de Dios puedan manifestarse en él. Esta ceguera le ha sobrevenido en el curso natural de los acontecimientos, pero ahora, mientras que aún es de día, debemos hacer las obras de Aquel que me ha enviado, porque la noche llegará con seguridad, y entonces será imposible hacer el trabajo que estamos a punto de realizar. Mientras estoy en el mundo, yo soy la luz del mundo, pero dentro de poco tiempo ya no estaré con vosotros>>.

\par 
%\textsuperscript{(1812.2)}
\textsuperscript{164:3.8} Cuando Jesús terminó de hablar, dijo a Natanael y a Tomás: <<Vamos a crear la vista de este ciego en este día de sábado, para que los escribas y los fariseos tengan plenamente la oportunidad que buscan para acusar al Hijo del Hombre>>. Entonces se inclinó hacia adelante, escupió en la tierra y mezcló la arcilla con la saliva, mientras hablaba de todo esto para que el ciego pudiera oírle; luego se acercó a Josías y puso la arcilla sobre sus ojos ciegos, diciendo: <<Hijo mío, ve a lavar esta arcilla en el estanque de Siloé, y recibirás inmediatamente la vista>>. Y cuando Josías se hubo lavado así en el estanque de Siloé, volvió junto a sus amigos y su familia, viendo.

\par 
%\textsuperscript{(1812.3)}
\textsuperscript{164:3.9} Como siempre había sido mendigo, no sabía hacer otra cosa; así pues, en cuanto pasó la primera excitación por la creación de su vista, volvió al mismo sitio donde pedía limosnas. Cuando sus amigos, sus vecinos y todos los que lo habían conocido anteriormente observaron que podía ver, todos dijeron: <<¿No es éste Josías, el mendigo ciego?>> Algunos afirmaban que sí, mientras que otros decían: <<No, es uno que se parece a él, pero este hombre puede ver>>. Pero cuando le preguntaron a él mismo, respondió: <<Soy yo>>.

\par 
%\textsuperscript{(1812.4)}
\textsuperscript{164:3.10} Cuando empezaron a preguntarle cómo es que podía ver, les respondió: <<Un hombre llamado Jesús pasó por aquí, y mientras hablaba de mí con sus amigos, hizo arcilla con su saliva, me ungió los ojos y me ordenó que fuera a lavármelos en el estanque de Siloé. Hice lo que este hombre me había dicho, y recibí la vista inmediatamente. Esto ocurrió hace sólo unas horas. Todavía no conozco el significado de muchas cosas que veo>>. Cuando la gente que empezó a congregarse a su alrededor le preguntó dónde podían encontrar al extraño hombre que lo había curado, Josías sólo pudo responder que no lo sabía.

\par 
%\textsuperscript{(1812.5)}
\textsuperscript{164:3.11} Éste es uno de los milagros más extraños de todos los que hizo el Maestro. Este hombre no había pedido que lo curaran. No sabía que el Jesús que le había ordenado que se lavara en Siloé, y que le había prometido la visión, era el profeta de Galilea que había predicado en Jerusalén durante la fiesta de los tabernáculos. Este hombre tenía poca fe en recibir la vista, pero la gente de aquella época tenía mucha fe en la eficacia de la saliva de un gran hombre o de un santo; de la conversación de Jesús con Natanael y Tomás, Josías había concluido que su supuesto benefactor era un gran hombre, un instructor erudito o un santo profeta; por eso hizo lo que Jesús le había ordenado.

\par 
%\textsuperscript{(1812.6)}
\textsuperscript{164:3.12} Jesús tenía tres razones para utilizar la arcilla y la saliva, y para ordenar al ciego que se lavara en el estanque simbólico de Siloé:

\par 
%\textsuperscript{(1812.7)}
\textsuperscript{164:3.13} 1. Este milagro no era una respuesta a la fe personal. Era un prodigio que Jesús decidió realizar con una finalidad escogida por él mismo, pero lo preparó de tal manera que aquel hombre pudiera recibir un beneficio duradero.

\par 
%\textsuperscript{(1813.1)}
\textsuperscript{164:3.14} 2. Como el ciego no había pedido la curación, y puesto que su fe era pequeña, se le habían indicado estos actos materiales con la finalidad de estimularlo. Josías sí creía en la superstición de la eficacia de la saliva, y sabía que el estanque de Siloé era un lugar casi sagrado. Pero difícilmente hubiera ido allí si no hubiera sido necesario lavar la arcilla de la unción. En esta operación había la suficiente ceremonia como para incitarlo a actuar.

\par 
%\textsuperscript{(1813.2)}
\textsuperscript{164:3.15} 3. Pero Jesús tenía un tercer motivo para recurrir a estos medios materiales en relación con esta operación excepcional: Aquél fue un milagro efectuado simplemente en conformidad con su propia elección, y con ello deseaba enseñar a sus seguidores de aquella época, y de todos los siglos posteriores, a no despreciar u olvidar los medios materiales para curar a los enfermos. Quería enseñarles que debían dejar de considerar los milagros como el único método de curar las enfermedades humanas.

\par 
%\textsuperscript{(1813.3)}
\textsuperscript{164:3.16} Jesús concedió la vista a este hombre por medio de una acción milagrosa, este sábado por la mañana y cerca del templo en Jerusalén, con la finalidad principal de hacer que este acto fuera un desafío abierto al sanedrín y a todos los educadores y jefes religiosos judíos.
Ésta fue su manera de proclamar una ruptura abierta con los fariseos. Siempre era positivo en todo lo que hacía. Jesús había llevado a sus dos apóstoles hasta aquel hombre, a primeras horas de la tarde de este sábado, con el propósito de someter estas cuestiones al sanedrín, y provocó deliberadamente las discusiones que obligaron a los fariseos a tener en cuenta este milagro.

\section*{4. Josías ante el sanedrín}
\par 
%\textsuperscript{(1813.4)}
\textsuperscript{164:4.1} A media tarde, la curación de Josías había levantado tal debate alrededor del templo, que los dirigentes del sanedrín decidieron convocar al consejo en su lugar habitual de reunión en el templo. Hicieron esto violando una regla establecida que prohibía las reuniones del sanedrín los días del sábado. Jesús sabía que la violación del sábado sería una de las acusaciones principales que utilizarían contra él cuando llegara la prueba final, y deseaba comparecer ante el sanedrín para que se le juzgara por el cargo de haber curado a un ciego el día del sábado, en el mismo momento en que la alta corte judía, reunida para juzgarlo por este acto de misericordia, estaría deliberando sobre estas cuestiones el día del sábado, violando directamente las leyes que ellos mismos se habían impuesto.

\par 
%\textsuperscript{(1813.5)}
\textsuperscript{164:4.2} Pero no llamaron a Jesús para que se presentara ante ellos; temían hacerlo. En lugar de eso, enviaron a buscar inmediatamente a Josías. Después de algunas preguntas preliminares, el portavoz del sanedrín (estaban presentes unos cincuenta miembros) ordenó a Josías que les contara lo que le había sucedido. Desde que se había curado aquella mañana, Josías se había enterado por Tomás, Natanael y otras personas que los fariseos estaban irritados porque había sido curado un sábado, y que probablemente causarían dificultades a todos los interesados. Pero Josías no percibía todavía que Jesús era aquel a quien llamaban el Libertador. Por eso, cuando los fariseos le interrogaron, dijo: <<Ese hombre vino con otros, puso la arcilla en mis ojos, me dijo que fuera a lavarme en Siloé, y ahora veo>>.

\par 
%\textsuperscript{(1813.6)}
\textsuperscript{164:4.3} Uno de los más ancianos fariseos, después de pronunciar un largo discurso, dijo: <<Ese hombre no puede venir de Dios, porque, como podéis ver, no guarda el sábado. Viola la ley, en primer lugar preparando la arcilla, y luego enviando a este mendigo a lavarse en Siloé el día del sábado. Un hombre así no puede ser un maestro enviado por Dios>>.

\par 
%\textsuperscript{(1813.7)}
\textsuperscript{164:4.4} Entonces, uno de los más jóvenes, que creía en secreto en Jesús, dijo: <<Si ese hombre no ha sido enviado por Dios, ¿cómo puede hacer estas cosas? Sabemos que un vulgar pecador no puede realizar tales milagros. Todos conocemos a este mendigo y sabemos que nació ciego; pero ahora ve. ¿Vais a seguir diciendo que ese profeta realiza todos estos prodigios por el poder del príncipe de los demonios?>> Por cada fariseo que se atrevía a acusar y denunciar a Jesús, había otro que se levantaba para hacer preguntas embarazosas y desconcertantes, de manera que una grave división se produjo entre ellos. El presidente percibió adónde les llevaba el debate, y para apaciguar la discusión se dispuso a hacerle nuevas preguntas al mismo interesado. Volviéndose hacia Josías, le dijo: <<¿Qué tienes que decir de ese hombre, de ese Jesús que según tú te abrió los ojos?>> Y Josías respondió: <<Creo que es un profeta>>.

\par 
%\textsuperscript{(1814.1)}
\textsuperscript{164:4.5} Los dirigentes se quedaron muy inquietos, y no sabiendo qué otra cosa podían hacer, decidieron enviar a buscar a los padres de Josías para saber si éste había nacido realmente ciego. Eran reacios a creer que el mendigo había sido curado.

\par 
%\textsuperscript{(1814.2)}
\textsuperscript{164:4.6} En Jerusalén se sabía muy bien que, no sólo se había prohibido la entrada a Jesús en todas las sinagogas, sino que todos los que creían en su enseñanza también eran expulsados de la sinagoga, excomulgados de la congregación de Israel; esto significaba que se les privaba de todo tipo de derechos y de privilegios en todo el mundo judío, excepto del derecho a comprar lo necesario para vivir.

\par 
%\textsuperscript{(1814.3)}
\textsuperscript{164:4.7} Por esta razón, cuando los padres de Josías, unas pobres almas cargadas de temor, aparecieron ante el augusto sanedrín, tuvieron miedo de hablar libremente. El portavoz del tribunal dijo: <<¿Es éste vuestro hijo? ¿Entendemos acertadamente que nació ciego? Si eso es verdad, ¿cómo puede ser que ahora pueda ver?>> Entonces, el padre de Josías, secundado por la madre, contestó: <<Sabemos que éste es nuestro hijo y que nació ciego, pero en cuanto a la manera en que ha llegado a ver, o quién le ha abierto los ojos, no lo sabemos. Preguntadle a él; es mayor de edad; que hable por sí mismo>>.

\par 
%\textsuperscript{(1814.4)}
\textsuperscript{164:4.8} Entonces llamaron a Josías para que se presentara ante ellos por segunda vez. No conseguían avanzar en su proyecto de celebrar un juicio formal, y algunos empezaron a sentirse molestos por estar haciendo esto el día del sábado; en consecuencia, cuando volvieron a llamar a Josías, intentaron hacerlo caer en una trampa con otro método de ataque. El secretario del tribunal se dirigió al ex ciego, diciéndole: <<¿Por qué no das gloria a Dios por esto? ¿Por qué no nos dices toda la verdad sobre lo que sucedió? Todos sabemos que ese hombre es un pecador. ¿Por qué te niegas a discernir la verdad? Sabes que tanto tú como ese hombre sois culpables de quebrantar el sábado. ¿No quieres expiar tu pecado reconociendo que es Dios quien te ha curado, si todavía pretendes que tus ojos han sido abiertos en el día de hoy?>>

\par 
%\textsuperscript{(1814.5)}
\textsuperscript{164:4.9} Pero Josías no era tonto ni carecía de sentido del humor; por eso respondió al secretario del tribunal: <<No sé si ese hombre es un pecador; pero sí sé una cosa ---que antes era ciego, y que ahora veo>>. Como no podían hacer caer a Josías en una trampa, siguieron interrogándolo, y le preguntaron: <<¿De qué manera exactamente te abrió los ojos? ¿Qué te hizo realmente? ¿Qué te dijo? ¿Te pidió que creyeras en él?>>

\par 
%\textsuperscript{(1814.6)}
\textsuperscript{164:4.10} Josías respondió con un poco de impaciencia: <<Os he dicho exactamente todo lo que sucedió, y si no habéis creído en mi testimonio, ¿por qué queréis escucharlo de nuevo? ¿Acaso queréis también convertiros en discípulos suyos?>> Cuando Josías dijo esto, el sanedrín se disolvió en desorden, casi con violencia, pues los jefes se precipitaron sobre Josías, exclamando furiosamente: <<Tú puedes hablar de ser discípulo de ese hombre, pero nosotros somos discípulos de Moisés, y somos los que enseñamos las leyes de Dios. Sabemos que Dios habló a través de Moisés, pero en cuanto a ese Jesús, no sabemos de dónde viene>>.

\par 
%\textsuperscript{(1814.7)}
\textsuperscript{164:4.11} Entonces Josías se subió en un taburete y gritó a todos los que podían oírle, diciendo: <<Escuchad, vosotros que pretendéis ser los educadores de todo Israel; os aseguro que aquí hay una gran maravilla, puesto que confesáis que no sabéis de dónde viene ese hombre, y sin embargo sabéis con certeza, por el testimonio que habéis escuchado, que me ha abierto los ojos. Todos sabemos que Dios no hace este tipo de obras por los impíos; que Dios sólo haría una cosa así a petición de un adorador verdadero ---por alguien que sea santo y justo. Sabéis que, desde el principio del mundo, nunca se ha oído hablar de que se hayan abierto los ojos a alguien que naciera ciego. ¡Miradme pues, todos vosotros, y daos cuenta de lo que se ha hecho hoy en Jerusalén! Os lo digo, si ese hombre no viniera de Dios, no podría hacer esto>>. Y mientras los miembros del sanedrín se marchaban llenos de ira y de confusión, le gritaron: <<Naciste totalmente en pecado, y ¿ahora pretendes enseñarnos? Quizás no naciste realmente ciego, y aunque tus ojos hayan sido abiertos el día del sábado, ha sido gracias al poder del príncipe de los demonios>>. Y se dirigieron inmediatamente a la sinagoga para echar a Josías.

\par 
%\textsuperscript{(1815.1)}
\textsuperscript{164:4.12} Al principio de este interrogatorio, Josías tenía escasas ideas sobre Jesús y la naturaleza de su curación. La mayor parte del intrépido testimonio que dio con tanta habilidad y valentía, delante de este tribunal supremo de todo Israel, se desarrolló en su mente a medida que el interrogatorio avanzaba de esta manera injusta y desprovista de equidad.

\section*{5. La enseñanza en el Pórtico de Salomón}
\par 
%\textsuperscript{(1815.2)}
\textsuperscript{164:5.1} Mientras esta sesión del sanedrín, que violaba el sábado, se estaba celebrando en una de las cámaras del templo, Jesús estaba paseándose cerca de allí, enseñando a la gente en el Pórtico de Salomón; tenía la esperanza de ser citado ante el sanedrín, donde podría anunciarles la buena nueva de la libertad y la alegría de la filiación divina en el reino de Dios. Pero tenían miedo de enviar a buscarlo. Siempre se sentían desconcertados por estas repentinas apariciones públicas de Jesús en Jerusalén. Jesús les ofrecía ahora la oportunidad que habían buscado con tanto ardor, pero tenían miedo de traerlo ante el sanedrín, aunque fuera como testigo, y tenían aún mucho más miedo de arrestarlo.

\par 
%\textsuperscript{(1815.3)}
\textsuperscript{164:5.2} Se encontraban a mitad del invierno en Jerusalén, y la gente trataba de refugiarse parcialmente en el Pórtico de Salomón; mientras Jesús se demoraba allí, las multitudes le hicieron muchas preguntas, y él les enseñó durante más de dos horas. Algunos educadores judíos intentaron hacerlo caer en una trampa, preguntándole públicamente: <<¿Cuánto tiempo nos tendrás en la incertidumbre? Si eres el Mesías, ¿por qué no nos lo dices claramente?>> Jesús dijo: <<Os he hablado muchas veces de mí y de mi Padre, pero no queréis creerme. ¿No podéis ver que las obras que hago en nombre de mi Padre dan testimonio por mí? Pero muchos de vosotros no creéis porque no pertenecéis a mi rebaño. El instructor de la verdad atrae solamente a los que tienen hambre de verdad y sed de rectitud. Mis ovejas escuchan mi voz, yo las conozco y ellas me siguen. Y a todos los que siguen mi enseñanza, les concedo la vida eterna; nunca perecerán, y nadie los arrebatará de mis manos. Mi Padre, que me ha dado estos hijos, es más grande que todos, de manera que nadie puede arrebatarlos de las manos de mi Padre. El Padre y yo somos uno>>. Algunos judíos incrédulos se precipitaron hacia el lugar donde aún estaban construyendo el templo para coger piedras y arrojárselas a Jesús, pero los creyentes se lo impidieron.

\par 
%\textsuperscript{(1815.4)}
\textsuperscript{164:5.3} Jesús continuó su enseñanza: <<Os he mostrado muchas obras amorosas que provienen del Padre, y ahora quisiera preguntaros ¿por cuál de ésas buenas obras pensáis apedrearme?>> Entonces, uno de los fariseos respondió: <<No queremos apedrearte por ninguna buena obra, sino por blasfemia, porque tú, que eres un hombre, te atreves a igualarte con Dios>>. Y Jesús contestó: <<Acusáis al Hijo del Hombre de blasfemia porque os habéis negado a creerme cuando os he afirmado que Dios me ha enviado. Si no hago las obras de Dios, no me creáis, pero si hago las obras de Dios, aunque no creáis en mí, pensaba que creeríais en las obras. Pero para que estéis seguros de lo que proclamo, dejadme afirmar de nuevo que el Padre está en mí y yo en el Padre, y que de la misma manera que el Padre reside en mí, yo residiré en cada uno de los que creen en este evangelio>>. Cuando la gente escuchó estas palabras, muchos de ellos salieron precipitadamente a coger piedras para arrojárselas, pero él salió por los recintos del templo; se reunió con Natanael y Tomás, que habían asistido a la sesión del sanedrín, y esperó con ellos, cerca del templo, hasta que Josías saliera de la cámara del consejo.

\par 
%\textsuperscript{(1816.1)}
\textsuperscript{164:5.4} Jesús y los dos apóstoles no fueron a buscar a Josías a su casa hasta que se enteraron de que había sido expulsado de la sinagoga. Cuando llegaron a su casa, Tomás lo llamó para que saliera al patio, y Jesús le dijo: <<Josías, ¿crees en el Hijo de Dios?>> Y Josías contestó: <<Dime quién es, para que pueda creer en él>>. Jesús dijo: <<Lo has visto y oído, es el que te habla en este momento>>. Y Josías dijo: <<Señor, yo creo>>, y cayendo de rodillas, le adoró.

\par 
%\textsuperscript{(1816.2)}
\textsuperscript{164:5.5} Cuando Josías se enteró de que había sido expulsado de la sinagoga, al principio se deprimió enormemente, pero se animó mucho cuando Jesús le ordenó que se preparara inmediatamente para acompañarlos al campamento de Pella. Este hombre sencillo de Jerusalén había sido expulsado en verdad de una sinagoga judía, pero he aquí que el Creador de un universo se lo llevaba para asociarlo con la nobleza espiritual de aquel tiempo y de aquella generación.

\par 
%\textsuperscript{(1816.3)}
\textsuperscript{164:5.6} Jesús salió entonces de Jerusalén para no regresar allí hasta poco antes del momento en que se preparó para dejar este mundo. El Maestro volvió a Pella con Josías y los dos apóstoles. Josías demostró ser uno de los que recibieron el ministerio milagroso del Maestro que dió resultados fructíferos, pues se convirtió en un predicador del evangelio del reino durante el resto de su vida.


\chapter{Documento 165. Comienza la misión en Perea}
\par 
%\textsuperscript{(1817.1)}
\textsuperscript{165:0.1} ABNER, el antiguo jefe de los doce apóstoles de Juan el Bautista, nazareo, y en otro tiempo jefe de la escuela nazarea de En-Gedi, era ahora el jefe de los setenta mensajeros del reino. El martes 3 de enero del año 30 reunió a sus asociados y les dio las instrucciones finales antes de enviarlos en misión a todas las ciudades y pueblos de Perea. Esta misión en Perea continuó durante casi tres meses y fue el último ministerio del Maestro. Después de estos trabajos, Jesús fue directamente a Jerusalén para atravesar sus últimas experiencias en la carne. Con el complemento de la periódica labor de Jesús y de los doce apóstoles, los setenta trabajaron en las ciudades y poblaciones siguientes, así como en unos cincuenta pueblos adicionales: Zafón, Gadara, Macad, Arbela, Ramat, Edrei, Bosora, Caspin, Mispé, Gerasa, Ragaba, Sucot, Amatus, Adam, Penuel, Capitolias, Dion, Hatita, Gada, Filadelfia, Jogbeha, Galaad, Bet-Nimra, Tiro, Eleale, Livias, Hesbón, Callirhue, Bet-Peor, Sitim, Sibma, Medeba, Bet-Meón, Areópolis y Aroer.

\par 
%\textsuperscript{(1817.2)}
\textsuperscript{165:0.2} Durante toda esta gira por Perea, el cuerpo de mujeres, que ahora contaba con sesenta y dos miembros, se hizo cargo de la mayor parte de la tarea de cuidar a los enfermos. Éste fue el período final del desarrollo de los aspectos espirituales más elevados del evangelio del reino, y en consecuencia, no se produjo ninguna obra milagrosa. Los apóstoles y discípulos de Jesús no trabajaron tan a fondo en ninguna otra parte de Palestina, y las mejores clases de ciudadanos no aceptaron en ninguna otra región, de manera tan general, la enseñanza del Maestro.

\par 
%\textsuperscript{(1817.3)}
\textsuperscript{165:0.3} En aquella época, la población de Perea estaba compuesta casi por igual de gentiles y judíos, pues los judíos habían sido trasladados generalmente de estas regiones durante los tiempos de Judas Macabeo. Perea era la provincia más hermosa y pintoresca de toda Palestina. Los judíos se referían a ella habitualmente como <<la tierra más allá del Jordán>>.

\par 
%\textsuperscript{(1817.4)}
\textsuperscript{165:0.4} Durante todo este período, Jesús repartió su tiempo entre el campamento de Pella y los desplazamientos con los doce para ayudar a los setenta en las diversas ciudades donde enseñaban y predicaban. Siguiendo las instrucciones de Abner, los setenta bautizaban a todos los creyentes, aunque Jesús no les había encargado que lo hicieran.

\section*{1. En el campamento de Pella}
\par 
%\textsuperscript{(1817.5)}
\textsuperscript{165:1.1} A mediados de enero, más de mil doscientas personas se habían reunido en Pella. Cuando Jesús residía en el campamento, enseñaba a esta multitud al menos una vez al día, y hablaba generalmente a las nueve de la mañana si la lluvia no se lo impedía. Pedro y los demás apóstoles enseñaban todas las tardes. Jesús reservaba las noches para las sesiones habituales de preguntas y respuestas con los doce y otros discípulos avanzados. Los grupos nocturnos tenían un promedio de unas cincuenta personas.

\par 
%\textsuperscript{(1817.6)}
\textsuperscript{165:1.2} A mediados de marzo, momento en que Jesús inició su viaje hacia Jerusalén, más de cuatro mil personas componían la amplia audiencia que escuchaba cada mañana la predicación de Jesús o de Pedro. El Maestro decidió finalizar su obra en la Tierra cuando el interés por su mensaje había llegado a un alto grado, al punto más elevado que había alcanzado durante esta segunda fase del progreso del reino, una fase desprovista de milagros. Aunque tres cuartas partes de la multitud eran buscadores de la verdad, también estaba presente un gran número de fariseos de Jerusalén y de otros lugares, así como numerosos escépticos y sofistas.

\par 
%\textsuperscript{(1818.1)}
\textsuperscript{165:1.3} Jesús y los doce apóstoles consagraron mucho tiempo a la multitud congregada en el campamento de Pella. Los doce prestaron poca o ninguna atención al trabajo de campaña, limitándose a salir con Jesús de vez en cuando para visitar a los asociados de Abner. Abner conocía muy bien la comarca de Perea, puesto que era la zona donde su maestro anterior, Juan el Bautista, había realizado la mayor parte de su obra. Después de empezar la misión en Perea, Abner y los setenta no volvieron nunca más al campamento de Pella.

\section*{2. El sermón sobre el buen pastor}
\par 
%\textsuperscript{(1818.2)}
\textsuperscript{165:2.1} Un grupo de más de trescientos habitantes de Jerusalén, fariseos y otros, había seguido a Jesús hacia el norte hasta Pella cuando se alejó apresuradamente de la jurisdicción de los dirigentes judíos, al final de la fiesta de la consagración; Jesús predicó el sermón sobre el <<Buen Pastor>> en presencia de estos educadores y dirigentes judíos, así como de los doce apóstoles. Después de media hora de debate informal, Jesús dirigió la palabra a un grupo de unas cien personas, diciendo:

\par 
%\textsuperscript{(1818.3)}
\textsuperscript{165:2.2} <<Esta noche tengo muchas cosas que deciros, y puesto que muchos de vosotros sois mis discípulos y algunos de vosotros mis enemigos encarnizados, presentaré mi enseñanza en una parábola para que cada uno pueda coger para sí aquello que reciba su corazón>>.

\par 
%\textsuperscript{(1818.4)}
\textsuperscript{165:2.3} <<Esta noche, aquí delante de mí hay unos hombres que estarían dispuestos a morir por mí y por este evangelio del reino, y algunos de ellos se sacrificarán así en los años venideros; y también estáis aquí algunos de vosotros, esclavos de la tradición, que me habéis seguido desde Jerusalén, y que junto con vuestros jefes, que viven engañados y en las tinieblas, intentáis matar al Hijo del Hombre. La vida que estoy viviendo ahora en la carne os juzgará a los dos grupos, a los verdaderos pastores y a los falsos pastores. Si los falsos pastores fueran ciegos, no tendrían ningún pecado; pero afirmáis que veis; declaráis que sois los educadores de Israel; por eso vuestro pecado permanece en vosotros>>.

\par 
%\textsuperscript{(1818.5)}
\textsuperscript{165:2.4} <<El verdadero pastor reúne a su rebaño en el redil durante la noche en los momentos de peligro. Cuando llega la mañana, entra en el corral por la puerta, y cuando llama, las ovejas conocen su voz. Todo pastor que entra en el corral por otro medio que no sea la puerta, es un ladrón y un saqueador. El verdadero pastor entra en el corral después de que el guardián le ha abierto la puerta, y sus ovejas, que conocen su voz, salen cuando las llama; y cuando las ovejas que le pertenecen están todas fuera, el verdadero pastor va delante de ellas; él muestra el camino y las ovejas le siguen. Sus ovejas le siguen porque conocen su voz; no seguirán a un extraño. Huirán de un extraño porque no conocen su voz. Esta multitud reunida aquí alrededor de nosotros se parece a unas ovejas sin pastor, pero cuando les hablamos, conocen la voz del pastor y nos siguen; al menos lo hacen aquellos que tienen hambre de verdad y sed de rectitud. Algunos de vosotros no pertenecéis a mi redil; no conocéis mi voz y no me seguís. Y como sois falsos pastores, las ovejas no conocen vuestra voz y no quieren seguiros>>.

\par 
%\textsuperscript{(1819.1)}
\textsuperscript{165:2.5} Cuando Jesús hubo contado esta parábola, nadie le hizo ninguna pregunta. Después de un momento, empezó a hablar de nuevo y continuó examinando la parábola:

\par 
%\textsuperscript{(1819.2)}
\textsuperscript{165:2.6} <<Vosotros, que quisierais ser los pastores ayudantes de los rebaños de mi Padre, no solamente tenéis que ser unos jefes meritorios, sino que también tenéis que \textit{alimentar} al rebaño con buena comida; no seréis unos verdaderos pastores a menos que conduzcáis a vuestros rebaños a los verdes pastos y al lado de unas aguas tranquilas>>.

\par 
%\textsuperscript{(1819.3)}
\textsuperscript{165:2.7} <<Y ahora, por temor a que algunos de vosotros comprendáis demasiado fácilmente esta parábola, os declaro que soy ambas cosas a la vez, la puerta que conduce al redil del Padre, y al mismo tiempo el verdadero pastor de los rebaños de mi Padre. Todo pastor que intente entrar sin mí en el corral no lo conseguirá, y las ovejas no escucharán su voz. Yo soy la puerta, junto con aquellos que sirven conmigo. Toda alma que entra en el camino eterno por los medios que yo he creado y ordenado, podrá salvarse y será capaz de continuar hasta alcanzar los eternos pastos del Paraíso>>.

\par 
%\textsuperscript{(1819.4)}
\textsuperscript{165:2.8} <<Pero yo soy también el verdadero pastor que está dispuesto incluso a dar su vida por las ovejas. El ladrón irrumpe en el corral únicamente para robar, matar y destruir; pero yo he venido para que todos podáis tener la vida, y tenerla con más abundancia. Cuando surge el peligro, el asalariado huye y deja que las ovejas se dispersen y sean destruidas; pero el verdadero pastor no huye cuando se acerca el lobo; protege a su rebaño, y si es preciso, da su vida por sus ovejas. En verdad, en verdad os lo digo, amigos y enemigos, yo soy el verdadero pastor; conozco a los míos y los míos me conocen. No huiré frente al peligro. Terminaré este servicio completando la voluntad de mi Padre, y no abandonaré al rebaño que el Padre ha confiado a mi cuidado>>.

\par 
%\textsuperscript{(1819.5)}
\textsuperscript{165:2.9} <<Pero tengo otras muchas ovejas que no pertenecen a este redil, y estas palabras no solamente son verdaderas para este mundo. Esas otras ovejas también escuchan y conocen mi voz, y le he prometido al Padre que todas serán reunidas en un solo redil, en una sola fraternidad de los hijos de Dios. Entonces todos conoceréis la voz de un solo pastor, del verdadero pastor, y todos reconoceréis la paternidad de Dios>>.

\par 
%\textsuperscript{(1819.6)}
\textsuperscript{165:2.10} <<Así sabréis por qué el Padre me ama y ha puesto todos los rebaños de este dominio en mis manos para que los cuide; es porque el Padre sabe que no vacilaré en la protección del redil, que no abandonaré a mis ovejas y que, si es necesario, no dudaré en dar mi vida al servicio de sus múltiples rebaños. Pero, cuidado, si doy mi vida, la tomaré de nuevo. Ningún hombre y ninguna otra criatura pueden quitarme la vida. Tengo el derecho y el poder de dar mi vida, y tengo el mismo poder y el mismo derecho de tomarla de nuevo. No podéis comprender esto, pero he recibido esta autoridad de mi Padre antes incluso de que existiera este mundo>>.

\par 
%\textsuperscript{(1819.7)}
\textsuperscript{165:2.11} Cuando escucharon estas palabras, sus apóstoles se quedaron confundidos, sus discípulos estaban asombrados, mientras que los fariseos de Jerusalén y de los alrededores partieron de noche, diciendo: <<O bien está loco, o está poseído por un demonio>>. Pero sin embargo algunos educadores de Jerusalén dijeron: <<Habla como alguien que tiene autoridad; además, ¿quién ha visto nunca a un poseído por el demonio abrir los ojos de un ciego de nacimiento y hacer todas las cosas maravillosas que este hombre ha hecho?>>

\par 
%\textsuperscript{(1819.8)}
\textsuperscript{165:2.12} Al día siguiente, alrededor de la mitad de estos educadores judíos declararon abiertamente su creencia en Jesús, y la otra mitad regresó consternada a sus hogares de Jerusalén.

\section*{3. El sermón del sábado en Pella}
\par 
%\textsuperscript{(1819.9)}
\textsuperscript{165:3.1} A finales de enero, las multitudes de los sábados por la tarde sumaban casi tres mil personas. El sábado 28 de enero, Jesús predicó el memorable sermón sobre <<La confianza y el estado de preparación espiritual>>. Después de unas observaciones preliminares de Simón Pedro, el Maestro dijo:

\par 
%\textsuperscript{(1820.1)}
\textsuperscript{165:3.2} <<Lo que he dicho muchas veces a mis apóstoles y a mis discípulos, ahora lo proclamo a esta multitud: Guardaos de la influencia de los fariseos, que es la hipocresía nacida de los prejuicios y cultivada en la esclavitud a la tradición; sin embargo, muchos de esos fariseos son honrados de corazón y algunos de ellos permanecen aquí como discípulos míos. Dentro de poco todos comprenderéis mi enseñanza, porque no hay nada que ahora esté oculto que no pueda ser revelado. Lo que ahora está escondido para vosotros, será plenamente conocido cuando el Hijo del Hombre haya concluido su misión en la Tierra y en la carne>>.

\par 
%\textsuperscript{(1820.2)}
\textsuperscript{165:3.3} <<Pronto, muy pronto, las cosas que nuestros enemigos están tramando ahora en secreto y en la oscuridad, saldrán a la luz y serán proclamadas desde los tejados. Pero yo os digo, amigos míos, que no les tengáis miedo cuando traten de destruir al Hijo del Hombre. No temáis a aquellos que, aunque puedan ser capaces de matar el cuerpo, después ya no tienen ningún poder sobre vosotros. Os exhorto a que no temáis a nadie, ni en el cielo ni en la Tierra, sino que os regocijéis en el conocimiento de Aquel que tiene el poder de liberaros de toda injusticia, y de presentaros intachables ante el tribunal de un universo>>.

\par 
%\textsuperscript{(1820.3)}
\textsuperscript{165:3.4} <<¿No se venden cinco gorriones por dos céntimos? Y sin embargo, cuando esos pájaros revolotean buscando su alimento, ni uno de ellos existe sin que lo sepa el Padre, la fuente de toda vida. Para los guardianes seráficos, los cabellos mismos de vuestra cabeza están contados. Si todo esto es verdad, ¿por qué tenéis que vivir con el temor a las muchas pequeñeces que surgen en vuestra vida diaria? Os lo digo: No tengáis miedo; vosotros valéis mucho más que un gran número de gorriones>>.

\par 
%\textsuperscript{(1820.4)}
\textsuperscript{165:3.5} <<A todos los que habéis tenido el valor de confesar vuestra fe en mi evangelio delante de los hombres, yo os reconoceré dentro de poco delante de los ángeles del cielo; pero cualquiera que niegue a sabiendas la verdad de mis enseñanzas delante de los hombres, será renegado por el guardián de su destino hasta delante de los ángeles del cielo>>.

\par 
%\textsuperscript{(1820.5)}
\textsuperscript{165:3.6} <<Decid lo que queráis sobre el Hijo del Hombre, que eso os será perdonado; pero el que se atreva a blasfemar contra Dios, difícilmente encontrará perdón. Cuando los hombres llegan hasta el extremo de atribuir a sabiendas los actos de Dios a las fuerzas del mal, esos rebeldes deliberados difícilmente buscarán el perdón de sus pecados>>.

\par 
%\textsuperscript{(1820.6)}
\textsuperscript{165:3.7} <<Cuando nuestros enemigos os lleven delante de los jefes de las sinagogas y delante de otras altas autoridades, no os preocupéis por lo que tendréis que decir, y no os inquietéis por la manera en que deberéis contestar a sus preguntas, porque el espíritu que reside dentro de vosotros os enseñará sin duda, en esa misma hora, lo que deberéis decir en honor del evangelio del reino>>.

\par 
%\textsuperscript{(1820.7)}
\textsuperscript{165:3.8} <<¿Cuánto tiempo estaréis detenidos en el valle de la decisión? ¿Por qué vaciláis entre dos opiniones? ¿Por qué un judío o un gentil dudaría en aceptar la buena nueva de que es un hijo del Dios eterno? ¿Cuánto tiempo necesitaremos para persuadiros de que entréis con alegría en vuestra herencia espiritual? He venido a este mundo para revelaros el Padre y para conduciros hacia el Padre. Lo primero ya lo he hecho, pero no puedo hacer lo segundo sin vuestro consentimiento; el Padre nunca obliga a nadie a entrar en el reino. La invitación siempre ha sido, y será siempre: Cualquiera que quiera, que venga y comparta libremente el agua de la vida>>.

\par 
%\textsuperscript{(1820.8)}
\textsuperscript{165:3.9} Cuando Jesús hubo terminado de hablar, muchos salieron para ser bautizados por los apóstoles en el Jordán, mientras Jesús escuchaba las preguntas de los que se habían quedado con él.

\section*{4. La división de la herencia}
\par 
%\textsuperscript{(1821.1)}
\textsuperscript{165:4.1} Mientras los apóstoles bautizaban a los creyentes, el Maestro hablaba con los que permanecían allí. Y cierto joven le dijo: <<Maestro, mi padre ha muerto dejándonos muchos bienes a mi hermano y a mí, pero mi hermano se niega a darme lo que me pertenece. ¿Quieres pedirle a mi hermano que comparta esta herencia conmigo?>> A Jesús le indignó un poco que este joven materialista trajera al debate este asunto de negocios; pero aprovechó la ocasión para impartir una enseñanza adicional. Jesús dijo: <<Hombre, ¿quién me ha encargado de repartir vuestras cosas? ¿De dónde has sacado la idea de que me ocupo de los asuntos materiales de este mundo?>> Entonces, volviéndose hacia todos los que estaban a su alrededor, dijo: <<Tened cuidado y guardaos de la codicia; la vida de un hombre no consiste en la abundancia de los bienes que pueda poseer. La felicidad no procede del poder de la fortuna, y la alegría no proviene de las riquezas. La fortuna en sí misma no es una maldición, pero el amor a las riquezas conduce muchas veces a tal dedicación a las cosas de este mundo, que el alma se vuelve ciega ante los hermosos atractivos de las realidades espirituales del reino de Dios en la Tierra, y ante las alegrías de la vida eterna en el cielo>>.

\par 
%\textsuperscript{(1821.2)}
\textsuperscript{165:4.2} <<Dejadme que os cuente la historia de cierto hombre rico cuya tierra producía con mucha abundancia; cuando se volvió muy rico, empezó a razonar consigo mismo, diciendo: `¿Qué voy a hacer con todas mis riquezas? Ahora tengo tantas, que ya no tengo sitio para almacenar mi fortuna.' Después de meditar sobre este problema, dijo: `Voy a hacer esto: derribaré mis graneros y construiré unos más grandes, y así tendré sitio suficiente para guardar mis frutos y mis bienes. Entonces podré decir a mi alma: alma, tienes una gran fortuna acumulada para muchos años; descansa ahora; come, bebe y regocíjate, porque eres rica y con tus bienes en aumento.'>>

\par 
%\textsuperscript{(1821.3)}
\textsuperscript{165:4.3} <<Pero este hombre rico también era tonto. Mientras abastecía las necesidades materiales de su mente y de su cuerpo, había olvidado acumular tesoros en el cielo para la satisfacción de su espíritu y la salvación de su alma. E incluso así, tampoco iba a gozar del placer de consumir sus riquezas acumuladas, porque aquella misma noche se le requirió su alma. Aquella noche llegaron unos bandidos que irrumpieron en su casa para matarlo, y después de que hubieron saqueado sus graneros, incendiaron lo que quedaba. En cuanto a las propiedades que se salvaron de los ladrones, sus herederos se las disputaron entre ellos. Este hombre había acumulado tesoros para sí mismo en la Tierra, pero no era rico para con Dios>>.

\par 
%\textsuperscript{(1821.4)}
\textsuperscript{165:4.4} Jesús trató así al joven y su herencia porque sabía que su problema era la codicia. Pero si éste no hubiera sido el caso, el Maestro no habría intervenido, porque nunca se entrometía en los asuntos temporales ni siquiera de sus apóstoles, y mucho menos de sus discípulos.

\par 
%\textsuperscript{(1821.5)}
\textsuperscript{165:4.5} Cuando Jesús hubo terminado su relato, otro hombre se levantó y le preguntó: <<Maestro, sé que tus apóstoles han vendido todas sus posesiones terrenales para seguirte, y que tienen todas las cosas en común, como hacen los esenios; pero ¿quieres que todos nosotros, que somos tus discípulos, hagamos lo mismo? ¿Es pecado poseer una fortuna honesta?>> Jesús respondió a esta pregunta: <<Amigo mío, no es un pecado tener una fortuna honorable; pero sí es un pecado convertir la riqueza de las posesiones materiales en unos \textit{tesoros} que pueden absorber tus intereses y desviar tu afecto de la devoción a los asuntos espirituales del reino. No hay ningún pecado en tener posesiones honradas en la Tierra, con tal que tu \textit{tesoro} esté en el cielo, porque allí donde esté tu tesoro, allí estará también tu corazón. Existe una gran diferencia entre la riqueza que conduce a la avaricia y al egoísmo, y la riqueza que tienen y reparten con espíritu de administradores aquellos que poseen una abundancia de bienes de este mundo, y que contribuyen tan generosamente a sostener a los que dedican todas sus energías a la obra del reino. Muchos de vosotros, que estáis aquí presentes y sin dinero, recibís la comida y el alojamiento en esa ciudad de tiendas porque unos hombres y mujeres generosos, con medios económicos, han entregado sus fondos para esa finalidad a vuestro anfitrión David Zebedeo>>.

\par 
%\textsuperscript{(1822.1)}
\textsuperscript{165:4.6} <<Pero no olvidéis nunca que, después de todo, la riqueza no es duradera. Con demasiada frecuencia, el amor a las riquezas oscurece la visión espiritual, e incluso la destruye. No dejéis de reconocer el peligro de que el dinero se convierta en vuestro dueño, en lugar de ser vuestro servidor>>.

\par 
%\textsuperscript{(1822.2)}
\textsuperscript{165:4.7} Jesús no enseñó ni apoyó la imprevisión, la ociosidad, la indiferencia en satisfacer las necesidades materiales de nuestra familia, o la dependencia de las limosnas. Pero sí enseñó que las cosas materiales y temporales deben estar subordinadas al bienestar del alma y al progreso de la naturaleza espiritual en el reino de los cielos.

\par 
%\textsuperscript{(1822.3)}
\textsuperscript{165:4.8} Luego, mientras la gente bajaba al río para presenciar los bautismos, el primer joven vino a ver a Jesús en privado para hablar de su herencia, ya que consideraba que Jesús lo había tratado con dureza; después de haberle escuchado de nuevo, el Maestro dijo: <<Hijo mío, ¿por qué desaprovechas la ocasión de alimentarte con el pan de la vida en un día como éste, a fin de satisfacer tu tendencia a la codicia? ¿No sabes que las leyes judías sobre la herencia serán administradas con justicia si te presentas con tu queja en el tribunal de la sinagoga? ¿No puedes ver que mi trabajo consiste en asegurarme de que estás informado acerca de tu herencia celestial? No has leído en las Escrituras: `Hay quien se hace rico gracias a su precaución y a muchas privaciones, y ésta es la parte de su recompensa, puesto que dice: He encontrado el descanso y ahora podré comer continuamente mis bienes, pero sin embargo no sabe lo que el tiempo le traerá, y que también deberá abandonar todas esas cosas a otros cuando muera.' No has leído el mandamiento: `No codiciarás.' Y también: `Han comido, se han hartado y han engordado, y luego se han vuelto hacia otros dioses.' Has leído en los Salmos que `el Señor aborrece a los codiciosos' y que `lo poco que posee un hombre justo es mejor que las riquezas de muchos malvados.' `Si tus riquezas aumentan, no pongas tu corazón en ellas.' Has leído lo que dice Jeremías: `Que el rico no se glorifique en sus riquezas'; y Ezequiel expresó la verdad cuando dijo: `Con sus labios hacen alarde de amor, pero sus corazones están centrados en sus propios beneficios egoístas>>.

\par 
%\textsuperscript{(1822.4)}
\textsuperscript{165:4.9} Jesús despidió al joven, diciéndole: <<Hijo mío, ¿de qué te servirá ganar el mundo entero, si pierdes tu propia alma?>>

\par 
%\textsuperscript{(1822.5)}
\textsuperscript{165:4.10} Otro oyente que se hallaba cerca preguntó cómo serían tratados los ricos en el día del juicio, y Jesús respondió: <<No he venido para juzgar ni a los ricos ni a los pobres, sino que la vida que viven los hombres los juzgará a todos. Cualquier otra cosa que concierna al juicio de los ricos, todos los que hayan adquirido una gran fortuna deberán responder al menos a las tres preguntas siguientes:>>

\par 
%\textsuperscript{(1822.6)}
\textsuperscript{165:4.11} <<1. ¿Cuánta riqueza has acumulado?>>

\par 
%\textsuperscript{(1822.7)}
\textsuperscript{165:4.12} <<2. ¿Cómo has conseguido esa riqueza?>>

\par 
%\textsuperscript{(1822.8)}
\textsuperscript{165:4.13} <<3. ¿Cómo has empleado tu riqueza?>>

\par 
%\textsuperscript{(1822.9)}
\textsuperscript{165:4.14} Luego, Jesús se retiró a su tienda para descansar un rato antes de la cena. Cuando los apóstoles terminaron de bautizar, vinieron también y habrían conversado con él sobre la riqueza en la Tierra y los tesoros en el cielo, pero el Maestro estaba dormido.

\section*{5. Las conversaciones con los apóstoles sobre la riqueza}
\par 
%\textsuperscript{(1823.1)}
\textsuperscript{165:5.1} Aquella noche después de la cena, cuando Jesús y los doce se reunieron para celebrar su conferencia diaria, Andrés preguntó: <<Maestro, mientras bautizábamos a los creyentes, dijiste muchas cosas a la multitud que permanecía contigo, que nosotros no escuchamos. ¿Estarías dispuesto a repetir esas palabras en beneficio nuestro?>> En respuesta a la petición de Andrés, Jesús dijo:

\par 
%\textsuperscript{(1823.2)}
\textsuperscript{165:5.2} <<Sí, Andrés, voy a hablaros sobre estas cuestiones relacionadas con la riqueza y el sustento, pero lo que os voy a decir a vosotros, mis apóstoles, será un poco diferente a lo que dije a los discípulos y a la multitud, puesto que vosotros lo habéis abandonado todo, no sólo para seguirme, sino para ser ordenados como embajadores del reino. Ya habéis tenido una experiencia de varios años, y sabéis que el Padre, cuyo reino proclamáis, no os abandonará. Habéis dedicado vuestra vida al ministerio del reino; por ello, no os inquietéis ni os preocupéis por las cosas de la vida temporal, por lo que vais a comer ni tampoco por cómo vestiréis vuestro cuerpo. El bienestar del alma vale más que la comida y la bebida; el progreso en el espíritu está muy por encima de la necesidad de ropa. Cuando os sintáis tentados a poner en duda la seguridad de vuestro pan, pensad en los cuervos; no siembran ni cosechan, no tienen almacenes ni graneros, y sin embargo el Padre proporciona comida a todos aquellos que la buscan. ¡Y cuánto más valiosos sois vosotros que muchos pájaros! Además, toda vuestra ansiedad o las dudas que os corroan no podrán hacer nada por satisfacer vuestras necesidades materiales. ¿Quién de vosotros puede, con su ansiedad, añadir un palmo a su estatura o un día a su vida? Puesto que esas cuestiones no dependen de vosotros, ¿por qué pensáis ansiosamente en esos problemas?>>

\par 
%\textsuperscript{(1823.3)}
\textsuperscript{165:5.3} <<Contemplad los lirios y la manera en que crecen; no trabajan ni hilan; y sin embargo os afirmo que ni siquiera Salomón, con toda su gloria, estaba engalanado como uno de ellos. Si Dios viste así a la hierba del campo, que hoy está viva y mañana será cortada y echada al fuego, cuánto mejor os vestirá a vosotros, los embajadores del reino celestial. ¡Oh, hombres de poca fe! Si os dedicáis de todo corazón a proclamar el evangelio del reino, no deberíais tener dudas en vuestra mente sobre vuestro propio sustento o el de las familias que habéis abandonado. Si entregáis realmente vuestra vida al evangelio, viviréis por el evangelio. Si solamente sois unos discípulos creyentes, tendréis que ganaros vuestro pan y contribuir al sostén de todos los que enseñan, predican y curan. Si estáis inquietos a causa de vuestro pan y de vuestra agua, ¿en qué sois diferentes a las naciones del mundo que buscan esas necesidades con tanta diligencia? Consagraos a vuestro trabajo con el convencimiento de que tanto el Padre como yo sabemos que tenéis necesidad de todas esas cosas. Dejadme aseguraros, una vez por todas, que si dedicáis vuestra vida a la obra del reino, todas vuestras necesidades reales serán satisfechas. Buscad la cosa más grande, y encontraréis que las más pequeñas están contenidas en ella; pedid las cosas celestiales, y las cosas terrenales estarán incluidas. La sombra no puede dejar de seguir a la sustancia>>.

\par 
%\textsuperscript{(1823.4)}
\textsuperscript{165:5.4} <<Sólo sois un grupo pequeño, pero si tenéis fe, si el miedo no os hace tropezar, os declaro que mi Padre tendrá la satisfacción de daros este reino. Habéis guardado vuestros tesoros donde las bolsas no envejecen, donde ningún ladrón puede despojaros, y donde ninguna polilla puede destruir. Tal como se lo he dicho a la gente, allí donde esté vuestro tesoro, estará también vuestro corazón>>.

\par 
%\textsuperscript{(1824.1)}
\textsuperscript{165:5.5} <<Pero en la tarea que nos aguarda de inmediato, y en la que quedará para vosotros después de que yo regrese al Padre, pasaréis por pruebas muy penosas. Todos tendréis que estar alertas contra el miedo y las dudas. Que cada uno de vosotros se prepare mentalmente para la lucha y mantenga su lámpara encendida. Comportaos como unos hombres que están esperando a que regrese su señor de la fiesta nupcial, para que cuando vuelva y llame a la puerta, podáis abrirle rápidamente. El señor bendecirá a esos servidores vigilantes por encontrarlos fieles en un momento tan importante. Entonces el señor hará que sus servidores se sienten, y él mismo los servirá. En verdad, en verdad os digo que se avecina una crisis en vuestra vida, y os corresponde vigilar y estar preparados>>.

\par 
%\textsuperscript{(1824.2)}
\textsuperscript{165:5.6} <<Comprendéis bien que ningún hombre permitirá que su casa sea asaltada, si sabe a qué hora llegará el ladrón. Vigilaos también a vosotros mismos, porque a la hora que menos sospechéis y de una manera que no imagináis, el Hijo del Hombre se marchará>>.

\par 
%\textsuperscript{(1824.3)}
\textsuperscript{165:5.7} Los doce permanecieron sentados en silencio durante unos minutos. Algunas de estas advertencias las habían escuchado antes, pero no en el marco en que Jesús se las expuso en esta ocasión.

\section*{6. La respuesta a la pregunta de Pedro}
\par 
%\textsuperscript{(1824.4)}
\textsuperscript{165:6.1} Mientras estaban sentados pensando, Simón Pedro preguntó: <<¿Nos cuentas esta parábola a nosotros, tus apóstoles, o es para todos los discípulos?>> Y Jesús contestó:

\par 
%\textsuperscript{(1824.5)}
\textsuperscript{165:6.2} <<El alma del hombre se revela en los momentos de prueba; la prueba descubre lo que hay realmente en el corazón. Cuando el criado ha sido probado y experimentado, entonces el señor de la casa puede entregar a ese sirviente el gobierno de su casa, y confiar sin peligro a ese mayordomo fiel el encargo de alimentar y criar a sus hijos. Del mismo modo, yo sabré pronto a quién podré confiar el bienestar de mis hijos después de que haya regresado al Padre. Así como el señor de la casa entregará al servidor leal y probado los asuntos de su familia, yo también ensalzaré, en los asuntos de mi reino, a aquellos que resistan las pruebas de esta hora>>.

\par 
%\textsuperscript{(1824.6)}
\textsuperscript{165:6.3} <<Pero si el criado es perezoso y empieza a decirse en su interior: `Mi señor retrasa su llegada', y comienza a maltratar a los demás criados, y a comer y a beber con los borrachos, entonces el señor de ese sirviente llegará cuando menos lo espere y, al encontrarlo infiel, lo despedirá con ignominia. Por eso, haréis bien en prepararos para el día en que seréis visitados de repente y de manera inesperada. Recordad que a vosotros se os ha dado mucho; por eso se os pedirá mucho. Se avecinan duras pruebas para vosotros. Tengo que pasar por un bautismo, y estaré alerta hasta que se haya consumado. Predicáis la paz en la Tierra, pero mi misión no traerá la paz a los asuntos materiales de los hombres ---al menos, no durante un tiempo. Cuando dos miembros de una familia creen en mí y otros tres rechazan este evangelio, el único resultado es la división. Los amigos, los parientes y los seres queridos están destinados a indisponerse los unos con los otros a causa del evangelio que predicáis. Es verdad que cada uno de estos creyentes gozará de una gran paz duradera en su propio corazón, pero la paz en la Tierra no llegará hasta que todos estén dispuestos a creer y a entrar en la herencia gloriosa de su filiación con Dios. A pesar de eso, id por todo el mundo y proclamad este evangelio a todas las naciones, a cada hombre, mujer y niño>>.

\par 
%\textsuperscript{(1824.7)}
\textsuperscript{165:6.4} Así fue como terminó este día de sábado repleto y atareado. Al día siguiente, Jesús y los doce fueron a las ciudades del norte de Perea para charlar con los setenta, que estaban trabajando en estas regiones bajo la supervisión de Abner.


\chapter{Documento 166. La última visita a Perea del norte}
\par 
%\textsuperscript{(1825.1)}
\textsuperscript{166:0.1} DEL 11 al 20 de febrero, Jesús y los doce hicieron una gira por todas las ciudades y pueblos del norte de Perea donde trabajaban los asociados de Abner y los miembros del cuerpo de mujeres. Se encontraron con que estos mensajeros del evangelio tenían éxito, y Jesús llamó repetidas veces la atención de sus apóstoles sobre el hecho de que el evangelio del reino se podía difundir sin necesidad de milagros y prodigios.

\par 
%\textsuperscript{(1825.2)}
\textsuperscript{166:0.2} Toda esta misión de tres meses en Perea fue llevada a cabo con éxito y con poca ayuda por parte de los doce apóstoles; desde aquel momento en adelante, el evangelio reflejó más las \textit{enseñanzas} de Jesús que su personalidad. Pero sus discípulos no siguieron durante mucho tiempo sus instrucciones, pues poco después de la muerte y resurrección de Jesús, se desviaron de sus enseñanzas y empezaron a construir la iglesia primitiva alrededor de los conceptos milagrosos y de los recuerdos glorificados de su personalidad divina y humana.

\section*{1. Los fariseos en Ragaba}
\par 
%\textsuperscript{(1825.3)}
\textsuperscript{166:1.1} El sábado 18 de febrero, Jesús se encontraba en Ragaba, donde vivía un rico fariseo llamado Natanael; puesto que un número considerable de sus compañeros fariseos seguían a Jesús y a los doce por todo el país, este sábado por la mañana Natanael preparó un desayuno para todos ellos, unas veinte personas, e invitó a Jesús como huésped de honor.

\par 
%\textsuperscript{(1825.4)}
\textsuperscript{166:1.2} Cuando Jesús llegó a este desayuno, la mayoría de los fariseos, junto con dos o tres juristas, ya se encontraban allí sentados a la mesa. El Maestro tomó asiento de inmediato a la izquierda de Natanael, sin lavarse las manos en las palanganas. Muchos fariseos, especialmente los que estaban a favor de las enseñanzas de Jesús, sabían que sólo se lavaba las manos con fines higiénicos, y que detestaba estas prácticas puramente ceremoniales; por eso no se sorprendieron de que se dirigiera directamente a la mesa sin haberse lavado las manos dos veces. Pero Natanael se escandalizó porque el Maestro olvidó cumplir con las estrictas exigencias de las prácticas fariseas. Jesús tampoco se lavaba las manos, como hacían los fariseos, después del servicio de cada plato, ni al final de la comida.

\par 
%\textsuperscript{(1825.5)}
\textsuperscript{166:1.3} Después de mucho cuchicheo entre Natanael y un fariseo poco amistoso que estaba a su derecha, y después de muchos levantamientos de cejas y muecas burlonas de desprecio por parte de los que estaban sentados enfrente del Maestro, Jesús finalmente dijo: <<Creí que me habíais invitado a esta casa para partir el pan con vosotros y quizás para hacerme preguntas sobre la proclamación del nuevo evangelio del reino de Dios; pero percibo que me habéis traído aquí para presenciar una exhibición de devoción ceremonial a vuestra propia presunción. Ese servicio ya me lo habéis hecho; ¿con qué nueva cosa vais a honrarme como invitado vuestro en esta ocasión?>>

\par 
%\textsuperscript{(1826.1)}
\textsuperscript{166:1.4} Cuando el Maestro hubo hablado así, bajaron la mirada hacia la mesa y permanecieron en silencio. Como nadie hablaba, Jesús continuó: <<Muchos de vosotros, fariseos, estáis aquí conmigo como amigos, y algunos son incluso mis discípulos, pero la mayoría de los fariseos persisten en negarse a ver la luz y en reconocer la verdad, aunque la obra del evangelio se les presente con un gran poder. ¡Con cuánto cuidado limpiáis el exterior de las copas y de los platos, mientras que los recipientes del alimento espiritual están sucios y contaminados! Os aseguráis en mostrarle al pueblo una apariencia piadosa y santa, pero vuestra alma interior está llena de presunción, de codicia, de extorsión y de todo tipo de maldad espiritual. Vuestros dirigentes se atreven incluso a conspirar y a planear el asesinato del Hijo del Hombre. ¿No comprendéis, insensatos, que el Dios del cielo mira los móviles internos del alma, así como vuestros fingimientos exteriores y vuestras ostentaciones de piedad? No creáis que el hecho de dar limosnas y de pagar los diezmos os limpiará de vuestra injusticia y os permitirá aparecer puros ante el Juez de todos los hombres. ¡Ay de vosotros, fariseos, que habéis persistido en rechazar la luz de la vida! Sois meticulosos en el pago del diezmo y dais limosnas con ostentación, pero despreciáis a sabiendas la visita de Dios y rechazáis la revelación de su amor. Aunque hacéis bien en prestar atención a esos deberes menores, no deberíais haber dejado sin hacer otras exigencias más importantes. ¡Ay de todos los que rehuyen la justicia, desdeñan la misericordia y rechazan la verdad! ¡Ay de todos los que desprecian la revelación del Padre, mientras aspiran a conseguir los principales asientos en la sinagoga y anhelan los saludos halagadores en las plazas de los mercados!>>

\par 
%\textsuperscript{(1826.2)}
\textsuperscript{166:1.5} Cuando Jesús se levantó para marcharse, uno de los juristas sentados a la mesa le dirigió la palabra, diciendo: <<Pero, Maestro, en algunas de tus declaraciones también nos haces reproches. ¿No hay nada bueno en los escribas, los fariseos o los juristas?>> Jesús, que permanecía de pie, respondió al jurista: <<Vosotros, al igual que los fariseos, os deleitáis en ocupar los mejores lugares en las fiestas y en lucir largas túnicas, mientras que colocáis unas cargas pesadas, difíciles de llevar, sobre los hombros de la gente. Y cuando las almas de los hombres se tambalean debajo de esas pesadas cargas, no levantáis ni uno solo de vuestros dedos. ¡Ay de vosotros, que encontráis vuestra mayor satisfacción en construir tumbas para los profetas que vuestros padres mataron! Que vosotros aprobáis lo que hicieron vuestros padres se pone de manifiesto en el hecho de que ahora planeáis matar a los que vienen a hacer, en el día de hoy, lo que hicieron los profetas en su día ---proclamar la justicia de Dios y revelar la misericordia del Padre celestial. Pero de todas las generaciones pasadas, la sangre de los profetas y de los apóstoles será exigida a esta generación perversa y presuntuosa. ¡Ay de todos vosotros, juristas, que le habéis quitado la llave del conocimiento a la gente común! Vosotros mismos os negáis a entrar en el camino de la verdad, y al mismo tiempo quisierais impedir la entrada a todos los que la buscan. Pero no podéis cerrar así las puertas del reino de los cielos; las hemos abierto a todos los que tienen fe para entrar; y esos portales de misericordia no serán cerrados por los prejuicios y la arrogancia de los falsos educadores y de los pastores engañosos que se parecen a los sepulcros blanqueados, los cuales aparecen hermosos por fuera, pero por dentro están llenos de huesos de muertos y de todo tipo de impurezas espirituales>>.

\par 
%\textsuperscript{(1826.3)}
\textsuperscript{166:1.6} Cuando Jesús hubo terminado de hablar en la mesa de Natanael, salió de la casa sin haber participado en la comida. De todos los fariseos que habían escuchado estas palabras, algunos creyeron en su enseñanza y entraron en el reino, pero más numerosos fueron los que persistieron en el camino de las tinieblas, estando cada vez más decididos a espiarlo para poder atrapar algunas de sus palabras y utilizarlas para procesarlo y juzgarlo ante el sanedrín de Jerusalén.

\par 
%\textsuperscript{(1827.1)}
\textsuperscript{166:1.7} Había únicamente tres cosas a las que los fariseos prestaban una atención particular:

\par 
%\textsuperscript{(1827.2)}
\textsuperscript{166:1.8} 1. Practicar estrictamente el diezmo.

\par 
%\textsuperscript{(1827.3)}
\textsuperscript{166:1.9} 2. Cumplir escrupulosamente las reglas de purificación.

\par 
%\textsuperscript{(1827.4)}
\textsuperscript{166:1.10} 3. Evitar asociarse con todos los que no fueran fariseos.

\par 
%\textsuperscript{(1827.5)}
\textsuperscript{166:1.11} En aquel momento, Jesús trataba de poner al descubierto la esterilidad espiritual de las dos primeras prácticas; en cuanto a sus observaciones destinadas a reprender a los fariseos por su rechazo a mantener relaciones sociales con los no fariseos, las reservó para una ocasión posterior en la que cenaría de nuevo con muchos de estos mismos hombres.

\section*{2. Los diez leprosos}
\par 
%\textsuperscript{(1827.6)}
\textsuperscript{166:2.1} Al día siguiente, Jesús fue con los doce a Amatus, cerca de la frontera de Samaria. Al acercarse a la ciudad, se encontraron con un grupo de diez leprosos que residían por algún tiempo cerca de aquel lugar. Nueve de ellos eran judíos y uno samaritano. Normalmente, estos judíos habrían evitado toda asociación o todo contacto con este samaritano, pero la aflicción que tenían en común era más que suficiente para superar todos los prejuicios religiosos. Habían oído hablar mucho de Jesús y de sus primeras curaciones milagrosas, y como los setenta tenían la costumbre de anunciar la hora aproximada en que Jesús llegaría cuando el Maestro estaba de gira con los doce, los diez leprosos se habían enterado de que aparecería por estas inmediaciones hacia esta hora; en consecuencia, estaban apostados aquí en las afueras de la ciudad, con la esperanza de atraer su atención y pedirle la curación. Cuando los leprosos vieron llegar a Jesús, no se atrevieron a acercarse a él y se mantuvieron a distancia, gritándole: <<Maestro, ten piedad de nosotros; límpianos de nuestra aflicción. Cúranos como has curado a otros>>.

\par 
%\textsuperscript{(1827.7)}
\textsuperscript{166:2.2} Jesús acababa de explicar a los doce por qué los gentiles de Perea, junto con los judíos menos ortodoxos, estaban más dispuestos que los judíos de Judea, más ortodoxos y atados a la tradición, a creer en el evangelio predicado por los setenta. Había llamado su atención sobre el hecho de que su mensaje también había sido recibido más fácilmente por los galileos, e incluso por los samaritanos. Pero los doce apóstoles aún no estaban dispuestos a mantener sentimientos amistosos hacia los samaritanos, despreciados durante tanto tiempo.

\par 
%\textsuperscript{(1827.8)}
\textsuperscript{166:2.3} En consecuencia, cuando Simón Celotes observó al samaritano entre los leprosos, intentó persuadir al Maestro para que siguiera andando hasta la ciudad sin detenerse siquiera para intercambiar saludos con ellos. Jesús le dijo a Simón: <<Pero, ¿y si el samaritano ama a Dios tanto como los judíos? ¿Vamos a juzgar a nuestros semejantes? ¿Quién puede decirlo? Si curamos a estos diez hombres, quizás el samaritano resulte ser más agradecido incluso que los judíos. ¿Te sientes seguro de tus opiniones, Simón?>> Simón replicó de inmediato: <<Si los purificas, lo averiguarás enseguida>>. Y Jesús contestó: <<Así será, Simón, y pronto conocerás la verdad sobre la gratitud de los hombres y la misericordia amorosa de Dios>>.

\par 
%\textsuperscript{(1827.9)}
\textsuperscript{166:2.4} Jesús se acercó a los leprosos, y dijo: <<Si queréis recuperar la salud, id inmediatamente a mostraros a los sacerdotes, como lo exige la ley de Moisés>>. Y mientras iban de camino, recuperaron la salud. Cuando el samaritano vio que había sido curado, volvió sobre sus pasos buscando a Jesús, y empezó a glorificar a Dios en voz alta. Cuando hubo encontrado al Maestro, cayó de rodillas a sus pies y dio gracias por su purificación. Los otros nueve, los judíos, también habían descubierto que habían sido curados, y aunque también estaban agradecidos por su purificación, continuaron su camino para mostrarse a los sacerdotes.

\par 
%\textsuperscript{(1828.1)}
\textsuperscript{166:2.5} Mientras el samaritano permanecía arrodillado a los pies de Jesús, el Maestro miró sucesivamente a los doce, especialmente a Simón Celotes, y dijo: <<¿No han sido purificados los diez? ¿Dónde están entonces los otros nueve, los judíos? Solamente uno, este extranjero, ha regresado para dar gloria a Dios>>. Luego dijo al samaritano: <<Levántate y sigue tu camino; tu fe te ha curado>>.

\par 
%\textsuperscript{(1828.2)}
\textsuperscript{166:2.6} Jesús miró de nuevo a sus apóstoles mientras el extranjero se alejaba. Y todos los apóstoles miraron a Jesús, excepto Simón Celotes, que tenía la mirada baja. Los doce no dijeron ni una palabra. Y Jesús tampoco habló; no era necesario hacerlo.

\par 
%\textsuperscript{(1828.3)}
\textsuperscript{166:2.7} Aunque estos diez hombres creían realmente que tenían la lepra, solamente cuatro sufrían de ella. Los otros seis fueron curados de una enfermedad de la piel que había sido confundida con la lepra. Pero el samaritano tenía realmente la lepra.

\par 
%\textsuperscript{(1828.4)}
\textsuperscript{166:2.8} Jesús ordenó a los doce que no dijeran nada sobre la purificación de los leprosos, y cuando entraban en Amatus, comentó: <<Ya veis cómo los hijos de la casa, incluso cuando son desobedientes a la voluntad de su Padre, dan por sentadas sus bendiciones. Creen que es de poca importancia el dejar de dar las gracias cuando el Padre les concede la curación, pero cuando los extranjeros reciben los dones del dueño de la casa, se llenan de asombro y se sienten obligados a dar las gracias en reconocimiento por las buenas cosas que les han sido concedidas>>. Y los apóstoles continuaron sin decir nada en respuesta a las palabras del Maestro.

\section*{3. El sermón en Gerasa}
\par 
%\textsuperscript{(1828.5)}
\textsuperscript{166:3.1} Mientras Jesús y los doce conversaban con los mensajeros del reino en Gerasa, uno de los fariseos que creían en él hizo la pregunta siguiente: <<Señor, ¿en realidad se salvarán pocas o muchas personas?>> Y Jesús contestó:

\par 
%\textsuperscript{(1828.6)}
\textsuperscript{166:3.2} <<Os han enseñado que sólo los hijos de Abraham serán salvados, que sólo los gentiles de adopción pueden esperar la salvación. Como las Escrituras indican que de todas las multitudes que salieron de Egipto, sólo Caleb y Josué vivieron para entrar en la tierra prometida, algunos de vosotros habéis deducido que sólo un número relativamente pequeño de aquellos que buscan el reino de los cielos conseguirá entrar en él>>.

\par 
%\textsuperscript{(1828.7)}
\textsuperscript{166:3.3} <<También tenéis otro dicho entre vosotros, y es un dicho que contiene mucha verdad: El camino que conduce a la vida eterna es recto y estrecho, y la puerta de acceso es igualmente estrecha, de manera que, de aquellos que buscan la salvación, pocos son los que logran entrar por esa puerta. También tenéis una enseñanza que dice que el camino que conduce a la destrucción es amplio, que su entrada es ancha, y que muchos escogen seguir ese camino. Este proverbio no está desprovisto de significado. Pero yo declaro que la salvación es, en primer lugar, una cuestión de elección personal. Aunque la puerta que conduce al camino de la vida sea estrecha, es lo suficientemente ancha como para recibir a todos los que intentan entrar sinceramente, porque yo soy esa puerta. Y el Hijo nunca le negará la entrada a ningún hijo del universo que aspira, por la fe, a encontrar al Padre a través del Hijo>>.

\par 
%\textsuperscript{(1829.1)}
\textsuperscript{166:3.4} <<Pero he aquí el peligro para todos los que quisieran aplazar su entrada en el reino, a fin de continuar buscando los placeres de la inmadurez y permitirse las satisfacciones del egoísmo: Al haberse negado a entrar en el reino como experiencia espiritual, quizás intenten más tarde entrar en él cuando la gloria del mejor camino sea revelada en la era por venir. Por consiguiente, aquellos que despreciaron el reino cuando yo vine en la similitud de la humanidad, tratarán de encontrar una entrada cuando sea revelado en la similitud de la divinidad; pero entonces diré a todos esos egoístas: No sé de dónde venís. Tuvisteis la oportunidad de prepararos para esta ciudadanía celestial, pero rehusasteis todas estas ofertas de misericordia; rechazasteis todas las invitaciones para venir mientras que la puerta estaba abierta. Ahora, para vosotros que habéis rechazado la salvación, la puerta está cerrada. Esta puerta no está abierta para aquellos que quieren entrar en el reino para glorificarse egoístamente. La salvación no es para los que no están dispuestos a pagar el precio de una dedicación entusiasta a hacer la voluntad de mi Padre. Cuando en vuestro espíritu y en vuestra alma le habéis dado la espalda al reino del Padre, es inútil permanecer mental y corporalmente delante de esta puerta, y llamar diciendo: `Señor, ábrenos; nosotros también queremos ser grandes en el reino.' Entonces declararé que no pertenecéis a mi redil. No os recibiré para que estéis con los que han librado el buen combate de la fe y han ganado la recompensa del servicio desinteresado en el reino en la Tierra. Y cuando digáis: `¿No comimos y bebimos contigo, y no enseñaste en nuestras calles?', entonces declararé de nuevo que sois unos extranjeros espirituales; que no servimos juntos en el ministerio de misericordia del Padre en la Tierra; que no os conozco; y entonces, el Juez de toda la Tierra os dirá: `Apartaos de nosotros, todos los que habéis disfrutado con las obras de la iniquidad.'>>

\par 
%\textsuperscript{(1829.2)}
\textsuperscript{166:3.5} <<Pero no temáis; todo el que desee sinceramente encontrar la vida eterna entrando en el reino de Dios, hallará con seguridad esa salvación eterna. Pero vosotros, que rechazáis esta salvación, algún día veréis a los profetas de la semilla de Abraham sentarse en este reino glorificado con los creyentes de las naciones gentiles, para compartir el pan de la vida y refrescarse con el agua de la vida. Aquellos que se apoderen así del reino mediante el poder espiritual y los asaltos perseverantes de la fe viviente, vendrán del norte y del sur, del este y del oeste. Y mirad, muchos que son los primeros serán los últimos, y aquellos que son los últimos serán muchas veces los primeros>>.

\par 
%\textsuperscript{(1829.3)}
\textsuperscript{166:3.6} Ésta fue, en verdad, una versión nueva e insólita del viejo proverbio bien conocido sobre el camino recto y estrecho.

\par 
%\textsuperscript{(1829.4)}
\textsuperscript{166:3.7} Lentamente, los apóstoles y muchos discípulos aprendían el significado de la declaración inicial de Jesús: <<A menos que nazcáis de nuevo, que nazcáis del espíritu, no podréis entrar en el reino de Dios>>. Sin embargo, para todos los que son honrados de corazón y tienen una fe sincera, es eternamente cierto que: <<Mirad, permanezco en la puerta del corazón de los hombres y llamo; si alguien me abre, entraré, cenaré con él y lo alimentaré con el pan de la vida; seremos uno solo en espíritu y en propósito, y así seremos siempre hermanos en el largo y fructífero servicio de buscar al Padre Paradisiaco>>. Así pues, si los que se van a salvar son muchos o pocos, eso depende enteramente de que muchos o pocos hagan caso de la invitación: <<Yo soy la puerta, yo soy el camino nuevo y viviente, y cualquiera que lo desee puede entrar para emprender la búsqueda interminable de la verdad, que durará la vida eterna>>.

\par 
%\textsuperscript{(1829.5)}
\textsuperscript{166:3.8} Los mismos apóstoles eran incapaces de comprender plenamente su enseñanza sobre la necesidad de utilizar la fuerza espiritual a fin de vencer todas las resistencias materiales, y para superar todos los obstáculos terrenales que casualmente pudieran impedir la comprensión de los valores espirituales, sumamente importantes, de la nueva vida en el espíritu, como hijos liberados de Dios.

\section*{4. La enseñanza sobre los accidentes}
\par 
%\textsuperscript{(1830.1)}
\textsuperscript{166:4.1} Aunque la mayoría de los palestinos sólo hacían dos comidas al día, Jesús y los apóstoles tenían la costumbre, cuando iban de viaje, de detenerse al mediodía para descansar y tomar un refrigerio. En una de estas detenciones del mediodía, en el camino de Filadelfia, fue cuando Tomás le preguntó a Jesús: <<Maestro, después de haber escuchado tus comentarios mientras viajábamos esta mañana, me gustaría averiguar si los seres espirituales están implicados en la producción de acontecimientos extraños y extraordinarios en el mundo material, y preguntar además si los ángeles y otros seres espirituales son capaces de impedir los accidentes>>.

\par 
%\textsuperscript{(1830.2)}
\textsuperscript{166:4.2} En respuesta a la pregunta de Tomás, Jesús dijo: <<¿He estado tanto tiempo con vosotros, y sin embargo continuáis haciéndome estas preguntas? ¿No habéis observado que el Hijo del Hombre vive como uno de vosotros, y que se niega firmemente a emplear las fuerzas del cielo para su sostenimiento personal? ¿No vivimos todos con los mismos recursos que emplean todos los hombres para existir? ¿Acaso veis que el poder del mundo espiritual se manifieste en la vida material de este mundo, salvo en la revelación del Padre y en la curación esporádica de sus hijos afligidos?>>

\par 
%\textsuperscript{(1830.3)}
\textsuperscript{166:4.3} <<Vuestros antepasados han creído durante demasiado tiempo que la prosperidad era el signo de la aprobación divina, y que la adversidad era la prueba del desagrado de Dios. Afirmo que esas creencias son supersticiones. ¿No observáis que un número mucho mayor de pobres reciben el evangelio con regocijo y entran inmediatamente en el reino? Si las riquezas prueban el favor divino, ¿por qué los ricos se niegan tantas veces a creer en esta buena nueva que procede del cielo?>>

\par 
%\textsuperscript{(1830.4)}
\textsuperscript{166:4.4} <<El Padre hace caer su lluvia sobre los justos y los injustos; el Sol brilla de igual manera sobre los virtuosos y los perversos. Habéis oído hablar de aquellos galileos cuya sangre mezcló Pilatos con la de los sacrificios, pero yo os digo que esos galileos no eran de ninguna manera más pecadores que todos sus semejantes, simplemente porque esto les sucedió a ellos. También conocéis la historia de los dieciocho hombres que perecieron por la caída de la torre de Siloé. No creáis que esos hombres que fueron aniquilados así eran más pecadores que todos sus hermanos de Jerusalén. Esas personas fueron simplemente las víctimas inocentes de uno de los accidentes del tiempo>>.

\par 
%\textsuperscript{(1830.5)}
\textsuperscript{166:4.5} <<Existen tres tipos de acontecimientos que se pueden producir en vuestras vidas:>>

\par 
%\textsuperscript{(1830.6)}
\textsuperscript{166:4.6} <<1. Podéis participar en aquellos acontecimientos normales que forman parte de la vida que vosotros y vuestros compañeros vivís sobre la faz de la Tierra>>.

\par 
%\textsuperscript{(1830.7)}
\textsuperscript{166:4.7} <<2. Podéis ser víctimas por casualidad de uno de los accidentes de la naturaleza, de una de las desgracias humanas, sabiendo muy bien que esos sucesos no están de ninguna manera preparados de antemano ni son producidos de otro modo por las fuerzas espirituales del planeta>>.

\par 
%\textsuperscript{(1830.8)}
\textsuperscript{166:4.8} <<3. Podéis recoger la cosecha de vuestros esfuerzos directos por acatar las leyes naturales que gobiernan el mundo>>.

\par 
%\textsuperscript{(1830.9)}
\textsuperscript{166:4.9} <<Había un hombre que plantó una higuera en su patio, y después de ir muchas veces a buscar los frutos sin encontrar ninguno, llamó a los viñadores y les dijo: `He venido aquí durante tres temporadas para buscar los frutos de esta higuera y no he encontrado ninguno. Derribad este árbol estéril; ¿para qué tiene que estar estorbando en el suelo?' Pero el jardinero en jefe respondió a su señor: `Déjalo tranquilo durante un año más para que yo pueda cavar a su alrededor y echarle abono; si el año que viene no produce frutos, entonces lo cortaremos.' Y cuando se hubieron sometido así a las leyes de la fertilidad, fueron recompensados con una cosecha abundante, ya que el árbol estaba vivo y en buen estado>>.

\par 
%\textsuperscript{(1831.1)}
\textsuperscript{166:4.10} <<En las cosas de la enfermedad y de la salud, deberíais saber que esos estados físicos son el resultado de causas materiales; la salud no es la sonrisa del cielo, ni la aflicción el enojo de Dios>>.

\par 
%\textsuperscript{(1831.2)}
\textsuperscript{166:4.11} <<Los hijos humanos del Padre tienen la misma capacidad para recibir las bendiciones materiales; por eso, concede las cosas físicas a los hijos de los hombres sin discriminación. Cuando se trata de atribuir los dones espirituales, el Padre está limitado por la capacidad del hombre para recibir estos dones divinos. Aunque el Padre no hace acepción de personas, en la atribución de los dones espirituales está limitado por la fe del hombre y por su buena disposición para atenerse siempre a la voluntad del Padre>>.

\par 
%\textsuperscript{(1831.3)}
\textsuperscript{166:4.12} Mientras viajaban hacia Filadelfia, Jesús continuó enseñándoles y respondiendo a sus preguntas sobre los accidentes, las enfermedades y los milagros, pero no fueron capaces de comprender plenamente esta enseñanza. Una hora de enseñanza no es suficiente para cambiar por completo las creencias de toda una vida, y por eso Jesús creyó necesario reiterar su mensaje, decirles una y otra vez lo que deseaba hacerles comprender; y aún así, no lograron captar el significado de su misión terrenal hasta después de su muerte y resurrección.

\section*{5. La congregación de Filadelfia}
\par 
%\textsuperscript{(1831.4)}
\textsuperscript{166:5.1} Jesús y los doce iban de camino para visitar a Abner y a sus asociados, que predicaban y enseñaban en Filadelfia. De todas las ciudades de Perea, es en Filadelfia donde el grupo más numeroso de judíos y gentiles, ricos y pobres, eruditos e ignorantes, aceptó las enseñanzas de los setenta y entró así en el reino de los cielos. La sinagoga de Filadelfia nunca había estado sometida a la supervisión del sanedrín de Jerusalén, por lo que nunca había estado cerrada a las enseñanzas de Jesús y sus compañeros. En ese mismo momento, Abner enseñaba tres veces al día en la sinagoga de Filadelfia.

\par 
%\textsuperscript{(1831.5)}
\textsuperscript{166:5.2} Esta misma sinagoga se convirtió más tarde en una iglesia cristiana y fue el cuartel general de los misioneros que promulgaron el evangelio en las regiones situadas al este. Fue mucho tiempo la plaza fuerte de las enseñanzas del Maestro, y durante siglos se mantuvo sola en esta región como centro del conocimiento cristiano.

\par 
%\textsuperscript{(1831.6)}
\textsuperscript{166:5.3} Los judíos de Jerusalén siempre habían tenido problemas con los judíos de Filadelfia. Después de la muerte y resurrección de Jesús, la iglesia de Jerusalén, cuyo jefe era Santiago, el hermano del Señor, empezó a tener graves dificultades con la congregación de creyentes de Filadelfia. Abner se convirtió en el jefe de la iglesia de Filadelfia, y continuó siéndolo hasta su muerte. Este distanciamiento de Jerusalén explica por qué los relatos evangélicos del Nuevo Testamento no mencionan nada sobre Abner y su obra. Esta enemistad entre Jerusalén y Filadelfia permaneció durante toda la vida de Santiago y Abner, y continuó hasta algún tiempo después de la destrucción de Jerusalén. Filadelfia fue realmente el centro de la iglesia primitiva en el sur y en el este, como Antioquía lo fue en el norte y el oeste.

\par 
%\textsuperscript{(1831.7)}
\textsuperscript{166:5.4} Fue una aparente desgracia para Abner estar en desacuerdo con todos los jefes de la iglesia cristiana primitiva. Riñó con Pedro y Santiago (el hermano de Jesús) sobre cuestiones relacionadas con la administración y la jurisdicción de la iglesia de Jerusalén; se separó de Pablo por divergencias sobre filosofía y teología. Abner tenía una filosofía más babilónica que helenista, y se opuso obstinadamente a todos los intentos de Pablo por rehacer las enseñanzas de Jesús para que ocasionaran menos objeciones, primero entre los judíos, y luego entre los grecorromanos que creían en los misterios.

\par 
%\textsuperscript{(1832.1)}
\textsuperscript{166:5.5} Abner se vio obligado así a vivir una vida de aislamiento. Era el jefe de una iglesia que no gozaba de ninguna reputación en Jerusalén. Se había atrevido a desafiar a Santiago, el hermano del Señor, que posteriormente fue apoyado por Pedro. Esta conducta lo separó efectivamente de todos sus antiguos asociados. Luego se atrevió a oponerse a Pablo. Aunque estaba totalmente de acuerdo con la misión de Pablo entre los gentiles, y aunque lo apoyaba en sus disputas con la iglesia de Jerusalén, se opuso encarnizadamente a la versión de las enseñanzas de Jesús que Pablo había elegido predicar. En los últimos años de su vida, Abner denunció a Pablo como el <<hábil corruptor de las enseñanzas de la vida de Jesús de Nazaret, el Hijo del Dios viviente>>.

\par 
%\textsuperscript{(1832.2)}
\textsuperscript{166:5.6} Durante los últimos años de Abner y hasta algún tiempo después de su muerte, los creyentes de Filadelfia se atuvieron a la religión de Jesús, tal como éste la había vivido y enseñado, más estrictamente que cualquier otro grupo en la Tierra.

\par 
%\textsuperscript{(1832.3)}
\textsuperscript{166:5.7} Abner vivió hasta los 89 años de edad, y murió en Filadelfia el día 21 de noviembre del año 74. Hasta el final de su vida, fue un creyente fiel en el evangelio del reino celestial y un instructor del mismo.


\chapter{Documento 167. La visita a Filadelfia}
\par 
%\textsuperscript{(1833.1)}
\textsuperscript{167:0.1} A LO largo de todo este período de ministerio en Perea, cuando se menciona que Jesús y los apóstoles visitaban las diversas localidades donde trabajaban los setenta, es bueno recordar que, por regla general, sólo lo acompañaban diez apóstoles, pues tenía la costumbre de dejar en Pella al menos a dos apóstoles para instruir a la multitud. Mientras Jesús se preparaba para ir a Filadelfia, Simón Pedro y su hermano Andrés regresaron al campamento de Pella para enseñar a la muchedumbre allí reunida. Cuando el Maestro dejaba el campamento de Pella para recorrer Perea, no era raro que lo siguieran entre trescientas y quinientas personas que residían en el campamento. Cuando llegó a Filadelfia, iba acompañado por más de seiscientos seguidores.

\par 
%\textsuperscript{(1833.2)}
\textsuperscript{167:0.2} Ningún milagro se había producido durante la reciente gira de predicación a través de la Decápolis y, a excepción de la purificación de los diez leprosos, hasta ese momento no había habido ningún milagro en esta misión en Perea. Era un período en el que el evangelio se proclamaba con poder, sin milagros, y la mayor parte del tiempo sin la presencia personal de Jesús ni tampoco de sus apóstoles.

\par 
%\textsuperscript{(1833.3)}
\textsuperscript{167:0.3} Jesús y los diez apóstoles llegaron a Filadelfia el miércoles 22 de febrero, y el jueves y el viernes los pasaron descansando de sus viajes y trabajos recientes. Santiago habló en la sinagoga aquel viernes por la noche, y se convocó un consejo general para el día siguiente al anochecer. Estaban muy contentos por el progreso del evangelio en Filadelfia y en los pueblos cercanos. Los mensajeros de David también trajeron noticias de los nuevos progresos del reino por toda Palestina, así como buenas nuevas de Alejandría y Damasco.

\section*{1. El desayuno con los fariseos}
\par 
%\textsuperscript{(1833.4)}
\textsuperscript{167:1.1} En Filadelfia vivía un fariseo muy rico e influyente que había aceptado las enseñanzas de Abner, y que invitó a Jesús a desayunar en su casa el sábado por la mañana. Se sabía que a Jesús se le esperaba en Filadelfia a esa hora, por lo que un gran número de visitantes, entre ellos muchos fariseos, habían venido de Jerusalén y de otros lugares. En consecuencia, unos cuarenta de estos dirigentes y algunos juristas, fueron invitados a este desayuno que se había organizado en honor del Maestro.

\par 
%\textsuperscript{(1833.5)}
\textsuperscript{167:1.2} Mientras Jesús permanecía cerca de la puerta hablando con Abner, y después de que el mismo anfitrión se hubiera sentado, uno de los fariseos principales de Jerusalén, miembro del sanedrín, entró en la sala y, siguiendo su costumbre, se dirigió directamente al asiento de honor, a la izquierda del anfitrión. Pero como este lugar había sido reservado para el Maestro, y el de la derecha para Abner, el anfitrión señaló al fariseo de Jerusalén que se sentara en el cuarto asiento a la izquierda, y este dignatario se ofendió mucho por no haber recibido el asiento de honor.

\par 
%\textsuperscript{(1834.1)}
\textsuperscript{167:1.3} Pronto, todos estuvieron sentados y disfrutando de la conversación entre ellos, ya que la mayoría de los presentes eran discípulos de Jesús o bien eran favorables al evangelio. Sólo sus enemigos notaron el hecho de que no había cumplido con la ceremonia de lavarse las manos antes de sentarse para comer. Abner se lavó las manos al principio de la comida, pero no durante el servicio.

\par 
%\textsuperscript{(1834.2)}
\textsuperscript{167:1.4} Hacia el final de la comida, un hombre procedente de la calle entró en la sala; había estado afligido durante mucho tiempo con una enfermedad crónica, y ahora se encontraba en un estado hidrópico. Este hombre era creyente, y había sido bautizado recientemente por los compañeros de Abner. No le pidió a Jesús que lo curara, pero el Maestro sabía muy bien que este enfermo había acudido al desayuno con la esperanza de evitar las multitudes que se agolpaban a su alrededor, y así tener más posibilidades de atraer su atención. Este hombre sabía que por aquella época se realizaban pocos milagros; sin embargo, había razonado en su fuero interno que su estado lastimoso quizás atraería la compasión del Maestro. Y no estaba equivocado porque, en cuanto entró en la sala, tanto Jesús como el presuntuoso fariseo de Jerusalén advirtieron su presencia. El fariseo no tardó en expresar su indignación porque se permitiera a un individuo así entrar en la sala. Pero Jesús miró al enfermo y le sonrió con tanta bondad que éste se acercó y se sentó en el suelo. Cuando la comida estaba finalizando, el Maestro paseó su mirada sobre los demás invitados y luego, después de lanzar una mirada significativa al hombre con hidropesía, dijo: <<Amigos míos, educadores de Israel y expertos juristas, me gustaría haceros una pregunta: ¿Es lícito, o no, curar a los enfermos y a los afligidos en el día del sábado?>> Pero los que estaban allí presentes conocían muy bien a Jesús; guardaron silencio, y no contestaron a su pregunta.

\par 
%\textsuperscript{(1834.3)}
\textsuperscript{167:1.5} Jesús se dirigió entonces al lugar donde estaba sentado el enfermo, lo cogió de la mano, y le dijo: <<Levántate y sigue tu camino. No has pedido la curación, pero conozco el deseo de tu corazón y la fe de tu alma>>. Antes de que el hombre abandonara la sala, Jesús volvió a su sitio y dirigió la palabra a los que estaban en la mesa, diciendo: <<Mi Padre hace estas obras, no para induciros a entrar en el reino, sino para revelarse a los que ya están en el reino. Podéis percibir que sería muy propio del Padre hacer precisamente estas cosas, porque ¿quién de vosotros, si su animal favorito se cayera en el pozo el día del sábado, no saldría inmediatamente para sacarlo de allí?>> Como nadie quería contestarle, y ya que su anfitrión aprobaba evidentemente lo que estaba sucediendo, Jesús se levantó y dijo a todos los presentes: <<Hermanos míos, cuando os inviten a un banquete nupcial, no os sentéis en el asiento principal, no sea que un hombre más ilustre que vosotros haya sido invitado, y el anfitrión tenga que venir a rogaros que dejéis vuestro sitio al otro huésped de honor. En ese caso se os pedirá, para vuestra verg\"uenza, que ocupéis un sitio inferior en la mesa. Cuando os inviten a una fiesta, es una demostración de sabiduría, al llegar a la mesa del festín, buscar el asiento más humilde y sentaros allí, de tal manera que, cuando el anfitrión examine a los convidados, pueda deciros: `Amigo mío, ¿por qué te has sentado en el asiento más humilde? Ven más arriba'; y así ese hombre será glorificado en presencia de los demás invitados. No lo olvidéis: el que se ensalza a sí mismo será humillado, mientras que el que se humilla sinceramente será ensalzado. Por eso, cuando convidéis a almorzar o deis una cena, no invitéis siempre a vuestros amigos, vuestros hermanos, vuestros parientes o a vuestros vecinos ricos, para que ellos puedan invitaros a cambio a sus fiestas, y seáis así recompensados. Cuando ofrezcáis un banquete, invitad de vez en cuando a los pobres, a los mutilados y a los ciegos. De esa manera os sentiréis dichosos en vuestro corazón, porque sabéis muy bien que los cojos y los lisiados no pueden devolveros vuestra ayuda amorosa>>.

\section*{2. La parábola de la gran cena}
\par 
%\textsuperscript{(1835.1)}
\textsuperscript{167:2.1} Cuando Jesús terminó de hablar en la mesa del desayuno del fariseo, uno de los juristas presentes, deseando romper el silencio, dijo sin reflexionar: <<Bendito sea el que coma pan en el reino de Dios>> ---lo cual era un dicho corriente en aquella época. Jesús contó entonces una parábola, que incluso su amistoso anfitrión se vio obligado a tomar en serio. Dijo:

\par 
%\textsuperscript{(1835.2)}
\textsuperscript{167:2.2} <<Cierto dirigente ofreció una gran cena, y como había invitado a muchos huéspedes, a la hora de la cena envió a sus criados para que dijeran a los invitados:
`Venid, pues ya está todo preparado.' Pero todos empezaron a excusarse unánimemente. El primero dijo: `Acabo de comprar una finca, y es absolutamente necesario que vaya a examinarla; te ruego que me excuses.' Otro dijo: `He comprado cinco yuntas de bueyes, y tengo que ir a recibirlas; te ruego que me excuses.' Y otro dijo: `Acabo de contraer matrimonio, y por esta razón no puedo ir.' Así pues, los criados regresaron e informaron de esto a su señor. Cuando el dueño de la casa escuchó esto, se irritó mucho, y volviéndose hacia sus criados, les dijo: `He preparado este banquete de boda; los animales cebados han sido matados, y todo está preparado para mis huéspedes, pero han desdeñado mi invitación; cada cual se ha ido a sus tierras y a sus mercancías, e incluso han mostrado una falta de respeto a mis criados que les pedían que vinieran a mi festín. Salid pues rápidamente a las calles y callejuelas de la ciudad, a las carreteras y a los caminos, y traed aquí a los pobres y a los parias, a los ciegos y a los cojos, para que haya invitados en el banquete de boda.' Los criados hicieron lo que su señor les había ordenado, y aún así quedaba sitio para más invitados. Entonces el señor dijo a sus criados: `Salid ahora a los caminos y a los campos, y obligad a los que estén allí a que vengan, para que se llene mi casa. Os aseguro que ninguno de los que fueron invitados en primer lugar probará mi cena.' Los criados hicieron lo que les había ordenado su señor, y la casa se llenó>>.

\par 
%\textsuperscript{(1835.3)}
\textsuperscript{167:2.3} Cuando escucharon estas palabras, todos se marcharon, y cada uno se fue a su propia casa. De todos los fariseos despectivos que estaban presentes aquella mañana, al menos uno comprendió el significado de esta parábola, porque fue bautizado aquel mismo día y confesó públicamente su fe en el evangelio del reino. Aquella noche, Abner predicó sobre esta parábola en el consejo general de los creyentes.

\par 
%\textsuperscript{(1835.4)}
\textsuperscript{167:2.4} Al día siguiente, todos los apóstoles emprendieron el ejercicio filosófico de tratar de interpretar el significado de esta parábola de la gran cena. Aunque Jesús escuchó con interés todas las diversas interpretaciones, se negó firmemente a ofrecerles una ayuda adicional para comprender la parábola. Solamente dijo: <<Que cada uno encuentre el significado por sí mismo y en su propia alma>>.

\section*{3. La mujer de carácter débil}
\par 
%\textsuperscript{(1835.5)}
\textsuperscript{167:3.1} Abner se había encargado de que el Maestro enseñara en la sinagoga este sábado; era la primera vez que Jesús aparecía en una sinagoga desde que todas habían sido cerradas a sus enseñanzas por orden del sanedrín. Al final del servicio, Jesús observó delante de él a una mujer anciana que tenía una expresión abatida y el cuerpo muy encorvado. Esta mujer había estado durante mucho tiempo tiranizada por el miedo, y toda alegría había desaparecido de su vida. Cuando Jesús bajó del púlpito, se acercó a ella, tocó el hombro de su figura encorvada, y le dijo: <<Mujer, si tan sólo quisieras creer, te liberarías por completo de tu debilidad de carácter>>. Y esta mujer, que había estado encorvada y atada por las depresiones del miedo durante más de dieciocho años, creyó en las palabras del Maestro y, gracias a la fe, se puso derecha inmediatamente. Cuando esta mujer se dio cuenta de que estaba erguida, elevó la voz y glorificó a Dios.

\par 
%\textsuperscript{(1836.1)}
\textsuperscript{167:3.2} Aunque la aflicción de esta mujer era completamente mental, ya que su forma encorvada se debía a su mente deprimida, la gente creyó que Jesús había curado una verdadera enfermedad física. La asamblea de la sinagoga de Filadelfia era favorable a las enseñanzas de Jesús, pero el jefe principal de la sinagoga era un fariseo hostil. Como compartía la opinión de la asamblea de que Jesús había curado una enfermedad física, se indignó porque Jesús se hubiera atrevido a hacer una cosa así el día del sábado, y poniéndose de pie delante del auditorio, dijo: <<¿No hay seis días para que los hombres puedan hacer todo su trabajo? Venid pues para ser curados durante esos días laborables, pero no en el día del sábado>>.

\par 
%\textsuperscript{(1836.2)}
\textsuperscript{167:3.3} Cuando el jefe hostil hubo dicho esto, Jesús regresó a la tribuna de los oradores y dijo: <<¿Por qué jugar a ser hipócritas? Cada uno de vosotros, ¿no saca a su buey del establo el día del sábado, para llevarlo al abrevadero? Si esa buena acción es permisible el día del sábado, esta mujer, una hija de Abraham, que ha estado encogida por el mal durante estos dieciocho años, ¿no debería ser liberada de esa esclavitud y llevada a compartir las aguas de la libertad y de la vida, incluso en este día de sábado?>> Mientras la mujer continuaba glorificando a Dios, el detractor quedó puesto en evidencia, y la asamblea se regocijó con ella de que hubiera sido curada.

\par 
%\textsuperscript{(1836.3)}
\textsuperscript{167:3.4} Como consecuencia de haber criticado públicamente a Jesús este sábado, el jefe principal de la sinagoga fue destituido y reemplazado por un seguidor de Jesús.

\par 
%\textsuperscript{(1836.4)}
\textsuperscript{167:3.5} Jesús liberaba con frecuencia a estas víctimas del miedo de su debilidad de carácter, de su depresión mental y de su esclavitud al temor. Pero la gente creía que todas estas aflicciones eran, o bien enfermedades físicas, o posesiones por los espíritus malignos.

\par 
%\textsuperscript{(1836.5)}
\textsuperscript{167:3.6} El domingo, Jesús enseñó de nuevo en la sinagoga, y aquel día a mediodía Abner bautizó a muchas personas en el río que corría al sur de la ciudad. Al día siguiente, Jesús y los diez apóstoles habrían emprendido el viaje de vuelta al campamento de Pella si no hubiera sido por la llegada de uno de los mensajeros de David, que traía un mensaje urgente para Jesús de parte de sus amigos de Betania, cerca de Jerusalén.

\section*{4. El mensaje de Betania}
\par 
%\textsuperscript{(1836.6)}
\textsuperscript{167:4.1} El domingo 26 de febrero, a una hora muy tardía de la noche, un corredor procedente de Betania llegó a Filadelfia, trayendo un mensaje de Marta y María que decía: <<Señor, aquel que amas está muy enfermo>>. Este mensaje le llegó a Jesús al final de la conferencia de la tarde, justo en el momento en que se despedía de los apóstoles para pasar la noche. Al principio, Jesús no respondió nada. Entonces se produjo uno de esos extraños intervalos, un período de tiempo en el que parecía estar en comunicación con algo exterior a él y más allá de él. Luego levantó los ojos y se dirigió al mensajero, de manera que los apóstoles le oyeron decir: <<Esta enfermedad no le llevará realmente a la muerte. No dudéis de que será empleada para glorificar a Dios y exaltar al Hijo>>.

\par 
%\textsuperscript{(1837.1)}
\textsuperscript{167:4.2} Jesús estaba muy encariñado con Marta, María y su hermano Lázaro; los amaba con un afecto ferviente. Su primer pensamiento humano fue acudir inmediatamente en su ayuda, pero otra idea apareció en su mente combinada. Casi había perdido la esperanza de que los dirigentes judíos de Jerusalén aceptaran alguna vez el reino, pero seguía amando a su pueblo, y ahora se le había ocurrido un plan para que los escribas y fariseos de Jerusalén tuvieran una nueva oportunidad de aceptar sus enseñanzas; de este último llamamiento a Jerusalén decidió hacer, si su Padre quería, la obra exterior más profunda y asombrosa de toda su carrera terrenal. Los judíos estaban aferrados a la idea de un libertador que hiciera prodigios. Y aunque rehusaba rebajarse a realizar maravillas materiales o a llevar a cabo exhibiciones temporales de poder político, ahora buscó el consentimiento del Padre para manifestar su poder todavía no demostrado sobre la vida y la muerte.

\par 
%\textsuperscript{(1837.2)}
\textsuperscript{167:4.3} Los judíos tenían la costumbre de enterrar a sus muertos el día de su fallecimiento; era una práctica necesaria en un clima tan caluroso. A menudo sucedía que metían en la tumba a alguien que estaba simplemente en coma, de manera que al segundo, o incluso al tercer día, aquella persona salía de la tumba. Pero los judíos tenían la creencia de que, aunque el espíritu o el alma podía quedarse cerca del cuerpo durante dos o tres días, nunca permanecía allí después del tercer día; que la putrefacción ya estaba avanzada al cuarto día, y que nadie regresaba nunca de la tumba después de transcurrido ese tiempo. Debido a estas razones, Jesús permaneció dos días más en Filadelfia antes de prepararse para salir hacia Betania.

\par 
%\textsuperscript{(1837.3)}
\textsuperscript{167:4.4} En consecuencia, el miércoles por la mañana temprano dijo a sus apóstoles: <<Preparémonos inmediatamente para ir otra vez a Judea>>. Cuando los apóstoles escucharon estas palabras de su Maestro, se retiraron aparte durante un rato para consultarse entre ellos. Santiago tomó la dirección de la conversación, y todos estuvieron de acuerdo en que era una auténtica locura permitir a Jesús que regresara a Judea, por lo que volvieron como un solo hombre para comunicarselo. Santiago dijo: <<Maestro, has estado en Jerusalén hace unas semanas, y los dirigentes intentaron matarte, mientras que el pueblo estaba dispuesto a lapidarte. En aquel momento ya diste a esos hombres su oportunidad de recibir la verdad, y no te permitiremos que regreses a Judea>>.

\par 
%\textsuperscript{(1837.4)}
\textsuperscript{167:4.5} Jesús dijo entonces: <<Pero, ¿no comprendéis que el día tiene doce horas, durante las cuales se pueden hacer las tareas sin peligro? Si un hombre camina de día, no tropieza puesto que tiene luz. Si camina de noche, está expuesto a tropezar ya que no tiene luz. Mientras dure mi día, no tengo miedo a entrar en Judea. Quisiera realizar otra obra poderosa para esos judíos; quisiera darles otra oportunidad para creer, y en los términos que ellos prefieren ---con las condiciones de una gloria exterior y la manifestación visible del poder del Padre y del amor del Hijo. Además, ¡no os dais cuenta de que nuestro amigo Lázaro se ha dormido, y que quiero ir a despertarlo de ese sueño!>>

\par 
%\textsuperscript{(1837.5)}
\textsuperscript{167:4.6} A continuación, uno de los apóstoles dijo: <<Maestro, si Lázaro se ha dormido, entonces se restablecerá con más seguridad>>. En aquel tiempo, los judíos tenían la costumbre de hablar de la muerte como de una forma de sueño, pero como los apóstoles no habían comprendido que Jesús quería decir que Lázaro había partido de este mundo, ahora les dijo con toda claridad: <<Lázaro ha muerto. Pero por vuestro bien, y aunque esto no salve a los demás, me alegro de no haber estado allí, a fin de que ahora podáis tener una nueva razón para creer en mí; todos os sentiréis fortalecidos por lo que vais a presenciar, y os servirá de preparación para el día en que me despediré de vosotros para ir hacia el Padre>>.

\par 
%\textsuperscript{(1838.1)}
\textsuperscript{167:4.7} Como no pudieron persuadirlo para que se abstuviera de ir a Judea, y como algunos apóstoles eran incluso reacios a acompañarlo, Tomás se dirigió a sus compañeros, diciendo: <<Ya le hemos expresado nuestros temores al Maestro, pero él está decidido a ir a Betania. Estoy convencido de que esto significa el fin; lo matarán con toda seguridad, pero si ésta es la elección del Maestro, entonces comportémonos como unos hombres valientes; vamos también para poder morir con él>>. Siempre fue así; en las cuestiones que requerían un coraje deliberado y sostenido, Tomás fue siempre el sostén principal de los doce apóstoles.

\section*{5. En el camino de Betania}
\par 
%\textsuperscript{(1838.2)}
\textsuperscript{167:5.1} Un grupo de casi cincuenta amigos y enemigos siguió a Jesús en el camino hacia Judea. El miércoles, a la hora de la comida del mediodía, habló a sus apóstoles y a este grupo de seguidores sobre las <<Condiciones de la salvación>>, y al final de esta lección, contó la parábola del fariseo y el publicano (un recaudador de impuestos). Jesús dijo: <<Ya veis que el Padre concede la salvación a los hijos de los hombres, y esta salvación es un don gratuito para todos los que tienen la fe necesaria para recibir la filiación en la familia divina. El hombre no puede hacer nada para ganar esta salvación. Las obras presuntuosas no pueden comprar el favor de Dios, y una gran cantidad de oraciones en público no compensarán la falta de fe viviente en el corazón. Podéis engañar a los hombres con vuestro servicio aparente, pero Dios examina vuestra alma. Lo que os digo está bien ilustrado en el ejemplo de dos hombres, uno fariseo y el otro publicano, que entraron a orar en el templo. El fariseo permanecía de pie y oraba para sus adentros: `Oh Dios, te doy las gracias por no ser como los demás hombres, que son opresores, ignorantes, injustos, adúlteros, o incluso como ese publicano. Ayuno dos veces por semana, y doy el diezmo de todo lo que gano.' En cambio el publicano permanecía a lo lejos, sin atreverse siquiera a levantar los ojos al cielo, y se golpeaba el pecho, diciendo: `Dios, sé misericordioso con un pecador como yo.' Os digo que el publicano regresó a su casa con la aprobación de Dios más bien que el fariseo, porque todo aquel que se ensalza a sí mismo será humillado, pero aquel que se humilla será ensalzado>>.

\par 
%\textsuperscript{(1838.3)}
\textsuperscript{167:5.2} Aquella noche en Jericó, los fariseos hostiles trataron de coger al Maestro en una trampa, incitándolo a discutir sobre el matrimonio y el divorcio, como sus colegas habían hecho anteriormente en Galilea, pero Jesús evitó hábilmente sus esfuerzos por ponerlo en un conflicto con sus leyes relacionadas con el divorcio. Así como el publicano y el fariseo ilustraban la buena y la mala religión, sus prácticas del divorcio servían para establecer un contraste entre las mejores leyes matrimoniales del código judío y la relajación vergonzosa con que los fariseos interpretaban estas reglas mosaicas sobre el divorcio. El fariseo se juzgaba a sí mismo utilizando el criterio más bajo; el publicano se medía por el ideal más elevado. Para el fariseo, la devoción era un medio de inducirle a la inactividad presuntuosa y a la certeza de una falsa seguridad espiritual; para el publicano, la devoción era un medio de despertar su alma para que comprendiera la necesidad de arrepentirse, de confesarse y de aceptar, por la fe, un perdón misericordioso. El fariseo buscaba la justicia; el publicano buscaba la misericordia. La ley del universo es: pedid y recibiréis; buscad y encontraréis.

\par 
%\textsuperscript{(1838.4)}
\textsuperscript{167:5.3} Aunque Jesús se negó a dejarse arrastrar a una controversia con los fariseos sobre el divorcio, proclamó una enseñanza positiva de los ideales más elevados relativos al matrimonio. Exaltó el matrimonio como la relación más ideal y más elevada de todas las relaciones humanas. Asimismo, insinuó su enérgica desaprobación por las prácticas de divorcio relajadas e injustas de los judíos de Jerusalén, que en aquella época permitían que un hombre se divorciara de su esposa por las razones más insignificantes, tales como ser una mala cocinera, una ama de casa deficiente, o simplemente porque se había enamorado de una mujer más bonita.

\par 
%\textsuperscript{(1839.1)}
\textsuperscript{167:5.4} Los fariseos habían llegado incluso a enseñar que este tipo de divorcio fácil era una dispensa especial concedida al pueblo judío, y a los fariseos en particular. Y así, aunque Jesús se negó a hacer declaraciones sobre el matrimonio y el divorcio, censuró muy severamente estas burlas vergonzosas de la relación matrimonial, y señaló su injusticia para con las mujeres y los niños. Nunca aprobó una práctica de divorcio que proporcionara al hombre alguna ventaja sobre la mujer; el Maestro sólo apoyaba aquellas enseñanzas que concedían a las mujeres la igualdad con los hombres.

\par 
%\textsuperscript{(1839.2)}
\textsuperscript{167:5.5} Aunque Jesús no ofreció unos nuevos preceptos que rigieran el matrimonio y el divorcio, instó a los judíos a que cumplieran con sus propias leyes y con sus enseñanzas más elevadas. Recurrió constantemente a las Escrituras en su esfuerzo por mejorar las prácticas de todas estas conductas sociales. A la vez que defendía así los conceptos elevados e ideales del matrimonio, Jesús evitó hábilmente contradecir a sus interrogadores respecto a las prácticas sociales representadas en sus leyes escritas, o en sus privilegios de divorcio, tan apreciados por ellos.

\par 
%\textsuperscript{(1839.3)}
\textsuperscript{167:5.6} A los apóstoles les resultaba muy difícil comprender la desgana que mostraba el Maestro en hacer declaraciones positivas sobre los problemas científicos, sociales, económicos y políticos. No se daban plenamente cuenta de que su misión terrenal estaba exclusivamente interesada en las revelaciones de las verdades espirituales y religiosas.

\par 
%\textsuperscript{(1839.4)}
\textsuperscript{167:5.7} Después de que Jesús hubiera hablado sobre el matrimonio y el divorcio, más tarde aquella misma noche sus apóstoles le hicieron muchas preguntas adicionales en privado, y sus respuestas a estas preguntas liberaron sus mentes de muchos conceptos equivocados. Al final de esta conferencia, Jesús dijo: <<El matrimonio es honorable y todos los hombres deberían desearlo. El hecho de que el Hijo del Hombre continúe solo su misión terrenal, no es de ninguna manera un rechazo a la deseabilidad del matrimonio. Es voluntad del Padre que yo actúe de esta manera, pero el mismo Padre ha ordenado la creación del hombre y de la mujer, y es voluntad divina que los hombres y las mujeres encuentren su servicio más elevado, y la alegría consiguiente, estableciendo un hogar para recibir y criar a los hijos, en cuya creación estos padres se convierten en asociados de los Hacedores del cielo y de la Tierra. Por esta razón, el hombre dejará a su padre y a su madre para unirse a su mujer, y los dos se volverán como uno solo>>.

\par 
%\textsuperscript{(1839.5)}
\textsuperscript{167:5.8} De esta manera, Jesús liberó la mente de los apóstoles de un gran número de preocupaciones acerca del matrimonio, y aclaró muchos malentendidos relacionados con el divorcio; al mismo tiempo, contribuyó mucho a realzar sus ideales de unión social y a acrecentar su respeto por las mujeres, los niños y el hogar.

\section*{6. La bendición de los niños}
\par 
%\textsuperscript{(1839.6)}
\textsuperscript{167:6.1} El mensaje vespertino de Jesús sobre el matrimonio y la bendición que suponen los niños se difundió por todo Jericó, de manera que, a la mañana siguiente, mucho antes de que Jesús y los apóstoles se prepararan para partir, e incluso antes de la hora del desayuno, decenas de madres llegaron al lugar donde Jesús estaba alojado, trayendo a sus hijos en brazos o llevándolos de la mano, con el deseo de que bendijera a los pequeños. Cuando los apóstoles salieron y vieron esta multitud de madres con sus hijos, intentaron despedirlas, pero estas mujeres se negaron a marcharse hasta que el Maestro hubiera impuesto sus manos sobre sus hijos y los hubiera bendecido. Cuando los apóstoles reprendieron ruidosamente a estas madres, Jesús escuchó el alboroto, salió y los amonestó con indignación, diciendo: <<Dejad que los niños se acerquen a mí; no se lo impidáis, porque de ellos es el reino de los cielos. En verdad, en verdad os digo que el que no reciba el reino de Dios como un niño pequeño, difícilmente entrará en él para crecer hasta la plena estatura de la madurez espiritual>>.

\par 
%\textsuperscript{(1840.1)}
\textsuperscript{167:6.2} Cuando el Maestro hubo hablado a sus apóstoles, recibió a todos los niños y les impuso sus manos mientras decía a sus madres palabras de ánimo y de esperanza.

\par 
%\textsuperscript{(1840.2)}
\textsuperscript{167:6.3} Jesús habló con frecuencia a sus apóstoles de las mansiones celestiales y les enseñó que los hijos de Dios que progresan allí deben crecer espiritualmente, como los niños crecen físicamente en este mundo. De hecho, las cosas sagradas tienen muchas veces una apariencia corriente, y aquel día, aquellos niños y sus madres no se dieron cuenta de que las inteligencias espectadoras de Nebadon contemplaban a los niños de Jericó jugando con el Creador de un universo.

\par 
%\textsuperscript{(1840.3)}
\textsuperscript{167:6.4} La condición de la mujer en Palestina mejoró mucho gracias a las enseñanzas de Jesús; y lo mismo hubiera sucedido en todo el mundo si sus seguidores no se hubieran alejado tanto de lo que el Maestro se había esmerado en enseñarles.

\par 
%\textsuperscript{(1840.4)}
\textsuperscript{167:6.5} Fue también en Jericó, en conexión con una discusión sobre la temprana formación religiosa de los niños en los hábitos de la adoración divina, donde Jesús inculcó a sus apóstoles el gran valor de la belleza como influencia que conduce al impulso de adorar, especialmente entre los niños. Mediante sus preceptos y su ejemplo, el Maestro enseñó el valor de adorar al Creador en medio de los contornos naturales de la creación. Prefería comunicarse con el Padre celestial en medio de los árboles y entre las humildes criaturas del mundo natural. Sentía el regocijo de contemplar al Padre a través del espectáculo inspirador de los reinos estrellados de los Hijos Creadores.

\par 
%\textsuperscript{(1840.5)}
\textsuperscript{167:6.6} Cuando no es posible adorar a Dios en los tabernáculos de la naturaleza, los hombres deberían hacer todo lo posible por tener unas casas llenas de belleza, unos santuarios con una sencillez atrayente y una decoración artística, para que puedan despertarse las emociones humanas más elevadas en asociación con un acercamiento intelectual a la comunión espiritual con Dios. La verdad, la belleza y la santidad son unas ayudas poderosas y eficaces para la verdadera adoración. Pero la comunión espiritual no se fomenta con unos simples adornos masivos y el embellecimiento exagerado del arte esmerado y ostentoso del hombre. La belleza es más religiosa cuando es más sencilla y semejante a la naturaleza. ¡Es una pena que los niños pequeños tengan su primer contacto con los conceptos de la adoración en público en unas salas frías y estériles, tan desprovistas del atractivo de la belleza y tan vacías de toda insinuación a la alegría y a la santidad inspiradora! El niño debería ser iniciado a la adoración en el mundo de la naturaleza, y después acompañar a sus padres a los edificios públicos de las asambleas religiosas, que posean al menos tanto atractivo material y belleza artística como el hogar donde vive cada día.

\section*{7. La conversación sobre los ángeles}
\par 
%\textsuperscript{(1840.6)}
\textsuperscript{167:7.1} Mientras viajaban por las colinas desde Jericó a Betania, Natanael caminó casi todo el tiempo al lado de Jesús, y su discusión sobre la relación de los niños con el reino de los cielos les llevó indirectamente a reflexionar sobre el ministerio de los ángeles. Natanael le hizo finalmente al Maestro la pregunta siguiente: <<Puesto que el sumo sacerdote es un saduceo, y en vista de que los saduceos no creen en los ángeles, ¿qué vamos a enseñarle al pueblo sobre los ministros celestiales?>> Entonces Jesús, entre otras cosas, dijo:

\par 
%\textsuperscript{(1841.1)}
\textsuperscript{167:7.2} <<Las huestes angélicas son una orden distinta de seres creados; son enteramente diferentes a la orden material de criaturas mortales, y funcionan como un grupo distinto de inteligencias del universo. Los ángeles no pertenecen al grupo de criaturas llamadas en las Escrituras ``los Hijos de Dios''; no son tampoco los espíritus glorificados de los mortales que han continuado progresando a través de las mansiones de las alturas. Los ángeles son una creación directa, y no se reproducen. Las huestes angélicas solamente tienen un parentesco espiritual con la raza humana. A medida que el hombre progresa en el viaje hacia el Padre que está en el Paraíso, pasa en un momento dado por un estado de existencia semejante al de los ángeles, pero el hombre mortal nunca se convierte en un ángel>>.

\par 
%\textsuperscript{(1841.2)}
\textsuperscript{167:7.3} <<Los ángeles no mueren nunca, como mueren los hombres. Los ángeles son inmortales, a menos que se impliquen en el pecado, como hicieron algunos de ellos con los engaños de Lucifer. Los ángeles son los servidores espirituales en el cielo, y no son infinitamente sabios ni todopoderosos. Pero todos los ángeles leales son realmente puros y santos>>.

\par 
%\textsuperscript{(1841.3)}
\textsuperscript{167:7.4} <<¿No recuerdas que ya os he dicho en otra ocasión que si vuestros ojos espirituales fueran ungidos, entonces veríais los cielos abiertos y contemplaríais a los ángeles de Dios subiendo y bajando? El ministerio de los ángeles es el que hace posible que un mundo pueda mantenerse en contacto con otros mundos, porque ¿no os he dicho repetidas veces que tengo otras ovejas que no pertenecen a este redil? Estos ángeles no son los espías del mundo espiritual, que os vigilan, y luego van a contarle al Padre los pensamientos de vuestro corazón y a informarle de las acciones de la carne. El Padre no tiene necesidad de ese servicio, ya que su propio espíritu vive dentro de vosotros. Pero estos espíritus angélicos se ocupan de mantener informada a una parte de la creación celestial acerca de los acontecimientos que se producen en otras partes lejanas del universo. Muchos ángeles están asignados al servicio de las razas humanas, mientras ejercen su actividad en el gobierno del Padre y en los universos de los Hijos. Cuando os enseñé que muchos de estos serafines eran espíritus ministrantes, no os lo decía en un lenguaje figurado ni en términos poéticos. Todo esto es verdad, independientemente de vuestra dificultad para comprender estas cosas>>.

\par 
%\textsuperscript{(1841.4)}
\textsuperscript{167:7.5} <<Muchos de estos ángeles están ocupados en la tarea de salvar a los hombres, porque, ¿no os he hablado de la alegría seráfica, cuando un alma escoge abandonar el pecado y empezar la búsqueda de Dios? Os he hablado incluso de la alegría en la \textit{presencia de los ángeles} del cielo cuando un pecador se arrepiente, señalando de este modo la existencia de otras órdenes más elevadas de seres celestiales, que se ocupan igualmente del bienestar espiritual y del progreso divino del hombre mortal>>.

\par 
%\textsuperscript{(1841.5)}
\textsuperscript{167:7.6} <<Estos ángeles también están muy relacionados con los medios a través de los cuales el espíritu del hombre es liberado de los tabernáculos de la carne y su alma acompañada hasta las mansiones del cielo. Los ángeles son los guías seguros y celestiales del alma del hombre durante ese período de tiempo desconocido e indeterminado que transcurre entre la muerte física y la nueva vida en las moradas espirituales>>.

\par 
%\textsuperscript{(1841.6)}
\textsuperscript{167:7.7} Jesús hubiera continuado hablando con Natanael sobre el ministerio de los ángeles, pero fue interrumpido por la llegada de Marta, a quien unos amigos le habían informado que el Maestro se acercaba a Betania, pues lo habían visto subir las colinas del este. Y ahora Marta se apresuraba a recibirlo.


\chapter{Documento 168. La resurrección de Lázaro}
\par 
%\textsuperscript{(1842.1)}
\textsuperscript{168:0.1} POCO después del mediodía, Marta salió al encuentro de Jesús cuando éste atravesaba la cima de la colina cerca de Betania. Su hermano Lázaro había muerto hacía cuatro días y el domingo al anochecer había sido colocado en el sepulcro de la familia, situado en un extremo del jardín. Este mismo jueves por la mañana, habían hecho rodar la piedra a la entrada de la tumba.

\par 
%\textsuperscript{(1842.2)}
\textsuperscript{168:0.2} Cuando Marta y María enviaron a Jesús el aviso de la enfermedad de Lázaro, confiaban en que el Maestro haría algo al respecto. Sabían que su hermano estaba irremediablemente enfermo, y aunque apenas se atrevían a esperar que Jesús dejara su trabajo de enseñanza y predicación para venir a ayudarlos, tenían tanta confianza en su poder de curar las enfermedades, que pensaron que le bastaría con pronunciar las palabras curativas y Lázaro recuperaría inmediatamente la salud. Cuando Lázaro murió pocas horas después de que el mensajero saliera de Betania hacia Filadelfia, dedujeron que el Maestro no se había enterado de la enfermedad de su hermano hasta que fue demasiado tarde, hasta que ya estaba muerto desde hacía varias horas.

\par 
%\textsuperscript{(1842.3)}
\textsuperscript{168:0.3} Sin embargo, se sintieron muy desconcertadas, al igual que todos sus amigos creyentes, por el mensaje que trajo el corredor cuando llegó a Betania el martes por la mañana. El mensajero insistió en que había oído decir a Jesús: <<... esta enfermedad no le llevará realmente a la muerte>>. Tampoco podían comprender por qué Jesús no les había enviado ningún mensaje, ni les había ofrecido su ayuda de alguna otra manera.

\par 
%\textsuperscript{(1842.4)}
\textsuperscript{168:0.4} Muchos amigos de las aldeas vecinas, y otros de Jerusalén, vinieron a consolar a las hermanas que estaban muy afligidas. Lázaro y sus hermanas eran los hijos de un judío ilustre y acaudalado, que había sido el vecino principal del pueblecito de Betania. A pesar de que los tres habían sido, desde hacía tiempo, unos discípulos apasionados de Jesús, eran sumamente respetados por todos los que los conocían. Habían heredado unos grandes viñedos y huertos de olivos en aquellas proximidades, y el hecho de que pudieran permitirse un sepulcro privado en sus propias tierras era una prueba más de su riqueza. Sus padres ya habían sido enterrados en este sepulcro.

\par 
%\textsuperscript{(1842.5)}
\textsuperscript{168:0.5} María había renunciado a la idea de la venida de Jesús y se había entregado a su aflicción, pero Marta se aferró a la esperanza de que Jesús vendría, y la conservó hasta el momento en que hicieron rodar la piedra delante de la tumba, aquella misma mañana, y sellaron la entrada. E incluso entonces, encargó a un joven vecino que vigilara la carretera de Jericó desde la cima de la colina al este de Betania; éste fue el muchacho que le llevó a Marta la noticia de que Jesús y sus amigos se acercaban.

\par 
%\textsuperscript{(1842.6)}
\textsuperscript{168:0.6} Cuando Marta se encontró con Jesús, cayó a sus pies, exclamando: <<Maestro, ¡si hubieras estado aquí, mi hermano no habría muerto!>> Muchos temores atravesaban la mente de Marta, pero no expresó ninguna duda ni se atrevió a criticar o a poner en tela de juicio la conducta del Maestro en relación con la muerte de Lázaro. Cuando hubo terminado de hablar, Jesús se inclinó para levantarla, y le dijo: <<Ten fe únicamente, Marta, y tu hermano resucitará>>. Entonces Marta contestó: <<Sé que resucitará en la resurrección del último día; e incluso ahora creo que nuestro Padre te concederá todo lo que le pidas a Dios>>.

\par 
%\textsuperscript{(1843.1)}
\textsuperscript{168:0.7} Entonces Jesús miró a Marta fijamente a los ojos, y le dijo: <<Yo soy la resurrección y la vida; el que cree en mí, aunque muera, vivirá. En verdad, cualquiera que vive y cree en mí no morirá nunca realmente. Marta, ¿crees esto?>> Y Marta respondió al Maestro: <<Sí, creo desde hace mucho tiempo que tú eres el Libertador, el Hijo del Dios vivo, aquel que debía venir a este mundo>>.

\par 
%\textsuperscript{(1843.2)}
\textsuperscript{168:0.8} Cuando Jesús preguntó por María, Marta se dirigió inmediatamente a la casa y le dijo a su hermana en voz baja: <<El Maestro está aquí y ha preguntado por ti>>. Cuando María escuchó esto, se levantó en seguida y salió apresuradamente para ir a recibir a Jesús, que permanecía en el mismo lugar donde Marta lo había encontrado primero, a cierta distancia de la casa. Cuando los amigos que estaban con María, tratando de consolarla, vieron que se levantaba rápidamente y salía, la siguieron suponiendo que iba a la tumba para llorar.

\par 
%\textsuperscript{(1843.3)}
\textsuperscript{168:0.9} Muchos de los presentes eran enemigos encarnizados de Jesús. Por eso Marta había salido para encontrarse con él a solas, y por eso también había entrado para informar en secreto a María de que el Maestro había preguntado por ella. Aunque Marta anhelaba ver a Jesús, deseaba evitar que su llegada repentina en medio de un grupo numeroso de sus enemigos de Jerusalén pudiera ocasionar alguna posible situación desagradable. Marta había tenido la intención de permanecer en la casa con sus amigos mientras María iba a saludar a Jesús, pero no lo consiguió, porque todos siguieron a María, y se encontraron así de manera inesperada en presencia del Maestro.

\par 
%\textsuperscript{(1843.4)}
\textsuperscript{168:0.10} Marta llevó a María ante Jesús, y cuando ésta lo vio, cayó a sus pies, exclamando: <<¡Si tan sólo hubieras estado aquí, mi hermano no hubiera muerto!>> Cuando Jesús vio hasta qué punto estaban todos afligidos por la muerte de Lázaro, su alma se llenó de compasión.

\par 
%\textsuperscript{(1843.5)}
\textsuperscript{168:0.11} Cuando los acompañantes vieron que María había ido a saludar a Jesús, se apartaron a corta distancia, mientras Marta y María hablaban con el Maestro; recibieron palabras adicionales de consuelo y una exhortación a que conservaran una fe firme en el Padre y se conformaran por completo a la voluntad divina.

\par 
%\textsuperscript{(1843.6)}
\textsuperscript{168:0.12} La mente humana de Jesús se conmovió poderosamente debido al conflicto entre su amor por Lázaro y las desoladas hermanas, y su desprecio y desdén por las muestras exteriores de afecto que manifestaban algunos de estos judíos incrédulos y con intenciones asesinas. A Jesús le causaba indignación que algunos de estos supuestos amigos mostraran una aflicción forzada y externa por Lázaro, cuando esa falsa pena estaba acompañada en sus corazones por una enemistad tan implacable contra él. Sin embargo, algunos de estos judíos eran sinceros en su luto, pues eran verdaderos amigos de la familia.

\section*{1. En la tumba de Lázaro}
\par 
%\textsuperscript{(1843.7)}
\textsuperscript{168:1.1} Después de que Jesús hubiera pasado unos momentos consolando a Marta y María, apartados de los acompañantes, les preguntó: <<¿Dónde lo habéis puesto?>> Entonces Marta dijo: <<Ven a ver>>. Mientras el Maestro seguía en silencio a las dos hermanas afligidas, lloró. Cuando los judíos amistosos que los seguían vieron sus lágrimas, uno de ellos dijo: <<Mirad cómo lo amaba. El que abrió los ojos del ciego, ¿no podría haber impedido la muerte de este hombre?>> Para entonces ya se encontraban delante del sepulcro familiar, que era una pequeña cueva natural, o declive, en el saliente de una roca de unos diez metros de altura, situada en el extremo más alejado del jardín.

\par 
%\textsuperscript{(1844.1)}
\textsuperscript{168:1.2} Es difícil explicar a la mente humana por qué exactamente lloró Jesús. Aunque tenemos acceso al registro de las emociones humanas y de los pensamientos divinos conjuntos de Jesús, tal como constan en la mente del Ajustador Personalizado, no estamos totalmente seguros de la causa real de estas manifestaciones emocionales. Tendemos a creer que Jesús lloró debido a una cantidad de pensamientos y sentimientos que atravesaban su mente en aquel momento, tales como:

\par 
%\textsuperscript{(1844.2)}
\textsuperscript{168:1.3} 1. Sentía una compasión sincera y dolorosa por Marta y María; tenía un afecto humano real y profundo por estas hermanas que habían perdido a su hermano.

\par 
%\textsuperscript{(1844.3)}
\textsuperscript{168:1.4} 2. Se sentía mentalmente agitado por la presencia de la multitud de acompañantes, algunos sinceros y otros simplemente hipócritas. Siempre le molestaban estas manifestaciones exteriores de duelo. Sabía que las hermanas amaban a su hermano y que tenían fe en la supervivencia de los creyentes. Estas emociones contradictorias quizás explican por qué lloró cuando se acercaban a la tumba.

\par 
%\textsuperscript{(1844.4)}
\textsuperscript{168:1.5} 3. Dudaba sinceramente en devolverle a Lázaro la vida mortal. Sus hermanas lo necesitaban realmente, pero Jesús lamentaba tener que llamar a su amigo para que luego tuviera que experimentar una cruel persecución; sabía muy bien que Lázaro tendría que sufrirla por haber sido el objeto de la demostración más grande de poder divino del Hijo del Hombre.

\par 
%\textsuperscript{(1844.5)}
\textsuperscript{168:1.6} Y ahora podemos contar un hecho interesante e instructivo: Aunque este relato se desarrolla como un acontecimiento aparentemente natural y normal de los asuntos humanos, tiene algunos aspectos indirectos muy interesantes. Aunque el mensajero fue a ver a Jesús el domingo para informarle de la enfermedad de Lázaro, y aunque Jesús envió un mensaje indicando que <<no le llevaría a la muerte>>, sin embargo fue personalmente hasta Betania, e incluso preguntó a las hermanas: <<¿Dónde lo habéis puesto?>> Todo esto parece indicar que el Maestro actuaba a la manera de esta vida y de acuerdo con los conocimientos limitados de la mente humana. Sin embargo, los archivos del universo revelan que el Ajustador Personalizado de Jesús emitió unas órdenes para que se retuviera indefinidamente en el planeta al Ajustador del Pensamiento de Lázaro, después de su muerte, y que esta orden se registró apenas quince minutos antes de que Lázaro exhalara su último suspiro.

\par 
%\textsuperscript{(1844.6)}
\textsuperscript{168:1.7} ¿Sabía la mente divina de Jesús, incluso antes de que Lázaro muriera, que lo resucitaría de entre los muertos? No lo sabemos. Sólo sabemos lo que indicamos aquí.

\par 
%\textsuperscript{(1844.7)}
\textsuperscript{168:1.8} Muchos enemigos de Jesús tendían a mofarse de sus manifestaciones de afecto, y decían entre ellos: <<Si tanto apreciaba a este hombre, ¿por qué esperó tanto para venir a Betania? Si él es lo que ellos pretenden, ¿por qué no ha salvado a su querido amigo? ¿Para qué sirve curar a los desconocidos en Galilea, si no puede salvar a los que ama?>> Y de otras muchas maneras se burlaron y le restaron importancia a las obras y enseñanzas de Jesús.

\par 
%\textsuperscript{(1844.8)}
\textsuperscript{168:1.9} Y así, hacia las dos y media de la tarde de este jueves, todo el escenario estaba preparado en esta pequeña aldea de Betania para la representación de la obra más grande de todas las relacionadas con el ministerio terrenal de Miguel de Nebadon, para la manifestación más grande de poder divino que se produjo durante su encarnación, puesto que su propia resurrección tuvo lugar después de que hubiera sido liberado de las cadenas de la morada mortal.

\par 
%\textsuperscript{(1845.1)}
\textsuperscript{168:1.10} El pequeño grupo reunido delante de la tumba de Lázaro poco podía imaginar que allí cerca se encontraba presente una enorme multitud de todas las órdenes de seres celestiales, congregados bajo la dirección de Gabriel y ahora en espera por mandato del Ajustador Personalizado de Jesús, vibrando de expectación y preparados para ejecutar las órdenes de su amado Soberano.

\par 
%\textsuperscript{(1845.2)}
\textsuperscript{168:1.11} Cuando Jesús pronunció aquellas palabras, ordenando: <<Quitad la piedra>>, las huestes celestiales reunidas se prepararon para representar el drama de la resurrección de Lázaro en la similitud de su carne mortal. Esta forma de resurrección implica unas dificultades de ejecución que trascienden de lejos la técnica habitual de la resurrección de las criaturas mortales en el estado morontial, y necesita muchas más personalidades celestiales y una organización mucho mayor de recursos universales.

\par 
%\textsuperscript{(1845.3)}
\textsuperscript{168:1.12} Cuando Marta y María escucharon este mandato de Jesús ordenando que se quitara la piedra que estaba delante de la tumba, se llenaron de emociones contradictorias. María esperaba que Lázaro fuera resucitado de entre los muertos, pero Marta, aunque compartía hasta cierto punto la fe de su hermana, estaba más preocupada por el temor de que la apariencia de Lázaro no fuera presentable para Jesús, los apóstoles y sus amigos. Marta dijo: <<¿Tenemos que quitar la piedra? Mi hermano ya lleva muerto cuatro días, de manera que la descomposición del cuerpo ya ha empezado>>. Marta dijo esto también porque no estaba segura de la razón por la que el Maestro había pedido que se apartara la piedra; pensaba que Jesús quizás sólo quería echarle una última mirada a Lázaro. La actitud de Marta no era firme ni constante. Como dudaban en quitar la piedra, Jesús dijo: <<¿No os he dicho desde el principio que esta enfermedad no le llevaría a la muerte? ¿No he venido para cumplir mi promesa? Y después de llegar hasta vosotras, ¿no he dicho que, si tan sólo creyerais, veríais la gloria de Dios? ¿Por qué dudáis? ¿Cuánto tiempo necesitaréis para creer y obedecer?>>

\par 
%\textsuperscript{(1845.4)}
\textsuperscript{168:1.13} Cuando Jesús hubo terminado de hablar, sus apóstoles, con la ayuda de unos vecinos voluntarios, agarraron la piedra y la hicieron rodar hasta quitarla de la entrada de la tumba.

\par 
%\textsuperscript{(1845.5)}
\textsuperscript{168:1.14} Los judíos tenían la creencia común de que la gota de hiel situada en la punta de la espada del ángel de la muerte empezaba a actuar al final del tercer día, de manera que la totalidad de su efecto se producía al cuarto día. Admitían que el alma del hombre podía demorarse cerca de la tumba hasta el final del tercer día, tratando de reanimar el cadáver; pero creían firmemente que antes del amanecer del cuarto día, ese alma se había ido a la morada de los espíritus difuntos.

\par 
%\textsuperscript{(1845.6)}
\textsuperscript{168:1.15} Estas creencias y opiniones acerca de los muertos y de la partida de los espíritus de los muertos, sirvieron para asegurar en la mente de todos los que ahora estaban presentes en la tumba de Lázaro, y de todos los que pudieran enterarse posteriormente de lo que estaba a punto de suceder, que éste era un caso real y verdadero de resurrección de entre los muertos, debido a un acto personal de aquel que había declarado ser <<la resurrección y la vida>>.

\section*{2. La resurrección de Lázaro}
\par 
%\textsuperscript{(1845.7)}
\textsuperscript{168:2.1} Mientras este grupo de unos cuarenta y cinco mortales permanecía delante de la tumba, pudieron ver vagamente la forma de Lázaro, envuelta en unos vendajes de lino, descansando en el nicho inferior derecho de la cueva fúnebre. Mientras estas criaturas terrenales se hallaban allí en silencio, casi sin aliento, una enorme hueste de seres celestiales se había situado en sus puestos preliminares, para responder a la señal de actuar en cuanto la diera su comandante Gabriel.

\par 
%\textsuperscript{(1846.1)}
\textsuperscript{168:2.2} Jesús levantó los ojos y dijo: <<Padre, te doy las gracias por haber escuchado y concedido mi petición. Sé que me escuchas siempre, pero te hablo así a causa de aquellos que están aquí conmigo, para que puedan creer que me has enviado al mundo, y para que sepan que actúas conmigo en esto que estamos a punto de realizar>>. Cuando hubo terminado de orar, dijo en voz alta: <<Lázaro, ¡sal fuera!>>

\par 
%\textsuperscript{(1846.2)}
\textsuperscript{168:2.3} Los espectadores humanos permanecieron inmóviles, pero toda la inmensa hueste celestial bullía en una acción unificada, obedeciendo la palabra del Creador. En sólo doce segundos del tiempo terrestre, la forma hasta entonces inanimada de Lázaro empezó a moverse, y pronto se sentó en el borde de la plataforma de piedra donde había descansado. Su cuerpo estaba envuelto en las mortajas y su rostro cubierto con un paño. Mientras permanecía de pie delante de ellos ---vivo--- Jesús dijo: <<Desatadlo y dejadlo salir>>.

\par 
%\textsuperscript{(1846.3)}
\textsuperscript{168:2.4} Todos los espectadores, salvo los apóstoles así como Marta y María, huyeron hacia la casa. Estaban pálidos de terror y abrumados por el asombro. Aunque algunos permanecieron allí, muchos regresaron apresuradamente a sus hogares.

\par 
%\textsuperscript{(1846.4)}
\textsuperscript{168:2.5} Lázaro saludó a Jesús y a los apóstoles, preguntó por el significado de las mortajas y por qué se había despertado en el jardín. Jesús y los apóstoles se apartaron, mientras Marta le contaba a Lázaro su muerte, entierro y resurrección. Tuvo que explicarle que había muerto el domingo y que ahora había sido devuelto a la vida el jueves, ya que Lázaro no había tenido conciencia del tiempo desde que había caído en el sueño de la muerte.

\par 
%\textsuperscript{(1846.5)}
\textsuperscript{168:2.6} Mientras Lázaro salía de la tumba, el Ajustador Personalizado de Jesús, ahora jefe de su orden en este universo local, ordenó al antiguo Ajustador de Lázaro, entonces en espera, que volviera a residir en la mente y el alma del resucitado.

\par 
%\textsuperscript{(1846.6)}
\textsuperscript{168:2.7} Luego Lázaro se acercó a Jesús y, junto con sus hermanas, se arrodilló a los pies del Maestro para dar gracias y alabar a Dios. Jesús cogió a Lázaro de la mano, y lo levantó diciendo: <<Hijo mío, lo que te ha sucedido será experimentado también por todos los que creen en este evangelio, excepto que serán resucitados con una forma más gloriosa. Serás un testigo viviente de la verdad que he proclamado ---yo soy la resurrección y la vida. Pero ahora entremos todos en la casa y tomemos algún alimento para estos cuerpos físicos>>.

\par 
%\textsuperscript{(1846.7)}
\textsuperscript{168:2.8} Mientras caminaban hacia la casa, Gabriel disolvió los grupos adicionales de las huestes celestiales reunidas, y procedió a registrar el primer y último caso, sucedido en Urantia, en el que una criatura mortal había sido resucitada en la similitud de su cuerpo físico mortal.

\par 
%\textsuperscript{(1846.8)}
\textsuperscript{168:2.9} Lázaro apenas podía comprender lo que había sucedido. Sabía que había estado muy enfermo, pero sólo podía recordar que se había dormido y que había sido despertado. Nunca pudo decir nada sobre aquellos cuatro días en la tumba, porque había estado totalmente inconsciente. El tiempo no existe para aquellos que duermen el sueño de la muerte.

\par 
%\textsuperscript{(1846.9)}
\textsuperscript{168:2.10} Muchos creyeron en Jesús a consecuencia de esta obra poderosa, pero otros sólo endurecieron su corazón para rechazarlo aún más. Al día siguiente al mediodía, esta historia se había difundido por todo Jerusalén. Decenas de hombres y mujeres fueron a Betania para contemplar a Lázaro y hablar con él, y los fariseos, alarmados y desconcertados, convocaron apresuradamente una reunión del sanedrín para determinar lo que había que hacer con respecto a estos nuevos acontecimientos.

\section*{3. La reunión del sanedrín}
\par 
%\textsuperscript{(1847.1)}
\textsuperscript{168:3.1} Aunque el testimonio de este hombre resucitado de entre los muertos contribuyó mucho a consolidar la fe de la masa de creyentes en el evangelio del reino, tuvo poca o ninguna influencia sobre la actitud de los jefes y dirigentes religiosos de Jerusalén, excepto que apresuró su decisión de destruir a Jesús y de poner fin a su obra.

\par 
%\textsuperscript{(1847.2)}
\textsuperscript{168:3.2} Al día siguiente, viernes, el sanedrín se reunió a la una para deliberar de nuevo sobre la cuestión: <<¿Qué vamos a hacer con Jesús de Nazaret?>> Después de más de dos horas de discusiones y debates enconados, cierto fariseo propuso una resolución pidiendo la muerte inmediata de Jesús, proclamando que era una amenaza para todo Israel y comprometiendo formalmente al sanedrín para que decidiera su muerte, sin juicio y haciendo caso omiso de todo precedente.

\par 
%\textsuperscript{(1847.3)}
\textsuperscript{168:3.3} Este augusto cuerpo de dirigentes judíos había decretado una y otra vez que Jesús debía ser apresado y sometido a juicio, inculpado de blasfemia y de otras muchas acusaciones de desacato a la ley sagrada judía. En una ocasión anterior habían llegado incluso a declarar que debía morir, pero ésta era la primera vez que el sanedrín indicaba el deseo de decretar su muerte con antelación a todo juicio. Pero esta resolución no fue puesta a votación, ya que catorce miembros del sanedrín dimitieron en masa cuando se propuso esta acción inaudita. Aunque estas dimisiones no tuvieron efecto oficial durante casi dos semanas, este grupo de catorce se separó del sanedrín aquel día y no volvió a sentarse nunca más en el consejo. Cuando estas dimisiones fueron aceptadas posteriormente, cinco miembros más fueron expulsados porque sus colegas opinaban que albergaban sentimientos amistosos hacia Jesús. Con la expulsión de estos diecinueve hombres, el sanedrín estaba en disposiciones de juzgar y condenar a Jesús con una solidaridad que rozaba la unanimidad.

\par 
%\textsuperscript{(1847.4)}
\textsuperscript{168:3.4} A la semana siguiente, Lázaro y sus hermanas fueron convocados ante el sanedrín. Después de haberse escuchado el testimonio de los tres, no se podía albergar ninguna duda de que Lázaro había sido resucitado de entre los muertos. Aunque los anales del sanedrín admitían prácticamente la resurrección de Lázaro, el registro contenía una resolución que atribuía este prodigio, y todos los demás realizados por Jesús, al poder del príncipe de los demonios, declarándose que Jesús estaba aliado con él.

\par 
%\textsuperscript{(1847.5)}
\textsuperscript{168:3.5} Sea cual fuere el origen de su poder para realizar prodigios, estos dirigentes judíos estaban persuadidos de que si no lo paraban de inmediato, muy pronto toda la gente corriente creería en él, y que además surgirían graves complicaciones con las autoridades romanas, puesto que muchos de sus creyentes lo consideraban como el Mesías, el libertador de Israel.

\par 
%\textsuperscript{(1847.6)}
\textsuperscript{168:3.6} En esta misma reunión del sanedrín fue donde el sumo sacerdote Caifás expresó por primera vez el viejo dicho judío, que luego repitió tantas veces: <<Es mejor que muera un solo hombre, a que perezca la comunidad>>.

\par 
%\textsuperscript{(1847.7)}
\textsuperscript{168:3.7} Aunque Jesús había recibido aviso de las acciones del sanedrín durante este sombrío viernes por la tarde, no se inquietó en lo más mínimo y continuó descansando todo el sábado con unos amigos en Betfagé, una aldea cercana a Betania. El domingo por la mañana temprano, Jesús y los apóstoles se reunieron, como habían convenido, en la casa de Lázaro, se despidieron de la familia de Betania, y emprendieron su viaje de vuelta al campamento de Pella.

\section*{4. La respuesta a la oración}
\par 
%\textsuperscript{(1848.1)}
\textsuperscript{168:4.1} En el camino desde Betania a Pella, los apóstoles hicieron muchas preguntas a Jesús y el Maestro contestó sin reparos a todas ellas, excepto a las relacionadas con los detalles de la resurrección de los muertos. Estos problemas sobrepasaban la capacidad de comprensión de sus apóstoles, y por eso el Maestro rehusó discutir estas cuestiones con ellos. Como habían partido de Betania en secreto, nadie los acompañaba. Por consiguiente, Jesús aprovechó la ocasión para decirle muchas cosas a los diez que, en su opinión, los prepararía para los días difíciles que se avecinaban.

\par 
%\textsuperscript{(1848.2)}
\textsuperscript{168:4.2} Los apóstoles tenían la mente muy excitada y pasaron bastante tiempo discutiendo de sus experiencias recientes relacionadas con la oración y la respuesta a la oración. Todos recordaban la declaración que Jesús había hecho en Filadelfia al mensajero de Betania, cuando dijo claramente: <<Esta enfermedad no le llevará realmente a la muerte>>. Sin embargo, a pesar de esta promesa, Lázaro había muerto realmente. Durante todo aquel día, volvieron a hablar una y otra vez de este problema de la respuesta a la oración.

\par 
%\textsuperscript{(1848.3)}
\textsuperscript{168:4.3} Las respuestas de Jesús a sus numerosas preguntas se pueden resumir como sigue:

\par 
%\textsuperscript{(1848.4)}
\textsuperscript{168:4.4} 1. La oración es una expresión de la mente finita en su esfuerzo por acercarse al Infinito. Por consiguiente, la formulación de una oración está necesariamente limitada por el conocimiento, la sabiduría y los atributos de lo finito; del mismo modo, la respuesta ha de estar condicionada por la visión, los objetivos, los ideales y las prerrogativas del Infinito. Nunca se puede observar una continuidad ininterrumpida de fenómenos materiales entre la formulación de una oración y la recepción de la plena respuesta espiritual a la misma.

\par 
%\textsuperscript{(1848.5)}
\textsuperscript{168:4.5} 2. Cuando una oración se queda aparentemente sin respuesta, el retraso es a menudo el presagio de una respuesta mejor, aunque esa respuesta se demore considerablemente por alguna buena razón. Cuando Jesús dijo que la enfermedad de Lázaro no le llevaría realmente hasta la muerte, éste ya había muerto hacía once horas. Ninguna oración sincera se queda sin respuesta, salvo cuando el punto de vista superior del mundo espiritual ha concebido una respuesta mejor, una respuesta que satisface la petición del espíritu del hombre en contraposición con la oración de la simple mente humana.

\par 
%\textsuperscript{(1848.6)}
\textsuperscript{168:4.6} 3. Cuando las oraciones temporales son compuestas por el espíritu y expresadas con fe, a menudo son tan amplias y abarcan tantas cosas que sólo se pueden contestar en la eternidad; a veces, la súplica finita está tan llena del deseo de captar lo Infinito, que la respuesta debe ser aplazada durante mucho tiempo a fin de esperar la creación de la capacidad adecuada para recibirla; la oración de la fe puede abarcar tanto, que la respuesta sólo se puede recibir en el Paraíso.

\par 
%\textsuperscript{(1848.7)}
\textsuperscript{168:4.7} 4. Las respuestas a la oración de la mente mortal son a menudo de tal naturaleza, que sólo se pueden recibir y reconocer después de que esa misma mente que ora ha alcanzado el estado inmortal. Muchas veces, la oración de un ser material sólo se puede contestar cuando ese individuo ha progresado hasta el nivel del espíritu.

\par 
%\textsuperscript{(1848.8)}
\textsuperscript{168:4.8} 5. La oración de una persona que conoce a Dios puede estar tan distorsionada por la ignorancia y tan deformada por la superstición, que responder a la misma sería muy poco deseable. Los seres espirituales intermedios tienen entonces que traducir de tal manera esa oración que, cuando llega la respuesta, el peticionario no logra reconocer en absoluto que se trata de la respuesta a su oración.

\par 
%\textsuperscript{(1848.9)}
\textsuperscript{168:4.9} 6. Todas las oraciones verdaderas son dirigidas a los seres espirituales, y todas esas peticiones deben ser contestadas en términos espirituales, y todas esas respuestas deben consistir en realidades espirituales. Los seres espirituales no pueden ofrecer respuestas materiales ni siquiera a las súplicas espirituales de los seres materiales. Los seres materiales sólo pueden orar eficazmente cuando <<oran en espíritu>>.

\par 
%\textsuperscript{(1849.1)}
\textsuperscript{168:4.10} 7. Ninguna oración puede esperar una respuesta a menos que haya nacido del espíritu y haya sido alimentada por la fe. Vuestra fe sincera implica que habéis concedido prácticamente de antemano, a los que escuchan vuestra oración, el pleno derecho de contestar a vuestras súplicas de acuerdo con esa sabiduría suprema y ese amor divino que, según describe vuestra fe, impulsan siempre a esos seres a quienes dirigís vuestras oraciones.

\par 
%\textsuperscript{(1849.2)}
\textsuperscript{168:4.11} 8. El niño siempre está en su derecho cuando se atreve a dirigir una petición al padre; y el padre siempre cumple con sus obligaciones paternales hacia el niño inmaduro cuando su sabiduría superior le dicta que retrase la respuesta a la súplica del niño, la modifique, la divida, la trascienda o la aplace hasta otra fase de su ascensión espiritual.

\par 
%\textsuperscript{(1849.3)}
\textsuperscript{168:4.12} 9. No vaciléis en formular las oraciones que expresan los anhelos del espíritu; no dudéis de que vuestras súplicas recibirán una respuesta. Esas respuestas permanecerán en depósito, esperando a que hayáis alcanzado, en este mundo o en otros, esos niveles espirituales futuros de verdadera consecución cósmica, en los que os será posible reconocer y apropiaros de las respuestas tanto tiempo esperadas a vuestras peticiones anteriores pero inoportunas.

\par 
%\textsuperscript{(1849.4)}
\textsuperscript{168:4.13} 10. Todas las súplicas sinceras nacidas del espíritu recibirán, con certeza, una respuesta. Pedid y recibiréis. Pero debéis recordar que sois unas criaturas que progresan en el tiempo y el espacio; por eso tenéis que contar constantemente con el factor espacio-temporal en vuestra experiencia de recibir personalmente las respuestas completas a vuestras diversas oraciones y peticiones.

\section*{5. ¿Qué fue de Lázaro?}
\par 
%\textsuperscript{(1849.5)}
\textsuperscript{168:5.1} Lázaro permaneció en su casa de Betania, donde fue un centro de gran interés para muchos creyentes sinceros y numerosos curiosos, hasta la semana de la crucifixión de Jesús, momento en que recibió la advertencia de que el sanedrín había decretado su muerte. Los dirigentes de los judíos estaban decididos a poner fin a la difusión ulterior de las enseñanzas de Jesús, y estimaron acertadamente que sería inútil hacer morir a Jesús si permitían que Lázaro, el cual representaba el apogeo mismo de sus obras prodigiosas, viviera y diera testimonio del hecho de que Jesús lo había resucitado de entre los muertos. Lázaro ya había sufrido crueles persecuciones por parte de ellos.

\par 
%\textsuperscript{(1849.6)}
\textsuperscript{168:5.2} Así pues, Lázaro se despidió apresuradamente de sus hermanas en Betania, huyó hacia Jericó, atravesó el Jordán, y no se permitió ningún largo descanso hasta haber llegado a Filadelfia. Lázaro conocía bien a Abner, y aquí se sentía a salvo de las intrigas asesinas del malvado sanedrín.

\par 
%\textsuperscript{(1849.7)}
\textsuperscript{168:5.3} Poco después de esto, Marta y María vendieron sus tierras de Betania y se reunieron con su hermano en Perea. Entretanto, Lázaro se había convertido en el tesorero de la iglesia de Filadelfia. Apoyó firmemente a Abner en su controversia con Pablo y la iglesia de Jerusalén, y murió finalmente, a los 67 años de edad, de la misma enfermedad que se lo había llevado en Betania cuando era más joven.


\chapter{Documento 169. La última enseñanza en Pella}
\par 
%\textsuperscript{(1850.1)}
\textsuperscript{169:0.1} JESÚS y los diez apóstoles llegaron al campamento de Pella el lunes 6 de marzo al caer la tarde. Ésta fue la última semana que Jesús pasó allí, y estuvo muy activo enseñando a la muchedumbre e instruyendo a los apóstoles. Todas las tardes predicaba a las multitudes, y todas las noches respondía a las preguntas de los apóstoles y de algunos de los discípulos más avanzados que residían en el campamento.

\par 
%\textsuperscript{(1850.2)}
\textsuperscript{169:0.2} La noticia de la resurrección de Lázaro había llegado al campamento dos días antes de la llegada del Maestro, y toda la asamblea estaba llena de curiosidad. Desde el episodio de la alimentación de los cinco mil, no había sucedido nada que excitara tanto la imaginación de la gente. Así es como en la cumbre misma de la segunda fase del ministerio público del reino, Jesús planeó enseñar durante esta sola y corta semana en Pella, para luego empezar la gira por el sur de Perea, que conduciría directamente a las experiencias finales y trágicas de la última semana en Jerusalén.

\par 
%\textsuperscript{(1850.3)}
\textsuperscript{169:0.3} Los fariseos y los sacerdotes principales habían empezado a formular sus cargos y a cristalizar sus acusaciones. Se oponían a las enseñanzas del Maestro por los motivos siguientes:

\par 
%\textsuperscript{(1850.4)}
\textsuperscript{169:0.4} 1. Es amigo de los publicanos y de los pecadores; recibe a los impíos e incluso come con ellos.

\par 
%\textsuperscript{(1850.5)}
\textsuperscript{169:0.5} 2. Es un blasfemo; habla de Dios como si fuera su Padre y piensa que es igual a Dios.

\par 
%\textsuperscript{(1850.6)}
\textsuperscript{169:0.6} 3. Es un infractor de la ley. Cura las enfermedades durante el sábado y se burla de otras muchas maneras de la ley sagrada de Israel.

\par 
%\textsuperscript{(1850.7)}
\textsuperscript{169:0.7} 4. Está aliado con los demonios. Realiza prodigios y hace milagros aparentes por el poder de Belcebú, el príncipe de los demonios.

\section*{1. La parábola del hijo perdido}
\par 
%\textsuperscript{(1850.8)}
\textsuperscript{169:1.1} El jueves por la tarde, Jesús habló a la multitud sobre la <<Gracia de la salvación>>. En el transcurso de este sermón, volvió a contar la historia de la oveja perdida y de la moneda perdida, y luego añadió su parábola favorita del hijo pródigo. Jesús dijo:

\par 
%\textsuperscript{(1850.9)}
\textsuperscript{169:1.2} <<Desde Samuel hasta Juan, los profetas os han exhortado a buscar a Dios ---a buscar la verdad. Siempre os han dicho: `Buscad al Señor mientras se le puede encontrar.' Toda esta enseñanza debería tomarse en serio. Pero yo he venido a mostraros que, mientras tratáis de encontrar a Dios, Dios también trata de encontraros a vosotros. Os he contado muchas veces la historia del buen pastor que dejó a las noventa y nueve ovejas en el redil para salir a buscar a la que se había perdido, y cuando encontró a la oveja descarriada, cómo se la echó al hombro y la devolvió tiernamente al redil. Y cuando la oveja perdida estuvo de nuevo en el redil, recordaréis que el buen pastor llamó a sus amigos y los invitó a que se regocijaran con él porque había encontrado a la oveja que se había extraviado. Os digo de nuevo que hay más alegría en el cielo por un pecador que se arrepiente que por noventa y nueve justos que no necesitan arrepentimiento. El hecho de que unas almas estén \textit{perdidas} no hace más que acrecentar el interés del Padre celestial. He venido a este mundo para ejecutar el mandato de mi Padre, y se ha dicho con razón del Hijo del Hombre que es amigo de los publicanos y de los pecadores>>.

\par 
%\textsuperscript{(1851.1)}
\textsuperscript{169:1.3} <<Os han enseñado que la aceptación divina se produce después de que os hayáis arrepentido y como consecuencia de todas vuestras obras de sacrificio y de penitencia, pero os aseguro que el Padre os acepta incluso antes de que os hayáis arrepentido, y envía al Hijo y a sus asociados para encontraros y devolveros con regocijo al redil, al reino de la filiación y del progreso espiritual. Todos sois como ovejas extraviadas, y yo he venido para buscar y salvar a los que están perdidos>>.

\par 
%\textsuperscript{(1851.2)}
\textsuperscript{169:1.4} <<También deberíais recordar la historia de la mujer que, después de haber hecho un collar de adorno con diez monedas de plata, perdió una de las monedas; entonces encendió la lámpara, barrió cuidadosamente la casa y continuó buscando hasta que encontró la moneda de plata perdida. En cuanto encontró la moneda que había perdido, convocó a sus amigos y vecinos, diciendo: `Regocijaos conmigo porque he encontrado la moneda que se había perdido.' Así pues, os digo de nuevo que siempre hay alegría entre los ángeles del cielo por un pecador que se arrepiente y vuelve al redil del Padre. Os cuento esta historia para convenceros de que el Padre y su Hijo salen a \textit{buscar} a aquellos que están perdidos, y en esta búsqueda empleamos todas las influencias que puedan ayudarnos en nuestros esfuerzos diligentes por encontrar a los que se han perdido, a los que necesitan ser salvados. Y así, el Hijo del Hombre sale al desierto para buscar a la oveja extraviada, pero también busca la moneda que se ha perdido en la casa. La oveja se extravía de manera involuntaria; la moneda está cubierta por el polvo del tiempo y oscurecida por la acumulación de las cosas humanas>>.

\par 
%\textsuperscript{(1851.3)}
\textsuperscript{169:1.5} <<Ahora me gustaría contaros la historia del hijo atolondrado de un granjero acaudalado, que dejó \textit{deliberadamente} la casa de su padre y se fue a un país extranjero, donde sufrió muchas tribulaciones. Recordáis que la oveja se descarrió sin intención, pero este joven abandonó su hogar con premeditación. Esto fue lo que ocurrió:>>

\par 
%\textsuperscript{(1851.4)}
\textsuperscript{169:1.6} <<Había un hombre que tenía dos hijos; el más joven era alegre y despreocupado, y trataba siempre de pasarselo bien y de eludir las responsabilidades, mientras que su hermano mayor era serio, sobrio, trabajador y dispuesto a asumir las responsabilidades. Pero estos dos hermanos no se llevaban bien; discutían y reñían constantemente. El más joven era alegre y vivaz pero holgazán, y no se podía confiar en él; el hijo mayor era formal y trabajador, pero al mismo tiempo egocéntrico, hosco y engreído. El hijo más joven disfrutaba con el juego pero rehuía el trabajo; el mayor se consagraba al trabajo pero jugaba pocas veces. Esta asociación se volvió tan desagradable, que el hijo menor fue a ver a su padre y le dijo: `Padre, entrégame la tercera parte de los bienes que yo heredaría, y permíteme salir al mundo para buscar mi propia fortuna.' El padre sabía lo infeliz que era el joven en el hogar con su hermano mayor, y cuando escuchó esta petición, dividió sus bienes y le entregó al joven su parte>>.

\par 
%\textsuperscript{(1851.5)}
\textsuperscript{169:1.7} <<El joven reunió todos sus fondos en pocas semanas y salió de viaje hacia un país lejano; como no encontró nada provechoso que hacer que fuera también agradable, pronto derrochó toda su herencia viviendo de manera desenfrenada. Cuando lo hubo gastado todo, una hambruna prolongada surgió en aquel país, y el joven se encontró en la miseria. Y así, cuando empezó a pasar hambre y a sufrir una gran angustia, encontró un empleo con uno de los ciudadanos de aquel país, que lo envió a los campos a dar de comer a los cerdos. El joven se hubiera saciado de buena gana con los desperdicios que comían los cerdos, pero nadie quería darle nada>>.

\par 
%\textsuperscript{(1852.1)}
\textsuperscript{169:1.8} <<Un día que tenía mucha hambre, se le ocurrió decir: `¡Cuántos criados a sueldo de mi padre tienen pan de sobra mientras yo me muero de hambre, alimentando cerdos aquí en un país extranjero! Me levantaré, iré a ver a mi padre y le diré: Padre, he pecado contra el cielo y contra ti. Ya no soy digno de ser llamado hijo tuyo; permíteme ser solamente como uno de tus criados a sueldo.' Y cuando el joven llegó a esta decisión, se levantó y partió hacia la casa de su padre>>.

\par 
%\textsuperscript{(1852.2)}
\textsuperscript{169:1.9} <<Pero aquel padre había llorado mucho por su hijo; había echado de menos al alegre pero irreflexivo muchacho. Este padre amaba a este hijo y vigilaba constantemente su regreso, de manera que el día que el hijo se acercó a la casa, aunque aún estaba muy lejos, el padre lo vio; impulsado por una compasión amorosa, corrió a su encuentro y, saludándolo afectuosamente, lo abrazó y lo besó. Después de haberse reunido así, el hijo levantó los ojos hacia el rostro lleno de lágrimas de su padre y dijo: `Padre, he pecado contra el cielo y ante tus ojos; ya no soy digno de ser llamado tu hijo' ---pero el joven no tuvo la posibilidad de terminar su confesión, porque el padre lleno de alegría dijo a los criados que para entonces habían llegado corriendo: `Traed enseguida su mejor vestido, aquel que guardé, y ponedselo, y poned en su mano el anillo de hijo y buscad unas sandalias para sus pies.'>>

\par 
%\textsuperscript{(1852.3)}
\textsuperscript{169:1.10} <<Luego, después de que el feliz padre hubiera llevado hasta la casa al muchacho cansado y con los pies doloridos, dijo a sus criados: `Traed el ternero engordado y matadlo; comamos y divirtámonos, porque este hijo mío estaba muerto y vive de nuevo; estaba perdido y lo he encontrado.' Y todos se reunieron alrededor del padre para regocijarse con él por la restitución de su hijo>>.

\par 
%\textsuperscript{(1852.4)}
\textsuperscript{169:1.11} <<En ese momento, mientras lo estaban celebrando, el hijo mayor regresó de su trabajo cotidiano en el campo y, al acercarse a la casa, escuchó la música y el baile. Cuando llegó a la puerta de atrás, llamó a uno de los criados y le preguntó por el significado de toda esta celebración. El criado dijo entonces: `Tu hermano perdido desde hace mucho tiempo ha regresado al hogar, y tu padre ha matado al ternero engordado para regocijarse porque su hijo ha regresado sano y salvo. Entra para que puedas saludar también a tu hermano y acogerlo a su vuelta a la casa de tu padre.'>>

\par 
%\textsuperscript{(1852.5)}
\textsuperscript{169:1.12} <<Pero cuando el hermano mayor escuchó esto, se sintió tan herido y enojado que no quiso entrar en la casa. Cuando su padre se enteró de su resentimiento por la bienvenida que le había dado a su hermano menor, salió para rogarle que entrara. Pero el hijo mayor no quiso ceder a la persuasión de su padre, y le contestó diciendo: `Te he servido aquí durante todos estos años sin transgredir nunca el más pequeño de tus mandamientos, y sin embargo, nunca me has dado ni siquiera un cabrito para poder divertirme con mis amigos. He permanecido aquí para cuidarte todos estos años y nunca has dado una fiesta por mi servicio fiel, pero cuando regresa este hijo tuyo, después de haber malgastado tu fortuna con las prostitutas, te apresuras a matar el ternero engordado y a festejar su regreso.'>>

\par 
%\textsuperscript{(1852.6)}
\textsuperscript{169:1.13} <<Como este padre amaba realmente a sus dos hijos, intentó razonar con el mayor: `Pero hijo mío, has estado conmigo todo este tiempo, y todo lo que poseo es tuyo. Hubieras podido coger un cabrito en cualquier momento que hubieras hecho amigos con quienes compartir tu alegría. Pero ahora es sencillamente apropiado que te unas a mí en la alegría y el regocijo por el regreso de tu hermano. Piensa en ello, hijo mío, tu hermano se había perdido y ha sido encontrado; ¡ha regresado vivo a nosotros!'>>

\par 
%\textsuperscript{(1853.1)}
\textsuperscript{169:1.14} Ésta fue una de las parábolas más conmovedoras y eficaces de todas las que Jesús presentó para convencer a sus oyentes de la buena voluntad del Padre en recibir a todos los que intentan entrar en el reino de los cielos.

\par 
%\textsuperscript{(1853.2)}
\textsuperscript{169:1.15} Jesús era muy aficionado a contar estas tres historias al mismo tiempo. Presentaba la historia de la oveja perdida para mostrar que, cuando los hombres se desvían involuntariamente del camino de la vida, el Padre tiene presentes a estos hijos \textit{perdidos}, y sale con sus Hijos, los verdaderos pastores del rebaño, a buscar a las ovejas perdidas. Luego narraba la historia de la moneda perdida en la casa para ilustrar cuán completa es la \textit{búsqueda} divina de todos los que están confusos, desconcertados, o cegados espiritualmente de otros modos por las preocupaciones materiales y las acumulaciones de la vida. Luego, Jesús se lanzaba a contar esta parábola del hijo perdido, la acogida del pródigo que regresa, para mostrar cuán completo es el \textit{restablecimiento} del hijo perdido en la casa y en el corazón de su padre.

\par 
%\textsuperscript{(1853.3)}
\textsuperscript{169:1.16} Durante sus años de enseñanza, Jesús contó y volvió a contar muchísimas veces esta historia del hijo pródigo. Esta parábola y la historia del buen samaritano eran sus medios preferidos para enseñar el amor del Padre y las buenas relaciones entre los hombres.

\section*{2. La parábola del administrador sagaz}
\par 
%\textsuperscript{(1853.4)}
\textsuperscript{169:2.1} Una tarde, al comentar una de las declaraciones de Jesús, Simón Celotes dijo: <<Maestro, ¿qué has querido decir hoy cuando has afirmado que muchos hijos del mundo son más sensatos en su generación que los hijos del reino, puesto que tienen la habilidad de hacer amigos con las riquezas conseguidas injustamente?>> Jesús respondió:

\par 
%\textsuperscript{(1853.5)}
\textsuperscript{169:2.2} <<Antes de entrar en el reino, algunos de vosotros erais muy astutos en el trato con vuestros asociados en los negocios. Si erais injustos y a menudo desleales, sin embargo erais prudentes y previsores, en el sentido de que realizabais vuestros negocios con el ojo puesto únicamente en vuestro beneficio presente y en vuestra seguridad futura. Del mismo modo, ahora deberíais ordenar vuestra vida en el reino de tal manera que os proporcione la alegría en el presente y os asegure también el disfrute futuro de los tesoros acumulados en el cielo. Si erais tan diligentes en la obtención de ganancias para vosotros mismos cuando estabais al servicio del ego, ¿por qué tendríais que mostrar menos diligencia en ganar almas para el reino, puesto que ahora sois los servidores de la fraternidad de los hombres y los administradores de Dios?>>

\par 
%\textsuperscript{(1853.6)}
\textsuperscript{169:2.3} <<Todos podéis aprender una lección de la historia de cierto hombre rico que tenía un administrador astuto, pero injusto. Este administrador no sólo había presionado a los clientes de su señor en su propio beneficio egoísta, sino que también había malgastado y disipado directamente los fondos de su señor. Cuando todo esto llegó finalmente a oídos del dueño, éste llamó al administrador a su presencia y le preguntó por el significado de aquellos rumores; le exigió que le rindiera cuentas inmediatamente de su administración y que se preparara para entregar los asuntos de su señor a otra persona>>.

\par 
%\textsuperscript{(1853.7)}
\textsuperscript{169:2.4} <<Pero este administrador infiel empezó a decirse para sí: `¿Qué va a ser de mí, puesto que estoy a punto de perder esta administración? No tengo fuerzas para cavar la tierra, y me da verg\"uenza mendigar. Ya sé lo que voy a hacer para asegurarme de que seré bien recibido, cuando me quiten esta administración, en las casas de todos los que hacen negocios con mi señor.' Luego llamó a todos los deudores de su señor, y le dijo al primero: `¿Cuánto le debes a mi señor?' Éste respondió: `Cien medidas de aceite.' Entonces dijo el administrador: `Coge la tablilla de cera de tu deuda, siéntate deprisa, y cámbiala a cincuenta.' Luego dijo a otro deudor: `¿Cuánto debes tú?' Y éste replicó: `Cien medidas de trigo.' Entonces dijo el administrador: `Coge tu cuenta y escribe ochenta.' E hizo esto mismo con otros numerosos deudores. Este administrador poco honrado trataba así de hacer amigos para cuando le quitaran la administración. Incluso su dueño y señor, cuando se enteró posteriormente de esto, se vio obligado a admitir que su infiel administrador al menos había mostrado sagacidad en la manera en que había intentado asegurarse el porvenir para los días futuros de miseria y de adversidad>>.

\par 
%\textsuperscript{(1854.1)}
\textsuperscript{169:2.5} <<Así es como los hijos de este mundo muestran a veces más sabiduría que los hijos de la luz en la preparación de su futuro. A vosotros que pretendéis adquirir un tesoro en el cielo, os digo: Aprended de los que hacen amigos con las riquezas conseguidas injustamente, y conducid vuestra vida de tal manera que entabléis una amistad eterna con las fuerzas de la rectitud para que, cuando fallen todas las cosas terrenales, seáis recibidos con alegría en las moradas eternas>>.

\par 
%\textsuperscript{(1854.2)}
\textsuperscript{169:2.6} <<Afirmo que aquel que es fiel en las cosas pequeñas también será fiel en las grandes, mientras que el que es injusto en las cosas pequeñas también lo será en las grandes. Si no habéis mostrado previsión e integridad en los asuntos de este mundo, ¿cómo podéis esperar ser fieles y prudentes cuando se os confíe la administración de las verdaderas riquezas del reino celestial? Si no sois unos buenos administradores y unos banqueros fieles, si no habéis sido fieles en lo que pertenece a otro, ¿quién será lo bastante insensato como para daros un gran tesoro en propiedad?>>

\par 
%\textsuperscript{(1854.3)}
\textsuperscript{169:2.7} <<Afirmo de nuevo que nadie puede servir a dos señores; o bien odiará a uno y amará al otro, o bien se quedará con uno mientras que despreciará al otro. No podéis servir a Dios y a las riquezas>>.

\par 
%\textsuperscript{(1854.4)}
\textsuperscript{169:2.8} Cuando los fariseos que estaban presentes escucharon esto, empezaron a burlarse y a reírse, puesto que eran muy dados a conseguir riquezas. Estos oyentes hostiles trataron de implicar a Jesús en un debate inútil, pero éste se negó a discutir con sus enemigos. Cuando los fariseos se pusieron a reñir entre ellos, sus fuertes voces atrajeron a una gran parte de la multitud que estaba acampada en los alrededores; y cuando empezaron a discutir entre sí, Jesús se retiró a su tienda para pasar la noche.

\section*{3. El hombre rico y el mendigo}
\par 
%\textsuperscript{(1854.5)}
\textsuperscript{169:3.1} Cuando la reunión se volvió demasiado ruidosa, Simón Pedro se levantó y se hizo cargo de la situación, diciendo: <<Hombres y hermanos, no es apropiado que discutáis así entre vosotros. El Maestro ha hablado, y haríais bien en sopesar sus palabras. No os ha proclamado ninguna nueva doctrina. ¿No habéis oído también la alegoría de los nazareos sobre el rico y el mendigo? Algunos de nosotros hemos escuchado a Juan el Bautista decir a voz en grito esta parábola de advertencia a todos los que aman las riquezas y codician los bienes fraudulentos. Aunque esta antigua parábola no es conforme al evangelio que predicamos, todos haríais bien en prestar atención a sus lecciones, hasta el momento en que podáis comprender la nueva luz del reino de los cielos. La historia, tal como Juan la contaba, era así:>>

\par 
%\textsuperscript{(1854.6)}
\textsuperscript{169:3.2} <<Había un hombre rico llamado Dives que, vestido de púrpura y de lino fino, vivía todos los días en el regocijo y el esplendor. Y había un mendigo llamado Lázaro, que estaba tendido en la puerta de aquel rico, cubierto de llagas y deseando alimentarse con las migajas que caían de la mesa del rico. Sí, incluso los perros venían y le lamían las llagas. Y sucedió que el mendigo murió y fue llevado por los ángeles a descansar en el seno de Abraham. El rico murió también enseguida y fue enterrado con una gran pompa y un esplendor real. Cuando el rico partió de este mundo, se despertó en el Hades, y al encontrarse atormentado, levantó los ojos y vio a Abraham a lo lejos y a Lázaro en su seno. Entonces Dives gritó: `Padre Abraham, ten misericordia de mí y envíame a Lázaro para que moje la punta de su dedo en el agua y me refresque la lengua, porque sufro un gran suplicio a causa de mi castigo.' Entonces Abraham replicó: `Hijo mío, recuerda que disfrutaste de las cosas buenas durante tu vida, mientras que Lázaro soportaba las malas. Pero ahora todo ha cambiado, pues Lázaro recibe consuelo mientras que tú estás atormentado. Además, existe un gran abismo entre tú y nosotros, de manera que no podemos ir hasta ti, ni tú puedes venir hasta nosotros.' Entonces Dives le dijo a Abraham: `Te ruego que hagas volver a Lázaro a la casa de mi padre, ya que tengo cinco hermanos, para que pueda dar tal testimonio que impida que mis hermanos vengan a este lugar de tormento.' Pero Abraham dijo: {}`Hijo mío, tienen a Moisés y a los profetas; que los escuchen.' Entonces Dives respondió: `¡No, no, padre Abraham! Pero si alguien que ha muerto se presenta ante ellos, se arrepentirán.' Y entonces dijo Abraham: `Si no escuchan a Moisés y a los profetas, tampoco se convencerán aunque alguien resucite de entre los muertos.'>>

\par 
%\textsuperscript{(1855.1)}
\textsuperscript{169:3.3} Después de que Pedro hubiera contado esta antigua parábola de la fraternidad nazarea, y como la multitud se había calmado, Andrés se levantó y disolvió la reunión para pasar la noche. Aunque tanto los apóstoles como los discípulos preguntaron con frecuencia a Jesús sobre la parábola de Dives y Lázaro, nunca consintió en comentarla.

\section*{4. El Padre y su reino}
\par 
%\textsuperscript{(1855.2)}
\textsuperscript{169:4.1} Jesús siempre tuvo dificultades cuando intentó explicar a los apóstoles que, aunque proclamaban el establecimiento del reino de Dios, el Padre que está en los cielos \textit{no era un rey}. En la época en que Jesús vivió en la Tierra y enseñó en la carne, los pueblos de Urantia conocían principalmente a reyes y emperadores en el gobierno de las naciones, y los judíos habían esperado durante mucho tiempo la llegada del reino de Dios. Por estas y otras razones, el Maestro pensó que era mejor llamar reino de los cielos a la fraternidad espiritual de los hombres, y \textit{Padre que está en los cielos} al jefe espiritual de esta fraternidad. Jesús nunca se refirió a su Padre como si fuera un rey. En sus conversaciones íntimas con los apóstoles, siempre se refería a sí mismo como el Hijo del Hombre, como el hermano mayor de ellos. Describía a todos sus seguidores como los <<servidores de la humanidad>> y como los <<mensajeros del evangelio del reino>>.

\par 
%\textsuperscript{(1855.3)}
\textsuperscript{169:4.2} Jesús nunca dio a sus apóstoles una lección sistemática sobre la personalidad y los atributos del Padre que está en los cielos. Nunca pidió a los hombres que creyeran en su Padre, pues daba por hecho que lo hacían. Jesús nunca se rebajó a ofrecer argumentos que probaran la realidad del Padre. Toda su enseñanza acerca del Padre estaba centrada en la declaración de que él y el Padre son uno solo; que aquel que ha visto al Hijo ha visto al Padre; que el Padre, al igual que el Hijo, conoce todas las cosas; que sólo el Hijo conoce realmente al Padre y aquel a quien el Hijo se lo revela; que aquel que conoce al Hijo conoce también al Padre; y que el Padre lo había enviado al mundo para revelar sus naturalezas combinadas y para dar a conocer su trabajo conjunto. Nunca hizo otras declaraciones sobre su Padre, excepto a la mujer de Samaria en el pozo de Jacob, cuando afirmó: <<Dios es espíritu>>.

\par 
%\textsuperscript{(1856.1)}
\textsuperscript{169:4.3} Aprendéis cosas sobre Dios a través de Jesús observando la divinidad de su vida, no dependiendo de sus enseñanzas. Cada uno puede asimilar, de la vida del Maestro, ese concepto de Dios que representa la medida de vuestra capacidad para percibir las realidades espirituales y divinas, las verdades reales y eternas. El finito nunca puede esperar comprender al Infinito, salvo cuando el Infinito estuvo focalizado en la personalidad espacio-temporal de la experiencia finita de la vida humana de Jesús de Nazaret.

\par 
%\textsuperscript{(1856.2)}
\textsuperscript{169:4.4} Jesús sabía muy bien que a Dios sólo se le puede conocer mediante las realidades de la experiencia; nunca se le puede comprender mediante la simple enseñanza de la mente. Jesús enseñó a sus apóstoles que, aunque nunca podrían comprender plenamente a Dios, podrían \textit{conocerlo} con toda certeza, tal como habían conocido al Hijo del Hombre. Podéis conocer a Dios, no comprendiendo lo que Jesús dijo, sino sabiendo lo que Jesús era. Jesús \textit{era} una revelación de Dios.

\par 
%\textsuperscript{(1856.3)}
\textsuperscript{169:4.5} Excepto cuando citaba las escrituras hebreas, Jesús sólo empleaba dos nombres para referirse a la Deidad: Dios y Padre. Cuando el Maestro se refería a su Padre como Dios, empleaba generalmente la palabra hebrea que significaba el Dios plural (la Trinidad), y no la palabra Yahvé, que representaba el concepto progresivo del Dios tribal de los judíos.

\par 
%\textsuperscript{(1856.4)}
\textsuperscript{169:4.6} Jesús nunca llamó rey al Padre, y lamentaba mucho que la esperanza de los judíos de poseer un reino restaurado y la proclamación de Juan sobre un reino venidero le hubieran obligado a denominar <<reino de los cielos>> a la fraternidad espiritual que se proponía establecer. Con una sola excepción ---la declaración de que <<Dios es espíritu>>--- Jesús nunca se refirió a la Deidad de manera distinta a los términos que describían su propia relación personal con la Fuente-Centro Primera del Paraíso.

\par 
%\textsuperscript{(1856.5)}
\textsuperscript{169:4.7} Jesús empleó la palabra Dios para designar la \textit{idea} de la Deidad, y la palabra Padre para designar la \textit{experiencia} de conocer a Dios. Cuando la palabra Padre se emplea para designar a Dios, se debería entender en su significado más amplio posible. La palabra Dios no se puede definir y representa por tanto el concepto infinito del Padre, pero como la palabra Padre se puede definir parcialmente, puede ser empleada para representar el concepto humano del Padre divino, tal como éste está asociado con el hombre en el transcurso de la existencia mortal.

\par 
%\textsuperscript{(1856.6)}
\textsuperscript{169:4.8} Elohim era para los judíos el Dios de los dioses, mientras que Yahvé era el Dios de Israel. Jesús aceptaba el concepto de Elohim y llamaba Dios a este grupo supremo de seres. En el lugar del concepto de Yahvé, la deidad racial, Jesús introdujo la idea de la paternidad de Dios y de la fraternidad mundial de los hombres. Elevó el concepto de Yahvé, el de un Padre racial deificado, hasta la idea de un Padre de todos los hijos de los hombres, un Padre divino del creyente individual. Y además enseñó que este Dios de los universos y este Padre de todos los hombres eran la misma y única Deidad Paradisiaca.

\par 
%\textsuperscript{(1856.7)}
\textsuperscript{169:4.9} Jesús nunca pretendió ser la manifestación de Elohim (Dios) en la carne. Nunca declaró que fuera una revelación de Elohim (Dios) para los mundos. Nunca enseñó que cualquiera que lo hubiera visto había visto a Elohim (Dios). Pero sí se proclamó como la revelación del Padre en la carne, y dijo también que cualquiera que lo hubiera visto había visto al Padre. Como Hijo divino afirmó que sólo representaba al Padre.

\par 
%\textsuperscript{(1857.1)}
\textsuperscript{169:4.10} En verdad, él era incluso el Hijo del Dios Elohim; pero en la similitud de la carne mortal y para los hijos mortales de Dios, escogió limitar la revelación de su vida a la descripción del carácter de su Padre hasta donde esta revelación pudiera ser comprensible por el hombre mortal. En cuanto al carácter de las otras personas de la Trinidad del Paraíso, deberemos contentarnos con la enseñanza de que son totalmente como el Padre, cuya descripción personal ha sido revelada en la vida de su Hijo encarnado, Jesús de Nazaret.

\par 
%\textsuperscript{(1857.2)}
\textsuperscript{169:4.11} Aunque Jesús reveló en su vida terrenal la verdadera naturaleza del Padre celestial, pocas cosas enseñó sobre él. De hecho, sólo enseñó dos cosas: que Dios es en sí mismo espíritu y que, en todas las cuestiones de las relaciones con sus criaturas, es un Padre. Aquella noche, Jesús efectuó la declaración final de su relación con Dios cuando afirmó: <<He salido del Padre y he venido al mundo; de nuevo, dejaré el mundo e iré al Padre>>.

\par 
%\textsuperscript{(1857.3)}
\textsuperscript{169:4.12} ¡Pero prestad atención! Jesús nunca dijo: <<Cualquiera que me ha escuchado, ha escuchado a Dios>>. Pero sí dijo: <<Aquel que me ha \textit{visto}, ha visto al Padre>>. Escuchar la enseñanza de Jesús no equivale a conocer a Dios, pero \textit{ver} a Jesús es una experiencia que es en sí misma una revelación del Padre al alma. El Dios de los universos gobierna la extensa creación, pero es el Padre que está en los cielos el que envía a su espíritu para que resida dentro de vuestra mente.

\par 
%\textsuperscript{(1857.4)}
\textsuperscript{169:4.13} Jesús es, en su semejanza humana, la lente espiritual que hace visible para la criatura material a Aquel que es invisible. Es vuestro hermano mayor que, en la carne, os hace \textit{conocer} a un Ser con atributos infinitos que ni siquiera las huestes celestiales pueden vanagloriarse de comprender plenamente. Pero todo esto debe consistir en la experiencia personal del \textit{creyente individual}. Dios, que es espíritu, sólo se puede conocer como experiencia espiritual. A los hijos finitos de los mundos materiales, el Hijo divino de los reinos espirituales sólo les puede revelar a Dios como \textit{Padre}. Podéis conocer al Eterno como Padre, pero podéis adorarlo como el Dios de los universos, el Creador infinito de todo lo que existe.


\chapter{Documento 170. El reino de los cielos}
\par 
%\textsuperscript{(1858.1)}
\textsuperscript{170:0.1} EL SÁBADO 11 de marzo por la tarde, Jesús predicó su último sermón en Pella. Fue una de las alocuciones más memorables de su ministerio público, que abarcó un examen pleno y completo del reino de los cielos. Era consciente de la confusión que existía en la mente de sus apóstoles y discípulos sobre el sentido y el significado de las expresiones <<reino de los cielos>> y <<reino de Dios>>, que él utilizaba indistintamente para designar su misión donadora. El término mismo de reino de los \textit{cielos} debería haber sido suficiente para separar lo que significaba de toda conexión con los reinos \textit{terrenales} y los gobiernos temporales, pero no era así. La idea de un rey temporal estaba arraigada demasiado profundamente en la mente de los judíos como para poder desalojarla en una sola generación. Por eso Jesús no se opuso abiertamente, al principio, a este concepto del reino que mantenían desde hacía mucho tiempo.

\par 
%\textsuperscript{(1858.2)}
\textsuperscript{170:0.2} Aquel sábado por la tarde, el Maestro intentó clarificar la enseñanza sobre el reino de los cielos; trató el tema desde todos los puntos de vista, y se esforzó por aclarar los numerosos sentidos diferentes en los que el término se había empleado. En esta narración, ampliaremos su discurso añadiendo numerosas declaraciones realizadas por Jesús en ocasiones anteriores, e incluiremos algunas observaciones hechas exclusivamente a los apóstoles durante las discusiones vespertinas de aquel mismo día. También efectuaremos algunos comentarios sobre la evolución ulterior de la idea del reino, tal como está relacionada con la iglesia cristiana posterior.

\section*{1. Los conceptos del reino de los cielos}
\par 
%\textsuperscript{(1858.3)}
\textsuperscript{170:1.1} En relación con la descripción del sermón de Jesús, es preciso señalar que en todas las escrituras hebreas figuraba un doble concepto del reino de los cielos. Los profetas habían presentado el reino de Dios como:

\par 
%\textsuperscript{(1858.4)}
\textsuperscript{170:1.2} 1. Una realidad presente; y como

\par 
%\textsuperscript{(1858.5)}
\textsuperscript{170:1.3} 2. Una esperanza futura ---cuando el reino llegara a realizarse en su plenitud en el momento de la aparición del Mesías. Este concepto del reino fue el que enseñó Juan el Bautista.

\par 
%\textsuperscript{(1858.6)}
\textsuperscript{170:1.4} Desde el principio, Jesús y los apóstoles enseñaron estos dos conceptos. Y habría que tener presentes en la memoria otras dos ideas del reino:

\par 
%\textsuperscript{(1858.7)}
\textsuperscript{170:1.5} 3. El concepto judío posterior de un reino mundial y trascendental, de origen sobrenatural e inauguración milagrosa.

\par 
%\textsuperscript{(1858.8)}
\textsuperscript{170:1.6} 4. Las enseñanzas persas que describían el establecimiento de un reino divino al fin del mundo, como consecución del triunfo del bien sobre el mal.

\par 
%\textsuperscript{(1858.9)}
\textsuperscript{170:1.7} Poco antes de la venida de Jesús a la Tierra, los judíos combinaban y confundían todas estas ideas del reino en su concepto apocalíptico de la llegada del Mesías para establecer la era del triunfo judío, la era eterna del gobierno supremo de Dios en la Tierra, el nuevo mundo, la era en que toda la humanidad adoraría a Yahvé. Al escoger utilizar este concepto del reino de los cielos, Jesús decidió apropiarse de la herencia más fundamental y culminante de las dos religiones, la judía y la persa.

\par 
%\textsuperscript{(1859.1)}
\textsuperscript{170:1.8} El reino de los cielos, tal como ha sido comprendido y malentendido durante todos los siglos de la era cristiana, abarcaba cuatro grupos distintos de ideas:

\par 
%\textsuperscript{(1859.2)}
\textsuperscript{170:1.9} 1. El concepto de los judíos.

\par 
%\textsuperscript{(1859.3)}
\textsuperscript{170:1.10} 2. El concepto de los persas.

\par 
%\textsuperscript{(1859.4)}
\textsuperscript{170:1.11} 3. El concepto de la experiencia personal de Jesús ---<<el reino de los cielos dentro de vosotros>>.

\par 
%\textsuperscript{(1859.5)}
\textsuperscript{170:1.12} 4. Los conceptos amalgamados y confusos que los fundadores y divulgadores del cristianismo han intentado inculcar al mundo.

\par 
%\textsuperscript{(1859.6)}
\textsuperscript{170:1.13} En momentos diferentes y en circunstancias diversas, parece ser que Jesús había presentado numerosos conceptos del <<reino>> en sus enseñanzas públicas, pero a sus apóstoles siempre les enseñó que el reino abarcaba la experiencia personal del hombre en relación con sus semejantes en la Tierra y con el Padre en el cielo. Sus últimas palabras con respecto al reino siempre eran: <<El reino está dentro de vosotros>>.

\par 
%\textsuperscript{(1859.7)}
\textsuperscript{170:1.14} Tres factores han causado siglos de confusión en lo que se refiere al significado de la expresión <<el reino de los cielos>>:

\par 
%\textsuperscript{(1859.8)}
\textsuperscript{170:1.15} 1. La confusión que ocasionó el observar que la idea del <<reino>> pasaba por diversas fases progresivas de modificación por parte de Jesús y sus apóstoles.

\par 
%\textsuperscript{(1859.9)}
\textsuperscript{170:1.16} 2. La confusión que acompañó de manera inevitable al trasplante del cristianismo primitivo desde un terreno judío a un terreno gentil.

\par 
%\textsuperscript{(1859.10)}
\textsuperscript{170:1.17} 3. La confusión inherente al hecho de que el cristianismo se convirtió en una religión organizada alrededor de la idea central de la persona de Jesús; el evangelio del reino se convirtió cada vez más en una religión \textit{acerca de} Jesús.

\section*{2. El concepto de Jesús sobre el reino}
\par 
%\textsuperscript{(1859.11)}
\textsuperscript{170:2.1} El Maestro indicó claramente que el reino de los cielos debe empezar por el doble concepto de la verdad de la paternidad de Dios y el hecho correlativo de la fraternidad de los hombres, y debe permanecer centrado en este doble concepto. Jesús declaró que la aceptación de esta enseñanza liberaría a los hombres de la esclavitud milenaria al miedo animal, y al mismo tiempo enriquecería la vida humana con los dones siguientes de la nueva vida de libertad espiritual:

\par 
%\textsuperscript{(1859.12)}
\textsuperscript{170:2.2} 1. La posesión de una nueva valentía y de un poder espiritual acrecentado. El evangelio del reino iba a liberar al hombre y a inspirarlo para que se atreviera a esperar la vida eterna.

\par 
%\textsuperscript{(1859.13)}
\textsuperscript{170:2.3} 2. El evangelio contenía un mensaje de nueva confianza y de verdadero consuelo para todos los hombres, incluso para los pobres.

\par 
%\textsuperscript{(1859.14)}
\textsuperscript{170:2.4} 3. Era en sí mismo una nueva norma de valores morales, una nueva vara ética para medir la conducta humana. Mostraba el ideal del nuevo orden de la sociedad humana que resultaría de él.

\par 
%\textsuperscript{(1859.15)}
\textsuperscript{170:2.5} 4. Enseñaba la preeminencia de lo espiritual comparado con lo material; glorificaba las realidades espirituales y exaltaba los ideales sobrehumanos.

\par 
%\textsuperscript{(1860.1)}
\textsuperscript{170:2.6} 5. Este nuevo evangelio presentaba el logro espiritual como la verdadera meta de la vida. La vida humana recibía una nueva dotación de valor moral y de dignidad divina.

\par 
%\textsuperscript{(1860.2)}
\textsuperscript{170:2.7} 6. Jesús enseñó que las realidades eternas eran el resultado (la recompensa) de los esfuerzos honrados en la Tierra. La estancia mortal del hombre en la Tierra adquirió nuevos significados como consecuencia del reconocimiento de un noble destino.

\par 
%\textsuperscript{(1860.3)}
\textsuperscript{170:2.8} 7. El nuevo evangelio afirmaba que la salvación humana es la revelación de un propósito divino de gran alcance, que debe cumplirse y realizarse en el destino futuro del servicio sin fin de los hijos salvados de Dios.

\par 
%\textsuperscript{(1860.4)}
\textsuperscript{170:2.9} Estas enseñanzas abarcan la idea ampliada del reino que Jesús enseñó. Este gran concepto apenas estaba contenido en las enseñanzas elementales y confusas de Juan el Bautista sobre el reino.

\par 
%\textsuperscript{(1860.5)}
\textsuperscript{170:2.10} Los apóstoles eran incapaces de captar el significado real de las declaraciones del Maestro acerca del reino. La deformación posterior de las enseñanzas de Jesús, tal como están registradas en el Nuevo Testamento, se debe a que el concepto de los escritores evangélicos estaba influido por la creencia de que Jesús sólo se había ausentado del mundo por un corto período de tiempo; que pronto regresaría para establecer el reino con poder y gloria ---exactamente la idea que habían mantenido mientras estaba con ellos en la carne. Pero Jesús no había asociado el establecimiento del reino con la idea de su regreso a este mundo. Que los siglos hayan pasado sin ningún signo de la aparición de la <<Nueva Era>>, no está de ninguna manera en desacuerdo con la enseñanza de Jesús.

\par 
%\textsuperscript{(1860.6)}
\textsuperscript{170:2.11} El gran esfuerzo incluido en este sermón fue la tentativa por trasladar el concepto del reino de los cielos al ideal de la idea de hacer la voluntad de Dios. Hacía tiempo que el Maestro había enseñado a sus seguidores a orar: <<Que venga tu reino; que se haga tu voluntad>>; en esta época intentó seriamente inducirlos a que abandonaran la utilización de la expresión \textit{reino de Dios} a favor de un equivalente más práctico: \textit{la voluntad de Dios}. Pero no lo consiguió.

\par 
%\textsuperscript{(1860.7)}
\textsuperscript{170:2.12} Jesús deseaba sustituir la idea de reino, de rey y de súbditos por el concepto de la familia celestial, del Padre celestial y de los hijos liberados de Dios, dedicados al servicio alegre y voluntario de sus semejantes, y a la adoración sublime e inteligente de Dios Padre.

\par 
%\textsuperscript{(1860.8)}
\textsuperscript{170:2.13} Hasta este momento, los apóstoles habían adquirido un doble punto de vista sobre el reino; lo consideraban como:

\par 
%\textsuperscript{(1860.9)}
\textsuperscript{170:2.14} 1. Un asunto de experiencia personal entonces presente en el corazón de los verdaderos creyentes, y

\par 
%\textsuperscript{(1860.10)}
\textsuperscript{170:2.15} 2. Una cuestión de fenómeno racial o mundial; el reino se encontraba en el futuro, algo a esperar con mucha ilusión.

\par 
%\textsuperscript{(1860.11)}
\textsuperscript{170:2.16} Consideraban la llegada del reino en el corazón de los hombres como un desarrollo gradual, semejante a la levadura en la masa o al crecimiento de la semilla de mostaza. Creían que la llegada del reino, en el sentido racial o mundial, sería al mismo tiempo repentina y espectacular. Jesús nunca se cansó de decirles que el reino de los cielos era su experiencia personal consistente en obtener las cualidades superiores de la vida espiritual; que esas realidades de la experiencia espiritual son transferidas progresivamente a unos niveles nuevos y superiores de certidumbre divina y de grandeza eterna.

\par 
%\textsuperscript{(1860.12)}
\textsuperscript{170:2.17} Aquella tarde, el Maestro enseñó claramente un nuevo concepto de la doble naturaleza del reino, en el sentido de que describió las dos fases siguientes:

\par 
%\textsuperscript{(1860.13)}
\textsuperscript{170:2.18} <<Primera, el reino de Dios en este mundo, el deseo supremo de hacer la voluntad de Dios, el amor desinteresado por los hombres, que produce los buenos frutos de una mejor conducta ética y moral>>.

\par 
%\textsuperscript{(1861.1)}
\textsuperscript{170:2.19} <<Segunda, el reino de Dios en el cielo, la meta de los creyentes mortales, el estado en el que el amor a Dios se ha perfeccionado y en el que se hace la voluntad de Dios de manera más divina>>.

\par 
%\textsuperscript{(1861.2)}
\textsuperscript{170:2.20} Jesús enseñó que, por medio de la fe, el creyente entra \textit{de inmediato} en el reino. Enseñó en sus diversos discursos que dos cosas son esenciales para entrar por la fe en el reino:

\par 
%\textsuperscript{(1861.3)}
\textsuperscript{170:2.21} 1. \textit{La fe, la sinceridad}. Venir como un niño pequeño, recibir el don de la filiación como un regalo; aceptar hacer la voluntad del Padre sin hacer preguntas, con una seguridad plena y una confianza sincera en la sabiduría del Padre; entrar en el reino libre de prejuicios y de ideas preconcebidas; tener una actitud abierta y estar dispuesto a aprender como un niño no mimado.

\par 
%\textsuperscript{(1861.4)}
\textsuperscript{170:2.22} 2. \textit{El hambre de la verdad}. La sed de rectitud, un cambio de mentalidad, la adquisición de la motivación de ser como Dios y de encontrar a Dios.

\par 
%\textsuperscript{(1861.5)}
\textsuperscript{170:2.23} Jesús enseñó que el pecado no es el producto de una naturaleza defectuosa, sino más bien el fruto de una mente instruida, dominada por una voluntad insumisa. Con respecto al pecado, enseñó que Dios \textit{ha} perdonado; que ese perdón lo ponemos a nuestra disposición personal mediante el acto de perdonar a nuestros semejantes. Cuando perdonáis a vuestro hermano en la carne, creáis así en vuestra propia alma la capacidad para recibir la realidad del perdón de Dios por vuestras propias fechorías.

\par 
%\textsuperscript{(1861.6)}
\textsuperscript{170:2.24} Cuando el apóstol Juan empezó a escribir la historia de la vida y las enseñanzas de Jesús, los primeros cristianos habían tenido tantos problemas con la idea del reino de Dios como generadora de persecuciones, que prácticamente habían abandonado la utilización de este término. Juan habla mucho sobre la <<vida eterna>>. Jesús habló a menudo de esta idea como el <<reino de la vida>>. También aludió con frecuencia al <<reino de Dios dentro de vosotros>>. Una vez calificó esta experiencia de <<comunión familiar con Dios Padre>>. Jesús intentó sustituir la palabra <<reino>> por otros muchos términos, pero siempre sin éxito. Utilizó entre otros: la familia de Dios, la voluntad del Padre, los amigos de Dios, la comunidad de los creyentes, la fraternidad de los hombres, el redil del Padre, los hijos de Dios, la comunidad de los fieles, el servicio del Padre, y los hijos liberados de Dios.

\par 
%\textsuperscript{(1861.7)}
\textsuperscript{170:2.25} Pero no pudo evitar la utilización de la idea de reino. Más de cincuenta años más tarde, después de la destrucción de Jerusalén por los ejércitos romanos, fue cuando este concepto del reino empezó a transformarse en el culto de la vida eterna, a medida que sus aspectos sociales e institucionales eran asumidos por la iglesia cristiana en rápida expansión y cristalización.

\section*{3. En relación con la rectitud}
\par 
%\textsuperscript{(1861.8)}
\textsuperscript{170:3.1} Jesús intentó siempre inculcar a sus apóstoles y discípulos que debían adquirir, por la fe, una rectitud que sobrepasara la rectitud de las obras serviles que algunos escribas y fariseos exhibían con tanta vanidad delante del mundo.

\par 
%\textsuperscript{(1861.9)}
\textsuperscript{170:3.2} Jesús enseñó que la fe, la simple creencia semejante a la de un niño, es la llave de la puerta del reino, pero también enseñó que después de haber pasado la puerta, están los peldaños progresivos de rectitud que todo niño creyente debe ascender para crecer hasta la plena estatura de los vigorosos hijos de Dios.

\par 
%\textsuperscript{(1861.10)}
\textsuperscript{170:3.3} En el estudio de la técnica para \textit{recibir} el perdón de Dios es donde se encuentra revelada la obtención de la rectitud del reino. La fe es el precio que pagáis por entrar en la familia de Dios; pero el perdón es el acto de Dios que acepta vuestra fe como precio de admisión. Y la recepción del perdón de Dios por parte de un creyente en el reino implica una experiencia precisa y real, que consiste en las cuatro etapas siguientes, las etapas del reino de la rectitud interior:

\par 
%\textsuperscript{(1862.1)}
\textsuperscript{170:3.4} 1. El hombre dispone realmente del perdón de Dios, y lo experimenta personalmente, en la medida exacta en que perdona a sus semejantes.

\par 
%\textsuperscript{(1862.2)}
\textsuperscript{170:3.5} 2. El hombre no perdona de verdad a sus semejantes a menos que los ame como a sí mismo.

\par 
%\textsuperscript{(1862.3)}
\textsuperscript{170:3.6} 3. Amar así al prójimo como a sí mismo \textit{es} la ética más elevada.

\par 
%\textsuperscript{(1862.4)}
\textsuperscript{170:3.7} 4. La conducta moral, la verdadera rectitud, se convierte entonces en el resultado natural de ese amor.

\par 
%\textsuperscript{(1862.5)}
\textsuperscript{170:3.8} Por eso es evidente que la verdadera religión interior del reino tiende a manifestarse infaliblemente, y cada vez más, en las vías prácticas del servicio social. Jesús enseñó una religión viva que impulsaba a sus creyentes a dedicarse a realizar un servicio amoroso. Pero Jesús no puso la ética en el lugar de la religión. Enseñó la religión como causa, y la ética como resultado.

\par 
%\textsuperscript{(1862.6)}
\textsuperscript{170:3.9} La rectitud de cualquier acto debe ser medida por el móvil; las formas más elevadas del bien son por tanto inconscientes. Jesús no se interesó nunca por la moral o la ética como tales. Se ocupó completamente de esa comunión interior y espiritual con Dios Padre que se manifiesta exteriormente de manera tan cierta y directa en el servicio amoroso a los hombres. Enseñó que la religión del reino es una experiencia personal auténtica que nadie puede reprimir dentro de sí mismo; que la conciencia de ser un miembro de la familia de los creyentes conduce inevitablemente a practicar los preceptos de la conducta familiar, el servicio a los propios hermanos y hermanas, en un esfuerzo por realzar y ampliar la fraternidad.

\par 
%\textsuperscript{(1862.7)}
\textsuperscript{170:3.10} La religión del reino es personal, individual; los frutos, los resultados, son familiares, sociales. Jesús nunca dejó de exaltar el carácter sagrado del individuo en contraposición con la comunidad. Pero también reconocía que el hombre desarrolla su carácter mediante el servicio desinteresado; que despliega su naturaleza moral en las relaciones afectuosas con sus semejantes.

\par 
%\textsuperscript{(1862.8)}
\textsuperscript{170:3.11} Al enseñar que el reino es interior, al exaltar al individuo, Jesús dio el golpe de gracia al antiguo orden social, en el sentido de que introdujo la nueva dispensación de la verdadera rectitud social. El mundo ha conocido poco este nuevo orden social, porque ha rehusado practicar los principios del evangelio del reino de los cielos. Cuando este reino de preeminencia espiritual llegue de hecho a la Tierra, no se manifestará simplemente mediante una mejora de las condiciones sociales y materiales, sino más bien mediante la gloria de esos valores espirituales, realzados y enriquecidos, que caracterizan a la era que se aproxima de mejores relaciones humanas y de logros espirituales progresivos.

\section*{4. La enseñanza de Jesús sobre el reino}
\par 
%\textsuperscript{(1862.9)}
\textsuperscript{170:4.1} Jesús nunca dio una definición precisa del reino. En ciertos momentos disertaba sobre una fase del reino, y en otros hablaba de un aspecto diferente de la fraternidad del reino de Dios en el corazón de los hombres. En el transcurso del sermón de este sábado por la tarde, Jesús señaló no menos de cinco fases, o épocas del reino, que fueron las siguientes:

\par 
%\textsuperscript{(1862.10)}
\textsuperscript{170:4.2} 1. La experiencia personal e interior de la vida espiritual del creyente individual que comulga con Dios Padre.

\par 
%\textsuperscript{(1863.1)}
\textsuperscript{170:4.3} 2. La fraternidad creciente de los creyentes en el evangelio, los aspectos sociales de la moral elevada y de la ética vivificada que son el resultado del reinado del espíritu de Dios en el corazón de los creyentes individuales.

\par 
%\textsuperscript{(1863.2)}
\textsuperscript{170:4.4} 3. La fraternidad supermortal de los seres espirituales invisibles que prevalece en la Tierra y en el cielo, el reino sobrehumano de Dios.

\par 
%\textsuperscript{(1863.3)}
\textsuperscript{170:4.5} 4. La perspectiva de una realización más perfecta de la voluntad de Dios, el progreso hacia el amanecer de un nuevo orden social en conexión con una vida espiritual mejorada ---la era siguiente de la humanidad.

\par 
%\textsuperscript{(1863.4)}
\textsuperscript{170:4.6} 5. El reino en su plenitud, la futura era espiritual de luz y de vida en la Tierra.

\par 
%\textsuperscript{(1863.5)}
\textsuperscript{170:4.7} Por eso tenemos siempre que examinar la enseñanza del Maestro para averiguar a cuál de estas cinco fases puede estar refiriéndose cuando utiliza la expresión <<el reino de los cielos>>. Mediante este proceso de cambiar gradualmente la voluntad del hombre, influyendo así en las decisiones humanas, Miguel y sus asociados están cambiando también, de manera gradual pero segura, todo el curso de la evolución humana, tanto social como en otros aspectos.

\par 
%\textsuperscript{(1863.6)}
\textsuperscript{170:4.8} En esta ocasión, el Maestro hizo hincapié en los cinco puntos siguientes que representan las características esenciales del evangelio del reino:

\par 
%\textsuperscript{(1863.7)}
\textsuperscript{170:4.9} 1. La preeminencia del individuo.

\par 
%\textsuperscript{(1863.8)}
\textsuperscript{170:4.10} 2. La voluntad como factor determinante en la experiencia del hombre.

\par 
%\textsuperscript{(1863.9)}
\textsuperscript{170:4.11} 3. La comunión espiritual con Dios Padre.

\par 
%\textsuperscript{(1863.10)}
\textsuperscript{170:4.12} 4. Las satisfacciones supremas de servir con amor a los hombres.

\par 
%\textsuperscript{(1863.11)}
\textsuperscript{170:4.13} 5. La trascendencia de lo espiritual sobre lo material en la personalidad humana.

\par 
%\textsuperscript{(1863.12)}
\textsuperscript{170:4.14} Este mundo nunca ha puesto a prueba de manera seria, sincera y honrada estas ideas dinámicas y estos ideales divinos de la doctrina del reino de los cielos enseñada por Jesús. Pero no deberíais desanimaros por el progreso aparentemente lento de la idea del reino en Urantia. Recordad que el orden de la evolución progresiva está sujeto a cambios periódicos, repentinos e inesperados, tanto en el mundo material como en el mundo espiritual. La donación de Jesús como Hijo encarnado fue precisamente uno de esos acontecimientos extraños e inesperados en la vida espiritual del mundo. Al buscar la manifestación del reino en la época presente, no cometáis tampoco el error fatal de olvidar establecerlo en vuestra propia alma.

\par 
%\textsuperscript{(1863.13)}
\textsuperscript{170:4.15} Aunque Jesús se refirió a una fase del reino situada en el futuro, y sugirió en numerosas ocasiones que dicho acontecimiento podría suceder como parte de una crisis mundial; y aunque en diversas ocasiones prometió con precisión que algún día regresaría con toda seguridad a Urantia, hay que indicar que nunca asoció explícitamente estas dos ideas entre sí. Prometió una nueva revelación del reino en la Tierra en algún momento del futuro; también prometió que volvería alguna vez en persona a este mundo; pero no dijo que estos dos acontecimientos tuvieran la misma significación. Por todo lo que sabemos, estas promesas pueden referirse, o no, al mismo acontecimiento.

\par 
%\textsuperscript{(1863.14)}
\textsuperscript{170:4.16} Sus apóstoles y discípulos asociaron con toda seguridad estas dos enseñanzas. Cuando el reino no se materializó tal como habían esperado, recordaron la enseñanza del Maestro sobre un reino futuro y se acordaron de su promesa de volver, apresurándose a deducir que aquellas promesas se referían a un mismo acontecimiento. Por eso vivieron con la esperanza de su segunda venida inmediata para establecer el reino en su plenitud, con poder y gloria. Y así han vivido las generaciones sucesivas de creyentes en la Tierra, albergando la misma esperanza inspiradora pero decepcionante.

\section*{5. Las ideas posteriores sobre el reino}
\par 
%\textsuperscript{(1864.1)}
\textsuperscript{170:5.1} Después de haber resumido las enseñanzas de Jesús sobre el reino de los cielos, se nos ha permitido describir algunas ideas posteriores que se agregaron al concepto del reino, y emprender un pronóstico profético del reino tal como podría evolucionar en la era venidera.

\par 
%\textsuperscript{(1864.2)}
\textsuperscript{170:5.2} Durante los primeros siglos de la propaganda cristiana, la idea del reino de los cielos estuvo enormemente influida por los conceptos del idealismo griego que entonces se estaban difundiendo rápidamente, la idea de lo natural como sombra de lo espiritual ---de lo temporal como sombra de lo eterno, en el tiempo.

\par 
%\textsuperscript{(1864.3)}
\textsuperscript{170:5.3} Pero el gran paso que marcó el trasplante de las enseñanzas de Jesús desde un terreno judío a un terreno gentil se produjo cuando el Mesías del reino se transformó en el Redentor de la iglesia, una organización religiosa y social nacida de las actividades de Pablo y de sus sucesores, y basada en las enseñanzas de Jesús tal como fueron complementadas con las ideas de Filón y las doctrinas persas del bien y del mal.

\par 
%\textsuperscript{(1864.4)}
\textsuperscript{170:5.4} Las ideas y los ideales de Jesús, incorporados en la enseñanza del evangelio del reino, casi no llegaron a realizarse cuando sus seguidores tergiversaron progresivamente sus declaraciones. El concepto del reino presentado por el Maestro fue notablemente modificado por dos grandes tendencias:

\par 
%\textsuperscript{(1864.5)}
\textsuperscript{170:5.5} 1. Los creyentes judíos persistieron en considerarlo como el \textit{Mesías}. Creían que Jesús regresaría muy pronto para establecer realmente un reino mundial más o menos material.

\par 
%\textsuperscript{(1864.6)}
\textsuperscript{170:5.6} 2. Los cristianos gentiles empezaron muy pronto a aceptar las doctrinas de Pablo, que condujeron cada vez más a la creencia general de que Jesús era el \textit{Redentor} de los hijos de la iglesia, la nueva sucesora institucional del concepto primitivo de la fraternidad puramente espiritual del reino.

\par 
%\textsuperscript{(1864.7)}
\textsuperscript{170:5.7} La iglesia, como consecuencia social del reino, hubiera sido enteramente natural e incluso deseable. El mal de la iglesia no fue su existencia, sino más bien el hecho de que sustituyó casi por completo el concepto del reino presentado por Jesús. La iglesia institucionalizada de Pablo se convirtió prácticamente en el sustituto del reino de los cielos que Jesús había proclamado.

\par 
%\textsuperscript{(1864.8)}
\textsuperscript{170:5.8} Pero no lo dudéis, este mismo reino de los cielos que el Maestro enseñó que existe en el corazón de los creyentes, será proclamado aún a esta iglesia cristiana, así como a todas las demás religiones, razas y naciones de la Tierra ---e incluso a cada individuo.

\par 
%\textsuperscript{(1864.9)}
\textsuperscript{170:5.9} El reino enseñado por Jesús, el ideal espiritual de la rectitud individual y el concepto de la comunión divina del hombre con Dios, se sumergió gradualmente en el concepto místico de la persona de Jesús como Redentor-Creador y jefe espiritual de una comunidad religiosa socializada. De esta manera, una iglesia oficial e institucional se volvió la sustituta de la fraternidad del reino dirigida individualmente por el espíritu.

\par 
%\textsuperscript{(1864.10)}
\textsuperscript{170:5.10} La iglesia fue un resultado \textit{social} inevitable y útil de la vida y de las enseñanzas de Jesús; la tragedia consistió en el hecho de que esta reacción social a las enseñanzas del reino desplazara tan completamente el concepto espiritual del verdadero reino, tal como Jesús lo había enseñado y vivido.

\par 
%\textsuperscript{(1865.1)}
\textsuperscript{170:5.11} Para los judíos, el reino era la \textit{comunidad} israelita; para los gentiles se convirtió en la \textit{iglesia} cristiana. Para Jesús, el reino era el conjunto de las \textit{personas} que habían confesado su fe en la paternidad de Dios, proclamando de ese modo su dedicación total a hacer la voluntad de Dios, volviéndose así miembros de la fraternidad espiritual de los hombres.

\par 
%\textsuperscript{(1865.2)}
\textsuperscript{170:5.12} El Maestro se daba plenamente cuenta de que algunos resultados sociales aparecerían en el mundo como consecuencia de la diseminación del evangelio del reino; pero su intención era que todas estas manifestaciones sociales deseables aparecieran como resultados inconscientes e inevitables, o frutos naturales, de la experiencia personal interior de los creyentes individuales, de esa asociación y comunión puramente espiritual con el espíritu divino que reside en todos esos creyentes y los moviliza.

\par 
%\textsuperscript{(1865.3)}
\textsuperscript{170:5.13} Jesús preveía que una organización social, o iglesia, seguiría al progreso del verdadero reino espiritual, y por eso no se opuso nunca a que los apóstoles practicaran el rito del bautismo de Juan. Enseñó que el alma que ama la verdad, el alma que tiene hambre y sed de rectitud, de Dios, es admitida por la fe en el reino espiritual; al mismo tiempo, los apóstoles enseñaban que dicho creyente es admitido en la organización social de los discípulos mediante el rito exterior del bautismo.

\par 
%\textsuperscript{(1865.4)}
\textsuperscript{170:5.14} Cuando los seguidores inmediatos de Jesús reconocieron que habían fracasado parcialmente en la realización del ideal del Maestro, consistente en establecer el reino en el corazón de los hombres mediante la dominación y la guía del espíritu en los creyentes individuales, se pusieron a salvar su enseñanza para que no se perdiera por completo, sustituyendo el ideal del reino que tenía el Maestro por la creación gradual de una organización social visible, la iglesia cristiana. Después de haber efectuado este programa de sustitución, procedieron a situar el reino en el futuro para mantener la coherencia y asegurar el reconocimiento de las enseñanzas del Maestro sobre el hecho del reino. En cuanto la iglesia estuvo bien establecida, empezó a enseñar que el reino aparecería en realidad cuando culminara la era cristiana, con la segunda venida de Cristo.

\par 
%\textsuperscript{(1865.5)}
\textsuperscript{170:5.15} De esta manera, el reino se convirtió en el concepto de una era, en la idea de una visita futura, y en el ideal de la redención final de los santos del Altísimo. Los primeros cristianos (y muchísimos cristianos posteriores) perdieron generalmente de vista la idea Padre-e-hijo incluida en la enseñanza de Jesús sobre el reino, sustituyéndola por la comunidad social bien organizada de la iglesia. Así, la iglesia se convirtió principalmente en una fraternidad \textit{social}, que desplazó eficazmente el concepto y el ideal de Jesús de una fraternidad \textit{espiritual}.

\par 
%\textsuperscript{(1865.6)}
\textsuperscript{170:5.16} El concepto ideal de Jesús fracasó en gran parte, pero sobre los fundamentos de la vida y de las enseñanzas personales del Maestro, complementados con los conceptos griegos y persas de la vida eterna, y acrecentados con la doctrina de Filón sobre el contraste de lo temporal con lo espiritual, Pablo se puso a construir una de las sociedades humanas más progresivas que jamás han existido en Urantia.

\par 
%\textsuperscript{(1865.7)}
\textsuperscript{170:5.17} El concepto de Jesús está todavía vivo en las religiones avanzadas del mundo. La iglesia cristiana de Pablo es la sombra socializada y humanizada del reino de los cielos que Jesús tenía en proyecto ---y que llegará a ser así con toda seguridad. Pablo y sus sucesores transfirieron parcialmente las cuestiones de la vida eterna desde el individuo a la iglesia. Cristo se convirtió así en la cabeza de la iglesia, en lugar de ser el hermano mayor de cada creyente individual dentro de la familia del reino del Padre. Pablo y sus contemporáneos aplicaron a la \textit{iglesia}, como grupo de creyentes, todas las implicaciones espirituales de Jesús relacionadas con él mismo y con el creyente individual; y al hacer esto, asestaron un golpe mortal al concepto de Jesús sobre el reino divino en el corazón de cada creyente.

\par 
%\textsuperscript{(1866.1)}
\textsuperscript{170:5.18} Y así, durante siglos, la iglesia cristiana ha trabajado en una situación muy embarazosa, porque se atrevió a reclamar para sí los misteriosos poderes y privilegios del reino, unos poderes y privilegios que sólo se pueden ejercer y experimentar entre Jesús y sus hermanos espirituales creyentes. De esta manera resulta evidente que la pertenencia a la iglesia no significa necesariamente comunión en el reino; ésta es espiritual, y la otra principalmente social.

\par 
%\textsuperscript{(1866.2)}
\textsuperscript{170:5.19} Tarde o temprano deberá surgir otro Juan el Bautista más grande, que proclamará que <<el reino de Dios está cerca>> ---que propondrá un retorno al elevado concepto espiritual de Jesús, el cual proclamó que el reino es la voluntad de su Padre celestial, dominante y trascendente, en el corazón del creyente--- y hará todo esto sin referirse para nada a la iglesia visible en la Tierra, ni a la esperada segunda venida de Cristo. Es preciso que se produzca un renacimiento de las \textit{verdaderas} enseñanzas de Jesús, que se expongan de nuevo de tal manera que anulen el efecto de la obra de sus primeros seguidores, los cuales se pusieron a crear un sistema sociofilosófico de creencias sobre el \textit{hecho} de la estancia de Miguel en la Tierra. En poco tiempo, la enseñanza de esta historia \textit{acerca de} Jesús sustituyó casi por completo la predicación del evangelio del reino de Jesús. De esta manera, una religión histórica desplazó la enseñanza en la que Jesús había mezclado las ideas morales y los ideales espirituales más elevados del hombre con sus esperanzas más sublimes para el futuro ---la vida eterna. Éste era todo el evangelio del reino.

\par 
%\textsuperscript{(1866.3)}
\textsuperscript{170:5.20} El evangelio de Jesús presentaba muchos aspectos diferentes, y precisamente por eso, en el transcurso de unos pocos siglos, los estudiosos de los relatos de sus enseñanzas se dividieron en tantos cultos y sectas. Esta lamentable subdivisión de los creyentes cristianos se debe a que no han sido capaces de discernir, en las múltiples enseñanzas del Maestro, la divina unidad de su vida incomparable. Pero algún día, los verdaderos creyentes en Jesús no estarán divididos espiritualmente de esta manera en su actitud ante los no creyentes. Siempre podemos tener diferencias de comprensión y de interpretación intelectuales, e incluso diversos grados de socialización, pero la falta de fraternidad espiritual es a la vez inexcusable y reprensible.

\par 
%\textsuperscript{(1866.4)}
\textsuperscript{170:5.21} ¡No os engañéis! Existe en las enseñanzas de Jesús una naturaleza eterna que no les permitirá permanecer estériles para siempre en el corazón de los hombres inteligentes. El reino, tal como Jesús lo concebía, ha fracasado en gran parte en la Tierra; por ahora, una iglesia exterior ha tomado su lugar. Pero deberíais comprender que esta iglesia es solamente el estado larvario del frustrado reino espiritual, que esta iglesia lo transportará a través de la presente era material y lo llevará hasta una dispensación más espiritual en la que las enseñanzas del Maestro gozarán de una mayor oportunidad para desarrollarse. La iglesia llamada cristiana se convierte así en el capullo donde duerme actualmente el concepto que Jesús tenía del reino. El reino de la fraternidad divina está todavía vivo, y saldrá sin duda finalmente de su largo letargo, con la misma seguridad con que la mariposa aparece finalmente como la hermosa manifestación de su crisálida metamórfica menos atractiva.


\chapter{Documento 171. En el camino de Jerusalén}
\par 
%\textsuperscript{(1867.1)}
\textsuperscript{171:0.1} UN DÍA después del memorable sermón sobre <<el reino de los cielos>>, Jesús anunció que partiría al día siguiente con los apóstoles para asistir a la Pascua en Jerusalén, visitando de camino numerosas ciudades del sur de Perea.

\par 
%\textsuperscript{(1867.2)}
\textsuperscript{171:0.2} La alocución sobre el reino y el anuncio de que iría a la Pascua, hicieron que todos sus seguidores creyeran que subía a Jerusalén para inaugurar el reino temporal de la supremacía judía. Independientemente de lo que Jesús dijera sobre el carácter no material del reino, no podía apartar por completo de la mente de sus oyentes judíos la idea de que el Mesías tenía que establecer algún tipo de gobierno nacionalista con sede en Jerusalén.

\par 
%\textsuperscript{(1867.3)}
\textsuperscript{171:0.3} Lo que Jesús dijo en su sermón del sábado sólo contribuyó a confundir a la mayoría de sus seguidores; muy pocos de ellos vieron las cosas más claras con el discurso del Maestro. Los líderes comprendieron algo de sus enseñanzas sobre el reino interior, <<el reino de los cielos dentro de vosotros>>, pero también sabían que había hablado de otro reino futuro, y creían que ahora iba a subir a Jerusalén para establecer dicho reino. Cuando esta expectativa sufrió una decepción, cuando el Maestro fue rechazado por los judíos, y cuando más tarde, Jerusalén fue literalmente destruida, continuaron aferrados a esta esperanza, creyendo sinceramente que el Maestro regresaría pronto al mundo, con un gran poder y una gloria majestuosa, para establecer el reino prometido.

\par 
%\textsuperscript{(1867.4)}
\textsuperscript{171:0.4} Este domingo por la tarde fue cuando Salomé, la madre de Santiago y de Juan Zebedeo, se acercó a Jesús con sus dos hijos apóstoles a la manera en que uno se acerca a un potentado oriental; intentó que Jesús le prometiera de antemano que le concedería cualquier cosa que ella le pidiera. Pero el Maestro no quiso prometer nada; en lugar de eso, le preguntó: <<¿Qué deseas que haga por ti?>> Entonces Salomé respondió: <<Maestro, ahora que vas a subir a Jerusalén para establecer el reino, quisiera pedirte que me prometas por anticipado que estos hijos míos serán honrados contigo, sentándose uno a tu derecha y el otro a tu izquierda en tu reino>>.

\par 
%\textsuperscript{(1867.5)}
\textsuperscript{171:0.5} Cuando Jesús escuchó la petición de Salomé, dijo: <<Mujer, no sabes lo que pides>>. Luego, clavando la mirada en los ojos de los dos apóstoles que buscaban honores, dijo: <<Porque os conozco y os amo desde hace mucho tiempo, porque he vivido incluso en la casa de vuestra madre, porque Andrés os ha encargado de que estéis conmigo en todo momento, por esa razón permitís que vuestra madre venga a verme en secreto para hacerme esta petición improcedente. Pero dejad que os pregunte: ¿Sois capaces de beber la copa que estoy a punto de beber?>> Y sin pararse a reflexionar, Santiago y Juan contestaron: <<Sí, Maestro, somos capaces>>. Jesús dijo: <<Me entristece ver que no sabéis por qué vamos a Jerusalén; me apena que no comprendáis la naturaleza de mi reino; me decepciona que traigáis a vuestra madre para que me haga esta petición; pero sé que me amáis en vuestro corazón; por eso os declaro que beberéis en verdad mi copa de amargura y compartiréis mi humillación, pero no me corresponde concederos que os sentéis a mi derecha o a mi izquierda. Esos honores están reservados para aquellos que han sido designados por mi Padre>>.

\par 
%\textsuperscript{(1868.1)}
\textsuperscript{171:0.6} Para entonces, alguien había comunicado la noticia de esta conversación a Pedro y a los demás apóstoles, y estaban muy indignados porque Santiago y Juan hubieran intentado ser preferidos antes que ellos, y hubieran ido en secreto con su madre para hacer esta petición. Cuando empezaron a discutir entre ellos, Jesús los reunió a todos y dijo: <<Comprendéis muy bien cómo los gobernantes de los gentiles tratan con prepotencia a sus súbditos, y cómo los grandes ejercen su autoridad. Pero no será así en el reino de los cielos. Si alguien quiere ser grande entre vosotros, que se vuelva primero vuestro servidor. El que quiera ser el primero en el reino, que se ponga a vuestro servicio. Os afirmo que el Hijo del Hombre no ha venido para ser servido, sino para servir. Y ahora voy a Jerusalén para dar mi vida haciendo la voluntad del Padre, y sirviendo a mis hermanos>>. Cuando los apóstoles escucharon estas palabras, se retiraron a solas para orar. Aquella noche, en respuesta a los esfuerzos de Pedro, Santiago y Juan se disculparon adecuadamente ante los diez y restablecieron sus buenas relaciones con sus hermanos.

\par 
%\textsuperscript{(1868.2)}
\textsuperscript{171:0.7} Al solicitar un lugar a la derecha y a la izquierda de Jesús en Jerusalén, los hijos de Zebedeo poco podían imaginar que en menos de un mes su amado maestro estaría colgado en una cruz romana, con un ladrón moribundo a un lado y otro infractor al otro lado. Y la madre de ellos, que estuvo presente en la crucifixión, recordó muy bien la tonta petición que había hecho a Jesús en Pella en relación con los honores que tan imprudentemente había buscado para sus hijos apóstoles.

\section*{1. La partida de Pella}
\par 
%\textsuperscript{(1868.3)}
\textsuperscript{171:1.1} El lunes 13 de marzo por la mañana, Jesús y sus doce apóstoles se despidieron definitivamente del campamento de Pella, y partieron hacia el sur en su gira por las ciudades de la Perea meridional, donde los asociados de Abner estaban trabajando. Pasaron más de dos semanas visitando a los setenta, y luego fueron directamente a Jerusalén para la Pascua.

\par 
%\textsuperscript{(1868.4)}
\textsuperscript{171:1.2} Cuando el Maestro salió de Pella, los discípulos que estaban acampados con los apóstoles, aproximadamente unos mil, lo siguieron. Casi la mitad de este grupo se separó de él en el vado del Jordán, camino de Jericó, cuando se enteraron que se dirigía a Hesbón, y después de que hubiera predicado el sermón sobre <<El cálculo del coste>>. Luego continuaron hasta Jerusalén, mientras que la otra mitad del grupo siguió a Jesús durante dos semanas, visitando las ciudades del sur de Perea.

\par 
%\textsuperscript{(1868.5)}
\textsuperscript{171:1.3} La mayor parte de los seguidores inmediatos de Jesús comprendió, de manera general, que el campamento de Pella había sido abandonado, pero creían realmente que esto indicaba que su Maestro se proponía, por fin, ir a Jerusalén y reclamar el trono de David. Una gran mayoría de sus seguidores nunca fue capaz de captar otro concepto del reino de los cielos; independientemente de lo que Jesús les enseñara, no querían renunciar a esta idea judía del reino.

\par 
%\textsuperscript{(1868.6)}
\textsuperscript{171:1.4} Siguiendo las instrucciones del apóstol Andrés, David Zebedeo cerró el campamento de los visitantes en Pella el miércoles 15 de marzo. En aquel momento, cerca de cuatro mil visitantes residían allí, sin incluir a más de mil personas que vivían con los apóstoles en un lugar conocido como el <<campamento de los instructores>>, y que acompañaron a Jesús y los doce hacia el sur. Aunque detestaba tener que hacerlo, David vendió todo el equipo a numerosos compradores y se dirigió con los fondos a Jerusalén, entregando posteriormente el dinero a Judas Iscariote.

\par 
%\textsuperscript{(1869.1)}
\textsuperscript{171:1.5} David estuvo presente en Jerusalén durante la última semana trágica, y se llevó a su madre con él a Betsaida después de la crucifixión. Mientras esperaba a Jesús y a los apóstoles, David se detuvo en casa de Lázaro en Betania y se sintió enormemente perturbado por la manera en que los fariseos habían empezado a perseguirlo y a agobiarlo desde su resurrección. Andrés había ordenado a David que suspendiera el servicio de mensajeros, y todos interpretaron esto como una indicación de que el reino se iba a establecer pronto en Jerusalén. David se encontraba sin ocupación, y casi tenía decidido convertirse en el defensor autodesignado de Lázaro, cuando de pronto el objeto de su indignada preocupación huyó precipitadamente a Filadelfia. En consecuencia, algún tiempo después de la resurrección de Jesús y también de la muerte de su madre, David se fue a Filadelfia, no sin antes haber ayudado a Marta y María a vender sus propiedades. Allí pasó el resto de su vida en asociación con Abner y Lázaro, convirtiéndose en el supervisor financiero de todos los numerosos intereses del reino que tuvieron su centro en Filadelfia durante la vida de Abner.

\par 
%\textsuperscript{(1869.2)}
\textsuperscript{171:1.6} Poco tiempo después de la destrucción de Jerusalén, Antioquía se volvió la sede del \textit{cristianismo paulino}, mientras que Filadelfia siguió siendo el centro del \textit{reino de los cielos según Abner}. Desde Antioquía, la versión paulina de las enseñanzas de Jesús y acerca de Jesús se difundió hacia todo el mundo occidental; desde Filadelfia, los misioneros de la versión abneriana del reino de los cielos se extendieron por toda Mesopotamia y Arabia, hasta la época posterior en que estos emisarios inflexibles de las enseñanzas de Jesús fueron arrollados por el ascenso súbito del islam.

\section*{2. El cálculo del coste}
\par 
%\textsuperscript{(1869.3)}
\textsuperscript{171:2.1} Cuando Jesús y el grupo de casi mil seguidores llegaron al vado de Betania en el Jordán, llamado a veces Betábara, sus discípulos empezaron a darse cuenta de que no se dirigía directamente a Jerusalén. Mientras dudaban y discutían entre ellos, Jesús se subió en una piedra gigantesca y pronunció el discurso que se conoce como <<El cálculo del coste>>. El Maestro dijo:

\par 
%\textsuperscript{(1869.4)}
\textsuperscript{171:2.2} <<De ahora en adelante, los que queréis seguirme debéis estar dispuestos a pagar el precio de una dedicación total a hacer la voluntad de mi Padre. Si queréis ser mis discípulos, debéis estar dispuestos a abandonar padre, madre, esposa, hijos, hermanos y hermanas. Si alguno de vosotros quiere ser ahora mi discípulo, debe estar dispuesto a renunciar incluso a su vida, de la misma manera que el Hijo del Hombre está a punto de ofrecer su vida para completar su misión de hacer la voluntad del Padre en la Tierra y en la carne>>.

\par 
%\textsuperscript{(1869.5)}
\textsuperscript{171:2.3} <<Si no estás dispuesto a pagar el precio íntegro, difícilmente puedes ser mi discípulo. Antes de que continuéis, cada uno de vosotros debería sentarse y calcular lo que le cuesta ser mi discípulo. ¿Quién de vosotros emprendería la construcción de una torre de vigilancia en sus tierras, sin sentarse primero a calcular el coste para ver si posee el dinero suficiente para terminarla? Si descuidáis así calcular el gasto, es posible que descubráis, después de haber echado los cimientos, que sois incapaces de terminar lo que habéis empezado. Entonces, todos vuestros vecinos se burlarán de vosotros, diciendo: `Mirad, este hombre ha empezado a construir, pero no ha sido capaz de terminar su obra.' Y también, ¿qué rey que se prepara para hacer la guerra a otro rey, no se sienta primero para consultar si con diez mil hombres podrá enfrentarse al que viene contra él con veinte mil? Si el rey no puede enfrentarse con su enemigo porque no está preparado, envía una embajada al otro rey, mientras éste se encuentra aún muy lejos, para preguntarle por las condiciones de paz>>.

\par 
%\textsuperscript{(1870.1)}
\textsuperscript{171:2.4} <<Ahora es preciso, pues, que cada uno de vosotros se siente y calcule lo que le cuesta ser mi discípulo. De ahora en adelante ya no podrás seguirnos, escuchando las enseñanzas y contemplando las obras; tendrás que enfrentarte con persecuciones encarnizadas y dar testimonio de este evangelio en medio de decepciones aplastantes. Si no estás dispuesto a renunciar a todo lo que eres, y a consagrar todo lo que posees, entonces no eres digno de ser mi discípulo. Si ya te has conquistado a ti mismo dentro de tu corazón, no necesitas tener ningún miedo a esa victoria exterior que pronto tendrás que conseguir cuando el Hijo del Hombre sea rechazado por los principales sacerdotes y los saduceos, y entregado a los incrédulos burlones>>.

\par 
%\textsuperscript{(1870.2)}
\textsuperscript{171:2.5} <<Ahora deberías examinarte y descubrir el motivo que tienes para ser mi discípulo. Si buscas honores y gloria, si tienes inclinaciones mundanas, eres como la sal que ha perdido su sabor. Y cuando aquello que se valora por su sabor salado ha perdido su sabor, ¿con qué se sazonará? Un condimento así es inútil; sólo sirve para ser tirado a la basura. Ya os he advertido que regreséis en paz a vuestros hogares si no estáis dispuestos a beber conmigo la copa que se está preparando. Os he dicho una y otra vez que mi reino no es de este mundo, pero no queréis creerme. El que tenga oídos para oír, que oiga lo que digo>>.

\par 
%\textsuperscript{(1870.3)}
\textsuperscript{171:2.6} Inmediatamente después de decir estas palabras, Jesús, a la cabeza de los doce, partió en dirección a Hesbón, seguido de unas quinientas personas. Después de un breve intervalo, la otra mitad de la multitud continuó hacia Jerusalén. Sus apóstoles, así como los discípulos principales, reflexionaron mucho sobre estas palabras, pero continuaban aferrados a la creencia de que, después de este breve período de adversidad y de prueba, el reino sería sin duda establecido de acuerdo en cierto modo con sus esperanzas tanto tiempo acariciadas.

\section*{3. La gira por Perea}
\par 
%\textsuperscript{(1870.4)}
\textsuperscript{171:3.1} Durante más de dos semanas, Jesús y los doce, seguidos por una multitud de varios cientos de discípulos, viajaron por el sur de Perea, visitando todas las ciudades donde trabajaban los setenta. En esta región vivían muchos gentiles, y puesto que pocos de ellos iban a la fiesta de la Pascua en Jerusalén, los mensajeros del reino continuaron sin interrupción su trabajo de enseñanza y de predicación.

\par 
%\textsuperscript{(1870.5)}
\textsuperscript{171:3.2} Jesús se encontró con Abner en Hesbón, y Andrés ordenó que no se interrumpieran los trabajos de los setenta por la fiesta de la Pascua; Jesús aconsejó a los mensajeros que continuaran con su obra, sin prestar ninguna atención a lo que estaba a punto de suceder en Jerusalén. También aconsejó a Abner que permitiera al cuerpo de mujeres, al menos a las que lo desearan, ir a Jerusalén para la Pascua. Ésta fue la última vez que Abner vio a Jesús en la carne. Se despidió de Abner diciéndole: <<Hijo mío, sé que serás fiel al reino, y ruego al Padre que te conceda sabiduría para que puedas amar y comprender a tus hermanos>>.

\par 
%\textsuperscript{(1870.6)}
\textsuperscript{171:3.3} Mientras viajaban de ciudad en ciudad, una gran cantidad de sus seguidores los abandonaron para continuar hacia Jerusalén, de tal manera que, cuando Jesús partió para la Pascua, el número de los que lo habían acompañado día tras día se había reducido a menos de doscientos.

\par 
%\textsuperscript{(1871.1)}
\textsuperscript{171:3.4} Los apóstoles comprendieron que Jesús iba a Jerusalén para la Pascua. Sabían que el sanedrín había difundido un mensaje por todo Israel anunciando que había sido condenado a muerte, y ordenando que cualquiera que supiera dónde estaba informara al sanedrín; sin embargo, a pesar de todo esto, no estaban tan alarmados como cuando Jesús les había anunciado, en Filadelfia, que iba a Betania para ver a Lázaro. Este cambio de actitud, que pasó de un miedo intenso a un estado de discreta expectativa, se debía principalmente a la resurrección de Lázaro. Habían llegado a la conclusión de que Jesús podría, en caso de emergencia, afirmar su poder divino y poner en evidencia a sus enemigos. Esta esperanza, unida a su fe más profunda y madura en la supremacía espiritual de su Maestro, explica el valor exterior demostrado por sus seguidores inmediatos, los cuales se preparaban ahora para seguirlo hasta Jerusalén, haciendo caso omiso de la declaración pública del sanedrín de que debía morir.

\par 
%\textsuperscript{(1871.2)}
\textsuperscript{171:3.5} La mayoría de los apóstoles y muchos de sus discípulos más allegados no creían que Jesús pudiera morir; como opinaban que él era <<la resurrección y la vida>>, lo consideraban como inmortal y ya triunfante sobre la muerte.

\section*{4. La enseñanza en Livias}
\par 
%\textsuperscript{(1871.3)}
\textsuperscript{171:4.1} El miércoles 29 de marzo al anochecer, Jesús y sus seguidores acamparon en Livias, camino de Jerusalén, después de haber completado su gira por las ciudades del sur de Perea. Durante esta noche en Livias fue cuando Simón Celotes y Simón Pedro, que se habían confabulado para que les entregaran en este lugar más de cien espadas, recibieron y distribuyeron estas armas a todos los que quisieron aceptarlas y llevarlas ocultas debajo de sus mantos. Simón Pedro todavía llevaba su espada la noche en que el Maestro fue traicionado en el jardín.

\par 
%\textsuperscript{(1871.4)}
\textsuperscript{171:4.2} El jueves por la mañana temprano, antes de que se despertaran los demás, Jesús llamó a Andrés y le dijo: <<¡Despierta a tus hermanos! Tengo algo que decirles>>. Jesús sabía lo de las espadas y qué apóstoles habían recibido y llevaban estas armas, pero nunca les reveló que conocía estas cosas. Cuando Andrés hubo despertado a sus compañeros y estos se hubieron reunido, Jesús les dijo: <<Hijos míos, habéis estado conmigo mucho tiempo, y os he enseñado muchas cosas que son útiles para esta época, pero ahora quisiera advertiros que no pongáis vuestra confianza en las incertidumbres de la carne ni en las debilidades de la defensa humana, contra las pruebas y aflicciones que nos esperan. Os he reunido aquí a solas para poder deciros una vez más, claramente, que vamos a Jerusalén, donde sabéis que el Hijo del Hombre ya ha sido condenado a muerte. Os digo de nuevo que el Hijo del Hombre será entregado a los principales sacerdotes y a los dirigentes religiosos, los cuales lo condenarán y luego lo entregarán a los gentiles. Y así, se burlarán del Hijo del Hombre, incluso le escupirán y lo azotarán, y lo entregarán a la muerte. Y cuando maten al Hijo del Hombre, no os sintáis consternados, porque os declaro que al tercer día resucitará. Cuidad de vosotros mismos y recordad que os he prevenido>>.

\par 
%\textsuperscript{(1871.5)}
\textsuperscript{171:4.3} Los apóstoles se quedaron de nuevo asombrados, anonadados; pero no se decidieron a considerar sus palabras al pie de la letra; no podían comprender que el Maestro quería decir exactamente lo que había dicho. Estaban tan cegados por su creencia persistente en un reino temporal en la Tierra, con sede en Jerusalén, que simplemente no podían ---no querían--- permitirse el aceptar literalmente las palabras de Jesús. Todo aquel día estuvieron reflexionando sobre lo que el Maestro había querido decir con estas extrañas declaraciones. Pero ninguno se atrevió a preguntarle sobre ellas. Hasta después de la muerte de Jesús, estos apóstoles desconcertados no llegaron a comprender que el Maestro les había hablado por anticipado, clara y directamente, de su crucifixión.

\par 
%\textsuperscript{(1872.1)}
\textsuperscript{171:4.4} Fue aquí en Livias donde algunos fariseos amistosos vinieron a ver a Jesús poco después del desayuno, y le dijeron: <<Huye deprisa de estos lugares, porque Herodes pretende ahora matarte tal como hizo con Juan. Teme un levantamiento del pueblo y ha decidido matarte. Te traemos esta advertencia para que puedas huir>>.

\par 
%\textsuperscript{(1872.2)}
\textsuperscript{171:4.5} Esto era parcialmente cierto. La resurrección de Lázaro había asustado y alarmado a Herodes, y sabiendo que el sanedrín se había atrevido a condenar a Jesús incluso antes de juzgarlo, Herodes había decidido o bien matar a Jesús, o echarlo fuera de su territorio. En realidad deseaba hacer lo segundo, pues le tenía tanto miedo que esperaba no verse obligado a ejecutarlo.

\par 
%\textsuperscript{(1872.3)}
\textsuperscript{171:4.6} Cuando escuchó lo que los fariseos tenían que decirle, Jesús respondió: <<Conozco bien a Herodes y el miedo que tiene a este evangelio del reino. Pero no os engañéis, preferiría mucho más que el Hijo del Hombre subiera a Jerusalén para sufrir y morir a manos de los jefes de los sacerdotes; como se ha manchado las manos con la sangre de Juan, no tiene el deseo de responsabilizarse de la muerte del Hijo del Hombre. Id a decirle a ese zorro que el Hijo del Hombre predica hoy en Perea, que mañana irá a Judea, y que dentro de unos días habrá terminado su misión en la Tierra y estará preparado para ascender hacia el Padre>>.

\par 
%\textsuperscript{(1872.4)}
\textsuperscript{171:4.7} Luego Jesús se volvió hacia sus apóstoles, y dijo: <<Desde los tiempos antiguos los profetas han perecido en Jerusalén, y es apropiado que el Hijo del Hombre vaya a la ciudad de la casa del Padre para ser sacrificado como precio del fanatismo humano, y como consecuencia de los prejuicios religiosos y de la ceguera espiritual. ¡Oh Jerusalén, Jerusalén, que matas a los profetas y lapidas a los instructores de la verdad! ¡Cuántas veces hubiera querido reunir a tus hijos como una gallina reúne a sus polluelos debajo de sus alas, pero no me has dejado hacerlo! ¡He aquí que tu casa está a punto de quedarse desolada! Muchas veces desearás verme, pero no podrás. Entonces me buscarás, pero no me encontrarás>>. Después de haber hablado así, se volvió hacia los que le rodeaban y dijo: <<Sin embargo, vayamos a Jerusalén para asistir a la Pascua y hacer lo que nos corresponda para llevar a cabo la voluntad del Padre que está en los cielos>>.

\par 
%\textsuperscript{(1872.5)}
\textsuperscript{171:4.8} Un grupo confundido y desconcertado de creyentes siguió aquel día a Jesús hasta Jericó. En las declaraciones de Jesús sobre el reino, los apóstoles sólo podían discernir la certidumbre del triunfo final; simplemente no se dejaban llevar hasta el punto de estar dispuestos a captar las advertencias de un revés inminente. Cuando Jesús habló de <<resucitar al tercer día>>, se aferraron a que esta declaración significaba un triunfo seguro del reino inmediatamente después de una desagradable escaramuza preliminar con los jefes religiosos de los judíos. El <<tercer día>> era una expresión corriente judía que significaba <<pronto>> o <<poco después>>. Cuando Jesús habló de <<resucitar>>, pensaron que se refería a la <<resurrección del reino>>.

\par 
%\textsuperscript{(1872.6)}
\textsuperscript{171:4.9} Estos creyentes habían aceptado a Jesús como el Mesías, y los judíos no sabían nada o casi nada sobre un Mesías sufriente. No comprendían que Jesús iba a conseguir con su muerte muchas cosas que nunca podría haber logrado con su vida. La resurrección de Lázaro es la que había armado de valor a los apóstoles para entrar en Jerusalén, pero el recuerdo de la transfiguración fue lo que sostuvo al Maestro durante este duro período de su donación.

\section*{5. El ciego de Jericó}
\par 
%\textsuperscript{(1873.1)}
\textsuperscript{171:5.1} El jueves 30 de marzo al atardecer, Jesús y sus apóstoles, a la cabeza de un grupo de unos doscientos seguidores, se aproximaron a los muros de Jericó. Al acercarse a la puerta de la ciudad se encontraron con una multitud de mendigos entre los que se hallaba un tal Bartimeo, un anciano que había estado ciego desde su juventud. Este mendigo ciego había oído hablar mucho de Jesús y lo sabía todo sobre la curación del ciego Josías en Jerusalén. No se había enterado de la última visita de Jesús a Jericó hasta que éste había partido hacia Betania. Bartimeo había decidido que nunca más permitiría que Jesús visitara Jericó sin recurrir a él para que le devolviera la vista.

\par 
%\textsuperscript{(1873.2)}
\textsuperscript{171:5.2} La noticia de la llegada de Jesús se había difundido por todo Jericó, y centenares de habitantes se habían congregado para salir a su encuentro. Cuando este gran gentío regresó escoltando al Maestro por la ciudad, Bartimeo escuchó el ruido de los pasos de la multitud y supo que ocurría algo fuera de lo normal, por lo que preguntó a los que estaban cerca de él qué era lo que sucedía. Uno de los mendigos le contestó: <<Está pasando Jesús de Nazaret>>. Cuando Bartimeo escuchó que Jesús estaba cerca, elevó la voz y empezó a gritar: <<¡Jesús, Jesús, ten piedad de mí!>> Como continuaba gritando cada vez más fuerte, algunos de los que estaban cerca de Jesús fueron hacia él y le reprendieron, pidiéndole que guardara silencio. Pero fue en vano; se limitó a gritar aún más y más fuerte todavía.

\par 
%\textsuperscript{(1873.3)}
\textsuperscript{171:5.3} Cuando Jesús escuchó los gritos del ciego, se detuvo. Y cuando lo vio, dijo a sus amigos: <<Traedme a ese hombre>>. Entonces se acercaron a Bartimeo, diciendo: <<Alégrate y ven con nosotros, porque el Maestro te llama>>. Cuando Bartimeo escuchó estas palabras, tiró a un lado su manto y saltó hacia el centro de la carretera, mientras que los que estaban cerca lo guiaban hacia Jesús. Dirigiéndose a Bartimeo, Jesús dijo: <<¿Qué quieres que haga por ti?>> Entonces el ciego contestó: <<Quisiera recobrar la vista>>. Cuando Jesús escuchó esta petición y vio su fe, dijo: <<Recobrarás la vista; sigue tu camino, tu fe te ha curado>>. Bartimeo recuperó inmediatamente la vista y permaneció cerca de Jesús, glorificando a Dios, hasta que el Maestro partió al día siguiente para Jerusalén; entonces precedió a la multitud, proclamando a todo el mundo cómo le habían devuelto la vista en Jericó.

\section*{6. La visita a Zaqueo}
\par 
%\textsuperscript{(1873.4)}
\textsuperscript{171:6.1} Cuando la procesión del Maestro entró en Jericó, el Sol estaba a punto de ponerse, y Jesús se dispuso a permanecer allí durante la noche. Mientras pasaba por delante de la aduana, Zaqueo, el jefe publicano o recaudador de impuestos, se encontraba allí por casualidad, y tenía muchos deseos de ver a Jesús. Este jefe publicano era muy rico y había oído hablar mucho de este profeta de Galilea. Había decidido ver qué tipo de hombre era Jesús la próxima vez que visitara Jericó. En consecuencia, Zaqueo trató de abrirse paso entre el gentío, pero éste era demasiado grande, y como era bajo de estatura, no podía ver por encima de las cabezas. Así pues, el jefe publicano siguió a la multitud hasta que llegaron cerca del centro de la ciudad, no lejos de donde él vivía. Cuando vio que no sería capaz de traspasar la multitud, y pensando que Jesús quizás atravesaría la ciudad sin detenerse, se adelantó corriendo y se subió a un sicomoro cuyas ramas extendidas colgaban por encima de la calzada. Sabía que de esta manera podría ver muy bien al Maestro cuando éste pasara. Y no quedó decepcionado porque, al pasar por allí, Jesús se detuvo, levantó la vista hacia Zaqueo, y dijo: <<Date prisa en bajar, Zaqueo, porque esta noche he de quedarme en tu casa>>. Cuando Zaqueo escuchó estas palabras sorprendentes, estuvo a punto de caerse del árbol en su prisa por bajar y, acercándose a Jesús, expresó su gran alegría porque el Maestro quisiera detenerse en su casa.

\par 
%\textsuperscript{(1874.1)}
\textsuperscript{171:6.2} Fueron inmediatamente a la casa de Zaqueo, y los habitantes de Jericó se quedaron muy sorprendidos de que Jesús consintiera en residir con el jefe publicano. Mientras el Maestro y sus apóstoles se demoraban con Zaqueo delante de la puerta de su casa, uno de los fariseos de Jericó que estaba cerca, dijo: <<Ya veis cómo este hombre ha ido a alojarse con un hijo apóstata de Abraham, con un pecador que es un opresor y roba a su propio pueblo>>. Cuando Jesús escuchó esto, bajó la mirada sobre Zaqueo y sonrió. Entonces Zaqueo se subió en un taburete y dijo: <<¡Hombres de Jericó, escuchadme! Quizás soy un publicano y un pecador, pero el gran Instructor ha venido a residir en mi casa. Antes de que entre, os digo que voy a dar la mitad de todos mis bienes a los pobres; y a partir de mañana, si he exigido algo a alguien de manera injusta, le devolveré el cuádruple. Voy a buscar la salvación con todo mi corazón, y a aprender a actuar con rectitud a los ojos de Dios>>.

\par 
%\textsuperscript{(1874.2)}
\textsuperscript{171:6.3} Cuando Zaqueo hubo terminado de hablar, Jesús dijo: <<Hoy ha llegado la salvación a esta casa, y te has vuelto en verdad un hijo de Abraham>>. Y volviéndose hacia la multitud congregada alrededor de ellos, Jesús dijo: <<No os maravilléis por lo que digo ni os ofendáis por lo que hacemos, pues he declarado desde el principio que el Hijo del Hombre ha venido a buscar y a salvar lo que estaba perdido>>.

\par 
%\textsuperscript{(1874.3)}
\textsuperscript{171:6.4} Se alojaron en casa de Zaqueo durante la noche. A la mañana siguiente se levantaron y se dirigieron por <<la ruta de los ladrones>> hacia Betania, camino de la Pascua en Jerusalén.

\section*{7. <<Mientras Jesús pasaba>>}
\par 
%\textsuperscript{(1874.4)}
\textsuperscript{171:7.1} Jesús sembraba la alegría por dondequiera que iba. Estaba lleno de benevolencia y de verdad. Sus compañeros nunca dejaron de maravillarse por las palabras agradables que salían de su boca. Podéis cultivar la gentileza, pero la dulzura es el aroma de la amistad que emana de un alma saturada de amor.

\par 
%\textsuperscript{(1874.5)}
\textsuperscript{171:7.2} La bondad impone siempre el respeto, pero cuando está desprovista de agrado, a menudo repele el afecto. La bondad sólo es universalmente atractiva cuando es agradable. La bondad sólo es eficaz cuando es atrayente.

\par 
%\textsuperscript{(1874.6)}
\textsuperscript{171:7.3} Jesús comprendía realmente a los hombres; por eso podía manifestar una simpatía verdadera y mostrar una compasión sincera. Pero rara vez se permitía la lástima. Mientras que su compasión era ilimitada, su simpatía era práctica, personal y constructiva. Su familiaridad con el sufrimiento nunca engendró su indiferencia, y era capaz de ayudar a las almas afligidas sin aumentar la lástima de sí mismas.

\par 
%\textsuperscript{(1874.7)}
\textsuperscript{171:7.4} Jesús podía ayudar tanto a los hombres porque también los amaba sinceramente. Amaba realmente a cada hombre, a cada mujer y a cada niño. Podía ser un amigo así de auténtico debido a su perspicacia extraordinaria ---conocía plenamente el contenido del corazón y de la mente del hombre. Era un observador penetrante y lleno de interés. Era experto en comprender las necesidades humanas y hábil en detectar los anhelos humanos.

\par 
%\textsuperscript{(1874.8)}
\textsuperscript{171:7.5} Jesús nunca tenía prisa. Tenía tiempo para confortar a sus semejantes <<mientras pasaba>>. Siempre procuraba que sus amigos se sintieran a gusto. Era un oyente encantador. Nunca se dedicaba a explorar de manera indiscreta el alma de sus compañeros. Cuando confortaba a las mentes hambrientas y ayudaba a las almas sedientas, los que recibían su misericordia no tenían el sentimiento de estar \textit{confesándose} con él, sino más bien de estar \textit{conversando} con él. Tenían una confianza ilimitada en él porque veían que él tenía también mucha fe en ellos.

\par 
%\textsuperscript{(1875.1)}
\textsuperscript{171:7.6} Nunca parecía tener curiosidad por la gente, y nunca manifestaba el deseo de dirigirlos, manejarlos o investigarlos. Inspiraba una profunda confianza en uno mismo y una sólida valentía a todos los que disfrutaban de su compañía. Cuando le sonreía a un hombre, ese mortal experimentaba una mayor capacidad para resolver sus múltiples problemas.

\par 
%\textsuperscript{(1875.2)}
\textsuperscript{171:7.7} Jesús amaba tanto a los hombres y de manera tan sabia, que nunca dudaba en ser severo con ellos cuando las circunstancias requerían dicha disciplina. Para ayudar a una persona, a menudo empezaba por pedirle ayuda. De esta manera suscitaba su interés, recurría a lo mejor que posee la naturaleza humana.

\par 
%\textsuperscript{(1875.3)}
\textsuperscript{171:7.8} El Maestro podía discernir la fe salvadora en la burda superstición de la mujer que buscaba la curación mediante el acto de tocar el borde de su manto. Siempre estaba preparado y dispuesto a interrumpir un sermón o a hacer esperar a una multitud mientras atendía las necesidades de una sola persona, o incluso de un niño pequeño. Sucedían grandes cosas no solamente porque la gente tenía fe en Jesús, sino también porque Jesús tenía mucha fe en ellos.

\par 
%\textsuperscript{(1875.4)}
\textsuperscript{171:7.9} La mayoría de las cosas realmente importantes que Jesús dijo o hizo parecieron suceder por casualidad, <<mientras pasaba>>. El ministerio terrenal del Maestro tuvo muy pocos aspectos profesionales, bien planeados o premeditados. Concedía la salud y sembraba la alegría con naturalidad y gentileza mientras viajaba por la vida. Era literalmente cierto que <<iba de un sitio para otro haciendo el bien>>.

\par 
%\textsuperscript{(1875.5)}
\textsuperscript{171:7.10} A los seguidores del Maestro de todos los tiempos les incumbe aprender a ayudar <<mientras pasan>> ---a hacer el bien desinteresadamente mientras se dirigen a sus obligaciones diarias.

\section*{8. La parábola de las minas}
\par 
%\textsuperscript{(1875.6)}
\textsuperscript{171:8.1} No salieron de Jericó hasta cerca del mediodía, pues la noche anterior se habían quedado levantados hasta tarde mientras Jesús enseñaba el evangelio del reino a Zaqueo y a su familia. El grupo se detuvo para almorzar casi a medio camino de la carretera que subía hasta Betania, mientras la multitud continuaba pasando hacia Jerusalén, sin saber que Jesús y los apóstoles iban a permanecer aquella noche en el Monte de los Olivos.

\par 
%\textsuperscript{(1875.7)}
\textsuperscript{171:8.2} A diferencia de la parábola de los talentos, que estaba destinada a todos los discípulos, la parábola de las minas fue contada más expresamente para los apóstoles, y estaba ampliamente basada en la experiencia de Arquelao y su inútil tentativa por conseguir el gobierno del reino de Judea. Ésta es una de las pocas parábolas del Maestro que estaba basada en un personaje histórico real. No era raro que hubieran pensado en Arquelao, ya que la casa de Zaqueo en Jericó estaba muy cerca del adornado palacio de Arquelao, y su acueducto bordeaba la carretera por la que habían salido de Jericó.

\par 
%\textsuperscript{(1875.8)}
\textsuperscript{171:8.3} Jesús dijo: <<Creéis que el Hijo del Hombre va a Jerusalén para recibir un reino, pero os aseguro que estáis destinados a sufrir una decepción. ¿No recordáis la historia de cierto príncipe que fue a un país lejano para recibir un reino? Antes incluso de que pudiera regresar, los ciudadanos de su provincia, que ya lo habían rechazado en su corazón, enviaron una embajada tras él, diciendo: `No queremos que este hombre reine sobre nosotros.' De la misma manera que la soberanía temporal de este rey fue rechazada, la soberanía espiritual del Hijo del Hombre también va a ser rechazada. Declaro de nuevo que mi reino no es de este mundo; pero si al Hijo del Hombre le hubieran concedido la soberanía espiritual de su pueblo, habría aceptado ese reino de las almas de los hombres y habría reinado sobre ese imperio de corazones humanos. A pesar de que rechazan mi soberanía espiritual sobre ellos, regresaré de nuevo para recibir de otras personas este reino del espíritu que ahora me niegan. Veréis que el Hijo del Hombre será rechazado ahora, pero en otra época, aquello que los hijos de Abraham rechazan ahora, será aceptado y exaltado>>.

\par 
%\textsuperscript{(1876.1)}
\textsuperscript{171:8.4} <<Y ahora, al igual que el noble rechazado de esta parábola, quisiera convocar ante mí a mis doce servidores, a mis administradores especiales, y entregaros a cada uno la suma de una mina. Os recomiendo a todos que prestéis mucha atención a mis instrucciones sobre cómo comerciar diligentemente con el capital que se os ha confiado durante mi ausencia, para que tengáis con qué justificar vuestra administración cuando yo regrese, cuando se os pida que rindáis cuentas>>.

\par 
%\textsuperscript{(1876.2)}
\textsuperscript{171:8.5} <<Pero aunque este Hijo rechazado no regrese, otro Hijo será enviado para recibir este reino, y entonces ese Hijo enviará a buscaros a todos para recibir el informe de vuestra administración y para regocijarse por vuestras ganancias>>.

\par 
%\textsuperscript{(1876.3)}
\textsuperscript{171:8.6} <<Cuando estos administradores fueron convocados posteriormente para rendir cuentas, el primero se adelantó, diciendo: `Señor, con tu mina he ganado diez minas más.' Y su señor le dijo: `Bien hecho; eres un buen servidor; como te has mostrado fiel en este asunto, te daré autoridad sobre diez ciudades.' El segundo vino, diciendo: `La mina que me dejaste Señor, ha producido cinco minas.' Y el señor dijo: `En consecuencia, te haré gobernante de cinco ciudades.' Y así sucesivamente con todos los demás, hasta que el último servidor fue llamado para rendir cuentas, y dijo: `Mira, Señor, he aquí tu mina que he guardado a salvo envuelta en esta servilleta. Hice esto porque tenía miedo de ti; creí que eras desrazonable, puesto que recoges allí donde no has depositado nada, y pretendes cosechar allí donde no has sembrado.' Entonces dijo su señor: `Eres un servidor negligente e infiel, y voy a juzgarte por tus propias palabras. Sabías que recojo la cosecha allí donde aparentemente no he sembrado; sabías por tanto que se te pediría esta rendición de cuentas. Sabiendo esto, al menos podrías haber entregado mi dinero al banquero, para poder recuperarlo a mi regreso con un interés adecuado.'>>

\par 
%\textsuperscript{(1876.4)}
\textsuperscript{171:8.7} <<Entonces este gobernante dijo a los que estaban allí: `Quitadle el dinero a este servidor perezoso y dadselo al que tiene diez minas.' Cuando le recordaron al señor que el primer servidor ya tenía diez minas, dijo: `A todo el que tiene se le dará más, pero al que no tiene nada, incluso lo que tiene se le quitará.'>>

\par 
%\textsuperscript{(1876.5)}
\textsuperscript{171:8.8} A continuación, los apóstoles trataron de conocer la diferencia entre el significado de esta parábola y el de la parábola anterior de los talentos, pero en respuesta a sus numerosas preguntas, Jesús se limitó a decir: <<Meditad bien estas palabras en vuestro corazón mientras cada uno descubre su verdadero significado>>.

\par 
%\textsuperscript{(1876.6)}
\textsuperscript{171:8.9} Natanael fue el que enseñó muy bien el significado de estas dos parábolas en los años posteriores, y resumió sus enseñanzas en las conclusiones siguientes:

\par 
%\textsuperscript{(1876.7)}
\textsuperscript{171:8.10} 1. La capacidad es la medida práctica de las oportunidades de la vida. Nunca seréis considerados responsables de tener que realizar algo que sobrepase vuestras capacidades.

\par 
%\textsuperscript{(1876.8)}
\textsuperscript{171:8.11} 2. La fidelidad es la medida infalible de la honradez humana. Es probable que el que es fiel en las cosas pequeñas, también mostrará fidelidad en todo lo que sea compatible con sus talentos.

\par 
%\textsuperscript{(1876.9)}
\textsuperscript{171:8.12} 3. El Maestro concede una recompensa menor por una fidelidad menor cuando las oportunidades son iguales.

\par 
%\textsuperscript{(1877.1)}
\textsuperscript{171:8.13} 4. Concede una recompensa igual por una fidelidad igual cuando las oportunidades son menores.

\par 
%\textsuperscript{(1877.2)}
\textsuperscript{171:8.14} Cuando hubieron terminado de almorzar, y después de que la multitud de seguidores hubiera continuado hacia Jerusalén, Jesús se hallaba de pie delante de los apóstoles a la sombra de una roca que sobresalía por encima del camino. Con una dignidad jovial y una graciosa majestad, señaló con el dedo hacia el oeste y dijo: <<Venid, hermanos míos, entremos en Jerusalén, para recibir allí lo que nos espera; así cumpliremos la voluntad del Padre celestial en todas las cosas>>.

\par 
%\textsuperscript{(1877.3)}
\textsuperscript{171:8.15} Y así, Jesús y sus apóstoles reanudaron este viaje, el último que hacía el Maestro a Jerusalén en la similitud de la carne del hombre mortal.


\chapter{Documento 172. La entrada en Jerusalén}
\par 
%\textsuperscript{(1878.1)}
\textsuperscript{172:0.1} JESÚS y los apóstoles llegaron a Betania poco después de las cuatro de la tarde del viernes 31 de marzo del año 30. Lázaro, sus hermanas y sus amigos los estaban esperando; en vista de que un gran número de personas venía diariamente para hablar con Lázaro sobre su resurrección, Jesús fue informado de que se había preparado todo para que se alojara con un creyente vecino, un tal Simón, el ciudadano principal de aquel pueblecito desde la muerte del padre de Lázaro.

\par 
%\textsuperscript{(1878.2)}
\textsuperscript{172:0.2} Aquella tarde, Jesús recibió a muchos visitantes, y la gente común de Betania y Betfagé hizo todo lo posible para que se sintiera bienvenido. Muchos creían que Jesús iba ahora a Jerusalén, desafiando por completo el decreto de muerte del sanedrín, para proclamarse rey de los judíos, pero la familia de Betania ---Lázaro, Marta y María--- comprendía más plenamente que el Maestro no era un rey de ese tipo; sentían vagamente que ésta podía ser su última visita a Jerusalén y Betania.

\par 
%\textsuperscript{(1878.3)}
\textsuperscript{172:0.3} Los jefes de los sacerdotes fueron informados de que Jesús estaba alojado en Betania, pero pensaron que sería mejor no intentar capturarlo entre sus amigos; decidieron esperar a que entrara en Jerusalén. Jesús sabía todo esto, pero conservaba una calma majestuosa; sus amigos nunca lo habían visto más tranquilo y agradable; incluso los apóstoles estaban sorprendidos de que estuviera tan indiferente, cuando el sanedrín había pedido a todos los judíos que se lo entregaran. Mientras el Maestro dormía aquella noche, los apóstoles estuvieron vigilando de dos en dos, y muchos de ellos se habían ceñido la espada. A la mañana siguiente temprano, fueron despertados por cientos de peregrinos que venían de Jerusalén, aunque fuera sábado, para ver a Jesús y a Lázaro, a quien había resucitado de entre los muertos.

\section*{1. El sábado en Betania}
\par 
%\textsuperscript{(1878.4)}
\textsuperscript{172:1.1} Los peregrinos que venían de fuera de Judea, así como las autoridades judías, se habían preguntado: <<¿Qué pensáis? ¿Vendrá Jesús a la fiesta?>> Por ello, la gente se alegró cuando escuchó que Jesús estaba en Betania, pero los jefes de los sacerdotes y de los fariseos estaban un poco perplejos. Se sentían contentos de tenerlo bajo su jurisdicción, pero estaban algo desconcertados por su audacia; recordaban que en su visita anterior a Betania, Lázaro había sido resucitado de entre los muertos, y Lázaro se estaba convirtiendo en un gran problema para los enemigos de Jesús.

\par 
%\textsuperscript{(1878.5)}
\textsuperscript{172:1.2} Seis días antes de la Pascua, la tarde después del sábado, todo Betania y todo Betfagé se reunió para celebrar la llegada de Jesús con un banquete público en la casa de Simón. Esta cena era en honor de Jesús y de Lázaro, y fue ofrecida desafiando al sanedrín. Marta dirigía el servicio de la comida; su hermana María se encontraba entre las espectadoras, porque era contrario a la costumbre de los judíos que una mujer se sentara en un banquete público. Los agentes del sanedrín estaban presentes, pero temían arrestar a Jesús en medio de sus amigos.

\par 
%\textsuperscript{(1879.1)}
\textsuperscript{172:1.3} Jesús conversó con Simón sobre el Josué de antaño, cuyo nombre era homónimo del suyo, y contó cómo Josué y los israelitas habían llegado a Jerusalén a través de Jericó. Al comentar la leyenda del derrumbamiento de las murallas de Jericó, Jesús dijo: <<No me ocupo de esas murallas de ladrillo y de piedra; pero quisiera que las murallas del prejuicio, de la presunción y del odio se desmoronaran delante de esta predicación del amor del Padre por todos los hombres>>.

\par 
%\textsuperscript{(1879.2)}
\textsuperscript{172:1.4} El banquete continuó de una manera muy alegre y normal, salvo que todos los apóstoles estaban más serios que de costumbre. Jesús estaba excepcionalmente alegre y había jugado con los niños hasta el momento de sentarse a la mesa.

\par 
%\textsuperscript{(1879.3)}
\textsuperscript{172:1.5} No sucedió nada extraordinario hasta cerca del final del festín, cuando María, la hermana de Lázaro, se salió del grupo de espectadoras, avanzó hasta el lugar donde Jesús estaba reclinado como huésped de honor, y se puso a abrir un gran frasco de alabastro que contenía un ung\"uento muy raro y costoso. Después de ungir la cabeza del Maestro, empezó a verterlo sobre sus pies, y luego se soltó los cabellos para secárselos con ellos. El olor del ung\"uento impregnó toda la casa, y todos los presentes se asombraron por lo que María había hecho. Lázaro no dijo nada, pero cuando alguna gente murmuró manifestando su indignación porque un ung\"uento tan caro se utilizara de esta manera, Judas Iscariote se dirigió al lugar donde Andrés estaba reclinado y dijo: <<¿Por qué no se ha vendido ese ung\"uento y se ha dado el dinero para alimentar a los pobres? Deberías decirle al Maestro que censure este derroche>>.

\par 
%\textsuperscript{(1879.4)}
\textsuperscript{172:1.6} Sabiendo lo que pensaban y escuchando lo que decían, Jesús puso su mano sobre la cabeza de María, que estaba arrodillada a su lado, y con una expresión de bondad en su rostro, dijo: <<Que cada uno de vosotros la deje en paz. ¿Por qué la molestáis con esto, ya que ha hecho una buena cosa según su corazón? A vosotros que murmuráis y decís que este ung\"uento debería haberse vendido y el dinero entregado a los pobres, dejad que os diga que a los pobres los tendréis siempre con vosotros, de manera que podréis ayudarlos en cualquier momento que os parezca bien. Pero yo no estaré siempre con vosotros; pronto iré hacia mi Padre. Esta mujer ha guardado este ung\"uento durante mucho tiempo para cuando entierren mi cuerpo; y puesto que le ha parecido bien efectuar esta unción anticipándose a mi muerte, esa satisfacción no le será denegada. Al hacer esto, María os ha reprendido a todos, en el sentido de que con este acto manifiesta su fe en lo que he dicho sobre mi muerte y ascensión hacia mi Padre que está en los cielos. Esta mujer no será recriminada por lo que ha hecho esta noche; os digo más bien que en las eras por venir, en cualquier parte del mundo que se predique este evangelio, lo que ella ha hecho se contará en memoria suya>>.

\par 
%\textsuperscript{(1879.5)}
\textsuperscript{172:1.7} A causa de esta reprimenda, tomada por una recriminación personal, Judas Iscariote se decidió finalmente a buscar venganza para sus sentimientos heridos. Muchas veces había albergado estas ideas de manera subconsciente, pero ahora se atrevía a considerar estos pensamientos perversos en su mente clara y consciente. Otras muchas personas lo animaron en esta actitud, pues el precio de este ung\"uento equivalía al salario de un hombre durante un año ---suficiente para abastecer de pan a cinco mil personas. Pero María amaba a Jesús; había adquirido este precioso ung\"uento para embalsamar su cuerpo después de muerto, pues creía en sus palabras cuando les advertía que tenía que morir; y no se le iba a privar de ello si había cambiado de idea y escogido otorgar esta ofrenda al Maestro mientras aún estaba vivo.

\par 
%\textsuperscript{(1879.6)}
\textsuperscript{172:1.8} Tanto Lázaro como Marta sabían que María había tardado mucho tiempo en ahorrar el dinero destinado a comprar este frasco de nardo, y aprobaban por completo que actuara en este asunto según los deseos de su corazón, pues eran ricos y podían permitirse fácilmente hacer esta ofrenda.

\par 
%\textsuperscript{(1880.1)}
\textsuperscript{172:1.9} Cuando los jefes de los sacerdotes tuvieron noticia de esta cena en Betania en honor de Jesús y Lázaro, empezaron a consultarse para ver lo que debían hacer con Lázaro. Decidieron enseguida que Lázaro también tenía que morir. Concluyeron, con toda la razón, que sería inútil ejecutar a Jesús si dejaban vivir a Lázaro, a quien Jesús había resucitado de entre los muertos.

\section*{2. El domingo por la mañana con los apóstoles}
\par 
%\textsuperscript{(1880.2)}
\textsuperscript{172:2.1} Aquel domingo por la mañana, en el hermoso jardín de Simón, el Maestro convocó a sus doce apóstoles a su alrededor y les dio sus instrucciones finales antes de entrar en Jerusalén. Les dijo que probablemente pronunciaría muchos discursos y enseñaría numerosas lecciones antes de volver hacia el Padre, pero aconsejó a los apóstoles que se abstuvieran de hacer cualquier trabajo público durante esta estancia para pasar la Pascua en Jerusalén. Les indicó que permanecieran cerca de él y que <<vigilaran y oraran>>. Jesús sabía que muchos de sus apóstoles y seguidores inmediatos llevaban sus espadas escondidas en aquel mismo momento, pero no hizo ninguna alusión a este hecho.

\par 
%\textsuperscript{(1880.3)}
\textsuperscript{172:2.2} Estas instrucciones matutinas abarcaron un breve repaso del ministerio de los apóstoles desde el día de su ordenación, cerca de Cafarnaúm, hasta este día en que se preparaban para entrar en Jerusalén. Los apóstoles escucharon en silencio, y no hicieron ninguna pregunta.

\par 
%\textsuperscript{(1880.4)}
\textsuperscript{172:2.3} Aquella mañana temprano, David Zebedeo había entregado a Judas los fondos obtenidos con la venta del equipo del campamento de Pella, y Judas a su vez había puesto la mayor parte de este dinero en manos de Simón, su anfitrión, para que lo guardara en lugar seguro en previsión de las necesidades de su entrada en Jerusalén.

\par 
%\textsuperscript{(1880.5)}
\textsuperscript{172:2.4} Después de la conferencia con los apóstoles, Jesús mantuvo una conversación con Lázaro y le indicó que evitara sacrificar su vida al espíritu vengativo del sanedrín. Obedeciendo esta recomendación, Lázaro huyó unos días después a Filadelfia, cuando los agentes del sanedrín enviaron a unos hombres para que lo arrestaran.

\par 
%\textsuperscript{(1880.6)}
\textsuperscript{172:2.5} En cierto modo, todos los seguidores de Jesús sentían la crisis inminente, pero la jovialidad inhabitual y el buen humor excepcional del Maestro impidieron que se dieran plenamente cuenta de la gravedad de la situación.

\section*{3. La partida hacia Jerusalén}
\par 
%\textsuperscript{(1880.7)}
\textsuperscript{172:3.1} Betania estaba a unos tres kilómetros del templo, y era la una y media de aquel domingo por la tarde cuando Jesús se preparó para salir hacia Jerusalén. Sentía un profundo afecto por Betania y su gente sencilla. Nazaret, Cafarnaúm y Jerusalén lo habían rechazado, pero Betania lo había aceptado, había creído en él. Fue en este pueblecito, en el que casi todos los hombres, mujeres y niños eran creyentes, donde Jesús escogió realizar la obra más poderosa de su donación terrenal: la resurrección de Lázaro. No resucitó a Lázaro para que los habitantes pudieran creer, sino más bien porque ya creían.

\par 
%\textsuperscript{(1880.8)}
\textsuperscript{172:3.2} Jesús había reflexionado toda la mañana sobre su entrada en Jerusalén. Hasta ese momento, siempre se había esforzado por impedir que el público lo aclamara como el Mesías, pero ahora la situación era diferente. Se estaba acercando al final de su carrera en la carne, el sanedrín había decretado su muerte, y no iba a pasar nada porque permitiera a sus discípulos que expresaran libremente sus sentimientos, tal como hubiera ocurrido si hubiera elegido hacer una entrada oficial y pública en la ciudad.

\par 
%\textsuperscript{(1881.1)}
\textsuperscript{172:3.3} Jesús no decidió efectuar esta entrada pública en Jerusalén como un último intento por hacerse con el favor popular, ni como una tentativa final para obtener el poder. Tampoco lo hizo del todo para satisfacer los anhelos humanos de sus discípulos y apóstoles. Jesús no albergaba ninguna de las ilusiones de un soñador fantasioso; sabía muy bien cuál iba a ser el desenlace de esta visita.

\par 
%\textsuperscript{(1881.2)}
\textsuperscript{172:3.4} Después de haber decidido hacer una entrada pública en Jerusalén, el Maestro se vio enfrentado a la necesidad de escoger un método apropiado para ejecutar esta resolución. Jesús reflexionó sobre las numerosas profecías, más o menos contradictorias, llamadas mesiánicas, pero sólo parecía haber una que pudiera seguir de manera apropiada. La mayoría de estas declaraciones proféticas describían a un rey, el hijo y sucesor de David, un hombre audaz y enérgico que liberaría temporalmente a todo Israel del yugo de la dominación extranjera. Pero había un pasaje en las Escrituras que a veces había sido asociado con el Mesías por parte de aquellos que más defendían el concepto espiritual de su misión; Jesús consideró que podría utilizar coherentemente este pasaje como guía para la entrada que proyectaba hacer en Jerusalén. Este escrito se encontraba en Zacarías y decía: <<Regocíjate mucho, oh hija de Sión; da gritos de júbilo, oh hija de Jerusalén. He aquí que tu rey viene hacia ti. Es justo y trae la salvación. Viene como alguien humilde, montado en un asno, en un pollino, el hijo de una burra>>.

\par 
%\textsuperscript{(1881.3)}
\textsuperscript{172:3.5} Un rey guerrero siempre entraba en una ciudad montado a caballo; un rey en misión de paz y de amistad siempre entraba montado en un asno. Jesús no quería entrar en Jerusalén a lomos de un caballo, pero estaba dispuesto a entrar pacíficamente y con buena voluntad, subido en un burro, como el Hijo del Hombre.

\par 
%\textsuperscript{(1881.4)}
\textsuperscript{172:3.6} Jesús había intentado durante mucho tiempo, mediante una enseñanza directa, inculcar a sus apóstoles y a sus discípulos que su reino no era de este mundo, que se trataba de un asunto puramente espiritual; pero no había tenido éxito en este esfuerzo. Ahora quería intentar realizar, mediante un gesto simbólico, aquello que no había conseguido hacer por medio de una enseñanza clara y personal. En consecuencia, inmediatamente después del almuerzo, Jesús llamó a Pedro y a Juan y les ordenó que fueran a Betfagé, un pueblo vecino un poco retirado de la carretera principal, a corta distancia al noroeste de Betania. Les dijo además: <<Id a Betfagé, y cuando lleguéis al cruce de los caminos, encontraréis el pollino de una burra atado allí. Desatad el pollino y traedlo con vosotros. Si alguien os pregunta por qué hacéis esto, decid simplemente: `El Maestro lo necesita.'>> Cuando los dos apóstoles fueron a Betfagé tal como el Maestro les había ordenado, encontraron al pollino atado en la calle al lado de su madre y cerca de una casa de esquina. Mientras Pedro empezó a desatar el pollino, llegó el dueño y preguntó por qué hacían eso. Cuando Pedro le contestó lo que Jesús les había ordenado, el hombre dijo: <<Si vuestro Maestro es Jesús de Galilea, el pollino está a su disposición>>. Y así regresaron llevando al pollino con ellos.

\par 
%\textsuperscript{(1881.5)}
\textsuperscript{172:3.7} Entretanto, varios cientos de peregrinos se habían reunido alrededor de Jesús y de sus apóstoles. Desde media mañana, los visitantes que pasaban camino de la Pascua se habían detenido allí. Mientras tanto, David Zebedeo y algunos de sus antiguos mensajeros decidieron dirigirse apresuradamente a Jerusalén, donde difundieron eficazmente la noticia, entre las multitudes de peregrinos que visitaban el templo, de que Jesús de Nazaret iba a hacer una entrada triunfal en la ciudad. En consecuencia, varios miles de estos visitantes acudieron en masa para saludar a este profeta, autor de prodigios, del que tanto se hablaba, y que algunos creían que era el Mesías. Esta multitud que salía de Jerusalén encontró a Jesús y al gentío que se dirigía hacia la ciudad poco después de que hubieran pasado la cima del Olivete, y hubieran empezado a descender hacia la ciudad.

\par 
%\textsuperscript{(1882.1)}
\textsuperscript{172:3.8} Cuando la procesión partió de Betania, había un gran entusiasmo en la alegre multitud de discípulos, creyentes y peregrinos visitantes, muchos de ellos procedentes de Galilea y Perea. Justo antes de partir, las doce mujeres del cuerpo femenino original, acompañadas por algunas de sus asociadas, llegaron al lugar y se unieron a esta procesión excepcional que se dirigía alegremente hacia la ciudad.

\par 
%\textsuperscript{(1882.2)}
\textsuperscript{172:3.9} Antes de partir, los gemelos Alfeo colocaron sus mantos encima del asno y lo sujetaron mientras se subía el Maestro. A medida que la procesión avanzaba hacia la cima del Olivete, la alegre multitud echaba al suelo sus prendas de vestir y traía ramas de los árboles cercanos para hacerle una alfombra de honor al asno que llevaba al Hijo real, al Mesías prometido. Mientras la multitud jubilosa continuaba avanzando hacia Jerusalén, empezaron a cantar, o más bien a gritar al unísono, el salmo: <<Hosanna al hijo de David; bendito es el que viene en nombre del Señor. Hosanna en las alturas. Bendito sea el reino que desciende del cielo>>.

\par 
%\textsuperscript{(1882.3)}
\textsuperscript{172:3.10} Jesús se mostró alegre y jovial durante el trayecto hasta que llegó a la cumbre del Olivete, desde donde se tenía una vista panorámica sobre la ciudad y las torres del templo; el Maestro detuvo allí la procesión, y un gran silencio se apoderó de todos mientras lo veían llorar. Bajando la mirada sobre la inmensa multitud que salía de la ciudad para recibirlo, el Maestro, con mucha emoción y una voz llorosa, dijo: <<¡Oh Jerusalén, si tan sólo hubieras conocido, tú también, al menos en este día tuyo, las cosas que pertenecen a tu paz, y que podrías haber tenido con tanta profusión! Pero ahora estas glorias están a punto de ocultarse a tus ojos. Estás a punto de rechazar al Hijo de la Paz y de volverle la espalda al evangelio de la salvación. Pronto vendrán los días en que tus enemigos abrirán una trinchera a tu alrededor, y te asediarán por todas partes; te destruirán por completo, de manera que no quedará piedra sobre piedra. Y todo esto te sucederá porque no has reconocido la hora de tu visita divina. Estás a punto de rechazar el regalo de Dios, y todos los hombres te rechazarán>>.

\par 
%\textsuperscript{(1882.4)}
\textsuperscript{172:3.11} Cuando hubo terminado de hablar, empezaron a descender del Olivete y pronto se reunieron con la multitud de visitantes que venía de Jerusalén ondeando ramas de palmera, gritando hosannas y expresando de otras maneras su regocijo y su buena hermandad. El Maestro no había planeado que estas multitudes salieran de Jerusalén para encontrarse con ellos; fue obra de otras personas. Nunca premeditó nada que fuera teatral.

\par 
%\textsuperscript{(1882.5)}
\textsuperscript{172:3.12} Junto con la multitud que afluía para dar la bienvenida al Maestro, también venían muchos fariseos y otros enemigos suyos. Estaban tan perturbados por esta explosión repentina e inesperada de aclamación popular, que tuvieron miedo de arrestarlo, por temor a que esta acción precipitara una revuelta abierta del pueblo. Temían enormemente la actitud de la gran cantidad de visitantes, que habían oído hablar mucho de Jesús, y gran número de los cuales creían en él.

\par 
%\textsuperscript{(1882.6)}
\textsuperscript{172:3.13} Al acercarse a Jerusalén, la multitud se volvió más expresiva, tanto que algunos fariseos se abrieron paso hasta Jesús y dijeron: <<Instructor, deberías reprender a tus discípulos y exhortarlos a que se comporten de una manera más correcta>>. Jesús respondió: <<Es muy adecuado que estos hijos den la bienvenida al Hijo de la Paz, a quien los jefes de los sacerdotes han rechazado. Sería inútil detenerlos, no sea que estas piedras al borde del camino se pongan a gritar en su lugar>>.

\par 
%\textsuperscript{(1882.7)}
\textsuperscript{172:3.14} Los fariseos se apresuraron a adelantarse a la procesión para volver al sanedrín, que entonces estaba reunido en el templo, e informaron a sus colegas: <<Mirad, todo lo que hacemos no sirve para nada; estamos confundidos por ese galileo. La gente se ha vuelto loca por él; si no detenemos a esos ignorantes, todo el mundo le seguirá>>.

\par 
%\textsuperscript{(1883.1)}
\textsuperscript{172:3.15} En realidad, no había que atribuir ningún significado profundo a esta explosión superficial y espontánea de entusiasmo popular. Esta bienvenida, aunque alegre y sincera, no representaba ninguna convicción real o profunda en el corazón de esta multitud jubilosa. Esta misma muchedumbre estuvo igualmente dispuesta a rechazar rápidamente a Jesús, más tarde aquella misma semana, en cuanto el sanedrín hubo tomado una posición firme y decidida contra él, cuando perdieron sus ilusiones ---cuando se dieron cuenta de que Jesús no iba a establecer el reino de acuerdo con sus esperanzas albergadas durante mucho tiempo.

\par 
%\textsuperscript{(1883.2)}
\textsuperscript{172:3.16} Pero toda la ciudad estaba extraordinariamente agitada, de manera que todo el mundo preguntaba: <<¿Quién es ese hombre?>> Y la multitud contestaba: <<Es Jesús de Nazaret, el profeta de Galilea>>.

\section*{4. La visita al templo}
\par 
%\textsuperscript{(1883.3)}
\textsuperscript{172:4.1} Mientras los gemelos Alfeo devolvían el asno a su dueño, Jesús y los diez apóstoles se separaron de sus asociados inmediatos y se pasearon por el templo, observando los preparativos para la Pascua. No se hizo ningún intento por molestar a Jesús, ya que el sanedrín temía mucho al pueblo, y después de todo, ésa era una de las razones por las que Jesús había permitido que la multitud lo aclamara de aquella manera. Los apóstoles apenas comprendían que éste era el único procedimiento humano que podía impedir, de manera eficaz, que Jesús fuera arrestado inmediatamente en cuanto entrara en la ciudad. El Maestro deseaba dar a los habitantes de Jerusalén, destacados y humildes, así como a las decenas de miles de visitantes para la Pascua, esta última oportunidad adicional de escuchar el evangelio y de recibir, si querían, al Hijo de la Paz.

\par 
%\textsuperscript{(1883.4)}
\textsuperscript{172:4.2} Ahora, mientras avanzaba la tarde y las multitudes iban en busca de alimento, Jesús y sus seguidores inmediatos se quedaron solos. ¡Qué día tan extraño había sido! Los apóstoles estaban pensativos, pero mudos. En todos sus años de asociación con Jesús, nunca habían visto un día como éste. Se sentaron un rato cerca del tesoro del templo, observando cómo la gente dejaba caer sus contribuciones: los ricos ponían mayores cantidades en la caja de las ofrendas, y todos daban algo según sus posibilidades. Al final llegó una pobre viuda, vestida miserablemente, y observaron que echaba dos ébolos
(pequeñas monedas de cobre) en el embudo. Entonces Jesús llamó la atención de los apóstoles sobre la viuda, diciendo: <<Retened bien lo que acabáis de ver. Esa pobre viuda ha echado más que todos los demás, porque todos los demás han echado, como don, una pequeña parte de lo que les sobraba, pero esa pobre mujer, aunque está necesitada, ha dado todo lo que tenía, incluso su sustento>>.

\par 
%\textsuperscript{(1883.5)}
\textsuperscript{172:4.3} A medida que avanzaba la tarde, caminaron en silencio por los patios del templo, y después de haber observado una vez más estas escenas familiares, Jesús recordó las emociones asociadas a sus visitas anteriores, sin excluir las primeras, y dijo: <<Subamos a Betania para descansar>>. Jesús, con Pedro y Juan, fueron a la casa de Simón, mientras que los demás apóstoles se alojaron con sus amigos de Betania y Betfagé.

\section*{5. La actitud de los apóstoles}
\par 
%\textsuperscript{(1883.6)}
\textsuperscript{172:5.1} Este domingo por la tarde, mientras regresaban a Betania, Jesús caminó delante de los apóstoles. No se dijo ni una palabra hasta que se separaron después de llegar a la casa de Simón. Nunca hubo doce seres humanos que experimentaran unas emociones tan diversas e inexplicables como las que surgían ahora en la mente y en el alma de estos embajadores del reino. Estos robustos galileos estaban confusos y desconcertados; no sabían qué esperar inmediatamente después; estaban demasiado sorprendidos como para sentirse muy asustados. No sabían nada de los planes del Maestro para el día siguiente, y no hicieron ninguna pregunta. Se fueron a sus alojamientos, aunque no durmieron mucho, a excepción de los gemelos. Pero no mantuvieron una vigilia armada alrededor de Jesús en la casa de Simón.

\par 
%\textsuperscript{(1884.1)}
\textsuperscript{172:5.2} Andrés estaba totalmente desconcertado, casi desorientado. Fue el único apóstol que no intentó evaluar seriamente la explosión popular de aclamaciones. Estaba demasiado preocupado por la idea de su responsabilidad como jefe del cuerpo apostólico, como para analizar seriamente el sentido o el significado de los ruidosos hosannas de la multitud. Andrés estaba atareado vigilando a algunos de sus compañeros, pues temía que se dejaran llevar por sus emociones durante la agitación popular, especialmente Pedro, Santiago, Juan y Simón Celotes. Durante todo este día y los que siguieron inmediatamente después, Andrés estuvo preocupado con serias dudas, pero nunca expresó ninguno de estos recelos a sus compañeros apostólicos. Le inquietaba la actitud de algunos de los doce, pues sabía que estaban armados con espadas; pero ignoraba que su propio hermano Pedro llevaba una de aquellas armas. Así pues, la procesión hacia Jerusalén sólo causó en Andrés una impresión relativamente superficial; estaba demasiado atareado con las responsabilidades de su cargo como para sentirse afectado por otras cosas.

\par 
%\textsuperscript{(1884.2)}
\textsuperscript{172:5.3} Simón Pedro se sintió al principio casi arrebatado por esta manifestación popular de entusiasmo; pero se había serenado notablemente en el momento de regresar aquella noche a Betania. Pedro simplemente no podía imaginar qué es lo que pretendía hacer el Maestro. Estaba terriblemente desilusionado porque Jesús no había aprovechado esta oleada de favor popular para hacer algún tipo de declaración. Pedro no podía comprender por qué Jesús no había hablado a la multitud cuando llegaron al templo, o al menos permitido que uno de los apóstoles se dirigiera al gentío. Pedro era un gran predicador, y le disgustaba ver cómo se desaprovechaba un auditorio tan amplio, tan receptivo y tan entusiasta. Le hubiera gustado tanto predicar el evangelio del reino a este gentío allí mismo en el templo; pero el Maestro les había encargado expresamente que no debían enseñar ni predicar en Jerusalén durante esta semana de la Pascua. La reacción a la espectacular procesión hacia la ciudad fue desastrosa para Simón Pedro; cuando llegó la noche, estaba pensativo y con una tristeza indecible.

\par 
%\textsuperscript{(1884.3)}
\textsuperscript{172:5.4} Para Santiago Zebedeo, este domingo fue un día de perplejidad y de profunda confusión; no conseguía captar el significado de lo que estaba ocurriendo; no podía comprender la intención del Maestro, que permitía estas aclamaciones desenfrenadas, y luego se negaba a decir una palabra a la gente cuando llegaron al templo. Mientras la procesión descendía del Olivete hacia Jerusalén, y más particularmente cuando se encontraron con los miles de peregrinos que salían para acoger al Maestro, Santiago se sintió cruelmente desgarrado entre sus emociones contradictorias de exaltación y satisfacción por lo que veía, y su profundo sentimiento de temor por lo que podía ocurrir cuando llegaran al templo. Luego se sintió abatido y abrumado por la decepción cuando Jesús se bajó del asno y se puso a caminar tranquilamente por los patios del templo. Santiago no podía comprender por qué se desperdiciaba una oportunidad tan magnífica para proclamar el reino. Por la noche, una angustiosa y terrible incertidumbre dominaba su mente.

\par 
%\textsuperscript{(1884.4)}
\textsuperscript{172:5.5} Juan Zebedeo estuvo a punto de comprender por qué Jesús había actuado así; al menos captó parcialmente el significado espiritual de esta supuesta entrada triunfal en Jerusalén. Mientras la multitud se dirigía hacia el templo y Juan observaba a su Maestro sentado a horcajadas en el pollino, recordó que anteriormente había escuchado a Jesús citar el pasaje de las Escrituras, la declaración de Zacarías, que describía la llegada del Mesías como un hombre de paz que entraba en Jerusalén montado en un asno. Mientras Juan le daba vueltas a esta Escritura en su cabeza, empezó a comprender el significado simbólico del espectáculo de este domingo por la tarde. Al menos captó el suficiente significado de esta Escritura como para permitirle disfrutar un poco del episodio e impedir deprimirse con exceso por el final aparentemente sin sentido de la procesión triunfal. Juan tenía un tipo de mente que tendía de manera natural a pensar y a sentir en símbolos.

\par 
%\textsuperscript{(1885.1)}
\textsuperscript{172:5.6} Felipe estaba completamente trastornado por lo inesperado y la espontaneidad de la explosión. Mientras descendían del Olivete, no pudo ordenar suficientemente sus pensamientos como para llegar a una opinión determinada sobre el significado de toda esta manifestación. En cierto modo, disfrutó del espectáculo porque su Maestro estaba siendo honrado. Cuando llegaron al templo, le inquietó la idea de que Jesús quizás pudiera pedirle que alimentara a la multitud, de manera que el comportamiento de Jesús de apartarse deliberadamente del gentío, que tan amargamente había desilusionado a la mayoría de los apóstoles, fue un gran alivio para Felipe. Las multitudes habían sido a veces una gran prueba para el administrador de los doce. Después de haberse liberado de estos temores personales referentes a las necesidades materiales del gentío, Felipe se unió a Pedro para expresar su desilusión porque no se había hecho nada por enseñar a la multitud. Aquella noche, Felipe se puso a reflexionar sobre estas experiencias, y estuvo tentado de poner en duda toda la idea del reino; se preguntaba honradamente qué podían significar todas estas cosas, pero no expresó sus dudas a nadie; amaba demasiado a Jesús como para hacer una cosa así. Tenía una gran fe personal en el Maestro.

\par 
%\textsuperscript{(1885.2)}
\textsuperscript{172:5.7} Natanael, aparte de apreciar los aspectos simbólicos y proféticos, fue el que estuvo más cerca de comprender las razones que tenía el Maestro para ganarse el apoyo popular de los peregrinos de la Pascua. Antes de llegar al templo, estuvo razonando que, sin esta entrada espectacular en Jerusalén, Jesús hubiera sido arrestado por los agentes del sanedrín y arrojado en un calabozo en cuanto se hubiera atrevido a entrar en la ciudad. Así pues, no le sorprendió en absoluto que el Maestro dejara de utilizar a la alegre multitud en cuanto se encontró dentro de los muros de la ciudad, después de haber impresionado tan poderosamente a los dirigentes judíos como para que éstos se abstuvieran de proceder a su arresto inmediato. Al comprender la verdadera razón que tenía el Maestro para entrar en la ciudad de esta manera, Natanael siguió adelante con naturalidad y con más equilibrio, y se sintió menos perturbado y desilusionado que los otros apóstoles por la conducta posterior de Jesús. Natanael tenía una gran confianza en la aptitud de Jesús para comprender a los hombres, así como en su sagacidad y destreza para manejar las situaciones difíciles.

\par 
%\textsuperscript{(1885.3)}
\textsuperscript{172:5.8} Mateo se sintió al principio confundido por esta manifestación espectacular. No captó el significado de lo que veían sus ojos hasta que se acordó también del escrito de Zacarías, en el que el profeta aludía al regocijo de Jerusalén porque había llegado su rey trayendo la salvación y montado en el pollino de una burra. Mientras la procesión avanzaba en dirección a la ciudad y luego se dirigía hacia el templo, Mateo se quedó extasiado; estaba seguro de que algo extraordinario iba a suceder cuando el Maestro llegara al templo a la cabeza de esta multitud que lo aclamaba. Cuando uno de los fariseos se mofó de Jesús, diciendo: <<¡Mirad todos, mirad quién viene aquí: el rey de los judíos montado en un asno!>>, Mateo tuvo que hacer un gran esfuerzo para no ponerle las manos encima. Aquel atardecer, ninguno de los doce estaba más deprimido que él durante el camino de vuelta a Betania. Después de Simón Pedro y Simón Celotes, Mateo fue quien experimentó la mayor tensión nerviosa y por la noche estaba agotado. Pero por la mañana ya estaba mucho más animado; después de todo, era un buen perdedor.

\par 
%\textsuperscript{(1886.1)}
\textsuperscript{172:5.9} Tomás fue el hombre más desconcertado y confundido de los doce. La mayor parte del tiempo se limitó a seguir a los demás, contemplando el espectáculo y preguntándose honradamente cuál podía ser el motivo del Maestro para participar en una manifestación tan peculiar. En lo más profundo de su corazón, consideraba toda esta representación como un poco infantil, si no absolutamente disparatada. Nunca había visto a Jesús hacer una cosa semejante, y no sabía cómo explicar su extraña conducta de este domingo por la tarde. Cuando llegaron al templo, Tomás había deducido que la finalidad de esta demostración popular era asustar de tal manera al sanedrín que no se atrevieran a arrestar inmediatamente al Maestro. En el camino de vuelta a Betania, Tomás reflexionó mucho, pero no dijo nada. En el momento de acostarse, la habilidad del Maestro para organizar esta entrada tumultuosa en Jerusalén había empezado a despertar un poco su sentido del humor, y se sintió muy animado por esta reacción.

\par 
%\textsuperscript{(1886.2)}
\textsuperscript{172:5.10} Este domingo empezó siendo un gran día para Simón Celotes. Imaginaba las cosas maravillosas que se harían en Jerusalén los próximos días, y en esto tenía razón, pero Simón soñaba con el establecimiento de la nueva soberanía nacional de los judíos, con Jesús sentado en el trono de David. Simón veía a los nacionalistas entrar en acción en cuanto se anunciara el reino, y se veía a sí mismo al mando supremo de las fuerzas militares, en vías de congregarse, del nuevo reino. Durante el descenso del Olivete, llegó incluso a imaginar que el sanedrín y todos sus partidarios estarían muertos antes de que el Sol se pusiera aquel día. Creía realmente que algo extraordinario iba a suceder. Era el hombre más ruidoso de toda la multitud. Pero a las cinco de la tarde, era un apóstol silencioso, abatido y desilusionado. Nunca se recuperó por completo de la depresión que se apoderó de él a consecuencia de la conmoción de este día; al menos, no hasta mucho tiempo después de la resurrección del Maestro.

\par 
%\textsuperscript{(1886.3)}
\textsuperscript{172:5.11} Para los gemelos Alfeo, éste fue un día perfecto. Lo disfrutaron realmente hasta el fin, y como no estuvieron presentes durante la tranquila visita al templo, se libraron en gran parte de la decepción que siguió a la agitación popular. No podían comprender de ninguna manera el comportamiento abatido de los apóstoles cuando regresaban a Betania aquella noche. En la memoria de los gemelos, éste fue siempre el día en que se sintieron más cerca del cielo en la Tierra. Este día fue la culminación satisfactoria de toda su carrera como apóstoles. El recuerdo de la euforia de este domingo por la tarde los sostuvo durante toda la tragedia de esta semana memorable, hasta el mismo momento de la crucifixión. Fue la entrada real más apropiada que los gemelos podían imaginar; disfrutaron cada momento del espectáculo. Aprobaron plenamente todo lo que vieron y conservaron el recuerdo durante mucho tiempo.

\par 
%\textsuperscript{(1886.4)}
\textsuperscript{172:5.12} De todos los apóstoles, Judas Iscariote fue el que estuvo más desfavorablemente afectado por esta entrada procesional en Jerusalén. Su mente estaba desagradablemente agitada porque el Maestro le había reprendido el día anterior a causa de la unción de María durante la fiesta en casa de Simón. Judas estaba disgustado con todo el espectáculo. Le parecía infantil, si no francamente ridículo. Mientras este apóstol vengativo contemplaba los acontecimientos de este domingo por la tarde, le daba la impresión de que Jesús se parecía más a un payaso que a un rey. Le molestaba enormemente todo el espectáculo. Compartía el punto de vista de los griegos y de los romanos, que despreciaban a todo el que consintiera en montarse en un asno o en el pollino de una burra. Cuando la procesión triunfal hubo entrado en la ciudad, Judas casi había decidido abandonar toda idea de un reino semejante; estaba casi resuelto a renunciar a todas estas tentativas absurdas para establecer el reino de los cielos. Luego se acordó de la resurrección de Lázaro y de otras muchas cosas, y decidió permanecer con los doce, al menos un día más. Además, llevaba la bolsa, y no quería desertar con los fondos apostólicos en su poder. Aquella noche, durante el camino de vuelta a Betania, su conducta no pareció extraña puesto que todos los apóstoles estaban igualmente deprimidos y silenciosos.

\par 
%\textsuperscript{(1887.1)}
\textsuperscript{172:5.13} Judas se dejó influir enormemente por las burlas de sus amigos saduceos. En su determinación final de abandonar a Jesús y a sus compañeros apóstoles, ningún otro factor ejerció una influencia tan poderosa sobre él como cierto episodio que se produjo en el preciso momento en que Jesús llegaba a la puerta de la ciudad: Un distinguido saduceo (amigo de la familia de Judas) se precipitó hacia éste con el ánimo de burlarse jovialmente de él, le dio una palmada en la espalda, y le dijo: <<¿Por qué tienes tan mala cara, mi buen amigo? Anímate y únete a todos nosotros para aclamar a ese Jesús de Nazaret, el rey de los judíos, mientras atraviesa las puertas de Jerusalén montado en un burro>>. Judas nunca había retrocedido ante las persecuciones, pero no podía soportar este tipo de burlas. A su sentimiento de venganza, alimentado durante largo tiempo, se sumaba ahora este miedo mortal al ridículo, este sentimiento terrible y espantoso de sentir verg\"uenza de su Maestro y de sus compañeros apóstoles. En su corazón, este embajador ordenado del reino ya era un desertor; sólo le quedaba encontrar una excusa plausible para romper abiertamente con el Maestro.


\chapter{Documento 173. El lunes en Jerusalén}
\par 
%\textsuperscript{(1888.1)}
\textsuperscript{173:0.1} TAL COMO habían planeado de antemano, este lunes por la mañana temprano Jesús y los apóstoles se reunieron en la casa de Simón en Betania y, después de una breve conferencia, partieron para Jerusalén. Los doce estaban extrañamente silenciosos mientras se dirigían hacia el templo; no se habían recuperado de la experiencia del día anterior. Estaban expectantes, temerosos y profundamente afectados por cierto sentimiento de distanciamiento que tenía su origen en el repentino cambio de táctica del Maestro, unido a sus instrucciones de que no debían efectuar ningún tipo de enseñanza pública durante toda esta semana de la Pascua.

\par 
%\textsuperscript{(1888.2)}
\textsuperscript{173:0.2} Mientras este grupo descendía del Monte de los Olivos, Jesús iba delante y los apóstoles le seguían de cerca en un silencio meditativo. Un sólo pensamiento predominaba en la mente de todos, salvo en Judas Iscariote, y era el siguiente: ¿Qué hará hoy el Maestro? El único pensamiento que absorbía a Judas era: ¿Qué voy a hacer? ¿Voy a continuar con Jesús y mis compañeros, o voy a retirarme? Y si los dejo, ¿cómo voy a romper?

\par 
%\textsuperscript{(1888.3)}
\textsuperscript{173:0.3} Eran cerca de las nueve de esta hermosa mañana cuando estos hombres llegaron al templo. Se dirigieron enseguida al gran patio donde Jesús enseñaba con tanta frecuencia, y después de saludar a los creyentes que lo estaban esperando, Jesús se subió a uno de los estrados para educadores y empezó a hablarle a la multitud que se estaba congregando. Los apóstoles se apartaron a corta distancia y esperaron los acontecimientos.

\section*{1. La depuración del templo}
\par 
%\textsuperscript{(1888.4)}
\textsuperscript{173:1.1} Un inmenso tráfico comercial se había desarrollado en asociación con los servicios y las ceremonias de culto en el templo. Existía el comercio de suministrar los animales apropiados para los diversos sacrificios. Aunque estaba permitido que los fieles aportaran sus propias ofrendas, persistía el hecho de que los animales debían estar libres de todo <<defecto>> en el sentido de la ley levítica, según la interpretaban los inspectores oficiales del templo. Muchos fieles habían sufrido la humillación de ver cómo los examinadores del templo rechazaban su animal supuestamente perfecto. Por esta razón, se había generalizado la práctica de adquirir los animales propiciatorios en el mismo templo, y aunque había diversos lugares cerca del Olivete donde se podían comprar, se había puesto de moda comprar estos animales directamente en los corrales del templo. Esta costumbre de vender todo tipo de animales propiciatorios en los patios del templo se había desarrollado gradualmente. Así había surgido a la existencia un importante comercio que reportaba unos beneficios enormes. Una parte de estas ganancias estaba reservada para el tesoro del templo, pero la mayoría iba a parar indirectamente a las manos de las familias de los altos sacerdotes en el poder.

\par 
%\textsuperscript{(1888.5)}
\textsuperscript{173:1.2} Esta venta de animales en el templo prosperó porque cuando un fiel compraba un animal, aunque el precio fuera un poco alto, ya no tenía que pagar ningún tributo más, y podía estar seguro de que el sacrificio propuesto no sería rechazado con el pretexto de que el animal tenía defectos reales o imaginarios. De vez en cuando, los precios se recargaban de una manera exorbitante a la gente del pueblo, en particular durante las grandes fiestas nacionales. En un momento dado, los codiciosos sacerdotes llegaron a exigir el equivalente de una semana de trabajo por un par de palomas que deberían haberse vendido a los pobres por unos pocos céntimos. Los <<hijos de Anás>> ya habían empezado a instalar sus bazares en los recintos del templo, unos mercados de géneros que sobrevivieron hasta que fueron finalmente derribados por una muchedumbre tres años antes de la destrucción del templo mismo.

\par 
%\textsuperscript{(1889.1)}
\textsuperscript{173:1.3} Pero el tráfico de animales propiciatorios y de otras mercancías no era la única manera en que se profanaban los patios del templo. En esta época se había fomentado un amplio sistema de intercambio bancario y comercial, que se realizaba directamente dentro de los recintos del templo. Todo esto había sucedido de la manera siguiente: Durante la dinastía de los Asmoneos, los judíos acuñaron su propia moneda de plata, y se había establecido la práctica de exigir que el tributo de medio siclo, y todos los demás derechos del templo, se pagaran con esta moneda judía. Esta reglamentación hacía necesario autorizar a unos cambistas para que intercambiaran este siclo ortodoxo de acuñación judía por los numerosos tipos de monedas que circulaban en toda Palestina y en otras provincias del imperio romano. El impuesto del templo por persona, pagadero por todo el mundo a excepción de las mujeres, los esclavos y los menores, era de medio siclo, una moneda de casi dos centímetros de diámetro, pero bastante gruesa. En los tiempos de Jesús, los sacerdotes también estaban exentos de pagar los impuestos del templo. En consecuencia, entre los días 15 y 25 del mes anterior a la Pascua, los cambistas acreditados instalaban sus puestos en las principales ciudades de Palestina, con el fin de proporcionar a los judíos la moneda apropiada para pagar los impuestos del templo cuando llegaran a Jerusalén. Después de este período de diez días, estos cambistas se trasladaban a Jerusalén y montaban sus mostradores de cambio en los patios del templo. Estaban autorizados a cobrar una comisión equivalente a tres o cuatro céntimos por el cambio de una moneda valorada en unos diez céntimos, y en el caso de que se deseara cambiar una moneda de mayor valor, tenían permiso para cobrar el doble. Estos banqueros del templo también se lucraban cambiando todo el dinero destinado a comprar los animales propiciatorios y a pagar los votos y las ofrendas.

\par 
%\textsuperscript{(1889.2)}
\textsuperscript{173:1.4} Estos cambistas del templo no sólo dirigían un negocio regular de banca para obtener beneficios con el intercambio de más de veinte tipos de monedas que los peregrinos visitantes traían periódicamente a Jerusalén, sino que también se dedicaban a todas las otras clases de operaciones relacionadas con el oficio de banquero. Tanto el tesoro del templo como los jefes del mismo obtenían unos beneficios enormes con estas actividades comerciales. No era raro que el tesoro del templo contuviera más de diez millones de dólares (de 1935), mientras que la gente común y corriente languidecía en la miseria y continuaba pagando estas recaudaciones injustas.

\par 
%\textsuperscript{(1889.3)}
\textsuperscript{173:1.5} Este lunes por la mañana, Jesús intentó enseñar el evangelio del reino celestial en medio de esta multitud ruidosa de cambistas, mercaderes y vendedores de ganado. No era el único que se sentía molesto por esta profanación del templo; la gente corriente, y en especial los visitantes judíos de las provincias extranjeras, también se sentían completamente contrariados por esta profanación especulativa de su templo nacional de culto. En esta época, el mismo sanedrín celebraba sus reuniones regulares en una sala que estaba rodeada por todo este murmullo y confusión del comercio y del trueque.

\par 
%\textsuperscript{(1890.1)}
\textsuperscript{173:1.6} Cuando Jesús estaba a punto de empezar su alocución, se produjeron dos incidentes que atrajeron su atención. En el mostrador de un cambista cercano había surgido una discusión violenta y acalorada porque al parecer se le había cobrado con exceso a un judío de Alejandría, y en el mismo momento, el aire se desgarró con los mugidos de una manada de unos cien bueyes que estaban siendo conducidos de una sección de los corrales a otra. Mientras Jesús se detenía, contemplando de manera silenciosa pero meditativa esta escena de comercio y de confusión, observó cerca de él a un galileo sencillo, un hombre con quien había hablado una vez en Irón, que estaba siendo ridiculizado y empujado por unos judeos arrogantes que se consideraban superiores. Todo esto se combinó para que se produjera en el alma de Jesús uno de esos extraños arrebatos periódicos de indignada emoción.

\par 
%\textsuperscript{(1890.2)}
\textsuperscript{173:1.7} Ante el asombro de sus apóstoles, que estaban allí cerca y que se abstuvieron de participar en lo que siguió a continuación, Jesús bajó del estrado de los instructores, se dirigió hacia el muchacho que conducía el ganado a través del patio, le quitó el látigo de cuerdas y sacó rápidamente a los animales del templo. Pero esto no fue todo. Ante la mirada asombrada de las miles de personas reunidas en el patio del templo, se dirigió a grandes zancadas majestuosas hacia el corral más alejado, y se puso a abrir las puertas de cada establo y a expulsar a los animales encerrados. Para entonces los peregrinos reunidos se habían entusiasmado, y con un griterío tumultuoso se dirigieron a los bazares y empezaron a volcar las mesas de los cambistas. En menos de cinco minutos, todo comercio había sido barrido del templo. En el momento en que los guardias romanos cercanos aparecieron en escena, todo estaba tranquilo y las multitudes habían recuperado la calma. Jesús regresó a la tribuna de los oradores, y dijo a la multitud: <<Hoy habéis presenciado lo que está escrito en las Escrituras: `Mi casa será llamada una casa de oración para todas las naciones, pero habéis hecho de ella una cueva de ladrones.'>>

\par 
%\textsuperscript{(1890.3)}
\textsuperscript{173:1.8} Antes de que pudiera decir una palabra más, la gran asamblea estalló en hosannas de alabanza, y un gran grupo de jóvenes salió enseguida de la multitud para cantar himnos de gratitud porque los mercaderes profanos y usureros habían sido echados del templo sagrado. Mientras tanto, algunos sacerdotes habían llegado al lugar, y uno de ellos dijo a Jesús: <<¿No oyes lo que dicen los hijos de los levitas?>> Y el Maestro respondió: <<¿No has leído nunca que `la alabanza ha salido perfecta de la boca de los niños y de los lactantes?'>> Durante todo el resto del día, mientras Jesús estuvo enseñando, unos guardianes establecidos por el pueblo estuvieron vigilando todos los arcos de entrada, y no permitieron que nadie transportara ni siquiera una vasija vacía a través de los patios del templo.

\par 
%\textsuperscript{(1890.4)}
\textsuperscript{173:1.9} Cuando los principales sacerdotes y los escribas se enteraron de estos acontecimientos, se quedaron sin habla. Tenían mucho más miedo del Maestro, y estaban aún más decididos a destruirlo. Pero estaban confundidos. No sabían cómo disponer su muerte, porque tenían mucho miedo de las multitudes, que ahora expresaban tan abiertamente su aprobación por la expulsión de los especuladores profanos. Durante todo este día, un día tranquilo y pacífico en los patios del templo, el pueblo escuchó la enseñanza de Jesús y estuvo literalmente pendiente de sus palabras.

\par 
%\textsuperscript{(1890.5)}
\textsuperscript{173:1.10} Este acto sorprendente de Jesús sobrepasaba la comprensión de sus apóstoles. Estaban tan desconcertados por esta acción repentina e inesperada de su Maestro, que durante todo el episodio permanecieron agrupados cerca de la tribuna de los oradores; no levantaron ni un dedo para ayudar a esta depuración del templo. Si este acontecimiento espectacular hubiera ocurrido el día anterior, cuando Jesús llegó triunfalmente al templo al final de la tumultuosa procesión a través de las puertas de la ciudad, todo el tiempo aclamado ruidosamente por la multitud, hubieran estado dispuestos a actuar; pero dada la manera en que se desarrollaron las cosas, no estaban preparados en absoluto para participar.

\par 
%\textsuperscript{(1891.1)}
\textsuperscript{173:1.11} Esta depuración del templo revela la actitud del Maestro hacia la comercialización de las prácticas religiosas, así como su abominación por todas las formas de injusticia y de especulación a expensas de los pobres y de los ignorantes. Este episodio demuestra también que Jesús no aprobaba que se rehusara emplear la fuerza para proteger a la mayoría de un grupo humano determinado contra las prácticas desleales y esclavizantes de unas minorías injustas que pudieran parapetarse detrás del poder político, financiero o eclesiástico. No se debe permitir que los hombres astutos, perversos e insidiosos se organicen para explotar y oprimir a aquellos que, a causa de su idealismo, no están dispuestos a recurrir a la violencia para protegerse o para promover sus proyectos de vida dignos de alabanza.

\section*{2. El desafío a la autoridad del Maestro}
\par 
%\textsuperscript{(1891.2)}
\textsuperscript{173:2.1} El domingo, la entrada triunfal de Jesús en Jerusalén intimidó tanto a los dirigentes judíos que se abstuvieron de arrestarlo. Hoy lunes, esta depuración espectacular del templo también retrasó eficazmente la captura del Maestro. Día tras día, los jefes de los judíos estaban más decididos a destruirlo, pero se sentían aturdidos por dos temores, que se conjugaban para demorar la hora de asestar el golpe. Los principales sacerdotes y los escribas eran reacios a arrestar a Jesús en público, por miedo a que la multitud se revolviera contra ellos con un furioso resentimiento; también temían la posibilidad de tener que llamar a los guardias romanos para sofocar una revuelta popular.

\par 
%\textsuperscript{(1891.3)}
\textsuperscript{173:2.2} En su sesión del mediodía, el sanedrín acordó por unanimidad, ya que ningún amigo del Maestro había asistido a esta reunión, que Jesús debía ser destruido rápidamente. Pero no pudieron ponerse de acuerdo en cuanto al momento y a la manera en que debía ser arrestado. Finalmente acordaron designar a cinco grupos para que salieran a mezclarse entre la gente, e intentaran enredarlo en sus enseñanzas o desacreditarlo de otras maneras a los ojos de los que escuchaban su instrucción. En consecuencia, a eso de las dos, cuando Jesús acababa de empezar su discurso sobre <<La libertad de la filiación>>, un grupo de estos ancianos de Israel se abrió paso hasta llegar cerca de él, lo interrumpieron como de costumbre, y le hicieron esta pregunta: ¿<<Con qué autoridad haces estas cosas? ¿Quién te ha dado esa autoridad?>>

\par 
%\textsuperscript{(1891.4)}
\textsuperscript{173:2.3} Era completamente correcto que los dirigentes del templo y los funcionarios del sanedrín judío hicieran esta pregunta a cualquiera que se atreviera a enseñar y a actuar de la manera extraordinaria característica de Jesús, especialmente en lo referente a su reciente conducta de eliminar todo comercio del templo. Todos estos mercaderes y cambistas operaban con una licencia otorgada directamente por los dirigentes más elevados, y se suponía que un porcentaje de sus ganancias iba directamente al tesoro del templo. No olvidéis que la \textit{autoridad} era la contraseña de toda la sociedad judía. Los profetas siempre provocaban problemas porque tenían la audacia de atreverse a enseñar sin autoridad, sin haber sido debidamente instruidos en las academias rabínicas, ni haber recibido después la ordenación regular del sanedrín. La carencia de esta autoridad para enseñar pretenciosamente en público se consideraba como indicación de una arrogancia ignorante o de una rebelión abierta. En esta época, sólo el sanedrín podía ordenar a un anciano o a un instructor, y la ceremonia debía tener lugar en presencia de al menos tres personas que hubieran sido previamente ordenadas de la misma manera. Esta ordenación confería al educador el título de <<rabino>>, y también lo capacitaba para actuar como juez, <<atando y desatando aquellas cuestiones que le fueran presentadas para que emitiera su fallo>>.

\par 
%\textsuperscript{(1892.1)}
\textsuperscript{173:2.4} Los dirigentes del templo se presentaron ante Jesús a esta hora de la tarde desafiando no solamente su enseñanza, sino sus actos. Jesús sabía muy bien que estos mismos hombres habían afirmado públicamente durante mucho tiempo que su autoridad para enseñar era satánica, y que todas sus obras poderosas habían sido realizadas por el poder del príncipe de los demonios. Por consiguiente, el Maestro empezó su respuesta a aquella pregunta haciéndoles otra pregunta. Jesús dijo: <<Me gustaría también haceros una pregunta, y si me la contestáis, os diré igualmente con qué autoridad hago estas obras. ¿De dónde venía el bautismo de Juan? ¿Recibió Juan su autoridad del cielo o de los hombres?>>

\par 
%\textsuperscript{(1892.2)}
\textsuperscript{173:2.5} Cuando sus interrogadores escucharon esto, se apartaron a un lado para consultarse entre ellos acerca de la respuesta que debían dar. Habían pensado en desconcertar a Jesús delante de la multitud, pero ahora eran ellos los que se encontraban bastante confundidos ante todos los que estaban congregados en ese momento en el patio del templo. Y su desconcierto fue aun más evidente cuando regresaron ante Jesús, diciendo: <<Respecto al bautismo de Juan, no podemos responder; no sabemos>>. Contestaron de esta manera al Maestro porque habían razonado entre ellos: Si decimos que viene del cielo, entonces Jesús dirá: `¿Por qué no creísteis en él?', y quizás añada que su autoridad la ha recibido de Juan. Y si decimos que viene de los hombres, entonces la multitud podría revolverse contra nosotros, porque la mayoría piensa que Juan era un profeta. Y así se vieron obligados a presentarse ante Jesús y la gente para confesar que ellos, los educadores y dirigentes religiosos de Israel, no podían (o no querían) expresar una opinión sobre la misión de Juan. Cuando terminaron de hablar, Jesús bajó la mirada hacia ellos, y dijo: <<Yo tampoco os diré con qué autoridad hago estas cosas>>.

\par 
%\textsuperscript{(1892.3)}
\textsuperscript{173:2.6} Jesús nunca tuvo la intención de recurrir a Juan para respaldar su autoridad. El sanedrín nunca había ordenado a Juan. La autoridad de Jesús residía en él mismo y en la supremacía eterna de su Padre.

\par 
%\textsuperscript{(1892.4)}
\textsuperscript{173:2.7} Al emplear esta manera de comportarse con sus adversarios, Jesús no pretendía eludir la pregunta. A primera vista podría parecer que era culpable de responder con una evasiva magistral, pero no era así. Jesús nunca estaba dispuesto a aprovecharse injustamente de nadie, ni siquiera de sus enemigos. Con esta aparente evasiva, en realidad proporcionó a todos sus oyentes la respuesta a la pregunta de los fariseos sobre la autoridad que había detrás de su misión. Ellos habían afirmado que él actuaba con la autoridad del príncipe de los demonios. Jesús había repetido muchas veces que todas sus enseñanzas y obras las realizaba con el poder y la autoridad de su Padre que está en los cielos. Los dirigentes judíos se negaban a aceptar esto, y trataban de acorralarlo para que admitiera que era un educador irregular, puesto que nunca había sido autorizado por el sanedrín. Al contestarles como lo hizo, sin pretender que su autoridad viniera de Juan, satisfizo también a la gente con la conclusión de que el esfuerzo de sus enemigos por hacerlo caer en una trampa recayó eficazmente sobre ellos y los desacreditó considerablemente a los ojos de todos los presentes.

\par 
%\textsuperscript{(1892.5)}
\textsuperscript{173:2.8} Este talento que tenía el Maestro para tratar a sus adversarios era lo que tanto les asustaba de él. Aquel día ya no intentaron hacer más preguntas, y se retiraron para consultarse de nuevo entre ellos. Pero la gente no tardó en discernir la falta de honradez y de sinceridad que había en estas preguntas realizadas por los dirigentes judíos. Incluso la gente común no podía dejar de diferenciar entre la majestad moral del Maestro y la hipocresía insidiosa de sus enemigos. Pero la depuración del templo había llevado a los saduceos a unirse con los fariseos para perfeccionar los planes destinados a destruir a Jesús. Y los saduceos representaban ahora la mayoría del sanedrín.

\section*{3. La parábola de los dos hijos}
\par 
%\textsuperscript{(1893.1)}
\textsuperscript{173:3.1} Mientras los críticos fariseos permanecían allí en silencio delante de Jesús, éste bajó la mirada hacia ellos y dijo: <<Puesto que dudáis de la misión de Juan y sois hostiles a la enseñanza y a las obras del Hijo del Hombre, prestad oído a la parábola que os voy a contar: Un gran terrateniente respetado tenía dos hijos, y como deseaba la ayuda de sus hijos para administrar sus grandes posesiones, fue a ver a uno de ellos, diciendo: `Hijo, ve hoy a trabajar a mi viñedo.' Este hijo irreflexivo le contestó a su padre: `No voy a ir', pero luego se arrepintió, y fue. Cuando encontró a su hijo mayor, le dijo igualmente: `Hijo, ve a trabajar a mi viñedo.' Y este hijo hipócrita e infiel le contestó: `Sí, padre mío, voy a ir.' Pero cuando su padre se marchó, no fue. Permitidme que os pregunte, ¿cuál de estos hijos hizo realmente la voluntad de su padre?>>

\par 
%\textsuperscript{(1893.2)}
\textsuperscript{173:3.2} Y la gente respondió al unísono, diciendo: <<El primer hijo>>. Entonces Jesús dijo: <<Así es; y ahora os afirmo que los publicanos y las prostitutas, aunque parezcan rechazar la llamada al arrepentimiento, verán el error de su estilo de vida y entrarán en el reino de Dios antes que vosotros, que hacéis grandes ostentaciones de servir al Padre que está en los cielos, mientras os negáis a hacer las obras del Padre. No habéis sido vosotros, los fariseos y los escribas, los que habéis creído en Juan, sino más bien los publicanos y los pecadores; tampoco creéis en mi enseñanza, pero la gente corriente escucha mis palabras con mucho gusto>>.

\par 
%\textsuperscript{(1893.3)}
\textsuperscript{173:3.3} Jesús no despreciaba personalmente a los fariseos ni a los saduceos. Lo que trataba de desacreditar era sus sistemas de enseñanza y de prácticas. No sentía hostilidad hacia nadie, pero aquí se estaba produciendo la colisión inevitable entre una religión del espíritu nueva y viviente, y la antigua religión de las ceremonias, la tradición y la autoridad.

\par 
%\textsuperscript{(1893.4)}
\textsuperscript{173:3.4} Los doce apóstoles permanecieron todo este tiempo cerca del Maestro, pero no participaron en absoluto en estas acciones. Cada uno de los doce reaccionaba según su propia manera particular ante los acontecimientos de estos últimos días del ministerio de Jesús en la carne, y cada uno obedecía igualmente el mandato del Maestro de abstenerse de toda enseñanza y de toda predicación en público durante esta semana de la Pascua.

\section*{4. La parábola del propietario ausente}
\par 
%\textsuperscript{(1893.5)}
\textsuperscript{173:4.1} Cuando los principales fariseos y los escribas que habían intentado enredar a Jesús con sus preguntas hubieron terminado de escuchar la historia de los dos hijos, se retiraron para consultarse de nuevo. El Maestro volvió su atención hacia la atenta multitud, y contó otra parábola:

\par 
%\textsuperscript{(1893.6)}
\textsuperscript{173:4.2} <<Había un hombre de bien que poseía una propiedad, y plantó una viña. La rodeó de un seto, cavó un hoyo para el lagar y construyó una torre para los guardas. Luego alquiló esta viña a unos arrendatarios y partió para un largo viaje a otro país. Cuando se acercó la temporada de los frutos, envió a unos servidores a los arrendatarios para que cobraran su alquiler. Pero los arrendatarios se consultaron entre ellos y se negaron a entregar a estos servidores los frutos que le debían al señor; en lugar de eso, atacaron a los sirvientes, golpearon a uno, lapidaron a otro, y despidieron a los demás con las manos vacías. Cuando el propietario se enteró de todo esto, envió a otros servidores de más confianza para que trataran con estos malvados arrendatarios, pero éstos hirieron a los nuevos sirvientes y los trataron de una manera vergonzosa. Entonces el señor envió a su servidor favorito, a su administrador, y los arrendatarios lo mataron. Sin embargo, con paciencia e indulgencia, el propietario envió a otros muchos servidores, pero no quisieron recibir a ninguno. A unos los golpearon y a otros los mataron. Cuando el propietario se sintió tratado de esta manera, decidió enviar a su hijo para que tratara con aquellos arrendatarios ingratos, diciéndose: `Pueden maltratar a mis servidores, pero seguramente mostrarán respeto por mi amado hijo.' Pero cuando aquellos arrendatarios malvados e impenitentes vieron venir al hijo, razonaron entre ellos: `Éste es el heredero; vamos a matarlo y entonces la herencia será nuestra.' Así pues lo agarraron, y después de echarlo fuera de la viña, lo mataron. Cuando el dueño de esta viña se entere de que han rechazado y matado a su hijo, ¿qué hará con aquellos arrendatarios ingratos y perversos?>>

\par 
%\textsuperscript{(1894.1)}
\textsuperscript{173:4.3} Cuando la gente escuchó esta parábola y la pregunta que Jesús había hecho, contestaron: <<Destruirá a esos miserables y alquilará su viña a otros arrendatarios honrados, que le entregarán los frutos a su debido tiempo>>. Algunos de los oyentes percibieron que esta parábola se refería a la nación judía, a la manera en que había tratado a los profetas y al rechazo inminente de Jesús y del evangelio del reino; entonces dijeron con tristeza: <<Quiera Dios que no sigamos haciendo estas cosas>>.

\par 
%\textsuperscript{(1894.2)}
\textsuperscript{173:4.4} Jesús vio que un grupo de saduceos y fariseos se abría paso a través del gentío, y se calló un momento hasta que se acercaron a él; entonces dijo: <<Sabéis cómo vuestros padres rechazaron a los profetas, y sabéis muy bien que habéis decidido en vuestro corazón rechazar al Hijo del Hombre>>. Luego, mirando con una mirada escrutadora a los sacerdotes y a los ancianos que estaban cerca de él, Jesús dijo: <<¿No habéis leído nunca en las Escrituras acerca de la piedra que rechazaron los constructores, y que se convirtió en la piedra angular cuando el pueblo la descubrió? Por eso, os advierto una vez más que si continuáis rechazando este evangelio, el reino de Dios será pronto apartado de vosotros, y se entregará a un pueblo dispuesto a recibir la buena nueva y a producir los frutos del espíritu. Esta piedra contiene un misterio, pues el que cae sobre ella, aunque se rompa en pedazos por su causa, se salvará; pero aquel sobre quien caiga esta piedra se convertirá en polvo, y sus cenizas se dispersarán a los cuatro vientos>>.

\par 
%\textsuperscript{(1894.3)}
\textsuperscript{173:4.5} Cuando los fariseos escucharon estas palabras, comprendieron que Jesús se refería a ellos y a los demás dirigentes judíos. Tenían enormes deseos de agarrarlo en aquel mismo momento, pero tenían miedo de la multitud. Sin embargo, estaban tan irritados por las palabras del Maestro que se retiraron para consultarse de nuevo entre ellos acerca de cómo provocar su muerte. Aquella noche, tanto los saduceos como los fariseos se unieron para planear la manera de hacerlo caer en una trampa al día siguiente.

\section*{5. La parábola del banquete de boda}
\par 
%\textsuperscript{(1894.4)}
\textsuperscript{173:5.1} Después de que los escribas y los dirigentes se hubieron retirado, Jesús se dirigió de nuevo a la multitud reunida y contó la parábola del banquete de boda. Dijo:

\par 
%\textsuperscript{(1894.5)}
\textsuperscript{173:5.2} <<El reino de los cielos se puede comparar con un rey que preparó un banquete de boda para su hijo, y envió a unos mensajeros para que llamaran a los que habían sido previamente invitados a venir a la fiesta, diciendo: `Todo está preparado para la cena nupcial en el palacio del rey.' Sin embargo, muchos de los que habían prometido asistir se negaron a venir en aquel momento. Cuando el rey escuchó que rechazaban su invitación, envió a otros servidores y mensajeros, diciendo: `Decid que vengan todos los que estaban invitados, porque mirad, mi cena está preparada. Mis bueyes y mis cebones han sido matados, y todo está preparado para celebrar la boda inminente de mi hijo.' Pero de nuevo, aquellos invitados desconsiderados no le dieron importancia a la llamada de su rey, y se fueron por su camino, uno a su granja, otro a su cerámica y otros a sus negocios. Y otros además no se contentaron con menospreciar así la llamada del rey, sino que se rebelaron abiertamente, pegaron a los mensajeros del rey, los maltrataron vergonzosamente, e incluso mataron a algunos de ellos. Cuando el rey observó que sus convidados elegidos, incluídos aquellos que habían aceptado su invitación preliminar y habían prometido asistir al banquete de boda, rechazaban finalmente su llamada, y en rebeldía habían atacado y matado a sus mensajeros elegidos, se encolerizó extremadamente. Entonces, este rey ultrajado mandó salir a sus ejércitos y a los ejércitos de sus aliados, y les ordenó que destruyeran a aquellos asesinos rebeldes y que incendiaran su ciudad>>.

\par 
%\textsuperscript{(1895.1)}
\textsuperscript{173:5.3} <<Después de haber castigado a los que habían despreciado su invitación, fijó un nuevo día para el banquete de bodas y dijo a sus mensajeros: `Los primeros invitados a la boda no eran dignos; id pues ahora a los cruces de los caminos y a las carreteras, e incluso más allá de los límites de la ciudad, e invitad a todos los que encontréis, incluídos los extranjeros, para que vengan y asistan a este banquete de bodas.' Los servidores salieron entonces a las carreteras y a los lugares apartados, y reunieron a todos los que encontraron, buenos y malos, ricos y pobres, de manera que por fin la sala nupcial se llenó de convidados de buena voluntad. Cuando todo estuvo dispuesto, el rey entró para examinar a sus huéspedes, y se sorprendió mucho al ver allí a un hombre sin vestido nupcial. Puesto que el rey había proporcionado generosamente vestidos nupciales a todos sus huéspedes, se dirigió a este hombre y le dijo: `Amigo, ¿cómo puede ser que entres en la sala de mis invitados, en esta ocasión, sin el vestido nupcial'? Aquel hombre descuidado se quedó callado. Entonces, el rey dijo a sus servidores: `Echad a este invitado desconsiderado de mi casa, y que comparta la misma suerte que todos los demás que despreciaron mi hospitalidad y rechazaron mi llamada. Sólo quiero tener aquí a los que se regocijan de aceptar mi invitación, y que me hacen el honor de llevar los vestidos nupciales que tan generosamente se han proporcionado a todos.'>>

\par 
%\textsuperscript{(1895.2)}
\textsuperscript{173:5.4} Después de contar esta parábola, Jesús estaba a punto de despedir a la multitud cuando un creyente simpatizante se abrió paso hacia él a través del gentío, y preguntó: <<Pero, Maestro, ¿cómo nos enteraremos de esas cosas? ¿Cómo estaremos preparados para la invitación del rey? ¿Qué signo nos darás para que sepamos que eres el Hijo de Dios?>> Cuando el Maestro escuchó estas palabras, dijo: <<Sólo se os dará un signo>>. Luego, señalando a su propio cuerpo, continuó: <<Destruid este templo, y en tres días lo levantaré>>. Pero no lo comprendieron, y se dispersaron diciéndose entre ellos: <<Este templo ha estado en construcción casi cincuenta años, y sin embargo dice que lo destruirá y lo levantará en tres días>>. Ni siquiera sus propios apóstoles comprendieron el significado de esta declaración, pero posteriormente, después de su resurrección, recordaron lo que el Maestro había dicho.

\par 
%\textsuperscript{(1895.3)}
\textsuperscript{173:5.5} Hacia las cuatro de esta tarde, Jesús hizo señas a sus apóstoles y les indicó que deseaba dejar el templo e ir a Betania para cenar y descansar durante la noche. Mientras subían el Olivete, Jesús indicó a Andrés, Felipe y Tomás que al día siguiente debían establecer un campamento más cerca de la ciudad, para poder ocuparlo durante el resto de la semana pascual. Siguiendo estas instrucciones, a la mañana siguiente montaron sus tiendas de campaña en una hondonada de la ladera que dominaba el parque de acampamiento público de Getsemaní, en un pequeño terreno que pertenecía a Simón de Betania.

\par 
%\textsuperscript{(1896.1)}
\textsuperscript{173:5.6} De nuevo, un grupo silencioso de judíos ascendió la pendiente occidental del Olivete este lunes por la noche. Estos doce hombres empezaban a sentir, como nunca lo habían sentido antes, que algo trágico estaba a punto de suceder. La espectacular depuración del templo, durante las primeras horas de la mañana, había despertado sus esperanzas de ver cómo el Maestro se imponía y manifestaba sus grandes poderes, pero los acontecimientos de toda la tarde estuvieron caracterizados por un descenso de la tensión, en el sentido de que todos apuntaban a un rechazo seguro de las enseñanzas de Jesús por parte de las autoridades judías. Los apóstoles se sentían oprimidos por la duda y prisioneros de una terrible incertidumbre. Se daban cuenta de que podían transcurrir sólo unos breves días entre los acontecimientos del día que acababa de terminar y el estallido de una fatalidad inminente. Todos sentían que algo temible estaba a punto de suceder, pero no sabían qué esperar. Cada uno se fue a su sitio para descansar, pero durmieron muy poco. Incluso los gemelos Alfeo empezaron por fin a comprender que los acontecimientos de la vida del Maestro se dirigían velozmente hacia su culminación final.


\chapter{Documento 174. El martes por la mañana en el templo}
\par 
%\textsuperscript{(1897.1)}
\textsuperscript{174:0.1} HACIA las siete de la mañana de este martes, Jesús se reunió, en la casa de Simón, con los apóstoles, el cuerpo de mujeres y unas dos docenas de otros discípulos destacados. En esta reunión se despidió de Lázaro, y le dio las instrucciones que le indujeron a huir rápidamente a Filadelfia en Perea, donde se unió más tarde al movimiento misionero que tenía su sede en aquella ciudad. Jesús también se despidió del anciano Simón, y dio sus consejos de despedida al cuerpo de mujeres, pues nunca más se dirigió a ellas de manera oficial.

\par 
%\textsuperscript{(1897.2)}
\textsuperscript{174:0.2} Esta mañana, saludó a cada uno de lo doce con unas palabras personales. A Andrés le dijo: <<No te desanimes por los acontecimientos inminentes. Controla firmemente a tus hermanos y procura que no te vean abatido>>. A Pedro le dijo: <<No pongas tu confianza en el vigor de tu brazo ni en las armas de acero. Asiéntate sobre los fundamentos espirituales de las rocas eternas>>. A Santiago le dijo: <<No vaciles ante las apariencias externas. Permanece firme en tu fe, y pronto conocerás la realidad de aquello en lo que crees>>. A Juan le dijo: <<Sé dulce; ama incluso a tus enemigos; sé tolerante. Y recuerda que te he confiado muchas cosas>>. A Natanael le dijo: <<No juzgues por las apariencias; permanece firme en tu fe cuando todo parezca desvanecerse; sé fiel a tu misión de embajador del reino>>. A Felipe le dijo: <<No te dejes conmover por los acontecimientos inminentes. Permanece impasible, aunque no puedas ver el camino. Sé fiel a tu juramento de consagración>>. A Mateo le dijo: <<No olvides la misericordia que te recibió en el reino. No dejes que nadie te robe tu recompensa eterna. Puesto que has resistido las tendencias de la naturaleza humana, dispónte a ser firme>>. A Tomás le dijo: <<Por muy difícil que sea, ahora tienes que caminar por la fe y no por la vista. No dudes de que yo sea capaz de terminar la obra que he empezado, y de que finalmente veré a todos mis fieles embajadores en el reino del más allá>>. A los gemelos Alfeo les dijo: <<No permitáis que os abrumen las cosas que no podéis comprender. Sed fieles a los afectos de vuestro corazón, y no pongáis vuestra confianza ni en los grandes hombres ni en la actitud cambiante de la gente. Permaneced al lado de vuestros hermanos>>. A Simón Celotes le dijo: <<Simón, quizás te sientas abrumado por la decepción, pero tu espíritu se elevará por encima de todo lo que pueda sucederte. Lo que no has conseguido aprender de mí, mi espíritu te lo enseñará. Busca las verdaderas realidades del espíritu, y deja de sentirte atraído por las sombras irreales y materiales>>. Y a Judas Iscariote le dijo: <<Judas, te he amado y he rogado para que ames a tus hermanos. No te canses de hacer el bien; y deseo prevenirte que te guardes de los senderos resbaladizos de la adulación y de los dardos envenenados del ridículo>>.

\par 
%\textsuperscript{(1897.3)}
\textsuperscript{174:0.3} Una vez que hubo concluido estos saludos, partió para Jerusalén con Andrés, Pedro, Santiago y Juan, mientras que los demás apóstoles se ocupaban de establecer el campamento de Getsemaní, donde iban a dirigirse aquella noche, y donde instalaron su cuartel general durante el resto de la vida mortal del Maestro. Aproximadamente a medio camino del descenso del Olivete, Jesús se detuvo y conversó durante más de una hora con los cuatro apóstoles.

\section*{1. El perdón divino}
\par 
%\textsuperscript{(1898.1)}
\textsuperscript{174:1.1} Durante varios días, Pedro y Santiago habían estado discutiendo sus diferencias de opinión sobre la enseñanza del Maestro acerca del perdón de los pecados. Los dos habían acordado plantear el asunto a Jesús, y Pedro aprovechó esta ocasión como una oportunidad adecuada para obtener el consejo del Maestro. En consecuencia, Simón Pedro interrumpió la conversación sobre las diferencias entre la alabanza y la adoración, y preguntó: <<Maestro, Santiago y yo no estamos de acuerdo sobre tus enseñanzas relacionadas con el perdón de los pecados. Santiago afirma que, según tu enseñanza, el Padre nos perdona incluso antes de que se lo pidamos, y yo sostengo que el arrepentimiento y la confesión deben preceder al perdón. ¿Quién de nosotros tiene razón? ¿Qué dices tú?>>

\par 
%\textsuperscript{(1898.2)}
\textsuperscript{174:1.2} Después de un breve silencio, Jesús miró de manera significativa a los cuatro y contestó: <<Hermanos míos, os equivocáis en vuestras opiniones porque no comprendéis la naturaleza de las relaciones íntimas y amorosas entre la criatura y el Creador, entre el hombre y Dios. No lográis captar la simpatía comprensiva que un padre sabio alberga por su hijo inmaduro y a veces equivocado. En verdad es dudoso que unos padres inteligentes y afectuosos se vean nunca en la necesidad de perdonar a un hijo normal y corriente. Las relaciones comprensivas, asociadas con las actitudes amorosas, impiden eficazmente todos los distanciamientos que necesitan posteriormente un reajuste mediante el arrepentimiento del hijo y el perdón del padre>>.

\par 
%\textsuperscript{(1898.3)}
\textsuperscript{174:1.3} <<En cada hijo vive una fracción de su padre. El padre disfruta de una prioridad y de una superioridad de comprensión en todas las cuestiones relacionadas con la relación entre padre e hijo. El padre es capaz de percibir la inmadurez del hijo a la luz de la madurez paternal más elevada, de la experiencia más madura que posee el compañero de más edad. En el caso del hijo terrestre y del Padre celestial, el padre divino posee, de una manera infinita y divina, la compasión y la capacidad para comprender con amor. El perdón divino es inevitable; es inherente e inalienable a la comprensión infinita de Dios, a su conocimiento perfecto de todo lo relacionado con el juicio erróneo y la elección equivocada del hijo. La justicia divina es tan eternamente equitativa que engloba infaliblemente una misericordia comprensiva>>.

\par 
%\textsuperscript{(1898.4)}
\textsuperscript{174:1.4} <<Cuando un hombre sensato comprende los impulsos internos de sus semejantes, los ama. Y cuando amáis a vuestro hermano, ya lo habéis perdonado. Esta capacidad para comprender la naturaleza del hombre y para perdonar sus aparentes fechorías, es divina. Si sois unos padres sabios, así es como amaréis y comprenderéis a vuestros hijos, e incluso los perdonaréis cuando los malentendidos pasajeros os hayan separado aparentemente. El hijo es inmaduro y no comprende plenamente la profundidad de la relación entre padre e hijo; por eso experimenta con frecuencia un sentimiento de separación culpable cuando no tiene la plena aprobación de su padre, pero un verdadero padre nunca tiene conciencia de una separación semejante. El pecado es una experiencia de la conciencia de la criatura; no forma parte de la conciencia de Dios>>.

\par 
%\textsuperscript{(1898.5)}
\textsuperscript{174:1.5} <<Vuestra incapacidad o vuestra mala disposición para perdonar a vuestros semejantes es la medida de vuestra inmadurez, de vuestro fracaso en alcanzar el nivel adulto de compasión, de comprensión y de amor. Vuestros rencores y vuestras ideas de venganza son directamente proporcionales a vuestra ignorancia de la naturaleza interior y de los verdaderos anhelos de vuestros hijos y de vuestros semejantes. El amor es la manifestación exterior del impulso de vida interior y divino. Está basado en la comprensión, alimentado por el servicio desinteresado y perfeccionado con la sabiduría>>.

\section*{2. Las preguntas de los dirigentes judíos}
\par 
%\textsuperscript{(1899.1)}
\textsuperscript{174:2.1} El lunes por la noche se había celebrado un consejo entre el sanedrín y unos cincuenta dirigentes adicionales seleccionados entre los escribas, los fariseos y los saduceos. Esta asamblea llegó al consenso de que sería peligroso arrestar a Jesús en público a causa de su influencia sobre los sentimientos de la gente común. La mayoría opinaba también que había que hacer un esfuerzo decidido para desacreditarlo a los ojos de la multitud, antes de arrestarlo y de llevarlo a juicio. En consecuencia, se designaron diversos grupos de hombres eruditos para que estuvieran disponibles a la mañana siguiente en el templo, a fin de intentar hacerlo caer en una trampa con preguntas difíciles, y tratar de desconcertarlo de otras maneras delante de la gente. Al fin, los fariseos, los saduceos e incluso los herodianos se encontraban todos unidos en este esfuerzo por desacreditar a Jesús a los ojos de las multitudes pascuales.

\par 
%\textsuperscript{(1899.2)}
\textsuperscript{174:2.2} El martes por la mañana, cuando Jesús llegó al patio del templo y empezó a enseñar, sólo había pronunciado algunas palabras cuando un grupo de los estudiantes más jóvenes de las academias, que habían sido preparados de antemano con esta finalidad, se adelantaron y se dirigieron a Jesús a través de su portavoz, diciendo: <<Maestro, sabemos que eres un instructor honrado; sabemos que proclamas los caminos de la verdad y que sólo sirves a Dios, porque no temes a ningún hombre y no haces acepción de personas. Sólo somos unos estudiantes, y quisiéramos conocer la verdad sobre una cuestión que nos preocupa. Nuestra dificultad es la siguiente: ¿Es lícito que paguemos tributo al César? ¿Hemos de pagarlo o no?>> Percibiendo su hipocresía y su astucia, Jesús les dijo: <<¿Por qué venís a tentarme de esta manera? Mostradme el dinero del tributo, y os contestaré>>. Cuando los estudiantes le entregaron un denario, lo examinó y dijo: <<¿De quién es la imagen y la inscripción que lleva esta moneda?>> Cuando le contestaron: <<Del César>>, Jesús dijo: <<Dad al César las cosas que son del César, y dad a Dios las cosas que son de Dios>>.

\par 
%\textsuperscript{(1899.3)}
\textsuperscript{174:2.3} Después de haber contestado así, los jóvenes escribas y sus cómplices herodianos se retiraron de su presencia, y la gente, incluídos los saduceos, disfrutaron de su turbación. Incluso los jóvenes que habían intentado hacer caer al Maestro en una trampa, se maravillaron enormemente de la inesperada sagacidad de su respuesta.

\par 
%\textsuperscript{(1899.4)}
\textsuperscript{174:2.4} El día anterior, los dirigentes habían intentado que cometiera un desliz delante de la multitud en cuestiones de autoridad eclesiástica, y como habían fracasado, ahora intentaban implicarlo en una discusión perjudicial sobre la autoridad civil. Tanto Pilatos como Herodes se encontraban en Jerusalén en aquel momento, y los enemigos de Jesús supusieron que si se atrevía a aconsejar que no se pagara el tributo al César, podrían ir inmediatamente a las autoridades romanas y acusarlo de sedición. Por otra parte, si aconsejaba expresamente el pago del tributo, calculaban con razón que dicha declaración heriría profundamente el orgullo nacional de sus oyentes judíos, desviando así la buena voluntad y el afecto de la multitud.

\par 
%\textsuperscript{(1899.5)}
\textsuperscript{174:2.5} Los enemigos de Jesús fueron derrotados en todo esto puesto que una orden bien conocida del sanedrín, emitida para orientar a los judíos dispersos por las naciones gentiles, precisaba que el <<derecho de acuñar moneda comportaba el derecho de exigir impuestos>>. De esta manera, Jesús había evitado la trampa. Si hubiera contestado <<no>> a su pregunta, hubiera sido el equivalente de incitar a la rebelión; si hubiera contestado <<sí>>, habría conmocionado los sentimientos nacionalistas profundamente arraigados de aquella época. El Maestro no eludió la pregunta; simplemente utilizó la sabiduría de ofrecer una respuesta doble. Jesús nunca era evasivo, pero siempre era sabio en su trato con los que intentaban acosarlo y destruirlo.

\section*{3. Los saduceos y la resurrección}
\par 
%\textsuperscript{(1900.1)}
\textsuperscript{174:3.1} Antes de que Jesús pudiera empezar su enseñanza, otro grupo se adelantó para hacerle preguntas, en esta ocasión un grupo de saduceos eruditos y astutos. Su portavoz se acercó y le dijo: <<Maestro, Moisés dijo que si un hombre casado moría sin dejar hijos, su hermano se casaría con la mujer y engendraría una descendencia a su hermano muerto. Pues bien, se ha producido un caso en el que un hombre que tenía seis hermanos murió sin hijos; el hermano siguiente se casó con su mujer, pero también murió pronto sin dejar hijos. El segundo hermano tomó asimismo a la mujer, pero también murió sin dejar descendencia. Y así sucesivamente hasta que los seis hermanos se casaron con ella, y los seis murieron sin dejar hijos. Luego, la mujer murió después de todos ellos. Pues bien, lo que quisiéramos preguntarte es lo siguiente: Cuando llegue la resurrección, ¿de quién será la esposa, puesto que los siete hermanos se casaron con ella?>>

\par 
%\textsuperscript{(1900.2)}
\textsuperscript{174:3.2} Jesús sabía, y la gente también, que estos saduceos no eran sinceros al hacer esta pregunta, porque no era probable que un caso así se produjera realmente; además, esta costumbre de que los hermanos de un muerto trataran de engendrarle hijos, era prácticamente letra muerta entre los judíos de esta época. Sin embargo, Jesús condescendió a contestar a su pregunta maliciosa. Dijo: <<Todos os equivocáis al hacer este tipo de preguntas, porque no conocéis ni las Escrituras ni el poder viviente de Dios. Sabéis que los hijos de este mundo pueden casarse y ser dados en matrimonio, pero no parecéis comprender que aquellos que son considerados dignos de alcanzar los mundos venideros, mediante la resurrección de los justos, no se casan ni son dados en matrimonio. Los que experimentan la resurrección de entre los muertos se parecen más a los ángeles del cielo, y no mueren nunca. Esos resucitados son eternamente los hijos de Dios; son los hijos de la luz resucitados para el progreso de la vida eterna. Incluso vuestro padre Moisés comprendió esto porque, en conexión con sus experiencias junto a la zarza ardiente, oyó decir al Padre:
`Yo \textit{soy} el Dios de Abraham, el Dios de Isaac y el Dios de Jacob.' Y así, junto con Moisés, declaro que mi Padre no es el Dios de los muertos, sino de los vivos. En él todos vivís, os reproducís y poseéis vuestra existencia mortal>>.

\par 
%\textsuperscript{(1900.3)}
\textsuperscript{174:3.3} Cuando Jesús hubo terminado de contestar estas preguntas, los saduceos se retiraron, y algunos fariseos se olvidaron tanto de sí mismos que exclamaron: <<Es verdad, es verdad, Maestro, has contestado bien a esos saduceos incrédulos>>. Los saduceos no se atrevieron a hacerle más preguntas, y la gente común se maravilló de la sabiduría de su enseñanza.

\par 
%\textsuperscript{(1900.4)}
\textsuperscript{174:3.4} En su choque con los saduceos, Jesús sólo recurrió a Moisés porque esta secta político-religiosa únicamente reconocía la validez de los llamados cinco libros de Moisés; no aceptaban que las enseñanzas de los profetas sirvieran de base para los dogmas doctrinales. En su respuesta, el Maestro afirmó categóricamente el hecho de la supervivencia de las criaturas mortales mediante la técnica de la resurrección, pero no aprobó en ningún sentido las creencias fariseas en la resurrección del cuerpo humano físico. El punto que Jesús deseaba recalcar era que el Padre había dicho:
`Yo \textit{soy} el Dios de Abraham, de Isaac y de Jacob', y no yo \textit{era} su Dios.

\par 
%\textsuperscript{(1900.5)}
\textsuperscript{174:3.5} Los saduceos habían querido someter a Jesús a la influencia debilitante del \textit{ridículo}, sabiendo muy bien que toda persecución en público crearía sin duda una mayor simpatía hacia él en la mente de la multitud.

\section*{4. El gran mandamiento}
\par 
%\textsuperscript{(1901.1)}
\textsuperscript{174:4.1} Otro grupo de saduceos había recibido instrucciones para hacerle a Jesús unas preguntas enredosas sobre los ángeles, pero cuando observaron la suerte de sus compañeros que habían intentado hacerlo caer en una trampa con preguntas relacionadas con la resurrección, decidieron muy juiciosamente permanecer en silencio; se retiraron sin hacer una sola pregunta. Los fariseos, los escribas, los saduceos y los herodianos aliados habían premeditado el plan de pasarse todo el día haciéndole estas preguntas enredosas, esperando así desacreditar a Jesús delante de la gente, y al mismo tiempo impedirle eficazmente que tuviera tiempo para proclamar sus enseñanzas perturbadoras.

\par 
%\textsuperscript{(1901.2)}
\textsuperscript{174:4.2} Uno de los grupos de fariseos se adelantó entonces para hacerle preguntas embarazosas; el portavoz hizo señas a Jesús, y dijo: <<Maestro, soy jurista, y me gustaría preguntarte cuál es, en tu opinión, el mandamiento más grande>>. Jesús respondió: <<No hay más que un solo mandamiento, que es el más grande de todos, y ese mandamiento es: `Escucha, oh Israel, al Señor nuestro Dios; el Señor es uno; y amarás al Señor tu Dios con todo tu corazón y con toda tu alma, con toda tu mente y con todas tus fuerzas.' Éste es el primer gran mandamiento. Y el segundo mandamiento se parece a este primero; en efecto, proviene directamente de él, y dice: `Amarás a tu prójimo como a ti mismo.' No hay otros mandamientos más grandes que estos; en estos dos mandamientos se apoyan toda la ley y los profetas>>.

\par 
%\textsuperscript{(1901.3)}
\textsuperscript{174:4.3} Cuando el jurista percibió que Jesús no solamente había respondido de acuerdo con el concepto más elevado de la religión judía, sino que también había contestado sabiamente a los ojos de la multitud reunida, pensó que era mejor tener el valor de alabar abiertamente la respuesta del Maestro. En consecuencia, dijo: <<En verdad, Maestro, has dicho bien que Dios es uno y que no hay nadie aparte de él; y que el primer gran mandamiento es amarlo con todo el corazón, con toda la inteligencia y con todas nuestras fuerzas, y también amar al prójimo como a uno mismo. Estamos de acuerdo en que este gran mandamiento tiene mucha más importancia que todos los holocaustos y sacrificios>>. Cuando el jurista contestó de esta manera tan prudente, Jesús bajó la mirada hacia él y dijo: <<Amigo mío, percibo que no estás muy lejos del reino de Dios>>.

\par 
%\textsuperscript{(1901.4)}
\textsuperscript{174:4.4} Jesús dijo la verdad cuando indicó que este jurista <<no estaba muy lejos del reino>>, porque aquella misma noche fue al campamento del Maestro, cerca de Getsemaní, confesó su fe en el evangelio del reino y fue bautizado por Josías, uno de los discípulos de Abner.

\par 
%\textsuperscript{(1901.5)}
\textsuperscript{174:4.5} Otros dos o tres grupos de escribas y fariseos estaban presentes y habían tenido la intención de hacerle preguntas, pero se sentían desarmados por la respuesta de Jesús al jurista, o bien estaban acobardados por la derrota de todos los que habían intentado enredarlo. Después de esto, nadie se atrevió a hacerle más preguntas en público.

\par 
%\textsuperscript{(1901.6)}
\textsuperscript{174:4.6} Como no había más preguntas y se estaba acercando la hora del mediodía, Jesús no reanudó su enseñanza, sino que se contentó simplemente con hacer una pregunta a los fariseos y a sus asociados. Jesús dijo: <<Puesto que no hacéis más preguntas, me gustaría haceros una. ¿Qué pensáis del Libertador? Es decir, ¿de quién es hijo?>> Después de una breve pausa, uno de los escribas contestó: <<El Mesías es el hijo de David>>. Puesto que Jesús sabía que se había discutido mucho, incluso entre sus propios discípulos, sobre si él era o no el hijo de David, hizo esta otra pregunta: <<Si el Libertador es en verdad el hijo de David, ¿cómo puede ser que en el salmo que atribuís a David, él mismo dice, hablando según el espíritu: `El Señor dijo a mi señor: Siéntate a mi derecha hasta que ponga a tus enemigos de banqueta para tus pies?' Si David le llama Señor, entonces ¿cómo puede ser su hijo?>> Los dirigentes, los escribas y los principales sacerdotes no contestaron a esta pregunta, pero también se abstuvieron de hacerle más preguntas para intentar enredarlo. Nunca contestaron a la pregunta que Jesús les había hecho, pero después de la muerte del Maestro, intentaron eludir la dificultad cambiando la interpretación de este salmo para que se refiriera a Abraham en lugar del Mesías. Otros trataron de evitar este dilema negando que David fuera el autor de este salmo llamado mesiánico.

\par 
%\textsuperscript{(1902.1)}
\textsuperscript{174:4.7} Un rato antes, los fariseos habían disfrutado con la manera en que el Maestro había acallado a los saduceos; ahora los saduceos se regocijaban con el fracaso de los fariseos; pero esta rivalidad sólo era momentánea; rápidamente se olvidaron de sus diferencias tradicionales, en un esfuerzo común por poner fin a las enseñanzas y a las obras de Jesús. Pero durante todas estas experiencias, la gente común le escuchó con agrado.

\section*{5. Los griegos indagadores}
\par 
%\textsuperscript{(1902.2)}
\textsuperscript{174:5.1} Alrededor del mediodía, mientras Felipe compraba unas provisiones para el nuevo campamento que se estaba estableciendo aquel día cerca de Getsemaní, fue abordado por una delegación de extranjeros, un grupo de creyentes griegos de Alejandría, Atenas y Roma, cuyo portavoz le dijo al apóstol: <<Los que te conocen nos han dicho que nos dirijamos a ti; por eso venimos a ti, Señor, con la petición de ver a Jesús, tu Maestro>>. A Felipe le cogió de sorpresa el encontrarse así, en la plaza del mercado, con estos gentiles griegos eminentes e indagadores. Puesto que Jesús había encargado explícitamente a los doce que no efectuaran ninguna enseñanza pública durante la semana de la Pascua, Felipe estaba un poco confuso sobre la manera correcta de manejar esta situación. También estaba desconcertado porque estos hombres eran gentiles extranjeros. Si hubieran sido judíos, o gentiles conocidos de los alrededores, no hubiera dudado tanto. Lo que hizo fue lo siguiente: Pidió a aquellos griegos que permanecieran allí donde estaban. Mientras se alejaba deprisa, los griegos supusieron que había ido a buscar a Jesús, pero en realidad corrió a la casa de José, donde sabía que Andrés y los otros apóstoles estaban almorzando. Llamó a Andrés para que saliera, le explicó el motivo de su venida, y luego regresó con Andrés al lugar donde esperaban los griegos.

\par 
%\textsuperscript{(1902.3)}
\textsuperscript{174:5.2} Como Felipe casi había terminado de comprar las provisiones, regresó con Andrés y los griegos a la casa de José, donde Jesús los recibió. Se sentaron cerca del Maestro, mientras éste hablaba a sus apóstoles y a un grupo de discípulos principales reunidos en este almuerzo. Jesús dijo:

\par 
%\textsuperscript{(1902.4)}
\textsuperscript{174:5.3} <<Mi Padre me ha enviado a este mundo para revelar su bondad a los hijos de los hombres, pero los primeros a quienes me he dirigido se han negado a recibirme. Es verdad que muchos de vosotros habéis creído en mi evangelio por vosotros mismos, pero los hijos de Abraham y sus dirigentes están a punto de rechazarme, y al hacerlo, rechazarán a Aquél que me ha enviado. He proclamado sin reservas el evangelio de la salvación a este pueblo; les he hablado de la filiación acompañada de alegría, de libertad y de una vida más abundante en el espíritu. Mi Padre ha realizado muchas obras maravillosas entre estos hijos de los hombres tiranizados por el miedo. Pero el profeta Isaías se refirió con razón a este pueblo cuando escribió: `Señor, ¿quién ha creído en nuestras enseñanzas? ¿Y a quién ha sido revelado el Señor?' En verdad, los dirigentes de mi pueblo se han cegado deliberadamente para no ver, y han endurecido su corazón por temor a creer y a ser salvados. Todos estos años he tratado de curarlos de su incredulidad, para que puedan recibir la salvación eterna del Padre. Sé que no todos me han fallado; algunos de vosotros habéis creído de verdad en mi mensaje. En esta sala hay ahora veinte hombres que han sido anteriormente miembros del sanedrín, o que han ocupado altos puestos en los consejos de la nación, aunque algunos de ellos evitan todavía confesar abiertamente la verdad, por temor a ser expulsados de la sinagoga. Algunos de vosotros tenéis la tentación de amar más la gloria de los hombres que la gloria de Dios. Pero me veo obligado a mostrar paciencia, puesto que temo incluso por la seguridad y la lealtad de algunos de los que han estado tanto tiempo junto a mí, y que han vivido tan cerca de mi>>.

\par 
%\textsuperscript{(1903.1)}
\textsuperscript{174:5.4} <<Observo que en esta sala de banquetes están reunidos los judíos y los gentiles en un número aproximadamente igual, y os dirigiré la palabra como al primer y último grupo de este tipo que voy a instruir en los asuntos del reino antes de ir hacia mi Padre>>.

\par 
%\textsuperscript{(1903.2)}
\textsuperscript{174:5.5} Estos griegos habían asistido fielmente a las enseñanzas de Jesús en el templo. El lunes por la noche habían celebrado una conferencia en la casa de Nicodemo, que se había prolongado hasta el amanecer, y treinta de ellos habían elegido entrar en el reino.

\par 
%\textsuperscript{(1903.3)}
\textsuperscript{174:5.6} Mientras Jesús permanecía delante de ellos en aquel momento, percibió el final de una dispensación y el principio de otra. Volviendo su atención hacia los griegos, el Maestro dijo:

\par 
%\textsuperscript{(1903.4)}
\textsuperscript{174:5.7} <<El que cree en este evangelio, no solamente cree en mí, sino en Aquel que me ha enviado. Cuando me miráis, no veis solamente al Hijo del Hombre, sino también a Aquel que me ha enviado. Yo soy la luz del mundo, y cualquiera que crea en mi enseñanza ya no permanecerá más tiempo en las tinieblas. Si vosotros, los gentiles, queréis escucharme, recibiréis las palabras de la vida y entraréis inmediatamente en la gozosa libertad de la verdad de la filiación con Dios. Si mis compatriotas, los judíos, escogen rechazarme y rehusar mis enseñanzas, no los juzgaré, porque no he venido para juzgar al mundo, sino para ofrecerle la salvación. Sin embargo, los que me rechazan y rehúsan recibir mi enseñanza, serán llevados a juicio a su debido tiempo por mi Padre y por aquellos que él ha designado para que juzguen a los que rechazan el don de la misericordia y las verdades de la salvación. Recordad todos que no hablo por mí mismo, sino que os he proclamado fielmente lo que el Padre mandó que yo debía revelar a los hijos de los hombres. Y estas palabras que el Padre me ordenó que dijera al mundo son palabras de verdad divina, de misericordia perpetua y de vida eterna>>.

\par 
%\textsuperscript{(1903.5)}
\textsuperscript{174:5.8} <<Pero declaro tanto a los judíos como a los gentiles, que está a punto de llegar la hora en que el Hijo del Hombre será glorificado. Sabéis muy bien que un grano de trigo permanece solitario, a menos que caiga en la tierra y muera; pero si muere en una buena tierra, surge de nuevo a la vida y produce mucho fruto. Aquel que ama egoístamente su vida, corre el peligro de perderla; pero aquel que está dispuesto a dar su vida por mí y por el evangelio, gozará de una existencia más abundante en la Tierra, y de la vida eterna en el cielo. Si queréis seguirme sinceramente, incluso después de que haya regresado al Padre, entonces os convertiréis en mis discípulos y en los sinceros servidores de vuestros semejantes>>.

\par 
%\textsuperscript{(1903.6)}
\textsuperscript{174:5.9} <<Sé que se acerca mi hora, y estoy preocupado. Me doy cuenta de que mi pueblo está decidido a despreciar el reino, pero me alegra recibir a estos gentiles que buscan la verdad, y que hoy están aquí para preguntar por el camino de la luz. Sin embargo, mi corazón sufre por mi pueblo, y mi alma está angustiada por lo que me espera. ¿Qué puedo decir cuando miro hacia adelante y percibo lo que está a punto de sucederme? ¿Acaso diré: Padre, sálvame de esta hora terrible? ¡No! Precisamente con esta finalidad he venido al mundo, e incluso he llegado hasta esta hora. Diré más bien, orando para que os unáis a mí: Padre, glorifica tu nombre; que se haga tu voluntad>>.

\par 
%\textsuperscript{(1904.1)}
\textsuperscript{174:5.10} Cuando Jesús hubo hablado así, el Ajustador Personalizado que había residido en él antes de su bautismo apareció delante de él, y mientras hacía una pausa de manera perceptible, este espíritu ahora poderoso que representaba al Padre le habló a Jesús de Nazaret, diciendo:<<He glorificado mi nombre muchas veces en tus donaciones, y lo glorificaré una vez más>>.

\par 
%\textsuperscript{(1904.2)}
\textsuperscript{174:5.11} Aunque los judíos y los gentiles allí reunidos no escucharon ninguna voz, no pudieron dejar de percibir que el Maestro se había detenido en su discurso mientras le llegaba un mensaje de alguna fuente sobrehumana. Cada uno le dijo al que tenía a su lado: <<Un ángel le ha hablado>>.

\par 
%\textsuperscript{(1904.3)}
\textsuperscript{174:5.12} Entonces Jesús continuó diciendo: <<Todo esto no ha sucedido por mi bien, sino por el vuestro. Sé con certeza que el Padre me recibirá y aceptará mi misión en vuestro favor, pero es necesario que os sintáis estimulados y preparados para la prueba de fuego que se avecina. Dejadme aseguraros que la victoria terminará por coronar nuestros esfuerzos unidos por iluminar al mundo y liberar a la humanidad. El antiguo orden de cosas se está juzgando a sí mismo; he derribado al Príncipe de este mundo, y todos los hombres llegarán a ser libres gracias a la luz del espíritu que yo derramaré sobre toda carne, después de haber ascendido hasta mi Padre que está en los cielos>>.

\par 
%\textsuperscript{(1904.4)}
\textsuperscript{174:5.13} <<Y ahora os afirmo que, si soy elevado en la Tierra y en vuestras vidas, atraeré a todos los hombres hacia mí y hacia la comunidad de mi Padre. Habéis creído que el Libertador residiría para siempre en la Tierra, pero declaro que el Hijo del Hombre será rechazado por los hombres, y que regresará al Padre. Sólo estaré con vosotros un corto período de tiempo; la luz viviente sólo estará poco tiempo en medio de esta generación tenebrosa. Caminad mientras tengáis esta luz, para que las tinieblas y la confusión venideras no os cojan por sorpresa. El que camina en las tinieblas, no sabe adonde va; pero si escogéis caminar en la luz, todos os convertiréis en verdad en los hijos liberados de Dios. Y ahora, venid conmigo todos vosotros mientras regresamos al templo, donde voy a decir mis palabras de adiós a los jefes de los sacerdotes, a los escribas, a los fariseos, a los saduceos, a los herodianos y a los dirigentes ignorantes de Israel>>.

\par 
%\textsuperscript{(1904.5)}
\textsuperscript{174:5.14} Después de haber hablado así, Jesús condujo al grupo de regreso hacia el templo por las estrechas calles de Jerusalén. Acababan de oír decir al Maestro que éste iba a ser su discurso de adiós en el templo, y le siguieron en silencio, meditando profundamente.


\chapter{Documento 175. El último discurso en el templo}
\par 
%\textsuperscript{(1905.1)}
\textsuperscript{175:0.1} POCO después de las dos de la tarde de este martes, Jesús llegó al templo en compañía de once apóstoles, José de Arimatea, los treinta griegos y algunos otros discípulos, y empezó a pronunciar su última alocución en los patios del edificio sagrado. Este discurso estaba destinado a ser su último llamamiento al pueblo judío y la acusación final contra sus vehementes enemigos que trataban de destruirlo: los escribas, los fariseos, los saduceos y los dirigentes principales de Israel. A lo largo de toda la mañana, los diversos grupos habían tenido la oportunidad de hacerle preguntas a Jesús; esta tarde, nadie le preguntó nada.

\par 
%\textsuperscript{(1905.2)}
\textsuperscript{175:0.2} Cuando el Maestro empezó a hablar, el patio del templo estaba tranquilo y en orden. Los cambistas y los mercaderes no se habían atrevido a entrar de nuevo en el templo desde que Jesús y la multitud excitada los habían echado el día anterior. Antes de empezar su discurso, Jesús miró con ternura a este auditorio que pronto iba a escuchar su alocución pública de despedida, su mensaje de misericordia para la humanidad, unido a su última denuncia de los falsos educadores y de los fanáticos dirigentes de los judíos.

\section*{1. El discurso}
\par 
%\textsuperscript{(1905.3)}
\textsuperscript{175:1.1} <<He estado con vosotros durante mucho tiempo, recorriendo el país de un lado a otro, y proclamando el amor del Padre por los hijos de los hombres. Muchos han visto la luz y han entrado, por la fe, en el reino de los cielos. En conexión con esta enseñanza y esta predicación, el Padre ha realizado muchas obras maravillosas, llegando incluso a resucitar a los muertos. Muchos enfermos y afligidos han recuperado la salud porque creían; pero toda esta proclamación de la verdad y esta curación de enfermedades no ha abierto los ojos a aquellos que se niegan a ver la luz, a aquellos que están decididos a rechazar este evangelio del reino>>.

\par 
%\textsuperscript{(1905.4)}
\textsuperscript{175:1.2} <<De todas las maneras compatibles con la realización de la voluntad de mi Padre, mis apóstoles y yo hemos hecho todo lo posible por vivir en paz con nuestros hermanos, por cumplir con las exigencias razonables de las leyes de Moisés y de las tradiciones de Israel. Hemos buscado la paz constantemente, pero los dirigentes de Israel no la quieren. Al rechazar la verdad de Dios y la luz del cielo, se alinean al lado del error y de las tinieblas. No puede haber paz entre la luz y las tinieblas, entre la vida y la muerte, entre la verdad y el error>>.

\par 
%\textsuperscript{(1905.5)}
\textsuperscript{175:1.3} <<Muchos de vosotros os habéis atrevido a creer en mis enseñanzas y ya habéis entrado en la alegría y la libertad de la conciencia de la filiación con Dios. Y daréis testimonio de que he ofrecido esta misma filiación con Dios a toda la nación judía, incluso a esos mismos hombres que ahora tratan de destruirme. Incluso ahora, mi Padre recibiría a esos educadores ciegos y a esos dirigentes hipócritas, sólo con que se volvieran hacia él y aceptaran su misericordia. Incluso ahora no es demasiado tarde para que esta gente reciba la palabra del cielo y acoja con agrado al Hijo del Hombre>>.

\par 
%\textsuperscript{(1906.1)}
\textsuperscript{175:1.4} <<Mi Padre ha tratado a este pueblo con misericordia durante mucho tiempo. Generación tras generación, hemos enviado a nuestros profetas para enseñarles y advertirles, y generación tras generación, han matado a estos instructores enviados por el cielo. Y ahora, vuestros altos sacerdotes obstinados y vuestros dirigentes testarudos continúan haciendo exactamente lo mismo. Del mismo modo que Herodes ha provocado la muerte de Juan, vosotros también os preparáis ahora para destruir al Hijo del Hombre>>.

\par 
%\textsuperscript{(1906.2)}
\textsuperscript{175:1.5} <<Mientras exista una posibilidad de que los judíos se vuelvan hacia mi Padre y busquen la salvación, el Dios de Abraham, de Isaac y de Jacob mantendrá extendidas sus manos misericordiosas hacia vosotros; pero una vez que hayáis llenado vuestra copa de impenitencia, y una vez que hayáis rechazado finalmente la misericordia de mi Padre, esta nación será abandonada a sí misma y llegará rápidamente a un final ignominioso. Este pueblo estaba destinado a convertirse en la luz del mundo, a mostrar la gloria espiritual de una raza que conocía a Dios, pero os habéis desviado tanto de la realización de vuestros privilegios divinos, que vuestros dirigentes están a punto de cometer la locura suprema de todos los tiempos, en el sentido de que están a punto de rechazar finalmente el don de Dios a todos los hombres y para todos los tiempos ---la revelación del amor del Padre que está en los cielos por todas sus criaturas de la Tierra>>.

\par 
%\textsuperscript{(1906.3)}
\textsuperscript{175:1.6} <<Una vez que hayáis rechazado esta revelación de Dios al hombre, el reino de los cielos será entregado a otros pueblos, a aquellos que lo reciban con alegría y felicidad. En nombre del Padre que me ha enviado, os advierto solemnemente que estáis a punto de perder vuestra posición en el mundo como portaestandartes de la verdad eterna y custodios de la ley divina. En este momento os ofrezco vuestra última oportunidad de adelantaros y arrepentiros, para anunciar vuestra intención de buscar a Dios con todo vuestro corazón y entrar, como niños pequeños y con una fe sincera, en la seguridad y la salvación del reino de los cielos>>.

\par 
%\textsuperscript{(1906.4)}
\textsuperscript{175:1.7} <<Mi Padre ha trabajado durante mucho tiempo por vuestra salvación, y yo he descendido para vivir entre vosotros y mostraros personalmente el camino. Muchos judíos y samaritanos, e incluso los gentiles, han creído en el evangelio del reino, pero los que deberían ser los primeros en adelantarse para aceptar la luz del cielo, se han negado resueltamente a creer en la revelación de la verdad de Dios ---Dios revelado en el hombre y el hombre elevado a Dios>>.

\par 
%\textsuperscript{(1906.5)}
\textsuperscript{175:1.8} <<Esta tarde, mis apóstoles están aquí delante de vosotros en silencio, pero pronto escucharéis sus voces anunciando la llamada a la salvación y la incitación a unirse al reino celestial como hijos del Dios vivo. Y ahora, tomo por testigos a mis discípulos y a los creyentes en el evangelio del reino, así como a los mensajeros invisibles que están a su lado, de que he ofrecido una vez más, a Israel y a sus dirigentes, la liberación y la salvación. Pero todos observáis que la misericordia del Padre es despreciada y que los mensajeros de la verdad son rechazados. Sin embargo, os advierto que esos escribas y fariseos aún están sentados en el puesto de Moisés; por lo tanto, hasta que los Altísimos que gobiernan en los reinos de los hombres no hayan demolido finalmente esta nación y destruido el lugar donde se encuentran sus dirigentes, os pido que cooperéis con esos ancianos de Israel. No es necesario que os unáis a ellos en sus planes para destruir al Hijo del Hombre, pero en todo lo relacionado con la paz de Israel, debéis someteros a ellos. En todas esas cuestiones, haced todo lo que os ordenen y guardad lo esencial de la ley, pero no imitéis sus malas acciones. Recordad que éste es el pecado de esos gobernantes: Dicen lo que es bueno, pero no lo hacen. Sabéis bien que esos dirigentes echan sobre vuestros hombros unas cargas pesadas, unas cargas penosas de llevar, y que no levantarán ni un solo dedo para ayudaros a llevar esas pesadas cargas. Os han oprimido con ceremonias y esclavizado con tradiciones>>.

\par 
%\textsuperscript{(1907.1)}
\textsuperscript{175:1.9} <<Además, a esos dirigentes egocéntricos les deleita hacer sus buenas obras de manera que puedan ser vistos por los hombres. Agrandan sus filacterias y ensanchan los bordes de sus vestidos oficiales. Anhelan los sitios principales en los banquetes y exigen los asientos de honor en las sinagogas. Codician los saludos elogiosos en las plazas públicas y desean que todos los hombres los llamen rabinos. Y mientras buscan ser honrados así por los hombres, se apoderan en secreto de las casas de las viudas y sacan provecho de los servicios del templo sagrado. Esos hipócritas simulan hacer largas oraciones en público, y dan limosnas para atraer la atención de sus semejantes>>.

\par 
%\textsuperscript{(1907.2)}
\textsuperscript{175:1.10} <<Aunque debéis honrar a vuestros dirigentes y respetar a vuestros educadores, no debéis llamar Padre a ningún hombre en el sentido espiritual, porque uno solo es vuestro Padre, y ese es Dios. No tratéis tampoco de dominar a vuestros hermanos en el reino. Recordad que os he enseñado que aquel que quiera ser el más grande entre vosotros, debe convertirse en el servidor de todos. Si os atrevéis a exaltaros delante de Dios, sin duda seréis humillados; pero aquel que se humilla sinceramente, será exaltado con toda seguridad. En vuestra vida diaria, no busquéis vuestra propia glorificación, sino la gloria de Dios. Someted inteligentemente vuestra propia voluntad a la voluntad del Padre que está en los cielos>>.

\par 
%\textsuperscript{(1907.3)}
\textsuperscript{175:1.11} <<No interpretéis mal mis palabras. No albergo ninguna mala intención hacia esos jefes de los sacerdotes y los dirigentes que en este mismo momento intentan destruirme; no tengo ninguna aversión contra esos escribas y fariseos que rechazan mis enseñanzas. Sé que muchos de vosotros creéis en secreto, y sé que confesaréis abiertamente vuestra lealtad hacia el reino cuando llegue mi hora. Pero, ¿cómo se justificarán vuestros rabinos, que declaran hablar con Dios y luego se atreven a rechazar y destruir a aquel que viene a revelar el Padre a los mundos?>>

\par 
%\textsuperscript{(1907.4)}
\textsuperscript{175:1.12} <<¡Ay de vosotros, escribas y fariseos, hipócritas! Quisierais cerrar las puertas del reino de los cielos a los hombres sinceros, sólo porque ignoran los caminos de vuestra enseñanza. Os negáis a entrar en el reino, y al mismo tiempo hacéis todo lo que podéis para impedir que entren todos los demás. Permanecéis de espaldas a las puertas de la salvación, y lucháis contra todos los que quieren entrar>>.

\par 
%\textsuperscript{(1907.5)}
\textsuperscript{175:1.13} <<¡Ay de vosotros, escribas y fariseos, tan hipócritas como sois! Porque recorréis en verdad la tierra y el mar para hacer un prosélito, y cuando lo habéis conseguido, no os sentís satisfechos hasta hacerlo dos veces peor de lo que era como hijo de los paganos>>.

\par 
%\textsuperscript{(1907.6)}
\textsuperscript{175:1.14} <<¡Ay de vosotros, sacerdotes principales y dirigentes, que os adueñáis de los bienes de los pobres y exigís impuestos opresivos a los que quieren servir a Dios como creen que Moisés lo ordenó! Vosotros, que os negáis a mostrar misericordia, ¿podéis esperar misericordia en los mundos venideros?>>

\par 
%\textsuperscript{(1907.7)}
\textsuperscript{175:1.15} <<¡Ay de vosotros, falsos educadores y guías ciegos! ¿Qué se puede esperar de una nación cuando los ciegos conducen a los ciegos? Los dos tropezarán y caerán al abismo de la destrucción>>.

\par 
%\textsuperscript{(1907.8)}
\textsuperscript{175:1.16} <<¡Ay de vosotros que disimuláis cuando prestáis juramento! Sois unos tramposos, porque enseñáis que un hombre puede jurar por el templo y violar su juramento; pero que si cualquiera jura por el oro del templo, debe permanecer atado a su juramento. Todos sois necios y ciegos. Ni siquiera sois consistentes en vuestra deshonestidad, porque, ¿que es más grande, el oro o el templo que supuestamente ha santificado al oro? También enseñáis que si un hombre jura por el altar, no significa nada; pero que si alguien jura por la ofrenda que está en el altar, entonces será tenido por deudor. De nuevo estáis ciegos ante la verdad, porque ¿qué es más grande, la ofrenda o el altar que santifica la ofrenda? ¿Cómo podéis justificar una hipocresía y una deshonestidad semejantes a los ojos del Dios del cielo?>>

\par 
%\textsuperscript{(1908.1)}
\textsuperscript{175:1.17} <<¡Ay de vosotros, escribas y fariseos, y todos los demás hipócritas, que os aseguráis de pagar el diezmo de la menta, el anís y el comino, y al mismo tiempo descuidáis los asuntos más importantes de la ley ---la fe, la misericordia y el juicio! Dentro de lo razonable, deberíais hacer lo primero sin dejar de hacer lo segundo. Sois realmente unos guías ciegos y unos educadores estúpidos; filtráis los mosquitos y os tragáis los camellos>>.

\par 
%\textsuperscript{(1908.2)}
\textsuperscript{175:1.18} <<¡Ay de vosotros, escribas, fariseos e hipócritas! pues limpiáis escrupulosamente el exterior de la copa y del plato, pero dentro permanece la inmundicia de la extorsión, los excesos y el engaño. Estáis espiritualmente ciegos. ¿No reconocéis que sería mucho mejor limpiar primero el interior de la copa, y luego lo que rebosa limpiaría por sí mismo el exterior? ¡Réprobos perversos! Ejecutáis los actos exteriores de vuestra religión para cumplir literalmente con vuestra interpretación de la ley de Moisés, mientras que vuestras almas están impregnadas de iniquidad y llenas de intenciones asesinas>>.

\par 
%\textsuperscript{(1908.3)}
\textsuperscript{175:1.19} <<¡Ay de todos vosotros que rechazáis la verdad y despreciáis la misericordia! Muchos de vosotros os parecéis a los sepulcros blanqueados, que aparecen hermosos por fuera, pero por dentro están llenos de huesos de muertos y de todo tipo de impurezas. Así es como vosotros, que rechazáis a sabiendas el consejo de Dios, aparecéis exteriormente ante los hombres como santos y rectos, pero por dentro vuestro corazón está lleno de hipocresía y de iniquidad>>.

\par 
%\textsuperscript{(1908.4)}
\textsuperscript{175:1.20} <<¡Ay de vosotros, guías falsos de una nación! Habéis construido allí un monumento a los antiguos profetas martirizados, mientras conspiráis para destruir a Aquel de quien ellos hablaban. Adornáis las tumbas de los justos y presumís de que si hubierais vivido en la época de vuestros padres, no hubierais matado a los profetas; y luego, a pesar de este pensamiento presuntuoso, os preparáis para asesinar a aquel de quien hablaban los profetas: el Hijo del Hombre. En vista de que hacéis estas cosas, testificáis contra vosotros mismos de que sois los hijos perversos de aquellos que mataron a los profetas. ¡Continuad pues, y llenad hasta el borde la copa de vuestra condenación!>>

\par 
%\textsuperscript{(1908.5)}
\textsuperscript{175:1.21} <<¡Ay de vosotros, hijos del mal! Juan os llamó con razón los hijos de las víboras, y yo os pregunto: ¿cómo podéis escapar al juicio que Juan pronunció contra vosotros?>>

\par 
%\textsuperscript{(1908.6)}
\textsuperscript{175:1.22} <<Pero incluso ahora os ofrezco, en nombre de mi Padre, la misericordia y el perdón; incluso ahora os tiendo la mano amorosa de la hermandad eterna. Mi Padre os ha enviado a los sabios y a los profetas; habéis perseguido a unos y habéis matado a los otros. Luego apareció Juan, proclamando la llegada del Hijo del Hombre, y lo destruisteis después de que muchos hubieran creído en sus enseñanzas. Y ahora os preparáis para derramar más sangre inocente. ¿No comprendéis que llegará un día terrible de rendición de cuentas, cuando el Juez de toda la Tierra exija a este pueblo que explique por qué ha rechazado, perseguido y destruido a estos mensajeros del cielo? ¿No comprendéis que debéis rendir cuentas por toda esta sangre justa, desde el primer profeta asesinado hasta la época de Zacarías, a quien le quitaron la vida entre el santuario y el altar? Si continuáis por ese camino perverso, quizás esta rendición de cuentas le sea requerida a esta misma generación>>.

\par 
%\textsuperscript{(1908.7)}
\textsuperscript{175:1.23} <<¡Oh Jerusalén e hijos de Abraham, vosotros que habéis lapidado a los profetas y matado a los instructores que os fueron enviados, incluso ahora quisiera reunir a vuestros hijos como la gallina reúne a sus polluelos debajo de sus alas, pero no queréis!>>

\par 
%\textsuperscript{(1908.8)}
\textsuperscript{175:1.24} <<Y ahora me despido de vosotros. Habéis escuchado mi mensaje y habéis tomado vuestra decisión. Aquellos que han creído en mi evangelio ya están a salvo en el reino de Dios. A vosotros, que habéis escogido rechazar el regalo de Dios, os digo que no me veréis más enseñar en el templo. Mi trabajo a favor de vosotros ha terminado. ¡Mirad, ahora salgo con mis hijos, y os dejo vuestra casa desolada!>>

\par 
%\textsuperscript{(1908.9)}
\textsuperscript{175:1.25} A continuación, el Maestro hizo señas a sus seguidores para que salieran del templo.

\section*{2. La condición de los judíos}
\par 
%\textsuperscript{(1909.1)}
\textsuperscript{175:2.1} El hecho de que los dirigentes espirituales y los educadores religiosos de la nación judía rechazaran en otra época las enseñanzas de Jesús y conspiraran para provocar su muerte cruel, no afecta para nada a la situación de cada judío en su posición ante Dios. Este hecho no debería incitar a los que afirman ser seguidores de Cristo a tener prejuicios contra los judíos como compañeros mortales. Los judíos como nación y como grupo sociopolítico pagaron plenamente el precio terrible de rechazar al Príncipe de la Paz. Hace mucho tiempo que dejaron de ser los portadores espirituales de la verdad divina para las razas de la humanidad, pero esto no constituye una razón válida para que los descendientes individuales de aquellos antiguos judíos tengan que sufrir las persecuciones que les han infligido algunos seguidores declarados, intolerantes, indignos y fanáticos de Jesús de Nazaret, el cual era también judío de nacimiento.

\par 
%\textsuperscript{(1909.2)}
\textsuperscript{175:2.2} Este odio y esta persecución irrazonables, tan diferentes al espíritu de Cristo, contra los judíos modernos, ha terminado muchas veces en el sufrimiento y la muerte de algún judío inocente e inofensivo, cuyos mismos antepasados, en los tiempos de Jesús, habían aceptado sinceramente su evangelio y luego murieron sin vacilar por aquella verdad en la que creían de todo corazón. ¡Qué estremecimiento de horror recorre a los seres celestiales espectadores, cuando contemplan a los seguidores declarados de Jesús entregarse a perseguir, acosar e incluso asesinar a los descendientes actuales de Pedro, Felipe y Mateo, y de otros judíos palestinos que dieron sus vidas tan gloriosamente como primeros mártires del evangelio del reino de los cielos!

\par 
%\textsuperscript{(1909.3)}
\textsuperscript{175:2.3} ¡Cuán cruel e irrazonable es obligar a unos niños inocentes a que sufran por los pecados de sus progenitores, por unos delitos que ignoran por completo, de los que no pueden ser responsables de ninguna manera! ¡Y llevar a cabo estas acciones perversas en nombre de aquel que enseñó a sus discípulos a que amaran incluso a sus enemigos! En este relato de la vida de Jesús, ha sido necesario describir la manera en que algunos de sus compatriotas judíos lo rechazaron y conspiraron para provocar su muerte ignominiosa; pero queremos advertir a todos los que lean esta narración que la presentación de este relato histórico no justifica de ninguna manera el odio injusto, ni perdona la actitud mental sin equidad, que tantos cristianos declarados han mantenido durante muchos siglos contra personas judías. Los creyentes en el reino, los que siguen las enseñanzas de Jesús, deben dejar de maltratar al judío individual como alguien culpable del rechazo y de la crucifixión de Jesús. El Padre y su Hijo Creador nunca han dejado de amar a los judíos. Dios no hace acepción de personas, y la salvación es tanto para los judíos como para los gentiles.

\section*{3. La nefasta reunión del sanedrín}
\par 
%\textsuperscript{(1909.4)}
\textsuperscript{175:3.1} La nefasta reunión del sanedrín fue convocada para este martes a las ocho de la noche. En muchas ocasiones anteriores, este tribunal supremo de la nación judía había decretado de manera no oficial la muerte de Jesús. Este augusto cuerpo gobernante había decidido muchas veces poner fin a la obra del Maestro, pero nunca antes había resuelto arrestarlo y provocar su muerte a toda costa. Poco antes de la medianoche de este martes 4 de abril del año 30, el sanedrín, tal como estaba constituido en ese momento, votó oficialmente y por \textit{unanimidad} imponer la sentencia de muerte tanto a Jesús como a Lázaro. Ésta fue la respuesta al último llamamiento del Maestro a los dirigentes de los judíos, un llamamiento que había hecho en el templo tan sólo unas horas antes; esta respuesta representaba su reacción de amargo resentimiento hacia Jesús por su última y vigorosa acusación contra estos mismos sacerdotes principales, y saduceos y fariseos impenitentes. La condena a muerte del Hijo de Dios (incluso antes de su juicio) fue la contestación del sanedrín a la última oferta de misericordia celestial que jamás fuera concedida a la nación judía como tal.

\par 
%\textsuperscript{(1910.1)}
\textsuperscript{175:3.2} A partir de aquel momento, los judíos fueron dejados solos para que terminaran su breve y corto período de vida nacional, en total acuerdo con su posición puramente humana entre las naciones de Urantia. Israel había repudiado al Hijo del Dios que había hecho una alianza con Abraham; y el plan de que los hijos de Abraham fueran los portadores de la luz de la verdad en el mundo se había hecho añicos. La alianza divina se había anulado, y el final de la nación hebrea se aproximaba rápidamente.

\par 
%\textsuperscript{(1910.2)}
\textsuperscript{175:3.3} Los funcionarios del sanedrín recibieron la orden de arrestar a Jesús a la mañana siguiente temprano, pero con las instrucciones de que no debía ser apresado en público. Les dijeron que planearan arrestarlo en secreto, preferiblemente de manera repentina y de noche. Comprendiendo que quizás aquel día (miércoles) no regresaría para enseñar en el templo, indicaron a aquellos oficiales del sanedrín que <<lo trajeran ante el alto tribunal judío en cualquier momento antes de la medianoche del jueves>>.

\section*{4. La situación en Jerusalén}
\par 
%\textsuperscript{(1910.3)}
\textsuperscript{175:4.1} Al final del último discurso de Jesús en el templo, los apóstoles se quedaron una vez más confundidos y consternados. Judas había regresado al templo antes de que el Maestro empezara su terrible acusación contra los dirigentes judíos, de manera que los doce al completo escucharon la segunda mitad del último discurso de Jesús en el templo. Es una pena que Judas Iscariote no pudiera escuchar la primera mitad de este discurso de despedida, que ofrecía la misericordia. No escuchó esta última oferta de misericordia a los dirigentes judíos porque aún estaba conferenciando con un grupo de parientes y amigos saduceos con quienes había almorzado, y con quienes estaba conversando sobre la manera más adecuada de separarse de Jesús y de sus compañeros apóstoles. Mientras escuchaba la acusación final del Maestro contra los jefes y dirigentes judíos, Judas tomó su decisión final y completa de abandonar el movimiento evangélico y de lavarse las manos de toda esta empresa. Sin embargo, salió del templo en compañía de los doce y se dirigió con ellos al Monte de los Olivos, donde escuchó, con sus compañeros apóstoles, el discurso fatídico sobre la destrucción de Jerusalén y el final de la nación judía. Aquella noche del martes, Judas permaneció con ellos en el nuevo campamento cerca de Getsemaní.

\par 
%\textsuperscript{(1910.4)}
\textsuperscript{175:4.2} La multitud se quedó atónita y desconcertada cuando escuchó a Jesús pasar de su llamamiento misericordioso a los dirigentes judíos, a una reprimenda repentina y mordaz que rozaba la acusación sin piedad. Aquella noche, mientras el sanedrín pronunciaba la sentencia de muerte contra Jesús, y el Maestro estaba sentado con sus apóstoles y algunos de sus discípulos en el Monte de los Olivos, prediciendo la muerte de la nación judía, todo Jerusalén se había entregado a la discusión seria y callada de una sola pregunta: <<¿Qué van a hacer con Jesús?>>

\par 
%\textsuperscript{(1910.5)}
\textsuperscript{175:4.3} En la casa de Nicodemo, más de treinta judíos eminentes que creían en secreto en el reino se reunieron para debatir la conducta a seguir en el caso de que se produjera una ruptura abierta con el sanedrín. Todos los presentes acordaron que reconocerían abiertamente su lealtad al Maestro en cuanto se enteraran de su arresto. Y eso fue exactamente lo que hicieron.

\par 
%\textsuperscript{(1911.1)}
\textsuperscript{175:4.4} Los saduceos, que ahora controlaban y dominaban el sanedrín, deseaban eliminar a Jesús por las razones siguientes:

\par 
%\textsuperscript{(1911.2)}
\textsuperscript{175:4.5} 1. Temían que el creciente favor popular con que la multitud consideraba a Jesús amenazara con poner en peligro la existencia de la nación judía, debido a una posible complicación con las autoridades romanas.

\par 
%\textsuperscript{(1911.3)}
\textsuperscript{175:4.6} 2. El ardor de Jesús por la reforma del templo afectaba directamente a sus ingresos; la depuración del templo perjudicaba sus bolsillos.

\par 
%\textsuperscript{(1911.4)}
\textsuperscript{175:4.7} 3. Se sentían responsables de la preservación del orden social, y temían las consecuencias de una expansión posterior de la nueva y extraña doctrina de Jesús sobre la fraternidad de los hombres.

\par 
%\textsuperscript{(1911.5)}
\textsuperscript{175:4.8} Los fariseos tenían unos motivos diferentes para desear quitarle la vida a Jesús. Le tenían miedo porque:

\par 
%\textsuperscript{(1911.6)}
\textsuperscript{175:4.9} 1. Se había opuesto eficazmente al dominio tradicional que los fariseos ejercían sobre el pueblo. Los fariseos eran ultraconservadores, y les encolerizaba amargamente estos ataques, supuestamente radicales, contra su prestigio establecido como instructores religiosos.

\par 
%\textsuperscript{(1911.7)}
\textsuperscript{175:4.10} 2. Sostenían que Jesús violaba la ley; que había mostrado un desprecio total por el sábado y por otras numerosas exigencias legales y ceremoniales.

\par 
%\textsuperscript{(1911.8)}
\textsuperscript{175:4.11} 3. Lo acusaban de blasfemia porque se refería a Dios como si fuera su Padre.

\par 
%\textsuperscript{(1911.9)}
\textsuperscript{175:4.12} 4. Y ahora estaban profundamente irritados contra él a causa del último discurso que había pronunciado aquel día en el templo, donde en la parte final de su alocución de despedida los acusaba severamente.

\par 
%\textsuperscript{(1911.10)}
\textsuperscript{175:4.13} Una vez que hubo decretado oficialmente la muerte de Jesús y dado las órdenes para su arresto, el sanedrín levantó la sesión este martes cerca de la medianoche, después de acordar una reunión para las diez de la mañana del día siguiente en la casa del sumo sacerdote Caifás, con el fin de formular las acusaciones que permitirían llevar a Jesús a juicio.

\par 
%\textsuperscript{(1911.11)}
\textsuperscript{175:4.14} Un pequeño grupo de saduceos había llegado a proponer que se deshicieran de Jesús mediante el asesinato, pero los fariseos se negaron terminantemente a apoyar este procedimiento.

\par 
%\textsuperscript{(1911.12)}
\textsuperscript{175:4.15} Ésta era la situación en Jerusalén y entre los hombres en este día lleno de acontecimientos, mientras una enorme multitud de seres celestiales se cernía sobre esta importante escena en la Tierra, impacientes por hacer algo para ayudar a su amado Soberano, pero sin poder actuar porque los superiores que los dirigían se lo habían prohibido expresamente.


\chapter{Documento 176. El martes por la noche en el Monte de los Olivos}
\par 
%\textsuperscript{(1912.1)}
\textsuperscript{176:0.1} ESTE martes por la tarde, cuando Jesús y los apóstoles salían del templo para ir al campamento de Getsemaní, Mateo llamó la atención sobre la estructura del templo, y dijo: <<Maestro, observa el aspecto de estos edificios. Mira las piedras macizas y los hermosos adornos; ¿es posible que estos edificios vayan a ser destruidos?>> Mientras continuaban hacia el Olivete, Jesús dijo: <<Estáis viendo estas piedras y este templo macizo; en verdad, en verdad os digo que en los días que pronto llegarán, no quedará piedra sobre piedra. Todas serán derribadas>>. Estas observaciones que describían la destrucción del templo sagrado despertaron la curiosidad de los apóstoles mientras caminaban detrás del Maestro; no podían concebir ningún acontecimiento, como no fuera el fin del mundo, que pudiera ocasionar la destrucción del templo.

\par 
%\textsuperscript{(1912.2)}
\textsuperscript{176:0.2} Para evitar las multitudes que pasaban por el valle de Cedrón hacia Getsemaní, Jesús y sus compañeros tenían la intención de subir la pendiente occidental del Olivete durante una corta distancia, y luego seguir un sendero que conducía a su campamento privado, situado cerca de Getsemaní, a corta distancia por encima del campamento público. Cuando se desviaban para abandonar el camino que conducía a Betania, contemplaron el templo, glorificado por los rayos del Sol poniente; y mientras se detenían en el monte, vieron aparecer las luces de la ciudad y contemplaron la belleza del templo iluminado; y allí, bajo la suave luz de la Luna llena, Jesús y los doce se sentaron. El Maestro conversó con ellos, y Natanael hizo enseguida la pregunta siguiente: <<Dinos Maestro, ¿cómo sabremos que esos acontecimientos están a punto de suceder?>>

\section*{1. La destrucción de Jerusalén}
\par 
%\textsuperscript{(1912.3)}
\textsuperscript{176:1.1} En respuesta a la pregunta de Natanael, Jesús dijo: <<Sí, voy a hablaros de la época en que este pueblo habrá llenado la copa de su iniquidad, cuando la justicia caerá rápidamente sobre esta ciudad de nuestros padres. Estoy a punto de dejaros; voy hacia el Padre. Después de que me haya ido, tened cuidado de que nadie os engañe, porque muchos vendrán como liberadores y conducirán a mucha gente por el camino equivocado. Cuando escuchéis hablar de guerras y de rumores de guerras, no os preocupéis, porque aunque todas esas cosas sucederán, el fin de Jerusalén aún no está cerca. No os inquietéis por la hambruna o los terremotos; tampoco debéis preocuparos cuando seáis entregados a las autoridades civiles y seáis perseguidos a causa del evangelio. Seréis expulsados de la sinagoga e iréis a la cárcel por mi causa, y algunos de vosotros seréis ejecutados. Cuando seáis llevados ante los gobernadores y los dirigentes, será para dar testimonio de vuestra fe y para mostrar vuestra firmeza en el evangelio del reino. Cuando estéis en presencia de los jueces, no os inquietéis de antemano por lo que vais a decir, porque el espíritu os enseñará en esa misma hora lo que deberéis contestar a vuestros adversarios. En esos días de dolor, incluso vuestros propios parientes, bajo la dirección de los que han rechazado al Hijo del Hombre, os entregarán a la cárcel y a la muerte. Durante un tiempo, puede ser que todos los hombres os odien por mi causa, pero incluso durante esas persecuciones, no os abandonaré; mi espíritu no os dejará. ¡Tened paciencia! No dudéis de que este evangelio del reino triunfará sobre todos sus enemigos, y será proclamado finalmente a todas las naciones>>.

\par 
%\textsuperscript{(1913.1)}
\textsuperscript{176:1.2} Jesús hizo una pausa mientras contemplaba la ciudad. El Maestro se daba cuenta de que el rechazo del concepto espiritual del Mesías, la determinación de aferrarse de manera ciega y perseverante a la misión material del libertador esperado, pronto llevaría a los judíos a un conflicto directo con los poderosos ejércitos romanos, y que esa lucha sólo podía terminar con la destrucción final y completa de la nación judía. Cuando el pueblo de Jesús rechazó su donación espiritual y se negó a recibir la luz del cielo que brillaba de manera tan misericordiosa sobre ellos, sellaron así su perdición como pueblo independiente con una misión espiritual especial en la Tierra. Los mismos dirigentes judíos reconocieron posteriormente que esta idea laica del Mesías fue la que condujo directamente al alboroto que provocó finalmente su destrucción.

\par 
%\textsuperscript{(1913.2)}
\textsuperscript{176:1.3} Puesto que Jerusalén iba a ser la cuna del movimiento evangélico primitivo, Jesús no quería que sus instructores y predicadores perecieran en la terrible derrota del pueblo judío, asociada a la destrucción de Jerusalén; por eso dio estas instrucciones a sus seguidores. A Jesús le preocupaba mucho que algunos de sus discípulos se implicaran en estas revueltas venideras, y perecieran así en la caída de Jerusalén.

\par 
%\textsuperscript{(1913.3)}
\textsuperscript{176:1.4} Andrés preguntó entonces: <<Pero, Maestro, si la ciudad santa y el templo van a ser destruidos, y si tú no estás aquí para dirigirnos, ¿cuándo deberemos abandonar Jerusalén?>> Jesús dijo: <<Podéis permanecer en la ciudad después de mi partida, e incluso durante esos tiempos de dolor y de crueles persecuciones, pero cuando veáis finalmente que Jerusalén está siendo rodeada por los ejércitos romanos, después de la revuelta de los falsos profetas, entonces sabréis que su desolación está próxima; entonces deberéis huir a las montañas. Que nadie que esté en la ciudad y en sus alrededores se detenga para salvar nada, y que los que estén fuera no se atrevan a entrar. Habrá una gran tribulación, porque serán los días de la venganza de los gentiles. Después de que hayáis abandonado la ciudad, este pueblo desobediente caerá derribado por el filo de la espada y será llevado cautivo por todas las naciones; Jerusalén será así pisoteada por los gentiles. Mientras tanto, os lo advierto, no os dejéis engañar. Si alguien viene hasta vosotros diciendo: `Mirad, aquí está el Libertador', o `Mirad, está allí', no le creáis, porque surgirán muchos falsos educadores y descarriarán a mucha gente; pero vosotros no deberíais dejaros engañar, porque os he dicho todo esto por anticipado>>.

\par 
%\textsuperscript{(1913.4)}
\textsuperscript{176:1.5} Los apóstoles permanecieron mucho tiempo sentados en silencio a la luz de la Luna, mientras estas sorprendentes predicciones del Maestro se grababan en sus mentes confusas. Y fue en conformidad con esta advertencia como prácticamente todo el grupo de creyentes y discípulos huyó de Jerusalén en cuanto aparecieron las tropas romanas, encontrando un refugio seguro al norte de Pella.

\par 
%\textsuperscript{(1913.5)}
\textsuperscript{176:1.6} Incluso después de esta advertencia explícita, muchos seguidores de Jesús interpretaron estas predicciones como alusivas a los cambios que ocurrirían evidentemente en Jerusalén, cuando a la reaparición del Mesías le siguiera el establecimiento de la Nueva Jerusalén y la ampliación de la ciudad para que se convirtiera en la capital del mundo. En su mente, estos judíos estaban decididos a relacionar la destrucción del templo con el <<fin del mundo>>. Creían que esta Nueva Jerusalén ocuparía toda Palestina; que después del fin del mundo vendría la aparición inmediata de los <<nuevos cielos y de la nueva tierra>>. Por eso no es de extrañar que Pedro dijera: <<Maestro, sabemos que todas las cosas se desvanecerán cuando aparezcan los nuevos cielos y la nueva tierra, pero, ¿cómo sabremos cuándo regresarás para efectuar todo esto?>>

\par 
%\textsuperscript{(1914.1)}
\textsuperscript{176:1.7} Cuando Jesús escuchó esto, se quedó pensativo durante unos momentos y luego dijo: <<Os equivocáis continuamente porque siempre tratáis de conectar la nueva enseñanza con la antigua; estáis decididos a tergiversar toda mi enseñanza; insistís en interpretar el evangelio de acuerdo con vuestras creencias establecidas. Sin embargo, trataré de iluminaros>>.

\section*{2. La segunda venida del Maestro}
\par 
%\textsuperscript{(1914.2)}
\textsuperscript{176:2.1} En diversas ocasiones, Jesús había hecho declaraciones que condujeron a sus oyentes a deducir que, aunque se proponía dejar este mundo dentro de poco, regresaría con toda seguridad para consumar la obra del reino celestial. A medida que sus seguidores estaban más convencidos de que los iba a dejar, y después de haber partido de este mundo, era muy natural que todos los creyentes se aferraran firmemente a estas promesas de regresar. Y así, la doctrina de la segunda venida de Cristo se incorporó pronto en las enseñanzas de los cristianos, y casi todas las generaciones posteriores de discípulos han creído devotamente en esta verdad y han esperado con confianza que regresaría algún día.

\par 
%\textsuperscript{(1914.3)}
\textsuperscript{176:2.2} Puesto que debían separarse de su Maestro e Instructor, estos primeros discípulos y los apóstoles se aferraron mucho más a esta promesa de regresar, y no tardaron en asociar la vaticinada destrucción de Jerusalén con esta segunda venida prometida. Y continuaron interpretando de esta manera sus palabras, a pesar de que el Maestro, durante todo este anochecer de enseñanza en el Monte de los Olivos, se tomó el enorme trabajo de impedir precisamente este error.

\par 
%\textsuperscript{(1914.4)}
\textsuperscript{176:2.3} En su contestación adicional a la pregunta de Pedro, Jesús dijo: <<¿Por qué continuáis creyendo que el Hijo del Hombre se sentará en el trono de David, y esperáis que se cumplan los sueños materiales de los judíos? ¿No os he dicho todos estos años que mi reino no es de este mundo? Las cosas que ahora contempláis a vuestros pies están llegando a su fin, pero éste será un nuevo comienzo, a partir del cual el evangelio del reino se extenderá por todo el mundo, y esta salvación se difundirá a todos los pueblos. Cuando el reino haya llegado a su plena madurez, estad seguros de que el Padre que está en los cielos no dejará de visitaros con una revelación ampliada de la verdad y con una demostración realzada de la rectitud, tal como ya ha otorgado a este mundo a aquel que se convirtió en el príncipe de las tinieblas, y luego a Adán, que fue seguido por Melquisedek, y en nuestros días, al Hijo del Hombre. Mi Padre continuará así manifestando su misericordia y mostrando su amor, incluso a este mundo oscuro y malvado. Después de que mi Padre me haya investido con todo el poder y la autoridad, yo también continuaré siguiendo vuestra suerte y guiándoos en los asuntos del reino mediante la presencia de mi espíritu, que pronto será derramado sobre todo el género humano. Aunque así estaré presente con vosotros en espíritu, también prometo que regresaré algún día a este mundo donde he vivido esta vida en la carne y he logrado la experiencia simultánea de revelar a Dios a los hombres y de conducir los hombres hacia Dios. Tengo que dejaros muy pronto y reemprender el trabajo que el Padre me ha confiado, pero tened buen ánimo, porque volveré algún día. Mientras tanto, mi Espíritu de la Verdad de un universo os confortará y os guiará>>.

\par 
%\textsuperscript{(1915.1)}
\textsuperscript{176:2.4} <<Ahora me veis débil y en la carne, pero cuando regrese será con poder y en el espíritu. Los ojos de la carne contemplan al Hijo del Hombre en la carne, pero sólo los ojos del espíritu contemplarán al Hijo del Hombre glorificado por el Padre y apareciendo en la Tierra en su propio nombre>>.

\par 
%\textsuperscript{(1915.2)}
\textsuperscript{176:2.5} <<Pero la época de la reaparición del Hijo del Hombre sólo se conoce en los consejos del Paraíso; ni siquiera los ángeles del cielo saben cuándo sucederá esto. Sin embargo, deberíais comprender que cuando este evangelio del reino haya sido proclamado en el mundo entero para la salvación de todos los pueblos, y cuando la era haya alcanzado su plenitud, el Padre os enviará otra donación dispensacional, o si no, el Hijo del Hombre regresará para juzgar la era>>.

\par 
%\textsuperscript{(1915.3)}
\textsuperscript{176:2.6} <<Y ahora, en lo que se refiere a las tribulaciones de Jerusalén, de las cuales os he hablado, esta generación no pasará hasta que se cumplan mis palabras; pero en lo que respecta a la época de la nueva venida del Hijo del Hombre, nadie en el cielo o en la Tierra puede atreverse a hablar de ello. Pero deberíais ser sabios en lo que se refiere a la maduración de una era; deberíais estar alertas para discernir los signos de los tiempos. Cuando la higuera muestra sus ramas tiernas y brotan sus hojas, sabéis que el verano está cerca. De la misma manera, cuando el mundo haya pasado por el largo invierno de la mentalidad materialista y discernáis la venida de la primavera espiritual de una nueva dispensación, deberíais saber que se acerca el verano de una nueva visita>>.

\par 
%\textsuperscript{(1915.4)}
\textsuperscript{176:2.7} <<Pero, ¿cuál es el significado de esta enseñanza relacionada con la venida de los Hijos de Dios? ¿No os dais cuenta de que cuando cada uno de vosotros sea llamado a abandonar la lucha de la vida y a traspasar la puerta de la muerte estará en la presencia inmediata del juicio, frente a frente con los hechos de una nueva dispensación de servicio en el plan eterno del Padre infinito? Aquello a lo que el mundo entero debe de hecho enfrentarse literalmente al final de una era, cada uno de vosotros, como individuo, tiene que enfrentarse con toda seguridad, como experiencia personal, cuando llegue al final de su vida física, y con ello pase a enfrentarse a las condiciones y a las exigencias inherentes a la revelación siguiente de la evolución eterna del reino del Padre>>.

\par 
%\textsuperscript{(1915.5)}
\textsuperscript{176:2.8} De todos los discursos que el Maestro dio a sus apóstoles, ninguno causó nunca tanta confusión en sus mentes como éste, pronunciado este martes por la noche en el Monte de los Olivos, sobre el doble tema de la destrucción de Jerusalén y de su propia segunda venida. Por consiguiente, las narraciones escritas posteriormente, basadas en los recuerdos de lo que el Maestro había dicho en esta ocasión extraordinaria, concordaron poco entre sí. En consecuencia, como los relatos dejaron en blanco muchas cosas que se dijeron este martes por la noche, surgieron muchas tradiciones. A principios del siglo segundo, un apocalipsis judío sobre el Mesías, escrito por un tal Selta, que estaba ligado a la corte del emperador Calígula, fue íntegramente copiado en el Evangelio según Mateo, y posteriormente añadido (en parte) a los relatos de Marcos y de Lucas. En estos escritos de Selta fue donde apareció la parábola de las diez vírgenes. Ninguna parte de los escritos evangélicos sufrió nunca una interpretación errónea tan confusa como la enseñanza de esta noche. Pero el apóstol Juan nunca se dejó confundir de esta manera.

\par 
%\textsuperscript{(1915.6)}
\textsuperscript{176:2.9} Mientras estos trece hombres reanudaban su camino hacia el campamento, permanecían callados y bajo los efectos de una gran tensión emocional. Judas había ratificado finalmente su decisión de abandonar a sus compañeros. Ya era tarde cuando David Zebedeo, Juan Marcos y cierto número de discípulos principales recibieron a Jesús y a los doce en el nuevo campamento, pero los apóstoles no querían dormir; querían saber más cosas sobre la destrucción de Jerusalén, la partida del Maestro y el fin del mundo.

\section*{3. La conversación posterior en el campamento}
\par 
%\textsuperscript{(1916.1)}
\textsuperscript{176:3.1} Mientras unos veinte de ellos se reunían alrededor del fuego del campamento, Tomás preguntó: <<Puesto que tienes que volver para terminar la obra del reino, ¿cuál ha de ser nuestra actitud mientras estás lejos, ocupado en los asuntos del Padre?>> Jesús los miró a la luz del fuego y respondió:

\par 
%\textsuperscript{(1916.2)}
\textsuperscript{176:3.2} <<Tomás, tú tampoco logras comprender lo que he estado diciendo. ¿No te he enseñado todo este tiempo que tu relación con el reino es espiritual e individual, que es totalmente un asunto de experiencia personal en el espíritu mediante la comprensión, por la fe, de que eres un hijo de Dios? ¿Qué puedo decir más? La caída de las naciones, el desplome de los imperios, la destrucción de los judíos incrédulos, el final de una era e incluso el fin del mundo, ¿qué tienen que ver estas cosas con alguien que cree en este evangelio, y que ha refugiado su vida en la seguridad del reino eterno? Vosotros que conocéis a Dios y que creéis en el evangelio, ya habéis recibido las seguridades de la vida eterna. Puesto que vuestra vida ha sido vivida en el espíritu y para el Padre, nada os puede preocupar seriamente. Los constructores del reino, los ciudadanos acreditados de los mundos celestiales, no deben inquietarse por los trastornos temporales o perturbarse por los cataclismos terrestres. A vosotros que creéis en este evangelio del reino, ¿qué os importa que se derrumben las naciones, que se termine la era o que estallen todas las cosas visibles, puesto que sabéis que vuestra vida es el don del Hijo, y que está eternamente segura en el Padre? Como habéis vivido la vida temporal por la fe, y habéis producido los frutos del espíritu con la rectitud del servicio amoroso hacia vuestros semejantes, podéis contemplar con confianza el siguiente paso de la carrera eterna, con la misma fe en la supervivencia que os ha hecho atravesar vuestra primera aventura terrenal de filiación con Dios>>.

\par 
%\textsuperscript{(1916.3)}
\textsuperscript{176:3.3} <<Cada generación de creyentes debería continuar su trabajo con vistas al posible regreso del Hijo del Hombre, exactamente como cada creyente individual lleva adelante el trabajo de su vida con vistas a la inevitable muerte natural siempre amenazante. Una vez que os habéis establecido por la fe como hijos de Dios, no importa ninguna otra cosa en lo que respecta a la seguridad de la supervivencia. ¡Pero no os engañéis! Esta fe en la supervivencia es una fe viva, y manifiesta cada vez más los frutos de ese espíritu divino que al principio la inspiró en el corazón humano. El hecho de que hayáis aceptado anteriormente la filiación en el reino celestial, no os salvará si rechazáis a sabiendas y de manera persistente las verdades relacionadas con la producción progresiva de los frutos espirituales de los hijos de Dios en la carne. Vosotros, que habéis estado conmigo en los asuntos terrestres del Padre, incluso ahora podéis abandonar el reino si descubrís que no amáis el camino del servicio del Padre para la humanidad>>.

\par 
%\textsuperscript{(1916.4)}
\textsuperscript{176:3.4} <<Como individuos y como generación de creyentes, escuchadme mientras os cuento una parábola: Había un hombre importante que, antes de partir para un largo viaje a otro país, convocó a todos sus servidores de confianza y les entregó todos sus bienes. A uno le dio cinco talentos, a otro dos y a otro uno, y así sucesivamente a todo el grupo de fieles administradores. A cada uno le confió sus bienes según sus capacidades variadas, y luego salió de viaje. Cuando este señor hubo partido, sus servidores se pusieron a trabajar para sacarle provecho a las riquezas que les habían confiado. El que había recibido cinco talentos empezó inmediatamente a negociar con ellos, y muy pronto obtuvo un beneficio de otros cinco talentos. De la misma manera, el que había recibido dos talentos pronto había ganado dos más. Y así, todos aquellos servidores consiguieron beneficios para su señor, excepto aquel que sólo había recibido un talento. Se marchó solo y cavó un hoyo en la tierra, donde escondió el dinero de su señor. Pronto, el señor de aquellos servidores regresó inesperadamente y llamó a sus administradores para que le rindieran cuentas. Cuando todos se encontraron delante de su amo, el que había recibido los cinco talentos se adelantó con el dinero que se le había confiado y aportó cinco talentos adicionales, diciendo: `Señor, me diste cinco talentos para invertirlos, y me alegra entregarte otros cinco talentos que he ganado.' Entonces su señor le dijo: `Bien hecho, mi buen y fiel servidor, has sido fiel en las pocas cosas; ahora te estableceré como administrador de muchas cosas; comparte inmediatamente la alegría de tu señor.' Luego, el que había recibido los dos talentos se adelantó diciendo: `Señor, me entregaste dos talentos; mira, he ganado estos otros dos talentos.' Y su señor le dijo entonces: `Bien hecho, mi buen y fiel administrador; tú también has sido fiel en las pocas cosas y ahora te pondré a cargo de muchas; comparte la alegría de tu señor.' Entonces se presentó para rendir cuentas el que había recibido un solo talento. Este servidor se adelantó, diciendo:`Señor, yo te conocía y me daba cuenta de que eras un hombre astuto, en el sentido de que esperabas unos beneficios allí donde no habías trabajado personalmente; por eso tenía miedo de arriesgar algo de lo que se me había confiado. Escondí tu talento en un lugar seguro en la tierra; aquí está; ahora tienes lo que es tuyo.' Pero su señor respondió: `Eres un administrador indolente y perezoso. Confiesas con tus propias palabras que sabías que yo te exigiría una rendición de cuentas con unos beneficios razonables, como las que me han rendido hoy tus diligentes compañeros. Por lo tanto, sabiendo esto, al menos deberías haber entregado mi dinero a los banqueros para que, a mi regreso, pudiera recibir lo que es mío más los intereses.' Entonces este señor dijo al administrador principal: `Quítale ese único talento a este servidor inútil y dáselo al que tiene diez talentos.'>>

\par 
%\textsuperscript{(1917.1)}
\textsuperscript{176:3.5} <<A todo el que tiene, se le dará más y poseerá en abundancia; pero a aquel que no tiene, incluso lo que tiene se le quitará. No podéis permanecer inmóviles en los asuntos del reino eterno. Mi Padre exige que todos sus hijos crezcan en la gracia y en el conocimiento de la verdad. Vosotros, que conocéis estas verdades, debéis producir cada vez más frutos del espíritu y manifestar una devoción creciente al servicio desinteresado de vuestros compañeros servidores. Y recordad que, en la medida en que ayudáis al más humilde de mis hermanos, ese servicio me lo habréis hecho a mí>>.

\par 
%\textsuperscript{(1917.2)}
\textsuperscript{176:3.6} <<Así es como deberíais ocuparos de los asuntos del Padre, ahora y en el futuro, e incluso para siempre. Continuad hasta que yo regrese. Haced fielmente lo que se os ha confiado, y así estaréis preparados para la rendición de cuentas que acompaña a la muerte. Habiendo vivido así para la gloria del Padre y la satisfacción del Hijo, entraréis con alegría y un placer extremo al servicio eterno del reino perpetuo>>.

\par 
%\textsuperscript{(1917.3)}
\textsuperscript{176:3.7} La verdad es viviente; el Espíritu de la Verdad siempre está conduciendo a los hijos de la luz a unos nuevos dominios de realidad espiritual y de servicio divino. La verdad no se os da para que la cristalicéis en unas formas establecidas, seguras y veneradas. Vuestra revelación de la verdad debe ser tan realzada al pasar por vuestra experiencia personal, que ha de descubrir una nueva belleza y unos beneficios espirituales reales a todos aquellos que contemplan vuestros frutos espirituales, viéndose inducidos en consecuencia a glorificar al Padre que está en los cielos. Únicamente aquellos fieles servidores que crecen así en el conocimiento de la verdad, y que gracias a ello desarrollan la capacidad de apreciar divinamente las realidades espirituales, pueden esperar <<compartir plenamente la alegría de su Señor>>. Es triste ver a las generaciones sucesivas de seguidores declarados de Jesús, decir a propósito de su administración de la verdad divina: <<Maestro, he aquí la verdad que nos confiaste hace cien o mil años. No hemos perdido nada; hemos conservado fielmente todo lo que nos diste; no hemos permitido que se haga ningún cambio en lo que nos enseñaste; aquí está la verdad que nos diste>>. Pero este pretexto relativo a la indolencia espiritual no justificará, en presencia del Maestro, al administrador estéril de la verdad. El Maestro de la verdad os exigirá una rendición de cuentas de acuerdo con la verdad que os ha sido confiada.

\par 
%\textsuperscript{(1918.1)}
\textsuperscript{176:3.8} En el mundo siguiente se os pedirá que deis cuenta de vuestros dones y de vuestras gestiones en este mundo. Que vuestros talentos inherentes sean pocos o muchos, será necesario enfrentarse a una rendición de cuentas justa y misericordiosa. Si los dones sólo se utilizan con fines egoístas y no se presta ninguna atención al deber superior de obtener una producción creciente de los frutos del espíritu, tal como éstos se manifiestan en el servicio a los hombres y en la adoración a Dios en constante expansión, esos administradores egoístas deben aceptar las consecuencias de su elección deliberada.

\par 
%\textsuperscript{(1918.2)}
\textsuperscript{176:3.9} Cuánto se parece este servidor infiel provisto de un solo talento a todos los mortales egoístas, en el sentido de que acusó directamente a su señor de su propia pereza. Cuando un hombre se enfrenta con sus propios fracasos, ¡cuánta tendencia tiene a inculpar a los demás, con mucha frecuencia a quienes menos se lo merecen!

\par 
%\textsuperscript{(1918.3)}
\textsuperscript{176:3.10} Aquella noche, cuando se retiraban para descansar, Jesús les dijo: <<Habéis recibido gratuitamente; por eso deberíais dar gratuitamente la verdad del cielo, y al darla, esta verdad se multiplicará y mostrará la luz creciente de la gracia salvadora a medida que la prodiguéis>>.

\section*{4. El regreso de Miguel}
\par 
%\textsuperscript{(1918.4)}
\textsuperscript{176:4.1} De todas las enseñanzas del Maestro, ninguna fase ha sido tan mal comprendida como su promesa de regresar algún día en persona a este mundo. No es de extrañar que Miguel estuviera interesado en regresar algún día al planeta donde había experimentado su séptima y última donación como mortal del reino. Es muy natural creer que Jesús de Nazaret, ahora gobernante soberano de un inmenso universo, esté interesado en regresar, no solamente una vez sino muchas veces, al mundo en el que vivió una vida tan excepcional y donde ganó finalmente para sí mismo el poder y la autoridad universales que el Padre le había otorgado de manera ilimitada. Urantia será eternamente una de las siete esferas de nacimiento de Miguel, en su proceso de ganar la soberanía de un universo.

\par 
%\textsuperscript{(1918.5)}
\textsuperscript{176:4.2} Jesús declaró en numerosas ocasiones y a muchas personas su intención de regresar a este mundo. A medida que sus seguidores despertaban al hecho de que su Maestro no iba a ejercer su actividad como libertador temporal, y a medida que escuchaban sus predicciones sobre la destrucción de Jerusalén y la ruina de la nación judía, empezaron a asociar de la manera más natural su regreso prometido con estos acontecimientos catastróficos. Pero cuando los ejércitos romanos arrasaron los muros de Jerusalén, destruyeron el templo y dispersaron a los judíos de Judea, y el Maestro seguía sin revelarse con poder y gloria, sus seguidores empezaron a formular la creencia que acabó por asociar la segunda venida de Cristo con el final de la era, e incluso con el fin del mundo.

\par 
%\textsuperscript{(1918.6)}
\textsuperscript{176:4.3} Jesús prometió hacer dos cosas después de haber ascendido hacia el Padre, y una vez que todos los poderes en el cielo y en la Tierra hubieran sido puestos entre sus manos. Primero, prometió enviar al mundo, en su lugar, a otro instructor, al Espíritu de la Verdad, y lo hizo el día de Pentecostés. Y segundo, prometió con toda seguridad a sus seguidores que algún día regresaría personalmente a este mundo. Pero no dijo cómo, dónde ni cuándo volvería a visitar este planeta donde había vivido su experiencia donadora en la carne. En una ocasión insinuó que, como los ojos de la carne lo habían contemplado mientras vivía aquí, a su regreso (al menos en una de sus posibles visitas) sólo sería percibido por el ojo de la fe espiritual.

\par 
%\textsuperscript{(1919.1)}
\textsuperscript{176:4.4} Muchos de nosotros tienden a creer que Jesús regresará muchas veces a Urantia durante las eras por venir. No tenemos su promesa expresa de que hará estas múltiples visitas, pero parece muy probable que aquel que lleva entre sus títulos universales el de Príncipe Planetario de Urantia, visitará muchas veces el mundo cuya conquista le ha conferido este título tan excepcional.

\par 
%\textsuperscript{(1919.2)}
\textsuperscript{176:4.5} Creemos firmemente que Miguel volverá en persona a Urantia, pero no tenemos la menor idea de cuándo o de qué manera elegirá hacerlo. Su segunda venida a la Tierra ¿se calculará para que ocurra en conexión con el juicio final de la era presente, con o sin la aparición concomitante de un Hijo Magistral? ¿Vendrá en conexión con el final de alguna era urantiana posterior? ¿Vendrá sin anunciarse y como un acontecimiento aislado? No lo sabemos. Sólo estamos seguros de una cosa, y es que cuando regrese, probablemente todo el mundo lo sabrá, porque deberá venir como jefe supremo de un universo, y no como el oscuro recién nacido de Belén. Pero si todos los ojos han de contemplarlo, y si sólo los ojos espirituales podrán discernir su presencia, entonces su venida deberá retrasarse durante mucho tiempo.

\par 
%\textsuperscript{(1919.3)}
\textsuperscript{176:4.6} Por lo tanto, haríais bien en no asociar el regreso personal del Maestro a la Tierra con ningún acontecimiento previsto y con ninguna época determinada. Sólo estamos seguros de una cosa: Ha prometido que volverá. No tenemos ninguna idea de cuándo cumplirá esta promesa ni en relación con qué acontecimiento. Que nosotros sepamos, puede aparecer en la Tierra en cualquier momento, y puede no venir hasta que hayan pasado unas eras tras otras y hayan sido debidamente juzgadas por sus Hijos asociados del cuerpo Paradisiaco.

\par 
%\textsuperscript{(1919.4)}
\textsuperscript{176:4.7} La segunda venida de Miguel a la Tierra es un acontecimiento con un enorme valor sentimental, tanto para los intermedios como para los humanos; pero por otra parte, no tiene una importancia inmediata para los intermedios ni más importancia práctica para los seres humanos que el acontecimiento común de la muerte natural, la cual precipita repentinamente al hombre mortal en la influencia inmediata de esa sucesión de acontecimientos universales que le conducen directamente a la presencia de este mismo Jesús, el gobernante soberano de nuestro universo. Todos los hijos de la luz están destinados a verlo, y no tiene ninguna importancia que nosotros vayamos hacia él o que se dé la circunstancia de que él venga primero hacia nosotros. Estad pues siempre dispuestos a acogerlo en la Tierra, tal como él está dispuesto a acogeros en el cielo. Esperamos con confianza su gloriosa aparición, e incluso sus repetidas visitas, pero ignoramos por completo cuándo, cómo, o en relación con qué acontecimiento está destinado a aparecer.


\chapter{Documento 177. El miércoles, día de descanso}
\par 
%\textsuperscript{(1920.1)}
\textsuperscript{177:0.1} CUANDO la tarea de enseñar al pueblo no les apremiaba, Jesús y sus apóstoles tenían la costumbre de descansar de sus trabajos todos los miércoles. Este miércoles en particular tomaron el desayuno un poco más tarde que de costumbre, y el campamento estaba impregnado de un silencio de mal ag\"uero; se dijeron muy pocas palabras durante la primera mitad de esta comida matutina. Por fin, Jesús habló: <<Deseo que descanséis hoy. Dedicad tiempo para reflexionar sobre todo lo que ha sucedido desde que llegamos a Jerusalén, y para meditar en lo que se avecina, de todo lo cual os he informado claramente. Aseguraos de que la verdad permanece en vuestra vida, y de que crecéis diariamente en la gracia>>.

\par 
%\textsuperscript{(1920.2)}
\textsuperscript{177:0.2} Después del desayuno, el Maestro informó a Andrés que tenía la intención de ausentarse durante todo el día, y sugirió que se autorizara a los apóstoles para que pasaran el tiempo según sus propios deseos, excepto que no debían, en ninguna circunstancia, atravesar las puertas de Jerusalén.

\par 
%\textsuperscript{(1920.3)}
\textsuperscript{177:0.3} Cuando Jesús se preparó para partir solo hacia las colinas, David Zebedeo se le acercó diciendo: <<Maestro, sabes bien que los fariseos y los dirigentes intentan destruirte, y sin embargo te preparas para salir solo hacia las colinas. Eso es una locura; por ello, enviaré a tres hombres contigo, bien preparados para que vigilen que no te suceda nada malo>>. Jesús miró a los tres galileos robustos y bien armados, y dijo a David: <<Tu intención es buena, pero te equivocas en el sentido de que no logras comprender que el Hijo del Hombre no necesita a nadie que lo defienda. Nadie me pondrá la mano encima hasta el momento en que esté preparado para abandonar mi vida de acuerdo con la voluntad de mi Padre. Estos hombres no pueden acompañarme. Deseo ir solo, para poder comulgar con el Padre>>.

\par 
%\textsuperscript{(1920.4)}
\textsuperscript{177:0.4} Al escuchar estas palabras, David y sus guardianes armados se retiraron; pero mientras Jesús partía solo, Juan Marcos se adelantó con una pequeña cesta que contenía alimentos y agua, y sugirió que si Jesús tenía la intención de estar fuera todo el día, podría tener hambre. El Maestro le sonrió a Juan y bajó la mano para coger la cesta.

\section*{1. Un día a solas con Dios}
\par 
%\textsuperscript{(1920.5)}
\textsuperscript{177:1.1} Cuando Jesús estaba a punto de coger la cesta del almuerzo de las manos de Juan, el joven se aventuró a decir: <<Pero, Maestro, quizás dejes la cesta en el suelo mientras te alejas para orar y te vayas sin ella. Además, si te acompaño para llevar el almuerzo, estarás más libre para adorar, y permaneceré callado con toda seguridad. No haré ninguna pregunta, y me quedaré con la cesta cuando te apartes para orar a solas>>.

\par 
%\textsuperscript{(1920.6)}
\textsuperscript{177:1.2} Mientras daba este discurso, cuya temeridad sorprendió a algunos oyentes que se encontraban cerca, Juan tuvo la audacia de retener la cesta. Allí estaban los dos, Juan y Jesús, agarrados a la cesta. Enseguida el Maestro la soltó, bajó la mirada hacia el muchacho, y le dijo: <<Puesto que anhelas acompañarme con todo tu corazón, no te será negado. Nos marcharemos juntos y tendremos una buena conversación. Podrás hacerme todas las preguntas que surjan en tu corazón, y nos confortaremos y nos consolaremos mutuamente. Puedes empezar llevando el almuerzo, y cuando te canses, te ayudaré. Sígueme pues>>.

\par 
%\textsuperscript{(1921.1)}
\textsuperscript{177:1.3} Aquella noche, Jesús no regresó al campamento hasta después de la puesta del Sol. El Maestro pasó este último día de tranquilidad en la Tierra charlando con este joven hambriento de verdad, y hablando con su Padre Paradisiaco. Este acontecimiento se conoce en las alturas como <<el día que un joven pasó con Dios en las colinas>>. Este suceso ejemplifica para siempre la buena voluntad del Creador para fraternizar con la criatura. Hasta un adolescente, si el deseo de su corazón es realmente supremo, puede atraer la atención y disfrutar de la compañía amorosa del Dios de un universo, experimentar realmente el éxtasis inolvidable de estar a solas con Dios en las colinas, y todo ello durante un día entero. Y ésta fue la extraordinaria experiencia de Juan Marcos durante este miércoles en las colinas de Judea.

\par 
%\textsuperscript{(1921.2)}
\textsuperscript{177:1.4} Jesús charló mucho con Juan, y habló libremente sobre los asuntos de este mundo y del siguiente. Juan le dijo a Jesús que lamentaba mucho no haber tenido la edad suficiente para ser uno de los apóstoles, y expresó su gran reconocimiento porque se le había permitido seguir al grupo apostólico desde su primera predicación en el vado del Jordán cerca de Jericó, a excepción del viaje a Fenicia. Jesús le advirtió al joven que no se desanimara por los acontecimientos inminentes, y le aseguró que viviría para convertirse en un poderoso mensajero del reino.

\par 
%\textsuperscript{(1921.3)}
\textsuperscript{177:1.5} Juan Marcos estaba emocionado por el recuerdo de este día con Jesús en las colinas, pero nunca olvidó la recomendación final del Maestro. Cuando estaban a punto de regresar al campamento de Getsemaní, Jesús le dijo: <<Bien, Juan, hemos tenido una buena conversación, un verdadero día de descanso, pero procura no contarle a nadie las cosas que te he dicho>>. Y Juan Marcos nunca reveló nada de lo que había sucedido este día que pasó con Jesús en las colinas.

\par 
%\textsuperscript{(1921.4)}
\textsuperscript{177:1.6} Durante las pocas horas que le quedaban a Jesús por vivir en la Tierra, Juan Marcos nunca dejó que el Maestro estuviera lejos de su vista durante mucho tiempo. El muchacho siempre estaba oculto cerca de él; sólo durmió cuando Jesús dormía.

\section*{2. La infancia en el hogar}
\par 
%\textsuperscript{(1921.5)}
\textsuperscript{177:2.1} En el transcurso de las conversaciones de este día con Juan Marcos, Jesús pasó bastante tiempo comparando sus experiencias de la infancia y de la adolescencia. Aunque los padres de Juan poseían más bienes terrenales que los padres de Jesús, habían tenido en su niñez muchas experiencias muy similares. Jesús dijo muchas cosas que ayudaron a Juan a comprender mejor a sus padres y a otros miembros de su familia. Cuando el muchacho le preguntó al Maestro cómo podía saber que se convertiría en un <<poderoso mensajero del reino>>, Jesús dijo:

\par 
%\textsuperscript{(1921.6)}
\textsuperscript{177:2.2} <<Sé que te mostrarás fiel al evangelio del reino, porque puedo contar con la fe y el amor que tienes ahora, ya que estas cualidades están basadas en una educación tan temprana como la que has recibido en el hogar. Eres el producto de un hogar donde los padres se tienen un afecto mutuo y sincero, por lo que no has sido amado con exceso como para exaltar perjudicialmente tu concepto de tu propia importancia. Tu personalidad tampoco ha sufrido una deformación a consecuencia de unas maniobras sin amor efectuadas por tus padres, enfrentados el uno contra el otro para ganar tu confianza y tu lealtad. Has disfrutado de ese amor parental que asegura una loable confianza en sí mismo y que fomenta unos sentimientos normales de seguridad. Pero también has tenido la suerte de que tus padres poseyeran sabiduría al mismo tiempo que amor; fue la sabiduría la que les condujo a negarte la mayoría de las satisfacciones y de los múltiples lujos que se pueden comprar con la riqueza; te enviaron a la escuela de la sinagoga con tus compañeros de juego de la vecindad, y también te animaron a aprender la manera de vivir en este mundo permitiéndote efectuar una experiencia original. Viniste con tu joven amigo Amós al Jordán, donde nosotros predicábamos y los discípulos de Juan bautizaban. Los dos deseabais acompañarnos. Cuando regresasteis a Jerusalén, tus padres dieron su consentimiento; los padres de Amós se negaron; amaban tanto a su hijo que le negaron la experiencia bendita que tú has tenido, incluida la que hoy estás disfrutando. Amós podría haberse escapado de su casa para unirse a nosotros, pero si lo hubiera hecho, habría herido el amor y sacrificado la fidelidad. Aunque esta conducta hubiera sido sabia, hubiera pagado un precio terrible por la experiencia, la independencia y la libertad. Los padres sabios, como los tuyos, procuran que sus hijos no tengan que herir el amor o ahogar la fidelidad para desarrollar su independencia y disfrutar de una libertad vigorizante cuando han llegado a tu edad>>.

\par 
%\textsuperscript{(1922.1)}
\textsuperscript{177:2.3} <<El amor, Juan, es la realidad suprema del universo cuando es otorgado por unos seres infinitamente sabios, pero presenta un rasgo peligroso y a veces semiegoísta tal como es manifestado en la experiencia de los padres mortales. Cuando te cases y tengas que criar tus propios hijos, asegúrate de que tu amor esté aconsejado por la sabiduría y guiado por la inteligencia>>.

\par 
%\textsuperscript{(1922.2)}
\textsuperscript{177:2.4} <<Tu joven amigo Amós cree en este evangelio del reino tanto como tú, pero no puedo contar plenamente con él; no estoy seguro de lo que va a hacer en los años venideros. Su infancia en el hogar no se desarrolló como para producir una persona enteramente digna de confianza. Amós se parece demasiado a uno de mis apóstoles que no pudo disfrutar de una educación familiar normal, amorosa y sabia. Toda tu vida futura será más feliz y digna de confianza porque pasaste tus primeros ocho años en un hogar normal y bien regulado. Posees un carácter fuerte y bien integrado porque creciste en un hogar donde prevalecía el amor y reinaba la sabiduría. Este tipo de formación durante la infancia produce un tipo de fidelidad que me asegura que continuarás en el camino que has empezado>>.

\par 
%\textsuperscript{(1922.3)}
\textsuperscript{177:2.5} Durante más de una hora, Jesús y Juan continuaron esta conversación sobre la vida familiar. El Maestro siguió explicándole a Juan que un niño depende totalmente de sus padres y de la vida asociada en el hogar para formarse sus primeros conceptos sobre todas las cosas intelectuales, sociales, morales e incluso espirituales, puesto que la familia representa para el niño pequeño todo lo que puede conocer al principio sobre las relaciones humanas o divinas. El niño debe obtener, de los cuidados de su madre, sus primeras impresiones sobre el universo; depende totalmente de su padre terrenal para sus primeras ideas sobre el Padre celestial. La vida mental y emocional de los primeros años, condicionada por estas relaciones sociales y espirituales del hogar, determina si la vida posterior del niño será feliz o infeliz, fácil o difícil. Toda la vida de un ser humano está enormemente influida por lo que sucede durante los primeros años de la existencia.

\par 
%\textsuperscript{(1922.4)}
\textsuperscript{177:2.6} Creemos sinceramente que el evangelio contenido en las enseñanzas de Jesús, basado como lo está en la relación entre padre e hijo, difícilmente podrá disfrutar de una aceptación mundial hasta el momento en que la vida familiar de los pueblos modernos civilizados contenga más amor y más sabiduría. A pesar de que los padres del siglo veinte poseen un gran conocimiento y una mayor verdad para mejorar el hogar y ennoblecer la vida familiar, sigue siendo un hecho que para educar a los niños y a las niñas, muy pocos hogares modernos son tan buenos como el hogar de Jesús en Galilea y el de Juan Marcos en Judea; sin embargo, la aceptación del evangelio de Jesús tendrá como resultado una mejora inmediata de la vida familiar. La vida de amor de un hogar sabio y la devoción fiel a la verdadera religión ejercen una profunda influencia recíproca. Una vida hogareña así realza la religión, y la auténtica religión siempre glorifica el hogar.

\par 
%\textsuperscript{(1923.1)}
\textsuperscript{177:2.7} Es verdad que muchas influencias censurables atrofiadas y otras características restrictivas de estos antiguos hogares judíos han sido prácticamente eliminadas de muchos hogares modernos mejor organizados. Existe en verdad más independencia espontánea y mucha más libertad personal, pero esta libertad no está refrenada por el amor, motivada por la fidelidad, ni dirigida por la disciplina inteligente de la sabiduría. Mientras enseñemos al niño a rezar <<Padre nuestro que estás en los cielos>>, todos los padres terrenales tendrán la inmensa responsabilidad de vivir y ordenar sus hogares de tal manera que la palabra \textit{padre} quede guardada dignamente en la mente y en el corazón de todos los niños que crecen.

\section*{3. El día en el campamento}
\par 
%\textsuperscript{(1923.2)}
\textsuperscript{177:3.1} Los apóstoles pasaron la mayor parte de este día caminando por el Monte de los Olivos y conversando con los discípulos que acampaban con ellos, pero al principio de la tarde sintieron el vivo deseo de ver regresar a Jesús. A medida que pasaba el día, se inquietaron cada vez más por su seguridad; se sentían inexpresablemente solos sin él. Durante todo el día estuvieron discutiendo sobre si deberían haberle permitido al Maestro partir solo hacia las colinas, acompañado solamente por el muchacho de los recados. Aunque nadie expresaba abiertamente sus pensamientos, no había ninguno de ellos, salvo Judas Iscariote, que no hubiera deseado estar en el lugar de Juan Marcos.

\par 
%\textsuperscript{(1923.3)}
\textsuperscript{177:3.2} Fue hacia mediados de la tarde cuando Natanael dio su discurso sobre el <<Deseo supremo>> a una media docena de apóstoles y a un número igual de discípulos, concluyendo de la manera siguiente: <<En lo que estamos equivocados la mayoría de nosotros es en que somos poco entusiastas. No amamos al Maestro como él nos ama. Si todos hubiéramos querido ir con él tanto como Juan Marcos lo deseaba, seguramente nos hubiera llevado a todos. Nos quedamos mirando mientras el muchacho se acercaba al Maestro y le ofrecía la cesta, pero cuando el Maestro la cogió, el muchacho no la soltó. Por eso el Maestro nos dejó aquí mientras partía hacia las colinas con la cesta, el niño y todo>>.

\par 
%\textsuperscript{(1923.4)}
\textsuperscript{177:3.3} Hacia las cuatro, unos corredores llegaron hasta David Zebedeo trayéndole un mensaje de su madre en Betsaida y de la madre de Jesús. Varios días antes, David había llegado a la conclusión de que los jefes de los sacerdotes y los dirigentes iban a matar a Jesús. David sabía que estaban decididos a destruir al Maestro, y estaba casi convencido de que Jesús no ejercería su poder divino para salvarse, ni permitiría que sus seguidores emplearan la fuerza para defenderlo. Habiendo llegado a estas conclusiones, no tardó en enviar un mensajero a su madre, instándola a que viniera enseguida a Jerusalén y que trajera a María, la madre de Jesús, y a todos los miembros de su familia.

\par 
%\textsuperscript{(1923.5)}
\textsuperscript{177:3.4} La madre de David hizo lo que su hijo le pedía, y los corredores regresaron ahora hasta David trayendo la noticia de que su madre y toda la familia de Jesús estaban de camino hacia Jerusalén, y que llegarían tarde en cualquier momento del día siguiente, o muy temprano la mañana después. Puesto que David había hecho esto por su propia iniciativa, pensó que sería prudente guardarse esta información para sí mismo. Por lo tanto, no le dijo a nadie que la familia de Jesús estaba de camino hacia Jerusalén.

\par 
%\textsuperscript{(1924.1)}
\textsuperscript{177:3.5} Poco después del mediodía, más de veinte de los griegos que se habían encontrado con Jesús y los doce en la casa de José de Arimatea llegaron al campamento, y Pedro y Juan pasaron varias horas conversando con ellos. Estos griegos, o al menos algunos de ellos, tenían un buen conocimiento del reino, pues habían sido instruidos por Rodán en Alejandría.

\par 
%\textsuperscript{(1924.2)}
\textsuperscript{177:3.6} Aquella noche, después de regresar al campamento, Jesús conversó con los griegos, y habría ordenado a estos veinte hombres tal como había hecho con los setenta si no hubiera sido porque esta acción habría perturbado profundamente a sus apóstoles y a muchos de sus discípulos principales.

\par 
%\textsuperscript{(1924.3)}
\textsuperscript{177:3.7} Mientras todo esto sucedía en el campamento, en Jerusalén los jefes de los sacerdotes y los ancianos estaban sorprendidos de que Jesús no regresara para dirigir la palabra a las multitudes. Es verdad que el día anterior había dicho, al abandonar el templo: <<Os dejo vuestra casa desolada>>. Pero no podían comprender por qué estaba dispuesto a renunciar a la gran ventaja que había conseguido con la actitud amistosa de las multitudes. Aunque temían que produjera un tumulto en el pueblo, las últimas palabras del Maestro a la multitud habían sido una exhortación a que se conformaran, de todas las maneras razonables, a la autoridad de aquellos <<que estaban sentados en el puesto de Moisés>>. Pero aquel día estaban muy ocupados en la ciudad, preparándose simultáneamente para la Pascua y para perfeccionar sus planes de destruir a Jesús.

\par 
%\textsuperscript{(1924.4)}
\textsuperscript{177:3.8} Al campamento no acudió mucha gente, porque su ubicación se había mantenido como un secreto bien guardado por todos los que sabían que Jesús contaba con quedarse allí en lugar de dirigirse todas las noches a Betania.

\section*{4. Judas y los jefes de los sacerdotes}
\par 
%\textsuperscript{(1924.5)}
\textsuperscript{177:4.1} Poco después de que Jesús y Juan Marcos dejaran el campamento, Judas Iscariote desapareció del grupo de sus hermanos y no regresó hasta el final de la tarde. A pesar de la recomendación expresa de su Maestro de que no entraran en Jerusalén, este apóstol confundido y descontento se dirigió apresuradamente a su cita con los enemigos de Jesús, en la casa del sumo sacerdote Caifás. Se trataba de una reunión informal del sanedrín, fijada para poco después de las diez de aquella mañana. Esta reunión se había convocado para discutir la naturaleza de las acusaciones que se iban a presentar contra Jesús, y decidir el procedimiento a seguir para llevarlo ante las autoridades romanas a fin de conseguir la confirmación civil necesaria para la sentencia de muerte que ya habían decretado.

\par 
%\textsuperscript{(1924.6)}
\textsuperscript{177:4.2} El día anterior, Judas había revelado a algunos de sus parientes, y a ciertos amigos saduceos de la familia de su padre, que había llegado a la conclusión de que, aunque Jesús era un soñador y un idealista bien intencionado, no era el libertador esperado de Israel. Judas declaró que le gustaría mucho encontrar una manera airosa de retirarse de todo el movimiento. Sus amigos le aseguraron halagadoramente que su retirada sería saludada por los dirigentes judíos como un gran acontecimiento, y que podría lograr cualquier cosa. Le indujeron a creer que recibiría inmediatamente grandes honores del sanedrín, y que por fin se encontraría en condiciones de borrar el estigma de su <<asociación bien intencionada, aunque desafortunada, con unos galileos ignorantes>>.

\par 
%\textsuperscript{(1924.7)}
\textsuperscript{177:4.3} Judas no podía creer del todo que las grandes obras del Maestro habían sido realizadas por el poder del príncipe de los demonios, pero ahora estaba plenamente convencido de que Jesús no ejercería su poder para engrandecerse; al final se había convencido de que Jesús se dejaría destruir por los dirigentes judíos, y no podía soportar la idea humillante de ser identificado con un movimiento condenado al fracaso. Se negaba a considerar la idea de un fracaso aparente. Comprendía plenamente el carácter firme de su Maestro y la agudeza de su mente majestuosa y misericordiosa, pero sin embargo le causaba placer aceptar, aunque fuera parcialmente, la sugerencia de uno de sus parientes de que Jesús, aunque fuera un fanático bien intencionado, es probable que no estuviera realmente bien de la cabeza; que siempre había parecido ser una persona extraña y mal comprendida.

\par 
%\textsuperscript{(1925.1)}
\textsuperscript{177:4.4} Y ahora más que nunca, Judas empezó a sentir un extraño resentimiento porque Jesús nunca le había asignado una posición más honorífica. Durante todo este tiempo había apreciado el honor de ser el tesorero apostólico, pero ahora empezaba a sentir que no era apreciado, que sus capacidades no eran reconocidas. Repentinamente se sintió dominado por la indignación porque Pedro, Santiago y Juan habían sido honrados con una asociación estrecha con Jesús, y en aquel momento, mientras se dirigía a la casa del sumo sacerdote, estaba más resuelto a desquitarse de Pedro, Santiago y Juan que a preocuparse por la idea de traicionar a Jesús. Pero por encima de todo, en aquel preciso momento, una nueva idea dominante empezó a ocupar el primer lugar en su mente consciente: Se había puesto en marcha para conseguir honores para sí mismo, y si podía asegurárselos al mismo tiempo que se desquitaba de los que habían contribuido a la mayor desilusión de su vida, mucho mejor. Cayó presa de una terrible confabulación de confusión, orgullo, desesperación y resolución. Así pues, debe quedar claro que no era por dinero por lo que Judas se dirigía en aquel momento hacia la casa de Caifás para preparar la traición a Jesús.

\par 
%\textsuperscript{(1925.2)}
\textsuperscript{177:4.5} Mientras Judas se acercaba a la casa de Caifás, tomó la decisión definitiva de abandonar a Jesús y a sus compañeros apóstoles; habiendo decidido dejar así la causa del reino de los cielos, estaba resuelto a asegurarse para sí mismo el máximo de honor y de gloria que había esperado recibir algún día, cuando se identificó por primera vez con Jesús y el nuevo evangelio del reino. Todos los apóstoles habían compartido alguna vez esta ambición con Judas, pero a medida que pasaba el tiempo habían aprendido a admirar la verdad y a amar a Jesús, al menos más que Judas.

\par 
%\textsuperscript{(1925.3)}
\textsuperscript{177:4.6} El traidor fue presentado a Caifás y a los dirigentes judíos por su primo. Éste explicó que Judas había descubierto el error que había cometido al dejarse engañar por la sutil enseñanza de Jesús, y había llegado a un punto en que deseaba renunciar pública y oficialmente a su asociación con el galileo; al mismo tiempo, pedía que se le restableciera en la confianza y la hermandad de sus hermanos judeos. El portavoz de Judas continuó explicando que Judas reconocía que sería mejor, para la paz de Israel, que Jesús fuera arrestado. Como demostración de su pesar por haber participado en este movimiento erróneo, y como prueba de la sinceridad de su presente regreso a las enseñanzas de Moisés, había venido para ofrecerse al sanedrín como alguien que podía colaborar con el capitán que tenía la orden de arrestar a Jesús, para que éste pudiera ser detenido discretamente, evitando así el peligro de excitar a las multitudes, o la necesidad de retrasar su arresto hasta después de la Pascua.

\par 
%\textsuperscript{(1925.4)}
\textsuperscript{177:4.7} Cuando hubo terminado de hablar, el primo presentó a Judas, el cual se acercó al sumo sacerdote, y dijo: <<Haré todo lo que mi primo ha prometido, pero ¿qué estáis dispuestos a darme por este servicio?>> Judas no pareció percibir la expresión de desdén, e incluso de disgusto, que cruzó por el rostro del insensible y vanidoso Caifás; el corazón de Judas estaba demasiado centrado en su propia glorificación y en el anhelo de satisfacer la exaltación de su ego.

\par 
%\textsuperscript{(1926.1)}
\textsuperscript{177:4.8} Caifás bajó entonces la mirada hacia el traidor mientras decía: <<Judas, ve a ver al capitán de la guardia y ponte de acuerdo con ese oficial para traernos a tu Maestro esta noche o mañana por la noche. Y cuando nos lo hayas entregado, recibirás tu recompensa por este servicio>>. Cuando Judas escuchó esto, se retiró de la presencia de los sacerdotes y dirigentes principales, y fue a consultar con el capitán de los guardias del templo sobre la manera en que debían apresar a Jesús. Judas sabía que Jesús estaba entonces ausente del campamento, y no tenía ni idea de la hora en que volvería aquella noche, por lo que acordaron detener a Jesús a la noche siguiente (jueves), después de que el pueblo de Jerusalén y todos los peregrinos visitantes se hubieran retirado a descansar.

\par 
%\textsuperscript{(1926.2)}
\textsuperscript{177:4.9} Judas regresó al campamento con sus compañeros, embriagado con unas ideas de grandeza y de gloria como no había tenido desde hacía mucho tiempo. Se había enrolado con Jesús esperando convertirse algún día en un gran hombre en el nuevo reino, y al final se había dado cuenta de que no habría ningún nuevo reino tal como él lo había esperado. Pero se regocijaba por ser lo bastante sagaz como para canjear la decepción de no conseguir la gloria en el nuevo reino esperado por la obtención inmediata de honores y recompensas en el viejo orden de cosas; ahora creía que este viejo orden sobreviviría, y estaba seguro de que destruiría a Jesús y a todo lo que él representaba. En el móvil final de su intención consciente, la traición de Judas a Jesús fue el acto cobarde de un desertor egoísta cuya única preocupación era su propia seguridad y su glorificación, cualquiera que fueran los resultados de su conducta para su Maestro y sus antiguos compañeros.

\par 
%\textsuperscript{(1926.3)}
\textsuperscript{177:4.10} Pero siempre había sido así. Hacía mucho tiempo que Judas alimentaba esta conciencia deliberada, persistente, egoísta y vengativa de construir progresivamente en su mente, y de albergar en su corazón, estos deseos odiosos y malvados de venganza y deslealtad. Jesús amaba y confiaba en Judas tal como amaba y confiaba en los otros apóstoles, pero Judas no logró desarrollar a cambio una confianza leal ni experimentar un amor sincero. ¡Cuán peligrosa puede ser la ambición cuando está enteramente unida al egoísmo y motivada de manera suprema por la venganza sombría tanto tiempo reprimida! Qué aplastante es la decepción en la vida de aquellas personas necias que fijan sus miras en los atractivos oscuros y evanescentes del tiempo, y se vuelven ciegas a los logros superiores y más reales de las conquistas perpetuas de los mundos eternos de los valores divinos y de las verdaderas realidades espirituales. Judas ansiaba en su mente los honores mundanos y llegó a amar este deseo con todo su corazón; los otros apóstoles también ansiaban en su mente estos mismos honores mundanos, pero amaban a Jesús con el corazón y hacían todo lo posible por aprender a amar las verdades que él les enseñaba.

\par 
%\textsuperscript{(1926.4)}
\textsuperscript{177:4.11} Judas no se daba cuenta de ello en este momento, pero había criticado subconscientemente a Jesús desde que Juan el Bautista había sido decapitado por Herodes. En lo más profundo de su corazón, a Judas siempre le había indignado el hecho de que Jesús no salvara a Juan. No debéis olvidar que Judas había sido discípulo de Juan antes de convertirse en seguidor de Jesús. Toda esta acumulación de resentimiento humano y de amarga decepción que Judas había conservado en su alma con atuendos de odio, se encontraba ahora bien organizada en su mente subconsciente, lista para brotar y sumergirlo en cuanto se atreviera a separarse de la influencia protectora de sus hermanos, exponiéndose al mismo tiempo a las hábiles insinuaciones y a las burlas sutiles de los enemigos de Jesús. Cada vez que Judas permitía que sus esperanzas se elevaran muy alto, y Jesús decía o hacía algo que las hacía añicos, siempre quedaba en el corazón de Judas una cicatriz de amargo resentimiento; y a medida que estas cicatrices se multiplicaron, aquel corazón herido con tanta frecuencia perdió enseguida todo afecto real por aquel que había infligido esta experiencia desagradable a una personalidad bien intencionada, pero cobarde y egocéntrica. Judas no se daba cuenta de ello, pero era un cobarde. En consecuencia, siempre tenía la tendencia de atribuir a la cobardía de Jesús los móviles que le llevaron con tanta frecuencia a no coger el poder o la gloria cuando estaban en apariencia fácilmente a su alcance. Y todo hombre mortal sabe muy bien que el amor, aunque al principio haya sido sincero, puede convertirse finalmente en un odio real a causa de las decepciones, los celos y un resentimiento constante.

\par 
%\textsuperscript{(1927.1)}
\textsuperscript{177:4.12} Los jefes de los sacerdotes y los ancianos pudieron por fin respirar tranquilamente durante algunas horas. No tendrían que arrestar a Jesús en público, y los servicios de Judas como aliado traidor les aseguraba que Jesús no se escaparía de su jurisdicción como lo había hecho tantas veces en el pasado.

\section*{5. Las últimas horas de reunión social}
\par 
%\textsuperscript{(1927.2)}
\textsuperscript{177:5.1} Puesto que era miércoles, aquella noche en el campamento fueron horas de reunión social. El Maestro intentó animar a sus apóstoles abatidos, pero era casi imposible. Todos empezaban a darse cuenta de que se acercaban unos acontecimientos desconcertantes y abrumadores. No podían estar alegres, ni siquiera cuando el Maestro recordó sus años de asociación afectuosa y llena de acontecimientos. Jesús se interesó cuidadosamente por las familias de todos los apóstoles y, mirando a David Zebedeo, preguntó si alguien tenía noticias recientes de su madre, de su hermana menor o de otros miembros de su familia. David bajó la mirada hacia sus pies; tenía miedo de responder.

\par 
%\textsuperscript{(1927.3)}
\textsuperscript{177:5.2} Ésta fue la ocasión en que Jesús advirtió a sus seguidores que desconfiaran del apoyo de la multitud. Recordó sus experiencias en Galilea, cuando las grandes muchedumbres los habían seguido con entusiasmo una y otra vez, y luego se habían predispuesto contra ellos con el mismo ardor, para volver a sus creencias y maneras de vivir anteriores. Luego dijo: <<Así pues, no os dejéis engañar por las grandes muchedumbres que nos escucharon en el templo y que parecían creer en nuestras enseñanzas. Esas multitudes escuchan la verdad y la creen superficialmente con su mente, pero pocos de ellos dejan que la palabra de la verdad se fije en su corazón con raíces vivientes. Cuando se presentan las dificultades reales, no se puede contar con el apoyo de aquellos que sólo conocen el evangelio en su mente, y no lo han experimentado en su corazón. Cuando los dirigentes de los judíos lleguen a un acuerdo para destruir al Hijo del Hombre, y golpeen al unísono, veréis que la multitud huirá aterrada o bien permanecerá allí asombrada en silencio, mientras esos dirigentes enloquecidos y ciegos conducen a la muerte a los instructores de la verdad evangélica. Luego, cuando la adversidad y las persecuciones caigan sobre vosotros, otros que creéis que aman la verdad también se dispersarán, y algunos renunciarán al evangelio y os abandonarán. Algunos que han estado muy cerca de nosotros ya han decidido desertar. Habéis descansado hoy como preparación para los acontecimientos inminentes. Vigilad pues, y orad para que mañana os podáis sentir fortalecidos para los días que se acercan>>.

\par 
%\textsuperscript{(1927.4)}
\textsuperscript{177:5.3} El ambiente del campamento estaba cargado de una tensión inexplicable. Unos mensajeros silenciosos iban y venían, comunicándose únicamente con David Zebedeo. Antes de que terminara la noche, algunos sabían que Lázaro había huido precipitadamente de Betania. Juan Marcos guardaba un silencio siniestro después de regresar al campamento, a pesar de haber pasado todo el día en compañía del Maestro. Todo esfuerzo por persuadirlo para que hablara sólo indicaba claramente que Jesús le había dicho que no hablara.

\par 
%\textsuperscript{(1928.1)}
\textsuperscript{177:5.4} Incluso el buen humor y la sociabilidad poco común del Maestro asustó a los apóstoles. Todos sentían la clara proximidad del terrible aislamiento que estaba a punto de caer sobre ellos con una prontitud arrolladora y un terror ineludible. Sospechaban vagamente lo que iba a suceder, y ninguno se sentía preparado para enfrentarse a la prueba. El Maestro había estado ausente todo el día, y lo habían echado enormemente de menos.

\par 
%\textsuperscript{(1928.2)}
\textsuperscript{177:5.5} Este miércoles por la noche marcó el punto más bajo del estado espiritual de los apóstoles hasta el momento mismo de la muerte del Maestro. Aunque el día siguiente era un día que les acercaba más al viernes trágico, al menos él todavía estaba con ellos, y pudieron pasar esas horas de inquietud más airosamente.

\par 
%\textsuperscript{(1928.3)}
\textsuperscript{177:5.6} Jesús sabía que ésta sería la última noche que podría dormir tranquilo con la familia que había elegido en la Tierra; un poco antes de la medianoche, los despidió diciendo: <<Id a dormir, hermanos míos, y que la paz sea con vosotros hasta que nos levantemos mañana, un día más para hacer la voluntad del Padre y experimentar la alegría de saber que somos sus hijos>>.


\chapter{Documento 178. El último día en el campamento}
\par 
%\textsuperscript{(1929.1)}
\textsuperscript{178:0.1} JESÚS pensaba pasar este jueves, su último día de libertad en la Tierra como Hijo divino encarnado, con sus apóstoles y algunos discípulos leales y fervientes. Poco después de la hora del desayuno de esta hermosa mañana, el Maestro los condujo a un lugar apartado, a poca distancia por encima de su campamento, y allí les enseñó muchas nuevas verdades. Aunque Jesús pronunció otros discursos a los apóstoles durante las primeras horas de la noche de este día, esta charla del jueves por la mañana fue su alocución de despedida al grupo del campamento compuesto por los apóstoles y los discípulos escogidos, tanto judíos como gentiles. Los doce estaban todos presentes, salvo Judas. Pedro y varios apóstoles mencionaron su ausencia, y algunos pensaron que Jesús lo había enviado a la ciudad para ocuparse de algún asunto, probablemente para arreglar los detalles de su próxima celebración de la Pascua. Judas no regresó al campamento hasta media tarde, poco antes de que Jesús condujera a los doce a Jerusalén para compartir la Última Cena.

\section*{1. El discurso sobre la filiación y la ciudadanía}
\par 
%\textsuperscript{(1929.2)}
\textsuperscript{178:1.1} Jesús habló durante casi dos horas a unos cincuenta seguidores suyos de confianza, y respondió a una veintena de preguntas sobre la relación entre el reino de los cielos y los reinos de este mundo, sobre la relación entre la filiación con Dios y la ciudadanía en los gobiernos terrenales. Esta disertación, así como sus respuestas a las preguntas, se pueden resumir y exponer en lenguaje moderno de la manera siguiente:

\par 
%\textsuperscript{(1929.3)}
\textsuperscript{178:1.2} Los reinos de este mundo, como son materiales, a menudo pueden juzgar necesario emplear la fuerza física para hacer cumplir sus leyes y mantener el orden. En el reino de los cielos, los verdaderos creyentes no recurrirán al empleo de la fuerza física. El reino de los cielos es una fraternidad espiritual de los hijos de Dios nacidos del espíritu, y sólo se puede promulgar por el poder del espíritu. Esta diferencia de procedimiento se refiere a las relaciones entre el reino de los creyentes y los reinos de los gobiernos laicos, y no anula el derecho que tienen los grupos sociales de creyentes a mantener el orden en sus filas y a administrar la disciplina a sus miembros ingobernables e indignos.

\par 
%\textsuperscript{(1929.4)}
\textsuperscript{178:1.3} No hay nada que sea incompatible entre la filiación en el reino espiritual y la ciudadanía en un gobierno laico o civil. El creyente tiene el deber de dar al César las cosas que son del César, y a Dios las cosas que son de Dios. No puede haber discrepancia entre estas dos exigencias, pues una es material y la otra espiritual, a menos que un César se atreva a usurpar las prerrogativas de Dios y exija que se le rinda un homenaje espiritual y un culto supremo. En ese caso, sólo adoraréis a Dios y trataréis al mismo tiempo de iluminar a esos dirigentes terrenales equivocados, conduciéndolos de esta manera a reconocer también al Padre que está en los cielos. No rendiréis culto espiritual a los dirigentes terrenales; tampoco emplearéis la fuerza física de los gobiernos terrestres, cuyos jefes puedan volverse creyentes algún día, en la tarea de promover la misión del reino espiritual.

\par 
%\textsuperscript{(1930.1)}
\textsuperscript{178:1.4} Desde el punto de vista de una civilización que progresa, la filiación en el reino debería ayudaros a convertiros en los ciudadanos ideales de los reinos de este mundo, puesto que la fraternidad y el servicio son las piedras angulares del evangelio del reino. La llamada al amor del reino espiritual debería llegar a ser el destructor efectivo de la incitación al odio de los ciudadanos incrédulos y belicosos de los reinos terrestres. Pero esos hijos materialistas, que se hallan en las tinieblas, nunca sabrán nada de vuestra luz espiritual de la verdad a menos que os acerquéis mucho a ellos con ese servicio social desinteresado que es el resultado natural de producir los frutos del espíritu en la experiencia de la vida de cada creyente individual.

\par 
%\textsuperscript{(1930.2)}
\textsuperscript{178:1.5} Como hombres mortales y materiales, sois en verdad los ciudadanos de los reinos terrestres, y deberíais ser buenos ciudadanos, mucho mejores por haberos convertido en los hijos renacidos de espíritu del reino celestial. Como hijos iluminados por la fe y liberados por el espíritu del reino de los cielos, os enfrentáis con la doble responsabilidad del deber hacia los hombres y del deber hacia Dios, mientras que asumís voluntariamente una tercera obligación sagrada: el servicio a la fraternidad de los creyentes que conocen a Dios.

\par 
%\textsuperscript{(1930.3)}
\textsuperscript{178:1.6} No es lícito que adoréis a vuestros gobernantes temporales, y no deberíais emplear el poder temporal para hacer progresar el reino espiritual; pero deberíais manifestar por igual, a los creyentes y a los incrédulos, el ministerio equitativo del servicio amoroso. El poderoso Espíritu de la Verdad reside en el evangelio del reino, y pronto derramaré este mismo espíritu sobre todo el género humano. Los frutos del espíritu, vuestro servicio sincero y amoroso, son la poderosa palanca social que eleva a las razas que están en las tinieblas, y este Espíritu de la Verdad se convertirá en el punto de apoyo que multiplicará vuestro poder.

\par 
%\textsuperscript{(1930.4)}
\textsuperscript{178:1.7} Mostrad sabiduría y manifestad sagacidad en vuestras relaciones con los gobernantes civiles incrédulos. Con vuestra prudencia, mostrad que sois expertos en allanar los desacuerdos menores y en ajustar los pequeños malentendidos. De todas las maneras posibles ---en todas las cosas, salvo en vuestra lealtad espiritual a los gobernantes del universo--- tratad de vivir en paz con todos los hombres. Sed siempre tan prudentes como las serpientes, pero tan inofensivos como las palomas.

\par 
%\textsuperscript{(1930.5)}
\textsuperscript{178:1.8} Deberíais ser mucho mejores ciudadanos del gobierno laico como consecuencia de haberos convertido en los hijos iluminados del reino; de la misma manera, los jefes de los gobiernos terrestres dirigirán mucho mejor los asuntos civiles como consecuencia de creer en este evangelio del reino celestial. La actitud de servir desinteresadamente a los hombres y de adorar a Dios de manera inteligente debería hacer que todos los creyentes en el reino sean mejores ciudadanos del mundo, mientras que la actitud de ser un ciudadano honrado y de consagrarse sinceramente a sus deberes temporales debería ayudar a ese ciudadano a ser más receptivo a la llamada espiritual de la filiación en el reino celestial.

\par 
%\textsuperscript{(1930.6)}
\textsuperscript{178:1.9} Mientras los jefes de los gobiernos terrestres intenten ejercer la autoridad de los dictadores religiosos, vosotros que creéis en este evangelio sólo podéis esperar dificultades, persecuciones e incluso la muerte. Pero la luz misma que aportáis al mundo, e incluso la manera misma en que sufriréis y moriréis por este evangelio del reino, iluminarán finalmente, por sí mismas, al mundo entero, y acabarán separando gradualmente la política de la religión. La continua predicación de este evangelio del reino traerá algún día, a todas las naciones, una liberación nueva e increíble, la independencia intelectual y la libertad religiosa.

\par 
%\textsuperscript{(1931.1)}
\textsuperscript{178:1.10} Durante las persecuciones inminentes que sufriréis por parte de aquellos que odian este evangelio de alegría y de libertad, vosotros floreceréis y el reino prosperará. Pero correréis un grave peligro, en épocas posteriores, cuando la mayoría de la gente hable bien de los creyentes en el reino, y muchos que ocupan puestos importantes acepten nominalmente el evangelio del reino celestial. Aprended a ser fieles al reino, incluso en tiempos de paz y de prosperidad. No tentéis a los ángeles que os supervisan a conduciros por caminos turbulentos como disciplina amorosa destinada a salvar vuestra alma indolente.

\par 
%\textsuperscript{(1931.2)}
\textsuperscript{178:1.11} Recordad que estáis encargados de predicar este evangelio del reino ---el deseo supremo de hacer la voluntad del Padre, unido a la alegría suprema de comprender, por la fe, que sois hijos de Dios--- y no debéis permitir que nada desvíe vuestra consagración a este único deber. Que toda la humanidad se beneficie del desbordamiento de vuestro afectuoso ministerio espiritual, de vuestra comunión intelectual iluminadora, y de vuestro servicio social edificante; pero no se debe permitir que ninguna de estas labores humanitarias, ni todas a la vez, reemplacen la proclamación del evangelio. Estos grandes servicios son los productos sociales secundarios de los ministerios y transformaciones aun más grandes y sublimes, forjados en el corazón del creyente en el reino por el Espíritu viviente de la Verdad y por la comprensión personal de que la fe de un hombre nacido del espíritu confiere la seguridad de una comunión viviente con el Dios eterno.

\par 
%\textsuperscript{(1931.3)}
\textsuperscript{178:1.12} No debéis intentar promulgar la verdad ni establecer la rectitud mediante el poder de los gobiernos civiles o por medio de la promulgación de las leyes laicas. Siempre podéis esforzaros por persuadir la mente de los hombres, pero no debéis atreveros nunca a forzarlos. No debéis olvidar la gran ley de la equidad humana que os he enseñado de manera positiva: Todo aquello que queréis que los hombres hagan por vosotros, hacedlo por ellos.

\par 
%\textsuperscript{(1931.4)}
\textsuperscript{178:1.13} Cuando un creyente en el reino es llamado a servir al gobierno civil, que preste ese servicio como ciudadano temporal de ese gobierno, aunque ese creyente debería mostrar en su servicio civil todas las características comunes de los ciudadanos tal como han sido realzadas por la iluminación espiritual de la asociación ennoblecedora de la mente del hombre mortal con el espíritu interior del Dios eterno. Si a un no creyente se le puede calificar de servidor civil superior, deberíais examinar seriamente si las raíces de la verdad que están en vuestro corazón no se han secado por falta del agua viva de la comunión espiritual combinada con el servicio social. La conciencia de la filiación con Dios debería vivificar toda la vida de servicio de cada hombre, de cada mujer y de cada niño que posee ese poderoso estimulante de todos los poderes inherentes a una personalidad humana.

\par 
%\textsuperscript{(1931.5)}
\textsuperscript{178:1.14} No debéis ser unos místicos pasivos ni unos ascetas anodinos; no os convirtáis en unos soñadores ni en unos vagabundos, que confían pasivamente en una Providencia ficticia para que les proporcione hasta las necesidades de la vida. En verdad, debéis ser dulces en vuestras relaciones con los mortales equivocados, pacientes en vuestro trato con los ignorantes, e indulgentes cuando os provoquen; pero también debéis ser valientes en la defensa de la rectitud, poderosos en la promulgación de la verdad y dinámicos en la predicación de este evangelio del reino, incluso hasta los confines de la Tierra.

\par 
%\textsuperscript{(1931.6)}
\textsuperscript{178:1.15} Este evangelio del reino es una verdad viviente. Os he dicho que se parece a la levadura en la masa, y al grano de la semilla de mostaza; y ahora os afirmo que se parece a la semilla del ser vivo, que sigue siendo la misma de generación en generación, pero que se desarrolla infaliblemente en nuevas manifestaciones, y crece de manera aceptable en canales que se adaptan de nuevo a las necesidades y condiciones particulares de cada generación sucesiva. La revelación que os he hecho es una \textit{revelación viva}, y deseo que produzca los frutos apropiados en cada individuo y en cada generación, de acuerdo con las leyes del crecimiento espiritual, de la mejora y del desarrollo adaptativo. De generación en generación, este evangelio debe mostrar una vitalidad creciente y demostrar una mayor profundidad de poder espiritual. No se debe permitir que se convierta en un simple recuerdo sagrado, en una simple tradición acerca de mí y de la época en que vivimos ahora.

\par 
%\textsuperscript{(1932.1)}
\textsuperscript{178:1.16} Y no lo olvidéis: No hemos atacado directamente a las personas ni a la autoridad de los que están sentados en el puesto de Moisés; sólo les hemos ofrecido la nueva luz, que ellos han rechazado tan enérgicamente. Sólo les hemos atacado denunciando su deslealtad espiritual hacia las mismas verdades que pretenden enseñar y salvaguardar. Sólo hemos entrado en conflicto con esos dirigentes establecidos y esos jefes reconocidos cuando se han opuesto directamente a la predicación del evangelio del reino a los hijos de los hombres. E incluso ahora, no somos nosotros quienes les atacamos, sino que son ellos los que buscan nuestra destrucción. No olvidéis que sólo estáis encargados de salir a predicar la buena nueva. No debéis atacar las viejas costumbres; debéis introducir hábilmente la levadura de la nueva verdad en medio de las antiguas creencias. Dejad que el Espíritu de la Verdad efectúe su propio trabajo. Que la controversia sólo surja cuando los que desprecian la verdad os fuercen a ella. Pero cuando el incrédulo obstinado os ataque, no vaciléis en defender vigorosamente la verdad que os ha salvado y santificado.

\par 
%\textsuperscript{(1932.2)}
\textsuperscript{178:1.17} A lo largo de todas las vicisitudes de la vida, recordad siempre que debéis amaros los unos a los otros. No luchéis contra los hombres, ni siquiera contra los incrédulos. Mostrad misericordia incluso a los que abusan de vosotros maliciosamente. Mostrad que sois unos ciudadanos leales, unos artesanos honrados, unos vecinos dignos de elogio, unos parientes dedicados, unos padres comprensivos y unos creyentes sinceros en la fraternidad del reino del Padre. Y mi espíritu estará con vosotros, ahora e incluso hasta el fin del mundo.

\par 
%\textsuperscript{(1932.3)}
\textsuperscript{178:1.18} Cuando Jesús hubo terminado su enseñanza, era casi la una, y regresaron inmediatamente al campamento, donde David y sus compañeros tenían preparado el almuerzo para ellos.

\section*{2. Después del almuerzo}
\par 
%\textsuperscript{(1932.4)}
\textsuperscript{178:2.1} Pocos oyentes del Maestro fueron capaces de entender ni siquiera una parte de su alocución matutina. De todos los que le escucharon, los griegos fueron quienes le comprendieron mejor. Incluso los once apóstoles se sintieron desconcertados por sus alusiones a futuros reinos políticos y a generaciones sucesivas de creyentes en el reino. Los seguidores más fervientes de Jesús no podían conciliar el final inminente de su ministerio terrenal con estas referencias a un futuro lejano de actividades evangélicas. Algunos de estos creyentes judíos empezaban a intuir que la tragedia más grande del mundo estaba a punto de suceder, pero no podían conciliar este desastre inminente con la actitud personal alegremente indiferente del Maestro, ni con su discurso matutino, en el que había aludido repetidas veces a las actividades futuras del reino celestial, que abarcarían enormes períodos de tiempo y englobarían relaciones con muchos reinos temporales sucesivos en la Tierra.

\par 
%\textsuperscript{(1932.5)}
\textsuperscript{178:2.2} Al mediodía de este día, todos los apóstoles y discípulos se habían enterado de que Lázaro había huido precipitadamente de Betania. Empezaron a intuir que los dirigentes judíos estaban implacablemente resueltos a exterminar a Jesús y sus enseñanzas.

\par 
%\textsuperscript{(1932.6)}
\textsuperscript{178:2.3} Gracias al trabajo de sus agentes secretos en Jerusalén, David Zebedeo estaba plenamente informado de los progresos del plan para detener y matar a Jesús. Lo sabía todo acerca del papel de Judas en este complot, pero nunca reveló este conocimiento a los otros apóstoles ni a ninguno de los discípulos. Poco después del almuerzo, llevó a Jesús aparte, y se atrevió a preguntarle si sabía... Pero nunca pudo terminar su pregunta. El Maestro levantó la mano para interrumpirle, diciendo: <<Sí, David, lo sé todo, y sé que tú lo sabes, pero procura no decírselo a nadie. Solamente, no dudes en tu propio corazón de que la voluntad de Dios acabará por prevalecer>>.

\par 
%\textsuperscript{(1933.1)}
\textsuperscript{178:2.4} Esta conversación con David fue interrumpida por la llegada de un mensajero de Filadelfia, que traía la noticia de que Abner había oído hablar del complot para matar a Jesús, y preguntaba si debía venir a Jerusalén. Este corredor salió apresuradamente hacia Filadelfia con el siguiente mensaje para Abner: <<Continúa con tu obra. Si me separo físicamente de vosotros, sólo es para poder regresar en espíritu. No os abandonaré. Estaré con vosotros hasta el fin>>.

\par 
%\textsuperscript{(1933.2)}
\textsuperscript{178:2.5} En ese momento, Felipe se acercó al Maestro y preguntó: <<Maestro, puesto que se acerca la hora de la Pascua, ¿dónde quieres que preparemos lo necesario para comerla?>> Cuando Jesús escuchó la pregunta de Felipe, respondió: <<Ve y trae a Pedro y a Juan, y os daré instrucciones para la cena que vamos a compartir esta noche. En cuanto a la Pascua, tendréis que deliberarlo después de que primero hayamos hecho esto>>.

\par 
%\textsuperscript{(1933.3)}
\textsuperscript{178:2.6} Cuando Judas escuchó al Maestro hablar de estas cuestiones con Felipe, se acercó para poder escuchar su conversación. Pero David Zebedeo, que estaba cerca, se adelantó y emprendió una conversación con Judas, mientras Felipe, Pedro y Juan se apartaban a un lado para hablar con el Maestro.

\par 
%\textsuperscript{(1933.4)}
\textsuperscript{178:2.7} Jesús dijo a los tres: <<Id inmediatamente a Jerusalén y cuando franqueéis la puerta, encontraréis a un hombre llevando un cántaro de agua. Él os hablará, y entonces lo seguiréis. Os conducirá hasta cierta casa, entrad detrás de él, y preguntadle al digno dueño de esa casa: `¿Dónde está la sala de los invitados donde el Maestro va a cenar con sus apóstoles?' Cuando hayáis preguntado esto, el dueño de la casa os enseñará una gran sala en la parte superior, provista de todo lo necesario y preparada para nosotros>>.

\par 
%\textsuperscript{(1933.5)}
\textsuperscript{178:2.8} Cuando los apóstoles llegaron a la ciudad, encontraron al hombre con el cántaro de agua cerca de la puerta, y lo siguieron hasta la casa de Juan Marcos, donde el padre del muchacho los recibió y les mostró la habitación de arriba preparada para la cena.

\par 
%\textsuperscript{(1933.6)}
\textsuperscript{178:2.9} Todo esto sucedió como resultado de un acuerdo concluido entre el Maestro y Juan Marcos durante la tarde del día anterior, cuando estaban solos en las colinas. Jesús quería estar seguro de que esta última comida con sus apóstoles transcurriría sin inquietudes. Pensaba que si Judas conocía de antemano el lugar de la reunión, podría ponerse de acuerdo con sus enemigos para arrestarlo, y por eso hizo este arreglo secreto con Juan Marcos. De esta manera, Judas no se enteró del lugar de la reunión hasta más tarde, cuando llegó allí en compañía de Jesús y de los otros apóstoles.

\par 
%\textsuperscript{(1933.7)}
\textsuperscript{178:2.10} David Zebedeo tenía muchos asuntos que tratar con Judas, por lo que resultó fácil impedir que siguiera a Pedro, Juan y Felipe, tal como deseaba hacerlo con tanta intensidad. Cuando Judas le dio a David cierta cantidad de dinero para las provisiones, David le dijo: <<Judas, dadas las circunstancias, ¿no sería oportuno que me proporcionaras un poco de dinero por adelantado para mis necesidades reales?>> Después de reflexionar un momento, Judas respondió: <<Sí, David, creo que sería sensato. De hecho, en vista de las condiciones inquietantes en Jerusalén, creo que sería mejor para mí que te entregue todo el dinero. Hay un complot contra el Maestro, y en el caso de que me sucediera algo, no tendrías dificultades>>.

\par 
%\textsuperscript{(1934.1)}
\textsuperscript{178:2.11} David recibió pues todos los fondos apostólicos en efectivo y los recibos del dinero en depósito. Los apóstoles no se enteraron de esta operación hasta el día siguiente por la noche.

\par 
%\textsuperscript{(1934.2)}
\textsuperscript{178:2.12} Eran aproximadamente las cuatro y media cuando los tres apóstoles regresaron e informaron a Jesús de que todo estaba dispuesto para la cena. El Maestro se preparó inmediatamente para conducir a sus doce apóstoles por el sendero que llevaba a la carretera de Betania, y desde allí hasta Jerusalén. Este fue el último desplazamiento que hizo con los doce.

\section*{3. Camino de la cena}
\par 
%\textsuperscript{(1934.3)}
\textsuperscript{178:3.1} Procurando de nuevo evitar las multitudes que cruzaban el valle de Cedrón de acá para allá entre el parque de Getsemaní y Jerusalén, Jesús y los doce pasaron por la cresta occidental del Monte de los Olivos para llegar a la carretera que descendía desde Betania hasta la ciudad. Cuando se acercaron al lugar donde Jesús se había detenido la noche anterior para hablar de la destrucción de Jerusalén, se detuvieron inconscientemente y permanecieron allí contemplando en silencio la ciudad. Como iban un poco temprano, y puesto que Jesús no deseaba atravesar la ciudad hasta después de la puesta del Sol, dijo a sus compañeros:

\par 
%\textsuperscript{(1934.4)}
\textsuperscript{178:3.2} <<Sentaos y descansad mientras hablo con vosotros sobre lo que dentro de poco ha de suceder. Todos estos años he vivido con vosotros como hermanos; os he enseñado la verdad sobre el reino de los cielos y os he revelado los misterios del mismo. Mi Padre ha hecho en verdad muchas obras maravillosas en conexión con mi misión en la Tierra. Habéis sido testigos de todo esto y habéis participado en la experiencia de ser compañeros de trabajo de Dios. Y sois testigos de que os he advertido durante algún tiempo que dentro de poco tendré que regresar a la tarea que el Padre me ha asignado; os he dicho claramente que debo dejaros en el mundo para continuar la obra del reino. Con esta finalidad os seleccioné en las colinas de Cafarnaúm. Ahora debéis prepararos para compartir con otros la experiencia que habéis tenido conmigo. Al igual que el Padre me envió a este mundo, estoy a punto de enviaros para que me representéis y terminéis la obra que he empezado>>.

\par 
%\textsuperscript{(1934.5)}
\textsuperscript{178:3.3} <<Contempláis esa ciudad con tristeza, porque habéis escuchado mis palabras sobre el fin de Jerusalén. Os he prevenido de antemano para que no perezcáis en su destrucción y se retrase así la proclamación del evangelio del reino. Os advierto asimismo que tengáis cuidado y no os expongáis innecesariamente al peligro cuando vengan a llevarse al Hijo del Hombre. Es indispensable que me vaya, pero vosotros debéis quedaros para dar testimonio de este evangelio cuando yo me haya ido, tal como le ordené a Lázaro que huyera de la ira de los hombres, para que pudiera vivir y dar a conocer la gloria de Dios. Si es voluntad del Padre que me vaya, nada de lo que hagáis podrá frustrar el plan divino. Cuidad de vosotros mismos para que no os maten también. Que vuestras almas defiendan valientemente el evangelio con el poder del espíritu, pero no os equivoquéis tratando tontamente de defender al Hijo del Hombre. No necesito ninguna protección humana; los ejércitos del cielo están cerca en este mismo momento; pero estoy decidido a hacer la voluntad de mi Padre que está en los cielos, y por eso debemos someternos a lo que muy pronto nos va a suceder>>.

\par 
%\textsuperscript{(1934.6)}
\textsuperscript{178:3.4} <<Cuando veáis esta ciudad destruida, no olvidéis que ya habéis entrado en la vida eterna de servicio perpetuo en el reino siempre en progreso del cielo, e incluso del cielo de los cielos. Deberíais saber que hay muchas moradas en el universo de mi Padre y en el mío, y que a los hijos de la luz les espera allí la revelación de unas ciudades cuyo constructor es Dios y de unos mundos cuyas costumbres de vida son la rectitud y la alegría en la verdad. Os he traído el reino de los cielos aquí a la Tierra, pero declaro que todos aquellos de vosotros que entren en él por la fe y permanezcan en él mediante el servicio viviente de la verdad, ascenderán con seguridad a los mundos superiores y se sentarán conmigo en el reino espiritual de nuestro Padre. Pero primero debéis ceñiros y completar la obra que habéis empezado conmigo. Primero debéis pasar por muchas tribulaciones y soportar muchas penas ---y esas pruebas son ahora inminentes--- y cuando hayáis terminado vuestro trabajo en la Tierra, vendréis a mi alegría, al igual que yo he terminado la obra de mi Padre en la Tierra, y estoy a punto de regresar a su abrazo>>.

\par 
%\textsuperscript{(1935.1)}
\textsuperscript{178:3.5} Cuando el Maestro terminó de hablar, se levantó y todos le siguieron mientras descendían el Olivete y entraban con él en la ciudad. Ninguno de los apóstoles, salvo tres, sabía adónde iban mientras caminaban por las estrechas calles a la caída de la noche. Las multitudes los empujaban, pero nadie los reconoció ni supo que el Hijo de Dios pasaba por allí camino de su última reunión como ser mortal con sus embajadores escogidos del reino. Y los apóstoles tampoco sabían que uno de ellos mismos ya había empezado a conspirar para traicionar al Maestro y entregarlo a sus enemigos.

\par 
%\textsuperscript{(1935.2)}
\textsuperscript{178:3.6} Juan Marcos los había seguido todo el camino hasta la ciudad, y después de que hubieron entrado por la puerta, corrió por otra calle, de manera que los estaba esperando para recibirlos cuando llegaran a la casa de su padre.


\chapter{Documento 179. La Última cena}
\par 
%\textsuperscript{(1936.1)}
\textsuperscript{179:0.1} DURANTE la tarde de este jueves, cuando Felipe le recordó al Maestro que se acercaba la Pascua y le preguntó sobre sus planes para celebrarla, estaba pensando en la cena pascual que debía tener lugar al día siguiente, viernes, por la noche. Era costumbre empezar los preparativos para la celebración de la Pascua, como muy tarde, al mediodía del día anterior. Como los judíos consideraban que el día comenzaba con la puesta del Sol, esto significaba que la cena pascual del sábado se celebraba el viernes por la noche, poco antes de la medianoche.

\par 
%\textsuperscript{(1936.2)}
\textsuperscript{179:0.2} Por esta razón, los apóstoles no lograban comprender en absoluto el anuncio del Maestro de que celebrarían la Pascua un día antes. Pensaban, al menos algunos de ellos, que Jesús sabía que sería arrestado antes de la hora de la cena pascual del viernes por la noche y que, por consiguiente, los reunía para una cena especial este jueves por la noche. Otros pensaban que se trataba simplemente de una ocasión especial, que precedería la celebración regular de la Pascua.

\par 
%\textsuperscript{(1936.3)}
\textsuperscript{179:0.3} Los apóstoles sabían que Jesús había celebrado otras Pascuas sin cordero; sabían que no participaba personalmente en ningún oficio del sistema judío que incluyera sacrificios. Había compartido muchas veces el cordero pascual como invitado, pero siempre que él era el anfitrión no se servía cordero. Para los apóstoles no habría sido una gran sorpresa que se hubiera suprimido el cordero incluso la noche de la Pascua, y puesto que esta cena tenía lugar un día antes, la falta de cordero pasó desapercibida.

\par 
%\textsuperscript{(1936.4)}
\textsuperscript{179:0.4} Después de que el padre y la madre de Juan Marcos les ofrecieron sus saludos de bienvenida, los apóstoles subieron inmediatamente a la sala de arriba, mientras Jesús se quedaba atrás charlando con la familia Marcos.

\par 
%\textsuperscript{(1936.5)}
\textsuperscript{179:0.5} Se había acordado de antemano que el Maestro celebraría este acontecimiento a solas con sus doce apóstoles; por lo tanto, no se había previsto que hubiera ningún criado para servirles.

\section*{1. El deseo de ser preferido}
\par 
%\textsuperscript{(1936.6)}
\textsuperscript{179:1.1} Cuando los apóstoles fueron conducidos al piso superior por Juan Marcos, contemplaron una sala amplia y cómoda que estaba completamente preparada la cena, y observaron que el pan, el vino, el agua y las hierbas estaban dispuestos en un extremo de la mesa. Salvo en este extremo donde se encontraban el pan y el vino, esta larga mesa estaba rodeada por trece triclinios, tal como hubiera estado preparada para la celebración de la Pascua en una familia judía adinerada.

\par 
%\textsuperscript{(1936.7)}
\textsuperscript{179:1.2} Mientras los doce entraban en esta habitación de arriba, observaron justo por dentro de la puerta los cántaros de agua, las palanganas y las toallas para lavar sus pies polvorientos; y puesto que no estaba previsto que ningún criado hiciera este servicio, los apóstoles empezaron a mirarse entre sí en cuanto Juan Marcos los hubo dejado, y cada uno empezó a pensar para sus adentros: ¿Quién va a lavarnos los pies? Y cada cual también pensó que él no sería el que actuaría así como servidor de los demás.

\par 
%\textsuperscript{(1937.1)}
\textsuperscript{179:1.3} Mientras permanecían allí de pie con este dilema en el corazón, examinaron la disposición de los asientos en la mesa, y observaron el diván más elevado del anfitrión, con un lecho a la derecha y los otros once dispuestos alrededor de la mesa hasta llegar al asiento opuesto a este segundo asiento de honor situado a la derecha del anfitrión.

\par 
%\textsuperscript{(1937.2)}
\textsuperscript{179:1.4} Esperaban la llegada del Maestro en cualquier momento, pero tenían la incertidumbre de si debían sentarse o esperar a que viniera para que les asignara sus sitios. Mientras titubeaban, Judas se dirigió al asiento de honor, a la izquierda del anfitrión, y manifestó que tenía la intención de recostarse allí como convidado preferido. Este acto de Judas provocó inmediatamente una violenta disputa entre los demás apóstoles. Apenas acababa Judas de ocupar el asiento de honor cuando Juan Zebedeo reclamó para sí el siguiente asiento preferido, el que se encontraba a la derecha del anfitrión. Simón Pedro se enfureció tanto con esta presunción de Judas y de Juan por ocupar los lugares de preferencia que, mientras los demás apóstoles observaban irritados, caminó alrededor de la mesa y se situó en el lecho más bajo, al final de la fila de asientos, exactamente enfrente del que había elegido Juan Zebedeo. Puesto que otros apóstoles habían ocupado los asientos elevados, Pedro pensó en elegir el más bajo, y lo hizo no solamente para protestar contra el orgullo indecente de sus hermanos, sino con la esperanza de que Jesús, cuando entrara y lo viera en el lugar menos honorífico, lo hiciera subir a uno más elevado, desplazando así a otro que se había atrevido a honrarse a sí mismo.

\par 
%\textsuperscript{(1937.3)}
\textsuperscript{179:1.5} Con las posiciones más elevadas y más bajas ya ocupadas, los demás apóstoles escogieron sus sitios, algunos cerca de Judas y otros cerca de Pedro, hasta que todos estuvieron instalados. Estaban sentados alrededor de la mesa en forma de U, en estos divanes reclinados, en el orden siguiente: a la derecha del Maestro, Juan; a la izquierda, Judas, Simón Celotes, Mateo, Santiago Zebedeo, Andrés, los gemelos Alfeo, Felipe, Natanael, Tomás y Simón Pedro.

\par 
%\textsuperscript{(1937.4)}
\textsuperscript{179:1.6} Están reunidos para celebrar, al menos en espíritu, una institución que databa incluso de un período anterior a Moisés y que se refería a la época en que sus antepasados eran esclavos en Egipto. Esta cena es su último encuentro con Jesús, e incluso en esta ocasión solemne, bajo la dirección de Judas, los apóstoles se dejan llevar una vez más por su vieja predilección por el honor, la preferencia y la exaltación personal.

\par 
%\textsuperscript{(1937.5)}
\textsuperscript{179:1.7} Aún estaban diciéndose recriminaciones irritadas cuando el Maestro apareció en la puerta, donde vaciló un instante mientras una expresión de desencanto se deslizaba lentamente por su rostro. Sin hacer ningún comentario se dirigió a su sitio, y no cambió la distribución de los asientos.

\par 
%\textsuperscript{(1937.6)}
\textsuperscript{179:1.8} Ahora estaban preparados para empezar la cena, salvo que aún no se habían lavado los pies, y que su estado de ánimo era de todo menos agradable. Cuando el Maestro llegó, aún se estaban haciendo comentarios desfavorables unos a otros, por no decir nada de los pensamientos de algunos de ellos, que tenían el suficiente control emocional como para abstenerse de expresar públicamente sus sentimientos.

\section*{2. El comienzo de la cena}
\par 
%\textsuperscript{(1937.7)}
\textsuperscript{179:2.1} Después de que el Maestro hubiera ocupado su lugar, no se dijo ni una palabra durante unos momentos. Jesús los examinó a todos y suavizó la tensión con una sonrisa, diciendo: <<He deseado mucho comer esta Pascua con vosotros. Quería comer una vez más con vosotros antes de mi sufrimiento, y sabiendo que mi hora ha llegado, he organizado esta cena con vosotros para esta noche porque, en cuanto al mañana, todos estamos en las manos del Padre, cuya voluntad he venido a hacer. No volveré a comer con vosotros hasta que os sentéis conmigo en el reino que mi Padre me dará cuando haya terminado aquello para lo que me envió a este mundo>>.

\par 
%\textsuperscript{(1938.1)}
\textsuperscript{179:2.2} Después de haber mezclado el agua y el vino, trajeron la copa a Jesús, y cuando la hubo recibido de las manos de Tadeo, la sostuvo mientras daba gracias. Cuando hubo terminado de dar gracias, dijo: <<Tomad esta copa y compartidla entre vosotros, y cuando la bebáis, sabed que no volveré a beber con vosotros el fruto de la vid puesto que ésta es nuestra última cena. Cuando nos sentemos de nuevo de esta manera, será en el reino venidero>>.

\par 
%\textsuperscript{(1938.2)}
\textsuperscript{179:2.3} Jesús empezó a hablar así a sus apóstoles porque sabía que su hora había llegado. Comprendía que había llegado el momento en que debía regresar al Padre, y que su obra en la Tierra estaba casi terminada. El Maestro sabía que había revelado el amor del Padre en la Tierra y había mostrado su misericordia a la humanidad, y que había completado aquello para lo que había venido al mundo, incluido el recibir todo el poder y la autoridad en el cielo y en la Tierra. Asimismo, sabía que Judas Iscariote había decidido plenamente entregarlo esta noche en manos de sus enemigos. Se daba completamente cuenta de que esta pérfida traición era obra de Judas, pero que también agradaba a Lucifer, Satanás y Caligastia, el príncipe de las tinieblas. Pero no le temía a ninguno de los que perseguían su derrota espiritual, así como tampoco a los que buscaban su muerte física. El Maestro sólo tenía una inquietud, y era la seguridad y la salvación de sus seguidores escogidos. Y así, sabiendo por completo que el Padre había puesto todas las cosas bajo su autoridad, el Maestro se preparó ahora para poner en práctica la parábola del amor fraterno.

\section*{3. El lavado de pies de los apóstoles}
\par 
%\textsuperscript{(1938.3)}
\textsuperscript{179:3.1} Después de beber la primera copa de la Pascua, era costumbre judía que el anfitrión se levantara de la mesa y se lavara las manos. En el transcurso de la comida y después de la segunda copa, todos los invitados se levantaban igualmente y se lavaban las manos. Puesto que los apóstoles sabían que su Maestro nunca guardaba estos ritos de lavado ceremonial de las manos, tenían mucha curiosidad por saber qué se proponía hacer después de que hubieran compartido esta primera copa. Jesús se levantó de la mesa y se dirigió silenciosamente hacia el lado de la puerta donde habían sido colocados los cántaros de agua, las palanganas y las toallas. Y su curiosidad se transformó en asombro cuando vieron que el Maestro se quitaba su manto, se ceñía una toalla y empezaba a echar agua en una de las palanganas para los pies. Imaginad la sorpresa de estos doce hombres, que se habían negado tan recientemente a lavarse los pies los unos a los otros, y que se habían enredado en disputas indecentes acerca de los lugares de honor en la mesa, cuando le vieron rodear el extremo libre de la mesa hasta llegar al asiento más bajo del festín, donde Simón Pedro estaba recostado, y arrodillándose como si fuera un criado, se preparó para lavarle los pies a Simón. Cuando el Maestro se arrodilló, los doce se levantaron como un solo hombre; incluso el traidor Judas olvidó por un momento su infamia hasta el punto de que se levantó con sus compañeros apóstoles en esta expresión de sorpresa, de respeto y de asombro total.

\par 
%\textsuperscript{(1938.4)}
\textsuperscript{179:3.2} Allí estaba de pie Simón Pedro, bajando la mirada hacia el rostro alzado de su Maestro. Jesús no dijo nada; no era necesario que hablara. Su actitud revelaba claramente que tenía la intención de lavar los pies de Simón Pedro. A pesar de sus debilidades humanas, Pedro amaba al Maestro. Este pescador galileo fue el primer ser humano que creyó de todo corazón en la divinidad de Jesús \textit{y} que confesó plena y públicamente esta creencia. Y desde entonces, Pedro nunca había dudado realmente de la naturaleza divina del Maestro. Puesto que Pedro veneraba y honraba así a Jesús en su corazón, no es de extrañar que a su alma le molestara la idea de que Jesús estuviera arrodillado allí delante de él como un vulgar criado, con el propósito de lavarle los pies como lo hubiera hecho un esclavo. Cuando Pedro recuperó las suficientes facultades como para dirigirse al Maestro, expresó los sentimientos internos de todos sus compañeros apóstoles.

\par 
%\textsuperscript{(1939.1)}
\textsuperscript{179:3.3} Después de unos momentos de gran desconcierto, Pedro dijo: <<Maestro, ¿tienes realmente la intención de lavarme los pies?>> Entonces, levantando la mirada hacia la cara de Pedro, Jesús dijo: <<Quizás no comprendes plenamente lo que estoy a punto de hacer, pero más adelante conocerás el significado de todas estas cosas>>. Entonces, Simón Pedro respiró profundamente y dijo: <<Maestro, ¡nunca me lavarás los pies!>> Y cada uno de los apóstoles aprobó con la cabeza la firme declaración de Pedro de negarse a permitir que Jesús se humillara de esta manera delante de ellos.

\par 
%\textsuperscript{(1939.2)}
\textsuperscript{179:3.4} El atractivo dramático de esta escena insólita conmovió al principio el corazón incluso de Judas Iscariote; pero cuando su intelecto vanidoso juzgó el espectáculo, concluyó que este gesto de humildad era simplemente un episodio más que probaba de manera concluyente que Jesús nunca estaría capacitado para ser el libertador de Israel, y que él, Judas, no había cometido un error al decidir abandonar la causa del Maestro.

\par 
%\textsuperscript{(1939.3)}
\textsuperscript{179:3.5} Mientras todos permanecían allí de pie sin aliento por el asombro, Jesús dijo: <<Pedro, te aseguro que si no te lavo los pies, no participarás conmigo en lo que estoy a punto de realizar>>. Cuando Pedro escuchó esta declaración, unida al hecho de que Jesús continuaba arrodillado allí a sus pies, tomó una de esas decisiones de sumisión ciega consistente en obedecer el deseo de aquel a quien respetaba y amaba. Cuando Simón Pedro empezó a darse cuenta de que este acto de servicio propuesto comportaba algún significado que determinaría la unión futura del interesado con la obra del Maestro, no solamente admitió la idea de permitir que Jesús le lavara los pies, sino que con su manera de ser característica e impetuosa, dijo: <<Entonces, Maestro, no me laves solamente los pies, sino también las manos y la cabeza>>.

\par 
%\textsuperscript{(1939.4)}
\textsuperscript{179:3.6} Mientras el Maestro se preparaba para empezar a lavar los pies de Pedro, dijo: <<El que ya está limpio, sólo necesita que le laven los pies. Vosotros que estáis sentados conmigo esta noche, estáis limpios ---pero no todos. Pero el polvo de vuestros pies debería haberse lavado antes de sentaros a comer conmigo. Además, quisiera hacer este servicio por vosotros como una parábola, para ilustrar el significado de un nuevo mandamiento que pronto os daré>>.

\par 
%\textsuperscript{(1939.5)}
\textsuperscript{179:3.7} De la misma manera, el Maestro se desplazó alrededor de la mesa, en silencio, lavando los pies de sus doce apóstoles, sin excluir siquiera a Judas. Cuando Jesús hubo terminado de lavar los pies de los doce, se puso su manto, volvió a su asiento de anfitrión, y después de examinar a sus apóstoles desconcertados, dijo:

\par 
%\textsuperscript{(1939.6)}
\textsuperscript{179:3.8} <<¿Comprendéis realmente lo que os he hecho? Me llamáis Maestro, y decís bien, porque lo soy. Así pues, si el Maestro os ha lavado los pies, ¿por qué no estabais dispuestos a lavaros los pies los unos a los otros? ¿Qué lección deberíais aprender de esta parábola en la que el Maestro hace tan gustosamente el servicio que sus hermanos eran reacios a hacerse los unos a los otros? En verdad, en verdad os lo digo: Un servidor no es más grande que su señor; ni el enviado es más grande que aquel que lo envía. Habéis visto en mi vida entre vosotros cómo se ha de servir, y benditos sean los que tengan el coraje misericordioso de servir así. Pero, ¿por qué sois tan lentos en aprender que el secreto de la grandeza en el reino espiritual no se parece a los métodos de poder del mundo material?>>

\par 
%\textsuperscript{(1940.1)}
\textsuperscript{179:3.9} <<Cuando entré esta noche en esta sala, no os contentabais con negaros orgullosamente a lavaros los pies los unos a los otros, sino que también teníais que discutir entre vosotros sobre quiénes ocuparían los lugares de honor en mi mesa. Esos honores los buscan los fariseos y los hijos de este mundo, pero no debería ser así entre los embajadores del reino celestial. ¿No sabéis que en mi mesa no puede haber ningún lugar de preferencia? ¿No comprendéis que amo a cada uno de vosotros como a los demás? ¿No sabéis que el asiento más cercano a mí, considerado como un honor por los hombres, no significa nada en lo que respecta a vuestra posición en el reino de los cielos? Sabéis que los reyes de los gentiles tienen el dominio sobre sus súbditos, y que a veces se les llama benefactores a los que ejercen esta autoridad. Pero no será así en el reino de los cielos. El que quiera ser grande entre vosotros, que se vuelva como el más joven; y el que quiera ser el jefe, que se convierta en el que sirve. ¿Quién es más grande, el que se sienta a comer, o el que sirve? ¿No se considera generalmente que el que se sienta a comer es el más grande? Pero observaréis que estoy entre vosotros como alguien que sirve. Si estáis dispuestos a ser compañeros míos en el servicio para hacer la voluntad del Padre, os sentaréis conmigo con poder en el reino venidero, haciendo sin cesar la voluntad del Padre en la gloria futura>>.

\par 
%\textsuperscript{(1940.2)}
\textsuperscript{179:3.10} Cuando Jesús hubo terminado de hablar, los gemelos Alfeo trajeron el pan y el vino, con las hierbas amargas y la pasta de frutos secos, que componían el plato siguiente de la
Última Cena.

\section*{4. Las últimas palabras al traidor}
\par 
%\textsuperscript{(1940.3)}
\textsuperscript{179:4.1} Los apóstoles comieron en silencio durante algunos minutos, pero debido a la influencia de la conducta jovial del Maestro, pronto se sintieron incitados a la conversación, y en muy poco rato la cena continuó como si no hubiera ocurrido nada fuera de lo común que alterara el buen humor y la armonía social de esta extraordinaria ocasión. Después de haber transcurrido cierto tiempo, hacia la mitad de este segundo servicio de la comida, Jesús los miró a todos diciendo: <<Os he dicho cuánto deseaba compartir esta cena con vosotros, y sabiendo de qué manera las fuerzas malignas de las tinieblas han conspirado para provocar la muerte del Hijo del Hombre, he decidido tomar esta cena con vosotros en esta sala secreta, un día antes de la Pascua, porque mañana por la noche a esta hora ya no estaré con vosotros. Os he repetido muchas veces que debo regresar al Padre. Ahora ha llegado mi hora, pero no era necesario que uno de vosotros me traicionara entregándome a mis enemigos>>.

\par 
%\textsuperscript{(1940.4)}
\textsuperscript{179:4.2} La parábola del lavado de los pies y el discurso posterior del Maestro ya habían hecho perder a los doce una buena parte de su presunción y de su confianza en sí mismos. Cuando escucharon esto, empezaron a mirarse unos a otros y a preguntarse vacilantes con tono desconcertado: <<¿Soy yo?>> Cuando todos hubieron preguntado esto, Jesús dijo: <<Aunque es necesario que regrese al Padre, no hacía falta que uno de vosotros se convirtiera en un traidor para cumplir la voluntad del Padre. Esto es la maduración del mal escondido en el corazón de uno que no ha logrado amar la verdad con toda su alma. ¡Cuán engañoso es el orgullo intelectual que precede a la caída espiritual! Mi amigo de muchos años, que ahora mismo come mi pan, está dispuesto a traicionarme, incluso ahora que mete su mano conmigo en el mismo plato>>.

\par 
%\textsuperscript{(1940.5)}
\textsuperscript{179:4.3} Cuando Jesús hubo hablado así, todos empezaron de nuevo a preguntar: <<¿Soy yo?>>. Cuando Judas, que estaba sentado a la izquierda de su Maestro, preguntó de nuevo: <<¿Soy yo?>>, Jesús mojó el pan en el plato de las hierbas y se lo dio a Judas diciendo: <<Tú lo has dicho>>. Pero los demás no escucharon a Jesús hablarle a Judas. Juan, que estaba recostado a la derecha de Jesús, se inclinó y le preguntó al Maestro: <<¿Quién es? Deberíamos saber quién se ha mostrado infiel a su deber>>. Jesús respondió: <<Ya os he dicho que es aquel a quien le he dado el pan mojado>>. Pero era tan natural que el anfitrión diera el pan mojado al que estaba sentado a su izquierda, que ninguno le prestó atención a este hecho, aunque el Maestro se hubiera expresado con toda claridad. Pero Judas era dolorosamente consciente del significado de las palabras del Maestro unidas a su acción, y empezó a temer que sus hermanos también se dieran cuenta ahora de que él era el traidor.

\par 
%\textsuperscript{(1941.1)}
\textsuperscript{179:4.4} Pedro estaba bastante excitado por lo que se había dicho; se inclinó sobre la mesa y se dirigió a Juan: <<Pregúntale quién es, o si te lo ha dicho, dime quién es el traidor>>.

\par 
%\textsuperscript{(1941.2)}
\textsuperscript{179:4.5} Jesús puso fin a sus cuchicheos diciendo: <<Me apena que este mal haya tenido que ocurrir y he esperado hasta este mismo momento que el poder de la verdad pudiera triunfar sobre los engaños del mal, pero esas victorias no se ganan sin la fe del amor sincero a la verdad. No hubiera querido deciros estas cosas en nuestra última cena, pero deseo advertiros de estas penas y prepararos así para lo que nos espera dentro de poco. Os he dicho esto porque deseo que recordéis, después de mi partida, que conocía todos estos perversos complots, y que os avisé de que iba a ser traicionado. Hago todo esto únicamente para que os sintáis fortalecidos en las tentaciones y pruebas que os esperan>>.

\par 
%\textsuperscript{(1941.3)}
\textsuperscript{179:4.6} Después de haber hablado así, Jesús se inclinó hacia Judas y le dijo: <<Lo que has decidido hacer, hazlo enseguida>>. Cuando Judas escuchó estas palabras, se levantó de la mesa y abandonó apresuradamente la habitación, saliendo a la noche para hacer lo que había decidido llevar a cabo. Cuando los otros apóstoles vieron que Judas salía precipitadamente después de que Jesús le hubiera hablado, creyeron que había ido a buscar algo más para la cena o a hacer algún otro recado para el Maestro, pues suponían que aún tenía la bolsa.

\par 
%\textsuperscript{(1941.4)}
\textsuperscript{179:4.7} Jesús sabía ahora que no se podía hacer nada para impedir que Judas se convirtiera en un traidor. Había empezado con doce hombres ---ahora tenía once. Había elegido a seis de estos apóstoles, y aunque Judas se encontraba entre los que habían sido nombrados por sus primeros apóstoles escogidos, el Maestro lo había aceptado, y hasta este mismo momento había hecho todo lo posible por santificarlo y salvarlo, tal como había trabajado por la paz y la salvación de los demás.

\par 
%\textsuperscript{(1941.5)}
\textsuperscript{179:4.8} Esta cena, con sus tiernos episodios y sus detalles suaves, fue el último llamamiento de Jesús al desertor Judas, pero fue en vano. Una vez que el amor está realmente muerto, aunque las advertencias se hagan con el máximo de tacto y se transmitan con el espíritu más cariñoso, por regla general sólo intensifican el odio y encienden la malvada resolución de llevar a cabo íntegramente nuestros propios proyectos egoístas.

\section*{5. El establecimiento de la cena del recuerdo}
\par 
%\textsuperscript{(1941.6)}
\textsuperscript{179:5.1} Cuando trajeron a Jesús la tercera copa de vino, la <<copa de la bendición>>, se levantó del diván, tomó la copa en sus manos y la bendijo, diciendo: <<Tomad todos esta copa, y bebed de ella. Ésta será la copa de mi recuerdo. Ésta es la copa de la bendición de una nueva dispensación de gracia y de verdad. Será para vosotros el emblema de la donación y del ministerio del Espíritu divino de la Verdad. No volveré a beber esta copa con vosotros hasta que beba de una nueva forma con vosotros en el reino eterno del Padre>>.

\par 
%\textsuperscript{(1942.1)}
\textsuperscript{179:5.2} Mientras bebían esta copa de la bendición con un profundo respeto y en un silencio perfecto, todos los apóstoles sintieron que estaba teniendo lugar algo fuera de lo común. La vieja Pascua conmemoraba la salida de sus padres de un estado de esclavitud racial a otro de libertad individual; ahora, el Maestro instituía una nueva cena de conmemoración como símbolo de la nueva dispensación en la que el individuo esclavizado emerge del cautiverio del ceremonialismo y del egoísmo, y pasa a la alegría espiritual de la fraternidad y la comunidad de los hijos por la fe, liberados, que pertenecen al Dios vivo.

\par 
%\textsuperscript{(1942.2)}
\textsuperscript{179:5.3} Cuando terminaron de beber esta nueva copa del recuerdo, el Maestro cogió el pan y, después de dar gracias, lo rompió en pedazos y les pidió que lo pasaran, diciendo:<<Tomad este pan del recuerdo y comedlo. Os he dicho que yo soy el pan de la vida. Y este pan de la vida es la vida unida del Padre y del Hijo en un solo don. La palabra del Padre, tal como es revelada en el Hijo, es en verdad el pan de la vida>>. Cuando hubieron compartido el pan de la conmemoración, símbolo de la palabra viviente de la verdad encarnada en la similitud de la carne mortal, todos se sentaron.

\par 
%\textsuperscript{(1942.3)}
\textsuperscript{179:5.4} Al instituir esta cena del recuerdo, el Maestro recurrió, como siempre tenía costumbre, a las parábolas y a los símbolos. Empleó símbolos porque quería enseñar ciertas grandes verdades espirituales de tal manera que a sus sucesores les resultara difícil atribuir a sus palabras interpretaciones precisas y significados definidos. De esta manera, trataba de impedir que las generaciones siguientes cristalizaran su enseñanza y vincularan sus significados espirituales con las cadenas muertas de la tradición y de los dogmas. Al establecer la única ceremonia, o sacramento, asociada a la totalidad de la misión de su vida, Jesús se esmeró mucho en \textit{sugerir} sus significados, en lugar de recurrir a \textit{definicionesprecisas}. No quería destruir el concepto individual de la comunión divina, estableciendo una práctica precisa; tampoco deseaba limitar la imaginación espiritual del creyente, restringiéndola de manera formalista. Trataba más bien de liberar el alma renacida del hombre para que emprendiera el vuelo con las alas gozosas de una libertad espiritual nueva y viviente.

\par 
%\textsuperscript{(1942.4)}
\textsuperscript{179:5.5} A pesar del esfuerzo del Maestro por establecer así este nuevo sacramento de conmemoración, aquellos que le siguieron en los siglos posteriores se encargaron de frustrar eficazmente su deseo expreso, en el sentido de que este simple simbolismo espiritual de aquella última noche en la carne ha sido reducido a interpretaciones precisas y sometido a la precisión casi matemática de una fórmula fija. De todas las enseñanzas de Jesús, ninguna ha sido más reglamentada por la tradición.

\par 
%\textsuperscript{(1942.5)}
\textsuperscript{179:5.6} Cuando la cena del recuerdo es compartida por aquellos que creen en el Hijo y conocen a Dios, su simbolismo no necesita estar asociado a ninguna de las falsas interpretaciones pueriles del hombre sobre el significado de la presencia divina, porque en todas esas ocasiones, el Maestro está \textit{realmente presente}. La cena del recuerdo es el encuentro simbólico del creyente con Miguel. Cuando os volvéis así conscientes del espíritu, el Hijo está realmente presente, y su espíritu fraterniza con el fragmento interior de su Padre.

\par 
%\textsuperscript{(1942.6)}
\textsuperscript{179:5.7} Después de que hubieron meditado unos momentos, Jesús continuó hablando: <<Cuando hagáis estas cosas, recordad la vida que he vivido en la Tierra entre vosotros, y regocijaos con el hecho de que voy a continuar viviendo en la Tierra con vosotros y sirviendo a través de vosotros. Como individuos, no discutáis entre vosotros sobre quién será el más grande. Sed todos como hermanos. Cuando el reino crezca hasta abarcar grandes grupos de creyentes, deberíais absteneros también de luchar por la grandeza o de buscar la preferencia entre esos grupos>>.

\par 
%\textsuperscript{(1943.1)}
\textsuperscript{179:5.8} Este importante acontecimiento tuvo lugar en la habitación superior de un amigo. Ni la cena ni el edificio contenían ninguna forma sagrada o consagración ceremonial. La cena del recuerdo fue establecida sin aprobación eclesiástica.

\par 
%\textsuperscript{(1943.2)}
\textsuperscript{179:5.9} Cuando Jesús hubo establecido así la cena del recuerdo, dijo a sus apóstoles: <<Cada vez que hagáis esto, hacedlo en memoria mía. Y cuando os acordéis de mí, reflexionad primero sobre mi vida en la carne, recordad que en otro tiempo estuve con vosotros, y luego discernid por la fe que todos cenaréis conmigo algún día en el reino eterno del Padre. Ésta es la nueva Pascua que os dejo, el recuerdo mismo de mi vida de donación, la palabra de la verdad eterna; y de mi amor por vosotros, os dejo la efusión de mi Espíritu de la Verdad sobre todo el género humano>>.

\par 
%\textsuperscript{(1943.3)}
\textsuperscript{179:5.10} Y terminaron la celebración de esta antigua pero incruenta Pascua en conexión con la inauguración de la nueva cena del recuerdo, cantando todos juntos el salmo ciento dieciocho.


\chapter{Documento 180. El discurso de despedida}
\par 
%\textsuperscript{(1944.1)}
\textsuperscript{180:0.1} DESPUÉS de cantar el salmo al final de la última cena, los apóstoles pensaron que Jesús tenía la intención de regresar inmediatamente al campamento, pero les indicó que se sentaran. El Maestro dijo:

\par 
%\textsuperscript{(1944.2)}
\textsuperscript{180:0.2} <<Recordáis bien cuando os envié sin bolsa ni alforja, e incluso os aconsejé que no llevarais ninguna ropa de repuesto. Y todos recordaréis que no os faltó de nada. Pero ahora os encontráis en tiempos difíciles. Ya no podéis contar con la buena voluntad de las multitudes. De aquí en adelante, el que tenga una bolsa que la lleve con él. Cuando salgáis al mundo para proclamar este evangelio, encargaos de vuestro sostén como os parezca más conveniente. He venido para traer la paz, pero ésta no aparecerá durante un tiempo>>.

\par 
%\textsuperscript{(1944.3)}
\textsuperscript{180:0.3} <<Ha llegado la hora de que el Hijo del Hombre sea glorificado, y el Padre será glorificado en mí. Amigos míos, sólo voy a estar con vosotros un poco más de tiempo. Pronto me buscaréis, pero no me encontraréis, porque voy a un lugar donde no podéis venir en este momento. Pero cuando hayáis terminado vuestra obra en la Tierra tal como yo he terminado la mía, entonces vendréis a mí como yo me preparo ahora para ir hacia mi Padre. Voy a dejaros dentro de muy poco tiempo y no me veréis más en la Tierra, pero todos me veréis en la era venidera cuando ascendáis al reino que mi Padre me ha dado>>.

\section*{1. El nuevo mandamiento}
\par 
%\textsuperscript{(1944.4)}
\textsuperscript{180:1.1} Después de unos momentos de conversación informal, Jesús se levantó y dijo: <<Cuando representé para vosotros una parábola que indicaba de qué manera deberíais estar dispuestos a serviros los unos a los otros, dije que deseaba daros un nuevo mandamiento; quisiera hacerlo ahora que estoy a punto de dejaros. Conocéis bien el mandamiento que ordena que os améis los unos a los otros; que améis a vuestro prójimo como a vosotros mismos. Pero incluso esta dedicación sincera por parte de mis hijos no me satisface plenamente. Quisiera que realizarais unos actos de amor aún más grandes en el reino de la fraternidad de los creyentes. Y por eso os doy este nuevo mandamiento: Que os améis los unos a los otros como yo os he amado. De esta manera, si os amáis así los unos a los otros, todos los hombres sabrán que sois mis discípulos>>.

\par 
%\textsuperscript{(1944.5)}
\textsuperscript{180:1.2} <<Al daros este nuevo mandamiento, no pongo ninguna nueva carga sobre vuestra alma; os traigo más bien una nueva alegría y os doy la posibilidad de experimentar un nuevo placer, conociendo las delicias de dar el afecto de vuestro corazón a vuestros semejantes. Incluso soportando un dolor externo, estoy a punto de experimentar la alegría suprema de daros mi afecto a vosotros y a vuestros compañeros mortales>>.

\par 
%\textsuperscript{(1944.6)}
\textsuperscript{180:1.3} <<Cuando os invito a que os améis los unos a los otros como yo os he amado, os presento la medida suprema del verdadero afecto, porque nadie puede tener un amor más grande que éste: el de dar la vida por sus amigos. Y vosotros sois mis amigos; seguiréis siendo mis amigos con que sólo estéis dispuestos a hacer lo que os he enseñado. Me habéis llamado Maestro, pero yo no os llamo sirvientes. Si tan sólo os amáis los unos a los otros como yo os amo, seréis mis amigos y siempre os hablaré de lo que el Padre me revela>>.

\par 
%\textsuperscript{(1945.1)}
\textsuperscript{180:1.4} <<No simplemente me habéis elegido vosotros, sino que yo también os he elegido, y os he ordenado para que salgáis al mundo a fin de ofrecer el fruto del servicio amoroso a vuestros semejantes, tal como yo he vivido entre vosotros y os he revelado al Padre. El Padre y yo trabajaremos con vosotros, y vosotros experimentaréis la divina plenitud de la alegría con que sólo obedezcáis mi mandamiento de amaros los unos a los otros como yo os he amado>>.

\par 
%\textsuperscript{(1945.2)}
\textsuperscript{180:1.5} Si queréis compartir el gozo del Maestro, tenéis que compartir su amor. Y compartir su amor significa que habéis compartido su servicio. Esta experiencia de amor no os libera de las dificultades de este mundo; no crea un mundo nuevo, pero hace con toda seguridad que el viejo mundo resulte nuevo.

\par 
%\textsuperscript{(1945.3)}
\textsuperscript{180:1.6} Retened en la memoria: Lo que Jesús pide es la lealtad, no el sacrificio. La conciencia de hacer un sacrificio implica la ausencia de ese afecto sincero que hubiera convertido ese servicio amoroso en una alegría suprema. La idea del \textit{deber} significa que tenéis una mentalidad de sirvientes, y a consecuencia de ello no conseguís la grandísima emoción de hacer vuestro servicio como un amigo y por un amigo. El impulso de la amistad trasciende todas las convicciones del deber, y el servicio que un amigo hace por un amigo nunca se puede llamar sacrificio. El Maestro ha enseñado a los apóstoles que son hijos de Dios. Los ha llamado hermanos, y ahora, antes de irse, los llama sus amigos.

\section*{2. La vid y los sarmientos}
\par 
%\textsuperscript{(1945.4)}
\textsuperscript{180:2.1} Luego, Jesús se levantó de nuevo y continuó enseñando a sus apóstoles: <<Yo soy la verdadera vid, y mi Padre el viñador. Yo soy la vid, y vosotros los sarmientos. El Padre sólo me pide que produzcáis muchos frutos. La vid solamente se poda para aumentar la fecundidad de sus sarmientos. Todo sarmiento estéril que sale de mí, el Padre lo cortará. Todo sarmiento que produzca fruto, el Padre lo limpiará para que pueda producir más frutos. Vosotros ya estáis limpios por la palabra que he pronunciado, pero debéis continuar estando limpios. Tenéis que permanecer en mí, y yo en vosotros; el sarmiento muere si se le separa de la vid. Así como el sarmiento no puede producir frutos a menos que permanezca en la vid, vosotros tampoco podéis producir los frutos del servicio amoroso a menos que permanezcáis en mí. Recordad: Yo soy la verdadera vid, y vosotros los sarmientos vivientes. El que vive en mí, y yo en él, producirá muchos frutos del espíritu y experimentará la alegría suprema de dar esta cosecha espiritual. Si mantenéis esta unión espiritual viviente conmigo, produciréis un fruto abundante. Si permanecéis en mí y mis palabras viven en vosotros, podréis comulgar libremente conmigo, y entonces mi espíritu viviente se infiltrará en vosotros de tal manera que podréis pedir todo lo que mi espíritu quiere, y hacer todo esto con la seguridad de que el Padre nos concederá nuestra petición. El Padre es glorificado en esto: que la vid tenga muchos sarmientos vivientes, y que cada sarmiento produzca muchos frutos. Y cuando el mundo vea estos sarmientos fructíferos ---mis amigos que se aman los unos a los otros como yo los he amado--- todos los hombres sabrán que sois realmente mis discípulos>>.

\par 
%\textsuperscript{(1945.5)}
\textsuperscript{180:2.2} <<Así como el Padre me ha amado, yo os he amado. Vivid en mi amor como yo vivo en el amor del Padre. Si hacéis lo que os he enseñado, permaneceréis en mi amor al igual que yo he guardado la palabra del Padre y permanezco eternamente en su amor>>.

\par 
%\textsuperscript{(1946.1)}
\textsuperscript{180:2.3} Los judíos habían enseñado desde hacía mucho tiempo que el Mesías sería <<un tallo que surgiría de la vid>> de los antepasados de David, y en conmemoración de esta antigua enseñanza, un gran emblema de la uva unida a su vid decoraba la entrada del templo de Herodes. Todos los apóstoles recordaron estas cosas mientras el Maestro les hablaba esta noche en la habitación de arriba.

\par 
%\textsuperscript{(1946.2)}
\textsuperscript{180:2.4} Pero más adelante, las conclusiones del Maestro sobre la oración fueron malinterpretadas, lo que produjo una gran pesadumbre. Estas enseñanzas hubieran provocado pocas dificultades si se hubieran recordado las palabras exactas del Maestro y hubieran sido transcritas fielmente con posterioridad. Pero de la manera en que se escribió el relato, los creyentes terminaron por considerar la oración en nombre de Jesús como una especie de magia suprema, creyendo que recibirían del Padre todo lo que pidieran. Durante siglos, las almas sinceras han continuado haciendo naufragar su fe contra este escollo. ¿Cuánto tiempo necesitará el mundo de los creyentes para comprender que la oración no es un proceso para conseguir lo que uno desea, sino más bien un programa para emprender el camino de Dios, una experiencia para aprender a reconocer y a ejecutar la voluntad del Padre? Es enteramente cierto que, cuando vuestra voluntad se ha alineado verdaderamente con la suya, podéis pedir cualquier cosa concebida por esta unión de voluntades, y os será concedida. Esta unión de voluntades se efectúa por medio de Jesús y a través de él, al igual que la vida de la vid circula y atraviesa los sarmientos vivientes.

\par 
%\textsuperscript{(1946.3)}
\textsuperscript{180:2.5} Cuando existe esta conexión viviente entre la divinidad y la humanidad, si la humanidad reza sin reflexión y de manera ignorante por sus comodidades egoístas y sus éxitos vanidosos, sólo puede haber una respuesta divina: que los tallos de los sarmientos vivientes produzcan una mayor cantidad de frutos del espíritu. Cuando el sarmiento de la vid está vivo, todas sus peticiones sólo pueden recibir una respuesta: que produzca más uvas. De hecho, el sarmiento sólo existe para producir frutos, y no puede hacer otra cosa que producir uvas. Y así, el verdadero creyente sólo existe con la finalidad de producir los frutos del espíritu: amar a los hombres como él mismo ha sido amado por Dios ---que nos amemos los unos a los otros como Jesús nos ha amado.

\par 
%\textsuperscript{(1946.4)}
\textsuperscript{180:2.6} Cuando la mano disciplinaria del Padre se coloca sobre la vid, lo hace con amor, para que los sarmientos puedan producir muchos frutos. Un sabio viñador sólo corta las ramas muertas y estériles.

\par 
%\textsuperscript{(1946.5)}
\textsuperscript{180:2.7} Jesús tuvo grandes dificultades para hacer que sus mismos apóstoles reconocieran que la oración es una función de los creyentes nacidos del espíritu, en el reino dominado por el espíritu.

\section*{3. La enemistad del mundo}
\par 
%\textsuperscript{(1946.6)}
\textsuperscript{180:3.1} Apenas habían terminado los once sus comentarios sobre el discurso de la vid y los sarmientos, el Maestro les indicó que deseaba continuar hablándoles, pues sabía que le quedaba poco tiempo, y dijo: <<Cuando os haya dejado, no os desaniméis por la enemistad del mundo. No os sintáis abatidos ni siquiera cuando los creyentes pusilánimes se vuelvan contra vosotros y se alíen con los enemigos del reino. Si el mundo os odia, recordad que me ha odiado antes que a vosotros. Si fuerais de este mundo, entonces el mundo amaría lo que es suyo, pero como no lo sois, el mundo se niega a amaros. Estáis en este mundo, pero no debéis vivir a su manera. Os he elegido y apartado del mundo para que representéis el espíritu de otro mundo en este mismo mundo en el que habéis sido elegidos. Pero recordad siempre las palabras que os he dicho: El servidor no es más grande que su señor. Si se atreven a perseguirme, también os perseguirán a vosotros. Si mis palabras ofenden a los incrédulos, vuestras palabras también ofenderán a los impíos. Y os harán todo esto porque no creen en mí ni en Aquel que me ha enviado; por eso sufriréis muchas cosas a causa de mi evangelio; pero cuando estéis soportando esas tribulaciones, deberíais recordar que yo también he sufrido antes que vosotros a causa de este evangelio del reino celestial>>.

\par 
%\textsuperscript{(1947.1)}
\textsuperscript{180:3.2} <<Muchos de los que os atacarán ignoran la luz del cielo, pero éste no es el caso de algunos que nos persiguen ahora. Si no les hubiéramos enseñado la verdad, podrían hacer muchas cosas extrañas sin incurrir en la condenación, pero ahora, puesto que han conocido la luz y se han atrevido a rechazarla, no tienen excusas para su actitud. El que me odia, odia a mi Padre. No puede ser de otra manera; la luz que podría salvaros si la aceptáis, sólo puede condenaros si la rechazáis a sabiendas. ¿Qué les he hecho a esos hombres para que me odien con un odio tan terrible? Nada, salvo ofrecerles la hermandad en la Tierra y la salvación en el cielo. Pero ¿no habéis leído en las Escrituras el dicho: `Y me odiaron sin causa'?>>

\par 
%\textsuperscript{(1947.2)}
\textsuperscript{180:3.3} <<Pero no os dejaré solos en el mundo. Poco tiempo después de mi partida, os enviaré un ayudante espiritual. Tendréis con vosotros a alguien que ocupará mi lugar entre vosotros, a alguien que continuará enseñándoos el camino de la verdad, y que incluso os confortará>>.

\par 
%\textsuperscript{(1947.3)}
\textsuperscript{180:3.4} <<Que no se turbe vuestro corazón. Creéis en Dios; continuad creyendo también en mí. Aunque tenga que dejaros, no estaré lejos de vosotros. Ya os he dicho que hay muchas residencias en el universo de mi Padre. Si no fuera verdad, no os habría hablado repetidas veces de ellas. Voy a regresar a esos mundos de luz, a esas estaciones en el cielo del Padre a las que ascenderéis algún día. Desde esos lugares he venido a este mundo, y ahora ha llegado el momento en que debo regresar a la obra de mi Padre en las esferas de arriba>>.

\par 
%\textsuperscript{(1947.4)}
\textsuperscript{180:3.5} <<Si os precedo así en el reino celestial del Padre, os enviaré a buscar con seguridad para que podáis estar conmigo en los lugares que fueron preparados para los hijos mortales de Dios antes de que existiera este mundo. Aunque debo dejaros, estaré presente con vosotros en espíritu, y finalmente estaréis conmigo en persona cuando hayáis ascendido hasta mí en mi universo, tal como yo estoy a punto de ascender hasta mi Padre en su universo más grande. Lo que os he dicho es verdad y eterno, aunque no podáis comprenderlo plenamente. Voy hacia el Padre, y aunque ahora no podáis seguirme, me seguiréis sin duda en las eras venideras>>.

\par 
%\textsuperscript{(1947.5)}
\textsuperscript{180:3.6} Cuando Jesús se sentó, Tomás se levantó y dijo: <<Maestro, no sabemos adónde vas; por consiguiente, no conocemos el camino. Pero te seguiremos esta misma noche si nos muestras el camino>>.

\par 
%\textsuperscript{(1947.6)}
\textsuperscript{180:3.7} Cuando Jesús escuchó a Tomás, contestó: <<Tomás, yo soy el camino, la verdad y la vida. Nadie va hacia el Padre si no es a través de mí. Todos los que encuentran al Padre, primero me encuentran a mí. Si me conocéis, conocéis el camino hacia el Padre. Y me conocéis de hecho, porque habéis vivido conmigo y ahora me veis>>.

\par 
%\textsuperscript{(1947.7)}
\textsuperscript{180:3.8} Pero esta enseñanza era demasiado profunda para muchos de los apóstoles, especialmente para Felipe, quien después de hablar unas palabras con Natanael, se levantó y dijo: <<Maestro, muéstranos al Padre, y todo lo que nos has dicho se aclarará>>.

\par 
%\textsuperscript{(1947.8)}
\textsuperscript{180:3.9} Cuando Felipe hubo hablado, Jesús dijo: <<Felipe, ¿he estado tanto tiempo contigo y sin embargo ni siquiera me conoces ahora? Declaro de nuevo que aquel que me ha visto, ha visto al Padre. ¿Cómo puedes decir entonces: muéstranos al Padre? ¿Acaso no crees que yo estoy en el Padre y el Padre en mí? ¿No os he enseñado que las palabras que os digo no son mis palabras, sino las palabras del Padre? Hablo por el Padre y no por mí mismo. Estoy en este mundo para hacer la voluntad del Padre, y eso es lo que he hecho. Mi Padre permanece en mí y trabaja a través de mí. Creedme cuando digo que el Padre está en mí, y que yo estoy en el Padre, o si no, creedme por la vida misma que he vivido ---por mi obra>>.

\par 
%\textsuperscript{(1948.1)}
\textsuperscript{180:3.10} Mientras el Maestro se apartaba para refrescarse con agua, los once emprendieron una viva discusión sobre estas enseñanzas, y Pedro iba a empezar a pronunciar un largo discurso cuando Jesús regresó y les indicó que se sentaran.

\section*{4. El ayudante prometido}
\par 
%\textsuperscript{(1948.2)}
\textsuperscript{180:4.1} Jesús continuó su enseñanza, diciendo: <<Cuando haya regresado al Padre, y él haya aceptado plenamente la obra que he realizado por vosotros en la Tierra, y después de que haya recibido la soberanía final sobre mi propio dominio, le diré a mi Padre: Como he dejado solos a mis hijos en la Tierra, es conforme a mi promesa enviarles a otro instructor. Y cuando el Padre lo apruebe, derramaré el Espíritu de la Verdad sobre todo el género humano. El espíritu de mi Padre ya está en vuestro corazón, y cuando llegue ese día, también me tendréis con vosotros como ahora tenéis al Padre. Este nuevo don es el espíritu de la verdad viviente. Los incrédulos no escucharán al principio las enseñanzas de este espíritu, pero todos los hijos de la luz lo recibirán con placer y de todo corazón. Cuando llegue este espíritu, lo conoceréis como me habéis conocido a mí, recibiréis este don en vuestro corazón, y él permanecerá con vosotros. Podéis percibir que no os voy a dejar sin ayuda ni guía. No voy a dejaros abandonados. Actualmente sólo puedo estar con vosotros en persona. En los tiempos venideros estaré con vosotros y con todos los demás hombres que deseen mi presencia, dondequiera que estéis, y con cada uno de vosotros al mismo tiempo. ¿No discernís que es mejor que me vaya, que os deje físicamente, para poder estar mejor y más plenamente con vosotros en espíritu?>>

\par 
%\textsuperscript{(1948.3)}
\textsuperscript{180:4.2} <<Dentro de muy pocas horas el mundo dejará de verme; pero continuaréis conociéndome en vuestro corazón hasta que os envíe este nuevo instructor, el Espíritu de la Verdad. De la misma manera que he vivido con vosotros en persona, viviré entonces dentro de vosotros; seré una sola cosa con vuestra experiencia personal en el reino del espíritu. Cuando esto suceda, sabréis con seguridad que estoy en el Padre, y que, aunque vuestra vida está oculta con el Padre que está en mí, yo también estoy en vosotros. He amado al Padre y he guardado su palabra; vosotros me habéis amado, y guardaréis mi palabra. Al igual que mi Padre me ha dado de su espíritu, yo os daré del mío. Este Espíritu de la Verdad que os donaré os guiará y os confortará, y os conducirá finalmente a toda la verdad>>.

\par 
%\textsuperscript{(1948.4)}
\textsuperscript{180:4.3} <<Os cuento estas cosas mientras aún estoy con vosotros, a fin de que estéis mejor preparados para soportar las pruebas que ahora se avecinan. Cuando llegue ese nuevo día, estaréis habitados tanto por el Hijo como por el Padre. Y esos dones del cielo trabajarán siempre el uno con el otro, al igual que el Padre y yo hemos trabajado en la Tierra delante de vuestros propios ojos como una sola persona: el Hijo del Hombre. Este amigo espiritual os traerá a la memoria todo lo que os he enseñado>>.

\par 
%\textsuperscript{(1948.5)}
\textsuperscript{180:4.4} Mientras el Maestro se detenía un momento, Judas Alfeo se atrevió a hacer una de las pocas preguntas que él o su hermano hicieron a Jesús en público. Judas dijo: <<Maestro, siempre has vivido entre nosotros como un amigo; ¿cómo te conoceremos cuando ya no te manifiestes a nosotros, salvo a través de este espíritu? Si el mundo no te ve, ¿cómo estaremos seguros de ti? ¿Cómo te mostrarás a nosotros?>>

\par 
%\textsuperscript{(1949.1)}
\textsuperscript{180:4.5} Jesús los miró a todos, sonrió y dijo: <<Hijos míos, me voy, voy de vuelta hacia el Padre. Dentro de poco ya no me veréis como me veis aquí, en carne y hueso. Dentro de muy poco tiempo os enviaré a mi espíritu, que es como yo, a excepción de este cuerpo material. Este nuevo instructor es el Espíritu de la Verdad que vivirá con cada uno de vosotros, en vuestro corazón, y así todos los hijos de la luz serán como uno solo y serán atraídos los unos hacia los otros. De esta manera concreta mi Padre y yo podremos vivir en el alma de cada uno de vosotros, y también en el corazón de todos los demás hombres que nos aman y hacen real ese amor en sus experiencias, amándose los unos a los otros como yo os amo ahora>>.

\par 
%\textsuperscript{(1949.2)}
\textsuperscript{180:4.6} Judas Alfeo no comprendió plenamente lo que había dicho el Maestro, pero captó la promesa de un nuevo instructor, y por la expresión de la cara de Andrés, percibió que su pregunta había sido contestada satisfactoriamente.

\section*{5. El Espíritu de la Verdad}
\par 
%\textsuperscript{(1949.3)}
\textsuperscript{180:5.1} El nuevo ayudante que Jesús prometió enviar al corazón de los creyentes, derramar sobre todo el género humano, es el \textit{Espíritu de la Verdad}. Este don divino no es la letra o la ley de la verdad, ni tampoco está destinado a funcionar como la forma o la expresión de la verdad. El nuevo instructor es la \textit{convicciónde la verdad}, la conciencia y la seguridad de los verdaderos significados en los niveles espirituales reales. Este nuevo instructor es el espíritu de la verdad viviente y creciente, de la verdad que se expande, se desarrolla y se adapta.

\par 
%\textsuperscript{(1949.4)}
\textsuperscript{180:5.2} La verdad divina es una realidad viviente que es percibida por el espíritu. La verdad sólo existe en los niveles espirituales superiores de la comprensión de la divinidad y de la conciencia de la comunión con Dios. Podéis conocer la verdad, y podéis vivir la verdad; podéis experimentar el crecimiento de la verdad en el alma, y gozar de la libertad que su luz aporta a la mente, pero no podéis aprisionar la verdad en unas fórmulas, códigos, credos o modelos intelectuales de conducta humana. Cuando intentáis formular humanamente la verdad divina, ésta muere rápidamente. Incluso en el mejor de los casos, el salvamento póstumo de la verdad aprisionada sólo puede terminar en la realización de una forma particular de sabiduría intelectual glorificada. La verdad estática es una verdad muerta, y sólo la verdad muerta puede ser formulada en una teoría. La verdad viviente es dinámica y sólo puede gozar de una existencia experiencial en la mente humana.

\par 
%\textsuperscript{(1949.5)}
\textsuperscript{180:5.3} La inteligencia nace de una existencia material que está iluminada por la presencia de la mente cósmica. La sabiduría consta de la conciencia del conocimiento, elevada a nuevos niveles de significados, y activada por la presencia de la dotación universal del espíritu ayudante de la sabiduría. La verdad es un valor de la realidad espiritual que sólo lo experimentan los seres dotados de espíritu que ejercen su actividad en los niveles supermateriales de conciencia del universo, y que después de reconocer la verdad, permiten que su espíritu activador viva y reine en sus almas.

\par 
%\textsuperscript{(1949.6)}
\textsuperscript{180:5.4} El verdadero hijo que posee perspicacia universal busca el Espíritu viviente de la Verdad en toda palabra sabia. La persona que conoce a Dios eleva constantemente la sabiduría a los niveles de verdad viviente donde se alcanza la divinidad; el alma que no progresa espiritualmente arrastra todo el tiempo a la verdad viviente hacia los niveles muertos de la sabiduría y hacia los dominios de la simple exaltación del conocimiento.

\par 
%\textsuperscript{(1949.7)}
\textsuperscript{180:5.5} Cuando la regla de oro está despojada de la perspicacia suprahumana del Espíritu de la Verdad, no es nada más que una regla de conducta altamente ética. Cuando la regla de oro se interpreta literalmente, puede convertirse en un instrumento muy ofensivo para vuestros semejantes. Sin un discernimiento espiritual de la regla de oro de la sabiduría, podéis razonar que, puesto que deseáis que todos los hombres os digan con franqueza toda la verdad que tienen en su mente, vosotros deberíais expresarles de manera franca y total todos los pensamientos de vuestra mente. Una interpretación tan poco espiritual de la regla de oro podría ocasionar una infelicidad indecible y unas penas sin fin.

\par 
%\textsuperscript{(1950.1)}
\textsuperscript{180:5.6} Algunas personas disciernen e interpretan la regla de oro como una afirmación puramente intelectual de la fraternidad humana. Otras experimentan esta expresión de las relaciones humanas como una satisfacción emocional de los tiernos sentimientos de la personalidad humana. Otros mortales reconocen esta misma regla de oro como la vara que mide todas las relaciones sociales, el modelo de la conducta social. Y otros aún la consideran como el mandato positivo de un gran instructor moral, que incorporó en esta declaración el concepto más elevado de la obligación moral en lo concerniente a todas las relaciones fraternales. En la vida de esos seres morales, la regla de oro se convierte en el centro sabio y la circunferencia de toda su filosofía.

\par 
%\textsuperscript{(1950.2)}
\textsuperscript{180:5.7} En el reino de la fraternidad creyente de los amantes de la verdad que conocen a Dios, esta regla de oro adquiere cualidades vivientes de realización espiritual en aquellos niveles superiores de interpretación que inducen a los hijos mortales de Dios a considerar que este mandato del Maestro les exige que se relacionen con sus semejantes de tal manera, que éstos reciban el mayor bien posible como resultado de su contacto con los creyentes. Ésta es la esencia de la verdadera religión: que améis a vuestro prójimo como a vosotros mismos.

\par 
%\textsuperscript{(1950.3)}
\textsuperscript{180:5.8} Pero la comprensión más elevada y la interpretación más verdadera de la regla de oro consiste en la conciencia del espíritu de la verdad de la realidad perdurable y viviente de esta declaración divina. El verdadero significado cósmico de esta regla de las relaciones universales solamente se revela en su comprensión espiritual, en la interpretación que el espíritu del Hijo hace de la ley de la conducta al espíritu del Padre que reside en el alma del hombre mortal. Cuando esos mortales conducidos por el espíritu se dan cuenta del verdadero significado de esta regla de oro, se llenan a rebosar con la certeza de ser ciudadanos de un universo amistoso, y sus ideales de realidad espiritual sólo se satisfacen cuando aman a sus semejantes como Jesús nos amó a todos. Ésta es la realidad de la comprensión del amor de Dios.

\par 
%\textsuperscript{(1950.4)}
\textsuperscript{180:5.9} Esta misma filosofía de flexibilidad viviente y de adaptabilidad cósmica de la verdad divina a las necesidades y capacidades individuales de cada hijo de Dios, ha de ser percibida antes de que podáis esperar comprender adecuadamente la enseñanza y la práctica del Maestro de la no resistencia al mal. La enseñanza del Maestro es básicamente una declaración espiritual. Incluso las implicaciones materiales de su filosofía no pueden considerarse con utilidad independientemente de sus correlaciones espirituales. El espíritu del mandato del Maestro consiste en no oponer resistencia a todas las reacciones egoístas hacia el universo, y al mismo tiempo alcanzar de manera dinámica y progresiva los niveles rectos de los verdaderos valores espirituales: la belleza divina, la bondad infinita y la verdad eterna ---conocer a Dios y volverse cada vez más como él.

\par 
%\textsuperscript{(1950.5)}
\textsuperscript{180:5.10} El amor, el altruismo, debe sufrir una interpretación readaptativa constante y viviente de las relaciones de acuerdo con las directrices del Espíritu de la Verdad. El amor debe captar así los conceptos ampliados y siempre cambiantes del bien cósmico más elevado para la persona que es amada. Luego, el amor continúa adoptando esta misma actitud hacia todas las demás personas que quizás pudieran ser influidas por las relaciones crecientes y vivientes del amor que un mortal conducido por el espíritu siente por otros ciudadanos del universo. Toda esta adaptación viviente del amor debe efectuarse a la luz del entorno de mal presente y de la meta eterna de la perfección del destino divino.

\par 
%\textsuperscript{(1950.6)}
\textsuperscript{180:5.11} Y así, tenemos que reconocer claramente que ni la regla de oro ni la enseñanza de la no resistencia se pueden entender nunca correctamente como dogmas o preceptos. Sólo se pueden comprender viviéndolas, percatándose de sus significados en la interpretación viviente del Espíritu de la Verdad, que dirige el contacto afectuoso entre los seres humanos.

\par 
%\textsuperscript{(1951.1)}
\textsuperscript{180:5.12} Y todo esto indica claramente la diferencia entre la antigua religión y la nueva. La antigua religión enseñaba la abnegación; la nueva religión sólo enseña el olvido de sí mismo, una autorrealización elevada gracias al servicio social unido a la comprensión del universo. La antigua religión estaba motivada por la conciencia del miedo; el nuevo evangelio del reino está dominado por la convicción de la verdad, el espíritu de la verdad eterna y universal. En la experiencia de la vida de los creyentes en el reino, ninguna cantidad de piedad o de lealtad a un credo puede compensar la ausencia de esa amabilidad espontánea, generosa y sincera que caracteriza a los hijos del Dios viviente nacidos del espíritu. Ni la tradición, ni un sistema ceremonial de culto oficial, pueden compensar la falta de compasión auténtica por nuestros semejantes.

\section*{6. La necesidad de partir}
\par 
%\textsuperscript{(1951.2)}
\textsuperscript{180:6.1} Después de que Pedro, Santiago, Juan y Mateo hubieron hecho numerosas preguntas al Maestro, éste continuó su discurso de despedida, diciendo: <<Os cuento todo esto antes de dejaros, a fin de que podáis estar preparados de tal manera para lo que os va a suceder, que no cometáis graves errores. Las autoridades no se contentarán con expulsaros simplemente de las sinagogas; os advierto que se acerca la hora en que aquellos que os maten pensarán que están haciendo un servicio a Dios. Os harán todas estas cosas a vosotros y a los que conduzcáis al reino de los cielos, porque no conocen al Padre. Se han negado a conocer al Padre al negarse a recibirme; y se negarán a recibirme cuando os rechacen a vosotros, a condición de que hayáis guardado mi nuevo mandamiento de que os améis los unos a los otros como yo os he amado. Os cuento de antemano estas cosas para que cuando llegue vuestra hora, como ahora ha llegado la mía, os sintáis fortalecidos por el conocimiento de que yo sabía todo esto, y que mi espíritu estará con vosotros en todo lo que sufriréis por mi causa y a causa del evangelio. Por este motivo os he hablado tan claramente desde el principio. Os he advertido incluso que los enemigos de un hombre pueden ser los miembros de su propia familia. Aunque este evangelio del reino nunca deja de traer una gran paz al alma del creyente individual, no traerá la paz a la Tierra hasta que los hombres no estén dispuestos a creer de todo corazón en mis enseñanzas, y a establecer la práctica de hacer la voluntad del Padre como meta principal de su vida mortal>>.

\par 
%\textsuperscript{(1951.3)}
\textsuperscript{180:6.2} <<Ahora que os dejo, puesto que ha llegado la hora en que estoy a punto de ir hacia el Padre, me sorprende que ninguno de vosotros me haya preguntado: ¿Por qué nos dejas? Sin embargo, sé que os hacéis estas preguntas en vuestro corazón. Os hablaré claramente, como lo hace un amigo a otro. Es realmente beneficioso para vosotros que me vaya. Si no me voy, el nuevo instructor no podrá venir a vuestro corazón. Debo ser despojado de este cuerpo mortal, y restablecido en mi puesto en el cielo, antes de poder enviar a este instructor espiritual para que viva en vuestra alma y conduzca a vuestro espíritu a la verdad. Cuando mi espíritu llegue para residir en vosotros, iluminará la diferencia entre el pecado y la rectitud, y os permitirá juzgar sabiamente en vuestro corazón acerca de ambas cosas>>.

\par 
%\textsuperscript{(1951.4)}
\textsuperscript{180:6.3} <<Aún tengo que deciros muchas cosas, pero ahora no podéis soportar más. Sin embargo, cuando llegue el Espíritu de la Verdad, os guiará finalmente a toda la verdad a medida que paséis por las muchas moradas del universo de mi Padre>>.

\par 
%\textsuperscript{(1951.5)}
\textsuperscript{180:6.4} <<Este espíritu no hablará de sí mismo, pero os proclamará lo que el Padre le ha revelado al Hijo, e incluso os mostrará cosas por venir; me glorificará como yo he glorificado a mi Padre. Este espíritu sale de mí y os revelará mi verdad. Todo lo que el Padre posee en este dominio ahora es mío; por eso os he dicho que este nuevo instructor tomará lo que es mío y os lo revelará>>.

\par 
%\textsuperscript{(1952.1)}
\textsuperscript{180:6.5} <<Dentro de muy poco os dejaré por un corto período de tiempo. Después, cuando me veáis de nuevo, ya estaré camino del Padre, de manera que, incluso entonces, no me veréis por mucho tiempo>>.

\par 
%\textsuperscript{(1952.2)}
\textsuperscript{180:6.6} Mientras hacía una corta pausa, los apóstoles empezaron a hablar entre ellos: <<¿Qué es lo que nos está diciendo? `Dentro de muy poco os dejaré', y `cuando me veáis de nuevo no será por mucho tiempo, porque estaré camino del Padre'. ¿Qué quiere decir con este `poco tiempo' y `no por mucho tiempo'? No podemos comprender lo que nos dice>>.

\par 
%\textsuperscript{(1952.3)}
\textsuperscript{180:6.7} Puesto que Jesús sabía que se hacían estas preguntas, dijo: <<¿Os preguntáis unos a otros sobre lo que he querido decir cuando he indicado que dentro de muy poco ya no estaré con vosotros y que, cuando me veáis de nuevo, estaré camino del Padre? Os he dicho claramente que el Hijo del Hombre debe morir, pero que resucitará. ¿No podéis discernir pues el significado de mis palabras? Primero os sentiréis apenados, pero más tarde os regocijaréis con muchas personas que comprenderán estas cosas después de que hayan sucedido. En verdad, una mujer está angustiada a la hora del parto, pero una vez que ha dado a luz a su hijo, olvida inmediatamente su angustia ante la alegría de saber que un hombre ha nacido en el mundo. De la misma manera, os vais a entristecer por mi partida, pero os veré pronto de nuevo, y entonces vuestra pena se transformará en alegría, y os llegará una nueva revelación de la salvación de Dios que nadie podrá quitaros nunca. Y todos los mundos serán benditos en esta misma revelación de la vida que derrota a la muerte. Hasta ahora habéis hecho todas vuestras peticiones en nombre de mi Padre. Después de que me veáis de nuevo, podréis pedir también en mi nombre, y yo os escucharé>>.

\par 
%\textsuperscript{(1952.4)}
\textsuperscript{180:6.8} <<Aquí abajo os he enseñado con proverbios y os he hablado en parábolas. Lo he hecho así porque espiritualmente sólo erais niños; pero se acerca el momento en que os hablaré claramente sobre el Padre y su reino. Y lo haré porque el Padre mismo os ama y desea ser revelado más plenamente a vosotros. El hombre mortal no puede ver al Padre que es espíritu; por eso he venido al mundo para mostrar el Padre a vuestros ojos de criaturas. Pero cuando os hayáis perfeccionado en el crecimiento espiritual, entonces veréis al Padre mismo>>.

\par 
%\textsuperscript{(1952.5)}
\textsuperscript{180:6.9} Cuando los once le oyeron hablar así, se dijeron unos a otros: <<Mirad, ahora nos habla claramente. Es seguro que el Maestro ha venido de Dios. Pero, ¿por qué dice que debe regresar al Padre?>> Jesús vio que incluso entonces no le comprendían. Estos once hombres no podían liberarse de las ideas que habían abrigado durante mucho tiempo sobre el concepto judío del Mesías. Cuanto más plenamente creían en Jesús como Mesías, más embarazosas se volvían estas nociones profundamente arraigadas sobre el glorioso triunfo material del reino en la Tierra.


\chapter{Documento 181. Las últimas recomendaciones y advertencias}
\par 
%\textsuperscript{(1953.1)}
\textsuperscript{181:0.1} DESPUÉS de terminar el discurso de despedida a los once, Jesús conversó familiarmente con ellos y recordó muchas experiencias que les concernía como grupo y como individuos. Estos galileos empezaban por fin a darse cuenta de que su amigo e instructor iba a dejarlos, y su esperanza se aferraba a la promesa de que después de poco tiempo estaría de nuevo con ellos, pero eran propensos a olvidar que este regreso también sería por poco tiempo. Muchos apóstoles y discípulos principales creían realmente que esta promesa de volver durante una corta temporada (el corto intervalo entre la resurrección y la ascensión) indicaba que Jesús sólo se iba para conversar brevemente con su Padre, después de lo cual volvería para establecer el reino. Esta interpretación de su enseñanza concordaba tanto con sus creencias preconcebidas como con sus ardientes esperanzas. Puesto que sus creencias de toda la vida y sus esperanzas de ver realizados sus anhelos se armonizaban de esta manera, no les fue difícil encontrar una interpretación de las palabras del Maestro que justificara sus intensos deseos.

\par 
%\textsuperscript{(1953.2)}
\textsuperscript{181:0.2} Después de haber debatido el discurso de despedida y de haber empezado a asimilarlo, Jesús llamó de nuevo a los apóstoles al orden y empezó a impartirles sus últimas recomendaciones y advertencias.

\section*{1. Las últimas palabras de consuelo}
\par 
%\textsuperscript{(1953.3)}
\textsuperscript{181:1.1} Cuando los once se hubieron sentado, Jesús se levantó y les dirigió la palabra: <<Mientras que esté con vosotros en la carne, sólo puedo ser una persona en medio de vosotros o en el mundo entero. Pero cuando haya sido liberado de esta envoltura de naturaleza mortal, podré regresar como habitante espiritual a cada uno de vosotros y de todos los demás creyentes en este evangelio del reino. De esta manera, el Hijo del Hombre se volverá una encarnación espiritual en el alma de todos los verdaderos creyentes>>.

\par 
%\textsuperscript{(1953.4)}
\textsuperscript{181:1.2} <<Cuando haya regresado para vivir en vosotros y trabajar a través de vosotros, podré continuar conduciéndoos mejor por esta vida y guiaros a través de las muchas moradas en la vida futura en el cielo de los cielos. La vida en la creación eterna del Padre no es un descanso sin fin en la ociosidad ni un reposo egoísta, sino más bien una progresión continua en la gracia, la verdad y la gloria. Cada una de las muchísimas estaciones en la casa de mi Padre es una parada, una vida destinada a prepararos para la siguiente. Los hijos de la luz continuarán así de gloria en gloria hasta que alcancen el estado divino en el que estarán perfeccionados espiritualmente como el Padre es perfecto en todas las cosas>>.

\par 
%\textsuperscript{(1953.5)}
\textsuperscript{181:1.3} <<Si queréis seguir mis pasos cuando os haya dejado, esforzaos seriamente por vivir de acuerdo con el espíritu de mis enseñanzas y el ideal de mi vida ---hacer la voluntad de mi Padre. Haced esto, en lugar de intentar imitar mi vida sencilla en la carne tal como me he visto obligado a vivirla, necesariamente, en este mundo>>.

\par 
%\textsuperscript{(1954.1)}
\textsuperscript{181:1.4} <<El Padre me ha enviado a este mundo, pero sólo unos pocos de vosotros habéis elegido recibirme plenamente. Derramaré mi espíritu sobre todo el género humano, pero no todos los hombres escogerán recibir a este nuevo instructor como guía y consejero del alma. Pero todos los que lo reciban serán iluminados, purificados y confortados. Y este Espíritu de la Verdad se transformará en ellos en una fuente de agua viva que brotará hasta en la vida eterna>>.

\par 
%\textsuperscript{(1954.2)}
\textsuperscript{181:1.5} <<Y ahora que estoy a punto de dejaros, quisiera decir unas palabras de consuelo. Os dejo la paz; mi paz os doy. Os concedo estos dones, no como los ofrece el mundo ---por medidas--- sino que doy a cada uno de vosotros todo lo que quiera recibir. Que vuestro corazón no se perturbe ni sienta temor. Yo he vencido al mundo, y en mí todos triunfaréis por la fe. Os he advertido que el Hijo del Hombre será ejecutado, pero os aseguro que volveré antes de ir hacia el Padre, aunque sólo sea por poco tiempo. Y después de haber ascendido hasta el Padre, enviaré con seguridad al nuevo instructor para que esté con vosotros y resida en vuestro propio corazón. Cuando veáis que sucede todo esto, no os desalentéis, sino más bien creed, puesto que lo sabíais todo de antemano. Os he amado con un gran afecto y no quisiera dejaros, pero esa es la voluntad del Padre. Mi hora ha llegado>>.

\par 
%\textsuperscript{(1954.3)}
\textsuperscript{181:1.6} <<No dudéis de ninguna de estas verdades, incluso cuando estéis dispersos por las persecuciones y abatidos por numerosas tristezas. Cuando os sintáis solos en el mundo, yo conoceré vuestra soledad, al igual que vosotros conoceréis la mía cuando estéis dispersos cada uno por su lado, dejando al Hijo del Hombre en manos de sus enemigos. Pero nunca estoy solo; el Padre siempre está conmigo. Incluso en esos momentos rezaré por vosotros. Os he contado todas estas cosas para que podáis tener paz y tenerla más abundantemente. Tendréis tribulaciones en este mundo, pero tened buen ánimo; he triunfado en el mundo y os he mostrado el camino de la alegría eterna y del servicio perpetuo>>.

\par 
%\textsuperscript{(1954.4)}
\textsuperscript{181:1.7} Jesús da la paz a los que hacen con él la voluntad de Dios, pero esta paz no es semejante a las alegrías y satisfacciones de este mundo material. Los materialistas y los fatalistas incrédulos sólo pueden esperar disfrutar de dos tipos de paz y de consuelo del alma: o bien deben ser estoicos, decididos a enfrentarse a lo inevitable y a soportar lo peor con una resolución firme; o bien deben ser optimistas, contentándose siempre con esa esperanza que brota perpetuamente en el seno del hombre, anhelando en vano una paz que nunca llega realmente.

\par 
%\textsuperscript{(1954.5)}
\textsuperscript{181:1.8} Cierta cantidad de estoicismo y de optimismo son útiles para vivir la vida en la Tierra, pero ninguno de los dos tiene nada que ver con esa paz espléndida que el Hijo de Dios confiere a sus hermanos en la carne. La paz que Miguel da a sus hijos de la Tierra es la misma paz que llenaba su propia alma cuando él mismo vivía la vida mortal en la carne y en este mismo mundo. La paz de Jesús es la alegría y la satisfacción de una persona que conoce a Dios, y que ha logrado el triunfo de aprender plenamente a hacer la voluntad de Dios mientras vive la vida mortal en la carne. La paz mental de Jesús estaba fundada en una fe humana absoluta en la realidad de los cuidados sabios y compasivos del Padre divino. Jesús tuvo dificultades en la Tierra, incluso se le había llamado falsamente el <<hombre de dolores>>, pero en todas estas experiencias y a través de ellas, disfrutó del consuelo de esa confianza que siempre le dio fuerzas para seguir adelante con el objetivo de su vida, con la plena seguridad de que estaba realizando la voluntad del Padre.

\par 
%\textsuperscript{(1954.6)}
\textsuperscript{181:1.9} Jesús era decidido, perseverante y estaba completamente dedicado a realizar su misión, pero no era un estoico insensible y endurecido; siempre buscaba los aspectos alegres en las experiencias de su vida, pero no era un optimista ciego que se engañara a sí mismo. El Maestro sabía todo lo que le sucedería, y no tenía miedo. Después de haber otorgado esta paz a cada uno de sus seguidores, podía decir de manera coherente: <<Que vuestro corazón no se perturbe ni sienta temor>>.

\par 
%\textsuperscript{(1955.1)}
\textsuperscript{181:1.10} La paz de Jesús es pues la paz y la seguridad de un hijo que cree plenamente que su carrera en el tiempo y en la eternidad está totalmente a salvo bajo el cuidado y la vigilancia de un Padre espíritu infinitamente sabio, amoroso y poderoso. Ésta es, en verdad, una paz que sobrepasa el entendimiento de la mente mortal, pero que el corazón humano creyente puede disfrutar plenamente.

\section*{2. Las recomendaciones personales de despedida}
\par 
%\textsuperscript{(1955.2)}
\textsuperscript{181:2.1} El Maestro había terminado de dar sus instrucciones de despedida y de impartir sus exhortaciones finales a los apóstoles como grupo. Luego se dirigió a ellos para decirles adiós individualmente y para darle a cada uno sus consejos personales así como su bendición de despedida. Los apóstoles continuaban sentados alrededor de la mesa tal como se habían instalado al principio para compartir la Última Cena. A medida que el Maestro rodeaba la mesa y hablaba con ellos, cada uno se ponía de pie cuando Jesús se dirigía a él.

\par 
%\textsuperscript{(1955.3)}
\textsuperscript{181:2.2} A Juan, Jesús le dijo: <<Tú, Juan, eres el más joven de mis hermanos. Has estado muy cerca de mí, y aunque os amo a todos con el mismo amor que un padre tiene por sus hijos, Andrés te designó como uno de los tres que siempre debían estar cerca de mí. Además de esto, te has ocupado en mi nombre de muchos asuntos relacionados con mi familia terrenal, y debes continuar haciéndolo. Y voy hacia el Padre, Juan, teniendo la plena confianza de que continuarás cuidando de los que son míos en la carne. Procura que la confusión que sufren actualmente sobre mi misión no te impida en absoluto concederles toda la simpatía, los consejos y la ayuda necesarios, como sabes que yo lo haría si tuviera que permanecer en la carne. Y cuando todos lleguen a ver la luz y entren plenamente en el reino, aunque todos vosotros los recibiréis con regocijo, cuento contigo Juan para darles la bienvenida en mi nombre>>.

\par 
%\textsuperscript{(1955.4)}
\textsuperscript{181:2.3} <<Y ahora que comienzo las últimas horas de mi carrera terrenal, permanece cerca de mí para que pueda dejarte cualquier mensaje relacionado con mi familia. En lo que concierne a la obra que el Padre me confió, ahora está terminada a excepción de mi muerte en la carne, y estoy listo para beber esta última copa. Pero en cuanto a las responsabilidades que me dejó mi padre terrenal José, las he atendido durante mi vida, pero ahora debo contar contigo para que actúes en mi nombre en todos esos asuntos. Te he elegido para que hagas esto por mí, Juan, porque eres el más joven, y por consiguiente es muy probable que vivas más tiempo que los otros apóstoles>>.

\par 
%\textsuperscript{(1955.5)}
\textsuperscript{181:2.4} <<En otro tiempo os llamamos a ti y a tu hermano los hijos del trueno. Empezaste con nosotros siendo resuelto e intolerante, pero has cambiado mucho desde el día en que querías que hiciera bajar el fuego sobre la cabeza de los incrédulos ignorantes e irreflexivos. Y debes cambiar aún más. Deberías convertirte en el apóstol del nuevo mandamiento que os he dado esta noche. Dedica tu vida a enseñar a tus hermanos a amarse los unos a los otros como yo os he amado>>.

\par 
%\textsuperscript{(1955.6)}
\textsuperscript{181:2.5} Mientras Juan Zebedeo permanecía allí de pie en la habitación de arriba con las lágrimas corriendo por sus mejillas, miró de frente al Maestro y dijo: <<Así lo haré, Maestro mío, pero, ¿cómo puedo aprender a amar más a mis hermanos?>> Entonces Jesús respondió: <<Aprenderás a amar más a tus hermanos cuando primero aprendas a amar más a su Padre que está en los cielos, y después de que te intereses realmente más por su bienestar en el tiempo y en la eternidad. Todo interés humano de este tipo se fomenta mediante la simpatía comprensiva, el servicio desinteresado y el perdón sin límites. Nadie debería menospreciar tu juventud, pero te exhorto a que siempre consideres debidamente el hecho de que la edad representa muchas veces la experiencia, y que en los asuntos humanos nada puede reemplazar a la experiencia real. Esfuérzate por vivir en paz con todos los hombres, especialmente con tus amigos en la fraternidad del reino celestial. Y recuerda siempre, Juan, no luches con las almas que quisieras ganar para el reino>>.

\par 
%\textsuperscript{(1956.1)}
\textsuperscript{181:2.6} Luego el Maestro rodeó su propio asiento y se detuvo un momento al lado del sitio de Judas Iscariote. Los apóstoles estaban un poco sorprendidos de que Judas aún no hubiera regresado, y tenían mucha curiosidad por conocer el significado de la expresión de tristeza en el rostro de Jesús, mientras éste permanecía al lado del asiento vacío del traidor. Pero ninguno de ellos, a excepción quizás de Andrés, albergaba la más leve sospecha de que su tesorero había salido para traicionar a su Maestro, tal como Jesús les había dado a entender anteriormente por la tarde y durante la cena. Habían sucedido tantas cosas que, por el momento, habían olvidado por completo la declaración del Maestro de que uno de ellos lo traicionaría.

\par 
%\textsuperscript{(1956.2)}
\textsuperscript{181:2.7} Jesús se acercó entonces a Simón Celotes, que se levantó para escuchar la siguiente exhortación: <<Eres un verdadero hijo de Abraham, pero cuánto tiempo he estado intentando hacer de ti un hijo de este reino celestial. Te amo y todos tus hermanos también te aman. Sé que me amas, Simón, y que también amas al reino, pero aún tienes la idea fija de hacer venir este reino según tus preferencias. Sé muy bien que acabarás por captar la naturaleza y el significado espirituales de mi evangelio, y que trabajarás valientemente para proclamarlo, pero me preocupa lo que pueda sucederte cuando yo me vaya. Me alegraría saber que no vacilarás; sería feliz si pudiera saber que después de que me vaya hacia el Padre no dejarás de ser mi apóstol, y que te comportarás aceptablemente como embajador del reino celestial>>.

\par 
%\textsuperscript{(1956.3)}
\textsuperscript{181:2.8} Apenas había terminado Jesús de hablar a Simón Celotes, cuando el fogoso patriota, secándose los ojos, respondió: <<Maestro, no temas por mi lealtad. Le he dado la espalda a todo para poder dedicar mi vida al establecimiento de tu reino en la Tierra, y no titubearé. Hasta ahora he sobrevivido a todas las decepciones, y no te abandonaré>>.

\par 
%\textsuperscript{(1956.4)}
\textsuperscript{181:2.9} Entonces, poniendo su mano en el hombro de Simón, Jesús dijo: <<En verdad, es confortante oírte hablar así, especialmente en un momento como éste, pero mi buen amigo, aún no sabes de qué estás hablando. No dudo ni un instante de tu lealtad, de tu devoción. Sé que no dudarías en salir a luchar y morir por mí, como lo harían todos estos otros>> (y todos asintieron enérgicamente con la cabeza), <<pero no se te pedirá eso. Te he dicho repetidas veces que mi reino no es de este mundo, y que mis discípulos no lucharán para establecerlo. Te he dicho esto muchas veces, Simón, pero te niegas a enfrentarte a la verdad. No me preocupa tu lealtad hacia mí y hacia el reino, sino ¿qué harás cuando me vaya y caigas por fin en la cuenta de que no has sabido captar el significado de mi enseñanza, y que debes ajustar tus ideas erróneas a la realidad de una clase de asuntos, diferente y espiritual, en el reino?>>

\par 
%\textsuperscript{(1956.5)}
\textsuperscript{181:2.10} Simón quería hablar de nuevo, pero Jesús levantó la mano para detenerlo y continuó diciendo: <<Ninguno de mis apóstoles tiene un corazón más sincero y honrado que tú, pero después de mi partida, ninguno de ellos se sentirá tan trastornado y tan desanimado como tú. Durante todo tu desánimo mi espíritu permanecerá contigo, y éstos, tus hermanos, no te abandonarán. No olvides lo que te he enseñado en cuanto a la relación entre la ciudadanía en la Tierra y la filiación en el reino espiritual del Padre. Reflexiona bien sobre todo lo que te he dicho acerca de dar al César las cosas que son del César y a Dios las que son de Dios. Dedica tu vida, Simón, a mostrar que el hombre mortal puede cumplir aceptablemente mi mandato de reconocer simultáneamente el deber temporal hacia los poderes civiles y el servicio espiritual en la fraternidad del reino. Si te dejas enseñar por el Espíritu de la Verdad, nunca habrá conflicto entre las exigencias de la ciudadanía en la Tierra y las de la filiación en el cielo, a menos que los gobernantes temporales se atrevan a exigirte el homenaje y la adoración que sólo pertenecen a Dios>>.

\par 
%\textsuperscript{(1957.1)}
\textsuperscript{181:2.11} <<Y ahora, Simón, cuando veas finalmente todo esto, una vez que te hayas liberado de tu depresión y hayas salido a proclamar este evangelio con una gran energía, no olvides nunca que yo estaba contigo durante todo tu período de desánimo, y que continuaré contigo hasta el fin. Siempre serás mi apóstol, y una vez que estés dispuesto a ver con los ojos del espíritu y a someter más plenamente tu voluntad a la voluntad del Padre que está en los cielos, volverás a trabajar como embajador mío, y nadie te quitará la autoridad que te he conferido porque hayas sido lento en comprender las verdades que te he enseñado. Así pues, Simón, te advierto una vez más que los que combaten con la espada perecen por la espada, mientras que los que trabajan en el espíritu consiguen la vida eterna en el reino venidero, y la alegría y la paz en el reino presente. Cuando la obra que se te ha confiado haya terminado en la Tierra, tú, Simón, te sentarás conmigo en mi reino del más allá. Verás realmente el reino que has anhelado, pero no en esta vida. Continúa creyendo en mí y en lo que te he revelado, y recibirás el don de la vida eterna>>.

\par 
%\textsuperscript{(1957.2)}
\textsuperscript{181:2.12} Cuando Jesús hubo terminado de hablar a Simón Celotes, se acercó a Mateo Leví y dijo: <<Ya no tendrás la responsabilidad de abastecer la tesorería del grupo apostólico. Pronto, muy pronto, todos estaréis dispersos; ni siquiera te permitirán disfrutar de la asociación consoladora y confortante con uno solo de tus hermanos. A medida que continuéis predicando este evangelio del reino, tendréis que encontrar nuevos asociados. Os he enviado de dos en dos durante la época de vuestra preparación, pero ahora que os dejo, cuando os hayáis recuperado de la conmoción, saldréis solos hasta los confines de la Tierra, proclamando esta buena nueva: Que los mortales vivificados por la fe son hijos de Dios>>.

\par 
%\textsuperscript{(1957.3)}
\textsuperscript{181:2.13} Entonces Mateo dijo: <<Pero, Maestro, ¿quién nos va a enviar y cómo sabremos adónde ir? ¿Nos mostrará Andrés el camino?>> Y Jesús respondió: <<No, Leví, Andrés ya no os dirigirá para proclamar el evangelio. Continuará por supuesto siendo vuestro amigo y consejero hasta el día en que llegue el nuevo instructor, y entonces el Espíritu de la Verdad os conducirá por ahí a cada uno de vosotros en el trabajo de expansión del reino. Se han producido en ti muchos cambios desde aquel día en la aduana en que empezaste a seguirme por primera vez; pero deberán producirse muchos más antes de que puedas tener la visión de una fraternidad en la cual los gentiles se sentarán con los judíos en una asociación fraternal. Pero continúa con tu impulso de atraer a tus hermanos judíos hasta que estés plenamente satisfecho, y luego dirígete con energía hacia los gentiles. Leví, puedes estar seguro de una cosa: Te has ganado la confianza y el afecto de tus hermanos; todos te aman>>. (Y los diez indicaron su conformidad a las palabras del Maestro.)

\par 
%\textsuperscript{(1958.1)}
\textsuperscript{181:2.14} <<Leví, sé muchas cosas que tus hermanos ignoran sobre tus ansiedades, sacrificios y esfuerzos para mantener repleta la tesorería, y aunque el que llevaba la bolsa esté ausente, me alegra que el embajador publicano esté aquí en mi reunión de despedida con los mensajeros del reino. Ruego para que puedas discernir el significado de mi enseñanza con los ojos del espíritu. Cuando el nuevo instructor llegue a tu corazón, síguelo allá donde te conduzca y que tus hermanos vean ---e incluso el mundo entero--- lo que puede hacer el Padre por un detestado recaudador de impuestos que se ha atrevido a seguir al Hijo del Hombre y a creer en el evangelio del reino. Desde el principio, Leví, te he amado como he amado a estos otros galileos. Sabiendo pues muy bien que ni el Padre ni el Hijo hacen acepción de personas, procura no hacer este tipo de distinciones entre los que se hagan creyentes en el evangelio gracias a tu ministerio. Así pues, Mateo, dedica toda tu futura vida de servicio a mostrar a todos los hombres que Dios no hace acepción de personas; que a los ojos de Dios y en la hermandad del reino, todos los hombres son iguales, todos los creyentes son hijos de Dios>>.

\par 
%\textsuperscript{(1958.2)}
\textsuperscript{181:2.15} Jesús se dirigió entonces a Santiago Zebedeo, que permaneció de pie en silencio mientras el Maestro le decía: <<Santiago, cuando tú y tu hermano menor vinisteis a verme un día buscando preferencias en los honores del reino, os dije que esos honores sólo los podía otorgar el Padre, y os pregunté si erais capaces de beber mi copa, y los dos me contestasteis que sí. Aunque entonces no hubierais sido capaces de hacerlo, y aunque ahora tampoco lo seáis, pronto estaréis preparados para ese servicio gracias a la experiencia que estáis a punto de atravesar. En aquella ocasión enfadaste a tus hermanos con tu conducta. Si aún no te han perdonado del todo, lo harán cuando te vean beber mi copa. Que tu ministerio sea largo o breve, domina tu alma con paciencia. Cuando llegue el nuevo instructor, deja que te enseñe el equilibrio de la compasión y esa tolerancia comprensiva que nace de la confianza sublime en mí y de la sumisión perfecta a la voluntad del Padre. Dedica tu vida a demostrar que el afecto humano y la dignidad divina se pueden combinar en el discípulo que conoce a Dios y cree en el Hijo. Todos los que viven así revelarán el evangelio incluso por su manera de morir. Tú y tu hermano Juan seguiréis caminos diferentes, y es posible que uno de vosotros se siente conmigo en el reino eterno mucho antes que el otro. Te ayudaría mucho si pudieras aprender que la verdadera sabiduría abarca la prudencia así como la valentía. Debes aprender que tu agresividad ha de ir acompañada de sagacidad. Llegarán esos momentos supremos en los que mis discípulos no dudarán en dar su vida por este evangelio, pero en todas las circunstancias ordinarias sería mucho mejor aplacar la ira de los incrédulos para que puedas seguir viviendo y continuar predicando la buena nueva. En la medida en que dependa de ti, vive mucho tiempo en la Tierra para que tu larga vida pueda ser fecunda en almas ganadas para el reino celestial>>.

\par 
%\textsuperscript{(1958.3)}
\textsuperscript{181:2.16} Cuando el Maestro hubo terminado de hablar a Santiago Zebedeo, dio la vuelta hasta el extremo de la mesa donde estaba sentado Andrés, miró a su fiel asistente a los ojos, y dijo: <<Andrés, me has representado fielmente como jefe en funciones de los embajadores del reino celestial. Aunque a veces has dudado y en otras ocasiones has manifestado una timidez peligrosa, sin embargo siempre has sido sinceramente justo y eminentemente equitativo en tu trato con tus compañeros. Desde tu ordenación y la de tus hermanos como mensajeros del reino, habéis sido autónomos en todos los asuntos administrativos del grupo, salvo que te designé como jefe en funciones de estos escogidos. En ninguna otra cuestión temporal he actuado para dirigir o influir en tus decisiones. Y lo he hecho así a fin de asegurar la existencia de un jefe que dirija todas vuestras deliberaciones colectivas posteriores. En mi universo y en el universo de universos de mi Padre, nuestros hijos-hermanos son tratados como individuos en todas sus relaciones espirituales, pero en todas las relaciones colectivas, procuramos invariablemente que exista una persona determinada que dirija. Nuestro reino es un reino de orden, y cuando dos o más criaturas volitivas actúan en cooperación, siempre se prevé la autoridad de un jefe>>.

\par 
%\textsuperscript{(1959.1)}
\textsuperscript{181:2.17} <<Y ahora Andrés, puesto que eres el jefe de tus hermanos en virtud de la autoridad que te he conferido, puesto que has servido así como mi representante personal y como estoy a punto de dejaros para ir hacia mi Padre, te libero de toda responsabilidad relacionada con estos asuntos temporales y administrativos. De ahora en adelante ya no tendrás ninguna jurisdicción sobre tus hermanos, excepto la que te has ganado como jefe espiritual y que tus hermanos reconocen por tanto libremente. A partir de este momento ya no puedes ejercer ninguna autoridad sobre tus hermanos, a menos que ellos te restituyan esa potestad mediante un acto legislativo preciso, después de que me haya ido hacia el Padre. Pero el hecho de liberarte de tus responsabilidades como jefe administrativo de este grupo no disminuye de ninguna manera tu responsabilidad moral de hacer todo lo que esté en tu poder para mantener juntos a tus hermanos, con mano firme y afectuosa, durante el período difícil que se avecina, esos días que transcurrirán entre mi partida de la carne y el envío del nuevo instructor que vivirá en vuestro corazón y que os conducirá finalmente a toda la verdad. Mientras me preparo para dejarte, quiero liberarte de toda la responsabilidad administrativa que tuvo su comienzo y su autoridad en mi presencia entre vosotros como uno de vosotros. De ahora en adelante, sólo ejerceré una autoridad espiritual sobre ti y entre vosotros>>.

\par 
%\textsuperscript{(1959.2)}
\textsuperscript{181:2.18} <<Si tus hermanos desean conservarte como consejero, te ordeno que hagas todo lo posible, en todas las cuestiones temporales y espirituales, por promover la paz y la armonía entre los diversos grupos de creyentes sinceros en el evangelio. Dedica el resto de tu vida a fomentar los aspectos prácticos del amor fraternal entre tus hermanos. Sé amable con mis hermanos carnales cuando lleguen a creer plenamente en este evangelio; manifiesta una dedicación afectuosa e imparcial a los griegos en el oeste y a Abner en el este. Aunque estos apóstoles míos pronto se van a dispersar por todos los rincones de la Tierra para proclamar la buena nueva de la salvación mediante la filiación con Dios, debes mantenerlos unidos durante las horas difíciles que se avecinan, ese período de intensa prueba durante el cual deberéis aprender a creer en este evangelio sin mi presencia personal, mientras esperáis pacientemente la llegada del nuevo instructor, el Espíritu de la Verdad. Así pues, Andrés, aunque quizás no te corresponda realizar grandes obras a los ojos de los hombres, conténtate con ser el educador y el consejero de aquellos que las hacen. Continúa hasta el fin tu trabajo en la Tierra, y luego continuarás este ministerio en el reino eterno, porque ¿no te he dicho muchas veces que tengo otras ovejas que no son de este rebaño?>>

\par 
%\textsuperscript{(1959.3)}
\textsuperscript{181:2.19} Jesús se dirigió entonces hacia los gemelos Alfeo, se colocó entre ellos, y dijo: <<Queridos hijos míos, sois uno de los tres pares de hermanos que escogieron seguirme. Los seis habéis hecho bien en trabajar en paz con los de vuestra propia sangre, pero ninguno lo ha hecho mejor que vosotros. Se avecinan duros tiempos. Quizás no comprendáis todo lo que os sucederá a vosotros y a vuestros hermanos, pero no dudéis nunca de que un día fuisteis llamados para la obra del reino. Durante algún tiempo no habrá multitudes que dirigir, pero no os desaniméis; cuando el trabajo de vuestra vida haya terminado, os recibiré en el cielo, donde contaréis con gloria vuestra salvación a las huestes seráficas y a las multitudes de Hijos elevados de Dios. Dedicad vuestra vida a realzar las faenas vulgares. Mostrad a todos los hombres de la Tierra y a los ángeles del cielo cómo un hombre mortal puede volver con alegría y coraje a sus tareas de años atrás, después de haber sido llamado para trabajar durante una temporada en el servicio especial de Dios. Si vuestro trabajo en los asuntos exteriores del reino ha terminado por ahora, deberíais regresar a vuestros quehaceres anteriores con la iluminación nueva de la experiencia de la filiación con Dios, y con la elevada comprensión de que para aquel que conoce a Dios no existen trabajos vulgares ni faenas laicas. Para vosotros que habéis trabajado conmigo, todas las cosas se han vuelto sagradas, y toda labor terrestre se ha convertido también en un servicio para Dios Padre. Cuando escuchéis hablar de las actividades de vuestros antiguos asociados apostólicos, regocijaos con ellos y continuad vuestro trabajo diario como aquellos que esperan a Dios y sirven mientras esperan. Habéis sido mis apóstoles y lo seréis siempre, y me acordaré de vosotros en el reino venidero>>.

\par 
%\textsuperscript{(1960.1)}
\textsuperscript{181:2.20} Después, Jesús fue hacia Felipe, que se levantó para escuchar el siguiente mensaje de su Maestro: <<Felipe, me has hecho muchas preguntas tontas, y he hecho todo lo posible por contestar a cada una de ellas, y ahora quisiera contestar a la última que ha surgido en tu mente sumamente honrada, pero poco espiritual. Todo el tiempo que he tardado en rodear la mesa hasta llegar a ti te has estado diciendo a ti mismo: `¿Qué voy a hacer si el Maestro se va y nos deja solos en el mundo?' ¡Oh, hombre de poca fe! Y sin embargo tienes casi tanta como muchos de tus hermanos. Has sido un buen administrador, Felipe. Sólo nos has fallado algunas veces, y uno de esos fallos lo utilizamos para manifestar la gloria del Padre. Tu función como administrador está a punto de terminar. Pronto deberás dedicarte más plenamente al trabajo para el que fuiste llamado: la predicación de este evangelio del reino. Felipe, siempre has querido demostraciones, y muy pronto vas a ver grandes cosas. Habría sido mucho mejor que hubieras visto todo esto por la fe, pero como eras sincero incluso en tu visión material, vivirás para ver cómo se cumplen mis palabras. Después, cuando tengas la bendición de la visión espiritual, sal a hacer tu trabajo dedicando tu vida a la causa de guiar a la humanidad en la búsqueda de Dios, y a perseguir las realidades eternas con el ojo de la fe espiritual y no con los ojos de la mente material. Recuerda Felipe que tienes una gran misión en la Tierra, porque el mundo está lleno de gente que tiene la tendencia de ver la vida exactamente como tú. Tienes una gran tarea que hacer, y cuando haya sido terminada en la fe, vendrás hacia mí en mi reino, y tendré el gran placer de mostrarte lo que el ojo no ha visto, lo que el oído no ha escuchado y lo que la mente mortal no ha concebido. Mientras tanto, sé como un niño pequeño en el reino del espíritu y permíteme, como espíritu del nuevo instructor, conducirte hacia adelante en el reino espiritual. De esta manera podré hacer por ti muchas cosas que no he podido realizar mientras vivía contigo como un mortal del reino. Y recuerda siempre, Felipe, que el que me ha visto ha visto al Padre>>.

\par 
%\textsuperscript{(1960.2)}
\textsuperscript{181:2.21} Entonces el Maestro se acercó a Natanael. Cuando Natanael se levantó, Jesús le pidió que se sentara y sentándose a su lado, le dijo: <<Natanael, has aprendido a vivir por encima de los prejuicios y a practicar una tolerancia creciente desde que te convertiste en mi apóstol. Pero tienes que aprender muchas más cosas. Has sido una bendición para tus compañeros, porque tu constante sinceridad siempre les ha servido de aviso. Cuando me haya ido, es posible que tu franqueza te impida llevarte bien con tus hermanos, tanto antiguos como nuevos. Deberías aprender que incluso la expresión de un pensamiento bueno debe ser modulada de acuerdo con el estado intelectual y el desarrollo espiritual del oyente. La sinceridad es extremadamente útil en el trabajo del reino cuando está unida a la discreción>>.

\par 
%\textsuperscript{(1961.1)}
\textsuperscript{181:2.22} <<Si quisieras aprender a trabajar con tus hermanos, podrías realizar cosas más duraderas, pero si sales en busca de aquellos que piensan como tú, en ese caso dedica tu vida a probar que el discípulo que conoce a Dios puede convertirse en un constructor del reino, aunque esté solo en el mundo y completamente aislado de sus compañeros creyentes. Sé que serás fiel hasta el fin, y algún día te daré la bienvenida en el servicio más amplio de mi reino del cielo>>.

\par 
%\textsuperscript{(1961.2)}
\textsuperscript{181:2.23} Entonces habló Natanael, haciéndole a Jesús la pregunta siguiente: <<He escuchado tu enseñanza desde que me llamaste por primera vez al servicio de este reino, pero honradamente no puedo comprender el significado completo de todo lo que nos dices. No sé qué es lo próximo que va a suceder, y creo que la mayoría de mis hermanos están igualmente perplejos, aunque dudan en confesar su confusión. ¿Puedes ayudarme?>> Jesús puso su mano en el hombro de Natanael, y dijo: <<Amigo mío, no es raro que te sientas perplejo al intentar captar el significado de mis enseñanzas espirituales, puesto que estás muy trabado por tus ideas preconcebidas que tienen su origen en la tradición judía, y muy confundido por tu tendencia persistente a interpretar mi evangelio de acuerdo con las enseñanzas de los escribas y de los fariseos>>.

\par 
%\textsuperscript{(1961.3)}
\textsuperscript{181:2.24} <<Os he enseñado muchas cosas por medio de la palabra, y he vivido mi vida entre vosotros. He hecho todo lo que se puede hacer por iluminar vuestra mente y liberar vuestra alma, y lo que no habéis sido capaces de obtener de mis enseñanzas y de mi vida, ahora tenéis que prepararos para adquirirlo de la mano del maestro de todos los instructores: la experiencia real. En todas esas nuevas experiencias que ahora te esperan, iré delante de ti y el Espíritu de la Verdad estará contigo. No temas; cuando el nuevo instructor haya llegado, lo que ahora no logras comprender te lo revelará durante el resto de tu vida en la Tierra y a lo largo de toda tu formación durante las eras eternas>>.

\par 
%\textsuperscript{(1961.4)}
\textsuperscript{181:2.25} Luego el Maestro se volvió hacia todos ellos, y dijo: <<No os desaniméis si no lográis captar el pleno significado del evangelio. Sólo sois seres finitos, hombres mortales, y lo que os he enseñado es infinito, divino y eterno. Sed pacientes y tened buen ánimo, porque tenéis ante vosotros las eras eternas para continuar haciendo realidad progresivamente la experiencia de volveros perfectos, como vuestro Padre en el Paraíso es perfecto>>.

\par 
%\textsuperscript{(1961.5)}
\textsuperscript{181:2.26} Entonces Jesús se dirigió hacia Tomás, que se puso de pie para escucharle decir: <<Tomás, a menudo te ha faltado fe; sin embargo, cuando has tenido tus períodos de duda, nunca te ha faltado el coraje. Sé muy bien que los falsos profetas y los educadores impostores no te engañarán. Después de mi partida, tus hermanos apreciarán mucho más tu manera crítica de ver las nuevas enseñanzas. Cuando todos estéis dispersos hasta los confines de la Tierra en los tiempos venideros, recuerda que sigues siendo mi embajador. Dedica tu vida a la gran tarea de mostrar que la mente material crítica del hombre puede triunfar sobre la inercia de la duda intelectual cuando se enfrenta a la demostración de la manifestación de la verdad viviente, tal como ésta opera en la experiencia de los hombres y mujeres nacidos del espíritu, que producen en sus vidas los frutos del espíritu, y que se aman los unos a los otros como yo os he amado. Tomás, me alegro de que te unieras a nosotros, y sé que después de un corto período de perplejidad, continuarás al servicio del reino. Tus dudas han confundido a tus hermanos, pero nunca me han preocupado. Tengo confianza en ti, y te precederé hasta los rincones más alejados de la Tierra>>.

\par 
%\textsuperscript{(1962.1)}
\textsuperscript{181:2.27} Luego el Maestro fue hacia Simón Pedro, el cual se puso de pie mientras Jesús le dirigía la palabra: <<Pedro, sé que me amas, y que dedicarás tu vida a proclamar públicamente este evangelio del reino a los judíos y a los gentiles, pero me apena que tus años de asociación tan estrecha conmigo no hayan servido más para ayudarte a reflexionar antes de hablar. ¿Por qué experiencias tendrás que pasar para aprender a contener tu lengua? ¡Cuántas dificultades nos has causado con tus palabras irreflexivas, con tu presuntuosa confianza en ti mismo! Y estás destinado a crearte muchos más problemas si no dominas esta flaqueza. Sabes que tus hermanos te aman a pesar de esta debilidad, y también deberías comprender que este defecto no disminuye de ninguna manera mi afecto por ti, pero sí disminuye tu utilidad y no deja de crearte problemas. Pero la experiencia por la que vas a pasar esta misma noche será sin duda de gran ayuda para ti. Y lo que ahora voy a decirte, Simón Pedro, lo digo igualmente a todos tus hermanos aquí reunidos: Esta noche, todos vais a correr el gran peligro de tropezar por mi causa. Sabéis que está escrito: `Golpearán al pastor y las ovejas serán dispersadas.' Cuando ya no esté presente, existirá el gran peligro de que algunos de vosotros sucumban a las dudas y tropiecen a causa de lo que me suceda a mí. Pero os prometo ahora que regresaré por un corto período de tiempo, y que entonces os precederé en Galilea>>.

\par 
%\textsuperscript{(1962.2)}
\textsuperscript{181:2.28} Entonces Pedro, poniendo su mano en el hombro de Jesús, dijo: <<No importa que todos mis hermanos sucumban a las dudas por tu causa; prometo que no tropezaré por nada de lo que puedas hacer. Iré contigo y, si es preciso, moriré por ti>>.

\par 
%\textsuperscript{(1962.3)}
\textsuperscript{181:2.29} Mientras Pedro permanecía allí delante de su Maestro, temblando de intensa emoción y rebosante de amor sincero por él, Jesús miró directamente a sus ojos humedecidos mientras le decía: <<Pedro, en verdad, en verdad te digo que el gallo no cantará esta noche hasta que me hayas negado tres o cuatro veces. Y así, lo que no has logrado aprender mediante tu asociación pacífica conmigo, lo aprenderás a través de muchas dificultades y de grandes tristezas. Después de que hayas aprendido realmente esta lección indispensable, deberías fortalecer a tus hermanos y continuar viviendo una vida dedicada a la predicación de este evangelio, aunque puedas terminar en la cárcel y quizás sigas mis pasos, pagando el precio supremo del servicio amoroso en la construcción del reino del Padre>>.

\par 
%\textsuperscript{(1962.4)}
\textsuperscript{181:2.30} <<Pero recuerda mi promesa: Cuando haya resucitado, permaneceré algún tiempo con vosotros antes de ir hacia el Padre. Esta misma noche le suplicaré al Padre que fortalezca a cada uno de vosotros para la prueba que muy pronto tendréis que atravesar. Os amo a todos con el mismo amor que el Padre me ama, y por eso, de ahora en adelante, deberíais amaros los unos a los otros como yo os he amado>>.

\par 
%\textsuperscript{(1962.5)}
\textsuperscript{181:2.31} Luego, después de haber cantado un himno, partieron hacia el campamento del Monte de los Olivos.


\chapter{Documento 182. En Getsemaní}
\par 
%\textsuperscript{(1963.1)}
\textsuperscript{182:0.1} ERAN aproximadamente las diez de este jueves por la noche cuando Jesús llevó de regreso a los once apóstoles desde la casa de Elías y María Marcos hasta el campamento de Getsemaní. Desde el día que estuvo con el Maestro en las colinas, Juan Marcos se había ocupado de vigilar constantemente a Jesús. Como tenía necesidad de dormir, Juan había descansado varias horas mientras el Maestro estaba con sus apóstoles en la sala de arriba, pero al escuchar que bajaban las escaleras, se levantó y se puso rápidamente un manto de lino; luego los siguió a través de la ciudad, cruzó el arroyo Cedrón y continuó hasta su campamento privado que lindaba con el parque de Getsemaní. A lo largo de esta noche y del día siguiente, Juan Marcos permaneció tan cerca del Maestro que lo presenció todo y escuchó muchas cosas que dijo el Maestro desde este instante hasta el momento de la crucifixión.

\par 
%\textsuperscript{(1963.2)}
\textsuperscript{182:0.2} Mientras Jesús y los once regresaban al campamento, los apóstoles empezaron a preguntarse por el significado de la prolongada ausencia de Judas; hablaron entre sí acerca de la predicción del Maestro de que uno de ellos lo traicionaría, y sospecharon por primera vez que las cosas no iban bien con Judas Iscariote. Pero no se dedicaron abiertamente a hacer comentarios sobre Judas hasta que llegaron al campamento y observaron que no estaba allí esperándolos para recibirlos. Cuando todos acosaron a Andrés para saber qué le había pasado a Judas, su jefe se limitó a comentar: <<No sé dónde está Judas, pero me temo que nos ha abandonado>>.

\section*{1. La última oración en grupo}
\par 
%\textsuperscript{(1963.3)}
\textsuperscript{182:1.1} Poco después de llegar al campamento, Jesús les dijo: <<Amigos y hermanos míos, me queda muy poco tiempo que estar con vosotros, y deseo que nos aislemos mientras le rogamos a nuestro Padre que está en los cielos que nos dé fuerzas para sostenernos en esta hora y de aquí en adelante en todo el trabajo que tenemos que hacer en su nombre>>.

\par 
%\textsuperscript{(1963.4)}
\textsuperscript{182:1.2} Después de haber hablado así, Jesús los llevó un poco más arriba por el Olivete hasta una gran roca plana desde donde se veía todo Jerusalén, y les pidió que se arrodillaran en círculo a su alrededor como lo habían hecho el día de su ordenación; luego, mientras permanecía allí en medio de ellos, glorificado en la suave luz de la Luna, levantó los ojos al cielo y oró:

\par 
%\textsuperscript{(1963.5)}
\textsuperscript{182:1.3} <<Padre, mi hora ha llegado; glorifica ahora a tu Hijo para que el Hijo pueda glorificarte. Sé que me has dado plena autoridad sobre todas las criaturas vivientes de mi reino, y daré la vida eterna a todos los que se vuelvan hijos de Dios por la fe. Y la vida eterna consiste en que mis criaturas te conozcan como el único verdadero Dios y Padre de todos, y que crean en aquel que has enviado a este mundo. Padre, te he exaltado en la Tierra y he realizado la obra que me encargaste. Casi he terminado mi donación a los hijos de nuestra propia creación; sólo me queda abandonar mi vida en la carne. Ahora, oh Padre mío, glorifícame con la gloria que tenía contigo antes de que existiera este mundo y recíbeme una vez más a tu diestra>>.

\par 
%\textsuperscript{(1964.1)}
\textsuperscript{182:1.4} <<Te he manifestado a los hombres que escogiste en el mundo para dármelos. Son tuyos ---como toda vida está en tus manos--- tú me los diste y yo he vivido entre ellos enseñándoles el camino de la vida, y ellos han creído. Estos hombres están aprendiendo que todo lo que tengo procede de ti, y que la vida que vivo en la carne es para hacer que los mundos conozcan a mi Padre. La verdad que me has dado se la he revelado a ellos. Estos amigos y embajadores míos han querido recibir sinceramente tu palabra. Les he dicho que he salido de ti, que tú me has enviado a este mundo, y que estoy a punto de volver a ti. Padre, ruego de hecho por estos hombres escogidos. Y ruego por ellos, no como rogaría por el mundo, sino como por aquellos a quienes he elegido en el mundo para que me representen en el mundo después de que haya regresado a tu tarea, al igual que te he representado en este mundo durante mi estancia en la carne. Estos hombres son míos; tú me los has dado; pero todas las cosas que son mías son siempre tuyas, y has hecho que todo lo que era tuyo ahora sea mío. Has sido exaltado en mí, y ahora ruego para que yo pueda ser honrado en estos hombres. No puedo estar más tiempo en este mundo; estoy a punto de volver a la tarea que me has encargado. Tengo que dejar atrás a estos hombres para que nos representen y representen a nuestro reino entre los hombres. Padre, mantén fieles a estos hombres mientras me preparo para abandonar mi vida en la carne. Ayuda a estos amigos míos para que sean uno en espíritu, como nosotros también somos uno. Mientras podía estar con ellos, podía velar por ellos y guiarlos, pero ahora estoy a punto de irme. Permanece cerca de ellos, Padre, hasta que podamos enviar al nuevo instructor para que los consuele y los fortalezca>>.

\par 
%\textsuperscript{(1964.2)}
\textsuperscript{182:1.5} <<Me diste doce hombres, y los he conservado a todos salvo a uno, el hijo de la venganza, que no ha querido seguir asociado con nosotros. Estos hombres son débiles y frágiles, pero sé que podemos confiar en ellos; los he puesto a prueba; me aman al igual que te veneran a ti. Aunque deberán sufrir mucho por mí, deseo que también estén llenos de alegría ante la seguridad de la filiación en el reino celestial. He dado a estos hombres tu palabra y les he enseñado la verdad. El mundo puede odiarlos como me ha odiado a mí, pero no pido que los saques del mundo, sino que los protejas del mal que hay en el mundo. Santifícalos en la verdad; tu palabra es la verdad. Del mismo modo que me enviaste a este mundo, yo estoy a punto de enviar a estos hombres al mundo. Por el bien de ellos, he vivido entre los hombres y he consagrado mi vida a tu servicio, a fin de poder inspirarlos para que se purifiquen por medio de la verdad que les he enseñado y el amor que les he revelado. Sé muy bien, Padre mío, que no necesito pedirte que veles por estos hermanos después de que me haya ido; sé que los amas como yo, pero hago esto para que puedan darse cuenta mejor de que el Padre ama a los hombres mortales como el Hijo los ama>>.

\par 
%\textsuperscript{(1964.3)}
\textsuperscript{182:1.6} <<Y ahora, Padre mío, quisiera rogar no solamente por estos once hombres, sino también por todos los demás que ahora creen en el evangelio del reino, o que puedan creer más adelante gracias a la palabra del ministerio futuro de mis apóstoles. Quiero que todos sean uno solo, como tú y yo somos uno. Tú estás en mí y yo estoy en ti, y deseo que estos creyentes estén igualmente en nosotros; que nuestros dos espíritus residan en ellos. Si mis hijos son uno solo como nosotros somos uno, y si se aman los unos a los otros como yo los he amado, entonces todos los hombres creerán que he salido de ti y estarán dispuestos a recibir la revelación que he efectuado de la verdad y la gloria. He revelado a estos creyentes la gloria que tú me has dado. Así como tú has vivido conmigo en espíritu, yo he vivido con ellos en la carne. Así como tú has sido uno conmigo, yo he sido uno con ellos, y el nuevo instructor será siempre uno con ellos y en ellos. He hecho todo esto para que mis hermanos en la carne puedan saber que el Padre los ama como el Hijo los ama, y que tú los amas como me amas a mí. Padre, trabaja conmigo para salvar a estos creyentes a fin de que dentro de poco puedan estar conmigo en la gloria, y luego continúen hasta unirse contigo en el abrazo del Paraíso. A los que sirven conmigo en la humillación, quisiera tenerlos conmigo en la gloria para que puedan ver todo lo que has puesto entre mis manos como cosecha eterna de la siembra del tiempo en la similitud de la carne mortal. Anhelo mostrar a mis hermanos terrestres la gloria que tenía contigo antes de la fundación de este mundo. Este mundo sabe muy poco de ti, Padre justo, pero yo te conozco y te he hecho conocer a estos creyentes, y ellos harán conocer tu nombre a otras generaciones. Y ahora les prometo que estarás con ellos en el mundo al igual que has estado conmigo ---que así sea>>.

\par 
%\textsuperscript{(1965.1)}
\textsuperscript{182:1.7} Los once permanecieron arrodillados en círculo alrededor de Jesús durante varios minutos, antes de levantarse y regresar en silencio al campamento cercano.

\par 
%\textsuperscript{(1965.2)}
\textsuperscript{182:1.8} Jesús oró por la \textit{unidad} entre sus seguidores, pero no deseaba la uniformidad. El pecado crea un nivel muerto de inercia maligna, pero la rectitud alimenta el espíritu creativo de la experiencia individual en las realidades vivientes de la verdad eterna y en la comunión progresiva de los espíritus divinos del Padre y del Hijo. En la comunión espiritual de un hijo creyente con el Padre divino, nunca puede haber una finalidad doctrinal ni una superioridad sectaria de conciencia de grupo.

\par 
%\textsuperscript{(1965.3)}
\textsuperscript{182:1.9} En el transcurso de esta oración final con sus apóstoles, el Maestro aludió al hecho de que había manifestado al mundo el \textit{nombre} del Padre. Y esto es realmente lo que hizo al revelar a Dios mediante su vida perfeccionada en la carne. El Padre que está en los cielos había intentado revelarse a Moisés, pero no pudo ir más allá de hacer que se dijera: <<YO SOY>>. Y cuando se le instó a que revelara más cosas de sí mismo, sólo se reveló: <<YO SOY el que SOY>>. Pero cuando Jesús hubo terminado su vida terrenal, el nombre del Padre se había revelado de tal manera que el Maestro, que era el Padre encarnado, podía decir en verdad:

\par 
%\textsuperscript{(1965.4)}
\textsuperscript{182:1.10} Yo soy el pan de la vida.

\par 
%\textsuperscript{(1965.5)}
\textsuperscript{182:1.11} Yo soy el agua viva.

\par 
%\textsuperscript{(1965.6)}
\textsuperscript{182:1.12} Yo soy la luz del mundo.

\par 
%\textsuperscript{(1965.7)}
\textsuperscript{182:1.13} Yo soy el deseo de todos los tiempos.

\par 
%\textsuperscript{(1965.8)}
\textsuperscript{182:1.14} Yo soy la puerta abierta a la salvación eterna.

\par 
%\textsuperscript{(1965.9)}
\textsuperscript{182:1.15} Yo soy la realidad de la vida sin fin.

\par 
%\textsuperscript{(1965.10)}
\textsuperscript{182:1.16} Yo soy el buen pastor.

\par 
%\textsuperscript{(1965.11)}
\textsuperscript{182:1.17} Yo soy el sendero de la perfección infinita.

\par 
%\textsuperscript{(1965.12)}
\textsuperscript{182:1.18} Yo soy la resurrección y la vida.

\par 
%\textsuperscript{(1965.13)}
\textsuperscript{182:1.19} Yo soy el secreto de la supervivencia eterna.

\par 
%\textsuperscript{(1965.14)}
\textsuperscript{182:1.20} Yo soy el camino, la verdad y la vida.

\par 
%\textsuperscript{(1965.15)}
\textsuperscript{182:1.21} Yo soy el Padre infinito de mis hijos finitos.

\par 
%\textsuperscript{(1965.16)}
\textsuperscript{182:1.22} Yo soy la verdadera vid; vosotros sois los sarmientos.

\par 
%\textsuperscript{(1965.17)}
\textsuperscript{182:1.23} Yo soy la esperanza de todos los que conocen la verdad viviente.

\par 
%\textsuperscript{(1965.18)}
\textsuperscript{182:1.24} Yo soy el puente viviente que va de un mundo a otro.

\par 
%\textsuperscript{(1965.19)}
\textsuperscript{182:1.25} Yo soy el enlace viviente entre el tiempo y la eternidad.

\par 
%\textsuperscript{(1965.20)}
\textsuperscript{182:1.26} Jesús amplió así la revelación viviente del nombre de Dios para todas las generaciones. De la misma manera que el amor divino revela la naturaleza de Dios, la verdad eterna revela su nombre en unas proporciones siempre crecientes.

\section*{2. Las últimas horas antes de la traición}
\par 
%\textsuperscript{(1966.1)}
\textsuperscript{182:2.1} Los apóstoles se quedaron profundamente anonadados cuando regresaron a su campamento y comprobaron que Judas no estaba allí. Mientras los once emprendían una viva discusión sobre el asunto de su compañero apóstol traidor, David Zebedeo y Juan Marcos llevaron a Jesús a un lado y le revelaron que habían estado observando a Judas durante varios días, y que sabían que tenía la intención de traicionarlo poniéndolo en manos de sus enemigos. Jesús los escuchó pero se limitó a decir: <<Amigos míos, al Hijo del Hombre no puede sucederle nada a menos que lo quiera el Padre que está en los cielos. Que no se inquiete vuestro corazón; todas las cosas concurrirán para la gloria de Dios y la salvación de los hombres>>.

\par 
%\textsuperscript{(1966.2)}
\textsuperscript{182:2.2} La actitud jovial de Jesús iba decayendo. A medida que pasaba el tiempo se volvía cada vez más serio e incluso triste. Los apóstoles, que estaban muy agitados, eran reacios a regresar a sus tiendas aunque se lo pidiera el mismo Maestro. Al volver de su conversación con David y Juan, Jesús dirigió sus últimas palabras a los once, diciendo: <<Amigos míos, id a descansar. Preparaos para el trabajo de mañana. Recordad que todos deberíamos someternos a la voluntad del Padre que está en los cielos. Os dejo mi paz>>. Después de hablar así, les indicó que regresaran a sus tiendas, pero mientras se iban, llamó a Pedro, Santiago y Juan, diciendo: <<Deseo que permanezcáis un rato conmigo>>.

\par 
%\textsuperscript{(1966.3)}
\textsuperscript{182:2.3} Los apóstoles se durmieron únicamente porque estaban literalmente agotados. Habían estado escasos de sueño desde que llegaron a Jerusalén. Antes de ir a sus diferentes tiendas para dormir, Simón Celotes los condujo a todos a su tienda, donde estaban guardadas las espadas y otras armas, y entregó a cada uno su equipo de combate. Todos recibieron estas armas y se las ciñeron allí mismo, excepto Natanael. Al rehusar el arma, Natanael dijo: <<Hermanos míos, el Maestro nos ha dicho muchas veces que su reino no es de este mundo, y que sus discípulos no deberían luchar con la espada para establecerlo. Yo creo en esto, y no pienso que el Maestro necesite que utilicemos la espada para defenderlo. Todos hemos visto su enorme poder y sabemos que podría defenderse de sus enemigos si lo deseara. Si no quiere resistirse a sus enemigos, debe ser porque esa conducta representa su intento por realizar la voluntad de su Padre. Rezaré, pero no empuñaré la espada>>. Cuando Andrés escuchó el discurso de Natanael, devolvió su espada a Simón Celotes. Así pues, nueve de ellos estaban armados cuando se separaron para irse a dormir.

\par 
%\textsuperscript{(1966.4)}
\textsuperscript{182:2.4} El resentimiento que tenían porque Judas era un traidor eclipsó por el momento todo lo demás en la mente de los apóstoles. El comentario del Maestro alusivo a Judas, expresado en el transcurso de la última oración, había abierto sus ojos al hecho de que los había abandonado.

\par 
%\textsuperscript{(1966.5)}
\textsuperscript{182:2.5} Después de que los ocho apóstoles se hubieron retirado finalmente a sus tiendas, y mientras Pedro, Santiago y Juan estaban esperando recibir las órdenes del Maestro, Jesús le dijo a David Zebedeo: <<Envíame a tu mensajero más rápido y fiable>>. Cuando David trajo ante el Maestro a un tal Jacobo, en otro tiempo corredor al servicio de los mensajes nocturnos entre Jerusalén y Betsaida, Jesús se dirigió a él y le dijo: <<Ve a toda prisa hasta Abner en Filadelfia y dile: `El Maestro te envía sus saludos de paz y dice que ha llegado la hora en que será entregado en manos de sus enemigos, que le darán muerte, pero que resucitará de entre los muertos y pronto aparecerá ante ti antes de ir hacia el Padre, y que entonces te dará unas directrices hasta el momento en que el nuevo instructor venga a vivir en vuestro corazón.'>> Cuando Jacobo hubo repetido este mensaje a la satisfacción del Maestro, Jesús lo envió a su misión, diciendo: <<No temas por lo que alguien pueda hacerte, Jacobo, porque esta noche un mensajero invisible correrá a tu lado>>.

\par 
%\textsuperscript{(1967.1)}
\textsuperscript{182:2.6} Luego Jesús se volvió hacia el jefe de los visitantes griegos que estaban acampados con ellos y le dijo: <<Hermano mío, no te inquietes por lo que está a punto de suceder, puesto que te he avisado de antemano. El Hijo del Hombre será ejecutado a instigación de sus enemigos, los jefes de los sacerdotes y los dirigentes de los judíos, pero resucitaré para estar con vosotros un poco de tiempo antes de ir hacia el Padre. Cuando hayas visto que sucede todo esto, glorifica a Dios y fortalece a tus hermanos>>.

\par 
%\textsuperscript{(1967.2)}
\textsuperscript{182:2.7} En circunstancias normales, los apóstoles hubieran dado personalmente las buenas noches al Maestro, pero esta noche estaban tan preocupados por la conciencia repentina de la deserción de Judas y tan aturdidos por la naturaleza insólita de la oración de despedida del Maestro, que escucharon su saludo de adiós y se alejaron en silencio.

\par 
%\textsuperscript{(1967.3)}
\textsuperscript{182:2.8} Aquella noche, cuando Andrés se alejaba de su lado, Jesús le dijo lo siguiente: <<Andrés, haz lo que puedas para mantener juntos a tus hermanos hasta que yo regrese con vosotros después de haber bebido esta copa. Fortalece a tus hermanos, puesto que ya te lo he dicho todo. Que la paz sea contigo>>.

\par 
%\textsuperscript{(1967.4)}
\textsuperscript{182:2.9} Ninguno de los apóstoles esperaba que sucediera nada fuera de lo común aquella noche, puesto que ya era muy tarde. Trataron de dormirse para poder levantarse temprano por la mañana y estar preparados para lo peor. Pensaban que los jefes de los sacerdotes intentarían capturar a su Maestro por la mañana temprano, porque nunca se hacía ningún trabajo secular después del mediodía del día de la preparación de la Pascua. Sólo David Zebedeo y Juan Marcos comprendieron que los enemigos de Jesús vendrían con Judas aquella misma noche.

\par 
%\textsuperscript{(1967.5)}
\textsuperscript{182:2.10} David había acordado permanecer de guardia aquella noche en el sendero más elevado que conducía a la carretera de Betania a Jerusalén, mientras que Juan Marcos debía vigilar la carretera que subía del Cedrón a Getsemaní. Antes de que David se dirigiera a su tarea autoimpuesta de centinela en un puesto avanzado, se despidió de Jesús diciendo: <<Maestro, he tenido la gran alegría de servir contigo. Mis hermanos son tus apóstoles, pero yo he disfrutado haciendo las cosas menores tal como debían hacerse, y te echaré de menos con todo mi corazón cuando te hayas ido>>. Jesús le dijo entonces a David: <<David, hijo mío, los demás han hecho lo que se les ordenaba que hicieran, pero tú has hecho este servicio por tu propia voluntad, y he sido consciente de tu dedicación. Tú también servirás algún día conmigo en el reino eterno>>.

\par 
%\textsuperscript{(1967.6)}
\textsuperscript{182:2.11} Entonces, mientras se preparaba para ir a vigilar en el sendero de arriba, David le dijo a Jesús: <<Sabes, Maestro, he enviado a buscar a tu familia, y un mensajero me ha dado la noticia de que esta noche están en Jericó. Mañana por la mañana temprano estarán aquí, pues sería peligroso para ellos subir de noche por este maldito camino>>. Bajando la mirada hacia David, Jesús dijo solamente: <<Que así sea, David>>.

\par 
%\textsuperscript{(1967.7)}
\textsuperscript{182:2.12} Cuando David se marchó hacia la parte alta del Olivete, Juan Marcos empezó a vigilar cerca de la carretera que descendía a lo largo del arroyo hacia Jerusalén. Juan habría permanecido en su puesto si no hubiera sido por su gran deseo de estar cerca de Jesús y de saber qué estaba sucediendo. Poco después de que David lo dejara, y al observar que Jesús se retiraba con Pedro, Santiago y Juan hacia una hondonada cercana, Juan Marcos se sintió tan dominado por una mezcla de devoción y de curiosidad, que abandonó su puesto de centinela y los siguió, ocultándose entre los arbustos. Desde allí observó y escuchó todo lo que sucedió durante estos últimos momentos en el jardín, poco antes de que Judas y los guardias armados aparecieran para arrestar a Jesús.

\par 
%\textsuperscript{(1968.1)}
\textsuperscript{182:2.13} Mientras todo esto se desarrollaba en el campamento del Maestro, Judas Iscariote conversaba con el capitán de los guardias del templo, el cual había reunido a sus hombres antes de ponerse en camino, bajo la dirección del traidor, para arrestar a Jesús.

\section*{3. A solas en Getsemaní}
\par 
%\textsuperscript{(1968.2)}
\textsuperscript{182:3.1} Cuando todo estuvo silencioso y tranquilo en el campamento, Jesús se llevó a Pedro, Santiago y Juan, y subieron un corto trecho hasta una hondonada cercana donde había ido anteriormente con frecuencia para orar y comulgar. Los tres apóstoles no podían dejar de reconocer que el Maestro estaba dolorosamente abrumado. Nunca antes lo habían observado tan triste y agobiado. Cuando llegaron al lugar de sus devociones, pidió a los tres que se sentaran y velaran con él mientras se alejaba a casi un tiro de piedra para orar. Cuando se hubo postrado en el suelo, oró: <<Padre mío, he venido a este mundo para hacer tu voluntad, y la he hecho. Sé que ha llegado la hora de abandonar esta vida en la carne, y no rehuyo hacerlo, pero quisiera saber si es tu voluntad que yo beba esta copa. Envíame la seguridad de que te complaceré en mi muerte tal como lo he hecho en mi vida>>.

\par 
%\textsuperscript{(1968.3)}
\textsuperscript{182:3.2} El Maestro permaneció unos momentos en actitud de oración, y luego se acercó a los tres apóstoles; los encontró profundamente dormidos, pues tenían los párpados pesados y no podían permanecer despiertos. Cuando Jesús los despertó, dijo: <<¡Cómo! ¿No podéis velar conmigo ni siquiera una hora? ¿No podéis ver que mi alma está extremadamente afligida, afligida de muerte, y que anhelo vuestra compañía?>> Cuando los tres se despertaron de su sueño, el Maestro se alejó de nuevo a solas y, cayendo al suelo, oró otra vez: <<Padre, sé que es posible evitar esta copa ---todas las cosas son posibles para ti--- pero he venido para hacer tu voluntad, y aunque esta copa sea amarga, la beberé si es tu voluntad>>. Después de haber orado así, un ángel poderoso descendió a su lado, le habló, lo tocó y lo fortaleció.

\par 
%\textsuperscript{(1968.4)}
\textsuperscript{182:3.3} Cuando Jesús regresó para hablar con los tres apóstoles, los encontró de nuevo profundamente dormidos. Los despertó diciendo: <<En esta hora necesito que veléis y oréis conmigo ---necesitáis orar aún más para no caer en la tentación--- ¿por qué os dormís cuando os dejo?>>

\par 
%\textsuperscript{(1968.5)}
\textsuperscript{182:3.4} Entonces, el Maestro se retiró por tercera vez para orar: <<Padre, ves a mis apóstoles dormidos; ten misericordia de ellos. En verdad, el espíritu está dispuesto, pero la carne es débil. Y ahora, oh Padre, si esta copa no puede ser apartada, entonces la beberé. Que no se haga mi voluntad, sino la tuya>>. Cuando hubo terminado de orar, permaneció unos momentos postrado en el suelo. Cuando se levantó y regresó donde estaban sus apóstoles, los encontró dormidos una vez más. Los observó y, con un gesto de piedad, dijo tiernamente: <<Dormid ahora y descansad; el momento de la decisión ha pasado. Ha llegado la hora en que el Hijo del Hombre será traicionado y entregado a sus enemigos>>. Mientras se inclinaba y los sacudía para poder despertarlos, dijo: <<Levantaos, volvamos al campamento, porque he aquí que el que me traiciona está cerca, y ha llegado la hora en que mi rebaño va a ser dispersado. Pero ya os he hablado de estas cosas>>.

\par 
%\textsuperscript{(1968.6)}
\textsuperscript{182:3.5} Durante los años que Jesús vivió entre sus discípulos, éstos tuvieron en verdad muchas pruebas de su naturaleza divina, pero en este momento están a punto de presenciar nuevas evidencias de su humanidad. Justo antes de la más grande de todas las revelaciones de su divinidad, su resurrección, deben producirse las pruebas más grandes de su naturaleza mortal: su humillación y su crucifixión.

\par 
%\textsuperscript{(1969.1)}
\textsuperscript{182:3.6} Cada vez que había orado en el jardín, su humanidad se había aferrado más firmemente, por la fe, a su divinidad; su voluntad humana se había unificado más completamente con la voluntad divina de su Padre. Entre otras palabras que le había dicho el ángel poderoso, se encontraba el mensaje de que el Padre deseaba que su Hijo terminara su donación terrenal pasando por la experiencia de la muerte que atraviesan las criaturas, exactamente como todas las criaturas mortales deben experimentar la disolución material cuando pasan de la existencia en el tiempo a la progresión en la eternidad.

\par 
%\textsuperscript{(1969.2)}
\textsuperscript{182:3.7} Anteriormente aquella noche, no había parecido tan difícil beber la copa, pero cuando el Jesús humano se despidió de sus apóstoles y los envió a descansar, la prueba se volvió más espantosa. Jesús experimentaba esos sentimientos naturales de flujo y de reflujo que toda experiencia humana tiene en común, y en aquel momento estaba cansado de trabajar, agotado por las largas horas de esfuerzo tenaz y de penosa ansiedad a causa de la seguridad de sus apóstoles. Aunque ningún mortal puede atreverse a comprender los pensamientos y sentimientos del Hijo encarnado de Dios en un momento como éste, sabemos que soportó una gran angustia y sufrió una tristeza indecible, porque grandes gotas de sudor corrían por su rostro. Por fin estaba convencido de que el Padre tenía la intención de dejar que los acontecimientos naturales siguieran su curso; estaba plenamente decidido a no emplear, para salvarse, ninguno de sus poderes soberanos como jefe supremo de un universo.

\par 
%\textsuperscript{(1969.3)}
\textsuperscript{182:3.8} Las huestes reunidas de una inmensa creación se cernían ahora sobre esta escena, bajo el mando temporal conjunto de Gabriel y del Ajustador Personalizado de Jesús. Los jefes de división de estos ejércitos del cielo habían sido advertidos repetidas veces que no interfirieran en estas actividades terrenales, a menos que el mismo Jesús les ordenara que intervinieran.

\par 
%\textsuperscript{(1969.4)}
\textsuperscript{182:3.9} La experiencia de separarse de los apóstoles suponía una gran tensión para el corazón humano de Jesús; esta tristeza de amor pesaba sobre él y le hacía más difícil enfrentarse a una muerte como la que sabía muy bien que le esperaba. Se daba cuenta de cuán débiles e ignorantes eran sus apóstoles, y temía abandonarlos. Sabía muy bien que había llegado la hora de su partida, pero su corazón humano anhelaba descubrir si no existía la posibilidad de que hubiera alguna vía legítima para escapar de este trance terrible de sufrimiento y de pena. Cuando su corazón hubo buscado así una escapatoria, sin conseguirla, estuvo dispuesto a beber la copa. La mente divina de Miguel sabía que había hecho todo lo posible por los doce apóstoles; pero el corazón humano de Jesús deseaba haber hecho más por ellos antes de dejarlos solos en el mundo. El corazón de Jesús estaba destrozado; amaba sinceramente a sus hermanos. Estaba aislado de su familia carnal; uno de sus asociados escogidos lo estaba traicionando. El pueblo de su padre José lo había rechazado y había sellado así su destino como pueblo con una misión especial en la Tierra. Su alma estaba atormentada por el amor frustrado y la misericordia rechazada. Se trataba de uno de esos momentos terribles en la vida de un hombre en que todo parece aplastarlo con una crueldad demoledora y una agonía terrible.

\par 
%\textsuperscript{(1969.5)}
\textsuperscript{182:3.10} La naturaleza humana de Jesús no era insensible a esta situación de soledad personal, de oprobio público y de fracaso aparente de su causa. Todos estos sentimientos pesaban sobre él con una fuerza indescriptible. En medio de esta gran tristeza, su mente volvió a los tiempos de su infancia en Nazaret y de sus primeros trabajos en Galilea. En el momento de esta gran prueba, muchas escenas agradables de su ministerio terrenal surgieron en su mente. Gracias a estos antiguos recuerdos de Nazaret, Cafarnaúm, el Monte Hermón y las salidas y puestas de Sol en el resplandeciente mar de Galilea, logró calmarse mientras fortalecía y preparaba su corazón humano para salir al encuentro del traidor que tan pronto iba a traicionarlo.

\par 
%\textsuperscript{(1970.1)}
\textsuperscript{182:3.11} Antes de que Judas y los soldados llegaran, el Maestro había recuperado por completo su equilibrio habitual; el espíritu había triunfado sobre la carne; la fe se había afirmado sobre todas las tendencias humanas al temor y a albergar dudas. La prueba suprema del desarrollo completo de la naturaleza humana había sido afrontada y superada de manera aceptable. Una vez más, el Hijo del Hombre estaba preparado para enfrentarse a sus enemigos con serenidad y con la plena seguridad de que era invencible como hombre mortal dedicado sin reservas a hacer la voluntad de su Padre.


\chapter{Documento 183. La traición y el arresto de Jesús}
\par 
%\textsuperscript{(1971.1)}
\textsuperscript{183:0.1} CUANDO Jesús despertó finalmente a Pedro, Santiago y Juan, les sugirió que se fueran a sus tiendas y trataran de dormir a fin de prepararse para las tareas del día siguiente. Pero para entonces, los tres apóstoles estaban totalmente despiertos; habían descansado con las breves cabezadas que habían dado, y además se habían estimulado y excitado con la llegada al lugar de dos mensajeros agitados que preguntaron por David Zebedeo, y que partieron rápidamente en su búsqueda en cuanto Pedro les indicó dónde se encontraba de vigilancia.

\par 
%\textsuperscript{(1971.2)}
\textsuperscript{183:0.2} Aunque ocho de los apóstoles estaban profundamente dormidos, los griegos que estaban acampados junto a ellos tenían un mayor temor de que se produjeran disturbios, de tal manera que habían apostado un centinela para que diera la alarma en caso de que se presentara algún peligro. Cuando los dos mensajeros entraron precipitadamente en el campamento, el centinela griego procedió a despertar a todos sus compatriotas, los cuales salieron de sus tiendas completamente vestidos y armados. Todo el campamento estaba ahora despierto, salvo los ocho apóstoles; Pedro deseaba llamar a sus compañeros, pero Jesús se lo prohibió terminantemente. El Maestro recomendó dulcemente a todos que regresaran a sus tiendas, pero estaban poco dispuestos a someterse a su sugerencia.

\par 
%\textsuperscript{(1971.3)}
\textsuperscript{183:0.3} Como no logró dispersar a sus seguidores, el Maestro los dejó y descendió hacia el lagar, cerca de la entrada del parque de Getsemaní. Los tres apóstoles, los griegos y los otros miembros del campamento dudaron en seguirlo inmediatamente, pero Juan Marcos corrió entre los olivos y se ocultó en un pequeño cobertizo cerca del lagar. Jesús se había alejado del campamento y de sus amigos para que cuando llegaran sus captores pudieran arrestarlo sin molestar a sus apóstoles. El Maestro temía que sus apóstoles se despertaran y estuvieran presentes en el momento de su arresto, no sea que el espectáculo de la traición de Judas suscitara de tal manera su animosidad que ofrecieran resistencia a los soldados y fueran apresados con él. Temía que si eran arrestados con él, pudieran perecer también con él.

\par 
%\textsuperscript{(1971.4)}
\textsuperscript{183:0.4} Aunque Jesús sabía que el plan para matarlo se había originado en los consejos de los dirigentes de los judíos, también era consciente de que todos estos proyectos nefastos tenían la plena aprobación de Lucifer, Satanás y Caligastia. Sabía muy bien que estos rebeldes de los reinos también verían con placer que todos los apóstoles fueran exterminados con él.

\par 
%\textsuperscript{(1971.5)}
\textsuperscript{183:0.5} Jesús se sentó solo en el lagar, donde esperó la llegada del traidor, y en aquel momento solamente era visto por Juan Marcos y una multitud innumerable de observadores celestiales.

\section*{1. La voluntad del Padre}
\par 
%\textsuperscript{(1971.6)}
\textsuperscript{183:1.1} Existe el gran peligro de malinterpretar el significado de numerosos dichos y de muchos acontecimientos que acompañaron el final de la carrera del Maestro en la carne. El tratamiento cruel que los criados ignorantes y los soldados insensibles infligieron a Jesús, la manera injusta en que fue juzgado y la actitud sin piedad de los dirigentes religiosos declarados, no se deben confundir con el hecho de que al someterse pacientemente a todo este sufrimiento y humillación, Jesús estaba haciendo realmente la voluntad del Padre Paradisiaco. De hecho y en verdad, era voluntad del Padre que su Hijo bebiera por completo la copa de la experiencia humana desde el nacimiento hasta la muerte, pero el Padre que está en los cielos no instigó de ninguna manera la conducta bárbara de aquellos seres humanos, supuestamente civilizados, que torturaron tan brutalmente al Maestro y acumularon tan horriblemente unas indignidades tras otras sobre una persona que no ofrecía resistencia. Estas experiencias inhumanas e impactantes que Jesús tuvo que soportar durante las últimas horas de su vida mortal no formaban parte en ningún sentido de la voluntad divina del Padre, una voluntad que la naturaleza humana del Maestro se había comprometido tan triunfalmente a realizar en el momento de la rendición final del hombre a Dios, tal como lo indicaba la triple oración que formuló en el jardín mientras sus cansados apóstoles dormían el sueño del agotamiento físico.

\par 
%\textsuperscript{(1972.1)}
\textsuperscript{183:1.2} El Padre que está en los cielos deseaba que el Hijo donador terminara su carrera terrenal de manera \textit{natural}, exactamente como todos los mortales deben terminar su vida en la Tierra y en la carne. Los hombres y las mujeres corrientes no pueden esperar que una dispensación especial les facilite sus últimas horas en la Tierra y el episodio posterior de la muerte. En consecuencia, Jesús escogió abandonar su vida en la carne de una manera que fuera conforme con el proceso natural de los acontecimientos, y se negó firmemente a librarse de las garras crueles de una malvada conspiración de acontecimientos inhumanos que lo arrastraron, con horrible certeza, hacia su humillación increíble y su muerte ignominiosa. Cada detalle de toda esta asombrosa manifestación de odio y de esta demostración de crueldad sin precedentes fue obra de unos hombres malvados y de unos mortales perversos. Dios que está en el cielo no lo quiso así, ni tampoco fue dictado por los enemigos acérrimos de Jesús, aunque éstos hicieron muchas cosas para asegurarse de que los mortales malvados e irreflexivos rechazaran así al Hijo donador. Incluso el padre del pecado volvió su rostro ante el horror atroz de la escena de la crucifixión.

\section*{2. Judas en la ciudad}
\par 
%\textsuperscript{(1972.2)}
\textsuperscript{183:2.1} Después de abandonar tan precipitadamente la mesa durante la
Última Cena, Judas fue directamente a la casa de su primo, y luego los dos se dirigieron directamente a ver al capitán de los guardias del templo. Judas le pidió al capitán que reuniera a los guardias y le informó que estaba listo para conducirlos hasta Jesús. Como Judas había aparecido en escena un poco antes de lo esperado, hubo cierta demora hasta que partieron hacia la casa de Marcos, donde Judas esperaba que Jesús se encontraría todavía charlando con los apóstoles. El Maestro y los once salieron de la casa de Elías Marcos unos quince minutos antes de que llegaran el traidor y los guardias. Cuando los captores llegaron a la casa de Marcos, Jesús y los once estaban muy lejos de los muros de la ciudad, camino del campamento en el Olivete.

\par 
%\textsuperscript{(1972.3)}
\textsuperscript{183:2.2} A Judas le inquietó mucho el fracaso que supuso no encontrar a Jesús en el domicilio de Marcos y en compañía de once hombres, de los cuales sólo dos estaban armados para defenderse. Sabía por casualidad que, cuando salieron del campamento por la tarde, sólo Simón Pedro y Simón Celotes se habían ceñido sus espadas; Judas había esperado apresar a Jesús mientras la ciudad estaba tranquila y había pocas posibilidades de resistencia. El traidor temía tener que enfrentarse con más de sesenta discípulos fervientes si esperaba que regresaran a su campamento, y también sabía que Simón Celotes tenía en su poder una buena cantidad de armas. Judas se iba poniendo cada vez más nervioso a medida que pensaba en cómo lo detestarían los once apóstoles leales, y temía que todos intentaran aniquilarlo. No solamente era desleal, sino que en el fondo era un verdadero cobarde.

\par 
%\textsuperscript{(1973.1)}
\textsuperscript{183:2.3} Como no lograron encontrar a Jesús en la sala de arriba, Judas le pidió al capitán de los guardias que regresaran al templo. Mientras tanto, los dirigentes habían empezado a congregarse en la casa del sumo sacerdote, preparándose para recibir a Jesús, puesto que su pacto con el traidor exigía que Jesús fuera arrestado aquel día a medianoche. Judas explicó a sus asociados que no habían encontrado a Jesús en la casa de Marcos, y que sería necesario ir a Getsemaní para detenerlo. Luego el traidor continuó diciendo que más de sesenta seguidores fervientes estaban acampados con él, y que todos ellos estaban bien armados. Los dirigentes de los judíos recordaron a Judas que Jesús siempre había predicado la no resistencia, pero Judas replicó que no podían contar con que todos los seguidores de Jesús obedecieran esta enseñanza. Judas temía realmente por su vida, y por ello se atrevió a pedir una compañía de cuarenta soldados armados. Puesto que las autoridades judías no disponían de una fuerza semejante de hombres armados bajo su jurisdicción, se dirigieron inmediatamente a la fortaleza de Antonia y le pidieron al comandante romano que les diera esta guardia; pero cuando éste se enteró de que tenían la intención de arrestar a Jesús, rehusó rápidamente acceder a su petición y los envió a su oficial superior. De esta manera perdieron más de una hora yendo de una autoridad a otra, hasta que finalmente se vieron obligados a presentarse ante el mismo Pilatos para obtener el permiso de emplear los guardias armados romanos. Ya era tarde cuando llegaron a la casa de Pilatos, y éste se había retirado con su mujer a sus aposentos privados. Dudó en inmiscuirse de alguna manera en esta empresa, y aún más porque su mujer le había pedido que no concediera esta petición. Pero puesto que el presidente oficial del sanedrín judío estaba presente y solicitaba personalmente esta ayuda, el gobernador consideró que era sabio conceder la petición, pensando que más adelante podría enmendar cualquier injusticia que estuvieran dispuestos a cometer.

\par 
%\textsuperscript{(1973.2)}
\textsuperscript{183:2.4} En consecuencia, cuando Judas Iscariote salió del templo hacia las once y media de la noche, iba acompañado de más de sesenta personas ---los guardias del templo, los soldados romanos y los criados curiosos de los sacerdotes y dirigentes principales.

\section*{3. El arresto del Maestro}
\par 
%\textsuperscript{(1973.3)}
\textsuperscript{183:3.1} Mientras esta compañía de soldados y guardias armados, provistos de antorchas y linternas, se acercaba al jardín, Judas se adelantó considerablemente al grupo a fin de estar preparado para identificar rápidamente a Jesús, de manera que los captores pudieran prenderlo fácilmente antes de que sus compañeros acudieran a defenderlo. Había también otra razón por la que Judas escogió adelantarse a los enemigos del Maestro: Pensó que así parecería que había llegado a la escena antes que los soldados, de tal manera que los apóstoles y las otras personas reunidas alrededor de Jesús quizás no lo relacionarían directamente con los guardias armados que le seguían tan de cerca. Judas había pensado incluso en alardear de que se había apresurado para prevenirlos de la llegada de los captores, pero este plan fue desbaratado por el saludo sombrío con que Jesús recibió al traidor. Aunque el Maestro le habló a Judas con amabilidad, lo recibió como a un traidor.

\par 
%\textsuperscript{(1973.4)}
\textsuperscript{183:3.2} Tan pronto como Pedro, Santiago, Juan y unos treinta de sus compañeros de campamento vieron al grupo armado y sus antorchas girar en la cima de la colina, supieron que aquellos soldados venían a arrestar a Jesús, y todos descendieron precipitadamente hacia el lagar, donde el Maestro estaba sentado en una soledad iluminada por la Luna. Mientras la compañía de soldados se acercaba por un lado, los tres apóstoles y sus compañeros se acercaban por el otro. Cuando Judas avanzó a zancadas para acercarse al Maestro, los dos grupos se quedaron inmóviles con el Maestro entre ellos, mientras Judas se preparaba para estampar el beso traidor en la frente de Jesús.

\par 
%\textsuperscript{(1974.1)}
\textsuperscript{183:3.3} El traidor había esperado que, después de conducir a los guardias hasta Getsemaní, podría simplemente indicar a los soldados quién era Jesús, o a lo más llevar a cabo la promesa de saludarlo con un beso, y luego alejarse rápidamente de la escena. Judas tenía mucho miedo de que todos los apóstoles estuvieran presentes y que concentraran su ataque sobre él como castigo por haberse atrevido a traicionar a su amado instructor. Pero cuando el Maestro lo saludó como a un traidor, se sintió tan confundido que no hizo ningún intento por huir.

\par 
%\textsuperscript{(1974.2)}
\textsuperscript{183:3.4} Jesús hizo un último esfuerzo por evitarle a Judas que llevara a cabo el gesto efectivo de traicionarlo. Antes de que el traidor pudiera llegar hasta él, se apartó a un lado y se dirigió al soldado principal de la izquierda, el capitán de los romanos, diciendo: <<¿A quién buscáis?>> El capitán contestó: <<A Jesús de Nazaret>>. Entonces Jesús se presentó inmediatamente delante del oficial y, con la tranquila majestad del Dios de toda esta creación, dijo: <<Soy yo>>. Muchos miembros de este grupo armado habían escuchado a Jesús enseñar en el templo, otros se habían enterado de sus obras poderosas, y cuando le escucharon anunciar tan audazmente su identidad, los que se encontraban en las primeras filas retrocedieron repentinamente. Se quedaron aturdidos de sorpresa ante la tranquila y majestuosa declaración de su identidad. Judas no tenía pues ninguna necesidad de continuar con su plan de traición. El Maestro se había revelado audazmente a sus enemigos, y éstos podían haberlo arrestado sin la ayuda de Judas. Pero el traidor tenía que hacer algo para justificar su presencia con este grupo armado, y además quería hacer alarde de que estaba realizando su papel en el pacto de traición acordado con los jefes de los judíos, para hacerse digno de la gran recompensa y de los honores que creía que se acumularían sobre él como compensación por su promesa de entregarles a Jesús.

\par 
%\textsuperscript{(1974.3)}
\textsuperscript{183:3.5} Mientras los guardias se recuperaban de su primera vacilación al ver a Jesús y escuchar el sonido de su voz excepcional, y mientras los apóstoles y los discípulos se acercaban cada vez más, Judas avanzó hacia Jesús, le dio un beso en la frente, y dijo: <<Salve, Maestro e Instructor>>. Mientras Judas abrazaba así a su Maestro, Jesús dijo: <<Amigo, ¡no te basta con hacer esto! ¿Traicionarás también al Hijo del Hombre con un beso?>>

\par 
%\textsuperscript{(1974.4)}
\textsuperscript{183:3.6} Los apóstoles y los discípulos se quedaron literalmente anonadados por lo que estaban viendo. Durante un momento nadie se movió. Luego Jesús se desembarazó del abrazo traidor de Judas, se acercó a los guardias y soldados y preguntó de nuevo: <<¿A quién buscáis?>> El capitán dijo otra vez: <<A Jesús de Nazaret>>. Y Jesús contestó de nuevo: <<Os he dicho que soy yo. Así pues, si me buscáis a mí, dejad que estos otros se vayan. Estoy listo para ir con vosotros>>.

\par 
%\textsuperscript{(1974.5)}
\textsuperscript{183:3.7} Jesús estaba preparado para regresar a Jerusalén con los guardias, y el capitán de los soldados estaba enteramente dispuesto a permitir que los tres apóstoles y sus compañeros se fueran en paz. Pero antes de que pudieran partir, mientras Jesús estaba allí esperando las órdenes del capitán, un tal Malco, el guardaespaldas sirio del sumo sacerdote, se acercó a Jesús y se preparó para atarle las manos a la espalda, aunque el capitán romano no había ordenado que Jesús fuera atado así. Cuando Pedro y sus compañeros vieron que su Maestro era sometido a esta indignidad, ya no fueron capaces de contenerse más tiempo. Pedro sacó su espada y se abalanzó con los demás para golpear a Malco. Pero antes de que los soldados pudieran acudir en defensa del servidor del sumo sacerdote, Jesús levantó la mano delante de Pedro con gesto de prohibición y le habló severamente, diciendo: <<Pedro, guarda tu espada. Los que sacan la espada, perecerán por la espada. ¿No comprendes que es voluntad del Padre que yo beba esta copa? ¿Y no sabes además que incluso ahora podría ordenar a más de doce legiones de ángeles y a sus asociados que me liberaran de las manos de estos pocos hombres?>>

\par 
%\textsuperscript{(1975.1)}
\textsuperscript{183:3.8} Aunque Jesús puso así fin eficazmente a esta demostración de resistencia física por parte de sus seguidores, lo sucedido bastó para despertar los temores del capitán de los guardias, el cual, con la ayuda de sus soldados, puso sus pesadas manos sobre Jesús y lo ató rápidamente. Mientras le ataban las manos con fuertes cuerdas, Jesús les dijo: <<¿Por qué salís contra mí con espadas y palos como para capturar a un ladrón? He estado diariamente con vosotros en el templo, enseñando públicamente a la gente, y no habéis hecho ningún esfuerzo por apresarme>>.

\par 
%\textsuperscript{(1975.2)}
\textsuperscript{183:3.9} Cuando Jesús estuvo atado, el capitán, temiendo que los seguidores del Maestro intentaran rescatarlo, dio órdenes para que fueran capturados; pero los soldados no fueron lo suficientemente rápidos porque, como los seguidores de Jesús habían escuchado las órdenes del capitán de que fueran arrestados, huyeron precipitadamente por la hondonada. Durante todo este tiempo, Juan Marcos había permanecido recluido en el cobertizo cercano. Cuando los guardias emprendieron el regreso hacia Jerusalén con Jesús, Juan Marcos intentó salir a escondidas del cobertizo para unirse a los apóstoles y discípulos que habían huido; pero en el preciso momento en que salía, uno de los últimos soldados que regresaba de perseguir a los discípulos que huían pasó por allí y, al ver a este joven con su manto de lino, empezó a perseguirlo y casi llegó a atraparlo. De hecho, el soldado se acercó lo suficiente a Juan como para agarrar su manto, pero el joven se liberó de la ropa y se escapó desnudo mientras el soldado se quedaba con el manto vacío. Juan Marcos se dirigió a toda prisa hacia el sendero de arriba donde se encontraba David Zebedeo. Cuando le contó a David lo que había sucedido, los dos regresaron precipitadamente a las tiendas de los apóstoles dormidos e informaron a los ocho que el Maestro había sido traicionado y detenido.

\par 
%\textsuperscript{(1975.3)}
\textsuperscript{183:3.10} Casi en el mismo momento en que los ocho apóstoles eran despertados, los que habían huido por la hondonada arriba empezaron a regresar, y todos se reunieron cerca del lagar para discutir lo que había que hacer. Mientras tanto, Simón Pedro y Juan Zebedeo, que se habían ocultado entre los olivos, ya habían empezado a seguir al grupo de soldados, guardias y sirvientes, que ahora conducían a Jesús de regreso a Jerusalén como si llevaran a un criminal capaz de cualquier cosa. Juan seguía de cerca al grupo, pero Pedro iba detrás a más distancia. Después de escapar de las garras del soldado, Juan Marcos se procuró un manto que había encontrado en la tienda de Simón Pedro y Juan Zebedeo. Sospechaba que los guardias llevarían a Jesús a la casa de Anás, el sumo sacerdote jubilado; así pues, bordeó los huertos de olivos y llegó antes que el grupo al palacio del sumo sacerdote, donde se escondió cerca de la entrada principal.

\section*{4. La discusión en el lagar}
\par 
%\textsuperscript{(1975.4)}
\textsuperscript{183:4.1} Santiago Zebedeo se encontró separado de Simón Pedro y de su hermano Juan, de manera que se unió a los otros apóstoles y sus compañeros de campamento en el lagar para deliberar sobre lo que debían hacer en vista del arresto del Maestro.

\par 
%\textsuperscript{(1975.5)}
\textsuperscript{183:4.2} Andrés había sido liberado de toda responsabilidad como director del grupo de sus compañeros apóstoles; en consecuencia, en esta crisis, que era la más grave de sus vidas, permanecía en silencio. Después de una breve discusión informal, Simón Celotes se subió en el muro de piedra del lagar y, después de hacer una apasionada defensa a favor de la lealtad al Maestro y a la causa del reino, exhortó a sus compañeros apóstoles y a los otros discípulos a que corrieran detrás de la tropa y rescataran a Jesús. La mayoría del grupo habría estado dispuesta a seguir su conducta agresiva si no hubiera sido por la advertencia de Natanael, el cual se levantó en cuanto Simón terminó de hablar y llamó la atención de todos sobre las enseñanzas tantas veces repetidas de Jesús en relación con la no resistencia. Les recordó además que Jesús les había ordenado aquella misma noche que protegieran sus vidas hasta el momento en que salieran al mundo para proclamar la buena nueva del evangelio del reino celestial. Santiago Zebedeo apoyó esta actitud de Natanael, contando ahora cómo Pedro y otros habían sacado la espada para impedir el arresto del Maestro, y cómo Jesús había pedido a Simón Pedro y a sus compañeros armados que envainaran sus hojas. Mateo y Felipe también dieron sus discursos, pero nada concreto surgió de esta discusión hasta que Tomás llamó la atención de todos sobre el hecho de que Jesús había aconsejado a Lázaro que no se expusiera a la muerte; les indicó que no podían hacer nada por salvar a su Maestro puesto que éste se había negado a permitir que sus amigos lo defendieran, y persistía en abstenerse de utilizar sus poderes divinos para burlar a sus enemigos humanos. Tomás los persuadió para que se dispersaran cada uno por su lado, con el acuerdo de que David Zebedeo permanecería en el campamento para mantener un centro de intercambio de información y un cuartel general de mensajeros para el grupo. A las dos y media de aquella mañana, el campamento se quedaba desierto; sólo David permanecía allí con tres o cuatro mensajeros, después de haber enviado a los demás para que obtuvieran información sobre dónde habían llevado a Jesús y qué iban a hacer con él.

\par 
%\textsuperscript{(1976.1)}
\textsuperscript{183:4.3} Cinco apóstoles ---Natanael, Mateo, Felipe y los gemelos--- fueron a esconderse en Betfagé y Betania. Tomás, Andrés, Santiago y Simón Celotes se escondieron en la ciudad. Simón Pedro y Juan Zebedeo siguieron adelante hasta la casa de Anás.

\par 
%\textsuperscript{(1976.2)}
\textsuperscript{183:4.4} Poco después del amanecer, Simón Pedro, con una imagen abatida de profunda desesperación, regresó vagando al campamento de Getsemaní. David lo envió a cargo de un mensajero para que se reuniera con su hermano Andrés, que estaba en la casa de Nicodemo en Jerusalén.

\par 
%\textsuperscript{(1976.3)}
\textsuperscript{183:4.5} Hasta el final mismo de la crucifixión, Juan Zebedeo permaneció siempre cerca, tal como Jesús se lo había ordenado, y era él quien de hora en hora suministraba a los mensajeros la información que llevaban a David en el campamento del jardín, y que luego se transmitía a los apóstoles escondidos y a la familia de Jesús.

\par 
%\textsuperscript{(1976.4)}
\textsuperscript{183:4.6} Ciertamente, ¡el pastor es golpeado y las ovejas se dispersan! Aunque todos se dan vagamente cuenta de que Jesús les había avisado de esta precisa situación, están muy severamente conmocionados por la repentina desaparición del Maestro como para poder utilizar su mente de manera normal.

\par 
%\textsuperscript{(1976.5)}
\textsuperscript{183:4.7} Poco después del amanecer, y justo después de que Pedro hubiera sido enviado a reunirse con su hermano, Judá, el hermano carnal de Jesús, llegó al campamento casi sin aliento y por delante del resto de la familia de Jesús, para enterarse simplemente de que el Maestro ya había sido arrestado, y descendió apresuradamente la carretera de Jericó para llevar esta información a su madre y a sus hermanos y hermanas. David Zebedeo avisó a la familia de Jesús, por medio de Judá, de que se reunieran en la casa de Marta y María en Betania, y esperaran allí las noticias que sus mensajeros les llevarían con regularidad.

\par 
%\textsuperscript{(1976.6)}
\textsuperscript{183:4.8} Ésta era la situación durante la última mitad de la noche del jueves y las primeras horas de la mañana del viernes en lo que concierne a los apóstoles, los discípulos principales y la familia terrenal de Jesús. Todos estos grupos y personas se mantenían en contacto los unos con los otros gracias al servicio de mensajeros que David Zebedeo continuaba dirigiendo desde su cuartel general en el campamento de Getsemaní.

\section*{5. Camino del palacio del sumo sacerdote}
\par 
%\textsuperscript{(1977.1)}
\textsuperscript{183:5.1} Antes de partir del jardín con Jesús, se originó una discusión entre el capitán judío de los guardias del templo y el capitán romano de la compañía de soldados en cuanto al lugar donde debían llevar a Jesús. El capitán de los guardias del templo dio órdenes para que se le llevara ante Caifás, el sumo sacerdote en ejercicio. El capitán de los soldados romanos ordenó que Jesús fuera llevado al palacio de Anás, el antiguo sumo sacerdote y suegro de Caifás. Y lo hizo así porque los romanos tenían la costumbre de tratar directamente con Anás todas las cuestiones relacionadas con la aplicación de las leyes eclesiásticas judías. Y se obedecieron las órdenes del capitán romano; llevaron a Jesús a la casa de Anás para someterlo a un interrogatorio preliminar.

\par 
%\textsuperscript{(1977.2)}
\textsuperscript{183:5.2} Judas caminaba al lado de los capitanes, escuchando todo lo que se decía, pero sin participar en la discusión, porque ni el capitán judío ni el oficial romano querían siquiera hablar con el traidor ---de tal manera lo despreciaban.

\par 
%\textsuperscript{(1977.3)}
\textsuperscript{183:5.3} Casi en aquel momento, Juan Zebedeo recordó las instrucciones de su Maestro de que permaneciera siempre cerca, y se aproximó apresuradamente a Jesús que caminaba entre los dos capitanes. Al ver que Juan se ponía a su lado, el comandante de los guardias del templo dijo a su asistente: <<Coge a este hombre y átalo. Es uno de los seguidores de este tipo>>. Pero cuando el capitán romano escuchó esto, volvió la cabeza, vio a Juan, y dio órdenes para que el apóstol se pusiera a su lado y que nadie lo molestara. Luego el capitán romano le dijo al capitán judío: <<Este hombre no es ni un traidor ni un cobarde. Lo he visto en el jardín y no sacó la espada para oponer resistencia. Tiene el coraje de adelantarse para estar con su Maestro, y nadie le pondrá la mano encima. La ley romana permite que todo preso pueda tener al menos a un amigo que permanezca con él delante del tribunal, y no se impedirá que este hombre esté al lado de su Maestro, el detenido>>. Cuando Judas escuchó esto, se sintió tan avergonzado y humillado que se fue quedando detrás de la comitiva y llegó solo al palacio de Anás.

\par 
%\textsuperscript{(1977.4)}
\textsuperscript{183:5.4} Esto explica por qué se le permitió a Juan Zebedeo permanecer cerca de Jesús a lo largo de las duras experiencias de aquella noche y del día siguiente. Los judíos temían decirle algo a Juan o molestarlo de alguna manera, porque en cierto modo tenía la condición de un consejero romano designado para actuar como observador de las operaciones del tribunal eclesiástico judío. La posición privilegiada de Juan quedó aún más asegurada cuando, en el momento de entregar a Jesús al capitán de los guardias del templo en la puerta del palacio de Anás, el capitán romano se dirigió a su asistente y le dijo: <<Acompaña a este preso y asegúrate de que estos judíos no lo maten sin el consentimiento de Pilatos. Cuida de que no lo asesinen, y asegúrate de que a su amigo, el galileo, le permitan permanecer a su lado para observar todo lo que suceda>>. Así es como Juan pudo estar cerca de Jesús hasta el momento de su muerte en la cruz, aunque los otros diez apóstoles estuvieron obligados a permanecer ocultos. Juan actuaba bajo la protección romana, y los judíos no se atrevieron a molestarlo hasta después de la muerte del Maestro.

\par 
%\textsuperscript{(1977.5)}
\textsuperscript{183:5.5} Durante todo el trayecto hasta el palacio de Anás, Jesús no abrió la boca. Desde el momento de su arresto hasta su aparición delante de Anás, el Hijo del Hombre no dijo ni una palabra.


\chapter{Documento 184. Ante el tribunal del sanedrín}
\par 
%\textsuperscript{(1978.1)}
\textsuperscript{184:0.1} UNOS representantes de Anás habían ordenado en secreto al capitán de los soldados romanos que llevara a Jesús al palacio de Anás inmediatamente después de arrestarlo. El antiguo sumo sacerdote deseaba mantener su prestigio como principal autoridad eclesiástica de los judíos. También tenía otro objetivo al retener a Jesús en su casa durante varias horas, y era ganar tiempo para reunir legalmente al tribunal del sanedrín. No era legal convocar el tribunal del sanedrín antes de la hora de la ofrenda del sacrificio matutino en el templo, y este sacrificio se ofrecía hacia las tres de la mañana.

\par 
%\textsuperscript{(1978.2)}
\textsuperscript{184:0.2} Anás sabía que un tribunal de sanedristas estaba esperando en el palacio de su yerno Caifás. Unos treinta miembros del sanedrín se habían reunido a medianoche en la casa del sumo sacerdote a fin de estar preparados para juzgar a Jesús cuando fuera traído ante ellos. Únicamente se habían reunido aquellos miembros que se oponían enérgica y abiertamente a Jesús y a sus enseñanzas, puesto que sólo se necesitaban veintitrés para constituir un tribunal procesal.

\par 
%\textsuperscript{(1978.3)}
\textsuperscript{184:0.3} Jesús pasó unas tres horas en el palacio de Anás en el monte Olivete, no lejos del jardín de Getsemaní, donde lo habían arrestado. Juan Zebedeo estaba libre y a salvo en el palacio de Anás, no solamente debido a la palabra del capitán romano, sino también porque él y su hermano Santiago eran bien conocidos por los criados más viejos, pues habían sido invitados muchas veces al palacio, ya que el antiguo sumo sacerdote era un pariente lejano de su madre Salomé.

\section*{1. El interrogatorio de Anás}
\par 
%\textsuperscript{(1978.4)}
\textsuperscript{184:1.1} Enriquecido por los ingresos del templo, con su yerno como sumo sacerdote en ejercicio, y debido a sus relaciones con las autoridades romanas, Anás era en verdad el individuo más poderoso de la sociedad judía. Era un planificador y un conspirador sofisticado y diplomático. Deseaba dirigir este asunto para deshacerse de Jesús; temía confiar enteramente una empresa tan importante como ésta a su brusco y agresivo yerno. Anás quería asegurarse de que el juicio del Maestro permanecería entre las manos de los saduceos; temía la posible simpatía de algunos fariseos, ya que prácticamente todos los miembros del sanedrín que habían abrazado la causa de Jesús eran fariseos.

\par 
%\textsuperscript{(1978.5)}
\textsuperscript{184:1.2} Anás no había visto a Jesús desde hacía varios años, desde la época en que el Maestro se presentó en su casa y se marchó inmediatamente al observar la frialdad y la reserva con que lo recibió. Anás había pensado aprovecharse de esta antigua relación e intentar de este modo persuadir a Jesús para que abandonara sus pretensiones y se fuera de Palestina. Le repugnaba participar en el asesinato de un hombre bueno y había razonado que quizás Jesús escogería dejar el país en lugar de sufrir la muerte. Pero cuando Anás se encontró delante del fornido y resuelto galileo, supo enseguida que hacer tales proposiciones sería inútil. Jesús era aún más majestuoso y bien equilibrado de lo que Anás recordaba.

\par 
%\textsuperscript{(1979.1)}
\textsuperscript{184:1.3} Cuando Jesús era joven, Anás se había interesado mucho por él, pero ahora sus ingresos estaban amenazados por lo que Jesús había hecho tan recientemente echando del templo a los cambistas y a otros mercaderes. Este acto había suscitado la enemistad del antiguo sumo sacerdote mucho más que las enseñanzas de Jesús.

\par 
%\textsuperscript{(1979.2)}
\textsuperscript{184:1.4} Anás entró en su espaciosa sala de audiencias, se sentó en un gran sillón y ordenó que trajeran a Jesús ante él. Después de observar al Maestro en silencio durante unos momentos, dijo: <<Comprenderás que habrá que hacer algo con respecto a tu enseñanza, puesto que estás perturbando la paz y el orden de nuestro país>>. Mientras Anás miraba de manera indagadora a Jesús, el Maestro lo miró directamente a los ojos, pero no respondió. Anás dijo de nuevo: <<¿Cuáles son los nombres de tus discípulos, además de Simón Celotes, el agitador?>> Jesús lo miró de nuevo, pero no contestó.

\par 
%\textsuperscript{(1979.3)}
\textsuperscript{184:1.5} Anás estaba considerablemente molesto porque Jesús se negaba a contestar a sus preguntas, de tal manera que le dijo: <<¿No te preocupa que yo sea benévolo o no contigo? ¿No tienes consideración por el poder que tengo para determinar las cuestiones de tu próximo juicio?>> Cuando Jesús escuchó esto, dijo: <<Anás, sabes que no podrías tener ningún poder sobre mí si no fuera permitido por mi Padre. Algunos quisieran matar al Hijo del Hombre porque son ignorantes y no conocen nada mejor; pero tú, amigo, sabes lo que estás haciendo. ¿Cómo puedes, por tanto, rechazar la luz de Dios?>>

\par 
%\textsuperscript{(1979.4)}
\textsuperscript{184:1.6} Anás se quedó casi desconcertado por la manera amable en que Jesús le habló. Pero ya había decidido en su interior que Jesús debía irse de Palestina o morir; así pues, reunió su coraje y preguntó: <<¿Qué es exactamente lo que intentas enseñar a la gente? ¿Qué pretendes ser?>> Jesús contestó: <<Sabes muy bien que he hablado abiertamente al mundo. He enseñado en las sinagogas y muchas veces en el templo, donde todos los judíos y muchos gentiles me han escuchado. No he dicho nada en secreto; entonces, ¿por qué me preguntas sobre mi enseñanza? ¿Por qué no llamas a los que me han escuchado y les preguntas a ellos? Mira, todo Jerusalén ha oído lo que he dicho, aunque tú mismo no hayas escuchado estas enseñanzas>>. Pero antes de que Anás pudiera responder, el administrador principal del palacio, que estaba cerca, abofeteó a Jesús, diciendo: <<¿Cómo te atreves a contestarle así al sumo sacerdote?>> Anás no reprendió a su administrador, pero Jesús se dirigió a él, diciendo: <<Amigo mío, si he hablado mal, testifica contra el mal; pero si he dicho la verdad, entonces ¿por qué me golpeas?>>

\par 
%\textsuperscript{(1979.5)}
\textsuperscript{184:1.7} Anás lamentaba que su administrador hubiera abofeteado a Jesús, pero era demasiado orgulloso como para tener en cuenta el asunto. En su confusión, se fue a otra habitación y dejó solo a Jesús casi una hora con los criados de la casa y los guardias del templo.

\par 
%\textsuperscript{(1979.6)}
\textsuperscript{184:1.8} Cuando regresó, se puso al lado del Maestro y dijo: <<¿Pretendes ser el Mesías, el libertador de Israel?>> Jesús dijo: <<Anás, me conoces desde la época de mi juventud. Sabes que no pretendo ser nada más que lo que mi Padre ha decretado, y que he sido enviado a todos los hombres, tanto gentiles como judíos>>. Entonces Anás dijo: <<Me han dicho que has pretendido ser el Mesías; ¿es verdad?>> Jesús miró a Anás pero se limitó a contestar: <<Tú lo has dicho>>.

\par 
%\textsuperscript{(1980.1)}
\textsuperscript{184:1.9} Aproximadamente en ese momento, unos mensajeros del palacio de Caifás llegaron para preguntar a qué hora sería llevado Jesús ante el tribunal del sanedrín, y puesto que faltaba poco para el amanecer, Anás pensó que sería mejor enviar a Jesús, atado y custodiado por los guardias del templo, a Caifás. Él mismo los siguió un poco después.

\section*{2. Pedro en el patio}
\par 
%\textsuperscript{(1980.2)}
\textsuperscript{184:2.1} Cuando el grupo de guardias y soldados se acercaba a la entrada del palacio de Anás, Juan Zebedeo caminaba al lado del capitán de los soldados romanos. Judas se había quedado rezagado a cierta distancia, y Simón Pedro los seguía a lo lejos. Después de que Juan hubiera entrado en el patio del palacio con Jesús y los guardias, Judas se acercó a la cancela pero, al ver a Jesús y a Juan, continuó hacia la casa de Caifás, donde sabía que el verdadero juicio del Maestro se llevaría a cabo más tarde. Poco después de que Judas se hubiera marchado, llegó Simón Pedro, y mientras permanecía delante de la cancela, Juan lo vio en el momento en que estaban a punto de meter a Jesús en el palacio. La portera que estaba encargada de la cancela conocía a Juan, y cuando éste le pidió que dejara entrar a Pedro, ella asintió con placer.

\par 
%\textsuperscript{(1980.3)}
\textsuperscript{184:2.2} Al entrar en el patio, Pedro se dirigió hacia el fuego de carbón e intentó calentarse porque la noche era fría. Se sentía aquí totalmente fuera de lugar entre los enemigos de Jesús, y en verdad no estaba en su sitio. El Maestro no le había ordenado que se mantuviera cerca tal como se lo había recomendado a Juan. Pedro pertenecía al grupo de apóstoles que habían sido expresamente advertidos que no arriesgaran su vida durante estas horas del juicio y de la crucifixión de su Maestro.

\par 
%\textsuperscript{(1980.4)}
\textsuperscript{184:2.3} Pedro había tirado su espada poco antes de llegar a la cancela del palacio, de manera que entró desarmado en el patio de Anás. Su mente era un torbellino de confusión; apenas podía darse cuenta de que Jesús había sido arrestado. No conseguía captar la realidad de la situación ---que se encontraba allí en el patio de Anás, calentándose al lado de los criados del sumo sacerdote. Se preguntaba qué estarían haciendo los demás apóstoles y, al darle vueltas en la cabeza a la forma en que Juan había sido admitido en el palacio, llegó a la conclusión de que los criados lo conocían, puesto que Juan le había pedido a la portera que lo dejara entrar.

\par 
%\textsuperscript{(1980.5)}
\textsuperscript{184:2.4} Poco después de que la portera dejara entrar a Pedro, y mientras éste se calentaba junto al fuego, ella se le acercó y le dijo maliciosamente: <<¿No eres tú también uno de los discípulos de ese hombre?>> Pedro no debería haberse sorprendido de ser reconocido de esta manera, ya que había sido Juan quien le había pedido a la muchacha que le dejara traspasar las cancelas del palacio; pero estaba en tal estado de tensión nerviosa que el ser identificado como discípulo lo desequilibró, y con un solo pensamiento prioritario en su mente ---la idea de escapar con vida--- respondió de inmediato a la pregunta de la sirvienta, diciendo: <<No lo soy>>.

\par 
%\textsuperscript{(1980.6)}
\textsuperscript{184:2.5} Muy pronto, otro criado se acercó a Pedro y le preguntó: <<¿No te he visto en el jardín cuando arrestaron a ese tipo? ¿No eres tú también uno de sus seguidores?>> Pedro estaba ahora totalmente alarmado; no veía la manera de escapar sano y salvo de estos acusadores; negó pues con vehemencia toda conexión con Jesús, diciendo: <<No conozco a ese hombre, ni soy uno de sus seguidores>>.

\par 
%\textsuperscript{(1980.7)}
\textsuperscript{184:2.6} Casi en ese momento, la portera de la cancela llevó a Pedro a un lado y le dijo: <<Estoy segura de que eres un discípulo de ese Jesús, no solamente porque uno de sus seguidores me ha pedido que te dejara entrar en el patio, sino que mi hermana que está aquí te ha visto en el templo con ese hombre. ¿Por qué lo niegas?>> Cuando Pedro escuchó la acusación de la sirvienta, negó totalmente conocer a Jesús con muchas maldiciones y juramentos, diciendo de nuevo: <<No soy un seguidor de ese hombre; ni siquiera lo conozco; nunca he oído hablar de él anteriormente>>.

\par 
%\textsuperscript{(1981.1)}
\textsuperscript{184:2.7} Pedro se alejó del fuego durante un momento mientras caminaba por el patio. Le hubiera gustado escaparse, pero temía atraer la atención. Como tenía frío, regresó junto al fuego, y uno de los hombres que estaban cerca de él dijo: <<Tú eres sin duda uno de los discípulos de ese hombre. Ese Jesús es galileo, y tu manera de hablar te traiciona, pues hablas también como un galileo>>. Y Pedro negó de nuevo toda conexión con su Maestro.

\par 
%\textsuperscript{(1981.2)}
\textsuperscript{184:2.8} Pedro estaba tan inquieto que intentó evitar el contacto con sus acusadores alejándose del fuego y permaneciendo solo en el porche. Después de más de una hora de aislamiento, la portera y su hermana lo encontraron por casualidad, y las dos le tomaron el pelo de nuevo acusándolo de ser un seguidor de Jesús. Y otra vez negó la acusación. Justo cuando había negado una vez más toda conexión con Jesús, el gallo cantó, y Pedro recordó las palabras de advertencia que su Maestro le había dicho anteriormente aquella misma noche. Mientras permanecía allí, acongojado y abatido por el sentimiento de culpa, las puertas del palacio se abrieron y salieron los guardias conduciendo a Jesús hacia la casa de Caifás. Al pasar cerca de Pedro, el Maestro vio, a la luz de las antorchas, la cara de desesperación de su apóstol, anteriormente tan seguro de sí mismo y superficialmente valiente; volvió la cabeza y miró a Pedro. Pedro nunca olvidó aquella mirada durante toda su vida. Fue una mirada de compasión y de amor a la vez como ningún hombre mortal había visto nunca en el rostro del Maestro.

\par 
%\textsuperscript{(1981.3)}
\textsuperscript{184:2.9} Después de que Jesús y los guardias hubieron franqueado las cancelas del palacio, Pedro los siguió, pero sólo durante una corta distancia. No pudo ir más allá. Se sentó a un lado del camino y lloró amargamente. Después de derramar estas lágrimas de desesperación, volvió sobre sus pasos hacia el campamento con la esperanza de encontrar a su hermano, Andrés. Al llegar al campamento, sólo encontró a David Zebedeo, el cual envió a un mensajero para indicarle el camino hasta el lugar donde se había escondido su hermano en Jerusalén.

\par 
%\textsuperscript{(1981.4)}
\textsuperscript{184:2.10} Toda la experiencia de Pedro tuvo lugar en el patio del palacio de Anás en el monte Olivete. No siguió a Jesús hasta el palacio del sumo sacerdote Caifás. El hecho de que Pedro cayera en la cuenta, con el canto de un gallo, de que había negado repetidas veces a su Maestro, indica que todo esto sucedió fuera de Jerusalén, puesto que la ley prohibía tener aves de corral dentro de los límites de la ciudad.

\par 
%\textsuperscript{(1981.5)}
\textsuperscript{184:2.11} Hasta que el canto del gallo no devolvió a Pedro su sentido común, sólo había pensado, mientras iba y venía por el porche para entrar en calor, en la habilidad con que había eludido las acusaciones de los criados, y en cómo había frustrado sus intenciones de identificarlo con Jesús. De momento, sólo había considerado que aquellos criados no tenían el derecho moral ni legal de interrogarlo así, y se felicitaba realmente por la manera en que creía que había evitado ser identificado y quizás arrestado y encarcelado. A Pedro no se le ocurrió que había negado a su Maestro hasta que el gallo cantó. Únicamente cuando Jesús lo miró se dio cuenta de que no había estado a la altura de sus privilegios como embajador del reino.

\par 
%\textsuperscript{(1981.6)}
\textsuperscript{184:2.12} Después de dar el primer paso en el camino del compromiso y de la menor resistencia, a Pedro no parecía quedarle más salida que continuar con el tipo de conducta que había decidido. Se necesita un carácter grande y noble para cambiar de opinión y retomar el camino recto después de haber empezado mal. Demasiado a menudo, nuestra propia mente tiende a justificar nuestra permanencia en el camino erróneo después de haber entrado en él.

\par 
%\textsuperscript{(1982.1)}
\textsuperscript{184:2.13} Pedro nunca creyó por completo que podría ser perdonado hasta el momento en que se encontró con su Maestro después de la resurrección, y vio que era acogido como antes de las experiencias de esta trágica noche de negaciones.

\section*{3. Ante el tribunal de los sanedristas}
\par 
%\textsuperscript{(1982.2)}
\textsuperscript{184:3.1} Eran aproximadamente las tres y media de la madrugada de este viernes cuando el sumo sacerdote, Caifás, declaró constituido el tribunal sanedrista de investigación y pidió que Jesús fuera traído ante ellos para ser juzgado oficialmente. En tres ocasiones anteriores, el sanedrín, por una gran mayoría de votos, había decretado la muerte de Jesús, había decidido que merecía la muerte basándose en las acusaciones irregulares de transgredir la ley, blasfemar y burlarse de las tradiciones de los padres de Israel.

\par 
%\textsuperscript{(1982.3)}
\textsuperscript{184:3.2} Esta reunión del sanedrín no se había convocado de manera regular y no se celebraba en el lugar habitual, la cámara de piedras labradas del templo. Se trataba de un tribunal especial compuesto por unos treinta sanedristas, y se había convocado en el palacio del sumo sacerdote. Juan Zebedeo estuvo presente con Jesús durante todo este pretendido juicio.

\par 
%\textsuperscript{(1982.4)}
\textsuperscript{184:3.3} ¡Cuánto se vanagloriaban estos jefes de los sacerdotes, escribas, saduceos y algunos fariseos, de que este Jesús que había comprometido su posición social y desafiado su autoridad, estuviera ahora seguro entre sus manos! Y estaban decididos a no dejarlo escapar vivo de sus garras vengativas.

\par 
%\textsuperscript{(1982.5)}
\textsuperscript{184:3.4} Normalmente, cuando los judíos juzgaban a un hombre por un delito capital, procedían con una gran cautela y proporcionaban todas las garantías de la equidad en la selección de los testigos y en toda la conducta del juicio. Pero en esta ocasión, Caifás era más un fiscal que un juez imparcial.

\par 
%\textsuperscript{(1982.6)}
\textsuperscript{184:3.5} Jesús apareció ante este tribunal vestido con su ropa habitual y con las manos atadas detrás de la espalda. Todo el tribunal se quedó sorprendido y algo confuso por su aspecto majestuoso. Nunca habían contemplado a un preso semejante ni habían presenciado aquella sangre fría en un hombre que podía perder la vida.

\par 
%\textsuperscript{(1982.7)}
\textsuperscript{184:3.6} La ley judía exigía que al menos dos testigos estuvieran de acuerdo en un punto cualquiera antes de que se pudiera hacer una acusación contra un preso. Judas no podía servir de testigo contra Jesús, porque la ley judía prohibía expresamente el testimonio de un traidor. Más de veinte falsos testigos estaban disponibles para testificar contra Jesús, pero sus declaraciones eran tan contradictorias y tan evidentemente inventadas que los mismos sanedristas se sentían bastante avergonzados con el espectáculo. Jesús permanecía allí de pie, mirando con benignidad a aquellos perjuros, y su mismo semblante desconcertaba a los testigos mentirosos. Durante todos estos falsos testimonios, el Maestro no dijo ni una sola palabra; no respondió a ninguna de sus numerosas falsas acusaciones.

\par 
%\textsuperscript{(1982.8)}
\textsuperscript{184:3.7} La primera vez que dos de los testigos se acercaron algo a una apariencia de acuerdo fue cuando dos hombres atestiguaron que habían escuchado decir a Jesús, en el transcurso de uno de sus discursos en el templo, que <<destruiría este templo hecho por las manos del hombre y en tres días construiría otro templo sin emplear las manos del hombre>>. Aquello no era exactamente lo que Jesús había dicho, aparte del hecho de que había señalado su propio cuerpo cuando hizo aquel comentario.

\par 
%\textsuperscript{(1982.9)}
\textsuperscript{184:3.8} Aunque el sumo sacerdote le gritó a Jesús: <<¿No contestas a ninguna de estas acusaciones?>>, Jesús no abrió la boca. Permaneció allí en silencio mientras todos aquellos falsos testigos prestaban sus declaraciones. El odio, el fanatismo y la exageración sin escrúpulos caracterizaban de tal manera las palabras de aquellos perjuros, que sus testimonios caían por su propio peso. La mejor refutación de aquellas falsas acusaciones era el silencio sosegado y majestuoso del Maestro.

\par 
%\textsuperscript{(1983.1)}
\textsuperscript{184:3.9} Anás llegó poco después de que los falsos testigos empezaran a declarar, y tomó asiento al lado de Caifás. Anás se levantó entonces para argumentar que aquella amenaza de Jesús de destruir el templo era suficiente para justificar tres cargos contra él:

\par 
%\textsuperscript{(1983.2)}
\textsuperscript{184:3.10} 1. Que era un peligroso embaucador del pueblo. Que les enseñaba cosas imposibles y que los engañaba de otras maneras.

\par 
%\textsuperscript{(1983.3)}
\textsuperscript{184:3.11} 2. Que era un revolucionario fanático pues abogaba por el empleo de la violencia contra el templo sagrado, porque ¿cómo podría destruirlo de otra manera?

\par 
%\textsuperscript{(1983.4)}
\textsuperscript{184:3.12} 3. Que enseñaba la magia, puesto que prometía construir un nuevo templo, y sin ayudarse con las manos.

\par 
%\textsuperscript{(1983.5)}
\textsuperscript{184:3.13} Todo el sanedrín ya estaba de acuerdo en que Jesús era culpable de unas transgresiones de las leyes judías que merecían la muerte, pero ahora les preocupaba más preparar unas acusaciones relacionadas con su conducta y sus enseñanzas, que justificaran la sentencia de muerte que Pilatos debería pronunciar contra su preso. Sabían que tenían que obtener el consentimiento del gobernador romano antes de poder ejecutar legalmente a Jesús. Anás se sentía inclinado a seguir el método de hacer aparecer a Jesús como un peligroso educador que no podía estar por la calle entre la gente.

\par 
%\textsuperscript{(1983.6)}
\textsuperscript{184:3.14} Pero Caifás ya no podía soportar más la vista del Maestro, que permanecía allí de pie con una serenidad perfecta y en un silencio absoluto. Pensó que conocía al menos una manera de incitar al preso a hablar. En consecuencia, se precipitó hacia Jesús, agitó un dedo acusador delante del rostro del Maestro, y dijo: <<En nombre del Dios vivo, te ordeno que nos digas si eres el Libertador, el Hijo de Dios>>. Jesús contestó a Caifás: <<Lo soy. Pronto iré hacia el Padre, y dentro de poco el Hijo del Hombre será revestido de poder y reinará de nuevo sobre las huestes del cielo>>.

\par 
%\textsuperscript{(1983.7)}
\textsuperscript{184:3.15} Cuando el sumo sacerdote escuchó a Jesús pronunciar estas palabras, se encolerizó enormemente, y rasgando sus vestiduras exteriores, exclamó: <<¿Qué necesidad tenemos de más testigos? Mirad, ahora todos habéis escuchado la blasfemia de este hombre. ¿Qué pensáis ahora que podemos hacer con este blasfemo y transgresor de la ley?>> Y todos contestaron al unísono: <<Merece la muerte; que sea crucificado>>.

\par 
%\textsuperscript{(1983.8)}
\textsuperscript{184:3.16} Jesús no manifestó interés por ninguna de las preguntas que le hicieron cuando estaba delante de Anás o de los sanedristas, exceptuando la única pregunta relacionada con su misión donadora. Cuando se le preguntó si era el Hijo de Dios, contestó afirmativamente de manera instantánea e inequívoca.

\par 
%\textsuperscript{(1983.9)}
\textsuperscript{184:3.17} Anás deseaba que continuara el juicio y que se formularan unas acusaciones bien definidas en cuanto a la relación de Jesús con la ley y las instituciones romanas, para presentarlas posteriormente a Pilatos. Los consejeros estaban impacientes por terminar rápidamente este asunto, no sólo porque era el día de la preparación de la Pascua y no se podía hacer ningún trabajo seglar después del mediodía, sino también porque temían que Pilatos regresara en cualquier momento a Cesarea, la capital romana de Judea, puesto que sólo estaba en Jerusalén para la celebración de la Pascua.

\par 
%\textsuperscript{(1983.10)}
\textsuperscript{184:3.18} Pero Anás no logró conservar el control del tribunal. Después de que Jesús contestara tan inesperadamente a Caifás, el sumo sacerdote se adelantó y lo abofeteó. Anás se quedó verdaderamente impresionado cuando los otros miembros del tribunal escupieron a Jesús a la cara al salir de la sala, y muchos de ellos lo abofetearon burlonamente con la palma de la mano. Y así terminó, a las cuatro y media de la mañana, esta primera sesión del juicio de Jesús por parte de los sanedristas, en desorden y en medio de una confusión inaudita.

\par 
%\textsuperscript{(1984.1)}
\textsuperscript{184:3.19} Treinta falsos jueces llenos de prejuicios y cegados por la tradición, con sus falsos testigos, se atreven a sentarse a juzgar al justo Creador de un universo. Y estos acusadores apasionados están exasperados por el silencio majestuoso y el magnífico comportamiento de este Dios-hombre. Su silencio es terrible de soportar; su palabra es un reto intrépido. Permanece impasible ante sus amenazas e impávido ante sus ataques. El hombre juzga a Dios, pero incluso en ese momento Dios los ama y los salvaría si pudiera.

\section*{4. La hora de la humillación}
\par 
%\textsuperscript{(1984.2)}
\textsuperscript{184:4.1} En la cuestión de pronunciar una sentencia de muerte, la ley judía exigía que el tribunal celebrara dos sesiones. Esta segunda sesión debía tener lugar al día siguiente de la primera, y los miembros del tribunal debían pasar las horas intermedias ayunando y lamentándose. Pero estos hombres no podían esperar al día siguiente para confirmar su decisión de que Jesús debía morir. Sólo esperaron una hora. Mientras tanto, dejaron a Jesús en la sala de audiencia al cuidado de los guardias del templo, que junto con los criados del sumo sacerdote, se divirtieron acumulando todo tipo de indignidades sobre el Hijo del Hombre. Se burlaron de él, le escupieron y lo abofetearon cruelmente. Le golpeaban en la cara con una vara y luego le decían: <<Profetiza, Libertador, y dinos quién te ha golpeado>>. Continuaron así durante una hora entera, ultrajando y maltratando a este hombre de Galilea que no ofrecía resistencia.

\par 
%\textsuperscript{(1984.3)}
\textsuperscript{184:4.2} Durante esta hora trágica de sufrimientos y de juicios burlescos a manos de los guardias y criados ignorantes e insensibles, Juan Zebedeo estuvo esperando a solas, lleno de terror, en una habitación contigua. Cuando empezaron estos abusos, Jesús le indicó a Juan con un gesto de la cabeza que debía retirarse. El Maestro sabía muy bien que si permitía a su apóstol permanecer en la sala presenciando estas indignidades, se despertaría en Juan tal resentimiento que le hubiera conducido a una explosión de protesta indignada que probablemente le hubiera costado la vida.

\par 
%\textsuperscript{(1984.4)}
\textsuperscript{184:4.3} Durante esta hora espantosa, Jesús no pronunció ni una palabra. Para este alma humana dulce y sensible, unida en una relación de personalidad con el Dios de todo este universo, no hubo un período más amargo en la copa de su humillación que esta hora terrible a merced de estos guardias y criados ignorantes y crueles, que se habían sentido estimulados a maltratarlo debido al ejemplo de los miembros de este pretendido tribunal sanedrista.

\par 
%\textsuperscript{(1984.5)}
\textsuperscript{184:4.4} El corazón humano quizás no puede concebir el escalofrío de indignación que recorrió un enorme universo, mientras las inteligencias celestiales presenciaban este espectáculo de su amado Soberano sometiéndose a la voluntad de sus criaturas ignorantes y desviadas, en la esfera ensombrecida por el pecado de la desafortunada Urantia.

\par 
%\textsuperscript{(1984.6)}
\textsuperscript{184:4.5} ¿Qué es esa característica animal en el hombre que le conduce a querer insultar y atacar físicamente aquello que no puede alcanzar espiritualmente ni conseguir intelectualmente? Aún se esconde en el hombre medio civilizado una malvada brutalidad que intenta desahogarse en aquellos que son superiores en sabiduría y en logros espirituales. Observad la malvada tosquedad y la brutal ferocidad de estos hombres supuestamente civilizados, mientras obtenían cierta forma de placer animal atacando físicamente al Hijo del Hombre que no ofrecía resistencia. Mientras estos insultos, burlas y golpes caían sobre Jesús, él no se defendía, pero no estaba indefenso. Jesús no estaba derrotado, se limitaba a no luchar en el sentido material.

\par 
%\textsuperscript{(1985.1)}
\textsuperscript{184:4.6} Éstos son los momentos de las mayores victorias del Maestro en toda su larga y extraordinaria carrera como autor, sostén y salvador de un enorme y extenso universo. Después de vivir hasta su plenitud una vida revelando Dios al hombre, Jesús está dedicado ahora a revelar el hombre a Dios de una manera nueva y sin precedentes. Jesús está revelando ahora a los mundos la victoria final sobre todos los temores del aislamiento de la personalidad que siente la criatura. El Hijo del Hombre ha conseguido finalmente realizar su identidad como Hijo de Dios. Jesús no duda en afirmar que él y el Padre son uno; y basándose en el hecho y la verdad de esta experiencia suprema y celestial, exhorta a todo creyente en el reino a que se vuelva uno con él, como él y su Padre son uno. La experiencia viviente en la religión de Jesús se convierte así en la técnica cierta y segura mediante la cual los mortales de la Tierra, espiritualmente aislados y cósmicamente solitarios, consiguen escapar del aislamiento de la personalidad, con todos sus efectos de temores y de sentimientos de impotencia asociados. En las realidades fraternales del reino de los cielos, los hijos de Dios por la fe encuentran su liberación final del aislamiento del yo, tanto de manera personal como planetaria. El creyente que conoce a Dios experimenta cada vez más el éxtasis y la grandeza de la socialización espiritual a escala del universo ---la ciudadanía en el cielo asociada a la realización eterna del destino divino consistente en alcanzar la perfección.

\section*{5. La segunda reunión del tribunal}
\par 
%\textsuperscript{(1985.2)}
\textsuperscript{184:5.1} El tribunal se reunió de nuevo a las cinco y media de la mañana, y Jesús fue conducido a la habitación contigua donde estaba esperando Juan. Aquí, el soldado romano y los guardias del templo vigilaron a Jesús, mientras el tribunal empezaba a formular las acusaciones que se iban a presentar a Pilatos. Anás indicó claramente a sus asociados que la acusación de blasfemia no tendría ningún peso ante Pilatos. Judas estaba presente durante esta segunda reunión del tribunal, pero no prestó ninguna declaración.

\par 
%\textsuperscript{(1985.3)}
\textsuperscript{184:5.2} Esta sesión de la corte sólo duró media hora, y cuando levantaron la sesión para presentarse ante Pilatos, habían redactado la acusación contra Jesús estimando que merecía la muerte por tres razones:

\par 
%\textsuperscript{(1985.4)}
\textsuperscript{184:5.3} 1. Que pervertía a la nación judía; que engañaba al pueblo y lo incitaba a la rebelión.

\par 
%\textsuperscript{(1985.5)}
\textsuperscript{184:5.4} 2. Que enseñaba al pueblo a que se negara a pagar el tributo al César.

\par 
%\textsuperscript{(1985.6)}
\textsuperscript{184:5.5} 3. Que como pretendía ser rey y el fundador de un nuevo tipo de reino, incitaba a la traición contra el emperador.

\par 
%\textsuperscript{(1985.7)}
\textsuperscript{184:5.6} Todo este procedimiento era irregular y totalmente contrario a las leyes judías. No había dos testigos que se hubieran puesto de acuerdo en ninguna cuestión, excepto los que habían testificado en relación con la declaración de Jesús de que destruiría el templo y lo levantaría de nuevo en tres días. E incluso en este punto, ningún testigo había hablado en nombre de la defensa, y tampoco se le pidió a Jesús que explicara lo que había querido decir.

\par 
%\textsuperscript{(1985.8)}
\textsuperscript{184:5.7} El único punto sobre el que el tribunal podría haberlo juzgado coherentemente era el de la blasfemia, y hubiera estado basado enteramente en el propio testimonio del acusado. Incluso en lo que concierne a la blasfemia, no consiguieron votar oficialmente la pena de muerte.

\par 
%\textsuperscript{(1985.9)}
\textsuperscript{184:5.8} Y ahora, para presentarse ante Pilatos, se atrevían a formular tres cargos sobre los cuales ningún testigo había sido interrogado, y sobre los que se habían puesto de acuerdo en ausencia del acusado. Cuando todo estuvo hecho, tres de los fariseos se marcharon; querían que Jesús fuera aniquilado, pero no querían formular cargos contra él sin testigos y en su ausencia.

\par 
%\textsuperscript{(1986.1)}
\textsuperscript{184:5.9} Jesús no volvió a aparecer ante el tribunal de los sanedristas. Éstos no querían volver a contemplar su rostro mientras juzgaban su vida inocente. Jesús no se enteró (como hombre) de las acusaciones oficiales hasta que las escuchó de boca de Pilatos.

\par 
%\textsuperscript{(1986.2)}
\textsuperscript{184:5.10} Mientras Jesús estaba en la habitación con Juan y los guardias, y el tribunal celebraba su segunda sesión, algunas mujeres del palacio del sumo sacerdote vinieron con sus amigas para contemplar al extraño preso, y una de ellas le preguntó: <<¿Eres el Mesías, el Hijo de Dios?>> Y Jesús respondió: <<Si te lo digo, no me creerás; y si te lo pregunto, no contestarás>>.

\par 
%\textsuperscript{(1986.3)}
\textsuperscript{184:5.11} A las seis de aquella mañana, Jesús fue sacado de la casa de Caifás para aparecer ante Pilatos, a fin de que éste confirmara la sentencia de muerte que el tribunal de los sanedristas había decretado de manera tan injusta e irregular.


\chapter{Documento 185. El juicio ante Pilatos}
\par 
%\textsuperscript{(1987.1)}
\textsuperscript{185:0.1} POCO después de las seis de la mañana de este viernes 7 de abril del año 30, Jesús fue llevado ante Pilatos, el procurador romano que gobernaba Judea, Samaria e Idumea bajo la supervisión inmediata del legado de Siria. Los guardias del templo llevaron al Maestro, atado, a la presencia del gobernador romano, e iba acompañado por unos cincuenta de sus acusadores, incluyendo el tribunal sanedrista
(principalmente saduceos), Judas Iscariote, el sumo sacerdote Caifás y el apóstol Juan. Anás no se presentó ante Pilatos.

\par 
%\textsuperscript{(1987.2)}
\textsuperscript{185:0.2} Pilatos estaba levantado y preparado para recibir a este grupo de visitantes tan madrugadores, pues los hombres que habían conseguido su consentimiento la noche anterior para emplear los soldados romanos en el arresto del Hijo del Hombre le habían informado que traerían a Jesús temprano ante él. Se había acordado que este juicio tendría lugar frente al pretorio, un edificio adicional a la fortaleza de Antonia, donde Pilatos y su mujer establecían su cuartel general cuando se quedaban en Jerusalén.

\par 
%\textsuperscript{(1987.3)}
\textsuperscript{185:0.3} Aunque Pilatos dirigió una gran parte del interrogatorio de Jesús dentro de las salas del pretorio, el juicio público se celebró en el exterior, en los escalones que conducían a la entrada principal. Fue una concesión que hizo a los judíos, los cuales se negaban a entrar en cualquier edificio gentil donde quizás se había utilizado la levadura en este día de la preparación de la Pascua. Una conducta así no solamente los volvería ceremonialmente impuros, privándolos con ello de poder participar en la fiesta de acción de gracias de la tarde, sino que también necesitarían someterse a las ceremonias de purificación después de la puesta del Sol para poder compartir la cena pascual.

\par 
%\textsuperscript{(1987.4)}
\textsuperscript{185:0.4} Aunque a estos judíos no les molestaba en absoluto la conciencia cuando tramaban asesinar judicialmente a Jesús, sin embargo eran escrupulosos en lo referente a todas estas cuestiones de pureza ceremonial y de regularidad tradicional. Y estos judíos no han sido los únicos en dejar de reconocer sus altas y santas obligaciones de naturaleza divina, mientras prestaban una atención meticulosa a cosas de poca importancia por el bienestar humano tanto en el tiempo como en la eternidad.

\section*{1. Poncio Pilatos}
\par 
%\textsuperscript{(1987.5)}
\textsuperscript{185:1.1} Si Poncio Pilatos no hubiera sido un gobernador razonablemente bueno de las provincias menores, Tiberio difícilmente le hubiera permitido que permaneciera diez años como procurador de Judea. Aunque era un administrador razonablemente bueno, moralmente era un cobarde. No era un hombre lo bastante grande como para comprender la naturaleza de su tarea como gobernador de los judíos. No lograba captar el hecho de que estos hebreos tenían una religión \textit{real}, una fe por la que estaban dispuestos a morir, y que millones y millones de ellos, dispersos aquí y allá por todo el imperio, consideraban a Jerusalén como el santuario de su fe y respetaban al sanedrín como el tribunal más alto de la Tierra.

\par 
%\textsuperscript{(1988.1)}
\textsuperscript{185:1.2} Pilatos no amaba a los judíos, y este odio profundo empezó a manifestarse muy pronto. De todas las provincias romanas, ninguna era más difícil de gobernar que Judea. Pilatos nunca comprendió realmente los problemas implicados en la administración de los judíos y por esta razón, desde el principio de su experiencia como gobernador, cometió una serie de errores descomunales casi fatales y prácticamente suicidas. Estos errores fueron los que dieron a los judíos tanto poder sobre él. Cuando querían influir sobre sus decisiones, todo lo que tenían que hacer era amenazarlo con una insurrección, y Pilatos capitulaba rápidamente. Esta indecisión aparente, o falta de valor moral del procurador, se debía principalmente al recuerdo de una serie de controversias que había tenido con los judíos, y en cada caso habían sido ellos los que habían vencido. Los judíos sabían que Pilatos les tenía miedo, que temía por su posición ante Tiberio, y emplearon este conocimiento en gran perjuicio del gobernador en numerosas ocasiones.

\par 
%\textsuperscript{(1988.2)}
\textsuperscript{185:1.3} La desventaja de Pilatos ante los judíos se produjo a consecuencia de una serie de encuentros desafortunados. En primer lugar, no supo tomarse en serio el profundo prejuicio judío contra todas las imágenes, consideradas como símbolos de idolatría. Por consiguiente, permitió que sus soldados entraran en Jerusalén sin quitar las imágenes del César de sus banderas, como los soldados romanos habían tenido la costumbre de hacerlo bajo su predecesor. Una numerosa delegación de judíos esperó a Pilatos durante cinco días, implorándole que hiciera quitar aquellas imágenes de los estandartes militares. Se negó rotundamente a conceder su petición y los amenazó de muerte inmediata. Como él mismo era un escéptico, Pilatos no comprendía que unos hombres con unos fuertes sentimientos religiosos no dudarían en morir por sus convicciones religiosas; por eso, se sintió consternado cuando aquellos judíos se reunieron desafiantes delante de su palacio, inclinaron sus rostros hasta el suelo y enviaron a decir que estaban preparados para morir. Pilatos comprendió entonces que había hecho una amenaza que no quería llevar a cabo. Cedió, y ordenó que quitaran las imágenes de los estandartes de sus soldados en Jerusalén; desde aquel día en adelante, se encontró ampliamente sometido a los caprichos de los dirigentes judíos, que habían descubierto así su debilidad, la de hacer amenazas que temía ejecutar.

\par 
%\textsuperscript{(1988.3)}
\textsuperscript{185:1.4} Pilatos decidió posteriormente recuperar su prestigio perdido y, en consecuencia, hizo colocar los escudos del emperador, como los que se empleaban generalmente para adorar al César, en los muros del palacio de Herodes en Jerusalén. Cuando los judíos protestaron, se mantuvo inflexible. Como se negó a escuchar sus protestas, los judíos apelaron rápidamente a Roma, y el emperador ordenó con igual rapidez que se quitaran los escudos ofensivos. Y Pilatos gozó entonces de mucha menos estima que antes.

\par 
%\textsuperscript{(1988.4)}
\textsuperscript{185:1.5} Otra cosa que le granjeó una gran desaprobación entre los judíos fue el hecho de que se atrevió a coger dinero del tesoro del templo para financiar la construcción de un nuevo acueducto, a fin de proporcionar un mayor abastecimiento de agua a los millones de visitantes de Jerusalén en las épocas de las grandes fiestas religiosas. Los judíos estimaban que sólo el sanedrín podía gastar los fondos del templo, y nunca dejaron de arremeter contra Pilatos por esta orden arbitraria. Esta decisión provocó no menos de veinte motines y mucho derramamiento de sangre. El último de estos graves disturbios consistió en la matanza de un numeroso grupo de galileos cuando estaban rindiendo culto en el altar.

\par 
%\textsuperscript{(1988.5)}
\textsuperscript{185:1.6} Es significativo constatar que, aunque este gobernante romano indeciso sacrificó a Jesús por miedo a los judíos y para salvaguardar su posición personal, finalmente fue destituido a consecuencia de una matanza innecesaria de samaritanos en conexión con las pretensiones de un falso Mesías que había conducido unas tropas al Monte Gerizim, donde pretendía que estaban enterradas las vasijas del templo; y estallaron unos violentos motines cuando no logró revelar el escondite de las vasijas sagradas tal como lo había prometido. A consecuencia de este episodio, el legado de Siria ordenó a Pilatos que volviera a Roma. Tiberio murió mientras Pilatos iba camino de Roma, y no se le nombró de nuevo procurador de Judea. Nunca se recuperó por completo de la lamentable condena que hizo al haber consentido la crucifixión de Jesús. Como no encontró ningún favor a los ojos del nuevo emperador, se retiró a la provincia de Lausana, donde posteriormente se suicidó.

\par 
%\textsuperscript{(1989.1)}
\textsuperscript{185:1.7} Claudia Prócula, la mujer de Pilatos, había oído hablar mucho de Jesús por boca de su criada, una fenicia que creía en el evangelio del reino. Después de la muerte de Pilatos, Claudia se identificó de manera sobresaliente con la difusión de la buena nueva.

\par 
%\textsuperscript{(1989.2)}
\textsuperscript{185:1.8} Todo esto explica una gran parte de lo que sucedió este trágico viernes por la mañana. Es fácil comprender por qué los judíos se atrevían a darle órdenes a Pilatos ---a hacer que se levantara a las seis de la mañana para juzgar a Jesús--- y también por qué no dudaron en amenazarlo con acusarlo de traición ante el emperador si se atrevía a rehusar sus peticiones de ejecutar a Jesús.

\par 
%\textsuperscript{(1989.3)}
\textsuperscript{185:1.9} Un gobernador romano digno, que no hubiera estado implicado de manera desfavorable con los dirigentes de los judíos, nunca hubiera permitido que estos fanáticos religiosos sedientos de sangre provocaran la muerte de un hombre que él mismo había declarado sin falta e inocente de las falsas acusaciones. Roma cometió una gran equivocación, un error trascendental en los asuntos terrestres, cuando envió al mediocre Pilatos como gobernador de Palestina. Tiberio debería haber enviado a los judíos al mejor administrador provincial del imperio.

\section*{2. Jesús comparece ante Pilatos}
\par 
%\textsuperscript{(1989.4)}
\textsuperscript{185:2.1} Cuando Jesús y sus acusadores se hubieron congregado delante de la sala de juicios de Pilatos, el gobernador romano salió y se dirigió a la compañía reunida, preguntando: <<¿Qué acusación traéis contra este hombre?>> Los saduceos y los consejeros, que habían hecho suyo el deshacerse de Jesús, habían decidido presentarse ante Pilatos para pedirle la confirmación de la sentencia de muerte pronunciada contra él, sin ofrecer ninguna acusación definida. Por esta razón, el portavoz del tribunal de los sanedristas le contestó a Pilatos: <<Si este hombre no fuera un malhechor, no te lo habríamos entregado>>.

\par 
%\textsuperscript{(1989.5)}
\textsuperscript{185:2.2} Cuando Pilatos observó que eran reacios a exponer sus acusaciones contra Jesús, aunque sabía que habían pasado toda la noche deliberando sobre su culpabilidad, les contestó: <<Puesto que no estáis de acuerdo en unas acusaciones determinadas, ¿por qué no os lleváis a este hombre y lo juzgáis según vuestras propias leyes?>>

\par 
%\textsuperscript{(1989.6)}
\textsuperscript{185:2.3} Entonces, el actuario del tribunal del sanedrín le dijo a Pilatos: <<No nos está permitido ejecutar a nadie, y este perturbador de nuestra nación merece morir por las cosas que ha dicho y hecho. Por eso hemos venido ante ti para que confirmes esta sentencia>>.

\par 
%\textsuperscript{(1989.7)}
\textsuperscript{185:2.4} Presentarse ante el gobernador romano con este intento de evasión revela la inquina y el malhumor de los sanedristas hacia Jesús, así como su falta de respeto por la equidad, el honor y la dignidad de Pilatos. ¡Qué desfachatez la de estos ciudadanos sometidos, los cuales comparecían ante su gobernador provincial para pedirle un decreto de ejecución contra un hombre antes de concederle un juicio justo, e incluso sin presentar unas acusaciones criminales definidas contra él!

\par 
%\textsuperscript{(1989.8)}
\textsuperscript{185:2.5} Pilatos conocía algunas cosas del trabajo de Jesús entre los judíos, y supuso que las acusaciones que se podían presentar contra él estarían relacionadas con infracciones a las leyes eclesiásticas judías; por esta razón, trató de remitir el caso al propio tribunal judío. Además, Pilatos se deleitó en hacerles confesar públicamente que no tenían poder para pronunciar y ejecutar una sentencia de muerte, ni siquiera contra un miembro de su propia raza, al cual habían llegado a despreciar con un odio lleno de amargura y de envidia.

\par 
%\textsuperscript{(1990.1)}
\textsuperscript{185:2.6} Unas horas antes, poco antes de la medianoche y después de haber concedido el permiso de emplear los soldados romanos para detener en secreto a Jesús, Pilatos había escuchado más cosas sobre Jesús y sus enseñanzas de labios de su mujer, Claudia, que se había convertido parcialmente al judaísmo, y que más tarde creyó plenamente en el evangelio de Jesús.

\par 
%\textsuperscript{(1990.2)}
\textsuperscript{185:2.7} A Pilatos le hubiera gustado posponer esta audiencia, pero vio que los dirigentes judíos estaban decididos a continuar con el caso. Sabía que esta mañana no era solamente la de la preparación de la Pascua, sino que como era viernes, también era el día de la preparación para el sábado judío de descanso y de culto.

\par 
%\textsuperscript{(1990.3)}
\textsuperscript{185:2.8} Como Pilatos era extremadamente sensible a la manera irrespetuosa en que estos judíos lo trataban, no estaba dispuesto a satisfacer sus exigencias de sentenciar a muerte a Jesús sin un juicio. Por consiguiente, después de esperar unos momentos para que presentaran sus acusaciones contra el detenido, se volvió hacia ellos y dijo: <<No condenaré a muerte a este hombre sin un juicio; y tampoco consentiré en interrogarlo hasta que hayáis presentado por escrito vuestras acusaciones contra él>>.

\par 
%\textsuperscript{(1990.4)}
\textsuperscript{185:2.9} Cuando el sumo sacerdote y los demás escucharon a Pilatos decir esto, hicieron una señal al actuario del tribunal, el cual entregó entonces a Pilatos las acusaciones escritas contra Jesús. Estas acusaciones eran:

\par 
%\textsuperscript{(1990.5)}
\textsuperscript{185:2.10} <<El tribunal sanedrista estima que este hombre es un malhechor y un perturbador de nuestra nación, porque es culpable de:

\par 
%\textsuperscript{(1990.6)}
\textsuperscript{185:2.11} 1. Pervertir a nuestra nación e incitar a nuestro pueblo a la rebelión.

\par 
%\textsuperscript{(1990.7)}
\textsuperscript{185:2.12} 2. Prohibir al pueblo que pague el tributo al César.

\par 
%\textsuperscript{(1990.8)}
\textsuperscript{185:2.13} 3. Llamarse a sí mismo rey de los judíos y enseñar la fundación de un nuevo reino>>.

\par 
%\textsuperscript{(1990.9)}
\textsuperscript{185:2.14} Jesús no había sido juzgado de manera regular ni declarado legalmente culpable de ninguna de estas acusaciones. Ni siquiera las escuchó cuando fueron expresadas por primera vez, pero Pilatos lo hizo traer del pretorio, donde estaba a cargo de los guardias, e insistió para que estas acusaciones fueran repetidas delante de Jesús.

\par 
%\textsuperscript{(1990.10)}
\textsuperscript{185:2.15} Cuando Jesús escuchó estas acusaciones, sabía muy bien que no había sido interrogado sobre estas cuestiones ante el tribunal judío, y también lo sabían Juan Zebedeo y sus acusadores, pero no respondió nada a estos falsos cargos. Incluso cuando Pilatos le rogó que respondiera a sus acusadores, no abrió la boca. Pilatos se quedó tan sorprendido por la injusticia de todo el procedimiento y tan impresionado por el comportamiento silencioso y magistral de Jesús, que decidió llevar al preso al interior de la sala e interrogarlo en privado.

\par 
%\textsuperscript{(1990.11)}
\textsuperscript{185:2.16} Pilatos tenía la mente confusa, miedo a los judíos en su fuero interno, y su espíritu poderosamente agitado por el espectáculo que ofrecía Jesús, el cual permanecía majestuosamente allí de pie delante de sus acusadores sedientos de sangre, contemplándolos no con un desprecio silencioso, sino con una expresión de verdadera piedad y de afecto entristecido.

\section*{3. El interrogatorio privado de Pilatos}
\par 
%\textsuperscript{(1991.1)}
\textsuperscript{185:3.1} Pilatos llevó a Jesús y a Juan Zebedeo a una habitación privada, dejando a los guardias fuera en la sala; le rogó al preso que se sentara, se sentó a su lado y le hizo varias preguntas. Pilatos empezó su conversación con Jesús asegurándole que no creía en la primera acusación contra él: la de que pervertía a la nación e incitaba a la rebelión. Luego le preguntó: <<¿Has enseñado alguna vez que se debe negar el tributo al César?>> Jesús señaló a Juan y dijo: <<Pregúntale a él o a cualquier otra persona que haya escuchado mi enseñanza>>. Entonces Pilatos le preguntó a Juan sobre este asunto del tributo, y Juan testificó acerca de la enseñanza de su Maestro y explicó que Jesús y sus apóstoles pagaban los impuestos tanto al César como al templo. Cuando Pilatos hubo interrogado a Juan, dijo: <<Procura no decirle a nadie que he hablado contigo>>. Y Juan no reveló nunca este asunto.

\par 
%\textsuperscript{(1991.2)}
\textsuperscript{185:3.2} Pilatos se volvió entonces para hacerle nuevas preguntas a Jesús, diciendo: <<Y ahora, en cuanto a la tercera acusación contra ti, ¿eres el rey de los judíos?>> Puesto que en la voz de Pilatos había un tono de interrogación posiblemente sincera, Jesús le sonrió al procurador y dijo: <<Pilatos, ¿preguntas esto por ti mismo, o coges esta pregunta de esos otros, mis acusadores?>> Entonces, el gobernador respondió con un tono parcialmente indignado: <<¿Soy yo judío? Tu propio pueblo y los jefes de los sacerdotes te han entregado y me han pedido que te condene a muerte. Pongo en duda la validez de sus acusaciones y sólo intento descubrir por mí mismo qué has hecho. Dime, ¿has dicho que eres el rey de los judíos, y has tratado de fundar un nuevo reino?>>

\par 
%\textsuperscript{(1991.3)}
\textsuperscript{185:3.3} Jesús le dijo entonces a Pilatos: <<¿No percibes que mi reino no es de este mundo? Si mi reino fuera de este mundo, mis discípulos lucharían con toda seguridad para que yo no fuera entregado a los judíos. Mi presencia aquí delante de ti con estas ataduras es suficiente para mostrar a todos los hombres que mi reino es un dominio espiritual, la fraternidad misma de los hombres que se han vuelto hijos de Dios a través de la fe y por amor. Y esta salvación es tanto para los gentiles como para los judíos>>.

\par 
%\textsuperscript{(1991.4)}
\textsuperscript{185:3.4} <<Entonces, ¿después de todo eres rey?>> dijo Pilatos. Y Jesús respondió: <<Sí, soy un rey de ese tipo, y mi reino es la familia de los hijos por la fe de mi Padre que está en los cielos. Nací en este mundo con esa finalidad, para mostrar mi Padre a todos los hombres y dar testimonio de la verdad de Dios. E incluso ahora te afirmo que todo el que ama la verdad escucha mi voz>>.

\par 
%\textsuperscript{(1991.5)}
\textsuperscript{185:3.5} Entonces dijo Pilatos con una mezcla de burla y de sinceridad: <<La verdad, ¿cuál es la verdad ---quién la conoce?>>

\par 
%\textsuperscript{(1991.6)}
\textsuperscript{185:3.6} Pilatos no era capaz de profundizar en las palabras de Jesús ni de comprender la naturaleza de su reino espiritual, pero ahora estaba seguro de que el detenido no había hecho nada que mereciera la muerte. Una mirada a Jesús cara a cara era suficiente para convencer incluso a Pilatos de que este hombre dulce y cansado, pero justo y majestuoso, no era ningún revolucionario salvaje y peligroso que aspirara a establecerse en el trono temporal de Israel. Pilatos creía comprender algo de lo que Jesús había querido decir cuando se llamó a sí mismo rey, porque conocía las enseñanzas de los estoicos que proclamaban que <<el hombre sabio es rey>>. Pilatos estaba enteramente convencido de que en lugar de ser un sedicioso peligroso, Jesús no era ni más ni menos que un visionario inofensivo, un fanático inocente.

\par 
%\textsuperscript{(1991.7)}
\textsuperscript{185:3.7} Después de interrogar al Maestro, Pilatos regresó donde estaban los jefes de los sacerdotes y los acusadores de Jesús, y dijo: <<He interrogado a este hombre, y no encuentro ninguna falta en él. No creo que sea culpable de las acusaciones que habéis efectuado contra él; creo que debe ser puesto en libertad>>. Cuando los judíos escucharon esto, se encolerizaron enormemente, hasta el punto de que gritaron ferozmente que Jesús debía morir; y uno de los sanedristas subió con descaro hasta el lado de Pilatos, diciendo: <<Este hombre excita al pueblo, empezando por Galilea y continuando por toda Judea. Causa daño y es un malhechor. Si dejas en libertad a este hombre perverso, lo lamentarás durante mucho tiempo>>.

\par 
%\textsuperscript{(1992.1)}
\textsuperscript{185:3.8} Pilatos se veía en el apuro de no saber qué hacer con Jesús; por eso, cuando les oyó decir que había empezado su trabajo en Galilea, pensó en esquivar la responsabilidad de resolver el caso, o al menos ganar tiempo para reflexionar, enviando a Jesús a comparecer ante Herodes, que entonces estaba en la ciudad para asistir a la Pascua. Pilatos pensó también que este gesto serviría de antídoto contra algunos sentimientos desagradables que habían existido entre él y Herodes desde hacía algún tiempo, debidos a numerosos malentendidos sobre cuestiones de jurisdicción.

\par 
%\textsuperscript{(1992.2)}
\textsuperscript{185:3.9} Pilatos llamó a los guardias y les dijo: <<Este hombre es galileo. Llevadlo inmediatamente ante Herodes, y cuando lo haya interrogado, informadme de sus conclusiones>>. Y los guardias llevaron a Jesús ante Herodes.

\section*{4. Jesús ante Herodes}
\par 
%\textsuperscript{(1992.3)}
\textsuperscript{185:4.1} Cuando Herodes Antipas se quedaba en Jerusalén, residía en el viejo palacio macabeo de Herodes el Grande, y Jesús fue llevado ahora por los guardias del templo a esta residencia del anterior rey, seguido por sus acusadores y una multitud en aumento. Herodes había oído hablar de Jesús desde hacía tiempo, y tenía mucha curiosidad por conocerlo. Cuando el Hijo del Hombre estuvo ante él este viernes por la mañana, el malvado idumeo no recordó en ningún momento al muchacho de años atrás que se había presentado ante él en Séforis para rogarle una decisión justa sobre el dinero que le debían a su padre, el cual había muerto accidentalmente mientras trabajaba en uno de los edificios públicos. Que Herodes supiera, nunca había visto a Jesús, aunque se había inquietado mucho a causa de él cuando la actividad del Maestro estaba centrada en Galilea. Ahora que Jesús estaba bajo la custodia de Pilatos y de los judeos, Herodes ansiaba verlo, pues se sentía protegido contra cualquier problema que Jesús pudiera causar en el futuro. Herodes había oído hablar mucho de los milagros que Jesús había hecho, y esperaba realmente verle realizar algún prodigio.

\par 
%\textsuperscript{(1992.4)}
\textsuperscript{185:4.2} Cuando llevaron a Jesús ante Herodes, el tetrarca se quedó sorprendido de su apariencia majestuosa y de la serenidad de su semblante. Herodes le hizo preguntas a Jesús durante unos quince minutos, pero el Maestro no quiso responder. Herodes lo provocó y lo desafió a que realizara un milagro, pero Jesús no quiso contestar a sus numerosas preguntas ni responder a sus insultos.

\par 
%\textsuperscript{(1992.5)}
\textsuperscript{185:4.3} Herodes se volvió entonces hacia los jefes de los sacerdotes y los saduceos, prestó oído a sus acusaciones, y escuchó todo lo que Pilatos había oído y más aún acerca de las supuestas maldades del Hijo del Hombre. Finalmente, convencido de que Jesús no hablaría ni realizaría un prodigio para él, Herodes, después de burlarse de él durante un rato, le colocó un viejo manto de púrpura real y lo envió de vuelta a Pilatos. Herodes sabía que no tenía ninguna jurisdicción sobre Jesús en Judea. Aunque le alegraba creer que por fin se iba a desembarazar de Jesús en Galilea, estaba agradecido de que fuera Pilatos quien tenía la responsabilidad de quitarle la vida. Herodes nunca se había recuperado por completo del miedo que padecía por haber ejecutado a Juan el Bautista. En algunos momentos, Herodes había temido incluso que Jesús fuera Juan resucitado de entre los muertos. Ahora se había librado de este temor, puesto que observó que Jesús era un tipo de persona muy diferente al directo y fogoso profeta que se había atrevido a sacar a la luz y denunciar su vida privada.

\section*{5. Jesús vuelve ante Pilatos}
\par 
%\textsuperscript{(1993.1)}
\textsuperscript{185:5.1} Cuando los guardias volvieron a traer a Jesús ante Pilatos, éste salió a los escalones del pretorio donde se había colocado su asiento para el juicio, convocó a los principales sacerdotes y a los sanedristas, y les dijo: <<Habéis traído a este hombre ante mí acusándolo de que pervierte al pueblo, prohíbe el pago de los impuestos y pretende ser el rey de los judíos. Lo he interrogado y no lo he encontrado culpable de esas acusaciones. De hecho, no encuentro ninguna falta en él. Luego lo he enviado a Herodes, y el tetrarca debe haber llegado a la misma conclusión, puesto que nos lo ha enviado de vuelta. Sin duda este hombre no ha hecho nada que merezca la muerte. Si aún seguís pensando que necesita ser castigado, estoy dispuesto a darle un escarmiento antes de ponerlo en libertad>>.

\par 
%\textsuperscript{(1993.2)}
\textsuperscript{185:5.2} En el preciso momento en que los judíos se disponían a gritar sus protestas por la liberación de Jesús, una gran muchedumbre se acercó hasta el pretorio para pedirle a Pilatos que soltara a un preso en honor de la fiesta de la Pascua. Desde hacía algún tiempo, los gobernadores romanos habían tenido la costumbre de permitir que la plebe escogiera a un hombre encarcelado o condenado para que fuera indultado en la época de la Pascua. Ahora que este gentío se presentaba ante él para pedirle que liberara a un preso, y puesto que Jesús había gozado tan recientemente de una gran popularidad entre las multitudes, a Pilatos se le ocurrió que quizás podría salir de este apuro proponiéndole a este grupo que, ya que Jesús estaba ahora preso delante de su tribunal, les soltaría a este hombre de Galilea como prueba de la buena voluntad de la Pascua.

\par 
%\textsuperscript{(1993.3)}
\textsuperscript{185:5.3} Mientras la multitud invadía las escaleras del edificio, Pilatos les oyó gritar el nombre de un tal Barrabás. Barrabás era un conocido agitador político y ladrón asesino, hijo de un sacerdote, que había sido capturado recientemente in fraganti robando y asesinando en la carretera de Jericó. Este hombre había sido condenado a muerte y sería ejecutado en cuanto terminaran las fiestas de la Pascua.

\par 
%\textsuperscript{(1993.4)}
\textsuperscript{185:5.4} Pilatos se levantó y explicó a la multitud que Jesús había sido traído ante él por los jefes de los sacerdotes, los cuales querían que fuera condenado a muerte por ciertas acusaciones, y que él no creía que este hombre mereciera la muerte. Pilatos dijo: <<¿A quién preferís entonces que os suelte, a ese Barrabás, el asesino, o a este Jesús de Galilea?>> Cuando Pilatos hubo dicho esto, los jefes de los sacerdotes y los consejeros del sanedrín exclamaron a voz en grito: <<¡Barrabás, Barrabás!>> Cuando la gente vio que los jefes de los sacerdotes estaban dispuestos a conseguir la muerte de Jesús, se unieron rápidamente al clamor pidiendo su vida, mientras vociferaban ruidosamente que soltaran a Barrabás.

\par 
%\textsuperscript{(1993.5)}
\textsuperscript{185:5.5} Pocos días antes de esto, la multitud había sentido un respeto reverencial por Jesús, pero la muchedumbre no miraba con respeto a alguien que había pretendido ser el Hijo de Dios y ahora se encontraba preso de los principales sacerdotes y de los dirigentes, con el riesgo de ser condenado a muerte ante el tribunal de Pilatos. Jesús podía ser un héroe a los ojos del pueblo cuando echaba del templo a los cambistas y a los mercaderes, pero no cuando era un preso sin resistencia en manos de sus enemigos y con el riesgo de perder la vida.

\par 
%\textsuperscript{(1993.6)}
\textsuperscript{185:5.6} Pilatos se indignó al ver a los jefes de los sacerdotes pidiendo a voces el indulto de un asesino bien conocido mientras gritaban para conseguir la sangre de Jesús. Vio su maldad y su odio y percibió sus prejuicios y su envidia. Por eso les dijo: <<¿Cómo podéis escoger la vida de un asesino, en lugar de preferir la de este hombre cuyo peor crimen consiste en hacerse llamar en sentido figurado el rey de los judíos?>> Pero esta declaración que hizo Pilatos no fue sabia. Los judíos eran un pueblo orgulloso, ahora sometido al yugo político romano, pero que esperaban la venida de un Mesías que los liberaría de su esclavitud de los gentiles con una gran exhibición de poder y de gloria. Se sintieron más ofendidos de lo que Pilatos podía suponer, por la insinuación de que este instructor de modales suaves que enseñaba unas doctrinas extrañas, ahora arrestado y acusado de unos delitos que merecían la muerte, pudiera ser considerado como <<el rey de los judíos>>. Contemplaron esta observación como un insulto a todo lo que consideraban sagrado y honorable en su existencia nacional, y por esta razón todos se pusieron a gritar con todas sus fuerzas por la liberación de Barrabás y la muerte de Jesús.

\par 
%\textsuperscript{(1994.1)}
\textsuperscript{185:5.7} Pilatos sabía que Jesús era inocente de las acusaciones presentadas contra él, y si hubiera sido un juez justo y valiente, lo habría absuelto y puesto en libertad. Pero tenía miedo de desafiar a estos judíos encolerizados, y mientras titubeaba en cumplir con su deber, llegó un mensajero y le entregó un mensaje sellado de su mujer, Claudia.

\par 
%\textsuperscript{(1994.2)}
\textsuperscript{185:5.8} Pilatos indicó a los que estaban congregados ante él que deseaba leer la comunicación que acababa de recibir antes de proseguir con el asunto que tenía ante él. Pilatos abrió la carta de su mujer y leyó: <<Te ruego que no tengas nada que ver con este hombre justo e inocente a quien llaman Jesús. Esta noche he sufrido mucho en un sueño a causa de él>>. Esta nota de Claudia no sólo afectó mucho a Pilatos y retrasó así el juicio de este asunto, sino que desgraciadamente también proporcionó a los dirigentes judíos un tiempo considerable para circular libremente entre la multitud e incitar al pueblo a pedir la liberación de Barrabás y a gritar que crucificaran a Jesús.

\par 
%\textsuperscript{(1994.3)}
\textsuperscript{185:5.9} Finalmente, Pilatos se dedicó una vez más a solucionar el problema que tenía delante, preguntándole a la asamblea mixta compuesta por los dirigentes judíos y la multitud que buscaba el indulto: <<¿Qué he de hacer con el que llaman el rey de los judíos?>> Y todos gritaron al unísono: <<¡Crucifícalo! ¡Crucifícalo!>> La unanimidad de esta petición por parte de una gente de todo tipo sorprendió y alarmó a Pilatos, el juez injusto y dominado por el miedo.

\par 
%\textsuperscript{(1994.4)}
\textsuperscript{185:5.10} Entonces Pilatos dijo una vez más: <<¿Por qué queréis crucificar a este hombre? ¿Qué mal ha hecho? ¿Quién quiere adelantarse para testificar contra él?>> Pero cuando escucharon que Pilatos hablaba en defensa de Jesús, se limitaron a gritar aún más: <<¡Crucifícalo! ¡Crucifícalo!>>

\par 
%\textsuperscript{(1994.5)}
\textsuperscript{185:5.11} Entonces Pilatos recurrió de nuevo a ellos para el asunto relacionado con la liberación del preso de la Pascua, diciendo: <<Os pregunto una vez más, ¿cuál de estos presos debo soltaros en estas fechas de vuestra Pascua?>> Y el gentío gritó de nuevo: <<¡Danos a Barrabás!>>

\par 
%\textsuperscript{(1994.6)}
\textsuperscript{185:5.12} Entonces dijo Pilatos: <<Si suelto a Barrabás, el asesino, ¿qué he de hacer con Jesús?>> Y una vez más la multitud gritó al unísono: <<¡Crucifícalo! ¡Crucifícalo!>>

\par 
%\textsuperscript{(1994.7)}
\textsuperscript{185:5.13} Pilatos se sintió aterrorizado por el clamor insistente del gentío, que actuaba bajo la dirección inmediata de los jefes de los sacerdotes y los consejeros del sanedrín; sin embargo, decidió hacer al menos una última tentativa por apaciguar a la muchedumbre y salvar a Jesús.

\section*{6. El último llamamiento de Pilatos}
\par 
%\textsuperscript{(1994.8)}
\textsuperscript{185:6.1} Sólo los enemigos de Jesús participan en todo lo que está sucediendo este viernes por la mañana temprano ante Pilatos. Sus numerosos amigos o bien ignoran todavía su arresto nocturno y su juicio a primeras horas de la mañana, o están escondidos por temor a ser capturados también y condenados a muerte porque creen en las enseñanzas de Jesús. En la multitud que ahora vocifera pidiendo la muerte del Maestro sólo se encuentran sus enemigos declarados y la plebe irreflexiva fácilmente gobernable.

\par 
%\textsuperscript{(1995.1)}
\textsuperscript{185:6.2} Pilatos quería hacer un último llamamiento a la piedad de la gente. Como tenía miedo de desafiar el clamor de este gentío descarriado que gritaba para conseguir la sangre de Jesús, ordenó a los guardias judíos y a los soldados romanos que cogieran a Jesús y lo azotaran. Este modo de proceder era en sí mismo injusto e ilegal, ya que la ley romana estipulaba que únicamente los condenados a morir por crucifixión fueran sometidos así a la flagelación. Los guardias llevaron a Jesús al patio abierto del pretorio para este suplicio. Aunque sus enemigos no presenciaron esta flagelación, Pilatos sí lo hizo, y antes de que terminaran este abuso perverso, ordenó a los azotadores que se detuvieran e indicó que Jesús fuera llevado ante él. Antes de que los azotadores ataran a Jesús al poste de flagelación y lo golpearan con sus látigos de nudos, le pusieron de nuevo el manto de púrpura, trenzaron una corona de espinas y se la colocaron en la frente. Después de poner una caña en su mano simulando un cetro, se arrodillaron delante de él y se burlaron de él, diciendo: <<¡Salud, rey de los judíos!>> Luego le escupieron y lo abofetearon. Antes de devolverlo a Pilatos, uno de ellos le quitó la caña de la mano y lo golpeó con ella en la cabeza.

\par 
%\textsuperscript{(1995.2)}
\textsuperscript{185:6.3} Entonces, Pilatos condujo fuera a este preso sangrante y lacerado, y lo presentó a la variopinta multitud, diciendo: <<¡He aquí al hombre! Os declaro de nuevo que no encuentro ningún delito en él, y después de haberlo azotado, quisiera liberarlo>>.

\par 
%\textsuperscript{(1995.3)}
\textsuperscript{185:6.4} Jesús de Nazaret estaba allí, vestido con un viejo manto de púrpura real, con una corona de espinas que le hería su bondadosa frente. Su rostro estaba manchado de sangre y su cuerpo encorvado de sufrimiento y de pena. Pero nada puede conmover el corazón insensible de aquellos que son víctimas de un intenso odio emocional y esclavos de los prejuicios religiosos. Este espectáculo produjo un poderoso estremecimiento en los reinos de un inmenso universo, pero no enterneció el corazón de los que habían decidido llevar a cabo la destrucción de Jesús.

\par 
%\textsuperscript{(1995.4)}
\textsuperscript{185:6.5} Cuando se hubieron recobrado del primer impacto al ver el estado lastimoso del Maestro, sólo gritaron más fuerte y durante más tiempo: <<¡Crucifícalo! ¡Crucifícalo! ¡Crucifícalo!>>

\par 
%\textsuperscript{(1995.5)}
\textsuperscript{185:6.6} Pilatos comprendió ahora que era inútil apelar a sus supuestos sentimientos de piedad. Se adelantó y dijo: <<Percibo que estáis decididos a que este hombre muera, ¿pero qué ha hecho para merecer la muerte? ¿Quién quiere declarar su crimen?>>

\par 
%\textsuperscript{(1995.6)}
\textsuperscript{185:6.7} Entonces el sumo sacerdote en persona se adelantó, subió hasta Pilatos, y declaró con irritación: <<Tenemos una ley sagrada, y según esa ley este hombre debe morir porque se ha llamado a sí mismo Hijo de Dios>>. Cuando Pilatos escuchó esto, tuvo aún más miedo, no solamente de los judíos, sino que al recordar la nota de su mujer y la mitología griega en la que los dioses descendían a la Tierra, se puso a temblar ante la idea de que Jesús pudiera ser un personaje divino. Hizo señas a la multitud para que se calmara, mientras cogía a Jesús por el brazo y lo conducía de nuevo al interior del edificio para poder interrogarlo otra vez. Pilatos estaba ahora confuso por el miedo, desconcertado por la superstición y abrumado por la actitud testaruda de la muchedumbre.

\section*{7. La última entrevista con Pilatos}
\par 
%\textsuperscript{(1995.7)}
\textsuperscript{185:7.1} Cuando Pilatos, temblando con una temerosa emoción, se sentó al lado de Jesús, le preguntó: <<¿De dónde vienes? ¿Quién eres realmente? ¿Qué es eso que dicen de que eres el Hijo de Dios?>>

\par 
%\textsuperscript{(1996.1)}
\textsuperscript{185:7.2} Pero Jesús difícilmente podía contestar estas preguntas cuando eran efectuadas por un juez débil, vacilante, que temía a los hombres, y que era tan injusto como para hacerlo azotar incluso después de haberlo declarado inocente de todo delito, y antes de haber sido debidamente condenado a muerte. Jesús miró a Pilatos directamente a la cara, pero no le contestó. Entonces dijo Pilatos: <<¿Te niegas a hablarme? ¿No te das cuenta de que aún tengo el poder de liberarte o de crucificarte?>> Entonces Jesús le dijo: <<No podrías tener ningún poder sobre mí si no fuera consentido desde arriba. No podrías ejercer ninguna autoridad sobre el Hijo del Hombre a menos que lo permita el Padre que está en los cielos. Pero no eres tan culpable puesto que ignoras el evangelio. El que me ha traicionado y el que me ha entregado a ti son los que tienen el mayor pecado>>.

\par 
%\textsuperscript{(1996.2)}
\textsuperscript{185:7.3} Esta última conversación con Jesús aterrorizó completamente a Pilatos. Este hombre moralmente cobarde, este juez débil, tenía que luchar ahora contra el doble peso del temor supersticioso a Jesús y del miedo mortal a los dirigentes judíos.

\par 
%\textsuperscript{(1996.3)}
\textsuperscript{185:7.4} Pilatos apareció de nuevo ante el gentío, diciendo: <<Estoy seguro de que este hombre sólo es un delincuente religioso. Deberíais cogerlo y juzgarlo según vuestra ley. ¿Por qué esperáis que yo acceda a que muera porque se ha opuesto a vuestras tradiciones?>>

\par 
%\textsuperscript{(1996.4)}
\textsuperscript{185:7.5} Pilatos estaba casi dispuesto a soltar a Jesús cuando Caifás, el sumo sacerdote, se acercó al cobarde juez romano, agitó un dedo vengativo delante de la cara de Pilatos, y dijo estas palabras irritadas que toda la multitud pudo escuchar: <<Si sueltas a este hombre, no eres amigo del César, y procuraré que el emperador se entere de todo>>. Esta amenaza pública fue demasiado para Pilatos. El temor por sus bienes personales eclipsó ahora cualquier otra consideración, y el cobarde gobernador ordenó que Jesús fuera traído ante el tribunal. Cuando el Maestro estuvo allí delante de ellos, Pilatos lo señaló con el dedo y dijo en tono burlón: <<Aquí está vuestro rey>>. Y los judíos respondieron: <<¡Acaba con él! ¡Crucifícalo!>> Entonces dijo Pilatos, con mucha ironía y sarcasmo: <<¿Voy a crucificar a vuestro rey?>> Y los judíos respondieron: <<Sí, ¡crucifícalo! No tenemos más rey que al César>>. Entonces Pilatos se dio cuenta de que no había ninguna esperanza de salvar a Jesús, puesto que no estaba dispuesto a desafiar a los judíos.

\section*{8. El trágico abandono de Pilatos}
\par 
%\textsuperscript{(1996.5)}
\textsuperscript{185:8.1} Allí estaba el Hijo de Dios, encarnado como Hijo del Hombre. Había sido arrestado sin acusación, acusado sin pruebas, juzgado sin testigos, castigado sin veredicto, y pronto iba a ser condenado a muerte por un juez injusto que había confesado que no podía encontrar ninguna falta en él. Si Pilatos había creído apelar al patriotismo de la gente llamando a Jesús el <<rey de los judíos>>, se había equivocado por completo. Los judíos no esperaban ningún rey de este tipo. La declaración de los jefes de los sacerdotes y los saduceos <<No tenemos más rey que al César>> impactó incluso a la plebe irreflexiva, pero ya era demasiado tarde para salvar a Jesús, aunque el gentío se hubiera atrevido a abrazar la causa del Maestro.

\par 
%\textsuperscript{(1996.6)}
\textsuperscript{185:8.2} Pilatos temía un alboroto o un motín. No se atrevía a arriesgarse a tener este tipo de disturbios durante la época de la Pascua en Jerusalén. Recientemente había recibido una reprimenda del César, y no quería arriesgarse a recibir otra. El gentío aplaudió cuando ordenó que soltaran a Barrabás. Luego ordenó que le trajeran una palangana y un poco de agua, y se lavó las manos allí mismo delante de la multitud, diciendo: <<Soy inocente de la sangre de este hombre. Estáis decididos a que muera, pero no he encontrado ninguna culpa en él. Allá vosotros. Los soldados se lo llevarán>>. Entonces el gentío aplaudió y replicó: <<Que su sangre caiga sobre nosotros y sobre nuestros hijos>>.


\chapter{Documento 186. Poco antes de la crucifixión}
\par 
%\textsuperscript{(1997.1)}
\textsuperscript{186:0.1} CUANDO Jesús y sus acusadores salieron para ver a Herodes, el Maestro se volvió hacia el apóstol Juan y le dijo: <<Juan, ya no puedes hacer nada más por mí. Ve a buscar a mi madre y tráela para que me vea antes de morir>>. Cuando Juan escuchó la petición de su Maestro, aunque no quería dejarlo solo entre sus enemigos, se apresuró a ir a Betania, donde toda la familia de Jesús estaba reunida y esperando en la casa de Marta y María, las hermanas de Lázaro, a quien Jesús había resucitado de entre los muertos.

\par 
%\textsuperscript{(1997.2)}
\textsuperscript{186:0.2} Varias veces durante la mañana, los mensajeros habían llevado a Marta y María las noticias sobre el desarrollo del juicio de Jesús. Pero la familia de Jesús no llegó a Betania hasta unos minutos antes de que llegara Juan trayendo la petición de Jesús de ver a su madre antes de ser ejecutado. Después de que Juan Zebedeo les hubiera contado todo lo que había sucedido desde el arresto de Jesús a medianoche, su madre María partió inmediatamente en compañía de Juan para ver a su hijo mayor. Cuando María y Juan llegaron a la ciudad, Jesús, acompañado por los soldados romanos que iban a crucificarlo, ya había llegado al Gólgota.

\par 
%\textsuperscript{(1997.3)}
\textsuperscript{186:0.3} Cuando María, la madre de Jesús, se marchó con Juan para ver a su hijo, su hermana Rut se negó a quedarse atrás con el resto de la familia. Puesto que estaba decidida a acompañar a su madre, su hermano Judá fue con ella. El resto de la familia del Maestro permaneció en Betania bajo la dirección de Santiago, y prácticamente cada hora, los mensajeros de David Zebedeo les traían noticias sobre el desarrollo del terrible acontecimiento de la ejecución de su hermano mayor, Jesús de Nazaret.

\section*{1. El final de Judas Iscariote}
\par 
%\textsuperscript{(1997.4)}
\textsuperscript{186:1.1} El juicio de Jesús ante Pilatos terminó aproximadamente a las ocho y media de este viernes por la mañana, y el Maestro fue puesto en manos de los soldados romanos que iban a crucificarlo. En cuanto los romanos tomaron posesión de Jesús, el capitán de los guardias judíos regresó con sus hombres a su cuartel general en el templo. El sumo sacerdote y sus asociados sanedristas siguieron de cerca a los guardias, y fueron directamente a su lugar habitual de reunión en la sala de piedras labradas del templo. Allí encontraron a otros muchos miembros del sanedrín que esperaban para saber lo que se había hecho con Jesús. Mientras Caifás presentaba su informe al sanedrín sobre el juicio y la condena de Jesús, Judas apareció ante ellos para reclamar su recompensa por el papel que había jugado en el arresto y la sentencia de muerte de su Maestro.

\par 
%\textsuperscript{(1997.5)}
\textsuperscript{186:1.2} Todos estos judíos detestaban a Judas; sólo miraban al traidor con unos sentimientos de total desprecio. A lo largo de todo el juicio de Jesús ante Caifás y durante su aparición ante Pilatos, a Judas le había remordido la conciencia por su comportamiento traidor. Y también empezaba a perder un poco sus ilusiones sobre la recompensa que iba a recibir como pago por su traición a Jesús. No le gustaba la frialdad y la indiferencia de las autoridades judías; sin embargo, contaba con ser recompensado ampliamente por su cobarde conducta. Preveía que sería llamado ante el pleno del sanedrín y que allí escucharía sus elogios mientras le conferían los honores adecuados como prueba del gran servicio que se vanagloriaba de haber prestado a su nación. Imaginad pues la gran sorpresa de este traidor egoísta cuando un criado del sumo sacerdote le tocó en el hombro, lo llamó para que saliera de la sala y le dijo: <<Judas, me han encargado que te pague por haber traicionado a Jesús. Aquí está tu recompensa>>. Mientras le decía esto, el criado de Caifás le entregó a Judas una bolsa que contenía treinta monedas de plata ---el precio corriente de un buen esclavo sano.

\par 
%\textsuperscript{(1998.1)}
\textsuperscript{186:1.3} Judas se quedó atónito, confundido. Se abalanzó para entrar en la sala, pero el portero se lo impidió. Quería apelar al sanedrín, pero no quisieron recibirlo. Judas no podía creer que estos dirigentes de los judíos le habían permitido traicionar a sus amigos y a su Maestro para ofrecerle después como recompensa treinta monedas de plata. Se sentía humillado, desilusionado y totalmente abrumado. Se alejó del templo, por así decirlo, como enajenado. Como un autómata, dejó caer la bolsa de dinero en su profundo bolsillo, el mismo bolsillo en el que había transportado durante tanto tiempo la bolsa que contenía los fondos apostólicos. Y estuvo vagando por la ciudad detrás de la muchedumbre que se dirigía a presenciar las crucifixiones.

\par 
%\textsuperscript{(1998.2)}
\textsuperscript{186:1.4} Judas vio a lo lejos que levantaban el travesaño donde estaba clavado Jesús; al ver esto, volvió precipitadamente al templo, apartó a la fuerza al portero y se encontró en presencia del sanedrín, que estaba reunido todavía. El traidor estaba casi sin aliento y sumamente desconcertado, pero se las arregló para balbucear estas palabras: <<He pecado porque he traicionado una sangre inocente. Me habéis insultado. Me habéis ofrecido dinero como recompensa por mi servicio ---el precio de un esclavo. Me arrepiento de haber hecho esto; aquí está vuestro dinero. Quiero escapar de la culpabilidad de este acto>>.

\par 
%\textsuperscript{(1998.3)}
\textsuperscript{186:1.5} Cuando los dirigentes de los judíos escucharon a Judas, se mofaron de él. Uno de ellos, que estaba sentado cerca de donde se encontraba Judas, le hizo señas para que se fuera de la sala, y le dijo: <<Tu Maestro ya ha sido ejecutado por los romanos, y en cuanto a tu culpabilidad, ¿qué nos importa a nosotros? Ocúpate tú de ella ---y ¡fuera de aquí!>>

\par 
%\textsuperscript{(1998.4)}
\textsuperscript{186:1.6} Cuando dejó la sala del sanedrín, Judas sacó de la bolsa las treinta monedas de plata y las lanzó al voleo sobre el suelo del templo. Cuando el traidor salió del templo, estaba casi fuera de sí. Judas estaba pasando ahora por la experiencia de comprender la verdadera naturaleza del pecado. Todo el encanto, la fascinación y la embriaguez de las malas acciones se habían desvanecido. Ahora el malhechor se encontraba solo, frente a frente con el veredicto del juicio de su alma desilusionada y decepcionada. El pecado era fascinante y aventurero mientras se cometía, pero ahora había que hacer frente a la cosecha de los hechos desnudos y poco románticos.

\par 
%\textsuperscript{(1998.5)}
\textsuperscript{186:1.7} Este antiguo embajador del reino de los cielos en la Tierra caminaba ahora solo y abandonado por las calles de Jerusalén. Su desesperación era terrible y casi absoluta. Continuó caminando por la ciudad y por fuera de sus muros, hasta descender a la terrible soledad del valle de Hinom, donde subió por las rocas escarpadas; cogió el cinturón de su vestido, ató un extremo a un pequeño árbol, anudó el otro alrededor de su cuello, y se arrojó al precipicio. Antes de morir, el nudo que había atado con sus manos nerviosas se soltó, y el cuerpo del traidor se hizo trizas al caer sobre las rocas puntiagudas de abajo.

\section*{2. La actitud del Maestro}
\par 
%\textsuperscript{(1999.1)}
\textsuperscript{186:2.1} Cuando Jesús fue arrestado, sabía que su trabajo en la Tierra, en la similitud de la carne mortal, había terminado. Comprendía plenamente la clase de muerte que le esperaba, y le preocupaban poco los detalles de sus supuestos juicios.

\par 
%\textsuperscript{(1999.2)}
\textsuperscript{186:2.2} Delante del tribunal de los sanedristas, Jesús rehusó responder al testimonio de los testigos perjuros. Sólo había una pregunta que siempre atraía una respuesta, ya fuera hecha por amigos o enemigos, y era la que se refería a la naturaleza y a la divinidad de su misión en la Tierra. Cuando le preguntaban si era el Hijo de Dios, daba infaliblemente una respuesta. Se negó firmemente a hablar cuanto estuvo en presencia del curioso y malvado Herodes. Delante de Pilatos sólo habló cuando pensó que podría ayudar a Pilatos, o a alguna otra persona, a conocer mejor la verdad mediante lo que él decía. Jesús había enseñado a sus apóstoles que era inútil que echaran sus perlas a los cerdos, y ahora se atrevía a practicar lo que había enseñado. Su conducta en ese momento ejemplificó la paciente sumisión de la naturaleza humana unida al silencio majestuoso y a la solemne dignidad de la naturaleza divina. Estaba enteramente dispuesto a discutir con Pilatos cualquier cuestión relacionada con las acusaciones políticas presentadas contra él ---cualquier cuestión que Jesús reconocía que pertenecía a la jurisdicción del gobernador.

\par 
%\textsuperscript{(1999.3)}
\textsuperscript{186:2.3} Jesús estaba convencido de que la voluntad del Padre era que se sometiera al curso natural y normal de los acontecimientos humanos, tal como las demás criaturas mortales deben hacerlo, y por eso se negó incluso a emplear sus poderes puramente humanos de elocuencia persuasiva para influir sobre el resultado de las maquinaciones de sus semejantes mortales, socialmente miopes y espiritualmente ciegos. Aunque Jesús vivió y murió en Urantia, toda su carrera humana, desde el principio hasta el fin, fue un espectáculo destinado a influir e instruir a todo el universo que había creado y sostenido sin cesar.

\par 
%\textsuperscript{(1999.4)}
\textsuperscript{186:2.4} Estos judíos miopes gritaban de manera indecente pidiendo la muerte del Maestro, mientras éste permanecía allí en un silencio impresionante, contemplando la escena de la muerte de una nación ---el propio pueblo de su padre terrenal.

\par 
%\textsuperscript{(1999.5)}
\textsuperscript{186:2.5} Jesús había adquirido ese tipo de carácter humano que puede conservar la serenidad y afirmar su dignidad en medio de los insultos continuos e injustificados. No se le podía intimidar. Cuando fue atacado en primer lugar por el criado de Anás, sólo había sugerido la conveniencia de llamar a unos testigos que pudieran testificar debidamente contra él.

\par 
%\textsuperscript{(1999.6)}
\textsuperscript{186:2.6} Desde el principio hasta el fin de su supuesto juicio ante Pilatos, las huestes celestiales que presenciaban los acontecimientos no pudieron abstenerse de transmitir al universo la descripción de la escena de <<Pilatos procesado ante Jesús>>.

\par 
%\textsuperscript{(1999.7)}
\textsuperscript{186:2.7} Cuando compareció ante Caifás y todos los testimonios perjuros se habían derrumbado, Jesús no dudó en responder a la pregunta del sumo sacerdote, proporcionando así con su propio testimonio la base que deseaban para condenarlo por blasfemia.

\par 
%\textsuperscript{(1999.8)}
\textsuperscript{186:2.8} El Maestro nunca manifestó el menor interés por los esfuerzos bien intencionados, pero poco entusiastas, de Pilatos por conseguir su liberación. Compadecía realmente a Pilatos y se esforzó sinceramente por iluminar su mente ensombrecida. Permaneció enteramente pasivo ante todos los llamamientos del gobernador romano para que los judíos retiraran sus acusaciones criminales contra él. Durante toda esta penosa prueba, se comportó con una dignidad sencilla y una majestad sin ostentación. Ni siquiera quiso criticar la insinceridad de sus supuestos asesinos cuando éstos le preguntaron si era <<el rey de los judíos>>. Aceptó esta designación con un mínimo de explicación modificativa, sabiendo que aunque habían escogido rechazarlo, sería el último en representar para ellos un verdadero jefe nacional, incluso en el sentido espiritual.

\par 
%\textsuperscript{(2000.1)}
\textsuperscript{186:2.9} Jesús habló poco durante estos juicios, pero dijo lo suficiente como para mostrar a todos los mortales el tipo de carácter humano que un hombre puede perfeccionar en asociación con Dios, y para revelar a todo el universo la manera en que Dios se puede manifestar en la vida de la criatura cuando ésta escoge verdaderamente hacer la voluntad del Padre, volviéndose así un hijo activo del Dios vivo.

\par 
%\textsuperscript{(2000.2)}
\textsuperscript{186:2.10} Su amor por los mortales ignorantes se revela plenamente mediante su paciencia y su gran serenidad frente a las burlas, los golpes y las bofetadas de los toscos soldados y de los criados irreflexivos. Ni siquiera se irritó cuando le vendaron los ojos y le golpearon burlonamente en la cara, exclamando: <<Profetiza y dinos quién te ha golpeado>>.

\par 
%\textsuperscript{(2000.3)}
\textsuperscript{186:2.11} Pilatos estaba más cerca de la verdad de lo que podía suponer cuando, después de haber hecho azotar a Jesús, lo presentó ante la multitud exclamando: <<¡He aquí al hombre!>> En verdad, el temeroso gobernador romano poco podía imaginar que en aquel mismo momento el universo permanecía atento, contemplando esta escena única en la que su amado Soberano se sometía así a la humillación de las burlas y los golpes de sus súbditos mortales ignorantes y envilecidos. Y cuando Pilatos habló, la frase <<¡He aquí a Dios y al Hombre!>> resonó por todo Nebadon. Desde ese día, incalculables millones de criaturas han continuado contemplando a este hombre en todo un universo, mientras que el Dios de Havona, el gobernante supremo del universo de universos, acepta al hombre de Nazaret como que satisface el ideal de las criaturas mortales de este universo local del tiempo y del espacio. En su vida incomparable, Jesús no dejó nunca de revelar Dios al hombre. Ahora, durante estos episodios finales de su carrera mortal y de su muerte posterior, efectuó una nueva y conmovedora revelación del hombre a Dios.

\section*{3. El fiable David Zebedeo}
\par 
%\textsuperscript{(2000.4)}
\textsuperscript{186:3.1} Poco después de que Jesús fuera entregado a los soldados romanos al final de la audiencia ante Pilatos, un destacamento de guardias del templo se dirigió apresuradamente a Getsemaní para dispersar o arrestar a los seguidores del Maestro. Pero mucho antes de que llegaran, estos seguidores se habían dispersado. Los apóstoles se habían retirado a unos escondites designados; los griegos se habían separado y dirigido a diversas casas de Jerusalén; los demás discípulos habían desaparecido igualmente. David Zebedeo creía que los enemigos de Jesús regresarían, de manera que trasladó enseguida unas cinco o seis tiendas hacia la parte alta de la hondonada, cerca del lugar donde el Maestro se retiraba tan a menudo para orar y adorar. Tenía la intención de ocultarse aquí y de mantener al mismo tiempo un centro, o estación coordinadora, para su servicio de mensajeros. David apenas había abandonado el campamento cuando llegaron los guardias del templo. Como no encontraron a nadie allí, se contentaron con incendiar el campamento y luego regresaron apresuradamente al templo. Al escuchar el informe de los guardias, el sanedrín se convenció de que los seguidores de Jesús estaban tan asustados y sumisos, que ya no habría ningún peligro de motín o de cualquier intento por rescatar a Jesús de las manos de sus verdugos. Por fin podían respirar tranquilos; así pues levantaron la sesión y cada cual se fue por su lado para prepararse para la Pascua.

\par 
%\textsuperscript{(2000.5)}
\textsuperscript{186:3.2} Tan pronto como Pilatos entregó a Jesús a los soldados romanos para que lo crucificaran, un mensajero salió precipitadamente hacia Getsemaní para informar a David, y en menos de cinco minutos ya habían partido los corredores hacia Betsaida, Pella, Filadelfia, Sidón, Siquem, Hebrón, Damasco y Alejandría. Estos mensajeros llevaban la noticia de que Jesús estaba a punto de ser crucificado por los romanos a instancias insistentes de los dirigentes de los judíos.

\par 
%\textsuperscript{(2001.1)}
\textsuperscript{186:3.3} A lo largo de todo este trágico día, y hasta que salió el último mensaje indicando que el Maestro había sido colocado en el sepulcro, David envió a los mensajeros aproximadamente cada media hora con informes para los apóstoles, los griegos y la familia terrenal de Jesús, que estaba reunida en la casa de Lázaro en Betania. Cuando los mensajeros partieron con la noticia de que Jesús había sido sepultado, David despidió a su cuerpo de corredores locales para que celebraran la Pascua y descansaran el sábado que se avecinaba, dándoles instrucciones para que comparecieran discretamente ante él el domingo por la mañana en la casa de Nicodemo, donde tenía la intención de esconderse algunos días con Andrés y Simón Pedro.

\par 
%\textsuperscript{(2001.2)}
\textsuperscript{186:3.4} Este David Zebedeo, con su manera de pensar tan peculiar, era el único de los principales discípulos de Jesús que se sentía inclinado a tomar al pie de la letra y como un hecho verdadero la afirmación del Maestro de que moriría y <<resucitaría al tercer día>>. David le había escuchado una vez hacer esta predicción, y como tenía la inclinación de tomarse las cosas en sentido literal, ahora se proponía reunir a sus mensajeros el domingo por la mañana temprano en la casa de Nicodemo a fin de tenerlos cerca para difundir la noticia, en el caso de que Jesús resucitara de entre los muertos. David descubrió enseguida que ninguno de los seguidores de Jesús esperaba que regresara tan pronto de la tumba; por eso habló poco sobre su convicción, y no dijo que había movilizado a todo su ejército de mensajeros para el domingo por la mañana temprano, excepto a los corredores que habían sido enviados el viernes por la mañana a las ciudades lejanas y a los centros de creyentes.

\par 
%\textsuperscript{(2001.3)}
\textsuperscript{186:3.5} Y así, estos seguidores de Jesús, dispersos por todo Jerusalén y sus alrededores, compartieron la Pascua aquella noche, y al día siguiente permanecieron recluídos.

\section*{4. Los preparativos para la crucifixión}
\par 
%\textsuperscript{(2001.4)}
\textsuperscript{186:4.1} Después de que Pilatos se hubiera lavado las manos delante de la multitud, tratando así de escapar a la culpabilidad de haber entregado a un hombre inocente a la crucifixión simplemente porque temía resistirse al clamor de los dirigentes de los judíos, ordenó que el Maestro fuera entregado a los soldados romanos y dio instrucciones a su capitán para que fuera crucificado inmediatamente. Al hacerse cargo de Jesús, los soldados lo llevaron de nuevo al patio del pretorio, y después de quitarle el manto que Herodes le había puesto, lo vistieron con su propia ropa. Estos soldados se burlaron y se mofaron de él, pero no le infligieron nuevos castigos físicos. Jesús estaba ahora solo con estos soldados romanos. Sus amigos estaban escondidos, sus enemigos se habían ido por su camino, e incluso Juan Zebedeo ya no estaba a su lado.

\par 
%\textsuperscript{(2001.5)}
\textsuperscript{186:4.2} Pilatos entregó a Jesús a los soldados poco después de las ocho de la mañana, y poco antes de las nueve partieron para el lugar de la crucifixión. Durante este intervalo de más de media hora, Jesús no dijo ni una sola palabra. La actividad ejecutiva de un gran universo estaba prácticamente detenida. Gabriel y los principales dirigentes de Nebadon o bien se encontraban reunidos aquí en Urantia, o prestaban mucha atención a los informes espaciales de los arcángeles, en un esfuerzo por mantenerse informados de lo que le estaba sucediendo al Hijo del Hombre en Urantia.

\par 
%\textsuperscript{(2001.6)}
\textsuperscript{186:4.3} Cuando los soldados estuvieron preparados para salir con Jesús hacia el Gólgota, habían empezado a sentirse impresionados por su insólita serenidad y su dignidad extraordinaria, por su silencio sin queja.

\par 
%\textsuperscript{(2001.7)}
\textsuperscript{186:4.4} Una gran parte del retraso en partir con Jesús para el lugar de la crucifixión se debió a que el capitán decidió, a última hora, llevarse consigo a dos ladrones que habían sido condenados a muerte; puesto que Jesús iba a ser crucificado aquella mañana, el capitán romano pensó que estos dos también podían morir con él, en lugar de esperar hasta el fin de las festividades de la Pascua.

\par 
%\textsuperscript{(2002.1)}
\textsuperscript{186:4.5} En cuanto prepararon a los ladrones, fueron conducidos al patio, donde uno de ellos contempló a Jesús por primera vez, pero el otro le había oído hablar a menudo tanto en el templo como muchos meses antes en el campamento de Pella.

\section*{5. Relación entre la muerte de Jesús y la Pascua}
\par 
%\textsuperscript{(2002.2)}
\textsuperscript{186:5.1} No existe una relación directa entre la muerte de Jesús y la Pascua judía. Es verdad que el Maestro entregó su vida carnal en este día, el día de la preparación de la Pascua judía, y alrededor de la hora en que se sacrificaban los corderos pascuales en el templo. Pero la coincidencia de estos acontecimientos no indica de ninguna manera que la muerte del Hijo del Hombre en la Tierra tenga alguna relación con el sistema de los sacrificios judío. Jesús era judío pero, como Hijo del Hombre, era un mortal de los reinos. Los acontecimientos ya narrados, que condujeron a este momento en que el Maestro iba a ser crucificado de manera inminente, son suficientes para indicar que su muerte, que se produjo aproximadamente a esta hora, fue un asunto puramente natural y manejado por los hombres.

\par 
%\textsuperscript{(2002.3)}
\textsuperscript{186:5.2} Fue el hombre, y no Dios, el que planeó y ejecutó la muerte de Jesús en la cruz. Es verdad que el Padre se negó a entremeterse en la marcha de los acontecimientos humanos en Urantia, pero el Padre Paradisiaco no decretó, pidió ni exigió la muerte de su Hijo tal como se llevó a cabo en la Tierra. Es un hecho que, tarde o temprano y de alguna manera, Jesús habría tenido que despojarse de su cuerpo mortal, poniendo fin a su encarnación, pero podría haberlo hecho de muchas formas, sin tener que morir en una cruz entre dos ladrones. Todo esto fue obra del hombre, y no de Dios.

\par 
%\textsuperscript{(2002.4)}
\textsuperscript{186:5.3} En la época de su bautismo, el Maestro ya había completado la parte técnica de la experiencia requerida en la Tierra y en la carne, necesaria para finalizar su séptima y última donación en el universo. En aquel mismo momento, Jesús había realizado su deber en la Tierra. Toda la vida que vivió de allí en adelante, e incluso la manera en que murió, fueron un ministerio puramente personal por su parte por el bienestar y la elevación de sus criaturas mortales en este mundo y en otros mundos.

\par 
%\textsuperscript{(2002.5)}
\textsuperscript{186:5.4} El evangelio de la buena nueva de que el hombre mortal puede, por la fe, volverse espiritualmente consciente de que es hijo de Dios, no depende de la muerte de Jesús. Es verdad, en efecto, que todo este evangelio del reino ha sido enormemente iluminado por la muerte del Maestro, pero lo fue aun más por su vida.

\par 
%\textsuperscript{(2002.6)}
\textsuperscript{186:5.5} Todo lo que el Hijo del Hombre dijo o hizo en la Tierra embelleció enormemente las doctrinas de la filiación con Dios y de la fraternidad entre los hombres, pero estas relaciones esenciales entre Dios y los hombres son inherentes a los hechos universales del amor de Dios por sus criaturas y de la misericordia innata de los Hijos divinos. Estas relaciones conmovedoras y divinamente hermosas entre el hombre y su Hacedor, en este mundo y en todos los demás, en todo el universo de universos, han existido desde la eternidad; y no dependen en ningún sentido de las actuaciones donadoras periódicas de los Hijos Creadores de Dios, que asumen así la naturaleza y la semejanza de las inteligencias creadas por ellos, como una parte del precio que han de pagar para adquirir finalmente la soberanía ilimitada sobre sus universos locales respectivos.

\par 
%\textsuperscript{(2002.7)}
\textsuperscript{186:5.6} Antes de la vida y la muerte de Jesús en Urantia, el Padre que está en los cielos amaba al hombre mortal de la Tierra tanto como lo ama después de esta manifestación trascendente de la asociación entre el hombre y Dios. Esta poderosa operación de la encarnación del Dios de Nebadon como hombre en Urantia no podía aumentar los atributos del Padre eterno, infinito y universal, pero sí enriqueció e iluminó a todos los demás administradores y criaturas del universo de Nebadon. Aunque el Padre que está en los cielos no nos ama más debido a esta donación de Miguel, todas las demás inteligencias celestiales sí lo hacen. Y esto es así porque Jesús no solamente hizo una revelación de Dios al hombre, sino que también hizo una nueva revelación del hombre a los Dioses y a las inteligencias celestiales del universo de universos.

\par 
%\textsuperscript{(2003.1)}
\textsuperscript{186:5.7} Jesús no está a punto de morir como sacrificio por el pecado. No va a expiar la culpabilidad moral innata de la raza humana. La humanidad no tiene esta culpabilidad racial ante Dios. La culpabilidad es estrictamente una cuestión de pecado personal y de rebelión consciente y deliberada contra la voluntad del Padre y la administración de sus Hijos.

\par 
%\textsuperscript{(2003.2)}
\textsuperscript{186:5.8} El pecado y la rebelión no tienen nada que ver con el plan fundamental de donación de los Hijos Paradisiacos de Dios, aunque nos parezca que el plan de salvación es una característica provisional del plan de donación.

\par 
%\textsuperscript{(2003.3)}
\textsuperscript{186:5.9} La salvación de Dios para los mortales de Urantia habría sido exactamente igual de eficaz e infaliblemente segura si Jesús no hubiera sido ejecutado por las manos crueles de unos mortales ignorantes. Si el Maestro hubiera sido recibido favorablemente por los mortales de la Tierra y si hubiera partido de Urantia abandonando voluntariamente su vida en la carne, el hecho del amor de Dios y de la misericordia del Hijo ---el hecho de la filiación con Dios--- no hubiera sido afectado de ninguna manera. Vosotros los mortales sois los hijos de Dios, y para que esta verdad se convierta en un hecho en vuestra experiencia personal, sólo se os pide una cosa: vuestra fe nacida del espíritu.


\chapter{Documento 187. La crucifixión}
\par 
%\textsuperscript{(2004.1)}
\textsuperscript{187:0.1} DESPUÉS de que los dos bandidos hubieran sido preparados, los soldados partieron, bajo el mando de un centurión, para el lugar de la crucifixión. El centurión que estaba a cargo de estos doce soldados era el mismo capitán que había dirigido a los soldados romanos la noche anterior para arrestar a Jesús en Getsemaní. Los romanos tenían la costumbre de asignar cuatro soldados a cada persona que iba a ser crucificada. Los dos bandidos fueron debidamente azotados antes de llevarselos para ser crucificados, pero Jesús no recibió ningún castigo físico adicional; el capitán pensó sin duda que ya había sido suficientemente azotado, incluso antes de ser condenado.

\par 
%\textsuperscript{(2004.2)}
\textsuperscript{187:0.2} Los dos ladrones crucificados con Jesús eran cómplices de Barrabás y habrían sido ejecutados más tarde con su jefe si éste no hubiera sido liberado gracias al indulto pascual de Pilatos. Jesús fue pues crucificado en el lugar de Barrabás.

\par 
%\textsuperscript{(2004.3)}
\textsuperscript{187:0.3} Lo que Jesús está ahora a punto a hacer, sometiéndose a la muerte en la cruz, lo hace por su propio libre albedrío. Al predecir esta experiencia, había dicho: <<El Padre me ama y me sostiene porque estoy dispuesto a entregar mi vida. Pero la recuperaré de nuevo. Nadie me quita la vida ---la entrego por mí mismo. Tengo autoridad para entregarla, y tengo autoridad para recuperarla. Este poder lo he recibido de mi Padre>>.

\par 
%\textsuperscript{(2004.4)}
\textsuperscript{187:0.4} Justo antes de las nueve de esta mañana, los soldados salieron del pretorio con Jesús camino del Gólgota. Mucha gente que los seguía simpatizaba en secreto con Jesús, pero la mayor parte de este grupo de doscientas personas o más estaba compuesto por sus enemigos y por holgazanes curiosos que simplemente deseaban disfrutar del horror de presenciar las crucifixiones. Sólo algunos dirigentes judíos fueron a ver a Jesús morir en la cruz. Sabiendo que Pilatos lo había entregado a los soldados romanos y que estaba condenado a muerte, los demás se ocuparon de su reunión en el templo, donde discutieron qué iban a hacer con sus discípulos.

\section*{1. Camino del Gólgota}
\par 
%\textsuperscript{(2004.5)}
\textsuperscript{187:1.1} Antes de salir del patio del pretorio, los soldados colocaron el travesaño de la cruz sobre los hombros de Jesús. Era costumbre obligar al condenado a que llevara el travesaño de la cruz hasta el lugar de la crucifixión. El condenado no llevaba toda la cruz, sino únicamente este madero más corto. Los postes de madera más largos y verticales de las tres cruces ya habían sido transportados al Gólgota y, cuando llegaron los soldados con sus presos, estaban firmemente hincados en el suelo.

\par 
%\textsuperscript{(2004.6)}
\textsuperscript{187:1.2} De acuerdo con la costumbre, el capitán dirigió la procesión, llevando unas pequeñas tablillas blancas en las que se habían escrito con carbón los nombres de los criminales y la naturaleza de los crímenes por los que habían sido condenados. Para los dos ladrones, el centurión llevaba unos letreros con sus nombres, debajo de los cuales estaba escrita una sola palabra: <<Bandido>>. Después de que la víctima había sido clavada en el travesaño y levantada hasta su sitio en el poste vertical, tenían la costumbre de clavar este letrero en el extremo superior de la cruz, justo encima de la cabeza del criminal, para que todos los espectadores pudieran saber por qué crimen se crucificaba al condenado. La inscripción que llevaba el centurión para colocarla en la cruz de Jesús había sido escrita por el mismo Pilatos en latín, griego y arameo, y decía: <<Jesús de Nazaret ---el rey de los judíos>>.

\par 
%\textsuperscript{(2005.1)}
\textsuperscript{187:1.3} Algunas autoridades judías que aún estaban presentes cuando Pilatos escribió esta inscripción protestaron enérgicamente porque se calificara a Jesús de <<rey de los judíos>>. Pero Pilatos les recordó que esta acusación formaba parte de los cargos que habían llevado a condenarlo. Cuando vieron que no podían convencer a Pilatos para que cambiara de idea, los judíos le rogaron que al menos modificara la inscripción para que dijera: <<Él ha dicho: `yo soy el rey de los judíos'>>. Pero Pilatos se mantuvo inflexible y no quiso cambiar el letrero. A todas sus nuevas súplicas se limitó a contestar: <<Lo que he escrito, escrito está>>.

\par 
%\textsuperscript{(2005.2)}
\textsuperscript{187:1.4} Normalmente se tenía la costumbre de ir hasta el Gólgota por el camino más largo, para que un gran número de personas pudiera ver al criminal condenado, pero este día se dirigieron por el camino más directo hasta la puerta de Damasco, por donde se salía de la ciudad hacia el norte, y siguiendo esta carretera, pronto llegaron al Gólgota, el lugar oficial de las crucifixiones en Jerusalén. Más allá del Gólgota se encontraban las villas de los ricos, y al otro lado de la carretera estaban las tumbas de muchos judíos acaudalados.

\par 
%\textsuperscript{(2005.3)}
\textsuperscript{187:1.5} La crucifixión no era una forma de castigo judío. Tanto los griegos como los romanos habían aprendido este método de ejecución de los fenicios. Incluso Herodes, con toda su crueldad, no recurría a la crucifixión. Los romanos nunca crucificaban a un ciudadano romano; sólo sometían a este tipo de muerte deshonrosa a los esclavos y a los pueblos sometidos. Durante el asedio de Jerusalén, exactamente cuarenta años después de la crucifixión de Jesús, todo el Gólgota estuvo cubierto de miles y miles de cruces sobre las que pereció, día tras día, la flor de la raza judía. Fue en verdad una cosecha terrible por la semilla que se sembró este día.

\par 
%\textsuperscript{(2005.4)}
\textsuperscript{187:1.6} Mientras la procesión de la muerte pasaba por las estrechas calles de Jerusalén, muchas mujeres judías tiernas de corazón que habían escuchado las palabras de ánimo y de compasión de Jesús, y que conocían su vida de ministerio amoroso, no pudieron contener el llanto cuando vieron que lo llevaban a una muerte tan indigna. Mientras pasaba, muchas de estas mujeres lloraban y se lamentaban. Cuando algunas de ellas se atrevieron incluso a caminar a su lado, el Maestro volvió la cabeza hacia ellas y les dijo: <<Hijas de Jerusalén, no lloréis por mí, sino más bien por vosotras y por vuestros hijos. Mi obra está a punto de terminar ---pronto iré hacia mi Padre--- pero los tiempos de las terribles tribulaciones para Jerusalén acaban de empezar. Mirad, se acercan los días en que diréis: Bienaventuradas las estériles y aquellas cuyos pechos nunca han amamantado a sus pequeños. En esos días rogaréis a las rocas de las colinas que caigan sobre vosotras para que os liberen de los terrores de vuestras tribulaciones>>.

\par 
%\textsuperscript{(2005.5)}
\textsuperscript{187:1.7} Estas mujeres de Jerusalén eran en verdad valientes al manifestar su simpatía por Jesús, porque la ley prohibía estrictamente que se mostraran sentimientos amistosos por alguien que iba a ser crucificado. El populacho tenía permiso para burlarse, mofarse y ridiculizar al condenado, pero no estaba permitido expresar ninguna simpatía. Aunque Jesús apreciaba estas manifestaciones de simpatía en esta hora sombría en la que sus amigos estaban escondidos, no quería que estas mujeres de buen corazón se granjearan la indignación de las autoridades por atreverse a mostrar compasión por él. Incluso en un momento como éste, Jesús pensaba poco en sí mismo; sólo pensaba en los terribles días de tragedia que le esperaban a Jerusalén y a toda la nación judía.

\par 
%\textsuperscript{(2006.1)}
\textsuperscript{187:1.8} Mientras el Maestro caminaba con dificultad hacia la crucifixión, se sintió muy cansado; estaba casi agotado. No había comido ni bebido desde la Última Cena en la casa de Elías Marcos, y tampoco le habían permitido disfrutar de un instante de sueño. Además, había soportado un interrogatorio tras otro hasta el momento de su condena, sin mencionar los azotes abusivos con el sufrimiento físico y la pérdida de sangre consiguientes. A todo esto había que añadir su extremada angustia mental, su aguda tensión espiritual y un terrible sentimiento de soledad humana.

\par 
%\textsuperscript{(2006.2)}
\textsuperscript{187:1.9} Poco después de pasar por la puerta que conducía fuera de la ciudad, mientras Jesús se tambaleaba llevando el travesaño de la cruz, su fuerza física flaqueó por un momento y cayó bajo el peso de su pesada carga. Los soldados le gritaron y le dieron patadas, pero no pudo levantarse. Cuando el capitán vio esto, sabiendo lo que Jesús ya había soportado, ordenó a los soldados que se detuvieran. Luego le ordenó a un transeúnte, un tal Simón de Cirene, que cogiera el travesaño de los hombros de Jesús y le obligó a llevarlo durante el resto del camino hasta el Gólgota.

\par 
%\textsuperscript{(2006.3)}
\textsuperscript{187:1.10} Este Simón había efectuado todo el trayecto desde Cirene, en el norte de África, para asistir a la Pascua. Estaba alojado con otros cireneos justo fuera de los muros de la ciudad, y se dirigía hacia el templo para asistir a los oficios en la ciudad cuando el capitán romano le ordenó que llevara el travesaño de Jesús. Simón permaneció allí durante las horas que el Maestro tardó en morir en la cruz, hablando con muchos amigos de Jesús y con sus enemigos. Después de la resurrección y antes de marcharse de Jerusalén, se convirtió en un valeroso creyente en el evangelio del reino, y cuando regresó a su hogar, hizo entrar a su familia en el reino celestial. Sus dos hijos, Alejandro y Rufo, fueron unos instructores muy eficaces del nuevo evangelio en África. Pero Simón no supo nunca que Jesús, cuya carga había llevado, y el preceptor judío que en otro tiempo había favorecido a su hijo lesionado, eran la misma persona.

\par 
%\textsuperscript{(2006.4)}
\textsuperscript{187:1.11} Eran poco más de las nueve cuando esta procesión de la muerte llegó al Gólgota, y los soldados romanos se pusieron a la tarea de clavar a los dos bandidos y al Hijo del Hombre en sus cruces respectivas.

\section*{2. La crucifixión}
\par 
%\textsuperscript{(2006.5)}
\textsuperscript{187:2.1} Los soldados ataron primero los brazos del Maestro al travesaño con unas cuerdas, y luego clavaron sus manos a la madera. Después de haber izado este travesaño en el poste, y de haberlo clavado firmemente en el brazo vertical de la cruz, ataron y clavaron los pies de Jesús a la madera, utilizando un largo clavo para atravesar los dos pies. El poste vertical tenía un gran taco, colocado a la altura adecuada, que servía como una especie de sillín para sostener el peso del cuerpo. La cruz no era alta, y los pies del Maestro se encontraban aproximadamente a sólo un metro del suelo. Por eso podía escuchar todo lo que se decía de él con irrisión, y podía ver claramente la expresión de los rostros de todos los que se mofaban de él con tanta desconsideración. Los que estaban presentes también podían escuchar fácilmente todo lo que Jesús dijo durante estas horas de tortura prolongada y de muerte lenta.

\par 
%\textsuperscript{(2007.1)}
\textsuperscript{187:2.2} Se tenía la costumbre de quitarle toda la ropa a los que iban a ser crucificados, pero como los judíos ponían grandes objeciones a que se expusiera públicamente el cuerpo humano desnudo, los romanos siempre proporcionaban un taparrabos adecuado a todas las personas que se crucificaban en Jerusalén. En consecuencia, después de haberle quitado la ropa a Jesús, lo vistieron de esta manera antes de colocarlo en la cruz.

\par 
%\textsuperscript{(2007.2)}
\textsuperscript{187:2.3} Se recurría a la crucifixión para infligir un castigo cruel y prolongado, pues la víctima a veces tardaba varios días en morir. Había en Jerusalén una importante oposición a la crucifixión, y existía una asociación de mujeres judías que siempre enviaba a una representante a las crucifixiones, con el fin de ofrecerle a la víctima un vino mezclado con drogas para disminuir sus sufrimientos. Pero cuando Jesús probó este vino narcotizado, a pesar de la sed que tenía, se negó a beberlo. El Maestro escogió conservar su conciencia humana hasta el instante final. Deseaba enfrentarse a la muerte, incluso de esta manera cruel e inhumana, y vencerla sometiéndose voluntariamente a la plena experiencia humana.

\par 
%\textsuperscript{(2007.3)}
\textsuperscript{187:2.4} Antes de que Jesús fuera colocado en su cruz, los dos bandidos ya habían sido situados en las suyas, maldiciendo y escupiendo continuamente a sus verdugos. Las únicas palabras de Jesús mientras lo clavaban en el travesaño fueron: <<Padre, perdónalos porque no saben lo que hacen>>. No podría haber intercedido con tanto amor y misericordia a favor de sus verdugos, si estos pensamientos de devoción afectuosa no hubieran sido el móvil principal de toda su vida de servicio desinteresado. Las ideas, los móviles y los anhelos de toda una vida se revelan abiertamente en una crisis.

\par 
%\textsuperscript{(2007.4)}
\textsuperscript{187:2.5} Después de que el Maestro fuera izado en la cruz, el capitán clavó el letrero por encima de su cabeza, y se podía leer en tres idiomas: <<Jesús de Nazaret ---el rey de los Judíos>>. Los judíos estaban furiosos por este supuesto insulto. Pero los modales irrespetuosos de los judíos habían enfadado a Pilatos; sentía que había sido intimidado y humillado, y adoptó este método para obtener una mezquina venganza. Podría haber escrito: <<Jesús, un rebelde>>. Pero sabía muy bien que estos judíos de Jerusalén detestaban el nombre mismo de Nazaret, y estaba decidido a humillarlos de esta manera. Sabía que también se sentirían heridos en lo más vivo al ver que este galileo ejecutado era llamado <<el rey de los judíos>>.

\par 
%\textsuperscript{(2007.5)}
\textsuperscript{187:2.6} Muchos dirigentes judíos, cuando se enteraron de cómo Pilatos había intentado ridiculizarlos poniendo esta inscripción en la cruz de Jesús, se apresuraron a ir al Gólgota, pero no se atrevieron a quitar el letrero porque los soldados romanos estaban vigilando. Como no pudieron quitar el rótulo, estos dirigentes se mezclaron con la multitud e hicieron todo lo posible por incitarla a la burla y a la irrisión, a fin de que nadie se tomara en serio la inscripción.

\par 
%\textsuperscript{(2007.6)}
\textsuperscript{187:2.7} El apóstol Juan, con María la madre de Jesús, Rut y Judá, llegaron a la escena poco después de que Jesús hubiera sido izado a su posición en la cruz, y justo cuando el capitán estaba clavando el letrero por encima de la cabeza del Maestro. Juan fue el único de los once apóstoles que presenció la crucifixión, e incluso él no estuvo presente todo el tiempo, puesto que corrió a Jerusalén para traer a su madre y a las amigas de ésta poco después de haber llevado al Gólgota a la madre de Jesús.

\par 
%\textsuperscript{(2007.7)}
\textsuperscript{187:2.8} Cuando Jesús vio a su madre, junto con Juan, su hermano y su hermana, sonrió pero no dijo nada. Mientras tanto, los cuatro soldados asignados a la crucifixión del Maestro se habían repartido, como era costumbre, sus vestidos entre ellos; uno había cogido las sandalias, otro el turbante, otro el cinturón y el cuarto el manto. Sólo quedaba la túnica, el vestido sin costuras que llegaba hasta cerca de las rodillas, que iba a ser cortada en cuatro pedazos, pero cuando los soldados vieron que se trataba de una prenda tan insólita, decidieron echarla a suertes. Jesús los miraba desde arriba mientras se repartían sus vestiduras y la multitud irreflexiva se burlaba de él.

\par 
%\textsuperscript{(2008.1)}
\textsuperscript{187:2.9} Fue una suerte que los soldados romanos se apropiaran de las ropas del Maestro. De lo contrario, si sus seguidores hubieran conseguido estas vestimentas, hubieran tenido la tentación de utilizar estas reliquias para adorarlas de manera supersticiosa. El Maestro deseaba que sus seguidores no tuvieran ningún objeto material que pudiera asociarse con su vida en la Tierra. Quería dejar a la humanidad únicamente el recuerdo de una vida humana dedicada al alto ideal espiritual de estar consagrado a hacer la voluntad del Padre.

\section*{3. Los que vieron la crucifixión}
\par 
%\textsuperscript{(2008.2)}
\textsuperscript{187:3.1} Hacia las nueve y media de este viernes por la mañana, Jesús fue suspendido en la cruz. Antes de las once, más de mil personas se habían reunido para presenciar este espectáculo de la crucifixión del Hijo del Hombre. Durante estas horas espantosas, las huestes invisibles de un universo permanecieron en silencio mientras contemplaban este fenómeno extraordinario en el que el Creador estaba experimentando la muerte de la criatura, incluso la muerte más indigna de un criminal condenado.

\par 
%\textsuperscript{(2008.3)}
\textsuperscript{187:3.2} Las personas que permanecieron cerca de la cruz en un momento u otro de la crucifixión fueron: María, Rut, Judá, Juan, Salomé (la madre de Juan) y un grupo de fervorosas creyentes que incluía a María (la mujer de Clopas y hermana de la madre de Jesús), María Magdalena y Rebeca, que en otro tiempo había vivido en Séforis. Estos y otros amigos de Jesús guardaron silencio mientras presenciaban su gran paciencia y entereza, y contemplaban sus intensos sufrimientos.

\par 
%\textsuperscript{(2008.4)}
\textsuperscript{187:3.3} Muchos de los que pasaban por allí movían la cabeza y se burlaban de él diciendo: <<Tú que querías destruir el templo y reconstruirlo en tres días, sálvate a ti mismo. Si eres el Hijo de Dios, ¿por qué no bajas de tu cruz?>> De la misma manera, algunos dirigentes judíos se mofaban de él diciendo: <<Ha salvado a otros, pero no puede salvarse a sí mismo>>. Otros decían: <<Si eres el rey de los judíos, bájate de la cruz y creeremos en ti>>. Y más tarde se burlaron aun más de él, diciendo: <<Confiaba en que Dios lo liberaría. Incluso pretendía ser el Hijo de Dios ---miradlo ahora--- crucificado entre dos ladrones>>. Incluso los dos ladrones también se burlaron de él y lo llenaron de reproches.

\par 
%\textsuperscript{(2008.5)}
\textsuperscript{187:3.4} En vista de que Jesús no quería responder a sus insultos, y puesto que se acercaba la hora del mediodía de este día especial de preparación, la mayor parte de la multitud burlona y bromista se había ido a las once y media; menos de cincuenta personas permanecieron en el lugar. Los soldados se prepararon ahora para comer y beber su vino agrio y barato mientras se instalaban para el largo velatorio. Mientras compartían su vino, brindaron irrisoriamente a la salud de Jesús, diciendo: <<¡Salud y buena suerte al rey de los judíos!>> Y se quedaron sorprendidos de la mirada tolerante del Maestro ante sus burlas y mofas.

\par 
%\textsuperscript{(2008.6)}
\textsuperscript{187:3.5} Cuando Jesús los vio comer y beber, bajó la mirada hacia ellos y dijo: <<Tengo sed>>. Cuando el capitán de la guardia oyó decir a Jesús <<tengo sed>>, cogió un poco de vino de su botella, puso el tapón esponjoso empapado en la punta de una jabalina y lo levantó hasta Jesús para que pudiera humedecer sus labios resecos.

\par 
%\textsuperscript{(2008.7)}
\textsuperscript{187:3.6} Jesús había decidido vivir sin recurrir a sus poderes sobrenaturales, y del mismo modo escogió morir en la cruz como un mortal común y corriente. Había vivido como un hombre y quería morir como un hombre ---haciendo la voluntad del Padre.

\section*{4. El ladrón en la cruz}
\par 
%\textsuperscript{(2008.8)}
\textsuperscript{187:4.1} Uno de los bandidos recriminó a Jesús diciendo: <<Si eres el Hijo de Dios, ¿por qué no te salvas a ti mismo y nos salvas a nosotros?>> Pero cuando terminó de increpar a Jesús, el otro ladrón, que había escuchado muchas veces la enseñanza del Maestro, dijo: <<¿Es que ni siquiera temes a Dios? ¿No ves que sufrimos justamente por nuestras acciones, pero que este hombre sufre injustamente? Sería mejor que buscáramos el perdón de nuestros pecados y la salvación de nuestra alma>>. Cuando Jesús escuchó al ladrón decir esto, volvió la cara hacia él y le sonrió con aprobación. Al ver el rostro de Jesús vuelto hacia él, el malhechor reunió su valor, avivó la llama vacilante de su fe, y dijo: <<Señor, acuérdate de mí cuando entres en tu reino>>. Entonces Jesús dijo: <<En verdad, en verdad te digo hoy, algún día estarás conmigo en el Paraíso>>.

\par 
%\textsuperscript{(2009.1)}
\textsuperscript{187:4.2} En medio de los dolores de la muerte física, el Maestro tenía tiempo para escuchar la confesión de fe del bandido creyente. Cuando este ladrón intentó alcanzar la salvación, encontró la liberación. Muchas veces antes de este momento había sentido el impulso de creer en Jesús, pero sólo en estas últimas horas de conciencia se volvió de todo corazón hacia la enseñanza del Maestro. Cuando vio la manera en que Jesús afrontaba la muerte en la cruz, este ladrón ya no pudo resistir la convicción de que el Hijo del Hombre era en verdad el Hijo de Dios.

\par 
%\textsuperscript{(2009.2)}
\textsuperscript{187:4.3} Durante este episodio de la conversión del ladrón y de su recibimiento en el reino por parte de Jesús, el apóstol Juan estaba ausente, pues había ido a la ciudad para traer a su madre y a las amigas de ésta a la escena de la crucifixión. Lucas escuchó posteriormente esta anécdota de labios del capitán romano de la guardia, que se había convertido.

\par 
%\textsuperscript{(2009.3)}
\textsuperscript{187:4.4} El apóstol Juan habló de la crucifixión tal como recordaba el acontecimiento dos tercios de siglo después de haber ocurrido. Los otros escritos se basaron en el relato del centurión romano que estaba de servicio, el cual, a causa de lo que había visto y oído, creyó posteriormente en Jesús y entró plenamente en la hermandad del reino de los cielos en la Tierra.

\par 
%\textsuperscript{(2009.4)}
\textsuperscript{187:4.5} Este joven, el bandido arrepentido, había sido inducido a una vida de violencia y de fechorías por aquellos que ensalzaban esta carrera de pillaje como una protesta patriótica eficaz contra la opresión política y la injusticia social. Este tipo de enseñanza, unido al deseo de aventuras, conducía a muchos jóvenes, por otra parte bien intencionados, a alistarse en estas arriesgadas expediciones de robo a mano armada. Este joven había considerado a Barrabás como un héroe. Ahora veía que se había equivocado. Aquí en la cruz, a su lado, veía a un hombre realmente grande, a un verdadero héroe. Éste era un héroe que inflamaba su fervor e inspiraba sus ideas más elevadas de dignidad moral y vivificaba todos sus ideales de coraje, de virilidad y de valentía. Al contemplar a Jesús, brotó en su corazón un sentimiento irresistible de amor, de lealtad y de auténtica grandeza.

\par 
%\textsuperscript{(2009.5)}
\textsuperscript{187:4.6} Si cualquier otra persona de la burlona multitud hubiera experimentado el nacimiento de la fe en su alma y hubiera apelado a la misericordia de Jesús, habría sido recibida con la misma consideración afectuosa que se había mostrado al bandido creyente.

\par 
%\textsuperscript{(2009.6)}
\textsuperscript{187:4.7} Poco después de que el ladrón arrepentido hubiera escuchado la promesa del Maestro de que algún día se encontrarían en el Paraíso, Juan regresó de la ciudad trayendo con él a su madre y a un grupo de casi doce mujeres creyentes. Juan ocupó su lugar al lado de María, la madre de Jesús, sosteniéndola. Su hijo Judá se encontraba al otro lado. Cuando Jesús contempló esta escena, ya era mediodía, y dijo a su madre: <<Mujer, he aquí a tu hijo>>. Y hablándole a Juan, le dijo: <<Hijo mío, he aquí a tu madre>>. Luego se dirigió a los dos, diciendo: <<Deseo que os vayáis de este lugar>>. Y así, Juan y Judá alejaron a María del Gólgota. Juan llevó a la madre de Jesús al lugar donde él se alojaba en Jerusalén, y luego se apresuró en volver a la escena de la crucifixión. Después de la Pascua, María regresó a Betsaida, donde vivió en la casa de Juan durante el resto de su vida física. María no llegó a vivir más de un año después de la muerte de Jesús.

\par 
%\textsuperscript{(2010.1)}
\textsuperscript{187:4.8} Después de marcharse María, las otras mujeres se retiraron a corta distancia y permanecieron acompañando a Jesús hasta que éste expiró en la cruz. Y aún se hallaban allí cuando bajaron el cuerpo del Maestro para ser sepultado.

\section*{5. Las últimas horas en la cruz}
\par 
%\textsuperscript{(2010.2)}
\textsuperscript{187:5.1} Aunque era pronto para que se produjera un fenómeno así en esta estación del año, poco después de las doce el cielo se oscureció a causa de la presencia de arena fina en el aire. La población de Jerusalén sabía que esto significaba la llegada de una tormenta de arena con viento caliente procedente del desierto de Arabia. Antes de la una el cielo se puso tan oscuro que ocultó al Sol, y el resto de la multitud se apresuró a regresar a la ciudad. Cuando el Maestro abandonó su vida poco después de esta hora, menos de treinta personas estaban presentes, únicamente los trece soldados romanos y un grupo de unos quince creyentes. Todos estos creyentes eran mujeres, excepto dos: Judá el hermano de Jesús, y Juan Zebedeo, que regresó a la escena justo antes de que expirara el Maestro.

\par 
%\textsuperscript{(2010.3)}
\textsuperscript{187:5.2} Poco después de la una, en medio de la creciente oscuridad de la violenta tormenta de arena, Jesús empezó a perder su conciencia humana. Había pronunciado sus últimas palabras de misericordia, de perdón y de exhortación. Su último deseo ---acerca del cuidado de su madre--- había sido expresado. Durante esta hora en que la muerte se acercaba, la mente humana de Jesús recurrió a la repetición de numerosos pasajes de las escrituras hebreas, en particular los salmos. El último pensamiento consciente del Jesús humano estuvo ocupado en la repetición mental de una parte del Libro de los Salmos que ahora se conoce como los salmos veinte, veintiuno y veintidós. Aunque sus labios se movían a menudo, estaba demasiado débil como para pronunciar las palabras de estos pasajes, que tan bien conocía de memoria, a medida que cruzaban por su mente. Sólo en pocas ocasiones aquellos que estaban cerca lograron captar algunas palabras, tales como: <<Sé que el Señor salvará a su ungido>>, <<Tu mano descubrirá a todos mis enemigos>> y <<Dios mío, Dios mío, ¿por qué me has abandonado?>> Jesús no albergó en ningún momento la menor duda de que había vivido de acuerdo con la voluntad del Padre; y nunca dudó de que ahora abandonaba su vida carnal de acuerdo con la voluntad de su Padre. No tenía el sentimiento de que el Padre lo había abandonado; simplemente estaba recitando en su conciencia evanescente numerosos pasajes de las escrituras, entre ellos este salmo veintidós que comienza diciendo <<Dios mío, Dios mío, ¿por qué me has abandonado?>> Y dio la casualidad de que éste fue uno de los tres pasajes que pronunció con la suficiente claridad como para ser escuchado por aquellos que estaban cerca.

\par 
%\textsuperscript{(2010.4)}
\textsuperscript{187:5.3} La última petición que el Jesús mortal hizo a sus semejantes tuvo lugar alrededor de la una y media, cuando dijo por segunda vez: <<Tengo sed>>. Y el mismo capitán de la guardia le humedeció de nuevo los labios con la misma esponja mojada en el vino agrio, que en aquella época llamaban vulgarmente vinagre.

\par 
%\textsuperscript{(2010.5)}
\textsuperscript{187:5.4} La tormenta de arena se volvió más intensa y los cielos se oscurecieron cada vez más. Sin embargo, los soldados y el pequeño grupo de creyentes permanecían allí. Los soldados se habían agachado cerca de la cruz, acurrucados todos juntos para protegerse de la arena cortante. La madre de Juan y otras personas observaban desde cierta distancia, donde estaban un poco resguardadas bajo una roca saliente. Cuando el Maestro exhaló finalmente su último suspiro, al pie de su cruz se encontraban su hermano Judá, su hermana Rut, Juan Zebedeo, María Magdalena y Rebeca, la que había vivido en Séforis.

\par 
%\textsuperscript{(2011.1)}
\textsuperscript{187:5.5} Fue justo antes de las tres cuando Jesús, dando un grito, exclamó: <<¡Todo se ha consumado! Padre, en tus manos encomiendo mi espíritu>>. Cuando hubo dicho esto, inclinó la cabeza y abandonó la lucha por la vida. Cuando el centurión romano vio cómo Jesús había muerto, se golpeó el pecho y dijo: <<Éste era en verdad un hombre justo; debe haber sido realmente un Hijo de Dios>>. Y a partir de ese momento empezó a creer en Jesús.

\par 
%\textsuperscript{(2011.2)}
\textsuperscript{187:5.6} Jesús murió majestuosamente ---tal como había vivido. Admitió sin reservas su realeza y permaneció dueño de la situación durante todo este trágico día. Se dirigió voluntariamente a su muerte ignominiosa después de haber previsto la seguridad de sus apóstoles escogidos. Detuvo sabiamente la violencia alborotadora de Pedro y dispuso que Juan pudiera estar cerca de él hasta el fin de su existencia mortal. Reveló su verdadera naturaleza al sanguinario sanedrín y le recordó a Pilatos el origen de su autoridad soberana como Hijo de Dios. Partió para el Gólgota llevando el travesaño de su propia cruz y terminó su donación amorosa entregando el espíritu que había adquirido como mortal al Padre Paradisiaco. Después de una vida así ---y en el momento de una muerte semejante--- el Maestro podía decir en verdad: <<Se acabó>>.

\par 
%\textsuperscript{(2011.3)}
\textsuperscript{187:5.7} Como éste era el día tanto de la preparación de la Pascua como del sábado, los judíos no querían que estos cuerpos permanecieran expuestos en el Gólgota. Por eso, se presentaron ante Pilatos para pedirle que rompieran las piernas de estos tres hombres, que fueran rematados, para poder bajarlos de sus cruces y echarlos en la fosa común de los criminales antes de ponerse el Sol. Cuando Pilatos escuchó esta petición, envió inmediatamente a tres soldados para que rompieran las piernas y remataran a Jesús y a los dos bandidos.

\par 
%\textsuperscript{(2011.4)}
\textsuperscript{187:5.8} Cuando estos soldados llegaron al Gólgota, actuaron en consecuencia con los dos ladrones, pero para su gran sorpresa, se encontraron con que Jesús ya había muerto. Sin embargo, para asegurarse de su muerte, uno de los soldados le clavó su lanza en el costado izquierdo. Aunque era corriente que las víctimas de la crucifixión permanecieran vivas en la cruz incluso durante dos o tres días, la abrumadora agonía emocional y la aguda angustia espiritual de Jesús provocaron el final de su vida mortal en la carne en poco menos de cinco horas y media.

\section*{6. Después de la crucifixión}
\par 
%\textsuperscript{(2011.5)}
\textsuperscript{187:6.1} En medio de la oscuridad de la tormenta de arena, hacia las tres y media, David Zebedeo envió al último de sus mensajeros con la noticia de la muerte del Maestro. Despachó al último de sus corredores hacia la casa de Marta y María en Betania, donde suponía que estaba la madre de Jesús con el resto de la familia.

\par 
%\textsuperscript{(2011.6)}
\textsuperscript{187:6.2} Después de la muerte del Maestro, Juan envió a las mujeres, bajo la dirección de Judá, a la casa de Elías Marcos, donde permanecieron durante el sábado. En cuanto a Juan, que ya era bien conocido por el centurión romano, permaneció en el Gólgota hasta que José y Nicodemo llegaron a la escena con una orden de Pilatos autorizándolos a tomar posesión del cuerpo de Jesús.

\par 
%\textsuperscript{(2011.7)}
\textsuperscript{187:6.3} Así es como terminó un día de tragedia y de dolor para un inmenso universo, cuyos millares de inteligencias se habían estremecido ante el impresionante espectáculo de la crucifixión de la encarnación humana de su amado Soberano; estaban atónitas ante esta exhibición de insensibilidad y de perversidad humanas.


\chapter{Documento 188. El período en la tumba}
\par 
%\textsuperscript{(2012.1)}
\textsuperscript{188:0.1} EL DÍA y medio que el cuerpo mortal de Jesús estuvo en la tumba de José, el período entre su muerte en la cruz y su resurrección, es un capítulo de la carrera terrenal de Miguel que conocemos poco. Podemos narrar el entierro del Hijo del Hombre y contar en este relato los acontecimientos asociados con su resurrección, pero no podemos proporcionar mucha información auténtica sobre lo que sucedió realmente durante este intervalo de casi treinta y seis horas, desde las tres de la tarde del viernes hasta las tres de la mañana del domingo. Este período de la carrera del Maestro comenzó poco antes de que los soldados romanos lo bajaran de la cruz. Permaneció suspendido en la cruz cerca de una hora después de morir. Hubiera sido bajado antes si no se hubieran retrasado para rematar a los dos bandidos.

\par 
%\textsuperscript{(2012.2)}
\textsuperscript{188:0.2} Los dirigentes de los judíos habían planeado arrojar el cuerpo de Jesús a las fosas comunes abiertas de Gehena, al sur de la ciudad; era costumbre deshacerse así de las víctimas de la crucifixión. Si se hubiera seguido este plan, el cuerpo del Maestro habría estado expuesto a las bestias salvajes.

\par 
%\textsuperscript{(2012.3)}
\textsuperscript{188:0.3} Mientras tanto, José de Arimatea, acompañado de Nicodemo, había ido a ver a Pilatos para pedirle que les entregara el cuerpo de Jesús a fin de enterrarlo adecuadamente. No era raro que los amigos de las personas crucificadas ofrecieran sobornos a las autoridades romanas para obtener el privilegio de disponer de los cuerpos. José se presentó ante Pilatos con una gran suma de dinero, por si era necesario pagar la autorización de trasladar el cuerpo de Jesús a un sepulcro privado. Pero Pilatos no quiso aceptar dinero por esto. Cuando escuchó la petición, firmó rápidamente la orden que autorizaba a José a dirigirse al Gólgota y tomar inmediatamente plena posesión del cuerpo del Maestro. Mientras tanto, la tormenta de arena había amainado considerablemente, y un grupo de judíos que representaba al sanedrín había salido hacia el Gólgota con la intención de asegurarse de que el cuerpo de Jesús acompañaría a los de los bandidos hasta la fosa común pública y abierta.

\section*{1. El entierro de Jesús}
\par 
%\textsuperscript{(2012.4)}
\textsuperscript{188:1.1} Cuando José y Nicodemo llegaron al Gólgota, encontraron que los soldados estaban bajando a Jesús de la cruz y que los representantes del sanedrín estaban allí cerca para asegurarse de que ninguno de los seguidores de Jesús impediría que su cuerpo fuera llevado a la fosa común de los criminales. Cuando José presentó al centurión la orden de Pilatos para que le entregara el cuerpo del Maestro, los judíos causaron un alboroto y pidieron a gritos su posesión. En su frenesí, trataron de apoderarse del cuerpo por la fuerza; al ver esto, el centurión llamó a su lado a cuatro de sus soldados, y con las espadas desenvainadas permanecieron a horcajadas sobre el cuerpo del Maestro que yacía allí en el suelo. El centurión ordenó a los otros soldados que dejaran a los dos ladrones y que rechazaran a esta chusma irritada de judíos enfurecidos. Cuando se restableció el orden, el centurión leyó a los judíos el permiso de Pilatos, se apartó a un lado y le dijo a José: <<Este cuerpo es tuyo para que hagas con él lo que creas conveniente. Yo y mis soldados nos quedaremos aquí para asegurarnos de que nadie se entremeta>>.

\par 
%\textsuperscript{(2013.1)}
\textsuperscript{188:1.2} Una persona crucificada no podía ser enterrada en un cementerio judío; había una ley que prohibía estrictamente esta manera de proceder. José y Nicodemo conocían esta ley, y en el camino hacia el Gólgota habían decidido enterrar a Jesús en el nuevo sepulcro de la familia de José, tallado en la roca maciza y situado a corta distancia al norte del Gólgota, al otro lado de la carretera que conducía a Samaria. Nadie había sido nunca enterrado en este sepulcro, y consideraron apropiado que el Maestro reposara allí. José creía realmente que Jesús resucitaría de entre los muertos, pero Nicodemo tenía muchas dudas. Estos antiguos miembros del sanedrín habían mantenido más o menos en secreto su fe en Jesús, aunque sus colegas sanedristas habían desconfiado de ellos desde hacía tiempo, incluso antes de que se retiraran del consejo. A partir de este momento se convirtieron en los discípulos más abiertos de Jesús en todo Jerusalén.

\par 
%\textsuperscript{(2013.2)}
\textsuperscript{188:1.3} Hacia las cuatro y media, el cortejo fúnebre de Jesús de Nazaret partió del Gólgota hacia el sepulcro de José, situado al otro lado de la carretera. El cuerpo estaba envuelto en una sábana de lino y lo llevaban cuatro hombres, seguidos por las fieles mujeres de Galilea que habían estado vigilando. Los mortales que llevaron hasta la tumba el cuerpo material de Jesús fueron: José, Nicodemo, Juan y el centurión romano.

\par 
%\textsuperscript{(2013.3)}
\textsuperscript{188:1.4} Transportaron el cuerpo hasta el sepulcro, una cámara cuadrada de unos tres metros de lado, donde lo prepararon rápidamente para sepultarlo. En realidad, los judíos no enterraban a sus muertos; los embalsamaban. José y Nicodemo habían traído grandes cantidades de mirra y áloes, y entonces envolvieron el cuerpo con unos vendajes empapados en estas soluciones. Cuando terminaron de embalsamarlo, ataron un paño alrededor de la cara, envolvieron el cuerpo en una sábana de lino y lo depositaron respetuosamente en una plataforma del sepulcro.

\par 
%\textsuperscript{(2013.4)}
\textsuperscript{188:1.5} Después de colocar el cuerpo en la tumba, el centurión hizo señas a sus soldados para que ayudaran a rodar la piedra de cierre delante de la entrada del sepulcro. Los soldados partieron después para Gehena con los cuerpos de los ladrones, mientras los demás regresaban entristecidos a Jerusalén para guardar la fiesta de la Pascua según las leyes de Moisés.

\par 
%\textsuperscript{(2013.5)}
\textsuperscript{188:1.6} El entierro de Jesús se llevó a cabo con una prisa y una precipitación extremas, porque era el día de la preparación y el sábado se acercaba rápidamente. Los hombres se apresuraron en regresar a la ciudad, pero las mujeres se quedaron cerca de la tumba hasta que se hizo de noche.

\par 
%\textsuperscript{(2013.6)}
\textsuperscript{188:1.7} Mientras sucedía todo esto, las mujeres estaban ocultas cerca de allí, de manera que lo vieron todo y observaron el lugar donde había sido enterrado el Maestro. Se habían escondido así porque a las mujeres no les estaba permitido asociarse con los hombres en momentos como éste. Estas mujeres pensaban que Jesús no había sido preparado adecuadamente para ser enterrado, y se pusieron de acuerdo para regresar a la casa de José, descansar el sábado, preparar los aromas y los ung\"uentos, y volver el domingo por la mañana para preparar convenientemente el cuerpo del Maestro para el descanso de la muerte. Las mujeres que permanecieron así cerca de la tumba este viernes por la noche fueron: María Magdalena, María la mujer de Clopas, Marta (otra hermana de la madre de Jesús) y Rebeca de Séforis.

\par 
%\textsuperscript{(2013.7)}
\textsuperscript{188:1.8} Aparte de David Zebedeo y José de Arimatea, muy pocos discípulos de Jesús creían o comprendían realmente que iba a resucitar de la tumba al tercer día.

\section*{2. La protección de la tumba}
\par 
%\textsuperscript{(2014.1)}
\textsuperscript{188:2.1} Aunque los seguidores de Jesús no pensaban en su promesa de resucitar de la tumba al tercer día, sus enemigos no lo olvidaban. Los jefes de los sacerdotes, los fariseos y los saduceos recordaban que habían recibido informes según los cuales había dicho que resucitaría de entre los muertos.

\par 
%\textsuperscript{(2014.2)}
\textsuperscript{188:2.2} Este viernes por la noche, después de la cena pascual, un grupo de dirigentes judíos se reunió hacia la medianoche en la casa de Caifás, donde discutieron de sus temores acerca de las afirmaciones del Maestro de que al tercer día resucitaría de entre los muertos. Esta reunión terminó con el nombramiento de un comité de sanedristas que iría a visitar a Pilatos a primeras horas del día siguiente, llevando la petición oficial del sanedrín de que se apostara una guardia romana delante de la tumba de Jesús para impedir que sus amigos trataran de forzarla. El portavoz de este comité le dijo a Pilatos: <<Señor, nos acordamos de que ese farsante, Jesús de Nazaret, dijo mientras aún estaba vivo: `Al cabo de tres días resucitaré.' Por eso hemos venido ante ti para pedirte que des las órdenes oportunas para proteger el sepulcro contra sus seguidores, al menos hasta después del tercer día. Tenemos el gran temor de que sus discípulos vayan y lo roben durante la noche, para luego proclamar ante el pueblo que ha resucitado de entre los muertos. Si consentimos que suceda esto, este error podría ser mucho peor que haberle permitido seguir viviendo>>.

\par 
%\textsuperscript{(2014.3)}
\textsuperscript{188:2.3} Cuando Pilatos escuchó esta petición de los sanedristas, dijo: <<Os daré una guardia de diez soldados. Seguid vuestro camino y asegurad la tumba>>. Regresaron al templo, cogieron a diez de sus propios guardias, y luego se dirigieron hacia la tumba de José con estos diez guardias judíos y los diez soldados romanos, aunque fuera sábado por la mañana, para ponerlos de vigilancia delante de la tumba. Estos hombres rodaron otra piedra más delante del sepulcro, y colocaron el sello de Pilatos en estas piedras y alrededor de ellas para que no fueran removidas sin que ellos lo supieran. Estos veinte hombres permanecieron de guardia hasta el momento de la resurrección, y los judíos les trajeron de comer y de beber.

\section*{3. Durante el sábado}
\par 
%\textsuperscript{(2014.4)}
\textsuperscript{188:3.1} Durante todo este sábado, los discípulos y los apóstoles permanecieron escondidos, mientras todo Jerusalén hablaba de la muerte de Jesús en la cruz. En este momento había en Jerusalén casi un millón y medio de judíos procedentes de todos los lugares del imperio romano y de Mesopotamia. Era el comienzo de la semana de la Pascua, y todos estos peregrinos estarían en la ciudad para enterarse de la resurrección de Jesús y llevar la noticia a sus tierras natales.

\par 
%\textsuperscript{(2014.5)}
\textsuperscript{188:3.2} A últimas horas del sábado por la noche, Juan Marcos llamó a los once apóstoles para que fueran en secreto a la casa de su padre; poco antes de la medianoche, todos se habían reunido en la misma sala de arriba donde dos noches antes habían compartido la Última Cena con su Maestro.

\par 
%\textsuperscript{(2014.6)}
\textsuperscript{188:3.3} Este sábado por la tarde, poco antes de ponerse el Sol, María la madre de Jesús, acompañada de Rut y de Judá, regresó a Betania para reunirse con su familia. David Zebedeo permaneció en la casa de Nicodemo, donde había hecho los arreglos para que sus mensajeros se reunieran allí el domingo por la mañana temprano. Las mujeres de Galilea, que habían preparado los aromas para embalsamar mejor el cuerpo de Jesús, permanecieron en la casa de José de Arimatea.

\par 
%\textsuperscript{(2014.7)}
\textsuperscript{188:3.4} En realidad, no somos capaces de explicar lo que le sucedió exactamente a Jesús de Nazaret durante este período de un día y medio en el que se suponía que estaba descansando en la nueva tumba de José de Arimatea. Aparentemente, murió en la cruz de la misma muerte natural que hubiera muerto cualquier otro mortal en las mismas circunstancias. Le oímos decir: <<Padre, en tus manos encomiendo mi espíritu>>. No comprendemos plenamente el significado de esta declaración, puesto que su Ajustador del Pensamiento había sido personalizado desde hacía tiempo, y mantenía así una existencia separada del ser mortal de Jesús. Al Ajustador Personalizado del Maestro no le podía afectar de ninguna manera su muerte física en la cruz. Lo que Jesús puso por ahora en las manos del Padre debe haber sido el duplicado espiritual del trabajo inicial del Ajustador, consistente en espiritualizar la mente mortal para poder asegurar la transferencia de la transcripción de la experiencia humana a los mundos de las mansiones. En la experiencia de Jesús debe haber habido alguna realidad espiritual análoga a la naturaleza espiritual, o alma, de los mortales de las esferas que crecen en la fe. Pero esto es simplemente nuestra opinión ---en realidad no sabemos lo que Jesús encomendó a su Padre.

\par 
%\textsuperscript{(2015.1)}
\textsuperscript{188:3.5} Sabemos que la forma física del Maestro descansó en la tumba de José hasta cerca de las tres de la mañana del domingo, pero no tenemos ninguna certidumbre en lo que se refiere al estado de la personalidad de Jesús durante ese período de treinta y seis horas. A veces nos hemos atrevido a explicarnos estas cosas a nosotros mismos más o menos como sigue:

\par 
%\textsuperscript{(2015.2)}
\textsuperscript{188:3.6} 1. La conciencia de Miguel como Creador debe haber estado en libertad y totalmente independiente de su mente mortal asociada en la encarnación física.

\par 
%\textsuperscript{(2015.3)}
\textsuperscript{188:3.7} 2. Sabemos que el antiguo Ajustador del Pensamiento de Jesús estaba presente en la Tierra durante este período y dirigía personalmente las huestes celestiales reunidas.

\par 
%\textsuperscript{(2015.4)}
\textsuperscript{188:3.8} 3. El hombre de Nazaret había adquirido una identidad espiritual que había construido durante su vida en la carne, primero gracias a los esfuerzos directos de su Ajustador del Pensamiento, y después mediante la perfecta adaptación personal que efectuó entre las necesidades físicas y las exigencias espirituales de la existencia humana ideal, una adaptación que llevó a cabo escogiendo sin cesar la voluntad del Padre; esa identidad espiritual es la que debe haber sido confiada al cuidado del Padre Paradisiaco. No sabemos si esta realidad espiritual regresó o no para formar parte de la personalidad resucitada, pero creemos que sí. Pero en el universo están aquellos que sostienen que esta identidad de alma de Jesús descansa ahora en el <<seno del Padre>>, y que posteriormente será liberada para dirigir el Cuerpo de la Finalidad de Nebadon hacia su destino no revelado relacionado con los universos increados de los reinos inorganizados del espacio exterior.

\par 
%\textsuperscript{(2015.5)}
\textsuperscript{188:3.9} 4. Creemos que la conciencia humana o mortal de Jesús durmió durante estas treinta y seis horas. Tenemos razones para creer que el Jesús humano no sabía nada de lo que sucedía en el universo durante este período. Para la conciencia mortal no transcurrió ningún período de tiempo; la resurrección a la vida siguió instantáneamente al sueño de la muerte.

\par 
%\textsuperscript{(2015.6)}
\textsuperscript{188:3.10} Y esto es casi todo lo que podemos indicar sobre el estado de Jesús durante este período en la tumba. Existe una serie de hechos correlativos a los que podemos aludir, aunque no somos del todo competentes para emprender su interpretación.

\par 
%\textsuperscript{(2015.7)}
\textsuperscript{188:3.11} En el inmenso patio de las salas de resurrección del primer mundo de las mansiones de Satania, se puede observar actualmente un magnífico edificio material-morontial conocido con el nombre de <<Monumento conmemorativo de Miguel>>, que lleva ahora el sello de Gabriel. Este monumento fue creado poco después de que Miguel partiera de este mundo, y lleva esta inscripción: <<En conmemoración del tránsito humano de Jesús de Nazaret por Urantia>>.

\par 
%\textsuperscript{(2016.1)}
\textsuperscript{188:3.12} Existen documentos que muestran que, durante este período, el consejo supremo de Salvington, compuesto por cien miembros, celebró una reunión ejecutiva en Urantia bajo la presidencia de Gabriel. También hay archivos que muestran que, durante este período, los Ancianos de los Días de Uversa se comunicaron con Miguel en relación con el estado del universo de Nebadon.

\par 
%\textsuperscript{(2016.2)}
\textsuperscript{188:3.13} Sabemos que al menos un mensaje se cruzó entre Miguel y Emmanuel en Salvington, mientras el cuerpo del Maestro yacía en la tumba.

\par 
%\textsuperscript{(2016.3)}
\textsuperscript{188:3.14} Existen buenas razones para creer que cierta personalidad se sentó en el lugar de Caligastia, en el consejo sistémico de los Príncipes Planetarios que se convocó en Jerusem, mientras el cuerpo de Jesús descansaba en la tumba.

\par 
%\textsuperscript{(2016.4)}
\textsuperscript{188:3.15} Los archivos de Edentia indican que el Padre de la Constelación de Norlatiadek estaba en Urantia, y que recibió instrucciones de Miguel durante este período en que estaba en la tumba.

\par 
%\textsuperscript{(2016.5)}
\textsuperscript{188:3.16} Y existen otras muchas pruebas que sugieren que no toda la personalidad de Jesús estaba dormida e inconsciente durante este período de muerte física aparente.

\section*{4. El significado de la muerte en la cruz}
\par 
%\textsuperscript{(2016.6)}
\textsuperscript{188:4.1} Aunque Jesús no sufrió esta muerte en la cruz para expiar la culpabilidad racial del hombre mortal, ni para proporcionar algún tipo de acercamiento eficaz a un Dios por otra parte ofendido e implacable; aunque el Hijo del Hombre no se ofreció como sacrificio para apaciguar la ira de Dios y abrir a los pecadores el camino para obtener la salvación; a pesar de que estas ideas de expiación y de propiciación son erróneas, sin embargo existen unos significados ligados a esta muerte de Jesús en la cruz que no deberían ser pasados por alto. Es un hecho que a Urantia se le conoce, entre los otros planetas vecinos habitados, como <<el Mundo de la Cruz>>.

\par 
%\textsuperscript{(2016.7)}
\textsuperscript{188:4.2} Jesús deseaba vivir en Urantia una vida mortal plena en la carne. La muerte es, generalmente, una parte de la vida. La muerte es el último acto del drama de los mortales. En vuestros esfuerzos bien intencionados por evitar los errores supersticiosos de la falsa interpretación del significado de la muerte en la cruz, deberíais procurar no cometer el grave error de dejar de percibir el verdadero significado y la auténtica importancia de la muerte del Maestro.

\par 
%\textsuperscript{(2016.8)}
\textsuperscript{188:4.3} El hombre mortal nunca ha sido propiedad de los grandes farsantes. Jesús no murió para redimir al hombre de las garras de los gobernantes apóstatas y de los príncipes caídos de las esferas. El Padre que está en los cielos nunca ha concebido una injusticia tan burda como la de condenar al alma de un mortal por las malas acciones de sus antepasados. La muerte del Maestro en la cruz tampoco fue un sacrificio consistente en un esfuerzo por pagarle a Dios una deuda que la raza humana había contraído con él.

\par 
%\textsuperscript{(2016.9)}
\textsuperscript{188:4.4} Antes de que Jesús viviera en la Tierra, quizás podíais tener la justificación de creer en un Dios semejante, pero ya no es posible desde que el Maestro vivió y murió entre vuestros semejantes mortales. Moisés enseñó la dignidad y la justicia de un Dios Creador, pero Jesús describió el amor y la misericordia de un Padre celestial.

\par 
%\textsuperscript{(2016.10)}
\textsuperscript{188:4.5} La naturaleza animal ---la tendencia a la maldad--- puede ser hereditaria, pero el pecado no se transmite de padres a hijos. El pecado es un acto de rebelión consciente y deliberada contra la voluntad del Padre y las leyes de los Hijos, cometido por una criatura volitiva individual.

\par 
%\textsuperscript{(2017.1)}
\textsuperscript{188:4.6} Jesús vivió y murió para un universo entero, y no solamente para las razas de este único mundo. Aunque los mortales de los reinos disponían de la salvación antes incluso de que Jesús viviera y muriera en Urantia, sin embargo es un hecho que su donación en este mundo iluminó enormemente el camino de la salvación; su muerte contribuyó mucho a hacer evidente para siempre la certeza de la supervivencia de los mortales después de la muerte en la carne.

\par 
%\textsuperscript{(2017.2)}
\textsuperscript{188:4.7} Aunque no es muy adecuado hablar de Jesús como de un sacrificador, un rescatador o un redentor, es enteramente correcto referirse a él como un \textit{salvador}. Hizo que el camino de la salvación (de la supervivencia) fuera para siempre más claro y seguro; el camino de la salvación lo mostró mejor y con más seguridad para todos los mortales de todos los mundos del universo de Nebadon.

\par 
%\textsuperscript{(2017.3)}
\textsuperscript{188:4.8} La idea de Dios como Padre verdadero y amoroso es el único concepto que Jesús enseñó. Una vez que captáis esta idea, debéis, con toda coherencia, abandonar por completo y de manera inmediata todas esas nociones primitivas sobre Dios como monarca ofendido, como soberano severo y todopoderoso, cuyo placer principal consiste en sorprender a sus súbditos obrando mal y en asegurarse de que sean castigados adecuadamente, a menos que otro ser casi igual a él se ofrezca para sufrir por ellos, para morir como un sustituto y en lugar de ellos. Toda la idea de la redención y de la expiación es incompatible con el concepto de Dios tal como fue enseñado y ejemplificado por Jesús de Nazaret. El amor infinito de Dios ocupa el primer lugar en la naturaleza divina.

\par 
%\textsuperscript{(2017.4)}
\textsuperscript{188:4.9} Todo este concepto de la expiación y de la salvación a través del sacrificio está arraigado y apoyado en el egoísmo. Jesús enseñó que el \textit{servicio} al prójimo es el concepto más elevado de la fraternidad de los creyentes en el espíritu. La salvación deben darla por sentada aquellos que creen en la paternidad de Dios. La preocupación principal del creyente no debería ser el deseo egoísta de la salvación personal, sino más bien el impulso desinteresado de amar a los semejantes, y por tanto de servirlos tal como Jesús amó y sirvió a los hombres mortales.

\par 
%\textsuperscript{(2017.5)}
\textsuperscript{188:4.10} Los creyentes auténticos tampoco se inquietan mucho por el castigo futuro de los pecados. El verdadero creyente sólo se preocupa por su separación actual de Dios. Es verdad que los padres sabios pueden castigar a sus hijos, pero hacen todo esto con amor y con un propósito correctivo. No disciplinan llenos de indignación, ni tampoco castigan como represalia.

\par 
%\textsuperscript{(2017.6)}
\textsuperscript{188:4.11} Aunque Dios fuera el monarca severo y legal de un universo en el que la justicia reinara de manera suprema, sin duda no estaría satisfecho con el plan infantil de sustituir a un ofensor culpable por una víctima inocente.

\par 
%\textsuperscript{(2017.7)}
\textsuperscript{188:4.12} Lo importante de la muerte de Jesús, tal como está relacionada con el enriquecimiento de la experiencia humana y la ampliación del camino de la salvación, no es el \textit{hecho} de su muerte, sino más bien la manera magnífica y el espíritu incomparable con que se enfrentó a la muerte.

\par 
%\textsuperscript{(2017.8)}
\textsuperscript{188:4.13} Toda esta idea de la redención de la expiación sitúa a la salvación en un plano de irrealidad; un concepto así es puramente filosófico. La salvación humana es \textit{real;} está basada en dos realidades que las criaturas pueden captar por la fe e incorporarlas de este modo en la experiencia individual humana: el hecho de la paternidad de Dios y su verdad correlacionada, la fraternidad de los hombres. Después de todo, es verdad que se os <<perdonarán vuestras deudas, así como vosotros perdonáis a vuestros deudores>>.

\section*{5. Las lecciones de la cruz}
\par 
%\textsuperscript{(2017.9)}
\textsuperscript{188:5.1} La cruz de Jesús representa la medida total de la devoción suprema del verdadero pastor hacia aquellos miembros de su rebaño que incluso no se la merecen. Todas las relaciones entre Dios y el hombre las sitúa para siempre sobre la base de la familia. Dios es el Padre; el hombre es su hijo. El amor, el amor de un padre por su hijo, se convierte en la verdad central de las relaciones entre el Creador y la criatura en el universo ---y no la justicia de un rey que busca su satisfacción en los sufrimientos y el castigo de sus súbditos malvados.

\par 
%\textsuperscript{(2018.1)}
\textsuperscript{188:5.2} La cruz muestra para siempre que la actitud de Jesús hacia los pecadores no era ni una condena ni una remisión, sino más bien una salvación amorosa y eterna. Jesús es en verdad un salvador, en el sentido de que su vida y su muerte atraen a los hombres hacia la bondad y hacia una justa supervivencia. Jesús ama tanto a los hombres que su amor despierta una respuesta de amor en el corazón humano. El amor es realmente contagioso y eternamente creativo. La muerte de Jesús en la cruz ejemplifica un amor que es lo suficientemente fuerte y divino como para perdonar el pecado y absorber toda maldad. Jesús reveló a este mundo una calidad de rectitud superior a la justicia ---el simple concepto técnico del bien y del mal. El amor divino no se limita a perdonar las ofensas; las absorbe y las destruye realmente. El perdón del amor trasciende totalmente el perdón de la misericordia. La misericordia pone a un lado la culpabilidad del mal; pero el amor destruye para siempre el pecado y todas las debilidades que resultan de él. Jesús trajo a Urantia una nueva manera de vivir. Nos enseñó que no resistiéramos al mal, sino que encontráramos a través de él, de Jesús, una bondad que destruye eficazmente el mal. El perdón de Jesús no es una remisión; es una salvación de la condenación. La salvación no menosprecia las ofensas; \textit{lasenmienda}. El verdadero amor no transige con el odio ni lo perdona, lo destruye. El amor de Jesús nunca se siente satisfecho con el simple perdón. El amor del Maestro implica la rehabilitación, la supervivencia eterna. Es perfectamente correcto hablar de la salvación como de una redención, si con ello os referís a esta rehabilitación eterna.

\par 
%\textsuperscript{(2018.2)}
\textsuperscript{188:5.3} Con el poder de su amor personal por los hombres, Jesús pudo romper la influencia del pecado y del mal. De este modo liberó a los hombres para que escogieran mejores maneras de vivir. Jesús describió una liberación del pasado que prometía en sí misma un triunfo para el futuro. El perdón proporcionaba así la salvación. Cuando el amor divino ha sido aceptado plenamente en el corazón humano, su belleza destruye para siempre el encanto del pecado y el poder del mal.

\par 
%\textsuperscript{(2018.3)}
\textsuperscript{188:5.4} Los sufrimientos de Jesús no se limitaron a la crucifixión. En realidad, Jesús de Nazaret pasó más de veinticinco años en la cruz de una existencia humana real e intensa. El verdadero valor de la cruz consiste en el hecho de que fue la expresión suprema y final de su amor, la revelación culminante de su misericordia.

\par 
%\textsuperscript{(2018.4)}
\textsuperscript{188:5.5} En millones de mundos habitados, decenas de billones de criaturas evolutivas que podían haber tenido la tentación de renunciar a la lucha moral y de abandonar el buen combate de la fe, han mirado una vez más a Jesús en la cruz, y luego han continuado avanzando hacia adelante, inspirados por el espectáculo de un Dios que entrega su vida encarnada por devoción al servicio desinteresado de los hombres.

\par 
%\textsuperscript{(2018.5)}
\textsuperscript{188:5.6} Todo el triunfo de la muerte en la cruz está resumido en el espíritu de la actitud de Jesús hacia sus agresores. Convirtió la cruz en un símbolo eterno del triunfo del amor sobre el odio y de la victoria de la verdad sobre el mal, cuando oró: <<Padre, perdónalos, porque no saben lo que hacen>>. Esta devoción amorosa fue contagiosa en todo un inmenso universo; los discípulos se contagiaron de su Maestro. El primer instructor de su evangelio que fue llamado a entregar su vida en este servicio, dijo, mientras lo lapidaban a muerte: <<No los acuses de este pecado>>.

\par 
%\textsuperscript{(2018.6)}
\textsuperscript{188:5.7} La cruz hace un llamamiento supremo a lo mejor que hay en el hombre, porque nos revela a aquél que estuvo dispuesto a entregar su vida al servicio de sus semejantes. Nadie puede tener un amor más grande que éste: el de estar dispuesto a dar su vida por sus amigos ---y Jesús tenía tal amor, que estaba dispuesto a dar su vida por sus enemigos, un amor más grande que cualquier otro que se hubiera conocido hasta ese momento en la Tierra.

\par 
%\textsuperscript{(2019.1)}
\textsuperscript{188:5.8} En otros mundos, así como en Urantia, este sublime espectáculo de la muerte del Jesús humano en la cruz del Gólgota ha conmovido las emociones de los mortales, mientras que ha despertado la más alta devoción de los ángeles.

\par 
%\textsuperscript{(2019.2)}
\textsuperscript{188:5.9} La cruz es el símbolo superior del servicio sagrado, la consagración de vuestra vida al bienestar y la salvación de vuestros semejantes. La cruz no es el símbolo del sacrificio del Hijo inocente de Dios que se pone en el lugar de los pecadores culpables a fin de apaciguar la ira de un Dios ofendido. Pero sí se alza para siempre, en la Tierra y en todo un inmenso universo, como un símbolo sagrado de los buenos dándose a los malos, salvándolos así mediante esta devoción misma de amor. La cruz sí se alza como la prueba de la forma más elevada de servicio desinteresado, la devoción suprema de la plena donación de una vida recta al servicio de un ministerio incondicional, incluso en la muerte, la muerte en la cruz. La sola visión de este gran símbolo de la vida de donación de Jesús nos inspira realmente a todos a querer hacer lo mismo.

\par 
%\textsuperscript{(2019.3)}
\textsuperscript{188:5.10} Cuando los hombres y las mujeres inteligentes contemplan a Jesús ofreciendo su vida en la cruz, difícilmente se atreverán a quejarse de nuevo ni siquiera de las penalidades más duras de la vida, y mucho menos de las pequeñas incomodidades y sus muchas molestias puramente ficticias. Su vida fue tan gloriosa y su muerte tan triunfal, que todos nos sentimos atraídos a querer compartir las dos. Toda la donación de Miguel posee un verdadero poder de atracción, desde la época de su juventud hasta este espectáculo sobrecogedor de su muerte en la cruz.

\par 
%\textsuperscript{(2019.4)}
\textsuperscript{188:5.11} Aseguraos, pues, de que cuando contempléis la cruz como una revelación de Dios, no la miréis con los ojos del hombre primitivo ni desde el punto de vista de los bárbaros posteriores, pues ambos consideraban a Dios como un Soberano implacable de justicia severa que aplicaba la ley con rigidez. Aseguraos más bien de que veis en la cruz la manifestación final del amor y de la devoción de Jesús a la misión de donación de su vida sobre las razas mortales de su inmenso universo. Ved en la muerte del Hijo del Hombre la culminación de la manifestación del amor divino del Padre por sus hijos de las esferas donde viven los mortales. La cruz representa así la devoción de un afecto complaciente y la donación de la salvación voluntaria a aquellos que están dispuestos a recibir estos dones y esta devoción. En la cruz no hubo nada que el Padre exigiera ---sino únicamente lo que Jesús dio tan gustosamente y que rehusó evitar.

\par 
%\textsuperscript{(2019.5)}
\textsuperscript{188:5.12} Si el hombre no puede apreciar a Jesús de otra manera ni entender el significado de su donación en la Tierra, al menos puede comprender que fue compañero suyo en sus sufrimientos humanos. Nadie puede temer nunca que el Creador no conozca la naturaleza o el grado de sus aflicciones temporales.

\par 
%\textsuperscript{(2019.6)}
\textsuperscript{188:5.13} Sabemos que la muerte en la cruz no sirvió para reconciliar al hombre con Dios, sino para estimular en el hombre la \textit{comprensión} del amor eterno del Padre y de la misericordia sin fin de su Hijo, y para difundir estas verdades universales a un universo entero.


\chapter{Documento 189. La resurrección}
\par 
%\textsuperscript{(2020.1)}
\textsuperscript{189:0.1} POCO después de que Jesús hubiera sido enterrado el viernes por la tarde, el jefe de los arcángeles de Nebadon, en aquel momento presente en Urantia, convocó su consejo encargado de la resurrección de las criaturas volitivas dormidas y se puso a considerar una posible técnica para reconstruir a Jesús. Estos hijos reunidos del universo local, criaturas de Miguel, actuaban así bajo su propia responsabilidad; Gabriel no los había convocado. A medianoche, habían llegado a la conclusión de que la criatura no podía hacer nada para facilitar la resurrección del Creador. Estaban dispuestos a aceptar el consejo de Gabriel, el cual les indicó que, puesto que Miguel había <<entregado su vida por su propio libre albedrío, también tenía el poder de recuperarla de acuerdo con su propia decisión>>. Poco después de que se suspendiera este consejo de arcángeles, de Portadores de Vida y de sus diversos asociados en la tarea de rehabilitación de las criaturas y de creación morontial, el Ajustador Personalizado de Jesús, que dirigía personalmente las huestes celestiales reunidas en ese momento en Urantia, dijo lo siguiente a estos observadores que esperaban con ansiedad:

\par 
%\textsuperscript{(2020.2)}
\textsuperscript{189:0.2} <<Ninguno de vosotros puede hacer nada para ayudar a vuestro Creador-padre a volver a la vida. Como mortal del reino, ha experimentado la muerte humana; como Soberano de un universo, vive todavía. Lo que observáis es el tránsito humano de Jesús de Nazaret de la vida en la carne a la vida en la morontia. El tránsito espiritual de este Jesús concluyó el día en que me separé de su personalidad y me convertí en vuestro director temporal. Vuestro Creador-padre ha elegido atravesar toda la experiencia de sus criaturas mortales, desde el nacimiento en los mundos materiales hasta el estado de la verdadera existencia espiritual, pasando por la muerte natural y la resurrección de la morontia. Estáis a punto de observar una fase de esta experiencia, pero no podéis participar en ella. No podéis hacer por el Creador las cosas que habitualmente hacéis por la criatura. Un Hijo Creador posee en sí mismo el poder de donarse en la similitud de cualquiera de sus hijos creados; tiene en sí mismo el poder de abandonar su vida observable y de recuperarla de nuevo; tiene este poder a causa de la orden directa del Padre Paradisiaco, y sé de lo que hablo>>.

\par 
%\textsuperscript{(2020.3)}
\textsuperscript{189:0.3} Cuando escucharon al Ajustador Personalizado decir esto, todos adoptaron una actitud de ansiosa expectativa, desde Gabriel hasta el más humilde querubín. Veían el cuerpo mortal de Jesús en la tumba; detectaban pruebas de la actividad de su amado Soberano en el universo; y como no comprendían estos fenómenos, esperaron pacientemente el desarrollo de los acontecimientos.

\section*{1. El tránsito morontial}
\par 
%\textsuperscript{(2020.4)}
\textsuperscript{189:1.1} A las dos y cuarenta y cinco del domingo por la mañana, la comisión de encarnación del Paraíso, compuesta por siete personalidades paradisiacas no identificadas, llegó al lugar y se desplegó inmediatamente alrededor de la tumba. A las tres menos diez minutos, intensas vibraciones de actividades materiales y morontiales entremezcladas empezaron a emanar del sepulcro nuevo de José, y a las tres y dos minutos de este domingo por la mañana 9 de abril del año 30, la forma y la personalidad morontiales resucitadas de Jesús de Nazaret salieron de la tumba.

\par 
%\textsuperscript{(2021.1)}
\textsuperscript{189:1.2} Cuando el Jesús resucitado emergió de su tumba, el cuerpo de carne en el que había vivido y trabajado en la Tierra durante cerca de treinta y seis años yacía todavía allí en el nicho del sepulcro, intacto y envuelto en la sábana de lino, tal como había sido colocado para su descanso el viernes por la tarde por José y sus compañeros. La piedra que tapaba la entrada de la tumba tampoco había sido alterada para nada; el sello de Pilatos permanecía aún intacto; los soldados continuaban de guardia. Los guardias del templo habían estado de servicio sin interrupción; la guardia romana había sido cambiada a medianoche. Ninguno de estos vigilantes sospechaba que el objeto de su desvelo se había elevado a una forma de existencia nueva y superior, y que el cuerpo que estaban custodiando era ahora una envoltura exterior desechada, sin ninguna conexión con la personalidad morontial liberada y resucitada de Jesús.

\par 
%\textsuperscript{(2021.2)}
\textsuperscript{189:1.3} La humanidad es lenta en percibir que, en todo lo que es personal, la materia es el esqueleto de la morontia, y que ambos son la sombra reflejada de la realidad espiritual duradera. ¿Cuánto tiempo necesitaréis para considerar que el tiempo es la imagen móvil de la eternidad, y el espacio la sombra fugaz de las realidades del Paraíso?

\par 
%\textsuperscript{(2021.3)}
\textsuperscript{189:1.4} Por lo que podemos discernir, ninguna criatura de este universo y ninguna personalidad de otro universo tuvo nada que ver con esta resurrección morontial de Jesús de Nazaret. El viernes entregó su vida como un mortal del reino; el domingo por la mañana la recuperó de nuevo como un ser morontial del sistema de Satania en Norlatiadek. Hay muchas cosas sobre la resurrección de Jesús que no comprendemos. Pero sabemos que tuvo lugar tal como lo hemos contado y aproximadamente a la hora indicada. También podemos afirmar que todos los fenómenos conocidos asociados con este tránsito como mortal, o resurrección morontial, se produjeron allí mismo en la tumba nueva de José, donde los restos mortales materiales de Jesús yacían envueltos en los lienzos fúnebres.

\par 
%\textsuperscript{(2021.4)}
\textsuperscript{189:1.5} Sabemos que ninguna criatura del universo local participó en este despertar morontial. Percibimos que las siete personalidades del Paraíso rodearon la tumba, pero no les vimos hacer nada en relación con el despertar del Maestro. En cuanto Jesús apareció al lado de Gabriel, justo por encima del sepulcro, las siete personalidades del Paraíso señalaron su intención de partir inmediatamente para Uversa.

\par 
%\textsuperscript{(2021.5)}
\textsuperscript{189:1.6} Clarifiquemos para siempre el concepto de la resurrección de Jesús efectuando las declaraciones siguientes:

\par 
%\textsuperscript{(2021.6)}
\textsuperscript{189:1.7} 1. Su cuerpo material o físico no formaba parte de la personalidad resucitada. Cuando Jesús salió de la tumba, su cuerpo de carne permaneció intacto en el sepulcro. Emergió de la tumba sin desplazar las piedras que cerraban la entrada y sin romper los sellos de Pilatos.

\par 
%\textsuperscript{(2021.7)}
\textsuperscript{189:1.8} 2. No surgió de la tumba como un espíritu ni como Miguel de Nebadon; no apareció con la forma del Soberano Creador, como la que había tenido antes de su encarnación en la similitud de la carne mortal en Urantia.

\par 
%\textsuperscript{(2021.8)}
\textsuperscript{189:1.9} 3. Salió de esta tumba de José con el mismo aspecto que las personalidades morontiales de aquellos que emergen, como seres ascendentes morontiales resucitados, de las salas de resurrección del primer mundo de las mansiones de este sistema local de Satania. La presencia del monumento conmemorativo a Miguel en el centro del inmenso patio de las salas de resurrección de la mansonia número uno nos lleva a sospechar que la resurrección del Maestro en Urantia se promovió de alguna manera en este primer mundo de las mansiones del sistema.

\par 
%\textsuperscript{(2022.1)}
\textsuperscript{189:1.10} El primer acto de Jesús al salir de la tumba fue saludar a Gabriel e indicarle que continuara con el cargo ejecutivo de los asuntos del universo bajo la supervisión de Emmanuel; luego ordenó al jefe de los Melquisedeks que transmitiera sus saludos fraternales a Emmanuel. A continuación pidió al Altísimo de Edentia la certificación de los Ancianos de los Días en cuanto a su tránsito como mortal; luego se volvió hacia los grupos morontiales congregados de los siete mundos de las mansiones, reunidos allí para saludar a su Creador y darle la bienvenida como una criatura de su orden, y Jesús pronunció las primeras palabras de su carrera postmortal. El Jesús morontial dijo: <<Una vez terminada mi vida en la carne, quisiera detenerme aquí un poco de tiempo en mi forma de transición para poder conocer mejor la vida de mis criaturas ascendentes y revelar aún más la voluntad de mi Padre que está en el Paraíso>>.

\par 
%\textsuperscript{(2022.2)}
\textsuperscript{189:1.11} Después de haber hablado, Jesús hizo señas al Ajustador Personalizado y todas las inteligencias del universo, que se habían reunido en Urantia para presenciar la resurrección, fueron enviadas inmediatamente a sus respectivas asignaciones en el universo.

\par 
%\textsuperscript{(2022.3)}
\textsuperscript{189:1.12} Jesús empezó entonces a tomar contacto con el nivel morontial, y se le inició, como criatura, a las exigencias de la vida que había elegido vivir durante un corto período de tiempo en Urantia. Esta iniciación al mundo morontial necesitó más de una hora del tiempo terrestre, y fue interrumpida dos veces por su deseo de comunicarse con sus antiguos compañeros en la carne, cuando éstos vinieron de Jerusalén para asomarse con asombro a la tumba vacía y descubrir lo que consideraban una prueba de su resurrección.

\par 
%\textsuperscript{(2022.4)}
\textsuperscript{189:1.13} El tránsito de Jesús como ser mortal ---la resurrección morontial del Hijo del Hombre--- ya ha terminado. La experiencia transitoria del Maestro como personalidad a medio camino entre lo material y lo espiritual ha comenzado. Y ha hecho todo esto mediante un poder inherente a él mismo; ninguna personalidad le ha prestado ayuda alguna. Ahora vive como Jesús de morontia, y mientras comienza esta vida morontial, su cuerpo material de carne yace intacto allí en la tumba. Los soldados continúan vigilando, y aún no se ha roto el sello del gobernador colocado alrededor de las rocas.

\section*{2. El cuerpo material de Jesús}
\par 
%\textsuperscript{(2022.5)}
\textsuperscript{189:2.1} A las tres y diez, mientras el Jesús resucitado fraternizaba con las personalidades morontiales reunidas de los siete mundos de las mansiones de Satania, el jefe de los arcángeles ---los ángeles de la resurrección--- se acercó a Gabriel y le pidió el cuerpo mortal de Jesús. El jefe de los arcángeles dijo: <<No nos está permitido participar en la resurrección morontial de la experiencia de donación de nuestro soberano Miguel; pero quisiéramos que se nos entregaran sus restos mortales para disolverlos inmediatamente. No tenemos la intención de utilizar nuestra técnica de desmaterialización; deseamos simplemente invocar el proceso de la aceleración del tiempo. Ya es suficiente con haber visto al Soberano vivir y morir en Urantia; las huestes celestiales quisieran ahorrarse el recuerdo de soportar el espectáculo de la lenta putrefacción de la forma humana del Creador y Sostenedor de un universo. En nombre de las inteligencias celestiales de todo Nebadon, solicito un mandato que me confiera la custodia del cuerpo mortal de Jesús de Nazaret y que nos autorice a proceder a su disolución inmediata>>.

\par 
%\textsuperscript{(2023.1)}
\textsuperscript{189:2.2} Después de que Gabriel hubiera conversado con el decano de los Altísimos de Edentia, el arcángel portavoz de las huestes celestiales recibió el permiso de disponer de los restos físicos de Jesús tal como estimara conveniente.

\par 
%\textsuperscript{(2023.2)}
\textsuperscript{189:2.3} Cuando al jefe de los arcángeles le hubieron concedido esta petición, llamó en su ayuda a un gran número de sus semejantes, así como a una multitud de representantes de todas las órdenes de personalidades celestiales; luego, con la ayuda de los intermedios de Urantia, procedió a hacerse cargo del cuerpo físico de Jesús. Este cadáver era una creación puramente material; era literalmente físico; no podía ser sacado de la tumba tal como la forma morontial de la resurrección había podido escapar del sepulcro sellado. Con la ayuda de ciertas personalidades morontiales auxiliares, la forma morontial puede hacerse en ciertos momentos semejante a la del espíritu, de tal manera que puede volverse indiferente a la materia común, mientras que en otros momentos puede volverse discernible y contactable para los seres materiales tales como los mortales del reino.

\par 
%\textsuperscript{(2023.3)}
\textsuperscript{189:2.4} Mientras se preparaban para sacar el cuerpo de Jesús del sepulcro, antes de disponer de él de una manera digna y respetuosa mediante la disolución casi instantánea, los intermedios secundarios de Urantia recibieron la misión de apartar las piedras de la entrada de la tumba. La más grande de estas dos piedras era una enorme roca redonda, muy parecida a una rueda de molino, que se desplazaba dentro de una ranura cincelada en la roca, de tal manera que se la podía hacer rodar hacia adelante y hacia atrás para abrir o cerrar la tumba. Cuando los guardias judíos y los soldados romanos que estaban de vigilancia vieron, a la tenue luz de la madrugada, que esta enorme piedra empezaba a desplazarse aparentemente por sí sola para abrir la entrada de la tumba ---sin ningún medio visible que explicara este movimiento--- se sintieron dominados por el temor y el pánico, y huyeron precipitadamente del lugar. Los judíos huyeron a sus casas, y más tarde regresaron al templo para informar a su capitán de estos hechos. Los romanos huyeron hacia la fortaleza de Antonia e informaron al centurión de lo que habían visto en cuanto éste entró de servicio.

\par 
%\textsuperscript{(2023.4)}
\textsuperscript{189:2.5} Ofreciéndole sobornos al traidor Judas, los dirigentes judíos habían emprendido la sórdida tarea de desembarazarse supuestamente de Jesús, y ahora, al enfrentarse con esta situación embarazosa, en lugar de pensar en castigar a los guardias por haber abandonado su puesto, recurrieron a sobornar a estos guardias y a los soldados romanos. Pagaron una suma de dinero a cada uno de estos veinte hombres y les ordenaron que dijeran a todos: <<Mientras estábamos durmiendo por la noche, los discípulos de Jesús nos sorprendieron y se llevaron el cuerpo>>. Y los dirigentes judíos prometieron solemnemente a los soldados que los defenderían ante Pilatos en el caso de que el gobernador se enterara alguna vez que habían aceptado un soborno.

\par 
%\textsuperscript{(2023.5)}
\textsuperscript{189:2.6} La creencia cristiana en la resurrección de Jesús se ha basado en el hecho de la <<tumba vacía>>. En verdad es un \textit{hecho} que la tumba estaba vacía, pero ésta no es la \textit{verdad} de la resurrección. La tumba estaba realmente vacía cuando llegaron los primeros creyentes, y este hecho, unido al de la resurrección indudable del Maestro, les llevó a formular una creencia que no era cierta: la enseñanza de que el cuerpo material y mortal de Jesús había resucitado de la tumba. Puesto que la verdad está relacionada con las realidades espirituales y los valores eternos, no siempre se puede construir sobre una combinación de hechos aparentes. Aunque unos hechos individuales pueden ser materialmente ciertos, eso no significa que la asociación de un grupo de hechos deba conducir necesariamente a unas conclusiones espirituales verídicas.

\par 
%\textsuperscript{(2023.6)}
\textsuperscript{189:2.7} La tumba de José estaba vacía, no porque el cuerpo de Jesús había sido rehabilitado o resucitado, sino porque las huestes celestiales habían recibido el permiso solicitado para aplicarle una disolución especial y excepcional, una vuelta del <<polvo al polvo>>, sin la intervención del paso del tiempo y sin el funcionamiento de los procesos ordinarios y visibles de la descomposición mortal y la corrupción material.

\par 
%\textsuperscript{(2024.1)}
\textsuperscript{189:2.8} Los restos mortales de Jesús sufrieron el mismo proceso natural de desintegración elemental que caracteriza a todos los cuerpos humanos en la Tierra, excepto que, en lo que se refiere al tiempo, este modo natural de disolución fue enormemente acelerado, apresurado hasta tal punto que se volvió casi instantáneo.

\par 
%\textsuperscript{(2024.2)}
\textsuperscript{189:2.9} Las verdaderas pruebas de la resurrección de Miguel son de naturaleza espiritual, aunque esta enseñanza esté corroborada por el testimonio de numerosos mortales del reino que se encontraron con el Maestro morontial resucitado, lo reconocieron y conversaron con él. Jesús formó parte de la experiencia personal de casi mil seres humanos, antes de despedirse finalmente de Urantia.

\section*{3. La resurrección dispensacional}
\par 
%\textsuperscript{(2024.3)}
\textsuperscript{189:3.1} Poco después de las cuatro y media de este domingo por la mañana, Gabriel llamó a su lado a los arcángeles y se preparó para inaugurar en Urantia la resurrección general del final de la dispensación adámica. Cuando la enorme multitud de serafines y de querubines que participaban en este gran acontecimiento fue ordenada en formación apropiada, el Miguel morontial apareció ante Gabriel, diciendo: <<Así como mi Padre tiene la vida en sí mismo, también le ha dado al Hijo el tener la vida en sí mismo. Aunque todavía no he reasumido por completo el ejercicio de la jurisdicción universal, esta limitación autoimpuesta no restringe de ninguna manera la donación de la vida a mis hijos dormidos; que se empiece a pasar lista para la resurrección planetaria>>.

\par 
%\textsuperscript{(2024.4)}
\textsuperscript{189:3.2} El circuito de los arcángeles funcionó entonces por primera vez desde Urantia. Gabriel y las huestes de arcángeles se trasladaron al lugar de la polarización espiritual del planeta; y cuando Gabriel dio la señal, su voz se transmitió como un relámpago al primer mundo de las mansiones del sistema, diciendo: <<Por orden de Miguel, ¡que resuciten los muertos de una dispensación de Urantia!>> Entonces, todos los supervivientes de las razas humanas de Urantia que se habían dormido desde la época de Adán, y que aún no habían sido juzgados, aparecieron en las salas de resurrección de mansonia, dispuestos para la investidura morontial. Y en un instante, los serafines y sus asociados se prepararon para partir hacia los mundos de las mansiones. Normalmente, estos guardianes seráficos, asignados anteriormente a la custodia colectiva de estos mortales supervivientes, habrían estado presentes en el momento de su despertar en las salas de resurrección de mansonia, pero en este momento se encontraban en Urantia porque la presencia de Gabriel era necesaria aquí en relación con la resurrección morontial de Jesús.

\par 
%\textsuperscript{(2024.5)}
\textsuperscript{189:3.3} Aunque innumerables personas que tenían guardianes seráficos personales, y otras que habían alcanzado el nivel necesario de progreso espiritual de la personalidad, habían continuado hasta mansonia en las épocas posteriores a los tiempos de Adán y Eva, y aunque había habido muchas resurrecciones especiales y milenarias para los hijos de Urantia, ésta era la tercera vez que se pasaba lista a escala planetaria, o sea una resurrección dispensacional completa. La primera tuvo lugar en la época de la llegada del Príncipe Planetario, la segunda durante los tiempos de Adán, y esta tercera señalaba la resurrección morontial, el tránsito como mortal, de Jesús de Nazaret.

\par 
%\textsuperscript{(2024.6)}
\textsuperscript{189:3.4} Cuando el jefe de los arcángeles recibió la señal de la resurrección planetaria, el Ajustador Personalizado del Hijo del Hombre renunció a su autoridad sobre las huestes celestiales reunidas en Urantia, y a todos estos hijos del universo local los devolvió a la jurisdicción de sus jefes respectivos. Cuando hubo hecho esto, partió para Salvington a fin de registrar ante Emmanuel la finalización del tránsito como mortal de Miguel. Y todas las huestes celestiales cuyos servicios no se necesitaban en Urantia le siguieron de inmediato. Pero Gabriel permaneció en Urantia con el Jesús morontial.

\par 
%\textsuperscript{(2025.1)}
\textsuperscript{189:3.5} Y ésta es la narración de los acontecimientos de la resurrección de Jesús, tal como los vieron aquellos que los presenciaron mientras sucedían realmente, sin las limitaciones de la visión humana parcial y restringida.

\section*{4. El descubrimiento de la tumba vacía}
\par 
%\textsuperscript{(2025.2)}
\textsuperscript{189:4.1} Al acercarse el momento de la resurrección de Jesús este domingo de madrugada, hay que recordar que los diez apóstoles se alojaban en la casa de Elías y María Marcos, donde estaban durmiendo en la habitación de arriba, descansando en los mismos divanes en los que se habían reclinado durante la última cena con su Maestro. Este domingo por la mañana, todos estaban reunidos allí, excepto Tomás. Tomás estuvo con ellos durante unos minutos cuando se reunieron inicialmente a últimas horas del sábado por la noche, pero la visión de los apóstoles, unida a la idea de lo que le había sucedido a Jesús, fue demasiado para él. Echó una ojeada a sus compañeros y abandonó inmediatamente la habitación, encaminándose a la casa de Simón en Betfagé, donde pensaba lamentarse de sus penas en la soledad. Todos los apóstoles sufrían, no tanto debido a la duda y a la desesperación, como al temor, la pena y la verg\"uenza.

\par 
%\textsuperscript{(2025.3)}
\textsuperscript{189:4.2} En la casa de Nicodemo se encontraban reunidos, con David Zebedeo y José de Arimatea, unos doce o quince discípulos de Jesús de los más sobresalientes en Jerusalén. En la casa de José de Arimatea había unas quince o veinte de las principales mujeres creyentes. Estas mujeres eran las únicas que se encontraban en la casa de José, y habían permanecido encerradas durante las horas del sábado y la noche después del sábado, de manera que ignoraban que una guardia militar vigilaba la tumba; tampoco sabían que habían rodado una segunda piedra delante de la tumba, y que el sello de Pilatos había sido colocado en las dos piedras.

\par 
%\textsuperscript{(2025.4)}
\textsuperscript{189:4.3} Un poco antes de las tres de este domingo por la mañana, cuando los primeros signos del amanecer empezaron a aparecer hacia el este, cinco de estas mujeres partieron hacia la tumba de Jesús. Habían preparado en abundancia unas lociones especiales para embalsamar, y llevaban consigo numerosos vendajes de lino. Tenían la intención de aplicar con más esmero los ung\"uentos fúnebres en el cuerpo de Jesús y de envolverlo más cuidadosamente en los nuevos vendajes.

\par 
%\textsuperscript{(2025.5)}
\textsuperscript{189:4.4} Las mujeres que salieron con esta misión de ungir el cuerpo de Jesús fueron: María Magdalena, María la madre de los gemelos Alfeo, Salomé la madre de los hermanos Zebedeo, Juana la mujer de Chuza y Susana la hija de Ezra de Alejandría.

\par 
%\textsuperscript{(2025.6)}
\textsuperscript{189:4.5} Eran aproximadamente las tres y media cuando las cinco mujeres, cargadas con sus ung\"uentos, llegaron delante de la tumba vacía. En el momento de salir por la puerta de Damasco, se encontraron con algunos soldados más o menos sobrecogidos de terror que huían hacia el interior de la ciudad, y esto hizo que se detuvieran durante unos minutos; pero como no sucedía nada más, reanudaron su camino.

\par 
%\textsuperscript{(2025.7)}
\textsuperscript{189:4.6} Se quedaron enormemente sorprendidas cuando vieron que la piedra estaba apartada de la entrada de la tumba, ya que durante el camino habían comentado entre ellas: <<¿Quién nos ayudará a apartar la piedra?>> Depositaron su carga en el suelo y empezaron a mirarse unas a otras asustadas y con una gran estupefacción. Mientras permanecían allí, temblando de miedo, María Magdalena se aventuró a rodear la piedra más pequeña y se atrevió a entrar en el sepulcro abierto. Esta tumba de José estaba situada en su jardín, en la ladera de la parte oriental de la carretera, y también miraba hacia el este. A esta hora había la suficiente claridad de un nuevo día como para que María pudiera ver el lugar donde había reposado el cuerpo del Maestro, y percibir que ya no estaba allí. En el nicho de piedra donde habían puesto a Jesús, María sólo vio el paño doblado donde había reposado su cabeza y los vendajes con los que había sido envuelto, que yacían intactos y tal como habían descansado en la piedra antes de que las huestes celestiales sacaran el cuerpo. La sábana que lo cubría yacía a los pies del nicho fúnebre.

\par 
%\textsuperscript{(2026.1)}
\textsuperscript{189:4.7} Después de que María hubo permanecido unos momentos en la entrada de la tumba (al principio no distinguía con claridad cuando entró en ella), vio que el cuerpo de Jesús ya no estaba y que en su lugar sólo quedaban estos lienzos fúnebres, y dio un grito de alarma y de angustia. Todas las mujeres estaban extremadamente nerviosas; habían tenido los nervios de punta desde que encontraron a los soldados dominados por el pánico en la puerta de la ciudad, y cuando María dio este grito de angustia, se aterrorizaron y huyeron a toda prisa. No se detuvieron hasta que hubieron recorrido todo el camino hasta la puerta de Damasco. En ese momento, Juana tomó conciencia de que habían abandonado a María; reunió a sus compañeras y emprendieron el camino de vuelta hacia la tumba.

\par 
%\textsuperscript{(2026.2)}
\textsuperscript{189:4.8} Mientras se acercaban al sepulcro, la asustada Magdalena, que había sentido aun más terror cuando no encontró a sus hermanas esperándola al salir de la tumba, se precipitó ahora hacia ellas, exclamando con excitación: <<No está ahí ---¡se lo han llevado!>> Las llevó de vuelta a la tumba, y todas entraron y vieron que estaba vacía.

\par 
%\textsuperscript{(2026.3)}
\textsuperscript{189:4.9} Las cinco mujeres se sentaron entonces en la piedra cerca de la entrada y discutieron la situación. Aún no se les había ocurrido que Jesús había sido resucitado. Habían estado solas todo el sábado, y suponían que el cuerpo había sido trasladado a otro lugar de descanso. Pero cuando reflexionaban sobre esta solución a su dilema, no acertaban a explicarse la colocación ordenada de los lienzos fúnebres; ¿cómo podían haber sacado el cuerpo, si los mismos vendajes en los que estaba envuelto habían sido dejados en la misma posición, y aparentemente intactos, en la plataforma fúnebre?

\par 
%\textsuperscript{(2026.4)}
\textsuperscript{189:4.10} Mientras estas mujeres estaban sentadas allí a primeras horas del amanecer de este nuevo día, miraron hacia un lado y observaron a un desconocido silencioso e inmóvil. Por un momento se asustaron de nuevo, pero María Magdalena se precipitó hacia él y, pensando que podría ser el jardinero, le dijo: <<¿Dónde habéis llevado al Maestro? ¿Dónde lo han enterrado? Dínoslo para poder ir a buscarlo>>. Como el desconocido no le contestaba a María, ésta empezó a llorar. Entonces Jesús les habló, diciendo: <<¿A quién buscáis?>> María dijo: <<Buscamos a Jesús, que fue enterrado en la tumba de José, pero ya no está. ¿Sabes dónde lo han llevado?>> Entonces dijo Jesús: <<¿No os dijo este Jesús, incluso en Galilea, que moriría pero que resucitaría de nuevo?>> Estas palabras asustaron a las mujeres, pero el Maestro estaba tan cambiado que aún no lo reconocían a la tenue luz del contraluz. Mientras meditaban sus palabras, Jesús se dirigió a Magdalena con una voz familiar, diciendo: <<María>>. Cuando ella escuchó esta palabra de simpatía bien conocida y de saludo afectuoso, supo que era la voz del Maestro, y se precipitó para arrodillarse a sus pies, exclamando: <<¡Mi Señor y Maestro!>> Todas las demás mujeres reconocieron que era el Maestro el que se encontraba delante de ellas con una forma glorificada, y rápidamente se arrodillaron delante de él.

\par 
%\textsuperscript{(2027.1)}
\textsuperscript{189:4.11} Estos ojos humanos fueron capaces de ver la forma morontial de Jesús gracias al ministerio especial de los transformadores y de los intermedios, en asociación con algunas personalidades morontiales que en ese momento acompañaban a Jesús.

\par 
%\textsuperscript{(2027.2)}
\textsuperscript{189:4.12} Cuando María intentó abrazar sus pies, Jesús le dijo: <<No me toques, María, porque no soy como me has conocido en la carne. Con esta forma permaneceré con vosotros algún tiempo antes de ascender hacia el Padre. Pero id todas ahora y decid a mis apóstoles ---y a Pedro--- que he resucitado y que habéis hablado conmigo>>.

\par 
%\textsuperscript{(2027.3)}
\textsuperscript{189:4.13} Después de que estas mujeres se hubieron recobrado del impacto de su asombro, se apresuraron a regresar a la ciudad y a la casa de Elías Marcos, donde contaron a los diez apóstoles todo lo que les había sucedido; pero los apóstoles no estaban dispuestos a creerlas. Al principio pensaron que las mujeres habían visto una visión, pero cuando María Magdalena repitió las palabras que Jesús les había dicho, y cuando Pedro escuchó su nombre, salió precipitadamente de la habitación de arriba, seguido de cerca por Juan, para llegar a la tumba lo más rápidamente posible y ver estas cosas por sí mismo.

\par 
%\textsuperscript{(2027.4)}
\textsuperscript{189:4.14} Las mujeres repitieron a los otros apóstoles la historia de su conversación con Jesús, pero no querían creer; y no quisieron ir a averiguarlo por sí mismos como hicieron Pedro y Juan.

\section*{5. Pedro y Juan en la tumba}
\par 
%\textsuperscript{(2027.5)}
\textsuperscript{189:5.1} Mientras los dos apóstoles corrían hacia el Gólgota y la tumba de José, los pensamientos de Pedro alternaban entre el miedo y la esperanza; temía encontrar al Maestro, pero su esperanza se había despertado con la historia de que Jesús le había enviado un mensaje especial. Estaba casi persuadido de que Jesús estaba realmente vivo; se acordaba de la promesa de que resucitaría al tercer día. Aunque parezca extraño, no había pensado en esta promesa desde la crucifixión hasta este momento en que corría hacia el norte a través de Jerusalén. Mientras Juan salía precipitadamente de la ciudad, un extraño éxtasis de alegría y de esperanza brotaba en su alma. Estaba casi convencido de que las mujeres habían visto realmente al Maestro resucitado.

\par 
%\textsuperscript{(2027.6)}
\textsuperscript{189:5.2} Como Juan era más joven que Pedro, corrió más deprisa que él y llegó primero a la tumba. Juan permaneció en la entrada contemplando la tumba, que se encontraba tal como María la había descrito. Simón Pedro llegó corriendo poco después, entró, y vio la misma tumba vacía con los lienzos fúnebres dispuestos de manera tan particular. Cuando Pedro salió, Juan también entró y lo vio todo por sí mismo; luego se sentaron en la piedra para reflexionar sobre el significado de lo que habían visto y oído. Mientras estaban sentados allí, dieron vueltas en su cabeza a todas las cosas que les habían dicho sobre Jesús, pero no podían percibir claramente lo que había sucedido.

\par 
%\textsuperscript{(2027.7)}
\textsuperscript{189:5.3} Pedro sugirió al principio que la tumba había sido saqueada, que los enemigos habían robado el cuerpo, y quizás sobornado a los guardias. Pero Juan razonó que la tumba no habría sido dejada de manera tan ordenada si hubieran robado el cuerpo, y también planteó la cuestión de cómo podía ser que los vendajes hubieran sido dejados atrás, y aparentemente tan intactos. Y los dos volvieron a entrar en el sepulcro para examinar más atentamente los lienzos fúnebres. Cuando salieron de la tumba por segunda vez, encontraron a María Magdalena que había vuelto y estaba llorando delante de la entrada. María había ido a ver a los apóstoles con la creencia de que Jesús había resucitado de la tumba, pero cuando todos se negaron a creer su relato, se sintió abatida y desesperada. Anhelaba volver cerca de la tumba, donde pensaba que había escuchado la voz familiar de Jesús.

\par 
%\textsuperscript{(2027.8)}
\textsuperscript{189:5.4} Mientras María permanecía allí después de la partida de Pedro y Juan, el Maestro se le apareció de nuevo, diciendo: <<No dudes; ten el valor de creer en lo que has visto y oído. Vuelve a donde están mis apóstoles y diles de nuevo que he resucitado, que me apareceré a ellos, y que pronto los precederé en Galilea como les prometí>>.

\par 
%\textsuperscript{(2028.1)}
\textsuperscript{189:5.5} María se apresuró a volver a la casa de Marcos y contó a los apóstoles que había hablado de nuevo con Jesús, pero no quisieron creerla. Sin embargo, cuando Pedro y Juan regresaron, dejaron de burlarse y se llenaron de temor y de aprensión.


\chapter{Documento 190. Las apariciones morontiales de Jesús}
\par 
%\textsuperscript{(2029.1)}
\textsuperscript{190:0.1} EL JESÚS resucitado se prepara ahora para pasar un corto período en Urantia con el fin de experimentar la carrera morontial ascendente de un mortal de los reinos. Aunque este período de vida morontial deberá pasarlo en el mundo de su encarnación como mortal, sin embargo será equivalente en todos los sentidos a la experiencia de los mortales de Satania que pasan por la vida morontial progresiva de los siete mundos de las mansiones de Jerusem.

\par 
%\textsuperscript{(2029.2)}
\textsuperscript{190:0.2} Todo este poder inherente a Jesús ---el don de la vida--- que le permitió resucitar de entre los muertos, es el mismo don de la vida eterna que él concede a los creyentes en el reino, y que incluso ahora asegura la resurrección de éstos de las ataduras de la muerte natural.

\par 
%\textsuperscript{(2029.3)}
\textsuperscript{190:0.3} Los mortales de los reinos se levantarán, en la mañana de la resurrección, con el mismo tipo de cuerpo de transición, o morontial, que Jesús tenía cuando se levantó de la tumba este domingo por la mañana. Estos cuerpos no tienen circulación sanguínea, y estos seres no comen los alimentos materiales corrientes; sin embargo, estas formas morontiales son \textit{reales}. Cuando los diversos creyentes vieron a Jesús después de su resurrección, lo vieron realmente, no fueron víctimas del engaño de sus propias visiones o alucinaciones.

\par 
%\textsuperscript{(2029.4)}
\textsuperscript{190:0.4} Una fe permanente en la resurrección de Jesús fue la característica esencial de la fe de todas las ramas de la enseñanza primitiva del evangelio. En Jerusalén, Alejandría, Antioquía y Filadelfia, todos los educadores del evangelio se unieron en esta fe implícita en la resurrección del Maestro.

\par 
%\textsuperscript{(2029.5)}
\textsuperscript{190:0.5} Al examinar el papel sobresaliente que jugó María Magdalena en la proclamación de la resurrección del Maestro, hay que indicar que María era la portavoz principal del grupo femenino, tal como Pedro lo era de los apóstoles. María no era la directora de las mujeres que trabajaban para el reino, pero era su educadora principal y su portavoz pública. María se había convertido en una mujer muy prudente, de manera que la audacia que mostró al hablarle a un hombre que había tomado por el jardinero de José, sólo indica el horror que sintió cuando encontró la tumba vacía. La profundidad y la agonía de su amor, la plenitud de su devoción, fueron las que le hicieron olvidar por un momento las prohibiciones convencionales que tenía una mujer judía para dirigirse a un desconocido.

\section*{1. Los anunciadores de la resurrección}
\par 
%\textsuperscript{(2029.6)}
\textsuperscript{190:1.1} Los apóstoles no querían que Jesús los dejara; por eso no habían hecho caso de todas sus declaraciones sobre su muerte, así como de sus promesas de resucitar. No esperaban que la resurrección se produjera tal como ocurrió, y se negaron a creer hasta que tuvieron que hacer frente al apremio de una evidencia indiscutible y de la prueba absoluta de sus propias experiencias.

\par 
%\textsuperscript{(2030.1)}
\textsuperscript{190:1.2} Cuando los apóstoles se negaron a creer en el relato de las cinco mujeres que manifestaban que habían visto a Jesús y hablado con él, María Magdalena regresó al sepulcro, y las demás volvieron a la casa de José, donde relataron sus experiencias a la hija de José y a las otras mujeres. Y las mujeres creyeron en sus declaraciones. Poco después de las seis, la hija de José de Arimatea y las cuatro mujeres que habían visto a Jesús fueron a la casa de Nicodemo, donde contaron todos estos sucesos a José, Nicodemo, David Zebedeo y a los otros hombres que estaban allí reunidos. Nicodemo y los demás dudaron de esta historia, dudaron de que Jesús hubiera resucitado de entre los muertos; supusieron que los judíos habían trasladado el cuerpo. José y David estaban dispuestos a creer en el relato, de tal manera que se apresuraron a ir a inspeccionar la tumba, y lo encontraron todo tal como las mujeres lo habían descrito. Fueron los últimos en ver así el sepulcro, porque a las siete y media el sumo sacerdote envió al capitán de los guardias del templo a la tumba para que se llevara los lienzos fúnebres. El capitán los envolvió en la sábana de lino y los tiró por un barranco cercano.

\par 
%\textsuperscript{(2030.2)}
\textsuperscript{190:1.3} Desde la tumba, David y José fueron inmediatamente a la casa de Elías Marcos, donde mantuvieron una conferencia con los diez apóstoles en la habitación de arriba. Sólo Juan Zebedeo estaba dispuesto a creer, aunque débilmente, que Jesús había resucitado de entre los muertos. Pedro había creído al principio, pero como no logró encontrar al Maestro, empezó a tener grandes dudas. Todos estaban dispuestos a creer que los judíos se habían llevado el cuerpo. David no quiso discutir con ellos, pero en el momento de irse, dijo: <<Vosotros sois los apóstoles, y deberíais comprender estas cosas. No discutiré con vosotros; no obstante, ahora regreso a la casa de Nicodemo, donde he indicado a los mensajeros que nos reuniremos esta mañana. Cuando se hayan reunido, los enviaré a realizar su última misión, la de anunciar la resurrección del Maestro. Escuché decir al Maestro que, después de su muerte, resucitaría al tercer día, y yo le creo>>. Después de hablar así a los abatidos y desamparados embajadores del reino, este joven que se había nombrado a sí mismo jefe de las comunicaciones y de la información se despidió de los apóstoles. Al salir de la habitación de arriba, dejó caer la bolsa de Judas, que contenía todos los fondos apostólicos, en el regazo de Mateo Leví.

\par 
%\textsuperscript{(2030.3)}
\textsuperscript{190:1.4} Eran aproximadamente las nueve y media cuando el último de los veintiséis mensajeros de David llegó a la casa de Nicodemo. David los reunió enseguida en el espacioso patio y se dirigió a ellos, diciendo:

\par 
%\textsuperscript{(2030.4)}
\textsuperscript{190:1.5} <<Amigos y hermanos, me habéis servido todo este tiempo de acuerdo con vuestro juramento hacia mí y entre vosotros mismos, y os tomo por testigos de que hasta ahora nunca he enviado una falsa información por medio de vosotros. Estoy a punto de enviaros a vuestra última misión como mensajeros voluntarios del reino, y al hacer esto os libero de vuestro juramento, y con ello disuelvo este cuerpo de mensajeros. Amigos, os manifiesto que hemos terminado nuestra tarea. El Maestro ya no tiene necesidad de mensajeros mortales; ha resucitado de entre los muertos. Antes de que lo arrestaran nos dijo que moriría y que resucitaría al tercer día. Yo he visto la tumba ---está vacía. He hablado con María Magdalena y con otras cuatro mujeres que han conversado con Jesús. Ahora disuelvo este grupo, me despido de vosotros y os envío a vuestras misiones respectivas con el siguiente mensaje que llevaréis a los creyentes: `Jesús ha resucitado de entre los muertos; la tumba está vacía.'>>

\par 
%\textsuperscript{(2030.5)}
\textsuperscript{190:1.6} La mayoría de los que estaban presentes trataron de persuadir a David para que no hiciera esto. Pero no pudieron influir sobre él. Entonces intentaron disuadir a los mensajeros, pero éstos no quisieron prestar atención a sus palabras de duda. Y así, poco antes de las diez de este domingo por la mañana, estos veintiséis corredores salieron como los primeros anunciadores del hecho y de la verdad poderosos de la resurrección de Jesús. Y partieron para esta misión como lo habían hecho para tantas otras, para cumplir el juramento realizado a David Zebedeo y entre ellos mismos. Estos hombres tenían una gran confianza en David. Partieron para efectuar esta tarea sin detenerse siquiera para hablar con las mujeres que habían visto a Jesús; aceptaron la palabra de David. La mayoría de ellos creía en lo que David les había dicho, e incluso aquellos que dudaban un poco, llevaron el mensaje con la misma certeza y la misma rapidez que los demás.

\par 
%\textsuperscript{(2031.1)}
\textsuperscript{190:1.7} Este día, los apóstoles ---el cuerpo espiritual del reino--- están reunidos en la sala de arriba donde manifiestan su temor y expresan sus dudas, mientras que estos mensajeros laicos, que representan el primer intento de socialización del evangelio de la fraternidad de los hombres del Maestro, bajo las órdenes de su jefe audaz y eficiente, salen para proclamar que el Salvador de un mundo y de un universo ha resucitado. Y emprenden este servicio extraordinario antes de que los representantes escogidos del Maestro estén dispuestos a creer en su palabra o a aceptar el testimonio de los testigos oculares.

\par 
%\textsuperscript{(2031.2)}
\textsuperscript{190:1.8} Estos veintiséis fueron enviados a la casa de Lázaro en Betania y a todos los centros de creyentes, desde Beerseba en el sur hasta Damasco y Sidón en el norte, y desde Filadelfia en el este hasta Alejandría en el oeste.

\par 
%\textsuperscript{(2031.3)}
\textsuperscript{190:1.9} Cuando David se hubo despedido de sus hermanos, fue a buscar a su madre a la casa de José, y partieron entonces para Betania a fin de reunirse con la familia de Jesús que les estaba esperando. David permaneció en Betania con Marta y María hasta que éstas vendieron sus bienes terrenales, y luego las acompañó en su viaje para reunirse con su hermano Lázaro en Filadelfia.

\par 
%\textsuperscript{(2031.4)}
\textsuperscript{190:1.10} Cerca de una semana más tarde, Juan Zebedeo llevó a María la madre de Jesús a la casa que él tenía en Betsaida. Santiago, el hermano mayor de Jesús, permaneció con su familia en Jerusalén. Rut se quedó en Betania con las hermanas de Lázaro. El resto de la familia de Jesús regresó a Galilea. David Zebedeo salió de Betania con Marta y María hacia Filadelfia a primeros de junio, al día siguiente de casarse con Rut, la hermana menor de Jesús.

\section*{2. La aparición de Jesús en Betania}
\par 
%\textsuperscript{(2031.5)}
\textsuperscript{190:2.1} Desde el momento de su resurrección morontial hasta el instante de su ascensión espiritual a las alturas, Jesús efectuó diecinueve apariciones distintas de forma visible a sus creyentes en la Tierra. No se apareció a sus enemigos ni a aquellos que no podían hacer un uso espiritual de su manifestación en forma visible. Su primera aparición fue a las cinco mujeres cerca de la tumba; la segunda, a María Magdalena, también cerca de la tumba.

\par 
%\textsuperscript{(2031.6)}
\textsuperscript{190:2.2} La tercera aparición tuvo lugar alrededor del mediodía de este domingo en Betania. Poco después del mediodía, Santiago, el hermano mayor de Jesús, se encontraba en el jardín de Lázaro delante de la tumba vacía del hermano resucitado de Marta y María, dándole vueltas en su cabeza a las noticias que el mensajero de David les había traído una hora antes. Santiago siempre había tendido a creer en la misión de su hermano mayor en la Tierra, pero hacía mucho tiempo que había perdido el contacto con el trabajo de Jesús, y se había puesto a dudar seriamente de las afirmaciones posteriores de los apóstoles de que Jesús era el Mesías. Toda la familia estaba alarmada y casi confundida por la noticia que había traído el mensajero. Mientras Santiago permanecía delante de la tumba vacía de Lázaro, María Magdalena llegó a la casa y empezó a contar emocionadamente a la familia sus experiencias de las primeras horas de la mañana en la tumba de José. Antes de que terminara, David Zebedeo llegó con su madre. Rut creía, por supuesto, en el relato, y lo mismo le sucedió a Judá después de hablar con David y Salomé.

\par 
%\textsuperscript{(2032.1)}
\textsuperscript{190:2.3} Entretanto, mientras buscaban a Santiago y antes de que llegaran a encontrarlo, éste permanecía allí en el jardín cerca de la tumba, y se dio cuenta de una presencia cercana, como si alguien le hubiera tocado en el hombro. Cuando se volvió para mirar, contempló la aparición gradual de una forma extraña a su lado. Estaba demasiado asombrado para hablar y demasiado asustado para huir. Entonces, la extraña forma habló y dijo: <<Santiago, vengo para llamarte al servicio del reino. Únete sinceramente a tus hermanos y sígueme>>. Cuando Santiago escuchó su nombre, supo que era su hermano mayor, Jesús, el que le había dirigido la palabra. Todos tenían más o menos dificultades para reconocer la forma morontial del Maestro, pero pocos de ellos tenían el menor problema para reconocer su voz o identificar de otra manera su encantadora personalidad en cuanto empezaba a comunicarse con ellos.

\par 
%\textsuperscript{(2032.2)}
\textsuperscript{190:2.4} Cuando Santiago se dio cuenta de que Jesús le estaba hablando, empezó a ponerse de rodillas, exclamando: <<Padre y hermano mío>>, pero Jesús le pidió que permaneciera de pie mientras hablaba con él. Caminaron por el jardín y conversaron casi tres minutos; hablaron de las experiencias del pasado e hicieron planes para el futuro cercano. Mientras se acercaban a la casa, Jesús dijo: <<Adiós, Santiago, hasta que os salude a todos juntos>>.

\par 
%\textsuperscript{(2032.3)}
\textsuperscript{190:2.5} Santiago entró corriendo en la casa, mientras lo buscaban en Betfagé, exclamando: <<Acabo de ver a Jesús y he hablado con él, he charlado con él. No está muerto; ¡ha resucitado! Ha desaparecido delante de mí, diciendo: `Adiós, hasta que os salude a todos juntos'>>. Apenas había acabado de hablar cuando Judá regresó, y volvió a contar la experiencia del encuentro con Jesús en el jardín para que Judá la escuchara. Todos empezaron a creer en la resurrección de Jesús. Santiago anunció entonces que no volvería a Galilea, y David exclamó: <<No solamente lo ven las mujeres emocionadas; incluso los hombres valerosos han empezado a verlo. Espero verlo yo mismo>>.

\par 
%\textsuperscript{(2032.4)}
\textsuperscript{190:2.6} David no tuvo que esperar mucho tiempo, porque la cuarta aparición de Jesús en la que fue reconocido por los mortales, tuvo lugar poco antes de las dos de la tarde en esta misma casa de Marta y María, cuando apareció de manera visible delante de su familia terrenal y de los amigos de ésta, veinte personas en total. El Maestro apareció en la puerta de atrás, que estaba abierta, diciendo: <<Que la paz sea con vosotros. Saludos para aquellos que estuvieron cerca de mí en la carne, y fraternidad para mis hermanos y hermanas en el reino de los cielos. ¿Cómo habéis podido dudar? ¿Por qué habéis esperado tanto tiempo antes de escoger seguir de todo corazón la luz de la verdad? Entrad pues todos en la comunión del Espíritu de la Verdad en el reino del Padre>>. Cuando empezaron a recuperarse del primer impacto de su asombro y a acercarse a él como para abrazarlo, desapareció de su vista.

\par 
%\textsuperscript{(2032.5)}
\textsuperscript{190:2.7} Todos querían precipitarse hacia la ciudad para contarle a los incrédulos apóstoles lo que había sucedido, pero Santiago los detuvo. Sólo María Magdalena recibió permiso para regresar a la casa de José. Santiago les prohibió que anunciaran públicamente el hecho de esta visita morontial, debido a ciertas cosas que Jesús le había dicho mientras conversaban en el jardín. Pero Santiago nunca reveló más cosas sobre su conversación de este día con el Maestro resucitado en la casa de Lázaro en Betania.

\section*{3. En la casa de José}
\par 
%\textsuperscript{(2033.1)}
\textsuperscript{190:3.1} La quinta manifestación morontial de Jesús, reconocida por los ojos mortales, se produjo en presencia de unas veinticinco mujeres creyentes reunidas en la casa de José de Arimatea, hacia las cuatro y quince minutos de este mismo domingo por la tarde. María Magdalena había vuelto a la casa de José unos minutos antes de esta aparición. Santiago, el hermano de Jesús, había rogado que no se dijera nada a los apóstoles acerca de la aparición del Maestro en Betania, pero no le había pedido a María que se abstuviera de informar a sus hermanas creyentes sobre este acontecimiento. En consecuencia, después de que María hiciera prometer a todas las mujeres que guardarían el secreto, procedió a contarles lo que acababa de suceder mientras estaba con la familia de Jesús en Betania. Estaba precisamente en medio de este relato apasionante, cuando un silencio repentino y solemne se hizo entre ellas; vieron en medio de su grupo la forma enteramente visible de Jesús resucitado. Éste las saludó diciendo: <<Que la paz sea con vosotras. En la hermandad del reino no habrá ni judíos ni gentiles, ni ricos ni pobres, ni libres ni esclavos, ni hombres ni mujeres. Vosotras también estáis llamadas a divulgar la buena nueva de la liberación de la humanidad a través del evangelio de la filiación con Dios en el reino de los cielos. Id por el mundo entero proclamando este evangelio y confirmando a los creyentes en la fe del mismo. Y mientras lo hacéis, no olvidéis cuidar a los enfermos y fortalecer a los tímidos y a los que están dominados por el temor. Siempre estaré con vosotras, incluso hasta los confines de la Tierra>>. Cuando hubo hablado así, desapareció de su vista, mientras las mujeres caían de bruces y adoraban en silencio.

\par 
%\textsuperscript{(2033.2)}
\textsuperscript{190:3.2} De las cinco apariciones morontiales de Jesús acontecidas hasta este momento, María Magdalena había presenciado cuatro.

\par 
%\textsuperscript{(2033.3)}
\textsuperscript{190:3.3} A consecuencia de haber enviado a los mensajeros a media mañana, y debido a la filtración inconsciente de indicios relacionados con esta aparición de Jesús en la casa de José, los dirigentes de los judíos empezaron a recibir noticias al principio del anochecer de que se decía por la ciudad que Jesús había resucitado, y que muchas personas pretendían haberlo visto. Estos rumores excitaron enormemente a los sanedristas. Después de consultar apresuradamente con Anás, Caifás convocó una reunión del sanedrín para las ocho de aquella noche. En esta reunión se tomó la decisión de echar de las sinagogas a toda persona que mencionara la resurrección de Jesús. Se sugirió incluso que cualquiera que afirmara haberlo visto debía ser ejecutado; sin embargo, esta proposición no se sometió a votación ya que la reunión se disolvió en una confusión que rayaba en verdadero pánico. Se habían atrevido a pensar que habían acabado con Jesús. Estaban a punto de descubrir que sus verdaderas dificultades con el hombre de Nazaret sólo acababan de empezar.

\section*{4. La aparición a los griegos}
\par 
%\textsuperscript{(2033.4)}
\textsuperscript{190:4.1} Alrededor de las cuatro y media, el Maestro hizo su sexta aparición morontial a unos cuarenta creyentes griegos que estaban reunidos en la casa de un tal Flavio. Mientras estaban discutiendo las noticias sobre la resurrección del Maestro, éste se manifestó en medio de ellos, a pesar de que las puertas estaban bien cerradas, y les habló diciendo: <<Que la paz sea con vosotros. Aunque el Hijo del Hombre apareció en la Tierra entre los judíos, vino para aportar su ministerio a todos los hombres. En el reino de mi Padre no habrá ni judíos ni gentiles; todos seréis hermanos ---los hijos de Dios. Id pues a proclamar al mundo entero este evangelio de salvación tal como lo habéis recibido de los embajadores del reino, y yo os recibiré en la comunión de la fraternidad de los hijos de la fe y de la verdad del Padre>>. Cuando les hubo encargado esta misión, se despidió y no lo volvieron a ver. Permanecieron dentro de la casa toda la noche; estaban demasiado dominados por el pavor y el miedo como para atreverse a salir. Ninguno de estos griegos tampoco durmió aquella noche; se quedaron despiertos discutiendo estas cosas y esperando que el Maestro los visitara de nuevo. En este grupo había muchos griegos que estaban en Getsemaní cuando los soldados arrestaron a Jesús y Judas lo traicionó con un beso.

\par 
%\textsuperscript{(2034.1)}
\textsuperscript{190:4.2} Los rumores de la resurrección de Jesús y las noticias sobre las numerosas apariciones a sus seguidores se están difundiendo rápidamente, y toda la ciudad está alcanzando un alto grado de agitación. El Maestro ya se ha aparecido a su familia, a las mujeres y a los griegos, y dentro de poco se va a manifestar en medio de los apóstoles. El sanedrín pronto va a empezar a examinar estos nuevos problemas que se han impuesto tan repentinamente a los dirigentes judíos. Jesús piensa mucho en sus apóstoles, pero desea que sigan estando solos algunas horas más para que reflexionen seriamente y mediten cuidadosamente antes de visitarlos.

\section*{5. El paseo con los dos hermanos}
\par 
%\textsuperscript{(2034.2)}
\textsuperscript{190:5.1} En Emaús, a unos once kilómetros al oeste de Jerusalén, vivían dos hermanos, pastores, que habían pasado la semana de la Pascua en Jerusalén asistiendo a los sacrificios, las ceremonias y las fiestas. Cleofás, el mayor, creía parcialmente en Jesús; al menos había sido expulsado de la sinagoga. Su hermano, Jacobo, no era creyente, aunque estaba muy intrigado por las cosas que había escuchado acerca de las enseñanzas y las obras del Maestro.

\par 
%\textsuperscript{(2034.3)}
\textsuperscript{190:5.2} Este domingo por la tarde, a unos cinco kilómetros de Jerusalén y pocos minutos antes de las cinco, mientras estos dos hermanos caminaban por la carretera de Emaús, iban hablando con mucha seriedad de Jesús, de sus enseñanzas, de sus obras, y muy en particular de los rumores de que su tumba estaba vacía, y de que algunas mujeres habían hablado con él. Cleofás tenía una ligera inclinación a creer en estas noticias, pero Jacobo insistía en que todo el asunto era probablemente un engaño. Mientras razonaban y discutían así a medida que se dirigían hacia su casa, la manifestación morontial de Jesús, su séptima aparición, caminó con ellos mientras continuaban el viaje. Cleofás había escuchado a Jesús enseñar con frecuencia y había comido con él en diversas ocasiones en las casas de los creyentes de Jerusalén. Pero no reconoció al Maestro, ni siquiera cuando éste les habló con toda libertad.

\par 
%\textsuperscript{(2034.4)}
\textsuperscript{190:5.3} Después de acompañarlos durante un corto trayecto, Jesús dijo: <<¿De qué hablabais con tanta seriedad cuando me acerqué a vosotros?>> Cuando Jesús dijo esto, se detuvieron y le miraron con una sorpresa entristecida. Cleofás dijo: <<¿Es posible que vivas en Jerusalén y no conozcas las cosas que han sucedido recientemente?>> Entonces preguntó el Maestro: <<¿Qué cosas?>> Cleofás respondió: <<Si no sabes estas cosas, eres el único en Jerusalén que no ha escuchado los rumores sobre Jesús de Nazaret, que era un profeta poderoso en palabras y en acciones delante de Dios y de todo el pueblo. Los jefes de los sacerdotes y nuestros dirigentes lo entregaron a los romanos y les pidieron que lo crucificaran. Ahora bien, muchos de nosotros habíamos esperado que él fuera el que liberara a Israel del yugo de los gentiles. Pero esto no es todo. Ahora hace tres días que fue crucificado, y unas mujeres nos han sorprendido hoy declarando que esta mañana muy temprano fueron a su tumba y la encontraron vacía. Y estas mismas mujeres insisten en que han hablado con ese hombre; sostienen que ha resucitado de entre los muertos. Cuando las mujeres informaron de esto a los hombres, dos de sus apóstoles corrieron hasta la tumba y la encontraron igualmente vacía>> ---y aquí Jacobo interrumpió a su hermano para decir: <<pero no vieron a Jesús>>.

\par 
%\textsuperscript{(2035.1)}
\textsuperscript{190:5.4} Mientras seguían caminando, Jesús les dijo: <<¡Qué lentos sois en comprender la verdad! Puesto que me decís que estabais discutiendo de las enseñanzas y de las obras de este hombre, quizás yo pueda iluminaros, puesto que estoy más que familiarizado con esas enseñanzas. ¿No recordáis que ese Jesús siempre enseñó que su reino no era de este mundo, y que como todos los hombres son hijos de Dios, deberían encontrar la libertad y la independencia en la alegría espiritual de la comunión de la fraternidad del servicio amoroso en este nuevo reino de la verdad del amor del Padre celestial? ¿No recordáis cómo este Hijo del Hombre proclamó la salvación de Dios para todos los hombres, cuidando a los enfermos y a los afligidos, y liberando a los que estaban encadenados por el miedo y esclavizados por el mal? ¿No sabéis que este hombre de Nazaret dijo a sus discípulos que debía ir a Jerusalén, ser entregado a sus enemigos, que lo ejecutarían, y que resucitaría al tercer día? ¿No os han dicho todo esto? ¿Y no habéis leído nunca en las Escrituras acerca de este día de salvación para los judíos y los gentiles, donde dice que en él todas las familias de la Tierra serán benditas; que él escuchará el lamento de los necesitados y salvará el alma de los pobres que lo buscan; que todas las naciones lo llamarán bendito? Que este Libertador será como la sombra de una gran roca en una tierra agotada. Que alimentará al rebaño como un verdadero pastor, reuniendo a las ovejas en sus brazos y llevándolas tiernamente en su seno. Que abrirá los ojos de los ciegos espirituales y sacará a los presos de la desesperación a la plena luz y libertad; que todos los que están en las tinieblas verán la gran luz de la salvación eterna. Que curará a los que tienen el corazón destrozado, proclamará la libertad a los cautivos del pecado y abrirá la prisión a los que están esclavizados por el miedo y encadenados por el mal. Que consolará a los afligidos y les otorgará la alegría de la salvación en lugar de la pena y la tristeza. Que él será el deseo de todas las naciones y la alegría perpetua de los que buscan la rectitud. Que este Hijo de la verdad y de la rectitud se elevará sobre el mundo con una luz curativa y un poder salvador; e incluso salvará a su pueblo de sus pecados; que buscará y salvará realmente a los que están perdidos. Que no destruirá a los débiles, sino que aportará la salvación a todos los que tienen hambre y sed de rectitud. Que los que creen en él tendrán la vida eterna. Que derramará su espíritu sobre todo el género humano, y que este Espíritu de la Verdad será en cada creyente una fuente de agua que brotará hasta la vida eterna. ¿No habéis comprendido la grandeza del evangelio del reino que este hombre os entregó? ¿No percibís la grandeza de la salvación que os ha llegado?>>

\par 
%\textsuperscript{(2035.2)}
\textsuperscript{190:5.5} Para entonces habían llegado cerca del pueblo donde vivían estos hermanos. Estos dos hombres no habían dicho ni una palabra desde que Jesús empezó a enseñarlos mientras andaban por el camino. Pronto se detuvieron delante de su humilde morada, y Jesús estaba a punto de despedirse de ellos para continuar carretera abajo, pero le obligaron a entrar y a quedarse con ellos. Insistieron en que era casi de noche y que permaneciera con ellos. Jesús consintió finalmente, y poco después de entrar en la casa se sentaron para comer. Dieron el pan a Jesús para que lo bendijera, y cuando empezó a partirlo y a darlo a los hermanos, los ojos de éstos se abrieron, y Cleofás reconoció que su invitado era el Maestro mismo. Y cuando dijo: <<Es el Maestro..>>., el Jesús morontial desapareció de su vista.

\par 
%\textsuperscript{(2036.1)}
\textsuperscript{190:5.6} Entonces se dijeron el uno al otro: <<¡No es de extrañar que nuestro corazón ardiera por dentro cuando nos hablaba mientras caminábamos por la carretera, y mientras abría nuestra inteligencia a las enseñanzas de las Escrituras!>>

\par 
%\textsuperscript{(2036.2)}
\textsuperscript{190:5.7} Ni siquiera se detuvieron para comer. Habían visto al Maestro morontial y salieron precipitadamente de la casa, regresando rápidamente a Jerusalén para difundir la buena nueva del Salvador resucitado.

\par 
%\textsuperscript{(2036.3)}
\textsuperscript{190:5.8} Hacia las nueve de aquella noche y poco antes de que el Maestro se apareciera a los diez, estos dos hermanos excitados irrumpieron en la habitación de arriba donde estaban los apóstoles, declarando que habían visto a Jesús y que habían hablado con él. Contaron todo lo que Jesús les había dicho, y que no habían descubierto quién era hasta el momento en que partió el pan.


\chapter{Documento 191. Las apariciones a los apóstoles y a otros discípulos principales}
\par 
%\textsuperscript{(2037.1)}
\textsuperscript{191:0.1} EL DOMINGO de la resurrección fue un día terrible en la vida de los apóstoles; diez de ellos pasaron la mayor parte del día en la habitación de arriba detrás de las puertas atrancadas. Podían haber huido de Jerusalén, pero tenían miedo de ser arrestados por los agentes del sanedrín si los encontraban en la calle. Tomás rumiaba a solas sus problemas en Betfagé. Hubiera hecho mejor permaneciendo con sus compañeros apóstoles, y los hubiera ayudado a dirigir sus discusiones por unas vías más provechosas.

\par 
%\textsuperscript{(2037.2)}
\textsuperscript{191:0.2} A lo largo de todo el día, Juan sostuvo la idea de que Jesús había resucitado de entre los muertos. Recordó que en no menos de cinco ocasiones diferentes el Maestro había afirmado que resucitaría de nuevo, y que al menos tres veces había aludido al tercer día. La actitud de Juan tenía una influencia considerable sobre ellos, especialmente sobre su hermano Santiago y sobre Natanael. Juan los habría influido aún más si no hubiera sido el miembro más joven del grupo.

\par 
%\textsuperscript{(2037.3)}
\textsuperscript{191:0.3} Los problemas de los apóstoles estaban muy relacionados con su aislamiento. Juan Marcos los mantenía al corriente de lo que sucedía alrededor del templo y les informaba de los numerosos rumores que se difundían por la ciudad, pero no se le ocurrió recoger las noticias de los diferentes grupos de creyentes a los que Jesús ya se había aparecido. Este era el tipo de servicio que habían prestado hasta ahora los mensajeros de David, pero todos estaban ausentes realizando su última misión como anunciadores de la resurrección a los grupos de creyentes que vivían lejos de Jerusalén. Por primera vez en todos estos años, los apóstoles se dieron cuenta de cuánto habían dependido de los mensajeros de David para recibir su información diaria sobre los asuntos del reino.

\par 
%\textsuperscript{(2037.4)}
\textsuperscript{191:0.4} Como ya era típico en él, Pedro vaciló emocionalmente todo el día entre la fe y la duda con respecto a la resurrección del Maestro. Pedro no podía olvidar la visión de los lienzos fúnebres que yacían allí en la tumba como si el cuerpo de Jesús se hubiera evaporado desde dentro. <<Pero>>, razonaba Pedro, <<si ha resucitado y puede mostrarse a las mujeres, ¿por qué no se muestra a nosotros, sus apóstoles?>> Pedro se entristecía cuando pensaba que Jesús quizás no venía hacia ellos a causa de su presencia entre los apóstoles, porque lo había negado aquella noche en el patio de Anás. Luego se animaba con el mensaje que habían traído las mujeres: <<Id a decir a mis apóstoles ---y a Pedro>>. Pero estimularse con este mensaje implicaba que tenía que creer que las mujeres habían visto y oído realmente al Maestro resucitado. Pedro alternó así entre la fe y la duda durante todo el día, hasta poco después de las ocho, en que se atrevió a salir al patio. Pedro pensaba alejarse de los apóstoles para no impedir que Jesús viniera hasta ellos porque él había negado al Maestro.

\par 
%\textsuperscript{(2037.5)}
\textsuperscript{191:0.5} Santiago Zebedeo defendió al principio que todos debían ir a la tumba; estaba firmemente a favor de hacer algo para llegar hasta el fondo del misterio. Fue Natanael el que les impidió que se mostraran en público a consecuencia de los argumentos de Santiago, y lo hizo recordándoles la advertencia de Jesús de que no arriesgaran indebidamente sus vidas en estos momentos. Hacia el mediodía, Santiago se había calmado como los demás y permanecieron en una espera vigilante. Habló poco; estaba enormemente desilusionado porque Jesús no se les aparecía, y no sabía nada de las numerosas apariciones del Maestro a otros grupos y a otras personas.

\par 
%\textsuperscript{(2038.1)}
\textsuperscript{191:0.6} Andrés escuchó mucho este día. Estaba extremadamente perplejo por la situación y tenía más dudas de las que le correspondían, pero al menos disfrutaba de cierta sensación de libertad al no tener la responsabilidad de dirigir a los demás apóstoles. En verdad estaba agradecido al Maestro por haberle liberado de las cargas de la jefatura antes de que empezaran a vivir estas horas de confusión.

\par 
%\textsuperscript{(2038.2)}
\textsuperscript{191:0.7} Más de una vez durante las largas horas agotadoras de este día trágico, la única influencia que sostuvo al grupo fue la frecuente contribución de los consejos filosóficos característicos de Natanael. Él fue realmente la influencia que controló a los diez durante todo el día. Ni una sola vez expresó si creía o no en la resurrección del Maestro. Pero a medida que pasaba el día, se sintió cada vez más inclinado a creer que Jesús había cumplido su promesa de resucitar.

\par 
%\textsuperscript{(2038.3)}
\textsuperscript{191:0.8} Simón Celotes estaba demasiado abrumado como para participar en las discusiones. La mayor parte del tiempo permaneció recostado en un diván en un rincón de la habitación, mirando a la pared; no llegó a hablar media docena de veces en todo el día. Su concepto del reino se había derrumbado, y no lograba discernir que la resurrección del Maestro podía cambiar materialmente la situación. Su decepción era muy personal y demasiado aguda como para que pudiera reponerse a corto plazo, ni siquiera ante un hecho tan prodigioso como la resurrección.

\par 
%\textsuperscript{(2038.4)}
\textsuperscript{191:0.9} Aunque parezca extraño, Felipe, que habitualmente se expresaba poco, habló mucho durante toda la tarde de este día. Por la mañana tuvo poco que decir, pero se pasó toda la tarde haciendo preguntas a los demás apóstoles. Pedro se irritó a menudo con las preguntas de Felipe, pero los demás se las tomaron con buena disposición. Felipe deseaba saber en particular, en el caso de que Jesús hubiera resucitado realmente de la tumba, si su cuerpo tendría las marcas físicas de la crucifixión.

\par 
%\textsuperscript{(2038.5)}
\textsuperscript{191:0.10} Mateo estaba sumamente confundido; escuchó las discusiones de sus compañeros, pero pasó la mayor parte del tiempo dándole vueltas en la cabeza al problema de las finanzas futuras del grupo. Independientemente de la supuesta resurrección de Jesús, Judas ya no estaba, David le había entregado los fondos sin ceremonias, y no tenían un jefe con autoridad. Antes de que Mateo llegara a considerar seriamente los argumentos de los demás sobre la resurrección, ya había visto al Maestro cara a cara.

\par 
%\textsuperscript{(2038.6)}
\textsuperscript{191:0.11} Los gemelos Alfeo participaron poco en estos importantes debates; estaban plenamente ocupados en sus trabajos habituales. Uno de ellos expresó la actitud de los dos cuando dijo, en respuesta a una pregunta de Felipe: <<No comprendemos esto de la resurrección, pero nuestra madre dice que ha hablado con el Maestro, y nosotros la creemos>>.

\par 
%\textsuperscript{(2038.7)}
\textsuperscript{191:0.12} Tomás se encontraba en medio de uno de sus típicos períodos de depresión desesperante. Durmió una parte del día y se paseó por las colinas el resto del tiempo. Sentía el impulso de reunirse con sus compañeros apóstoles, pero el deseo de estar solo era más fuerte.

\par 
%\textsuperscript{(2038.8)}
\textsuperscript{191:0.13} El Maestro aplazó su primera aparición morontial a los apóstoles por varias razones. En primer lugar, después de que oyeran hablar de su resurrección, quería que tuvieran tiempo para reflexionar bien sobre lo que les había dicho acerca de su muerte y de su resurrección cuando aún estaba con ellos en la carne. El Maestro quería que Pedro venciera algunas de sus dificultades particulares antes de manifestarse a todos ellos. En segundo lugar, deseaba que Tomás estuviera con ellos en el momento de su primera aparición. Juan Marcos localizó a Tomás en la casa de Simón en Betfagé este domingo por la mañana temprano, e informó de ello a los apóstoles alrededor de las once. Tomás hubiera regresado con ellos en cualquier momento de este día si Natanael u otros dos apóstoles cualquiera hubieran ido a buscarlo. Tenía realmente el deseo de volver, pero como los había dejado la noche anterior de la manera que lo había hecho, era demasiado orgulloso como para regresar tan pronto por su propia cuenta. Al día siguiente estaba tan deprimido que necesitó casi una semana para decidirse a regresar. Los apóstoles le esperaban, y él esperaba que sus hermanos fueran a buscarlo para pedirle que volviera con ellos. Tomás permaneció así alejado de sus compañeros hasta el sábado siguiente por la noche cuando, después del anochecer, Pedro y Juan fueron a Betfagé y lo trajeron de vuelta con ellos. Ésta es también la razón por la que no partieron inmediatamente para Galilea después de que Jesús se les apareciera por primera vez; no querían irse sin Tomás.

\section*{1. La aparición a Pedro}
\par 
%\textsuperscript{(2039.1)}
\textsuperscript{191:1.1} Eran casi las ocho y media de la noche de este domingo cuando Jesús se apareció a Simón Pedro en el jardín de la casa de Marcos. Ésta era su octava manifestación morontial. Pedro había vivido con una pesada carga de dudas y de culpabilidad desde que había negado al Maestro. Toda la jornada del sábado y este domingo había luchado contra el temor de que quizás ya no era un apóstol. Se había estremecido de horror ante la suerte de Judas, e incluso había pensado que él también había traicionado a su Maestro. Toda esta tarde pensó que quizás su presencia entre los apóstoles era la que impedía que Jesús se les apareciera, a condición, por supuesto, de que hubiera resucitado realmente de entre los muertos. Y fue a Pedro, en estas condiciones mentales y con este estado de ánimo, a quien Jesús se apareció mientras el deprimido apóstol deambulaba entre las flores y los arbustos.

\par 
%\textsuperscript{(2039.2)}
\textsuperscript{191:1.2} Cuando Pedro pensó en la mirada afectuosa del Maestro mientras éste pasaba por el porche de Anás, cuando dio vueltas en su cabeza al maravilloso mensaje <<Id a decir a mis apóstoles ---y a Pedro>> que le habían traído aquella mañana temprano las mujeres que regresaban de la tumba vacía, cuando contempló estas muestras de misericordia, su fe empezó a vencer sus dudas. Entonces se detuvo, apretando los puños, mientras decía en voz alta: <<Creo que ha resucitado de entre los muertos; voy a decírselo a mis hermanos>>. Cuando pronunció estas palabras, la forma de un hombre apareció repentinamente delante de él y le habló con un tono de voz familiar, diciendo: <<Pedro, el enemigo deseaba poseerte, pero no he querido abandonarte. Sabía que no me habías negado con el corazón; por eso te había perdonado incluso antes de que me lo pidieras; pero ahora debes dejar de pensar en ti mismo y en los problemas del momento, y prepararte para llevar la buena nueva del evangelio a los que están en las tinieblas. Ya no debe importarte lo que puedas obtener del reino, sino que debes preocuparte más bien por lo que puedas dar a los que viven en una espantosa miseria espiritual. Cíñete, Simón, para la batalla de un nuevo día, para la lucha contra las tinieblas espirituales y las dudas perjudiciales de la mente común de los hombres>>.

\par 
%\textsuperscript{(2039.3)}
\textsuperscript{191:1.3} Pedro y el Jesús morontial caminaron por el jardín y hablaron de las cosas del pasado, del presente y del futuro durante cerca de cinco minutos. Luego el Maestro desapareció de su vista, diciendo: <<Adiós, Pedro, hasta que te vea con tus hermanos>>.

\par 
%\textsuperscript{(2039.4)}
\textsuperscript{191:1.4} Pedro se quedó aturdido durante un momento al darse cuenta de que había hablado con el Maestro resucitado, y que podía estar seguro de que continuaba siendo un embajador del reino. Acababa de escuchar al Maestro glorificado que le exhortaba a continuar predicando el evangelio. Con todo esto brotando en su corazón, se precipitó hacia la habitación de arriba donde estaban sus compañeros apóstoles, y jadeando de excitación exclamó: <<He visto al Maestro; estaba en el jardín. He hablado con él y me ha perdonado>>.

\par 
%\textsuperscript{(2040.1)}
\textsuperscript{191:1.5} La declaración de Pedro de que había visto a Jesús en el jardín causó una profunda impresión en sus compañeros apóstoles, y estaban casi dispuestos a abandonar sus dudas cuando Andrés se levantó y les advirtió que no se dejaran influir demasiado por el relato de su hermano. Andrés dio a entender que Pedro había visto cosas irreales anteriormente. Aunque Andrés no aludió directamente a la visión nocturna en el Mar de Galilea, donde Pedro afirmó que había visto al Maestro venir hacia ellos caminando sobre el agua, dijo lo suficiente como para mostrar a todos los presentes que guardaba este incidente en la memoria. Simón Pedro se sintió muy dolido por las insinuaciones de su hermano, y cayó inmediatamente en un silencio alicaído. Los gemelos sintieron mucha compasión por Pedro; los dos se acercaron para expresarle su simpatía y decirle que ellos le creían, y reafirmar que su propia madre también había visto al Maestro.

\section*{2. La primera aparición a los apóstoles}
\par 
%\textsuperscript{(2040.2)}
\textsuperscript{191:2.1} Aquella noche poco después de las nueve, después de la partida de Cleofás y Jacobo, mientras los gemelos Alfeo consolaban a Pedro, y Natanael le hacía reproches a Andrés, y mientras los diez apóstoles estaban reunidos allí en la habitación de arriba con todas las puertas cerradas con cerrojo por temor a ser arrestados, el Maestro apareció de pronto en su forma morontial en medio de ellos, diciendo: <<Que la paz sea con vosotros. ¿Por qué os asustáis tanto cuando aparezco, como si vierais a un espíritu? ¿No os he hablado de estas cosas cuando estaba presente con vosotros en la carne? ¿No os dije que los jefes de los sacerdotes y los dirigentes me entregarían para ser ejecutado, que uno de vosotros mismos me traicionaría, y que resucitaría al tercer día? ¿Por qué pues todas vuestras dudas y toda esta discusión acerca de los relatos de las mujeres, de Cleofás y de Jacobo, e incluso de Pedro? ¿Cuánto tiempo dudaréis de mis palabras y os negaréis a creer en mis promesas? Y ahora que me veis realmente, ¿vais a creer? Incluso ahora uno de vosotros está ausente. Cuando todos estéis juntos una vez más, y después de que todos sepáis con certeza que el Hijo del Hombre ha salido de la tumba, partid de aquí para Galilea. Tened fe en Dios; tened fe los unos en los otros; y así entraréis en el nuevo servicio del reino de los cielos. Permaneceré con vosotros en Jerusalén hasta que estéis preparados para ir a Galilea. Mi paz os dejo>>.

\par 
%\textsuperscript{(2040.3)}
\textsuperscript{191:2.2} Cuando el Jesús morontial les hubo dicho esto, desapareció de su vista en un instante. Todos cayeron de bruces, alabando a Dios y venerando a su desaparecido Maestro. Ésta fue la novena aparición morontial del Maestro.

\section*{3. Con los seres morontiales}
\par 
%\textsuperscript{(2040.4)}
\textsuperscript{191:3.1} Jesús pasó todo el día siguiente, lunes, con las criaturas morontiales entonces presentes en Urantia. Más de un millón de directores morontiales y sus asociados, así como los mortales de transición de diversas órdenes procedentes de los siete mundos de las mansiones de Satania, habían venido a Urantia para participar en la experiencia de transición morontial del Maestro. El Jesús morontial permaneció con estas espléndidas inteligencias durante cuarenta días. Los instruyó y aprendió de sus directores la vida de transición morontial tal como la atraviesan los mortales de los mundos habitados de Satania cuando pasan por las esferas morontiales del sistema.

\par 
%\textsuperscript{(2041.1)}
\textsuperscript{191:3.2} Alrededor de la medianoche de este lunes, la forma morontial del Maestro fue ajustada para la transición a la segunda fase de la evolución morontial. Cuando se apareció de nuevo a sus hijos mortales de la Tierra, era un ser morontial de la segunda fase. A medida que el Maestro progresaba en la carrera morontial, las inteligencias morontiales y sus asociados transformadores tenían cada vez más dificultades técnicas para hacer visible al Maestro a los ojos mortales y materiales.

\par 
%\textsuperscript{(2041.2)}
\textsuperscript{191:3.3} Jesús realizó el tránsito a la tercera fase morontial el viernes 14 de abril; a la cuarta el lunes 17; a la quinta el sábado 22; a la sexta el jueves 27; a la séptima el martes 2 de mayo; a la ciudadanía de Jerusem el domingo 7; y entró en el abrazo de los Altísimos de Edentia el domingo 14 de mayo.

\par 
%\textsuperscript{(2041.3)}
\textsuperscript{191:3.4} Miguel de Nebadon completó de esta manera su servicio de experiencia universal, puesto que en conexión con sus donaciones anteriores ya había experimentado por completo la vida de los mortales ascendentes del tiempo y del espacio, desde la estancia en la sede de la constelación hasta el servicio en la sede del superuniverso, y a través de dicho servicio. Precisamente gracias a estas experiencias morontiales, el Hijo Creador de Nebadon acabó realmente y terminó de manera aceptable su séptima y última donación en el universo.

\section*{4. La décima aparición (en Filadelfia)}
\par 
%\textsuperscript{(2041.4)}
\textsuperscript{191:4.1} La décima manifestación morontial de Jesús reconocida por los mortales tuvo lugar el martes 11 de abril, poco después de las ocho, en Filadelfia, donde se mostró a Abner, Lázaro y a unos ciento cincuenta de sus compañeros, incluídos más de cincuenta miembros del cuerpo evangélico de los setenta. Esta aparición se produjo en la sinagoga, poco después de la apertura de una reunión especial convocada por Abner para discutir la crucifixión de Jesús y la noticia más reciente de la resurrección, aportada por un mensajero de David. Puesto que el Lázaro resucitado ahora era miembro de este grupo de creyentes, no les resultaba difícil creer en la noticia de que Jesús había resucitado de entre los muertos.

\par 
%\textsuperscript{(2041.5)}
\textsuperscript{191:4.2} Abner y Lázaro, que estaban juntos en el púlpito, acababan de abrir la sesión en la sinagoga cuando toda la audiencia de creyentes vio aparecer repentinamente la forma del Maestro. Avanzó unos pasos desde donde había aparecido entre Abner y Lázaro, ninguno de los cuales lo había visto, saludó al grupo y dijo:

\par 
%\textsuperscript{(2041.6)}
\textsuperscript{191:4.3} <<Que la paz sea con vosotros. Todos sabéis que tenemos un solo Padre en el cielo y que sólo hay un evangelio del reino ---la buena nueva del don de la vida eterna que los hombres reciben por la fe. Mientras os regocijáis en vuestra lealtad al evangelio, rogad al Padre de la verdad que derrame en vuestro corazón un amor nuevo y más grande por vuestros hermanos. Debéis amar a todos los hombres como yo os he amado; debéis servir a todos los hombres como yo os he servido. Con una simpatía comprensiva y con un afecto fraternal, aceptad como compañeros a todos vuestros hermanos que se dedican a la proclamación de la buena nueva, ya sean judíos o gentiles, griegos o romanos, persas o etíopes. Juan proclamó el reino por adelantado; vosotros habéis predicado el evangelio con autoridad; los griegos enseñan ya la buena nueva; y yo voy a enviar pronto el Espíritu de la Verdad al alma de todos estos hermanos míos, que han dedicado su vida tan generosamente a iluminar a sus semejantes que están en las tinieblas espirituales. Todos sois los hijos de la luz; no tropecéis pues en los enredos de los malentendidos causados por la desconfianza y la intolerancia humana. Si la gracia de la fe os ennoblece para amar a los incrédulos, ¿no deberíais amar igualmente a aquellos que son vuestros compañeros creyentes en la gran familia de la fe? Recordad, en la medida en que os améis los unos a los otros, todos los hombres sabrán que sois mis discípulos>>.

\par 
%\textsuperscript{(2042.1)}
\textsuperscript{191:4.4} <<Id pues a proclamar por todo el mundo, a todas las naciones y razas, este evangelio de la paternidad de Dios y de la fraternidad de los hombres, y sed siempre sabios en la elección de vuestros métodos para presentar la buena nueva a las diferentes razas y tribus de la humanidad. Habéis recibido gratuitamente este evangelio del reino, y aportaréis gratuitamente la buena nueva a todas las naciones. No temáis la resistencia del mal porque siempre estoy con vosotros, incluso hasta el fin de los tiempos. Mi paz os dejo>>.

\par 
%\textsuperscript{(2042.2)}
\textsuperscript{191:4.5} Después de haber dicho <<Mi paz os dejo>>, desapareció de su vista. A excepción de una de sus apariciones en Galilea, donde más de quinientos creyentes lo vieron al mismo tiempo, este grupo de Filadelfia contenía la mayor cantidad de mortales que lo hubiera visto en una misma ocasión.

\par 
%\textsuperscript{(2042.3)}
\textsuperscript{191:4.6} A la mañana siguiente temprano, mientras los apóstoles permanecían en Jerusalén esperando que Tomás se recuperara emocionalmente, estos creyentes de Filadelfia salieron a proclamar que Jesús de Nazaret había resucitado de entre los muertos.

\par 
%\textsuperscript{(2042.4)}
\textsuperscript{191:4.7} El día siguiente, miércoles, Jesús lo pasó sin interrupción en compañía de sus asociados morontiales, y a media tarde recibió la visita de unos delegados morontiales procedentes de los mundos de las mansiones de todos los sistemas locales de esferas habitadas de toda la constelación de Norlatiadek. Y todos se regocijaron en reconocer a su Creador como miembro de su propia orden de inteligencias universales.

\section*{5. La segunda aparición a los apóstoles}
\par 
%\textsuperscript{(2042.5)}
\textsuperscript{191:5.1} Tomás pasó una triste semana completamente solo en las colinas que rodeaban al Olivete. Durante este tiempo sólo vio a Juan Marcos y a los que vivían en la casa de Simón. Eran alrededor de las nueve del sábado 15 de abril cuando los dos apóstoles lo encontraron y se lo llevaron de vuelta a su refugio en la casa de Marcos. Al día siguiente, Tomás escuchó el relato de las historias de las diversas apariciones del Maestro, pero se negó rotundamente a creer. Sostenía que Pedro, con su entusiasmo, los había convencido de que habían visto al Maestro. Natanael razonó con él, pero no sirvió de nada. Había una obstinación emotiva asociada a sus dudas habituales, y este estado mental, unido a su disgusto por haber huido de ellos, concurría a crear una situación de aislamiento que ni siquiera el mismo Tomás comprendía plenamente. Se había apartado de sus compañeros, había seguido su propio camino, y ahora, incluso estando de vuelta entre ellos, tendía inconscientemente a adoptar una actitud de desacuerdo. Era lento en rendirse; no le gustaba ceder. Aunque no tuviera esa intención, disfrutaba realmente con la atención que le prestaban; los esfuerzos de todos sus compañeros por convencerlo y convertirlo le producían una satisfacción inconsciente. Los había echado de menos durante toda una semana, y sus atenciones permanentes le causaban un gran placer.

\par 
%\textsuperscript{(2042.6)}
\textsuperscript{191:5.2} Poco después de las seis, estaban tomando la cena con Tomás sentado entre Pedro y Natanael, cuando el incrédulo apóstol dijo: <<No creeré hasta que haya visto al Maestro con mis propios ojos y haya metido mi dedo en la marca de los clavos>>. Mientras estaban así sentados cenando, con las puertas fuertemente cerradas y atrancadas, el Maestro morontial apareció repentinamente dentro de la curvatura de la mesa, y permaneciendo directamente delante de Tomás, dijo:

\par 
%\textsuperscript{(2043.1)}
\textsuperscript{191:5.3} <<Que la paz sea con vosotros. He esperado toda una semana a fin de poder aparecer de nuevo cuando todos estuvierais presentes para escuchar una vez más el encargo de ir a predicar por todo el mundo este evangelio del reino. Os lo digo de nuevo: Del mismo modo que el Padre me ha enviado al mundo, así os envío yo. Al igual que yo he revelado al Padre, vosotros revelaréis el amor divino, no solamente con las palabras, sino en vuestra vida diaria. Os envío, no para que améis el alma de los hombres, sino más bien para que \textit{améis a los hombres}. No debéis proclamar simplemente las alegrías del cielo, sino que debéis manifestar también en vuestra experiencia diaria estas realidades espirituales de la vida divina, puesto que gracias a la fe ya tenéis la vida eterna como un don de Dios. Cuando tengáis fe, cuando el poder de las alturas, el Espíritu de la Verdad, haya venido a vosotros, no esconderéis vuestra luz aquí detrás de unas puertas cerradas; haréis conocer a toda la humanidad el amor y la misericordia de Dios. Ahora huís, por miedo, de los hechos de una experiencia desagradable, pero cuando hayáis sido bautizados con el Espíritu de la Verdad, saldréis con valentía y alegría al encuentro de las nuevas experiencias que viviréis al proclamar la buena nueva de la vida eterna en el reino de Dios. Podéis permanecer aquí y en Galilea durante un corto período mientras os recobráis del impacto de la transición entre la falsa seguridad de la autoridad del tradicionalismo y el nuevo orden de la autoridad de los hechos, de la verdad y de la fe en las realidades supremas de la experiencia viviente. Vuestra misión en el mundo está basada en el hecho de que he vivido entre vosotros una vida revelando a Dios, está basada en la verdad de que vosotros y todos los demás hombres sois los hijos de Dios; y esta misión consistirá en la vida que viviréis entre los hombres ---en la experiencia real y viviente de amar y servir a los hombres como yo os he amado y servido. Que la fe revele vuestra luz al mundo; que la revelación de la verdad abra los ojos cegados por la tradición; que vuestro servicio amoroso destruya eficazmente los prejuicios engendrados por la ignorancia. Acercándoos así a vuestros semejantes con una simpatía comprensiva y con una dedicación desinteresada, los conduciréis al conocimiento salvador del amor del Padre. Los judíos han ensalzado la bondad; los griegos han exaltado la belleza; los hindúes predican la devoción; los lejanos ascetas enseñan la veneración; los romanos exigen la lealtad; pero yo exijo la vida de mis discípulos, incluso una vida de servicio amoroso para vuestros hermanos en la carne>>.

\par 
%\textsuperscript{(2043.2)}
\textsuperscript{191:5.4} Después de haber hablado así, el Maestro bajó la mirada hacia el rostro de Tomás y dijo: <<Y tú, Tomás, que has dicho que no creerías hasta que pudieras verme y meter tu dedo en las marcas de los clavos de mis manos, ahora me has contemplado y escuchado mis palabras; y aunque no veas ninguna marca de clavos en mis manos, puesto que he resucitado con una forma que tú también tendrás cuando te vayas de este mundo, ¿qué vas a decir a tus hermanos? Reconocerás la verdad, porque ya habías empezado a creer en tu corazón incluso cuando afirmabas tan categóricamente tu incredulidad. Tus dudas, Tomás, siempre se afirman con la mayor obstinación cuando están a punto de desmoronarse. Tomás, te ruego que no seas escéptico sino creyente ---y sé que creerás, incluso de todo corazón>>.

\par 
%\textsuperscript{(2043.3)}
\textsuperscript{191:5.5} Cuando Tomás escuchó estas palabras, cayó de rodillas delante del Maestro morontial y exclamó: <<¡Creo! ¡Señor mío y Maestro mío!>> Entonces Jesús le dijo a Tomás: <<Has creído, Tomás, porque me has visto y escuchado realmente. Benditos sean, en los siglos venideros, aquellos que creerán sin haber visto siquiera con los ojos de la carne ni haber escuchado con los oídos mortales>>.

\par 
%\textsuperscript{(2043.4)}
\textsuperscript{191:5.6} Luego, mientras la forma del Maestro se acercaba al extremo de la mesa, se dirigió a todos ellos diciendo: <<Y ahora, id todos a Galilea, donde pronto me apareceré a vosotros>>. Después de decir esto, desapareció de su vista.

\par 
%\textsuperscript{(2044.1)}
\textsuperscript{191:5.7} Los once apóstoles estaban ahora plenamente convencidos de que Jesús había resucitado de entre los muertos, y a la mañana siguiente muy temprano, antes del amanecer, partieron para Galilea.

\section*{6. La aparición en Alejandría}
\par 
%\textsuperscript{(2044.2)}
\textsuperscript{191:6.1} Mientras los once apóstoles iban camino de Galilea acercándose al final de su viaje, el martes 18 de abril hacia las ocho y media de la noche Jesús se apareció a Rodán y a unos ochenta creyentes más en Alejandría. Ésta era la duodécima aparición del Maestro en forma morontial. Jesús apareció ante estos griegos y judíos en el momento en que un mensajero de David terminaba su informe sobre la crucifixión. Este mensajero era el quinto corredor de relevo entre Jerusalén y Alejandría, y había llegado a Alejandría a últimas horas de aquella tarde; cuando hubo entregado su mensaje a Rodán, se decidió convocar a los creyentes para que recibieran esta trágica noticia de los labios mismos del mensajero. Alrededor de las ocho, el mensajero Natán de Busiris se presentó ante este grupo y les contó con detalle todo lo que el corredor anterior le había dicho a él. Natán terminó su conmovedor relato con estas palabras: <<Pero David, que nos envía esta noticia, informa que el Maestro, en el momento de predecir su muerte, declaró que resucitaría de nuevo>>. Mientras Natán hablaba todavía, el Maestro morontial apareció allí a la vista de todos. Y cuando Natán se sentó, Jesús dijo:

\par 
%\textsuperscript{(2044.3)}
\textsuperscript{191:6.2} <<Que la paz sea con vosotros. Lo que mi Padre me envió a establecer en el mundo no pertenece ni a una raza, ni a una nación, ni a un grupo especial de educadores o de predicadores. Este evangelio del reino pertenece tanto a los judíos como a los gentiles, a los ricos y a los pobres, a los libres y a los esclavos, a los hombres y a las mujeres, e incluso a los niños pequeños. Todos debéis proclamar este evangelio de amor y de verdad mediante la vida que vivís en la carne. Os amaréis los unos a los otros con un afecto nuevo y sorprendente, tal como yo os he amado. Serviréis a la humanidad con una devoción nueva y extraordinaria, tal como yo os he servido. Cuando los hombres vean que los amáis así, y cuando observen el fervor con que los servís, percibirán que sois hermanos por la fe en el reino de los cielos, y seguirán al Espíritu de la Verdad que verán en vuestra vida, hasta que encuentren la salvación eterna>>.

\par 
%\textsuperscript{(2044.4)}
\textsuperscript{191:6.3} <<Al igual que el Padre me ha enviado a este mundo, yo os envío a vosotros. Todos estáis llamados a llevar la buena nueva a aquellos que están en las tinieblas. Este evangelio del reino pertenece a todos los que crean en él; no será confiado al cuidado exclusivo de los sacerdotes. El Espíritu de la Verdad vendrá pronto a vosotros, y os conducirá a toda la verdad. Id pues por el mundo entero predicando este evangelio, y pensad que siempre estoy con vosotros, incluso hasta el fin de los tiempos>>.

\par 
%\textsuperscript{(2044.5)}
\textsuperscript{191:6.4} Después de haber hablado así, el Maestro desapareció de su vista. Estos creyentes permanecieron allí juntos toda la noche, contando sus experiencias como creyentes en el reino y escuchando las numerosas palabras de Rodán y de sus asociados. Y todos creyeron que Jesús había resucitado de entre los muertos. Un mensajero de David llegó dos días después para anunciarles la resurrección, e imaginad su sorpresa cuando respondieron a su anuncio: <<Sí, ya lo sabemos, porque le hemos visto. Anteayer se apareció a nosotros>>.


\chapter{Documento 192. Las apariciones en Galilea}
\par 
%\textsuperscript{(2045.1)}
\textsuperscript{192:0.1} CUANDO los apóstoles salieron de Jerusalén hacia Galilea, los dirigentes judíos se habían tranquilizado considerablemente. Puesto que Jesús sólo se aparecía a su familia de creyentes en el reino, y como los apóstoles estaban escondidos y no hacían ninguna predicación pública, los jefes de los judíos concluyeron que, después de todo, el movimiento del evangelio estaba eficazmente aplastado. Por supuesto, estaban desconcertados por la creciente difusión de los rumores de que Jesús había resucitado de entre los muertos, pero contaban con los guardias sobornados para contrarrestar eficazmente todas estas noticias repitiendo la historia de que un grupo de seguidores de Jesús se había llevado el cuerpo.

\par 
%\textsuperscript{(2045.2)}
\textsuperscript{192:0.2} A partir de este momento y hasta que los apóstoles fueron dispersados por la marea creciente de las persecuciones, Pedro fue reconocido de manera general como jefe del cuerpo apostólico. Jesús nunca le confirió esta autoridad, y sus compañeros apóstoles nunca lo eligieron oficialmente para este puesto de responsabilidad; Pedro lo asumió de manera natural y lo conservó por consentimiento general, y también porque era el principal predicador de todos ellos. Desde ahora en adelante, la predicación pública se convirtió en la tarea fundamental de los apóstoles. Después de regresar de Galilea, Matías, a quien habían elegido para sustituir a Judas, se convirtió en su tesorero.

\par 
%\textsuperscript{(2045.3)}
\textsuperscript{192:0.3} Durante la semana que permanecieron en Jerusalén, María la madre de Jesús pasó mucho tiempo con las mujeres creyentes que estaban alojadas en la casa de José de Arimatea.

\par 
%\textsuperscript{(2045.4)}
\textsuperscript{192:0.4} Cuando los apóstoles partieron para Galilea este lunes por la mañana temprano, Juan Marcos salió tras ellos. Los siguió fuera de la ciudad, y cuando se encontraban mucho más allá de Betania, se presentó audazmente entre ellos, confiando en que no lo enviarían para atrás.

\par 
%\textsuperscript{(2045.5)}
\textsuperscript{192:0.5} Los apóstoles se detuvieron varias veces en el camino de Galilea para contar la historia de su Maestro resucitado, y por eso no llegaron a Betsaida hasta el miércoles por la noche muy tarde. Ya era mediodía del jueves cuando todos se despertaron y se prepararon para tomar el desayuno.

\section*{1. La aparición cerca del lago}
\par 
%\textsuperscript{(2045.6)}
\textsuperscript{192:1.1} El viernes 21 de abril hacia las seis de la mañana, el Maestro morontial efectuó su decimotercera aparición, la primera en Galilea, a los diez apóstoles cuando acercaban su barca a la orilla, cerca del desembarcadero habitual de Betsaida.

\par 
%\textsuperscript{(2045.7)}
\textsuperscript{192:1.2} El jueves, después de que los apóstoles hubieron pasado la tarde y las primeras horas de la noche esperando en la casa de Zebedeo, Simón Pedro sugirió que fueran a pescar. Cuando Pedro propuso esta jornada de pesca, todos los apóstoles decidieron ir. Se afanaron toda la noche con las redes, pero no atraparon ningún pez. No se preocuparon mucho por no haber pescado nada, pues tenían muchas experiencias interesantes sobre las que hablar, todas las cosas que tan recientemente les habían sucedido en Jerusalén. Pero cuando llegó la luz del día, decidieron volver a Betsaida. Al acercarse a la orilla, vieron a alguien en la playa, cerca del desembarcadero, de pie al lado de un fuego. Al principio creyeron que se trataba de Juan Marcos, que había bajado a recibirlos cuando regresaban con la pesca, pero al acercarse más a la orilla vieron que se habían equivocado ---el hombre era demasiado alto para ser Juan. A ninguno se le había ocurrido que la persona que estaba en la playa fuera el Maestro. No comprendían del todo por qué Jesús quería encontrarse con ellos entre los paisajes de sus primeras relaciones y al aire libre en contacto con la naturaleza, lejos del ambiente cerrado de Jerusalén, con sus trágicas asociaciones de miedo, de traición y de muerte. Les había dicho que, si iban a Galilea, se encontraría con ellos allí, y estaba a punto de cumplir esta promesa.

\par 
%\textsuperscript{(2046.1)}
\textsuperscript{192:1.3} Mientras echaban el ancla y se preparaban para subir al bote pequeño con el fin de desembarcar, el hombre que estaba en la playa les gritó: <<Muchachos, ¿habéis pescado algo?>> Cuando respondieron que no, el hombre dijo de nuevo: <<Echad la red a la derecha de la barca y encontraréis los peces>>. Aunque no sabían que era Jesús el que les había orientado, echaron la red al unísono tal como les había indicado, y se llenó inmediatamente de tal manera que casi no podían sacarla. Pero Juan Zebedeo era de percepción rápida, y cuando vio la red cargada hasta los topes, percibió que era el Maestro el que les había hablado. Cuando este pensamiento le vino a la cabeza, se inclinó hacia Pedro y le dijo en voz baja: <<Es el Maestro>>. Pedro fue siempre un hombre de acción irreflexiva y de devoción impetuosa, de manera que, en cuanto Juan le susurró esto al oído, se levantó rápidamente y se arrojó al agua para poder llegar cuanto antes al lado del Maestro. Sus hermanos llegaron inmediatamente después de él, alcanzando la orilla en la barca pequeña y arrastrando la red de peces detrás de ellos.

\par 
%\textsuperscript{(2046.2)}
\textsuperscript{192:1.4} Mientras tanto, Juan Marcos se había levantado, y al ver que los apóstoles llegaban a la orilla con la red cargada hasta los topes, corrió por la playa abajo para saludarlos. Cuando vio a once hombres en lugar de diez, supuso que el desconocido era Jesús resucitado, y mientras los diez hombres asombrados permanecían allí en silencio, el joven se precipitó hacia el Maestro, se arrodilló a sus pies, y dijo: <<Señor mío y Maestro mío>>. Entonces Jesús habló, no como lo había hecho en Jerusalén cuando los saludó diciendo <<Que la paz sea con vosotros>>, sino que se dirigió a Juan Marcos en un tono familiar: <<Bien, Juan, me alegro de verte de nuevo en la despreocupada Galilea, donde podremos tener una buena conversación. Quédate con nosotros, Juan, y desayuna>>.

\par 
%\textsuperscript{(2046.3)}
\textsuperscript{192:1.5} Mientras Jesús hablaba con el joven, los diez estaban tan asombrados y sorprendidos que se olvidaron de arrastrar la red de peces hasta la playa. Jesús dijo entonces: <<Traed vuestros peces y preparad algunos para el desayuno. Ya tenemos el fuego y mucho pan>>.

\par 
%\textsuperscript{(2046.4)}
\textsuperscript{192:1.6} Mientras Juan Marcos rendía homenaje al Maestro, Pedro se sobresaltó por un momento a la vista de las brasas que resplandecían allí en la playa; la escena le recordó vivamente el fuego de carbón de leña a medianoche en el patio de Anás, donde había negado al Maestro. Pero se repuso, y arrodillándose a los pies del Maestro, exclamó: <<¡Señor mío y Maestro mío!>>

\par 
%\textsuperscript{(2046.5)}
\textsuperscript{192:1.7} Pedro se unió luego a sus compañeros que arrastraban la red. Cuando llevaron a tierra su captura, contaron los peces, y había 153 grandes. Y de nuevo se cometió el error de llamarle a esto otra pesca milagrosa. No hubo ningún milagro asociado a este episodio. El Maestro simplemente había ejercido su preconocimiento. Sabía que los peces estaban allí, y en consecuencia, indicó a los apóstoles dónde debían echar la red.

\par 
%\textsuperscript{(2047.1)}
\textsuperscript{192:1.8} Jesús les habló diciendo: <<Ahora, venid todos a desayunar. Incluso los gemelos deberían sentarse mientras charlo con vosotros; Juan Marcos preparará los peces>>. Juan Marcos trajo siete peces de buen tamaño que el Maestro puso en el fuego, y cuando estuvieron asados, el muchacho los sirvió a los diez. Entonces, Jesús partió el pan y se lo entregó a Juan que, a su vez, lo sirvió a los hambrientos apóstoles. Cuando todos estuvieron servidos, Jesús le rogó a Juan Marcos que se sentara mientras él mismo servía el pescado y el pan al muchacho. Mientras comían, Jesús charló con ellos, recordando sus numerosas experiencias comunes en Galilea y al lado de este mismo lago.

\par 
%\textsuperscript{(2047.2)}
\textsuperscript{192:1.9} Ésta era la tercera vez que Jesús se manifestaba a los apóstoles como grupo. Cuando Jesús se dirigió a ellos al principio preguntándoles si habían pescado, no sospecharon quien era porque para estos pescadores del Mar de Galilea era una experiencia corriente, cuando llegaban a la orilla, que los mercaderes de pescado de Tariquea los abordaran así, ya que habitualmente estaban dispuestos a comprar la pesca fresca para los establecimientos de desecación.

\par 
%\textsuperscript{(2047.3)}
\textsuperscript{192:1.10} Jesús conversó con los diez apóstoles y Juan Marcos durante más de una hora; luego se paseó de un lado a otro de la playa hablando con ellos de dos en dos ---pero no eran las mismas parejas que al principio había enviado juntas a enseñar. Los once apóstoles habían venido juntos desde Jerusalén, pero a medida que se acercaban a Galilea, Simón Celotes se había desalentado cada vez más, de manera que cuando llegaron a Betsaida, dejó a sus hermanos y regresó a su casa.

\par 
%\textsuperscript{(2047.4)}
\textsuperscript{192:1.11} Antes de despedirse de ellos esta mañana, Jesús les encargó que dos apóstoles se ofrecieran voluntarios para ir a por Simón Celotes y lo trajeran de vuelta aquel mismo día. Y esto es lo que hicieron Pedro y Andrés.

\section*{2. Las conversaciones con los apóstoles de dos en dos}
\par 
%\textsuperscript{(2047.5)}
\textsuperscript{192:2.1} Cuando terminaron de desayunar, y mientras los demás permanecían sentados al lado del fuego, Jesús hizo señas a Pedro y a Juan para que le acompañaran a dar un paseo por la playa. Mientras caminaban, Jesús le dijo a Juan: <<Juan, ¿me amas?>> Y cuando Juan contestó: <<Sí, Maestro, con todo mi corazón>>, el Maestro dijo: <<Entonces, Juan, abandona tu intolerancia y aprende a amar a los hombres como yo te he amado. Dedica tu vida a demostrar que el amor es la cosa más grande del mundo. Es el amor de Dios el que impulsa a los hombres a buscar la salvación. El amor es el padre de toda bondad espiritual, la esencia de lo verdadero y de lo bello>>.

\par 
%\textsuperscript{(2047.6)}
\textsuperscript{192:2.2} Jesús se volvió entonces hacia Pedro y le preguntó: <<Pedro, ¿me amas?>> Pedro contestó: <<Señor, tú sabes que te amo con toda mi alma>>. Entonces dijo Jesús: <<Si me amas, Pedro, apacienta mis corderos. No descuides ayudar a los débiles, a los pobres y a los jóvenes. Predica el evangelio sin temor ni favor; recuerda siempre que Dios no hace acepción de personas. Sirve a tus semejantes como yo te he servido; perdona a tus compañeros mortales como yo te he perdonado. Que la experiencia te enseñe el valor de la meditación y el poder de la reflexión inteligente>>.

\par 
%\textsuperscript{(2047.7)}
\textsuperscript{192:2.3} Después de caminar un poco más, el Maestro se volvió hacia Pedro y le preguntó: <<Pedro, ¿realmente me amas?>> Y entonces dijo Simón: <<Sí, Señor, tú sabes que te amo>>. Y Jesús dijo de nuevo: <<Entonces, cuida bien a mis ovejas. Sé un pastor bueno y verdadero para el rebaño. No traiciones su confianza en ti. No te dejes sorprender por el enemigo. Permanece alerta en todo momento ---vigila y ora>>.

\par 
%\textsuperscript{(2047.8)}
\textsuperscript{192:2.4} Cuando dieron unos cuantos pasos más, Jesús se volvió hacia Pedro y le preguntó por tercera vez: <<Pedro, ¿me amas de verdad?>> Entonces Pedro, ligeramente herido por la aparente desconfianza del Maestro, dijo con una gran emoción: <<Señor, tú lo sabes todo, y sabes por tanto que te amo realmente y de verdad>>. Entonces dijo Jesús: <<Apacienta mis ovejas. No abandones al rebaño. Sé un ejemplo y una inspiración para todos tus compañeros pastores. Ama al rebaño como yo te he amado y conságrate a su bienestar como yo he consagrado mi vida a tu bienestar. Y sígueme hasta el fin>>.

\par 
%\textsuperscript{(2048.1)}
\textsuperscript{192:2.5} Pedro interpretó esta última declaración al pie de la letra ---que debía continuar detrás de Jesús--- y volviéndose hacia él, señaló con el dedo a Juan y preguntó: <<Si continúo detrás de ti, ¿qué hará éste?>> Entonces, al percibir que Pedro había entendido mal sus palabras, Jesús dijo: <<Pedro, no te preocupes por lo que hagan tus hermanos. Si quiero que Juan se quede después de que te hayas ido, o incluso hasta que yo vuelva, ¿en qué te concierne a ti? Asegúrate solamente de que me sigues>>.

\par 
%\textsuperscript{(2048.2)}
\textsuperscript{192:2.6} Este comentario se difundió entre los hermanos y fue recibido como una declaración de Jesús de que Juan no moriría antes de que regresara el Maestro para establecer el reino con poder y gloria, como muchos pensaban y esperaban. Esta interpretación de lo que Jesús había dicho contribuyó mucho a que Simón Celotes regresara al servicio y permaneciera trabajando.

\par 
%\textsuperscript{(2048.3)}
\textsuperscript{192:2.7} Cuando regresaron donde estaban los demás, Jesús se fue a pasear y a hablar con Andrés y Santiago. Después de recorrer una corta distancia, Jesús le dijo a Andrés: <<Andrés, ¿confías en mí?>> Cuando el antiguo jefe de los apóstoles escuchó a Jesús hacerle esta pregunta, se detuvo y contestó: <<Sí, Maestro, es evidente que confío en ti, y sabes que es así>>. Entonces dijo Jesús: <<Andrés, si confías en mí, confía más en tus hermanos ---incluso en Pedro. Hace tiempo te confié la dirección de tus hermanos. Ahora debes confiar en los demás mientras os dejo para ir hacia el Padre. Cuando tus hermanos empiecen a dispersarse debido a las crueles persecuciones, sé un consejero sabio y prudente para Santiago, mi hermano carnal, cuando le asignen unas pesadas cargas que no está capacitado para llevar por falta de experiencia. Y luego continúa confiando, porque yo no te fallaré. Cuando hayas terminado en la Tierra, vendrás hacia mí>>.

\par 
%\textsuperscript{(2048.4)}
\textsuperscript{192:2.8} Luego Jesús se volvió hacia Santiago, preguntando: <<Santiago ¿confías en mí?>> Y Santiago contestó por supuesto: <<Sí, Maestro, confío en ti con todo mi corazón>>. Entonces dijo Jesús: <<Santiago, si confías más en mí, serás menos impaciente con tus hermanos. Si quieres confiar en mí, eso te ayudará a ser bondadoso con la fraternidad de los creyentes. Aprende a estimar las consecuencias de tus palabras y de tus actos. Recuerda que se cosecha aquello que se siembra. Reza por la tranquilidad de espíritu y cultiva la paciencia. Estas gracias, junto con la fe viviente, te sostendrán cuando llegue la hora de beber la copa del sacrificio. Pero no te desanimes nunca; cuando hayas terminado en la Tierra, también vendrás para estar conmigo>>.

\par 
%\textsuperscript{(2048.5)}
\textsuperscript{192:2.9} Jesús habló a continuación con Tomás y Natanael. A Tomás le dijo: <<Tomás, ¿me sirves?>> Tomás contestó: <<Sí, Señor, te sirvo ahora y siempre>>. Entonces dijo Jesús: <<Si quieres servirme, sirve a mis hermanos en la carne como yo te he servido. Y no te canses de obrar bien, sino que persevera como alguien que ha sido ordenado por Dios para realizar este servicio de amor. Cuando hayas terminado tu servicio conmigo en la Tierra, servirás conmigo en la gloria. Tomás, debes dejar de dudar; debes crecer en la fe y en el conocimiento de la verdad. Cree en Dios como un niño, pero deja de actuar de manera tan infantil. Ten coraje; sé fuerte en la fe y poderoso en el reino de Dios>>.

\par 
%\textsuperscript{(2049.1)}
\textsuperscript{192:2.10} Entonces el Maestro le dijo a Natanael: <<Natanael, ¿me sirves?>> Y el apóstol contestó: <<Sí, Maestro, y con todo mi afecto>>. Entonces dijo Jesús: <<Si me sirves pues de todo corazón, asegúrate de que te consagras al bienestar de mis hermanos en la Tierra con un afecto incansable. Incorpora la amistad a tu consejo y añade el amor a tu filosofía. Sirve a tus semejantes como yo te he servido. Sé fiel a los hombres como yo he velado por ti. Sé menos crítico; espera menos de algunos hombres y disminuye así la magnitud de tus decepciones. Y cuando el trabajo aquí abajo haya terminado, servirás conmigo en el cielo>>.

\par 
%\textsuperscript{(2049.2)}
\textsuperscript{192:2.11} Después de esto, el Maestro conversó con Mateo y Felipe. A Felipe le dijo: <<Felipe, ¿me obedeces?>> Felipe respondió: <<Sí, Señor, te obedeceré incluso con mi vida>>. Entonces dijo Jesús: <<Si quieres obedecerme, ve pues a los países de los gentiles y proclama este evangelio. Los profetas te han dicho que es mejor obedecer que sacrificar. Te has convertido, por la fe, en un hijo del reino que conoce a Dios. Sólo hay una ley que obedecer ---y es el mandamiento de salir a proclamar el evangelio del reino. Deja de temer a los hombres; no tengas miedo de predicar la buena nueva de la vida eterna a tus semejantes que languidecen en las tinieblas y ansían la luz de la verdad. Felipe, ya no tendrás que ocuparte del dinero ni de los bienes. Ahora eres libre de predicar la buena nueva exactamente igual que tus hermanos. Iré delante de ti y estaré contigo hasta el fin>>.

\par 
%\textsuperscript{(2049.3)}
\textsuperscript{192:2.12} Luego, el Maestro se dirigió a Mateo y le preguntó: <<Mateo, ¿albergas en tu corazón el deseo de obedecerme?>> Mateo respondió: <<Sí, Señor, estoy plenamente dedicado a hacer tu voluntad>>. Entonces dijo el Maestro: <<Mateo, si quieres obedecerme, sal a enseñar a todos los pueblos este evangelio del reino. Ya no proporcionarás más a tus hermanos las cosas materiales de la vida; de ahora en adelante también deberás proclamar la buena nueva de la salvación espiritual. A partir de ahora ten el único propósito de obedecer tu encargo de predicar este evangelio del reino del Padre. Al igual que yo he hecho la voluntad del Padre en la Tierra, tú cumplirás la misión divina. Recuerda, tanto los judíos como los gentiles son tus hermanos. No temas a nadie cuando proclames las verdades salvadoras del evangelio del reino de los cielos. Y allí donde voy, dentro de poco vendrás tú>>.

\par 
%\textsuperscript{(2049.4)}
\textsuperscript{192:2.13} Después se paseó y habló con Santiago y Judas, los gemelos Alfeo; dirigiéndose a los dos a la vez, les preguntó: <<Santiago y Judas, ¿creéis en mí?>> Y cuando los dos contestaron: <<Sí, Maestro, creemos>>, Jesús dijo: <<Pronto voy a dejaros. Veis que ya os he dejado de manera carnal. Sólo permaneceré poco tiempo con esta forma antes de ir hacia mi Padre. Creéis en mí ---sois mis apóstoles, y siempre lo seréis. Continuad creyendo y recordando vuestra asociación conmigo cuando me haya ido, y después de que quizás hayáis regresado al trabajo que hacíais antes de venir a vivir conmigo. No permitáis nunca que un cambio en vuestro trabajo exterior influya en vuestra lealtad. Tened fe en Dios hasta el final de vuestros días en la Tierra. No olvidéis nunca que cuando uno es un hijo de Dios por la fe, todo trabajo honrado en la Tierra es sagrado. Nada de lo que hace un hijo de Dios puede ser corriente. De ahora en adelante, haced pues vuestro trabajo como si fuera para Dios. Y cuando hayáis terminado en este mundo, tengo otros mundos mejores donde trabajaréis igualmente para mí. En todo este trabajo, en este mundo y en los otros, yo trabajaré con vosotros y mi espíritu residirá dentro de vosotros>>.

\par 
%\textsuperscript{(2049.5)}
\textsuperscript{192:2.14} Eran casi las diez cuando Jesús regresó de su conversación con los gemelos Alfeo. Al dejar a los apóstoles, les dijo: <<Adiós, hasta que os vea a todos mañana al mediodía en el monte de vuestra ordenación>>. Después de hablar así, desapareció de su vista.

\section*{3. En el monte de la ordenación}
\par 
%\textsuperscript{(2050.1)}
\textsuperscript{192:3.1} El sábado 22 de abril al mediodía, los once apóstoles se reunieron tal como habían acordado en la colina cerca de Cafarnaúm, y Jesús apareció entre ellos. Esta reunión tuvo lugar en el mismo monte donde el Maestro los había seleccionado como apóstoles suyos y como embajadores del reino del Padre en la Tierra. Ésta era la decimocuarta manifestación morontial del Maestro.

\par 
%\textsuperscript{(2050.2)}
\textsuperscript{192:3.2} En esta ocasión, los once apóstoles se arrodillaron en círculo alrededor del Maestro; le oyeron repetir sus misiones y le vieron reproducir la escena de la ordenación como cuando fueron seleccionados por primera vez para el trabajo especial del reino. Todo esto fue para ellos como un recordatorio de su consagración anterior al servicio del Padre, excepto la oración del Maestro. Cuando el Maestro ---el Jesús morontial--- oró este día, lo hizo con tal tono de majestad y con tales palabras de autoridad como los apóstoles no lo habían escuchado nunca anteriormente. Su Maestro hablaba ahora con los gobernantes de los universos como alguien en cuyas manos se habían depositado todos los poderes y toda la autoridad sobre su propio universo. Estos once hombres no olvidaron nunca esta experiencia de reconsagración morontial a sus compromisos anteriores como embajadores. El Maestro pasó exactamente una hora con sus embajadores en este monte, y después de despedirse afectuosamente de ellos, desapareció de su vista.

\par 
%\textsuperscript{(2050.3)}
\textsuperscript{192:3.3} Nadie vio a Jesús durante una semana entera. Los apóstoles no tenían realmente ninguna idea de lo que debían hacer, pues no sabían si el Maestro había ido hacia el Padre. En este estado de incertidumbre, permanecieron en Betsaida. No se atrevían a salir a pescar por temor a que viniera a visitarlos y no consiguieran verlo. Durante toda esta semana, Jesús estuvo ocupado con las criaturas morontiales que se encontraban en la Tierra y con los asuntos de la transición morontial que estaba experimentando en este mundo.

\section*{4. La reunión a la orilla del lago}
\par 
%\textsuperscript{(2050.4)}
\textsuperscript{192:4.1} La noticia de las apariciones de Jesús se estaba difundiendo por toda Galilea, y cada día llegaban más creyentes a la casa de Zebedeo para informarse sobre la resurrección del Maestro y averiguar la verdad sobre estas supuestas apariciones. A principios de la semana, Pedro hizo saber que el sábado siguiente a las tres de la tarde se celebraría una reunión pública a la orilla del mar.

\par 
%\textsuperscript{(2050.5)}
\textsuperscript{192:4.2} En consecuencia, el sábado 29 de abril a las tres de la tarde, más de quinientos creyentes de los alrededores de Cafarnaúm se reunieron en Betsaida para escuchar a Pedro predicar su primer sermón público desde la resurrección. El apóstol estaba en su mejor momento, y después de terminar su atractivo discurso, pocos oyentes suyos dudaron de que el Maestro había resucitado de entre los muertos.

\par 
%\textsuperscript{(2050.6)}
\textsuperscript{192:4.3} Pedro terminó su sermón diciendo: <<Afirmamos que Jesús de Nazaret no está muerto; declaramos que ha salido de la tumba; proclamamos que lo hemos visto y que hemos hablado con él>>. En el preciso momento en que terminaba de efectuar esta declaración de fe, el Maestro apareció en forma morontial allí a su lado, plenamente a la vista de toda aquella gente, y les habló en un tono familiar, diciendo: <<Que la paz sea con vosotros, y mi paz os dejo>>. Después de aparecer así y de hablarles de esta manera, desapareció de su vista. Ésta fue la decimoquinta manifestación morontial del Jesús resucitado.

\par 
%\textsuperscript{(2051.1)}
\textsuperscript{192:4.4} Debido a ciertas cosas que el Maestro había dicho a los once durante la conferencia en el monte de la ordenación, los apóstoles tuvieron la impresión de que su Maestro haría pronto una aparición pública delante de un grupo de creyentes galileos, y que después de esto debían regresar a Jerusalén. En consecuencia, al día siguiente, domingo 30 de abril, los once partieron temprano de Betsaida hacia Jerusalén. Enseñaron y predicaron bastante por el camino que descendía junto al Jordán, de manera que no llegaron a la casa de los Marcos, en Jerusalén, hasta el miércoles 3 de mayo ya tarde.

\par 
%\textsuperscript{(2051.2)}
\textsuperscript{192:4.5} Para Juan Marcos fue un triste regreso al hogar. Pocas horas antes de llegar a su casa, su padre, Elías Marcos, había muerto repentinamente de una hemorragia cerebral. La certidumbre de la resurrección de los muertos contribuyó mucho a consolar el dolor de los apóstoles, pero al mismo tiempo se afligieron sinceramente por la pérdida de su buen amigo, que los había apoyado incondicionalmente incluso en los momentos de las mayores dificultades y decepciones. Juan Marcos hizo todo lo que pudo por consolar a su madre, y hablando en nombre de ella, invitó a los apóstoles a que continuaran sintiéndose como en su hogar en la casa de ella. Y los once instalaron su cuartel general en la habitación de arriba hasta después del día de Pentecostés.

\par 
%\textsuperscript{(2051.3)}
\textsuperscript{192:4.6} Los apóstoles habían entrado adrede en Jerusalén después de la caída de la noche para no ser vistos por las autoridades judías. Tampoco aparecieron en público en el momento del funeral de Elías Marcos. Todo el día siguiente permanecieron aislados tranquilamente en esta memorable habitación de la parte superior.

\par 
%\textsuperscript{(2051.4)}
\textsuperscript{192:4.7} El jueves por la noche, los apóstoles tuvieron una maravillosa reunión en esta habitación de arriba, y todos se comprometieron a salir a predicar públicamente el nuevo evangelio del Señor resucitado, excepto Tomás, Simón Celotes y los gemelos Alfeo. Ya se estaban dando los primeros pasos para sustituir el evangelio del reino ---la filiación con Dios y la fraternidad con los hombres--- por la proclamación de la resurrección de Jesús. Natanael se opuso a este cambio en la esencia de su mensaje público, pero no pudo oponerse a la elocuencia de Pedro ni pudo vencer el entusiasmo de los discípulos, especialmente de las mujeres creyentes.

\par 
%\textsuperscript{(2051.5)}
\textsuperscript{192:4.8} Y así, bajo la vigorosa dirección de Pedro, y antes de que el Maestro ascendiera hacia el Padre, sus representantes bien intencionados emprendieron este proceso sutil de sustituir de manera gradual y segura la religión \textit{de} Jesús por una forma nueva y modificada de religión \textit{acerca de} Jesús.


\chapter{Documento 193. Las apariciones finales y la ascensión}
\par 
%\textsuperscript{(2052.1)}
\textsuperscript{193:0.1} LA DECIMOSEXTA manifestación morontial de Jesús tuvo lugar el viernes 5 de mayo, hacia las nueve de la noche, en el patio de Nicodemo. Esta noche, los creyentes de Jerusalén habían realizado su primer intento por reunirse después de la resurrección. En este momento se encontraban congregados aquí los once apóstoles, el cuerpo de mujeres y sus asociadas, y aproximadamente otros cincuenta discípulos principales del Maestro, incluyendo a varios griegos. Este grupo de creyentes había estado conversando familiarmente durante más de media hora cuando de pronto, el Maestro morontial apareció plenamente a la vista de todos y empezó de inmediato a instruirlos. Jesús dijo:

\par 
%\textsuperscript{(2052.2)}
\textsuperscript{193:0.2} <<Que la paz sea con vosotros. Éste es el grupo de creyentes más representativo ---apóstoles y discípulos, tanto hombres como mujeres--- al que me he aparecido desde el momento en que fui liberado de la carne. Ahora os tomo por testigos de que os había dicho de antemano que mi estancia entre vosotros debía llegar a su fin. Os dije que pronto debía regresar hacia el Padre. Y luego os dije claramente de qué manera los jefes de los sacerdotes y los dirigentes de los judíos me entregarían para ser ejecutado, y que saldría de la tumba. ¿Por qué, entonces, os habéis desconcertado tanto por todo esto, cuando ha sucedido? ¿Y por qué estabais tan sorprendidos cuando resucité de la tumba al tercer día? No lograsteis creerme porque escuchabais mis palabras sin comprender su significado>>.

\par 
%\textsuperscript{(2052.3)}
\textsuperscript{193:0.3} <<Ahora deberíais prestar oído a mis palabras para no cometer de nuevo el error de escuchar mi enseñanza con la mente, sin comprender su significado en vuestro corazón. Desde el principio de mi estancia aquí como uno de vosotros, os enseñé que mi única finalidad era revelar mi Padre que está en los cielos a sus hijos de la Tierra. He vivido la donación de revelar a Dios para que podáis experimentar la carrera de conocer a Dios. He revelado a Dios como vuestro Padre que está en los cielos; os he revelado que sois los hijos de Dios en la Tierra. Es un hecho que Dios os ama a vosotros, sus hijos. Por la fe en mis palabras, este hecho se vuelve una verdad eterna y viviente en vuestro corazón. Cuando, por la fe viviente, os volvéis divinamente conscientes de Dios, entonces nacéis del espíritu como hijos de la luz y de la vida, de la misma vida eterna con la que ascenderéis el universo de universos y lograréis la experiencia de encontrar a Dios Padre en el Paraíso>>.

\par 
%\textsuperscript{(2052.4)}
\textsuperscript{193:0.4} <<Os exhorto a que recordéis siempre que vuestra misión entre los hombres consiste en proclamar el evangelio del reino ---la realidad de la paternidad de Dios y la verdad de la filiación de los hombres. Proclamad la verdad total de la buena nueva, y no solamente una parte del evangelio salvador. Vuestro mensaje no ha cambiado debido a la experiencia de mi resurrección. La filiación con Dios, por la fe, sigue siendo la verdad salvadora del evangelio del reino. Debéis salir a predicar el amor de Dios y el servicio a los hombres. Lo que el mundo más necesita saber es que los hombres son hijos de Dios, y que pueden comprender realmente por la fe esta verdad ennoblecedora, y experimentarla diariamente. Mi donación debería ayudar a todos los hombres a saber que son hijos de Dios, pero este conocimiento será insuficiente si no logran captar personalmente, por la fe, la verdad salvadora de que son los hijos espirituales vivientes del Padre eterno. El evangelio del reino se ocupa del amor del Padre y del servicio a sus hijos en la Tierra>>.

\par 
%\textsuperscript{(2053.1)}
\textsuperscript{193:0.5} <<Aquí compartís entre vosotros el conocimiento de que he resucitado de entre los muertos, pero esto no es algo extraordinario. Tengo el poder de abandonar mi vida y de recuperarla de nuevo; el Padre confiere este poder a sus Hijos Paradisiacos. Vuestro corazón debería conmoverse más bien con el conocimiento de que los muertos de una era han emprendido la ascensión eterna poco después de que yo saliera de la tumba nueva de José. He vivido mi vida en la carne para mostraros cómo podéis ser, a través del servicio amoroso, una revelación de Dios para vuestros semejantes, al igual que yo he sido, amándoos y sirviéndoos, una revelación de Dios para vosotros. He vivido entre vosotros como el Hijo del Hombre para que vosotros, y todos los demás hombres, podáis saber que todos sois en verdad los hijos de Dios. Por eso, id ahora por el mundo entero predicando este evangelio del reino de los cielos a todos los hombres. Amad a todos los hombres como yo os he amado; servid a vuestros compañeros mortales como yo os he servido. Habéis recibido gratuitamente, dad gratuitamente. Permaneced aquí en Jerusalén solamente mientras voy hacia el Padre y hasta que os envíe el Espíritu de la Verdad. Él os guiará hacia una verdad más amplia, y yo iré con vosotros por todo el mundo. Siempre estoy con vosotros, y mi paz os dejo>>.

\par 
%\textsuperscript{(2053.2)}
\textsuperscript{193:0.6} Cuando el Maestro les hubo hablado, desapareció de su vista. Estos creyentes no se dispersaron hasta cerca del alba; permanecieron juntos toda la noche discutiendo seriamente las recomendaciones del Maestro y meditando sobre todo lo que les había sucedido. Santiago Zebedeo y otros apóstoles les contaron también sus experiencias con el Maestro morontial en Galilea, y refirieron cómo se les había aparecido tres veces.

\section*{1. La aparición en Sicar}
\par 
%\textsuperscript{(2053.3)}
\textsuperscript{193:1.1} El sábado 13 de mayo hacia las cuatro de la tarde, el Maestro se apareció a Nalda y a unos setenta y cinco creyentes samaritanos cerca del pozo de Jacob, en Sicar. Los creyentes tenían la costumbre de reunirse en este lugar, cerca del cual Jesús le había hablado a Nalda sobre el agua de la vida. Este día, justo en el momento en que habían terminado de discutir la noticia de la resurrección, Jesús apareció repentinamente delante de ellos, diciendo:

\par 
%\textsuperscript{(2053.4)}
\textsuperscript{193:1.2} <<Que la paz sea con vosotros. Os alegráis de saber que yo soy la resurrección y la vida, pero esto no os servirá de nada si no nacéis primero del espíritu eterno, llegando a poseer así, por la fe, el don de la vida eterna. Si sois los hijos de mi Padre por la fe, no moriréis nunca, no pereceréis. El evangelio del reino os ha enseñado que todos los hombres son hijos de Dios. Y esta buena nueva relativa al amor del Padre celestial por sus hijos de la Tierra debe ser llevada por el mundo entero. Ha llegado la hora en que no adoraréis a Dios ni en Gerizim ni en Jerusalén, sino allí donde estéis, tal como estéis, en espíritu y en verdad. Vuestra fe es la que salva vuestra alma. La salvación es el don de Dios para todos los que creen que son sus hijos. Pero no os engañéis; aunque la salvación es el don gratuito de Dios y se concede a todos los que la aceptan por la fe, a ello le sigue la experiencia de producir los frutos de la vida espiritual tal como ésta se vive en la carne. La aceptación de la doctrina de la paternidad de Dios implica que también aceptáis libremente la verdad asociada de la fraternidad de los hombres. Si el hombre es vuestro hermano, es aún más que vuestro prójimo, a quien el Padre os pide que améis como a vosotros mismos. Como vuestro hermano pertenece a vuestra propia familia, no solamente lo amaréis con un afecto familiar, sino que también lo serviréis como os servís a vosotros mismos. Y amaréis y serviréis así a vuestro hermano porque vosotros, que sois mis hermanos, habéis sido amados y servidos por mí de esa manera. Id pues por todo el mundo contando esta buena nueva a todas las criaturas de todas las razas, tribus y naciones. Mi espíritu os precederá, y yo estaré siempre con vosotros>>.

\par 
%\textsuperscript{(2054.1)}
\textsuperscript{193:1.3} Estos samaritanos se quedaron enormemente asombrados con esta aparición del Maestro, y se apresuraron a ir a las ciudades y pueblos vecinos, donde difundieron la noticia de que habían visto a Jesús y que éste les había hablado. Ésta fue la decimoséptima aparición morontial del Maestro.

\section*{2. La aparición en Fenicia}
\par 
%\textsuperscript{(2054.2)}
\textsuperscript{193:2.1} La decimoctava aparición morontial del Maestro tuvo lugar en Tiro, el martes 16 de mayo, poco antes de las nueve de la noche. Apareció, una vez más, al término de una reunión de creyentes, cuando estaban a punto de dispersarse, y dijo:

\par 
%\textsuperscript{(2054.3)}
\textsuperscript{193:2.2} <<Que la paz sea con vosotros. Os alegráis de saber que el Hijo del Hombre ha resucitado de entre los muertos porque sabéis así que vosotros y vuestros hermanos sobreviviréis también a la muerte física. Pero esta supervivencia depende de que hayáis nacido previamente del espíritu que busca la verdad y encuentra a Dios. El pan y el agua de la vida sólo se conceden a los que tienen hambre de la verdad y sed de rectitud ---de Dios. El hecho de que los muertos resuciten no es el evangelio del reino. Estas grandes verdades y estos hechos universales están todos relacionados con este evangelio, en el sentido de que son una parte del resultado de creer en la buena nueva, y están contenidos en la experiencia posterior de aquellos que, por la fe, se convierten de hecho y en verdad en los hijos perpetuos del Dios eterno. Mi Padre me envió a este mundo para proclamar a todos los hombres esta salvación de la filiación. Y yo os envío también en todas direcciones para que prediquéis esta salvación de la filiación. La salvación es un don gratuito de Dios, pero aquellos que nacen del espíritu empiezan a manifestar inmediatamente los frutos del espíritu en el servicio amoroso a sus semejantes. Y los frutos del espíritu divino, producidos en la vida de los mortales nacidos del espíritu y que conocen a Dios, son: servicio amoroso, consagración desinteresada, lealtad valiente, equidad sincera, honradez iluminada, esperanza imperecedera, confianza fiel, ministerio misericordioso, bondad inagotable, tolerancia indulgente y paz duradera. Si unos creyentes declarados no producen estos frutos del espíritu divino en sus vidas, están muertos; el Espíritu de la Verdad no está en ellos; son unas ramas inútiles de la vid viviente, y pronto serán cortadas. Mi Padre pide a los hijos de la fe que produzcan muchos frutos del espíritu. Por consiguiente, si no sois fecundos, él cavará alrededor de vuestras raíces y cortará vuestras ramas estériles. A medida que progreséis hacia el cielo en el reino de Dios, deberéis producir cada vez más los frutos del espíritu. Podéis entrar en el reino como un niño, pero el Padre exige que crezcáis, por la gracia, hasta la plena estatura de un adulto espiritual. Cuando salgáis por ahí a contarle a todas las naciones la buena nueva de este evangelio, iré delante de vosotros, y mi Espíritu de la Verdad residirá en vuestro corazón. Mi paz os dejo>>.

\par 
%\textsuperscript{(2054.4)}
\textsuperscript{193:2.3} Entonces, el Maestro desapareció de su vista. Al día siguiente, los creyentes salieron de Tiro para llevar esta historia hasta Sidón e incluso hasta Antioquía y Damasco. Jesús había estado con estos creyentes cuando vivía en la carne, y lo reconocieron rápidamente en cuanto empezó a enseñarlos. Aunque sus amigos no podían reconocer fácilmente su forma morontial cuando ésta se hacía visible, no tardaban en reconocer su personalidad en cuanto les hablaba.

\section*{3. La última aparición en Jerusalén}
\par 
%\textsuperscript{(2055.1)}
\textsuperscript{193:3.1} El jueves 18 de mayo por la mañana temprano, Jesús hizo su última aparición en la Tierra como personalidad morontial. Cuando los once apóstoles estaban a punto de sentarse para desayunar en la habitación superior de la casa de María Marcos, Jesús se les apareció y les dijo:

\par 
%\textsuperscript{(2055.2)}
\textsuperscript{193:3.2} <<Que la paz sea con vosotros. Os he pedido que os quedéis aquí en Jerusalén hasta que yo ascienda hacia el Padre, e incluso hasta que os envíe el Espíritu de la Verdad, que pronto será derramado sobre todo el género humano y que os dotará de un poder de las alturas>>. Simón Celotes interrumpió a Jesús para preguntarle: <<Entonces, Maestro, ¿restablecerás el reino y veremos la gloria de Dios manifestada en la Tierra?>> Cuando Jesús escuchó la pregunta de Simón, contestó: <<Simón, continúas aferrado a tus viejas ideas sobre el Mesías judío y el reino material. Pero recibirás un poder espiritual cuando el espíritu haya descendido sobre vosotros, y pronto iréis por todo el mundo predicando este evangelio del reino. Al igual que el Padre me ha enviado al mundo, yo os envío a vosotros. Y deseo que os améis y confiéis los unos en los otros. Judas ya no está con vosotros porque su amor se enfrió y porque se negó a confiar en vosotros, sus leales hermanos. ¿No habéis leído en las Escrituras el pasaje que dice: `No es bueno que el hombre esté solo. Nadie vive para sí mismo'? ¿Y también donde dice: `El que quiera tener amigos debe mostrarse amistoso'? ¿Y no os envié a enseñar de dos en dos para que no os sintierais solos y no cayerais en los perjuicios y las desgracias del aislamiento? También sabéis muy bien que, cuando vivía en la carne, nunca me permití estar solo durante mucho tiempo. Desde el principio mismo de nuestra asociación, siempre tuve a dos o tres de vosotros constantemente a mi lado o muy cerca de mí, incluso cuando comulgaba con el Padre. Confiad, pues, y tened confianza los unos en los otros. Esto es tanto más necesario cuanto que en el día de hoy voy a dejaros solos en el mundo. Ha llegado la hora; estoy a punto de ir hacia el Padre>>.

\par 
%\textsuperscript{(2055.3)}
\textsuperscript{193:3.3} Cuando terminó de hablar, les hizo señas para que lo acompañaran y los condujo hasta el Monte de los Olivos, donde se despidió de ellos antes de partir de Urantia. Este recorrido hasta el Olivete fue solemne. Ninguno dijo ni una palabra desde el momento en que salieron de la habitación de arriba hasta que Jesús se detuvo con ellos en el Monte de los Olivos.

\section*{4. Las causas de la caída de Judas}
\par 
%\textsuperscript{(2055.4)}
\textsuperscript{193:4.1} En la primera parte de su mensaje de despedida a sus apóstoles, el Maestro aludió a la pérdida de Judas y resaltó el trágico destino de su compañero de trabajo traidor como una advertencia solemne contra los peligros del aislamiento social y fraternal. Quizás sea útil para los creyentes de este siglo y de los siglos futuros, analizar brevemente las causas de la caída de Judas a la luz de las observaciones del Maestro y en vista de las aclaraciones acumuladas de los siglos posteriores.

\par 
%\textsuperscript{(2055.5)}
\textsuperscript{193:4.2} Cuando recordamos esta tragedia, pensamos que Judas se desvió, principalmente, porque era una personalidad solitaria muy notoria, una personalidad cerrada y alejada de los contactos sociales corrientes. Se negó insistentemente a confiar en sus compañeros apóstoles, o a fraternizar libremente con ellos. Pero el hecho de ser una personalidad de tipo solitario, en sí mismo y por sí mismo, no le hubiera causado tanto daño a Judas si no hubiera sido porque tampoco logró acrecentar su amor ni crecer en gracia espiritual. Y además, para empeorar más las cosas, guardó rencores persistentes y alimentó enemigos psicológicos tales como la venganza y el ansia generalizada de <<desquitarse>> de alguien por todas sus decepciones.

\par 
%\textsuperscript{(2056.1)}
\textsuperscript{193:4.3} Esta desdichada combinación de peculiaridades individuales y de tendencias mentales se conjugó para destruir a un hombre bien intencionado que no logró subyugar estos males por medio del amor, la fe y la confianza. El hecho de que Judas no tenía necesidad de ir por mal camino está bien demostrado en los casos de Tomás y de Natanael, los cuales estaban aquejados de este mismo tipo de desconfianza y tenían superdesarrolladas sus tendencias individualistas. Incluso Andrés y Mateo tenían muchas inclinaciones en este sentido; pero todos estos hombres experimentaron por Jesús y sus compañeros apóstoles un amor que iba creciendo con el tiempo, y no disminuyendo. Crecieron en la gracia y en el conocimiento de la verdad. Confiaron cada vez más en sus hermanos y desarrollaron lentamente la capacidad de fiarse de sus compañeros. Judas se negó insistentemente a fiarse de sus hermanos. Cuando la acumulación de sus conflictos emocionales le obligaba a buscar alivio en la expresión personal, buscaba invariablemente el consejo y recibía el consuelo poco sensato de sus parientes no espirituales o de aquellos que conocía por casualidad, que eran indiferentes o realmente hostiles al bienestar y al progreso de las realidades espirituales del reino celestial, del que Judas era uno de los doce embajadores consagrados en la Tierra.

\par 
%\textsuperscript{(2056.2)}
\textsuperscript{193:4.4} Judas encontró la derrota en los combates de su lucha terrenal a causa de los factores siguientes relacionados con sus tendencias personales y sus debilidades de carácter:

\par 
%\textsuperscript{(2056.3)}
\textsuperscript{193:4.5} 1. Era un ser humano de tipo solitario. Era sumamente individualista y eligió convertirse en una clase de persona firmemente <<encerrada en sí misma>> e insociable.

\par 
%\textsuperscript{(2056.4)}
\textsuperscript{193:4.6} 2. Cuando era niño, le habían hecho la vida demasiado fácil. Se indignaba amargamente cuando le contrariaban. Siempre esperaba ganar; era muy mal perdedor.

\par 
%\textsuperscript{(2056.5)}
\textsuperscript{193:4.7} 3. Nunca adquirió una técnica filosófica para enfrentarse con las decepciones. En lugar de aceptar las desilusiones como un aspecto normal y común de la existencia humana, recurría infaliblemente a la práctica de acusar a alguien en particular, o a sus compañeros como grupo, de todas sus dificultades y decepciones personales.

\par 
%\textsuperscript{(2056.6)}
\textsuperscript{193:4.8} 4. Tendía a guardar rencor; alimentaba constantemente la idea de venganza.

\par 
%\textsuperscript{(2056.7)}
\textsuperscript{193:4.9} 5. No le gustaba enfrentarse francamente a los hechos; era poco honrado en su actitud ante las situaciones de la vida.

\par 
%\textsuperscript{(2056.8)}
\textsuperscript{193:4.10} 6. Detestaba discutir sus problemas personales con sus asociados inmediatos; se negaba a hablar de sus dificultades con sus verdaderos amigos y con aquellos que lo amaban realmente. En todos sus años de asociación con el Maestro, ni una sola vez se presentó ante él con un problema puramente personal.

\par 
%\textsuperscript{(2056.9)}
\textsuperscript{193:4.11} 7. No aprendió nunca que, después de todo, las verdaderas recompensas de una noble vida consisten en premios espirituales, que no siempre se distribuyen durante esta corta y única vida en la carne.

\par 
%\textsuperscript{(2056.10)}
\textsuperscript{193:4.12} A consecuencia del aislamiento persistente de su personalidad, sus penas se multiplicaron, sus aflicciones crecieron, sus ansiedades aumentaron y su desesperación alcanzó una profundidad casi insoportable.

\par 
%\textsuperscript{(2057.1)}
\textsuperscript{193:4.13} Aunque este apóstol egocéntrico y ultraindividualista tenía muchos problemas psíquicos, emocionales y espirituales, sus dificultades principales eran las siguientes: Como personalidad, estaba aislado. Mentalmente, era desconfiado y vengativo. Por temperamento, era hosco y rencoroso. Emocionalmente, estaba desprovisto de amor y era incapaz de perdonar. Socialmente, no confiaba en nadie y estaba casi enteramente encerrado en sí mismo. En espíritu, se volvió arrogante y egoístamente ambicioso. En la vida, ignoró a los que le amaban, y en la muerte, no tuvo ningún amigo.

\par 
%\textsuperscript{(2057.2)}
\textsuperscript{193:4.14} Éstos son, pues, los factores mentales y las influencias nocivas que, tomados en su conjunto, explican por qué un creyente en Jesús bien intencionado y por otra parte anteriormente sincero, incluso después de varios años de asociación íntima con la personalidad transformadora de Jesús, abandonó a sus compañeros, repudió una causa sagrada, renunció a su santa vocación y traicionó a su divino Maestro.

\section*{5. La ascensión del Maestro}
\par 
%\textsuperscript{(2057.3)}
\textsuperscript{193:5.1} Eran casi las siete y media de la mañana de este jueves 18 de mayo cuando Jesús llegó a la ladera occidental del Monte Olivete con sus once apóstoles silenciosos y un poco desconcertados. Desde este lugar, situado a unos dos tercios de la subida hasta la cima, podían contemplar Jerusalén y, debajo de ellos, Getsemaní. Jesús se preparó ahora para decir su último adiós a los apóstoles antes de despedirse de Urantia. Mientras estaba allí de pie delante de ellos, y sin que él lo pidiera, se arrodillaron en círculo a su alrededor, y el Maestro dijo:

\par 
%\textsuperscript{(2057.4)}
\textsuperscript{193:5.2} <<Os he pedido que permanezcáis en Jerusalén hasta que seáis dotados de un poder de las alturas. Ahora estoy a punto de despedirme de vosotros; estoy a punto de ascender hacia mi Padre, y pronto, muy pronto, enviaremos al Espíritu de la Verdad a este mundo donde he residido; cuando haya venido, empezaréis la nueva proclamación del evangelio del reino, primero en Jerusalén, y luego hasta los lugares más alejados del mundo. Amad a los hombres con el amor con que yo os he amado, y servid a vuestros semejantes mortales como yo os he servido. Mediante los frutos espirituales de vuestra vida, impulsad a las almas a creer en la verdad de que el hombre es un hijo de Dios, y de que todos los hombres son hermanos. Recordad todo lo que os he enseñado y la vida que he vivido entre vosotros. Mi amor os cubre con su sombra, mi espíritu residirá con vosotros y mi paz permanecerá en vosotros. Adiós>>.

\par 
%\textsuperscript{(2057.5)}
\textsuperscript{193:5.3} Después de hablar así, el Maestro morontial desapareció de su vista. Esta supuesta ascensión de Jesús no se diferenció en nada de sus otras desapariciones de la visión humana durante los cuarenta días de su carrera morontial en Urantia.

\par 
%\textsuperscript{(2057.6)}
\textsuperscript{193:5.4} El Maestro pasó por Jerusem para dirigirse a Edentia, donde los Altísimos, bajo la observación del Hijo Paradisiaco, liberaron a Jesús de Nazaret del estado morontial, y a través de los canales espirituales de ascensión, lo restituyeron al estado de filiación paradisiaca y de soberanía suprema en Salvington.

\par 
%\textsuperscript{(2057.7)}
\textsuperscript{193:5.5} Eran aproximadamente las siete y cuarenta y cinco de esta mañana cuando el Jesús morontial desapareció del campo de observación de sus once apóstoles para empezar la ascensión hacia la diestra de su Padre, y recibir allí la confirmación oficial de su completa soberanía sobre el universo de Nebadon.

\section*{6. Pedro convoca una reunión}
\par 
%\textsuperscript{(2057.8)}
\textsuperscript{193:6.1} Siguiendo las instrucciones de Pedro, Juan Marcos y otras personas salieron para convocar a los discípulos principales a una reunión en la casa de María Marcos. A las diez y media, ciento veinte de los discípulos más destacados de Jesús que vivían en Jerusalén se habían congregado para escuchar el relato del mensaje de adiós del Maestro y para enterarse de su ascensión. María, la madre de Jesús, se encontraba en este grupo. Había regresado a Jerusalén con Juan Zebedeo cuando los apóstoles volvieron de su reciente estancia en Galilea. Poco después de Pentecostés, María regresó a la casa de Salomé en Betsaida. Santiago, el hermano de Jesús, también estaba presente en esta reunión, la primera conferencia de discípulos que se convocaba después de finalizar la carrera planetaria del Maestro.

\par 
%\textsuperscript{(2058.1)}
\textsuperscript{193:6.2} Simón Pedro se encargó de hablar en nombre de sus compañeros apóstoles, e hizo un relato emocionante de la última reunión de los once con su Maestro; describió de la manera más conmovedora el adiós final del Maestro y su desaparición para emprender la ascensión. Nunca había tenido lugar en este mundo una reunión como ésta. Esta parte de la reunión duró poco menos de una hora. Pedro explicó entonces que habían decidido elegir a un sucesor de Judas Iscariote, y que se haría un descanso para permitir que los apóstoles decidieran entre los dos hombres que habían sido propuestos para esta función: Matías y Justo.

\par 
%\textsuperscript{(2058.2)}
\textsuperscript{193:6.3} Los once apóstoles descendieron entonces al piso de abajo, donde acordaron echar a suertes a fin de determinar cuál de estos hombres se convertiría en apóstol para servir en el lugar de Judas. La suerte cayó sobre Matías, que fue proclamado nuevo apóstol. Fue debidamente instalado en su cargo, y luego nombrado tesorero. Pero Matías participó poco en las actividades posteriores de los apóstoles.

\par 
%\textsuperscript{(2058.3)}
\textsuperscript{193:6.4} Poco después de Pentecostés, los gemelos regresaron a sus casas en Galilea. Simón Celotes se retiró durante algún tiempo antes de salir a predicar el evangelio. Tomás estuvo preocupado durante un período de tiempo más corto, y luego reanudó su enseñanza. Natanael discrepó cada vez más con Pedro respecto a la cuestión de predicar acerca de Jesús, en lugar de proclamar el evangelio original del reino. A mediados del mes siguiente, este desacuerdo se volvió tan agudo que Natanael se retiró y se fue a Filadelfia para visitar a Abner y Lázaro. Después de permanecer allí durante más de un año, se dirigió hacia los países situados más allá de Mesopotamia, predicando el evangelio tal como él lo entendía.

\par 
%\textsuperscript{(2058.4)}
\textsuperscript{193:6.5} De esta manera sólo quedaron seis apóstoles, de los doce originales, para actuar en el escenario de la proclamación inicial del evangelio en Jerusalén: Pedro, Andrés, Santiago, Juan, Felipe y Mateo.

\par 
%\textsuperscript{(2058.5)}
\textsuperscript{193:6.6} Poco antes del mediodía, los apóstoles regresaron junto a sus hermanos en la habitación de arriba, y anunciaron que Matías había sido elegido como nuevo apóstol. Luego, Pedro invitó a todos los creyentes a ponerse en oración, a orar a fin de estar preparados para recibir el don del espíritu que el Maestro había prometido enviar.


\chapter{Documento 194. La donación del Espíritu de la Verdad}
\par 
%\textsuperscript{(2059.1)}
\textsuperscript{194:0.1} ALREDEDOR DE la una, mientras los ciento veinte creyentes estaban orando, todos se dieron cuenta de una extraña presencia en la sala. Al mismo tiempo, todos estos discípulos se volvieron conscientes de un nuevo y profundo sentimiento de alegría, de seguridad y de confianza espirituales. Esta nueva conciencia de fuerza espiritual fue seguida de inmediato por un poderoso impulso a salir y proclamar públicamente el evangelio del reino y la buena nueva de que Jesús había resucitado de entre los muertos.

\par 
%\textsuperscript{(2059.2)}
\textsuperscript{194:0.2} Pedro se puso de pie y declaró que esto debía ser la llegada del Espíritu de la Verdad que el Maestro les había prometido, y propuso que fueran al templo para empezar a proclamar la buena nueva que les había sido confiada. Y todos hicieron lo que Pedro había sugerido.

\par 
%\textsuperscript{(2059.3)}
\textsuperscript{194:0.3} A estos hombres se les había educado y enseñado que el evangelio que debían predicar era la paternidad de Dios y la filiación de los hombres, pero en este preciso momento de éxtasis espiritual y de triunfo personal, la mejor nueva, la noticia más importante en la que estos hombres podían pensar era el \textit{hecho} de que el Maestro había resucitado. Dotados de un poder de las alturas, salieron pues a predicar la buena nueva al pueblo ---e incluso la salvación a través de Jesús--- pero cayeron involuntariamente en el error de sustituir el mensaje mismo del evangelio por algunos hechos asociados con el evangelio. Pedro dio comienzo sin saberlo a este error, y otros le siguieron después hasta llegar a Pablo, el cual creó una nueva religión basada en esta nueva versión de la buena nueva.

\par 
%\textsuperscript{(2059.4)}
\textsuperscript{194:0.4} El evangelio del reino es: el hecho de la paternidad de Dios, unido a la verdad consiguiente de la filiación y la fraternidad de los hombres. El cristianismo, tal como se desarrolló desde aquel día, es: el hecho de Dios como Padre del Señor Jesucristo, en asociación con la experiencia de la comunión del creyente con el Cristo resucitado y glorificado.

\par 
%\textsuperscript{(2059.5)}
\textsuperscript{194:0.5} No es de extrañar que estos hombres infundidos por el espíritu aprovecharan esta oportunidad para expresar sus sentimientos de triunfo sobre las fuerzas que habían intentado destruir a su Maestro y poner fin a la influencia de sus enseñanzas. En un momento como éste, era más fácil recordar su asociación personal con Jesús y sentirse emocionados con la seguridad de que el Maestro vivía todavía, que su amistad con él no había terminado y que el espíritu había descendido en verdad sobre ellos tal como él les había prometido.

\par 
%\textsuperscript{(2059.6)}
\textsuperscript{194:0.6} Estos creyentes se sentían de pronto transportados a otro mundo, a una nueva existencia de alegría, de poder y de gloria. El Maestro les había dicho que el reino vendría con poder, y algunos de ellos creían que empezaban a discernir lo que él había querido decir.

\par 
%\textsuperscript{(2059.7)}
\textsuperscript{194:0.7} Cuando todo esto se toma en consideración, no es difícil comprender cómo estos hombres llegaron a predicar un \textit{nuevo evangelio acerca de Jesús}, en lugar de su mensaje inicial de la paternidad de Dios y de la fraternidad de los hombres.

\section*{1. El sermón de Pentecostés}
\par 
%\textsuperscript{(2060.1)}
\textsuperscript{194:1.1} Los apóstoles habían estado escondidos durante cuarenta días. Este día resultó ser la fiesta judía de Pentecostés, y miles de visitantes de todas las partes del mundo se encontraban en Jerusalén. Muchos habían llegado para esta fiesta, pero la mayoría había permanecido en la ciudad desde la Pascua. Ahora, estos apóstoles asustados surgían de sus semanas de reclusión para aparecer audazmente en el templo, donde empezaron a predicar el nuevo mensaje de un Mesías resucitado. Y todos los discípulos eran igualmente conscientes de haber recibido una nueva dotación espiritual de perspicacia y de poder.

\par 
%\textsuperscript{(2060.2)}
\textsuperscript{194:1.2} Eran alrededor de las dos cuando Pedro se levantó en el mismo lugar donde su Maestro había enseñado por última vez en este templo, y pronunció el llamamiento apasionado que consiguió ganar a más de dos mil almas. El Maestro se había ido, pero ellos descubrieron repentinamente que esta historia acerca de él ejercía un gran poder sobre el pueblo. No es de extrañar que se sintieran inducidos a continuar proclamando lo que justificaba su anterior devoción a Jesús y que, al mismo tiempo, tanto forzaba a los hombres a creer en él. Seis apóstoles participaron en esta reunión: Pedro, Andrés, Santiago, Juan, Felipe y Mateo. Hablaron durante más de hora y media, y expresaron sus mensajes en griego, hebreo y arameo, diciendo incluso algunas palabras en otras lenguas que conocían un poco.

\par 
%\textsuperscript{(2060.3)}
\textsuperscript{194:1.3} Los dirigentes de los judíos se quedaron asombrados de la audacia de los apóstoles, pero tuvieron miedo de molestarlos a causa de la gran cantidad de gente que creía en su relato.

\par 
%\textsuperscript{(2060.4)}
\textsuperscript{194:1.4} Hacia las cuatro y media, más de dos mil nuevos creyentes siguieron a los apóstoles hasta el estanque de Siloé, donde Pedro, Andrés, Santiago y Juan los bautizaron en nombre del Maestro. Ya era de noche cuando terminaron de bautizar a la multitud.

\par 
%\textsuperscript{(2060.5)}
\textsuperscript{194:1.5} Pentecostés era la gran fiesta del bautismo, el momento en que se aceptaban como miembros a los prosélitos del exterior, a aquellos gentiles que deseaban servir a Yahvé. Por consiguiente, para gran cantidad de judíos y de gentiles creyentes, era mucho más fácil someterse al bautismo en este día. Al hacer esto, no se separaban de ninguna manera de la fe judía. Incluso durante algún tiempo después de esto, los creyentes en Jesús fueron una secta dentro del judaísmo. Todos ellos, incluídos los apóstoles, seguían siendo leales a las exigencias esenciales del sistema ceremonial judío.

\section*{2. El significado de Pentecostés}
\par 
%\textsuperscript{(2060.6)}
\textsuperscript{194:2.1} Jesús vivió en la Tierra y enseñó un evangelio que liberaba al hombre de la superstición de que era un hijo del demonio, y lo elevaba a la dignidad de un hijo de Dios por la fe. El mensaje de Jesús, tal como lo predicó y lo vivió en su día, fue una solución eficaz para las dificultades espirituales del hombre en la época en que fue expuesto. Y ahora que el Maestro se ha ido personalmente de este mundo, envía en su lugar a su Espíritu de la Verdad, que está destinado a vivir en el hombre y a exponer de nuevo el mensaje de Jesús para cada nueva generación. Así, cada nuevo grupo de mortales que aparezca sobre la faz de la Tierra tendrá una versión nueva y actualizada del evangelio, precisamente esa iluminación personal y esa guía colectiva que resultará ser una solución eficaz para las dificultades espirituales, siempre nuevas y variadas, del hombre.

\par 
%\textsuperscript{(2060.7)}
\textsuperscript{194:2.2} La primera misión de este espíritu es, por supuesto, fomentar y personalizar la verdad, porque la comprensión de la verdad es lo que constituye la forma más elevada de libertad humana. A continuación, la finalidad de este espíritu es destruir el sentimiento de orfandad del creyente. Como Jesús había estado entre los hombres, todos los creyentes experimentarían un sentimiento de soledad si el Espíritu de la Verdad no hubiera venido a residir en el corazón de los hombres.

\par 
%\textsuperscript{(2061.1)}
\textsuperscript{194:2.3} Esta donación del espíritu del Hijo preparó eficazmente la mente de todos los hombres normales para la donación universal posterior del espíritu del Padre (el Ajustador) a toda la humanidad. En cierto sentido, este Espíritu de la Verdad es el espíritu tanto del Padre Universal como del Hijo Creador.

\par 
%\textsuperscript{(2061.2)}
\textsuperscript{194:2.4} No cometáis el error de esperar que llegaréis a tener una fuerte conciencia intelectual del Espíritu de la Verdad derramado. El espíritu nunca crea una conciencia de sí mismo, sino sólo una conciencia de Miguel, el Hijo. Desde el principio, Jesús enseñó que el espíritu no hablaría de sí mismo. Por consiguiente, la prueba de vuestra comunión con el Espíritu de la Verdad no se puede encontrar en vuestra conciencia de este espíritu, sino más bien en vuestra experiencia de una elevada comunión con Miguel.

\par 
%\textsuperscript{(2061.3)}
\textsuperscript{194:2.5} El espíritu vino también para ayudar a los hombres a recordar y a comprender las palabras del Maestro, así como para iluminar y reinterpretar su vida en la Tierra.

\par 
%\textsuperscript{(2061.4)}
\textsuperscript{194:2.6} A continuación, el Espíritu de la Verdad vino para ayudar al creyente a atestiguar las realidades de las enseñanzas de Jesús y de su vida tal como la vivió en la carne, y tal como la vive ahora de nuevo una y otra vez en el creyente individual de cada generación sucesiva de hijos de Dios llenos de espíritu.

\par 
%\textsuperscript{(2061.5)}
\textsuperscript{194:2.7} Así pues, parece ser que el Espíritu de la Verdad viene para conducir realmente a todos los creyentes a toda la verdad, al conocimiento en expansión de la experiencia de la conciencia espiritual, viviente y creciente, de la realidad de la filiación eterna y ascendente con Dios.

\par 
%\textsuperscript{(2061.6)}
\textsuperscript{194:2.8} Jesús vivió una vida que es una revelación del hombre sometido a la voluntad del Padre, y no un ejemplo que cada hombre deba intentar seguir al pie de la letra. Su vida en la carne, junto con su muerte en la cruz y su resurrección posterior, pronto se convirtieron en un nuevo evangelio del rescate que se había pagado así a fin de recuperar al hombre de las garras del maligno ---de la condenación de un Dios ofendido. Sin embargo, aunque el evangelio fue enormemente distorsionado, sigue siendo un hecho que este nuevo mensaje acerca de Jesús llevaba consigo muchas verdades y enseñanzas fundamentales de su evangelio inicial del reino. Tarde o temprano, estas verdades ocultas de la paternidad de Dios y de la fraternidad de los hombres emergerán para transformar eficazmente la civilización de toda la humanidad.

\par 
%\textsuperscript{(2061.7)}
\textsuperscript{194:2.9} Pero estos errores del intelecto no interfirieron de ninguna manera con los grandes progresos de los creyentes en crecimiento espiritual. En menos de un mes, después de la donación del Espíritu de la Verdad, los apóstoles hicieron individualmente más progresos espirituales que durante sus casi cuatro años de asociación personal y afectuosa con el Maestro. Esta sustitución de la \textit{verdad} del evangelio salvador de la filiación con Dios por el \textit{hecho} de la resurrección de Jesús tampoco impidió de ninguna manera la rápida difusión de sus enseñanzas; al contrario, el hecho de que el mensaje de Jesús fuera eclipsado por las nuevas enseñanzas sobre su persona y su resurrección pareció facilitar enormemente la predicación de la buena nueva.

\par 
%\textsuperscript{(2061.8)}
\textsuperscript{194:2.10} La expresión <<bautismo de espíritu>>, que empezó a emplearse de manera tan generalizada hacia esta época, significaba simplemente la recepción consciente de este don del Espíritu de la Verdad, y el reconocimiento personal de este nuevo poder espiritual como un acrecentamiento de todas las influencias espirituales experimentadas previamente por las almas que conocían a Dios.

\par 
%\textsuperscript{(2061.9)}
\textsuperscript{194:2.11} Desde la donación del Espíritu de la Verdad, el hombre está sujeto a la enseñanza y a la guía de una triple dotación espiritual: el espíritu del Padre (el Ajustador del Pensamiento), el espíritu del Hijo (el Espíritu de la Verdad), y el espíritu del Espíritu (el Espíritu Santo).

\par 
%\textsuperscript{(2062.1)}
\textsuperscript{194:2.12} En cierto modo, la humanidad está sujeta a la doble influencia del séptuple llamamiento de las influencias espirituales del universo. Las primeras razas evolutivas de mortales están sometidas al contacto progresivo con los siete espíritus ayudantes de la mente procedentes del Espíritu Madre del universo local. A medida que el hombre progresa hacia arriba en la escala de la inteligencia y de la percepción espiritual, siete influencias espirituales superiores vienen finalmente a cernirse sobre él y a residir dentro de él. Y estos siete espíritus de los mundos que progresan son:

\par 
%\textsuperscript{(2062.2)}
\textsuperscript{194:2.13} 1. El espíritu otorgado por el Padre Universal ---los Ajustadores del Pensamiento.

\par 
%\textsuperscript{(2062.3)}
\textsuperscript{194:2.14} 2. La presencia espiritual del Hijo Eterno ---la gravedad espiritual del universo de universos y el canal seguro para toda comunión espiritual.

\par 
%\textsuperscript{(2062.4)}
\textsuperscript{194:2.15} 3. La presencia espiritual del Espíritu Infinito ---la mente-espíritu universal de toda la creación, la fuente espiritual del parentesco intelectual de todas las inteligencias progresivas.

\par 
%\textsuperscript{(2062.5)}
\textsuperscript{194:2.16} 4. El espíritu del Padre Universal y del Hijo Creador ---el Espíritu de la Verdad, considerado generalmente como el espíritu del Hijo del Universo.

\par 
%\textsuperscript{(2062.6)}
\textsuperscript{194:2.17} 5. El espíritu del Espíritu Infinito y del Espíritu Madre del Universo ---el Espíritu Santo, considerado generalmente como el espíritu del Espíritu del Universo.

\par 
%\textsuperscript{(2062.7)}
\textsuperscript{194:2.18} 6. El espíritu-mente del Espíritu Madre del Universo ---los siete espíritus ayudantes de la mente del universo local.

\par 
%\textsuperscript{(2062.8)}
\textsuperscript{194:2.19} 7. El espíritu del Padre, de los Hijos y de los Espíritus ---el espíritu con un nuevo nombre que llega a los mortales ascendentes de los reinos después de la fusión del alma mortal nacida del espíritu con el Ajustador del Pensamiento del Paraíso, y después de alcanzar posteriormente la divinidad y la glorificación de pertenecer al Cuerpo Paradisiaco de la Finalidad.

\par 
%\textsuperscript{(2062.9)}
\textsuperscript{194:2.20} Y así, la donación del Espíritu de la Verdad aportó al mundo y a sus pueblos la última dotación espiritual destinada a ayudarles en la búsqueda ascendente de Dios.

\section*{3. Lo que sucedió en Pentecostés}
\par 
%\textsuperscript{(2062.10)}
\textsuperscript{194:3.1} Muchas enseñanzas raras y extrañas fueron asociadas a los relatos iniciales del día de Pentecostés. En épocas posteriores, los sucesos de este día en que el Espíritu de la Verdad, el nuevo instructor, vino a residir en la humanidad, se han confundido con los necios estallidos de una emotividad desenfrenada. La misión principal de este espíritu, derramado por el Padre y el Hijo, consiste en enseñar a los hombres las verdades sobre el amor del Padre y la misericordia del Hijo. Éstas son las verdades de la divinidad que los hombres pueden comprender mucho mejor que todos los demás rasgos del carácter divino. El Espíritu de la Verdad se interesa principalmente por revelar la naturaleza espiritual del Padre y el carácter moral del Hijo. El Hijo Creador, en la carne, reveló Dios a los hombres; el Espíritu de la Verdad, en el corazón, revela el Hijo Creador a los hombres. Cuando un hombre produce en su vida los <<frutos del espíritu>>, muestra simplemente los rasgos que el Maestro manifestó en su propia vida terrenal. Cuando Jesús estuvo en la Tierra, vivió su vida como una personalidad única ---Jesús de Nazaret. Desde Pentecostés, el Maestro, como espíritu interno del <<nuevo instructor>>, ha podido vivir su vida de nuevo en la experiencia de cada creyente que ha sido enseñado por la verdad.

\par 
%\textsuperscript{(2062.11)}
\textsuperscript{194:3.2} Muchas cosas que suceden en el transcurso de una vida humana son duras de comprender, difíciles de conciliar con la idea de que éste es un universo en el que prevalece la verdad y triunfa la rectitud. Muy a menudo se tiene la impresión de que prevalece la calumnia, la mentira, la deshonestidad y la falta de rectitud ---el pecado. Después de todo, ¿triunfa la fe sobre el mal, el pecado y la iniquidad? Sí que triunfa. La vida y la muerte de Jesús son la prueba eterna de que la verdad de la bondad y la fe de la criatura conducida por el espíritu serán siempre justificadas. Se mofaron de Jesús en la cruz, diciendo: <<Veamos si Dios viene a liberarlo>>. El día de la crucifixión pareció sombrío, pero la mañana de la resurrección fue gloriosamente brillante, y el día de Pentecostés fue aun más radiante y gozoso. Las religiones de desesperación pesimista tratan de liberarse de las cargas de la vida; anhelan la extinción en un sueño y un reposo sin fin. Son las religiones del miedo y del temor primitivos. La religión de Jesús es un nuevo evangelio de fe que se ha de proclamar a una humanidad que lucha. Esta nueva religión está fundada en la fe, la esperanza y el amor.

\par 
%\textsuperscript{(2063.1)}
\textsuperscript{194:3.3} La vida mortal le había asestado a Jesús sus golpes más duros, más crueles y más amargos; y este hombre se había enfrentado a estas situaciones desesperantes con fe, coraje y la férrea determinación de hacer la voluntad de su Padre. Jesús afrontó la vida en toda su terrible realidad, y la venció ---incluso en la muerte. No utilizó la religión para liberarse de la vida. La religión de Jesús no intenta eludir esta vida para disfrutar de la felicidad que espera en otra existencia. La religión de Jesús proporciona la alegría y la paz de una nueva existencia espiritual para realzar y ennoblecer la vida que los hombres viven ahora en la carne.

\par 
%\textsuperscript{(2063.2)}
\textsuperscript{194:3.4} Si la religión es un opio para el pueblo, no es la religión de Jesús. En la cruz, se negó a beber la droga adormecedora, y su espíritu, derramado sobre todo el género humano, es una poderosa influencia mundial que conduce al hombre hacia arriba y lo impulsa hacia adelante. El impulso espiritual hacia adelante es la fuerza motriz más poderosa que existe en este mundo; el creyente que aprende la verdad es la única alma progresiva y dinámica de la Tierra.

\par 
%\textsuperscript{(2063.3)}
\textsuperscript{194:3.5} El día de Pentecostés, la religión de Jesús rompió todas las restricciones nacionales y todas las cadenas raciales. Es eternamente cierto que <<allí donde se encuentra el espíritu del Señor, está la libertad>>. Aquel día, el Espíritu de la Verdad se convirtió en el don personal del Maestro para cada mortal. Este espíritu se otorgó con la finalidad de cualificar a los creyentes para que predicaran más eficazmente el evangelio del reino, pero confundieron la experiencia de recibir el espíritu derramado con una parte del nuevo evangelio que inconscientemente estaban formulando.

\par 
%\textsuperscript{(2063.4)}
\textsuperscript{194:3.6} No paséis por alto el hecho de que el Espíritu de la Verdad fue otorgado a todos los creyentes sinceros; este don del espíritu no vino solamente a los apóstoles. Los ciento veinte hombres y mujeres congregados en la habitación de arriba recibieron todos el nuevo instructor, así como todos los honrados de corazón del mundo entero. Este nuevo instructor fue otorgado a la humanidad, y cada alma lo recibió según su amor por la verdad y su capacidad para captar y comprender las realidades espirituales. Por fin, la verdadera religión se libera de la custodia de los sacerdotes y de todas las clases sagradas, y encuentra su manifestación real en el alma individual de los hombres.

\par 
%\textsuperscript{(2063.5)}
\textsuperscript{194:3.7} La religión de Jesús fomenta el tipo más elevado de civilización humana, en el sentido de que crea el tipo más elevado de personalidad espiritual y proclama la condición sagrada de esa persona.

\par 
%\textsuperscript{(2063.6)}
\textsuperscript{194:3.8} La llegada del Espíritu de la Verdad en Pentecostés hizo posible una religión que no es ni radical ni conservadora; no es ni antigua ni nueva; no debe estar dominada ni por los viejos ni por los jóvenes. El hecho de la vida terrenal de Jesús proporciona un punto fijo para el ancla del tiempo, mientras que la donación del Espíritu de la Verdad asegura la expansión perpetua y el crecimiento sin fin de la religión que Jesús vivió y del evangelio que proclamó. El espíritu conduce a \textit{toda} la verdad; enseña la expansión y el constante crecimiento de una religión de progreso sin fin y de descubrimiento divino. Este nuevo instructor estará revelando siempre al creyente que busca la verdad aquello que estaba tan divinamente contenido en la persona y en la naturaleza del Hijo del Hombre.

\par 
%\textsuperscript{(2064.1)}
\textsuperscript{194:3.9} Las manifestaciones que acompañaron a la donación del <<nuevo instructor>>, y la acogida que los hombres de las diversas razas y naciones, reunidos en Jerusalén, hicieron a la predicación de los apóstoles, indican la universalidad de la religión de Jesús. El evangelio del reino no debía ser identificado con ninguna raza, cultura o idioma particular. Este día de Pentecostés fue testigo del gran esfuerzo del espíritu por liberar a la religión de Jesús de las trabas judías que había heredado. Incluso después de esta demostración en la que el espíritu fue derramado sobre todo el género humano, los apóstoles trataron al principio de imponer a sus conversos las exigencias del judaísmo. El mismo Pablo tuvo dificultades con sus hermanos de Jerusalén, porque se negaba a someter a los gentiles a estas prácticas judías. Ninguna religión revelada puede difundirse por todo el mundo si comete el grave error de dejarse impregnar por alguna cultura nacional, o asociarse con unas prácticas raciales, sociales o económicas ya establecidas.

\par 
%\textsuperscript{(2064.2)}
\textsuperscript{194:3.10} La donación del Espíritu de la Verdad fue independiente de todas las formalidades, ceremonias, lugares sagrados y comportamiento especial de aquellos que recibieron la plenitud de su manifestación. Cuando el espíritu descendió sobre las personas congregadas en la habitación de arriba, simplemente estaban sentadas allí y acababan de ponerse a orar en silencio. El espíritu fue otorgado en el campo así como en la ciudad. Los apóstoles no necesitaron retirarse a un lugar aislado durante años de meditación solitaria a fin de recibir el espíritu. Pentecostés disocia para siempre la idea de experiencia espiritual, de la noción de un entorno especialmente favorable.

\par 
%\textsuperscript{(2064.3)}
\textsuperscript{194:3.11} Pentecostés, con su dotación espiritual, estuvo destinado a liberar para siempre la religión del Maestro de toda dependencia de la fuerza física; los instructores de esta nueva religión ahora están provistos de armas espirituales. Deben partir a la conquista del mundo con una indulgencia inagotable, una buena voluntad incomparable y un amor abundante. Están equipados para dominar el mal con el bien, para vencer el odio con el amor, para destruir el miedo con una fe valiente y viviente en la verdad. Jesús ya había enseñado a sus seguidores que su religión nunca era pasiva; sus discípulos debían ser siempre activos y positivos en su ministerio de misericordia y en sus manifestaciones de amor. Estos creyentes ya no contemplaban a Yahvé como <<el Señor de los Ejércitos>>. Ahora consideraban a la Deidad eterna como el <<Dios y el Padre del Señor Jesucristo>>. Al menos hicieron este progreso, aunque en cierta medida no lograron captar plenamente la verdad de que Dios es también el Padre espiritual de cada individuo.

\par 
%\textsuperscript{(2064.4)}
\textsuperscript{194:3.12} Pentecostés dotó al hombre mortal del poder de perdonar las ofensas personales, de conservar la dulzura en medio de las peores injusticias, de permanecer impasible ante unos peligros aterradores, y de desafiar los males del odio y de la ira mediante los actos intrépidos del amor y la indulgencia. A lo largo de su historia, Urantia ha sufrido las devastaciones de grandes guerras destructivas. Todos los que participaron en estas luchas terribles encontraron la derrota. Sólo hubo un vencedor; sólo hubo uno que salió de estas amargas luchas con un prestigio realzado ---y éste fue Jesús de Nazaret y su evangelio de vencer el mal con el bien. El secreto de una civilización mejor está encerrado en las enseñanzas del Maestro sobre la fraternidad de los hombres, la buena voluntad del amor y de la confianza mutua.

\par 
%\textsuperscript{(2065.1)}
\textsuperscript{194:3.13} Hasta Pentecostés, la religión no había revelado más que el hombre a la búsqueda de Dios; a partir de Pentecostés, el hombre continúa buscando a Dios, pero también brilla sobre el mundo el espectáculo de Dios a la búsqueda del hombre y enviando su espíritu para que resida en él cuando lo ha encontrado.

\par 
%\textsuperscript{(2065.2)}
\textsuperscript{194:3.14} Antes de las enseñanzas de Jesús, que culminaron en Pentecostés, las mujeres tenían poca o ninguna posición espiritual en los credos de las religiones más antiguas. Después de Pentecostés, la mujer se encontró ante Dios, en la fraternidad del reino, en igualdad de condiciones que el hombre. Entre las ciento veinte personas que recibieron esta visita especial del espíritu se encontraban muchas discípulas, y compartieron estas bendiciones en la misma medida que los creyentes masculinos. Los hombres ya no pueden atreverse a monopolizar el ministerio del servicio religioso. Los fariseos podían continuar dando gracias a Dios por <<no haber nacido mujer, ni leproso, ni gentil>>, pero entre los seguidores de Jesús, las mujeres han sido liberadas para siempre de toda discriminación religiosa basada en el sexo. Pentecostés borró toda discriminación religiosa fundada en la distinción racial, las diferencias culturales, las castas sociales o los prejuicios relacionados con el sexo. No es de extrañar que estos creyentes en la nueva religión exclamaran: <<Allí donde se encuentra el espíritu del Señor, está la libertad>>.

\par 
%\textsuperscript{(2065.3)}
\textsuperscript{194:3.15} Tanto la madre como un hermano de Jesús estaban presentes entre los ciento veinte creyentes, y como miembros de este grupo común de discípulos, recibieron también el espíritu derramado. No recibieron de este buen don una cantidad mayor que sus compañeros. No se concedió ningún don especial a los miembros de la familia terrenal de Jesús. Pentecostés marcó el final de los sacerdocios especiales y de toda creencia en las familias sagradas.

\par 
%\textsuperscript{(2065.4)}
\textsuperscript{194:3.16} Antes de Pentecostés, los apóstoles habían renunciado a muchas cosas por Jesús. Habían sacrificado sus hogares, sus familias, sus amigos, sus bienes terrenales y su posición social. En Pentecostés se entregaron a Dios, y el Padre y el Hijo respondieron entregándose a los hombres ---enviando a sus espíritus para que vivieran en los hombres. Esta experiencia de perder el yo y de encontrar el espíritu no fue una experiencia emocional; fue un acto de autoentrega inteligente y de consagración sin reservas.

\par 
%\textsuperscript{(2065.5)}
\textsuperscript{194:3.17} Pentecostés fue el llamamiento a la unidad espiritual entre los creyentes en el evangelio. Cuando el espíritu descendió sobre los discípulos en Jerusalén, lo mismo sucedió en Filadelfia, en Alejandría y en todos los demás lugares donde vivían los creyentes sinceros. Fue literalmente cierto que <<había un solo corazón y una sola alma entre la multitud de creyentes>>. La religión de Jesús es la influencia unificadora más poderosa que el mundo ha conocido jamás.

\par 
%\textsuperscript{(2065.6)}
\textsuperscript{194:3.18} Pentecostés estaba destinado a disminuir la presunción de las personas, los grupos, las naciones y las razas. La tensión de este espíritu de presunción es la que se acrecienta tanto que periódicamente se desata en guerras destructivas. La humanidad sólo puede unificarse mediante el acercamiento espiritual, y el Espíritu de la Verdad es una influencia mundial común para todos.

\par 
%\textsuperscript{(2065.7)}
\textsuperscript{194:3.19} La llegada del Espíritu de la Verdad purifica el corazón humano y conduce a la persona que lo recibe a formular un proyecto de vida dedicado a la voluntad de Dios y al bienestar de los hombres. El espíritu de egoísmo material ha sido absorbido en esta nueva donación espiritual de altruismo. Pentecostés, en aquel entonces como ahora, significa que el Jesús histórico se ha convertido en el Hijo divino de la experiencia viviente. Cuando la alegría de este espíritu derramado se experimenta conscientemente en la vida humana, es un tónico para la salud, un estímulo para la mente y una energía inagotable para el alma.

\par 
%\textsuperscript{(2065.8)}
\textsuperscript{194:3.20} La oración no hizo venir al espíritu el día de Pentecostés, pero contribuyó mucho a determinar la capacidad receptiva que caracterizó a los creyentes individuales. La oración no incita al corazón divino a donarse generosamente, pero muy a menudo cava unos canales más amplios y más profundos por los cuales los dones divinos pueden fluir hasta el corazón y el alma de aquellos que se acuerdan de mantener así, mediante la oración sincera y la verdadera adoración, una comunión ininterrumpida con su Hacedor.

\section*{4. Los principios de la iglesia cristiana}
\par 
%\textsuperscript{(2066.1)}
\textsuperscript{194:4.1} Cuando los enemigos de Jesús lo apresaron tan repentinamente y lo crucificaron con tanta rapidez entre dos ladrones, sus apóstoles y sus discípulos se sintieron completamente desmoralizados. La idea de que el Maestro había sido arrestado, atado, azotado y crucificado, era demasiado incluso para los apóstoles. Olvidaron sus enseñanzas y sus advertencias. Jesús podía haber sido en verdad <<un profeta poderoso en obras y en palabras delante de Dios y de todo el pueblo>>, pero difícilmente podía ser el Mesías que esperaban que restauraría el reino de Israel.

\par 
%\textsuperscript{(2066.2)}
\textsuperscript{194:4.2} Luego llega la resurrección, que los libera de la desesperación y les devuelve su fe en la divinidad del Maestro. Lo ven y hablan con él una y otra vez, y Jesús los lleva hasta el Olivete, donde se despide de ellos y les dice que regresa hacia el Padre. Les ha dicho que permanezcan en Jerusalén hasta que sean dotados de poder ---hasta que venga el Espíritu de la Verdad. Este nuevo instructor llega el día de Pentecostés, y los apóstoles salen inmediatamente a predicar su evangelio con una nueva energía. Son los seguidores audaces y valientes de un Señor vivo, y no de un jefe muerto y vencido. El Maestro vive en el corazón de estos evangelistas; Dios no es una doctrina en sus mentes; se ha vuelto una presencia viviente en sus almas.

\par 
%\textsuperscript{(2066.3)}
\textsuperscript{194:4.3} <<Día tras día, perseveraban de común acuerdo en el templo y partían el pan en la casa. Comían con alegría y unidad de corazón, alabando a Dios y teniendo el favor de todo el pueblo. Todos estaban llenos del espíritu, y proclamaban con audacia la palabra de Dios. Las multitudes de creyentes tenían un solo corazón y una sola alma; ninguno decía que los bienes que poseía eran suyos, y todas las cosas las tenían en común>>.

\par 
%\textsuperscript{(2066.4)}
\textsuperscript{194:4.4} ¿Que les ha sucedido a estos hombres a quienes Jesús había ordenado para que salieran a predicar el evangelio del reino ---la paternidad de Dios y la fraternidad de los hombres? Tienen un nuevo evangelio; arden con una nueva experiencia; están llenos de una nueva energía espiritual. Su mensaje ha sido sustituido repentinamente por la proclamación del Cristo resucitado: <<Jesús de Nazaret, ese hombre a quien Dios dio su aprobación mediante obras y prodigios poderosos, que fue entregado por el dictamen resuelto y la presciencia de Dios, vosotros lo habéis crucificado y ejecutado. Ha cumplido así las cosas que Dios había anunciado por boca de todos los profetas. A este Jesús es a quien Dios ha resucitado. Dios lo ha hecho Señor y Cristo a la vez. Como ha sido elevado a la diestra de Dios y ha recibido del Padre la promesa del espíritu, ha derramado esto que veis y oís. Arrepentíos, para que vuestros pecados puedan ser borrados, para que el Padre pueda enviar al Cristo que ha sido designado para vosotros, al mismo Jesús, a quien el cielo ha de recibir hasta los tiempos del restablecimiento de todas las cosas>>.

\par 
%\textsuperscript{(2066.5)}
\textsuperscript{194:4.5} El evangelio del reino, el mensaje de Jesús, había sido transformado repentinamente en el evangelio acerca del Señor Jesucristo. Ahora proclamaban los hechos de su vida, de su muerte y de su resurrección, y predicaban la esperanza de que regresaría rápidamente a este mundo para terminar la obra que había empezado. El mensaje de los primeros creyentes consistió pues en predicar los hechos de su primera venida y en enseñar la esperanza de su segunda venida, un acontecimiento que suponían que estaba muy próximo.

\par 
%\textsuperscript{(2067.1)}
\textsuperscript{194:4.6} Cristo estaba a punto de convertirse en el credo de la iglesia que se formaba rápidamente. Jesús vive; murió por los hombres; ha dado el espíritu; va a regresar de nuevo. Jesús llenaba todos sus pensamientos y determinaba todos sus nuevos conceptos sobre Dios y sobre todo lo demás. Estaban demasiado entusiasmados con la nueva doctrina de que <<Dios es el Padre del Señor Jesús>> como para preocuparse del antiguo mensaje de que <<Dios es el Padre amoroso de todos los hombres>>, e incluso de cada persona en particular. Es verdad que una maravillosa manifestación de amor fraternal y de buena voluntad inigualable nació en estas primeras comunidades de creyentes. Pero eran unas comunidades de creyentes en Jesús, y no una confraternidad de hermanos en el reino de la familia del Padre que está en los cielos. Su buena voluntad provenía del amor nacido del concepto de la donación de Jesús, y no del reconocimiento de la fraternidad de los mortales. Sin embargo, estaban llenos de alegría y vivían unas vidas tan nuevas y excepcionales, que todos los hombres se sentían atraídos hacia sus enseñanzas acerca de Jesús. Cometieron el gran error de utilizar la interpretación viviente e ilustrativa del evangelio del reino, en lugar del evangelio mismo, pero incluso esto representaba la religión más asombrosa que la humanidad hubiera conocido jamás.

\par 
%\textsuperscript{(2067.2)}
\textsuperscript{194:4.7} Evidentemente, una nueva comunidad estaba apareciendo en el mundo. <<La multitud que creía perseveraba en la enseñanza y la comunión de los apóstoles, en la partición del pan y en las oraciones>>. Se llamaban unos a otros hermanos y hermanas; se saludaban unos a otros con un beso puro; ayudaban a los pobres. Era una comunidad de vida así como de adoración. No eran comunitarios por decreto, sino por el deseo de compartir sus bienes con sus compañeros creyentes. Esperaban con confianza que Jesús regresaría durante su generación para terminar de establecer el reino del Padre. El hecho de compartir espontáneamente las posesiones terrenales no era una característica directa de las enseñanzas de Jesús; sucedió porque estos hombres y mujeres creían de manera muy sincera y confiada que el Maestro iba a regresar en cualquier momento para terminar su obra y consumar el reino. Pero los resultados finales de este experimento bien intencionado de amor fraternal irreflexivo fueron desastrosos y causaron muchos pesares. Miles de creyentes sinceros vendieron sus propiedades y distribuyeron todos sus bienes capitales y otros activos rentables. Con el paso del tiempo, los recursos menguantes de este <<compartir por igual>> de los cristianos \textit{se acabaron} ---pero el mundo no se acabó. Muy pronto, los creyentes de Antioquía empezaron a hacer colectas para impedir que sus compañeros creyentes de Jerusalén se murieran de hambre.

\par 
%\textsuperscript{(2067.3)}
\textsuperscript{194:4.8} En aquellos días, los creyentes celebraban la Cena del Señor de la manera que había sido establecida, es decir, que se reunían para participar en una comida social de buena hermandad y compartían el sacramento al final de la comida.

\par 
%\textsuperscript{(2067.4)}
\textsuperscript{194:4.9} Al principio bautizaron en el nombre de Jesús; pero casi veinte años después empezaron a bautizar <<en el nombre del Padre, del Hijo y del Espíritu Santo>>. El bautismo era todo lo que se exigía para ser admitido en la comunidad de los creyentes. Hasta ahora no tenían ninguna organización; era simplemente la fraternidad de Jesús.

\par 
%\textsuperscript{(2067.5)}
\textsuperscript{194:4.10} Esta secta de Jesús crecía rápidamente, y una vez más los saduceos les prestaron atención. Los fariseos se molestaron poco con esta situación, ya que ninguna de las enseñanzas interfería de manera alguna con el cumplimiento de la leyes judías. Pero los saduceos empezaron a encarcelar a los dirigentes de la secta de Jesús hasta que se decidieron a aceptar el consejo de Gamaliel, uno de los rabinos principales, el cual les había advertido: <<Absteneos de tocar a esos hombres y dejadlos en paz, porque si este consejo o esta obra procede de los hombres, será destruido, pero si procede de Dios, no seréis capaces de destruirlos, y quizás os encontréis incluso luchando contra Dios>>. Decidieron seguir el consejo de Gamaliel, y sobrevino un período de paz y de tranquilidad en Jerusalén, durante el cual el nuevo evangelio acerca de Jesús se difundió rápidamente.

\par 
%\textsuperscript{(2068.1)}
\textsuperscript{194:4.11} Y así, todo fue bien en Jerusalén hasta el momento en que una gran cantidad de griegos vino desde Alejandría. Dos alumnos de Rodán llegaron a Jerusalén e hicieron muchos conversos entre los helenistas. Entre sus primeros conversos se encontraban Esteban y Bernabé. Estos hábiles griegos no compartían tanto el punto de vista judío, y no se amoldaban tan bien a la manera de adorar de los judíos ni a otras prácticas ceremoniales. Las actividades de estos creyentes griegos fueron las que pusieron fin a las pacíficas relaciones entre la fraternidad de Jesús y los fariseos y saduceos. Esteban y su compañero griego empezaron a predicar de manera más acorde a como Jesús había enseñado, y esto les llevó a un conflicto inmediato con los dirigentes judíos. En uno de los sermones públicos de Esteban, cuando éste llegó a la parte inaceptable de su discurso, prescindieron de todas las formalidades jurídicas y procedieron a lapidarlo a muerte allí mismo.

\par 
%\textsuperscript{(2068.2)}
\textsuperscript{194:4.12} Esteban, el jefe de la colonia griega de los creyentes en Jesús de Jerusalén, se convirtió así en el primer mártir de la nueva fe y en la causa específica de la organización oficial de la iglesia cristiana primitiva. Los creyentes hicieron frente a esta nueva crisis reconociendo que ya no podían continuar como una secta dentro de la religión judía. Todos estuvieron de acuerdo en que debían separarse de los no creyentes. Un mes después de la muerte de Esteban, la iglesia de Jerusalén había sido organizada bajo la dirección de Pedro, y Santiago, el hermano de Jesús, había sido nombrado jefe titular.

\par 
%\textsuperscript{(2068.3)}
\textsuperscript{194:4.13} Entonces estallaron las nuevas e implacables persecuciones por parte de los judíos, de manera que los instructores activos de la nueva religión acerca de Jesús, llamada posteriormente cristianismo en Antioquía, salieron hasta los confines del imperio proclamando a Jesús. Antes de la época de Pablo, los griegos fueron los que se encargaron de difundir este mensaje. Estos primeros misioneros, así como los que vinieron después, siguieron los pasos del antiguo itinerario de Alejandro, dirigiéndose por el camino de Gaza y Tiro hasta Antioquía, luego desde Asia Menor hasta Macedonia, y después continuaron hasta Roma y las partes más distantes del imperio.


\chapter{Documento 195. Después de Pentecostés}
\par 
%\textsuperscript{(2069.1)}
\textsuperscript{195:0.1} LOS resultados de la predicación de Pedro, el día de Pentecostés, tuvieron tales efectos que decidieron la política futura y determinaron los planes de la mayoría de los apóstoles en sus esfuerzos por proclamar el evangelio del reino. Pedro fue el verdadero fundador de la iglesia cristiana; Pablo llevó el mensaje cristiano a los gentiles, y los creyentes griegos lo propagaron por todo el imperio romano.

\par 
%\textsuperscript{(2069.2)}
\textsuperscript{195:0.2} Los hebreos, atados por la tradición y tiranizados por los sacerdotes, se negaron a aceptar, como pueblo, tanto el evangelio de Jesús sobre la paternidad de Dios y la fraternidad de los hombres, como la proclamación de Pedro y de Pablo sobre la resurrección y la ascensión de Cristo (el cristianismo posterior), pero el resto del imperio romano resultó ser receptivo a las enseñanzas cristianas en desarrollo. En esta época, la civilización occidental era intelectual, estaba cansada de la guerra y era totalmente escéptica respecto a todas las religiones y filosofías universales existentes. Los pueblos del mundo occidental, beneficiarios de la cultura griega, tenían una tradición venerada de un magnífico pasado. Podían contemplar la herencia de las grandes realizaciones conseguidas en filosofía, arte, literatura y progreso político. Pero a pesar de todos estos logros, no tenían una religión satisfactoria para el alma. Sus anhelos espirituales continuaban insatisfechos.

\par 
%\textsuperscript{(2069.3)}
\textsuperscript{195:0.3} Las enseñanzas de Jesús, contenidas en el mensaje cristiano, fueron introducidas repentinamente en esta etapa de la sociedad humana. Un nuevo orden de vida fue presentado así a los corazones hambrientos de estos pueblos occidentales. Esta situación significó un conflicto inmediato entre las antiguas prácticas religiosas y la nueva versión cristianizada del mensaje de Jesús al mundo. Este conflicto tenía que terminar o bien en una victoria inequívoca de lo antiguo o de lo nuevo, o en algún tipo de \textit{compromiso}. La historia demuestra que la lucha terminó en un compromiso. El cristianismo se atrevió a abarcar demasiadas cosas como para que un pueblo cualquiera pudiera asimilarlas en una o dos generaciones. No se trataba de un simple llamamiento espiritual, tal como Jesús lo había presentado a las almas de los hombres; el cristianismo adoptó muy pronto una actitud decidida sobre los ritos religiosos, la educación, la magia, la medicina, el arte, la literatura, la ley, el gobierno, la moral, la reglamentación sexual, la poligamia y, en menor grado, incluso la esclavitud. El cristianismo no se presentó simplemente como una nueva religión ---cosa que estaban esperando todo el imperio romano y todo Oriente--- sino como un \textit{nuevoorden de sociedad humana}. Y esta pretensión como tal precipitó rápidamente el conflicto sociomoral de los siglos. Los ideales de Jesús, tal como estaban reinterpretados por la filosofía griega y socializados en el cristianismo, ahora desafiaron audazmente las tradiciones de la raza humana incorporadas en la ética, la moral y las religiones de la civilización occidental.

\par 
%\textsuperscript{(2069.4)}
\textsuperscript{195:0.4} Al principio, el cristianismo sólo hizo conversiones en las capas sociales y económicas más bajas. Pero desde el comienzo del siglo segundo, lo mejor de la cultura grecorromana se orientó cada vez más hacia este nuevo orden de creencia cristiana, este nuevo concepto del propósito de la vida y de la meta de la existencia.

\par 
%\textsuperscript{(2070.1)}
\textsuperscript{195:0.5} Este nuevo mensaje de origen judío, que casi había fracasado en su país natal, ¿cómo pudo captar de manera tan rápida y eficaz las mejores mentes del imperio romano? El triunfo del cristianismo sobre las religiones filosóficas y los cultos de misterio se debió a los factores siguientes:

\par 
%\textsuperscript{(2070.2)}
\textsuperscript{195:0.6} 1. La organización. Pablo era un gran organizador y sus sucesores se mantuvieron a su altura.

\par 
%\textsuperscript{(2070.3)}
\textsuperscript{195:0.7} 2. El cristianismo estaba totalmente helenizado. Englobaba lo mejor de la filosofía griega así como la crema de la teología hebrea.

\par 
%\textsuperscript{(2070.4)}
\textsuperscript{195:0.8} 3. Pero por encima de todo, contenía un nuevo y gran \textit{ideal}, el eco de la vida de donación de Jesús y el reflejo de su mensaje de salvación para toda la humanidad.

\par 
%\textsuperscript{(2070.5)}
\textsuperscript{195:0.9} 4. Los dirigentes cristianos estaban dispuestos a hacer tales compromisos con el mitracismo, que la mitad más valiosa de sus partidarios fue conquistada para el culto de Antioquía.

\par 
%\textsuperscript{(2070.6)}
\textsuperscript{195:0.10} 5. Asimismo, la generación siguiente y las generaciones posteriores de dirigentes cristianos hicieron tales compromisos adicionales con el paganismo, que incluso el emperador romano Constantino fue conquistado para la nueva religión.

\par 
%\textsuperscript{(2070.7)}
\textsuperscript{195:0.11} Pero los cristianos hicieron un astuto trato con los paganos, porque adoptaron la pompa de sus ritos, y al mismo tiempo les obligaron a aceptar la versión helenizada del cristianismo paulino. El acuerdo que hicieron con los paganos fue mejor que el que concluyeron con el culto mitríaco, pero incluso en este compromiso inicial salieron más que vencedores, porque consiguieron eliminar las vergonzosas inmoralidades así como otras numerosas prácticas reprensibles del misterio persa.

\par 
%\textsuperscript{(2070.8)}
\textsuperscript{195:0.12} Con acierto o sin él, estos primeros dirigentes del cristianismo comprometieron deliberadamente los \textit{ideales} de Jesús en un esfuerzo por salvar y promover muchas de sus \textit{ideas;} y tuvieron un éxito notable. ¡Pero no os engañéis! Estos ideales comprometidos del Maestro continúan latentes en su evangelio, y terminarán por afirmar todos sus poderes en el mundo.

\par 
%\textsuperscript{(2070.9)}
\textsuperscript{195:0.13} Mediante esta paganización del cristianismo, el antiguo orden consiguió muchas victorias menores de naturaleza ritualista, pero los cristianos obtuvieron la supremacía, por cuanto:

\par 
%\textsuperscript{(2070.10)}
\textsuperscript{195:0.14} 1. Hicieron resonar una nota nueva y enormemente más elevada en la moral humana.

\par 
%\textsuperscript{(2070.11)}
\textsuperscript{195:0.15} 2. Dieron al mundo un nuevo concepto de Dios mucho más ampliado.

\par 
%\textsuperscript{(2070.12)}
\textsuperscript{195:0.16} 3. La esperanza de la inmortalidad se volvió una parte de las seguridades de una religión reconocida.

\par 
%\textsuperscript{(2070.13)}
\textsuperscript{195:0.17} 4. Jesús de Nazaret fue ofrecido al alma hambrienta de los hombres.

\par 
%\textsuperscript{(2070.14)}
\textsuperscript{195:0.18} Muchas grandes verdades enseñadas por Jesús estuvieron a punto de perderse en estos primeros compromisos, pero continúan adormecidas en esta religión de cristianismo paganizado, que era a su vez la versión paulina de la vida y las enseñanzas del Hijo del Hombre. Antes de ser paganizado, el cristianismo fue primero completamente helenizado. El cristianismo le debe mucho, muchísimo a los griegos. Un griego de Egipto fue el que se levantó en Nicea con tanta valentía, y desafió a esta asamblea con tanta intrepidez, que el concilio no se atrevió a oscurecer el concepto de la naturaleza de Jesús hasta el punto de que la auténtica verdad de su donación hubiera corrido el peligro de perderse para el mundo. Este griego se llamaba Atanasio, y si no hubiera sido por la elocuencia y la lógica de este creyente, las opiniones religiosas de Arrio habrían triunfado.

\section*{1. La influencia de los griegos}
\par 
%\textsuperscript{(2071.1)}
\textsuperscript{195:1.1} La helenización del cristianismo empezó realmente el día memorable en que el apóstol Pablo se presentó ante el consejo del Areópago de Atenas y habló a los atenienses sobre el <<Dios Desconocido>>. Allí, a la sombra del Acrópolis, este ciudadano romano proclamó a aquellos griegos su versión de la nueva religión que había nacido en la tierra judía de Galilea. Había una extraña similitud entre la filosofía griega y muchas enseñanzas de Jesús. Tenían una meta común: las dos aspiraban al \textit{surgimiento del individuo}. Los griegos, a su surgimiento social y político; Jesús, a su surgimiento moral y espiritual. Los griegos enseñaban el liberalismo intelectual que conducía a la libertad política; Jesús enseñaba el liberalismo espiritual que conducía a la libertad religiosa. Estas dos ideas reunidas formaban una nueva y poderosa carta constitucional para la libertad humana; presagiaban la libertad social, política y espiritual del hombre.

\par 
%\textsuperscript{(2071.2)}
\textsuperscript{195:1.2} El cristianismo surgió a la existencia y triunfó sobre todas las religiones rivales debido principalmente a dos factores:

\par 
%\textsuperscript{(2071.3)}
\textsuperscript{195:1.3} 1. La mente griega estaba dispuesta a sacar ideas nuevas y buenas incluso de los judíos.

\par 
%\textsuperscript{(2071.4)}
\textsuperscript{195:1.4} 2. Pablo y sus sucesores estaban dispuestos a hacer compromisos, y sabían hacerlo con astucia y sagacidad; eran unos negociadores perspicaces en materia teológica.

\par 
%\textsuperscript{(2071.5)}
\textsuperscript{195:1.5} Cuando Pablo se levantó en Atenas para predicar <<Cristo y Aquel que fue crucificado>>, los griegos estaban espiritualmente hambrientos; eran investigadores, estaban interesados y buscaban realmente la verdad espiritual. No olvidéis nunca que al principio los romanos combatieron el cristianismo, mientras que los griegos lo abrazaron, y que fueron los griegos los que posteriormente forzaron literalmente a los romanos a aceptar esta nueva religión, tal como ya estaba modificada, como parte de la cultura griega.

\par 
%\textsuperscript{(2071.6)}
\textsuperscript{195:1.6} Los griegos veneraban la belleza y los judíos la santidad, pero los dos pueblos amaban la verdad. Durante siglos, los griegos habían examinado seriamente y discutido con sinceridad todos los problemas humanos ---sociales, económicos, políticos y filosóficos--- excepto la religión. Pocos griegos habían prestado mucha atención a la religión; ni siquiera tomaban muy en serio la suya propia. Durante siglos, los judíos habían descuidado estas otras esferas del pensamiento, consagrando su atención a la religión. Se tomaban muy en serio su religión, demasiado en serio. Iluminado por el contenido del mensaje de Jesús, el producto unificado de los siglos de pensamiento de estos dos pueblos se convirtió entonces en la fuerza motriz de un nuevo orden de sociedad humana y, hasta cierto punto, de un nuevo orden de creencias y de prácticas religiosas humanas.

\par 
%\textsuperscript{(2071.7)}
\textsuperscript{195:1.7} Cuando Alejandro propagó la civilización helenista por el Cercano Oriente, la influencia de la cultura griega ya había penetrado en los países del Mediterráneo occidental. A los griegos les fue muy bien con su religión y su política mientras vivieron en pequeñas ciudades-Estado, pero cuando el rey de Macedonia se atrevió a expandir Grecia en un imperio que se extendía desde el Adriático hasta el Indo, los problemas empezaron. El arte y la filosofía de Grecia estaban completamente a la altura de la expansión imperial, pero no sucedía lo mismo con su administración política o su religión. Después de que las ciudades-Estado de Grecia se expandieron en un imperio, sus dioses más bien parroquiales parecieron un poco raros. Los griegos estaban buscando realmente a \textit{un solo Dios}, a un Dios más grande y mejor, cuando les llegó la versión cristianizada de la religión judía más antigua.

\par 
%\textsuperscript{(2072.1)}
\textsuperscript{195:1.8} El imperio heleno, como tal, no podía durar. Su influencia cultural continuó, pero solamente perduró después de adquirir de occidente el genio político romano para administrar un imperio, y después de obtener de oriente una religión cuyo Dios único poseía una dignidad imperial.

\par 
%\textsuperscript{(2072.2)}
\textsuperscript{195:1.9} La cultura helenista ya había alcanzado sus niveles más altos en el siglo primero después de Cristo; su retroceso había empezado; el conocimiento avanzaba, pero el genio declinaba. En este preciso momento fue cuando las ideas y los ideales de Jesús, que estaban parcialmente incorporados en el cristianismo, contribuyeron en parte a salvar la cultura y el conocimiento griegos.

\par 
%\textsuperscript{(2072.3)}
\textsuperscript{195:1.10} Alejandro había atacado oriente con el don cultural de la civilización griega; Pablo invadió occidente con la versión cristiana del evangelio de Jesús. Y el cristianismo helenizado echó raíces en todos los lugares de occidente donde prevaleció la cultura griega.

\par 
%\textsuperscript{(2072.4)}
\textsuperscript{195:1.11} Aunque la versión oriental del mensaje de Jesús permaneció más fiel a sus enseñanzas, continuó siguiendo la actitud intransigente de Abner. Nunca progresó como la versión helenizada, y acabó por perderse en el movimiento islámico.

\section*{2. La influencia romana}
\par 
%\textsuperscript{(2072.5)}
\textsuperscript{195:2.1} Los romanos se apoderaron en su totalidad de la cultura griega, sustituyendo el gobierno echado a suertes por un gobierno representativo. Este cambio favoreció pronto al cristianismo, ya que Roma introdujo en todo el mundo occidental una nueva tolerancia por los idiomas y los pueblos extranjeros, e incluso por las religiones ajenas.

\par 
%\textsuperscript{(2072.6)}
\textsuperscript{195:2.2} En Roma, muchas de las primeras persecuciones contra los cristianos se debieron únicamente a la desafortunada utilización, en sus predicaciones, de la palabra <<reino>>. Los romanos eran tolerantes con todas y cada una de las religiones, pero muy susceptibles ante cualquier cosa que tuviera sabor a rivalidad política. Por eso, cuando estas primeras persecuciones ---debidas tan ampliamente a los malentendidos--- desaparecieron, el campo para la propaganda religiosa se encontró completamente abierto. A los romanos les interesaba la administración política; el arte o la religión les resultaban indiferentes, pero eran excepcionalmente tolerantes con los dos.

\par 
%\textsuperscript{(2072.7)}
\textsuperscript{195:2.3} La ley oriental era rígida y arbitraria; la ley griega era fluida y artística; la ley romana tenía dignidad y causaba respeto. La educación romana engendraba una lealtad inaudita e imperturbable. Los primeros romanos eran unos individuos políticamente dedicados y sublimemente consagrados. Eran honrados, incondicionales y entregados a sus ideales, pero sin una religión digna de ese nombre. No es de extrañar que sus educadores griegos fueran capaces de persuadirlos para que aceptaran el cristianismo de Pablo.

\par 
%\textsuperscript{(2072.8)}
\textsuperscript{195:2.4} Estos romanos eran un gran pueblo. Podían gobernar Occidente porque se gobernaban a sí mismos. Esta honradez sin igual, esta devoción y este firme autocontrol constituían un terreno ideal para la recepción y el crecimiento del cristianismo.

\par 
%\textsuperscript{(2072.9)}
\textsuperscript{195:2.5} A estos grecorromanos les resultaba igual de fácil consagrarse espiritualmente a una iglesia institucional, como hacerlo políticamente al Estado. Los romanos sólo lucharon contra la iglesia cuando temieron que ésta le hiciera la competencia al Estado. Como Roma tenía poca filosofía nacional o cultura nativa, se apoderó de la cultura griega como si fuera suya y adoptó audazmente a Cristo como filosofía moral. El cristianismo se convirtió en la cultura moral de Roma pero difícilmente en su religión, en el sentido de ser una experiencia individual de crecimiento espiritual para aquellos que abrazaron la nueva religión de una manera tan masiva. Es verdad que muchas personas penetraron bajo la superficie de toda esta religión estatal y encontraron, para alimento de su alma, los verdaderos valores de los significados ocultos contenidos en las verdades latentes del cristianismo helenizado y paganizado.

\par 
%\textsuperscript{(2073.1)}
\textsuperscript{195:2.6} Los estoicos y su vigoroso llamamiento a <<la naturaleza y la conciencia>> habían preparado mucho mejor toda Roma para recibir a Cristo, al menos en un sentido intelectual. El romano era un jurista por naturaleza y por educación; veneraba incluso las leyes de la naturaleza. Y ahora, en el cristianismo, discernía las leyes de Dios en las leyes de la naturaleza. Un pueblo que podía dar a un Cicerón y a un Virgilio estaba maduro para el cristianismo helenizado de Pablo.

\par 
%\textsuperscript{(2073.2)}
\textsuperscript{195:2.7} Y así, estos griegos romanizados forzaron tanto a los judíos como a los cristianos a hacer filosófica su religión, a coordinar sus ideas y sistematizar sus ideales, a adaptar las prácticas religiosas a la marcha existente de la vida. Todo esto fue enormemente favorecido por la traducción al griego de las escrituras hebreas y la redacción posterior del Nuevo Testamento en lengua griega.

\par 
%\textsuperscript{(2073.3)}
\textsuperscript{195:2.8} Durante largo tiempo, los griegos, a diferencia de los judíos y de otros muchos pueblos, habían creído provisionalmente en la inmortalidad, en alguna clase de supervivencia después de la muerte. Puesto que éste era el centro mismo de la enseñanza de Jesús, era seguro que el cristianismo ejercería un poderoso atractivo sobre ellos.

\par 
%\textsuperscript{(2073.4)}
\textsuperscript{195:2.9} Una sucesión de victorias de la cultura griega y de la política romana había consolidado a los países mediterráneos en un solo imperio, con un solo idioma y una sola cultura, y había preparado al mundo occidental para un solo Dios. El judaísmo proporcionaba este Dios, pero el judaísmo era inaceptable como religión para estos griegos romanizados. Filón ayudó a algunos a mitigar sus objeciones, pero el cristianismo les reveló un concepto aún mejor de un solo Dios, y lo aceptaron inmediatamente.

\section*{3. Bajo el imperio romano}
\par 
%\textsuperscript{(2073.5)}
\textsuperscript{195:3.1} Después de la consolidación del régimen político romano y tras la propagación del cristianismo, los cristianos se encontraron con un solo Dios, un gran concepto religioso, pero sin imperio. Los grecorromanos se encontraron con un gran imperio, pero sin un Dios que sirviera como concepto religioso satisfactorio para el culto del imperio y la unificación espiritual. Los cristianos aceptaron el imperio, y el imperio adoptó el cristianismo. Los romanos proporcionaron una unidad de gobierno político; los griegos, una unidad de cultura y de instrucción; y el cristianismo, una unidad de pensamiento y de práctica religiosos.

\par 
%\textsuperscript{(2073.6)}
\textsuperscript{195:3.2} Roma venció la tradición del nacionalismo mediante un universalismo imperial, y por primera vez en la historia hizo posible que diversas razas y naciones aceptaran, al menos nominalmente, una misma religión.

\par 
%\textsuperscript{(2073.7)}
\textsuperscript{195:3.3} El cristianismo tuvo la aceptación de Roma en un momento en que había grandes discusiones entre las vigorosas enseñanzas de los estoicos y las promesas de salvación de los cultos de misterio. El cristianismo aportó un consuelo reconfortante y un poder liberador a un pueblo espiritualmente hambriento cuyo idioma no contenía la palabra <<desinterés>>.

\par 
%\textsuperscript{(2073.8)}
\textsuperscript{195:3.4} Lo que dio mayor poder al cristianismo fue la manera en que sus creyentes vivieron una vida de servicio, e incluso la forma en que murieron por su fe durante los primeros tiempos de persecuciones radicales.

\par 
%\textsuperscript{(2073.9)}
\textsuperscript{195:3.5} La enseñanza acerca del amor de Cristo por los niños pronto puso fin a la práctica generalizada de exponer a la muerte a los niños no deseados, en particular a las niñas.

\par 
%\textsuperscript{(2074.1)}
\textsuperscript{195:3.6} El primer modelo de culto cristiano fue ampliamente tomado de las sinagogas judías, y modificado por el ritual mitríaco; más tarde se añadió mucha pompa pagana. Los griegos cristianizados, prosélitos del judaísmo, componían la columna vertebral de la iglesia cristiana primitiva.

\par 
%\textsuperscript{(2074.2)}
\textsuperscript{195:3.7} El siglo segundo después de Cristo fue el mejor período de toda la historia mundial para que una buena religión progresara en el mundo occidental. Durante el siglo primero, el cristianismo se había preparado, mediante la lucha y los compromisos, para echar raíces y difundirse rápidamente. El cristianismo adoptó al emperador, y más tarde éste adoptó el cristianismo. Fue una gran época para la difusión de una nueva religión. Había libertad religiosa, los viajes se habían generalizado y el libre pensamiento no tenía trabas.

\par 
%\textsuperscript{(2074.3)}
\textsuperscript{195:3.8} El ímpetu espiritual de aceptar nominalmente el cristianismo helenizado llegó a Roma demasiado tarde para impedir su decadencia moral bien avanzada, o para compensar el deterioro racial ya bien establecido y en aumento. Esta nueva religión era una necesidad cultural para la Roma imperial, y es extremadamente desafortunado que no se convirtiera en un medio de salvación espiritual en un sentido más amplio.

\par 
%\textsuperscript{(2074.4)}
\textsuperscript{195:3.9} Ni siquiera una buena religión podía salvar a un gran imperio de los resultados inevitables de la falta de participación individual en los asuntos del gobierno, del paternalismo excesivo, del exceso de impuestos y de los abusos flagrantes en su recaudación, de un comercio desequilibrado con el Levante que agotaba el oro, de la locura por las diversiones, de la estandarización romana, de la degradación de la mujer, de la esclavitud y la decadencia racial, de las calamidades físicas y de una iglesia estatal que se institucionalizó hasta el punto de llegar casi a la esterilidad espiritual.

\par 
%\textsuperscript{(2074.5)}
\textsuperscript{195:3.10} Sin embargo, las condiciones no eran tan malas en Alejandría. Las primeras escuelas siguieron conservando muchas enseñanzas de Jesús libres de compromisos. Pantaenos enseñó a Clemente, y luego siguió a Natanael para proclamar a Cristo en la India. Aunque algunos ideales de Jesús fueron sacrificados para construir el cristianismo, hay que indicar con toda justicia que a finales del siglo segundo prácticamente todas las grandes mentes del mundo grecorromano se habían vuelto cristianas. El triunfo se acercaba a su culminación.

\par 
%\textsuperscript{(2074.6)}
\textsuperscript{195:3.11} Y este imperio romano duró el tiempo suficiente como para asegurar la supervivencia del cristianismo, incluso después de que se derrumbara el imperio. Pero a menudo hemos conjeturado sobre qué hubiera sucedido en Roma y en el mundo si se hubiera aceptado el evangelio del reino en lugar del cristianismo griego.

\section*{4. La edad de las tinieblas en Europa}
\par 
%\textsuperscript{(2074.7)}
\textsuperscript{195:4.1} Como la iglesia era una agregada de la sociedad y la aliada de la política, estaba destinada a compartir la decadencia intelectual y espiritual de la llamada <<edad de las tinieblas>> en Europa. Durante este período, la religión se volvió cada vez más monástica, ascética y legalizada. En un sentido espiritual, el cristianismo estaba en hibernación. Durante todo este período existió, al lado de esta religión adormecida y secularizada, una corriente continua de misticismo, una experiencia espiritual fantástica que rayaba en la irrealidad y filosóficamente similar al panteísmo.

\par 
%\textsuperscript{(2074.8)}
\textsuperscript{195:4.2} Durante estos siglos sombríos y desesperantes, la religión volvió a ser prácticamente de segunda mano. El individuo se encontraba casi perdido ante la autoridad, la tradición y el dictado de una iglesia que lo eclipsaba todo. Una nueva amenaza espiritual surgió con la creación de una constelación de <<santos>> que se suponía tenían una influencia especial en los tribunales divinos y que, por consiguiente, si se recurría eficazmente a ellos, podían interceder ante los Dioses a favor de los hombres.

\par 
%\textsuperscript{(2075.1)}
\textsuperscript{195:4.3} Aunque era impotente para detener la edad de las tinieblas que se aproximaba, el cristianismo estaba suficientemente socializado y paganizado como para encontrarse mejor preparado para sobrevivir a este largo período de tinieblas morales y de estancamiento espiritual. Siguió viviendo durante la larga noche de la civilización occidental y aún desempeñaba su función como influencia moral en el mundo en los albores del renacimiento. Después de atravesar la edad de las tinieblas, la rehabilitación del cristianismo se tradujo en la aparición de numerosas sectas de enseñanzas cristianas, cuyas creencias estaban adaptadas a unos tipos especiales ---intelectuales, emocionales y espirituales--- de personalidades humanas. Muchos de estos grupos cristianos especiales, o familias religiosas, continúan existiendo en el momento de efectuar esta presentación.

\par 
%\textsuperscript{(2075.2)}
\textsuperscript{195:4.4} El cristianismo muestra en su historia que tuvo su origen en la transformación no intencionada de la religión de Jesús en una religión acerca de Jesús. Además, su historia indica que experimentó la helenización, la paganización, la secularización, la institucionalización, el deterioro intelectual, la decadencia espiritual, la hibernación moral, la amenaza de extinción, el rejuvenecimiento posterior, la fragmentación y una rehabilitación relativa más reciente. Este historial indica una vitalidad inherente y la posesión de inmensos recursos de recuperación. Y este mismo cristianismo está ahora presente en el mundo civilizado de los pueblos occidentales, haciendo frente a una lucha por la existencia que es aún más inquietante que aquellas crisis memorables que caracterizaron sus pasadas batallas por conseguir el dominio.

\par 
%\textsuperscript{(2075.3)}
\textsuperscript{195:4.5} La religión se enfrenta ahora con el desafío de una nueva era de mentalidad científica y de tendencias materialistas. En este conflicto gigantesco entre lo secular y lo espiritual, la religión de Jesús acabará por triunfar.

\section*{5. El problema moderno}
\par 
%\textsuperscript{(2075.4)}
\textsuperscript{195:5.1} El siglo veinte ha traído al cristianismo y a todas las demás religiones unos nuevos problemas que tienen que resolver. Cuanto más se eleva una civilización, mayor es el deber que tiene el hombre de <<buscar primero las realidades del cielo>> en todos sus esfuerzos por estabilizar la sociedad y facilitar la solución de sus problemas materiales.

\par 
%\textsuperscript{(2075.5)}
\textsuperscript{195:5.2} La verdad se vuelve a veces confusa e incluso engañosa cuando es fragmentada, segregada, aislada y analizada con exceso. La verdad viviente sólo enseña bien al buscador de la verdad cuando es abrazada en su totalidad y como una realidad espiritual viviente, no como un hecho de la ciencia material o una inspiración de un arte intermedio.

\par 
%\textsuperscript{(2075.6)}
\textsuperscript{195:5.3} La religión es la revelación al hombre de su destino divino y eterno. La religión es una experiencia puramente personal y espiritual, y siempre se debe diferenciar de las otras formas elevadas de pensamiento humano, tales como:

\par 
%\textsuperscript{(2075.7)}
\textsuperscript{195:5.4} 1. La actitud lógica hacia las cosas de la realidad material.

\par 
%\textsuperscript{(2075.8)}
\textsuperscript{195:5.5} 2. La apreciación estética de la belleza, en contraste con la fealdad.

\par 
%\textsuperscript{(2075.9)}
\textsuperscript{195:5.6} 3. El reconocimiento ético de las obligaciones sociales y del deber político.

\par 
%\textsuperscript{(2075.10)}
\textsuperscript{195:5.7} 4. Incluso el sentido de la moral humana, en sí mismo y por sí mismo, no es religioso.

\par 
%\textsuperscript{(2075.11)}
\textsuperscript{195:5.8} La religión está destinada a encontrar en el universo aquellos valores que inspiran la fe, la confianza y la seguridad; la religión culmina en la adoración. La religión descubre para el alma aquellos valores supremos que contrastan con los valores relativos descubiertos por la mente. Esta perspicacia sobrehumana sólo se puede obtener mediante una experiencia religiosa auténtica.

\par 
%\textsuperscript{(2075.12)}
\textsuperscript{195:5.9} Mantener un sistema social duradero sin una moral basada en las realidades espirituales es igual de imposible que mantener el sistema solar sin la gravedad.

\par 
%\textsuperscript{(2076.1)}
\textsuperscript{195:5.10} No intentéis satisfacer la curiosidad o contentar todas las aventuras latentes que surgen dentro del alma, en una corta vida en la carne. ¡Tened paciencia! No caigáis en la tentación de zambulliros de manera desordenada en aventuras baratas y sórdidas. Aprovechad vuestras energías y refrenad vuestras pasiones; permaneced tranquilos mientras esperáis el desarrollo majestuoso de una carrera sin fin de aventuras progresivas y de descubrimientos emocionantes.

\par 
%\textsuperscript{(2076.2)}
\textsuperscript{195:5.11} En la confusión sobre el origen del hombre, no perdáis de vista su destino eterno. No olvidéis que Jesús amaba incluso a los niños pequeños, y que indicó claramente para siempre el gran valor de la personalidad humana.

\par 
%\textsuperscript{(2076.3)}
\textsuperscript{195:5.12} Al observar el mundo, recordad que las manchas oscuras de maldad que veis resaltan sobre un fondo blanco de bondad última. No observáis unas simples manchas blancas de bondad que destacan pobremente sobre un fondo oscuro de maldad.

\par 
%\textsuperscript{(2076.4)}
\textsuperscript{195:5.13} Puesto que hay tantas verdades buenas que publicar y proclamar, ¿por qué los hombres habrían de hacer tanto hincapié en el mal que hay en el mundo, simplemente porque el mal parece ser un hecho? Los encantos de los valores espirituales de la verdad son más agradables y edificantes que el fenómeno del mal.

\par 
%\textsuperscript{(2076.5)}
\textsuperscript{195:5.14} En religión, Jesús defendió y siguió el método de la experiencia, al igual que la ciencia moderna utiliza la técnica experimental. Encontramos a Dios mediante las directrices de la perspicacia espiritual, pero nos acercamos a esta perspicacia del alma mediante el amor de lo bello, la búsqueda de la verdad, la fidelidad al deber y la adoración de la bondad divina. Pero de todos estos valores, el amor es el verdadero guía que conduce a la perspicacia auténtica.

\section*{6. El materialismo}
\par 
%\textsuperscript{(2076.6)}
\textsuperscript{195:6.1} Los científicos han precipitado involuntariamente a la humanidad hacia un pánico materialista; han desencadenado un asedio irreflexivo al banco moral de los siglos, pero este banco de la experiencia humana tiene enormes recursos espirituales; puede soportar las demandas que se le hagan. Sólo los hombres irreflexivos se dejan llevar por el pánico con respecto a los activos espirituales de la raza humana. Cuando el pánico laico-materialista haya pasado, la religión de Jesús no se encontrará en bancarrota. El banco espiritual del reino de los cielos pagará con fe, esperanza y seguridad moral a todos los que recurran a él <<en Su nombre>>.

\par 
%\textsuperscript{(2076.7)}
\textsuperscript{195:6.2} Cualquiera que sea el conflicto aparente entre el materialismo y las enseñanzas de Jesús, podéis estar seguros de que las enseñanzas del Maestro triunfarán plenamente en las eras por venir. En realidad, la verdadera religión no puede meterse en ninguna controversia con la ciencia, pues no se ocupa en absoluto de las cosas materiales. A la religión, la ciencia le resulta sencillamente indiferente, aunque es comprensiva con ella, mientras que se interesa supremamente por el \textit{científico}.

\par 
%\textsuperscript{(2076.8)}
\textsuperscript{195:6.3} La búsqueda del simple conocimiento, sin la interpretación concomitante de la sabiduría y la perspicacia espiritual de la experiencia religiosa, conduce finalmente al pesimismo y a la desesperación humana. Un conocimiento limitado es realmente desconcertante.

\par 
%\textsuperscript{(2076.9)}
\textsuperscript{195:6.4} En el momento de escribir este documento, lo peor de la era materialista ha pasado; ya está empezando a despuntar el día de una mejor comprensión. Las mejores mentes del mundo científico han dejado de tener una filosofía totalmente materialista, pero la gente común y corriente se inclina todavía en esa dirección a consecuencia de las enseñanzas anteriores. Pero esta era de realismo físico sólo es un episodio transitorio en la vida del hombre en la Tierra. La ciencia moderna ha dejado intacta a la verdadera religión ---las enseñanzas de Jesús tal como se traducen en la vida de sus creyentes. Todo lo que la ciencia ha hecho es destruir las ilusiones infantiles de las falsas interpretaciones de la vida.

\par 
%\textsuperscript{(2077.1)}
\textsuperscript{195:6.5} En lo que se refiere a la vida del hombre en la Tierra, la ciencia es una experiencia cuantitativa y la religión una experiencia cualitativa. La ciencia se ocupa de los fenómenos; la religión, de los orígenes, los valores y las metas. Indicar que las \textit{causas} son una explicación de los fenómenos físicos equivale a confesar que se ignoran los factores últimos, y al final sólo conduce al científico directamente de vuelta a la gran causa primera ---al Padre Universal del Paraíso.

\par 
%\textsuperscript{(2077.2)}
\textsuperscript{195:6.6} El paso violento de una era de milagros a una era de máquinas ha resultado ser enteramente perturbador para el hombre. El ingenio y la habilidad de las falsas filosofías mecanicistas desmienten sus mismas opiniones mecanicistas. La agilidad fatalista de la mente de un materialista contradice para siempre sus afirmaciones de que el universo es un fenómeno energético ciego y carente de finalidad.

\par 
%\textsuperscript{(2077.3)}
\textsuperscript{195:6.7} Tanto el naturalismo mecanicista de algunos hombres supuestamente instruidos como el laicismo irreflexivo del hombre de la calle se ocupan exclusivamente de \textit{cosas;} están desprovistos de todo verdadero valor, sanción y satisfacción de naturaleza espiritual, y también están exentos de fe, de esperanza y de seguridades eternas. Uno de los grandes problemas de la vida moderna es que el hombre se cree demasiado ocupado como para encontrar tiempo para la meditación espiritual y la devoción religiosa.

\par 
%\textsuperscript{(2077.4)}
\textsuperscript{195:6.8} El materialismo reduce al hombre a un estado de autómata sin alma, y lo convierte en un simple símbolo aritmético que ocupa un lugar impotente en la fórmula matemática de un universo realista y mecanicista. Pero, ¿de dónde viene todo este inmenso universo de matemáticas, sin un Maestro Matemático? La ciencia puede discurrir sobre la conservación de la materia, pero la religión valida la conservación del alma de los hombres ---se ocupa de su experiencia con las realidades espirituales y los valores eternos.

\par 
%\textsuperscript{(2077.5)}
\textsuperscript{195:6.9} El sociólogo materialista de hoy examina una comunidad, hace un informe sobre ella y deja a la gente tal como las encontró. Hace mil novecientos años, unos galileos ignorantes observaron a Jesús dar su vida como aportación espiritual a la experiencia interior del hombre, y luego salieron y pusieron boca abajo todo el imperio romano.

\par 
%\textsuperscript{(2077.6)}
\textsuperscript{195:6.10} Pero los dirigentes religiosos cometen un grave error cuando intentan llamar al hombre moderno a la lucha espiritual al son de las trompetas de la Edad Media. La religión debe proveerse de lemas nuevos y actualizados. Ni la democracia ni ninguna otra panacea política podrán reemplazar el progreso espiritual. Las falsas religiones pueden representar una evasión de la realidad, pero Jesús, en su evangelio, puso al hombre mortal en la entrada misma de una realidad eterna de progreso espiritual.

\par 
%\textsuperscript{(2077.7)}
\textsuperscript{195:6.11} Decir que la mente <<surgió>> de la materia no explica nada. Si el universo fuera simplemente un mecanismo y la mente fuera inseparable de la materia, nunca tendríamos dos interpretaciones diferentes de cualquier fenómeno observado. Los conceptos de la verdad, la belleza y la bondad no son inherentes ni a la física ni a la química. Una máquina no puede \textit{conocer}, y mucho menos conocer la verdad, tener hambre de rectitud y apreciar la bondad.

\par 
%\textsuperscript{(2077.8)}
\textsuperscript{195:6.12} La ciencia puede ser física, pero la mente del científico que discierne la verdad es al mismo tiempo supermaterial. La materia no conoce la verdad, ni puede amar la misericordia ni deleitarse con las realidades espirituales. Las convicciones morales basadas en la iluminación espiritual y arraigadas en la experiencia humana son tan reales y seguras como las deducciones matemáticas basadas en las observaciones físicas, pero se encuentran en un nivel diferente y más elevado.

\par 
%\textsuperscript{(2077.9)}
\textsuperscript{195:6.13} Si los hombres sólo fueran unas máquinas, reaccionarían de manera más o menos uniforme a un universo material. No existiría la individualidad, y mucho menos la personalidad.

\par 
%\textsuperscript{(2077.10)}
\textsuperscript{195:6.14} El hecho del mecanismo absoluto del Paraíso en el centro del universo de universos, en presencia de la volición incondicionada de la Fuente-Centro Segunda, asegura para siempre que los determinantes no son la ley exclusiva del cosmos. El materialismo está ahí, pero no es exclusivo; el mecanismo está ahí, pero no es incondicionado; el determinismo está ahí, pero no está solo.

\par 
%\textsuperscript{(2078.1)}
\textsuperscript{195:6.15} El universo finito de la materia se volvería finalmente uniforme y determinista si no fuera por la presencia combinada de la mente y el espíritu. La influencia de la mente cósmica inyecta constantemente espontaneidad incluso en los mundos materiales.

\par 
%\textsuperscript{(2078.2)}
\textsuperscript{195:6.16} En cualquier aspecto de la existencia, la libertad o la iniciativa es directamente proporcional al grado de influencia espiritual y de control de la mente cósmica; es decir, en la experiencia humana, al grado en que se hace realmente <<la voluntad del Padre>>. Así pues, una vez que habéis empezado a descubrir a Dios, ésta es la prueba decisiva de que Dios ya os ha encontrado.

\par 
%\textsuperscript{(2078.3)}
\textsuperscript{195:6.17} La búsqueda sincera de la bondad, la belleza y la verdad conduce a Dios. Y todo descubrimiento científico demuestra la existencia tanto de la libertad como de la uniformidad en el universo. El descubridor era libre de hacer su descubrimiento. La cosa descubierta es real y aparentemente uniforme, pues de otro modo no hubiera podido ser conocida como \textit{cosa}.

\section*{7. La vulnerabilidad del materialismo}
\par 
%\textsuperscript{(2078.4)}
\textsuperscript{195:7.1} Qué insensatez la del hombre con mentalidad materialista cuando permite que unas teorías tan vulnerables como las de un universo mecanicista le priven de los enormes recursos espirituales de la experiencia personal de la verdadera religión. Los hechos nunca están reñidos con la auténtica fe espiritual; las teorías sí pueden estarlo. La ciencia haría mejor en dedicarse a destruir la superstición, en lugar de intentar aniquilar la fe religiosa ---la creencia humana en las realidades espirituales y los valores divinos.

\par 
%\textsuperscript{(2078.5)}
\textsuperscript{195:7.2} La ciencia debería hacer materialmente por el hombre lo que la religión hace espiritualmente por él: ampliar el horizonte de la vida y engrandecer su personalidad. La verdadera ciencia no puede tener ninguna discrepancia duradera con la verdadera religión. El <<método científico>> es simplemente una vara intelectual para medir las aventuras materiales y los logros físicos. Pero como es material y enteramente intelectual, es totalmente inútil para evaluar las realidades espirituales y las experiencias religiosas.

\par 
%\textsuperscript{(2078.6)}
\textsuperscript{195:7.3} La contradicción del mecanicista moderno es la siguiente: Si este universo fuera simplemente material y el hombre sólo fuera una máquina, ese hombre sería enteramente incapaz de reconocerse como tal máquina; además, un hombre-máquina así sería totalmente inconsciente del hecho de que existe dicho universo material. El desaliento y la desesperación materialista de una ciencia mecanicista no han logrado reconocer el hecho de que la mente del científico está habitada por el espíritu, aunque la perspicacia supermaterial del científico es precisamente la que formula estos \textit{conceptos} erróneos y contradictorios en sí mismos de un universo materialista.

\par 
%\textsuperscript{(2078.7)}
\textsuperscript{195:7.4} Los valores paradisiacos de eternidad e infinidad, de verdad, belleza y bondad, están escondidos dentro de los hechos de los fenómenos de los universos del tiempo y del espacio. Pero es necesario el ojo de la fe de un mortal nacido del espíritu para detectar y discernir estos valores espirituales.

\par 
%\textsuperscript{(2078.8)}
\textsuperscript{195:7.5} Las realidades y los valores del progreso espiritual no son una <<proyección psicológica>> ---un simple sueño despierto y glorificado de la mente material. Estas cosas son las previsiones espirituales del Ajustador interior, del espíritu de Dios que vive en la mente del hombre. No dejéis que vuestros escarceos en los descubrimientos ligeramente vislumbrados de la <<relatividad>> alteren vuestros conceptos de la eternidad y de la infinidad de Dios. Y en todas vuestras tentativas relacionadas con la necesidad de \textit{expresaros}, no cometáis el error de omitir la \textit{expresión del Ajustador}, la manifestación de vuestro yo real y mejor.

\par 
%\textsuperscript{(2079.1)}
\textsuperscript{195:7.6} Si este universo sólo fuera material, el hombre material nunca sería capaz de llegar al concepto del carácter mecanicista de una existencia tan exclusivamente material. Este mismo \textit{concepto mecanicista} del universo es, en sí mismo, un fenómeno no material de la mente, y toda mente es de origen no material, por mucho que pueda dar la impresión de estar condicionada materialmente y controlada mecánicamente.

\par 
%\textsuperscript{(2079.2)}
\textsuperscript{195:7.7} El mecanismo mental parcialmente evolucionado del hombre mortal no está muy dotado de coherencia ni de sabiduría. La presunción del hombre sobrepasa a menudo su razón y elude su lógica.

\par 
%\textsuperscript{(2079.3)}
\textsuperscript{195:7.8} El mismo pesimismo del materialista más pesimista es, en sí y por sí mismo, una prueba suficiente de que el universo del pesimista no es totalmente material. Tanto el optimismo como el pesimismo son unas reacciones conceptuales que se producen en una mente que es consciente de los \textit{valores} así como de los \textit{hechos}. Si el universo fuera realmente lo que el materialista considera que es, entonces el hombre, como máquina humana, estaría privado de todo reconocimiento consciente de ese mismo \textit{hecho}. Sin la conciencia del concepto de los \textit{valores} dentro de la mente nacida del espíritu, el hombre no podría reconocer de ninguna manera el hecho del materialismo universal ni los fenómenos mecanicistas de la acción del universo. Una máquina no puede ser consciente de la naturaleza ni del valor de otra máquina.

\par 
%\textsuperscript{(2079.4)}
\textsuperscript{195:7.9} Una filosofía mecanicista de la vida y del universo no puede ser científica, porque la ciencia sólo reconoce y trata de los objetos materiales y de los hechos. La filosofía es inevitablemente supercientífica. El hombre es un hecho material de la naturaleza, pero su \textit{vida} es un fenómeno que trasciende los niveles materiales de la naturaleza, porque manifiesta los atributos controladores de la mente y las cualidades creativas del espíritu.

\par 
%\textsuperscript{(2079.5)}
\textsuperscript{195:7.10} El esfuerzo sincero del hombre por volverse mecanicista representa el fenómeno trágico del empeño inútil de ese hombre por suicidarse intelectual y moralmente. Pero no puede conseguirlo.

\par 
%\textsuperscript{(2079.6)}
\textsuperscript{195:7.11} Si el universo sólo fuera material y el hombre solamente una máquina, no existiría ninguna ciencia que animara al científico a postular esta mecanización del universo. Las máquinas no pueden medirse, clasificarse ni evaluarse a sí mismas. Esta tarea científica sólo podría ejecutarla una entidad con estatus de supermáquina.

\par 
%\textsuperscript{(2079.7)}
\textsuperscript{195:7.12} Si la realidad del universo no es más que una inmensa máquina, entonces el hombre debe estar fuera del universo y separado de él para poder reconocer este \textit{hecho} y ser consciente de la \textit{perspicacia} de esta \textit{evaluación}.

\par 
%\textsuperscript{(2079.8)}
\textsuperscript{195:7.13} Si el hombre sólo es una máquina, ¿qué técnica utiliza para llegar a \textit{creer} o a pretender \textit{saber} que sólo es una máquina? La experiencia de evaluarse conscientemente a sí mismo nunca es atributo de una simple máquina. Un mecanicista declarado y consciente de sí mismo es la mejor respuesta posible al mecanismo. Si el materialismo fuera un hecho, no podría existir ningún mecanicista consciente de sí mismo. También es cierto que primero hay que ser una persona moral antes de poder realizar actos inmorales.

\par 
%\textsuperscript{(2079.9)}
\textsuperscript{195:7.14} La pretensión misma del materialismo implica una conciencia supermaterial de la mente que se atreve a afirmar tales dogmas. Un mecanismo puede deteriorarse, pero nunca puede progresar. Las máquinas no piensan, ni crean, ni sueñan, ni aspiran a algo, ni idealizan, ni tienen hambre de verdad o sed de rectitud. No motivan su vida con la pasión de servir a otras máquinas y escoger como meta de su progreso eterno la sublime tarea de encontrar a Dios y de esforzarse en ser como él. Las máquinas nunca son intelectuales, emotivas, estéticas, éticas, morales ni espirituales.

\par 
%\textsuperscript{(2079.10)}
\textsuperscript{195:7.15} El arte prueba que el hombre no es mecánico, pero no prueba que sea espiritualmente inmortal. El arte es la morontia humana, el terreno intermedio entre el hombre material y el hombre espiritual. La poesía es un esfuerzo por huir de las realidades materiales hacia los valores espirituales.

\par 
%\textsuperscript{(2080.1)}
\textsuperscript{195:7.16} En una civilización elevada, el arte humaniza a la ciencia, y es espiritualizado a su vez por la verdadera religión ---la comprensión de los valores espirituales y eternos. El arte representa la evaluación humana y espacio-temporal de la realidad. La religión \textit{es} el abrazo divino de los valores cósmicos y conlleva un progreso eterno en la ascensión y la expansión espirituales. El arte temporal sólo es peligroso cuando se vuelve ciego a los modelos espirituales de los arquetipos divinos que la eternidad refleja como sombras temporales de la realidad. El arte verdadero es la manipulación eficaz de las cosas materiales de la vida; la religión es la transformación ennoblecedora de los hechos materiales de la vida, y nunca deja de evaluar el arte en el sentido espiritual.

\par 
%\textsuperscript{(2080.2)}
\textsuperscript{195:7.17} ¡Cuán insensato es suponer que un autómata pueda concebir una filosofía del automatismo, y cuán ridículo es creer que podría formarse un concepto así de otros compañeros autómatas!

\par 
%\textsuperscript{(2080.3)}
\textsuperscript{195:7.18} Cualquier interpretación científica del universo material carece de valor a menos que asegure un debido reconocimiento al \textit{científico}. Ninguna apreciación del arte es auténtica a menos que conceda un reconocimiento al \textit{artista}. Ninguna evaluación de la moral es válida a menos que incluya al \textit{moralista}. Ningún reconocimiento de la filosofía es edificante si ignora al \textit{filósofo}, y la religión no puede existir sin la experiencia real de la \textit{persona religiosa} que, en esta experiencia misma y a través de ella, intenta encontrar a Dios y conocerlo. Del mismo modo, el universo de universos carece de trascendencia separado del YO SOY, el Dios infinito que lo ha hecho y lo gobierna sin cesar.

\par 
%\textsuperscript{(2080.4)}
\textsuperscript{195:7.19} Los mecanicistas ---los humanistas--- tienden a ir a la deriva con las corrientes materiales. Los idealistas y los espiritualistas \textit{se atreven} a utilizar sus remos con inteligencia y vigor a fin de modificar el curso, en apariencia puramente material, de las corrientes de energía.

\par 
%\textsuperscript{(2080.5)}
\textsuperscript{195:7.20} La ciencia vive gracias a las matemáticas de la mente; la música expresa el ritmo de las emociones. La religión es el ritmo espiritual del alma, en armonía espacio-temporal con las medidas melódicas superiores y eternas de la Infinidad. La experiencia religiosa es algo verdaderamente supermatemático en la vida humana.

\par 
%\textsuperscript{(2080.6)}
\textsuperscript{195:7.21} En el lenguaje, el alfabeto representa el mecanismo del materialismo, mientras que las palabras que expresan el significado de mil pensamientos, grandes ideas y nobles ideales ---de amor y de odio, de cobardía y de valor--- representan las actuaciones de la mente dentro del alcance definido por la ley tanto material como espiritual, unas actuaciones dirigidas por la afirmación de la voluntad de la personalidad, y limitadas por la dotación inherente a la situación.

\par 
%\textsuperscript{(2080.7)}
\textsuperscript{195:7.22} El universo no se parece a las leyes, los mecanismos y las constantes que descubre el científico, y que llega a considerar como ciencia, sino que se parece más bien al \textit{científico} curioso que piensa, escoge, crea, combina y discrimina, que observa así los fenómenos del universo y clasifica los hechos matemáticos inherentes a las fases mecanicistas del aspecto material de la creación. El universo tampoco se parece al arte del artista, sino más bien al \textit{artista} que se esfuerza, sueña, aspira, progresa e intenta trascender el mundo de las cosas materiales, en un esfuerzo por alcanzar una meta espiritual.

\par 
%\textsuperscript{(2080.8)}
\textsuperscript{195:7.23} Es el científico, y no la ciencia, el que percibe la realidad de un universo de energía y materia en evolución y progreso. Es el artista, y no el arte, el que demuestra la existencia del mundo morontial transitorio interpuesto entre la existencia material y la libertad espiritual. Es la persona religiosa, y no la religión, la que prueba la existencia de las realidades del espíritu y de los valores divinos que se habrán de encontrar durante el progreso en la eternidad.

\section*{8. El totalitarismo laico}
\par 
%\textsuperscript{(2081.1)}
\textsuperscript{195:8.1} Pero incluso después de que el materialismo y el mecanicismo hayan sido más o menos derrotados, la influencia devastadora del laicismo del siglo veinte continuará marchitando la experiencia espiritual de millones de almas confiadas.

\par 
%\textsuperscript{(2081.2)}
\textsuperscript{195:8.2} El laicismo moderno ha sido fomentado por dos influencias mundiales. El padre del laicismo fue la actitud atea y de ideas limitadas de la llamada ciencia de los siglos diecinueve y veinte ---la ciencia atea. La madre del laicismo moderno fue la iglesia cristiana totalitaria de la Edad Media. El laicismo tuvo su comienzo como una protesta que se elevó contra la dominación casi completa de la civilización occidental por parte de la iglesia cristiana institucionalizada.

\par 
%\textsuperscript{(2081.3)}
\textsuperscript{195:8.3} En el momento de esta revelación, el clima intelectual y filosófico que prevalece tanto en la vida europea como en la americana es decididamente laico ---humanista. Durante trescientos años, el pensamiento occidental ha sido progresivamente laicizado. La religión se ha convertido cada vez más en una influencia nominal, se ha vuelto mayormente un ejercicio ritualista. La mayoría de los cristianos declarados de la civilización occidental son, sin saberlo, realmente laicos.

\par 
%\textsuperscript{(2081.4)}
\textsuperscript{195:8.4} Fue necesario un gran poder, una poderosa influencia, para liberar el pensamiento y la vida de los pueblos occidentales de la garra marchitante de una dominación eclesiástica totalitaria. El laicismo rompió las ataduras del control de la iglesia, y ahora amenaza a su vez con establecer un nuevo tipo de dominio ateo en el corazón y la mente del hombre moderno. El Estado político tiránico y dictatorial es el descendiente directo del materialismo científico y del laicismo filosófico. El laicismo apenas libera al hombre de la dominación de la iglesia institucionalizada, cuando lo vende a la esclavitud servil del Estado totalitario. El laicismo sólo libera al hombre de la esclavitud eclesiástica para traicionarlo entregándolo a la tiranía de la esclavitud política y económica.

\par 
%\textsuperscript{(2081.5)}
\textsuperscript{195:8.5} El materialismo niega a Dios, el laicismo se limita a ignorarlo; al menos ésta fue su actitud primitiva. Más recientemente, el laicismo ha tomado una actitud más militante, pretendiendo ocupar el lugar de la religión, cuya esclavitud totalitaria rechazó anteriormente. El laicismo del siglo veinte tiende a afirmar que el hombre no necesita a Dios. ¡Pero cuidado! Esta filosofía atea de la sociedad humana sólo conducirá a la inquietud, a la animosidad, a la infelicidad, a la guerra y a un desastre mundial.

\par 
%\textsuperscript{(2081.6)}
\textsuperscript{195:8.6} El laicismo nunca podrá traer la paz a la humanidad. Nada puede sustituir a Dios en la sociedad humana. ¡Pero poned mucha atención! No os apresuréis a abandonar las ventajas beneficiosas de la sublevación laica que os ha liberado del totalitarismo eclesiástico. La civilización occidental disfruta hoy de muchas libertades y satisfacciones debido a la sublevación laica. El gran error del laicismo fue el siguiente: Al sublevarse contra el control casi total de la vida por parte de la autoridad religiosa, y después de conseguir liberarse de esta tiranía eclesiástica, los laicos continuaron adelante iniciando una sublevación contra el mismo Dios, a veces tácitamente y a veces de manera manifiesta.

\par 
%\textsuperscript{(2081.7)}
\textsuperscript{195:8.7} A la sublevación laica le debéis la asombrosa creatividad de la industria americana y el progreso material sin precedentes de la civilización occidental. Como la sublevación laica ha ido demasiado lejos y ha perdido de vista a Dios y a la \textit{verdadera} religión, también le ha seguido una cosecha inesperada de guerras mundiales y de inestabilidad internacional.

\par 
%\textsuperscript{(2081.8)}
\textsuperscript{195:8.8} No es necesario sacrificar la fe en Dios para disfrutar de las bendiciones de la sublevación laica moderna: tolerancia, servicio social, gobierno democrático y libertades civiles. Los laicos no tenían necesidad de oponerse a la verdadera religión para promover la ciencia y hacer progresar la educación.

\par 
%\textsuperscript{(2082.1)}
\textsuperscript{195:8.9} Pero el laicismo no es el único autor de todas estas ventajas recientes en la expansión del modo de vivir. Detrás de los logros del siglo veinte están no solamente la ciencia y el laicismo, sino también los efectos espirituales no reconocidos ni admitidos de la vida y las enseñanzas de Jesús de Nazaret.

\par 
%\textsuperscript{(2082.2)}
\textsuperscript{195:8.10} Sin Dios, sin religión, el laicismo científico nunca podrá coordinar sus fuerzas, ni armonizar sus intereses, razas y nacionalismos divergentes y rivales. A pesar de sus logros materialistas incomparables, esta sociedad humana laicista se está desintegrando lentamente. La principal fuerza de cohesión que se resiste a esta desintegración de antagonismos es el nacionalismo. Y el nacionalismo es el obstáculo principal para la paz mundial.

\par 
%\textsuperscript{(2082.3)}
\textsuperscript{195:8.11} La debilidad inherente al laicismo consiste en que desecha la ética y la religión a favor de la política y del poder. Es simplemente imposible establecer la fraternidad de los hombres cuando se ignora o se niega la paternidad de Dios.

\par 
%\textsuperscript{(2082.4)}
\textsuperscript{195:8.12} El optimismo laico en materia social y política es una ilusión. Sin Dios, ni la independencia y la libertad, ni los bienes y la riqueza conducirán a la paz.

\par 
%\textsuperscript{(2082.5)}
\textsuperscript{195:8.13} La secularización completa de la ciencia, la educación, la industria y la sociedad sólo pueden conducir al desastre. Durante el primer tercio del siglo veinte, los urantianos han matado a más seres humanos que durante toda la dispensación cristiana hasta ese momento. Y éste sólo es el principio de la espantosa cosecha del materialismo y del laicismo; una destrucción aún más terrible está todavía por venir.

\section*{9. El problema del cristianismo}
\par 
%\textsuperscript{(2082.6)}
\textsuperscript{195:9.1} No paséis por alto el valor de vuestra herencia espiritual, el río de verdad que fluye a través de los siglos, incluso hasta la época estéril de una era materialista y laica. En todos vuestros esfuerzos meritorios por desembarazaros de los credos supersticiosos de las épocas pasadas, aseguraos de conservar firmemente la verdad eterna. ¡Pero tened paciencia! Cuando la sublevación actual contra la superstición haya terminado, las verdades del evangelio de Jesús sobrevivirán gloriosamente para iluminar un camino nuevo y mejor.

\par 
%\textsuperscript{(2082.7)}
\textsuperscript{195:9.2} Pero el cristianismo paganizado y socializado necesita un nuevo contacto con las enseñanzas no comprometidas de Jesús; languidece por falta de una visión nueva de la vida del Maestro en la Tierra. Una revelación nueva y más completa de la religión de Jesús está destinada a conquistar un imperio de laicismo materialista y a derrocar un influjo mundial de naturalismo mecanicista. Urantia se estremece actualmente al borde mismo de una de sus épocas más asombrosas y apasionantes de reajuste social, de reanimación moral y de iluminación espiritual.

\par 
%\textsuperscript{(2082.8)}
\textsuperscript{195:9.3} Las enseñanzas de Jesús, aunque enormemente modificadas, sobrevivieron a los cultos de misterio de su época natal, a la ignorancia y la superstición de la edad de las tinieblas, e incluso ahora están venciendo lentamente al materialismo, al mecanicismo y al laicismo del siglo veinte. Estas épocas de grandes pruebas y de derrotas amenazantes siempre son períodos de gran revelación.

\par 
%\textsuperscript{(2082.9)}
\textsuperscript{195:9.4} La religión necesita nuevos dirigentes, hombres y mujeres espirituales que se atrevan a depender únicamente de Jesús y de sus enseñanzas incomparables. Si el cristianismo insiste en olvidar su misión espiritual mientras continúa ocupándose de los problemas sociales y materiales, el renacimiento espiritual tendrá que esperar la llegada de esos nuevos instructores de la religión de Jesús que se consagrarán exclusivamente a la regeneración espiritual de los hombres. Entonces, esas almas nacidas del espíritu proporcionarán rápidamente la dirección y la inspiración necesarias para la reorganización social, moral, económica y política del mundo.

\par 
%\textsuperscript{(2083.1)}
\textsuperscript{195:9.5} La era moderna rehusará aceptar una religión que sea incompatible con los hechos y que no se armonice con sus conceptos más elevados de la verdad, la belleza y la bondad. Ha llegado la hora de volver a descubrir los verdaderos fundamentos originales del cristianismo de hoy deformado y comprometido ---la vida y las enseñanzas reales de Jesús.

\par 
%\textsuperscript{(2083.2)}
\textsuperscript{195:9.6} El hombre primitivo vivía una vida de esclavitud supersticiosa al miedo religioso. El hombre civilizado moderno teme la idea de caer bajo el dominio de fuertes convicciones religiosas. El hombre inteligente siempre ha tenido miedo de estar \textit{sujeto} a una religión. Cuando una religión fuerte y activa amenaza con dominarlo, intenta invariablemente racionalizarla, institucionalizarla y convertirla en una tradición, esperando de este modo poder controlarla. Mediante este procedimiento, incluso una religión revelada se convierte en una religión elaborada y dominada por el hombre. Los hombres y las mujeres modernos e inteligentes rehuyen la religión de Jesús por temor a lo que ésta \textit{les} hará ---y a lo que hará \textit{con} ellos. Y todos estos temores están bien fundados. En verdad, la religión de Jesús domina y transforma a sus creyentes, pidiendo a los hombres que dediquen su vida a buscar el conocimiento de la voluntad del Padre que está en los cielos, y exigiendo que las energías de la vida se consagren al servicio desinteresado de la fraternidad de los hombres.

\par 
%\textsuperscript{(2083.3)}
\textsuperscript{195:9.7} Los hombres y las mujeres egoístas simplemente no quieren pagar este precio, ni siquiera a cambio del mayor tesoro espiritual que se haya ofrecido nunca al hombre mortal. Cuando el hombre se haya sentido suficientemente desilusionado por las tristes decepciones que acompañan la búsqueda insensata y engañosa del egoísmo, y después de que haya descubierto la esterilidad de la religión formalizada, sólo entonces estará dispuesto a volverse de todo corazón hacia el evangelio del reino, la religión de Jesús de Nazaret.

\par 
%\textsuperscript{(2083.4)}
\textsuperscript{195:9.8} El mundo necesita más que nada una religión de primera mano. Incluso el cristianismo ---la mejor religión del siglo veinte--- no es solamente una religión \textit{acerca de} Jesús, sino que es una religión que los hombres experimentan ampliamente de segunda mano. Éstos cogen su religión íntegramente tal como se la transmiten sus educadores religiosos aceptados. ¡Qué despertar experimentaría el mundo si tan sólo pudiera ver a Jesús tal como vivió realmente en la Tierra, y conocer de primera mano sus enseñanzas dadoras de vida! Las palabras que describen las cosas bellas no pueden conmover tanto como la visión de esas cosas, y las palabras de un credo tampoco pueden inspirar el alma de los hombres como la experiencia de conocer la presencia de Dios. Pero la fe expectante mantendrá siempre abierta la puerta de la esperanza del alma del hombre, para que entren las realidades espirituales eternas de los valores divinos de los mundos del más allá.

\par 
%\textsuperscript{(2083.5)}
\textsuperscript{195:9.9} El cristianismo se ha atrevido a rebajar sus ideales ante el desafío de la avidez humana, la locura de la guerra y la codicia del poder; pero la religión de Jesús se mantiene como la citación espiritual inmaculada y trascendente, apelando a lo mejor que hay en el hombre para que se eleve por encima de todos estos legados de la evolución animal, y alcance por la gracia las alturas morales del verdadero destino humano.

\par 
%\textsuperscript{(2083.6)}
\textsuperscript{195:9.10} El cristianismo está amenazado de muerte lenta por el formalismo, el exceso de organización, el intelectualismo y otras tendencias no espirituales. La iglesia cristiana moderna no es esa fraternidad de creyentes dinámicos a la que Jesús encargó que efectuara la transformación espiritual contínua de las generaciones sucesivas de la humanidad.

\par 
%\textsuperscript{(2083.7)}
\textsuperscript{195:9.11} El llamado cristianismo se ha convertido en un movimiento social y cultural, así como en una creencia y una práctica religiosas. El arroyo del cristianismo moderno desagua más de un antiguo pantano pagano y más de una ciénaga bárbara; muchas antiguas cuencas culturales vierten sus aguas en esta corriente cultural de hoy, además de las altas mesetas galileas que se supone que son su fuente exclusiva.

\section*{10. El futuro}
\par 
%\textsuperscript{(2084.1)}
\textsuperscript{195:10.1} En verdad, el cristianismo ha hecho un gran servicio a este mundo, pero a quien más se necesita ahora es a Jesús. El mundo necesita ver a Jesús viviendo de nuevo en la Tierra en la experiencia de los mortales nacidos del espíritu que revelan el Maestro eficazmente a todos los hombres. Es inútil hablar de un renacimiento del cristianismo primitivo; tenéis que avanzar desde el lugar donde os encontráis. La cultura moderna debe bautizarse espiritualmente con una nueva revelación de la vida de Jesús, e iluminarse con una nueva comprensión de su evangelio de salvación eterna. Y cuando Jesús sea elevado así, atraerá a todos los hombres hacia él. Los discípulos de Jesús deberían de ser más que conquistadores, e incluso fuentes desbordantes de inspiración y de vida realzada para todos los hombres. La religión no es más que un humanismo elevado hasta que se hace divina mediante el descubrimiento de la realidad de la presencia de Dios en la experiencia personal.

\par 
%\textsuperscript{(2084.2)}
\textsuperscript{195:10.2} La belleza y la sublimidad, la humanidad y la divinidad, la sencillez y la singularidad de la vida de Jesús en la Tierra presentan un cuadro tan sorprendente y atractivo de la salvación del hombre y de la revelación de Dios, que los teólogos y los filósofos de todos los tiempos deberían reprimir eficazmente el atrevimiento de formular credos o de crear sistemas teológicos de esclavitud espiritual partiendo de esta donación trascendental de Dios en la forma del hombre. En Jesús, el universo produjo un hombre mortal en quien el espíritu de amor triunfó sobre los obstáculos materiales del tiempo y superó el hecho del origen físico.

\par 
%\textsuperscript{(2084.3)}
\textsuperscript{195:10.3} Tened siempre presente que Dios y el hombre se necesitan el uno al otro. Son mutuamente necesarios para alcanzar de manera plena y final la experiencia de la personalidad eterna en el destino divino de la finalidad del universo.

\par 
%\textsuperscript{(2084.4)}
\textsuperscript{195:10.4} <<El reino de Dios está dentro de vosotros>> fue probablemente la proclamación más grande que Jesús hiciera nunca, después de la declaración de que su Padre es un espíritu vivo y amoroso.

\par 
%\textsuperscript{(2084.5)}
\textsuperscript{195:10.5} Para ganar almas para el Maestro, no es la primera legua recorrida por coacción, deber o convencionalismo la que transformará al hombre y a su mundo, sino que es más bien la \textit{segunda} legua de servicio libre y de devoción amante de la libertad la que revela que el discípulo de Jesús ha alargado la mano para coger a su hermano con amor y llevarlo, bajo la guía espiritual, hacia la meta superior y divina de la existencia mortal. Ahora mismo, el cristianismo recorre con gusto la \textit{primera} legua, pero la humanidad languidece y tropieza en las tinieblas morales porque hay muy pocos discípulos auténticos que recorran la segunda legua ---muy pocos seguidores declarados de Jesús que vivan y amen realmente como él enseñó a sus discípulos a vivir, amar y servir.

\par 
%\textsuperscript{(2084.6)}
\textsuperscript{195:10.6} La llamada a la aventura de construir una sociedad humana nueva y transformada mediante el renacimiento espiritual de la fraternidad del reino de Jesús debería emocionar a todos los que creen en él como los hombres no se han conmovido desde la época en que caminaban por la Tierra como compañeros suyos en la carne.

\par 
%\textsuperscript{(2084.7)}
\textsuperscript{195:10.7} Ningún sistema social o régimen político que niegue la realidad de Dios puede contribuir de manera constructiva y duradera al progreso de la civilización humana. Pero el cristianismo, tal como hoy está subdividido y secularizado, representa el mayor de todos los obstáculos para su propio progreso ulterior; esto es especialmente cierto en lo que concierne a oriente.

\par 
%\textsuperscript{(2084.8)}
\textsuperscript{195:10.8} El poder eclesiástico es ahora y siempre incompatible con la fe viviente, el espíritu creciente y la experiencia de primera mano de los compañeros, por la fe, de Jesús en la fraternidad de los hombres, en la asociación espiritual del reino de los cielos. El deseo loable de preservar las tradiciones de los logros pasados conduce a menudo a defender unos sistemas de adoración obsoletos. El deseo bien intencionado de fomentar antiguos sistemas de pensamiento impide eficazmente patrocinar unos medios y unos métodos nuevos y adecuados destinados a satisfacer los anhelos espirituales de la mente en expansión y en progreso del hombre moderno. Asímismo, las iglesias cristianas del siglo veinte se alzan como enormes obstáculos, aunque enteramente inconscientes, para el progreso inmediato del verdadero evangelio ---las enseñanzas de Jesús de Nazaret.

\par 
%\textsuperscript{(2085.1)}
\textsuperscript{195:10.9} Muchas personas serias que ofrecerían gustosamente su lealtad al Cristo del evangelio, encuentran muy difícil apoyar con entusiasmo a una iglesia que da tan pocas muestras del espíritu de su vida y de sus enseñanzas, y a estas personas se les ha enseñado erróneamente que él la fundó. Jesús no fundó la llamada iglesia cristiana, pero de todas las maneras compatibles con su naturaleza, la ha \textit{fomentado} como la mejor representante existente de la obra de su vida en la Tierra.

\par 
%\textsuperscript{(2085.2)}
\textsuperscript{195:10.10} Si la iglesia cristiana se atreviera tan sólo a abrazar el programa del Maestro, miles de jóvenes aparentemente indiferentes se precipitarían para alistarse en esta empresa espiritual, y no dudarían en llevar a cabo hasta el fin esta gran aventura.

\par 
%\textsuperscript{(2085.3)}
\textsuperscript{195:10.11} El cristianismo se enfrenta seriamente con la sentencia incluida en uno de sus propios lemas: <<Una casa dividida contra sí misma no puede subsistir>>. El mundo no cristiano difícilmente capitulará ante una cristiandad dividida en sectas. El Jesús vivo es la única esperanza de una posible unificación del cristianismo. La verdadera iglesia ---la fraternidad de Jesús--- es invisible, espiritual y está caracterizada por la \textit{unidad}, pero no necesariamente por la \textit{uniformidad}. La uniformidad es la marca distintiva del mundo físico de naturaleza mecanicista. La unidad espiritual es el fruto de la unión por la fe con el Jesús vivo. La iglesia visible debería negarse a continuar obstaculizando el progreso de la fraternidad invisible y espiritual del reino de Dios. Esta fraternidad está destinada a convertirse en un \textit{organismo viviente}, en contraste con una organización social institucionalizada. Puede utilizar muy bien estas organizaciones sociales, pero no debe ser sustituida por ellas.

\par 
%\textsuperscript{(2085.4)}
\textsuperscript{195:10.12} Pero incluso el cristianismo del siglo veinte no debe ser despreciado. Es el producto del genio moral combinado de los hombres que conocían a Dios pertenecientes a muchas razas y durante muchas épocas; ha sido realmente uno de los más grandes poderes benéficos de la Tierra, y por consiguiente nadie debería considerarlo a la ligera, a pesar de sus defectos inherentes y adquiridos. El cristianismo continúa ingeniándoselas para incitar, con poderosas emociones morales, la mente de los hombres reflexivos.

\par 
%\textsuperscript{(2085.5)}
\textsuperscript{195:10.13} Pero la implicación de la iglesia en el comercio y la política no tiene excusa; estas alianzas profanas son una flagrante traición al Maestro. Y los auténticos amantes de la verdad tardarán mucho tiempo en olvidar que esta poderosa iglesia institucionalizada se ha atrevido con frecuencia a sofocar una fe recién nacida, y a perseguir a los portadores de la verdad que aparecían por casualidad con vestiduras no ortodoxas.

\par 
%\textsuperscript{(2085.6)}
\textsuperscript{195:10.14} Es demasiado cierto que esta iglesia no habría sobrevivido si no hubiera habido hombres en el mundo que prefirieran esta forma de culto. Muchas almas espiritualmente indolentes anhelan una religión antigua y autoritaria de rituales y de tradiciones consagradas. La evolución humana y el progreso espiritual apenas son suficientes para hacer que todos los hombres prescindan de una autoridad religiosa. Y la fraternidad invisible del reino puede muy bien incluir a estos grupos familiares de diversas clases sociales y temperamentales, con tal que estén dispuestos a convertirse en unos hijos de Dios realmente conducidos por el espíritu. Pero en esta fraternidad de Jesús no hay sitio para las rivalidades sectarias, el resentimiento entre los grupos, ni para las afirmaciones de superioridad moral e infalibilidad espiritual.

\par 
%\textsuperscript{(2086.1)}
\textsuperscript{195:10.15} Estas diversas agrupaciones de cristianos pueden servir para albergar a los numerosos tipos diferentes de supuestos creyentes entre los diversos pueblos de la civilización occidental, pero esta división de la cristiandad muestra una grave debilidad cuando intenta llevar el evangelio de Jesús a los pueblos orientales. Esas razas no comprenden todavía que existe una \textit{religión de Jesús} separada, y un poco apartada, del cristianismo, el cual se ha vuelto cada vez más una \textit{religión acerca de Jesús}.

\par 
%\textsuperscript{(2086.2)}
\textsuperscript{195:10.16} La gran esperanza de Urantia reside en la posibilidad de una nueva revelación de Jesús, con una presentación nueva y ampliada de su mensaje salvador, que uniría espiritualmente en un servicio amoroso a las numerosas familias de sus seguidores declarados de hoy en día.

\par 
%\textsuperscript{(2086.3)}
\textsuperscript{195:10.17} Incluso la educación laica podría ayudar a este gran renacimiento espiritual, si prestara más atención a la tarea de enseñar a los jóvenes cómo acometer la planificación de la vida y el desarrollo del carácter. La meta de toda educación debería consistir en fomentar y promover el objetivo supremo de la vida, el desarrollo de una personalidad majestuosa y bien equilibrada. Existe una gran necesidad de enseñar la disciplina moral en lugar de tantas satisfacciones egoístas. Sobre esta base, la religión puede aportar su estímulo espiritual para ampliar y enriquecer la vida humana, e incluso para asegurar y realzar la vida eterna.

\par 
%\textsuperscript{(2086.4)}
\textsuperscript{195:10.18} El cristianismo es una religión improvisada, y por eso debe funcionar a baja velocidad. Las actuaciones espirituales a gran velocidad deben esperar la nueva revelación y la aceptación más generalizada de la verdadera religión de Jesús. Pero el cristianismo es una religión poderosa, puesto que los discípulos corrientes de un carpintero crucificado pusieron en marcha las enseñanzas que conquistaron el mundo romano en trescientos años, y luego continuaron hasta vencer a los bárbaros que derrocaron a Roma. Este mismo cristianismo conquistó ---absorbió y exaltó--- toda la corriente de la teología hebrea y de la filosofía griega. Luego, cuando esta religión cristiana cayó en estado de coma durante más de mil años a causa de una dosis excesiva de misterios y de paganismo, se resucitó a sí misma y reconquistó virtualmente todo el mundo occidental. El cristianismo contiene suficientes enseñanzas de Jesús como para volverse inmortal.

\par 
%\textsuperscript{(2086.5)}
\textsuperscript{195:10.19} Si el cristianismo tan sólo pudiera captar una mayor cantidad de enseñanzas de Jesús, podría hacer mucho más para ayudar al hombre moderno a resolver sus problemas nuevos y cada vez más complejos.

\par 
%\textsuperscript{(2086.6)}
\textsuperscript{195:10.20} El cristianismo sufre una gran desventaja porque ha sido identificado, en la mente de todo el mundo, como una parte del sistema social, la vida industrial y los criterios morales de la civilización occidental; de este modo, el cristianismo ha parecido patrocinar, sin ser consciente de ello, una sociedad que se tambalea bajo la culpabilidad de tolerar una ciencia sin idealismo, una política sin principios, una riqueza sin trabajo, un placer sin restricción, un conocimiento sin carácter, un poder sin conciencia y una industria sin moralidad.

\par 
%\textsuperscript{(2086.7)}
\textsuperscript{195:10.21} La esperanza del cristianismo moderno consiste en dejar de patrocinar los sistemas sociales y las políticas industriales de la civilización occidental, e inclinarse humildemente ante la cruz que ensalza tan valientemente, para aprender allí otra vez de Jesús de Nazaret las verdades más grandes que el hombre mortal pueda escuchar jamás ---el evangelio viviente de la paternidad de Dios y de la fraternidad de los hombres.


\chapter{Documento 196. La fe de Jesús}
\par 
%\textsuperscript{(2087.1)}
\textsuperscript{196:0.1} JESÚS gozaba de una fe sublime y sin reservas en Dios. Experimentó los altibajos normales y corrientes de la existencia mortal, pero nunca puso religiosamente en duda la certidumbre de la vigilancia y la guía de Dios. Su fe era el fruto de la perspicacia nacida de la actividad de la presencia divina, su Ajustador interior. Su fe no era ni tradicional ni simplemente intelectual; era enteramente personal y puramente espiritual.

\par 
%\textsuperscript{(2087.2)}
\textsuperscript{196:0.2} El Jesús humano veía a Dios como santo, justo y grande, así como verdadero, bello y bueno. Todos estos atributos de la divinidad los enfocó en su mente como <<la voluntad del Padre que está en los cielos>>. El Dios de Jesús era al mismo tiempo <<el Santo de Israel>> y <<el Padre vivo y amoroso que está en los cielos>>. El concepto de Dios como Padre no era original de Jesús, pero exaltó y elevó la idea hasta el nivel de una experiencia sublime mediante la realización de una nueva revelación de Dios y la proclamación de que toda criatura mortal es hija de este Padre del amor, un hijo de Dios.

\par 
%\textsuperscript{(2087.3)}
\textsuperscript{196:0.3} Jesús no se aferró a la fe en Dios como un alma que lucha en una guerra contra el universo y en una pelea a muerte con un mundo hostil y pecaminoso; no recurrió a la fe simplemente para consolarse en medio de las dificultades o para animarse cuando lo amenazaba la desesperación; la fe no era para él una simple compensación ilusoria ante las realidades desagradables y las tristezas de la vida. En presencia misma de todas las dificultades naturales y de todas las contradicciones temporales de la existencia mortal, experimentó la tranquilidad de una confianza suprema e incontestable en Dios y sintió la formidable emoción de vivir, por la fe, en la presencia misma del Padre celestial. Esta fe triunfante era la experiencia viviente de un logro espiritual real. La gran contribución de Jesús a los valores de la experiencia humana no fue la de revelar tantas ideas nuevas sobre el Padre que está en los cielos, sino más bien la de demostrar de manera tan magnífica y humana un tipo nuevo y superior de \textit{fe viviente en Dios}. En ningún mundo de este universo, ni en la vida de ningún otro mortal, Dios no se ha vuelto nunca una \textit{realidad tan viviente} como en la experiencia humana de Jesús de Nazaret.

\par 
%\textsuperscript{(2087.4)}
\textsuperscript{196:0.4} Este mundo y todos los demás mundos de la creación local descubren, en la vida del Maestro en Urantia, un tipo de religión nuevo y superior, una religión basada en las relaciones espirituales personales con el Padre Universal, y totalmente validada por la autoridad suprema de una experiencia personal auténtica. Esta fe viviente de Jesús era más que una reflexión intelectual, y no era una meditación mística.

\par 
%\textsuperscript{(2087.5)}
\textsuperscript{196:0.5} La teología puede fijar, formular, definir y dogmatizar la fe, pero en la vida humana de Jesús, la fe era personal, viviente, original, espontánea y puramente espiritual. Esta fe no era una veneración por la tradición ni una simple creencia intelectual que él mantenía como un credo sagrado, sino más bien una experiencia sublime y una convicción profunda que lo \textit{mantenían en la seguridad}. Su fe era tan real e inclusiva que erradicó absolutamente todas las dudas espirituales y destruyó eficazmente todo deseo contradictorio. Nada era capaz de arrancar a Jesús del anclaje espiritual de esta fe ferviente, sublime e intrépida. Incluso en presencia de una derrota aparente o en medio de la decepción y de una desesperación amenazante, se mantenía sereno en la presencia divina, libre de temores y plenamente consciente de ser espiritualmente invencible. Jesús disfrutaba de la seguridad vigorizante de poseer una fe a toda prueba, y en cada una de las situaciones difíciles de la vida, mostró infaliblemente una lealtad incondicional a la voluntad del Padre. Esta fe magnífica no se dejó intimidar ni siquiera por la amenaza cruel y abrumadora de una muerte ignominiosa.

\par 
%\textsuperscript{(2088.1)}
\textsuperscript{196:0.6} En un genio religioso, una poderosa fe espiritual conduce muchas veces directamente a un fanatismo desastroso, a la exageración del ego religioso, pero esto no le sucedió a Jesús. Su vida práctica no se vio afectada desfavorablemente por su fe extraordinaria y sus logros espirituales, porque esta exaltación espiritual era una expresión enteramente inconsciente y espontánea que hacía su alma de su experiencia personal con Dios.

\par 
%\textsuperscript{(2088.2)}
\textsuperscript{196:0.7} La fe espiritual de Jesús, arrolladora e indomable, nunca se volvió fanática porque nunca intentó dejarse llevar por sus juicios intelectuales bien equilibrados sobre los valores proporcionales de las situaciones sociales, económicas y morales, prácticas y corrientes, de la vida. El Hijo del Hombre era una personalidad humana espléndidamente unificada; era un ser divino perfectamente dotado; también estaba magníficamente coordinado como ser humano y divino combinados, ejerciendo su actividad en la Tierra como una sola personalidad. El Maestro siempre coordinaba la fe del alma con las sabias evaluaciones de una experiencia madurada. La fe personal, la esperanza espiritual y la devoción moral siempre estaban correlacionadas en una unidad religiosa incomparable de asociación armoniosa con la comprensión penetrante de la realidad y el carácter sagrado de todas las lealtades humanas ---honor personal, amor familiar, obligaciones religiosas, deberes sociales y necesidades económicas.

\par 
%\textsuperscript{(2088.3)}
\textsuperscript{196:0.8} La fe de Jesús visualizaba que todos los valores espirituales se encontraban en el reino de Dios; por eso decía: <<Buscad primero el reino de los cielos>>. Jesús veía en la hermandad avanzada e ideal del reino la realización y el cumplimiento de la <<voluntad de Dios>>. La esencia misma de la oración que enseñó a sus discípulos fue: <<Que venga tu reino; que se haga tu voluntad>>. Una vez que concibió así que el reino incluía la voluntad de Dios, se consagró a la causa de hacerlo realidad con un asombroso olvido de sí mismo y un entusiasmo ilimitado. Pero durante toda su intensa misión y a lo largo de su vida extraordinaria, nunca se manifestó el furor del fanático ni la frivolidad superficial del egotista religioso.

\par 
%\textsuperscript{(2088.4)}
\textsuperscript{196:0.9} Toda la vida del Maestro estuvo constantemente condicionada por esta fe viviente, esta experiencia religiosa sublime. Esta actitud espiritual dominaba totalmente sus pensamientos y sentimientos, su creencia y su oración, su enseñanza y su predicación. Esta fe personal de un hijo en la certidumbre y la seguridad de la guía y la protección del Padre celestial confirió a su vida excepcional una profunda dotación de realidad espiritual. Sin embargo, a pesar de esta conciencia profundísima de su estrecha relación con la divinidad, este Galileo, este Galileo de Dios, cuando le llamaron Maestro Bueno, replicó instantáneamente: <<¿Por qué me llamas bueno?>> Cuando nos encontramos ante un olvido de sí mismo tan espléndido, empezamos a comprender cómo le resultó posible al Padre Universal manifestarse tan plenamente a Jesús y revelarse a través de él a los mortales de los mundos.

\par 
%\textsuperscript{(2088.5)}
\textsuperscript{196:0.10} Jesús le entregó a Dios, como hombre del reino, la más grande de todas las ofrendas: la consagración y la dedicación de su propia voluntad al servicio majestuoso de hacer la voluntad divina. Jesús siempre interpretó la religión, de manera sistemática, totalmente en función de la voluntad del Padre. Cuando estudiéis la carrera del Maestro, en lo referente a la oración o a cualquier otra característica de la vida religiosa, no busquéis tanto lo que enseñó como lo que hizo. Jesús nunca oraba porque fuera un deber religioso. Para él, la oración era una expresión sincera de la actitud espiritual, una declaración de la lealtad del alma, una recitación de devoción personal, una expresión de acción de gracias, una manera de evitar la tensión emocional, una prevención de los conflictos, una exaltación del intelecto, un ennoblecimiento de los deseos, una confirmación de las decisiones morales, un enriquecimiento del pensamiento, una estimulación de las tendencias más elevadas, una consagración del impulso, una clarificación de un punto de vista, una declaración de fe, una rendición trascendental de la voluntad, una sublime afirmación de confianza, una revelación de valentía, la proclamación de un descubrimiento, una confesión de devoción suprema, la validación de una consagración, una técnica para ajustar las dificultades y la poderosa movilización de los poderes combinados del alma para resistir todas las tendencias humanas al egoísmo, al mal y al pecado. Vivió precisamente este tipo de vida consagrada piadosamente a hacer la voluntad de su Padre, y terminó su vida triunfalmente con una oración de este tipo. El secreto de su incomparable vida religiosa fue esta conciencia de la presencia de Dios; y la consiguió mediante oraciones inteligentes y una adoración sincera ---una comunión ininterrumpida con Dios--- y no por medio de directrices, voces, visiones, apariciones o prácticas religiosas extraordinarias.

\par 
%\textsuperscript{(2089.1)}
\textsuperscript{196:0.11} En la vida terrestre de Jesús, la religión fue una experiencia viviente, un movimiento directo y personal desde la veneración espiritual hasta la rectitud práctica. La fe de Jesús produjo los frutos trascendentes del espíritu divino. Su fe no era inmadura y crédula como la de un niño, pero en muchos aspectos se parecía a la confianza sin sospechas de la mente de un niño; Jesús confiaba en Dios como un niño confía en su padre. Tenía una profunda confianza en el universo ---la misma confianza que tiene un niño en el ambiente de sus padres. La fe incondicional de Jesús en la bondad fundamental del universo se parecía mucho a la confianza del niño en la seguridad de su entorno terrestre. Dependía del Padre celestial como un niño se apoya en su padre terrenal, y su fe ferviente nunca dudó ni un momento de la certeza de los grandes cuidados del Padre celestial. No le perturbaron seriamente los temores, las dudas ni el escepticismo. La incredulidad no inhibió la expresión libre y original de su vida. Combinó el coraje inquebrantable e inteligente de un adulto con el optimismo sincero y confiado de un niño creyente. Su fe había crecido hasta tales niveles de confianza que estaba desprovista de temor.

\par 
%\textsuperscript{(2089.2)}
\textsuperscript{196:0.12} La fe de Jesús alcanzó la pureza de la confianza de un niño. Su fe era tan absoluta y estaba tan desprovista de dudas que era sensible al encanto del contacto con los semejantes y a las maravillas del universo. Su sentimiento de dependencia de lo divino era tan completo y tan confiado que le producía la alegría y la certeza de una seguridad personal absoluta. No había ningún fingimiento vacilante en su experiencia religiosa. En este intelecto gigantesco de adulto, la fe del niño reinaba de manera suprema en todos los asuntos relacionados con la conciencia religiosa. No es extraño que dijera una vez: <<A menos que os volváis como un niño pequeño, no entraréis en el reino>>. Aunque la fe de Jesús era \textit{ingenua}, no era en ningún sentido \textit{infantil}.

\par 
%\textsuperscript{(2089.3)}
\textsuperscript{196:0.13} Jesús no le pide a sus discípulos que crean en él, sino más bien que crean \textit{con} él, que crean en la realidad del amor de Dios y que acepten con toda confianza la seguridad de su filiación con el Padre celestial. El Maestro desea que todos sus seguidores compartan plenamente su fe trascendente. Jesús desafió a sus seguidores, de la manera más enternecedora, no sólo a creer \textit{lo que} él creía, sino también a creer \textit{como} él creía. Éste es el significado completo de su única exigencia suprema: <<Sígueme>>.

\par 
%\textsuperscript{(2090.1)}
\textsuperscript{196:0.14} La vida terrenal de Jesús estuvo consagrada a una sola gran finalidad ---hacer la voluntad del Padre, vivir la vida humana religiosamente y por la fe. La fe de Jesús era confiada como la de un niño, pero sin la menor presunción. Tomó decisiones firmes y valientes, se enfrentó con intrepidez a múltiples decepciones, superó resueltamente dificultades extraordinarias, e hizo frente sin vacilar a las duras exigencias del deber. Se necesitaba una fuerte voluntad y una confianza indefectible para creer lo que Jesús creía, y \textit{como} él lo creía.

\section*{1. Jesús ---el hombre}
\par 
%\textsuperscript{(2090.2)}
\textsuperscript{196:1.1} La devoción de Jesús a la voluntad del Padre y al servicio del hombre era mucho más que una decisión como mortal y que una determinación humana; era una consagración total de sí mismo a esta donación ilimitada de amor. Por muy grande que sea el hecho de la soberanía de Miguel, no debéis apartar de los hombres al Jesús humano. El Maestro subió a los cielos no sólo como hombre, sino también como Dios; él pertenece a los hombres, y los hombres le pertenecen. ¡Es muy lamentable que la religión misma sea tan mal interpretada, que aparte al Jesús humano de los mortales que luchan! Que las discusiones sobre la humanidad o la divinidad de Cristo no oscurezcan la verdad salvadora de que Jesús de Nazaret fue un hombre religioso que consiguió, por la fe, conocer y hacer la voluntad de Dios; fue realmente el hombre más religioso que haya vivido jamás en Urantia.

\par 
%\textsuperscript{(2090.3)}
\textsuperscript{196:1.2} Los tiempos están maduros para presenciar la resurrección simbólica del Jesús humano, saliendo de la tumba de las tradiciones teológicas y de los dogmas religiosos de diecinueve siglos. Jesús de Nazaret ya no debe ser sacrificado, ni siquiera por el espléndido concepto del Cristo glorificado. ¡Qué servicio trascendente prestaría la presente revelación si, a través de ella, el Hijo del Hombre fuera rescatado de la tumba de la teología tradicional, y fuera presentado como el Jesús vivo a la iglesia que lleva su nombre y a todas las demás religiones! La hermandad cristiana de creyentes no dudará seguramente en reajustar su fe y sus costumbres de vida para poder <<seguir>> al Maestro en la manifestación de su vida real de devoción religiosa a la tarea de hacer la voluntad de su Padre, y de consagración al servicio desinteresado de los hombres. ¿Temen los cristianos declarados que se ponga al descubierto a una hermandad autosuficiente y no consagrada, que tiene respetabilidad social y una inadaptación económica egoísta? ¿Teme el cristianismo institucional que la autoridad eclesiástica tradicional esté posiblemente en peligro, o incluso sea derrocada, si el Jesús de Galilea es reinstalado en la mente y el alma de los hombres mortales como el ideal de la vida religiosa personal? En verdad, los reajustes sociales, las transformaciones económicas, los rejuvenecimientos morales y las revisiones religiosas de la civilización cristiana serían drásticas y revolucionarias si la religión viviente de Jesús sustituyera repentinamente a la religión teológica acerca de Jesús.

\par 
%\textsuperscript{(2090.4)}
\textsuperscript{196:1.3} <<Seguir a Jesús>> significa compartir personalmente su fe religiosa y entrar en el espíritu de la vida del Maestro, consagrada al servicio desinteresado de los hombres. Una de las cosas más importantes de la vida humana consiste en averiguar lo que Jesús creía, en descubrir sus ideales, y en esforzarse por alcanzar el elevado objetivo de su vida. De todos los conocimientos humanos, el que posee mayor valor es el de conocer la vida religiosa de Jesús y la manera en que la vivió.

\par 
%\textsuperscript{(2090.5)}
\textsuperscript{196:1.4} La gente corriente escuchaba a Jesús con placer, y responderán de nuevo a la presentación de su vida humana sincera de motivación religiosa consagrada, si estas verdades se proclaman de nuevo en el mundo. La gente lo escuchaba con placer porque era uno de ellos, un laico sin pretensiones; el instructor religioso más grande del mundo fue en verdad un laico.

\par 
%\textsuperscript{(2091.1)}
\textsuperscript{196:1.5} Los creyentes en el reino no deberían tener el objetivo de imitar literalmente la vida exterior de Jesús en la carne, sino más bien de compartir su fe; confiar en Dios como él confiaba en Dios, y creer en los hombres como él creía en ellos. Jesús nunca discutió sobre la paternidad de Dios o la fraternidad de los hombres; él era una ilustración viviente de lo primero y una profunda demostración de lo segundo.

\par 
%\textsuperscript{(2091.2)}
\textsuperscript{196:1.6} Al igual que los hombres deben progresar desde la conciencia de lo humano hasta la comprensión de lo divino, Jesús se elevó desde la naturaleza del hombre hasta la conciencia de la naturaleza de Dios. Y el Maestro efectuó esta gran ascensión desde lo humano hasta lo divino mediante el logro conjunto de la fe de su intelecto mortal y los actos de su Ajustador interior. El hecho de llevar a cabo la conquista de la totalidad de su divinidad (siendo en todo momento plenamente consciente de la realidad de su humanidad) pasó por siete fases de conciencia, por la fe, de su divinización progresiva. Los siguientes acontecimientos extraordinarios marcaron estas fases de desarrollo progresivo de sí mismo en la experiencia donadora del Maestro:

\par 
%\textsuperscript{(2091.3)}
\textsuperscript{196:1.7} 1. La llegada del Ajustador del Pensamiento.

\par 
%\textsuperscript{(2091.4)}
\textsuperscript{196:1.8} 2. El mensajero de Emmanuel que se le apareció en Jerusalén cuando tenía unos doce años.

\par 
%\textsuperscript{(2091.5)}
\textsuperscript{196:1.9} 3. Las manifestaciones que acompañaron a su bautismo.

\par 
%\textsuperscript{(2091.6)}
\textsuperscript{196:1.10} 4. Las experiencias en el Monte de la Transfiguración.

\par 
%\textsuperscript{(2091.7)}
\textsuperscript{196:1.11} 5. La resurrección morontial.

\par 
%\textsuperscript{(2091.8)}
\textsuperscript{196:1.12} 6. La ascensión en espíritu.

\par 
%\textsuperscript{(2091.9)}
\textsuperscript{196:1.13} 7. El abrazo final del Padre Paradisiaco, que le confirió la soberanía ilimitada sobre su universo.

\section*{2. La religión de Jesús}
\par 
%\textsuperscript{(2091.10)}
\textsuperscript{196:2.1} Algún día, una reforma en la iglesia cristiana podría causar un impacto lo suficientemente profundo como para regresar a las enseñanzas religiosas puras de Jesús, el autor y consumador de nuestra fe. Podéis \textit{predicar} una religión \textit{acerca de} Jesús, pero la religión \textit{de} Jesús, forzosamente, tenéis que \textit{vivirla}. En el entusiasmo de Pentecostés, Pedro inauguró involuntariamente una nueva religión, la religión del Cristo resucitado y glorificado. El apóstol Pablo transformó más tarde este nuevo evangelio en el cristianismo, una religión que incluye sus propias opiniones teológicas y describe su propia \textit{experienciapersonal} con el Jesús del camino de Damasco. El evangelio del reino está fundado en la experiencia religiosa personal de Jesús de Galilea; el cristianismo está fundado casi exclusivamente en la experiencia religiosa personal del apóstol Pablo. Casi todo el Nuevo Testamento está dedicado, no a describir la vida religiosa significativa e inspiradora de Jesús, sino a examinar la experiencia religiosa de Pablo y a describir sus convicciones religiosas personales. Las únicas excepciones notables a esta afirmación son el Libro de los Hebreos y la Epístola de Santiago, además de algunos fragmentos de Mateo, Marcos y Lucas. El mismo Pedro sólo volvió una vez, en sus escritos, a la vida religiosa personal de su Maestro. El Nuevo Testamento es un magnífico documento cristiano, pero sólo refleja pobremente la religión de Jesús.

\par 
%\textsuperscript{(2091.11)}
\textsuperscript{196:2.2} La vida de Jesús en la carne describe un crecimiento religioso trascendente que empezó por las antiguas ideas del temor primitivo y de la veneración humana, y pasó por los años de comunión espiritual personal, hasta que llegó finalmente al estado avanzado y elevado de la conciencia de su unidad con el Padre. Y así, en una sola corta vida, Jesús atravesó esa experiencia de evolución espiritual religiosa que los hombres empiezan en la Tierra y que sólo terminan generalmente al final de su larga estancia en las escuelas de educación espiritual de los niveles sucesivos de la carrera preparadisiaca. Jesús progresó desde una conciencia puramente humana en la que tenía la certidumbre, por la fe, de una experiencia religiosa personal, hasta las sublimes alturas espirituales de la comprensión definitiva de su naturaleza divina, y hasta la conciencia de su estrecha asociación con el Padre Universal en la administración de un universo. Progresó desde el humilde estado de dependencia mortal que le impulsó a decir espontáneamente a aquel que le había llamado Maestro Bueno: <<¿Por qué me llamas bueno? Nadie es bueno salvo Dios>>, hasta esa conciencia sublime de una divinidad consumada que le condujo a exclamar: <<¿Quién de vosotros me declara culpable de pecado?>> Esta ascensión progresiva de lo humano a lo divino fue un logro exclusivamente mortal. Cuando hubo alcanzado así la divinidad, continuó siendo el mismo Jesús humano, el Hijo del Hombre así como el Hijo de Dios.

\par 
%\textsuperscript{(2092.1)}
\textsuperscript{196:2.3} Marcos, Mateo y Lucas retienen algunos aspectos del Jesús humano empeñado en el magnífico esfuerzo por averiguar la voluntad divina y por hacer dicha voluntad. Juan presenta la imagen de un Jesús triunfante que caminaba por la Tierra plenamente consciente de su divinidad. El gran error que han cometido aquellos que han estudiado la vida del Maestro es que algunos lo han concebido como enteramente humano, mientras que otros lo han considerado exclusivamente divino. A lo largo de toda su experiencia, el Maestro fue realmente ambas cosas, humano y divino, como lo sigue siendo ahora.

\par 
%\textsuperscript{(2092.2)}
\textsuperscript{196:2.4} Pero el error más grande se cometió cuando, aunque se reconocía que el Jesús humano \textit{tenía} una religión, el Jesús divino (Cristo) se convirtió casi de la noche a la mañana en una religión. El cristianismo de Pablo aseguró la adoración del Cristo divino, pero casi perdió de vista por completo al Jesús humano de Galilea, luchador y valiente, que gracias a la intrepidez de su fe religiosa personal y al heroísmo de su Ajustador interior, ascendió desde los humildes niveles de la humanidad hasta volverse uno con la divinidad, convirtiéndose así en el nuevo camino viviente por el que todos los mortales pueden elevarse de esta manera desde la humanidad hasta la divinidad. En todos los grados de espiritualidad y en todos los mundos, los mortales pueden encontrar en la vida personal de Jesús aquello que les fortalecerá e inspirará a medida que progresan desde los niveles espirituales más bajos hasta los valores divinos más elevados, desde el principio hasta el fin de toda la experiencia religiosa personal.

\par 
%\textsuperscript{(2092.3)}
\textsuperscript{196:2.5} En la época en que se escribió el Nuevo Testamento, los autores no sólo creían profundamente en la divinidad del Cristo resucitado, sino que también creían de manera ferviente y sincera en su inmediato regreso a la Tierra para consumar el reino celestial. Esta sólida fe en el regreso inmediato del Señor tuvo mucha relación con la tendencia a omitir en los escritos aquellas referencias que describían las experiencias y los atributos puramente humanos del Maestro. Todo el movimiento cristiano tendió a alejarse de la imagen humana de Jesús de Nazaret hacia la exaltación del Cristo resucitado, el Señor Jesucristo glorificado que pronto iba a volver.

\par 
%\textsuperscript{(2092.4)}
\textsuperscript{196:2.6} Jesús fundó la religión de la experiencia personal haciendo la voluntad de Dios y sirviendo a la fraternidad humana; Pablo fundó una religión en la que el Jesús glorificado se volvió el objeto de adoración, y la fraternidad estaba compuesta por los compañeros creyentes en el Cristo divino. En la donación de Jesús, estos dos conceptos existían en potencia en su vida humano-divina, y es en verdad una lástima que sus seguidores no lograran crear una religión unificada que hubiera reconocido adecuadamente tanto la naturaleza humana como la naturaleza divina del Maestro, tal como estaban inseparablemente unidas en su vida terrenal y tan gloriosamente expuestas en el evangelio original del reino.

\par 
%\textsuperscript{(2093.1)}
\textsuperscript{196:2.7} Algunas declaraciones enérgicas de Jesús no os impresionarían ni os perturbarían si tan sólo quisierais recordar que fue el hombre religioso más entusiasta y apasionado del mundo. Fue un mortal totalmente consagrado, dedicado sin reserva a hacer la voluntad de su Padre. Muchas de sus aserciones aparentemente duras eran más bien una confesión personal de fe y una promesa de devoción, que unos mandatos para sus seguidores. Esta misma determinación y esta devoción desinteresada fueron las que le permitieron efectuar, en una corta vida, un progreso tan extraordinario en la conquista de su mente humana. Muchas de sus declaraciones deberían ser consideradas como una confesión de lo que se exigía a sí mismo, en lugar de una exigencia para todos sus seguidores. En su devoción a la causa del reino, Jesús quemó todos los puentes detrás de él; sacrificó todo lo que fuera un obstáculo para hacer la voluntad de su Padre.

\par 
%\textsuperscript{(2093.2)}
\textsuperscript{196:2.8} Jesús bendecía a los pobres porque generalmente eran sinceros y piadosos; condenaba a los ricos porque habitualmente eran libertinos e irreligiosos. Pero hubiera condenado igualmente a los indigentes irreligiosos y alabado a los hombres de dinero consagrados y honorables.

\par 
%\textsuperscript{(2093.3)}
\textsuperscript{196:2.9} Jesús inducía a los hombres a sentirse en el mundo como en su hogar; los liberaba de la esclavitud de los tabúes y les enseñaba que el mundo no es fundamentalmente malo. No anhelaba huir de su vida terrenal; dominó una técnica para hacer aceptablemente la voluntad del Padre mientras vivía en la carne. Alcanzó una vida religiosa idealista en medio de un mundo realista. Jesús no compartía la opinión pesimista de Pablo sobre la humanidad. El Maestro consideraba a los hombres como hijos de Dios y preveía un futuro magnífico y eterno para aquellos que escogieran sobrevivir. No era un escéptico moral; miraba al hombre de manera positiva, no negativa. Veía que la mayoría de los hombres eran más bien débiles que malvados, más bien aturdidos que depravados. Pero cualquiera que fuera su condición, todos eran hijos de Dios y sus hermanos.

\par 
%\textsuperscript{(2093.4)}
\textsuperscript{196:2.10} Enseñó a los hombres a que se atribuyeran un alto valor en el tiempo y en la eternidad. Como Jesús tenía esta alta estima por los hombres, estaba dispuesto a dedicarse al servicio incansable de la humanidad. Este valor infinito que atribuía a lo finito es lo que hacía que la regla de oro fuera un factor vital en su religión. ¿Qué mortal puede dejar de sentirse elevado por la fe extraordinaria que Jesús tiene en él?

\par 
%\textsuperscript{(2093.5)}
\textsuperscript{196:2.11} Jesús no ofreció ninguna regla para el progreso social; su misión era religiosa, y la religión es una experiencia exclusivamente individual. La meta última del logro más avanzado de la sociedad nunca puede esperar trascender la fraternidad de los hombres enseñada por Jesús, basada en el reconocimiento de la paternidad de Dios. El ideal de todo logro social sólo se puede realizar con la llegada de este reino divino.

\section*{3. La supremacía de la religión}
\par 
%\textsuperscript{(2093.6)}
\textsuperscript{196:3.1} La experiencia religiosa espiritual personal resuelve eficientemente la mayoría de las dificultades de los mortales; clasifica, evalúa y ajusta eficazmente todos los problemas humanos. La religión no aleja ni destruye las dificultades humanas, pero las disuelve, las absorbe, las ilumina y las trasciende. La verdadera religión unifica la personalidad para que se ajuste eficazmente a todas las necesidades de los mortales. La fe religiosa ---la guía positiva de la presencia divina interior--- permite indefectiblemente al hombre que conoce a Dios salvar ese abismo que existe entre la lógica intelectual que reconoce a la Primera Causa Universal como \textit{Eso}, y las afirmaciones positivas del alma que afirman que esta Primera Causa es \textit{Él}, el Padre celestial del evangelio de Jesús, el Dios personal de la salvación humana.

\par 
%\textsuperscript{(2094.1)}
\textsuperscript{196:3.2} Hay exactamente tres elementos en la realidad universal: los hechos, las ideas y las relaciones. La conciencia religiosa identifica estas realidades como ciencia, filosofía y verdad. La filosofía se siente inclinada a considerar estas actividades como razón, sabiduría y fe ---la realidad física, la realidad intelectual y la realidad espiritual. Nosotros tenemos la costumbre de distinguir estas realidades como cosas, significados y valores.

\par 
%\textsuperscript{(2094.2)}
\textsuperscript{196:3.3} La comprensión progresiva de la realidad equivale a acercarse a Dios. El descubrimiento de Dios, la conciencia de identificarse con la realidad, equivale a experimentar el yo completo ---el yo entero, el yo total. Experimentar la realidad total es comprender plenamente a Dios, la finalidad de la experiencia de conocer a Dios.

\par 
%\textsuperscript{(2094.3)}
\textsuperscript{196:3.4} La suma total de la vida humana consiste en el conocimiento de que el hombre es educado por los hechos, ennoblecido por la sabiduría y salvado ---justificado--- por la fe religiosa.

\par 
%\textsuperscript{(2094.4)}
\textsuperscript{196:3.5} La certidumbre física consiste en la lógica de la ciencia; la certidumbre moral, en la sabiduría de la filosofía; la certidumbre espiritual, en la verdad de la experiencia religiosa auténtica.

\par 
%\textsuperscript{(2094.5)}
\textsuperscript{196:3.6} La mente del hombre puede alcanzar unos niveles elevados de perspicacia espiritual y las esferas correspondientes de divinidad de valores porque no es enteramente material. Existe un núcleo espiritual en la mente del hombre ---el Ajustador de la presencia divina. Hay tres pruebas distintas de que este espíritu habita en la mente humana:

\par 
%\textsuperscript{(2094.6)}
\textsuperscript{196:3.7} 1. La comunión humanitaria ---el amor. La mente puramente animal puede ser gregaria para protegerse, pero sólo el intelecto habitado por el espíritu es generosamente altruista e incondicionalmente amoroso.

\par 
%\textsuperscript{(2094.7)}
\textsuperscript{196:3.8} 2. La interpretación del universo ---la sabiduría. Sólo la mente habitada por el espíritu puede comprender que el universo es amistoso para el individuo.

\par 
%\textsuperscript{(2094.8)}
\textsuperscript{196:3.9} 3. La evaluación espiritual de la vida ---la adoración. Sólo el hombre habitado por el espíritu puede darse cuenta de la presencia divina y tratar de alcanzar una experiencia más completa en y con este anticipo de la divinidad.

\par 
%\textsuperscript{(2094.9)}
\textsuperscript{196:3.10} La mente humana no crea valores reales; la experiencia humana no ofrece una perspicacia del universo. En lo que concierne a la perspicacia, el reconocimiento de los valores morales y el discernimiento de los significados espirituales, todo lo que la mente humana puede hacer es descubrir, reconocer, interpretar y \textit{elegir}.

\par 
%\textsuperscript{(2094.10)}
\textsuperscript{196:3.11} Los valores morales del universo se vuelven posesiones intelectuales mediante el ejercicio de los tres criterios básicos, o elecciones, de la mente mortal:

\par 
%\textsuperscript{(2094.11)}
\textsuperscript{196:3.12} 1. El criterio de sí mismo ---la elección moral.

\par 
%\textsuperscript{(2094.12)}
\textsuperscript{196:3.13} 2. El criterio social ---la elección ética.

\par 
%\textsuperscript{(2094.13)}
\textsuperscript{196:3.14} 3. El criterio de Dios ---la elección religiosa.

\par 
%\textsuperscript{(2094.14)}
\textsuperscript{196:3.15} Así pues, parece ser que todo progreso humano se efectúa mediante una técnica de \textit{evolución revelatoria} conjunta.

\par 
%\textsuperscript{(2094.15)}
\textsuperscript{196:3.16} Si un amante divino no viviera en él, el hombre no podría amar de manera desinteresada y espiritual. Si un intérprete no viviera en su mente, el hombre no podría comprender realmente la unidad del universo. Si un evaluador no residiera en él, al hombre le sería totalmente imposible apreciar los valores morales y reconocer los significados espirituales. Y este amante procede de la fuente misma del amor infinito; este intérprete es una parte de la Unidad Universal; este evaluador es el hijo del Centro y la Fuente de todos los valores absolutos de la realidad divina y eterna.

\par 
%\textsuperscript{(2095.1)}
\textsuperscript{196:3.17} La evaluación moral con un significado religioso ---la perspicacia espiritual--- conlleva la elección del individuo entre el bien y el mal, la verdad y el error, lo material y lo espiritual, lo humano y lo divino, el tiempo y la eternidad. La supervivencia humana depende, en gran parte, de que la voluntad humana se consagre a escoger los valores elegidos por este clasificador de los valores espirituales ---el intérprete y unificador interior. La experiencia religiosa personal consta de dos fases: el descubrimiento en la mente humana, y la revelación por el espíritu divino interior. Debido a una sofisticación excesiva o a consecuencia de la conducta impía de unas personas supuestamente religiosas, un hombre o incluso una generación de hombres pueden elegir interrumpir sus esfuerzos por descubrir al Dios que vive en ellos; pueden dejar de progresar en la revelación divina y no llegar a alcanzarla. Pero estas actitudes desprovistas de progreso espiritual no pueden durar mucho tiempo debido a la presencia y a la influencia de los Ajustadores interiores del Pensamiento.

\par 
%\textsuperscript{(2095.2)}
\textsuperscript{196:3.18} Esta profunda experiencia de la realidad de la presencia divina interior trasciende para siempre la rudimentaria técnica materialista de las ciencias físicas. No podéis colocar la alegría espiritual debajo de un microscopio; no podéis pesar el amor en una balanza; no podéis medir los valores morales; ni tampoco podéis calcular la calidad de la adoración espiritual.

\par 
%\textsuperscript{(2095.3)}
\textsuperscript{196:3.19} Los hebreos tenían una religión de sublimidad moral; los griegos desarrollaron una religión de belleza; Pablo y sus compañeros fundaron una religión de fe, esperanza y caridad. Jesús reveló y ejemplificó una religión de amor: la seguridad en el amor del Padre, con la alegría y la satisfacción consiguientes de compartir este amor al servicio de la fraternidad humana.

\par 
%\textsuperscript{(2095.4)}
\textsuperscript{196:3.20} Cada vez que el hombre hace una elección moral reflexiva, experimenta de inmediato una nueva invasión divina de su alma. La elección moral constituye la religión porque es el motivo de la reacción interior a las condiciones exteriores. Pero esta religión real no es una experiencia puramente subjetiva. Significa que la totalidad subjetiva del individuo está ocupada en una respuesta significativa e inteligente a la objetividad total ---al universo y a su Hacedor.

\par 
%\textsuperscript{(2095.5)}
\textsuperscript{196:3.21} La experiencia exquisita y trascendente de amar y ser amado es puramente subjetiva, pero eso no significa que sea solamente una ilusión psíquica. La única realidad verdaderamente divina y objetiva que está asociada con los seres mortales, el Ajustador del Pensamiento, funciona aparentemente para la observación humana como un fenómeno exclusivamente subjetivo. El contacto del hombre con la realidad objetiva más elevada ---Dios--- sólo se efectúa a través de la experiencia puramente subjetiva de conocerlo, adorarlo y comprender la filiación con él.

\par 
%\textsuperscript{(2095.6)}
\textsuperscript{196:3.22} La verdadera adoración religiosa no es un monólogo inútil en el que uno se engaña a sí mismo. La adoración es una comunión personal con lo que es divinamente real, con lo que es la fuente misma de la realidad. Mediante la adoración, el hombre aspira a ser mejor, y por medio de ella, alcanza finalmente lo \textit{mejor}.

\par 
%\textsuperscript{(2095.7)}
\textsuperscript{196:3.23} La idealización de la verdad, la belleza y la bondad, y el intento de servirlas, no son un sustituto de la experiencia religiosa auténtica ---la realidad espiritual. La psicología y el idealismo no son el equivalente de la realidad religiosa. Las proyecciones del intelecto humano pueden originar en verdad falsos dioses ---dioses a la imagen del hombre--- pero la verdadera conciencia de Dios no se origina de esta manera. La conciencia de Dios reside en el espíritu interior. Muchos sistemas religiosos del hombre provienen de las formulaciones del intelecto humano, pero la conciencia de Dios no forma parte necesariamente de estos sistemas grotescos de esclavitud religiosa.

\par 
%\textsuperscript{(2095.8)}
\textsuperscript{196:3.24} Dios no es una simple invención del idealismo del hombre; él es la fuente misma de todas estas perspicacias y valores superanimales. Dios no es una hipótesis formulada para unificar los conceptos humanos de la verdad, la belleza y la bondad; él es la personalidad de amor de la que proceden todas estas manifestaciones universales. La verdad, la belleza y la bondad del mundo del hombre están unificadas por la espiritualidad creciente de la experiencia de los mortales que ascienden hacia las realidades del Paraíso. La unión de la verdad, la belleza y la bondad sólo se puede realizar en la experiencia espiritual de la personalidad que conoce a Dios.

\par 
%\textsuperscript{(2096.1)}
\textsuperscript{196:3.25} La moralidad es el terreno preexistente esencial de la conciencia personal de Dios, la comprensión personal de la presencia interior del Ajustador, pero esta moralidad no es el origen de la experiencia religiosa ni de la perspicacia espiritual resultante. La naturaleza moral es superanimal pero subespiritual. La moralidad equivale a reconocer el deber, a comprender la existencia del bien y del mal. La zona moral se interpone entre el tipo de mente animal y el tipo de mente humana, al igual que la morontia desempeña su función entre las esferas materiales y las esferas espirituales que alcanza la personalidad.

\par 
%\textsuperscript{(2096.2)}
\textsuperscript{196:3.26} La mente evolutiva es capaz de descubrir la ley, la moral y la ética; pero el espíritu otorgado, el Ajustador interior, revela a la mente humana en evolución el legislador, el Padre-fuente de todo lo que es verdadero, bello y bueno. Un hombre iluminado así tiene una religión y está espiritualmente equipado para empezar la larga e intrépida búsqueda de Dios.

\par 
%\textsuperscript{(2096.3)}
\textsuperscript{196:3.27} La moralidad no es necesariamente espiritual; puede ser total y puramente humana, aunque la auténtica religión realza todos los valores morales, los hace más significativos. La moralidad sin religión no logra revelar la bondad última y tampoco consigue asegurar la supervivencia de ni siquiera sus propios valores morales. La religión asegura el engrandecimiento, la glorificación y la supervivencia indudable de todo lo que la moralidad reconoce y aprueba.

\par 
%\textsuperscript{(2096.4)}
\textsuperscript{196:3.28} La religión se encuentra por encima de la ciencia, el arte, la filosofía, la ética y la moral, pero no es independiente de ellas. Todas están indisolublemente interrelacionadas en la experiencia humana, personal y social. La religión es la experiencia suprema del hombre en su estado natural como ser mortal, pero el lenguaje finito hace imposible para siempre que la teología pueda describir adecuadamente la auténtica experiencia religiosa.

\par 
%\textsuperscript{(2096.5)}
\textsuperscript{196:3.29} La perspicacia religiosa posee el poder de transformar una derrota en deseos superiores y en nuevas determinaciones. El amor es la motivación más elevada que el hombre puede utilizar en su ascensión por el universo. Pero el amor, cuando está despojado de la verdad, la belleza y la bondad, sólo es un sentimiento, una deformación filosófica, una ilusión psíquica, un engaño espiritual. El amor ha de ser siempre definido de nuevo en los niveles sucesivos de la evolución morontial y espiritual.

\par 
%\textsuperscript{(2096.6)}
\textsuperscript{196:3.30} El arte surge del intento del hombre por huir de la falta de belleza de su entorno material; es un gesto hacia el nivel morontial. La ciencia es el esfuerzo del hombre por resolver los enigmas aparentes del universo material. La filosofía es la tentativa del hombre por unificar la experiencia humana. La religión es el gesto supremo del hombre, su esfuerzo magnífico por alcanzar la realidad final, su determinación de encontrar a Dios y de parecerse a él.

\par 
%\textsuperscript{(2096.7)}
\textsuperscript{196:3.31} En el terreno de la experiencia religiosa, la posibilidad espiritual es una realidad potencial. El impulso espiritual hacia adelante del hombre no es una ilusión psíquica. Toda la fantasía del hombre sobre el universo puede no ser un hecho, pero una parte, una gran parte es verdad.

\par 
%\textsuperscript{(2096.8)}
\textsuperscript{196:3.32} La vida de algunos hombres es demasiado grande y noble como para descender al bajo nivel de un simple éxito. El animal debe adaptarse al entorno, pero el hombre religioso trasciende su entorno y elude así las limitaciones del presente mundo material mediante esta perspicacia del amor divino. Este concepto del amor produce en el alma del hombre ese esfuerzo superanimal por encontrar la verdad, la belleza y la bondad; y cuando las encuentra, es glorificado en su abrazo; le consume el deseo de vivirlas, de actuar con rectitud.

\par 
%\textsuperscript{(2097.1)}
\textsuperscript{196:3.33} No os desaniméis; la evolución humana continúa avanzando, y la revelación de Dios al mundo, en Jesús y por Jesús, no fracasará.

\par 
%\textsuperscript{(2097.2)}
\textsuperscript{196:3.34} El gran desafío para el hombre moderno consiste en conseguir una mejor comunicación con el Monitor divino que reside en la mente humana. La aventura más grande del hombre en la carne consiste en el esfuerzo sano y bien equilibrado por elevar los límites de la conciencia de sí a través de los reinos imprecisos de la conciencia embrionaria del alma, en un esfuerzo sincero por alcanzar la zona fronteriza de la conciencia espiritual ---el contacto con la presencia divina. Esta experiencia constituye la conciencia de Dios, una experiencia que confirma poderosamente la verdad preexistente de la experiencia religiosa de conocer a Dios. Esta conciencia del espíritu equivale a conocer la realidad de la filiación con Dios. De otro modo, la seguridad de la filiación es la experiencia de la fe.

\par 
%\textsuperscript{(2097.3)}
\textsuperscript{196:3.35} La conciencia de Dios equivale a la integración del yo en el universo y en sus niveles más elevados de realidad espiritual. Únicamente el contenido espiritual de un valor cualquiera es imperecedero. Incluso aquello que es verdadero, bello y bueno no puede perecer en la experiencia humana. Si el hombre no elige sobrevivir, entonces el Ajustador sobreviviente conservará esas realidades nacidas del amor y alimentadas en el servicio. Todas estas cosas forman parte del Padre Universal. El Padre es amor viviente, y esta vida del Padre se encuentra en sus Hijos. Y el espíritu del Padre reside en los hijos de sus Hijos ---los hombres mortales. Cuando todo ha sido dicho y hecho, la idea de Padre continúa siendo el concepto humano más elevado de Dios.

\newpage
\pagestyle{empty}

\par {\huge Abreviaturas}
\bigbreak
\bigbreak
\begin{multicols}{2}
	\par LU \textit{(El Libro de Urantia)}
	\bigbreak
	\par Libros bíblicos:
	\bigbreak
	\par Abd \textit{(Abdías)}
	\par Am \textit{(Amós)}
	\par Ap \textit{(Apocalipsis)}
	\par Bar \textit{(Baruc)}
	\par Co \textit{(Epístola a los Corintios)}
	\par Cnt \textit{(El Cantar de los Cantares)}
	\par Col \textit{(Epístola a los Colosenses)}
	\par Cr \textit{(Crónicas)}
	\par Dn \textit{(Daniel)}
	\par Dt \textit{(Deuteronomio)}
	\par Ec \textit{(Eclesiastés)}
	\par Eclo \textit{(Ecclesiástico)}
	\par Ef \textit{(Epístola a los Efesios)}
	\par Esd \textit{(Esdras)}
	\par Est \textit{(Ester)}
	\par Ex \textit{(Éxodo)}
	\par Ez \textit{(Ezequiel)} 
	\par Flm \textit{(Epístola a Filemón)}
	\par Flp \textit{(Epístola a los Filipenses)}
	\par Gl \textit{(Epítosla a los Gálatas)}
	\par Gn \textit{(Génesis)}
	\par Hab \textit{(Habacuc)} 
	\par Hag \textit{(Ageo)}
	\par Hch \textit{(Hechos de los Apóstoles)}
	\par Heb \textit{(Epístola a los Hebreos)}
	\par Is \textit{(Isaías)}
	\par Jer \textit{(Jeremías)}
	\par Jl \textit{(Joel)}
	\par Jn \textit{(Juan, evangelio y epístolas)}
	\par Job \textit{(Job)}
	\par Jon \textit{(Jonás)}
	\par Jos \textit{(Josué)}
	\par Jud \textit{(Epístola de Judas)}
	\par Jue \textit{(Jueces)}
	\par Lc \textit{(Lucas)}
	\par Lm \textit{(Lamentaciones)}
	\par Lv \textit{(Levítico)}
	\par Mac \textit{(Macabeos)}
	\par Mal \textit{(Malaquías)}
	\par Mc \textit{(Marcos)}
	\par Miq \textit{(Miqueas)} 
	\par Mt \textit{(Mateo)}
	\par Nah \textit{(Nahúm)}
	\par Neh \textit{(Nehemías)} 
	\par Nm \textit{(Números)}
	\par Os \textit{(Oseas)}
	\par P \textit{(Epístola de Pedro)}
	\par Pr \textit{(Proverbios)}
	\par Re \textit{(Reyes)}
	\par Ro \textit{(Epístola a los Romanos)}
	\par Rt \textit{(Rut)}
	\par Sab \textit{(Sabiduría)}
	\par Sal \textit{(Salmos)}
	\par Sam \textit{(Samuel)}
	\par Sof \textit{(Sofonías)}
	\par Stg \textit{(Epístola a Santiago)}
	\par Ti \textit{(Epístola a Timoteo)}
	\par Tit \textit{(Epítosla a Tito)}
	\par Ts \textit{(Epístola a los Tesalonicenses)}
	\par Zac \textit{(Zacarías)}
	\bigbreak
	\par Libros bíblicos apócrifos:
	\bigbreak 
	\par AsMo \textit{(Asunción de Moisés)}
	\par Bel \textit{(Bel y el Dragón)} 
	\par Hen \textit{(Enoc)} 
	\par Man \textit{(Oración de Manasés)} 
	\par Tb \textit{(Tobit)}
	\bigbreak
	\par Libros de otras religiones: 
	\bigbreak
	\par XXX \textit{(YYYY)}
	
	
\end{multicols}

\end{document}
