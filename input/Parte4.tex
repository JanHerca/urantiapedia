% Author of this conversion to LaTeX format: Jan Herca, 2017
\documentclass[twoside, 11pt]{book}
\usepackage[T1]{fontenc} % indica al procesador cómo imprimir los caracteres
\usepackage{fontspec} % permite definir fuentes a partir de las instaladas en el SO
\usepackage{geometry}
\usepackage{graphicx}
\usepackage{float}
\usepackage{tocloft}
\usepackage{titleps}
\usepackage{emptypage}
\usepackage[spanish]{babel}
\usepackage{multicol}
% Text styles
\geometry{paperwidth=16cm, paperheight=24cm, top=2.5cm, bottom=1.7cm, inner=2.5cm, outer=1.2cm}

\makeatletter
\def\@makechapterhead#1{%
	\vspace*{50\p@}%
	{\parindent \z@ \raggedright \normalfont
		\interlinepenalty\@M
		\huge \bfseries #1\par\nobreak
		\vskip 40\p@
}}
\def\@makeschapterhead#1{%
	\vspace*{50\p@}%
	{\parindent \z@ \raggedright
		\normalfont
		\interlinepenalty\@M
		\huge \bfseries  #1\par\nobreak
		\vskip 40\p@
}}
\makeatother

\renewcommand{\cftchapleader}{\cftdotfill{\cftdotsep}}
\renewcommand{\thechapter}{}
\renewcommand{\cftchapfont}{\large}
\cftsetpnumwidth{3em}
\renewcommand{\cftchappagefont}{\large}


\title{La Quinta Revelación \newline Cuarto Volumen \newline La evolución de la civilización humana}
\date{}
\begin{document}
	
	\begin{titlepage}
		\centering
		{\Huge\bfseries El Libro de Urantia\par}
		{\huge\bfseries La Quinta Revelación\par}
		\vspace{1cm}
		{\huge\bfseries Cuarto Volumen\par}
		\vspace{1cm}
		{\huge\bfseries La evolución de la civilización humana\par}
		\vfill
		{\scshape\Large URANTIA FOUNDATION\par}
		{\scshape\Large CHICAGO ILLINOIS\par}
		{\Large 2009 Traducción al español Europea\par}
	\end{titlepage}
	
	
\par {\textcopyright} 2019 Jan Herca, de la edición
\par {\textcopyright} 2009 Urantia Foundation, de la traducción
\par {\textcopyright} 1993 Urantia Foundation, de otros materiales
\bigbreak
\par Jan Herca
\par Correo electrónico: janherca@gmail.com
\bigbreak
\par Urantia Foundation
\par 533 West Diversey Parkway
\par Chicago, IL 60614 EE.UU.A
\par Oficina: 1+(773) 525-3319
\par Fax: 1 +(773) 525-7739
\par Website: http://www.urantia.org
\par Correo electrónico: urantia@urantia.org
\bigbreak
\par Todos los derechos reservados, incluyendo el de traducción en los Estados Unidos de América, Canadá y en los demás países de la Unión Internacional de copyright. Todos los derechos reservados en los paises firmantes de la Union Panamericana de la Union internacional de copyright.
\par No todo el libro ni parte de él pueden ser copiados, reproducidos o traducidos en forma alguna, ya sea por medio electrónico, mecánico u otra forma, como fotocopia, grabación o archivo computerizado sin autorización por escrito del editor.
\par URANTIA,'' ``URANTIAN,'' ``EL LIBRO DE URANTIA'' y son marcas registradas de Urantia Foundation y su uso está sujeto a licencia.
\bigbreak
\par La Quinta Revelación es una reedición de El Libro de Urantia (Edición Europea). Está dividido en siete volúmenes para hacerlo más manejable y dispone de contenido adicional en forma de ayudas a la lectura integradas en el texto. El Libro de Urantia (Edición Europea) es una traducción de The Urantia Book realizada por la Fundación Urantia en 2009. 
\newpage

\begin{center}
	{\huge\bfseries Las partes del libro\par}
	\vspace{1cm}
	{\scshape\large PRIMER VOLUMEN\par}
	{\scshape\Large DIOS, EL UNIVERSO CENTRAL Y LOS SUPERUNIVERSOS\par}
	\vspace{1cm}
	
	{\scshape\large SEGUNDO VOLUMEN \par}
	{\scshape\Large EL UNIVERSO LOCAL\par}
	\vspace{1cm}
	
	{\scshape\large TERCER VOLUMEN \par}
	{\scshape\Large LA HISTORIA DE NUESTRO PLANETA, URANTIA\par}
	\vspace{1cm}
	
	{\scshape\large CUARTO VOLUMEN \par}
	{\scshape\Large LA EVOLUCIÓN DE LA CIVILIZACIÓN HUMANA\par}
	\vspace{1cm}
	
	{\scshape\large QUINTO VOLUMEN \par}
	{\scshape\Large LA RELIGIÓN, LA SOBREVIVENCIA A LA MUERTE Y LA DEIDAD EXPERIENCIAL\par}
	\vspace{1cm}
	
	{\scshape\large SEXTO VOLUMEN \par}
	{\scshape\Large LA VIDA Y LAS ENSEÑANZAS DE JESÚS - I\par}
	\vspace{1cm}
	
	{\scshape\large SÉPTIMO VOLUMEN \par}
	{\scshape\Large LA VIDA Y LAS ENSEÑANZAS DE JESÚS - II\par}
\end{center}
	
\newpage
\begin{center}
	{\small \textit {Intencionadamente en blanco}\par}
\end{center}
\newpage

\pagestyle{empty}


\tableofcontents

\newpagestyle{main}{
	%\setheadrule{1pt}% Header rule
	%\setfootrule{.4pt}% Footer rule
	\sethead[\small \thepage]% odd-left
	[]% odd-center
	[\begin{minipage}{0.9\textwidth}\begin{flushright}\scriptsize \MakeUppercase{\chaptertitle}\end{flushright}\end{minipage}]% odd-right
	{\begin{minipage}{0.9\textwidth}\scriptsize \MakeUppercase{\chaptertitle}\end{minipage}}% even-left
	{}% even-center
	{\small \thepage}% even-right
	\setfoot[]% odd-left
	[]% odd-center
	[]% odd-right
	{}% even-left
	{}% even-center
	{}% even-right
}

\pagestyle{main}
\renewcommand{\makeheadrule}{\rule[-.6\baselineskip]{\linewidth}{.4pt}}



\chapter{Documento 68. Los albores de la civilización}
\par
%\textsuperscript{(763.1)}
\textsuperscript{68:0.1} HE AQUÍ el comienzo de la narración de la larguísima lucha hacia adelante de la especie humana, partiendo de un estado apenas mejor que el de la existencia animal, y pasando por las épocas intermedias hasta llegar a los tiempos más recientes durante los cuales una civilización real, aunque imperfecta, se ha desarrollado entre las razas superiores de la humanidad.

\par
%\textsuperscript{(763.2)}
\textsuperscript{68:0.2} La civilización es una adquisición racial; no es inherente a la biología; por eso todos los niños deben criarse en un entorno de cultura, mientras que la juventud de cada generación sucesiva debe recibir de nuevo su educación. Las cualidades superiores de la civilización ---científicas, filosóficas y religiosas--- no se transmiten de una generación a otra por herencia directa. Estos logros culturales sólo se pueden preservar mediante la conservación inteligente de la herencia social.

\par
%\textsuperscript{(763.3)}
\textsuperscript{68:0.3} Los instructores de Dalamatia introdujeron la evolución social de tipo cooperativo, y durante trescientos mil años, la humanidad fue educada en la idea de las actividades colectivas. El hombre azul se benefició más que los demás de estas primeras enseñanzas sociales, el hombre rojo hasta cierto punto, y el hombre negro menos que los demás. En tiempos más recientes, las razas amarilla y blanca han manifestado el desarrollo social más avanzado de Urantia.

\section*{1. La socialización protectora}
\par
%\textsuperscript{(763.4)}
\textsuperscript{68:1.1} Cuando los hombres tienen que vivir estrechamente unidos, a menudo aprenden a amarse mutuamente, pero el hombre primitivo no rebosaba por naturaleza de sentimientos fraternales ni del deseo de tener contactos sociales con sus semejantes. Las razas primitivas aprendieron más bien a través de experiencias dolorosas que <<la unión hace la fuerza>>; y esta falta de atracción fraternal natural es la que obstaculiza actualmente la realización inmediata de la fraternidad entre los hombres en Urantia.

\par
%\textsuperscript{(763.5)}
\textsuperscript{68:1.2} La asociación se convirtió pronto en el precio de la supervivencia. El hombre solitario estaba indefenso, a menos que llevara una marca tribal que demostrara que pertenecía a un grupo, el cual se vengaría indudablemente de cualquier ataque contra él. Incluso en la época de Caín resultaba muy peligroso salir solo al exterior sin llevar alguna marca de asociación a un grupo\footnote{\textit{Marca de Caín}: Gn 4:14-15.}. La civilización se ha convertido en el seguro del hombre contra una muerte violenta, y las primas que hay que pagar son el sometimiento a las numerosas exigencias legales de la sociedad.

\par
%\textsuperscript{(763.6)}
\textsuperscript{68:1.3} La sociedad primitiva se fundó así sobre las necesidades recíprocas y sobre el aumento de la seguridad que proporcionaba la asociación. La sociedad humana ha evolucionado durante ciclos milenarios como consecuencia de este temor al aislamiento y gracias a una cooperación ofrecida a disgusto.

\par
%\textsuperscript{(763.7)}
\textsuperscript{68:1.4} Los seres humanos primitivos aprendieron pronto que los grupos son mucho más grandes y más fuertes que la simple suma de los individuos que los componen. Cien hombres unidos y trabajando al unísono pueden mover una piedra muy grande; una veintena de guardianes de la paz bien entrenados pueden contener a una muchedumbre enfurecida. Así es como nació la sociedad, no de una simple asociación numérica, sino más bien como consecuencia de la \textit{organización} de unos cooperadores inteligentes. Pero la cooperación no es una característica natural del hombre; éste aprende a cooperar, en primer lugar, a causa del miedo, y más tarde porque descubre que es muy beneficioso para hacer frente a las dificultades del tiempo y para protegerse contra los supuestos peligros de la eternidad.

\par
%\textsuperscript{(764.1)}
\textsuperscript{68:1.5} Los pueblos que pronto se organizaron así en una sociedad primitiva tuvieron más éxito en su lucha contra la naturaleza así como en su defensa contra sus semejantes; tenían mayores posibilidades de supervivencia; de ahí que la civilización haya progresado continuamente en Urantia, a pesar de sus múltiples retrocesos. Hasta ahora, el hecho de que los numerosos desatinos del hombre no hayan conseguido detener ni destruir la civilización humana se debe únicamente a que el valor de la supervivencia aumenta por medio de la asociación.

\par
%\textsuperscript{(764.2)}
\textsuperscript{68:1.6} La sociedad cultural contemporánea es más bien un fenómeno reciente, y este hecho está bien demostrado en la supervivencia actual de unas condiciones sociales tan primitivas como las que caracterizan a los aborígenes australianos y a los bosquimanos y pigmeos de África. Entre estos pueblos atrasados se puede observar algo de la antigua hostilidad tribal, la desconfianza personal y otros rasgos extremadamente antisociales tan característicos de todas las razas primitivas. Estos restos deplorables de los pueblos asociales de los tiempos antiguos atestiguan elocuentemente el hecho de que la tendencia individualista natural del hombre no puede competir con éxito con las organizaciones y asociaciones más potentes y poderosas que promueven el progreso social. Estas razas antisociales atrasadas y desconfiadas, que hablan un dialecto diferente cada sesenta u ochenta kilómetros, demuestran en qué tipo de mundo estaríais viviendo ahora si no hubiera sido por las enseñanzas combinadas del estado mayor corpóreo del Príncipe Planetario y los trabajos posteriores del grupo adámico de mejoradores raciales.

\par
%\textsuperscript{(764.3)}
\textsuperscript{68:1.7} La expresión moderna <<regreso a la naturaleza>> es una ilusión de la ignorancia, una creencia en la realidad de una antigua <<edad de oro>> ficticia. La única base que tiene la leyenda de la edad de oro es el hecho histórico de la existencia de Dalamatia y del Edén. Pero aquellas sociedades mejoradas estaban lejos de haber realizado los sueños utópicos.

\section*{2. Los factores del progreso social}
\par
%\textsuperscript{(764.4)}
\textsuperscript{68:2.1} La sociedad civilizada es el resultado de los primeros esfuerzos del hombre por superar su aversión al \textit{aislamiento}. Pero esto no indica necesariamente un afecto mutuo; y el estado turbulento actual de ciertos grupos primitivos ilustra muy bien las dificultades que tuvieron que vencer las primeras tribus. Pero aunque los individuos de una civilización puedan chocar entre sí y luchar entre ellos, y aunque la civilización misma pueda parecer un conjunto inconsistente de esfuerzos y de luchas, manifiesta de hecho un esfuerzo decidido, y no la monotonía mortal del estancamiento.

\par
%\textsuperscript{(764.5)}
\textsuperscript{68:2.2} Aunque el nivel de inteligencia ha contribuido considerablemente al ritmo del progreso cultural, la sociedad está fundamentalmente concebida para disminuir el elemento riesgo en el modo de vivir del individuo, y ha progresado con la misma rapidez que ha logrado disminuir el dolor y aumentar el elemento placer en la vida. Todo el cuerpo social avanza así lentamente hacia la meta de su destino ---la supervivencia o la extinción--- dependiendo de que esa meta sea la preservación de sí o la satisfacción propia. La preservación de sí da origen a la sociedad, mientras que el exceso de satisfacciones personales destruye la civilización.

\par
%\textsuperscript{(764.6)}
\textsuperscript{68:2.3} La sociedad se ocupa de perpetuarse, de conservarse y de satisfacerse, pero la autorrealización humana es digna de convertirse en el objetivo inmediato de muchos grupos culturales.

\par
%\textsuperscript{(765.1)}
\textsuperscript{68:2.4} El instinto gregario del hombre sencillo apenas es suficiente para explicar el desarrollo de una organización social como la que existe actualmente en Urantia. Aunque esta tendencia gregaria innata yace en la base de la sociedad humana, una gran parte de la sociabilidad del hombre es adquirida. El hambre y el deseo sexual fueron las dos grandes influencias que contribuyeron a que los seres humanos se asociaran pronto; el hombre comparte estos impulsos instintivos con el mundo animal. La vanidad y el temor, y más concretamente el miedo a los fantasmas, fueron otras dos emociones que empujaron a los seres humanos a unirse y a \textit{mantenerse} unidos.

\par
%\textsuperscript{(765.2)}
\textsuperscript{68:2.5} La historia no es más que la narración de la lucha milenaria del hombre por la comida. \textit{El hombre primitivo sólo pensaba cuando tenía hambre}; guardar la comida fue su primer acto de abnegación, de autodisciplina. Con el desarrollo de la sociedad, el hambre dejó de ser el único motivo para asociarse mutuamente. Otros muchos tipos de hambre, la satisfacción de diversas necesidades, condujeron a una asociación más estrecha de la humanidad. Pero la sociedad de hoy es inestable debido al crecimiento excesivo de unas supuestas necesidades humanas. La civilización occidental del siglo veinte se queja de cansancio bajo la enorme sobrecarga del lujo y la multiplicación desordenada de los deseos y anhelos humanos. La sociedad moderna sufre la tensión de una de sus fases más peligrosas debido a una extensa interasociación y a una interdependencia extremadamente complicada.

\par
%\textsuperscript{(765.3)}
\textsuperscript{68:2.6} La presión social del hambre, la vanidad y el miedo a los fantasmas era continua, pero el placer sexual era transitorio e irregular. El deseo sexual por sí solo no impulsó a los hombres y mujeres primitivos a asumir las pesadas cargas del mantenimiento de un hogar. El hogar primitivo estaba fundado en el desasosiego sexual que experimentaba el varón cuando estaba privado de satisfacciones frecuentes, y en el abnegado amor maternal de la mujer, que ésta comparte en cierta medida con las hembras de todos los animales superiores. La presencia de un bebé indefenso determinó la primera diferenciación entre las actividades masculinas y femeninas; la mujer tenía que mantener una residencia fija donde poder cultivar la tierra. Y desde los tiempos más primitivos, el lugar donde se halla la mujer siempre ha sido considerado como el hogar.

\par
%\textsuperscript{(765.4)}
\textsuperscript{68:2.7} De este modo, la mujer pronto se volvió indispensable para el sistema social en evolución, no tanto a causa de una pasión sexual efímera como a consecuencia de la \textit{necesidad de comida}; la mujer era una asociada esencial para poder alimentarse. Era una proveedora de alimentos, una bestia de carga y una compañera que podía soportar grandes abusos sin resentimientos violentos, y además de todas estas características deseables, era un medio siempre presente de satisfacción sexual.

\par
%\textsuperscript{(765.5)}
\textsuperscript{68:2.8} Casi todos los valores duraderos de la civilización tienen sus raíces en la familia. La familia fue el primer grupo pacífico con éxito, pues el hombre y la mujer aprendieron a ajustar sus antagonismos al mismo tiempo que enseñaban a sus hijos ocupaciones pacíficas.

\par
%\textsuperscript{(765.6)}
\textsuperscript{68:2.9} La función del matrimonio, en la evolución, es asegurar la supervivencia de la raza, y no simplemente realizar la felicidad personal; la preservación y la perpetuación de sí mismo son los verdaderos objetivos del hogar. El placer personal es secundario y no es esencial salvo como estímulo para asegurar la asociación entre los sexos. La naturaleza exige la supervivencia, pero las artes de la civilización continúan acrecentando los placeres del matrimonio y las satisfacciones de la vida familiar.

\par
%\textsuperscript{(765.7)}
\textsuperscript{68:2.10} Si ampliamos la noción de vanidad hasta incluir el orgullo, la ambición y el honor, entonces podremos discernir no solamente la manera en que estas tendencias contribuyen a la formación de las asociaciones humanas, sino también cómo mantienen unidos a los hombres, puesto que estas emociones son inútiles sin un público ante quien poder alardear. A la vanidad se le unieron pronto otras emociones e impulsos que necesitaban un campo social donde poder exhibirse y satisfacerse. Este grupo de emociones dio nacimiento a las primeras manifestaciones de todas las artes, ceremoniales, y a todas las formas de juegos deportivos y competiciones.

\par
%\textsuperscript{(766.1)}
\textsuperscript{68:2.11} La vanidad contribuyó poderosamente al nacimiento de la sociedad; pero en el momento de estas revelaciones, los esfuerzos tortuosos de una generación jactanciosa amenazan con anegar y sumergir toda la complicada estructura de una civilización extremadamente especializada. Hace mucho tiempo que la necesidad de placer ha sustituido al hambre; los objetivos sociales legítimos de la preservación de sí se están transformando rápidamente en unas formas viles y amenazadoras de satisfacción egoísta. La preservación de sí edifica la sociedad; la satisfacción egoísta desenfrenada destruye infaliblemente la civilización.

\section*{3. La influencia socializadora del miedo a los fantasmas}
\par
%\textsuperscript{(766.2)}
\textsuperscript{68:3.1} Los deseos primitivos produjeron la sociedad original, pero el miedo a los fantasmas la mantuvo unida y confirió a su existencia un aspecto extrahumano. El miedo corriente tenía un origen fisiológico: miedo al dolor físico, al hambre insatisfecha o a alguna calamidad terrestre; pero el miedo a los fantasmas era una clase de terror nueva y suprema.

\par
%\textsuperscript{(766.3)}
\textsuperscript{68:3.2} El factor individual más importante en la evolución de la sociedad humana fue probablemente soñar con fantasmas. Aunque la mayoría de los sueños inquietaba profundamente a la mente primitiva, soñar con fantasmas aterrorizó realmente a los hombres primitivos, y estos soñadores supersticiosos se echaron los unos en brazos de los otros dispuestos a asociarse en serio para protegerse mutuamente contra los peligros imaginarios, vagos e invisibles, del mundo de los espíritus. Soñar con fantasmas fue una de las primeras diferencias que aparecieron entre la mente animal y la mente humana. Los animales no se imaginan la supervivencia después de la muerte.

\par
%\textsuperscript{(766.4)}
\textsuperscript{68:3.3} A excepción de este factor de los fantasmas, toda la sociedad se construyó sobre las necesidades fundamentales y los instintos biológicos básicos. Pero el miedo a los fantasmas introdujo un nuevo factor en la civilización, un miedo que trascendía las necesidades elementales del individuo y que se elevaba muy por encima incluso de las luchas por conservar el grupo. El terror a los espíritus de los difuntos reveló una nueva y asombrosa forma de miedo, un terror espantoso y poderoso que contribuyó a fustigar a las clases sociales relajadas de los primeros tiempos para convertirlas en los grupos primitivos más completamente disciplinados y mejor controlados de los tiempos antiguos. Esta superstición insensata, que todavía sobrevive en parte, preparó la mente de los hombres, a través del miedo supersticioso a lo irreal y a lo sobrenatural, para el descubrimiento posterior del <<temor al Señor, que es el comienzo de la sabiduría>>\footnote{\textit{El temor al Señor, comienzo de la sabiduría}: Job 28:28; Sal 111:10; Pr 1:7; 9:10.}. Los miedos infundados de la evolución están destinados a ser sustituidos por el temor a la Deidad inspirado por la revelación. El culto primitivo del miedo a los fantasmas se convirtió en un poderoso lazo social, y desde aquel día tan lejano la humanidad siempre se ha estado más o menos esforzando por alcanzar la espiritualidad.

\par
%\textsuperscript{(766.5)}
\textsuperscript{68:3.4} El hambre y el amor obligaron a los hombres a juntarse; la vanidad y el miedo a los fantasmas los mantuvieron unidos. Pero estas emociones por sí solas, sin la influencia de las revelaciones que promueven la paz, son incapaces de soportar las tensiones de las desconfianzas e irritaciones de las interasociaciones humanas. Sin la ayuda de las fuentes superhumanas, la tensión social estalla cuando alcanza ciertos límites, y estas mismas influencias que movilizan a la sociedad ---el hambre, el amor, la vanidad y el miedo--- se conjuran para sumergir a la humanidad en la guerra y el derramamiento de sangre.

\par
%\textsuperscript{(766.6)}
\textsuperscript{68:3.5} La tendencia a la paz de la raza humana no es una dotación natural; tiene su origen en las enseñanzas de la religión revelada, en la experiencia acumulada de las razas progresivas, y principalmente en las enseñanzas de Jesús, el Príncipe de la Paz\footnote{\textit{Príncipe de la Paz}: Is 9:6.}.

\section*{4. La evolución de las costumbres}
\par
%\textsuperscript{(767.1)}
\textsuperscript{68:4.1} Todas las instituciones sociales modernas proceden de la evolución de las costumbres primitivas de vuestros antepasados salvajes; los convencionalismos de hoy son las costumbres modificadas y ampliadas de ayer. Lo que el hábito es para el individuo, la costumbre lo es para el grupo; y las costumbres de los grupos se convierten en culturas populares o en tradiciones tribales ---en los convencionalismos de las masas. Todas las instituciones de la sociedad humana actual tienen su origen humilde en estos primeros comienzos.

\par
%\textsuperscript{(767.2)}
\textsuperscript{68:4.2} Debe recordarse que las costumbres tuvieron su origen en el esfuerzo por adaptar la vida de los grupos a las condiciones de la existencia colectiva; las costumbres fueron la primera institución social del hombre. Todas estas reacciones tribales surgieron del esfuerzo por evitar el dolor y la humillación, procurando al mismo tiempo disfrutar del placer y del poder. El origen de las culturas populares, al igual que el origen de las lenguas, siempre es inconsciente y no deliberado, y por lo tanto siempre está envuelto en un velo de misterio.

\par
%\textsuperscript{(767.3)}
\textsuperscript{68:4.3} El miedo a los fantasmas condujo al hombre primitivo a imaginar lo sobrenatural, y estableció así unas bases sólidas para las poderosas influencias sociales de la ética y la religión, que a su vez preservaron intactas, de generación en generación, las costumbres y tradiciones de la sociedad. Al principio, la única cosa que estableció y cristalizó las costumbres fue la creencia de que los difuntos deseaban conservar celosamente la manera de vivir y de morir que habían tenido; por consiguiente, enviarían un castigo terrible a los mortales vivos que se atrevieran a tratar con un desprecio negligente las reglas de vida que ellos habían respetado cuando vivían en la carne. Todo esto está perfectamente ilustrado en la veneración que la raza amarilla tiene actualmente por sus antepasados. La religión primitiva que se desarrolló más tarde reforzó enormemente el miedo a los fantasmas mediante la estabilización de las costumbres, pero la civilización en progreso ha liberado cada vez más a la humanidad de la servidumbre del miedo y de la esclavitud de la superstición.

\par
%\textsuperscript{(767.4)}
\textsuperscript{68:4.4} Antes de las enseñanzas liberadoras y liberalizadoras de los instructores de Dalamatia, el hombre antiguo era una víctima indefensa del ritual de las costumbres; el salvaje primitivo estaba rodeado de un ceremonial interminable. Todo lo que hacía desde el momento en que se despertaba por la mañana hasta la hora de dormirse en su caverna por la noche, tenía que hacerlo exactamente de una manera determinada ---de acuerdo con la cultura popular de su tribu. Era un esclavo de la tiranía de la usanza; su vida no contenía nada libre, espontáneo ni original. No había ningún progreso natural hacia una existencia mental, moral o social superior.

\par
%\textsuperscript{(767.5)}
\textsuperscript{68:4.5} El hombre primitivo estaba extremadamente sujeto a la costumbre; el salvaje era un verdadero esclavo de la usanza; pero de vez en cuando surgieron diferentes tipos de personas que se atrevieron a introducir nuevas maneras de pensar y mejores métodos de vida. Sin embargo, la inercia del hombre primitivo constituye el freno de seguridad biológico contra la acción de precipitarse demasiado repentinamente en las inadaptaciones ruinosas de una civilización que progresa demasiado deprisa.

\par
%\textsuperscript{(767.6)}
\textsuperscript{68:4.6} Sin embargo, estas costumbres no son un mal absoluto; su evolución debe continuar. Emprender su modificación global mediante una revolución radical es casi fatal para la continuación de la civilización. La costumbre ha sido el hilo de continuidad que ha mantenido unida a la civilización. El sendero de la historia humana está sembrado de restos de costumbres desechadas y de prácticas sociales obsoletas; pero ninguna civilización que haya abandonado sus costumbres ha perdurado, a menos que haya adoptado unas costumbres mejores y más adecuadas.

\par
%\textsuperscript{(767.7)}
\textsuperscript{68:4.7} La supervivencia de una sociedad depende principalmente de la evolución progresiva de sus costumbres. El proceso de la evolución de las costumbres surge del deseo de experimentar; se proponen ideas nuevas ---y se origina la rivalidad. Una civilización que progresa abraza las ideas avanzadas y perdura; el tiempo y las circunstancias seleccionan finalmente al grupo más apto para sobrevivir. Pero esto no significa que cada uno de los distintos cambios aislados en la composición de la sociedad humana haya sido para mejorar. ¡No! ¡Claro que no!, pues ha habido muchísimos retrocesos en la larga lucha de la civilización de Urantia por el progreso.

\section*{5. El uso del territorio ---las artes para sustentarse}
\par
%\textsuperscript{(768.1)}
\textsuperscript{68:5.1} La tierra es el teatro de la sociedad; los hombres son los actores. El hombre debe adaptar constantemente su forma de actuar para ajustarse a las condiciones de la tierra. La evolución de las costumbres depende siempre de la proporción entre el hombre y la tierra. Esto es cierto, aunque sea difícil discernirlo. Las técnicas del hombre para utilizar el territorio, o artes para sustentarse, más su nivel de vida, son iguales a la suma total de las culturas populares, de las costumbres. Y la suma de la adaptación del hombre a las exigencias de la vida es igual a su civilización cultural.

\par
%\textsuperscript{(768.2)}
\textsuperscript{68:5.2} Las primeras culturas humanas aparecieron a lo largo de los ríos del hemisferio oriental, y hubo cuatro grandes etapas en la marcha hacia adelante de la civilización, a saber:

\par
%\textsuperscript{(768.3)}
\textsuperscript{68:5.3} 1. \textit{La etapa de la recogida}. La coacción alimenticia, el hambre, condujo a la primera forma de organización industrial, a las filas primitivas para recoger alimentos. A veces, estas filas de caminantes hambrientos que atravesaban una región rebuscando alimentos medían quince kilómetros de longitud. Fue la etapa de la cultura nómada primitiva y es la forma de vida que siguen actualmente los bosquimanos de África.

\par
%\textsuperscript{(768.4)}
\textsuperscript{68:5.4} 2. \textit{La etapa de la caza}. La invención de los utensilios para defenderse permitió al hombre convertirse en cazador y liberarse así considerablemente de la esclavitud de la comida. Un andonita reflexivo que se había magullado gravemente el puño en un violento combate redescubrió la idea de utilizar un largo palo en lugar de su brazo, y un trozo de duro sílex atado con tendones en la punta para reemplazar el puño. Muchas tribus hicieron descubrimientos independientes de esta índole, y estas diversas formas de martillos representaron uno de los grandes pasos hacia adelante de la civilización humana. En la actualidad, algunos indígenas australianos no han progresado mucho más allá de esta etapa.

\par
%\textsuperscript{(768.5)}
\textsuperscript{68:5.5} Los hombres azules se convirtieron en unos cazadores y tramperos expertos; cercaban los ríos y atrapaban grandes cantidades de peces, desecando el excedente para utilizarlo durante el invierno. Se empleaban muchas formas de cepos y trampas ingeniosos para atrapar las presas, pero las razas más primitivas no cazaban los animales más grandes.

\par
%\textsuperscript{(768.6)}
\textsuperscript{68:5.6} 3. \textit{La etapa del pastoreo}. La domesticación de los animales hizo posible esta fase de la civilización. Los árabes y los indígenas de África figuran entre los pueblos pastores más recientes.

\par
%\textsuperscript{(768.7)}
\textsuperscript{68:5.7} La vida pastoril permitió un alivio adicional de la esclavitud de la comida; el hombre aprendió a vivir de los beneficios de su capital, del aumento de sus rebaños, y esto le proporcionó más tiempo libre para la cultura y el progreso.

\par
%\textsuperscript{(768.8)}
\textsuperscript{68:5.8} La sociedad prepastoril había sido una sociedad de cooperación entre los sexos, pero la diseminación de la ganadería sumió a la mujer en un abismo de esclavitud social. En las épocas más primitivas, el hombre tenía la obligación de garantizar la alimentación animal, y la mujer tenía la ocupación de proporcionar los comestibles vegetales. Por consiguiente, la dignidad de la mujer cayó enormemente cuando el hombre entró en la era pastoril de su existencia. La mujer tenía que continuar trabajando para producir los alimentos vegetales necesarios para la vida, mientras que el hombre sólo necesitaba recurrir a sus rebaños para proporcionar abundante comida animal. El hombre se volvió así relativamente independiente de la mujer; y la situación de la mujer declinó continuamente durante toda la época pastoril. Hacia el final de este período, la mujer apenas era más que un animal humano, relegada a trabajar y a dar a luz a la descendencia humana, en gran medida tal como se esperaba que los animales del rebaño trabajaran y parieran sus crías. Los hombres de la época pastoril tenían un gran amor por su ganado, y es aún más lamentable que no hayan sabido desarrollar un afecto más profundo por sus esposas.

\par
%\textsuperscript{(769.1)}
\textsuperscript{68:5.9} 4. \textit{La etapa agrícola}. Esta era se originó debido a la aclimatación de las plantas, y representa el tipo más elevado de civilización material. Tanto Caligastia como Adán se esforzaron por enseñar la horticultura y la agricultura. Adán y Eva fueron horticultores y no pastores, pues el cultivo de la huerta era una forma avanzada de cultura en aquellos tiempos. El cultivo de las plantas ejerce una influencia ennoblecedora sobre todas las razas de la humanidad.

\par
%\textsuperscript{(769.2)}
\textsuperscript{68:5.10} La agricultura multiplicó por más de cuatro veces la proporción entre las tierras y los hombres en el mundo. Puede combinarse con las ocupaciones pastoriles de la etapa cultural anterior. Cuando las tres etapas se superponen, los hombres cazan y las mujeres cultivan la tierra.

\par
%\textsuperscript{(769.3)}
\textsuperscript{68:5.11} Siempre ha habido fricciones entre los pastores y los labradores. El cazador y el pastor eran belicosos, guerreros; el agricultor es más pacífico. El trato con los animales sugiere la lucha y la fuerza; la relación con las plantas inculca la paciencia, el sosiego y la paz. La agricultura y la industria son las actividades de la paz. Pero la debilidad de las dos, como actividades sociales mundiales, es que carecen de emoción y de aventura.

\par
%\textsuperscript{(769.4)}
\textsuperscript{68:5.12} La sociedad humana ha evolucionado desde la etapa de la caza, pasando por la de los pastores, hasta la etapa territorial de la agricultura. Cada etapa de esta civilización progresiva estuvo acompañada de una disminución constante del nomadismo; el hombre empezó a vivir cada vez más en el hogar.

\par
%\textsuperscript{(769.5)}
\textsuperscript{68:5.13} En la actualidad, la industria complementa a la agricultura, con el consiguiente aumento de la urbanización y la multiplicación de los grupos no agrícolas entre las clases de ciudadanos. Pero una era industrial no puede esperar sobrevivir si sus dirigentes no logran reconocer que los desarrollos sociales, incluso los más elevados, deben siempre descansar sobre una base agrícola sana.

\section*{6. La evolución de la cultura}
\par
%\textsuperscript{(769.6)}
\textsuperscript{68:6.1} El hombre es una criatura de la tierra, un hijo de la naturaleza; por mucho ardor que ponga en intentar liberarse de la tierra, a fin de cuentas puede estar seguro de que no lo logrará. <<Polvo eres y al polvo volverás>>\footnote{\textit{Polvo eres y al polvo volverás}: Gn 2:7; 3:19; Ec 3:20; 12:7.} se aplica al pie de la letra a toda la humanidad. La lucha básica del hombre era, es y siempre será por la tierra. Las primeras asociaciones sociales de seres humanos primitivos tuvieron por objetivo ganar estas batallas por la tierra. La proporción entre la tierra y el hombre es la base de toda la civilización social.

\par
%\textsuperscript{(769.7)}
\textsuperscript{68:6.2} La inteligencia del hombre acrecentó el rendimiento de la tierra por medio de las artes y las ciencias; al mismo tiempo, el aumento natural de su descendencia se pudo controlar un poco, y así se dispuso de los medios para subsistir y del tiempo libre para construir una civilización cultural.

\par
%\textsuperscript{(769.8)}
\textsuperscript{68:6.3} La sociedad humana está regulada por una ley que decreta que la población debe variar en proporción directa a las artes de la tierra y en proporción inversa a un nivel de vida determinado. A lo largo de todas estas épocas primitivas, mucho más que en la actualidad, la ley de la oferta y la demanda, en lo concerniente a los hombres y la tierra, determinaba el valor aproximado de los dos. Durante los períodos en que las tierras abundaban ---territorios despoblados--- la necesidad de hombres era grande, y por consiguiente el valor de la vida humana era muy elevado; de ahí que las pérdidas de vidas fueran consideradas con más horror. Durante los períodos de escasez de tierras y de la correspondiente superpoblación, el precio de la vida humana era comparativamente más bajo, de manera que la guerra, el hambre y la peste se consideraban con menos inquietud.

\par
%\textsuperscript{(770.1)}
\textsuperscript{68:6.4} Cuando disminuye el rendimiento de la tierra o aumenta la población, la inevitable lucha comienza de nuevo, y los peores rasgos de la naturaleza humana emergen a la superficie. El aumento del rendimiento de la tierra, la extensión de las artes mecánicas y la reducción de la población tienden a fomentar el desarrollo del lado mejor de la naturaleza humana.

\par
%\textsuperscript{(770.2)}
\textsuperscript{68:6.5} Una sociedad de pioneros produce obreros no cualificados; las bellas artes y el verdadero progreso científico, junto con la cultura espiritual, han prosperado mejor en los centros habitados más grandes, cuando han estado sostenidos por una población agrícola e industrial ligeramente por debajo de la proporción entre la tierra y el hombre. Las ciudades siempre multiplican el poder de sus habitantes para bien o para mal.

\par
%\textsuperscript{(770.3)}
\textsuperscript{68:6.6} El nivel de vida siempre ha influido sobre el tamaño de la familia. Cuanto más alto es el nivel más pequeña es la familia, hasta que se llega al punto en que la familia se estabiliza o se extingue gradualmente.

\par
%\textsuperscript{(770.4)}
\textsuperscript{68:6.7} A lo largo de todos los tiempos, los niveles de vida han determinado la calidad de una población sobreviviente en contraste con la simple cantidad. Los niveles de vida de una clase local dan nacimiento a nuevas castas sociales, a nuevas costumbres. Cuando los niveles de vida se vuelven demasiado complicados o excesivamente lujosos, tienden rápidamente al suicidio. Las castas son el resultado directo de la intensa presión social de una fuerte competencia producida por la densidad de la población.

\par
%\textsuperscript{(770.5)}
\textsuperscript{68:6.8} Las razas primitivas recurrieron a menudo a prácticas destinadas a restringir la población; todas las tribus primitivas mataban a los niños deformes o enfermizos. Antes de la época en que se compraban a las esposas, a las recién nacidas las mataban con frecuencia. A los niños los estrangulaban a veces al nacer, pero el método favorito era el abandono. El padre de unos gemelos insistía generalmente para que se matara a uno de los dos, porque se creía que los nacimientos múltiples eran causados por la magia o la infidelidad. Sin embargo, a los gemelos del mismo sexo se les perdonaba generalmente la vida. Aunque estos tabúes sobre los gemelos fueron en otro tiempo casi universales, nunca formaron parte de las costumbres de los andonitas; estos pueblos siempre consideraron a los gemelos como presagios de buena suerte.

\par
%\textsuperscript{(770.6)}
\textsuperscript{68:6.9} Muchas razas aprendieron la técnica del aborto, y esta práctica se volvió muy común después de que se estableciera el tabú sobre el alumbramiento entre las no casadas. Las solteras tuvieron durante mucho tiempo la costumbre de matar a sus hijos, pero entre los grupos más civilizados estos hijos ilegítimos se ponían bajo la tutela de la madre de la joven. Muchos clanes primitivos estuvieron a punto de exterminarse debido a la práctica conjunta del aborto y el infanticidio. Sin embargo, a pesar de los dictados de las costumbres, a muy pocos niños les quitaban la vida una vez que habían sido amamantados ---el amor maternal es demasiado fuerte.

\par
%\textsuperscript{(770.7)}
\textsuperscript{68:6.10} En el siglo veinte sobreviven todavía algunos restos de estas regulaciones primitivas de la población. Existe una tribu en Australia donde las madres se niegan a criar a más de dos o tres hijos. No hace mucho tiempo, una tribu caníbal se comía a cada quinto hijo que nacía. En Madagascar, algunas tribus siguen matando a todos los niños que nacen durante ciertos días nefastos, ocasionando la muerte de casi el veinticinco por ciento de todos los recién nacidos.

\par
%\textsuperscript{(770.8)}
\textsuperscript{68:6.11} Desde el punto de vista mundial, la superpoblación nunca ha sido un grave problema en el pasado, pero si las guerras disminuyen y la ciencia controla cada vez más las enfermedades humanas, puede convertirse en un problema serio en el futuro cercano. En ese momento se presentará la gran prueba de sabiduría para los dirigentes del mundo. Los gobernantes de Urantia ¿tendrán la perspicacia y la valentía de fomentar la multiplicación de los seres humanos de tipo medio o estabilizados, en lugar de favorecer la de los grupos extremos compuestos por los que son superiores a la normalidad y por los grupos cada vez más grandes de seres inferiores a la normalidad? Se debería fomentar el hombre normal; él es la espina dorsal de la civilización y la fuente de los genios mutantes de la raza. El hombre inferior a la normalidad debería estar sujeto al control de la sociedad; no se deberían tener más de los que se necesitan para atender los niveles inferiores de la industria, aquellas tareas que requieren una inteligencia por encima del nivel animal, pero que precisan unos esfuerzos tan pequeños que resultan una verdadera esclavitud y una servidumbre para los tipos superiores de la humanidad.

\par
%\textsuperscript{(771.1)}
\textsuperscript{68:6.12} [Presentado por un Melquisedek destinado en otro tiempo en Urantia.]


\chapter{Documento 69. Las instituciones humanas primitivas}
\par
%\textsuperscript{(772.1)}
\textsuperscript{69:0.1} EN el plano emocional, el hombre trasciende a sus antepasados animales por su capacidad para apreciar el humor, el arte y la religión. En el plano social, el hombre muestra su superioridad fabricando herramientas, comunicándose con los demás y estableciendo instituciones.

\par
%\textsuperscript{(772.2)}
\textsuperscript{69:0.2} Cuando los seres humanos mantienen sus grupos sociales durante mucho tiempo, estos colectivos siempre ocasionan la creación de ciertas tendencias a la actividad que culminan en la institucionalización. La mayoría de las instituciones del hombre han demostrado que economizan trabajo y al mismo tiempo contribuyen en cierta medida a mejorar la seguridad colectiva.

\par
%\textsuperscript{(772.3)}
\textsuperscript{69:0.3} El hombre civilizado está muy orgulloso del carácter, la estabilidad y la continuidad de sus instituciones establecidas, pero todas las instituciones humanas son simplemente las costumbres acumuladas del pasado, tal como han sido conservadas por los tabúes y dignificadas por la religión. Estos legados se convierten en tradiciones, y las tradiciones se transforman finalmente en convenciones.

\section*{1. Las instituciones humanas fundamentales}
\par
%\textsuperscript{(772.4)}
\textsuperscript{69:1.1} Todas las instituciones humanas sirven para alguna necesidad social, pasada o presente, aunque su desarrollo excesivo resta méritos infaliblemente al individuo, eclipsando su personalidad y disminuyendo sus iniciativas. El hombre debería controlar sus instituciones, en lugar de dejarse dominar por estas creaciones de la civilización en progreso.

\par
%\textsuperscript{(772.5)}
\textsuperscript{69:1.2} Las instituciones humanas son generalmente de tres clases:

\par
%\textsuperscript{(772.6)}
\textsuperscript{69:1.3} 1. \textit{Las instituciones de autoconservación}. Estas instituciones abarcan las prácticas nacidas del hambre y de sus instintos asociados de autopreservación. Incluyen a la industria, la propiedad, la guerra de intereses y toda la maquinaria reguladora de la sociedad. Tarde o temprano, el instinto del miedo fomenta el establecimiento de estas instituciones de supervivencia mediante los tabúes, las convenciones y las sanciones religiosas. Pero el miedo, la ignorancia y la superstición han jugado un papel sobresaliente en el origen inicial y en el desarrollo posterior de todas las instituciones humanas.

\par
%\textsuperscript{(772.7)}
\textsuperscript{69:1.4} 2. \textit{Las instituciones de autoperpetuación}. Son las organizaciones de la sociedad que surgen del apetito sexual, del instinto maternal y de los sentimientos afectivos superiores de las razas. Abarcan las salvaguardias sociales del hogar y la escuela, de la vida familiar, la educación, la ética y la religión. Incluyen las costumbres matrimoniales, la guerra defensiva y el establecimiento del hogar.

\par
%\textsuperscript{(772.8)}
\textsuperscript{69:1.5} 3. \textit{Las instituciones de satisfacción personal}. Son las prácticas que surgen de las tendencias a la vanidad y de los sentimientos de orgullo; abarcan las costumbres de la vestimenta y del adorno personal, las usanzas sociales, las guerras de prestigio, el baile, la diversión, los juegos y otras formas de placeres sensuales. Pero la civilización nunca ha producido por evolución unas instituciones definidas para las satisfacciones personales.

\par
%\textsuperscript{(773.1)}
\textsuperscript{69:1.6} Estos tres grupos de prácticas sociales están íntimamente interrelacionados y son minuciosamente interdependientes los unos de los otros. En Urantia representan una organización compleja que funciona como un solo mecanismo social.

\section*{2. Los albores de la industria}
\par
%\textsuperscript{(773.2)}
\textsuperscript{69:2.1} La industria primitiva se desarrolló lentamente como un seguro contra los terrores del hambre. Desde el principio de su existencia, el hombre empezó a tomar lecciones de algunos animales que almacenaban la comida durante las cosechas abundantes para los períodos de escasez.

\par
%\textsuperscript{(773.3)}
\textsuperscript{69:2.2} Antes de la aparición de la frugalidad inicial y de la industria primitiva, la suerte que corrían las tribus de tipo medio era la miseria y los auténticos sufrimientos. El hombre primitivo tenía que competir con todo el reino animal para conseguir su comida. La presión de la competitividad siempre arrastra al hombre hacia el nivel de la bestia; la pobreza es su estado natural y tiránico. La riqueza no es un don natural; es el resultado del trabajo, del conocimiento y de la organización.

\par
%\textsuperscript{(773.4)}
\textsuperscript{69:2.3} El hombre primitivo no tardó en reconocer las ventajas de la asociación. La asociación condujo a la organización, y el primer resultado de la organización fue la división del trabajo, con su ahorro inmediato de tiempo y de materiales. Estas especializaciones del trabajo surgieron de la adaptación a las presiones ---siguiendo las líneas de menor resistencia. Los salvajes primitivos no realizaron nunca un trabajo real con alegría o de buena gana. La conformidad que tenían se debía a la fuerza de la necesidad.

\par
%\textsuperscript{(773.5)}
\textsuperscript{69:2.4} El hombre primitivo tenía aversión por el trabajo duro, y no se apresuraba a menos que tuviera que enfrentarse con algún peligro grave. El tiempo, considerado como un elemento del trabajo, la idea de realizar una tarea determinada dentro de un cierto límite de tiempo, es una noción totalmente moderna. Los antiguos nunca tenían prisa. La doble exigencia de la intensa lucha por la existencia y del progreso constante de los niveles de vida fue lo que empujó a las razas de hombres primitivos, ociosas por naturaleza, por los caminos de la industria.

\par
%\textsuperscript{(773.6)}
\textsuperscript{69:2.5} El trabajo, los esfuerzos creativos, distinguen al hombre de la bestia, cuyos esfuerzos son ampliamente instintivos. La necesidad de trabajar es la bendición suprema del hombre. Todo el estado mayor del Príncipe trabajaba; contribuyeron mucho a ennoblecer el trabajo físico en Urantia. Adán fue horticultor; el Dios de los hebreos trabajaba ---era el creador y el sostén de todas las cosas. Los hebreos fueron la primera tribu que dio un valor supremo a la industria; fueron el primer pueblo que decretó que <<el que no trabaje no comerá>>\footnote{\textit{El que no trabaje no comerá}: Gn 3:19; Pr 20:4; 2 Ts 3:10.}. Pero muchas religiones del mundo volvieron al ideal primitivo de la ociosidad. Júpiter era un juerguista y Buda se convirtió en un partidario meditabundo del ocio.

\par
%\textsuperscript{(773.7)}
\textsuperscript{69:2.6} Las tribus sangiks fueron bastante trabajadoras mientras residieron lejos de los trópicos. Pero hubo una larguísima lucha entre los adeptos perezosos de la magia y los apóstoles del trabajo ---los que practicaban la previsión.

\par
%\textsuperscript{(773.8)}
\textsuperscript{69:2.7} La primera previsión humana tuvo como finalidad la conservación del fuego, el agua y la comida. Pero el hombre primitivo era un jugador nato; siempre quería obtener algo a cambio de nada, y durante aquellos tiempos primitivos, los éxitos procedentes de un trabajo asiduo se atribuían con demasiada frecuencia a los hechizos. La magia tardó mucho tiempo en ceder su lugar a la previsión, la abnegación y la industria.

\section*{3. La especialización del trabajo}
\par
%\textsuperscript{(773.9)}
\textsuperscript{69:3.1} Las divisiones del trabajo, en la sociedad primitiva, estuvieron determinadas, primero, por las circunstancias naturales, y luego por las sociales. El orden primitivo de la especialización del trabajo fue el siguiente:

\par
%\textsuperscript{(774.1)}
\textsuperscript{69:3.2} 1. \textit{La especialización basada en el sexo}. El trabajo de la mujer tuvo su origen en la presencia selectiva de los hijos; las mujeres, por naturaleza, aman a los bebés más que los hombres. La mujer se convirtió así en la trabajadora rutinaria, mientras que el hombre se hizo cazador y luchador, pasando por períodos muy diferenciados de trabajo y de descanso.

\par
%\textsuperscript{(774.2)}
\textsuperscript{69:3.3} A lo largo de todas las épocas, los tabúes han funcionado para mantener a la mujer estrictamente en su propio campo. El hombre ha escogido, de la manera más egoísta, el trabajo más agradable, dejando a la mujer el pesado trabajo rutinario. Al hombre siempre le ha avergonzado hacer el trabajo de la mujer, pero la mujer nunca ha mostrado la menor reticencia en hacer el trabajo del hombre. Y un hecho extraño a indicar es que tanto el hombre como la mujer siempre han trabajado juntos para construir y amueblar el hogar.

\par
%\textsuperscript{(774.3)}
\textsuperscript{69:3.4} 2. \textit{Las modificaciones debidas a la edad y las enfermedades}. Estas diferencias determinaron la siguiente división del trabajo. A los ancianos y los lisiados los pusieron pronto a fabricar las herramientas y las armas. Más tarde se les asignó la construcción de las obras de regadío.

\par
%\textsuperscript{(774.4)}
\textsuperscript{69:3.5} 3. \textit{La diferenciación basada en la religión}. Los curanderos fueron los primeros seres humanos que estuvieron exentos del trabajo físico; fueron los pioneros de las clases profesionales. Los herreros formaban un pequeño grupo que competía con los curanderos como magos. Su habilidad en el trabajo de los metales hizo que la gente tuviera miedo de ellos. Los <<herreros blancos>> (hojalateros) y los <<herreros negros>> (forjadores) dieron origen a las creencias primitivas en la magia blanca y la magia negra. Estas creencias se mezclaron más tarde con la superstición de los fantasmas buenos y malos, de los buenos y malos espíritus.

\par
%\textsuperscript{(774.5)}
\textsuperscript{69:3.6} Los herreros fueron el primer grupo no religioso que disfrutó de privilegios especiales. Eran considerados neutrales durante las guerras, y este tiempo libre adicional los llevó a convertirse, como clase, en los políticos de la sociedad primitiva. Pero debido a los grandes abusos que hicieron de estos privilegios, los herreros fueron odiados universalmente, y los curanderos se apresuraron en fomentar este odio por sus rivales. En esta primera contienda entre la ciencia y la religión, la religión (la superstición) fue la que triunfó. Después de ser arrojados fuera de los pueblos, los herreros mantuvieron las primeras posadas, las primeras casas de huéspedes públicas, en las afueras de las poblaciones.

\par
%\textsuperscript{(774.6)}
\textsuperscript{69:3.7} 4. \textit{Los amos y los esclavos}. La siguiente diferenciación del trabajo tuvo su origen en las relaciones entre los conquistadores y los conquistados, lo que significó el comienzo de la esclavitud humana.

\par
%\textsuperscript{(774.7)}
\textsuperscript{69:3.8} 5. \textit{La diferenciación basada en los diversos dones físicos y mentales}. Las diferencias intrínsecas entre los hombres favorecieron las nuevas divisiones del trabajo, pues todos los seres humanos no nacen iguales.

\par
%\textsuperscript{(774.8)}
\textsuperscript{69:3.9} Los primeros especialistas de la industria fueron los tallistas de sílex y los albañiles; a continuación vinieron los herreros. Posteriormente se desarrollaron las especializaciones colectivas; las familias y los clanes enteros se dedicaron a ciertos tipos de trabajos. El origen de una de las primeras castas sacerdotales, aparte de los curanderos tribales, se debió a la exaltación supersticiosa de una familia de expertos fabricantes de espadas.

\par
%\textsuperscript{(774.9)}
\textsuperscript{69:3.10} Los primeros especialistas colectivos de la industria fueron los exportadores de sal gema y los alfareros. Las mujeres fabricaban la alfarería sencilla y los hombres la de fantasía. En algunas tribus, la tejeduría y la costura las realizaban las mujeres, y en otras los hombres.

\par
%\textsuperscript{(774.10)}
\textsuperscript{69:3.11} Los primeros comerciantes fueron las mujeres; se las empleaba como espías, ejerciendo el comercio como actividad suplementaria. El comercio se expandió enseguida y las mujeres actuaron como intermediarias ---como corredoras. Luego surgió la clase mercantil, que cobraba una comisión, un beneficio, por sus servicios. El crecimiento del trueque entre los grupos dio nacimiento al comercio, y al intercambio de las mercancías le siguió el intercambio de la mano de obra especializada.

\section*{4. Los principios del comercio}
\par
%\textsuperscript{(775.1)}
\textsuperscript{69:4.1} De la misma manera que el matrimonio por contrato siguió al matrimonio por captura, el comercio de trueque siguió a las incautaciones de los ataques por sorpresa. Pero transcurrió un largo período de piratería entre las primeras prácticas del trueque silencioso y el comercio posterior realizado con métodos de intercambio modernos.

\par
%\textsuperscript{(775.2)}
\textsuperscript{69:4.2} Los primeros trueques estuvieron dirigidos por comerciantes armados que dejaban sus mercancías en un sitio neutral. Las mujeres mantuvieron los primeros mercados; fueron las primeras comerciantes, y esto se produjo porque eran ellas las que llevaban las cargas; los hombres eran guerreros. Los mostradores de venta aparecieron muy pronto; se trataba de unos muros lo bastante anchos como para impedir que los comerciantes se alcanzaran con sus armas.

\par
%\textsuperscript{(775.3)}
\textsuperscript{69:4.3} Se utilizaba un fetiche para montar la guardia en los depósitos de mercancías destinados al trueque silencioso. Estos lugares de mercado estaban protegidos contra el robo; no se podía retirar nada, a menos que se hiciera mediante la permuta o la compra; con un fetiche de guardia, las mercancías siempre estaban a salvo. Los primeros comerciantes eran escrupulosamente honrados dentro de sus propias tribus, pero consideraban totalmente correcto engañar a los extraños que venían de lejos. Incluso los primeros hebreos admitían la utilización de un código ético distinto para sus transacciones con los gentiles.

\par
%\textsuperscript{(775.4)}
\textsuperscript{69:4.4} El trueque silencioso continuó existiendo durante miles de años, antes de que los hombres aceptaran reunirse sin armas en la plaza sagrada del mercado. Estas mismas plazas de los mercados se convirtieron en los primeros refugios, y en algunas regiones se conocieron más tarde como <<ciudades de refugio>>\footnote{\textit{Ciudades de refugio}: 1 Cr 6:57,67; Nm 35:6,11,14; Jos 20:2; Jos 21:13,21,27; Jos 21:32,38.}. Cualquier fugitivo que alcanzara la plaza del mercado estaba a salvo y protegido contra todo ataque.

\par
%\textsuperscript{(775.5)}
\textsuperscript{69:4.5} Los primeros pesos que se utilizaron fueron los granos de trigo y de otros cereales. El primer medio de cambio fue un pescado o una cabra. Más tarde, la vaca se convirtió en una unidad de trueque.

\par
%\textsuperscript{(775.6)}
\textsuperscript{69:4.6} La escritura moderna tuvo su origen en los primeros registros comerciales; la primera literatura del hombre fue un documento de propaganda comercial, una publicidad para la sal. Muchas guerras primitivas se libraron por la posesión de los depósitos naturales, tales como el sílex, la sal y los metales. El primer tratado oficial entre tribus estuvo relacionado con la explotación en común de un depósito de sal. Estos lugares protegidos por un tratado proporcionaron la oportunidad, a las diversas tribus, de entremezclarse e intercambiar sus ideas de manera amistosa y pacífica.

\par
%\textsuperscript{(775.7)}
\textsuperscript{69:4.7} La escritura progresó desde las etapas del <<bastón mensajero>>, las cuerdas anudadas, la pictografía, los jeroglíficos y los cinturones de cuentas de concha, hasta llegar a los primeros alfabetos simbólicos. El envío de los mensajes evolucionó desde las señales de humo primitivas hasta los corredores, los jinetes, los ferrocarriles y los aviones, así como el telégrafo, el teléfono y la comunicación radiofónica.

\par
%\textsuperscript{(775.8)}
\textsuperscript{69:4.8} Los comerciantes de la antig\"uedad llevaron nuevas ideas y métodos mejores por todo el mundo habitado. El comercio, unido a la aventura, condujo a la exploración y al descubrimiento. Y todo esto dio nacimiento al transporte. El comercio ha sido el gran civilizador al estimular la fecundación cruzada de las culturas.

\section*{5. Los principios del capital}
\par
%\textsuperscript{(775.9)}
\textsuperscript{69:5.1} El capital es un trabajo realizado, al que se renuncia en el presente en favor del futuro. Los ahorros representan una forma de seguridad para poder mantenerse y sobrevivir. La acumulación de la comida desarrolló el autocontrol y creó los primeros problemas del capital y del trabajo. El hombre que tenía comida, a condición de que pudiera protegerla contra los ladrones, poseía una clara ventaja sobre el que no la tenía.

\par
%\textsuperscript{(775.10)}
\textsuperscript{69:5.2} El banquero primitivo era el hombre más valiente de la tribu. Guardaba en depósito los tesoros del grupo y todo el clan defendía su choza en caso de ataque. De esta manera, la acumulación del capital individual y de la riqueza colectiva condujo inmediatamente a la organización militar. Al principio, estas precauciones estaban destinadas a defender la propiedad contra los invasores exteriores; pero más tarde se estableció la costumbre de mantener entrenada a la organización militar efectuando ataques por sorpresa contra la propiedad y la riqueza de las tribus vecinas.

\par
%\textsuperscript{(776.1)}
\textsuperscript{69:5.3} Los impulsos fundamentales que condujeron a la acumulación del capital fueron los siguientes:

\par
%\textsuperscript{(776.2)}
\textsuperscript{69:5.4} 1. \textit{El hambre ---asociada a la previsión}. Guardar y conservar la comida significaba poder y comodidad para aquellos que tenían la suficiente \textit{previsión} como para precaverse así contra las necesidades futuras. El almacenamiento de los alimentos era un seguro adecuado contra el hambre y los desastres. Todo el conjunto de las costumbres primitivas estaba realmente diseñado para ayudar al hombre a subordinar el presente al futuro.

\par
%\textsuperscript{(776.3)}
\textsuperscript{69:5.5} 2. \textit{El amor a la familia} ---el deseo de asegurar sus necesidades. El capital representa el ahorro de unos bienes a pesar de la presión de las necesidades del presente, a fin de asegurarse contra las exigencias del futuro. Una parte de estas necesidades futuras puede estar relacionada con la posteridad del interesado.

\par
%\textsuperscript{(776.4)}
\textsuperscript{69:5.6} 3. \textit{La vanidad} ---el vivo deseo de mostrar la acumulación de sus bienes. La ropa suplementaria fue uno de los primeros signos de distinción. La vanidad de coleccionar atrajo pronto el orgullo del hombre.

\par
%\textsuperscript{(776.5)}
\textsuperscript{69:5.7} 4. \textit{La posición social} ---el ansia de comprar el prestigio social y político. Pronto surgió una nobleza comercializada, y el ser admitido en ella dependía de la prestación de algún servicio especial a la realeza, o simplemente se concedía a cambio de dinero.

\par
%\textsuperscript{(776.6)}
\textsuperscript{69:5.8} 5. \textit{El poder} ---el ansia de ser el amo. Prestar tesoros se empleó como un medio de esclavizar, pues en aquellos tiempos antiguos el interés de los préstamos era del cien por cien al año. Los prestamistas se convertían en reyes al crearse un ejército permanente de deudores. Los criados hipotecados se encontraron entre las primeras formas de propiedad que se acumularon, y en la antig\"uedad, la esclavitud ocasionada por las deudas se extendía incluso hasta tener autoridad sobre el cuerpo después de la muerte.

\par
%\textsuperscript{(776.7)}
\textsuperscript{69:5.9} 6. \textit{El miedo a los fantasmas de los muertos} ---los honorarios que se pagaban a los sacerdotes para protegerse. Los hombres empezaron pronto a hacer regalos fúnebres a los sacerdotes con la idea de que estos bienes se utilizaran para facilitar su progreso en la próxima vida. Los sacerdotes se volvieron así muy ricos; fueron los principales capitalistas antiguos.

\par
%\textsuperscript{(776.8)}
\textsuperscript{69:5.10} 7. \textit{El impulso sexual} ---el deseo de comprar una o varias esposas. La primera forma de comercio entre los hombres fue el intercambio de mujeres; éste comenzó mucho tiempo antes que el comercio de los caballos. Pero el trueque de esclavos por motivos sexuales nunca ha hecho progresar a la sociedad; este tráfico era y es una verg\"uenza racial, porque obstaculizó el desarrollo de la vida familiar y, al mismo tiempo, contaminó la aptitud biológica de los pueblos superiores.

\par
%\textsuperscript{(776.9)}
\textsuperscript{69:5.11} 8. \textit{Las numerosas formas de placeres personales}. Algunos buscaron las riquezas porque conferían poder; otros trabajaron duro para conseguir propiedades porque significaban una vida fácil. Los hombres primitivos (y otros después de ellos) tendían a derrochar sus recursos en lujos. Las bebidas alcohólicas y las drogas intrigaban a las razas primitivas.

\par
%\textsuperscript{(776.10)}
\textsuperscript{69:5.12} A medida que se desarrollaba la civilización, los hombres encontraron nuevos motivos para ahorrar; al hambre original se agregaron rápidamente otras nuevas necesidades. La pobreza se volvió tan detestable que se suponía que los ricos eran los únicos que iban directamente al cielo después de morir. La propiedad se volvió tan apreciada que bastaba dar un festín presuntuoso para borrar el deshonor de un nombre.

\par
%\textsuperscript{(777.1)}
\textsuperscript{69:5.13} La acumulación de las riquezas se convirtió pronto en el símbolo de la distinción social. En algunas tribus, los individuos acumulaban propiedades durante años únicamente para causar impresión quemándolas algún día de fiesta o repartiéndolas gratuitamente entre los miembros de su tribu. Esto los convertía en grandes hombres. Incluso los pueblos modernos se deleitan distribuyendo pródigamente los regalos de Navidad, mientras que los hombres ricos hacen donaciones a las grandes instituciones filantrópicas y educativas. Las técnicas del hombre varían, pero su naturaleza no cambia mucho.

\par
%\textsuperscript{(777.2)}
\textsuperscript{69:5.14} Pero es justo indicar que muchos hombres ricos de la antig\"uedad distribuyeron una gran parte de su fortuna a causa del miedo a que los mataran los que codiciaban sus tesoros. Los ricos sacrificaban generalmente docenas de esclavos para demostrar su desdén por las riquezas.

\par
%\textsuperscript{(777.3)}
\textsuperscript{69:5.15} Aunque el capital ha contribuido a liberar al hombre, ha complicado enormemente su organización social e industrial. El empleo abusivo del capital por parte de unos capitalistas injustos no invalida el hecho de que es la base de la sociedad industrial moderna. Gracias al capital y a los inventos, la generación actual disfruta de un alto grado de libertad que nunca se había alcanzado anteriormente en la Tierra. Esto lo hacemos constar como un hecho, y no para justificar los numerosos abusos que los custodios irreflexivos y egoístas hacen del capital.

\section*{6. El fuego en relación con la civilización}
\par
%\textsuperscript{(777.4)}
\textsuperscript{69:6.1} La sociedad primitiva con sus cuatro divisiones ---industrial, reguladora, religiosa y militar--- nació gracias al papel decisivo que jugaron el fuego, los animales, los esclavos y la propiedad.

\par
%\textsuperscript{(777.5)}
\textsuperscript{69:6.2} Saber encender el fuego separó para siempre, de un solo salto, al hombre del animal; es el invento o descubrimiento humano fundamental. El fuego permitió al hombre permanecer en el suelo durante la noche ya que todos los animales le temen. El fuego estimuló las relaciones sociales a la caída de la tarde; no solamente protegía del frío y de las bestias feroces, sino que también se empleaba como protección contra los fantasmas. Al principio se utilizaba más para alumbrar que para calentar; muchas tribus atrasadas se niegan a dormir a menos que esté ardiendo una llama durante toda la noche.

\par
%\textsuperscript{(777.6)}
\textsuperscript{69:6.3} El fuego fue un gran civilizador, proporcionando al hombre el primer medio para ser altruista sin perder nada, pues le permitía ofrecer unas brasas ardientes a un vecino sin despojarse de nada. El fuego de la casa, que era cuidado por la madre o la hija mayor, fue el primer educador, pues necesitaba vigilancia y seriedad. El hogar primitivo no era un edificio, sino que la familia se reunía alrededor del fuego, del hogar familiar. Cuando un hijo fundaba un nuevo hogar, se llevaba una tea del hogar familiar.

\par
%\textsuperscript{(777.7)}
\textsuperscript{69:6.4} Aunque Andón, el descubridor del fuego, evitó tratarlo como si fuera un objeto de adoración, muchos de sus descendientes consideraron la llama como un fetiche o un espíritu\footnote{\textit{El fetiche del fuego}: Ex 13:21-22.}. No lograron cosechar los beneficios higiénicos del fuego porque no querían quemar los residuos. El hombre primitivo tenía miedo del fuego y siempre procuraba mantenerlo de buen humor, de ahí que lo rociara de incienso. Los antiguos no hubieran escupido en el fuego bajo ningún concepto, ni tampoco hubieran pasado nunca entre una persona y un fuego encendido. La humanidad primitiva tenía por sagrados incluso las piritas de hierro y los pedernales que se utilizaban para encender el fuego.

\par
%\textsuperscript{(777.8)}
\textsuperscript{69:6.5} Apagar una llama era un pecado; si una choza se incendiaba, se dejaba que se quemara. Los fuegos de los templos y de los santuarios eran sagrados\footnote{\textit{Fuego sagrado}: 2 Mac 1:18-22; Lv 6:12-13.} y nunca se permitía que se apagaran, salvo que existía la costumbre de encender nuevos fuegos cada año o después de alguna calamidad. Las mujeres fueron escogidas como sacerdotisas porque eran las que custodiaban los fuegos caseros.

\par
%\textsuperscript{(778.1)}
\textsuperscript{69:6.6} Los primeros mitos sobre la manera en que el fuego descendió de los dioses\footnote{\textit{Fuego de los dioses}: Gn 19:24; 1 Re 18:38; Lv 9:24; 10:1-2.} nacieron de la observación de los incendios provocados por los rayos. Estas ideas sobre el origen sobrenatural del fuego condujeron directamente a su adoración, y la adoración del fuego llevó a la costumbre de <<pasar por el fuego>>\footnote{\textit{Pasar por el fuego}: 2 Re 16:3.}, una práctica que se conservó hasta los tiempos de Moisés. Todavía persiste la idea de que se pasa a través del fuego después de la muerte. El mito del fuego fue un gran vínculo en los tiempos primitivos, y aún perdura todavía en el simbolismo de los parsis.

\par
%\textsuperscript{(778.2)}
\textsuperscript{69:6.7} El fuego condujo a la cocción, y <<come crudo>> se convirtió en una expresión desdeñosa. La cocción disminuyó el gasto de energía vital necesaria para digerir la comida, y dejó así al hombre primitivo algunas fuerzas para cultivarse socialmente; al mismo tiempo, la cría de ganado redujo el esfuerzo necesario para conseguir alimentos, y proporcionó tiempo para las actividades sociales.

\par
%\textsuperscript{(778.3)}
\textsuperscript{69:6.8} Se debe recordar que el fuego abrió las puertas de la metalurgia y condujo al descubrimiento posterior de la energía del vapor y al empleo actual de la electricidad.

\section*{7. La utilización de los animales}
\par
%\textsuperscript{(778.4)}
\textsuperscript{69:7.1} Al principio, todo el reino animal era enemigo del hombre; los seres humanos tuvieron que aprender a protegerse de las bestias. El hombre empezó primero a comerse a los animales, pero más tarde aprendió a domesticarlos y a ponerlos a su servicio.

\par
%\textsuperscript{(778.5)}
\textsuperscript{69:7.2} La domesticación de los animales se produjo por casualidad. El salvaje cazaba las manadas poco más o menos como los indios norteamericanos cazaban el bisonte. Rodeaban la manada y podían mantener así el control de los animales, pudiendo matarlos entonces a medida que necesitaban comida. Más tarde construyeron corrales y capturaron manadas enteras.

\par
%\textsuperscript{(778.6)}
\textsuperscript{69:7.3} Fue fácil domar a algunos animales, pero muchos de ellos, al igual que el elefante, no se reproducían en cautividad. Posteriormente se descubrió además que algunas especies de animales se sometían a la presencia del hombre y se reproducían en cautividad. La domesticación de los animales se desarrolló así mediante la cría selectiva, un arte que ha hecho grandes progresos desde los tiempos de Dalamatia.

\par
%\textsuperscript{(778.7)}
\textsuperscript{69:7.4} El perro fue el primer animal que se domesticó, y la difícil experiencia de domarlo empezó cuando cierto perro, después de seguir a un cazador durante todo el día, lo acompañó efectivamente hasta su casa. Durante miles de años, los perros se utilizaron como alimento, para la caza y el transporte, y como animales de compañía. Al principio los perros se limitaban a aullar, pero más tarde aprendieron a ladrar. El agudo sentido del olfato del perro condujo a la idea de que podía ver los espíritus, y así es como surgieron los cultos de los perros fetiches. El empleo de perros guardianes permitió por primera vez que todo el clan pudiera dormir por la noche. Entonces se estableció la costumbre de emplear los perros guardianes para proteger el hogar contra los espíritus, así como contra los enemigos materiales. Cuando el perro ladraba, algún hombre o alguna bestia se acercaba, pero cuando aullaba, los espíritus andaban cerca. Incluso hoy en día, mucha gente cree todavía que el aullido de un perro por la noche es un presagio de muerte.

\par
%\textsuperscript{(778.8)}
\textsuperscript{69:7.5} Cuando el hombre era cazador, era bastante amable con la mujer, pero después de la domesticación de los animales, unido a la confusión ocasionada por Caligastia, muchas tribus trataron a sus mujeres de manera vergonzosa. Las trataron en conjunto de manera muy similar a como trataban a sus animales. El tratamiento brutal que los hombres han infligido a las mujeres constituye uno de los capítulos más sombríos de la historia humana.

\section*{8. La esclavitud como factor de la civilización}
\par
%\textsuperscript{(778.9)}
\textsuperscript{69:8.1} El hombre primitivo no dudó nunca en esclavizar a sus semejantes. La mujer fue la primera esclava, una esclava familiar. Los pastores esclavizaron a sus mujeres como si fueran unas compañeras sexuales inferiores. Este tipo de esclavitud sexual surgió directamente del hecho de que el hombre dependió cada vez menos de la mujer.

\par
%\textsuperscript{(779.1)}
\textsuperscript{69:8.2} No hace mucho tiempo, la esclavitud era el destino de los prisioneros de guerra que se negaban a aceptar la religión de sus conquistadores. En épocas anteriores, los prisioneros habían sido comidos, torturados hasta morir, obligados a luchar entre sí, sacrificados a los espíritus o esclavizados. La esclavitud fue un gran progreso sobre las masacres y el canibalismo.

\par
%\textsuperscript{(779.2)}
\textsuperscript{69:8.3} La esclavitud fue un paso hacia adelante en el tratamiento más clemente de los prisioneros de guerra. La emboscada de Hai\footnote{\textit{Emboscada de Hai}: Jos 8:1-29.}, con la matanza total de hombres, mujeres y niños, en la que sólo se salvó el rey para satisfacer la vanidad del vencedor, es una imagen fiel de las masacres bárbaras que practicaban incluso los pueblos supuestamente civilizados. El ataque por sorpresa a Og\footnote{\textit{Ataque por sorpresa a Og}: Dt 3:1-7.}, el rey de Basan, fue igual de brutal e impresionante. Los hebreos <<destruían por completo>>\footnote{\textit{Los hebreos ``destruían por completo''}: Jer 50:21; Nm 21:2-3; Dt 12:2; 20:17; Jos 2:10; Jue 21:11; 1 Sam 15:3,9,18.} a sus enemigos, y se apoderaban de todos sus bienes como botín. Imponían un tributo a todas las ciudades, so pena de <<destruir a todos los varones>>\footnote{\textit{Destruir a todos los varones}: Nm 31:7.}. Pero muchas tribus de la misma época, que tenían menos egoísmo tribal, habían empezado a practicar desde hacía mucho tiempo la adopción de los cautivos superiores.

\par
%\textsuperscript{(779.3)}
\textsuperscript{69:8.4} Los cazadores, al igual que los hombres rojos americanos, no practicaban la esclavitud. O bien adoptaban a sus cautivos, o los mataban. La esclavitud no estaba extendida entre los pueblos pastoriles porque necesitaban poca mano de obra. Durante las guerras, los pastores tenían la costumbre de matar a todos los hombres cautivos, y sólo se llevaban como esclavos a las mujeres y los niños\footnote{\textit{Capturar mujeres}: Nm 31:9,15-18; Dt 20:14.}. El código de Moisés contenía instrucciones específicas para que estas cautivas se convirtieran en esposas\footnote{\textit{Reglas para esposas cautivas}: Dt 21:10-14.}. Si no eran satisfactorias, podían echarlas, pero a los hebreos no se les permitía vender como esclavas a estas consortes rechazadas ---al menos fue un progreso en la civilización. Aunque las normas sociales de los hebreos eran rudimentarias, estaban muy por encima de las de las tribus circundantes.

\par
%\textsuperscript{(779.4)}
\textsuperscript{69:8.5} Los pastores fueron los primeros capitalistas; sus rebaños representaban un capital, y vivían de los intereses ---de los incrementos naturales. Estaban poco dispuestos a confiar esta riqueza a los esclavos o a las mujeres. Pero más adelante hicieron prisioneros varones y los forzaron a cultivar el suelo. Éste es el origen primitivo de la servidumbre ---el hombre atado a la tierra. A los africanos se les podía enseñar fácilmente a cultivar la tierra, y por eso se convirtieron en la gran raza esclava.

\par
%\textsuperscript{(779.5)}
\textsuperscript{69:8.6} La esclavitud fue un eslabón indispensable en la cadena de la civilización humana. Fue el puente por el que la sociedad pasó del caos y la indolencia al orden y a las actividades civilizadas; obligó a los pueblos atrasados y perezosos a trabajar y a proporcionar así a sus superiores la riqueza y el tiempo libre necesarios para el progreso social.

\par
%\textsuperscript{(779.6)}
\textsuperscript{69:8.7} La institución de la esclavitud obligó al hombre a inventar el mecanismo regulador de la sociedad primitiva; dio nacimiento a los inicios del gobierno. La esclavitud necesita una fuerte reglamentación, y desapareció prácticamente durante la Edad Media europea porque los señores feudales no podían controlar a los esclavos. Las tribus atrasadas de los tiempos antiguos, al igual que los aborígenes australianos de hoy, nunca tuvieron esclavos.

\par
%\textsuperscript{(779.7)}
\textsuperscript{69:8.8} Es verdad que la esclavitud era opresiva, pero en las escuelas de la opresión es donde el hombre aprendió la diligencia. Los esclavos compartieron finalmente las ventajas de una sociedad superior que habían ayudado a crear de manera tan involuntaria. La esclavitud crea una organización de cultura y de logros sociales, pero pronto ataca insidiosamente a la sociedad desde el interior como la enfermedad social destructiva más grave de todas.

\par
%\textsuperscript{(779.8)}
\textsuperscript{69:8.9} Los inventos mecánicos modernos han dejado obsoleto al esclavo. La esclavitud, al igual que la poligamia, está desapareciendo porque no es rentable. Pero siempre ha sido desastroso liberar repentinamente a una gran cantidad de esclavos; su emancipación paulatina origina menos dificultades.

\par
%\textsuperscript{(780.1)}
\textsuperscript{69:8.10} Hoy día los hombres ya no son unos esclavos sociales, pero miles de ellos permiten que la ambición los haga esclavos de las deudas. La esclavitud involuntaria ha cedido el paso a una forma nueva y mejorada de servidumbre industrial modificada.

\par
%\textsuperscript{(780.2)}
\textsuperscript{69:8.11} Aunque el ideal de la sociedad sea la libertad universal, la ociosidad no debería tolerarse nunca. Todas las personas sanas deberían ser obligadas a realizar una cantidad de trabajo que al menos les permita vivir.

\par
%\textsuperscript{(780.3)}
\textsuperscript{69:8.12} La sociedad moderna está dando marcha atrás. La esclavitud casi ha desaparecido; los animales domésticos se están extinguiendo. La civilización está volviendo al fuego ---al mundo inorgánico--- en busca de energía. El hombre salió del estado salvaje por medio del fuego, los animales y la esclavitud; hoy vuelve hacia atrás, descartando la ayuda de los esclavos y la asistencia de los animales, e intentando arrebatar nuevos secretos y nuevas fuentes de riqueza y energía a los depósitos elementales de la naturaleza.

\section*{9. La propiedad privada}
\par
%\textsuperscript{(780.4)}
\textsuperscript{69:9.1} Aunque la sociedad primitiva era prácticamente comunal, el hombre primitivo no practicaba las doctrinas modernas del comunismo. El comunismo de aquellos primeros tiempos no era una mera teoría o una doctrina social; era una adaptación automática simple y práctica. Aquel comunismo impedía el pauperismo y la miseria; la mendicidad y la prostitución eran casi desconocidas en aquellas tribus antiguas.

\par
%\textsuperscript{(780.5)}
\textsuperscript{69:9.2} El comunismo primitivo no niveló especialmente a los hombres por abajo, ni tampoco ensalzó a la mediocridad, pero sí dio un gran valor a la inactividad y a la pereza, y ahogó la diligencia y destruyó la ambición. El comunismo fue un andamiaje indispensable para el crecimiento de la sociedad primitiva, pero cedió el paso a la evolución de un orden social más elevado porque iba en contra de cuatro poderosas inclinaciones humanas:

\par
%\textsuperscript{(780.6)}
\textsuperscript{69:9.3} 1. \textit{La familia}. El hombre no solamente anhela acumular propiedades, sino que desea legar sus bienes de equipo a sus descendientes. Pero en la sociedad comunal primitiva, el capital que un hombre dejaba a su muerte era consumido inmediatamente o bien se repartía entre los miembros del grupo. La propiedad no se heredaba ---el impuesto sobre la herencia era del cien por cien. Las costumbres posteriores de acumular capitales y heredar propiedades representaron un progreso social indudable. Y esto es cierto a pesar de los grandes abusos posteriores que han acompañado al mal uso del capital.

\par
%\textsuperscript{(780.7)}
\textsuperscript{69:9.4} 2. \textit{Las tendencias religiosas}. El hombre primitivo también quería conservar sus propiedades como base para empezar su vida en la siguiente existencia. Este motivo explica por qué existió durante tanto tiempo la costumbre de enterrar con el difunto sus efectos personales. Los antiguos creían que sólo los ricos sobrevivían a la muerte con algún tipo de placer y dignidad inmediatos. Los instructores de la religión revelada, y en particular los educadores cristianos, fueron los primeros que proclamaron que los pobres podían salvarse en las mismas condiciones que los ricos.

\par
%\textsuperscript{(780.8)}
\textsuperscript{69:9.5} 3. \textit{El deseo de libertad y de tiempo libre}. En los primeros tiempos de la evolución social, el reparto de los ingresos individuales entre los miembros del grupo era prácticamente una forma de esclavitud; el trabajador se convertía en el esclavo del holgazán. La debilidad suicida de este comunismo fue que el imprevisor vivía habitualmente a expensas del ahorrativo. Incluso en los tiempos modernos, los imprevisores cuentan con el Estado (con los contribuyentes ahorrativos) para que cuide de ellos. Los que no tienen ningún capital esperan todavía que los que lo tienen les den de comer.

\par
%\textsuperscript{(780.9)}
\textsuperscript{69:9.6} 4. \textit{La necesidad de seguridad y de poder}. El comunismo se destruyó finalmente debido a las estratagemas engañosas de los individuos prósperos y progresistas, que recurrieron a diversos subterfugios para evitar convertirse en los esclavos de los holgazanes indolentes de sus tribus. Pero al principio todo atesoramiento se hacía en secreto; la inseguridad que reinaba en los tiempos primitivos impedía que se acumulara abiertamente el capital. Incluso en una época más tardía fue sumamente peligroso amasar demasiadas riquezas; el rey no dejaría de inventar alguna acusación para confiscar las propiedades de un hombre rico; cuando un rico moría, los funerales se retrasaban hasta que la familia donaba una gran suma para el bienestar público o al rey, un impuesto sobre la herencia.

\par
%\textsuperscript{(781.1)}
\textsuperscript{69:9.7} En los tiempos más primitivos, las mujeres eran propiedad de la comunidad y la madre dominaba la familia. Los caciques primitivos poseían todas las tierras y eran propietarios de todas las mujeres; para casarse se necesitaba el consentimiento del jefe de la tribu. Cuando el comunismo desapareció, las mujeres se volvieron propiedad individual, y el padre asumió gradualmente el poder doméstico. Así es como nació el hogar, y las costumbres polígamas imperantes fueron reemplazadas paulatinamente por la monogamia. (La poligamia es la supervivencia del concepto de esclavitud femenina en el matrimonio. La monogamia es el ideal, libre de toda esclavitud, de la asociación incomparable entre un hombre y una mujer en la delicada empresa de formar un hogar, criar a los hijos, cultivarse mutuamente y mejorarse.)

\par
%\textsuperscript{(781.2)}
\textsuperscript{69:9.8} Al principio, todos los bienes, incluidas las herramientas y las armas, eran propiedad común de la tribu. La propiedad privada consistió en primer lugar en todas las cosas que había tocado una persona. Si un extraño bebía en una copa, desde ese momento en adelante la copa era suya. Más adelante, todo lugar donde se había derramado sangre se convirtió en la propiedad del herido o de su grupo.

\par
%\textsuperscript{(781.3)}
\textsuperscript{69:9.9} La propiedad privada se respetó así en un principio porque se suponía que estaba cargada con alguna parte de la personalidad de su dueño. La honradez con respecto a la propiedad descansaba sin peligro sobre este tipo de superstición; no se necesitaba ninguna policía para proteger los efectos personales. No había robos en el interior del grupo, pero los hombres no dudaban en apropiarse de los bienes de otras tribus. Las relaciones con la propiedad no terminaban con la muerte; al principio, los efectos personales se quemaban, luego se enterraban con el difunto, y más tarde los heredaban la familia sobreviviente o la tribu.

\par
%\textsuperscript{(781.4)}
\textsuperscript{69:9.10} Los efectos personales de tipo ornamental tuvieron su origen en el uso de los amuletos. La vanidad, unida al miedo de los fantasmas, condujeron al hombre primitivo a resistirse a todos los intentos por liberarlo de sus amuletos favoritos, ya que estas posesiones las valoraba por encima de sus necesidades vitales.

\par
%\textsuperscript{(781.5)}
\textsuperscript{69:9.11} Una de las primeras propiedades del hombre fue el lugar donde dormía. Más tarde, el domicilio familiar era asignado por el jefe de la tribu, el cual tenía en fideicomiso todos los bienes raíces del grupo. Luego, el lugar donde estaba un fuego confería su propiedad; y más tarde aún, un pozo constituyó un título de propiedad sobre las tierras adyacentes\footnote{\textit{Pozos y tierras adyacentes}: Gn 21:25-30; 26:19-22.}.

\par
%\textsuperscript{(781.6)}
\textsuperscript{69:9.12} Los abrevaderos y los pozos figuraron entre las primeras posesiones privadas. Se utilizaron todas las prácticas fetichistas para proteger los abrevaderos, los pozos, los árboles, los cultivos y la miel. Cuando desapareció la fe en los fetiches, se desarrollaron leyes para proteger las pertenencias privadas. Pero las leyes de la caza, el derecho a cazar, fueron muy anteriores a las leyes sobre los bienes raíces. El hombre rojo americano nunca entendió la propiedad privada de las tierras; no pudo comprender el punto de vista del hombre blanco.

\par
%\textsuperscript{(781.7)}
\textsuperscript{69:9.13} La propiedad privada pronto llevó la marca de la insignia familiar, y éste es el origen lejano de los emblemas familiares. Los bienes raíces también se podían poner bajo la custodia de los espíritus. Los sacerdotes <<consagraban>> un terreno, que luego quedaba bajo la protección de los tabúes mágicos erigidos sobre él. Se decía que los propietarios de estos terrenos poseían una <<escritura de propiedad sacerdotal>>\footnote{\textit{Respeto por las marcas territoriales}: Pr 22:28; 23:10; Dt 19:14.}. Los hebreos tenían un gran respeto por estas marcas familiares: <<Maldito sea el que quite la marca de su vecino>>\footnote{\textit{Maldiciones si se destruían marcas territoriales}: Dt 27:17.}. Estos indicadores de piedra llevaban las iniciales del sacerdote. Incluso los árboles se convertían en propiedad privada cuando se les ponían unas iniciales.

\par
%\textsuperscript{(782.1)}
\textsuperscript{69:9.14} En los tiempos primitivos, sólo los cultivos eran privados, pero las cosechas sucesivas conferían un derecho; la agricultura fue así la génesis de la propiedad privada de las tierras. Al principio los individuos sólo recibían un arrendamiento de por vida; a su muerte, la tierra volvía a ser de la tribu. Las primeras titularidades de tierras que las tribus concedieron a los individuos fueron las tumbas ---los cementerios familiares. En tiempos posteriores, la tierra perteneció a quien la cercara. Pero las ciudades siempre reservaron cierta cantidad de tierras para pastos y para utilizarlas en caso de asedio; estos <<ejidos>> representan la supervivencia de las formas primitivas de propiedad colectiva.

\par
%\textsuperscript{(782.2)}
\textsuperscript{69:9.15} Con el tiempo, el Estado asignó la propiedad a los individuos, reservándose el derecho de cobrar impuestos. Una vez que habían asegurado sus títulos, los propietarios podían cobrar alquileres, y la tierra se convirtió en una fuente de ingresos ---en un capital. Finalmente la tierra se volvió realmente negociable, con ventas, traspasos, hipotecas y ejecuciones hipotecarias.

\par
%\textsuperscript{(782.3)}
\textsuperscript{69:9.16} La propiedad privada acrecentó la libertad y aumentó la estabilidad; pero la propiedad privada de la tierra sólo recibió la aprobación social después de que el control y la dirección comunales hubieron fracasado, a lo cual pronto le siguió una sucesión de esclavos, de siervos y de clases sociales sin tierras. Pero el perfeccionamiento de las máquinas está liberando gradualmente al hombre del duro trabajo servil.

\par
%\textsuperscript{(782.4)}
\textsuperscript{69:9.17} El derecho a la propiedad no es absoluto; es puramente social. Pero todos los gobiernos, las leyes, el orden, los derechos civiles, las libertades sociales, las convenciones, la paz y la felicidad que disfrutan los pueblos modernos se han desarrollado alrededor de la propiedad privada de los bienes.

\par
%\textsuperscript{(782.5)}
\textsuperscript{69:9.18} El orden social actual no es necesariamente justo ---no es ni divino ni sagrado--- pero la humanidad hará bien en proceder lentamente a efectuar sus cambios. El sistema que tenéis es muy superior a todos los que conocieron vuestros antepasados. Cuando cambiéis el orden social, aseguraos de que lo cambiáis por otro mejor. No os dejéis persuadir de que hay que experimentar con las fórmulas desechadas por vuestros antecesores ¡Avanzad, no retrocedáis! ¡Dejad que continúe la evolución! ¡No deis un paso atrás!

\par
%\textsuperscript{(782.6)}
\textsuperscript{69:9.19} [Presentado por un Melquisedek de Nebadon.]


\chapter{Documento 70. La evolución del gobierno humano}
\par
%\textsuperscript{(783.1)}
\textsuperscript{70:0.1} EN cuanto el hombre resolvió parcialmente el problema de ganarse la vida, tuvo que hacer frente a la tarea de reglamentar las relaciones humanas. El desarrollo de la industria exigía unas leyes, orden y un ajuste social; la propiedad privada necesitaba un gobierno.

\par
%\textsuperscript{(783.2)}
\textsuperscript{70:0.2} En un mundo evolutivo, los antagonismos son naturales; la paz sólo se consigue mediante algún tipo de sistema social regulador. La reglamentación social es inseparable de la organización social; la asociación implica alguna autoridad que controle. El gobierno obliga a coordinar los antagonismos entre las tribus, los clanes, las familias y los individuos.

\par
%\textsuperscript{(783.3)}
\textsuperscript{70:0.3} El gobierno es un desarrollo inconsciente; evoluciona a base de aciertos y errores. Posee un valor de supervivencia, y por esta razón se vuelve tradicional. La anarquía aumentaba la miseria; por eso los gobiernos, la ley y el orden relativos, surgieron lentamente o están surgiendo. Las exigencias coactivas de la lucha por la existencia empujaron literalmente a la raza humana por el camino progresivo de la civilización.

\section*{1. La génesis de la guerra}
\par
%\textsuperscript{(783.4)}
\textsuperscript{70:1.1} La guerra es el estado y la herencia naturales del hombre en evolución; la paz es la vara social que mide el progreso de la civilización. Antes de que las razas progresivas se socializaran parcialmente, el hombre era enormemente individualista, extremadamente desconfiado e increíblemente pendenciero. La violencia es la ley de la naturaleza, la hostilidad es la reacción automática de los hijos de la naturaleza, mientras que la guerra no es más que estas mismas actividades pero realizadas de manera colectiva. En todas las circunstancias en que las complicaciones del progreso de la sociedad ponen en tensión la estructura de la civilización, siempre se produce una vuelta inmediata y ruinosa a estos métodos primitivos para ajustar, por medio de la violencia, las irritaciones que se producen en las interasociaciones humanas.

\par
%\textsuperscript{(783.5)}
\textsuperscript{70:1.2} La guerra es una reacción animal ante los malentendidos y las irritaciones; la paz acompaña a la solución civilizada de todos estos problemas y dificultades. Las razas sangiks, así como los adamitas y los noditas degenerados posteriores, eran todos belicosos. A los andonitas se les enseñó pronto la regla de oro, y hoy todavía sus descendientes esquimales viven en gran parte siguiendo este código; las costumbres están muy arraigadas entre ellos y se encuentran relativamente libres de antagonismos violentos.

\par
%\textsuperscript{(783.6)}
\textsuperscript{70:1.3} Andón enseñó a sus hijos a resolver sus disputas golpeando cada uno de ellos un árbol con un palo, mientras maldecían el árbol; el primero que rompía el palo era el vencedor. Los andonitas posteriores tenían la costumbre de arreglar sus controversias organizando un espectáculo público, durante el cual los adversarios se reían del otro y se ridiculizaban mutuamente, mientras que el público decidía con sus aplausos quién era el ganador.

\par
%\textsuperscript{(783.7)}
\textsuperscript{70:1.4} Pero un fenómeno como la guerra no podía existir hasta que la sociedad hubiera evolucionado lo suficiente como para experimentar auténticos períodos de paz y aprobar las prácticas bélicas. El concepto mismo de la guerra implica cierto grado de organización.

\par
%\textsuperscript{(784.1)}
\textsuperscript{70:1.5} Con la aparición de las agrupaciones sociales, las irritaciones personales empezaron a sumergirse en los sentimientos colectivos, lo cual fomentó la tranquilidad dentro de las tribus, pero a costa de la paz entre ellas. Así pues, la paz se disfrutó primero dentro del grupo interno, o tribu, que siempre detestaba y odiaba al grupo externo, a los extranjeros. El hombre primitivo consideraba que derramar sangre extranjera era una virtud.

\par
%\textsuperscript{(784.2)}
\textsuperscript{70:1.6} Pero incluso esto no dio resultado al principio. Cuando los primeros jefes intentaron allanar los malentendidos, a menudo se vieron en la necesidad de autorizar los combates a pedradas en la tribu al menos una vez al año. El clan se dividía en dos grupos y emprendían una batalla que duraba todo el día, sin ninguna otra razón que la de divertirse; en verdad les gustaba pelear.

\par
%\textsuperscript{(784.3)}
\textsuperscript{70:1.7} La guerra continúa existiendo porque el hombre es humano, desciende por evolución del animal, y todos los animales son belicosos. Entre las primeras causas de la guerra figuran las siguientes:

\par
%\textsuperscript{(784.4)}
\textsuperscript{70:1.8} 1. \textit{El hambre} ---que conducía a los saqueos para conseguir alimentos. La escasez de tierras siempre ha llevado a la guerra, y durante estas luchas, las tribus pacíficas primitivas fueron prácticamente exterminadas.

\par
%\textsuperscript{(784.5)}
\textsuperscript{70:1.9} 2. \textit{La escasez de mujeres} ---el intento de mitigar la falta de ayuda doméstica. El rapto de las mujeres siempre ha provocado guerras.

\par
%\textsuperscript{(784.6)}
\textsuperscript{70:1.10} 3. \textit{La vanidad} ---el deseo de demostrar las proezas de la tribu. Los grupos superiores combatían para imponer su manera de vivir a los pueblos inferiores.

\par
%\textsuperscript{(784.7)}
\textsuperscript{70:1.11} 4. \textit{Los esclavos} ---la necesidad de nuevos miembros como mano de obra.

\par
%\textsuperscript{(784.8)}
\textsuperscript{70:1.12} 5. \textit{La venganza} era un motivo de guerra cuando una tribu creía que otra tribu vecina había ocasionado la muerte de uno de los suyos. El luto se prolongaba hasta que se traía una cabeza a la tribu. La guerra por venganza ha estado bien vista hasta una época relativamente reciente.

\par
%\textsuperscript{(784.9)}
\textsuperscript{70:1.13} 6. \textit{El divertimiento} ---los jóvenes de aquellos tiempos antiguos consideraban la guerra como una forma de diversión. Cuando no surgía ningún pretexto válido y suficiente como para desencadenar una guerra, cuando la paz se volvía agobiante, las tribus vecinas tenían la costumbre de salir a combatir de manera semi-amistosa, emprendiendo una incursión de carácter festivo para disfrutar de una batalla simulada.

\par
%\textsuperscript{(784.10)}
\textsuperscript{70:1.14} 7. \textit{La religión} ---el deseo de hacer conversos a un culto. Todas las religiones primitivas aprobaban la guerra. La religión sólo ha empezado a desaprobar la guerra en los tiempos recientes. Por desgracia, los cleros primitivos estaban habitualmente aliados con el poder militar. Uno de los grandes pasos que se han dado en todos los tiempos a favor de la paz ha sido el intento de separar la iglesia del Estado.

\par
%\textsuperscript{(784.11)}
\textsuperscript{70:1.15} Estas tribus antiguas siempre hacían la guerra por orden de sus dioses, a petición de sus jefes o de sus curanderos. Los hebreos creían en este tipo de <<Dios de las batallas>>\footnote{\textit{Dios de las batallas}: 1 Cr 14:15; 2 Cr 20:15; 32:8; Sal 24:8; Dt 7:21-23; 20:1-4; 1 Sam 17:47.}, y la narración de su ataque repentino a los madianitas es un típico relato de la crueldad atroz de las antiguas guerras entre tribus\footnote{\textit{Ataques bárbaros}: Nm 31:3-31.}; este ataque, con la masacre de todos los varones y la matanza posterior de todos los niños varones y de todas las mujeres que no eran vírgenes\footnote{\textit{Conservación de las vírgenes}: Jue 21:10-12.}, hubiera hecho honor a las costumbres de un jefe tribal de hace doscientos mil años. Y todo esto se llevó a cabo en <<nombre del Señor Dios de Israel>>\footnote{\textit{En nombre del Señor Dios de Israel}: Nm 31:1-2; Dt 7:16-24; 20:1; Jue 11:21,23; 1 Sam 15:2-3.}.

\par
%\textsuperscript{(784.12)}
\textsuperscript{70:1.16} Esta narración describe la evolución de la sociedad ---la solución natural de los problemas de las razas--- el hombre elaborando su propio destino en la Tierra. La Deidad no instiga este tipo de atrocidades, a pesar de la tendencia del hombre a responsabilizar a sus dioses.

\par
%\textsuperscript{(784.13)}
\textsuperscript{70:1.17} La clemencia militar ha tardado en manifestarse en la humanidad. Incluso cuando una mujer, Débora, gobernaba a los hebreos, continuaba existiendo la misma crueldad sistemática. Cuando su general venció a los gentiles, hizo que <<todo el ejército cayera bajo la espada; no quedó ni uno vivo>>\footnote{\textit{Matanza del ejército de Sísara}: Jue 4:16.}.

\par
%\textsuperscript{(785.1)}
\textsuperscript{70:1.18} Las armas envenenadas se utilizaron muy pronto en la historia de la raza. Se practicaron todo tipo de mutilaciones. Saúl no dudó en exigir a David cien prepucios de filisteos como dote a pagar por su hija Mical\footnote{\textit{Dote con prepucios}: 1 Sam 18:25-27.}.

\par
%\textsuperscript{(785.2)}
\textsuperscript{70:1.19} Las primeras guerras tenían lugar entre tribus enteras, pero en épocas posteriores, cuando dos individuos de tribus diferentes tenían una disputa, en lugar de permitir que lucharan las dos tribus, los dos rivales se batían en duelo. También se estableció la costumbre de que dos ejércitos lo arriesgaran todo al resultado del combate entre los representantes escogidos por cada lado, como en el caso de David y Goliat\footnote{\textit{Lucha de campeones}: 1 Sam 17:1-51.}.

\par
%\textsuperscript{(785.3)}
\textsuperscript{70:1.20} El primer refinamiento de la guerra fue hacer prisioneros. Después, a las mujeres se les eximió de las hostilidades, y luego vino el reconocimiento de los no combatientes. Pronto se desarrollaron las castas militares y los ejércitos permanentes para mantenerse al mismo ritmo que la creciente complejidad del combate. A estos guerreros se les prohibió pronto que se asociaran con las mujeres, y hace mucho tiempo que las mujeres dejaron de combatir, aunque siempre han alimentado y curado a los soldados y los han incitado a luchar.

\par
%\textsuperscript{(785.4)}
\textsuperscript{70:1.21} La práctica de declarar la guerra representó un gran progreso. Estas declaraciones de intención de combatir anunciaron la llegada de un sentido de la equidad, a lo cual le siguió el desarrollo gradual de las reglas de la guerra <<civilizada>>. Muy pronto se estableció la costumbre de no combatir cerca de los lugares religiosos, y aún más tarde, de no luchar durante ciertos días sagrados. Luego vino el reconocimiento general del derecho de asilo; los refugiados políticos recibieron protección.

\par
%\textsuperscript{(785.5)}
\textsuperscript{70:1.22} La guerra evolucionó así paulatinamente desde la primitiva caza del hombre hasta el sistema un poco más ordenado de las naciones <<civilizadas>> más recientes. Pero la actitud social de amistad tarda mucho tiempo en reemplazar a la actitud de enemistad.

\section*{2. El valor social de la guerra}
\par
%\textsuperscript{(785.6)}
\textsuperscript{70:2.1} En las épocas pasadas, una guerra feroz provocaba tales cambios sociales y facilitaba la adopción de tales nuevas ideas, que éstos no habrían aparecido de manera natural en diez mil años. El precio terrible que se pagaba por estas ventajas indudables de la guerra era el retroceso temporal de la sociedad al estado salvaje; la razón civilizada tenía que abdicar. La guerra es un remedio poderoso, muy costoso y sumamente peligroso; aunque cura a menudo ciertos males sociales, a veces mata al paciente, destruye la sociedad.

\par
%\textsuperscript{(785.7)}
\textsuperscript{70:2.2} La necesidad constante de la defensa nacional produce muchos ajustes sociales nuevos y avanzados. La sociedad disfruta hoy de los beneficios de una larga lista de innovaciones útiles que al principio eran totalmente militares; y a la guerra le debe incluso la danza, una de cuyas primeras formas fue un ejercicio militar.

\par
%\textsuperscript{(785.8)}
\textsuperscript{70:2.3} La guerra ha tenido un valor social para las civilizaciones pasadas porque:

\par
%\textsuperscript{(785.9)}
\textsuperscript{70:2.4} 1. Imponía la disciplina, forzaba a la cooperación.

\par
%\textsuperscript{(785.10)}
\textsuperscript{70:2.5} 2. Premiaba la entereza y la valentía.

\par
%\textsuperscript{(785.11)}
\textsuperscript{70:2.6} 3. Fomentaba y consolidaba el nacionalismo.

\par
%\textsuperscript{(785.12)}
\textsuperscript{70:2.7} 4. Destruía a los pueblos débiles e ineptos.

\par
%\textsuperscript{(785.13)}
\textsuperscript{70:2.8} 5. Deshacía la ilusión de la igualdad primitiva y estratificaba selectivamente a la sociedad.

\par
%\textsuperscript{(785.14)}
\textsuperscript{70:2.9} La guerra ha tenido cierto valor evolutivo y selectivo, pero al igual que la esclavitud, deberá abandonarse alguna vez a medida que la civilización progrese lentamente. Las guerras antiguas favorecían los viajes y los intercambios culturales; los métodos modernos de transporte y de comunicación sirven ahora mejor para estos fines. Las guerras de antaño fortalecían a las naciones, pero las luchas modernas trastornan la cultura civilizada. Las guerras antiguas conducían a diezmar a los pueblos inferiores; el resultado neto de los conflictos modernos es la destrucción selectiva de los mejores linajes humanos. Las guerras primitivas estimulaban la organización y la eficacia, pero éstas últimas se han convertido ahora en los objetivos de la industria moderna. Durante las épocas pasadas, la guerra era un fermento social que empujaba a la civilización hacia adelante; este resultado ahora se logra mejor mediante la ambición y la invención. Las guerras antiguas sostenían el concepto de un Dios de las batallas, pero al hombre moderno se le ha informado de que Dios es amor\footnote{\textit{Dios es amor}: 1 Jn 4:8,16.}. La guerra ha servido para muchos fines valiosos en el pasado; ha sido un andamiaje indispensable para construir la civilización, pero se está declarando rápidamente en quiebra cultural ---es incapaz de producir, en beneficios sociales, los dividendos de alguna forma proporcionales a las terribles pérdidas que acompañan a su invocación.

\par
%\textsuperscript{(786.1)}
\textsuperscript{70:2.10} En otra época, los médicos creían que la sangría curaba numerosas enfermedades, pero desde entonces han descubierto remedios más eficaces para la mayoría de estas dolencias. La sangría internacional de la guerra deberá también ceder el paso indudablemente al descubrimiento de mejores métodos para curar los males de las naciones.

\par
%\textsuperscript{(786.2)}
\textsuperscript{70:2.11} Las naciones de Urantia ya han emprendido la lucha gigantesca entre el militarismo nacionalista y el industrialismo, y este conflicto es análogo en muchos aspectos a la lucha secular entre los pastores-cazadores y los agricultores. Pero si el industrialismo ha de triunfar sobre el militarismo, debe evitar los peligros que le acechan. Los peligros para la industria incipiente de Urantia son:

\par
%\textsuperscript{(786.3)}
\textsuperscript{70:2.12} 1. La fuerte tendencia hacia el materialismo, la ceguera espiritual.

\par
%\textsuperscript{(786.4)}
\textsuperscript{70:2.13} 2. La adoración del poder de las riquezas, la deformación de los valores.

\par
%\textsuperscript{(786.5)}
\textsuperscript{70:2.14} 3. Los vicios del lujo, la inmadurez cultural.

\par
%\textsuperscript{(786.6)}
\textsuperscript{70:2.15} 4. Los peligros crecientes de la indolencia, la insensibilidad al servicio.

\par
%\textsuperscript{(786.7)}
\textsuperscript{70:2.16} 5. El desarrollo de una debilidad racial indeseable, la degeneración biológica.

\par
%\textsuperscript{(786.8)}
\textsuperscript{70:2.17} 6. La amenaza de una esclavitud industrial estandarizada, el estancamiento de la personalidad. El trabajo ennoblece, pero las faenas monótonas embrutecen.

\par
%\textsuperscript{(786.9)}
\textsuperscript{70:2.18} El militarismo es autocrático y cruel ---salvaje. Favorece la organización social entre los vencedores, pero desintegra a los vencidos. El industrialismo es más civilizado y debería promoverse de tal manera que favorezca la iniciativa y estimule el individualismo. La sociedad debería fomentar la originalidad por todos los medios.

\par
%\textsuperscript{(786.10)}
\textsuperscript{70:2.19} No cometáis el error de glorificar la guerra; discernid más bien lo que ha hecho por la sociedad, para que podáis imaginar con más exactitud lo que deben proporcionar sus sustitutos a fin de que continúe el progreso de la civilización. Si no se proveen esos sustitutos adecuados, entonces podéis estar seguros de que la guerra continuará existiendo durante mucho tiempo.

\par
%\textsuperscript{(786.11)}
\textsuperscript{70:2.20} El hombre nunca aceptará la paz como una manera normal de vivir hasta que no se haya convencido repetidas veces y por completo de que la paz es lo mejor para su bienestar material, y hasta que la sociedad no haya facilitado sabiamente los sustitutos pacíficos para satisfacer la tendencia inherente a dar rienda suelta periódicamente al impulso colectivo destinado a liberar las emociones y energías que se acumulan constantemente, y que forman parte de las reacciones autopreservatorias de la especie humana.

\par
%\textsuperscript{(786.12)}
\textsuperscript{70:2.21} Pero la guerra debería ser reconocida, aunque sea de paso, como la escuela experiencial que ha obligado a una raza de individualistas arrogantes a someterse a una autoridad extremadamente concentrada ---a un jefe ejecutivo. La guerra a la antigua usanza escogía como jefes a los hombres que eran eminentes por naturaleza, pero la guerra moderna ya no lo hace. Para descubrir a sus dirigentes, la sociedad debe recurrir ahora a las conquistas de la paz: la industria, la ciencia y las realizaciones sociales.

\section*{3. Las asociaciones humanas primitivas}
\par
%\textsuperscript{(787.1)}
\textsuperscript{70:3.1} En la sociedad más primitiva, la \textit{horda} lo es todo; incluso los niños son su propiedad común. La familia evolutiva sustituyó a la horda en la crianza de los hijos, mientras que los clanes y las tribus emergentes la reemplazaron como unidad social.

\par
%\textsuperscript{(787.2)}
\textsuperscript{70:3.2} El apetito sexual y el amor maternal establecen la familia. Pero el gobierno real no aparece hasta que no se han empezado a formar los grupos superfamiliares. En los tiempos prefamiliares de la horda, los individuos escogidos sin ceremonias eran los que aseguraban el caudillaje. Los bosquimanos africanos nunca han sobrepasado este estado primitivo; no tienen jefes en la horda.

\par
%\textsuperscript{(787.3)}
\textsuperscript{70:3.3} Las familias se unieron por lazos de sangre en clanes, en conjuntos de parientes, y estos clanes se convirtieron más tarde en tribus, en comunidades territoriales. La guerra y la presión externa forzaron a los clanes de parientes a organizarse en tribus, pero el comercio y los negocios son los que mantuvieron unidos a estos grupos primitivos iniciales con cierto grado de paz interna.

\par
%\textsuperscript{(787.4)}
\textsuperscript{70:3.4} Las organizaciones comerciales internacionales favorecerán la paz en Urantia mucho más que toda la sofistería sensiblera de los planes quiméricos de paz. El desarrollo del lenguaje y los métodos más perfectos de comunicación, así como la mejora del transporte, han facilitado las relaciones comerciales.

\par
%\textsuperscript{(787.5)}
\textsuperscript{70:3.5} La ausencia de un lenguaje común siempre ha obstaculizado el crecimiento de los grupos pacíficos, pero el dinero se ha convertido en el lenguaje universal del comercio moderno. La sociedad moderna se mantiene unida en gran parte gracias al mercado industrial. El afán de lucro es un poderoso civilizador cuando contiene además el deseo de servir.

\par
%\textsuperscript{(787.6)}
\textsuperscript{70:3.6} En las épocas primitivas, cada tribu estaba rodeada por unos círculos concéntricos de miedo y de desconfianza crecientes; de ahí que en otro tiempo fuera costumbre matar a todos los extraños, y más adelante, esclavizarlos. La idea antigua de la amistad significaba la adopción por parte del clan; y se creía que uno continuaba perteneciendo al clan después de la muerte ---fue uno de los primeros conceptos de la vida eterna.

\par
%\textsuperscript{(787.7)}
\textsuperscript{70:3.7} La ceremonia de adopción consistía en beber uno la sangre del otro. En algunos grupos se intercambiaban la saliva en lugar de beber la sangre, y éste es el antiguo origen de la costumbre de besarse en sociedad. Y todas las ceremonias de asociación, ya se tratara de casamientos o de adopciones, siempre terminaban en un banquete.

\par
%\textsuperscript{(787.8)}
\textsuperscript{70:3.8} En tiempos posteriores se utilizó la sangre diluida en vino tinto, y finalmente sólo se bebió el vino para sellar la ceremonia de adopción, la cual se notificaba poniendo en contacto las copas de vino y se consumaba tragando la bebida. Los hebreos emplearon una forma modificada de esta ceremonia de adopción. Sus antepasados árabes utilizaban un juramento que se prestaba mientras la mano del candidato descansaba en el órgano genital del nativo de la tribu. Los hebreos trataban a los extranjeros adoptados con amabilidad y fraternidad. <<El extranjero que vive con vosotros será como alguien que ha nacido entre vosotros, y lo amaréis como a vosotros mismos>>\footnote{\textit{Extranjero tratado como local}: Lv 19:34.}.

\par
%\textsuperscript{(787.9)}
\textsuperscript{70:3.9} <<La amistad con los huéspedes>> era una relación de hospitalidad temporal. Cuando los huéspedes que estaban de visita se marchaban, se rompía un plato en dos mitades y se entregaba una de ellas al amigo que partía, para que sirviera de introducción apropiada a una tercera persona que pudiera llegar de visita en el futuro. Existía la costumbre de que los huéspedes pagaran su estancia contando las historias de sus viajes y aventuras. Los narradores de antaño se volvieron tan populares, que las costumbres terminaron por prohibirles que ejercieran su actividad durante las temporadas de caza o de cosecha.

\par
%\textsuperscript{(788.1)}
\textsuperscript{70:3.10} Los primeros tratados de paz fueron los <<lazos de sangre>>. Los embajadores de la paz de dos tribus en guerra se reunían, se rendían homenaje, y luego procedían a pincharse la piel hasta que ésta sangraba; después de lo cual se chupaban mutuamente la sangre y declaraban la paz.

\par
%\textsuperscript{(788.2)}
\textsuperscript{70:3.11} Las primeras misiones de paz consistieron en delegaciones de hombres que llevaban a sus doncellas escogidas para la satisfacción sexual de sus antiguos enemigos, y utilizaban este apetito sexual para combatir los impulsos bélicos. La tribu honrada de este modo devolvía la visita, con su ofrenda de doncellas; después de esto la paz se establecía firmemente. Al poco tiempo se autorizaban los matrimonios entre las familias de los jefes.

\section*{4. Los clanes y las tribus}
\par
%\textsuperscript{(788.3)}
\textsuperscript{70:4.1} El primer grupo pacífico fue la familia, luego el clan, la tribu, y más tarde la nación, que con el tiempo se convertiría en el Estado territorial moderno. Es sumamente alentador el hecho de que los grupos pacíficos de hoy en día se hayan ampliado desde hace mucho tiempo más allá de los lazos de sangre hasta englobar a las naciones, a pesar del hecho de que las naciones de Urantia continúan gastando inmensas sumas en preparativos de guerra.

\par
%\textsuperscript{(788.4)}
\textsuperscript{70:4.2} Los clanes eran los grupos consanguíneos dentro de la tribu, y debían su existencia a ciertos intereses comunes, tales como:

\par
%\textsuperscript{(788.5)}
\textsuperscript{70:4.3} 1. Su origen se remontaba a un antepasado común.

\par
%\textsuperscript{(788.6)}
\textsuperscript{70:4.4} 2. Eran leales a un tótem religioso común.

\par
%\textsuperscript{(788.7)}
\textsuperscript{70:4.5} 3. Hablaban el mismo dialecto.

\par
%\textsuperscript{(788.8)}
\textsuperscript{70:4.6} 4. Compartían un lugar de residencia común.

\par
%\textsuperscript{(788.9)}
\textsuperscript{70:4.7} 5. Temían a los mismos enemigos.

\par
%\textsuperscript{(788.10)}
\textsuperscript{70:4.8} 6. Tenían una experiencia militar común.

\par
%\textsuperscript{(788.11)}
\textsuperscript{70:4.9} Los jefes de los clanes estaban siempre subordinados al jefe de la tribu, y los primeros gobiernos tribales fueron una vaga confederación de clanes. Los aborígenes australianos nunca han desarrollado una forma de gobierno tribal.

\par
%\textsuperscript{(788.12)}
\textsuperscript{70:4.10} Los jefes pacíficos de los clanes gobernaban generalmente por la línea materna; los jefes guerreros de las tribus establecieron la línea paterna. Las cortes de los jefes tribales y de los primeros reyes estaban compuestas por los jefes de los clanes, y era costumbre invitarlos a que se presentaran ante el rey varias veces al año. Esto permitía a este último vigilarlos y asegurarse mejor su cooperación. Los clanes desempeñaron un valioso servicio en los gobiernos locales, pero retrasaron enormemente el desarrollo de naciones grandes y fuertes.

\section*{5. Los principios del gobierno}
\par
%\textsuperscript{(788.13)}
\textsuperscript{70:5.1} Toda institución humana ha tenido un comienzo, y el gobierno civil es un producto de la evolución progresiva, al igual que lo son el matrimonio, la industria y la religión. A partir de los primeros clanes y de las tribus primitivas, se desarrollaron gradualmente los tipos sucesivos de gobiernos humanos que han aparecido y desaparecido, hasta llegar a las formas de reglamentación civil y social que caracterizan al segundo tercio del siglo veinte.

\par
%\textsuperscript{(788.14)}
\textsuperscript{70:5.2} Con la aparición gradual de las unidades familiares, las bases del gobierno se establecieron en la organización del clan, en la agrupación de las familias consanguíneas. El primer cuerpo verdaderamente gubernamental fue el \textit{consejo de ancianos}\footnote{\textit{Consejo de ancianos}: Gn 50:7; Ex 3:16-18.}. Este grupo regulador estaba compuesto por los ancianos que se habían distinguido de alguna manera eficaz. Incluso el hombre bárbaro supo apreciar pronto la sabiduría y la experiencia, y el resultado fue un largo período de dominación por parte de los ancianos. Este reinado oligárquico de la edad se convirtió gradualmente en la idea del patriarcado.

\par
%\textsuperscript{(789.1)}
\textsuperscript{70:5.3} En el consejo primitivo de ancianos residía el potencial de todas las funciones gubernamentales: la ejecutiva, la legislativa y la judicial. Cuando el consejo interpretaba las costumbres vigentes, era un tribunal; cuando establecía las nuevas formas de usanzas sociales, era un cuerpo legislativo; en la medida en que hacía cumplir estos decretos y promulgaciones, era el poder ejecutivo. El presidente del consejo fue uno de los precursores del jefe tribal posterior.

\par
%\textsuperscript{(789.2)}
\textsuperscript{70:5.4} Algunas tribus tenían consejos femeninos y, de vez en cuando, muchas tribus fueron gobernadas por mujeres. Algunas tribus de hombres rojos conservaron la enseñanza de Onamonalontón consistente en seguir las decisiones unánimes del <<consejo de los siete>>.

\par
%\textsuperscript{(789.3)}
\textsuperscript{70:5.5} A la humanidad le ha costado trabajo aprender que un club de debates no puede dirigir ni la guerra ni la paz. Las <<palabrerías>> primitivas raras veces fueron útiles. La raza aprendió pronto que un ejército dirigido por un grupo de jefes de clanes no tenía ninguna posibilidad ante un fuerte ejército mandado por un solo hombre. La guerra siempre ha producido reyes.

\par
%\textsuperscript{(789.4)}
\textsuperscript{70:5.6} Al principio, los jefes de guerra se elegían exclusivamente para el servicio militar, y solían renunciar a una parte de su autoridad durante los períodos de paz, cuando sus deberes tenían un carácter más bien social. Pero poco a poco empezaron a inmiscuirse en los intervalos de paz, tendiendo a continuar gobernando de una guerra a la siguiente. A menudo procuraron que una guerra no tardara mucho tiempo en seguir a la otra. A estos primitivos señores de la guerra no les gustaba la paz.

\par
%\textsuperscript{(789.5)}
\textsuperscript{70:5.7} En tiempos posteriores, algunos jefes fueron escogidos para otros servicios no militares, siendo seleccionados debido a una constitución física excepcional o a unas aptitudes personales sobresalientes. Los hombres rojos tenían a menudo dos clases de jefes ---los sachems, o jefes de la paz, y los jefes de guerra hereditarios. Los jefes de la paz también eran jueces y educadores.

\par
%\textsuperscript{(789.6)}
\textsuperscript{70:5.8} Algunas comunidades primitivas estaban gobernadas por los curanderos, que a menudo ejercían como jefes. Un solo hombre desempeñaba las funciones de sacerdote, médico y jefe ejecutivo. Con mucha frecuencia, las primeras insignias reales habían sido al principio los símbolos o emblemas de las vestiduras sacerdotales.

\par
%\textsuperscript{(789.7)}
\textsuperscript{70:5.9} La rama ejecutiva del gobierno nació gradualmente a través de estas etapas. Los consejos de los clanes y de las tribus continuaron existiendo en calidad de asesores y como precursores de las ramas legislativa y judicial que aparecieron más tarde. Hoy día, en África, todas estas formas de gobiernos primitivos existen realmente entre las diversas tribus.

\section*{6. El gobierno monárquico}
\par
%\textsuperscript{(789.8)}
\textsuperscript{70:6.1} El gobierno estatal eficaz sólo apareció con la llegada de un jefe que tenía plena autoridad ejecutiva. El hombre descubrió que sólo se podía tener un gobierno eficaz confiriendo el poder a una personalidad, y no sosteniendo una idea.

\par
%\textsuperscript{(789.9)}
\textsuperscript{70:6.2} La soberanía tuvo su origen en la idea de la autoridad o de la riqueza familiar. Cuando un reyezuelo patriarcal se convertía en un verdadero rey, a veces se le llamaba el <<padre de su pueblo>>\footnote{\textit{Padre de su pueblo}: Gn 19:37-38; 36:9,43; Nm 3:24.}. Más adelante se creyó que los reyes habían surgido de los héroes. Y más tarde aún, la soberanía se volvió hereditaria, debido a la creencia en el origen divino de los reyes.

\par
%\textsuperscript{(789.10)}
\textsuperscript{70:6.3} La monarquía hereditaria evitó la anarquía que anteriormente había causado tantos estragos entre la muerte de un rey y la elección de su sucesor. La familia tenía un jefe biológico y el clan un jefe natural escogido; pero la tribu, y más tarde el Estado, no tenían ningún dirigente natural, y éste fue un motivo adicional para hacer que los jefes-reyes fueran hereditarios. La idea de las familias reales y de la aristocracia también estaba basada en las costumbres de <<poseer un nombre>> en los clanes.

\par
%\textsuperscript{(790.1)}
\textsuperscript{70:6.4} La sucesión de los reyes se consideró finalmente como sobrenatural, pues se creía que la sangre real se remontaba a los tiempos del estado mayor materializado del Príncipe Caligastia. Los reyes se convirtieron así en personalidades fetiche y se les tuvo un miedo desmesurado, adoptándose una forma especial de lenguaje para utilizarlo en la corte. Incluso en épocas recientes se ha creído que tocar a un rey curaba las enfermedades, y algunos pueblos de Urantia consideran todavía que sus soberanos han tenido un origen divino.

\par
%\textsuperscript{(790.2)}
\textsuperscript{70:6.5} Al rey fetiche primitivo se le mantenía a menudo aislado; se le consideraba demasiado sagrado como para ser visto, salvo los días de fiesta y los días sagrados. Habitualmente se escogía a un representante para que actuara en su lugar, y éste es el origen de los primeros ministros. El primer funcionario ministerial fue un administrador de alimentos; otros le siguieron poco después. Los soberanos nombraron pronto a unos representantes para que se encargaran del comercio y de la religión; el desarrollo de los gabinetes ministeriales supuso un paso directo hacia la despersonalización de la autoridad ejecutiva. Estos ayudantes de los primeros reyes se convirtieron en la nobleza reconocida, y la esposa del rey ascendió gradualmente a la dignidad de reina a medida que las mujeres gozaron de mayor estima.

\par
%\textsuperscript{(790.3)}
\textsuperscript{70:6.6} Los soberanos sin escrúpulos consiguieron un gran poder gracias al descubrimiento del veneno. La magia de las cortes primitivas era diabólica; los enemigos del rey morían pronto. Pero incluso el tirano más déspota se encontraba sometido a algunas restricciones; al menos se sentía refrenado por el miedo constante a ser asesinado. Los curanderos, los hechiceros y los sacerdotes han sido siempre un freno poderoso para los reyes. Los terratenientes, la aristocracia, ejercieron posteriormente una influencia restrictiva. Y de vez en cuando, los clanes y las tribus sencillamente se sublevaban y derrocaban a sus déspotas y tiranos. Cuando los soberanos depuestos eran condenados a muerte, a menudo se les concedía la alternativa de suicidarse, lo cual dio origen a la antigua moda social de suicidarse en ciertas circunstancias.

\section*{7. Los clubes primitivos y las sociedades secretas}
\par
%\textsuperscript{(790.4)}
\textsuperscript{70:7.1} La consanguinidad determinó los primeros grupos sociales; los clanes consanguíneos se agrandaron mediante la asociación. Los matrimonios entre los clanes fueron la etapa siguiente en la ampliación de los grupos, y la tribu compleja resultante fue el primer organismo verdaderamente político. El progreso siguiente en el desarrollo social fue la evolución de los cultos religiosos y de los clubes políticos. Éstos aparecieron primero como sociedades secretas e inicialmente eran totalmente religiosas; después se volvieron reguladoras. Al principio eran clubes de hombres; más tarde aparecieron grupos de mujeres. Luego se dividieron en dos clases: sociopolítica y místico-religiosa.

\par
%\textsuperscript{(790.5)}
\textsuperscript{70:7.2} Estas sociedades tenían muchas razones para permanecer secretas, tales como:

\par
%\textsuperscript{(790.6)}
\textsuperscript{70:7.3} 1. El temor a atraer la indignación de los dirigentes por haber violado algún tabú.

\par
%\textsuperscript{(790.7)}
\textsuperscript{70:7.4} 2. La finalidad de practicar unos ritos religiosos minoritarios.

\par
%\textsuperscript{(790.8)}
\textsuperscript{70:7.5} 3. La intención de preservar valiosos secretos <<espirituales>> o comerciales.

\par
%\textsuperscript{(790.9)}
\textsuperscript{70:7.6} 4. Disfrutar de algún hechizo o magia especial.

\par
%\textsuperscript{(790.10)}
\textsuperscript{70:7.7} El hecho mismo de que estas sociedades fueran secretas confería a todos sus miembros el poder del misterio frente al resto de la tribu. El secreto atrae también la vanidad; los iniciados formaban la aristocracia social de su época. Después de su iniciación, los muchachos cazaban con los hombres, mientras que anteriormente recogían las verduras con las mujeres. La humillación suprema, la deshonra ante la tribu, consistía en no lograr pasar las pruebas de la pubertad, y verse así obligado a permanecer fuera de la vivienda de los hombres en compañía de las mujeres y los niños, en ser considerado como afeminado. Además, a los no iniciados no se les permitía casarse.

\par
%\textsuperscript{(791.1)}
\textsuperscript{70:7.8} Los pueblos primitivos enseñaron muy pronto a sus jóvenes adolescentes a controlar sus impulsos sexuales. Se estableció la costumbre de separar a los muchachos de sus padres desde la pubertad hasta el matrimonio, confiando su educación y formación a las sociedades secretas de los hombres. Una de las funciones principales de estos clubes era conservar el control de los jóvenes adolescentes para evitar así los hijos ilegítimos.

\par
%\textsuperscript{(791.2)}
\textsuperscript{70:7.9} La prostitución comercializada empezó cuando estos clubes de hombres pagaron con dinero el derecho a utilizar las mujeres de otras tribus. Pero los grupos más primitivos permanecieron notablemente libres de laxitud sexual.

\par
%\textsuperscript{(791.3)}
\textsuperscript{70:7.10} La ceremonia de iniciación de la pubertad se prolongaba generalmente durante un período de cinco años. Estas ceremonias contenían muchas torturas y cortes dolorosos que se infligían a sí mismos. La circuncisión se practicó al principio como un rito de iniciación en una de estas cofradías secretas. Las marcas de la tribu se grababan en el cuerpo como parte de la iniciación de la pubertad; el tatuaje se originó así, como un símbolo de pertenencia. Estas torturas, así como muchas privaciones, estaban destinadas a endurecer a estos jóvenes, a inculcarles la realidad de la vida y sus penurias inevitables. Este objetivo se logra mejor mediante los juegos atléticos y las competiciones físicas que aparecieron más tarde.

\par
%\textsuperscript{(791.4)}
\textsuperscript{70:7.11} Pero las sociedades secretas intentaban mejorar de verdad la moral de los adolescentes; una de las metas principales de las ceremonias de la pubertad era inculcar a los muchachos que debían dejar en paz a las esposas de los otros hombres.

\par
%\textsuperscript{(791.5)}
\textsuperscript{70:7.12} Después de estos años de disciplina y entrenamiento rigurosos, y justo antes de casarse, a los jóvenes se les dejaba salir durante un corto período de ocio y de libertad, después del cual volvían para casarse y someterse a la sujeción de los tabúes de su tribu durante el resto de su vida. Esta antigua costumbre ha subsistido hasta los tiempos modernos en la idea descabellada de <<correrla mientras se es joven>>.

\par
%\textsuperscript{(791.6)}
\textsuperscript{70:7.13} Muchas tribus posteriores autorizaron la formación de clubes secretos de mujeres, cuya finalidad consistía en preparar a las muchachas adolescentes para ser esposas y madres. Después de su iniciación, las jóvenes estaban capacitadas para el matrimonio y se les permitía asistir a la <<presentación de las novias>>, la fiesta de presentación en sociedad de aquellos tiempos. Las órdenes de mujeres con votos de celibato empezaron a aparecer muy pronto.

\par
%\textsuperscript{(791.7)}
\textsuperscript{70:7.14} Los clubes no secretos hicieron luego su aparición cuando los grupos de hombres solteros y de mujeres no comprometidas formaron sus organizaciones separadas. Estas asociaciones fueron en realidad las primeras escuelas. Mientras que los clubes masculinos y femeninos se dedicaban con frecuencia a perseguirse mutuamente, algunas tribus avanzadas, después de haber estado en contacto con los educadores de Dalamatia, experimentaron con la enseñanza mixta, disponiendo de internados para ambos sexos.

\par
%\textsuperscript{(791.8)}
\textsuperscript{70:7.15} Las sociedades secretas contribuyeron a la formación de las castas sociales, principalmente debido al carácter misterioso de sus iniciaciones. Al principio, los miembros de estas sociedades utilizaban máscaras para asustar a los curiosos y alejarlos de sus ritos de duelo ---el culto a los antepasados. Este ritual se convirtió más tarde en una seudo sesión de espiritismo en la que se suponía que aparecían fantasmas. Las antiguas sociedades del <<nuevo nacimiento>> utilizaban signos y empleaban un lenguaje secreto especial; también renunciaban solemnemente a ciertos alimentos y bebidas. Actuaban como policía nocturna y, por lo demás, ejercían sus funciones en una amplia gama de actividades sociales.

\par
%\textsuperscript{(792.1)}
\textsuperscript{70:7.16} Todas las asociaciones secretas imponían un juramento, prescribían la confianza entre sus miembros y enseñaban que había que guardar los secretos. Estas agrupaciones atemorizaban y controlaban a las muchedumbres; también actuaban como sociedades de vigilancia, y practicaban linchamientos. Fueron los primeros espías de las tribus que estaban en guerra y la primera policía secreta en tiempos de paz. Lo mejor de todo fue que mantuvieron a los reyes poco escrupulosos en un estado de inquietud. Para compensar este hecho, los reyes patrocinaron su propia policía secreta.

\par
%\textsuperscript{(792.2)}
\textsuperscript{70:7.17} Estas sociedades dieron nacimiento a los primeros partidos políticos. El primer gobierno partidista fue el de <<los fuertes>> \textit{contra} <<los débiles>>. En los tiempos antiguos, un cambio de administración sólo se producía después de una guerra civil, probando así sobradamente que los débiles se habían vuelto fuertes.

\par
%\textsuperscript{(792.3)}
\textsuperscript{70:7.18} Los comerciantes emplearon estos clubes para cobrar sus deudas, y los soberanos para recaudar sus impuestos. El sistema tributario ha supuesto una larga lucha, y una de sus primeras formas fue el diezmo, la décima parte de la caza o del botín. Al principio los impuestos se cobraban para mantener la casa del rey, pero se descubrió que era más fácil recaudarlos cuando se disfrazaban bajo la forma de ofrendas para sostener el servicio del templo.

\par
%\textsuperscript{(792.4)}
\textsuperscript{70:7.19} Estas asociaciones secretas se convirtieron después en las primeras organizaciones caritativas y más tarde evolucionaron en sociedades religiosas primitivas ---las precursoras de las iglesias. Finalmente, algunas de estas sociedades se volvieron intertribales, formando las primeras cofradías internacionales.

\section*{8. Las clases sociales}
\par
%\textsuperscript{(792.5)}
\textsuperscript{70:8.1} La desigualdad mental y física de los seres humanos asegura la aparición de las clases sociales. Los únicos mundos que no tienen estratos sociales son los más primitivos y los más avanzados. Una civilización en sus albores aún no ha empezado la diferenciación de los niveles sociales, mientras que un mundo establecido en la luz y la vida ha borrado en gran parte estas divisiones de la humanidad, tan características de todas las etapas evolutivas intermedias.

\par
%\textsuperscript{(792.6)}
\textsuperscript{70:8.2} A medida que la sociedad salió del salvajismo para entrar en la barbarie, sus componentes humanos tendieron a agruparse en clases por las razones generales siguientes:

\par
%\textsuperscript{(792.7)}
\textsuperscript{70:8.3} 1. \textit{Razones naturales} ---contacto, parentesco y matrimonio; las primeras distinciones sociales estuvieron basadas en el sexo, la edad y la sangre ---en el parentesco con el jefe.

\par
%\textsuperscript{(792.8)}
\textsuperscript{70:8.4} 2. \textit{Razones personales} ---el reconocimiento de la capacidad, la resistencia, la habilidad y la entereza, a lo que pronto le siguió el reconocimiento del dominio del lenguaje, el saber y la inteligencia general.

\par
%\textsuperscript{(792.9)}
\textsuperscript{70:8.5} 3. \textit{Razones fortuitas} ---la guerra y la emigración ocasionaron la separación de los grupos humanos. Las conquistas, las relaciones entre los vencedores y los vencidos, influyeron poderosamente en la evolución de las clases, mientras que la esclavitud provocó la primera división general de la sociedad en hombres libres y cautivos.

\par
%\textsuperscript{(792.10)}
\textsuperscript{70:8.6} 4. \textit{Razones económicas} ---los ricos y los pobres. La riqueza y la posesión de esclavos fue una base que generó una de las clases de la sociedad.

\par
%\textsuperscript{(792.11)}
\textsuperscript{70:8.7} 5. \textit{Razones geográficas} ---ciertas clases surgieron a consecuencia del establecimiento de la población en zonas urbanas o rurales. Las ciudades y el campo han contribuido respectivamente a la diferenciación entre los pastores-agricultores y los comerciantes-industriales, con sus reacciones y puntos de vista divergentes.

\par
%\textsuperscript{(792.12)}
\textsuperscript{70:8.8} 6. \textit{Razones sociales} ---algunas clases se han formado gradualmente según la apreciación popular del valor social de diversos grupos. Entre las primeras divisiones de esta índole se encontraron las distinciones entre los sacerdotes-educadores, los gobernantes-guerreros, los capitalistas-comerciantes, los obreros comunes y los esclavos. El esclavo nunca podía convertirse en capitalista, pero a veces el asalariado podía optar por unirse a los capitalistas.

\par
%\textsuperscript{(793.1)}
\textsuperscript{70:8.9} 7. \textit{Razones profesionales} ---a medida que las profesiones se multiplicaron, tendieron a establecer castas y gremios. Los trabajadores se dividieron en tres grupos: las clases profesionales, incluídos los curanderos, luego los trabajadores especializados, seguidos de los obreros no especializados.

\par
%\textsuperscript{(793.2)}
\textsuperscript{70:8.10} 8. \textit{Razones religiosas} ---los primeros clubes de culto dieron nacimiento a sus propias clases dentro de los clanes y las tribus; la piedad y el misticismo de los sacerdotes las han perpetuado durante mucho tiempo como un grupo social distinto.

\par
%\textsuperscript{(793.3)}
\textsuperscript{70:8.11} 9. \textit{Razones raciales} ---la presencia de dos o más razas dentro de una nación o unidad territorial determinada produce generalmente castas de color. El sistema original de las castas de la India estaba basado en el color, así como el del antiguo Egipto.

\par
%\textsuperscript{(793.4)}
\textsuperscript{70:8.12} 10. \textit{Razones de edad} ---la juventud y la madurez. En las tribus, los niños permanecían bajo la custodia de su padre mientras éste vivía, y en cambio las niñas se quedaban a cargo de su madre hasta que se casaban.

\par
%\textsuperscript{(793.5)}
\textsuperscript{70:8.13} Unas clases sociales flexibles y cambiantes son indispensables para una civilización en evolución, pero cuando las \textit{clases} se convierten en \textit{castas}, cuando los niveles sociales se petrifican, el mejoramiento de la estabilidad social se consigue mediante la disminución de la iniciativa personal. La casta social resuelve el problema de encontrar uno su lugar en la industria, pero también reduce claramente el desarrollo del individuo e impide prácticamente la cooperación social.

\par
%\textsuperscript{(793.6)}
\textsuperscript{70:8.14} Como las clases de la sociedad se han formado de manera natural, continuarán existiendo hasta que el hombre consiga eliminarlas gradualmente por evolución mediante la manipulación inteligente de los recursos biológicos, intelectuales y espirituales de una civilización en progreso, tales como:

\par
%\textsuperscript{(793.7)}
\textsuperscript{70:8.15} 1. La renovación biológica de los linajes raciales ---la eliminación selectiva de las cepas humanas inferiores. Esto tenderá a erradicar muchas desigualdades humanas.

\par
%\textsuperscript{(793.8)}
\textsuperscript{70:8.16} 2. La formación educativa de la mayor capacidad cerebral que surgirá de este mejoramiento biológico.

\par
%\textsuperscript{(793.9)}
\textsuperscript{70:8.17} 3. La estimulación religiosa de los sentimientos de parentesco y de fraternidad humanos.

\par
%\textsuperscript{(793.10)}
\textsuperscript{70:8.18} Pero estas medidas sólo pueden dar sus verdaderos frutos en los lejanos milenios del futuro, aunque la manipulación inteligente, sabia y \textit{paciente} de estos factores aceleradores del progreso cultural producirá inmediatamente muchas mejoras sociales. La religión es la palanca poderosa que levanta a la civilización por encima del caos, pero se encuentra impotente sin el punto de apoyo de una mente sana y normal, que descanse firmemente sobre una herencia sana y normal.

\section*{9. Los derechos humanos}
\par
%\textsuperscript{(793.11)}
\textsuperscript{70:9.1} La naturaleza no le confiere ningún derecho al hombre; sólo le concede la vida y un mundo donde vivirla. La naturaleza ni siquiera le confiere el derecho de vivir, tal como se puede deducir si consideramos lo que le sucedería probablemente a un hombre desarmado que se encontrara frente a frente con un tigre hambriento en un bosque primitivo. El don fundamental que la sociedad le otorga al hombre es la seguridad.

\par
%\textsuperscript{(793.12)}
\textsuperscript{70:9.2} La sociedad ha afirmado gradualmente sus derechos y, en el momento presente, son los siguientes:

\par
%\textsuperscript{(793.13)}
\textsuperscript{70:9.3} 1. La seguridad en el abastecimiento de los alimentos.

\par
%\textsuperscript{(793.14)}
\textsuperscript{70:9.4} 2. La defensa militar ---la seguridad mediante el estado de preparación.

\par
%\textsuperscript{(793.15)}
\textsuperscript{70:9.5} 3. La conservación de la paz interna ---la prevención de la violencia personal y del desorden social.

\par
%\textsuperscript{(794.1)}
\textsuperscript{70:9.6} 4. El control sexual ---el matrimonio, la institución de la familia.

\par
%\textsuperscript{(794.2)}
\textsuperscript{70:9.7} 5. La propiedad ---el derecho de poseer.

\par
%\textsuperscript{(794.3)}
\textsuperscript{70:9.8} 6. El fomento de la competitividad entre los individuos y los grupos.

\par
%\textsuperscript{(794.4)}
\textsuperscript{70:9.9} 7. Las disposiciones para educar y formar a la juventud.

\par
%\textsuperscript{(794.5)}
\textsuperscript{70:9.10} 8. La promoción del intercambio y del comercio ---el desarrollo industrial.

\par
%\textsuperscript{(794.6)}
\textsuperscript{70:9.11} 9. El mejoramiento de las condiciones y las remuneraciones de los trabajadores.

\par
%\textsuperscript{(794.7)}
\textsuperscript{70:9.12} 10. La garantía de la libertad de las prácticas religiosas para que la motivación espiritual pueda exaltar todas estas otras actividades sociales.

\par
%\textsuperscript{(794.8)}
\textsuperscript{70:9.13} Cuando los derechos son tan antiguos que no se conocen sus orígenes, a menudo se denominan \textit{derechos naturales}. Pero los derechos humanos no son realmente naturales; son enteramente sociales. Son relativos y cambian continuamente, pues no son más que las reglas del juego ---los ajustes admitidos en las relaciones que gobiernan los fenómenos siempre cambiantes de la competitividad humana.

\par
%\textsuperscript{(794.9)}
\textsuperscript{70:9.14} Aquello que se puede considerar como un derecho en una época, puede que no lo sea en otra. La supervivencia de un gran número de personas anormales y degeneradas no se debe a que tengan el derecho natural de sobrecargar la civilización del siglo veinte, sino simplemente porque la sociedad de la época, las costumbres, lo decretan así.

\par
%\textsuperscript{(794.10)}
\textsuperscript{70:9.15} La Edad Media europea reconocía pocos derechos humanos; todo hombre pertenecía entonces a algún otro, y los derechos no eran más que privilegios o favores concedidos por la iglesia o el Estado. La sublevación contra este error fue igualmente un error, ya que condujo a la creencia de que todos los hombres nacen iguales.

\par
%\textsuperscript{(794.11)}
\textsuperscript{70:9.16} Los débiles y los inferiores siempre han luchado por tener los mismos derechos que los demás; siempre han insistido para que el Estado obligue a los fuertes y superiores a satisfacer sus necesidades y a compensar de otras maneras aquellas carencias que son muy a menudo el resultado natural de su propia indiferencia e indolencia.

\par
%\textsuperscript{(794.12)}
\textsuperscript{70:9.17} Pero este ideal de igualdad es el fruto de la civilización; no se encuentra en la naturaleza. La cultura misma demuestra también de manera concluyente la desigualdad intrínseca que existe entre los hombres mediante el hecho de que poseen unas capacidades muy desiguales para asimilarla. La realización repentina y no evolutiva de una supuesta igualdad natural haría retroceder rápidamente al hombre civilizado a las costumbres rudimentarias de las épocas primitivas. La sociedad no puede ofrecer los mismos derechos a todos, pero puede comprometerse a administrar los derechos variables de cada uno con justicia y equidad. La sociedad tiene la obligación y el deber de proporcionar a los hijos de la naturaleza una oportunidad justa y pacífica para luchar por su autopreservación, para participar en su autoperpetuación, y para disfrutar al mismo tiempo de cierto grado de satisfacción, ya que la suma de estos tres factores constituye la felicidad humana.

\section*{10. La evolución de la justicia}
\par
%\textsuperscript{(794.13)}
\textsuperscript{70:10.1} La justicia natural es una teoría elaborada por el hombre; no es una realidad. En la naturaleza, la justicia es puramente teórica, totalmente ficticia. La naturaleza sólo proporciona una clase de justicia ---la adaptación inevitable de los resultados a las causas.

\par
%\textsuperscript{(794.14)}
\textsuperscript{70:10.2} La justicia, tal como la conciben los hombres, significa conseguir sus derechos, y por eso es una cuestión de evolución progresiva. El concepto de justicia puede muy bien formar parte constitutiva de una mente dotada de espíritu, pero no nace plenamente desarrollado en los mundos del espacio.

\par
%\textsuperscript{(794.15)}
\textsuperscript{70:10.3} El hombre primitivo atribuía todos los fenómenos a una persona. En caso de muerte, el salvaje no se preguntaba \textit{qué} lo había matado, sino \textit{quién}. Por consiguiente, el homicidio accidental no se reconocía, y cuando se castigaba un crimen, no se tenía en cuenta en absoluto el móvil del criminal; la sentencia se pronunciaba de acuerdo con los daños ocasionados.

\par
%\textsuperscript{(795.1)}
\textsuperscript{70:10.4} En las sociedades más primitivas, la opinión pública actuaba de manera directa; no se necesitaban agentes de la ley. En la vida primitiva no había ninguna intimidad. Los vecinos de un hombre eran responsables de su conducta; tenían pues derecho a entrometerse en sus asuntos personales. La sociedad estaba reglamentada sobre la teoría de que los miembros de un grupo debían interesarse por la conducta de cada individuo, y tener cierto grado de control sobre ella.

\par
%\textsuperscript{(795.2)}
\textsuperscript{70:10.5} Muy pronto se creyó que los fantasmas administraban la justicia por medio de los curanderos y los sacerdotes; estos grupos se constituyeron así en los primeros detectives y agentes de la ley. Sus métodos primitivos para descubrir los crímenes consistían en utilizar las ordalías del veneno, el fuego y el dolor. Estos suplicios salvajes no eran más que unas técnicas rudimentarias de arbitraje; no resolvían necesariamente las controversias de manera justa. Por ejemplo: cuando se administraba un veneno, si el acusado lo vomitaba, era inocente.

\par
%\textsuperscript{(795.3)}
\textsuperscript{70:10.6} El Antiguo Testamento relata una de estas ordalías, una prueba de culpabilidad matrimonial: Si un hombre sospechaba que su esposa le era infiel, la llevaba ante el sacerdote y exponía sus sospechas, después de lo cual el sacerdote preparaba un brebaje compuesto de agua bendita y barreduras del suelo del templo. Después de la debida ceremonia, que incluía maldiciones amenazadoras, a la esposa acusada se le hacía beber la repugnante pócima. Si era culpable, <<el agua que causa la maldición entrará en ella y se volverá amarga, y su vientre se hinchará, y sus muslos se pudrirán, y la mujer será maldita para su pueblo>>\footnote{\textit{Prueba de fidelidad antigua}: Nm 5:12-31.}. Si, por casualidad, alguna mujer podía beber este inmundo brebaje sin manifestar síntomas de enfermedad física, era absuelta de las acusaciones presentadas por su marido celoso.

\par
%\textsuperscript{(795.4)}
\textsuperscript{70:10.7} Casi todas las tribus en evolución practicaron en una época u otra estos métodos atroces para detectar los crímenes. Batirse en duelo es una supervivencia moderna del juicio por medio de las ordalías.

\par
%\textsuperscript{(795.5)}
\textsuperscript{70:10.8} No tiene nada de extraño que los hebreos y otras tribus semicivilizadas practicaran hace tres mil años estas técnicas primitivas para administrar la justicia, pero es sumamente asombroso que unos hombres racionales conservaran posteriormente esta reliquia de la barbarie en las páginas de una colección de escritos sagrados. La simple reflexión debería clarificar que ningún ser divino ha dado nunca al hombre mortal unas instrucciones tan injustas sobre la detección y el juicio de unas supuestas infidelidades matrimoniales.

\par
%\textsuperscript{(795.6)}
\textsuperscript{70:10.9} La sociedad adoptó pronto la actitud de pagar con represalias: ojo por ojo\footnote{\textit{Ojo por ojo}: Ex 21:23-24; Lv 24:20; Dt 19:21; Mt 5:38.}, vida por vida\footnote{\textit{Vida por vida}: Ex 21:23.}. Todas las tribus en evolución reconocieron este derecho a la venganza sangrienta\footnote{\textit{Venganza de sangre}: Dt 19:6,12; Jos 20:3-9.}. La venganza se convirtió en la meta de la vida primitiva, pero desde entonces la religión ha modificado considerablemente estas prácticas tribales iniciales. Los instructores de la religión revelada siempre han proclamado: <<`La venganza es mía', dice el Señor>>\footnote{\textit{La venganza es de Dios}: Sal 94:1; Is 35:4; Nah 1:2; Dt 32:35,41,43.}. Los asesinatos por venganza de los tiempos primitivos no eran tan diferentes de los que se cometen en la actualidad con el pretexto de la ley no escrita.

\par
%\textsuperscript{(795.7)}
\textsuperscript{70:10.10} El suicidio era una forma corriente de represalia. Si un hombre era incapaz de vengarse durante su vida, moría con la creencia de que podría volver como fantasma y descargar su ira sobre su enemigo. Puesto que esta creencia estaba generalizada, la amenaza de suicidarse en el umbral de un enemigo era habitualmente suficiente para hacerlo ceder. El hombre primitivo no apreciaba mucho la vida; el suicidio por nimiedades era corriente, pero las enseñanzas de los dalamatianos redujeron mucho esta costumbre, mientras que en los tiempos más recientes, el ocio, las comodidades, la religión y la filosofía se han unido para hacer la vida más agradable y más deseable. Sin embargo, las huelgas de hambre suponen la analogía moderna de estos métodos antiguos de represalias.

\par
%\textsuperscript{(796.1)}
\textsuperscript{70:10.11} Una de las primeras formulaciones de la ley tribal en progreso consistió en asumir la enemistad sangrienta como un asunto de la tribu. Pero por extraño que parezca, incluso entonces un hombre podía matar a su esposa sin ser castigado, a condición de que la hubiera pagado íntegramente. Sin embargo, los esquimales actuales permiten todavía que la familia perjudicada sea la que pronuncie y administre el castigo por un crimen, incluso si se trata de un asesinato.

\par
%\textsuperscript{(796.2)}
\textsuperscript{70:10.12} Otro progreso consistió en la imposición de multas por violar los tabúes, en la estipulación de penas pecuniarias. Estas multas constituyeron las primeras rentas públicas. La costumbre de pagar el <<dinero compensatorio>>\footnote{\textit{Dinero compensatorio}: Ex 21:28.} también se puso de moda como sustituto de la venganza sangrienta. Estos daños se pagaban habitualmente en mujeres o en ganado; transcurrió mucho tiempo antes de que se impusieran unas multas reales, una compensación monetaria, como castigo por los crímenes. Puesto que la idea de castigo era esencialmente la de una compensación, todas las cosas, incluyendo la vida humana, terminaron por tener un precio que se podía pagar como daños y perjuicios. Los hebreos fueron los primeros que abolieron la práctica de pagar dinero a la familia de una víctima de asesinato. Moisés les enseñó que no debían <<aceptar ninguna compensación a cambio de la vida de un asesino que fuera culpable de haber matado; será ejecutado con toda seguridad>>\footnote{\textit{No aceptar compensación por el asesinato}: Nm 35:31.}.

\par
%\textsuperscript{(796.3)}
\textsuperscript{70:10.13} Así pues, la justicia fue administrada primero por la familia, luego por el clan y más tarde por la tribu. La administración de la verdadera justicia data del momento en que a los grupos privados y emparentados se les privó de la venganza para depositarla en manos del grupo social, del Estado.

\par
%\textsuperscript{(796.4)}
\textsuperscript{70:10.14} El castigo consistente en quemar vivo a alguien fue en otro tiempo una práctica común. Muchos jefes antiguos lo admitieron, incluyendo a Hamurabi y Moisés; éste último ordenó que muchos crímenes, en particular los de naturaleza sexual grave, se castigaran quemando al culpable en la hoguera. Si <<la hija de un sacerdote>> o de otro ciudadano importante se dedicaba a la prostitución pública, los hebreos tenían la costumbre de <<quemarla en el fuego>>\footnote{\textit{Quemar a la adúltera en el fuego}: Gn 38:24; Lv 21:9.}.

\par
%\textsuperscript{(796.5)}
\textsuperscript{70:10.15} La traición ---el hecho de <<vender>> o traicionar a los miembros de la tribu--- fue el primer crimen capital. El robo de ganado se castigaba universalmente con una ejecución sumaria, e incluso recientemente el robo de caballos se ha castigado de manera similar. Pero con el paso del tiempo se aprendió que la severidad del castigo no era tan válida para disuadir a los criminales, como la certidumbre y la rapidez en su ejecución.

\par
%\textsuperscript{(796.6)}
\textsuperscript{70:10.16} Cuando una sociedad no consigue castigar los crímenes, el resentimiento colectivo se afirma generalmente bajo la forma de linchamiento; el establecimiento de refugios fue un medio de eludir esta cólera colectiva repentina. El linchamiento y el batirse en duelo representan la resistencia del individuo a ceder su desagravio privado al Estado.

\section*{11. Las leyes y los tribunales}
\par
%\textsuperscript{(796.7)}
\textsuperscript{70:11.1} Hacer distinciones nítidas entre las costumbres y las leyes es tan difícil como indicar en qué momento exacto del amanecer el día sucede a la noche. Las costumbres son las leyes y los reglamentos policiales en gestación. Cuando las costumbres no definidas llevan mucho tiempo establecidas, tienden a cristalizarse en leyes precisas, en reglas concretas y en convenciones sociales bien definidas.

\par
%\textsuperscript{(796.8)}
\textsuperscript{70:11.2} Al principio, la ley siempre es negativa y prohibitiva; en las civilizaciones que progresan se va volviendo cada vez más positiva y directiva. La sociedad primitiva funcionaba de manera negativa; concedía al individuo el derecho de vivir, imponiendo a todos los demás el mandamiento de <<no matarás>>\footnote{\textit{No matarás}: Ex 20:13; Dt 5:17.}. Toda concesión de derechos o de libertades a un individuo implica una reducción de las libertades de todos los demás, y esto se lleva a cabo mediante el tabú, la ley primitiva. Toda la idea del tabú es intrínsecamente negativa, pues la organización de la sociedad primitiva era totalmente negativa, y la administración primitiva de la justicia consistía en la aplicación de los tabúes. Pero al principio, estas leyes sólo se aplicaban a los miembros de la tribu, tal como está ilustrado en los hebreos de los tiempos posteriores, que tenían un código ético diferente para tratar con los gentiles.

\par
%\textsuperscript{(797.1)}
\textsuperscript{70:11.3} El juramento tuvo su origen en los tiempos de Dalamatia en un esfuerzo por hacer que los testimonios fueran más verídicos. Estos juramentos consistían en pronunciar una maldición sobre sí mismo. En los tiempos pasados, ningún individuo quería testificar en contra de su grupo nativo.

\par
%\textsuperscript{(797.2)}
\textsuperscript{70:11.4} El crimen era un ataque a las costumbres de la tribu, el pecado era la transgresión de aquellos tabúes que gozaban de la aprobación de los fantasmas, y existió una larga confusión debido a que no se lograba separar el crimen del pecado.

\par
%\textsuperscript{(797.3)}
\textsuperscript{70:11.5} El interés personal estableció el tabú sobre el asesinato, la sociedad lo santificó como costumbre tradicional, mientras que la religión consagró esta costumbre como ley moral, y las tres cosas contribuyeron así a hacer la vida humana más segura y sagrada. La sociedad no habría podido mantenerse unida durante los primeros tiempos si los derechos no hubieran tenido la aprobación de la religión; la superstición fue la policía moral y social de las largas épocas evolutivas. Todos los antiguos afirmaban que los dioses habían dado a sus antepasados las viejas leyes que poseían, los tabúes.

\par
%\textsuperscript{(797.4)}
\textsuperscript{70:11.6} La ley es un registro codificado de la larga experiencia humana, la opinión pública cristalizada y legalizada. Las costumbres fueron la materia prima de la experiencia acumulada, a partir de la cual las inteligencias dirigentes posteriores formularon las leyes escritas. Los jueces antiguos no tenían leyes. Cuando anunciaban una decisión, decían simplemente: <<Es la costumbre>>\footnote{\textit{Es la costumbre}: Jer 32:11; Jue 11:39.}.

\par
%\textsuperscript{(797.5)}
\textsuperscript{70:11.7} En los fallos de los tribunales, la referencia a la jurisprudencia representa el esfuerzo de los jueces por adaptar las leyes escritas a las condiciones cambiantes de la sociedad. Esto asegura una adaptación progresiva a las condiciones sociales cambiantes, unido al carácter impresionante de la continuidad tradicional.

\par
%\textsuperscript{(797.6)}
\textsuperscript{70:11.8} Los litigios sobre la propiedad se trataban de muchas maneras, tales como:

\par
%\textsuperscript{(797.7)}
\textsuperscript{70:11.9} 1. Destruyendo la propiedad en discusión.

\par
%\textsuperscript{(797.8)}
\textsuperscript{70:11.10} 2. Por la fuerza ---los contendientes luchaban hasta llegar a una decisión.

\par
%\textsuperscript{(797.9)}
\textsuperscript{70:11.11} 3. Por medio del arbitraje ---una tercera persona decidía.

\par
%\textsuperscript{(797.10)}
\textsuperscript{70:11.12} 4. Apelando a los ancianos ---y más tarde a los tribunales.

\par
%\textsuperscript{(797.11)}
\textsuperscript{70:11.13} Los primeros tribunales fueron encuentros pugilísticos reglamentados; los jueces no eran más que unos árbitros. Se encargaban de que la pelea se desarrollara de acuerdo con las reglas aprobadas. Antes de emprender un combate ante los tribunales, cada una de las partes entregaba una fianza al juez para pagar los gastos y la multa después de que uno hubiera sido derrotado por el otro. <<La fuerza era todavía el derecho>>. Más adelante, los argumentos verbales sustituyeron a los golpes físicos.

\par
%\textsuperscript{(797.12)}
\textsuperscript{70:11.14} Todo el concepto de la justicia primitiva no consistía tanto en ser justo como en arreglar la controversia e impedir así el desorden público y la violencia privada. Pero el hombre primitivo no experimentaba mucho resentimiento por lo que hoy se consideraría como una injusticia; se daba por sentado que los que tenían el poder lo utilizarían de manera egoísta. No obstante, la categoría de cualquier civilización se puede determinar con mucha exactitud analizando la minuciosidad y la equidad de sus tribunales, y la integridad de sus jueces.

\section*{12. La asignación de la autoridad civil}
\par
%\textsuperscript{(797.13)}
\textsuperscript{70:12.1} En la evolución del gobierno, la gran lucha ha estado relacionada con la concentración del poder. Los administradores del universo han aprendido por experiencia que los pueblos evolutivos de los mundos habitados están mejor reglamentados por un gobierno civil de tipo representativo, cuando se mantiene un equilibrio de poder adecuado entre las ramas ejecutiva, legislativa y judicial bien coordinadas.

\par
%\textsuperscript{(798.1)}
\textsuperscript{70:12.2} Aunque la autoridad primitiva estaba basada en la fuerza, en el poder físico, el gobierno ideal es el sistema representativo donde la jefatura está basada en la capacidad; pero en los tiempos de la barbarie, había demasiadas guerras como para permitir que un gobierno representativo funcionara de manera eficaz. En la larga lucha entre la división de la autoridad y la unidad de mando, los dictadores fueron los que ganaron. Los poderes iniciales y difusos del consejo primitivo de ancianos se concentraron gradualmente en la persona de un monarca absoluto. Después de la llegada de los verdaderos reyes, los grupos de ancianos sobrevivieron como cuerpos consultivos casi legislativo-judiciales; más tarde aparecieron los cuerpos legislativos de carácter coordinado, y finalmente se establecieron los tribunales supremos de justicia, separados de los cuerpos legislativos.

\par
%\textsuperscript{(798.2)}
\textsuperscript{70:12.3} El rey hacía cumplir las costumbres, la ley original o no escrita. Más tarde hizo respetar los decretos legislativos, la cristalización de la opinión pública. La asamblea popular, como expresión de la opinión pública, apareció lentamente, pero supuso un gran progreso social.

\par
%\textsuperscript{(798.3)}
\textsuperscript{70:12.4} Los primeros reyes estaban enormemente limitados por las costumbres ---por la tradición o la opinión pública. En una época más reciente, algunas naciones de Urantia han codificado estas costumbres en unas bases documentales que sirven para gobernar.

\par
%\textsuperscript{(798.4)}
\textsuperscript{70:12.5} Los mortales de Urantia tienen derecho a la libertad; deben crear sus sistemas de gobierno; deben adoptar sus constituciones u otras cartas constitucionales de autoridad civil y de procedimiento administrativo. Una vez hecho esto, deben elegir como jefes del ejecutivo a sus compañeros más competentes y dignos. Sólo deben elegir como representantes en la rama legislativa a aquellas personas intelectual y moralmente cualificadas para desempeñar estas responsabilidades sagradas. Como jueces de sus tribunales superiores y supremos sólo deben escoger a aquellas personas que estén dotadas de una aptitud natural y que hayan adquirido sabiduría a través de una profunda experiencia.

\par
%\textsuperscript{(798.5)}
\textsuperscript{70:12.6} Después de haber elegido su carta constitucional de libertad, si los hombres quieren conservar su libertad deben tomar sus precauciones para que esa carta sea interpretada de manera sabia, inteligente y audaz, a fin de poder impedir:

\par
%\textsuperscript{(798.6)}
\textsuperscript{70:12.7} 1. La usurpación de un poder injustificado por parte de la rama ejecutiva o legislativa.

\par
%\textsuperscript{(798.7)}
\textsuperscript{70:12.8} 2. Las maquinaciones de los agitadores ignorantes y supersticiosos.

\par
%\textsuperscript{(798.8)}
\textsuperscript{70:12.9} 3. El retraso del progreso científico.

\par
%\textsuperscript{(798.9)}
\textsuperscript{70:12.10} 4. El estancamiento debido al predominio de la mediocridad.

\par
%\textsuperscript{(798.10)}
\textsuperscript{70:12.11} 5. La dominación ejercida por minorías corrompidas.

\par
%\textsuperscript{(798.11)}
\textsuperscript{70:12.12} 6. El control por parte de los aspirantes a dictadores ambiciosos y hábiles.

\par
%\textsuperscript{(798.12)}
\textsuperscript{70:12.13} 7. Los trastornos desastrosos debidos al pánico.

\par
%\textsuperscript{(798.13)}
\textsuperscript{70:12.14} 8. La explotación por parte de hombres sin escrúpulos.

\par
%\textsuperscript{(798.14)}
\textsuperscript{70:12.15} 9. La transformación de los ciudadanos en esclavos fiscales del Estado.

\par
%\textsuperscript{(798.15)}
\textsuperscript{70:12.16} 10. La falta de justicia social y económica.

\par
%\textsuperscript{(798.16)}
\textsuperscript{70:12.17} 11. La unión de la iglesia y el Estado.

\par
%\textsuperscript{(798.17)}
\textsuperscript{70:12.18} 12. La pérdida de la libertad personal.

\par
%\textsuperscript{(798.18)}
\textsuperscript{70:12.19} Éstos son los objetivos y las metas de los tribunales constitucionales que actúan como reguladores sobre los mecanismos de un gobierno representativo en un mundo evolutivo.

\par
%\textsuperscript{(799.1)}
\textsuperscript{70:12.20} La lucha de la humanidad por perfeccionar el gobierno en Urantia consiste en optimizar los canales de la administración, en adaptarlos a las necesidades corrientes en continuo cambio, en mejorar la distribución de los poderes dentro del gobierno, y luego en seleccionar a unos dirigentes administrativos que sean realmente sabios. Aunque existe una forma de gobierno divina e ideal, no podemos revelarla, sino que debe ser descubierta de manera lenta y laboriosa por los hombres y las mujeres de cada planeta en todos los universos del tiempo y del espacio.

\par
%\textsuperscript{(799.2)}
\textsuperscript{70:12.21} [Presentado por un Melquisedek de Nebadon.]


\chapter{Documento 71. El desarrollo del Estado}
\par
%\textsuperscript{(800.1)}
\textsuperscript{71:0.1} EL ESTADO es un desarrollo beneficioso de la civilización; representa el beneficio neto que la sociedad ha obtenido de los estragos y sufrimientos de la guerra. Incluso el arte de gobernar no es más que una acumulación de técnicas para ajustar las pruebas competitivas de fuerza entre las tribus y las naciones en lucha.

\par
%\textsuperscript{(800.2)}
\textsuperscript{71:0.2} El Estado moderno es la institución que ha sobrevivido a la larga lucha por el poder colectivo. Un poder superior ha prevalecido finalmente y ha dado nacimiento a una criatura de hecho ---el Estado--- junto con el mito moral de que el ciudadano tiene la obligación absoluta de vivir y morir por el Estado. Pero el Estado no tiene una génesis divina; ni siquiera ha sido causado por una acción humana volitivamente inteligente; es una institución puramente evolutiva y tuvo un origen totalmente automático.

\section*{1. El Estado embrionario}
\par
%\textsuperscript{(800.3)}
\textsuperscript{71:1.1} El Estado es una organización reguladora social y territorial, y el Estado más fuerte, más eficaz y más duradero está compuesto por una sola nación cuya población posee una lengua, unas costumbres y unas instituciones comunes.

\par
%\textsuperscript{(800.4)}
\textsuperscript{71:1.2} Los primeros Estados eran pequeños y todos fueron el resultado de las conquistas. No tuvieron su origen en las asociaciones voluntarias. Muchos fueron fundados por conquistadores nómadas que se precipitaban sobre los pastores pacíficos o los agricultores asentados para dominarlos y esclavizarlos. Estos Estados, productos de las conquistas, estaban forzosamente estratificados; las clases eran inevitables, y las luchas de clases siempre han sido selectivas.

\par
%\textsuperscript{(800.5)}
\textsuperscript{71:1.3} Las tribus nórdicas de hombres rojos americanos nunca consiguieron organizarse en un auténtico Estado. Nunca progresaron más allá de una vaga confederación de tribus, una forma de Estado muy primitiva. La que más se aproximó fue la federación de los iroqueses, pero este grupo de seis naciones nunca funcionó exactamente como un Estado, y no logró sobrevivir debido a la ausencia de ciertos elementos esenciales para la vida nacional moderna, tales como:

\par
%\textsuperscript{(800.6)}
\textsuperscript{71:1.4} 1. La adquisición y la herencia de la propiedad privada.

\par
%\textsuperscript{(800.7)}
\textsuperscript{71:1.5} 2. La existencia de ciudades, además de la agricultura y la industria.

\par
%\textsuperscript{(800.8)}
\textsuperscript{71:1.6} 3. Animales domésticos útiles.

\par
%\textsuperscript{(800.9)}
\textsuperscript{71:1.7} 4. Una organización familiar práctica. Estos hombres rojos se aferraban a la familia materna y a la herencia de tíos a sobrinos.

\par
%\textsuperscript{(800.10)}
\textsuperscript{71:1.8} 5. Un territorio definido.

\par
%\textsuperscript{(800.11)}
\textsuperscript{71:1.9} 6. Un jefe ejecutivo fuerte.

\par
%\textsuperscript{(800.12)}
\textsuperscript{71:1.10} 7. La esclavitud de los cautivos ---los adoptaban o los mataban en masa.

\par
%\textsuperscript{(800.13)}
\textsuperscript{71:1.11} 8. Unas conquistas decisivas.

\par
%\textsuperscript{(800.14)}
\textsuperscript{71:1.12} Los hombres rojos eran demasiado democráticos; tenían un buen gobierno, pero fracasó. Con el tiempo habrían desarrollado un Estado si no hubieran tropezado prematuramente con la civilización más avanzada del hombre blanco, que empleaba los métodos gubernamentales de los griegos y los romanos.

\par
%\textsuperscript{(801.1)}
\textsuperscript{71:1.13} El éxito del Estado romano estuvo basado en:

\par
%\textsuperscript{(801.2)}
\textsuperscript{71:1.14} 1. La familia patriarcal.

\par
%\textsuperscript{(801.3)}
\textsuperscript{71:1.15} 2. La agricultura y la domesticación de los animales.

\par
%\textsuperscript{(801.4)}
\textsuperscript{71:1.16} 3. La concentración de la población ---las ciudades.

\par
%\textsuperscript{(801.5)}
\textsuperscript{71:1.17} 4. La propiedad privada de las cosas y la tierra.

\par
%\textsuperscript{(801.6)}
\textsuperscript{71:1.18} 5. La esclavitud ---las clases de ciudadanos.

\par
%\textsuperscript{(801.7)}
\textsuperscript{71:1.19} 6. La conquista y la reorganización de los pueblos débiles y atrasados.

\par
%\textsuperscript{(801.8)}
\textsuperscript{71:1.20} 7. Un territorio definido y con carreteras.

\par
%\textsuperscript{(801.9)}
\textsuperscript{71:1.21} 8. Unos gobernantes personales y fuertes.

\par
%\textsuperscript{(801.10)}
\textsuperscript{71:1.22} La gran debilidad de la civilización romana, y uno de los factores que contribuyeron a la caída final del imperio, fue la disposición supuestamente liberal y avanzada de emancipar a los muchachos a los veintiún años, y de liberar incondicionalmente a las jóvenes para que tuvieran la libertad de casarse con un hombre de su propia elección o recorrer el país dedicándose a la inmoralidad. El perjuicio para la sociedad no provino de estas reformas mismas, sino más bien de la manera repentina y general en que fueron adoptadas. La caída de Roma demuestra lo que se puede esperar cuando un Estado experimenta una expansión demasiado rápida, acompañada de una degeneración interna.

\par
%\textsuperscript{(801.11)}
\textsuperscript{71:1.23} La decadencia de los lazos consanguíneos a favor de los lazos territoriales hizo posible el Estado embrionario, y en general las conquistas cimentaban firmemente estas federaciones tribales. Aunque la característica del verdadero Estado es una soberanía que trasciende todas las luchas menores y todas las diferencias entre los grupos, sin embargo muchas clases y castas sobreviven en las organizaciones estatales posteriores, como vestigios de los clanes y las tribus de los tiempos pasados. Los Estados territoriales posteriores más grandes sostuvieron una larga lucha encarnizada contra estos grupos de clanes consanguíneos más pequeños, y el gobierno tribal resultó ser una valiosa transición entre la autoridad familiar y la del Estado. En épocas más tardías, muchos clanes tuvieron su origen en las asociaciones de profesionales y en otras asociaciones laborales.

\par
%\textsuperscript{(801.12)}
\textsuperscript{71:1.24} Cuando el Estado no logra integrarse, se produce un retroceso a las técnicas gubernamentales que prevalecían antes de la existencia del Estado, como sucedió con el feudalismo de la Edad Media europea. Durante estos siglos de tinieblas, el Estado territorial se desplomó y se produjo una reversión a los grupos pequeños de los castillos, a la reaparición de las etapas de desarrollo del clan y de la tribu. Incluso ahora existen unos semi-Estados similares en Asia y África, pero no todos son unas reversiones evolutivas; muchos de ellos forman los núcleos embrionarios de los Estados del futuro.

\section*{2. La evolución del gobierno representativo}
\par
%\textsuperscript{(801.13)}
\textsuperscript{71:2.1} Aunque la democracia sea un ideal, es un producto de la civilización, no de la evolución. ¡Id despacio! ¡Elegid con cuidado! Porque los peligros de la democracia son los siguientes:

\par
%\textsuperscript{(801.14)}
\textsuperscript{71:2.2} 1. La glorificación de la mediocridad.

\par
%\textsuperscript{(801.15)}
\textsuperscript{71:2.3} 2. La elección de unos gobernantes viles e ignorantes.

\par
%\textsuperscript{(801.16)}
\textsuperscript{71:2.4} 3. La incapacidad para reconocer los hechos fundamentales de la evolución social.

\par
%\textsuperscript{(801.17)}
\textsuperscript{71:2.5} 4. El peligro de un sufragio universal en manos de unas mayorías incultas e indolentes.

\par
%\textsuperscript{(801.18)}
\textsuperscript{71:2.6} 5. La esclavitud a la opinión pública; la mayoría no siempre tiene razón.

\par
%\textsuperscript{(802.1)}
\textsuperscript{71:2.7} La opinión pública, la opinión común y corriente, siempre ha retrasado la sociedad; sin embargo, es valiosa porque aunque frena la evolución social, protege la civilización. La educación de la opinión pública es el único método efectivo y seguro para acelerar la civilización; la fuerza no es más que un recurso temporal, y el desarrollo cultural se acelerará cada vez más a medida que las balas cedan su lugar a las papeletas electorales. La opinión pública, las costumbres, es la energía básica y primordial para la evolución social y el desarrollo del Estado, pero para que tenga un valor estatal, tiene que expresarse de manera no violenta.

\par
%\textsuperscript{(802.2)}
\textsuperscript{71:2.8} La medida del progreso de una sociedad está directamente determinada por el grado en que la opinión pública puede controlar la conducta personal y la reglamentación estatal sin tener que recurrir a la violencia. El gobierno realmente civilizado apareció cuando la opinión pública fue investida de los poderes del derecho al voto personal. Las elecciones populares puede que no siempre decidan las cosas como es debido, pero representan la manera correcta de cometer incluso un error. La evolución no produce de inmediato una perfección superlativa, sino más bien un ajuste práctico comparativo y progresivo.

\par
%\textsuperscript{(802.3)}
\textsuperscript{71:2.9} La evolución de una forma práctica y eficaz de gobierno representativo comporta las diez fases o etapas siguientes:

\par
%\textsuperscript{(802.4)}
\textsuperscript{71:2.10} 1. \textit{La libertad de la persona}. La esclavitud, la servidumbre y todas las formas de cautiverio humano tienen que desaparecer.

\par
%\textsuperscript{(802.5)}
\textsuperscript{71:2.11} 2. \textit{La libertad de la mente}. A menos que un pueblo libre esté educado ---que le hayan enseñado a pensar con inteligencia y a hacer proyectos con sabiduría--- la libertad hace generalmente más daño que bien.

\par
%\textsuperscript{(802.6)}
\textsuperscript{71:2.12} 3. \textit{El reinado de la ley}. Sólo se puede disfrutar de la libertad cuando la voluntad y los caprichos de los gobernantes humanos son reemplazados por unos decretos legislativos conformes a la ley fundamental aceptada.

\par
%\textsuperscript{(802.7)}
\textsuperscript{71:2.13} 4. \textit{La libertad de expresión}. Un gobierno representativo es impensable si las aspiraciones y las opiniones humanas no tienen la libertad de expresarse de todas las formas..

\par
%\textsuperscript{(802.8)}
\textsuperscript{71:2.14} 5. \textit{La seguridad de la propiedad}. Ningún gobierno puede durar mucho tiempo si no logra asegurar el derecho a disfrutar, de alguna manera, de la propiedad personal. El hombre anhela tener el derecho de utilizar, controlar, conferir, vender, arrendar y legar su propiedad personal.

\par
%\textsuperscript{(802.9)}
\textsuperscript{71:2.15} 6. \textit{El derecho de petición}. Un gobierno representativo asume el derecho que tienen los ciudadanos a ser escuchados. El privilegio de la petición es inherente a la ciudadanía libre.

\par
%\textsuperscript{(802.10)}
\textsuperscript{71:2.16} 7. \textit{El derecho de gobernar}. No basta con ser escuchado; la fuerza de la petición debe ascender hasta la dirección misma del gobierno.

\par
%\textsuperscript{(802.11)}
\textsuperscript{71:2.17} 8. \textit{El sufragio universal}. Un gobierno representativo presupone un electorado inteligente, eficiente y universal. El carácter de un gobierno semejante siempre estará determinado por el carácter y la capacidad de aquellos que lo componen. A medida que progrese la civilización, aunque el sufragio siga siendo universal para ambos sexos, será eficazmente modificado, reagrupado y diferenciado de otras maneras.

\par
%\textsuperscript{(802.12)}
\textsuperscript{71:2.18} 9. \textit{El control de los funcionarios públicos}. Ningún gobierno civil será útil y eficaz a menos que los ciudadanos posean y utilicen unas técnicas acertadas para guiar y controlar a los titulares de los cargos públicos y a los funcionarios.

\par
%\textsuperscript{(802.13)}
\textsuperscript{71:2.19} 10. \textit{Unos representantes inteligentes y cualificados}. La supervivencia de la democracia depende del éxito del gobierno representativo, y este éxito está condicionado por la práctica de elegir únicamente para los cargos públicos a aquellas personas que estén técnicamente cualificadas, y sean intelectualmente competentes, socialmente leales y moralmente idóneas. El gobierno del pueblo, por el pueblo y para el pueblo sólo se puede conservar mediante estas disposiciones.

\section*{3. Los ideales del Estado}
\par
%\textsuperscript{(803.1)}
\textsuperscript{71:3.1} La forma política o administrativa de un gobierno tiene poca importancia con tal que proporcione los elementos esenciales del progreso civil: la libertad, la seguridad, la educación y la coordinación social. Lo que determina el curso de la evolución social es lo que el Estado hace, no lo que el Estado es. Después de todo, ningún Estado puede trascender los valores morales de sus ciudadanos, que se manifiestan en sus dirigentes escogidos. La ignorancia y el egoísmo aseguran la caída de cualquier gobierno, incluso del tipo más elevado.

\par
%\textsuperscript{(803.2)}
\textsuperscript{71:3.2} Por muy lamentable que sea, el egoísmo nacional ha sido esencial para la supervivencia social. La doctrina del pueblo elegido ha sido un factor primordial para unir a las tribus y edificar las naciones hasta los tiempos modernos. Pero ningún Estado puede alcanzar unos niveles ideales de funcionamiento hasta que todas las formas de intolerancia hayan sido dominadas; la intolerancia es la eterna enemiga del progreso humano. La mejor manera de combatirla es coordinando la ciencia, el comercio, las diversiones y la religión.

\par
%\textsuperscript{(803.3)}
\textsuperscript{71:3.3} El Estado ideal funciona con el impulso de tres poderosas fuerzas coordinadas:

\par
%\textsuperscript{(803.4)}
\textsuperscript{71:3.4} 1. Una lealtad amorosa nacida de la realización de la fraternidad humana.

\par
%\textsuperscript{(803.5)}
\textsuperscript{71:3.5} 2. Un patriotismo inteligente basado en unos ideales sabios.

\par
%\textsuperscript{(803.6)}
\textsuperscript{71:3.6} 3. Una perspicacia cósmica interpretada en función de los hechos, las necesidades y las metas planetarias.

\par
%\textsuperscript{(803.7)}
\textsuperscript{71:3.7} Las leyes del Estado ideal son poco numerosas; han dejado atrás la época negativa de los tabúes para entrar en la era del progreso positivo de una libertad individual que es consecuencia de un mejor autocontrol. Un Estado superior no solamente obliga a sus ciudadanos a trabajar, sino que también los incita a utilizar de manera provechosa y edificante el creciente tiempo libre que les proporciona la liberación de los trabajos agotadores, gracias a los progresos de una época de máquinas. El ocio debe producir además de consumir.

\par
%\textsuperscript{(803.8)}
\textsuperscript{71:3.8} Ninguna sociedad ha progresado mucho permitiendo la pereza o tolerando la miseria. Pero la pobreza y la dependencia nunca se podrán eliminar si se apoyan abundantemente los linajes defectuosos y degenerados, y se les permite que se reproduzcan sin restricción.

\par
%\textsuperscript{(803.9)}
\textsuperscript{71:3.9} Una sociedad moral debe aspirar a mantener la autoestima de sus ciudadanos, y proporcionar a todo individuo normal unas oportunidades adecuadas para autorrealizarse. Un proyecto así de realización social produciría una sociedad cultural del tipo más elevado. La evolución social debe ser estimulada por una supervisión gubernamental que ejerza un mínimo de control regulador. El mejor Estado es aquel que coordina más y gobierna menos.

\par
%\textsuperscript{(803.10)}
\textsuperscript{71:3.10} Los ideales del Estado deben alcanzarse por evolución, mediante el lento crecimiento de la conciencia cívica, el reconocimiento de que el servicio social es una obligación y un privilegio. Después del final de la administración de los oportunistas políticos, los hombres comienzan por asumir las cargas del gobierno como un deber, pero más tarde buscan este servicio como un privilegio, como el honor más grande. La capacidad de los ciudadanos que se ofrecen para aceptar las responsabilidades del Estado retrata fielmente la categoría de cualquier nivel de civilización.

\par
%\textsuperscript{(803.11)}
\textsuperscript{71:3.11} En un Estado auténtico de bien público, los expertos dirigen la tarea de gobernar las ciudades y las provincias, y éstas son administradas de la misma manera que todas las otras formas de asociaciones económicas y comerciales entre personas.

\par
%\textsuperscript{(803.12)}
\textsuperscript{71:3.12} En los Estados evolucionados, el servicio político es considerado como la entrega más elevada de los ciudadanos. La ambición suprema de los ciudadanos más sabios y nobles es conseguir el reconocimiento civil, ser elegido o nombrado para algún puesto gubernamental de confianza, y estos gobiernos confieren sus máximos honores, en reconocimiento por los servicios prestados, a sus funcionarios civiles y sociales. A continuación se conceden honores, en el orden que se menciona, a los filósofos, educadores, científicos, industriales y militares. A los padres se les recompensa debidamente por la excelencia de sus hijos; y como los dirigentes puramente religiosos son los embajadores de un reino espiritual, reciben sus verdaderas recompensas en otro mundo.

\section*{4. La civilización progresiva}
\par
%\textsuperscript{(804.1)}
\textsuperscript{71:4.1} La economía, la sociedad y el gobierno tienen que evolucionar si desean seguir existiendo. Las condiciones estáticas en un mundo evolutivo son signos de decadencia; sólo sobreviven aquellas instituciones que avanzan con la corriente evolutiva.

\par
%\textsuperscript{(804.2)}
\textsuperscript{71:4.2} El programa progresivo de una civilización en expansión abarca:

\par
%\textsuperscript{(804.3)}
\textsuperscript{71:4.3} 1. La conservación de las libertades individuales.

\par
%\textsuperscript{(804.4)}
\textsuperscript{71:4.4} 2. La protección del hogar.

\par
%\textsuperscript{(804.5)}
\textsuperscript{71:4.5} 3. La promoción de la seguridad económica.

\par
%\textsuperscript{(804.6)}
\textsuperscript{71:4.6} 4. La prevención de las enfermedades.

\par
%\textsuperscript{(804.7)}
\textsuperscript{71:4.7} 5. La educación obligatoria.

\par
%\textsuperscript{(804.8)}
\textsuperscript{71:4.8} 6. El empleo obligatorio.

\par
%\textsuperscript{(804.9)}
\textsuperscript{71:4.9} 7. La utilización provechosa del tiempo libre.

\par
%\textsuperscript{(804.10)}
\textsuperscript{71:4.10} 8. La asistencia a los desafortunados.

\par
%\textsuperscript{(804.11)}
\textsuperscript{71:4.11} 9. El mejoramiento de la raza.

\par
%\textsuperscript{(804.12)}
\textsuperscript{71:4.12} 10. El fomento de las ciencias y las artes.

\par
%\textsuperscript{(804.13)}
\textsuperscript{71:4.13} 11. El fomento de la filosofía ---la sabiduría.

\par
%\textsuperscript{(804.14)}
\textsuperscript{71:4.14} 12. El aumento de la perspicacia cósmica ---la espiritualidad.

\par
%\textsuperscript{(804.15)}
\textsuperscript{71:4.15} Estos progresos en las artes de la civilización conducen directamente a la realización de las metas humanas y divinas más elevadas que persiguen los mortales ---la consecución social de la fraternidad de los hombres y la situación personal de ser consciente de Dios, la cual se manifiesta en el deseo supremo de cada individuo de hacer la voluntad del Padre que está en los cielos.

\par
%\textsuperscript{(804.16)}
\textsuperscript{71:4.16} La aparición de la auténtica fraternidad significa que ha llegado un orden social en el que todos los hombres se complacen en llevar las cargas de los demás; desean practicar realmente la regla de oro. Pero esta sociedad ideal no se puede llevar a cabo mientras los débiles o los malvados estén al acecho para aprovecharse de manera injusta e impía de aquellos que se sienten impulsados principalmente por su dedicación al servicio de la verdad, la belleza y la bondad. En una situación así sólo existe un camino práctico: los seguidores de la regla de oro pueden establecer una sociedad progresiva en la que puedan vivir de acuerdo con sus ideales, manteniendo al mismo tiempo una defensa adecuada contra sus compañeros ignorantes, que podrían intentar, o bien explotar sus predilecciones pacíficas, o destruir su civilización en progreso.

\par
%\textsuperscript{(804.17)}
\textsuperscript{71:4.17} El idealismo nunca puede sobrevivir en un planeta evolutivo si los idealistas de cada generación se dejan exterminar por los grupos más abyectos de la humanidad. La gran prueba del idealismo es la siguiente: Una sociedad avanzada, ¿puede mantener un estado de preparación militar que la proteja de todos los ataques de sus vecinos belicosos, sin caer en la tentación de emplear esta fuerza militar en operaciones ofensivas contra otros pueblos para obtener beneficios egoístas o un engrandecimiento nacional? La supervivencia nacional exige un estado de preparación, y únicamente el idealismo religioso puede impedir que la preparación se prostituya y se convierta en agresión. Sólo el amor, la fraternidad, puede impedir que los fuertes opriman a los débiles.

\section*{5. La evolución de la competencia}
\par
%\textsuperscript{(805.1)}
\textsuperscript{71:5.1} La competencia es imprescindible para el progreso social, pero la competencia no regulada engendra violencia. En la sociedad actual, la competencia está desplazando lentamente a la guerra en la medida en que determina el lugar del individuo en la industria, al mismo tiempo que decreta la supervivencia de las industrias mismas. (El asesinato y la guerra ocupan lugares diferentes ante las costumbres; el asesinato fue declarado fuera de la ley desde los primeros días de la sociedad, mientras que la guerra nunca ha sido proscrita todavía por la totalidad de la humanidad.)

\par
%\textsuperscript{(805.2)}
\textsuperscript{71:5.2} Un Estado ideal no se encarga de regular la conducta social más que lo suficiente como para eliminar la violencia en la competencia entre los individuos e impedir la injusticia en la iniciativa personal. He aquí un gran problema para el Estado: ¿Cómo se puede garantizar la paz y la tranquilidad en la industria, pagar los impuestos para mantener el poder del Estado, y al mismo tiempo impedir que el sistema tributario obstaculice la industria y evitar que el Estado se vuelva parasitario o tiránico?

\par
%\textsuperscript{(805.3)}
\textsuperscript{71:5.3} Durante las épocas primitivas de un mundo cualquiera, la competencia es imprescindible para la civilización progresiva. A medida que progresa la evolución del hombre, la cooperación se vuelve cada vez más real. En las civilizaciones avanzadas, la cooperación es más eficaz que la competencia. La competencia estimula al hombre primitivo. La evolución primitiva está caracterizada por la supervivencia de los seres biológicamente capacitados, pero la mejor manera de fomentar las civilizaciones posteriores es a través de la cooperación inteligente, la fraternidad comprensiva y la hermandad espiritual.

\par
%\textsuperscript{(805.4)}
\textsuperscript{71:5.4} Es verdad que la competitividad en la industria es extremadamente despilfarradora y sumamente ineficaz, pero no se debería favorecer ningún intento por eliminar esta actividad económica desperdiciada, si tales ajustes ocasionan la más leve anulación de cualquiera de las libertades fundamentales del individuo.

\section*{6. El afán de lucro}
\par
%\textsuperscript{(805.5)}
\textsuperscript{71:6.1} La economía actual, motivada por el lucro, está condenada al fracaso a menos que los móviles del servicio se añadan a los móviles del lucro. La competencia implacable, basada en el egoísmo de miras estrechas, termina finalmente por destruir aquellas mismas cosas que pretendía conservar. La motivación que busca un beneficio exclusivo para sí mismo es incompatible con los ideales cristianos ---y mucho más con las enseñanzas de Jesús.

\par
%\textsuperscript{(805.6)}
\textsuperscript{71:6.2} En la economía, el móvil del lucro es con relación al móvil del servicio lo que, en la religión, el miedo es con relación al amor. Pero el afán de lucro no se debe destruir o eliminar de manera repentina; mantiene trabajando arduamente a muchos mortales que de otra manera serían perezosos. Sin embargo, no es necesario que los objetivos de este estimulador de la energía social sean permanentemente egoístas.

\par
%\textsuperscript{(805.7)}
\textsuperscript{71:6.3} En un tipo avanzado de sociedad, el afán de lucro en las actividades económicas es totalmente despreciable y enteramente indigno; sin embargo, es un factor indispensable durante todas las fases iniciales de la civilización. A los hombres no se les debe quitar el móvil del lucro hasta que posean firmemente unos móviles no lucrativos de tipo superior que puedan emplear en la competencia económica y en el servicio social ---la motivación trascendente de una sabiduría superlativa, una fraternidad fascinante y una consecución espiritual magnífica.

\section*{7. La educación}
\par
%\textsuperscript{(806.1)}
\textsuperscript{71:7.1} Un Estado duradero está basado en la cultura, dominado por los ideales y motivado por el servicio. La finalidad de la educación debería consistir en adquirir habilidad, buscar la sabiduría, desarrollar la individualidad y alcanzar los valores espirituales.

\par
%\textsuperscript{(806.2)}
\textsuperscript{71:7.2} En el Estado ideal, la educación continúa durante toda la vida, y la filosofía se convierte algunas veces en el objetivo principal de sus ciudadanos. Los ciudadanos de un Estado de bien público semejante buscan la sabiduría para comprender mejor el significado de las relaciones humanas, el sentido de la realidad, la nobleza de los valores, las metas de la vida y las glorias del destino cósmico.

\par
%\textsuperscript{(806.3)}
\textsuperscript{71:7.3} Los urantianos deberían tener una visión de una sociedad cultural nueva y superior. La educación se elevará a nuevos niveles de valor cuando desaparezca el sistema económico motivado puramente por el lucro. La educación ha sido demasiado tiempo provinciana, militarista, para exaltar el ego y buscar el éxito; con el tiempo deberá volverse mundial, idealista, para el desarrollo del individuo y la comprensión del cosmos.

\par
%\textsuperscript{(806.4)}
\textsuperscript{71:7.4} La educación ha pasado recientemente del control del clero al de los juristas y los hombres de negocios. Con el tiempo deberá ser confiada a los filósofos y a los científicos. Los educadores deben ser unos seres libres, unos auténticos dirigentes, para que la filosofía, la búsqueda de la sabiduría, pueda convertirse en el objetivo principal de la educación.

\par
%\textsuperscript{(806.5)}
\textsuperscript{71:7.5} La educación es la ocupación de la vida; debe continuar durante toda la vida para que la humanidad pueda experimentar gradualmente los niveles ascendentes de la sabiduría mortal, que son los siguientes:

\par
%\textsuperscript{(806.6)}
\textsuperscript{71:7.6} 1. El conocimiento de las cosas.

\par
%\textsuperscript{(806.7)}
\textsuperscript{71:7.7} 2. La comprensión de los significados.

\par
%\textsuperscript{(806.8)}
\textsuperscript{71:7.8} 3. La apreciación de los valores.

\par
%\textsuperscript{(806.9)}
\textsuperscript{71:7.9} 4. La nobleza del trabajo ---el deber.

\par
%\textsuperscript{(806.10)}
\textsuperscript{71:7.10} 5. La motivación de las metas ---la moralidad.

\par
%\textsuperscript{(806.11)}
\textsuperscript{71:7.11} 6. El amor al servicio ---el carácter.

\par
%\textsuperscript{(806.12)}
\textsuperscript{71:7.12} 7. La perspicacia cósmica ---el discernimiento espiritual.

\par
%\textsuperscript{(806.13)}
\textsuperscript{71:7.13} Luego, gracias a estos logros, muchas personas se elevarán hasta el nivel último que la mente humana puede alcanzar: la conciencia de Dios.

\section*{8. El carácter del Estado}
\par
%\textsuperscript{(806.14)}
\textsuperscript{71:8.1} La única característica sagrada de cualquier gobierno humano es la división del Estado en tres ámbitos, los de las funciones ejecutivas, legislativas y judiciales. El universo está administrado con arreglo a este plan que separa las funciones y la autoridad. Aparte de este concepto divino sobre la reglamentación social eficaz, o gobierno civil, poco importa la forma de Estado que un pueblo pueda elegir, con tal que los ciudadanos progresen siempre hacia la meta de un mayor autocontrol y un servicio social creciente. La agudeza intelectual, la sabiduría económica, la habilidad social y el vigor moral de un pueblo se reflejan fielmente en la categoría de su Estado.

\par
%\textsuperscript{(806.15)}
\textsuperscript{71:8.2} La evolución del Estado ocasiona un progreso de nivel en nivel, como sigue:

\par
%\textsuperscript{(806.16)}
\textsuperscript{71:8.3} 1. La creación de un gobierno triple, con sus ramas ejecutiva, legislativa y judicial.

\par
%\textsuperscript{(806.17)}
\textsuperscript{71:8.4} 2. La libertad de las actividades sociales, políticas y religiosas.

\par
%\textsuperscript{(807.1)}
\textsuperscript{71:8.5} 3. La abolición de todas las formas de esclavitud y de cautiverio humano.

\par
%\textsuperscript{(807.2)}
\textsuperscript{71:8.6} 4. La capacidad de los ciudadanos para controlar la recaudación de los impuestos.

\par
%\textsuperscript{(807.3)}
\textsuperscript{71:8.7} 5. El establecimiento de una educación universal ---una enseñanza que abarque desde la cuna hasta la tumba.

\par
%\textsuperscript{(807.4)}
\textsuperscript{71:8.8} 6. El ajuste adecuado entre los gobiernos locales y el gobierno nacional.

\par
%\textsuperscript{(807.5)}
\textsuperscript{71:8.9} 7. El fomento de la ciencia y la derrota de las enfermedades.

\par
%\textsuperscript{(807.6)}
\textsuperscript{71:8.10} 8. El debido reconocimiento de la igualdad entre los sexos y el funcionamiento coordinado de los hombres y las mujeres en el hogar, la escuela y la iglesia, con servicios femeninos especializados en la industria y el gobierno.

\par
%\textsuperscript{(807.7)}
\textsuperscript{71:8.11} 9. La eliminación de la esclavitud del trabajo duro mediante la invención de máquinas y el dominio posterior de la época mecánica.

\par
%\textsuperscript{(807.8)}
\textsuperscript{71:8.12} 10. La victoria sobre los dialectos ---el triunfo de una lengua universal.

\par
%\textsuperscript{(807.9)}
\textsuperscript{71:8.13} 11. El fin de las guerras ---las sentencias internacionales sobre las discrepancias nacionales y raciales serán emitidas por los tribunales continentales de naciones, presididos por un tribunal supremo planetario reclutado automáticamente entre los presidentes de los tribunales continentales que se jubilan periódicamente. Los tribunales continentales tienen autoridad; el tribunal mundial es consultivo ---moral.

\par
%\textsuperscript{(807.10)}
\textsuperscript{71:8.14} 12. La tendencia mundial a buscar la sabiduría ---la exaltación de la filosofía. La evolución de una religión mundial, que presagiará la entrada del planeta en las fases iniciales del establecimiento en la luz y la vida.

\par
%\textsuperscript{(807.11)}
\textsuperscript{71:8.15} Éstos son los requisitos previos para un gobierno progresivo y las marcas distintivas de un Estado ideal. Urantia está lejos de hacer realidad estos ideales elevados, pero las razas civilizadas han empezado a caminar ---la humanidad está en marcha hacia unos destinos evolutivos superiores.

\par
%\textsuperscript{(807.12)}
\textsuperscript{71:8.16} [Patrocinado por un Melquisedek de Nebadon.]


\chapter{Documento 72. Un gobierno en un planeta vecino}
\par
%\textsuperscript{(808.1)}
\textsuperscript{72:0.1} CON el permiso de Lanaforge y la aprobación de los Altísimos de Edentia, estoy autorizado para describir algunos aspectos de la vida social, moral y política de la raza humana más avanzada que vive en un planeta no muy alejado que pertenece al sistema de Satania.

\par
%\textsuperscript{(808.2)}
\textsuperscript{72:0.2} De todos los mundos de Satania que fueron aislados por haber participado en la rebelión de Lucifer, este planeta es el que ha experimentado una historia más similar a la de Urantia. La similitud entre las dos esferas explica sin duda por qué se concedió el permiso para que se hiciera esta exposición extraordinaria, ya que es muy poco frecuente que los gobernantes del sistema permitan que los asuntos de un planeta se relaten en otro.

\par
%\textsuperscript{(808.3)}
\textsuperscript{72:0.3} Este planeta, al igual que Urantia, fue descarriado por la deslealtad de su Príncipe Planetario en conexión con la rebelión de Lucifer. Recibió un Hijo Material poco después de la llegada de Adán a Urantia, y este Hijo tampoco cumplió con su deber, quedando la esfera aislada puesto que nunca se ha otorgado un Hijo Magistral a sus razas mortales.

\section*{1. La nación continental}
\par
%\textsuperscript{(808.4)}
\textsuperscript{72:1.1} A pesar de todas estas desventajas planetarias, una civilización muy superior está evolucionando en un continente aislado que tiene aproximadamente el tamaño de Australia. Esta nación contiene unos 140 millones de habitantes. Su población es de raza mixta, con predominio de las razas azul y amarilla, teniendo una proporción de sangre violeta ligeramente superior a la llamada raza blanca de Urantia. Estas diferentes razas aún no se han mezclado por completo, pero fraternizan y se relacionan socialmente de manera muy aceptable. La duración media de la vida en este continente es ahora de noventa años, un quince por ciento superior a la de cualquier otro pueblo del planeta.

\par
%\textsuperscript{(808.5)}
\textsuperscript{72:1.2} El mecanismo industrial de esta nación disfruta de una gran ventaja debido a la topografía excepcional de su continente. Las altas montañas, sobre las que llueve torrencialmente durante ocho meses al año, están situadas en el centro mismo del país. Esta disposición natural favorece el empleo de la energía hidráulica y facilita enormemente el riego de la cuarta parte occidental más árida del continente.

\par
%\textsuperscript{(808.6)}
\textsuperscript{72:1.3} Este pueblo es autosuficiente, es decir, que puede vivir de manera indefinida sin importar nada de las naciones circundantes. Sus recursos naturales son abundantes, y han aprendido mediante técnicas científicas la manera de compensar sus carencias en elementos esenciales para la vida. Disfrutan de un comercio interior muy activo, pero tienen poco comercio exterior debido a la hostilidad universal de sus vecinos menos progresivos.

\par
%\textsuperscript{(808.7)}
\textsuperscript{72:1.4} Esta nación continental siguió, en términos generales, la tendencia evolutiva del planeta: Su desarrollo desde la etapa tribal hasta la aparición de unos jefes y reyes poderosos duró miles de años. A los monarcas absolutos les siguieron muchos tipos de gobiernos diferentes ---las repúblicas frustradas, los estados comunales y los dictadores entraron y salieron en una profusión interminable. Este crecimiento continuó hasta hace aproximadamente quinientos años cuando, durante un período de fermentación política, uno de los poderosos triunviros-dictadores de la nación cambió de idea. Se ofreció a abdicar voluntariamente a condición de que uno de los otros gobernantes, el más vil de los dos que quedaban, renunciara también a su dictadura. De esta manera la soberanía del continente quedó depositada entre las manos de un solo gobernante. El Estado unificado progresó más de cien años bajo un fuerte régimen monárquico, y durante este período se confeccionó una carta magistral de libertades.

\par
%\textsuperscript{(809.1)}
\textsuperscript{72:1.5} La transición posterior entre la monarquía y una forma de gobierno representativo se produjo de manera gradual; los reyes permanecieron como simples figuras sociales o sentimentales, y finalmente desaparecieron cuando se extinguió la línea de sus descendientes varones. La república actual existe ahora desde hace exactamente doscientos años, durante los cuales ha progresado continuamente hacia las técnicas gubernamentales que estamos a punto de describir. Los últimos desarrollos en los ámbitos industrial y político se han efectuado en el transcurso de la década pasada.

\section*{2. La organización política}
\par
%\textsuperscript{(809.2)}
\textsuperscript{72:2.1} Esta nación continental posee ahora un gobierno representativo con una capital nacional situada en el centro del país. El gobierno central consiste en una sólida federación de cien Estados relativamente libres. Estos Estados eligen a sus gobernadores y legisladores por diez años, y ninguno de ellos puede ser reelegido. Los jueces estatales son nombrados de por vida por los gobernadores y confirmados por sus asambleas legislativas, que están compuestas de un representante por cada cien mil ciudadanos.

\par
%\textsuperscript{(809.3)}
\textsuperscript{72:2.2} Existen cinco tipos diferentes de gobiernos urbanos, dependiendo de las dimensiones de la ciudad, pero a ninguna ciudad se le permite sobrepasar el millón de habitantes. En general, estos modelos de gobiernos municipales son muy sencillos, directos y económicos. Los pocos cargos públicos de la administración urbana son muy anhelados por los tipos de ciudadanos más elevados.

\par
%\textsuperscript{(809.4)}
\textsuperscript{72:2.3} El gobierno federal contiene tres divisiones coordinadas: la ejecutiva, la legislativa y la judicial. El jefe del ejecutivo federal es elegido cada seis años por sufragio territorial universal. No puede ser reelegido salvo a petición de un mínimo de setenta y cinco asambleas legislativas estatales y la aprobación de sus gobernadores estatales respectivos, y en este caso sólo por un mandato más. Recibe el asesoramiento de un supergabinete que está compuesto por todos los antiguos jefes del ejecutivo que viven todavía.

\par
%\textsuperscript{(809.5)}
\textsuperscript{72:2.4} La división legislativa abarca tres cámaras:

\par
%\textsuperscript{(809.6)}
\textsuperscript{72:2.5} 1. \textit{La cámara alta} es elegida por los grupos de trabajadores de la industria, las profesiones liberales, la agricultura y otros oficios, y votan según su función económica.

\par
%\textsuperscript{(809.7)}
\textsuperscript{72:2.6} 2. \textit{La cámara baja} es elegida por ciertas organizaciones de la sociedad que abarcan a los grupos sociales, políticos y filosóficos no incluídos en la industria o en las otras profesiones. Todos los ciudadanos de buena reputación participan en la elección de las dos clases de representantes, pero se agrupan de manera diferente dependiendo de que la elección se refiera a la cámara alta o a la cámara baja.

\par
%\textsuperscript{(809.8)}
\textsuperscript{72:2.7} 3. \textit{La tercera cámara} ---los ancianos estadistas--- engloba a los veteranos del servicio cívico e incluye a muchas personas ilustres nombradas por el jefe del ejecutivo, por los jefes ejecutivos regionales (subfederales), por el presidente del tribunal supremo y por los funcionarios que presiden cada una de las otras cámaras legislativas. Este grupo está limitado a cien personas, y sus miembros son elegidos por el voto mayoritario de los mismos ancianos estadistas. El nombramiento es de por vida, y cuando se produce una vacante, se elige debidamente a la persona que figura en la lista de candidatos y que recibe el mayor número de votos. Las competencias de este cuerpo son puramente consultivas, pero es un poderoso regulador de la opinión pública y ejerce una gran influencia sobre todas las ramas del gobierno.

\par
%\textsuperscript{(810.1)}
\textsuperscript{72:2.8} Una gran parte del trabajo administrativo federal es realizado por las diez autoridades regionales (subfederales), consistiendo cada una de ellas en la asociación de diez estados. Estas divisiones regionales son totalmente ejecutivas y administrativas, careciendo de funciones legislativas y judiciales. Los diez jefes ejecutivos regionales son nombrados personalmente por el jefe del ejecutivo federal, y la duración de sus mandatos coincide con la del suyo propio ---seis años. El tribunal federal supremo aprueba el nombramiento de estos diez ejecutivos regionales, y aunque no pueden ser reelegidos, el ejecutivo saliente se convierte automáticamente en el asociado y consejero de su sucesor. Por otra parte, estos jefes regionales eligen sus propios gabinetes de funcionarios administrativos.

\par
%\textsuperscript{(810.2)}
\textsuperscript{72:2.9} La justicia se administra en esta nación mediante dos sistemas principales de tribunales ---los tribunales de justicia y los tribunales socioeconómicos. Los tribunales de justicia funcionan en los tres niveles siguientes:

\par
%\textsuperscript{(810.3)}
\textsuperscript{72:2.10} 1. \textit{Los tribunales menores} con jurisdicción local y municipal, cuyas decisiones pueden ser apeladas ante los tribunales estatales superiores.

\par
%\textsuperscript{(810.4)}
\textsuperscript{72:2.11} 2. \textit{Los tribunales supremos estatales}, cuyas decisiones son definitivas en todas las cuestiones que no afectan al gobierno federal o pongan en peligro los derechos y las libertades de los ciudadanos. Los ejecutivos regionales están facultados para llevar inmediatamente cualquier caso ante el tribunal federal supremo.

\par
%\textsuperscript{(810.5)}
\textsuperscript{72:2.12} 3. \textit{El tribunal federal supremo} ---el alto tribunal que juzga las controversias nacionales y los casos apelados procedentes de los tribunales estatales. Este tribunal supremo está compuesto de doce hombres mayores de cuarenta años y menores de setenta y cinco, que han servido dos años o más en algún tribunal estatal, y que han sido nombrados para este alto cargo por el jefe del ejecutivo con la aprobación mayoritaria del supergabinete y de la tercera cámara de la asamblea legislativa. Todas las decisiones que toma este órgano judicial supremo necesitan al menos dos tercios de los votos.

\par
%\textsuperscript{(810.6)}
\textsuperscript{72:2.13} Los tribunales socioeconómicos funcionan en las tres divisiones siguientes:

\par
%\textsuperscript{(810.7)}
\textsuperscript{72:2.14} 1. \textit{Los tribunales de los padres}, que están asociados con las divisiones legislativa y ejecutiva del sistema familiar y social.

\par
%\textsuperscript{(810.8)}
\textsuperscript{72:2.15} 2. \textit{Los tribunales de la enseñanza} ---los órganos jurídicos conectados con los sistemas escolares de los Estados y las regiones, y asociados con las ramas ejecutiva y legislativa del mecanismo administrativo de la enseñanza.

\par
%\textsuperscript{(810.9)}
\textsuperscript{72:2.16} 3. \textit{Los tribunales de la industria} ---los tribunales jurisdiccionales investidos con plena autoridad para arreglar todos los malentendidos económicos.

\par
%\textsuperscript{(810.10)}
\textsuperscript{72:2.17} El tribunal federal supremo no juzga los casos socioeconómicos, a menos que así lo decidan las tres cuartas partes de los votos de la tercera rama legislativa del gobierno nacional, la cámara de los ancianos estadistas. Por lo demás, todas las decisiones de los altos tribunales de los padres, de la enseñanza y de la industria son definitivas.

\section*{3. La vida de familia}
\par
%\textsuperscript{(811.1)}
\textsuperscript{72:3.1} En este continente, la ley prohíbe que dos familias vivan bajo el mismo techo. Puesto que las viviendas colectivas han sido proscritas, la mayoría de las casas de vecindad se han demolido. Pero los solteros viven todavía en los clubes, los hoteles y otras viviendas colectivas. El solar más pequeño que se permite para una vivienda debe tener unos cuatro mil seiscientos metros cuadrados de tierra. Todos los terrenos y otras propiedades destinados a viviendas están libres de impuestos hasta diez veces más de la superficie mínima permitida para una vivienda.

\par
%\textsuperscript{(811.2)}
\textsuperscript{72:3.2} La vida de familia de este pueblo ha mejorado enormemente durante el último siglo. Es obligatorio que tanto los padres como las madres asistan a las escuelas de puericultura para padres. Incluso los agricultores que residen en los pueblecitos del campo siguen estos cursos por correspondencia, desplazándose hasta los centros de instrucción oral más cercanos una vez cada diez días ---cada dos semanas, pues la semana es de cinco días.

\par
%\textsuperscript{(811.3)}
\textsuperscript{72:3.3} Cada familia tiene una media de cinco hijos y éstos permanecen bajo la completa autoridad de sus padres, o en caso de fallecimiento de uno de ellos o de los dos, bajo la de los tutores designados por los tribunales de padres. Cualquier familia considera como un gran honor que se le conceda la tutela de un huérfano de padre y madre. Los padres se presentan a unas oposiciones y el huérfano es adjudicado al hogar de aquellos que muestran las mejores aptitudes paternales.

\par
%\textsuperscript{(811.4)}
\textsuperscript{72:3.4} Este pueblo considera el hogar como la institución fundamental de su civilización. Se espera que los padres proporcionen a sus hijos, en el hogar, la parte más valiosa de su educación y de la formación de su carácter, y los padres consagran casi tanta atención como las madres a la cultura de sus hijos.

\par
%\textsuperscript{(811.5)}
\textsuperscript{72:3.5} Los padres o los tutores legítimos imparten en el hogar toda la educación sexual. Los profesores ofrecen la enseñanza moral durante los períodos de descanso en los talleres escuela, pero no sucede lo mismo con la educación religiosa, que se estima que es el privilegio exclusivo de los padres, pues la religión es considerada como una parte integrante de la vida familiar. La enseñanza puramente religiosa sólo se imparte públicamente en los templos de filosofía, pues entre estas gentes no se han desarrollado unas instituciones exclusivamente religiosas como las iglesias de Urantia. En su filosofía, la religión es el esfuerzo por conocer a Dios y por manifestar el amor a los semejantes a través del servicio, pero esto no es característico de la condición religiosa de las otras naciones de este planeta. Para este pueblo, la religión es un asunto tan completamente familiar que no existen lugares públicos consagrados exclusivamente a las asambleas religiosas. Como suele decirse en Urantia, la iglesia y el Estado están, políticamente, totalmente separados, pero existe una extraña superposición entre la religión y la filosofía.

\par
%\textsuperscript{(811.6)}
\textsuperscript{72:3.6} Hasta hace veinte años, los instructores espirituales
(comparables a los pastores de Urantia) que visitan periódicamente cada familia para examinar a los niños y comprobar si sus padres los han instruido de manera adecuada, estaban bajo la supervisión del gobierno. Estos consejeros y examinadores espirituales están ahora bajo la dirección de la Fundación del Progreso Espiritual, una institución recién creada y sostenida por aportaciones voluntarias. Es posible que esta institución no evolucione más hasta después de la llegada de un Hijo Magistral del Paraíso.

\par
%\textsuperscript{(811.7)}
\textsuperscript{72:3.7} Los niños permanecen sometidos legalmente a sus padres hasta la edad de quince años, momento en que tiene lugar su primera iniciación a las responsabilidades cívicas. Después, cada cinco años y durante cinco períodos sucesivos, se celebran unos ejercicios públicos similares para estos grupos de la misma edad, durante los cuales disminuyen sus obligaciones hacia los padres, mientras que asumen nuevas responsabilidades cívicas y sociales hacia el Estado. El derecho al voto se confiere a los veinte años, el derecho a casarse sin el consentimiento de los padres no se concede hasta los veinticinco años, y los hijos deben abandonar el hogar cuando llegan a la edad de treinta años.

\par
%\textsuperscript{(812.1)}
\textsuperscript{72:3.8} Las leyes del matrimonio y del divorcio son uniformes en toda la nación. El matrimonio antes de los veinte años ---la edad de la emancipación civil--- no está permitido. El permiso para casarse sólo se concede un año después de haber anunciado la intención de hacerlo, y después de que el novio y la novia han presentado los certificados que demuestran que han sido debidamente instruidos en las escuelas de padres acerca de las responsabilidades de la vida conyugal.

\par
%\textsuperscript{(812.2)}
\textsuperscript{72:3.9} Los reglamentos del divorcio son poco exigentes, pero la sentencia de separación que emite el tribunal de padres no se puede obtener hasta un año después de haberse registrado la solicitud, y los años de este planeta son considerablemente más largos que los de Urantia. A pesar de estas leyes que facilitan el divorcio, el índice actual de divorcios sólo es la décima parte del de las razas civilizadas de Urantia.

\section*{4. El sistema educativo}
\par
%\textsuperscript{(812.3)}
\textsuperscript{72:4.1} El sistema educativo de esta nación es obligatorio y mixto en las escuelas preuniversitarias a las que los estudiantes asisten desde la edad de cinco años hasta los dieciocho. Estas escuelas son muy diferentes a las de Urantia. No hay aulas, se estudia una sola materia a la vez, y después de los tres primeros años, todos los alumnos se convierten en profesores auxiliares, enseñando a los que están por debajo de ellos. Los libros sólo se utilizan para conseguir la información que ayude a resolver los problemas que surgen en los talleres escuela y en las granjas escuela. En estos talleres se produce una gran parte de los muebles que se utilizan en el continente y numerosos aparatos mecánicos ---es una gran época de inventos y de mecanización. Al lado de cada taller se encuentra una biblioteca laboral donde los estudiantes pueden consultar los libros de referencia necesarios. Durante todo el período educativo también se enseña la agricultura y la horticultura en las grandes granjas que lindan con todas las escuelas locales.

\par
%\textsuperscript{(812.4)}
\textsuperscript{72:4.2} A los débiles mentales sólo se les enseña la agricultura y la ganadería, y son internados de por vida en unas colonias tutelares especiales, donde se les separa por sexos para impedir la procreación, que está prohibida para todos los subnormales. Estas medidas restrictivas están en vigor desde hace setenta y cinco años; las sentencias de reclusión son promulgadas por los tribunales de padres.

\par
%\textsuperscript{(812.5)}
\textsuperscript{72:4.3} Todo el mundo coge un mes de vacaciones por año. El año tiene diez meses; las escuelas preuniversitarias funcionan durante nueve meses, y las vacaciones se pasan viajando con los padres o los amigos. Estos viajes forman parte del programa de educación de adultos y continúan durante toda la vida; los fondos para sufragar estos gastos se acumulan de la misma manera que los que se emplean para las pensiones de jubilación.

\par
%\textsuperscript{(812.6)}
\textsuperscript{72:4.4} Una cuarta parte del tiempo escolar se dedica a los juegos ---a las competiciones atléticas--- y los estudiantes progresan desde estos concursos locales, luego estatales y regionales, hasta las pruebas nacionales de habilidad y de proezas. Los concursos oratorios y musicales, así como los de ciencia y filosofía, ocupan igualmente la atención de los estudiantes desde las divisiones sociales inferiores hasta las competiciones con honores nacionales.

\par
%\textsuperscript{(812.7)}
\textsuperscript{72:4.5} La dirección de las escuelas es una réplica del gobierno nacional, con sus tres ramas correlacionadas, y el profesorado funciona como la tercera división legislativa, o consultiva. En este continente, el objetivo principal de la educación es hacer de cada alumno un ciudadano económicamente independiente.

\par
%\textsuperscript{(813.1)}
\textsuperscript{72:4.6} Todos los jóvenes que salen diplomados del sistema escolar preuniversitario a los dieciocho años son unos expertos artesanos. Entonces empieza el estudio de los libros y la búsqueda de los conocimientos especiales, ya sea en las escuelas de adultos o bien en las universidades. Cuando un estudiante brillante termina su trabajo antes de tiempo, se le concede como recompensa el tiempo y los medios para que pueda llevar a cabo algún proyecto favorito de su propia invención. Todo el sistema educativo está diseñado para formar adecuadamente al individuo.

\section*{5. La organización industrial}
\par
%\textsuperscript{(813.2)}
\textsuperscript{72:5.1} La situación industrial de este pueblo está muy lejos de sus ideales; el capital y los trabajadores tienen todavía sus conflictos, pero los dos se van ajustando a un proyecto de cooperación sincera. En este continente excepcional, los trabajadores se están convirtiendo cada vez más en los accionistas de todas las empresas industriales; todo trabajador inteligente se transforma lentamente en un pequeño capitalista.

\par
%\textsuperscript{(813.3)}
\textsuperscript{72:5.2} Los antagonismos sociales disminuyen, y la buena voluntad aumenta rápidamente. La abolición de la esclavitud (hace más de cien años) no ha provocado ningún problema económico grave, ya que esta adaptación se realizó gradualmente liberando cada año el dos por ciento de los esclavos. Aquellos esclavos que superaron satisfactoriamente unas pruebas físicas, mentales y morales, obtuvieron la cuidadanía; una gran parte de estos esclavos superiores eran prisioneros de guerra o hijos de estos cautivos. Esta nación deportó hace unos cincuenta años a los últimos esclavos inferiores, y aún más recientemente ha emprendido la tarea de reducir el número de las clases degeneradas y viciosas.

\par
%\textsuperscript{(813.4)}
\textsuperscript{72:5.3} Este pueblo ha desarrollado recientemente unas nuevas técnicas para solucionar los malentendidos industriales y para corregir los abusos económicos; representan unas mejoras apreciables frente a los antiguos métodos que empleaban para resolver estos problemas. La violencia ha sido proscrita como procedimiento para arreglar las discrepancias personales o industriales. Los salarios, los beneficios y otros problemas económicos no están rígidamente regulados, pero en general están controlados por los cuerpos legislativos industriales, mientras que todos los conflictos que surgen en la industria se deciden en los tribunales de la industria.

\par
%\textsuperscript{(813.5)}
\textsuperscript{72:5.4} Los tribunales de la industria sólo tienen treinta años de existencia, pero funcionan de manera muy satisfactoria. El progreso más reciente estipula que desde ahora en adelante los tribunales de la industria reconocerán que las remuneraciones legales están contempladas en tres divisiones:

\par
%\textsuperscript{(813.6)}
\textsuperscript{72:5.5} 1. Los tipos legales de interés sobre el capital invertido.

\par
%\textsuperscript{(813.7)}
\textsuperscript{72:5.6} 2. Los salarios razonables para los especialistas empleados en las obras industriales.

\par
%\textsuperscript{(813.8)}
\textsuperscript{72:5.7} 3. Los sueldos justos y equitativos para los obreros.

\par
%\textsuperscript{(813.9)}
\textsuperscript{72:5.8} Al principio, estas remuneraciones se pagarán con arreglo a un contrato, pero ante una disminución de los beneficios, compartirán una reducción transitoria proporcional. A partir de entonces, todos los beneficios que superen estas cargas fijas se considerarán como dividendos, y se repartirán proporcionalmente entre las tres divisiones indicadas: capital, especialistas y obreros.

\par
%\textsuperscript{(813.10)}
\textsuperscript{72:5.9} Los jefes ejecutivos regionales adaptan y decretan cada diez años las horas legales de trabajo diario remunerado. La industria funciona actualmente a base de semanas de cinco días, trabajando cuatro de ellos y descansando uno. Esta gente trabaja seis horas cada día laborable y, al igual que los estudiantes, durante nueve meses de los diez que tiene el año. Las vacaciones las suelen pasar viajando, y como recientemente se han desarrollado nuevos medios de transporte, toda la nación tiende a viajar. El clima favorece los viajes durante unos ocho meses al año, y los habitantes aprovechan al máximo sus oportunidades.

\par
%\textsuperscript{(813.11)}
\textsuperscript{72:5.10} Hace doscientos años, la industria estaba completamente dominada por el afán de lucro, pero hoy está siendo reemplazado rápidamente por otros impulsos superiores. La competencia es fuerte en este continente, pero una gran parte de ella se ha transferido de la industria a los juegos, a la destreza, a las realizaciones científicas y a los logros intelectuales. Está muy activa en los servicios sociales y en la lealtad al gobierno. Entre esta gente, el servicio público se está convirtiendo rápidamente en la meta principal de la ambición. El hombre más rico del continente trabaja seis horas diarias en la oficina de su taller mecánico, y luego se apresura a ir a la rama local de la escuela para estadistas, donde intenta capacitarse para el servicio público.

\par
%\textsuperscript{(814.1)}
\textsuperscript{72:5.11} El trabajo está siendo mejor considerado en este continente, y todos los ciudadanos sanos de más de dieciocho años trabajan o bien en su casa y en las granjas, o en alguna industria reconocida, o en las obras públicas que absorben a los desempleados temporales, o bien en el cuerpo de trabajadores obligatorios en las minas.

\par
%\textsuperscript{(814.2)}
\textsuperscript{72:5.12} Esta gente también ha empezado a experimentar una nueva forma de repugnancia social ---la repugnancia por la ociosidad así como por la riqueza inmerecida. Están venciendo a sus máquinas de manera lenta pero segura. Ellos también lucharon en otro tiempo por la libertad política y posteriormente por la libertad económica. Ahora comienzan a disfrutar de las dos y además empiezan a apreciar sus ratos de ocio bien merecidos, los cuales pueden dedicarlos a autorrealizarse cada vez más.

\section*{6. El seguro de vejez}
\par
%\textsuperscript{(814.3)}
\textsuperscript{72:6.1} Esta nación está haciendo un esfuerzo decidido por reemplazar el tipo de caridad destructora de la autoestima por unas garantías de seguridad para la vejez basadas en unos seguros gubernamentales dignos. Esta nación proporciona una educación a todos los niños y un trabajo a todos los hombres, por lo que puede llevar a cabo con éxito este sistema de seguros que protege a los enfermizos y a los ancianos.

\par
%\textsuperscript{(814.4)}
\textsuperscript{72:6.2} En esta nación, todas las personas tienen que jubilarse de los trabajos remunerados a los sesenta y cinco años de edad, a menos que obtengan un permiso del comisario estatal de trabajo que les dé derecho a seguir trabajando hasta los setenta años. Este límite de edad no se aplica a los funcionarios públicos ni a los filósofos. Los discapacitados físicos o los lisiados permanentes pueden ser inscritos en la lista de jubilados a cualquier edad, necesitándose una orden judicial ratificada por el comisario de pensiones del gobierno regional.

\par
%\textsuperscript{(814.5)}
\textsuperscript{72:6.3} Los fondos para las pensiones de vejez proceden de cuatro fuentes:

\par
%\textsuperscript{(814.6)}
\textsuperscript{72:6.4} 1. El gobierno federal requisa el sueldo de un día por mes con esta finalidad, y en este país todo el mundo trabaja.

\par
%\textsuperscript{(814.7)}
\textsuperscript{72:6.5} 2. Los legados ---muchos ciudadanos ricos entregan fondos con esta finalidad.

\par
%\textsuperscript{(814.8)}
\textsuperscript{72:6.6} 3. Los salarios del trabajo obligatorio en las minas del Estado. Después de que los trabajadores reclutados se mantienen a sí mismos y apartan las cuotas para su propia jubilación, todo los excedentes de los beneficios de su trabajo son entregados para este fondo de pensiones.

\par
%\textsuperscript{(814.9)}
\textsuperscript{72:6.7} 4. Los ingresos de los recursos naturales. El gobierno federal posee como depósito social todas las riquezas naturales del continente, y los ingresos de éstas se utilizan con fines sociales tales como la prevención de las enfermedades, la educación de los genios y los gastos de los individuos especialmente prometedores que estudian en las escuelas para estadistas. La mitad de los ingresos de los recursos naturales se destina al fondo de pensiones para la vejez.

\par
%\textsuperscript{(814.10)}
\textsuperscript{72:6.8} Aunque las fundaciones actuariales estatales y regionales proporcionan muchas formas de seguros protectores, las pensiones de vejez son administradas exclusivamente por el gobierno federal a través de los diez departamentos regionales.

\par
%\textsuperscript{(814.11)}
\textsuperscript{72:6.9} Estos fondos gubernamentales se han administrado honradamente desde hace mucho tiempo. Después de la traición y el asesinato, los castigos más severos que imponen los tribunales recaen sobre la traición a la confianza pública. La deslealtad social y política es ahora considerada como el más atroz de todos los crímenes.

\section*{7. El sistema tributario}.
\par
%\textsuperscript{(815.1)}
\textsuperscript{72:7.1} El gobierno federal sólo es paternalista en la administración de las pensiones para la vejez y en la promoción del talento y de la originalidad creativa; los gobiernos estatales se interesan un poco más por el ciudadano individual, mientras que los gobiernos locales son mucho más paternalistas o socialistas. La ciudad (o alguna de sus subdivisiones) se ocupa de los asuntos tales como la salud, la higiene, el urbanismo, el embellecimiento, el suministro de agua, el alumbrado, la calefacción, el esparcimiento, la música y las comunicaciones.

\par
%\textsuperscript{(815.2)}
\textsuperscript{72:7.2} En todas las industrias, la primera preocupación es la salud; ciertas fases del bienestar físico son consideradas como prerrogativas de la industria y de la comunidad, pero los problemas de la salud individual y familiar son cuestiones de interés exclusivamente personal. En la medicina, al igual que en todos los demás asuntos puramente personales, el plan del gobierno consiste en abstenerse cada vez más de intervenir.

\par
%\textsuperscript{(815.3)}
\textsuperscript{72:7.3} Las ciudades no tienen el poder de imponer tributos, y tampoco pueden contraer deudas. Reciben una subvención per cápita de la tesorería del Estado, y estos ingresos deben completarlos con los beneficios de sus empresas socializadas y mediante la concesión de licencias para las diversas actividades comerciales.

\par
%\textsuperscript{(815.4)}
\textsuperscript{72:7.4} Los servicios de ferrocarriles metropolitanos, que permiten ampliar considerablemente los límites de la ciudad, se encuentran bajo el control municipal. Las fundaciones de protección y seguros contra incendios son las que mantienen a los cuerpos de bomberos urbanos, y todos los edificios de la ciudad o del campo están a prueba de incendios ---lo han estado desde hace más de setenta y cinco años.

\par
%\textsuperscript{(815.5)}
\textsuperscript{72:7.5} No existen agentes del orden público nombrados por los municipios; los cuerpos de policía son mantenidos por los gobiernos estatales. Los agentes de este departamento se reclutan casi exclusivamente entre los solteros de veinticinco a cincuenta años. La mayor parte de los Estados grava a los solteros con unos impuestos más bien importantes, pero todos los hombres que entran en la policía estatal están exonerados de pagarlos. En los Estados de tipo medio, el cuerpo de policía sólo tiene ahora una décima parte de los efectivos que tenía hace cincuenta años.

\par
%\textsuperscript{(815.6)}
\textsuperscript{72:7.6} Los sistemas tributarios de los cien Estados relativamente libres y soberanos tienen poca o ninguna uniformidad entre sí, ya que las condiciones económicas y de otro tipo varían enormemente en los diferentes sectores del continente. Cada Estado posee diez disposiciones constitucionales fundamentales que no se pueden modificar, salvo con el consentimiento del tribunal federal supremo, y uno de estos artículos impide que se pueda exigir un impuesto de más del uno por ciento por año sobre el valor de una propiedad cualquiera, y los solares urbanos o rurales para viviendas están exentos.

\par
%\textsuperscript{(815.7)}
\textsuperscript{72:7.7} El gobierno federal no puede contraer deudas, y para que un Estado pueda pedir un préstamo se necesita un referéndum con la mayoría de las tres cuartas partes de los votos, salvo por razones de guerra. Puesto que el gobierno federal no puede endeudarse, en caso de guerra el Consejo de la Defensa Nacional está facultado para exigir a los Estados que entreguen dinero, así como hombres y materiales, a medida que se necesiten. Pero ninguna deuda puede permanecer sin saldarse durante más de veinticinco años.

\par
%\textsuperscript{(815.8)}
\textsuperscript{72:7.8} Los ingresos destinados a sostener al gobierno federal proceden de las cinco fuentes siguientes:

\par
%\textsuperscript{(815.9)}
\textsuperscript{72:7.9} 1. \textit{Los derechos de importación}. Todas las importaciones están sujetas a un arancel destinado a proteger el nivel de vida de este continente, que es mucho más elevado que el de cualquier otra nación del planeta. El tribunal superior de la industria es el que establece estos aranceles después de que las dos cámaras del congreso industrial han ratificado las recomendaciones del jefe ejecutivo de asuntos económicos, el cual es nombrado conjuntamente por estos dos cuerpos legislativos. La cámara alta industrial es elegida por los trabajadores, y la cámara baja por los capitalistas.

\par
%\textsuperscript{(816.1)}
\textsuperscript{72:7.10} 2. \textit{Los derechos de autor}. El gobierno federal estimula la invención y las creaciones originales en los diez laboratorios regionales, ayudando a todos los tipos de genios ---artistas, autores y científicos--- y protegiendo sus patentes. El gobierno se queda a cambio con la mitad de los beneficios procedentes de todos estos inventos y creaciones, ya se trate de máquinas, libros, obras de arte, plantas o animales.

\par
%\textsuperscript{(816.2)}
\textsuperscript{72:7.11} 3. \textit{El impuesto sobre sucesiones}. El gobierno federal percibe un impuesto gradual sobre la herencia, que varía entre el uno y el cincuenta por ciento, dependiendo del tamaño de la fortuna así como de otras condiciones.

\par
%\textsuperscript{(816.3)}
\textsuperscript{72:7.12} 4. \textit{El equipo militar}. El gobierno gana una cantidad considerable con el arrendamiento de los equipos militares y navales para usos comerciales y recreativos.

\par
%\textsuperscript{(816.4)}
\textsuperscript{72:7.13} 5. \textit{Los recursos naturales}. Los ingresos procedentes de los recursos naturales, cuando no se necesitan en su totalidad para los fines específicos designados en la carta del Estado federal, se ingresan en el tesoro nacional.

\par
%\textsuperscript{(816.5)}
\textsuperscript{72:7.14} Las asignaciones federales, excepto los fondos de guerra gravados por el Consejo de la Defensa Nacional, se originan en la cámara legislativa alta, se acuerdan en la cámara baja, reciben la aprobación del jefe del ejecutivo, y son validadas finalmente por la comisión presupuestaria federal de los cien. Los cien miembros de esta comisión son nombrados por los gobernadores de los Estados y elegidos por los cuerpos legislativos estatales para prestar sus servicios durante veinticuatro años, eligiéndose a una cuarta parte de ellos cada seis años. Este cuerpo escoge como presidente a uno de sus miembros cada seis años por una mayoría de las tres cuartas partes de los votos, convirtiéndose de este modo en el director-controlador de la tesorería federal.

\section*{8. Los colegios especiales}
\par
%\textsuperscript{(816.6)}
\textsuperscript{72:8.1} Además del programa de educación básica obligatoria que se extiende desde los cinco hasta los dieciocho años, las escuelas especiales están organizadas como sigue:

\par
%\textsuperscript{(816.7)}
\textsuperscript{72:8.2} 1. \textit{Las escuelas para estadistas}. Estas escuelas son de tres clases: nacionales, regionales y estatales. Las oficinas públicas de la nación están agrupadas en cuatro divisiones. La primera división del servicio público está relacionada principalmente con la administración nacional, y todos los funcionarios de este grupo tienen que ser diplomados de las escuelas para estadistas tanto regionales como nacionales. En la segunda división, los individuos pueden aceptar un cargo político, electivo o por nombramiento después de haberse diplomado en cualquiera de las diez escuelas regionales para estadistas; su trabajo está relacionado con las responsabilidades de la administración regional y de los gobiernos estatales. La tercera división incluye las responsabilidades estatales, y a estos funcionarios sólo se les exige que posean un título estatal de estadista. Los funcionarios de la cuarta y última división no necesitan un título de estadista, pues todos sus cargos son de libre designación. Representan los puestos menores de auxiliares, secretarios y técnicos, y son desempeñados por los miembros de las diversas profesiones liberales que trabajan en calidad de administradores gubernamentales.

\par
%\textsuperscript{(816.8)}
\textsuperscript{72:8.3} Los jueces de los tribunales menores y estatales poseen un título de las escuelas estatales para estadistas. Los jueces de los tribunales jurisdiccionales para asuntos sociales, educativos e industriales poseen un título de las escuelas regionales. Los jueces del tribunal federal supremo deben estar licenciados en todas estas escuelas para estadistas.

\par
%\textsuperscript{(817.1)}
\textsuperscript{72:8.4} 2. \textit{Las escuelas de filosofía}. Estas escuelas están afiliadas a los templos de filosofía y están más o menos asociadas con la religión como función pública.

\par
%\textsuperscript{(817.2)}
\textsuperscript{72:8.5} 3. \textit{Las instituciones científicas}. Estas escuelas técnicas se encuentran más coordinadas con la industria que con el sistema educativo, y están administradas en quince divisiones.

\par
%\textsuperscript{(817.3)}
\textsuperscript{72:8.6} 4. \textit{Las escuelas de formación profesional}. Estas instituciones especiales proporcionan la formación técnica de las diversas profesiones liberales, las cuales son doce en total.

\par
%\textsuperscript{(817.4)}
\textsuperscript{72:8.7} 5. \textit{Las escuelas militares y navales}. Cerca del cuartel general nacional y en los veinticinco centros militares costeros están en funcionamiento unas instituciones dedicadas a la preparación militar de los ciudadanos voluntarios entre dieciocho y treinta años de edad. Los menores de veinticinco años necesitan el consentimiento de los padres para ser admitidos en estas escuelas.

\section*{9. El sistema del sufragio universal}
\par
%\textsuperscript{(817.5)}
\textsuperscript{72:9.1} Aunque todos los cargos públicos están reservados para los candidatos diplomados en las escuelas para estadistas tanto estatales como regionales o federales, los dirigentes progresivos de esta nación descubrieron un defecto grave en su sistema de sufragio universal, y hace unos cincuenta años prepararon una disposición constitucional para adoptar un sistema de votación modificado que contiene las características siguientes:

\par
%\textsuperscript{(817.6)}
\textsuperscript{72:9.2} 1. Cada hombre y cada mujer de más de veinte años posee un voto. Cuando llegan a esta edad, todos los ciudadanos tienen que aceptar pertenecer a dos grupos de votantes: Se inscribirán en el primero de acuerdo con su función económica ---industrial, profesional, agrícola o comercial; y entrarán en el segundo grupo según sus inclinaciones políticas, filosóficas y sociales. Todos los trabajadores pertenecen así a algún grupo electoral económico, y al igual que las asociaciones no económicas, estos gremios poseen unos reglamentos muy similares a los del gobierno nacional con su triple división de poderes. La inscripción en estos grupos no se puede cambiar durante doce años.

\par
%\textsuperscript{(817.7)}
\textsuperscript{72:9.3} 2. A propuesta de los gobernadores estatales o de los jefes ejecutivos regionales, y por mandato de los consejos regionales supremos, las personas que han prestado un gran servicio a la sociedad o que han demostrado una sabiduría extraordinaria al servicio del gobierno, pueden disponer de votos adicionales, pero sólo una vez cada cinco años y sin que estos votos adicionales sobrepasen de nueve. El máximo número de votos que posee cualquier votante múltiple es de diez. Los científicos, inventores, educadores, filósofos y dirigentes espirituales también son reconocidos y honrados de esta manera con un mayor poder político. Los consejos supremos estatales y regionales confieren estos elevados privilegios cívicos de manera muy similar a los títulos que otorgan los colegios especiales, y los beneficiarios se sienten orgullosos de añadir estos símbolos de reconocimiento cívico, junto con sus otros títulos, a la lista de sus logros personales.

\par
%\textsuperscript{(817.8)}
\textsuperscript{72:9.4} 3. Todos los individuos condenados al trabajo obligatorio en las minas y todos los funcionarios del gobierno que perciben sus sueldos de los fondos procedentes de los impuestos, pierden su derecho al voto durante los períodos en que realizan estos servicios. Esto no se aplica a las personas mayores que cobran una pensión después de haberse jubilado a los sesenta y cinco años.

\par
%\textsuperscript{(817.9)}
\textsuperscript{72:9.5} 4. Hay cinco categorías de sufragio que reflejan los impuestos anuales medios que se han pagado durante cada período quinquenal. Los contribuyentes que han pagado más reciben votos adicionales hasta un máximo de cinco. Esta concesión es independiente de cualquier otro reconocimiento, pero una persona no puede disponer en ningún caso de más de diez votos.

\par
%\textsuperscript{(818.1)}
\textsuperscript{72:9.6} 5. En el momento en que se adoptó este plan electoral, el método territorial de votar fue abandonado a favor del sistema económico o funcional. Todos los ciudadanos votan ahora como miembros de sus grupos industriales, sociales o profesionales, independientemente de donde residan. El electorado está compuesto así de grupos consolidados, unificados e inteligentes, que eligen únicamente a sus mejores miembros para los puestos de confianza y de responsabilidad gubernamental. Este sistema de sufragio funcional o colectivo contiene una excepción: La elección del jefe del ejecutivo federal cada seis años se lleva a cabo mediante una votación nacional en la que ningún ciudadano dispone de más de un voto.

\par
%\textsuperscript{(818.2)}
\textsuperscript{72:9.7} Las agrupaciones económicas, profesionales, intelectuales y sociales de ciudadanos ejercen de esta manera el sufragio, salvo para elegir al jefe del ejecutivo. El Estado ideal es orgánico, y cada grupo libre e inteligente de ciudadanos representa un órgano vital y funcional dentro del organismo gubernamental más grande.

\par
%\textsuperscript{(818.3)}
\textsuperscript{72:9.8} Las escuelas para estadistas tienen el poder de emprender cualquier proceso en los tribunales estatales para que se prive del derecho al voto a todo individuo anormal, perezoso, indiferente o criminal. Este pueblo reconoce que cuando el cincuenta por ciento de una nación es inferior o anormal y posee el derecho de voto, esa nación está condenada. Creen que el dominio de la mediocridad significa la ruina de cualquier nación. Votar es obligatorio, y se imponen multas importantes a todos aquellos que no depositan su papeleta.

\section*{10. El tratamiento del crimen}
\par
%\textsuperscript{(818.4)}
\textsuperscript{72:10.1} Los métodos que utiliza este pueblo para enfrentarse con el crimen, la locura y la degeneración, aunque en algunos aspectos agradarán a la mayoría de los urantianos, en otros les resultarán sin duda espantosos. Los criminales corrientes y los anormales son colocados por sexos en las diferentes colonias agrícolas, donde viven sobradamente con sus propios recursos. Los criminales empedernidos más peligrosos y los locos incurables son condenados por los tribunales a morir en las cámaras de gas letal. Numerosos crímenes, además del asesinato, incluyendo la traición a la confianza del gobierno, sufren también la pena de muerte, y el castigo de la justicia es rápido y seguro.

\par
%\textsuperscript{(818.5)}
\textsuperscript{72:10.2} Este pueblo está saliendo de la era negativa de la ley para entrar en la era positiva. Recientemente han llegado al extremo de intentar prevenir el crimen condenando al trabajo de por vida, en las colonias de detención, a aquellos que se cree que podrían ser asesinos potenciales y criminales importantes. Si estos presidiarios demuestran posteriormente que se han vuelto más normales, pueden ser puestos en libertad condicional o bien indultados. El índice de homicidios en este continente sólo representa el uno por ciento del de las otras naciones.

\par
%\textsuperscript{(818.6)}
\textsuperscript{72:10.3} Hace más de cien años que se emprendieron esfuerzos para impedir la procreación de los criminales y los anormales, y ya han dado resultados satisfactorios. No existen cárceles ni hospitales para los locos. Y esto es así por una buena razón, ya que estos grupos sólo representan aproximadamente el diez por ciento de los que se encuentran en Urantia.

\section*{11. El estado de preparación militar}
\par
%\textsuperscript{(818.7)}
\textsuperscript{72:11.1} El presidente del Consejo de la Defensa Nacional puede nombrar a los diplomados de las escuelas militares federales como <<guardianes de la civilización>> en siete grados, según la capacidad y la experiencia. Este consejo está compuesto de veinticinco miembros, nombrados por los tribunales de padres, educativos e industriales más elevados, confirmados por el tribunal federal supremo, y está presidido de oficio por el jefe del estado mayor de los asuntos militares coordinados. Estos miembros prestan su servicio hasta la edad de setenta años.

\par
%\textsuperscript{(819.1)}
\textsuperscript{72:11.2} Los cursos que siguen estos oficiales designados duran cuatro años y están relacionados invariablemente con el dominio de algún oficio o profesión. La formación militar nunca se imparte sin esta enseñanza industrial, científica o profesional asociada. Cuando termina la preparación militar, el interesado ha recibido, durante sus cuatro años de cursos, la mitad de la educación que se imparte en cualquier escuela especial, donde los cursos duran también cuatro años. De esta manera se evita la creación de una clase militar profesional, proporcionando a una gran cantidad de hombres la oportunidad de ganarse la vida al mismo tiempo que adquieren la primera mitad de una formación técnica o profesional.

\par
%\textsuperscript{(819.2)}
\textsuperscript{72:11.3} El servicio militar en tiempos de paz es puramente voluntario, y el alistamiento en todas las ramas del servicio es por cuatro años, durante los cuales todo hombre sigue algún tipo de estudio especial, además del dominio de las tácticas militares. La formación musical es una de las ocupaciones principales de las escuelas militares centrales y de los veinticinco campos de entrenamiento repartidos por la periferia del continente. Durante los períodos de inactividad industrial, muchos miles de desempleados son utilizados automáticamente para reforzar las defensas militares del continente tanto en la tierra como en el mar y en el aire.

\par
%\textsuperscript{(819.3)}
\textsuperscript{72:11.4} Aunque esta nación mantiene una poderosa organización militar para defenderse de las invasiones de los pueblos hostiles que la rodean, se puede indicar a su favor que desde hace más de cien años no ha empleado estos recursos militares en ninguna guerra ofensiva. Se han civilizado hasta tal punto que pueden defender vigorosamente su civilización sin caer en la tentación de utilizar su poder militar con fines agresivos. No se ha producido ninguna guerra civil desde que se estableció el Estado continental unificado, pero durante los dos últimos siglos, este pueblo se ha visto obligado a sostener nueve conflictos defensivos encarnizados, tres de ellos contra poderosas confederaciones de potencias mundiales. Aunque esta nación mantiene una defensa adecuada contra cualquier ataque de sus vecinos hostiles, consagra mucha más atención a la formación de sus estadistas, científicos y filósofos.

\par
%\textsuperscript{(819.4)}
\textsuperscript{72:11.5} Cuando está en paz con el mundo, todos los mecanismos móviles de defensa se emplean íntegramente en los negocios, el comercio y el esparcimiento. Cuando se declara la guerra, toda la nación se moviliza. Durante el período de las hostilidades, todas las industrias pagan a sus empleados un salario militar, y los jefes de todos los departamentos militares se convierten en miembros del gabinete del jefe del ejecutivo.

\section*{12. Las otras naciones}
\par
%\textsuperscript{(819.5)}
\textsuperscript{72:12.1} Aunque la sociedad y el gobierno de este pueblo excepcional son superiores en muchos aspectos a los de las naciones de Urantia, debemos indicar que en los otros continentes (hay once en este planeta), los gobiernos son decididamente inferiores a los de las naciones más avanzadas de Urantia.

\par
%\textsuperscript{(819.6)}
\textsuperscript{72:12.2} En el momento actual, este gobierno superior tiene el proyecto de establecer relaciones diplomáticas con los pueblos inferiores, y ha surgido por primera vez un gran jefe religioso que recomienda el envío de misioneros a estas naciones circundantes. Nos tememos que estén a punto de cometer el mismo error que tantos otros han realizado intentando imponer una cultura y una religión superiores a otras razas.
¡Qué cosa tan admirable se podría hacer en este mundo si esta nación continental, con una cultura avanzada, se limitara a salir al exterior para traer hasta su territorio a los mejores elementos de los pueblos vecinos, y luego, después de haberlos educado, enviarlos de vuelta como emisarios de cultura a sus hermanos sumidos en la ignorancia! Si un Hijo Magistral viniera pronto a esta nación avanzada, es indudable que se podrían producir grandes acontecimientos en este mundo.

\par
%\textsuperscript{(820.1)}
\textsuperscript{72:12.3} Esta narración de los asuntos de un planeta vecino se lleva a cabo debido a un permiso especial y con la intención de hacer progresar la civilización y acelerar la evolución gubernamental en Urantia. Se podrían narrar muchas más cosas que interesarían y sorprenderían sin duda a los urantianos, pero esta revelación abarca los límites que nos marca el mandato que hemos recibido.

\par
%\textsuperscript{(820.2)}
\textsuperscript{72:12.4} Sin embargo, los urantianos deberían tomar nota de que su esfera hermana en la familia de Satania no se ha beneficiado ni de las misiones magistrales ni de las misiones de donación de los Hijos Paradisiacos. Los diversos pueblos de Urantia tampoco están separados los unos de los otros por la disparidad cultural que diferencia a esta nación continental de sus vecinos planetarios.

\par
%\textsuperscript{(820.3)}
\textsuperscript{72:12.5} El derramamiento del Espíritu de la Verdad proporciona la base espiritual para llevar a cabo grandes logros a favor de la raza humana del mundo sobre el que se otorga. Urantia está por lo tanto mucho mejor preparada para hacer realidad más inmediatamente un gobierno planetario con sus leyes, mecanismos, símbolos, convenciones e idioma ---lo cual podría contribuir de manera muy poderosa al establecimiento de una paz mundial bajo el imperio de la ley, y podría conducir algún día a los albores de una verdadera época de esfuerzos espirituales. Una época así es el umbral planetario hacia las épocas utópicas de luz y de vida.

\par
%\textsuperscript{(820.4)}
\textsuperscript{72:12.6} [Presentado por un Melquisedek de Nebadon.]


\chapter{Documento 81. El desarrollo de la civilización moderna}
\par
%\textsuperscript{(900.1)}
\textsuperscript{81:0.1} A PESAR de los altibajos sufridos debido al fracaso de los planes para el mejoramiento del mundo previstos en las misiones de Caligastia y Adán, la evolución orgánica básica de la especie humana continuó llevando a las razas hacia adelante en la escala del progreso humano y del desarrollo racial. Es posible retrasar la evolución, pero no puede ser detenida.

\par
%\textsuperscript{(900.2)}
\textsuperscript{81:0.2} Aunque los miembros de la raza violeta fueron menos numerosos de lo que se había planeado, su influencia produjo, desde la época de Adán, un avance en la civilización que sobrepasó con mucho el progreso que la humanidad había hecho a lo largo de toda su existencia anterior de casi un millón de años.

\section*{1. La cuna de la civilización}
\par
%\textsuperscript{(900.3)}
\textsuperscript{81:1.1} Durante cerca de treinta y cinco mil años después de la época de Adán, la cuna de la civilización estuvo en el suroeste de Asia, extendiéndose desde el valle del Nilo hacia el este y ligeramente hacia el norte a través del norte de Arabia, por toda Mesopotamia y continuando hasta el Turquestán. El \textit{clima} fue el factor decisivo para el establecimiento de la civilización en esta zona.

\par
%\textsuperscript{(900.4)}
\textsuperscript{81:1.2} Los grandes cambios climáticos y geológicos que se produjeron en África del norte y en el oeste de Asia fueron los que pusieron fin a las emigraciones iniciales de los adamitas, impidiéndoles llegar a Europa debido a la expansión del Mediterráneo, y desviando la oleada de emigrantes hacia el norte y el este hasta el Turquestán. Hacia la época en que finalizaron estas elevaciones de tierras y los cambios climáticos asociados, en torno al año 15.000 a. de J.C., la civilización había llegado en el mundo entero a un punto muerto, a excepción de los fermentos culturales y de las reservas biológicas de los anditas, los cuales permanecían confinados al este por las montañas de Asia y al oeste por los bosques en expansión de Europa.

\par
%\textsuperscript{(900.5)}
\textsuperscript{81:1.3} La evolución climática estaba a punto de conseguir ahora lo que todos los demás esfuerzos no habían logrado realizar, es decir, obligar al hombre eurasiático a abandonar la caza a favor de las ocupaciones más avanzadas del pastoreo y la agricultura. La evolución puede ser lenta, pero es enormemente eficaz.

\par
%\textsuperscript{(900.6)}
\textsuperscript{81:1.4} Puesto que los primeros agricultores utilizaban esclavos de manera muy generalizada, los campesinos eran menospreciados tanto por los cazadores como por los pastores. Durante miles de años se consideró que el cultivo de la tierra\footnote{\textit{Trabajo servil}: Gn 3:17-19.} era una ocupación inferior; de ahí la idea de que el trabajo de la tierra es una maldición, aunque se trata de la más grande de todas las bendiciones. Incluso en la época de Caín y Abel, los sacrificios de la vida pastoril se tenían en mucha mayor estima que las ofrendas de la agricultura\footnote{\textit{Ofrendas pastoriles preferidas}: Gn 4:2-5.}.

\par
%\textsuperscript{(900.7)}
\textsuperscript{81:1.5} El hombre evolucionó, en general, del estado de cazador al de agricultor, pasando por un período de transición como pastor, y esto mismo sucedió también entre los anditas; pero mucho más a menudo, la coacción evolutiva de las necesidades climáticas hizo que las tribus enteras pasaran directamente del estado de cazadores al de agricultores prósperos. Pero este fenómeno de pasar inmediatamente de la caza a la agricultura sólo se produjo en aquellas regiones donde había un alto grado de mezcla racial con el linaje violeta.

\par
%\textsuperscript{(901.1)}
\textsuperscript{81:1.6} Los pueblos evolutivos (principalmente los chinos) aprendieron pronto a plantar semillas y a cultivar las cosechas mediante la observación del crecimiento de las semillas que se humedecían accidentalmente, o que habían sido colocadas en las tumbas como alimento para los fallecidos. Pero en todo el suroeste de Asia, a lo largo de los fértiles fondos fluviales y de las llanuras adyacentes, los anditas llevaron a cabo las técnicas agrícolas perfeccionadas que habían heredado de sus antepasados, los cuales habían tenido la agricultura y la horticultura como ocupación principal dentro de los límites del segundo jardín.

\par
%\textsuperscript{(901.2)}
\textsuperscript{81:1.7} Durante miles de años, los descendientes de Adán habían cultivado el trigo y la cebada, que habían mejorado en el Jardín, en todas las tierras altas del borde superior de Mesopotamia. Los descendientes de Adán y Adanson se reunían allí, comerciaban y se relacionaban socialmente.

\par
%\textsuperscript{(901.3)}
\textsuperscript{81:1.8} Estos cambios forzosos en las condiciones de vida fueron los que provocaron que una proporción tan grande de la raza humana practicara un régimen alimenticio omnívoro. La combinación de una dieta de trigo, arroz y legumbres con la carne de los rebaños marcó un gran paso hacia adelante en la salud y el vigor de estos pueblos antiguos.

\section*{2. Los instrumentos de la civilización}
\par
%\textsuperscript{(901.4)}
\textsuperscript{81:2.1} El crecimiento de la cultura está basado en el desarrollo de los instrumentos de la civilización. Y los instrumentos que el hombre utilizó para salir del estado salvaje fueron eficaces en la medida exacta en que liberaron las capacidades del hombre para poder realizar otras tareas más elevadas.

\par
%\textsuperscript{(901.5)}
\textsuperscript{81:2.2} Vosotros que vivís ahora en un ambiente moderno de cultura en ciernes y de progreso incipiente en asuntos sociales, vosotros que disponéis realmente de algunos ratos libres para \textit{pensar} acerca de la sociedad y la civilización, no debéis pasar por alto el hecho de que vuestros antepasados primitivos tenían poco o ningún tiempo libre para poder dedicarlo a la reflexión cuidadosa y a la meditación social.

\par
%\textsuperscript{(901.6)}
\textsuperscript{81:2.3} Los cuatro primeros grandes progresos de la civilización humana fueron:

\par
%\textsuperscript{(901.7)}
\textsuperscript{81:2.4} 1. El dominio del fuego.

\par
%\textsuperscript{(901.8)}
\textsuperscript{81:2.5} 2. La domesticación de los animales.

\par
%\textsuperscript{(901.9)}
\textsuperscript{81:2.6} 3. La esclavización de los cautivos.

\par
%\textsuperscript{(901.10)}
\textsuperscript{81:2.7} 4. La propiedad privada.

\par
%\textsuperscript{(901.11)}
\textsuperscript{81:2.8} Aunque el fuego, el primer gran descubrimiento, abrió finalmente las puertas del mundo científico, en ese sentido tenía poco valor para el hombre primitivo. Éste se negaba a reconocer que las causas naturales explican los fenómenos vulgares.

\par
%\textsuperscript{(901.12)}
\textsuperscript{81:2.9} Cuando se le preguntaba de dónde venía el fuego, la simple historia de Andón y el pedernal fue rápidamente sustituida por la leyenda de cómo cierto Prometeo lo había robado del cielo. Los antiguos buscaban una explicación sobrenatural para todos los fenómenos naturales que no se encontraban al alcance de su comprensión personal, y muchos modernos continúan haciendo lo mismo. La despersonalización de los fenómenos llamados naturales ha necesitado miles de años, y aún no ha finalizado. Pero la búsqueda sincera, honrada y audaz de las causas verdaderas dio origen a la ciencia moderna: convirtió la astrología en astronomía, la alquimia en química y la magia en medicina.

\par
%\textsuperscript{(901.13)}
\textsuperscript{81:2.10} Durante la era anterior a las máquinas, la única manera que tenía el hombre de realizar un trabajo sin hacerlo él mismo consistía en utilizar un animal. La domesticación de los animales puso en sus manos unas herramientas vivientes cuya utilización inteligente preparó el camino para la agricultura y el transporte. Sin estos animales, el hombre no podría haberse elevado desde su estado primitivo hasta los niveles de la civilización posterior.

\par
%\textsuperscript{(902.1)}
\textsuperscript{81:2.11} La mayoría de los animales que convenían mejor para la domesticación se encontraban en Asia, especialmente en las regiones centrales y del suroeste. Ésta fue una de las razones por las cuales la civilización progresó más rápidamente en esta zona que en otras partes del mundo. Muchos de estos animales habían sido domesticados anteriormente dos veces, y en la época de los anditas fueron domesticados una vez más. Pero el perro había permanecido con los cazadores desde que había sido adoptado por el hombre azul muchísimo tiempo antes.

\par
%\textsuperscript{(902.2)}
\textsuperscript{81:2.12} Los anditas del Turquestán fueron los primeros pueblos que domesticaron una gran cantidad de caballos, y ésta es otra razón por la que su cultura predominó durante tanto tiempo. Hacia el año 5000 a. de J.C., los campesinos de Mesopotamia, el Turquestán y China habían empezado a criar ovejas, cabras, vacas, camellos, caballos, aves de corral y elefantes. Empleaban como bestias de carga el buey, el camello, el caballo y el yak. El hombre mismo fue en cierto momento la bestia de carga. Un jefe de la raza azul tuvo en cierta ocasión una colonia de porteadores de cargas de cien mil hombres.

\par
%\textsuperscript{(902.3)}
\textsuperscript{81:2.13} El establecimiento de la esclavitud y la propiedad privada de la tierra llegó con la agricultura. La esclavitud elevó el nivel de vida de los amos y les procuró más tiempo libre para cultivarse socialmente.

\par
%\textsuperscript{(902.4)}
\textsuperscript{81:2.14} El salvaje es un esclavo de la naturaleza, pero la civilización científica está confiriendo lentamente una mayor libertad a la humanidad. El hombre se ha liberado, y continuará liberándose, de la necesidad de trabajar sin descanso gracias a los animales, el fuego, el viento, el agua, la electricidad y otras fuentes de energía no descubiertas. A pesar de las dificultades transitorias ocasionadas por la invención prolífica de maquinarias, los beneficios finales que se derivarán de estos inventos mecánicos son inestimables. La civilización nunca puede florecer, y mucho menos establecerse, hasta que el hombre no dispone de \textit{tiempo libre} para pensar, planear e imaginar formas nuevas y mejores de hacer las cosas.

\par
%\textsuperscript{(902.5)}
\textsuperscript{81:2.15} Al principio, el hombre se apropió simplemente de su refugio, vivía debajo de las cornisas o habitaba en las cuevas. Luego adaptó los materiales naturales, tales como la madera y la piedra, para construir sus cabañas familiares. Finalmente entró en la etapa creativa de la construcción de viviendas, y aprendió a fabricar ladrillos y otros materiales de construcción.

\par
%\textsuperscript{(902.6)}
\textsuperscript{81:2.16} Entre las razas más modernas, los pueblos de las regiones montañosas del Turquestán fueron los primeros que construyeron sus viviendas de madera; sus casas se parecían mucho a las primeras cabañas de troncos de los pioneros americanos. En todas las llanuras, las viviendas humanas estaban hechas de ladrillos, y más tarde de ladrillos cocidos.

\par
%\textsuperscript{(902.7)}
\textsuperscript{81:2.17} Las antiguas razas fluviales construían sus cabañas clavando en la tierra unos palos altos en forma de círculo; luego juntaban los extremos superiores de los palos, formando así el armazón para la cabaña, el cual lo entrelazaban con cañas transversales, de manera que el conjunto se parecía a un enorme cesto invertido. Esta estructura se podía recubrir entonces con arcilla, y después de secarse al Sol, formaba una vivienda muy práctica y resistente a la intemperie.

\par
%\textsuperscript{(902.8)}
\textsuperscript{81:2.18} La idea posterior de trenzar todo tipo de cestos se originó independientemente a partir de estas cabañas primitivas. La idea de fabricar objetos de alfarería surgió en uno de los grupos al observar los efectos que se producían cuando estos armazones de palos se untaban con arcilla húmeda. La práctica de endurecer la cerámica mediante la cocción se descubrió cuando una de estas cabañas primitivas cubiertas de arcilla se incendió accidentalmente. Las artes de la antig\"uedad tenían muchas veces su origen en los sucesos fortuitos que acompañaban la vida diaria de los pueblos primitivos. Al menos esto es casi totalmente cierto en lo que se refiere al progreso evolutivo de la humanidad hasta la llegada de Adán.

\par
%\textsuperscript{(903.1)}
\textsuperscript{81:2.19} Aunque el estado mayor del Príncipe había introducido la alfarería por primera vez hace aproximadamente medio millón de años, la fabricación de recipientes de arcilla se había interrumpido prácticamente durante más de ciento cincuenta mil años. Sólo los noditas presumerios de la costa del golfo continuaron fabricando recipientes de arcilla. El arte de la alfarería se restableció durante la época de Adán. La diseminación de este arte tuvo lugar al mismo tiempo que se extendían las áreas desérticas de África, Arabia y Asia central, y se propagó en oleadas sucesivas con unas técnicas cada vez mejores desde Mesopotamia hacia el hemisferio oriental.

\par
%\textsuperscript{(903.2)}
\textsuperscript{81:2.20} No siempre se puede seguir la pista de estas civilizaciones de la época andita por las etapas de su alfarería o de sus otras artes. Los regímenes de Dalamatia y del Edén complicaron enormemente el curso tranquilo de la evolución humana. A menudo sucede que las vasijas y los utensilios más tardíos son inferiores a los productos anteriores de los pueblos anditas más puros.

\section*{3. Las ciudades, la manufactura y el comercio}
\par
%\textsuperscript{(903.3)}
\textsuperscript{81:3.1} La destrucción climática de las ricas praderas abiertas de caza y de las tierras de pastoreo del Turquestán, que empezó hacia el año 12.000 a. de J.C., obligó a los hombres de estas regiones a recurrir a nuevas formas de industria y de manufacturas rudimentarias. Algunos se orientaron hacia la cría de rebaños domesticados, otros se volvieron agricultores o colectores de alimentos de origen acuático, pero los tipos superiores de intelectos anditas escogieron dedicarse al comercio y la manufactura. Algunas tribus enteras cogieron la costumbre de dedicarse al desarrollo de una sola industria. Desde el valle del Nilo hasta el Hindu-Kusch y desde el Ganges hasta el Río Amarillo, la ocupación principal de las tribus superiores se volvió el cultivo del suelo, con el comercio como actividad suplementaria.

\par
%\textsuperscript{(903.4)}
\textsuperscript{81:3.2} El incremento del comercio y de la transformación de las materias primas en diversos artículos comerciales jugó directamente un papel decisivo en el nacimiento de las primeras comunidades semipacíficas que tuvieron tanta influencia en la diseminación de la cultura y las artes de la civilización. Antes de la era de un abundante comercio mundial, las comunidades sociales eran tribales ---eran grupos familiares ampliados. El comercio llevó a los diferentes tipos de seres humanos a asociarse, contribuyendo así a una fecundación cruzada más rápida de la cultura.

\par
%\textsuperscript{(903.5)}
\textsuperscript{81:3.3} Hace unos doce mil años, la era de las ciudades independientes estaba en sus albores. Estas ciudades primitivas, comerciantes y manufactureras, siempre estaban rodeadas de zonas de agricultura y ganadería. Aunque es cierto que la elevación del nivel de vida fomentó la industria, no debéis haceros una idea falsa de los refinamientos de la vida urbana inicial. Las razas primitivas no eran demasiado pulcras ni limpias, y las comunidades medias primitivas se elevaban entre treinta y sesenta centímetros cada veinticinco años a consecuencia de la simple acumulación de la suciedad y la basura. Algunas de estas ciudades antiguas también se elevaron muy rápidamente por encima de las tierras circundantes porque sus cabañas de barro no cocido duraban poco tiempo, y tenían la costumbre de construir sus nuevas viviendas directamente sobre las ruinas de las anteriores.

\par
%\textsuperscript{(903.6)}
\textsuperscript{81:3.4} El empleo generalizado de los metales fue una de las características de esta era de las primeras ciudades industriales y comerciales. Ya habéis descubierto en el Turquestán una cultura del bronce que es anterior al año 9000 a. de J.C., y los anditas aprendieron pronto a trabajar también el hierro, el oro y el cobre. Pero lejos de los centros más avanzados de la civilización, las condiciones eran muy diferentes. No había períodos bien diferenciados como la Edad de Piedra, del Bronce y del Hierro; los tres existían simultáneamente en diferentes localidades.

\par
%\textsuperscript{(904.1)}
\textsuperscript{81:3.5} El oro fue el primer metal que buscaron los hombres; era fácil de trabajar y al principio sólo se utilizó como adorno. Luego se empleó el cobre, pero no de manera abundante hasta que se mezcló con el estaño para fabricar el bronce más duro. El descubrimiento de la mezcla del cobre con el estaño para hacer el bronce fue realizado por un adansonita del Turquestán, cuya mina de cobre en las tierras altas se encontraba situada por casualidad al lado de un yacimiento de estaño.

\par
%\textsuperscript{(904.2)}
\textsuperscript{81:3.6} Con la aparición de una manufactura rudimentaria y de una industria incipiente, el comercio se convirtió rápidamente en la influencia más poderosa para la diseminación de la civilización cultural. La apertura de las rutas comerciales por tierra y por mar facilitó enormemente los viajes y la mezcla de las culturas, así como la fusión de las civilizaciones. Hacia el año 5000 a. de J.C., el caballo era de uso común en todos los países civilizados y semicivilizados. Estas razas más recientes no sólo poseían caballos domesticados, sino también diversos tipos de carros y carrozas. La rueda se utilizaba desde hacía miles de años, pero ahora los vehículos provistos de ruedas se emplearon de manera universal tanto en el comercio como en la guerra.

\par
%\textsuperscript{(904.3)}
\textsuperscript{81:3.7} Los comerciantes viajeros y los exploradores errantes hicieron más por el progreso de la civilización histórica que todas las demás influencias combinadas. Las conquistas militares, la colonización y las empresas misioneras patrocinadas por las religiones posteriores fueron también otros factores que contribuyeron a la difusión de la cultura; pero todos ellos fueron secundarios en comparación con las relaciones comerciales, continuamente en aumento gracias a las artes y las ciencias de la industria que se desarrollaban con rapidez.

\par
%\textsuperscript{(904.4)}
\textsuperscript{81:3.8} La inyección del linaje adámico en las razas humanas no sólo aceleró el ritmo de la civilización sino que también estimuló enormemente sus tendencias a la aventura y la exploración, de manera que la mayor parte de Eurasia y el norte de África se encontraron pronto ocupadas por los descendientes mixtos de los anditas que se multiplicaban rápidamente.

\section*{4. Las razas mezcladas}
\par
%\textsuperscript{(904.5)}
\textsuperscript{81:4.1} En el momento de contactar con los albores de los tiempos históricos, toda Eurasia, el norte de África y las islas del Pacífico están pobladas por las razas compuestas de la humanidad. Y estas razas actuales son el resultado de la mezcla y la remezcla de los cinco linajes humanos básicos de Urantia.

\par
%\textsuperscript{(904.6)}
\textsuperscript{81:4.2} Cada una de las razas de Urantia se podía identificar por ciertas características físicas distintivas. Los adamitas y los noditas tenían la cabeza alargada; los andonitas eran de cabeza ancha. Las razas sangiks tenían una cabeza mediana, aunque los hombres amarillos y azules tendían a ser de cabeza ancha. Cuando las razas azules se mezclaban con los linajes andonitas, eran claramente de cabeza ancha. Los sangiks secundarios tenían una cabeza entre mediana y alargada.

\par
%\textsuperscript{(904.7)}
\textsuperscript{81:4.3} Aunque estas dimensiones craneanas ayudan a descifrar los orígenes raciales, el esqueleto en su totalidad es mucho más fiable. En el desarrollo primitivo de las razas de Urantia había originalmente cinco tipos distintos de estructuras esqueléticas:

\par
%\textsuperscript{(904.8)}
\textsuperscript{81:4.4} 1. Andonitas ---los aborígenes de Urantia.

\par
%\textsuperscript{(904.9)}
\textsuperscript{81:4.5} 2. Sangiks primarios ---rojos, amarillos y azules.

\par
%\textsuperscript{(904.10)}
\textsuperscript{81:4.6} 3. Sangiks secundarios ---anaranjados, verdes e índigos.

\par
%\textsuperscript{(904.11)}
\textsuperscript{81:4.7} 4. Noditas ---los descendientes de los dalamatianos.

\par
%\textsuperscript{(904.12)}
\textsuperscript{81:4.8} 5. Adamitas --- la raza violeta.

\par
%\textsuperscript{(904.13)}
\textsuperscript{81:4.9} A medida que estos cinco grandes grupos raciales se entremezclaron ampliamente, las mezclas continuas tendieron a eclipsar el tipo andonita debido al predominio de la herencia sangik. Los lapones y los esquimales son una mezcla de andonitas y de la raza azul sangik. La estructura de su esqueleto es la que conserva mejor el tipo andónico aborigen. Pero los adamitas y los noditas se han mezclado tanto con las otras razas que sólo se pueden detectar como un tipo caucasoide generalizado.

\par
%\textsuperscript{(905.1)}
\textsuperscript{81:4.10} Por consiguiente, a medida que se desentierren los restos humanos de los últimos veinte mil años, será imposible, en general, distinguir claramente los cinco tipos originales. El estudio de las estructuras de estos esqueletos revelará que la humanidad está dividida ahora aproximadamente en tres clases:

\par
%\textsuperscript{(905.2)}
\textsuperscript{81:4.11} 1. \textit{La caucasoide} ---la mezcla andita de los linajes noditas y adamitas, modificada además por la unión con los sangiks primarios y (una parte de los) secundarios y por un cruce considerable con los andonitas. Las razas blancas occidentales, junto con algunos pueblos hindúes y turanianos, están incluidas en este grupo. El factor unificante de esta división es la mayor o menor proporción de herencia andita.

\par
%\textsuperscript{(905.3)}
\textsuperscript{81:4.12} 2. \textit{La mongoloide} ---el tipo sangik primario, que incluye a las razas roja, amarilla y azul originales. Los chinos y los amerindios pertenecen a este grupo. En Europa, el tipo mongoloide se ha modificado mediante una mezcla con los sangiks secundarios y los andonitas, y más aún debido a la inyección andita. Los malayos y otros pueblos indonesios están incluídos en esta clasificación, aunque contienen un porcentaje elevado de sangre sangik secundaria.

\par
%\textsuperscript{(905.4)}
\textsuperscript{81:4.13} 3. \textit{La negroide} ---el tipo sangik secundario, que incluía originalmente a las razas anaranjada, verde e índiga. El mejor ejemplo de este tipo es el negro, y se puede encontrar en África, la India e Indonesia, en todos los lugares donde se establecieron las razas sangiks secundarias.

\par
%\textsuperscript{(905.5)}
\textsuperscript{81:4.14} En el norte de China existe cierta mezcla de los tipos caucasoide y mongoloide; en el Levante, los caucasoides y los negroides se han entremezclado; en la India, así como en América del Sur, los tres tipos están representados. Las características del esqueleto de los tres tipos sobrevivientes subsisten todavía y ayudan a identificar a los antepasados más recientes de las razas humanas de hoy.

\section*{5. La sociedad cultural}
\par
%\textsuperscript{(905.6)}
\textsuperscript{81:5.1} La evolución biológica y la civilización cultural no están necesariamente correlacionadas; en cualquier época, la evolución orgánica puede seguir adelante sin obstáculos en medio mismo de una decadencia cultural. Pero cuando se examinan largos períodos de la historia humana, se puede observar que al final la evolución y la cultura se encuentran conectadas como causa y efecto. La evolución puede avanzar en ausencia de la cultura, pero la civilización cultural no florece sin un trasfondo adecuado de progreso racial anterior. Adán y Eva no introdujeron ningún arte de la civilización ajeno al progreso de la sociedad humana, pero la sangre adámica aumentó la capacidad inherente de las razas y aceleró el ritmo del desarrollo económico y del progreso industrial. La donación de Adán mejoró la capacidad cerebral de las razas, acelerando así enormemente los procesos de la evolución natural.

\par
%\textsuperscript{(905.7)}
\textsuperscript{81:5.2} Gracias a la agricultura, la domesticación de los animales y a una arquitectura más perfeccionada, la humanidad se liberó gradualmente de las peores fases de la lucha constante por la vida, y empezó a buscar el modo de dulcificar su manera de vivir; éste fue el principio de sus esfuerzos por conseguir unos niveles de bienestar material cada vez más elevados. Por medio de la manufactura y la industria, el hombre está aumentando gradualmente el contenido placentero de su vida como mortal.

\par
%\textsuperscript{(906.1)}
\textsuperscript{81:5.3} Pero la sociedad cultural no es ninguna gran asociación benéfica de privilegios heredados, en la que todos los hombres nacen con el derecho adquirido de pertenecer a ella y con una igualdad total. Es más bien una corporación elevada y progresiva de trabajadores terrestres, que sólo admite en sus filas a los operarios más nobles que se esfuerzan por hacer del mundo un lugar mejor en el que sus hijos, y los hijos de sus hijos, puedan vivir y avanzar en los siglos por venir. Y esta corporación de la civilización exige unos derechos de admisión muy costosos, impone unas disciplinas estrictas y rigurosas, inflige grandes penalizaciones a todos los disidentes y no conformistas, mientras que confiere pocas licencias o privilegios personales, excepto los de una seguridad creciente contra los peligros comunes y los riesgos raciales.

\par
%\textsuperscript{(906.2)}
\textsuperscript{81:5.4} La asociación social es una forma de seguro de supervivencia, y los seres humanos han aprendido que es beneficiosa; por eso la mayoría de los individuos está dispuesta a pagar las primas de sacrificio de sí mismo y de reducción de la libertad personal que la sociedad exige a sus miembros, a cambio de esta protección colectiva cada vez mayor. En resumen, el mecanismo social de hoy en día es un plan de seguro a base de ensayos y errores, destinado a proporcionar cierto grado de seguridad y protección contra un retorno a las terribles condiciones antisociales que caracterizaban las experiencias iniciales de la raza humana.

\par
%\textsuperscript{(906.3)}
\textsuperscript{81:5.5} La sociedad se convierte así en un sistema cooperativo que sirve para asegurar la libertad civil a través de las instituciones, la libertad económica a través del capital y la invención, la libertad social a través de la cultura, y la protección contra la violencia a través de la reglamentación penal.

\par
%\textsuperscript{(906.4)}
\textsuperscript{81:5.6} \textit{La fuerza no crea el derecho, pero hace respetar los derechos comúnmentereconocidos de cada generación sucesiva}. La misión principal del gobierno consiste en definir el derecho, la reglamentación justa y equitativa de las diferencias de clases, y la aplicación de una igualdad de oportunidades bajo el imperio de la ley. Cada derecho humano está asociado a un deber social; el privilegio colectivo es el mecanismo de un seguro que exige infaliblemente el pago total de las primas rigurosas de servicio al grupo. Y los derechos colectivos, así como los del individuo, deben ser protegidos, incluida la reglamentación de las inclinaciones sexuales.

\par
%\textsuperscript{(906.5)}
\textsuperscript{81:5.7} La libertad sometida a las reglas colectivas es la meta legítima de la evolución social. La libertad sin restricción es el sueño vano e imaginario de las mentes humanas inestables y caprichosas.

\section*{6. La conservación de la civilización}
\par
%\textsuperscript{(906.6)}
\textsuperscript{81:6.1} Aunque la evolución biológica ha continuado siempre hacia adelante, una gran parte de la evolución cultural salió del valle del Éufrates en unas oleadas que se debilitaron sucesivamente con el paso del tiempo, hasta que por fin la totalidad de los descendientes de puro linaje adámico hubo salido para enriquecer las civilizaciones de Asia y Europa. Las razas no se mezclaron por completo, pero sus civilizaciones sí lo hicieron en una medida considerable. La cultura se extendió lentamente por todo el mundo. Y esta civilización debe ser conservada y fomentada, porque hoy ya no existen nuevas fuentes de cultura, ni anditas que fortifiquen y estimulen el lento progreso de la evolución de la civilización.

\par
%\textsuperscript{(906.7)}
\textsuperscript{81:6.2} La civilización que se desarrolla actualmente en Urantia tuvo su origen, y está basada, en los factores siguientes:

\par
%\textsuperscript{(906.8)}
\textsuperscript{81:6.3} 1. \textit{Las circunstancias naturales}. La naturaleza y el alcance de una civilización material están determinados en gran medida por los recursos naturales disponibles. El clima, el tiempo atmosférico y numerosas condiciones físicas son factores en la evolución de la cultura.

\par
%\textsuperscript{(907.1)}
\textsuperscript{81:6.4} Al principio de la era andita sólo había dos zonas abiertas de caza, extensas y fértiles, en todo el mundo. Una se encontraba en América del Norte y estaba ocupada por los amerindios; la otra se hallaba al norte del Turquestán y estaba parcialmente ocupada por una raza andónico-amarilla. Los factores decisivos en la evolución de una cultura superior en el suroeste de Asia fueron la raza y el clima. Los anditas eran un gran pueblo, pero el factor decisivo que determinó el curso de su civilización fue la aridez creciente del Irán, el Turquestán y el Sinkiang, que los \textit{forzó} a inventar y a adoptar métodos nuevos y avanzados para arrancarle el sustento a sus tierras cada vez menos fértiles.

\par
%\textsuperscript{(907.2)}
\textsuperscript{81:6.5} La configuración de los continentes y otras disposiciones geográficas ejercen una gran influencia en la determinación de la paz o la guerra. Muy pocos urantianos han tenido nunca una oportunidad tan favorable para desarrollarse de manera continua y tranquila como la que disfrutaron los pueblos de América del Norte ---protegidos prácticamente por todos lados por inmensos océanos.

\par
%\textsuperscript{(907.3)}
\textsuperscript{81:6.6} 2. \textit{Los bienes de equipo}. La cultura no se desarrolla nunca en situaciones de pobreza; el tiempo libre es esencial para el progreso de la civilización. Los individuos pueden adquirir un carácter con un valor moral y espiritual en ausencia de riquezas materiales, pero una civilización cultural sólo puede derivarse de unas condiciones de prosperidad material que favorezcan los momentos de ocio combinados con la ambición.

\par
%\textsuperscript{(907.4)}
\textsuperscript{81:6.7} Durante los tiempos primitivos, la vida en Urantia era un asunto serio y grave. La humanidad tendió constantemente a encaminarse hacia los climas salubres de los trópicos precisamente para escapar de esta lucha incesante y de este trabajo interminable. Aunque estas zonas más cálidas para vivir disminuyeron un poco la intensa lucha por la existencia, las razas y las tribus que buscaron así la facilidad raras veces utilizaron su tiempo libre no ganado para hacer avanzar la civilización. El progreso social ha venido invariablemente de las ideas y los proyectos de las razas que han aprendido, por medio de sus esfuerzos inteligentes, a arrancarle a la tierra su sustento con menos esfuerzo y jornadas de trabajo reducidas, pudiendo disfrutar así de un margen beneficioso de tiempo libre bien merecido.

\par
%\textsuperscript{(907.5)}
\textsuperscript{81:6.8} 3. \textit{Los conocimientos científicos}. Los aspectos materiales de la civilización deben siempre esperar la acumulación de los datos científicos. Después del descubrimiento del arco y la flecha y de la utilización de los animales como fuerza motriz, pasó mucho tiempo antes de que el hombre aprendiera la manera de aprovechar la fuerza del viento y el agua, seguidos después por el empleo del vapor y la electricidad. Sin embargo, los instrumentos de la civilización mejoraron lentamente. La tejeduría, la alfarería, la domesticación de los animales y el trabajo de los metales fueron seguidos por una era de escritura y de imprenta.

\par
%\textsuperscript{(907.6)}
\textsuperscript{81:6.9} El conocimiento es poder. Los inventos preceden siempre la aceleración del desarrollo cultural a escala mundial. La ciencia y la invención fueron las que más se beneficiaron de las máquinas de imprimir, y la interacción de todas estas actividades culturales e inventivas ha acelerado enormemente el ritmo del progreso cultural.

\par
%\textsuperscript{(907.7)}
\textsuperscript{81:6.10} La ciencia enseña al hombre a hablar el nuevo lenguaje de las matemáticas y disciplina sus pensamientos según unas líneas de precisión rigurosa. La ciencia estabiliza también la filosofía mediante la eliminación de los errores, y al mismo tiempo purifica la religión gracias a la destrucción de las supersticiones.

\par
%\textsuperscript{(907.8)}
\textsuperscript{81:6.11} 4. \textit{Los recursos humanos}. Un gran número de hombres es indispensable para la diseminación de la civilización. En igualdad de condiciones en todos los aspectos, un pueblo numeroso dominará la civilización de una raza más reducida. En consecuencia, si una nación no logra aumentar el número de sus habitantes hasta cierto punto, eso le impedirá realizar plenamente su destino nacional, pero llega un momento en que un crecimiento adicional de la densidad de la población se vuelve suicida. La multiplicación de los habitantes más allá de la proporción óptima normal entre los hombres y las tierras disponibles significa o bien una disminución del nivel de vida, o una expansión inmediata de las fronteras territoriales mediante la penetración pacífica o la conquista militar ---la ocupación por la fuerza.

\par
%\textsuperscript{(908.1)}
\textsuperscript{81:6.12} A veces os sentís impresionados por los estragos de la guerra, pero deberíais reconocer que es necesario que nazca un gran número de mortales para permitir que el desarrollo social y moral tenga una amplia oportunidad de manifestarse; pero con esta fecundidad planetaria surge pronto el grave problema de la superpoblación. La mayoría de los mundos habitados son pequeños. Urantia está dentro de la media, quizás un poco más pequeña de lo normal. La estabilización óptima de la población nacional aumenta la cultura e impide la guerra. Y es sabia la nación que sabe cuándo detener su crecimiento.

\par
%\textsuperscript{(908.2)}
\textsuperscript{81:6.13} Pero el continente más rico en depósitos naturales y el más avanzado en equipos mecánicos hará pocos progresos si la inteligencia de su pueblo está en decadencia. El conocimiento se puede obtener mediante la educación, pero la sabiduría, que es indispensable para la verdadera cultura, sólo se puede conseguir a través de la experiencia y por parte de unos hombres y mujeres que son inteligentes de manera innata. Un pueblo así es capaz de aprender por experiencia, y puede volverse realmente sabio.

\par
%\textsuperscript{(908.3)}
\textsuperscript{81:6.14} 5. \textit{La eficacia de los recursos materiales}. Muchas cosas dependen de la sabiduría demostrada en la utilización de los recursos naturales, el conocimiento científico, los bienes de equipo y los potenciales humanos. El factor principal de la civilización primitiva era la \textit{fuerza} que ejercían los sabios jefes sociales; los hombres primitivos tenían la civilización que les imponían literalmente sus contemporáneos superiores. Las minorías superiores y bien organizadas han gobernado ampliamente este mundo.

\par
%\textsuperscript{(908.4)}
\textsuperscript{81:6.15} La fuerza no crea el derecho, pero la fuerza crea lo que existe y lo que ha existido en la historia. Urantia acaba de alcanzar recientemente el punto en que la sociedad está dispuesta a discutir la ética de la fuerza y del derecho.

\par
%\textsuperscript{(908.5)}
\textsuperscript{81:6.16} 6. \textit{La eficacia del idioma}. La civilización tiene que esperar al idioma para diseminarse. Las lenguas vivas y que se enriquecen aseguran la expansión de las ideas y los proyectos civilizados. Durante las épocas primitivas se hicieron progresos importantes en el lenguaje. Hoy existe la gran necesidad de un desarrollo ling\"uístico adicional que facilite la expresión del pensamiento en evolución.

\par
%\textsuperscript{(908.6)}
\textsuperscript{81:6.17} El idioma surgió en las asociaciones colectivas, donde cada grupo local desarrolló su propio sistema de intercambio de palabras. El lenguaje creció a través de los gestos, los signos, los gritos, los sonidos imitativos, la entonación y el acento, hasta llegar a la vocalización de los alfabetos posteriores. El idioma es la herramienta para pensar más importante y útil que posee el hombre, pero sólo pudo florecer cuando los grupos sociales consiguieron tener algún tiempo libre. La tendencia a jugar con el lenguaje desarrolla nuevas palabras ---el argot. Si la mayoría adopta el argot, entonces el uso lo convierte en idioma. Un ejemplo del origen de los dialectos es la condescendencia a <<hablar como los niños>> dentro de un grupo familiar.

\par
%\textsuperscript{(908.7)}
\textsuperscript{81:6.18} Las diferencias de idiomas siempre han sido el obstáculo principal para la extensión de la paz. La diseminación de una cultura sobre una raza, un continente o un mundo entero debe estar precedida por la eliminación de los dialectos. Un lenguaje universal favorece la paz, asegura la cultura y aumenta la felicidad. Incluso cuando las lenguas de un mundo se reducen a unas pocas, su dominio por parte de los pueblos cultos dirigentes influye poderosamente sobre la realización de la paz y la prosperidad mundiales\footnote{\textit{Una lengua}: Gn 11:1-9.}.

\par
%\textsuperscript{(908.8)}
\textsuperscript{81:6.19} Urantia ha hecho muy pocos progresos en el desarrollo de un idioma internacional, pero se han logrado muchas cosas gracias al establecimiento de un intercambio comercial internacional. Todas estas relaciones internacionales deberían fomentarse, ya se trate de los idiomas, el comercio, el arte, la ciencia, los juegos competitivos o la religión.

\par
%\textsuperscript{(909.1)}
\textsuperscript{81:6.20} 7. \textit{La eficacia de los dispositivos mecánicos}. El progreso de la civilización está relacionado directamente con el desarrollo y la posesión de las herramientas, las máquinas y los canales de distribución. Unas herramientas mejores, unas máquinas ingeniosas y eficaces, determinan la supervivencia de los grupos competidores en el marco de la civilización que progresa.

\par
%\textsuperscript{(909.2)}
\textsuperscript{81:6.21} En los tiempos primitivos, la única energía que se empleaba para cultivar la tierra era la energía humana. Fue precisa una larga lucha para sustituir a los hombres por los bueyes, ya que esto le quitaba el trabajo a los hombres. Más recientemente, las máquinas han empezado a reemplazar a los hombres, y cada avance de este tipo contribuye directamente al progreso de la sociedad, porque libera la energía humana para la realización de tareas más valiosas.

\par
%\textsuperscript{(909.3)}
\textsuperscript{81:6.22} La ciencia, guiada por la sabiduría, puede convertirse en la gran liberadora social del hombre. Una época mecánica sólo puede resultar desastrosa para aquella nación cuyo nivel intelectual es demasiado bajo como para descubrir los métodos sabios y las técnicas acertadas que le permitan adaptarse con éxito a las dificultades de transición que aparecen a consecuencia de la pérdida repentina de un gran número de empleos debido a la invención demasiado rápida de nuevos tipos de máquinas que economizan mano de obra.

\par
%\textsuperscript{(909.4)}
\textsuperscript{81:6.23} 8. \textit{El carácter de los abanderados}. La herencia social permite al hombre subirse en los hombros de todos los que lo han precedido y que han contribuido en algo a la suma de la cultura y el conocimiento. En esta tarea de pasar la antorcha cultural a la generación siguiente, el hogar será siempre la institución fundamental. Vienen a continuación el esparcimiento y la vida social, con la escuela en último lugar, pero igualmente indispensable en una sociedad compleja y muy bien organizada.

\par
%\textsuperscript{(909.5)}
\textsuperscript{81:6.24} Los insectos nacen plenamente educados y equipados para la vida ---una existencia en verdad muy limitada y puramente instintiva. El bebé humano nace sin educación; por consiguiente, al controlar la formación educativa de las generaciones más jóvenes, el hombre posee el poder de modificar enormemente el curso evolutivo de la civilización.

\par
%\textsuperscript{(909.6)}
\textsuperscript{81:6.25} Las influencias más importantes que contribuyen en el siglo veinte al fomento de la civilización y al progreso de la cultura son el incremento notable de los viajes por el mundo y las mejoras sin precedentes de los métodos de comunicación. Pero el desarrollo de la educación no ha seguido el mismo ritmo que la estructura social en expansión; la apreciación moderna de la ética tampoco se ha desarrollado en proporción al crecimiento de los ámbitos más puramente intelectuales y científicos. Y la civilización moderna se encuentra estancada en su desarrollo espiritual y en la salvaguardia de la institución del hogar.

\par
%\textsuperscript{(909.7)}
\textsuperscript{81:6.26} 9. \textit{Los ideales raciales}. Los ideales de una generación labran los canales del destino para la posteridad inmediata. La \textit{calidad} de los abanderados sociales determinará si la civilización avanza o retrocede. Los hogares, las iglesias y las escuelas de una generación determinan de antemano la tendencia del carácter de la generación siguiente. El impulso moral y espiritual de una raza o una nación determina en gran parte la velocidad cultural de esa civilización.

\par
%\textsuperscript{(909.8)}
\textsuperscript{81:6.27} Los ideales elevan la fuente de la corriente social. Y ninguna corriente puede elevarse por encima de su fuente, cualquiera que sea la técnica de presión o el control direccional que se pueda emplear. La fuerza motriz de los aspectos incluso más materiales de una civilización cultural reside en las realizaciones menos materiales de la sociedad. La inteligencia puede controlar el mecanismo de la civilización, la sabiduría puede dirigirlo, pero el idealismo espiritual es la energía que eleva realmente la cultura humana y la hace progresar de un nivel de realización al siguiente.

\par
%\textsuperscript{(910.1)}
\textsuperscript{81:6.28} Al principio, la vida era una lucha por la existencia; hoy es una lucha por el nivel de vida, y en el futuro lo será por la calidad del pensamiento, la próxima meta terrestre de la existencia humana.

\par
%\textsuperscript{(910.2)}
\textsuperscript{81:6.29} 10. \textit{La coordinación de los especialistas}. La civilización ha avanzado enormemente gracias a la temprana división del trabajo y a su corolario posterior de la especialización. La civilización depende ahora de la coordinación eficaz de los especialistas. A medida que se expande la sociedad, se deberá encontrar algún método que agrupe a los diversos especialistas.

\par
%\textsuperscript{(910.3)}
\textsuperscript{81:6.30} Los especialistas en los temas sociales, artísticos, técnicos e industriales continuarán multiplicando y acrecentando su habilidad y su destreza. Esta diversificación de las aptitudes y esta diferencia de trabajos debilitará y desintegrará finalmente la sociedad humana si no se desarrollan unos medios eficaces de coordinación y cooperación. Pero unas inteligencias que son capaces de tal inventiva y de una especialización semejante deberían ser enteramente competentes para idear unos métodos adecuados de control y de adaptación para todos los problemas derivados del rápido crecimiento de la invención y del ritmo acelerado de la expansión cultural.

\par
%\textsuperscript{(910.4)}
\textsuperscript{81:6.31} 11. \textit{Los mecanismos para encontrar empleo}. La próxima época de desarrollo social estará materializada en una cooperación y una coordinación mejores y más eficaces de la creciente especialización en plena expansión. A medida que el trabajo se diversifique cada vez más, será preciso idear alguna técnica para dirigir a los individuos hacia un empleo adecuado. Las máquinas no son la única causa de desempleo entre los pueblos civilizados de Urantia. La complejidad económica y el incremento continuo de la especialización industrial y profesional se añaden a los problemas de la colocación laboral.

\par
%\textsuperscript{(910.5)}
\textsuperscript{81:6.32} No es suficiente con preparar a los hombres para el trabajo; una sociedad compleja debe proporcionar también unos métodos eficaces para encontrar empleo. Antes de formar a los ciudadanos en las técnicas sumamente especializadas de ganarse la vida, se les debería enseñar uno o más métodos de trabajo, oficios o profesiones no especializados, que podrían utilizar cuando estuvieran desempleados temporalmente en sus oficios especializados. Ninguna civilización que alberga grandes clases de desempleados puede sobrevivir durante mucho tiempo. Con el tiempo, la aceptación de la ayuda del Tesoro público deformará y desmoralizará incluso a los mejores ciudadanos. La caridad privada misma se vuelve perniciosa cuando se concede mucho tiempo a unos ciudadanos sanos.

\par
%\textsuperscript{(910.6)}
\textsuperscript{81:6.33} Una sociedad tan sumamente especializada no aceptará con gusto las antiguas prácticas comunales y feudales de los pueblos antiguos. Es verdad que muchos servicios comunes pueden ser socializados de manera aceptable y beneficiosa, pero la mejor manera de dirigir a unos seres humanos extremadamente capacitados y ultraespecializados es mediante una técnica de cooperación inteligente. La coordinación modernizada y la reglamentación fraternal producirán una cooperación más duradera que los métodos comunistas más antiguos y primitivos o que las instituciones reguladoras dictatoriales basadas en la fuerza.

\par
%\textsuperscript{(910.7)}
\textsuperscript{81:6.34} 12. \textit{La buena voluntad para cooperar}. Uno de los grandes obstáculos para el progreso de la sociedad humana es el conflicto entre los intereses y el bienestar de los grupos humanos más grandes y socializados, y los de las asociaciones humanas más pequeñas con ideas contrarias y asociales, sin mencionar a los individuos aislados con una mentalidad antisocial.

\par
%\textsuperscript{(910.8)}
\textsuperscript{81:6.35} Ninguna civilización nacional dura mucho tiempo a menos que sus métodos educativos y sus ideales religiosos inspiren un patriotismo inteligente y una devoción nacional de tipo elevado. Sin este tipo de patriotismo inteligente y de solidaridad cultural, todas las naciones tienden a desintegrarse a consecuencia de los celos regionales y de los egoísmos locales.

\par
%\textsuperscript{(911.1)}
\textsuperscript{81:6.36} Para mantener una civilización mundial es preciso que los seres humanos aprendan a vivir juntos en paz y fraternidad. Sin una coordinación eficaz, la civilización industrial se encuentra en peligro a causa de los riesgos de la ultraespecialización: la monotonía, la estrechez de miras y la tendencia a engendrar la desconfianza y los celos.

\par
%\textsuperscript{(911.2)}
\textsuperscript{81:6.37} 13. \textit{Los dirigentes sabios y eficaces}. La civilización depende mucho, muchísimo, de un espíritu de cooperación entusiasta y eficaz. Diez hombres no valen mucho más que uno solo para levantar un gran peso, a menos que lo levanten todos juntos ---todos al mismo tiempo. Este trabajo de equipo ---la cooperación social--- depende de los dirigentes. Las civilizaciones culturales del pasado y del presente han estado basadas en la cooperación inteligente de los ciudadanos con unos jefes sabios y progresivos; y hasta que el hombre no alcance por evolución unos niveles más elevados, la civilización continuará dependiendo de una autoridad sabia y vigorosa.

\par
%\textsuperscript{(911.3)}
\textsuperscript{81:6.38} Las civilizaciones elevadas nacen de la correlación sagaz entre la riqueza material, la grandeza intelectual, el valor moral, la habilidad social y la perspicacia cósmica.

\par
%\textsuperscript{(911.4)}
\textsuperscript{81:6.39} 14. \textit{Los cambios sociales}. La sociedad no es una institución divina; es un fenómeno de la evolución progresiva. Una civilización que progresa siempre sufre retrasos cuando sus dirigentes son lentos en efectuar los cambios esenciales en la organización social que le permitan seguir el mismo ritmo que los desarrollos científicos de esa época. Sin embargo, no se deben menospreciar ciertas cosas simplemente porque sean viejas, ni tampoco hay que abrazar incondicionalmente una idea sólo porque sea nueva y original.

\par
%\textsuperscript{(911.5)}
\textsuperscript{81:6.40} El hombre debería experimentar sin miedo con los mecanismos de la sociedad. Pero estas aventuras de adaptación cultural deberían estar siempre controladas por aquellos que conocen plenamente la historia de la evolución social; y estos innovadores deberían estar siempre aconsejados por la sabiduría de aquellos que tienen una experiencia práctica en el ámbito del experimento social o económico en proyecto. \textit{No se debería intentar ningún gran cambiosocial o económico de manera repentina}. El tiempo es esencial para todos los tipos de adaptaciones humanas ---físicas, sociales o económicas. Únicamente los ajustes morales y espirituales se pueden efectuar bajo el impulso del momento, e incluso éstos también necesitan el paso del tiempo para que se manifiesten plenamente sus repercusiones sociales y materiales. Los ideales de la raza son el apoyo y la seguridad principales durante los períodos críticos en que una civilización se encuentra en tránsito entre un nivel y el siguiente.

\par
%\textsuperscript{(911.6)}
\textsuperscript{81:6.41} 15. \textit{Las medidas preventivas contra los desmoronamientos en los períodosde transición}. La sociedad es el fruto de innumerables épocas de ensayos y errores; representa lo que ha sobrevivido a los ajustes y reajustes selectivos en las etapas sucesivas de la ascensión secular de la humanidad desde el nivel animal hasta el nivel humano de categoría planetaria. El gran peligro para cualquier civilización ---en cualquier momento--- es la amenaza de su derrumbamiento durante el período de transición entre los métodos establecidos del pasado y los procedimientos nuevos y mejores, pero aún no probados, del futuro.

\par
%\textsuperscript{(911.7)}
\textsuperscript{81:6.42} El liderazgo es vital para el progreso. La sabiduría, la perspicacia y la previsión son indispensables para que duren las naciones. La civilización nunca está realmente en peligro hasta que sus dirigentes capaces empiezan a desaparecer. Y la cantidad de estos jefes sabios nunca ha sobrepasado el uno por ciento de la población.

\par
%\textsuperscript{(911.8)}
\textsuperscript{81:6.43} La civilización se ha elevado por estos peldaños de la escala evolutiva hasta alcanzar el nivel en que se podían poner en marcha las poderosas influencias que han culminado en la cultura en rápida expansión del siglo veinte. Los hombres sólo pueden esperar mantener sus civilizaciones actuales por medio de su adhesión a estos elementos esenciales, y asegurando al mismo tiempo su continuo desarrollo y su supervivencia indudable.

\par
%\textsuperscript{(912.1)}
\textsuperscript{81:6.44} Ésta es la esencia de la larguísima lucha de los pueblos de la Tierra por establecer la civilización desde la época de Adán. La cultura de hoy en día es el resultado neto de esta ardua evolución. Antes del descubrimiento de la imprenta, el progreso era relativamente lento porque los hombres de una generación no podían beneficiarse tan rápidamente de los logros de sus predecesores. Pero actualmente la sociedad humana se lanza hacia adelante con la fuerza del impulso acumulado de todas las épocas durante las cuales ha luchado la civilización.

\par
%\textsuperscript{(912.2)}
\textsuperscript{81:6.45} [Patrocinado por un Arcángel de Nebadon.]


\chapter{Documento 82. La evolución del matrimonio}
\par
%\textsuperscript{(913.1)}
\textsuperscript{82:0.1} EL MATRIMONIO ---el emparejamiento--- surge de la bisexualidad. El matrimonio es la reacción del hombre para adaptarse a esta bisexualidad, mientras que la vida familiar es el total resultante de todos estos ajustes evolutivos y adaptativos. El matrimonio es duradero; no es inherente a la evolución biológica, pero es la base de toda la evolución social, y por eso la continuidad de su existencia está asegurada de alguna manera. El matrimonio ha dado el hogar a la humanidad, y el hogar es la gloria que corona toda la larga y ardua lucha evolutiva.

\par
%\textsuperscript{(913.2)}
\textsuperscript{82:0.2} Aunque las instituciones religiosas, sociales y educativas son todas esenciales para la supervivencia de la civilización cultural, \textit{la familia es la civilizadoraprincipal}. Un niño aprende de su familia y de sus vecinos la mayor parte de las cosas esenciales de la vida.

\par
%\textsuperscript{(913.3)}
\textsuperscript{82:0.3} Los humanos de los tiempos pasados no poseían una civilización social muy rica, pero aquella que tenían la pasaban de manera fiel y eficaz a las generaciones siguientes. Y debéis reconocer que la mayoría de estas civilizaciones del pasado continuaron evolucionando con un mínimo estricto de otras influencias institucionales, porque el hogar funcionaba de manera eficaz. Hoy, las razas humanas poseen una rica herencia social y cultural, que debería ser pasada sabia y eficazmente a las generaciones venideras. La familia, como institución educativa, debe conservarse.

\section*{1. El instinto de apareamiento}
\par
%\textsuperscript{(913.4)}
\textsuperscript{82:1.1} A pesar del abismo que existe entre la personalidad del hombre y la de la mujer, el impulso sexual es suficiente para asegurar su unión con vistas a la reproducción de la especie. Este instinto funcionaba eficazmente mucho antes de que los humanos experimentaran una gran parte de lo que más tarde se ha llamado amor, devoción y fidelidad conyugal. El apareamiento es una propensión innata, y el matrimonio es su repercusión social evolutiva.

\par
%\textsuperscript{(913.5)}
\textsuperscript{82:1.2} El interés y el deseo sexuales no eran pasiones dominantes en los pueblos primitivos; simplemente los daban por sentados. Toda la experiencia reproductora estaba desprovista de embellecimientos imaginativos. La pasión sexual absorbente de los pueblos más civilizados se debe principalmente a las mezclas de razas, especialmente allí donde la naturaleza evolutiva fue estimulada por la imaginación asociativa y la apreciación de la belleza de los noditas y los adamitas. Pero las razas evolutivas absorbieron tan poca cantidad de herencia andita, que ésta no logró proporcionar el suficiente autocontrol sobre las pasiones animales así despertadas y estimuladas a consecuencia de la dotación de una conciencia más aguda del sexo y de unos impulsos de apareamiento más intensos. De todas las razas evolutivas, el hombre rojo es el que tenía el código sexual más elevado.

\par
%\textsuperscript{(913.6)}
\textsuperscript{82:1.3} La reglamentación sexual en relación con el matrimonio indica:

\par
%\textsuperscript{(913.7)}
\textsuperscript{82:1.4} 1. El progreso relativo de la civilización. La civilización ha exigido cada vez más que la satisfacción sexual se canalice de una manera útil y de acuerdo con las costumbres.

\par
%\textsuperscript{(914.1)}
\textsuperscript{82:1.5} 2. La cantidad de sangre andita en un pueblo cualquiera. En estos grupos, el sexo se ha vuelto la expresión más alta y más baja tanto de la naturaleza física como de la naturaleza emocional.

\par
%\textsuperscript{(914.2)}
\textsuperscript{82:1.6} Las razas sangiks tenían pasiones animales normales, pero mostraban poca imaginación o apreciación por la belleza y el atractivo físico del sexo opuesto. Aquello que se denomina atractivo sexual está prácticamente ausente incluso entre las razas primitivas de hoy en día; estos pueblos no mezclados poseen un instinto de apareamiento bien definido, pero una atracción sexual insuficiente como para crear problemas serios que necesiten un control social.

\par
%\textsuperscript{(914.3)}
\textsuperscript{82:1.7} El instinto de apareamiento es una de las fuerzas físicas dominantes que impulsan a los seres humanos; es la única emoción que, bajo la apariencia de una satisfacción individual, engaña eficazmente al hombre egoísta para que coloque el bienestar y la perpetuación de la raza muy por encima de la comodidad individual y de la ausencia de responsabilidades personales.

\par
%\textsuperscript{(914.4)}
\textsuperscript{82:1.8} Desde sus primeros comienzos hasta los tiempos modernos, el matrimonio como institución describe la evolución social de la tendencia biológica a perpetuarse. La perpetuación de la especie humana en evolución está asegurada por la presencia de este impulso racial al apareamiento, una necesidad que se denomina vagamente atracción sexual. Esta gran necesidad biológica se vuelve el eje impulsor de todo tipo de instintos, emociones y costumbres asociadas ---físicas, intelectuales, morales y sociales.

\par
%\textsuperscript{(914.5)}
\textsuperscript{82:1.9} Entre los salvajes, el acopio de alimentos era la motivación impulsora, pero cuando la civilización asegura una abundancia de alimentos, el deseo sexual se vuelve muchas veces un impulso dominante, y por eso necesita siempre una reglamentación social. En los animales, la periodicidad instintiva refrena la propensión al apareamiento, pero como el hombre es un ser que se controla en gran parte a sí mismo, el deseo sexual no es del todo periódico; por eso es necesario que la sociedad imponga a los individuos el control sobre sí mismos.

\par
%\textsuperscript{(914.6)}
\textsuperscript{82:1.10} Ninguna emoción o impulso a los que el ser humano se entregue sin freno y con exceso puede producir tanto daño y aflicción como esta poderosa necesidad sexual. El sometimiento inteligente de este impulso a las reglamentaciones de la sociedad es la prueba suprema de la realidad de cualquier civilización. El autocontrol, un autocontrol cada vez mayor, es lo que necesita cada vez más la humanidad que progresa. El secreto, la falta de sinceridad y la hipocresía pueden ocultar los problemas sexuales, pero no proporcionan soluciones ni mejoran la ética.

\section*{2. Los tabúes restrictivos}
\par
%\textsuperscript{(914.7)}
\textsuperscript{82:2.1} La historia de la evolución del matrimonio es simplemente la historia del control sexual bajo la presión de las restricciones sociales, religiosas y civiles. La naturaleza apenas reconoce a los individuos; no tiene en cuenta la llamada moralidad; está única y exclusivamente interesada en la reproducción de la especie. La naturaleza insiste irresistiblemente en la reproducción, pero deja con indiferencia que la sociedad resuelva los problemas consiguientes, creando así un problema enorme y siempre presente para la humanidad evolutiva. Este conflicto social consiste en una guerra sin fin entre los instintos básicos y la ética en evolución.

\par
%\textsuperscript{(914.8)}
\textsuperscript{82:2.2} Las relaciones entre los sexos estaban poco o nada reglamentadas entre las razas primitivas. Debido a esta licencia sexual, la prostitución no existía. Actualmente, los pigmeos y otras tribus atrasadas no poseen la institución del matrimonio; el estudio de estos pueblos revela las simples costumbres de emparejamiento que practicaban las razas primitivas. Pero siempre hay que estudiar y juzgar a todos los pueblos antiguos a la luz de las reglas morales de las costumbres de su propia época.

\par
%\textsuperscript{(915.1)}
\textsuperscript{82:2.3} Sin embargo, el amor libre nunca ha tenido buena reputación entre los pueblos que se encuentran por encima de la escala del salvajismo más completo. Los códigos matrimoniales y las restricciones conyugales comenzaron a desarrollarse en cuanto los grupos sociales empezaron a formarse. El emparejamiento ha progresado así a través de una multitud de transiciones, desde el estado de un libertinaje sexual casi total hasta los criterios morales del siglo veinte que implican una restricción sexual relativamente completa.

\par
%\textsuperscript{(915.2)}
\textsuperscript{82:2.4} En las primeras etapas del desarrollo tribal, las costumbres y los tabúes restrictivos eran muy rudimentarios, pero mantenían separados a los sexos ---lo cual favorecía la tranquilidad, el orden y la laboriosidad--- y la larga evolución del matrimonio y del hogar había empezado. Las costumbres de la vestimenta, el adorno y las prácticas religiosas, según el sexo, tuvieron su origen en estos tabúes primitivos que definieron el alcance de las libertades sexuales y terminaron así por crear los conceptos de vicio, crimen y pecado. Pero la costumbre de suspender todas las reglamentaciones sexuales durante los días de fiesta importantes, especialmente el Primero de Mayo, perduró durante mucho tiempo\footnote{\textit{Sesgo femenino}: Lv 12:2-5; 1 Co 14:34.}.

\par
%\textsuperscript{(915.3)}
\textsuperscript{82:2.5} Las mujeres siempre han estado sometidas a unos tabúes más restrictivos que los hombres. Las costumbres primitivas concedían a las mujeres no casadas el mismo grado de libertad sexual que a los hombres, pero siempre se ha exigido a las esposas que sean fieles a sus maridos. El matrimonio primitivo no restringía mucho las libertades sexuales del hombre, pero sí hacía que una mayor licencia sexual fuera tabú para la mujer. Las mujeres casadas siempre han llevado alguna marca que las destacaba como una clase aparte, tales como el peinado, la vestimenta, el velo, el aislamiento, los adornos y los anillos.

\section*{3. Las costumbres matrimoniales primitivas}
\par
%\textsuperscript{(915.4)}
\textsuperscript{82:3.1} El matrimonio es la respuesta institucional del organismo social a la tensión biológica siempre presente del instinto de reproducción ---la multiplicación de sí mismo--- que el hombre experimenta sin cesar. El apareamiento es universalmente natural, y a medida que la sociedad evolucionó de lo simple a lo complejo, hubo una evolución correspondiente de las costumbres relacionadas con el emparejamiento, la génesis de la institución matrimonial. Dondequiera que la evolución social ha progresado hasta la etapa en que se generan las costumbres, el matrimonio se podrá encontrar como una institución evolutiva.

\par
%\textsuperscript{(915.5)}
\textsuperscript{82:3.2} En el matrimonio siempre ha habido, y siempre habrá, dos ámbitos diferentes: las costumbres, las leyes que regulan los aspectos exteriores del emparejamiento, y las relaciones por otra parte secretas y personales entre los hombres y las mujeres. El individuo siempre se ha rebelado contra las reglamentaciones sexuales impuestas por la sociedad, y he aquí la razón de este problema sexual secular: la preservación de sí mismo es individual, pero está sostenida por la colectividad; la perpetuación de sí mismo es social, pero está asegurada por el impulso individual.

\par
%\textsuperscript{(915.6)}
\textsuperscript{82:3.3} Cuando las costumbres son respetadas, poseen un amplio poder para restringir y controlar el impulso sexual, tal como se ha demostrado en todas las razas. Los criterios sobre el matrimonio siempre han sido un indicador verídico del poder presente de las costumbres y de la integridad funcional del gobierno civil. Pero las costumbres primitivas relacionadas con el sexo y el emparejamiento eran una masa de reglamentaciones contradictorias y rudimentarias. Los padres, los hijos, los parientes y la sociedad, todos tenían intereses contrapuestos en la reglamentación del matrimonio. Pero a pesar de todo esto, las razas que ensalzaron y practicaron el matrimonio evolucionaron con naturalidad hasta unos niveles más elevados y sobrevivieron en mayor número.

\par
%\textsuperscript{(915.7)}
\textsuperscript{82:3.4} En los tiempos primitivos, el matrimonio era el precio de la posición social; la posesión de una esposa era un símbolo de distinción\footnote{\textit{La mujer como propiedad}: Gn 29:18-20; Rt 4:10.}. El salvaje consideraba que el día de su boda señalaba el comienzo de sus responsabilidades y de su madurez. En cierta época, el matrimonio fue considerado como un deber social; en otra, como una obligación religiosa; y en otra aún, como una necesidad política para proporcionar ciudadanos al Estado.

\par
%\textsuperscript{(916.1)}
\textsuperscript{82:3.5} Muchas tribus primitivas exigían que se llevara a cabo un robo notable como requisito para poder casarse; los pueblos posteriores sustituyeron estos saqueos e incursiones por los concursos atléticos y los juegos competitivos. Los vencedores de estas competiciones recibían el primer premio ---la posibilidad de elegir entre las novias del momento. Entre los cazadores de cabezas, un joven no podía casarse hasta que poseyera al menos una cabeza, aunque a veces se podían comprar estos cráneos. A medida que decayó la costumbre de comprar a las esposas, éstas se consiguieron mediante concursos de adivinanzas, una práctica que sobrevive todavía en muchos grupos de hombres negros.

\par
%\textsuperscript{(916.2)}
\textsuperscript{82:3.6} Con el avance de la civilización, algunas tribus pusieron en manos de las mujeres las duras pruebas matrimoniales de resistencia masculina; las mujeres pudieron así favorecer a los hombres de su elección. Estas pruebas matrimoniales incluían la habilidad en la caza, en la lucha y la capacidad para mantener una familia. Durante mucho tiempo se exigió que el novio viviera con la familia de la novia al menos un año, para trabajar allí y demostrar que era digno de la esposa que deseaba.

\par
%\textsuperscript{(916.3)}
\textsuperscript{82:3.7} Los requisitos de una esposa consistían en la aptitud para realizar los trabajos penosos y para tener hijos. Se le exigía que ejecutara cierta cantidad de trabajo agrícola en un período de tiempo determinado. Y si había tenido un hijo antes de casarse, era mucho más valiosa, porque su fertilidad estaba así asegurada.

\par
%\textsuperscript{(916.4)}
\textsuperscript{82:3.8} El hecho de que los pueblos antiguos consideraran como una deshonra, e incluso como un pecado, el no estar casado, explica el origen de los matrimonios entre los niños; puesto que uno tenía que casarse, cuanto antes lo hiciera, mejor. También existía la creencia generalizada de que las personas solteras no podían entrar en el mundo de los espíritus, y esto fue un motivo adicional para casar a los niños incluso en el momento de nacer, y a veces antes, en espera del sexo que tuvieran. Los antiguos creían que incluso los muertos tenían que estar casados. Los casamenteros originales se empleaban para gestionar los matrimonios de las personas fallecidas. Uno de los padres encargaba a estos intermediarios que llevaran a cabo el casamiento entre un hijo muerto y la hija muerta de otra familia.

\par
%\textsuperscript{(916.5)}
\textsuperscript{82:3.9} Entre los pueblos más recientes, la pubertad era la edad normal para casarse, pero esta edad ha avanzado en proporción directa a los progresos de la civilización. Al principio de la evolución social surgieron unas órdenes peculiares, tanto de hombres como de mujeres, que practicaban el celibato; estas órdenes fueron creadas y mantenidas por personas más o menos desprovistas de necesidades sexuales normales.

\par
%\textsuperscript{(916.6)}
\textsuperscript{82:3.10} Muchas tribus permitían que los miembros de su grupo dirigente tuvieran relaciones sexuales con la novia poco antes de que fuera entregada a su marido. Cada uno de estos hombres le entregaba un regalo a la muchacha, y éste es el origen de la costumbre de hacer los regalos de boda. Algunos grupos contaban con que la joven se ganaría su propia dote, la cual consistía en los regalos que recibía como recompensa por sus servicios sexuales en la sala de exhibición de las novias.

\par
%\textsuperscript{(916.7)}
\textsuperscript{82:3.11} Algunas tribus casaban a los muchachos con las viudas y las mujeres de edad, y luego, cuando más tarde se quedaban viudos, les permitían casarse con las chicas jóvenes. De esta manera se aseguraban, según decían, de que los dos padres no serían unos insensatos, tal como pensaban que ocurriría si permitían que se casaran dos jóvenes. Otras tribus limitaban el emparejamiento a los grupos que tenían una edad similar. Esta limitación del matrimonio a los grupos de una edad determinada fue la que primero dio origen a las ideas de incesto. (En la India no existe, incluso en la actualidad, ningún límite de edad para casarse.)

\par
%\textsuperscript{(916.8)}
\textsuperscript{82:3.12} Según ciertas costumbres, la viudedad era algo muy de temer, ya que las viudas eran ejecutadas o bien se les permitía que se suicidaran sobre las tumbas de sus maridos, pues se creía que debían entrar con sus esposos en el mundo de los espíritus. A la viuda sobreviviente se le culpaba casi invariablemente de la muerte de su marido. Algunas tribus las quemaban vivas. Si una viuda seguía viviendo, llevaba una vida de luto continuo y de restricciones sociales insoportables, ya que un nuevo casamiento se veía generalmente con desaprobación.

\par
%\textsuperscript{(917.1)}
\textsuperscript{82:3.13} En los tiempos antiguos se fomentaban muchas prácticas que ahora se consideran como inmorales. No era raro que las esposas primitivas se enorgullecieran de las aventuras de sus maridos con otras mujeres. La castidad de las muchachas era un gran obstáculo para casarse; dar a luz a un hijo antes del matrimonio incrementaba considerablemente el atractivo de una joven como esposa, puesto que el hombre estaba seguro de tener una compañera fértil.

\par
%\textsuperscript{(917.2)}
\textsuperscript{82:3.14} Muchas tribus primitivas autorizaban el matrimonio a prueba hasta que la mujer se quedaba embarazada, y entonces se llevaba a cabo la ceremonia regular de la boda; en otros grupos, la boda no se celebraba hasta que nacía el primer hijo. Si una esposa era estéril, sus padres tenían que recuperarla, y el matrimonio era anulado. Las costumbres exigían que cada pareja tuviera hijos.

\par
%\textsuperscript{(917.3)}
\textsuperscript{82:3.15} Estos matrimonios a prueba primitivos estaban enteramente desprovistos de toda semejanza de licencia; se trataban simplemente de unas pruebas sinceras de fecundidad. Las personas contrayentes se casaban de manera permanente en cuanto quedaba probada la fertilidad. Cuando las parejas modernas se casan con la idea, en el fondo de su mente, de divorciarse cómodamente si su vida conyugal no les satisface plenamente, contraen en realidad una forma de matrimonio a prueba, que además es muy inferior al de las honradas aventuras de sus antepasados menos civilizados.

\section*{4. El matrimonio y las costumbres sobre la propiedad}
\par
%\textsuperscript{(917.4)}
\textsuperscript{82:4.1} El matrimonio siempre ha estado estrechamente vinculado con la propiedad y la religión. La propiedad ha estabilizado el matrimonio, y la religión lo ha moralizado.

\par
%\textsuperscript{(917.5)}
\textsuperscript{82:4.2} El matrimonio primitivo era una inversión, una especulación económica; era más una cuestión comercial que un asunto de coqueteo. Los antiguos se casaban en beneficio y por el bienestar del grupo; por esta razón sus matrimonios eran planeados y concertados por el grupo, por los padres y los ancianos. Las costumbres relacionadas con la propiedad estabilizaban eficazmente la institución matrimonial, y esto está corroborado por el hecho de que el matrimonio era más permanente entre las tribus primitivas que entre muchos pueblos modernos.

\par
%\textsuperscript{(917.6)}
\textsuperscript{82:4.3} A medida que la civilización avanzó y que la propiedad privada consiguió un reconocimiento mayor dentro de las costumbres, el robo se convirtió en el crimen más grave. El adulterio se consideraba como una forma de robo, una violación de los derechos de propiedad del marido; por eso no se menciona de manera específica en los códigos y costumbres primitivos. La mujer empezaba siendo propiedad de su padre, quien transfería sus derechos al marido, y todas las relaciones sexuales legalizadas surgieron de estos derechos de propiedad preexistentes. El Antiguo Testamento trata a las mujeres como una forma de propiedad. El Corán enseña su inferioridad. El hombre tenía el derecho de prestar su esposa a un amigo o a un invitado, y esta costumbre prevalece todavía entre algunos pueblos.

\par
%\textsuperscript{(917.7)}
\textsuperscript{82:4.4} Los celos sexuales modernos no son innatos; son un producto de las costumbres en evolución. El hombre primitivo no tenía celos de su mujer; se limitaba a defender su propiedad. La razón de mantener a la esposa en una consideración sexual más estricta que al marido se debía a que su infidelidad conyugal implicaba una descendencia y una herencia. En la marcha de la civilización, el hijo ilegítimo cayó muy pronto en descrédito. Al principio sólo la mujer era castigada por el adulterio; más tarde, las costumbres decretaron también que se castigara a su compañero, y durante muchos milenios, el marido ofendido o el padre protector tuvieron el pleno derecho de matar al transgresor masculino. Los pueblos modernos conservan estas costumbres, las cuales reconocen los llamados crímenes de honor en el derecho consuetudinario.

\par
%\textsuperscript{(917.8)}
\textsuperscript{82:4.5} Puesto que el tabú de la castidad tuvo su origen como una fase de las costumbres relacionadas con la propiedad, al principio se aplicó a las mujeres casadas, pero no a las jóvenes solteras. En años posteriores, la castidad fue más una exigencia del padre que del pretendiente; una virgen era un activo comercial para el padre ---representaba un precio más elevado. A medida que aumentó la demanda de la castidad, se estableció la costumbre de pagarle al padre una recompensa nupcial en reconocimiento por el servicio de haber educado adecuadamente a una novia casta para el futuro marido. Una vez que surgió esta idea de la castidad femenina, se arraigó tanto en las razas que emprendieron la costumbre de enjaular literalmente a las muchachas, de encarcelarlas realmente durante años a fin de asegurar su virginidad. Así es como los principios morales más recientes y las pruebas de virginidad dieron origen automáticamente a las clases de prostitutas profesionales; eran las novias rechazadas, las mujeres que las madres de los novios habían descubierto que no eran vírgenes.

\section*{5. La endogamia y la exogamia}
\par
%\textsuperscript{(918.1)}
\textsuperscript{82:5.1} Los salvajes observaron muy pronto que las mezclas raciales mejoraban la calidad de la descendencia. No se trataba de que la endogamia fuera siempre mala, sino que la exogamia era siempre comparativamente mejor; por eso las costumbres tendieron a cristalizar la restricción de las relaciones sexuales entre los parientes cercanos. Se reconoció que la exogamia acrecentaba enormemente las oportunidades selectivas para la variación y el progreso evolutivos. Los individuos nacidos de matrimonios exogámicos eran más polifacéticos y tenían una mayor capacidad para sobrevivir en un mundo hostil; los engendrados por endogamia, así como sus costumbres, desaparecieron gradualmente. Todo esto se desarrolló lentamente; los salvajes no razonaban conscientemente sobre estos problemas. Pero los pueblos progresivos posteriores sí lo hicieron, y observaron también que la endogamia excesiva a veces provocaba una debilidad generalizada.

\par
%\textsuperscript{(918.2)}
\textsuperscript{82:5.2} Aunque una endogamia con buenos linajes produjo a veces la formación de fuertes tribus, los casos espectaculares de los malos resultados observados en la endogamia de los anormales hereditarios se grabaron con más fuerza en la mente de los hombres, lo que provocó que las costumbres progresivas formularan cada vez más tabúes contra todos los matrimonios entre parientes cercanos.

\par
%\textsuperscript{(918.3)}
\textsuperscript{82:5.3} La religión ha sido mucho tiempo una barrera eficaz contra la exogamia; muchas enseñanzas religiosas han proscrito los matrimonios fuera de la fe. La mujer ha favorecido generalmente la práctica de la endogamia, y el hombre la de la exogamia. La propiedad siempre ha influido sobre el matrimonio, y a veces, en un esfuerzo por conservar las propiedades en el interior de un clan, han surgido costumbres que obligaban a las mujeres a elegir sus maridos dentro de la tribu de sus padres. Las reglas de este tipo condujeron a una gran multiplicación de los matrimonios entre primos. La endogamia también se practicó en un esfuerzo por preservar los secretos artesanales; los artesanos expertos trataban de conservar el conocimiento de su oficio dentro de su familia.

\par
%\textsuperscript{(918.4)}
\textsuperscript{82:5.4} Cuando los grupos superiores se encontraban aislados, volvían siempre a los emparejamientos consanguíneos. Durante más de ciento cincuenta mil años, los noditas fueron uno de los grandes grupos endogámicos. Las costumbres endogámicas más recientes sufrieron la enorme influencia de las tradiciones de la raza violeta, en la que los emparejamientos se producían al principio, forzosamente, entre hermanos y hermanas. Los matrimonios entre hermanos y hermanas fueron frecuentes en el Egipto primitivo, Siria, Mesopotamia, y en todos los países ocupados en otro tiempo por los anditas. Los egipcios practicaron mucho tiempo los matrimonios entre hermanos y hermanas en un esfuerzo por conservar la pureza de la sangre real, una costumbre que sobrevivió más tiempo aún en Persia. Antes de la época de Abraham, los matrimonios entre primos eran obligatorios en Mesopotamia; los primos tenían el derecho prioritario de casarse con sus primas. Abraham mismo se casó con su hermanastra\footnote{\textit{La hermana y mujer de Abraham}: Gn 20:12.}, pero las costumbres posteriores de los judíos ya no permitieron estas uniones\footnote{\textit{Uniones prohibidas}: Lv 18:9.}.

\par
%\textsuperscript{(919.1)}
\textsuperscript{82:5.5} Los primeros pasos para suprimir los matrimonios entre hermanos y hermanas se dieron bajo la influencia de las costumbres polígamas, porque la esposa-hermana solía dominar con arrogancia a la otra u otras esposas\footnote{\textit{La esposa-hermana solía dominar a las otras}: Gn 16:6; 21:9-10.}. Algunas costumbres tribales prohibían el matrimonio con la viuda de un hermano muerto, pero exigían que el hermano vivo engendrara los hijos de su hermano fallecido\footnote{\textit{Matrimonio del levirato}: Dt 25:5-6; Mt 22:24; Mc 12:19; Lc 20:28.}. No existe ningún instinto biológico que vaya en contra de algún grado de endogamia; tales restricciones son únicamente una cuestión de tabúes.

\par
%\textsuperscript{(919.2)}
\textsuperscript{82:5.6} La exogamia\footnote{\textit{Exogamia}: Gn 16:2; Ex 2:16-21; Nm 12:1.} terminó por dominar porque los hombre la preferían; conseguir una esposa en el exterior les aseguraba una mayor libertad con respecto a su familia política. La familiaridad produce el menosprecio; así pues, a medida que el factor de la elección individual empezó a dominar el emparejamiento, se estableció la costumbre de elegir a la pareja fuera de la tribu.

\par
%\textsuperscript{(919.3)}
\textsuperscript{82:5.7} Muchas tribus prohibieron finalmente el matrimonio dentro del clan, y otras limitaron el emparejamiento a ciertas castas. El tabú en contra del matrimonio con una mujer del mismo tótem que el interesado impulsó la costumbre de raptar a las mujeres de las tribus vecinas. Posteriormente, los matrimonios se reglamentaron más de acuerdo con la residencia territorial que según el parentesco. La evolución de la endogamia pasó por muchas etapas hasta transformarse en las prácticas modernas de la exogamia. Incluso después de que el tabú sobre la endogamia pesara sobre la gente del pueblo, a los jefes y los reyes les estaba permitido casarse con sus parientes cercanos a fin de conservar la sangre real concentrada y pura. Las costumbres han permitido generalmente a los dirigentes soberanos ciertas licencias en materia sexual.

\par
%\textsuperscript{(919.4)}
\textsuperscript{82:5.8} La presencia de los pueblos anditas posteriores tuvo mucho que ver con el aumento del deseo de las razas sangiks de casarse fuera de sus propias tribus. Pero a la exogamia no le fue posible volverse predominante hasta que los grupos vecinos aprendieron a convivir en una paz relativa.

\par
%\textsuperscript{(919.5)}
\textsuperscript{82:5.9} La exogamia en sí misma promovía la paz; los matrimonios entre tribus reducían las hostilidades. La exogamia condujo a la coordinación tribal y a las alianzas militares; se volvió predominante porque proporcionaba un aumento de fuerzas; fue una constructora de naciones. Las relaciones comerciales crecientes también favorecieron enormemente la exogamia; la aventura y la exploración contribuyeron a ampliar los límites del emparejamiento y facilitaron mucho la fecundación cruzada de las culturas raciales.

\par
%\textsuperscript{(919.6)}
\textsuperscript{82:5.10} Las contradicciones, por otra parte inexplicables, de las costumbres raciales sobre el matrimonio se deben ampliamente a esta tradición de la exogamia, acompañada del rapto y la compra de las esposas en las tribus ajenas, todo lo cual se tradujo en una mezcla de las distintas costumbres tribales. Estos tabúes sobre la endogamia eran sociológicos y no biológicos, y este hecho está bien ilustrado en los tabúes sobre los matrimonios entre parientes, los cuales abarcaban muchos grados de relaciones con las familias políticas, en unos casos en los que no existía ningún parentesco consanguíneo.

\section*{6. Las mezclas raciales}
\par
%\textsuperscript{(919.7)}
\textsuperscript{82:6.1} Hoy ya no existe ninguna raza pura en el mundo. Los primeros pueblos originales y evolutivos de color sólo tienen dos razas representativas que sobreviven en el mundo ---los hombres amarillos y los hombres negros--- e incluso estas dos razas están muy mezcladas con los pueblos de color ya desaparecidos. Aunque la llamada raza blanca desciende predominantemente de los antiguos hombres azules, está más o menos mezclada con todas las demás razas, al igual que los hombres rojos de las Américas.

\par
%\textsuperscript{(919.8)}
\textsuperscript{82:6.2} De las seis razas sangiks de color, tres eran primarias y tres secundarias. Aunque las razas primarias ---azul, roja y amarilla--- eran superiores en muchos aspectos a los tres pueblos secundarios, se debe recordar que estas razas secundarias poseían muchas características deseables que habrían mejorado considerablemente a los pueblos primarios si éstos hubieran podido absorber sus mejores linajes.

\par
%\textsuperscript{(920.1)}
\textsuperscript{82:6.3} Los prejuicios actuales contra los <<mestizos>>, los <<híbridos>> y los <<mixtos>> han surgido porque la mayor parte de los cruces raciales modernos se producen entre los linajes extremadamente inferiores de las razas interesadas. También se consigue una progenie poco satisfactoria cuando los linajes degenerados de la misma raza se casan entre sí.

\par
%\textsuperscript{(920.2)}
\textsuperscript{82:6.4} Si las razas actuales de Urantia pudieran liberarse de la maldición de sus estratos más bajos de especímenes degenerados, antisociales, mentalmente débiles y marginados, habría pocas objeciones para llevar a cabo una fusión racial limitada. Y si estas mezclas raciales pudieran producirse entre los tipos más elevados de las diversas razas, habría aún menos objeciones.

\par
%\textsuperscript{(920.3)}
\textsuperscript{82:6.5} La hibridación de los linajes superiores y diferentes es el secreto para crear estirpes nuevas y más vigorosas, y esto es tan cierto para las plantas y los animales como para la especie humana. La hibridación aumenta el vigor y acrecienta la fecundidad. Las mezclas raciales de los estratos medios o superiores de los diversos pueblos aumentan considerablemente el potencial \textit{creativo}, tal como está demostrado en la población actual de los Estados Unidos de América del Norte. Cuando estos emparejamientos tienen lugar entre los estratos inferiores o más bajos, la creatividad disminuye, tal como se puede observar en los pueblos de hoy en día del sur de la India.

\par
%\textsuperscript{(920.4)}
\textsuperscript{82:6.6} La mezcla de las razas contribuye enormemente a la aparición repentina de características \textit{nuevas}, y si esta hibridación es la unión de los linajes superiores, entonces estas nuevas características serán también peculiaridades \textit{superiores}.

\par
%\textsuperscript{(920.5)}
\textsuperscript{82:6.7} Mientras las razas actuales continúen tan sobrecargadas de linajes inferiores y degenerados, las mezclas raciales a gran escala serán sumamente perjudiciales, pero la mayoría de las objeciones a estos experimentos están basadas en prejuicios sociales y culturales más bien que en consideraciones biológicas. Incluso entre las estirpes inferiores, los híbridos son con frecuencia una mejora con respecto a sus antepasados. La hibridación contribuye a mejorar la especie debido al papel de los \textit{genes dominantes}. La mezcla racial aumenta la probabilidad de que un mayor número de \textit{dominantes} deseables estén presentes en el híbrido.

\par
%\textsuperscript{(920.6)}
\textsuperscript{82:6.8} En los últimos cien años ha tenido lugar más hibridación racial en Urantia de la que se había producido durante miles de años. Se ha exagerado mucho el peligro de que surjan grandes discordancias a causa de los cruces de los linajes humanos. Las dificultades principales de los <<mestizos>> se deben a los prejuicios sociales.

\par
%\textsuperscript{(920.7)}
\textsuperscript{82:6.9} El experimento de Pitcairn, consistente en mezclar las razas blanca y polinesia, salió bastante bien porque los hombres blancos y las mujeres polinesias poseían unos linajes raciales relativamente buenos. El cruce entre los tipos más elevados de las razas blanca, roja y amarilla traería inmediatamente a la existencia muchas características nuevas y biológicamente eficaces. Estos tres pueblos pertenecen a las razas sangiks primarias. Los resultados inmediatos de las mezclas entre las razas blanca y negra no son tan deseables, ni sus descendientes mulatos son tan inaceptables como pretenden hacerlo creer los prejuicios sociales y raciales. Estos híbridos blanco-negros son, físicamente, unos excelentes especímenes de la humanidad, a pesar de su ligera inferioridad en algunos otros aspectos.

\par
%\textsuperscript{(920.8)}
\textsuperscript{82:6.10} Cuando una raza sangik primaria se fusiona con una raza sangik secundaria, esta última mejora considerablemente a expensas de la primera. Y a pequeña escala ---que se extienda durante largos períodos de tiempo--- esta contribución sacrificatoria de las razas primarias para mejorar a los grupos secundarios debe encontrar pocos inconvenientes serios. Desde el punto de vista biológico, los sangiks secundarios eran, en algunos aspectos, superiores a las razas primarias.

\par
%\textsuperscript{(921.1)}
\textsuperscript{82:6.11} Después de todo, el verdadero riesgo para la especie humana reside en la multiplicación desmedida de los linajes inferiores y degenerados de los diversos pueblos civilizados, más bien que en el supuesto peligro de sus cruces raciales.

\par
%\textsuperscript{(921.2)}
\textsuperscript{82:6.12} [Presentado por el Jefe de los Serafines estacionado en Urantia.]


\chapter{Documento 83. La institución del matrimonio}
\par
%\textsuperscript{(922.1)}
\textsuperscript{83:0.1} ÉSTA es la narración de los primeros comienzos de la institución del matrimonio. Éste ha progresado continuamente desde los apareamientos licenciosos y promiscuos dentro de la horda, pasando por muchas variaciones y adaptaciones, hasta la aparición de las normas matrimoniales que culminaron finalmente en la realización de las uniones en parejas, la unión de un hombre y una mujer para establecer un hogar del orden social más elevado.

\par
%\textsuperscript{(922.2)}
\textsuperscript{83:0.2} El matrimonio ha estado muchas veces en peligro, y las costumbres matrimoniales han recurrido muy a menudo tanto a la propiedad privada como a la religión en busca de apoyo; pero la verdadera influencia que protege constantemente al matrimonio y a la familia resultante es el hecho biológico simple e innato de que los hombres y las mujeres no pueden vivir realmente los unos sin los otros, ya se trate de los salvajes más primitivos o de los mortales más cultos.

\par
%\textsuperscript{(922.3)}
\textsuperscript{83:0.3} A causa del impulso sexual, el hombre egoísta es atraído a convertirse en algo mejor que un animal fuera de sí. Las relaciones sexuales gratificantes y dignas implican las consecuencias seguras de la abnegación, y aseguran la asunción de deberes altruistas y de numerosas responsabilidades familiares beneficiosas para la raza. En esto es en lo que el sexo ha sido el civilizador desconocido e insospechado de los salvajes, porque este mismo impulso sexual \textit{obliga al hombre} automática e infaliblemente \textit{a pensary lo conduce} finalmente \textit{a amar}.

\section*{1. El matrimonio como institución social}
\par
%\textsuperscript{(922.4)}
\textsuperscript{83:1.1} El matrimonio es el mecanismo que la sociedad ha concebido para regular y controlar las múltiples relaciones humanas que se originan por el hecho físico de la bisexualidad. Como tal institución, el matrimonio funciona en dos direcciones:

\par
%\textsuperscript{(922.5)}
\textsuperscript{83:1.2} 1. En la reglamentación de las relaciones sexuales personales.

\par
%\textsuperscript{(922.6)}
\textsuperscript{83:1.3} 2. En la reglamentación de la descendencia, la herencia, la sucesión y el orden social, siendo ésta su función original más antigua.

\par
%\textsuperscript{(922.7)}
\textsuperscript{83:1.4} La familia, que nace del matrimonio, es en sí misma una estabilizadora de la institución matrimonial, junto con las costumbres sobre la propiedad. Otros factores poderosos en la estabilidad del matrimonio son el orgullo, la vanidad, la caballerosidad, el deber y las convicciones religiosas. Pero, aunque los matrimonios puedan ser aprobados o desaprobados en las alturas, difícilmente se concluyen en el cielo. La familia humana es una institución claramente humana, un desarrollo evolutivo. El matrimonio es una institución de la sociedad, no un negociado de la iglesia. Es verdad que la religión debería influir poderosamente sobre él, pero no debería intentar controlarlo y reglamentarlo de manera exclusiva.

\par
%\textsuperscript{(922.8)}
\textsuperscript{83:1.5} El matrimonio primitivo era principalmente laboral, e incluso en los tiempos modernos, es a menudo un asunto social o comercial. Gracias a la influencia de la mezcla del linaje andita y a consecuencia de las costumbres de la civilización progresiva, el matrimonio se está volviendo lentamente mutuo, romántico, parental, poético, afectuoso, ético e incluso idealista. Sin embargo, la elección y el amor llamado romántico jugaban un papel mínimo en el emparejamiento primitivo. En los tiempos antiguos, el marido y la mujer no pasaban mucho tiempo juntos; ni siquiera comían juntos muy a menudo. Pero entre los antiguos, el afecto personal no estaba estrechamente vinculado con la atracción sexual; se tomaban cariño el uno al otro debido principalmente a la vida y al trabajo en común.

\section*{2. El cortejo y los esponsales}
\par
%\textsuperscript{(923.1)}
\textsuperscript{83:2.1} Los matrimonios primitivos eran siempre planeados por los padres del muchacho y de la joven. La etapa de transición entre esta costumbre y la de la época de la libre elección estuvo ocupada por los agentes matrimoniales o casamenteros profesionales. Al principio, estos casamenteros fueron los barberos, y más adelante los sacerdotes. El matrimonio fue, originariamente, un asunto del grupo, y luego una cuestión familiar; sólo recientemente se ha convertido en una aventura individual.

\par
%\textsuperscript{(923.2)}
\textsuperscript{83:2.2} La coacción, y no la atracción, era el camino de acceso al matrimonio primitivo. En los tiempos antiguos, la mujer no tenía ninguna actitud sexual distante, sino únicamente la inferioridad sexual que le inculcaban las costumbres. De la misma manera que las incursiones precedieron al comercio, el matrimonio por captura precedió al matrimonio por contrato. Algunas mujeres fingían ser capturadas para escapar de la dominación de los hombres más viejos de su tribu. Preferían caer en manos de los hombres de su propia edad pertenecientes a otra tribu. Estas supuestas fugas fueron la etapa de transición entre la captura por la fuerza y el posterior cortejo por atracción.

\par
%\textsuperscript{(923.3)}
\textsuperscript{83:2.3} Había un tipo primitivo de ceremonia nupcial que consistía en la huida fingida, una especie de simulacro de fuga que en otro tiempo se había practicado habitualmente. Más tarde, la captura simulada se convirtió en una parte de la ceremonia regular de la boda. Las pretensiones que manifiesta una chica moderna de oponerse a la <<captura>>, de mostrarse reticente al matrimonio, no son más que reliquias de costumbres antiguas. Cruzar el umbral con la novia en brazos es una reminiscencia de numerosas prácticas antiguas, entre otras las de los tiempos en que se robaban las esposas.

\par
%\textsuperscript{(923.4)}
\textsuperscript{83:2.4} A la mujer se le negó durante mucho tiempo la plena libertad de decidir por sí misma en el asunto del matrimonio, pero las mujeres más inteligentes siempre han sido capaces de burlar esta restricción mediante el hábil ejercicio de su ingenio. El hombre ha tomado generalmente la delantera en el cortejo, pero no siempre. La mujer, unas veces formalmente y otras de manera encubierta, inicia el proceso del casamiento. Y a medida que la civilización ha progresado, las mujeres han participado cada vez más en todas las fases del cortejo y del matrimonio.

\par
%\textsuperscript{(923.5)}
\textsuperscript{83:2.5} El amor, el romanticismo y la elección personal crecientes del cortejo prenupcial son una aportación de los anditas a las razas del mundo. Las relaciones entre los sexos evolucionan favorablemente; muchos pueblos progresivos están sustituyendo gradualmente los antiguos móviles de la utilidad y la propiedad por los conceptos un poco idealizados de la atracción sexual. El impulso sexual y los sentimientos afectivos están empezando a desplazar a la manera fría y calculadora de elegir a los compañeros de vida.

\par
%\textsuperscript{(923.6)}
\textsuperscript{83:2.6} Al principio, los esponsales equivalían al matrimonio, y entre los pueblos primitivos, las relaciones sexuales eran habituales durante el noviazgo. En tiempos más recientes, la religión ha establecido un tabú sexual sobre el período comprendido entre los esponsales y el casamiento.

\section*{3. La compra y la dote}
\par
%\textsuperscript{(923.7)}
\textsuperscript{83:3.1} Los antiguos desconfiaban del amor y de las promesas; pensaban que las uniones duraderas tenían que estar garantizadas por alguna seguridad tangible, por la propiedad. Por este motivo, el precio de adquisición de una esposa era considerado como una prenda o depósito, que el marido estaba condenado a perder en caso de divorcio o abandono. Una vez que se había pagado el precio de adquisición de una novia, muchas tribus permitían que le pusieran con hierro candente la marca del marido. Los africanos todavía compran a sus esposas. A una esposa que se casa por amor, o a la esposa de un hombre blanco, la comparan con un gato porque no cuesta nada.

\par
%\textsuperscript{(924.1)}
\textsuperscript{83:3.2} Los desfiles de novias eran un motivo para vestir elegantemente y adornar a las hijas, a fin de mostrarlas en público con la idea de conseguir un precio más alto como esposas\footnote{\textit{Comprar esposas}: Gn 29:18-20.}. Pero no las vendían como animales ---en las tribus más tardías, estas esposas no eran transferibles. Su adquisición tampoco era siempre una transacción monetaria efectuada a sangre fría; los servicios prestados equivalían al dinero en efectivo en la adquisición de una esposa. Si un hombre, por otra parte deseable, no podía pagar el precio de su esposa, podía ser adoptado como hijo por el padre de la muchacha, y luego podía casarse. Y si un hombre pobre aspiraba a tener una esposa y no podía satisfacer el precio exigido por un padre codicioso, los ancianos solían con frecuencia presionar al padre para que éste modificara sus exigencias, o de lo contrario su hija podía fugarse.

\par
%\textsuperscript{(924.2)}
\textsuperscript{83:3.3} A medida que progresó la civilización, los padres no quisieron dar la impresión de que vendían a sus hijas, y así, aunque continuaban aceptando el precio de adquisición de la novia, introdujeron la costumbre de dar a la pareja unos regalos valiosos que equivalían prácticamente al dinero de la compra. Más tarde, cuando se dejó de pagar para obtener una esposa, estos regalos se convirtieron en la dote de la novia.

\par
%\textsuperscript{(924.3)}
\textsuperscript{83:3.4} La idea de la dote consistía en transmitir la impresión de que la novia era independiente, en insinuar que se estaba muy lejos de los tiempos de las esposas esclavas y de las compañeras consideradas como una propiedad. Un hombre no podía divorciarse de una esposa con dote sin devolver toda la dote. En algunas tribus se entregaba un depósito mutuo a los padres del novio y de la novia, el cual se perdía en caso de que uno de ellos abandonara al otro; se trataba en verdad de una fianza matrimonial. Durante el período de transición entre la compra y la dote, si la esposa había sido comprada, los hijos pertenecían al padre; en caso contrario pertenecían a la familia de la madre.

\section*{4. La ceremonia nupcial}
\par
%\textsuperscript{(924.4)}
\textsuperscript{83:4.1} La ceremonia de la boda surgió del hecho de que el matrimonio era en un principio un asunto de la comunidad, y no simplemente la culminación de una decisión de dos personas. El emparejamiento era una preocupación del grupo, así como un acto personal.

\par
%\textsuperscript{(924.5)}
\textsuperscript{83:4.2} Toda la vida de los antiguos estaba rodeada de magia, de rituales y de ceremonias, y el matrimonio no era una excepción. A medida que avanzó la civilización, a medida que el matrimonio se consideró con más seriedad, la ceremonia de la boda se volvió cada vez más presuntuosa. El matrimonio primitivo era un factor en los intereses relacionados con la propiedad, tal como lo es hoy en día, y por eso necesitaba una ceremonia legal, mientras que la posición social de los hijos por venir exigía la mayor publicidad posible. El hombre primitivo no tenía archivos; por eso la ceremonia del matrimonio tenía que ser presenciada por muchas personas.

\par
%\textsuperscript{(924.6)}
\textsuperscript{83:4.3} Al principio, la ceremonia nupcial tenía más bien el carácter de unos esponsales, y sólo consistía en la notificación pública de la intención de vivir juntos; más tarde consistió en compartir formalmente una comida. En algunas tribus los padres se limitaban a entregar su hija al marido; en otros casos, la única ceremonia era el intercambio formal de los regalos, después de lo cual el padre de la novia la entregaba al novio. Muchos pueblos levantinos tenían la costumbre de prescindir de toda formalidad, y el matrimonio se consumaba mediante las relaciones sexuales. El hombre rojo fue el primero que desarrolló las celebraciones nupciales más elaboradas.

\par
%\textsuperscript{(924.7)}
\textsuperscript{83:4.4} Se tenía mucho miedo a no tener hijos, y como la esterilidad se atribuía a las maquinaciones de los espíritus, los esfuerzos por asegurar la fecundidad condujeron también a asociar el matrimonio con ciertos ceremoniales mágicos o religiosos. En este esfuerzo por asegurar un matrimonio fecundo y feliz se empleaban muchos hechizos; incluso se consultaba a los astrólogos para que averiguaran las estrellas de la buena suerte bajo las que habían nacido las partes contrayentes. En cierta época, los sacrificios humanos fueron una característica habitual en todas las bodas de la gente adinerada.

\par
%\textsuperscript{(925.1)}
\textsuperscript{83:4.5} Se buscaban los días que traían suerte, y el jueves se consideraba como el más favorable; se creía que las bodas que se celebraban en Luna llena eran excepcionalmente afortunadas. Muchos pueblos del Cercano Oriente tenían la costumbre de arrojar granos sobre los recién casados; era un rito mágico que se suponía que aseguraba la fecundidad. Algunos pueblos orientales utilizaban el arroz con esta finalidad.

\par
%\textsuperscript{(925.2)}
\textsuperscript{83:4.6} El fuego y el agua siempre fueron considerados como los mejores medios de oponer resistencia a los fantasmas y a los espíritus malignos; en consecuencia, los fuegos sobre el altar y las velas encendidas, así como las aspersiones bautismales con agua bendita, estaban generalmente de manifiesto en las bodas. Durante mucho tiempo se tuvo la costumbre de fijar un día falso para la boda, y luego se aplazaba repentinamente el acontecimiento para despistar a los fantasmas y los espíritus.

\par
%\textsuperscript{(925.3)}
\textsuperscript{83:4.7} Todas las tomaduras de pelo a los recién casados y las bromas que se gastan a las parejas en luna de miel son reliquias de aquellos días lejanos en que se pensaba que era mejor parecer desgraciado e incómodo a los ojos de los espíritus, para evitar despertar su envidia. El uso del velo nupcial es una reliquia de los tiempos en que se consideraba necesario disfrazar a la novia para que los fantasmas no pudieran reconocerla, y también para ocultar su belleza a las miradas, por otra parte celosas y envidiosas, de los espíritus. Los pies de la novia nunca debían tocar el suelo justo antes de la ceremonia. Incluso en el siglo veinte sigue siendo tradición, bajo las costumbres cristianas, extender una alfombra desde el vehículo nupcial hasta el altar de la iglesia.

\par
%\textsuperscript{(925.4)}
\textsuperscript{83:4.8} Una de las formas más antiguas de la ceremonia nupcial consistía en que un sacerdote bendijera el lecho nupcial para asegurar la fecundidad de la unión; esto se hacía mucho tiempo antes de que se estableciera cualquier rito nupcial formal. Durante este período de la evolución de las costumbres matrimoniales, se contaba con que los invitados a la boda desfilarían de noche por la cámara nupcial, convirtiéndose así en los testigos legales de la consumación del matrimonio.

\par
%\textsuperscript{(925.5)}
\textsuperscript{83:4.9} El elemento suerte, que hacía que algunos matrimonios salieran mal a pesar de todas las pruebas prenupciales, condujo al hombre primitivo a buscar una seguridad para protegerse contra el fracaso matrimonial, induciéndole a recurrir a los sacerdotes y la magia. Este movimiento culminó directamente en los casamientos modernos en la iglesia. Pero durante mucho tiempo se admitió generalmente que el matrimonio consistía en la decisión de los padres contratantes ---y más tarde de la pareja--- mientras que en los últimos quinientos años, la iglesia y el Estado han asumido la jurisdicción y se atreven a hacer pronunciamientos sobre el matrimonio.

\section*{5. Los matrimonios múltiples}
\par
%\textsuperscript{(925.6)}
\textsuperscript{83:5.1} Al principio de la historia del matrimonio, las mujeres solteras pertenecían a los hombres de la tribu. Más tarde, las mujeres sólo tenían un marido a la vez. Esta práctica de \textit{un-solo-hombre-a-la-vez} fue el primer paso para alejarse de la promiscuidad de la horda. Aunque a la mujer sólo se le permitía tener un solo hombre, su marido podía romper a voluntad estas relaciones temporales\footnote{\textit{Divorcio para el hombre}: Dt 24:1.}. Pero estas asociaciones reglamentadas de manera imprecisa fueron el primer paso hacia la vida en pareja, en contraste con la vida en la horda. En esta etapa del desarrollo del matrimonio, los hijos pertenecían generalmente a la madre.

\par
%\textsuperscript{(925.7)}
\textsuperscript{83:5.2} El paso siguiente en la evolución del emparejamiento fue el \textit{matrimonio colectivo}. Esta fase comunal del matrimonio tuvo que existir en el desarrollo de la vida familiar, porque las costumbres matrimoniales no eran todavía lo bastante fuertes como para hacer que las asociaciones en pareja fueran permanentes. Los matrimonios de hermanos y hermanas pertenecieron a este grupo; cinco hermanos de una familia solían casarse con cinco hermanas de otra. En todo el mundo, las formas más imprecisas de matrimonios comunales se transformaron gradualmente en diversos tipos de matrimonios colectivos. Estas asociaciones colectivas fueron reglamentadas principalmente por las costumbres del tótem. La vida familiar se desarrolló de manera lenta y segura porque la reglamentación del sexo y del matrimonio favoreció la supervivencia de la tribu misma al asegurar la supervivencia de un mayor número de niños.

\par
%\textsuperscript{(926.1)}
\textsuperscript{83:5.3} Los matrimonios colectivos fueron reemplazados gradualmente por las prácticas emergentes de la poligamia ---la poliginia y la poliandria--- en las tribus más avanzadas. Pero la poliandria nunca estuvo generalizada, limitándose normalmente a las reinas y a las mujeres ricas; además, se trataba habitualmente de un asunto de familia, una esposa para varios hermanos. Las restricciones económicas y de casta hicieron a veces necesario que varios hombres se contentaran con una sola esposa. Incluso entonces, la mujer sólo se casaba con uno, y los otros eran tolerados vagamente como <<tíos>> de la progenie conjunta.

\par
%\textsuperscript{(926.2)}
\textsuperscript{83:5.4} La costumbre judía de exigir que un hombre se uniera con la viuda de su hermano fallecido a fin de <<conseguir una descendencia para su hermano>>\footnote{\textit{Matrimonio del levirato}: Gn 38:6-10; Dt 25:5-6; Mt 22:24; Mc 12:19; Lc 20:28.}, era una costumbre que existía en más de la mitad del mundo antiguo. Era una reliquia de la época en que el matrimonio era un asunto de familia más bien que una asociación individual.

\par
%\textsuperscript{(926.3)}
\textsuperscript{83:5.5} La institución de la poliginia reconoció, en épocas diversas, cuatro tipos de esposas:

\par
%\textsuperscript{(926.4)}
\textsuperscript{83:5.6} 1. Las esposas ceremoniales o legales.

\par
%\textsuperscript{(926.5)}
\textsuperscript{83:5.7} 2. Las esposas amadas y permitidas.

\par
%\textsuperscript{(926.6)}
\textsuperscript{83:5.8} 3. Las concubinas, las esposas contractuales.

\par
%\textsuperscript{(926.7)}
\textsuperscript{83:5.9} 4. Las esposas esclavas.

\par
%\textsuperscript{(926.8)}
\textsuperscript{83:5.10} La verdadera poliginia, en la que todas las esposas tenían la misma categoría y todos los hijos eran iguales, ha sido muy rara. Habitualmente, incluso en los matrimonios múltiples, el hogar estaba dominado por la esposa principal, la compañera reconocida. Sólo ella tenía derecho a la ceremonia de boda ritual, y sólo los hijos de esta esposa comprada o con dote podían heredar, a menos que se hiciera un acuerdo especial con ella.

\par
%\textsuperscript{(926.9)}
\textsuperscript{83:5.11} La esposa legal no era necesariamente la esposa amada; en los tiempos primitivos generalmente no lo era. La esposa amada, o dulce amor, no apareció hasta que las razas hubieron avanzado considerablemente, y más específicamente después de la mezcla de las tribus evolutivas con los noditas y los adamitas.

\par
%\textsuperscript{(926.10)}
\textsuperscript{83:5.12} La esposa tabú ---la única esposa con una situación legal--- creó las costumbres de las concubinas. Bajo estas costumbres, un hombre sólo podía tener una esposa, pero podía mantener relaciones sexuales con un número indeterminado de concubinas. El concubinato fue el trampolín hacia la monogamia, el primer paso para alejarse de la franca poliginia. Las concubinas de los judíos, los romanos y los chinos eran con mucha frecuencia las criadas de la esposa. Más tarde, tal como sucedió entre los judíos, la esposa legal fue considerada como la madre de todos los hijos engendrados por el marido.

\par
%\textsuperscript{(926.11)}
\textsuperscript{83:5.13} Los antiguos tabúes sobre las relaciones sexuales con una esposa embarazada o lactante tendieron a fomentar enormemente la poliginia. Las mujeres primitivas envejecían muy pronto debido a sus frecuentes maternidades unidas al duro trabajo que realizaban. (Estas esposas sobrecargadas sólo se las ingeniaban para existir gracias al hecho de que se las aislaba una semana por mes cuando no estaban embarazadas). Estas esposas se cansaban con frecuencia de tener hijos y le pedían a su marido que tomara una segunda esposa más joven, capaz de ayudar tanto en la procreación como en el trabajo doméstico. Por esta razón, las nuevas esposas eran acogidas generalmente con regocijo por las más antiguas; no existía nada que se pareciera a los celos sexuales.

\par
%\textsuperscript{(926.12)}
\textsuperscript{83:5.14} El número de esposas sólo estaba limitado por la capacidad del hombre para mantenerlas. Los hombres ricos y capaces querían un gran número de hijos, y como la mortalidad infantil era muy elevada, se necesitaba un grupo de esposas para conseguir una familia numerosa. Muchas de estas esposas múltiples eran simples trabajadoras, esposas esclavas.

\par
%\textsuperscript{(927.1)}
\textsuperscript{83:5.15} Las costumbres humanas evolucionan, pero muy lentamente. La finalidad del harén consistía en crear un grupo fuerte y numeroso de parientes consanguíneos para que apoyaran el trono. Cierto jefe se convenció una vez de que no debía tener un harén, de que debía contentarse con una sola esposa; así pues, se deshizo inmediatamente de su harén. Las esposas descontentas regresaron a sus hogares, y sus parientes ofendidos se abalanzaron enfurecidos sobre el jefe y lo mataron de inmediato.

\section*{6. La verdadera monogamia ---el matrimonio de una pareja}
\par
%\textsuperscript{(927.2)}
\textsuperscript{83:6.1} La monogamia es un monopolio; es buena para aquellos que alcanzan este estado deseable, pero tiende a causar dificultades biológicas a aquellos que no son tan afortunados. Pero independientemente de su efecto sobre el individuo, la monogamia es indudablemente lo mejor para los hijos.

\par
%\textsuperscript{(927.3)}
\textsuperscript{83:6.2} La monogamia más primitiva se debía a la fuerza de las circunstancias, a la pobreza. La monogamia es cultural y social, artificial y antinatural, es decir, antinatural para el hombre evolutivo. Era totalmente natural para los noditas y adamitas más puros y ha sido de un gran valor cultural para todas las razas avanzadas.

\par
%\textsuperscript{(927.4)}
\textsuperscript{83:6.3} Las tribus caldeas reconocían el derecho que tenía una esposa de imponer a su marido la promesa prenupcial de que no tomaría una segunda esposa o una concubina. Tanto los griegos como los romanos favorecieron el matrimonio monógamo. El culto a los antepasados ha fomentado siempre la monogamia, así como el error cristiano de considerar el matrimonio como un sacramento. Incluso la elevación del nivel de vida ha militado firmemente en contra de las esposas múltiples. En la época de la venida de Miguel a Urantia, prácticamente todo el mundo civilizado había alcanzado el nivel de la monogamia teórica. Pero esta monogamia pasiva no significaba que la humanidad se hubiera habituado a la práctica de los verdaderos matrimonios en pareja.

\par
%\textsuperscript{(927.5)}
\textsuperscript{83:6.4} Al mismo tiempo que persigue la meta monógama del matrimonio ideal en pareja, que se parece, después de todo, a una asociación sexual monopolizadora, la sociedad no debe pasar por alto la situación poco envidiable de aquellos hombres y mujeres desafortunados que no logran encontrar su lugar en este orden social nuevo y mejor, incluso después de haber hecho todo lo posible por cooperar con sus exigencias y cumplir con ellas. La imposibilidad de conseguir una pareja en el terreno social de la competencia puede deberse a dificultades insuperables o a restricciones múltiples que han sido impuestas por las costumbres corrientes. En verdad, la monogamia es ideal para aquellos que están dentro de ella, pero ha de causar inevitablemente grandes dificultades a aquellos que se quedan fuera en el frío de una existencia solitaria.

\par
%\textsuperscript{(927.6)}
\textsuperscript{83:6.5} Unos pocos desafortunados siempre han tenido que sufrir para que la mayoría pueda avanzar bajo las costumbres en desarrollo de la civilización evolutiva; pero la mayoría favorecida debería mirar siempre con bondad y consideración a sus compañeros menos afortunados, que deben pagar el precio de no conseguir entrar en las filas de esas asociaciones sexuales ideales que proporcionan la satisfacción de todos los impulsos biológicos bajo la autorización de las costumbres más elevadas de la evolución social en progreso.

\par
%\textsuperscript{(927.7)}
\textsuperscript{83:6.6} La monogamia ha sido siempre, es ahora, y será siempre, la meta idealista de la evolución sexual humana. Este ideal del verdadero matrimonio en pareja implica la abnegación, y por eso fracasa tan a menudo, simplemente porque una de las partes contrayentes, o las dos, carecen de la más grande de todas las virtudes humanas: el riguroso control de sí mismo.

\par
%\textsuperscript{(927.8)}
\textsuperscript{83:6.7} La monogamia es la vara que mide el avance de la civilización social, en contraste con la evolución puramente biológica. La monogamia no es necesariamente biológica o natural, pero es indispensable para el mantenimiento inmediato y el desarrollo ulterior de la civilización social. Contribuye a una delicadeza de sentimientos, a un refinamiento del carácter moral y a un crecimiento espiritual que son totalmente imposibles en la poligamia. Una mujer no puede convertirse nunca en una madre ideal cuando se ve todo el tiempo obligada a competir por el afecto de su marido.

\par
%\textsuperscript{(928.1)}
\textsuperscript{83:6.8} El matrimonio en pareja favorece y fomenta la comprensión íntima y la cooperación eficaz, que son las mejores cosas para la felicidad de los padres, el bienestar de los hijos y la eficiencia social. El matrimonio, que empezó siendo una vulgar coacción, evoluciona gradualmente hacia una magnífica institución de refinamiento de sí mismo, de autocontrol, de expresión personal y de perpetuación de sí mismo.

\section*{7. La disolución del matrimonio}
\par
%\textsuperscript{(928.2)}
\textsuperscript{83:7.1} En la evolución primitiva de las costumbres maritales, el matrimonio era una unión vaga que podía finalizar a voluntad, y los hijos siempre seguían a la madre; el vínculo entre la madre y el hijo es instintivo y ha funcionado sin tener en cuenta el grado de desarrollo de las costumbres.

\par
%\textsuperscript{(928.3)}
\textsuperscript{83:7.2} En los pueblos primitivos, aproximadamente sólo la mitad de los matrimonios resultaban satisfactorios. La causa más frecuente de separación era la esterilidad, de la que siempre se culpaba a la esposa; y se creía que las esposas sin hijos se volvían serpientes en el mundo del espíritu. Bajo las costumbres más primitivas, el divorcio se concedía únicamente a petición del hombre, y estas normas han subsistido en algunos pueblos hasta el siglo veinte.

\par
%\textsuperscript{(928.4)}
\textsuperscript{83:7.3} A medida que evolucionaron las costumbres, algunas tribus desarrollaron dos tipos de matrimonios: el matrimonio corriente, que permitía el divorcio, y el matrimonio ante un sacerdote, que no autorizaba la separación. La introducción de la compra y de la dote de las esposas contribuyó mucho a reducir las separaciones, mediante la imposición de una multa sobre la propiedad por el fracaso del matrimonio. Y en verdad, muchas uniones modernas están estabilizadas gracias a este antiguo factor de la propiedad.

\par
%\textsuperscript{(928.5)}
\textsuperscript{83:7.4} La presión social ejercida por la posición dentro de la comunidad y por los privilegios que otorga la propiedad siempre ha tenido el poder de mantener los tabúes y las costumbres sobre el matrimonio. A lo largo de las épocas, el matrimonio ha hecho progresos continuos y se encuentra en una posición avanzada en el mundo moderno, a pesar de que está siendo atacado de manera amenazadora por una insatisfacción generalizada en aquellos pueblos donde la elección individual ---una nueva libertad--- juega un papel preponderante. Aunque estos trastornos de adaptación aparecen entre las razas más progresivas a consecuencia de la aceleración repentina de la evolución social, el matrimonio continúa prosperando y mejorando lentamente entre los pueblos menos avanzados, bajo la dirección de las antiguas costumbres.

\par
%\textsuperscript{(928.6)}
\textsuperscript{83:7.5} La sustitución nueva y repentina, en el matrimonio, del antiguo móvil de la propiedad establecido durante mucho tiempo, por el móvil del amor, más ideal pero extremadamente individualista, ha provocado inevitablemente una inestabilidad temporal en la institución del matrimonio. Los móviles del hombre para casarse han trascendido siempre de lejos la moral matrimonial efectiva, y en los siglos diecinueve y veinte, el ideal occidental del matrimonio ha sobrepasado repentinamente con mucho los impulsos sexuales egocéntricos, pero sólo parcialmente controlados, de las razas. La presencia en cualquier sociedad de una gran cantidad de personas no casadas indica la crisis temporal o la transición de las costumbres.

\par
%\textsuperscript{(928.7)}
\textsuperscript{83:7.6} A lo largo de todas las épocas, la verdadera prueba del matrimonio ha sido esa continua intimidad que es inevitable en toda vida familiar. Dos jóvenes mimados y consentidos, educados para contar con todo tipo de complacencias y la plena satisfacción de su vanidad y su ego, difícilmente pueden esperar tener un gran éxito en su matrimonio y en la construcción de un hogar ---una asociación para toda una vida de abnegación, compromiso, devoción y dedicación desinteresada a la educación de los hijos.

\par
%\textsuperscript{(929.1)}
\textsuperscript{83:7.7} El alto grado de imaginación y de romanticismo fantástico que se introducen en el noviazgo es en gran parte responsable de las tendencias crecientes al divorcio de los pueblos occidentales modernos, todo lo cual se complica aún más debido a la mayor libertad personal de la mujer y a su independencia económica creciente. El divorcio fácil, cuando es el resultado de una falta de autocontrol o de un fallo de adaptación normal de la personalidad, sólo conduce directamente a las antiguas etapas sociales rudimentarias de las que el hombre ha surgido tan recientemente como consecuencia de tantas angustias personales y sufrimientos raciales.

\par
%\textsuperscript{(929.2)}
\textsuperscript{83:7.8} Pero mientras la sociedad no logre educar convenientemente a los niños y a los jóvenes, mientras el orden social no consiga proporcionar una formación prematrimonial adecuada, y mientras el idealismo de una juventud sin sabiduría ni madurez sea el árbitro para entrar en el matrimonio, el divorcio continuará predominando. En la medida en que el grupo social no consiga proporcionar una preparación matrimonial a los jóvenes, el divorcio deberá funcionar como una válvula de seguridad de la sociedad para impedir situaciones aún peores durante los períodos de rápido crecimiento de las costumbres en evolución.

\par
%\textsuperscript{(929.3)}
\textsuperscript{83:7.9} Los antiguos parecen haber considerado el matrimonio casi con tanta seriedad como algunos pueblos actuales. Y muchos matrimonios apresurados y fracasados de los tiempos modernos no parecen ser una mejora con respecto a las prácticas antiguas que capacitaban a los chicos y las chicas para el emparejamiento. La gran contradicción de la sociedad moderna consiste en ensalzar el amor e idealizar el matrimonio, desaprobando al mismo tiempo un examen profundo de los dos.

\section*{8. La idealización del matrimonio}
\par
%\textsuperscript{(929.4)}
\textsuperscript{83:8.1} El matrimonio que culmina en un hogar es en verdad la institución más sublime del hombre, pero es esencialmente humano; nunca debería haber sido calificado de sacramento. Los sacerdotes setitas hicieron del matrimonio un ritual religioso; pero durante miles de años después del Edén, el emparejamiento continuó siendo una institución puramente social y civil.

\par
%\textsuperscript{(929.5)}
\textsuperscript{83:8.2} La comparación entre las asociaciones humanas y las asociaciones divinas es sumamente desacertada. La unión del marido y la mujer en la relación del matrimonio y del hogar es una función material de los mortales de los mundos evolutivos. Es verdad, naturalmente, que se pueden conseguir muchos progresos espirituales a consecuencia de los sinceros esfuerzos humanos del marido y la mujer por evolucionar, pero esto no significa que el matrimonio sea necesariamente sagrado. El progreso espiritual acompaña a la dedicación sincera en otros campos del empeño humano.

\par
%\textsuperscript{(929.6)}
\textsuperscript{83:8.3} El matrimonio tampoco puede compararse realmente con la relación entre el Ajustador y el hombre, ni con la fraternidad entre Cristo Miguel y sus hermanos humanos. Estas relaciones apenas son comparables en ningún punto con la asociación entre marido y mujer. Y es muy lamentable que el concepto erróneo humano de estas relaciones haya producido tanta confusión en lo referente al estado del matrimonio.

\par
%\textsuperscript{(929.7)}
\textsuperscript{83:8.4} También es lamentable que ciertos grupos de mortales hayan imaginado que el matrimonio era consumado por un acto divino. Estas creencias conducen directamente al concepto de la indisolubilidad del estado matrimonial, sin tener en cuenta las circunstancias o los deseos de las partes contrayentes. Pero el hecho mismo de que el matrimonio pueda disolverse indica que la Deidad no es una parte conjunta de estas uniones. Una vez que Dios ha unido dos cosas o dos personas cualquiera, éstas permanecerán unidas así hasta el momento en que la voluntad divina decrete su separación. Pero en lo que se refiere al matrimonio, que es una institución humana, ¿quién se atreverá a juzgarlo para decir cuáles son las uniones que pueden ser aprobadas por los supervisores del universo, en contraste con aquellas cuya naturaleza y origen son puramente humanos?

\par
%\textsuperscript{(930.1)}
\textsuperscript{83:8.5} Sin embargo, existe un ideal del matrimonio en las esferas de las alturas. En la capital de cada sistema local, los Hijos e Hijas Materiales de Dios describen de hecho el punto culminante de los ideales de la unión de un hombre y una mujer en los lazos del matrimonio y con la finalidad de procrear y criar una descendencia. Después de todo, el matrimonio ideal de los mortales es \textit{humanamente} sagrado.

\par
%\textsuperscript{(930.2)}
\textsuperscript{83:8.6} El matrimonio ha sido siempre, y continua siendo, el sueño supremo del ideal temporal del hombre. Aunque este hermoso sueño se realiza muy pocas veces en su totalidad, perdura como un glorioso ideal, atrayendo siempre a la humanidad en evolución hacia unos esfuerzos más grandes por la felicidad humana. Pero a los jóvenes de ambos sexos se les debería enseñar algunas cosas sobre las realidades del matrimonio, antes de sumergirse en las exigencias rigurosas de las interasociaciones de la vida familiar; la idealización juvenil debería ser moderada con cierto grado de desilusión prematrimonial.

\par
%\textsuperscript{(930.3)}
\textsuperscript{83:8.7} Sin embargo, la idealización juvenil del matrimonio no debería ser desalentada; estos sueños constituyen la visualización de la meta futura de la vida familiar. Esta actitud es estimulante y útil a la vez, a condición de que no produzca una insensibilidad para llevar a cabo las exigencias prácticas y corrientes del matrimonio y de la vida familiar ulterior.

\par
%\textsuperscript{(930.4)}
\textsuperscript{83:8.8} Los ideales del matrimonio han hecho recientemente grandes progresos; en algunos pueblos, la mujer disfruta prácticamente de los mismos derechos que su consorte. La familia se está convirtiendo, al menos en concepto, en una asociación leal para criar a los hijos, acompañada de fidelidad sexual. Pero incluso esta versión más nueva del matrimonio no tiene necesidad de atreverse a llegar hasta el extremo de conferir un monopolio mutuo de toda la personalidad y la individualidad. El matrimonio no es simplemente un ideal individualista; es la asociación social evolutiva de un hombre y una mujer, que existe y funciona bajo las costumbres admitidas, limitada por los tabúes y reforzada por las leyes y las reglamentaciones de la sociedad.

\par
%\textsuperscript{(930.5)}
\textsuperscript{83:8.9} Los matrimonios del siglo veinte se encuentran en un nivel elevado en comparación con los de los tiempos pasados, a pesar de que la institución del hogar está pasando actualmente por una dura prueba a causa de los problemas que el aumento precipitado de las libertades de la mujer ha impuesto tan repentinamente a la organización social, unos derechos que le han sido negados durante tanto tiempo a lo largo de la lenta evolución de las costumbres de las generaciones pasadas.

\par
%\textsuperscript{(930.6)}
\textsuperscript{83:8.10} [Presentado por el Jefe de los Serafines estacionado en Urantia.]


\chapter{Documento 84. El matrimonio y la vida familiar}
\par
%\textsuperscript{(931.1)}
\textsuperscript{84:0.1} LA NECESIDAD material fundó el matrimonio, el apetito sexual lo embelleció, la religión lo aprobó y lo ensalzó, el Estado lo exigió y lo reglamentó, mientras que en tiempos más recientes, el amor en evolución empieza a justificar y a glorificar el matrimonio como el antepasado y el creador de la institución más útil y sublime de la civilización: el hogar. La formación del hogar debería ser el centro y la esencia de todos los esfuerzos educativos.

\par
%\textsuperscript{(931.2)}
\textsuperscript{84:0.2} El apareamiento es puramente un acto de perpetuación de sí mismo, asociado con grados variables de satisfacción de sí mismo; el matrimonio, la formación de un hogar, es en gran parte una cuestión de preservación de sí mismo, e implica la evolución de la sociedad. La sociedad misma es la estructura global de las unidades familiares. Como factores planetarios, los individuos son muy transitorios ---sólo las familias son los agentes continuos en la evolución social. La familia es el canal por el que fluye el río de la cultura y del conocimiento de una generación a la siguiente.

\par
%\textsuperscript{(931.3)}
\textsuperscript{84:0.3} El hogar es básicamente una institución sociológica. El matrimonio surgió de la cooperación para sustentarse y de la asociación para perpetuarse, siendo la satisfacción de sí mismo un elemento ampliamente accesorio. Sin embargo, el hogar abarca las tres funciones esenciales de la existencia humana, mientras que la propagación de la vida lo convierte en la institución fundamental humana, y el sexo lo separa de todas las demás actividades sociales.

\section*{1. Las asociaciones primitivas en pareja}
\par
%\textsuperscript{(931.4)}
\textsuperscript{84:1.1} El matrimonio no se construyó sobre las relaciones sexuales; éstas eran accesorias en el mismo. El hombre primitivo no tenía necesidad del matrimonio; daba rienda suelta libremente a su apetito sexual sin cargarse con las responsabilidades de una esposa, unos hijos y un hogar.

\par
%\textsuperscript{(931.5)}
\textsuperscript{84:1.2} A causa de su apego físico y emocional a sus hijos, la mujer depende de la cooperación del hombre, y esto la incita a buscar el refugio protector del matrimonio. Pero ningún impulso biológico directo condujo al hombre al matrimonio ---y mucho menos lo retuvo allí. El amor no fue lo que hizo atractivo el matrimonio para el hombre; fue el hambre lo que atrajo primero al hombre salvaje hacia la mujer y hacia el refugio primitivo que compartía con sus hijos.

\par
%\textsuperscript{(931.6)}
\textsuperscript{84:1.3} El matrimonio ni siquiera fue ocasionado por la comprensión consciente de las obligaciones que se derivan de las relaciones sexuales. El hombre primitivo no comprendía la conexión existente entre la satisfacción sexual y el nacimiento posterior de un niño. Antiguamente se creía de manera universal que una virgen podía quedarse embarazada. Los salvajes concibieron muy pronto la idea de que los bebés se originaban en el mundo de los espíritus; se creía que el embarazo era el resultado de la introducción de un espíritu, de un fantasma en evolución, dentro de la mujer. También se creía que tanto la alimentación como el mal de ojo podían dejar embarazada a una virgen o a una mujer no casada, mientras que las creencias posteriores asociaron el comienzo de la vida con el aliento\footnote{\textit{Aliento de vida}: Gn 2:7.} y con la luz del Sol.

\par
%\textsuperscript{(932.1)}
\textsuperscript{84:1.4} Muchos pueblos primitivos asociaban a los fantasmas con el mar, y por eso se imponían grandes restricciones a los baños de las vírgenes; las chicas jóvenes tenían mucho más miedo de bañarse en el mar con la marea alta que mantener relaciones sexuales. Los bebés deformes o prematuros eran considerados como las crías de unos animales que habían encontrado la manera de entrar en el cuerpo de una mujer a consecuencia de un baño imprudente o debido a la actividad malévola de un espíritu. A los salvajes, por supuesto, no les suponía nada estrangular a estos bebés en el momento de nacer.

\par
%\textsuperscript{(932.2)}
\textsuperscript{84:1.5} El primer paso aclaratorio se produjo con la creencia de que las relaciones sexuales abrían el camino al fantasma fecundador para entrar en la mujer. El hombre ha descubierto desde entonces que el padre y la madre contribuyen por igual a los factores hereditarios vivientes que dan comienzo a la progenie. Pero incluso en el siglo veinte, muchos padres se esfuerzan todavía por mantener a sus hijos en una mayor o menor ignorancia sobre el origen de la vida humana.

\par
%\textsuperscript{(932.3)}
\textsuperscript{84:1.6} El hecho de que la función reproductora trae consigo la relación entre madre e hijo aseguró la existencia de una especie de familia simple. El amor materno es instintivo; no tuvo su origen en las costumbres como fue el caso del matrimonio. El amor materno de todos los mamíferos es el don inherente de los espíritus ayudantes de la mente del universo local; la fuerza y la devoción de este amor siempre son directamente proporcionales a la duración de la infancia indefensa de las especies.

\par
%\textsuperscript{(932.4)}
\textsuperscript{84:1.7} La relación entre madre e hijo es natural, fuerte e instintiva, y por eso es una relación que obligó a las mujeres primitivas a someterse a muchas condiciones extrañas y a soportar dificultades indecibles. Este amor materno imperioso es la emoción obstaculizadora que siempre ha colocado a la mujer en una desventaja tan enorme en todas sus luchas con el hombre. A pesar de todo, el instinto materno no es irresistible en la especie humana; puede ser contrarrestado por la ambición, el egoísmo y las convicciones religiosas.

\par
%\textsuperscript{(932.5)}
\textsuperscript{84:1.8} Aunque la asociación entre madre e hijo no es el matrimonio ni el hogar, es el núcleo a partir del cual nacieron los dos. El gran progreso en la evolución del emparejamiento se produjo cuando estas asociaciones temporales duraron lo suficiente como para criar a la descendencia resultante, pues en esto consiste la creación de un hogar.

\par
%\textsuperscript{(932.6)}
\textsuperscript{84:1.9} Sin tener en cuenta los antagonismos de estas parejas primitivas, y a pesar de la falta de firmeza de su asociación, las posibilidades de supervivencia mejoraron enormemente gracias a estas asociaciones entre un varón y una hembra. Un hombre y una mujer que cooperan, incluso aparte de la familia y de los hijos, son muy superiores en casi todos los aspectos a dos hombres o dos mujeres. Este emparejamiento de los sexos incrementó la supervivencia y fue el principio mismo de la sociedad humana. La división del trabajo entre los sexos también contribuyó a la comodidad y a una felicidad creciente.

\section*{2. El matriarcado primitivo}
\par
%\textsuperscript{(932.7)}
\textsuperscript{84:2.1} La hemorragia periódica de la mujer y su pérdida de sangre adicional en el momento del parto pronto hicieron creer que la sangre era la creadora del hijo (e incluso la sede del alma\footnote{\textit{Sangre de la vida}: Lv 15:19; 17:11.}) y dieron origen al concepto de los lazos de sangre en las relaciones humanas. En los tiempos primitivos, toda la descendencia se contaba según el linaje femenino, porque era la única parte de la herencia de la que se estaba totalmente seguro.

\par
%\textsuperscript{(932.8)}
\textsuperscript{84:2.2} La familia primitiva, que nació del vínculo sanguíneo biológico e instintivo entre la madre y el hijo, fue inevitablemente un matriarcado; muchas tribus mantuvieron durante mucho tiempo esta organización. El matriarcado fue la única transición posible entre la etapa del matrimonio colectivo en la horda y la vida hogareña posterior y mejor de las familias patriarcales polígamas y monógamas. El matriarcado era natural y biológico; el patriarcado es social, económico y político. La supervivencia del matriarcado entre los hombres rojos de América del Norte es una de las razones principales por las cuales los iroqueses, por lo demás progresivos, no fundaron nunca un verdadero Estado.

\par
%\textsuperscript{(933.1)}
\textsuperscript{84:2.3} Bajo las costumbres matriarcales, la madre de la esposa gozaba en el hogar de una autoridad prácticamente suprema; incluso los hermanos de la esposa y los hijos de éstos eran más activos que el marido en la supervisión de la familia. A los padres les cambiaban a menudo el nombre por el de uno de sus propios hijos.

\par
%\textsuperscript{(933.2)}
\textsuperscript{84:2.4} Las razas más primitivas daban poco crédito al padre, pues consideraban que el niño provenía enteramente de la madre. Creían que los hijos se parecían al padre a causa de la asociación, o que estaban <<marcados>> de esta manera porque la madre deseaba que tuvieran el aspecto del padre. Más tarde, cuando se efectuó el cambio del matriarcado al patriarcado, el padre se atribuyó todo el mérito del hijo, y muchos tabúes sobre la mujer embarazada se extendieron posteriormente hasta incluir a su marido. Cuando se acercaba el alumbramiento, el futuro padre dejaba de trabajar, y en el momento del parto se acostaba con su mujer, permaneciendo en la cama entre tres y ocho días. La esposa podía levantarse al día siguiente y emprender su duro trabajo, pero el marido continuaba en la cama para recibir las felicitaciones; todo esto formó parte de las costumbres primitivas destinadas a establecer el derecho del padre sobre el hijo.

\par
%\textsuperscript{(933.3)}
\textsuperscript{84:2.5} Al principio, la costumbre exigía que el hombre se fuera a vivir con la familia de su mujer, pero en tiempos posteriores, una vez que el hombre había pagado en dinero o con su trabajo el precio de la novia, podía llevarse a su esposa y a sus hijos con su propia familia. La transición entre el matriarcado y el patriarcado explica las prohibiciones, por lo demás sin sentido, de algunos tipos de matrimonios entre primos, mientras que otros con el mismo parentesco eran aprobados.

\par
%\textsuperscript{(933.4)}
\textsuperscript{84:2.6} Con la desaparición de las costumbres de los cazadores, cuando el pastoreo dio al hombre el control sobre la principal fuente de alimentación, el matriarcado llegó a su fin rápidamente. Simplemente fracasó porque no podía competir con éxito con la nueva familia gobernada por el padre. El poder depositado en los parientes masculinos de la madre no podía competir con el poder concentrado en el marido-padre. La mujer no tenía fuerzas para las tareas combinadas de dar a luz a los hijos y de ejercer una autoridad continua y un poder doméstico cada vez mayor. La aparición del robo de las esposas y la compra posterior de las mujeres aceleraron la desaparición del matriarcado.

\par
%\textsuperscript{(933.5)}
\textsuperscript{84:2.7} El cambio prodigioso del matriarcado al patriarcado es uno de los cambios adaptativos más radicales y completos que haya realizado nunca la raza humana. Este cambio condujo inmediatamente a una expresión social más grande y a una aventura familiar cada vez mayor.

\section*{3. La familia bajo el dominio del padre}
\par
%\textsuperscript{(933.6)}
\textsuperscript{84:3.1} Puede ser que el instinto maternal condujera a la mujer al matrimonio, pero la fuerza superior del hombre, unida a la influencia de las costumbres, fueron las que la obligaron prácticamente a permanecer casada. La vida pastoril tendió a crear un nuevo sistema de costumbres, el tipo patriarcal de vida familiar; y la base de la unidad familiar bajo las costumbres de los pastores y de los agricultores primitivos era la autoridad incuestionable y arbitraria del padre. Toda la sociedad, ya sea nacional o familiar, pasó por la etapa de la autoridad autocrática de tipo patriarcal.

\par
%\textsuperscript{(934.1)}
\textsuperscript{84:3.2} La poca cortesía que se manifestaba a las mujeres durante la era del Antiguo Testamento es un auténtico reflejo de las costumbres de los pastores. Todos los patriarcas hebreos eran pastores, tal como lo demuestra el dicho: <<El Señor es mi pastor>>\footnote{\textit{El Señor es mi pastor}: Sal 23:1.}.

\par
%\textsuperscript{(934.2)}
\textsuperscript{84:3.3} Pero el hombre no era más culpable de la baja opinión que tenía de la mujer, durante las épocas pasadas, que la mujer misma. Ella no logró obtener el reconocimiento social durante los tiempos primitivos porque no actuaba en caso de emergencia; no era una heroína espectacular ni sobresalía en caso de crisis. La maternidad era una clara desventaja en la lucha por la existencia; el amor materno era un impedimento para las mujeres a la hora de defender la tribu.

\par
%\textsuperscript{(934.3)}
\textsuperscript{84:3.4} Las mujeres primitivas también crearon involuntariamente su dependencia del varón mediante la admiración y la alabanza que manifestaban por su belicosidad y virilidad. Esta exaltación del guerrero elevó el ego masculino y disminuyó en igual medida el de la mujer, haciéndola más dependiente; un uniforme militar excita poderosamente todavía las emociones femeninas.

\par
%\textsuperscript{(934.4)}
\textsuperscript{84:3.5} Entre las razas más avanzadas, las mujeres no son tan grandes ni tan fuertes como los hombres. Al ser la más débil, la mujer se volvió por tanto más discreta; pronto aprendió a aprovecharse de sus encantos sexuales. Se volvió más despierta y conservadora que el hombre, aunque ligeramente menos profunda. El hombre era superior a la mujer en el campo de batalla y en la caza; pero en el hogar, la mujer ha superado generalmente incluso al más primitivo de los hombres.

\par
%\textsuperscript{(934.5)}
\textsuperscript{84:3.6} El pastor cuidaba de sus rebaños para poder sustentarse, pero durante todas estas épocas pastoriles, la mujer tuvo que seguir suministrando los alimentos vegetales. El hombre primitivo rehuía el trabajo de la tierra, que era demasiado pacífico, muy poco arriesgado. Había también una antigua superstición que aseguraba que las mujeres podían conseguir mejores plantas; eran madres. En muchas tribus atrasadas de hoy en día, los hombres cocinan la carne y las mujeres las verduras. Cuando las tribus primitivas de Australia se trasladan de un lado a otro, las mujeres no cazan nunca, mientras que un hombre no se agacharía para desenterrar una raíz.

\par
%\textsuperscript{(934.6)}
\textsuperscript{84:3.7} La mujer siempre ha tenido que trabajar; ha sido una verdadera productora, al menos hasta los tiempos modernos. El hombre ha elegido habitualmente el camino más fácil, y esta desigualdad ha existido durante toda la historia de la raza humana. La mujer siempre ha sido la portadora de las cargas; transportaba las propiedades de la familia y se ocupaba de los hijos, dejando así las manos libres al hombre para combatir o cazar.

\par
%\textsuperscript{(934.7)}
\textsuperscript{84:3.8} La primera liberación de la mujer tuvo lugar cuando el hombre consintió en cultivar la tierra, cuando consintió en hacer lo que hasta ese momento se había considerado como un trabajo de la mujer. Se produjo un gran paso hacia adelante cuando los prisioneros masculinos ya no fueron ejecutados, sino que fueron esclavizados como agricultores. Esto provocó la liberación de la mujer, que así pudo dedicar más tiempo a ocuparse de la casa y de la educación de los hijos.

\par
%\textsuperscript{(934.8)}
\textsuperscript{84:3.9} El suministro de leche para los pequeños condujo a un destete más prematuro de los bebés, y por tanto, las madres así liberadas de estos períodos de esterilidad temporal pudieron tener más hijos, mientras que el empleo de la leche de vaca y de cabra redujo considerablemente la mortalidad infantil. Antes de la etapa pastoril de la sociedad, las madres solían amamantar a sus bebés hasta la edad de cuatro o cinco años.

\par
%\textsuperscript{(934.9)}
\textsuperscript{84:3.10} La disminución de las guerras primitivas redujo enormemente la disparidad entre la división del trabajo basada en el sexo. Pero las mujeres tuvieron que seguir haciendo el trabajo real, mientras que los hombres se dedicaban a la tarea de vigilar. Ningún campamento o aldea podía quedarse sin vigilancia ni de día ni de noche, pero incluso esta tarea fue aliviada por la domesticación del perro. La aparición de la agricultura aumentó en general el prestigio y la posición social de la mujer; al menos esto fue así hasta el momento en que el hombre mismo se volvió agricultor. En cuanto el hombre mismo se puso a cultivar la tierra, inmediatamente se produjo un gran progreso en los métodos agrícolas, que se prolongó durante las generaciones sucesivas. El hombre había aprendido el valor de la organización en la caza y en la guerra; estas técnicas las introdujo en la industria y, más tarde, cuando se hizo cargo de una gran parte de las tareas de la mujer, mejoró considerablemente sus métodos de trabajo poco precisos.

\section*{4. La situación de la mujer en la sociedad primitiva}
\par
%\textsuperscript{(935.1)}
\textsuperscript{84:4.1} En términos generales, la situación de la mujer en una época cualquiera constituye un criterio acertado del progreso evolutivo del matrimonio como institución social, mientras que el progreso del matrimonio mismo es un indicador razonablemente preciso de los avances de la civilización humana.

\par
%\textsuperscript{(935.2)}
\textsuperscript{84:4.2} La situación de la mujer ha sido siempre una paradoja social; siempre ha sabido dirigir hábilmente a los hombres; siempre ha sacado partido del impulso sexual más fuerte del hombre a favor de sus propios intereses y de su propio ascenso. Explotando sutilmente sus encantos sexuales, a menudo ha sido capaz de ejercer un poder dominante sobre el hombre, incluso cuando éste la mantenía en una esclavitud abyecta.

\par
%\textsuperscript{(935.3)}
\textsuperscript{84:4.3} La mujer primitiva no era para el hombre una amiga, un dulce amor, una amante y una compañera, sino más bien una parte de su propiedad, una sirvienta o una esclava y, más tarde, una asociada económica, un juguete y una productora de hijos. Sin embargo, las relaciones sexuales adecuadas y satisfactorias han requerido siempre el elemento de la elección y la cooperación de la mujer, y esto siempre ha proporcionado a las mujeres inteligentes una influencia considerable sobre su situación personal e inmediata, sin tener en cuenta su posición social como sexo. Pero el hecho de que las mujeres se vieran constantemente obligadas a recurrir a la astucia en un esfuerzo por aliviar su esclavitud no ayudó a disipar el recelo y la desconfianza del hombre.

\par
%\textsuperscript{(935.4)}
\textsuperscript{84:4.4} Los sexos han tenido grandes dificultades para comprenderse mutuamente. El hombre ha encontrado difícil comprender a la mujer, y la miraba con una extraña mezcla de desconfianza ignorante y de fascinación temerosa, cuando no con recelo y desdén. Muchas tradiciones tribales y raciales relegan todas las dificultades a Eva, Pandora o alguna otra representante del sexo femenino. Estos relatos siempre fueron desvirtuados para dar la impresión de que la mujer había traído el mal sobre el hombre\footnote{\textit{Culpabilidad sobre la mujer}: Gn 3:12-17.}; y todo esto indica que la desconfianza hacia la mujer fue en otro tiempo universal. Entre las razones que se alegaban a favor del celibato de los sacerdotes, la principal era la bajeza de la mujer. El hecho de que la mayoría de las supuestas brujas fueran mujeres no mejoró la antigua reputación de este sexo.

\par
%\textsuperscript{(935.5)}
\textsuperscript{84:4.5} Los hombres han considerado durante mucho tiempo a las mujeres como extrañas, e incluso anormales. Han creído incluso que las mujeres no tenían alma, y por esta razón no les ponían un nombre. Durante los tiempos primitivos existía un gran temor a la primera relación sexual con una mujer; por eso se estableció la costumbre de que un sacerdote tuviera el primer contacto sexual con una virgen. Se pensaba que incluso la sombra de una mujer era peligrosa.

\par
%\textsuperscript{(935.6)}
\textsuperscript{84:4.6} En otros tiempos se consideraba generalmente que la maternidad volvía peligrosa e impura a una mujer\footnote{\textit{``Impureza'' de la mujer puérpera}: Lv 12:2-8; Lc 2:22-24.}. Muchas costumbres tribales decretaron que la madre debía pasar por largas ceremonias de purificación después del nacimiento de un hijo. Excepto en aquellos grupos donde el hombre participaba en el parto, la futura madre era rechazada, la dejaban sola. Los antiguos evitaban incluso que el niño naciera dentro de la casa. Finalmente se permitió que las mujeres de edad asistieran a la madre durante el parto, y esta práctica dio origen a la profesión de comadrona. Durante el parto se decían y se hacían decenas de tonterías para facilitar el alumbramiento. Tenían la costumbre de rociar al recién nacido con agua bendita para impedir la injerencia de los fantasmas.

\par
%\textsuperscript{(935.7)}
\textsuperscript{84:4.7} El parto era relativamente fácil entre las tribus de sangre pura, necesitándose sólo dos o tres horas; es raro que sea tan fácil entre las razas mezcladas. Si una mujer moría de parto, especialmente durante el alumbramiento de gemelos, se creía que había sido culpable de adulterio con un espíritu. Posteriormente, las tribus más evolucionadas consideraron la muerte durante el parto como la voluntad del cielo; se estimaba que estas madres habían perecido por una noble causa.

\par
%\textsuperscript{(936.1)}
\textsuperscript{84:4.8} La supuesta modestia de las mujeres con respecto a la ropa y a mostrar su persona nació del miedo mortal a ser observadas durante el período menstrual. Dejarse ver en este estado era un grave pecado, la violación de un tabú. Bajo las costumbres de los tiempos antiguos, toda mujer, desde la adolescencia hasta la menopausia, estaba sometida a una cuarentena\footnote{\textit{Cuarentena}: Lv 15:19-20.} familiar y social completa durante una semana entera cada mes. Todas las cosas que pudiera tocar, o sobre las que se había sentado a acostado, estaban <<manchadas>>\footnote{\textit{Contaminación por contacto de mujer menstruante}: Lv 15:21-22.}. Durante mucho tiempo se tuvo la costumbre de golpear brutalmente a las muchachas después de cada período menstrual, para intentar expulsar de su cuerpo al espíritu maligno. Pero cuando una mujer pasaba la menopausia, la trataban generalmente con más consideración, concediéndole más derechos y privilegios. En vista de todo esto, no es de extrañar que las mujeres fueran contempladas con desprecio. Incluso los griegos consideraban que la mujer con la menstruación era una de las tres grandes causas de contaminación, siendo las otras dos la carne de cerdo y el ajo.

\par
%\textsuperscript{(936.2)}
\textsuperscript{84:4.9} Por muy descabelladas que fueran estas ideas antiguas, hicieron algún bien, puesto que concedieron a las mujeres sobrecargadas de trabajo, al menos durante su juventud, una semana cada mes para dedicarla a un bienvenido descanso y a meditaciones provechosas. Así pudieron aguzar su ingenio para tratar con sus compañeros masculinos el resto del tiempo. Esta cuarentena de las mujeres también protegió a los hombres contra los excesos sexuales, contribuyendo indirectamente de este modo a restringir la población y a aumentar el dominio de sí mismo.

\par
%\textsuperscript{(936.3)}
\textsuperscript{84:4.10} Un gran progreso tuvo lugar cuando se le negó al hombre el derecho de matar a su mujer a voluntad. También se realizó un paso hacia adelante cuando la mujer tuvo el derecho de poseer sus regalos de boda. Más tarde consiguió el derecho legal de poseer, controlar e incluso disponer de sus propiedades, pero estuvo mucho tiempo privada del derecho a ocupar un puesto en la iglesia o el Estado. La mujer siempre ha sido tratada más o menos como una propiedad hasta el siglo veinte después de Cristo, y durante este mismo siglo. Todavía no ha conseguido liberarse, a nivel mundial, de la exclusión impuesta por el control del hombre. Incluso entre los pueblos avanzados, el intento del hombre por proteger a la mujer ha sido siempre una afirmación tácita de superioridad.

\par
%\textsuperscript{(936.4)}
\textsuperscript{84:4.11} Pero las mujeres primitivas no se compadecían de sí mismas, como sus hermanas más recientemente liberadas acostumbran a hacer. Después de todo, se sentían realmente felices y satisfechas; no se atrevían a imaginar una forma de existencia diferente o mejor.

\section*{5. La mujer bajo las costumbres en evolución}
\par
%\textsuperscript{(936.5)}
\textsuperscript{84:5.1} En la perpetuación de sí mismo, la mujer está en un plano de igualdad con el hombre, pero en la asociación para sustentarse, trabaja con una clara desventaja, y este obstáculo de la maternidad forzada sólo puede ser compensado por las costumbres iluminadas de una civilización progresiva, y por la adquisición de un sentido creciente de la equidad por parte del hombre.

\par
%\textsuperscript{(936.6)}
\textsuperscript{84:5.2} A medida que evolucionó la sociedad, los criterios sexuales de las mujeres se elevaron más porque también sufrían más las consecuencias de la transgresión de las costumbres sexuales. Los criterios sexuales del hombre sólo están mejorando tardíamente a consecuencia del puro sentido de esa equidad que exige la civilización. La naturaleza no sabe nada de equidad ---hace que la mujer sufra sola los dolores del parto.

\par
%\textsuperscript{(936.7)}
\textsuperscript{84:5.3} La idea moderna de la igualdad de los sexos es hermosa, y digna de una civilización en expansión, pero no se encuentra en la naturaleza. Cuando la fuerza es el derecho, el hombre domina a la mujer; cuando la justicia, la paz y la equidad prevalecen más, la mujer emerge gradualmente de la esclavitud y la oscuridad. La posición social de la mujer ha variado generalmente de manera inversa al grado de militarismo existente en cualquier época o nación.

\par
%\textsuperscript{(937.1)}
\textsuperscript{84:5.4} Pero el hombre no se apoderó de forma consciente e intencional de los derechos de la mujer, para luego devolvérselos gradualmente a regañadientes; todo esto fue un episodio inconsciente e imprevisto de la evolución social. Cuando llegó realmente el momento en que la mujer tenía que disfrutar de unos derechos adicionales, los obtuvo, y sin tener en cuenta para nada la actitud consciente del hombre. Las costumbres cambian de manera lenta pero segura para proporcionar los ajustes sociales que forman parte de la evolución continua de la civilización. Las costumbres progresivas proporcionaron lentamente un trato cada vez mejor a las mujeres; las tribus que continuaron tratándolas con crueldad no sobrevivieron.

\par
%\textsuperscript{(937.2)}
\textsuperscript{84:5.5} Los adamitas y los noditas concedieron a las mujeres un reconocimiento cada vez mayor, y los grupos que fueron influidos por los anditas migratorios tendieron a adoptar las enseñanzas edénicas relacionadas con el lugar de las mujeres en la sociedad.

\par
%\textsuperscript{(937.3)}
\textsuperscript{84:5.6} Los antiguos chinos y los griegos trataron a las mujeres mejor que la mayoría de los pueblos circundantes. Pero los hebreos desconfiaban extremadamente de ellas. En occidente, la mujer ha tenido un ascenso difícil debido a las doctrinas paulinas que se enlazaron con el cristianismo, aunque el cristianismo hizo progresar las costumbres imponiendo a los hombres unas obligaciones sexuales más rigurosas. El estado de la mujer es poco menos que desesperado ante la degradación especial que sufre en el mahometismo, y le va aún peor bajo las enseñanzas de otras diversas religiones orientales.

\par
%\textsuperscript{(937.4)}
\textsuperscript{84:5.7} La ciencia, y no la religión, ha emancipado realmente a la mujer; la fábrica moderna es la que la ha liberado principalmente de los límites del hogar. Las aptitudes físicas del hombre ya no son un elemento esencial en el nuevo mecanismo para sustentarse; la ciencia ha cambiado tanto las condiciones de vida que la fuerza masculina ya no es tan superior a la fuerza femenina.

\par
%\textsuperscript{(937.5)}
\textsuperscript{84:5.8} Estos cambios han tendido a liberar a la mujer de la esclavitud doméstica, y han producido tal modificación en su situación, que actualmente disfruta de un grado de libertad personal y de decisión sexual que son prácticamente iguales a las del hombre. En otro tiempo, el valor de una mujer consistía en su capacidad para producir alimentos, pero los inventos y la prosperidad le han permitido crear un nuevo mundo en el cual actuar ---el ámbito de la gracia y el encanto. La industria ha ganado así su batalla inconsciente y no intencional para la emancipación social y económica de la mujer. La evolución ha logrado hacer una vez más lo que ni siquiera la revelación pudo realizar.

\par
%\textsuperscript{(937.6)}
\textsuperscript{84:5.9} La reacción de los pueblos progresistas ante las costumbres injustas que gobernaban la posición de la mujer en la sociedad ha oscilado en verdad de un extremo a otro como un péndulo. Entre las razas industrializadas, la mujer ha recibido casi todos los derechos y disfruta de la exención de numerosas obligaciones, tales como el servicio militar. Cada disminución de la lucha por la existencia ha contribuido a liberar a la mujer, y ésta se ha beneficiado directamente de todos los progresos hacia la monogamia. Los más débiles siempre obtienen unos beneficios desproporcionados en cada ajuste de las costumbres en la evolución progresiva de la sociedad.

\par
%\textsuperscript{(937.7)}
\textsuperscript{84:5.10} En cuanto a los ideales del matrimonio en pareja, la mujer ha conseguido finalmente reconocimiento, dignidad, independencia, igualdad y educación; pero, ¿se mostrará merecedora de todos estos logros nuevos y sin precedentes? ¿Responderá la mujer moderna a esta gran liberación social con la pereza, la indiferencia, la esterilidad y la infidelidad? ¡Hoy, en el siglo veinte, la mujer está pasando por la prueba decisiva de su larga existencia en el mundo!

\par
%\textsuperscript{(938.1)}
\textsuperscript{84:5.11} La mujer participa en un plano de igualdad con el hombre en la reproducción de la raza, por lo que es tan importante como él en el desarrollo de la evolución racial; por esta razón la evolución ha trabajado cada vez más por hacer realidad los derechos de la mujer. Pero los derechos de la mujer no son de ninguna manera los derechos del hombre. La mujer no puede progresar a costa de los derechos del hombre, como el hombre tampoco puede prosperar a expensas de los derechos de la mujer.

\par
%\textsuperscript{(938.2)}
\textsuperscript{84:5.12} Cada sexo tiene su propia esfera de existencia particular, con sus propios derechos dentro de dicha esfera. Si la mujer aspira a disfrutar literalmente de todos los derechos del hombre, entonces una competencia despiadada y desprovista de sentimientos reemplazará con seguridad, tarde o temprano, esa caballerosidad y esa consideración especial que muchas mujeres disfrutan en la actualidad, y que han conseguido tan recientemente de los hombres.

\par
%\textsuperscript{(938.3)}
\textsuperscript{84:5.13} La civilización nunca podrá eliminar el abismo que existe entre la conducta de los dos sexos. Las costumbres cambian de una época a la siguiente, pero el instinto jamás. El amor materno innato nunca permitirá a la mujer emancipada rivalizar seriamente con el hombre en la industria. Cada sexo permanecerá siempre supremo en su propio ámbito, un ámbito determinado por la diferenciación biológica y la disparidad mental.

\par
%\textsuperscript{(938.4)}
\textsuperscript{84:5.14} Cada sexo tendrá siempre su propia esfera especial, aunque de vez en cuando se superpongan. Los hombres y las mujeres sólo competirán en términos de igualdad en el terreno social.

\section*{6. La asociación del hombre y la mujer}
\par
%\textsuperscript{(938.5)}
\textsuperscript{84:6.1} El impulso reproductor reúne infaliblemente a los hombres y las mujeres para perpetuarse, pero, por sí solo, no asegura que permanecerán juntos en una cooperación mutua ---para la fundación de un hogar.

\par
%\textsuperscript{(938.6)}
\textsuperscript{84:6.2} Toda institución humana coronada de éxito contiene unos antagonismos de intereses personales que han sido ajustados para conseguir una armonía práctica de trabajo, y la creación del hogar no es una excepción. El matrimonio, la base para formar un hogar, es la manifestación más elevada de esa cooperación antagonista que caracteriza con tanta frecuencia los contactos entre la naturaleza y la sociedad. El conflicto es inevitable. El emparejamiento es inherente, es natural. El matrimonio sin embargo no es biológico, es sociológico. La pasión asegura que el hombre y la mujer se reunirán, pero el instinto parental más débil y las costumbres sociales son las que los mantienen unidos.

\par
%\textsuperscript{(938.7)}
\textsuperscript{84:6.3} Considerados en la práctica, el hombre y la mujer son dos variedades distintas de la misma especie, que viven en una asociación íntima y estrecha. Sus puntos de vista y todas sus reacciones ante la vida son esencialmente diferentes; son totalmente incapaces de comprenderse plena y realmente el uno al otro. La comprensión completa entre los sexos es imposible de alcanzar.

\par
%\textsuperscript{(938.8)}
\textsuperscript{84:6.4} Las mujeres parecen tener más intuición que los hombres, pero también parecen ser un poco menos lógicas. Sin embargo, la mujer ha sido siempre la abanderada moral y la dirigente espiritual de la humanidad. La mano que mece la cuna fraterniza todavía con el destino.

\par
%\textsuperscript{(938.9)}
\textsuperscript{84:6.5} Las diferencias de naturaleza, reacción, puntos de vista y pensamientos entre los hombres y las mujeres, en lugar de producir inquietud, deberían ser consideradas como altamente beneficiosas para la humanidad, tanto individual como colectivamente. Muchas órdenes de criaturas del universo son creadas en fases duales de manifestación de la personalidad. Entre los mortales, los Hijos Materiales y los midsonitarios, esta diferencia se describe como masculina y femenina; entre los serafines, los querubines y los Compañeros Morontiales, ha sido denominada positiva o dinámica, y negativa o reservada. Estas asociaciones duales multiplican enormemente la diversidad de talentos y vencen las limitaciones inherentes, tal como lo hacen ciertas asociaciones trinas en el sistema Paraíso-Havona.

\par
%\textsuperscript{(939.1)}
\textsuperscript{84:6.6} Los hombres y las mujeres se necesitan mutuamente en sus carreras morontiales y espirituales tanto como en sus carreras como mortales. Las diferencias de puntos de vista entre el varón y la hembra subsisten incluso más allá de la primera vida y a lo largo de toda la ascensión del universo local y del superuniverso. Incluso en Havona, los peregrinos que en otro tiempo fueron hombres y mujeres continuarán ayudándose unos a otros en el ascenso al Paraíso. Hasta en el Cuerpo de la Finalidad, la metamorfosis de la criatura nunca será tan grande como para borrar las tendencias de la personalidad que los humanos llaman masculinas y femeninas; estas dos variantes fundamentales de la humanidad siempre continuarán intrigándose, estimulándose, alentándose y ayudándose una a la otra; siempre dependerán mutuamente de su cooperación para resolver los complicados problemas del universo y para superar las numerosas dificultades cósmicas.

\par
%\textsuperscript{(939.2)}
\textsuperscript{84:6.7} Aunque los sexos nunca pueden esperar comprenderse plenamente el uno al otro, son efectivamente complementarios, y aunque su cooperación sea a menudo más o menos antagonista en el plano personal, es capaz de mantener y reproducir la sociedad. El matrimonio es una institución destinada a ajustar las diferencias sexuales, llevando a cabo al mismo tiempo la continuación de la civilización y asegurando la reproducción de la raza.

\par
%\textsuperscript{(939.3)}
\textsuperscript{84:6.8} El matrimonio es la madre de todas las instituciones humanas, pues conduce directamente a la fundación y al mantenimiento del hogar, que es la base estructural de la sociedad. La familia está unida vitalmente al mecanismo de la preservación de sí mismo; constituye la única esperanza de perpetuar la raza bajo las costumbres de la civilización, mientras que al mismo tiempo proporciona de manera muy eficaz ciertas formas altamente satisfactorias de placer personal. La familia es la realización puramente humana más importante del hombre, pues combina, tal como lo hace, la evolución de las relaciones biológicas entre el varón y la hembra con las relaciones sociales entre el marido y la mujer.

\section*{7. Los ideales de la vida familiar}
\par
%\textsuperscript{(939.4)}
\textsuperscript{84:7.1} La unión sexual es instintiva, los hijos son el resultado natural, y la familia nace así de manera automática. Según sean las familias de una raza o nación, así será su sociedad. Si las familias son buenas, la sociedad será igualmente buena. La gran estabilidad cultural de los pueblos judío y chino reside en la fuerza de sus grupos familiares.

\par
%\textsuperscript{(939.5)}
\textsuperscript{84:7.2} El instinto femenino de amar y cuidar a los hijos se confabuló para hacer de la mujer la parte interesada en promover el matrimonio y la vida familiar primitiva. Sólo la presión de las costumbres y las convenciones sociales posteriores obligaron al hombre a formar el hogar; fue lento en interesarse por el establecimiento del matrimonio y el hogar porque el acto sexual no conlleva ninguna consecuencia biológica para él.

\par
%\textsuperscript{(939.6)}
\textsuperscript{84:7.3} La asociación sexual es natural, pero el matrimonio es social y siempre ha estado reglamentado por las costumbres. Las costumbres (religiosas, morales y éticas), así como la propiedad, el orgullo y la caballerosidad, estabilizan las instituciones del matrimonio y la familia. Cada vez que fluctúan las costumbres se produce una oscilación en la estabilidad de la institución hogar-matrimonio. El matrimonio está saliendo ahora de la etapa de la propiedad para entrar en la era de lo personal. Antiguamente, el hombre protegía a la mujer porque era su pertenencia, y ella obedecía por la misma razón. Independientemente de sus méritos, este sistema proporcionaba estabilidad. Ahora, la mujer ya no es considerada como una propiedad, y están surgiendo nuevas costumbres destinadas a estabilizar la institución matrimonio-hogar:

\par
%\textsuperscript{(939.7)}
\textsuperscript{84:7.4} 1. El nuevo papel de la religión ---la enseñanza de que la experiencia parental es esencial, la idea de procrear ciudadanos cósmicos, la comprensión más amplia del privilegio de la procreación ---dar hijos al Padre.

\par
%\textsuperscript{(940.1)}
\textsuperscript{84:7.5} 2. El nuevo papel de la ciencia ---la procreación se está volviendo cada vez más voluntaria, sometida al control del hombre. En los tiempos antiguos, la falta de conocimientos aseguraba la aparición de los hijos en ausencia de todo deseo de tenerlos.

\par
%\textsuperscript{(940.2)}
\textsuperscript{84:7.6} 3. La nueva función del aliciente del placer ---esto introduce un nuevo factor en la supervivencia racial; los antiguos dejaban morir a los hijos no deseados; los modernos se niegan a traerlos al mundo.

\par
%\textsuperscript{(940.3)}
\textsuperscript{84:7.7} 4. La mejora del instinto parental. Cada generación tiende ahora a eliminar de la corriente reproductora de la raza a aquellos individuos cuyo instinto parental no es lo suficientemente fuerte como para asegurar la procreación de hijos, los futuros padres de la siguiente generación.

\par
%\textsuperscript{(940.4)}
\textsuperscript{84:7.8} Pero el hogar como institución, la asociación entre un solo hombre y una sola mujer, data más específicamente de los tiempos de Dalamatia, hace aproximadamente medio millón de años, ya que las costumbres monógamas de Andón y sus descendientes inmediatos habían sido abandonadas mucho tiempo antes. Sin embargo, la vida familiar no era muy digna de alabanza antes de la época de los noditas y de los adamitas que llegaron después. Adán y Eva ejercieron una influencia duradera sobre toda la humanidad; por primera vez en la historia del mundo se pudo observar a los hombres y las mujeres trabajando juntos en el Jardín. El ideal edénico, toda la familia trabajando como horticultores, era una idea nueva en Urantia.

\par
%\textsuperscript{(940.5)}
\textsuperscript{84:7.9} La familia primitiva englobaba a un grupo relacionado por el trabajo, que incluía a los esclavos, y todos vivían en una sola vivienda. El matrimonio y la vida familiar no siempre han sido la misma cosa, pero han estado necesariamente muy asociados. La mujer siempre ha deseado una familia individual, y al final se salió con la suya.

\par
%\textsuperscript{(940.6)}
\textsuperscript{84:7.10} El amor a los hijos es casi universal y tiene un claro valor de supervivencia. Los antiguos sacrificaban siempre los intereses de la madre a favor del bienestar del hijo; las madres esquimales lamen todavía a sus bebés en lugar de lavarlos. Pero las madres primitivas sólo alimentaban y cuidaban a sus hijos mientras eran muy pequeños; al igual que hacen los animales, en cuanto crecían se desentendían de ellos. Las asociaciones humanas duraderas y continuas nunca han estado basadas en el solo afecto biológico. Los animales aman a sus crías; el hombre ---el hombre civilizado--- ama a los hijos de sus hijos\footnote{\textit{Amor por los nietos}: Pr 17:6.}. Cuanto más elevada es una civilización, mayor es la alegría de los padres ante el progreso y el éxito de sus hijos; así es como surge una conciencia nueva y superior del orgullo del \textit{apellido}.

\par
%\textsuperscript{(940.7)}
\textsuperscript{84:7.11} Entre los pueblos antiguos, las familias grandes no eran necesariamente el resultado del afecto. Se deseaban muchos hijos porque:

\par
%\textsuperscript{(940.8)}
\textsuperscript{84:7.12} 1. Eran valiosos como trabajadores.

\par
%\textsuperscript{(940.9)}
\textsuperscript{84:7.13} 2. Eran un seguro para la vejez.

\par
%\textsuperscript{(940.10)}
\textsuperscript{84:7.14} 3. Las hijas se podían vender.

\par
%\textsuperscript{(940.11)}
\textsuperscript{84:7.15} 4. El orgullo familiar exigía la extensión del apellido.

\par
%\textsuperscript{(940.12)}
\textsuperscript{84:7.16} 5. Los hijos proporcionaban protección y defensa.

\par
%\textsuperscript{(940.13)}
\textsuperscript{84:7.17} 6. El miedo a los fantasmas engendró el temor a la soledad.

\par
%\textsuperscript{(940.14)}
\textsuperscript{84:7.18} 7. Algunas religiones exigían una descendencia.

\par
%\textsuperscript{(940.15)}
\textsuperscript{84:7.19} Los practicantes del culto a los antepasados consideran el no tener hijos como la calamidad suprema de todos los tiempos y de la eternidad. Desean por encima de todo tener hijos para que oficien en los festines post mortem, para que ofrezcan los sacrificios necesarios para el progreso del fantasma a través del mundo del espíritu.

\par
%\textsuperscript{(941.1)}
\textsuperscript{84:7.20} Los antiguos salvajes empezaban muy pronto a disciplinar a sus hijos; los niños no tardaban en comprender que la desobediencia significaba el fracaso o incluso la muerte, exactamente igual que para los animales. La civilización protege al niño contra las consecuencias naturales de una conducta insensata, y esto es lo que contribuye tanto a la insubordinación moderna.

\par
%\textsuperscript{(941.2)}
\textsuperscript{84:7.21} Los niños esquimales se desarrollan con tan poca necesidad de disciplina y corrección simplemente porque son por naturaleza unos pequeños animales dóciles; tanto los hijos de los hombres rojos como los de los amarillos son casi igual de manejables. Pero en las razas que contienen la herencia andita, los niños no son tan apacibles; estos jóvenes más imaginativos y aventureros necesitan más educación y disciplina. Los problemas modernos de la educación de los niños se han vuelto cada vez más difíciles debido a:

\par
%\textsuperscript{(941.3)}
\textsuperscript{84:7.22} 1. El alto grado de las mezclas raciales.

\par
%\textsuperscript{(941.4)}
\textsuperscript{84:7.23} 2. La educación artificial y superficial.

\par
%\textsuperscript{(941.5)}
\textsuperscript{84:7.24} 3. La incapacidad de los niños para cultivarse imitando a sus padres ---éstos están ausentes de la escena familiar una gran parte del tiempo.

\par
%\textsuperscript{(941.6)}
\textsuperscript{84:7.25} Las antiguas ideas sobre la disciplina familiar eran biológicas y tenían su origen en la comprensión de que los padres eran los creadores del ser del hijo. Los ideales progresivos de la vida familiar conducen al concepto de que traer un hijo al mundo, en lugar de conferir ciertos derechos a los padres, implica la responsabilidad suprema de la existencia humana.

\par
%\textsuperscript{(941.7)}
\textsuperscript{84:7.26} La civilización considera que los padres asumen todos los deberes, y que el hijo tiene todos los derechos. El respeto del hijo por sus padres no surge del conocimiento de la obligación implícita que conlleva la procreación parental, sino que crece de manera natural a consecuencia de los cuidados, la educación y el afecto que manifiestan con amor ayudando al hijo a ganar la batalla de la vida. Los padres auténticos están dedicados a un continuo ministerio de servicio que el hijo juicioso termina por reconocer y apreciar.

\par
%\textsuperscript{(941.8)}
\textsuperscript{84:7.27} En la era industrial y urbana actual, la institución del matrimonio está evolucionando por unas vías económicas nuevas. La vida familiar se ha vuelto cada vez más costosa, mientras que los hijos, que solían ser un activo, se han convertido en un pasivo económico. Pero la seguridad de la civilización misma depende todavía de la buena voluntad creciente de cada generación en invertir en el bienestar de la próxima generación y de las siguientes. Cualquier intento por transferir la responsabilidad parental al Estado o la iglesia resultará suicida para el bienestar y el progreso de la civilización.

\par
%\textsuperscript{(941.9)}
\textsuperscript{84:7.28} El matrimonio, con los hijos y la vida familiar consiguiente, estimula los potenciales más elevados de la naturaleza humana, y proporciona simultáneamente el canal ideal para expresar los atributos avivados de la personalidad mortal. La familia asegura la perpetuación biológica de la especie humana. El hogar es el marco social natural donde los hijos que crecen pueden captar la ética de la fraternidad de la sangre. La familia es la unidad fundamental de fraternidad donde los padres y los hijos aprenden las lecciones de paciencia, altruismo, tolerancia e indulgencia que son tan esenciales para realizar la fraternidad entre todos los hombres.

\par
%\textsuperscript{(941.10)}
\textsuperscript{84:7.29} La sociedad humana mejoraría enormemente si las razas civilizadas volvieran de manera más general a las costumbres de los consejos de familia de los anditas. Éstos no mantenían la forma patriarcal o autocrática de gobierno familiar. Eran muy fraternales y asociativos, discutiendo con franqueza y libertad todas las propuestas y reglamentaciones de naturaleza familiar. Eran idealmente fraternales en todos sus gobiernos de familia. En una familia ideal, tanto el afecto filial como el amor de los padres aumentan a través de la devoción fraternal.

\par
%\textsuperscript{(942.1)}
\textsuperscript{84:7.30} La vida familiar\footnote{\textit{Crianza de los hijos}: Pr 22:6.} es el progenitor de la verdadera moralidad, el antepasado de la conciencia de la lealtad al deber. Las asociaciones forzosas de la vida familiar estabilizan la personalidad y estimulan su crecimiento mediante la obligación de amoldarse necesariamente a otras personalidades diferentes. Pero hay aún más: una verdadera familia ---una buena familia--- revela a los padres procreadores la actitud del Creador hacia sus hijos, mientras que al mismo tiempo estos auténticos padres representan para sus hijos la primera de una larga serie de revelaciones progresivas acerca del amor del Padre Paradisiaco de todos los hijos del universo.

\section*{8. Los peligros de la satisfacción de sí mismo}
\par
%\textsuperscript{(942.2)}
\textsuperscript{84:8.1} El gran peligro que acecha a la vida familiar reside en la amenazadora marea creciente de la satisfacción de sí mismo, en la manía moderna del placer. El aliciente principal que llevaba al matrimonio solía ser el económico; la atracción sexual era secundaria. El matrimonio, basado en la preservación de sí mismo, conducía a la perpetuación de sí mismo y proporcionaba al mismo tiempo una de las formas más deseables de satisfacción de sí mismo. Es la única institución de la sociedad humana que abarca los tres grandes alicientes de la vida.

\par
%\textsuperscript{(942.3)}
\textsuperscript{84:8.2} En un principio, la propiedad era la institución fundamental para sustentarse, mientras que el matrimonio funcionaba como la única institución para perpetuarse. Aunque la satisfacción de las necesidades alimenticias, las diversiones y el humor, junto con la gratificación sexual periódica, eran medios de satisfacerse, sigue siendo un hecho que las costumbres en evolución no han logrado crear una institución bien determinada para la satisfacción de sí mismo. Debido a este fracaso en desarrollar unas técnicas especializadas para los placeres agradables, todas las instituciones humanas están completamente impregnadas de esta búsqueda del placer. La acumulación de los bienes se está convirtiendo en un instrumento para aumentar todas las formas de satisfacción de sí mismo, mientras que el matrimonio a menudo se considera únicamente como un medio de placer. Esta indulgencia excesiva, esta manía tan extendida del placer, constituye en la actualidad la amenaza más grande que se haya dirigido jamás contra la institución social evolutiva de la vida familiar: el hogar.

\par
%\textsuperscript{(942.4)}
\textsuperscript{84:8.3} La raza violeta introdujo en la experiencia de la humanidad una característica nueva y aún no realizada por completo ---el instinto de la diversión unido al sentido del humor. Este instinto existía en cierta medida en los sangiks y los andonitas, pero la estirpe adámica elevó esta tendencia primitiva hasta el nivel de un \textit{potencial de placer}, una forma nueva y glorificada de satisfacción de sí mismo. Aparte del aplacamiento del hambre, el tipo básico de satisfacción de sí mismo es la gratificación sexual, y esta forma de placer sensual fue acrecentada enormemente por la mezcla de los sangiks y los anditas.

\par
%\textsuperscript{(942.5)}
\textsuperscript{84:8.4} La combinación de la impaciencia, la curiosidad, la aventura y el abandono a los placeres, característica de las razas posteriores a los anditas, comporta un verdadero peligro. Los placeres físicos no pueden satisfacer el hambre del alma; la búsqueda insensata del placer no aumenta el amor por el hogar y los hijos. Aunque agotéis los recursos del arte, el color, el sonido, el ritmo, la música y el adorno personal, no podéis esperar de ese modo elevar el alma o alimentar el espíritu. La vanidad y la moda no pueden ayudar a establecer el hogar ni a educar a los hijos; el orgullo y la rivalidad son impotentes para realzar las cualidades de supervivencia de las generaciones venideras.

\par
%\textsuperscript{(942.6)}
\textsuperscript{84:8.5} Todos los seres celestiales que progresan disfrutan del descanso y del ministerio de los directores de la reversión. Todos los esfuerzos por conseguir una diversión sana y por dedicarse a un entretenimiento que eleve son acertados; el sueño reparador, el descanso, el esparcimiento y todos los pasatiempos que impiden el aburrimiento de la monotonía valen la pena. Los juegos competitivos, la narración de historias e incluso la afición a la buena comida pueden servir como formas de satisfacerse. (Cuando empleáis la sal para dar sabor a los alimentos, deteneos a pensar que durante cerca de un millón de años, el hombre sólo podía obtener la sal metiendo sus alimentos en las cenizas.)

\par
%\textsuperscript{(943.1)}
\textsuperscript{84:8.6} Que los hombres disfruten de la vida; que la raza humana encuentre placer de mil y una maneras; que la humanidad evolutiva explore todas las formas de satisfacciones legítimas, los frutos de su larga lucha biológica por elevarse. El hombre se ha ganado bien algunas de sus alegrías y placeres de hoy. ¡Pero mirad bien por la meta del destino! Los placeres son realmente suicidas si consiguen destruir la propiedad, que se ha convertido en la institución para la preservación de sí mismo; y la satisfacción de sí mismo habrá costado en verdad un precio funesto si ocasiona el derrumbamiento del matrimonio, la decadencia de la vida familiar y la destrucción del hogar ---la adquisición evolutiva suprema del hombre y la única esperanza de supervivencia de la civilización.

\par
%\textsuperscript{(943.2)}
\textsuperscript{84:8.7} [Presentado por el Jefe de Serafines estacionado en Urantia.]


\chapter{Documento 85. Los orígenes de la adoración}
\par
%\textsuperscript{(944.1)}
\textsuperscript{85:0.1} LA RELIGIÓN primitiva tuvo un origen biológico, un desarrollo evolutivo natural, al margen de las asociaciones morales y aparte de toda influencia espiritual. Los animales superiores tienen miedos, pero no ilusiones, y en consecuencia ninguna religión. El hombre crea sus religiones primitivas de sus miedos y por medio de sus ilusiones.

\par
%\textsuperscript{(944.2)}
\textsuperscript{85:0.2} En la evolución de la especie humana, las manifestaciones primitivas de la adoración aparecen mucho antes de que la mente del hombre sea capaz de formular los conceptos más complejos sobre la vida presente y en el más allá que merezcan el nombre de religión. La naturaleza de la religión primitiva era completamente intelectual y estaba basada íntegramente en circunstancias asociativas. Los objetos de adoración eran totalmente evocadores; consistían en las cosas de la naturaleza que estaban al alcance de la mano, o que tenían mucha importancia en la experiencia corriente de los urantianos primitivos y sencillos.

\par
%\textsuperscript{(944.3)}
\textsuperscript{85:0.3} Una vez que la religión evolucionó más allá de la adoración de la naturaleza, adquirió raíces de origen espiritual, pero sin embargo siempre estuvo condicionada por el entorno social. A medida que se desarrolló la adoración de la naturaleza, el hombre imaginó la idea de una división del trabajo en el mundo supermortal; había espíritus de la naturaleza para los lagos, los árboles, las cascadas, la lluvia y centenares de otros fenómenos terrestres corrientes.

\par
%\textsuperscript{(944.4)}
\textsuperscript{85:0.4} El hombre mortal ha adorado, en uno u otro momento, todo lo que se encuentra sobre la faz de la Tierra, incluyéndose a sí mismo. También ha adorado todo lo que podía imaginar que se encontraba en el cielo y bajo la superficie de la Tierra. El hombre primitivo temía todas las manifestaciones de poder; adoraba todos los fenómenos naturales que no podía comprender. La observación de las poderosas fuerzas de la naturaleza tales como las tormentas, las inundaciones, los terremotos, los corrimientos de tierras, los volcanes, el fuego, el calor y el frío, causaban una enorme impresión en la mente humana en expansión\footnote{\textit{Adoración de las obras de Dios}: Sab 13:2.}. Las cosas inexplicables de la vida todavía reciben el nombre de <<actos de Dios>> y de <<dispensaciones misteriosas de la Providencia>>.

\section*{1. La adoración de las piedras y las colinas}
\par
%\textsuperscript{(944.5)}
\textsuperscript{85:1.1} El primer objeto que adoró el hombre en evolución fue una piedra. En la actualidad, el pueblo kateri del sur de la India adora todavía una piedra, tal como lo hacen numerosas tribus del norte de la India. Jacob durmió sobre una piedra porque la veneraba; incluso llegó a ungirla\footnote{\textit{Piedra de Jacob}: Gn 28:18.}. Raquel escondía numerosas piedras sagradas en su tienda\footnote{\textit{Piedras de Raquel}: Gn 31:19,30-35.}.

\par
%\textsuperscript{(944.6)}
\textsuperscript{85:1.2} Las piedras impresionaron primero al hombre primitivo como si fueran objetos extraordinarios debido a la manera en que aparecían tan repentinamente en la superficie de un campo cultivado o de una pradera. Los hombres no tenían en cuenta ni la erosión ni los resultados de remover la tierra. Las piedras también impresionaban profundamente a los pueblos primitivos a causa de su frecuente parecido con los animales. La atención del hombre civilizado se detiene ante las numerosas formaciones rocosas de las montañas que tanto se parecen a las facciones de los animales e incluso de los hombres. Pero las piedras meteóricas fueron las que ejercieron la influencia más profunda; los humanos primitivos las veían pasar a toda velocidad por la atmósfera con un esplendor llameante. Las estrellas fugaces aterrorizaban al hombre primitivo, y éste creía con facilidad que estas señales brillantes indicaban el paso de un espíritu camino de la Tierra. No es de extrañar que los hombres se sintieran inducidos a adorar estos fenómenos, especialmente cuando más tarde descubrieron los meteoros. Esto condujo a una mayor veneración por todas las demás piedras. En Bengala, mucha gente adora un meteoro que cayó en la Tierra en el año 1880 d.de J.C.

\par
%\textsuperscript{(945.1)}
\textsuperscript{85:1.3} Todos los clanes y tribus antiguos tenían sus piedras sagradas, y la mayoría de los pueblos modernos manifiestan cierto grado de veneración por algunos tipos de piedras ---sus joyas. En la India se veneraba un grupo de cinco piedras; en Grecia era un grupo de treinta; entre los hombres rojos se trataba generalmente de un círculo de piedras. Los romanos siempre tiraban una piedra al aire cuando invocaban a Júpiter. En la India, incluso hoy en día, se puede utilizar una piedra como testigo\footnote{\textit{Una piedra como testigo}: Jos 24:27.}. En algunas regiones se puede emplear una piedra como talismán de la ley y, por su prestigio, un delincuente puede ser llevado ante el tribunal. Pero los mortales sencillos no siempre identifican a la Deidad con un objeto de culto reverente. Estos fetiches son muchas veces simples símbolos del verdadero objeto de adoración.

\par
%\textsuperscript{(945.2)}
\textsuperscript{85:1.4} Los antiguos tenían una consideración especial por los agujeros en las piedras. Se suponía que estas rocas porosas eran excepcionalmente eficaces para curar las enfermedades. Las orejas no se las perforaban para colgarse unas piedras, pero éstas sí se las colocaban en los agujeros de las orejas para mantenerlos abiertos. Incluso en los tiempos modernos, las personas supersticiosas hacen un agujero en las monedas. En África, los nativos hacen mucho ruido alrededor de sus piedras fetiches. De hecho, todas las tribus y pueblos atrasados conservan todavía una veneración supersticiosa por las piedras. Incluso en la actualidad, la adoración de las piedras está muy difundida por el mundo. Las lápidas sepulcrales son un símbolo sobreviviente de las imágenes y los ídolos que se esculpían en las piedras en conexión con las creencias en los fantasmas y los espíritus de los compañeros fallecidos.

\par
%\textsuperscript{(945.3)}
\textsuperscript{85:1.5} La adoración de las colinas siguió a la de las piedras, y las primeras colinas que se veneraron fueron las grandes formaciones rocosas\footnote{\textit{Lugares elevados}: 1 Re 3:2-3; 2 Re 18:4,34; Lv 26:30; Nm 21:28; 22:41; Dt 33:29.}. Poco después se cogió la costumbre de creer que los dioses vivían en las montañas\footnote{\textit{Montañas sagradas}: Ex 3:1,12; 19:3,11-23; 24:12-18; Sal 74:2; 121:1; Is 2:2; Jer 3:23; Nm 10:33; Dt 11:29; Jn 4:20-21.}, de manera que las altas elevaciones de tierra fueron adoradas por esta razón adicional. A medida que pasó el tiempo, algunas montañas fueron asociadas con ciertos dioses, y por lo tanto se volvieron sagradas. Los aborígenes ignorantes y supersticiosos creían que las cuevas conducían al infierno, con sus espíritus y demonios malignos, en contraste con las montañas, que eran identificadas con los conceptos que evolucionaron posteriormente sobre las deidades y los espíritus buenos.

\section*{2. La adoración de las plantas y los árboles}
\par
%\textsuperscript{(945.4)}
\textsuperscript{85:2.1} Las plantas fueron primero temidas, y después adoradas, a causa de los licores embriagadores que se obtenían de ellas. El hombre primitivo creía que la embriaguez lo volvía a uno divino. Se suponía que esta experiencia tenía algo de inhabitual y de sagrado. Incluso en los tiempos modernos, las bebidas alcohólicas se conocen con el nombre de <<bebidas espirituosas>>.

\par
%\textsuperscript{(945.5)}
\textsuperscript{85:2.2} El hombre primitivo miraba con temor y respeto supersticioso los granos que germinaban. El apóstol Pablo no fue el primero en extraer profundas lecciones espirituales\footnote{\textit{Lecciones espirituales}: 1 Co 15:35-38; 2 Co 9:10.} de los granos que brotaban, y en basar en ellos unas creencias religiosas.

\par
%\textsuperscript{(945.6)}
\textsuperscript{85:2.3} Los cultos de la adoración de los árboles se encuentran en los grupos religiosos más antiguos. Todas las bodas primitivas se celebraban debajo de los árboles, y cuando las mujeres deseaban tener hijos, a veces se las podía encontrar en el bosque abrazando afectuosamente a un robusto roble. Muchas plantas y árboles eran venerados a causa de sus poderes medicinales reales o imaginarios. Los salvajes creían que todos los efectos químicos se debían a la actividad directa de la fuerzas sobrenaturales.

\par
%\textsuperscript{(945.7)}
\textsuperscript{85:2.4} Las ideas sobre los espíritus de los árboles variaban considerablemente entre las diferentes tribus y razas. Algunos árboles estaban habitados por espíritus bondadosos; otros contenían espíritus engañosos y crueles. Los finlandeses creían que la mayoría de los árboles estaban ocupados por espíritus benévolos. Los suizos desconfiaron durante mucho tiempo de los árboles, creyendo que contenían espíritus astutos. Los habitantes de la India y de la Rusia oriental consideran que los espíritus de los árboles son crueles. Los patagones adoran todavía a los árboles, tal como lo hacían los semitas primitivos. Mucho tiempo después de que los hebreos dejaran de adorar a los árboles, continuaron venerando a sus diversas deidades en los bosquecillos\footnote{\textit{Arboledas sagradas}: Gn 21:33; 2 Re 17:9-11,16; 2 Cr 33:19; Dt 16:21.}. Salvo en China, en otro tiempo existió un culto universal al \textit{árbol de la vida}\footnote{\textit{Árbol de la vida}: Gn 2:9; 3:22; Ap 2:7; 22:2,14.}.

\par
%\textsuperscript{(946.1)}
\textsuperscript{85:2.5} La creencia de que el agua o los metales preciosos que se encuentran debajo de la superficie de la Tierra se pueden detectar con una varilla adivinatoria de madera es una reliquia de los antiguos cultos a los árboles. El mayo, los árboles de Navidad y la práctica supersticiosa de tocar madera perpetúan algunas costumbres antiguas de adoración de los árboles y de los cultos más recientes a los árboles.

\par
%\textsuperscript{(946.2)}
\textsuperscript{85:2.6} Muchas de estas formas iniciales de veneración de la naturaleza se mezclaron con las técnicas de adoración que evolucionaron más tarde, pero los primeros tipos de adoración activados por los espíritus ayudantes de la mente funcionaban mucho antes de que la naturaleza religiosa recién despierta de la humanidad se volviera plenamente sensible al estímulo de las influencias espirituales.

\section*{3. La adoración de los animales}
\par
%\textsuperscript{(946.3)}
\textsuperscript{85:3.1} El hombre primitivo tenía un sentimiento peculiar de compañerismo hacia los animales superiores. Sus antepasados habían vivido con ellos e incluso se habían apareado con ellos. En el sur de Asia se creyó muy pronto que las almas de los hombres volvían a la Tierra en forma de animales. Esta creencia era una supervivencia de la costumbre aún más antigua de adorar a los animales.

\par
%\textsuperscript{(946.4)}
\textsuperscript{85:3.2} Los hombres primitivos veneraban a los animales por su fuerza y su astucia. Creían que el agudo sentido del olfato y la vista penetrante de algunas bestias denotaban que estaban guiadas por los espíritus. Todos los animales han sido adorados por una u otra raza, en uno u otro momento. Entre estos objetos de adoración figuraban criaturas que eran consideradas como mitad humanas y mitad animales, tales como los centauros y las sirenas.

\par
%\textsuperscript{(946.5)}
\textsuperscript{85:3.3} Los hebreos adoraron a las serpientes\footnote{\textit{Adoración de las serpientes}: Ex 4:2-4; 7:9-12; 2 Re 18:4; Nm 21:8-9; Da 14:23 (Bel 1:23); Jn 3:14.} hasta la época del rey Ezequías, y los hindúes mantienen todavía relaciones amistosas con sus serpientes domésticas. La adoración de los chinos por el dragón es una supervivencia de los cultos a las serpientes. La sabiduría de la serpiente era un símbolo de la medicina griega y los médicos modernos lo emplean todavía como emblema. El arte de encantar las serpientes ha sido trasmitido desde los tiempos del \textit{culto del amor a las serpientes} de las mujeres chamanes, las cuales estaban inmunizadas a consecuencia de las mordeduras diarias de las serpientes; de hecho, se volvían auténticas adictas al veneno y no podían prescindir de esta ponzoña.

\par
%\textsuperscript{(946.6)}
\textsuperscript{85:3.4} La adoración de los insectos y de otros animales fue fomentada por una falsa interpretación posterior de la regla de oro ---hacer a los demás (a todas las formas de vida) lo que queréis que os hagan a vosotros. Los antiguos creían en otro tiempo que todos los vientos eran producidos por las alas de los pájaros, y por lo tanto temían y adoraban a la vez a todas las criaturas aladas. Los nórdicos primitivos pensaban que los eclipses eran causados por un lobo que devoraba una parte del Sol o de la Luna. Los hindúes muestran con frecuencia a Vichnú con una cabeza de caballo. Un símbolo animal representa muchas veces a un dios olvidado o un culto desaparecido. Al principio de la religión evolutiva, el cordero\footnote{\textit{Cordero sacrificial}: Gn 22:7-8; Ex 12:3-5; 34:20; Lv 4:32.} se convirtió en el típico animal sacrificatorio y la paloma en el símbolo de la paz y del amor\footnote{\textit{Paloma de la paz}: Gn 8:8-12; Lv 1:14; Mt 3:16; 10:16; Mc 1:10; Lc 3:22; Jn 1:32.}.

\par
%\textsuperscript{(946.7)}
\textsuperscript{85:3.5} En la religión, el simbolismo puede ser bueno o malo en la medida exacta en que el símbolo sustituya o no a la idea original de adoración. Y no se debe confundir el simbolismo con la idolatría directa, en la cual el objeto material es adorado de manera directa y real.

\section*{4. La adoración de los elementos}
\par
%\textsuperscript{(946.8)}
\textsuperscript{85:4.1} La humanidad ha adorado la tierra, el aire, el agua y el fuego. Las razas primitivas veneraban los manantiales y adoraban los ríos. En Mongolia florece, incluso en la actualidad, un influyente culto a los ríos. El bautismo\footnote{\textit{Bautismo}: Mt 3:6,11; Mc 1:4-5,8; Lc 3:3,7,16; Jn 1:25-26,31,33; 3:22-23; Hch 1:5.} se volvió un ceremonial religioso en Babilonia, y los creeks practicaban el baño ritual anual. A los antiguos les resultaba fácil imaginar que los espíritus vivían en los manantiales burbujeantes, en las fuentes que brotaban, en los ríos que fluían y en los torrentes impetuosos. Las aguas en movimiento\footnote{\textit{Aguas en movimiento}: Jn 5:2-4.} impresionaban intensamente a estas mentes sencillas, haciéndoles creer que estaban animadas por los espíritus y que tenían poderes sobrenaturales. A veces se negaban a socorrer a un hombre que se ahogaba por temor a ofender a algún dios del río.

\par
%\textsuperscript{(947.1)}
\textsuperscript{85:4.2} Muchas cosas y numerosos acontecimientos han actuado como estímulos religiosos para diferentes pueblos en distintas épocas. Muchas tribus de las colinas de la India adoran todavía el arco iris. Tanto en la India como en África se cree que el arco iris es una gigantesca serpiente celeste; los hebreos y los cristianos lo consideran como <<el arco de la promesa>>\footnote{\textit{Arco iris, arco de la promesa}: Gn 9:9-17.}. Del mismo modo, unas influencias consideradas como benéficas en una parte del mundo, pueden ser contempladas como perjudiciales en otras regiones. El viento del este es un dios en América del Sur porque trae la lluvia; en la India es un demonio porque trae el polvo y provoca la sequía. Los antiguos beduinos creían que un espíritu de la naturaleza producía los remolinos de arena, e incluso en la época de Moisés, la creencia en los espíritus de la naturaleza era lo suficientemente fuerte como para asegurar su perpetuación en la teología hebrea bajo la forma de los ángeles del fuego\footnote{\textit{Ángeles del fuego}: Ex 3:2; 13:21-22; 19:18; 2 Cr 7:1-3; Sal 104:4; Nm 9:15-16; Dt 4:12; Hch 7:30; Jue 6:20-21.}, del agua\footnote{\textit{Ángeles del agua}: Gn 16:7; Jn 5:4.} y del aire\footnote{\textit{Ángeles del aire}: Gn 22:15; Ef 2:2.}.

\par
%\textsuperscript{(947.2)}
\textsuperscript{85:4.3} Las nubes\footnote{\textit{Nubes amenazadoras}: Ex 16:10; 19:9,16; Lm 2:1.}, la lluvia\footnote{\textit{Lluvias inoportunas}: Gn 7:4,12; Jer 10:13; Ez 38:22; 1 Sam 12:17-18.} y el granizo\footnote{\textit{El granizo}: Ex 9:18-26; Sal 18:12; Ap 11:19.} han sido todos temidos y adorados por numerosas tribus primitivas y en muchos cultos iniciales de la naturaleza. Las tempestades con truenos y relámpagos aterrorizaban al hombre primitivo. Estas perturbaciones de los elementos le impresionaban tanto que el trueno\footnote{\textit{El trueno}: Ex 9:23; Dt 5:22; 1 Sam 7:10; 2 Sam 22:14-15.} era considerado como la voz de un dios encolerizado. La adoración del fuego y el miedo al relámpago estaban enlazados y muy difundidos entre numerosos grupos primitivos.

\par
%\textsuperscript{(947.3)}
\textsuperscript{85:4.4} El fuego\footnote{\textit{Reverencia al fuego}: 2 Re 16:3; 2 Cr 33:6; 2 Mac 1:18-22; Lv 18:21; Jer 32:35; Ez 16:21.} y la magia estaban mezclados en la mente de los mortales primitivos dominados por el miedo. Los partidarios de la magia recordarán vívidamente un resultado positivo obtenido por casualidad mediante la práctica de sus fórmulas mágicas, mientras que olvidan con indiferencia decenas de resultados negativos, de fracasos totales. La veneración del fuego alcanzó su punto culminante en Persia, donde sobrevivió durante mucho tiempo. Algunas tribus adoraban el fuego como una deidad en sí misma, otras lo reverenciaban como el símbolo llameante del espíritu purificador y purgador de las deidades que veneraban. Las vírgenes vestales tenían el deber de vigilar los fuegos sagrados, y en el siglo veinte se siguen encendiendo cirios como parte del ritual de muchos servicios religiosos\footnote{\textit{Reverencia al fuego (Hanuka)}: 1 Mac 4:52-59.}.

\section*{5. La adoración de los cuerpos celestes}
\par
%\textsuperscript{(947.4)}
\textsuperscript{85:5.1} La adoración de las piedras, las colinas, los árboles y los animales progresó de manera natural a través de la veneración temerosa de los elementos hasta llegar a la deificación del Sol, la Luna y las estrellas. En la India y en otros lugares, las estrellas eran consideradas como las almas glorificadas de los grandes hombres que habían dejado la vida en la carne. Los adeptos caldeos del culto a las estrellas pensaban que eran hijos del padre cielo y de la madre Tierra.

\par
%\textsuperscript{(947.5)}
\textsuperscript{85:5.2} La adoración de la Luna precedió a la del Sol. La veneración de la Luna alcanzó su apogeo durante la era de la caza, mientras que la adoración del Sol se convirtió en la ceremonia religiosa principal de las épocas agrícolas posteriores. La adoración del Sol se arraigó primero ampliamente en la India, y es allí donde sobrevivió más tiempo. En Persia, la veneración del Sol dio origen al culto mitríaco posterior. Muchos pueblos consideraban al Sol como el antepasado de sus reyes. Los caldeos colocaban al Sol en el centro de <<los siete círculos del universo>>. Las civilizaciones más tardías honraron al Sol poniendo su nombre al primer día de la semana.

\par
%\textsuperscript{(947.6)}
\textsuperscript{85:5.3} Se suponía que el dios Sol era el padre místico de los hijos del destino nacidos de una virgen, y se creía que éstos se donaban de vez en cuando como salvadores a las razas favorecidas. Estos niños sobrenaturales siempre eran abandonados a la deriva en algún río sagrado\footnote{\textit{Bebés abandonados en botes}: Ex 3:2.}, para ser salvados de una manera extraordinaria y crecer a continuación hasta convertirse en unas personalidades milagrosas y en los libertadores de sus pueblos.

\section*{6. La adoración del hombre}
\par
%\textsuperscript{(948.1)}
\textsuperscript{85:6.1} Después de haber adorado todo lo que se encontraba en la superficie de la Tierra y arriba en los cielos, el hombre no dudó en honrarse a sí mismo con esta adoración. El salvaje de mente sencilla no distingue claramente entre los animales, los hombres y los dioses.

\par
%\textsuperscript{(948.2)}
\textsuperscript{85:6.2} El hombre primitivo consideraba que todas las personas fuera de lo común eran sobrehumanas, y tenía tanto miedo de estos seres que les manifestaba un temor reverencial; en cierta medida, los adoraba literalmente. El hecho mismo de tener gemelos era considerado como una gran suerte o una gran desgracia. Los lunáticos, los epilépticos y los débiles mentales eran a menudo adorados por sus compañeros mentalmente normales, los cuales creían que estos seres anormales estaban habitados por los dioses. Se adoraba a los sacerdotes, los reyes y los profetas; se pensaba que los hombres santos de la antig\"uedad estaban inspirados por las deidades.

\par
%\textsuperscript{(948.3)}
\textsuperscript{85:6.3} Los jefes tribales morían y luego eran \textit{deificados}. Más tarde se \textit{canonizó} a las almas eminentes que habían pasado a mejor vida. La evolución, sin ayuda, nunca ha inventado unos dioses que fueran superiores a los espíritus glorificados, ensalzados y evolucionados de los humanos fallecidos. Al principio de la evolución, la religión crea sus propios dioses. En el transcurso de la revelación, los Dioses formulan la religión. La religión evolutiva crea sus dioses a imagen y semejanza del hombre mortal; la religión revelada intenta que el hombre mortal evolucione y se transforme a imagen y semejanza de Dios.

\par
%\textsuperscript{(948.4)}
\textsuperscript{85:6.4} Los dioses fantasmas, que tienen un supuesto origen humano, deben distinguirse de los dioses de la naturaleza, pues la adoración de la naturaleza produjo un panteón ---los espíritus de la naturaleza elevados a la posición de dioses. Los cultos de la naturaleza continuaron desarrollándose junto con los cultos a los fantasmas que aparecieron más tarde, y cada uno ejerció su influencia sobre el otro. Muchos sistemas religiosos contenían un doble concepto de la deidad: los dioses de la naturaleza y los dioses fantasmas; en algunas teologías estos dos conceptos están entrelazados de manera confusa, tal como sucede en el ejemplo de Thor, el héroe fantasma que era también el señor del rayo.

\par
%\textsuperscript{(948.5)}
\textsuperscript{85:6.5} Pero la adoración del hombre por el hombre alcanzó su punto culminante cuando los gobernantes temporales ordenaron a sus súbditos que los veneraran así y, para justificar estas exigencias, pretendieron que habían descendido de la deidad.

\section*{7. Los ayudantes de la adoración y la sabiduría}
\par
%\textsuperscript{(948.6)}
\textsuperscript{85:7.1} La adoración de la naturaleza puede parecer que surgió de manera natural y espontánea en la mente de los hombres y las mujeres primitivos, y así es como ocurrió; pero durante todo este tiempo estuvo actuando en estas mismas mentes primitivas el sexto espíritu ayudante, que había sido conferido a estos pueblos como influencia directriz para esta fase de la evolución humana. Este espíritu estimulaba constantemente el impulso a la adoración en la especie humana, por muy primitivas que fueran sus primeras manifestaciones. El espíritu de adoración dio claramente origen al impulso humano de adorar, a pesar de que el miedo animal fue el que motivó la expresión de la adoración, y de que sus prácticas iniciales se centraron en las cosas de la naturaleza.

\par
%\textsuperscript{(948.7)}
\textsuperscript{85:7.2} Debéis recordar que fue el sentimiento, y no el pensamiento, la influencia que dirigió y controló todo el desarrollo evolutivo. Para la mente primitiva existe poca diferencia entre tener miedo, rehuir, honrar y adorar.

\par
%\textsuperscript{(948.8)}
\textsuperscript{85:7.3} Cuando el impulso de adoración está animado y dirigido por la sabiduría ---por el pensamiento meditativo y experiencial--- entonces empieza a convertirse en el fenómeno de la verdadera religión. Cuando el séptimo espíritu ayudante, el espíritu de la sabiduría, consigue ejercer eficazmente su ministerio, el hombre empieza entonces a desviar su adoración de la naturaleza y de los objetos naturales, para dirigirla hacia el Dios de la naturaleza y hacia el Creador eterno de todas las cosas naturales.

\par
%\textsuperscript{(949.1)}
\textsuperscript{85:7.4} [Presentado por una Brillante Estrella Vespertina de Nebadon.]


\chapter{Documento 86. La evolución inicial de la religión}
\par
%\textsuperscript{(950.1)}
\textsuperscript{86:0.1} LA EVOLUCIÓN de la religión a partir del impulso precedente y primitivo a la adoración no depende de la revelación. El funcionamiento normal de la mente humana bajo la influencia directriz del sexto y séptimo ayudantes de la mente, que son una parte de la concesión universal del espíritu, es enteramente suficiente para asegurar dicho desarrollo.

\par
%\textsuperscript{(950.2)}
\textsuperscript{86:0.2} El miedo prerreligioso inicial del hombre a las fuerzas de la naturaleza se volvió gradualmente religioso a medida que la naturaleza fue personalizada, convertida en espíritu y finalmente deificada en la conciencia humana. La religión de tipo primitivo fue por tanto una consecuencia biológica natural de la inercia psicológica de la mente animal en evolución, después de que esta mente hubo albergado por primera vez el concepto de lo sobrenatural.

\section*{1. La casualidad: la buena y la mala suerte}
\par
%\textsuperscript{(950.3)}
\textsuperscript{86:1.1} Aparte del impulso natural a la adoración, la religión evolutiva primitiva tuvo sus raíces originales en las experiencias humanas con la casualidad: la llamada suerte, los acontecimientos corrientes. El hombre primitivo cazaba para alimentarse. Los resultados de la caza son siempre necesariamente variables, y esto da origen inevitablemente a esas experiencias que el hombre interpreta como \textit{buena suerte} y \textit{mala suerte}. La desgracia era un factor importante en la vida de unos hombres y mujeres que vivían constantemente al borde de una existencia precaria y agobiada.

\par
%\textsuperscript{(950.4)}
\textsuperscript{86:1.2} El horizonte intelectual limitado del salvaje concentra tanto la atención en la casualidad que la suerte se vuelve un factor constante en su vida. Los urantianos primitivos luchaban por la existencia, no por un nivel de vida; vivían una vida llena de peligros en la que la casualidad jugaba un papel importante. La aprensión constante de que se produjera una calamidad desconocida e invisible se cernía sobre estos salvajes como una nube de desesperación que eclipsaba eficazmente todos los placeres; vivían con el miedo constante de hacer algo que atrajera la mala suerte. Los salvajes supersticiosos siempre temían una racha de buena suerte; consideraban esta buena fortuna como un presagio seguro de calamidades.

\par
%\textsuperscript{(950.5)}
\textsuperscript{86:1.3} Este terror siempre presente a la mala suerte era paralizante. ¿Para qué trabajar duro y cosechar la mala suerte ---dar algo por nada--- cuando uno puede dejarse llevar por los acontecimientos y encontrar la buena suerte ---obtener algo por nada? Los hombres irreflexivos olvidan la buena suerte ---la dan por sentada--- pero recuerdan dolorosamente la mala suerte.

\par
%\textsuperscript{(950.6)}
\textsuperscript{86:1.4} El hombre primitivo vivía en la incertidumbre y el miedo constante a la casualidad ---a la mala suerte. La vida era un emocionante juego de azar; la existencia era una lotería. No es de extrañar que la gente parcialmente civilizada crea todavía en la casualidad y manifieste una predisposición persistente por los juegos de azar. El hombre primitivo alternaba entre dos poderosos intereses: la pasión de conseguir algo por nada y el temor a no conseguir nada por algo. Este juego de azar de la existencia era el interés principal y la fascinación suprema de la mente salvaje primitiva.

\par
%\textsuperscript{(951.1)}
\textsuperscript{86:1.5} Más tarde, los pastores tuvieron el mismo punto de vista sobre la casualidad y la suerte, mientras que los agricultores aun más tardíos fueron cada vez más conscientes de que las cosechas sufrían la influencia inmediata de muchos factores sobre los que el hombre tenía poco o ningún control. Los campesinos eran víctimas de la sequía, las inundaciones, el granizo, las tormentas, las plagas y las enfermedades de las plantas, así como del calor y del frío. Y en la medida en que todas estas influencias naturales afectaban la prosperidad individual, eran consideradas como buena o mala suerte.

\par
%\textsuperscript{(951.2)}
\textsuperscript{86:1.6} Este concepto de la casualidad y la suerte impregnó poderosamente la filosofía de todos los pueblos antiguos. Incluso en una época reciente, en la sabiduría de Salomón se dice: <<Me volví y observé que la carrera no es de los ligeros, ni la batalla de los fuertes, ni tampoco de los sabios el pan, ni de los entendidos las riquezas, ni de los hábiles el favor; sino que el destino y la casualidad les acontece a todos. Porque el hombre no conoce su destino; al igual que los peces son cogidos en una red destructora, y los pájaros atrapados con el lazo, los hijos de los hombres caen en la trampa de una mala época cuando ésta les sobreviene de repente>>\footnote{\textit{La carrera no la ganan los ligeros}: Ec 9:11-12.}.

\section*{2. La personificación de la casualidad}
\par
%\textsuperscript{(951.3)}
\textsuperscript{86:2.1} La ansiedad era el estado natural de la mente salvaje. Cuando los hombres y las mujeres caen víctimas de una ansiedad excesiva, vuelven simplemente al estado natural de sus lejanos antepasados; y cuando la ansiedad se vuelve realmente dolorosa, inhibe la actividad y produce infaliblemente cambios evolutivos y adaptaciones biológicas. El dolor y el sufrimiento son esenciales para la evolución progresiva.

\par
%\textsuperscript{(951.4)}
\textsuperscript{86:2.2} La lucha por la vida es tan dolorosa que incluso en la actualidad algunas tribus atrasadas dan alaridos y se lamentan cada nuevo amanecer. El hombre primitivo se preguntaba constantemente: <<¿Quién me atormenta?>>. Al no encontrar la fuente material de sus sufrimientos, se decidió por la explicación de que eran causados por los espíritus. La religión nació así del miedo a lo misterioso, del temor a lo invisible y del terror a lo desconocido. El miedo a la naturaleza se volvió así un factor en la lucha por la existencia, primero debido a la casualidad y luego a causa del misterio.

\par
%\textsuperscript{(951.5)}
\textsuperscript{86:2.3} La mente primitiva era lógica, pero contenía pocas ideas para asociarlas de manera inteligente; la mente del salvaje era inculta, totalmente ingenua. Si un acontecimiento seguía a otro, el salvaje los consideraba como causa y efecto. Aquello que el hombre civilizado considera como una superstición, sólo era pura ignorancia en el salvaje. La humanidad ha sido lenta en aprender que no hay necesariamente una relación entre las intenciones y los resultados. Los seres humanos acaban de empezar a darse cuenta de que las reacciones de la existencia aparecen entre los actos y sus consecuencias. El salvaje se esfuerza por personalizar todo lo que es intangible y abstracto, y así es como la naturaleza y la casualidad fueron personalizadas como fantasmas ---espíritus--- y más tarde como dioses.

\par
%\textsuperscript{(951.6)}
\textsuperscript{86:2.4} El hombre tiende a creer de manera natural en aquello que considera lo mejor para él, en aquello que forma parte de sus intereses cercanos o lejanos; el interés personal oscurece ampliamente la lógica. La diferencia entre la mente del salvaje y la del hombre civilizado reside más en el contenido que en la naturaleza, en el grado más bien que en la calidad.

\par
%\textsuperscript{(951.7)}
\textsuperscript{86:2.5} Pero continuar atribuyendo las cosas difíciles de comprender a las causas sobrenaturales no es más que una manera perezosa y cómoda de evitar todas las formas de esfuerzo intelectual. La suerte es simplemente un término acuñado para abarcar lo inexplicable en cualquier época de la existencia humana; designa aquellos fenómenos que los hombres son incapaces o no tienen deseos de descubrir. La casualidad es una palabra que significa que el hombre es demasiado ignorante o demasiado indolente como para determinar las causas. Los hombres sólo consideran un acontecimiento natural como un accidente o como mala suerte cuando están desprovistos de curiosidad e imaginación, cuando las razas carecen de iniciativa y de espíritu aventurero. La investigación de los fenómenos de la vida destruye tarde o temprano la creencia del hombre en la casualidad, la suerte y los supuestos accidentes, sustituyéndola por un universo de ley y de orden donde todos los efectos están precedidos por unas causas definidas. El miedo a la existencia es así reemplazado por la alegría de vivir.

\par
%\textsuperscript{(952.1)}
\textsuperscript{86:2.6} El salvaje consideraba que toda la naturaleza estaba viva, poseída por algo. El hombre civilizado todavía maldice y da un puntapié a los objetos inanimados con los que se tropieza en su camino. El hombre primitivo nunca consideraba que algo fuera accidental; todo era siempre intencional. Para el hombre primitivo, el ámbito del destino, la función de la suerte, el mundo de los espíritus, estaban tan desorganizados y dirigidos al azar como la sociedad primitiva. La suerte era considerada como la reacción caprichosa y temperamental del mundo de los espíritus y, más tarde, como el estado de ánimo de los dioses.

\par
%\textsuperscript{(952.2)}
\textsuperscript{86:2.7} Pero no todas las religiones se desarrollaron a partir del animismo. Otros conceptos de lo sobrenatural fueron contemporáneos del animismo, y estas creencias condujeron también a la adoración. El naturalismo no es una religión ---es el fruto de la religión.

\section*{3. La muerte ---lo inexplicable}
\par
%\textsuperscript{(952.3)}
\textsuperscript{86:3.1} La muerte era para el hombre evolutivo el impacto supremo, la combinación más confusa de casualidad y de misterio. No fue la santidad de la vida, sino el horror a la muerte, lo que inspiró el miedo y fomentó así eficazmente la religión. Entre los pueblos salvajes, la muerte se debía generalmente a la violencia, de manera que la muerte no violenta se volvió cada vez más misteriosa. La muerte como fin natural y esperado de la vida no estaba clara en la conciencia de la gente primitiva, y el hombre ha necesitado siglos y siglos para darse cuenta de su inevitabilidad.

\par
%\textsuperscript{(952.4)}
\textsuperscript{86:3.2} El hombre primitivo aceptaba la vida como un hecho, mientras que consideraba la muerte como algún tipo de castigo. Todas las razas tienen sus leyendas sobre hombres que no han muerto, tradiciones residuales de la actitud inicial ante la muerte. En la mente humana ya existía el concepto nebuloso de un mundo espiritual vago y desorganizado, un ámbito de donde procedía todo lo que es inexplicable en la vida humana, y la muerte se añadió a esta larga lista de fenómenos inexplicados.

\par
%\textsuperscript{(952.5)}
\textsuperscript{86:3.3} Al principio se creía que todas las enfermedades humanas y la muerte natural se debían a la influencia de los espíritus. Incluso en la época actual, algunas razas civilizadas consideran que la enfermedad ha sido producida por <<el enemigo>>, y cuentan con las ceremonias religiosas para llevar a cabo la curación. Algunos sistemas teológicos más recientes y complejos continúan atribuyendo la muerte a la acción del mundo de los espíritus, y todo ello ha conducido a doctrinas tales como el pecado original y la caída del hombre.

\par
%\textsuperscript{(952.6)}
\textsuperscript{86:3.4} La comprensión de su impotencia ante las fuerzas poderosas de la naturaleza, junto con el reconocimiento de la debilidad humana ante los azotes de la enfermedad y la muerte, fue lo que impulsó al salvaje a buscar ayuda en el mundo supermaterial, que él imaginaba vagamente como la fuente de estas misteriosas vicisitudes de la vida.

\section*{4. El concepto de la supervivencia después de la muerte}
\par
%\textsuperscript{(952.7)}
\textsuperscript{86:4.1} El concepto de una fase supermaterial de la personalidad mortal nació de la asociación inconsciente y puramente accidental entre los acontecimientos de la vida diaria y el hecho de soñar con los fantasmas. El hecho de que varios miembros de una tribu soñaran simultáneamente con un jefe fallecido parecía constituir una prueba convincente de que el viejo jefe había regresado realmente bajo alguna forma. Todo esto era muy real para el salvaje, que solía despertarse de estos sueños bañado en sudor, temblando y gritando.

\par
%\textsuperscript{(953.1)}
\textsuperscript{86:4.2} El origen onírico de la creencia en una existencia futura explica la tendencia a imaginar siempre las cosas invisibles en términos de las cosas visibles. Este nuevo concepto de la vida futura, surgido de los sueños con los fantasmas, pronto empezó a servir de antídoto eficaz contra el miedo a la muerte asociado al instinto biológico de conservación.

\par
%\textsuperscript{(953.2)}
\textsuperscript{86:4.3} El hombre primitivo también se preocupaba mucho por su respiración, especialmente en los climas fríos, donde ésta aparecía como un vaho en el momento de exhalar. El \textit{aliento de la vida}\footnote{\textit{Aliento de la vida}: Gn 2:7; 6:17; 7:15,22; Job 33:4.} fue considerado como el único fenómeno que diferenciaba a los vivos de los muertos. El hombre primitivo sabía que su aliento podía abandonar su cuerpo, y sus sueños, en los que hacía todo tipo de cosas extrañas mientras dormía, le convencieron de que el ser humano poseía algo inmaterial. La idea más primitiva del alma humana, el fantasma, tuvo su origen en el sistema de ideas relacionado con el sueño y la respiración.

\par
%\textsuperscript{(953.3)}
\textsuperscript{86:4.4} El salvaje se imaginó finalmente a sí mismo como un ser doble ---cuerpo y aliento. El aliento menos el cuerpo equivalía a un espíritu, a un fantasma. Aunque los fantasmas, o los espíritus, tuvieron un origen humano muy preciso, se les consideraba como superhumanos. Esta creencia en la existencia de espíritus incorpóreos parecía explicar la presencia de lo insólito, lo extraordinario, lo infrecuente y lo inexplicable.

\par
%\textsuperscript{(953.4)}
\textsuperscript{86:4.5} La doctrina primitiva de la supervivencia después de la muerte no era necesariamente una creencia en la inmortalidad. Unos seres que no sabían contar más allá de veinte difícilmente podían concebir la infinidad y la eternidad; pensaban más bien en encarnaciones periódicas.

\par
%\textsuperscript{(953.5)}
\textsuperscript{86:4.6} La raza anaranjada tenía una inclinación especial por la creencia en la transmigración y la reencarnación. Esta idea de la reencarnación tuvo su origen en la observación del parecido hereditario y de los rasgos entre los descendientes y sus antepasados. La costumbre de poner a los niños el nombre de sus abuelos y de otros antepasados se debía a la creencia en la reencarnación. Algunas razas más recientes creían que el hombre moría entre tres y siete veces. Esta creencia (residuo de las enseñanzas de Adán sobre los mundos de las mansiones), y otros muchos vestigios de la religión revelada, se pueden encontrar entre las doctrinas, por otra parte absurdas, de los bárbaros del siglo veinte.

\par
%\textsuperscript{(953.6)}
\textsuperscript{86:4.7} El hombre primitivo no albergaba ninguna idea sobre el infierno o los castigos futuros. El salvaje consideraba que la vida futura era exactamente como ésta, menos toda la mala suerte. Más tarde se concibió un destino separado para los buenos y los malos fantasmas ---el cielo y el infierno. Pero como muchas razas primitivas creían que el hombre empezaba en la vida siguiente en el mismo estado en que había dejado ésta, no les hacía ninguna gracia la idea de volverse viejos y decrépitos. Los ancianos preferían con mucho que los mataran antes de volverse demasiado débiles.

\par
%\textsuperscript{(953.7)}
\textsuperscript{86:4.8} Casi todos los grupos tenían ideas diferentes sobre el destino del alma fantasma. Los griegos creían que los hombres débiles debían tener almas débiles; así pues inventaron el Hades como lugar adecuado para recibir estas almas anémicas; también suponían que estos especímenes poco vigorosos tenían unas sombras más pequeñas. Los primeros anditas pensaban que sus fantasmas volvían a las tierras natales de sus antepasados. Los chinos y los egipcios creyeron en otro tiempo que el alma y el cuerpo permanecían juntos. Esto condujo a los egipcios a construir cuidadosamente las tumbas y a esforzarse por preservar los cuerpos. Incluso los pueblos modernos tratan de detener la descomposición de los muertos. Los hebreos imaginaban que una réplica fantasmal del individuo bajaba al Sheol\footnote{\textit{Sheol}: Lc 16:17-26.}, y no podía regresar al mundo de los vivos. Hicieron este progreso importante en la doctrina de la evolución del alma.

\section*{5. El concepto del alma fantasma}
\par
%\textsuperscript{(953.8)}
\textsuperscript{86:5.1} La parte no material del hombre ha sido llamada diversamente fantasma, espíritu, sombra, aparecido, espectro, y más recientemente \textit{alma}. Cuando el hombre primitivo soñaba, el alma era su doble; era en todos los aspectos exactamente igual al mortal mismo, salvo que no era sensible al tacto. La creencia en los dobles oníricos condujo directamente a la idea de que todas las cosas animadas e inanimadas tenían un alma, igual que los hombres. Este concepto tendió a perpetuar durante mucho tiempo las creencias en los espíritus de la naturaleza. Los esquimales piensan todavía que todas las cosas de la naturaleza tienen un espíritu.

\par
%\textsuperscript{(954.1)}
\textsuperscript{86:5.2} El alma fantasma podía verse y oírse, pero no se podía tocar. La vida onírica de la raza desarrolló y amplió gradualmente las actividades de este mundo evolutivo de los espíritus hasta el punto de que la muerte fue finalmente considerada como <<entregar el alma>>\footnote{\textit{Entregar el alma}: Gn 25:8; Job 3:11; 13:19; Jer 15:9; Lm 1:19; Jn 19:30.}. Todas las tribus primitivas, salvo aquellas que apenas se encontraban por encima de los animales, han desarrollado algún concepto del alma. A medida que avanza la civilización, este concepto supersticioso del alma es destruido, y el hombre depende enteramente de la revelación y de la experiencia religiosa personal para hacerse una nueva idea del alma como creación conjunta de la mente mortal que conoce a Dios y del espíritu divino que la habita, el Ajustador del Pensamiento.

\par
%\textsuperscript{(954.2)}
\textsuperscript{86:5.3} Los mortales primitivos no lograban generalmente diferenciar los conceptos de un espíritu interior y de un alma de naturaleza evolutiva. El salvaje tenía mucha confusión en cuanto a si el alma fantasma existía de manera innata en el cuerpo o se trataba de un agente externo en posesión del cuerpo. La ausencia de un pensamiento razonado en presencia de la perplejidad explica las grandes contradicciones del punto de vista de los salvajes sobre las almas, los fantasmas y los espíritus.

\par
%\textsuperscript{(954.3)}
\textsuperscript{86:5.4} Se creía que el alma estaba asociada al cuerpo como el perfume a la flor. Los antiguos creían que el alma podía abandonar el cuerpo de diversas maneras, tales como:

\par
%\textsuperscript{(954.4)}
\textsuperscript{86:5.5} 1. El desmayo corriente y transitorio.

\par
%\textsuperscript{(954.5)}
\textsuperscript{86:5.6} 2. Durmiendo, durante el sueño natural.

\par
%\textsuperscript{(954.6)}
\textsuperscript{86:5.7} 3. El coma y la inconsciencia que acompañan a la enfermedad y los accidentes.

\par
%\textsuperscript{(954.7)}
\textsuperscript{86:5.8} 4. La muerte, la partida definitiva.

\par
%\textsuperscript{(954.8)}
\textsuperscript{86:5.9} El salvaje consideraba que el estornudo era un intento frustrado del alma por escapar del cuerpo. Como estaba despierto y vigilante, el cuerpo era capaz de impedir el intento de huida del alma. Más tarde, los estornudos siempre estuvieron acompañados de alguna expresión religiosa, tales como <<¡Jesús, María y José!>>

\par
%\textsuperscript{(954.9)}
\textsuperscript{86:5.10} Al principio de la evolución, el sueño era considerado como la prueba de que el alma fantasma podía ausentarse del cuerpo, y se creía que se la podía hacer regresar diciendo o gritando el nombre de la persona que dormía. En otras formas de inconsciencia, se pensaba que el alma se había alejado más, intentando quizás escaparse para siempre ---la muerte inminente. Se estimaba que los sueños eran las experiencias del alma mientras ésta se encontraba temporalmente ausente del cuerpo que dormía. El salvaje cree que sus sueños son tan reales como cualquier otra parte de su experiencia consciente. Los antiguos tenían la costumbre de despertar gradualmente a las personas que dormían, para que el alma tuviera tiempo de regresar al cuerpo.

\par
%\textsuperscript{(954.10)}
\textsuperscript{86:5.11} A lo largo de todas las épocas, los hombres han tenido un miedo pavoroso a las apariciones durante el período nocturno, y los hebreos no fueron una excepción. Creían realmente que Dios les hablaba en sueños\footnote{\textit{Dios ``hablando'' en sueños}: Gn 20:3; 28:12-16; Jer 23:25-26; Nm 12:6; Dn 1:17; Mt 1:20; 1 Sam 28:15.}, a pesar de los preceptos de Moisés en contra de esta idea\footnote{\textit{Mandato de Moisés sobre los sueños}: Dt 13:1-5.}. Y Moisés tenía razón, porque los sueños ordinarios no son los métodos que emplean las personalidades del mundo espiritual cuando intentan comunicarse con los seres materiales.

\par
%\textsuperscript{(954.11)}
\textsuperscript{86:5.12} Los antiguos creían que las almas podían introducirse en los animales e incluso en los objetos inanimados. Esto culminó en las ideas de la identificación con los animales, como por ejemplo la del hombre lobo. Una persona podía ser un ciudadano respetuoso de las leyes durante el día, pero cuando se dormía, su alma podía meterse en un lobo o en cualquier otro animal y merodear cometiendo depredaciones nocturnas.

\par
%\textsuperscript{(955.1)}
\textsuperscript{86:5.13} Los hombres primitivos creían que el alma estaba asociada a la respiración, y que sus cualidades se podían comunicar o transferir por medio del aliento. El jefe valeroso solía echar su aliento sobre el niño recién nacido para conferirle la valentía. Entre los primeros cristianos, la ceremonia de donación del Espíritu Santo estaba acompañada de un soplo sobre los candidatos\footnote{\textit{Respiración y espíritu}: Jn 20:22.}. El salmista dijo: <<Los cielos han sido creados por la palabra del Señor, y todas las huestes que lo componen por el soplo de su boca>>\footnote{\textit{Creación por la palabra}: Sal 33:6.}. Durante mucho tiempo, el hijo mayor tuvo la costumbre de intentar atrapar el último suspiro de su padre moribundo.

\par
%\textsuperscript{(955.2)}
\textsuperscript{86:5.14} Más tarde se llegó a temer y a venerar la sombra de la misma manera que el aliento. La imagen de sí mismo reflejada en el agua también era considerada a veces como prueba de la dualidad del ser, y los espejos eran contemplados con un temor supersticioso. Incluso hoy en día, muchas personas civilizadas vuelven el espejo hacia la pared en caso de muerte. Algunas tribus atrasadas creen todavía que hacer retratos, dibujos, modelos o imágenes saca toda el alma del cuerpo, o una parte de ella, y por eso este tipo de cosas están prohibidas.

\par
%\textsuperscript{(955.3)}
\textsuperscript{86:5.15} Se creía generalmente que el alma estaba identificada con el aliento, pero diversos pueblos la situaron también en la cabeza, el cabello, el corazón, el hígado, la sangre y la grasa. <<La sangre de Abel que clama desde la tierra>>\footnote{\textit{La sangre de Abel que clama desde la tierra}: Gn 4:10.} expresa la antigua creencia en la presencia del fantasma en la sangre. Los semitas enseñaban que el alma residía en la grasa del cuerpo, y para muchas tribus era tabú comer la grasa animal\footnote{\textit{Tabú de la grasa animal}: Lv 3:17; 7:22-25.}. Cazar cabezas era un método de apresar el alma del enemigo, tal como lo era quitarle el cuero cabelludo. En tiempos más recientes, los ojos han sido considerados como las ventanas del alma\footnote{\textit{Ojos como ventanas del alma}: Mt 6:22-23; Lc 11:34.}.

\par
%\textsuperscript{(955.4)}
\textsuperscript{86:5.16} Aquellos que sostenían la doctrina de que existían tres o cuatro almas creían que la pérdida de una de ellas significaba malestar, la pérdida de dos, enfermedad, y la pérdida de tres, la muerte. Un alma vivía en el aliento, otra en la cabeza, otra en el cabello y otra en el corazón. Se aconsejaba a los enfermos que se pasearan al aire libre con la esperanza de recuperar sus almas extraviadas. Se suponía que los curanderos más importantes intercambiaban el alma sin salud de una persona enferma por un alma nueva, el <<nuevo nacimiento>>.

\par
%\textsuperscript{(955.5)}
\textsuperscript{86:5.17} Los hijos de Badonán desarrollaron la creencia en dos almas: el aliento y la sombra. Las primeras razas noditas estimaban que el hombre consistía en dos personas: el alma y el cuerpo. Esta filosofía de la existencia humana se reflejó más tarde en el punto de vista griego. Los griegos mismos creían en tres almas; la vegetativa residía en el estómago, la animal en el corazón y la intelectual en la cabeza. Los esquimales creen que el hombre está compuesto de tres partes: el cuerpo, el alma y el nombre.

\section*{6. El entorno de espíritus y fantasmas}
\par
%\textsuperscript{(955.6)}
\textsuperscript{86:6.1} El hombre heredó un entorno natural, adquirió un entorno social e imaginó un entorno fantasmal. El Estado es la reacción del hombre hacia su entorno natural, el hogar, hacia su entorno social, y la iglesia, hacia su entorno ilusorio de fantasmas.

\par
%\textsuperscript{(955.7)}
\textsuperscript{86:6.2} Al principio de la historia de la humanidad, la creencia en las realidades del mundo imaginario de los fantasmas y los espíritus se volvió universal, y este mundo de espíritus recién imaginado se convirtió en una fuerza en la sociedad primitiva. La vida mental y moral de toda la humanidad fue modificada para siempre mediante la aparición de este nuevo factor en el pensamiento y la actuación de los hombres.

\par
%\textsuperscript{(955.8)}
\textsuperscript{86:6.3} El miedo humano ha amontonado todas las supersticiones y religiones posteriores de los pueblos primitivos dentro de esta premisa principal de ilusiones e ignorancia. Ésta fue la única religión del hombre hasta los tiempos de la revelación, y hoy en día, muchas razas del mundo sólo poseen esta religión evolutiva rudimentaria.

\par
%\textsuperscript{(955.9)}
\textsuperscript{86:6.4} A medida que progresó la evolución, la buena suerte fue relacionada con los buenos espíritus y la mala suerte con los espíritus malignos. La incomodidad de tener que adaptarse a la fuerza a un entorno cambiante era considerada como mala suerte, el desagrado de los fantasmas espíritus. El hombre primitivo desarrolló lentamente la religión a partir de su impulso innato a la adoración y de su concepto erróneo sobre la casualidad. El hombre civilizado establece unos sistemas de seguros para vencer estos sucesos del azar; la ciencia moderna coloca un actuario versado en cálculos matemáticos en el lugar de los espíritus ficticios y los dioses caprichosos.

\par
%\textsuperscript{(956.1)}
\textsuperscript{86:6.5} Cada generación que pasa sonríe ante las supersticiones descabelladas de sus antepasados, mientras que continúa manteniendo aquellos sofismas de pensamiento y de adoración que harán sonreír a su vez a la posteridad más ilustrada.

\par
%\textsuperscript{(956.2)}
\textsuperscript{86:6.6} Pero, por fin, la mente del hombre primitivo estaba ocupada con unas ideas que trascendían todos sus impulsos biológicos inherentes; por fin el hombre estaba a punto de desarrollar un arte de vivir basado en algo más que la reacción a los estímulos materiales. Los principios de un primitivo sistema filosófico de vida empezaban a emerger. Un criterio de vida sobrenatural estaba a punto de aparecer porque, si el fantasma espíritu infligía la mala suerte cuando estaba enojado, y la buena suerte cuando estaba contento, entonces la conducta humana tenía que regularse en consecuencia. El concepto del bien y del mal había aparecido finalmente por evolución; y todo ello mucho antes de que se efectuara ninguna revelación en la Tierra.

\par
%\textsuperscript{(956.3)}
\textsuperscript{86:6.7} Con la aparición de estos conceptos empezó la larga lucha ruinosa por apaciguar a los espíritus siempre descontentos, la esclavitud servil al miedo religioso evolutivo, esa larga pérdida de esfuerzos humanos en tumbas, templos, sacrificios y sacerdotes. El precio que hubo que pagar fue terrible y espantoso, pero valió la pena todo lo que costó, porque gracias a ello el hombre alcanzó una conciencia natural del bien y del mal relativos; ¡la ética humana había nacido!

\section*{7. La función de la religión primitiva}
\par
%\textsuperscript{(956.4)}
\textsuperscript{86:7.1} El salvaje sentía la necesidad de un seguro, y por lo tanto pagaba gustosamente sus onerosas primas de miedo, superstición, terror y regalos a los sacerdotes por su póliza de seguro mágico contra la mala suerte. La religión primitiva consistía simplemente en el pago de las primas del seguro contra los peligros del bosque; el hombre civilizado paga unas primas materiales contra los accidentes de la industria y las exigencias de las formas de vida modernas.

\par
%\textsuperscript{(956.5)}
\textsuperscript{86:7.2} La sociedad moderna le está quitando el negocio de los seguros al dominio de los sacerdotes y de la religión, para colocarlo en el ámbito de la economía. La religión se interesa cada vez más por el seguro de vida más allá de la tumba. Los hombres modernos, al menos aquellos que piensan, ya no pagan unas primas ruinosas para controlar la suerte. La religión está ascendiendo lentamente a unos niveles filosóficos más elevados, en contraste con su antigua función como sistema de seguro contra la mala suerte.

\par
%\textsuperscript{(956.6)}
\textsuperscript{86:7.3} Pero estas antiguas ideas religiosas impidieron que los hombres se volvieran fatalistas y desesperadamente pesimistas; creían que al menos podían hacer algo para influir sobre el destino. La religión del miedo a los fantasmas inculcó a los hombres que debían \textit{reglamentar su conducta}, que existía un mundo supermaterial que controlaba el destino humano.

\par
%\textsuperscript{(956.7)}
\textsuperscript{86:7.4} Las razas civilizadas modernas están empezando a salir del miedo a los fantasmas como explicación de la suerte y de las desigualdades corrientes de la existencia. La humanidad está logrando emanciparse de la esclavitud a los espíritus-fantasmas como explicación de la mala suerte. Pero al mismo tiempo que los hombres abandonan la doctrina errónea de que las vicisitudes de la vida están causadas por los espíritus, manifiestan una inclinación sorprendente a aceptar una enseñanza casi igual de falaz que les invita a atribuir todas las desigualdades humanas a la mala adaptación política, a la injusticia social y a la competencia industrial. Pero una nueva legislación, una filantropía cada vez mayor y una reorganización industrial más extensa, por muy buenas que sean en sí mismas y por sí mismas, no remediarán los hechos del nacimiento ni los accidentes de la vida. Únicamente la comprensión de los hechos y una sabia manipulación dentro de los límites de las leyes de la naturaleza, permitirán al hombre conseguir lo que quiere y evitar lo que no desea. El conocimiento científico, que conduce a la acción científica, es el único antídoto que existe contra las llamadas desgracias accidentales.

\par
%\textsuperscript{(957.1)}
\textsuperscript{86:7.5} La industria, la guerra, la esclavitud y el gobierno civil aparecieron en respuesta a la evolución social del hombre en su entorno natural. La religión surgió igualmente como la respuesta del hombre al entorno ilusorio del mundo imaginario de los fantasmas. La religión fue un desarrollo evolutivo de la preservación de sí mismo, y surtió efecto, a pesar de que al principio partió de un concepto erróneo y era totalmente ilógica.

\par
%\textsuperscript{(957.2)}
\textsuperscript{86:7.6} Gracias a la fuerza poderosa e impresionante del falso miedo, la religión primitiva preparó el terreno de la mente humana para la concesión de una auténtica fuerza espiritual de origen sobrenatural, el Ajustador del Pensamiento. Y los Ajustadores divinos han trabajado siempre desde entonces para transmutar el temor de Dios en amor por Dios. La evolución puede ser lenta, pero es infaliblemente eficaz.

\par
%\textsuperscript{(957.3)}
\textsuperscript{86:7.7} [Presentado por una Estrella Vespertina de Nebadon.]


\chapter{Documento 87. Los cultos a los fantasmas}
\par
%\textsuperscript{(958.1)}
\textsuperscript{87:0.1} EL CULTO a los fantasmas se desarrolló como una compensación a los riesgos de la mala suerte; sus prácticas religiosas primitivas fueron el resultado de la preocupación por la mala suerte y del miedo desmesurado a los muertos. Ninguna de estas religiones primitivas tuvo mucho que ver con el reconocimiento de la Deidad ni con la veneración de lo sobrehumano; sus ritos eran principalmente negativos, destinados a evitar, expulsar o coaccionar a los fantasmas. El culto a los fantasmas no era ni más ni menos que un seguro contra los desastres; no tenía nada que ver con una inversión destinada a conseguir unos ingresos más elevados en el futuro.

\par
%\textsuperscript{(958.2)}
\textsuperscript{87:0.2} El hombre ha sostenido una larga y encarnizada lucha contra el culto a los fantasmas. No hay nada en la historia humana que despierte más compasión que esta imagen de la esclavitud abyecta del hombre al miedo a los espíritus-fantasmas. Con el nacimiento de este miedo mismo, la humanidad empezó a subir la pendiente de la evolución religiosa. La imaginación humana abandonó las orillas del yo y no volverá a echar el ancla hasta llegar al concepto de una verdadera Deidad, de un Dios real.

\section*{1. El miedo a los fantasmas}
\par
%\textsuperscript{(958.3)}
\textsuperscript{87:1.1} Se tenía miedo a la muerte porque la muerte significaba que otro fantasma se había liberado de su cuerpo físico. Los antiguos hacían todo lo que podían por impedir la muerte, por evitar el problema de tener que luchar con otro fantasma más. Siempre estaban ansiosos por inducir al fantasma a que abandonara el escenario de la defunción y emprendiera el viaje hacia el reino de los muertos. Al fantasma se le temía más que nada durante el supuesto período de transición entre su aparición en el momento de la muerte y su partida posterior hacia la tierra de los fantasmas, un concepto vago y primitivo de un supuesto cielo.

\par
%\textsuperscript{(958.4)}
\textsuperscript{87:1.2} Aunque los salvajes atribuían a los fantasmas unos poderes sobrenaturales, apenas imaginaban que tuvieran una inteligencia sobrenatural. Se practicaban muchos trucos y estratagemas en un esfuerzo por engañar y burlar a los fantasmas; el hombre civilizado deposita todavía mucha fe en la esperanza de que una manifestación exterior de piedad engañará de alguna manera a una Deidad incluso omnisciente.

\par
%\textsuperscript{(958.5)}
\textsuperscript{87:1.3} Los primitivos temían la enfermedad porque habían observado que era con frecuencia precursora de la muerte. Si el curandero de la tribu no lograba curar al afligido, normalmente sacaban al enfermo de la cabaña familiar y lo llevaban a otra más pequeña o lo dejaban al aire libre para que muriera solo. Habitualmente destruían la casa donde se había producido una defunción; si no lo hacían, siempre la esquivaban, y este miedo impidió que el hombre primitivo construyera viviendas duraderas. También obró en contra del establecimiento de pueblos y ciudades permanentes.

\par
%\textsuperscript{(958.6)}
\textsuperscript{87:1.4} Cuando un miembro del clan moría, los salvajes permanecían levantados toda la noche conversando; tenían miedo de morir también si se quedaban dormidos cerca de un cadáver. El contagio del cadáver justificaba el miedo a los muertos, y todos los pueblos, en uno u otro momento, han empleado complicadas ceremonias de purificación destinadas a limpiar a los individuos después del contacto con los muertos. Los antiguos creían que se debía suministrar luz a un cadáver; nunca se permitía que un cuerpo muerto permaneciera en la oscuridad. En el siglo veinte se siguen encendiendo cirios en las cámaras mortuorias, y los hombres continúan velando a los muertos. El hombre llamado civilizado aún no ha eliminado por completo de su filosofía de la vida el miedo a los cadáveres.

\par
%\textsuperscript{(959.1)}
\textsuperscript{87:1.5} Pero a pesar de todo este miedo, los hombres siguieron intentando engañar a los fantasmas. Si la cabaña donde alguien había muerto no era destruida, el cadáver se sacaba por un agujero en la pared, pero nunca por la puerta. Estas medidas se tomaban para confundir al fantasma, para impedir que se rezagara, y para asegurarse contra su regreso. Los dolientes también volvían del entierro por un camino diferente para que el fantasma no los siguiera. Se practicaba el caminar de espaldas y decenas de otras tácticas para asegurarse de que el fantasma no regresaría de la tumba. A menudo se intercambiaban la ropa entre los sexos con objeto de engañar al fantasma. Los vestidos de luto estaban destinados a disfrazar a los supervivientes y, más tarde, a mostrar respeto por los muertos y apaciguar así a los fantasmas.

\section*{2. El apaciguamiento de los fantasmas}
\par
%\textsuperscript{(959.2)}
\textsuperscript{87:2.1} En la religión, el programa negativo del apaciguamiento de los fantasmas precedió de lejos al programa positivo de la coacción y la súplica a los espíritus. Los primeros actos de adoración humana fueron fenómenos de defensa, no de veneración. El hombre moderno estima que es sabio asegurarse contra los incendios; el salvaje pensaba también que la mejor sabiduría consistía en asegurarse contra la mala suerte provocada por los fantasmas. Los esfuerzos por conseguir esta protección dieron forma a las técnicas y los rituales del culto a los fantasmas.

\par
%\textsuperscript{(959.3)}
\textsuperscript{87:2.2} Antiguamente se pensaba que el deseo más grande de un fantasma consistía en ser <<conjurado>> rápidamente a fin de poder dirigirse tranquilamente hacia el reino de los muertos. Cualquier error de ejecución u omisión por parte de los vivos en los actos del ritual para conjurar al fantasma, retrasaba ciertamente su marcha hacia el reino de los fantasmas. Se creía que esto desagradaba al fantasma, y se suponía que un fantasma enojado era una fuente de calamidades, desgracias e infelicidad.

\par
%\textsuperscript{(959.4)}
\textsuperscript{87:2.3} Los funerales tuvieron su origen en el esfuerzo del hombre por inducir al alma fantasmal a partir hacia su futuro hogar, y el sermón fúnebre estuvo en un principio destinado a instruir al nuevo fantasma sobre la manera de llegar hasta allí. Se tenía la costumbre de suministrar alimentos y vestidos para el viaje del fantasma, y estos artículos se colocaban dentro o cerca de la tumba. Los salvajes creían que se necesitaban de tres días a un año para <<conjurar al fantasma>> ---para apartarlo de los alrededores de la tumba. Los esquimales creen todavía que el alma permanece con el cuerpo durante tres días.

\par
%\textsuperscript{(959.5)}
\textsuperscript{87:2.4} Después de un fallecimiento se guardaba silencio o luto para que el fantasma no se sintiera atraído a regresar al hogar. Una forma corriente de luto consistía en torturarse a sí mismo ---en hacerse heridas. Muchos educadores avanzados intentaron poner fin a esta práctica, pero no lo consiguieron. Se pensaba que el ayuno y otras formas de abnegación agradaban a los fantasmas, que disfrutaban con la aflicción de los vivos durante el período de transición en que rondaban por los alrededores antes de su partida real hacia el reino de los muertos.

\par
%\textsuperscript{(959.6)}
\textsuperscript{87:2.5} Uno de los grandes obstáculos para el progreso de la civilización fueron los largos y frecuentes períodos de inactividad debidos al luto. Cada año se malgastaban semanas e incluso meses en estos lutos improductivos e inútiles. El hecho de que se contrataran plañideras profesionales para los acontecimientos fúnebres indica que el luto era un rito, no una prueba de tristeza. Los modernos tal vez lleven luto por respeto a los muertos y a causa de la pérdida sufrida, pero los antiguos lo hacían por \textit{miedo}.

\par
%\textsuperscript{(959.7)}
\textsuperscript{87:2.6} Los nombres de los muertos no se pronunciaban nunca. De hecho, a menudo se les desterraba del idioma. Estos nombres se volvían tabúes, y los idiomas se empobrecieron constantemente de esta manera. Esto produjo finalmente una multiplicación de palabras simbólicas y de expresiones figuradas tales como <<el nombre o el día que nunca se menciona>>.

\par
%\textsuperscript{(960.1)}
\textsuperscript{87:2.7} Los antiguos tenían tanta ansia por deshacerse de un fantasma que le ofrecían todo lo que hubiera podido desear durante su vida. Los fantasmas querían esposas y criados; un salvaje acaudalado esperaba que al menos una esposa esclava sería enterrada viva con él cuando muriera. Más tarde se convirtió en costumbre que la viuda se suicidara sobre la tumba de su marido. Cuando un niño moría se estrangulaba con frecuencia a la madre, una tía o la abuela para que un fantasma adulto pudiera acompañar y cuidar al fantasma infantil. Aquellos que renunciaban así a su vida lo hacían generalmente de buena gana; en verdad, si hubieran vivido violando esta costumbre, su miedo a la cólera del fantasma habría despojado su vida de los pocos placeres que podían disfrutar los primitivos.

\par
%\textsuperscript{(960.2)}
\textsuperscript{87:2.8} Se tenía la costumbre de matar a un gran número de súbditos para que acompañaran a un jefe difunto; los esclavos eran ejecutados cuando moría su amo para que pudieran servirle en el reino de los fantasmas. Los indígenas de Borneo todavía suministran un compañero que sirva de guía; se atraviesa a un esclavo con una lanza para que haga el viaje fantasmal con su amo fallecido. Se creía que a los fantasmas de las personas asesinadas les encantaba tener como esclavos a los fantasmas de sus asesinos; esta idea incitó a los hombres a convertirse en cazadores de cabezas.

\par
%\textsuperscript{(960.3)}
\textsuperscript{87:2.9} Se suponía que los fantasmas disfrutaban con el olor de la comida; las ofrendas de alimentos en los banquetes fúnebres fueron en otro tiempo universales. El método primitivo de acción de gracias consistía en arrojar al fuego un trozo de alimento, antes de comer, a fin de apaciguar a los espíritus, murmurando al mismo tiempo una fórmula mágica.

\par
%\textsuperscript{(960.4)}
\textsuperscript{87:2.10} Se creía que los muertos utilizaban los fantasmas de las herramientas y las armas que habían poseído en vida. Romper uno de estos objetos significaba <<matarlo>>, lo cual liberaba a su fantasma para que pasara a ser utilizado en el reino de los fantasmas. Los bienes también se sacrificaban, quemándolos o enterrándolos. El despilfarro en los funerales antiguos era enorme. Las razas posteriores fabricaron modelos de papel, y a las personas y los objetos reales los sustituyeron por dibujos en estos sacrificios mortuorios. La civilización realizó un gran progreso cuando la herencia destinada a los familiares reemplazó al incendio y al entierro de los bienes. Los indios iroqueses efectuaron muchas reformas en los despilfarros fúnebres. Esta conservación de la propiedad les permitió convertirse en los hombres rojos más poderosos del norte. Se supone que los hombres modernos no temen a los fantasmas, pero las costumbres son poderosas, y todavía se consumen muchas riquezas terrestres en ritos fúnebres y ceremonias mortuorias.

\section*{3. El culto a los antepasados}
\par
%\textsuperscript{(960.5)}
\textsuperscript{87:3.1} El culto progresivo a los fantasmas hizo inevitable el culto a los antepasados, pues se convirtió en el lazo de unión entre los fantasmas corrientes y los espíritus más elevados, los dioses en evolución. Los dioses primitivos eran simplemente los humanos difuntos glorificados.

\par
%\textsuperscript{(960.6)}
\textsuperscript{87:3.2} Al principio, el culto a los antepasados estaba mucho más compuesto de miedo que de adoración, pero estas creencias contribuyeron definitivamente a la propagación ulterior del miedo y la adoración a los fantasmas. Los partidarios de los cultos primitivos a los fantasmas de los antepasados tenían incluso miedo de bostezar, por temor a que un fantasma maligno aprovechara ese momento para entrar en su cuerpo.

\par
%\textsuperscript{(960.7)}
\textsuperscript{87:3.3} La costumbre de adoptar a los niños surgió para asegurarse de que alguien realizaría las ofrendas, después de la muerte, por la paz y el progreso del alma. El salvaje vivía con el miedo a los fantasmas de sus semejantes, y pasaba su tiempo libre haciendo planes para la protección de su propio fantasma después de la muerte.

\par
%\textsuperscript{(960.8)}
\textsuperscript{87:3.4} La mayoría de las tribus instituyeron una fiesta de todas las almas al menos una vez al año. Los romanos tenían cada año doce fiestas para los fantasmas, con sus ceremonias correspondientes. La mitad de los días del año estaba dedicada a algún tipo de ceremonia relacionada con estos cultos antiguos. Un emperador romano intentó reformar estas prácticas reduciendo el número de días festivos anuales a 135.

\par
%\textsuperscript{(961.1)}
\textsuperscript{87:3.5} El culto a los fantasmas evolucionó continuamente. A medida que se imaginó que los fantasmas pasaban de una fase incompleta a otra fase superior de existencia, el culto progresó finalmente hasta la adoración de los espíritus, e incluso de los dioses. Pero sin tener en cuenta las creencias variables en espíritus más avanzados, todas las tribus y razas creyeron en otro tiempo en los fantasmas.

\section*{4. Los espíritus fantasmas buenos y malos}
\par
%\textsuperscript{(961.2)}
\textsuperscript{87:4.1} El miedo a los fantasmas fue la fuente de todas las religiones del mundo; muchas tribus se aferraron durante miles de años a la vieja creencia en una sola clase de fantasmas. Enseñaban que el hombre tenía buena suerte cuando el fantasma estaba contento, y mala suerte cuando estaba enojado.

\par
%\textsuperscript{(961.3)}
\textsuperscript{87:4.2} A medida que se extendió el culto del miedo a los fantasmas, se produjo el reconocimiento de tipos superiores de espíritus, unos espíritus que no eran claramente identificables con ningún individuo humano. Se trataba de fantasmas diplomados o glorificados que habían progresado más allá del ámbito del reino de los fantasmas hasta los reinos superiores donde residen los espíritus.

\par
%\textsuperscript{(961.4)}
\textsuperscript{87:4.3} El concepto de dos tipos de espíritus fantasmas se desarrolló de manera lenta pero segura en todo el mundo. Este nuevo espiritismo doble no tuvo que extenderse de tribu en tribu; nació de forma independiente en todas partes. Para influir sobre la mente evolutiva en expansión, el poder de una idea no reside en su realidad o en su sensatez, sino más bien en su \textit{intensidad} y en su pronta y simple aplicación universal.

\par
%\textsuperscript{(961.5)}
\textsuperscript{87:4.4} Más tarde aún, la imaginación del hombre concibió el concepto de agentes sobrenaturales buenos y malos; algunos fantasmas no evolucionaban nunca hasta el nivel de los espíritus buenos. El monoespiritismo primitivo del miedo a los fantasmas evolucionó gradualmente hacia un espiritismo doble, hacia un concepto nuevo del control invisible de los asuntos terrestres. Finalmente se llegó a imaginar que la buena y la mala suerte tenían sus controladores respectivos. Y se creía que, de las dos clases, el grupo que traía la mala suerte era el más activo y numeroso.

\par
%\textsuperscript{(961.6)}
\textsuperscript{87:4.5} Cuando la doctrina de los espíritus buenos y malos maduró finalmente, se convirtió en la creencia religiosa más difundida y persistente de todas. Este dualismo representaba un gran avance filosófico-religioso porque permitía al hombre explicar tanto la buena como la mala suerte, creyendo al mismo tiempo en unos seres supermortales que tenían un comportamiento hasta cierto punto coherente. Se podía contar con que los espíritus eran buenos o malos; ya no se pensaba que fueran totalmente caprichosos como los primeros fantasmas del monoespiritismo de la mayoría de las religiones primitivas. El hombre era capaz por fin de concebir unas fuerzas supermortales que tenían un comportamiento coherente, y éste fue uno de los descubrimientos más importantes de la verdad en toda la historia de la evolución de la religión y en la expansión de la filosofía humana.

\par
%\textsuperscript{(961.7)}
\textsuperscript{87:4.6} Sin embargo, la religión evolutiva ha pagado un precio terrible por el concepto del doble espiritismo. La filosofía primitiva del hombre sólo podía conciliar la invariabilidad de los espíritus con las vicisitudes de la fortuna temporal admitiendo la existencia de dos tipos de espíritus, uno bueno y otro malo. Esta creencia permitió al hombre conciliar las variables de la casualidad con un concepto de fuerzas supermortales inmutables, pero esta doctrina siempre ha hecho difícil desde entonces que las personas religiosas puedan concebir la unidad cósmica. Los dioses de la religión evolutiva se han encontrado generalmente con la oposición de las fuerzas de las tinieblas.

\par
%\textsuperscript{(962.1)}
\textsuperscript{87:4.7} La tragedia de todo esto reside en el hecho de que cuando estas ideas echaban raíces en la mente primitiva del hombre, no había en realidad ningún espíritu malo o discordante en todo el mundo. Esta situación lamentable no se desarrolló hasta después de la rebelión de Caligastia y sólo duró hasta Pentecostés. Incluso en el siglo veinte, el concepto del bien y del mal como semejantes cósmicos permanece muy vivo en la filosofía humana; la mayor parte de las religiones del mundo llevan todavía esta marca cultural de nacimiento de los tiempos lejanos cuando surgieron los cultos a los fantasmas.

\section*{5. El culto progresivo a los fantasmas}
\par
%\textsuperscript{(962.2)}
\textsuperscript{87:5.1} El hombre primitivo consideraba que los espíritus y los fantasmas tenían unos derechos casi ilimitados, pero ningún deber; se pensaba que los espíritus estimaban que el hombre tenía numerosos deberes, pero ningún derecho. Se creía que los espíritus menospreciaban a los hombres porque éstos fracasaban constantemente en el cumplimiento de sus deberes espirituales. La humanidad creía en general que los fantasmas imponían un tributo continuo de servicio como precio a pagar por no interferir en los asuntos humanos, y la más pequeña desgracia se atribuía a las actividades de los fantasmas. Los humanos primitivos tenían tanto miedo de pasar por alto algún honor que le debieran a los dioses que, después de haber hecho sacrificios a todos los espíritus conocidos, hacían otra serie de ellos a los <<dioses desconocidos>>\footnote{\textit{Dioses desconocidos}: Hch 17:22-23.}, sólo para sentirse completamente a salvo.

\par
%\textsuperscript{(962.3)}
\textsuperscript{87:5.2} El culto simple a los fantasmas fue seguido después por las prácticas del culto más avanzado y relativamente complejo a los espíritus-fantasmas, el servicio y la adoración a los espíritus superiores tal como éstos evolucionaban en la imaginación primitiva del hombre. El ceremonial religioso tenía que seguir el mismo ritmo que la evolución y el progreso de los espíritus. Este culto ampliado no era más que el arte de la preservación de sí mismo practicado en relación con la creencia en unos seres sobrenaturales, una adaptación del yo a un entorno de espíritus. Las organizaciones industriales y militares eran adaptaciones al entorno natural y social. Y de la misma manera que el matrimonio surgió para satisfacer las exigencias de la bisexualidad, la organización religiosa se desarrolló en respuesta a la creencia en unas fuerzas y unos seres espirituales superiores. La religión representa la adaptación del hombre a sus ilusiones sobre el misterio de la casualidad. El miedo a los espíritus, y su adoración posterior, fueron adoptados como un seguro contra las desgracias, como una póliza de prosperidad.

\par
%\textsuperscript{(962.4)}
\textsuperscript{87:5.3} Los salvajes imaginan que los espíritus buenos se dedican a sus asuntos, y que exigen pocas cosas a los seres humanos. Los fantasmas y los espíritus malos son los que hay que mantener de buen humor. En consecuencia, los pueblos primitivos prestaban más atención a sus fantasmas malévolos que a sus espíritus benévolos.

\par
%\textsuperscript{(962.5)}
\textsuperscript{87:5.4} Se suponía que la prosperidad humana provocaba especialmente la envidia de los espíritus malignos, y que su método de represalias consistía en devolver el golpe a través de un agente humano y mediante la técnica del \textit{mal de ojo}\footnote{\textit{Mal de ojo}: Pr 23:6; 28:22; Dt 15:9; 28:54,56; Mc 7:22.}. Esta fase del culto consistente en evitar a los espíritus se preocupaba mucho por las maquinaciones del mal de ojo, y el miedo al mal de ojo se volvió casi mundial. A las mujeres bonitas se las cubría con un velo para protegerlas contra el mal de ojo; posteriormente, muchas mujeres que deseaban ser consideradas como hermosas adoptaron esta práctica. Debido a este miedo a los malos espíritus, a los niños raramente se les permitía salir al exterior después del anochecer, y las oraciones primitivas siempre incluían la súplica: <<líbranos del mal de ojo>>\footnote{\textit{Líbranos del mal de ojo}: Mt 6:13; Lc 11:4.}.

\par
%\textsuperscript{(962.6)}
\textsuperscript{87:5.5} El Corán contiene un capítulo entero dedicado al mal de ojo y a los sortilegios mágicos, y los judíos creían plenamente en ellos. Todo el culto fálico se desarrolló como una protección contra el mal de ojo. Se creía que los órganos de la reproducción eran el único fetiche que podía volverlo ineficaz. El mal de ojo dio origen a las primeras supersticiones sobre las marcas prenatales de los niños, las señales maternas, y este culto fue en cierto momento casi universal.

\par
%\textsuperscript{(963.1)}
\textsuperscript{87:5.6} La envidia es una característica humana profundamente arraigada; por eso los hombres primitivos la atribuyeron a sus dioses iniciales. Puesto que el hombre ya había practicado el engaño con los fantasmas, pronto empezó a engañar a los espíritus. Se dijo a sí mismo: <<Si los espíritus están celosos de nuestra belleza y prosperidad, nos afearemos y hablaremos a la ligera de nuestros éxitos.>> La humildad primitiva no era pues una degradación del ego, sino más bien un intento por frustrar y engañar a los espíritus envidiosos.

\par
%\textsuperscript{(963.2)}
\textsuperscript{87:5.7} Para impedir que los espíritus se sintieran celosos de la prosperidad humana, se adoptó el método de llenar de injurias a una cosa o persona afortunada o muy amada. La costumbre de menospreciar los comentarios halagadores sobre uno mismo o su familia se originó de esta manera, y con el tiempo se transformó en la modestia, la moderación y la cortesía civilizadas. Por el mismo motivo se puso de moda parecer feo. La belleza despertaba la envidia de los espíritus; denotaba un orgullo humano pecaminoso. El salvaje trataba de encontrar un nombre feo. Esta característica del culto obstaculizó enormemente el progreso de las artes, y mantuvo al mundo durante mucho tiempo sombrío y feo.

\par
%\textsuperscript{(963.3)}
\textsuperscript{87:5.8} Durante la época del culto a los espíritus, la vida era como mucho una lotería, el resultado del control de los espíritus. El futuro de una persona no dependía de sus esfuerzos, su laboriosidad o su talento, salvo que pudiera utilizarlos para influir sobre los espíritus. Las ceremonias de propiciación de los espíritus constituyeron una carga pesada e hicieron la vida tediosa y prácticamente insoportable. De época en época y de generación en generación, las razas han intentado mejorar, unas tras otras, esta doctrina de los superfantasmas, pero ninguna generación se ha atrevido todavía a rechazarla por completo.

\par
%\textsuperscript{(963.4)}
\textsuperscript{87:5.9} La intención y la voluntad de los espíritus se estudiaban por medio de los presagios, los oráculos y los signos. Estos mensajes de los espíritus se interpretaban mediante la adivinación, las predicciones, la magia, las ordalías y la astrología. Todo el culto era un programa destinado a apaciguar, satisfacer y comprar a los espíritus mediante este soborno disfrazado.

\par
%\textsuperscript{(963.5)}
\textsuperscript{87:5.10} Así es como nació una visión del mundo nueva y más amplia que consistía en:

\par
%\textsuperscript{(963.6)}
\textsuperscript{87:5.11} 1. \textit{El deber} ---las cosas que se deben hacer para mantener a los espíritus en una disposición favorable, o al menos neutral.

\par
%\textsuperscript{(963.7)}
\textsuperscript{87:5.12} 2. \textit{El derecho} ---la conducta y las ceremonias correctas destinadas a poner activamente a los espíritus a favor de nuestros intereses personales.

\par
%\textsuperscript{(963.8)}
\textsuperscript{87:5.13} 3. \textit{La verdad} ---la comprensión exacta de los espíritus y la actitud correcta hacia ellos, y en consecuencia, hacia la vida y la muerte.

\par
%\textsuperscript{(963.9)}
\textsuperscript{87:5.14} Los antiguos no trataban de conocer el futuro simplemente por curiosidad; querían esquivar la mala suerte. La adivinación era simplemente un intento por evitar las dificultades. En aquellos tiempos los sueños se consideraban como proféticos, y todo lo que se salía de lo normal era estimado como un presagio. Incluso hoy en día, las razas civilizadas están aquejadas de la creencia en los signos, las señales y otros vestigios supersticiosos del antiguo culto progresivo a los fantasmas. El hombre es lento, muy lento en abandonar aquellos métodos que le sirvieron para ascender de manera tan penosa y gradual por la escala evolutiva de la vida.

\section*{6. La coacción y el exorcismo}
\par
%\textsuperscript{(963.10)}
\textsuperscript{87:6.1} Cuando los hombres sólo creían en los fantasmas, el ritual religioso era más personal, menos organizado, pero el reconocimiento de unos espíritus más elevados necesitó el empleo de unos <<métodos espirituales superiores>> para relacionarse con ellos. Esta tentativa por mejorar y ampliar la técnica de la propiciación de los espíritus condujo directamente a la creación de unas defensas contra los espíritus. En verdad, el hombre se sentía impotente ante las fuerzas incontrolables que actuaban en la vida terrestre, y su sentimiento de inferioridad le llevó a intentar encontrar alguna adaptación compensatoria, alguna técnica para nivelar las probabilidades en esta lucha unilateral del hombre contra el cosmos.

\par
%\textsuperscript{(964.1)}
\textsuperscript{87:6.2} En los primeros tiempos del culto, los esfuerzos del hombre por influir sobre la actividad de los fantasmas se limitaban a la propiciación, a los intentos de soborno para librarse de la mala suerte. A medida que la evolución del culto a los fantasmas progresó hasta el concepto de los espíritus tanto buenos como malos, estas ceremonias se transformaron en tentativas de naturaleza más positiva, en esfuerzos por atraer la buena suerte. La religión del hombre ya no era completamente negativa, ni el hombre tampoco se detuvo en sus esfuerzos por conseguir la buena suerte; poco después empezó a idear proyectos para forzar a los espíritus a cooperar. Las personas religiosas ya no están indefensas ante las exigencias incesantes de los fantasmas espíritus imaginados por ellas mismas; el salvaje empieza a inventar armas para obligar a los espíritus a actuar y forzarlos a que le ayuden.

\par
%\textsuperscript{(964.2)}
\textsuperscript{87:6.3} Los primeros esfuerzos defensivos del hombre estuvieron dirigidos contra los fantasmas. A medida que pasaron los siglos, los vivos empezaron a inventar métodos para oponer resistencia a los muertos. Se desarrollaron muchas técnicas para asustar y alejar a los fantasmas, entre las cuales se pueden citar las siguientes:

\par
%\textsuperscript{(964.3)}
\textsuperscript{87:6.4} 1. Cortar la cabeza y atar el cuerpo en la tumba.

\par
%\textsuperscript{(964.4)}
\textsuperscript{87:6.5} 2. Apedrear la casa donde se había producido la defunción.

\par
%\textsuperscript{(964.5)}
\textsuperscript{87:6.6} 3. Castrar el cadáver o quebrarle las piernas.

\par
%\textsuperscript{(964.6)}
\textsuperscript{87:6.7} 4. Enterrarlo debajo de las piedras, uno de los orígenes de las lápidas sepulcrales modernas.

\par
%\textsuperscript{(964.7)}
\textsuperscript{87:6.8} 5. Incinerarlo, un invento más tardío para impedir los problemas causados por los fantasmas.

\par
%\textsuperscript{(964.8)}
\textsuperscript{87:6.9} 6. Arrojar el cuerpo al mar.

\par
%\textsuperscript{(964.9)}
\textsuperscript{87:6.10} 7. Dejar el cuerpo al descubierto para que se lo comieran los animales salvajes.

\par
%\textsuperscript{(964.10)}
\textsuperscript{87:6.11} Se suponía que a los fantasmas les molestaba y asustaba el ruido, que los gritos, las campanas y los tambores los alejaban de los vivos; estos métodos antiguos están todavía de moda en los <<velatorios>> de los muertos. Se utilizaban mezclas nauseabundas para ahuyentar a los espíritus inoportunos. Se construían imágenes espantosas de los espíritus para que éstos huyeran apresuradamente cuando se contemplaran a sí mismos. Se creía que los perros podían detectar la proximidad de los fantasmas, y que lo avisaban mediante aullidos; que los gallos solían cantar cuando los fantasmas estaban cerca. El empleo del gallo como veleta es una perpetuación de esta superstición.

\par
%\textsuperscript{(964.11)}
\textsuperscript{87:6.12} El agua se consideraba como la mejor protección contra los fantasmas. El agua bendita era superior a todas las demás; era el agua donde los sacerdotes se habían lavado los pies. Se creía que tanto el fuego como el agua constituían unas barreras infranqueables para los fantasmas. Los romanos daban tres vueltas con agua alrededor de un cadáver; en el siglo veinte, los cadáveres se rocían con agua bendita, y los judíos conservan todavía el ritual de lavarse las manos en el cementerio. El bautismo fue una característica del ritual posterior del agua\footnote{\textit{Bautismo con agua}: Mt 3:6,11; Mc 1:4-5,8; Lc 3:3,7,16; Jn 1:25-26,31,33; 3:22-23; Hch 1:5.}. Los baños primitivos eran una ceremonia religiosa. El baño sólo se ha convertido en una práctica higiénica en los tiempos recientes.

\par
%\textsuperscript{(964.12)}
\textsuperscript{87:6.13} Pero el hombre no se contentó con coaccionar a los fantasmas; pronto intentó forzar a los espíritus a actuar mediante los rituales religiosos y otras prácticas. El exorcismo consistía en emplear un espíritu para que controlara o desterrara a otro, y estas tácticas se utilizaron también para asustar a los fantasmas y los espíritus. El concepto de las fuerzas buenas y malas, contenido en el doble espiritismo, ofreció al hombre amplias ocasiones para intentar oponer un agente a otro, porque si un hombre fuerte podía vencer a uno más débil, entonces un espíritu poderoso podía dominar sin duda a un fantasma inferior. Las maldiciones primitivas eran una práctica coercitiva destinada a intimidar a los espíritus menores. Más tarde, esta costumbre se utilizó como base para proferir maldiciones contra los enemigos.

\par
%\textsuperscript{(965.1)}
\textsuperscript{87:6.14} Durante mucho tiempo se creyó que a los espíritus y semidioses se les podía forzar a actuar de manera deseable si se volvía a los usos de las costumbres más antiguas. El hombre moderno es culpable de emplear el mismo procedimiento. Os dirigís los unos a los otros en el lenguaje corriente de todos los días, pero cuando os ponéis a rezar, recurrís al estilo anticuado de otra generación, al estilo llamado solemne.

\par
%\textsuperscript{(965.2)}
\textsuperscript{87:6.15} Esta doctrina explica también muchas reversiones religioso-rituales de naturaleza sexual, tales como la prostitución en los templos. Estas reversiones a las costumbres primitivas se consideraban como protecciones seguras contra muchas calamidades. Entre estos pueblos sencillos, todas estas actuaciones estaban totalmente libres de lo que el hombre moderno podría llamar promiscuidad.

\par
%\textsuperscript{(965.3)}
\textsuperscript{87:6.16} Luego surgió la costumbre de los votos rituales, seguida poco después de los compromisos religiosos y los juramentos sagrados\footnote{\textit{Votos sagrados y juramentos}: Gn 31:13; Ex 22:11; 2 Cr 15:12-15; Lv 5:4; Sal 105:9-10; Nm 6:2-15; 30:2-16; Dt 29:12-15.}. Casi todos estos juramentos iban acompañados de torturas y mutilaciones que se infligían a sí mismos\footnote{\textit{Penas por violar juramentos}: Jue 11:30-39.}, y más tarde aún, de ayunos y oraciones. La abnegación fue considerada posteriormente como un método coercitivo seguro; esto era especialmente cierto en materia de continencia sexual. Así es como el hombre primitivo desarrolló pronto una austeridad resuelta en sus prácticas religiosas, una creencia en la eficacia de la tortura de sí mismo y la abnegación como ritos capaces de forzar a los espíritus mal dispuestos a reaccionar favorablemente ante todos estos sufrimientos y privaciones.

\par
%\textsuperscript{(965.4)}
\textsuperscript{87:6.17} El hombre moderno ya no intenta coaccionar abiertamente a los espíritus, aunque todavía manifiesta cierta predisposición a negociar con la Deidad. Y continúa blasfemando, tocando madera, cruzando los dedos y diciendo una frase trivial después de una expectoración; en otro tiempo era una fórmula mágica.

\section*{7. La naturaleza del culto}
\par
%\textsuperscript{(965.5)}
\textsuperscript{87:7.1} La organización social de tipo cultual perduró porque proporcionaba un simbolismo que preservaba y estimulaba los sentimientos morales y las lealtades religiosas. El culto tuvo su origen en las tradiciones de las <<antiguas familias>> y se perpetuó como institución establecida; todas las familias tienen un culto de algún tipo. Todo ideal inspirador se apodera de algún simbolismo que lo perpetúe ---busca alguna técnica de manifestación cultural que asegure su supervivencia y aumente su desarrollo--- y el culto consigue esta finalidad mediante el fomento y la satisfacción de las emociones.

\par
%\textsuperscript{(965.6)}
\textsuperscript{87:7.2} Desde los albores de la civilización, todo movimiento atractivo de cultura social o de progreso religioso ha desarrollado un ritual, un ceremonial simbólico. Cuanto más inconsciente ha sido el crecimiento de este ritual, más intensamente ha cautivado a sus adeptos. El culto preservaba los sentimientos y satisfacía las emociones, pero siempre ha sido el mayor obstáculo para la reconstrucción social y el progreso espiritual.

\par
%\textsuperscript{(965.7)}
\textsuperscript{87:7.3} A pesar de que el culto siempre ha retrasado el progreso social, es lamentable que tantos creyentes modernos en las normas morales y en los ideales espirituales no posean un simbolismo adecuado ---un culto donde apoyarse mutuamente--- nada a lo que puedan \textit{pertenecer}. Pero un culto religioso no se puede fabricar; tiene que crecer. Y los cultos de dos grupos distintos nunca serán idénticos, a menos que sus rituales sean uniformados arbitrariamente por alguna autoridad.

\par
%\textsuperscript{(965.8)}
\textsuperscript{87:7.4} El culto cristiano primitivo era el más eficaz, atractivo y duradero de todos los rituales que se hayan concebido o inventado jamás, pero una gran parte de su valor ha sido aniquilada en la era científica mediante la destrucción de muchos de sus principios originales subyacentes. El culto cristiano se ha debilitado debido a la pérdida de muchas ideas fundamentales.

\par
%\textsuperscript{(965.9)}
\textsuperscript{87:7.5} En el pasado, la verdad ha crecido rápidamente y se ha extendido con libertad cuando el culto ha sido flexible, y el simbolismo expansible. Una verdad abundante y un culto adaptable han favorecido la rapidez del progreso social. Un culto sin sentido vicia la religión cuando intenta suplantar la filosofía y esclavizar la razón; un culto auténtico crece.

\par
%\textsuperscript{(966.1)}
\textsuperscript{87:7.6} A pesar de los inconvenientes y las desventajas, cada nueva revelación de la verdad ha dado nacimiento a un nuevo culto, e incluso la nueva exposición de la religión de Jesús debe desarrollar un simbolismo nuevo y apropiado. El hombre moderno debe encontrar un simbolismo adecuado para sus nuevos ideales, ideas y lealtades en expansión. Este símbolo realzado debe surgir de la vida religiosa, de la experiencia espiritual. Este simbolismo superior de una civilización más elevada debe estar basado en el concepto de la Paternidad de Dios y estar cargado del poderoso ideal de la fraternidad de los hombres.

\par
%\textsuperscript{(966.2)}
\textsuperscript{87:7.7} Los antiguos cultos eran demasiado egocéntricos; el nuevo culto debe ser la consecuencia del amor aplicado. Al igual que los antiguos, el nuevo culto debe favorecer los sentimientos, satisfacer las emociones y promover la lealtad; pero debe hacer algo más: Debe facilitar el progreso espiritual, realzar los significados cósmicos, aumentar los valores morales, animar el desarrollo social y estimular un tipo elevado de vida religiosa personal. El nuevo culto debe proporcionar unos objetivos supremos de vida que sean temporales y eternos a la vez ---sociales y espirituales.

\par
%\textsuperscript{(966.3)}
\textsuperscript{87:7.8} Ningún culto puede durar ni contribuir al progreso de la civilización social y a la consecución espiritual individual a menos que esté basado en la importancia biológica, sociológica y religiosa del \textit{hogar}. Un culto que sobrevive debe simbolizar aquello que es permanente en presencia del cambio continuo; debe glorificar aquello que unifica la corriente de las metamorfosis sociales en constante cambio. Debe reconocer los verdaderos significados, ensalzar las relaciones hermosas y alabar los valores buenos de la auténtica nobleza.

\par
%\textsuperscript{(966.4)}
\textsuperscript{87:7.9} Pero la gran dificultad que existe para encontrar un simbolismo nuevo y satisfactorio reside en que los hombres modernos, como grupo, se adhieren a la actitud científica, evitan las supersticiones y aborrecen la ignorancia, mientras que como individuos, todos ansían el misterio y veneran lo desconocido. Ningún culto puede sobrevivir a menos que incorpore un misterio dominante y oculte una meta inaccesible digna de alcanzarse. Además, el nuevo simbolismo no sólo debe ser significativo para el grupo, sino que también debe tener sentido para el individuo. Las formas de cualquier simbolismo útil deben ser aquellas que el individuo pueda llevar a cabo por su propia iniciativa, y que también pueda disfrutar con sus semejantes. Si el nuevo culto pudiera ser dinámico en lugar de estático, podría efectuar una contribución realmente valiosa al progreso tanto temporal como espiritual de la humanidad.

\par
%\textsuperscript{(966.5)}
\textsuperscript{87:7.10} Pero un culto ---un simbolismo de ritos, lemas u objetivos--- no funcionará si es demasiado complejo. Y debe estar presente la exigencia de la devoción, la respuesta de la lealtad. Toda religión eficaz desarrolla infaliblemente un simbolismo valioso, y sus partidarios harían bien en impedir que ese ritual se cristalice en ceremoniales estereotipados obstaculizadores, deformantes y sofocantes, que lo único que pueden hacer es perjudicar y retrasar todo progreso social, moral y espiritual. No existe un culto que pueda sobrevivir si retrasa el crecimiento moral y no logra fomentar el progreso espiritual. El culto es la estructura esquelética alrededor de la cual crece el cuerpo vivo y dinámico de la experiencia espiritual personal ---la verdadera religión.

\par
%\textsuperscript{(966.6)}
\textsuperscript{87:7.11} [Presentado por una Brillante Estrella Vespertina de Nebadon.]


\chapter{Documento 88. Fetiches, amuletos y magia}
\par
%\textsuperscript{(967.1)}
\textsuperscript{88:0.1} EL CONCEPTO de la introducción de un espíritu en un objeto inanimado, un animal o un ser humano, es una creencia muy antigua y respetable que ha prevalecido desde el comienzo de la evolución de la religión. Esta doctrina de la posesión por los espíritus no es más ni menos que el \textit{fetichismo}. El salvaje no adora necesariamente al fetiche; adora y venera con mucha lógica al espíritu que reside en el fetiche.

\par
%\textsuperscript{(967.2)}
\textsuperscript{88:0.2} Al principio se creía que el espíritu de un fetiche era el fantasma de un hombre fallecido; más tarde se supuso que los espíritus superiores residían en los fetiches. El culto a los fetiches terminó así por incorporar todas las ideas primitivas sobre los fantasmas, las almas, los espíritus y la posesión demoníaca.

\section*{1. La creencia en los fetiches}
\par
%\textsuperscript{(967.3)}
\textsuperscript{88:1.1} El hombre primitivo necesitaba siempre convertir todas las cosas extraordinarias en un fetiche; la casualidad dio origen pues a muchos fetiches. Un hombre está enfermo, sucede algo, y recupera la salud. Esto mismo ocurre también con la fama de numerosos medicamentos y con los métodos casuales para tratar las enfermedades. Los objetos que aparecían en los sueños tenían la posibilidad de ser convertidos en fetiches. Los volcanes, pero no las montañas, los cometas, pero no las estrellas, se volvieron fetiches. El hombre primitivo consideraba que las estrellas fugaces y los meteoros indicaban la llegada a la Tierra de unos espíritus visitantes especiales.

\par
%\textsuperscript{(967.4)}
\textsuperscript{88:1.2} Los primeros fetiches fueron los guijarros que tenían unas marcas peculiares, y el hombre ha buscado siempre desde entonces las <<piedras sagradas>>; un collar era en otro tiempo una colección de piedras sagradas, una serie de amuletos. Muchas tribus tenían piedras fetiches, pero pocas han sobrevivido como la Caaba y la Piedra de Scone. El fuego y el agua figuraron también entre los primeros fetiches, y la adoración del fuego, así como la creencia en el agua bendita, sobreviven todavía.

\par
%\textsuperscript{(967.5)}
\textsuperscript{88:1.3} Los árboles fetiches aparecieron más tarde, pero en algunas tribus, la persistencia de la adoración de la naturaleza condujo a la creencia en los amuletos habitados por algún tipo de espíritu de la naturaleza. Cuando las plantas y las frutas se convertían en fetiches, eran tabúes como alimento. La manzana fue una de las primeras en caer en esta categoría; los pueblos levantinos no la comían jamás.

\par
%\textsuperscript{(967.6)}
\textsuperscript{88:1.4} Si un animal comía carne humana, se volvía un fetiche. El perro se convirtió de esta manera en el animal sagrado de los parsis. Si el fetiche es un animal y el fantasma reside en él de manera permanente, el fetichismo puede rayar en la reencarnación. Los salvajes envidiaban a los animales en muchos aspectos; no se sentían superiores a ellos y a menudo se ponían el nombre de sus bestias favoritas.

\par
%\textsuperscript{(967.7)}
\textsuperscript{88:1.5} Cuando los animales se volvieron fetiches, surgieron los tabúes sobre la consumición de la carne de dichos animales. A causa de su parecido con el hombre, los monos y los simios se volvieron pronto animales fetiches; más tarde, las serpientes, los pájaros y los cerdos fueron considerados también de la misma manera. La vaca fue un fetiche en cierta época, y su leche era tabú mientras que sus excrementos eran muy apreciados. La serpiente era venerada en Palestina\footnote{\textit{Veneración de las serpientes}: Ex 4:2-4; 7:9-12; 2 Re 18:4; Nm 21:8-9; (Da 14:23-27): Bel 1:23-27; Jn 3:14.}, especialmente por los fenicios que, junto con los judíos, la consideraban como el portavoz de los espíritus malignos\footnote{\textit{Serpientes, portavoces de los espíritus malignos}: Gn 3:1-5.}. Muchas personas modernas creen aún en los poderes mágicos de los reptiles. La serpiente ha sido venerada desde Arabia, pasando por la India, hasta la danza de la serpiente de la tribu moqui de los hombres rojos.

\par
%\textsuperscript{(968.1)}
\textsuperscript{88:1.6} Ciertos días de la semana eran fetiches. El viernes ha sido considerado durante miles de años como el día de la mala suerte, y el número trece como nefasto. Los números afortunados tres y siete procedían de revelaciones posteriores; el cuatro era el número de la suerte de los hombres primitivos y tuvo su origen en el reconocimiento temprano de los cuatro puntos cardinales. Se consideraba que el hecho de contar el ganado u otras posesiones traía mala suerte; los antiguos siempre se opusieron al empadronamiento, a <<contar al pueblo>>\footnote{\textit{Contar al pueblo}: Ex 30:12-16; 1 Cr 21:1-4; 1 Cr 27:23-24; 2 Sam 24:1-4.}.

\par
%\textsuperscript{(968.2)}
\textsuperscript{88:1.7} El hombre primitivo no hizo del sexo un fetiche indebido; la función reproductora sólo recibió una atención limitada. El salvaje tenía una mentalidad natural, que no era ni obscena ni lasciva.

\par
%\textsuperscript{(968.3)}
\textsuperscript{88:1.8} La saliva era un poderoso fetiche; se podían expulsar los demonios de una persona escupiendo sobre ella. El mayor cumplido que alguien podía recibir era que un anciano o un superior escupiera sobre él. Algunas partes del cuerpo humano fueron consideradas como fetiches potenciales, en particular el cabello y las uñas. Las largas uñas de los jefes eran muy apreciadas, y sus recortes constituían unos poderosos fetiches. La creencia en las calaveras como fetiches explica una gran parte de la actividad posterior de los cazadores de cabezas. El cordón umbilical era un fetiche muy apreciado, y así es como se considera en África incluso en la actualidad. El primer juguete de la humanidad fue un cordón umbilical conservado. Adornado con perlas, tal como se hacía a menudo, fue el primer collar que tuvo el hombre.

\par
%\textsuperscript{(968.4)}
\textsuperscript{88:1.9} Los niños jorobados y tullidos eran considerados como fetiches; se creía que los locos estaban influidos por la Luna. El hombre primitivo no podía distinguir entre el genio y la locura; a los tontos los golpeaban hasta morir o eran venerados como personalidades fetiches. La histeria confirmó cada vez más la creencia popular en la brujería; los epilépticos eran con frecuencia sacerdotes y curanderos. La embriaguez se consideraba como una forma de posesión por los espíritus; cuando un salvaje se iba de juerga, se colocaba una hoja en el pelo con el fin de negarse a aceptar la responsabilidad de sus actos. Los venenos y las bebidas alcohólicas se volvieron fetiches; se suponía que estaban poseídos.

\par
%\textsuperscript{(968.5)}
\textsuperscript{88:1.10} Mucha gente consideraba que los genios eran personalidades fetiches poseídas por un espíritu sabio. Estos humanos talentosos aprendieron pronto a recurrir al fraude y al engaño para promover sus intereses egoístas. Se creía que un hombre fetiche era más que humano; era divino, e incluso infalible. Así es como los jefes, reyes, sacerdotes, profetas y dirigentes de la iglesia consiguieron finalmente un gran poder y ejercieron una autoridad ilimitada.

\section*{2. La evolución de los fetiches}
\par
%\textsuperscript{(968.6)}
\textsuperscript{88:2.1} Se suponía que los fantasmas preferían residir en un objeto que les había pertenecido cuando vivían en la carne. Esta creencia explica la eficacia de muchas reliquias modernas. Los antiguos siempre veneraban los huesos de sus dirigentes, y muchas personas contemplan todavía los restos óseos de los santos y los héroes con un temor supersticioso. Incluso hoy en día se hacen peregrinajes a la tumba de los grandes hombres.

\par
%\textsuperscript{(968.7)}
\textsuperscript{88:2.2} La creencia en las reliquias es una consecuencia del antiguo culto a los fetiches. Las reliquias de las religiones modernas representan una tentativa por racionalizar los fetiches de los salvajes, y elevarlos así a una posición de dignidad y respetabilidad en los sistemas religiosos modernos. La creencia en los fetiches y en la magia se considera como pagana, pero se supone que es muy correcto aceptar las reliquias y los milagros.

\par
%\textsuperscript{(969.1)}
\textsuperscript{88:2.3} El hogar ---el sitio donde estaba el fuego--- se convirtió más o menos en un fetiche, en un lugar sagrado. Los santuarios y los templos fueron al principio unos lugares fetiches porque los muertos eran enterrados allí. La cabaña fetiche de los hebreos fue elevada por Moisés a la posición de albergar un superfetiche\footnote{\textit{Cabaña fetiche elevada por Moisés}: Ex 25:8-10; 33:7-10; 40:1-8; 1 Re 5:2-5.}, el concepto entonces existente de la ley de Dios. Pero los israelitas no abandonaron nunca la creencia peculiar de los cananeos en los altares de piedra: <<Y esta piedra que he levantado como un pilar será la casa de Dios>>\footnote{\textit{Piedra levantada como un pilar}: Gn 28:22.}. Creían sinceramente que el espíritu de su Dios residía en estos altares de piedra, que en realidad eran fetiches.

\par
%\textsuperscript{(969.2)}
\textsuperscript{88:2.4} Las primeras imágenes se hicieron para conservar la apariencia y la memoria de los muertos ilustres; en realidad eran monumentos. Los ídolos fueron un refinamiento del fetichismo. Los primitivos creían que una ceremonia de consagración hacía que el espíritu entrara en la imagen; del mismo modo, cuando se bendecían ciertos objetos, éstos se volvían amuletos.

\par
%\textsuperscript{(969.3)}
\textsuperscript{88:2.5} Cuando Moisés añadió el segundo mandamiento al antiguo código moral de Dalamatia, lo hizo en un esfuerzo por controlar la adoración de los fetiches entre los hebreos\footnote{\textit{Nuevos fetiches}: Ex 25:10-16; 37:1-9; 40:20; 1 Re 8:9; 2 Re 18:4; Dt 10:1-5; Heb 9:4.}. Les ordenó cuidadosamente que no debían hacer ningún tipo de imágenes que pudieran ser consagradas como fetiches. Les indicó claramente: <<No haréis imágenes talladas ni ningún retrato de lo que se encuentra arriba en el cielo, ni abajo en la Tierra, ni en las aguas de la Tierra>>\footnote{\textit{Ninguna imagen tallada de nada}: Ex 20:4; 34:17; Lv 26:1; Dt 5:8.}. Aunque este mandamiento contribuyó mucho a retrasar el arte entre los judíos, redujo la adoración de los fetiches. Pero Moisés era demasiado sabio como para intentar desplazar repentinamente los antiguos fetiches, y consintió pues en colocar ciertas reliquias al lado de la ley en el arca, que era una mezcla de altar de guerra y santuario religioso.

\par
%\textsuperscript{(969.4)}
\textsuperscript{88:2.6} Las palabras se volvieron finalmente fetiches, en particular aquellas que se consideraban como las palabras de Dios; los libros sagrados de muchas religiones se han convertido de esta manera en prisiones fetichistas que encarcelan la imaginación espiritual de los hombres. El esfuerzo mismo de Moisés contra los fetiches se convirtió en un supremo fetiche; su mandamiento fue utilizado más tarde para aniquilar el arte y retrasar el disfrute y la adoración de lo hermoso.

\par
%\textsuperscript{(969.5)}
\textsuperscript{88:2.7} En los tiempos antiguos, la palabra de autoridad fetiche era una \textit{doctrina} que inspiraba temor, el más terrible de todos los tiranos que esclavizan al hombre. Un fetiche doctrinal conducirá al hombre mortal a echarse en las garras de la mojigatería, el fanatismo, la superstición, la intolerancia y las crueldades bárbaras más atroces. El respeto moderno por la sabiduría y la verdad no es más que una huida reciente de la tendencia a crear fetiches hacia unos niveles más elevados de pensamiento y razonamiento. En lo que concierne a los escritos fetiches acumulados que diversos practicantes de la religión consideran como \textit{libros sagrados}\footnote{\textit{Libros sagrados como fetiches}: Is 11:12; Ap 7:1.}, no solamente creen que lo que figura en el libro es verdad, sino que el libro contiene toda la verdad. Si uno de estos libros sagrados dice por casualidad que la Tierra es plana, entonces, durante largas generaciones, los hombres y las mujeres por otra parte sensatos se negarán a aceptar las pruebas positivas de que el planeta es redondo.

\par
%\textsuperscript{(969.6)}
\textsuperscript{88:2.8} La costumbre de abrir uno de estos libros sagrados para leer un pasaje al azar cuya puesta en práctica podría condicionar importantes decisiones o proyectos de vida, no es ni más ni menos que un fetichismo redomado. Prestar juramento sobre <<un libro sagrado>>, o jurar por algún objeto de veneración suprema, es una forma de fetichismo refinado.

\par
%\textsuperscript{(969.7)}
\textsuperscript{88:2.9} En cambio, sí representa un progreso evolutivo real pasar del miedo fetichista de los recortes de uñas de un jefe salvaje a la adoración de una espléndida colección de cartas, leyes, leyendas, alegorías, mitos, poemas y crónicas que, después de todo, reflejan la sabiduría moral cribada de muchos siglos, al menos hasta el día y la hora en que fueron reunidos en un <<libro sagrado>>.

\par
%\textsuperscript{(970.1)}
\textsuperscript{88:2.10} Para volverse fetiches, las palabras tenían que ser consideradas como inspiradas, y la invocación de unos escritos supuestamente inspirados por la divinidad condujo directamente al establecimiento de la \textit{autoridad} de la iglesia, mientras que la evolución de las formas civiles condujo a la instauración de la \textit{autoridad} del Estado.

\section*{3. El totemismo}
\par
%\textsuperscript{(970.2)}
\textsuperscript{88:3.1} El fetichismo estuvo presente en todos los cultos primitivos, desde las primeras creencias en las piedras sagradas, pasando por la idolatría, el canibalismo y la adoración de la naturaleza, hasta el totemismo.

\par
%\textsuperscript{(970.3)}
\textsuperscript{88:3.2} El totemismo es una combinación de prácticas sociales y religiosas. Al principio se creía que respetar al animal totémico, que era el supuesto progenitor biológico de la tribu, aseguraba la provisión de alimentos. Los tótemes eran al mismo tiempo los símbolos de los grupos y su dios. Dicho dios era el clan personificado. El totemismo fue una fase de la tentativa por socializar la religión que, por lo demás, es personal. El tótem evolucionó con el tiempo hasta convertirse en la bandera, o símbolo nacional de los diversos pueblos modernos.

\par
%\textsuperscript{(970.4)}
\textsuperscript{88:3.3} Una bolsa fetiche, una bolsa de medicinas, era un saquito que contenía un acreditado surtido de artículos impregnados por los fantasmas, y el curandero de la antig\"uedad nunca permitía que su bolsa, el símbolo de su poder, tocara el suelo. Los pueblos civilizados del siglo veinte procuran que sus banderas, emblemas de la conciencia nacional, tampoco toquen nunca el suelo.

\par
%\textsuperscript{(970.5)}
\textsuperscript{88:3.4} Las insignias de los cargos sacerdotales y reales fueron consideradas finalmente como fetiches, y el fetiche del Estado supremo ha pasado por muchas fases de desarrollo: de los clanes a las tribus, del señorío feudal a la soberanía, de los tótemes a las banderas. Los reyes fetiches han gobernado por <<derecho divino>>, y han prevalecido otras muchas formas de gobierno. Los hombres han hecho también un fetiche de la democracia, la exaltación y adoración de las ideas del hombre de la calle, cuando son calificadas colectivamente de <<opinión pública>>. La opinión aislada de un hombre solo no se considera que tenga mucho valor, pero cuando muchos hombres actúan colectivamente en democracia, este mismo juicio mediocre es considerado como el árbitro de la justicia y el modelo de la rectitud.

\section*{4. La magia}
\par
%\textsuperscript{(970.6)}
\textsuperscript{88:4.1} El hombre civilizado se enfrenta a los problemas de un entorno real a través de su ciencia; el hombre salvaje intentaba resolver los problemas reales de un entorno ilusorio de fantasmas por medio de la magia. La magia era una técnica para manipular el entorno imaginario de espíritus cuyas maquinaciones explicaban interminablemente lo inexplicable; era el arte de obtener la cooperación voluntaria de los espíritus y de forzarlos a ofrecer su ayuda involuntaria mediante la utilización de los fetiches u otros espíritus más poderosos.

\par
%\textsuperscript{(970.7)}
\textsuperscript{88:4.2} El objetivo de la magia, la brujería y la nigromancia era doble:

\par
%\textsuperscript{(970.8)}
\textsuperscript{88:4.3} 1. Obtener un atisbo del futuro.

\par
%\textsuperscript{(970.9)}
\textsuperscript{88:4.4} 2. Influir favorablemente sobre el entorno.

\par
%\textsuperscript{(970.10)}
\textsuperscript{88:4.5} Las metas de la ciencia son idénticas a las de la magia. La humanidad progresa de la magia a la ciencia, no por medio de la meditación y la razón, sino más bien de manera gradual y penosa a través de una larga experiencia. El hombre llega paulatinamente de espaldas a la verdad; empieza en el error, progresa en el error, y alcanza finalmente el umbral de la verdad. Sólo se ha puesto a mirar hacia adelante con la llegada del método científico. Pero el hombre primitivo tenía que experimentar o perecer.

\par
%\textsuperscript{(970.11)}
\textsuperscript{88:4.6} La fascinación de las supersticiones primitivas fue la madre de la curiosidad científica posterior. En estas supersticiones primitivas había una emoción dinámica progresista ---miedo además de curiosidad; la antigua magia poseía una fuerza motriz progresista. Estas supersticiones representaban la aparición del deseo humano por conocer y controlar el entorno planetario.

\par
%\textsuperscript{(971.1)}
\textsuperscript{88:4.7} La magia consiguió un dominio tan fuerte sobre los salvajes porque éstos no podían captar el concepto de la muerte natural. La idea posterior del pecado original ayudó mucho a debilitar el poder de la magia sobre la raza, porque explicaba la muerte natural. En cierta época, no era raro que diez personas inocentes fueran ejecutadas porque se suponía que eran responsables de una muerte natural. Ésta es una de las razones por las cuales los pueblos antiguos no crecieron más rápidamente, y aún sigue sucediendo en algunas tribus africanas. El individuo acusado confesaba generalmente su culpabilidad, aún sabiendo que se enfrentaba a la muerte.

\par
%\textsuperscript{(971.2)}
\textsuperscript{88:4.8} La magia es natural para un salvaje. Cree que se puede matar realmente a un enemigo practicando la brujería sobre un mechón de sus cabellos o unos recortes de sus uñas. La muerte por mordedura de serpiente se atribuía a la magia del brujo. La dificultad para combatir la magia surge del hecho de que el miedo puede matar. Los pueblos primitivos temían tanto la magia que ésta mataba realmente, y estos resultados eran suficientes para justificar esta creencia errónea. En caso de fracaso, siempre existía alguna explicación plausible; el remedio para una magia defectuosa era más magia.

\section*{5. Los amuletos mágicos}
\par
%\textsuperscript{(971.3)}
\textsuperscript{88:5.1} Puesto que todo lo relacionado con el cuerpo podía volverse un fetiche, la magia más primitiva tuvo que ver con el cabello y las uñas. El secreto que acompaña las excreciones corporales nació del miedo a que un enemigo pudiera tomar posesión de algo que procediera del cuerpo y emplearlo en una magia perjudicial; por lo tanto, todos los excrementos del cuerpo se enterraban cuidadosamente. La gente se abstenía de escupir en público por miedo a que la saliva se pudiera utilizar en una magia mortífera; el escupitajo siempre se tapaba. Incluso los restos de comida, la ropa y los adornos podían volverse instrumentos de la magia. Los salvajes nunca dejaban restos de comida en la mesa. Todo esto lo hacían por miedo a que los enemigos pudieran utilizar estas cosas en sus ritos mágicos, y no porque apreciaran el valor higiénico de estas prácticas.

\par
%\textsuperscript{(971.4)}
\textsuperscript{88:5.2} Los amuletos mágicos se preparaban mezclando una gran variedad de cosas: carne humana, garras de tigre, dientes de cocodrilo, semillas de plantas venenosas, veneno de serpiente y cabellos humanos. Los huesos de los muertos eran muy mágicos. Incluso el polvo de las pisadas se podía utilizar en la magia. Los antiguos creían mucho en los amuletos de amor. La sangre y otras formas de secreciones corporales eran capaces de asegurar la influencia mágica del amor.

\par
%\textsuperscript{(971.5)}
\textsuperscript{88:5.3} Se suponía que las imágenes eran eficaces en la magia. Se hacían efigies y, cuando se las trataba bien o mal, se creía que estos mismos efectos alcanzaban a la persona real. Cuando iban a comprar, las personas supersticiosas masticaban un trozo de madera dura para ablandar el corazón del vendedor.

\par
%\textsuperscript{(971.6)}
\textsuperscript{88:5.4} La leche de una vaca negra era sumamente mágica, así como los gatos negros. El palo o varita eran mágicos, junto con los tambores, las campanas y los nudos. Todos los objetos antiguos eran amuletos mágicos. Las costumbres de una civilización nueva o más elevada eran consideradas con desaprobación a causa de su supuesta naturaleza mágica nociva. La escritura, los impresos y las imágenes fueron considerados así durante mucho tiempo.

\par
%\textsuperscript{(971.7)}
\textsuperscript{88:5.5} El hombre primitivo creía que los nombres debían ser tratados con respeto, especialmente los nombres de los dioses\footnote{\textit{Nombres como fetiches}: Ex 20:7; Lv 19:12; 24:11-16; Dt 5:11.}. El nombre era considerado como una entidad, una influencia distinta a la de la personalidad física; se le tenía en la misma estima que al alma y a la sombra. El nombre se empeñaba para obtener un préstamo; un hombre no podía utilizar su nombre hasta que lo hubiera desempeñado pagando el préstamo. Hoy en día la gente firma con su nombre en los pagarés. El nombre de una persona no tardó en volverse importante en la magia. El salvaje tenía dos nombres; el más importante se consideraba demasiado sagrado como para utilizarlo en circunstancias corrientes, de ahí el segundo nombre o nombre de todos los días ---un apodo. El salvaje nunca decía su verdadero nombre a los extraños. Cualquier experiencia de naturaleza insólita le inducía a cambiar de nombre; a veces lo hacía en un esfuerzo por curar una enfermedad o detener la mala suerte. El salvaje podía conseguir un nuevo nombre comprándoselo al jefe de la tribu. Los hombres todavía invierten dinero en títulos y rangos. Pero en las tribus más primitivas, tales como los bosquimanos de África, los nombres individuales no existen.

\section*{6. La práctica de la magia}
\par
%\textsuperscript{(972.1)}
\textsuperscript{88:6.1} La magia se practicaba mediante la utilización de varitas, rituales <<medicinales>> y conjuros, y el curandero tenía la costumbre de trabajar desnudo. Entre los magos primitivos, las mujeres eran más numerosas que los hombres. En magia, la palabra <<medicina>> significa misterio, no tratamiento. El salvaje nunca se curaba a sí mismo; nunca tomaba medicamentos a menos que se lo aconsejaran los especialistas en magia. Los médicos vudúes del siglo veinte son un ejemplo típico de los magos antiguos.

\par
%\textsuperscript{(972.2)}
\textsuperscript{88:6.2} La magia tenía una fase pública y una fase privada. Se suponía que la magia practicada por el curandero, el chamán o el sacerdote era para el bien de toda la tribu. Las brujas, los brujos y los hechiceros realizaban la magia privada, la magia personal y egoísta que se empleaba como método coercitivo para perjudicar a los enemigos. El concepto del doble espiritismo, de los espíritus buenos y malos, dio nacimiento a las creencias posteriores en la magia blanca y la magia negra. A medida que la religión evolucionó, el término magia se aplicó a las operaciones con los espíritus ajenas al culto propio, y también se refirió a las creencias más antiguas en los fantasmas.

\par
%\textsuperscript{(972.3)}
\textsuperscript{88:6.3} Las combinaciones de palabras, el ritual de los cantos y los conjuros, eran extremadamente mágicos. Algunos conjuros primitivos se transformaron finalmente en oraciones. La magia imitativa se practicó pronto; las oraciones se representaban; las danzas mágicas no eran más que oraciones teatrales. La oración desplazó gradualmente a la magia como asociada en los sacrificios.

\par
%\textsuperscript{(972.4)}
\textsuperscript{88:6.4} Como los gestos eran más antiguos que el habla, eran más sagrados y mágicos, y se creía que la mímica poseía un fuerte poder mágico. Los hombres rojos ponían a menudo en escena una danza del búfalo en la que uno de ellos interpretaba el papel del búfalo que, al ser capturado, aseguraba el éxito de la caza inminente. Las celebraciones sexuales del Primero de Mayo eran simplemente una magia imitativa, un llamamiento sugestivo a las pasiones sexuales del mundo vegetal. Las muñecas fueron empleadas al principio como talismanes mágicos por las esposas estériles.

\par
%\textsuperscript{(972.5)}
\textsuperscript{88:6.5} La magia fue la rama que salió del árbol religioso evolutivo y que dio finalmente el fruto de la era científica. La creencia en la astrología condujo al desarrollo de la astronomía; la creencia en la piedra filosofal llevó al dominio de los metales, mientras que la creencia en los números mágicos fundó la ciencia de las matemáticas.

\par
%\textsuperscript{(972.6)}
\textsuperscript{88:6.6} Pero un mundo tan lleno de hechizos contribuyó mucho a destruir toda ambición e iniciativa personal. Los frutos del trabajo suplementario o de la diligencia eran considerados como mágicos. Si un hombre tenía en su campo más grano que su vecino, lo podían llevar a rastras ante el jefe y acusarlo de que atraía este grano adicional del campo de su vecino indolente. En verdad, en los tiempos de la barbarie era peligroso saber demasiado; siempre existía la posibilidad de ser ejecutado como practicante de la magia negra.

\par
%\textsuperscript{(972.7)}
\textsuperscript{88:6.7} La ciencia elimina gradualmente de la vida el factor de juego de azar. Pero si los métodos modernos de educación fracasaran, se produciría una vuelta casi inmediata a las creencias primitivas en la magia. Estas supersticiones subsisten todavía en la mente de muchas personas llamadas civilizadas. El lenguaje contiene muchos fósiles que revelan que la raza ha estado impregnada durante mucho tiempo de la superstición mágica, teniendo palabras tales como hechizado, augurio, poseso, inspiración, desaparecer como por arte de magia, genial, encantador, adivinanza y embrujo. Los seres humanos inteligentes creen todavía en la buena suerte, el mal de ojo y la astrología.

\par
%\textsuperscript{(973.1)}
\textsuperscript{88:6.8} La magia antigua fue el capullo de la ciencia moderna, indispensable en su tiempo pero inútil en la actualidad. Los fantasmas de la superstición ignorante agitaron así la mente primitiva de los hombres hasta que los conceptos de la ciencia pudieron nacer. Urantia se encuentra hoy en el crepúsculo de esta evolución intelectual. Una mitad del mundo se aferra ávidamente a la luz de la verdad y a los hechos de los descubrimientos científicos, mientras que la otra mitad languidece en los brazos de las antiguas supersticiones y de una magia apenas disfrazada.

\par
%\textsuperscript{(973.2)}
\textsuperscript{88:6.9} [Presentado por una Brillante Estrella Vespertina de Nebadon.]


\chapter{Documento 89. Pecado, sacrificio y expiación}
\par
%\textsuperscript{(974.1)}
\textsuperscript{89:0.1} EL HOMBRE primitivo se consideraba como endeudado con los espíritus, como teniendo necesidad de redención. Desde el punto de vista de los salvajes, los espíritus les podían haber enviado con justa razón mucha más mala suerte. Con el paso del tiempo, este concepto se transformó en la doctrina del pecado y la salvación. Se consideraba que el alma venía al mundo con una deuda ---el pecado original. El alma tenía que ser redimida; había que proporcionar un chivo expiatorio. Los cazadores de cabezas, además de practicar el culto de la adoración a las calaveras, podían proporcionar una víctima propiciatoria como sustituta de sus propias vidas.

\par
%\textsuperscript{(974.2)}
\textsuperscript{89:0.2} El salvaje se obsesionó muy pronto con la idea de que los espíritus obtenían una satisfacción suprema con el espectáculo de la miseria, el sufrimiento y la humillación humanos. Al principio el hombre sólo se inquietó por los pecados de obra, pero más tarde se preocupó por los pecados de omisión. Todo el sistema sacrificatorio\footnote{\textit{Sistema sacrificial}: Lv 1:1-9:24.} posterior se desarrolló alrededor de estas dos ideas. Este nuevo ritual estaba relacionado con el cumplimiento de las ceremonias propiciatorias de los sacrificios. El hombre primitivo creía que había que hacer algo especial para ganarse el favor de los dioses; sólo una civilización avanzada reconoce a un Dios coherentemente ecuánime y benévolo. La propiciación era un seguro contra la mala suerte cercana, en lugar de ser una inversión para una dicha futura. Todos los ritos de evitación, exorcismo, coacción y propiciación se fundieron los unos en los otros.

\section*{1. El tabú}
\par
%\textsuperscript{(974.3)}
\textsuperscript{89:1.1} El acatamiento de un tabú era el esfuerzo del hombre por esquivar la mala suerte, por evitar ofender a los fantasmas espíritus absteniéndose de hacer algo. Al principio los tabúes no eran religiosos, pero muy pronto consiguieron la aprobación de los fantasmas o los espíritus\footnote{\textit{Tabús aprobados por los espíritus}: Lv 11:1-47.}, y cuando estuvieron reforzados de esta manera, se convirtieron en los legisladores y constructores de las instituciones. El tabú es la fuente de las reglas ceremoniales y el predecesor del autocontrol primitivo. Fue la primera forma de reglamentación social y, durante mucho tiempo, la única; todavía sigue siendo un elemento básico de la estructura regulativa social.

\par
%\textsuperscript{(974.4)}
\textsuperscript{89:1.2} El respeto que infundían estas prohibiciones en la mente de los salvajes equivalía exactamente al miedo que tenían a los poderes que supuestamente las imponían\footnote{\textit{Miedo a violar tabús}: Lv 10:1-2.}. Los tabúes surgieron primero a causa de las experiencias casuales con la mala suerte. Más tarde fueron propuestos por los jefes y los chamanes ---los hombres fetiches que, según se creía, estaban dirigidos por un fantasma espíritu, o incluso por un dios. El miedo al castigo de los espíritus es tan grande en la mente de un primitivo, que a veces muere de miedo cuando ha violado un tabú, y estos episodios dramáticos refuerzan enormemente el poder del tabú sobre la mente de los supervivientes.

\par
%\textsuperscript{(974.5)}
\textsuperscript{89:1.3} Entre las primeras prohibiciones se encontraron las restricciones sobre la apropiación de las mujeres y otros bienes. A medida que la religión empezó a jugar un papel más importante en la evolución del tabú, el objeto que estaba prohibido era considerado como impuro, y posteriormente como profano. Los anales de los hebreos están repletos de menciones sobre cosas puras e impuras, sagradas y profanas, pero sus creencias en este sentido eran mucho menos engorrosas y abundantes que las de otros muchos pueblos.

\par
%\textsuperscript{(975.1)}
\textsuperscript{89:1.4} Los siete mandamientos de Dalamatia y Edén, así como los diez mandatos de los hebreos, eran unos tabúes precisos\footnote{\textit{Los diez mandamientos como tabúes}: Ex 20:3-17; Dt 5:7-21.}, todos expresados de la misma forma negativa que la mayoría de las prohibiciones antiguas. Pero estos códigos más nuevos eran realmente emancipadores, ya que sustituían a miles de tabúes preexistentes. Y además de esto, estos mandamientos más tardíos prometían claramente algo a cambio de la obediencia.

\par
%\textsuperscript{(975.2)}
\textsuperscript{89:1.5} Los tabúes primitivos sobre la comida se originaron en el fetichismo y el totemismo. El cerdo era sagrado para los fenicios\footnote{\textit{El tabú del cerdo}: Lv 11:7-8.}, y la vaca para los hindúes. El tabú egipcio sobre la carne de cerdo se ha perpetuado en la fe hebrea e islámica. Una variante del tabú sobre la comida era la creencia de que una mujer embarazada podía pensar tanto en cierto alimento que, cuando naciera el hijo, sería el reflejo de ese alimento. Tales viandas serían tabúes para el niño.

\par
%\textsuperscript{(975.3)}
\textsuperscript{89:1.6} Las maneras de comer pronto se volvieron tabúes, y así es como se originó el protocolo antiguo y moderno en la mesa. Los sistemas de castas y los niveles sociales son vestigios residuales de las prohibiciones antiguas. Los tabúes fueron muy eficaces para organizar la sociedad, pero fueron enormemente gravosos; el sistema negativo de la prohibición no solamente mantenía unas reglas útiles y constructivas, sino también unos tabúes anticuados, caducos e inútiles.

\par
%\textsuperscript{(975.4)}
\textsuperscript{89:1.7} Sin embargo, ninguna sociedad civilizada podría criticar al hombre primitivo salvo por estos tabúes extensos y variados, y los tabúes nunca hubieran perdurado si no hubieran tenido la aprobación y el apoyo de la religión primitiva. Muchos factores esenciales para la evolución del hombre han sido extremadamente costosos, han costado inmensos tesoros en esfuerzos, sacrificios y abnegación, pero estos logros en el dominio de sí mismo fueron los verdaderos peldaños por los que el hombre subió la escala ascendente de la civilización.

\section*{2. El concepto del pecado}
\par
%\textsuperscript{(975.5)}
\textsuperscript{89:2.1} El miedo a la casualidad y el terror a la mala suerte empujaron literalmente al hombre a inventar la religión primitiva como un supuesto seguro contra estas calamidades. Partiendo de la magia y los fantasmas, la religión evolucionó pasando por los espíritus y los fetiches hasta los tabúes\footnote{\textit{``No harás''}: Ex 20:4; Dt 5:8.}. Todas las tribus primitivas tenían su árbol del fruto prohibido\footnote{\textit{Fruto prohibido}: Gn 2:17; 3:2-3.}, literalmente la manzana, pero en sentido figurado consistía en un millar de ramas sobrecargadas de todo tipo de tabúes. Y el árbol prohibido siempre decía: <<No harás>>.

\par
%\textsuperscript{(975.6)}
\textsuperscript{89:2.2} Cuando la mente del salvaje evolucionó hasta el punto de imaginar tanto a los buenos como a los malos espíritus, y cuando el tabú recibió la solemne aprobación de la religión en evolución, todo el escenario estuvo preparado para la aparición del nuevo concepto del \textit{pecado}. La idea del pecado se estableció en el mundo de manera universal antes de que entrara la religión revelada. La muerte natural sólo se volvió lógica para la mente primitiva gracias al concepto del pecado\footnote{\textit{Muerte y pecado}: Ro 6:23.}. El pecado era la transgresión del tabú, y la muerte era el castigo del pecado.

\par
%\textsuperscript{(975.7)}
\textsuperscript{89:2.3} El pecado era ritual, no racional; era un acto, no un pensamiento. Las tradiciones sobrevivientes de Dilmun y de los tiempos de un pequeño paraíso en la Tierra fomentaron todo este concepto del pecado. La tradición de Adán y del Jardín del Edén también dio consistencia a la ilusión de una antigua <<era de oro>> en los albores de las razas. Todo esto confirmaba las ideas expresadas más tarde en la creencia de que el hombre tenía su origen en una creación especial, de que había empezado su carrera siendo perfecto\footnote{\textit{Creencia en un hombre original perfecto}: Gn 1:27,31; 2:7,22.}, y que la transgresión de los tabúes ---el pecado--- lo había rebajado a su triste condición posterior\footnote{\textit{La caída de la humanidad}: Gn 3:1-19; Ro 5:12-19.}.

\par
%\textsuperscript{(976.1)}
\textsuperscript{89:2.4} La violación habitual de un tabú se volvió un vicio; la ley primitiva hizo del vicio un crimen; la religión determinó que era un pecado. Entre las tribus primitivas, la violación de un tabú era una combinación de crimen y de pecado. Las calamidades que caían sobre la comunidad\footnote{\textit{Calamidades de la comunidad}: Lv 26:14-39.} siempre eran consideradas como un castigo por un pecado de la tribu. Para aquellos que creían que la prosperidad y la rectitud iban unidas, la aparente prosperidad de los malvados causó tanta preocupación que fue necesario inventar los infiernos para castigar a los que violaban los tabúes; el número de estos lugares de castigo futuro ha variado de uno a cinco.

\par
%\textsuperscript{(976.2)}
\textsuperscript{89:2.5} La idea de confesión y de perdón apareció pronto en la religión primitiva. Los hombres solían pedir perdón en una reunión pública por los pecados que tenían la intención de cometer la semana siguiente. La confesión era simplemente un rito de remisión, y también una notificación pública de deshonra, un ritual que consistía en gritar <<¡impuro, impuro!>>. Luego venían a continuación todas las formas rituales de purificación\footnote{\textit{Ritual de la confesión como purificación}: Lv 13:45.}. Todos los pueblos antiguos practicaban estas ceremonias sin sentido. Muchas costumbres aparentemente higiénicas de las tribus primitivas eran sobre todo ceremoniales.

\section*{3. La renuncia y la humillación}
\par
%\textsuperscript{(976.3)}
\textsuperscript{89:3.1} La renuncia fue la etapa siguiente de la evolución religiosa; el ayuno se practicaba de manera habitual\footnote{\textit{Ayuno ritual}: 1 Re 21:9,12; Is 58:3.6; Mt 9:14; Mc 2:18; Lc 5:33; 2 Sam 12:16,21-23.}. Pronto se estableció la costumbre de renunciar a muchas formas de placer físico, especialmente de naturaleza sexual. El ritual del ayuno estaba profundamente arraigado en muchas religiones antiguas, y ha sido transmitido prácticamente a todos los sistemas teológicos modernos de pensamiento.

\par
%\textsuperscript{(976.4)}
\textsuperscript{89:3.2} Justo en la época en que los hombres bárbaros se recobraban de la práctica ruinosa consistente en quemar y enterrar los bienes con los muertos, justo en el momento en que la estructura económica de las razas empezaba a tomar forma, apareció esta nueva doctrina religiosa de la renuncia, y decenas de miles de almas sinceras empezaron a buscar la pobreza. Los bienes fueron considerados como un obstáculo espiritual. Estas ideas sobre los peligros espirituales de las posesiones materiales\footnote{\textit{La propiedad vista como un peligro espiritual}: Hch 2:44-45; 1 Ti 6:8-11.} estaban ampliamente difundidas en los tiempos de Filón y Pablo, y desde entonces han influido notablemente sobre la filosofía europea.

\par
%\textsuperscript{(976.5)}
\textsuperscript{89:3.3} La pobreza era simplemente una parte del ritual de la mortificación de la carne que, lamentablemente, quedó incorporada en los escritos y las enseñanzas de muchas religiones, principalmente del cristianismo. La penitencia es la forma negativa de este ritual, a menudo insensato, de la renuncia\footnote{\textit{Pobreza y penitencia}: Gn 37:34; 1 Re 21:27; Est 4:1-4; Is 37:1-2; 58:3-5; Jon 3:5-6.}. Pero todo esto enseñó a los salvajes el \textit{dominio de sí mismo}, y constituyó un progreso digno de consideración en la evolución social. La abnegación y el dominio de sí mismo fueron dos de los beneficios sociales más importantes procedentes de la religión evolutiva primitiva. El dominio de sí mismo proporcionó al hombre una nueva filosofía de la vida; le enseñó el arte de aumentar su fracción de vida disminuyendo el denominador de las exigencias personales, en lugar de intentar acrecentar siempre el numerador de las satisfacciones egoístas.

\par
%\textsuperscript{(976.6)}
\textsuperscript{89:3.4} Estas ideas antiguas sobre la autodisciplina incluían la flagelación y todo tipo de torturas físicas. Los sacerdotes del culto a la madre eran especialmente activos enseñando la virtud de los sufrimientos físicos, y daban ejemplo sometiéndose a la castración. Los hebreos, los hindúes y los budistas eran unos partidarios sinceros de esta doctrina de la humillación física.

\par
%\textsuperscript{(976.7)}
\textsuperscript{89:3.5} A lo largo de todos los tiempos antiguos, los hombres trataron de conseguir por estos medios unos saldos adicionales a su favor en los libros contables sobre la abnegación que llevaban sus dioses. Cuando se experimentaba alguna tensión emocional, en otros tiempos se tenía la costumbre de hacer votos de abnegación y de tortura de sí mismo. Con el tiempo, estos votos adoptaron la forma de contratos con los dioses y, en este sentido, representaron un verdadero progreso evolutivo, ya que se suponía que los dioses harían algo concreto en recompensa por esta tortura y mortificación de la carne. Los votos eran tanto negativos como positivos. Las promesas de esta naturaleza tan nociva y extrema se pueden observar hoy mucho mejor en algunos grupos de la India.

\par
%\textsuperscript{(977.1)}
\textsuperscript{89:3.6} Era muy natural que el culto de la renuncia y la humillación prestara atención a las satisfacciones sexuales. El culto de la continencia se originó como un ritual que practicaban los soldados antes de entrar en combate; en épocas posteriores se convirtió en la práctica de los <<santos>>. Este culto sólo toleraba el matrimonio como un mal menor\footnote{\textit{El matrimonio visto como mal menor}: 1 Co 7:2,9.} que la fornicación. Muchas grandes religiones del mundo han sufrido la influencia desfavorable de este antiguo culto, pero ninguna de manera más acusada que el cristianismo. El apóstol Pablo era un adepto de este culto, y sus opiniones personales están reflejadas en las enseñanzas que introdujo en la teología cristiana: <<Es bueno para el hombre no tocar ninguna mujer>>\footnote{\textit{No tocar mujer}: 1 Co 7:1.}. <<Quisiera que todos los hombres fueran como yo>>\footnote{\textit{Pablo quiere que todos sean solteros como él}: 1 Co 7:7.}. <<Digo pues a los no casados y a las viudas que es bueno para ellos permanecer como yo>>\footnote{\textit{Bueno para solteros y viudas}: 1 Co 7:8.}. Pablo sabía muy bien que estas enseñanzas no formaban parte del evangelio de Jesús, y así lo reconoció, tal como queda demostrado en su declaración: <<Digo esto por concesión, no por mandato>>\footnote{\textit{Pablo habla desde su opinión}: 1 Co 7:6.}. Pero este culto condujo a Pablo a menospreciar a las mujeres. La pena de todo esto es que sus opiniones personales han influido durante mucho tiempo sobre las enseñanzas de una gran religión mundial. Si los consejos de este instructor y fabricante de tiendas fueran obedecidos de manera literal y universal, la raza humana llegaría a un fin repentino e ignominioso. Además, la relación de una religión con el antiguo culto de la continencia conduce directamente a una guerra contra el matrimonio y el hogar, que son los verdaderos fundamentos de la sociedad y las instituciones básicas del progreso humano. No es de extrañar que todas estas creencias favorecieran la formación de cleros célibes en las diversas religiones de distintos pueblos.

\par
%\textsuperscript{(977.2)}
\textsuperscript{89:3.7} Algún día, el hombre deberá aprender a disfrutar de la libertad sin licencia, de la alimentación sin glotonería, y del placer sin libertinaje. Para regular el comportamiento personal, el dominio de sí mismo es una política humana mucho mejor que la abnegación extrema. Jesús tampoco enseñó nunca estas ideas desrazonables a sus seguidores.

\section*{4. Los orígenes del sacrificio}
\par
%\textsuperscript{(977.3)}
\textsuperscript{89:4.1} Al igual que otros muchos rituales de adoración, el sacrificio, como parte de las devociones religiosas, no tuvo un origen simple y único. La tendencia a doblegarse ante el poder y a postrarse en devota adoración en presencia del misterio se encuentra prefigurada en el servilismo del perro ante su amo. Entre el impulso a adorar y el acto del sacrificio no hay más que un paso. El hombre primitivo medía el valor de su sacrificio por el dolor que padecía. Cuando la idea de sacrificio se vinculó por primera vez al ceremonial religioso, no se concebía ninguna ofrenda que no produjera dolor. Los primeros sacrificios consistieron en actos tales como arrancarse los cabellos, cortarse, mutilarse, partirse los dientes y amputarse los dedos. A medida que avanzó la civilización, estos conceptos rudimentarios del sacrificio fueron elevados al nivel de los rituales de la abnegación, el ascetismo, el ayuno, las privaciones y la doctrina cristiana posterior de la santificación a través de la tristeza, el sufrimiento y la mortificación de la carne.

\par
%\textsuperscript{(977.4)}
\textsuperscript{89:4.2} Al principio de la evolución de la religión existieron dos concepciones del sacrificio: la idea del sacrificio mediante las ofrendas, que implicaba una actitud de acción de gracias, y el sacrificio debido a la deuda, que englobaba la idea de redención. Más adelante se desarrolló el concepto de la sustitución.

\par
%\textsuperscript{(977.5)}
\textsuperscript{89:4.3} Más tarde aún, el hombre concibió que cualquiera que fuera la naturaleza de su sacrificio, podría servir como portador de un mensaje ante los dioses; podría ser como un aroma agradable para el olfato de la deidad\footnote{\textit{Aroma agradable para el olfato de Dios}: Gn 8:21; Ex 29:18,25,41; Lv 1:9,13,17; Nm 15:3,7,10,13; 15:14,24; 2 Co 2:15; Ef 5:2.}. Esto introdujo la utilización del incienso y otras características estéticas en los rituales de los sacrificios, los cuales se convirtieron con el tiempo en unas fiestas religiosas sacrificatorias que se volvieron cada vez más elaboradas y adornadas.

\par
%\textsuperscript{(978.1)}
\textsuperscript{89:4.4} A medida que la religión evolucionó, los ritos sacrificatorios de la conciliación y la propiciación reemplazaron a los métodos más antiguos de la evitación, el apaciguamiento y el exorcismo.

\par
%\textsuperscript{(978.2)}
\textsuperscript{89:4.5} La idea inicial del sacrificio fue la de un gravamen de neutralidad impuesto por los espíritus ancestrales; la idea de la expiación sólo se desarrolló más tarde. A medida que el hombre se alejó de la noción del origen evolutivo de la raza, a medida que las tradiciones de la época del Príncipe Planetario y de la estancia de Adán fueron filtradas por el tiempo, el concepto del pecado y del pecado original se difundió ampliamente, de manera que el sacrificio por un pecado accidental y personal evolucionó hacia la doctrina del sacrificio para expiar el pecado racial. La expiación por medio del sacrificio era un mecanismo de seguro a todo riesgo que protegía incluso contra el rencor y los celos de un dios desconocido.

\par
%\textsuperscript{(978.3)}
\textsuperscript{89:4.6} Rodeado de tantos espíritus susceptibles y dioses codiciosos, el hombre primitivo se enfrentaba con tal multitud de deidades acreedoras que se necesitaban todos los sacerdotes, el ritual y los sacrificios de una vida entera para liberarlo de sus deudas espirituales. La doctrina del pecado original, o de la culpabilidad racial, hacía que cada persona empezara su vida con una deuda importante hacia los poderes espirituales.

\par
%\textsuperscript{(978.4)}
\textsuperscript{89:4.7} A los hombres les entregan regalos y sobornos; pero cuando éstos son ofrecidos a los dioses, se les califica de consagrados, sagrados, o se les llama sacrificios. La renuncia era la forma negativa de la propiciación; el sacrificio se volvió la forma positiva. El acto de la propiciación incluía la alabanza, la glorificación, la adulación e incluso la diversión. Los restos de estas prácticas positivas del antiguo culto de la propiciación son los que constituyen las formas modernas de adoración divina. Las formas actuales de adoración son simplemente la ritualización de estas antiguas técnicas sacrificatorias de la propiciación positiva.

\par
%\textsuperscript{(978.5)}
\textsuperscript{89:4.8} El sacrificio de un animal significaba para el hombre primitivo mucho más de lo que podría significar nunca para las razas modernas. Aquellos bárbaros consideraban a los animales como sus verdaderos parientes cercanos. A medida que pasó el tiempo, el hombre se volvió más astuto en sus sacrificios y dejó de ofrecer sus animales de trabajo. Al principio sacrificaba lo \textit{mejor} de todo\footnote{\textit{Sacrificio de los mejor de todo}: Ex 12:5; 29:1; Lv 1:3,10; 1 P 1:19.}, incluyendo a sus animales domésticos.

\par
%\textsuperscript{(978.6)}
\textsuperscript{89:4.9} Cierto soberano egipcio no se jactaba en vano cuando afirmaba que había sacrificado 113.433 esclavos, 493.386 cabezas de ganado, 88 barcos, 2.756 imágenes de oro, 331.702 jarras de miel y de aceite, 228.380 jarras de vino, 680.714 gansos, 6.744.428 barras de pan y 5.740.352 sacos de monedas. Para poder hacer esto no había tenido más remedio que gravar con enormes impuestos a sus fatigados súbditos.

\par
%\textsuperscript{(978.7)}
\textsuperscript{89:4.10} La pura necesidad forzó finalmente a estos semisalvajes a comerse la parte material de sus sacrificios\footnote{\textit{Comerse los sacrificios}: Lv 2:3,10; 6:16-18,29.}, pues los dioses ya habían disfrutado del alma de los mismos. Esta costumbre se vio justificada bajo el pretexto del antiguo banquete sagrado, un culto de comunión según los usos modernos.

\section*{5. Los sacrificios y el canibalismo}
\par
%\textsuperscript{(978.8)}
\textsuperscript{89:5.1} Las ideas modernas sobre el canibalismo primitivo son totalmente falsas; éste formaba parte de las costumbres de la sociedad primitiva. Aunque el canibalismo es tradicionalmente horrible para la civilización moderna, formaba parte de la estructura social y religiosa de la sociedad primitiva. Los intereses colectivos dictaron la práctica del canibalismo. Surgió debido al impulso de la necesidad y perduró a causa de la esclavitud a la superstición y a la ignorancia. Era una costumbre social, económica, religiosa y militar.

\par
%\textsuperscript{(979.1)}
\textsuperscript{89:5.2} El hombre primitivo era caníbal. Disfrutaba con la carne humana, y por eso la ofrecía como ofrenda alimenticia a los espíritus y a sus dioses primitivos. Puesto que los espíritus fantasmas no eran más que hombres modificados, y puesto que la comida era la necesidad más grande del hombre, entonces la comida debía ser también la necesidad más grande de un espíritu.

\par
%\textsuperscript{(979.2)}
\textsuperscript{89:5.3} El canibalismo fue en otro tiempo casi universal entre las razas en evolución. Todos los sangiks eran caníbales, pero al principio los andonitas no lo eran, ni tampoco los noditas ni los adamitas; los anditas no lo fueron hasta después de mezclarse enormemente con las razas evolutivas.

\par
%\textsuperscript{(979.3)}
\textsuperscript{89:5.4} El gusto por la carne humana aumenta. Una vez que se ha empezado a comer carne humana debido al hambre, la amistad, la venganza, o el ritual religioso, llega a convertirse en un canibalismo habitual. La antropofagia surgió a causa de la escasez de alimentos, aunque ésta ha sido raras veces la razón fundamental. Sin embargo, los esquimales y los andonitas primitivos muy pocas veces fueron caníbales, salvo en períodos de escasez. Los hombres rojos, especialmente en América Central, eran caníbales. Las madres primitivas tuvieron en otro tiempo la costumbre general de matar y comerse a sus propios hijos a fin de renovar las fuerzas que habían perdido en el parto; en Queensland, al hijo primogénito todavía se le mata y se le devora así con frecuencia. En tiempos recientes, muchas tribus africanas han recurrido deliberadamente al canibalismo como medida de guerra, como una especie de atrocidad para aterrorizar a sus vecinos.

\par
%\textsuperscript{(979.4)}
\textsuperscript{89:5.5} Cierto canibalismo fue el resultado de la degeneración de algunos linajes en otro tiempo superiores, pero éste predominaba principalmente entre las razas evolutivas. La antropofagia empezó en una época en que los hombres experimentaban unas intensas y amargas emociones hacia sus enemigos. Comer carne humana llegó a formar parte de una ceremonia solemne de venganza; se creía que, de esta manera, el fantasma de un enemigo se podía destruir o fusionar con el de la persona que se lo comía. La creencia de que los hechiceros conseguían sus poderes comiendo carne humana estuvo en otro tiempo muy extendida.

\par
%\textsuperscript{(979.5)}
\textsuperscript{89:5.6} Algunos grupos de antropófagos solían consumir únicamente a los miembros de sus propias tribus, una endogamia seudoespiritual que acentuaba supuestamente la solidaridad tribal. Pero también se comían a los enemigos para vengarse, con la idea de apropiarse de su fuerza. Se consideraba que para el alma de un amigo o de un compañero de tribu era un honor que su cuerpo fuera comido, mientras que devorar así a un enemigo no era más que infligirle un justo castigo. La mente del salvaje no tenía ninguna pretensión de ser coherente.

\par
%\textsuperscript{(979.6)}
\textsuperscript{89:5.7} En algunas tribus, los padres ancianos solían aspirar a ser comidos por sus hijos; en otras tenían la costumbre de abstenerse de comer a los parientes cercanos; sus cuerpos se vendían o se intercambiaban por los de los desconocidos. Existía un comercio considerable de mujeres y niños que eran engordados para la matanza. Cuando la enfermedad o la guerra no lograban restringir la población, el excedente era comido sin ceremonias.

\par
%\textsuperscript{(979.7)}
\textsuperscript{89:5.8} El canibalismo ha desaparecido paulatinamente debido a las influencias siguientes:

\par
%\textsuperscript{(979.8)}
\textsuperscript{89:5.9} 1. A veces se convertía en una ceremonia comunal, en la asunción de la responsabilidad colectiva para infligir la pena de muerte a un miembro de la misma tribu. La culpabilidad de la sangre deja de ser un crimen cuando todos participan en ella, cuando participa la sociedad. En Asia, la última manifestación de canibalismo fue la de comerse a estos criminales ajusticiados.

\par
%\textsuperscript{(979.9)}
\textsuperscript{89:5.10} 2. El canibalismo se convirtió muy pronto en un rito religioso, pero el miedo creciente a los fantasmas no siempre surtió el efecto de reducir la antropofagia.

\par
%\textsuperscript{(979.10)}
\textsuperscript{89:5.11} 3. Con el tiempo progresó hasta el punto en que sólo se comían ciertas partes u órganos del cuerpo, aquellas partes que se suponía que contenían el alma o porciones del espíritu. Beber sangre se volvió algo corriente, y existía la costumbre de mezclar las partes <<comestibles>> del cuerpo con medicinas.

\par
%\textsuperscript{(980.1)}
\textsuperscript{89:5.12} 4. Fue limitado a los hombres; a las mujeres se les prohibió que comieran carne humana.

\par
%\textsuperscript{(980.2)}
\textsuperscript{89:5.13} 5. Luego fue limitado a los jefes, sacerdotes y chamanes.

\par
%\textsuperscript{(980.3)}
\textsuperscript{89:5.14} 6. Después se volvió tabú entre las tribus superiores. El tabú sobre la antropofagia tuvo su origen en Dalamatia y se difundió lentamente por el mundo. Los noditas fomentaron la incineración como medio de combatir el canibalismo, ya que en otro tiempo era práctica normal desenterrar a los cadáveres para comerlos.

\par
%\textsuperscript{(980.4)}
\textsuperscript{89:5.15} 7. Los sacrificios humanos anunciaron el fin del canibalismo. Como la carne humana se había convertido en el alimento de los hombres superiores, de los jefes, finalmente fue reservada para los espíritus aún más superiores; y así, la ofrenda de los sacrificios humanos puso fin eficazmente al canibalismo, excepto entre las tribus más inferiores. Cuando los sacrificios humanos estuvieron plenamente establecidos, la antropofagia se volvió tabú; la carne humana sólo era una comida para los dioses; los hombres sólo podían comer un pequeño trozo ceremonial, un sacramento.

\par
%\textsuperscript{(980.5)}
\textsuperscript{89:5.16} Finalmente se generalizó la práctica de emplear animales como sustitutos para los fines sacrificatorios; los perros eran comidos incluso entre las tribus más atrasadas, lo que redujo considerablemente la antropofagia. El perro era el primer animal que se había domesticado, y se tenía en gran estima como animal doméstico y como alimento.

\section*{6. La evolución de los sacrificios humanos}
\par
%\textsuperscript{(980.6)}
\textsuperscript{89:6.1} Los sacrificios humanos fueron un resultado indirecto del canibalismo, así como su curación. El hecho de proporcionar un séquito de espíritus al mundo de los espíritus condujo también a la disminución de la antropofagia, porque nunca se tuvo la costumbre de comer estos muertos sacrificados. Ninguna raza ha estado completamente exenta de la práctica de los sacrificios humanos en alguna de sus formas y en algún momento, aunque los andonitas, los noditas y los adamitas fueron los menos adictos al canibalismo.

\par
%\textsuperscript{(980.7)}
\textsuperscript{89:6.2} Los sacrificios humanos\footnote{\textit{Sacrificios humanos}: Gn 22:1-10.} han sido prácticamente universales; sobrevivieron en las costumbres religiosas de los chinos, hindúes, egipcios, hebreos, mesopotámicos, griegos, romanos y otros muchos pueblos, y en los tiempos recientes se encuentran todavía entre las tribus atrasadas de África y Australia. Los indios americanos más recientes tuvieron una civilización surgida del canibalismo y, por ello, sumida en los sacrificios humanos, sobre todo en América Central y del Sur. Los caldeos fueron de los primeros que abandonaron los sacrificios humanos en circunstancias corrientes, sustituyéndolos por animales. Hace unos dos mil años, un compasivo emperador japonés introdujo las imágenes de arcilla para sustituir a los sacrificios humanos, pero hace sólo menos de mil años que estos sacrificios desaparecieron del norte de Europa. En ciertas tribus atrasadas, el sacrificio humano es practicado todavía por algunos voluntarios, como una especie de suicidio religioso o ritual. Un chamán ordenó en cierta ocasión el sacrificio de un anciano muy respetado de cierta tribu. El pueblo se sublevó; se negó a obedecer. Entonces el anciano hizo que su propio hijo lo matara; los antiguos creían realmente en esta costumbre.

\par
%\textsuperscript{(980.8)}
\textsuperscript{89:6.3} Entre las historias que ilustran las controversias desgarradoras entre las antiguas costumbres religiosas consagradas por la tradición y las exigencias contrarias de la civilización en progreso, no existe un relato más trágico y patético que la narración hebrea de Jefté y su única hija\footnote{\textit{La hija de Jefté}: Jue 11:30-39.}. Siguiendo la costumbre habitual, este hombre bienintencionado había hecho una promesa descabellada, había negociado con el <<dios de las batallas>>\footnote{\textit{Dios de las batallas}: 1 Cr 14:15; 2 Cr 20:15; 32:8; Sal 24:8; Dt 7:21-23; 20:1-4; 1 Sam 17:47.}, aceptando pagar cierto precio por la victoria sobre sus enemigos. Este precio consistía en sacrificar lo primero que saliera de su casa a su encuentro cuando volviera al hogar. Jefté pensó que uno de sus esclavos leales se acercaría para recibirlo, pero resultó que su hija, la única que tenía, salió para darle la bienvenida al hogar. Así pues, incluso en esta fecha reciente y en un pueblo supuestamente civilizado, esta hermosa doncella, después de dos meses lamentándose sobre su destino, fue ofrecida realmente como sacrificio humano por su padre, y con la aprobación de los hombres de su tribu. Todo esto se llevó a cabo a pesar de los estrictos mandatos de Moisés contra las ofrendas de sacrificios humanos. Pero los hombres y las mujeres son adictos a hacer votos insensatos e inútiles, y los hombres de la antig\"uedad consideraban que todas estas promesas solemnes eran sumamente sagradas.

\par
%\textsuperscript{(981.1)}
\textsuperscript{89:6.4} Cuando en los tiempos antiguos se empezaba a construir un edificio de alguna importancia, la costumbre exigía que se matara a un ser humano como <<sacrificio fundacional>>\footnote{\textit{Sacrificio fundacional}: Jos 6:26.}. Esto suministraba un espíritu fantasma para que vigilara y protegiera la estructura. Cuando los chinos se disponían a fundir una campana, la costumbre decretaba que se sacrificara al menos una doncella con el fin de mejorar el tono de la campana; la muchacha seleccionada era arrojada viva en el metal fundido.

\par
%\textsuperscript{(981.2)}
\textsuperscript{89:6.5} Numerosos grupos tuvieron durante mucho tiempo la costumbre de empotrar vivos a los esclavos en las murallas importantes. En tiempos posteriores, las tribus del norte de Europa se limitaron a emparedar la sombra de un transeúnte para sustituir la costumbre de sepultar vivas a las personas entre los muros de los nuevos edificios. Los chinos enterraban en un muro a aquellos obreros que habían muerto mientras lo construían.

\par
%\textsuperscript{(981.3)}
\textsuperscript{89:6.6} En el momento de construir las murallas de Jericó, un reyezuelo de Palestina <<echó los cimientos sobre Abiram, su hijo primogénito, y edificó las puertas sobre Segub, su hijo menor>>\footnote{\textit{Sacrifico de los hijos de Hiel}: 1 Re 16:34.}. En esta fecha tan tardía, este padre no solamente puso a dos de sus hijos vivos en los agujeros de los cimientos de las puertas de la ciudad, sino que su acción fue también registrada como <<conforme a la palabra del Señor>>\footnote{\textit{Conforme a la palabra del Señor}: 1 Re 16:34.}. Moisés había prohibido estos sacrificios fundacionales, pero los israelitas volvieron a practicarlos poco después de su muerte. Las ceremonias del siglo veinte consistentes en depositar baratijas y recuerdos en la piedra angular de un nuevo edificio, es una reminiscencia de los sacrificios fundacionales primitivos.

\par
%\textsuperscript{(981.4)}
\textsuperscript{89:6.7} Numerosos pueblos tuvieron durante mucho tiempo la costumbre de dedicar a los espíritus los primeros frutos. Todas estas prácticas, ahora más o menos simbólicas, son supervivencias de las ceremonias primitivas que incluían los sacrificios humanos. La idea de ofrecer al hijo primogénito como sacrificio estaba muy extendida entre los antiguos, especialmente entre los fenicios, que fueron los últimos en abandonarla. En el momento del sacrificio se solía decir: <<una vida por una vida>>\footnote{\textit{Una vida por una vida}: Ex 21:23.}. Ahora decís después de una muerte: <<el polvo vuelve al polvo>>\footnote{\textit{El polvo vuelve al polvo}: Gn 3:14,19; Job 34:15; Ec 3:20.}.

\par
%\textsuperscript{(981.5)}
\textsuperscript{89:6.8} Aunque resulte chocante para la sensibilidad civilizada, el espectáculo de Abraham obligado a sacrificar a su hijo Isaac\footnote{\textit{Sacrificio de Isaac}: Gn 22:1-10.} no era una idea nueva o extraña para los hombres de aquella época. En los momentos de una gran tensión emocional, los padres habían recurrido durante mucho tiempo a la práctica frecuente de sacrificar a sus hijos primogénitos. Muchos pueblos poseen una tradición análoga a esta historia, pues antiguamente existía la creencia profunda y generalizada de que era necesario ofrecer un sacrificio humano cada vez que sucedía algo extraordinario o fuera de lo común.

\section*{7. Las modificaciones de los sacrificios humanos}
\par
%\textsuperscript{(981.6)}
\textsuperscript{89:7.1} Moisés intentó poner fin a los sacrificios humanos, introduciendo el rescate como sustituto\footnote{\textit{Dios acepta el rescate}: Ex 13:12-13.}. Estableció un programa sistemático que permitía a su pueblo eludir las peores consecuencias de sus promesas imprudentes e insensatas\footnote{\textit{Sistema sacrificial}: Lv 27:1-34.}. Las tierras, las propiedades y los hijos se podían recomprar de acuerdo con los honorarios establecidos, que se pagaban a los sacerdotes. Aquellos grupos que dejaron de sacrificar a sus primogénitos pronto poseyeron grandes ventajas sobre sus vecinos menos avanzados que continuaron practicando estas atrocidades. Muchas tribus atrasadas de este tipo no sólo se debilitaron enormemente debido a esta pérdida de sus hijos, sino que a menudo se rompió incluso la línea de sucesión en el mando\footnote{\textit{Sustitutos y rescate}: Ex 21:30.}.

\par
%\textsuperscript{(982.1)}
\textsuperscript{89:7.2} Una consecuencia del sacrificio pasajero de los hijos\footnote{\textit{Sacrificio de niños}: Ex 12:7,12-13.} fue la costumbre de manchar con sangre las jambas de la puerta de la casa para proteger a los primogénitos. Esto se hacía a menudo en conexión con una de las fiestas sagradas del año, y esta ceremonia prevaleció en otro tiempo en la mayor parte del mundo, desde Méjico hasta Egipto.

\par
%\textsuperscript{(982.2)}
\textsuperscript{89:7.3} Incluso después de que la mayoría de los grupos hubieron dejado de practicar el asesinato ritual de los niños, conservaron la costumbre de abandonar a un niño en el desierto o en una pequeña embarcación en el agua. Si el niño sobrevivía, se creía que los dioses habían intervenido para protegerlo, como en las tradiciones de Sargón, Moisés\footnote{\textit{Niños abandonados en ríos (Moisés)}: Ex 2:1-10.}, Ciro y Rómulo. Luego se estableció la práctica de consagrar a los hijos primogénitos como sagrados o sacrificatorios, permitiéndoles crecer y después los exiliaban en lugar de quitarles la vida; éste fue el origen de la colonización. Los romanos adoptaron esta costumbre en sus proyectos de colonización.

\par
%\textsuperscript{(982.3)}
\textsuperscript{89:7.4} Muchas asociaciones peculiares entre el libertinaje sexual y la adoración primitiva tuvieron su origen en conexión con los sacrificios humanos. En los tiempos antiguos, si una mujer se encontraba con los cazadores de cabezas, podía salvar su vida entregándose sexualmente a ellos. Más tarde, una doncella destinada a ser sacrificada a los dioses podía elegir recomprar su vida, dedicando su cuerpo de por vida al servicio sexual sagrado del templo; de esta manera podía ganar el dinero de su redención. Los antiguos consideraban que era algo muy elevado mantener relaciones sexuales con una mujer dedicada así a rescatar su vida. Tener trato con estas doncellas sagradas era una ceremonia religiosa, y todo este ritual proporcionaba además una excusa aceptable para las satisfacciones sexuales corrientes. Era una manera sutil de engañarse a sí mismo, que tanto a las doncellas como a sus parejas les encantaba practicar. Las costumbres siempre se quedan rezagadas con respecto al progreso evolutivo de la civilización, tolerando así las prácticas sexuales más primitivas y salvajes de las razas en evolución.

\par
%\textsuperscript{(982.4)}
\textsuperscript{89:7.5} La prostitución en los templos se extendió finalmente por toda Europa del sur y Asia. El dinero que ganaban las prostitutas de los templos era considerado como sagrado por todos los pueblos ---un regalo valioso para ofrecerlo a los dioses. Las mujeres de tipo superior atestaban el mercado sexual del templo y dedicaban sus ganancias a todo tipo de servicios sagrados y de obras de utilidad pública. Muchas mujeres de las mejores clases acumulaban su dote mediante un servicio sexual temporal en los templos, y la mayoría de los hombres preferían tener como esposas a estas mujeres.

\section*{8. La redención y las alianzas}
\par
%\textsuperscript{(982.5)}
\textsuperscript{89:8.1} La redención sacrificatoria y la prostitución en los templos eran en realidad modificaciones de los sacrificios humanos. Después se estableció el sacrificio simulado de las hijas. Esta ceremonia consistía en una sangría, acompañada de la dedicación a la virginidad durante toda la vida, y fue una reacción moral contra la antigua prostitución en los templos. En una época más reciente, las vírgenes se dedicaron al servicio de vigilar los fuegos sagrados de los templos.

\par
%\textsuperscript{(982.6)}
\textsuperscript{89:8.2} Los hombres concibieron finalmente la idea de que la ofrenda de una parte del cuerpo podía sustituir al antiguo sacrificio humano completo. Se consideró que la mutilación física era también un sustituto aceptable. Se sacrificaban los cabellos, las uñas, la sangre e incluso los dedos de las manos y de los pies. El antiguo rito posterior y casi universal de la circuncisión\footnote{\textit{Circuncisión}: Gn 17:10-14.} fue una consecuencia del culto del sacrificio parcial; era simplemente sacrificatorio y no se le atribuía ninguna finalidad higiénica. A los hombres los circuncidaban y a las mujeres les agujereaban las orejas.

\par
%\textsuperscript{(983.1)}
\textsuperscript{89:8.3} Posteriormente se estableció la costumbre de atarse los dedos en lugar de amputárselos. Afeitarse la cabeza y cortarse el pelo fueron igualmente unas formas de devoción religiosa. La castración fue al principio una modificación de la idea de los sacrificios humanos. En África se practica todavía el agujerear la nariz y los labios, y el tatuaje es una evolución artística de las brutales cicatrices que primitivamente se hacían en el cuerpo.

\par
%\textsuperscript{(983.2)}
\textsuperscript{89:8.4} Como consecuencia de enseñanzas más avanzadas, la costumbre de los sacrificios se asoció finalmente con la idea de la alianza. Al final se concibió que los dioses efectuaban verdaderos acuerdos con los hombres; éste fue un paso importante en la estabilización de la religión. La ley, una alianza, sustituyó a la suerte, al miedo y a la superstición\footnote{\textit{La alianza reemplaza al miedo}: Gn 6:18; 9:9,11; 17:4,7-10,21.}.

\par
%\textsuperscript{(983.3)}
\textsuperscript{89:8.5} El hombre nunca había podido soñar siquiera con celebrar un contrato con la Deidad hasta que su concepto de Dios hubo avanzado hasta el nivel en que imaginó que los controladores del universo eran dignos de confianza. La idea primitiva que el hombre tenía de Dios era tan antropomorfa que fue incapaz de concebir una Deidad digna de confianza hasta que él mismo no se volvió relativamente digno de confianza, moral y ético.

\par
%\textsuperscript{(983.4)}
\textsuperscript{89:8.6} Pero la idea de efectuar un pacto con los dioses acabó por llegar. \textit{El hombreevolutivo adquirió finalmente la dignidad moral suficiente como para atreversea negociar con sus dioses}. Y así, el asunto de ofrecer sacrificios se transformó gradualmente en el juego del regateo filosófico del hombre con Dios. Todo esto representaba una nueva estratagema para asegurarse contra la mala suerte, o más bien una técnica mejor para obtener con más seguridad la prosperidad. No alberguéis la idea errónea de que estos sacrificios primitivos eran regalos que se ofrecían gratuitamente a los dioses, unas ofrendas espontáneas de gratitud o de acción de gracias; no eran expresiones de auténtica adoración.

\par
%\textsuperscript{(983.5)}
\textsuperscript{89:8.7} Las formas primitivas de oración no eran ni más ni menos que unos regateos con los espíritus, una discusión con los dioses. Era una especie de trueque en el que las súplicas y la persuasión fueron sustituidas por algo más tangible y costoso. El desarrollo del comercio entre las razas había inculcado el espíritu comercial y había desarrollado la astucia en los trueques; estas características empezaron a aparecer entonces en los métodos de adoración del hombre. Al igual que algunos hombres eran mejores comerciantes que otros, también se consideraba que algunos rezadores eran mejores que otros. La oración de un hombre justo se tenía en gran estima\footnote{\textit{Oración del hombre justo}: 2 Cr 30:27; Stg 5:16.}. El hombre justo era aquel que había saldado todas sus deudas con los espíritus, que había cumplido plenamente con todas sus obligaciones rituales hacia los dioses.

\par
%\textsuperscript{(983.6)}
\textsuperscript{89:8.8} La oración primitiva se parecía poco a la adoración; era una petición negociadora para conseguir la salud, la riqueza y la vida. En numerosos aspectos, las oraciones no han cambiado mucho con el paso de los siglos. Continúan leyéndose en voz alta en los libros, recitándose de manera solemne, y copiándose para colocarlas en las ruedas y colgarlas en los árboles, donde el soplido de los vientos ahorra al hombre la molestia de emplear su propio aliento.

\section*{9. Los sacrificios y los sacramentos}
\par
%\textsuperscript{(983.7)}
\textsuperscript{89:9.1} En el transcurso de la evolución de los rituales urantianos, los sacrificios humanos han progresado desde las manifestaciones sangrientas de la antropofagia hasta unos niveles superiores y más simbólicos. Los ritos primitivos de los sacrificios engendraron las ceremonias posteriores de los sacramentos. En tiempos más recientes, el sacerdote era el único que tomaba un trozo del sacrificio caníbal o una gota de sangre humana, y luego todos los asistentes comían el animal sustitutorio. Estas ideas primitivas sobre el rescate, la redención y las alianzas han evolucionado hasta convertirse en los servicios sacramentales de nuestros días. Toda esta evolución ceremonial ha ejercido una enorme influencia socializadora.

\par
%\textsuperscript{(984.1)}
\textsuperscript{89:9.2} En conexión con el culto de la Madre de Dios, en Méjico y en otros lugares se utilizó finalmente un sacramento de pasteles y vino, en lugar de la carne y la sangre de los antiguos sacrificios humanos. Los hebreos practicaron durante mucho tiempo este ritual como parte de sus ceremonias pascuales, y en este ceremonial es donde tuvo su origen la versión cristiana posterior del sacramento.

\par
%\textsuperscript{(984.2)}
\textsuperscript{89:9.3} Las antiguas fraternidades sociales estaban basadas en el rito de beber sangre; la fraternidad judía primitiva era un sacrificio de sangre. Pablo empezó a construir un nuevo culto cristiano sobre <<la sangre de la alianza eterna>>\footnote{\textit{La sangre de la alianza eterna}: Heb 13:20.}. Y aunque haya sobrecargado innecesariamente el cristianismo con enseñanzas sobre la sangre y el sacrificio, puso fin de una vez por todas a las doctrinas de la redención a través de los sacrificios humanos o de animales. Sus compromisos teológicos indican que incluso la revelación debe someterse al control gradual de la evolución. Según Pablo, Cristo fue el sacrificio humano último y definitivo\footnote{\textit{Jesús visto como sacrificio humano}: 1 Co 5:7; Ef 5:2; Col 1:14; Tit 2:14; Heb 9:11-28; 10:1-20; 13:12; 1 P 1:18-19.}; el Juez divino está ahora plenamente satisfecho para siempre.

\par
%\textsuperscript{(984.3)}
\textsuperscript{89:9.4} Y así, después de largos milenios, el culto del sacrificio se ha convertido por evolución en el culto del sacramento\footnote{\textit{Culto al sacramento}: 1 Co 11:23-27.}. Los sacramentos de las religiones modernas son así los sucesores legítimos de aquellas horribles ceremonias primitivas de sacrificios humanos y de los rituales caníbales aún más primitivos. Muchas personas cuentan todavía con la sangre para salvarse, pero ésta se ha vuelto al menos figurativa, simbólica y mística.

\section*{10. El perdón de los pecados}
\par
%\textsuperscript{(984.4)}
\textsuperscript{89:10.1} Los hombres antiguos sólo llegaban a tener conciencia del favor de Dios a través del sacrificio. Los hombres modernos deben desarrollar unas técnicas nuevas para alcanzar la conciencia personal de la salvación. La conciencia del pecado subsiste en la mente de los mortales, pero los modelos de pensamiento sobre cómo salvarse del pecado se han vuelto caducos y anticuados. La realidad de la necesidad espiritual subsiste, pero el progreso intelectual ha destruido las antiguas maneras de conseguir la paz y el consuelo para la mente y el alma.

\par
%\textsuperscript{(984.5)}
\textsuperscript{89:10.2} \textit{Hay que volver a definir el pecado como una deslealtad deliberada haciala Deidad}. Existen diversos grados de deslealtad: la lealtad parcial debida a la indecisión; la lealtad dividida debida a los conflictos; la lealtad moribunda debida a la indiferencia y la muerte de la lealtad, que se manifiesta en la consagración a los ideales impíos.

\par
%\textsuperscript{(984.6)}
\textsuperscript{89:10.3} El sentido o sentimiento de culpa es la conciencia de haber violado las costumbres; no es necesariamente un pecado. No existe pecado real en ausencia de una deslealtad consciente hacia la Deidad.

\par
%\textsuperscript{(984.7)}
\textsuperscript{89:10.4} La posibilidad de reconocer el sentimiento de culpa es una señal de distinción trascendente para la humanidad. No califica al hombre de despreciable, sino más bien lo separa como una criatura de una grandeza potencial y de una gloria siempre ascendente. Ese sentimiento de indignidad es el estímulo inicial que debería conducir de manera rápida y segura a esas conquistas de la fe que trasladan a la mente mortal a los magníficos niveles de la nobleza moral, la perspicacia cósmica y la vida espiritual; todos los significados de la existencia humana cambian así de lo temporal a lo eterno, y todos los valores se elevan de lo humano a lo divino.

\par
%\textsuperscript{(984.8)}
\textsuperscript{89:10.5} La confesión del pecado es un rechazo valiente de la deslealtad, pero no atenúa de ninguna manera las consecuencias espacio-temporales de esa deslealtad. Pero la confesión ---el reconocimiento sincero de la naturaleza del pecado--- es esencial para el crecimiento religioso y el progreso espiritual.

\par
%\textsuperscript{(985.1)}
\textsuperscript{89:10.6} Cuando los pecados son perdonados por la Deidad, se produce la reanudación de las relaciones leales después de un período durante el cual el hombre es consciente de la interrupción de dichas relaciones como consecuencia de una rebelión consciente. No es necesario buscar el perdón, sino únicamente recibirlo teniendo conciencia del restablecimiento de las relaciones leales entre la criatura y el Creador. Y todos los hijos leales de Dios son felices, aman el servicio y progresan constantemente en la ascensión hacia el Paraíso.

\par
%\textsuperscript{(985.2)}
\textsuperscript{89:10.7} [Presentado por una Brillante Estrella Vespertina de Nebadon.]


\chapter{Documento 90. El chamanismo ---los curanderos y los sacerdotes}
\par
%\textsuperscript{(986.1)}
\textsuperscript{90:0.1} LA EVOLUCIÓN de las prácticas religiosas progresó desde el apaciguamiento, la evitación, el exorcismo, la coacción, la conciliación y la propiciación hasta el sacrificio, la expiación y la redención. La técnica del ritual religioso pasó desde las formas del culto primitivo, a través de los fetiches, hasta la magia y los milagros. A medida que el ritual se volvió más complejo en respuesta al concepto cada vez más complejo que el hombre se formaba de los reinos supermateriales, estuvo inevitablemente dominado por los curanderos, los chamanes y los sacerdotes.

\par
%\textsuperscript{(986.2)}
\textsuperscript{90:0.2} El hombre primitivo terminó por considerar, en sus conceptos progresivos, que el mundo de los espíritus era insensible hacia los mortales corrientes. Únicamente los seres humanos excepcionales podían atraer la atención de los dioses; sólo el hombre o la mujer extraordinarios podían ser escuchados por los espíritus. La religión entra así en una nueva fase, en una etapa en la que se vuelve gradualmente de segunda mano; un curandero, un chamán o un sacerdote interviene siempre entre la persona religiosa y el objeto de su adoración. Hoy día, la mayor parte de los sistemas urantianos de creencias religiosas organizadas están pasando por este nivel de desarrollo evolutivo.

\par
%\textsuperscript{(986.3)}
\textsuperscript{90:0.3} La religión evolutiva nace de un miedo simple y todopoderoso, el miedo que se apodera de la mente humana cuando ésta se enfrenta a lo desconocido, lo inexplicable y lo incomprensible. La religión alcanza finalmente la comprensión profundamente sencilla de un amor todopoderoso, el amor que invade irresistiblemente el alma humana cuando ésta se despierta a la idea del afecto ilimitado del Padre Universal por los hijos del universo. Pero entre el comienzo y la consumación de la evolución religiosa se encuentran las largas épocas de los chamanes, los cuales se atreven a colocarse entre el hombre y Dios como intermediarios, intérpretes e intercesores.

\section*{1. Los primeros chamanes ---los curanderos}
\par
%\textsuperscript{(986.4)}
\textsuperscript{90:1.1} El chamán era el curandero de mayor categoría, el hombre fetiche de las ceremonias y la personalidad central en todas las prácticas de la religión evolutiva. En muchos grupos, el chamán estaba jerárquicamente por encima del jefe militar, señalando el comienzo del dominio de la iglesia sobre el Estado. El chamán actuaba a veces como sacerdote\footnote{\textit{Chamán sacerdote}: Ex 2:16; 28:1-43; 29:4-9,29; Nm 18:1-10.} e incluso como sacerdote-rey. Algunas tribus posteriores tuvieron al mismo tiempo a los chamanes-curanderos (videntes\footnote{\textit{Videntes}: 2 Re 17:13; 1 Cr 29:29; 1 Sam 9:8-11,18-19; 2 Sam 15:27.}) iniciales y a los chamanes-sacerdotes que aparecieron después. En muchos casos, el cargo de chamán se volvió hereditario.

\par
%\textsuperscript{(986.5)}
\textsuperscript{90:1.2} Puesto que en los tiempos antiguos cualquier cosa anormal era atribuida a la posesión por los espíritus, cualquier anormalidad mental o física notable constituía una aptitud para ser curandero. Muchos de estos hombres eran epilépticos, muchas mujeres eran histéricas, y estos dos tipos explican una gran parte de la inspiración antigua así como la posesión por los espíritus y los demonios. Un gran número de estos sacerdotes más primitivos pertenecían a una clase que desde entonces se ha denominado paranoica.

\par
%\textsuperscript{(987.1)}
\textsuperscript{90:1.3} Aunque puedan haber practicado el engaño en asuntos menores, la gran mayoría de los chamanes creían en el hecho de que estaban poseídos por los espíritus. Las mujeres que eran capaces de caer en trance o en un ataque cataléptico se volvieron poderosas chamanesas; más tarde, estas mujeres fueron profetisas y médiums espiritistas. Sus trances catalépticos consistían habitualmente en supuestas comunicaciones con los fantasmas de los muertos. Muchas chamanesas eran también bailarinas profesionales.

\par
%\textsuperscript{(987.2)}
\textsuperscript{90:1.4} Pero no todos los chamanes se engañaban a sí mismos; muchos eran unos estafadores hábiles y astutos. A medida que se desarrolló la profesión, a los principiantes se les exigió que hicieran un aprendizaje de diez años de dificultades y de abnegación para capacitarse como curanderos. Los chamanes desarrollaron una manera profesional de vestirse y adoptaban una conducta misteriosa. Empleaban drogas con frecuencia para provocar ciertos estados físicos que solían impresionar y desconcertar a los miembros de su tribu. La gente común consideraba que las proezas de la prestidigitación eran sobrenaturales, y algunos sacerdotes astutos utilizaron por primera vez la ventriloquia. Muchos chamanes antiguos descubrieron sin querer el hipnotismo; otros se provocaban la autohipnosis mirándose fijamente el ombligo durante largo tiempo.

\par
%\textsuperscript{(987.3)}
\textsuperscript{90:1.5} Aunque muchos de ellos recurrieron a estos trucos y engaños, su reputación como clase se basaba después de todo en sus éxitos aparentes. Cuando un chamán fracasaba en su empresa, si no podía presentar una coartada plausible, lo degradaban o bien lo mataban. Así pues, los chamanes honrados perecieron pronto; sólo sobrevivieron los actores astutos.

\par
%\textsuperscript{(987.4)}
\textsuperscript{90:1.6} El chamanismo fue el que quitó a los ancianos y a los fuertes la dirección exclusiva de los asuntos de la tribu, y la puso en manos de los astutos, los hábiles y los perspicaces.

\section*{2. Las prácticas chamanísticas}
\par
%\textsuperscript{(987.5)}
\textsuperscript{90:2.1} El conjuro de los espíritus era un procedimiento muy preciso y bastante complicado, comparable a los rituales eclesiásticos actuales dirigidos en una lengua antigua. La raza humana buscó muy pronto la ayuda sobrehumana, la \textit{revelación}, y los hombres creían que los chamanes recibían realmente estas revelaciones. Aunque los chamanes utilizaban en su trabajo el gran poder de la sugestión, se trataba casi invariablemente de una sugestión negativa; la técnica de la sugestión positiva sólo se ha empleado en tiempos muy recientes. Al principio del desarrollo de su profesión, los chamanes empezaron a especializarse en labores tales como provocar la lluvia, curar las enfermedades y detectar los crímenes. Sin embargo, curar las enfermedades no era la ocupación principal de un curandero chamánico; ésta consistía más bien en conocer y controlar los riesgos de la vida.

\par
%\textsuperscript{(987.6)}
\textsuperscript{90:2.2} La antigua magia negra, tanto religiosa como laica, se llamaba magia blanca cuando la practicaban los sacerdotes, los videntes, los chamanes o los curanderos. Los que practicaban la magia negra eran calificados de brujos, magos, hechiceros, brujas, encantadores, nigromantes, prestidigitadores y adivinos. A medida que pasó el tiempo, todos estos pretendidos contactos con lo sobrenatural fueron clasificados como brujería o bien como chamanismo.

\par
%\textsuperscript{(987.7)}
\textsuperscript{90:2.3} La brujería\footnote{\textit{Brujería}: Ex 22:18; Dt 18:10-12; 1 Sam 15:23.} abarcaba la \textit{magia} que realizaban los espíritus primitivos, irregulares y no identificados; el chamanismo estaba relacionado con los \textit{milagros} que realizaban los espíritus regulares y los dioses reconocidos de la tribu. En tiempos posteriores, las brujas fueron relacionadas con el diablo, y el escenario estuvo así preparado para las numerosas manifestaciones relativamente recientes de intolerancia religiosa. La brujería era una religión para muchas tribus primitivas.

\par
%\textsuperscript{(987.8)}
\textsuperscript{90:2.4} Los chamanes creían profundamente en la misión de la casualidad como reveladora de la voluntad de los espíritus; con frecuencia lo echaban a suertes para llegar a una decisión. Las supervivencias modernas de esta tendencia a echarlo a suertes no sólo se encuentran en los numerosos juegos de azar, sino también en las canciones <<eliminatorias>> infantiles bien conocidas. Antiguamente, la persona eliminada debía morir; ahora se limitan a decir \textit{túte quedas} en algunos juegos infantiles. Aquello que constituía un asunto serio para los hombres primitivos, ha sobrevivido como una diversión para los niños modernos.

\par
%\textsuperscript{(988.1)}
\textsuperscript{90:2.5} Los curanderos tenían una gran confianza en los signos y los presagios tales como <<Cuando oigas el ruido de un susurro en las copas de las moreras, entonces muévete>>\footnote{\textit{Cuando oigas el sonido del viento}: 1 Cr 14:15; 2 Sam 5:24.}. Muy pronto en la historia de la raza, los chamanes dirigieron su atención hacia las estrellas. La astrología\footnote{\textit{Astrología}: Is 47:13; Dn 2:2; Am 5:26; Mt 2:7,9-10.} primitiva se creía y se practicaba en todo el mundo; la interpretación de los sueños también se difundió ampliamente. Todo esto fue pronto seguido por la aparición de las chamanesas\footnote{\textit{Chamanesas}: 1 Sam 28:7-19.} inestables que pretendían poder comunicarse con los espíritus de los muertos.

\par
%\textsuperscript{(988.2)}
\textsuperscript{90:2.6} Aunque su origen es antiguo, los artífices de la lluvia, o chamanes del tiempo, han sobrevivido a lo largo de todas las épocas. Una grave sequía significaba la muerte para los agricultores primitivos; controlar el tiempo era el objetivo de una gran parte de la magia antigua. Los hombres civilizados aún hacen del tiempo un tema corriente de conversación. Todos los pueblos antiguos creían en el poder del chamán como artífice de la lluvia, pero tenían la costumbre de matarlo cuando fracasaba, a menos que pudiera ofrecer una excusa plausible que justificara su fracaso.

\par
%\textsuperscript{(988.3)}
\textsuperscript{90:2.7} Los césares desterraron a los astrólogos una y otra vez, pero éstos volvieron invariablemente a causa de la creencia popular en sus poderes. No pudieron expulsarlos, e incluso en el siglo dieciséis después de Cristo, los administradores de la iglesia y de los Estados occidentales eran los patrocinadores de la astrología. Miles de personas supuestamente inteligentes creen todavía que uno puede nacer bajo el dominio de una buena o mala estrella\footnote{\textit{La buena estrella}: Mt 2:2.}, que la yuxtaposición de los cuerpos celestes determina el resultado de las diversas aventuras terrestres. Los adivinos cuentan todavía con el favor de los crédulos.

\par
%\textsuperscript{(988.4)}
\textsuperscript{90:2.8} Los griegos creían en la eficacia del consejo de los oráculos, los chinos utilizaban la magia para protegerse contra los demonios, el chamanismo floreció en la India, y todavía sobrevive abiertamente en Asia central. Es una práctica que sólo se ha abandonado recientemente en una gran parte del mundo.

\par
%\textsuperscript{(988.5)}
\textsuperscript{90:2.9} De vez en cuando surgieron auténticos profetas e instructores para denunciar y desenmascarar al chamanismo. Incluso los hombres rojos en vías de desaparición tuvieron un profeta de este tipo en los últimos cien años, el tenskwatawa shawnee, que predijo el eclipse de Sol de 1806 y denunció los vicios del hombre blanco. Muchos verdaderos educadores han aparecido en las diversas tribus y razas durante las largas épocas de la historia evolutiva. Y continuarán apareciendo siempre para desafiar a los chamanes o los sacerdotes de cualquier época que se opongan a la educación general e intenten contrarrestar el progreso científico.

\par
%\textsuperscript{(988.6)}
\textsuperscript{90:2.10} Los antiguos chamanes establecieron su reputación como voces de Dios y guardianes de la providencia de muchas maneras y por métodos tortuosos. Asperjaban con agua a los recién nacidos y les conferían el nombre; circuncidaban a los varones. Presidían todas las ceremonias fúnebres y anunciaban debidamente la feliz llegada de los muertos al reino de los espíritus.

\par
%\textsuperscript{(988.7)}
\textsuperscript{90:2.11} Los sacerdotes y curanderos chamánicos se volvieron a menudo muy ricos debido a la acumulación de sus diversos honorarios que eran, aparentemente, ofrendas para los espíritus. No era raro que un chamán acumulara prácticamente toda la riqueza material de su tribu. Cuando moría un hombre rico, se tenía la costumbre de dividir sus bienes por igual entre el chamán y alguna empresa pública u obra de beneficencia. Esta práctica existe todavía en algunas partes del Tíbet, donde la mitad de la población masculina pertenece a esta clase de no productores.

\par
%\textsuperscript{(989.1)}
\textsuperscript{90:2.12} Los chamanes se vestían bien y tenían generalmente varias esposas; fueron la aristocracia original, y estaban exentos de todas las restricciones tribales. Su mente y su moral eran con mucha frecuencia de baja calidad. Suprimían a sus rivales acusándolos de brujas o brujos, y ascendían muy a menudo a tales posiciones de influencia y de poder que podían dominar a los jefes o a los reyes.

\par
%\textsuperscript{(989.2)}
\textsuperscript{90:2.13} El hombre primitivo consideraba al chamán como un mal necesario; le tenía miedo pero no le amaba. El hombre primitivo respetaba el conocimiento; honraba y premiaba la sabiduría. El chamán era la mayoría de las veces un impostor, pero la veneración por el chamanismo ilustra muy bien el gran valor que se daba a la sabiduría en la evolución de la raza.

\section*{3. La teoría chamánica de la enfermedad y la muerte}
\par
%\textsuperscript{(989.3)}
\textsuperscript{90:3.1} Puesto que el hombre de la antig\"uedad consideraba que él mismo y su entorno material eran directamente sensibles a los caprichos de los fantasmas y a los antojos de los espíritus, no es de extrañar que su religión se ocupara tan exclusivamente de los asuntos materiales. El hombre moderno ataca directamente sus problemas materiales; reconoce que la materia es sensible a la manipulación inteligente de la mente. El hombre primitivo deseaba también modificar, e incluso controlar, la vida y las energías del ámbito físico; y puesto que su comprensión limitada del cosmos le condujo a creer que los fantasmas, los espíritus y los dioses se ocupaban personal y directamente del control pormenorizado de la vida y la materia, dirigió lógicamente sus esfuerzos a conseguir el favor y el apoyo de estos agentes superhumanos.

\par
%\textsuperscript{(989.4)}
\textsuperscript{90:3.2} Considerado desde este punto de vista, una gran parte de los elementos inexplicables e irracionales de los cultos antiguos se vuelve comprensible. Las ceremonias del culto eran las tentativas del hombre primitivo por controlar el mundo material en el cual se encontraba. Y una gran parte de sus esfuerzos estaban dirigidos hacia el objetivo de prolongar la vida y asegurar la salud. Puesto que todas las enfermedades y la muerte misma fueron consideradas en un principio como fenómenos causados por los espíritus, era inevitable que los chamanes, a la vez que ejercían como curanderos y sacerdotes, trabajaran también como médicos y cirujanos.

\par
%\textsuperscript{(989.5)}
\textsuperscript{90:3.3} La mente primitiva puede encontrarse en situación de inferioridad por falta de datos, pero a pesar de todo ello es lógica. Cuando los hombres reflexivos observan la enfermedad y la muerte, se dedican a determinar las causas de estas calamidades, y de acuerdo con su comprensión, los chamanes y los científicos han propuesto las siguientes teorías sobre la aflicción:

\par
%\textsuperscript{(989.6)}
\textsuperscript{90:3.4} 1. \textit{Los fantasmas} ---\textit{las influencias directas de los espíritus}. La hipótesis más primitiva que se ofreció para explicar la enfermedad y la muerte fue que los espíritus causaban las enfermedades atrayendo el alma fuera del cuerpo; si ésta no regresaba, se producía la muerte. Los antiguos temían tanto la actividad malévola de los fantasmas productores de enfermedades, que solían abandonar a menudo a las personas enfermas sin dejarles siquiera ni alimentos ni agua. Sin tener en cuenta las bases erróneas de estas creencias, éstas aislaban eficazmente a las personas aquejadas e impedían la propagación de las enfermedades contagiosas.

\par
%\textsuperscript{(989.7)}
\textsuperscript{90:3.5} 2. \textit{La violencia} ---\textit{las causas evidentes}. Las causas de algunos accidentes y fallecimientos eran tan fáciles de identificar que fueron pronto eliminadas de la categoría de las actividades de los fantasmas. Las calamidades y las heridas que acompañaban a la guerra, los combates con los animales y otros agentes fácilmente identificables fueron consideradas como sucesos naturales. Pero durante mucho tiempo se creyó que los espíritus seguían siendo responsables del retraso de las curaciones o de la infección de las heridas producidas incluso por una causa <<natural>>. Si no se podía descubrir ningún agente natural observable, los fantasmas espíritus seguían siendo considerados como responsables de la enfermedad y la muerte.

\par
%\textsuperscript{(990.1)}
\textsuperscript{90:3.6} Hoy se pueden encontrar, en África y en otros lugares, pueblos primitivos que matan a alguien cada vez que se produce una muerte no violenta. Sus curanderos les indican quiénes son los individuos culpables. Si una madre muere de parto, el niño es estrangulado inmediatamente ---vida por vida.

\par
%\textsuperscript{(990.2)}
\textsuperscript{90:3.7} 3. \textit{La magia} ---\textit{la influencia de los enemigos}. Se creía que muchas enfermedades eran causadas por los hechizos, por la acción del mal de ojo y la inclinación mágica señalando a alguien. En cierta época era realmente peligroso señalar con el dedo a una persona; todavía se considera que señalar es de mala educación. En los casos de enfermedad y de muerte poco claras, los antiguos solían realizar una investigación oficial, diseccionaban el cuerpo y, basándose en algún descubrimiento, decidían que éste era la causa de la muerte; de lo contrario, la muerte solía atribuírse a la brujería, siendo necesario ejecutar entonces a la bruja responsable. Estas antiguas investigaciones judiciales salvaron la vida de muchas supuestas brujas. En algunas tribus se creía que un hombre podía morir a consecuencia de su propia brujería, en cuyo caso no se acusaba a nadie.

\par
%\textsuperscript{(990.3)}
\textsuperscript{90:3.8} 4. \textit{El pecado} ---\textit{el castigo por la violación de un tabú}. En una época relativamente reciente se ha creído que la enfermedad es un castigo por el pecado, personal o racial. Entre los pueblos que atraviesan este nivel de evolución, la teoría predominante es que uno no puede sufrir a menos que haya violado un tabú. Una forma típica de estas creencias consiste en considerar que la enfermedad y el sufrimiento son las <<flechas del Todopoderoso dentro del cuerpo>>\footnote{\textit{Flechas del Todopoderoso}: Job 6:4; Sal 38:1-2.}. Los chinos y los mesopotámicos consideraron durante mucho tiempo que la enfermedad era el resultado de la actividad de los demonios malignos, aunque los caldeos también estimaban que las estrellas eran la causa del sufrimiento. Esta teoría de que la enfermedad es la consecuencia de la cólera divina predomina todavía entre muchos grupos de urantianos supuestamente civilizados.

\par
%\textsuperscript{(990.4)}
\textsuperscript{90:3.9} 5. \textit{Las causas naturales}. La humanidad ha aprendido muy lentamente los secretos materiales de la relación entre las causas y los efectos en los ámbitos físicos de la energía, la materia y la vida. Los antiguos griegos, que habían conservado las tradiciones de las enseñanzas de Adanson, figuran entre los primeros en reconocer que todas las enfermedades son el resultado de unas causas naturales. El desarrollo de la era científica está destruyendo de manera lenta pero segura las teorías seculares del hombre sobre la enfermedad y la muerte. La fiebre fue uno de los primeros malestares humanos que se eliminaron de la categoría de los desórdenes sobrenaturales, y la era de la ciencia ha roto progresivamente las cadenas de la ignorancia que tanto tiempo han aprisionado a la mente humana. La comprensión de la vejez y del contagio está destruyendo gradualmente el miedo del hombre a los fantasmas, los espíritus y los dioses como autores personales de las desgracias humanas y del sufrimiento de los mortales.

\par
%\textsuperscript{(990.5)}
\textsuperscript{90:3.10} La evolución consigue infaliblemente sus fines: Infunde al hombre ese temor supersticioso a lo desconocido y ese terror a lo invisible que constituyen el andamiaje para alcanzar el concepto de Dios. Y después de haber presenciado el nacimiento de una comprensión avanzada de la Deidad, mediante la acción coordinada de la revelación, esta misma técnica de la evolución pone entonces infaliblemente en movimiento esas fuerzas del pensamiento que destruirán inexorablemente el andamiaje, que ha cumplido con su misión.

\section*{4. La medicina bajo el dominio de los chamanes}
\par
%\textsuperscript{(990.6)}
\textsuperscript{90:4.1} Toda la vida de los hombres antiguos estaba basada en la prevención; su religión era en gran medida una técnica para prevenir las enfermedades. A pesar del error de sus teorías, las ponían sinceramente en práctica; tenían una fe ilimitada en sus métodos de tratamiento y esto, en sí mismo, es un poderoso remedio.

\par
%\textsuperscript{(991.1)}
\textsuperscript{90:4.2} La fe que se necesitaba para restablecerse con los cuidados descabellados de uno de estos antiguos chamanes no era, después de todo, materialmente diferente de la que se necesita para experimentar la curación por obra de alguno de sus sucesores más recientes que se dedican a tratar las enfermedades de manera no científica.

\par
%\textsuperscript{(991.2)}
\textsuperscript{90:4.3} Las tribus más primitivas tenían mucho miedo a los enfermos, y durante largas épocas los evitaron cuidadosamente, los desatendieron vergonzosamente. El humanitarismo avanzó enormemente cuando la evolución del chamanismo dio nacimiento a sacerdotes y curanderos que consintieron en tratar las enfermedades. Entonces todo el clan cogió la costumbre de reunirse en el cuarto del enfermo para ayudar al chamán a expulsar a gritos a los fantasmas de la enfermedad. No era raro que el chamán que hacía el diagnóstico fuera una mujer, mientras que un hombre administraba el tratamiento. El método habitual para diagnosticar las enfermedades consistía en examinar las entrañas de un animal.

\par
%\textsuperscript{(991.3)}
\textsuperscript{90:4.4} La enfermedad se trataba por medio de cantos, gritos, imposiciones de manos, soplando sobre el paciente y otras muchas técnicas. En tiempos posteriores se recurrió a que el enfermo durmiera en el templo, suponiéndose que durante ese período se producía la curación, y esta costumbre se difundió mucho. Los curanderos terminaron por intentar verdaderas operaciones quirúrgicas en conexión con el sueño en el templo; una de las primeras operaciones consistió en trepanar el cráneo para permitir que huyera el espíritu que producía el dolor de cabeza. Los chamanes aprendieron a tratar las fracturas y las dislocaciones, a abrir los furúnculos y los abscesos; las chamanesas se volvieron comadronas expertas.

\par
%\textsuperscript{(991.4)}
\textsuperscript{90:4.5} Un método corriente de tratamiento consistía en frotar alguna cosa mágica sobre una parte infectada o manchada del cuerpo, arrojar fuera el amuleto, y suponer que se producía la curación. Si alguien recogía por casualidad el amuleto desechado, se creía que contraía inmediatamente la infección o la mancha. Pasó mucho tiempo antes de que se introdujeran las hierbas y otros verdaderos medicamentos. El masaje se desarrolló en conexión con el conjuro, frotando el cuerpo para expulsar al espíritu, y estuvo precedido por los esfuerzos para aplicar los medicamentos mediante fricciones, al igual que los modernos intentan hacer penetrar los linimentos frotando. Se creía que aplicar ventosas y chupar las partes afectadas, así como la sangría, eran valiosos para desembarazarse de un espíritu causante de enfermedades.

\par
%\textsuperscript{(991.5)}
\textsuperscript{90:4.6} Puesto que el agua era un poderoso fetiche, se utilizaba en el tratamiento de muchos malestares. Durante mucho tiempo se creyó que el espíritu que causaba la enfermedad se podía eliminar a través del sudor. Los baños de vapor eran muy apreciados; los manantiales naturales de agua caliente florecieron pronto como balnearios primitivos. El hombre primitivo descubrió que el calor solía aliviar el dolor; utilizó la luz del Sol, los órganos de los animales recién sacrificados, la arcilla caliente y las piedras recalentadas, y muchos de estos métodos se emplean todavía. Los ritmos se practicaban en un esfuerzo por influir sobre los espíritus; los tantanes eran universales.

\par
%\textsuperscript{(991.6)}
\textsuperscript{90:4.7} Algunos pueblos creían que la enfermedad era causada por una conspiración perversa entre los espíritus y los animales. Esto dio nacimiento a la creencia de que existía un remedio vegetal benéfico para cada una de las enfermedades causadas por los animales. Los hombres rojos eran especialmente fieles a la teoría de las plantas como remedios universales; siempre ponían una gota de sangre en el agujero que dejaba la raíz cuando arrancaban una planta.

\par
%\textsuperscript{(991.7)}
\textsuperscript{90:4.8} El ayuno, la dieta y los revulsivos se utilizaban a menudo como medidas curativas. Las secreciones humanas, como eran claramente mágicas, se tenían en gran estima; la sangre y la orina figuraron pues entre los primeros medicamentos, y pronto se añadieron las raíces y diversas sales. Los chamanes creían que se podía expulsar del cuerpo a los espíritus de la enfermedad con medicamentos nauseabundos y de mal gusto. Los purgantes se convirtieron muy pronto en un tratamiento rutinario, y los valores del cacao y de la quinina puros figuraron entre los primeros descubrimientos farmacéuticos.

\par
%\textsuperscript{(992.1)}
\textsuperscript{90:4.9} Los griegos fueron los primeros que desarrollaron unos métodos realmente racionales para curar a los enfermos. Tanto los griegos como los egipcios recibieron sus conocimientos médicos del valle del Éufrates. El aceite y el vino se utilizaron muy pronto como medicinas para curar las heridas; los sumerios empleaban el aceite de ricino y el opio. Muchos de estos remedios secretos, antiguos y eficaces, perdieron su poder cuando fueron conocidos; el secreto siempre ha sido esencial para practicar con éxito el engaño y la superstición. Sólo los hechos y la verdad buscan la plena luz de la comprensión y se regocijan con la iluminación y la aclaración de la investigación científica.

\section*{5. Los sacerdotes y los rituales}
\par
%\textsuperscript{(992.2)}
\textsuperscript{90:5.1} La esencia del ritual consiste en la perfección de su ejecución; entre los salvajes ha de practicarse con una precisión exacta. La ceremonia sólo posee un poder irresistible sobre los espíritus cuando el ritual ha sido realizado correctamente. Si el ritual es defectuoso, lo único que hace es despertar la ira y el resentimiento de los dioses. Por consiguiente, puesto que la mente en lenta evolución del hombre concebía que la \textit{técnica del ritual} era el factor decisivo para su eficacia, era inevitable que los primeros chamanes se convirtieran tarde o temprano en un clero entrenado para dirigir la práctica meticulosa del ritual. Y así, durante decenas de miles de años, los rituales interminables han obstaculizado a la sociedad y han afligido a la civilización, han sido una carga intolerable para cada acto de la vida, para cada empresa racial.

\par
%\textsuperscript{(992.3)}
\textsuperscript{90:5.2} El ritual es la técnica para santificar la costumbre; el ritual crea y perpetúa los mitos, al mismo tiempo que contribuye a conservar las costumbres sociales y religiosas. Además, el ritual mismo ha sido engendrado por los mitos. Al principio los rituales son a menudo sociales, luego se vuelven económicos y finalmente adquieren la santidad y la dignidad de un ceremonial religioso. La práctica del ritual puede ser personal o colectiva ---o las dos--- tal como lo ilustran la oración, la danza y las manifestaciones dramáticas.

\par
%\textsuperscript{(992.4)}
\textsuperscript{90:5.3} Las palabras se volvieron una parte del ritual, con la utilización de términos tales como amén\footnote{\textit{``Amen'' como ritual}: Nm 5:22.} y selah\footnote{\textit{``Selah''  como ritual}: Sal 3:2.}. La costumbre de decir palabrotas, la blasfemia, representa una prostitución de la antigua repetición ritual de los nombres sagrados. El hacer peregrinajes a los santuarios sagrados es un ritual muy antiguo. Los rituales se convirtieron después en complicadas ceremonias de purificación\footnote{\textit{Purificación: Ritual de la novilla roja}: Nm 19:1-22; \textit{del altar}: Lv 8:15; \textit{del botín de guerra}: Nm 31:12-24; \textit{de las vestiduras sacerdotales}: Ex 29:4; \textit{del tabernáculo}: Ex 29:41-44; \textit{de la puérpera}: Lv 12:1-7.}, limpieza\footnote{\textit{Limpieza: de los pecados de la congregación}: Lv 4:13-21; \textit{de los pecados individuales}: Lv 5:1-13; \textit{de leprosos}: Lv 14:2-11; \textit{de los delitos}: Lv 6:8-18.} y santificación\footnote{\textit{Sanctificación por unción}: Ex 40:9-11; \textit{por el Señor}: Lv 22:9,16; \textit{de los niños primogénitos}: Ex 13:2; \textit{de las personas}: Ex 19:10,14; \textit{del tabernáculo}: Ex 29:41-44.}. Las ceremonias de iniciación de las sociedades secretas de las tribus primitivas eran en realidad un rito religioso rudimentario. La técnica de adoración de los antiguos cultos de misterio era simplemente una larga representación de rituales religiosos acumulados. El ritual se convirtió finalmente en los tipos modernos de ceremonias sociales y de cultos religiosos, unos servicios que abarcan la oración, los cánticos, la lectura con respuestas y otras devociones espirituales individuales y colectivas.

\par
%\textsuperscript{(992.5)}
\textsuperscript{90:5.4} Los sacerdotes evolucionaron desde los chamanes, pasando por los oráculos, adivinos, cantores, bailarines, artífices del tiempo, guardianes de las reliquias religiosas, custodios de los templos y pronosticadores de acontecimientos, hasta el estado de auténticos directores del culto religioso. El cargo se volvió finalmente hereditario, y así surgió una casta sacerdotal permanente.

\par
%\textsuperscript{(992.6)}
\textsuperscript{90:5.5} A medida que evolucionaba la religión, los sacerdotes empezaron a especializarse de acuerdo con sus talentos innatos o sus predilecciones especiales. Algunos se volvieron cantores, otros rezadores y otros aún sacrificadores; más tarde aparecieron los oradores ---los predicadores. Y cuando la religión se institucionalizó, estos sacerdotes pretendieron <<poseer las llaves del cielo>>.

\par
%\textsuperscript{(992.7)}
\textsuperscript{90:5.6} Los sacerdotes siempre han intentado impresionar y atemorizar a la gente corriente, dirigiendo el ritual religioso en una lengua muerta y haciendo diversos pases mágicos tanto para desconcertar a los fieles como para realzar su propia piedad y autoridad. El gran peligro que tiene todo esto es que el ritual tiende a convertirse en el sustituto de la religión.

\par
%\textsuperscript{(993.1)}
\textsuperscript{90:5.7} Los cleros han contribuido mucho a retrasar el desarrollo científico y a entorpecer el progreso espiritual, pero han contribuido a estabilizar la civilización y a realzar ciertos tipos de cultura. Sin embargo, muchos sacerdotes modernos han dejado de ejercer como directores del ritual de la adoración de Dios, y han desviado su atención hacia la teología ---el intento por definir a Dios.

\par
%\textsuperscript{(993.2)}
\textsuperscript{90:5.8} No se puede negar que los sacerdotes han sido una piedra de molino atada al cuello de las razas, pero los verdaderos dirigentes religiosos han resultado inestimables señalando el camino hacia otras realidades más elevadas y mejores.

\par
%\textsuperscript{(993.3)}
\textsuperscript{90:5.9} [Presentado por un Melquisedek de Nebadon.]


\chapter{Documento 91. La evolución de la oración}
\par
%\textsuperscript{(994.1)}
\textsuperscript{91:0.1} LA ORACIÓN, como actividad de la religión, surgió de unas expresiones anteriores no religiosas consistentes en monólogos y diálogos. Cuando el hombre primitivo alcanzó la conciencia de sí mismo, se produjo la consecuencia inevitable de la conciencia de los demás, el doble potencial de la reacción hacia la sociedad y el reconocimiento de Dios.

\par
%\textsuperscript{(994.2)}
\textsuperscript{91:0.2} Las primeras formas de oración no estaban dirigidas a la Deidad. Estas expresiones se parecían mucho a lo que le diríais a un amigo en el momento de emprender una empresa importante: <<Deséame suerte>>. El hombre primitivo era esclavo de la magia; la suerte, buena o mala, formaba parte de todos los asuntos de la vida. Al principio, estas peticiones de suerte eran monólogos ---una especie de reflexión en voz alta del practicante de la magia. Luego, estos creyentes en la suerte buscaron el apoyo de sus amigos y familias, y poco después se realizaron ciertas formas de ceremonias que incluían a todo el clan o la tribu.

\par
%\textsuperscript{(994.3)}
\textsuperscript{91:0.3} Cuando los conceptos de los fantasmas y los espíritus evolucionaron, estas peticiones se dirigieron a las fuerzas superhumanas, y con la aparición de la conciencia de los dioses, estas expresiones alcanzaron los niveles de auténticas oraciones. Como ejemplo de esto, en algunas tribus de Australia las oraciones religiosas primitivas precedieron a la creencia en los espíritus y en las personalidades superhumanas.

\par
%\textsuperscript{(994.4)}
\textsuperscript{91:0.4} La tribu de los Todas de la India conserva actualmente esta práctica de no rezarle a nadie en particular, tal como lo hacían los pueblos primitivos antes de la época de la conciencia religiosa. Pero entre los Todas, esto representa un retroceso de su religión degenerativa hacia este nivel primitivo. Los rituales actuales de los sacerdotes lecheros de los Todas no equivalen a una ceremonia religiosa, ya que estas oraciones impersonales no contribuyen en nada a conservar ni a elevar los valores sociales, morales o espirituales.

\par
%\textsuperscript{(994.5)}
\textsuperscript{91:0.5} La oración prerreligiosa formaba parte de las prácticas mana de los melanesios, de las creencias oudah de los pigmeos africanos y de las supersticiones manitú de los indios norteamericanos. Las tribus baganda de África acaban de salir recientemente del nivel de oración mana. Durante esta confusión evolutiva primitiva, los hombres rezan a los dioses ---locales y nacionales--- a los fetiches, los amuletos, los fantasmas, los gobernantes y a la gente corriente.

\section*{1. La oración primitiva}
\par
%\textsuperscript{(994.6)}
\textsuperscript{91:1.1} La función de la religión evolutiva primitiva consiste en conservar y aumentar los valores sociales, morales y espirituales esenciales que van tomando forma lentamente. La humanidad no observa conscientemente esta misión de la religión, pero es llevada a cabo principalmente por la función de la oración. La práctica de la oración representa el esfuerzo no deliberado, pero sin embargo personal y colectivo, de un grupo cualquiera por asegurar (por realizar) esta conservación de los valores superiores. Sin la salvaguardia de la oración, todos los días de fiesta volverían rápidamente a la categoría de simples días de vacaciones.

\par
%\textsuperscript{(995.1)}
\textsuperscript{91:1.2} La religión y sus actividades, la principal de las cuales es la oración, sólo están aliadas con aquellos valores que gozan de un reconocimiento social general, de una aprobación colectiva. Por ello, cuando el hombre primitivo intentaba satisfacer sus emociones más bajas o conseguir sus ambiciones egoístas desenfrenadas, se quedaba privado del consuelo de la religión y de la ayuda de la oración. Si el individuo pretendía realizar algo antisocial, estaba obligado a buscar la ayuda de la magia no religiosa, a recurrir a los brujos y privarse así de la ayuda de la oración. Por consiguiente, la oración se volvió muy pronto una poderosa promotora de la evolución social, el progreso moral y la consecución espiritual.

\par
%\textsuperscript{(995.2)}
\textsuperscript{91:1.3} Pero la mente primitiva no era ni lógica ni coherente. Los hombres primitivos no percibían que las cosas materiales no pertenecían al ámbito de la oración. Estas almas sencillas razonaban que la comida, el refugio, la lluvia, la caza y otros bienes materiales acrecentaban el bienestar social, y por eso empezaron a rogar por estas bendiciones físicas. Aunque esto constituía una desnaturalización de la oración, estimulaba el esfuerzo por conseguir estos objetivos materiales mediante acciones sociales y éticas. Aunque esta prostitución de la oración degradaba los valores espirituales de un pueblo, sin embargo elevaba directamente sus costumbres económicas, sociales y éticas.

\par
%\textsuperscript{(995.3)}
\textsuperscript{91:1.4} La oración solamente es un monólogo para el tipo de mente más primitivo. Pronto se vuelve un diálogo y se amplía rápidamente hasta el nivel de culto colectivo. La oración significa que los conjuros premágicos de la religión primitiva han evolucionado hasta el nivel en que la mente humana reconoce la realidad de unos poderes o seres benéficos que son capaces de realzar los valores sociales y aumentar los ideales morales, y además, que estas influencias son superhumanas y distintas del ego humano consciente de sí mismo y sus compañeros mortales. Por lo tanto, la verdadera oración no aparece hasta que la acción del ministerio religioso llega a ser imaginada como \textit{personal}.

\par
%\textsuperscript{(995.4)}
\textsuperscript{91:1.5} La oración está poco relacionada con el animismo, pero estas creencias pueden existir al lado de los sentimientos religiosos emergentes. Muchas veces, la religión y el animismo han tenido orígenes totalmente distintos.

\par
%\textsuperscript{(995.5)}
\textsuperscript{91:1.6} Para aquellos mortales que no se han liberado de la esclavitud primitiva del miedo, existe un verdadero peligro de que todas las oraciones puedan conducir a un sentido mórbido del pecado, a unas convicciones injustificadas de culpabilidad, real o imaginaria. Pero en los tiempos modernos es poco probable que muchas personas dediquen el suficiente tiempo a la oración como para llegar a estas reflexiones perjudiciales sobre su indignidad o culpabilidad. Los peligros que acompañan a la distorsión y la perversión de la oración consisten en la ignorancia, la superstición, la cristalización, la desvitalización, el materialismo y el fanatismo.

\section*{2. La oración en evolución}
\par
%\textsuperscript{(995.6)}
\textsuperscript{91:2.1} Las primeras oraciones fueron unos simples anhelos expresados con palabras, la expresión de unos deseos sinceros. La oración se volvió después una técnica para conseguir la cooperación de los espíritus. Luego alcanzó la función superior de ayudar a la religión a conservar todos los valores dignos de consideración.

\par
%\textsuperscript{(995.7)}
\textsuperscript{91:2.2} La oración y la magia surgieron como resultado de las reacciones adaptativas humanas al entorno urantiano. Pero aparte de esta relación general, tienen pocas cosas en común. La oración siempre ha indicado una acción positiva por parte del ego que oraba; siempre ha sido psíquica y a veces espiritual. La magia ha significado generalmente un intento por manipular la realidad sin afectar al ego del manipulador, al practicante de la magia. A pesar de sus orígenes independientes, la magia y la oración han estado relacionadas con frecuencia en sus períodos posteriores de desarrollo. Mediante la elevación de sus objetivos, la magia a veces ha ascendido desde las fórmulas, pasando por los rituales y los conjuros, hasta el umbral de la verdadera oración. La oración se ha vuelto a veces tan materialista que ha degenerado en una técnica seudomágica para evitar el empleo del esfuerzo que se necesita para solucionar los problemas de Urantia.

\par
%\textsuperscript{(996.1)}
\textsuperscript{91:2.3} Cuando el hombre aprendió que la oración no podía coaccionar a los dioses, entonces ésta se convirtió más a menudo en una petición, en la búsqueda de un favor. Pero la oración más auténtica es en realidad una comunión entre el hombre y su Hacedor.

\par
%\textsuperscript{(996.2)}
\textsuperscript{91:2.4} La aparición de la idea de sacrificio en cualquier religión reduce infaliblemente la eficacia superior de la verdadera oración, ya que los hombres intentan sustituir la ofrenda de consagrar su propia voluntad a hacer la voluntad de Dios por las ofrendas de las posesiones materiales.

\par
%\textsuperscript{(996.3)}
\textsuperscript{91:2.5} Cuando la religión se encuentra despojada de un Dios personal, sus oraciones se trasladan a los niveles de la teología y la filosofía. Cuando el concepto más elevado de Dios que tiene una religión es el de una Deidad impersonal, como sucede en el idealismo panteísta, aunque este concepto proporcione las bases para ciertas formas de comunión mística, resulta funesto para el poder de la verdadera oración, que siempre representa la comunión del hombre con un ser personal y superior.

\par
%\textsuperscript{(996.4)}
\textsuperscript{91:2.6} En la experiencia cotidiana de los mortales corrientes durante los primeros tiempos de la evolución racial, e incluso en la actualidad, la oración es en gran medida un fenómeno de relaciones entre el hombre y su propio subconsciente. Pero también existe un ámbito en la oración en el que la persona intelectualmente despierta y espiritualmente progresiva consigue más o menos contactar con los niveles superconscientes de la mente humana, el dominio del Ajustador del Pensamiento interior. Además, existe una fase espiritual concreta de la verdadera oración que incumbe a su recepción y reconocimiento por parte de las fuerzas espirituales del universo, y que es totalmente distinta a todas las asociaciones humanas e intelectuales.

\par
%\textsuperscript{(996.5)}
\textsuperscript{91:2.7} La oración contribuye enormemente al desarrollo del sentimiento religioso de una mente humana en evolución. Es una influencia poderosa que actúa para impedir el aislamiento de la personalidad.

\par
%\textsuperscript{(996.6)}
\textsuperscript{91:2.8} La oración representa una técnica asociada a las religiones naturales de la evolución racial, que también forma parte de los valores experienciales de las religiones superiores con una ética excelente, las religiones reveladas.

\section*{3. La oración y el álter ego}
\par
%\textsuperscript{(996.7)}
\textsuperscript{91:3.1} Cuando los niños aprenden por primera vez a utilizar el lenguaje, tienen tendencia a pensar en voz alta, a expresar sus pensamientos en palabras, aunque no haya nadie para escucharlos. En los albores de su imaginación creativa, manifiestan la tendencia a conversar con unos compañeros imaginarios. De esta manera, el ego en ciernes trata de mantenerse en comunión con un \textit{álter ego} ficticio. El niño aprende pronto, por medio de esta técnica, a convertir sus conversaciones a base de monólogos en unos seudodiálogos en los que este álter ego contesta a sus pensamientos verbales y a la expresión de sus deseos. Una gran parte de las reflexiones de los adultos se lleva a cabo mentalmente bajo la forma de conversaciones.

\par
%\textsuperscript{(996.8)}
\textsuperscript{91:3.2} La forma de oración inicial y primitiva se parecía mucho a las recitaciones semimágicas de la tribu de los Todas de hoy en día, unas oraciones que no se dirigían a nadie en particular. Pero estas técnicas de oración tienden a transformarse en un tipo de comunicación dialogada gracias a la aparición de la idea del álter ego. Con el tiempo, el concepto del álter ego es elevado a una posición superior de dignidad divina, y la oración como acto religioso hace su aparición. Este tipo primitivo de oración está destinado a evolucionar a través de muchas fases y durante largas épocas, antes de alcanzar el nivel de la oración inteligente y realmente ética.

\par
%\textsuperscript{(997.1)}
\textsuperscript{91:3.3} Tal como lo conciben las generaciones sucesivas de mortales que practican la oración, el álter ego evoluciona desde los fantasmas, los fetiches y los espíritus hasta los dioses politeístas, y finalmente hasta el Dios Único, un ser divino que personifica los ideales superiores y las aspiraciones más elevadas del ego en oración. La oración funciona así como la acción más poderosa de la religión para conservar los valores e ideales superiores de las personas que oran. Desde el momento en que se concibe un álter ego hasta la aparición del concepto de un Padre divino y celestial, la oración es siempre una práctica socializadora, moralizadora y espiritualizadora.

\par
%\textsuperscript{(997.2)}
\textsuperscript{91:3.4} La oración sencilla de la fe demuestra una poderosa evolución en la experiencia humana, por medio de la cual las antiguas conversaciones con el símbolo ficticio del álter ego de la religión primitiva se han elevado hasta el nivel de la comunión con el espíritu del Infinito, y hasta el de una auténtica conciencia de la realidad del Dios eterno y Padre Paradisiaco de toda la creación inteligente.

\par
%\textsuperscript{(997.3)}
\textsuperscript{91:3.5} Aparte de todo lo que supone el yo superior en la experiencia de la oración, se debe recordar que la oración ética es una manera magnífica de elevar el propio ego y de reforzar el yo con vistas a una vida mejor y a unas consecuciones más elevadas. La oración induce al ego humano a buscar asistencia en dos direcciones: ayuda material en el depósito subconsciente de la experiencia humana, e inspiración y guía en las fronteras superconscientes donde lo material se pone en contacto con lo espiritual, con el Monitor de Misterio.

\par
%\textsuperscript{(997.4)}
\textsuperscript{91:3.6} La oración ha sido siempre, y siempre será, una experiencia humana doble: es un procedimiento psicológico, interasociado con una técnica espiritual. Estas dos funciones de la oración nunca se pueden separar por completo.

\par
%\textsuperscript{(997.5)}
\textsuperscript{91:3.7} La oración iluminada no solamente debe reconocer a un Dios externo y personal, sino también a una Divinidad interna e impersonal, el Ajustador interior. Cuando el hombre reza, es muy conveniente que se esfuerce por captar el concepto del Padre Universal del Paraíso; pero, para la mayoría de los efectos prácticos, la técnica más eficaz consistirá en volver al concepto del álter ego cercano, tal como solía hacer la mente primitiva, y luego reconocer que la idea de este álter ego ha evolucionado desde la simple ficción hasta la verdad de que Dios reside en el hombre mortal mediante la presencia real del Ajustador, de manera que el hombre puede hablar cara a cara, por así decirlo, con un divino álter ego real y auténtico que reside en él, y que es la presencia y la esencia mismas del Dios vivo, del Padre Universal.

\section*{4. La oración ética}
\par
%\textsuperscript{(997.6)}
\textsuperscript{91:4.1} Ninguna oración puede ser ética cuando el suplicante busca una ventaja egoísta sobre sus semejantes. La oración egoísta y materialista es incompatible con las religiones éticas que están basadas en el amor desinteresado y divino. Todas estas oraciones poco éticas vuelven a los niveles primitivos de la seudomagia, y son indignas de las civilizaciones que progresan y de las religiones iluminadas. La oración egoísta viola el espíritu de todas las éticas basadas en una justicia amorosa.

\par
%\textsuperscript{(997.7)}
\textsuperscript{91:4.2} La oración nunca debe prostituirse hasta el punto de convertirse en un sustituto de la acción. Toda oración ética es un estímulo para la acción y una guía para la lucha progresiva por las metas idealistas que desea alcanzar el yo superior.

\par
%\textsuperscript{(998.1)}
\textsuperscript{91:4.3} En todas vuestras oraciones, sed \textit{equitativos}; no esperéis que Dios muestre predilecciones, que os ame más que a sus otros hijos, vuestros amigos, vecinos e incluso vuestros enemigos. Pero la oración de las religiones naturales o evolucionadas no empieza siendo ética, como lo es en las religiones reveladas posteriores. Toda oración, ya sea individual o comunal, puede ser egoísta o altruista. Es decir, que la oración puede estar centrada en el yo o en los demás. Cuando la oración no busca nada para el que reza ni para sus semejantes, esta actitud del alma tiende entonces hacia los niveles de la verdadera adoración. Las oraciones egoístas incluyen confesiones y súplicas, y a menudo consisten en peticiones de favores materiales. La oración es un poco más ética cuando se ocupa del perdón y busca la sabiduría para acrecentar el dominio de sí mismo.

\par
%\textsuperscript{(998.2)}
\textsuperscript{91:4.4} Mientras que la oración de tipo altruista fortalece y consuela, la oración materialista está destinada a aportar decepción y desilusión a medida que los descubrimientos científicos en progreso demuestran que el hombre vive en un universo físico de ley y de orden. La infancia de un individuo o de una raza está caracterizada por oraciones primitivas, egoístas y materialistas. Y, hasta cierto punto, todas estas súplicas son eficaces, ya que conducen invariablemente a los esfuerzos y diligencias que contribuyen a conseguir las respuestas a esas oraciones. La verdadera oración de la fe siempre contribuye a mejorar la técnica de vida, aunque estas peticiones no sean dignas del reconocimiento espiritual. Pero las personas espiritualmente avanzadas deberían proceder con gran cautela al intentar recomendar a las mentes primitivas o inmaduras que no efectúen este tipo de oraciones.

\par
%\textsuperscript{(998.3)}
\textsuperscript{91:4.5} Recordad que, aunque la oración no cambia a Dios, realiza con mucha frecuencia unos cambios importantes y duraderos en aquel que ora con fe y una esperanza confiada. La oración ha engendrado mucha paz mental, alegría, calma, valor, dominio de sí mismo y equidad en los hombres y las mujeres de las razas en evolución.

\section*{5. Las repercusiones sociales de la oración}
\par
%\textsuperscript{(998.4)}
\textsuperscript{91:5.1} En el culto a los antepasados, la oración conduce a cultivar los ideales ancestrales. Pero como característica del culto a la Deidad, la oración trasciende todas las demás prácticas de este tipo, ya que conduce a cultivar los ideales divinos. A medida que el concepto del álter ego de la oración se vuelve supremo y divino, los ideales del hombre se elevan en consecuencia desde el nivel simplemente humano hacia los niveles celestiales y divinos, y el resultado de todas estas oraciones es el realce del carácter humano y la profunda unificación de la personalidad humana.

\par
%\textsuperscript{(998.5)}
\textsuperscript{91:5.2} Pero no es necesario que la oración sea siempre individual. La oración en grupo o en asamblea es muy eficaz ya que sus repercusiones son extremadamente socializadoras. Cuando un grupo se dedica a orar en común por el acrecentamiento moral y la elevación espiritual, estas devociones producen efecto en los individuos que componen el grupo; todos se vuelven mejores gracias a esta participación. Estas devociones piadosas pueden incluso ayudar a una ciudad entera o a toda una nación. La confesión, el arrepentimiento y la oración han conducido a los individuos, las ciudades, las naciones y las razas enteras a extraordinarios esfuerzos de reforma y a acciones intrépidas realizadas con valentía.

\par
%\textsuperscript{(998.6)}
\textsuperscript{91:5.3} Si deseáis realmente vencer la costumbre de criticar a un amigo, la manera más rápida y segura de conseguir este cambio de actitud consiste en establecer la costumbre de rezar por esa persona cada día de vuestra vida. Pero las repercusiones sociales de estas oraciones dependen en gran parte de dos condiciones:

\par
%\textsuperscript{(998.7)}
\textsuperscript{91:5.4} 1. La persona por la que se reza debe saber que se reza por ella.

\par
%\textsuperscript{(999.1)}
\textsuperscript{91:5.5} 2. La persona que reza debe entrar en contacto social íntimo con la persona por la que reza.

\par
%\textsuperscript{(999.2)}
\textsuperscript{91:5.6} La oración es la técnica por la cual toda religión se convierte tarde o temprano en una institución. Y con el tiempo, la oración se asocia a numerosas acciones secundarias, algunas útiles y otras decididamente perjudiciales, tales como los sacerdotes, los libros sagrados, los rituales de adoración y las ceremonias.

\par
%\textsuperscript{(999.3)}
\textsuperscript{91:5.7} Pero las mentes con una mayor iluminación espiritual deberían ser pacientes y tolerantes con los intelectos menos dotados que desean ardientemente un simbolismo para movilizar su débil perspicacia espiritual. Los fuertes no deben mirar con desdén a los débiles. Aquellos que son conscientes de Dios sin necesidad de simbolismos no deben negarle el ministerio de gracia de los símbolos a aquellos que encuentran difícil adorar a la Deidad y venerar la verdad, la belleza y la bondad sin formas ni ritos. En la adoración piadosa, la mayoría de los mortales imaginan algún símbolo del objeto y meta de sus devociones.

\section*{6. La esfera de acción de la oración}
\par
%\textsuperscript{(999.4)}
\textsuperscript{91:6.1} La oración, a menos que esté coordinada con la voluntad y las actividades de las fuerzas espirituales personales y de los supervisores materiales de un mundo, no puede tener ningún efecto directo sobre vuestro entorno físico. Aunque existe un límite muy definido en el terreno de las peticiones de la oración, estos límites no se aplican por igual a la \textit{fe} de aquellos que oran.

\par
%\textsuperscript{(999.5)}
\textsuperscript{91:6.2} La oración no es una técnica para curar las enfermedades orgánicas reales, pero ha contribuido enormemente al disfrute de una salud abundante y a la curación de numerosos malestares mentales, emocionales y nerviosos. Incluso en el caso de enfermedades bacterianas reales, la oración ha acrecentado muchas veces la eficacia de otros procedimientos curativos. La oración ha transformado a muchos inválidos irritables y quejumbrosos en modelos de paciencia, y ha hecho de ellos una inspiración para todos los demás enfermos humanos.

\par
%\textsuperscript{(999.6)}
\textsuperscript{91:6.3} Por muy difícil que sea conciliar las dudas científicas sobre la eficacia de la oración con el impulso siempre presente de buscar la ayuda y la guía de las fuentes divinas, no olvidéis nunca que la oración sincera de la fe es una fuerza poderosa para fomentar la felicidad personal, el autocontrol individual, la armonía social, el progreso moral y los logros espirituales.

\par
%\textsuperscript{(999.7)}
\textsuperscript{91:6.4} La oración, incluso como práctica puramente humana, como un diálogo con vuestro álter ego, constituye una técnica de aproximación de las más eficaces para hacer realidad aquellos poderes de reserva de la naturaleza humana que están almacenados y conservados en las zonas inconscientes de la mente humana. La oración es una práctica psicológica sana, aparte de sus implicaciones religiosas y de su significado espiritual. Es un hecho de la experiencia humana que la mayoría de las personas, si se sienten lo bastante apremiadas, rezan de alguna manera a alguna fuente de ayuda.

\par
%\textsuperscript{(999.8)}
\textsuperscript{91:6.5} No seáis tan perezosos como para pedirle a Dios que resuelva vuestras dificultades, pero no dudéis nunca en pedirle sabiduría y fuerza espiritual para que os guíen y os sostengan mientras atacáis con resolución y valor los problemas diarios.

\par
%\textsuperscript{(999.9)}
\textsuperscript{91:6.6} La oración ha sido un factor indispensable para el progreso y la conservación de la civilización religiosa, y todavía puede contribuir enormemente a una mayor elevación y espiritualización de la sociedad si aquellos que oran lo hacen a la luz de los hechos científicos, la sabiduría filosófica, la sinceridad intelectual y la fe espiritual. Orad como Jesús lo enseñaba a sus discípulos ---con sinceridad, desinterés, equidad, y sin dudar.

\par
%\textsuperscript{(1000.1)}
\textsuperscript{91:6.7} Pero la eficacia de la oración en la experiencia espiritual personal de aquel que ora no depende de ninguna manera de la comprensión intelectual de dicho fiel, de su perspicacia filosófica, su nivel social, su situación cultural o de sus otros conocimientos humanos. Los efectos psicológicos y espirituales que acompañan a la oración de la fe son inmediatos, personales y experienciales. No existe ninguna otra técnica que permita a cualquier hombre, sin tener en cuenta todos sus demás logros mortales, acercarse de manera tan inmediata y eficaz al umbral de ese reino donde puede comunicarse con su Hacedor, donde la criatura se pone en contacto con la realidad del Creador, con el Ajustador del Pensamiento interior.

\section*{7. El misticismo, el éxtasis y la inspiración}
\par
%\textsuperscript{(1000.2)}
\textsuperscript{91:7.1} El misticismo, como técnica para cultivar la conciencia de la presencia de Dios, es totalmente digno de elogio, pero cuando tales prácticas conducen al aislamiento social y culminan en el fanatismo religioso, son casi censurables. Con demasiada frecuencia, aquello que el místico sobreexcitado interpreta como una inspiración divina es algo que emerge de su propia mente profunda. Aunque una meditación ferviente favorece a menudo el contacto de la mente mortal con su Ajustador interior, el servicio sincero y amoroso de un ministerio desinteresado hacia vuestros semejantes lo facilita con más frecuencia.

\par
%\textsuperscript{(1000.3)}
\textsuperscript{91:7.2} Los grandes educadores religiosos y los profetas de las épocas pasadas no eran místicos extremos. Eran hombres y mujeres que conocían a Dios y que servían mejor a su Dios ayudando desinteresadamente a sus compañeros mortales. Jesús se llevaba con frecuencia a sus apóstoles a solas durante cortos períodos para dedicarse a meditar y a orar, pero la mayor parte del tiempo los mantenía en contacto servicial con las multitudes. El alma del hombre tiene necesidad de ejercicio espiritual así como de alimento espiritual.

\par
%\textsuperscript{(1000.4)}
\textsuperscript{91:7.3} El éxtasis religioso es aceptable cuando resulta de unos antecedentes sanos, pero estas experiencias son con más frecuencia la consecuencia de influencias puramente emocionales que la manifestación de un carácter espiritual profundo. Las personas religiosas no deben considerar cada presentimiento psicológico fuerte y cada experiencia emocional intensa como una revelación divina o una comunicación espiritual. El éxtasis espiritual auténtico está generalmente acompañado de una gran calma exterior y de un control emocional casi perfecto. Pero la verdadera visión profética es un presentimiento super-psicológico. Estas experiencias no son ni seudo-alucinaciones ni éxtasis semejantes a los trances.

\par
%\textsuperscript{(1000.5)}
\textsuperscript{91:7.4} La mente humana puede actuar en respuesta a la pretendida inspiración cuando es sensible a lo que emerge del subconsciente o al estímulo del superconsciente. En cualquiera de los dos casos, al individuo le parece que estos incrementos del contenido de la conciencia son más o menos exteriores. El entusiasmo místico desmedido y el éxtasis religioso desenfrenado no son las cartas credenciales de la inspiración, las cartas credenciales supuestamente divinas.

\par
%\textsuperscript{(1000.6)}
\textsuperscript{91:7.5} La prueba práctica para todas estas extrañas experiencias religiosas de misticismo, éxtasis e inspiración consiste en observar si estos fenómenos hacen que un individuo:

\par
%\textsuperscript{(1000.7)}
\textsuperscript{91:7.6} 1. Disfrute de una salud física mejor y más completa.

\par
%\textsuperscript{(1000.8)}
\textsuperscript{91:7.7} 2. Actúe de una manera más práctica y eficaz en su vida mental.

\par
%\textsuperscript{(1000.9)}
\textsuperscript{91:7.8} 3. Adapte su experiencia religiosa con más plenitud y alegría a la vida social.

\par
%\textsuperscript{(1000.10)}
\textsuperscript{91:7.9} 4. Espiritualice de una forma más completa su vida cotidiana, mientras cumple fielmente con los deberes corrientes de la existencia humana rutinaria.

\par
%\textsuperscript{(1001.1)}
\textsuperscript{91:7.10} 5. Aumente su amor y su apreciación de la verdad, la belleza y la bondad.

\par
%\textsuperscript{(1001.2)}
\textsuperscript{91:7.11} 6. Conserve los valores sociales, morales, éticos y espirituales generalmente reconocidos.

\par
%\textsuperscript{(1001.3)}
\textsuperscript{91:7.12} 7. Incremente su perspicacia espiritual ---su conciencia de Dios.

\par
%\textsuperscript{(1001.4)}
\textsuperscript{91:7.13} Pero la oración no está relacionada realmente con estas experiencias religiosas excepcionales. Cuando la oración se vuelve demasiado estética, cuando consiste casi exclusivamente en una hermosa y feliz contemplación de la divinidad paradisiaca, pierde una gran parte de su influencia socializadora y tiende hacia el misticismo y el aislamiento de sus adeptos. El exceso de oración en privado implica cierto peligro que se puede corregir e impedir mediante la oración en grupo, las devociones colectivas.

\section*{8. La oración como experiencia personal}
\par
%\textsuperscript{(1001.5)}
\textsuperscript{91:8.1} La oración posee un aspecto realmente espontáneo, pues el hombre primitivo empezó a orar mucho antes de que tuviera un concepto claro de un Dios. Los primeros hombres solían rezar en dos situaciones diferentes: cuando tenían una necesidad extrema, experimentaban el impulso de buscar ayuda; y cuando se sentían alborozados, daban rienda suelta a la expresión impulsiva de su alegría.

\par
%\textsuperscript{(1001.6)}
\textsuperscript{91:8.2} La oración no es una evolución de la magia; cada una de ellas surgió de manera independiente. La magia era un intento por adaptar la Deidad a las circunstancias; la oración es el esfuerzo por adaptar la personalidad a la voluntad de la Deidad. La verdadera oración es al mismo tiempo moral y religiosa; la magia no es ninguna de las dos.

\par
%\textsuperscript{(1001.7)}
\textsuperscript{91:8.3} La oración puede convertirse en una costumbre establecida; muchas personas rezan porque otras lo hacen. Otras rezan también porque temen que pueda sucederles algo terrible si no presentan sus súplicas habituales.

\par
%\textsuperscript{(1001.8)}
\textsuperscript{91:8.4} Para algunos individuos, la oración es la expresión sosegada de la gratitud; para otros, una expresión colectiva de alabanza, las devociones sociales; a veces consiste en la imitación de la religión de otras personas, mientras que la verdadera oración es la comunicación sincera y confiada entre la naturaleza espiritual de la criatura y la presencia ubicua del espíritu del Creador.

\par
%\textsuperscript{(1001.9)}
\textsuperscript{91:8.5} La oración puede ser una expresión espontánea de la conciencia de Dios, o una recitación sin sentido de fórmulas teológicas. Puede ser la alabanza extática de un alma que conoce a Dios, o el homenaje servil de un mortal dominado por el miedo. A veces consiste en la expresión patética de un anhelo espiritual, y a veces en el grito estridente de unas frases piadosas. La oración puede ser una alabanza gozosa o una humilde petición de perdón.

\par
%\textsuperscript{(1001.10)}
\textsuperscript{91:8.6} La oración puede ser la petición infantil de lo imposible, o la súplica madura por el crecimiento moral y el poder espiritual. Una petición puede ser por el pan de cada día, o puede expresar el anhelo sincero de encontrar a Dios y hacer su voluntad\footnote{\textit{Oración personal}: Mt 6:9-13; Lc 11:3-4.}. Puede tratarse de un ruego totalmente egoísta, o de un gesto sincero y magnífico hacia la realización de la fraternidad desinteresada.

\par
%\textsuperscript{(1001.11)}
\textsuperscript{91:8.7} La oración puede ser un grito airado de venganza, o una intercesión misericordiosa por vuestros enemigos. Puede ser la expresión de la esperanza de cambiar a Dios, o la técnica poderosa de cambiarse a sí mismo. Puede ser la súplica acobardada de un pecador perdido ante un Juez supuestamente severo, o la alegre expresión de un hijo, liberado, del Padre celestial vivo y misericordioso.

\par
%\textsuperscript{(1001.12)}
\textsuperscript{91:8.8} El hombre moderno se siente desconcertado ante la idea de hablar de sus asuntos con Dios de una manera puramente personal. Muchos han abandonado la oración asidua; sólo rezan cuando se encuentran bajo una presión inhabitual ---en casos de urgencia. El hombre no debería tener miedo de hablar con Dios, pero sólo una persona espiritualmente infantil intentaría persuadir, o atreverse a cambiar, a Dios.

\par
%\textsuperscript{(1002.1)}
\textsuperscript{91:8.9} Pero la verdadera oración alcanza de hecho la realidad. Incluso cuando las corrientes de aire son ascendentes, ningún pájaro puede elevarse a menos que extienda sus alas. La oración eleva al hombre porque es una técnica para progresar mediante la utilización de las corrientes espirituales ascendentes del universo.

\par
%\textsuperscript{(1002.2)}
\textsuperscript{91:8.10} La oración auténtica aumenta el crecimiento espiritual, modifica las actitudes y produce la satisfacción que proviene de la comunión con la divinidad. Es una explosión espontánea de conciencia de Dios.

\par
%\textsuperscript{(1002.3)}
\textsuperscript{91:8.11} Dios contesta a la oración del hombre dándole una mayor revelación de la verdad, una apreciación realzada de la belleza, y un concepto acrecentado de la bondad. La oración es un gesto subjetivo, pero se pone en contacto con unas poderosas realidades objetivas en los niveles espirituales de la experiencia humana; es un intento significativo de lo humano por alcanzar los valores superhumanos. Es el estímulo más poderoso para el crecimiento espiritual.

\par
%\textsuperscript{(1002.4)}
\textsuperscript{91:8.12} Las palabras no tienen ninguna importancia en el rezo; son simplemente el canal intelectual por el que fluye casualmente el río de la súplica espiritual. El valor verbal de una plegaria es puramente autosugestivo en las devociones privadas, y sociosugestivo en las devociones colectivas. Dios responde a la actitud del alma, no a las palabras.

\par
%\textsuperscript{(1002.5)}
\textsuperscript{91:8.13} La oración no es una técnica para huir de los conflictos, sino más bien un estímulo para crecer en presencia misma de los conflictos. Orad sólo por los valores, no por las cosas; por el crecimiento, no por la satisfacción.

\section*{9. Condiciones para que la oración sea eficaz}
\par
%\textsuperscript{(1002.6)}
\textsuperscript{91:9.1} Si queréis orar de manera eficaz, debéis tener en cuenta las leyes de las peticiones comunes:

\par
%\textsuperscript{(1002.7)}
\textsuperscript{91:9.2} 1. Tenéis que capacitaros como rezadores poderosos, enfrentándoos sincera y valientemente con los problemas de la realidad universal. Debéis poseer vigor cósmico.

\par
%\textsuperscript{(1002.8)}
\textsuperscript{91:9.3} 2. Tenéis que haber agotado honradamente todas las capacidades humanas de adaptación. Tenéis que haber sido laboriosos.

\par
%\textsuperscript{(1002.9)}
\textsuperscript{91:9.4} 3. Tenéis que abandonar todos los deseos de la mente y todos los anhelos del alma al abrazo transformador del crecimiento espiritual. Tenéis que haber experimentado un realce de los significados y una elevación de los valores.

\par
%\textsuperscript{(1002.10)}
\textsuperscript{91:9.5} 4. Tenéis que elegir sinceramente la voluntad divina. Tenéis que eliminar el punto muerto de la indecisión.

\par
%\textsuperscript{(1002.11)}
\textsuperscript{91:9.6} 5. No solamente reconocéis la voluntad del Padre y escogéis hacerla, sino que habéis efectuado una consagración incondicional y una dedicación dinámica a hacer realmente la voluntad del Padre.

\par
%\textsuperscript{(1002.12)}
\textsuperscript{91:9.7} 6. Vuestra oración estará dirigida exclusivamente a obtener sabiduría divina para resolver los problemas humanos específicos que encontraréis en la ascensión al Paraíso ---la conquista de la perfección divina.

\par
%\textsuperscript{(1002.13)}
\textsuperscript{91:9.8} 7. Y debéis tener fe ---una fe viviente.

\par
%\textsuperscript{(1002.14)}
\textsuperscript{91:9.9} [Presentado por el Jefe de los Intermedios de Urantia.]


\chapter{Documento 92. La evolución posterior de la religión}
\par
%\textsuperscript{(1003.1)}
\textsuperscript{92:0.1} EL HOMBRE poseía una religión de origen natural, que formaba parte de su experiencia evolutiva, mucho antes de que se hiciera cualquier revelación sistemática en Urantia. Pero esta religión de origen \textit{natural} era, en sí misma, el producto de los dones superanimales del hombre. La religión evolutiva surgió lentamente a lo largo de todos los milenios de la carrera experiencial de la humanidad gracias al ministerio de las influencias siguientes, que actuaban en el interior del hombre salvaje, del bárbaro y del civilizado, e incidían en ellos:

\par
%\textsuperscript{(1003.2)}
\textsuperscript{92:0.2} 1. \textit{El ayudante de la adoración} ---la aparición en la conciencia animal de unos potenciales superanimales destinados a percibir la realidad. Esto podría denominarse el instinto humano primordial de búsqueda de la Deidad.

\par
%\textsuperscript{(1003.3)}
\textsuperscript{92:0.3} 2. \textit{El ayudante de la sabiduría} ---la manifestación en una mente adoradora de la tendencia a dirigir su adoración en unos canales superiores de expresión y hacia unos conceptos siempre más amplios de la realidad de la Deidad.

\par
%\textsuperscript{(1003.4)}
\textsuperscript{92:0.4} 3. \textit{El Espíritu Santo} ---éste es el don supermental inicial y aparece infaliblemente en todas las personalidades humanas de buena fe. Este ministerio crea en la mente anhelante de adoración y deseosa de sabiduría la capacidad de desarrollar por sí misma el postulado de la supervivencia humana, a la vez como concepto teológico y como una experiencia real y objetiva de la personalidad.

\par
%\textsuperscript{(1003.5)}
\textsuperscript{92:0.5} El funcionamiento coordinado de estos tres ministerios divinos es totalmente suficiente para iniciar y llevar a cabo el crecimiento de la religión evolutiva. Estas influencias reciben la ayuda posterior de los Ajustadores del Pensamiento, los serafines y el Espíritu de la Verdad, y todos ellos aceleran el ritmo del desarrollo religioso. Estos agentes funcionan desde hace mucho tiempo en Urantia, y continuarán aquí mientras este planeta siga siendo una esfera habitada. Una gran parte del potencial de estos agentes divinos nunca ha tenido todavía la oportunidad de expresarse; muchas cosas se revelarán en las épocas venideras a medida que la religión de los mortales se eleve, de nivel en nivel, hacia las alturas celestiales de los valores morontiales y de las verdades espirituales.

\section*{1. La naturaleza evolutiva de la religión}
\par
%\textsuperscript{(1003.6)}
\textsuperscript{92:1.1} La evolución de la religión se remonta al miedo primitivo y a los fantasmas, y ha pasado por numerosas etapas sucesivas de desarrollo, incluyendo los esfuerzos que se hicieron, primero para coaccionar a los espíritus, y luego para engatusarlos. Los fetiches de las tribus se convirtieron en los tótemes y los dioses tribales; las fórmulas mágicas se transformaron en las oraciones modernas. La circuncisión, que al principio era un sacrificio, se volvió un procedimiento higiénico.

\par
%\textsuperscript{(1003.7)}
\textsuperscript{92:1.2} A lo largo de la infancia salvaje de las razas, la religión progresó desde la adoración de la naturaleza hasta el fetichismo, pasando por el culto a los fantasmas. En los albores de la civilización, la raza humana abrazó las creencias más místicas y simbólicas, mientras que ahora, al acercarse a su madurez, la humanidad se prepara para apreciar la verdadera religión, e incluso un comienzo de la revelación de la verdad misma.

\par
%\textsuperscript{(1004.1)}
\textsuperscript{92:1.3} La religión surge como una reacción biológica de la mente a las creencias espirituales y al entorno; es lo último que perece o cambia en una raza. La religión es la adaptación de la sociedad, en cualquier época, a aquello que es misterioso. Como institución social abarca ritos, símbolos, cultos, escrituras, altares, santuarios y templos. El agua bendita, las reliquias, los fetiches, los amuletos, las vestiduras, las campanas, los tambores y los sacerdotes son frecuentes en todas las religiones. Es imposible separar por completo la religión puramente evolutiva de la magia o la brujería.

\par
%\textsuperscript{(1004.2)}
\textsuperscript{92:1.4} El misterio y el poder siempre han estimulado los sentimientos y los temores religiosos, mientras que la emoción ha funcionado continuamente como un poderoso factor que ha condicionado el desarrollo de ambos. El miedo ha sido siempre el estímulo religioso fundamental. El miedo da forma a los dioses de la religión evolutiva y motiva el ritual religioso de los creyentes primitivos. A medida que avanza la civilización, el temor es modificado por la veneración, la admiración, el respeto y la simpatía, y luego es condicionado además por el remordimiento y el arrepentimiento.

\par
%\textsuperscript{(1004.3)}
\textsuperscript{92:1.5} Un pueblo asiático enseñaba que <<Dios es un gran temor>>\footnote{\textit{Dios es un gran temor}: Gn 20:11; 22:12; Job 1:1.}; éste es el resultado de la religión puramente evolutiva. Jesús, la revelación del tipo más elevado de vida religiosa, proclamó que <<Dios es amor>>\footnote{\textit{Dios es amor}: 1 Jn 4:8,16.}.

\section*{2. La religión y las costumbres}
\par
%\textsuperscript{(1004.4)}
\textsuperscript{92:2.1} La religión es la más rígida e inflexible de todas las instituciones humanas, pero se adapta con retraso a la sociedad cambiante. La religión evolutiva refleja finalmente las costumbres cambiantes que, a su vez, pueden haber sido afectadas por la religión revelada. De una manera lenta, segura, pero a regañadientes, la religión (el culto) sigue las huellas de la sabiduría ---del conocimiento dirigido por la razón experiencial e iluminado por la revelación divina.

\par
%\textsuperscript{(1004.5)}
\textsuperscript{92:2.2} La religión se aferra a las costumbres; aquello que \textit{era} es antiguo y supuestamente sagrado. Es por esta razón, y no por otra, por la que las herramientas de piedra sobrevivieron durante mucho tiempo en la edad del bronce y del hierro. Vuestros archivos contienen esta declaración: <<Y si me hacéis un altar de piedra, no lo construyáis con piedras talladas, porque si utilizáis vuestras herramientas para hacerlo, lo habréis profanado>>\footnote{\textit{Altar de piedra no labrada}: Ex 20:25; Dt 27:5-6.}. Incluso hoy en día, los hindúes encienden el fuego de sus altares utilizando un instrumento primitivo para hacer fuego. En el transcurso de la religión evolutiva, la novedad siempre ha sido considerada como un sacrilegio. El sacramento debe estar compuesto, no de alimentos nuevos y manufacturados, sino de las viandas más primitivas: <<La carne asada al fuego y el pan sin levadura servido con hierbas amargas>>\footnote{\textit{Carne asada y pan}: Ex 12:8.}. Todos los tipos de usos sociales, e incluso los procedimientos legales, se aferran a las formas antiguas.

\par
%\textsuperscript{(1004.6)}
\textsuperscript{92:2.3} Cuando el hombre moderno se asombra de que las escrituras de diferentes religiones presenten tantos pasajes que se podrían juzgar como obscenos, debería detenerse a considerar que las generaciones que pasan han temido eliminar lo que sus antepasados creían que era santo y sagrado. Una generación puede estimar como obscenas muchas cosas que las generaciones precedentes consideraban como una parte de sus costumbres aceptadas, e incluso como rituales religiosos aprobados. Una gran cantidad de controversias religiosas han tenido lugar debido a los intentos sin fin por conciliar las prácticas antiguas, pero censurables, con los nuevos progresos de la razón, por encontrar unas teorías plausibles que justifiquen la perpetuación, en los credos, de unas costumbres antiguas y caducas.

\par
%\textsuperscript{(1004.7)}
\textsuperscript{92:2.4} Pero tratar de acelerar con demasiada rapidez el crecimiento religioso no es más que una insensatez. Una raza o una nación sólo puede asimilar, de cualquier religión avanzada, aquello que es razonablemente coherente y compatible con su estado evolutivo en curso, además de su don especial para adaptarse. Todas las condiciones sociales, climáticas, políticas y económicas ejercen su influencia para determinar el curso y el progreso de la evolución religiosa. La moralidad social no está determinada por la religión, es decir, por la religión evolutiva; la moralidad racial es más bien la que dicta las formas de la religión.

\par
%\textsuperscript{(1005.1)}
\textsuperscript{92:2.5} Las razas de los hombres sólo aceptan una religión nueva y extraña de forma superficial; en realidad, la adaptan a sus costumbres y a sus antiguas maneras de creer. Este hecho está bien ilustrado en el ejemplo de una tribu de Nueva Zelanda cuyos sacerdotes, después de haber aceptado nominalmente el cristianismo, afirmaron haber recibido unas revelaciones directas de Gabriel especificando que esta misma tribu se había convertido en el pueblo elegido de Dios, y ordenando que se permitiera a sus miembros entregarse libremente a las relaciones sexuales licenciosas y a otras muchas de sus costumbres antiguas y censurables. Todos los cristianos recién convertidos se pasaron inmediatamente a esta versión nueva y menos exigente del cristianismo.

\par
%\textsuperscript{(1005.2)}
\textsuperscript{92:2.6} La religión ha autorizado, en una época u otra, todo tipo de comportamientos contrarios e inconsecuentes, ha aprobado en algún momento prácticamente todo lo que ahora se considera como inmoral o pecaminoso. La conciencia, sin la enseñanza de la experiencia ni la ayuda de la razón, no ha sido nunca y nunca podrá ser una guía infalible y segura para la conducta humana. La conciencia no es una voz divina que le habla al alma humana. Es solamente la suma total del contenido moral y ético de las costumbres de cualquier etapa corriente de la existencia; representa simplemente la reacción ideal concebida por el ser humano en cualquier conjunto dado de circunstancias.

\section*{3. La naturaleza de la religión evolutiva}
\par
%\textsuperscript{(1005.3)}
\textsuperscript{92:3.1} El estudio de la religión humana es el examen de los estratos sociales fosilíferos de las épocas pasadas. Las costumbres de los dioses antropomórficos son un reflejo fiel de la moral de los hombres que concibieron por primera vez estas deidades. Las religiones antiguas y la mitología describen fielmente las creencias y tradiciones de unos pueblos perdidos desde hace mucho tiempo en la oscuridad. Estas antiguas prácticas cultuales sobreviven al lado de las costumbres económicas y los desarrollos sociales nuevos y, por supuesto, parecen enormemente contradictorias. Los restos de un culto ofrecen una imagen auténtica de las religiones raciales del pasado. Recordad siempre que los cultos no se forman para descubrir la verdad, sino más bien para promulgar sus credos.

\par
%\textsuperscript{(1005.4)}
\textsuperscript{92:3.2} La religión ha sido siempre sobre todo un asunto de ritos, rituales, prácticas, ceremonias y dogmas. Normalmente se ha contaminado con un error sembrador de discordias permanentes, la ilusión del pueblo elegido. Todas las ideas religiosas cardinales ---conjuro, inspiración, revelación, propiciación, arrepentimiento, expiación, intercesión, sacrificio, oración, confesión, adoración, supervivencia después de la muerte, sacramento, ritual, rescate, salvación, redención, alianza, impureza, purificación, profecía, pecado original--- se remontan a los tiempos primitivos del miedo primordial a los fantasmas.

\par
%\textsuperscript{(1005.5)}
\textsuperscript{92:3.3} La religión primitiva no es ni más ni menos que la lucha por la existencia material, ampliada hasta abarcar la existencia más allá de la tumba. Las prácticas de este credo representaban la extensión de la lucha por la subsistencia hasta el ámbito de un mundo imaginario de espíritus fantasmas. Pero cuando tengáis la tentación de criticar la religión evolutiva, tened cuidado. Recordad que ella representa \textit{lo que sucedió}; es un hecho histórico. Y recordad también que el poder de una idea cualquiera no reside en su certidumbre o en su verdad, sino más bien en su fuerza de atracción sobre los hombres.

\par
%\textsuperscript{(1006.1)}
\textsuperscript{92:3.4} La religión evolutiva no prevé llevar a cabo cambios o revisiones; a diferencia de la ciencia, no asegura su propia corrección progresiva. La religión evolucionada infunde respeto porque sus seguidores creen que es \textit{La Verdad}; <<la fe entregada a los santos en otro tiempo>>\footnote{\textit{La fe entregada a los santos}: Jud 1:3.} debe ser, en teoría, definitiva e infalible a la vez. El culto se resiste al desarrollo porque el auténtico progreso modificará o destruirá con toda seguridad al culto mismo; por eso la revisión siempre ha de serle impuesta.

\par
%\textsuperscript{(1006.2)}
\textsuperscript{92:3.5} Únicamente dos influencias pueden modificar y elevar los dogmas de la religión natural: la presión de las costumbres que progresan lentamente y la iluminación periódica de las revelaciones de época. Y no es de extrañar que el progreso haya sido lento; en los tiempos antiguos, ser progresista o inventivo significaba ser ejecutado como brujo. El culto avanza lentamente a través de las épocas generacionales y los ciclos seculares. Pero avanza de hecho. La creencia evolutiva en los fantasmas colocó los cimientos para una filosofía de la religión revelada que destruirá con el tiempo la superstición que le dio origen.

\par
%\textsuperscript{(1006.3)}
\textsuperscript{92:3.6} La religión ha obstaculizado el desarrollo social de muchas maneras, pero sin religión no habría habido ninguna moral ni ética duraderas, ninguna civilización digna de ese nombre. La religión dio nacimiento a mucha cultura no religiosa: la escultura se originó en la fabricación de los ídolos, la arquitectura en la construcción de los templos, la poesía en los conjuros, la música en los cantos de adoración, el teatro en las interpretaciones para conseguir la guía de los espíritus, y la danza en los festivales estacionales de adoración.

\par
%\textsuperscript{(1006.4)}
\textsuperscript{92:3.7} Pero, aunque llamamos la atención sobre el hecho de que la religión fue esencial para el desarrollo y la preservación de la civilización, hay que indicar que la religión natural también ha contribuido mucho a paralizar y detener a la misma civilización que por otra parte fomentaba y mantenía. La religión ha obstaculizado las actividades industriales y el desarrollo económico; ha desperdiciado el trabajo y ha malgastado el capital; no siempre ha ayudado a la familia; no ha fomentado de manera adecuada la paz y la buena voluntad; a veces ha descuidado la educación y retrasado la ciencia; ha empobrecido indebidamente la vida a cambio de un supuesto enriquecimiento de la muerte. La religión evolutiva, la religión humana, ha sido realmente culpable de todas estas equivocaciones, errores y desatinos, y de muchos más; sin embargo, ha mantenido una ética cultural, una moralidad civilizada, y una cohesión social, y ha hecho posible que la religión revelada posterior compensara estos numerosos defectos evolutivos.

\par
%\textsuperscript{(1006.5)}
\textsuperscript{92:3.8} La religión evolutiva ha sido la institución humana más costosa, pero su eficacia ha sido incomparable. La religión humana sólo se puede justificar a la luz de la civilización evolutiva. Si el hombre no fuera el producto ascendente de la evolución animal, entonces este recorrido del desarrollo religioso permanecería sin justificación.

\par
%\textsuperscript{(1006.6)}
\textsuperscript{92:3.9} La religión facilitó la acumulación del capital; fomentó ciertos tipos de trabajos; el tiempo libre de los sacerdotes favoreció el arte y el conocimiento; al final, la raza ganó mucho como consecuencia de todos estos errores iniciales de la técnica ética. Los chamanes, honrados y fraudulentos, fueron enormemente costosos, pero valieron la pena todo lo que costaron. Las profesiones liberales y la ciencia misma surgieron de los cleros parasitarios. La religión fomentó la civilización y facilitó la continuidad social; ha sido la policía moral de todos los tiempos. La religión proporcionó la disciplina humana y el dominio de sí mismo que hicieron posible la \textit{sabiduría}. La religión es el látigo eficaz de la evolución que obliga implacablemente a la humanidad indolente y sufriente a salir de su estado natural de inercia intelectual y a elevarse hasta los niveles superiores de la razón y la sabiduría.

\par
%\textsuperscript{(1006.7)}
\textsuperscript{92:3.10} La religión evolutiva, esta herencia sagrada de la ascensión animal, debe continuar siempre refinándose y ennobleciéndose por medio de la censura constante de la religión revelada y del horno ardiente de la ciencia auténtica.

\section*{4. El don de la revelación}
\par
%\textsuperscript{(1007.1)}
\textsuperscript{92:4.1} La revelación es evolutiva pero siempre progresiva. A lo largo de las épocas de la historia de un mundo, las revelaciones de la religión son cada vez más extensas y sucesivamente más instructivas. La misión de la revelación consiste en clasificar y censurar las religiones sucesivas de la evolución. Pero si la revelación ha de engrandecer y elevar las religiones de la evolución, entonces estas visitas divinas deben presentar unas enseñanzas que no estén demasiado alejadas de las ideas y reacciones de la época en que son presentadas. Por eso la revelación debe mantenerse siempre en contacto con la evolución, y lo hace de hecho. La religión revelada ha de estar siempre limitada por la capacidad del hombre para recibirla.

\par
%\textsuperscript{(1007.2)}
\textsuperscript{92:4.2} Pero sin tener en cuenta sus conexiones o derivaciones aparentes, las religiones reveladas siempre están caracterizadas por una creencia en alguna Deidad de valor final y en algún concepto de la supervivencia de la identidad de la personalidad después de la muerte.

\par
%\textsuperscript{(1007.3)}
\textsuperscript{92:4.3} La religión evolutiva es sentimental, pero no lógica. Es la reacción del hombre a la creencia en un mundo hipotético de espíritus fantasmas ---el reflejo humano en forma de creencia provocado por la conciencia de, y el miedo a, lo desconocido. La religión revelada es presentada por el verdadero mundo espiritual; es la respuesta del cosmos superintelectual a la sed que tienen los mortales de creer y confiar en las Deidades universales. La religión evolutiva describe los titubeos tortuosos de la humanidad en busca de la verdad; la religión revelada \textit{es} esa verdad misma.

\par
%\textsuperscript{(1007.4)}
\textsuperscript{92:4.4} Se han producido muchos casos de revelaciones religiosas, pero sólo cinco han tenido una importancia que ha hecho época. Y fueron los siguientes:

\par
%\textsuperscript{(1007.5)}
\textsuperscript{92:4.5} 1. \textit{Las enseñanzas de Dalamatia}. El verdadero concepto de la Fuente-Centro Primera fue promulgado por primera vez en Urantia por los cien miembros corpóreos del estado mayor del Príncipe Caligastia. Esta revelación creciente de la Deidad duró más de trescientos mil años, hasta que fue interrumpida repentinamente por la secesión planetaria y la ruptura del régimen educativo. A excepción del trabajo de Van, la influencia de la revelación dalamatiana se perdió prácticamente para el mundo entero. Incluso los noditas habían olvidado esta verdad en la época de la llegada de Adán. De todos aquellos que recibieron las enseñanzas de los cien, los hombres rojos fueron los que las conservaron durante más tiempo, pero la idea del Gran Espíritu no era más que un concepto nebuloso en la religión amerindia cuando el contacto con el cristianismo lo clarificó y lo reforzó enormemente.

\par
%\textsuperscript{(1007.6)}
\textsuperscript{92:4.6} 2. \textit{Las enseñanzas del Edén}. Adán y Eva describieron de nuevo el concepto del Padre de todos a los pueblos evolutivos. La disgregación del primer Edén detuvo el curso de la revelación adámica antes de que hubiera empezado a efectuarse plenamente. Pero los sacerdotes setitas continuaron las enseñanzas abortadas de Adán, y algunas de estas verdades nunca se han perdido por completo para el mundo. Toda la tendencia de la evolución religiosa levantina fue modificada por las enseñanzas de los setitas. Pero hacia el año 2500 a. de J. C., la humanidad había perdido ampliamente de vista la revelación patrocinada en los tiempos del Edén.

\par
%\textsuperscript{(1007.7)}
\textsuperscript{92:4.7} 3. \textit{Melquisedek de Salem}. Este Hijo de Nebadon, enviado en misión de urgencia al planeta, inauguró la tercera revelación de la verdad en Urantia. Los preceptos cardinales de sus enseñanzas fueron la \textit{confianza} y la \textit{fe}. Enseñó la confianza en la beneficencia omnipotente de Dios y proclamó que la fe era el acto por el cual los hombres conseguían el favor de Dios. Sus enseñanzas se mezclaron gradualmente con las creencias y las prácticas de diversas religiones evolutivas, y finalmente se convirtieron en los sistemas teológicos presentes en Urantia al principio del primer milenio después de Cristo.

\par
%\textsuperscript{(1008.1)}
\textsuperscript{92:4.8} 4. \textit{Jesús de Nazaret}. Cristo Miguel presentó por cuarta vez en Urantia el concepto de Dios como Padre Universal, y esta enseñanza ha perdurado en general desde entonces. La esencia de su enseñanza era el \textit{amor} y el \textit{servicio}, la adoración amorosa que un hijo creado ofrece voluntariamente en reconocimiento al ministerio afectuoso de su Padre Dios, y en respuesta al mismo; el servicio por propia voluntad que estos hijos creados dispensan a sus hermanos, con la alegre comprensión de que mediante este servicio están sirviendo igualmente a Dios Padre.

\par
%\textsuperscript{(1008.2)}
\textsuperscript{92:4.9} 5. \textit{Los documentos de Urantia}. Los documentos, de los cuales éste mismo forma parte, constituyen la presentación más reciente de la verdad a los mortales de Urantia. Estos documentos difieren de todas las revelaciones anteriores, ya que no son el trabajo de una sola personalidad del universo, sino una presentación compuesta realizada por numerosos seres. Pero ninguna revelación puede ser nunca completa hasta que no se alcanza al Padre Universal. Todos los demás ministerios celestiales no son más que parciales, transitorios y prácticamente adaptados a las condiciones locales en el tiempo y el espacio. Aunque una confesión como ésta quizás pueda reducir la fuerza y la autoridad inmediatas de todas las revelaciones, ha llegado la hora en que es conveniente hacer estas sinceras declaraciones incluso a riesgo de debilitar la influencia y la autoridad futuras de esta obra, que es la revelación más reciente de la verdad para las razas mortales de Urantia.

\section*{5. Los grandes dirigentes religiosos}
\par
%\textsuperscript{(1008.3)}
\textsuperscript{92:5.1} En la religión evolutiva se concibe que los dioses existen a imagen y semejanza de los hombres; en la religión revelada se enseña a los hombres que son hijos de Dios ---que incluso están hechos a la imagen finita de la divinidad; en las creencias sintetizadas compuestas por las enseñanzas de la revelación y los productos de la evolución, el concepto de Dios es una mezcla de:

\par
%\textsuperscript{(1008.4)}
\textsuperscript{92:5.2} 1. Las ideas preexistentes de los cultos evolutivos.

\par
%\textsuperscript{(1008.5)}
\textsuperscript{92:5.3} 2. Los ideales sublimes de la religión revelada.

\par
%\textsuperscript{(1008.6)}
\textsuperscript{92:5.4} 3. Los puntos de vista personales de los grandes dirigentes religiosos, los profetas e instructores de la humanidad.

\par
%\textsuperscript{(1008.7)}
\textsuperscript{92:5.5} La mayor parte de las grandes épocas religiosas han sido inauguradas por la vida y las enseñanzas de alguna personalidad sobresaliente; las directrices de un jefe han originado la mayoría de los movimientos morales, dignos de consideración, de la historia. Los hombres siempre han tenido la tendencia de venerar al dirigente, incluso a costa de sus enseñanzas; de reverenciar su personalidad, incluso perdiendo de vista las verdades que proclamaba. Y esto no sucede sin razón; el corazón del hombre evolutivo posee el deseo instintivo de recibir la ayuda de arriba y del más allá. Este anhelo está diseñado para esperar la aparición en la Tierra del Príncipe Planetario y de los Hijos Materiales posteriores. En Urantia, los hombres han estado privados de estos jefes y gobernantes superhumanos, y por eso intentan constantemente compensar esta pérdida envolviendo a sus dirigentes humanos en leyendas relacionadas con sus orígenes sobrenaturales y sus carreras milagrosas.

\par
%\textsuperscript{(1008.8)}
\textsuperscript{92:5.6} Muchas razas han imaginado que sus dirigentes habían nacido de vírgenes; sus carreras están generosamente salpicadas de episodios milagrosos, y sus grupos respectivos continúan esperando su retorno. Los miembros de las tribus de Asia central esperan todavía el regreso de Gengis Kan; en el Tíbet, China y la India esperan a Buda, y en el islam, a Mahoma; entre los amerindios, a Hesunanín Onamonalontón; entre los hebreos se trataba en general del regreso de Adán como gobernante material. En Babilonia, el dios Marduc era una perpetuación de la leyenda de Adán, la idea del hijo de Dios, el eslabón entre el hombre y Dios. Después de la aparición de Adán en la Tierra, los supuestos hijos de Dios fueron frecuentes entre las razas del mundo.

\par
%\textsuperscript{(1009.1)}
\textsuperscript{92:5.7} Pero sin tener en cuenta el temor supersticioso que a menudo inspiraban, sigue siendo un hecho que estos instructores fueron las personalidades temporales que sirvieron de puntos de apoyo sobre los que dependieron las palancas de la verdad revelada para hacer progresar la moralidad, la filosofía y la religión de la humanidad.

\par
%\textsuperscript{(1009.2)}
\textsuperscript{92:5.8} Ha habido centenares de dirigentes religiosos a lo largo del millón de años de la historia humana de Urantia, desde Onagar hasta el Gurú Nanek. Durante este tiempo se han producido muchos flujos y reflujos en la marea de la verdad religiosa y de la fe espiritual, y cada renacimiento de la religión urantiana ha estado identificado, en el pasado, con la vida y las enseñanzas de algún dirigente religioso. Al examinar los instructores de los tiempos recientes, puede resultar útil agruparlos en siete épocas religiosas mayores de la Urantia postadámica:

\par
%\textsuperscript{(1009.3)}
\textsuperscript{92:5.9} 1. \textit{El período setita}. Los sacerdotes setitas, regenerados bajo la dirección de Amosad, se convirtieron en los grandes educadores postadámicos. Ejercieron su actividad en todas las tierras de los anditas, y su influencia sobrevivió durante más tiempo entre los griegos, los sumerios y los hindúes. Entre estos últimos han continuado hasta la época actual bajo la forma de los brahmanes de la fe hindú. Los setitas y sus seguidores nunca perdieron por completo el concepto de la Trinidad revelado por Adán.

\par
%\textsuperscript{(1009.4)}
\textsuperscript{92:5.10} 2. \textit{La era de los misioneros de Melquisedek}. La religión de Urantia fue regenerada en gran medida por los esfuerzos de los educadores que fueron nombrados por Maquiventa Melquisedek cuando éste vivía y enseñaba en Salem, cerca de dos mil años antes de Cristo. Estos misioneros proclamaron que la fe era el precio del favor de Dios, y aunque sus enseñanzas no produjeron la aparición inmediata de religiones, sin embargo formaron las bases sobre las cuales los instructores posteriores de la verdad construyeron las religiones de Urantia.

\par
%\textsuperscript{(1009.5)}
\textsuperscript{92:5.11} 3. \textit{La era posterior a Melquisedek}. Tanto Amenemope como Akenatón enseñaron durante este período, pero el genio religioso sobresaliente de la era posterior a Melquisedek fue el jefe de un grupo de beduinos levantinos, el fundador de la religión hebrea ---Moisés. Moisés enseñó el monoteísmo\footnote{\textit{Un único Dios}: 2 Re 19:19; 1 Cr 17:20; Neh 9:6; Sal 86:10; Eclo 36:5; Is 37:16; 44:6,8; 45:5-6,21; Dt 4:35,39; Jn 17:3; Ro 3:30; 1 Co 8:4-6; Gl 3:20; Ef 4:6; 1 Ti 2:5; Stg 2:19; 1 Sam 2:2; 2 Sam 7:22.}. Dijo: <<Escucha, oh Israel, el Señor nuestro Dios es un solo Dios>>\footnote{\textit{Escucha, oh Israel, un solo Dios}: Dt 6:4; Mc 12:29,32.}. <<Es el Señor el que es Dios. No hay ningún otro además de él>>\footnote{\textit{El Señor es Dios, no otro sino él}: Dt 4:35,39.}. Trató insistentemente de desarraigar de su pueblo los vestigios del culto a los fantasmas, llegando incluso a establecer la pena de muerte para los que lo practicaran. El monoteísmo de Moisés fue adulterado por sus sucesores, pero en tiempos posteriores éstos volvieron a muchas de sus enseñanzas. La grandeza de Moisés reside en su sabiduría y su sagacidad. Otros hombres han tenido unos conceptos más grandes de Dios, pero ninguno ha tenido nunca tanto éxito convenciendo a grandes cantidades de personas para que adoptaran unas creencias tan avanzadas.

\par
%\textsuperscript{(1009.6)}
\textsuperscript{92:5.12} 4. \textit{El siglo sexto antes de Cristo}. Éste fue uno de los siglos de despertar religioso más grandes que se haya visto jamás en Urantia. Muchos hombres surgieron para proclamar la verdad, y entre ellos se puede citar a Gautama, Confucio, Lao-Tse, Zoroastro y los educadores jainistas. Las enseñanzas de Gautama se han difundido ampliamente por Asia, y millones de personas lo veneran como Buda. Confucio supuso para la moral china lo mismo que Platón para la filosofía griega, y aunque las enseñanzas de los dos tuvieron repercusiones religiosas, ninguno de ellos era en realidad un educador religioso; Lao-Tse concibió más cosas sobre Dios en el Tao que Confucio en las humanidades o que Platón en el idealismo. Aunque Zoroastro estaba muy afectado por el concepto predominante del dualismo espiritual, de los espíritus buenos y malos, al mismo tiempo exaltó claramente la idea de una Deidad eterna y de la victoria final de la luz sobre la oscuridad.

\par
%\textsuperscript{(1010.1)}
\textsuperscript{92:5.13} 5. \textit{El primer siglo después de Cristo}. Como instructor religioso, Jesús de Nazaret partió del culto que había establecido Juan el Bautista y se alejó tanto como pudo de los ayunos y las formas. Aparte de Jesús, Pablo de Tarso y Filón de Alejandría fueron los educadores más grandes de esta era. Sus conceptos de la religión han jugado un papel predominante en la evolución de la fe que lleva el nombre de Cristo.

\par
%\textsuperscript{(1010.2)}
\textsuperscript{92:5.14} 6. \textit{El siglo sexto después de Cristo}. Mahoma fundó una religión que era superior a muchos credos de su época. Su religión fue una protesta contra las exigencias sociales de las doctrinas extranjeras y contra la incoherencia de la vida religiosa de su propio pueblo.

\par
%\textsuperscript{(1010.3)}
\textsuperscript{92:5.15} 7. \textit{El siglo quince después de Cristo}. Este período presenció dos movimientos religiosos: la ruptura de la unidad del cristianismo en occidente y la síntesis de una nueva religión en oriente. En Europa, el cristianismo institucionalizado había alcanzado el grado de rigidez que hacía que cualquier crecimiento adicional resultara incompatible con la unidad. En oriente, las enseñanzas combinadas del Islam, el hinduismo y el budismo fueron sintetizadas por Nanek y sus seguidores en el sijismo, una de las religiones más avanzadas de Asia.

\par
%\textsuperscript{(1010.4)}
\textsuperscript{92:5.16} El futuro de Urantia estará caracterizado sin duda por la aparición de instructores de la verdad religiosa ---la Paternidad de Dios y la fraternidad de todas las criaturas. Pero es de esperar que los esfuerzos ardientes y sinceros de esos futuros profetas estén menos dirigidos hacia el reforzamiento de las barreras entre las religiones, y más encaminados hacia el acrecentamiento de una fraternidad religiosa de adoración espiritual entre los numerosos seguidores de las diferentes teologías intelectuales que tanto caracterizan al planeta Urantia de Satania.

\section*{6. Las religiones compuestas}
\par
%\textsuperscript{(1010.5)}
\textsuperscript{92:6.1} Las religiones urantianas del siglo veinte ofrecen un estudio interesante sobre la evolución social del impulso humano a la adoración. Muchas doctrinas han progresado muy poco desde los tiempos del culto a los fantasmas. Los pigmeos de África no tienen reacciones religiosas como tales, aunque algunos de ellos creen un poco en un entorno de espíritus. Hoy están exactamente en el punto en que se encontraba el hombre primitivo cuando empezó la evolución de la religión. La creencia fundamental de la religión primitiva era la supervivencia después de la muerte. La idea de adorar a un Dios personal indica un desarrollo evolutivo avanzado, e incluso la primera etapa de la revelación. Los dayacs sólo han desarrollado las prácticas religiosas más primitivas. Los esquimales y amerindios relativamente recientes tenían unos conceptos muy pobres de Dios; creían en los fantasmas y tenían una idea imprecisa de algún tipo de supervivencia después de la muerte. Los indígenas australianos de hoy en día sólo tienen el miedo a los fantasmas, el temor a la oscuridad y una veneración rudimentaria de los antepasados. Los zulúes están precisamente desarrollando una religión de miedo a los fantasmas y de sacrificios. Muchas tribus africanas, excepto aquellas que han recibido el trabajo misionero de los cristianos y los mahometanos, no han sobrepasado todavía el estado fetichista de la evolución religiosa. Pero algunos grupos se han mantenido fieles durante mucho tiempo a la idea del monoteísmo, como los antiguos tracios, que también creían en la inmortalidad.

\par
%\textsuperscript{(1010.6)}
\textsuperscript{92:6.2} En Urantia, la religión evolutiva y la religión revelada progresan una al lado de la otra, mezclándose y fundiéndose en los diversos sistemas teológicos que se encontraban en el mundo en la época de la redacción de estos documentos. Estas religiones, las del siglo veinte de Urantia, se pueden enumerar como sigue:

\par
%\textsuperscript{(1011.1)}
\textsuperscript{92:6.3} 1. El hinduismo ---la más antigua.

\par
%\textsuperscript{(1011.2)}
\textsuperscript{92:6.4} 2. La religión hebrea.

\par
%\textsuperscript{(1011.3)}
\textsuperscript{92:6.5} 3. El budismo.

\par
%\textsuperscript{(1011.4)}
\textsuperscript{92:6.6} 4. Las enseñanzas de Confucio.

\par
%\textsuperscript{(1011.5)}
\textsuperscript{92:6.7} 5. Las creencias taoistas.

\par
%\textsuperscript{(1011.6)}
\textsuperscript{92:6.8} 6. El zoroastrismo.

\par
%\textsuperscript{(1011.7)}
\textsuperscript{92:6.9} 7. El sintoísmo.

\par
%\textsuperscript{(1011.8)}
\textsuperscript{92:6.10} 8. El jainismo.

\par
%\textsuperscript{(1011.9)}
\textsuperscript{92:6.11} 9. El cristianismo.

\par
%\textsuperscript{(1011.10)}
\textsuperscript{92:6.12} 10. El islam.

\par
%\textsuperscript{(1011.11)}
\textsuperscript{92:6.13} 11. El sijismo ---la más reciente.

\par
%\textsuperscript{(1011.12)}
\textsuperscript{92:6.14} Las religiones más avanzadas de los tiempos antiguos eran el judaísmo y el hinduismo, y cada una de ellas ha tenido respectivamente una gran influencia sobre el curso del desarrollo religioso en oriente y occidente. Tanto los hindúes como los hebreos creían que sus religiones eran inspiradas y reveladas, y que todas las demás eran formas decadentes de la única fe verdadera.

\par
%\textsuperscript{(1011.13)}
\textsuperscript{92:6.15} La India está dividida entre los hindúes, los sijs, los mahometanos y los jaínes, y cada uno describe a Dios, al hombre y al universo según sus conceptos diferentes. China sigue las enseñanzas del Tao y de Confucio; el sintoísmo se venera en el Japón.

\par
%\textsuperscript{(1011.14)}
\textsuperscript{92:6.16} Las grandes doctrinas internacionales, interraciales, son la hebrea, la budista, la cristiana y la islámica. El budismo se extiende desde Ceilán y Birmania, a través del Tíbet y China, hasta el Japón. Ha demostrado una facultad de adaptación a las costumbres de numerosos pueblos que sólo ha sido igualada por el cristianismo.

\par
%\textsuperscript{(1011.15)}
\textsuperscript{92:6.17} La religión hebrea engloba la transición filosófica entre el politeísmo y el monoteísmo\footnote{\textit{Un único Dios}: 2 Re 19:19; 1 Cr 17:20; Neh 9:6; Sal 86:10; Eclo 36:5; Is 37:16; 44:6,8; 45:5-6,21; Dt 4:35,39; 6:4; Mc 12:29,32; Jn 17:3; Ro 3:30; 1 Co 8:4-6; Gl 3:20; Ef 4:6; 1 Ti 2:5; Stg 2:19; 1 Sam 2:2; 2 Sam 7:22.}; es un eslabón evolutivo entre las religiones de la evolución y las religiones reveladas. Los hebreos fueron el único pueblo occidental que siguió a sus dioses evolutivos primitivos desde el principio hasta el fin, hasta el Dios de la revelación. Pero esta verdad nunca fue ampliamente aceptada hasta la época de Isaías, que enseñó de nuevo la idea mixta de una deidad racial fusionada con un Creador Universal: <<Oh Señor de los ejércitos\footnote{\textit{Señor de los ejércitos}: Sal 46:7,11; Is 8:13; Jer 35:17; 2 Sam 7:26. \textit{Oh Señor de los ejércitos, solo tú eres Dios}: Is 37:16.}, Dios de Israel, tú eres Dios, sólo tú lo eres; tú has creado el cielo y la Tierra>>\footnote{\textit{Dios creó el cielo y la tierra}: Gn 1:1; 2:4; Ex 20:11; 31:17; 2 Re 19:15; 2 Cr 2:12; Neh 9:6; Sal 115:15-16; 121:2; 124:8; 134:3; 146:6; Is 37:16; 42:5; 45:12,18; Jer 10:11-12; 32:17; 51:15-16; Hch 4:24; 14:15; Col 1:16; Ap 4:11; 10:6.}. En un momento dado, la esperanza de supervivencia de la civilización occidental residió en los sublimes conceptos hebreos de la bondad y en los avanzados conceptos helénicos de la belleza.

\par
%\textsuperscript{(1011.16)}
\textsuperscript{92:6.18} La religión cristiana es la religión acerca de la vida y las enseñanzas de Cristo, basada en la teología del judaísmo, modificada además por la asimilación de algunas enseñanzas de Zoroastro y de la filosofía griega, y formulada principalmente por tres personalidades: Filón, Pedro y Pablo. Ha pasado por muchas fases en su evolución desde los tiempos de Pablo, y se ha occidentalizado tanto que muchos pueblos no europeos consideran naturalmente al cristianismo como la extraña revelación de un Dios extraño, destinada a los extraños.

\par
%\textsuperscript{(1011.17)}
\textsuperscript{92:6.19} El islam es la conexión religioso-cultural entre África del norte, el Levante y el sudeste de Asia. La teología judía, en unión con las enseñanzas cristianas posteriores, fue la que hizo monoteísta al islam. Los seguidores de Mahoma tropezaron con las enseñanzas avanzadas sobre la Trinidad; no podían comprender la doctrina de tres personalidades divinas y una sola Deidad. Siempre es difícil inducir a la mente evolutiva a que acepte \textit{repentinamente} una verdad revelada avanzada. El hombre es una criatura evolutiva y, en general, debe conseguir su religión por medio de técnicas evolutivas.

\par
%\textsuperscript{(1012.1)}
\textsuperscript{92:6.20} El culto a los antepasados constituyó antiguamente un progreso indudable en la evolución religiosa, pero es a la vez sorprendente y lamentable que este concepto primitivo continúe existiendo en China, el Japón y la India en medio de otras creencias relativamente más avanzadas, tales como el budismo y el hinduismo. En occidente, el culto a los antepasados se convirtió en la veneración de los dioses nacionales y en el respeto por los héroes de la raza. En el siglo veinte, esta religión nacionalista de veneración de los héroes hace su aparición en los diversos laicismos radicales y nacionalistas que caracterizan a muchas razas y naciones occidentales. Esta misma actitud se encuentra también en gran parte en las grandes universidades y en las comunidades industriales más importantes de los pueblos de habla inglesa. La idea de que la religión no es más que <<una búsqueda en común de la buena vida>> no difiere mucho de estos conceptos. Las <<religiones nacionales>> no son más que una reversión a la adoración primitiva romana de los emperadores, y al sintoísmo ---la adoración del Estado en la familia imperial.

\section*{7. La evolución ulterior de la religión}
\par
%\textsuperscript{(1012.2)}
\textsuperscript{92:7.1} La religión no puede volverse nunca un hecho científico. La filosofía puede descansar en verdad sobre una base científica, pero la religión seguirá siendo siempre evolutiva o revelada, o una posible combinación de las dos, tal como sucede en el mundo de hoy en día.

\par
%\textsuperscript{(1012.3)}
\textsuperscript{92:7.2} No se pueden inventar nuevas religiones; o éstas se desarrollan por evolución, o son \textit{reveladas repentinamente}. Todas las religiones evolutivas nuevas son simplemente las expresiones progresivas de creencias antiguas, nuevas adaptaciones y nuevos ajustes. Lo antiguo no deja de existir; está fundido en lo nuevo, tal como el sijismo brotó y floreció de la tierra y las formas del hinduismo, el budismo, el islam y otros cultos contemporáneos. La religión primitiva era muy democrática; el salvaje prestaba o pedía prestado rápidamente. El egotismo teológico autocrático e intolerante sólo apareció con la religión revelada.

\par
%\textsuperscript{(1012.4)}
\textsuperscript{92:7.3} Las numerosas religiones de Urantia son todas buenas en la medida en que llevan al hombre hacia Dios y aportan al hombre la comprensión del Padre. Es una falacia, para cualquier grupo de personas religiosas, imaginar que su credo es \textit{La Verdad}; esta actitud demuestra más arrogancia teológica que certidumbre en la fe. No existe una religión en Urantia que no pueda estudiar y asimilar provechosamente lo mejor de las verdades contenidas en todas las otras doctrinas, porque todas contienen verdades. Los practicantes de la religión harían mejor en tomar prestado lo mejor de la fe espiritual viviente de sus vecinos, en lugar de denunciar lo peor de sus supersticiones sobrevivientes y de sus rituales anticuados.

\par
%\textsuperscript{(1012.5)}
\textsuperscript{92:7.4} Todas estas religiones han surgido como consecuencia de la reacción intelectual variable de los hombres a sus directrices espirituales idénticas. Los hombres nunca pueden esperar alcanzar una uniformidad de credos, dogmas y ritos ---pues éstos son intelectuales; pero sí pueden, y algún día lo lograrán, conseguir la unidad en la adoración sincera del Padre de todos, porque ésta es espiritual, y es eternamente cierto que en espíritu todos los hombres son iguales.

\par
%\textsuperscript{(1012.6)}
\textsuperscript{92:7.5} La religión primitiva era sobre todo una conciencia de los valores materiales, pero la civilización eleva los valores religiosos, porque la verdadera religión es la dedicación del yo al servicio de los valores significativos y supremos. A medida que evoluciona la religión, la ética se convierte en la filosofía de la moral, y la moralidad se vuelve la disciplina del yo gracias a los criterios de los significados superiores y de los valores supremos ---de los ideales divinos y espirituales. La religión se convierte así en una devoción espontánea y delicada, en la experiencia viviente de la fidelidad del amor.

\par
%\textsuperscript{(1013.1)}
\textsuperscript{92:7.6} La calidad de una religión se puede apreciar por:

\par
%\textsuperscript{(1013.2)}
\textsuperscript{92:7.7} 1. La altura de sus valores ---las fidelidades.

\par
%\textsuperscript{(1013.3)}
\textsuperscript{92:7.8} 2. La profundidad de sus significados ---la sensibilización del individuo a la apreciación idealista de estos valores superiores.

\par
%\textsuperscript{(1013.4)}
\textsuperscript{92:7.9} 3. La intensidad de la consagración ---el grado de devoción a estos valores divinos.

\par
%\textsuperscript{(1013.5)}
\textsuperscript{92:7.10} 4. El progreso sin trabas de la personalidad en este camino cósmico de vida espiritual idealista, de comprensión de la filiación con Dios y de ciudadanía progresiva sin fin en el universo.

\par
%\textsuperscript{(1013.6)}
\textsuperscript{92:7.11} Los significados religiosos progresan en la conciencia personal cuando el niño transfiere sus ideas de la omnipotencia desde sus padres hasta Dios. Toda la experiencia religiosa de ese niño dependerá considerablemente de si la relación con sus padres ha estado dominada por el miedo o por el amor. Los esclavos siempre han tenido grandes dificultades para transformar el miedo a sus amos en conceptos de amor por Dios. La civilización, la ciencia y las religiones avanzadas deben liberar a la humanidad de los miedos procedentes del temor a los fenómenos naturales. Una cultura más amplia debería liberar así a los mortales instruidos de tener que depender totalmente de los intermediarios para comulgar con la Deidad.

\par
%\textsuperscript{(1013.7)}
\textsuperscript{92:7.12} Estas etapas intermedias de titubeo idólatra en el proceso de transferir la veneración de lo humano y visible a lo divino e invisible son inevitables, pero la conciencia de las facilidades aportadas por el ministerio del espíritu divino interior debería abreviar estas etapas. Sin embargo, el hombre ha sido profundamente influido no sólo por sus conceptos sobre la Deidad, sino también por el carácter de los héroes que ha escogido honrar. Es muy lamentable que aquellos que han llegado a venerar al Cristo divino y resucitado hayan pasado por alto al hombre ---al héroe valiente e intrépido--- a Josué ben José.

\par
%\textsuperscript{(1013.8)}
\textsuperscript{92:7.13} El hombre moderno tiene una conciencia suficiente de la religión, pero sus costumbres devotas están confusas y desacreditadas debido a su metamorfosis social acelerada y a sus desarrollos científicos sin precedentes. Los hombres y las mujeres pensantes quieren que la religión sea definida de nuevo, y esta exigencia obligará a la religión a volverse a evaluar a sí misma.

\par
%\textsuperscript{(1013.9)}
\textsuperscript{92:7.14} El hombre moderno se enfrenta a la tarea de hacer más reajustes en los valores humanos en una sola generación que en dos mil años. Y todo esto influye sobre la actitud social hacia la religión, porque la religión es una manera de vivir así como una técnica de pensamiento.

\par
%\textsuperscript{(1013.10)}
\textsuperscript{92:7.15} La verdadera religión debe ser siempre y al mismo tiempo el eterno fundamento y la estrella orientadora de todas las civilizaciones duraderas.

\par
%\textsuperscript{(1013.11)}
\textsuperscript{92:7.16} [Presentado por un Melquisedek de Nebadon.]


\newpage
\pagestyle{empty}

\par {\huge Abreviaturas}
\bigbreak
\bigbreak
\begin{multicols}{2}
	\par LU \textit{(El Libro de Urantia)}
	\bigbreak
	\par Libros bíblicos:
	\bigbreak
	\par Abd \textit{(Abdías)}
	\par Am \textit{(Amós)}
	\par Ap \textit{(Apocalipsis)}
	\par Bar \textit{(Baruc)}
	\par Co \textit{(Epístola a los Corintios)}
	\par Cnt \textit{(El Cantar de los Cantares)}
	\par Col \textit{(Epístola a los Colosenses)}
	\par Cr \textit{(Crónicas)}
	\par Dn \textit{(Daniel)}
	\par Dt \textit{(Deuteronomio)}
	\par Ec \textit{(Eclesiastés)}
	\par Eclo \textit{(Ecclesiástico)}
	\par Ef \textit{(Epístola a los Efesios)}
	\par Esd \textit{(Esdras)}
	\par Est \textit{(Ester)}
	\par Ex \textit{(Éxodo)}
	\par Ez \textit{(Ezequiel)} 
	\par Flm \textit{(Epístola a Filemón)}
	\par Flp \textit{(Epístola a los Filipenses)}
	\par Gl \textit{(Epítosla a los Gálatas)}
	\par Gn \textit{(Génesis)}
	\par Hab \textit{(Habacuc)} 
	\par Hag \textit{(Ageo)}
	\par Hch \textit{(Hechos de los Apóstoles)}
	\par Heb \textit{(Epístola a los Hebreos)}
	\par Is \textit{(Isaías)}
	\par Jer \textit{(Jeremías)}
	\par Jl \textit{(Joel)}
	\par Jn \textit{(Juan, evangelio y epístolas)}
	\par Job \textit{(Job)}
	\par Jon \textit{(Jonás)}
	\par Jos \textit{(Josué)}
	\par Jud \textit{(Epístola de Judas)}
	\par Jue \textit{(Jueces)}
	\par Lc \textit{(Lucas)}
	\par Lm \textit{(Lamentaciones)}
	\par Lv \textit{(Levítico)}
	\par Mac \textit{(Macabeos)}
	\par Mal \textit{(Malaquías)}
	\par Mc \textit{(Marcos)}
	\par Miq \textit{(Miqueas)} 
	\par Mt \textit{(Mateo)}
	\par Nah \textit{(Nahúm)}
	\par Neh \textit{(Nehemías)} 
	\par Nm \textit{(Números)}
	\par Os \textit{(Oseas)}
	\par P \textit{(Epístola de Pedro)}
	\par Pr \textit{(Proverbios)}
	\par Re \textit{(Reyes)}
	\par Ro \textit{(Epístola a los Romanos)}
	\par Rt \textit{(Rut)}
	\par Sab \textit{(Sabiduría)}
	\par Sal \textit{(Salmos)}
	\par Sam \textit{(Samuel)}
	\par Sof \textit{(Sofonías)}
	\par Stg \textit{(Epístola a Santiago)}
	\par Ti \textit{(Epístola a Timoteo)}
	\par Tit \textit{(Epítosla a Tito)}
	\par Ts \textit{(Epístola a los Tesalonicenses)}
	\par Zac \textit{(Zacarías)}
	\bigbreak
	\par Libros bíblicos apócrifos:
	\bigbreak 
	\par AsMo \textit{(Asunción de Moisés)}
	\par Bel \textit{(Bel y el Dragón)} 
	\par Hen \textit{(Enoc)} 
	\par Man \textit{(Oración de Manasés)} 
	\par Tb \textit{(Tobit)}
	\bigbreak
	\par Libros de otras religiones: 
	\bigbreak
	\par XXX \textit{(YYYY)}
	
	
\end{multicols}

\end{document}

