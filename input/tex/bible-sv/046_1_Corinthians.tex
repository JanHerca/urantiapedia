\begin{document}

\title{1 Korintierbrevet}


\chapter{1}

\par 1 Paulus, genom Guds vilja kallad till Kristi Jesu apostel, så ock brodern Sostenes,
\par 2 hälsar den Guds församling som finnes i Korint, de i Kristus Jesus helgade, dem som äro kallade och heliga, jämte alla andra som åkalla vår Herres, Jesu Kristi, namn, på alla orter där de eller vi bo.
\par 3 Nåd vare med eder och frid ifrån Gud, vår Fader, och Herren Jesus Kristus.
\par 4 Jag tackar Gud alltid för eder skull, för den Guds nåd som har blivit eder given i Kristus Jesus,
\par 5 att I haven i honom blivit rikligen begåvade i alla stycken, i fråga om allt vad tal och kunskap heter.
\par 6 Så har ju ock vittnesbördet om Kristus blivit befäst hos eder,
\par 7 så att I icke stån tillbaka i fråga om någon nådegåva, medan I vänten på vår Herres, Jesu Kristi, uppenbarelse.
\par 8 Han skall ock göra eder ståndaktiga intill änden, så att I ären ostraffliga på vår Herres, Jesu Kristi, dag.
\par 9 Gud är trofast, han genom vilken I haven blivit kallade till gemenskap med hans Son, Jesus Kristus, vår Herre.
\par 10 Men jag förmanar eder, mina bröder, vid vår Herres, Jesu Kristi, namn, att alla vara eniga i edert tal och att icke låta söndringar finnas bland eder, utan hålla fast tillhopa i samma sinnelag och samma tänkesätt.
\par 11 Det har nämligen av Kloes husfolk blivit mig berättat om eder, mina bröder, att tvister hava uppstått bland eder.
\par 12 Härmed menar jag att bland eder den ene säger: "Jag håller mig till Paulus", den andre: "Jag håller mig till Apollos", en annan: "Jag håller mig till Cefas", åter en annan: "Jag håller mig till Kristus." -
\par 13 Är då Kristus delad? Icke blev väl Paulus korsfäst för eder? Och icke bleven I väl döpta i Paulus' namn?
\par 14 Jag tackar Gud för att jag icke har döpt någon bland eder utom Krispus och Gajus,
\par 15 så att ingen kan säga att I haven blivit döpta i mitt namn.
\par 16 Dock, jag har döpt också Stefanas' husfolk; om jag eljest har döpt någon vet jag icke.
\par 17 Ty Kristus har icke sänt mig till att döpa, utan till att förkunna evangelium, och detta icke med en visdom som består i ord, för att Kristi kors icke skall berövas sin kraft.
\par 18 Ty talet om korset är visserligen en dårskap för dem som gå förlorade, men för oss som bliva frälsta är det en Guds kraft.
\par 19 Det är ju skrivet: "Jag skall göra de visas vishet om intet, och de förståndigas förstånd skall jag slå ned."
\par 20 Ja, var äro de visa? Var äro de skriftlärda? Var äro denna tidsålders klyftiga män? Har icke Gud gjort denna världens visdom till dårskap?
\par 21 Jo, eftersom världen icke genom sin visdom lärde känna Gud i hans visdom, behagade det Gud att genom den dårskap han lät predikas frälsa dem som tro.
\par 22 Ty judarna begära tecken, och grekerna åstunda visdom,
\par 23 vi åter predika en korsfäst Kristus, en som för judarna är en stötesten och för hedningarna en dårskap,
\par 24 men som för de kallade, vare sig judar eller greker, är en Kristus som är Guds kraft och Guds visdom.
\par 25 Ty Guds dårskap är visare än människor, och Guds svaghet är starkare än människor.
\par 26 Ty betänken, mina bröder, huru det var vid eder kallelse: icke många som voro visa efter köttet blevo kallade, icke många mäktiga, icke många av förnämlig släkt.
\par 27 Men det som för världen var dåraktigt, det utvalde Gud, för att han skulle låta de visa komma på skam.
\par 28 Och det som i världen var svagt, det utvalde Gud, för att han skulle låta det starka komma på skam. Och det som i världen var ringa och föraktat, det utvalde Gud - ja, det som ingenting var - för att han skulle göra det till intet, som någonting var.
\par 29 Ty han ville icke att något kött skulle kunna berömma sig inför Gud.
\par 30 Men hans verk är det, att I ären i Kristus Jesus, som för oss har blivit till visdom från Gud, till rättfärdighet och helgelse och till förlossning,
\par 31 för att så skall ske, som det är skrivet: "Den som vill berömma sig, han berömme sig av Herren."

\chapter{2}

\par 1 När jag kom till eder, mina bröder, var det också icke med höga ord eller hög visdom som jag kom och frambar för eder Guds vittnesbörd.
\par 2 Ty jag hade beslutit mig för, att medan jag var bland eder icke veta om något annat än Jesus Kristus, och honom såsom korsfäst.
\par 3 Och jag uppträdde hos eder i svaghet och med fruktan och mycken bävan.
\par 4 Och mitt tal och min predikan framställdes icke med övertalande visdomsord, utan med en bevisning i ande och kraft;
\par 5 ty eder tro skulle icke vara grundad på människors visdom, utan på Guds kraft.
\par 6 Visdom tala vi dock bland dem som äro fullmogna, men en visdom som icke tillhör denna tidsålder eller denna tidsålders mäktige, vilkas makt bliver till intet.
\par 7 Nej, vi tala Guds hemliga visdom, den fördolda, om vilken Gud, redan före tidsåldrarnas begynnelse, har bestämt att den skall bliva oss till härlighet,
\par 8 och som ingen av denna tidsålders mäktige har känt; ty om de hade känt den, så hade de icke korsfäst härlighetens Herre.
\par 9 Vi tala - såsom det heter i skriften - "vad intet öga har sett och intet öra har hört, och vad ingen människas hjärta har kunnat tänka, vad Gud har berett åt dem som älska honom".
\par 10 Ty för oss har Gud uppenbarat det genom sin Ande. Anden utrannsakar ju allt, ja ock Guds djuphet.
\par 11 Ty vilken människa vet vad som är i en människa, utom den människans egen ande? Likaså känner ingen vad som är i Gud, utom Guds Ande.
\par 12 Men vi hava icke fått världens ande, utan den Ande som är av Gud, för att vi skola veta vad som har blivit oss skänkt av Gud.
\par 13 Om detta tala vi ock, icke med sådana ord som mänsklig visdom lär oss, utan med sådana ord som Anden lär oss; vi hava ju att tyda andliga ting för andliga människor.
\par 14 Men en "själisk" människa tager icke emot vad som hör Guds Ande till. Det är henne en dårskap, och hon kan icke förstå det, ty det måste utgrundas på ett andligt sätt.
\par 15 Den andliga människan åter kan utgrunda allt, men själv kan hon icke utgrundas av någon.
\par 16 Ty "vem har lärt känna Herrens sinne, så att han skulle kunna undervisa honom?" Men vi hava Kristi sinne.

\chapter{3}

\par 1 Och jag kunde icke tala till eder, mina bröder, såsom till andliga människor, utan måste tala såsom till människor av köttslig natur, såsom till dem som ännu äro barn i Kristus.
\par 2 Mjölk gav jag eder att dricka; fast föda gav jag eder icke, ty det fördrogen I då ännu icke. Ja, icke ens nu fördragen I det,
\par 3 eftersom I ännu haven ett köttsligt sinne. Ty om avund och kiv finnes bland eder, haven I icke då ett köttsligt sinne, och vandren I icke då på vanligt människosätt?
\par 4 När den ene säger: "Jag håller mig till Paulus" och den andre: "Jag håller mig till Apollos", ären I icke då lika hopen av människor?
\par 5 Vad är då Apollos? Vad är Paulus? Allenast tjänare, genom vilka I haven kommit till tro; och de äro det i mån av vad Herren har beskärt åt var och en av dem.
\par 6 Jag planterade, Apollos vattnade, men Gud gav växten.
\par 7 Alltså kommer det icke an på den som planterar, ej heller på den som vattnar, utan på Gud, som giver växten.
\par 8 Den som planterar och den som vattnar - den ene är såsom den andre, dock så, att var och en skall få sin särskilda lön efter sitt särskilda arbete.
\par 9 Ty vi äro Guds medarbetare; I ären ett Guds åkerfält, en Guds byggnad.
\par 10 Efter den Guds nåd som blev mig given lade jag grunden såsom en förfaren byggmästare, och en annan bygger nu vidare därpå. Men var och en må se till, huru han bygger därpå.
\par 11 Ty en annan grund kan ingen lägga, än den som är lagd, nämligen Jesus Kristus;
\par 12 men om någon bygger på den grunden med guld, silver och dyrbara stenar eller med trä, hö och strå,
\par 13 så skall det en gång visa sig huru det är med vars och ens verk. "Den dagen" skall göra det kunnigt; ty den skall uppenbaras i eld, och hurudant vars och ens verk är, det skall elden pröva.
\par 14 Om det byggnadsverk, som någon har uppfört på den grunden, bliver beståndande, så skall han undfå lön;
\par 15 men om hans verk brännes upp, så skall han gå miste om lönen. Själv skall han dock bliva frälst, men såsom igenom eld.
\par 16 Veten I icke att I ären ett Guds tempel och att Guds Ande bor i eder?
\par 17 Om nu någon fördärvar Guds tempel, så skall Gud fördärva honom; ty Guds tempel är heligt, och det templet ären I.
\par 18 Ingen bedrage sig själv. Om någon bland eder menar sig vara vis genom denna tidsålders visdom, så blive han en dåre, för att han skall kunna bliva vis.
\par 19 Ty denna världens visdom är dårskap inför Gud. Det är ju skrivet: "Han fångar de visa i deras klokskap";
\par 20 så ock: "Herren känner de visas tankar, han vet att de äro fåfängliga."
\par 21 Så berömme sig då ingen av människor. Allt hör ju eder till;
\par 22 det må vara Paulus eller Apollos eller Cefas eller hela världen, det må vara liv eller död, vad som nu är, eller vad som skall komma, alltsammans hör eder till.
\par 23 Men I hören Kristus till, och Kristus hör Gud till.

\chapter{4}

\par 1 Såsom Kristi tjänare och såsom förvaltare av Guds hemligheter, så må man anse oss.
\par 2 Vad man nu därutöver söker hos förvaltare är att en sådan må befinnas vara trogen.
\par 3 För mig betyder det likväl föga att I - eller överhuvud någon mänsklig domstol - sätten eder till doms över mig. Ja, jag vill icke ens sätta mig till doms över mig själv.
\par 4 Ty väl vet jag intet med mig, men därigenom är jag icke rättfärdigad; det är Herren som sitter till doms över mig.
\par 5 Dömen därför icke förrän tid är, icke förrän Herren kommer, han som skall draga fram i ljuset vad som är fördolt i mörker och uppenbara alla hjärtans rådslag. Och då skall var och en undfå av Gud den berömmelse som honom tillkommer.
\par 6 Detta, mina bröder, har jag nu för eder skull så framställt, som gällde det mig och Apollos; ty jag vill att I skolen i fråga om oss lära eder detta: "Icke utöver vad skrivet är." Jag vill icke att I skolen stå emot varandra, uppblåsta var och en över sin lärare.
\par 7 Vem säger då att du har något företräde? Och vad äger du, som du icke har fått dig givet? Men har du nu fått dig givet vad du har, huru kan du då berömma dig, såsom om du icke hade fått det dig givet?
\par 8 I ären kantänka redan mätta, I haven redan blivit rika; oss förutan haven I blivit sannskyldiga konungar! Ja, jag skulle önska att I verkligen haden blivit konungar, så att vi kunde få bliva edra medkonungar.
\par 9 Mig tyckes nämligen att Gud har ställt oss apostlar här såsom de ringaste bland alla, såsom livdömda män; ett skådespel hava vi ju blivit för världen, för både änglar och människor.
\par 10 Vi äro dårar för Kristi skull, men I ären kloka i Kristus; vi äro svaga, men I ären starka; I ären ärade, men vi äro föraktade.
\par 11 Ännu i denna stund lida vi både hunger och törst, vi måste gå nakna, vi få uppbära hugg och slag, vi hava intet stadigt hemvist,
\par 12 vi måste möda oss och arbeta med våra händer. Vi bliva smädade och välsigna likväl; vi lida förföljelse och härda dock ut;
\par 13 man talar illa om oss, men vi tala goda ord. Vi hava blivit såsom världens avskum, såsom var mans avskrap, och vi äro så ännu alltjämt.
\par 14 Detta skriver jag, icke för att komma eder att blygas, utan såsom en förmaning till mina älskade barn.
\par 15 Ty om I än haden tio tusen uppfostrare i Kristus, så haven I dock icke många fäder; det var ju jag som i Kristus Jesus genom evangelium födde eder till liv.
\par 16 Därför förmanar jag eder: Bliven mina efterföljare.
\par 17 Just för denna saks skull sänder jag nu till eder Timoteus, min älskade och trogne son i Herren; han skall påminna eder om huru jag går till väga i Kristus, i enlighet med den lära jag förkunnar allestädes, i alla församlingar.
\par 18 Nu är det väl så, att somliga hava blivit uppblåsta, under förmenande att jag icke skulle komma till eder.
\par 19 Men om Herren så vill, skall jag snart komma till eder; och då skall jag lära känna, icke dessa uppblåsta människors ord, utan deras kraft.
\par 20 Ty Guds rike består icke i ord, utan i kraft.
\par 21 Vilketdera viljen I nu: skall jag komma till eder med ris eller i kärlek och saktmods ande?

\chapter{5}

\par 1 Det förljudes såväl att överhuvud otukt bedrives bland eder, som ock att sådan otukt förekommer, som man icke ens finner bland hedningarna, nämligen att en son har sin faders hustru.
\par 2 Och ändå ären I uppblåsta och haven icke fastmer blivit uppfyllda av sådan sorg, att I haven drivit ut ur eder krets den som har gjort detta.
\par 3 Jag, som väl till kroppen är frånvarande, men till anden närvarande, har för min del redan, såsom vore jag närvarande, fällt domen över den som har förövat en sådan ogärning:
\par 4 i Herren Jesu namn skola vi komma tillsammans, I och min ande, med vår Herre Jesu kraft,
\par 5 och överlämna den mannen åt Satan till köttets fördärv, för att anden skall bliva frälst på Herren Jesu dag.
\par 6 Det är icke väl beställt med eder berömmelse. Veten I icke att litet surdeg syrar hela degen?
\par 7 Rensen bort den gamla surdegen, så att I bliven en ny deg. I ären ju osyrade; ty vi hava ock ett påskalamm, som är slaktat, nämligen Kristus.
\par 8 Låtom oss därför hålla högtid, icke med gammal surdeg, icke med elakhetens och ondskans surdeg, utan med renhetens och sanningens osyrade bröd.
\par 9 Jag skrev till eder i mitt brev att I icke skullen hava något umgänge med otuktiga människor -
\par 10 detta icke sagt i allmänhet, om alla denna världens otuktiga människor eller om giriga och roffare eller om avgudadyrkare; annars måsten I ju rymma ur världen.
\par 11 Nej, då jag skrev så till eder, menade jag, att om någon som kallades broder vore en otuktig människa eller en girig eller en avgudadyrkare eller en smädare eller en drinkare eller en roffare, så skullen I icke hava något umgänge med en sådan eller äta tillsammans med honom.
\par 12 Ty icke tillkommer det väl mig att döma dem som äro utanför? Dem som äro innanför haven I ju att döma;
\par 13 dem som äro utanför skall Gud döma. "I skolen driva ut ifrån eder den som är ond."

\chapter{6}

\par 1 Huru kan någon av eder taga sig för, att när han har sak med en annan, gå till rätta icke inför de heliga, utan inför de orättfärdiga?
\par 2 Veten I då icke att de heliga skola döma världen? Men om nu I skolen sitta till doms över världen, ären I då icke goda nog att döma i helt ringa mål?
\par 3 I veten ju att vi skola döma änglar; huru mycket mer böra vi icke då kunna döma i timliga ting?
\par 4 Och likväl, när I nu haven före något mål som gäller sådana ting, sätten I till domare just dem som äro ringa aktade i församlingen!
\par 5 Eder till blygd säger jag detta. Är det då så omöjligt att bland eder finna någon vis man, som kan bliva skiljedomare mellan sina bröder?
\par 6 Måste i stället den ene brodern gå till rätta med den andre, och det inför de otrogna?
\par 7 Överhuvud är redan det en brist hos eder, att I gån till rätta med varandra. Varför liden I icke hellre orätt? Varför låten I icke hellre andra göra eder skada?
\par 8 I stället gören I nu själva orätt och skada, och detta mot bröder.
\par 9 Veten I då icke att de orättfärdiga icke skola få Guds rike till arvedel? Faren icke vilse. Varken otuktiga människor eller avgudadyrkare eller äktenskapsbrytare, varken de som låta bruka sig till synd mot naturen eller de som själva öva sådan synd,
\par 10 varken tjuvar eller giriga eller drinkare eller smädare eller roffare skola få Guds rike till arvedel.
\par 11 Sådana voro ock somliga bland eder, men I haven låtit två eder rena, I haven blivit helgade, I haven blivit rättfärdiggjorda i Herrens, Jesu Kristi, namn och i vår Guds Ande.
\par 12 "Allt är mig lovligt"; ja, men icke allt är nyttigt. "Allt är mig lovligt"; ja, men jag bör icke låta något få makt över mig.
\par 13 Maten är för buken och buken för maten, men bådadera skall Gud göra till intet. Däremot är kroppen icke för otukt, utan för Herren, och Herren för kroppen;
\par 14 och Gud, som har uppväckt Herren, skall ock genom sin kraft uppväcka oss.
\par 15 Veten I icke att edra kroppar äro Kristi lemmar? Skall jag nu taga Kristi lemmar och göra dem till en skökas lemmar? Bort det!
\par 16 Veten I då icke att den som håller sig till en sköka, han bliver en kropp med henne? Det heter ju: "De tu skola varda ett kött."
\par 17 Men den som håller sig till Herren, han är en ande med honom.
\par 18 Flyn otukten. All annan synd som en människa kan begå är utom kroppen; men den som bedriver otukt, han syndar på sin egen kropp.
\par 19 Veten I då icke att eder kropp är ett tempel åt den helige Ande, som bor i eder, och som I haven undfått av Gud, och att I icke ären edra egna?
\par 20 I ären ju köpta, och betalning är given. Så förhärligen då Gud i eder kropp.

\chapter{7}

\par 1 Vad nu angår det I haven skrivit om, så svarar jag detta: En man gör visserligen väl i att icke komma vid någon kvinna;
\par 2 men för att undgå otuktssynder må var man hava sin egen hustru, och var kvinna sin egen man.
\par 3 Mannen give sin hustru vad han är henne pliktig, sammalunda ock hustrun sin man.
\par 4 Hustrun råder icke själv över sin kropp, utan mannen; sammalunda råder ej heller mannen över sin kropp, utan hustrun.
\par 5 Dragen eder icke undan från varandra, om icke möjligen, med bådas samtycke, till en tid, för att I skolen hava ledighet till bönen. Kommen sedan åter tillsammans, så att Satan icke frestar eder, då I nu icke kunnen leva återhållsamt.
\par 6 Detta säger jag likväl såsom en tillstädjelse, icke såsom en befallning.
\par 7 Jag skulle dock vilja att alla människor vore såsom jag. Men var och en har fått sin särskilda nådegåva från Gud, den ene så, den andre så.
\par 8 Till de ogifta åter och till änkorna säger jag att de göra väl, om de förbliva i samma ställning som jag.
\par 9 Men kunna de icke leva återhållsamt, så må de gifta sig; ty det är bättre att gifta sig än att vara upptänd av begär.
\par 10 Men dem som äro gifta bjuder jag - dock icke jag, utan Herren; En hustru må icke skilja sig från sin man
\par 11 (om hon likväl skulle skilja sig, så förblive hon ogift eller förlike sig åter med mannen), ej heller må en man förskjuta sin hustru.
\par 12 Till de andra åter säger jag själv, icke Herren: Om någon som hör till bröderna har en hustru som icke är troende, och denna är villig att leva tillsammans med honom, så må han icke förskjuta henne.
\par 13 Likaså, om en hustru har en man som icke är troende, och denne är villig att leva tillsammans med henne, så må hon icke förskjuta mannen.
\par 14 Ty den icke troende mannen är helgad i och genom sin hustru, och den icke troende hustrun är helgad i och genom sin man, då han är en broder; annars vore ju edra barn orena, men nu äro de heliga. -
\par 15 Om däremot den icke troende vill skiljas, så må han få skiljas. En broder eller syster är i sådana fall intet tvång underkastad, och Gud har kallat oss till att leva i frid.
\par 16 Ty huru kan du veta, du hustru, om du skall frälsa din man? Eller du man, huru vet du om du skall frälsa din hustru?
\par 17 Må allenast var och en vandra den väg fram, som Herren har bestämt åt honom, var och en i den ställning vari Gud har kallat honom. Den ordningen stadgar jag för alla församlingar.
\par 18 Har någon blivit kallad såsom omskuren, så göre han sig icke åter lik de oomskurna; har någon blivit kallad såsom oomskuren, så låte han icke omskära sig.
\par 19 Det kommer icke an på om någon är omskuren eller oomskuren; allt beror på huruvida han håller Guds bud.
\par 20 Var och en förblive i den kallelse vari han var, när han blev kallad.
\par 21 Har du blivit kallad såsom träl, så låt detta icke gå dig till sinnes; dock, om du kan bliva fri, så begagna dig hellre därav.
\par 22 Ty den träl som har blivit kallad till att vara i Herren, han är en Herrens frigivne; sammalunda är ock den frie, som har blivit kallad, en Kristi livegne.
\par 23 I ären köpta, och betalningen är given; bliven icke människors trälar.
\par 24 Ja, mina bröder, var och en förblive inför Gud i den ställning vari han har blivit kallad.
\par 25 Vad vidare angår dem som äro jungfrur, så har jag icke att åberopa någon befallning av Herren, utan giver allenast ett råd, såsom en som genom Herrens barmhärtighet har blivit förtroende värd.
\par 26 Jag menar alltså, med tanke på den nöd som står för dörren, att den människa gör väl, som förbliver såsom hon är.
\par 27 Är du bunden vid hustru, så sök icke att bliva lös. Är du utan hustru, så sök icke att få hustru.
\par 28 Om du likväl skulle gifta dig, så syndar du icke därmed; ej heller syndar en jungfru, om hon gifter sig. Dock komma de som så göra att draga över sig lekamliga vedermödor; och jag skulle gärna vilja skona eder.
\par 29 Men det säger jag, mina bröder: Tiden är kort; därför må härefter de som hava hustrur vara såsom hade de inga,
\par 30 och de som gråta såsom gräte de icke, och de som glädja sig såsom gladde de sig icke, och de som köpa något såsom finge de icke behålla det,
\par 31 och de som bruka denna världen såsom gjorde de icke något bruk av den. Ty den nuvarande världsordningen går mot sitt slut;
\par 32 och jag skulle gärna vilja att I voren fria ifrån omsorger. Den man som icke är gift ägnar nämligen sin omsorg åt vad som hör Herren till, huru han skall behaga Herren;
\par 33 men den gifte mannen ägnar sin omsorg åt vad som hör världen till, huru han skall behaga sin hustru,
\par 34 och så är hans hjärta delat. Likaså ägnar den kvinna, som icke längre är gift eller som är jungfru, sin omsorg åt vad som hör Herren till, att hon må vara helig till både kropp och ande; men den gifta kvinnan ägnar sin omsorg åt vad som hör världen till, huru hon skall behaga sin man.
\par 35 Detta säger jag till eder egen nytta, och icke för att lägga något band på eder, utan för att I skolen föra en hövisk vandel och stadigt förbliva vid Herren.
\par 36 Men om någon menar sig handla otillbörligt mot sin ogifta dotter därmed att hon får bliva överårig, då må han göra såsom han vill, om det nu måste så vara; han begår därmed ingen synd. Må hon få gifta sig.
\par 37 Om däremot någon är fast i sitt sinne och icke bindes av något nödtvång, utan kan följa sin egen vilja, och så i sitt sinne är besluten att låta sin ogifta dotter förbliva såsom hon är, då gör denne väl.
\par 38 Alltså: den som gifter bort sin dotter, han gör väl; och den som icke gifter bort henne, han gör ännu bättre.
\par 39 En hustru är bunden så länge hennes man lever; men när hennes man är avsomnad, står det henne fritt att gifta sig med vem hon vill, blott det sker i Herren.
\par 40 Men lyckligare är hon, om hon förbliver såsom hon är. Så är min mening, och jag tror att också jag har Guds Ande.

\chapter{8}

\par 1 Vad åter angår kött från avgudaoffer, så känna vi nog det talet: "Alla hava vi 'kunskap'." "Kunskapen" uppblåser, men kärleken uppbygger.
\par 2 Om någon menar sig hava fått någon "kunskap", så har han ännu icke fått kunskap på sådant sätt som han borde hava.
\par 3 Men den som älskar Gud, han är känd av honom.
\par 4 Vad alltså angår ätandet av kött från avgudaoffer, så säger jag detta: Vi veta visserligen att ingen avgud finnes till i världen, och att det icke finnes mer än en enda Gud.
\par 5 Ty om ock några så kallade gudar skulle finnas, vare sig i himmelen eller på jorden - och det finnes ju många "gudar" och många "herrar" -
\par 6 så finnes dock för oss allenast en enda Gud: Fadern, av vilken allt är, och till vilken vi själva äro, och en enda Herre: Jesus Kristus, genom vilken allt är, och genom vilken vi själva äro.
\par 7 Dock, icke alla hava denna kunskap, utan somliga, som äro vana att ännu alltjämt tänka på avguden, äta köttet såsom avgudaofferskött. Och eftersom deras samvete är svagt, bliver det härigenom befläckat.
\par 8 Men maten skall icke avgöra vår ställning till Gud. Avhålla vi oss från att äta, så bliva vi icke därigenom sämre; äta vi, så bliva vi icke därigenom bättre.
\par 9 Sen likväl till, att denna eder frihet icke till äventyrs bliver en stötesten för de svaga.
\par 10 Ty om någon får se dig, som har undfått "kunskap", ligga till bords i ett avgudahus, skall då icke hans samvete, om han är svag, därav "bliva uppbyggt" på det sätt att han äter köttet från avgudaoffer?
\par 11 Genom din "kunskap" går ju då den svage förlorad - han, din broder, som Kristus har lidit döden för.
\par 12 Om I på sådant sätt synden mot bröderna och såren deras svaga samveten, då synden I mot Kristus själv.
\par 13 Därför, om maten kan bliva min broder till fall, så vill jag sannerligen hellre för alltid avstå från att äta kött, på det att jag icke må bliva min broder till fall.

\chapter{9}

\par 1 Är jag icke fri? Är jag icke en apostel? Har jag icke sett Jesus, vår Herre? Ären icke I mitt verk i Herren?
\par 2 Om jag icke för andra är en apostel, så är jag det åtminstone för eder, ty I själva ären i Herren inseglet på mitt apostlaämbete.
\par 3 Detta är mitt försvar mot dem som sätta sig till doms över mig.
\par 4 Skulle vi kanhända icke hava rätt att få mat och dryck?
\par 5 Skulle vi icke hava rätt att få såsom hustru föra med oss på våra resor någon som är en syster, vi likaväl som de andra apostlarna och Herrens bröder och särskilt Cefas?
\par 6 Eller äro jag och Barnabas de enda som icke hava rätt att vara fritagna ifrån kroppsarbete?
\par 7 Vem tjänar någonsin i krig på egen sold? Vem planterar en vingård och äter icke dess frukt? Eller vem vaktar en hjord och förtär icke mjölk från hjorden?
\par 8 Icke talar jag väl detta därför att människor pläga så tala? Säger icke själva lagen detsamma?
\par 9 I Moses' lag är ju skrivet: "Du skall icke binda munnen till på oxen som tröskar." Månne det är om oxarna som Gud har sådan omsorg?
\par 10 Eller säger han det icke i alla händelser med tanke på oss? Jo, för vår skull blev det skrivet, att den som plöjer bör plöja med en förhoppning, och att den som tröskar bör göra det i förhoppning om att få sin del.
\par 11 Om vi hava sått åt eder ett utsäde av andligt gott, är det då för mycket, om vi få inbärga från eder en skörd av lekamligt gott?
\par 12 Om andra hava en viss rättighet över eder, skulle då icke vi än mer hava det? Och likväl hava vi icke gjort bruk av den rättigheten, utan vi fördraga allt, för att icke lägga något hinder i vägen för Kristi evangelium.
\par 13 I veten ju att de som förrätta tjänsten i helgedomen få sin föda ifrån helgedomen, och att de som äro anställda vid altaret få sin del, när altaret får sin.
\par 14 Så har ock Herren förordnat att de som förkunna evangelium skola hava sitt uppehälle av evangelium.
\par 15 Men jag för min del har icke gjort bruk av någon sådan förmån. Detta skriver jag nu icke, för att jag själv skall få någon sådan; långt hellre ville jag dö. Nej, ingen skall göra min berömmelse om intet.
\par 16 Ty om jag förkunnar evangelium, så är detta ingen berömmelse för mig. Jag måste ju så göra; och ve mig, om jag icke förkunnade evangelium!
\par 17 Gör jag det av egen drift, så har jag rätt till lön; men då jag nu icke gör det av egen drift, så är den syssla som jag är betrodd med allenast en livegen förvaltares. -
\par 18 Vilken är alltså min lön? Jo, just den, att när jag förkunnar evangelium, så gör jag detta utan kostnad för någon, i det att jag avstår från att göra bruk av den rättighet jag har såsom förkunnare av evangelium.
\par 19 Ty fastän jag är fri och oberoende av alla, har jag dock gjort mig till allas tjänare, för att jag skall vinna dess flera.
\par 20 För judarna har jag blivit såsom en jude, för att kunna vinna judar; för dom som stå under lagen har jag, som själv icke står under lagen, blivit såsom stode jag under lagen, för att kunna vinna dem som stå under lagen.
\par 21 För dem som äro utan lag har jag, som icke är utan Guds lag, men är i Kristi lag, blivit såsom vore jag utan lag, för att jag skall vinna dem som äro utan lag.
\par 22 För de svaga har jag blivit svag, för att kunna vinna de svaga; för alla har jag blivit allt, för att jag i alla händelser skall frälsa några.
\par 23 Men allt gör jag för evangelii skull, för att också jag skall bliva delaktig av dess goda.
\par 24 I veten ju, att fastän de som löpa på tävlingsbanan allasammans löpa, så vinner allenast en segerlönen. Löpen såsom denne, för att I mån vinna lönen.
\par 25 Men alla som vilja deltaga i en sådan tävlan pålägga sig återhållsamhet i alla stycken: dessa för att vinna en förgänglig segerkrans, men vi för att vinna en oförgänglig.
\par 26 Jag för min del löper alltså icke såsom gällde det ett ovisst mål; jag kämpar icke likasom en man som hugger i vädret.
\par 27 Fastmer tuktar jag min kropp och kuvar den, för att jag icke, när jag predikar för andra, själv skall komma till korta vid provet.

\chapter{10}

\par 1 Ty jag vill säga eder detta, mina bröder: Våra fäder voro alla under molnskyn och gingo alla genom havet;
\par 2 alla blevo de i molnskyn och i havet döpta till Moses;
\par 3 alla åto de samma andliga mat,
\par 4 och alla drucko de samma andliga dryck - de drucko nämligen ur en andlig klippa, som åtföljde dem, och den klippan var Kristus.
\par 5 Men de flesta av dem hade Gud icke behag till; de blevo ju nedgjorda i öknen.
\par 6 Detta skedde oss till en varnagel, för att vi icke skulle hava begärelse till det onda, såsom de hade begärelse därtill.
\par 7 Ej heller skolen I bliva avgudadyrkare, såsom somliga av dem blevo; så är ju skrivet: "Folket satte sig ned till att äta och dricka, och därpå stodo de upp till all leka."
\par 8 Låtom oss icke heller bedriva otukt, såsom somliga av dem gjorde, varför ock tjugutre tusen föllo på en enda dag.
\par 9 Låtom oss icke heller fresta Kristus, såsom somliga av dem gjorde, varför de ock blevo dödade av ormarna.
\par 10 Knorren icke heller, såsom somliga av dem gjorde, varför de ock blevo dödade av "Fördärvaren".
\par 11 Men detta vederfors dem för att tjäna till en varnagel, och det blev upptecknat till lärdom för oss, som hava tidernas ände inpå oss.
\par 12 Därför, den som menar sig stå, han må se till, att han icke faller.
\par 13 Inga andra frestelser hava mött eder än sådana som vanligen möta människor. Och Gud är trofast; han skall icke tillstädja att I bliven frestade över eder förmåga, utan när han låter frestelsen komma, skall han ock bereda en utväg därur, så att I kunnen härda ut i den.
\par 14 Alltså, mina älskade, undflyn avgudadyrkan.
\par 15 Jag säger detta till eder såsom till förståndiga människor; själva mån I döma om det som jag säger.
\par 16 Välsignelsens kalk, över vilken vi uttala välsignelsen, är icke den en delaktighet av Kristi blod? Brödet, som vi bryta, är icke det en delaktighet av Kristi kropp?
\par 17 Eftersom det är ett enda bröd, så äro vi, fastän många, en enda kropp, ty alla få vi vår del av detta ena bröd.
\par 18 Sen på det lekamliga Israel: äro icke de som äta av offren delaktiga i altaret?
\par 19 Vad vill jag då säga härmed? Månne att avgudaofferskött är någonting, eller att en avgud är någonting?
\par 20 Nej, det vill jag säga, att vad hedningarna offra, det offra de åt onda andar och icke åt Gud; och jag vill icke att I skolen hava någon gemenskap med de onda andarna.
\par 21 I kunnen icke dricka Herrens kalk och tillika onda andars kalk; I kunnen icke hava del i Herrens bord och tillika i onda andars bord.
\par 22 Eller vilja vi reta Herren? Äro då vi starkare än han?
\par 23 "Allt är lovligt"; ja, men icke allt är nyttigt. "Allt är lovligt"; ja, men icke allt uppbygger.
\par 24 Ingen söke sitt eget bästa, utan envar den andres.
\par 25 Allt som säljes i köttboden mån I äta; I behöven icke för samvetets skull göra någon undersökning därom.
\par 26 Ty "jorden är Herrens, och allt vad därpå är".
\par 27 Om någon av dem som icke äro troende bjuder eder till sig och I viljen gå till honom, så mån I äta av allt som sättes fram åt eder; I behöven icke för samvetets skull göra någon undersökning därom.
\par 28 Men om någon då säger till eder: "Detta är offerkött", så skolen I avhålla eder från att äta, för den mans skull, som gav saken till känna, och för samvetets skull -
\par 29 jag menar icke ditt eget samvete, utan den andres; ty varför skulle jag låta min frihet dömas av en annans samvete?
\par 30 Om jag äter därav med tacksägelse, varför skulle jag då bliva smädad för det som jag tackar Gud för?
\par 31 Alltså, vare sig I äten eller dricken, eller vadhelst annat I gören, så gören allt till Guds ära.
\par 32 Bliven icke för någon till en stötesten, varken för judar eller för greker eller för Guds församling;
\par 33 varen såsom jag, som i alla stycken fogar mig efter alla och icke söker min egen nytta, utan de mångas, för att de skola bliva frälsta.

\chapter{11}

\par 1 Varen I mina efterföljare, såsom jag är Kristi.
\par 2 Jag prisar eder för det att I i alla stycken haven mig i minne och hållen fast vid mina lärdomar, såsom de äro eder givna av mig.
\par 3 Men jag vill att I skolen inse detta, att Kristus är envar mans huvud, och att mannen är kvinnans huvud, och att Gud är Kristi huvud.
\par 4 Var och en man som har sitt huvud betäckt, när han beder eller profeterar, han vanärar sitt huvud.
\par 5 Men var kvinna som beder eller profeterar med ohöljt huvud, hon vanärar sitt huvud, ty det är då alldeles som om hon hade sitt hår avrakat.
\par 6 Om en kvinna icke vill hölja sig, så kan hon lika väl låta skära av sitt hår; men eftersom det är en skam för en kvinna att låta skära av sitt hår eller att låta raka av det, så må hon hölja sig.
\par 7 En man är icke pliktig att hölja sitt huvud, eftersom han är Guds avbild och återspeglar hans härlighet, då kvinnan däremot återspeglar mannens härlighet.
\par 8 Ty mannen är icke av kvinnan, utan kvinnan av mannen.
\par 9 Icke heller skapades mannen för kvinnans skull, utan kvinnan för mannens skull.
\par 10 Därför bör kvinnan på sitt huvud hava en "makt", för änglarnas skull.
\par 11 Dock är det i Herren så, att varken kvinnan är till utan mannen, eller mannen utan kvinnan.
\par 12 Ty såsom kvinnan är av mannen, så är ock mannen genom kvinnan; men alltsammans är av Gud. -
\par 13 Dömen själva: höves det en kvinnan att ohöljd bedja till Gud?
\par 14 Lär icke själva naturen eder att det länder en man till vanheder, om han har långt hår,
\par 15 men att det länder en kvinna till ära, om hon har långt hår? Håret är ju henne givet såsom slöja.
\par 16 Om nu likväl någon vill vara genstridig, så mån han veta att vi för vår del icke hava en sådan sedvänja, ej heller andra Guds församlingar.
\par 17 Detta bjuder jag eder nu. Men vad jag icke kan prisa är att I kommen tillsammans, icke till förbättring, utan till försämring.
\par 18 Ty först och främst hör jag sägas att vid edra församlingsmöten söndringar yppa sig bland eder. Och till en del tror jag att så är.
\par 19 Ty partier måste ju finnas bland eder, för att det skall bliva uppenbart vilka bland eder som hålla provet.
\par 20 När I alltså kommen tillsammans med varandra, kan ingen Herrens måltid hållas;
\par 21 ty vid måltiden tager var och en i förväg själv den mat han har medfört, och så får den ene hungra, medan den andre får för mycket.
\par 22 Haven I då icke edra hem, där I kunnen äta ock dricka? Eller är det så, att I förakten Guds församling och viljen komma dem att blygas, som intet hava? Vad skall jag då säga till eder? Skall jag prisa eder? Nej, i detta stycke prisar jag eder icke.
\par 23 Ty jag har från Herren undfått detta, som jag ock har meddelat eder: I den natt, då Herren Jesus blev förrådd, tog han ett bröd
\par 24 och tackade Gud och bröt det och sade: "Tagen, äten. Detta är min lekamen, som varder utgiven för eder. Gören detta till min åminnelse."
\par 25 Sammalunda tog han ock kalken, efter måltiden, och sade: "Denna kalk är det nya förbundet, i mitt blod. Så ofta I dricken den, så gören detta till min åminnelse."
\par 26 Ty så ofta I äten detta bröd och dricken kalken, förkunnen I Herrens död, till dess att han kommer.
\par 27 Den som nu på ett ovärdigt sätt äter detta bröd eller dricker Herrens kalk, han försyndar sig på Herrens lekamen och blod.
\par 28 Pröve då människan sig själv, och äte så av brödet och dricke av kalken.
\par 29 Ty den som äter och dricker, utan att göra åtskillnad mellan Herrens lekamen och annan spis, han äter och dricker en dom över sig.
\par 30 Därför finnas ock bland eder många som äro svaga och sjuka, och ganska många äro avsomnade.
\par 31 Om vi ginge till doms med oss själva, så bleve vi icke dömda.
\par 32 Men då vi nu bliva dömda, så är detta en Herrens tuktan, som drabbar oss, för att vi icke skola bliva fördömda tillika med världen.
\par 33 Alltså, mina bröder, när I kommen tillsammans för att hålla måltid, så vänten på varandra.
\par 34 Om någon är hungrig, då må han äta hemma, så att eder sammankomst icke bliver eder till en dom. Om det övriga skall jag förordna, när jag kommer.

\chapter{12}

\par 1 Vad nu angår dem som hava andliga gåvor, så vill jag säga eder, mina bröder, huru med dem förhåller sig.
\par 2 I veten att I, medan I voren hedningar, läten eder blindvis föras bort till de stumma avgudarna.
\par 3 Därför vill jag nu förklara för eder, att likasom ingen som talar i Guds Ande säger: "Förbannad vare Jesus", så kan ej heller någon säga: "Jesus är Herre" annat än i den helige Ande.
\par 4 Nådegåvorna äro mångahanda, men Anden är en och densamme.
\par 5 Tjänsterna äro mångahanda, men Herren är en och densamme.
\par 6 Kraftverkningarna äro mångahanda, men Gud är en och densamme, han som verkar allt i alla.
\par 7 Men de gåvor i vilka Anden uppenbarar sig givas åt var och en så, att de kunna bliva till nytta.
\par 8 Så gives genom Anden åt den ene att tala visdomens ord, åt en annan att efter samme Ande tala kunskapens ord,
\par 9 åt en annan gives tro i samme Ande, åt en annan givas helbrägdagörelsens gåvor i samme ene Ande,
\par 10 åt en annan gives gåvan att utföra kraftgärningar, åt en annan att profetera, åt en annan att skilja mellan andar, åt en annan att tala tungomål på olika sätt, åt en annan att uttyda, när någon talar tungomål.
\par 11 Men allt detta verkar densamme ene Anden, i det han, alltefter sin vilja, tilldelar åt var och en någon särskild gåva.
\par 12 Ty likasom kroppen är en och likväl har många lemmar, och likasom kroppens alla lemmar, fastän de äro många, likväl utgöra en enda kropp, likaså är det med Kristus.
\par 13 Ty i en och samme Ande äro vi alla döpta till att utgöra en och samma kropp, vare sig vi äro judar eller greker, vare sig vi äro trälar eller fria; och alla hava vi fått en och samme Ande utgjuten över oss.
\par 14 Kroppen utgöres ju icke heller av en enda lem, utan av många.
\par 15 Om foten ville säga: "Jag är icke hand, därför hör jag icke till kroppen", så skulle den icke dess mindre höra till kroppen.
\par 16 Och om örat ville säga: "Jag är icke öga, därför hör jag icke till kroppen", så skulle det icke dess mindre höra till kroppen.
\par 17 Om hela kroppen vore öga, var funnes då hörseln? Och om den hel och hållen vore öra, var funnes då lukten?
\par 18 Men nu har Gud insatt lemmarna i kroppen, var och en av dem på det sätt som han har velat.
\par 19 Om åter allasammans utgjorde en enda lem, var funnes då själva kroppen?
\par 20 Men nu är det så, att lemmarna äro många, och att kroppen dock är en enda.
\par 21 Ögat kan icke säga till handen: "Jag behöver dig icke", ej heller huvudet till fötterna: "Jag behöver eder icke."
\par 22 Nej, just de kroppens lemmar som tyckas vara svagast äro som mest nödvändiga.
\par 23 Och de delar av kroppen, som tyckas oss vara mindre hedersamma, dem bekläda vi med så mycket större heder; och dem som vi blygas för, dem skyla vi med så mycket större blygsamhet,
\par 24 under det att de andra icke behöva något sådant. Men när Gud sammanfogade kroppen av olika delar och därvid lät den ringare delen få en så mycket större heder,
\par 25 så skedde detta, för att söndring icke skulle uppstå i kroppen, utan alla lemmar endräktigt hava omsorg om varandra.
\par 26 Om nu en lem lider, så lida alla de andra lemmarna med den; om åter en lem äras, så glädja sig alla de andra lemmarna med den.
\par 27 Men nu ären I Kristi kropp och hans lemmar, var och en i sin mån.
\par 28 Och Gud har i församlingen satt först och främst några till apostlar, för det andra några till profeter, för det tredje några till lärare, vidare några till att utföra kraftgärningar, ytterligare några till att hava helbrägdagörelsens gåvor, eller till att taga sig an de hjälplösa, eller till att vara styresmän, eller till att på olika sätt tala tungomål.
\par 29 Icke äro väl alla apostlar? Icke äro väl alla profeter? Icke äro väl alla lärare? Icke utföra väl alla kraftgärningar?
\par 30 Icke hava väl alla helbrägdagörelsens gåvor? Icke tala väl alla tungomål? Icke kunna väl alla uttyda?
\par 31 Men varen ivriga att undfå de nådegåvor som äro de största. Och nu vill jag ytterligare visa eder en väg, en övermåttan härlig väg.

\chapter{13}

\par 1 Om jag talade både människors och änglars tungomål, men icke hade kärlek, så vore jag allenast en ljudande malm eller en klingande cymbal.
\par 2 Och om jag hade profetians gåva och visste alla hemligheter och ägde all kunskap, och om jag hade all tro, så att jag kunde förflytta berg, men icke hade kärlek, så vore jag intet.
\par 3 Och om jag gåve bort allt vad jag ägde till bröd åt de fattiga, ja, om jag offrade min kropp till att brännas upp, men icke hade kärlek, så vore detta mig till intet gagn.
\par 4 Kärleken är tålig och mild. Kärleken avundas icke, kärleken förhäver sig icke, den uppblåses icke.
\par 5 Den skickar sig icke ohöviskt, den söker icke sitt, den förtörnas icke, den hyser icke agg för en oförrätts skull.
\par 6 Den gläder sig icke över orättfärdigheten, men har sin glädje i sanningen.
\par 7 Den fördrager allting, den tror allting, den hoppas allting, den uthärdar allting.
\par 8 Kärleken förgår aldrig. Men profetians gåva, den skall försvinna, och tungomålstalandet, det skall taga slut, och kunskapen, den skall försvinna.
\par 9 Ty vår kunskap är ett styckverk, och vårt profeterande är ett styckverk;
\par 10 men när det kommer, som är fullkomligt, då skall det försvinna, som är ett styckverk.
\par 11 När jag var barn, talade jag såsom ett barn, mitt sinne var såsom ett barns, jag hade barnsliga tankar; men sedan jag blev man, har jag lagt bort vad barnsligt var.
\par 12 Nu se vi ju på ett dunkelt sätt, såsom i en spegel, men då skola vi se ansikte mot ansikte. Nu är min kunskap ett styckverk, men då skall jag känna till fullo, såsom jag själv har blivit till fullo känd.
\par 13 Så bliva de då beståndande, tron, hoppet, kärleken, dessa tre; men störst bland dem är kärleken.

\chapter{14}

\par 1 Faren efter kärleken, men varen ock ivriga att undfå de andliga gåvorna, framför allt profetians gåva.
\par 2 Ty den som talar tungomål, han talar icke för människor, utan för Gud; ingen förstår honom ju, han talar i andehänryckning hemlighetsfulla ord.
\par 3 Men den som profeterar, han talar för människor, dem till uppbyggelse och förmaning och tröst.
\par 4 Den som talar tungomål uppbygger allenast sig själv, men den som profeterar, han uppbygger en hel församling.
\par 5 Jag skulle väl vilja att I alla taladen tungomål, men ännu hellre ville jag att I profeteraden. Den som profeterar är förmer än den som talar tungomål, om nämligen den senare icke därjämte uttyder sitt tal, så att församlingen får någon uppbyggelse.
\par 6 Ja, mina bröder, om jag komme till eder och talade tungomål, vad gagn gjorde jag eder därmed, såframt jag icke därjämte genom mitt tal meddelade eder antingen någon uppenbarelse eller någon kunskap eller någon profetia eller någon undervisning?
\par 7 Gäller det icke jämväl om livlösa ting som giva ljud ifrån sig, det må nu vara en flöjt eller en harpa, att vad som spelas på dem icke kan uppfattas, om de icke giva ifrån sig toner som kunna skiljas från varandra?
\par 8 Likaså, om den signal som basunen giver är otydlig, vem gör sig då redo till strid?
\par 9 Detsamma gäller nu för eder; om I icke med edra tungor frambringen begripliga ord, huru skall man då kunna förstå vad I talen? Då bliver det ju ett tal i vädret.
\par 10 Det finnes här i världen olika språk, vem vet huru många, och bland dem finnes intet vars ljud äro utan mening.
\par 11 Men om jag nu icke förstår språket, så bliver jag en främling för den som talar, och den som talar bliver en främling för mig.
\par 12 Detta gäller ock för eder; när I ären ivriga att undfå andliga gåvor, så må eder strävan efter att dessa hos eder skola överflöda hava församlingens uppbyggelse till mål.
\par 13 Därför må den som talar tungomål bedja om att han ock må kunna uttyda.
\par 14 Ty om jag talar tungomål, när jag beder, så beder visserligen min ande, men mitt förstånd kommer ingen frukt åstad.
\par 15 Vad följer då härav? Jo, jag skall väl bedja med anden, men jag skall ock bedja med förståndet; jag skall väl lovsjunga med anden, men jag skall ock lovsjunga med förståndet.
\par 16 Eljest, om du lovar Gud med anden, huru skola de som sitta på de olärdas plats då kunna säga sitt "amen" till din tacksägelse? De förstå ju icke vad du säger.
\par 17 Om än din tacksägelse är god, så bliva de andra dock icke uppbyggda därav. -
\par 18 Gud vare tack, jag talar tungomål mer än I alla;
\par 19 och dock vill jag hellre i församlingen tala fem ord med mitt förstånd, till undervisning jämväl för andra, än tio tusen ord i tungomål.
\par 20 Mina bröder, varen icke barn till förståndet; nej varen barn i ondskan, men varen fullmogna till förståndet.
\par 21 Det är skrivet i lagen: "Genom människor med främmande tungomål och genom främlingars läppar skall jag tala till detta folk, men icke ens så skola de höra på mig, säger Herren."
\par 22 Alltså äro "tungomålen" ett tecken, ej för dem som tro, utan för dem som icke tro; profetian däremot är ett tecken, ej för dem som icke tro, utan för dem som tro.
\par 23 Om nu hela församlingen komme tillhopa till gemensamt möte, och alla där talade tungomål, och så några som vore olärda komme ditin, eller några som icke trodde, skulle då icke dessa säga att I voren ifrån edra sinnen?
\par 24 Om åter alla profeterade, och så någon som icke trodde, eller som vore olärd komme ditin, då skulle denne känna sig avslöjad av alla och av alla utrannsakad.
\par 25 Vad som vore fördolt i hans hjärta bleve då uppenbart, och så skulle han falla ned på sitt ansikte och tillbedja Gud och betyga att "Gud verkligen är i eder".
\par 26 Vad följer då härav, mina bröder? Jo, när I kommen tillsammans, så har var och en något särskilt att meddela: den ene har en psalm, den andre något till undervisning, en annan åter någon uppenbarelse, en talar tungomål, en annan uttyder; allt detta må nu ske så, att det länder till uppbyggelse.
\par 27 Vill man tala tungomål, så må för var gång två eller högst tre få tala, och av dessa en i sänder, och en må uttyda det.
\par 28 Är ingen uttydare tillstädes, så må de tiga i församlingen och tala allenast för sig själva och för Gud.
\par 29 Av dem som vilja profetera må två eller tre få tala, och de andra må döma om det som talas.
\par 30 Men om någon annan som sitter där får en uppenbarelse, då må den förste tiga.
\par 31 Ty I kunnen alla få profetera, den ene efter den andre, så att alla bliva undervisade och alla förmanade;
\par 32 och profeters andar äro profeterna underdåniga.
\par 33 Gud är ju icke oordningens Gud, utan fridens.
\par 34 Såsom kvinnorna tiga i alla andra de heligas församlingar, så må de ock tiga i edra församlingar. Det är dem icke tillstatt att tala, utan de böra underordna sig, såsom lagen bjuder.
\par 35 Vilja de hava upplysning om något, så må de hemma fråga sina män; ty det är en skam för en kvinna att tala i församlingen. -
\par 36 Eller är det från eder som Guds ord har utgått? Eller har det kommit allenast till eder?
\par 37 Om någon menar sig vara en profet eller en man med andegåva, så må han ock inse att vad jag skriver till eder är Herrens bud.
\par 38 Men vill någon icke inse detta, så vare det hans egen sak.
\par 39 Alltså, mina bröder, varen ivriga att undfå profetians gåva och förmenen ej heller någon att tala tungomål.
\par 40 Men låten allt tillgå på höviskt sätt och med ordning.

\chapter{15}

\par 1 Mina bröder, jag vill påminna eder om det evangelium som jag förkunnade för eder, som I jämväl togen emot, och som I ännu stån kvar i,
\par 2 genom vilket I ock bliven frälsta; jag vill påminna eder om huru jag förkunnade det för eder, såframt I eljest hållen fast därvid - om nu icke så är att I förgäves haven kommit till tro.
\par 3 Jag meddelade eder ju såsom ett huvudstycke vad jag själv hade undfått: att Kristus dog för våra synder, enligt skrifterna,
\par 4 och att han blev begraven, och att han har uppstått på tredje dagen, enligt skrifterna,
\par 5 och att han visade sig för Cefas och sedan för de tolv.
\par 6 Därefter visade han sig för mer än fem hundra bröder på en gång, av vilka de flesta ännu leva kvar, medan några äro avsomnade.
\par 7 Därefter visade han sig för Jakob och sedan för alla apostlarna.
\par 8 Allra sist visade han sig också för mig, som är att likna vid ett ofullgånget foster.
\par 9 Ty jag är den ringaste bland apostlarna, ja, icke ens värdig att kallas apostel, jag som har förföljt Guds församling.
\par 10 Men genom Guds nåd är jag vad jag är, och hans nåd mot mig har icke varit fåfäng, utan jag har arbetat mer än de alla - dock icke jag, utan Guds nåd, som har varit med mig.
\par 11 Det må nu vara jag eller de andra, så är det på det sättet vi predika, och på det sättet I haven kommit till tro.
\par 12 Om det nu predikas om Kristus att han har uppstått från de döda, huru kunna då somliga bland eder säga att det icke finnes någon uppståndelse från de döda?
\par 13 Om det åter icke finnes någon uppståndelse från de döda, då har icke heller Kristus uppstått.
\par 14 Men om Kristus icke har uppstått, då är ju vår predikan fåfäng, då är ock eder tro fåfäng;
\par 15 då befinnas vi ock vara falska Guds vittnen, eftersom vi hava vittnat mot Gud att han har uppväckt Kristus, som han icke har uppväckt, om det är sant att döda icke uppstå.
\par 16 Ja, om döda icke uppstå, så har ej heller Kristus uppstått.
\par 17 Men om Kristus icke har uppstått, så är eder tro förgäves; I ären då ännu kvar i edra synder.
\par 18 Då hava ju ock de gått förlorade, som hava avsomnat i Kristus.
\par 19 Om vi i detta livet hava i Kristus haft vårt hopp, och därav intet bliver, då äro vi de mest ömkansvärda av alla människor.
\par 20 Men nu har Kristus uppstått från de döda, såsom förstlingen av de avsomnade.
\par 21 Ty eftersom döden kom genom en människa, så kom ock genom en människa de dödas uppståndelse.
\par 22 Och såsom i Adam alla dö, så skola ock i Kristus alla göras levande.
\par 23 Men var och en i sin ordning: Kristus såsom förstlingen, därnäst, vid Kristi tillkommelse, de som höra honom till.
\par 24 Därefter kommer änden, då när han överlämnar riket åt Gud och Fadern, sedan han från andevärldens alla furstar och alla väldigheter och makter har tagit all deras makt.
\par 25 Ty han måste regera "till dess han har lagt alla sina fiender under sina fötter".
\par 26 Sist bland hans fiender bliver ock döden berövad all sin makt;
\par 27 ty "allt har han lagt under hans fötter". Men när det heter att "allt är honom underlagt", då är uppenbarligen den undantagen, som har lagt allt under honom.
\par 28 Och sedan allt har blivit Sonen underlagt, då skall ock Sonen själv giva sig under den som har lagt allt under honom. Och så skall Gud bliva allt i alla.
\par 29 Vad kunna annars de som låta döpa sig för de dödas skull vinna därmed? Om så är att döda alls icke uppstå, varför låter man då döpa sig för deras skull?
\par 30 Och varför undsätta vi oss själva var stund för faror?
\par 31 Ty - så sant jag i Kristus Jesus, vår Herre, kan berömma mig av eder, mina bröder - jag lider döden dag efter dag.
\par 32 Om jag hade tänkt såsom människor pläga tänka, när jag i Efesus kämpade mot vilddjuren, vad gagnade mig då det jag gjorde? Om döda icke uppstå - "låtom oss då äta och dricka, ty i morgon måste vi dö".
\par 33 Faren icke vilse: "För goda seder dåligt sällskap är fördärv."
\par 34 Vaknen upp till rätt nykterhet, och synden icke. Somliga finnas ju, som leva i okunnighet om Gud; eder till blygd säger jag detta.
\par 35 Nu torde någon fråga: "På vad sätt uppstå då de döda, och med hurudan kropp skola de träda fram?"
\par 36 Du oförståndige! Det frö du sår, det får ju icke liv, om det icke först har dött.
\par 37 Och när du sår, då är det du sår icke den växt som en gång skall komma upp, utan ett naket korn, kanhända ett vetekorn, kanhända något annat.
\par 38 Men Gud giver det en kropp, en sådan som han vill, och åt vart frö dess särskilda kropp.
\par 39 Icke allt kött är av samma slag, utan människors har sin art, boskapsdjurs kött en annan art, fåglars kött åter en annan, fiskars återigen en annan.
\par 40 Så finnas ock både himmelska kroppar och jordiska kroppar, men de himmelska kropparnas härlighet är av ett slag, de jordiska kropparnas av ett annat slag.
\par 41 En härlighet har solen, en annan härlighet har månen, åter en annan härlighet hava stjärnorna; ja, den ena stjärnan är icke lik den andra i härlighet. -
\par 42 Så är det ock med de dödas uppståndelse: vad som bliver sått förgängligt, det uppstår oförgängligt;
\par 43 vad som bliver sått i ringhet, det uppstår i härlighet; vad som bliver sått i svaghet, det uppstår i kraft;
\par 44 här sås en "själisk" kropp, där uppstår en andlig kropp. Så visst som det finnes en "själisk" kropp, så visst finnes det ock en andlig.
\par 45 Så är ock skrivet: "Den första människan, Adam, blev en levande varelse med själ." Den siste Adam åter blev en levandegörande ande.
\par 46 Men icke det andliga är det första, utan det "själiska"; sedan kommer det andliga.
\par 47 Den första människan var av jorden och jordisk, den andra människan är av himmelen.
\par 48 Sådan som den jordiska var, sådana äro ock de jordiska; och sådan som den himmelska är, sådana äro ock de himmelska.
\par 49 Och såsom vi hava burit den jordiskas gestalt, så skola vi ock bära den himmelskas gestalt.
\par 50 Mina bröder, vad jag nu vill säga är detta, att kött och blod icke kunna få Guds rike till arvedel; ej heller får förgängligheten oförgängligheten till arvedel.
\par 51 Se, jag säger eder en hemlighet: Vi skola icke alla avsomna, men alla skola vi bliva förvandlade,
\par 52 och det i ett nu, i ett ögonblick, vid den sista basunens ljud. Ty basunen skall ljuda, och de döda skola uppstå till oförgänglighet, och då skola vi bliva förvandlade.
\par 53 Ty detta förgängliga måste ikläda sig oförgänglighet, och detta dödliga ikläda sig odödlighet.
\par 54 Men när detta förgängliga har iklätt sig oförgänglighet, och detta dödliga har iklätt sig odödlighet, då skall det ord fullbordas, som står skrivet: "Döden är uppslukad och seger vunnen."
\par 55 Du död, var är din seger? Du död, var är din udd?
\par 56 Dödens udd är synden, och syndens makt kommer av lagen.
\par 57 Men Gud vare tack, som giver oss segern genom vår Herre Jesus Kristus!
\par 58 Alltså, mina älskade bröder, varen fasta, orubbliga, alltid överflödande i Herrens verk, eftersom I veten att edert arbete icke är fåfängt i Herren.

\chapter{16}

\par 1 Vad nu angår insamlingen till de heliga, så mån I förfara på samma sätt som jag har förordnat för församlingarna i Galatien.
\par 2 Var och en av eder må spara ihop vad han får tillfälle till, och på första dagen i var vecka må han lägga av detta hemma hos sig, så att insamlingen icke göres först vid min ankomst.
\par 3 Men när jag kommer, skall jag sända åstad de män som I själva pröven vara lämpliga, med brev till Jerusalem, för att där frambära eder kärleksgåva.
\par 4 Och om saken befinnes vara värd att också jag reser, så skola de få åtfölja mig.
\par 5 Jag tänker nämligen komma till eder, sedan jag har farit genom Macedonien. Ty Macedonien vill jag allenast fara igenom,
\par 6 men hos eder skall jag kanhända stanna något, möjligen vintern över, för att I därefter mån hjälpa mig till vägs, dit jag kan vilja begiva mig.
\par 7 Jag vill icke besöka eder nu strax, på genomresa, ty jag hoppas att någon tid få stanna hos eder, om Herren så tillstädjer.
\par 8 Men i Efesus vill jag stanna ända till pingst.
\par 9 Ty en dörr till stor och fruktbärande verksamhet har öppnats för mig; jag har ock många motståndare.
\par 10 Men när Timoteus kommer, så sen till, att han utan fruktan må kunna vistas hos eder. Han utför ju Herrens verk, han såväl som jag;
\par 11 må därför ingen förakta honom. Hjälpen honom sedan till vägs i frid, så att han kommer åter till mig; ty jag väntar honom med bröderna.
\par 12 Vad angår brodern Apollos, så har jag ivrigt uppmanat honom att med de andra bröderna begiva sig till eder. Han var dock alls icke hågad att komma just nu; men när det bliver honom lägligt, skall han komma.
\par 13 Vaken, stån fasta i tron, skicken eder såsom män, varen starka.
\par 14 Låten allt hos eder ske i kärlek.
\par 15 Mina bröder, jag vill giva eder en förmaning: I kännen ju Stefanas' husfolk och veten att de äro förstlingen i Akaja, och att de hava ägnat sig åt de heligas tjänst;
\par 16 därför mån I å eder sida underordna eder under dessa män och under envar som bistår dem i deras arbete och själv gör sig möda.
\par 17 Jag gläder mig över att Stefanas och Fortunatus och Akaikus hava kommit hit, ty dessa hava givit mig ersättning för vad jag har måst sakna genom att vara skild från eder;
\par 18 de hava vederkvickt min ande såväl som eder ande. Så lären eder nu att rätt uppskatta sådana män.
\par 19 Församlingarna i provinsen Asien hälsar eder. Akvila och Priska, tillika med den församling som kommer tillhopa i deras hus, hälsa eder mycket i Herren.
\par 20 Ja, alla bröderna hälsa eder. Hälsen varandra med en helig kyss.
\par 21 Här skriver jag, Paulus, min hälsning med egen hand.
\par 22 Om någon icke har Herren kär, så vare han förbannad. Marana, ta!
\par 23 Herren Jesu nåd vare med eder.
\par 24 Min kärlek är med eder alla, i Kristus Jesus.


\end{document}