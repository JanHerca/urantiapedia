\begin{document}

\title{1 Tessalonikerbrevet}


\chapter{1}

\par 1 Paulus och Silvanus och Timoteus hälsa tessalonikernas församling i Gud, Fadern, och Herren Jesus Kristus. Nåd och frid (av Gud, vår Fader, och Herren Jesu Kristus) vare med eder.
\par 2 Vi tacka Gud alltid för eder alla, när vi tänka på eder i våra böner.
\par 3 Ty oavlåtligen ihågkomma vi inför vår Gud och Fader edra gärningar i tron och edert arbete i kärleken och eder ståndaktighet i hoppet, i vår Herre Jesus Kristus.
\par 4 Vi veta ju, käre bröder, i Guds älskade, huru det var, när I bleven utvalda:
\par 5 vårt evangelium kom till eder icke med ord allenast, utan i kraft och helig ande och med full visshet. I veten ock på vad sätt vi uppträdde bland eder, till edert bästa.
\par 6 Och I å eder sida bleven våra efterföljare och därmed Herrens, i det att I, mitt under stort betryck, togen emot ordet med glädje i helig ande.
\par 7 Så bleven I själva ett föredöme för alla de troende i Macedonien och Akaja.
\par 8 Ty från eder har genljudet av Herrens ord gått vidare ut; icke allenast i Macedonien och Akaja, utan allestädes har eder tro på Gud blivit känd, så att vi för vår del icke behöva tala något därom.
\par 9 Ty själva förkunna de om oss, med vilken framgång vi begynte vårt arbete hos eder, och huru I från avgudarna omvänden eder till Gud, till att tjäna den levande och sanne Guden,
\par 10 och till att vänta hans Son från himmelen, honom som han har uppväckt från de döda, Jesus, som frälsar oss undan den kommande vredesdomen.

\chapter{2}

\par 1 I veten ju själva, käre bröder, att det icke var utan kraft vi begynte vårt arbete hos eder.
\par 2 Nej, fastän vi, såsom i veten, i Filippi förut hade fått utstå lidande och misshandling, hade vi dock frimodighet i vår Gud till att förkunna för eder Guds evangelium, under mycken kamp.
\par 3 Ty vad vi tala till tröst och förmaning, det har icke sin grund i villfarelse eller i orent uppsåt, ej heller sker det med svek;
\par 4 utan därför att vi av Gud hava prövats värdiga att få evangelium oss betrott, tala vi i enlighet därmed, icke för att vara människor till behag, utan för att vara Gud till behag, honom som prövar våra hjärtan.
\par 5 Aldrig någonsin hava vi uppträtt med smickrets ord, det veten I, ej heller så, att vi skulle få en förevändning att bereda oss vinning - Gud är vårt vittne.
\par 6 Ej heller hava vi sökt pris av människor, vare sig av eder eller av andra,
\par 7 fastän vi såsom Kristi apostlar väl hade kunnat uppträda med myndighet. Tvärtom hava vi visat oss milda bland eder, såsom när en moder omhuldar sina späda barn.
\par 8 I sådan ömhet om eder ville vi gärna icke allenast göra också eder delaktiga av Guds evangelium, utan till och med offra våra liv för eder, ty I haden blivit oss kära.
\par 9 I kommen ju ihåg, käre bröder, vårt arbete och vår möda, huru vi, under det att vi predikade för eder Guds evangelium, strävade natt och dag, för att icke bliva någon av eder till tunga.
\par 10 I själva ären våra vittnen, och Gud är vårt vittne, I veten, och han vet huru heligt och rättfärdigt och ostraffligt vi förhöllo oss mot eder, I som tron.
\par 11 Likaledes veten I huru vi förmanade och uppmuntrade var och en av eder, såsom en fader sina barn,
\par 12 och huru vi uppfordrade eder att föra en vandel som vore värdig Gud, honom som kallar eder till sitt rike och sin härlighet.
\par 13 Därför tacka vi ock oavlåtligen Gud för att I, när I undfingen det Guds ord som vi predikade, icke mottogen det såsom människoord, utan såsom Guds ord, vilket det förvisso är, ett ord som ock är verksamt i eder som tron.
\par 14 I, käre bröder, haven ju blivit efterföljare till de Guds församlingar i Kristus Jesus som äro i Judeen. Ty I haven av edra egna landsmän fått lida detsamma som de hava lidit av judarna -
\par 15 av dem som dödade både Herren Jesus och profeterna och förjagade oss, och som äro misshagliga för Gud och fiender till alla människor,
\par 16 i det att de söka hindra oss att tala till hedningarna, så att dessa kunna bliva frälsta. Så uppfylla de alltjämt sina synders mått. Dock, vredesdomen har kommit över dem i all sin stränghet.
\par 17 Men då vi nu hava måst vara skilda från eder, käre bröder - visserligen allenast för en kort tid och i utvärtes måtto, icke till hjärtat - hava vi blivit så mycket mer angelägna att få se edra ansikten och känt stor åstundan därefter.
\par 18 Ty vi hava varit redo att komma till eder - jag, Paulus, för min del både en och två gånger - men Satan har hindrat oss.
\par 19 Ty vem är vårt hopp och vår glädje och vår berömmelses krona inför vår Herre Jesus vid hans tillkommelse, vem, om icke just I?
\par 20 Ja, I ären vår ära och vår glädje.

\chapter{3}

\par 1 Därför, när vi icke mer kunde uthärda, beslöto vi att stanna ensamma kvar i Aten,
\par 2 och sände åstad Timoteus, vår broder och Guds tjänare vid förkunnandet av evangelium om Kristus, för att han skulle styrka och uppmuntra eder i eder tro,
\par 3 så att ingen bleve vacklande under dessa lidanden. Ty I veten själva att sådana äro oss förelagda.
\par 4 Redan när vi voro hos eder, sade vi ju eder förut att vi skulle komma att utstå lidanden. Så har nu ock skett, det veten I.
\par 5 Det var också därför som jag sände honom åstad, när jag icke mer kunde uthärda; ty jag ville veta något om eder tro, eftersom jag fruktade att frestaren till äventyrs hade så frestat eder, att vårt arbete skulle bliva utan frukt.
\par 6 Men nu, då Timoteus har kommit till oss från eder och förkunnat för oss det glada budskapet om eder tro och kärlek, och sagt oss att I alltjämt haven oss i god hågkomst, och att I längten efter att se oss, likasom vi längta efter eder,
\par 7 nu hava vi i fråga om eder, käre bröder, genom eder tro fått hugnad i all vår nöd och allt vårt lidande.
\par 8 Ty nu leva vi, eftersom I stån fasta i Herren.
\par 9 Ja, huru skola vi nog kunna tacka Gud för eder, till gengäld för all den glädje som vi genom eder hava inför vår Gud?
\par 10 Natt och dag är det vår innerligaste bön, att vi må få se edra ansikten och avhjälpa vad som kan brista i eder tro.
\par 11 Men vår Gud och Fader själv och vår Herre Jesus må för oss jämna vägen till eder.
\par 12 Och eder må Herren giva en allt större och mer överflödande kärlek till varandra, ja, till alla människor, en sådan kärlek som vi hava till eder,
\par 13 så att edra hjärtan styrkas till att vara ostraffliga i helighet inför vår Gud och Fader vid vår Herre Jesu tillkommelse, när han kommer med alla sina heliga.

\chapter{4}

\par 1 Ytterligare, käre bröder, bedja vi nu och förmana eder i Herren Jesus att allt mer förkovra eder i en sådan vandel som I haven fått lära av oss att I skolen föra, Gud till behag - en sådan vandel som I redan fören.
\par 2 I veten ju vilka bud vi hava givit eder genom Herren Jesus.
\par 3 Ty detta är Guds vilja, detta som hör till eder helgelse, att I avhållen eder från otukt,
\par 4 och att var och en av eder vet att hava sin egen maka i helgelse och ära,
\par 5 icke i begärelses lusta såsom hedningarna - vilka icke känna Gud -
\par 6 och att ingen i sitt förhållande till sin broder kränker honom eller gör honom något förfång, ty Herren är en hämnare över allt detta, såsom vi redan förut hava sagt och betygat för eder.
\par 7 Gud har ju icke kallat oss till orenhet, utan till att leva i helgelse.
\par 8 Den som icke vill veta av detta, han förkastar alltså icke en människa, utan Gud, honom som giver sin helige Ande till att bo i eder.
\par 9 Om broderlig kärlek är det icke behövligt att skriva till eder, ty I haven själva fått lära av Gud att älska varandra;
\par 10 så handlen I ju ock mot alla bröderna i hela Macedonien. Men vi förmana eder, käre bröder, att allt mer förkovra eder häri
\par 11 och att sätta eder ära i att leva i stillhet och sköta vad eder åligger och arbeta med edra händer, enligt vad vi hava bjudit eder,
\par 12 så att I skicken eder höviskt mot dem som stå utanför och icke behöven anlita någons hjälp.
\par 13 Vi vilja icke lämna eder, käre bröder, i okunnighet om huru det förhåller sig med dem som avsomna, för att I icke skolen sörja såsom de andra, de som icke hava något hopp.
\par 14 Ty lika visst som Jesus, såsom vi tro, har dött och har uppstått, lika visst skall ock Gud genom Jesus föra dem som äro avsomnade fram jämte honom.
\par 15 Såsom ett ord från Herren säga vi eder nämligen detta, att vi som leva och lämnas kvar till Herrens tillkommelse ingalunda skola komma före dem som äro avsomnade.
\par 16 Ty Herren skall själv stiga ned från himmelen, och ett maktbud skall ljuda, en överängels röst och en Guds basun. Och först skola de i Kristus döda uppstå;
\par 17 sedan skola vi som då ännu leva och hava lämnats kvar bliva jämte dem bortryckta på skyar upp i luften, Herren till mötes; och så skola vi alltid få vara hos Herren.
\par 18 Så trösten nu varandra med dessa ord.

\chapter{5}

\par 1 Vad åter angår tid och stund härför, så är det icke behövligt att därom skriva till eder, käre bröder.
\par 2 Ty I veten själva nogsamt att Herrens dag kommer såsom en tjuv om natten.
\par 3 Bäst de säga: "Allt står väl till, och ingen fara är på färde", då kommer plötsligt fördärv över dem, såsom födslovåndan över en havande kvinna, och de skola förvisso icke kunna fly undan.
\par 4 Men I, käre bröder, I leven icke mörker, så att den dagen kan komma över eder såsom en tjuv;
\par 5 I ären ju alla ljusets barn och dagens barn. Ja, vi höra icke natten eller mörkret till;
\par 6 låtom oss alltså icke sova såsom de andra, utan låtom oss vaka och vara nyktra.
\par 7 De som sova, de sova om natten, och de som dricka sig druckna, de äro druckna om natten;
\par 8 men vi som höra dagen till, vi må vara nyktra, iklädda trons och kärlekens pansar, med frälsningens hopp såsom hjälm.
\par 9 Ty Gud har icke bestämt oss till att drabbas av vrede, utan till att vinna frälsning genom vår Herre, Jesus Kristus,
\par 10 som har dött för oss, på det att vi må leva tillika med honom, vare sig vi ännu äro vakna eller vi äro avsomnade.
\par 11 Trösten därför varandra, och uppbyggen varandra inbördes, såsom I ock redan gören.
\par 12 Vi bedja eder, käre bröder, att rätt uppskatta de män som arbeta bland eder, och som äro edra föreståndare i Herren och förmana eder.
\par 13 Låten dem vara eder övermåttan kära, för det verks skull som de utföra. Hållen frid inbördes.
\par 14 Vi bjuda eder, käre bröder: Förmanen de oordentliga, uppmuntren de klenmodiga, tagen eder an de svaga, visen tålamod mot var man.
\par 15 Sen till, att ingen vedergäller någon med ont för ont; faren fastmer alltid efter att göra vad gott är mot varandra och mot var man.
\par 16 Varen alltid glada.
\par 17 Bedjen oavlåtligen.
\par 18 Tacken Gud i alla livets förhållanden. Ty att I så gören är Guds vilja i Kristus Jesus.
\par 19 Utsläcken icke Anden,
\par 20 förakten icke profetisk tal,
\par 21 men pröven allt, behållen vad gott är,
\par 22 avhållen eder från allt ont, av vad slag det vara må.
\par 23 Men fridens Gud själv helge eder till hela eder varelse, så att hela eder ande och eder själ och eder kropp finnas bevarade ostraffliga vid vår Herres, Jesu Kristi, tillkommelse.
\par 24 Trofast är han som kallar eder; han skall ock utföra sitt verk.
\par 25 Käre bröder, bedjen för oss.
\par 26 Hälsen alla bröderna med en helig kyss.
\par 27 Jag besvär eder vid Herren att låta uppläsa detta brev för alla bröderna.
\par 28 Vår Herres, Jesu Kristi, nåd vare med eder.


\end{document}