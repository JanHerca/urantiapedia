\begin{document}

\title{Nahum}


\chapter{1}

\par 1 Detta är en utsaga om Nineve, den bok som innehåller elkositen Nahums syn.
\par 2 HERREN är en nitälskande Gud och en hämnare, ja, en hämnare är HERREN, en som kan vredgas. En hämnare är HERREN mot sina ovänner, vrede behåller han mot sina fiender.
\par 3 HERREN är långmodig, men han är stor i kraft, och ingalunda låter han någon bliva ostraffad. HERREN har sin väg i storm och oväder och molnen äro dammet efter hans fötter.
\par 4 Han näpser havet och låter det uttorka och alla strömmar låter han sina bort. Då försmäkta Basan och Karmel, Libanons grönska försmäktar.
\par 5 Bergen bäva för honom, och höjderna försmälta av ångest. Jorden röres upp för hans ansikte, jordens krets med alla som bo därpå.
\par 6 Vem kan bestå för hans ogunst, och vem kan uthärda hans vrede glöd? Hans förtörnelse utgjuter sig såsom eld, och klipporna rämna inför honom.
\par 7 HERREN är god, ett värn i nödens tid, och han låter sig vårda om dem som förtrösta på honom.
\par 8 Men genom en störtflod gör han ände på platsen där den staden står, och hans fiender förföljas av mörker.
\par 9 Ja, på edert anslag mot HERREN gör han ände, icke två gånger behöver hemsökelsen drabba.
\par 10 Ty om de ock äro hopslingrade såsom törnsnår och så fulla av livssaft, som deras dryck är av must, skola de likväl alla förbrännas såsom torrt strå.
\par 11 Ty från dig drog ut en man som hade onda anslag mot HERREN, en vilkens rådslag voro fördärv.
\par 12 Så säger HERREN: "Huru starka och huru många de ock må vara, skola de ändå mejas av och försvinna; och om jag förr har plågat dig, så skall jag nu ej göra det mer.
\par 13 Ty nu skall jag bryta sönder de ok han har lagt på dig, och hans band skall jag slita av."
\par 14 Men om dig bjuder HERREN så "Ingen avkomma av ditt namn skall mer få finnas. Ur dina gudars hus skall jag utrota alla beläten, både skurna och gjutna. En grav bereder jag åt dig, ty på skam har du kommit."
\par 15 Se, över bergen nalkas glädjebudbärarens fötter hans som förkunnar frid: "Fira dina högtider, Juda, infria dina löften. Ty ej mer skall fördärvaren draga fram mot dig; han varder förgjord i grund."

\chapter{2}

\par 1 En folkförskingrare drager upp mot dig; bevaka dina fästen. Speja utåt vägen, omgjorda dina länder bruka din kraft, så mycket du förmår.
\par 2 Ty HERREN vill återställa Jakobs höghet såsom Israels höghet, då nu plundrare så hava ödelagt dem och så fördärvat deras vinträd.
\par 3 Hans hjältars sköldar äro färgade röda, stridsmännen gå klädda i scharlakan; vagnarna gnistra av eld, när han gör dem redo till strid; och man skakar lansar av cypressträ.
\par 4 På vägarna storma vagnarna fram, de köra om varandra på fälten; såsom bloss äro de att skåda lika ljungeldar fara de åstad.
\par 5 Han vet nogsamt vilka väldiga kämpar han äger; de störta överända, där de rusa framåt. De hasta mot stadens murar, och stormtaken göras redo.
\par 6 Strömportarna måste öppna sig, och palatset försmälter av ångest.
\par 7 Ja, domen står fast: hon bliver blottad, bortsläpad; hennes tärnor måste sucka likasom duvor och slå sig för sitt bröst.
\par 8 I all sin tid var Nineve lik en vattenrik damm, men nu flyr vattnet bort. "Stannen! Stannen!" - Nej, ingen vänder sig om.
\par 9 Röven nu silver, röven guld. Här finnas skatter utan ände, överflöd på alla dyrbara håvor.
\par 10 Ödeläggelse och förödelse och förstörelse! Förfärade hjärtan och skälvande knän! Darrande länder allestädes! Allas ansikten hava skiftat färg.
\par 11 Var är nu lejonens kula, den plats där de unga lejonen förtärde sitt rov, där lejonet och lejoninnan hade sin gång, där lejonungen gick omkring, utan att någon skrämde bort den?
\par 12 Var är lejonet som tog rov, så mycket dess ungar ville hava, och dödade åt sina lejoninnor, ja, uppfyllde sina hålor med rov och sina kulor med rövat gods?
\par 13 Se, jag skall vända mig mot dig, säger HERREN Sebaot; dina vagnar skall jag låta gå upp i rök, och dina unga lejon skall svärdet förtära. Jag skall utrota ditt rövade gods från jorden och man skall ej mer höra dina sändebuds röst

\chapter{3}

\par 1 Ve dig, du blodstad, alltigenom så full av lögn och våld, du som aldrig upphör att röva!
\par 2 Hör, piskor smälla! Hör, vagnshjul dåna! Hästar jaga fram, och vagnar rulla åstad.
\par 3 Ryttare komma i fyrsprång; svärden ljunga, och spjuten blixtra. Slagna ser man i mängd och lik i stora hopar; igen ände är på döda, man stupar över döda.
\par 4 Allt detta för den myckna otukt hon bedrev, hon, den fagra och trollkunniga skökan, som prisgav folkslag genom sin otukt och folkstammar genom sina trolldomskonster.
\par 5 Se, jag skall vända mig mot dig, säger HERREN Sebaot; jag skall lyfta upp ditt mantelsläp över ditt ansikte och låta folkslag se din blygd och konungariken din skam.
\par 6 Och jag skall kasta på dig vad styggeligt är, jag skall låta dig bliva föraktad, ja, göra dig till ett skådespel.
\par 7 Var och en som ser dig skall sky dig och skall säga: "Nineve är ödelagt, men vem kan ömka det?" Ja, var finner man någon som vill trösta dig?
\par 8 Är du då bättre än No-Amon, hon som tronade vid Nilens strömmar, omsluten av vatten - ett havets fäste, som hade ett hav till mur?
\par 9 Etiopier i mängd och egyptier utan ände, putéer och libyer voro dig till hjälp.
\par 10 Också hon måste ju gå i landsflykt och fångenskap, också hennes barn blevo krossade i alla gathörn; om hennes ädlingar kastade man lott, och alla hennes stormän blevo fängslade med kedjor
\par 11 Så skall ock du bliva drucken och sjunka i vanmakt; också du skall få leta efter något värn mot fienden.
\par 12 Alla dina fästen likna fikonträd med brådmogen frukt: vid minsta skakning falla de i munnen på den som vill äta dem.
\par 13 Se, ditt manskap är hos dig såsom kvinnor; ditt lands portar stå vidöppna för dina fiender; eld förtär dina bommar.
\par 14 Hämta dig vatten till förråd under belägringen, förstärk dina fästen. Stig ned i leran och trampa i murbruket; grip till tegelformen.
\par 15 Bäst du står där, skall elden förtära dig och svärdet utrota dig. Ja, såsom av gräsmaskar skall du bliva uppfrätt, om du ock själv samlar skaror så talrika som gräsmaskar, skaror så talrika som gräshoppor.
\par 16 Om du ock har krämare flera än himmelens stjärnor, så vet: gräsmaskarna fälla sina vingars höljen och flyga bort.
\par 17 Ja, dina furstar äro såsom gräshoppor och dina hövdingar såsom gräshoppssvärmar: de stanna inom murarna, så länge det är svalt, men när solen kommer fram, då fly de bort, och sedan vet ingen var de finnas.
\par 18 Dina herdar hava slumrat in, du Assurs konung; dina väldige ligga i ro. Ditt folk är förstrött uppe på bergen, och ingen församlar det.
\par 19 Det finnes ingen bot för din skada oläkligt är ditt sår. Alla som höra vad som har hänt dig klappa i händerna över dig. Ty över vem gick ej din ondska beständigt?


\end{document}