\begin{document}

\title{Apostlagärningarna}


\chapter{1}

\par 1 I min förra skrift, gode Teofilus, har jag berättat om allt vad Jesus gjorde och lärde,
\par 2 ända till den dag då han blev upptagen, sedan han genom helig ande hade givit sina befallningar åt apostlarna som han hade utvalt.
\par 3 För dem hade han ock genom många säkra bevis tett sig såsom levande, efter utståndet lidande; ty under fyrtio dagar lät han sig ses av dem och talade med dem om Guds rike.
\par 4 När han då var tillsammans med dem, bjöd han dem och sade: "Lämnen icke Jerusalem, utan förbiden där vad Fadern har utlovat, det varom I haven hört av mig.
\par 5 Ty Johannes döpte med vatten, men få dagar härefter skolen I bliva döpta i helig ande."
\par 6 Då de nu hade kommit tillhopa, frågade de honom och sade: "Herre, skall du nu i denna tid upprätta igen riket åt Israel?"
\par 7 Han svarade dem: "Det tillkommer icke eder att få veta tider eller stunder som Fadern i sin makt har fastställt.
\par 8 Men när den helige Ande kommer över eder, skolen I undfå kraft och bliva mina vittnen, både i Jerusalem och i hela Judeen och Samarien, och sedan intill jordens ända."
\par 9 När han hade sagt detta, lyftes han inför deras ögon upp i höjden, och en sky tog honom bort ur deras åsyn.
\par 10 Och medan de skådade mot himmelen, under det han for upp, se, då stodo hos dem två män i vita kläder.
\par 11 Och dessa sade: "I galileiske män, varför stån I och sen mot himmelen? Denne Jesus, som har blivit upptagen från eder till himmelen, han skall komma igen på samma sätt som I haven sett honom fara upp till himmelen."
\par 12 Sedan vände de tillbaka till Jerusalem från det berg som kallas Oljeberget, vilket ligger nära Jerusalem, icke längre därifrån, än man får färdas på en sabbat.
\par 13 Och när de hade kommit dit, gingo de upp i den sal i övre våningen, där de plägade vara tillsammans: Petrus och Johannes och Jakob och Andreas, Filippus och Tomas, Bartolomeus och Matteus, Jakob, Alfeus' son, och Simon ivraren och Judas, Jakobs son.
\par 14 Alla dessa höllo endräktigt ut i bön tillika med Maria, Jesu moder, och några andra kvinnor samt Jesu bröder.
\par 15 En av de dagarna stod Petrus upp och talade bland bröderna, som då voro församlade till ett antal av omkring etthundratjugu; han sade:
\par 16 "Mina bröder, det skriftens ord skulle fullbordas, som den helige Ande genom Davids mun hade profetiskt talat om Judas, vilken blev vägvisare åt de män som grepo Jesus.
\par 17 Han var ju räknad bland oss och hade också fått detta ämbete på sin lott.
\par 18 Och med de penningar han hade fått såsom lön för sin ogärning förvärvade han sig en åker. Men han störtade framstupa ned, och hans kropp brast mitt itu, så att alla hans inälvor gåvo sig ut.
\par 19 Detta blev bekant för alla Jerusalems invånare, och så blev den åkern på deras tungomål kallad Akeldamak (det betyder Blodsåkern).
\par 20 Så är ju skrivet i Psalmernas bok: 'Hans gård blive öde, och ingen må finnas, som bor däri'; och vidare: 'Hans ämbete tage en annan.'
\par 21 Därför bör nu någon av de män som följde oss under hela den tid då Herren Jesus gick ut och in bland oss,
\par 22 allt ifrån den dag då han döptes av Johannes ända till den dag då han blev upptagen och skildes ifrån oss - någon av dessa män bör insättas till att jämte oss vittna om hans uppståndelse."
\par 23 Därefter ställde de fram två: Josef (som kallades Barsabbas och hade tillnamnet Justus) och Mattias.
\par 24 Och de bådo och sade: "Herre, du som känner allas hjärtan, visa oss vilken av dessa två du har utvalt
\par 25 till att få den plats såsom tjänare och apostel, vilken Judas övergav, för att gå till den plats som var hans."
\par 26 Och de drogo lott om dem, och lotten föll på Mattias. Och så blev denne, jämte de elva, räknad såsom apostel.

\chapter{2}

\par 1 När sedan pingstdagen var inne, voro de alla församlade med varandra.
\par 2 Då kom plötsligt från himmelen ett dån, såsom om en våldsam storm hade dragit fram; och det uppfyllde hela huset där de sutto.
\par 3 Och tungor såsom av eld visade sig för dem och fördelade sig och satte sig på dem, en på var av dem.
\par 4 Och de blevo alla uppfyllda av helig ande och begynte tala andra tungomål, efter som Anden ingav dem att tala.
\par 5 Nu bodde i Jerusalem fromma judiska män från allahanda folk under himmelen.
\par 6 Och när dånet hördes, församlade sig hela hopen, och en stor rörelse uppstod, ty var och en hörde sitt eget tungomål talas av dem.
\par 7 Och de uppfylldes av häpnad och förundran och sade: "Äro de icke galiléer, alla dessa som här tala?
\par 8 Huru kommer det då till, att var och en av oss hör sitt eget modersmål talas?
\par 9 Vi må vara parter eller meder eller elamiter, vi må hava vårt hem i Mesopotamien eller Judeen eller Kappadocien, i Pontus eller provinsen Asien,
\par 10 i Frygien eller Pamfylien, i Egypten eller i Libyens bygder, åt Cyrene till, eller vara hitflyttade främlingar från Rom,
\par 11 vi må vara judar eller proselyter, kretenser eller araber, alla höra vi dem på våra egna tungomål tala om Guds väldiga gärningar."
\par 12 Så uppfylldes de alla av häpnad och visste icke vad de skulle tänka. Och de sade, den ene till den andre: "Vad kan detta betyda?"
\par 13 Men somliga drevo gäck med dem och sade: "De äro fulla av sött vin."
\par 14 Då trädde Petrus fram, jämte de elva, och hov upp sin röst och talade till dem: "I judiske män och I alla Jerusalems invånare, detta mån I veta, och lyssnen nu till mina ord:
\par 15 Det är icke så som I menen, att dessa äro druckna; det är ju blott tredje timmen på dagen.
\par 16 Nej, här uppfylles det som är sagt genom profeten Joel:
\par 17 'Och det skall ske i de yttersta dagarna, säger Gud, att jag skall utgjuta av min Ande över allt kött, och edra söner och edra döttrar skola profetera, och edra ynglingar skola se syner, och edra gamla män skola hava drömmar;
\par 18 ja, över mina tjänare och mina tjänarinnor skall jag i de dagarna utgjuta av min Ande, och de skola profetera.
\par 19 Och jag skall låta undertecken synas uppe på himmelen och tecken nere på jorden: blod och eld och rökmoln.
\par 20 Solen skall vändas i mörker och månen i blod, förrän Herrens dag kommer, den stora och härliga.
\par 21 Och det skall ske att var och en som åkallar Herrens namn, han skall varda frälst.'
\par 22 I män av Israel, hören dessa ord: Jesus från Nasaret, en man som inför eder fick vittnesbörd av Gud genom kraftgärningar och under och tecken, vilka Gud genom honom gjorde bland eder, såsom I själva veten,
\par 23 denne som blev given i edert våld, enligt vad Gud i sitt rådslut och sin försyn hade bestämt, honom haven I genom män som icke veta av lagen låtit fastnagla vid korset och döda.
\par 24 Men Gud gjorde en ände på dödens vånda och lät honom uppstå, eftersom det icke var möjligt att han skulle kunna behållas av döden.
\par 25 Ty David säger med tanke på honom: 'Jag har haft Herren för mina ögon alltid, ja, han är på min högra sida, för att jag icke skall vackla.
\par 26 Fördenskull gläder sig mitt hjärta, och min tunga fröjdar sig, och jämväl min kropp får vila med en förhoppning:
\par 27 den, att du icke skall lämna min själ åt dödsriket och icke låta din Helige se förgängelse.
\par 28 Du har kungjort mig livets vägar; du skall uppfylla mig med glädje inför ditt ansikte.'
\par 29 Mina bröder, jag kan väl fritt säga till eder om vår stamfader David att han är både död och begraven; hans grav finnes ju ibland oss ännu i dag.
\par 30 Men eftersom han var en profet och visste att Gud med ed hade lovat honom att 'av hans livs frukt sätta en konung på hans tron',
\par 31 därför förutsåg han att Messias skulle uppstå, och talade därom och sade att Messias icke skulle lämnas åt dödsriket, och att hans kropp icke skulle se förgängelse.
\par 32 Denne - Jesus - har nu Gud låtit uppstå; därom kunna vi alla vittna.
\par 33 Och sedan han genom Guds högra hand har blivit upphöjd och av Fadern undfått den utlovade helige Anden, har han utgjutit vad I här sen och hören.
\par 34 Ty icke har David farit upp till himmelen; fastmer säger han själv: 'Herren sade till min herre: Sätt dig på min högra sida,
\par 35 till dess jag har lagt dina fiender dig till en fotapall.
\par 36 Så må nu hela Israels hus veta och vara förvissat om att denne Jesus som I haven korsfäst, honom har Gud gjort både till Herre och till Messias."
\par 37 När de hörde detta, kände de ett styng i hjärtat. Och de sade till Petrus och de andra apostlarna: "Bröder, vad skola vi göra?"
\par 38 Petrus svarade dem: "Gören bättring, och låten alla döpa eder i Jesu Kristi namn till edra synders förlåtelse; då skolen I såsom gåva undfå den helige Ande.
\par 39 Ty eder gäller löftet och edra barn, jämväl alla dem som äro i fjärran, så många som Herren, vår Gud, kallar."
\par 40 Också med många andra ord bad och förmanade han dem, i det han sade: "Låten frälsa eder från detta vrånga släkte."
\par 41 De som då togo emot hans ort läto döpa sig; och så ökades församlingen på den dagen med vid pass tre tusen personer.
\par 42 Och dessa höllo fast vid apostlarnas undervisning och brödragemenskapen, vid brödsbrytelsen och bönerna.
\par 43 Och fruktan kom över var och en; och många under och tecken gjordes genom apostlarna.
\par 44 Men alla de som trodde höllo sig tillsammans och hade allting gemensamt;
\par 45 de sålde sina jordagods och vad de eljest ägde och delade med sig därav åt alla, eftersom var och en behövde.
\par 46 Och ständigt, var dag, voro de endräktigt tillsammans i helgedomen; och hemma i husen bröto de bröd och åto med fröjd och i hjärtats enfald, och lovade Gud.
\par 47 Och allt folket vad dem väl bevåget. Och Herren ökade församlingen, dag efter dag, med dem som läto sig frälsas.

\chapter{3}

\par 1 Och Petrus och Johannes gingo upp till helgedomen, till den bön som hölls vid nionde timmen.
\par 2 Och där bars fram en man som hade varit ofärdig allt ifrån moderlivet, och som man var dag plägade sätta vid den port i helgedomen, som kallades Sköna porten, för att han skulle kunna begära allmosor av dem som gingo in i helgedomen.
\par 3 När denne nu fick se Petrus och Johannes, då de skulle gå in i helgedomen, bad han dem om en allmosa.
\par 4 Då fäste Petrus och Johannes sina ögon på honom, och Petrus sade: "Se på oss."
\par 5 När han då gav akt på dem, i förväntan att få något av dem,
\par 6 sade Petrus: "Silver och guld har jag icke; men vad jag har, det giver jag dig. I Jesu Kristi, nasaréens namn: stå upp och gå."
\par 7 Och så fattade han honom vid högra handen och reste upp honom. Och strax fingo hans fötter och fotleder styrka,
\par 8 och han sprang upp och stod upprätt och begynte gå och följde dem in i helgedomen, alltjämt gående och springande, under det att han lovade Gud.
\par 9 Och allt folket såg honom, där han gick omkring och lovade Gud.
\par 10 Och när de kände igen honom och sågo att det var samme man som plägade sitta och begära allmosor vid Sköna porten i helgedomen, blevo de uppfyllda av häpnad och bestörtning över det som hade vederfarits honom.
\par 11 Då han nu höll sig till Petrus och Johannes, strömmade allt folket, utom sig av häpnad, tillsammans till dem på den plats som kallades Salomos pelargång.
\par 12 När Petrus såg detta, tog han till orda och talade till folket så: "I män av Israel, varför undren I över denne man, och varför sen I så på oss, likasom hade vi genom någon vår kraft eller fromhet åstadkommit att han kan gå?
\par 13 Nej, Abrahams och Isaks och Jakobs Gud, våra fäders Gud, har förhärligat sin tjänare Jesus, honom som I utlämnaden, och som I förnekaden inför Pilatus, när denne redan hade beslutit att giva honom lös.
\par 14 Ja, I förnekaden honom, den helige och rättfärdige, och begärden att en dråpare skulle givas åt eder.
\par 15 Och livets furste dräpten I, men Gud uppväckte honom från de döda; därom kunna vi själva vittna.
\par 16 Och det är på grund av tron på hans namn som denne man, vilken I sen och kännen, har undfått styrka av hans namn; och den tro som verkas genom Jesus har, i allas eder åsyn, gjort att han nu kan bruka alla lemmar.
\par 17 Nu vet jag väl, mina bröder, att I såväl som edra rådsherrar haven gjort detta, därför att I icke vissten bättre.
\par 18 Men Gud har på detta sätt låtit det gå i fullbordan, som han förut genom alla sina profeters mun hade förkunnat, nämligen att hans Smorde skulle lida.
\par 19 Gören därför bättring och omvänden eder, så att edra synder bliva utplånade,
\par 20 på det att tider av vederkvickelse må komma från Herren, i det att han sänder den Messias som han har utsett åt eder, nämligen Jesus,
\par 21 vilken dock himmelen måste behålla intill de tider nå allt skall bliva upprättat igen, varom Gud har talat genom sina forntida heliga profeters mun.
\par 22 Moses har ju sagt: 'En profet skall Herren Gud låta uppstå åt eder, av edra bröder, en som är mig lik; honom skolen I lyssna till i allt vad han talar till eder.
\par 23 Och det skall ske att var och en som icke lyssnar till den profeten, han skall utrotas ur folket.'
\par 24 Och sedan hava alla profeterna, både Samuel och de som följde efter honom, så många som hava talat, också bebådat dessa tider.
\par 25 I ären själva barn av profeterna och delaktiga i det förbund som Gud slöt med edra fäder, när han sade till Abraham: 'Och i din säd skola alla släkter på jorden varda välsignade.'
\par 26 För eder först och främst har Gud låtit sin tjänare uppstå, och han har sänt honom för att välsigna eder, när I, en och var, omvänden eder från eder ondska."

\chapter{4}

\par 1 Medan de ännu talade till folket, kommo prästerna och tempelvaktens befälhavare och sadducéerna över dem.
\par 2 Ty det förtröt dem att du undervisade folket och i Jesus förkunnade uppståndelsen från de döda.
\par 3 Därför grepo de dem nu och satte dem i fängsligt förvar till följande dag, eftersom det redan var afton.
\par 4 Men många av dem som hade hört vad som hade talats kommo till tro; och antalet av männen uppgick nu till vid pass fem tusen.
\par 5 Dagen därefter församlade sig deras rådsherrar och äldste och skriftlärde i Jerusalem;
\par 6 där voro då ock Hannas, översteprästen, och Kaifas och Johannes och Alexander och alla som voro av översteprästerlig släkt.
\par 7 Och de läto föra fram dem inför sig och frågade dem: "Av vilken makt eller i genom vilket namn haven I gjort detta?"
\par 8 Då sade Petrus till dem, uppfylld av helig ande: "I folkets rådsherrar och äldste,
\par 9 eftersom vi i dag underkastas rannsakning för en god gärning mot en sjuk man och tillfrågas varigenom denne har blivit botad,
\par 10 så mån I veta, I alla och hela Israels folk, att det är genom Jesu Kristi, nasaréens, namn, hans som I haven korsfäst, men som Gud har uppväckt från de döda - att det är genom det namnet som denne man står inför eder frisk och färdig.
\par 11 Han är 'den stenen som av byggningsmännen' - av eder själva - 'aktades för intet, men som har blivit en hörnsten'.
\par 12 Och i ingen annan finnes frälsning; ej heller finnes under himmelen något annat namn, bland människor givet, genom vilket vi kunna bliva frälsta."
\par 13 När de sågo Petrus och Johannes vara så frimodiga och förnummo att de voro olärda män ur folket, förundrade de sig. Men så kände de igen dem och påminde sig att de hade varit med Jesus.
\par 14 Och när de sågo mannen som hade blivit botad stå där bredvid dem, kunde de icke säga något däremot.
\par 15 De befallde dem alltså att gå ut från rådsförsamlingen. Sedan överlade de med varandra
\par 16 och sade: "Vad skola vi göra med dessa män? Att ett märkligt tecken har blivit gjort av dem, det är ju uppenbart för alla Jerusalems invånare, och vi kunna icke förneka det.
\par 17 Men för att detta icke ännu mer skall komma ut bland folket, må vi strängeligen förbjuda dem att hädanefter i det namnet tala för någon människa."
\par 18 Därefter kallade de in dem och förbjödo dem helt och hållet att tala eller undervisa i Jesu namn.
\par 19 Men Petrus och Johannes svarade och sade till dem: "Om det är rätt inför Gud att vi hörsamma eder mer är Gud, därom mån I själva döma;
\par 20 vi för vår del kunna icke underlåta att tala vad vi hava sett och hört."
\par 21 Då förbjödo de dem detsamma ännu strängare, men läto dem sedan gå. Ty eftersom alla prisade Gud för det som hade skett, kunde de, för folkets skull, icke finna någon utväg genom att straffa dem.
\par 22 Mannen som genom detta tecken hade blivit botad var nämligen över fyrtio år gammal.
\par 23 När de alltså hade blivit lösgivna, kommo de till sina egna och omtalade för dem allt vad översteprästerna och de äldste hade sagt dem.
\par 24 Då de hörde detta, ropade de endräktigt till Gud och sade: "Herre, det är du som har gjort himmelen och jorden och havet och allt vad i dem är.
\par 25 Och du har genom vår fader Davids, din tjänares, mun sagt genom helig ande: 'Varför larmade hedningarna och tänkte folken fåfänglighet?
\par 26 Jordens konungar trädde fram, och furstarna samlade sig tillhopa mot Herren och hans Smorde.'
\par 27 Ja, i sanning, de församlade sig i denna stad mot din helige tjänare Jesus, mot honom som du har smort: Herodes och Pontius Pilatus med hedningarna och Israels folkstammar;
\par 28 de församlade sig till att utföra allt vad din hand och ditt rådslut förut hade bestämt skola ske.
\par 29 Och nu, Herre, se till deras hotelser, och giv dina tjänare att de med all frimodighet må förkunna ditt ord,
\par 30 i det att du uträcker din hand till att bota de sjuka, och till att låta tecken och under ske genom din helige tjänare Jesu namn."
\par 31 När de hade slutat att bedja, skakades platsen där de voro församlade, och de blevo alla uppfyllda av den helige Ande, och de förkunnade Guds ord med frimodighet.
\par 32 Och i hela skaran av dem som trodde var ett hjärta och en själ. Ingen enda kallade något av det han ägde för sitt, utan du hade allting gemensamt.
\par 33 Och med stor kraft framburo apostlarna vittnesbördet om Herren Jesu uppståndelse; och stor nåd var över dem alla.
\par 34 Bland dem fanns ingen som led nöd; ty alla som ägde något jordstycke eller något hus sålde detta och buro fram betalningen för det sålda
\par 35 och lade den för apostlarnas fötter, och man delade ut därav, så att var och en fick efter som han behövde.
\par 36 Josef, som av apostlarna ock kallades Barnabas (det betyder förmanaren), en levit som var bördig från Cypern,
\par 37 också han sålde en åker som han ägde och bar fram penningarna och lade dem för apostlarnas fötter.

\chapter{5}

\par 1 Men en ung man vid namn Ananias och hans hustru Safira sålde ett jordagods,
\par 2 och han tog därvid, med sin hustrus vetskap, undan något av betalningen därför; allenast en del bar han fram och lade för apostlarnas fötter.
\par 3 Då sade Petrus: "Ananias, varför har Satan fått uppfylla ditt hjärta, så att du har velat bedraga den helige Ande och taga undan något av betalningen för jordstycket?
\par 4 Detta var ju din egendom, medan du hade det kvar; och när det var sålt, voro ju penningarna i din makt. Huru kunde du få något sådant i sinnet? Du har ljugit, icke för människor, utan för Gud."
\par 5 När Ananias hörde dessa ord, föll han ned och gav upp andan. Och stor fruktan kom över alla som hörde detta.
\par 6 Och de yngre männen stodo upp och höljde in honom och buro ut honom och begrovo honom.
\par 7 Vid pass tre timmar därefter kom hans hustru in, utan att veta om, vad som hade skett.
\par 8 Petrus sade då till henne: "Säg mig, var det för den summan I sålden jordstycket?" Hon svarade: "Ja, för den summan."
\par 9 Då sade Petrus till henne: "Huru kunden I vilja komma överens om att fresta Herrens Ande? Se, härutanför dörren höras nu fotstegen av de män som hava begravt din man; och de skola bära ut också dig."
\par 10 Och strax föll hon ned vid hans fötter och gav upp andan; och när de unge männen kommo in, funno de henne död. De buro då ut henne och begrovo henne bredvid hennes man.
\par 11 Och stor fruktan kom över hela församlingen och över alla andra som hörde detta.
\par 12 Och genom apostlarna gjordes många tecken och under bland folket; och de höllo sig alla endräktigt tillsammans i Salomos pelargång.
\par 13 Av de andra dristade sig ingen att närma sig dem, men folket höll dem i ära.
\par 14 Och ännu flera trodde och slöto sig till Herren, hela skaror av både män och kvinnor.
\par 15 Ja, man bar de sjuka ut på gatorna och lade dem på bårar och i sängar, för att, när Petrus kom gående, åtminstone hans skugga måtte falla på någon av dem.
\par 16 Och jämväl från städerna runt omkring Jerusalem kom folket i skaror och förde med sig sjuka och sådana som voro plågade av orena andar; och alla blevo botade.
\par 17 Då stod översteprästen upp och alla som höllo med honom - de som hörde till sadducéernas parti - och de uppfylldes av nitälskan
\par 18 och läto gripa apostlarna och sätta dem i allmänt häkte.
\par 19 Men en Herrens ängel öppnade om natten fängelsets portar och förde ut dem och sade:
\par 20 "Gån åstad och träden upp i helgedomen, och talen till folket alla det sanna livets ord."
\par 21 När de hade hört detta, gingo de inemot dagbräckningen in i helgedomen och undervisade. Emellertid kommo översteprästen och de som höllo med honom och sammankallade Stora rådet, alla Israels barns äldste. Därefter sände de åstad till fängelset för att hämta dem.
\par 22 Men när rättstjänarna kommo dit, funno de dem icke i fängelset. De vände då tillbaka och omtalade detta
\par 23 och sade: "Fängelset funno vi stängt med all omsorg och väktarna stående utanför portarna, men då vi öppnade, funno vi ingen därinne."
\par 24 När tempelvaktens befälhavare och översteprästerna hörde detta, visst de icke vad de skulle tänka därom, eller vad som skulle bliva av detta.
\par 25 Då kom någon och berättade för den: "De män som I haven insatt i fängelset, de stå nu i helgedomen och undervisa folket."
\par 26 Befälhavaren gick då med rättstjänarna åstad och hämtade dem; dock brukade de icke våld, ty de fruktade att bliva stenade av folket.
\par 27 Och sedan de hade hämtat dem, förde de dem fram inför Stora rådet. Och översteprästen anställde förhör med dem
\par 28 och sade: "Vi hava ju allvarligen förbjudit eder att undervisa i det namnet, och likväl haven I uppfyllt Jerusalem med eder undervisning, och I viljen nu låta den mannens blod komma över oss."
\par 29 Men Petrus och de andra apostlarna svarade och sade: "Man måste lyda Gud mer än människor.
\par 30 Våra fäders Gud har uppväckt Jesus, som I haden upphängt på trä och dödat.
\par 31 Och Gud har med sin högra hand upphöjt honom till en hövding och frälsare, för att åt Israel förläna bättring och syndernas förlåtelse.
\par 32 Om allt detta kunna vi själva vittna, så ock den helige Ande, vilken Gud har givit åt dem som äro honom lydiga."
\par 33 När de hörde detta, blevo de mycket förbittrade och ville döda dem.
\par 34 Men en farisé, en laglärare vid namn Gamaliel, som var aktad av allt folket, stod då upp i Rådet och tillsade att man för en kort stund skulle föra ut männen.
\par 35 Sedan sade han till de andra: "I män av Israel, sen eder för vad I tänken göra med dessa män.
\par 36 För en tid sedan uppträdde ju Teudas och gav sig ut för att något vara, och till honom slöt sig en hop av vid pass fyra hundra män. Och han blev dödad, och alla som hade trott på honom förskingrades och blevo till intet.
\par 37 Efter honom uppträdde Judas från Galileen, vid den tid då skattskrivningen pågick; denne förledde en hop folk till avfall, så att de följde honom. Också han förgicks, och alla som hade trott på honom blevo förskingrade.
\par 38 Och nu säger jag eder: Befatten eder icke med dessa män, utan låten dem vara; ty skulle detta vara ett rådslag eller ett verk av människor, så kommer det att slås ned;
\par 39 men är det av Gud, så kunnen I icke slå ned dessa män. Sen till, att I icke mån befinnas strida mot Gud själv."
\par 40 Och de lydde hans råd; de kallade in apostlarna, och sedan de hade låtit gissla dem, förbjödo de dem att tala i Jesu namn och läto dem därefter gå.
\par 41 Och de gingo ut från rådsförsamlingen, glada över att de hade aktats värdiga att lida smälek för det namnets skull.
\par 42 Och de upphörde icke att var dag undervisa i helgedomen och hemma i husen och förkunna evangelium om Kristus Jesus.

\chapter{6}

\par 1 Vid denna tid, då nu lärjungarnas antal förökades, begynte de grekiska judarna knorra mot de infödda hebréerna över att deras änkor blevo förbisedda vid den dagliga utdelningen.
\par 2 Då sammankallade de tolv hela lärjungaskaran och sade: "Det är icke tillbörligt att vi försumma Guds ord för att göra tjänst vid borden.
\par 3 Så utsen nu bland eder, I bröder, sju män som hava gott vittnesbörd om sig och äro fulla av ande och vishet, män som vi kunna sätta till att sköta denna syssla.
\par 4 Vi skola då helt få ägna oss åt bönen och åt ordets tjänst."
\par 5 Det talet behagade hela menigheten. Och de utvalde Stefanus, en man som var full av tro och helig ande, vidare Filippus och Prokorus och Nikanor och Timon och Parmenas, slutligen Nikolaus, en proselyt från Antiokia.
\par 6 Dem läto de träda fram för apostlarna, och dessa bådo och lade händerna på dem.
\par 7 Och Guds ord hade framgång, och lärjungarnas antal förökades mycket i Jerusalem; och en stor hop av prästerna blevo lydiga och trodde.
\par 8 Och Stefanus var full av nåd och kraft och gjorde stora under och tecken bland folket.
\par 9 Men av dem som hörde till den synagoga som kallades "De frigivnes och cyrenéernas och alexandrinernas synagoga", så ock av dem som voro från Cilicien och provinsen Asien, stodo några upp för att disputera med Stefanus.
\par 10 Dock förmådde de icke stå emot den vishet och den ande som här talade.
\par 11 Då skaffade de några män som föregåvo att de hade hört honom tala hädiska ord mot Moses och mot Gud.
\par 12 De uppeggade så folket och de äldste och de skriftlärde och överföllo honom och grepo honom och förde honom inför Stora rådet.
\par 13 Där läto de falska vittnen träda fram, vilka sade: "Denne man upphör icke att tala mot vår heliga plats och mot lagen.
\par 14 Ty vi hava hört honom säga att Jesus, han från Nasaret, skall bryta ned denna byggnad och förändra de stadgar som Moses har givit oss."
\par 15 Då nu alla som sutto i Rådet fäste sina ögon på honom, syntes dem hans ansikte vara såsom en ängels ansikte.

\chapter{7}

\par 1 Och översteprästen frågade: "Förhåller detta sig så?"
\par 2 Då sade han: "Bröder och fäder, hören mig. Härlighetens Gud uppenbarade sig för vår fader Abraham, medan han ännu var i Mesopotamien, och förrän han bosatte sig i Karran,
\par 3 och sade till honom: 'Gå ut ur ditt land och från din släkt, och drag till det land som jag skall visa dig.'
\par 4 Då gick han åstad ut ur kaldéernas land och bosatte sig i Karran. Sedan, efter han faders död, bjöd Gud honom att flytta därifrån till detta land, där I nu bon.
\par 5 Han gav honom ingen arvedel däri, icke ens så mycket som en fotsbredd, men lovade att giva det till besittning åt honom och åt hans säd efter honom; detta var på den tid då han ännu icke hade någon son.
\par 6 Och vad Gud sade var detta, att hans säd skulle leva såsom främlingar i ett land som icke tillhörde dem, och att man skulle göra dem till trälar och förtrycka dem i fyra hundra år.
\par 7 'Men det folk vars trälar de bliva skall jag döma', sade Gud; 'sedan skola de draga ut och hålla gudstjänst åt mig på denna plats.'
\par 8 Och han upprättade ett omskärelsens förbund med honom. Och så födde han Isak och omskar honom på åttonde dagen, och Isak födde Jakob, och Jakob födde våra tolv stamfäder.
\par 9 Och våra stamfäder avundades Josef och sålde honom till Egypten. Men Gud var med honom
\par 10 och frälste honom ur allt hans betryck och lät honom finna nåd och gav honom vishet inför Farao, konungen i Egypten; och denne satte honom till herre över Egypten och över hela sitt hus.
\par 11 Och hungersnöd kom över hela Egypten och Kanaan med stort betryck, och våra fäder kunde icke få något att äta.
\par 12 Men när Jakob fick höra att bröd fanns i Egypten, sände han våra fäder åstad dit, en första gång.
\par 13 Sedan, när de för andra gången kommo dit, blev Josef igenkänd av sina bröder, och Farao fick kunskap och Josefs släkt.
\par 14 Därefter sände Josef åstad och kallade till sig sin fader Jakob och hela sin släkt, sjuttiofem personer.
\par 15 Och Jakob for ned till Egypten; och han dog där, han såväl som våra fäder.
\par 16 Och man förde dem därifrån till Sikem och lade dem i den grav som Abraham för en summa penningar hade köpt av Emmors barn i Sikem.
\par 17 Och alltefter som tiden nalkades att det löfte skulle uppfyllas, som Gud hade givit Abraham, växte folket till och förökade sig i Egypten,
\par 18 till dess en ny konung över Egypten uppstod, en som icke visste av Josef.
\par 19 Denne konung gick listigt till väga mot vårt folk och förtryckte våra fäder och drev dem till att utsätta sina späda barn, för att dessa icke skulle bliva vid liv.
\par 20 Vid den tiden föddes Moses, och han 'var ett vackert barn' inför Gud. Under tre månader fostrades han i sin faders hus;
\par 21 sedan, när han hade blivit utsatt, lät Faraos dotter hämta honom till sig och uppfostra honom såsom sin egen son.
\par 22 Och Moses blev undervisad i all egyptiernas visdom och var mäktig i ord och gärningar.
\par 23 Men när han blev fyrtio år gammal, fick han i sinnet att besöka sina bröder, Israels barn.
\par 24 När han då såg att en av dem led orätt, tog han den misshandlade i försvar och hämnades honom, i det att han slog ihjäl egyptiern.
\par 25 Nu menade han att hans bröder skulle förstå att Gud genom honom ville bereda dem frälsning; men de förstodo det icke.
\par 26 Dagen därefter kom han åter fram till dem, där de tvistade, och ville förlika dem och sade: 'I män, I ären ju bröder; varför gören I då varandra orätt?'
\par 27 Men den som gjorde orätt mot sin landsman stötte bort honom och sade: 'Vem har satt dig till hövding och domare över oss?
\par 28 Kanske du vill döda mig, såsom du i går dödade egyptiern?'
\par 29 Vid det talet flydde Moses bort och levde sedan såsom främling i Madiams land och födde där två söner.
\par 30 Och när fyrtio är äter voro förlidna, uppenbarade sig för honom, i öknen vid berget Sinai, en ängel i en brinnande törnbuske.
\par 31 När Moses såg detta, förundrade han sig över synen; och då han gick fram för att se vad det var, hördes Herrens röst:
\par 32 'Jag är dina fäders Gud, Abrahams, Isaks och Jakobs Gud.' Då greps Moses av bävan och dristade sig icke att se dit.
\par 33 Och Herren sade till honom: 'Lös dina skor av dina fötter, ty platsen där du står är helig mark.
\par 34 Jag har nogsamt sett mitt folks betryck i Egypten, och deras suckan har jag hört, och jag har stigit ned för att rädda dem. Därför må du nu gå åstad; jag vill sända dig till Egypten.
\par 35 Denne Moses, som de hade förnekat, i det de sade: 'Vem har satt dig till hövding och domare?', honom sände Gud att vara både en hövding och en förlossare, genom ängeln som uppenbarade sig för honom i törnbusken.
\par 36 Det var han som förde ut dem, och som gjorde under och tecken i Egyptens land och i Röda havet och i öknen, under fyrtio år.
\par 37 Det var samme Moses som sade till Israels barn: 'En profet skall Gud låta uppstå åt eder, av edra bröder, en som är mig lik.'
\par 38 Det var och han, som under den tid då menigheten levde i öknen, både var hos ängeln, som talade med honom på berget Sinai, och tillika hos våra fäder; och han undfick levande ord för att giva dem åt eder.
\par 39 Men våra fäder ville icke bliva honom lydiga, utan stötte bort honom och vände sig med sina hjärtan mot Egypten
\par 40 och sade till Aron: 'Gör oss gudar, som kunna gå framför oss; ty vi veta icke vad som har vederfarits denne Moses, som förde oss ut ur Egyptens land.'
\par 41 Och de gjorde i de dagarna en kalv och buro sedan fram offer åt avguden och gladde sig över sina händers verk.
\par 42 Då vände Gud sig bort och prisgav dem till att dyrka himmelens härskara, såsom det är skrivet i Profeternas bok: 'Framburen I väl åt mig slaktoffer och spisoffer under de fyrtio åren i öknen, I av Israels hus?
\par 43 Nej, I buren Moloks tält och guden Romfas stjärna, de bilder som I hade gjort för att tillbedja. Därför skall jag låta eder föras åstad ända bortom Babylon.'
\par 44 Våra fäder hade vittnesbördets tabernakel i öknen, så inrättat, som han som talade till Moses hade förordnat att denne skulle göra det, efter den mönsterbild som han hade fått se.
\par 45 Och våra fäder togo det i arv och förde det sedan under Josua hitin, när de togo landet i besittning, efter de folk som Gud fördrev för våra fäder. Så var det ända till Davids tid.
\par 46 Denne fann nåd inför Gud och bad att han måtte finna 'ett rum till boning' åt Jakobs Gud.
\par 47 Men det var Salomo som fick bygga ett hus åt honom.
\par 48 Dock, den Högste bor icke i hus som äro gjorda med händer, ty det är såsom profeten säger:
\par 49 'Himmelen är min tron, och jorden är min fotapall; vad för ett hus skullen I då kunna bygga åt mig, säger Herren, och vad för en plats skulle tjäna mig till vilostad?
\par 50 Min hand har ju gjort allt detta.'
\par 51 I hårdnackade, med oomskurna hjärtan och öron, I stån alltid emot den helige Ande, I likaväl som edra fäder.
\par 52 Vilken av profeterna hava icke edra fäder förföljt? De hava ju dräpt dem som förkunnade att den Rättfärdige skulle komma, han som I själva nu haven förrått och dräpt,
\par 53 I som fingen lagen eder given genom änglars försorg, men icke haven hållit den."
\par 54 När de hörde detta, blevo de mycket förbittrade i sina hjärtan och beto sina tänder samman mot honom.
\par 55 Men han, full av helig ande, skådade upp mot himmelen och fick se Guds härlighet och såg Jesus stå på Guds högra sida.
\par 56 Och han sade: "Jag ser himmelen öppen och Människosonen stå på Guds högra sida."
\par 57 Då skriade de med hög röst och höllo för sina öron och stormade alla på en gång emot honom
\par 58 och förde honom ut ur staden och stenade honom. Och vittnena lade av sina mantlar vid en ung mans fötter, som hette Saulus.
\par 59 Så stenade de Stefanus, under det att han åkallade och sade: "Herre Jesus, tag emot min ande."
\par 60 Och han föll ned på sina knän och ropade med hög röst: "Herre, tillräkna dem icke denna synd." Och när han hade sagt detta, avsomnade han.

\chapter{8}

\par 1 Och jämväl Saulus hade gillat att man dödade honom. Samma dag utbröt en svår förföljelse mot församlingen i Jerusalem; och alla, utom apostlarna, blevo kringspridda över Judeens och Samariens landsbygd.
\par 2 Några fromma män begrovo dock Stefanus och höllo en stor dödsklagan efter honom.
\par 3 Saulus åter for våldsamt fram mot församlingen; han gick omkring i husen och drog fram män och kvinnor och lät sätta dem i fängelse.
\par 4 Men de som hade blivit kringspridda gingo omkring och förkunnade evangelii ord.
\par 5 Och Filippus kom så ned till huvudstaden i Samarien och predikade Kristus för folket där.
\par 6 Och när de hörde Filippus och sågo de tecken som han gjorde, aktade de endräktigt på det som han talade.
\par 7 Ty från många som voro besatta av orena andar foro andarna ut under höga rop, och många lama och ofärdiga blevo botade.
\par 8 Och det blev stor glädje i den staden.
\par 9 Nu var där i staden före honom en man vid namn Simon, som hade övat trolldom, så att han hade slagit det samaritiska folket med häpnad, och som sade sig vara något stort.
\par 10 Till honom höllo sig alla, både små och stora, och sade: "Denne är vad man kallar 'Guds stora kraft.'"
\par 11 Och de höllo sig till honom, därför att han genom sina trollkonster under ganska lång tid hade slagit dem med häpnad.
\par 12 Men nu, då de satte tro till Filippus, som förkunnade evangelium om Guds rike och om Jesu Kristi namn, läto de döpa sig, både män och kvinnor.
\par 13 Ja, Simon själv kom till tro; och sedan han hade blivit döpt, höll han sig ständigt till Filippus. Och när han såg de stora tecken och kraftgärningar som denne gjorde, betogs han av häpnad.
\par 14 Då nu apostlarna i Jerusalem fingo höra att Samarien hade tagit emot Guds ord, sände de dit Petrus och Johannes.
\par 15 Och när dessa kommo ditned, bådo de för dem, att de måtte undfå helig ande;
\par 16 ty helig ande hade ännu icke fallit på någon av dem, utan de voro allenast döpta i Herren Jesu namn.
\par 17 De lade då händerna på dem, och de undfingo helig ande.
\par 18 När då Simon såg att det var genom apostlarnas handpåläggning som Anden blev given, bjöd han dem penningar
\par 19 och sade: "Given ock mig den makten, så att var och en som jag lägger händerna på undfår helig ande."
\par 20 Då sade Petrus till honom: "Må dina penningar med dig själv gå i fördärvet, eftersom du menar att Guds gåva kan köpas för penningar.
\par 21 Du har ingen del eller lott i det som här är fråga om, ty ditt hjärta är icke rättsinnigt inför Gud.
\par 22 Gör fördenskull bättring och upphör med denna din ondska, och bed till Herren att den tanke som har uppstått i ditt hjärta må, om möjligt är, bliva dig förlåten.
\par 23 Ty jag ser att du är förgiftad av ondska och fången i orättfärdighetens bojor."
\par 24 Då svarade Simon och sade: "Bedjen I till Herren för mig, att intet av det som I haven sagt må komma över mig."
\par 25 Och sedan de hade framburit sitt vittnesbörd och talat Herrens ord, begåvo de sig tillbaka till Jerusalem och förkunnade därvid evangelium i många samaritiska byar.
\par 26 Men en Herrens ängel talade till Filippus och sade: "Stå upp och begiv dig vid middagstiden ut på den väg som leder ned från Jerusalem till Gasa; den är tom på folk."
\par 27 Då stod han upp och begav sig åstad. Och se, en etiopisk man for där fram, en hovman som var en mäktig herre hos Kandace, drottningen i Etiopien, och var satt över hela hennes skattkammare. Denne hade kommit till Jerusalem för att där tillbedja,
\par 28 men var nu stadd på hemvägen och satt i sin vagn och läste profeten Esaias.
\par 29 Då sade Anden till Filippus: "Gå fram och närma dig till denna vagn."
\par 30 Filippus skyndade fram och hörde att han läste profeten Esaias. Då frågade han: "Förstår du vad du läser?"
\par 31 Han svarade: "Huru skulle jag väl kunna förstå det, om ingen vägleder mig?" Och han bad Filippus stiga upp och sätta sig bredvid honom.
\par 32 Men det ställe i skriften som han läste var detta: "Såsom ett får fördes han bort till att slaktas; och såsom ett lamm som är tyst inför den som klipper det, så öppnade han icke sin mun.
\par 33 Genom hans förnedring blev hans dom borttagen. Vem kan räkna hans släkte? Ty hans liv ryckes undan från jorden."
\par 34 Och hovmannen frågade Filippus och sade: "Jag beder dig, säg mig om vilken profeten talar detta, om sig själv eller om någon annan?"
\par 35 Då öppnade Filippus sin mun och begynte med detta skriftens ord och förkunnade för honom evangelium om Jesus.
\par 36 Och medan de färdades vägen fram, kommo de till ett vatten. Då sade hovmannen: "Se, här finnes vatten. Vad hindrar att jag döpes?"
\par 37 Filippus sade till honom: "Om du tro av hela ditt hjärta, så kan det ske. Han svarade och sade: "Jag tror att Jesus Kristus är Guds Son."
\par 38 Och han lät vagnen stanna; och de stego båda ned i vattnet, Filippus och hovmannen, och han döpte honom.
\par 39 Men när de hade stigit upp ur vattnet, ryckte Herrens Ande bort Filippus, och hovmannen såg honom icke mer, då han nu glad fortsatte sin färd.
\par 40 Men Filippus blev efteråt sedd i Asdod. Därefter vandrade han omkring och förkunnade evangelium i alla städer, till dess han kom till Cesarea.

\chapter{9}

\par 1 Men Saulus, som alltjämt andades hot och mordlust mot Herrens lärjungar, gick till översteprästen
\par 2 och utbad sig av honom brev till synagogorna i Damaskus, för att, om han funne några som voro på "den vägen", vare sig män eller kvinnor, han skulle kunna föra dem bundna till Jerusalem.
\par 3 Men när han på sin färd nalkades Damaskus, hände sig att ett sken från himmelen plötsligt kringstrålade honom.
\par 4 Och han föll ned till jorden och hörde då en röst som sade till honom: "Saul, Saul, varför förföljer du mig?"
\par 5 Då sade han: "Vem är du, Herre?" Han svarade: "Jag är Jesus, den som du förföljer.
\par 6 Men stå nu upp och gå in i staden, så skall där bliva dig sagt vad du har att göra."
\par 7 Och männen som voro med honom på färden stodo mållösa av skräck, ty de hörde väl rösten, men sågo ingen.
\par 8 Och Saulus reste sig upp från jorden, men när han öppnade sina ögon, kunde han icke mer se något. De togo honom därför vid handen och ledde honom in i Damaskus.
\par 9 Och under tre dagar såg han intet; och han varken åt eller drack.
\par 10 Men i Damaskus fanns en lärjunge vid namn Ananias. Till honom sade Herren i en syn: "Ananias!" Han svarade: "Här är jag, Herre."
\par 11 Och Herren sade till honom: "Stå upp och gå till den gata som kallas Raka gatan och fråga i Judas' hus efter en man vid namn Saulus, från Tarsus. Ty se, han beder.
\par 12 Och i en syn har han sett huru en man vid namn Ananias kom in och lade händerna på honom, för att han skulle få sin syn igen."
\par 13 Då svarade Ananias: "Herre, jag har av många hört huru mycket ont den mannen har gjort dina heliga i Jerusalem.
\par 14 Och han har nu här med sig fullmakt ifrån översteprästerna att fängsla alla dem som åkalla ditt namn."
\par 15 Men Herren sade till honom: "Gå åstad; ty denne man är mig ett utvalt redskap till att bära fram mitt namn inför hedningar och konungar och inför Israels barn;
\par 16 och jag skall visa honom huru mycket han måste lida för mitt namns skull."
\par 17 Då gick Ananias åstad och kom in i huset; och han lade sina händer på honom och sade: "Saul, min broder, Herren har sänt mig, Jesus, som visade sig för dig på vägen där du färdades; han har sänt mig, för att du skall få din syn igen och bliva uppfylld av helig ande."
\par 18 Då var det strax såsom om fjäll föllo ifrån hans ögon, och han fick sin syn igen. Och han stod upp och lät döpa sig.
\par 19 Sedan tog han sig mat och blev därav stärkt. Därefter var han någon tid tillsammans med lärjungarna i Damaskus.
\par 20 Och strax begynte han i synagogorna predika om Jesus, att han var Guds Son.
\par 21 Och alla som hörde honom blevo uppfyllda av häpnad och sade: "Var det icke denne som i Jerusalem förgjorde dem som åkallade det namnet? Och hade han icke kommit hit, för att han skulle föra sådana människor bundna till översteprästerna?"
\par 22 Men Saulus uppträdde med allt större kraft och gjorde de judar som bodde i Damaskus svarslösa, i det han bevisade att Jesus var Messias.
\par 23 När så en längre tid hade förgått, rådslogo judarna om att röja honom ur vägen;
\par 24 men deras anslag blev bekant för Saulus. Och då de nu, för att kunna röja honom ur vägen, till och med höllo vakt vid stadsportarna både dag och natt,
\par 25 togo hans lärjungar honom en natt och släppte honom ut genom muren, i det att de sänkte ned honom i en korg.
\par 26 När han sedan kom till Jerusalem, försökte han att närma sig lärjungarna; men alla fruktade för honom, ty de trodde icke att han verkligen var en lärjunge.
\par 27 Då tog Barnabas sig an honom och förde honom till apostlarna och förtäljde för dem huru han på vägen hade sett Herren, som hade talat till honom, och huru han i Damaskus hade frimodigt predikat i Jesu namn.
\par 28 Sedan gick han fritt ut och in bland dem i Jerusalem och predikade frimodigt i Herrens namn;
\par 29 och han talade och disputerade med de grekiska judarna. Men de gjorde försök att röja honom ur vägen.
\par 30 När bröderna förnummo detta, förde de honom ned till Cesarea och sände honom därifrån vidare till Tarsus.
\par 31 Så hade nu församlingen frid i hela Judeen och Galileen och Samarien; och den blev uppbyggd och vandrade i Herrens fruktan och växte till genom den helige Andes tröst och förmaning.
\par 32 Medan nu Petrus vandrade omkring bland dem alla, hände sig att han ock kom ned till de heliga som bodde i Lydda.
\par 33 Där träffade han på en man vid namn Eneas, som i åtta år hade legat till sängs; han var nämligen lam.
\par 34 Och Petrus sade till honom: "Eneas, Jesus Kristus botar dig. Stå upp och lägg ihop din bädd." Då stod han strax upp.
\par 35 Och alla som bodde i Lydda och i Saron sågo honom; och de omvände sig till Herren.
\par 36 I Joppe bodde då en lärjunginna vid namn Tabita (det betyder detsamma som Dorkas). Hon överflödade i goda gärningar och gav allmosor rikligen.
\par 37 Men just i de dagarna hände sig att hon blev sjuk och dog. Och man tvådde henne och lade henne i en sal i övre våningen.
\par 38 Då nu Lydda låg nära Joppe och lärjungarna hade hört att Petrus var där, sände de två män till honom och bådo honom att utan dröjsmål komma till dem.
\par 39 Petrus stod då upp och följde med dem. Och när han kom dit, förde de honom upp i salen; och alla änkorna kommo där omkring honom gråtande och visade honom alla livklädnader och mantlar som Dorkas hade gjord, medan hon ännu levde ibland dem.
\par 40 Då tillsade Petrus dem allasammans att gå ut och föll ned på sina knän och bad; sedan vände han sig mot den döda och sade: "Tabita, stå upp." Då slog hon upp ögonen, och när hon fick se Petrus, satte hon sig upp.
\par 41 Och han räckte henne handen och reste upp henne och kallade sedan in de heliga, jämte änkorna, och ställde henne levande framför den.
\par 42 Detta blev bekant i hela Joppe, och många kommo till tro på Herren.
\par 43 Därefter stannade han en längre tid i Joppe hos en garvare vid namn Simon.

\chapter{10}

\par 1 I Cesarea bodde en man vid namn Kornelius, en hövitsman vid den så kallade italiska krigsskaran.
\par 2 Han var en from man, som "fruktade Gud" tillika med hela sitt hus; han utdelade rikligen allmosor åt folket och bad alltid till Gud.
\par 3 En dag omkring nionde timmen såg denne tydligt i en syn en Guds ängel, som kom in till honom och sade till honom: "Kornelius!"
\par 4 Han betraktade honom förskräckt och frågade: "Vad är det, herre?" Då sade ängeln till honom: "Dina böner och dina allmosor hava uppstigit till Gud och äro i åminnelse hos honom.
\par 5 Så sänd nu några män till Joppe och låt hämta en viss Simon, som ock kallas Petrus.
\par 6 Han gästar hos en garvare vid namn Simon, som har ett hus vid havet."
\par 7 När ängeln som hade talat med honom var borta, kallade han till sig två av sina tjänare och en from krigsman, en av dem som hörde till hans närmaste följe,
\par 8 och förtäljde alltsammans för dem och sände dem åstad till Joppe.
\par 9 Men dagen därefter, medan dessa voro på vägen och nalkades staden, gick Petrus vid sjätte timmen upp på taket för att bedja.
\par 10 Och han blev hungrig och ville hava något att äta. Medan man nu tillredde maten, föll han i hänryckning.
\par 11 Han såg himmelen öppen och någonting komma ned som liknade en stor linneduk, och som fasthölls vid de fyra hörnen och sänktes ned till jorden.
\par 12 Och däri funnos alla slags fyrfota och krälande djur som leva på jorden, och alla slags himmelens fåglar.
\par 13 Och en röst kom till honom: "Stå upp, Petrus, slakta och ät."
\par 14 Men Petrus svarade: "Bort det, Herre! Jag har aldrig ätit något oheligt och orent."
\par 15 Åter, för andra gången, kom en röst till honom: "Vad Gud har förklarat för rent, det må du icke hålla för oheligt."
\par 16 Detta skedde tre gånger efter varandra; sedan blev duken strax åter upptagen till Himmelen.
\par 17 Medan Petrus i sitt sinne undrade över vad den syn han hade sett skulle betyda, hade de män som voro utsända av Kornelius redan frågat sig fram till Simons hus; och de stannade nu vid porten
\par 18 och ropade på någon för att få veta om Simon, som ock kallades Petrus, gästade där.
\par 19 Men under det Petrus alltjämt begrundade synen, sade Anden till honom: "Här äro ett par män som fråga efter dig.
\par 20 Stå upp, och gå ditned och följ med dem, utan att tveka; ty det är jag som har sänt dem."
\par 21 Då steg Petrus ned till männen och sade: "Se här är jag, den som I frågen efter. Av vilken orsak haven I kommit hit?"
\par 22 De svarade: "Hövitsmannen Kornelius, en rättfärdig och gudfruktig man, som har gott vittnesbörd om sig av hela det judiska folket, har i en uppenbarelse fått befallning av en helig ängel att hämta dig till sig och höra vad du har att säga."
\par 23 Då bjöd han dem komma in och beredde dem härbärge. Dagen därefter stod han upp och begav sig åstad med dem; och några av bröderna i Joppe följde med honom.
\par 24 Följande dag kommo de fram till Cesarea. Och Kornelius väntade på dem och hade kallat tillhopa sina fränder och närmaste vänner.
\par 25 Då nu Petrus skulle träda in, gick Kornelius emot honom och betygade honom sin vördnad, i det att han föll ned för hans fötter.
\par 26 Men Petrus reste upp honom och sade: "Stå upp; också jag är en människa."
\par 27 Och under samtal med honom trädde Petrus in och fann många vara där församlade.
\par 28 Och han sade till dem: "I veten själva att det är en judisk man förbjudet att hava något umgänge med en utlänning eller att besöka en sådan; men mig har Gud lärt att icke räkna någon människa för ohelig eller oren.
\par 29 Därför kom jag ock utan invändning hit, när jag blev hämtad. Och nu frågar jag eder av vilket skäl I haven låtit hämta mig."
\par 30 Då svarade Kornelius: "Det är nu fjärde dagen sedan jag, just vid denna timme, förrättade i mitt hus den bön som man beder vid nionde timmen. Då fick jag se en man i skinande klädnad stå framför mig,
\par 31 och han sade: 'Kornelius, din bön är hörd, och dina allmosor hava kommit i åminnelse inför Gud.
\par 32 Så sänd nu bud till Joppe och kalla till dig Simon, som ock kallas Petrus: han gästar i garvaren Simons hus vid havet.'
\par 33 Då sände jag strax bud till dig, och du gjorde väl i att du kom. Och nu äro vi alla här tillstädes inför Gud för att höra allt som har blivit dig befallt av Herren."
\par 34 Då öppnade Petrus sin mun och sade: "Nu förnimmer jag i sanning att 'Gud icke har anseende till personen',
\par 35 utan att den som fruktar honom och övar rättfärdighet, han tages emot av honom, vilket folk han än må tillhöra.
\par 36 Det ord som han har sänt till Israels barn för att genom Jesus Kristus, som är allas Herre, förkunna det glada budskapet om frid, det ordet kännen I,
\par 37 den förkunnelse som gick ut över hela Judeen, sedan den hade begynt i Galileen efter den döpelse Johannes predikade -
\par 38 förkunnelsen om Jesus från Nasaret och om huru Gud hade smort honom med helig ande och kraft, honom som vandrade omkring och gjorde gott och botade alla som voro under djävulens våld; ty Gud var med honom.
\par 39 Vi kunna själva vittna om allt vad han gjorde både på den judiska landsbygden och i Jerusalem; likväl upphängde man honom på trä och dödade honom.
\par 40 Men honom har Gud uppväckt på tredje dagen och låtit honom bliva uppenbar,
\par 41 väl icke för allt folket, men för oss, som redan förut av Gud hade blivit utvalda till vittnen, och som åto och drucko med honom, sedan han hade uppstått från de döda.
\par 42 Och han bjöd oss predika för folket och betyga att han är den som av Gud har blivit bestämd till att vara domare över levande och döda.
\par 43 Om honom bära alla profeterna vittnesbörd och betyga att var och en som tror på honom skall få syndernas förlåtelse genom hans namn."
\par 44 Medan Petrus ännu så talade, föll den helige Ande på alla dem som hörde hans tal.
\par 45 Och alla de omskurna troende män som hade kommit dit med Petrus blevo uppfyllda av häpnad över att den helige Ande hade blivit utgjuten jämväl över hedningarna, såsom en gåva åt dem.
\par 46 De hörde dem nämligen tala tungomål och storligen prisa Gud.
\par 47 Då tog Petrus till orda och sade: "Icke kan väl någon hindra att dessa döpas med vatten, då de hava undfått den helige Ande, de likaväl som vi?"
\par 48 Och så bjöd han att man skulle döpa dem i Jesu Kristi namn. Därefter bådo de honom att han skulle stanna hos dem några dagar.

\chapter{11}

\par 1 Men apostlarna och de bröder som voro i Judeen fingo höra att också hedningarna hade tagit emot Guds ord.
\par 2 När så Petrus kom upp till Jerusalem, begynte de som voro omskurna gå till rätta med honom;
\par 3 de sade: "Du har ju besökt oomskurna män och ätit med dem."
\par 4 Då begynte Petrus från början och omtalade för dem allt i följd och ordning; han sade:
\par 5 "Jag var i staden Joppe, stadd i bön; då såg jag under hänryckning i en syn någonting komma ned, som liknade en stor linneduk, vilken fasthölls vid de fyra hörnen och sänktes ned från himmelen; och det kom ända ned till mig.
\par 6 Och jag betraktade det och gav akt därpå; då fick jag däri se fyrfota djur, sådana som leva på jorden, tama och vilda, så ock krälande djur och himmelens fåglar.
\par 7 Jag hörde ock en röst säga till mig: 'Stå upp, Petrus, slakta och ät.'
\par 8 Men jag svarade: 'Bort det, Herre! Aldrig har något oheligt eller orent kommit i min mun.'
\par 9 För andra gången talade en röst från himmelen: 'Vad Gud har förklarat för rent, det må du icke hålla för oheligt.'
\par 10 Detta skedde tre gånger efter varandra; sedan drogs alltsammans åter upp till himmelen.
\par 11 Och i detsamma kommo tre män, som hade blivit sända till mig från Cesarea, och stannade framför huset där vi voro.
\par 12 Och Anden sade till mig att jag skulle följa med dem, utan att göra någon åtskillnad mellan folk och folk. Också de sex bröder som äro här kommo med mig; och vi gingo in i mannens hus.
\par 13 Och han berättade för oss huru han hade sett ängeln träda in i hans hus, och att denne hade sagt: 'Sänd åstad till Joppe och låt hämta Simon, som ock kallas Petrus.
\par 14 Han skall tala till dig ord genom vilka du skall bliva frälst, du själv och hela ditt hus.'
\par 15 Och när jag hade begynt tala, föll den helige Ande på dem, alldeles såsom det under den första tiden skedde med oss.
\par 16 Då kom jag ihåg Herrens ord, huru han hade sagt: 'Johannes döpte med vatten, men I skolen bliva döpta i helig ande.'
\par 17 Då alltså Gud åt dem hade givit samma gåva som åt oss, som hava kommit till tro på Herren Jesus Kristus, huru skulle då jag hava kunnat sätta mig emot Gud?"
\par 18 När de hade hört detta, gåvo de sig till freds och prisade Gud och sade: "Så har då Gud också åt hedningarna förlänat den bättring som för till liv."
\par 19 De som hade blivit kringspridda genom den förföljelse som utbröt för Stefanus' skull drogo emellertid omkring ända till Fenicien och Cypern och Antiokia, men förkunnade icke ordet för andra än för judar.
\par 20 Dock funnos bland dem några män från Cypern och Cyrene, som när de kommo till Antiokia, också talade till grekerna och för dem förkunnade evangelium om Herren Jesus.
\par 21 Och Herrens hand var med dem, och en stor skara kom till tro och omvände sig till Herren.
\par 22 Ryktet härom nådde församlingen i Jerusalem; och de sände då Barnabas till Antiokia.
\par 23 När han kom dit och fick se vad Guds nåd hade verkat, blev han glad och förmanade dem alla att med hjärtats fasta föresats stadigt hålla sig till Herren.
\par 24 Ty han var en god man och full av helig ande och tro. Och ganska mycket folk blev ytterligare fört till Herren.
\par 25 Sedan begav han sig åstad till Tarsus för att uppsöka Saulus.
\par 26 Och när han hade träffat honom, tog han honom med sig till Antiokia. Ett helt år hade de sedan sin umgängelse inom församlingen och undervisade ganska mycket folk. Och det var i Antiokia som lärjungarna först begynte kallas "kristna".
\par 27 Vid den tiden kommo några profeter från Jerusalem ned till Antiokia.
\par 28 Och en av dem, vid namn Agabus, trädde upp och gav genom Andens ingivelse till känna att en stor hungersnöd skulle komma över hela världen; den kom också på Klaudius' tid.
\par 29 Då bestämde lärjungarna att de, var och en efter sin förmåga, skulle sända något till understöd åt de bröder som bodde i Judeen.
\par 30 Detta gjorde de också, och genom Barnabas och Saulus översände de det till de äldste.

\chapter{12}

\par 1 Vid den tiden lät konung Herodes gripa och misshandla några av dem som hörde till församlingen.
\par 2 Och Jakob, Johannes' broder, lät han avrätta med svärd.
\par 3 När han såg att detta behagade judarna, fortsatte han och lät fasttaga också Petrus. Detta skedde under det osyrade brödets högtid.
\par 4 Och sedan han hade gripit honom, satte han honom i fängelse och uppdrog åt fyra vaktavdelningar krigsmän, vardera på fyra man, att bevaka honom; och hans avsikt var att efter påsken ställa honom fram inför folket.
\par 5 Under tiden förvarades Petrus i fängelset, men församlingen bad enträget till Gud för honom.
\par 6 Natten före den dag då Herodes tänkte draga honom inför rätta låg Petrus och sov mellan två krigsmän, fängslad med två kedjor; och utanför dörren voro väktare utsatta till att bevaka fängelset.
\par 7 Då stod plötsligt en Herrens ängel där, och ett sken lyste i rummet. Och han stötte Petrus i sidan och väckte honom och sade: "Stå nu strax upp"; och kedjorna föllo ifrån hans händer.
\par 8 Ängeln sade ytterligare till honom: "Omgjorda dig, och tag på dig dina sandaler." Och han gjorde så. därefter sade ängeln till honom: "Tag din mantel på dig och följ mig."
\par 9 Och Petrus gick ut och följde honom; men han förstod icke att det som skedde genom ängeln var något verkligt, utan trodde att det var en syn han såg.
\par 10 När de så hade gått genom första och andra vakten, kommo de till järnporten som ledde ut till staden. Den öppnade sig för dem av sig själv, och de trädde ut och gingo en gata fram; och i detsamma försvann ängeln ifrån honom.
\par 11 När sedan Petrus kom till sig igen, sade han: "Nu vet jag och är förvissad om att Herren har utsänt sin ängel och räddat mig ur Herodes' hand och undan allt det som det judiska folket hade väntat sig."
\par 12 När han alltså hade förstått huru det var, gick han till det hus där Maria bodde, hon som var moder till den Johannes som ock kallades Markus; där voro ganska många församlade och bådo.
\par 13 Då han nu klappade på portdörren, kom en tjänsteflicka, vid namn Rode, för att höra vem det var.
\par 14 Och när hon kände igen Petrus' röst, öppnade hon i sin glädje icke porten, utan skyndade in och berättade att Petrus stod utanför porten.
\par 15 Då sade de till henne: "Du är från dina sinnen." Men hon bedyrade att det var såsom hon hade sagt. Då sade de: "Det är väl hans ängel."
\par 16 Men Petrus fortfor att klappa; och när de öppnade, sågo de med häpnad att det var han.
\par 17 Och han gav tecken åt dem med handen att de skulle tiga, och förtäljde för dem huru Herren hade fört honom ut ur fängelset. Och han tillade: "Låten Jakob och de andra bröderna få veta detta." Sedan gick han därifrån och begav sig till en annan ort.
\par 18 Men när det hade blivit dag, uppstod bland krigsmännen en ganska stor oro och undran över vad som hade blivit av Petrus.
\par 19 När så Herodes ville hämta honom, men icke fann honom, anställde han rannsakning med väktarna och bjöd att de skulle föras bort till bestraffning. Därefter for han ned från Judeen till Cesarea och vistades sedan där.
\par 20 Men han hade fattat stor ovilja mot tyrierna och sidonierna. Dessa infunno sig nu gemensamt hos honom; och sedan de hade fått Blastus, konungens kammarherre, på sin sida, bådo de om fred, ty deras land hade sin näring av konungens.
\par 21 På utsatt dag klädde sig då Herodes i konungslig skrud och satte sig på tronen och höll ett tal till dem.
\par 22 Då ropade folket: "En guds röst är detta, och icke en människas."
\par 23 Men i detsamma slog honom en Herrens ängel, därför att han icke gav Gud äran. Och han föll i en sjukdom som bestod däri att han uppfrättes av maskar, och så gav han upp andan.
\par 24 Men Guds ord hade framgång och utbredde sig.
\par 25 Och sedan Barnabas och Saulus hade fullgjort sitt uppdrag i Jerusalem och avlämnat understödet, vände de tillbaka och togo då med sig Johannes, som ock kallades Markus.

\chapter{13}

\par 1 I den församling som fanns i Antiokia verkade nu såsom profeter och lärare Barnabas och Simeon, som kallades Niger, och Lucius från Cyrene, så ock Manaen, landsfursten Herodes' fosterbroder, och Saulus.
\par 2 När dessa förrättade Herrens tjänst och fastade, sade den helige Ande: "Avskiljen åt mig Barnabas och Saulus för det verk som jag har kallat dem till."
\par 3 Då fastade de och bådo och lade händerna på dem och läto dem begiva sig åstad.
\par 4 Dessa, som så hade blivit utsända av den helige Ande, foro nu ned till Seleucia och seglade därifrån till Cypern.
\par 5 Och när de hade kommit till Salamis, förkunnade de Guds ord i judarnas synagogor. De hade också med sig Johannes såsom tjänare.
\par 6 Och sedan de hade färdats över hela ön ända till Pafos, träffade de där på en judisk trollkarl och falsk profet, vid namn Barjesus,
\par 7 som vistades hos landshövdingen Sergius Paulus. Denne var en förståndig man. Han kallade till sig Barnabas och Saulus och begärde att få höra Guds ord.
\par 8 Men Elymas (eller trollkarlen, ty namnet har den betydelsen) stod emot dem och ville hindra landshövdingen från att komma till tro.
\par 9 Saulus, som ock kallades Paulus, uppfylldes då av helig ande och fäste ögonen på honom
\par 10 och sade: "O du djävulens barn, du som är full av allt slags svek och arglistighet och en fiende till allt vad rätt är, skall du då icke upphöra att förvrida Herrens raka vägar?
\par 11 Se, nu kommer Herrens hand över dig, och du skall till en tid bliva blind och icke kunna se solen." Och strax föll töcken och mörker över honom; och han gick omkring och sökte efter någon som kunde leda honom.
\par 12 När då landshövdingen såg vad som hade skett, häpnade han över Herrens lära och kom till tro.
\par 13 Paulus och hans följeslagare lade sedan ut ifrån Pafos och foro till Perge i Pamfylien. Där skilde sig Johannes ifrån dem och vände tillbaka till Jerusalem.
\par 14 Men själva foro de vidare från Perge och kommo till Antiokia i Pisidien. Och på sabbatsdagen gingo de in i synagogan och satte sig där.
\par 15 Och sedan man hade föreläst ur lagen och profeterna, sände synagogföreståndarna till dem och läto säga: "Bröder, haven I något förmaningens ord att säga till folket, så sägen det."
\par 16 Då stod Paulus upp och gav tecken med handen och sade: "I män av Israels hus och I som 'frukten Gud', hören mig.
\par 17 Detta folks, Israels, Gud utvalde våra fäder, och han upphöjde detta folk, medan de ännu bodde såsom främlingar i Egyptens land, och förde dem sedan ut därifrån 'med upplyft arm'.
\par 18 Och under en tid av vid pass fyrtio år hade han fördrag med dem i öknen.
\par 19 Och sedan han hade utrotat sju folk i Kanaans land, utskiftade han dessas land till arvedelar åt dem.
\par 20 Därunder förgick en tid av vid pass fyra hundra femtio år. Sedan gav han dem domare, ända till profeten Samuels tid.
\par 21 Därefter begärde de en konung; och Gud gav dem Saul, Kis' son, en man av Benjamins stam, för en tid av fyrtio år.
\par 22 Men denne avsatte han och gjorde David till konung över dem. Honom gav han ock sitt vittnesbörd, i det han sade: 'Jag har funnit David, Jessais son, en man efter mitt hjärta. Han skall i alla stycken göra min vilja.'
\par 23 Av dennes säd har Gud efter sitt löfte låtit Jesus komma, såsom Frälsare åt Israel.
\par 24 Men redan innan han uppträdde, hade Johannes predikat bättringens döpelse för hela Israels folk.
\par 25 Och när Johannes höll på att fullborda sitt lopp, sade han: 'Vad I menen mig vara, det är jag icke. Men se, efter mig kommer den vilkens skor jag icke är värdig att lösa av han fötter.'
\par 26 Mina bröder, I som ären barn av Abrahams släkt, så ock I andra här, I som 'frukten Gud', till oss har ordet om denna frälsning blivit sänt.
\par 27 Ty eftersom Jerusalems invånare och deras rådsherrar icke kände honom, uppfyllde de ock genom sin dom över honom profeternas utsagor, vilka var sabbat föreläses;
\par 28 och fastän de icke funno honom skyldig till något som förtjänade döden, bådo de likväl Pilatus att han skulle låta döda honom.
\par 29 När de så hade fört till fullbordan allt som var skrivet om honom, togo de honom ned från korsets trä och lade honom i en grav.
\par 30 Men Gud uppväckte honom från de döda.
\par 31 Sedan visade han sig under många dagar för dem som med honom hade gått upp från Galileen till Jerusalem, och som nu äro hans vittnen inför folket.
\par 32 Och vi förkunna för eder det glada budskapet, att det löfte som gavs åt våra fäder, det har Gud låtit gå i fullbordan för oss, deras barn, därigenom att han har låtit Jesus uppstå,
\par 33 såsom ock är skrivet i andra psalmen: 'Du är min Son, jag har i dag fött dig.'
\par 34 Och att han har låtit honom uppstå från de döda, så att han icke mer skall vända tillbaka till förgängelsen, det har han sagt med dessa ord: 'Jag skall uppfylla åt eder de heliga löften som jag i trofasthet har givit åt David.'
\par 35 Därför säger han ock i en annan psalm: 'Du skall icke låta din Helige se förgängelsen.'
\par 36 När David i sin tid hade tjänat Guds vilja, avsomnade han ju och blev samlad till sina fäder och såg förgängelsen;
\par 37 men den som Gud har uppväckt, han har icke sett förgängelsen.
\par 38 Så mån I nu veta, mena bröder, att genom honom syndernas förlåtelse förkunnas för eder,
\par 39 och att i honom var och en som tror bliver rättfärdig och friad ifrån allt det varifrån I icke under Moses' lag kunden bliva friade.
\par 40 Sen därför till, att över eder icke må komma det som är sagt hos profeterna:
\par 41 'Sen här, I föraktare, och förundren eder, och bliven till intet; ty en gärning utför jag i edra dagar, en gärning som I alls icke skullen tro, om den förtäljdes för eder.'"
\par 42 När de sedan gingo därifrån, bad men dem att de nästa sabbat skulle tala för dem om samma sak.
\par 43 Och när församlingen åtskildes, följde många judar och gudfruktiga proselyter med Paulus och Barnabas. Dessa talade då till dem och förmanade dem att stadigt hålla sig till Guds nåd.
\par 44 Följande sabbat kom nästan hela staden tillsammans för att höra Guds ord.
\par 45 Då nu judarna sågo det myckna folket, uppfylldes de av nitälskan och foro ut i smädelser och motsade det som Paulus talade.
\par 46 Då togo Paulus och Barnabas mod till sig och sade: "Guds ord måste i första rummet förkunnas för eder. Men eftersom I stöten det bort ifrån eder och icke akten eder själva värdiga det eviga livet, så vända vi oss nu till hedningarna.
\par 47 Ty så har Herren bjudit oss: 'Jag har satt dig till ett ljus för hednafolken, för att du skall bliva till frälsning intill jordens ända.'"
\par 48 När hedningarna hörde detta, blevo de glada och prisade Herrens ord; och de kommo till tro, så många det var beskärt att få evigt liv.
\par 49 Och Herrens ord utbredde sig över hela landet.
\par 50 Men judarna uppeggade de ansedda kvinnor som "fruktade Gud", så ock de förnämsta männen i staden, och uppväckte en förföljelse mot Paulus och Barnabas och drevo dem bort ifrån sin stads område.
\par 51 Dessa skuddade då stoftet av sina fötter mot dem och begåvo sig till Ikonium.
\par 52 Och lärjungarna uppfylldes alltmer av glädje och helig ande.

\chapter{14}

\par 1 På samma sätt tillgick det i Ikonium: de gingo in i judarnas synagoga och talade så, att en stor hop av både judar och greker kommo till tro;
\par 2 men de judar som voro ohörsamma retade upp hedningarna och väckte deras förbittring mot bröderna.
\par 3 Så vistades de där en längre tid och predikade frimodigt, i förtröstan på Herren, och han gav vittnesbörd åt sitt nådesord, i det att han lät tecken och under ske genom dem.
\par 4 Men folket i staden delade sig, så att somliga höllo med judarna, andra åter med apostlarna.
\par 5 Och när sedan, både ibland hedningar och ibland judar med deras föreståndare, en storm hade blivit uppväckt emot dem, och man ville misshandla och stena dem,
\par 6 flydde de, så snart de förstodo huru det var, till städerna Lystra och Derbe i Lykaonien och till trakten omkring dem.
\par 7 Och där förkunnade de evangelium.
\par 8 I Lystra fanns nu en man som satt där oförmögen att bruka sina fötter, ty allt ifrån sin moders liv hade han varit ofärdig och hade aldrig kunnat gå.
\par 9 Denne hörde på, när Paulus talade. Och då Paulus fäste sina ögon på honom och såg att han hade tro, så att han kunde bliva botad,
\par 10 sade han med hög röst: "Res dig upp och stå på dina fötter." Då sprang mannen upp och begynte gå.
\par 11 När folket såg vad Paulus hade gjort, hovo de upp sin röst och ropade på lykaoniskt tungomål: "Gudarna hava stigit ned till oss i människogestalt."
\par 12 Och de kallade Barnabas för Jupiter, men Paulus kallade de för Merkurius, eftersom det var han som förde ordet.
\par 13 Och prästen vid det Jupiterstempel som låg utanför staden förde fram tjurar och kransar till portarna och ville jämte folket anställa ett offer.
\par 14 Men när apostlarna, Barnabas och Paulus, fingo höra detta, revo de sönder sina kläder och sprungo ut bland folket och ropade
\par 15 och sade: "I män, vad är det I gören? Också vi äro människor, av samma natur som I, och vi förkunna för eder evangelium, att I måsten omvända eder från dessa fåfängliga avgudar till den levande Guden, 'som har gjort himmelen och jorden och havet och allt vad i dem är'.
\par 16 Han har under framfarna släktens tider tillstatt alla hedningar att gå sina egna vägar.
\par 17 Dock har han icke låtit sig vara utan vittnesbörd, ty han har bevisat eder välgärningar, i det han har givit eder regn och fruktbara tider från himmelen och så vederkvickt edra hjärtan med mat och glädje."
\par 18 Genom sådana ord stillade de med knapp nöd folket, så att man icke offrade åt dem.
\par 19 Men några judar kommo dit från Antiokia och Ikonium. Dessa drogo folket över på sin sida och stenade Paulus och släpade honom ut ur staden, i tanke att han var död.
\par 20 Men sedan lärjungarna hade samlat sig omkring honom, reste han sig upp och gick in i staden. Dagen därefter begav han sig med Barnabas åstad därifrån till Derbe.
\par 21 Och de förkunnade evangelium i den staden och vunno ganska många lärjungar. Sedan vände de tillbaka till Lystra och Ikonium och Antiokia
\par 22 och styrkte lärjungarnas själar, i det de förmanade dem att stå fasta i tron och sade dem, att det är genom mycken bedrövelse som vi måste ingå i Guds rike.
\par 23 Därefter utvalde de åt dem "äldste" för var särskild församling och anbefallde dem efter bön och fastor åt Herren, som de nu trodde på.
\par 24 Sedan färdades de vidare genom Pisidien och kommo till Pamfylien.
\par 25 Där förkunnade de ordet i Perge och foro sedan ned till Attalia.
\par 26 Därifrån avseglade de till Antiokia, samma ort varifrån de hade blivit utsända, sedan man hade anbefallt dem åt Guds nåd, för det verk som de nu hade fullbordat.
\par 27 Och när de hade kommit dit, kallade de tillhopa församlingen och omtalade för dem huru stora ting Gud hade gjort med dem, och huru han för hedningarna hade öppnat en dörr till tro.
\par 28 Sedan vistades de där hos lärjungarna en ganska lång tid.

\chapter{15}

\par 1 Men från Judeen kommo några män ditned och lärde bröderna så: "Om I icke låten omskära eder, såsom Moses har stadgat, så kunnen I icke bliva frälsta."
\par 2 Då uppstod söndring, och Paulus och Barnabas kommo i ett ganska skarpt ordskifte med dem. Det bestämdes därför, att Paulus och Barnabas och några andra av dem skulle, för denna tvistefrågas skull, fara upp till apostlarna och de äldste i Jerusalem.
\par 3 Och församlingen utrustade dem för resan, och de foro genom Fenicien och Samarien och förtäljde utförligt om hedningarnas omvändelse och gjorde därmed alla bröderna stor glädje.
\par 4 När de sedan kommo fram till Jerusalem, mottogos de av församlingen och av apostlarna och de äldste och omtalade huru stora ting Gud hade gjort med dem.
\par 5 Men några ifrån fariséernas parti, vilka hade kommit till tro, stodo upp och sade att man borde omskära dem och bjuda dem att hålla Moses' lag.
\par 6 Då trädde apostlarna och de äldste tillsammans för att överlägga om denna sak.
\par 7 Och sedan man länge hade förhandlat därom, stod Petrus upp och sade till dem: "Mina bröder, I veten själva att Gud, för lång tid sedan, bland eder utvalde mig att vara den genom vilkens mun hedningarna skulle få höra evangelii ord och komma till tro.
\par 8 Och Gud, som känner allas hjärtan, gav dem sitt vittnesbörd, därigenom att han lät dem, likaväl som oss, undfå den helige Ande.
\par 9 Och han gjorde ingen åtskillnad mellan oss och dem, i det att han genom tron renade deras hjärtan.
\par 10 Varför fresten I då nu Gud, genom att på lärjungarnas hals vilja lägga ett ok som varken våra fäder eller vi hava förmått bära?
\par 11 Vi tro ju fastmer att det är genom Herren Jesu nåd som vi bliva frälsta, vi likaväl som de."
\par 12 Då teg hela menigheten, och man hörde på Barnabas och Paulus, som förtäljde om huru stora tecken och under Gud genom dem hade gjort bland hedningarna.
\par 13 När de hade slutat att tala, tog Jakob till orda och sade: "Mina bröder, hören mig.
\par 14 Simeon har förtäljt huru Gud först så skickade, att han bland hedningarna fick ett folk som kunde kallas efter hans namn.
\par 15 Därmed stämmer ock överens vad profeterna hava talat; ty så är skrivet:
\par 16 'Därefter skall jag komma tillbaka och åter bygga upp Davids förfallna hydda; ja, dess ruiner skall jag bygga upp och så upprätta den igen,
\par 17 för att ock övriga människor skola söka Herren, alla hedningar som hava uppkallats efter mitt namn. Så säger Herren, han som skall göra detta,
\par 18 såsom han ock har vetat det förut av evighet.'
\par 19 Därför är min mening att man icke bör betunga sådana som hava varit hedningar, men omvänt sig till Gud,
\par 20 utan allenast skriva till dem att de skola avhålla sig från avgudastyggelser och från otukt och från köttet av förkvävda djur och från blod.
\par 21 Ty Moses har av ålder sina förkunnare i alla städer, då han ju var sabbat föreläses i synagogorna."
\par 22 Därefter beslöto apostlarna och de äldste, tillika med hela församlingen, att bland sig utvälja några män, som jämte Paulus och Barnabas skulle sändas till Antiokia; och de valde Judas, som kallades Barsabbas, och Silas, vilka bland bröderna voro ledande män.
\par 23 Och man översände genom dem följande skrivelse: "Apostlarna och de äldste, edra bröder, hälsa eder, I bröder av hednisk börd, som bon i Antiokia, Syrien och Cilicien.
\par 24 Alldenstund vi hava hört att några som hava kommit från oss hava förvirrat eder med sitt tal och väckt oro i edra själar, utan att de hava haft något uppdrag av oss,
\par 25 så hava vi enhälligt kommit till det beslutet att utvälja några män, som vi skulle sända till eder jämte Barnabas och Paulus, våra älskade bröder,
\par 26 vilka hava vågat sina liv för vår Herres, Jesu Kristi, namns skull.
\par 27 Alltså sända vi nu Judas och Silas, vilka ock muntligen skola kungöra detsamma för eder.
\par 28 Den helige Ande och vi hava nämligen beslutit att icke pålägga eder någon ytterligare börda, utöver följande nödvändiga föreskrifter:
\par 29 att I skolen avhålla eder från avgudaofferskött och från blod och från köttet av förkvävda djur och från otukt. Om I noga tagen eder till vara för detta, så skall det gå eder väl. Faren väl."
\par 30 De fingo så begiva sig åstad och kommo ned till Antiokia. Där kallade de tillsammans menigheten och lämnade fram brevet.
\par 31 Och när menigheten läste detta, blevo de glada över det hugnesamma budskapet.
\par 32 Judas och Silas, som själva voro profeter, talade därefter många förmaningens ord till bröderna och styrkte dem.
\par 33 Och sedan de hade uppehållit sig där någon tid, fingo de i frid fara ifrån bröderna tillbaka till dem som hade sänt dem.
\par 34 Men silas beslöt att stanna där och Judas avreste ensam.
\par 35 Men Paulus och Barnabas vistades fortfarande i Antiokia, där de undervisade och, jämte många andra förkunnade evangelii ord från Herren.
\par 36 Efter någon tid sade Paulus till Barnabas: "Låt oss nu fara tillbaka och besöka våra bröder, i alla de städer där vi hava förkunnat Herrens ord, och se till, huru det är med dem."
\par 37 Barnabas ville då att de skulle taga med sig Johannes, som ock kallades Markus.
\par 38 Men Paulus fann icke skäligt att taga med sig en man som hade övergivit dem i Pamfylien och icke följt med dem till deras arbete.
\par 39 Och så skarp blev deras tvist att de skilde sig ifrån varandra; och Barnabas tog med sig Markus och avseglade till Cypern.
\par 40 Men Paulus utvalde åt sig Silas; och sedan han av bröderna hade blivit anbefalld åt Herrens nåd, begav han sig åstad
\par 41 och färdades genom Syrien och Cilicien och styrkte församlingarna.

\chapter{16}

\par 1 Han kom då också till Derbe och till Lystra. Där fanns en lärjunge vid namn Timoteus, som var son av en troende judisk kvinna och en grekisk fader,
\par 2 och som hade gott vittnesbörd om sig av bröderna i Lystra och Ikonium.
\par 3 Paulus ville nu att denne skulle fara med honom. För de judars skull som bodde i dessa trakter tog han honom därför till sig och omskar honom, ty alla visste att hans fader var grek.
\par 4 Och när de sedan foro genom städerna, meddelade de församlingarna till efterföljd de stadgar som voro fastställda av apostlarna och de äldste i Jerusalem.
\par 5 Så styrktes nu församlingarna i tron, och brödernas antal förökades för var dag.
\par 6 Sedan togo de vägen genom Frygien och det galatiska landet; de förhindrades nämligen av den helige Ande att förkunna ordet i provinsen Asien.
\par 7 Och när de hade kommit fram emot Mysien, försökte de att fara in i Bitynien, men Jesu Ande tillstadde dem det icke.
\par 8 Då begåvo de sig över Mysien ned till Troas.
\par 9 Här visade sig för Paulus i en syn om natten en macedonisk man, som stod där och bad honom och sade: "Far över till Macedonien och hjälp oss."
\par 10 När han hade sett denna syn, sökte vi strax någon lägenhet att fara därifrån till Macedonien, ty vi förstodo nu att Gud hade kallat oss att förkunna evangelium för dem.
\par 11 Vi lade alltså ut från Troas och foro raka vägen till Samotrace och dagen därefter till Neapolis
\par 12 och sedan därifrån till Filippi. Denna stad, en romersk koloni, är den första i denna del av Macedonien. I den staden vistades vi någon tid.
\par 13 På sabbatsdagen gingo vi utom stadsporten, längs med en flod, till en plats som gällde såsom böneställe. Där satte vi oss ned och talade med de kvinnor som hade samlats dit.
\par 14 Och en kvinna som "fruktade Gud", en purpurkrämerska från staden Tyatira, vid namn Lydia, lyssnade till samtalet; och Herren öppnade hennes hjärta, så att hon aktade på det som Paulus talade.
\par 15 Och sedan hon jämte sitt husfolk hade låtit döpa sig, bad hon oss och sade: "Eftersom I ansen mig vara en kvinna som tror på Herren, så kommen in i mitt hus och stannen där." Och hon nödgade oss därtill.
\par 16 Och det hände sig en gång, då vi gingo ned till bönestället, att vi mötte en tjänsteflicka, som hade en spådomsande i sig och genom sina spådomar skaffade sina herrar mycken inkomst.
\par 17 Denne följde efter Paulus och oss andra och ropade och sade: "Dessa män äro Guds, den Högstes, tjänare, och de förkunna för eder frälsningens väg."
\par 18 Så gjorde hon under många dagar. Men Paulus tog illa vid sig och vände sig om och sade till anden: "I Jesu Kristi namn bjuder jag dig att fara ut ur henne." Och anden for ut i samma stund.
\par 19 Men när hennes herrar sågo att det för dem var slut med allt hopp om vidare inkomst, grepo de Paulus och Silas och släpade dem till torget inför överhetspersonerna.
\par 20 Och sedan de hade fört dem tid fram, till domarna, sade de: "Dessa män uppväcka stor oro i vår stad; de äro judar
\par 21 och vilja införa stadgar som det för oss, såsom romerska medborgare, icke är lovligt att antaga eller hålla."
\par 22 Också folket reste sig upp emot dem, och domarna läto slita av dem deras kläder och bjödo att man skulle piska dem med spön.
\par 23 Och sedan de hade låtit giva dem många slag, kastade de dem i fängelse och bjödo fångvaktaren att hålla dem i säkert förvar.
\par 24 Då denne fick en så sträng befallning, satte han in dem i det innersta fängelserummet och fastgjorde deras fötter i stocken.
\par 25 Vid midnattstiden voro Paulus och Silas stadda i bön och lovade Gud med sång, och de andra fångarna hörde på dem.
\par 26 Då kom plötsligt en stark jordstöt, så att fängelsets grundvalar skakades; och i detsamma öppnades alla dörrar, och allas bojor löstes.
\par 27 Då vaknade fångvaktaren; och när han fick se fängelsets dörrar öppna, drog han sitt svärd och ville döda sig själv, i tanke att fångarna hade kommit undan.
\par 28 Men Paulus ropade med hög röst och sade: "Gör dig intet ont; ty vi äro alla här."
\par 29 Då lät han hämta ljus och sprang in och föll ned för Paulus och Silas, bävande.
\par 30 Därefter förde han ut dem och sade: "I herrar, vad skall jag göra för att bliva frälst?"
\par 31 De svarade: "Tro på Herren Jesus, så bliver du med ditt hus frälst."
\par 32 Och de förkunnade Guds ord för honom och för alla dem som voro i hans hus.
\par 33 Och redan under samma timme på natten tog han dem till sig och tvådde deras sår och lät strax döpa sig med allt sitt husfolk.
\par 34 Och han förde dem upp i sitt hus och dukade ett bord åt dem och fröjdade sig över att han med allt sitt hus hade kommit till tro på Gud.
\par 35 Men när det hade blivit dag, sände domarna åstad rättstjänarna och läto säga: "Släpp ut männen."
\par 36 Fångvaktaren underrättade då Paulus härom och sade: "Domarna hava sänt bud att I skolen släppas ut. Gån därför nu eder väg i frid."
\par 37 Men Paulus sade till dem: "De hava offentligen låtit gissla oss, utan dom och rannsakning, oss som äro romerska medborgare, och hava kastat oss i fängelse; nu vilja de också i tysthet släppa oss ut! Nej, icke så; de måste själva komma och taga oss ut."
\par 38 Rättstjänarna inberättade detta för domarna. När dessa hörde att de voro romerska medborgare, blevo de förskräckta.
\par 39 Och de gingo dit och talade goda ord till dem och togo dem ut och bådo dem lämna staden.
\par 40 När de så hade kommit ut ur fängelset, begåvo de sig hem till Lydia. Och sedan de där hade träffat bröderna och talat förmaningens ord till dem, drogo de vidare.

\chapter{17}

\par 1 Och de foro över Amfipolis och Apollonia och kommo så till Tessalonika. Där hade judarna en synagoga;
\par 2 i den gick Paulus in, såsom hans sed var. Och under tre sabbater talade han där med dem, i det han utgick ifrån skrifterna
\par 3 och utlade dem och bevisade att Messias måste lida och uppstå från de döda; och han sade: "Denne Jesus som jag förkunnar för eder är Messias."
\par 4 Och några av dem läto övertyga sig och slöto sig till Paulus och Silas; så gjorde ock en stor hop greker som "fruktade Gud", likaså ganska många av de förnämsta kvinnorna.
\par 5 Då grepos judarna av nitälskan och togo med sig allahanda dåligt folk ifrån gatan och ställde till folkskockning och oroligheter i staden och trängde fram mot Jasons hus och ville draga dem ut inför folket.
\par 6 Men när de icke funno dem, släpade de Jason och några av bröderna inför stadens styresmän och ropade: "Dessa män, som hava uppviglat hela världen, hava nu också kommit hit;
\par 7 och Jason har tagit emot dem i sitt hus. De göra alla tvärtemot kejsarens påbud och säga att en annan, en som heter Jesus, är konung.
\par 8 Så väckte de oro bland folket och hos stadens styresmän, när de hörde detta.
\par 9 Dessa läto då Jason och de andra ställa borgen för sig och släppte dem därefter lösa.
\par 10 Men strax om natten blevo Paulus och Silas av bröderna sända åstad till Berea. Och när de hade kommit dit, gingo de till judarnas synagoga.
\par 11 Dessa voro ädlare till sinnes än judarna i Tessalonika; de togo emot ordet med all villighet och rannsakade var dag skrifterna, för att se om det förhölle sig såsom nu sades.
\par 12 Många av dem kommo därigenom till tro, likaså ganska många ansedda grekiska kvinnor och jämväl män.
\par 13 Men när judarna i Tessalonika fingo veta att Guds ord förkunnades av Paulus också i Berea, kommo de dit och uppviglade också där folket och väckte oro bland dem.
\par 14 Strax sände då bröderna Paulus åstad ända ned till havet, men både Silas och Timoteus stannade kvar på platsen.
\par 15 De som ledsagade Paulus förde honom vidare till Aten och foro så därifrån tillbaka, med bud till Silas och Timoteus att dessa med det snaraste skulle komma till honom.
\par 16 Men, Paulus nu väntade på dem i Aten, upprördes han i sin ande, när han såg huru uppfylld staden var med avgudabilder.
\par 17 Han höll därför i synagogan samtal med judarna och med dem som "fruktade Gud", så ock på torget, var dag, med dem som han träffade där
\par 18 Också några filosofer, dels av epikuréernas skola, dels av stoikernas, gåvo sig i ordskifte med honom. Och somliga sade: "Vad kan väl denne pratmakare vilja säga?" Andra åter: "Han tyckes vara en förkunnare av främmande gudar." De evangelium om Jesus och om uppståndelsen.
\par 19 Och de grepo honom och förde honom till Areopagen och sade: "Kunna vi få veta vad det är för en ny lära som du förkunnar?
\par 20 Ty det är förunderliga ting som du talar oss i öronen. Vi vilja nu veta vad detta skall betyda."
\par 21 Det var nämligen så med alla atenare, likasom ock med de främlingar som hade bosatt sig bland dem, att de icke hade tid och håg för annat än att tala om eller höra på något nytt för dagen.
\par 22 Då trädde Paulus fram mitt på Areopagen och sade: "Atenare, jag ser av allting att I ären mycket ivriga gudsdyrkare.
\par 23 Ty medan jag har gått omkring och betraktat edra helgedomar, har jag ock funnit ett altare med den inskriften: 'Åt en okänd Gud.' Om just detta väsende, som I sålunda dyrken utan att känna det, är det jag nu kommer med budskap till eder.
\par 24 Den Gud som har gjort världen och allt vad däri är, han som är Herre över himmel och jord, han bor icke i tempel som äro gjorda med händer,
\par 25 ej heller låter han betjäna sig av människohänder, såsom vore han i behov av något, han som själv åt alla giver liv, anda och allt.
\par 26 Och han har skapat människosläktets alla folk, alla från en enda stamfader, till att bosätta sig utöver hela jorden; och han har fastställt för dem bestämda tider och utstakat de gränser inom vilka de skola bo -
\par 27 detta för att de skola söka Gud, om de till äventyrs skulle kunna treva sig fram till honom och finna honom; fastän han ju icke är långt ifrån någon enda av oss.
\par 28 Ty i honom är det som vi leva och röra oss och äro till, såsom ock några av edra egna skalder hava sagt: 'Vi äro ju ock av hans släkt.'
\par 29 Äro vi nu av Guds släkt, så böra vi icke mena att gudomen är lik någonting av guld eller silver eller sten, något som är danat genom mänsklig konst och uppfinning.
\par 30 Med sådana okunnighetens tider har Gud hittills haft fördrag, men nu bjuder han människorna att de alla allestädes skola göra bättring.
\par 31 Ty han har fastställt en dag då han skall 'döma världen med rättfärdighet', genom en man som han har bestämt därtill; och han har åt alla givit en bekräftelse härpå, i det att han har låtit honom uppstå från de döda."
\par 32 När de hörde talas om att "uppstå från de döda", drevo somliga gäck därmed, andra åter sade: "Vi vilja höra dig tala härom ännu en gång."
\par 33 Med detta besked gick Paulus bort ifrån dem.
\par 34 Dock slöto sig några män till honom och kommo till tro. Bland dessa var Dionysius, han som tillhörde Areopagens domstol, så ock en kvinna vid namn Damaris och några andra jämte dem.

\chapter{18}

\par 1 Därefter lämnade Paulus Aten och kom till Korint.
\par 2 Där träffade han en jude vid namn Akvila, bördig från Pontus, vilken nyligen hade kommit från Italien med sin hustru Priscilla. (Klaudius hade nämligen påbjudit att alla judar skulle lämna Rom.) Till dessa båda slöt han sig nu,
\par 3 och eftersom han hade samma hantverk som de, stannade han kvar hos dem, och de arbetade tillsammans; de voro nämligen till yrket tältmakare.
\par 4 Och i synagogan höll han var sabbat samtal och övertygade både judar och greker..
\par 5 När sedan Silas och Timoteus kommo ditned från Macedonien, var Paulus helt upptagen av att förkunna ordet, i det att han betygade för judarna att Jesus var Messias.
\par 6 Men när dessa stodo emot honom och foro ut i smädelser, skakade han stoftet av sina kläder och sade till dem: "Edert blod komme över edra egna huvuden. Jag är utan skuld och går nu till hedningarna."
\par 7 Och han gick därifrån och tog in hos en man vid namn Titius Justus, som "fruktade Gud"; denne hade sitt hus invid synagogan.
\par 8 Men Krispus, synagogföreståndaren, kom med hela sitt hus till tro på Herren; också många andra korintier som hörde honom trodde och läto döpa sig.
\par 9 Och i en syn om natten sade Herren till Paulus: "Frukta icke, utan tala och tig icke;
\par 10 ty jag är med dig, och ingen skall komma vid dig och göra dig skada. Jag har ock mycket folk i denna stad."
\par 11 Så uppehöll han sig där bland dem ett år och sex månader och undervisade i Guds ord.
\par 12 Men när Gallio var landshövding i Akaja, reste sig judarna, alla tillhopa, upp mot Paulus och förde honom inför domstolen
\par 13 och sade: "Denne man förleder människorna att dyrka Gud på ett sätt som är emot lagen."
\par 14 När då Paulus ville öppna sin mun och tala, sade Gallio till judarna: "Vore något brott eller något ont och arglistigt dåd begånget, då kunde väl vara skäligt att jag tålmodigt hörde på eder, I judar.
\par 15 Men är det någon tvistefråga om ord och namn eller om eder egen lag, så mån I själva avgöra saken; i sådana mål vill jag icke vara domare."
\par 16 Och så visade han bort dem från domstolen.
\par 17 Då grepo de alla gemensamt Sostenes, synagogföreståndaren, och slogo honom inför domstolen; och Gallio frågade alls icke därefter.
\par 18 Men Paulus stannade där ännu ganska länge. Därpå tog han avsked av bröderna och avseglade till Syrien, åtföljd av Priscilla och Akvila, sedan han i Kenkrea hade låtit raka sitt huvud; han hade nämligen bundit sig genom ett löfte.
\par 19 Så kommo de till Efesus, och där lämnade Paulus dem. Själv gick han in i synagogan och gav sig i samtal med judarna.
\par 20 Och de bådo honom att han skulle stanna där något längre; men han samtyckte icke därtill,
\par 21 utan tog avsked av dem med de orden: "Om Gud vill, skall jag vända tillbaka till eder." Och så lämnade han Efesus.
\par 22 Och när han hade kommit till Cesarea, begav han sig upp och hälsade på hos församlingen och for därefter ned till Antiokia.
\par 23 Sedan han hade uppehållit sig där någon tid, for han vidare, och färdades först genom det galatiska landet och därefter genom Frygien och styrkte alla lärjungarna.
\par 24 Men till Efesus kom en jude vid namn Apollos, bördig från Alexandria, en lärd man, mycket förfaren i skrifterna.
\par 25 Denne sade blivit undervisad om "Herrens väg" och talade, brinnande i anden, och undervisade grundligt om Jesus, fastän han allenast hade kunskap om Johannes' döpelse.
\par 26 Han begynte ock att frimodigt tala i synagogan. När Priscilla och Akvila hörde honom, togo de honom till sig och undervisade honom grundligare om "Guds väg".
\par 27 Och då han sedan ville fara till Akaja, skrevo bröderna till lärjungarna där och uppmanade dem att taga vänligt emot honom. Och när han hade kommit fram, blev han dem som trodde till mycken hjälp, genom den nåd han hade undfått.
\par 28 Ty med stor kraft vederlade han judarna offentligen och bevisade genom skrifterna att Jesus

\chapter{19}

\par 1 Medan Apollos var i Korint, kom Paulus, sedan han hade farit genom de övre delarna av landet, ned till Efesus. Där träffade han några lärjungar.
\par 2 Och han frågade dessa: "Undfingen I helig ande, när I kommen till tro?" De svarade honom: "Nej, vi hava icke ens hört att helig ande är given."
\par 3 Han frågade: "Vilken döpelse bleven I då döpta med?" De svarade: "Vi döptes med Johannes' döpelse"
\par 4 Då sade Paulus: "Johannes' döpelse var en döpelse till bättring; och han sade därvid till folket, att det var på den som skulle komma efter honom, det är på Jesus, som de skulle tro."
\par 5 Sedan de hade hört detta, läto de döpa sig i Herren Jesu namn.
\par 6 Och när Paulus lade händerna på dem, kom den helige Ande över dem, och de talade tungomål och profeterade.
\par 7 Och tillsammans voro de vid pass tolv män.
\par 8 Därefter gick han in i synagogan; och under tre månader samtalade han där, frimodigt och övertygande, med dem om Guds rike.
\par 9 Men när några av dem förhärdade sig och voro ohörsamma och inför menigheten talade illa om "den vägen", vände han sig ifrån dem och avskilde lärjungarna och samtalade sedan dagligen med dessa i Tyrannus' lärosal.
\par 10 Så fortgick det i två år, och alla provinsen Asiens inbyggare, både judar och greker, fingo på detta sätt höra Herrens ord.
\par 11 Och Gud gjorde genom Paulus kraftgärningar av icke vanligt slag.
\par 12 Man till och med tog handkläden och förkläden, som hade varit i beröring med hans kropp, och lade dem på de sjuka; och sjukdomarna veko då ifrån dem, och de onda andarna foro ut.
\par 13 Men också några kringvandrande judiska besvärjare företogo sig nu att över dem som voro besatta av onda andar nämna Herren Jesu namn; de sade: "Jag besvär eder vid den Jesus som Paulus predikar.
\par 14 Bland dem som så gjorde voro sju söner av en viss Skevas, en judisk överstepräst.
\par 15 Men den onde anden svarade då och sade till dem: "Jesus känner jag, Paulus är mig ock väl bekant men vilka ären I?"
\par 16 Och mannen som var besatt av den onde anden störtade sig på dem och övermannade både den ene och den andre; han betedde sig så våldsamt mot dem, att de måste fly ut ur huset, nakna och sargade.
\par 17 Och detta blev bekant för alla Efesus' invånare, både judar och greker, och fruktan föll över dem alla, och Herren Jesu namn blev storligen prisat.
\par 18 Och många av dem som hade kommit till tro trädde fram och bekände sin synd och omtalade vad de hade gjort.
\par 19 Och ganska många av dem som hade övat vidskepliga konster samlade ihop sina böcker och brände upp dem i allas åsyn. Och när man räknade tillsammans vad böckerna voro värda, fann man att värdet uppgick till femtio tusen silverpenningar.
\par 20 På detta sätt hade Herrens ord mäktig framgång och visade sin kraft.
\par 21 Efter allt detta bestämde sig Paulus genom Andens tillskyndelse, att över Macedonien och Akaja fara till Jerusalem. Och han sade: "Sedan jag har varit där, måste jag ock se Rom."
\par 22 Han sände då två av sina medhjälpare, Timoteus och Erastus, åstad till Macedonien, men själv stannade han ännu någon tid i provinsen Asien.
\par 23 Vid den tiden uppstod ganska mycket oväsen angående "den vägen".
\par 24 Där fanns nämligen en guldsmed, vid namn Demetrius, som förfärdigade Dianatempel av silver och därmed skaffade hantverkarna en ganska stor inkomst.
\par 25 Han kallade tillhopa dessa, jämte andra som hade liknande arbete, och sade: "I man, I veten att det är detta arbete som giver oss vår goda bärgning;
\par 26 men nu sen och hören I att denne Paulus icke allenast i Efesus, utan i nästan hela provinsen Asien genom sitt tal har förlett ganska mycket folk, i det han säger att de gudar som göras med människohänder icke äro gudar.
\par 27 Och det är fara värt, icke allenast att denna vår hantering kommer i missaktning, utan ock att den stora gudinnan Dianas helgedom bliver räknad för intet, och att jämväl denna gudinna, som hela provinsen Asien, ja, hela världen dyrkar, kommer att lida avbräck i sitt stora anseende."
\par 28 När de hörde detta, blevo de fulla av vrede och skriade: "Stor är efesiernas Diana!"
\par 29 Och hela staden kom i rörelse, och alla stormade på en gång till skådebanan och släpade med sig Gajus och Aristarkus, två macedonier som voro Paulus' följeslagare
\par 30 Paulus ville då gå in bland folket men lärjungarna tillstadde honom det icke.
\par 31 Också några asiarker, som voro hans vänner, sände bud till honom och bådo honom att han icke skulle giva sig in på skådebanan.
\par 32 Och de skriade, den ene så och den andre så; ty menigheten var upprörd, och de flesta visste icke varför de hade kommit tillsammans.
\par 33 Då drog man ur folkhopen fram Alexander, som judarna sköto framför sig. Och Alexander gav tecken med handen att han ville hålla ett försvarstal inför folket.
\par 34 Men när de märkte att han var jude, begynte de ropa, alla med en mun, och skriade under ett par timmars tid: "Stor är efesiernas Diana!"
\par 35 Men stadens kansler lugnade folket och sade: "Efesier, finnes då någon människa som icke vet, att efesiernas stad är vårdare av den stora Dianas tempel och den bild av henne, som har fallit ned från himmelen?
\par 36 Eftersom ju ingen kan bestrida detta, bören I hålla eder lugna och icke företaga eder något förhastat.
\par 37 Emellertid haven I dragit fram dessa män, som icke äro helgerånare, ej heller smäda vår gudinna,
\par 38 Om nu Demetrius och de hantverkare som hålla ihop med honom hava sak mot någon, så finnas ju domstolssammanträden och landshövdingar. Må de alltså göra upp saken med varandra inför rätta.
\par 39 Och haven I något att andraga som går därutöver, så må sådant avgöras i den lagliga folkförsamlingen.
\par 40 På grund av det som i dag har skett löpa vi ju till och med fara att bliva anklagade för upplopp, fastän vi icke hava gjort något ont; och någon giltig anledning till denna folkskockning kunna vi icke heller uppgiva."

\chapter{20}

\par 1 Då nu oroligheterna voro stillade, kallade Paulus lärjungarna till sig och talade till dem förmaningens ord; och sedan han hade tagit avsked av dem, begav han sig åstad för att fara till Macedonien.
\par 2 Och när han hade färdats genom det landet och jämväl där talat många förmaningens ord, kom han till Grekland.
\par 3 Där uppehöll han sig i tre månader. När han sedan tänkte avsegla därifrån till Syrien, beslöt han, eftersom judarna förehade något anslag mot honom, att göra återfärden genom Macedonien.
\par 4 Och med honom följde Sopater, Pyrrus' son, från Berea, och av tessalonikerna Aristarkus och Sekundus, vidare Gajus från Derbe och Timoteus, slutligen Tykikus och Trofimus från provinsen Asien.
\par 5 Men dessa foro i förväg och inväntade oss i Troas.
\par 6 Sedan, efter det osyrade brödets högtid, avseglade vi andra ifrån Filippi och träffade dem på femte dagen åter i Troas; och där vistades vi i sju dagar.
\par 7 På första veckodagen voro vi församlade till brödsbrytelse, och Paulus, som tänkte fara vidare dagen därefter, samtalade med bröderna. Och samtalet drog ut ända till midnattstiden;
\par 8 och ganska många lampor voro tända i den sal i övre våningen, där vi voro församlade.
\par 9 Invid fönstret satt då en yngling vid namn Eutykus, och när Paulus talade så länge, föll denne i djup sömn och blev så överväldigad av sömnen, att han störtade ned från tredje våningen; och när man tog upp honom, var han död
\par 10 Då gick Paulus ned och lade sig över honom och fattade om honom och sade: "Klagen icke så; ty livet är ännu kvar i honom."
\par 11 Sedan gick han åter upp, och bröt brödet och åt, och samtalade ytterligare ganska länge med dem, ända till dess att det dagades; först då begav han sig i väg.
\par 12 Och de förde ynglingen hem levande och kände sig nu icke litet tröstade.
\par 13 Men vi andra gingo i förväg ombord på skeppet och avseglade till Assos, där vi tänkte taga Paulus ombord; ty så hade han förordnat, eftersom han själv tänkte fara land vägen.
\par 14 Och när han sammanträffade med oss i Assos, togo vi honom ombord och kommo sedan till Mitylene.
\par 15 Därifrån seglade vi vidare och kommo följande dag mitt för Kios. Dagen därefter lade vi till vid Samos; och sedan vi hade legat över i Trogyllium, kommo vi nästföljande dag till Miletus.
\par 16 Paulus hade nämligen beslutit att segla förbi Efesus, för att icke fördröja sig i provinsen Asien; ty han påskyndade sin färd, för att, om det bleve honom möjligt, till pingstdagen kunna vara i Jerusalem.
\par 17 Men från Miletus sände han bud till Efesus och kallade till sig församlingens äldste.
\par 18 Och när de hade kommit till honom, sade han till dem: "I veten själva på vad sätt jag hela tiden, ifrån första dagen då jag kom till provinsen Asien, har umgåtts med eder:
\par 19 huru jag har tjänat Herren i all ödmjukhet, under tårar och prövningar, som hava vållats mig genom judarnas anslag.
\par 20 Och I veten att jag icke har dragit mig undan, när det gällde något som kunde vara eder nyttigt, och att jag icke har försummat att offentligen och hemma i husen predika för eder och undervisa eder.
\par 21 Ty jag har allvarligt uppmanat både judar och greker att göra bättring och vända sig till Gud och tro på vår Herre Jesus.
\par 22 Och se, bunden i anden begiver jag mig nu till Jerusalem, utan att veta vad där skall vederfaras mig;
\par 23 allenast det vet jag, att den helige Ande i den ene staden efter den andra betygar för mig och säger att bojor och bedrövelser vänta mig.
\par 24 Dock anser jag mitt liv icke vara av något värde för mig själv, om jag blott får väl fullborda mitt lopp och vad som hör till det ämbete jag har mottagit av Herren Jesus: att vittna om Guds nåds evangelium.
\par 25 Och se, jag vet nu att I icke mer skolen få se mitt ansikte, I alla bland vilka jag har gått omkring och predikat om riket.
\par 26 Därför betygar jag för eder nu i dag att jag icke bär skuld för någons blod.
\par 27 Ty jag har icke undandragit mig att förkunna för eder allt Guds rådslut.
\par 28 Så haven nu akt på eder själva och på hela den hjord i vilken den helige Ande har satt eder till föreståndare, till att vara herdar för Guds församling, som han har vunnit med sitt eget blod.
\par 29 Jag vet, att sedan jag har skilts från eder svåra ulvar skola komma in bland eder, och att de icke skola skona hjorden.
\par 30 Ja, bland eder själva skola män uppträda, som tala vad förvänt är, för att locka lärjungarna att följa sig.
\par 31 Vaken därför, och kommen ihåg att jag i tre års tid, natt och dag, oavlåtligen under tårar har förmanat var och en särskild av eder.
\par 32 Och nu anbefaller jag eder åt Gud och hans nådesord, åt honom som förmår uppbygga eder och giva åt eder eder arvedel bland alla som äro helgade.
\par 33 Silver eller guld eller kläder har jag icke åstundat av någon.
\par 34 I veten själva att dessa mina händer hava gjort tjänst, för att skaffa nödtorftigt uppehälle åt mig och åt dem som hava varit med mig.
\par 35 I allt har jag genom mitt föredöme visat eder att man så, under eget arbete, bör taga sig an de svaga och komma ihåg Herren Jesu ord, huru han själv sade: 'Saligare är att giva än att taga.'"
\par 36 När han hade sagt detta, föll han ned på sina knän och bad med dem alla.
\par 37 Och de begynte alla att gråta bitterligen och föllo Paulus om halsen och kysste honom innerligt;
\par 38 och mest sörjde de för det ordets skull som han hade sagt, att de icke mer skulle få se hans ansikte. Och så ledsagade de honom till skeppet.

\chapter{21}

\par 1 Sedan vi hade skilts ifrån dem, lade vi ut och foro raka vägen till Kos och kommo dagen därefter till Rodus och därifrån till Patara.
\par 2 Där funno vi ett skepp som skulle fara över till Fenicien; på det gingo vi ombord och lade ut.
\par 3 Och när vi hade fått Cypern i sikte, lämnade vi denna ö på vänster hand och seglade till Syrien och landade vid Tyrus; ty där skulle skeppet lossa sin last.
\par 4 Och vi uppsökte där lärjungarna och stannade hos dem i sju dagar. Dessa sade nu genom Andens tillskyndelse till Paulus att han icke borde begiva sig till Jerusalem.
\par 5 Men när vi hade stannat där de dagarna ut, bröto vi upp därifrån och gåvo oss i väg, ledsagade av dem alla, med hustrur och barn, ända utom staden. Och på stranden föllo vi ned på våra knän och bådo
\par 6 och togo sedan avsked av varandra. Därefter stego vi ombord på skeppet, och de andra vände tillbaka hem igen.
\par 7 Från Tyrus kommo vi till Ptolemais, och därmed avslutade vi sjöresan. Och vi hälsade på hos bröderna där och stannade hos dem en dag.
\par 8 Men följande dag begåvo vi oss därifrån och kommo till Cesarea. Där togo vi in hos evangelisten Filippus, en av de sju, och stannade kvar hos honom.
\par 9 Denne hade fyra ogifta döttrar, som ägde profetisk gåva.
\par 10 Under den tid av flera dagar, som vi stannade där, kom en profet, vid namn Agabus, dit ned från Judeen.
\par 11 När denne hade kommit till oss, tog han Paulus' bälte och band därmed sina händer och fötter och sade: "Så säger den helige Ande: 'Den man som detta bälte tillhör, honom skola judarna så binda i Jerusalem, och sedan skola de överlämna honom i hedningarnas händer.'"
\par 12 När vi hörde detta, bådo såväl vi själva som bröderna i staden honom att han icke skulle begiva sig upp till Jerusalem.
\par 13 Men då svarade Paulus: "Varför gråten I så och sargen mitt hjärta? Jag är ju redo icke allenast att låta mig bindas, utan ock att dö i Jerusalem, för Herren Jesu namns skull."
\par 14 Då han alltså icke lät övertala sig, gåvo vi oss till freds och sade: "Ske Herrens vilja."
\par 15 Efter de dagarnas förlopp gjorde vi oss i ordning och begåvo oss upp till Jerusalem.
\par 16 Från Cesarea följde också några av lärjungarna med oss, och dessa förde oss till en viss Mnason från Cypern, en gammal lärjunge, som vi skulle gästa hos.
\par 17 Och när vi kommo till Jerusalem, togo bröderna emot oss med glädje.
\par 18 Dagen därefter gick Paulus med oss andra till Jakob; dit kommo ock alla de äldste.
\par 19 Och sedan han hade hälsat dem förtäljde han för dem alltsammans, det ena med det andra, som Gud genom hans arbete hade gjort bland hedningarna.
\par 20 När de hörde detta, prisade de Gud. Och de sade till honom: "Du ser, käre broder, huru många tusen judar det är som hava kommit till tro, och alla nitälska de för lagen.
\par 21 Nu har det blivit dem sagt om dig, att du lär alla judar som bo spridda bland hedningarna att avfalla från Moses, i det du säger att de icke behöva omskära sina barn, ej heller i övrigt vandra efter vad stadgat är.
\par 22 Vad är då att göra? Helt visst skall man få höra att du har kommit hit.
\par 23 Gör därför såsom vi nu vilja säga dig. Vi hava här fyra män som hava bundit sig genom ett löfte.
\par 24 Tag med dig dessa, och låt helga dig tillsammans med dem, och åtag dig omkostnaderna för dem, så att de kunna låta raka sina huvuden. Då skola alla förstå att intet av allt det som har blivit dem sagt om dig äger någon grund, utan att också du vandrar efter lagen och håller den.
\par 25 Vad åter angår de hedningar som hava kommit till tro, så hava vi här beslutit och jämväl skrivit till dem, att de böra taga sig till vara för kött från avgudaoffer och för blod och för köttet av förkvävda djur och för otukt."
\par 26 Så tog då Paulus männen med sig och lät följande dag helga sig tillsammans med dem; sedan gick han in i helgedomen och gav till känna när den tid skulle gå till ända, för vilken de hade låtit helga sig, den tid före vars utgång offer skulle frambäras för var och en särskild av dem.
\par 27 När de sju dagarna nästan voro ute, fingo judarna från provinsen Asien se honom i helgedomen och uppviglade då allt folket. Och de grepo honom
\par 28 och ropade: "I män av Israel, kommen till hjälp! Här är den man som allestädes lär alla sådant som är emot vårt folk och emot lagen och emot denna plats. Därtill har han nu ock fört greker in i helgedomen och oskärat denna heliga plats."
\par 29 De hade nämligen förut sett efesiern Trofimus i staden tillsammans med honom och menade att Paulus hade fört denne in i helgedomen.
\par 30 Och hela staden kom i rörelse, och folket skockade sig tillsammans. Och då de nu hade gripit Paulus, släpade de honom ut ur helgedomen, varefter portarna genast stängdes igen.
\par 31 Men just som de stodo färdiga att dräpa honom, anmäldes det hos översten för den romerska vakten att hela Jerusalem var i uppror.
\par 32 Denne tog då strax med sig krigsmän och hövitsmän och skyndade ned till dem. Och när de fingo se översten och krigsmännen, upphörde de att slå Paulus
\par 33 Översten gick då fram och tog honom i förvar och bjöd att man skulle fängsla honom med två kedjor. Och han frågade vem han var och vad han hade gjort.
\par 34 Men bland folket ropade den ene så, den andre så. Då han alltså för larmets skull icke kunde få något säkert besked, bjöd han att man skulle föra honom till kasernen.
\par 35 Och när han kom fram till trappan, trängde folket så våldsamt på, att han måste bäras av krigsmännen,
\par 36 ty folkhopen följde efter och skriade: "Bort med honom!"
\par 37 Då nu Paulus skulle föras in i kasernen, sade han till översten: "Tillstädjes det mig att säga något till dig?" Han svarade: "Kan du tala grekiska?
\par 38 Är du då icke den egyptier som för en tid sedan ställde till 'dolkmännens' uppror, de fyra tusens, och förde dem ut i öknen?"
\par 39 Då svarade Paulus: "Nej, jag är en judisk man från Tarsus, medborgare alltså i en betydande stad i Cilicien. Men jag beder dig, tillstäd mig att tala till folket."
\par 40 Och han tillstadde honom det. Då gav Paulus från trappan, där han stod, med handen ett tecken åt folket. Och sedan där hade blivit helt tyst, talade han till dem på hebreiska och sade:

\chapter{22}

\par 1 "Bröder och fäder, hören vad jag nu inför eder vill tala till mitt försvar."
\par 2 När de hörde att han talade till dem på hebreiska, blevo de ännu mer stilla. Och han fortsatte:
\par 3 "Jag är en judisk man, född i Tarsus i Cilicien, men uppfostrad här i staden och undervisad vid Gamaliels fötter, efter fädernas lag i all dess stränghet. Och jag var en man som nitälskade för Gud, såsom I allasammans i dag gören.
\par 4 Jag förföljde 'den vägen' ända till döds, och både män och kvinnor lät jag binda och sätta i fängelse;
\par 5 det vittnesbördet kan översteprästen och de äldstes hela råd giva mig. Också fick jag av dem brev till bröderna i Damaskus; och jag begav mig dit, för att fängsla jämväl dem som voro där och föra dem till Jerusalem, så att de kunde bliva straffade.
\par 6 Men när jag var på vägen och nalkades Damaskus, hände sig vid middagstiden att ett starkt sken från himmelen plötsligt kringstrålade mig.
\par 7 Och jag föll ned till marken och hörde då en röst som sade till mig: 'Saul, Saul, varför förföljer du mig?'
\par 8 Då svarade jag: 'Vem är du, Herre?' Han sade till mig: 'Jag är Jesus från Nasaret, den som du förföljer.'
\par 9 Och de som voro med mig sågo väl skenet, men hörde icke rösten av den som talade till mig.
\par 10 Då frågade jag: 'Vad skall jag göra, Herre?' Och Herren svarade mig: 'Stå upp och gå in i Damaskus; där skall allt det bliva dig sagt, som är dig förelagt att göra.'
\par 11 Men eftersom jag, till följd av det starka skenet, icke mer kunde se togo mina följeslagare mig vid handen och ledde mig, så att jag kom in i Damaskus.
\par 12 Där fanns en efter lagen fram man, Ananias, vilken hade gott vittnesbörd om sig av alla judar som bodde där.
\par 13 Denne kom nu och trädde fram till mig och sade: 'Saul, min broder, hav din syn igen.' Och i samma stund fick jag min syn igen och såg upp på honom.
\par 14 Då sade han: 'Våra fäders Gud har utsett dig till att känna hans vilja och till att se den Rättfärdige och höra ord från hans mun.
\par 15 Ty du skall vara hans vittne inför alla människor och vittna om vad du har sett och hört.
\par 16 Varför dröjer du då nu? Stå upp och låt döpa dig och avtvå dina synder, och åkalla därvid hans namn.'
\par 17 Men när jag hade kommit tillbaka till Jerusalem, hände sig, medan jag bad i helgedomen, att jag föll i hänryckning
\par 18 och såg honom och hörde honom säga till mig: 'Skynda dig med hast bort ifrån Jerusalem; ty de skola icke här taga emot ditt vittnesbörd om mig.'
\par 19 Men jag sade: 'Herre, de veta själva att det var jag som överallt i synagogorna lät fängsla och gissla dem som trodde på dig.
\par 20 Och när Stefanus', ditt vittnes, blod utgöts, var ock jag tillstädes och gillade vad som skedde och vaktade de mäns kläder, som dödade honom.'
\par 21 Då sade han till mig: Gå; jag vill sända dig åstad långt bort till hedningarna.'"
\par 22 Ända till dess att han sade detta hade de hört på honom. Men nu hovo de upp sin röst och ropade: "Bort ifrån jorden med den människan! Det är icke tillbörligt att en sådan får leva."
\par 23 Då de så skriade och därvid revo av sig sina kläder och kastade stoft upp i luften,
\par 24 bjöd översten att man skulle föra in honom i kasernen, och gav befallning om att man skulle förhöra honom under gisselslag, så att han finge veta varför de så ropade mot honom.
\par 25 Men när de redan hade sträckt ut honom till gissling, sade Paulus till den hövitsman som stod där: "Är det lovligt för eder att gissla en romersk medborgare, och det utan dom och rannsakning?"
\par 26 När hövitsmannen hörde detta, gick han till översten och underrättade honom härom och sade: "Vad är det du tänker göra? Mannen är ju romersk medborgare."
\par 27 Då gick översten dit och frågade honom: "Säg mig, är du verkligen romersk medborgare?" Han svarade: "Ja."
\par 28 Översten sade då: Mig har det kostat en stor summa penningar att köpa den medborgarrätten." Men Paulus sade: "Jag däremot har den redan genom födelsen."
\par 29 Männen som skulle hava förhört honom drogo sig då strax undan och lämnade honom. Och när översten nu hade fått veta att han var romersk medborgare, blev också han förskräckt, vid tanken på att han hade låtit fängsla honom.
\par 30 Då han emellertid ville få säkert besked om varför Paulus anklagades av judarna, låt han dagen därefter taga av honom bojorna och bjöd översteprästerna och hela Stora rådet att komma tillsammans. Sedan lät han föra Paulus ditned och ställde honom inför dem.

\chapter{23}

\par 1 Och Paulus fäste ögonen på Rådet och sade: "Mina bröder, allt intill denna dag har jag vandrat inför Gud med ett i allo gott samvete."
\par 2 Då befallde översteprästen Ananias dem som stodo bredvid honom, att de skulle slå honom på munnen.
\par 3 Paulus sade då till honom: "Gud skall slå dig, du vitmenade vägg. Du sitter här för att döma mig efter lagen, och ändå bjuder du, tvärtemot lagen, att man skall slå mig!"
\par 4 Då sade de som stodo därbredvid: "Smädar du Guds överstepräst?"
\par 5 Paulus svarade: "Jag visste icke, mina bröder, att han var överstepräst. Det är ju skrivet: 'Mot en hövding i ditt folk skall du icke tala onda ord.'"
\par 6 Nu hade Paulus märkt att den ena delen av dem utgjordes av sadducéer och den andra av fariséer. Därför sade han med ljudelig röst inför Rådet: "Mina bröder, jag är farisé, en avkomling av fariséer. Det är för vårt hopps skull, för de dödas uppståndelses skull, som jag står här inför rätta."
\par 7 Knappt hade han sagt detta, förrän en strid uppstod mellan fariséerna och sadducéerna, så att hopen blev delad.
\par 8 Sadducéerna säga nämligen att det icke finnes någon uppståndelse, ej heller någon ängel eller ande, men fariséerna bekänna sig tro på både det ena och det andra.
\par 9 Och man begynte ropa och larma; och några skriftlärde som hörde till fariséernas parti stodo upp och begynte ivrigt disputera med de andra och sade: "Vi finna intet ont hos denne man. Kanhända har en ande eller en ängel verkligen talat med honom."
\par 10 Då nu en så häftig strid hade uppstått, fruktade översten att de skulle slita Paulus i stycken, och bjöd manskapet gå ned och rycka honom undan dem och föra honom till kasernen.
\par 11 Natten därefter kom Herren och stod framför honom och sade: "Var vid gott mod; ty såsom du har vittnat om mig i Jerusalem, så måste du ock vittna i Rom."
\par 12 När det sedan hade blivit dag, sammangaddade sig judarna och förpliktade sig med dyr ed att varken äta eller dricka, förrän de hade dräpt Paulus.
\par 13 Och det var mer än fyrtio män som så hade sammansvurit sig.
\par 14 Dessa gingo till översteprästerna och de äldste och sade: "Vi hava med dyr ed förpliktat oss att ingenting smaka, förrän vi hava dräpt Paulus.
\par 15 Så mån I nu, tillsammans med Rådet, hemställa hos översten att han låter föra honom ned till eder, detta under föregivande att I tänken grundligare undersöka hans sak. Vi skola då vara redo att röja honom ur vägen, innan han hinner fram."
\par 16 Men Paulus' systerson fick höra om försåtet. Han kom därför till kasernen och gick ditin och omtalade för Paulus vad han hade hört.
\par 17 Paulus bad då att en av hövitsmännen skulle komma till honom, och sade: "För denne yngling till översten; ty han har en underrättelse att lämna honom."
\par 18 Denne tog honom då med sig och förde honom till översten och sade: "Fången Paulus har kallat mig till sig och bett mig föra denne yngling till dig, ty han har något att säga dig."
\par 19 Då tog översten honom vid handen och gick avsides med honom och frågade honom: "Vad är det för en underrättelse du har att lämna mig?"
\par 20 Han svarade: "Judarna hava kommit överens om att bedja dig att du i morgon låter föra Paulus ned till Rådet, detta under föregivande att det tänker skaffa sig grundligare kunskap om honom.
\par 21 Gör dem nu icke till viljes häri; ty mer än fyrtio av dem ligga i försåt för honom och hava med dyr ed förpliktat sig att varken äta eller dricka, förrän de hava röjt honom ur vägen. Och nu äro de redo och vänta allenast på att du skall bevilja deras begäran."
\par 22 Översten bjöd då ynglingen att icke för någon omtala att han hade yppat detta för honom, och lät honom sedan gå.
\par 23 Därefter kallade han till sig två av hövitsmännen och sade till dem: "Låten två hundra krigsmän göra sig redo att i natt vid tredje timmen avgå till Cesarea, så ock sjuttio ryttare och två hundra spjutbärare."
\par 24 Och han tillsade dem att skaffa åsnor, som de skulle låta Paulus rida på så att han oskadd kunde föras till landshövdingen Felix.
\par 25 Och han skrev ett brev, så lydande:
\par 26 "Klaudius Lysias hälsar den ädle landshövdingen Felix.
\par 27 Denne man blev gripen av judarna, och det var nära att han hade blivit dödad av dem. Då kom jag tillstädes med mitt manskap och tog honom ifrån dem, sedan jag hade fått veta att han var romersk medborgare.
\par 28 Men då jag också ville veta vad de anklagade honom för, lät jag ställa honom inför deras Stora råd.
\par 29 Jag fann då att anklagelsen mot honom gällde några tvistefrågor i deras lag, men att han icke var anklagad för något som förtjänade död eller fängelse.
\par 30 Sedan har jag fått kännedom om att något anslag förehaves mot honom, och därför sänder jag honom nu strax till dig. Jag har jämväl bjudit hans anklagare att inför dig föra sin talan mot honom."
\par 31 Så togo nu krigsmännen Paulus, såsom det hade blivit dem befallt, och förde honom om natten till Antipatris.
\par 32 Dagen därefter vände de själva tillbaka till kasernen och läto ryttarna färdas vidare med honom.
\par 33 När dessa kommo till Cesarea, lämnade de fram brevet till landshövdingen och förde jämväl Paulus fram inför honom.
\par 34 Sedan han hade läst brevet, frågade han från vilket landskap han var; och när han hade fått veta att han var från Cilicien, sade han:
\par 35 "Jag skall höra vad du har att säga, när också dina anklagare hava kommit tillstädes." Och så bjöd han att man skulle förvara honom i Herodes' borg.

\chapter{24}

\par 1 Fem dagar därefter for översteprästen Ananias ditned med några av de äldste och en sakförare, Tertullus; dessa anmälde inför landshövdingen klagomål mot Paulus.
\par 2 Och sedan denne hade blivit förekallad, begynte Tertullus sitt anklagelsetal; han sade:
\par 3 "Att vi genom dig åtnjuta mycken frid och ro, och att genom din försorg, ädle Felix, goda åtgärder hava blivit vidtagna för detta folk, det erkänna vi på allt sätt och allestädes, med många tacksägelser.
\par 4 Men för att icke alltför länge besvära dig beder jag att du, i din mildhet, ville höra allenast några få ord av oss.
\par 5 Vi hava funnit att denne är en fördärvlig man, som uppväcker strid bland alla judar i hela världen, och att han är en huvudman för nasaréernas parti.
\par 6 Han har ock försökt att oskära helgedomen; därför grepo vi honom, och vi ville döma honom efter vår lag.
\par 7 Men överste Lysias kom och ryckte honom med våld ur våra händer och bortförde honom
\par 8 och befallde hans anklagare att komma inför dig. och du kan nu själv anställa rannsakning med honom och så skaffa dig kännedom om allt det som vi anklaga honom för."
\par 9 De andra judarna instämde häri och påstodo att det förhöll sig så.
\par 10 Då landshövdingen nu gav tecken åt Paulus att han skulle tala, tog han till orda och sade: "Eftersom jag vet att du nu i många år har varit domare över detta folk, försvarar jag min sak med frimodighet.
\par 11 Du kan själv lätt förvissa dig om att det icke är mer än tolv dagar sedan jag kom upp till Jerusalem för att tillbedja.
\par 12 Och varken i helgedomen eller i synagogorna eller ute i staden har man funnit mig tvista med någon eller ställa till folkskockning.
\par 13 Ej heller kunna de inför dig bevisa det som de nu anklaga mig för.
\par 14 Men det bekänner jag för dig att jag, i enlighet med 'den vägen', vilken de kalla en partimening, så tjänar mina fäders Gud, att jag tror allt vad som är skrivet i lagen och i profeterna,
\par 15 och att jag har samma hopp till Gud som dessa hysa, att de döda skola uppstå, både rättfärdiga och orättfärdiga.
\par 16 Därför lägger också jag mig vinn om att alltid hava ett okränkt samvete inför Gud och människor.
\par 17 Så kom jag nu, efter flera års förlopp, tillbaka för att överlämna några allmosor till mitt folk och för att frambära offer.
\par 18 Därunder påträffades jag i helgedomen, sedan jag hade låtit helga mig, utan att hava vållat någon folkskockning eller något larm,
\par 19 av några judar från provinsen Asien, vilka nu borde vara här tillstädes inför dig och framställa sina klagomål, om de hava något att anklaga mig för.
\par 20 Eller ock må dessa som äro här tillstädes säga vad orätt de funno mig skyldig till, när jag stod inför Stora rådet,
\par 21 om det icke skulle vara i fråga om detta enda ord, som jag ljudeligen uttalade, där jag stod ibland dem: 'Det är för de dödas uppståndelses skull som jag i dag står inför rätta här bland eder.'"
\par 22 Men Felix, som mycket väl kände till "den vägen", uppsköt målet och sade: "När översten Lysias kommer hit ned, vill jag undersöka eder sak."
\par 23 Och han befallde hövitsmannen att hålla honom i förvar, dock så, att man skulle behandla honom milt och icke hindra någon av hans närmaste från att vara honom till tjänst.
\par 24 Någon tid därefter infann sig Felix tillsammans med sin hustru Drusilla, som var judinna; och han lät hämta Paulus och hörde honom om tron på Kristus Jesus.
\par 25 Men när Paulus talade med dem om rättfärdighet och återhållsamhet och om den tillstundande domen, blev Felix förskräckt och sade: "Gå din väg för denna gång; när jag får läglig tid, vill jag kalla dig till mig."
\par 26 Han hoppades också att han skulle få penningar av Paulus, varför han ock ganska ofta lät hämta honom och samtalade med honom.
\par 27 När två år voro förlidna, fick Felix till efterträdare Porcius Festus. Och eftersom Felix ville göra judarna sig bevågna, lämnade han Paulus kvar i fängelset.

\chapter{25}

\par 1 Tre dagar efter det att Festus hade tillträtt hövdingdömet for han från Cesarea upp till Jerusalem.
\par 2 Översteprästerna och de förnämsta bland judarna anmälde då inför honom klagomål mot Paulus.
\par 3 För att få denne i sitt våld anhöllo de hos Festus och begärde såsom en ynnest, att han skulle låta hämta honom till Jerusalem. De ville nämligen lägga försåt för honom, så att de kunde döda honom under vägen.
\par 4 Festus svarade då att Paulus hölls i förvar i Cesarea, och att han själv tänkte inom kort fara dit tillbaka.
\par 5 Och han tillade: "De bland eder som det vederbör må alltså fara dit ned med mig och framlägga sin anklagelse mot mannen, om han är skyldig till något ont."
\par 6 Sedan han hade vistats hos dem högst åtta eller tio dagar, kom han åter ned till Cesarea. Dagen därefter satte han sig på domarsätet och bjöd att Paulus skulle föras fram.
\par 7 När denne hade infunnit sig, omringades han av de judar som hade kommit ned från Jerusalem, och dessa framställde nu många svåra beskyllningar. Men de förmådde icke bevisa dem,
\par 8 ty Paulus försvarade sig och visade att han icke på något sätt hade försyndat sig, vare sig mot judarnas lag eller mot helgedomen eller mot kejsaren.
\par 9 Men Festus ville göra judarna sig bevågna och frågade Paulus och sade: "Vill du fara upp till Jerusalem och där stå till rätta inför mig i denna sak?"
\par 10 Paulus svarade: "Jag står här inför kejserlig domstol, och av sådan domstol bör jag dömas. Mot judarna har jag intet orätt gjort, såsom du själv mycket väl vet.
\par 11 Om jag nu eljest är skyldig till något orätt och har gjort något som förtjänar döden, så vill jag icke undandraga mig att dö; men om deras anklagelser mot mig äro utan grund, så kan ingen giva mig till pris åt dem. Jag vädjar till kejsaren." -
\par 12 Sedan Festus därefter hade överlagt med sitt råd, svarade han: "Till kejsaren har du vädjat, till kejsaren skall du ock få fara."
\par 13 Efter några dagars förlopp kommo konung Agrippa och Bernice till Cesarea och hälsade på hos Festus.
\par 14 Medan de nu i flera dagar vistades där, framlade Festus Paulus' sak för konungen och sade: "Felix har här lämnat efter sig en man såsom fånge;
\par 15 och när jag var i Jerusalem, anmälde judarnas överstepräster och äldste klagomål mot honom och begärde att han skulle dömas skyldig.
\par 16 Men jag svarade dem att det icke var romersk sed att prisgiva någon människa, förrän den anklagade hade fått stå ansikte mot ansikte med sina anklagare och haft tillfälle att försvara sig mot anklagelsen.
\par 17 Sedan de hade kommit med mig hit, satte jag mig alltså utan uppskov, dagen därefter, på domarsätet och bjöd att mannen skulle föras fram.
\par 18 Men när hans anklagare uppträdde, anförde de mot honom ingen beskyllning för sådana förbrytelser som jag hade tänkt mig;
\par 19 de voro allenast i tvist med honom om några frågor som rörde deras särskilda gudsdyrkan, och angående en viss Jesus, som är död, men om vilken Paulus påstod att han lever.
\par 20 Då jag var villrådig huru jag skulle göra med undersökningen härom, frågade jag om han ville fara till Jerusalem och där stå till rätta i denna sak.
\par 21 När Paulus då sade sig vilja vädja till kejsaren och begärde att bliva hållen i förvar, för att sedan undergå rannsakning inför honom, bjöd jag att han skulle hållas i förvar, till dess jag kunde sända honom till kejsaren."
\par 22 Då sade Agrippa till Festus: "Jag skulle också själv gärna vilja höra den mannen." Han svarade: "I morgon skall du få höra honom."
\par 23 Dagen därefter kommo alltså Agrippa och Bernice, med stor ståt, och gingo in i domsalen, tillika med överstarna och de förnämsta männen i staden; och på Festus' befallning blev Paulus införd.
\par 24 Då sade Festus: "Konung Agrippa, och alla I andra som ären har tillstädes med oss, I sen här den man för vilkens skull hela hopen av judar, både i Jerusalem och här, har legat över mig med sina rop att han icke borde få leva längre.
\par 25 Jag för min del har kommit till insikt om att han icke har gjort något som förtjänar döden;
\par 26 men då han nu själv har vädjat till kejsaren, har jag beslutit att sända honom till denne. Jag har emellertid icke något säkert besked om honom att giva min höge herre, när jag skriver. Därför har jag fört honom fram inför eder, och först och främst inför dig, konung Agrippa, för att jag, efter det att rannsakning har; blivit hållen, skall få veta vad jag bör skriva.
\par 27 Ty det synes mig vara orimligt att sända åstad en fånge, utan att på samma gång giva till känna vad han är anklagad för."

\chapter{26}

\par 1 Agrippa sade nu till Paulus: "Det tillstädjes dig att tala för din sak." Då räckte Paulus ut handen och talade så till sitt försvar:
\par 2 "Jag skattar mig lycklig att jag, i fråga om allt det som judarna anklaga mig för, i dag skall försvara mig inför dig, konung Agrippa,
\par 3 som så väl känner judarnas alla stadgar och tvistefrågor. Därför beder jag dig höra mig med tålamod.
\par 4 Hurudant mitt liv allt ifrån ungdomen har varit, det veta alla judar, ty jag har ju från tidiga år framlevat det bland mitt folk och i Jerusalem.
\par 5 Och sedan lång tid tillbaka känna de om mig - såframt de nu vilja tillstå det - att jag har tillhört det strängaste partiet i vår gudsdyrkan och levat såsom farisé.
\par 6 Och nu står jag här till rätta för vårt hopp om det som Gud har lovat våra fäder,
\par 7 det vartill ock våra tolv stammar, under det de tjäna Gud med iver både natt och dag, hoppas att nå fram. För det hoppets skull, o konung, är jag anklagad av judarna.
\par 8 Varför hålles det då bland eder för otroligt att Gud uppväcker döda?
\par 9 Jag för min del menade alltså att jag med all makt borde strida mot Jesu, nasaréens, namn;
\par 10 så gjorde jag ock i Jerusalem. Och många av de heliga inspärrade jag i fängelse, sedan jag av översteprästerna hade fått fullmakt därtill; och när man ville döda dem, röstade ock jag därför.
\par 11 Och överallt i synagogorna försökte jag, gång på gång, att genom straff tvinga dem till hädelse. I mitt raseri mot dem gick jag så långt, att jag förföljde dem till och med ända in i utländska städer.
\par 12 När jag nu i detta ärende var på väg till Damaskus, med fullmakt och uppdrag från översteprästerna,
\par 13 fick jag under min färd, o konung, mitt på dagen se ett sken från himmelen, klarare än solens glans, kringstråla mig och mina följeslagare.
\par 14 Och vi föllo alla ned till jorden, och jag hörde då en röst säga till mig på hebreiska: 'Saul, Saul, varför förföljer du mig'? Det är dig svårt att spjärna mot udden.'
\par 15 Då sade jag: 'Vem är du, Herre?' Herren svarade: 'Jag är Jesus, den som du förföljer.
\par 16 Men res dig upp och stå på dina fötter; ty därför har jag visat mig för dig, att jag har velat utse dig till en tjänare och ett vittne, som skall vittna både om huru du nu har sett mig, och om huru jag vidare skall uppenbara mig för dig.
\par 17 Och jag skall rädda dig såväl från ditt eget folk som från hedningarna. Ty till dem sänder jag dig,
\par 18 för att du skall öppna deras ögon, så att de omvända sig från mörkret till ljuset, och från Satans makt till Gud, på det att de må, genom tron på mig, undfå syndernas förlåtelse och få sin lott bland dem som äro helgade.'
\par 19 Så blev jag då, konung Agrippa, icke ohörsam mot den himmelska synen,
\par 20 utan predikade först för dem som voro i Damaskus och i Jerusalem, och sedan över hela judiska landet och för hedningarna, att de skulle göra bättring och omvända sig till Gud och göra sådana gärningar som tillhöra bättringen.
\par 21 För denna saks skull var det som judarna grepo mig i helgedomen och försökte att döda mig.
\par 22 Genom den hjälp som jag har undfått av Gud står jag alltså ännu i dag såsom ett vittne inför både små och stora; och jag säger intet annat, än vad profeterna och Moses hava sagt skola ske,
\par 23 nämligen att Messias skulle lida och, såsom förstlingen av dem som uppstå från de döda, bära budskap om ljuset, såväl till vårt eget folk som till hedningarna."
\par 24 När han på detta satt försvarade sig, utropade Festus: "Du är från dina sinnen, Paulus; den myckna lärdomen gör dig förryckt."
\par 25 Men Paulus svarade: "Jag är icke från mina sinnen, ädle Festus; jag talar sanna ord med lugn besinning.
\par 26 Konungen känner väl till dessa ting; därför talar jag också frimodigt inför honom. Ty jag kan icke tro att något av detta är honom obekant; det har ju icke tilldragit sig i någon undangömd vrå.
\par 27 Tror du profeterna, konung Agrippa? Jag vet att du tror dem."
\par 28 Då sade Agrippa till Paulus: "Föga fattas att du övertalar mig och gör mig till kristen."
\par 29 Paulus svarade: "Vare sig det fattas litet eller fattas mycket, skulle jag önska inför Gud att icke allenast du, utan alla som i dag höra mig, måtte bliva sådana som jag är, dock med undantag av dessa bojor."
\par 30 Därefter stod konungen upp, och med honom landshövdingen och Bernice och de som sutto där tillsammans med dem.
\par 31 Och när de gingo därifrån, talade de med varandra och sade: "Den mannen har icke gjort något som förtjänar död eller fängelse."
\par 32 Och Agrippa sade till Festus: "Denne man hade väl kunnat frigivas, om han icke hade vädjat till kejsaren."

\chapter{27}

\par 1 När det nu var beslutet att vi skulle avsegla till Italien, blev Paulus jämte några andra fångar överlämnad åt en hövitsman, vid namn Julius, som tillhörde den kejserliga vakten.
\par 2 Och vi gingo ombord på ett skepp från Adramyttium, som skulle anlöpa provinsen Asiens kuststäder. Så lade vi ut, och vi hade med oss Aristarkus, en macedonier från Tessalonika.
\par 3 Dagen därefter lade vi till vid Sidon. Och Julius, som bemötte Paulus med välvilja, tillstadde honom att besöka sina vänner där och åtnjuta deras omvårdnad.
\par 4 När vi hade lagt ut därifrån, seglade vi under Cypern, eftersom vinden låg emot.
\par 5 Och sedan vi hade seglat över havet, utanför Cilicien och Pamfylien, landade vi vid Myrra i Lycien.
\par 6 Där träffade hövitsmannen på ett skepp från Alexandria, som skulle segla till Italien, och på det förde han oss ombord.
\par 7 Under en längre tid gick nu seglingen långsamt, och vi kommo med knapp nöd inemot Knidus. Och då vinden icke var oss gynnsam, seglade vi in under Kreta vid Salmone.
\par 8 Det var med knapp nöd som vi kommo där förbi och hunno fram till en ort som kallades Goda hamnarna, icke långt från staden Lasea.
\par 9 Härunder hade ganska lång tid hunnit förflyta, och sjöfarten begynte redan vara osäker; fastedagen var nämligen redan förbi. Paulus varnade dem då
\par 10 och sade: "I mån, jag ser att denna sjöresa kommer att medföra vedervärdigheter och stor olycka, icke allenast för last och skepp, utan ock för våra liv."
\par 11 Men hövitsmannen trodde mer på styrmannen och skepparen än på det som Paulus sade.
\par 12 Och då hamnen icke låg väl till för övervintring, var flertalet av den meningen att man borde lägga ut därifrån och försöka om man kunde komma fram till Fenix, en hamn på Kreta, som ligger skyddad mot sydväst och nordväst; där skulle de sedan stanna över vintern.
\par 13 Och då nu en lindrig sunnanvind blåste upp, menade de sig hava målet vunnet, och lyfte ankar och foro tätt utmed Kreta.
\par 14 Men icke långt därefter kom en våldsam stormvind farande ned från ön; det var den så kallade nordostorkanen.
\par 15 Då skeppet av denna rycktes med och icke kunde hållas upp mot vinden, gåvo vi efter och läto det driva.
\par 16 När vi kommo under en liten ö som hette Kauda, förmådde vi dock, fastän med knapp nöd, bärga skeppsbåten.
\par 17 Sedan manskapet hade dragit upp den, tillgrepo de nödhjälpsmedel och slogo tåg om skeppet. Och då de fruktade att bliva kastade på Syrtenrevlarna, lade de ut drivankare och läto skeppet så driva.
\par 18 Och eftersom vi alltjämt hårt ansattes av stormen, vräkte de dagen därefter en del av lasten över bord.
\par 19 På tredje dagen kastade de med egna händer ut skeppsredskapen.
\par 20 Och då under flera dagar varken sol eller stjärnor hade synts, och stormen låg ganska hårt på, hade vi icke mer något hopp om räddning.
\par 21 Då nu många funnos som ingenting ville förtära, trädde Paulus upp mitt ibland dem och sade: "I män, I haden bort lyda mig och icke avsegla från Kreta; I haden då kunnat spara eder dessa vedervärdigheter och denna olycka.
\par 22 Men nu uppmanar jag eder att vara vid gott mod, ty ingen av eder skall förlora sitt liv; allenast skeppet skall gå förlorat.
\par 23 Ty i natt kom en ängel från den Gud som jag tillhör, och som jag också tjänar, och stod bredvid mig och sade:
\par 24 'Frukta icke, Paulus. Du skall komma att stå inför kejsaren; och se, Gud har skänkt dig alla dem som segla med dig.'
\par 25 Varen därför vid gott mod, I män; ty jag har den förtröstan till Gud, att så skall ske som mig är sagt.
\par 26 Men på en ö måste vi bliva kastade."
\par 27 När vi nu den fjortonde natten drevo omkring på Adriatiska havet, tyckte sjömännen sig vid midnattstiden finna att de närmade sig något land.
\par 28 De lodade då och funno tjugu famnars djup. När de hade kommit ett litet stycke längre fram lodade de åter och funno femton famnars djup.
\par 29 Då fruktade de att vi skulle stöta på något skarpt grund, och kastade därför ut fyra ankaren från akterskeppet och längtade efter att det skulle dagas.
\par 30 Sjömännen ville emellertid fly ifrån skeppet och firade ned skeppsbåten i havet, under föregivande att de tänkte föra ut ankaren ifrån förskeppet.
\par 31 Då sade Paulus till hövitsmannen och krigsmännen: "Om icke dessa stanna kvar på skeppet, så kunnen I icke räddas."
\par 32 Då höggo krigsmännen av de tåg som höllo skeppsbåten, och läto den fara.
\par 33 Medan det nu höll på att dagas, uppmanade Paulus alla att taga sig mat och sade: "Det är i dag fjorton dagar som I haven väntat och förblivit fastande, utan att förtära något.
\par 34 Därför uppmanar jag eder att taga eder mat; detta skall förhjälpa eder till räddning. Ty på ingen av eder skall ett huvudhår gå förlorat.
\par 35 När han hade sagt detta, tog han ett bröd och tackade Gud i allas åsyn och bröt det och begynte äta.
\par 36 Då blevo alla de andra vid gott mod och togo sig mat, också de.
\par 37 Och vi voro på skeppet tillsammans två hundra sjuttiosex personer.
\par 38 Sedan de hade ätit sig mätta, lättade de skeppet genom att kasta vetelasten i havet.
\par 39 När det blev dag, kände de icke igen landet; men de blevo varse en vik med låg strand och beslöto då att, om möjligt, låta skeppet driva upp på denna.
\par 40 De kapade så ankartågen på båda sidor och lämnade ankarna kvar i havet; tillika lösgjorde de rodren och hissade förseglet för vinden och styrde mot stranden.
\par 41 De stötte då på ett rev och läto skeppet gå upp på det. Där fastnade förskeppet och blev stående orörligt, men akterskeppet begynte brytas sönder av vågsvallet.
\par 42 Då ville krigsmännen döda fångarna, för att ingen skulle kunna fly undan simmande.
\par 43 Men hövitsmannen ville rädda Paulus och hindrade dem därför i deras uppsåt, och bjöd att de simkunniga först skulle kasta sig i vattnet och söka komma i land,
\par 44 och att därefter de övriga skulle giva sig ut, somliga på plankor, andra på spillror av skeppet. Så lyckades det för alla att komma välbehållna i land.

\chapter{28}

\par 1 Först sedan vi hade blivit räddade, fingo vi veta att ön hette Malta.
\par 2 Och infödingarna visade oss en icke vanlig välvilja; de tände upp en eld och togo oss alla med sig dit, för det påkommande regnets och för köldens skull.
\par 3 När Paulus då tog upp ett fång torra kvistar som han lade på elden, kom, i följd av hettan, en huggorm fram därur och högg sig fast vid hans hand.
\par 4 Då infödingarna fingo se ormen hänga där vid hans hand, sade de till varandra: "Helt visst är denne man en dråpare, som rättvisans gudinna icke tillstädjer att leva, om han nu ock har blivit räddad undan havet."
\par 5 Men han skakade ormen ifrån sig i elden och led ingen skada.
\par 6 De väntade att han skulle svälla upp eller helt plötsligt falla ned död; men när de efter lång väntan fingo se att intet ont vederfors honom, ändrade de mening och sade att han var en gud.
\par 7 I närheten av detta ställe var en lantgård, som tillhörde den förnämste mannen på ön, en som hette Publius; denne tog välvilligt emot oss och gav oss härbärge i tre dagar.
\par 8 Nu hände sig att Publius' fader låg sjuk i en magsjukdom med feberanfall. Paulus gick då in till honom och bad och lade händerna på honom och gjorde honom frisk.
\par 9 Men när detta hade skett, kommo också de av öns övriga inbyggare som hade någon sjukdom till honom och blevo botade.
\par 10 Och de bevisade oss ära på mångahanda sätt; och när vi skulle avsegla, försågo de oss med vad vi behövde.
\par 11 Då tre månader voro förlidna, avseglade vi på ett skepp som hade legat vid ön över vintern; det var från Alexandria och bar Tvillinggudarnas bilder.
\par 12 Och vi lade till vid Syrakusa och stannade där i tre dagar.
\par 13 Därifrån foro vi längs kusten och kommo till Regium. Dagen därefter fingo vi sunnanvind, och vi kommo så redan på andra dagen till Puteoli.
\par 14 Där träffade vi på bröder, och hos dem stannade vi, på deras inbjudning, i sju dagar. På detta sätt kommo vi till Rom.
\par 15 Så snart bröderna där fingo höra om oss, gingo de oss till mötes ända till Forum Appii och Tres Taberne. När Paulus fick se dem, tackade han Gud och fick nytt mod.
\par 16 Och då vi hade kommit in i Rom, tillstaddes det Paulus att bo för sig själv, med den krigsman som skulle bevaka honom.
\par 17 Tre dagar därefter kallade han tillhopa de förnämsta av judarna; och när de hade kommit tillsammans, sade han till dem: "Mina bröder, fastän jag icke har gjort något mot vårt folk eller mot fädernas stadgar, blev jag likväl i Jerusalem överlämnad i romarnas händer och fördes bort därifrån såsom fånge.
\par 18 Och när de hade anställt rannsakning med mig, ville de giva mig lös, eftersom jag icke hade gjort något som förtjänade döden.
\par 19 Men då judarna satte sig däremot, nödgades jag vädja till kejsaren; dock, icke som om jag hade någon anklagelse att göra mot mitt folk.
\par 20 Av denna orsak har jag kallat eder hit till mig, för att få se eder och tala med eder, ty det är för Israels hopps skull som jag är bunden med denna kedja."
\par 21 Då svarade de honom: "Vi hava icke från Judeen mottagit någon skrivelse om dig, ej heller har någon av våra bröder kommit och berättat eller sagt något ont om dig.
\par 22 Men vi finna skäligt att du låter oss höra huru du tänker. Ty om det partiet är oss bekant att det allestädes mötes med gensägelse.
\par 23 Sedan utsatte de en viss dag för honom, och på den kommo ännu flera till honom i hans härbärge. Då vittnade han för dem om Guds rike och utlade vad därtill hör, och försökte att övertyga dem i fråga om Jesus, med bevis både ur Moses' lag och ur profeterna; därmed höll han på från morgonen ända till aftonen.
\par 24 Och somliga läto övertyga sig av det som han sade, men andra trodde icke.
\par 25 Och då de icke kunde komma överens med varandra, gingo de sin väg, och därvid sade Paulus allenast detta ord: "Rätt talade den helige Ande genom profeten Esaias till edra fäder,
\par 26 när han sade: 'Gå åstad och säg till detta folk: Med hörande öron skolen I höra, och dock alls intet förstå, och med seende ögon skolen I se, och dock alls intet förnimma.
\par 27 Ty detta folks hjärta har blivit förstockat; och med öronen höra de illa, och sina ögon hava de tillslutit, så att de icke se med sina ögon eller höra med sina öron eller förstå med sina hjärtan och omvända sig och bliva helade av mig'
\par 28 Det mån I därför veta: till hedningarna bar denna Guds frälsning blivit sänd; de skola ock akta därpå."
\par 29 Och då han hade sagt detta, gingo judarna bort. Och de tvistade häftigt sinsemellan.
\par 30 I två hela år bodde han sedan kvar i en bostad som han själv hade hyrt. Och alla som kommo till honom tog han emot;
\par 31 och han predikade om Guds rike och undervisade om Herren Jesus Kristus med all frimodighet, utan att någon hindrade honom däri.


\end{document}