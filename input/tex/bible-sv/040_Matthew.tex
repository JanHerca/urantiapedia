\begin{document}

\title{Matteus}


\chapter{1}

\par 1 Detta är Jesu Kristi, Davids sons, Abrahams sons, släkttavla.
\par 2 Abraham födde Isak, Isak födde Jakob, Jakob födde Judas och hans bröder;
\par 3 Judas födde Fares och Sara med Tamar, Fares födde Esrom, Esrom födde Aram;
\par 4 Aram födde Aminadab, Aminadab födde Naasson, Naasson födde Salmon;
\par 5 Salmon födde Boes med Rakab, Boes födde Jobed med Rut, Jobed födde Jessai;
\par 6 Jessai födde David, konungen, David födde Salomo med Urias' hustru;
\par 7 Salomo födde Roboam, Roboam födde Abia. Abia födde Asaf;
\par 8 Asaf födde Josafat, Josafat födde Joram, Joram födde Osias;
\par 9 Osias födde Joatam, Joatam födde Akas, Akas födde Esekias;
\par 10 Esekias födde Manasses, Manasses födde Amos, Amos födde Josias;
\par 11 Josias födde Jekonias och hans bröder, vid den tid då folket blev bortfört i fångenskap till Babylonien.
\par 12 Sedan folket hade blivit bortfört i fångenskap till Babylonien, födde Jekonias Salatiel, Salatiel födde Sorobabel;
\par 13 Sorobabel födde Abiud, Abiud födde Eljakim, Eljakim födde Asor;
\par 14 Asor födde Sadok, Sadok födde Akim, Akim födde Eliud;
\par 15 Eliud födde Eleasar, Eleasar födde Mattan, Mattan födde Jakob;
\par 16 Jakob födde Josef, Marias man, och av henne föddes Jesus, som kallas Kristus.
\par 17 Så utgöra släktlederna från Abraham intill David tillsammans fjorton leder, och från David intill dess att folket blev bortfört i fångenskap till Babylonien fjorton leder, och från det att folket blev bortfört i fångenskap till Babylonien intill Kristus fjorton leder.
\par 18 Med Jesu Kristi födelse gick det så till. Sedan Maria, hans moder, hade blivit trolovad med Josef, befanns hon, förrän de kommo tillsammans, vara havande av helig ande.
\par 19 Nu var Josef, hennes man, en rättsinnig man och ville icke utsätta henne for vanära; därför beslöt han att hemligen skilja sig från henne.
\par 20 Men när han hade fått detta i sinnet, se, då visade sig i drömmen en Herrens ängel för honom och sade: "Josef, Davids son, frukta icke att taga till dig Maria, din hustru; ty det som är avlat i henne är av helig ande.
\par 21 Och hon skall föda en son, och honom skall du giva namnet Jesus, ty han skall frälsa sitt folk ifrån deras synder."
\par 22 Allt detta har skett, för att det skulle fullbordas, som var sagt av Herren genom profeten som sade:
\par 23 "Se, jungfrun skall bliva havande och föda en son, och man skall giva honom namnet Emmanuel" (det betyder Gud med oss).
\par 24 När Josef hade vaknat upp ur sömnen, gjorde han som Herrens ängel hade befallt honom och tog sin hustru till sig.
\par 25 Och han kände henne icke, förrän hon hade fött en son; och honom gav han namnet Jesus.

\chapter{2}

\par 1 När nu Jesus var född i Betlehem i Judeen, på konung Herodes' tid, då kommo vise män från österns länder till Jerusalem
\par 2 och sade: "Var är den nyfödde judakonungen? Vi hava nämligen sett hans stjärna i östern och hava kommit för att giva honom vår hyllning."
\par 3 När konung Herodes hörde detta, blev han förskräckt, och hela Jerusalem med honom.
\par 4 Och han församlade alla överstepräster och skriftlärde bland folket och frågade dem var Messias skulle födas.
\par 5 De svarade honom: "I Betlehem i Judeen; ty så är skrivet genom profeten:
\par 6 'Och du Betlehem, du judiska bygd, ingalunda är du minst bland Juda furstar, ty av dig skall utgå en furste som skall vara en herde för mitt folk Israel.'"
\par 7 Då kallade Herodes hemligen till sig de vise männen och utfrågade dem noga om tiden då stjärnan hade visat sig.
\par 8 Sedan lät han dem fara till Betlehem och sade: "Faren åstad och forsken noga efter barnet; och när I haven funnit det, så låten mig veta detta, för att också jag må komma och giva det min hyllning."
\par 9 När de hade hört konungens ord, foro de åstad; och se, stjärnan som de hade sett i östern gick framför dem, till dess att den kom över det ställe där barnet var. Där stannade den.
\par 10 Och när de sågo stjärnan, uppfylldes de av mycket stor glädje.
\par 11 Och de gingo in i huset och fingo se barnet med Maria, dess moder. Då föllo de ned och gåvo det sin hyllning; och de togo fram sina skatter och framburo åt det skänker: guld, rökelse och myrra.
\par 12 Sedan fingo de, genom en uppenbarelse i drömmen, befallning att icke återvända till Herodes; och de drogo så en annan väg tillbaka till sitt land.
\par 13 Men när de hade dragit åstad, se, då visade sig i drömmen en Herrens ängel för Josef och sade: "Stå upp och tag barnet och dess moder med dig, och fly till Egypten, och bliv kvar där, till dess jag säger dig till; ty Herodes tänker söka efter barnet för att förgöra det."
\par 14 Då stod han upp och tog barnet och dess moder med sig om natten, och drog bort till Egypten.
\par 15 Där blev han kvar intill Herodes' död, för att det skulle fullbordas, som var sagt av Herren genom profeten som sade: "Ut ur Egypten kallade jag min son."
\par 16 När Herodes nu såg att han hade blivit gäckad av de vise männen, blev han mycket vred. Och han sände åstad och lät döda alla de gossebarn i Betlehem och hela området däromkring, som voro två år gamla och därunder, detta enligt den uppgift om tiden, som han hade fått genom att utfråga de vise männen.
\par 17 Då fullbordades det som var sagt genom profeten Jeremias, när han sade:
\par 18 "Ett rop hördes i Rama, gråt och mycken jämmer; det var Rakel som begrät sina barn, och hon ville icke låta trösta sig, eftersom de icke mer voro till."
\par 19 Men när Herodes var död, se, då visade sig i drömmen en Herrens ängel for Josef, i Egypten,
\par 20 och sade: "Stå upp och tag barnet och dess moder med dig, och begiv dig till Israels land; ty de som traktade efter barnets liv äro nu döda."
\par 21 Då stod han upp och tog barnet och dess moder med sig, och kom så till Israels land.
\par 22 Men när han hörde att Arkelaus regerade över Judeen; efter sin fader Herodes, fruktade han att begiva sig dit; och på grund av en uppenbarelse i drömmen drog han bort till Galileens bygder.
\par 23 Och när han hade kommit dit, bosatte han sig i en stad som hette Nasaret, för att det skulle fullbordas, som var sagt genom profeterna, att han skulle kallas nasaré.

\chapter{3}

\par 1 Vid den tiden uppträdde Johannes döparen och predikade i Judeens öken
\par 2 och sade: "Gören bättring, ty himmelriket är nära."
\par 3 Det var om denne som profeten Esaias talade, när han sade: "Hör rösten av en som ropar i öknen: 'Bereden vägen för Herren, gören stigarna jämna för honom.'"
\par 4 Och Johannes hade kläder av kamelhår och bar en lädergördel om sina länder, och hans mat var gräshoppor och vildhonung.
\par 5 Och från Jerusalem och hela Judeen och hela trakten omkring Jordan gick då folket ut till honom
\par 6 och lät döpa sig av honom i floden Jordan, och bekände därvid sina synder.
\par 7 Men när han såg många fariséer och sadducéer komma för att låta döpa sig, sade han till dem: "I huggormars avföda, vem har ingivit eder att söka komma undan den tillstundande vredesdomen?
\par 8 Bären då ock sådan frukt som tillhör bättringen.
\par 9 Och menen icke att I kunnen säga vid eder själva: 'Vi hava ju Abraham till fader'; ty jag säger eder att Gud av dessa stenar kan uppväcka barn åt Abraham.
\par 10 Och redan är yxan satt till roten på träden; så bliver då vart träd som icke bär god frukt avhugget och kastat i elden.
\par 11 Jag döper eder i vatten till bättring, men den som kommer efter mig, han är starkare än jag, och jag är icke ens värdig att bära hans skor; han skall döpa eder i helig ande och eld.
\par 12 Han har sin kastskovel i handen, och han skall noga rensa sin loge och samla in sitt vete i ladan; men agnarna skall han bränna upp i en eld som icke utsläckes."
\par 13 Därefter kom Jesus från Galileen till Johannes, vid Jordan, för att låta döpa sig av honom;
\par 14 men denne ville hindra honom och sade: "Jag behövde döpas av dig, och du kommer till mig?"
\par 15 Då svarade Jesus och sade till honom: "Låt det nu så ske; ty det höves oss att så uppfylla all rättfärdighet. Då tillstadde han honom det.
\par 16 Och när Jesus var döpt, steg han strax upp ur vattnet; och se, då öppnades himmelen, och han såg Guds Ande sänka sig ned såsom en duva och komma över honom.
\par 17 Och från himmelen kom en röst, som sade: "Denne är min älskade Son, i vilken jag har funnit behag."

\chapter{4}

\par 1 Därefter blev Jesus av Anden förd upp i öknen, för att han skulle frestas av djävulen.
\par 2 Och när han hade fastat i fyrtio dagar och fyrtio nätter, blev han omsider hungrig.
\par 3 Då trädde frestaren fram och sade till honom: "Är du Guds Son, så bjud att dessa stenar bliva bröd."
\par 4 Men han svarade och sade: "Det är skrivet: 'Människan skall leva icke allenast av bröd, utan av allt det som utgår av Guds mun.'"
\par 5 Därefter tog djävulen honom med sig till den heliga staden och ställde honom uppe på helgedomens mur
\par 6 och sade till honom: "Är du Guds Son, så kasta dig ned; det är ju skrivet: 'Han skall giva sina änglar befallning om dig, och de skola bära dig på händerna, så att du icke stöter din fot mot någon sten.'"
\par 7 Jesus sade till honom: "Det är ock skrivet: 'Du skall icke fresta Herren, din Gud.'"
\par 8 Åter tog djävulen honom med sig, upp på ett mycket högt berg, och visade honom alla riken i världen och deras härlighet
\par 9 och sade till honom: "Allt detta vill jag giva dig, om du faller ned och tillbeder mig."
\par 10 Då sade Jesus till honom: "Gå bort, Satan; ty det är skrivet: 'Herren, din Gud, skall du tillbedja, och honom allena skall du tjäna.'"
\par 11 Då lämnade djävulen honom; och se, änglar trädde fram och betjänade honom.
\par 12 Men när han hörde att Johannes hade blivit satt i fängelse, drog han sig tillbaka till Galileen.
\par 13 Och han lämnade Nasaret och begav sig till Kapernaum, som ligger vid sjön, på Sabulons och Neftalims område, och bosatte sig där,
\par 14 för att det skulle fullbordas, som var sagt genom profeten Esaias, när han sade:
\par 15 "Sabulons land och Neftalims land, trakten åt havet till, landet på andra sidan Jordan, hedningarnas Galileen -
\par 16 det folk som där satt i mörker fick se ett stort ljus; ja, de som sutto i dödens ängd och skugga, för dem gick upp ett ljus."
\par 17 Från den tiden begynte Jesus predika och säga: "Gören bättring, ty himmelriket är nära."
\par 18 Då han nu vandrade utmed Galileiska sjön, fick han se två bröder, Simon, som kallas Petrus, och Andreas, hans broder, kasta ut nät i sjön, ty de voro fiskare.
\par 19 Och han sade till dem: "Följen mig så skall jag göra eder till människofiskare."
\par 20 Strax lämnade de näten och följde honom.
\par 21 När han hade gått därifrån ett stycke längre fram, fick han se två andra bröder, Jakob, Sebedeus' son, och Johannes, hans broder, där de jämte sin fader Sebedeus sutto i båten och ordnade sina nät; och han kallade dem till sig.
\par 22 Och strax lämnade de båten och sin fader och följde honom.
\par 23 Och han gick omkring i hela Galileen och undervisade i deras synagogor och predikade evangelium om riket och botade alla slags sjukdomar och allt slags skröplighet bland folket.
\par 24 Och ryktet om honom gick ut över hela Syrien, och man förde till honom alla sjuka som voro hemsökta av olika slags lidanden och plågor, alla som voro besatta eller månadsrasande eller lama; och han botade dem.
\par 25 Och honom följde mycket folk ifrån Galileen och Dekapolis och Jerusalem och Judeen och från landet på andra sidan Jordan.

\chapter{5}

\par 1 När han nu såg folket, gick han upp på berget; och sedan han hade satt sig ned, trädde hans lärjungar fram till honom.
\par 2 Då öppnade han sin mun och undervisade dem och sade:
\par 3 "Saliga äro de som äro fattiga i anden, ty dem hör himmelriket till.
\par 4 Saliga äro de som sörja, ty de skola bliva tröstade.
\par 5 Saliga äro de saktmodiga, ty de skola besitta jorden.
\par 6 Saliga äro de som hungra och törsta efter rättfärdighet, ty de skola bliva mättade.
\par 7 Saliga äro de barmhärtiga, ty dem skall vederfaras barmhärtighet.
\par 8 Saliga äro de renhjärtade, ty de skola se Gud.
\par 9 Saliga äro de fridsamma, ty de skola kallas Guds barn.
\par 10 Saliga äro de som lida förföljelse för rättfärdighets skull, ty dem hör himmelriket till.
\par 11 Ja, saliga ären I, när människorna för min skull smäda och förfölja eder och sanningslöst säga allt ont mot eder.
\par 12 Glädjens och fröjden eder, ty eder lön är stor i himmelen. Så förföljde man ju ock profeterna, som voro före eder.
\par 13 I ären jordens salt; men om saltet mister sin sälta, varmed skall man då giva det sälta igen? Till intet annat duger det än till att kastas ut och trampas ned av människorna.
\par 14 I ären världens ljus. Icke kan en stad döljas, som ligger uppe på ett berg?
\par 15 Ej heller tänder man ett ljus och sätter det under skäppan, utan man sätter det på ljusstaken, så att det lyser för alla dem som äro i huset.
\par 16 På samma sätt må ock edert ljus lysa inför människorna, så att de se edra goda gärningar och prisa eder Fader, som är i himmelen.
\par 17 I skolen icke mena att jag har kommit för att upphäva lagen eller profeterna. Jag har icke kommit för att upphäva, utan för att fullborda.
\par 18 Ty sannerligen säger jag eder: Intill dess himmel och jord förgås, skall icke den minsta bokstav, icke en enda prick av lagen förgås, förrän det allt har fullbordats.
\par 19 Därför, den som upphäver ett av de minsta bland dessa bud och lär människorna så, han skall räknas för en av de minsta i himmelriket; men den som håller dem och lär människorna så, han skall räknas för stor i himmelriket.
\par 20 Ty jag säger eder, att om eder rättfärdighet icke övergår de skriftlärdes och fariséernas, så skolen I icke komma in i himmelriket.
\par 21 I haven hört att det är sagt till de gamle: 'Du skall icke dräpa; och den som dräper, han är hemfallen åt Domstolens dom.'
\par 22 Men jag säger eder: Var och en som vredgas på sin broder, han är hemfallen åt Domstolens dom; men den som säger till sin broder: 'Du odåga', han är hemfallen åt Stora rådets dom; och den som säger: 'Du dåre', han är hemfallen åt det brinnande Gehenna.
\par 23 Därför, om du kommer med din gåva till altaret, och där drager dig till minnes att din broder har något emot dig,
\par 24 så lägg ned din gåva där framför altaret, och gå först bort och förlik dig med din broder, och kom sedan och bär fram din gåva.
\par 25 Var villig till snar förlikning med din motpart, medan du ännu är med honom på vägen, så att din motpart icke drager dig inför domaren, och domaren överlämnar dig åt rättstjänaren, och du bliver kastad i fängelse.
\par 26 Sannerligen säger jag dig: Du skall icke slippa ut därifrån, förrän du har betalt den yttersta skärven.
\par 27 I haven hört att det är sagt: 'Du skall icke begå äktenskapsbrott.'
\par 28 Men jag säger eder: Var och en som med begärelse ser på en annans hustru, han har redan begått äktenskapsbrott med henne i sitt hjärta.
\par 29 Om nu ditt högra öga är dig till förförelse, så riv ut det och kasta det ifrån dig; ty det är bättre för dig att en av dina lemmar fördärvas, än att hela din kropp kastas i Gehenna.
\par 30 Och om din högra hand är dig till förförelse, så hugg av den och kasta den ifrån dig; ty det är bättre för dig att en av dina lemmar fördärvas, än att hela din kropp kommer till Gehenna.
\par 31 Det är ock sagt: 'Den som vill skilja sig från sin hustru han skall giva henne skiljebrev.'
\par 32 Men jag säger eder: Var och en som skiljer sig från sin hustru för någon annan saks skull än för otukt, han bliver orsak till att äktenskapsbrott begås med henne. Och den som tager en frånskild kvinna till hustru, han begår äktenskapsbrott.
\par 33 Ytterligare haven I hört att det är sagt till de gamle: 'Du skall icke svärja falskt' och 'Du skall hålla din ed inför Herren.'
\par 34 Men jag säger eder att I alls icke skolen svärja, varken vid himmelen, ty den är 'Guds tron',
\par 35 ej heller vid jorden, ty den är 'hans fotapall', ej heller vid Jerusalem, ty det är 'den store Konungens stad';
\par 36 ej heller må du svärja vid ditt huvud, ty du kan icke göra ett enda hår vare sig vitt eller svart;
\par 37 utan sådant skall edert tal vara, att ja är ja, och nej är nej. Vad därutöver är, det är av ondo.
\par 38 I haven hört att det är sagt: 'Öga för öga och tand för tand.'
\par 39 Men jag säger eder att I icke skolen stå emot en oförrätt; utan om någon slår dig på den högra kinden, så vänd ock den andra till åt honom;
\par 40 och om någon vill gå till rätta med dig för att beröva dig din livklädnad, så låt honom få manteln med;
\par 41 och om någon tvingar dig att till hans tjänst gå med en mil, så gå två med honom.
\par 42 Giv åt den som beder dig, och vänd dig icke bort ifrån den som vill låna av dig.
\par 43 I haven hört att det är sagt: 'Du skall älska din nästa och hata din ovän.'
\par 44 Men jag säger eder: Älsken edra ovänner, och bedjen för dem som förfölja eder,
\par 45 och varen så eder himmelske Faders barn; han låter ju sin sol gå upp över både onda och goda och låter det regna över både rättfärdiga och orättfärdiga.
\par 46 Ty om I älsken dem som älska eder, vad lön kunnen I få därför? Göra icke publikanerna detsamma?
\par 47 Och om I visen vänlighet mot edra bröder allenast, vad synnerligt gören I därmed? Göra icke hedningarna detsamma?
\par 48 Varen alltså I fullkomliga, såsom eder himmelske Fader är fullkomlig."

\chapter{6}

\par 1 "Tagen eder till vara för att öva eder rättfärdighet inför människorna, för att bliva sedda av dem; annars haven I ingen lön hos eder Fader, som är i himmelen.
\par 2 Därför, när du giver en allmosa, så låt icke stöta i basun för dig, såsom skrymtarna göra i synagogorna och på gatorna, för att de skola bliva prisade av människorna. Sannerligen säger jag eder: De hava fått ut sin lön.
\par 3 Nej, när du giver en allmosa, låt då din vänstra hand icke få veta vad den högra gör,
\par 4 så att din allmosa gives i det fördolda. Då skall din Fader, som ser i det fördolda, vedergälla dig.
\par 5 Och när I bedjen, skolen I icke vara såsom skrymtarna, vilka gärna stå i synagogorna och i gathörnen och bedja, för att bliva sedda av människorna. Sannerligen säger jag eder: De hava fått ut sin lön.
\par 6 Nej, när du vill bedja, gå då in i din kammare, och stäng igen din dörr, och bed till din Fader i det fördolda. Då skall din Fader, som ser i det fördolda, vedergälla dig.
\par 7 Men i edra böner skolen I icke hopa tomma ord såsom hedningarna, vilka mena att de skola bliva bönhörda för sina många ords skull.
\par 8 Så varen då icke lika dem; eder Fader vet ju vad I behöven, förrän I bedjen honom.
\par 9 I skolen alltså bedja sålunda: 'Fader vår, som är i himmelen! Helgat varde ditt namn;
\par 10 tillkomme ditt rike; ske din vilja, såsom i himmelen, så ock på jorden;
\par 11 vårt dagliga bröd giv oss i dag;
\par 12 och förlåt oss våra skulder, såsom ock vi förlåta dem oss skyldiga äro;
\par 13 och inled oss icke i frestelse, utan fräls oss ifrån ondo.'
\par 14 Ty om I förlåten människorna deras försyndelser, så skall ock eder himmelske Fader förlåta eder;
\par 15 men om I icke förlåten människorna, så skall ej heller eder Fader förlåta edra försyndelser.
\par 16 Och när I fasten, skolen I icke visa en bedrövad uppsyn såsom skrymtarna, vilka vanställa sina ansikten för att bliva sedda av människorna med sin fasta. Sannerligen säger jag eder: De hava fått ut sin lön.
\par 17 Nej, när du fastar, smörj då ditt huvud och två ditt ansikte,
\par 18 för att du icke må bliva sedd av människorna med din fasta, utan allenast av din Fader, som är i det fördolda. Då skall din Fader, som ser i det fördolda, vedergälla dig.
\par 19 Samlen eder icke skatter på jorden, där mott och mal förstöra, och där tjuvar bryta sig in och stjäla,
\par 20 utan samlen eder skatter i himmelen, där mott och mal icke förstöra, och där inga tjuvar bryta sig in och stjäla.
\par 21 Ty där din skatt är, där kommer ock ditt hjärta att vara.
\par 22 Ögat är kroppens lykta. Om nu ditt öga är friskt, så får hela din kropp ljus.
\par 23 Men om ditt öga är fördärvat, då bliver hela din kropp höljd i mörker. Är det nu så, att ljuset, som du har i dig, är mörker, huru djupt bliver då icke mörkret!
\par 24 Ingen kan tjäna två herrar; ty antingen kommer han då att hata den ene och älska den andre, eller kommer han att hålla sig till den förre och förakta den senare. I kunnen icke tjäna både Gud och Mamon.
\par 25 Därför säger jag eder: Gören eder icke bekymmer för edert liv, vad I skolen äta eller dricka, ej heller för eder kropp, vad I skolen kläda eder med. Är icke livet mer än maten, och kroppen mer än kläderna?
\par 26 Sen på fåglarna under himmelen: de så icke, ej heller skörda de, ej heller samla de in i lador; och likväl föder eder himmelske Fader dem. Ären I icke mycket mer än de?
\par 27 Vilken av eder kan, med allt sitt bekymmer, lägga en enda aln till sin livslängd?
\par 28 Och varför bekymren I eder för kläder? Beskåden liljorna på marken, huru de växa: de arbeta icke, ej heller spinna de;
\par 29 och likväl säger jag eder att icke ens Salomo i all sin härlighet var så klädd som en av dem.
\par 30 Kläder nu Gud så gräset på marken, vilket i dag står och i morgon kastas i ugnen, skulle han då icke mycket mer kläda eder, I klentrogne?
\par 31 Så gören eder nu icke bekymmer, och sägen icke: 'Vad skola vi äta?' eller: 'Vad skola vi dricka?' eller: 'Vad skola vi kläda oss med?'
\par 32 Efter allt detta söka ju hedningarna, och eder himmelske Fader vet att I behöven allt detta.
\par 33 Nej, söken först efter hans rike och hans rättfärdighet, så skall också allt detta andra tillfalla eder.
\par 34 Gören eder alltså icke bekymmer för morgondagen, ty morgondagen skall själv bära sitt bekymmer. Var dag har nog av sin egen plåga."

\chapter{7}

\par 1 "Dömen icke, på det att I icke mån bliva dömda;
\par 2 ty med den dom varmed I dömen skolen I bliva dömda, och med det mått som I mäten med skall ock mätas åt eder.
\par 3 Huru kommer det till, att du ser grandet i din broders öga, men icke bliver varse bjälken i ditt eget öga?
\par 4 Eller huru kan du säga till din broder: 'Låt mig taga ut grandet ur ditt öga', du som har en bjälke i ditt eget öga?
\par 5 Du skrymtare, tag först ut bjälken ur ditt eget öga; därefter må du se till, att du kan taga ut grandet ur din broders öga.
\par 6 Given icke åt hundarna vad heligt är, och kasten icke edra pärlor för svinen, på det att dessa icke må trampa dem under fötterna och sedan vända sig om och sarga eder.
\par 7 Bedjen, och eder skall varda givet; söken, och I skolen finna; klappen, och för eder skall varda upplåtet.
\par 8 Ty var och en som beder, han får; och den som söker, han finner; och för den som klappar skall varda upplåtet.
\par 9 Eller vilken är den man bland eder, som räcker sin son en sten, när han beder honom om bröd,
\par 10 eller som räcker honom en orm, när han beder om fisk?
\par 11 Om nu I, som ären onda, förstån att giva edra barn goda gåvor, huru mycket mer skall icke då eder Fader, som är i himmelen, giva vad gott är åt dem som bedja honom!
\par 12 Därför, allt vad I viljen att människorna skola göra eder, det skolen I ock göra dem; ty detta är lagen och profeterna.
\par 13 Gån in genom den trånga porten. Ty vid och bred är den väg som leder till fördärvet, och många äro de som gå fram på den;
\par 14 och den port är trång och den väg är smal, som leder till livet, och få äro de som finna den.
\par 15 Tagen eder till vara för falska profeter, som komma till eder i fårakläder, men invärtes äro glupande ulvar.
\par 16 Av deras frukt skolen I känna dem. Icke hämtar man väl vindruvor från törnen, eller fikon från tistlar?
\par 17 Så bär vart och ett gott träd god frukt, men ett dåligt träd bär ond frukt.
\par 18 Ett gott träd kan icke bära ond frukt, ej heller kan ett dåligt träd bära god frukt.
\par 19 Vart träd som icke bär god frukt bliver avhugget och kastat i elden.
\par 20 Alltså skolen I känna dem av deras frukt. -
\par 21 Icke kommer var och en in i himmelriket, som säger till mig: 'Herre, Herre', utan den som gör min himmelske Faders vilja.
\par 22 Många skola på 'den dagen' säga till mig: 'Herre, Herre, hava vi icke profeterat i ditt namn och genom ditt namn drivit ut onda andar och genom ditt namn gjort många kraftgärningar?'
\par 23 Men då skall jag betyga för dem: 'Jag har aldrig känt eder; gån bort ifrån mig, I ogärningsmän.'
\par 24 Därför, var och en som hör dessa mina ord och gör efter dem, han må liknas vid en förståndig man som byggde sitt hus på hälleberget.
\par 25 Och slagregn föll, och vattenströmmarna kommo, och vindarna blåste och kastade sig mot det huset; och likväl föll det icke omkull, eftersom det var grundat på hälleberget.
\par 26 Men var och en som hör dessa mina ord och icke gör efter dem, han må liknas vid en oförståndig man som byggde sitt hus på sanden.
\par 27 Och slagregn föll, och vattenströmmarna kommo, och vindarna blåste och slogo mot det huset; och det föll omkull, och dess fall var stort."
\par 28 När Jesus hade slutat detta tal, häpnade folket över hans förkunnelse;
\par 29 ty han förkunnade sin lära för dem med makt och myndighet, och icke såsom deras skriftlärde.

\chapter{8}

\par 1 Sedan han hade kommit ned från berget, följde honom mycket folk.
\par 2 Då trädde en spetälsk man fram och föll ned för honom och sade: "Herre, vill du, så kan du göra mig ren."
\par 3 Då räckte han ut handen och rörde vid honom och sade: "Jag vill; bliv ren." Och strax blev han ren från sin spetälska.
\par 4 Och Jesus sade till honom: "Se till, att du icke säger detta för någon; men gå bort och visa dig för prästen, och frambär den offergåva som Moses har påbjudit, till ett vittnesbörd för dem."
\par 5 När han därefter kom in i Kapernaum, trädde en hövitsman fram till honom och bad honom
\par 6 och sade: "Herre, min tjänare ligger därhemma lam och lider svårt."
\par 7 Han sade till honom: "Skall då jag komma och bota honom?"
\par 8 Hövitsmannen svarade och sade: "Herre, jag är icke värdig att du går in under mitt tak. Men säg allenast ett ord, så bliver min tjänare frisk.
\par 9 Jag är ju själv en man som står under andras befäl: jag har ock krigsmän under mig, och om jag säger till en av dem: 'Gå', så går han, eller till en annan: 'Kom', så kommer han; och om jag säger till min tjänare: 'Gör det', då gör han så."
\par 10 När Jesus hörde detta, förundrade han sig och sade till dem som följde honom: "Sannerligen säger jag eder: I Israel har jag icke hos någon funnit så stor tro.
\par 11 Och jag säger eder: Många skola komma från öster och väster och få vara med Abraham, Isak och Jakob till bords i himmelriket,
\par 12 men rikets barn skola bliva utkastade i mörkret därutanför; där skall vara gråt och tandagnisslan."
\par 13 Och Jesus sade till hövitsmannen: "Gå; såsom du tror, så må det ske dig." Och i samma stund blev tjänaren frisk.
\par 14 När Jesus sedan kom in i Petrus' hus, fick han se hans svärmoder ligga sjuk i feber.
\par 15 Då rörde han vid hennes hand, och febern lämnade henne; och hon stod upp och betjänade honom.
\par 16 Men när det hade blivit afton, förde man till honom många som voro besatta; och han drev ut andarna med sitt blotta ord, och alla som voro sjuka botade han,
\par 17 för att det skulle fullbordas, som var sagt genom profeten Esaias, när han sade: "Han tog på sig våra krankheter, och våra sjukdomar bar han."
\par 18 Då nu Jesus såg mycket folk omkring sig, bjöd han att man skulle fara över till andra stranden.
\par 19 Och en skriftlärd kom fram och sade till honom: "Mästare, jag vill följa dig varthelst du går."
\par 20 Då svarade Jesus honom: "Rävarna hava kulor, och himmelens fåglar hava nästen; men Människosonen har ingen plats där han kan vila sitt huvud."
\par 21 Och en annan av hans lärjungar sade till honom: "Herre, tillstäd mig att först gå bort och begrava min fader."
\par 22 Då svarade Jesus honom: "Följ du mig, och låt de döda begrava sina döda."
\par 23 Och han steg i båten, och hans lärjungar följde honom.
\par 24 Och se, då uppstod en häftig storm på sjön, så att vågorna slogo över båten; men han låg och sov.
\par 25 Då gingo de fram och väckte honom och sade: "Herre, hjälp oss; vi förgås."
\par 26 Han sade till dem: "I klentrogne, varför rädens I?" Därefter stod han upp och näpste vindarna och sjön, och det blev alldeles lugnt.
\par 27 Och människorna förundrade sig och sade: "Vad är denne för en, eftersom både vindarna och sjön äro honom lydiga?"
\par 28 När han så hade kommit över till gadarenernas land på andra stranden, kommo två besatta emot honom, ut från gravarna där. Och de voro mycket våldsamma, så att ingen kunde färdas den vägen fram.
\par 29 Dessa ropade då och sade: "Vad har du med oss att göra, du Guds Son? Har du kommit hit för att plåga oss, förrän tid är?"
\par 30 Nu gick där långt ifrån dem en stor svinhjord i bet.
\par 31 Och de onda andarna bådo honom och sade: "Om du vill driva ut oss så låt oss fara in i svinhjorden."
\par 32 Då sade han till dem: "Faren åstad." Och de gåvo sig åstad och foro in i svinen. Och se, då störtade sig hela hjorden utför branten ned i sjön och omkom i vattnet.
\par 33 Men herdarna flydde; och när de hade kommit in i staden, omtalade de alltsammans, och särskilt vad som hade skett med de besatta.
\par 34 Då gick hela staden ut för att möta Jesus; och när de fingo se honom, bådo de att han skulle begiva sig bort ifrån deras område.

\chapter{9}

\par 1 Och han steg i en båt och for över och kom till sin egen stad.
\par 2 Då förde de till honom en lam man, som låg på en säng. När Jesus såg deras tro sade han till den lame: "Var vid gott mod, min son; dina synder förlåtas dig."
\par 3 Då sade några av de skriftlärde vid sig själva: "Denne hädar."
\par 4 Men Jesus förstod deras tankar och sade: "Varför tänken I i edra hjärtan vad ont är?
\par 5 Vilket är lättare, att säga: 'Dina synder förlåtas dig' eller att säga: 'Stå upp och gå'?
\par 6 Men för att I skolen veta att Människosonen har makt här på jorden att förlåta synder, så stå upp" - sade han nu till den lame - "och tag din säng och gå hem."
\par 7 Då stod han upp och gick hem.
\par 8 När folket såg detta, blevo de häpna och prisade Gud, som hade givit sådan makt åt människor.
\par 9 När Jesus därifrån gick vidare fram, fick han se en man, som hette Matteus, sitta vid tullhuset. Och han sade till denne: "Följ mig." Då stod han upp och följde honom.
\par 10 När han därefter låg till bords i hans hus, kommo många publikaner och syndare dit och voro bordsgäster där, jämte Jesus och hans lärjungar.
\par 11 Men då fariséerna sågo detta, sade de till hans lärjungar: "Huru kan eder mästare äta med publikaner och syndare?"
\par 12 När han hörde detta, sade han: "Det är icke de friska som behöva läkare, utan de sjuka.
\par 13 Men gån I åstad och lären eder vad de orden betyda: 'Jag har behag till barmhärtighet, och icke till offer.' Ty jag har icke kommit för att kalla rättfärdiga, utan för att kalla syndare."
\par 14 Därefter kommo Johannes' lärjungar till honom och sade: "Varför fasta icke dina lärjungar då vi och fariséerna ofta fasta?"
\par 15 Jesus svarade dem: "Icke kunna väl bröllopsgästerna sörja, så länge brudgummen är hos dem? Men den tid skall komma, då brudgummen tages ifrån dem, och då skola de fasta. -
\par 16 Ingen sätter en lapp av okrympt tyg på en gammal mantel, ty det isatta stycket skulle riva bort ännu mer av manteln, och hålet skulle bliva värre.
\par 17 Ej heller slår man nytt vin I gamla skinnläglar; om någon så gjorde, skulle läglarna sprängas sönder och vinet spillas ut, jämte det att läglarna fördärvades. Nej, man slår nytt vin i nya läglar, så bliva båda delarna bevarade."
\par 18 Medan han talade detta till dem, trädde en synagogföreståndare fram och föll ned för honom och sade: "Min dotter har just nu dött, men kom och lägg din hand på henne, så bliver hon åter levande."
\par 19 Då stod Jesus upp och följde honom med sina lärjungar.
\par 20 Men en kvinna, som i tolv år hade lidit av blodgång, närmade sig honom bakifrån och rörde vid hörntofsen på hans mantel.
\par 21 Ty hon sade vid sig själv: "Om jag allenast får röra vid hans mantel, så bliver jag hulpen."
\par 22 Då vände Jesus sig om, och när han fick se henne, sade han: "Var vid gott mod, min dotter; din tro har hjälpt dig." Och kvinnan var hulpen från den stunden.
\par 23 När Jesus sedan kom in i föreståndarens hus och fick se flöjtblåsarna och folket som höjde klagolåt,
\par 24 sade han: "Gån bort härifrån; ty flickan är icke död, hon sover." Då hånlogo de åt honom.
\par 25 Men sedan folket hade blivit utvisat, gick han in och tog flickan vid handen. Då stod hon upp.
\par 26 Och ryktet härom gick ut över hela det landet.
\par 27 När Jesus gick därifrån, följde honom två blinda som ropade och sade: "Davids son, förbarma dig över oss."
\par 28 Och då han kom hem, trädde de blinda fram till honom; och Jesus frågade dem: "Tron I att jag kan göra detta?" De svarade honom: "Ja, Herre."
\par 29 Då rörde han vid deras ögon och sade: "Ske eder efter eder tro."
\par 30 Och deras ögon öppnades. Och Jesus tillsade dem strängeligen att se till, att ingen finge veta detta.
\par 31 Men de gingo åstad och utspridde ryktet om honom över hela det landet.
\par 32 När dessa voro på väg ut, förde man till honom en dövstum som var besatt.
\par 33 Och när den onde anden hade blivit utdriven, talade den dövstumme. Och folket förundrade sig och sade: "Sådant har aldrig förut varit sett i Israel."
\par 34 Men fariséerna sade: "Det är med de onda andarnas furste som han driver ut de onda andarna."
\par 35 Och Jesus gick omkring i alla städer och byar och undervisade i deras synagogor och predikade evangelium om riket och botade alla slags sjukdomar och allt slags skröplighet.
\par 36 Och när han såg folkskarorna, ömkade han sig över dem, eftersom de voro så illa medfarna och uppgivna, "lika får som icke hava någon herde."
\par 37 Därför sade han till sina lärjungar: "Skörden är mycken, men arbetarna äro få.
\par 38 Bedjen fördenskull skördens Herre att han sänder ut arbetare till sin skörd."

\chapter{10}

\par 1 Och han kallade till sig sina tolv lärjungar och gav dem makt över orena andar, till att driva ut dem, så ock makt att bota alla slags sjukdomar och allt slags skröplighet.
\par 2 Och dessa äro de tolv apostlarnas namn: först Simon, som kallas Petrus, och Andreas, hans broder; vidare Jakob, Sebedeus' son, och Johannes, hans broder;
\par 3 Filippus och Bartolomeus; Tomas och Matteus, publikanen; Jakob, Alfeus' son, och Lebbeus;
\par 4 Simon ivraren och Judas Iskariot, densamme som förrådde honom.
\par 5 Dessa tolv sände Jesus ut; och han bjöd dem och sade: "Ställen icke eder färd till hedningarna, och gån icke in i någon samaritisk stad,
\par 6 utan gån hellre till de förlorade fåren av Israels hus.
\par 7 Och där I gån fram skolen I predika och säga: 'Himmelriket är nära.'
\par 8 Boten sjuka, uppväcken döda, gören spetälska rena, driven ut onda andar. I haven fått för intet; så given ock för intet.
\par 9 Skaffen eder icke guld eller silver eller koppar i edra bälten,
\par 10 icke någon ränsel för eder färd, ej heller dubbla livklädnader, ej heller skor eller stav; ty arbetaren är värd sin mat.
\par 11 Men när I haven kommit in i någon stad eller by, så utforsken vilken därinne som är värdig, och stannen hos honom, till dess I lämnen den orten.
\par 12 Och när I kommen in i ett hus, så hälsen det.
\par 13 Om då det huset är värdigt, så må den frid I tillönsken det komma däröver; men om det icke är värdigt, då må den frid I tillönsken det vända tillbaka till eder.
\par 14 Och om man på något ställe icke tager emot eder och icke hör på edra ord, så gån ut ur det huset eller den staden, och skudden stoftet av edra fötter.
\par 15 Sannerligen säger jag eder: För Sodoms och Gomorras land skall det på domens dag bliva drägligare än för den staden.
\par 16 Se, jag sänder eder åstad såsom får mitt in ibland ulvar. Varen fördenskull kloka såsom ormar och menlösa såsom duvor.
\par 17 Tagen eder till vara för människorna; ty de skola draga eder inför domstolar, och i sina synagogor skola de gissla eder;
\par 18 och I skolen föras fram också inför landshövdingar och konungar, för min skull, till ett vittnesbörd för dem och för hedningarna.
\par 19 Men när man drager eder inför rätta, gören eder då icke bekymmer för huru eller vad I skolen tala; ty vad I skolen tala skall bliva eder givet i den stunden.
\par 20 Det är icke I som skolen tala, utan det är eder Faders Ande som skall tala i eder.
\par 21 Och den ene brodern skall då överlämna den andre till att dödas, ja ock fadern sitt barn; och barn skola sätta sig upp mot sina föräldrar och skola döda dem.
\par 22 Och I skolen bliva hatade av alla, för mitt namns skull. Men den som är ståndaktig intill änden, han skall bliva frälst. -
\par 23 När de nu förfölja eder i en stad, så flyn till en annan; och om de också där förfölja eder, så flyn till ännu en annan. Ty sannerligen säger jag eder: I skolen icke hava hunnit igenom alla Israels städer, förrän Människosonen kommer.
\par 24 Lärjungen är icke förmer än sin mästare, ej heller är tjänaren förmer än sin herre.
\par 25 Det må vara lärjungen nog, om det går honom såsom hans mästare, och tjänaren, om det går honom såsom hans herre. Om de hava kallat husbonden för Beelsebul, huru mycket mer skola de icke så kalla hans husfolk!
\par 26 Frukten alltså icke för dem; ty intet är förborgat, som icke skall bliva uppenbarat, och intet är fördolt, som icke skall bliva känt.
\par 27 Vad jag säger eder i mörkret, det skolen säga i ljuset, och vad I hören viskas i edert öra, det skolen I predika på taken.
\par 28 Och frukten icke för dem som väl kunna dräpa kroppen, men icke hava makt att dräpa själen, utan frukten fastmer honom som har makt att förgöra både själ och kropp i Gehenna. -
\par 29 Säljas icke två sparvar för en skärv? Och icke en av dem faller till jorden utan eder Faders vilja.
\par 30 Men på eder äro till och med huvudhåren allasammans räknade.
\par 31 Frukten alltså icke; I ären mer värda än många sparvar.
\par 32 Därför, var och en som bekänner mig inför människorna, honom skall ock jag kännas vid inför min Fader, som är i himmelen.
\par 33 Men den som förnekar mig inför människorna, honom skall ock jag förneka inför min Fader, som är i himmelen.
\par 34 I skolen icke mena att jag har kommit för att sända frid på jorden. Jag har icke kommit för att sända frid, utan svärd.
\par 35 Ja, jag har kommit för att uppväcka söndring, så att 'sonen sätter sig upp mot sin fader och dottern mot sin moder och sonhustrun mot sin svärmoder,
\par 36 och envar får sitt eget husfolk till fiender'.
\par 37 Den som älskar fader eller moder mer än mig, han är mig icke värdig, och den som älskar son eller dotter mer än mig, han är mig icke värdig;
\par 38 och den som icke tager sitt kors på sig och efterföljer mig, han är mig icke värdig.
\par 39 Den som finner sitt liv, han skall mista det, och den som mister sitt liv, för min skull, han skall finna det. -
\par 40 Den som tager emot eder, han tager emot mig, och den som tager emot mig, han tager emot honom som har sänt mig.
\par 41 Den som tager emot en profet, därför att det är en profet, han skall få en profets lön; och den som tager emot en rättfärdig man, därför att det är en rättfärdig man, han skall få en rättfärdig mans lön.
\par 42 Och den som giver en av dessa små allenast en bägare friskt vatten att dricka, därför att det är en lärjunge - sannerligen säger jag eder: Han skall ingalunda gå miste om sin lön."

\chapter{11}

\par 1 När Jesus hade givit sina tolv lärjungar alla dessa bud, gick han därifrån vidare, för att undervisa och predika i deras städer.
\par 2 Men när Johannes i sitt fängelse fick höra om Kristi gärningar, sände han bud med sina lärjungar
\par 3 och lät fråga honom: "Är du den som skulle komma, eller skola vi förbida någon annan?"
\par 4 Då svarade Jesus och sade till dem: "Gån tillbaka och omtalen för Johannes vad I hören och sen:
\par 5 blinda få sin syn, halta gå, spetälska bliva rena, döva höra, döda uppstå, och 'för fattiga förkunnas glädjens budskap'.
\par 6 Och salig är den för vilken jag icke bliver en stötesten."
\par 7 När dessa sedan gingo bort, begynte Jesus tala till folket om Johannes: "Varför var det I gingen ut i öknen? Var det för att se ett rör som drives hit och dit av vinden?
\par 8 Eller varför gingen I ut? Var det för att se en människa klädd i fina kläder? De som bära fina kläder, dem finnen I ju i konungapalatsen.
\par 9 Varför gingen I då ut? Var det för att se en profet? Ja, jag säger eder: Ännu mer än en profet är han.
\par 10 Han är den om vilken det är skrivet: 'Se, jag sänder ut min ängel framför dig, och han skall bereda vägen för dig.'
\par 11 Sannerligen säger jag eder: Bland dem som äro födda av kvinnor har ingen uppstått, som har varit större än Johannes döparen; men den som är minst i himmelriket är likväl större än han.
\par 12 Och från Johannes döparens dagar intill denna stund tränger himmelriket fram med storm, och människor storma fram och rycka det till sig.
\par 13 Ty alla profeterna och lagen hava profeterat intill Johannes;
\par 14 och om I viljen tro det: han är Elias, den som skulle komma.
\par 15 Den som har öron, han höre.
\par 16 Men vad skall jag likna detta släkte vid? Det är likt barn som sitta på torgen och ropa till andra barn
\par 17 och säga: 'Vi hava spelat för eder, och I haven icke dansat; vi hava sjungit sorgesång, och I haven icke jämrat eder.'
\par 18 Ty Johannes kom, och han varken äter eller dricker, och så säger man: 'Han är besatt av en ond ande.'
\par 19 Människosonen kom, och han både äter och dricker, och nu säger man: 'Se vilken frossare och vindrinkare han är, en publikaners och syndares vän!' Men Visheten har fått rätt av sina barn."
\par 20 Därefter begynte han tala bestraffande ord till de städer i vilka han hade utfört så många av sina kraftgärningar, och förehålla dem att de icke hade gjort bättring:
\par 21 "Ve dig, Korasin! Ve dig, Betsaida! Ty om de kraftgärningar som äro gjorda i eder hade blivit gjorda i Tyrus och Sidon, så skulle de för länge sedan hava gjort bättring i säck och aska.
\par 22 Men jag säger eder: För Tyrus och Sidon skall det på domens dag bliva drägligare än för eder.
\par 23 Och du, Kapernaum, skall väl du bliva upphöjt till himmelen? Nej, ned till dödsriket måste du fara. Ty om de kraftgärningar som äro gjorda i dig hade blivit gjorda i Sodom, så skulle det hava stått ännu i dag.
\par 24 Men jag säger eder att det för Sodoms land skall på domens dag bliva drägligare än för dig."
\par 25 Vid den tiden talade Jesus och sade: "Jag prisar dig, Fader, du himmelens och jordens Herre, för att du väl har dolt detta för de visa och kloka, men uppenbarat det för de enfaldiga.
\par 26 Ja, Fader; så har ju varit ditt behag.
\par 27 Allt har av min Fader blivit förtrott åt mig. Och ingen känner Sonen utom Fadern, ej heller känner någon Fadern utom Sonen och den för vilken Sonen vill göra honom känd. -
\par 28 Kommen till mig, I alla som arbeten och ären betungade, så skall jag giva eder ro.
\par 29 Tagen på eder mitt ok och lären av mig, ty jag är saktmodig och ödmjuk i hjärtat; 'så skolen I finna ro för edra själar'.
\par 30 Ty mitt ok är milt, och min börda är lätt."

\chapter{12}

\par 1 Vid den tiden tog Jesus på sabbaten vägen genom ett sädesfält; och hans lärjungar blevo hungriga och begynte rycka av ax och äta.
\par 2 När fariséerna sågo detta, sade de till honom: "Se, dina lärjungar göra vad som icke är lovligt att göra på en sabbat."
\par 3 Han svarade dem: "Haven I icke läst vad David gjorde, när han och de som följde honom blevo hungriga:
\par 4 huru han då gick in i Guds hus, och huru de åto skådebröden, fastän det ju varken för honom eller för dem som följde honom var lovligt att äta sådant bröd, utan allenast för prästerna?
\par 5 Eller haven I icke läst i lagen att prästerna på sabbaten bryta sabbaten i helgedomen, och likväl äro utan skuld?
\par 6 Men jag säger eder: Här är vad som är förmer än helgedomen.
\par 7 Och haden I förstått vad det är: 'Jag har behag till barmhärtighet, och icke till offer', så skullen I icke hava dömt dem skyldiga, som äro utan skuld.
\par 8 Ty Människosonen är herre över sabbaten."
\par 9 Och han gick därifrån vidare och kom in i deras synagoga.
\par 10 Och se, där var en man som hade en förvissnad hand. Då frågade de honom och sade: "Är det lovligt att bota sjuka på sabbaten?" De ville nämligen få något att anklaga honom för.
\par 11 Men han sade till dem: "Om någon bland eder har ett får, och detta på sabbaten faller i en grop, fattar han icke då i det och drager upp det?
\par 12 Huru mycket mer värd är nu icke en människa än ett får! Alltså är det lovligt att på sabbaten göra vad gott är."
\par 13 Därefter sade han till mannen: "Räck ut din hand." Och han räckte ut den, och den blev frisk igen och färdig såsom den andra. -
\par 14 Då gingo fariséerna bort och fattade det beslutet om honom, att de skulle förgöra honom.
\par 15 Men när Jesus fick veta detta, gick han bort därifrån; och många följde honom, och han botade dem alla,
\par 16 men förbjöd dem strängeligen att utbreda ryktet om honom.
\par 17 Ty det skulle fullbordas, som var sagt genom profeten Esaias, när han sade:
\par 18 "Se, över min tjänare, som jag har utvalt, min älskade, i vilken min själ har funnit behag, över honom skall jag låta min Ande komma, och han skall förkunna rätten bland folken.
\par 19 Han skall icke kiva eller skria, och hans röst skall man icke höra på gatorna,
\par 20 Ett brutet rör skall han icke sönderkrossa, och en rykande veke skall han icke utsläcka, intill dess att han har fört rätten fram till seger.
\par 21 Och till hans namn skola folken sätta sitt hopp."
\par 22 Då förde man till honom en besatt, som var blind och dövstum. Och han botade honom, så att den dövstumme talade och såg.
\par 23 Och allt folket uppfylldes av häpnad och sade: "Månne icke denne är Davids son?"
\par 24 Men när fariséerna hörde detta, sade de: "Det är allenast med Beelsebul, de onda andarnas furste, som denne driver ut de onda andarna."
\par 25 Men han förstod deras tankar och sade till dem: "Vart rike som har kommit i strid med sig självt bliver förött, och intet samhälle eller hus som har kommit i strid med sig självt kan hava bestånd.
\par 26 Om nu Satan driver ut Satan, så har han kommit i strid med sig själv. Huru kan då hans rike hava bestånd?
\par 27 Och om det är med Beelsebul som jag driver ut de onda andarna, med vem driva då edra egna anhängare ut dem? De skola alltså vara edra domare.
\par 28 Om det åter är med Guds Ande som jag driver ut de onda andarna, så har ju Guds rike kommit till eder. -
\par 29 Eller huru kan någon gå in i en stark mans hus och beröva honom hans bohag, såframt han icke förut har bundit den starke? Först därefter kan han plundra hans hus.
\par 30 Den som icke är med mig, han är emot mig, och den som icke församlar med mig, han förskingrar.
\par 31 Därför säger jag eder: All annan synd och hädelse skall bliva människorna förlåten, men hädelse mot Anden skall icke bliva förlåten.
\par 32 Ja, om någon säger något mot Människosonen, så skall det bliva honom förlåtet; men om någon säger något mot den helige Ande, så skall det icke bliva honom förlåtet, varken i denna tidsåldern eller i den tillkommande.
\par 33 I måsten döma så: antingen är trädet gott, och då måste dess frukt vara god; eller är trädet dåligt, och då måste dess frukt vara dålig. Ty av frukten känner man trädet.
\par 34 I huggormars avföda, huru skullen I kunna tala något gott, då I själva ären onda? Vad hjärtat är fullt av, det talar ju munnen.
\par 35 En god människa bär ur sitt goda förråd fram vad gott är, och en ond människa bär ur sitt onda förråd fram vad ont är.
\par 36 Men jag säger eder, att för vart fåfängligt ord som människorna tala skola de göra räkenskap på domens dag.
\par 37 Ty efter dina ord skall du dömas rättfärdig, och efter dina ord skall du dömas skyldig."
\par 38 Då togo några av de skriftlärde och fariséerna till orda och sade till honom: "Mästare, vi skulle vilja se något tecken av dig."
\par 39 Men han svarade och sade till dem: "Ett ont och trolöst släkte är detta! Det åstundar ett tecken, men intet annat tecken skall givas det än profeten Jonas' tecken.
\par 40 Ty likasom Jonas tre dagar och tre nätter var i den stora fiskens buk, så skall ock Människosonen tre dagar och tre nätter vara i jordens sköte.
\par 41 Ninevitiska män skola vid domen träda fram tillsammans med detta släkte och bliva det till dom. Ty de gjorde bättring vid Jonas' predikan; och se, här är vad som är mer än Jonas.
\par 42 Drottningen av Söderlandet skall vid domen träda fram tillsammans med detta släkte och bliva det till dom. Ty hon kom från jordens ända för att höra Salomos visdom; och se, här är vad som är mer än Salomo.
\par 43 När en oren ande har farit ut ur en människa, vandrar han omkring i ökentrakter och söker efter ro, men finner ingen.
\par 44 Då säger han: 'Jag vill vända tillbaka till mitt hus, som jag gick ut ifrån.' Och när han kommer dit och finner det stå ledigt och vara fejat och prytt,
\par 45 då går han åstad och tager med sig sju andra andar, som äro värre än han själv, och de gå ditin och bo där; och så bliver för den människan det sista värre än det första. Så skall det ock gå med detta onda släkte."
\par 46 Medan han ännu talade till folket, kommo hans moder och hans bröder och stannade därutanför och ville tala med honom.
\par 47 Då sade någon till honom: "Se, din moder och dina bröder stå härutanför och vilja tala med dig."
\par 48 Men han svarade och sade till den som omtalade detta för honom: "Vilken är min moder, och vilka äro mina bröder?"
\par 49 Och han räckte ut handen mot sina lärjungar och sade: "Se här är min moder, och här äro mina bröder!
\par 50 Ty var och en som gör min himmelske Faders vilja, den är min broder och min syster och min moder."

\chapter{13}

\par 1 Samma dag gick Jesus ut från huset där han bodde och satte sig vid sjön.
\par 2 Då församlade sig mycket folk omkring honom. Därför steg han i en båt; och han satt i den, medan allt folket stod på stranden.
\par 3 Och han talade till dem mycket i liknelser; han sade: "En såningsman gick ut för att så.
\par 4 Och när han sådde, föll somt vid vägen, och fåglarna kommo och åto upp det.
\par 5 Och somt föll på stengrund, där det icke hade mycket jord, och det kom strax upp, eftersom det icke hade djup jord;
\par 6 men när solen hade gått upp, förbrändes det, och eftersom det icke hade någon rot, torkade det bort.
\par 7 Och somt föll bland törnen, och törnena sköto upp och förkvävde det.
\par 8 Men somt föll i god jord, och det gav frukt, dels hundrafalt, dels sextiofalt, dels trettiofalt.
\par 9 Den som har öron, han höre."
\par 10 Då trädde lärjungarna fram och sade till honom: "Varför talar du till dem i liknelser?"
\par 11 Han svarade och sade: "Eder är givet att lära känna himmelrikets hemligheter, men dem är det icke givet.
\par 12 Ty den som har, åt honom skall varda givet, så att han får över nog; men den som icke har, från honom skall tagas också det han har.
\par 13 Därför talar jag till dem i liknelser, eftersom de med seende ögon intet se, och med hörande öron intet höra, och intet heller förstå.
\par 14 Så fullbordas på dem Esaias' profetia, den som säger: 'Med hörande öron skolen I höra, och dock alls intet förstå, och med seende ögon skolen I se, och dock alls intet förnimma.
\par 15 Ty detta folks hjärta har blivit förstockat, och med öronen höra de illa, och sina ögon hava de tillslutit, så att de icke se med sina ögon, eller höra med sina öron, eller förstå med sina hjärtan, och omvända sig och bliva helade av mig.
\par 16 Men saliga äro edra ögon, som se, och edra öron, som höra.
\par 17 Ty sannerligen säger jag eder: Många profeter och rättfärdiga män åstundade att se det som I sen, men fingo dock icke se det, och att höra det som I hören, men fingo dock icke höra det.
\par 18 Hören alltså I vad som menas med liknelsen om såningsmannen.
\par 19 När någon hör ordet om riket, men icke förstår det, då kommer den onde och river bort det som såddes i hans hjärta. Om en sådan människa kan det sägas att säden såddes vid vägen.
\par 20 Och att den såddes på stengrunden, det är sagt om den som väl hör ordet och strax tager emot det med glädje,
\par 21 men som icke har någon rot i sig, utan bliver beståndande allenast till en tid, och när bedrövelse eller förföljelse påkommer för ordets skull, då kommer han strax på fall.
\par 22 Och att den såddes bland törnena, det är sagt om den som väl hör ordet, men låter tidens omsorger och rikedomens bedrägliga lockelse förkväva det, så att han bliver utan frukt.
\par 23 Men att den såddes i den goda jorden, det är sagt om den som både hör ordet och förstår det, och som jämväl bär frukt och giver dels hundrafalt, dels sextiofalt, dels trettiofalt."
\par 24 En annan liknelse framställde han för dem; han sade: "Med himmelriket är det, såsom när en man sådde god säd i sin åker;
\par 25 men när folket sov, kom hans ovän och sådde ogräs mitt ibland vetet och gick sedan sin väg.
\par 26 När nu säden sköt upp och satte frukt, så visade sig ock ogräset.
\par 27 Då trädde husbondens tjänare fram och sade till honom: 'Herre, du sådde ju god säd i din åker; varifrån har den då fått ogräs?
\par 28 Han svarade dem: 'En ovän har gjort detta.' Tjänarna sade till honom: 'Vill du alltså att vi skola gå åstad och samla det tillhopa?'
\par 29 Men han svarade: 'Nej; ty då kunden I rycka upp vetet jämte ogräset, när I samlen detta tillhopa.
\par 30 Låten båda slagen växa tillsammans intill skördetiden; och när skördetiden är inne, vill jag säga till skördemännen: 'Samlen först tillhopa ogräset, och binden det i knippor till att brännas upp, och samlen sedan in vetet i min lada.'"
\par 31 En annan liknelse framställde han för dem; han sade: "Himmelriket är likt ett senapskorn som en man tager och lägger ned i sin åker.
\par 32 Det är minst av alla frön, men när det har växt upp, är det störst bland kryddväxter; ja, det bliver ett träd, så att himmelens fåglar komma och bygga sina nästen på dess grenar."
\par 33 En annan liknelse framställde han för dem: "Himmelriket är likt en surdeg som en kvinna tager och blandar in i tre skäppor mjöl, till dess alltsammans bliver syrat."
\par 34 Allt detta talade Jesus i liknelser till folket, och utan liknelser talade han intet till dem.
\par 35 Ty det skulle fullbordas, som var sagt genom profeten som sade: "Jag vill öppna min mun till liknelser, uppenbara vad förborgat har varit från världens begynnelse."
\par 36 Därefter lät han folket gå och gick själv hem. Och hans lärjungar trädde fram till honom och sade: "Uttyd för oss liknelsen om ogräset i åkern."
\par 37 Han svarade och sade: "Den som sår den goda säden är Människosonen.
\par 38 Åkern är världen. Den goda säden, det är rikets barn, men ogräset är ondskans barn.
\par 39 Ovännen, som sådde det, är djävulen. Skördetiden är tidens ände. Skördemännen är änglar.
\par 40 Såsom nu ogräset samlas tillhopa och brännes upp i eld, så skall det ock ske vid tidens ände.
\par 41 Människosonen skall då sända ut sina änglar, och de skola samla tillhopa och föra bort ur hans rike alla dem som äro andra till fall, och dem som göra vad orätt är,
\par 42 och skola kasta dem i den brinnande ugnen; där skall vara gråt och tandagnisslan.
\par 43 Då skola de rättfärdiga lysa såsom solen, i sin Faders rike. Den som har öron, han höre.
\par 44 Himmelriket är likt en skatt som har blivit gömd i en åker. Och en man finner den, men håller det hemligt; och i sin glädje går han bort och säljer allt vad han äger och köper den åkern.
\par 45 Ytterligare är det med himmelriket, såsom när en köpman söker efter goda pärlor;
\par 46 och då han har funnit en dyrbar pärla, går han bort och säljer vad han äger och köper den.
\par 47 Ytterligare är det med himmelriket, såsom när en not kastas i havet och samlar tillhopa fiskar av alla slag.
\par 48 När den så bliver full, drager man upp den på stranden och sätter sig ned och samlar de goda i kärl, men de dåliga kastar man bort. -
\par 49 Så skall det ock ske vid tidens ände: änglarna skola gå ut och skilja de onda från de rättfärdiga
\par 50 och kasta dem i den brinnande ugnen; där skall vara gråt och tandagnisslan.
\par 51 Haven I förstått allt detta?" De svarade honom: "Ja."
\par 52 Då sade han till dem: "Så är nu var skriftlärd, som har blivit en lärjunge för himmelriket, lik en husbonde som ur sitt förråd bär fram nytt och gammalt."
\par 53 När Jesus hade framställt alla dessa liknelser, drog han bort därifrån.
\par 54 Och han kom till sin fädernestad, och där undervisade han folket i deras synagoga, så att de häpnade och sade: "Varifrån har han fått denna vishet? Och hans kraftgärningar, varifrån komma de?
\par 55 Är då denne icke timmermannens son? Heter icke hans moder Maria, och heta icke hans bröder Jakob och Josef och Simon och Judas?
\par 56 Och hans systrar, bo de icke alla hos oss? Varifrån har han då fått allt detta?"
\par 57 Så blev han för dem en stötesten. Men Jesus sade till dem: "En profet är icke föraktad utom i sin fädernestad och i sitt eget hus."
\par 58 Och för deras otros skull gjorde han där icke många kraftgärningar.

\chapter{14}

\par 1 Vid den tiden fick Herodes, landsfursten, höra ryktet om Jesus.
\par 2 Då sade han till sina tjänare: "Det är Johannes döparen. Han har uppstått från de döda, och därför verka dessa krafter i honom."
\par 3 Herodes hade nämligen låtit gripa Johannes och binda honom och sätta honom i fängelse, för Herodias', sin broder Filippus' hustrus, skull.
\par 4 Ty Johannes hade sagt till honom: "Det är icke lovligt för dig att hava henne."
\par 5 Och han hade velat döda honom, men han fruktade för folket, eftersom de höllo honom för en profet.
\par 6 Men så kom Herodes' födelsedag. Då dansade Herodias' dotter inför dem; och hon behagade Herodes så mycket,
\par 7 att han med en ed lovade att giva henne vad helst hon begärde.
\par 8 Hon sade då, såsom hennes moder ingav henne: "Giv mig här på ett fat Johannes döparens huvud."
\par 9 Då blev konungen bekymrad, men för edens och för bordsgästernas skull bjöd han att man skulle giva henne det,
\par 10 och sände åstad och lät halshugga Johannes i fängelset.
\par 11 Och hans huvud blev framburet på ett fat och givet åt flickan; och hon bar det till sin moder.
\par 12 Men hans lärjungar kommo och togo hans döda kropp och begrovo honom. Sedan gingo de och omtalade det för Jesus.
\par 13 Då Jesus hörde detta, for han i en båt därifrån bort till en öde trakt, där de kunde vara allena. Men när folket fick höra härom, kommo de landvägen efter honom från städerna.
\par 14 Och då han steg i land, fick han se att där var mycket folk; och han ömkade sig över dem och botade deras sjuka.
\par 15 Men när det led mot aftonen, trädde hans lärjungar fram till honom och sade: "Trakten är öde, och tiden är redan framskriden. Låt folket skiljas åt, så att de kunna gå bort i byarna och köpa sig mat."
\par 16 Men Jesus sade till dem: "De behöva icke gå bort; given I dem att äta."
\par 17 De svarade honom: "Vi hava här icke mer än fem bröd och två fiskar."
\par 18 Då sade han: "Bären dem hit till mig."
\par 19 Därefter tillsade han folket att lägga sig ned i gräset. Och han tog de fem bröden och de två fiskarna och såg upp till himmelen och välsignade dem. Och han bröt bröden och gav dem åt lärjungarna, och lärjungarna gåvo åt folket.
\par 20 Och de åto alla och blevo mätta Sedan samlade man upp de överblivna styckena, tolv korgar fulla.
\par 21 Men de som hade ätit voro vid pass fem tusen män, förutom kvinnor och barn.
\par 22 Strax därefter nödgade han sina lärjungar att stiga i båten och före honom fara över till andra stranden, medan han tillsåg att folket skildes åt.
\par 23 Och sedan detta hade skett, gick han upp på berget för att vara allena och bedja. När det så hade blivit afton, var han där ensam.
\par 24 Båten var då redan många stadier från land och hårt ansatt av vågorna, ty vinden låg emot.
\par 25 Men under fjärde nattväkten kom Jesus till dem, gående fram över sjön.
\par 26 När då lärjungarna fingo se honom gå på sjön, blevo de förfärade och sade: "Det är en vålnad", och ropade av förskräckelse.
\par 27 Men Jesus begynte strax tala till dem och sade: "Varen vid gott mod; det är jag, varen icke förskräckta."
\par 28 Då svarade Petrus honom och sade: "Herre, är det du, så bjud mig att komma till dig på vattnet."
\par 29 Han sade: "Kom." Då steg Petrus ut ur båten och begynte gå på vattnet och kom till Jesus.
\par 30 Men när han såg huru stark vinden var, blev han förskräckt; och då han nu begynte sjunka, ropade han och sade: "Herre, hjälp mig."
\par 31 Och strax räckte Jesus ut handen och fattade i honom och sade till honom: "Du klentrogne, varför tvivlade du?"
\par 32 När de sedan hade kommit upp i båten, lade sig vinden.
\par 33 Men de som voro i båten föllo ned för honom och sade: "Förvisso är du Guds Son."
\par 34 När de hade farit över, kommo de till Gennesarets land.
\par 35 Då nu folket där på orten kände igen honom, sände de ut bud i hela trakten däromkring, och man förde till honom alla som voro sjuka.
\par 36 Och de bådo honom att allenast få röra vid hörntofsen på hans mantel; och alla som rörde vid den blevo hulpna.

\chapter{15}

\par 1 Härefter kommo fariséer och skriftlärde från Jerusalem till Jesus och sade:
\par 2 "Varför överträda dina lärjungar de äldstes stadgar? De två ju icke sina händer, när de skola äta."
\par 3 Men han svarade och sade till dem: "Varför överträden I själva Guds bud, för edra stadgars skull?
\par 4 Gud har ju sagt: 'Hedra din fader och din moder' och: 'Den som smädar sin fader eller sin moder, han skall döden dö.'
\par 5 Men I sägen, att om någon säger till sin fader eller sin moder: 'Vad du av mig kunde hava fått till hjälp, det giver jag i stället såsom offergåva', då behöver han alls icke hedra sin fader eller sin moder.
\par 6 I haven så gjort Guds budord om intet, för edra stadgars skull.
\par 7 I skrymtare, rätt profeterade Esaias om eder, när han sade:
\par 8 'Detta folk ärar mig med sina läppar, men deras hjärtan äro långt ifrån mig;
\par 9 och fåfängt dyrka de mig, eftersom de läror de förkunna äro människobud.'"
\par 10 Och han kallade folket till sig och sade till dem: "Hören och förstån.
\par 11 Icke vad som går in i munnen orenar människan, men vad som går ut ifrån munnen, det orenar människan."
\par 12 Då trädde hans lärjungar fram och sade till honom: "Vet du, att när fariséerna hörde det du nu sade, var det för dem en stötesten?"
\par 13 Men han svarade och sade: "Var planta som min himmelske Fader icke har planterat skall ryckas upp med rötterna.
\par 14 Frågen icke efter dem. De äro blinda ledare; och om en blind leder en blind, så falla de båda i gropen."
\par 15 Då tog Petrus till orda och sade till honom: "Uttyd för oss detta bildliga tal."
\par 16 Han sade: "Ären då också I ännu utan förstånd?
\par 17 Insen I icke att allt som går in i munnen, det går ned i buken och har sin naturliga utgång?
\par 18 Men vad som går ut ifrån munnen, det kommer från hjärtat, och det är detta som orenar människan.
\par 19 Ty från hjärtat komma onda tankar, mord, äktenskapsbrott, otukt, tjuveri, falskt vittnesbörd, hädelse.
\par 20 Det är detta som orenar människan; men att äta med otvagna händer, det orenar icke människan."
\par 21 Och Jesus begav sig bort därifrån och drog sig undan till trakten av Tyrus och Sidon.
\par 22 Då kom en kananeisk kvinna från det området och ropade och sade: "Herre, Davids son, förbarma dig över mig. Min dotter plågas svårt av en ond ande."
\par 23 Men han svarade henne icke ett ord. Då trädde hans lärjungar fram och bådo honom och sade: "Giv henne besked; hon förföljer oss ju med sitt ropande."
\par 24 Han svarade och sade: "Jag är icke utsänd till andra än till de förlorade fåren av Israels hus."
\par 25 Men hon kom fram och föll ned för honom och sade: "Herre, hjälp mig."
\par 26 Då svarade han och sade: "Det är otillbörligt att taga brödet från barnen och kasta det åt hundarna."
\par 27 Hon sade: "Ja, Herre. Också äta ju hundarna allenast av de smulor som falla ifrån deras herrars bord."
\par 28 Då svarade Jesus och sade till henne: "O kvinna, din tro är stor. Ske dig såsom du vill." Och hennes dotter var frisk ifrån den stunden.
\par 29 Men Jesus gick därifrån vidare och kom till Galileiska sjön. Och han gick upp på berget och satte sig där.
\par 30 Då kom mycket folk till honom, och de hade med sig halta, blinda, dövstumma, lytta och många andra; dem lade de ned för hans fötter, och han botade dem,
\par 31 så att folket förundrade sig, när de funno dövstumma tala, lytta vara friska och färdiga, halta gå och blinda se. Och man prisade Israels Gud.
\par 32 Och Jesus kallade sina lärjungar till sig och sade: "Jag ömkar mig över folket, ty det är redan tre dagar som de hava dröjt kvar hos mig, och de hava intet att äta; och jag vill icke låta dem gå ifrån mig fastande, för att de icke skola uppgivas på vägen."
\par 33 Då sade lärjungarna till honom: "Varifrån skola vi här, i en öken, få så mycket bröd, att vi kunna mätta så mycket folk?"
\par 34 Jesus frågade dem: "Huru många bröd haven I?" De svarade: "Sju, och därtill några få småfiskar."
\par 35 Då tillsade han folket att lägra sig på marken.
\par 36 Och han tog de sju bröden, så ock fiskarna, och tackade Gud och bröt bröden och gav åt lärjungarna, och lärjungarna gåvo åt folket.
\par 37 Så åto de alla och blevo mätta. Och de överblivna styckena samlade man sedan upp, sju korgar fulla.
\par 38 Men de som hade ätit voro fyra tusen män, förutom kvinnor och barn.
\par 39 Sedan lät han folket skiljas åt och steg i båten och for till Magadans område.

\chapter{16}

\par 1 Och fariséerna och sadducéerna kommo dit och ville sätta honom på prov; de begärde att han skulle låta dem se något tecken från himmelen.
\par 2 Men han svarade och sade till dem: "Om aftonen sägen I: 'Det bliver klart väder, ty himmelen är röd',
\par 3 och om morgonen: 'Det bliver oväder i dag, ty himmelen är mulen och röd.' Ja, om himmelens utseende förstån I att döma, men om tidernas tecken kunnen I icke döma.
\par 4 Ett ont och trolöst släkte är detta! Det åstundar ett tecken, men intet annat tecken skall givas det än Jonas' tecken." Och så lämnade han dem och gick sin väg.
\par 5 När sedan lärjungarna foro åstad, över till andra stranden, hade de förgätit att taga med sig bröd.
\par 6 Och Jesus sade till dem: "Sen till, att I tagen eder till vara för fariséernas och sadducéernas surdeg."
\par 7 Då talade de med varandra och sade: "Det är därför att vi icke hava tagit med oss bröd."
\par 8 Men när Jesus märkte detta, sade han: "I klentrogne, varför talen I eder emellan om att I icke haven bröd med eder?
\par 9 Förstån I ännu ingenting? Och kommen I icke ihåg de fem bröden åt de fem tusen, och huru många korgar I då togen upp?
\par 10 Ej heller de sju bröden åt de fyra tusen, och huru många korgar I då togen upp?
\par 11 Huru kommer det då till, att I icke förstån att det ej var om bröd som jag talade till eder? Tagen eder till vara för fariséernas och sadducéernas surdeg."
\par 12 Då förstodo de att det icke var för surdeg i bröd som han hade bjudit dem att taga sig till vara, utan för fariséernas och sadducéernas lära.
\par 13 Men när Jesus kom till trakten av Cesarea Filippi, frågade han sina lärjungar och sade: "Vem säger folket Människosonen vara?"
\par 14 De svarade: "Somliga säga Johannes döparen, andra Elias, andra åter Jeremias eller en annan av profeterna."
\par 15 Då frågade han dem: "Vem sägen då I mig vara?"
\par 16 Simon Petrus svarade och sade: "Du är Messias, den levande Gudens Son."
\par 17 Då svarade Jesus och sade till honom: "Salig är du, Simon, Jonas' son; ty kött och blod har icke uppenbarat detta för dig, utan min Fader, som är i himmelen.
\par 18 Så säger ock jag dig att du är Petrus; och på denna klippa skall jag bygga min församling, och dödsrikets portar skola icke bliva henne övermäktiga.
\par 19 Jag skall giva dig himmelrikets nycklar: allt vad du binder på jorden, det skall vara bundet i himmelen; och allt vad du löser på jorden, det skall vara löst i himmelen."
\par 20 Därefter förbjöd han lärjungarna att för någon säga att han var Messias.
\par 21 Från den tiden begynte Jesus förklara för sina lärjungar, att han måste gå till Jerusalem och lida mycket av de äldste och översteprästerna och de skriftlärde, och att han skulle bliva dödad, men att han på tredje dagen skulle uppstå igen.
\par 22 Då tog Petrus honom avsides och begynte ivrigt motsäga honom och sade: "Bevare dig Gud, Herre! Ingalunda får detta vederfaras dig."
\par 23 Men han vände sig om och sade till Petrus: "Gå bort, Satan, och stå mig icke i vägen; du är för mig en stötesten, ty dina tankar äro icke Guds tankar, utan människotankar."
\par 24 Därefter sade Jesus till sina lärjungar: "Om någon vill efterfölja mig, så försake han sig själv och tage sitt kors på sig: så följe han mig.
\par 25 Ty den som vill bevara sitt liv, han skall mista det; men den som mister sitt liv, för min skull, han skall finna det.
\par 26 Och vad hjälper det en människa, om hon vinner hela världen, men förlorar sin själ? Eller vad kan en människa giva till lösen för sin själ?
\par 27 Människosonen skall komma i sin Faders härlighet med sina änglar, och då skall han vedergälla var och en efter hans gärningar.
\par 28 Sannerligen säger jag eder: Bland dem som här stå finnas några som icke skola smaka döden, förrän de få se Människosonen komma i sitt rike."

\chapter{17}

\par 1 Sex dagar därefter tog Jesus med sig Petrus och Jakob och Johannes, Jakobs broder, och förde dem upp på ett högt berg, där de voro allena.
\par 2 Och hans utseende förvandlades inför dem: hans ansikte sken såsom solen, och hans kläder blevo vita såsom ljuset.
\par 3 Och se, för dem visade sig Moses och Elias, i samtal med honom.
\par 4 Då tog Petrus till orda och sade till Jesus: "Herre, här är oss gott att vara; vill du, så skall jag här göra tre hyddor, åt dig en och åt Moses en och åt Elias en."
\par 5 Och se, medan han ännu talade, överskyggade dem en ljus sky, och ur skyn kom en röst som sade: "Denne är min älskade Son, i vilken jag har funnit behag; hören honom."
\par 6 När lärjungarna hörde detta, föllo de ned på sina ansikten i stor förskräckelse.
\par 7 Men Jesus gick fram och rörde vid dem och sade: "Stån upp, och varen icke förskräckta."
\par 8 När de då lyfte upp sina ögon, sågo de ingen utom Jesus allena.
\par 9 Då de sedan gingo ned från berget, bjöd Jesus dem och sade: "Omtalen icke för någon denna syn, förrän Människosonen har uppstått från de döda."
\par 10 Men lärjungarna frågade honom och sade: "Huru kunna då de skriftlärde säga att Elias först måste komma?"
\par 11 Han svarade och sade: "Elias måste visserligen komma och upprätta allt igen;
\par 12 men jag säger eder att Elias redan har kommit. Men de ville icke veta av honom, utan förforo mot honom alldeles såsom de ville. Sammalunda skall ock Människosonen få lida genom dem."
\par 13 Då förstodo lärjungarna att det var om Johannes döparen som han talade till dem.
\par 14 När de därefter kommo till folket, trädde en man fram till honom och föll på knä för honom
\par 15 och sade: "Herre, förbarma dig över min son ty han är månadsrasande och plågas svårt; ofta faller han i elden och ofta i vattnet.
\par 16 Och jag förde honom till dina lärjungar, men de kunde icke bota honom."
\par 17 Då svarade Jesus och sade: "O du otrogna och vrånga släkte, huru länge måste jag vara bland eder? Huru länge måste jag härda ut med eder? Fören honom hit till mig."
\par 18 Och Jesus tilltalade honom strängt, och den onde anden for ut ur honom; och gossen var botad från den stunden.
\par 19 Sedan, när de voro allena, trädde lärjungarna fram till Jesus och frågade: "Varför kunde icke vi driva ut honom?"
\par 20 Han svarade dem: "För eder otros skull. Ty sannerligen säger jag eder: Om I haven tro, vore den ock blott såsom ett senapskorn, så skolen I kunna säga till detta berg: 'Flytta dig härifrån dit bort', och det skall flytta sig; ja, intet skall då vara omöjligt för eder."
\par 21 Detta slag kan icke drivas ut genom något annat än bön och fasta.
\par 22 Medan de nu tillsammans vandrade omkring i Galileen, sade Jesus till dem: "Människosonen skall bliva överlämnad i människors händer,
\par 23 och man skall döda honom, men på tredje dagen skall han uppstå igen." Då blevo de mycket bedrövade.
\par 24 Och när de hade kommit till Kapernaum, trädde de som uppburo tempelskatten fram till Petrus och sade: "Plägar icke eder mästare betala tempelskatt?"
\par 25 Han svarade: "Jo." När han sedan hade kommit hem, förekom honom Jesus med frågan: "Vad synes dig, Simon? Av vilka taga jordens konungar tull eller skatt, av sina söner eller av andra människor?"
\par 26 Han svarade: "Av andra människor." Då sade Jesus till honom: "Alltså äro då sönerna fria.
\par 27 Men för att vi icke skola bliva dem till en stötesten, så gå ned till sjön och kasta ut en krok. Tag så den första fisk som du drager upp, och när du öppnar munnen på den skall du där finna en silverpenning. Tag den, och giv den åt dem for mig och dig."

\chapter{18}

\par 1 I samma stund trädde lärjungarna fram till Jesus och frågade: "Vilken är den störste i himmelriket?"
\par 2 Då kallade han fram ett barn och ställde det mitt ibland dem
\par 3 och sade: "Sannerligen säger jag eder: Om I icke omvänden eder och bliven såsom barn, skolen I icke komma in i himmelriket.
\par 4 Den som nu så ödmjukar sig, att han bliver såsom detta barn, han är den störste i himmelriket.
\par 5 Och den som tager emot ett sådant barn I mitt namn, han tager emot mig.
\par 6 Men den som förför en av dessa små som tro på mig, för honom vore det bättre att en kvarnsten hängdes om hans hals och han sänktes ned i havets djup.
\par 7 Ve världen för förförelsers skull! Förförelser måste ju komma; men ve den människa genom vilken förförelsen kommer!
\par 8 Om nu din hand eller din fot är dig till förförelse, så hugg av den och kasta den ifrån dig. Det är bättre för dig att ingå i livet lytt eller halt, än att hava båda händerna eller båda fötterna i behåll och kastas i den eviga elden.
\par 9 Och om ditt öga är dig till förförelse, så riv ut det och kasta det ifrån dig. Det är bättre för dig att ingå i livet enögd, än att hava båda ögonen i behåll och kastas i det brinnande Gehenna.
\par 10 Sen till, att I icke förakten någon av dessa små; ty jag säger eder att deras änglar i himmelen alltid se min himmelske Faders ansikte.
\par 11 Ty Människosonen har kommit för att frälsa det som var förlorat.
\par 12 Vad synes eder? Om en man har hundra får, och ett av dem har kommit vilse, lämnar han icke då de nittionio på bergen och går åstad och söker efter det som har kommit vilse?
\par 13 Och händer det då att han finner det - sannerligen säger jag eder: då gläder han sig mer över det fåret än över de nittionio som icke hade kommit vilse.
\par 14 Så är det ej heller eder himmelske Faders vilja att någon av dessa små skall gå förlorad.
\par 15 Men om din broder försyndar sig, så gå åstad och förehåll honom det enskilt. Om han då lyssnar till dig, så har du vunnit din broder.
\par 16 Men om han icke lyssnar till dig, så tag med dig ännu en eller två, för att 'var sak må avgöras efter två eller tre vittnens utsago'.
\par 17 Lyssnar han icke till dem, så säg det till församlingen. Lyssnar han ej heller till församlingen, så vare han för dig såsom en hedning och en publikan.
\par 18 Sannerligen säger jag eder: Allt vad I binden på jorden, det skall vara bundet i himmelen; och allt vad I lösen på jorden, det skall vara löst i himmelen.
\par 19 Ytterligare säger jag eder, att om två av eder här på jorden komma överens att bedja om något, vad det vara må, så skall det beskäras dem av min Fader, som är i himmelen.
\par 20 Ty var två eller tre är församlade i mitt namn, där är jag mitt ibland dem."
\par 21 Då trädde Petrus fram och sade till honom: "Herre, huru många gånger skall jag förlåta min broder, om han försyndar sig mot mig? Är sju gånger nog?"
\par 22 Jesus svarade honom: "Jag säger dig: Icke sju gånger, utan sjuttio gånger sju gånger.
\par 23 Alltså är det med himmelriket, såsom när en konung ville hålla räkenskap med sina tjänare.
\par 24 Och när han begynte hålla räkenskap, förde man fram till honom en som var skyldig honom tio tusen pund.
\par 25 Men då denna icke kunde betala, bjöd hans herre att han skulle säljas, så ock hans hustru och barn och allt vad han ägde, för att skulden måtte bliva betald.
\par 26 Då föll tjänaren ned för hans fötter och sade: 'Hav tålamod med mig, så skall jag betala dig alltsammans.'
\par 27 Och tjänarens herre ömkade sig över honom och gav honom fri och efterskänkte honom hans skuld.
\par 28 Men när samme tjänare kom ut, träffade han på en av sina medtjänare, som var skyldig honom hundra silverpenningar; och han tog fast denne och grep honom vid strupen och sade: 'Betala vad du är skyldig.'
\par 29 Då föll hans medtjänare ned och bad honom och sade: 'Hav tålamod med mig, så skall jag betala dig.'
\par 30 Men han ville icke, utan gick åstad och lät sätta honom i fängelse, till dess han hade betalt vad han var skyldig.
\par 31 Då nu hans medtjänare sågo det som skedde, togo de mycket illa vid sig och gingo och berättade för sin herre allt som hade skett.
\par 32 Då kallade hans herre honom till sig och sade till honom: 'Du onde tjänare, allt vad du var skyldig efterskänkte jag dig, eftersom du bad mig därom.
\par 33 Borde då icke också du hava förbarmat dig över din medtjänare, såsom jag förbarmade mig över dig?'
\par 34 Och i sin vrede överlämnade hans herre honom i fångknektarnas våld, intill dess han hade betalt allt vad han var skyldig.
\par 35 Så skall ock min himmelske Fader göra med eder, om I icke av hjärtat förlåten var och en sin broder."

\chapter{19}

\par 1 När Jesus hade slutat detta tal, drog han bort ifrån Galileen och begav sig, genom landet på andra sidan Jordan, till Judeens område.
\par 2 Och mycket folk följde honom, och han botade där de sjuka.
\par 3 Då ville några fariséer snärja honom och trädde fram till honom och sade: "Är det lovligt att skilja sig från sin hustru av vilken orsak som helst?"
\par 4 Men han svarade och sade: "Haven I icke läst att Skaparen redan i begynnelsen 'gjorde dem till man och kvinna'
\par 5 och sade: 'Fördenskull skall en man övergiva sin fader och sin moder och hålla sig till sin hustru, och de tu skola varda ett kött'?
\par 6 Så äro de icke mer två, utan ett kött. Vad nu Gud har sammanfogat, det må människan icke åtskilja."
\par 7 Då sade de till honom: "Huru kunde då Moses bjuda att man skulle giva hustrun skiljebrev och så skilja sig från henne?"
\par 8 Han svarade dem: "För edra hjärtans hårdhets skull tillstadde Moses eder att skiljas från edra hustrur, men från begynnelsen har det icke varit så.
\par 9 Och jag säger eder: Den som för någon annan orsaks skull än för otukt skiljer sig från sin hustru och tager sig en annan hustru, han begår äktenskapsbrott."
\par 10 Då sade lärjungarna till honom: "Är det så med mannens ställning till hustrun, då är det icke rådligt att gifta sig."
\par 11 Men han svarade dem: "Icke alla kunna taga emot det ordet, utan allenast de åt vilka sådant är givet.
\par 12 Ty visserligen finnas somliga som genom födelsen, allt ifrån moderlivet, äro oskickliga till äktenskap, andra åter som av människor hava gjorts oskickliga därtill, men somliga finnas ock, som för himmelrikets skull självmant hava gjort sig oskickliga därtill. Den som kan taga emot detta, han tage emot det."
\par 13 Därefter buros barn fram till honom, för att han skulle lägga händerna på dem och bedja; men lärjungarna visade bort dem.
\par 14 Då sade Jesus: "Låten barnen vara, och förmenen dem icke att komma till mig; ty himmelriket hör sådana till."
\par 15 Och han lade händerna på dem och gick sedan därifrån.
\par 16 Då trädde en man fram till honom och sade: "Mästare, vad gott skall jag göra för att få evigt liv?"
\par 17 Han sade till honom: "Varför frågar du mig om vad som är gott? En finnes som är god. Men vill du ingå i livet, så håll buden."
\par 18 Han frågade: "Vilka?" Jesus svarade: "'Du skall icke dräpa', 'Du skall icke begå äktenskapsbrott', 'Du skall icke stjäla', 'Du skall icke bära falskt vittnesbörd',
\par 19 'Hedra din fader och din moder' och 'Du skall älska din nästa såsom dig själv.'"
\par 20 Då sade den unge mannen till honom: "Allt detta har jag hållit. Vad fattas mig ännu?"
\par 21 Jesus svarade honom: "Vill du vara fullkomlig, så gå bort och sälj vad du äger och giv åt de fattiga; då skall du få en skatt i himmelen. Och kom sedan och följ mig."
\par 22 Men när den unge mannen hörde detta, gick han bedrövad bort, ty han hade många ägodelar.
\par 23 Då sade Jesus till sina lärjungar: "Sannerligen säger jag eder: För den som är rik är det svårt att komma in i himmelriket.
\par 24 Ja, jag säger eder: Det är lättare för en kamel att komma in genom ett nålsöga, än för den som är rik att komma in i Guds rike."
\par 25 När lärjungarna hörde detta, blevo de mycket häpna och sade: "Vem kan då bliva frälst?"
\par 26 Men Jesus såg på dem och sade till dem: "För människor är detta omöjligt, men för Gud är allting möjligt."
\par 27 Då tog Petrus till orda och sade till honom: "Se, vi hava övergivit allt och följt dig; vad skola vi få därför?"
\par 28 Jesus svarade dem: "Sannerligen säger jag eder: När världen födes på nytt, då när Människosonen sätter sig på sin härlighets tron, då skolen också I, som haven efterföljt mig, få sitta på tolv troner såsom domare över Israels tolv stammar.
\par 29 Och var och en som har övergivit hus, eller bröder eller systrar, eller fader eller moder, eller barn, eller jordagods, för mitt namns skull, han skall få mångfaldigt igen, och skall få evigt liv till arvedel.
\par 30 Men många som äro de första skola bliva de sista, och många som äro de sista skola bliva de första."

\chapter{20}

\par 1 "Ty med himmelriket är det, såsom när en husbonde bittida om morgonen gick ut för att leja åt sig arbetare till sin vingård.
\par 2 Och när han hade kommit överens med arbetarna om en viss dagspenning, sände han dem till sin vingård.
\par 3 När han sedan gick ut vid tredje timmen, fick han se några andra stå sysslolösa på torget;
\par 4 och han sade till dem: 'Gån ock I till min vingård, så skall jag giva eder vad skäligt är.'
\par 5 Och de gingo. Åter gick han ut vid sjätte timmen och vid nionde och gjorde sammalunda.
\par 6 Också vid elfte timmen gick han ut och fann då några andra stå där; och han sade till dem: 'Varför stån I här hela dagen sysslolösa?'
\par 7 De svarade honom: 'Därför att ingen har lejt oss.' Då sade han till dem: 'Gån ock I till min vingård.'
\par 8 När det så hade blivit afton, sade vingårdens herre till sin förvaltare: 'Kalla fram arbetarna och giv dem deras lön, men begynn med de sista och gå så tillbaka ända till de första.'
\par 9 Då nu de kommo fram, som voro lejda vid elfte timmen, fick var och en av dem full dagspenning.
\par 10 När sedan de första kommo, trodde de att de skulle få mer, men också var och en av dem fick samma dagspenning.
\par 11 När de så fingo, knorrade de mot husbonden.
\par 12 och sade: 'Dessa sista hava arbetat allenast en timme, och du har ändå ställt dem lika med oss, som hava burit dagens tunga och solens hetta?'
\par 13 Då svarade han en av dem och sade: 'Min vän, jag gör dig ingen orätt. Kom du icke överens med mig om den dagspenningen?
\par 14 Tag vad dig tillkommer och gå. Men åt denne siste vill jag giva lika mycket som åt dig.
\par 15 Har jag icke lov att göra såsom jag vill med det som är mitt? Eller skall du med onda ögon se på att jag är så god?' -
\par 16 Så skola de sista bliva de första, och de första bliva de sista."
\par 17 Då nu Jesus ville gå upp till Jerusalem, tog han till sig de tolv, så att de voro allena; och under vägen sade han till dem:
\par 18 "Se, vi gå nu upp till Jerusalem, och Människosonen skall bliva överlämnad åt översteprästerna och de skriftlärde, och de skola döma honom till döden
\par 19 och överlämna honom åt hedningarna till att begabbas och gisslas och korsfästas; men på tredje dagen skall han uppstå igen."
\par 20 Då trädde Sebedeus' söners moder fram till honom med sina söner och föll ned för honom och ville begära något av honom.
\par 21 Han frågade henne: "Vad vill du?" Hon svarade honom: "Säg att i ditt rike den ene av dessa mina två söner skall få sitta på din högra sida, och den andre på din vänstra."
\par 22 Men Jesus svarade och sade: "I veten icke vad I begären. Kunnen I dricka den kalk som jag skall dricka?" De svarade honom: "Det kunna vi."
\par 23 Då sade han till dem: "Ja, väl skolen I få dricka min kalk, men platsen på min högra sida och platsen på min vänstra tillkommer det icke mig att bortgiva, utan de skola tillfalla dem för vilka så är bestämt av min Fader."
\par 24 När de tio andra hörde detta, blevo de misslynta på de två bröderna.
\par 25 Då kallade Jesus dem till sig och sade: "I veten att furstarna uppträda mot sina folk såsom herrar, och att de mäktige låta folken känna sin myndighet.
\par 26 Så är det icke bland eder; utan den som vill bliva störst bland eder, han vare de andras tjänare,
\par 27 och den som vill vara främst bland eder, han vare de andras dräng,
\par 28 likasom Människosonen har kommit, icke för att låta tjäna sig, utan för att tjäna och giva sitt liv till lösen för många."
\par 29 När de sedan gingo ut ifrån Jeriko, följde honom mycket folk.
\par 30 Och se, två blinda sutto där vid vägen. När dessa hörde att det var Jesus som gick där fram, ropade de och sade: "Herre, förbarma dig över oss, du Davids son."
\par 31 Och folket tillsade dem strängeligen att de skulle tiga; men de ropade dess mer och sade: "Herre, förbarma dig över oss, du Davids son."
\par 32 Då stannade Jesus och kallade dem till sig och sade: "Vad viljen I att jag skall göra eder?"
\par 33 De svarade honom: "Herre, låt våra ögon bliva öppnade."
\par 34 Då förbarmade sig Jesus över dem och rörde vid deras ögon, och strax fingo de sin syn och följde honom.

\chapter{21}

\par 1 När de nu nalkades Jerusalem och kommo till Betfage vid Oljeberget, då sände Jesus åstad två lärjungar
\par 2 och sade till dem: "Gån in i byn som ligger mitt framför eder, så skolen I strax finna en åsninna stå där bunden och en fåle bredvid henne; lösen dem och fören dem till mig.
\par 3 Och om någon säger något till eder, så skolen I svara: 'Herren behöver dem'; då skall han strax släppa dem."
\par 4 Detta har skett, för att det skulle fullbordas, som var sagt genom profeten som sade:
\par 5 "Sägen till dottern Sion: 'Se, din konung kommer till dig, saktmodig, ridande på en åsna, på en arbetsåsninnas fåle.'"
\par 6 Och lärjungarna gingo åstad och gjorde såsom Jesus hade befallt dem
\par 7 och ledde till honom åsninnan och fålen; och de lade sina mantlar på denne, och han satte sig därovanpå.
\par 8 Och folkskaran, som var mycket stor, bredde ut sina mantlar på vägen; men somliga skuro kvistar av träden och strödde på vägen.
\par 9 Och folket, både de som gingo före honom och de som följde efter, ropade och sade: "Hosianna Davids son! Välsignad vare han som kommer, i Herrens namn. Hosianna i höjden!"
\par 10 När han så drog in i Jerusalem, kom hela staden i rörelse, och man frågade: "Vem är denne?"
\par 11 Och folket sade: "Det är Jesus, profeten, från Nasaret i Galileen."
\par 12 Och Jesus gick in i helgedomen. Och han drev ut alla dem som sålde och köpte i helgedomen, och han stötte omkull växlarnas bord och duvomånglarnas säten.
\par 13 Och han sade till dem: "Det är skrivet: 'Mitt hus skall kallas ett bönehus.' Men I gören det till en rövarkula."
\par 14 Och blinda och halta kommo fram till honom i helgedomen, och han botade dem.
\par 15 Men när översteprästerna och de skriftlärde sågo de under som han gjorde, och sågo barnen som ropade i helgedomen och sade: "Hosianna Davids son!", då förtröt detta dem;
\par 16 och de sade till honom: "Hör du vad dessa säga?" Då svarade Jesus dem: "Ja; haven I aldrig läst: 'Av barns och spenabarns mun har du berett dig lov'?"
\par 17 Därefter lämnade han dem och gick ut ur staden till Betania och stannade där över natten.
\par 18 När han sedan på morgonen gick in till staden igen, blev han hungrig.
\par 19 Och då han fick se ett fikonträd vid vägen, gick han fram till det, men fann intet därpå, utom allenast löv. Då sade han till det: "Aldrig någonsin mer skall frukt växa på dig." Och strax förtorkades fikonträdet.
\par 20 När lärjungarna sågo detta, förundrade de sig och sade: "Huru kunde fikonträdet så i hast förtorkas?"
\par 21 Då svarade Jesus och sade till dem: "Sannerligen säger jag eder: Om I haven tro och icke tvivlen, så skolen I icke allenast kunna göra sådant som skedde med fikonträdet, utan I skolen till och med kunna säga till detta berg: 'Häv dig upp och kasta dig i havet', och det skall ske.
\par 22 Och allt vad I med tro bedjen om i eder bön, det skolen I få."
\par 23 När han därefter hade kommit in i helgedomen, trädde översteprästerna och folkets äldste fram till honom, där han undervisade; och de sade: "Med vad myndighet gör du detta? Och vem har givit dig sådan myndighet?"
\par 24 Jesus svarade och sade till dem: "Också jag vill ställa en fråga till eder; om I svaren mig på den, så skall ock jag säga eder med vad myndighet jag gör detta".
\par 25 Johannes' döpelse, varifrån var den: från himmelen eller från människor?" Då överlade de med varandra och sade: "Om vi svara: 'Från himmelen', så frågar han oss: 'Varför trodden I honom då icke?'
\par 26 Men om vi svara: 'Från människor', då måste vi frukta för folket, ty alla hålla de Johannes för en profet."
\par 27 De svarade alltså Jesus och sade: "Vi veta det icke." Då sade ock han till dem: "Så säger icke heller jag eder med vad myndighet jag gör detta.
\par 28 Men vad synes eder? En man hade två söner. Och han kom till den förste och sade: 'Min son, gå i dag och arbeta i vingården.'
\par 29 Han svarade och sade: 'Jag vill icke'; men efteråt ångrade han sig och gick.
\par 30 Och han kom till den andre och sade sammalunda. Då svarade denne och sade: 'Ja, herre'; men han gick icke,
\par 31 Vilken av de två gjorde vad fadern ville?" De svarade: "Den förste." Jesus sade till dem: "Ja, sannerligen säger jag eder: Publikaner och skökor skola förr gå in i Guds rike än I.
\par 32 Ty Johannes kom och lärde eder rättfärdighetens väg, och I trodden honom icke, men publikaner och skökor trodde honom. Och fastän I sågen detta, ångraden I eder icke heller efteråt, så att I trodden honom.
\par 33 Hören nu en annan liknelse: En husbonde planterade en vingård, och han satte stängsel omkring den och högg ut ett presskar därinne och byggde ett vakttorn; därefter lejde han ut den åt vingårdsmän och for utrikes.
\par 34 När sedan frukttiden nalkades, sände han sina tjänare till vingårdsmännen för att uppbära frukten åt honom.
\par 35 Men vingårdsmännen togo fatt på hans tjänare, och en misshandlade de, en annan dräpte de, en tredje stenade de.
\par 36 Åter sände han åstad andra tjänare, flera än de förra, men de gjorde med dem sammalunda.
\par 37 Slutligen sände han till dem sin son, ty han tänkte: 'De skola väl hava försyn för min son.'
\par 38 Men när vingårdsmännen fingo se hans son, sade de till varandra: 'Denne är arvingen; kom, låt oss dräpa honom, så få vi hans arv.'
\par 39 Och de togo fatt på honom och förde honom ut ur vingården och dräpte honom.
\par 40 När nu vingårdens herre kommer, vad skall han då göra med de vingårdsmännen?"
\par 41 De svarade honom: "Eftersom de hava illa gjort, skall han illa förgöra dem, och vingården skall han lämna åt andra vingårdsmän, som giva honom frukten, när tiden därtill är inne."
\par 42 Jesus sade till dem: "Ja, haven I aldrig läst i skrifterna: 'Den sten som byggningsmännen förkastade. den har blivit en hörnsten; av Herren har den blivit detta, och underbar är den i våra ögon'?
\par 43 Därför säger jag eder att Guds rike skall tagas ifrån eder, och givas åt ett folk som bär dess frukt."
\par 44 "Och var och en som faller på den stenen, han skall bliva krossad, men den som stenen faller på, honom skall den söndersmula."
\par 45 Då nu översteprästerna och fariséerna hörde hans liknelser, förstodo de att det var om dem som han talade.
\par 46 Och de hade gärna velat gripa honom, men de fruktade för folket, eftersom man höll honom för en profet.

\chapter{22}

\par 1 Och Jesus begynte åter tala till dem i liknelser och sade:
\par 2 "Med himmelriket är det, såsom när en konung gjorde bröllop åt sin son.
\par 3 Han sände ut sina tjänare för att kalla till bröllopet dem som voro bjudna; men de ville icke komma.
\par 4 Åter sände han ut andra tjänare och befallde dem att säga till dem som voro bjudna: 'Jag har nu tillrett min måltid, mina oxar och min gödboskap äro slaktade, och allt är redo; kommen till bröllopet.'
\par 5 Men de aktade icke därpå, utan gingo bort, den ene till sitt jordagods, den andre till sin köpenskap.
\par 6 Och de övriga grepo hans tjänare och misshandlade och dräpte dem.
\par 7 Då blev konungen vred och sände ut sitt krigsfolk och förgjorde dråparna och brände upp deras stad.
\par 8 Därefter sade han till sina tjänare: 'Bröllopet är tillrett, men de som voro bjudna voro icke värdiga.
\par 9 Gån därför ut till vägskälen och bjuden till bröllopet alla som I träffen på.'
\par 10 Och tjänarna gingo ut på vägarna och samlade tillhopa alla som de träffade på, både onda och goda, och bröllopssalen blev full av bordsgäster.
\par 11 Men när konungen nu kom in för att se på gästerna, fick han där se en man som icke var klädd i bröllopskläder.
\par 12 Då sade han till honom: 'Min vän, huru har du kommit hitin, då du icke bär bröllopskläder?' Och han kunde intet svara.
\par 13 Då sade konungen till tjänarna: 'Gripen honom vid händer och fötter, och kasten honom ut i mörkret härutanför.' Där skall vara gråt och tandagnisslan.
\par 14 Ty många äro kallade, men få utvalda."
\par 15 Därefter gingo fariséerna bort och fattade det beslutet att de skulle söka snärja honom genom något hans ord.
\par 16 Och de sände till honom sina lärjungar, tillika med herodianerna, och läto dem säga: "Mästare, vi veta att du är sannfärdig och lär om Guds väg vad sant är, utan att fråga efter någon; ty du ser icke till personen.
\par 17 Så säg oss då: Vad synes dig? Är det lovligt att giva kejsaren skatt, eller är det icke lovligt?"
\par 18 Men Jesus märkte deras ondska och sade: "Varför söken I att snärja mig, I skrymtare?
\par 19 Låten mig se skattepenningen." Då lämnade de fram till honom en penning.
\par 20 Därefter frågade han dem: "Vems bild och överskrift är detta?"
\par 21 De svarade: "Kejsarens." Då sade han till dem: "Så given då kejsaren vad kejsaren tillhör, och Gud vad Gud tillhör."
\par 22 När de hörde detta, förundrade de sig. Och de lämnade honom och gingo sin väg.
\par 23 Samma dag trädde några sadducéer fram till honom och ville påstå att det icke gives någon uppståndelse; de frågade honom
\par 24 och sade: "Mästare, Moses har sagt: 'Om någon dör barnlös, så skall hans broder i hans ställe äkta hans hustru och skaffa avkomma åt sin broder.'
\par 25 Nu voro hos oss sju bröder. Den förste tog sig hustru och dog, och eftersom han icke hade någon avkomma, lämnade han sin hustru efter sig åt sin broder.
\par 26 Sammalunda ock den andre och den tredje, allt intill den sjunde.
\par 27 Sist av alla dog hustrun.
\par 28 Vilken av de sju skall då vid uppståndelsen få henne till hustru? De hade ju alla äktat henne."
\par 29 Jesus svarade och sade till dem: "I faren vilse, ty I förstån icke skrifterna, ej heller Guds kraft.
\par 30 Vid uppståndelsen taga män sig icke hustrur, ej heller givas hustrur åt män, utan de äro då såsom änglarna i himmelen.
\par 31 Men vad nu angår de dödas uppståndelse, haven I icke läst vad eder är sagt av Gud:
\par 32 'Jag är Abrahams Gud och Isaks Gud och Jakobs Gud'? Han är en Gud icke för döda, utan för levande."
\par 33 När folket hörde detta, häpnade de över hans undervisning.
\par 34 Men när fariséerna fingo höra att han hade stoppat munnen till på sadducéerna, samlade de sig tillhopa;
\par 35 och en av dem, som var lagklok, ville snärja honom och frågade:
\par 36 "Mästare, vilket är det yppersta budet i lagen?"
\par 37 Då svarade han honom: "'Du skall älska HERREN, din Gud, av allt ditt hjärta och av all din själ och av allt ditt förstånd.'
\par 38 Detta är det yppersta och förnämsta budet.
\par 39 Därnäst kommer ett som är detta likt: 'Du skall älska din nästa såsom dig själv.'
\par 40 På dessa två bud hänger hela lagen och profeterna."
\par 41 Men då nu fariséerna voro församlade, frågade Jesus dem
\par 42 och sade: "Vad synes eder om Messias, vems son är han?" De svarade honom: "Davids."
\par 43 Då sade han till dem: "Huru kan då David, genom andeingivelse, kalla honom 'herre'? Han säger ju:
\par 44 'Herren sade till min herre: Sätt dig på min högra sida, till dess jag har lagt dina fiender under dina fötter.'
\par 45 Om nu David kallar honom 'herre', huru kan han då vara hans son?"
\par 46 Och ingen förmådde svara honom ett ord. Ej heller dristade sig någon från den dagen att vidare ställa någon fråga till honom.

\chapter{23}

\par 1 Därefter talade Jesus till folket och till sina lärjungar
\par 2 och sade: "På Moses' stol hava de skriftlärde och fariséerna satt sig.
\par 3 Därför, allt vad de säga eder, det skolen I göra och hålla, men efter deras gärningar skolen I icke göra; ty de säga, men göra icke.
\par 4 De binda ihop tunga bördor och lägga dem på människornas skuldror, men själva vilja de icke röra ett finger för att flytta dem.
\par 5 Och alla sina gärningar göra de för att bliva sedda av människorna. De göra sina böneremsor breda och hörntofsarna på sina mantlar stora.
\par 6 De vilja gärna hava de främsta platserna vid gästabuden och sitta främst i synagogorna
\par 7 och vilja gärna bliva hälsade på torgen och av människorna kallas 'rabbi'.
\par 8 Men I skolen icke låta kalla eder 'rabbi', ty en är eder Mästare, och I ären alla bröder.
\par 9 Ej heller skolen I kalla någon på jorden eder 'fader', ty en är eder Fader, han som är i himmelen.
\par 10 Ej heller skolen I låta kalla eder 'läromästare', ty en är eder läromästare, Kristus.
\par 11 Den som är störst bland eder, han vare de andras tjänare.
\par 12 Men den som upphöjer sig, han skall bliva förödmjukad, och den som ödmjukar sig, han skall bliva upphöjd.
\par 13 Ve eder, I skriftlärde och fariséer, I skrymtare, som tillsluten himmelriket för människorna! Själva kommen I icke ditin, och dem som vilja komma dit tillstädjen I icke att komma in.
\par 14 Ve eder, I skriftlärde och fariseér, I skrymtare, att I utsugen änkors hus, medan I för syns skull hållen långa böner. Därför skolen I få en dess hårdare dom.
\par 15 Ve eder, I skriftlärde och fariséer, I skrymtare, som faren omkring över vatten och land för att göra en proselyt, och när någon har blivit det, gören I honom till ett Gehennas barn, dubbelt värre än I själva ären!
\par 16 Ve eder, I blinde ledare, som sägen: 'Om någon svär vid templet, så betyder det intet; men om någon svär vid guldet i templet, då är han bunden av sin ed'!
\par 17 I dåraktige och blinde, vilket är då förmer, guldet eller templet, som har helgat guldet?
\par 18 Så ock: 'Om någon svär vid altaret, så betyder det intet; men om någon svär vid offergåvan som ligger därpå, då är han bunden av sin ed.'
\par 19 I blinde, vilket är då förmer, offergåvan eller altaret, som helgar offergåvan?
\par 20 Den som svär vid altaret, han svär alltså både vid detta och vid allt som ligger därpå.
\par 21 Och den som svär vid templet, han svär både vid detta och vid honom som bor däri.
\par 22 Och den som svär vid himmelen, han svär både vid Guds tron och vid honom som sitter därpå.
\par 23 Ve eder, I skriftlärde och fariséer. I skrymtare, som given tionde av mynta och dill och kummin, men underlåten det som är viktigast i lagen, nämligen rätten och barmhärtigheten och troheten! Det ena borden I göra, men icke underlåta det andra.
\par 24 I blinde ledare, som silen bort myggan och sväljen kamelen!
\par 25 Ve eder, I skriftlärde och fariséer, I skrymtare, som gören det yttre av bägaren och fatet rent, medan de inuti äro fulla av vad I haven förvärvat genom rofferi och omättlig ondska!
\par 26 Du blinde farisé, gör först insidan av bägaren ren, för att sedan också dess utsida må bliva ren.
\par 27 Ve eder, I skriftlärde och fariséer, I skrymtare, som ären lika vitmenade gravar, vilka väl utanpå synas prydliga, men inuti äro fulla av de dödas ben och allt slags orenlighet!
\par 28 Så synens ock I utvärtes för människorna rättfärdiga, men invärtes ären I fulla av skrymteri och orättfärdighet.
\par 29 Ve eder, I skriftlärde och fariséer, I skrymtare, som byggen upp profeternas gravar och pryden de rättfärdigas grifter
\par 30 och sägen: 'Om vi hade levat på våra fäders tid, så skulle vi icke hava varit delaktiga med dem i profeternas blod'!
\par 31 Så vittnen I då om eder själva, att I ären barn av dem som dräpte profeterna.
\par 32 Nåväl, uppfyllen då I edra fäders mått.
\par 33 I ormar, I huggormars avföda, huru skullen I kunna söka undgå att dömas till Gehenna?
\par 34 Se, därför sänder jag till eder profeter och vise och skriftlärde. Somliga av dem skolen I dräpa och korsfästa, och somliga av dem skolen I gissla i edra synagogor och förfölja ifrån den ena staden till den andra.
\par 35 Och så skall över eder komma allt rättfärdigt blod som är utgjutet på jorden, ända ifrån den rättfärdige Abels blod intill Sakarias', Barakias' sons blod, hans som I dräpten mellan templet och altaret.
\par 36 Sannerligen säger jag eder: Allt detta skall komma över detta släkte.
\par 37 Jerusalem, Jerusalem, du som dräper profeterna och stenar dem som äro sända till dig! Huru ofta har jag icke velat församla dina barn, likasom hönan församlar sina kycklingar under sina vingar! Men I haven icke velat.
\par 38 Se, edert hus skall komma att stå övergivet och öde.
\par 39 Ty jag säger eder: Härefter skolen I icke få se mig, intill den tid då I sägen: 'Välsignad vare han som kommer, i Herrens namn.'"

\chapter{24}

\par 1 Och Jesus gick därifrån, ut ur helgedomen. Hans lärjungar trädde då fram och bådo honom giva akt på helgedomens byggnader.
\par 2 Då svarade han och sade till dem: "Ja, I sen nu allt detta; men sannerligen säger jag eder: Här skall icke lämnas sten på sten; allt skall bliva nedbrutet."
\par 3 När han sedan satt på Oljeberget, trädde hans lärjungar fram till honom, medan de voro allena, och sade: "Säg oss när detta skall ske, och vad som bliver tecknet till din tillkommelse och tidens ände."
\par 4 Då svarade Jesus och sade till dem: "Sen till, att ingen förvillar eder.
\par 5 Ty många skola komma under mitt namn och säga: 'Jag är Messias' och skola förvilla många.
\par 6 Och I skolen få höra krigslarm och rykten om krig; sen då till, att I icke förloren besinningen, ty sådant måste komma, men därmed är ännu icke änden inne.
\par 7 Ja, folk skall resa sig upp mot folk och rike mot rike, och det skall bliva hungersnöd och jordbävningar på den ena orten efter den andra;
\par 8 men allt detta är allenast begynnelsen till 'födslovåndorna'.
\par 9 Då skall man prisgiva eder till misshandling, och man skall dräpa eder, och I skolen bliva hatade av alla folk, för mitt namns skull.
\par 10 Och då skola många komma på fall, och den ene skall förråda den andre, och den ene skall hata den andre.
\par 11 Och många falska profeter skola uppstå och skola förvilla många.
\par 12 Och därigenom att laglösheten förökas, skall kärleken hos de flesta kallna.
\par 13 Men den som är ståndaktig intill änden, han skall bliva frälst.
\par 14 Och detta evangelium om riket skall bliva predikat i hela världen, till ett vittnesbörd för alla folk. Och sedan skall änden komma.
\par 15 När I nu fån se 'förödelsens styggelse', om vilken är talat genom profeten Daniel, stå på helig plats - den som läser detta, han give akt därpå -
\par 16 då må de som äro i Judeen fly bort till bergen,
\par 17 och den som är på taket må icke stiga ned för att hämta vad som finnes i hans hus,
\par 18 och den som är ute på marken må icke vända tillbaka för att hämta sin mantel.
\par 19 Och ve dem som äro havande, eller som giva di på den tiden!
\par 20 Men bedjen att eder flykt icke må ske om vintern eller på sabbaten.
\par 21 Ty då skall det bliva 'en stor vedermöda, vars like icke har förekommit allt ifrån världens begynnelse intill nu', ej heller någonsin skall förekomma.
\par 22 Och om den tiden icke bleve förkortad, så skulle intet kött bliva frälst; men för de utvaldas skull skall den tiden bliva förkortad.
\par 23 Om någon då säger till eder: 'Se här är Messias', eller: 'Där är han', så tron det icke.
\par 24 Ty människor som falskeligen säga sig vara Messias skola uppstå, så ock falska profeter, och de skola göra stora tecken och under, för att, om möjligt förvilla jämväl de utvalda.
\par 25 Jag har nu sagt eder det förut.
\par 26 Därför, om man säger till eder: 'Se, han är i öknen', så gån icke ditut, eller: 'Se, han är inne i huset', så tron det icke.
\par 27 Ty såsom ljungelden, när den går ut från öster, synes ända till väster, så skall Människosonens tillkommelse vara. -
\par 28 Där åteln är, dit skola rovfåglarna församla sig.
\par 29 Men strax efter den tidens vedermöda skall solen förmörkas och månen upphöra att giva sitt sken, och stjärnorna skola falla ifrån himmelen, och himmelens makter skola bäva.
\par 30 Och då skall Människosonens tecken visa sig på himmelen, och alla släkter på jorden skola då jämra sig. Och man skall få se 'Människosonen komma på himmelens skyar' med stor makt och härlighet.
\par 31 Och han skall sända ut sina änglar med starkt basunljud, och de skola församla hans utvalda från de fyra väderstrecken, från himmelens ena ända till den andra.
\par 32 Ifrån fikonträdet mån I här hämta en liknelse. När dess kvistar begynna att få save och löven spricka ut, då veten I att sommaren är nära.
\par 33 Likaså, när I sen allt detta, då kunnen I ock veta att han är nära och står för dörren.
\par 34 Sannerligen säger jag eder: Detta släkte skall icke förgås, förrän allt detta sker.
\par 35 Himmel och jord skola förgås, men mina ord skola aldrig förgås.
\par 36 Men om den dagen och den stunden vet ingen något, icke ens änglarna i himmelen, ingen utom Fadern allena.
\par 37 Ty såsom det skedde på Noas tid, så skall det ske vid Människosonens tillkommelse.
\par 38 Såsom människorna levde på den tiden, före floden: de åto och drucko, män togo sig hustrur, och hustrur gåvos åt män, ända till den dag då Noa gick in i arken;
\par 39 och de visste av intet, förrän floden kom och tog dem allasammans bort - så skall det ske vid Människosonens tillkommelse.
\par 40 Då skola två män vara tillsammans ute på marken; en skall bliva upptagen, och en skall lämnas kvar.
\par 41 Två kvinnor skola mala på samma kvarn; en skall bliva upptagen, och en skall lämnas kvar.
\par 42 Vaken fördenskull; ty I veten icke vilken dag vår Herre kommer.
\par 43 Men det förstån I väl, att om husbonden visste under vilken nattväkt tjuven skulle komma, så vakade han och tillstadde icke att någon bröt sig in i hans hus.
\par 44 Varen därför ock I redo; ty i en stund då I icke vänten det skall Människosonen komma.
\par 45 Finnes nu någon trogen och förståndig tjänare, som av sin herre har blivit satt över hans husfolk för att giva dem mat i rätt tid -
\par 46 salig är då den tjänaren, om hans herre, när han kommer, finner honom göra så.
\par 47 Sannerligen säger jag eder: Han skall sätta honom över allt vad han äger.
\par 48 Men om så är, att tjänaren är en ond man, som säger i sitt hjärta: 'Min herre kommer icke så snart',
\par 49 och han begynner slå sina medtjänare och äter och dricker med dem som äro druckna,
\par 50 då skall den tjänarens herre komma på en dag då han icke väntar det, och i en stund då han icke tänker sig det,
\par 51 och han skall låta hugga honom i stycken och låta honom få sin del med skrymtare. Där skall vara gråt och tandagnisslan."

\chapter{25}

\par 1 "Då skall det vara med himmelriket, såsom när tio jungfrur togo sina lampor och gingo ut för att möta brudgummen.
\par 2 Men fem av dem voro oförståndiga, och fem voro förståndiga.
\par 3 De oförståndiga togo väl sina lampor, men togo ingen olja med sig.
\par 4 De förståndiga åter togo olja i sina kärl, tillika med lamporna.
\par 5 Då nu brudgummen dröjde, blevo de alla sömniga och somnade.
\par 6 Men vid midnattstiden ljöd ett anskri: 'Se brudgummen kommer! Gån ut och möten honom.'
\par 7 Då stodo alla jungfrurna upp och redde till sina lampor.
\par 8 Och de oförståndiga sade till de förståndiga: 'Given oss av eder olja, ty våra lampor slockna.'
\par 9 Men de förståndiga svarade och sade: 'Nej, den skulle ingalunda räcka till för både oss och eder. Gån hellre bort till dem som sälja, och köpen åt eder.'
\par 10 Men när de gingo bort för att köpa, kom brudgummen, och de som voro redo gingo in med honom till bröllopet, och dörren stängdes igen.
\par 11 Omsider kommo ock de andra jungfrurna och sade: 'Herre, herre, låt upp för oss.'
\par 12 Men han svarade och sade: 'Sannerligen säger jag eder: Jag känner eder icke.'
\par 13 Vaken fördenskull; ty I veten icke dagen, ej heller stunden.
\par 14 Ty det skall ske, likasom när en man som ville fara utrikes kallade till sig sina tjänare och överlämnade åt dem sina ägodelar;
\par 15 åt en gav han fem pund, åt en annan två och åt en tredje ett pund, åt var och en efter hans förmåga, och for utrikes.
\par 16 Strax gick då den som hade fått de fem punden bort och förvaltade dem så, att han med dem vann andra fem pund.
\par 17 Den som hade fått de två punden vann på samma sätt andra två.
\par 18 Men den som hade fått ett pund gick bort och grävde en grop i jorden och gömde där sin herres penningar.
\par 19 En lång tid därefter kom tjänarnas herre hem och höll räkenskap med dem.
\par 20 Då trädde den fram, som hade fått de fem punden, och bar fram andra fem pund och sade: 'Herre, du överlämnade åt mig fem pund; se, andra fem pund har jag vunnit.'
\par 21 Hans herre svarade honom: 'Rätt så, du gode och trogne tjänare! När du var satt över det som ringa är, var du trogen; jag skall sätta dig över mycket. Gå in i din herres glädje.'
\par 22 Så trädde ock den fram, som hade fått de två punden, och sade: 'Herre, du överlämnade åt mig två pund; se, andra två pund har jag vunnit.'
\par 23 Hans herre svarade honom: 'Rätt så, du gode och trogne tjänare! När du var satt över det som ringa är, var du trogen; jag skall sätta dig över mycket. Gå in i din herres glädje.'
\par 24 Sedan trädde ock den fram, som hade fått ett pund, och sade: 'Herre, jag hade lärt känna dig såsom en sträng man, som vill skörda, där du icke har sått, och inbärga, där du icke har utstrött;
\par 25 och av fruktan för dig gick jag bort och gömde ditt pund i jorden. Se här har du vad dig tillhör.'
\par 26 Då svarade hans herre och sade till honom: 'Du onde och late tjänare, du visste att jag vill skörda, där jag icke har sått, och inbärga, där jag icke har utstrött?
\par 27 Då borde du också hava satt in mina penningar i en bank, så att jag hade fått igen mitt med ränta, när jag kom hem.
\par 28 Tagen därför ifrån honom hans pund, och given det åt den som har de tio punden.
\par 29 Ty var och en som har, åt honom skall varda givet, så att han får över nog; men den som icke har, från honom skall tagas också det han har.
\par 30 Och kasten den oduglige tjänaren ut i mörkret härutanför.' Där skall vara gråt och tandagnisslan.
\par 31 Men när Människosonen kommer i sin härlighet, och alla änglar med honom, då skall han sätta sig på sin härlighets tron.
\par 32 Och inför honom skola församlas alla folk och han skall skilja dem från varandra, såsom en herde skiljer fåren ifrån getterna.
\par 33 Och fåren skall han ställa på sin högra sida, och getterna på den vänstra.
\par 34 Därefter skall Konungen säga till dem som stå på hans högra sida: 'Kommen, I min Faders välsignade, och tagen i besittning det rike som är tillrett åt eder från världens begynnelse.
\par 35 Ty jag var hungrig, och I gåven mig att äta; jag var törstig, och I gåven mig att dricka; jag var husvill, och I gåven mig härbärge,
\par 36 naken, och I klädden mig; jag var sjuk, och I besökten mig; jag var i fängelse, och I kommen till mig.'
\par 37 Då skola de rättfärdiga svara honom och säga: 'Herre, när sågo vi dig hungrig och gåvo dig mat, eller törstig och gåvo dig att dricka?
\par 38 Och när sågo vi dig husvill och gåvo dig härbärge, eller naken och klädde dig?
\par 39 Och när sågo vi dig sjuk eller i fängelse och kommo till dig?'
\par 40 Då skall Konungen svara och säga till dem: 'Sannerligen säger jag eder: Vadhelst I haven gjort mot en av dessa mina minsta bröder, det haven I gjort mot mig.'
\par 41 Därefter skall han ock säga till dem som stå på hans vänstra sida: 'Gån bort ifrån mig, I förbannade, till den eviga elden, som är tillredd åt djävulen och hans änglar.
\par 42 Ty jag var hungrig, och I gåven mig icke att äta; jag var törstig, och I gåven mig icke att dricka;
\par 43 jag var husvill, och I gåven mig icke härbärge, naken, och I klädden mig icke, sjuk och i fängelse, och I besökten mig icke.'
\par 44 Då skola också de svara och säga: 'Herre, när sågo vi dig hungrig eller törstig eller husvill eller naken eller sjuk eller i fängelse och tjänade dig icke?'
\par 45 Då skall han svara dem och säga: 'Sannerligen säger jag eder: Vadhelst I icke haven gjort mot en av dessa minsta, det haven I ej heller gjort mot mig.'
\par 46 Och dessa skola då då bort till evigt straff, men de rättfärdiga till evigt liv."

\chapter{26}

\par 1 När nu Jesus hade talat allt detta till slut, sade han till sina lärjungar:
\par 2 "I veten att det två dagar härefter är påsk; då skall Människosonen bliva förrådd och utlämnad till att korsfästas."
\par 3 Därefter församlade sig översteprästerna och folkets äldste hos översteprästen, som hette Kaifas, i hans hus,
\par 4 och rådslogo om att låta gripa Jesus med list och döda honom.
\par 5 Men de sade: "Icke under högtiden, för att ej oroligheter skola uppstå bland folket."
\par 6 Men när Jesus var i Betania, i Simon den spetälskes hus,
\par 7 framträdde till honom en kvinna som hade med sig en alabasterflaska med dyrbar smörjelse; denna göt hon ut över hans huvud, där han låg till bords.
\par 8 Då lärjungarna sågo detta, blevo de misslynta och sade: "Varför skulle detta förspillas?
\par 9 Man hade ju kunnat sälja det för mycket penningar och giva dessa åt de fattiga."
\par 10 När Jesus märkte detta, sade han till dem: "Varför oroen I kvinnan? Det är en god gärning som hon har gjort mot mig.
\par 11 De fattiga haven I ju alltid ibland eder, men mig haven I icke alltid.
\par 12 När hon göt ut denna smörjelse på min kropp, gjorde hon det såsom en tillredelse till min begravning.
\par 13 Sannerligen säger jag eder: Varhelst i hela världen detta evangelium bliver predikat, där skall ock det som hon nu har gjort bliva omtalat, henne till åminnelse."
\par 14 Därefter gick en av de tolv, den som hette Judas Iskariot, bort till översteprästerna
\par 15 och sade: "Vad viljen I giva mig för att jag skall överlämna honom åt eder?" Då vägde de upp åt honom trettio silverpenningar.
\par 16 Och från den stunden sökte han efter lägligt tillfälle att förråda honom.
\par 17 Men på första dagen i det osyrade brödets högtid trädde lärjungarna fram till Jesus och frågade: "Var vill du att vi skola reda till åt dig att äta påskalammet?"
\par 18 Han svarade: "Gån in i staden till den och den och sägen till honom: 'Mästaren låter säga: Min tid är nära; hos dig vill jag hålla påskhögtid med mina lärjungar.'"
\par 19 Och lärjungarna gjorde såsom Jesus hade befallt dem och redde till påskalammet.
\par 20 När det nu hade blivit afton, lade han sig till bords med de tolv.
\par 21 Och medan de åto, sade han: "Sannerligen säger jag eder: En av eder skall förråda mig."
\par 22 Då blevo de mycket bedrövade och begynte fråga honom, var efter annan: "Icke är det väl jag, Herre?"
\par 23 Då svarade han och sade: "Den som jämte mig nu doppade handen i fatet, han skall förråda mig.
\par 24 Människosonen skall gå bort, såsom det är skrivet om honom; men ve den människa genom vilken Människosonen bliver förrådd! Det hade varit bättre för den människan, om hon icke hade blivit född."
\par 25 Judas, han som förrådde honom, tog då till orda och frågade: "Rabbi, icke är det väl jag?" Han svarade honom: "Du har själv sagt det."
\par 26 Medan de nu åto, tog Jesus ett bröd och välsignade det och bröt det och gav åt lärjungarna och sade: "Tagen och äten; detta är min lekamen."
\par 27 Och han tog en kalk och tackade Gud och gav åt dem och sade: "Dricken härav alla;
\par 28 ty detta är mitt blod, förbundsblodet, som varder utgjutet för många till syndernas förlåtelse.
\par 29 Och jag säger eder: Härefter skall jag icke mer dricka av det som kommer från vinträd, förrän på den dag då jag dricker det nytt med eder i min Faders rike."
\par 30 När de sedan hade sjungit lovsången, gingo de ut till Oljeberget.
\par 31 Då sade Jesus till dem: "I denna natt skolen I alla komma på fall för min skull, ty det är skrivet: 'Jag skall slå herden, och fåren i hjorden skola förskingras.'
\par 32 Men efter min uppståndelse skall jag före eder gå till Galileen."
\par 33 Då svarade Petrus och sade till honom: "Om än alla andra komma på fall för din skull, så skall dock jag aldrig komma på fall."
\par 34 Jesus sade till honom: "Sannerligen säger jag dig: I denna natt, förrän hanen har galit, skall du tre gånger förneka mig."
\par 35 Petrus svarade honom: "Om jag än måste dö med dig, så skall jag dock förvisso icke förneka dig." Sammalunda sade ock alla de andra lärjungarna.
\par 36 Därefter kom Jesus med dem till ett ställe som hette Getsemane. Och han sade till lärjungarna: "Bliven kvar här, medan jag går dit bort och beder."
\par 37 Och han tog med sig Petrus och Sebedeus' två söner; och han begynte bedrövas och ängslas.
\par 38 Då sade han till dem: "Min själ är djupt bedrövad, ända till döds; stannen kvar här och vaken med mig."
\par 39 Därefter gick han litet längre bort och föll ned på sitt ansikte och bad och sade: "Min Fader, om det är möjligt, så gånge denna kalk ifrån mig. Dock icke såsom jag vill, utan såsom du vill!"
\par 40 Sedan kom han tillbaka till lärjungarna och fann dem sovande. Då sade han till Petrus: "Så litet förmådden I då vaka en kort stund med mig!
\par 41 Vaken, och bedjen att I icke mån komma i frestelse. Anden är villig, men köttet är svagt."
\par 42 Åter gick han bort, för andra gången, och bad och sade: "Min Fader, om detta icke kan gå ifrån mig, utan jag måste dricka denna kalk, så ske din vilja."
\par 43 När han sedan kom tillbaka, fann han dem åter sovande, ty deras ögon voro förtyngda.
\par 44 Då lät han dem vara och gick åter bort och bad, för tredje gången, och sade återigen samma ord.
\par 45 Därefter kom han tillbaka till lärjungarna och sade till dem: "Ja, I soven ännu alltjämt och vilen eder! Se, stunden är nära då Människosonen skall bliva överlämnad i syndares händer.
\par 46 Stån upp, låt oss gå; se, den är nära, som förråder mig."
\par 47 Och se, medan han ännu talade, kom Judas, en av de tolv, och jämte honom en stor folkskara, med svärd och stavar, utsänd från översteprästerna och folkets äldste.
\par 48 Men förrädaren hade givit dem ett tecken; han hade sagt: "Den som jag kysser, den är det; honom skolen I gripa."
\par 49 Och han trädde nu strax fram till Jesus och sade: "Hell dig, rabbi!" och kysste honom häftigt.
\par 50 Jesus sade till honom: "Min vän, gör vad du är här för att göra." Då stego de fram och grepo Jesus och togo honom fången.
\par 51 Men en av dem som voro med Jesus förde handen till sitt svärd och drog ut det och högg till översteprästens tjänare och högg så av honom örat.
\par 52 Då sade Jesus till honom: "Stick ditt svärd tillbaka i skidan: ty alla som taga till svärd skola förgöras genom svärd.
\par 53 Eller menar du att jag icke kunde utbedja mig av min Fader, att han nu sände till min tjänst mer än tolv legioner änglar?
\par 54 Men huru bleve då skrifterna fullbordade, som säga att så måste ske?"
\par 55 I samma stund sade Jesus till folkskaran: "Såsom mot en rövare haven I gått ut med svärd och stavar för att fasttaga mig. Var dag har jag suttit i helgedomen och undervisat, utan att I haven gripit mig.
\par 56 Men allt detta har skett, för att profeternas skrifter skola fullbordas." Då övergåvo alla lärjungarna honom och flydde.
\par 57 Men de som hade gripit Jesus förde honom bort till översteprästen Kaifas, hos vilken de skriftlärde och de äldste hade församlat sig.
\par 58 Och Petrus följde honom på avstånd ända till översteprästens gård; där gick han in och satte sig bland rättstjänarna för att se vad slutet skulle bliva.
\par 59 Och översteprästerna och hela Stora rådet sökte efter något falskt vittnesbörd mot Jesus, för att kunna döda honom;
\par 60 men fastän många falska vittnen trädde fram, funno de likväl intet. Slutligen trädde dock två män fram
\par 61 och sade: "Denne har sagt: 'Jag kan bryta ned Guds tempel och på tre dagar bygga upp det igen.'"
\par 62 Då stod översteprästen upp och sade till honom: "Svarar du intet? Huru är det med det som dessa vittna mot dig?"
\par 63 Men Jesus teg. Då sade översteprästen till honom: "Jag besvär dig vid den levande Guden, att du säger oss om du är Messias, Guds Son."
\par 64 Jesus svarade honom: "Du har själv sagt det. Men jag säger eder: Härefter skolen I få se Människosonen sitta på Maktens högra sida och komma på himmelens skyar."
\par 65 Då rev översteprästen sönder sina kläder och sade: "Han har hädat. Vad behöva vi mer några vittnen? I haven nu hört hädelsen.
\par 66 Vad synes eder?" De svarade och sade: "Han är skyldig till döden."
\par 67 Därefter spottade man honom i ansiktet och slog honom på kinderna, den ene med knytnäven, den andre med flata handen,
\par 68 och sade: "Profetera för oss, Messias: vem var det som slog dig?"
\par 69 Men Petrus satt utanför på gården. Då kom en tjänstekvinna fram till honom och sade: "Också du var med Jesus från Galileen."
\par 70 Men han nekade inför alla och sade: "Jag förstår icke vad du menar."
\par 71 När han sedan hade kommit ut i porten, fick en annan kvinna se honom och sade till dem som voro där: "Denne var med Jesus från Nasaret."
\par 72 Åter nekade han med en ed och sade: "Jag känner icke den mannen."
\par 73 Litet därefter kommo de kringstående fram och sade till Petrus: "Förvisso är också du en av dem; redan ditt uttal röjer dig ju."
\par 74 Då begynte han förbanna sig och svärja: "Jag känner icke den mannen." Och i detsamma gol hanen.
\par 75 Då kom Petrus ihåg Jesu ord, huru han hade sagt: "Förrän hanen gal, skall du tre gånger förneka mig." Och han gick ut och grät bitterligen.

\chapter{27}

\par 1 Men när det hade blivit morgon, fattade alla översteprästerna och folkets äldste det beslutet angående Jesus, att de skulle döda honom.
\par 2 Och de läto binda honom och förde honom bort och överlämnade honom åt Pilatus, landshövdingen.
\par 3 När då Judas, som hade förrått honom, såg att han var dömd, ångrade han sig och bar de trettio silverpenningarna tillbaka till översteprästerna och de äldste
\par 4 och sade: "Jag har syndat därigenom att jag har förrått oskyldigt blod." Men de svarade: "Vad kommer det oss vid? Du får själv svara därför."
\par 5 Då kastade han silverpenningarna i templet och gick sin väg. Sedan gick han bort och hängde sig.
\par 6 Men översteprästerna togo silverpenningarna och sade: "Det är icke lovligt att lägga dem i offerkistan, eftersom det är blodspenningar."
\par 7 Och sedan de hade fattat sitt beslut, köpte de för dem Krukmakaråkern till begravningsplats för främlingar.
\par 8 Därför kallas den åkern ännu i dag Blodsåkern.
\par 9 Så fullbordades det som var sagt genom profeten Jeremias, när han sade: "Och jag tog de trettio silverpenningarna - priset för den man vilkens värde hade blivit bestämt, den som israelitiska män hade värderat -
\par 10 och jag gav dem till betalning för Krukmakaråkern, i enlighet med Herrens befallning till mig."
\par 11 Men Jesus ställdes fram inför landshövdingen. Och landshövdingen frågade honom och sade: "Är du judarnas konung?" Jesus svarade honom: "Du säger det själv."
\par 12 Men när översteprästerna och de äldste framställde sina anklagelser mot honom, svarade han intet.
\par 13 Då sade Pilatus till honom: "Hör du icke huru mycket de hava att vittna mot dig?"
\par 14 Men han svarade honom icke på en enda fråga, så att landshövdingen mycket förundrade sig.
\par 15 Nu var det sed att landshövdingen vid högtiden gav folket en fånge lös, vilken de ville.
\par 16 Och man hade då en beryktad fånge, som hette Barabbas.
\par 17 När de nu voro församlade, frågade Pilatus dem: "Vilken viljen I att jag skall giva eder lös, Barabbas eller Jesus, som kallas Messias?"
\par 18 Han visste nämligen att det var av avund som man hade dragit Jesus inför rätta.
\par 19 Och medan han satt på domarsätet, hade hans hustru sänt bud till honom och låtit säga: "Befatta dig icke med denne rättfärdige man; ty jag har i natt lidit mycket i drömmen för hans skull."
\par 20 Men översteprästerna och de äldste hade övertalat folket att begära Barabbas och låta förgöra Jesus.
\par 21 När alltså landshövdingen nu frågade dem och sade: "Vilken av de två viljen I att jag skall giva eder lös?", så svarade de: "Barabbas."
\par 22 Då frågade Pilatus dem: "Vad skall jag då göra med Jesus, som kallas Messias?" De svarade alla: "Låt korsfästa honom."
\par 23 Men han frågade: "Vad ont har han då gjort?" Då skriade de ännu ivrigare: "Låt korsfästa honom."
\par 24 När nu Pilatus såg att han intet kunde uträtta, utan att larmet blev allt starkare, lät han hämta vatten och tvådde sina händer i folkets åsyn och sade: "Jag är oskyldig till denne mans blod. I fån själva svara därför."
\par 25 Och allt folket svarade och sade: "Hans blod komme över oss och över våra barn."
\par 26 Då gav han dem Barabbas lös; men Jesus lät han gissla och utlämnade honom sedan till att korsfästas.
\par 27 Då togo landshövdingens krigsmän Jesus med sig in i pretoriet och församlade hela den romerska vakten omkring honom.
\par 28 Och de togo av honom hans kläder och satte på honom en röd mantel
\par 29 och vredo samman en krona av törnen och satte den på hans huvud, och i hans högra hand satte de ett rör. Sedan böjde de knä inför honom och begabbade honom och sade: "Hell dig, judarnas konung!"
\par 30 Och de spottade på honom och togo röret och slogo honom därmed i huvudet.
\par 31 Och när de så hade begabbat honom, klädde de av honom manteln och satte på honom hans egna kläder och förde honom bort till att korsfästas.
\par 32 Då de nu voro på väg ditut, träffade de på en man från Cyrene, som hette Simon. Honom tvingade de att gå med och bära hans kors.
\par 33 Och när de hade kommit till en plats som kallades Golgata (det betyder huvudskalleplats),
\par 34 räckte de honom vin att dricka, blandat med galla; men då han hade smakat därpå, ville han icke dricka det.
\par 35 Och när de hade korsfäst honom, delade de hans kläder mellan sig genom att kasta lott om dem.
\par 36 Sedan sutto de där och höllo vakt om honom.
\par 37 Och över hans huvud hade man satt upp en överskrift, som angav vad han var anklagad för, och den lydde så: "Denne är Jesus, judarnas konung."
\par 38 Med honom korsfästes då ock två rövare, den ene på högra sidan och den andre på vänstra.
\par 39 Och de som gingo där förbi bespottade honom och skakade huvudet
\par 40 och sade: "Du som bryter ned templet och inom tre dagar bygger upp det igen, hjälp dig nu själv, om du är Guds Son, och stig ned från korset."
\par 41 Sammalunda talade ock översteprästerna, jämte de skriftlärde och de äldste, begabbande ord och sade:
\par 42 "Andra har han hjälpt; sig själv kan han icke hjälpa. Han är ju Israels konung; han stige nu ned från korset, så vilja vi tro på honom.
\par 43 Han har satt sin förtröstan på Gud, må nu han frälsa honom, om han har behag till honom, han har ju sagt: 'Jag är Guds Son.'"
\par 44 På samma sätt smädade honom också rövarna som voro korsfästa med honom.
\par 45 Men vid sjätte timmen kom över hela landet ett mörker, som varade ända till nionde timmen.
\par 46 Och vid nionde timmen ropade Jesus med hög röst och sade: "Eli, Eli, lema sabaktani?"; det betyder: "Min Gud, min Gud, varför har du övergivit mig?"
\par 47 Men när några av dem som stodo där borde detta, sade de: "Han kallar på Elias."
\par 48 Och strax skyndade en av dem fram och tog en svamp och fyllde den med ättikvin och satte den på ett rör och gav honom att dricka.
\par 49 Men de andra sade: "Låt oss se om Elias kommer och hjälper honom."
\par 50 Åter ropade Jesus med hög röst och gav upp andan.
\par 51 Och se, då rämnade förlåten i templet i två stycken, uppifrån och ända ned, och jorden skalv, och klipporna rämnade,
\par 52 och gravarna öppnades, och många avsomnade heligas kroppar stodo upp.
\par 53 De gingo ut ur sina gravar och kommo efter hans uppståndelse in i den heliga staden och uppenbarade sig för många.
\par 54 Men när hövitsmannen och de som med honom höllo vakt om Jesus sågo jordbävningen och det övriga som skedde, blevo de mycket förskräckta och sade: "Förvisso var denne Guds Son."
\par 55 Och många kvinnor som hade följt Jesus från Galileen och tjänat honom, stodo där på avstånd och sågo vad som skedde.
\par 56 Bland dessa voro Maria från Magdala och den Maria som var Jakobs och Joses' moder, så ock Sebedeus' söners moder.
\par 57 Men när det hade blivit afton, kom en rik man från Arimatea, vid namn Josef, som ock hade blivit en Jesu lärjunge;
\par 58 denne gick till Pilatus och utbad sig att få Jesu kropp. Då bjöd Pilatus att man skulle lämna ut den åt honom.
\par 59 Och Josef tog hans kropp och svepte den i en ren linneduk
\par 60 och lade den i den nya grav som han hade låtit hugga ut åt sig i klippan; och sedan han hade vältrat en stor sten för ingången till graven, gick han därifrån.
\par 61 Men Maria från Magdala och den andra Maria voro där, och de sutto gent emot graven.
\par 62 Följande dag, som var dagen efter tillredelsedagen, församlade sig översteprästerna och fariséerna och gingo till Pilatus
\par 63 och sade: "Herre, vi hava dragit oss till minnes att den villoläraren sade, medan han ännu levde: 'Efter tre dagar skall jag uppstå.'
\par 64 Bjud fördenskull att man skyddar graven intill tredje dagen, så att hans lärjungar icke komma och stjäla bort honom, och sedan säga till folket att han har uppstått från de döda. Då bliver den sista villan värre än den första."
\par 65 Pilatus svarade dem: "Där haven I vakt; gån åstad och skydden graven så gott I kunnen."
\par 66 Och de gingo åstad och skyddade graven, i det att de icke allenast satte ut vakten, utan ock förseglade stenen.

\chapter{28}

\par 1 När sabbaten hade gått till ända, i gryningen till första veckodagen, kommo Maria från Magdala och den andra Maria för att se graven.
\par 2 Då blev det en stor jordbävning; ty en Herrens ängel steg ned från himmelen och gick fram och vältrade bort stenen och satte sig på den.
\par 3 Och han var att skåda såsom en ljungeld, och hans kläder voro vita såsom snö.
\par 4 Och väktarna skälvde av förskräckelse för honom och blevo såsom döda.
\par 5 Men ängeln talade och sade till kvinnorna: "Varen I icke förskräckta; jag vet att I söken Jesus, den korsfäste.
\par 6 Han är icke här, ty han är uppstånden, såsom han hade förutsagt. Kommen hit, och sen platsen där han har legat.
\par 7 Och gån så åstad med hast, och sägen till hans lärjungar att han är uppstånden från de döda. Och han skall före eder gå till Galileen; där skolen I få se honom. Jag har nu sagt eder det."
\par 8 Och de gingo med hast bort ifrån graven, under fruktan och med stor glädje, och skyndade åstad för att omtala det för hans lärjungar.
\par 9 Men se, då kom Jesus emot dem och sade: "Hell eder!" Och de gingo fram och fattade om hans fötter och tillbådo honom.
\par 10 Då sade Jesus till dem: "Frukten icke; gån åstad och omtalen detta för mina bröder, på det att de må gå till Galileen; där skola de få se mig."
\par 11 Men under det att de voro på vägen, kommo några av väktarna till staden och underrättade översteprästerna om allt det som hade hänt.
\par 12 Då församlade sig dessa jämte de äldste; och sedan de hade fattat sitt beslut, gåvo de en ganska stor summa penningar åt krigsmännen
\par 13 och sade: "Så skolen I säga: 'Hans lärjungar kommo om natten och stulo bort honom, medan vi sovo.'
\par 14 Och om saken kommer för landshövdingens öron, så skola vi ställa honom till freds och sörja för, att I kunnen vara utan bekymmer."
\par 15 Och de togo emot penningarna och gjorde såsom man hade lärt dem. Och det talet utspriddes bland judarna och är gängse bland dem ännu i denna dag.
\par 16 Men de elva lärjungarna begåvo sig till det berg i Galileen, dit Jesus hade bjudit dem att gå.
\par 17 Och när de fingo se honom, tillbådo de honom. Dock funnos några som tvivlade.
\par 18 Då trädde Jesus fram och talade till dem och sade: "Mig är given all makt i himmelen och på jorden.
\par 19 Gån fördenskull ut och gören alla folk till lärjungar, döpande dem i Faderns och Sonens och den helige Andes namn,
\par 20 lärande dem att hålla allt vad jag har befallt eder. Och se, jag är med eder alla dagar intill tidens ände."


\end{document}