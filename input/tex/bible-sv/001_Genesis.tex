\begin{document}

\title{1 Moseboken}


\chapter{1}

\par 1 I begynnelsen skapade Gud himmel och jord.
\par 2 Och jorden var öde och tom, och mörker var över djupet, och Guds Ande svävade över vattnet.
\par 3 Och Gud sade: "Varde ljus"; och det vart ljus.
\par 4 Och Gud såg att ljuset var gott; och Gud skilde ljuset från mörkret.
\par 5 Och Gud kallade ljuset dag, och mörkret kallade han natt. Och det vart afton, och det vart morgon, den första dagen.
\par 6 Och Gud sade: "Varde mitt i vattnet ett fäste som skiljer vatten från vatten."
\par 7 Och Gud gjorde fästet, och skilde vattnet under fästet från vattnet ovan fästet; och det skedde så.
\par 8 Och Gud kallade fästet himmel. Och det vart afton, och det vart morgon, den andra dagen.
\par 9 Och Gud sade: "Samle sig det vatten som är under himmelen till en särskild plats, så att det torra bliver synligt." Och det skedde så.
\par 10 Och Gud kallade det torra jord, och vattensamlingen kallade han hav. Och Gud såg att det var gott.
\par 11 Och Gud sade: "Frambringe jorden grönska, fröbärande örter och fruktträd, som efter sina arter bära frukt, vari de hava sitt frö, på jorden." Och det skedde så;
\par 12 jorden frambragte grönska, fröbärande örter, efter deras arter, och träd som efter sina arter buro frukt, vari de hade sitt frö. Och Gud såg att det var gott.
\par 13 Och det vart afton, och det vart morgon, den tredje dagen.
\par 14 Och Gud sade: "Varde på himmelens fäste ljus som skilja dagen från natten, och vare de till tecken och till att utmärka särskilda tider, dagar och år,
\par 15 och vare de på himmelens fäste till ljus som lysa över jorden." Och det skedde så;
\par 16 Gud gjorde de två stora ljusen, det större ljuset till att råda över dagen, och det mindre ljuset till att råda över natten, så ock stjärnorna.
\par 17 Och Gud satte dem på himmelens fäste till att lysa över jorden,
\par 18 och till att råda över dagen och över natten, och till att skilja ljuset från mörkret. Och Gud såg att det var gott.
\par 19 Och det vart afton, och det vart morgon, den fjärde dagen.
\par 20 Och Gud sade: "Frambringe vattnet ett vimmel av levande varelser; flyge ock fåglar över jorden under himmelens fäste."
\par 21 Och Gud skapade de stora havsdjuren och hela det stim av levande varelser, som vattnet vimlar av, efter deras arter, så ock alla bevingade fåglar, efter deras arter. Och Gud såg att det var gott.
\par 22 Och Gud välsignade dem och sade: "Varen fruktsamma och föröken eder, och uppfyllen vattnet i haven; föröke sig ock fåglarna på jorden."
\par 23 Och det vart afton, och det vart morgon, den femte dagen.
\par 24 Och Gud sade: "Frambringe jorden levande varelser, efter deras arter, boskapsdjur och kräldjur och vilda djur, efter deras arter." Och det skedde så;
\par 25 Gud gjorde de vilda djuren, efter deras arter, och boskapsdjuren, efter deras arter, och alla kräldjur på marken, efter deras arter. Och Gud såg att det var gott.
\par 26 Och Gud sade: "Låt oss göra människor till vår avbild, till att vara oss lika; och må de råda över fiskarna i havet och över fåglarna under himmelen och över boskapsdjuren och över hela jorden och över alla kräldjur som röra sig på jorden."
\par 27 Och Gud skapade människan till sin avbild, till Guds avbild skapade han henne, till man och kvinna skapade han dem.
\par 28 Och Gud välsignade dem; Gud sade till dem: "Varen fruktsamma och föröken eder, och uppfyllen jorden och läggen den under eder; och råden över fiskarna i havet och över fåglarna under himmelen och över alla djur som röra sig på jorden."
\par 29 Och Gud sade: "Se, jag giver eder alla fröbärande örter på hela jorden och alla träd med fröbärande trädfrukt; detta skolen I hava till föda.
\par 30 Men åt alla djur på jorden och åt alla fåglar under himmelen och åt allt som krälar på jorden, vad som i sig har en levande själ, åt dessa giver jag alla gröna örter till föda." Och det skedde så.
\par 31 Och Gud såg på allt som han hade gjort, och se, det var mycket gott. Och det vart afton, och det vart morgon, den sjätte dagen.

\chapter{2}

\par 1 Så blevo nu himmelen och jorden fullbordade med hela sin härskara.
\par 2 Och Gud fullbordade på sjunde dagen det verk som han hade gjort; och han vilade på sjunde dagen från allt det verk som han hade gjort.
\par 3 Och Gud välsignade den sjunde dagen och helgade den, därför att han på den dagen vilade från allt sitt verk, det som Gud hade gjort, när han skapade.
\par 4 Detta är berättelsen om den ordning i vilken allt blev till på himmelen och jorden, när de skapades, då när HERREN Gud gjorde jord och himmel.
\par 5 Då bar jorden ännu ingen buske på marken, och ingen ört hade ännu skjutit upp på marken, ty HERREN Gud hade icke låtit regna på jorden, och ingen människa fanns, som kunde bruka jorden;
\par 6 men en dimma steg upp från jorden och vattnade hela marken.
\par 7 Och HERREN Gud danade människan av stoft från jorden och inblåste livsande i hennes näsa, och så blev människan en levande varelse.
\par 8 Och HERREN Gud planterade en lustgård i Eden österut och satte däri människan som han hade danat.
\par 9 HERREN Gud lät nämligen alla slags träd som voro ljuvliga att se på och goda att äta av växa upp ur marken, och livets träd mitt i lustgården, så ock kunskapens träd på gott och ont.
\par 10 Och från Eden gick en flod ut, som vattnade lustgården; sedan delade den sig i fyra grenar.
\par 11 Den första heter Pison; det är den som flyter omkring hela landet Havila, där guld finnes,
\par 12 och det landets guld är gott; där finnes ock bdelliumharts och onyxsten.
\par 13 Den andra floden heter Gihon; det är den som flyter omkring hela landet Kus.
\par 14 Den tredje floden heter Hiddekel; det är den som har sitt lopp öster om Assyrien. Den fjärde floden är Frat.
\par 15 Så tog nu HERREN Gud mannen och satte honom i Edens lustgård, till att bruka och bevara den.
\par 16 Och HERREN Gud bjöd mannen och sade: "Av alla andra träd i lustgården må du fritt äta,
\par 17 men av kunskapens träd på gott och ont skall du icke äta, ty när du äter därav, skall du döden dö."
\par 18 Och HERREN Gud sade: "Det är icke gott att mannen är allena. Jag vill göra åt honom en hjälp, en sådan som honom höves."
\par 19 Och HERREN Gud danade av jord alla markens djur och alla himmelens fåglar, och förde dem fram till mannen för att se huru denne skulle kalla dem; ty såsom mannen kallade var levande varelse, så skulle den heta.
\par 20 Och mannen gav namn åt alla boskapsdjur, åt fåglarna under himmelen och åt alla markens djur. Men för Adam fann han icke någon hjälp, sådan som honom hövdes.
\par 21 Då lät HERREN Gud en tung sömn falla på mannen, och när han hade somnat, tog han ut ett av hans revben och fyllde dess plats med kött.
\par 22 Och HERREN Gud byggde en kvinna av revbenet som han hade tagit av mannen, och förde henne fram till mannen.
\par 23 Då sade mannen: "Ja, denna är nu ben av mina ben och kött av mitt kött. Hon skall heta maninna, ty av man är hon tagen."
\par 24 Fördenskull skall en man övergiva sin fader och sin moder och hålla sig till sin hustru, och de skola varda ett kött.
\par 25 Och mannen och hans hustru voro båda nakna och blygdes icke för varandra.

\chapter{3}

\par 1 Men ormen var listigare än alla andra markens djur som HERREN Gud hade gjort; och han sade till kvinnan: "Skulle då Gud hava sagt: 'I skolen icke äta av något träd i lustgården'?"
\par 2 Kvinnan svarade ormen: "Vi få äta av frukten på de andra träden i lustgården,
\par 3 men om frukten på det träd som står mitt i lustgården har Gud sagt: 'I skolen icke äta därav, ej heller komma därvid, på det att I icke mån dö.'"
\par 4 Då sade ormen till kvinnan: "Ingalunda skolen I dö;
\par 5 men Gud vet, att när I äten därav, skola edra ögon öppnas, så att I bliven såsom Gud och förstån vad gott och ont är."
\par 6 Och kvinnan såg att trädet var gott att äta av, och att det var en lust för ögonen, och att det var ett ljuvligt träd, eftersom man därav fick förstånd, och hon tog av dess frukt och åt; och hon gav jämväl åt sin man, som var med henne, och han åt.
\par 7 Då öppnades bådas ögon, och de blevo varse att de voro nakna; och de fäste ihop fikonlöv och bundo omkring sig.
\par 8 Och de hörde HERREN Gud vandra i lustgården, när dagen begynte svalkas; då gömde sig mannen med sin hustru för HERREN Guds ansikte bland träden i lustgården.
\par 9 Men HERREN Gud kallade på mannen och sade till honom: "Var är du?"
\par 10 Han svarade: "Jag hörde dig i lustgården; då blev jag förskräckt, eftersom jag är naken; därför gömde jag mig."
\par 11 Då sade han: "Vem har låtit dig förstå att du är naken? Har du icke ätit av det träd som jag förbjöd dig att äta av?"
\par 12 Mannen svarade: "Kvinnan som du har givit mig till att vara med mig, hon gav mig av trädet, så att jag åt."
\par 13 Då sade HERREN Gud till kvinnan: "Vad är det du har gjort!" Kvinnan svarade: "Ormen bedrog mig, så att jag åt."
\par 14 Då sade HERREN Gud till ormen: "Eftersom du har gjort detta, vare du förbannad bland alla djur, boskapsdjur och vilda djur. På din buk skall du gå, och stoft skall du äta i alla dina livsdagar.
\par 15 Och jag skall sätta fiendskap mellan dig och kvinnan, och mellan din säd och hennes säd. Denna skall söndertrampa ditt huvud, och du skall stinga den i hälen."
\par 16 Och till kvinnan sade han: "Jag skall låta dig utstå mycken vedermöda, när du bliver havande; med smärta skall du föda dina barn. Men till din man skall din åtrå vara, och han skall råda över dig."
\par 17 Och till Adam sade han: "Eftersom du lyssnade till din hustrus ord och åt av det träd om vilket jag hade bjudit dig och sagt: 'Du skall icke äta därav', därför vare marken förbannad för din skull. Med vedermöda skall du nära dig av den i alla dina livsdagar;
\par 18 törne och tistel skall den bära åt dig, men markens örter skola vara din föda.
\par 19 I ditt anletes svett skall du äta ditt bröd, till dess du vänder åter till jorden; ty av den är du tagen. Ty du är stoft, och till stoft skall du åter varda."
\par 20 Och mannen gav sin hustru namnet Eva, ty hon blev en moder åt allt levande.
\par 21 Och HERREN Gud gjorde åt Adam och hans hustru kläder av skinn och satte på dem.
\par 22 Och HERREN Gud sade: "Se, mannen har blivit såsom en av oss, så att han förstår vad gott och ont är. Må han nu icke räcka ut sin hand och taga jämväl av livets träd och äta, och så leva evinnerligen."
\par 23 Och HERREN Gud förvisade honom ur Edens lustgård, för att han skulle bruka jorden, varav han var tagen.
\par 24 Och han drev ut mannen, och satte öster om Edens lustgård keruberna jämte det ljungande svärdets lågor, för att bevaka vägen till livets träd.

\chapter{4}

\par 1 Och mannen kände sin hustru Eva, och hon blev havande och födde Kain; då sade hon: "Jag har fött en man genom HERRENS hjälp."
\par 2 Och hon födde åter en son, Abel, den förres broder. Och Abel blev en fårherde, men Kain blev en åkerman.
\par 3 Och efter någon tid hände sig att Kain av markens frukt bar fram en offergåva åt HERREN.
\par 4 Också Abel bar fram sin gåva, av det förstfödda i hans hjord, av djurens fett. Och HERREN såg till Abel och hans offergåva;
\par 5 men till Kain och hans offergåva såg han icke. Då blev Kain mycket vred, och hans blick blev mörk.
\par 6 Och HERREN sade till Kain: "Varför är du vred, och varför är din blick så mörk?
\par 7 Är det icke så: om du har gott i sinnet, då ser du frimodigt upp; men om du icke har gott i sinnet, då lurar synden vid dörren; till dig står hennes åtrå, men du bör råda över henne."
\par 8 Och Kain talade med sin broder Abel; och när de voro ute på marken, överföll Kain sin broder Abel och dräpte honom.
\par 9 Då sade HERREN till Kain: "Var är din broder Abel?" Han svarade: "Jag vet icke; skall jag taga vara på min broder?"
\par 10 Då sade han: "Vad har du gjort! Hör, din broders blod ropar till mig från jorden.
\par 11 Så vare du nu förbannad och förvisad ifrån åkerjorden, som har öppnat sin mun för att mottaga din broders blod av din hand.
\par 12 När du brukar jorden, skall den icke mer giva dig sin gröda. Ostadig och flyktig skall du bliva på jorden."
\par 13 Då sade Kain till HERREN: "Min missgärning är större än att jag kan bära den.
\par 14 Se, du driver mig nu bort ifrån åkerjorden, och jag måste gömma mig undan för ditt ansikte. Ostadig och flyktig skall jag bliva på jorden, och så skall ske att vemhelst som möter mig, han dräper mig."
\par 15 Men HERREN sade till honom: "Nej, ty Kain skall bliva hämnad sjufalt, vemhelst som dräper honom." Och HERREN satte ett tecken till skydd för Kain, så att ingen som mötte honom skulle slå honom ihjäl.
\par 16 Så gick Kain bort ifrån HERRENS ansikte och bosatte sig i landet Nod, öster om Eden.
\par 17 Och Kain kände sin hustru, och hon blev havande och födde Hanok. Och han byggde en stad och kallade den staden Hanok, efter sin sons namn.
\par 18 Och åt Hanok föddes Irad, och Irad födde Mehujael, och Mehujael födde Metusael, och Metusael födde Lemek.
\par 19 Men Lemek tog sig två hustrur; den ena hette Ada, den andra Silla.
\par 20 Och Ada födde Jabal; han blev stamfader för dem som bo i tält och idka boskapsskötsel.
\par 21 Och hans broder hette Jubal; han blev stamfader för alla dem som hantera harpa och pipa.
\par 22 Men Silla födde ock en son, Tubal-Kain; han var smed och gjorde alla slags redskap av koppar och järn. Och Tubal-Kains syster var Naama.
\par 23 Och Lemek sade till sina hustrur: "Ada och Silla, hören mina ord; I Lemeks hustrur, lyssnen till mitt tal: Se, en man dräper jag för vart sår jag får, och en yngling för var blånad jag får.
\par 24 Ja, sjufalt hämnad bliver Kain, men Lemek sju- och sjuttiofalt."
\par 25 Och Adam kände åter sin hustru, och hon födde en son och gav honom namnet Set, i det hon sade: "Gud har beskärt mig en annan livsfrukt, till ersättning för Abel, eftersom Kain dräpte honom."
\par 26 Men åt Set föddes ock en son, och han gav honom namnet Enos. Vid denna tid begynte man åkalla HERRENS namn.

\chapter{5}

\par 1 Detta är stycket om Adams släkt. När Gud skapade människor, gjorde han dem lika Gud.
\par 2 Till man och kvinna skapade han dem; och han välsignade dem och gav dem namnet människa, när de blevo skapade.
\par 3 När Adam var ett hundra trettio år gammal, födde han en son som var honom lik, hans avbild, och gav honom namnet Set.
\par 4 Och sedan Adam hade fött Set, levde han åtta hundra år och födde söner och döttrar.
\par 5 Alltså blev Adams hela levnadsålder nio hundra trettio år; därefter dog han.
\par 6 När Set var ett hundra fem år gammal, födde han Enos.
\par 7 Och sedan Set hade fött Enos, levde han åtta hundra sju år och födde söner och döttrar.
\par 8 Alltså blev Sets hela ålder nio hundra tolv år; därefter dog han.
\par 9 När Enos var nittio år gammal, födde han Kenan.
\par 10 Och sedan Enos hade fött Kenan, levde han åtta hundra femton år och födde söner och döttrar.
\par 11 Alltså blev Enos' hela ålder nio hundra fem år; därefter dog han.
\par 12 När Kenan var sjuttio år gammal, födde han Mahalalel.
\par 13 Och sedan Kenan fött Mahalalel, levde han åtta hundra fyrtio år och födde söner och döttrar.
\par 14 Alltså blev Kenans hela ålder nio hundra tio år; därefter dog han.
\par 15 När Mahalalel var sextiofem år gammal, födde han Jered.
\par 16 Och sedan Mahalalel hade fött Jered, levde han åtta hundra trettio år och födde söner och döttrar.
\par 17 Alltså blev Mahalalels hela ålder åtta hundra nittiofem år; därefter dog han.
\par 18 När Jered var ett hundra sextiotvå år gammal, födde han Hanok.
\par 19 Och sedan Jered hade fött Hanok, levde han åtta hundra år och födde söner och döttrar.
\par 20 Alltså blev Jereds hela ålder nio hundra sextiotvå år; därefter dog han.
\par 21 När Hanok var sextiofem år gammal, födde han Metusela.
\par 22 Och Hanok vandrade i umgängelse med Gud i tre hundra år, sedan han hade fött Metusela, och han födde söner och döttrar.
\par 23 Alltså blev Hanoks hela ålder tre hundra sextiofem år.
\par 24 Sedan Hanok så hade vandrat i umgängelse med Gud, såg man honom icke mer, ty Gud tog honom bort.
\par 25 När Metusela var ett hundra åttiosju år gammal, födde han Lemek.
\par 26 Och sedan Metusela hade fött Lemek, levde han sju hundra åttiotvå år och födde söner och döttrar.
\par 27 Alltså blev Metuselas hela ålder nio hundra sextionio år; därefter dog han.
\par 28 När Lemek var ett hundra åttiotvå år gammal, födde han en son.
\par 29 Och han gav honom namnet Noa, i det han sade: "Denne skall trösta oss vid vårt arbete och våra händers möda, när vi bruka jorden, som HERREN har förbannat."
\par 30 Och sedan Lemek hade fött Noa, levde han fem hundra nittiofem år och födde söner och döttrar.
\par 31 Alltså blev Lemeks hela ålder sju hundra sjuttiosju år; därefter dog han.
\par 32 När Noa var fem hundra år gammal, födde han Sem, Ham och Jafet.

\chapter{6}

\par 1 Då nu människorna begynte föröka sig på jorden och döttrar föddes åt dem
\par 2 sågo Guds söner att människornas döttrar voro fagra, och de togo till hustrur dem som de funno mest behag i.
\par 3 Då sade HERREN: "Min ande skall icke bliva kvar i människorna för beständigt, eftersom de dock äro kött; så vare nu deras tid bestämd till ett hundra tjugu år."
\par 4 Vid den tiden, likasom ock efteråt, levde jättarna på jorden, sedan Guds söner begynte gå in till människornas döttrar och dessa födde barn åt dem; detta var forntidens väldiga män, som voro så namnkunniga.
\par 5 Men när HERREN såg att människornas ondska var stor på jorden, och att deras hjärtans alla uppsåt och tankar beständigt voro allenast onda,
\par 6 då ångrade HERREN att han hade gjort människorna på jorden, och han blev bedrövad i sitt hjärta.
\par 7 Och HERREN sade: "Människorna, som jag skapade, vill jag utplåna från jorden, ja, både människor och fyrfotadjur och kräldjur och himmelens fåglar; ty jag ångrar att jag har gjort dem."
\par 8 Men Noa hade funnit nåd för HERRENS ögon.
\par 9 Detta är berättelsen om Noas släkt. Noa var en rättfärdig man och ostrafflig bland sitt släkte; i umgängelse med Gud vandrade Noa.
\par 10 Och Noa födde tre söner: Sem, Ham och Jafet.
\par 11 Men jorden blev alltmer fördärvad för Guds åsyn, och jorden uppfylldes av våld.
\par 12 Och Gud såg att jorden var fördärvad, eftersom allt kött vandrade i fördärv på jorden.
\par 13 Då sade Gud till Noa: "Jag har beslutit att göra ände på allt kött, ty jorden är uppfylld av våld som de öva; se, jag vill fördärva dem tillika med jorden.
\par 14 Så gör dig nu en ark av goferträ, och inred arken med kamrar, och bestryk den med jordbeck innan och utan.
\par 15 Och så skall du göra arken: Den skall vara tre hundra alnar lång, femtio alnar bred och trettio alnar hög;
\par 16 en öppning för ljuset, en aln hög alltigenom, skall du göra ovantill på arken; och en dörr till arken skall du sätta på dess sida; och du skall inreda den så, att den får en undervåning, en mellanvåning och en övervåning.
\par 17 Ty se, jag skall låta floden komma med vatten över jorden, till att fördärva allt kött som har i sig någon livsande, under himmelen; allt som finnes på jorden skall förgås.
\par 18 Men med dig vill jag upprätta ett förbund: du skall gå in i arken med dina söner och din hustru och dina söners hustrur.
\par 19 Och av allt levande, vad kött det vara må, skall du föra in i arken ett par av vart slag, för att behålla dem vid liv med dig; hankön och honkön skola de vara.
\par 20 Av fåglarna, efter deras arter, av fyrfotadjuren, efter deras arter, av alla kräldjur på marken, efter deras arter, skall ett par av vart slag gå in till dig, för att du må behålla dem vid liv.
\par 21 Och du skall taga till dig alla slags livsmedel, sådant som kan ätas, och samla det till dig, för att det må vara dig och dem till föda.
\par 22 Och Noa gjorde så; han gjorde i alla stycken såsom Gud hade bjudit honom.

\chapter{7}

\par 1 Och HERREN sade till Noa: "Gå in i arken med hela ditt hus, ty dig har jag funnit rättfärdig inför mig bland detta släkte.
\par 2 Av alla rena fyrfotadjur skall du taga till dig sju par, hanne och hona, men av sådana fyrfotadjur som icke äro rena ett par, hanne och hona,
\par 3 sammalunda av himmelens fåglar sju par, hankön och honkön, för att behålla deras släkten vid liv på hela jorden.
\par 4 Ty sju dagar härefter skall jag låta det regna på jorden, i fyrtio dagar och fyrtio nätter, och jag skall utplåna från jorden alla varelser som jag har gjort."
\par 5 Och Noa gjorde i alla stycken såsom HERREN hade bjudit honom.
\par 6 Noa var sex hundra år gammal, när floden kom med sitt vatten över jorden.
\par 7 Och Noa gick in i arken med sina söner och sin hustru och sina söners hustrur, undan flodens vatten.
\par 8 Och av fyrfotadjur, både rena och orena, och av fåglar och av allt som krälar på marken
\par 9 gingo två och två, hankön och honkön, in till Noa i arken, såsom Gud hade bjudit Noa.
\par 10 Och efter de sju dagarna kom flodens vatten över jorden.
\par 11 I det år då Noa var sex hundra år gammal, i andra månaden, på sjuttonde dagen i månaden, den dagen bröto alla det stora djupets källor fram, och himmelens fönster öppnade sig,
\par 12 och ett regn kom över jorden i fyrtio dagar och fyrtio nätter.
\par 13 På denna samma dag gick Noa in i arken, så ock Sem, Ham och Jafet, Noas söner, vidare Noas hustru och hans söners tre hustrur med dem,
\par 14 därtill alla vilda djur, efter sina arter, och alla boskapsdjur, efter sina arter, och alla kräldjur som röra sig på jorden, efter sina arter, och alla flygande djur, efter sina arter, allt vad fåglar heter, av alla slag.
\par 15 De gingo in till Noa i arken, två och två av allt kött som hade i sig någon livsande.
\par 16 Och de som gingo ditin voro hankön och honkön av allt slags kött, såsom Gud hade bjudit honom. Och HERREN stängde igen om honom.
\par 17 Och floden kom över jorden i fyrtio dagar, och vattnet förökade sig och lyfte arken, så att den flöt högt uppe över jorden.
\par 18 Och vattnet steg och förökade sig mycket på jorden, och arken drev på vattnet.
\par 19 Och vattnet steg mer och mer över jorden, och alla höga berg allestädes under himmelen övertäcktes.
\par 20 Femton alnar högt steg vattnet över bergen, så att de övertäcktes.
\par 21 Då förgicks allt kött som rörde sig på jorden, fåglar och boskapsdjur och vilda djur och alla smådjur som rörde sig på jorden, så ock alla människor.
\par 22 Allt som fanns på det torra omkom, allt som där hade en fläkt av livsande i sin näsa.
\par 23 Så utplånade han alla varelser på jorden, både människor och fyrfotadjur och kräldjur och himmelens fåglar; de utplånades från jorden, och allenast Noa räddades, jämte det som var med honom i arken.
\par 24 Och vattnet fortfor att stiga över jorden i hundra femtio dagar.

\chapter{8}

\par 1 Då tänkte Gud på Noa och på alla de vilda djur och alla de boskapsdjur som voro med honom i arken. Och Gud lät en vind gå fram över jorden, så att vattnet sjönk undan;
\par 2 och djupets källor och himmelens fönster tillslötos, och regnet från himmelen upphörde.
\par 3 Och vattnet vek bort ifrån jorden mer och mer; efter hundra femtio dagar begynte vattnet avtaga.
\par 4 Och i sjunde månaden, på sjuttonde dagen i månaden, stannade arken på Ararats berg.
\par 5 Och vattnet avtog mer och mer intill tionde månaden. I tionde månaden, på första dagen i månaden, blevo bergstopparna synliga.
\par 6 Och efter fyrtio dagar öppnade Noa fönstret som han hade gjort på arken,
\par 7 och lät en korp flyga ut; denne flög fram och åter, till dess vattnet hade torkat bort ifrån jorden.
\par 8 Sedan lät han en duva flyga ut, för att få se om vattnet hade sjunkit undan från marken.
\par 9 Men duvan fann ingen plats där hon kunde vila sin fot, utan kom tillbaka till honom i arken, ty vatten betäckte hela jorden. Då räckte han ut sin hand och tog henne in till sig i arken.
\par 10 Sedan väntade han ännu ytterligare sju dagar och lät så duvan än en gång flyga ut ur arken.
\par 11 Och duvan kom till honom mot aftonen, och se, då hade hon ett friskt olivlöv i sin näbb. Då förstod Noa att vattnet hade sjunkit undan från jorden.
\par 12 Men han väntade ännu ytterligare sju dagar och lät så duvan åter flyga ut; då kom hon icke mer tillbaka till honom.
\par 13 I det sexhundraförsta året, i första månaden, på första dagen i månaden, hade vattnet sinat bort ifrån jorden. Då tog Noa av taket på arken och såg nu att marken var fri ifrån vatten.
\par 14 Och i andra månaden, på tjugusjunde dagen i månaden, var jorden alldeles torr.
\par 15 Då talade Gud till Noa och sade:
\par 16 "Gå ut ur arken med din hustru och dina söner och dina söners hustrur.
\par 17 Alla djur som du har hos dig, vad slags kött det vara må, både fåglar och fyrfotadjur och alla kräldjur som röra sig på jorden, skall du låta gå ut med dig, för att de må växa till på jorden och vara fruktsamma och föröka sig på jorden."
\par 18 Så gick då Noa ut med sina söner och sin hustru och sina söners hustrur.
\par 19 Och alla fyrfotadjur, alla kräldjur och alla fåglar, alla slags djur som röra sig på jorden, gingo ut ur arken, efter sina släkten.
\par 20 Och Noa byggde ett altare åt HERREN och tog av alla rena fyrfotadjur och av alla rena fåglar och offrade brännoffer på altaret.
\par 21 När HERREN kände den välbehagliga lukten, sade han vid sig själv: "Jag skall härefter icke mer förbanna marken för människans skull, eftersom ju människans hjärtas uppsåt är ont allt ifrån ungdomen. Och jag skall härefter icke mer dräpa allt levande, såsom jag nu har gjort.
\par 22 Så länge jorden består, skola härefter sådd och skörd, köld och värme, sommar och vinter, dag och natt aldrig upphöra."

\chapter{9}

\par 1 Och Gud välsignade Noa och hans söner och sade till dem: "Varen fruktsamma och föröken eder, och uppfyllen jorden.
\par 2 Och må fruktan och förskräckelse för eder komma över alla djur på jorden och alla fåglar under himmelen; jämte allt som krälar på marken och alla fiskar i havet vare de givna i eder hand.
\par 3 Allt som rör sig och har liv skolen I hava till föda; såsom jag har givit eder gröna örter, så giver jag eder allt detta.
\par 4 Kött som har i sig sin själ, det är sitt blod, skolen I dock icke äta.
\par 5 Men edert eget blod, vari eder själ är, skall jag utkräva. Jag skall utkräva det av vilket djur det vara må. Jag skall ock av den ena människan utkräva den andres själ;
\par 6 den som utgjuter människoblod, hans blod skall av människor bliva utgjutet, ty Gud har gjort människan till sin avbild.
\par 7 Och varen I fruktsamma och föröken eder; växen till på jorden och föröken eder på den."
\par 8 Ytterligare sade Gud till Noa och till hans söner med honom:
\par 9 "Se, jag vill upprätta ett förbund med eder, och med edra efterkommande efter eder,
\par 10 och med alla levande varelser som I haven hos eder: fåglar, boskapsdjur och alla vilda djur hos eder, alla jordens djur som hava gått ut ur arken.
\par 11 Jag vill upprätta ett förbund med eder: härefter skall icke mer ske att allt kött utrotas genom flodens vatten; ingen flod skall mer komma och fördärva jorden."
\par 12 Och Gud sade: "Detta skall vara tecknet till det förbund som jag gör mellan mig och eder, jämte alla levande varelser hos eder, för eviga tider:
\par 13 min båge sätter jag i skyn; den skall vara tecknet till förbundet mellan mig och jorden.
\par 14 Och när jag härefter låter skyar stiga upp över jorden och bågen då synes i skyn,
\par 15 skall jag tänka på det förbund som har blivit slutet mellan mig och eder, jämte alla levande varelser, vad slags kött det vara må; och vattnet skall då icke mer bliva en flod som fördärvar allt kött.
\par 16 När alltså bågen synes i skyn och jag ser på den, skall jag tänka på det eviga förbund som har blivit slutet mellan Gud och alla levande varelser, vad slags kött det vara må på jorden."
\par 17 Så sade nu Gud till Noa: "Detta skall vara tecknet till det förbund som jag har upprättat mellan mig och allt kött på jorden."
\par 18 Noas söner, som gingo ut ur arken, voro Sem, Ham och Jafet; men Ham var Kanaans fader.
\par 19 Dessa tre voro Noas söner och från dessa hava alla jordens folk utgrenat sig.
\par 20 Och Noa var en åkerman och var den förste som planterade en vingård.
\par 21 Men när han drack av vinet, blev han drucken och låg blottad i sitt tält.
\par 22 Och Ham, Kanaans fader, såg då sin faders blygd och berättade det för sina båda bröder, som voro utanför.
\par 23 Men Sem och Jafet togo en mantel och lade den på sina skuldror, båda tillsammans, och gingo så baklänges in och täckte över sin faders blygd; de höllo därvid sina ansikten bortvända, så att de icke sågo sin faders blygd.
\par 24 När sedan Noa vaknade upp från ruset och fick veta vad hans yngste son hade gjort honom, sade han:
\par 25 "Förbannad vare Kanaan, en trälars träl vare han åt sina bröder!"
\par 26 Ytterligare sade han: "Välsignad vare HERREN, Sems Gud, och Kanaan vare deras träl!
\par 27 Gud utvidge Jafet, han tage sin boning i Sems hyddor, och Kanaan vare deras träl."
\par 28 Och Noa levde efter floden tre hundra femtio år;
\par 29 alltså blev Noas hela ålder nio hundra femtio år; därefter dog han.

\chapter{10}

\par 1 Detta är berättelsen om Noas söners släkt. De voro Sem, Ham och Jafet; och åt dem föddes söner efter floden.
\par 2 Jafets söner voro Gomer, Magog, Madai, Javan, Tubal, Mesek och Tiras.
\par 3 Gomers söner voro Askenas, Rifat och Togarma.
\par 4 Javans söner voro Elisa och Tarsis, kittéerna och dodanéerna.
\par 5 Från dessa hava inbyggarna i hedningarnas Havsländer utbrett sig i sina länder, var efter sitt tungomål, efter sina släkter, i sina folk.
\par 6 Hams söner voro Kus, Misraim, Put och Kanaan.
\par 7 Kus' söner voro Seba, Havila, Sabta, Raema och Sabteka. Raemas söner voro Saba och Dedan.
\par 8 Men Kus födde Nimrod; han var den förste som upprättade ett välde på jorden.
\par 9 Han var ock en väldig jägare inför HERREN; därför plägar man säga: "En väldig jägare inför HERREN såsom Nimrod."
\par 10 Och hans rike hade sin begynnelse i Babel, Erek, Ackad och Kalne, i Sinears land.
\par 11 Från det landet drog han sedan ut till Assyrien och byggde Nineve, Rehobot-Ir och Kela,
\par 12 och därtill Resen mellan Nineve och Kela; detta är "den stora staden".
\par 13 Och Misraim födde ludéerna, anaméerna, lehabéerna, naftuhéerna,
\par 14 patroséerna, kasluhéerna, från vilka filistéerna hava utgått, och kaftoréerna.
\par 15 Och Kanaan födde Sidon, som var hans förstfödde, och Het,
\par 16 så ock jebuséerna, amoréerna, girgaséerna,
\par 17 hivéerna, arkéerna, sinéerna,
\par 18 arvadéerna, semaréerna och hamatéerna. Sedan utgrenade sig kananéernas släkter allt vidare,
\par 19 så att kananéernas område sträckte sig från Sidon fram emot Gerar ända till Gasa, och fram emot Sodom, Gomorra, Adma och Seboim ända till Lesa.
\par 20 Dessa voro Hams söner, efter deras släkter och tungomål, i deras länder och folk.
\par 21 Söner föddes ock åt Sem, Jafets äldre broder, som blev stamfader för alla Ebers söner.
\par 22 Sems söner voro Elam, Assur, Arpaksad, Lud och Aram.
\par 23 Arams söner voro Us, Hul, Geter och Mas.
\par 24 Arpaksad födde Sela, och Sela födde Eber.
\par 25 Men åt Eber föddes två söner; den ene hette Peleg, ty i hans tid blev jorden fördelad; och hans broder hette Joktan.
\par 26 Och Joktan födde Almodad, Selef, Hasarmavet, Jera,
\par 27 Hadoram, Usal, Dikla,
\par 28 Obal, Abimael, Saba,
\par 29 Ofir, Havila och Jobab; alla dessa voro Joktans söner.
\par 30 Och de hade sina boningsorter från Mesa fram emot Sefar, emot Östra berget.
\par 31 Dessa voro Sems söner, efter deras släkter och tungomål, i deras länder, efter deras folk.
\par 32 Dessa voro Noas söners släkter, efter deras ättföljd, i deras folk. Och från dem hava folken efter floden utbrett sig på jorden.

\chapter{11}

\par 1 Och hela jorden hade enahanda tungomål och talade på enahanda sätt.
\par 2 Men när de bröto upp och drogo österut, funno de en lågslätt i Sinears land och bosatte sig där.
\par 3 Och de sade till varandra: "Kom, låt oss slå tegel och bränna det." Och teglet begagnade de såsom sten, och såsom murbruk begagnade de jordbeck.
\par 4 Och de sade: "Kom, låt oss bygga en stad åt oss och ett torn vars spets räcker upp i himmelen, och så göra oss ett namn; vi kunde eljest bliva kringspridda över hela jorden."
\par 5 Då steg HERREN ned för att se staden och tornet som människobarnen byggde.
\par 6 Och HERREN sade: "Se, de äro ett enda folk och hava alla enahanda tungomål, och detta är deras första tilltag; härefter skall intet bliva dem omöjligt, vad de än besluta att göra.
\par 7 Välan, låt oss stiga dit ned och förbistra deras tungomål, så att den ene icke förstår den andres tungomål."
\par 8 Och så spridde HERREN dem därifrån ut över hela jorden, så att de måste upphöra att bygga på staden.
\par 9 Därav fick den namnet Babel, eftersom HERREN där förbistrade hela jordens tungomål; därifrån spridde ock HERREN ut dem över hela jorden.
\par 10 Detta är berättelsen om Sems släkt. När Sem var hundra år gammal, födde han Arpaksad, två år efter floden.
\par 11 Och sedan Sem hade fött Arpaksad, levde han fem hundra år och födde söner och döttrar.
\par 12 När Arpaksad var trettiofem år gammal, födde han Sela.
\par 13 Och sedan Arpaksad hade fött Sela, levde han fyra hundra tre år och födde söner och döttrar.
\par 14 När Sela var trettio år gammal, födde han Eber.
\par 15 Och sedan Sela hade fött Eber, levde han fyra hundra tre år och födde söner och döttrar.
\par 16 När Eber var trettiofyra år gammal, födde han Peleg.
\par 17 Och sedan Eber hade fött Peleg, levde han fyra hundra trettio år och födde söner och döttrar.
\par 18 När Peleg var trettio år gammal, födde han Regu.
\par 19 Och sedan Peleg hade fött Regu, levde han två hundra nio år och födde söner och döttrar.
\par 20 När Regu var trettiotvå år gammal, födde han Serug.
\par 21 Och sedan Regu hade fött Serug, levde han två hundra sju år och födde söner och döttrar.
\par 22 När Serug var trettio år gammal, födde han Nahor.
\par 23 Och sedan Serug hade fött Nahor, levde han två hundra år och födde söner och döttrar.
\par 24 När Nahor var tjugunio år gammal, födde han Tera.
\par 25 Och sedan Nahor hade fött Tera, levde han ett hundra nitton år och födde söner och döttrar.
\par 26 När Tera var sjuttio år gammal, födde han Abram, Nahor och Haran.
\par 27 Och detta är berättelsen om Teras släkt. Tera födde Abram, Nahor och Haran. Och Haran födde Lot.
\par 28 Och Haran dog hos sin fader Tera i sitt fädernesland, i det kaldeiska Ur.
\par 29 Och Abram och Nahor togo sig hustrur; Abrams hustru hette Sarai, och Nahors hustru hette Milka, dotter till Haran, som var fader till Milka och Jiska.
\par 30 Men Sarai var ofruktsam och hade inga barn.
\par 31 Och Tera tog med sig sin son Abram och sin sonson Lot, Harans son, och sin sonhustru Sarai, som var hans son Abrams hustru; och de drogo tillsammans ut från det kaldeiska Ur på väg till Kanaans land; men när de kommo till Haran, bosatte de sig där.
\par 32 Och Teras ålder blev två hundra fem år; därefter dog Tera i Haran.

\chapter{12}

\par 1 Och HERREN sade till Abram: "Gå ut ur ditt land och från din släkt och från din faders hus, bort till det land som jag skall visa dig.
\par 2 Så skall jag göra dig till ett stort folk; jag skall välsigna dig och göra ditt namn stort, och du skall bliva en välsignelse.
\par 3 Och jag skall välsigna dem som välsigna dig, och den som förbannar dig skall jag förbanna, och i dig skola alla släkter på jorden varda välsignade."
\par 4 Och Abram gick åstad, såsom HERREN hade tillsagt honom, och Lot gick med honom. Och Abram var sjuttiofem år gammal, när han drog ut från Haran.
\par 5 Och Abram tog sin hustru Sarai och sin brorson Lot och alla ägodelar som de hade förvärvat och tjänarna som de hade skaffat sig i Haran; och de drogo åstad på väg mot Kanaans land
\par 6 och kommo så till Kanaans land. Och Abram drog fram i landet ända till den heliga platsen vid Sikem, till Mores terebint. Och på den tiden bodde kananéerna där i landet.
\par 7 Men HERREN uppenbarade sig för Abram och sade: "Åt din säd skall jag giva detta land." Då byggde han där ett altare åt HERREN, som hade uppenbarat sig för honom.
\par 8 Sedan flyttade han därifrån till bergsbygden öster om Betel och slog där upp sitt tält, så att han hade Betel i väster och Ai i öster; och han byggde där ett altare åt HERREN och åkallade HERRENS namn.
\par 9 Sedan bröt Abram upp därifrån och drog sig allt längre mot Sydlandet.
\par 10 Men hungersnöd uppstod i landet, och Abram drog ned till Egypten för att bo där någon tid, eftersom hungersnöden var så svår i landet.
\par 11 Men när han nalkades Egypten sade han till sin hustru Sarai: "Jag vet ju att du är en skön kvinna.
\par 12 Om nu egyptierna tänka, när de få se dig: 'Hon är hans hustru', så skola de dräpa mig, under det att de låta dig leva.
\par 13 Säg därför att du är min syster, så att det går mig väl för din skull, och så att jag för din skull får leva."
\par 14 Då nu Abram kom till Egypten, sågo egyptierna att hon var en mycket skön kvinna.
\par 15 Och när Faraos hövdingar fingo se henne, prisade de henne för Farao, och så blev kvinnan tagen in i Faraos hus.
\par 16 Och Abram blev av honom väl behandlad för hennes skull, så att han fick får, fäkreatur och åsnor, tjänare och tjänarinnor, åsninnor och kameler.
\par 17 Men HERREN hemsökte Farao och hans hus med stora plågor för Sarais, Abrams hustrus, skull.
\par 18 Då kallade Farao Abram till sig och sade: "Vad har du gjort mot mig! Varför lät du mig icke veta att hon var din hustru?
\par 19 Varför sade du: 'Hon är min syster' och vållade så, att jag tog henne till hustru åt mig? Se, här har du nu din hustru, tag henne och gå."
\par 20 Och Farao gav sina män befallning om honom, att de skulle ledsaga honom till vägs med hans hustru och allt vad han ägde.

\chapter{13}

\par 1 Så drog då Abram upp från Egypten med sin hustru och allt vad han ägde, och Lot jämte honom, till Sydlandet.
\par 2 Och Abram var mycket rik på boskap och på silver och guld.
\par 3 Och han färdades ifrån lägerplats till lägerplats och kom så från Sydlandet ända till Betel, till det ställe där hans tält förut hade stått, mellan Betel och Ai,
\par 4 dit där han förra gången hade rest ett altare. Och där åkallade Abram HERRENS namn.
\par 5 Men Lot, som drog med Abram, hade också får och fäkreatur och tält.
\par 6 Och landet räckte icke till för dem, så att de kunde bo tillsammans; ty deras ägodelar voro för stora för att de skulle kunna bo tillsammans;
\par 7 och tvister uppstodo mellan Abrams och Lots boskapsherdar. Tillika bodde på den tiden kananéerna och perisséerna där i landet.
\par 8 Då sade Abram till Lot: "Icke skall någon tvist vara mellan mig och dig, och mellan mina herdar och dina herdar; vi äro ju fränder.
\par 9 Ligger icke hela landet öppet för dig? Skilj dig ifrån mig; vill du åt vänster, så går jag åt höger, och vill du åt höger, så går jag åt vänster."
\par 10 Då lyfte Lot upp sina ögon och såg att hela Jordanslätten överallt var vattenrik. Innan HERREN fördärvade Sodom och Gomorra, var den nämligen såsom HERRENS lustgård, såsom Egyptens land, ända fram emot Soar.
\par 11 Så utvalde då Lot åt sig hela Jordanslätten. Och Lot bröt upp och drog österut, och de skildes så från varandra.
\par 12 Abram förblev boende i Kanaans land, och Lot bodde i städerna på Slätten och drog med sina tält ända inemot Sodom.
\par 13 Men folket i Sodom var mycket ont och syndigt inför HERREN.
\par 14 Och HERREN sade till Abram, sedan Lot hade skilt sig från honom: "Lyft upp dina ögon och se, från den plats där du står, mot norr och söder och öster och väster."
\par 15 Ty hela det land som du nu ser skall jag giva åt dig och din säd för evärdlig tid.
\par 16 Och jag skall låta din säd bliva såsom stoftet på jorden; kan någon räkna stoftet på jorden, så skall ock din säd kunna räknas.
\par 17 Stå upp och drag igenom landet efter dess längd och dess bredd, ty åt dig skall jag giva det."
\par 18 Och Abram drog åstad med sina tält och kom och bosatte sig vid Mamres terebintlund invid Hebron; och han byggde där ett altare åt HERREN.

\chapter{14}

\par 1 På den tid då Amrafel var konung i Sinear, Arjok konung i Ellasar, Kedorlaomer konung i Elam och Tideal konung över Goim, hände sig
\par 2 att dessa begynte krig mot Bera, konungen i Sodom, Birsa, konungen i Gomorra, Sinab, konungen i Adma, Semeber, konungen i Seboim, och mot konungen i Bela, det är Soar.
\par 3 De förenade sig alla och tågade till Siddimsdalen, där Salthavet nu är.
\par 4 I tolv år hade de varit under Kedorlaomer, men i det trettonde året hade de avfallit.
\par 5 Så kom nu i det fjortonde året Kedorlaomer med de konungar som voro på hans sida; och de slogo rafaéerna i Asterot-Karnaim, suséerna i Ham, eméerna i Save-Kirjataim
\par 6 och horéerna på deras berg Seir och drevo dem ända till El-Paran vid öknen.
\par 7 Sedan vände de om och kommo till En-Mispat, det är Kades, och härjade amalekiternas hela land; de slogo ock amoréerna som bodde i Hasason-Tamar.
\par 8 Då drogo konungen i Sodom, konungen i Gomorra, konungen i Adma, konungen i Seboim och konungen i Bela, det är Soar, ut och ställde upp sig i Siddimsdalen till strid mot dem -
\par 9 mot Kedorlaomer, konungen i Elam, Tideal, konungen över Goim, Amrafel, konungen i Sinear, och Arjok, konungen i Ellasar, fyra konungar mot de fem.
\par 10 Men Siddimsdalen var full av jordbecksgropar. Och konungarna i Sodom och Gomorra måste fly och föllo då i dessa, och de som kommo undan flydde till bergsbygden.
\par 11 Så togo de allt gods som fanns i Sodom och Gomorra, och alla livsmedel där, och tågade bort;
\par 12 de togo ock med sig Lot, Abrams brorson, och hans ägodelar, när de tågade bort; ty denne bodde i Sodom.
\par 13 Men en av de räddade kom och berättade detta för Abram, hebréen; denne bodde vid den terebintlund som tillhörde amoréen Mamre, Eskols och Aners broder, och dessa voro i förbund med Abram.
\par 14 Då nu Abram hörde att hans frände var fången, lät han sina mest beprövade tjänare, sådana som voro födda i hans hus, tre hundra aderton män, rycka ut, och förföljde fienderna ända till Dan.
\par 15 Och han delade sitt folk och överföll dem så om natten med sina tjänare och slog dem, och förföljde dem sedan ända till Hoba, norr om Damaskus,
\par 16 och tog tillbaka allt godset; sin frände Lot och hans ägodelar tog han ock tillbaka, ävensom kvinnorna och det övriga folket.
\par 17 Då han nu var på återvägen, sedan han hade slagit Kedorlaomer och de konungar som voro på hans sida, gick konungen i Sodom honom till mötes i Savedalen, det är Konungsdalen.
\par 18 Och Melki-Sedek, konungen i Salem, lät bära ut bröd och vin; denne var präst åt Gud den Högste.
\par 19 Och han välsignade honom och sade: "Välsignad vare Abram av Gud den Högste, himmelens och jordens skapare!
\par 20 Och välsignad vare Gud den Högste, som har givit dina ovänner i din hand!" Och Abram gav honom tionde av allt.
\par 21 Och konungen i Sodom sade till Abram: "Giv mig folket; godset må du behålla för dig själv."
\par 22 Men Abram svarade konungen i Sodom: "Jag lyfter min hand upp till HERREN, till Gud den Högste, himmelens och jordens skapare, och betygar
\par 23 att jag icke vill taga ens en tråd eller en skorem, än mindre något annat som tillhör dig. Du skall icke kunna säga: 'Jag har riktat Abram.'
\par 24 Jag vill intet hava; det är nog med vad mina män hava förtärt och den del som tillkommer mina följeslagare. Aner, Eskol och Mamre, de må få sin del."

\chapter{15}

\par 1 En tid härefter kom HERRENS ord i en syn till Abram; han sade: "Frukta icke, Abram, jag är din sköld, din lön skall bliva mycket stor."
\par 2 Men Abram sade: "Herre, HERRE, vad vill du då giva mig? Jag går ju barnlös bort, och arvinge till mitt hus bliver en man från Damaskus, Elieser."
\par 3 Och Abram sade ytterligare: "Mig har du icke givit någon livsfrukt; en av mitt husfolk skall bliva min arvinge."
\par 4 Men se, HERRENS ord kom till honom; han sade: "Nej, denne skall icke bliva din arvinge, utan en som utgår från ditt eget liv skall bliva din arvinge."
\par 5 Och han förde honom ut och sade: "Skåda upp till himmelen, och räkna stjärnorna, om du kan räkna dem." Och han sade till honom: "Så skall din säd bliva."
\par 6 Och han trodde på HERREN; och han räknade honom det till rättfärdighet.
\par 7 Och han sade till honom: "Jag är HERREN, som har fört dig ut från det kaldeiska Ur för att giva dig detta land till besittning."
\par 8 Han svarade: "Herre, HERRE, varav skall jag veta att jag skall besitta det?"
\par 9 Då sade han till honom: "Tag åt mig en treårig kviga, en treårig get och en treårig vädur, därtill en turturduva och en ung duva."
\par 10 Och han tog åt honom alla dessa djur och styckade dem mitt itu och lade styckena mitt emot varandra; dock styckade han icke fåglarna.
\par 11 Och rovfåglarna slogo ned på de döda kropparna, men Abram drev bort dem.
\par 12 När nu solen var nära att gå ned och en tung sömn hade fallit på Abram, se, då kom en förskräckelse över honom och ett stort mörker.
\par 13 Och han sade till Abram: "Det skall du veta, att din säd skall komma att leva såsom främlingar i ett land som icke tillhör dem, och de skola där vara trälar, och man skall förtrycka dem; så skall ske i fyra hundra år."
\par 14 Men det folk vars trälar de bliva skall jag ock döma. Sedan skola de draga ut med stora ägodelar.
\par 15 Men du själv skall gå till dina fäder i frid och bliva begraven i en god ålder.
\par 16 Och i det fjärde släktet skall din säd komma hit tillbaka. Ty ännu hava icke amoréerna fyllt sin missgärnings mått."
\par 17 Då nu solen hade gått ned och det hade blivit alldeles mörkt, syntes en rykande ugn med flammande låga, som for fram mellan styckena.
\par 18 På den dagen slöt HERREN ett förbund med Abram och sade: "Åt din säd skall jag giva detta land, från Egyptens flod ända till den stora floden, till floden Frat:
\par 19 kainéernas, kenaséernas, kadmonéernas,
\par 20 hetiternas, perisséernas, rafaéernas,
\par 21 amoréernas, kananéernas, girgaséernas och jebuséernas land."

\chapter{16}

\par 1 Och Sarai, Abrams hustru, hade icke fött barn åt honom. Men hon hade en egyptisk tjänstekvinna, som hette Hagar;
\par 2 och Sarai sade till Abram: "Se, HERREN har gjort mig ofruktsam, så att jag icke föder barn; gå in till min tjänstekvinna, kanhända skall jag få avkomma genom henne." Abram lyssnade till Sarais ord;
\par 3 och Sarai, Abrams hustru, tog sin egyptiska tjänstekvinna Hagar och gav henne till hustru åt sin man Abram, sedan denne hade bott tio år i Kanaans land.
\par 4 Och han gick in till Hagar, och hon blev havande. När hon nu såg att hon var havande, ringaktade hon sin fru.
\par 5 Då sade Sarai till Abram: "Den orätt mig sker komme över dig. Jag själv lade min tjänstekvinna i din famn, men då hon nu ser att hon är havande, ringaktar hon mig. HERREN döme mellan mig och dig."
\par 6 Abram sade till Sarai: "Din tjänstekvinna är ju i din hand, gör med henne vad du finner för gott." När då Sarai tuktade henne, flydde hon bort ifrån henne.
\par 7 Men HERRENS ängel kom emot henne vid en vattenkälla i öknen, den källa som ligger vid vägen till Sur.
\par 8 Och han sade: "Hagar, Sarais tjänstekvinna, varifrån kommer du, och vart går du?" Hon svarade: "Jag är stadd på flykt ifrån min fru Sarai."
\par 9 Då sade HERRENS ängel till henne: "Vänd tillbaka till din fru, och ödmjuka dig under henne."
\par 10 Och HERRENS ängel sade till henne: "Jag skall göra din säd mycket talrik, så att man icke skall kunna räkna den för dess myckenhets skull."
\par 11 Ytterligare sade HERRENS ängel till henne: "Se, du är havande och skall föda en son; honom skall du giva namnet Ismael, därför att HERREN har hört ditt lidande.
\par 12 Och han skall bliva lik en vildåsna; hans hand skall vara emot var man, och var mans hand emot honom; och han skall ligga i strid med alla sina bröder."
\par 13 Och hon gav HERREN, som hade talat med henne, ett namn, i det hon sade: "Du är Seendets Gud." Hon tänkte nämligen: "Har jag då verkligen här fått se en skymt av honom som ser mig?"
\par 14 Därav kallades brunnen Beer-Lahai-Roi; den ligger mellan Kades och Bered.
\par 15 Och Hagar födde åt Abram en son; och Abram gav den son som Hagar hade fött åt honom namnet Ismael.
\par 16 Och Abram var åttiosex år gammal, när Hagar födde Ismael åt Abram.

\chapter{17}

\par 1 När Abram var nittionio år gammal, uppenbarade sig HERREN för honom och sade till honom: "Jag är Gud den Allsmäktige. Vandra inför mig och var ostrafflig.
\par 2 Jag vill göra ett förbund mellan mig och dig, och jag skall föröka dig övermåttan."
\par 3 Då föll Abram ned på sitt ansikte, och Gud talade så med honom:
\par 4 "Se, det förbund som jag å min sida gör med dig är detta, att du skall bliva en fader till många folk.
\par 5 Därför skall du icke mer heta Abram, utan Abraham skall vara ditt namn, ty jag skall låta dig bliva en fader till många folk.
\par 6 Och jag skall göra dig övermåttan fruktsam och låta folkslag komma av dig, och konungar skola utgå från dig.
\par 7 Och jag skall upprätta ett förbund mellan mig och dig och din säd efter dig, från släkte till släkte, ett evigt förbund, så att jag skall vara din Gud och din säds efter dig;
\par 8 och jag skall giva dig och din säd efter dig det land där du nu bor såsom främling, hela Kanaans land, till evärdlig besittning, och jag skall vara deras Gud.
\par 9 Och Gud sade ytterligare till Abraham: "Du åter skall hålla mitt förbund, du och din säd efter dig, från släkte till släkte."
\par 10 Och detta är det förbund mellan mig och eder och din säd efter dig, som I skolen hålla: allt mankön bland eder skall omskäras;
\par 11 på eder förhud skolen I omskäras, och detta skall vara tecknet till förbundet mellan mig och eder.
\par 12 Släkte efter släkte skall vart gossebarn bland eder omskäras, när det är åtta dagar gammalt, jämväl den hemfödde tjänaren och den som är köpt för penningar från något främmande folk, och som icke är av din säd.
\par 13 Omskäras skall både din hemfödde tjänare och den som du har köpt för penningar; och så skall mitt förbund vara på edert kött betygat såsom ett evigt förbund.
\par 14 Men en oomskuren av mankön, en vilkens förhud icke har blivit omskuren, han skall utrotas ur sin släkt; han har brutit mitt förbund."
\par 15 Och Gud sade åter till Abraham: "Din hustru Sarai skall du icke mer kalla Sarai, utan Sara skall vara hennes namn.
\par 16 Och jag skall välsigna henne och skall också med henne giva dig en son; ja, jag skall välsigna henne, och folkslag skola komma av henne, konungar över folk skola härstamma från henne."
\par 17 Då föll Abraham ned på sitt ansikte och log, ty han sade vid sig själv: "Skulle barn födas åt en man som är hundra år gammal? Och skulle Sara föda barn, hon som är nittio år gammal?"
\par 18 Och Abraham sade till Gud: "Måtte allenast Ismael få leva inför dig!"
\par 19 Då sade Gud: "Nej, din hustru Sara skall föda dig en son, och du skall giva honom namnet Isak; och med honom skall jag upprätta mitt förbund, ett evigt förbund, som skall gälla hans säd efter honom.
\par 20 Men angående Ismael har jag ock hört din bön; se, jag skall välsigna honom och göra honom fruktsam och föröka honom övermåttan. Tolv hövdingar skall han få till söner, och jag skall göra honom till ett stort folk.
\par 21 Men mitt förbund skall jag upprätta med Isak, honom som Sara skall föda åt dig vid denna tid nästa år."
\par 22 Då Gud nu hade talat ut med Abraham, for han upp från honom.
\par 23 Och Abraham tog sin son Ismael och alla sina tjänare, de hemfödda och de som voro köpta för penningar, allt mankön bland Abrahams husfolk, och omskar på denna samma dag deras förhud, såsom Gud hade tillsagt honom.
\par 24 Och Abraham var nittionio år gammal, när hans förhud blev omskuren.
\par 25 Och hans son Ismael var tretton år gammal, när hans förhud blev omskuren.
\par 26 På denna samma dag omskuros Abraham och hans son Ismael;
\par 27 och alla män i hans hus, de hemfödda tjänarna och de som voro köpta för penningar ifrån främmande folk, blevo omskurna tillika med honom.

\chapter{18}

\par 1 Och HERREN uppenbarade sig för honom vid Mamres terebintlund, där han satt vid ingången till sitt tält, då det var som hetast på dagen.
\par 2 När han lyfte upp sina ögon, fick han se tre män stå framför sig. Och då han såg dem, skyndade han emot dem från tältets ingång och bugade sig ned till jorden
\par 3 och sade: "Herre, har jag funnit nåd för dina ögon, så gå icke förbi din tjänare.
\par 4 Låt mig hämta litet vatten, så att I kunnen två edra fötter; och vilen eder under trädet.
\par 5 Jag vill ock hämta ett stycke bröd, så att I kunnen vederkvicka eder, innan I gån vidare, eftersom I nu haven tagit vägen förbi eder tjänare." De sade: "Gör såsom du har sagt."
\par 6 Och Abraham skyndade in i tältet till Sara och sade: "Skynda dig och tag tre sea-mått fint mjöl, knåda det och baka kakor."
\par 7 Men själv hastade Abraham bort till boskapen och tog en god ungkalv och gav den åt sin tjänare, och denne skyndade sig att tillreda den.
\par 8 Och han tog gräddmjölk och söt mjölk och kalven, som han hade låtit tillreda, och satte fram för dem; och han stod själv hos dem under trädet, medan de åto.
\par 9 Och de sade till honom: "Var är din hustru Sara?" Han svarade: "Därinne i tältet."
\par 10 Då sade han: "Jag skall komma tillbaka till dig nästa år vid denna tid, och se, då skall din hustru Sara hava en son." Detta hörde Sara, där hon stod i ingången till tältet, som var bakom honom.
\par 11 Men Abraham och Sara voro gamla och komna till hög ålder, och Sara hade icke mer, såsom kvinnor pläga hava.
\par 12 Därför log Sara vid sig själv och tänkte: "Skulle jag väl nu på min ålderdom giva mig till lusta, nu då också min herre är gammal?"
\par 13 Men HERREN sade till Abraham: "Varför log Sara och tänkte: 'Skulle jag verkligen föda barn, så gammal som jag är?'
\par 14 Är då något så underbart, att HERREN icke skulle förmå det? På den bestämda tiden skall jag komma tillbaka till dig, vid denna tid nästa år, och då skall Sara hava en son."
\par 15 Då nekade Sara och sade: "Jag log icke"; ty hon blev förskräckt. Men han sade: "Jo, du log."
\par 16 Och männen stodo upp för att gå därifrån och vände sina blickar ned mot Sodom, och Abraham gick med för att ledsaga dem.
\par 17 Och HERREN sade: "Kan jag väl dölja för Abraham vad jag tänker göra?
\par 18 Av Abraham skall ju bliva ett stort och mäktigt folk, och i honom skola alla folk på jorden varda välsignade.
\par 19 Ty därtill har jag utvalt honom, för att han skall bjuda sina barn och sitt hus efter sig att hålla HERRENS väg och öva rättfärdighet och rätt, på det att HERREN må låta det komma över Abraham, som han har lovat honom."
\par 20 Och HERREN sade: "Ropet från Sodom och Gomorra är stort, och deras synd är mycket svår;
\par 21 därför vill jag gå ditned och se om de verkligen i allt hava gjort efter det rop som har kommit till mig; om så icke är, vill jag veta det."
\par 22 Och männen begåvo sig därifrån och gingo mot Sodom; men Abraham stod ännu kvar inför HERREN.
\par 23 Och Abraham trädde närmare och sade: "Vill du då förgöra den rättfärdige tillika med den ogudaktige?
\par 24 Kanhända finnas femtio rättfärdiga i staden; vill du då förgöra den och icke skona orten för de femtio rättfärdigas skull som finnas där?
\par 25 Bort det, att du skulle så göra och döda den rättfärdige tillika med den ogudaktige, så att det skulle gå den rättfärdige likasom den ogudaktige; bort det ifrån dig! Skulle han som är hela jordens domare icke göra vad rätt är?"
\par 26 HERREN sade: "Om jag i Sodom finner femtio rättfärdiga inom staden, så vill jag skona orten för deras skull."
\par 27 Men Abraham svarade och sade: "Se, jag har dristat mig att tala till Herren, fastän jag är stoft och aska."
\par 28 Kanhända skall det fattas fem i de femtio rättfärdiga; vill du då för de fems skull fördärva hela staden?" Han sade: "Om jag där finner fyrtiofem; så skall jag icke fördärva den."
\par 29 Men han fortfor att tala till honom och sade: "Kanhända skola fyrtio finnas där." Han svarade: "Jag skall då icke göra det, för de fyrtios skull."
\par 30 Då sade han: "Herre, vredgas icke över att jag ännu talar något. Kanhända skola trettio finnas där." Han svarade: "Om jag där finner trettio, så skall jag icke göra det."
\par 31 Men han sade: "Se, jag har dristat mig att tala till Herren. Kanhända skola tjugu finnas där." Han svarade: "Jag skall då icke fördärva den, för de tjugus skull."
\par 32 Då sade han: "Herre, vredgas icke över att jag talar allenast ännu en gång. Kanhända skola tio finnas där." Han svarade: "Jag skall då icke fördärva den, för de tios skull."
\par 33 Och HERREN gick bort, sedan han hade talat ut med Abraham; och Abraham vände tillbaka hem.

\chapter{19}

\par 1 Och de två änglarna kommo om aftonen till Sodom, och Lot satt då i Sodoms port. När Lot fick se dem, stod han upp och gick emot dem och föll ned till jorden på sitt ansikte
\par 2 och sade: "I herrar, tagen härbärge i eder tjänares hus och stannen där över natten, och tvån edra fötter; sedan kunnen I i morgon bittida fortsätta eder färd." De svarade: "Nej, vi vilja stanna på gatan över natten."
\par 3 Men han bad dem så enträget, att de togo härbärge hos honom och kommo in i hans hus. Och han tillredde en måltid åt dem och bakade osyrat bröd, och de åto.
\par 4 Men innan de hade lagt sig, omringades huset av männen i staden, Sodoms män, både unga och gamla, allt folket, så många de voro.
\par 5 Dessa kallade på Lot och sade till honom: "Var äro de män som hava kommit till dig i natt? För dem ut till oss, så att vi få känna dem."
\par 6 Då gick Lot ut till dem i porten och stängde dörren efter sig
\par 7 och sade: "Mina bröder, gören icke så illa.
\par 8 Se, jag har två döttrar, som ännu icke veta av någon man. Dem vill jag föra ut till eder, så kunnen I göra med dem vad I finnen för gott. Gören allenast icke något mot dessa män, eftersom de nu hava gått in under skuggan av mitt tak."
\par 9 Men de svarade: "Bort med dig!" Och de sade ytterligare: "Denne, en ensam man, har kommit hit och bor här såsom främling, och han vill dock ständigt upphäva sig som domare. Men nu skola vi göra dig mer ont än dem." Och de trängde med våld in på mannen Lot och stormade fram för att spränga dörren.
\par 10 Då räckte männen ut sina händer och togo Lot in till sig i huset och stängde dörren.
\par 11 Och de män som stodo utanför husets port slogo de med blindhet, både små och stora, så att de förgäves sökte finna porten.
\par 12 Och männen sade till Lot: "Har du någon mer här, någon måg, eller några söner eller döttrar, eller någon annan som tillhör dig i staden, så för dem bort ifrån detta ställe.
\par 13 Ty vi skola fördärva detta ställe; ropet från dem har blivit så stort inför HERREN, att HERREN har utsänt oss till att fördärva dem."
\par 14 Då gick Lot ut och talade till sina mågar, som skulle få hans döttrar, och sade: "Stån upp och gån bort ifrån detta ställe; ty HERREN skall fördärva staden." Men hans mågar menade att han skämtade.
\par 15 När nu morgonrodnaden gick upp, manade änglarna på Lot och sade: "Stå upp och tag med dig din hustru och dina båda döttrar, som du har hos dig, på det att du icke må förgås genom stadens missgärning."
\par 16 Och då han ännu dröjde, togo männen honom vid handen jämte hans hustru och hans båda döttrar, ty HERREN ville skona honom; och de förde honom ut, och när de voro utanför staden, släppte de honom.
\par 17 Och medan de förde dem ut, sade den ene: "Fly för ditt livs skull; se dig icke tillbaka, och dröj ingenstädes på Slätten. Fly undan till bergen, så att du icke förgås."
\par 18 Men Lot sade till dem: "Ack nej, Herre.
\par 19 Se, din tjänare har ju funnit nåd för dina ögon, och stor är den barmhärtighet som du gör med mig, då du vill rädda mitt liv; men jag förmår icke fly undan till bergen; jag rädes att olyckan hinner mig, så att jag omkommer.
\par 20 Se, staden därborta ligger helt nära, och det är lätt att fly dit, och den är liten; låt mig fly undan dit - den är ju så liten - på det att jag må bliva vid liv."
\par 21 Då svarade han honom: "Välan, jag skall ock häri göra dig till viljes; jag skall icke omstörta den stad som du talar om.
\par 22 Men skynda att fly undan dit; ty jag kan intet göra, förrän du har kommit dit." Därav fick staden namnet Soar.
\par 23 Då nu solen hade gått upp över jorden och Lot hade kommit till Soar,
\par 24 lät HERREN svavel och eld regna från himmelen, från HERREN, över Sodom och Gomorra;
\par 25 och han omstörtade dessa städer med hela Slätten och alla dem som bodde i städerna och det som växte på marken.
\par 26 Och Lots hustru, som följde efter honom, såg sig tillbaka; då blev hon en saltstod.
\par 27 Och när Abraham bittida följande morgon gick till den plats där han hade stått inför HERREN,
\par 28 och blickade ned över Sodom och Gomorra och över hela Slättlandet, då fick han se en rök stiga upp från landet, lik röken från en smältugn.
\par 29 Så skedde då, att när Gud fördärvade städerna på Slätten, tänkte han på Abraham och lät Lot komma undan omstörtningen, då han omstörtade städerna där Lot hade bott.
\par 30 Och Lot drog upp från Soar till bergsbygden och bodde där med sina båda döttrar, ty han fruktade för att bo kvar i Soar; och han bodde med sina båda döttrar i en grotta.
\par 31 Då sade den äldre till den yngre: "Vår fader är gammal, och ingen man finnes i landet, som kan gå in till oss efter all världens sedvänja.
\par 32 Kom, låt oss giva vår fader vin att dricka och lägga oss hos honom, för att vi må skaffa oss livsfrukt genom vår fader."
\par 33 Så gåvo de sin fader vin att dricka den natten, och den äldre gick in och lade sig hos sin fader, och han märkte icke när hon lade sig, ej heller när hon stod upp.
\par 34 Dagen därefter sade den äldre till den yngre: "Se, jag låg i natt hos min fader; låt oss också denna natt giva honom vin att dricka, och gå du in och lägg dig hos honom, för att vi må skaffa oss livsfrukt genom vår fader."
\par 35 Så gåvo de också den natten sin fader vin att dricka; och den yngre gick och lade sig hos honom, och han märkte icke när hon lade sig, ej heller när hon stod upp.
\par 36 Så blevo Lots båda döttrar havande genom sin fader.
\par 37 Och den äldre födde en son, och hon gav honom namnet Moab; från honom härstamma moabiterna ända till denna dag.
\par 38 Den yngre födde ock en son, och hon gav honom namnet Ben-Ammi; från honom härstamma Ammons barn ända till denna dag.

\chapter{20}

\par 1 Och Abraham bröt upp därifrån och drog till Sydlandet; där uppehöll han sig mellan Kades och Sur, och någon tid bodde han i Gerar.
\par 2 Och Abraham sade om sin hustru Sara att hon var hans syster. Då sände Abimelek, konungen i Gerar, och lät hämta Sara till sig.
\par 3 Men Gud kom till Abimelek i en dröm om natten och sade till honom: "Se, du måste dö för den kvinnas skull som du har tagit till dig, fast hon är en annan mans äkta hustru."
\par 4 Men Abimelek hade icke kommit vid henne. Och han svarade: "Herre, vill du då dräpa också rättfärdiga människor?
\par 5 Sade han icke själv till mig: 'Hon är min syster'? Och likaså sade hon: 'Han är min broder.' I mitt hjärtas oskuld och med rena händer har jag gjort detta."
\par 6 Då sade Gud till honom i drömmen: "Ja, jag vet att du har gjort detta i ditt hjärtas oskuld, och jag har själv hindrat dig från att synda mot mig; därför har jag icke tillstatt dig att komma vid henne.
\par 7 Men giv nu mannen hans hustru tillbaka; ty han är en profet. Och han må bedja för dig, så att du får leva. Men om du icke giver henne tillbaka, så vet att du skall döden dö, du själv och alla som tillhöra dig.
\par 8 Då stod Abimelek upp bittida om morgonen och kallade till sig alla sina tjänare och berättade allt detta för dem; och männen blevo mycket förskräckta.
\par 9 Sedan kallade Abimelek Abraham till sig och sade till honom: "Vad har du gjort mot oss! Vari har jag försyndat mig mot dig, eftersom du har velat komma mig och mitt rike att begå en så stor synd? På otillbörligt sätt har du handlat mot mig."
\par 10 Och Abimelek sade ytterligare till Abraham: "Vad var din mening, när du gjorde detta?"
\par 11 Abraham svarade: "Jag tänkte: 'På denna ort fruktar man nog icke Gud; de skola dräpa mig för min hustrus skull.'
\par 12 Hon är också verkligen min syster, min faders dotter, fastän icke min moders dotter; och så blev hon min hustru.
\par 13 Men när Gud sände mig ut på vandring bort ifrån min faders hus, sade jag till henne: 'Bevisa mig din kärlek därmed att du säger om mig, varthelst vi komma, att jag är din broder.'"
\par 14 Då tog Abimelek får och fäkreatur, tjänare och tjänarinnor och gav dem åt Abraham. Han gav honom ock hans hustru Sara tillbaka.
\par 15 Och Abimelek sade: "Se, mitt land ligger öppet för dig; du må bo var du finner för gott."
\par 16 Och till Sara sade han: "Se, jag giver åt din broder tusen siklar silver; det skall för dig vara en försoningsgåva inför allt ditt folk. Så har du inför alla fått upprättelse."
\par 17 Och Abraham bad till Gud, och Gud botade Abimelek och hans hustru och hans tjänstekvinnor, så att de åter kunde föda barn.
\par 18 HERREN hade nämligen gjort alla kvinnor i Abimeleks hus ofruktsamma, för Saras, Abrahams hustrus, skull.

\chapter{21}

\par 1 Och HERREN såg till Sara, såsom han hade lovat, och HERREN gjorde med Sara såsom han hade sagt.
\par 2 Sara blev havande och födde åt Abraham en son på hans ålderdom, vid den bestämda tid som Gud hade sagt honom.
\par 3 Och Abraham gav den son som var född åt honom, den som Sara hade fött åt honom, namnet Isak.
\par 4 Och Abraham omskar sin son Isak, när denne var åtta dagar gammal, såsom Gud hade bjudit honom.
\par 5 Och Abraham var hundra år gammal, när hans son Isak föddes åt honom.
\par 6 Och Sara sade: "Gud har berett mig ett löje; var och en som får höra detta skall le mot mig."
\par 7 Och hon sade: "Vem skulle hava sagt Abraham att Sara skulle giva barn di? Och nu har jag fött honom en son på hans ålderdom!"
\par 8 Och barnet växte upp och blev avvant; och den dag då Isak avvandes gjorde Abraham ett stort gästabud.
\par 9 Då fick Sara se Hagars, den egyptiska kvinnans, son, som denna hade fött åt Abraham, leka och skämta;
\par 10 och hon sade till Abraham: "Driv ut denna tjänstekvinna och hennes son, ty denna tjänstekvinnas son skall icke ärva med min son Isak."
\par 11 Det talet misshagade Abraham mycket för hans sons skull.
\par 12 Men Gud sade till Abraham: "Du må icke för gossens och för din tjänstekvinnas skull låta detta misshaga dig. Lyssna till Sara i allt vad hon säger dig; ty genom Isak är det som säd skall uppkallas efter dig.
\par 13 Men också tjänstekvinnans son skall jag göra till ett folk, därför att han är din säd."
\par 14 Bittida följande morgon tog Abraham bröd och en lägel med vatten och gav det åt Hagar; han lade det på hennes rygg och gav henne barnet med och lät henne gå. Och hon begav sig åstad och irrade omkring i Beer-Sebas öken.
\par 15 Men när vattnet i lägeln hade tagit slut, kastade hon barnet ifrån sig under en buske
\par 16 och gick bort och satte sig ett stycke därifrån, på ett bågskotts avstånd, ty hon tänkte: "Jag förmår icke se på, huru barnet dör." Och där hon nu satt, på något avstånd, brast hon ut i gråt.
\par 17 Då hörde Gud gossens röst, och Guds ängel ropade till Hagar från himmelen och sade till henne: "Vad fattas dig, Hagar? Frukta icke; ty Gud har hört gossens röst, där han ligger.
\par 18 Gå och lyft upp gossen, och tag honom vid handen; jag skall göra honom till ett stort folk."
\par 19 Och Gud öppnade hennes ögon, så att hon blev varse en vattenbrunn. Och hon gick dit och fyllde sin lägel med vatten och gav gossen att dricka.
\par 20 Och Gud var med gossen, och han växte upp och bodde i öknen och blev med tiden en bågskytt.
\par 21 Han bodde i öknen Paran; och hans moder tog en hustru åt honom från Egyptens land.
\par 22 Vid den tiden kom Abimelek med Pikol, sin härhövitsman, och talade med Abraham och sade: "Gud är med dig i allt vad du gör.
\par 23 Så lova mig nu här med ed vid Gud att du icke skall göra dig skyldig till något svek mot mig eller mina barn och efterkommande, utan att du skall bevisa mig och det land där du nu bor såsom främling samma godhet som jag har bevisat dig."
\par 24 Abraham sade: "Det vill jag lova dig."
\par 25 Dock gjorde Abraham Abimelek förebråelser angående en vattenbrunn som Abimeleks tjänare hade tagit ifrån honom.
\par 26 Men Abimelek svarade: "Jag vet icke vem som har gjort detta; själv har du ingenting sagt mig, och jag har icke hört något därom förrän i dag."
\par 27 Då tog Abraham får och fäkreatur och gav åt Abimelek; och de slöto förbund med varandra.
\par 28 Men Abraham ställde sju lamm av hjorden avsides.
\par 29 Då sade Abimelek till Abraham: "Vad betyda de sju lammen som du har ställt där avsides?"
\par 30 Han svarade: "Dessa sju lamm skall du taga emot av mig, för att detta må vara mig till ett vittnesbörd därom att det är jag som har grävt denna brunn."
\par 31 Därav kallades det stället Beer-Seba, eftersom de båda där gingo eden.
\par 32 När de så hade slutit förbund vid Beer-Seba, stodo Abimelek och hans härhövitsman Pikol upp och vände tillbaka till filistéernas land.
\par 33 Och Abraham planterade en tamarisk vid Beer-Seba och åkallade där HERRENS, den evige Gudens, namn.
\par 34 Och Abraham bodde i filistéernas land en lång tid.

\chapter{22}

\par 1 En tid härefter hände sig att Gud satte Abraham på prov. Han sade till honom: "Abraham!" Han svarade: "Här är jag."
\par 2 Då sade han: "Tag din son Isak, din ende son, som du har kär, och gå bort till Moria land, och offra honom där såsom brännoffer, på ett berg som jag skall säga dig."
\par 3 Bittida följande morgon lastade Abraham sin åsna och tog med sig två sina tjänare och sin son Isak; och sedan han hade huggit sönder ved till brännoffer, bröt han upp och begav sig på väg till den plats som Gud hade sagt honom.
\par 4 När nu Abraham på tredje dagen lyfte upp sina ögon och fick se platsen på avstånd,
\par 5 sade han till sina tjänare: "Stannen I här med åsnan; jag och gossen vilja gå ditbort. När vi hava tillbett, skola vi komma tillbaka till eder."
\par 6 Och Abraham tog veden till brännoffret och lade den på sin son Isak, men själv tog han elden och kniven, och de gingo så båda tillsammans.
\par 7 Då talade Isak till sin fader Abraham och sade: "Min fader!" Han svarade: "Vad vill du, min son?" Han sade: "Se, här är elden och veden, men var är fåret till brännoffret?"
\par 8 Abraham svarade: "Gud utser nog åt sig fåret till brännoffret, min son." Så gingo de båda tillsammans.
\par 9 När de nu hade kommit till den plats som Gud hade sagt Abraham, byggde han där ett altare och lade veden därpå, sedan band han sin son Isak och lade honom på altaret ovanpå veden.
\par 10 Och Abraham räckte ut sin hand och tog kniven för att slakta sin son.
\par 11 Då ropade HERRENS ängel till honom från himmelen och sade: "Abraham! Abraham!" Han svarade: "Här är jag."
\par 12 Då sade han: "Låt icke din hand komma vid gossen, och gör honom intet; ty nu vet jag att du fruktar Gud, nu då du icke har undanhållit mig din ende son."
\par 13 När då Abraham lyfte upp sina ögon, fick han bakom sig se en vädur, som hade fastnat med sina horn i ett snår; och Abraham gick dit och tog väduren och offrade den till brännoffer i sin sons ställe.
\par 14 Och Abraham gav den platsen namnet HERREN utser; nu för tiden heter den Berget där HERREN låter se sig.
\par 15 Och HERRENS ängel ropade för andra gången till Abraham från himmelen
\par 16 och sade: "Jag svär vid mig själv, säger HERREN: Eftersom du har gjort detta och icke undanhållit mig din ende son
\par 17 därför skall jag rikligen välsigna dig och göra din säd talrik såsom stjärnorna på himmelen och såsom sanden på havets strand; och din säd skall intaga sina fienders portar.
\par 18 Och i din säd skola alla folk på jorden välsigna sig, därför att du lyssnade till mina ord."
\par 19 Sedan vände Abraham tillbaka till sina tjänare; och de stodo upp och gingo tillsammans till Beer-Seba. Och Abraham bodde i Beer-Seba.
\par 20 En tid härefter blev så berättat för Abraham: "Se, Milka har ock fött barn åt din broder Nahor."
\par 21 Barnen voro Us, hans förstfödde, och Bus, dennes broder, och Kemuel, Arams fader,
\par 22 vidare Kesed, Haso, Pildas, Jidlaf och Betuel.
\par 23 Men Betuel födde Rebecka. Dessa åtta föddes av Milka åt Nahor, Abrahams broder.
\par 24 Och hans bihustru, som hette Reuma, födde ock barn, nämligen Teba, Gaham, Tahas och Maaka.

\chapter{23}

\par 1 Och Sara blev ett hundra tjugusju år gammal; så gammal blev Sara.
\par 2 Och Sara dog i Kirjat-Arba, det är Hebron, i Kanaans land. Och Abraham kom och höll dödsklagan efter Sara och begrät henne.
\par 3 Därefter stod Abraham upp och gick bort ifrån den döda och talade så till Hets barn:
\par 4 "Jag är en främling och gäst hos eder. Låten mig nu få en egen grav hos eder, så att jag kan föra min döda dit och begrava henne."
\par 5 Då svarade Hets barn Abraham och sade till honom:
\par 6 "Hör oss, herre. Du är en Guds hövding bland oss; begrav din döda i den förnämligaste av våra gravar. Ingen av oss skall vägra att giva dig sin grav till att där begrava din döda."
\par 7 Men Abraham stod upp och bugade sig för landets folk, Hets barn;
\par 8 och han talade med dem och sade: "Om I tillstädjen att jag för ut min döda och begraver henne, så hören mig och läggen eder ut för mig hos Efron, Sohars son,
\par 9 så att han giver mig den grotta i Makpela, som tillhör honom, och som ligger vid ändan av hans åker. Mot full betalning i eder krets må han giva mig den till egen grav."
\par 10 Men Efron satt där bland Hets barn. Och Efron, hetiten, svarade Abraham i närvaro av Hets barn, alla som bodde inom hans stadsport; han sade:
\par 11 "Icke så, min herre. Hör mig: Jag skänker dig åkern; grottan som finnes där skänker jag dig ock; jag skänker dig den inför mina landsmäns ögon; begrav där din döda."
\par 12 Men Abraham bugade sig för landets folk;
\par 13 och han talade till Efron i närvaro av landets folk och sade: "Värdes dock höra mig. Jag vill betala åkerns värde; tag emot det av mig, och låt mig där begrava min döda."
\par 14 Då svarade Efron Abraham och sade till honom:
\par 15 "Min herre, hör mig. Ett jordstycke till ett värde av fyra hundra siklar silver, vad betyder det mellan mig och dig? Begrav du din döda."
\par 16 Och Abraham förstod Efron och vägde upp åt honom den summa som Efron hade uppgivit i närvaro av Hets barn, fyra hundra siklar silver, sådant silver som var gångbart i handel.
\par 17 Så skedde det att Efrons åker i Makpela, gent emot Mamre, själva åkern med grottan som fanns där och alla träd på åkern, så långt dess område sträckte sig runt omkring, blev överlåten åt Abraham till egendom
\par 18 inför Hets barns ögon, inför alla som bodde inom hans stadsport.
\par 19 Därefter begrov Abraham sin hustru Sara i grottan på åkern i Makpela, gent emot Mamre, det är Hebron, i Kanaans land.
\par 20 Åkern med grottan som fanns där blev så av Hets barn överlåten åt Abraham till egen grav.

\chapter{24}

\par 1 Abraham var nu gammal och kommen till hög ålder, och HERREN hade välsignat Abraham i alla stycken.
\par 2 Då sade han till sin äldste hustjänare, den som förestod all hans egendom: "Lägg din hand under min länd;
\par 3 jag vill av dig taga en ed vid HERREN, himmelens Gud och jordens Gud, att du icke till hustru åt min son skall taga en dotter till någon av kananéerna bland vilka jag bor,
\par 4 utan att du skall gå till mitt eget land och till min släkt och där taga hustru åt min son Isak."
\par 5 Tjänaren sade till honom: "Men om så händer, att kvinnan icke vill följa mig hit till landet, måste jag då föra din son tillbaka till det land som du har kommit ifrån?"
\par 6 Abraham svarade honom: "Tag dig till vara för att föra min son dit tillbaka.
\par 7 HERREN, himmelens Gud, som har fört mig bort ifrån min faders hus och ifrån mitt fädernesland, han som har talat till mig och svurit och sagt: 'Åt din säd skall jag giva detta land', han skall sända sin ängel framför dig, så att du därifrån skall kunna få en hustru åt min son.
\par 8 Men om kvinnan icke vill följa dig, så är du fri ifrån denna din ed till mig. Allenast må du icke föra min son dit tillbaka."
\par 9 Då lade tjänaren sin hand under sin herre Abrahams länd och lovade honom detta med ed.
\par 10 Och tjänaren tog tio av sin herres kameler och drog åstad med allahanda dyrbara gåvor från sin herre; han stod upp och drog åstad till Nahors stad i Aram-Naharaim.
\par 11 Där lät han kamelerna lägra sig utanför staden, vid en vattenbrunn; och det led mot aftonen, den tid då kvinnorna plägade komma ut för att hämta vatten.
\par 12 Och han sade: "HERRE, min herre Abrahams Gud, låt mig i dag få ett lyckosamt möte, och gör nåd med min herre Abraham.
\par 13 Se, jag står här vid vattenkällan, och stadsbornas döttrar komma hitut för att hämta vatten.
\par 14 Om jag nu säger till en flicka: 'Håll hit din kruka, och låt mig få dricka' och hon då svarar: 'Drick; dina kameler vill jag ock vattna', må hon då vara den som du har utsett åt din tjänare Isak, så skall jag därav veta att du har gjort nåd med min herre."
\par 15 Och se, innan han hade slutat att tala, kom Rebecka ditut, en dotter till Betuel, som var son till Milka, Abrahams broder Nahors hustru; och hon bar sin kruka på axeln.
\par 16 Och flickan var mycket fager att skåda, en jungfru som ingen man hade känt. Hon gick nu ned till källan och fyllde sin kruka och steg så upp igen.
\par 17 Då skyndade tjänaren emot henne och sade: "Låt mig få dricka litet vatten ur din kruka."
\par 18 Hon svarade: "Drick, min herre" och lyfte strax ned krukan på sin hand och gav honom att dricka.
\par 19 Och sedan hon hade givit honom att dricka, sade hon: "Jag vill ock ösa upp vatten åt dina kameler, till dess att de alla hava fått dricka."
\par 20 Och hon tömde strax sin kruka i vattenhon och skyndade åter till brunnen för att hämta vatten och öste så upp åt alla hans kameler.
\par 21 Men mannen såg på henne under tystnad och undrade om HERREN hade gjort hans resa lyckosam eller icke.
\par 22 Och när alla kamelerna hade druckit, tog mannen fram en näsring av guld, en halv sikel i vikt, och två armband av guld, tio siklar i vikt,
\par 23 och frågade: "Vems dotter är du? Säg mig det. Och säg mig om vi kunna få natthärbärge i din faders hus?"
\par 24 Hon svarade honom: "Jag är dotter till Betuel, Milkas son, som av henne föddes åt Nahor."
\par 25 Och hon sade ytterligare till honom: "Vi hava rikligt med både halm och foder; natthärbärge kan du ock få."
\par 26 Då böjde mannen sig ned och tillbad HERREN
\par 27 och sade: "Lovad vare HERREN, min herre Abrahams Gud, som icke har tagit sin nåd och trofasthet ifrån min herre! Mig har HERREN ledsagat på vägen, hem till min herres fränder."
\par 28 Och flickan skyndade åstad och berättade allt detta i sin moders hus.
\par 29 Men Rebecka hade en broder som hette Laban. Och Laban skyndade åstad till mannen därute vid källan.
\par 30 När han nämligen såg näsringen och armbanden som hans syster bar, och när han hörde huru hans syster Rebecka berättade: 'Så och så talade mannen till mig', då begav han sig ut till mannen, där denne stod hos kamelerna vid källan.
\par 31 Och han sade: "Kom in, du HERRENS välsignade; varför står du härute? Jag har berett plats i huset, och rum finnes för kamelerna."
\par 32 Så kom då mannen in i huset; och man lastade av kamelerna, och tog fram halm och foder åt kamelerna, och vatten till att två hans och hans följeslagares fötter.
\par 33 Och man satte fram mat för honom; men han sade: "Jag vill icke äta, förrän jag har framfört mitt ärende." Laban svarade: "Så tala då."
\par 34 Då sade han: "Jag är Abrahams tjänare.
\par 35 Och HERREN har rikligen välsignat min herre, så att han har blivit en mäktig man; han har givit honom får och fäkreatur, silver och guld, tjänare och tjänarinnor, kameler och åsnor.
\par 36 Och Sara, min herres hustru, har fött åt min herre en son på sin ålderdom, och åt denne har han givit allt vad han äger.
\par 37 Och min herre har tagit en ed av mig och sagt: 'Till hustru åt min son skall du icke taga en dotter till någon av kananéerna i vilkas land jag bor,
\par 38 utan du skall gå till min faders hus och till min släkt och där taga hustru åt min son.'
\par 39 Då sade jag till min herre: 'Men om nu kvinnan icke vill följa med mig?'
\par 40 Han svarade mig: 'HERREN, inför vilken jag har vandrat, skall sända sin ängel med dig och göra din resa lyckosam, så att du åt min son får en hustru av min släkt och av min faders hus;
\par 41 i sådant fall skall du vara löst från din ed till mig, när du har kommit till min släkt. Också om de icke giva henne åt dig, skall du vara fri ifrån eden till mig.'
\par 42 Så kom jag i dag till källan, och jag sade: HERRE, min herre Abrahams Gud, om du vill låta den resa på vilken jag är stadd bliva lyckosam,
\par 43 må det då ske, när jag nu står här vid vattenkällan, att om en ung kvinna kommer ut för att hämta vatten och jag säger till henne: 'Låt mig få dricka litet vatten ur din kruka'
\par 44 och hon då svarar mig: 'Drick du; åt dina kameler vill jag ock ösa upp vatten' - må hon då vara den kvinna som HERREN har utsett åt min herres son.
\par 45 Och innan jag hade slutat att så tala för mig själv, se, då kom Rebecka ut med sin kruka på axeln och gick ned till källan för att hämta vatten. Då sade jag till henne: 'Låt mig få dricka.'
\par 46 Och strax lyfte hon ned sin kruka från axeln och sade: 'Drick; dina kameler vill jag ock vattna.' Så drack jag, och hon vattnade också kamelerna.
\par 47 Och jag frågade henne och sade: 'Vems dotter är du?' Hon svarade: 'Jag är dotter till Betuel, Nahors son, som föddes åt honom av Milka.' Då satte jag ringen i hennes näsa och armbanden på hennes armar.
\par 48 Och jag böjde mig ned och tillbad HERREN och lovade HERREN, min herre Abrahams Gud, som hade ledsagat mig på den rätta vägen, så att jag åt hans son skulle få min herres frändes dotter.
\par 49 Om I nu viljen visa min herre kärlek och trofasthet, så sägen mig det; varom icke, så sägen mig ock det, för att jag då må vända mig åt annat håll, till höger eller till vänster."
\par 50 Då svarade Laban och Betuel och sade: "Från HERREN har detta utgått; vi kunna i den saken intet säga till dig, varken ont eller gott.
\par 51 Se, där står Rebecka inför dig, tag henne och drag åstad; må hon bliva hustru åt din herres son, såsom HERREN har sagt."
\par 52 När Abrahams tjänare hörde deras ord, föll han ned på jorden och tillbad HERREN.
\par 53 Sedan tog tjänaren fram smycken av silver och guld, så ock kläder, och gav detta åt Rebecka. Jämväl åt hennes broder och hennes moder gav han dyrbara skänker.
\par 54 Och de åto och drucko, han och hans följeslagare, och stannade sedan där över natten. Men om morgonen, när de hade stått upp, sade han: "Låten mig nu fara till min herre."
\par 55 Då sade hennes broder och hennes moder: "Låt flickan stanna hos oss några dagar, tio eller så; sedan må du fara."
\par 56 Men han svarade dem: "Uppehållen mig icke, eftersom HERREN har gjort min resa lyckosam. Låten mig fara; jag vill resa hem till min herre."
\par 57 Då sade de: "Vi vilja kalla hit flickan och fråga henne själv."
\par 58 Och de kallade Rebecka till sig och sade till henne: "Vill du resa med denne man?" Hon svarade: "Ja."
\par 59 Då bestämde de att deras syster Rebecka jämte sin amma skulle fara med Abrahams tjänare och dennes män.
\par 60 Och de välsignade Rebecka och sade till henne: "Av dig, du vår syster, komme tusen gånger tio tusen, och må dina avkomlingar intaga sina fienders portar."
\par 61 Och Rebecka och hennes tärnor stodo upp och satte sig på kamelerna och följde med mannen; så tog tjänaren Rebecka med sig och for sin väg.
\par 62 Men Isak var på väg hem från Beer-Lahai-Roi, ty han bodde i Sydlandet.
\par 63 Och mot aftonen hade Isak gått ut på fältet i sorgsna tankar. När han då lyfte upp sina ögon, fick han se kameler komma.
\par 64 Då nu också Rebecka lyfte upp sina ögon och fick se Isak, steg hon med hast ned från kamelen;
\par 65 och hon frågade tjänaren: "Vem är den mannen som kommer emot oss där på fältet?" Tjänaren svarade: "Det är min herre." Då tog hon sin slöja och höljde sig i den.
\par 66 Och tjänaren förtäljde för Isak huru han hade uträttat allt.
\par 67 Och Isak förde henne in i sin moder Saras tält; och han tog Rebecka till sig, och hon blev hans hustru, och han hade henne kär. Så blev Isak tröstad i sorgen efter sin moder.

\chapter{25}

\par 1 Och Abraham tog sig ännu en hustru, och hon hette Ketura.
\par 2 Hon födde åt honom Simran, Joksan, Medan, Midjan, Jisbak och Sua.
\par 3 Och Joksan födde Saba och Dedan, och Dedans söner voro assuréerna, letuséerna och leumméerna.
\par 4 Och Midjans söner voro Efa, Efer, Hanok, Abida och Eldaa. Alla dessa voro Keturas söner.
\par 5 Och Abraham gav allt vad han ägde åt Isak.
\par 6 Men åt sönerna till sina bihustrur gav Abraham skänker och skilde dem, medan han själv ännu levde, från sin son Isak och lät dem draga österut, bort till Österlandet.
\par 7 Och detta är antalet av Abrahams levnadsår: ett hundra sjuttiofem år;
\par 8 därefter gav Abraham upp andan och dog i en god ålder, gammal och mätt på livet, och blev samlad till sina fäder.
\par 9 Och hans söner Isak och Ismael begrovo honom i grottan i Makpela, på hetiten Efrons, Sohars sons, åker gent emot Mamre,
\par 10 den åker som Abraham hade köpt av Hets barn; där blev Abraham begraven, såväl som hans hustru Sara.
\par 11 Och efter Abrahams död välsignade Gud hans son Isak. Och Isak bodde vid Beer-Lahai-Roi.
\par 12 Och detta är berättelsen om Ismaels släkt, Abrahams sons, som föddes åt Abraham av Hagar, Saras egyptiska tjänstekvinna.
\par 13 Dessa äro namnen på Ismaels söner, med deras namn, efter deras ättföljd: Nebajot, Ismaels förstfödde, vidare Kedar, Adbeel, Mibsam,
\par 14 Misma, Duma och Massa,
\par 15 Hadad och Tema, Jetur, Nafis och Kedma.
\par 16 Dessa voro Ismaels söner och dessa deras namn, i deras byar och tältläger, tolv hövdingar efter deras stammar.
\par 17 Och detta är antalet av Ismaels levnadsår: ett hundra trettiosju år; därefter gav han upp andan och dog och blev samlad till sina fäder.
\par 18 Och de hade sina boningsplatser från Havila ända till Sur, som ligger gent emot Egypten, fram emot Assyrien. Han kom i strid med alla sina bröder.
\par 19 Och detta är berättelsen om Isaks, Abrahams sons, släkt. Abraham födde Isak;
\par 20 och Isak var fyrtio år gammal, när han till hustru åt sig tog Rebecka, som var dotter till araméen Betuel från Paddan-Aram och syster till araméen Laban.
\par 21 Och Isak bad till HERREN för sin hustru Rebecka, ty hon var ofruktsam; och HERREN bönhörde honom, så att hans hustru Rebecka blev havande.
\par 22 Men barnen stötte varandra i hennes liv; då sade hon: "Om det skulle gå så, varför skulle jag då vara till?" Och hon gick bort för att fråga HERREN.
\par 23 Och HERREN svarade henne: "Två folk finnas i ditt liv, två folkstammar skola ur ditt sköte söndras från varandra; den ena stammen skall vara den andra övermäktig, och den äldre skall tjäna den yngre."
\par 24 När sedan tiden var inne att hon skulle föda, se, då funnos tvillingar i hennes liv.
\par 25 Den som först kom fram var rödlätt och över hela kroppen såsom en hårmantel; och de gåvo honom namnet Esau.
\par 26 Därefter kom hans broder fram, och denne höll med sin hand i Esaus häl; och han fick namnet Jakob. Men Isak var sextio år gammal, när de föddes.
\par 27 Och barnen växte upp, och Esau blev en skicklig jägare, som höll sig ute på marken; Jakob åter blev en fromsint man, som bodde i tält.
\par 28 Och Isak hade Esau kärast, ty han hade smak för villebråd; men Rebecka hade Jakob kärast.
\par 29 En gång, då Jakob höll på att koka något till soppa, kom Esau hem från marken, uppgiven av hunger.
\par 30 Och Esau sade till Jakob: "Låt mig få till livs av det röda, det röda du har där; ty jag är uppgiven av hunger." Därav fick han namnet Edom.
\par 31 Men Jakob sade: "Sälj då nu åt mig din förstfödslorätt."
\par 32 Esau svarade: "Jag är ju döden nära; vartill gagnar mig då min förstfödslorätt?"
\par 33 Jakob sade: "Så giv mig nu din ed därpå." Och han gav honom sin ed och sålde så sin förstfödslorätt till Jakob.
\par 34 Men Jakob gav Esau bröd och linssoppa; och han åt och drack och stod sedan upp och gick sin väg. Så ringa aktade Esau sin förstfödslorätt.

\chapter{26}

\par 1 Men en hungersnöd uppstod i landet, en ny hungersnöd, efter den som hade varit förut, i Abrahams tid. Då begav sig Isak till Abimelek, filistéernas konung, i Gerar.
\par 2 Och HERREN uppenbarade sig för honom och sade: "Drag icke ned till Egypten; bo kvar i det land som jag skall säga dig.
\par 3 Stanna såsom främling här i landet; jag skall vara med dig och välsigna dig, ty åt dig och din säd skall jag giva alla dessa länder, och skall hålla den ed som jag har svurit din fader Abraham.
\par 4 Jag skall göra din säd talrik såsom stjärnorna på himmelen, och jag skall giva åt din säd alla dessa länder; och i din säd skola alla folk på jorden välsigna sig,
\par 5 därför att Abraham har lyssnat till mina ord och hållit vad jag har bjudit honom hålla, mina bud, mina stadgar och mina lagar."
\par 6 Så stannade Isak kvar i Gerar.
\par 7 Och när männen på orten frågade honom om hans hustru, sade han: "Hon är min syster." Han fruktade nämligen för att säga att hon var hans hustru, ty han tänkte: "Männen här på orten kunde då dräpa mig för Rebeckas skull, eftersom hon är så fager att skåda."
\par 8 Men när han hade varit där en längre tid, hände sig en gång, då Abimelek, filistéernas konung, blickade ut genom fönstret, att han fick se Isak kärligt skämta med sin hustru Rebecka.
\par 9 Då kallade Abimelek Isak till sig och sade: "Hon är ju din hustru; huru har du då kunnat säga: 'Hon är min syster'?" Isak svarade honom: "Jag fruktade att jag annars skulle bliva dödad för hennes skull."
\par 10 Då sade Abimelek: "Vad har du gjort mot oss! Huru lätt kunde det icke hava skett att någon av folket hade lägrat din hustru? Och så hade du dragit skuld över oss."
\par 11 Sedan bjöd Abimelek allt folket och sade: "Den som kommer vid denne man eller vid hans hustru, han skall straffas med döden."
\par 12 Och Isak sådde där i landet och fick det året hundrafalt, ty HERREN välsignade honom.
\par 13 Och han blev en mäktig man; hans makt blev större och större, så att han till slut var mycket mäktig.
\par 14 Han ägde så många får och fäkreatur och så många tjänare, att filistéerna begynte avundas honom.
\par 15 Och alla de brunnar som hans faders tjänare hade grävt i hans fader Abrahams tid, dem hade filistéerna kastat igen och fyllt med grus.
\par 16 Och Abimelek sade till Isak: "Drag bort ifrån oss; ty du har blivit oss alltför mäktig."
\par 17 Då drog Isak bort därifrån och slog upp sitt läger i Gerars dal och bodde där.
\par 18 Och Isak lät åter gräva ut de vattenbrunnar som hade blivit grävda i hans fader Abrahams tid, men som filistéerna efter Abrahams död hade kastat igen; och han gav dem åter de namn som hans fader hade givit dem.
\par 19 Och Isaks tjänare grävde i dalen och funno där en brunn med rinnande vatten.
\par 20 Men herdarna i Gerar begynte tvista med Isaks herdar och sade: "Vattnet är vårt." Då gav han den brunnen namnet Esek, eftersom de hade kivat med honom.
\par 21 Därefter grävde de en annan brunn, men om den kommo de ock i tvist; då gav han den namnet Sitna.
\par 22 Sedan begav han sig därifrån till en annan plats och grävde åter en brunn; om den tvistade de icke. Därför gav han denna namnet Rehobot, i det han sade: "Nu har ju HERREN givit oss utrymme, så att vi kunna föröka oss i landet."
\par 23 Sedan drog han därifrån upp till Beer-Seba.
\par 24 Och HERREN uppenbarade sig för honom den natten och sade: "Jag är Abrahams, din faders, Gud. Frukta icke, ty jag är med dig, och jag skall välsigna dig och göra din säd talrik, för min tjänare Abrahams skull."
\par 25 Då byggde han där ett altare och åkallade HERRENS namn och slog där upp sitt tält. Och Isaks tjänare grävde där en brunn.
\par 26 Och Abimelek begav sig till honom från Gerar med Ahussat, sin vän, och Pikol, sin härhövitsman.
\par 27 Men Isak sade till dem: "Varför kommen I till mig, I som haten mig och haven drivit mig ifrån eder?"
\par 28 De svarade: "Vi hava tydligt sett att HERREN är med dig; därför tänkte vi: 'Låt oss giva varandra en ed, vi och du, så att vi sluta ett förbund med dig,
\par 29 att du icke skall göra oss något ont, likasom vi å vår sida icke hava kommit vid dig, och likasom vi icke hava gjort dig annat än gott och hava låtit dig fara i frid.' Du är nu HERRENS välsignade."
\par 30 Då gjorde han ett gästabud för dem, och de åto och drucko.
\par 31 Bittida följande morgon svuro de varandra eden; sedan lät Isak dem gå, och de foro ifrån honom i frid.
\par 32 Samma dag kommo Isaks tjänare och berättade för honom om den brunn som de hade grävt och sade till honom: "Vi hava funnit vatten."
\par 33 Och han kallade den Sibea. Därav heter staden Beer-Seba ännu i dag.
\par 34 När Esau var fyrtio år gammal, tog han till hustrur Judit, dotter till hetiten Beeri, och Basemat, dotter till hetiten Elon.
\par 35 Men dessa blevo en hjärtesorg för Isak och Rebecka.

\chapter{27}

\par 1 När Isak hade blivit gammal och hans ögon voro skumma, så att han icke kunde se, kallade han till sig Esau, sin äldste son, och sade till honom: "Min son!" Han svarade honom: "Vad vill du?"
\par 2 Då sade han: "Se, jag är gammal och vet icke när jag skall dö.
\par 3 Så tag nu dina jaktredskap, ditt koger och din båge, och gå ut i marken och jaga villebråd åt mig;
\par 4 red sedan till åt mig en smaklig rätt, en sådan som jag tycker om, och bär in den till mig till att äta, på det att min själ må välsigna dig, förrän jag dör."
\par 5 Men Rebecka hörde huru Isak talade till sin son Esau. Och medan Esau gick ut i marken för att jaga villebråd till att föra hem,
\par 6 sade Rebecka till sin son Jakob: "Se, jag har hört din fader tala så till din broder Esau:
\par 7 'Hämta mig villebråd och red till åt mig en smaklig rätt, på det att jag må äta och sedan välsigna dig inför HERREN, förrän jag dör.'
\par 8 Så hör nu vad jag säger, min son, och gör vad jag bjuder dig.
\par 9 Gå bort till hjorden och hämta mig därifrån två goda killingar, så vill jag av dem tillreda en smaklig rätt åt din fader, en sådan som han tycker om.
\par 10 Och du skall bära in den till din fader till att äta, på det att han må välsigna dig, förrän han dör."
\par 11 Men Jakob sade till sin moder Rebecka: "Min broder Esau är ju luden, och jag är slät.
\par 12 Kanhända tager min fader på mig, och jag bliver då av honom hållen för en bespottare och skaffar mig förbannelse i stället för välsignelse."
\par 13 Då sade hans moder till honom: "Den förbannelsen komme över mig, min son; hör nu allenast vad jag säger, och gå och hämta dem åt mig."
\par 14 Då gick han och hämtade dem och bar dem till sin moder; och hans moder tillredde en smaklig rätt, en sådan som hans fader tyckte om.
\par 15 Och Rebecka tog Esaus, sin äldre sons, högtidskläder, som hon hade hos sig i huset, och satte dem på Jakob, sin yngre son.
\par 16 Och med skinnen av killingarna beklädde hon hans händer och den släta delen av hans hals.
\par 17 Sedan lämnade hon åt sin son Jakob den smakliga rätten och brödet som hon hade tillrett.
\par 18 Och han gick in till sin fader och sade: "Min fader!" Han svarade: "Vad vill du? Vem är du, min son?"
\par 19 Då sade Jakob till sin fader: "Jag är Esau, din förstfödde. Jag har gjort såsom du tillsade mig; sätt dig upp och ät av mitt villebråd, på det att din själ må välsigna mig."
\par 20 Men Isak sade till sin son: "Huru har du så snart kunnat finna något, min son?" Han svarade: "HERREN, din Gud, skickade det i min väg."
\par 21 Då sade Isak till Jakob: "Kom hit, min son, och låt mig taga på dig och känna om du är min son Esau eller icke."
\par 22 Och Jakob gick fram till sin fader Isak; och när denne hade tagit på honom, sade han: "Rösten är Jakobs röst, men händerna äro Esaus händer."
\par 23 Och han kände icke igen honom, ty hans händer voro ludna såsom hans broder Esaus händer; och han välsignade honom.
\par 24 Men han frågade: "Är du verkligen min son Esau?" Han svarade: "Ja."
\par 25 Då sade han: "Bär hit maten åt mig och låt mig äta av min sons villebråd, på det att min själ må välsigna dig." Och han bar fram den till honom, och han åt; och han räckte honom vin, och han drack.
\par 26 Därefter sade hans fader Isak till honom: "Kom hit och kyss mig, min son."
\par 27 När han då gick fram och kysste honom, kände han lukten av hans kläder och välsignade honom; han sade: "Se, av min son utgår doft, lik doften av en mark, som HERREN har välsignat.
\par 28 Så give dig Gud av himmelens dagg och av jordens fetma och säd och vin i rikligt mått.
\par 29 Folk tjäne dig, och folkslag falle ned för dig. Bliv en herre över dina bröder, och må din moders söner falla ned för dig. Förbannad vare den som förbannar dig, och välsignad vare den som välsignar dig!"
\par 30 Men när Isak hade givit Jakob sin välsignelse och Jakob just hade gått ut från sin fader Isak, kom hans broder Esau hem från jakten.
\par 31 Därefter tillredde också han en smaklig rätt och bar in den till sin fader och sade till sin fader: "Må min fader stå upp och äta av sin sons villebråd, på det att din själ må välsigna mig."
\par 32 Hans fader Isak frågade honom: "Vem är du?" Han svarade: "Jag är Esau, din förstfödde son."
\par 33 Då blev Isak övermåttan häpen och sade: "Vem var då den jägaren som bar in till mig sitt villebråd, så att jag åt av allt, förrän du kom, och sedan välsignade honom? Välsignad skall han ock förbliva."
\par 34 När Esau hörde sin faders ord, brast han ut i högljudd och bitter klagan och sade till sin fader: "Välsigna också mig, min fader."
\par 35 Men han svarade: "Din broder har kommit med svek och tagit din välsignelse."
\par 36 Då sade han: "Han heter ju Jakob, och han har nu också två gånger bedragit mig. Min förstfödslorätt har han tagit, och se, nu har han ock tagit min välsignelse." Och han frågade: "Har du då ingen välsignelse kvar för mig?"
\par 37 Isak svarade och sade till Esau: "Se, jag har satt honom till en herre över dig, och alla hans bröder har jag givit honom till tjänare, och med säd och vin har jag begåvat honom; vad skall jag då nu göra för dig, min son?"
\par 38 Esau sade till sin fader: "Har du då allenast den enda välsignelsen, min fader? Välsigna också mig, min fader." Och Esau brast ut i gråt.
\par 39 Då svarade hans fader Isak och sade till honom: "Se, fjärran ifrån jordens fetma skall din boning vara och utan dagg från himmelen ovanefter.
\par 40 Av ditt svärd skall du leva, och du skall tjäna din broder. Men det skall ske, när du samlar din kraft, att du river hans ok från din hals."
\par 41 Och Esau blev hätsk mot Jakob för den välsignelses skull som hans fader hade givit honom. Och Esau sade vid sig själv: "Snart skola de dagar komma, då vi få sörja vår fader; då skall jag dräpa min broder Jakob."
\par 42 När man nu berättade för Rebecka vad hennes äldre son Esau hade sagt, sände hon och lät kalla till sig sin yngre son Jakob och sade till honom: "Se, din broder Esau vill hämnas på dig och dräpa dig.
\par 43 Så hör nu vad jag säger, min son: stå upp och fly till min broder Laban i Haran,
\par 44 och stanna någon tid hos honom, till dess din broders förbittring har upphört,
\par 45 ja, till dess din broders vrede mot dig har upphört och han förgäter vad du har gjort mot honom. Då skall jag sända åstad och hämta dig därifrån. Varför skall jag mista eder båda på samma gång?"
\par 46 Och Rebecka sade till Isak: "Jag är led vid livet för Hets döttrars skull. Om Jakob tager hustru bland Hets döttrar, en sådan som dessa, någon bland landets döttrar, varför skulle jag då leva?"

\chapter{28}

\par 1 Då kallade Isak till sig Jakob och välsignade honom; och han bjöd honom och sade till honom: "Tag dig icke till hustru någon av Kanaans döttrar,
\par 2 utan stå upp och begiv dig till Paddan-Aram, till Betuels, din morfaders, hus, och tag dig en hustru därifrån, någon av Labans, din morbroders, döttrar.
\par 3 Och må Gud den Allsmäktige välsigna dig och göra dig fruktsam och föröka dig, så att skaror av folk komma av dig;
\par 4 må han giva åt dig Abrahams välsignelse, åt dig och din säd med dig, så att du får taga i besittning det land som Gud har givit åt Abraham, och där du nu bor såsom främling."
\par 5 Så sände Isak åstad Jakob, och denne begav sig till Paddan-Aram, till araméen Laban, Betuels son, som var broder till Rebecka, Jakobs och Esaus moder.
\par 6 När nu Esau såg att Isak hade välsignat Jakob och sänt honom till Paddan-Aram för att därifrån taga sig hustru - ty han hade välsignat honom och bjudit honom och sagt: "Du skall icke taga till hustru någon av Kanaans döttrar" -
\par 7 och när han såg att Jakob hade lytt sin fader och moder och begivit sig till Paddan-Aram,
\par 8 då märkte Esau att Kanaans döttrar misshagade hans fader Isak;
\par 9 och Esau gick bort till Ismael och tog Mahalat, Abrahams son Ismaels dotter, Nebajots syster, till hustru åt sig, utöver de hustrur han förut hade.
\par 10 Men Jakob begav sig från Beer-Seba på väg till Haran.
\par 11 Och han kom då till den heliga platsen och stannade där över natten, ty solen hade gått ned; och han tog en av stenarna på platsen för att hava den till huvudgärd och lade sig att sova där.
\par 12 Då hade han en dröm. Han såg en stege vara rest på jorden, och dess övre ände räckte upp till himmelen, och Guds änglar stego upp och ned på den.
\par 13 Och se, HERREN stod framför honom och sade: "Jag är HERREN, Abrahams, din faders, Gud och Isaks Gud. Det land där du ligger skall jag giva åt dig och din säd.
\par 14 Och din säd skall bliva såsom stoftet på jorden, och du skall utbreda dig åt väster och öster och norr och söder, och alla släkter på jorden skola varda välsignade i dig och i din säd.
\par 15 Och se, jag är med dig och skall bevara dig, varthelst du går, och jag skall föra dig tillbaka till detta land; ty jag skall icke övergiva dig, till dess jag har gjort vad jag har lovat dig."
\par 16 När Jakob vaknade upp ur sömnen sade han: "HERREN är sannerligen på denna plats, och jag visste det icke!"
\par 17 Och han betogs av fruktan och sade: "Detta måste vara en helig plats, här bor förvisso Gud, och här är himmelens port."
\par 18 Och bittida om morgonen stod Jakob upp och tog stenen som han hade haft till huvudgärd och reste den till en stod och göt olja därovanpå.
\par 19 Och han gav den platsen namnet Betel; förut hade staden hetat Lus.
\par 20 Och Jakob gjorde ett löfte och sade: "Om Gud är med mig och bevarar mig under den resa som jag nu är stadd på och giver mig bröd till att äta och kläder till att kläda mig med,
\par 21 så att jag kommer i frid tillbaka till min faders hus, då skall HERREN vara min Gud;
\par 22 och denna sten som jag har rest till en stod skall bliva ett Guds hus, och av allt vad du giver mig skall jag giva dig tionde."

\chapter{29}

\par 1 Och Jakob begav sig åstad på väg till Österlandet.
\par 2 Där fick han se en brunn på fältet, och vid den lågo tre fårhjordar, ty ur denna brunn plägade man vattna hjordarna. Och stenen som låg över brunnens öppning var stor;
\par 3 därför plägade man låta alla hjordarna samlas dit och vältrade så stenen från brunnens öppning och vattnade fåren; sedan lade man stenen tillbaka på sin plats över brunnens öppning.
\par 4 Och Jakob sade till männen: "Mina bröder, varifrån ären I?" De svarade: "Vi äro från Haran."
\par 5 Då sade han till dem: "Kännen I Laban, Nahors son?" De svarade: "Ja."
\par 6 Han frågade dem vidare: "Står det väl till med honom?" De svarade: "Ja; och se, där kommer hans dotter Rakel med fåren."
\par 7 Han sade: "Det är ju ännu full dag; ännu är det icke tid att samla boskapen. Vattnen fåren, och fören dem åter i bet."
\par 8 Men de svarade: "Vi kunna icke göra det, förrän alla hjordarna hava blivit samlade och man har vältrat stenen från brunnens öppning; då vattna vi fåren."
\par 9 Medan han ännu talade med dem, hade Rakel kommit dit med sin faders får; ty hon plägade vakta dem.
\par 10 När Jakob fick se sin morbroder Labans dotter Rakel komma med Labans, hans morbroders, får, gick han fram och vältrade stenen från brunnens öppning och vattnade sin morbroder Labans får.
\par 11 Och Jakob kysste Rakel och brast ut i gråt.
\par 12 Och Jakob omtalade för Rakel att han var hennes faders frände, och att han var Rebeckas son; och hon skyndade åstad och omtalade det för sin fader.
\par 13 Då nu Laban fick höras talas om sin systerson Jakob, skyndade han emot honom och tog honom i famn och kysste honom och förde honom in i sitt hus; och han förtäljde för Laban allt som hade hänt honom.
\par 14 Och Laban sade till honom: "Ja, du är mitt kött och ben." Och han stannade hos honom en månads tid.
\par 15 Och Laban sade till Jakob: "Du är ju min frände. Skulle du då tjäna mig för intet? Säg mig vad du vill hava i lön?"
\par 16 Nu hade Laban två döttrar; den äldre hette Lea, och den yngre hette Rakel.
\par 17 Och Leas ögon voro matta, men Rakel hade en skön gestalt och var skön att skåda.
\par 18 Och Jakob hade fattat kärlek till Rakel; därför sade han: "Jag vill tjäna dig i sju år för Rakel, din yngre dotter."
\par 19 Laban svarade: "Det är bättre att jag giver henne åt dig, än att jag skulle giva henne åt någon annan; bliv kvar hos mig."
\par 20 Så tjänade Jakob för Rakel i sju år, och det tycktes honom vara allenast några dagar; så kär hade han henne.
\par 21 Därefter sade Jakob till Laban: "Giv mig min hustru, ty min tid är nu förlupen; låt mig gå in till henne."
\par 22 Då bjöd Laban tillhopa allt folket på orten och gjorde ett gästabud.
\par 23 Men när aftonen kom, tog han sin dotter Lea och förde henne till honom, och han gick in till henne.
\par 24 Och Laban gav sin tjänstekvinna Silpa åt sin dotter Lea till tjänstekvinna.
\par 25 Om morgonen fick Jakob se att det var Lea. Då sade han till Laban: "Vad har du gjort mot mig? Var det icke för Rakel jag tjänade hos dig? Varför har du så bedragit mig?"
\par 26 Laban svarade: "Det är icke sed på vår ort att man giver bort den yngre före den äldre.
\par 27 Låt nu dennas bröllopsvecka gå till ända, så vilja vi giva dig också den andra, mot det att du gör tjänst hos mig i ännu ytterligare sju år."
\par 28 Och Jakob samtyckte härtill och lät hennes bröllopsvecka gå till ända. Sedan gav han honom sin dotter Rakel till hustru.
\par 29 Och Laban gav sin tjänstekvinna Bilha åt sin dotter Rakel till tjänstekvinna.
\par 30 Så gick han in också till Rakel, och han hade Rakel kärare än Lea. Sedan tjänade han hos honom i ännu ytterligare sju år.
\par 31 Men då HERREN såg att Lea var försmådd, gjorde han henne fruktsam, medan Rakel var ofruktsam.
\par 32 Och Lea blev havande och födde en son, och hon gav honom namnet Ruben, ty hon tänkte: "HERREN har sett till mitt lidande; ja, nu skall min man hava mig kär."
\par 33 Och hon blev åter havande och födde en son. Då sade hon: "HERREN har hört att jag har varit försmådd, därför har han givit mig också denne." Och hon gav honom namnet Simeon.
\par 34 Och åter blev hon havande och födde en son. Då sade hon: "Nu skall väl ändå min man hålla sig till mig; jag har ju fött honom tre söner." Därav fick denne namnet Levi.
\par 35 Åter blev hon havande och födde en son. Då sade hon: "Nu vill jag tacka HERREN." Därför gav hon honom namnet Juda. Sedan upphörde hon att föda.

\chapter{30}

\par 1 Då nu Rakel såg att hon icke födde barn åt Jakob, avundades hon sin syster och sade till Jakob: "Skaffa mig barn, eljest dör jag."
\par 2 Då upptändes Jakobs vrede mot Rakel, och han svarade: "Håller du då mig för Gud? Det är ju han som förmenar dig livsfrukt."
\par 3 Hon sade: "Se, där är min tjänarinna Bilha; gå in till henne, för att hon må föda barn i mitt sköte, så att genom henne också jag får avkomma."
\par 4 Så gav hon honom sin tjänstekvinna Bilha till hustru, och Jakob gick in till henne.
\par 5 Och Bilha blev havande och födde åt Jakob en son.
\par 6 Då sade Rakel: "Gud har skaffat rätt åt mig; han har hört min röst och givit mig en son." Därför gav hon honom namnet Dan.
\par 7 Åter blev Bilha, Rakels tjänstekvinna, havande, och hon födde åt Jakob en andre son.
\par 8 Då sade Rakel: "Strider om Gud har jag stritt med min syster och har vunnit seger." Och hon gav honom namnet Naftali.
\par 9 Då Lea nu såg att hon hade upphört att föda, tog hon sin tjänstekvinna Silpa och gav henne åt Jakob till hustru.
\par 10 Och Silpa, Leas tjänstekvinna, födde åt Jakob en son.
\par 11 Då sade Lea: "Till lycka!" Och hon gav honom namnet Gad.
\par 12 Och Silpa, Leas tjänstekvinna, födde åt Jakob en andre son.
\par 13 Då sade Lea: "Till sällhet för mig! Ja, jungfrur skola prisa mig säll." Och hon gav honom namnet Aser.
\par 14 Men Ruben gick ut en gång vid tiden för veteskörden och fann då kärleksäpplen på marken och bar dem till sin moder Lea. Då sade Rakel till Lea: "Giv mig några av din sons kärleksäpplen."
\par 15 Men hon svarade henne: "Är det icke nog att du har tagit min man? Vill du ock taga min sons kärleksäpplen?" Rakel sade: "Må han då i natt ligga hos dig, om jag får din sons kärleksäpplen."
\par 16 När nu Jakob om aftonen kom hem från marken, gick Lea honom till mötes och sade: "Till mig skall du gå in; ty jag har givit min sons kärleksäpplen såsom lön för dig." Så låg han hos henne den natten.
\par 17 Och Gud hörde Lea, så att hon blev havande, och hon födde åt Jakob en femte son.
\par 18 Då sade Lea: "Gud har givit mig min lön, för det att jag gav min tjänstekvinna åt min man." Och hon gav honom namnet Isaskar.
\par 19 Åter blev Lea havande, och hon födde åt Jakob en sjätte son.
\par 20 Då sade Lea: "Gud har givit mig en god gåva. Nu skall min man förbliva boende hos mig, ty jag har fött honom sex söner." Och hon gav honom namnet Sebulon.
\par 21 Därefter födde hon en dotter och gav henne namnet Dina.
\par 22 Men Gud tänkte på Rakel; Gud hörde henne och gjorde henne fruktsam.
\par 23 Hon blev havande och födde en son. Då sade hon: "Gud har tagit bort min smälek."
\par 24 Och hon gav honom namnet Josef, i det hon sade: "HERREN give mig ännu en son."
\par 25 Då nu Rakel hade fött Josef, sade Jakob till Laban: "Låt mig fara; jag vill draga hem till min ort och till mitt land.
\par 26 Giv mig mina hustrur och mina barn, som jag har tjänat dig för, och låt mig draga hem; du vet ju själv huru jag har tjänat dig."
\par 27 Laban svarade honom: "Låt mig finna nåd för dina ögon; jag vet genom hemliga tecken att HERREN för din skull har välsignat mig."
\par 28 Och han sade ytterligare: "Bestäm vad du vill hava i lön av mig, så skall jag giva dig det."
\par 29 Han svarade honom: "Du vet själv huru jag har tjänat dig, och vad det har blivit av din boskap under min vård.
\par 30 Ty helt litet var det som du hade, förrän jag kom, men det har förökat sig och blivit mycket, ty HERREN har välsignat dig, varhelst jag har gått fram. Men när skall jag nu också få göra något för mitt eget hus?"
\par 31 Han svarade: "Vad skall jag giva dig?" Och Jakob sade: "Du skall icke alls giva mig något. Om du vill göra mot mig såsom jag nu säger, så skall jag fortfara att vara herde för din hjord och vakta den.
\par 32 Jag vill i dag gå igenom hela din hjord och avskilja ur den alla spräckliga och brokiga såväl som alla svarta djur bland fåren, så ock vad som är brokigt och spräckligt bland getterna; sådant må sedan bliva min lön.
\par 33 Och när du framdeles kommer för att med egna ögon se vad som har blivit min lön, då skall min rättfärdighet vara mitt vittne; alla getter hos mig, som icke äro spräckliga eller brokiga, och alla får hos mig, som icke äro svarta, de skola räknas såsom stulna."
\par 34 Då sade Laban: "Välan, blive det såsom du har sagt."
\par 35 Och samma dag avskilde han de strimmiga och brokiga bockarna och alla spräckliga och brokiga getter - alla djur som något vitt fanns på - och alla svarta djur bland fåren; och detta lämnade han i sina söners vård.
\par 36 Och han lät ett avstånd av tre dagsresor vara mellan sig och Jakob. Och Jakob fick Labans övriga hjord att vakta.
\par 37 Men Jakob tog sig friska käppar av poppel, mandelträd och lönn och skalade på dem vita ränder, i det han blottade det vita på käpparna.
\par 38 Sedan lade han käpparna, som han hade skalat, i rännorna eller vattenhoarna dit hjordarna kommo för att dricka, så att djuren hade dem framför sig; och de hade just sin parningstid, när de nu kommo för att dricka.
\par 39 Och djuren parade sig vid käpparna, och så blev djurens avföda strimmig, spräcklig och brokig.
\par 40 Därefter avskilde Jakob lammen och ordnade djuren så, att de vände huvudena mot det som var strimmigt och mot allt som var svart i Labans hjord; så skaffade han sig egna hjordar, som han icke lät komma ihop med Labans hjord.
\par 41 Och så ofta de kraftigare djuren skulle para sig, lade Jakob käpparna framför djurens ögon i rännorna, så att de parade sig vid käpparna.
\par 42 Men när det var de svagare djuren, lade han icke dit dem. Härigenom tillföllo de svaga Laban och de kraftiga Jakob.
\par 43 Så blev mannen övermåttan rik; han fick mycken småboskap, därtill ock tjänarinnor och tjänare, kameler och åsnor.

\chapter{31}

\par 1 Men han fick höra huru Labans söner talade så: "Jakob har tagit allt vad vår fader ägde; av det vår fader ägde är det som han har skaffat sig all denna rikedom."
\par 2 Jakob märkte också att Laban icke såg på honom med samma ögon som förut.
\par 3 Och HERREN sade till Jakob: "Vänd tillbaka till dina fäders land och till din släkt; jag skall vara med dig."
\par 4 Då sände Jakob och lät kalla Rakel och Lea ut på marken till sin hjord;
\par 5 och han sade till dem: "Jag märker att eder fader icke ser på mig med samma ögon som förut, nu då min faders Gud har varit med mig.
\par 6 Och I veten själva att jag har tjänat eder fader av alla mina krafter;
\par 7 men eder fader har handlat svikligt mot mig och tio gånger förändrat min lön. Dock har Gud icke tillstatt honom att göra mig något ont.
\par 8 När han sade: 'De spräckliga skola vara din lön', då fick hela hjorden spräcklig avföda; och när han sade: 'De strimmiga skola vara din lön', då fick hela jorden strimmig avföda.
\par 9 Så tog Gud eder faders boskap och gav den åt mig.
\par 10 Ty när parningstiden kom, lyfte jag upp mina ögon och fick se i drömmen att hannarna som betäckte småboskapen voro strimmiga, spräckliga och fläckiga.
\par 11 Och Guds ängel sade till mig i drömmen: 'Jakob!' Jag svarade: 'Här är jag.'
\par 12 Då sade han: 'Lyft upp dina ögon och se huru alla hannar som betäcka småboskapen äro strimmiga, spräckliga och fläckiga. Jag har ju sett allt vad Laban gör mot dig.
\par 13 Jag är den Gud som du såg i Betel, där du smorde en stod, och där du gjorde mig ett löfte. Stå nu upp och drag ut ur detta land, och vänd tillbaka till ditt fädernesland.'"
\par 14 Då svarade Rakel och Lea och sade till honom: "Hava vi numera någon lott eller arvedel i vår faders hus?
\par 15 Blevo vi icke av honom aktade såsom främlingar, när han sålde oss? Sedan har han ju ock förtärt vad han fick i betalning för oss.
\par 16 Ja, all den rikedom som Gud har avhänt vår fader tillhör oss och våra barn. Så gör nu allt vad Gud har sagt dig."
\par 17 Då stod Jakob upp och satte sina barn och sina hustrur på kamelerna
\par 18 och förde bort med sig all boskap och alla ägodelar som han hade förvärvat, den boskap han ägde, och som han hade förvärvat i Paddan-Aram, och begav sig på väg till sin fader Isak i Kanaans land.
\par 19 Men Laban hade gått bort för att klippa sina får; då stal Rakel sin faders husgudar,
\par 20 och Jakob stal sig undan från araméen Laban, så att han icke lät denne märka att han ämnade fly.
\par 21 Så flydde han med allt sitt; han bröt upp och gick över floden och ställde sin färd mot Gileads berg.
\par 22 Men på tredje dagen fick Laban veta att Jakob hade flytt.
\par 23 Då tog han med sig sina fränder och satte efter honom sju dagsresor och hann upp honom på Gileads berg.
\par 24 Men Gud kom till araméen Laban i en dröm om natten och sade till honom: "Tag dig till vara för att tala något mot Jakob, vad det vara må."
\par 25 Och Laban hann upp Jakob. Denne hade då slagit upp sitt tält på berget, och Laban med sina fränder hade ock sitt tält uppslaget på Gileads berg.
\par 26 Då sade Laban till Jakob: "Vad är detta för ett tilltag, att du har stulit dig undan från mig och fört bort mina döttrar, likasom vore de tagna med svärd?
\par 27 Varför dolde du din flykt och stal dig undan från mig? Därigenom att du icke lät mig veta något därom hindrades jag att ledsaga dig till vägs med jubel och sång, med pukor och harpor.
\par 28 Du förunnade mig icke ens att kyssa mina barnbarn och mina döttrar. Du har handlat dåraktigt.
\par 29 Det stode nu i min makt att göra eder ont; men eder faders Gud sade till mig i natt: 'Tag dig till vara för att tala något mot Jakob, vad det vara må.'
\par 30 Och då du nu äntligen ville fara, eftersom du längtade så mycket till din faders hus, varför skulle du stjäla mina gudar?"
\par 31 Då svarade Jakob och sade till Laban: "Jag fruktade för dig, ty jag tänkte att du skulle med våld taga dina döttrar ifrån mig.
\par 32 Men den som du finner dina gudar hos, han skall icke få behålla livet. I våra fränders närvaro må du se efter, om något är ditt av det jag har i min ägo, och i så fall taga det." Ty Jakob visste icke att Rakel hade stulit dem.
\par 33 Då gick Laban in i Jakobs tält, därefter i Leas tält och i de båda tjänstekvinnornas tält, men fann intet. Och när han hade kommit ut ur Leas tält, gick han in i Rakels tält.
\par 34 Men Rakel hade tagit husgudarna och lagt dem i kamelsadeln och satt sig därovanpå. Och Laban sökte igenom hela tältet, men fann dem icke.
\par 35 Och hon sade till sin fader: "Vredgas icke, min herre, över att jag ej kan stiga upp för dig, ty det är med mig på kvinnors vis." Så sökte han efter husgudarna, men fann dem icke.
\par 36 Då blev Jakob vred och for ut mot Laban; Jakob tog till orda och sade till Laban: "Vari har jag då förbrutit mig eller syndat, eftersom du så häftigt förföljer mig?
\par 37 Nu har du genomsökt allt mitt bohag; vad har du där funnit av bohagsting som tillhöra dig? Lägg det fram här inför mina fränder och dina fränder, så att de få döma mellan oss båda.
\par 38 I tjugu år har jag nu varit hos dig; dina tackor och dina getter hava icke fött i otid, och av vädurarna i din hjord har jag icke ätit.
\par 39 Intet ihjälrivet djur förde jag till dig; jag måste själv ersätta det; du utkrävde det av mig, evad det var stulet om dagen eller stulet om natten.
\par 40 Sådan var min lott: om dagen förtärdes jag av hetta och om natten av köld, och sömnen flydde mina ögon.
\par 41 I tjugu år har jag nu varit i ditt hus; jag har tjänat dig i fjorton år för dina båda döttrar och i sex år för din boskap, men du har tio gånger förändrat min lön.
\par 42 Om icke min faders Gud hade varit med mig, Abrahams Gud, han som ock Isak fruktar, så hade du nu säkert låtit mig fara med tomma händer. Men Gud såg mitt lidande och min möda, och han fällde domen i natt."
\par 43 Då svarade Laban och sade till Jakob: "Döttrarna äro mina döttrar, och barnen äro mina barn, och hjordarna äro mina hjordar, och allt det du ser är mitt; vad skulle jag då nu kunna göra mot dessa mina döttrar eller mot barnen som de hava fött?
\par 44 Så kom nu och låt oss sluta ett förbund med varandra, och må det vara ett vittne mellan mig och dig."
\par 45 Då tog Jakob en sten och reste den till en stod.
\par 46 Och Jakob sade till sina fränder: "Samlen tillhopa stenar." Och de togo stenar och gjorde ett röse och höllo måltid där på röset.
\par 47 Och Laban kallade det Jegar-Sahaduta, men Jakob kallade det Galed.
\par 48 Och Laban sade: "Detta röse vare i dag vittne mellan mig och dig." Därav fick det namnet Galed;
\par 49 men det kallades ock Mispa, ty han sade: "HERREN vare väktare mellan mig och dig, när vi icke mer se varandra.
\par 50 Om du behandlar mina döttrar illa eller tager andra hustrur jämte mina döttrar, så vet, att om ock ingen människa är tillstädes, så är dock Gud vittne mellan mig och dig."
\par 51 Och Laban sade ytterligare till Jakob: "Se, detta röse och stoden som jag har rest mellan mig och dig -
\par 52 detta röse vare ett vittne, och stoden vare ett vittne, att jag icke skall draga till dig förbi detta röse, och att icke heller du skall draga till mig förbi detta röse och denna stod, med ont uppsåt.
\par 53 Abrahams Gud och Nahors Gud, han som var deras faders Gud, han vare domare mellan oss." Och Jakob svor eden vid honom som hans fader Isak fruktade.
\par 54 Och Jakob offrade ett slaktoffer på berget och inbjöd sina fränder att hålla måltid med sig. Och de åto och stannade sedan på berget över natten.
\par 55 Men om morgonen stod Laban bittida upp, och sedan han hade kysst sina barnbarn och sina döttrar och välsignat dem, for han sin väg hem igen.

\chapter{32}

\par 1 Men när Jakob drog sin väg fram, mötte honom Guds änglar;
\par 2 och då Jakob såg dem, sade han: "Detta är Guds skara." Och han gav den platsen namnet Mahanaim.
\par 3 Och Jakob sände budbärare framför sig till sin broder Esau i Seirs land, på Edoms mark;
\par 4 och han bjöd dem och sade: "Så skolen I säga till min herre Esau: Din tjänare Jakob låter säga: Jag har vistats borta hos Laban och dröjt kvar där ända till nu;
\par 5 och jag har fått oxar, åsnor, får, tjänare och tjänarinnor. Och jag har nu velat sända bud för att låta min herre veta detta, på det att jag må finna nåd för dina ögon."
\par 6 När sedan budbärarna kommo tillbaka till Jakob, sade de: "Vi träffade din broder Esau, som redan drager emot dig med fyra hundra man."
\par 7 Då blev Jakob mycket förskräckt och betogs av ångest; och han delade sitt folk och fåren och fäkreaturen och kamelerna i två skaror.
\par 8 Ty han tänkte: "Om Esau överfaller den ena skaran och slår den, så kan dock den andra skaran undkomma."
\par 9 Och Jakob sade: "Min fader Abrahams Gud och min fader Isaks Gud, HERRE, du som sade till mig: 'Vänd tillbaka till ditt land och till din släkt, så skall jag göra dig gott',
\par 10 jag är för ringa till all den nåd och all den trofasthet som du har bevisat din tjänare; ty jag hade icke mer än min stav, när jag gick över denna Jordan, och nu har jag förökats till två skaror.
\par 11 Rädda mig undan min broder Esaus hand, ty jag fruktar att han kommer och förgör mig, utan att ens skona mödrar och barn.
\par 12 Du har själv sagt: 'Jag skall göra dig mycket gott och låta din säd bliva såsom havets sand, som man icke kan räkna för dess myckenhets skull.'"
\par 13 Och han stannade där den natten. Och av det han hade förvärvat tog han ut till skänker åt sin broder Esau
\par 14 två hundra getter och tjugu bockar, två hundra tackor och tjugu vädurar,
\par 15 trettio kamelston som gåvo di, jämte deras föl, därtill fyrtio kor och tio tjurar samt tjugu åsninnor med tio föl.
\par 16 Och han lämnade detta i sina tjänares vård, var hjord för sig, och sade till sina tjänare: "Gån framför mig och låten ett mellanrum vara mellan hjordarna."
\par 17 Och han bjöd den förste och sade: "När min broder Esau möter dig och frågar dig: 'Vem tillhör du, och vart går du, och vem tillhöra djuren som du driver framför dig?',
\par 18 då skall du svara: 'De tillhöra din tjänare Jakob; de äro skänker som han sänder till min herre Esau, och själv kommer han här efter oss.'"
\par 19 Och han bjöd likaledes den andre och den tredje och alla de övriga som drevo hjordarna: "Såsom jag nu har sagt eder skolen I säga till Esau, när I kommen fram till honom.
\par 20 Och I skolen vidare säga: 'Också din tjänare Jakob kommer här efter oss.'" Ty han tänkte: "Jag vill blidka honom med de skänker som gå före mig; sedan vill jag själv komma inför hans ansikte; kanhända tager han då nådigt emot mig."
\par 21 Så kommo nu skänkerna före honom, medan han själv den natten stannade i lägret.
\par 22 Men under natten stod han upp och tog sina båda hustrur och sina båda tjänstekvinnor och sina elva söner och gick över Jabboks vad.
\par 23 Han tog dem och förde dem över bäcken och förde tillika över vad han eljest ägde.
\par 24 Och Jakob blev ensam kvar. Då brottades en man med honom, till dess morgonrodnaden gick upp.
\par 25 Och när denne såg att han icke kunde övervinna Jakob, gav han honom ett slag på höftleden, så att höften gick ur led, under det han brottades med honom.
\par 26 Och mannen sade: "Släpp mig, ty morgonrodnaden går upp." Men han svarade: "Jag släpper dig icke, med mindre du välsignar mig."
\par 27 Då sade han till honom: "Vad är ditt namn?" Han svarade: "Jakob."
\par 28 Han sade: "Du skall icke mer heta Jakob, utan Israel, ty du har kämpat med Gud och med människor och vunnit seger."
\par 29 Då frågade Jakob och sade: "Låt mig veta ditt namn." Han svarade: "Varför frågar du efter mitt namn?" Och han välsignade honom där.
\par 30 Men Jakob gav platsen namnet Peniel, "ty", sade han, "jag har sett Gud ansikte mot ansikte, och dock har mitt liv blivit räddat".
\par 31 Och när han hade kommit förbi Penuel, såg han solen gå upp; men han haltade på höften.
\par 32 Fördenskull äta Israels barn ännu i dag icke höftsenan som ligger på höftleden, därför nämligen, att han gav Jakob ett slag på höftleden, på höftsenan.

\chapter{33}

\par 1 Och Jakob lyfte upp sina ögon och fick se Esau komma med fyra hundra man. Då fördelade han sina barn på Lea och Rakel och de båda tjänstekvinnorna.
\par 2 Och han lät tjänstekvinnorna med deras barn gå främst, Lea med hennes barn därnäst, och Rakel med Josef sist.
\par 3 Och själv gick han framför dem och bugade sig sju gånger ned till jorden, till dess han kom fram till sin broder.
\par 4 Men Esau skyndade emot honom och tog honom i famn och föll honom om halsen och kysste honom; och de gräto.
\par 5 Och när han lyfte upp sina ögon och fick se kvinnorna och barnen, sade han: "Vilka äro dessa som du har med dig?" Han svarade: "Det är barnen som Gud har beskärt din tjänare."
\par 6 Och tjänstekvinnorna gingo fram med sina barn och bugade sig.
\par 7 Därefter gick ock Lea fram med sina barn, och de bugade sig. Slutligen gingo Josef och Rakel fram och bugade sig.
\par 8 Sedan frågade han: "Vad ville du med hela den skara som jag mötte?" Han svarade: "Jag ville finna nåd för min herres ögon."
\par 9 Men Esau sade: "Jag har nog; behåll du vad du har, min broder."
\par 10 Jakob svarade: "Ack nej; om jag har funnit nåd för dina ögon, så tag emot skänkerna av mig, eftersom jag har fått se ditt ansikte, likasom såge jag ett gudaväsens ansikte, då du nu så gunstigt har tagit emot mig.
\par 11 Tag hälsningsskänkerna som jag har skickat emot dig; ty Gud har varit mig nådig, och jag har allt fullt upp." Och han bad honom så enträget, att han tog emot dem.
\par 12 Och Esau sade: "Låt oss bryta upp och draga vidare; jag vill gå framför dig."
\par 13 Men han svarade honom: "Min herre ser själv att barnen äro späda, och att jag har med mig får och kor som giva di; driver man dessa för starkt en enda dag, så dör hela hjorden.
\par 14 Må därför min herre draga åstad före sin tjänare, så vill jag komma efter i sakta mak, i den mån boskapen, som drives framför mig, och barnen orka följa med, till dess jag kommer till min herre i Seir."
\par 15 Då sade Esau: "Så vill jag åtminstone lämna kvar hos dig en del av mitt folk." Men han svarade: "Varför så? Må jag allenast finna nåd för min herres ögon."
\par 16 Så vände Esau om, samma dag, och tog vägen till Seir.
\par 17 Men Jakob bröt upp och drog till Suckot och byggde sig där ett hus. Och åt sin boskap gjorde han lövhyddor; därav fick platsen namnet Suckot.
\par 18 Och Jakob kom på sin färd ifrån Paddan-Aram välbehållen till Sikems stad i Kanaans land och slog upp sitt läger utanför staden.
\par 19 Och det jordstycke där han hade slagit upp sitt tält köpte han av Hamors, Sikems faders, barn för hundra kesitor.
\par 20 Och han reste där ett altare och kallade det El-Elohe-Israel.

\chapter{34}

\par 1 Men Dina, den dotter som Lea hade fött åt Jakob, gick ut för att besöka landets döttrar.
\par 2 Och Sikem, som var son till hivéen Hamor, hövdingen i landet, fick se henne, och han tog henne till sig och lägrade henne och kränkte henne.
\par 3 Och hans hjärta fäste sig vid Dina, Jakobs dotter, och flickan blev honom kär, och han talade vänligt med flickan.
\par 4 Och Sikem sade till sin fader Hamor: "Skaffa mig denna flicka till hustru."
\par 5 Och Jakob hade fått höra att hans dotter Dina hade blivit skändad. Men eftersom hans söner voro med hans boskap ute på marken, teg Jakob, till dess de kommo hem.
\par 6 Så gick nu Hamor, Sikems fader, ut till Jakob för att tala med honom.
\par 7 Men när Jakobs söner kommo hem från marken, sedan de hade fått höra vad som hade hänt, blevo de förbittrade och vredgades högeligen över att han hade gjort vad som var en galenskap i Israel, i det han hade lägrat Jakobs dotter - en otillbörlig gärning.
\par 8 Då talade Hamor med dem och sade: "Min son Sikems hjärta har fäst sig vid eder syster; given henne åt honom till hustru.
\par 9 Och befrynden eder med oss; given edra döttrar åt oss, och tagen I våra döttrar till hustrur,
\par 10 och bosätten eder hos oss, ty landet skall ligga öppet för eder; där mån I bo och draga omkring och förvärva besittningar."
\par 11 Och Sikem sade till hennes fader och hennes bröder: "Låten mig finna nåd för edra ögon; vad I fordren av mig vill jag giva.
\par 12 Begären av mig huru stor brudgåva och skänk som helst; jag vill giva vad I fordren av mig; given mig allenast flickan till hustru."
\par 13 Då svarade Jakobs söner Sikem och hans fader Hamor med listiga ord, eftersom han hade skändat deras syster Dina,
\par 14 och sade till dem: "Vi kunna icke samtycka till att giva vår syster åt en man som har förhud; ty sådant hålla vi för skamligt.
\par 15 Allenast på det villkoret skola vi göra eder till viljes, att I bliven såsom vi, därigenom att allt mankön bland eder omskäres.
\par 16 Då skola vi giva våra döttrar åt eder och själva taga edra döttrar till hustrur; och vi skola då bo hos eder och bliva med eder ett enda folk.
\par 17 Men om I icke viljen lyssna till oss och låta omskära eder, så skola vi taga vår syster och draga bort."
\par 18 Och Hamor och Sikem, Hamors son, voro till freds med vad de begärde.
\par 19 Och den unge mannen dröjde icke att göra så, ty han hade fått behag till Jakobs dotter. Och han hade större myndighet än någon annan i hans faders hus.
\par 20 Så trädde då Hamor och hans son Sikem upp i sin stads port och talade till männen i staden och sade:
\par 21 "Dessa män äro fredligt sinnade mot oss; må vi alltså låta dem bo i landet och draga omkring där; landet har ju utrymme nog för dem. Vi vilja taga deras döttrar till hustrur åt oss och giva dem våra döttrar.
\par 22 Men allenast på det villkoret skola männen göra oss till viljes och bo hos oss och bliva ett enda folk med oss, att allt mankön bland oss omskäres, likasom de själva äro omskurna.
\par 23 Och då bliva ju deras boskap och deras egendom och alla deras dragare vår tillhörighet. Må vi fördenskull allenast göra dem till viljes, så skola de bo kvar hos oss."
\par 24 Och folket lydde Hamor och hans son Sikem, alla de som bodde inom hans stadsport; allt mankön, så många som bodde inom hans stadsport, läto omskära sig.
\par 25 Men på tredje dagen, då de voro sjuka av såren, togo Jakobs två söner Simeon och Levi, Dinas bröder, var sitt svärd och överföllo staden oförtänkt och dräpte allt mankön.
\par 26 Också Hamor och hans son Sikem dräpte de med svärdsegg och togo Dina ut ur Sikems hus och gingo sin väg.
\par 27 Och Jakobs söner kommo över de slagna och plundrade staden, därför att deras syster hade blivit skändad;
\par 28 de togo deras får och fäkreatur och åsnor, både vad som fanns i staden och vad som fanns på fältet.
\par 29 Och allt deras gods och alla deras barn och deras kvinnor förde de bort såsom byte, tillika med allt annat som fanns i husen.
\par 30 Men Jakob sade till Simeon och Levi: "I haven dragit olycka över mig, då I nu haven gjort mig förhatlig för landets inbyggare, kananéerna och perisséerna. Mitt folk är allenast en ringa hop; man skall nu församla sig mot mig och slå mig ihjäl; så skall jag med mitt hus förgöras."
\par 31 Men de svarade: "Skulle man då få behandla vår syster såsom en sköka?"

\chapter{35}

\par 1 Och Gud sade till Jakob: "Stå upp, drag till Betel och stanna där, och res där ett altare åt den Gud som uppenbarade sig för dig, när du flydde för din broder Esau."
\par 2 Då sade Jakob till sitt husfolk och till alla som voro med honom: "Skaffen bort de främmande gudar som I haven bland eder, och renen eder och byten om kläder,
\par 3 och låt oss så stå upp och draga till Betel; där vill jag resa ett altare åt den Gud som bönhörde mig, när jag var i nöd, och som var med mig på den väg jag vandrade."
\par 4 Då gåvo de åt Jakob alla de främmande gudar som de hade hos sig, därtill ock sina örringar; och Jakob grävde ned detta under terebinten vid Sikem.
\par 5 Sedan bröto de upp; och en förskräckelse ifrån Gud kom över de kringliggande städerna, så att man icke förföljde Jakobs söner.
\par 6 Och Jakob kom till Lus, det är Betel, i Kanaans land, jämte allt det folk som var med honom.
\par 7 Och han byggde där ett altare och kallade platsen El-Betel, därför att Gud där hade uppenbarat sig för honom, när han flydde för sin broder.
\par 8 Och Debora, Rebeckas amma, dog och blev begraven nedanför Betel, under en ek; den fick namnet Gråtoeken.
\par 9 Och Gud uppenbarade sig åter för Jakob, när han hade kommit tillbaka från Paddan-Aram, och välsignade honom.
\par 10 Och Gud sade till honom: "Ditt namn är Jakob; men du skall icke mer heta Jakob, utan Israel skall vara ditt namn." Så fick han namnet Israel.
\par 11 Och Gud sade till honom: "Jag är Gud den Allsmäktige; var fruktsam och föröka dig. Ett folk, ja, skaror av folk skola komma av dig, och konungar skola utgå från din länd.
\par 12 Och det land som jag har givit åt Abraham och Isak skall jag giva åt dig; åt din säd efter dig skall jag ock giva det landet.
\par 13 Och Gud for upp från honom, på den plats där han hade talat med honom.
\par 14 Men Jakob reste en stod på den plats där han hade talat med honom, en stod av sten; och han offrade drickoffer därpå och göt olja över den.
\par 15 Och Jakob gav åt platsen där Gud hade talat med honom namnet Betel.
\par 16 Sedan bröto de upp från Betel. Och när det ännu var ett stycke väg fram till Efrat, kom Rakel i barnsnöd, och barnsnöden blev henne svår.
\par 17 Då nu hennes barnsnöd var som svårast, sade hjälpkvinnan till henne: "Frukta icke; ty också denna gång får du en son."
\par 18 Men när hon höll på att giva upp andan, ty hon skulle nu dö, gav hon honom namnet Ben-Oni; men hans fader kallade honom Benjamin.
\par 19 Så dog Rakel, och hon blev begraven vid vägen till Efrat, det är Bet-Lehem.
\par 20 Och Jakob reste en vård på hennes grav; det är den som ännu i dag kallas Rakels gravvård.
\par 21 Och Israel bröt upp därifrån och slog upp sitt tält på andra sidan om Herdetornet.
\par 22 Och medan Israel bodde där i landet, gick Ruben åstad och lägrade Bilha, sin faders bihustru; och Israel fick höra det. Och Jakob hade tolv söner.
\par 23 Leas söner voro Ruben, Jakobs förstfödde, vidare Simeon, Levi, Juda, Isaskar och Sebulon.
\par 24 Rakels söner voro Josef och Benjamin.
\par 25 Bilhas, Rakels tjänstekvinnas, söner voro Dan och Naftali.
\par 26 Silpas, Leas tjänstekvinnas, söner voro Gad och Aser. Dessa voro Jakobs söner, och de föddes åt honom i Paddan-Aram.
\par 27 Och Jakob kom till sin fader Isak i Mamre vid Kirjat-Arba, det är Hebron, där Abraham och Isak hade bott såsom främlingar.
\par 28 Och Isak levde ett hundra åttio år;
\par 29 därefter gav Isak upp andan och dog och blev samlad till sina fäder, gammal och mätt på att leva. Och hans söner Esau och Jakob begrovo honom.

\chapter{36}

\par 1 Detta är berättelsen om Esaus, det är Edoms, släkt.
\par 2 Esau hade tagit sina hustrur bland Kanaans döttrar: Ada, hetiten Elons dotter, och Oholibama, dotter till Ana och sondotter till hivéen Sibeon,
\par 3 så ock Basemat, Ismaels dotter, Nebajots syster.
\par 4 Och Ada födde Elifas åt Esau, men Basemat födde Reguel.
\par 5 Och Oholibama födde Jeus, Jaelam och Kora. Dessa voro Esaus söner, vilka föddes åt honom i Kanaans land.
\par 6 Och Esau tog sina hustrur, sina söner och döttrar och allt sitt husfolk, sin boskap och alla sina dragare och all annan egendom som han hade förvärvat i Kanaans land och drog till ett annat land och skilde sig så från sin broder Jakob.
\par 7 Ty deras ägodelar voro så stora att de icke kunde bo tillsammans; landet där de uppehöllo sig räckte icke till åt dem, för deras boskapshjordars skull.
\par 8 Och Esau bosatte sig i Seirs bergsbygd. Esau, det är densamme som Edom.
\par 9 Och detta är berättelsen om Esaus släkt, hans som var stamfader för edoméerna, i Seirs bergsbygd.
\par 10 Dessa äro namnen på Esaus söner: Elifas, son till Ada, Esaus hustru, och Reguel, son till Basemat, Esaus hustru.
\par 11 Men Elifas' söner voro Teman, Omar, Sefo, Gaetam och Kenas.
\par 12 Och Timna, som var Elifas', Esaus sons, bihustru, födde Amalek åt Elifas. Dessa voro söner till Ada, Esaus hustru.
\par 13 Men Reguels söner voro dessa: Nahat och Sera, Samma och Missa. Dessa voro söner till Basemat, Esaus hustru.
\par 14 Men söner till Oholibama, Esaus hustru, dotter till Ana och sondotter till Sibeon, voro dessa, som hon födde åt Esau: Jeus, Jaelam och Kora.
\par 15 Dessa voro stamfurstarna bland Esaus söner: Elifas', Esaus förstföddes, söner voro dessa: fursten Teman, fursten Omar, fursten Sefo, fursten Kenas,
\par 16 fursten Kora, fursten Gaetam, fursten Amalek. Dessa voro de furstar som härstammade från Elifas, i Edoms land; dessa voro Adas söner.
\par 17 Och dessa voro Reguels, Esaus sons, söner: fursten Nahat, fursten Sera, fursten Samma, fursten Missa. Dessa voro de furstar som härstammade från Reguel, i Edoms land; dessa voro söner till Basemat, Esaus hustru.
\par 18 Och dessa voro Oholibamas, Esaus hustrus, söner: fursten Jeus, fursten Jaelam, fursten Kora. Dessa voro de furstar som härstammade från Oholibama, Anas dotter och Esaus hustru.
\par 19 Dessa voro Esaus söner, och dessa deras stamfurstar. Han är densamme som Edom.
\par 20 Dessa voro horéen Seirs söner, landets förra inbyggare: Lotan, Sobal, Sibeon, Ana,
\par 21 Dison, Eser och Disan. Dessa voro horéernas, Seirs söners, stamfurstar i Edoms land.
\par 22 Men Lotans söner voro Hori och Hemam; och Lotans syster var Timna.
\par 23 Och dessa voro Sobals söner: Alvan, Manahat och Ebal, Sefo och Onam.
\par 24 Och dessa voro Sibeons söner: Aja och Ana; det var denne Ana som fann de varma källorna i öknen, när han vaktade sin fader Sibeons åsnor.
\par 25 Men dessa voro Anas barn: Dison och Oholibama, Anas dotter.
\par 26 Och dessa voro Disans söner: Hemdan, Esban, Jitran och Keran.
\par 27 Och dessa voro Esers söner: Bilhan, Saavan och Akan.
\par 28 Dessa voro Disans söner: Us och Aran.
\par 29 Dessa voro horéernas stamfurstar: fursten Lotan, fursten Sobal, fursten Sibeon, fursten Ana,
\par 30 fursten Dison, fursten Eser, fursten Disan. Dessa voro horéernas stamfurstar i Seirs land, var furste för sig.
\par 31 Och dessa voro de konungar som regerade i Edoms land, innan ännu någon israelitisk konung var konung där:
\par 32 Bela, Beors son, var konung i Edom, och hans stad hette Dinhaba.
\par 33 När Bela dog, blev Jobab, Seras son, från Bosra, konung efter honom.
\par 34 När Jobab dog, blev Husam från temanéernas land konung efter honom.
\par 35 När Husam dog, blev Hadad, Bedads son, konung efter honom, han som slog midjaniterna på Moabs mark; och hans stad hette Avit.
\par 36 När Hadad dog, blev Samla från Masreka konung efter honom.
\par 37 När Samla dog, blev Saul från Rehobot vid floden konung efter honom.
\par 38 När Saul dog, blev Baal-Hanan, Akbors son, konung efter honom.
\par 39 När Baal-Hanan, Akbors son, dog, blev Hadar konung efter honom; och hans stad hette Pagu, och hans hustru hette Mehetabel, dotter till Matred, som var dotter till Me-Sahab.
\par 40 Och dessa äro namnen på Esaus stamfurstar, efter deras släkter och orter, med deras namn: fursten Timna, fursten Alva, fursten Jetet,
\par 41 fursten Oholibama, fursten Ela, fursten Pinon,
\par 42 fursten Kenas, fursten Teman, fursten Mibsar,
\par 43 fursten Magdiel, fursten Iram. Dessa voro Edoms stamfurstar, efter deras boningsorter i det land de hade tagit i besittning - hans som ock kallas Esau, edoméernas stamfader.

\chapter{37}

\par 1 Men Jakob bosatte sig i det land där hans fader hade bott såsom främling, nämligen i Kanaans land.
\par 2 Detta är berättelsen om Jakobs släkt. När Josef var sjutton år gammal, gick han, jämte sina bröder, i vall med fåren; han följde då såsom yngling med Bilhas och Silpas, sin faders hustrurs, söner. Och Josef bar fram till deras fader vad ont som sades om dem.
\par 3 Men Israel hade Josef kärare än alla sina andra söner, eftersom han hade fött honom på sin ålderdom; och han lät göra åt honom en fotsid livklädnad.
\par 4 Då nu hans bröder sågo att deras fader hade honom kärare än alla hans bröder, blevo de hätska mot honom och kunde icke tala vänligt till honom.
\par 5 Därtill hade Josef en gång en dröm, som han omtalade för sina bröder; sedan hatade de honom ännu mer.
\par 6 Han sade nämligen till dem: "Hören vilken dröm jag har haft.
\par 7 Jag tyckte att vi bundo kärvar på fältet; och se, min kärve reste sig upp och blev stående, och edra kärvar ställde sig runt omkring och bugade sig för min kärve."
\par 8 Då sade hans bröder till honom: "Skulle du bliva vår konung, och skulle du råda över oss?" Och de hatade honom ännu mer för hans drömmars skull och för vad han hade sagt.
\par 9 Sedan hade han ännu en annan dröm som han förtäljde för sina bröder; han sade: "Hören, jag har haft ännu en dröm. Jag tyckte att solen och månen och elva stjärnor bugade sig för mig."
\par 10 När han förtäljde detta för sin fader och sina bröder, bannade hans fader honom och sade till honom: "Vad är detta för en dröm som du har haft? Skulle då jag och din moder och dina bröder komma och buga oss ned till jorden för dig?"
\par 11 Och hans bröder avundades honom; men hans fader bevarade detta i sitt minne.
\par 12 Då nu en gång hans bröder hade gått bort för att vakta sin faders får i Sikem,
\par 13 sade Israel till Josef: "Se, dina bröder vakta fåren i Sikem; gör dig redo, jag vill sända dig till dem." Han svarade honom: "Jag är redo."
\par 14 Då sade han till honom: "Gå och se efter, om det står väl till med dina bröder, och om det står väl till med fåren, och kom tillbaka till mig med svar." Så sände han honom åstad från Hebrons dal, och han kom till Sikem.
\par 15 Där mötte han en man, medan han gick omkring villrådig på fältet; och mannen frågade honom: "Vad söker du?"
\par 16 Han svarade: "Jag söker efter mina bröder; säg mig var de vakta sin hjord."
\par 17 Mannen svarade: "De hava brutit upp härifrån; ty jag hörde dem säga: 'Låt oss gå till Dotain.'" Då gick Josef vidare efter sina bröder och fann dem i Dotan.
\par 18 När de nu på avstånd fingo se honom, innan han ännu hade hunnit fram till dem, lade de råd om att döda honom.
\par 19 De sade till varandra: "Se, där kommer drömmaren.
\par 20 Upp, låt oss dräpa honom och kasta honom i en brunn; sedan kunna vi säga att ett vilddjur har ätit upp honom. Så få vi se huru det går med hans drömmar."
\par 21 Men när Ruben hörde detta, ville han rädda honom undan deras händer och sade: "Låt oss icke slå ihjäl honom."
\par 22 Ytterligare sade Ruben till dem: "Utgjuten icke blod; kasten honom i brunnen här i öknen, men bären icke hand på honom." Han ville nämligen rädda honom undan deras händer och föra honom tillbaka till hans fader.
\par 23 Då nu Josef kom fram till sina bröder, togo de av honom hans livklädnad, den fotsida klädnaden som han hade på sig,
\par 24 och grepo honom och kastade honom i brunnen; men brunnen var tom, intet vatten fanns däri.
\par 25 Därefter satte de sig ned för att äta. När de då lyfte upp sina ögon, fingo de se ett tåg av ismaeliter komma från Gilead, och deras kameler voro lastade med dragantgummi, balsam och ladanum; de voro på väg med detta ned till Egypten.
\par 26 Då sade Juda till sina bröder: "Vad gagn hava vi därav att vi dräpa vår broder och dölja hans blod?"
\par 27 Nej, låt oss sälja honom till ismaeliterna; må vår hand icke komma vid honom, ty han är ju vår broder, vårt eget kött." Och hans bröder lydde honom.
\par 28 Då nu midjanitiska köpmän kommo där förbi, drogo de upp Josef ur brunnen; och de sålde Josef för tjugu siklar silver till ismaeliterna. Dessa förde så Josef till Egypten.
\par 29 När sedan Ruben kom tillbaka till brunnen, se, då fanns Josef icke i brunnen. Då rev han sönder sina kläder
\par 30 och vände tillbaka till sina bröder och sade: "Gossen är icke där, vart skall jag nu taga vägen?"
\par 31 Men de togo Josefs livklädnad och slaktade en bock och doppade klädnaden i blodet;
\par 32 därefter sände de den fotsida livklädnaden hem till sin fader och läto säga: "Denna har vi funnit; se efter, om det är din sons livklädnad eller icke."
\par 33 Och han kände igen den och sade: "Det är min sons livklädnad; ett vilddjur har ätit upp honom, förvisso är Josef ihjälriven."
\par 34 Och Jakob rev sönder sina kläder och svepte säcktyg om sina länder och sörjde sin son i lång tid.
\par 35 Och alla hans söner och alla hans döttrar kommo för att trösta honom; men han ville icke låta trösta sig, utan sade: "Jag skall med sorg fara ned i dödsriket till min son." Så begrät hans fader honom.
\par 36 Men medaniterna förde honom till Egypten och sålde honom till Potifar, som var hovman hos Farao och hövitsman för drabanterna.

\chapter{38}

\par 1 Vid den tiden begav sig Juda åstad bort ifrån sina bröder och slöt sig till en man i Adullam, som hette Hira.
\par 2 Där fick Juda se dottern till en kananeisk man som hette Sua, och han tog henne till sig och gick in till henne.
\par 3 Och hon blev havande och födde en son, och han fick namnet Er.
\par 4 Åter blev hon havande och födde en son och gav honom namnet Onan.
\par 5 Och hon födde ännu en son, och åt denne gav hon namnet Sela; och när han föddes, var Juda i Kesib.
\par 6 Och Juda tog åt Er, sin förstfödde, en hustru som hette Tamar.
\par 7 Men Er, Judas förstfödde, misshagade HERREN; därför dödade HERREN honom.
\par 8 Då sade Juda till Onan: "Gå in till din broders hustru, äkta henne i din broders ställe och skaffa avkomma åt din broder."
\par 9 Men eftersom Onan visste att avkomman icke skulle bliva hans egen, lät han, när han gick in till sin broders hustru, det spillas på jorden, för att icke giva avkomma åt sin broder.
\par 10 Men det misshagade HERREN att han gjorde så; därför dödade han också honom.
\par 11 Då sade Juda till sin sonhustru Tamar: "Stanna såsom änka i din faders hus, till dess min son Sela bliver fullvuxen." Han fruktade nämligen att annars också denne skulle dö, likasom hans bröder. Så gick Tamar bort och stannade i sin faders hus.
\par 12 En lång tid därefter dog Suas dotter, Judas hustru. Och efter sorgetidens slut gick Juda med sin vän adullamiten Hira upp till Timna, för att se efter dem som klippte hans får.
\par 13 När man nu berättade för Tamar att hennes svärfader gick upp till Timna för att klippa sina får,
\par 14 lade hon av sig sina änkekläder och betäckte sig med en slöja och höljde in sig och satte sig vid porten till Enaim på vägen till Timna. Ty hon såg, att fastän Sela var fullvuxen, blev hon likväl icke given åt honom till hustru.
\par 15 Då nu Juda fick se henne, trodde han att hon var en sköka; hon hade ju nämligen sitt ansikte betäckt.
\par 16 Och han vek av till henne, där hon satt vid vägen, och sade: "Kom, låt mig gå in till dig." Ty han visste icke att det var hans sonhustru. Hon svarade: "Vad vill du giva mig för att få gå in till mig?"
\par 17 Han sade: "Jag vill sända dig en killing ur min hjord." Hon svarade: "Ja, om du giver mig pant, till dess du sänder den."
\par 18 Han sade: "Vad skall jag då giva dig i pant?" Hon svarade: "Din signetring, din snodd och staven som du har i din hand." Då gav han henne detta och gick in till henne, och hon blev havande genom honom.
\par 19 Och hon stod upp och gick därifrån och lade av sin slöja och klädde sig åter i sina änkekläder.
\par 20 Och Juda sände killingen med sin vän adullamiten, för att få igen panten av kvinnan; men denne fann henne icke.
\par 21 Och han frågade folket där på orten och sade: "Var är tempeltärnan, hon som satt i Enaim vid vägen?" De svarade: "Här har ingen tempeltärna varit."
\par 22 Och han kom tillbaka till Juda och sade: "Jag har icke funnit henne; därtill säger folket på orten att ingen tempeltärna har varit där."
\par 23 Då sade Juda: "Må hon då behålla det, så att vi icke draga smälek över oss. Jag har nu sänt killingen, men du har icke funnit henne."
\par 24 Vid pass tre månader därefter blev så berättat för Juda: "Din sonhustru Tamar har bedrivit otukt, och i otukt har hon blivit havande." Juda sade: "Fören ut henne till att brännas."
\par 25 Men när hon fördes ut, sände hon bud till sin svärfader och lät säga: "Genom en man som är ägare till detta har jag blivit havande." Och hon lät säga: "Se efter, vem denna signetring, dessa snodder och denna stav tillhöra."
\par 26 Och Juda kände igen dem och sade: "Hon är i sin rätt mot mig, eftersom jag icke har givit henne åt min son Sela." Men han kom icke mer vid henne.
\par 27 När hon nu skulle föda, se, då funnos tvillingar i hennes liv.
\par 28 Och i födslostunden stack den ene fram en hand; då tog hjälpkvinnan en röd tråd och band den om hans hand och sade: "Denne kom först fram."
\par 29 Men när han därefter åter drog sin hand tillbaka, se, då kom hans broder fram; och hon sade: "Varför har du trängt dig fram?" Och han fick namnet Peres.
\par 30 Därefter kom hans broder fram, han som hade den röda tråden om sin hand, och han fick namnet Sera.

\chapter{39}

\par 1 Och Josef fördes ned till Egypten; och Potifar, som var hovman hos Farao och hövitsman för drabanterna, en egyptisk man, köpte honom av ismaeliterna som hade fört honom ditned.
\par 2 Och HERREN var med Josef, så att han blev en lyckosam man. Och han vistades i sin herres, egyptierns, hus;
\par 3 och hans herre såg att HERREN var med honom, ty allt vad han gjorde lät HERREN lyckas väl under hans hand.
\par 4 Och Josef fann nåd för hans ögon och fick betjäna honom. Och han satte honom över sitt hus, och allt vad han ägde lämnade han i hans vård.
\par 5 Och från den stund då han hade satt honom över sitt hus och över allt vad han ägde, välsignade HERREN egyptierns hus, för Josefs skull; och HERRENS välsignelse vilade över allt vad han ägde, hemma och på marken.
\par 6 Därför överlät han i Josefs vård allt vad han ägde, och sedan han hade fått honom till sin hjälp, bekymrade han sig icke om något, utom maten som han själv åt. Men Josef hade en skön gestalt och var skön att skåda.
\par 7 Och efter en tid hände sig att hans herres hustru kastade sina ögon på Josef och sade: "Ligg hos mig."
\par 8 Men han ville icke, utan sade till sin herres hustru: "Se, alltsedan min herre har tagit mig till sin hjälp, bekymrar han sig icke om något i huset, och allt vad han äger har han lämnat i min vård.
\par 9 Han har i detta hus icke större makt än jag, och intet annat har han förbehållit sig än dig allena, eftersom du är hans hustru. Huru skulle jag då kunna göra så mycket ont och synda mot Gud?"
\par 10 Och fastän hon talade sådant dag efter dag till Josef, hörde han dock icke på henne och ville icke ligga hos henne eller vara med henne.
\par 11 Men en dag då han kom in i huset för att förrätta sina sysslor, och ingen av husfolket var tillstädes därinne,
\par 12 fattade hon honom i manteln och sade: "Ligg hos mig." Men han lämnade manteln i hennes hand och flydde och kom ut.
\par 13 Då hon nu såg att han hade lämnat sin mantel i hennes hand och flytt ut,
\par 14 ropade hon på sitt husfolk och sade till dem: "Sen här, han har fört hit till oss en hebreisk man, för att denne skulle locka oss till lättfärdighet. Han kom in till mig och ville ligga hos mig; men jag ropade med hög röst.
\par 15 Och när han hörde att jag hov upp min röst och ropade, lämnade han sin mantel kvar hos mig och flydde och kom ut."
\par 16 Och hon lät hans mantel ligga kvar hos sig, till dess hans herre kom hem;
\par 17 då berättade hon för honom detsamma; hon sade: "Den hebreiske tjänaren som du har fört hit till oss kom in till mig, och ville locka mig till lättfärdighet.
\par 18 Men då jag hov upp min röst och ropade, lämnade han sin mantel kvar hos mig och flydde ut."
\par 19 När nu hans herre hörde vad hans hustru berättade för honom, nämligen att hans tjänare hade betett sig mot henne på detta sätt, blev hans vrede upptänd.
\par 20 Och Josefs herre tog honom och lät sätta honom i det fängelse där konungens fångar sutto fängslade; där fick han då vara i fängelse.
\par 21 Men HERREN var med Josef och förskaffade honom ynnest och lät honom finna nåd hos föreståndaren för fängelset.
\par 22 Och föreståndaren för fängelset lät alla fångar som sutto i fängelset stå under Josefs uppsikt; och allt vad där skulle göras, det gjordes genom honom.
\par 23 Föreståndaren för fängelset tog sig alls icke av något som Josef hade om hand, eftersom HERREN var med denne; och vad han gjorde, det lät HERREN lyckas väl.

\chapter{40}

\par 1 En tid härefter hände sig att den egyptiske konungens munskänk och hans bagare försyndade sig mot sin herre, konungen av Egypten.
\par 2 Och Farao blev förtörnad på sina två hovmän, överste munskänken och överste bagaren,
\par 3 och lät sätta dem i förvar i drabanthövitsmannens hus, i samma fängelse där Josef satt fången.
\par 4 Och hövitsmannen för drabanterna anställde Josef hos dem till att betjäna dem; och de sutto där i förvar en tid.
\par 5 Medan nu den egyptiske konungens munskänk och bagare sutto fångna i fängelset, hade de båda under samma natt var sin dröm, vardera med sin särskilda betydelse.
\par 6 Och när Josef om morgonen kom in till dem, fick han se att de voro bedrövade.
\par 7 Då frågade han Faraos hovmän, som med honom sutto i förvar i hans herres hus: "Varför sen I så sorgsna ut i dag?"
\par 8 De svarade honom: "Vi hava haft en dröm, och ingen finnes, som kan uttyda den." Josef sade till dem: "Att giva uttydningen är ju Guds sak; förtäljen drömmen för mig."
\par 9 Då förtäljde överste munskänken sin dröm för Josef och sade till honom: "Jag drömde att ett vinträd stod framför mig;
\par 10 på vinträdet voro tre rankor, och knappt hade det skjutit skott, så slogo dess blommor ut och dess klasar buro mogna druvor.
\par 11 Och jag hade Faraos bägare i min hand, och jag tog druvorna och pressade ut dem i Faraos bägare och gav Farao bägaren i handen."
\par 12 Då sade Josef till honom: "Detta är uttydningen: de tre rankorna betyda tre dagar;
\par 13 om tre dagar skall Farao upphöja ditt huvud och sätta dig åter på din plats, så att du får giva Farao bägaren i handen likasom förut, då du var hans munskänk.
\par 14 Men tänk på mig, när det går dig väl, så att du gör barmhärtighet med mig och nämner om mig för Farao och skaffar mig ut från detta hus;
\par 15 ty jag är med orätt bortförd från hebréernas land, och icke heller här har jag gjort något varför jag borde sättas i fängelse."
\par 16 Då nu överste bagaren såg att Josef hade givit en god uttydning, sade han till honom: "Också jag hade en dröm. Jag tyckte att jag bar tre vetebrödskorgar på mitt huvud.
\par 17 Och i den översta korgen funnos bakverk av alla slag, sådant som Farao plägar äta; men fåglarna åto därav ur korgen på mitt huvud."
\par 18 Då svarade Josef och sade: "Detta är uttydningen: de tre korgarna betyda tre dagar;
\par 19 om tre dagar skall Farao upphöja ditt huvud och taga det av dig; han skall upphänga dig på trä, och fåglarna skola äta ditt kött."
\par 20 På tredje dagen därefter, då det var Faraos födelsedag, gjorde denne ett gästabud för alla sina tjänare. Då upphöjde han, bland sina tjänare, såväl överste munskänkens huvud som överste bagarens.
\par 21 Han insatte överste munskänken åter i hans ämbete, så att han fick giva Farao bägaren i handen;
\par 22 men överste bagaren lät han upphänga, såsom Josef hade sagt dem i sin uttydning.
\par 23 Men överste munskänken tänkte icke på Josef, utan glömde honom.

\chapter{41}

\par 1 Två år därefter hände sig att Farao hade en dröm. Han tyckte sig stå vid Nilfloden.
\par 2 Och han såg sju kor, vackra och feta, stiga upp ur floden, och de betade i vassen.
\par 3 Sedan såg han sju andra kor, fula och magra, stiga upp ur floden; och de ställde sig bredvid de förra korna på stranden av floden.
\par 4 Och de fula och magra korna åto upp de sju vackra och feta korna. Därefter vaknade Farao.
\par 5 Men han somnade åter in och såg då i drömmen sju ax, frodiga och vackra, växa på samma strå.
\par 6 Sedan såg han sju andra ax skjuta upp, tunna och svedda av östanvinden;
\par 7 och de tunna axen uppslukade de sju frodiga och fulla axen. Därefter vaknade Farao och fann att det var en dröm.
\par 8 Då han nu om morgonen var orolig till sinnes, sände han ut och lät kalla till sig alla spåmän och alla vise i Egypten. Och Farao förtäljde sina drömmar för dem; men ingen fanns, som kunde uttyda dem för Farao.
\par 9 Då talade överste munskänken till Farao och sade: "Jag måste i dag påminna om mina synder.
\par 10 När Farao en gång var förtörnad på sina tjänare, satte han mig jämte överste bagaren i fängelse i drabanthövitsmannens hus.
\par 11 Då hade vi båda, jag och han, under samma natt en dröm, och våra drömmar hade var sin särskilda betydelse.
\par 12 Och jämte oss var där en ung hebré, som var tjänare hos hövitsmannen för drabanterna. För honom förtäljde vi våra drömmar, och han uttydde dem för oss; efter som var och en hade drömt gav han en uttydning.
\par 13 Och såsom han uttydde för oss, så gick det. Jag blev åter insatt på min plats, och den andre blev upphängd."
\par 14 Då sände Farao och lät kalla Josef till sig; och man skyndade att föra honom ut ur fängelset. Och han lät raka sig och bytte om kläder och kom inför Farao.
\par 15 Och Farao sade till Josef: "Jag har haft en dröm, och ingen finnes, som kan uttyda den. Men jag har hört sägas om dig, att allenast du får höra en dröm, kan du uttyda den."
\par 16 Josef svarade Farao och sade: "I min makt står det icke; men Gud skall giva Farao ett lyckosamt svar."
\par 17 Då sade Farao till Josef: "Jag drömde att jag stod på stranden av Nilfloden.
\par 18 Och jag såg sju kor stiga upp ur floden, feta och vackra, och de betade i vassen.
\par 19 Sedan såg jag sju andra kor stiga upp, avfallna och mycket fula och magra; i hela Egyptens land har jag icke sett några så fula som dessa.
\par 20 Och de magra och fula korna åto upp de sju första, feta korna.
\par 21 Men när de hade sväljt ned dem, kunde man icke märka att de hade sväljt ned dem, utan de förblevo fula såsom förut. Därefter vaknade jag.
\par 22 Åter drömde jag och såg då sju ax, fulla och vackra, växa på samma strå.
\par 23 Sedan såg jag sju andra ax skjuta upp, förtorkade, tunna och svedda av östanvinden;
\par 24 och de tunna axen uppslukade de sju vackra axen. Detta omtalade jag för spåmännen; men ingen fanns, som kunde förklara det för mig."
\par 25 Då sade Josef till Farao: "Faraos drömmar hava en och samma betydelse; vad Gud ämnar göra, det har han förkunnat för Farao.
\par 26 De sju vackra korna betyda sju år, de sju vackra axen betyda ock sju år; drömmarna hava en och samma betydelse.
\par 27 Och de sju magra och fula korna som stego upp efter dessa betyda sju år, så ock de sju tomma axen, de som voro svedda av östanvinden; sju hungerår skola nämligen komma.
\par 28 Detta menade jag, när jag sade till Farao: Vad Gud ämnar göra, det har han låtit Farao veta.
\par 29 Se, sju år skola komma med stor ymnighet över hela Egyptens land.
\par 30 Men efter dem skola sju hungerår inträffa, sådana, att man skall förgäta all den förra ymnigheten i Egyptens land, och hungersnöden skall förtära landet.
\par 31 Och man skall icke hava något minne av den förra ymnigheten i landet, för den hungersnöds skull som sedan kommer, ty den skall bliva mycket svår.
\par 32 Men att Farao har haft drömmen två gånger, det betyder att detta är av Gud bestämt, och att Gud skall låta det ske snart.
\par 33 Må nu alltså Farao utse en förståndig och vis man, som han kan sätta över Egyptens land.
\par 34 Må Farao göra så; må han ock förordna tillsyningsmän över landet och taga upp femtedelen av avkastningen i Egyptens land under de sju ymniga åren.
\par 35 Må man under dessa kommande goda år samla in allt som kan tjäna till föda och hopföra säd under Faraos vård i städerna, för att tjäna till föda, och må man sedan förvara den,
\par 36 så att dessa födoämnen finnas att tillgå för landet under de sju hungerår som skola komma över Egyptens land. Så skall landet icke behöva förgås genom hungersnöden."
\par 37 Det talet behagade Farao och alla hans tjänare.
\par 38 Och Farao sade till sina tjänare: "Kunna vi finna någon i vilken Guds Ande så är som i denne?"
\par 39 Och Farao sade till Josef: "Eftersom Gud har kungjort för dig allt detta, finnes ingen som är så förståndig och vis som du.
\par 40 Du skall förestå mitt hus, och efter dina befallningar skall allt mitt folk rätta sig; allenast däri att tronen förbliver min vill jag vara förmer än du."
\par 41 Ytterligare sade Farao till Josef: "Jag sätter dig nu över hela Egyptens land."
\par 42 Och Farao tog ringen av sin hand och satte den på Josefs hand och lät kläda honom i kläder av fint linne och hängde den gyllene kedjan om hans hals.
\par 43 Och han lät honom åka i vagnen närmast efter sin egen, och man utropade framför honom "abrek". Och han satte honom över hela Egyptens land.
\par 44 Och Farao sade till Josef: "Jag är Farao; utan din vilja skall ingen i hela Egyptens land lyfta hand eller fot."
\par 45 Och Farao gav Josef namnet Safenat-Panea och gav honom till hustru Asenat, dotter till Poti-Fera, prästen i On. Och Josef begav sig ut och besåg Egyptens land.
\par 46 Josef var trettio år gammal, när han stod inför Farao, konungen i Egypten. Och Josef gick ut ifrån Farao och färdades omkring i hela Egyptens land.
\par 47 Och landet gav under de sju ymniga åren avkastning i överflöd
\par 48 och under dessa sju år som kommo i Egyptens land samlade han in allt som kunde tjäna till föda och lade upp det i städerna. I var särskild stad lade han upp de födoämnen som man hämtade ifrån fälten däromkring.
\par 49 Så hopförde Josef säd i stor myckenhet, såsom sanden i havet, till dess man måste upphöra att hålla räkning på den, eftersom det var omöjligt att hålla räkning på den.
\par 50 Och åt Josef föddes två söner, innan något hungerår kom; de föddes åt honom av Asenat, dotter till Poti-Fera, prästen i On.
\par 51 Och Josef gav åt den förstfödde namnet Manasse, "ty", sade han, "Gud har låtit mig förgäta all min olycka och hela min faders hus."
\par 52 Och åt den andre gav han namnet Efraim, "ty", sade han, "Gud har gjort mig fruktsam i mitt lidandes land".
\par 53 Men de sju ymniga åren som först hade kommit i Egyptens land gingo till ända;
\par 54 sedan begynte de sju hungeråren, såsom Josef hade förutsagt. Och hungersnöd uppstod i alla andra länder; men i Egyptens land fanns bröd överallt.
\par 55 Och när hela Egyptens land begynte hungra och folket ropade till Farao efter bröd, sade Farao till alla egyptier: "Gån till Josef, och gören vad han säger eder."
\par 56 När nu alltså hungersnöd var över hela landet, öppnade Josef alla förrådshus och sålde säd åt egyptierna. Men hungersnöden blev allt större i Egyptens land;
\par 57 och från alla länder kom man till Josef i Egypten för att köpa säd, ty hungersnöden blev allt större i alla länder.

\chapter{42}

\par 1 Men när Jakob förnam att säd fanns i Egypten, sade han till sina söner: "Varför stån I så rådlösa?"
\par 2 Och han sade vidare: "Se, jag har hört att i Egypten finnes säd; faren ditned och köpen därifrån säd åt oss, för att vi må leva och icke dö."
\par 3 Då foro tio av Josefs bröder ned för att köpa säd i Egypten.
\par 4 Men Benjamin, Josefs broder, blev icke av Jakob sänd åstad med sina bröder, ty han fruktade att någon olycka kunde hända honom.
\par 5 Så kommo då, bland de andra, också Israels söner för att köpa säd; ty hungersnöd rådde i Kanaans land.
\par 6 Och Josef var den som hade att befalla i landet; det var han som sålde säd åt allt folket i landet. Då nu Josefs bröder kommo dit, föllo de ned till jorden på sitt ansikte inför honom.
\par 7 När då Josef fick se sina bröder, kände han igen dem; men han ställde sig främmande mot dem och tilltalade dem hårt och frågade dem: "Varifrån kommen I?" De svarade: "Från Kanaans land, för att köpa säd till föda åt oss."
\par 8 Och fastän Josef kände igen sina bröder, kände de icke igen honom.
\par 9 Men Josef tänkte på de drömmar som han hade drömt om dem. Och han sade till dem: "I ären spejare, I haven kommit för att se efter, var landet är utan skydd."
\par 10 De svarade honom: "Nej, herre, dina tjänare hava kommit för att köpa säd till föda åt sig.
\par 11 Vi äro alla söner till en och samma man; vi äro redliga män, dina tjänare äro inga spejare."
\par 12 Men han sade till dem: "Jo, I haven kommit för att se efter, var landet är utan skydd."
\par 13 De svarade: "Vi, dina tjänare, äro tolv bröder, söner till en och samma man i Kanaans land; men den yngste är nu hemma hos vår fader, och en är icke mer till."
\par 14 Josef sade till dem: "Det är såsom jag sade eder: I ären spejare.
\par 15 Och på detta sätt vill jag pröva eder: så sant Farao lever, I skolen icke slippa härifrån, med mindre eder yngste broder kommer hit.
\par 16 En av eder må fara och hämta hit eder broder. Men I andra skolen stanna såsom fångar, för att jag så må pröva om I haven talat sanning. Ty om så icke är, då ären I spejare, så sant Farao lever."
\par 17 Därefter lät han hålla dem allasammans i fängelse under tre dagar.
\par 18 Men på tredje dagen sade Josef till dem: "Om I viljen leva, så gören på detta sätt, ty jag fruktar Gud:
\par 19 ären I redliga män, så må en av eder, I bröder, stanna såsom fånge i huset där I haven suttit fängslade; men I andra mån fara eder väg, och föra hem med eder den säd som I haven köpt till hjälp mot hungersnöden hemma hos eder.
\par 20 Fören sedan eder yngste broder hit till mig; om så edra ord visa sig vara sanna, skolen I slippa att dö." Och de måste göra så.
\par 21 Men de sade till varandra: "Förvisso hava vi dragit skuld över oss genom det som vi gjorde mot vår broder; ty vi sågo hans själs ångest, när han bad oss om misskund, och vi ville dock icke lyssna till honom. Därför hava vi själva kommit i denna ångest."
\par 22 Ruben svarade dem: "Sade jag icke till eder: 'Försynden eder icke på gossen'? Men I lyssnaden icke till mig; se, därför utkräves nu hans blod."
\par 23 Men de visste icke att Josef förstod detta, ty han talade med dem genom tolk.
\par 24 Och han vände sig bort ifrån dem och grät. Sedan vände han sig åter till dem och talade med dem; och han tog Simeon ut ur deras krets och lät fängsla honom inför deras ögon.
\par 25 Och Josef bjöd att man skulle fylla deras säckar med säd, och lägga vars och ens penningar tillbaka i hans säck, och giva dem kost för resan. Och man gjorde så med dem.
\par 26 Och de lastade säden på sina åsnor och foro därifrån.
\par 27 Men när vid ett viloställe en av dem öppnade sin säck för att giva foder åt sin åsna, fick han se sina penningar ligga överst i säcken.
\par 28 Då sade han till sina bröder: "Mina penningar hava blivit lagda hit tillbaka; se, de äro här i min säck." Då blevo de utom sig av häpnad och sågo förskräckta på varandra och sade: "Vad har Gud gjort mot oss!"
\par 29 När de kommo hem till sin fader Jakob i Kanaans land, berättade de för honom allt vad som hade hänt dem och sade:
\par 30 "Mannen som var herre där i landet tilltalade oss hårt och behandlade oss såsom om vi ville bespeja landet.
\par 31 Men vi sade till honom: 'Vi äro redliga män och inga spejare;
\par 32 vi äro tolv bröder, samma faders söner; en är icke mer till, och den yngste är nu hemma hos vår fader i Kanaans land.'
\par 33 Men mannen som var herre i landet svarade oss: 'Därav skall jag veta att I ären redliga män: lämnen kvar hos mig en av eder, I bröder; tagen så vad I haven köpt till hjälp mot hungersnöden hemma hos eder, och faren eder väg.
\par 34 Sedan mån I föra eder yngste broder hit till mig, så kan jag veta att I icke ären spejare, utan redliga män. Då skall jag giva eder broder tillbaka åt eder, och I skolen fritt få draga omkring i landet.
\par 35 När de sedan tömde sina säckar, fann var och en sin penningpung i sin säck. Och då de och deras fader fingo se penningpungarna, blevo de förskräckta.
\par 36 Och Jakob, deras fader, sade till dem: "I gören mig barnlös; Josef är borta, Simeon är borta, Benjamin viljen I ock taga ifrån mig; över mig kommer allt detta."
\par 37 Då svarade Ruben sin fader och sade: "Mina båda söner må du döda, om jag icke för honom åter till dig. Anförtro honom åt mig, jag skall föra honom tillbaka till dig."
\par 38 Men han svarade: "Min son får icke fara ditned med eder. Hans broder är ju död, och han är allena kvar; om nu någon olycka hände honom på den resa I viljen företaga, så skullen I bringa mina grå hår med sorg ned i dödsriket."

\chapter{43}

\par 1 Men hungersnöden var svår i landet.
\par 2 Och när de hade förtärt den säd som de hade hämtat från Egypten, sade deras fader till dem: "Faren tillbaka och köpen litet säd till föda åt oss."
\par 3 Men Juda svarade honom och sade: "Mannen betygade högtidligt och sade till oss: 'I fån icke komma inför mitt ansikte, med mindre eder broder är med eder.'
\par 4 Om du nu låter vår broder följa med oss, så skola vi fara ned och köpa säd till föda åt dig.
\par 5 Men om du icke låter honom följa med oss, så vilja vi icke fara, ty mannen sade till oss: 'I fån icke komma inför mitt ansikte, med mindre eder broder är med eder.'
\par 6 Då sade Israel: "Varför gjorden I så illa mot mig och berättaden för mannen att I haden ännu en broder?"
\par 7 De svarade: "Mannen frågade noga om oss och vår släkt; han sade: 'Lever eder fader ännu? Haven I någon broder?' Då omtalade vi för honom huru det förhöll sig. Kunde vi veta att han skulle säga: 'Fören eder broder hitned'?"
\par 8 Och Juda sade till sin fader Israel: "Låt ynglingen följa med mig, så vilja vi stå upp och begiva oss åstad, för att vi må leva och icke dö, vi själva och du och våra kvinnor och barn.
\par 9 Jag vill ansvara för honom; av min hand må du utkräva honom. Om jag icke för honom åter till dig och ställer honom inför ditt ansikte, så vill jag vara en syndare inför dig i all min tid.
\par 10 Sannerligen, om vi icke hade dröjt så länge, så skulle vi redan hava varit tillbaka för andra gången."
\par 11 Då svarade deras fader Israel dem: "Måste det så vara, så gören nu på detta sätt: tagen av landets bästa frukt i edra säckar och fören det till mannen såsom skänk, litet balsam och litet honung, dragantgummi och ladanum, pistacienötter och mandlar.
\par 12 Och tagen dubbla summan penningar med eder, så att I fören tillbaka dit med eder de penningar som I haven fått igen överst i edra säckar. Kanhända var det ett misstag.
\par 13 Tagen ock eder broder med eder, och stån upp och faren tillbaka till mannen.
\par 14 Men Gud den Allsmäktige låte eder finna barmhärtighet inför mannen, så att han tillstädjer eder andre broder och Benjamin att återvända med eder. Men skall jag bliva barnlös, så må det då ske."
\par 15 Då togo männen de nämnda skänkerna och togo med sig dubbla summan penningar, därtill ock Benjamin, och stodo upp och foro ned till Egypten och trädde inför Josef.
\par 16 Då nu Josef såg att Benjamin var med dem, sade han till sin hovmästare: "För dessa män in i mitt hus; och låt slakta och tillreda en måltid, ty männen skola äta middag med mig."
\par 17 Och mannen gjorde såsom Josef hade sagt och förde männen in i Josefs hus.
\par 18 Och männen blevo förskräckta, när de fördes in i Josefs hus; de sade: "Det är på grund av penningarna vi föras hitin, de penningar som förra gången kommo tillbaka i våra säckar; ty han vill nu störta sig på oss och överfalla oss och göra oss själva till trälar och taga ifrån oss våra åsnor."
\par 19 Och de trädde fram till Josefs hovmästare och talade med honom vid ingången till huset
\par 20 och sade: "Hör oss, herre. När vi förra gången voro härnere för att köpa säd till föda åt oss
\par 21 och sedan kommo till ett viloställe och öppnade våra säckar, då fann var och en av oss sina penningar överst i sin säck, penningarna till deras fulla vikt; dem hava vi nu fört tillbaka med oss.
\par 22 Och vi hava tagit andra penningar med oss för att köpa säd till föda åt oss. Vi veta icke vem som hade lagt penningarna i våra säckar."
\par 23 Då svarade han: "Varen vid gott mod, frukten icke; det är eder Gud och eder faders Gud som har låtit eder finna en skatt i edra säckar; edra penningar har jag fått." Sedan hämtade han Simeon ut till dem.
\par 24 Och han förde männen in i Josefs hus och gav dem vatten till att två sina fötter och gav foder åt deras åsnor.
\par 25 Och de ställde i ordning sina skänker, till dess Josef skulle komma hem om middagen; ty de hade fått höra att de skulle äta där.
\par 26 När sedan Josef hade kommit hem, förde de skänkerna, som de hade med sig, in till honom i huset och föllo ned för honom till jorden.
\par 27 Och han hälsade dem och frågade: "Står det väl till med eder fader, den gamle, som I taladen om? Lever han ännu?"
\par 28 De svarade: "Ja, det står väl till med vår fader, din tjänare; han lever ännu." Och de bugade sig och föllo ned för honom.
\par 29 Och när han lyfte upp sina ögon och fick se sin broder Benjamin, sin moders son, frågade han: "Är detta eder yngste broder, den som I taladen om med mig?" Därpå sade han: "Gud vare dig nådig, min son."
\par 30 Men Josef bröt av sitt tal, ty hans hjärta upprördes av kärlek till brodern, och han sökte tillfälle att gråta ut och gick in i sin kammare och grät där.
\par 31 Därefter, sedan han hade tvagit sitt ansikte, gick han åter ut och betvang sig och sade: "Sätten fram mat."
\par 32 Och de satte fram särskilt för honom och särskilt för dem och särskilt för de egyptier som åto tillsammans med honom; ty egyptierna få icke äta tillsammans med hebréerna; sådant är nämligen en styggelse för egyptierna.
\par 33 Och de fingo sina platser mitt emot honom, den förstfödde främst såsom den förstfödde, sedan de yngre, var och en efter sin ålder; och männen sågo med förundran på varandra.
\par 34 Och han lät bära till dem av rätterna på sitt bord, och Benjamin fick fem gånger så mycket som var och en av de andra. Och de drucko sig glada med honom.

\chapter{44}

\par 1 Därefter bjöd han sin hovmästare och sade: "Fyll männens säckar med säd, så mycket de kunna rymma, och lägg vars och ens penningar överst i hans säck.
\par 2 Och min bägare, silverbägaren, skall du lägga överst i den yngstes säck, tillika med penningarna för hans säd." Och han gjorde såsom Josef hade sagt.
\par 3 Om morgonen, då det blev dager, fingo männen fara med sina åsnor.
\par 4 Men när de hade kommit ett litet stycke utom staden, sade Josef till sin hovmästare: "Stå upp och sätt efter männen; och när du hinner upp dem, så säg till dem: 'Varför haven I lönat gott med ont?
\par 5 Det är ju just den bägaren som min herre dricker ur, och som han plägar spå med. Det är en ond gärning I haven gjort.'"
\par 6 När han nu hann upp dem, sade han detta till dem.
\par 7 Då svarade de honom: "Varför talar min herre så? Bort det, att dina tjänare skulle göra sådant!
\par 8 De penningar som vi funno överst i våra säckar hava vi ju fört tillbaka till dig från Kanaans land. Huru skulle vi då kunna vilja stjäla silver eller guld ur din herres hus?
\par 9 Den bland dina tjänare, som den finnes hos, han må dö; därtill vilja vi andra bliva min herres trälar."
\par 10 Han svarade: "Ja, vare det såsom I haven sagt; den som den finnes hos, han skall bliva min träl. Men I andra skolen vara utan skuld."
\par 11 Och de skyndade sig att lyfta ned var och en sin säck på jorden, och öppnade var och en sin säck.
\par 12 Och han begynte att söka hos den äldste och slutade hos den yngste; och bägaren fanns i Benjamins säck.
\par 13 Då revo de sönder sina kläder och lastade åter var och en sin åsna och vände tillbaka till staden.
\par 14 Och Juda och hans bröder gingo in i Josefs hus, där denne ännu var kvar; och de föllo ned till jorden för honom.
\par 15 Då sade Josef till dem: "Vad haven I gjort! Förstoden I icke att en man sådan som jag kan spå?"
\par 16 Juda svarade: Vad skola vi säga till min herre, vad skola vi tala, och huru skola vi rättfärdiga oss? Gud har funnit dina tjänares missgärning. Se, vi äro min herres trälar, vi andra såväl som den som bägaren har blivit funnen hos."
\par 17 Men han sade: "Bort det, att jag skulle så göra! Den som bägaren har blivit funnen hos, han skall bliva min träl. Men I andra mån i frid fara hem till eder fader."
\par 18 Då trädde Juda fram till honom och sade: "Hör mig, herre; låt din tjänare tala ett ord inför min herre, och må din vrede icke upptändas mot din tjänare; ty du är såsom Farao.
\par 19 Min herre frågade sina tjänare och sade: 'Haven I eder fader eller någon broder ännu därhemma?'
\par 20 Och vi svarade min herre: 'Vi hava en åldrig fader och en son till honom, en som är född på hans ålderdom och ännu är ung; men en broder till denne är död, så att han allena är kvar efter sin moder, och hans fader har honom kär.'
\par 21 Då sade du till dina tjänare: 'Fören honom hitned till mig, så att jag kan låta mitt öga vila på honom.'
\par 22 Och vi svarade min herre: 'Ynglingen kan icke lämna sin fader, ty om han lämnade sin fader, så skulle denne dö.'
\par 23 Men du sade till dina tjänare: 'Om eder yngste broder icke följer med eder hitned, så fån I icke mer komma inför mitt ansikte.'
\par 24 När vi därefter hade kommit hem till din tjänare, min fader, berättade vi för honom vad min herre hade sagt.
\par 25 Och när sedan vår fader sade: 'Faren tillbaka och köpen litet säd till föda åt oss',
\par 26 svarade vi: 'Vi kunna icke fara ditned; allenast på det villkoret vilja vi fara, att vår yngste broder följer med oss; ty vi få icke komma inför mannens ansikte om vår yngste broder icke är med oss.
\par 27 Men din tjänare, min fader, sade till oss: 'I veten själva att min hustru har fött åt mig två söner,
\par 28 och den ene gick bort ifrån mig, och jag sade: förvisso är han ihjälriven. Och jag har icke sett honom sedan den tiden.
\par 29 Om I nu tagen också denne ifrån mig och någon olycka händer honom, så skolen I bringa mina grå hår med jämmer ned i dödsriket.'
\par 30 Om jag alltså kommer hem till din tjänare, min fader, utan att vi hava med oss ynglingen, som vår faders hjärta är så fäst vid,
\par 31 då bliver det hans död, när han ser att ynglingen icke är med; och dina tjänare skulle så bringa din tjänares, vår faders, grå hår med sorg ned i dödsriket.
\par 32 Ty jag, din tjänare, har lovat min fader att ansvara för ynglingen och har sagt, att om jag icke för denne till honom igen, så vill jag vara en syndare inför min fader i all min tid.
\par 33 Låt nu därför din tjänare stanna kvar hos min herre såsom träl, i ynglingens ställe, men låt ynglingen fara hem med sina bröder.
\par 34 Ty huru skulle jag kunna fara hem till min fader utan att hava ynglingen med mig? Jag förmår icke se den jämmer som då skulle komma över min fader."

\chapter{45}

\par 1 Då kunde Josef icke längre betvinga sig inför alla dem som stodo omkring honom. Han ropade: "Må alla gå ut härifrån." Och ingen fick stanna inne hos Josef, när han gav sig till känna för sina bröder.
\par 2 Och han brast ut i högljudd gråt, så att egyptierna hörde det; också Faraos husfolk hörde det.
\par 3 Och Josef sade till sina bröder: "Jag är Josef. Lever min fader ännu?" Men hans bröder kunde icke svara honom, så förskräckta blevo de för honom.
\par 4 Då sade Josef till sina bröder: "Kommen hitfram till mig." Och när de kommo fram, sade han: "Jag är Josef, eder broder, som I sålden till Egypten.
\par 5 Men varen nu icke bedrövade och grämen eder icke däröver att I haven sålt mig hit: ty för att bevara människors liv har Gud sänt mig hit före eder.
\par 6 I två år har nu hungersnöd varit i landet, och ännu återstå fem år under vilka man varken skall plöja eller skörda.
\par 7 Men Gud sände mig hit före eder, för att I skullen bliva kvar på jorden och behållas vid liv, ja, till räddning för många.
\par 8 Så haven nu icke I sänt mig hit, utan Gud; och han har gjort mig till Faraos högste rådgivare och till en herre över hela hans hus och till en furste över hela Egyptens land.
\par 9 Skynden eder nu och faren hem till min fader, och sägen till honom: 'Så säger din son Josef: Gud har satt mig till en herre över hela Egypten; kom ned till mig, dröj icke.
\par 10 Du skall få bo i landet Gosen och vara mig nära, du med dina barn och barnbarn, dina får och fäkreatur och allt vad som tillhör dig.
\par 11 Jag vill där försörja dig - ty ännu återstå fem hungerår - så att varken du eller ditt hus eller någon som hör dig till skall lida nöd.
\par 12 I sen ju med egna ögon, också min broder Benjamin ser med egna ögon, att det är jag, som med egen mun talar till eder.
\par 13 Berätten nu för min fader om all min härlighet i Egypten och om allt vad I haven sett, och skynden eder att föra min fader hitned."
\par 14 Så föll han sin broder Benjamin om halsen och grät, och Benjamin grät vid hans hals.
\par 15 Och han kysste alla sina bröder och grät i deras armar. Sedan samtalade hans bröder med honom.
\par 16 När nu det ryktet spordes i Faraos hus, att Josefs bröder hade kommit, behagade detta Farao och hans tjänare väl.
\par 17 Och Farao sade till Josef: "Säg till dina bröder: 'Detta skolen I göra: lasten edra djur och faren hem till Kanaans land;
\par 18 hämten så eder fader och edert folk och kommen hit till mig, så skall jag giva eder det bästa som finnes i Egyptens land, och I skolen få äta av landets fetma.'
\par 19 Alltså bjuder jag dig nu att säga: 'Detta skolen I göra: tagen eder vagnar i Egyptens land för edra späda barn och edra hustrur, och hämten eder fader och kommen hit.
\par 20 Och bekymren eder icke om edert bohag; ty det bästa som finnes i hela Egyptens land skall höra eder till.'"
\par 21 Israels söner gjorde så, och Josef gav dem vagnar, efter Faraos befallning, och gav dem kost för resan.
\par 22 Och han gav åt dem alla var sin högtidsdräkt, men åt Benjamin gav han tre hundra siklar silver och fem högtidsdräkter.
\par 23 Och till sin fader sände han likaledes gåvor: tio åsnor, lastade med det bästa Egypten hade, och tio åsninnor, lastade med säd och bröd och andra livsmedel åt hans fader för resan.
\par 24 Därefter lät han sina bröder fara, och de begåvo sig åstad; och han sade till dem: "Kiven icke på vägen."
\par 25 Så foro de upp från Egypten och kommo till sin fader Jakob i Kanaans land;
\par 26 och de berättade för honom och sade: "Josef lever ännu, och han är en furste över hela Egyptens land." Då greps hans hjärta av vanmakt, ty han kunde icke tro dem.
\par 27 Men när de omtalade för honom allt vad Josef hade sagt till dem, och när han såg vagnarna som Josef hade sänt för att hämta honom, då fick deras fader Jakobs ande åter liv.
\par 28 Och Israel sade: "Det är nog; min son Josef lever ännu. Jag vill fara och se honom, förrän jag dör."

\chapter{46}

\par 1 Och Israel bröt upp med allt vad honom tillhörde. Och när han kom till Beer-Seba, offrade han slaktoffer åt sin fader Isaks Gud.
\par 2 Och Gud talade till Israel i en syn om natten; han sade: "Jakob! Jakob!" Han svarade: "Här är jag."
\par 3 Då sade han: "Jag är Gud, din faders Gud; frukta icke för att draga ned till Egypten, ty där skall jag göra dig till ett stort folk.
\par 4 Jag skall själv draga ned med dig till Egypten, jag skall ock föra dig åter upp därifrån; och Josefs hand skall tillsluta dina ögon."
\par 5 Och Jakob bröt upp från Beer-Seba; och Israels söner satte sin fader Jakob och sina späda barn och sina hustrur på vagnarna som Farao hade sänt för att hämta honom.
\par 6 Och de togo sin boskap och de ägodelar som de hade förvärvat i Kanaans land och kommo så till Egypten, Jakob och alla hans avkomlingar med honom.
\par 7 Sina söner och sonsöner, sina döttrar och sondöttrar, alla sina avkomlingar, förde han med sig till Egypten.
\par 8 Dessa äro namnen på Israels barn som kommo till Egypten: Jakob och hans söner. Jakobs förstfödde var Ruben,
\par 9 och Rubens söner voro Hanok, Pallu, Hesron och Karmi.
\par 10 Simeons söner voro Jemuel, Jamin, Ohad, Jakin, Sohar och Saul, den kananeiska kvinnans son.
\par 11 Levis söner voro Gerson, Kehat och Merari.
\par 12 Judas söner voro Er, Onan, Sela, Peres och Sera - men Er och Onan dogo i Kanaans land - och Peres' söner voro Hesron och Hamul.
\par 13 Isaskars söner voro Tola, Puva, Job och Simron.
\par 14 Sebulons söner voro Sered, Elon och Jaleel.
\par 15 Dessa voro Leas söner, de som hon födde åt Jakob i Paddan-Aram; tillika födde hon åt honom dottern Dina. Söner och döttrar utgjorde tillsammans trettiotre personer.
\par 16 Gads söner voro Sifjon och Haggi, Suni och Esbon, Eri och Arodi och Areli.
\par 17 Asers söner voro Jimna, Jisva, Jisvi och Beria; och deras syster var Sera; men Berias söner voro Heber och Malkiel.
\par 18 Dessa voro söner till Silpa, som Laban hade givit åt sin dotter Lea, och dessa födde hon åt Jakob, sexton personer.
\par 19 Rakels, Jakobs hustrus, söner voro Josef och Benjamin.
\par 20 Och de söner som föddes åt Josef i Egyptens land voro Manasse och Efraim; de föddes åt honom av Asenat, dotter till Poti-Fera, prästen i On.
\par 21 Och Benjamins söner voro Bela, Beker och Asbel, Gera och Naaman, Ehi och Ros, Muppim och Huppim och Ard.
\par 22 Dessa voro Rakels söner, de som föddes åt Jakob, tillsammans fjorton personer.
\par 23 Dans söner voro Husim.
\par 24 Naftalis söner voro Jaseel, Guni, Jeser och Sillem.
\par 25 Dessa voro söner till Bilha, som Laban hade givit åt sin dotter Rakel, och dessa födde hon åt Jakob, tillsammans sju personer.
\par 26 De som kommo med Jakob till Egypten, de som hade utgått från hans länd, utgjorde alla tillsammans sextiosex personer, förutom Jakobs sonhustrur.
\par 27 Och Josefs söner, vilka föddes åt honom i Egypten, voro två. De personer av Jakobs hus, som kommo till Egypten, utgjorde tillsammans sjuttio.
\par 28 Och han sände Juda framför sig till Josef, för att denne skulle visa honom vägen till Gosen. Så kommo de till landet Gosen.
\par 29 Och Josef lät spänna för sin vagn och for upp till Gosen för att möta sin fader Israel. Och när han kom fram till honom, föll han honom om halsen och grät länge vid hans hals.
\par 30 Och Israel sade till Josef: "Nu vill jag gärna dö, sedan jag har sett ditt ansikte och sett att du ännu lever."
\par 31 Därefter sade Josef till sina bröder och sin faders folk: "Jag vill fara upp och berätta för Farao och säga till honom: 'Mina bröder och min faders folk, som hittills hava bott i Kanaans land, hava kommit till mig.
\par 32 Och dessa män äro fårherdar, ty de hava idkat boskapsskötsel; och sina får och fäkreatur och allt vad de äga hava de fört med sig.'
\par 33 När sedan Farao kallar eder till sig och frågar: 'Vad är edert yrke?',
\par 34 skolen I svara: 'Vi, dina tjänare, hava idkat boskapsskötsel från vår ungdom ända till nu, vi såväl som våra fäder.' Så skolen I få bo i landet Gosen; ty alla fårherdar äro en styggelse för egyptierna."

\chapter{47}

\par 1 Och Josef kom och berättade för Farao och sade: "Min fader och mina bröder hava kommit från Kanaans land med sina får och fäkreatur och allt vad de äga; och de äro nu i landet Gosen."
\par 2 Och han hade bland sina bröder tagit ut fem män; dem ställde han fram inför Farao.
\par 3 Då frågade Farao hans bröder: "Vad är edert yrke?" De svarade Farao: "Dina tjänare äro fårherdar, såsom ock våra fäder hava varit."
\par 4 Och de sade ytterligare till Farao: "Vi hava kommit för att bo någon tid här i landet; ty dina tjänare hade intet bete för sina får, eftersom hungersnöden är så svår i Kanaans land. Så låt nu dina tjänare bo i landet Gosen."
\par 5 Då sade Farao till Josef: "Din fader och dina bröder hava alltså nu kommit till dig.
\par 6 Egyptens land ligger öppet för dig; i den bästa delen av landet må du låta din fader och dina bröder bo. Må de bo i landet Gosen, och ifall du vet om några bland dem att de äro dugande män, så sätt dessa till uppsyningsmän över min boskap."
\par 7 Sedan hämtade Josef sin fader Jakob och förde honom fram inför Farao, och Jakob hälsade Farao.
\par 8 Men Farao frågade Jakob: "Huru hög är din ålder?"
\par 9 Jakob svarade Farao: "Min vandringstid har varat ett hundra trettio år. Få och onda hava mina levnadsår varit, de nå icke upp till antalet av mina fäders levnadsår under deras vandringstid."
\par 10 Och Jakob tog avsked av Farao och gick ut ifrån honom.
\par 11 Men Josef lät sin fader och sina bröder bo i Egyptens land och gav dem besittning där, i den bästa delen av landet, i landet Rameses, såsom Farao hade bjudit.
\par 12 Och Josef försörjde sin fader och sina bröder och hela sin faders hus, och gav var och en underhåll efter antalet av hans barn.
\par 13 Men ingenstädes i landet fanns bröd, ty hungersnöden var mycket svår, så att Egyptens land och Kanaans land försmäktade av hunger.
\par 14 Och för den säd som folket köpte samlade Josef till sig alla penningar som funnos i Egyptens land och i Kanaans land; och Josef lät föra penningarna in i Faraos hus.
\par 15 Men när penningarna togo slut i Egyptens land och i Kanaans land, kommo alla egyptier till Josef och sade: "Giv oss bröd; icke vill du väl att vi skola dö i din åsyn? Vi hava ju inga penningar mer."
\par 16 Josef svarade: "Fören hit eder boskap, så skall jag giva eder bröd i utbyte mot eder boskap, om I icke mer haven några penningar."
\par 17 Då förde de sin boskap till Josef, och Josef gav dem bröd i utbyte mot deras hästar, får, fäkreatur och åsnor. Så underhöll han dem det året och gav dem bröd i utbyte mot all deras boskap.
\par 18 Så gick detta år till ända. Men det följande året kommo de åter till honom och sade till honom: "Vi vilja icke dölja det för min herre: penningarna äro slut, och den boskap vi ägde har kommit i min herres ägo; intet annat finnes nu kvar att giva åt min herre än våra kroppar och vår jord.
\par 19 Icke vill du att vi skola förgås inför dina ögon, vi med vår åkerjord? Köp oss och vår jord för bröd, så vilja vi med vår jord bliva Faraos trälar; giv oss allenast utsäde, för att vi må leva och icke dö, och för att jorden icke må läggas öde."
\par 20 Då köpte Josef all jord i Egypten åt Farao; ty egyptierna sålde var och en sin åker, eftersom hungersnöden så svårt tryckte dem. Så blev jorden Faraos egendom.
\par 21 Och folket förflyttade han till städerna, från den ena ändan av Egyptens område till den andra.
\par 22 Allenast prästernas jord köpte han icke, ty prästerna hade sitt bestämda underhåll av Farao, och de levde av det bestämda underhåll som Farao gav dem; därför behövde de icke sälja sin jord.
\par 23 Och Josef sade till folket: "Se, jag har nu köpt eder och eder jord åt Farao. Där haven I utsäde; besån nu jorden.
\par 24 Och när grödan kommer in, skolen I giva en femtedel åt Farao; men fyra femtedelar skolen I själva hava till utsäde på åkern och till föda för eder och dem som I haven i edra hus och till föda för edra barn."
\par 25 De svarade: "Du har behållit oss vid liv; låt oss finna nåd för min herres ögon, så vilja vi vara Faraos trälar."
\par 26 Så gjorde Josef det till en stadga, som ännu i dag gäller för Egyptens jord, att man skulle giva femtedelen åt Farao. Allenast prästernas jord blev icke Faraos egendom.
\par 27 Så bodde nu Israel i Egyptens land, i landet Gosen; och de fingo sina besittningar där och voro fruktsamma och förökade sig storligen.
\par 28 Och Jakob levde sjutton år i Egyptens land, så att hans levnadsålder blev ett hundra fyrtiosju år.
\par 29 Då nu tiden närmade sig att Israel skulle dö, kallade han till sig sin son Josef och sade till honom: "Om jag har funnit nåd för dina ögon, så lägg din hand under min länd och lova att visa mig din kärlek och trofasthet därmed att du icke begraver mig i Egypten;
\par 30 fastmer, när jag har gått till vila hos mina fäder, skall du föra mig från Egypten och begrava mig i deras grav." Han svarade: "Jag skall göra såsom du har sagt."
\par 31 Men han sade: "Giv mig din ed därpå." Och han gav honom sin ed. Då tillbad Israel, böjd mot sängens huvudgärd.

\chapter{48}

\par 1 En tid härefter blev det sagt till Josef: "Din fader är nu sjuk." Då tog han med sig sina båda söner, Manasse och Efraim.
\par 2 Och man berättade för Jakob och sade: "Din son Josef har nu kommit till dig." Då tog Israel styrka till sig och satte sig upp i sängen.
\par 3 Och Jakob sade till Josef: "Gud den Allsmäktige uppenbarade sig för mig i Lus i Kanaans land och välsignade mig
\par 4 och sade till mig: 'Se, jag skall göra dig fruktsam och föröka dig och låta skaror av folk komma av dig, och skall giva åt din säd efter dig detta land till evärdlig besittning.'
\par 5 Dina båda söner, som äro födda åt dig i Egyptens land, innan jag kom hit till dig i Egypten, de skola nu vara mina: Efraim och Manasse skola vara mina, likasom Ruben och Simeon.
\par 6 Men de barn som du har fött efter dem skola vara dina; de skola bära sina bröders namn i dessas arvedel.
\par 7 Se, när jag kom från Paddan, dog Rakel ifrån mig i Kanaans land, under resan, då det ännu var ett stycke väg fram till Efrat; och jag begrov henne där vid vägen till Efrat." Stället heter nu Bet-Lehem.
\par 8 Då nu Israel fick se Josefs söner, sade han: "Vilka äro dessa?"
\par 9 Josef svarade sin fader: "Det är mina söner, som Gud har givit mig här." Då sade han: "För dem hit till mig, på det att jag må välsigna dem."
\par 10 Och Israels ögon voro skumma av ålder, så att han icke kunde se. Så förde han dem då fram till honom, och han kysste dem och tog dem i famn.
\par 11 Och Israel sade till Josef: "Jag hade icke tänkt att jag skulle få se ditt ansikte, men nu har Gud låtit mig se till och med avkomlingar av dig."
\par 12 Och Josef förde dem bort ifrån hans knän och föll ned till jorden på sitt ansikte.
\par 13 Sedan tog Josef dem båda vid handen, Efraim i sin högra hand, till vänster framför Israel, och Manasse i sin vänstra hand, till höger framför Israel, och förde dem så fram till honom.
\par 14 Men Israel räckte ut sin högra hand och lade den på Efraims huvud, fastän han var den yngre, och sin vänstra hand på Manasses huvud; han lade alltså sina händer korsvis, ty Manasse var den förstfödde.
\par 15 Och han välsignade Josef och sade: "Den Gud inför vilken mina fäder, Abraham och Isak, hava vandrat, den Gud som har varit min herde från min födelse ända till denna dag,
\par 16 den ängel som har förlossat mig från allt ont, han välsigne dessa barn; och må de uppkallas efter mitt och mina fäders, Abrahams och Isaks, namn, och må de föröka sig och bliva talrika på jorden."
\par 17 Men när Josef såg att hans fader lade sin högra hand på Efraims huvud, misshagade detta honom, och han fattade sin faders hand och ville flytta den från Efraims huvud på Manasses huvud.
\par 18 Och Josef sade till sin fader: "Icke så, min fader; denne är den förstfödde, lägg din högra hand på hans huvud."
\par 19 Men hans fader ville icke; han sade: "Jag vet det, min son, jag vet det; också av honom skall ett folk komma, också han skall bliva stor; men hans yngre broder skall dock bliva större än han, och hans avkomma skall bliva ett talrikt folk."
\par 20 Så välsignade han dem på den dagen och sade: "Med ditt namn skall Israel välsigna, så att man skall säga: Gud göre dig lik Efraim och Manasse." Så satte han Efraim framför Manasse.
\par 21 Och Israel sade till Josef: "Se, jag dör; men Gud skall vara med eder och föra eder tillbaka till edra fäders land.
\par 22 Och utöver vad jag giver dina bröder giver jag dig en särskild höjdsträcka som jag med mitt svärd och min båge har tagit från amoréerna."

\chapter{49}

\par 1 Och Jakob kallade sina söner till sig och sade: Församlen eder, på det att jag må förkunna eder vad som skall hända eder i kommande dagar:
\par 2 Kommen tillhopa och hören, I Jakobs söner; hören på eder fader Israel.
\par 3 Ruben, min förstfödde är du, min kraft och min styrkas förstling, främst i myndighet och främst i makt.
\par 4 Du sjuder över såsom vatten, du skall icke bliva den främste, ty du besteg din faders läger; då gjorde du vad skändligt var. Ja, min bädd besteg han!
\par 5 Simeon och Levi äro bröder; deras vapen äro våldets verktyg.
\par 6 Min själ inlåte sig ej i deras råd, min ära tage ingen del i deras samkväm; ty i sin vrede dräpte de män, och i sitt överdåd stympade de oxar.
\par 7 Förbannad vare deras vrede, som är så våldsam, och deras grymhet, som är så hård! Jag skall förströ dem i Jakob, jag skall förskingra dem i Israel.
\par 8 Dig, Juda, dig skola dina bröder prisa; din hand skall vara på dina fienders nacke, för dig skola din faders söner buga sig.
\par 9 Ett ungt lejon är Juda; från rivet byte har du dragit ditupp, min son. Han har lagt sig ned, han vilar såsom ett lejon, såsom en lejoninna - vem vågar oroa honom?
\par 10 Spiran skall icke vika ifrån Juda, icke härskarstaven ifrån hans fötter, till dess han kommer till Silo (Eller: till dess Silo kommer.) och folken bliva honom hörsamma.
\par 11 Han binder vid vinträdet sin åsna, vid ädla rankan sin åsninnas fåle. Han tvår sina kläder i vin, sin mantel i druvors blod.
\par 12 Hans ögon äro dunkla av vin och hans tänder vita av mjölk.
\par 13 Sebulon skall bo vid havets strand, vid stranden, där skeppen ligga; sin sida skall han vända mot Sidon.
\par 14 Isaskar är en stark åsna, som ligger i ro i sin inhägnad.
\par 15 Och han såg att viloplatsen var god, och att landet var ljuvligt; då böjde han sin rygg under bördor och blev en arbetspliktig tjänare.
\par 16 Dan skall skaffa rätt åt sitt folk, han såväl som någon av Israels stammar.
\par 17 Dan skall vara en orm på vägen, en huggorm på stigen, en som biter hästen i foten, så att ryttaren faller baklänges av.
\par 18 HERRE, jag bidar efter din frälsning!
\par 19 Gad skall trängas av skaror, men själv skall han tränga dem på hälarna.
\par 20 Från Aser kommer fetma, honom till mat; konungsliga läckerheter har han att giva.
\par 21 Naftali är en snabb hind; han har sköna ord att giva.
\par 22 Ett ungt fruktträd är Josef, ett ungt fruktträd vid källan; dess grenar nå upp över muren.
\par 23 Bågskyttar oroa honom, de skjuta på honom och ansätta honom;
\par 24 dock förbliver hans båge fast, och hans händer och armar spänstiga, genom dens händer, som är den Starke i Jakob, genom honom som är herden, Israels klippa,
\par 25 genom din faders Gud - han skall hjälpa dig. genom den Allsmäktige - han skall välsigna dig med välsignelser från himmelen därovan, välsignelser från djupet som utbreder sig därnere, välsignelser från bröst och sköte.
\par 26 Din faders välsignelser nå högt, högre än mina förfäders välsignelser, de nå upp till de eviga höjdernas härlighet. De skola komma över Josefs huvud, över dens hjässa, som är en furste bland sina bröder.
\par 27 Benjamin är en glupande ulv; om morgonen förtär han rov, och om aftonen utskiftar han byte."
\par 28 Alla dessa äro Israels stammar, tolv till antalet, och detta är vad deras fader talade till dem, när han välsignade dem; åt var och en av dem gav han sin särskilda välsignelse.
\par 29 Och han bjöd dem och sade till dem: "Jag skall nu samlas till mitt folk; begraven mig bredvid mina fäder, i grottan på hetiten Efrons åker,
\par 30 i den grotta som ligger på åkern i Makpela, gent emot Mamre, i Kanaans land, den åker som Abraham köpte till egen grav av hetiten Efron,
\par 31 där de hava begravit Abraham och hans hustru Sara, där de ock hava begravit Isak och hans hustru Rebecka, och där jag själv har begravit Lea,
\par 32 på den åkern som jämte grottan där köptes av Hets barn."
\par 33 När Jakob hade givit sina söner denna befallning, drog han sina fötter upp i sängen; och han gav upp andan och blev samlad till sina fäder.

\chapter{50}

\par 1 Då föll Josef ned över sin faders ansikte och grät över honom och kysste honom.
\par 2 Och Josef bjöd läkarna som han hade i sin tjänst att de skulle balsamera hans fader; och läkarna balsamerade Israel.
\par 3 Därtill åtgingo fyrtio dagar; så många dagar åtgå nämligen för balsamering. Och egyptierna begräto honom i sjuttio dagar.
\par 4 Men när gråtodagarna efter honom voro förbi, talade Josef till Faraos husfolk och sade: "Om jag har funnit nåd för edra ögon, så framfören till Farao dessa mina ord:
\par 5 Min fader har tagit en ed av mig och sagt: 'När jag är död, begrav mig då i den grav som jag har låtit gräva åt mig i Kanaans land.' Så låt mig nu fara ditupp och begrava min fader; sedan skall jag komma tillbaka igen."
\par 6 Farao svarade: "Far ditupp och begrav din fader, efter den ed som han har tagit av dig."
\par 7 Då for Josef upp för att begrava sin fader, och med honom foro alla Faraos tjänare, de äldste i hans hus och alla de äldste i Egyptens land,
\par 8 därtill allt Josefs husfolk och hans bröder och hans faders husfolk; allenast sina kvinnor och barn, och sina får och fäkreatur lämnade de kvar i landet Gosen.
\par 9 Och med honom foro ditupp både vagnar och ryttare; och det var en mycket stor skara.
\par 10 När de nu kommo till Goren-Haatad, på andra sidan Jordan, höllo de där en mycket stor och högtidlig dödsklagan, och han anställde en sorgefest efter sin fader i sju dagar.
\par 11 Och när landets inbyggare, kananéerna, sågo sorgefesten i Goren-Haatad, sade de: "Det är en högtidlig sorgefest som egyptierna här hålla." Därav fick stället namnet Abel-Misraim; det ligger på andra sidan Jordan.
\par 12 Och hans söner gjorde med honom såsom han hade bjudit dem:
\par 13 hans söner förde honom till Kanaans land och begrovo honom i grottan på åkern i Makpela, den åker som Abraham hade köpt till egen grav av hetiten Efron, gent emot Mamre.
\par 14 Och sedan Josef hade begravit sin fader, vände han tillbaka till Egypten med sina bröder och alla dem som hade farit upp med honom för att begrava hans fader.
\par 15 Men när Josefs bröder sågo att deras fader var död, tänkte de: "Kanhända skall Josef nu bliva hätsk mot oss och vedergälla oss allt det onda som vi hava gjort mot honom."
\par 16 Därför sände de bud till Josef och läto säga: "Din fader bjöd oss så före sin död:
\par 17 'Så skolen I säga till Josef: Käre, förlåt dina bröder vad de hava brutit och syndat, i det att de hava handlat så illa mot dig.' Förlåt alltså nu din faders Guds tjänare vad de hava brutit." Och Josef grät, när de läto säga detta till honom.
\par 18 Sedan kommo ock hans bröder själva och föllo ned för honom och sade: "Se, vi vilja vara tjänare åt dig."
\par 19 Men Josef sade till dem: "Frukten icke. Hållen I då mig för Gud?
\par 20 I tänkten ont mot mig, men Gud har tänkt det till godo, för att låta det ske, som nu har skett, och så behålla mycket folk vid liv.
\par 21 Frukten därför nu icke; jag skall försörja eder och edra kvinnor och barn." Och han tröstade dem och talade vänligt med dem.
\par 22 Och Josef bodde kvar i Egypten med sin faders hus; och Josef blev ett hundra tio år gammal.
\par 23 Och Josef fick se Efraims barn till tredje led; också av Makir, Manasses son, föddes barn i Josefs sköte.
\par 24 Och Josef sade till sina bröder: "Jag dör, men Gud skall förvisso se till eder, och föra eder upp från detta land till det land som han med ed har lovat åt Abraham, Isak och Jakob."
\par 25 Och Josef tog en ed av Israels barn och sade: "När nu Gud ser till eder, fören då mina ben härifrån."
\par 26 Och Josef dog, när han var ett hundra tio år gammal. Och man balsamerade honom, och han lades i en kista, i Egypten.


\end{document}