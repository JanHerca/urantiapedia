\begin{document}

\title{Mark}

Mar 1:1  Detta är begynnelsen av evangelium om Jesus Kristus, Guds Son.
Mar 1:2  Så är skrivet hos profeten Esaias: "Se, jag sänder ut min ängel framför dig, och han skall bereda vägen för dig.
Mar 1:3  Hör rösten av en som ropar i öknen: 'Bereden vägen för Herren, gören stigarna jämna för honom.'"
Mar 1:4  I enlighet härmed uppträdde Johannes döparen i öknen och predikade bättringens döpelse till syndernas förlåtelse.
Mar 1:5  Och hela judiska landet och alla Jerusalems invånare gingo ut till honom och läto döpa sig av honom i floden Jordan, och bekände därvid sina synder.
Mar 1:6  Och Johannes hade kläder av kamelhår och bar en lädergördel om sina länder och levde av gräshoppor och vildhonung.
Mar 1:7  Och han predikade och sade: "Efter mig kommer den som är starkare än jag; jag är icke ens värdig att böja mig ned för att upplösa hans skorem.
Mar 1:8  Jag döper eder med vatten, men han skall döpa eder med helig ande."
Mar 1:9  Och det hände sig vid den tiden att Jesus kom från Nasaret i Galileen. Och han lät döpa sig i Jordan av Johannes.
Mar 1:10  Och strax då han steg upp ur vattnet, såg han himmelen dela sig och Anden såsom en duva sänka sig ned över honom.
Mar 1:11  Och en röst kom från himmelen: "Du är min älskade Son; i dig har jag funnit behag."
Mar 1:12  Strax därefter förde Anden honom ut i öknen.
Mar 1:13  Och han var i öknen i fyrtio dagar och frestades av Satan och levde bland vilddjuren; och änglarna betjänade honom.
Mar 1:14  Men sedan Johannes hade blivit satt i fängelse, kom Jesus till Galileen och predikade Guds evangelium
Mar 1:15  och sade: "Tiden är fullbordad, och Guds rike är nära; gören bättring, och tron evangelium."
Mar 1:16  När han nu gick fram utmed Galileiska sjön, fick han se Simon och Simons broder Andreas kasta ut nät i sjön, ty de voro fiskare.
Mar 1:17  Och Jesus sade till dem: "Följen mig, så skall jag göra eder till människofiskare."
Mar 1:18  Strax lämnade de näten och följde honom.
Mar 1:19  När han hade gått litet längre fram, fick han se Jakob, Sebedeus' son, och Johannes, hans broder, där de sutto i båten, också de, och ordnade sina nät.
Mar 1:20  Och strax kallade han dem till sig; och de lämnade sin fader Sebedeus med legodrängarna kvar i båten och följde honom.
Mar 1:21  Sedan begåvo de sig in i Kapernaum; och strax, på sabbaten, gick han in i synagogan och undervisade.
Mar 1:22  Och folket häpnade över hans förkunnelse; ty han förkunnade sin lära för dem med makt och myndighet, och icke såsom de skriftlärde.
Mar 1:23  Strax härefter befann sig i deras synagoga en man som var besatt av en oren ande. Denne ropade
Mar 1:24  och sade: "Vad har du med oss att göra, Jesus från Nasaret? Har du kommit för att förgöra oss? Jag vet vem du är, du Guds Helige."
Mar 1:25  Men Jesus tilltalade honom strängt och sade: "Tig, och far ut ur honom."
Mar 1:26  Då slet och ryckte den orene anden honom och ropade med hög röst och for ut ur honom.
Mar 1:27  Och alla häpnade, så att de begynte fråga varandra och säga: "Vad är detta? Det är ju en ny lära, med makt och myndighet. Till och med de orena andarna befaller han, och de lyda honom."
Mar 1:28  Och ryktet om honom gick strax ut överallt i hela den kringliggande trakten av Galileen.
Mar 1:29  Och strax då de hade kommit ut ur synagogan, begåvo de sig med Jakob och Johannes till Simons och Andreas' hus.
Mar 1:30  Men Simons svärmoder låg sjuk i feber, och de talade strax med honom om henne.
Mar 1:31  Då gick han fram och tog henne vid handen och reste upp henne; och febern lämnade henne, och hon betjänade dem.
Mar 1:32  Men när solen hade gått ned och det hade blivit afton, förde man till honom alla som voro sjuka eller besatta;
Mar 1:33  och hela staden var församlad utanför dörren.
Mar 1:34  Och han botade många som ledo av olika slags sjukdomar; och han drev ut många onda andar, men tillstadde icke de onda andarna att tala, eftersom de kände honom.
Mar 1:35  Och bittida om morgonen, medan det ännu var mörkt, stod han upp och gick åstad bort till en öde trakt, och bad där.
Mar 1:36  Men Simon och de som voro med honom skyndade efter honom.
Mar 1:37  Och när de funno honom, sade de till honom: "Alla fråga efter dig."
Mar 1:38  Då sade han till dem: "Låt oss draga bort åt annat håll, till de närmaste småstäderna, för att jag också där må predika; ty därför har jag begivit mig ut."
Mar 1:39  Och han gick åstad och predikade i hela Galileen, i deras synagogor, och drev ut de onda andarna.
Mar 1:40  Och en spetälsk man kom fram till honom och föll på knä och bad honom och sade till honom: "Vill du, så kan du göra mig ren."
Mar 1:41  Då förbarmade han sig och räckte ut handen och rörde vid honom och sade till honom: "Jag vill; bliv ren."
Mar 1:42  Och strax vek spetälskan ifrån honom, och han blev ren.
Mar 1:43  Sedan vände Jesus strax bort honom med stränga ord
Mar 1:44  och sade till honom: "Se till, att du icke säger något härom för någon; men gå bort och visa dig för prästen, och frambär för din rening det offer som Moses har påbjudit, till ett vittnesbörd för dem."
Mar 1:45  Men när han kom ut, begynte han ivrigt förkunna och utsprida vad som hade skett, så att Jesus icke mer kunde öppet gå in i någon stad, utan måste hålla sig ute i öde trakter; och dit kom man till honom från alla håll.
Mar 2:1  Några dagar därefter kom han åter till Kapernaum; och när det spordes att han var hemma,
Mar 2:2  församlade sig så mycket folk, att icke ens platsen utanför dörren mer kunde rymma dem; och han förkunnade ordet för dem.
Mar 2:3  Då kommo de till honom med en lam man, som bars dit av fyra män.
Mar 2:4  Och då de för folkets skull icke kunde komma fram till honom med mannen, togo de bort taket över platsen där han var; och sedan de så hade gjort en öppning, släppte de ned sängen, som den lame låg på.
Mar 2:5  När Jesus såg deras tro, sade han till den lame: "Min son, dina synder förlåtas dig."
Mar 2:6  Nu sutto där några skriftlärde, och dessa tänkte i sina hjärtan:
Mar 2:7  "Huru kan denne tala så? Han hädar ju. Vem kan förlåta synder, utom Gud allena?"
Mar 2:8  Strax förnam då Jesus i sin ande att de tänkte så vid sig själva; och han sade till dem: "Huru kunnen I tänka sådant i edra hjärtan?
Mar 2:9  Vilket är lättare, att säga till den lame: 'Dina synder förlåtas dig' eller att säga: 'Stå upp, tag din säng och gå'?
Mar 2:10  Men för att I skolen veta att Människosonen har makt här på jorden att förlåta synder,
Mar 2:11  så säger jag dig" (och härmed vände han sig till den lame): "Stå upp, tag din säng och gå hem."
Mar 2:12  Då stod han upp och tog strax sin säng och gick ut i allas åsyn, så att de alla uppfylldes av häpnad och prisade Gud och sade: "Sådant hava vi aldrig sett."
Mar 2:13  Åter begav han sig ut och gick längs med sjön. Och allt folket kom till honom, och han undervisade dem.
Mar 2:14  När han nu gick där fram, fick han se Levi, Alfeus' son, sitta vid tullhuset. Och han sade till denne: "Följ mig." Då steg han upp och följde honom.
Mar 2:15  När Jesus därefter låg till bords i hans hus, voro där såsom bordsgäster, jämte Jesus och hans lärjungar, också många publikaner och syndare; ty många sådana funnos bland dem som följde honom.
Mar 2:16  Men när de skriftlärde bland fariséerna sågo att han åt med publikaner och syndare, sade de till hans lärjungar: "Huru kan han äta med publikaner och syndare?"
Mar 2:17  När Jesus hörde detta, sade han till dem: "Det är icke de friska som behöva läkare, utan de sjuka. Jag har icke kommit för att kalla rättfärdiga, utan för att kalla syndare."
Mar 2:18  Och Johannes' lärjungar och fariséerna höllo fasta. Och man kom och sade till honom: "Varför fasta icke dina lärjungar, då Johannes' lärjungar och fariséernas lärjungar fasta?"
Mar 2:19  Jesus svarade dem: "Kunna väl bröllopsgästerna fasta, medan brudgummen ännu är hos dem? Nej, så länge de hava brudgummen hos sig, kunna de icke fasta.
Mar 2:20  Men den tid skall komma, då brudgummen tages ifrån dem, och då, på den tiden, skola de fasta. -
Mar 2:21  Ingen syr en lapp av okrympt tyg på en gammal mantel; om någon så gjorde, skulle det isatta nya stycket riva bort ännu mer av den gamla manteln, och hålet skulle bliva värre.
Mar 2:22  Ej heller slår någon nytt vin i gamla skinnläglar; om någon så gjorde, skulle vinet spränga sönder läglarna, så att både vinet och läglarna fördärvades. Nej, nytt vin slår man i nya läglar."
Mar 2:23  Och det hände sig på sabbaten att han tog vägen genom ett sädesfält; och hans lärjungar begynte rycka av axen, medan de gingo.
Mar 2:24  Då sade fariséerna till honom: "Se! Huru kunna de på sabbaten göra vad som icke är lovligt?"
Mar 2:25  Han svarade dem: "Haven I aldrig läst vad David gjorde, när han själv och de som följde honom kommo i nöd och blevo hungriga:
Mar 2:26  huru han då, på den tid Abjatar var överstepräst, gick in i Guds hus och åt skådebröden - fastän det ju icke är lovligt för andra än för prästerna att äta sådant bröd - och huru han jämväl gav åt dem som följde honom?"
Mar 2:27  Därefter sade han till dem: "Sabbaten blev gjord för människans skull, och icke människan för sabbatens skull.
Mar 2:28  Så år då Människosonen herre också över sabbaten."
Mar 3:1  Och han gick åter in i en synagoga. Där var då en man som hade en förvissnad hand.
Mar 3:2  Och de vaktade på honom, för att se om han skulle bota denne på sabbaten; de ville nämligen få något att anklaga honom för.
Mar 3:3  Då sade han till mannen som hade den förvissnade handen: "Stå upp, och kom fram."
Mar 3:4  Sedan sade han till dem: "Vilketdera är lovligt på sabbaten: att göra vad gott är, eller att göra vad ont är, att rädda någons liv, eller att döda?" Men de tego.
Mar 3:5  Då såg han sig omkring på dem med vrede, bedrövad över deras hjärtans förstockelse, och sade till mannen: "Räck ut din hand." Och han räckte ut den; och hans hand blev frisk igen. -
Mar 3:6  Då gingo fariséerna bort och fattade strax, tillsammans med herodianerna, det beslutet att de skulle förgöra honom.
Mar 3:7  Och Jesus drog sig med sina lärjungar undan till sjön, och en stor hop folk följde honom från Galileen.
Mar 3:8  Och från Judeen och Jerusalem och Idumeen och från landet på andra sidan Jordan och från trakterna omkring Tyrus och Sidon kom en stor hop folk till honom, när de fingo höra huru stora ting han gjorde.
Mar 3:9  Och han tillsade sina lärjungar att en båt skulle hållas tillreds åt honom, för folkets skull, för att de icke skulle tränga sig inpå honom.
Mar 3:10  Ty han botade många och blev därför överlupen av alla som hade någon plåga och fördenskull ville röra vid honom.
Mar 3:11  Och när de orena andarna sågo honom, föllo de ned för honom och och ropade och sade: "Du är Guds Son."
Mar 3:12  Men han förbjöd dem strängeligen, åter och åter, att röja honom.
Mar 3:13  Och han gick upp på berget och kallade till sig några som han själv utsåg; och de kommo till honom.
Mar 3:14  Så förordnade han tolv som skulle följa honom, och som han ville sända ut till att predika,
Mar 3:15  och de skulle hava makt att bota sjuka och driva ut onda andar.
Mar 3:16  Han förordnade alltså dessa tolv: Simon, åt vilken han gav tillnamnet Petrus;
Mar 3:17  vidare Jakob, Sebedeus' son, och Johannes, Jakobs broder, åt vilka han gav tillnamnet Boanerges (det betyder tordönsmän);
Mar 3:18  vidare Andreas och Filippus och Bartolomeus och Matteus och Tomas och Jakob, Alfeus' son, och Taddeus och Simon ivraren
Mar 3:19  och Judas Iskariot, densamme som förrådde honom.
Mar 3:20  Och när han kom hem, församlade sig folket åter, så att de icke ens fingo tillfälle att äta.
Mar 3:21  Då nu hans närmaste fingo höra härom, gingo de åstad för att taga vara på honom; ty de menade att han var från sina sinnen.
Mar 3:22  Och de skriftlärde som hade kommit ned från Jerusalem sade att han var besatt av Beelsebul, och att det var med de onda andarnas furste som han drev ut de onda andarna.
Mar 3:23  Då kallade han dem till sig och sade till dem i liknelser: "Huru skulle Satan kunna driva ut Satan?
Mar 3:24  Om ett rike har kommit i strid med sig självt, så kan det riket ju icke hava bestånd;
Mar 3:25  och om ett hus har kommit i strid med sig självt, så skall icke heller det huset kunna äga bestånd.
Mar 3:26  Om alltså Satan har satt sig upp mot sig själv och kommit i strid med sig själv, så kan han icke äga bestånd, utan det är då ute med honom. -
Mar 3:27  Nej, ingen kan gå in i en stark mans hus och plundra honom på hans bohag, såframt han icke förut har bundit den starke. Först därefter kan han plundra hans hus.
Mar 3:28  Sannerligen säger jag eder: Alla andra synder skola bliva människors barn förlåtna, ja ock alla andra hädelser, huru hädiskt de än må tala;
Mar 3:29  men den som hädar den helige Ande, han får icke någonsin förlåtelse, utan är skyldig till evig synd."
Mar 3:30  De hade ju nämligen sagt att han var besatt av en oren ande.
Mar 3:31  Så kommo nu hans moder och hans bröder; och de stannade därutanför och sände bud in till honom för att kalla honom ut.
Mar 3:32  Och mycket folk satt där omkring honom; och man sade till honom: "Se, din moder och dina bröder stå härutanför och fråga efter dig."
Mar 3:33  Då svarade han dem och sade: Vilken är min moder, och vilka äro mina bröder?"
Mar 3:34  Och han såg sig omkring på dem som sutto där runt omkring honom, och han sade: "Se här är min moder, och här äro mina bröder!
Mar 3:35  Den som gör Guds vilja, den är min broder och min syster och min moder."
Mar 4:1  Och han begynte åter undervisa vid sjön. Och där församlade sig en stor hop folk omkring honom. Därför steg han i en båt; och han satt i den ute på sjön, under det att allt folket stod på land utmed sjön.
Mar 4:2  Och han undervisade dem mycket i liknelser och sade till dem i sin undervisning:
Mar 4:3  "Hören! En såningsman gick ut för att så.
Mar 4:4  Då hände sig, när han sådde, att somt föll vid vägen, och fåglarna kommo och åto upp det.
Mar 4:5  Och somt föll på stengrund, där det icke hade mycket jord, och det kom strax upp, eftersom det icke hade djup jord;
Mar 4:6  men när solen hade gått upp, förbrändes det, och eftersom det icke hade någon rot, torkade det bort.
Mar 4:7  Och somt föll bland törnen, och törnena sköto upp och förkvävde det, så att det icke gav någon frukt.
Mar 4:8  Men somt föll i god jord, och det sköt upp och växte och gav frukt och bar trettiofalt och sextiofalt och hundrafalt."
Mar 4:9  Och han tillade: "Den som har öron till att höra, han höre."
Mar 4:10  När han sedan hade dragit sig undan ifrån folket, frågade honom de tolv, och med dem de andra som följde honom, om liknelserna.
Mar 4:11  Då sade han till dem: "Åt eder är Guds rikes hemlighet given, men åt dem som stå utanför meddelas alltsammans i liknelser,
Mar 4:12  för att de 'med seende ögon skola se, och dock intet förnimma, och med hörande öron höra, och dock intet förstå, så att de icke omvända sig och undfå förlåtelse'."
Mar 4:13  Sedan sade han till dem: "Förstån I icke denna liknelse, huru skolen I då kunna fatta alla de andra liknelserna? -
Mar 4:14  Vad såningsmannen sår är ordet.
Mar 4:15  Och att säden såddes vid vägen, det är sagt om dem i vilka ordet väl bliver sått, men när de hava hört det, kommer strax Satan ock tager bort ordet som såddes i dem.
Mar 4:16  Sammalunda förhåller det sig med det som sås på stengrunden: det är sagt om dem, som när de få höra ordet, väl strax taga emot det med glädje,
Mar 4:17  men icke hava någon rot i sig, utan bliva beståndande allenast till en tid; när sedan bedrövelse eller förföljelse påkommer för ordets skull, då komma de strax på fall.
Mar 4:18  Annorlunda förhåller det sig med det som sås bland törnena: det är sagt om dem som väl höra ordet,
Mar 4:19  men låta tidens omsorger och rikedomens bedrägliga lockelse, och begärelser efter andra ting, komma därin och förkväva ordet, så att det bliver utan frukt.
Mar 4:20  Men att det såddes i den goda jorden, det är sagt om dem som både höra ordet och taga emot det, och som bära frukt, trettiofalt och sextiofalt och hundrafalt."
Mar 4:21  Och han sade till dem: "Icke tager man väl fram ett ljus, för att det skall sättas under skäppan eller under bänken; man gör det ju, för att det skall sättas på ljusstaken.
Mar 4:22  Ty intet är fördolt, utom för att det skall bliva uppenbarat; ej heller har något blivit undangömt, utom för att det skall komma i dagen.
Mar 4:23  Om någon har öron till att höra, så höre han."
Mar 4:24  Och han sade till dem: "Akten på vad I hören. Med det mått som I mäten med skall ock mätas åt eder, och ännu mer skall bliva eder tilldelat.
Mar 4:25  Ty den som har, åt honom skall varda givet; men den som icke har, från honom skall tagas också det han har."
Mar 4:26  Och han sade: "Så är det med Guds rike, som när en man sår säd i jorden;
Mar 4:27  och han sover, och han vaknar, och nätter och dagar gå, och säden skjuter upp och växer i höjden, han vet själv icke huru.
Mar 4:28  Av sig själv bär jorden frukt, först strå och sedan ax, och omsider finnes fullbildat vete i axet.
Mar 4:29  När så frukten är mogen, låter han strax lien gå, ty skördetiden är då inne."
Mar 4:30  Och han sade: "Vad skola vi likna Guds rike vid, eller med vilken liknelse skola vi framställa det?
Mar 4:31  Det är såsom ett senapskorn, som när det lägges ned i jorden, är minst av alla frön på jorden;
Mar 4:32  men sedan det är nedlagt, skjuter det upp och bliver störst bland alla kryddväxter och får så stora grenar, att himmelens fåglar kunna bygga sina nästen i dess skugga."
Mar 4:33  I många sådana liknelser förkunnade han ordet för dem, efter deras förmåga att fatta det;
Mar 4:34  och utan liknelse talade han icke till dem. Men för sina lärjungar uttydde han allt, när de voro allena.
Mar 4:35  Samma dag, om aftonen, sade han till dem: "Låt oss fara över till andra stranden."
Mar 4:36  Så läto de folket gå och togo honom med i båten, där han redan förut var; och jämväl andra båtar följde med honom.
Mar 4:37  Då kom en häftig stormvind, och vågorna slogo in i båten, så att båten redan begynte fyllas.
Mar 4:38  Men han själv låg i bakstammen och sov, lutad mot huvudgärden. Då väckte de honom och sade till honom: "Mästare, frågar du icke efter att vi förgås?"
Mar 4:39  När han så hade vaknat, näpste han vinden och sade till sjön: "Tig, var stilla." Och vinden lade sig, och det blev alldeles lugnt.
Mar 4:40  Därefter sade han till dem: "Varför rädens I? Haven I ännu ingen tro?"
Mar 4:41  Och de hade blivit mycket häpna och sade till varandra: "Vem är då denne, eftersom både vinden och sjön äro honom lydiga?"
Mar 5:1  Så kommo de över till gerasenernas land, på andra sidan sjön.
Mar 5:2  Och strax då han hade stigit ur båten, kom en man, som var besatt av en oren ande, emot honom från gravarna där;
Mar 5:3  han hade nämligen sitt tillhåll bland gravarna. Och icke ens med kedjor kunde man numera fängsla honom;
Mar 5:4  ty väl hade han många gånger blivit fängslad med fotbojor och kedjor, men han hade slitit itu kedjorna och brutit sönder fotbojorna, och ingen kunde få makt med honom.
Mar 5:5  Och han vistades alltid, dag och natt, bland gravarna och på bergen och skriade och sargade sig själv med stenar.
Mar 5:6  När denne nu fick se Jesus på avstånd, skyndade han fram och föll ned för honom
Mar 5:7  och ropade med hög röst och sade: "Vad har du med mig att göra, Jesus, du Guds, den Högstes, Son? Jag besvär dig vid Gud, plåga mig icke."
Mar 5:8  Jesus skulle nämligen just säga till honom: "Far ut ur mannen, du orena ande."
Mar 5:9  Då frågade han honom: "Vad är ditt namn?" Han svarade honom: "Legion är mitt namn, ty vi äro många."
Mar 5:10  Och han bad honom enträget att icke driva dem bort ifrån den trakten.
Mar 5:11  Nu gick där vid berget en stor svinhjord i bet.
Mar 5:12  Och de bådo honom och sade: "Sänd oss åstad in i svinen; låt oss få fara in i dem."
Mar 5:13  Och han tillstadde dem det. Då gåvo sig de orena andarna åstad och foro in i svinen. Och hjorden, vid pass två tusen svin, störtade sig utför branten ned i sjön och drunknade i sjön.
Mar 5:14  Men de som vaktade dem flydde och berättade härom i staden och på landsbygden; och folket gick åstad för att se vad det var som hade skett.
Mar 5:15  När de då kommo till Jesus, fingo de se den som hade varit besatt, mannen som hade haft legionen i sig, sitta där klädd och vid sina sinnen; och de betogos av häpnad.
Mar 5:16  Och de som hade åsett händelsen förtäljde för dem vad som hade vederfarits den besatte, och vad som hade skett med svinen.
Mar 5:17  Då begynte folket bedja honom att han skulle gå bort ifrån deras område.
Mar 5:18  När han sedan steg i båten, bad honom mannen som hade varit besatt, att han skulle få följa honom.
Mar 5:19  Men han tillstadde honom det icke, utan sade till honom: "Gå hem till de dina, och berätta för dem huru stora ting Herren har gjort med dig, och huru han har förbarmat sig över dig."
Mar 5:20  Då gick han åstad och begynte förkunna i Dekapolis huru stora ting Jesus hade gjort med honom; och alla förundrade sig.
Mar 5:21  Och när Jesus hade farit över i båten, tillbaka till andra stranden, församlade sig mycket folk omkring honom, där han stod vid sjön.
Mar 5:22  Då kom en synagogföreståndare, vid namn Jairus, dit; och när denne fick se honom, föll han ned för hans fötter
Mar 5:23  och bad honom enträget och sade: "Min dotter ligger på sitt yttersta. Kom och lägg händerna på henne, så att hon bliver hulpen och får leva."
Mar 5:24  Då gick han med mannen; och honom följde mycket folk, som trängde sig inpå honom.
Mar 5:25  Nu var där en kvinna som hade haft blodgång i tolv år,
Mar 5:26  och som hade lidit mycket hos många läkare och kostat på sig allt vad hon ägde, utan att det hade varit henne till något gagn; snarare hade det blivit värre med henne.
Mar 5:27  Hon hade fått höra om Jesus och kom nu i folkhopen, bakom honom, och rörde vid hans mantel.
Mar 5:28  Ty hon tänkte: "Om jag åtminstone får röra vid hans kläder, så bliver jag hulpen."
Mar 5:29  Och strax uttorkade hennes blods källa, och hon kände i sin kropp att hon var botad från sin plåga.
Mar 5:30  Men strax då Jesus inom sig förnam vilken kraft som hade gått ut ifrån honom, vände han sig om i folkhopen och frågade: "Vem rörde vid mina kläder?"
Mar 5:31  Hans lärjungar sade till honom: "Du ser huru folket tränger på, och ändå frågar du: 'Vem rörde vid mig?'"
Mar 5:32  Då såg han sig omkring för att få se den som hade gjort detta.
Mar 5:33  Men kvinnan fruktade och bävade, ty hon visste vad som hade skett med henne; och hon kom fram och föll ned för honom och sade honom hela sanningen.
Mar 5:34  Då sade han till henne: "Min dotter, din tro har hjälpt dig. Gå i frid, och var botad från din plåga."
Mar 5:35  Medan han ännu talade, kommo några från synagogföreståndarens hus och sade: "Din dotter är död; du må icke vidare göra mästaren omak."
Mar 5:36  Men när Jesus märkte vad som talades, sade han till synagogföreståndaren: "Frukta icke, tro allenast."
Mar 5:37  Och han tillstadde ingen att följa med, utom Petrus och Jakob och Johannes, Jakobs broder.
Mar 5:38  Så kommo de till synagogföreståndarens hus, och han fick där se en hop människor som höjde klagolåt och gräto och jämrade sig högt.
Mar 5:39  Och han gick in och sade till dem: "Varför klagen I och gråten? Flickan är icke död, hon sover."
Mar 5:40  Då hånlogo de åt honom. Men han visade ut dem allasammans; och han tog med sig allenast flickans fader och moder och dem som hade fått följa med honom, och gick in dit där flickan låg.
Mar 5:41  Och han tog flickan vid handen och sade till henne: "Talita, kum" (det betyder: "Flicka, jag säger dig, stå upp").
Mar 5:42  Och strax stod flickan upp och begynte gå omkring (hon var nämligen tolv år gammal); och de blevo strax uppfyllda av stor häpnad.
Mar 5:43  Men han förbjöd dem strängeligen att låta någon få veta vad som hade skett. Därefter tillsade han att man skulle giva henne något att äta.
Mar 6:1  Och han gick bort därifrån och begav sig till sin fädernestad; och hans lärjungar följde honom.
Mar 6:2  Och när det blev sabbat, begynte han undervisa i synagogan. Och folket häpnade, när de hörde honom; de sade: "Varifrån har han fått detta? Och vad är det för vishet som har blivit honom given? Och dessa stora kraftgärningar som göras genom honom, varifrån komma de?
Mar 6:3  Är då denne icke timmermannen, han som är Marias son och broder till Jakob och Joses och Judas och Simon? Och bo icke hans systrar här hos oss?" Så blev han för dem en stötesten.
Mar 6:4  Då sade Jesus till dem: "En profet är icke föraktad utom i sin fädernestad och bland sina fränder och i sitt eget hus."
Mar 6:5  Och han kunde icke där göra någon kraftgärning, utom att han botade några få sjuka, genom att lägga händerna på dem.
Mar 6:6  Och han förundrade sig över deras otro. Sedan gick han omkring i byarna, från den ena byn till den andra, och undervisade.
Mar 6:7  Och han kallade till sig de tolv och sände så ut dem, två och två, och gav dem makt över de orena andarna.
Mar 6:8  Och han bjöd dem att icke taga något med sig på vägen, utom allenast en stav: icke bröd, icke ränsel, icke penningar i bältet.
Mar 6:9  Sandaler finge de dock hava på fötterna, men de skulle icke bära dubbla livklädnader.
Mar 6:10  Och han sade till dem: "När I haven kommit in i något hus, så stannen där, till dess I lämnen den orten.
Mar 6:11  Och om man på något ställe icke tager emot eder och icke hör på eder, så gån bort därifrån, och skudden av stoftet som är under edra fötter, till ett vittnesbörd mot dem."
Mar 6:12  Och de gingo ut och predikade att man skulle göra bättring;
Mar 6:13  och de drevo ut många onda andar och smorde många sjuka med olja och botade dem.
Mar 6:14  Och konung Herodes fick höra om honom, ty hans namn hade blivit känt. Man sade: "Det är Johannes döparen, som har uppstått från de döda, och därför verka dessa krafter i honom."
Mar 6:15  Men andra sade: "Det är Elias." Andra åter sade: "Det är en profet, lik de andra profeterna."
Mar 6:16  Men när Herodes hörde detta, sade han: "Det är Johannes, den som jag lät halshugga. Han bar uppstått från de döda."
Mar 6:17  Herodes hade nämligen sänt åstad och låtit gripa Johannes och binda honom och sätta honom i fängelse, för Herodias', sin broder Filippus' hustrus, skull. Ty henne hade Herodes tagit till äkta,
Mar 6:18  och Johannes hade då sagt till honom: "Det är icke lovligt för dig att hava din broders hustru."
Mar 6:19  Därför hyste nu Herodes agg till honom och ville döda honom, men han hade icke makt därtill.
Mar 6:20  Ty Herodes förstod att Johannes var en rättfärdig och helig man, och han fruktade för honom och gav honom sitt beskydd. Och när han hade hört honom, blev han betänksam i många stycken; och han hörde honom gärna.
Mar 6:21  Men så kom en läglig dag, i det att Herodes på sin födelsedag gjorde ett gästabud för sina stormän och för krigsöverstarna och de förnämsta männen i Galileen.
Mar 6:22  Då gick Herodias' dotter ditin och dansade; och hon behagade Herodes och hans bordsgäster. Och konungen sade till flickan: "Begär av mig vadhelst du vill, så skall jag giva dig det."
Mar 6:23  Ja, han lovade henne detta med ed och sade: "Vadhelst du begär av mig, det skall jag giva dig, ända till hälften av mitt rike."
Mar 6:24  Då gick hon ut och frågade sin moder: "Vad skall jag begära?" Hon svarade: "Johannes döparens huvud."
Mar 6:25  Och strax skyndade hon in till konungen och framställde sin begäran och sade: "Jag vill att du nu genast giver mig på ett fat Johannes döparens huvud."
Mar 6:26  Då blev konungen mycket bekymrad, men för edens och för bordsgästernas skull ville han icke avvisa henne.
Mar 6:27  Alltså sände konungen strax en drabant med befallning att hämta hans huvud. Och denne gick åstad och halshögg honom i fängelset
Mar 6:28  och bar sedan fram hans huvud på ett fat och gav det åt flickan, och flickan gav det åt sin moder.
Mar 6:29  Men när hans lärjungar fingo höra härom, kommo de och togo hans döda kropp och lade den i en grav.
Mar 6:30  Och apostlarna församlade sig hos Jesus och omtalade för honom allt vad de hade gjort, och allt vad de hade lärt folket.
Mar 6:31  Då sade han till dem: "Kommen nu I med mig bort till en öde trakt, där vi få vara allena, och vilen eder något litet." Ty de fingo icke ens tid att äta; så många voro de som kommo och gingo.
Mar 6:32  De foro alltså i båten bort till en öde trakt, där de kunde vara allena.
Mar 6:33  Men man såg dem fara sin väg, och många fingo veta det; och från alla städer strömmade då människor tillsammans dit landvägen och kommo fram före dem.
Mar 6:34  När han så steg i land, fick han se att där var mycket folk. Då ömkade han sig över dem, eftersom de voro "lika får som icke hade någon herde"; och han begynte undervisa dem i mångahanda stycken.
Mar 6:35  Men när det redan var långt lidet på dagen, trädde hans lärjungar fram till honom och sade: "Trakten är öde, och det är redan långt lidet på dagen.
Mar 6:36  Låt dem skiljas åt, så att de kunna gå bort i gårdarna och byarna häromkring och köpa sig något att äta."
Mar 6:37  Men han svarade och sade till dem: "Given I dem att äta." De svarade honom: "Skola vi då gå bort och köpa bröd för två hundra silverpenningar och giva dem att äta?"
Mar 6:38  Men han sade till dem: "Huru många bröd haven I? Gån och sen efter." Sedan de hade gjort så, svarade de: "Fem, och därtill två fiskar."
Mar 6:39  Då befallde han dem att låta alla i skilda matlag lägga sig ned i gröna gräset.
Mar 6:40  Och de lägrade sig där i skilda hopar, hundra eller femtio i var.
Mar 6:41  Därefter tog han de fem bröden och de två fiskarna och såg upp till himmelen och välsignade dem. Och han bröt bröden och gav dem åt lärjungarna, för att de skulle lägga fram åt folket; också de två fiskarna delade han mellan dem alla.
Mar 6:42  Och de åto alla och blevo mätta.
Mar 6:43  Sedan samlade man upp överblivna brödstycken, tolv korgar fulla, därtill ock kvarlevor av fiskarna.
Mar 6:44  Och det var fem tusen män som hade ätit.
Mar 6:45  Strax därefter nödgade han sina lärjungar att stiga i båten och i förväg fara över till Betsaida på andra stranden, medan han själv tillsåg att folket skildes åt.
Mar 6:46  Och när han hade tagit avsked av folket, gick han därifrån upp på berget för att bedja.
Mar 6:47  När det så hade blivit afton, var båten mitt på sjön, och han var ensam kvar på land.
Mar 6:48  Och han såg dem vara hårt ansatta, där de rodde fram, ty vinden låg emot dem. Vid fjärde nattväkten kom han då till dem, gående på sjön, och skulle just gå förbi dem.
Mar 6:49  Men när de fingo se honom gå på sjön, trodde de att det var en vålnad och ropade högt;
Mar 6:50  ty de sågo honom alla och blevo förfärade. Men han begynte strax tala med dem och sade till dem: "Varen vid gott mod; det är jag, varen icke förskräckta."
Mar 6:51  Därefter steg han upp till dem i båten, och vinden lade sig. Och de blevo uppfyllda av stor häpnad;
Mar 6:52  ty de hade icke kommit till förstånd genom det som hade skett med bröden, utan deras hjärtan voro förstockade.
Mar 6:53  När de hade farit över till andra stranden, kommo de till Gennesarets land och lade till där.
Mar 6:54  Och när de stego ur båten, kände man strax igen honom;
Mar 6:55  och man skyndade omkring med bud i hela den trakten, och folket begynte då överallt bära de sjuka på sängar dit där man hörde att han var.
Mar 6:56  Och varhelst han gick in i någon by eller någon stad eller någon gård, där lade man de sjuka på de öppna platserna. Och de bådo honom att åtminstone få röra vid hörntofsen på hans mantel; och alla som rörde vid den blevo hulpna.
Mar 7:1  Och fariséerna, så ock några skriftlärde som hade kommit från Jerusalem, församlade sig omkring honom;
Mar 7:2  och de fingo då se några av hans lärjungar äta med "orena", det är otvagna, händer.
Mar 7:3  Nu är det så med fariséerna och alla andra judar, att de icke äta något utan att förut, till åtlydnad av de äldstes stadgar, noga hava tvagit sina händer,
Mar 7:4  likasom de icke heller, när de komma från torget, äta något utan att förut hava tvagit sig; många andra stadgar finnas ock, som de av ålder pläga hålla, såsom att skölja bägare och träkannor och kopparskålar.
Mar 7:5  Därför frågade honom nu fariséerna och de skriftlärde: "Varför vandra icke dina lärjungar efter de äldstes stadgar, utan äta med orena händer?"
Mar 7:6  Men han svarade dem: "Rätt profeterade Esaias om eder, I skrymtare, såsom det är skrivet: 'Detta folk ärar mig med sina läppar, men deras hjärtan äro långt ifrån mig;
Mar 7:7  och fåfängt dyrka de mig, eftersom de läror de förkunna äro människobud.'
Mar 7:8  I sätten Guds bud å sido och hållen människors stadgar."
Mar 7:9  Ytterligare sade han till dem: "Rätt så; I upphäven Guds bud för att hålla edra egna stadgar!
Mar 7:10  Moses har ju sagt: 'Hedra din fader och din moder' och 'Den som smädar sin fader eller sin moder, han skall döden dö.'
Mar 7:11  Men I sägen: om en son säger till sin fader eller sin moder: 'Vad du av mig kunde hava fått till hjälp, det giver jag i stället såsom korban' (det betyder offergåva),
Mar 7:12  då kunnen I icke tillstädja honom att vidare göra något för sin fader eller sin moder.
Mar 7:13  På detta sätt gören I Guds budord om intet genom edra fäderneärvda stadgar. Och mycket annat sådant gören I."
Mar 7:14  Därefter kallade han åter folket till sig och sade till dem: "Hören mig alla och förstån.
Mar 7:15  Intet som utifrån går in i människan kan orena henne, men vad som går ut ifrån människan, detta är det som orenar henne."
Mar 7:16  Den som har öron till att höra, han höre.
Mar 7:17  När han sedan hade lämnat folket och kommit inomhus, frågade hans lärjungar honom om detta bildliga tal.
Mar 7:18  Han svarade dem: "Ären då också I så utan förstånd? Insen I icke att intet som utifrån går in i människan kan orena henne,
Mar 7:19  eftersom det icke går in i hennes hjärta, utan ned i buken, och har sin naturliga utgång?" Härmed förklarade han all mat för ren.
Mar 7:20  Och han tillade: "Vad som går ut ifrån människan, detta är det som orenar människan.
Mar 7:21  Ty inifrån, från människornas hjärtan, utgå deras onda tankar, otukt, tjuveri, mord,
Mar 7:22  äktenskapsbrott, girighet, ondska, svek, lösaktighet, avund, hädelse, övermod, oförsynt väsende.
Mar 7:23  Allt detta onda går inifrån ut, och det orenar människan."
Mar 7:24  Och han stod upp och begav sig bort därifrån till Tyrus' område. Där gick han in i ett hus och ville icke att någon skulle få veta det. Dock kunde han icke förbliva obemärkt,
Mar 7:25  utan en kvinna, vilkens dotter var besatt av en oren ande, kom, strax då hon hade fått höra om honom, och föll ned för hans fötter;
Mar 7:26  det var en grekisk kvinna av syrofenicisk härkomst. Och hon bad honom att han skulle driva ut den onde anden ur hennes dotter.
Mar 7:27  Men han sade till henne: "Låt barnen först bliva mättade; det är ju otillbörligt att taga brödet från barnen och kasta det åt hundarna."
Mar 7:28  Hon svarade och sade till honom: "Ja, Herre; också äta hundarna under bordet allenast av barnens smulor."
Mar 7:29  Då sade han till henne: "För det ordets skull säger jag dig: Gå; den onde anden har farit ut ur din dotter."
Mar 7:30  Och när hon kom hem, fann hon flickan ligga på sängen och såg att den onde anden hade farit ut.
Mar 7:31  Sedan begav han sig åter bort ifrån Tyrus' område och tog vägen över Sidon och kom, genom Dekapolis' område, till Galileiska sjön.
Mar 7:32  Och man förde till honom en som var döv och nästan stum och bad honom att lägga handen på denne.
Mar 7:33  Då tog han honom avsides ifrån folket och satte sina fingrar i hans öron och spottade och rörde vid hans tunga
Mar 7:34  och såg upp till himmelen, suckade och sade till honom: "Effata" (det betyder: "Upplåt dig").
Mar 7:35  Då öppnades hans öron, och hans tungas band löstes, och han talade redigt och klart.
Mar 7:36  Och Jesus förbjöd dem att omtala detta för någon; men ju mer han förbjöd dem, dess mer förkunnade de det.
Mar 7:37  Och folket häpnade övermåttan och sade: "Allt har han väl beställt: de döva låter han höra och de stumma tala."
Mar 8:1  Då vid samma tid åter mycket folk hade kommit tillstädes, och de icke hade något att äta, kallade han sina lärjungar till sig och sade till dem:
Mar 8:2  "Jag ömkar mig över folket, ty det är redan tre dagar som de hava dröjt kvar hos mig, och de hava intet att äta.
Mar 8:3  Om jag nu låter dem fastande gå ifrån mig hem, så uppgivas de på vägen; somliga av dem hava ju kommit långväga ifrån."
Mar 8:4  Då svarade hans lärjungar honom: "Varifrån skall man här i en öken kunna få bröd till att mätta dessa med?"
Mar 8:5  Han frågade dem: "Huru många bröd haven I?" De svarade: "Sju."
Mar 8:6  Då tillsade han folket att lägra sig på marken. Ock han tog de sju bröden, tackade Gud och bröt dem och gav åt sina lärjungar, för att de skulle lägga fram dem; och de lade fram åt folket.
Mar 8:7  De hade ock några få småfiskar; och när han hade välsignat dessa, bjöd han att man likaledes skulle lägga fram dem.
Mar 8:8  Så åto de och blevo mätta. Och man samlade sedan upp sju korgar med överblivna stycken.
Mar 8:9  Men antalet av dem som voro tillstädes var vid pass fyra tusen. Sedan lät han dem skiljas åt.
Mar 8:10  Och strax därefter steg han i båten med sina lärjungar och for till trakten av Dalmanuta.
Mar 8:11  Och fariséerna kommo ditut och begynte disputera med honom; de ville sätta honom på prov och begärde av honom något tecken från himmelen.
Mar 8:12  Då suckade han ur sin andes djup och sade: "Varför begär detta släkte ett tecken? Sannerligen säger jag eder: Åt detta släkte skall intet tecken givas."
Mar 8:13  Så lämnade han dem och steg åter i båten och for över till andra stranden.
Mar 8:14  Och de hade förgätit att taga med sig bröd; icke mer än ett enda bröd hade de med sig i båten.
Mar 8:15  Och han bjöd dem och sade: "Sen till, att I tagen eder till vara för fariséernas surdeg och för Herodes' surdeg."
Mar 8:16  Då talade de med varandra om att de icke hade bröd med sig.
Mar 8:17  Men när han märkte detta, sade han till dem: "Varför talen I om att I icke haven bröd med eder? Fatten och förstån I då ännu ingenting? Äro edra hjärtan så förstockade?
Mar 8:18  I haven ju ögon; sen I då icke? I haven ju öron; hören I då icke?
Mar 8:19  Och kommen I icke ihåg huru många korgar fulla av stycken I samladen upp, när jag bröt de fem bröden åt de fem tusen?" De svarade honom: "Tolv."
Mar 8:20  "Och när jag bröt de sju bröden åt de fyra tusen, huru många korgar fulla av stycken samladen I då upp?" De svarade: "Sju."
Mar 8:21  Då sade han till dem: "Förstån I då ännu ingenting?"
Mar 8:22  Därefter kommo de till Betsaida. Och man förde till honom en som var blind och bad honom att han skulle röra vid denne.
Mar 8:23  Då tog han den blinde vid handen och ledde honom utanför byn; sedan spottade han på hans ögon och lade händerna på honom och frågade honom: "Ser du något?"
Mar 8:24  Han såg då upp och svarade: "Jag kan urskilja människorna; jag ser dem gå omkring, men de likna träd."
Mar 8:25  Därefter lade han åter händerna på hans ögon, och nu såg han tydligt och var botad och kunde jämväl på långt håll se allting klart.
Mar 8:26  Och Jesus bjöd honom gå hem och sade: "Gå icke ens in i byn."
Mar 8:27  Och Jesus gick med sina lärjungar bort till byarna vid Cesarea Filippi. På vägen dit frågade han sina lärjungar och sade till dem: "Vem säger folket mig vara?"
Mar 8:28  De svarade och sade: "Johannes döparen; andra säga dock Elias, andra åter säga: 'Det är en av profeterna.'"
Mar 8:29  Då frågade han dem: "Vem sägen då I mig vara?" Petrus svarade och sade till honom: "Du är Messias."
Mar 8:30  Då förbjöd han dem strängeligen att för någon säga detta om honom.
Mar 8:31  Sedan begynte han undervisa dem om att Människosonen måste lida mycket, och att han skulle bliva förkastad av de äldste och översteprästerna och de skriftlärde, och att han skulle bliva dödad, men att han tre dagar därefter skulle uppstå igen.
Mar 8:32  Och han talade detta i oförtäckta ordalag. Då tog Petrus honom avsides och begynte ivrigt motsäga honom.
Mar 8:33  Men han vände sig om, och när han då såg sina lärjungar, talade han strängt till Petrus och sade: "Gå bort, Satan, och stå mig icke i vägen; ty dina tankar äro icke Guds tankar, utan människotankar."
Mar 8:34  Och han kallade till sig folket jämte sina lärjungar och sade till dem: "Om någon vill efterfölja mig, så försake han sig själv och tage sitt kors på sig; så följe han mig.
Mar 8:35  Ty den som vill bevara sitt liv, han skall mista det; men den som mister sitt liv, för min och för evangelii skull, han skall bevara det.
Mar 8:36  Och vad hjälper det en människa, om hon vinner hela världen, men förlorar sin själ?
Mar 8:37  Och vad kan en människa giva till lösen för sin själ?
Mar 8:38  Den som blyges för mig och för mina ord, i detta trolösa och syndiga släkte, för honom skall ock Människosonen blygas, när han kommer i sin Faders härlighet med de heliga änglarna."
Mar 9:1  Ytterligare sade han till dem: "Sannerligen säger jag eder: Bland dem som här stå finnas några som icke skola smaka döden, förrän de få se Guds rike vara kommet i sin kraft."
Mar 9:2  Sex dagar därefter tog Jesus med sig Petrus och Jakob och Johannes och förde dem ensamma upp på ett högt berg, där de voro allena. Och hans utseende förvandlades inför dem;
Mar 9:3  och hans kläder blevo glänsande och mycket vita, så att ingen valkare på jorden kan göra kläder så vita.
Mar 9:4  Och för dem visade sig Elias jämte Moses, och dessa samtalade med Jesus.
Mar 9:5  Då tog Petrus till orda och sade till Jesus: "Rabbi, här är oss gott att vara; låt oss göra tre hyddor, åt dig en och åt Moses en och åt Elias en."
Mar 9:6  Han visste nämligen icke vad han skulle säga; så stor var deras förskräckelse.
Mar 9:7  Då kom en sky som överskyggde dem, och ur skyn kom en röst: "Denne är min älskade Son; hören honom."
Mar 9:8  Och plötsligt märkte de, när de sågo sig omkring, att där icke mer fanns någon hos dem utom Jesus allena.
Mar 9:9  Då de sedan gingo ned från berget, bjöd han dem att de icke, förrän Människosonen hade uppstått från de döda, skulle för någon omtala vad de hade sett.
Mar 9:10  Och de lade märke till det ordet och begynte tala med varandra om vad som kunde menas med att han skulle uppstå från de döda.
Mar 9:11  Och de frågade honom och sade: "De skriftlärde säga ju att Elias först måste komma?"
Mar 9:12  Han svarade dem: "Elias måste visserligen först komma och upprätta allt igen. Men huru kan det då vara skrivet om Människosonen att han skall lida mycket och bliva föraktad?
Mar 9:13  Dock, jag säger eder att Elias redan har kommit; och de förforo mot honom alldeles såsom de ville, och såsom det var skrivet att det skulle gå honom."
Mar 9:14  När de därefter kommo till lärjungarna, sågo de att mycket folk var samlat omkring dem, och att några skriftlärde disputerade med dem.
Mar 9:15  Och strax då allt folket fick se honom, blevo de mycket häpna och skyndade fram och hälsade honom.
Mar 9:16  Då frågade han dem: "Varom disputeren I med dem?"
Mar 9:17  Och en man i folkhopen svarade honom: "Mästare, jag har fört till dig min son, som är besatt av en stum ande.
Mar 9:18  Och varhelst denne får fatt i honom kastar han omkull honom, och fradgan står gossen om munnen, och han gnisslar med tänderna och bliver såsom livlös. Nu bad jag dina lärjungar att de skulle driva ut honom, men de förmådde det icke."
Mar 9:19  Då svarade han dem och sade: "O du otrogna släkte, huru länge måste jag vara hos eder? Huru länge måste jag härda ut med eder? Fören honom till mig."
Mar 9:20  Och de förde honom till Jesus. Och strax då han fick se Jesus, slet och ryckte anden honom, och han föll ned på jorden och vältrade sig, under det att fradgan stod honom om munnen.
Mar 9:21  Jesus frågade då hans fader: "Huru länge har det varit så med honom?" Han svarade: "Alltsedan han var ett litet barn;
Mar 9:22  och det har ofta hänt att han har kastat honom än i elden, än i vattnet, för att förgöra honom. Men om du förmår något, så förbarma dig över oss och hjälp oss."
Mar 9:23  Då sade Jesus till honom: "Om jag förmår, säger du. Allt förmår den som tror."
Mar 9:24  Strax ropade gossens fader och sade: "Jag tror! Hjälp min otro."
Mar 9:25  Men när Jesus såg att folk strömmade tillsammans dit, tilltalade han den orene anden strängt och sade till honom: "Du stumme och döve ande, jag befaller dig: Far ut ur honom, och kom icke mer in i honom."
Mar 9:26  Då skriade han och slet och ryckte gossen svårt och for ut; och gossen blev såsom död, så att folket menade att han verkligen var död.
Mar 9:27  Men Jesus tog honom vid handen och reste upp honom; och han stod då upp.
Mar 9:28  När Jesus därefter hade kommit inomhus, frågade hans lärjungar honom, då de nu voro allena: "Varför kunde icke vi driva ut honom?"
Mar 9:29  Han svarade dem: "Detta slag kan icke drivas ut genom något annat än bön och fasta."
Mar 9:30  Och de gingo därifrån och vandrade genom Galileen; men han ville icke att någon skulle få veta det.
Mar 9:31  Han undervisade nämligen sina lärjungar och sade till dem: "Människosonen skall bliva överlämnad i människors händer, och man skall döda honom; men tre dagar efter det att han har blivit dödad skall han uppstå igen."
Mar 9:32  Och de förstodo icke vad han sade, men de fruktade att fråga honom.
Mar 9:33  Och de kommo till Kapernaum. Och när han hade kommit dit där han bodde, frågade han dem: "Vad var det I samtaladen om på vägen?"
Mar 9:34  Men de tego, ty de hade på vägen talat med varandra om vilken som vore störst.
Mar 9:35  Då satte han sig ned och kallade till sig de tolv och sade till dem: "Om någon vill vara den förste, så vare han den siste av alla och allas tjänare."
Mar 9:36  Och han tog ett barn och ställde det mitt ibland dem; sedan tog han det upp i famnen och sade till dem:
Mar 9:37  "Den som tager emot ett sådant barn i mitt namn, han tager emot mig, och den som tager emot mig, han tager icke emot mig, utan honom som har sänt mig."
Mar 9:38  Johannes sade till honom: "Mästare, vi sågo huru en man som icke följer oss drev ut onda andar genom ditt namn; och vi ville hindra honom, eftersom han icke följde oss."
Mar 9:39  Men Jesus sade: "Hindren honom icke; ty ingen som genom mitt namn har gjort en kraftgärning kan strax därefter tala illa om mig.
Mar 9:40  Ty den som icke är emot oss, han är för oss.
Mar 9:41  Ja, den som giver eder en bägare vatten att dricka, därför att I hören Kristus till - sannerligen säger jag eder: Han skall ingalunda gå miste om sin lön.
Mar 9:42  Och den som förför en av dessa små som tro, för honom vore det bättre, om en kvarnsten hängdes om hans hals och han kastades i havet.
Mar 9:43  Om nu din hand är dig till förförelse, så hugg av den. Det är bättre för dig att ingå i livet lytt, än att hava båda händerna i behåll och komma till Gehenna, till den eld som icke utsläckes.
Mar 9:44  Där 'deras mask icke dör och elden icke utsläckes'.
Mar 9:45  Och om din fot är dig till förförelse, så hugg av den. Det är bättre för dig att ingå i livet halt, än att hava båda fötterna i behåll och kastas i Gehenna.
Mar 9:46  Där 'deras mask icke dör och elden icke utsläckes'.
Mar 9:47  Och om ditt öga är dig till förförelse, så riv ut det. Det är bättre för dig att ingå i Guds rike enögd, än att hava båda ögonen i behåll och kastas i Gehenna,
Mar 9:48  där 'deras mask icke dör och elden icke utsläckes'.
Mar 9:49  Ty var människa måste saltas med eld. Ty varje offer skall med salt saltas.
Mar 9:50  Saltet är en god sak; men om saltet mister sin sälta, varmed skolen I då återställa dess kraft? - Haven salt i eder och hållen frid inbördes."
Mar 10:1  Och han stod upp och begav sig därifrån, genom landet på andra sidan Jordan, till Judeens område. Och mycket folk församlades åter omkring honom, och åter undervisade han dem, såsom hans sed var.
Mar 10:2  Då ville några fariséer snärja honom, och de trädde fram och frågade honom om det vore lovligt för en man att skilja sig från sin hustru.
Mar 10:3  Men han svarade och sade till dem: "Vad har Moses bjudit eder?"
Mar 10:4  De sade: "Moses tillstadde att en man fick skriva skiljebrev åt sin hustru och så skilja sig från henne."
Mar 10:5  Då sade Jesus till dem: "För edra hjärtans hårdhets skull skrev han åt eder detta bud.
Mar 10:6  Men redan vid världens begynnelse 'gjorde Gud dem till man och kvinna'.
Mar 10:7  'Fördenskull skall en man övergiva sin fader och sin moder.
Mar 10:8  Och de tu skola varda ett kött.' Så äro de icke mer två, utan ett kött.
Mar 10:9  Vad nu Gud har sammanfogat, det må människan icke åtskilja."
Mar 10:10  När de sedan hade kommit hem, frågade hans lärjungar honom åter om detsamma.
Mar 10:11  Och han svarade dem: "Den som skiljer sig från sin hustru och tager sig en annan hustru, han begår äktenskapsbrott mot henne.
Mar 10:12  Och om en hustru skiljer sig från sin man och tager sig en annan man, då begår hon äktenskapsbrott.
Mar 10:13  Och man bar fram barn till honom, för att han skulle röra vid dem; men lärjungarna visade bort dem.
Mar 10:14  När Jesus såg detta, blev han misslynt och sade till dem: "Låten barnen komma till mig, och förmenen dem det icke; ty Guds rike hör sådana till.
Mar 10:15  Sannerligen säger jag eder: Den som icke tager emot Guds rike såsom ett barn, han kommer aldrig ditin."
Mar 10:16  Och han tog dem upp i famnen och lade händerna på dem och välsignade dem.
Mar 10:17  När han sedan begav sig åstad för att fortsätta sin väg, skyndade en man fram och föll på knä för honom och frågade honom: "Gode Mästare, vad skall jag göra för att få evigt liv till arvedel?"
Mar 10:18  Jesus sade till honom: "Varför kallar du mig god? Ingen är god utom Gud allena.
Mar 10:19  Buden känner du: 'Du skall icke dräpa', 'Du skall icke begå äktenskapsbrott', 'Du skall icke stjäla', 'Du skall icke bära falskt vittnesbörd', 'Du skall icke undanhålla någon vad honom tillkommer', Hedra din fader och din moder.'"
Mar 10:20  Då svarade han honom: "Mästare, allt detta har jag hållit från min ungdom."
Mar 10:21  Då såg Jesus på honom och fick kärlek till honom och sade till honom: "Ett fattas dig: gå bort och sälj allt vad du äger och giv åt de fattiga; då skall du få en skatt i himmelen. Och kom sedan och följ mig."
Mar 10:22  Men han blev illa till mods vid det talet och gick bedrövad bort, ty han hade många ägodelar.
Mar 10:23  Då såg Jesus sig omkring och sade till sina lärjungar: "Huru svårt är det icke för dem som hava penningar att komma in i Guds rike!"
Mar 10:24  Men lärjungarna häpnade vid hans ord. Då tog Jesus åter till orda och sade till dem: "Ja, mina barn, huru svårt är det icke att komma in i Guds rike!
Mar 10:25  Det är lättare för en kamel att komma igenom ett nålsöga, än för den som är rik att komma in i Guds rike."
Mar 10:26  Då blevo de ännu mer häpna och sade till varandra: "Vem kan då bliva frälst?"
Mar 10:27  Jesus såg på dem och sade: "För människor är det omöjligt, men icke för Gud, ty för Gud är allting möjligt."
Mar 10:28  Då tog Petrus till orda och sade till honom: "Se, vi hava övergivit allt och följt dig."
Mar 10:29  Jesus svarade: "Sannerligen säger jag eder: Ingen som för min och evangelii skull har övergivit hus, eller bröder eller systrar, eller moder eller fader, eller barn, eller jordagods,
Mar 10:30  ingen sådan finnes, som icke skall få hundrafalt igen: redan här i tiden hus, och bröder och systrar, och mödrar och barn, och jordagods, mitt under förföljelser, och i den tillkommande tidsåldern evigt liv.
Mar 10:31  Men många som äro de första skola bliva de sista, medan de sista bliva de första."
Mar 10:32  Och de voro på vägen upp till Jerusalem. Och Jesus gick före dem, och de gingo där bävande; och de som följde med dem voro uppfyllda av fruktan. Då tog han åter till sig de tolv och begynte tala till dem om vad som skulle övergå honom:
Mar 10:33  "Se, vi gå nu upp till Jerusalem, och Människosonen skall bliva överlämnad åt översteprästerna och de skriftlärde, och de skola döma honom till döden och överlämna honom åt hedningarna,
Mar 10:34  och dessa skola begabba honom och bespotta honom och gissla honom och döda honom; men tre dagar därefter skall han uppstå igen."
Mar 10:35  Då trädde Jakob och Johannes, Sebedeus' söner, fram till honom och sade till honom: "Mästare, vi skulle vilja att du läte oss få vad vi nu tänka begära av dig."
Mar 10:36  Han frågade dem: "Vad viljen I då att jag skall låta eder få?"
Mar 10:37  De svarade honom: "Låt den ene av oss i din härlighet få sitta på din högra sida, och den andre på din vänstra."
Mar 10:38  Men Jesus sade till dem: "I veten icke vad I begären. Kunnen I dricka den kalk som jag dricker, eller genomgå det dop som jag genomgår?"
Mar 10:39  De svarade honom: "Det kunna vi." Då sade Jesus till dem: "Ja, den kalk jag dricker skolen I få dricka, och det dop jag genomgår skolen I genomgå,
Mar 10:40  men platsen på min högra sida och platsen på min vänstra tillkommer det icke mig att bortgiva, utan de skola tillfalla dem för vilka så är bestämt."
Mar 10:41  När de tio andra hörde detta, blevo de misslynta på Jakob och Johannes.
Mar 10:42  Då kallade Jesus dem till sig och sade till dem: "I veten att de som räknas för folkens furstar uppträda mot dem såsom herrar, och att deras mäktige låta dem känna sin myndighet.
Mar 10:43  Men så är det icke bland eder; utan den som vill bliva störst bland eder, han vare de andras tjänare,
Mar 10:44  och den som vill vara främst bland eder, han vare allas dräng.
Mar 10:45  Också Människosonen har ju kommit, icke för att låta tjäna sig, utan för att tjäna och giva sitt liv till lösen för många."
Mar 10:46  Och de kommo till Jeriko. Men när han åter gick ut ifrån Jeriko, följd av sina lärjungar och en ganska stor hop folk, satt där vid vägen en blind tiggare, Bartimeus, Timeus' son.
Mar 10:47  När denne hörde att det var Jesus från Nasaret, begynte han ropa och säga: "Jesus, Davids son, förbarma dig över mig."
Mar 10:48  Och många tillsade honom strängeligen att han skulle tiga; men han ropade ännu mycket mer: "Davids son, förbarma dig över mig."
Mar 10:49  Då stannade Jesus och sade: "Kallen honom hit." Och de kallade på den blinde och sade till honom: "Var vid gott mod, stå upp; han kallar dig till sig."
Mar 10:50  Då kastade han av sig sin mantel och stod upp med hast och kom fram till Jesus.
Mar 10:51  Och Jesus talade till honom och sade: "Vad vill du att jag skall göra dig?" Den blinde svarade honom: "Rabbuni, låt mig få min syn."
Mar 10:52  Jesus sade till honom: "Gå; din tro har hjälpt dig." Och strax fick han sin syn och följde honom på vägen.
Mar 11:1  När de nu nalkades Jerusalem och voro nära Betfage och Betania vid Oljeberget, sände han åstad två av sina lärjungar
Mar 11:2  och sade till dem: "Gån in i byn som ligger mitt framför eder, så skolen I, strax då I kommen ditin, finna en åsnefåle stå där bunden, som ännu ingen människa har suttit på; lösen den och fören den hit.
Mar 11:3  Och om någon frågar eder varför I gören detta, så skolen I svara: 'Herren behöver den, men han skall strax sända den tillbaka hit."
Mar 11:4  Då gingo de åstad och funno en åsnefåle stå där bunden utanför en port vid vägen, och de löste den.
Mar 11:5  Och några som stodo där bredvid sade till dem: "Vad gören I? Varför lösen I fålen?"
Mar 11:6  Men de svarade dem såsom Jesus hade bjudit. Då lät man dem vara.
Mar 11:7  Och de förde fålen till Jesus och lade sina mantlar på den, och han satte sig upp på den.
Mar 11:8  Och många bredde ut sina mantlar på vägen, andra åter skuro av kvistar och löv på fälten och strödde på vägen.
Mar 11:9  Och de som gingo före och de som följde efter ropade: "Hosianna! Välsignad vare han som kommer, i Herrens namn.
Mar 11:10  Välsignat vare vår fader Davids rike, som nu kommer. Hosianna i höjden!"
Mar 11:11  Så drog han in i Jerusalem och kom in i helgedomen; och när han hade sett sig omkring överallt och det redan var sent på dagen, gick han med de tolv ut till Betania.
Mar 11:12  När de dagen därefter voro på väg tillbaka från Betania, blev han hungrig.
Mar 11:13  Och då han på avstånd fick se ett fikonträd som hade löv, gick han dit för att se om han till äventyrs skulle finna något därpå; men när han kom fram till det, fann han intet annat än löv, det var icke då fikonens tid.
Mar 11:14  Då talade han och sade till trädet: "Aldrig någonsin mer äte någon frukt av dig." Och hans lärjungar hörde detta.
Mar 11:15  När de sedan kommo fram till Jerusalem, gick han in i helgedomen och begynte driva ut dem som sålde och köpte i helgedomen. Och han stötte omkull växlarnas bord och duvomånglarnas säten;
Mar 11:16  han tillstadde icke heller att man bar någonting genom helgedomen.
Mar 11:17  Och han undervisade dem och sade: "Det är ju skrivet: 'Mitt hus skall kallas ett bönehus för alla folk.' Men I haven gjort det till en rövarkula."
Mar 11:18  Då översteprästerna och de skriftlärde fingo höra härom, sökte de efter tillfälle att förgöra honom; ty de fruktade för honom, eftersom allt folket häpnade över hans undervisning.
Mar 11:19  När det blev afton, begåvo de sig ut ur staden.
Mar 11:20  Men då de nu på morgonen åter gingo där fram, fingo de se fikonträdet vara förtorkat ända från roten.
Mar 11:21  Då kom Petrus ihåg vad som hade skett och sade till honom: "Rabbi, se, fikonträdet som du förbannade är förtorkat."
Mar 11:22  Jesus svarade och sade till dem: "Haven tro på Gud.
Mar 11:23  Sannerligen säger jag eder: Om någon säger till detta berg: 'Häv dig upp, och kasta dig i havet' och därvid icke tvivlar i sitt hjärta, utan tror att det han säger skall ske, då skall det ske honom så.
Mar 11:24  Därför säger jag eder: Allt vad I bedjen om och begären, tron att det är eder givet; och det skall ske eder så.
Mar 11:25  Och när I stån och bedjen, så förlåten, om I haven något emot någon, för att också eder Fader, som är i himmelen, må förlåta eder edra försyndelser."
Mar 11:26  Men om I icke förlåten, så skall ej heller eder Fader, som är i himmelen, förlåta edra försyndelser.
Mar 11:27  Så kommo de åter till Jerusalem. Och medan han gick omkring i helgedomen, kommo översteprästerna och de skriftlärde och de äldste fram till honom;
Mar 11:28  och de sade till honom: "Med vad myndighet gör du detta? Och vem har givit dig myndighet att göra detta?"
Mar 11:29  Jesus svarade dem: "Jag vill ställa en fråga till eder; svaren mig på den, så skall ock jag säga eder med vad myndighet jag gör detta.
Mar 11:30  Johannes' döpelse, var den från himmelen eller från människor? Svaren mig härpå."
Mar 11:31  Då överlade de med varandra och sade: "Om vi svara: 'Från himmelen', så frågar han: 'Varför trodden I honom då icke?'
Mar 11:32  Eller skola vi svara: 'Från människor'?" - det vågade de icke av fruktan för folket, ty alla höllo före att Johannes verkligen var en profet.
Mar 11:33  De svarade alltså Jesus och sade: "Vi veta det icke." Då sade Jesus till dem: "Så säger icke heller jag eder med vad myndighet jag gör detta."
Mar 12:1  Och han begynte tala till dem i liknelser: "En man planterade en vingård och satte stängsel däromkring och högg ut ett presskar och byggde ett vakttorn; därefter lejde han ut den åt vingårdsmän och for utrikes.
Mar 12:2  När sedan rätta tiden var inne, sände han en tjänare till vingårdsmännen, för att denne av vingårdsmännen skulle uppbära någon del av vingårdens frukt.
Mar 12:3  Men de togo fatt på honom och misshandlade honom och läto honom gå tomhänt tillbaka.
Mar 12:4  Åter sände han till dem en annan tjänare. Honom slogo de i huvudet och skymfade.
Mar 12:5  Sedan sände han åstad ännu en annan, men denne dräpte de. Likaså gjorde de med många andra: somliga misshandlade de, och andra dräpte de.
Mar 12:6  Nu hade han ock en enda son, vilken han älskade. Honom sände han slutligen åstad till dem, ty han tänkte: 'De skola väl hava försyn för min son.'
Mar 12:7  Men vingårdsmännen sade till varandra: 'Denne är arvingen; kom, låt oss dräpa honom, så bliver arvet vårt.'
Mar 12:8  Och de togo fatt på honom och dräpte honom och kastade honom ut ur vingården. -
Mar 12:9  Vad skall nu vingårdens herre göra? Jo, han skall komma och förgöra vingårdsmännen och lämna vingården åt andra.
Mar 12:10  Haven I icke läst detta skriftens ord: 'Den sten som byggningsmännen förkastade, den har blivit en hörnsten;
Mar 12:11  av Herren har den blivit detta, och underbar är den i våra ögon'?"
Mar 12:12  De hade nu gärna velat gripa honom, men de fruktade för folket; ty de förstodo att det var om dem som han hade talat i denna liknelse. Så läto de honom vara och gingo sin väg.
Mar 12:13  Därefter sände de till honom några fariséer och herodianer, för att dessa skulle fånga honom genom något hans ord.
Mar 12:14  Dessa kommo nu och sade till honom: "Mästare, vi veta att du är sannfärdig och icke frågar efter någon, ty du ser icke till personen, utan lär om Guds väg vad sant är. Är det lovligt att giva kejsaren skatt, eller är det icke lovligt? Skola vi giva skatt, eller icke giva?"
Mar 12:15  Men han förstod deras skrymteri och sade till dem: "Varför söken I att snärja mig? Tagen hit en penning, så att jag får se den."
Mar 12:16  Då lämnade de fram en sådan. Därefter frågade han dem: "Vems bild och överskrift är detta?" De svarade honom: "Kejsarens."
Mar 12:17  Då sade Jesus till dem: "Så given kejsaren vad kejsaren tillhör, och Gud vad Gud tillhör." Och de förundrade sig högeligen över honom.
Mar 12:18  Sedan kommo till honom några av sadducéerna, vilka mena att det icke gives någon uppståndelse. Dessa frågade honom och sade:
Mar 12:19  "Mästare, Moses har givit oss den föreskriften, att om någon har en broder som dör, och som efterlämnar hustru, men icke lämnar barn efter sig, så skall han taga sin broders hustru till äkta och skaffa avkomma åt sin broder.
Mar 12:20  Nu voro här sju bröder. Den förste tog sig en hustru, men dog utan att lämna någon avkomma efter sig.
Mar 12:21  Då tog den andre i ordningen henne, men också han dog utan att lämna någon avkomma efter sig; sammalunda den tredje.
Mar 12:22  Så skedde med alla sju: ingen av dem lämnade någon avkomma efter sig. Sist av alla dog ock hustrun.
Mar 12:23  Vilken av dem skall nu vid uppståndelsen, när de uppstå, få henne till hustru? De hade ju alla sju tagit henne till hustru."
Mar 12:24  Jesus svarade dem: "Visar icke eder fråga att I faren vilse och varken förstån skrifterna, ej heller Guds kraft?
Mar 12:25  Efter uppståndelsen från de döda taga män sig icke hustrur, ej heller givas hustrur åt män, utan de äro då såsom änglarna i himmelen.
Mar 12:26  Men vad nu det angår, att de döda uppstå, haven I icke läst i Moses' bok, på det ställe där det talas om törnbusken, huru Gud sade till honom så: 'Jag är Abrahams Gud och Isaks Gud och Jakobs Gud'?
Mar 12:27  Han är en Gud icke för döda, utan för levande. I faren mycket vilse."
Mar 12:28  Då trädde en av de skriftlärde fram, en som hade hört deras ordskifte och förstått att han hade svarat dem väl. Denne frågade honom: "Vilket är det förnämsta av alla buden?"
Mar 12:29  Jesus svarade: "Det förnämsta är detta: 'Hör, Israel! Herren, vår Gud, Herren är en.
Mar 12:30  Och du skall älska Herren, din Gud, av allt ditt hjärta och av all din själ och av allt ditt förstånd och av all din kraft.'
Mar 12:31  Därnäst kommer detta: 'Du skall älska din nästa såsom dig själv.' Intet annat bud är större än dessa."
Mar 12:32  Då svarade den skriftlärde honom: "Mästare, du har i sanning rätt i vad du säger, att han är en, och att ingen annan är än han.
Mar 12:33  Och att älska honom av allt sitt hjärta och av allt sitt förstånd och av all sin kraft och att älska sin nästa såsom sig själv, det är 'förmer än alla brännoffer och slaktoffer'."
Mar 12:34  Då nu Jesus märkte att han hade svarat förståndigt, sade han till honom: "Du är icke långt ifrån Guds rike." Sedan dristade sig ingen att vidare ställa någon fråga på honom.
Mar 12:35  Medan Jesus undervisade i helgedomen, framställde han denna fråga: "Huru kunna de skriftlärde säga att Messias är Davids son?
Mar 12:36  David själv har ju sagt genom den helige Andes ingivelse: 'Herren sade till min herre: Sätt dig på min högra sida, till dess jag har lagt dina fiender dig till en fotapall.'
Mar 12:37  Så kallar nu David själv honom 'herre'; huru kan han då vara hans son?" Och folkskarorna hörde honom gärna.
Mar 12:38  Och han undervisade dem och sade till dem: "Tagen eder till vara för de skriftlärde, som gärna gå omkring i fotsida kläder och gärna vilja bliva hälsade på torgen
Mar 12:39  och gärna sitta främst i synagogorna och på de främsta platserna vid gästabuden -
Mar 12:40  detta under det att de utsuga änkors hus, medan de för syns skull hålla långa böner. De skola få en dess hårdare dom."
Mar 12:41  Och han satte sig mitt emot offerkistorna och såg huru folket lade ned penningar i offerkistorna. Och många rika lade dit mycket.
Mar 12:42  Men en fattig änka kom och lade ned två skärvar, det är ett öre.
Mar 12:43  Då kallade han sina lärjungar till sig och sade till dem: "Sannerligen säger jag eder: Denna fattiga änka lade dit mer än alla de andra som lade något i offerkistorna.
Mar 12:44  Ty dessa lade alla dit av sitt överflöd, men hon lade dit av sitt armod allt vad hon hade, så mycket som fanns i hennes ägo."
Mar 13:1  Då han nu gick ut ur helgedomen, sade en av hans lärjungar till honom: "Mästare, se hurudana stenar och hurudana byggnader!"
Mar 13:2  Jesus svarade honom: "Ja, du ser nu dessa stora byggnader; men här skall förvisso icke lämnas sten på sten; allt skall bliva nedbrutet."
Mar 13:3  När han sedan satt på Oljeberget, mitt emot helgedomen, frågade honom Petrus och Jakob och Johannes och Andreas, då de voro allena:
Mar 13:4  "Säg oss när detta skall ske, och vad som bliver tecknet till att tiden är inne, då allt detta skall gå i fullbordan."
Mar 13:5  Då begynte Jesus tala till dem och sade: "Sen till, att ingen förvillar eder.
Mar 13:6  Många skola komma under mitt namn och säga: 'Det är jag' och skola förvilla många.
Mar 13:7  Men när I fån höra krigslarm och rykten om krig, så förloren icke besinningen; sådant måste komma, men därmed är ännu icke änden inne.
Mar 13:8  Ja, folk skall resa sig upp mot folk och rike mot rike, och det skall bliva jordbävningar på den ena orten efter den andra, och hungersnöd skall uppstå; detta är begynnelsen till 'födslovåndorna'.
Mar 13:9  Men tagen I eder till vara. Man skall då draga eder inför domstolar, och I skolen bliva gisslade i synagogor och ställas fram inför landshövdingar och konungar, för min skull, till ett vittnesbörd för dem.
Mar 13:10  Men evangelium måste först bliva predikat för alla folk.
Mar 13:11  När man nu för eder åstad och drager eder inför rätta, så gören eder icke förut bekymmer om vad I skolen tala; utan vad som bliver eder givet i den stunden, det mån I tala. Ty det är icke I som skolen tala, utan den helige Ande.
Mar 13:12  Och den ene brodern skall då överlämna den andre till att dödas, ja ock fadern sitt barn; och barn skola sätta sig upp mot sina föräldrar och skola döda dem.
Mar 13:13  Och I skolen bliva hatade av alla, för mitt namns skull. Men den som är ståndaktig intill änden, han skall bliva frälst.
Mar 13:14  Men när I fån se 'förödelsens styggelse' stå där han icke borde stå - den som läser detta, han give akt därpå - då må de som äro i Judeen fly bort till bergen,
Mar 13:15  och den som är på taket må icke stiga ned och gå in för att hämta något ur sitt hus,
Mar 13:16  och den som är ute på marken må icke vända tillbaka för att hämta sin mantel.
Mar 13:17  Och ve de som äro havande, eller som giva di på den tiden!
Mar 13:18  Men bedjen att det icke må ske om vintern.
Mar 13:19  Ty den tiden skall bliva 'en tid av vedermöda, så svår att dess like icke har förekommit allt ifrån världens begynnelse, från den tid då Gud skapade världen, intill nu', ej heller någonsin skall förekomma.
Mar 13:20  Och om Herren icke förkortade den tiden, så skulle intet kött bliva frälst; men för de utvaldas skull, för de människors skull, som han har utvalt, har han förkortat den tiden.
Mar 13:21  Och om någon då säger till eder: 'Se här är Messias', eller: 'Se där är han', så tron det icke.
Mar 13:22  Ty människor som falskeligen säga sig vara Messias skola uppstå, så ock falska profeter, och de skola göra tecken och under, för att, om möjligt, förvilla de utvalda.
Mar 13:23  Men tagen I eder till vara. Jag har nu sagt eder allt förut.
Mar 13:24  Men på den tiden, efter den vedermödan, skall solen förmörkas och månen upphöra att giva sitt sken,
Mar 13:25  och stjärnorna skola falla ifrån himmelen, och makterna i himmelen skola bäva.
Mar 13:26  Och då skall man få se 'Människosonen komma i skyarna' med stor makt och härlighet.
Mar 13:27  Och han skall då sända ut sina änglar och församla sina utvalda från de fyra väderstrecken, från jordens ända till himmelens ända.
Mar 13:28  Ifrån fikonträdet mån I här hämta en liknelse. När dess kvistar begynna att få save och löven spricka ut, då veten I att sommaren är nära.
Mar 13:29  Likaså, när I sen detta ske, då kunnen I ock veta att han är nära och står för dörren.
Mar 13:30  Sannerligen säger jag eder: Detta släkte skall icke förgås, förrän allt detta sker.
Mar 13:31  Himmel och jord skola förgås, men mina ord skola icke förgås.
Mar 13:32  Men om den dagen och den stunden vet ingen något, icke änglarna i himmelen, icke ens Sonen - ingen utom Fadern.
Mar 13:33  Tagen eder till vara, vaken; ty I veten icke när tiden är inne.
Mar 13:34  Såsom när en man reser utrikes och lämnar sitt hus och giver sina tjänare makt och myndighet däröver, åt var och en hans särskilda syssla, och därvid ock bjuder portvaktaren att vaka.
Mar 13:35  likaså bjuder jag eder: Vaken; ty I veten icke när husets herre kommer, om han kommer på aftonen eller vid midnattstiden eller i hanegället eller på morgonen;
Mar 13:36  vaken, så att han icke finner eder sovande, när han oförtänkt kommer.
Mar 13:37  Men vad jag säger till eder, det säger jag till alla: Vaken!"
Mar 14:1  Två dagar därefter var det påsk och det osyrade brödets högtid. Och översteprästerna och de skriftlärde sökte efter tillfälle att gripa honom med list och döda honom.
Mar 14:2  De sade nämligen: "Icke under högtiden, för att ej oroligheter skola uppstå bland folket."
Mar 14:3  Men när han var i Betania, i Simon den spetälskes hus, och där låg till bords, kom en kvinna som hade med sig en alabasterflaska med smörjelse av dyrbar äkta nardus. Och hon bröt sönder flaskan och göt ut smörjelsen över hans huvud.
Mar 14:4  Några som voro där blevo då misslynta och sade till varandra: "Varför skulle denna smörjelse förspillas?
Mar 14:5  Man hade ju kunnat sälja den för mer än tre hundra silverpenningar och giva dessa åt de fattiga." Och de talade hårda ord till henne.
Mar 14:6  Men Jesus sade: "Låten henne vara. Varför oroen I henne? Det är en god gärning som hon har gjort mot mig.
Mar 14:7  De fattiga haven I ju alltid ibland eder, och närhelst I viljen kunnen I göra dem gott, men mig haven I icke alltid.
Mar 14:8  Vad hon kunde, det gjorde hon. Hon har i förväg smort min kropp såsom en tillredelse till min begravning.
Mar 14:9  Och sannerligen säger jag eder: Varhelst i hela världen evangelium bliver predikat, där skall ock det som hon nu har gjort bliva omtalat, henne till åminnelse."
Mar 14:10  Och Judas Iskariot, han som var en av de tolv, gick bort till översteprästerna och ville förråda honom åt dem.
Mar 14:11  När de hörde detta, blevo de glada och lovade att giva honom en summa penningar. Sedan sökte han efter tillfälle att förråda honom, då lägligt var.
Mar 14:12  På första dagen i det osyrade brödets högtid, när man slaktade påskalammet, sade hans lärjungar till honom: "Vart vill du att vi skola gå och reda till, så att du kan äta påskalammet?"
Mar 14:13  Då sände han åstad två av sina lärjungar och sade till dem: "Gån in i staden; där skolen I möta en man som bär en kruka vatten. Följen honom.
Mar 14:14  Och sägen till husbonden i det hus där han går in: 'Mästaren frågar: Var finnes härbärget där jag skall äta påskalammet med mina lärjungar?'
Mar 14:15  Då skall han visa eder en stor sal i övre våningen, tillredd och ordnad för måltid; reden till åt oss där."
Mar 14:16  Och lärjungarna begåvo sig i väg och kommo in i staden och funno det så som han hade sagt dem; och de redde till påskalammet.
Mar 14:17  När det sedan hade blivit afton, kom han dit med de tolv.
Mar 14:18  Och medan de lågo till bords och åto, sade Jesus: "Sannerligen säger jag eder: En av eder skall förråda mig, 'den som äter med mig'."
Mar 14:19  Då begynte de bedrövas och fråga honom, den ene efter den andre: "Icke är det väl jag?"
Mar 14:20  Och han sade till dem: "Det är en av de tolv, den som jämte mig doppar i fatet.
Mar 14:21  Ja, Människosonen skall gå bort, såsom det är skrivet om honom; men ve den människa genom vilken Människosonen bliver förrådd! Det hade varit bättre för den människan, om hon icke hade blivit född."
Mar 14:22  Medan de nu åto, tog han ett bröd och välsignade det och bröt det och gav åt dem och sade: "Tagen detta; detta är min lekamen."
Mar 14:23  Och han tog en kalk och tackade Gud ock gav åt dem; och de drucko alla därav.
Mar 14:24  Och han sade till dem: "Detta är mitt blod, förbundsblodet, som varder utgjutet för många.
Mar 14:25  Sannerligen säger jag eder: Jag skall icke mer dricka av det som kommer från vinträd, förrän på den dag då jag dricker det nytt i Guds rike."
Mar 14:26  När de sedan hade sjungit lovsången, gingo de ut till Oljeberget.
Mar 14:27  Då sade Jesus till dem: "I skolen alla komma på fall; ty det är skrivet: 'Jag skall slå herden, och fåren skola förskingras.'
Mar 14:28  Men efter min uppståndelse skall jag före eder gå till Galileen."
Mar 14:29  Då svarade Petrus honom: "Om än alla andra komma på fall, så skall dock jag det icke."
Mar 14:30  Jesus sade till honom: "Sannerligen säger jag dig: Redan i denna natt, förrän hanen har galit två gånger, skall du tre gånger förneka mig."
Mar 14:31  Då försäkrade han ännu ivrigare: "Om jag än måste dö med dig, så skall jag dock icke förneka dig." Sammalunda sade ock alla de andra. Och de kommo till ett ställe som kallades Getsemane.
Mar 14:32  Då sade han till sina lärjungar: "Bliven kvar här, medan jag beder."
Mar 14:33  Och han tog med sig Petrus och Jakob och Johannes; och han begynte bäva och ängslas.
Mar 14:34  Och han sade till dem: "Min själ är djupt bedrövad, ända till döds; stannen kvar här och vaken."
Mar 14:35  Därefter gick han litet längre bort och föll ned på jorden och bad, att om möjligt vore, den stunden skulle bliva honom besparad.
Mar 14:36  Och han sade: "Abba, Fader, allt är möjligt för dig. Tag denna kalk ifrån mig. Dock icke vad jag vill, utan vad du vill!"
Mar 14:37  Sedan kom han tillbaka och fann dem sovande. Då sade han till Petrus: "Simon, sover du? Förmådde du då icke vaka en kort stund?
Mar 14:38  Vaken, och bedjen att I icke mån komma i frestelse. Anden är villig, men köttet är svagt."
Mar 14:39  Och han gick åter bort och bad och sade samma ord.
Mar 14:40  När han sedan kom tillbaka, fann han dem åter sovande, ty deras ögon voro förtyngda. Och de visste icke vad de skulle svara honom.
Mar 14:41  För tredje gången kom han tillbaka och sade då till dem: "Ja, I soven ännu alltjämt och vilen eder! Det är nog. Stunden är kommen. Människosonen skall nu bliva överlämnad i syndarnas händer.
Mar 14:42  Stån upp, låt oss gå; se, den som förråder mig är nära."
Mar 14:43  Och i detsamma, medan han ännu talade, kom Judas, en av de tolv, och jämte honom en folkskara med svärd och stavar, utsänd från översteprästerna och de skriftlärde och de äldste.
Mar 14:44  Men förrädaren hade kommit överens med dem om ett tecken och sagt: "Den som jag kysser, den är det; honom skolen I gripa och föra bort under säker bevakning."
Mar 14:45  Och när han nu kom dit, trädde han strax fram till honom och sade: "Rabbi!" och kysste honom häftigt.
Mar 14:46  Då grepo de Jesus och togo honom fången.
Mar 14:47  Men en av dem som stodo där bredvid drog sitt svärd och högg till översteprästens tjänare och högg så av honom örat.
Mar 14:48  Och Jesus talade till dem och sade: "Såsom mot en rövare haven I gått ut med svärd och stavar för att fasttaga mig.
Mar 14:49  Var dag har jag varit ibland eder i helgedomen och undervisat, utan att I haven gripit mig. Men skrifterna skulle ju fullbordas."
Mar 14:50  Då övergåvo de honom alla och flydde.
Mar 14:51  Och bland dem som hade följt med honom var en ung man, höljd i ett linnekläde, som var kastat över blotta kroppen; honom grepo de.
Mar 14:52  Men han lämnade linneklädet kvar och flydde undan naken.
Mar 14:53  Så förde de nu Jesus bort till översteprästen, och där församlade sig alla översteprästerna och de äldste och de skriftlärde.
Mar 14:54  Och Petrus följde honom på avstånd ända in på översteprästens gård; där satt han sedan tillsammans med tjänarna och värmde sig vid elden.
Mar 14:55  Och översteprästerna och hela Stora rådet sökte efter något vittnesbörd mot Jesus, för att kunna döda honom; men de funno intet.
Mar 14:56  Ty väl vittnade många falskt mot honom, men vittnesbörden stämde icke överens.
Mar 14:57  Och några stodo upp och vittnade falskt mot honom och sade:
Mar 14:58  "Vi hava själva hört honom säga: 'Jag skall bryta ned detta tempel, som är gjort med händer, och skall sedan på tre dagar bygga upp ett annat, som icke är gjort med händer.'"
Mar 14:59  Men icke ens i det stycket stämde deras vittnesbörd överens.
Mar 14:60  Då stod översteprästen upp ibland dem och frågade Jesus och sade: "Svarar du intet? Huru är det med det som dessa vittna mot dig?"
Mar 14:61  Men han teg och svarade intet. Åter frågade översteprästen honom och sade till honom: "Är du Messias, den Högtlovades Son?"
Mar 14:62  Jesus svarade: "Jag är det. Och I skolen få se Människosonen sitta på Maktens högra sida och komma med himmelens skyar."
Mar 14:63  Då rev översteprästen sönder sina kläder och sade: "Vad behöva vi mer några vittnen?
Mar 14:64  I hörden hädelsen. Vad synes eder?" Då dömde de alla honom skyldig till döden.
Mar 14:65  Och några begynte spotta på honom; och sedan de hade höljt över hans ansikte, slogo de honom på kinderna med knytnävarna och sade till honom: "Profetera." Också rättstjänarna slogo honom på kinderna.
Mar 14:66  Medan nu Petrus befann sig därnere på gården, kom en av översteprästens tjänstekvinnor dit.
Mar 14:67  Och när hon fick se Petrus, där han satt och värmde sig, såg hon på honom och sade: "Också du var med nasaréen, denne Jesus."
Mar 14:68  Men han nekade och sade: "Jag varken vet eller förstår vad du menar." Sedan gick han ut på den yttre gården.
Mar 14:69  När tjänstekvinnan då fick se honom där, begynte hon åter säga till dem som stodo bredvid: "Denne är en av dem."
Mar 14:70  Då nekade han åter. Litet därefter sade återigen de som stodo där bredvid till Petrus: "Förvisso är du en av dem; du är ju också en galilé."
Mar 14:71  Då begynte han förbanna sig och svärja: "Jag känner icke den man som I talen om."
Mar 14:72  Och i detsamma gol hanen för andra gången. Då kom Petrus ihåg Jesu ord, huru han hade sagt till honom: "Förrän hanen har galit två gånger, skall du tre gånger förneka mig." Och han brast ut i gråt.
Mar 15:1  Sedan nu översteprästerna, tillsammans med de äldste och de skriftlärde, hela Stora rådet, på morgonen hade fattat sitt beslut, läto de strax binda Jesus och förde honom bort och överlämnade honom åt Pilatus.
Mar 15:2  Då frågade Pilatus honom: "Är du judarnas konung?" Han svarade honom och sade: "Du säger det själv."
Mar 15:3  Och översteprästerna framställde många anklagelser mot honom.
Mar 15:4  Pilatus frågade honom då åter och sade: "Svarar du intet? Du hör ju huru mycket det är som de anklaga dig för."
Mar 15:5  Men Jesus svarade intet mer, så att Pilatus förundrade sig.
Mar 15:6  Nu plägade han vid högtiden giva dem en fånge lös, den som de begärde.
Mar 15:7  Och där fanns då en man, han som kallades Barabbas, vilken satt fängslad jämte de andra som hade gjort upplopp och under upploppet begått dråp.
Mar 15:8  Folket kom ditupp och begynte begära att han skulle göra åt dem såsom han plägade göra.
Mar 15:9  Pilatus svarade dem och sade: "Viljen I att jag skall giva eder 'judarnas konung' lös?"
Mar 15:10  Han förstod nämligen att det var av avund som översteprästerna hade dragit Jesus inför rätta.
Mar 15:11  Men översteprästerna uppeggade folket till att begära att han hellre skulle giva dem Barabbas lös.
Mar 15:12  När alltså Pilatus åter tog till orda och frågade dem: "Vad skall jag då göra med den som I kallen 'judarnas konung'?",
Mar 15:13  så skriade de åter: "Korsfäst honom!"
Mar 15:14  Men Pilatus frågade dem: "Vad ont har han då gjort?" Då skriade de ännu ivrigare: "Korsfäst honom!"
Mar 15:15  Och eftersom Pilatus ville göra folket till viljes, gav han dem Barabbas lös; men Jesus lät han gissla och utlämnade honom sedan till att korsfästas.
Mar 15:16  Och krigsmännen förde honom in i palatset, eller pretoriet, och kallade tillhopa hela den romerska vakten.
Mar 15:17  Och de klädde på honom en purpurfärgad mantel och vredo samman en krona av törnen och satte den på honom.
Mar 15:18  Sedan begynte de hälsa honom: "Hell dig, judarnas konung!"
Mar 15:19  Och de slogo honom i huvudet med ett rör och spottade på honom; därvid böjde de knä och gåvo honom sin hyllning.
Mar 15:20  Och när de hade begabbat honom, klädde de av honom den purpurfärgade manteln och satte på honom hans egna kläder och förde honom ut för att korsfästa honom.
Mar 15:21  Och en man som kom utifrån marken gick där fram, Simon från Cyrene, Alexanders och Rufus' fader; honom tvingade de att gå med och bära hans kors.
Mar 15:22  Och de förde honom till Golgataplatsen (det betyder huvudskalleplatsen).
Mar 15:23  Och de räckte honom vin, blandat med myrra, men han tog icke emot det.
Mar 15:24  Och de korsfäste honom och delade sedan hans kläder mellan sig, genom att kasta lott om vad var och en skulle få.
Mar 15:25  Och det var vid tredje timmen som de korsfäste honom.
Mar 15:26  Och den överskrift som man hade satt upp över honom, för att angiva vad han var anklagad för, hade denna lydelse: "Judarnas konung."
Mar 15:27  Och de korsfäste med honom två rövare, den ene på hans högra sida och den andre på hans vänstra.
Mar 15:28  Och så fullbordades detta skriftens ord: "Han blev räknad bland ogärningsmän."
Mar 15:29  Och de som gingo där förbi bespottade honom och skakade huvudet och sade: "Tvi dig, du som 'bryter ned templet och bygger upp det igen inom tre dagar'!
Mar 15:30  Hjälp dig nu själv, och stig ned från korset."
Mar 15:31  Sammalunda talade ock översteprästerna, jämte de skriftlärde, begabbande ord med varandra och sade: "Andra har han hjälpt; sig själv kan han icke hjälpa.
Mar 15:32  Han som är Messias, Israels konung, han stige nu ned från korset, så att vi få se det och tro." Också de män som voro korsfästa med honom smädade honom.
Mar 15:33  Men vid sjätte timmen kom över hela landet ett mörker, som varade ända till nionde timmen.
Mar 15:34  Och vid nionde timmen ropade Jesus med hög röst: "Eloi, Eloi, lema sabaktani?"; det betyder: "Min Gud, min Gud, varför har du övergivit mig?"
Mar 15:35  Då några av dem som stodo där bredvid hörde detta, sade de: "Hör, han kallar på Elias."
Mar 15:36  Men en av dem skyndade fram och fyllde en svamp med ättikvin och satte den på ett rör och gav honom att dricka, i det han sade: "Låt oss se om Elias kommer och tager honom ned."
Mar 15:37  Men Jesus ropade med hög röst och gav upp andan.
Mar 15:38  Då rämnade förlåten i templet i två stycken, uppifrån och ända ned.
Mar 15:39  Men när hövitsmannen, som stod där mitt emot honom, såg att han på sådant sätt gav upp andan, sade han: "Förvisso var denne man Guds Son."
Mar 15:40  Också några kvinnor stodo där på avstånd och sågo vad som skedde. Bland dessa voro jämväl Maria från Magdala och den Maria som var Jakob den yngres och Joses' moder, så ock Salome
Mar 15:41  - vilka hade följt honom och tjänat honom, medan han var i Galileen - därtill många andra kvinnor, de som med honom hade vandrat upp till Jerusalem.
Mar 15:42  Det var nu tillredelsedag (det är dagen före sabbaten), och det hade blivit afton.
Mar 15:43  Josef från Arimatea, en ansedd rådsherre och en av dem som väntade på Guds rike, tog därför nu mod till sig och gick in till Pilatus och utbad sig att få Jesu kropp.
Mar 15:44  Då förundrade sig Pilatus över att Jesus redan skulle vara död, och han kallade till sig hövitsmannen och frågade honom om det var länge sedan han hade dött.
Mar 15:45  Och när han av hövitsmannen hade fått veta huru det var, skänkte han åt Josef hans döda kropp.
Mar 15:46  Denne köpte då en linneduk och tog honom ned och svepte honom i linneduken och lade honom i en grav som var uthuggen i en klippa; sedan vältrade han en sten för ingången till graven.
Mar 15:47  Men Maria från Magdala och den Maria som var Joses' moder sågo var han lades.
Mar 16:1  Och när sabbaten var förliden, köpte Maria från Magdala och den Maria som var Jakobs moder och Salome välluktande kryddor, för att sedan gå åstad och smörja honom.
Mar 16:2  Och bittida om morgonen på första veckodagen kommo de till graven, redan vid soluppgången.
Mar 16:3  Och de sade till varandra: "Vem skall åt oss vältra bort stenen från ingången till graven?"
Mar 16:4  Men när de sågo upp, fingo de se att stenen redan var bortvältrad. Den var nämligen mycket stor.
Mar 16:5  Och när de hade kommit in i graven, fingo de se en ung man sitta där på högra sidan, klädd i en vit fotsid klädnad; och de blevo förskräckta.
Mar 16:6  Men han sade till dem: "Varen icke förskräckta. I söken Jesus från Nasaret, den korsfäste. Han är uppstånden, han är icke här. Se där är platsen där de lade honom.
Mar 16:7  Men gån bort och sägen till hans lärjungar, och särskilt till Petrus: 'Han skall före eder gå till Galileen; där skolen I få se honom, såsom han bar sagt eder.'"
Mar 16:8  Då gingo de ut och flydde bort ifrån graven, ty bävan och bestörtning hade kommit över dem. Och i sin fruktan sade de intet till någon.
Mar 16:9  Men efter sin uppståndelse visade han sig på första veckodagens morgon först för Maria från Magdala, ur vilken han hade drivit ut sju onda andar.
Mar 16:10  Hon gick då och omtalade det för dem som hade följt med honom, och som nu sörjde och gräto.
Mar 16:11  Men när dessa hörde sägas att han levde och hade blivit sedd av henne, trodde de det icke.
Mar 16:12  Därefter uppenbarade han sig i en annan skepnad för två av dem, medan de voro stadda på vandring utåt landsbygden.
Mar 16:13  Också dessa gingo bort och omtalade det för de andra; men icke heller dem trodde man.
Mar 16:14  Sedan uppenbarade han sig också för de elva, när de lågo till bords; och han förebrådde dem då deras otro och deras hjärtans hårdhet, i det att de icke hade trott dem som hade sett honom vara uppstånden.
Mar 16:15  Och han sade till dem: "Gån ut i hela världen och prediken evangelium för allt skapat.
Mar 16:16  Den som tror och bliver döpt, han skall bliva frälst; men den som icke tror, han skall bliva fördömd.
Mar 16:17  Och dessa tecken skola åtfölja dem som tro: genom mitt namn skola de driva ut onda andar, de skola tala nya tungomål,
Mar 16:18  ormar skola de taga i händerna, och om de dricka något dödande gift, så skall det alls icke skada dem; på sjuka skola de lägga händerna, och de skola då bliva friska."
Mar 16:19  Därefter, sedan Herren Jesus hade talat med dem, blev han upptagen i himmelen och satte sig på Guds högra sida.
Mar 16:20  Men de gingo ut och predikade allestädes. Och Herren verkade med dem och stadfäste ordet genom de tecken som åtföljde det.


\end{document}