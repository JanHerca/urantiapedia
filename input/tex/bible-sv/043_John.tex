\begin{document}

\title{John}

Joh 1:1  I begynnelsen var Ordet, och Ordet var hos Gud, och Ordet var Gud.
Joh 1:2  Detta var i begynnelsen hos Gud.
Joh 1:3  Genom det har allt blivit till, och utan det har intet blivit till, som är till.
Joh 1:4  I det var liv, och livet var människornas ljus.
Joh 1:5  Och ljuset lyser i mörkret, och mörkret har icke fått makt därmed.
Joh 1:6  En man uppträdde, sänd av Gud; hans namn var Johannes.
Joh 1:7  Han kom såsom ett vittne, för att vittna om ljuset, på det att alla skulle komma till tro genom honom.
Joh 1:8  Icke var han ljuset, men han skulle vittna om ljuset.
Joh 1:9  Det sanna ljuset, det som lyser över alla människor, skulle nu komma i världen.
Joh 1:10  I världen var han, och genom honom hade världen blivit till, men världen ville icke veta av honom.
Joh 1:11  Han kom till sitt eget, och hans egna togo icke emot honom.
Joh 1:12  Men åt alla dem som togo emot honom gav han makt att bliva Guds barn, åt dem som tro på hans namn;
Joh 1:13  och de hava blivit födda, icke av blod, ej heller av köttslig vilja, ej heller av någon mans vilja, utan av Gud.
Joh 1:14  Och Ordet vart kött och tog sin boning ibland oss, och vi sågo hans härlighet, vi sågo likasom en enfödd Sons härlighet från sin Fader, och han var full av nåd och sanning.
Joh 1:15  Johannes vittnar om honom, han ropar och säger: "Det var om denne jag sade: 'Den som kommer efter mig, han är före mig; ty han var förr än jag.'"
Joh 1:16  Av hans fullhet hava vi ju alla fått, ja, nåd utöver nåd;
Joh 1:17  ty genom Moses blev lagen given, men nåden och sanningen hava kommit genom Jesus Kristus.
Joh 1:18  Ingen har någonsin sett Gud; den enfödde Sonen, som är i Faderns sköte, han har kungjort vad Gud är.
Joh 1:19  Och detta är vad Johannes vittnade, när judarna hade sänt till honom präster och leviter från Jerusalem för att fråga honom vem han var.
Joh 1:20  Han svarade öppet och förnekade icke; han sade öppet: "Jag är icke Messias."
Joh 1:21  Åter frågade de honom: "Vad är du då? Är du Elias?" Han svarade: "Det är jag icke." - "Är du Profeten?" Han svarade: "Nej."
Joh 1:22  Då sade de till honom: "Vem är du då? Säg oss det, så att vi kunna giva dem svar, som hava sänt oss. Vad säger du om dig själv?"
Joh 1:23  Han svarade: "Jag är rösten av en som ropar i öknen: 'Jämnen vägen för Herren', såsom profeten Esaias sade."
Joh 1:24  Och männen voro utsända ifrån fariséerna.
Joh 1:25  Och de frågade honom och sade till honom: "Varför döper du då, om du icke är Messias, ej heller Elias, ej heller Profeten?"
Joh 1:26  Johannes svarade dem och sade: "Jag döper i vatten; men mitt ibland eder står en som I icke kännen:
Joh 1:27  han som kommer efter mig (eller: vilken har varit före mig), vilkens skorem jag icke är värdig att upplösa."
Joh 1:28  Detta skedde i Betania, på andra sidan Jordan, där Johannes döpte.
Joh 1:29  Dagen därefter såg han Jesus nalkas; då sade han: "Se, Guds Lamm, som borttager världens synd!
Joh 1:30  Om denne var det som jag sade: 'Efter mig kommer en man som är före mig; ty han var förr än jag.'
Joh 1:31  Och jag kände honom icke; men för att han skall bliva uppenbar för Israel, därför är jag kommen och döper i vatten."
Joh 1:32  Och Johannes vittnade och sade: "Jag såg Anden såsom en duva sänka sig ned från himmelen; och han förblev över honom.
Joh 1:33  Och jag kände honom icke; men den som sände mig till att döpa i vatten, han sade till mig: 'Den över vilken du får se Anden sänka sig ned och förbliva, han är den som döper i helig ande.'
Joh 1:34  Och jag har sett det, och jag har vittnat att denne är Guds Son."
Joh 1:35  Dagen därefter stod Johannes åter där med två av sina lärjungar.
Joh 1:36  När då Jesus kom gående, såg Johannes på honom och sade: "Se, Guds Lamm!"
Joh 1:37  Och de två lärjungarna hörde hans ord och följde Jesus.
Joh 1:38  Då vände sig Jesus om, och när han såg att de följde honom, frågade han dem: "Vad viljen I?" De svarade honom: "Rabbi" (det betyder mästare) "var bor du?"
Joh 1:39  Han sade till dem: "Kommen och sen." Då gingo de med honom och sågo var han bodde; och de stannade den dagen hos honom. - Detta skedde vid den tionde timmen.
Joh 1:40  En av de två som hade hört var Johannes sade, och som hade följt Jesus, var Andreas, Simon Petrus' broder.
Joh 1:41  Denne träffade först sin broder Simon och sade till honom: "Vi hava funnit Messias" (det betyder detsamma som Kristus).
Joh 1:42  Och han förde honom till Jesus. Då såg Jesus på honom och sade: "Du är Simon, Johannes' son; du skall heta Cefas" (det betyder detsamma som Petrus).
Joh 1:43  Dagen därefter ville Jesus gå därifrån till Galileen, och han träffade då Filippus. Och Jesus sade till honom: "Följ mig."
Joh 1:44  Och Filippus var från Betsaida, Andreas' och Petrus' stad.
Joh 1:45  Filippus träffade Natanael och sade till honom: "Den som Moses har skrivit om i lagen och som profeterna hava skrivit om, honom hava vi funnit, Jesus, Josefs son, från Nasaret."
Joh 1:46  Natanael sade till honom: "Kan något gott komma från Nasaret?" Filippus svarade honom: "Kom och se."
Joh 1:47  När nu Jesus såg Natanael nalkas, sade han om honom: "Se, denne är en rätt israelit, i vilken icke finnes något svek."
Joh 1:48  Natanael frågade honom: "Huru kunna du känna mig?" Jesus svarade och sade till honom: "Förrän Filippus kallade dig, såg jag dig, där du var under fikonträdet."
Joh 1:49  Natanael svarade honom: "Rabbi, du är Guds Son, du är Israels konung."
Joh 1:50  Jesus svarade och sade till honom: "Eftersom jag sade dig att jag såg dig under fikonträdet, tror du? Större ting än vad detta är skall du få se."
Joh 1:51  Därefter sade han till honom: "Sannerligen, sannerligen säger jag eder: I skolen få se himmelen öppen och Guds änglar fara upp och fara ned över Människosonen."
Joh 2:1  På tredje dagen var ett bröllop i Kana i Galileen, och Jesu moder var där.
Joh 2:2  Också Jesus och hans lärjungar blevo bjudna till bröllopet.
Joh 2:3  Och vinet begynte taga slut. Då sade Jesu moder till honom: "De hava intet vin."
Joh 2:4  Jesus svarade henne: "Låt mig vara, moder; min stund är ännu icke kommen."
Joh 2:5  Hans moder sade då till tjänarna: "Vadhelst han säger till eder, det skolen I göra."
Joh 2:6  Nu stodo där sex stenkrukor, sådana som judarna hade för sina reningar; de rymde två eller tre bat-mått var.
Joh 2:7  Jesus sade till dem: "Fyllen krukorna med vatten." Och de fyllde dem ända till brädden.
Joh 2:8  Sedan sade han till dem: "Ösen nu upp och bären till övertjänaren." Och de gjorde så.
Joh 2:9  Och övertjänaren smakade på vattnet, som nu hade blivit vin; och han visste icke varifrån det hade kommit, vilket däremot tjänarna visste, de som hade öst upp vattnet. Då kallade övertjänaren på brudgummen.
Joh 2:10  och sade till honom: "Man brukar eljest alltid sätta fram det goda vinet, och sedan, när gästerna hava fått för mycket, det som är sämre. Du har gömt det goda vinet ända tills nu."
Joh 2:11  Detta var det första tecknet som Jesus gjorde. Han gjorde det i Kana i Galileen och uppenbarade så sin härlighet; och hans lärjungar trodde på honom.
Joh 2:12  Därefter begav han sig ned till Kapernaum med sin moder och sina bröder och sina lärjungar; och där stannade de några få dagar.
Joh 2:13  Judarnas påsk var nu nära, och Jesus begav sig då upp till Jerusalem.
Joh 2:14  Och när han fick i helgedomen se huru där sutto män som sålde fäkreatur och får och duvor, och huru växlare sutto där.
Joh 2:15  Då gjorde han sig ett gissel av tåg och drev dem alla ut ur helgedomen, med får och fäkreatur, och slog ut växlarnas penningar och stötte omkull deras bord.
Joh 2:16  Och till duvomånglarna sade han: "Tagen bort detta härifrån; gören icke min Faders hus till ett marknadshus."
Joh 2:17  Hans lärjungar kommo då ihåg att det var skrivet: "Nitälskan för ditt hus skall förtära mig."
Joh 2:18  Då togo judarna till orda och sade till honom: "Vad för tecken låter du oss se, eftersom du gör på detta sätt?"
Joh 2:19  Jesus svarade och sade till dem: "Bryten ned detta tempel, så skall jag inom tre dagar låta det uppstå igen."
Joh 2:20  Då sade judarna: "I fyrtiosex år har man byggt på detta tempel, och du skulle låta det uppstå igen inom tre dagar?"
Joh 2:21  Men det var om sin kropps tempel han talade.
Joh 2:22  Sedan, när han hade uppstått från de döda, kommo hans lärjungar ihåg att han hade sagt detta; och de trodde då skriften och det ord som Jesus hade sagt.
Joh 2:23  Medan han nu var i Jerusalem, under påsken, vid högtiden, kommo många till tro på hans namn, när de sågo de tecken som han gjorde.
Joh 2:24  Men själv betrodde sig Jesus icke åt dem, eftersom han kände alla
Joh 2:25  och icke behövde någon annans vittnesbörd om människorna; ty av sig själv visste han vad i människan var.
Joh 3:1  Men bland fariséerna var en man som hette Nikodemus, en av judarnas rådsherrar.
Joh 3:2  Denne kom till Jesus om natten och sade till honom: "Rabbi, vi veta att det är från Gud du har kommit såsom lärare; ty ingen kan göra sådana tecken som du gör, om icke Gud är med honom."
Joh 3:3  Jesus svarade och sade till honom: "Sannerligen, sannerligen säger jag dig: Om en människa icke bliver född på nytt, så kan hon icke få se Guds rike."
Joh 3:4  Nikodemus sade till honom: "Huru kan en människa födas, när hon är gammal? Icke kan hon väl åter gå in i sin moders liv och födas?"
Joh 3:5  Jesus svarade: "Sannerligen, sannerligen säger jag dig: Om en människa icke bliver född av vatten och ande, så kan hon icke komma in i Guds rike.
Joh 3:6  Det som är fött av kött, det är kött; och det som är fött av Anden, det är ande.
Joh 3:7  Förundra dig icke över att jag sade dig att I måsten födas på nytt.
Joh 3:8  Vinden blåser vart den vill, och du hör dess sus, men du vet icke varifrån den kommer, eller vart den far; så är det med var och en som är född av Anden."
Joh 3:9  Nikodemus svarade och sade till honom: "Huru kan detta ske?"
Joh 3:10  Jesus svarade och sade till honom: "Är du Israels lärare och förstår icke detta?
Joh 3:11  Sannerligen, sannerligen säger jag dig: Vad vi veta, det tala vi, och vad vi hava sett, det vittna vi om, men vårt vittnesbörd tagen I icke emot.
Joh 3:12  Tron i icke, när jag talar till eder om jordiska ting, huru skolen I då kunna tro, när jag talar till eder om himmelska ting?
Joh 3:13  Och likväl har ingen stigit upp till himmelen, utom den som steg ned från himmelen, Människosonen, som var i himmelen.
Joh 3:14  Och såsom Moses upphöjde ormen i öknen, så måste Människosonen bliva upphöjd,
Joh 3:15  så att var och en som tror skall i honom hava evigt liv.
Joh 3:16  Ty så älskade Gud världen, att han utgav sin enfödde Son, på det att var och en som tror på honom skall icke förgås, utan hava evigt liv.
Joh 3:17  Ty icke sände Gud sin Son i världen för att döma världen, utan för att världen skulle bliva frälst genom honom.
Joh 3:18  Den som tror på honom, han bliver icke dömd, men den som icke tror, han är redan dömd, eftersom han icke tror på Guds enfödde Sons namn.
Joh 3:19  Och detta är domen, att när ljuset hade kommit i världen, människorna dock älskade mörkret mer än ljuset, eftersom deras gärningar voro onda,
Joh 3:20  Ty var och en som gör vad ont är, han hatar ljuset och kommer icke till ljuset, på det att hans gärningar icke skola bliva blottade.
Joh 3:21  Men den som gör sanningen, han kommer till ljuset, för att det skall bliva uppenbart att hans gärningar äro gjorda i Gud."
Joh 3:22  Därefter begav sig Jesus med sina lärjungar till den judiska landsbygden, och där vistades han med dem och döpte.
Joh 3:23  Men också Johannes döpte, i Enon, nära Salim, ty där fanns mycket vatten; och folket kom dit och lät döpa sig.
Joh 3:24  Johannes hade nämligen ännu icke blivit kastad i fängelse.
Joh 3:25  Då uppstod mellan Johannes' lärjungar och en jude en tvist om reningen.
Joh 3:26  Och de kommo till Johannes och sade till honom: "Rabbi, se, den som var hos dig på andra sidan Jordan, den som du har vittnat om, han döper, och alla komma till honom."
Joh 3:27  Johannes svarade och sade: "En människa kan intet taga, om det icke bliver henne givet från himmelen."
Joh 3:28  I kunnen själva giva mig det vittnesbördet att jag sade: 'Icke är jag Messias; jag är allenast sänd framför honom.'
Joh 3:29  Brudgum är den som har bruden; men brudgummens vän, som står där och hör honom, han gläder sig storligen åt brudgummens röst. Den glädjen är mig nu given i fullt mått.
Joh 3:30  Det är såsom sig bör att han växer till, och att jag förminskas. -
Joh 3:31  Den som kommer ovanifrån, han är över alla; den som är från jorden, han är av jorden, och av jorden talar han. Ja, den som kommer från himmelen, han är över alla,
Joh 3:32  och vad han har sett och hört, det vittnar han om; och likväl tager ingen emot hans vittnesbörd.
Joh 3:33  Men om någon tager emot hans vittnesbörd, så bekräftar han därmed att Gud är sannfärdig.
Joh 3:34  Ty den som Gud har sänt, han talar Guds ord; Gud giver nämligen icke Anden efter mått.
Joh 3:35  Fadern älskar Sonen, och allt har han givit i hans hand.
Joh 3:36  Den som tror på Sonen, han har evigt liv; men den som icke hörsammar Sonen, han skall icke få se livet, utan Guds vrede förbliver över honom."
Joh 4:1  Men Herren fick nu veta att fariséerna hade hört hurusom Jesus vann flera lärjungar och döpte flera än Johannes;
Joh 4:2  dock var det icke Jesus själv som döpte, utan hans lärjungar.
Joh 4:3  Då lämnade han Judeen och begav sig åter till Galileen.
Joh 4:4  Därvid måste han taga vägen genom Samarien.
Joh 4:5  Så kom han till en stad i Samarien som hette Sykar, nära det jordstycke som Jakob gav åt sin son Josef.
Joh 4:6  Och där var Jakobs brunn. Eftersom nu Jesus var trött av vandringen, satte han sig strax ned vid brunnen. Det var vid den sjätte timmen.
Joh 4:7  Då kom en samaritisk kvinna för att hämta vatten. Jesus sade till henne: "Giv mig att dricka."
Joh 4:8  Hans lärjungar hade nämligen gått in i staden för att köpa mat.
Joh 4:9  Då sade den samaritiska kvinnan till honom: "Huru kan du, som är jude, bedja mig, som är en samaritisk kvinna, om något att dricka?" Judarna hava nämligen ingen umgängelse med samariterna.
Joh 4:10  Jesus svarade och sade till henne: "Förstode du Guds gåva, och vem den är som säger till dig: 'Giv mig att dricka', så skulle i stället du hava bett honom, och han skulle då hava givit dig levande vatten."
Joh 4:11  Kvinnan sade till honom: "Herre, du har ju intet att hämta upp vatten med, och brunnen är djup. Varifrån får du då det friska vattnet?"
Joh 4:12  Icke är du väl förmer än vår fader Jakob, som gav oss brunnen och själv med sina barn och sin boskap drack ur den?"
Joh 4:13  Jesus svarade och sade till henne: "Var och en som dricker av detta vatten, han bliver törstig igen;
Joh 4:14  men den som dricker av det vatten som jag giver honom, han skall aldrig någonsin törsta, utan det vatten jag giver honom skall bliva i honom en källa vars vatten springer upp med evigt liv."
Joh 4:15  Kvinnan sade till honom: "Herre, giv mig det vattnet, så att jag icke mer behöver törsta och komma hit för att hämta vatten."
Joh 4:16  Han sade till henne: "Gå och hämta din man, och kom sedan tillbaka."
Joh 4:17  Kvinnan svarade och sade: "Jag har ingen man." Jesus sade till henne: "Du har rätt i vad du säger, att du icke har någon man."
Joh 4:18  Ty fem män har du haft, och den du nu har är icke din man; däri sade du sant.
Joh 4:19  Då sade kvinnan till honom: "Herre, jag ser att du är en profet.
Joh 4:20  Våra fäder hava tillbett på detta berg, men I sägen att i Jerusalem den plats finnes, där man bör tillbedja."
Joh 4:21  Jesus sade till henne: "Tro mig, kvinna: den tid kommer, då det varken är på detta berg eller i Jerusalem som I skolen tillbedja Fadern.
Joh 4:22  I tillbedjen vad I icke kännen, vi tillbedja vad vi känna - ty frälsningen kommer från judarna -
Joh 4:23  men den tid skall komma, ja, den är redan inne, då sanna tillbedjare skola tillbedja Fadern i ande och sanning; ty sådana tillbedjare vill Fadern hava.
Joh 4:24  Gud är ande, och de som tillbedja måste tillbedja i ande och sanning."
Joh 4:25  Kvinnan sade till honom: "Jag vet att Messias skall komma, han som ock kallas Kristus; när han kommer, skall han förkunna oss allt."
Joh 4:26  Jesus svarade henne: "Jag, som talar med dig, är den du nu nämnde."
Joh 4:27  I detsamma kommo hans lärjungar; och de förundrade sig över att han talade med en kvinna. Dock frågade ingen vad han ville henne, eller varför han talade med henne.
Joh 4:28  Men kvinnan lät sin kruka stå och gick in i staden och sade till folket:
Joh 4:29  "Kommen och sen en man som har sagt mig allt vad jag har gjort. Månne icke han är Messias?"
Joh 4:30  Då gingo de ut ur staden och kommo till honom.
Joh 4:31  Under tiden bådo lärjungarna honom och sade: "Rabbi, tag och ät."
Joh 4:32  Men han svarade dem: "Jag har mat att äta som I icke veten om."
Joh 4:33  Då sade lärjungarna till varandra: "Kan väl någon hava burit mat till honom?"
Joh 4:34  Jesus sade till dem: "Min mat är att göra dens vilja, som har sänt mig, och att fullborda hans verk."
Joh 4:35  I sägen ju att det ännu är fyra månader innan skördetiden kommer. Men se, jag säger eder: Lyften upp edra ögon, och sen på fälten, huru de hava vitnat till skörd.
Joh 4:36  Redan nu får den som skördar uppbära sin lön och samla in frukt till evigt liv; så kunna den som sår och den som skördar tillsammans glädja sig.
Joh 4:37  Ty här sannas det ordet, att en är den som sår och en annan den som skördar.
Joh 4:38  Jag har sänt eder att skörda, där I icke haven arbetat. Andra hava arbetat, och I haven fått gå in i deras arbete."
Joh 4:39  Och många samariter från den staden kommo till tro på honom för kvinnans ords skull, då hon vittnade att han hade sagt henne allt vad hon hade gjort.
Joh 4:40  När sedan samariterna kommo till honom, både de honom att stanna kvar hos dem. Så stannade han där i två dagar.
Joh 4:41  Och långt flera kommo då till tro för hans egna ords skull.
Joh 4:42  Och de sade till kvinnan: "Nu är det icke mer för dina ords skull som vi tro, ty vi hava nu själva hört honom, och vi veta nu att han i sanning är världens Frälsare."
Joh 4:43  Men efter de två dagarna gick han därifrån till Galileen.
Joh 4:44  Ty Jesus vittnade själv att en profet icke är aktad i sitt eget fädernesland.
Joh 4:45  När han nu kom till Galileen, togo galiléerna vänligt emot honom, eftersom de hade sett allt vad han hade gjort i Jerusalem vid högtiden. Också de hade nämligen varit där vid högtiden.
Joh 4:46  Så kom han åter till Kana i Galileen, där han hade gjort vattnet till vin. I Kapernaum fanns då en man i konungens tjänst, vilkens son låg sjuk.
Joh 4:47  När han nu hörde att Jesus hade kommit från Judeen till Galileen, begav han sig åstad till honom och bad att han skulle komma ned och bota hans son; ty denne låg för döden.
Joh 4:48  Då sade Jesus till honom: "Om I icke sen tecken och under, så tron I icke."
Joh 4:49  Mannen sade till honom: "Herre, kom ned, förrän mitt barn dör."
Joh 4:50  Jesus svarade honom: "Gå, din son får leva." Då trodde mannen det ord som Jesus sade till honom, och gick.
Joh 4:51  Och medan han ännu var på vägen hem, mötte honom hans tjänare och sade: "Din son kommer att leva."
Joh 4:52  Då frågade han dem vid vilken timme det hade blivit bättre med honom. De svarade honom: "I går vid den sjunde timmen lämnade febern honom."
Joh 4:53  Då märkte fadern att det hade skett just den timme då Jesus sade till honom: "Din son får leva." Och han kom till tro, så ock hela hans hus.
Joh 4:54  Detta var nu åter ett tecken, det andra i ordningen som Jesus gjorde, sedan han hade kommit från Judeen till Galileen.
Joh 5:1  Därefter inföll en av judarnas högtider, och Jesus for upp till Jerusalem.
Joh 5:2  Vid Fårporten i Jerusalem ligger en damm, på hebreiska kallad Betesda, och invid den finnas fem pelargångar.
Joh 5:3  I dessa lågo många sjuka, blinda, halta, förtvinade. som väntade på att vattnet skulle uppröras.
Joh 5:4  Ty en ängel steg tidtals ned i dammen och upprörde vattnet. Den som nu först steg ned i vattnet, sedan det hade blivit upprört, han blev frisk, med vilken jukdom han än var behäftad.
Joh 5:5  Där fanns nu en man som hade varit sjuk i trettioåtta år.
Joh 5:6  Då Jesus fick se denne, där han låg, och fick veta att han redan lång tid hade varit sjuk, sade han till honom: "Vill du bliva frisk?"
Joh 5:7  Den sjuke svarade honom: "Herre, jag har ingen som hjälper mig ned i dammen, när vattnet har kommit i rörelse; och så stiger en annan ditned före mig, medan jag ännu är på väg."
Joh 5:8  Jesus sade till honom: "Stå upp, tag din säng och gå."
Joh 5:9  Och strax blev mannen frisk och tog sin säng och gick. Men det var sabbat den dagen.
Joh 5:10  Därför sade judarna till mannen som hade blivit botad: "Det är sabbat; det är icke lovligt för dig att bära sängen."
Joh 5:11  Men han svarade dem: "Den som gjorde mig frisk, han sade till mig: 'Tag din säng och gå.'"
Joh 5:12  Då frågade de honom: "Vem var den mannen som sade till dig att du skulle taga sin säng och gå?"
Joh 5:13  Men mannen som hade blivit botad visste icke vem det var; ty Jesus hade dragit sig undan, eftersom mycket folk var där på platsen. -
Joh 5:14  Sedan träffade Jesus honom i helgedomen och sade till honom: "Se, du har blivit frisk; synda icke härefter, på det att icke något värre må vederfaras dig."
Joh 5:15  Mannen gick då bort och omtalade för judarna, att det var Jesus som hade gjort honom frisk.
Joh 5:16  Därför förföljde nu judarna Jesus (och sökte att döda honom.), eftersom han gjorde sådant på sabbaten.
Joh 5:17  Men han svarade dem: "Min Fader verkar ännu alltjämt; så verkar ock jag."
Joh 5:18  Och därför stodo judarna ännu mer efter att döda honom, eftersom han icke allenast ville göra sabbaten om intet, utan ock kallade Gud sin Fader och gjorde sig själv lik Gud.
Joh 5:19  Då talade Jesus åter och sade till dem: "Sannerligen, sannerligen säger jag eder: Sonen kan icke göra något av sig själv, utan han gör allenast vad han ser Fadern göra; ty vad han gör, det gör likaledes ock Sonen.
Joh 5:20  Ty Fadern älskar Sonen och låter honom se allt vad han själv gör; och större gärningar, än dessa äro, skall han låta honom se, så att I skolen förundra eder.
Joh 5:21  Ty såsom Fadern uppväcker döda och gör dem levande, så gör ock Sonen levande vilka han vill.
Joh 5:22  Icke heller dömer Fadern någon, utan all dom har han överlåtit åt Sonen,
Joh 5:23  för att alla skola ära Sonen såsom de ära Faderns. Den som icke ärar Sonen, han ärar icke heller Fadern, som har sänt honom.
Joh 5:24  Sannerligen, sannerligen säger jag eder: Den som hör mina ord och tror honom som har sänt mig, han har evigt liv och kommer icke under någon dom, utan har övergått från döden till livet.
Joh 5:25  Sannerligen säger jag eder: Den stund kommer, jag, den är redan inne, så de döda skola höra Guds Sons röst, och de som höra den skola bliva levande.
Joh 5:26  Ty såsom Fadern har liv i sig själv, så har han ock givit åt Sonen att hava liv i sig själv.
Joh 5:27  Och han har givit honom makt att hålla dom, eftersom han är Människoson.
Joh 5:28  Förundren eder icke över detta. Ty den stund kommer, då alla som äro i gravarna skola höra hans röst
Joh 5:29  och gå ut ur dem: de som hava gjort vad gott är skola uppstå till liv, och de som hava gjort vad ont är skola uppstå till dom.
Joh 5:30  Jag kan icke göra något av mig själv. Såsom jag hör, så dömer jag; och min dom är rättvis, ty jag söker icke min vilja, utan dens vilja, som har sänt mig.
Joh 5:31  Om jag själv vittnar om mig, så gäller icke mitt vittnesbörd.
Joh 5:32  Men det är en annan som vittnar om mig, och jag vet att hans vittnesbörd om mig är sant.
Joh 5:33  I haven sänt bud till Johannes, och han har vittnat för sanningen,
Joh 5:34  Dock, det är icke av någon människa som jag tager emot vittnesbörd om mig; men jag säger detta, för att I skolen bliva frälsta.
Joh 5:35  Han var den brinnande, skinande lampan, och för en liten stund villen I fröjdas i dess ljus.
Joh 5:36  Men jag har ett vittnesbörd om mig, som är förmer än Johannes' vittnesbörd: de gärningar som Fadern har givit mig att fullborda, just de gärningar som jag gör, de vittna om mig, att Fadern har sänt mig.
Joh 5:37  Ja, Fadern, som har sänt mig, han har själv vittnat om mig. Hans röst haven I aldrig någonsin hört, ej heller haven I sett hans gestalt,
Joh 5:38  och hans ord haven I icke låtit förbliva i eder. Ty den han har sänt, honom tron I icke.
Joh 5:39  I rannsaken skrifterna, därför att I menen eder i dem hava evigt liv; och det är dessa som vittna om mig.
Joh 5:40  Men I viljen icke komma till mig för att få liv.
Joh 5:41  Jag tager icke emot pris av människor;
Joh 5:42  men jag känner eder och vet att I icke haven Guds kärlek i eder.
Joh 5:43  Jag har kommit i min Faders namn, och I tagen icke emot mig; kommer en annan i sitt eget namn, honom skolen I nog mottaga.
Joh 5:44  Huru skullen I kunna tro, I som tagen emot pris av varandra och icke söken det pris som kommer från honom som allena är Gud?
Joh 5:45  Menen icke att det är jag som skall anklaga eder hos Fadern. Den som anklagar eder är Moses, han till vilken I sätten edert hopp.
Joh 5:46  Trodden I Moses, så skullen I ju tro mig, ty om mig har han skrivit.
Joh 5:47  Men tron I icke hans skrifter, huru skolen I då kunna tro mina ord?"
Joh 6:1  Därefter for Jesus över Galileiska sjön, "Tiberias' sjö".
Joh 6:2  Och mycket folk följde efter honom, därför att de sågo de tecken som han gjorde med de sjuka.
Joh 6:3  Men Jesus gick upp på berget och satte sig där med sina lärjungar.
Joh 6:4  Och påsken, judarnas högtid, var nära.
Joh 6:5  Då nu Jesus lyfte upp sina ögon och såg att mycket folk kom till honom, sade han till Filippus: "Varifrån skola vi köpa bröd, så att dessa få äta?"
Joh 6:6  Men detta sade han för att sätta honom på prov, ty själv visste han vad han skulle göra.
Joh 6:7  Filippus svarade honom: "Bröd för två hundra silverpenningar vore icke nog för att var och en skulle få ett litet stycke."
Joh 6:8  Då sade till honom en annan av hans lärjungar, Andreas, Simon Petrus' broder:
Joh 6:9  "Här är en gosse som har fem kornbröd och två fiskar; men vad förslår det för så många?"
Joh 6:10  Jesus sade: "Låten folket lägga sig här." Och på det stället var mycket gräs. Då lägrade sig männen där, och deras antal var vid pass fem tusen.
Joh 6:11  Därefter tog Jesus bröden och tackade Gud och delade ut åt dem som hade lagt sig ned där, likaledes ock av fiskarna, så mycket de ville hava.
Joh 6:12  Och när de voro mätta, sade han till sina lärjungar: "Samlen tillhopa de överblivna styckena, så att intet förfares."
Joh 6:13  Då samlade de dem tillhopa och fyllde tolv korgar med stycken, som av de fem kornbröden hade blivit över efter dem som hade ätit.
Joh 6:14  Då nu människorna hade det tecken som han hade gjort, sade de: "Denne är förvisso Profeten som skulle komma i världen."
Joh 6:15  När då Jesus märkte att de tänkte komma och med våld föra honom med sig och göra honom till konung, drog han sig åter undan till berget, helt allena.
Joh 6:16  Men när det blev afton, gingo hans lärjungar ned till sjön
Joh 6:17  och stego i en båt för att fara över sjön till Kapernaum. Det hade då redan blivit mörkt, och Jesus hade ännu icke kommit till dem;
Joh 6:18  och sjön gick hög, ty det blåste hårt.
Joh 6:19  När de så hade rott vid pass tjugufem eller trettio stadier, fingo de se Jesus komma gående på sjön och nalkas båten. Då blevo de förskräckta.
Joh 6:20  Men han sade till dem: "Det är jag; varen icke förskräckta."
Joh 6:21  De ville då taga honom upp i båten; och strax var båten framme vid landet dit de foro.
Joh 6:22  Dagen därefter hände sig detta. Folket som stod kvar på andra sidan sjön hade lagt märke till att där icke fanns mer än en enda båt, och att Jesus icke hade stigit i den båten med sina lärjungar, utan att lärjungarna hade farit bort allena.
Joh 6:23  Andra båtar hade likväl kommit från Tiberias och lagt till nära det ställe där folket bespisades efter det att Herren hade uttalat tacksägelsen.
Joh 6:24  När alltså folket nu såg att Jesus icke var där, ej heller hans lärjungar, stego de själva i båtarna och foro till Kapernaum för att söka efter Jesus.
Joh 6:25  Och då de funno honom där på andra sidan sjön, frågade de honom: "Rabbi, när kom du hit?"
Joh 6:26  Jesus svarade dem och sade: "Sannerligen, sannerligen säger jag eder: I söken mig icke därför att I haven sett tecken, utan därför att I fingen äta av bröden och bleven mätta.
Joh 6:27  Verken icke för att få den mat som förgås, utan för att få den mat som förbliver och har med sig evigt liv, den som Människosonen skall giva eder; ty honom har Fadern, Gud själv, låtit undfå sitt insegel."
Joh 6:28  Då sade de till honom: "Vad skola vi göra för att utföra Guds gärningar?"
Joh 6:29  Jesus svarade och sade till dem: "Detta är Guds gärning, att I tron på den han har sänt."
Joh 6:30  De sade till honom: "Vad för tecken gör du då? Låt oss se något tecken, så att vi kunna tro dig. Vilken gärning utför du?
Joh 6:31  Våra fäder fingo äta manna i öknen, såsom det är skrivet: 'Han gav dem bröd från himmelen att äta.'"
Joh 6:32  Då svarade Jesus dem: "Sannerligen, sannerligen säger jag eder: Det är icke Moses som har givit eder brödet från himmelen, men det är min Fader som giver eder det rätta brödet från himmelen.
Joh 6:33  Ty Guds bröd är det bröd som kommer ned från himmelen och giver världen liv."
Joh 6:34  Då sade de till honom: "Herre, giv oss alltid det brödet."
Joh 6:35  Jesus svarade: "Jag är livets bröd. Den som kommer till mig, han skall aldrig hungra, och den som tror på mig, han skall aldrig törsta.
Joh 6:36  Men det är såsom jag har sagt eder: fastän I haven sett mig, tron I dock icke.
Joh 6:37  Allt vad min Fader giver mig, det kommer till mig; och den som kommer till mig, honom skall jag sannerligen icke kasta ut.
Joh 6:38  Ty jag har kommit ned från himmelen, icke för att göra min vilja, utan för att göra dens vilja, som har sänt mig.
Joh 6:39  Och detta är dens vilja, som har sänt mig, att jag icke skall låta någon enda gå förlorad av dem som han har givit mig, utan att jag skall låta dem uppstå på den yttersta dagen.
Joh 6:40  Ja, detta är min Faders vilja, att var och en som ser Sonen och tror på honom, han skall hava evigt liv, och att jag skall låta honom uppstå på den yttersta dagen."
Joh 6:41  Då knorrade judarna över honom, därför att han hade sagt: "Jag är det bröd som har kommit ned från himmelen."
Joh 6:42  Och de sade: "Är denne icke Jesus, Josefs son, vilkens fader och moder vi känna? Huru kan han då säga: 'Jag har kommit ned från himmelen'?"
Joh 6:43  Jesus svarade och sade till dem: "Knorren icke eder emellan.
Joh 6:44  Ingen kan komma till mig, om icke Fadern, som har sänt mig, drager honom; och jag skall låta honom uppstå på den yttersta dagen.
Joh 6:45  Det är skrivet hos profeterna: 'De skola alla hava fått lärdom av Gud.' Var och en som har lyssnat till Fadern och lärt av honom, han kommer till mig.
Joh 6:46  Icke som om någon skulle hava sett Fadern, utom den som är från Gud; han har sett Fadern.
Joh 6:47  Sannerligen, sannerligen säger jag eder: Den som tror, han har evigt liv.
Joh 6:48  Jag är livets bröd.
Joh 6:49  Edra fäder åto manna i öknen, och de dogo.
Joh 6:50  Men med det bröd som kommer ned från himmelen är det så, att om någon äter därav, så skall han icke dö.
Joh 6:51  Jag är det levande brödet som har kommit ned från himmelen. Om någon äter av det brödet, så skall han leva till evig tid. Och det bröd som jag skall giva är mitt kött; och jag giver det, för att världen skall leva."
Joh 6:52  Då tvistade judarna med varandra och sade: "Huru skulle denne kunna giva oss sitt kött att äta?"
Joh 6:53  Jesus sade då till dem: "Sannerligen, sannerligen säger jag eder: Om I icke äten Människosonens kött och dricken hans blod, så haven I icke liv i eder.
Joh 6:54  Den som äter mitt kött och dricker mitt blod, han har evigt liv, och jag skall låta honom uppstå på den yttersta dagen.
Joh 6:55  Ty mitt kött är sannskyldig mat, och mitt blod är sannskyldig dryck.
Joh 6:56  Den som äter mitt kött och dricker mitt blod, han förbliver i mig, och jag förbliver i honom.
Joh 6:57  Såsom Fadern, han som är den levande, har sänt mig, och såsom jag lever genom Fadern, så skall ock den som äter mig leva genom mig.
Joh 6:58  Så är det med det bröd som har kommit ned från himmelen. Det är icke såsom det fäderna fingo äta, vilka sedan dogo; den som äter detta bröd, han skall leva till evig tid."
Joh 6:59  Detta sade han, när han undervisade i synagogan i Kapernaum.
Joh 6:60  Många av hans lärjungar, som hörde detta, sade då: "Detta är ett hårt tal; vem står ut med att höra på honom?"
Joh 6:61  Men Jesus visste inom sig att hans lärjungar knorrade över detta; och han sade till dem: "Är detta för eder en stötesten?
Joh 6:62  Vad skolen I då säga, om I fån se Människosonen uppstiga dit där han förut var? -
Joh 6:63  Det är anden som gör levande; köttet är till intet gagneligt. De ord som jag har talat till eder äro ande och äro liv.
Joh 6:64  Men bland eder finnas några som icke tro." Jesus visste nämligen från begynnelsen vilka de voro som icke trodde, så ock vilken den var som skulle förråda honom.
Joh 6:65  Och han tillade: "Fördenskull har jag sagt eder att ingen kan komma till mig, om det icke bliver honom givet av Fadern."
Joh 6:66  För detta tals skull drogo sig många av hans lärjungar tillbaka, så att de icke längre vandrade med honom.
Joh 6:67  Då sade Jesus till de tolv: "Icke viljen väl också I gå bort?"
Joh 6:68  Simon Petrus svarade honom: "Herre, till vem skulle vi gå? Du har det eviga livets ord,
Joh 6:69  och vi tro och förstå att du är Guds helige."
Joh 6:70  Jesus svarade dem: "Har icke jag själv utvalt eder, I tolv? Och likväl är en av eder en djävul."
Joh 6:71  Detta sade han om Judas, Simon Iskariots son; ty det var denne som skulle förråda honom, och han var en av de tolv.
Joh 7:1  Därefter vandrade Jesus omkring i Galileen, ty i Judeen ville han icke vandra omkring, då nu judarna stodo efter att döda honom.
Joh 7:2  Men judarnas lövhyddohögtid var nu nära.
Joh 7:3  Då sade hans bröder till honom: "Begiv dig härifrån och gå till Judeen, så att också dina lärjungar få se de gärningar som du gör.
Joh 7:4  Ty ingen som vill vara känd bland människor utför sitt verk i hemlighet. Då du nu gör sådana gärningar, så träd öppet fram för världen."
Joh 7:5  Det var nämligen så, att icke ens hans bröder trodde på honom.
Joh 7:6  Då sade Jesus till dem: "Min tid är ännu icke kommen, men för eder är tiden alltid läglig.
Joh 7:7  Världen kan icke hata eder, men mig hatar hon, eftersom jag vittnar om henne, att hennes gärningar äro onda.
Joh 7:8  Gån I upp till högtiden; jag är icke stadd på väg upp till denna högtid, ty min tid är ännu icke fullbordad."
Joh 7:9  Detta sade han till dem och stannade så kvar i Galileen.
Joh 7:10  Men när hans bröder hade gått upp till högtiden, då gick också han ditupp, dock icke öppet, utan likasom i hemlighet.
Joh 7:11  Och judarna sökte efter honom under högtiden och sade: "Var är han?"
Joh 7:12  Och bland folket talades i tysthet mycket om honom. Somliga sade: "Han är en rättsinnig man", men andra sade: "Nej, han förvillar folket."
Joh 7:13  Dock talade ingen öppet om honom, av fruktan för judarna.
Joh 7:14  Men när redan halva högtiden var förliden, gick Jesus upp i helgedomen och undervisade.
Joh 7:15  Då förundrade sig judarna och sade: "Varifrån har denne sin lärdom, han som icke har fått undervisning?"
Joh 7:16  Jesus svarade dem och sade: "Min lära är icke min, utan hans som har sänt mig.
Joh 7:17  Om någon vill göra hans vilja, så skall han förstå om denna lära är från Gud, eller om jag talar av mig själv.
Joh 7:18  Den som talar av sig själv, han söker sin egen ära; men den som söker dens ära, som har sänt honom, han är sannfärdig, och orättfärdighet finnes icke i honom. -
Joh 7:19  Har icke Moses givit eder lagen? Och likväl fullgör ingen av eder lagen. Varför stån I efter att döda mig?"
Joh 7:20  Folket svarade: "Du är besatt av en ond ande. Vem står efter att döda dig?"
Joh 7:21  Jesus svarade och sade till dem: "En gärning allenast gjorde jag, och alla förundren I eder över den.
Joh 7:22  Moses har givit eder omskärelsen - icke som om den vore ifrån Moses, ty den är ifrån fäderna - och så omskären I människor också på en sabbat.
Joh 7:23  Om nu en människa undfår omskärelsen på en sabbat, för att Moses' lag icke skall göras om intet, huru kunnen I då vredgas på mig, därför att jag på en sabbat gjorde en människa hel och frisk?
Joh 7:24  Dömen icke efter skenet, utan dömen en rätt dom."
Joh 7:25  Då sade några av folket i Jerusalem: "Är det icke denne som de stå efter att döda?
Joh 7:26  Och ändå får han tala fritt, utan att de säga något till honom. Hava då rådsherrarna verkligen blivit förvissade om att denne är Messias?
Joh 7:27  Dock, denne känna vi, och vi veta varifrån han är; men när Messias kommer, känner ingen varifrån han är."
Joh 7:28  Då sade Jesus med hög röst, där han undervisade i helgedomen: "Javäl, I kännen mig, och I veten varifrån jag är. Likväl har jag icke kommit av mig själv, men han som har sänt mig är en som verkligen har myndighet att sända, han som I icke kännen.
Joh 7:29  Men jag känner honom, ty från honom är jag kommen, och han har sänt mig."
Joh 7:30  Då ville de gripa honom; dock kom ingen med sin hand vid honom, ty hans stund var ännu icke kommen.
Joh 7:31  Men många av folket trodde på honom, och de sade: "Icke skall väl Messias, när han kommer, göra flera tecken än denne har gjort?"
Joh 7:32  Sådant fingo fariséerna höra folket i tysthet tala om honom. Då sände översteprästerna och fariséerna ut rättstjänare för att gripa honom.
Joh 7:33  Men Jesus sade: "Ännu en liten tid är jag hos eder; sedan går jag bort till honom som har sänt mig.
Joh 7:34  I skolen då söka efter mig, men I skolen icke finna mig, och där jag är, dit kunnen I icke komma."
Joh 7:35  Då sade judarna till varandra: "Vart tänker denne gå, eftersom vi icke skola kunna finna honom? Månne han tänker gå till dem som bo kringspridda bland grekerna? Tänker han då undervisa grekerna?
Joh 7:36  Vad betyder det ord som han sade: 'I skolen söka efter mig, men I skolen icke finna mig, och där jag är, dit kunnen I icke komma'?"
Joh 7:37  På den sista dagen i högtiden, som ock var den förnämsta, stod Jesus där och ropade och sade: "Om någon törstar, så komme han till mig och dricke.
Joh 7:38  Den som tror på mig, av hans innersta skola strömmar av levande vatten flyta fram, såsom skriften säger."
Joh 7:39  Detta sade han om Anden, vilken de som trodde på honom skulle undfå; ty ande var då ännu icke given, eftersom Jesus ännu icke hade blivit förhärligad.
Joh 7:40  Några av folket, som hörde dessa ord, sade då: "Denne är förvisso Profeten."
Joh 7:41  Andra sade: "Han är Messias." Andra åter sade: "Icke kommer väl Messias från Galileen?
Joh 7:42  Säger icke skriften att Messias skall komma av Davids säd och från den lilla staden Betlehem, där David bodde?
Joh 7:43  Så uppstodo för hans skull stridiga meningar bland folket,
Joh 7:44  och somliga av dem ville gripa honom; dock kom ingen med sin hand vid honom.
Joh 7:45  När sedan rättstjänarna kommo tillbaka till översteprästerna och fariséerna, frågade dessa dem: "Varför haven I icke fört honom hit?"
Joh 7:46  Tjänarna svarade: "Aldrig har någon människa talat, som den mannen talar."
Joh 7:47  Då svarade fariséerna dem: "Haven nu också I blivit förvillade?
Joh 7:48  Har då någon av rådsherrarna trott på honom? Eller någon av fariséerna?
Joh 7:49  Nej; men detta folk, som icke känner lagen, det är förbannat.
Joh 7:50  Då sade Nikodemus till dem, han som förut hade besökt honom, och som själv var en av dem:
Joh 7:51  "Icke dömer väl vår lag någon, utan att man först har förhört honom och utrönt vad han förehar?"
Joh 7:52  De svarade och sade till honom: "Kanske också du är från Galileen? Rannsaka, så skall du finna att ingen profet kommer från Galileen."
Joh 7:53  Och de gingo hem, var och en till sitt. om natten.
Joh 8:1  Och Jesus gick ut till Oljeberget.
Joh 8:2  Men i dagbräckningen kom han åter till helgedomen (och allt folket kom till honom, och han satte sig och lärde dem.).
Joh 8:3  Då förde översteprästerna och fariséerna dit en kvinna som hade blivit beträdd med äktenskapsbrott; och när de hade lett henne fram,
Joh 8:4  sade de till honom: "Mästare, denna kvinna har på bar gärning blivit beträdd med äktenskapsbrott.
Joh 8:5  Nu bjuder Moses i lagen att sådana skola stenas. Vad säger då du?"
Joh 8:6  Detta sade de för att snärja honom, på det att de skulle få något att anklaga honom för. Då böjde Jesus sig ned och skrev med fingret på jorden.
Joh 8:7  Men när de stodo fast vid sin fråga, reste han sig upp och sade till dem: "Den av eder som är utan synd, han kaste första stenen på henne."
Joh 8:8  Sedan böjde han sig åter ned och skrev på jorden.
Joh 8:9  När de hörde detta (och kände sig överbevisade av samvetet.), gingo de ut, den ene efter den andre, först de äldsta, och Jesus blev lämnad allena med kvinnan, som stod där kvar.
Joh 8:10  Då såg Jesus upp och sade till kvinnan: "Var äro de andra? Har ingen dömt dig?"
Joh 8:11  Hon svarade: "Herre, ingen." Då sade han till henne: "Icke heller jag dömer dig. Gå, och synda icke härefter."
Joh 8:12  Åter talade Jesus till dem och sade: "Jag är världens ljus; den som följer mig, han skall förvisso icke vandra i mörkret, utan skall hava livets ljus."
Joh 8:13  Då sade fariséerna till honom: "Du vittnar om dig själv; ditt vittnesbörd gäller icke."
Joh 8:14  Jesus svarade och sade till dem: "Om jag än vittnar om mig själv, så gäller dock mitt vittnesbörd, ty jag vet varifrån jag har kommit, och vart jag går; men I veten icke varifrån jag kommer, eller vart jag går.
Joh 8:15  I dömen efter köttet; jag dömer ingen.
Joh 8:16  Och om jag än dömer, så är min dom en rätt dom, ty jag är därvid icke ensam, utan med mig är han som har sänt mig.
Joh 8:17  I eder lag är ju ock skrivet att vad två människor vittna, det gäller såsom sant.
Joh 8:18  Här är nu jag som vittnar om mig; om mig vittnar också Fadern, som har sänt mig.
Joh 8:19  Då sade de till honom: "Var är då din Fader?" Jesus svarade: "I kännen varken mig eller min Fader. Om I känden mig, så känden I ock min Fader."
Joh 8:20  Det var på det ställe där offerkistorna stodo som han talade dessa ord, medan han undervisade i helgedomen; men ingen bar hand på honom, ty hans stund var ännu icke kommen.
Joh 8:21  Åter sade han till dem: "Jag går bort, och I skolen då söka efter mig; men I skolen dö i eder synd. Dig jag går, dit kunnen I icke komma."
Joh 8:22  Då sade judarna: "Icke vill han väl dräpa sig själv, eftersom han säger: 'Dit jag går, dit kunnen I icke komma'?"
Joh 8:23  Men han svarade dem: "I ären härnedifrån, jag är ovanifrån; I ären av denna världen, jag är icke av denna världen.
Joh 8:24  Därför sade jag till eder att I skullen dö i edra synder; ty om I icke tron att jag är den jag är, så skolen I dö i edra synder."
Joh 8:25  Då frågade de honom: "Vem är du då?" Jesus svarade dem: "Det som jag redan från begynnelsen har uttalat för eder.
Joh 8:26  Mycket har jag ännu att tala och att döma i fråga om eder. Men han som har sänt mig är sannfärdig, och vad jag har hört av honom, det talar jag ut inför världen."
Joh 8:27  Men de förstodo icke att det var om Fadern som han talade till dem.
Joh 8:28  Då sade Jesus: "När I haven upphöjt Människosonen, då skolen I första att jag är den jag är, och att jag icke gör något av mig själv, utan talar detta såsom Fadern har lärt mig.
Joh 8:29  Och han som har sänt mig är med mig; han har icke lämnat mig allena, eftersom jag alltid gör vad honom behagar."
Joh 8:30  När han talade detta, kommo många till tro på honom.
Joh 8:31  Då sade Jesus till de judar som hade satt tro till honom: "Om I förbliven i mitt ord, så ären I i sanning mina lärjungar;
Joh 8:32  Och I skolen då förstå sanningen, och sanningen skall göra eder fria."
Joh 8:33  De svarade honom: "Vi äro Abrahams säd och hava aldrig varit trälar under någon. Huru kan du då säga: 'I skolen bliva fria'?"
Joh 8:34  Jesus svarade dem: "Sannerligen, sannerligen säger jag eder: Var och en som gör synd, han är syndens träl.
Joh 8:35  Men trälen får icke förbliva i huset för alltid; sonen får förbliva där för alltid.
Joh 8:36  Om nu Sonen gör eder fria, så bliven i verkligen fria.
Joh 8:37  Jag vet att I ären Abrahams säd; men I stån efter att döda mig, eftersom mitt ord icke får någon ingång i eder.
Joh 8:38  Jag talar vad jag har sett hos min Fader; så gören ock I vad I haven hört av eder fader."
Joh 8:39  De svarade och sade till honom: "Vår fader är ju Abraham." Jesus svarade till dem: "Ären I Abrahams barn, så gören ock Abrahams gärningar.
Joh 8:40  Men nu stån I efter att döda mig, en man som har sagt eder sanningen, såsom jag har hört den av Gud. Så handlade icke Abraham.
Joh 8:41  Nej, I gören eder faders gärningar." De sade till honom: "Vi äro icke födda i äktenskapsbrott. Vi hava Gud till fader och ingen annan."
Joh 8:42  Jesus svarade dem: "Vore Gud eder fader, så älskaden I ju mig, ty från Gud har jag utgått, och från honom är jag kommen. Ja, jag har icke kommit av mig själv, utan det är han som har sänt mig.
Joh 8:43  Varför fatten I då icke vad jag talar? Jo, därför att I icke 'stån ut med' att höra på mitt ord.
Joh 8:44  I haven djävulen till eder fader, och vad eder fader har begär till, det viljen i göra. Han har varit en mandråpare från begynnelsen, och i sanningen står han icke, ty sanning finnes icke i honom. När han talar lögn, då talar han av sitt eget, ty han är en lögnare, ja, lögnens fader.
Joh 8:45  Men mig tron I icke, just därför att jag talar sanning.
Joh 8:46  Vilken av eder kan överbevisa mig om någon synd? Om jag alltså talar sanning, varför tron I mig då icke?
Joh 8:47  Den som är av Gud, han lyssnar till Guds ord; och det är därför att I icke ären av Gud som I icke lyssnen därtill.
Joh 8:48  Judarna svarade och sade till honom: "Hava vi icke rätt, då vi säga att du är en samarit och är besatt av en ond ande?"
Joh 8:49  Jesus svarade: "Jag är icke besatt av någon ond ande; fastmer hedrar jag min Fader. I åter skymfen mig.
Joh 8:50  Men jag söker icke min egen ära; en finnes dock som söker den och som dömer.
Joh 8:51  Sannerligen, sannerligen säger jag eder: Den som håller mitt ord, han skall aldrig någonsin se döden."
Joh 8:52  Judarna sade till honom: "Nu förstå vi att du är besatt av en ond ande. Abraham har dött, så ock profeterna, och likväl säger du: 'Den som håller mitt ord, han skall aldrig någonsin smaka döden.'
Joh 8:53  Icke är väl du förmer än vår Fader Abraham? Och han har ju dött. Profeterna hava också dött. Till vem gör du då dig själv?"
Joh 8:54  Jesus svarade: "Om jag själv ville skaffa mig ära, så vore min ära intet; men det är min Fader som förlänar mig ära, han som I säger vara eder Gud.
Joh 8:55  Dock, I haven icke lärt känna honom, men jag känner honom; och om jag sade att jag icke kände honom, så bleve jag en lögnare likasom I; men jag känner honom och håller hans ord.
Joh 8:56  Abraham, eder fader, fröjdade sig över att han skulle få se min dag. Han fick se den och blev glad."
Joh 8:57  Då sade judarna till honom: "Femtio år gammal är du icke ännu, och Abraham har du sett!"
Joh 8:58  Jesus sade till dem: "Sannerligen, sannerligen säger jag eder: Förrän Abraham blev till, är jag."
Joh 8:59  Då togo de upp stenar för att kasta på honom. Men Jesus gömde sig undan och gick sedan ut ur helgedomen.
Joh 9:1  När han nu gick vägen fram, fick han se en man som var född blind.
Joh 9:2  Då frågade hans lärjungar honom och sade: "Rabbi, vilken har syndat, denne eller hans föräldrar, så att han har blivit född blind?"
Joh 9:3  Jesus svarade: "Det är varken denne som har syndat eller hans föräldrar, utan så har skett, för att Guds gärningar skulle uppenbaras på honom.
Joh 9:4  Medan dagen varar, måste vi göra dens gärningar, som har sänt mig; natten kommer, då ingen kan verka.
Joh 9:5  Så länge jag är i världen, är jag världens ljus."
Joh 9:6  När han hade sagt detta, spottade han på jorden och gjorde en deg av spotten och lade degen på mannens ögon
Joh 9:7  och sade till honom: "Gå bort och två dig i dammen Siloam" (det betyder utsänd). Mannen gick då dit och tvådde sig; och när han kom igen, kunde han se.
Joh 9:8  Då sade grannarna och andra som förut hade sett honom såsom tiggare: "Är detta icke den man som att och tiggde?"
Joh 9:9  Somliga svarade: "Det är han." Andra sade: "Nej, men han är lik honom." Själv sade han: "Jag är den mannen."
Joh 9:10  Och de frågade honom: "Huru blevo då dina ögon öppnade?"
Joh 9:11  Han svarade: "Den man som heter Jesus gjorde en deg och smorde därmed mina ögon och sade till mig: 'Gå bort till Siloam och två dig.' Jag gick då dit och tvådde mig, och så fick jag min syn."
Joh 9:12  De frågade honom: "Var är den mannen?" Han svarade: "Det vet jag icke."
Joh 9:13  Då förde de honom, mannen som förut hade varit blind, bort till fariséerna.
Joh 9:14  Och det var sabbat den dag då Jesus gjorde degen och öppnade hans ögon.
Joh 9:15  När nu jämväl fariséerna i sin ordning frågade honom huru han hade fått sin syn, svarade han dem: "Han lade en deg på mina ögon, och jag fick två mig, och nu kan jag se."
Joh 9:16  Då sade några av fariséerna: "Den mannen är icke från Gud, eftersom han icke håller sabbaten." Andra sade: "Huru skulle någon som är en syndare kunna göra sådana tecken?" Så funnos bland dem stridiga meningar.
Joh 9:17  Då frågade de åter den blinde: "Vad säger du själv om honom, då det ju var dina ögon han öppnade?" Han svarade: "En profet är han."
Joh 9:18  Men judarna trodde icke att han hade varit blind och fått sin syn, förrän de hade kallat till sig mannen föräldrar, hans som hade fått sin syn.
Joh 9:19  Dem frågade de och sade: "Är detta eder son, den som I sägen vara född blind? Huru kommer det då till, att han nu kan se?"
Joh 9:20  Då svarade han föräldrar och sade: "Att denne är vår son, och att han föddes blind, det veta vi.
Joh 9:21  Men huru han nu kan se, det veta vi icke, ej heller veta vi vem som har öppnat hans ögon. Frågen honom själv; han är gammal nog, han må själv tala för sig."
Joh 9:22  Detta sade hans föräldrar, därför att de fruktade judarna; ty judarna hade redan kommit överens om att den som bekände Jesus vara Messias, han skulle utstötas ur synagogan.
Joh 9:23  Därför var det som hans föräldrar sade: "Han är gammal nog; frågen honom själv."
Joh 9:24  Då kallade de för andra gången till sig mannen som hade varit blind och sade till honom: "Säg nu sanningen, Gud till pris. Vi veta att denne man är en syndare."
Joh 9:25  Han svarade: "Om han är en syndare vet jag icke; ett vet jag: att jag, som var blind, nu kan se."
Joh 9:26  Då frågade de honom: "Vad gjorde han med dig? På vad sätt öppnade han dina ögon?"
Joh 9:27  Han svarade dem: "Jag har ju redan sagt eder det, men I hörden icke på mig. Varför viljen I då åter höra det? Kanske viljen också I bliva hans lärjungar?"
Joh 9:28  Då bannade de honom och sade: "Du är själv hans lärjunge; vi äro Moses' lärjungar.
Joh 9:29  Till Moses har Gud talat, det veta vi; men varifrån denne är, det veta vi icke."
Joh 9:30  Mannen svarade och sade till dem: "Ja, däri ligger det förunderliga, att I icke veten varifrån han är, och ändå har han öppnat mina ögon.
Joh 9:31  Vi veta ju att Gud icke hör syndare, men också att om någon är gudfruktig och gör hans vilja, då hör han honom.
Joh 9:32  Aldrig förut har man hört att någon har öppnat ögonen på en som föddes blind.
Joh 9:33  Vore denne icke från Gud, så kunde han intet göra."
Joh 9:34  De svarade och sade till honom: "Du är hel och hållen född i synd, och du vill undervisa oss!" Och så drevo de ut honom.
Joh 9:35  Jesus fick sedan höra att de hade drivit ut honom, och när han så träffade honom, sade han: "Tror du på Människosonen?"
Joh 9:36  Han svarade och sade: "Herre, vem är han då? Säg mig det, så att jag kan tro på honom."
Joh 9:37  Jesus sade till honom: "Du har sett honom; det är han som talar med dig."
Joh 9:38  Då sade han: "Herre, jag tror." Och han föll ned för honom.
Joh 9:39  Och Jesus sade: "Till en dom har jag kommit hit i världen, för att de som icke se skola varda seende, och för att de som se skola varda blinda."
Joh 9:40  När några fariséer som voro i hans närhet hörde detta, sade de till honom: "Äro då kanske också vi blinda?"
Joh 9:41  Jesus svarade dem: "Voren I blinda, så haden I icke synd. Men nu sägen I: 'Vi se', därför står eder synd kvar."
Joh 10:1  "Sannerligen, sannerligen säger jag eder: Den som icke går in i fårahuset genom dörren, utan stiger in någon annan väg, han är en tjuv och en rövare.
Joh 10:2  Men den som går in genom dörren, han är fårens herde.
Joh 10:3  För honom öppnar dörrvaktaren, och fåren lyssna till hans röst, och han kallar sina får vid namn och för dem ut.
Joh 10:4  Och när han har släppt ut alla sina får, går han framför dem, och fåren följa honom, ty de känna hans röst.
Joh 10:5  Men en främmande följa de alls icke, utan fly bort ifrån honom, ty de känna icke de främmandes röst."
Joh 10:6  Så talade Jesus till dem i förtäckta ord; men de förstodo icke vad det var som han talade till dem.
Joh 10:7  Åter sade Jesus till dem: "Sannerligen, sannerligen säger jag eder: Jag är dörren in till fåren.
Joh 10:8  Alla de som hava kommit före mig äro tjuvar och rövare, men fåren hava icke lyssnat till dem.
Joh 10:9  Jag är dörren; den som går in genom mig, han skall bliva frälst, och han skall få gå ut och in och skall finna bete.
Joh 10:10  Tjuven kommer allenast för att stjäla och slakta och förgöra. Jag har kommit, för att de skola hava liv och hava över nog.
Joh 10:11  Jag är den gode herden. En god herde giver sitt liv för fåren.
Joh 10:12  Men den som är lejd och icke är herden själv, när han, den som fåren icke tillhöra, ser ulven komma, då övergiver han fåren och flyr, och ulven rövar bort dem och förskingrar dem.
Joh 10:13  Han är ju lejd och frågar icke efter fåren.
Joh 10:14  Jag är den gode herden, och jag känner mina får, och mina får känna mig,
Joh 10:15  såsom Fadern känner mig, och såsom jag känner Fadern; och jag giver mitt liv för fåren.
Joh 10:16  Jag har ock andra får, som icke höra till detta fårahus; också dem måste jag draga till mig, och de skola lyssna till min röst. Så skall det bliva en hjord och en herde.
Joh 10:17  Därför älskar Fadern mig, att jag giver mitt liv - för att sedan taga igen det.
Joh 10:18  Ingen tager det ifrån mig, utan jag giver det av fri vilja. Jag har makt att giva det, och jag har makt att taga igen det. Det budet har jag fått av min Fader."
Joh 10:19  För dessa ords skull uppstodo åter stridiga meningar bland judarna.
Joh 10:20  Många av dem sade: "Han är besatt av en ond ande och är från sina sinnen. Varför hören I på honom?"
Joh 10:21  Andra åter sade: "Sådana ord talar icke den som är besatt. Icke kan väl en ond ande öppna blindas ögon?"
Joh 10:22  Därefter inföll tempelinvigningens högtid i Jerusalem. Det var nu vinter,
Joh 10:23  och Jesus gick fram och åter i Salomos pelargång i helgedomen.
Joh 10:24  Då samlade sig judarna omkring honom och sade till honom: "Huru länge vill du hålla oss i ovisshet? Om du är Messias, så säg oss det öppet."
Joh 10:25  Jesus svarade dem: "Jag har sagt eder det, men I tron mig icke. De gärningar som jag gör i min Faders namn, de vittna om mig.
Joh 10:26  Men I tron mig icke, ty I ären icke av mina får.
Joh 10:27  Mina får lyssna till min röst, och jag känner dem, och de följa mig.
Joh 10:28  Och jag giver dem evigt liv, och de skola aldrig någonsin förgås, och ingen skall rycka dem ur min hand.
Joh 10:29  Min Fader, som har givit mig dem, är större än alla, och ingen kan rycka dem ur min Faders hand.
Joh 10:30  Jag och Fadern äro ett."
Joh 10:31  Då togo judarna åter upp stenar för att stena honom.
Joh 10:32  Men Jesus sade till dem: "Många goda gärningar, som komma från min Fader, har jag låtit eder se. För vilken av dessa gärningar är det som I viljen stena mig?"
Joh 10:33  Judarna svarade honom: "Det är icke för någon god gärnings skull som vi vilja stena dig, utan därför att du hädar och gör dig själv till Gud, du som är en människa."
Joh 10:34  Jesus svarade dem: "Det är ju så skrivet i eder lag: 'Jag har sagt att I ären gudar'.
Joh 10:35  Om han nu har kallat för gudar dem som Guds ord kom till - och skriften kan ju icke bliva om intet -
Joh 10:36  huru kunnen då I, på den grund att jag har sagt mig vara Guds Son, anklaga mig för hädelse, mig som Fadern har helgat och sänt i världen?
Joh 10:37  Gör jag icke min Faders gärningar, så tron mig icke.
Joh 10:38  Men gör jag dem, så tron gärningarna, om I än icke tron mig; då skolen I fatta och förstå att Fadern är i mig, och att jag är i Fadern."
Joh 10:39  Då ville de åter gripa honom, men han gick sin väg, undan deras händer.
Joh 10:40  Sedan begav han sig åter bort till det ställe på andra sidan Jordan, där Johannes först hade döpt, och stannade kvar där.
Joh 10:41  Och många kommo till honom. Och de sade: "Väl gjorde Johannes intet tecken, men allt vad Johannes sade om denne var sant."
Joh 10:42  Och många kommo där till tro på honom.
Joh 11:1  Och en man vid namn Lasarus låg sjuk; han var från Betania, den by där Maria och hennes syster Marta bodde.
Joh 11:2  Det var den Maria som smorde Herren med smörjelse och torkade hans fötter med sitt hår. Och nu låg hennes broder Lasarus sjuk.
Joh 11:3  Då sände systrarna bud till Jesus och läto säga: "Herre, se, han som du har så kär ligger sjuk."
Joh 11:4  När Jesus hörde detta, sade han: "Den sjukdomen är icke till döds, utan till Guds förhärligande, så att Guds Son genom den bliver förhärligad."
Joh 11:5  Och Jesus hade Marta och hennes syster och Lasarus kära.
Joh 11:6  När han nu hörde att denne låg sjuk, stannade han först två dagar där han var;
Joh 11:7  men därefter sade han till lärjungarna: "Låt oss gå tillbaka till Judeen."
Joh 11:8  Lärjungarna sade till honom: "Rabbi, nyligen ville judarna stena dig, och åter går du dit?"
Joh 11:9  Jesus svarade: "Dagen har ju tolv timmar; den som vandrar om dagen, han stöter sig icke, ty han ser då denna världens ljus.
Joh 11:10  Men den som vandrar om natten, han stöter sig, ty han har då intet som lyser honom."
Joh 11:11  Sedan han hade talat detta, sade han ytterligare till dem: "Lasarus, vår vän, har somnat in; men jag går för att väcka upp honom ur sömnen."
Joh 11:12  Då sade hans lärjungar till honom: "Herre, sover han, så bliver han frisk igen."
Joh 11:13  Men Jesus hade talat om hans död; de åter menade att han talade om vanlig sömn.
Joh 11:14  Då sade Jesus öppet till dem: "Lasarus är död.
Joh 11:15  Och för eder skull, för att I skolen tro, gläder jag mig över att jag icke var där. Men låt oss nu gå till honom."
Joh 11:16  Då sade Tomas, som kallades Didymus, till de andra lärjungarna: "Låt oss gå med, för att vi må dö med honom."
Joh 11:17  När så Jesus kom dit, fann han att den döde redan hade legat fyra dagar i graven.
Joh 11:18  Nu låg Betania nära Jerusalem, vid pass femton stadier därifrån,
Joh 11:19  och många judar hade kommit till Marta och Maria för att trösta dem i sorgen över deras broder.
Joh 11:20  Då nu Maria fick höra att Jesus kom, gick hon honom till mötes; men Maria satt kvar hemma.
Joh 11:21  Och Marta sade till Jesus: "Herre, hade du varit här, så vore min broder icke död.
Joh 11:22  Men jag vet ändå att allt vad du beder Gud om, det skall Gud giva dig."
Joh 11:23  Jesus sade till henne: "Din broder skall stå upp igen."
Joh 11:24  Marta svarade honom: "Jag vet att han skall stå upp, vid uppståndelsen på den yttersta dagen."
Joh 11:25  Jesus svarade till henne: "Jag är uppståndelsen och livet. Den som tror på mig, han skall leva, om han än dör;
Joh 11:26  och var och en som lever och tror på mig, han skall aldrig någonsin dö. Tror du detta?"
Joh 11:27  Hon svarade honom: "Ja, Herre, jag tror att du är Messias, Guds Son, han som skulle komma i världen."
Joh 11:28  När hon hade sagt detta, gick hon bort och kallade på Maria, sin syster, och sade hemligen till henne: "Mästaren är här och kallar dig till sig."
Joh 11:29  När hon hörde detta, stod hon strax upp och gick åstad till honom.
Joh 11:30  Men Jesus hade ännu icke kommit in i byn, utan var kvar på det ställe där Marta hade mött honom.
Joh 11:31  Då nu de judar, som voro inne i huset hos Maria för att trösta henne, sågo att hon så hastigt stod upp och gick ut, följde de henne, i tanke att hon gick till graven för att gråta där.
Joh 11:32  När så Maria kom till det ställe där Jesus var och fick se honom, föll hon ned för hans fötter och sade till honom: "Herre, hade du varit där, så vore min broder icke död."
Joh 11:33  Då nu Jesus såg henne gråta och såg jämväl att de judar, som hade kommit med henne, gräto, upptändes han i sin ande och blev upprörd
Joh 11:34  och frågade: "Var haven I lagt honom?" De svarade honom: "Herre, kom och se." Och Jesus grät.
Joh 11:35  Då sade judarna: "Se huru kär han hade honom!"
Joh 11:36  Men somliga av dem sade:
Joh 11:37  "Kunde icke han, som öppnade den blindes ögon, ock hava så gjort att denne icke hade dött?"
Joh 11:38  Då upptändes Jesus åter i sitt innersta och gick bort till graven. Den var urholkad i berget, och en sten låg framför ingången.
Joh 11:39  Jesus sade: "Tagen bort stenen." Då sade den dödes syster Marta till honom: "Herre, han luktar redan, ty han har varit död i fyra dygn."
Joh 11:40  Jesus svarade henne: "Sade jag dig icke, att om du trodde, skulle du få se Guds härlighet?"
Joh 11:41  Då togo de bort stenen. Och Jesus lyfte upp sina ögon och sade: "Fader, jag tackar dig för att du har hört mig.
Joh 11:42  Jag visste ju förut att du alltid hör mig; men för folkets skull, som står här omkring, säger jag detta, för att de skola tro att det är du som har sänt mig."
Joh 11:43  När han hade sagt detta, ropade han med hög röst: "Lasarus, kom ut."
Joh 11:44  Och han som hade varit död kom ut, med händer och fötter inlindade i bindlar och med ansiktet inhöljt i en duk. Jesus sade till dem: "Lösen honom, och låten honom gå."
Joh 11:45  Många judar, som hade kommit till Maria och hade sett vad Jesus hade gjort, trodde då på honom.
Joh 11:46  Men några av dem gingo bort till fariséerna och omtalade för dem vad Jesus hade gjort.
Joh 11:47  Då sammankallade översteprästerna och fariséerna en rådsförsamling och sade: "Vad skola vi taga oss till? Denne man gör ju många tecken.
Joh 11:48  Om vi skola låta honom så fortfara, skola alla tro på honom, och romarna komma då att taga ifrån oss både land och folk."
Joh 11:49  Men en av dem, Kaifas, som var överstepräst för det året, sade till dem: "I förstån intet,
Joh 11:50  och I besinnen icke huru mycket bättre det är för eder att en man dör för folket, än att hela folket förgås."
Joh 11:51  Detta sade han icke av sig själv, utan genom profetisk ingivelse, eftersom han var överstepräst för det året; ty Jesus skulle dö för folket.
Joh 11:52  Ja, icke allenast "för folket"; han skulle dö också för att samla och förena Guds förskingrade barn.
Joh 11:53  Från den dagen var deras beslut fattat att döda honom.
Joh 11:54  Så vandrade då Jesus icke längre öppet bland judarna, utan drog sig undan till en stad som hette Efraim, på landsbygden, i närheten av öknen; där stannade han kvar med sina lärjungar.
Joh 11:55  Men judarnas påsk var nära, och många begåvo sig då, före påsken, från landsbygden upp till Jerusalem för att helga sig.
Joh 11:56  Och de sökte efter Jesus och sade till varandra, där de stodo i helgedom: "Vad menen I? Skall han då alls icke komma till högtiden?"
Joh 11:57  Och översteprästerna och fariséerna hade utfärdat påbud om att den som finge veta var han fanns skulle giva det till känna, för att de måtte kunna gripa honom.
Joh 12:1  Sex dagar före påsk kom nu Jesus till Betania, där Lasarus bodde, han som av Jesus hade blivit uppväckt från de döda.
Joh 12:2  Där gjorde man då för honom ett gästabud, och Marta betjänade dem, men Lasarus var en av dem som lågo till bords jämte honom.
Joh 12:3  Då tog Maria ett skålpund smörjelse av dyrbar äkta nardus och smorde därmed Jesu fötter; sedan torkade hon hans fötter med sitt hår. Och huset uppfylldes med vällukt av smörjelsen.
Joh 12:4  Men Judas Iskariot, en av hans lärjungar, den som skulle förråda honom, sade då:
Joh 12:5  "Varför sålde man icke hellre denna smörjelse för tre hundra silverpenningar och gav dessa åt de fattiga?"
Joh 12:6  Detta sade han, icke därför, att han frågade efter de fattiga, utan därför, att han var en tjuv och plägade taga vad som lades i penningpungen, vilken han hade om hand.
Joh 12:7  Men Jesus sade: "Låt henne vara; må hon få fullgöra detta för min begravningsdag.
Joh 12:8  De fattiga haven I ju alltid ibland eder, men mig haven I icke alltid."
Joh 12:9  Nu hade det blivit känt för den stora hopen av judarna att Jesus var där, och de kommo dit, icke allenast för hans skull, utan ock för att se Lasarus, som han hade uppväckt från de döda.
Joh 12:10  Då beslöto översteprästerna att döda också Lasarus.
Joh 12:11  Ty för hans skull gingo många judar bort och trodde på Jesus.
Joh 12:12  När dagen därefter det myckna folk som hade kommit till högtiden fick höra att Jesus var på väg till Jerusalem,
Joh 12:13  togo de palmkvistar och gingo ut för att möta honom och ropade: "Hosianna! Välsignad vare han som kommer, i Herrens namn, han som är Israels konung."
Joh 12:14  Och Jesus fick sig en åsnefåle och satte sig upp på den, såsom det är skrivet:
Joh 12:15  "Frukta icke, du dotter Sion. Se, din konung kommer, sittande på en åsninnas fåle."
Joh 12:16  Detta förstodo hans lärjungar icke då strax, men när Jesus hade blivit förhärligad, då kommo de ihåg att detta var skrivet om honom, och att man hade gjort detta med honom.
Joh 12:17  Så gav nu folket honom sitt vittnesbörd, de som hade varit med honom, när han kallade Lasarus ut ur graven och uppväckte honom från de döda.
Joh 12:18  Därför kom också det övriga folket emot honom, eftersom de hörde att han hade gjort det tecknet.
Joh 12:19  Då sade fariséerna till varandra: "I sen att I alls intet kunnen uträtta; hela världen löper ju efter honom."
Joh 12:20  Nu voro där ock några greker, av dem som plägade fara upp för att tillbedja under högtiden.
Joh 12:21  Dessa kommo till Filippus, som var från Betsaida i Galileen, och bådo honom och sade: "Herre, vi skulle vilja se Jesus."
Joh 12:22  Filippus gick och sade detta till Andreas; Andreas och Filippus gingo och sade det till Jesus.
Joh 12:23  Jesus svarade dem och sade: "Stunden är kommen att Människosonen skall förhärligas.
Joh 12:24  Sannerligen, sannerligen säger jag eder: Om icke vetekornet faller i jorden och dör, så förbliver det ett ensamt korn; men om det dör, så bär det mycken frukt.
Joh 12:25  Den som älskar sitt liv, han mister det, men den som hatar sitt liv i denna världen, han skall behålla det och skall hava evigt liv.
Joh 12:26  Om någon vill tjäna mig, så följe han mig; och där jag är, där skall också min tjänare få vara. Om någon tjänar mig, så skall min Fader ära honom.
Joh 12:27  Nu är min själ i ångest; vad skall jag väl säga? Fader, fräls mig undan denna stund. Dock, just därför har jag kommit till denna stund.
Joh 12:28  Fader, förhärliga ditt namn." Då kom en röst från himmelen: "Jag har redan förhärligat det, och jag skall ytterligare förhärliga det."
Joh 12:29  Folket, som stod där och hörde detta, sade då: "Det var ett tordön." Andra sade: "Det var en ängel som talade med honom."
Joh 12:30  Då svarade Jesus och sade: "Denna röst kom icke för min skull, utan för eder skull."
Joh 12:31  Nu går en dom över denna världen, nu skall denna världens furste utkastas.
Joh 12:32  Och när jag har blivit upphöjd från jorden, skall jag draga alla till mig."
Joh 12:33  Med dessa ord gav han till känna på vad sätt han skulle dö.
Joh 12:34  Då svarade folket honom: "Vi hava hört av lagen att Messias skall stanna kvar för alltid. Huru kan du då säga att Människosonen måste bliva upphöjd? Vad är väl detta för en Människoson?"
Joh 12:35  Jesus sade till dem: "Ännu en liten tid är ljuset ibland eder. Vandren medan I haven ljuset, på det att mörkret icke må få makt med eder; den som vandrar i mörkret, han vet ju icke var han går.
Joh 12:36  Tron på ljuset, medan I haven ljuset, så att I bliven ljusets barn." Detta talade Jesus och gick sedan bort och dolde sig för dem.
Joh 12:37  Men fastän han hade gjort så många tecken inför dem, trodde de icke på honom.
Joh 12:38  Ty det ordet skulle fullbordas, som profeten Esaias säger: "Herre, vem trodde, vad som predikades för oss, och för vem var Herrens arm uppenbar?"
Joh 12:39  Alltså kunde de icke tro; Esaias säger ju ytterligare:
Joh 12:40  "Han har förblindat deras ögon och förstockat deras hjärtan, så att de icke kunna se med sina ögon eller förstå med sina hjärtan och omvända sig och bliva helade av mig."
Joh 12:41  Detta kunde Esaias säga, eftersom han hade sett hans härlighet, när han talade med honom. -
Joh 12:42  Dock funnos jämväl bland rådsherrarna många som trodde på honom; men för fariséernas skulle ville de icke bekänna det, för att de icke skulle bliva utstötta ur synagogan.
Joh 12:43  Ty de skattade högre att bliva ärade av människor än att bliva ärade av Gud.
Joh 12:44  Men Jesus sade med hög röst: "Den som tror på mig, han tror icke på mig, utan på honom som har sänt mig.
Joh 12:45  Och den som ser mig, han ser honom som har sänt mig.
Joh 12:46  Såsom ett ljus har jag kommit i världen, för att ingen av dem som tro på mig skall förbliva i mörkret.
Joh 12:47  Om någon hör mina ord, men icke håller dem, så dömer icke jag honom; ty jag har icke kommit för att döma världen, utan för att frälsa världen.
Joh 12:48  Den som förkastar mig och icke tager emot mina ord, han har dock en domare över sig; det ord som jag har talat, det skall döma honom på den yttersta dagen.
Joh 12:49  Ty jag har icke talat av mig själv, utan Fadern, som har sänt mig, han har bjudit mig vad jag skall säga, och vad jag skall tala.
Joh 12:50  Och jag vet att hans bud är evigt liv; därför, vad jag talar, det talar jag såsom Fadern har sagt mig."
Joh 13:1  Före påskhögtiden hände sig detta. Jesus visste att stunden var kommen för honom att gå bort ifrån denna världen till Fadern; och såsom han allt hittills hade älskat sina egna här i världen, så gav han dem nu ett yttersta bevis på sin kärlek.
Joh 13:2  De höllo nu aftonmåltid, och djävulen hade redan ingivit Judas Iskariot, Simons son, i hjärtat att förråda Jesus.
Joh 13:3  Och Jesus visste att Fadern hade givit allt i hans händer, och att han hade gått ut från Gud och skulle gå till Gud.
Joh 13:4  Men han stod upp från måltiden och lade av sig överklädnaden och tog en linneduk och band den om sig.
Joh 13:5  Sedan slog han vatten i ett bäcken och begynte två lärjungarnas fötter och torkade dem med linneduken som han hade bundit om sig.
Joh 13:6  Så kom han till Simon Petrus. Denne sade då till honom: "Herre, skulle du två mina fötter?"
Joh 13:7  Jesus svarade och sade till honom: "Vad jag gör förstår du icke nu, men framdeles skall du fatta det."
Joh 13:8  Petrus sade till honom: "Aldrig någonsin skall du två mina fötter!" Jesus svarade honom: "Om jag icke tvår dig, så har du ingen del med mig."
Joh 13:9  Då sade Simon Petrus till honom: "Herre, icke allenast mina fötter, utan ock händer och huvud!"
Joh 13:10  Jesus svarade honom: "Den som är helt tvagen, han behöver allenast två fötterna; han är ju i övrigt hel och hållen ren. Så ären ock I rena - dock icke alla."
Joh 13:11  Han visste nämligen vem det var som skulle förråda honom; därför sade han att de icke alla voro rena.
Joh 13:12  Sedan han nu hade tvagit deras fötter och tagit på sig överklädnaden och åter lagt sig ned vid bordet, sade han till dem: "Förstån I vad jag har gjort med eder?
Joh 13:13  I kallen mig 'Mästare' och 'Herre', och I säger rätt, ty jag är så.
Joh 13:14  Har nu jag, eder Herre och Mästare, tvagit edra fötter, så ären ock I pliktiga att två varandras fötter.
Joh 13:15  Jag har ju givit eder ett föredöme, för att I skolen göra såsom jag har gjort mot eder.
Joh 13:16  Sannerligen, sannerligen säger jag eder: Tjänaren är icke förmer än sin herre, ej heller sändebudet förmer än den som har sänt honom.
Joh 13:17  Då I veten detta, saliga ären I, om I ock gören det.
Joh 13:18  Jag talar icke om eder alla; jag vet vilka jag har utvalt. Men detta skriftens ord skulle ju fullbordas: 'Den som åt mitt bröd, han lyfte mot mig sin häl.'
Joh 13:19  Redan nu, förrän det sker, säger jag eder det, för att I, när det har skett, skolen tro att jag är den jag är.
Joh 13:20  Sannerligen, sannerligen säger jag eder: Den som tager emot den jag sänder, han tager emot mig; och den som tager emot mig, han tager emot honom som har sänt mig."
Joh 13:21  När Jesus hade sagt detta, blev han upprörd i sin ande och betygade och sade: "Sannerligen, sannerligen säger jag eder: En av eder skall förråda mig."
Joh 13:22  Då sågo lärjungarna på varandra och undrade vilken han talade om.
Joh 13:23  Nu var där bland lärjungarna en som låg till bords invid Jesu bröst, den lärjunge som Jesus älskade.
Joh 13:24  Åt denne gav då Simon Petrus ett tecken och sade till honom: "Säg vilken det är som han talar om."
Joh 13:25  Han lutade sig då mot Jesu bröst och frågade honom: "Herre, vilken är det?"
Joh 13:26  Då svarade Jesus: "Det är den åt vilken jag räcker brödstycket som jag nu doppar." Därvid doppade han brödstycket och räckte det åt Judas, Simon Iskariots son.
Joh 13:27  Då, när denne hade tagit emot brödstycket, for Satan in i honom. Och Jesus sade till honom: "Gör snart vad du gör."
Joh 13:28  Men ingen av dem som lågo där till bords förstod varför han sade detta till honom.
Joh 13:29  Ty eftersom Judas hade penningpungen om hand, menade några att Jesus hade velat säga till honom: "Köp vad vi behöva till högtiden", eller ock att han hade tillsagt honom att giva något åt de fattiga.
Joh 13:30  Då han nu hade tagit emot brödstycket, gick han strax ut; och det var natt.
Joh 13:31  Och när han hade gått ut, sade Jesus: "Nu är Människosonen förhärligad, och Gud är förhärligad i honom.
Joh 13:32  Är nu Gud förhärligad i honom, så skall ock Gud förhärliga honom i sig själv, och han skall snart förhärliga honom.
Joh 13:33  Kära barn, allenast en liten tid är jag ännu hos eder; I skolen sedan söka efter mig, men det som jag sade till judarna: 'Dit jag går, dit kunnen I icke komma', detsamma säger jag nu ock till eder.
Joh 13:34  Ett nytt bud giver jag eder, att I skolen älska varandra; ja, såsom jag har älskat eder, så skolen ock I älska varandra.
Joh 13:35  Om I haven kärlek inbördes, så skola alla därav förstå att I ären mina lärjungar."
Joh 13:36  Då frågade Simon Petrus honom: "Herre, vart går du?" Jesus svarade: "Dit jag går, dit kan du icke nu följa mig; men framdeles skall du följa mig."
Joh 13:37  Petrus sade till honom: "Herre, varför kan jag icke följa dig nu? Mitt liv vill jag giva för dig."
Joh 13:38  Jesus svarade: "Ditt liv vill du giva för mig? Sannerligen, sannerligen säger jag dig: Hanen skall icke gala, förrän du tre gånger har förnekat mig."
Joh 14:1  "Edra hjärtan vare icke oroliga. Tron på Gud; tron ock på mig.
Joh 14:2  I min Faders hus äro många boningar; om så icke voro, skulle jag nu säga eder att jag går bort för att bereda eder rum.
Joh 14:3  Och om jag än går bort för att bereda eder rum, så skall jag dock komma igen och taga eder till mig; ty jag vill att där jag är, där skolen I ock vara.
Joh 14:4  Och vägen som leder dit jag går, den veten I."
Joh 14:5  Tomas sade till honom: "Herre, vi veta icke vart du går; huru kunna vi då veta vägen?"
Joh 14:6  Jesus svarade honom: "Jag är vägen och sanningen och livet; ingen kommer till Fadern utom genom mig.
Joh 14:7  Haden I känt mig, så haden I ock känt min Fader; nu kännen I honom och haven sett honom."
Joh 14:8  Filippus sade till honom: "Herre, låt oss se Fadern, så hava vi nog."
Joh 14:9  Jesus svarade honom: "Så lång tid har jag varit hos eder, och du har icke lärt känna mig, Filippus? Den som har sett mig, han har sett Fadern. Huru kan du då säga: 'Låt oss se Fadern'?
Joh 14:10  Tror du icke att jag är i Fadern, och att Fadern är i mig? De ord jag talar till eder talar jag icke av mig själv. Och gärningarna, dem gör Fadern, som bor i mig; de äro hans verk.
Joh 14:11  Tron mig; jag är i Fadern, och Fadern i mig. Varom icke, så tron för själva gärningarnas skull.
Joh 14:12  Sannerligen, sannerligen säger jag eder: Den som tror på mig, han skall ock själv göra de gärningar som jag gör; och ännu större än dessa skall han göra. Ty jag går till Fadern,
Joh 14:13  och vadhelst I bedjen om i mitt namn, det skall jag göra, på det att Fadern må bliva förhärligad i Sonen.
Joh 14:14  Ja, om I bedjen om något i mitt namn, så skall jag göra det.
Joh 14:15  Älsken I mig, så hållen I mina bud,
Joh 14:16  och jag skall bedja Fadern, och han skall giva eder en annan Hjälpare, som för alltid skall vara hos eder:
Joh 14:17  sanningens Ande, som världen icke kan taga emot, ty hon ser honom icke och känner honom icke. Men I kännen honom, ty han bor hos eder och skall vara i eder.
Joh 14:18  Jag skall icke lämna eder faderlösa; jag skall komma till eder.
Joh 14:19  Ännu en liten tid, och världen ser mig icke mer, men I sen mig. Ty jag lever; I skolen ock leva.
Joh 14:20  På den dagen skolen I förstå att jag är i min Fader, och att I ären i mig, och att jag är i eder.
Joh 14:21  Den som har mina bud och håller dem, han är den som älskar mig; och den som älskar mig, han skall bliva älskad av min Fader, och jag skall älska honom och jag skall uppenbara mig för honom."
Joh 14:22  Judas - icke han som kallades Iskariot - sade då till honom: "Herre, varav kommer det att du tänker uppenbara dig för oss, men icke för världen?"
Joh 14:23  Jesus svarade och sade till honom: "Om någon älskar mig, så håller han mitt ord; och min Fader skall älska honom, och vi skola komma till honom och taga vår boning hos honom.
Joh 14:24  Den som icke älskar mig, han håller icke mina ord; och likväl är det ord som I hören icke mitt, utan Faderns, som har sänt mig.
Joh 14:25  Detta har jag talat till eder, medan jag ännu är kvar hos eder.
Joh 14:26  Men Hjälparen, den helige Ande, som Fadern skall sända i mitt namn, han skall lära eder allt och påminna eder om allt vad jag har sagt eder.
Joh 14:27  Frid lämnar jag efter mig åt eder, min frid giver jag eder; icke giver jag eder den såsom världen giver. Edra hjärtan vare icke oroliga eller försagda.
Joh 14:28  I hörden att jag sade till eder: 'Jag går bort, men jag kommer åter till eder.' Om I älskaden mig, så skullen I ju glädjas över att jag går bort till Fadern, ty Fadern är större än jag.
Joh 14:29  Och nu har jag sagt eder det, förrän det sker, på det att I mån tro, när det har skett.
Joh 14:30  Härefter talar jag icke mycket med eder, ty denna världens furste kommer. I mig finnes intet som hör honom till;
Joh 14:31  men detta sker, för att världen skall förstå att jag älskar Fadern och gör såsom Fadern har bjudit mig. Stån upp, låt oss gå härifrån."
Joh 15:1  "Jag är det sanna vinträdet, och min Fader är vingårdsmannen.
Joh 15:2  Var gren i mig, som icke bär frukt, den tager han bort; och var och en som bär frukt, den rensar han, för att den skall bära mer frukt.
Joh 15:3  I ären redan nu rena, i kraft av det ord som jag har talat till eder.
Joh 15:4  Förbliven i mig, så förbliver ock jag i eder. Såsom grenen icke kan bära frukt av sig själv, utan allenast om den förbliver i vinträdet, så kunnen I det ej heller, om I icke förbliven i mig.
Joh 15:5  Jag är vinträdet, I ären grenarna. Om någon förbliver i mig, och jag i honom, så bär han mycken frukt; ty mig förutan kunnen I intet göra.
Joh 15:6  Om någon icke förbliver i mig, så kastas han ut såsom en avbruten gren och förtorkas; och man samlar tillhopa sådana grenar och kastar dem i elden, och de brännas upp.
Joh 15:7  Om I förbliven i mig, och mina ord förbliva i eder, så mån I bedja om vadhelst I viljen, och det skall vederfaras eder.
Joh 15:8  Därigenom bliver min Fader förhärligad, att i bären mycken frukt och bliven mina lärjungar.
Joh 15:9  Såsom Fadern har älskat mig, så har ock jag älskat eder; förbliven i min kärlek.
Joh 15:10  Om I hållen mina bud, så förbliven I i min kärlek, likasom jag har hållit min Faders bud och förbliver i hans kärlek.
Joh 15:11  Detta har jag talat till eder, för att min glädje skall bo i eder, och för att eder glädje skall bliva fullkomlig.
Joh 15:12  Detta är mitt bud, att I skolen älska varandra, såsom jag har älskat eder.
Joh 15:13  Ingen har större kärlek, än att han giver sitt liv för sina vänner.
Joh 15:14  I ären mina vänner, om I gören vad jag bjuder eder.
Joh 15:15  Jag kallar eder nu icke längre tjänare, ty tjänaren får icke veta vad hans herre gör; vänner kallar jag eder, ty allt vad jag har hört av min Fader har jag kungjort för eder.
Joh 15:16  I haven icke utvalt mig, utan jag har utvalt eder; och jag har bestämt om eder att I skolen gå åstad och bära frukt, sådan frukt som bliver beståndande, på det att Fadern må giva eder vadhelst I bedjen honom om i mitt namn.
Joh 15:17  Ja, det bjuder jag eder, att I skolen älska varandra.
Joh 15:18  Om världen hatar eder, så betänken att hon har hatat mig förr än eder.
Joh 15:19  Voren I av världen, så älskade ju världen vad henne tillhörde; men eftersom I icke ären av världen, utan av mig haven blivit utvalda och tagna ut ur världen, därför hatar världen eder.
Joh 15:20  Kommen ihåg det ord som jag sade till eder: 'Tjänaren är icke förmer än sin herre.' Hava de förföljt mig, så skola de ock förfölja eder; hava de hållit mitt ord, så skola de ock hålla edert.
Joh 15:21  Men allt detta skola de göra mot eder för mitt namns skull, eftersom de icke känna honom som har sänt mig.
Joh 15:22  Hade jag icke kommit och talat till dem, så skulle de icke hava haft synd; men nu hava de ingen ursäkt för sin synd.
Joh 15:23  Den som hatar mig, han hatar ock min Fader.
Joh 15:24  Hade jag icke bland dem gjort sådana gärningar, som ingen annan har gjort, så skulle de icke hava haft synd; men nu hava de sett dem, och hava likväl hatat både mig och min Fader.
Joh 15:25  Men det ordet skulle ju fullbordas, som är skrivet i deras lag: 'De hava hatat mig utan sak.'
Joh 15:26  Dock, när Hjälparen kommer, som jag skall sända eder ifrån Fadern, sanningens Ande, som utgår ifrån Fadern, då skall han vittna om mig.
Joh 15:27  Också I kunnen vittna, eftersom I haven varit med mig från begynnelsen."
Joh 16:1  "Detta har jag talat till eder, för att I icke skolen komma på fall.
Joh 16:2  Man skall utstöta eder ur synagogorna; ja, den tid kommer, då vemhelst som dräper eder skall mena sig därmed förrätta offertjänst åt Gud.
Joh 16:3  Och så skola de göra, därför att de icke hava lärt känna Fadern, ej heller mig.
Joh 16:4  Men detta har jag talat till eder, för att I, när den tiden är inne, skolen komma ihåg att jag har sagt eder det. Jag sade eder det icke från begynnelsen, ty jag var ju hos eder.
Joh 16:5  Och nu går jag bort till honom som har sänt mig; och ingen av eder frågar mig vart jag går.
Joh 16:6  Men edra hjärtan äro uppfyllda av bedrövelse, därför att jag har sagt eder detta.
Joh 16:7  Dock säger jag eder sanningen: Det är nyttigt för eder att jag går bort, ty om jag icke ginge bort, så komme icke Hjälparen till eder; men då jag nu går bort, skall jag sända honom till eder.
Joh 16:8  Och när han kommer, skall han låta världen få veta sanningen i fråga om synd och rättfärdighet och dom:
Joh 16:9  i fråga om synd, ty de tro icke på mig;
Joh 16:10  i fråga om rättfärdighet, ty jag går till Fadern, och I sen mig icke mer;
Joh 16:11  i fråga om dom, ty denna världens furste är nu dömd.
Joh 16:12  Jag hade ännu mycket att säga eder, men I kunnen icke nu bära det.
Joh 16:13  Men när han kommer, som är sanningens Ande, då skall han leda eder fram till hela sanningen. Ty han skall icke tala av sig själv, utan vad han hör, allt det skall han tala; och han skall förkunna för eder vad komma skall.
Joh 16:14  Han skall förhärliga mig, ty av mitt skall han taga och skall förkunna det för eder.
Joh 16:15  Allt vad Fadern har, det är mitt; därför sade jag att han skall taga av mitt och förkunna det för eder.
Joh 16:16  En liten tid, och I sen mig icke mer; och åter en liten tid, och I fån se mig, ty jag går till Fadern."
Joh 16:17  Då sade några av hans lärjungar till varandra: "Vad är detta som han säger till oss: 'En liten tid, och I sen mig icke; och åter en liten tid, och I fån se mig', så ock: 'Jag går till Fadern'?"
Joh 16:18  De sade alltså: "Vad är detta som han säger: 'En liten tid'? Vi förstå icke vad han talar."
Joh 16:19  Då märkte Jesus att de ville fråga honom, och han sade till dem: "I talen med varandra om detta som jag sade: 'En liten tid, och I sen mig icke; och åter en liten tid, och I fån se mig.'
Joh 16:20  Sannerligen, sannerligen säger jag eder: I skolen bliva bedrövade, men eder bedrövelse skall vändas i glädje.
Joh 16:21  När en kvinna föder barn, har hon bedrövelse, ty hennes stund är kommen; men när hon har fött barnet, kommer hon icke mer ihåg sin vedermöda, ty hon gläder sig över att en människa är född till världen.
Joh 16:22  Så haven ock I nu bedrövelse; men jag skall se eder åter, och då skola edra hjärtan glädja sig, och ingen skall taga eder glädje ifrån eder.
Joh 16:23  Och på den dagen skolen I icke fråga mig om något. Sannerligen, sannerligen säger jag eder: Vad I bedjen Fadern om, det skall han giva eder i mitt namn.
Joh 16:24  Hittills haven I icke bett om något i mitt namn; bedjen, och I skolen få, för att eder glädje skall bliva fullkomlig.
Joh 16:25  Detta har jag talat till eder i förtäckta ord; den tid kommer, då jag icke mer skall tala till eder i förtäckta ord, utan öppet förkunna för eder om Fadern.
Joh 16:26  På den dagen skolen I bedja i mitt namn. Och jag säger eder icke att jag skall bedja Fadern för eder,
Joh 16:27  ty Fadern själv älskar eder, eftersom I haven älskat mig och haven trott att jag är utgången från Gud.
Joh 16:28  Ja, jag har gått ut ifrån Fadern och har kommit i världen; åter lämnar jag världen och går till Fadern."
Joh 16:29  Då sade hans lärjungar: "Se, nu talar du öppet och brukar inga förtäckta ord.
Joh 16:30  Nu veta vi att du vet allt, och att det icke är behövligt för dig att man frågar dig; därför tro vi att du är utgången från Gud."
Joh 16:31  Jesus svarade dem: "Nu tron I?
Joh 16:32  Se, den stund kommer, ja, den är redan kommen, så I skolen förskingras, var och en åt sitt håll, och lämna mig allena. Dock, jag är icke allena, ty Fadern är med mig.
Joh 16:33  Detta har jag talat till eder, för att I skolen hava frid i mig. I världen liden i betryck; men varen vid gott mod, jag har övervunnit världen."
Joh 17:1  Sedan Jesus hade talat detta, lyfte han upp sina ögon mot himmelen och sade: "Fader, stunden är kommen; förhärliga din Son, på det att din Son må förhärliga dig,
Joh 17:2  eftersom du har givit honom makt över allt kött, för att han skall giva evigt liv åt alla dem som du har givit åt honom.
Joh 17:3  Och detta är evigt liv, att de känna dig, den enda sanne Guden, och den du har sänt, Jesus Kristus.
Joh 17:4  Jag har förhärligat dig på jorden, genom att fullborda det verk som du har givit mig att utföra.
Joh 17:5  Och nu, Fader, förhärliga du mig hos dig själv, med den härlighet som jag hade hos dig, förrän världen var till.
Joh 17:6  Jag har uppenbarat ditt namn för de människor som du har tagit ut ur världen och givit åt mig. De voro dina, och du har givit dem åt mig, och de hava hållit ditt ord.
Joh 17:7  Nu hava de förstått att allt vad du har givit åt mig, det kommer från dig.
Joh 17:8  Ty de ord som du har givit åt mig har jag givit åt dem: och de hava tagit emot dem och hava i sanning förstått att jag är utgången från dig, och de tro att du har sänt mig.
Joh 17:9  Jag beder för dem; det är icke för världen jag beder, utan för dem som du har givit åt mig, ty de äro dina
Joh 17:10  - såsom allt mitt är ditt, och ditt är mitt - och jag är förhärligad i dem.
Joh 17:11  Jag är nu icke längre kvar i världen, men de äro kvar i världen, när jag går till dig. Helige Fader, bevara dem i ditt namn - det som du har förtrott åt mig - för att de må vara ett, likasom vi äro ett.
Joh 17:12  Medan jag var hos dem, bevarade jag dem i ditt namn, det som du har förtrott åt mig; jag vakade över dem, och ingen av dem gick i fördärvet, ingen utom fördärvets man, ty skriften skulle ju fullbordas.
Joh 17:13  Nu går jag till tid; dock talar jag detta, medan jag ännu är här i världen, för att de skola hava min glädje fullkomlig i sig.
Joh 17:14  Jag har givit dem ditt ord; och världen har hatat dem, eftersom de icke äro av världen, likasom icke heller jag är av världen.
Joh 17:15  Jag beder icke att du skall taga dem bort ur världen, utan att du skall bevara dem från det onda.
Joh 17:16  De äro icke av världen, likasom icke heller jag är av världen.
Joh 17:17  Helga dem i sanningen; ditt ord är sanning.
Joh 17:18  Såsom du har sänt mig i världen, så har ock jag sänt dem i världen.
Joh 17:19  Och jag helgar mig till ett offer för dem, på det att ock de må vara i sanning helgade.
Joh 17:20  Men icke för dessa allenast beder jag, utan ock för dem som genom deras ord komma till tro på mig;
Joh 17:21  jag beder att de alla må vara ett, och att, såsom du, Fader, är i mig, och jag i dig, också de må vara i oss, för att världen skall tro att du har sänt mig.
Joh 17:22  Och den härlighet som du har givit mig, den har jag givit åt dem, för att de skola vara ett, såsom vi äro ett
Joh 17:23  - jag i dem, och du i mig - ja, för att de skola vara fullkomligt förenade till ett, så att världen kan förstå att du har sänt mig, och att du har älskat dem, såsom du har älskat mig.
Joh 17:24  Fader, jag vill att där jag är, där skola ock de som du har givit mig vara med mig, så att de få se min härlighet, som du har givit mig; ty du har älskat mig före världens begynnelse.
Joh 17:25  Rättfärdige Fader, världen har icke lärt känna dig, men jag känner dig, och dessa hava förstått att du har sänt mig.
Joh 17:26  Och jag har kungjort för dem ditt namn och skall kungöra det, på det att den kärlek, som du har älskat mig med, må vara i dem, och jag i dem."
Joh 18:1  När Jesus hade sagt detta, begav han sig med sina lärjungar därifrån och gick över bäcken Kidron till andra sidan. Där var en örtagård, och i den gick han in med sina lärjungar.
Joh 18:2  Men också Judas, han som förrådde honom, kände till det stället, ty där hade Jesus och hans lärjungar ofta kommit tillsammans.
Joh 18:3  Och Judas tog nu med sig den romerska vakten, så ock några av översteprästernas och fariséernas tjänare, och kom dit med bloss och lyktor och vapen.
Joh 18:4  Och Jesus, som visste allt vad som skulle övergå honom, gick fram och sade till dem: "Vem söken I?"
Joh 18:5  De svarade honom: "Jesus från Nasaret." Jesus sade till dem: "Det är jag." Och Judas, förrädaren, stod också där ibland dem.
Joh 18:6  När Jesus nu sade till dem: "Det är jag", veko de tillbaka och föllo till marken.
Joh 18:7  Åter frågade han dem då: "Vem söken I?" De svarade: "Jesus från Nasaret."
Joh 18:8  Jesus sade: "Jag har sagt eder att det är jag; om det alltså är mig I söken, så låten dessa gå."
Joh 18:9  Ty det ordet skulle fullbordas, som han hade sagt: "Av dem som du har givit mig har jag icke förlorat någon."
Joh 18:10  Och Simon Petrus, som hade ett svärd, drog ut det och högg till översteprästens tjänare och högg så av honom högra örat; och tjänarens namn var Malkus.
Joh 18:11  Då sade Jesus till Petrus: "Stick ditt svärd i skidan. Skulle jag icke dricka den kalk som min Fader har givit mig?"
Joh 18:12  Den romerska vakten med sin överste och de judiska rättstjänarna grepo då Jesus och bundo honom
Joh 18:13  och förde honom bort, först till Hannas; denne var nämligen svärfader till Kaifas, som var överstepräst det året.
Joh 18:14  Och det var Kaifas som under rådplägningen hade sagt till judarna, att det vore bäst om en man finge dö för folket.
Joh 18:15  Och Simon Petrus jämte en annan lärjunge följde efter Jesus. Den lärjungen var bekant med översteprästen och gick med Jesus in på översteprästens gård;
Joh 18:16  men Petrus stod utanför vid porten. Den andre lärjungen, den som var bekant med översteprästen, gick då ut och talade med portvakterskan och fick så föra Petrus ditin.
Joh 18:17  Tjänstekvinnan som vaktade porten sade därvid till Petrus: "Är icke också du en av den mannens lärjungar?" Han svarade: "Nej, det är jag icke."
Joh 18:18  Men tjänarna och rättsbetjänterna hade gjort upp en koleld, ty det var kallt, och de stodo där och värmde sig; bland dem stod också Petrus och värmde sig.
Joh 18:19  Översteprästen frågade nu Jesus om hans lärjungar och om hans lära.
Joh 18:20  Jesus svarade honom: "Jag har öppet talat för världen, jag har alltid undervisat i synagogan eller i helgedomen, på ställen där alla judar komma tillsammans; hemligen har jag intet talat.
Joh 18:21  Varför frågar du då mig? Dem som hava hört mig må du fråga om vad jag har talat till dem. De veta ju vad jag har sagt."
Joh 18:22  När Jesus sade detta, gav honom en av rättstjänarna, som stod där bredvid, ett slag på kinden och sade: "Skall du så svara översteprästen?"
Joh 18:23  Jesus svarade honom: "Har jag talat orätt, så bevisa att det var orätt; men har jag talat rätt, varför slår du mig då?"
Joh 18:24  Och Hannas sände honom bunden till översteprästen Kaifas.
Joh 18:25  Men Simon Petrus stod och värmde sig. Då sade de till honom: "Är icke också du en av hans lärjungar?" Han nekade och sade: "Det är jag icke."
Joh 18:26  Då sade en av översteprästens tjänare, en frände till den som Petrus hade huggit örat av: "Såg jag icke själv att du var med honom i örtagården?"
Joh 18:27  Då nekade Petrus åter. Och i detsamma gol hanen.
Joh 18:28  Sedan förde de Jesus från Kaifas till pretoriet; och det var nu morgon. Men själva gingo de icke in i pretoriet, för att de icke skulle bliva orenade, utan skulle kunna äta påskalammet.
Joh 18:29  Då gick Pilatus ut till dem och sade: "Vad haven I för anklagelse att frambära mot denne man?"
Joh 18:30  De svarade och sade till honom: "Vore han icke en illgärningsman, så hade vi icke överlämnat honom åt dig."
Joh 18:31  Då sade Pilatus till dem: "Tagen I honom, och dömen honom efter eder lag." Judarna svarade honom: "För oss är det icke lovligt att avliva någon."
Joh 18:32  Ty Jesu ord skulle fullbordas, det som han hade sagt för att giva till känna på vad sätt han skulle dö.
Joh 18:33  Pilatus gick åter in i pretoriet och kallade Jesus till sig och sade till honom: "Är du judarnas konung?"
Joh 18:34  Jesus svarade: "Säger du detta av dig själv, eller hava andra sagt dig det om mig?"
Joh 18:35  Pilatus svarade: "Jag är väl icke en jude! Ditt eget folk och översteprästerna hava överlämnat dig åt mig. Vad har du gjort?"
Joh 18:36  Jesus svarade: "Mitt rike är icke av denna världen. Om mitt rike vore av denna världen, så hade väl mina tjänare kämpat för att jag icke skulle bliva överlämnad åt judarna. Men nu är mitt rike icke av denna världen."
Joh 18:37  Så sade Pilatus till honom: "Så är du dock en konung?" Jesus svarade: "Du säger det själv, att jag är en konung. Ja, därtill är jag född, och därtill har jag kommit i världen, att jag skall vittna för sanningen. Var och en som är av sanningen, han hör min röst."
Joh 18:38  Pilatus sade till honom: "Vad är sanning?" När han hade sagt detta, gick han åter ut till judarna och sade till dem: "Jag finner honom icke skyldig till något brott.
Joh 18:39  Nu är det en sedvänja hos eder, att jag vid påsken skall giva eder en fånge lös. Viljen I då att jag skall giva eder 'judarnas konung' lös?"
Joh 18:40  Då skriade de åter och sade: "Icke honom, utan Barabbas." Men Barabbas var en rövare.
Joh 19:1  Så tog då Pilatus Jesus och lät gissla honom.
Joh 19:2  Och krigsmännen vredo samman en krona av törnen och satte den på hans huvud och klädde på honom en purpurfärgad mantel.
Joh 19:3  Sedan trädde de fram till honom och sade: "Hell dig, du judarnas konung!" och slogo honom på kinden.
Joh 19:4  Åter gick Pilatus ut och sade till folket: "Se, jag vill föra honom ut till eder, på det att I mån förstå att jag icke finner honom skyldig till något brott."
Joh 19:5  Och Jesus kom då ut, klädd i törnekronan och den purpurfärgade manteln. Och han sade till dem: "Se mannen!"
Joh 19:6  Då nu översteprästerna och rättstjänarna fingo se honom, skriade de: "Korsfäst! Korsfäst!" Pilatus sade till dem: "Tagen I honom, och korsfästen honom; jag finner honom icke skyldig till något brott."
Joh 19:7  Judarna svarade honom: "Vi hava själva en lag, och efter den lagen måste han dö, ty han har gjort sig till Guds Son."
Joh 19:8  När Pilatus hörde dem tala så, blev hans fruktan ännu större.
Joh 19:9  Och han gick åter in i pretoriet och frågade Jesus: "Varifrån är du?" Men Jesus gav honom intet svar.
Joh 19:10  Då sade Pilatus till honom: "Svarar du mig icke? Vet du då icke att jag har makt att giva dig lös och makt att korsfästa dig?"
Joh 19:11  Jesus svarade honom: "Du hade alls ingen makt över mig, om den icke vore dig given ovanifrån. Därför har den större synd, som har överlämnat mig åt dig."
Joh 19:12  Från den stunden sökte Pilatus efter någon utväg att giva honom lös. Men judarna ropade och sade: "Giver du honom lös, så är du icke kejsarens vän. Vemhelst som gör sig till konung, han sätter sig upp mot kejsaren."
Joh 19:13  När Pilatus hörde de orden, lät han föra ut Jesus och satte sig på domarsätet, på en plats som kallades Litostroton, på hebreiska Gabbata.
Joh 19:14  Och det var tillredelsedagen före påsken, vid sjätte timmen. Och han sade till judarna: "Se här är eder konung!"
Joh 19:15  Då skriade de: "Bort med honom! Bort med honom! Korsfäst honom!" Pilatus sade till dem: "Skall jag korsfästa eder konung?" Översteprästerna svarade: "Vi hava ingen annan konung än kejsaren."
Joh 19:16  Då gjorde han dem till viljes och bjöd att han skulle korsfästas. Och de togo Jesus med sig.
Joh 19:17  Och han bar själv sitt kors och kom så ut till det ställe som kallades Huvudskalleplatsen, på hebreiska Golgata.
Joh 19:18  Där korsfäste de honom, och med honom två andra, en på vardera sidan, och Jesus i mitten.
Joh 19:19  Men Pilatus lät ock göra en överskrift och sätta upp den på korset; och den lydde så: "Jesus från Nasaret, judarnas konung."
Joh 19:20  Den överskriften läste många av judarna, ty det ställe där Jesus var korsfäst låg nära staden: och den var avfattad på hebreiska, på latin och på grekiska.
Joh 19:21  Då sade judarnas överstepräster till Pilatus: "Skriv icke: 'Judarnas konung', utan skriv att han har sagt sig vara judarnas konung."
Joh 19:22  Pilatus svarade: "Vad jag har skrivit, det har jag skrivit."
Joh 19:23  Då nu krigsmännen hade korsfäst Jesus, togo de hans kläder och delade dem i fyra delar, en del åt var krigsman. Också livklädnaden togo de. Men livklädnaden hade inga sömmar, utan var vävd i ett stycke, uppifrån och alltigenom.
Joh 19:24  Därför sade de till varandra: "Låt oss icke skära sönder den, utan kasta lott om vilken den skall tillhöra." Ty skriftens ord skulle fullbordas: "De delade mina kläder mellan sig och kastade lott om min klädnad." Så gjorde nu krigsmännen.
Joh 19:25  Men vid Jesu kors stodo hans moder och hans moders syster, Maria, Klopas' hustru, och Maria från Magdala.
Joh 19:26  När Jesus nu fick se sin moder och bredvid henne den lärjunge som han älskade, sade han till sin moder: "Moder, se din son."
Joh 19:27  Sedan sade han till lärjungen: "Se din moder." Och från den stunden tog lärjungen henne hem till sig.
Joh 19:28  Eftersom nu Jesus visste att allt annat redan var fullbordat, sade han därefter, då ju skriften skulle i allt uppfyllas: "Jag törstar."
Joh 19:29  Där stod då en kärl som var fullt av ättikvin. Med det vinet fyllde de en svamp, som de satte på en isopsstängel och förde till hans mun.
Joh 19:30  Och när Jesus hade tagit emot vinet, sade han: "Det är fullbordat." Sedan böjde han ned huvudet och gav upp andan.
Joh 19:31  Men eftersom det var tillredelsedag och judarna icke ville att kropparna skulle bliva kvar på korset över sabbaten (det var nämligen en stor sabbatsdag), bådo de Pilatus att han skulle låta sönderslå de korsfästas ben och taga bort kropparna.
Joh 19:32  Så kommo då krigsmännen och slogo sönder den förstes ben och sedan den andres som var korsfäst med honom.
Joh 19:33  När de därefter kommo till Jesus och sågo honom redan vara död, slogo de icke sönder hans ben;
Joh 19:34  men en av krigsmännen stack upp han sida med ett spjut, och strax kom därifrån ut blod och vatten.
Joh 19:35  Och den som har sett detta, han har vittnat därom, för att ock I skolen tro; och hans vittnesbörd är sant, och han vet att han talar sanning.
Joh 19:36  Ty detta skedde, för att skriftens ord skulle fullbordas: "Intet ben skall sönderslås på honom."
Joh 19:37  Och åter ett annat skriftens ord lyder så: "De skola se upp till honom som de hava stungit."
Joh 19:38  Men Josef från Arimatea, som var en Jesu lärjunge - fastän i hemlighet, av fruktan för judarna - kom därefter och bad Pilatus att få taga Jesu kropp; och Pilatus tillstadde honom det. Då gick han åstad och tog hans kropp.
Joh 19:39  Och jämväl Nikodemus kom dit, han som första gången hade besökt honom om natten; denne förde med sig en blandning av myrra och aloe, vid pass hundra skålpund.
Joh 19:40  Och de togo Jesu kropp och omlindade den med linnebindlar och lade dit de välluktande kryddorna, såsom judarna hava för sed vid tillredelse till begravning.
Joh 19:41  Men invid det ställe där han hade blivit korsfäst var en örtagård, och i örtagården fanns en ny grav, som ännu ingen hade varit lagd i.
Joh 19:42  Där lade de nu Jesus, eftersom det var judarnas tillredelsedag och graven låg nära.
Joh 20:1  Men på första veckodagen, medan det ännu var mörkt, kom Maria från Magdala dit till graven och fick se stenen vara borttagen från graven.
Joh 20:2  Då skyndade hon därifrån och kom till Simon Petrus och till den andre lärjungen, den som Jesus älskade, och sade till dem: "De hava tagit Herren bort ur graven, och vi veta icke var de hava lagt honom."
Joh 20:3  Då begåvo sig Petrus och den andre lärjungen åstad på väg till graven.
Joh 20:4  Och de sprungo båda på samma gång; men den andre lärjungen sprang fortare än Petrus och kom först fram till graven.
Joh 20:5  Och när han lutade sig ditin, så han linnebindlarna ligga där; dock gick han icke in.
Joh 20:6  Sedan, efter honom, kom ock Simon Petrus dit. Han gick in i graven och fick så se huru bindlarna lågo där,
Joh 20:7  och huru duken som hade varit höljd över hans huvud icke låg tillsammans med bindlarna, utan för sig själv på ett särskilt ställe, hopvecklad.
Joh 20:8  Då gick ock den andre lärjungen ditin, han som först hade kommit till graven; och han såg och trodde.
Joh 20:9  De hade nämligen ännu icke förstått skriftens ord, att han skulle uppstå från de döda.
Joh 20:10  Och lärjungarna gingo så hem till sitt igen.
Joh 20:11  Men Maria stod och grät utanför graven. Och under det hon grät, lutade hon sig in i graven
Joh 20:12  och fick då se två änglar i vita kläder sitta där Jesu kropp hade legat, den ene vid huvudets plats, den andre vid fötternas.
Joh 20:13  Och de sade till henne: "Kvinna, varför gråter du?" Hon svarade dem: "De hava tagit bort min Herre, och jag vet icke var de hava lagt honom."
Joh 20:14  Vid det hon sade detta, vände hon sig om och fick se Jesus stå där; men hon visste icke att det var Jesus.
Joh 20:15  Jesus sade till henne: "Kvinna, varför gråter du? Vem söker du?" Hon trodde att det var örtagårdsmästaren och svarade honom: "Herre, om det är du som har burit bort honom, så säg mig var du har lagt honom, så att jag kan hämta honom."
Joh 20:16  Jesus sade till henne: "Maria!" Då vände hon sig om och sade till honom på hebreiska: "Rabbuni!" (det betyder mästare).
Joh 20:17  Jesus sade till henne: "Rör icke vid mig; jag har ju ännu icke farit upp till Fadern. Men gå till mina bröder, och säg till dem att jag far upp till min Fader och eder Fader, till min Gud och eder Gud."
Joh 20:18  Maria från Magdala gick då och omtalade för lärjungarna att hon hade sett Herren, och att han hade sagt detta till henne.
Joh 20:19  På aftonen samma dag, den första veckodagen, medan lärjungarna av fruktan för judarna voro samlade inom stängda dörrar, kom Jesus och stod mitt ibland dem och sade till dem: "Frid vare med eder!"
Joh 20:20  Och när han hade sagt detta, visade han dem sina händer och sin sida. Och lärjungarna blevo glada, när de sågo Herren.
Joh 20:21  Åter sade Jesus till dem: "Frid vare med eder! Såsom Fadern har sänt mig, så sänder ock jag eder."
Joh 20:22  Och när han hade sagt detta, andades han på dem och sade till dem: "Tagen emot helig ande!
Joh 20:23  Om I förlåten någon hans synder, så äro de honom förlåtna; och om I binden någon i hans synder, så är han bunden i dem."
Joh 20:24  Men Tomas, en av de tolv, han som kallades Didymus, var icke med dem, när Jesus kom.
Joh 20:25  Då nu de andra lärjungarna sade till honom att de hade sett Herren, svarade han dem: "Om jag icke ser hålen efter spikarna i hans händer och sticker mitt finger i hålen efter spikarna och sticker min hand i hans sida, så kan jag icke tro det."
Joh 20:26  Åtta dagar därefter voro hans lärjungar åter därinne, och Tomas var med bland dem. Då kom Jesus, medan dörrarna voro stängda, och stod mitt ibland de, och sade: "Frid vare med eder!"
Joh 20:27  Sedan sade han till Tomas: "Räck hit dit finger, se här äro mina händer; och räck hit din hand, och stick den i min sida. Och tvivla icke, utan tro."
Joh 20:28  Tomas svarade och sade till honom: "Min Herre och min Gud!"
Joh 20:29  Jesus sade till honom: "Eftersom du har sett mig, tror du? Saliga äro de som icke se och dock tro."
Joh 20:30  Ännu många andra tecken, som icke äro uppskrivna i denna bok, gjorde Jesus i sina lärjungars åsyn.
Joh 20:31  Men dessa hava blivit uppskrivna, för att I skolen tro att Jesus är Messias, Guds Son, och för att I genom tron skolen hava liv i hans namn.
Joh 21:1  Därefter uppenbarade sig Jesus åter för lärjungarna, vid Tiberias' sjö; och vid den uppenbarelsen gick så till:
Joh 21:2  Simon Petrus och Tomas, som kallades Didymus, och Natanael, han som var från Kana i Galileen, och Sebedeus' söner voro tillsammans, och med dem två andra av hans lärjungar.
Joh 21:3  Simon Petrus sade då till dem: "Jag vill gå åstad och fiska." De sade till honom: "Vi gå också med dig." Så begåvo de sig åstad och stego i båten. Men den natten fingo de intet.
Joh 21:4  När det sedan hade blivit morgon, stod Jesus där på stranden; dock visste lärjungarna icke att det var Jesus.
Joh 21:5  Och Jesus sade till dem: "Mina barn, haven I något att äta?" De svarade honom: "Nej."
Joh 21:6  Han sade till dem: "Kasten ut nätet på högra sidan om båten, så skolen I få." Då kastade de ut; och nu fingo de en så stor hop fiskar, att de icke förmådde draga upp nätet.
Joh 21:7  Den lärjunge som Jesus älskade sade då till Petrus: "Det är Herren." När Simon Petrus hörde att det var Herren, tog han på sig sin överklädnad - ty han var oklädd - och gav sig i sjön.
Joh 21:8  Men de andra lärjungarna kommo med båten och drogo efter sig nätet med fiskarna; de voro nämligen icke längre från land än vid pass två hundra alnar.
Joh 21:9  När de sedan hade stigit i land, sågo de glöd ligga där och fisk, som låg därpå, och bröd.
Joh 21:10  Jesus sade till dem: "Tagen hit av de fiskar som I nu fingen."
Joh 21:11  Då steg Simon Petrus i båten och drog nätet upp på land, och det var fullt av stora fiskar, ett hundra femtiotre stycken. Och fastän de voro så många, hade nätet icke gått sönder.
Joh 21:12  Därefter sade Jesus till dem: "Kommen hit och äten." Och ingen av lärjungarna dristade sig att fråga honom vem han var, ty de förstodo att det var Herren.
Joh 21:13  Jesus gick då fram och tog brödet och gav dem, likaledes ock av fiskarna.
Joh 21:14  Detta var nu tredje gången som Jesus uppenbarade sig för sina lärjungar, sedan han hade uppstått från de döda.
Joh 21:15  När de hade ätit, sade Jesus till Simon Petrus: "Simon, Johannes' son, älskar du mig mer än dessa göra?" Han svarade honom: "Ja, Herre; du vet att jag har dig kär." Då sade han till honom: "Föd mina lamm."
Joh 21:16  Åter frågade han honom, för andra gången: "Simon, Johannes' son, älskar du mig?" Han svarade honom: "Ja, Herre; du vet att jag har dig kär." Då sade han till honom: "Var en herde för mina får."
Joh 21:17  För tredje gången frågade han honom: "Simon, Johannes' son, har du mig kär?" Petrus blev bedrövad över att han för tredje gången frågade honom: "Har du mig kär?" Och han svarade honom: "Herre, du vet allting; du vet att jag har dig kär." Då sade Jesus till honom: "Föd mina får.
Joh 21:18  Sannerligen, sannerligen säger jag dig: När du var yngre, omgjordade du dig själv och gick vart du ville; men när du bliver gammal, skall du nödgas sträcka ut dina händer, och en annan skall omgjorda dig och föra dig dit du icke vill."
Joh 21:19  Detta sade han för att giva till känna med hurudan död Petrus skulle förhärliga Gud. Och sedan han hade sagt detta, sade han till honom: "Följ mig."
Joh 21:20  När Petrus vände sig om, fick han se att den lärjunge som Jesus älskade följde med, densamme som under aftonmåltiden hade lutat sig mot hans bröst och frågat honom: "Herre, vilken är det som skall förråda dig?"
Joh 21:21  Då nu Petrus såg den lärjungen, frågade han Jesus: "Herre, huru bliver det då med denne?"
Joh 21:22  Jesus svarade honom: "Om jag vill att han skall leva kvar, till dess jag kommer, vad kommer det dig vid? Följ du mig."
Joh 21:23  Så kom det talet ut ibland bröderna, att den lärjungen icke skulle dö. Men Jesus hade icke sagt till honom att han icke skulle dö, utan allenast: "Om jag vill att han skall leva kvar, till dess jag kommer, vad kommer det dig vid?"
Joh 21:24  Det är den lärjungen som vittnar om detta, och som har skrivit detta; och vi veta att hans vittnesbörd är sant.
Joh 21:25  Ännu mycket annat var det som Jesus gjorde; och om allt detta skulle uppskrivas, det ena med det andra, så tror jag att icke ens hela världen skulle kunna rymma de böcker som då bleve skrivna.


\end{document}