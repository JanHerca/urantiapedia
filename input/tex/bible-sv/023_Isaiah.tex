\begin{document}

\title{Jesaja}


\chapter{1}

\par 1 Detta är Jesajas, Amos' sons, syner, vad han skådade angående Juda och Jerusalem i Ussias, Jotams, Ahas' och Hiskias, Juda konungars, tid.
\par 2 Hören, I himlar, och lyssna, du jord; ty HERREN talar. Barn har jag uppfött och fostrat, men de hava avfallit från mig.
\par 3 En oxe känner sin ägare och en åsna sin herres krubba, men Israel känner intet, mitt folk förstår intet.
\par 4 Ve dig, du syndiga släkte, du skuldbelastade folk, du ogärningsmäns avföda, I vanartiga barn, som haven övergivit HERREN, föraktat Israels Helige och vikit bort ifrån honom!
\par 5 Var skall man mer slå eder, då I så fortgån i avfällighet? Hela huvudet är ju krankt, och hela hjärtat är sjukt.
\par 6 Ifrån fotbladet ända upp till huvudet finnes intet helt, blott sårmärken och blånader och friska sår, icke utkramade eller förbundna eller lenade med olja.
\par 7 Edert land är en ödemark, edra städer äro uppbrända i eld, edra åkrar bliva i eder åsyn förtärda av främlingar; en ödeläggelse är det, såsom där främlingar hava omstörtat allt.
\par 8 Allenast dottern Sion står kvar där, såsom en hydda i en vingård, såsom ett vaktskjul på ett gurkfält, såsom en inspärrad stad.
\par 9 Om HERREN Sebaot icke hade lämnat en liten återstod kvar åt oss, då vore vi såsom Sodom, vi vore Gomorra lika.
\par 10 Hören HERRENS ord, I Sodomsfurstar, lyssna till vår Guds lag, du Gomorra-folk.
\par 11 Vad skall jag med edra många slaktoffer? säger HERREN. Jag är mätt på brännoffer av vädurar och på gödkalvars fett, och till blod av tjurar och lamm och bockar har jag intet behag.
\par 12 När I kommen för att träda fram inför mitt ansikte, vem begär då av eder det, att mina förgårdar trampas ned?
\par 13 Bären ej vidare fram fåfängliga spisoffer; ångan av dem är en styggelse för mig. Nymånader och sabbater och utlysta fester, ondska i förening med högtidsförsamlingar, sådant kan jag icke lida.
\par 14 Edra nymånader och högtider hatar min själ; de hava blivit mig en börda, jag orkar ej bära den.
\par 15 Ja, huru I än uträcken edra händer, så gömmer jag mina ögon för eder, och om I än mycket bedjen, så hör jag icke därpå. Edra händer äro fulla av blod;
\par 16 tvån eder då, och renen eder. Skaffen edert onda väsende bort ifrån mina ögon. Hören upp att göra, vad ont är.
\par 17 Lären att göra vad gott är, faren efter det rätt är, visen förtryckaren på bättre vägar, skaffen den faderlöse rätt, utfören änkans sak.
\par 18 Kom, låt oss gå till rätta med varandra, säger HERREN. Om edra synder än äro blodröda, så kunna de bliva snövita, och om de äro röda såsom scharlakan, så kunna de bliva såsom vit ull.
\par 19 Om I ären villiga att höra, skolen I få äta av landets goda.
\par 20 Men ären I ovilliga och gensträviga, skolen I förtäras av svärd; ty så har HERRENS mun talat.
\par 21 Huru har hon icke blivit en sköka, den trogna staden! Den var full av rätt, rättfärdighet bodde därinne, men nu bo där mördare.
\par 22 Ditt silver har blivit slagg, ditt ädla vin är utspätt med vatten.
\par 23 Dina styresmän äro upprorsmän och tjuvars stallbröder. Alla älskar de mutor och fara efter vinning. Den faderlöse skaffa de icke rätt, och änkans sak kommer icke inför dem.
\par 24 Därför säger Herren, HERREN Sebaot, den Starke i Israel: Ve! Jag vill släcka min harm på mina ovänner och hämnas på mina fiender.
\par 25 Jag vill vända min hand emot dig och bortrensa ditt slagg såsom med lutsalt och skaffa bort all din oädla malm.
\par 26 Jag vill åter giva dig sådana domare som tillförne, och sådana rådsherrar som du förut ägde. Därefter skall du kallas "rättfärdighetens stad", "en trogen stad".
\par 27 Sion skall genom rätt bliva förlossad och dess omvända genom rättfärdighet.
\par 28 Men fördärv skall drabba alla överträdare och syndare, och de som övergiva HERREN, de skola förgås.
\par 29 Ja, de skola komma på skam med de terebinter som voro eder fröjd; och I skolen få blygas över de lustgårdar som I haden så kära.
\par 30 Ty I skolen bliva såsom en terebint med vissnade löv och varda lika en lustgård utan något vatten.
\par 31 Och de väldige skola varda såsom blår, och deras verk såsom en gnista, och de skola tillsammans brinna, och ingen skall kunna släcka.

\chapter{2}

\par 1 Detta är vad Jesaja, Amos' son, skådade angående Juda och Jerusalem.
\par 2 Och det skall ske i kommande dagar att det berg där HERRENS hus är skall stå där fast grundat och vara det yppersta ibland bergen och upphöjt över andra höjder; och alla hednafolk skola strömma dit,
\par 3 ja, många folk skola gå åstad och skola säga: "Upp, låt oss draga åstad till HERRENS berg, upp till Jakobs Guds hus, för att han må undervisa oss om sina vägar, så att vi kunna vandra på hans stigar." Ty från Sion skall lag utgå, och HERRENS ord från Jerusalem.
\par 4 Och han skall döma mellan hednafolken och skipa rätt åt många folk. Då skola de smida sina svärd till plogbillar och sina spjut till vingårdsknivar. Folken skola ej mer lyfta svärd mot varandra och icke mer lära sig att strida.
\par 5 I av Jakobs hus, kommen, låtom oss vandra i HERRENS ljus.
\par 6 Ty du har förskjutit ditt folk, Jakobs hus, därför att de äro fulla av Österlandets väsende och öva teckentyderi såsom filistéerna; ja, med främlingar förbinda de sig.
\par 7 Deras land är fullt av silver och guld, och på deras skatter är ingen ände; deras land är fullt av hästar, och på deras vagnar är ingen ände;
\par 8 deras land är ock fullt av avgudar, och sina egna händers verk tillbedja de, det som deras fingrar hava gjort.
\par 9 Därför bliva människorna nedböjda och männen ödmjukade; du kan icke förlåta dem.
\par 10 Fly in i klippan, och göm dig i jorden, för HERRENS fruktansvärda makt och för hans höga majestät.
\par 11 Ty människornas högmodiga ögon skola bliva ödmjukade, och männens övermod skall bliva nedböjt, och HERREN allena skall vara hög på den dagen.
\par 12 Ty en dag har HERREN Sebaot bestämt, som skall komma över allt stolt och övermodigt och över allt som är upphöjt, och det skall bliva ödmjukat,
\par 13 ja, över alla Libanons cedrar, de höga och stolta, och över alla Basans ekar;
\par 14 över alla höga berg och alla stolta höjder,
\par 15 över alla höga torn och alla fasta murar,
\par 16 över alla Tarsis-skepp, ja, över allt som är skönt att skåda.
\par 17 Och människornas högmod skall bliva nedböjt och männens övermod nedbrutet, och HERREN allena skall vara hög på den dagen.
\par 18 Men avgudarna skola alldeles förgås.
\par 19 Och man skall fly in i klippgrottor och in i jordhålor, för HERRENS fruktansvärda makt och för hans höga majestät, när han står upp för att förskräcka jorden.
\par 20 På den dagen skola människorna kasta bort åt mullvadar och flädermöss de avgudar av silver och de avgudar av guld, som de hava gjort åt sig för att tillbedja.
\par 21 Ja, de skola fly in i klipprämnor och in i bergsklyftor, för HERRENS fruktansvärda makt och för hans höga majestät, när han står upp för att förskräcka jorden.
\par 22 Så förliten eder nu ej mer på människor, i vilkas näsa är allenast en fläkt; ty huru ringa äro icke de att akta!

\chapter{3}

\par 1 Ty se, Herren, HERREN Sebaot skall taga bort ifrån Jerusalem och Juda allt slags stöd och uppehälle - all mat till uppehälle och all dryck till uppehälle -
\par 2 hjältar och krigsmän, domare och profeter, spåmän och äldste,
\par 3 underhövitsmän och högtuppsatta män, rådsherrar och slöjdkunnigt folk och män som äro förfarna i besvärjelsekonst.
\par 4 Och jag skall giva dem ynglingar till furstar, och barnsligt självsvåld skall få råda över dem.
\par 5 Av folket skall den ene förtrycka den andre, var och en sin nästa; den unge skall sätta sig upp mot den gamle, den ringe mot den högt ansedde.
\par 6 När då så sker, att någon fattar tag i en annan i hans faders hus och säger: "Du äger en mantel, du skall bliva vår styresman; tag du hand om detta vacklande rike" -
\par 7 då skall denne svara och säga: "Jag kan icke skaffa bot; i mitt hus finnes varken bröd eller mantel. Mig skolen I icke sätta till styresman över folket."
\par 8 Ty Jerusalem vacklar, och Juda faller, då de nu med sitt tal och sina gärningar stå emot HERREN och äro gensträviga mot hans härlighets blickar.
\par 9 Deras uppsyn vittnar emot dem; och likasom Sodoms folk bedriva de sina synder uppenbart och dölja dem icke. Ve över deras själar, ty själva hava de berett sig olycka!
\par 10 Om den rättfärdige mån I tänka att det skall gå honom väl, ty sådana skola äta sina gärningars frukt.
\par 11 Men ve över den ogudaktige! Honom skall det gå illa, ty efter hans gärningar skall hans vedergällning bliva.
\par 12 Mitt folks behärskare är ett barn, och kvinnor råda över det. Mitt folk, dina ledare föra dig vilse och fördärva den väg, som du skulle gå.
\par 13 Men HERREN står redo att gå till rätta, han träder fram för att döma folken;
\par 14 HERREN vill gå till doms med sitt folks äldste och med dess furstar. "I haven skövlat vingården; rov från de fattiga är i edra hus.
\par 15 Huru kunnen I så krossa mitt folk och söndermala de fattiga?" Så säger Herren, HERREN Sebaot.
\par 16 Och HERREN säger: Eftersom Sions döttrar äro så högmodiga, och gå med rak hals och spela med ögonen, och gå där och trippa och pingla med sina fotringar,
\par 17 därför skall Herren låta Sions döttrars hjässor bliva fulla av skorv, och HERREN skall blotta deras blygd.
\par 18 På den dagen skall Herren taga bort all deras ståt: fotringar, pannband och halsprydnader,
\par 19 örhängen, armband och slöjor,
\par 20 huvudprydnader, fotstegskedjor, gördlar, luktflaskor och amuletter,
\par 21 fingerringar och näsringar,
\par 22 högtidsdräkter, kåpor, mantlar och pungar,
\par 23 speglar, fina linneskjortor, huvudbindlar och flor.
\par 24 Och där skall vara stank i stället för vällukt, rep i stället för bälte, skalligt huvud i stället för krusat hår, hölje av säcktyg i stället för högtidsmantel, märken av brännjärn i stället för skönhet.
\par 25 Dina män skola falla för svärd och dina hjältar i krig:
\par 26 hennes portar skola klaga och sörja, och övergiven skall hon sitta på marken.

\chapter{4}

\par 1 Och på den tiden skola sju kvinnor fatta i en och samma man och säga: "Vi vilja själva föda oss och själva kläda oss; låt oss allenast få bära ditt namn, och tag så bort vår smälek."
\par 2 På den tiden skall det som HERREN låter växa bliva till prydnad och härlighet, och vad landet alstrar bliva till berömmelse och ära, för den räddade skaran i Israel.
\par 3 Och det skall ske att den som lämnas övrig i Sion och den som bliver kvar i Jerusalem, han skall då kallas helig, var och en som är upptecknad till liv i Jerusalem -
\par 4 när en gång Herren har avtvått Sions döttrars orenlighet och bortsköljt ur Jerusalem dess blodskulder genom rättens och reningens ande.
\par 5 Och HERREN skall över hela Sions bergs område och över dess högtidsskaror skapa en molnsky och en rök om dagen, och skenet av en lågande eld om natten; ty ett beskärmande täckelse skall vila över all dess härlighet.
\par 6 Och ett skygd skall vara däröver till skugga under dagens hetta, och till en tillflykt och ett värn mot störtskurar och regn.

\chapter{5}

\par 1 Jag vill sjunga om min vän, min väns sång om hans vingård. Min vän hade en vingård på en bördig bergskulle.
\par 2 Och han hackade upp den och rensade den från stenar och planterade där ädla vinträd; han byggde ett vakttorn därinne, han högg ock ut ett presskar däri. Så väntade han att den skulle bära äkta druvor, men den bar vilddruvor.
\par 3 Och nu, I Jerusalems invånare och I Juda män, fällen nu eder dom mellan mig och min vingård.
\par 4 Vad kunde mer göras för min vingård, än vad jag har gjort för den? Varför bar den då vilddruvor, när jag väntade att den skulle bära äkta druvor?
\par 5 Så vill jag nu kungöra för eder vad jag skall göra med min vingård: Jag skall taga bort dess hägnad, och den skall givas till skövling; jag skall bryta ned dess mur, och den skall bliva nedtrampad.
\par 6 Jag skall i grund fördärva den, ingen skall skära den eller gräva däri. Den skall fyllas med tistel och törne; och molnen skall jag förbjuda att sända ned regn på den.
\par 7 Ty HERREN Sebaots vingård, det är Israels hus; och Juda folk är hans älsklingsplantering. Men när han väntade laglydnad, då fann han lagbrott, och när han väntade rättfärdighet, fann han skriande orättfärdighet. -
\par 8 Ve eder som läggen hus till hus och fogen åker till åker, intill dess att rum ej mer finnes och I ären de enda som bo i landet!
\par 9 Från HERREN Sebaot ljuder det så i mina öron: Sannerligen, de många husen skola bliva öde; huru stora och sköna de än äro, skola de bliva tomma på invånare.
\par 10 Ty en vingård på tio plogland skall giva allenast ett batmått, och en homers utsäde skall giva blott en efa.
\par 11 Ve dem som stå bittida upp för att hasta till starka drycker, och som sitta intill sena natten för att upphetta sig med vin!
\par 12 Harpor och psaltare, pukor och flöjter och vin hava de vid sina dryckeslag, men på HERRENS gärningar akta de icke, på hans händers verk se de icke.
\par 13 Därför skall mitt folk oförtänkt föras bort i fångenskap; dess ädlingar skola lida hunger och dess larmande skaror försmäkta av törst.
\par 14 Ja, därför spärrar dödsriket upp sitt gap, det öppnar sina käftar utan allt mått, och stadens yperste måste fara ditned, jämte dess larmande och sorlande skaror, envar som fröjdar sig därinne.
\par 15 Så bliva människorna nedböjda och männen ödmjukade, ja, ödmjukade varda de högmodigas ögon.
\par 16 Men HERREN Sebaot bliver hög genom sin dom, Gud, den helige, bevisar sig helig genom rättfärdighet.
\par 17 Och lamm gå där i bet såsom på sin egen mark, och på de rikas ödetomter söka vandrande herdar sin föda.
\par 18 Ve dem som draga fram missgärningsstraff med lögnens tåg och syndastraff såsom med vagnslinor,
\par 19 dem som säga: "Må han hasta, må han skynda med sitt verk, så att vi få se det; må det som Israels Helige har beslutit nalkas och komma, så att vi förnimma det!"
\par 20 Ve dem som kalla det onda gott, och det goda ont, dem som göra mörker till ljus, och ljus till mörker, dem som göra surt till sött, och sött till surt!
\par 21 Ve dem som äro visa i sina egna ögon och hålla sig själva för kloka!
\par 22 Ve dem som äro hjältar i att dricka vin och som äro tappra i att blanda starka drycker,
\par 23 dem som giva den skyldige rätt för mutors skull, men beröva den oskyldige vad som är hans rätt!
\par 24 Därför, såsom eldsflamman förtär strå, och såsom halm sjunker tillsammans i lågan, så skall deras rot förruttna, och deras löv skola flyga bort såsom stoft, eftersom de förkastade HERREN Sebaots lag och föraktade Israels Heliges ord.
\par 25 Därför har HERRENS vrede upptänts mot hans folk, och han uträcker sin hand emot det och slår det, så att bergen darra, och så att döda kroppar ligga såsom orenlighet på gatorna. Vid allt detta vänder hans vrede icke åter, hans hand är ännu uträckt.
\par 26 Och han reser upp ett baner för hednafolken i fjärran, och lockar på dem att de skola komma från jordens ända; och se, snart och med hast komma de dit.
\par 27 Ingen finnes bland dem, som är trött, ingen som är stapplande. Ingen unnar sig slummer och ingen sömn; på ingen lossnar bältet omkring hans länder, och för ingen brister en skorem sönder.
\par 28 Deras pilar äro skarpa, och deras bågar äro alla spända; deras hästars hovar äro såsom av flinta, och deras vagnshjul likna stormvinden.
\par 29 Deras skriande är såsom en lejoninnas; de skria såsom unga lejon, rytande gripa de sitt rov och bära bort det, och ingen finnes, som räddar.
\par 30 Ett rytande över folket höres på den dagen, likt rytandet av ett hav; och skådar man ned på jorden, se, då är där mörker och nöd, och ljuset är förmörkat genom töcken.

\chapter{6}

\par 1 I det år då konung Ussia dog såg jag Herren sitta på en hög och upphöjd tron, och släpet på hans mantel uppfyllde templet.
\par 2 Serafer stodo omkring honom. Var och en av dem hade sex vingar: med två betäckte de sina ansikten, med två betäckte de sina fötter, och med två flögo de.
\par 3 Och den ene ropade till den andre och sade: "Helig, helig, helig är HERREN Sebaot; hela jorden är full av hans härlighet."
\par 4 Och dörrtrösklarnas fästen darrade, när ropet ljöd; och huset blev uppfyllt av rök.
\par 5 Då sade jag: "Ve mig, jag förgås! Ty jag har orena läppar, och jag bor ibland ett folk som har orena läppar, och mina ögon hava sett Konungen, HERREN Sebaot."
\par 6 Men en av seraferna flög fram till mig, och han hade i sin hand ett glödande kol, som han med en tång hade tagit på altaret.
\par 7 Och han rörde därmed vid min mun. Därefter sade han: "Se, då nu detta har rört vid dina läppar, har din missgärning blivit tagen ifrån dig, och din synd är försonad."
\par 8 Och jag hörde Herren tala, och han sade: "Vem skall jag sända, och vem vill vara vår budbärare?" Och jag sade: "Se, här är jag, sänd mig."
\par 9 Då sade han: "Gå åstad och säg till detta folk: 'Hören alltjämt, men förstån intet; sen alltjämt, men förnimmen intet'.
\par 10 Förstocka detta folks hjärta, och tillslut dess öron, och förblinda dess ögon, så att det icke kan se med sina ögon, eller höra med sina öron, eller förstå med sitt hjärta, och omvända sig och bliva helat."
\par 11 Men jag sade: "För huru lång tid, Herre?" Han svarade: "Till dess att städerna bliva öde och utan någon invånare, och husen utan folk, och till dess att fälten ligga öde och förhärjade.
\par 12 Och när HERREN har fört folket bort i fjärran och ödsligheten bliver stor i landet,
\par 13 och allenast en tiondedel ännu är kvar däri, då skall denna ytterligare förödas såsom en terebint eller en ek av vilken en stubbe har lämnats kvar, när den fälldes. Den stubben skall vara en helig säd."

\chapter{7}

\par 1 Och i Ahas', Jotams sons, Ussias sons, Juda konungs, tid hände sig att Resin, konungen i Aram, och Peka, Remaljas son, Israels konung, drogo upp mot Jerusalem för att erövra det (vilket de likväl icke förmådde göra).
\par 2 Och när det blev berättat för Davids hus att araméerna hade lägrat sig i Efraim, då skälvde hans och hans folks hjärtan, såsom skogens träd skälva för vinden.
\par 3 Men HERREN sade till Jesaja: "Gå åstad med din son Sear-Jasub och möt Ahas vid ändan av Övre dammens vattenledning, på vägen till Valkarfältet,
\par 4 och säg till honom: Tag dig till vara och håll dig stilla; frukta icke och var icke försagd i ditt hjärta för dessa två rykande brandstumpar, för Resin med araméerna och för Remaljas son, i deras förgrymmelse.
\par 5 Eftersom Aram med Efraim och Remaljas son hava gjort upp onda planer mot dig och sagt:
\par 6 'Vi vilja draga upp mot Juda och slå det med skräck och erövra det åt oss och göra Tabals son till konung där',
\par 7 därför säger Herren, HERREN: Det skall icke lyckas, det skall icke ske.
\par 8 Ty Damaskus är Arams huvud, och Resin är Damaskus' huvud; och om sextiofem år skall Efraim vara krossat, så att det icke mer är ett folk.
\par 9 Och Samaria är Efraims huvud, och Remaljas son är Samarias huvud. Om I icke haven tro, skolen I icke hava ro."
\par 10 Och HERREN talade ytterligare till Ahas och sade:
\par 11 "Begär ett tecken från HERREN, din Gud; du må begära det vare sig nedifrån djupet eller uppifrån höjden."
\par 12 Men Ahas svarade: "Jag begär intet, jag vill icke fresta HERREN."
\par 13 Då sade han: "Så hören då, I av Davids hus: Är det eder icke nog att I sätten människors tålamod på prov? Viljen I ock pröva min Guds tålamod?
\par 14 Så skall då Herren själv giva eder ett tecken: Se, den unga kvinnan skall varda havande och föda en son, och hon skall giva honom namnet Immanuel.
\par 15 Gräddmjölk och honung skall bliva hans mat inemot den tid då han förstår att förkasta vad ont är och utvälja vad gott är.
\par 16 Ty innan gossen förstår att förkasta vad ont är och utvälja vad gott är, skall det land för vars båda konungar du gruvar dig vara öde.
\par 17 Och över dig och över ditt folk och över din faders hus skall HERREN låta dagar komma, sådana som icke hava kommit allt ifrån den tid då Efraim skilde sig från Juda: konungen i Assyrien.
\par 18 Ty på den tiden skall HERREN locka på flugorna längst borta vid Egyptens strömmar och på bisvärmarna i Assyriens land;
\par 19 och de skola komma och slå ned, alla tillhopa, i bergsdälder och stenklyftor, i alla törnsnår och på alla betesmarker.
\par 20 På den tiden skall HERREN med en rakkniv som tingas på andra sidan floden - nämligen med konungen i Assyrien - raka av allt hår både på huvudet och nedtill; ja, också skägget skall den taga bort.
\par 21 På den tiden skall en kviga och två tackor vara vad en man föder upp.
\par 22 Men han skall få mjölk i sådan myckenhet att han kan leva av gräddmjölk; ja, alla som finnas kvar i landet skola leva av gräddmjölk och honung.
\par 23 Och det skall ske på den tiden, att där nu tusen vinträd stå, värda tusen siklar silver, där skall överallt växa tistel och törne.
\par 24 Med pilar och båge skall man gå dit, ty hela landet skall vara tistel och törne.
\par 25 Och alla de berg där man nu arbetar med hackan, dem skall man ej mer beträda, av fruktan för tistel och törne; de skola bliva platser dit oxar drivas, och marker som trampas ned av får."

\chapter{8}

\par 1 Och HERREN sade till mig: "Tag dig en stor tavla och skriv på den med tydlig stil Maher-salal Has-bas.
\par 2 Och jag vill taga mig pålitliga vittnen: prästen Uria och Sakarja, Jeberekjas son."
\par 3 Och jag gick in till profetissan, och hon blev havande och födde en son. Och HERREN sade till mig: "Giv honom namnet Maher-salal Has-bas.
\par 4 Ty förrän gossen kan säga 'fader' och 'moder' skall man bära Damaskus' skatter och byte från Samaria fram för konungen i Assyrien."
\par 5 Och HERREN talade vidare till mig och sade:
\par 6 "Eftersom detta folk föraktar Siloas vatten, som flyter så stilla, och har sin fröjd med Resin och Remaljas son,
\par 7 se, därför skall HERREN låta komma över dem flodens vatten, de väldiga och stora, nämligen konungen i Assyrien med all hans härlighet. Och den skall stiga över alla sina bräddar och gå över alla sina stränder.
\par 8 Den skall tränga fram i Juda, svämma över och utbreda sig och räcka ända upp till halsen; och med sina utbredda vingar, skall den uppfylla ditt land, Immanuel, så vitt det är."
\par 9 Rasen, I folk; I skolen dock krossas. Lyssnen, alla I fjärran länder. Rusten eder; I skolen dock krossas. Ja, rusten eder; I skolen dock krossas.
\par 10 Gören upp planer; de varda dock om intet. Avtalen, vad I viljen; det skall dock ej lyckas. Ty Gud är med oss.
\par 11 Ty så sade HERREN till mig, när hans hand kom över mig med makt och han varnade mig för att vandra på detta folks väg:
\par 12 I skolen icke kalla för sammansvärjning allt vad detta folk kallar sammansvärjning, ej heller skolen I frukta vad det fruktar, I skolen icke förskräckas därför.
\par 13 Nej, HERREN Sebaot skolen I hålla helig; honom skolen I frukta, och för honom skolen I förskräckas.
\par 14 Så skall han varda för eder något heligt; men för de två Israels hus skall han bliva en stötesten och en klippa till fall och för Jerusalems invånare en snara och ett giller.
\par 15 Många av dem skola stupa därpå, de skola falla och krossas, de skola snärjas och varda fångade.
\par 16 Lägg vittnesbördet ombundet och lagen förseglad i mina lärjungars hjärtan.
\par 17 Så vill jag förbida HERREN, då han nu döljer sitt ansikte för Jakobs hus; jag vill vänta efter honom.
\par 18 Se, jag och barnen som HERREN har givit mig, vi äro tecken och förebilder i Israel, från HERREN Sebaot, som bor på Sions berg.
\par 19 Och när man säger till eder: "Frågen andebesvärjare och spåmän, dem som viska och mumla", så svaren: "Skall icke ett folk fråga sin Gud? Skall man fråga de döda för de levande?"
\par 20 "Nej, hållen eder till lagen, till vittnesbördet!" Så skola förvisso en gång de nödgas mana, för vilka nu ingen morgonrodnad finnes.
\par 21 De skola draga omkring i landet, nedtryckta och hungrande, och i sin hunger skola de förbittras och skola förbanna sin konung och sin Gud. Och de skola vända blicken uppåt, de skola ock skåda ned på jorden;
\par 22 men se, där är nöd och mörker och natt av ångest. Ja, tjockt mörker är de fördrivnas liv.

\chapter{9}

\par 1 Dock, natt skall icke förbliva där nu ångest råder. I den förgångna tiden har har han låtit Sebulons och Naftalis land vara ringa aktat, men i framtiden skall han låta det komma till ära, trakten utmed Havsvägen, landet på andra sidan Jordan, hedningarnas område.
\par 2 Det folk som vandrar i mörkret skall se ett stort ljus; ja, över dem som bo i dödsskuggans land skall ett ljus skina klart.
\par 3 Du skall göra folket talrikt, du skall göra dess glädje stor; inför dig skola de glädja sig, såsom man glädes under skördetiden, såsom man fröjdar sig, när man utskiftar byte.
\par 4 Ty du skall bryta sönder deras bördors ok och deras skuldrors gissel och deras plågares stav, likasom i Midjans tid.
\par 5 Och skon som krigaren bar i stridslarmet, och manteln som sölades i blod, allt sådant skall brännas upp och förtäras av eld.
\par 6 Ty ett barn varder oss fött, en son bliver oss given, och på hans skuldror skall herradömet vila; och hans namn skall vara: Underbar i råd, Väldig Gud, Evig fader, Fridsfurste.
\par 7 Så skall herradömet varda stort och friden utan ände över Davids tron och över hans rike; så skall det befästas och stödjas med rätt och rättfärdighet, från nu och till evig tid. HERREN Sebaots nitälskan skall göra detta.
\par 8 Ett ord sänder Herren mot Jakob, och det slår ned i Israel,
\par 9 och allt folket får förnimma det, Efraim och Samarias invånare, de som säga i sitt övermod och i sitt hjärtas stolthet:
\par 10 "Tegelmurar hava fallit, men med huggen sten bygga vi upp nya; mullbärsfikonträd har man huggit ned, men cederträd sätta vi i deras ställe."
\par 11 Och HERREN uppreser mot dem Resins ovänner och uppeggar deras fiender,
\par 12 araméerna från den ena sidan och filistéerna från den andra, och de äta upp Israel med glupska gap. Vid allt detta vänder hans vrede icke åter, hans hand är ännu uträckt.
\par 13 Men folket vänder ej åter till honom som slår dem; Herren Sebaot söka de icke.
\par 14 Därför avhugger HERREN på Israel både huvud och svans, han hugger av både palmtopp och sävstrå, allt på en dag -
\par 15 de äldste och högst uppsatte de äro huvudet, och profeterna, de falska vägvisarna, de äro svansen.
\par 16 Ty detta folks ledare föra det vilse, och de som låta leda sig gå i fördärvet.
\par 17 Därför kan Herren icke glädja sig över dess unga män, ej heller hava förbarmande med dess faderlösa och änkor; ty de äro allasammans gudlösa ogärningsmän, och var mun talar dårskap. Vid allt detta vänder hans vrede icke åter, hans hand är ännu uträckt.
\par 18 Ty ogudaktigheten förbränner såsom en eld, den förtär tistel och törne; den tänder på den tjocka skogen, så att den går upp i höga virvlar av rök.
\par 19 Genom HERREN Sebaots förgrymmelse har landet råkat i brand, och folket är likasom eldsmat; den ene skonar icke den andre.
\par 20 Man river åt sig till höger och förbliver dock hungrig, man tager för sig till vänster och bliver dock ej mätt; envar äter köttet på sin egen arm:
\par 21 Manasse äter Efraim, och Efraim Manasse, och båda tillhopa vända sig mot Juda. Vid allt detta vänder hans vrede icke åter, hans hand är ännu uträckt.

\chapter{10}

\par 1 Ve eder som stadgen orättfärdiga stadgar! I skriven, men våldslagar skriven I
\par 2 för att vränga de ringas sak och beröva de fattiga i mitt folk deras rätt, för att göra änkor till edert byte och plundra de faderlösa.
\par 3 Vad viljen I göra på hemsökelsens dag, när ovädret kommer fjärran ifrån? Till vem viljen I fly för att få hjälp, och var viljen I lämna edra skatter i förvar?
\par 4 Om man ej böjer knä såsom fånge, så måste man falla bland de dräpta. Vid allt detta vänder hans vrede icke åter, hans hand är ännu uträckt.
\par 5 Ve över Assur, min vredes ris, som bär min ogunst såsom en stav i sin hand!
\par 6 Mot ett gudlöst folk sänder jag honom, och mot min förgrymmelses folk bjuder jag honom gå, för att taga rov och göra byte, och för att nedtrampa det såsom orenlighet på gatorna.
\par 7 Men så menar icke han, och i sitt hjärta tänker han ej så, utan hans hjärta står efter att förgöra och efter att utrota folk i mängd.
\par 8 Han säger: "Äro mina hövdingar ej allasammans konungar?
\par 9 Har det icke gått Kalno såsom Karkemis, och Hamat såsom Arpad, och Samaria såsom Damaskus?
\par 10 Då min hand har träffat de andra gudarnas riken, vilkas beläten voro förmer än Jerusalems och Samarias,
\par 11 skulle jag då ej kunna göra med Jerusalem och dess gudabilder vad jag har gjort med Samaria och dess gudar?"
\par 12 Men när Herren har fullbordat allt sitt verk på Sions berg och i Jerusalem, då skall jag hemsöka den assyriske konungens hjärtas högmodsfrukt och hans stolta ögons förhävelse.
\par 13 Ty han säger: "Med min hands kraft har jag utfört detta och genom min vishet, ty jag har förstånd. Jag flyttade folkens gränser, deras förråd utplundrade jag, och i min väldighet stötte jag härskarna från tronen.
\par 14 Och min hand grep efter folkens skatter såsom efter fågelnästen, och såsom man samlar övergivna ägg, så samlade jag jordens alla länder; ingen fanns, som rörde vingen eller öppnade näbben till något ljud."
\par 15 Skall då yxan berömma sig mot honom som hugger med den, eller sågen förhäva sig mot honom som sätter den i rörelse? Som om käppen satte i rörelse honom som lyfter den, eller staven lyfte en som dock är förmer än ett stycke trä!
\par 16 Så skall då Herren, HERREN Sebaot sända tärande sjukdom i hans feta kropp, och under hans härlighet skall brinna en brand likasom en brinnande eld.
\par 17 Och Israels ljus skall bliva en eld och hans Helige en låga, och den skall bränna upp och förtära dess törnen och dess tistlar, allt på en dag.
\par 18 Och på hans skogars och parkers härlighet skall han alldeles göra en ände; det skall vara, såsom när en sjuk täres bort.
\par 19 De träd som bliva kvar i hans skog skola vara lätt räknade; ett barn skall kunna teckna upp dem.
\par 20 På den tiden skall kvarlevan av Israel och den räddade skaran av Jakobs hus ej vidare stödja sig vid honom som slog dem; i trohet skola de stödja sig vid HERREN, Israels Helige.
\par 21 En kvarleva skall omvända sig, en kvarleva av Jakob, till Gud, den väldige.
\par 22 Ty om än ditt folk, Israel, vore såsom sanden i havet, så skall dock allenast en kvarleva där omvända sig. Förödelsen är oryggligt besluten, den kommer med rättfärdighet såsom en flod.
\par 23 Ty förstöring och oryggligt besluten straffdom skall Herren, HERREN Sebaot låta komma över hela jorden.
\par 24 Därför säger Herren, HERREN Sebaot så: Frukta icke, mitt folk, du som bor i Sion, för Assur, när han slår dig med riset och upplyfter sin stav mot dig, såsom man gjorde i Egypten.
\par 25 Ty ännu allenast en liten tid, och ogunsten skall hava en ände, och min vrede skall vända sig till deras fördärv.
\par 26 Och HERREN Sebaot skall svänga sitt gissel över dem, såsom när han slog Midjan vid Orebsklippan; och sin stav, som han räckte ut över havet, skall han åter upplyfta, såsom han gjorde i Egypten.
\par 27 På den tiden skall hans börda tagas bort ifrån din skuldra och hans ok ifrån din hals, ty oket skall brista sönder för fetmas skull.
\par 28 Han kommer över Ajat, han drager fram genom Migron; i Mikmas lämnar han sin tross.
\par 29 De draga fram över passet; i Geba taga de nattkvarter. Rama bävar; Sauls Gibea flyr.
\par 30 Ropa högt, du dotter Gallim. Giv akt, du Laisa. Arma Anatot!
\par 31 Madmena flyktar; Gebims invånare bärga sitt gods.
\par 32 Ännu samma dag står han i Nob; han lyfter sin hand mot dottern Sions berg, mot Jerusalems höjd.
\par 33 Men se, då avhugger Herren, HERREN Sebaot den lummiga kronan, med förskräckande makt; de resliga stammarna ligga fällda, de höga träden störta ned.
\par 34 Den tjocka skogen nedhugges med järnet; Libanons skogar falla för den väldige.

\chapter{11}

\par 1 Men ett skott skall skjuta upp ur Isais avhuggna stam, och en telning från dess rötter skall bära frukt.
\par 2 Och på honom skall HERRENS Ande vila, vishets och förstånds Ande, råds och starkhets Ande, HERRENS kunskaps och fruktans Ande.
\par 3 Han skall hava sitt välbehag i HERRENS fruktan; och han skall icke döma efter som ögonen se eller skipa lag efter som öronen höra.
\par 4 Utan med rättfärdighet skall han döma de arma och med rättvisa skipa lag åt de ödmjuka på jorden. Och han skall slå jorden med sin muns stav, och med sina läppars anda döda de ogudaktiga.
\par 5 Rättfärdighet skall vara bältet omkring hans länder och trofasthet bältet omkring hans höfter.
\par 6 Då skola vargar bo tillsammans med lamm och pantrar ligga tillsammans med killingar; och kalvar och unga lejon och gödboskap skola sämjas tillhopa, och en liten gosse skall valla dem.
\par 7 Kor och björnar skola gå och beta, deras ungar skola ligga tillhopa, och lejon skola äta halm likasom oxar.
\par 8 Ett spenabarn skall leka invid en huggorms hål och ett avvant barn sträcka ut sin hand efter basiliskens öga.
\par 9 Ingenstädes på mitt heliga berg skall man då göra vad ont och fördärvligt är, ty landet skall vara fullt av HERRENS kunskap, likasom havsdjupet är fyllt av vattnet.
\par 10 Och det skall ske på den tiden att hednafolken skola söka telningen från Isais rot, där han står såsom ett baner för folken; och hans boning skall vara idel härlighet.
\par 11 Och HERREN skall på den tiden ännu en gång räcka ut sin hand, för att förvärva åt sig kvarlevan av sitt folk, vad som har blivit räddat från Assyrien, Egypten, Patros, Etiopien, Elam, Sinear, Hamat och havsländerna.
\par 12 Och han skall resa upp ett baner för hednafolken och samla Israels fördrivna män; och Juda förskingrade kvinnor skall han hämta tillhopa från jordens fyra hörn.
\par 13 Då skall Efraims avund upphöra och Juda ovänskap bliva utrotad; Efraim skall ej hysa avund mot Juda, och Juda icke ovänskap mot Efraim.
\par 14 Men såsom rovfåglar skola de slå ned på filistéernas skuldra västerut, tillsammans skola de taga byte av österlänningarna; Edom och Moab skola gripas av deras hand, och Ammons barn skola bliva dem hörsamma.
\par 15 Och HERREN skall giva till spillo Egyptens havsvik och lyfta sin hand mot floden i förgrymmelse; och han skall klyva den i sju bäckar och göra så, att man torrskodd kan gå däröver.
\par 16 Så skall där bliva en banad väg för den kvarleva av hans folk, som har blivit räddad från Assur, likasom det var för Israel på den dag då de drogo upp ur Egyptens land.

\chapter{12}

\par 1 På den tiden skall du säga: "Jag tackar dig, HERRE, ty väl var du vred på mig, men din vrede har upphört, och du tröstar mig.
\par 2 Se, Gud är min frälsning, jag är trygg och fruktar icke; ty HERREN, HERREN är min starkhet och min lovsång, och han blev mig till frälsning."
\par 3 Och I skolen ösa vatten med fröjd ur frälsningens källor
\par 4 och skolen säga på den tiden: "Tacken HERREN, åkallen hans namn, gören hans gärningar kunniga bland folken; förtäljen att hans namn är högt.
\par 5 Lovsjungen HERREN, ty han har gjort härliga ting; detta vare kunnigt över hela jorden.
\par 6 Ropen av fröjd och jublen, I Sions invånare, ty Israels Helige är stor bland eder.

\chapter{13}

\par 1 Detta är en utsaga om Babel, vad som uppenbarades för Jesaja, Amos' son.
\par 2 Resen upp ett baner på ett kalt berg, ropen högt till dem; viften med handen att de må draga in genom de mäktiges portar.
\par 3 Jag själv har bådat upp mina invigda, ja, kallat mina hjältar till mitt vredesverk, min stolta skara, som jublar.
\par 4 Hör, det larmar på bergen såsom av ett stort folk. Hör, det sorlar av riken med hopade hednafolk. HERREN Sebaot mönstrar sin krigarskara.
\par 5 Ifrån fjärran land komma de, ifrån himmelens ända, HERREN och hans vredes redskap, för att fördärva hela jorden.
\par 6 Jämren eder, ty nära är HERRENS dag; såsom våld från den Allsvåldige kommer den.
\par 7 Därför sjunka alla händer ned, och alla människohjärtan förfäras.
\par 8 Man förskräckes, man gripes av ångest och kval, ja, våndas såsom en barnaföderska. Häpen stirrar den ene på den andre; röda såsom eldslågor äro deras ansikten.
\par 9 Se, HERRENS dag kommer, gruvlig och med förgrymmelse och med vredesglöd, för att göra jorden till en ödemark och utrota syndarna som där bo.
\par 10 Ty himmelens stjärnor och stjärnbilder sända ej mer ut sitt ljus, solen går mörk upp, och månens ljus skiner icke.
\par 11 Jag skall hemsöka jordens krets för dess ondska och de ogudaktiga för deras missgärning; jag skall göra slut på de fräckas övermod och slå ned våldsverkarnas högmod.
\par 12 Jag skall göra en man mer sällsynt än fint guld, en människa mer sällsynt än guld från Ofir.
\par 13 Därför skall jag komma himmelen att darra, och jorden skall bäva och vika från sin plats - genom HERREN Sebaots förgrymmelse, på hans glödande vredes dag.
\par 14 Och likasom jagade gaseller och en hjord som ingen samlar vända de då hem, var och en till sitt folk, och fly, var och en till sitt land.
\par 15 Men envar som upphinnes bliver genomborrad, och envar som gripes faller för svärd.
\par 16 Deras späda barn krossas inför deras ögon, deras hus plundras, och deras kvinnor skändas.
\par 17 Ty se, jag vill uppväcka mot dem mederna, som akta silver för intet och icke fråga efter guld.
\par 18 Deras bågar skola fälla de unga männen, med frukten i moderlivet hava de intet förbarmande, och barnen skona de icke.
\par 19 Och det skall gå med Babel, rikenas krona, kaldéernas ära och stolthet, likasom när Gud omstörtade Sodom och Gomorra.
\par 20 Aldrig mer skall det bliva bebyggt, från släkte till släkte skall det ligga obebott; ingen arab skall där slå upp sitt tält, ingen herde lägra sig där med sin hjord.
\par 21 Nej, öknens djur skola lägra sig där, och dess hus skola fyllas av uvar; strutsar skola bo där, och gastar skola hoppa där.
\par 22 Schakaler skola tjuta i dess palatser och ökenhundar i praktbyggnaderna. Snart kommer dess tid; dess dagar skola ej fördröjas.

\chapter{14}

\par 1 Ty HERREN skall förbarma sig över Jakob och ännu en gång utvälja Israel och låta dem komma till ro i deras land; och främlingar skola sluta sig till dem och hålla sig till Jakobs hus.
\par 2 Och folk skola taga dem och föra dem hem igen; men Israels hus skall lägga dem under sig såsom sin arvedel i HERRENS land, och skall göra dem till trälar och trälinnor. Så skola de få sina fångvaktare till fångar och råda över sina plågare.
\par 3 Och på den dag då HERREN låter dig få ro från din vedermöda och ångest, och från den hårda träldom som har varit dig pålagd,
\par 4 då skall du stämma upp denna visa över konungen i Babel, du skall säga: "Vilken ände har icke plågaren fått, vilken ände pinoorten!
\par 5 HERREN har brutit sönder de ogudaktigas stav, tyrannernas ris,
\par 6 det ris som i grymhet slog folken med slag på slag, och i vrede härskade över folkslagen med skoningslös förföljelse.
\par 7 Hela jorden har nu fått vila och ro; man brister ut i jubel.
\par 8 Själva cypresserna glädja sig över ditt fall, så ock Libanons cedrar: 'Sedan du nu ligger där, drager ingen hitupp för att hugga ned oss.'
\par 9 Dödsriket därnere störes i sin ro för din skull, när det måste taga emot dig. Skuggorna där väckas upp för din skull, jordens alla väldige; folkens alla konungar måste stå upp från sina troner.
\par 10 De upphäva alla sin röst och säga till dig: 'Så har då också du blivit maktlös såsom vi, ja, blivit vår like.'
\par 11 Ned till dödsriket har din härlighet måst fara, och dina harpors buller; förruttnelse är bädden under dig, och maskar äro ditt täcke.
\par 12 Huru har du icke fallit ifrån himmelen, du strålande morgonstjärna! Huru har du icke blivit fälld till jorden, du folkens förgörare!
\par 13 Det var du som sade i ditt hjärta: 'Jag vill stiga upp till himmelen; högt ovanför Guds stjärnor vill jag ställa min tron; jag vill sätta mig på gudaförsamlingens berg längst uppe i norr.
\par 14 Jag vill stiga upp över molnens höjder, göra mig lik den Högste.'
\par 15 Nej, ned till dödsriket måste du fara, längst ned i graven.
\par 16 De som se dig stirra på dig, de betrakta dig och säga: 'Är detta den man som kom jorden att darra och riken att bäva,
\par 17 den som förvandlade jordkretsen till en öken och förstörde dess städer, den som aldrig frigav sina fångar, så att de fingo återvända hem?'
\par 18 Folkens alla konungar ligga allasammans med ära var och en i sitt vilorum;
\par 19 men du ligger obegraven och bortkastad, lik en föraktad gren; du ligger där överhöljd av dräpta, av svärdsslagna män, av döda som hava farit ned i gravkammaren, lik ett förtrampat as.
\par 20 Du skulle icke såsom de få vila i en grav, ty du fördärvade ditt land och dräpte ditt folk. Om ogärningsmännens avföda skall man aldrig mer tala.
\par 21 Anställen ett blodbad på hans söner för deras fäders missgärning. De få ej stå upp och besitta jorden och fylla jordkretsens yta med städer."
\par 22 Nej, jag skall stå upp emot dem, säger HERREN Sebaot; och jag skall utrota ur Babel både namn och kvarleva, både barn och efterkommande, säger HERREN.
\par 23 Och jag skall göra det till ett tillhåll för rördrommar och fylla det med sumpsjöar; ja, jag skall bortsopa det med ödeläggelsens kvast, säger HERREN Sebaot.
\par 24 HERREN Sebaot har svurit och sagt: Sannerligen, såsom jag har tänkt, så skall det ske, och vad jag har beslutit, det skall fullbordas.
\par 25 Jag skall krossa Assur i mitt land, och på mina berg skall jag förtrampa honom. Så skall hans ok bliva borttaget ifrån dem och hans börda tagas av deras skuldra.
\par 26 Detta är det beslut som är fattat mot hela jorden; detta är den hand som är uträckt mot alla folk.
\par 27 Ty HERREN Sebaot har beslutit det; vem kan då göra det om intet? Hans hand är det som är uträckt; vem kan avvända den?
\par 28 I det år då konung Ahas dog förkunnades följande utsaga:
\par 29 Gläd dig icke, du filistéernas hela land, över att det ris som slog dig är sönderbrutet; ty från ormens rot skall en basilisk komma fram, och dennes avkomma bliver en flygande drake.
\par 30 De utarmade skola sedan få bete och de fattiga få lägra sig i trygghet; men telningarna från din rot skall jag döda genom hunger, och vad som bliver kvar av dig skall dräpas.
\par 31 Jämra dig, du port; ropa, du stad; försmält av ångest, du filistéernas hela land. Ty norrifrån kommer en rök; i fiendeskarornas tåg bliver ingen efter.
\par 32 Vad skall man då svara det främmande folkets sändebud? Jo, att det är HERREN som har grundat Sion, och att de betryckta bland hans folk där hava sin tillflykt.

\chapter{15}

\par 1 Utsaga om Moab. Ja, med Ar-Moab är det ute den natt då det förstöres. Ja, med Kir-Moab är det ute den natt då det förstöres.
\par 2 Habbait och Dibon stiga upp på offerhöjderna för att gråta; uppe i Nebo och Medeba jämrar sig Moab; alla huvuden där äro skalliga, alla skägg avskurna.
\par 3 På dess gator bär man sorgdräkt, så ock på dess tak; på dess torg jämra sig alla och flyta i tårar.
\par 4 Hesbon och Eleale höja klagorop, så att det höres ända till Jahas. Därför skria ock Moabs krigare; hans själ våndas i honom.
\par 5 Mitt hjärta klagar över Moab, ty hans flyktingar fly ända till Soar, till Eglat-Selisia; uppför Halluhits höjd stiger man under gråt, och på vägen till Horonaim höjas klagorop över förstörelsen.
\par 6 Nimrims vatten bliva torr ökenmark, gräset förtorkas, brodden vissnar, intet grönt lämnas kvar.
\par 7 Återstoden av sitt förvärv, sitt sparda gods, bär man därför nu bort över Pilträdsbäcken.
\par 8 Ja, klagoropen ljuda runtom i Moabs land; till Eglaim når dess jämmer och till Beer-Elim dess jämmer.
\par 9 Dimons vatten äro fulla av blod. Ja, ännu något mer skall jag låta komma över Dimon; ett lejon över Moabs räddade, över det som bliver kvar av landet.

\chapter{16}

\par 1 "Sänden åstad de lamm som landets herre bör hava från Sela genom öknen till dottern Sions berg."
\par 2 Och såsom flyktande fåglar, lika skrämda fågelungar komma Moabs döttrar till Arnons vadställen.
\par 3 De säga: "Giv oss råd, bliv medlare för oss. Låt din skugga vara såsom natten, nu mitt i middagshettan. Skydda de fördrivna; röj icke de flyktande.
\par 4 Låt mina fördrivna finna härbärge hos dig, var för Moab ett beskärm mot fördärvaren, till dess att utpressaren ej mer är till och fördärvet får en ände och förtryckarna försvinna bort ur landet.
\par 5 Så skall genom eder mildhet eder tron bliva befäst, och på den skall sitta trygg i Davids hydda en furste som far efter vad rätt är och främjar rättfärdighet."
\par 6 Vi hava hört om Moabs högmod, det övermåttan höga, om hans högfärd, högmod och övermod och opålitligheten i hans lösa tal.
\par 7 Därför måste nu Moab jämra sig över Moab, hela landet måste jämra sig. Över Kir-Haresets druvkakor måsten I sucka i djup bedrövelse.
\par 8 Ty Hesbons fält äro förvissnade, så ock Sibmas vinträd, vilkas ädla druvor slogo folkens herrar till marken, vilkas rankor nådde till Jaeser och förirrade sig i öknen, vilkas skott bredde ut sig och gingo över havet.
\par 9 Därför gråter jag över Sibmas vinträd, såsom Jaeser gråter; med mina tårar vattnar jag dig, Hesbon, och dig, Eleale. Ty mitt i din sommar och din bärgningstid har ett skördeskri slagit ned.
\par 10 Glädje och fröjd är nu avbärgad från de bördiga fälten, och i vingårdarna höjes intet glädjerop, höres intet jubel; ingen trampar vin i pressarna, på skördeskriet har jag gjort slut.
\par 11 Därför klagar mitt hjärta såsom en harpa över Moab, ja, mitt innersta över Kir-Heres.
\par 12 Ty huru än Moab ävlas att träda upp på offerhöjden och huru han än går in i sin helgedom och beder, så uträttar han intet därmed.
\par 13 Detta är det ord, som HERREN tillförne talade till Moab.
\par 14 Men nu har HERREN åter talat och sagt: Inom tre år, såsom dagakarlen räknar åren, skall Moab i sin härlighet, med alla sina stora skaror, varda aktad för intet; och vad som bliver kvar skall vara litet och ringa, icke mycket värt.

\chapter{17}

\par 1 Utsaga om Damaskus. Se, Damaskus skall upphöra att vara en stad; det skall falla och bliva en stenhop.
\par 2 Aroers städer varda övergivna; de bliva tillhåll för hjordar, som lägra sig där ostörda.
\par 3 Det är förbi med Efraims värn, med Damaskus' konungadöme och med kvarlevan av Aram. Det skall gå med dem såsom med Israels barns härlighet, säger HERREN Sebaot.
\par 4 Och det skall ske på den tiden att Jakobs härlighet vändes i armod, och att hans feta kropp bliver mager.
\par 5 Det går, såsom när skördemannen samlar ihop säden och med sin arm skördar axen; det går, såsom när man plockar ax i Refaims-dalen:
\par 6 en ringa efterskörd lämnas kvar där, såsom när man slår ned oliver, två eller tre bär lämnas kvar högst uppe i toppen, fyra eller fem på trädets kvistar, säger HERREN, Israels Gud.
\par 7 På den tiden skola människorna blicka upp till sin Skapare och deras ögon se upp till Israels Helige.
\par 8 Människorna skola ej vända sin blick till de altaren som deras händer hava gjort; på sina fingrars verk skola de icke se, icke på Aserorna eller på solstoderna.
\par 9 På den tiden skola deras fasta städer bliva lika de övergivna fästen i skogarna och på bergstopparna, som övergåvos, när Israels barn drogo in; allt skall bliva ödelagt.
\par 10 Ty du har förgätit din frälsnings Gud, och du tänker icke på din fasta klippa. Därför planterar du ljuvliga planteringar och sätter i dem främmande vinträd.
\par 11 Och väl får du dem att växa högt samma dag du planterar dem, och morgonen därefter får du dina plantor att blomma, men skörden försvinner på hemsökelsens dag, då plågan bliver olidlig.
\par 12 Hör, det brusar av många folk, det brusar, såsom havet brusar. Det dånar av folkslag, det dånar, såsom väldiga vatten dåna.
\par 13 Ja, det dånar av folkslag, såsom stora vatten dåna. Men han näpser dem, och de fly bort i fjärran; de jagas bort såsom agnar för vinden, uppe på bergen, och såsom virvlande löv för stormen.
\par 14 När aftonen är inne, se, då kommer förskräckelsen, och förrän morgonen gryr, äro de sin kos. Detta bliver våra skövlares del, våra plundrares lott.

\chapter{18}

\par 1 Hör, du land där flygfän surra, du land bortom Etiopiens strömmar,
\par 2 du som har sänt budbärare över havet, i rörskepp hän över vattnet! Gån åstad, I snabba sändebud, till det resliga folket med glänsande hy, till folket som är så fruktat vida omkring, det starka och segerrika folket, vars land genomskäres av strömmar.
\par 3 I jordkretsens alla inbyggare, I som bon på jorden: sen till, när man reser upp baner på bergen, och när man stöter i basun, så lyssnen därtill.
\par 4 Ty så har HERREN sagt till mig: "I stillhet vill jag skåda ned från min boning, såsom solglans glöder från en klar himmel, såsom molnet utgjuter dagg under skördetidens hetta."
\par 5 Ty förrän skördetiden är inne, just när blomningen är slut och blomman förbytes i mognad druva, skall han avskära rankorna med vingårdskniv och hugga av rotskotten och skaffa dem bort.
\par 6 Alltsammans skall lämnas till pris åt rovfåglarna på bergen och åt djuren på marken; rovfåglarna skola där hava sina nästen över sommaren och markens alla djur ligga där om vintern.
\par 7 På den tiden skola skänker bäras fram till HERREN Sebaot från det resliga folket med glänsande hy, från folket som är så fruktat vida omkring, det starka och segerrika folket, vars land genomskäres av strömmar - till den plats där HERREN Sebaots namn bor, till Sions berg..

\chapter{19}

\par 1 Utsaga om Egypten. Se, HERREN far fram på ett ilande moln och kommer till Egypten. Egyptens avgudar bäva då för honom, och egyptiernas hjärtan förfäras i deras bröst.
\par 2 Och jag skall uppegga egyptier mot egyptier, så att broder skall strida mot broder och vän mot vän, stad mot stad och rike mot rike.
\par 3 Och egyptiernas förstånd skall försvinna ur deras hjärtan, och deras råd skall jag göra om intet; de skola då fråga sina avgudar och signare, sina andebesvärjare och spåmän.
\par 4 Men jag skall giva egyptierna i en hård herres hand, och en grym konung skall få råda över dem, säger Herren, HERREN Sebaot.
\par 5 Och vattnet skall försvinna ur havet, och floden skall sina bort och uttorka.
\par 6 Strömmarna skola utbreda stank, Egyptens kanaler skola förminskas och sina bort; rör och vass skall förvissna.
\par 7 Ängarna vid Nilfloden, längs flodens strand, och alla sädesfält vid floden, de skola förtorka, fördärvas och varda till intet.
\par 8 Dess fiskare skola klaga, alla som kasta ut krok i floden skola sörja; och de som lägga ut nät i vattnet skola stå där modlösa.
\par 9 De som arbeta i häcklat lin skola komma på skam, så ock de som väva fina tyger.
\par 10 Landets stödjepelare skola bliva krossade och alla de som arbeta för lön gripas av ångest.
\par 11 Såsom idel dårar stå då Soans furstar; Faraos visaste rådgivare giva blott oförnuftiga råd. Huru kunnen I då säga till Farao: "Jag är en son av visa män, en son av forntidens konungar"?
\par 12 Ja, var är dina vise? Må de förkunna för dig - ty de veta det ju - vad HERREN Sebaot har beslutit över Egypten.
\par 13 Nej, Soans furstar hava blivit dårar, Nofs furstar äro bedragna, Egypten föres vilse av dem som voro hörnstenar i dess stammar.
\par 14 HERREN har där utgjutit en förvirringens ande, så att de komma Egypten att ragla, vadhelst det företager sig, såsom en drucken raglar i sina spyor.
\par 15 Och Egypten skall icke hava framgång i vad någon där gör, evad han är huvud eller svans, evad han är palmtopp eller sävstrå.
\par 16 På den tiden skola egyptierna vara såsom kvinnor: de skola bäva och förskräckas för HERREN Sebaots upplyfta hand, när han lyfter den mot dem.
\par 17 Och Juda land skall bliva en skräck för egyptierna; så ofta man nämner det för dem, skola de förskräckas, för det besluts skull som HERREN Sebaot har fattat över dem.
\par 18 På den tiden skola i Egyptens land finnas fem städer som tala Kanaans tungomål, och som svärja vid HERREN Sebaot; en av dem skall heta Ir-Haheres.
\par 19 På den tiden skall ett altare vara rest åt HERREN mitt i Egyptens land, så ock en stod åt HERREN vid landets gräns.
\par 20 Och de skola vara till tecken och vittnesbörd för HERREN Sebaot i egyptiernas land: när de ropa till HERREN om hjälp mot förtryckare, då skall han sända dem en frälsare och försvarare, och han skall rädda dem.
\par 21 Och HERREN skall göra sig känd för egyptierna, ja, egyptierna skola lära känna HERREN på den tiden; och de skola tjäna honom med slaktoffer och spisoffer, de skola göra löften åt HERREN och få infria dem.
\par 22 Så skall då HERREN slå Egypten - slå, men ock hela; när de omvända sig till HERREN, skall han bönhöra dem och hela dem.
\par 23 På den tiden skall en banad väg leda från Egypten till Assyrien, och assyrierna skola komma in i Egypten, och egyptierna in i Assyrien; och egyptierna skola hålla gudstjänst tillsammans med assyrierna.
\par 24 På den tiden skall Israel, såsom den tredje i förbundet, stå vid sidan av Egypten och Assyrien, till en välsignelse på jorden.
\par 25 Och HERREN Sebaot skall välsigna dem och säga: Välsignad vare du Egypten, mitt folk, och du Assyrien, mina händer verk, och du Israel, min arvedel!

\chapter{20}

\par 1 I det år då Tartan kom till Asdod, utsänd av Sargon, konungen i Assyrien - varefter han ock belägrade Asdod och intog det -
\par 2 på den tiden talade HERREN genom Jesaja, Amos' son, och sade: "Upp, lös säcktygsklädnaden från dina länder, och drag dina skor av dina fötter." Och denne gjorde så och gick naken och barfota.
\par 3 Sedan sade HERREN: "Likasom min tjänare Jesaja har gått naken och barfota och nu i tre år varit till tecken och förebild angående Egypten och Etiopien,
\par 4 så skall konungen i Assyrien låta fångarna ifrån Egypten och de bortförda från Etiopien, både unga och gamla, vandra åstad nakna och barfota, med blottad bak, Egypten till blygd.
\par 5 Då skola de häpna och blygas över Etiopien, som var deras hopp, och över Egypten, som var deras stolthet.
\par 6 På den dagen skola inbyggarna här i kustlandet säga: 'Se, så gick det med dem som voro vårt hopp, med dem till vilka vi flydde, för att få hjälp och bliva räddade undan konungen i Assyrien; huru skola vi då själva kunna undkomma?'"

\chapter{21}

\par 1 Utsaga om Öknen vid havet. Likasom en storm som far fram i Sydlandet kommer det från öknen, från det fruktansvärda landet.
\par 2 En gruvlig syn har blivit mig kungjord: "Härjare härja, rövare röva. Drag upp, du Elam! Träng på, du Mediens folk! På all suckan vill jag göra slut."
\par 3 Fördenskull darra nu mina länder, ångest griper mig, lik en barnaföderskas ångest; förvirring kommer över mig, så att jag icke kan höra, förskräckelse fattar mig, så att jag icke kan se.
\par 4 Mitt hjärta är utom sig, jag kväljes av förfäran; skymningen, som jag längtade efter, vållar mig nu skräck.
\par 5 Man dukar bord, man breder ut täcken, man äter och dricker. Nej, stån upp, I furstar; smörjen edra sköldar!
\par 6 Ty så har Herren sagt till mig: "Gå och ställ ut en väktare; vad han får se, det må han förkunna.
\par 7 Och om han ser ett tåg, ryttare par efter par, ett tåg av åsnor, ett tåg av kameler, då må han giva akt, noga giva akt."
\par 8 Och denne ropade, såsom ett lejon ryter: "Herre, här står jag på vakt beständigt, dagen igenom, och jag förbliver här på min post natt efter natt.
\par 9 Och se, nu kommer här ett tåg av män, ryttare par efter par!" Och åter talade han och sade: "Fallet, fallet är Babel! Alla dess gudabeläten äro nedbrutna till jorden."
\par 10 O du mitt krossade, mitt söndertröskade folk, vad jag har hört av HERREN Sebaot, Israels Gud, det har jag förkunnat för eder.
\par 11 Utsaga om Duma. Man ropar till mig från Seir: "Väktare, vad lider natten? Väktare, vad lider natten?"
\par 12 Väktaren svarar: "Morgon har kommit, och likväl är det natt. Viljen I fråga mer, så frågen; kommen tillbaka igen."
\par 13 Utsaga över Arabien. Tagen natthärbärge i Arabiens vildmark, I karavaner från Dedan.
\par 14 Må man komma emot de törstande och giva dem vatten. Ja, inbyggarna i Temas land gå de flyktande till mötes med bröd.
\par 15 Ty de fly undan svärd, undan draget svärd, och undan spänd båge och undan krigets tunga.
\par 16 Ty så har Herren sagt till mig: Om ett år, såsom dagakarlen räknar året, skall all Kedars härlighet vara förgången,
\par 17 och föga skall då vara kvar av Kedars hjältars bågar, så många de äro. Ty så har HERREN, Israels Gud, talat.

\chapter{22}

\par 1 Utsaga om Synernas dal. Vad är då på färde, eftersom allt ditt folk stiger upp på taken?
\par 2 Du larmuppfyllda, du bullrande stad, du glada stad! Dina slagna hava icke blivit slagna med svärd, ej dödats i strid.
\par 3 Alla dina furstar hava samfällt flytt undan, utan bågskott blevo de fångar. Ja, så många som påträffades hos dig blevo allasammans fångar, huru långt bort de än flydde.
\par 4 Därför säger jag: Vänden blicken ifrån mig, jag måste gråta bitterligen; trugen icke på mig tröst för att dottern mitt folk har blivit förstörd.
\par 5 Ty en dag med förvirring, nedtrampning och bestörtning kommer från Herren, HERREN Sebaot, i Synernas dal, med nedbrutna murar och med rop upp mot berget.
\par 6 Elam hade fattat kogret, vagnskämpar och ryttare följde honom; Kir hade blottat skölden.
\par 7 Dina skönaste dalar voro fyllda med vagnar, och ryttarna hade fattat stånd vid porten.
\par 8 Juda blev blottat och låg utan skydd. Då skådade du bort efter vapnen i Skogshuset.
\par 9 Och I sågen att Davids stad hade många rämnor, och I samladen upp vattnet i Nedre dammen.
\par 10 Husen i Jerusalem räknaden I, och I bröten ned husen för att befästa muren.
\par 11 Och mellan de båda murarna gjorden I en behållare för vattnet från Gamla dammen. Men I skådaden icke upp till honom som hade verkat detta; till honom som för länge sedan hade bestämt det sågen I icke.
\par 12 Herren, HERREN Sebaot kallade eder på den dagen till gråt och klagan, till att raka edra huvuden och hölja eder i sorgdräkt.
\par 13 Men i stället hängåven I eder åt fröjd och glädje; I dödaden oxar och slaktaden får, I åten kött och drucken vin, I saden: "Låtom oss äta och dricka, ty i morgon måste vi dö."
\par 14 Därför ljuder från HERREN Sebaot denna uppenbarelse i mina öron: Sannerligen, denna eder missgärning skall icke bliva försonad, så länge I leven, säger Herren, HERREN Sebaot.
\par 15 Så sade Herren, HERREN Sebaot: Gå bort till honom där, förvaltaren, överhovmästaren Sebna, och säg till honom:
\par 16 Vad gör du här, och vem tänker du lägga här, eftersom du här hugger ut en grav åt dig? Du som hugger ut din grav så högt uppe, du som i klippan urholkar en boning åt dig,
\par 17 du må veta att HERREN skall slunga dig långt bort, en sådan man som du är. Han skall rulla dig tillhopa till en klump,
\par 18 han skall hopnysta dig såsom ett nystan, och kasta dig såsom en boll bort till ett land som har utrymme nog för dig; där skall du dö, och dit skola dina härliga vagnar komma, du skamfläck för din herres hus.
\par 19 Ja, jag skall stöta dig bort ifrån din plats, och från din tjänst skall du bliva avsatt.
\par 20 Och på den dagen skall jag kalla på min tjänare Eljakim, Hilkias son;
\par 21 honom skall jag ikläda din livklädnad, och med ditt bälte skall jag omgjorda honom, och skall lägga ditt välde i hans hand, så att han bliver en fader för Jerusalems invånare och för Juda hus.
\par 22 Och jag skall giva honom Davids hus' nyckel att bära; när han upplåter, skall ingen tillsluta, och när han tillsluter, skall ingen upplåta.
\par 23 Och jag skall slå in honom till en stadig spik i en fast vägg, och han skall bliva ett äresäte för sin faders hus.
\par 24 Men om då hans faders hus, så tungt det är, hänger sig på honom, med ättlingar och avkomlingar - alla slags småkärl av vad slag som helst, skålar eller allahanda krukor -
\par 25 då, på den dagen, säger HERREN Sebaot, skall spiken, som var inslagen i den fasta väggen lossna; den skall gå sönder och falla ned, och bördan som hängde därpå, skall krossas. Ty så har HERREN talat.

\chapter{23}

\par 1 Utsaga om Tyrus. Jämren eder, I Tarsis-skepp! Ty det är ödelagt, utan hus och utan gäster; från kittéernas land når dem budskapet härom.
\par 2 Sitten stumma, I kustlandets invånare! Köpmän från Sidon, sjöfarande män, uppfyllde dig;
\par 3 av Sihors säd och Nilflodens skördar skaffade du dig vinning, i det du for över stora vatten och drev handel därmed bland folken.
\par 4 Men stå där nu med skam, du Sidon; ty så säger havet, havets fäste: "Så är jag då utan avkomma och har icke fött några barn, icke uppfött ynglingar, icke fostrat jungfrur."
\par 5 När man får höra detta i Egypten, då bävar man vid ryktet om Tyrus.
\par 6 Dragen bort till Tarsis och jämren eder, I kustlandets invånare.
\par 7 Är detta eder glada stad, hon den urgamla, som av sina fötter bars till fjärran land, för att gästa där?
\par 8 Vem beslöt detta över Tyrus, henne som delade ut kronor, vilkens köpmän voro furstar, vilkens krämare voro stormän på jorden?
\par 9 HERREN Sebaot var den som beslöt det, för att slå ned all den stolta härligheten och ödmjuka alla stormän på jorden.
\par 10 Bred nu ut dig över ditt land såsom Nilfloden, du dotter Tarsis; du bär ingen boja mer.
\par 11 Han räckte ut sin hand över havet, han kom konungariken att darra; HERREN bjöd om Kanaans fästen att de skulle ödeläggas.
\par 12 Han sade: "Du skall ej allt framgent få leva i fröjd, du kränkta jungfru, du dotter Sidon. Stå upp och drag bort till kittéernas land; dock, ej heller där får du ro.
\par 13 Se, kaldéernas land, folket som förr ej var till, de vilkas land Assyrien gjorde till boning åt öknens djur, de resa där sina belägringstorn och omstörta stadens platser och göra den till en grushög.
\par 14 Jämren eder, I Tarsis-skepp, ty edert fäste är förstört."
\par 15 På den tiden skall Tyrus ligga förgätet i sjuttio år, såsom rådde där alltjämt en och samma konung; men efter sjuttio år skall det gå med Tyrus, såsom det heter i visan om skökan:
\par 16 "Tag din harpa och gå omkring i staden, du förgätna sköka; spela vackert och sjung flitigt, så att man kommer ihåg dig."
\par 17 Ty efter sjuttio år skall HERREN se till Tyrus, och det skall åter få begynna att taga emot skökolön och bedriva otukt med jordens alla konungariken i den vida världen.
\par 18 Men hennes handelsförvärv och vad hon får såsom skökolön skall vara helgat åt HERREN; det skall icke läggas upp och icke gömmas, utan de som bo inför HERRENS ansikte skola av hennes handelsförvärv hava mat till fyllest och präktiga kläder.

\chapter{24}

\par 1 Se, HERREN ödelägger jorden och föröder den; han omvälver, vad därpå är, och förströr dess inbyggare.
\par 2 Och det går prästen såsom folket, husbonden såsom tjänaren, husfrun såsom tjänarinnan, säljaren såsom köparen, låntagaren såsom långivaren, gäldenären såsom borgenären.
\par 3 Jorden bliver i grund ödelagd och i grund utplundrad; ty HERREN har talat detta ord.
\par 4 Jorden sörjer och tvinar bort, jordkretsen försmäktar och tvinar bort, vad högt är bland jordens folk försmäktar.
\par 5 Ty jorden har blivit ohelgad under sina inbyggare; de hava överträtt lagarna, de hava förvandlat rätten, brutit det eviga förbundet.
\par 6 Därför uppfräter förbannelse jorden, och de som bo där måste lida, vad de hava förskyllt; därför förtäras jordens inbyggare av hetta, så att ej många människor finnas kvar.
\par 7 Vinmusten sörjer, vinträdet försmäktar; de som voro så hjärteglada sucka nu alla.
\par 8 Det är förbi med fröjden vid pukornas ljud, de gladas larm ha tystnat; det är förbi med fröjden vid harpans klang.
\par 9 Vin dricker man icke mer under sång, rusdrycken kännes bitter för dem som dricka den.
\par 10 Nedbruten ligger den öde staden; vart hus är stängt, så att ingen kommer därin.
\par 11 Därute höres klagorop över vinet; all glädje är såsom en nedgången sol, all jordens fröjd har flyktat.
\par 12 Ödeläggelse allenast är kvar i staden, och porten är slagen i spillror.
\par 13 Ty det måste så gå på jorden bland folken, såsom det går, när man slår ned oliver, såsom när man gör en efterskörd, sedan vinbärgningen är slut.
\par 14 Dessa häva då upp sin röst och jubla; fröjderop över HERRENS höghet ljuda borta i väster:
\par 15 "Ären därför HERREN i österns bygder, även i havsländerna HERRENS, Israels Guds, namn."
\par 16 Från jordens ända höra vi lovsånger: "En härlig lott får den rättfärdige!" Men jag säger: Jag arme, jag arme, ve mig! Härjare härja, ja härjande fara härjare fram.
\par 17 Faror, fallgropar och fällor vänta eder, I jordens inbyggare.
\par 18 Och om någon flyr undan farlighetsropen, så störtar han i fallgropen, och om han kommer upp ur fallgropen, så fångas han i fällan. Ty fönstren i höjden äro öppnade, och jordens grundvalar bäva.
\par 19 Jorden brister, ja, den brister; jorden rämnar, ja, den rämnar; jorden vacklar, ja, den vacklar;
\par 20 jorden raglar, ja, den raglar såsom en drucken; den gungar såsom vaktskjulet i trädets topp. Dess överträdelse vilar tung på den, och den faller och kan icke mer stå upp.
\par 21 På den tiden skall HERREN hemsöka höjdens här uppe i höjden och jordens konungar nere på jorden.
\par 22 Och de skola samlas tillhopa, såsom fångar hopsamlas i fånggropen, och skola inneslutas i fängelse; sent omsider når dem hemsökelsen.
\par 23 Då skall månen blygas och solen skämmas; ty HERREN Sebaot skall då vara konung på Sions berg och i Jerusalem, och hans äldste skola skåda härlighet.

\chapter{25}

\par 1 HERRE, du är min Gud; jag vill upphöja dig, jag vill prisa ditt namn, ty du gör underfulla ting, du utför rådslut ifrån fordom tid, fasta och beståndande.
\par 2 Ja, du har gjort staden till en stenhop, den befästa staden till en grushög; främlingarnas palats står ej mer där såsom en stad, aldrig skall det byggas upp igen.
\par 3 Därför måste nu det vilda folket ära dig, den grymma hednastaden frukta dig.
\par 4 Ty du har varit ett värn för den arme, ett värn för den fattige i hans nöd, en tillflykt mot störtskurar, ett skygd under hettan. Ty våldsverkarnas raseri är likasom en störtskur mot en vägg.
\par 5 Och såsom du kuvar hettan, när det är som torrast, så kuvar du främlingarnas larm; ja, såsom hettan dämpas genom molnens skugga, så dämpas de grymmas segersång.
\par 6 Och HERREN Sebaot skall på detta berg göra ett gästabud för alla folk, ett gästabud med feta rätter, ett gästabud med starkt vin, ja, med feta, märgfulla rätter, med starkt vin, väl klarat.
\par 7 Och han skall på detta berg göra om intet det dok som höljer alla folk, och det täckelse som betäcker alla folkslag.
\par 8 Han skall för alltid göra döden om intet; och Herren, HERREN skall avtorka tårarna från alla ansikten, och skall taga bort sitt folks smälek överallt på jorden. Ty så har HERREN talat.
\par 9 På den tiden skall man säga: Se, där är vår Gud, som vi förbidade och som skulle frälsa oss. Ja, där är HERREN, som vi förbidade; låtom oss fröjdas och vara glada över hans frälsning.
\par 10 Ty HERRENS hand skall vila över detta berg, men Moab skall bliva nedtrampad i sitt eget land, likasom strå trampas ned i gödselpölen.
\par 11 Och huru han än där breder ut sina händer, lik simmaren, när han simmar, så skall dock hans högmod bliva nedbrutet, trots hans händers alla konster.
\par 12 Ja, dina murars höga fäste störtar han omkull och ödmjukar, han slår det till jorden, ned i stoftet.

\chapter{26}

\par 1 På den tiden skall man sjunga denna sång i Juda land: "Vår stad giver oss styrka; murar och värn bereda oss frälsning.
\par 2 Låten upp portarna, så att ett rättfärdigt folk får draga därin, ett som håller tro.
\par 3 Den som är fast i sitt sinne bevarar du i frid, i frid; ty på dig förtröstar han.
\par 4 Förtrösten då på HERREN till evig tid; ty HERREN, HERREN är en evig klippa.
\par 5 Ty dem som trona i höjden, dem störtar han ned, ja, den höga staden; han ödmjukar den, ödmjukar den till jorden, han slår den ned i stoftet.
\par 6 Den trampas under fötterna, under de förtrycktas fötter, under de armas steg."
\par 7 Men den rättfärdiges väg är jämn; åt den rättfärdige bereder du en jämnad stig.
\par 8 Ja, på dina domars väg, HERRE, förbida vi dig; till ditt namn och ditt pris står vår själs trängtan.
\par 9 Min själ trängtar efter dig om natten, och anden i mig söker dig bittida; ty när dina domar drabbar jorden, lära sig jordkretsens inbyggare rättfärdighet.
\par 10 Om nåd bevisas mot den ogudaktige, så lär han sig icke rättfärdighet; i det land, där rätt skulle övas, gör han då vad orätt är och ser icke HERRENS höghet.
\par 11 HERRE, din hand är upplyft, men de se det icke; må de nu med blygsel se din nitälskan för folket; ja, må eld förtära dina ovänner.
\par 12 HERRE, du skall skaffa frid åt oss, ty allt vad vi hava uträttat har du utfört åt oss.
\par 13 HERREN, vår Gud, andra herrar än du hava härskat över oss, men allenast dig prisa vi, allenast ditt namn.
\par 14 De döda få icke liv igen, skuggorna stå ej åter upp; därför hemsökte och förgjorde du dem och utrotade all deras åminnelse.
\par 15 Du förökade en gång folket, HERRE; du förökade folket och bevisade dig härlig; du utvidgade landets alla gränser.
\par 16 HERRE, i nöden hava de nu sökt dig, de hava utgjutit tysta böner, när din tuktan kom över dem.
\par 17 Såsom en havande kvinna, då hon är nära att föda, våndas och ropar i sina kval, så var det med oss inför ditt ansikte, o HERRE.
\par 18 Vi voro också havande och våndades; men när vi födde, var det vind. Vi kunde icke bereda frälsning åt landet; inga människor födas mer till att bo på jordens krets.
\par 19 Men dina döda må få liv igen; mina dödas kroppar må åter stå upp. Vaknen upp och jublen, I som liggen i graven; ty din dagg är en ljusets dagg, och jorden skall giva igen de avsomnade.
\par 20 Välan då, mitt folk, gå in i dina kamrar och stäng igen dörrarna om dig; göm dig ett litet ögonblick, till dess att vreden har gått förbi.
\par 21 Ty se, HERREN träder ut ur sin boning, för att hemsöka jordens inbyggare för deras missgärning; och jorden skall låta komma i dagen allt blod som där har blivit utgjutet, och skall icke längre betäcka dem som där hava blivit dräpta.

\chapter{27}

\par 1 På den tiden skall HERREN med sitt svärd, det hårda, det stora och starka, hemsöka Leviatan, den snabba ormen, och Leviatan, den ringlande ormen, och skall dräpa draken, som ligger i havet.
\par 2 På den tiden skall finnas en vingård, rik på vin, och man skall sjunga om den:
\par 3 Jag, HERREN, är dess vaktar, åter och åter vattnar jag den. För att ingen skall skada den, vaktar jag den natt och dag.
\par 4 Jag vredgas icke på den; nej, om tistel och törne ville begynna strid, så skulle jag gå löst därpå och bränna upp alltsammans.
\par 5 Eller ock måste man söka skydd hos mig och göra fred med mig; ja, fred måste man göra med mig.
\par 6 I tider som komma skall Jakob skjuta rötter och Israel grönska och blomstra; jordkretsen skola de uppfylla med sin frukt.
\par 7 Har man väl plågat dem så, som han plågade deras plågare? Eller dräptes de så, som deras dräpta fiender blevo dräpta?
\par 8 Nej, väl näpste han folket, när han förkastade och försköt det, väl ryckte han bort det med sin hårda vind, på östanstormens dag;
\par 9 men därför kan ock Jakobs missgärning då bliva försonad och deras synds borttagande då giva fullmogen frukt, när alla stenar i deras altaren äro förstörda - såsom då man krossar sönder kalkstycken - och när Aseror och solstoder ej mer resas upp.
\par 10 Se, den fasta staden ligger öde, den har blivit en folktom plats, övergiven såsom en öken, kalvar gå där i bet och lägra sig där och avbita de kvistar där finnas.
\par 11 Och när grenarna äro torra, bryter man av dem, och kvinnor komma och göra upp eld med dem. Ty detta är icke ett folk med förstånd; därför visar deras skapare dem intet förbarmande, och deras danare dem ingen misskund.
\par 12 Och det skall ske på den tiden att HERREN anställer en inbärgning, från den strida floden intill Egyptens bäck; och I skolen varda insamlade, en och en, I Israels barn.
\par 13 Och det skall ske på den tiden att man stöter i en stor basun; och de som hava varit borttappade i Assyriens land och fördrivna till Egyptens land, de skola då komma; och de skola tillbedja HERREN på det heliga berget i Jerusalem.

\chapter{28}

\par 1 Ve dig, du Efraims druckna mäns stolta krona, du hans strålande härlighets vissnande blomster på bergshjässan ovan de vinberusades bördiga dal!
\par 2 Se, från Herren kommer en som är stark och väldig, lik en hagelskur, en förödande storm, lik en störtskur med väldiga, översvämmande vatten, som slår allt till jorden med mat.
\par 3 Under fötterna bliver den då trampad, Efraims druckna mäns stolta krona.
\par 4 Och det går med hans strålande härlighets vissnande blomma på bergshjässan ovan den bördiga dalen, såsom det går med ett fikon därnere, ett som har mognat före sommarskörden: så snart någon får syn därpå, slukar han det, medan han ännu har det i sin hand.
\par 5 På den tiden skall HERREN Sebaot bliva en härlig krona och en strålande krans för kvarlevan av sitt folk;
\par 6 och han skall bliva en rättens ande för den som skipar rätt, och en starkhetsmakt för dem som driva fienden på porten.
\par 7 Men också här raglar man av vin, stapplar man av starka drycker; både präster och profeter ragla av starka drycker, de äro överlastade av vin, de stappla av starka drycker; de ragla, när de profetera, de vackla, när de skipa rätt.
\par 8 Ja, alla bord äro fulla av vämjeliga spyor, ingen ren fläck finnes.
\par 9 - "Vem är det då han vill lära förstånd, och vem skall han få att giva akt på sin predikan? Äro vi då nyss avvanda från modersmjölken, nyss tagna från modersbröstet?
\par 10 Det är ju gnat på gnat, gnat på gnat, prat på prat, prat på prat, litet här, litet där!"
\par 11 - Ja väl, genom stammande läppar och på ett främmande tungomål skall han tala till detta folk,
\par 12 han som en gång sade till dem: "Här är vilostaden, låten den trötte få vila; här är vederkvickelsens ort." Men sådant ville de icke höra.
\par 13 Så skall då HERRENS ord bliva för dem "gnat på gnat, gnat på gnat, prat på prat, prat på prat, litet här, litet där". Och så skola de, bäst de gå där, falla baklänges och krossas, varda snärjda och fångade.
\par 14 Hören därför HERRENS ord, I bespottare, I som råden över folket här i Jerusalem.
\par 15 Eftersom I sägen: "Vi hava slutit ett förbund med döden, med dödsriket hava vi ingått ett fördrag; om ock gisslet far fram likt en översvämmande flod, skall det icke nå oss, ty vi hava gjort lögnen till vår tillflykt och falskheten till vårt beskärm",
\par 16 därför säger Herren, HERREN så: Se, jag har lagt i Sion en grundsten, en beprövad sten, en dyrbar hörnsten, fast grundad; den som tror på den behöver icke fly.
\par 17 Och jag skall låta rätten vara mätsnöret och rättfärdigheten sänklodet. Och hagel skall slå ned eder lögntillflykt, och vatten skall skölja bort edert beskärm.
\par 18 Och edert förbund med döden skall bliva utplånat, och edert fördrag med dödsriket skall icke bestå; när gisslet far fram likt en översvämmande flod, då solen I varda nedtrampade.
\par 19 Så ofta det far fram, skall det träffa eder; ty morgon efter morgon skall det fara fram, ja, både dag och natt. Idel förskräckelse bliver det, när I måsten akta på den predikan.
\par 20 Ty sängen bliver då för kort att sträcka ut sig på och täcket för knappt att svepa in sig i.
\par 21 Ty HERREN skall stå upp likasom på Perasims berg, och han skall låta se sin vrede likasom i Gibeons dal. Han skall utföra sitt verk, ett sällsamt verk; han skall förrätta sitt arbete, ett förunderligt arbete.
\par 22 Så hören nu upp med eder bespottelse, för att edra band ej må bliva än hårdare; ty om förstöring och oryggligt besluten straffdom över hela jorden har jag hört från Herren, HERREN Sebaot.
\par 23 Lyssnen och hören min röst, akten härpå och hören mitt tal.
\par 24 När åkermannen vill så, plöjer han då beständigt och hackar upp och harvar sin mark?
\par 25 Nej, fastmer: sedan han har jämnat fältet, strör han ju där svartkummin och kastar dit kryddkummin och sår vete i rader och korn på dess särskilda plats och spält i kanten.
\par 26 Ty hans Gud har undervisat honom och lärt honom det rätta sättet.
\par 27 Man tröskar ju ej heller svartkummin med tröskvagn och låter ej vagnshjul gå över kryddkummin, utan klappar ut svartkummin med stav och kryddkummin med käpp.
\par 28 Och brödsäden, tröskar man sönder den? Nej, man plägar icke oavlåtligt tröska den och driva sina vagnshjul och hästar däröver; man vill ju icke tröska sönder den.
\par 29 Också detta kommer från HERREN Sebaot; han är underbar i råd och stor i vishet.

\chapter{29}

\par 1 Ve dig, Ariel, Ariel, du stad, där David slog upp sitt läger! Läggen år till år och låten högtiderna fullborda sitt kretslopp,
\par 2 så skall jag bringa Ariel i trångmål; jämmer skall följa på jämmer, och då bliver det för mig ett verkligt Ariel.
\par 3 Jag skall slå läger runt omkring dig och omsluta dig med vallar och resa upp bålverk emot dig.
\par 4 Då skall du tala djupt nedifrån jorden, och dina ord skola dämpade komma fram ur stoftet; din röst skall höras såsom en andes ur jorden, och ur stoftet skall du viska fram dina ord.
\par 5 Men främlingshopen skall bliva såsom fint damm och våldsverkarhopen såsom bortflyende agnar; plötsligt och med hast skall detta ske.
\par 6 Från HERREN Sebaot skall hemsökelsen komma, med tordön och jordbävning och stort dunder, med storm och oväder och med lågor av förtärande eld.
\par 7 Och hela hopen av alla de folk som drogo i strid mot Ariel, de skola vara såsom en drömsyn om natten, alla som drogo i strid mot det och dess borg och bragte det i trångmål.
\par 8 Såsom när den hungrige drömmer att han äter, men vaknar och känner sin buk vara tom, och såsom när den törstande drömmer att han dricker, men vaknar och känner sig törstig och försmäktande, så skall det gå med hela hopen av alla de folk som drogo i strid mot Sions berg.
\par 9 Stån där med häpnad, ja, varen häpna; stirren eder blinda, ja, varen blinda, I som ären druckna, men icke av vin, I som raglen, men icke av starka drycker.
\par 10 Ty HERREN har utgjutit över eder en tung sömns ande och har tillslutit edra ögon; han har höljt mörker över profeterna och över siarna, edra ledare.
\par 11 Och så har profetsynen om allt detta blivit för eder lik orden i en förseglad bok: om man räcker en sådan åt någon som kan läsa och säger: "Läs detta", så svarar han: "Jag kan det icke, den är ju förseglad",
\par 12 och om man räcker den åt någon som icke kan läsa och säger: "Läs detta", så svarar han: "Jag kan icke läsa."
\par 13 Och HERREN har sagt: Eftersom detta folk nalkas mig med sin mun och ärar mig med sina läppar, men låter sitt hjärta vara långt ifrån mig, så att deras fruktan för mig består i inlärda människobud,
\par 14 därför skall jag ännu en gång göra underbara ting mot detta folk, ja, underbara och förunderliga; de visas vishet skall förgås, och de förståndigas förstånd skall bliva förmörkat.
\par 15 Ve eder, I som söken att dölja edra rådslag för HERREN i djupet, och som bedriven edra verk i mörkret, I som sägen: "Vem ser oss, och vem känner oss?"
\par 16 Huru förvända ären I icke! Skall då leret aktas lika med krukmakaren? Skall verket säga om sin mästare: "Han har icke gjort mig"? Eller skall bilden säga om honom som har format den: "Han förstår intet"?
\par 17 Se, ännu allenast en liten tid, och Libanon skall förvandlas till ett bördigt fält och det bördiga fältet räknas såsom vildmark.
\par 18 De döva skola på den tiden höra vad som läses för de, och de blindas ögon skola se och vara fria ifrån dunkel och mörker;
\par 19 och de ödmjuka skola då känna allt större glädje i HERREN; och de fattigaste bland människor skola fröjda sig i Israels Helige.
\par 20 Ty våldsverkarna äro då icke mer till, bespottarna hava fått en ände, och de som stodo efter fördärv äro alla utrotade,
\par 21 de som genom sitt tal gjorde att oskyldiga blevo fällda och snärjde den som skulle skipa rätt i porten och genom lögn vrängde rätten för den rättfärdige.
\par 22 Därför säger HERREN så till Jakobs hus, han som förlossade Abraham: Nu skall Jakob icke mer behöva blygas, nu skall hans ansikte ej vidare blekna;
\par 23 ty när han - hans barn - få se mina händer verk ibland sig, då skola de hålla mitt namn heligt, de skola hålla Jakobs Helige helig och förskräckas för Israels Gud.
\par 24 De förvillade skola då få förstånd, och de knorrande skola taga emot lärdom.

\chapter{30}

\par 1 Ve eder, I vanartiga barn, säger HERREN, I som gören upp rådslag som icke komma från mig, och sluten förbund, utan att min Ande är med, så att I därigenom hopen synd på synd,
\par 2 I som dragen ned till Egypten, utan att hava rådfrågat min mun, för att söka eder ett värn hos Farao och en tillflykt under Egyptens skugga!
\par 3 Se, Faraos värn skall bliva eder till skam, och tillflykten under Egyptens skugga skall bliva eder till blygd.
\par 4 Ty om ock hans furstar äro i Soan, och om än hans sändebud komma ända till Hanes,
\par 5 så skall dock var man få blygas över detta folk, som icke kan hjälpa dem, icke vara till bistånd och hjälp, utan allenast till skam och smälek.
\par 6 Utsaga om Söderlandets odjur. Genom ett farornas och ångestens land, där lejoninnor och lejon hava sitt tillhåll, jämte huggormar och flygande drakar, där föra de på åsnors ryggar sina rikedomar och på kamelers pucklar sina skatter till ett folk som icke kan hjälpa dem.
\par 7 Ty Egyptens bistånd är fåfänglighet och tomhet; därför kallar jag det landet "Rahab, som ingenting uträttar".
\par 8 Så gå nu in och skriv detta på en tavla, som må förvaras bland dem, och teckna upp det i en bok, så att det bevaras för kommande dagar, alltid och evinnerligen.
\par 9 Ty det är ett gensträvigt folk, trolösa barn, barn som icke vilja höra HERRENS lag,
\par 10 utan säga till siarna: "Upphören med edra syner", och till profeterna: "Profeteren icke för oss vad sant är; talen till oss sådant som är oss välbehagligt, profeteren bedrägliga ting;
\par 11 viken av ifrån vägen, gån åt sidan från stigen, skaffen bort ur vår åsyn Israel Helige."
\par 12 Därför säger Israels Helige så: "Eftersom I förakten detta ord och förtrösten på våld och vrånghet och stödjen eder på sådant,
\par 13 därför skall denna missgärning bliva för eder såsom ett fallfärdigt stycke på en hög mur, vilket mer och mer giver sig ut, till dess att muren plötsligt och med hast störtar ned och krossas;
\par 14 den krossa, såsom när man våldsamt slår en lerkruka i bitar, så våldsamt att man bland bitarna icke kan finna en skärva stor nog att därmed taga eld från eldstaden eller ösa upp vatten ur dammen."
\par 15 Ty så säger Herren, HERREN, Israels Helige: "Om i vänden om och ären stilla, skolen I bliva frälsta, genom stillhet och förtröstan varden I starka." Men i viljen icke.
\par 16 I sägen: "Nej, på hästar vilja vi jaga fram" - därför skolen I också bliva jagade; "på snabba springare vilja vi rida åstad" - därför skola ock edra förföljare vara snabba.
\par 17 Tusen av eder skola fly för en enda mans hot eller för fem mäns hot, till dess att vad som är kvar av eder bliver såsom en ensam stång på bergets topp, såsom ett baner på höjden.
\par 18 Ja, därför väntar HERREN, till dess att han kan vara eder nådig; därför tronar han i höghet, till dess att han kan förbarma sig över eder. Ty en domens Gud är HERREN; saliga äro alla de som vänta efter honom.
\par 19 Ja, du folk som bor på Sion, i Jerusalem, ingalunda må du gråta. Han skall förvisso vara dig nådig, när du ropar; så snart han hör din röst, skall han svara dig.
\par 20 Ty väl skall Herren giva eder nödens bröd och fångenskapens dryck, men sedan skola dina lärare icke mer sättas å sido, utan dina ögon skola se upp till dina lärare.
\par 21 Och om du viker av, vare sig åt höger eller åt vänster, så skola dina öron höra detta ord ljuda bakom dig; "Här är vägen, vandren på en."
\par 22 Då skolen I akta för orent silvret varmed edra skurna beläten äro överdragna, och guldet varmed edra gjutna beläten äro belagda; du skall kasta ut det såsom orenlighet och säga till det: "Bort härifrån!"
\par 23 Och han skall giva regn åt säden som du har sått på din mark, och han skall av markens gröda giva dig bröd som är kraftigt och närande; och din boskap skall på den tiden gå i bet på vida ängar.
\par 24 Och oxarna och åsnorna med vilka man brukar jorden, de skola äta saltad blandsäd om man har kastat med vanna och kastskovel.
\par 25 Och på alla höga berg och alla stora höjder skola bäckar rinna upp med strömmande vatten - när den stora slaktningens dag kommer, då torn skola falla.
\par 26 Och månens ljus skall bliva såsom solens ljus, och solens ljus skall varda sju gånger klarare, såsom ett sjufaldigt dagsljus, när den tid kommer, då HERREN förbinder sitt folks skador och helar såren efter slagen som det har fått.
\par 27 Se, HERRENS namn kommer fjärran ifrån, med brinnande vrede och med tunga rökmoln; hans läppar äro fulla av förgrymmelse, och hans tunga är såsom förtärande eld;
\par 28 hans andedräkt är lik en ström som svämmar över, så att den når ändå upp till halsen. Ty han vill sålla folken i förintelsens såll och lägga i folkslagens mun ett betsel, till att leda dem vilse.
\par 29 Då skolen I sjunga såsom i en natt då man firar helig högtid, och edra hjärtan skola glädja sig, såsom när man under flöjters ljud tågar upp på HERRENS berg, upp till Israels klippa.
\par 30 Och HERREN skall låta höra sin röst i majestät och visa huru hans arm drabbar, i vredesförgrymmelse och med förtärande eldslåga, med storm och störtskurar och hagelstenar.
\par 31 Ty för HERRENS röst skall Assur bliva förfärad, när han slår honom med sitt ris.
\par 32 Och så ofta staven far fram och HERREN efter sitt rådslut låter den falla på honom skola pukor och harpor ljuda. Gång på gång skall han lyfta sin arm till att strida mot honom.
\par 33 Ty en Tofetplats är längesedan tillredd, ja ock för konungen är den gjord redo, och djup och vid är den; dess rund är fylld av eld och av ved i myckenhet, och lik en svavelström skall HERRENS Ande sätta den i brand.

\chapter{31}

\par 1 Ve dem som draga åstad ned till Egypten för att få hjälp, i det de förlita sig på hästar, dem som sätta sin förtröstan på vagnar, därför att där finnas så många, och på ryttare, därför att mängden är så stor, men som ej vända sin blick till Israels Helige och icke fråga efter HERREN!
\par 2 Också han är ju vis; han låter olyckan komma, och han ryggar icke sina ord. Han reser sig upp mot de ondas hus och mot den hjälp som ogärningsmännen sända.
\par 3 Ty egyptierna äro människor och äro icke Gud, deras hästar äro kött och icke ande. Och HERREN skall räcka ut sin hand, och då skall hjälparen vackla och den hjälpte falla, och båda skola tillhopa förgås.
\par 4 Ty så har HERREN sagt till mig: Såsom ett lejon ryter, ett ungt lejon över sitt rov, och icke skrämmes bort av herdarnas rop eller rädes för deras larm, när de i mängd samlas dit, så skall HERREN Sebaot fara ned för att strida på Sions berg och uppe på dess höjd.
\par 5 Såsom fågeln breder ut sina vingar, så skall HERREN Sebaot beskärma Jerusalem; han skall beskärma och hjälpa, han skall skona och rädda.
\par 6 Så vänden nu om till honom, från vilken I haven avfallit genom ett så djupt fall, I Israels barn.
\par 7 Ty på den tiden skall var och en av eder kasta bort de avgudar av silver och de avgudar av guld, som edra händer hava gjort åt eder till synd.
\par 8 Och Assur skall falla, men icke för en mans svärd; ett svärd, som icke är en människas, skall förtära honom. Han skall fly för svärd, och hans unga män skola bliva trälar.
\par 9 Och hans klippa skall förgås av skräck, och hans furstar skola i förfäran fly ifrån baneret. Så säger HERREN, han som har sin eld på Sion och sin ugn i Jerusalem.

\chapter{32}

\par 1 En konung skall uppstå, som skall regera med rättfärdighet, och härskare, som skola härska med rättvisa.
\par 2 Var och en av dem skall vara såsom en tillflykt i stormen, ett skydd mot störtskuren; de skola vara såsom vattenbäckar i en ödemark, såsom skuggan av en väldig klippa i ett törstigt land.
\par 3 Då skola de seendes ögon icke vara förblindade, och de hörandes öron skola lyssna till.
\par 4 Då skola de lättsinnigas hjärtan bliva förståndiga och vinna kunskap, och de stammandes tungor skola tala flytande och tydligt.
\par 5 Dåren skall då icke mer heta ädling, ej heller bedragaren kallas herre.
\par 6 Ty en dåre talar dårskap, och hans hjärta reder till fördärv; så övar han gudlöshet och talar, vad förvänt är, om HERREN, så låter han den hungrige svälta och nekar den törstige en dryck vatten.
\par 7 Och bedragaren brukar onda vapen, han tänker ut skändliga anslag till att fördärva de betryckta genom lögnaktiga ord, fördärva en fattig, som har rätt i sin talan.
\par 8 Men en ädling tänker ädla tankar och står fast vid det som ädelt är.
\par 9 I kvinnor, som ären så säkra, stån upp och hören min röst; I sorglösa jungfrur, lyssnen till mitt tal.
\par 10 När år och dagar hava gått, då skolen I darra, I som ären så sorglösa, ty då är det slut med all vinbärgning, och ingen fruktskörd kommer mer.
\par 11 Bäven, I som ären så säkra, darren, I som ären så sorglösa, läggen av edra kläder och blotten eder, kläden edra länder med säcktyg.
\par 12 Slån eder för bröstet och klagen över de sköna fälten, över de fruktsamma vinträden,
\par 13 över mitt folks åkrar som fyllas av törne och tistel, ja, över alla glädjens boningar i den yra staden.
\par 14 Ty palatsen äro övergivna, den folkrika staden ligger öde, Ofelhöjden med vakttornet är förvandlad till grotthålor för evig tid, till en plats, där vildåsnor hava sin fröjd och där hjordar beta -
\par 15 detta intill dess att ande från höjden bliver utgjuten över oss. Då skall öknen bliva ett bördigt fält och det bördiga fältet räknas såsom vildmark;
\par 16 då skall rätten taga sin boning i öknen och rättfärdigheten bo på det bördiga fältet.
\par 17 Och rättfärdighetens frukt skall vara frid och rättfärdighetens vinning vara ro med trygghet till evig tid.
\par 18 Och mitt folk skall bo i fridshyddor, i trygga boningar och på säkra viloplatser.
\par 19 Men under hagelskurar skall skogen fällas, och djupt skall staden bliva ödmjukad.
\par 20 Sälla ären då I som fån så vid alla vatten, I som kunnen låta edra oxar och åsnor fritt ströva omkring.

\chapter{33}

\par 1 Ve dig, du fördärvare, som själv har gått fri ifrån fördärvet! Ve dig, du härjare, som själv har undgått förhärjning! När du har fyllt ditt mått att fördärva, drabbas du själv av fördärvet; när du har fullbordat till härjande drabbas du själv av förhärjning.
\par 2 HERRE, var oss nådig, dig förbida vi. Var dessas arm var morgon; ja, var vår frälsning i nödens tid.
\par 3 För ditt väldiga dån fly folken bort; när du reser dig upp, förskingras folkslagen.
\par 4 Och man får skövla och taga byte efter eder, såsom gräsmaskar skövla; såsom gräshoppor störta fram, så störtar man över det.
\par 5 HERREN är hög, ty han bor i höjden; han uppfyller Sion med rätt och rättfärdighet.
\par 6 Ja, trygga tider skola komma för dig! Vishet och kunskap bereda Sion frälsning i rikt mått, och HERRENS fruktan skall vara deras skatt.
\par 7 Hör, deras hjältar klaga därute, fredsbudbärarna gråta bitterligen.
\par 8 Vägarna äro öde, ingen går mer på stigarna. Han bryter förbund, han aktar städer ringa, människor räknar han för intet.
\par 9 Landet ligger sörjande och försmäktar, Libanon blyges och står förvissnat, Saron har blivit likt en hedmark, Basans och Karmels skogar fälla sina löv.
\par 10 Men nu vill jag stå upp, säger HERREN, nu vill jag resa mig upp, nu vill jag upphäva mig.
\par 11 Med halm gån I havande, och strå föden I; edert raseri är en eld, som skall förtära eder själva.
\par 12 Folken skola förbrännas och bliva till aska, ja, likna avhugget törne, som brinner upp i eld.
\par 13 Så hören nu, I som fjärran ärer, vad jag har gjort; förnimmen min makt, I som nära ären.
\par 14 Syndarna i Sion bliva förskräckta, bävan griper de gudlösa. "Vem av oss kan härda ut vid en förtärande eld, vem av oss kan bo vid en evig glöd?"
\par 15 Den som vandrar i rättfärdighet och talar, vad rätt är, den som föraktar, vad som vinnes genom orätt och våld, och den som avhåller sina händer från att taga mutor, den som tillstoppar sina öron för att icke höra om blodsgärningar och tillsluter sina ögon för att icke se, vad ont är,
\par 16 han skall bo på höjderna, klippfästen skola vara hans värn, sitt bröd skall han få, och vatten skall han hava beständigt.
\par 17 Ja, dina ögon skola skåda en konung i hans härlighet, de skola blicka ut över ett vidsträckt land.
\par 18 Då skall ditt hjärta tänka tillbaka på förskräckelsens tid: "Var är nu skatteräknaren, var är nu skattevägaren, var är den som räknade tornen?"
\par 19 Du slipper då att se det fräcka folket, folket, vars obegripliga språk man ej kunde förstå, vars stammande tungomål ingen kunde tyda.
\par 20 Men skåda på Sion, våra högtiders stad, låt dina ögon betrakta Jerusalem: det är en säker boning, ett tält, som icke flyttas bort, ett vars pluggar aldrig ryckas upp och av vars streck intet enda brister sönder.
\par 21 Ja, vi hava där HERREN, den väldige; han är för oss såsom floder och breda strömmar; ingen roddflotta kommer där fram, och det väldigaste skepp kan ej fara däröver.
\par 22 Ty HERREN är vår domare, HERREN är vår härskare, HERREN är vår konung, han frälsar oss.
\par 23 Dina tåg hänga slappa, de hålla ej masten stadig, ej seglet spänt. Men då skall rövat gods utskiftas i myckenhet, ja, också de lama skola då taga byte.
\par 24 Och ingen av invånarna skall säga: "Jag är svag", ty folket, som där bor, har fått sin missgärning förlåten.

\chapter{34}

\par 1 Träden fram, I folk, och hören; I folkslag, akten härpå. Jorden höre och allt vad på den är, jordens krets och vad som alstras därav.
\par 2 Ty HERREN är förtörnad på alla folk och vred på all deras här; han giver dem till spillo, han överlämnar dem till att slaktas.
\par 3 Deras slagna kämpar ligga bortkastade, och stank stiger upp från deras döda kroppar, och bergen flyta av deras blod.
\par 4 Himmelens hela härskara förgås, och himmelen själv hoprullas såsom en bokrulle; hela dess härskara faller förvissnad ned, lik vissnade löv från vinrankan, lik vissnade blad ifrån fikonträdet.
\par 5 Ty mitt svärd har druckit sig rusigt i himmelen; se, det far ned på Edom till dom, på det folk jag har givit till spillo.
\par 6 Ja, ett svärd har HERREN, det dryper av blod och är dränkt i fett, i lamms och bockars blod, i fett ifrån vädurars njurar; ty HERREN anställer ett offer i Bosra, ett stort slaktande i Edoms land.
\par 7 Vildoxar fällas ock därvid, tjurar, både små och stora. Deras land dricker sig rusigt av blod, och deras jord bliver dränkt i fett.
\par 8 Ty detta är en HERRENS hämndedag, ett vedergällningens år, då han utför Sions sak.
\par 9 Då bliva Edoms bäckar förvandlade till tjära och dess jord till svavel; ja, dess land bliver förbytt i brinnande tjära.
\par 10 Varken natt eller dag skall den branden slockna, evinnerligen skall röken därav stiga upp. Från släkte till släkte skall landet ligga öde, aldrig i evighet skall någon gå där fram.
\par 11 Pelikaner och rördrommar skola taga det i besittning, uvar och korpar skola bo däri; ty förödelsens mätsnöre och förstörelsens murlod skall han låta komma däröver.
\par 12 Av dess ädlingar skola inga finnas kvar där, som kunna utropa någon till konung; och alla dess furstar få en ände.
\par 13 Dess palatser fyllas av törne, nässlor och tistlar växa i dess fästen; och det bliver en boning för ökenhundar och ett tillhåll för strutsar.
\par 14 Schakaler bo där tillsammans med andra ökendjur, och gastar ropa där till varandra; ja, där kan Lilit få ro, där kan hon finna en vilostad.
\par 15 Där reder pilormen sitt bo och lägger sina ägg och kläcker så ut ynglet och samlar det i sitt skygd; ja, där komma gamarna tillhopa, den ene möter där den andre.
\par 16 Söken efter i HERRENS bok och läsen däri; icke ett enda av de djuren skall utebliva, det ena skall icke fåfängt söka det andra. Ty det är hans mun, som bjuder det, det är hans Ande, som samlar dem tillhopa.
\par 17 Det är han, som kastar lott för dem, hans hand tillskiftar dem deras mark efter mätsnöre; till evig tid skola de hava den till besittning, från släkte till släkte bo därpå.

\chapter{35}

\par 1 Öknen och ödemarken skola glädja sig, och hedmarken skall fröjdas och blomstra såsom en lilja.
\par 2 Den skall blomstra skönt och fröjda sig, ja, fröjda sig och jubla; Libanons härlighet skall bliva den given, Karmels och Sarons prakt. Ja, de skola få se HERRENS härlighet, vår Guds prakt.
\par 3 Stärken maktlösa händer, given kraft åt vacklande knän.
\par 4 Sägen till de försagda: "Varen frimodiga, frukten icke." Se, eder Gud kommer med hämnd; vedergällning kommer från Gud, ja, själv kommer han och frälsar eder.
\par 5 Då skola de blindas ögon öppnas och de dövas öron upplåtas.
\par 6 Då skall den lame hoppa såsom en hjort, och den stummes tunga skall jubla. Ty vatten skola bryta fram i öknen och strömmar på hedmarken.
\par 7 Av förbränt land skall bliva en sjö och av torr mark vattenkällor; på den plats, där ökenhundar lägrade sig, skall växa gräs jämte vass och rör.
\par 8 Och en banad väg, en farväg, skall gå där fram, och den skall kallas "den heliga vägen"; ingen oren skall färdas därpå, den skall vara för dem själva. Den som vandrar den vägen skall icke gå vilse, om han ock hör till de fåkunniga.
\par 9 Där skall icke vara något lejon, ej heller skall något annat vilddjur komma dit. Intet sådant skall finnas där, men ett frälsat folk skall vandra på den.
\par 10 Ja, HERRENS förlossade skola vända tillbaka och komma till Sion med jubel; evig glädje skall kröna deras huvuden, fröjd och glädje skola de undfå, men sorg och suckan skola fly bort.

\chapter{36}

\par 1 Och i konung Hiskias fjortonde regeringsår hände sig, att Sanherib, konungen i Assyrien, drog upp och angrep alla befästa städer i Juda och intog dem.
\par 2 Och konungen i Assyrien sände från Lakis åstad Rab-Sake med en stor här till Jerusalem mot konung Hiskia; och han stannade vid Övre dammens vattenledning, på vägen till Valkarfältet.
\par 3 Då gingo överhovmästaren Eljakim, Hilkias son, och sekreteraren Sebna och kansleren Joa, Asafs son, ut till honom.
\par 4 Och Rab-Sake sade till dem: "Sägen till Hiskia: Så säger den store konungen, konungen i Assyrien: Vad är det för en förtröstan, som du nu har hängivit dig åt?
\par 5 Jag säger: Det är allenast munväder, att du vet råd och har makt att föra kriget. På vem förtröstar du då, eftersom du har satt dig upp mot mig?
\par 6 Du förtröstar väl på den bräckta rörstaven Egypten, men se, när någon stöder sig på den, går den i i hans hand och genomborrar den. Ty sådan är Farao, konungen i Egypten, för alla som förtrösta på honom.
\par 7 Eller säger du kanhända till mig: 'Vi förtrösta på HERREN, vår Gud?' Var det då icke hans offerhöjder och altaren Hiskia avskaffade, när han sade till Juda och Jerusalem: 'Inför detta altare skolen I tillbedja'?
\par 8 Men ingå nu ett vad med min herre, konungen i Assyrien: jag vill giva dig två tusen hästar, om du kan skaffa dig ryttare till dem.
\par 9 Huru skulle du då kunna slå tillbaka en enda ståthållare, en av min herres ringaste tjänare? Och du sätter din förtröstan till Egypten i hopp om att så få vagnar och ryttare!
\par 10 Menar du då att jag utan HERRENS vilja har dragit upp till detta land för att fördärva det? Nej, det är HERREN, som har sagt till mig: Drag upp mot detta land och fördärva det."
\par 11 Då sade Eljakim och Sebna och Joa till Rab-Sake: "Tala till dina tjänare på arameiska, ty vi förstå det språket, och tala icke till oss på judiska inför folket som står på muren."
\par 12 Men Rab-Sake svarade: "Är det då till din herre och till dig, som min herre har sänt mig att tala dessa ord? Är det icke fastmer till de män som sitta på muren och som jämte eder skola nödgas äta sin egen träck och dricka sitt eget vatten?"
\par 13 Därefter trädde Rab-Sake närmare och ropade med hög röst på judiska och sade: "Hören den store konungens, den assyriske konungens, ord.
\par 14 Så säger konungen: Låten icke Hiskia bedraga eder, ty han förmår icke rädda eder.
\par 15 Och låten icke Hiskia förleda eder att förtrösta på HERREN, därmed att han säger: 'HERREN skall förvisso rädda oss; denna stad skall icke bliva given i den assyriske konungens hand.'
\par 16 Hören icke på Hiskia. Ty så säger konungen i Assyrien: Gören upp i godo med mig och given eder åt mig, så skolen I få äta var och en av sitt vinträd och av sitt fikonträd och dricka var och en ur sin brunn,
\par 17 till dess jag kommer och hämtar eder till ett land som är likt edert eget land, ett land med säd och vin, ett land med bröd och vingårdar.
\par 18 Låten icke Hiskia förleda eder, när han säger: 'HERREN skall rädda oss.' Har väl någon av de andra folkens gudar räddat sitt land ur den assyriske konungens hand?
\par 19 Var äro Hamats och Arpads gudar? Var äro Sefarvaims gudar? Eller hava de räddat Samaria ur min hand?
\par 20 Vilken bland dessa länders alla gudar har väl räddat sitt land ur min hand, eftersom I menen, att HERREN skall rädda Jerusalem ur min hand?"
\par 21 Men de tego och svarade honom icke ett ord, ty konungen hade så bjudit och sagt: "Svaren honom icke."
\par 22 Och överhovmästaren Eljakim, Hilkias son, och sekreteraren Sebna och kansleren Joa, Asafs son, kommo till Hiskia med sönderrivna kläder och berättade för honom, vad Rab-Sake hade sagt.

\chapter{37}

\par 1 Då nu konung Hiskia hörde detta, rev han sönder sina kläder och höljde sig i sorgdräkt och gick in i HERRENS hus.
\par 2 Och överhovmästaren Eljakim och sekreteraren Sebna och de äldste bland prästerna sände han, höljda i sorgdräkt, till profeten Jesaja, Amos' son.
\par 3 Och de sade till denne: "Så säger Hiskia: En nödens, tuktans och smälekens dag är denne dag, ty fostren hava väl kommit fram till födseln, men kraft att föda finnes icke.
\par 4 Kanhända skall HERREN, din Gud, höra Rab-Sakes ord, med vilka hans herre, konungen i Assyrien, har sänt honom till att smäda den levande Guden, så att han straffar honom för dessa ord som han, HERREN, din Gud, har hört. Så bed nu en bön för den kvarleva som ännu finnes."
\par 5 När nu konung Hiskias tjänare kommo till Jesaja,
\par 6 sade Jesaja till dem: "Så skolen I säga till eder herre: Så säger HERREN: Frukta icke för de ord som du har hört, dem, med vilka den assyriske konungens tjänare hava hädat mig.
\par 7 Se, jag skall låta en sådan ande komma in i honom, att han på grund av ett rykte som han skall få höra vänder tillbaka till sitt land; och jag skall låta honom falla för svärd i hans eget land."
\par 8 Och Rab-Sake vände tillbaka och fann den assyriske konungen upptagen med att belägra Libna; ty han hade hört, att han hade brutit upp från Lakis.
\par 9 Men när Sanherib fick höra sägas om Tirhaka, konungen i Etiopien, att denne hade dragit ut för att strida mot honom, skickade han, så snart han hörde detta, sändebud till Hiskia och sade:
\par 10 "Så skolen I säga till Hiskia, Juda konung: Låt icke din Gud, som du förtröstar på, bedraga dig, i det att du tänker: 'Jerusalem skall icke bliva givet i den assyriske konungens hand.'
\par 11 Du har ju hört, vad konungarna i Assyrien hava gjort med alla andra länder, huru de hava givit dem till spillo. Och du skulle nu bliva räddad!
\par 12 Hava väl de folk, som mina fäder fördärvade, Gosan, Haran, Resef och Edens barn i Telassar, blivit räddade av sina gudar?
\par 13 Var är Hamats konung och Arpads konung och konungen över Sefarvaims stad, över Hena och Iva?"
\par 14 När Hiskia hade mottagit brevet av sändebuden och läst det, gick han upp i HERRENS hus, och där bredde Hiskia ut det inför HERRENS ansikte.
\par 15 Och Hiskia bad till HERREN och sade:
\par 16 "HERRE Sebaot, Israels Gud, du som tronar på keruberna, du allena är Gud, den som råder över alla riken på jorden; du har gjort himmel och jord.
\par 17 HERRE, böj ditt öra härtill och hör; HERRE, öppna dina ögon och se. Ja, hör alla Sanheribs ord, det budskap, varmed han har smädat den levande Guden.
\par 18 Det är sant, HERRE, att konungarna i Assyrien hava förött alla länder såsom ock sitt eget land.
\par 19 Och de hava kastat deras gudar i elden; ty dessa voro inga gudar, utan verk av människohänder, trä och sten; därför kunde de förgöra dem.
\par 20 Men fräls oss nu, HERRE, vår Gud, ur hans hand, så att alla riken på jorden förnimma, att du, HERRE, är den ende."
\par 21 Då sände Jesaja, Amos' son, bud till Hiskia och lät säga: "Så säger HERREN, Israels Gud, jag, till vilken du har bett angående Sanherib, konungen i Assyrien:
\par 22 Detta är det ord, som HERREN har talat om honom: Hon föraktar dig och bespottar dig, jungfrun dottern Sion; hon skakar huvudet efter dig, dottern Jerusalem.
\par 23 Vem har du smädat och hädat, och mot vem har du upphävt din röst? Alltför högt har du upplyft dina ögon - ja, mot Israels Helige.
\par 24 Genom dina tjänare smädade du HERREN, när du sade: 'Med mina många vagnar drog jag upp på bergens höjder, längst upp på Libanon; jag högg ned dess höga cedrar och väldiga cypresser; jag trängde fram till dess översta höjder, dess frodigaste skog;
\par 25 jag grävde brunnar och drack ut vatten, och med min fot uttorkade jag alla Egyptens strömmar.'
\par 26 Har du icke hört, att jag för länge sedan beredde detta? Av ålder bestämde jag ju så; och nu har jag fört det fram: du fick makt att ödelägga befästa städer till grusade stenhopar.
\par 27 Deras invånare blevo maktlösa, de förfärades och stodo med skam. Det gick dem såsom gräset på marken och gröna örter, såsom det som växer på taken, och säd, som förgås, förrän strået har vuxit upp.
\par 28 Om du sitter eller går ut eller går in, så vet jag det, och huru du rasar mot mig.
\par 29 Men då du nu så rasar mot mig och då ditt övermod har nått till mina öron, skall jag sätta min krok i din näsa och mitt betsel i din mun och föra dig tillbaka samma väg, som du har kommit på.
\par 30 Och detta skall för dig vara tecknet: man skall detta år äta, vad som växer upp av spillsäd, och nästa år självvuxen säd, men det tredje året skolen I få så och skörda och plantera vingårdar och äta deras frukt.
\par 31 Och den räddade skaran av Juda hus, som bliver kvar, skall åter skjuta rot nedtill och bära frukt upptill.
\par 32 Ty från Jerusalem skall utgå en kvarleva, en räddad skara från Sions berg. HERREN Sebaots nitälskan skall göra detta.
\par 33 Därför säger HERREN så om konungen i Assyrien: Han skall icke komma in i denna stad och icke skjuta någon pil ditin; han skall icke mot den föra fram någon sköld eller kasta upp någon vall mot den.
\par 34 Samma väg han kom skall han vända tillbaka, och in i denna stad skall han icke komma, säger HERREN.
\par 35 Ty jag skall beskärma och frälsa denna stad för min tjänare Davids skull."
\par 36 Och HERRENS ängel gick ut och slog i assyriernas läger ett hundra åttiofem tusen man; och när man bittida följande morgon kom ut, fick man se döda kroppar ligga där överallt.
\par 37 Då bröt Sanherib, konungen i Assyrien, upp och tågade tillbaka; och han stannade sedan i Nineve.
\par 38 Men när han en gång tillbad i sin gud Nisroks tempel, blev han dräpt med svärd av sina söner Adrammelek och Sareser; därefter flydde dessa undan till Ararats land. Och hans son Esarhaddon blev konung efter honom.

\chapter{38}

\par 1 Vid den tiden blev Hiskia dödssjuk; och profeten Jesaja, Amos' son, kom till honom och sade till honom: "Så säger HERREN: Beställ om ditt hus; ty du måste dö och skall icke tillfriskna."
\par 2 Då vände Hiskia sitt ansikte mot väggen och bad till HERREN.
\par 3 Och han sade: "Ack HERRE, tänk dock på huru jag har vandrat inför dig i trohet och med hängivet hjärta och gjort, vad gott är i dina ögon." Och Hiskia grät bitterligen.
\par 4 Då kom HERRENS ord till Jesaja: han sade:
\par 5 "Gå och säg till Hiskia: Så säger HERREN, din fader Davids Gud: Jag har hört din bön, jag har sett dina tårar. Se, jag skall föröka din livstid med femton år;
\par 6 jag skall ock rädda dig och denna stad ur den assyriske konungens hand. Ja, jag skall beskärma denna stad.
\par 7 Och detta skall för dig vara tecknet från HERREN därpå att HERREN skall göra, vad han nu har lovat:
\par 8 se, solvisarskuggan, som nu på Ahas' solvisare har gått nedåt med solen, skall jag låta gå tio steg tillbaka." Så gick solen tillbaka på solvisaren de tio steg, som den reda hade lagt till rygga.
\par 9 En sång, skriven av Hiskia, Juda konung, när han hade varit sjuk och tillfrisknat från sin sjukdom:
\par 10 Jag tänkte: Jag går hädan i mina bästa dagar, in genom dödsrikets portar; jag varder berövad återstoden av mina år.
\par 11 Jag tänkte: Jag får icke mer se HERREN, HERREN i de levandes land. Hos dem som bo i förgängelsens rike får jag ej mer skåda människor.
\par 12 Min hydda ryckes upp och flyttas bort ifrån mig såsom en herdes tält. Jag har vävt mitt liv till slut såsom en vävare sin väv, och jag skäres nu ned från bommen; innan dagen har gått över till natt, är du färdig med mig.
\par 13 Jag måste ryta såsom ett lejon intill morgonen; så krossas alla bin i min kropp. Ja, innan dagen har gått över till natt, är du färdig med mig.
\par 14 Jag klagade såsom en svala, såsom en trana, jag suckade såsom en duva; matta blickade mina ögon mot höjden: "HERRE, jag lider nöd; tag dig an min sak."
\par 15 Men vad skall jag väl säga, då han nu har talat till mig och själv utfört sitt verk? I ro får jag nu leva alla mina år till slut efter all min själs bedrövelse.
\par 16 Herre, sådant länder till liv, min ande har i allo sitt liv därav. Och så helar du mig - ja, giv mig liv!
\par 17 Se till mitt bästa kom denna bittra bedrövelse över mig. I din kärlek räddade du min själ ifrån förintelsens grop, i det du kastade alla mina synder bakom din rygg.
\par 18 Ty dödsriket tackar dig icke, döden prisar dig icke, och de som hava farit ned i graven hoppas ej mer på din trofasthet.
\par 19 De som leva, de som leva, de tacka dig, såsom ock jag nu gör; och fäderna göra din trofasthet kunnig för barnen.
\par 20 HERREN skall frälsa mig, och mina sånger skola vi då spela i alla våra livsdagar däruppe i HERRENS hus.
\par 21 Och Jesaja tillsade, att man skulle taga en fikonkaka och lägga den såsom plåster på bulnaden, så skulle han tillfriskna.
\par 22 Men Hiskia sade: "Vad för ett tecken gives mig därpå att jag skall få gå upp i HERRENS hus?"

\chapter{39}

\par 1 Vid samma tid sände Merodak-Baladan, Baladans son, konungen i Babel, brev och skänker till Hiskia; och han fick höra, att denne hade varit sjuk, men blivit återställd.
\par 2 Och Hiskia gladde sig över deras ankomst och visade dem sitt förrådshus, sitt silver och guld, sina välluktande kryddor och sina dyrbara oljor och hela sitt tyghus och allt vad som fanns i hans skattkamrar. Intet fanns i Hiskias hus eller eljest i hans ägo, som han icke visade dem.
\par 3 Men profeten Jesaja kom till konung Hiskia och sade till honom: "Vad hava dessa män sagt, och varifrån hava de kommit till dig?" Hiskia svarade: "De hava kommit till mig ifrån fjärran land, ifrån Babel."
\par 4 Han sade vidare: "Vad hava de sett i ditt hus?" Hiskia svarade: "Allt som är i mitt hus hava de sett: intet finnes i mina skattkamrar, som jag icke har visat dem."
\par 5 Då sade Jesaja till Hiskia: "Hör HERREN Sebaots ord:
\par 6 Se, dagar skola komma, då allt som finnes i ditt hus och som dina fäder hava samlat ända till denna dag skall föras bort till Babel; intet skall bliva kvar, säger HERREN.
\par 7 Och söner till dig, de som skola utgå av dig och som du skall föda, dem skall man taga, och de skola bliva hovtjänare i den babyloniske konungens palats."
\par 8 Hiskia sade till Jesaja: "Gott är det HERRENS ord, som du har talat." Och han sade ytterligare: "Frid och trygghet skola ju få råda i min tid."

\chapter{40}

\par 1 Trösten, trösten mitt folk, säger eder Gud.
\par 2 Talen ljuvligt till Jerusalem och prediken för det, att dess vedermöda är slut, att dess missgärning är försonad och att det har fått dubbelt igen av HERRENS hand för all sina synder.
\par 3 Hör, man ropar; "Bereden väg för HERREN i öknen, banen på hedmarken en jämn väg för vår Gud.
\par 4 Alla dalar skola höjas och alla berg och höjder sänkas; vad ojämnt är skall jämnas, och vad oländigt är skall bliva slät mark.
\par 5 HERRENS härlighet skall varda uppenbarad, och allt kött skall tillsammans se den. Ty så har HERRENS mun talat."
\par 6 Hör, någon talar: "Predika!", och en annan svarar: "Vad skall jag predika?" "Allt kött är gräs och all dess härlighet såsom ett blomster på marken.
\par 7 Gräset torkar bort, blomstret förvissnar, när HERRENS andedräkt blåser därpå.
\par 8 Gräset torkar bort, blomstret förvissnar, men vår Guds ord förbliver evinnerligen."
\par 9 Stig upp på ett högt berg, Sion, du glädjens budbärarinna; häv upp din röst med kraft, Jerusalem, du glädjens budbärarinna: häv upp den utan fruktan, säg till Juda stöder: "Se, där är eder Gud!"
\par 10 Ja, Herren, HERREN kommer med väldighet, och hans arm visar sin makt. Se, han har med sig sin lön, och hans segerbyte går framför honom.
\par 11 Han för sin hjord i bet såsom en herde, han samlar lammen i sin famn och bär dem i sitt sköte och sakta för han moderfåren fram.
\par 12 Vem är det, som mäter upp havens vatten i sin hand och märker ut himmelens vidd med sina utspända fingrar? Vem mäter upp stoftet på jorden med ett tredingsmått? Vem väger bergen på en våg och höjderna på en viktskål?
\par 13 Vem kan utrannsaka HERRENS Ande, och vem kan giva honom råd och undervisa honom?
\par 14 Går han till råds med någon, för att denne skall giva honom förstånd och lära honom den rätta stigen, lära honom kunskap och visa honom förståndets väg?
\par 15 Nej, folken äro att akta såsom en droppe ur ämbaret och såsom ett grand på vågskålen; se, havsländerna lyfter han såsom ett stoftkorn.
\par 16 Libanons skog vore icke nog till offerved och dess djur icke nog till brännoffer.
\par 17 Folken äro allasammans såsom ett intet inför honom; såsom alls intet och idel tomhet aktas de av honom.
\par 18 Vid vem viljen I då likna Gud, och vad finnes honom likt att ställa vid hans sida?
\par 19 Månne ett avgudabeläte? - det gjutes av någon konstnär, och guldsmeden överdrager det sedan med guld, och med silverkedjor pryder så guldsmeden det.
\par 20 Den som icke har råd att offra så mycket, han väljer ut ett stycke trä, som icke ruttnar, och söker sig en förfaren konstnär, som kan förfärdiga ett beläte, som ej faller omkull.
\par 21 Haven I då intet förstånd? Hören I då intet? Blev detta icke förkunnat för eder från begynnelsen? Haven I icke förstått, vad jordens grundvalar säga?
\par 22 Han är den som tronar över jordens rund, och dess inbyggare äro såsom gräshoppor; han är den som utbreder himmelen såsom ett flor och spänner ut den såsom ett tält att bo inunder.
\par 23 Han är den som gör furstarna till intet, förvandlar domarna på jorden till idel tomhet.
\par 24 Knappt äro de planterade, knappt äro de sådda, knappt har deras stam slagit rot i jorden, så blåser han på dem, och de förtorka, och en stormvind för dem bort såsom strå.
\par 25 Vid vem viljen I då likna mig, så agg jag skulle vara såsom han? säger den Helige.
\par 26 Lyften upp edra ögon mot höjden och sen: vem har skapat allt detta? Det har han som för härskaran däruppe fram i räknade hopar; han nämner dem alla vid namn. Så stor är hans makt, så väldig hans kraft, att icke en enda utebliver.
\par 27 Huru kan du säga sådant, du Jakob, och tala så, du Israel: "Min väg är fördold för HERREN, och min rätt är försvunnen för min Gud"?
\par 28 Vet du då icke, har du ej hört det, att HERREN är en evig Gud, han som har skapat jordens ändar? Han bliver ej trött och uppgives icke, hans förstånd är outrannsakligt.
\par 29 Han giver den trötte kraft och förökar den maktlöses styrka.
\par 30 Ynglingar kunna bliva trötta och uppgivas, och unga män kunna falla;
\par 31 men de som bida efter HERREN hämta ny kraft, de få nya vingfjädrar såsom örnarna. Så hasta de åstad utan att uppgivas, de färdas framåt utan att bliva trötta.

\chapter{41}

\par 1 Tigen, I havsländer, och lyssnen till mig, och må folken hämta ny kraft; må de så komma fram och tala, ja, låt oss med varandra träda inför rätta.
\par 2 Vem har i öster låtit denne uppstå, som mötes av seger, var han går fram? Vem giver folkslag i hans våld och gör honom till härskare över konungar? Vem gör deras svärd till stoft och deras bågar till strå som föres bort av vinden?
\par 3 Han förjagar dem, där han går lyckosam fram, vanliga vägar trampar icke hans fot.
\par 4 Vem har verkat och utfört detta? Det har han som från begynnelsen kallade människors släkten fram: jag, HERREN, som är den förste och som intill det sista är densamme.
\par 5 Havsländerna se det och frukta, och jordens ändar förskräckas. Man närmar sig till varandra och kommer tillhopa.
\par 6 Den ene vill hjälpa den andre; han säger till den andre: "Fatta mod!"
\par 7 Träsnidaren sätter mod i guldsmeden, bleckslagaren i den som hamrar på städ. Han säger om lödningen: "Den är god" och fäster bilden med spikar, så att den ej faller omkull.
\par 8 Men du Israel, min tjänare, du Jakob, som jag har utvalt, du ättling av Abraham, min vän,
\par 9 du som jag har hämtat från jordens ändar och kallat hit från dess yttersta hörn och till vilken jag har sagt: "Du är min tjänare, dig har jag utvalt och icke försmått",
\par 10 frukta icke, ty jag är med dig; var ej försagd, ty jag är din Gud. Jag styrker dig, jag hjälper dig ock, jag uppehåller dig med min rättfärdighets högra hand.
\par 11 Se, alla som äro dig hätska skola komma på skam och blygas; dina motståndare skola bliva till intet och skola förgås.
\par 12 Du skall söka efter dina vedersakare, men icke finna dem; ja, de som strida mot dig skola bliva till intet och få en ände.
\par 13 Ty jag är HERREN, din Gud, som håller dig vid din högra hand och som säger till dig: Frukta icke, jag hjälper dig.
\par 14 Så frukta nu icke, du mask Jakob, du Israels lilla hop. Jag hjälper dig, säger HERREN; din förlossare är Israels Helige.
\par 15 Se, jag gör dig till en tröskvagn, ny och med skarpa taggar, så att du skall söndertröska berg och krossa dem till stoft och göra höjder lika agnar.
\par 16 Du skall kasta dem med kastskovel, och vinden skall föra dem bort och stormen förskingra dem; men du själv skall fröjda dig i HERREN och berömma dig av Israels Helige.
\par 17 De betryckta och fattiga söka förgäves efter vatten, deras tunga försmäktar av törst; men jag, HERREN, skall bönhöra dem, jag, Israels Gud, skall icke övergiva dem.
\par 18 Jag skall låta strömmar rinna upp på höjderna och källor i dalarna; jag skall göra öknen till en vattenrik sjö och torrt land till källsprång.
\par 19 Och jag skall låta cedrar och akacieträd växa upp i öknen jämte myrten och olivträd och skall på hedmarken plantera cypress tillsammans med alm och buxbom,
\par 20 för att man skall både se och veta och akta på och förstå, att HERRENS hand har gjort detta, att Israels Helige har skapat det.
\par 21 Så träden nu fram med eder sak, säger HERREN; kommen med edra bevis, säger Jakobs konung.
\par 22 Ja, må man komma med dem och förkunna för oss, vad som skall ske. Var äro edra forna utsagor? Läggen fram dem, för att vi må akta på dem och se till, huru de hava gått i fullbordan. Eller låten oss höra, vad som nu skall komma,
\par 23 förkunnen, vad framdeles skall hända, för att vi må se, att I ären gudar. Ja, gören någonting, vad det nu vara må, så att vi alla häpna, när vi se det.
\par 24 Men se, I ären ett intet, och edert verk är alls intet; den som utväljer eder är en styggelse.
\par 25 Jag lät i norr en man uppstå, och han kom, ja, i öster en som skulle åkalla mitt namn; och han skulle gå fram över landsherrarna, såsom vore de lerjord, lik en krukmakare, som trampar lera.
\par 26 Vem förkunnade detta förut, så att vi fingo veta det, eller i förväg, så att vi kunde säga: "Du hade rätt"? Ingen fanns, som förkunnade det, ingen, som lät oss höra det, ingen, som hörde eder tala därom.
\par 27 Jag är den förste, som säger till Sion: "Se, se där äro de", den förste, som bringar Jerusalem detta glädjens budskap.
\par 28 Jag ser mig om, men här finnes ingen, ingen bland dessa, som kan giva besked; ingen som kan giva ett svar på min fråga.
\par 29 Se, de äro allasammans fåfänglighet, deras verk äro ett intet, deras beläten vind och tomhet.

\chapter{42}

\par 1 Se, över min tjänare som jag uppehåller, min utkorade, till vilken min själ har behag, över honom har jag låtit min Ande komma; han skall utbreda rätten bland folken.
\par 2 Han skall icke skria eller ropa och icke låta höra sin röst på gatorna.
\par 3 Ett brutet rör skall han icke sönderkrossa, och en tynande veke skall han icke utsläcka; han skall i trofasthet utbreda rätten.
\par 4 Hans kraft skall icke förtyna eller brytas, intill dess att han har grundat rätten på jorden; havsländerna vänta efter hans lag.
\par 5 Så säger Gud, HERREN, han som har skapat himmelen och utspänt den, han som har utbrett jorden med vad som alstras därav, han som har givit liv åt folket som är därpå och ande åt dem som vandra där:
\par 6 Jag, HERREN, har kallat dig i rättfärdighet, och jag vill fatta dig vid handen och bevara dig och fullborda i dig förbundet med folket och sätta dig till ett ljus för folkslagen,
\par 7 för att du må öppna blinda ögon och föra fångar ut ur fängelset, ja, ur fångenskapen dem som sitta i mörkret.
\par 8 Jag, HERREN, det är mitt namn; och jag giver icke min ära åt någon annan eller mitt lov åt belätena.
\par 9 Se, vad jag förut förkunnade, det har nu kommit. Nu förkunnar jag nya ting; förrän de visa sig, låter jag eder höra om dem.
\par 10 Sjungen till HERRENS ära en ny sång, hans lov från jordens ända, I som faren på havet, så ock allt vad däri är, I havsländer med edra inbyggare;
\par 11 stämmen upp, du öken med dina städer och I byar, där Kedar bor; jublen, I klippornas invånare, ropen från bergens toppar.
\par 12 Given HERREN ära och förkunnen hans lov i havsländerna.
\par 13 HERREN drager ut såsom en hjälte, han eggar upp sig till iver såsom en krigare; han uppgiver härskri, han ropar högt och visar sin makt mot sina fiender.
\par 14 I lång tid har jag tegat, jag höll mig stilla och betvang mig; men nu skall jag höja rop såsom en barnaföderska, jag vill skaffa mig luft och andas ut.
\par 15 Jag skall föröda berg och höjder och låta allt gräs på dem förtorka; jag skall göra strömmar till land och låta allt gräs på dem förtorka; jag skall göra strömmar till land och låta sjöar torka ut.
\par 16 Och de blinda skall jag leda på en väg som de icke känna; på stigar som de icke känna skall jag föra dem. Jag skall göra mörkret framför dem till ljus och det som är ojämnt till jämn mark. Detta är, vad jag skall göra, och jag skall ej rygga mitt ord.
\par 17 Men de som förtrösta på skurna beläten och som säga till gjutna beläten: "I ären våra gudar", de skola vika tillbaka och stå där med skam.
\par 18 Hören, I döve; I blinde, skåden och sen.
\par 19 Vem är blind, om icke min tjänare, och så döv som den budbärare jag sänder åstad?
\par 20 Du har fått se mycket, men du aktar icke därpå; fastän öronen hava blivit öppnade, lyssnar ingen till.
\par 21 Det är HERRENS behag, för hans rättfärdighets skull, att han vill låta sin lag komma till makt och ära.
\par 22 Men detta är ett plundrat och skövlat folk; dess ynglingar äro alla lagda i bojor, och i fängelser hållas de gömda, de hava blivit givna till plundring, och ingen finnes, som räddar, till skövling, och ingen säger: "Giv tillbaka."
\par 23 Ack att någon bland eder ville lyssna härtill, för framtiden giva akt och höra härpå!
\par 24 Vem har lämnat Jakob till skövling och Israel i plundrares våld? Har icke HERREN gjort det; han, mot vilken vi hava syndat, han, på vilkens vägar man icke ville vandra och på vilkens lag man icke ville höra?
\par 25 Därför utgöt han över dem i sin vrede förtörnelse och krigets raseri. Och de förbrändes därav runt omkring, men besinnade det icke; de förtärdes därav, men aktade icke därpå.

\chapter{43}

\par 1 Men nu säger HERREN så, han som har skapat dig, Jakob, han som har danat dig, Israel: Frukta icke, ty jag har förlossat dig, jag har kallat dig vid ditt namn, du är min.
\par 2 Om du ock måste gå genom vatten, så är jag med dig, eller genom strömmar, så skola de icke fördränka dig; måste du än gå genom eld, så skall du ej bliva svedd, och lågorna skola ej förtära dig.
\par 3 Ty jag är HERREN, din Gud, Israels Helige, din frälsare; jag giver Egypten till lösepenning för dig, Etiopien och Seba i ditt ställe.
\par 4 Eftersom du är så dyrbar i mina ögon, så högt aktad och så älskad av mig, därför giver jag människor till lösen för dig och folk till lösen för ditt liv.
\par 5 Frukta då icke, ty jag är med dig. Jag skall låta dina barn komma från öster, och från väster skall jag samla dig tillhopa.
\par 6 Jag skall säga till Norden: "Giv hit" och till södern: "Förhåll mig dem icke; för hit mina söner ifrån fjärran och mina döttrar ifrån jordens ända,
\par 7 envar som är uppkallad efter mitt namn och som jag har skapat till min ära, envar som jag har danat och gjort."
\par 8 För hitut det blinda folket, som dock har ögon, och de döva, som dock hava öron.
\par 9 Alla folk hava kommit tillsammans, folkslagen samla sig tillhopa. Vem bland dem finnes, som skulle kunna förutsäga sådant? Må de låta oss höra sina forna utsagor. Må de ställa fram sina vittnen och bevisa sin rätt, så att dessa, när de höra det, kunna säga: "Det är sant."
\par 10 Men I ären mina vittnen, säger HERREN, I ären min tjänare, den som jag har utvalt, på det att I mån veta och tro mig och förstå, att det är jag; före mig är ingen Gud danad, och efter mig skall ingen komma.
\par 11 Jag, jag är HERREN, och förutom mig finnes ingen frälsare.
\par 12 Jag har förkunnat det och skaffat frälsning, jag har kungjort det och ingen främmande gud bland eder. I ären mina vittnen, säger HERREN; och jag är Gud.
\par 13 Ja, allt framgent är jag densamme, och ingen kan rädda från min hand. När jag vill göra något, vem kan då avvända det?
\par 14 Så säger HERREN, eder förlossare, Israels Helige: För eder skull sänder jag mitt bud mot Babel, och jag skall driva dem allasammans på flykten, jag skall driva kaldéerna ned på skeppen som voro deras fröjd.
\par 15 Jag är HERREN, eder Helige, Israels skapare, eder konung.
\par 16 Så säger HERREN, han som gör en väg i havet, en stig i väldiga vatten,
\par 17 han som för vagnar och hästar ditut, ja, härskara och och stridsmakt, sedan ligga de där tillhopa och kunna icke stå upp, de äro utsläckta, de hava slocknat såsom en veke:
\par 18 Tänken icke på vad förr har varit, akten icke på vad fordom har skett.
\par 19 Se, jag vill göra något nytt. Redan nu visar det sig; märken I det icke? Ja, jag skall göra en väg i öknen och strömmar i ödemarken,
\par 20 så att markens djur skola ära mig, schakaler och strutsar, därför att jag låter vatten flyta i öknen, strömmar i ödemarken, så att mitt folk, min utkorade, kan få dricka.
\par 21 Det folk, som jag har danat åt mig, skall förtälja mitt lov.
\par 22 Men icke har du, Jakob, kallat mig hit, i det du har gjort dig möda för min skull, du Israel.
\par 23 Icke har du framburit åt mig dina brännoffersfår eller ärat mig med dina slaktoffer; icke har jag vållat dig arbete med spisoffer, ej heller möda med rökelse.
\par 24 Icke har du köpt kalmus åt mig för dina penningar eller mättat mig med dina slaktoffers fett. Nej, du har vållat mig arbete genom dina synder och möda genom dina missgärningar.
\par 25 Jag, jag är den som utplånar dina överträdelser för min egen skull, och dina synder kommer jag icke mer ihåg.
\par 26 Låt mig höra, vad du har att säga, och låt oss gå till rätta med varandra; tala du, för att du må finnas rättfärdig.
\par 27 Men se, redan din stamfader syndade, och de som förde din talan begingo överträdelser mot mig.
\par 28 Därför har jag måst låta helgedomens furstar utstå vanära och har överlämnat Jakob åt tillspillogivning, Israel åt försmädelse.

\chapter{44}

\par 1 Men hör nu, du Jakob, min tjänare, du Israel, som jag har utvalt.
\par 2 Så säger HERREN, han som har skapat dig, han som danade dig redan i moderlivet och som hjälper dig: Frukta icke, du min tjänare Jakob, du Jesurun, som jag har utvalt.
\par 3 Ty jag skall utgjuta vatten över de törstiga och strömmar över det torra; jag skall utgjuta min Ande över din barn och min välsignelse över dina telningar,
\par 4 så att de växa upp mitt ibland gräset såsom pilträd vid vattenbäckar.
\par 5 Då skall den ene säga: "HERREN tillhör jag", och den andre skall åberopa Jakobs namn, och en tredje skall skriva på sin hand: "HERRENS egen" och skall bruka Israel såsom ett ärenamn.
\par 6 Så säger HERREN, Israels konung, och hans förlossare, HERREN Sebaot: Jag är den förste, och jag är den siste, och förutom mig finnes ingen Gud.
\par 7 Och vem talar, såsom jag har gjort, alltsedan jag lät urtidsfolket framträda? Må han förkunna det och lägga det fram för mig. Ja, må de förkunna det tillkommande, vad som skall ske.
\par 8 Frukten icke och varen icke förskräckta. Har jag icke för länge sedan låtit dig höra om detta och förkunnat det? I ären ju mina vittnen. Finnes väl någon Gud förutom mig? Nej, ingen annan klippa finnes, jag vet av ingen.
\par 9 Avgudamakarna äro allasammans idel tomhet, och deras kära gudar kunna icke hjälpa. Deras bekännare se själva intet och förstå intet; därför måste de ock komma på skam.
\par 10 Om någon formar en gud och gjuter ett beläte, så är det honom till intet gagn.
\par 11 Se, hela dess följe skall komma på skam; konstnärerna själva äro ju allenast människor. Må de församlas, så många de äro, och träda fram; de skola då alla tillhopa med förskräckelse komma på skam.
\par 12 Smeden tager sitt verktyg och bearbetar sitt smide i glöden, han formar det med hammare, han bearbetar det med kraftig arm; till äventyrs får han därvid svälta, så att han bliver vanmäktig, och försaka att dricka, så att han bliver matt.
\par 13 Träsnidaren spänner ut sitt mätsnöre och gör märken på trästycket med sitt ritstift, han arbetar därpå med sina eggjärn och märker ut det med passaren; och han gör så därav en mansbild, en prydlig människogestalt, som får bo i ett hus.
\par 14 Man fäller åt sig cedrar; man tager plantor av stenek och vanlig ek och uppdrager dem åt sig bland skogens träd; man planterar åt sig lärkträd, och regnet giver dem växt.
\par 15 Detta hava människorna till bränsle; och man tager därav och värmer sig därmed, man tänder på det och bakar bröd därvid. Men därjämte förfärdigar man en gud därav och tillbeder den, man gör därav ett beläte och faller ned för det.
\par 16 En del av träet bränner man alltså upp i eld, över en annan del därav tillagar man kött till att äta, steker sin stek och äter sig mätt; när man så har värmt sig, säger man: "Gott, nu är jag varm, nu njuter jag av brasan."
\par 17 Men av det som är kvar gör man en gud, man gör sig ett beläte, och för det faller man ned och tillbeder, man bönfaller inför det och säger: "Rädda mig, ty du är min gud." -
\par 18 Ja, sådana veta intet och förstå intet, ty igentäppta äro deras ögon, så att de icke se, och deras hjärtan, så att de intet begripa.
\par 19 Ingen har så mycken eftertanke, så mycket vett eller förstånd, att han säger: "En del därav har jag bränt upp i eld, och på kolen har jag bakat bröd och stekt kött och har så ätit; skulle jag då av återstoden göra en styggelse? Skulle jag falla ned för ett stycke trä?"
\par 20 Den som så håller sig till vad som blott är aska, han är förledd av ett dårat hjärta, så att han icke förstår att rädda sin själ, icke att tänka: "Blott fåfänglighet är, vad jag håller i min högra hand."
\par 21 Tänk härpå, du Jakob, du Israel, ty du är min tjänare; jag har danat dig, ja, du är min tjänare. Israel, du varder icke förgäten av mig.
\par 22 Jag utplånar dina överträdelser såsom ett moln och dina synder såsom en sky. Vänd om till mig, ty jag förlossar dig.
\par 23 Jublen, I himlar, ty HERREN utför sitt verk; höjen glädjerop, I jordens djup, bristen ut i jubel, I berg, du skog med alla dina träd; ty HERREN förlossar Jakob, han bevisar sig härlig i Israel.
\par 24 Så säger HERREN, din förlossare, han som danade dig redan i moderlivet: "Jag, HERREN, är den som för allt, den som ensam utspänner himmelen och utan någons hjälp breder ut jorden.
\par 25 Jag är den som gör lögnprofeternas tecken om intet och gör spåmännen till dårar, den som låter de vise komma till korta och gör deras klokhet till dårskap,
\par 26 men som låter sin tjänares ord bliva beståndande och fullbordar sina sändebuds rådslag. Jag är den som säger om Jerusalem: "Det skall bliva bebott" och om Juda städer: "De skola varda uppbyggda; jag skall upprätta ruinerna där."
\par 27 Jag är den som säger till havsdjupet: "Sina ut; dina strömmar vill jag låta uttorka."
\par 28 Jag är den som säger om Kores: "Han är min herde, han skall fullborda all min vilja, och han skall säga om Jerusalem: 'Det skall bliva uppbyggt' och till templet: 'Din grund skall åter varda lagd.'"

\chapter{45}

\par 1 Så säger HERREN till sin smorde, till Kores som jag har fattat vid hans högra hand, då jag nu vill slå ned folken inför honom och lösa svärdet från konungarnas länd, då jag vill öppna dörrarna för honom så att inga portar mer äro stängda:
\par 2 Själv skall jag gå framför dig, backarna skall jag jämna ut; kopparportarna skall jag krossa, och järnbommarna skall jag bryta sönder.
\par 3 Och jag skall giva dig dolda skatter och bortgömda rikedomar, för att du må förnimma, att jag, HERREN, är den som har kallat dig vid ditt namn, jag, Israels Gud.
\par 4 För min tjänare Jakobs skull, för Israels, min utkorades, skull kallade jag dig vid ditt namn och gav dig ärenamn, innan du kände mig.
\par 5 Jag är HERREN och eljest ingen, utom mig finnes ingen Gud; innan du kände mig, omgjordade jag dig,
\par 6 för att man skulle förnimma både i öster och i väster, att alls ingen finnes förutom mig, att jag är HERREN och eljest ingen,
\par 7 jag som danar ljuset och skapar mörkret, jag som giver lyckan och skapar olyckan. Jag, HERREN, är den som gör allt detta.
\par 8 Drypen, I himlar därovan, och må skyarna låta rättfärdighet strömma ned. Må jorden öppna sig, och må dess frukt bliva frälsning; rättfärdighet låte den ock växa upp. Jag, HERREN, skapar detta.
\par 9 Ve dig som vill gå till rätta med din Skapare, ja, ve dig, du skärva bland andra skärvor av jord! Skall väl leret säga till krukmakaren: "Vad kan du göra?" Och skall ditt verk säga om dig: "Han har inga händer"?
\par 10 Ve dig som säger till din fader: "Icke kan du avla barn" och till hans hustru: "Icke kan du föda barn"!
\par 11 Så säger HERREN, Israels Helige, som ock är hans skapare: Frågen mig om det tillkommande; lämnen åt mig omsorgen om mina söner, mina händer verk.
\par 12 Det är jag, som har gjort jorden och skapat människorna därpå; det är mina händer, som hava utspänt himmelen, och hela dess härskara har jag bådat upp.
\par 13 Det är ock jag, som har låtit denne uppstå i rättfärdighet, och alla hans vägar skall jag göra jämna. Han skall bygga upp min stad och släppa mina fångar lösa, och det icke för betalning eller för gåvor, säger HERREN Sebaot.
\par 14 Så säger HERREN: Vad egyptierna hava förvärvat med sitt arbete och etiopiernas och Sebas resliga folk med sin handel, det skall allt övergå i din hand och höra dig till. De skola följa bakom dig, i kedjor skola de gå. Och de skola falla ned inför dig och ställa sin bön till dig: "Allenast i dig är Gud, och eljest finnes ingen, alls ingen annan Gud."
\par 15 Ja, du är sannerligen en outgrundlig Gud, du Israels Gud, du frälsare
\par 16 De komma alla på skam och varda till blygd, de måste allasammans gå där med blygd, alla avgudamakarna.
\par 17 Men Israel bliver frälst genom HERREN med en evig frälsning; aldrig i evighet skolen I komma på skam och varda till blygd.
\par 18 Ty så säger HERREN, han som har skapat himmelen, han som är Gud, han som har danat jorden och gjort den, han som har berett den och som icke har skapat den till att vara öde, utan danat den till att bebos: Jag är HERREN och eljest ingen.
\par 19 Jag har icke talat i det fördolda, någonstädes i ett mörkt land; jag har icke sagt till Jakobs släkt: Förgäves skolen I söka mig. Jag är HERREN, som talar sanning, som förkunnar, vad rätt är.
\par 20 Så församlen eder nu och kommen hit, träden fram allasammans, I räddade, som ären kvar av folken. Ty de hava intet förstånd, de som föra sina träbeläten omkring i högtidståg och bedja till en gud som icke kan frälsa.
\par 21 Förkunnen något och läggen fram det; alla tillhopa må rådslå därom. Vem har långt förut låtit eder höra detta och för länge sedan förkunnat det? Har icke jag, HERREN, gjort det jag, förutom vilken ingen Gud mer finnes, ingen Gud, som är rättfärdig och som frälsar, nej, ingen finnes jämte mig.
\par 22 Vänden eder till mig, så varden I frälsta, I jordens alla ändar; ty jag är Gud och eljest ingen.
\par 23 Jag har svurit vid mig själv, från min mun har utgått ett sanningsord, ett ord, som icke skall ryggas: För mig skola alla knän böja sig, och mig skola alla tungor giva sin ed.
\par 24 Så har man betygat om mig: Allenast hos HERREN finnes rättfärdighet och makt. Till honom skola komma med blygd alla de som hava varit honom hätska.
\par 25 Ja, genom HERREN får all Israels släkt sin rätt, och av honom skola de berömma sig.

\chapter{46}

\par 1 Bel sjunker ned, Nebo måste böja sig, deras bilder lämnas åt djur och fänad; de som I förden omkring i högtidståg, de lastas nu på ök som bära sig trötta av bördan.
\par 2 Ja, de måste båda böja sig och sjunka ned; de kunna icke rädda någon börda, själva vandra de bort i fångenskap.
\par 3 Så hören nu på mig, I av Jakobs hus, I alla som ären kvar av Israels hus, I som haven varit lastade på mig allt ifrån moderlivet och burna av mig allt ifrån modersskötet.
\par 4 Ända till eder ålderdom är jag densamme, och intill dess I varden grå, skall jag bära eder; så har jag hittills gjort, och jag skall också framgent hålla eder uppe, jag skall bära och rädda eder.
\par 5 Med vem viljen I likna och jämföra mig, och med vem viljen I sammanställa mig, så att jag skulle vara honom lik?
\par 6 Man skakar ut guld ur pungen och väger upp silver på vågen, och så lejer man en guldsmed att göra det till en gud, för vilken man kan falla ned och tillbedja.
\par 7 Den lyfter man på axeln och bär den bort och sätter ned den på dess plats, för att den skall stå där och ej vika från stället. Men ropar någon till den, så svarar den icke och frälsar honom icke ur hans nöd.
\par 8 Tänken härpå och kommen till förnuft; besinnen eder, I överträdare.
\par 9 Tänken på vad förr var, redan i forntiden; ty jag är Gud och eljest ingen, en Gud, vilkens like icke finnes;
\par 10 jag som i förväg förkunnar, vad komma skall, och långt förut, vad ännu ej har skett; jag som säger: "Mitt rådslut skall gå i fullbordan, och allt vad jag vill, det gör jag";
\par 11 jag som kallar på örnen från öster och ifrån fjärran land på mitt rådsluts man. Vad jag har bestämt, det sätter jag ock i verket.
\par 12 Så hören nu på mig, I stormodige, I som menen, att hjälpen är långt borta.
\par 13 Se, jag låter min hjälp nalkas, den är ej långt borta, och min frälsning dröjer icke; jag giver frälsning i Sion och min härlighet åt Israel.

\chapter{47}

\par 1 Stig ned och sätt dig i stoftet, du jungfru dotter Babel, sätt dig på jorden utan tron, du kaldéernas dotter; ty man skall icke mer kalla dig "den bortklemade och yppiga".
\par 2 Tag till kvarnen och mal mjöl, lägg av din slöja, lyft upp släpet, blotta benet, vada genom strömmarna.
\par 3 Din blygd skall varda blottad, och din skam skall ses. Hämnd skall jag utkräva och ej skona någon människa.
\par 4 Vår förlossares namn är HERREN Sebaot, Israels Helige!
\par 5 Sitt tyst och drag dig undan i mörkret, du kaldéernas dotter; ty du skall icke mer bliva kallad "konungarikenas drottning".
\par 6 Jag förtörnades på mitt folk, jag ohelgade min arvedel och gav dem i din hand. Och du visade dem intet förbarmande; på gamla män lät du ditt ok tynga hårt.
\par 7 Du tänkte: "Jag skall evinnerligen förbliva en drottning" därför ville du ej akta på och tänkte ej på änden.
\par 8 Så hör nu detta, du som lever i vällust, du som tronar så trygg, du som säger i ditt hjärta: "Jag och ingen annan; aldrig skall jag sitta såsom änka, aldrig veta av, vad barnlöshet är."
\par 9 Se, båda dessa olyckor skola komma över dig med hast, på en och samma dag: både barnlöshet och änkestånd skola komma över dig i fullaste mått, trots myckenheten av dina trolldomskonster, trots dina besvärjelsers starka kraft.
\par 10 Du kände dig trygg i din ondska, du tänkte: "Ingen ser mig." Din vishet och din kunskap var det, som förförde dig, så att du så sade i ditt hjärta: "Jag och ingen annan."
\par 11 Därför skall en olycka komma över dig, som du ej förmår besvärja bort, och ett fördärv skall falla över dig, som du icke skall kunna avvända; ja, plötsligt skall ödeläggelse komma över dig, när du minst anar det.
\par 12 Träd fram med de besvärjelser och många trolldomskonster som du har mödat dig med från din ungdom; se till, om du så kan skaffa hjälp, om du så kan skrämma bort faran.
\par 13 Du har arbetat dig trött med dina många rådslag. Må de nu träda fram, må de frälsa dig, dessa som avmäta himmelen och spana i stjärnorna och var nymånad kungöra, varifrån ditt öde skall komma över dig.
\par 14 Men se, de äro att likna vid strå som brännes upp i eld, de kunna icke rädda sitt liv ur lågornas våld. Detta är ju ingen koleld att värma sig framför, ingen brasa att sitta vid.
\par 15 Ja, så går det för dig med dem som du mödade dig för. Och dina handelsvänner från ungdomstiden draga bort, var och en åt sitt håll och ingen finnes, som frälsar dig.

\chapter{48}

\par 1 Hören detta, I av Jakobs hus, I som ären uppkallade med Israels namn och flutna ur Juda källa, I som svärjen vid HERRENS namn och prisen Israels Gud - dock icke i sanning och rättfärdighet,
\par 2 allt medan I kallen eder efter den heliga staden och stödjen eder på Israels Gud, på honom vilkens namn är HERREN Sebaot.
\par 3 Vad förut skedde, det hade jag för länge sedan förkunnat; av min mun var det förutsagt, och jag hade låtit eder höra därom. Plötsligt satte jag det i verket, och det inträffade.
\par 4 Eftersom jag visste, att du var så styvsint, ja, att din nacksena var av järn och din panna av koppar,
\par 5 därför förkunnade jag det för länge sedan och lät dig höra därom, innan det skedde, på det att du icke skulle kunna säga: "Min gudastod har gjort det, min gudabild, den skurna eller den gjutna har skickat det så."
\par 6 Du hade hört det, nu kan du se alltsammans; viljen I då icke erkänna det? Nu låter jag dig åter höra om nya ting, om fördolda ting som du ej har vetat av.
\par 7 Först nu hava de blivit skapade, icke tidigare, och förrän i dag fick du icke höra om dem, på det att du ej skulle kunna säga: "Det visste jag ju förut."
\par 8 Du fick icke förr höra något därom eller veta något därav, ej heller kom det tidigare för dina öron, eftersom jag visste, huru trolös du var och att du hette "överträdare" allt ifrån moderlivet.
\par 9 Men för mitt namns skull är jag långmodig, och för min äras skull håller jag tillbaka min vrede, så att du icke bliver utrotad.
\par 10 Se, jag har smält dig, men silver har jag icke fått; jag har prövat dig i lidandets ugn.
\par 11 För min egen skull, ja, för min egen skull gör jag så, ty huru skulle jag kunna låta mitt namn bliva ohelgat? Jag giver icke min ära åt någon annan.
\par 12 Hör på mig, du Jakob, du Israel, som jag har kallat. Jag är det; jag är den förste, jag är ock den siste.
\par 13 Min hand har lagt jordens grund, och min högra hand har utspänt himmelen; jag kallar på dem, då stå de där båda.
\par 14 Församlen eder, I alla, och hören: Vem bland dessa andra har förutsagt detta, att den man, som HERREN älskar, skall utföra hans vilja mot Babel och vara hans arm mot kaldéerna?
\par 15 Jag, jag har talat detta, jag har ock kallat honom, jag har fört honom fram, så att hans väg har blivit lyckosam.
\par 16 Träden hit till mig och hören detta; Mina förutsägelser har jag icke talat i det fördolda; när tiden kom, att något skulle ske, då var jag där. Och nu har Herren, HERREN sänt mig och sänt sin Ande.
\par 17 Så säger HERREN, din förlossare, Israels Helige: Jag är HERREN, din Gud, den som lär dig, vad nyttigt är, den som leder dig på den väg du skall vandra.
\par 18 O att du ville akta på mina bud! Då skulle frid tillflyta dig såsom en ström och din rätt såsom havets böljor;
\par 19 dina barn skulle då vara såsom sanden och din livsfrukt såsom sandkornen, dess namn skulle aldrig bliva utrotat eller utplånat ur min åsyn.
\par 20 Dragen ut från Babel, flyn ifrån kaldéernas land; förkunnen det med fröjderop och låten det bliva känt, utbreden ryktet därom till jordens ända; sägen: "HERREN har förlossat sin tjänare Jakob."
\par 21 De ledo ingen törst, när han förde dem genom ödemarker, ty han lät vatten strömma fram ur klippan åt dem, han klöv sönder klippan, och vattnet flödade.
\par 22 Men de ogudaktiga få ingen frid, säger HERREN.

\chapter{49}

\par 1 Hören på mig, I havsländer, och akten härpå, I folk, som bon i fjärran. HERREN kallade mig, när jag ännu var i moderlivet, han nämnde mitt namn, medan jag låg i min moders sköte.
\par 2 Och han gjorde min mun lik ett skarpt svärd och gömde mig under sin hands skugga; han gjorde mig till en vass pil och dolde mig i sitt koger.
\par 3 Och han sade till mig: "Du är min tjänare, Israel, genom vilken jag vill förhärliga mig."
\par 4 Men jag tänkte: "Förgäves har jag mödat mig, fruktlöst och fåfängt har jag förtärt min kraft; dock, min rätt är hos HERREN och min lön hos min Gud."
\par 5 Och nu säger HERREN, han som danade mig till sin tjänare, när jag ännu var i moderlivet, på det att jag måtte föra Jakob tillbaka till honom, så att Israel icke rycktes bort - ty jag är ärad i HERRENS ögon, och min Gud har blivit min starkhet -
\par 6 han säger: Det är för litet för dig, då du är min tjänare, att allenast upprätta Jakobs stammar och föra tillbaka de bevarade av Israel; jag vill sätta dig till ett ljus för hednafolken, för att min frälsning må nå till jordens ända.
\par 7 Så säger HERREN, Israels förlossare, hans Helige, till den djupt föraktade som är en styggelse för människor, en träl under tyranner: Konungar skola se det och stå upp, furstar skola se det och buga sig för HERRENS skull, som har bevisat sig trofast, för Israels Heliges skull, som har utvalt dig.
\par 8 Så säger HERREN: Jag bönhör dig i behaglig tid, och jag hjälper dig på frälsningens dag; jag skall bevara dig och fullborda i dig förbundet med folket, så att du skall upprätta landet och utskifta de förödda arvslotterna
\par 9 och säga till de fångna: "Dragen ut", till dem som sitta i mörkret: "Kommen fram." De skola finna bete utmed vägarna, ja, betesplatser på alla kala höjder;
\par 10 de skola varken hungra eller törsta, ökenhettan och solen skola icke skada dem, ty deras förbarmare skall leda dem och skall föra dem till vattenkällor.
\par 11 Och jag skall göra alla mina berg till öppna vägar, och mina farvägar skola byggas höga.
\par 12 Se, där komma de fjärran ifrån, ja, somliga från norr och andra från väster, somliga ock från sinéernas land.
\par 13 Jublen, I himlar, och fröjda dig, du jord, och bristen ut i jubel, I berg; ty HERREN tröstar sitt folk och förbarmar sig över sina betryckta.
\par 14 Men Sion säger: "HERREN har övergivit mig, Herren har förgätit mig."
\par 15 Kan då en moder förgäta sitt barn, så att hon icke har förbarmande med sin livsfrukt? Och om hon än kunde förgäta sitt barn, så skulle dock jag icke förgäta dig.
\par 16 Se, på mina händer har jag upptecknat dig; dina murar stå alltid inför mina ögon.
\par 17 Redan hasta dina söner fram, under det dina förstörare och härjare draga bort ifrån dig.
\par 18 Lyft upp dina ögon och se dig omkring: alla komma församlade till dig. Så sant jag lever, säger HERREN, du skall få ikläda dig dem alla såsom en skrud och lik en brud omgjorda dig med dem.
\par 19 Ty om du förut låg i ruiner och var ödelagd, ja, om ock ditt land var förhärjat, så skall du nu i stället bliva för trång för dina inbyggare, och dina fördärvare skola vara långt borta.
\par 20 Den tid stundar, då du skall få höra sägas av barnen som föddes under din barnlöshet: "Platsen är mig för trång, giv rum, så att jag kan bo här."
\par 21 Då skall du säga i ditt hjärta: "Vem har fött dessa åt mig? Jag var ju barnlös och ofruktsam, landsflyktig och fördriven; vem har då fostrat dessa? Se, jag var lämnad ensam kvar; varifrån komma då dessa?"
\par 22 Så säger Herren, HERREN: Se, jag skall upplyfta min hand till tecken åt folken och resa upp mitt baner till tecken åt folkslagen; då skola de bära dina söner hit i sin famn och föra dina döttrar fram på sina axlar.
\par 23 Och konungar skola vara dina barns vårdare och furstinnor deras ammor, de skola falla ned inför dig med ansiktet mot jorden och slicka dina fötters stoft. Och du skall förnimma, att jag är HERREN och att de som förbida mig icke komma på skam.
\par 24 Kan man taga ifrån hjälten hans byte eller rycka fångarna ifrån den som har segerns rätt?
\par 25 Och om än så vore, säger HERREN, om man än kunde taga ifrån hjälten hans fångar och rycka bytet ur den väldiges hand, så skulle jag dock själv stå emot dina motståndare, och själv skulle jag frälsa dina barn.
\par 26 Ja, jag skall tvinga dina förtryckare att äta sitt eget kött, och av sitt eget blod skola de bliva druckna såsom av druvsaft. Och allt kött skall då förnimma, att jag, HERREN, är din frälsare och att den Starke i Jakob är din förlossare.

\chapter{50}

\par 1 Så säger HERREN: Var är eder moders skiljebrev, det, varmed jag skulle hava förskjutit henne? Eller finnes bland mina borgenärer någon som jag har sålt eder åt? Nej, genom edra missgärningar bleven I sålda, och för edra överträdelsers skull blev eder moder förskjuten.
\par 2 Varför var ingen tillstädes, när jag kom? Varför svarade ingen, när jag ropade? Har då min arm blivit för kort, så att den ej kan förlossa, eller finnes hos mig ingen kraft till att hjälpa? Med min näpst uttorkar jag ju havet, och strömmarna gör jag till torrt land, så att fiskarna ruttna och dö av törst, eftersom vattnet är borta;
\par 3 själva himmelen kläder jag i mörker och giver den sorgdräkt att bära.
\par 4 Herren, HERREN har givit mig en tunga med lärdom, så att jag förstår att genom mina ord hugsvala den trötte; han väcker var morgon mitt öra, han väcker det till att höra på lärjungesätt.
\par 5 Ja, Herren, HERREN har öppnat mitt öra, och jag har ej varit gensträvig, jag har ej vikit tillbaka.
\par 6 Jag höll fram min rygg åt dem som slogo mig och mina kinder åt dem som ryckte mig i skägget; jag skylde icke mitt ansikte mot smädelse och spott.
\par 7 Men Herren, HERREN hjälper mig, därför kände jag ej smädelsen, därför gjorde jag min panna hård såsom sten; jag visste ju, att jag ej skulle komma på skam.
\par 8 Den som dömer mig fri är nära, vem vill då gå till rätta med mig? Må han träda fram jämte mig. Vem vill vara min anklagare? Må han komma hit till mig.
\par 9 Se, Herren, HERREN hjälper mig; vem vill då döma mig skyldig? Se, de skola allasammans falla sönder såsom en klädnad; mal skall förtära dem.
\par 10 Vem bland eder, som fruktar HERREN och hör hans tjänares röst? Om han än vandrar i mörkret och icke ser någon ljusning, så förtröste han dock på HERRENS namn och stödje sig vid sin Gud.
\par 11 Men se, I alla som tänden upp en brand och väpnen eder med glödande pilar, I hemfallen själva åt lågorna från eder brand och åt pilarna som I haven antänt. Av min hand skall detta vederfaras eder; i kval skolen I komma att ligga.

\chapter{51}

\par 1 Hören på mig, I som faren efter rättfärdighet, I som söken HERREN. Skåden på klippan, ur vilken I ären uthuggna, och på gruvan, ur vilken I haven framhämtats:
\par 2 ja, skåden på Abraham, eder fader, och på Sara som födde eder. Ty när han ännu var ensam, kallade jag honom och välsignade honom och förökade honom.
\par 3 Ja, HERREN skall varkunna sig över Sion, han skall varkunna sig över alla dess ruiner; han gör dess öken lik ett Eden och dess hedmark lik en HERRENS lustgård. Fröjd och glädje skall höras därinne, tacksägelse och lovsångs ljud.
\par 4 Akta på mig, du mitt folk; lyssna till mig, du min menighet. Ty från mig skall lag utgå, och min rätt skall jag sätta till ett ljus för folken.
\par 5 Min rättfärdighet är nära, min frälsning går fram, och mina armar skola skaffa rätt bland folken; havsländerna bida efter mig och hoppas på min arm.
\par 6 Lyften upp edra ögon till himmelen, skåden ock på jorden härnere: se, himmelen skall upplösa sig såsom rök och jorden nötas ut såsom en klädnad, och dess inbyggare skola dö såsom mygg; men min frälsning förbliver evinnerligen, och min rättfärdighet varder icke om intet.
\par 7 Hören på mig, I som kännen rättfärdigheten, du folk, som bär min lag i ditt hjärta; Frukten icke för människors smädelser och varen ej förfärade för deras hån.
\par 8 Ty mal skall förtära dem såsom en klädnad, och mott skall förtära dem såsom ull; men min rättfärdighet förbliver evinnerligen och min frälsning ifrån släkte till släkte.
\par 9 Vakna upp, vakna upp, kläd dig i makt, du HERRENS arm; vakna upp såsom i forna dagar, i förgångna tider. Var det icke du, som slog Rahab och genomborrade draken?
\par 10 Var det icke du, som uttorkade havet, det stora djupets vatten, och som gjorde havsbottnen till en väg, där ett frälsat folk kunde gå fram?
\par 11 Ja, HERRENS förlossade skola vända tillbaka och komma till Sion med jubel; evig glädje skall kröna deras huvuden, fröjd och glädje skola de undfå, sorg och suckan skola fly bort.
\par 12 Jag, jag är den som tröstar eder. Vem är då du, att du fruktar för dödliga människor, för människobarn som bliva såsom torrt gräs?
\par 13 Och därvid förgäter du HERREN, som har skapat dig, honom som har utspänt himmelen och lagt jordens grund. Ja, beständigt, dagen igenom, förskräckes du för förtryckarens vrede, såsom stode han just redo till att fördärva. Men vad bliver väl av förtryckarens vrede?
\par 14 Snart skall den fjättrade lösas ur sitt tvång; han skall icke dö och hemfalla åt graven, ej heller skall han lida brist på bröd.
\par 15 ty jag är HERREN, din Gud, han som rör upp havet, så att dess böljor brusa, han vilkens namn är HERREN Sebaot;
\par 16 och jag har lagt mina ord i din mun och övertäckt dig med min hands skugga för att plantera en himmel och grunda en jord och för att säga till Sion: Du är mitt folk.
\par 17 Vakna upp, vakna upp, stå upp, Jerusalem, du som av HERRENS hand har fått att dricka hans vredes bägare, ja, du som har tömt berusningens kalk till sista droppen.
\par 18 Bland alla de söner hon hade fött fanns ingen som ledde henne, bland alla de söner hon hade fostrat ingen som fattade henne vid handen.
\par 19 Dubbel är den olycka som har drabbat dig, och vem visar dig medlidande? Här är förödelse och förstöring, hunger och svärd. Huru skall jag trösta dig?
\par 20 Dina söner försmäktade, de lågo vid alla gathörn, lika antiloper i jägarens garn, drabbade i fullt mått av HERRENS vrede, av din Guds näpst.
\par 21 Därför må du höra detta, du arma, som är drucken, fastän icke av vin:
\par 22 Så säger HERREN, som är din Herre, och din Gud, som utför sitt folks sak: Se, jag tager bort ur din hand berusningens bägare; av min vredes kalk skall du ej vidare dricka.
\par 23 Och jag sätter den i dina plågares hand, deras som sade till dig: "Fall ned, så att vi få gå fram över dig"; och så nödgades du göra din rygg likasom till en mark och till en gata för dem som gingo där fram.

\chapter{52}

\par 1 Vakna upp, vakna upp, ikläd dig din makt, o Sion; ikläd dig din högtidsskrud, Jerusalem, du heliga stad; ty ingen oomskuren eller oren skall vidare komma in i dig.
\par 2 Skaka stoftet av dig, stå upp och intag din plats, Jerusalem; lös banden från din hals, du fångna dotter Sion.
\par 3 Ty så säger HERREN: I haven blivit sålda för intet; så skolen I ock utan penningar bliva lösköpta.
\par 4 Ja, så säger Herren, HERREN: Mitt folk drog i forna dagar ned till Egypten och bodde där såsom främlingar; sedan förtryckte Assur dem utan all rätt.
\par 5 Och vad skall jag nu göra här, säger HERREN, nu då man har fört bort mitt folk utan sak, nu då dess tyranner så skräna, säger HERREN, och mitt namn beständigt, dagen igenom, varder smädat?
\par 6 Jo, just därför skall mitt folk få lära känna mitt namn, just därför skall det förnimma på den dagen, att jag är den som talar; ja, se här är jag.
\par 7 Huru ljuvliga äro icke glädjebudbärarens fotsteg, när han kommer över bergen för att förkunna frid och frambära gott budskap och förkunna frälsning, i det han säger till Sion: "Din Gud är nu konung!"
\par 8 Hör, huru dina väktare upphäva sin röst och jubla allasammans, ty de se för sina ögon, huru HERREN vänder tillbaka till Sion.
\par 9 Ja, bristen ut i jubel tillsammans, I Jerusalems ruiner; ty HERREN tröstar sitt folk, han förlossar Israel.
\par 10 HERREN blottar sin heliga arm inför alla hedningars ögon, och alla jordens ändar få se vår Guds frälsning.
\par 11 Bort, bort, dragen ut därifrån, kommen icke vid det orent är; dragen ut ifrån henne, renen eder, I som bären HERRENS kärl.
\par 12 Se, I behöven icke draga ut med hast, icke vandra bort såsom flyktingar, ty HERREN går framför eder, och Israels Gud slutar edert tåg.
\par 13 Se, min tjänare skall hava framgång; han skall bliva upphöjd och stor och högt uppsatt.
\par 14 Såsom många häpnade över honom, därför att hans utseende var vanställt mer än andra människors och hans gestalt oansenligare än andra människobarns,
\par 15 så skall han ock väcka förundran hos många folk; ja, konungar skola förstummas i förundran över honom. Ty vad aldrig har varit förtäljt för dem, det få de se, och vad de aldrig hava hört, det få de förnimma.

\chapter{53}

\par 1 Men vem trodde, vad som predikades för oss, och för vem var HERRENS arm uppenbar?
\par 2 Han sköt upp såsom en ringa telning inför honom, såsom ett rotskott ur förtorkad jord. Han hade ingen gestalt eller fägring; när vi sågo på honom, kunde hans utseende ej behaga oss.
\par 3 Föraktat var han och övergiven av människor, en smärtornas man och förtrogen med krankhet; han var såsom en, för vilken man skyler sitt ansikte, så föraktat, att vi höllo honom för intet.
\par 4 Men det var våra krankheter han bar, våra smärtor, dem lade han på sig, medan vi höllo honom för att vara hemsökt, tuktad av Gud och pinad.
\par 5 Ja, han var sargad för våra överträdelsers skull och slagen för våra missgärningars skull; näpsten var lagd på honom, för att vi skulle få frid, och genom hans sår bliva vi helade.
\par 6 Vi gingo alla vilse såsom får, var och en av oss ville vandra sin egen väg, men HERREN lät allas vår missgärning drabba honom.
\par 7 Han blev plågad, fastän han ödmjukade sig och icke öppnade sin mun, lik ett lamm, som föres bort att slaktas, och lik ett får, som är tyst inför dem som klippa det ja, han öppnade icke sin mun.
\par 8 Undan våld och dom blev han borttagen, men vem i hans släkte betänker detta? Ja, han rycktes bort ifrån de levandes land, och för mitt folks överträdelses skull kom plåga över honom.
\par 9 Och bland de ogudaktiga fick hans in grav bland de rika kom han först, när han var död fastän han ingen orätt hade gjort och fastän svek icke fanns i hans mun.
\par 10 Det behagade HERREN att slå honom med krankhet: om hans liv så bleve ett skuldoffer, då skulle han få se avkomlingar och länge leva, och HERRENS vilja skulle genom honom hava framgång.
\par 11 Ja, av den vedermöda hans själ har utstått skall han se frukt och så bliva mättad; genom sin kunskap skall han göra många rättfärdiga, han, den rättfärdige, min tjänare, i det han bär deras missgärningar.
\par 12 Därför skall jag tillskifta honom hans lott bland de många, och med talrika skaror skall han få utskifta byte, eftersom han utgav sitt liv i döden och blev räknad bland överträdare, han som bar mångas synder och bad för överträdarna.

\chapter{54}

\par 1 Jubla, du ofruktsamma, du som icke har fött barn; brist ut i jubel och ropa av fröjd, du som icke har blivit moder. Ty den ensamma skall hava många barn, flera än den som har man, säger HERREN.
\par 2 Vidga ut platsen för ditt tjäll, låt spänna ut tältet, under vilket du bor, och spar icke; förläng dina tältstreck och gör dina tältpluggar fastare.
\par 3 Ty du skall utbreda dig både åt höger och vänster, och dina avkomlingar skola taga hedningarnas länder i besittning och åter befolka ödelagda städer.
\par 4 Frukta icke, ty du skall ej komma på skam; blygs icke, ty du skall ej varda utskämd. Nej, du skall få förgäta din ungdoms skam, och ditt änkestånds smälek skall du icke mer komma ihåg.
\par 5 Ty den som har skapat dig är din man, han vilkens namn är HERREN Sebaot; och Israels Helige är din förlossare, han som kallas hela jordens Gud.
\par 6 Ty såsom en övergiven kvinna i hjärtesorg kallades du av HERREN. Sin ungdomsbrud, vill någon förskjuta henne? säger din Gud.
\par 7 Ett litet ögonblick övergav jag dig, men i stor barmhärtighet vill jag åter församla dig.
\par 8 I min förtörnelses översvall dolde jag ett ögonblick mitt ansikte för dig, men med evig nåd vill jag nu förbarma mig över dig, säger HERREN, din förlossare.
\par 9 Ty såsom jag gjorde vid Noas flod, så gör jag ock nu: såsom jag då svor, att Noas flod icke mer skulle komma över jorden, så svär jag ock nu, att jag icke mer skall förtörnas på dig eller näpsa dig.
\par 10 Ja, om än bergen vika bort och höjderna vackla, så skall min nåd icke vika ifrån dig och mitt fridsförbund icke vackla, säger HERREN; din förbarmare.
\par 11 Du arma, som har blivit så hemsökt av stormar utan att få någon tröst, se, med spetsglans vill jag nu mura dina stenar och giva dig grundvalar av safirer,
\par 12 jag vill göra dina tinnar av rubiner och dina portar av kristall och hela din ringmur av ädla stenar.
\par 13 Och dina barn skola alla bliva HERRENS lärjungar, och stor frid skola dina barn då hava.
\par 14 Genom rättfärdighet skall du bliva befäst. All tanke på förtryck vare fjärran ifrån dig, ty du skall intet hava att frukta, och all tanke på fördärv, ty sådant skall icke nalkas dig.
\par 15 Om man då rotar sig samman till anfall, så kommer det ingalunda från mig; och vilka de än äro, som rota sig samman mot dig, så skola de falla för dig.
\par 16 Se, jag är den som skapar smeden, vilken blåser upp kolelden och så frambringar ett vapen, sådant han vill göra det; men jag är ock den som skapar fördärvaren, vilken förstör det.
\par 17 Och nu skall intet vapen, som smides mot dig, hava någon lycka; var tunga, som upphäver sig för att gå till rätta med dig, skall du få domfälld. Detta är HERRENS tjänares arvedel, den rätt de skola undfå av mig, säger HERREN.

\chapter{55}

\par 1 Upp, alla I som ären törstiga, kommen hit och fån vatten; och I som inga penningar haven, kommen hit och hämten säd och äten. Ja, kommen hit och hämten säd utan penningar och för intet både vin och mjölk.
\par 2 Varför given I ut penningar för det som ej är bröd och edert förvärv för det som icke kan mätta? Hören på mig, så skolen I få äta det gott är och förnöja eder med feta rätter.
\par 3 Böjen edra öron hit och kommen till mig; hören, så får eder själ leva. Jag vill sluta med eder ett evigt förbund: att I skolen undfå all den trofasta nåd jag har lovat David.
\par 4 Se, honom har jag satt till ett vittne för folken, till en furste och hövding för folken.
\par 5 Ja, du skall kalla på folkslag som du icke känner, och folkslag, som icke känna dig, skola hasta till dig för HERRENS, din Guds, skull, för Israels Heliges skull, när han förhärligar dig.
\par 6 Söken HERREN, medan han låter sig finnas; åkallen honom, medan han är nära.
\par 7 Den ogudaktige övergive sin väg och den orättfärdige sina tankar och vände om till HERREN, så skall han förbarma sig över honom, och till vår Gud, ty han skall beskära mycken förlåtelse.
\par 8 Se, mina tankar äro icke edra tankar, och edra vägar äro icke mina vägar, säger HERREN.
\par 9 Nej, så mycket som himmelen är högre än jorden, så mycket äro ock mina vägar högre än edra vägar och mina tankar högre än edra tankar.
\par 10 Ty likasom regnet och snön faller ifrån himmelen och icke vänder tillbaka dit igen, förrän det har vattnat jorden och gjort den fruktsam och bärande, så att den giver säd till att så och bröd till att äta,
\par 11 så skall det ock vara med ordet som utgår ur min mun; det skall icke vända tillbaka till mig fåfängt utan att hava verkat, vad jag vill, och utfört det, vartill jag hade sänt ut det.
\par 12 Ty med glädje skolen I draga ut, och i frid skolen I föras åstad. Bergen och höjderna skola brista ut i jubel, där I gån fram, och alla träd på marken skola klappa i händerna.
\par 13 Där törnsnår nu finnas skola cypresser växa upp, och där nässlor stå skall myrten uppväxa. Och detta skall bliva HERREN till ära och ett evigt tecken, som ej skall plånas ut.

\chapter{56}

\par 1 Så säger HERREN: Akten på vad rätt är och öven rättfärdighet, ty min frälsning kommer snart, och snart bliver min rättfärdighet uppenbarad.
\par 2 Säll är den människa, som gör detta, den människoson, som står fast därvid, den som håller sabbaten, så att han icke ohelgar den, och den som avhåller sin hand från att göra något ont.
\par 3 Främlingen, som har slutit sig till HERREN, må icke säga så: "Säkert skall HERREN avskilja mig från sitt folk." Ej heller må den snöpte säga: "Se, jag är ett förtorkat träd."
\par 4 Ty så säger HERREN: De snöpta, som hålla mina sabbater och utvälja det mig behagar och stå fast vid mitt förbund,
\par 5 åt dem skall jag i mitt hus och inom mina murar giva en åminnelse och ett namn, en välsignelse, som är förmer än söner och döttrar; jag skall giva dem ett evigt namn, som icke skall varda utrotat.
\par 6 Och främlingarna, som hava slutit sig till HERREN för att tjäna honom och för att älska HERRENS namn och så vara hans tjänare, alla som hålla sabbaten, så att de icke ohelga den, och som stå fast vid mitt förbund,
\par 7 dem skall jag låta komma till mitt heliga berg och giva dem glädje i mitt bönehus, och deras brännoffer och slaktoffer skola vara mig välbehagliga på mitt altare; ty mitt hus skall kallas ett bönehus för alla folk.
\par 8 Så säger Herren, HERREN, han som församlar de fördrivna av Israel: Jag skall församla ännu flera till honom, utöver dem som redan äro församlade till honom.
\par 9 I alla djur på marken, kommen och äten, ja, I alla skogens djur.
\par 10 Väktarna här äro allasammans blinda, de hava intet förstånd; de äro allasammans stumma hundar, som icke kunna skälla; de ligga och drömma och vilja gärna slumra.
\par 11 Men de hundarna äro ock glupska och kunna ej bliva mätta. Ja, sådana människor äro herdar, dessa som intet kunna förstå! De vilja allasammans vandra sin egen väg; var och en söker sin egen vinning, alla, så många de äro.
\par 12 "Kommen, jag skall hämta vin, och så skola vi dricka oss druckna av starka drycker. Och morgondagen skall bliva denna dag lik, en övermåttan härlig dag!"

\chapter{57}

\par 1 Den rättfärdige förgås, och ingen finnes, som tänker därpå; fromma människor ryckas bort, utan att någon lägger märke därtill. Ja, genom ondskans makt ryckes den rättfärdige bort
\par 2 och går då in i friden; de som hava vandrat sin väg rätt fram få ro i sina vilorum.
\par 3 Men träden fram hit, I söner av teckentyderskor, I barn av äktenskapsbrytare och skökor.
\par 4 Över vem gören I eder lustiga? Mot vem spärren I upp munnen och räcken I ut tungan? Sannerligen, I ären överträdelsens barn, en lögnens avföda,
\par 5 I som upptändens av brånad vid terebinterna, ja, under alla gröna träd, I som slakten edra barn i dalarna, i bergsklyftornas djup.
\par 6 Stenarna i din dal har du till din del, de, just de äro din lott; också åt dem utgjuter du drickoffer och frambär du spisoffer. Skulle jag giva mig till freds vid sådant?
\par 7 På höga och stora berg redde du dig läger; också upp på sådana begav du dig för att offra slaktoffer.
\par 8 Och bakom dörren och dörrposten satte du ditt märke. Du övergav mig; du klädde av dig och besteg ditt läger och beredde plats där. Du gjorde upp med dem, gärna delade du läger med dem vid första vink du såg.
\par 9 Du begav dig till Melek med olja och tog med dig dina många salvor; du sände dina budbärare till fjärran land, ja, ända ned till dödsriket.
\par 10 Om du än blev trött av din långa färd, sade du dock icke: "Förgäves!" Så länge du kunde röra din hand, mattades du icke.
\par 11 För vem räddes och fruktade du då, eftersom du var så trolös och eftersom du icke tänkte på mig och ej ville akta på? Är det icke så: eftersom jag har tegat, och det sedan länge, därför fruktar du mig icke?
\par 12 Men jag skall visa, huru det är med din rättfärdighet och med dina verk, de skola icke hjälpa dig.
\par 13 När du ropar, då må ditt avgudafölje rädda dig. Nej, en vind skall taga dem med sig allasammans och en fläkt föra dem bort. Men den som tager sin tillflykt till mig skall få landet till arvedel och få besitta mitt heliga berg.
\par 14 Ja, det skall heta: "Banen väg, banen och bereden väg; skaffen bort stötestenarna från mitt folks väg."
\par 15 Ty så säger den höge och upphöjde, han som tronar till evig tid och heter "den Helige": Jag bor i helighet uppe i höjden, men ock hos den som är förkrossad och har en ödmjuk ande; ty jag vill giva liv åt de ödmjukas ande och liv åt de förkrossades hjärtan.
\par 16 Ja, jag vill icke evinnerligen gå till rätta och icke ständigt förtörnas; eljest skulle deras ande försmäkta inför mig, de själar, som jag själv har skapat.
\par 17 För hans girighetssynd förtörnades jag; jag slog honom, och i min förtörnelse höll jag mig dold. Men i sin avfällighet fortfor han att vandra på sitt hjärtas väg.
\par 18 Hans vägar har jag sett, men nu vill jag hela honom och leda honom och giva honom och hans sörjande tröst.
\par 19 Jag skall skapa frukt ifrån hans läppar. Frid över dem som äro fjärran och frid över dem som äro nära! säger HERREN; jag skall hela honom.
\par 20 Men de ogudaktiga äro såsom ett upprört hav, ett som icke kan vara stilla, ett hav, vars vågor röra upp dy och orenlighet.
\par 21 De ogudaktiga hava ingen frid, säger min Gud.

\chapter{58}

\par 1 Ropa med full hals utan återhåll, häv upp din röst såsom en basun och förkunna för mitt folk deras överträdelse, för Jakobs hus deras synder.
\par 2 Väl söka de mig dag ut och dag in och vilja hava kunskap om mina vägar. Såsom vore de ett folk, som övade rättfärdighet och icke övergåve sin Guds rätt, så fråga de mig om rättfärdighetens rätter och vilja, att Gud skall komma till dem:
\par 3 "Vartill gagnar det, att vi fasta, när du icke ser det, vartill, att vi späka oss, när du icke märker det?" Men se, på edra fastedagar sköten I edra sysslor, och alla edra arbetare driven I blott på.
\par 4 Och se, I hållen eder fasta med kiv och split, med hugg och slag av gudlösa nävar. I hållen icke mer fasta på sådant sätt, att I kunnen göra eder röst hörd i höjden.
\par 5 Skulle detta vara en fasta, sådan som jag vill hava? Skulle detta vara en rätt späkningsdag? Att man hänger med huvudet såsom ett sävstrå och sätter sig i säck och aska, vill du kalla sådant att hålla fasta, att fira en dag till HERRENS behag?
\par 6 Nej, detta är den fasta, som jag vill hava: att I lossen orättfärdiga bojor och lösen okets band, att I given de förtryckta fria och krossen sönder alla ok,
\par 7 ja, att du bryter ditt bröd åt den hungrige och skaffar de fattiga och husvilla härbärge att du kläder den nakne, var du ser honom, och ej drager dig undan för den som är ditt kött och blod.
\par 8 Då skall ljus bryta fram för dig såsom en morgonrodnad, och dina sår skola läkas med hast, och din rätt skall då gå framför dig och HERRENS härlighet följa dina spår.
\par 9 Då skall HERREN svara, när du åkallar honom; när du ropar, skall han säga: "Se, här är jag." Om hos dig icke får finnas någon som pålägger ok och pekar finger och talar, vad fördärvligt är,
\par 10 om du delar med dig av din nödtorft åt den hungrige och mättar den som är i betryck, då skall ljus gå upp för dig i mörkret, och din natt skall bliva lik middagens sken.
\par 11 Och HERREN skall leda dig beständigt; han skall mätta dig mitt i ödemarken och giva styrka åt benen i din kropp. Och du skall vara lik en vattenrik trädgård och likna ett källsprång, vars vatten aldrig tryter.
\par 12 Och dina avkomlingar skola bygga upp de gamla ruinerna, du skall åter upprätta grundvalar ifrån forna släkten; och du skall kallas "han som murar igen revor", "han som återställer stigar, så att man kan bo i landet."
\par 13 Om du är varsam med din fot på sabbaten, så att du icke på min heliga dag utför dina sysslor; om du kallar sabbaten din lust och HERRENS helgdag en äredag, ja, om du ärar den, så att du icke går dina egna vägar eller sköter dina sysslor eller talar tomma ord,
\par 14 då skall du finna din lust i HERREN, och jag skall föra dig fram över landets höjder och giva dig till näring din fader Jakobs arvedel. Ja, så har HERRENS mun talat.

\chapter{59}

\par 1 Se, HERRENS arm är icke för kort, så att han ej kan frälsa, och hans öra är icke tillslutet, så att han ej kan höra.
\par 2 Nej, det är edra missgärningar, som skilja eder och eder Gud från varandra, och edra synder dölja hans ansikte för eder, så att han icke hör eder.
\par 3 Ty edra händer äro fläckade av blod och edra fingrar av missgärning, edra läppar tala lögn, och eder tunga frambär orättfärdighet.
\par 4 Ingen höjer sin röst i rättfärdighetens namn, och ingen visar redlighet i vad till rätten hör. De förtrösta på idel tomhet, de tala falskhet, de gå havande med olycka och föda fördärv.
\par 5 De kläcka ut basiliskägg och väva spindelnät. Om någon äter av deras ägg, så dör han, och trampas ett sådant sönder, så kommer en huggorm ut.
\par 6 Deras spindelnät duga icke till kläder, och de kunna ej skyla sig med vad de hava tillverkat; deras verk äro fördärvliga verk, och våldsgärningar öva deras händer.
\par 7 Deras fötter hasta till vad ont är och äro snara, när det gäller att utgjuta oskyldigt blod; deras tankar äro fördärvliga tankar, förödelse och förstöring är på deras vägar.
\par 8 Fridens väg känna de icke, och rätten följer ej i deras spår; de gå krokiga stigar, och ingen som vandrar så vet, vad frid är.
\par 9 Därför är rätten fjärran ifrån oss, och rättfärdighet tillfaller oss icke; vi bida efter ljus, men se, mörker råder, efter solsken, men vi få vandra i djupaste natt.
\par 10 Vi måste famla utefter väggen såsom blinda, famla, såsom hade vi inga ögon; vi stappla mitt på dagen, såsom vore det skymning, mitt i vår fulla kraft äro vi såsom döda.
\par 11 Vi brumma allasammans såsom björnar och sucka alltjämt såsom duvor; vi bida efter rätten, men den kommer icke, efter frälsningen, men den är fjärran ifrån oss.
\par 12 Ty många äro våra överträdelser inför dig, och våra synder vittna emot oss; ja, våra överträdelser hava vi för våra ögon, och våra missgärningar känna vi.
\par 13 Genom överträdelse och förnekelse hava vi felat mot HERREN, vi hava vikit bort ifrån vår Gud; vi hava talat förtryck och avfällighet, lögnläror hava vi förkunnat och hämtat fram ur våra hjärtan.
\par 14 Rätten tränges tillbaka, och rättfärdigheten står långt borta, ja, sanningen vacklar på torget, och vad rätt är kan ej komma fram.
\par 15 Så måste sanningen hålla sig undan, och den som vände sig ifrån det onda blev plundrad. Detta såg HERREN, och det misshagade honom, att det icke fanns någon rätt.
\par 16 Och han såg, att ingen trädde fram; han förundrade sig över att ingen grep in. Då hjälpte honom hans egen arm, och hans rättfärdighet understödde honom.
\par 17 Och han klädde sig i rättfärdighet såsom i ett pansar och satte frälsningens hjälm på sitt huvud; han klädde sig i hämndens dräkt såsom i en livklädnad och höljde sig i nitälskan såsom i en mantel.
\par 18 Efter deras gärningar skall han nu vedergälla dem; vrede skall han låta komma över sina ovänner och över sina fiender lönen för vad de hava gjort; ja, havsländerna skall han vedergälla, vad de hava gjort.
\par 19 Så skall HERRENS namn bliva fruktat i väster och hans härlighet, där solen går upp. När fienden bryter fram lik en ström, skall HERRENS andedräkt förjaga honom.
\par 20 Men såsom en förlossare kommer HERREN för Sion och för dem i Jakob, som omvända sig från sin överträdelse, säger HERREN.
\par 21 Och detta är det förbund, som jag å min sida gör med dem, säger HERREN: min Ande, som är över dig, och orden, som jag har lagt i din mun, de skola icke vika ur din mun, ej heller ur dina barns eller barnbarns mun från nu och till evig tid, säger HERREN.

\chapter{60}

\par 1 Stå upp, var ljus, ty ditt ljus kommer, och HERRENS härlighet går upp över dig.
\par 2 Se, mörker övertäcker jorden och töcken folken, men över dig uppgår HERREN, och hans härlighet uppenbaras över dig.
\par 3 Och folken skola vandra i ditt ljus och konungarna i glansen som går upp över dig.
\par 4 Lyft upp dina ögon och se dig omkring: alla komma församlade till dig; dina söner komma fjärran ifrån, och dina döttrar bäras fram på armen.
\par 5 Då, vid den synen skall du stråla av fröjd, och ditt hjärta skall bäva och vidga sig; ty havets rikedomar skola föras till dig, och folkens skatter skola falla dig till.
\par 6 Skaror av kameler skola övertäcka dig, kamelfålar från Midjan och Efa; från Saba skola de alla komma, guld och rökelse skola de bära och skola förkunna HERRENS lov.
\par 7 Alla Kedars hjordar skola församlas till dig, Nebajots vädurar skola vara dig till tjänst. Mig till välbehag skola de offras på mitt altare, och min härlighets hus skall jag så förhärliga.
\par 8 Vilka äro dessa som komma farande lika moln, lika duvor, som flyga till sitt duvslag?
\par 9 Se, havsländerna bida efter mig, och främst komma Tarsis' skepp; de vilja föra dina söner hem ifrån fjärran land, och de hava med sig silver och guld åt HERRENS, din Guds, namn, åt Israels Helige, ty han förhärligar dig.
\par 10 Och främlingar skola bygga upp dina murar, och deras konungar skola betjäna dig. Ty väl har jag slagit dig i min förtörnelse, men i min nåd förbarmar jag mig nu över dig.
\par 11 Och dina portar skola hållas öppna beständigt, varken dag eller natt skola de stängas, så att folkens skatter kunna föras in i dig, med deras konungar i hyllningståget.
\par 12 Ty det folk eller rike, som ej vill tjäna dig, skall förgås; ja, sådana folk skola i grund förgöras.
\par 13 Libanons härlighet skall komma till dig, både cypress och alm och buxbom, för att pryda platsen, där min helgedom är; ty den plats, där mina fötter stå, vill jag göra ärad.
\par 14 Och bugande skola dina förtryckares söner komma till dig, och dina föraktare skola allasammans falla ned för dina fötter. Och man skall kalla dig "HERRENS stad", "Israels Heliges Sion".
\par 15 I stället för att du var övergiven och hatad, så att ingen ville taga vägen genom dig, skall jag göra dig till en härlighetens boning evinnerligen och till en fröjdeort ifrån släkte till släkte.
\par 16 Och du skall dia folkens mjölk, ja, konungabröst skall du dia; och du skall förnimma, att jag, HERREN, är din frälsare och att den Starke i Jakob är din förlossare.
\par 17 Jag skall låta guld komma i stället för koppar och låta silver komma i stället för järn och koppar i stället för trä och järn i stället för sten. Och jag vill sätta frid till din överhet och rättfärdighet till din behärskare.
\par 18 Man skall icke mer höra talas om våld i ditt land, om ödeläggelse och förstöring inom dina gränser, utan du skall kalla dina murar för "frälsning" och dina portar för "lovsång".
\par 19 Solen skall icke mer vara ditt ljus om dagen, och månen skall icke mer lysa dig med sitt sken, utan HERREN skall vara ditt eviga ljus, och din Gud skall vara din härlighet.
\par 20 Din sol skall då icke mer gå ned och din måne icke mer taga av; ty HERREN skall vara ditt eviga ljus, och dina sorgedagar skola hava en ände.
\par 21 Och i ditt folk skola alla vara rättfärdiga, evinnerligen skola de besitta landet; de äro ju en telning, som jag har planterat, ett verk av mina händer, som jag vill förhärliga mig med.
\par 22 Av den minste skola komma tusen, och av den ringaste skall bliva ett talrikt folk. Jag är HERREN; när tiden är inne, skall jag med hast fullborda detta.

\chapter{61}

\par 1 Herrens, HERRENS Ande är över mig, ty HERREN har smort mig till att förkunna glädjens budskap för de ödmjuka; han har sänt mig till att läka dem som hava ett förkrossat hjärta, till att predika frihet för de fångna och förlossning för de bundna,
\par 2 till att predika ett nådens år från HERREN och en hämndens dag från vår Gud, en dag, då han skall trösta alla sörjande,
\par 3 då han skall låta de sörjande i Sion få huvudprydnad i stället för aska, glädjeolja i stället för sorg, högtidskläder i stället för en bedrövad ande; och de skola kallas "rättfärdighetens terebinter", "HERRENS plantering, som han vill förhärliga sig med".
\par 4 Och de skola bygga upp de gamla ruinerna och upprätta förfädernas ödeplatser; de skola återställa de förödda städerna, de platser, som hava legat öde släkte efter släkte.
\par 5 Främlingar skola stå redo att föra edra hjordar i bet, och utlänningar skola bruka åt eder åkrar och vingårdar.
\par 6 Men I skolen heta HERRENS präster, och man skall kalla eder vår Guds tjänare; I skolen få njuta av folkens skatter, och deras härlighet skall övergå till eder.
\par 7 För eder skam skolen I få dubbelt igen, och de som ledo smälek skola nu jubla över sin del. Så skola de få dubbelt att besitta i sitt land; evig glädje skola de undfå.
\par 8 Ty jag, HERREN, älskar, vad rätt är, och hatar orättfärdigt rov; och jag skall giva dem deras lön i trofasthet och sluta ett evigt förbund med dem.
\par 9 Och deras släkte skall bliva känt bland folken och deras avkomma bland folkslagen; alla som se dem skola märka på dem, att de äro ett släkte, som HERREN har välsignat.
\par 10 Jag gläder mig storligen i HERREN, och min själ fröjdar sig i min Gud, ty han har iklätt mig frälsningens klädnad och höljt mig i rättfärdighetens mantel, likasom när en brudgum sätter högtidsbindeln på sitt huvud eller likasom när en brud pryder sig med sina smycken.
\par 11 Ty likasom jorden låter sina växter spira fram och en trädgård sin sådd växa upp, så skall Herren, HERREN låta rättfärdighet uppväxa och lovsång inför alla folk.

\chapter{62}

\par 1 För Sions skull vill jag icke tiga, och för Jerusalems skull vill jag ej unna mig ro, förrän dess rätt går upp såsom solens sken och dess frälsning lyser såsom ett brinnande bloss.
\par 2 Och folken skola se din rätt och alla konungar din härlighet; och du skall få ett nytt namn, som HERRENS mun skall bestämma.
\par 3 Så skall du vara en härlig krona i HERRENS hand, en konungslig huvudbindel i din Guds hand.
\par 4 Du skall icke mer kallas "den övergivna", ej heller skall ditt land mer kallas "ödemark", utan du skall få heta "hon som jag har min lust i", och ditt land skall få heta "äkta hustrun"; ty HERREN har sin lust i dig, och ditt land har fått sin äkta man.
\par 5 Ty såsom när en ung man bliver en jungrus äkta herre, så skola dina barn bliva dina äkta herrar, och såsom en brudgum fröjdar sig över sin brud, så skall din Gud fröjda sig över dig.
\par 6 På dina murar, Jerusalem, har jag ställt väktare; varken dag eller natt få de någonsin tystna. I som skolen ropa till HERREN, given eder ingen ro.
\par 7 Och given honom ingen ro förrän han åter har byggt upp Jerusalem och låtit det bliva ett ämne till lovsång på jorden.
\par 8 HERREN har svurit vid sin högra hand och sin starka arm: Jag skall icke mer giva din säd till mat åt dina fiender, och främlingar skola icke dricka ditt vin, frukten av din möda.
\par 9 Nej, de som insamla säden skola ock äta den och skola lova HERREN, och de som inbärga vinet skola dricka det i min helgedoms gårdar.
\par 10 Dragen ut, dragen ut genom portarna, bereden väg för folket; banen, ja, banen en farväg rensen den från stenar, resen upp ett baner för folken.
\par 11 Hör, HERREN höjer ett rop, och det når till jordens ända: Sägen till dottern Sion: Se, din frälsning kommer. Se, han har med sig sin lön, och hans segerbyte går framför honom.
\par 12 Och man skall kalla dem "det heliga folket", "HERRENS förlossade"; och dig själv skall man kalla "den mångbesökta staden", "staden, som ej varder övergiven".

\chapter{63}

\par 1 Vem är han som kommer från Edom, från Bosra i högröda kläder, så präktig i sin dräkt, så stolt i sin stora kraft? "Det är jag, som talar i rättfärdighet, jag, som är en mästare till att frälsa."
\par 2 Varför är din dräkt så röd? Varför likna dina kläder en vintrampares?
\par 3 "Jo, en vinpress har jag trampat, jag själv allena, och ingen i folken bistod mig. Jag trampade dem i min vrede, trampade sönder dem i min förtörnelse. Då stänkte deras blod på mina kläder, och så fick jag hela min dräkt nedfläckad.
\par 4 Ty en hämndedag hade jag beslutit, och mitt förlossningsår hade kommit.
\par 5 Och jag skådade omkring mig, men ingen hjälpare fanns; jag stod där i förundran, men ingen fanns, som understödde mig. Då hjälpte mig min egen arm, och min förtörnelse understödde mig.
\par 6 Jag trampade ned folken i min vrede och gjorde dem druckna i min förtörnelse, och jag lät deras blod rinna ned på jorden."
\par 7 HERRENS nådegärningar vill jag förkunna, ja, HERRENS lov, efter allt vad HERREN har gjort mot oss, den nåderike mot Israels hus, vad han har gjort mot dem efter sin barmhärtighet och sin stora nåd.
\par 8 Ty han sade: "De äro ju mitt folk, barn, som ej svika." Och så blev han deras frälsare.
\par 9 I all deras nöd var ingen verklig nöd, ty hans ansiktes ängel frälste dem. Därför att han älskade dem och ville skona dem, förlossade han dem. Han lyfte dem upp och bar dem alltjämt, i forna tider.
\par 10 Men de voro gensträviga, och de bedrövade hans heliga Ande; därför förvandlades han till deras fiende, han själv stridde mot dem.
\par 11 Då tänkte hans folk på forna tider, de tänkte på Mose: Var är nu han som förde dem upp ur havet, jämte herdarna för hans hjord? Var är han som lade i deras bröst sin helige Ande,
\par 12 var är han som lät sin härliga arm gå fram vid Moses högra sida, han som klöv vattnet framför dem och så gjorde sig ett evigt namn,
\par 13 han som lät dem färdas genom djupen, såsom hästar färdas genom öknen, utan att stappla?
\par 14 Likasom när boskapen går ned i dalen så fördes de av HERRENS Ande till ro. Ja, så ledde du ditt folk och gjorde dig ett härligt namn.
\par 15 Skåda ned från himmelen och se härtill från din heliga och härliga boning. Var äro nu din nitälskan och dina väldiga gärningar, var är ditt hjärtas varkunnsamhet och din barmhärtighet? De hålla sig tillbaka från mig.
\par 16 Du är ju dock vår fader; ty Abraham vet icke av oss, och Israel känner oss icke. Men du, HERRE, är vår fader; "vår förlossare av evighet", det är ditt namn.
\par 17 Varför, o HERRE, låter du oss då gå vilse från dina vägar och förhärdar våra hjärtan, så att vi ej frukta dig? Vänd tillbaka för dina tjänares skull, för din arvedels stammars skull.
\par 18 Allenast helt kort fick ditt heliga folk behålla sin besittning; våra ovänner trampade ned din helgedom.
\par 19 Det är oss nu så, som om du aldrig hade varit herre över oss, om om vi ej hade blivit uppkallade efter ditt namn.

\chapter{64}

\par 1 O att du läte himmelen rämna och fore hitned, så att bergen skälvde inför dig,
\par 2 likasom när ris antändes av eld och vatten genom eld bliver sjudande, så att du gjorde ditt namn kunnigt bland dina ovänner och folken darrade för dig!
\par 3 O att du fore hitned med underbara gärningar som vi icke kunde vänta, så att bergen skälvde inför dig!
\par 4 Aldrig någonsin har man ju hört, aldrig har något öra förnummit, aldrig har något öga sett en annan Gud än dig handla så mot dem som vänta efter honom.
\par 5 Du kom dem till mötes, som övade rättfärdighet med fröjd, dem som på dina vägar tänkte på dig. Men se, du blev förtörnad, och vi stodo där såsom syndare. Så hava vi länge stått; skola vi väl bliva frälsta?
\par 6 Vi blevo allasammans lika orena människor, och all vår rättfärdighet var såsom en fläckad klädnad. Vi vissnade allasammans såsom löv, och våra missgärningar förde oss bort såsom vinden.
\par 7 Ingen fanns, som åkallade ditt namn, ingen, som vaknade upp för att hålla sig till dig; ty du dolde ditt ansikte för oss och lät oss försmäkta genom vår missgärning.
\par 8 Men HERRE, du är ju vår fader; vi äro leret, och du är den som har danat oss, vi äro allasammans verk av din hand.
\par 9 Var då ej så högeligen förtörnad, HERRE; och tänk icke evinnerligen på vår missgärning; nej, se därtill att vi allasammans äro ditt folk.
\par 10 Dina heliga städer hava blivit en öken, Sion har blivit en öken, Jerusalem en ödemark.
\par 11 Vårt heliga och härliga tempel, där våra fäder lovade dig, det har blivit uppbränt i eld; och allt vad dyrbart vi ägde har lämnats åt förödelsen.
\par 12 Kan du vid allt detta hålla dig tillbaka, o HERRE? Kan du tiga stilla och plåga oss så svårt?

\chapter{65}

\par 1 Jag har låtit mig bliva uppenbar för dem som icke frågade efter mig, jag har låtit mig finnas av dem som icke sökte mig; till ett folk som icke var uppkallat efter mitt namn har jag sagt: Se, här är jag, här är jag.
\par 2 Hela dagen har jag uträckt mina händer till ett gensträvigt folk som vandrar på den väg som icke är god, i det att de följa sina egna tankar -
\par 3 ett folk, som beständigt förtörnar mig utan att hava någon försyn, som frambär offer i lustgårdar och tänder offereld på tegelaltaren,
\par 4 som har sitt tillhåll bland gravar och tillbringar natten i undangömda nästen, som äter svinens kött och har vederstygglig spis i sina kärl,
\par 5 som säger: "Bort med dig, kom icke vid mig, ty jag är helig för dig." De äro såsom rök i min näsa, en eld, som brinner beständigt.
\par 6 Se, detta står upptecknat inför mina ögon; jag skall icke tiga, förrän jag har givit vedergällning, ja, vedergällning i deras sköte,
\par 7 både för deras egna missgärningar och för deras fäders, säger HERREN, vedergällning för att de tände offereld på bergen och för att de smädade mig på höjderna; ja, först skall jag mäta upp lönen åt dem i deras sköte.
\par 8 Så säger HERREN: Likasom man säger om en druvklase, när däri finnes saft: "Fördärva den icke, ty välsignelse är däri", så skall ock jag göra för mina tjänares skull: jag skall icke fördärva alltsammans.
\par 9 Jag skall låta en avkomma utgå från Jakob, från Juda en arvinge till mina berg; ty mina utkorade skola besitta landet, och mina tjänare skola bo däri.
\par 10 Saron skall bliva en betesmark för får och Akors dal en lägerplats för fäkreatur, och de skola givas åt mitt folk, när det söker mig.
\par 11 Men I som övergiven HERREN och förgäten mitt heliga berg, I som duken bord åt Gad och iskänken vindryck åt Meni,
\par 12 eder har jag bestämt åt svärdet, och I skolen alla få böja eder ned till att slaktas, därför att I icke svaraden, när jag kallade, och icke hörden, när jag talade, utan gjorden, vad ont var i mina ögon, och utvalden det som var mig misshagligt.
\par 13 Därför säger Herren, HERREN så: Se, mina tjänare skola äta, men I skolen hungra; se, mina tjänare skola dricka, men I skolen törsta; se, mina tjänare skola glädjas, men I skolen få blygas.
\par 14 Ja, mina tjänare skola jubla i sitt hjärtas fröjd, men I skolen ropa i edert hjärtas plåga och jämra eder i förtvivlan.
\par 15 Och I skolen lämna edert namn till ett förbannelsens ord, så att mina utkorade skola säga: "Sådan död give dig Herren, HERREN." Men åt sina tjänare skall han giva ett annat namn:
\par 16 den som då välsignar sig i landet skall välsigna sig i "den sannfärdige Guden", och den som svär i landet, han skall svärja vid "den sannfärdige Guden". Ty de förra bedrövelserna äro då förgätna och dolda för mina ögon.
\par 17 Ty se, jag vill skapa nya himlar och en ny jord; och man skall ej mer komma ihåg det förgångna eller tänka därpå.
\par 18 Nej, I skolen fröjdas och jubla till evig tid över det som jag skapar; ty se, jag vill skapa Jerusalem till jubel och dess folk till fröjd.
\par 19 Och jag skall jubla över Jerusalem och fröjda mig över mitt folk, och där skall icke mer höras gråt eller klagorop.
\par 20 Där skola icke mer finnas barn som leva allenast några dagar, ej heller gamla män, som icke fylla sina dagars mått; nej, den som dör ung skall dö först vid hundra års ålder, och först vid hundra års ålder skall syndaren drabbas av förbannelsen.
\par 21 När de bygga hus, skola de ock få bo i dem; när de plantera vingårdar, skola de ock få äta deras frukt.
\par 22 När de bygga hus, skall det ej bliva andra, som få bo i dem; när de plantera något, skall det ej bliva andra, som få äta därav. Ty samma ålder, som ett träd uppnår, skall man uppnå i mitt folk, och mina utkorade skola själva njuta av sina händers verk.
\par 23 De skola icke möda sig förgäves, och barnen, som de föda, drabbas ej av plötslig död; ty de äro ett släkte av HERRENS välsignade, och deras avkomlingar få leva kvar bland dem.
\par 24 Och det skall ske, att förrän de ropa, skall jag svara, och medan de ännu tala, skall jag höra.
\par 25 Då skola vargar gå i bet tillsammans med lamm, och lejon skola äta halm likasom oxar, och stoft skall vara ormens föda. Ingenstädes på mitt heliga berg skall man då göra, vad ont och fördärvligt är, säger HERREN.

\chapter{66}

\par 1 Så säger HERREN: Himmelen är min tron, och jorden är min fotapall; vad för ett hus skullen I då kunna bygga åt mig, och vad för en plats skulle tjäna mig till vilostad?
\par 2 Min hand har ju gjort allt detta, och så har allt detta blivit till, säger HERREN. Men till den skådar jag ned, som är betryckt och har en förkrossad ande, och till den som fruktar för mitt ord.
\par 3 Den däremot, som slaktar sin offertjur, men ock är en mandråpare, den som offrar sitt lamm, men tillika krossar nacken på en hund, den som frambär ett spisoffer, men därvid frambär svinblod, den som offrar rökelse, men därunder hyllar en fåfänglig avgud - likasom det lyster dessa att gå sina egna vägar och likasom deras själ har behag till deras styggelser,
\par 4 så lyster det ock mig att fara illa fram med dem och att låta förskräckelse komma över dem, eftersom ingen svarade, när jag kallade, och eftersom de icke hörde, när jag talade, utan gjorde, vad ont var i mina ögon, och hade sin lust i att göra, vad mig misshagligt var.
\par 5 Hören HERRENS ord, I som frukten för hans ord. Edra bröder, som hata eder och stöta eder bort för mitt namns skull, de säga: "Må HERREN förhärliga sig, så att vi få se eder glädje." Men de skola komma på skam.
\par 6 Hör, huru det larmar i staden, hör dånet i templet! Hör dånet, när HERREN vedergäller sina fiender, vad de hava gjort!
\par 7 Innan Sion har känt någon födslovånda, föder hon barnet; innan kval har kommit över henne, bliver hon förlöst med ett gossebarn.
\par 8 Vem har hört något sådant, vem har sett något dylikt? Kan då ett land komma till liv på en enda dag, eller kan ett folk födas i ett ögonblick, eftersom Sion födde fram sina barn, just då våndan begynte?
\par 9 Ja, ty skulle jag väl låta fostret bliva fullgånget, men icke giva kraft att föda fram det? säger HERREN. Eller skulle jag giva kraft att föda, men sedan hålla fostret tillbaka? säger din Gud.
\par 10 Glädjens med Jerusalem och fröjden eder över henne, alla I som haven henne kär; jublen högt med henne, alla I som haven sörjt över henne.
\par 11 Så skolen I få dia eder mätta vid hennes hugsvalelses bröst; så skolen I få suga med lust av hennes rika barm.
\par 12 Ty så säger HERREN: Se, jag vill låta frid komma över henne såsom en ström och folkens rikedomar såsom en översvämmande flod, och I skolen så få dia, I skolen bliva burna på armen och skolen få sitta i knäet och bliva smekta.
\par 13 Såsom en moder tröstar sin son, så skall jag trösta eder; ja, i Jerusalem skolen I få tröst.
\par 14 Och edra hjärtan skola glädja sig, när I fån se detta, och benen i edra kroppar skola hava livskraft såsom spirande gräs; och man skall förnimma, att HERRENS hand är med hans tjänare och att ogunst kommer över hans fiender.
\par 15 Ty se, HERREN skall komma i eld, och hans vagnar skola vara såsom en stormvind; och han skall låta sin vrede drabba med hetta och sin näpst med eldslågor.
\par 16 Ty HERREN skall hålla dom med eld, och med sitt svärd skall han slå allt kött, och många skola de vara, som bliva slagna av HERREN.
\par 17 De som låta inviga sig och rena sig till gudstjänst i lustgårdar, anförda av en som står där i mitten, de som äta svinkött och annan styggelse, ja, också möss, de skola allasammans förgås, säger HERREN.
\par 18 Jag känner deras gärningar och tankar. Den tid kommer, då jag skall församla alla folk och tungomål; och de skola komma och se min härlighet.
\par 19 Och jag skall göra ett tecken bland dem; och några av dem som bliva räddade skall jag sända såsom budbärare till hednafolken, till Tarsis, till Pul och Lud, bågskyttfolken, till Tubal och Javan, till havsländerna i fjärran, som icke hava hört något om mig eller sett min härlighet; och de skola förkunna min härlighet bland folken.
\par 20 Och på hästar och i vagnar och bärstolar och på mulåsnor och dromedarer skola de från alla folk föra alla edra bröder fram till mitt heliga berg i Jerusalem såsom ett spisoffer åt HERREN, säger HERREN, likasom Israels barn i rena kärl föra fram spisoffer till HERRENS hus.
\par 21 Och jämväl sådana skall jag taga till mina präster, till mina leviter, säger HERREN.
\par 22 Ty likasom de nya himlar och den nya jord, som jag vill göra, bliva beståndande inför mig, säger HERREN, så skall det ock vara med edra barn och med edert namn.
\par 23 Och nymånadsdag efter nymånadsdag och sabbatsdag efter sabbatsdag skall det ske, att allt kött kommer och tillbeder inför mig, säger HERREN.
\par 24 Och man skall gå ut och se med lust, huru de människor, som avföllo från mig, nu ligga där döda; ty deras mask skall icke dö, och deras eld skall icke utsläckas, och de skola vara till vämjelse för allt kött.


\end{document}