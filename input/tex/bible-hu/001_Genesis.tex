\begin{document}

\title{Genesis}


\chapter{1}

\par 1 Kezdetben teremté Isten az eget és a földet.
\par 2 A föld pedig kietlen és puszta vala, és setétség vala a mélység színén, és az Isten Lelke lebeg vala a vizek felett.
\par 3 És monda Isten: Legyen világosság: és lõn világosság.
\par 4 És látá Isten, hogy jó a világosság; és elválasztá Isten a világosságot a setétségtõl.
\par 5 És nevezé Isten a világosságot nappalnak, és a setétséget nevezé éjszakának: és lõn este és lõn reggel, elsõ nap.
\par 6 És monda Isten: Legyen mennyezet a víz között, a mely elválaszsza a vizeket a vizektõl.
\par 7 Teremté tehát Isten a mennyezetet, és elválasztá a mennyezet alatt való vizeket, a mennyezet felett való vizektõl. És úgy lõn.
\par 8 És nevezé Isten a mennyezetet égnek: és lõn este, és lõn reggel, második nap.
\par 9 És monda Isten: Gyûljenek egybe az ég alatt való vizek egy helyre, hogy tessék meg a száraz. És úgy lõn.
\par 10 És nevezé Isten a szárazat földnek; az egybegyûlt vizeket pedig tengernek nevezé. És látá Isten, hogy jó.
\par 11 Azután monda Isten: Hajtson a föld gyenge fûvet, maghozó fûvet, gyümölcsfát, a mely gyümölcsöt hozzon az õ neme szerint, a melyben legyen néki magva e földön. És úgy lõn.
\par 12 Hajta tehát a föld gyenge fûvet, maghozó fûvet az õ neme szerint, és gyümölcstermõ fát, a melynek gyümölcsében mag van az õ neme szerint. És látá Isten, hogy jó.
\par 13 És lõn este és lõn reggel, harmadik nap.
\par 14 És monda Isten: Legyenek világító testek az ég mennyezetén, hogy elválaszszák a nappalt az éjszakától, és legyenek jelek, és meghatározói ünnepeknek, napoknak és esztendõknek.
\par 15 És legyenek világítókul az ég mennyezetén hogy világítsanak a földre. És úgy lõn.
\par 16 Teremté tehát Isten a két nagy világító testet: a nagyobbik világító testet, hogy uralkodjék nappal és a kisebbik világító testet, hogy uralkodjék éjjel; és a csillagokat.
\par 17 És helyezteté Isten azokat az ég mennyezetére, hogy világítsanak a földre;
\par 18 És hogy uralkodjanak a nappalon és az éjszakán, és elválaszszák a világosságot a setétségtõl. És látá Isten, hogy jó.
\par 19 És lõn este és lõn reggel, negyedik nap.
\par 20 És monda Isten: Pezsdûljenek a vizek élõ állatok nyüzsgésétõl; és madarak repdessenek a föld felett, az ég mennyezetének színén.
\par 21 És teremté Isten a nagy vízi állatokat, és mindazokat a csúszó-mászó állatokat, a melyek nyüzsögnek a vizekben az õ nemök szerint, és mindenféle szárnyas repdesõt az õ neme szerint. És látá Isten, hogy jó.
\par 22 És megáldá azokat Isten, mondván: Szaporodjatok, és sokasodjatok, és töltsétek be a tenger vizeit; a madár is sokasodjék a földön.
\par 23 És lõn este és lõn reggel, ötödik nap.
\par 24 Azután monda az Isten: Hozzon a föld élõ állatokat nemök szerint: barmokat, csúszó-mászó állatokat és szárazföldi vadakat nemök szerint. És úgy lõn.
\par 25 Teremté tehát Isten a szárazföldi vadakat nemök szerint, a barmokat nemök szerint, és a földön csúszó-mászó mindenféle állatotokat nemök szerint. És látá Isten, hogy jó.
\par 26 És monda Isten: Teremtsünk embert a mi képünkre és hasonlatosságunkra; és uralkodjék a tenger halain, az ég madarain, a barmokon, mind az egész földön, és a földön csúszó-mászó mindenféle állatokon.
\par 27 Teremté tehát az Isten az embert az õ képére, Isten képére teremté õt: férfiúvá és asszonynyá teremté õket.
\par 28 És megáldá Isten õket, és monda nékik Isten: Szaporodjatok és sokasodjatok, és töltsétek be a földet és hajtsátok birodalmatok alá; és uralkodjatok a tenger halain, az ég madarain, és a földön csúszó-mászó mindenféle állatokon.
\par 29 És monda Isten: Imé néktek adok minden maghozó fûvet az egész föld színén, és minden fát, a melyen maghozó gyümölcs van; az legyen néktek eledül.
\par 30 A föld minden vadainak pedig, és az ég minden madarainak, és a földön csúszó-mászó mindenféle állatoknak, a melyekben élõ lélek van, a zöld fûveket adom eledelûl. És úgy lõn.
\par 31 És látá Isten, hogy minden a mit teremtett vala, ímé igen jó. És lõn este és lõn reggel, hatodik nap.

\chapter{2}

\par 1 És elvégezteték az ég és a föld, és azoknak minden serege.
\par 2 Mikor pedig elvégezé Isten hetednapon az õ munkáját, a melyet alkotott vala, megszûnék a hetedik napon minden munkájától, a melyet alkotott vala.
\par 3 És megáldá Isten a hetedik napot, és megszentelé azt; mivelhogy azon szûnt vala meg minden munkájától, melyet teremtve szerzett vala Isten.
\par 4 Ez az égnek és a földnek eredete, a mikor teremtettek. Mikor az Úr Isten a földet és az eget teremté,
\par 5 Még semmiféle mezei növény sem vala a földön, s még semmiféle mezei fû sem hajtott ki, mert az Úr Isten még nem bocsátott vala esõt a földre; és ember sem vala, ki a földet mívelje;
\par 6 Azonban pára szállott vala fel a földrõl, és megnedvesíté a föld egész színét.
\par 7 És formálta vala az Úr Isten az embert a földnek porából, és lehellett vala az õ orrába életnek lehelletét. Így lõn az ember élõ lélekké.
\par 8 És ültete az Úr Isten egy kertet Édenben, napkelet felõl, és abba helyezteté az embert, a kit formált vala.
\par 9 És nevele az Úr Isten a földbõl mindenféle fát, tekintetre kedvest és eledelre jót, az élet fáját is, a kertnek közepette, és a jó és gonosz tudásának fáját.
\par 10 Folyóvíz jõ vala pedig ki Édenbõl a kert megöntözésére; és onnét elágazik és négy fõágra szakad vala.
\par 11 Az elsõnek neve Pison, ez az, a mely megkerüli Havilah egész földét, a hol az arany terem.
\par 12 És annak a földnek aranya igen jó; ott van a Bdelliom és az Onix-kõ.
\par 13 A második folyóvíz neve pedig Gihon; ez az, a mely megkerüli az egész Khús földét.
\par 14 És a harmadik folyóvíz neve Hiddekel; ez az, a mely Assiria hosszában foly. A negyedik folyóvíz pedig az Eufrátes.
\par 15 És vevé az Úr Isten az embert, és helyezteté õt az Éden kertjébe, hogy mívelje és õrizze azt.
\par 16 És parancsola az Úr Isten az embernek, mondván: A kert minden fájáról bátran egyél.
\par 17 De a jó és gonosz tudásának fájáról, arról ne egyél; mert a mely napon ejéndel arról bizony meghalsz.
\par 18 És monda az Úr Isten: Nem jó az embernek egyedül lenni; szerzék néki segítõ társat, hozzá illõt.
\par 19 És formált vala az Úr Isten a földbõl mindenféle mezei vadat, és mindenféle égi madarat, és elvivé az emberhez, hogy lássa, minek nevezze azokat; mert a mely nevet adott az ember az élõ állatnak, az annak neve.
\par 20 És nevet ada az ember minden baromnak, az ég madarainak, és minden mezei vadnak; de az embernek hozzá illõ segítõ társat nem talált vala.
\par 21 Bocsáta tehát az Úr Isten mély álmot az emberre, és ez elaluvék. Akkor kivõn egyet annak oldalbordái közûl, és hússal tölté be annak helyét.
\par 22 És alkotá az Úr Isten azt az oldalbordát, a melyet kivett vala az emberbõl, asszonynyá, és vivé az emberhez.
\par 23 És monda az ember: Ez már csontomból való csont, és testembõl való test: ez asszonyembernek neveztessék, mert emberbõl vétetett.
\par 24 Annakokáért elhagyja a férfiú az õ atyját és az õ anyját, és ragaszkodik feleségéhez: és lesznek egy testté.
\par 25 Valának pedig mindketten mezítelenek, az ember és az õ felesége, és nem szégyenlik vala.

\chapter{3}

\par 1 A kígyó pedig ravaszabb vala minden mezei vadnál, melyet az Úr Isten teremtett vala, és monda az asszonynak: Csakugyan azt mondta az Isten, hogy a kertnek egy fájáról se egyetek?
\par 2 És monda az asszony a kígyónak: A kert fáinak gyümölcsébõl ehetünk;
\par 3 De annak a fának gyümölcsébõl, mely a kertnek közepette van, azt mondá Isten: abból ne egyetek, azt meg se illessétek, hogy meg ne haljatok.
\par 4 És És monda a kígyó az asszonynak: Bizony nem haltok meg;
\par 5 Hanem tudja az Isten, hogy a mely napon ejéndetek abból, megnyilatkoznak a ti szemeitek, és olyanok lésztek mint az Isten: jónak és gonosznak tudói.
\par 6 És látá az asszony, hogy jó az a fa eledelre, s hogy kedves a szemnek, és kivánatos az a fa a bölcseségért: szakaszta azért annak gyümölcsébõl és evék, és ada vele levõ férjének is, és az is evék.
\par 7 És megnyilatkozának mindkettõjöknek szemei s észrevevék, hogy mezítelenek; figefa levelet aggatának azért össze, és körülkötõket csinálának magoknak.
\par 8 És meghallák az Úr Isten szavát, a ki hûvös alkonyatkor a kertben jár vala; és elrejtõzék az ember és az õ felesége az Úr Isten elõl a kert fái között.
\par 9 Szólítá ugyanis az Úr Isten az embert és monda néki: Hol vagy?
\par 10 És monda: Szavadat hallám a kertben, és megfélemlém, mivelhogy mezítelen vagyok, és elrejtezém.
\par 11 És monda Õ: Ki mondá néked, hogy mezítelen vagy? Avagy talán ettél a fáról, melytõl tiltottalak, hogy arról ne egyél?
\par 12 És monda az ember: Az asszony, a kit mellém adtál vala, õ ada nékem arról a fáról, úgy evém.
\par 13 És monda az Úr Isten az asszonynak: Mit cselekedtél? Az asszony pedig monda: A kígyó ámított el engem, úgy evém.
\par 14 És monda az Úr Isten a kígyónak: Mivelhogy ezt cselekedted, átkozott légy minden barom és minden mezei vad között; hasadon járj, és és port egyél életed minden napjaiban.
\par 15 És ellenségeskedést szerzek közötted és az asszony között, a te magod között, és az õ magva között: az neked fejedre tapos, te pedig annak sarkát mardosod.
\par 16 Az asszonynak monda: Felette igen megsokasítom viselõsséged fájdalmait, fájdalommal szûlsz magzatokat; és epekedel a te férjed után, õ pedig uralkodik te rajtad.
\par 17 Az embernek pedig monda: Mivelhogy hallgattál a te feleséged szavára, és ettõl arról a fáról, a melyrõl azt parancsoltam, hogy ne egyél arról: Átkozott legyen a föld te miattad, fáradságos munkával élj belõle életednek minden napjaiban.
\par 18 Töviset és bogácskórót teremjen tenéked; s egyed a mezõnek fûvét.
\par 19 Orczád verítékével egyed a te kenyeredet, míglen visszatérsz a földbe, mert abból vétettél: mert por vagy te s ismét porrá leszesz.
\par 20 Nevezte vala pedig Ádám az õ feleségét Évának, mivelhogy õ lett anyja minden élõnek.
\par 21 És csinála az Úr Isten Ádámnak és az õ feleségének bõr ruhákat, és felöltözteté õket.
\par 22 És monda az Úr Isten: Ímé az ember olyanná lett, mint mi közûlünk egy, jót és gonoszt tudván. Most tehát, hogy ki ne nyújtsa kezét, hogy szakaszszon az élet fájáról is, hogy egyék, s örökké éljen:
\par 23 Kiküldé õt az Úr Isten az Éden kertjébõl, hogy mívelje a földet, a melybõl vétetett vala.
\par 24 És kiûzé az embert, és oda helyezteté az Éden kertjének keleti oldala felõl a Kerúbokat és a villogó pallos lángját, hogy õrizzék az élet fájának útját.

\chapter{4}

\par 1 Azután ismeré meg Ádám az õ feleségét Évát, a ki fogad vala méhében és szûli Kaint, és monda: Nyertem férfiat az Úrtól.
\par 2 És ismét szûlé annak atyjafiát, Ábelt. És Ábel juhok pásztora lõn, Kain pedig földmívelõ.
\par 3 Lõn pedig idõ multával, hogy Kain ajándékot vive az Úrnak a föld gyümölcsébõl.
\par 4 És Ábel is vive az õ juhainak elsõ fajzásából és azoknak kövérségébõl. És tekinte az Úr Ábelre és az õ ajándékára.
\par 5 Kainra pedig és az õ ajándékára nem tekinte, miért is Kain haragra gerjede és fejét lecsüggeszté.
\par 6 És monda az Úr Kainnak: Miért gerjedtél haragra? és miért csüggesztéd le fejedet?
\par 7 Hiszen, ha jól cselekszel, emelt fõvel járhatsz; ha pedig nem jól cselekszel, a bûn az ajtó elõtt leselkedik, és reád van vágyódása; de te uralkodjál rajta.
\par 8 És szól s beszél vala Kain Ábellel, az õ atyjafiával. És lõn, mikor a mezõn valának, támada Kain Ábelre az õ atyjafiára, és megölé õt.
\par 9 És monda az Úr Kainnak: Hol van Ábel a te atyádfia? Õ pedig monda: Nem tudom, avagy õrizõje vagyok-é én az én atyámfiának?
\par 10 Monda pedig az Úr: Mit cselekedtél? A te atyádfiának vére kiált én hozzám a földrõl.
\par 11 Mostan azért átkozott légy e földön, mely megnyitotta az õ száját, hogy befogadja a te atyádfiának vérét, a te kezedbõl.
\par 12 Mikor a földet míveled, ne adja az többé néked az õ termõ erejét, bujdosó és vándorló légy a földön.
\par 13 Akkor monda Kain az Úrnak: Nagyobb az én büntetésem, hogysem elhordozhatnám.
\par 14 Ímé elûzöl engem ma e földnek színérõl, és a te színed elõl el kell rejtõznöm; bujdosó és vándorló leszek a földön, és akkor akárki talál reám, megöl engemet.
\par 15 És monda néki az Úr: Sõt inkább, aki megöléndi Kaint, hétszerte megbüntettetik. És megbélyegzé az Úr Kaint, hogy senki meg ne ölje, a ki rátalál.
\par 16 És elméne Kain az Úr színe elõl, és letelepedék Nód földén, Édentõl keletre.
\par 17 És ismeré Kain az õ feleségét, az pedig fogada méhében, és szûlé Hánókhot. És építe várost, és nevezé azt az õ fiának nevérõl Hánókhnak.
\par 18 És lett Hánókhnak fia, Irád: És Irád nemzé Mekhujáelt: Mekhujáel pedig nemzé Methusáelt, és Methusáel nemzé Lámekhet.
\par 19 Lámekh pedig vett magának két feleséget: az egyiknek neve Háda, a másiknak neve Czilla.
\par 20 És szûlé Háda Jábált. Ez volt atyjok a sátorban-lakóknak, és a barompásztoroknak.
\par 21 Az õ atyjafiának pedig Jubál vala neve: ez volt atyja minden lantosnak és síposnak.
\par 22 Czilla pedig szûlé Tubálkaint, mindenféle réz- és vasszerszámok kovácsolóját: és Tubálkain hugát, Nahamát.
\par 23 Akkor monda Lámekh az õ feleségeinek: Oh Háda és Czilla, hallgassatok szómra, Lámekh feleségei, halljátok beszédem: embert öltem, mert megsebzett; ifjat öltem, mert megütött.
\par 24 Ha hétszeres a bosszú Kainért, hetvenszeres az Lámekhért.
\par 25 Ádám pedig ismét ismeré az õ feleségét, és az szûle néki fiat, és nevezé annak nevét Séthnek: mert adott úgymond, énnékem az Isten más magot Ábel helyett, kit megöle Kain.
\par 26 Séthnek is született fia, és nevezé annak nevét Énósnak. Akkor kezdték segítségül hívni az Úrnak nevét.

\chapter{5}

\par 1 Ez az Ádám nemzetségének könyve. A mely napon teremté Isten az embert, Isten hasonlatosságára teremté azt.
\par 2 Férfiúvá és asszonynyá teremté õket, és megáldá õket és nevezé az õ nevöket Ádámnak, a mely napon teremtetének.
\par 3 Élt vala pedig Ádám száz harmincz esztendõt, és nemze fiat az õ képére és hasonlatosságára és nevezé annak nevét Séthnek.
\par 4 És telének Ádám napjai, minekutánna Séthet nemzette, nyolczszáz esztendõre, és nemze fiakat és leányokat.
\par 5 És lõn Ádám egész életének ideje kilenczszáz harmincz esztendõ; És meghala.
\par 6 Éle pedig Séth száz öt esztendõt, és nemzé Énóst.
\par 7 És éle Séth, minekutánna Énóst nemzette, nyolczszáz hét esztendeig; és nemze fiakat és leányokat.
\par 8 És lõn Séth egész életének ideje kilenczszáz tizenkét esztendõ; és meghala.
\par 9 Éle pedig Énós kilenczven esztendõt, és nemzé Kénánt.
\par 10 És éle Énós, minekutánna Kénánt nemzette, nyolczszáz tizenöt esztendeig, és nemze fiakat és leányokat.
\par 11 És lõn Énós egész életének ideje kilenczszáz öt esztendõ; és meghala.
\par 12 Éle pedig Kénán hetven esztendõt, és nemzé Mahalálélt.
\par 13 És éle Kénán, minekutánna Mahalálélt nemzette, nyolczszáz negyven esztendeig; és nemze fiakat és leányokat.
\par 14 És lõn Kénán egész életének ideje kilenczszáz tíz esztendõ; és meghala.
\par 15 Éle pedig Mahalálél hatvanöt esztendõt, és nemzé Járedet.
\par 16 És éle Mahalálél, minekutánna Járedet nemzette, nyolczszáz harmincz esztendeig, és nemze fiakat és leányokat.
\par 17 És lõn Mahalálél egész életének ideje nyolczszáz kilenczvenöt esztendõ; és meghala.
\par 18 Éle pedig Járed száz hatvankét esztendõt, és nemzé Énókhot.
\par 19 És éle Járed, minekutánna Énókhot nemzette, nyolczszáz esztendõt; és nemze fiakat és leányokat.
\par 20 És lõn Járed egész életének ideje kilenczszáz hatvankét esztendõ, és meghala.
\par 21 Éle pedig Énókh hatvanöt esztendõt, és nemzé Methuséláht.
\par 22 És járt Énókh az Istennel, minekutánna Methuséláht nemzette, háromszáz esztendeig; és nemze fiakat és leányokat.
\par 23 És lõn Énókh egész életének ideje háromszáz hatvanöt esztendõ.
\par 24 És mivel Énókh Istennel járt vala; eltûnék, mert Isten magához vevé.
\par 25 Éle pedig Methusélah száz nyolczvanhét esztendõt és nemzé Lámekhet.
\par 26 És éle Methusélah, minekutánna Lámekhet nemzette, hétszáz nyolczvankét esztendõt; és nemze fiakat és leányokat.
\par 27 És lõn Methusélah egész életének ideje kilenczszáz hatvankilencz esztendõ; és meghala.
\par 28 Éle pedig Lámekh száz nyolczvankét esztendõt, és nemze fiat.
\par 29 És nevezé azt Noénak, mondván: Ez vígasztal meg minket munkálkodásunkban s kezünk terhes fáradozásában e földön, melyet megátkozott az Úr.
\par 30 És éle Lámekh, minekutánna Noét nemzette, ötszáz kilenczvenöt esztendõt; és nemze fiakat és leányokat.
\par 31 És lõn Lámekh egész életének ideje hétszázhetvenhét esztendõ; és meghala.
\par 32 És mikor Noé ötszáz esztendõs volt, nemzé Noé Sémet, Khámot és Jáfetet.

\chapter{6}

\par 1 Lõn pedig, hogy az emberek sokasodni kezdenének a föld színén, és leányaik születének.
\par 2 És láták az Istennek fiai az emberek leányait, hogy szépek azok, és vevének magoknak feleségeket mind azok közül, kiket megkedvelnek vala.
\par 3 És monda az Úr: Ne maradjon az én lelkem örökké az emberben, mivelhogy õ test; legyen életének ideje száz húsz esztendõ.
\par 4 Az óriások valának a földön abban az idõben, sõt még azután is, mikor az Isten fiai bémenének az emberek leányaihoz, és azok gyermekeket szûlének nékik. Ezek ama hatalmasok, kik eleitõl fogva híres-neves emberek voltak.
\par 5 És látá az Úr, hogy megsokasult az ember gonoszsága a földön, és hogy szíve gondolatának minden alkotása szüntelen csak gonosz.
\par 6 Megbáná azért az Úr, hogy teremtette az embert a földön, és bánkódék az õ szívében.
\par 7 És monda az Úr: Eltörlöm az embert, a kit teremtettem, a földnek színérõl; az embert, a barmot, a csúszó-mászó állatokat, és az ég madarait; mert bánom, hogy azokat teremtettem.
\par 8 De Noé kegyelmet talála az Úr elõtt.
\par 9 Noénak pedig ez a története: Noé igaz, tökéletes férfiú vala a vele egykorúak között. Istennel jár vala Noé.
\par 10 És nemze Noé három fiat: Sémet, Khámot és Jáfetet.
\par 11 A föld pedig romlott vala Isten elõtt és megtelék a föld erõszakoskodással.
\par 12 Tekinte azért Isten a földre, és ímé meg vala romolva, mert minden test megrontotta vala az õ útát a földön.
\par 13 Monda azért Isten Noénak: Minden testnek vége elérkezett elõttem, mivelhogy a föld erõszakoskodással telt meg általok: és ímé elvesztem õket a földdel egybe.
\par 14 Csinálj magadnak bárkát gófer fából, rekesztékeket csinálj a bárkában, és szurkozd meg belõl és kivûl szurokkal.
\par 15 Ekképpen csináld pedig azt: A bárka hoszsza háromszáz sing legyen, a szélessége ötven sing, és a magassága harmincz sing.
\par 16 Ablakot csinálj a bárkán, és egy singnyire hagyd azt felülrõl; a bárka ajtaját pedig oldalt csináld; alsó, közép, és harmad padlásúvá csináld azt.
\par 17 Én pedig ímé özönvizet hozok a földre, hogy elveszessek minden testet, a melyben élõ lélek van az ég alatt; valami a földön van, elvész.
\par 18 De te veled szövetséget kötök, és bemégy a bárkába, te és a te fiaid, feleséged és a te fiaidnak feleségei teveled.
\par 19 És minden élõbõl, s minden testbõl, mindenbõl kettõt-kettõt vígy be a bárkába, hogy veled együtt életben maradjanak: hímek és nõstények legyenek.
\par 20 A madarak közûl az õ nemök szerint, a barmok közûl az õ nemök szerint és a földnek minden csúszó-mászó állatjai közûl az õ nemök szerint; mindenbõl kettõ-kettõ menjen be hozzád, hogy életben maradjanak.
\par 21 Te pedig szerezz magadnak mindenféle eledelt, mely megehetõ, és takarítsd be magadhoz, hogy neked is, azoknak is legyen eledelûl.
\par 22 És úgy cselekedék Noé; a mint parancsolta vala néki Isten, mindent akképen cselekedék.

\chapter{7}

\par 1 Monda az Úr Noénak: Menj be te, és egész házadnépe a bárkába: mert téged láttalak igaznak elõttem ebben a nemzedékben.
\par 2 Minden tiszta baromból hetet-hetet vígy be, hímet és nõstényét; azokból a barmokból pedig, a melyek nem tiszták, kettõt-kettõt, hímet és nõstényét.
\par 3 Az égi madarakból is hetet-hetet, hímet és nõstényét, hogy magvok maradjon az egész föld színén.
\par 4 Mert hét nap múlva esõt bocsátok a földre negyven nap és negyven éjjel; és eltörlök a föld színérõl minden állatot, melyet teremtettem.
\par 5 Cselekedék azért Noé mind a szerint, a mint az Úr néki megparancsolta vala.
\par 6 Noé pedig hatszáz esztendõs vala, mikor az özönvíz volt a földön.
\par 7 Beméne azért Noé és az õ fiai, az õ felesége, és fiainak feleségei õ vele a bárkába, az özönvíz elõl.
\par 8 A tiszta barmok közûl, és a tisztátalan barmok közûl, a madarak közûl, és minden földön csúszó-mászó állat közûl,
\par 9 Kettõ-kettõ méne be Noéhoz a bárkába, hím és nõstény: a mint Isten megparancsolta vala Noénak.
\par 10 Lõn pedig hetednapra, hogy megjöve az özönvíz a földre.
\par 11 Noé életének hatszázadik esztendejében, a második hónapban, e hónap tizenhetedik napján, felfakadának ezen a napon a nagy mélység minden forrásai, és az ég csatornái megnyilatkozának.
\par 12 És esék az esõ a földre negyven nap és negyven éjjel.
\par 13 Ugyanezen a napon ment vala be Noé és Sém és Khám és Jáfet, Noénak fiai, és Noé felesége és az õ fiainak három felesége velök együtt a bárkába.
\par 14 Õk, és minden vad az õ neme szerint és minden barom az õ neme szerint és minden csúszó-mászó állat, mely csúsz-mász a földön, az õ neme szerint és minden repdesõ állat az õ neme szerint, minden madár, minden szárnyas állat.
\par 15 Kettõ-kettõ méne be Noéhoz a bárkába minden testbõl, melyben élõ lélek vala.
\par 16 A melyek pedig bemenének, hím és nõstény méne be minden testbõl, a mint parancsolta vala Isten õnéki: és az Úr bezára utána az ajtót.
\par 17 Mikor az özönvíz negyven napig volt a földön, annyira nevekedének a vizek, hogy felveheték a bárkát, és az felemelkedék a földrõl.
\par 18 A vizek pedig áradának és egyre nevekedének a földön, és a bárka jár vala a víz színén.
\par 19 Azután a vizek felette igen nagy erõt vevének a földön, és a legmagasabb hegyek is mind elboríttatának, melyek az egész ég alatt valának.
\par 20 Tizenöt singgel nevekedének a vizek feljebb, minekutánna a hegyek elboríttattak vala.
\par 21 És oda vesze minden földön járó test, madár, barom, vad, és a földön nyüzsgõ minden csúszó-mászó állat; és minden ember.
\par 22 Mindaz, a minek orrában élõ lélek lehellete vala, szárazon valók közûl mind meghala.
\par 23 És eltörle az Isten minden állatot, a mely a föld színén vala, az embertõl a baromig, a csúszó-mászó állatig, és az égi madárig; mindenek eltöröltetének a földrõl; és csak Noé marada meg, és azok akik vele valának a bárkában.
\par 24 És erõt vevének a vizek a földön, száz ötven napig.

\chapter{8}

\par 1 Megemlékezék pedig az Isten Noéról, és minden vadról, minden baromról, mely õ vele a bárkában vala: és szelet bocsáta az Isten a földre, és a vizek megapadának.
\par 2 És bezárulának a mélység forrásai s az ég csatornái; és megszûnt az esõ az égbõl.
\par 3 És elmenének a vizek a földrõl folyton fogyván, és száz ötven nap mulva megfogyatkozának a vizek.
\par 4 A bárka pedig a hetedik hónapban, a hónak tizenhetedik napján, megfeneklett az Ararát hegyén.
\par 5 A vizek pedig folyton fogyának a tizedik hónapig; a tizedikben, a hó elsõ napján meglátszának a hegyek csúcsai.
\par 6 És lõn negyven nap múlva, kinyitá Noé a bárka ablakát, melyet csinált vala.
\par 7 És kibocsátá a hollót, és az elrepûlt, meg visszaszállt, míg a vizek a földrõl felszáradának.
\par 8 Kibocsátá a galambot is, hogy meglássa, vajjon elfogytak-é a vizek a föld színérõl.
\par 9 De a galamb nem talála lábainak nyugvóhelyet és visszatére õ hozzá a bárkába, mert víz vala az egész föld színén; õ pedig kezét kinyujtá, megfogá, és bévevé azt magához a bárkába.
\par 10 És várakozék még másik hét napig, és ismét kibocsátá a galambot a bárkából.
\par 11 És megjöve õ hozzá a galamb estennen, és ímé leszakasztott olajfalevél vala annak szájában. És megtudá Noé, hogy elapadt a víz a földrõl.
\par 12 És ismét várakozék még másik hét napig, és kibocsátá a galambot, és az nem tére többé õ hozzá vissza.
\par 13 És lõn a hatszáz egyedik esztendõben, az elsõ hónak elsõ napján, felszáradának a vizek a földrõl, és elfordítá Noé a bárka fedelét, és látá, hogy ímé megszikkadt a földnek színe.
\par 14 A második hónapban pedig, a hónak huszonhetedik napján megszárada a föld.
\par 15 És szóla az Isten Noénak, mondván:
\par 16 Menj ki a bárkából te és a te feleséged, a te fiaid, és a te fiaid feleségei te veled.
\par 17 Minden vadat, mely veled van, minden testbõl madarat, barmot, és minden földön csúszó-mászó állatot vígy ki magaddal, hogy nyüzsögjenek a földön, szaporodjanak és sokasodjanak a földön.
\par 18 Kiméne azért Noé és az õ fiai, az õ felesége, és az õ fiainak feleségei õ vele.
\par 19 Minden állat, minden csúszó-mászó, minden madár, minden a mi mozog a földön, kijöve a bárkából az õ neme szerint.
\par 20 És oltárt építe Noé az Úrnak, és võn minden tiszta állatból és minden tiszta madárból, és áldozék égõáldozattal az oltáron.
\par 21 És megérezé az Úr a kedves illatot, és monda az Úr az õ szívében: Nem átkozom meg többé a földet az emberért, mert az ember szívének gondolatja gonosz az õ ifjúságától fogva; és többé nem vesztem el mind az élõ állatot, mint cselekedtem.
\par 22 Ennekutánna míg a föld lészen, vetés és aratás, hideg és meleg, nyár és tél, nap és éjszaka meg nem szünnek.

\chapter{9}

\par 1 Azután megáldá Isten Noét és az õ fiait, és azt mondá nékik: Szaporodjatok és sokasodjatok, és töltsétek be a földet.
\par 2 És féljen és rettegjen tõletek a földnek minden állatja az égnek minden madara: minden a mi nyüzsög a földön, és a tengernek minden hala kezetekbe adatott;
\par 3 Minden mozgó állat, a mely él legyen nektek eledelûl; a mint a zöld fûvet, nektek adtam mindazokat.
\par 4 Csak a húst az õt elevenítõ vérrel meg ne egyétek.
\par 5 De a ti véreteket, a melyben van a ti éltetek, számon kérem; számon kérem minden állattól, azonképen az embertõl, kinek-kinek atyjafiától számon kérem az ember életét.
\par 6 A ki ember-vért ont, annak vére ember által ontassék ki; mert Isten a maga képére teremté az embert.
\par 7 Ti pedig szaporodjatok és sokasodjatok, nyüzsögjetek a földön és sokasodjatok azon.
\par 8 És szóla az Isten Noénak és vele az õ fiainak, mondván:
\par 9 Én pedig ímé szövetséget szerzek ti veletek és a ti magvatokkal ti utánnatok.
\par 10 És minden élõ állattal, mely veletek van: madárral, barommal, minden mezei vaddal, mely veletek van; mindattól kezdve a mi a bárkából kijött, a földnek minden vadjáig.
\par 11 Szövetséget kötök ti veletek, hogy soha ezután el nem vész özönvíz miatt minden test; és soha sem lesz többé özönvíz a földnek elvesztésére.
\par 12 És monda az Isten: Ez a jele a szövetségnek, melyet én örök idõkre szerzek közöttem és ti köztetek, és minden élõ állat között, mely ti veletek van:
\par 13 Az én ívemet helyeztetem a felhõkbe, s ez lesz jele a szövetségnek közöttem és a föld között.
\par 14 És lészen, hogy mikor felhõvel borítom be a földet, meglátszik az ív a felhõben.
\par 15 És megemlékezem az én szövetségemrõl, mely van én közöttem és ti közöttetek, és minden testbõl való élõ állat között; és nem lesz többé a víz özönné minden testnek elvesztésére.
\par 16 Azért legyen tehát az ív a felhõben, hogy lássam azt és megemlékezzem az örökkévaló szövetségrõl Isten között és minden testbõl való élõ állat között, mely a földön van.
\par 17 És monda Isten Noénak: Ez ama szövetségnek jele, melyet szerzettem én közöttem és minden test között, mely a földön van.
\par 18 Valának pedig Noé fiai, kik a bárkából kijöttek vala: Sém és Khám és Jáfet. Khám pedig Kanaánnak atyja.
\par 19 Ezek hárman a Noé fiai s ezektõl népesedék meg az egész föld.
\par 20 Noé pedig földmívelõ kezde lenni, és szõlõt ültete.
\par 21 És ivék a borból, s megrészegedék, és meztelenen vala sátra közepén.
\par 22 Khám pedig, Kanaánnak atyja, meglátá az õ atyjának mezítelenségét, és hírûl adá künnlevõ két testvérének.
\par 23 Akkor Sém és Jáfet ruhát ragadván, azt mindketten vállokra veték, és háttal menve takarák be atyjok mezítelenségét; s arczczal hátra meg sem láták atyjok mezítelenségét.
\par 24 Hogy felserkene Noé mámorából, és megtudá a mit vele az õ kisebbik fia cselekedett vala:
\par 25 Monda: Átkozott Kanaán! Szolgák szolgája legyen atyjafiai közt.
\par 26 Azután monda: Áldott az Úr, Sémnek Istene, néki légyen szolgája Kanaán!
\par 27 Terjessze ki Isten Jáfetet, lakozzék Sémnek sátraiban; légyen néki szolgája a Kanaán!
\par 28 Éle pedig Noé az özönvíz után háromszáz ötven esztendeig.
\par 29 És vala Noé egész életének ideje kilenczszáz ötven esztendõ; és meghala.

\chapter{10}

\par 1 Ez pedig a Noé fiainak, Sémnek, Khámnak és Jáfetnek nemzetsége; és fiaik születének az özönvíz után.
\par 2 Jáfetnek fiai: Gómer, Mágog, Madai, Jáván, Thubál, Mésekh és Thirász.
\par 3 A Gómer fiai pedig: Askhenáz, Rifáth, és Tógármah.
\par 4 Jávánnak pedig fiai: Elisah, Thársis, Kitthim és Dodánim.
\par 5 Ezekbõl váltak ki a szigetlakó népek az õ országaikban, mindenik a maga nyelve, családja és nemzetsége szerint.
\par 6 Khámnak pedig fiai: Khús, Miczráim, Pút és Kanaán.
\par 7 Khúsnak pedig fiai: Széba, Hávilah, Szábthah, Rahmáh, Szabthékah. Rahmáhnak pedig fiai: Séba és Dédán.
\par 8 Khús nemzé Nimródot is; ez kezde hatalmassá lenni a földön.
\par 9 Ez hatalmas vadász vala az Úr elõtt, azért mondják: Hatalmas vadász az Úr elõtt, mint Nimród.
\par 10 Az õ birodalmának kezdete volt Bábel, Erekh, Akkád és Kálnéh a Sineár földén.
\par 11 E földrõl ment aztán Assiriába, és építé Ninivét, Rekhoboth városát, és Kaláht.
\par 12 És Reszent Ninivé között és Kaláh között: ez az a nagy város.
\par 13 Miczráim pedig nemzé: Lúdimot, Anámimot, Lehábimot és Naftukhimot.
\par 14 Pathruszimot és Kaszlukhimot, a honnan a Filiszteusok származtak, és Kafthorimot.
\par 15 Kanaán pedig nemzé Czídont, az õ elsõszülöttét, és Khétet.
\par 16 Jebuzeust, Emorreust, és Girgazeust.
\par 17 Khivveust, Harkeust, és Szineust.
\par 18 Arvadeust, Czemareust, Hamatheust. És azután elszéledének a Kananeusok nemzetségei.
\par 19 Vala pedig a Kananeusok határa, Czídonból Gérár felé menve Gázáig; Sodoma, Gomora, Ádmáh és Czeboim felé menve Lésáig.
\par 20 Ezek a Khám fiai családjok, nyelvök, földjök s nemzetségök szerint.
\par 21 Sémnek is lettek gyermekei; a ki Héber minden fiainak atyja, Jáfetnek testvérbátyja vala.
\par 22 Sém fiai: Élám, Assur, Arpaksád, Lúd és Arám.
\par 23 Arámnak fiai pedig: Úcz, Húl, Gether és Más.
\par 24 Arpaksád pedig nemzé Séláht, Séláh pedig Hébert.
\par 25 Hébernek is lett két fia: Az egyiknek neve Péleg, mivelhogy az õ idejében osztatott el a föld; testvérének neve pedig Joktán.
\par 26 Joktán pedig nemzé Almodádot, Sélefet, Haczarmávethet és Jerákhot.
\par 27 Hadórámot, Úzált és Dikláth.
\par 28 Obált, Abimáélt és Sébát.
\par 29 Ofirt, Havilát és Jóbábot. Ezek mind a Joktán fiai.
\par 30 És vala ezeknek lakása, Mésától fogva Séfárba menve a napkeleti hegyekig.
\par 31 Ezek a Sém fiai családjok, nyelvök, földjök és nemzetségök szerint.
\par 32 Ezek a Noé fiainak családjai az õ nemzetségeik szerint, az õ népeik között, és ezektõl szaporodának el a népek a földön, az özönvíz után.

\chapter{11}

\par 1 Mind az egész földnek pedig egy nyelve és egyféle beszéde vala.
\par 2 És lõn mikor kelet felõl elindultak vala, Sineár földén egy síkságot találának és ott letelepedének.
\par 3 És mondának egymásnak: Jertek, vessünk téglát és égessük ki jól; és lõn nékik a tégla kõ gyanánt, a szurok pedig ragasztó gyanánt.
\par 4 És mondának: Jertek, építsünk magunknak várost és tornyot, melynek teteje az eget érje, és szerezzünk magunknak nevet, hogy el ne széledjünk az egész földnek színén.
\par 5 Az Úr pedig leszálla, hogy lássa a várost és a tornyot, melyet építenek vala az emberek fiai.
\par 6 És monda az Úr: Imé e nép egy, s az egésznek egy a nyelve, és munkájának ez a kezdete; és bizony semmi sem gátolja, hogy véghez ne vigyenek mindent, a mit elgondolnak magukban.
\par 7 Nosza szálljunk alá, és zavarjuk ott össze nyelvöket, hogy meg ne értsék egymás beszédét.
\par 8 És elszéleszté õket onnan az Úr az egész földnek színére; és megszûnének építeni a várost.
\par 9 Ezért nevezék annak nevét Bábelnek; mert ott zavará össze az Úr az egész föld nyelvét, és onnan széleszté el õket az Úr az egész földnek színére.
\par 10 Ez a Sém nemzetsége: Sém száz esztendõs korában nemzé Arpaksádot, két esztendõvel az özönvíz után.
\par 11 És éle Sém, minekutánna nemzette Arpaksádot, ötszáz esztendeig és nemze fiakat és leányokat.
\par 12 Arpaksád pedig hatrminczöt esztendõs vala, és nemzé Séláht.
\par 13 És éle Arpaksád, minekutánna nemzette Séláht, négyszáz három esztendeig, és nemze fiakat és leányokat.
\par 14 Séláh pedig harmincz esztendõs vala, és nemzé Hébert.
\par 15 És éle Séláh, minekutánna nemzé Hébert, négyszáz három esztendeig, és nemze fiakat és leányokat.
\par 16 Héber pedig harmincznégy esztendõs vala és nemzé Péleget.
\par 17 És éle Héber, minekutánna nemzé Péleget, négyszáz harmincz esztendeig, és nemze fiakat és leányokat.
\par 18 Péleg pedig harmincz esztendõs vala, és nemzé Réut.
\par 19 És éle Péleg, minekutánna nemzé Réut, kétszáz kilencz esztendeig, és nemze fiakat és leányokat.
\par 20 Réu pedig harminczkét esztendõs vala, és nemzé Sérugot.
\par 21 És éle Réu, minekutánna nemzé Sérugot, kétszáz hét esztendeig és nemze fiakat és leányokat.
\par 22 Sérug pedig harmincz esztendõs vala, és nemzé Nákhort.
\par 23 És éle Sérug, minekutánna nemzé Nákhort, kétszáz esztendeig, és nemze fiakat és leányokat.
\par 24 Nákhor pedig huszonkilencz esztendõs vala, és nemzé Thárét.
\par 25 És éle Nákhor, minekutánna nemzé Thárét, száz tizenkilencz esztendeig, és nemze fiakat és leányokat.
\par 26 Tháré pedig hetven esztendõs vala, és nemzé Ábrámot, Nákhort, Háránt.
\par 27 Ez a Tháré nemzetsége: Tháré nemzé Ábrámot, Nákhort és Háránt. Hárán pedig nemzé Lótot.
\par 28 Meghala pedig Hárán, az õ atyjának Thárénak szemei elõtt, az õ születésének földjén, Úr-Kaszdimban.
\par 29 Ábrám pedig és Nákhor võnek magoknak feleséget: az Ábrám feleségének neve Szárai; a Nákhor feleségének neve Milkhah, Háránnak Milkhah atyjának és Jiszkáh atyjának leánya.
\par 30 Szárai pedig magtalan vala; nem vala néki gyermeke.
\par 31 És felvevé Tháré Ábrámot az õ fiát, és Lótot, Háránnak fiát, az õ unokáját, és Szárait, az õ menyét, Ábrámnak az õ fiának feleségét, és kiindulának együtt Úr-Kaszdimból, hogy Kanaán földére menjenek. És eljutának Háránig, és ott letelepedének.
\par 32 Vala pedig Tháré kétszáz öt esztendõs, és meghala Tháré Háránban.

\chapter{12}

\par 1 És monda az Úr Ábrámnak: Eredj ki a te földedbõl, és a te rokonságod közül, és a te atyádnak házából, a földre, a melyet én mutatok néked.
\par 2 És nagy nemzetté tészlek, és megáldalak téged, és felmagasztalom a te nevedet, és áldás leszesz.
\par 3 És megáldom azokat, a kik téged áldanak, és a ki téged átkoz, megátkozom azt: és megáldatnak te benned a föld minden nemzetségei.
\par 4 És kiméne Ábrám, a mint az Úr mondotta vala néki, és Lót is kiméne õ vele: Ábrám pedig hetvenöt esztendõs vala, mikor kiméne Háránból.
\par 5 És felvevé Ábrám az õ feleségét Szárait, és Lótot, az õ atyjafiának fiát, és minden szerzeményöket, a melyet szereztek vala, és a cselédeket, a kikre Háránban tettek vala szert, és elindulának, hogy Kanaán földére menjenek, és el is jutának a Kanaán földére.
\par 6 És általméne Ábrám a földön mind Sikhem vidékéig, Móréh tölgyeséig. Akkor Kananeusok valának azon a földön.
\par 7 És megjelenék az Úr Ábrámnak, és monda néki: A te magodnak adom ezt a földet. És Ábrám oltárt építe ott az Úrnak; a ki megjelent vala néki.
\par 8 Onnan azután a hegység felé méne Bétheltõl keletre és felüté sátorát: Béthel vala nyugatra, Hái pedig keletre, és ott oltárt építe az Úrnak, és segítségûl hívá az Úr nevét.
\par 9 És tovább költözék Ábrám: folyton dél felé húzódván.
\par 10 Azonban éhség lõn az országban, és Ábrám aláméne Égyiptomba, hogy ott tartózkodjék, mert nagy vala az éhség az országban.
\par 11 És lõn mikor közel vala, hogy bemenjen Égyiptomba, monda feleségének Szárainak: Ímé tudom, hogy szép ábrázatú asszony vagy.
\par 12 Azért mikor meglátnak téged az égyiptomiak, majd azt mondják: felesége ez; és engem megölnek, téged pedig életben tartanak.
\par 13 Mondd azért, kérlek, hogy húgom vagy; hogy jól legyen dolgom miattad, s életben maradjak te éretted.
\par 14 És lõn mikor Ábrám Égyiptomba érkezék, láták az égyiptomiak az asszonyt, hogy az nagyon szép.
\par 15 Mikor megláták õt a Faraó fõemberei, magasztalák a Faraó elõtt és elvivék az asszonyt a Faraó udvarába.
\par 16 És jól tõn érette Ábrámmal, és valának juhai, ökrei, szamarai, szolgái, szolgálói, nõstényszamarai és tevéi.
\par 17 De megveré az Úr a Faraót és az õ házát nagy csapásokkal, Száraiért, Ábrám feleségéért.
\par 18 Hívatá azért a Faraó Ábrámot és monda: Miért mívelted ezt velem? Miért nem mondottad meg énnékem, hogy ez néked feleséged?
\par 19 Miért mondottad: Húgom õ; azért vevém magamnak feleségûl. Most már imhol a te feleséged, vedd magadhoz és menj el.
\par 20 És parancsola felõle a Faraó némely embereknek, a kik elbocsáták õtet és az õ feleségét, és mindenét a mije vala.

\chapter{13}

\par 1 Feljöve azért Ábrám Égyiptomból, õ és az õ felesége, és mindene valamije vala, és Lót is õ vele, a déli tartományba.
\par 2 Ábrám pedig igen gazdag vala barmokkal, ezüsttel és aranynyal.
\par 3 És méne helyrõl helyre délfelõl mind Béthelig, oda a hol elõször vala az õ sátora, Béthel és Hái között.
\par 4 Annak az oltárnak a helyére, melyet ott elsõben készített vala: és segítségûl hívá ott Ábrám az Úrnak nevét.
\par 5 Lótnak is pedig a ki Ábrámmal jár vala, valának juhai, ökrei és sátorai.
\par 6 És nem bírá meg õket az a föld, hogy együtt lakjanak, mert sok jószáguk vala, és nem lakhatának együtt.
\par 7 Versengés is támada az Ábrám barompásztorai között, és a Lót barompásztorai között; és a Kananeusok és a Perizeusok is ott laknak vala akkor azon a földön.
\par 8 Monda azért Ábrám Lótnak: Ne legyen versengés közöttem és közötted, se az én pásztoraim között és a te pásztoraid között: hiszen atyafiak vagyunk.
\par 9 Avagy nincsen-é elõtted mind az egész föld? Válj el kérlek, tõlem; ha te balra tartasz, én jobbra megyek; ha te jobbra menéndesz, én balra térek.
\par 10 Felemelé azért Lót az õ szemeit, és látá, hogy a Jordán egész melléke bõvizû föld; mert minekelõtte elvesztette volna az Úr Sodomát és Gomorát, mind Czoárig olyan vala az, mint az Úr kertje, mint Égyiptom földe.
\par 11 És választá Lót magának a Jordán egész mellékét, és elköltözék Lót kelet felé: és elválának egymástól.
\par 12 Ábrám lakozik vala a Kanaán földén, Lót pedig lakozik vala a Jordán-melléki városokban, és sátoroz vala Sodomáig.
\par 13 Sodoma lakosai pedig nagyon gonoszok és bûnösök valának az Úr elõtt.
\par 14 Az Úr pedig monda Ábrámnak, minekutánna elválék tõle Lót: Emeld fel szemeidet és tekints arról a helyrõl, a hol vagy, északra, délre, keletre és nyugotra.
\par 15 Mert mind az egész földet, a melyet látsz, néked adom, és a te magodnak örökre.
\par 16 És olyanná tészem a te magodat, mint a földnek pora, hogyha valaki megszámlálhatja a földnek porát, a te magod is megszámlálható lesz.
\par 17 Kelj fel, járd be ez országot hosszában és széltében: mert néked adom azt.
\par 18 Elébb mozdítá azért sátorát Ábrám, és elméne, és lakozék Mamré tölgyesében, mely Hebronban van, és oltárt építe ott az Úrnak.

\chapter{14}

\par 1 Lõn pedig Amrafelnek, Sineár királyának és Ariókhnak, Elászár királyának, Khédorlaomernek, Élám királyának, és Thidálnak, a Góim királyának napjaiban:
\par 2 Hadat indítának ezek Bera, Sodoma királya ellen, Birsa, Gomora királya ellen, Sináb, Admáh királya ellen, Seméber, Czeboim királya ellen és Bélah, azaz Czoár királya ellen.
\par 3 Mind ezek a Sziddim völgyében egyesûlének; ez a Sóstenger.
\par 4 Tizenkét esztendeig szolgálták vala Khédorlaomert, és a tizenharmadik esztendõben ellene támadtak vala.
\par 5 A tizennegyedik esztendõben pedig eljöve Khédorlaomer, és a királyok, a kik õ vele valának, és megverék a Refeusokat Asztheroth Kárnajimban, és a Zuzeusokat Hámban, és az Emeusokat Sávé-Kirjáthajimban.
\par 6 És a Horeusokat az õ hegyökön, a Seiren, egész Él-Páránig, mely a puszta mellett van.
\par 7 És megtérének s menének Hén Mispátba, azaz Kádesbe, és elpusztíták az Amálekiták egész mezõségét, és az Emoreusokat is, kik laknak vala Háczaczon-Thámárban.
\par 8 Kiméne tehát Sodoma királya, Gomora királya, Admáh királya, Czeboim királya, és Bélah, azaz Czoár királya, és megütközének azokkal a Sziddim völgyében:
\par 9 Khédorlaomerrel, Élám királyával, és Thidállal, Gójim királyával, Amráfellel, Sineár királyával, és Ariókhkal, Elászár királyával: négy király öt ellen.
\par 10 A Sziddim völgye pedig tele vala szurok-forrásokkal. És megfutamodának Sodoma és Gomora királyai, és azokba esének: a megmaradottak pedig a hegységbe futának.
\par 11 És elvivék Sodomának és Gomorának minden jószágát és minden eleségét; és elmenének.
\par 12 Elvivék Lótot is az Ábrám atyjafiának fiát jószágostól együtt, és elmenének; mert Lót Sodomában lakik vala.
\par 13 Eljöve pedig egy menekûlt és hírûl hozá a héber Ábrámnak, a ki lakik vala az Emoreus Mamrénak, Eskol atyjafiának és Áner atyjafiának tölgyesében, a kik meg Ábrámnak szövetségesei valának.
\par 14 A mint meghallá Ábrám, hogy az õ atyjafia fogságba esett, felfegyverzé házában nevekedett háromszáz tizennyolcz próbált legényét és üldözve nyomula Dánig.
\par 15 És csapatokra oszolván ellenök éjszaka õ és szolgái, megveré õket, és ûzé õket mind Hóbáig, a mely Damaskustól balra esik.
\par 16 És visszahozá mind a jószágot; Lótot is, az õ atyjafiát jószágával egybe visszahozá, meg az asszonyokat és a népet.
\par 17 Minekutánna pedig visszatért Khédorlaomernek és a vele volt királyoknak megverésébõl, kiméne õ elébe Sodomának királya a Sáve völgyébe, azaz a király völgyébe.
\par 18 Melkhisédek pedig Sálem királya, kenyeret és bort hoza; õ pedig a Magasságos Istennek papja vala.
\par 19 És megáldá õt, és monda: Áldott legyen Ábrám a Magasságos Istentõl, ég és föld teremtõjétõl.
\par 20 Áldott a Magasságos Isten, a ki kezedbe adta ellenségeidet. És tizedet ada néki mindenbõl.
\par 21 És monda Sodoma királya Ábrámnak: Add nékem a népet, a jószágot pedig vedd magadnak.
\par 22 És monda Ábrám Sodoma királyának: Felemeltem az én kezemet az Úrhoz, a Magasságos Istenhez, ég és föld teremtõjéhez:
\par 23 Hogy én egy fonalszálat, vagy egy sarukötõt sem veszek el mindabból, a mi a tiéd, hogy ne mondjad: Én gazdagítottam meg Ábrámot.
\par 24 Semmi egyebet, csupán a legények élelmét, és ama férfiak részét, kik én velem eljöttek volt: Áner, Eskhol, Mamré, õk vegyék ki az õ részöket.

\chapter{15}

\par 1 E dolgok után lõn az Úr beszéde Ábrámhoz látomásban, mondván: Ne félj Ábrám: én paizsod vagyok tenéked, a te jutalmad felette igen bõséges.
\par 2 És monda Ábrám: Uram Isten, mit adnál énnékem, holott én magzatok nélkûl járok, és az, a kire az én házam száll, a Damaskusbeli Eliézer?
\par 3 És monda Ábrám: Ímé énnékem nem adtál magot, és ímé az én házam szolgaszülöttje lesz az én örökösöm.
\par 4 És ímé szóla az Úr õ hozzá, mondván: Nem ez lesz a te örökösöd: hanem a ki a te ágyékodból származik, az lesz a te örökösöd.
\par 5 És kivivé õt, és monda: Tekints fel az égre, és számláld meg a csillagokat, ha azokat megszámlálhatod; - és monda nékie: Igy lészen a te magod.
\par 6 És hitt az Úrnak és tulajdoníttaték az õnéki igazságul.
\par 7 És monda néki: Én vagyok az Úr, ki téged kihoztalak Úr-Kaszdimból, hogy néked adjam e földet, örökségedûl.
\par 8 És monda: Uram Isten, mirõl tudhatom meg, hogy öröklöm azt?
\par 9 És felele néki: Hozz nékem egy három esztendõs üszõt, egy három esztendõs kecskét, és egy három esztendõs kost, egy gerliczét és egy galambfiat.
\par 10 Elhozá azért mind ezeket, és kétfelé hasítá azokat, és mindeniknek fele részét a másik fele része átellenébe helyezteté; de a madarakat nem hasította vala kétfelé.
\par 11 És ragadozó madarak szállának e húsdarabokra, de Ábrám elûzi vala azokat.
\par 12 És lõn naplementekor, mély álom lepé meg Ábrámot, és ímé rémülés és nagy setétség szálla õ reá.
\par 13 És monda az Úr Ábrámnak: Tudván tudjad, hogy a te magod jövevény lesz a földön, mely nem övé, és szolgálatra szorítják, és nyomorgatják õket négyszáz esztendeig.
\par 14 De azt a népet, melyet szolgálnak, szintén megítélem én, és annakutánna kijõnek nagy gazdagsággal.
\par 15 Te pedig elmégy a te atyáidhoz békességgel, eltemettetel jó vénségben.
\par 16 Csak a negyedik nemzedék tér meg ide; mert az Emoreusok gonoszsága még nem tölt be.
\par 17 És mikor a nap leméne és setétség lõn, ímé egy füstölgõ kemencze, és tüzes fáklya, mely általmegyen vala a húsdarabok között.
\par 18 E napon kötött az Úr szövetséget Ábrámmal, mondván: A te magodnak adom ezt a földet Égyiptomnak folyóvizétõl fogva, a nagy folyóig, az Eufrátes folyóvízig.
\par 19 A Keneusokat, Kenizeusokat, és a Kadmoneusokat.
\par 20 A Hittheusokat, Perizeusokat, és a Refeusokat.
\par 21 Az Emoreusokat, Kananeusokat, Girgazeusokat, és a Jebuzeusokat.

\chapter{16}

\par 1 És Szárai, az Ábrám felesége nem szûle néki; de vala néki egy Égyiptomból való szolgálója, kinek neve Hágár vala.
\par 2 Monda azért Szárai Ábrámnak: Ímé az Úr bezárolta az én méhemet, hogy ne szûljek: kérlek, menj be az én szolgálómhoz, talán az által megépülök, és engede Ábrám a Szárai szavának.
\par 3 Vevé tehát Szárai, Ábrám felesége az Égyiptombeli Hágárt, az õ szolgálóját, tíz esztendõvel azután, hogy Ábrám a Kanaán földén letelepedék, és adá azt Ábrámnak az õ férjének feleségül.
\par 4 És béméne Hágárhoz, és az fogada az õ méhében; ez pedig a mint látta, hogy terhes, nem vala becsülete az õ asszonyának õ elõtte
\par 5 Monda azért Szárai Ábrámnak: Bántódásom van miattad. Én adtam öledbe szolgálómat, és mivelhogy látja, hogy teherbe esett, nincsen elõtte becsületem. Tegyen ítéletet az Úr én közöttem és te közötted.
\par 6 És monda Ábrám Szárainak: Imé a te szolgálód kezedben van, azt tedd vele a mit jónak látsz. Nyomorgatja vala azért Szárai, és az elfuta õ elõle.
\par 7 És találá õt az Úrnak angyala egy forrásnál a pusztában, annál a forrásnál, a mely a Súrba menõ úton van.
\par 8 És monda: Hágár, Szárai szolgálója! honnan jössz és hová mégy? És az monda: Az én asszonyomnak, Szárainak színe elõl futok én.
\par 9 Akkor monda néki az Úr angyala: Térj meg a te asszonyodhoz, és alázd meg magad az õ kezei alatt.
\par 10 És monda néki az Úrnak angyala: Felettébb megsokasítom a te magodat, hogy sokasága miatt megszámlálható se legyen.
\par 11 És monda néki az Úrnak angyala: Ímé te terhes vagy, és szûlsz fiat; és nevezd nevét Ismáelnek, mivelhogy meghallá Isten a te nyomorúságodat.
\par 12 Az pedig vadtermészetû ember lesz: az õ keze mindenek ellen, és mindenek keze õ ellene; és minden õ atyjafiának ellenébe üti fel sátorát.
\par 13 És nevezé Hágár az Úrnak nevét, a ki õ vele szólott vala: Te vagy a látomás Istene. Mert monda: Avagy nem e helyen láttam a látomás után?
\par 14 Annakokáért nevezé azt a forrást Lakhai Rói forrásának; ott van Kádes és Béred között.
\par 15 És fiat szûle Hágár Ábrámnak, és nevezé Ábrám az õ fiának nevét, a kit Hágár szûl vala néki, Ismáelnek.
\par 16 Ábrám pedig nyolczvanhat esztendõs vala, a mikor Hágár Ismáelt szûlé Ábrámnak.

\chapter{17}

\par 1 Mikor Ábrám kilenczvenkilencz esztendõs vala, megjelenék az Úr Ábrámnak, és monda néki: Én a mindenható Isten vagyok, járj én elõttem, és légy tökéletes.
\par 2 És megkötöm az én szövetségemet én közöttem és te közötted: és felette igen megsokasítlak téged.
\par 3 És arczára borúla Ábrám; az Isten pedig szóla õnéki, mondván:
\par 4 A mi engem illet, imhol az én szövetségem te veled, hogy népek sokaságának atyjává leszesz.
\par 5 És ne neveztessék ezután a te neved Ábrámnak, hanem legyen a te neved Ábrahám, mert népek sokaságának atyjává teszlek téged.
\par 6 És felette igen megsokasítalak téged; és népekké teszlek, és királyok is származnak tõled.
\par 7 És megállapítom az én szövetségemet én közöttem és te közötted, és te utánad a te magod között annak nemzedékei szerint örök szövetségûl, hogy legyek tenéked Istened, és a te magodnak te utánad.
\par 8 És adom tenéked és a te magodnak te utánnad a te bujdosásod földét, Kanaánnak egész földét, örök birtokul; és Istenök lészek nékik.
\par 9 Annakfelette monda Isten Ábrahámnak: Te pedig az én szövetségemet megõrizzed, te és a te magod te utánad az õ nemzedékei szerint.
\par 10 Ez pedig az én szövetségem, melyet meg kell tartanotok én közöttem és ti közöttetek, és a te utánnad való magod között: minden férfi körûlmetéltessék nálatok.
\par 11 És metéljétek körül a ti férfitestetek bõrének elejét és az lesz az én közöttem és ti közöttetek való szövetségnek jele.
\par 12 Nyolcznapos korában körûlmetéltessék nálatok minden férfigyermek nemzedékeiteknél; akár háznál született, akár pénzen vásároltatott valamely idegentõl, a ki nem a te magodból való.
\par 13 Körûlmetéltetvén körûlmetéltessék a házadban született és a pénzeden vett; és örökkévaló szövetségûl lesz az én szövetségem a ti testeteken.
\par 14 A körûlmetéletlen férfi pedig, a ki körûl nem metélteti az õ férfitestének bõrét, az ilyen lélek kivágattatik az õ népe közûl, mert felbontotta az én szövetségemet.
\par 15 És monda Isten Ábrahámnak: Szárainak, a te feleségednek nevét ne nevezd Szárainak, mert Sára az õ neve.
\par 16 És megáldom õt, és fiat is adok õ tõle néked, és megáldom, hogy legyen népekké; nemzetek királyai származzanak õ tõle.
\par 17 Ekkor arczára borúla Ábrahám, és nevete és gondolá az õ szívében: vajjon száz esztendõs embernek lesz-é gyermeke? avagy Sára kilenczven esztendõs lévén, szûlhet-é?
\par 18 És monda Ábrahám az Istennek: Vajha Ismáel élne te elõtted.
\par 19 Az Isten pedig monda: Kétségnélkûl a te feleséged Sára szûl néked fiat, és nevezed annak nevét Izsáknak, és megerõsítem az én szövetségemet õ vele örökkévaló szövetségûl az õ magvának õ utánna.
\par 20 Ismáel felõl is meghallgattalak: Ímé megáldom õt, és megszaporítom õt és megsokasítom õt felette nagyon; tizenkét fejedelmet nemz, és nagy néppé teszem õt.
\par 21 Az én szövetségemet pedig megerõsítem Izsákkal, kit néked szûl Sára ez idõkorban a következõ esztendõben.
\par 22 És elvégezé vele való beszédét, és felméne az Isten Ábrahámtól.
\par 23 Vevé azért Ábrahám Ismáelt az õ fiát, és háza minden szülöttét, és mind a pénzén vetteket, minden férfiat Ábrahám házanépe közûl és körûlmetélé férfitestöknek bõrét ugyanazon napon, a mikor szólott vala vele az Isten.
\par 24 Ábrahám pedig kilenczvenkilencz esztendõs vala, mikor körûlmetélé az õ férfitestének bõrét.
\par 25 Ismáel pedig az õ fia tizenhárom esztendõs vala, mikor körûlmetélék az õ férfitestének bõrét.
\par 26 Ugyanazon napon metéltetett körûl Ábrahám és Ismáel az õ fia.
\par 27 És házának minden férfi tagja, háza szülöttei és kik idegen embertõl pénzen vásároltattak, vele együtt körûlmetéltetének.

\chapter{18}

\par 1 Megjelenék pedig õ néki az Úr a Mamré tölgyesében, és õ ûl vala a sátor ajtajában, a hõ napon.
\par 2 És felemelé az õ szemeit, és látá, hogy ímé három férfiú áll õ elõtte. És látván, eléjök siete a sátor ajtajából, és földig meghajtá magát.
\par 3 És monda: Jó Uram, ha kedves vagyok te elõtted, kérlek, ne kerüld el a te szolgádat.
\par 4 Hadd hozzanak, kérlek, egy kevés vizet, és mossátok meg a ti lábaitokat, és dõljetek le a fa alatt.
\par 5 Én pedig hozok egy falat kenyeret, hogy erõsítsétek meg a ti szíveteket, azután menjetek tovább, mert azért tértetek be a ti szolgátokhoz. És mondának: Cselekedjél, a mint szólál.
\par 6 És besiete Ábrahám a sátorba Sárához és monda: Siess, gyúrj meg három mérték lisztlángot, és csinálj pogácsát.
\par 7 A baromhoz is elfuta Ábrahám, és hoza egy gyenge kövér borjút, és adá a szolgának, az pedig siete azt elkészíteni.
\par 8 És võn vajat és tejet, és a borjút, melyet elkészített vala, és eléjök tevé: és õ mellettök áll vala a fa alatt, azok pedig evének.
\par 9 És mondának néki: Hol van Sára a te feleséged? Õ pedig felele: Ímhol van a sátorban.
\par 10 És monda: Esztendõre ilyenkor bizonynyal megtérek hozzád és ímé akkor a te feleségednek Sárának fia lesz. Sára pedig hallgatózik vala a sátor ajtajában, mely annak háta megett vala.
\par 11 Ábrahám pedig és Sára élemedett korú öregek valának; megszünt vala Sáránál az asszonyi természet.
\par 12 Nevete azért Sára õ magában, mondván: Vénségemre lenne-é gyönyörûségem? meg az én uram is öreg!
\par 13 És monda az Úr Ábrahámnak: Miért nevetett Sára, ezt mondván: Vajjon csakugyan szûlhetek-é, holott én megvénhedtem?
\par 14 Avagy az Úrnak lehetetlen-é valami? Annak idején, esztendõre ilyenkor visszatérek hozzád, és fia lesz Sárának.
\par 15 Sára pedig megtagadá, mondván: Nem nevettem én; mivelhogy fél vala. De monda az Úr: Nem úgy van, mert bizony nevettél.
\par 16 Azután felkelvén onnan azok a férfiak, Sodoma felé tartanak vala. Ábrahám is velök méne, hogy elkisérje õket.
\par 17 És monda az Úr: Eltitkoljam-é én Ábrahámtól, a mit tenni akarok?
\par 18 Holott Ábrahám nagy és hatalmas néppé lesz; és benne megáldatnak a földnek minden nemzetségei.
\par 19 Mert tudom róla, hogy megparancsolja az õ fiainak és az õ házanépének õ utánna, hogy megõrizzék az Úrnak útát, igazságot és törvényt tévén, hogy beteljesítse az Úr Ábrahámon, a mit szólott felõle.
\par 20 Monda azután az Úr: Mivelhogy Sodomának és Gomorának kiáltása megsokasodott, és mivelhogy az õ bûnök felettébb megnehezedett:
\par 21 Alámegyek azért és meglátom, vajjon teljességgel a hozzám felhatott kiáltás szerint cselekedtek-é vagy nem? tudni akarom.
\par 22 És elfordulának onnan a férfiak, és menének Sodomába: Ábrahám pedig még az Úr elõtt áll vala.
\par 23 És hozzá járula Ábrahám és monda: Avagy elveszted-é az igazat is a gonoszszal egybe?
\par 24 Talán van ötven igaz abban a városban, avagy elveszted-é, és nem kedvezel-é a helynek az ötven igazért, a kik abban vannak?
\par 25 Távol legyen tõled, hogy ilyen dolgot cselekedjél, hogy megöld az igazat a gonoszszal, és úgy járjon az igaz mint a gonosz: Távol legyen tõled! Avagy az egész föld bírája nem szolgáltatna-é igazságot?
\par 26 És monda az Úr: Ha találok Sodomában a városon belõl ötven igazat, mind az egész helynek megkegyelmezek azokért.
\par 27 És felele Ábrahám, és monda: Immár merészkedtem szólani az én Uramnak, noha én por és hamu vagyok.
\par 28 Ha az ötven igaznak talán öt híja lesz, elveszted-é az öt miatt az egész várost? És monda: Nem vesztem el, ha találok ott negyvenötöt.
\par 29 És ismét szóla hozzá és monda: Hátha találtatnak ott negyvenen? És monda Õ: Nem teszem meg a negyvenért.
\par 30 Mégis monda: Kérlek, ne haragudjék meg az én Uram ha szólok: Hátha találtatnak ott harminczan? És Õ felele: Nem teszem meg, ha találok ott harminczat.
\par 31 És õ monda: Immár merészkedtem szólani az én Uramnak: Hátha találtatnak ott húszan? Felele: Nem vesztem el a húszért.
\par 32 És monda: Ne haragudjék kérlek az én Uram ha szólok még ez egyszer: Hátha találtatnak ott tízen? És Õ monda: Nem vesztem el a tízért.
\par 33 És elméne az Úr, minekutánna elvégezte Ábrahámmal való beszélgetését; Ábrahám pedig megtére az õ helyére.

\chapter{19}

\par 1 Mikor a két angyal estére Sodomába jutott, Lót Sodoma kapujában ûl vala, és a mint meglátá õket Lót, felkele eléjök, és arczczal a földre borúla.
\par 2 És monda: Ímé én Uraim kérlek, térjetek be a ti szolgátok házához, és háljatok ott, és mossátok meg lábaitokat; reggel korán felkelhettek és indulhattok útatokra. Azok pedig mondának: Nem, hanem az utczán hálunk meg.
\par 3 De nagyon unszolá õket, és betérének hozzá, és bemenének az õ házába; õ pedig szerze nékik vendégséget, és pogácsát is süte, és evének.
\par 4 Lefekvésök elõtt a város férfiai, Sodoma férfiai körûlvevék a házat, ifja, örege, mind az egész község egytõl egyig.
\par 5 És szólíták Lótot, mondván néki: Hol vannak a férfiak, a kik te hozzád jövének az éjjel? Hozd ki azokat mi hozzánk, hadd ismerjük õket.
\par 6 És kiméne Lót õ hozzájok az ajtó eleibe, és bezárá maga után az ajtót.
\par 7 És monda: Kérlek atyámfiai, ne cselekedjetek gonoszságot.
\par 8 Imé van énnékem két leányom, a kik még nem ismertek férfiat, kihozom azokat ti hozzátok, és cselekedjetek velök a mint néktek tetszik, csakhogy ezekkel az emberekkel ne csináljatok semmit, mivelhogy az én hajlékom árnyéka alá jöttenek.
\par 9 Azok pedig mondának: Eredj el innen. Ismét mondának: Ez egy maga nálunk a jövevény s õ szabja a törvényt? Majd gonoszbul cselekszünk veled, hogy nem azokkal. És reá rohanának a férfiúra, Lótra, felette igen, és azon valának, hogy betörik az ajtót.
\par 10 De kinyújták azok a férfiak kezeiket, és bevonák Lótot magokhoz a házba és bezárák az ajtót.
\par 11 Az embereket pedig, kik a ház ajtaja elõtt valának, vaksággal verék meg kicsinytõl nagyig, annyira, hogy elfáradának az ajtó keresésében.
\par 12 És mondának a férfiak Lótnak: Ki van még itt hozzád tartozó? võdet, fiaidat és leányaidat, és mindenedet, a mi a tied a városban, vidd ki e helyrõl.
\par 13 Mert mi elvesztjük e helyet, mivelhogy ezek kiáltása nagyra nõtt az Úr elõtt; és az Úr küldött minket, hogy elveszítsük ezt.
\par 14 Kiméne azért Lót, és szóla az õ võinek, kik az õ leányait elvették vala, és monda: Keljetek fel, menjetek ki e helybõl, mert elveszti az Úr e várost: de az õ võinek úgy tetszék, mintha tréfálna.
\par 15 És mikor a hajnal feljött, sürgetik vala az Angyalok Lótot, mondván: Kelj fel, vedd a te feleségedet és jelenlevõ két leányodat, hogy el ne veszsz a városnak bûne miatt.
\par 16 Mikor pedig késedelmeskedék, megragadák a férfiak az õ kezét és az õ feleségének kezét és két leánya kezét, az Úrnak iránta való irgalmából, és kivivék õt és ott hagyák a városon kívül.
\par 17 És lõn mikor kivivék õket, monda az egyik: Mentsd meg a te életedet, hátra ne tekints, és meg ne állj a környéken; a hegyre menekülj, hogy el ne veszsz.
\par 18 És monda Lót nékik: Ne oh Uram!
\par 19 Ímé a te szolgád kegyelmet talált te elõtted, és nagy a te irgalmasságod, melyet mutattál irántam, hogy életemet megtartotta: de én nem menekûlhetek a hegyre, nehogy utólérjen a veszedelem, és meghaljak.
\par 20 Ímhol az a város közel van, hogy oda fussak, kicsiny is, hadd menekûljek kérlek oda, lám kicsiny az; és én életben maradok.
\par 21 Monda azért néki: Ím tekintek rád e dologban is, és nem pusztítom el a várost, a melyrõl szólottál.
\par 22 Siess, menekülj oda, mert semmit sem tehetek addig, míg oda nem érsz. Azért nevezték annak a városnak nevét Czóárnak.
\par 23 A nap feljött vala a földre, mikor Lót Czóárba ére.
\par 24 És bocsáta az Úr Sodomára és Gomorára kénköves és tûzes esõt az Úrtól az égbõl.
\par 25 És elsûlyeszté ama városokat, és azt az egész vidéket, és a városok minden lakosait, és a föld növényeit is.
\par 26 És hátra tekinte az õ felesége, és sóbálványnyá lõn.
\par 27 Ábrahám pedig reggel arra a helyre indúla, a hol az Úr színe elõtt állott vala.
\par 28 És tekinte Sodoma és Gomora felé, és az egész környék földje felé; és látá, és ímé felszálla a földnek füstje, mint a kemencze füstje.
\par 29 És lõn mikor elveszté Isten annak a környéknek városait, megemlékezék az Isten Ábrahámról, és kiküldé Lótot a veszedelembõl, mikor elsûlyeszté a városokat, a melyekben lakott vala Lót.
\par 30 Lót pedig felméne Czóárból, és letelepedék a hegyen, és vele együtt az õ két leánya is, mert fél vala Czóárban lakni; lakozék tehát egy barlangban õ és az õ két leánya.
\par 31 És monda a nagyobbik a kisebbiknek: A mi atyánk megvénhedett, és nincsen a földön férfiú, a ki mi hozzánk bejöhetne az egész föld szokása szerint.
\par 32 Jer, adjunk bort inni a mi atyánknak, és háljunk õ vele, és támaszszunk magot a mi atyánktól.
\par 33 Adának azért inni bort az õ atyjoknak azon éjszaka, és beméne a nagyobbik, és hála az õ atyjával, ez pedig semmit sem tuda annak sem lefekvésérõl, sem fölkelésérõl.
\par 34 És lõn másodnapon, monda a nagyobbik a kisebbiknek: Ímé a mult éjjel én háltam atyámmal, adjunk néki bort inni ez éjjel is, és menj be te, hálj vele, és támaszszunk magot a mi atyánktól.
\par 35 Adának azért azon éjszaka is az õ atyjoknak bort inni, és felkele a kisebbik is és vele hála; õ pedig semmit sem tuda annak sem lefekvésérõl, sem fölkelésérõl.
\par 36 És teherbe esének Lót leányai mindketten az õ atyjoktól.
\par 37 És szûle a nagyobbik fiat, és nevezé annak nevét Moábnak; ez a Moábiták atyja mind e mai napig.
\par 38 A kisebbik is fiat szûle és nevezé annak nevét Benamminak. Ez az Ammoniták atyja mind a mai napig.

\chapter{20}

\par 1 És elköltözék onnan Ábrahám a déli tartományba, és letelepedék Kádes és Súr között, és tartózkodék Gérárban.
\par 2 És monda Ábrahám Sáráról az õ feleségérõl: Én húgom õ. Elkülde azért Abimélek Gérárnak királya, és elviteté Sárát.
\par 3 De Isten Abimélekhez jöve éjjeli álomban, és monda néki: Ímé meghalsz az asszonyért, a kit elvettél, holott férjnél van.
\par 4 Abimélek pedig nem illette vala õt, és monda: Uram, az ártatlan népet is megölöd-é?
\par 5 Avagy nem õ mondotta-é nékem: én húgom õ; s ez is azt mondotta: én bátyám õ. Szívem ártatlanságában, és kezeim tisztaságában cselekedtem ezt.
\par 6 És monda az Isten néki álomban: Én is tudom, hogy szívednek ártatlanságában mívelted ezt, azért tartóztattalak én is, hogy ne vétkezzél ellenem, azért nem engedtem, hogy illessed azt.
\par 7 Mostan azért add vissza az embernek az õ feleségét, mert Próféta õ: és imádkozik te éretted, és élsz; hogyha pedig vissza nem adod: tudd meg, hogy halállal halsz meg te, és minden hozzád tartozó.
\par 8 Felkele azért Abimélek reggel, és elõhívatá minden szolgáját, s fülök hallatára mindezeket elbeszélé és az emberek igen megfélemlének.
\par 9 És hívatá Abimélek Ábrahámot, és monda néki: Mit cselekedtél mi velünk? És mit vétettem te ellened, hogy én reám és az én országomra ilyen nagy bûnt hoztál? A miket cselekedni nem szabad, olyan dolgokat cselekedtél ellenem.
\par 10 És monda Abimélek Ábrahámnak: Mit láttál, hogy ezt a dolgot cselekedted?
\par 11 Felele Ábrahám: Bizony azt gondoltam: nincsen istenfélelem e helyen, és megölnek engem az én feleségemért.
\par 12 De valósággal húgom is, az én atyámnak leánya õ, csakhogy nem az én anyámnak leánya; és így lõn feleségemmé.
\par 13 És lõn hogy a mikor kibujdostata engem az Isten az én atyámnak házából, azt mondém néki: Ilyen kegyességet cselekedjél én velem, mindenütt valahová megyünk, azt mondjad én felõlem: én bátyám ez.
\par 14 Akkor Abimélek vett juhokat, ökröket, szolgákat és szolgálókat, és adá Ábrahámnak, és vissza adá néki Sárát is az õ feleségét.
\par 15 És monda Abimélek: Ímé elõtted van az én országom, a hol tenéked jónak tetszik, ott lakjál.
\par 16 Sárának pedig monda: Ímé ezer ezüst pénzt adtam a te bátyádnak, ímé az neked a szemek befedezõje mindazok elõtt, a kik veled vannak; és így mindenképpen igazolva vagy.
\par 17 Könyörge azért Ábrahám az Istennek, és meggyógyítá Isten Abiméleket, és az õ feleségét, és az õ szolgálóit, és szûlének.
\par 18 Mert az Úr erõsen bezárta vala az Abimélek háza népének méhét, Sáráért az Ábrahám feleségéért.

\chapter{21}

\par 1 Az Úr pedig meglátogatá Sárát, a mint mondotta vala, és akképen cselekedék az Úr Sárával, a miképen szólott vala.
\par 2 Mert fogada Sára az õ méhében, és szûle fiat Ábrahámnak az õ vénségében, abban az  idõben, melyet mondott vala néki az Isten.
\par 3 És nevezé Ábrahám az õ fiának nevét, a ki néki született vala, a kit szûlt vala néki Sára, Izsáknak:
\par 4 És körûlmetélé Ábrahám az õ fiát Izsákot, nyolcznapos korában, a mint parancsolta vala néki az Isten.
\par 5 Ábrahám pedig száz esztendõs vala, mikor születék néki az õ fia Izsák.
\par 6 És monda Sára: Nevetést szerzett az Isten, énnékem; a ki csak hallja, nevet rajtam.
\par 7 Ismét monda: Ki mondotta volna Ábrahámnak, hogy Sára fiakat szoptat? s ímé fiat szûltem vénségére.
\par 8 És felnevekedék a gyermek, és elválasztaték; Ábrahám pedig nagy vendégséget szerze azon a napon, a melyen Izsák elválasztaték.
\par 9 Mikor pedig Sára nevetgélni látá az Égyiptombéli Hágárnak fiát, kit Ábrahámnak szûlt vala,
\par 10 Monda Ábrahámnak: Kergesd el ezt a szolgálót az õ fiával egybe, mert nem lesz örökös e szolgáló fia az én fiammal, Izsákkal.
\par 11 Ábrahámnak pedig igen nehéznek látszék e dolog, az õ fiáért.
\par 12 De monda az Isten Ábrahámnak: Ne lássék elõtted nehéznek a gyermeknek és a szolgálónak a dolga; valamit mond néked Sára, engedj az õ szavának, mert Izsákról neveztetik a te magod.
\par 13 Mindazonáltal a szolgálóleány fiát is néppé teszem, mivelhogy a te magod õ.
\par 14 Felkele azért Ábrahám jó reggel, és võn kenyeret és egy tömlõ vizet, és adá Hágárnak, és feltevé azt és a gyermeket annak vállára s elbocsátá. Az pedig elméne, és bujdosék a Beérseba pusztájában.
\par 15 Hogy elfogyott a víz a tömlõbõl, letevé a gyermeket egy bokor alá.
\par 16 És elméne s leûle által ellenébe mintegy nyillövésnyi távolságra; mert azt mondja vala: Ne lássam mikor a gyermek meghal. Leûle tehát által ellenébe, és fölemelé szavát és síra.
\par 17 Meghallá pedig Isten a gyermeknek szavát és kiálta az Isten angyala az égbõl Hágárnak, és monda néki: Mi lelt téged Hágár? ne félj, mert az Isten meghallotta a gyermeknek szavát, ott a hol van.
\par 18 Kelj fel, vedd fel a gyermeket, és viseld gondját, mert nagy néppé teszem õt.
\par 19 És megnyitá Isten az õ szemeit, és láta egy vízforrást, oda méne azért, és megtölté a tömlõt vízzel, és inni ada a gyermeknek.
\par 20 És vala Isten a gyermekkel, s az felnövekedék, és lakik vala a pusztában, és lõn íjászszá.
\par 21 Lakozék pedig Párán pusztájában, és võn néki anyja feleséget Égyiptom földérõl.
\par 22 És lõn abban az idõben, hogy Abimélek és Pikhól annak hadvezére megszólíták Ábrahámot mondván: Az Isten van te veled mindenben a mit cselekszel.
\par 23 Mostan azért esküdj meg énnékem az Istenre itt, hogy sem én ellenem, sem fiam, sem unokám ellen álnokságot nem cselekszel, hanem azzal a szeretettel, a melylyel én te irántad viseltettem, viseltetel te is én irántam és az ország iránt, a melyben jövevény voltál.
\par 24 És monda Ábrahám: Én megesküszöm.
\par 25 Megdorgálá pedig Ábrahám Abiméleket a kútért, melyet erõvel elvettek vala az Abimélek szolgái.
\par 26 És monda Abimélek: Nem tudom kicsoda mívelte e dolgot; te sem jelentetted nekem, s én sem hallottam, hanem csak ma.
\par 27 Vett azért Ábrahám juhokat, barmokat és adá Abiméleknek; és egymással szövetséget kötének.
\par 28 És külön állíta Ábrahám a nyájból hét juhot.
\par 29 És monda Abimélek Ábrahámnak: Mire való e hét juh, melyet külön állítál?
\par 30 És felele Ábrahám: Ezt a hét juhot vedd tõlem, hogy bizonyságul legyenek nékem, hogy én ástam ezt a kútat.
\par 31 Azért nevezék azt a helyet Beérsebának, mivelhogy ott esküdtek vala meg mind a ketten.
\par 32 Megköték tehát a szövetséget Beérsebában, és felkele Abimélek és Pikhól annak hadvezére és visszatérének a Filiszteusok földére.
\par 33 Ábrahám pedig tamariskusfákat ültete Beérsebában, és segítségûl hívá ott az örökkévaló Úr Istennek nevét.
\par 34 És sok ideig tartózkodék Ábrahám a Filiszteusok földén.

\chapter{22}

\par 1 És lõn ezeknek utána, az Isten megkisérté Ábrahámot, és monda néki: Ábrahám! S az felele: Ímhol vagyok.
\par 2 És monda: Vedd a te fiadat, ama te egyetlenegyedet, a kit szeretsz, Izsákot, és menj el Mórijának földére és áldozd meg ott égõ áldozatul a hegyek közûl egyen, a melyet mondándok néked.
\par 3 Felkele azért Ábrahám jó reggel, és megnyergelé az õ szamarát, és maga mellé vevé két szolgáját, és az õ fiát Izsákot, és fát hasogatott az égõ áldozathoz. Akkor felkele és elindula a helyre, melyet néki az Isten mondott vala.
\par 4 Harmadnapon felemelé az õ szemeit Ábrahám. és látá a helyet messzirõl.
\par 5 És monda Ábrahám az õ szolgáinak: Maradjatok itt a szamárral, én pedig és ez a gyermek elmegyünk amoda és imádkozunk, azután visszatérünk hozzátok.
\par 6 Vevé azért Ábrahám az égõ áldozathoz való fákat, és feltevé az õ fiára Izsákra, õ maga pedig kezébe vevé a tüzet, és a kést, és mennek vala ketten együtt.
\par 7 És szóla Izsák Ábrahámhoz az õ atyjához, és monda: Atyám! Az pedig monda: Ímhol vagyok, fiam! És monda Izsák: Ímhol van a tûz és a fa; de hol van az égõ áldozatra való bárány?
\par 8 És monda Ábrahám: Az Isten majd gondoskodik az égõ áldozatra való bárányról, fiam; és mennek vala ketten együtt.
\par 9 Hogy pedig eljutának arra a helyre, melyet Isten néki mondott vala, megépíté ott Ábrahám az oltárt, és reá raká a fát, és megkötözé Izsákot az õ fiát, és feltevé az oltárra, a fa-rakás tetejére.
\par 10 És kinyújtá Ábrahám az õ kezét és vevé a kést, hogy levágja az õ fiát.
\par 11 Akkor kiálta néki az Úrnak Angyala az égbõl, és monda: Ábrahám! Ábrahám! Õ pedig felele: Ímhol vagyok.
\par 12 És monda: Ne nyujtsd ki a te kezedet a gyermekre, és ne bántsd õt: mert most már tudom, hogy istenfélõ vagy, és nem kedvezél a te fiadnak, a te egyetlenegyednek én érettem.
\par 13 És felemelé Ábrahám az õ szemeit, és látá hogy ímé háta megett egy kos akadt meg szarvánál fogva a szövevényben. Oda méne tehát Ábrahám, és elhozá a kost, és azt áldozá meg égõ áldozatul az õ fia helyett.
\par 14 És nevezé Ábrahám annak a helynek nevét Jehova-jire-nek. Azért mondják ma is: Az Úr hegyén a gondviselés.
\par 15 És kiálta az Úrnak Angyala Ábrahámnak másodszor is az égbõl.
\par 16 És monda: Én magamra esküszöm azt mondja az Úr: mivelhogy e dolgot cselekedéd, és nem kedvezél a te fiadnak, a te egyetlenegyednek:
\par 17 Hogy megáldván megáldalak tégedet, és bõségesen megsokasítom a te magodat mint az ég csillagait, és mint a fövényt, mely a tenger partján van, és a te magod örökség szerint fogja bírni az õ ellenségeinek kapuját.
\par 18 És megáldatnak a te magodban a földnek minden nemzetségei, mivelhogy engedtél az én beszédemnek.
\par 19 Megtére azért Ábrahám az õ szolgáihoz, és felkelének és együtt elmenének Beérsebába, mert lakozék Ábrahám Beérsebában.
\par 20 És lõn ezeknek utánna, hírt hozának Ábrahámnak mondván: Ímé Milkha is szûlt fiakat Nákhornak, a te atyádfiának:
\par 21 Úzt az õ elsõszülöttét, és Búzt annak testvérét, és Kemuélt Arámnak atyját.
\par 22 És Keszedet, Házót, Pildást, Jidláfot és Bethuélt.
\par 23 Bethuél pedig nemzé Rebekát. Ezt a nyolczat szûlé Milkha Nákhornak az Ábrahám atyjafiának.
\par 24 Az õ ágyasa is, kinek neve Reuma vala, szûlé néki Tebáhot, Gakhámot, Thakhást és Mahákhát.

\chapter{23}

\par 1 Vala pedig Sárának élete száz huszonhét esztendõ. Ezek Sára életének esztendei.
\par 2 És meghala Sára Kirját-Arbában azaz Hebronban a Kanaán földén, és beméne Ábrahám, hogy gyászolja Sárát és sirassa õt.
\par 3 Felkele azután Ábrahám az õ halottja elõl, és szóla a Khéth fiainak, mondván:
\par 4 Idegen és jövevény vagyok közöttetek: Adjatok nékem temetésre való örökséget ti nálatok, hadd temessem el az én halottamat én elõlem.
\par 5 Felelének pedig a Khéth fiai Ábrahámnak, mondván õnéki:
\par 6 Hallgass meg minket uram: Istentõl való fejedelem vagy te mi közöttünk, a mi temetõhelyeink közûl a mely legtisztességesebb, abba temesd el a te halottadat, közûlünk senki sem tiltja meg tõled az õ temetõhelyét, hogy eltemethesd a te halottadat.
\par 7 És felkele Ábrahám, és meghajtá magát a földnek népe elõtt, a Khéth fiai elõtt.
\par 8 És szóla õ velök mondván: Ha azt akarjátok, hogy eltemessem az én halottamat én elõlem: hallgassatok meg engemet, és esedezzetek én érettem Efron elõtt, Czohár fia elõtt.
\par 9 Hogy adja nékem Makpelá barlangját, mely az övé, mely az õ mezejének szélében van: igaz árán adja nékem azt, ti köztetek temetésre való örökségûl.
\par 10 Efron pedig ûl vala a Khéth fiai között. Felele azért Efron a Khitteus, Ábrahámnak, a Khéth fiainak és mindazoknak hallatára, a kik bemennek vala az õ városának kapuján, mondván:
\par 11 Nem úgy uram, hallgass meg engem: azt a mezõt néked adom, s a barlangot, mely abban van, azt is néked adom, népem fiainak szeme láttára adom azt néked, temesd el halottadat.
\par 12 És meghajtá magát Ábrahám a földnek népe elõtt.
\par 13 És szóla Efronhoz a föld népének hallatára, mondván: Ha mégis meghallgatnál engem! megadom a mezõnek árát, fogadd el tõlem; azután eltemetem ott az én halottamat.
\par 14 És felele Efron Ábrahámnak, mondván néki:
\par 15 Uram! hallgass meg engemet; négyszáz ezüst siklusos föld, micsoda az én köztem és te közötted? Csak temesd el a te halottadat.
\par 16 Engede azért Ábrahám Efronnak és odamérte Ábrahám Efronnak az ezüstöt, a melyet mondott vala a Khéth fiainak hallatára; kalmároknál kelendõ négyszáz ezüst siklust.
\par 17 Így lett Efronnak Makpelában levõ mezeje, mely Mamré átellenében van, a mezõ benne levõ barlanggal, és minden a mezõben levõ fa az egész határban köröskörûl
\par 18 Ábrahámnak birtoka, a Khéth fiainak, mind azoknak szeme elõtt, a kik az õ városának kapuján bemennek vala.
\par 19 Azután eltemeté Ábrahám az õ feleségét Sárát a Makpelá mezejének barlangjába Mamréval szemben. Ez Hebron a Kanaán földjén.
\par 20 Így erõsítteték meg a mezõ és a benne lévõ barlang Ábrahámnak temetésre való örökségûl a Khéth fiaitól.

\chapter{24}

\par 1 Ábrahám pedig vén élemedett ember vala, és az Úr mindenben megáldotta vala Ábrahámot.
\par 2 Monda azért Ábrahám az õ háza öregebb szolgájának, a ki õnéki mindenében gazda vala: Tedd a kezed a tomporom alá!
\par 3 Hogy megeskesselek téged az Úrra, a mennynek Istenére, és a földnek Istenére, hogy nem vészesz feleséget az én fiamnak a Kananeusok leányai közûl, a kik között lakom.
\par 4 Hanem elmégysz az én hazámba, és az én rokonságim közé és onnan vészesz feleséget az én fiamnak Izsáknak.
\par 5 Monda pedig õnéki a szolga: Hátha az a leányzó nem akar velem eljõni e földre, ugyan vissza vigyem-é a te fiadat arra a földre, a honnan kijöttél vala?
\par 6 Felele néki Ábrahám: Vigyázz, az én fiamat oda vissza ne vidd.
\par 7 Az Úr az égnek Istene, ki engemet kihozott az én atyámnak házából, és az én rokonságimnak földérõl, a ki szólt nékem, és megesküdött nékem mondván: A te magodnak adom ezt a földet; elbocsátja az õ Angyalát te elõtted, hogy onnan végy az én fiamnak feleséget.
\par 8 Hogyha pedig nem akar a leányzó teveled eljõni, ment lészesz az én esketésem alól; csakhogy az én fiamat oda vissza ne vidd.
\par 9 Veté azért a szolga az õ kezét az õ urának Ábrahámnak tompora alá, és megesküvék néki e dolog felõl.
\par 10 És võn a szolga tíz tevét az õ urának tevéi közûl, és elindula; (mert az urának minden gazdagsága az õ kezében vala). Felkele tehát és elméne Mésopotámiába, a Nákhor városába.
\par 11 És megpihenteté a tevéket a városon kivûl egy kútfõnél, este felé, mikor a leányok vizet meríteni járnak.
\par 12 És monda: Uram! én uramnak Ábrahámnak Istene, hozd elém még ma, és légy kegyelmes az én uram Ábrahám iránt.
\par 13 Ímé én a víz forrása mellé állok és e város lakosainak leányai kijõnek vizet meríteni.
\par 14 Legyen azért, hogy a mely leánynak ezt mondom: Hajtsd meg a te vedredet, hogy igyam, és az azt mondándja: igyál, sõt a te tevéidet is megitatom: hogy azt rendelted légyen a te szolgádnak Izsáknak, és errõl ismerjem meg, hogy irgalmasságot cselekedtél az én urammal.
\par 15 És lõn, minekelõtte elvégezte volna a beszédet, ímé jõ vala Rebeka, Bethuélnek leánya, a ki Milkhának, az Ábrahám testvérének Nákhor feleségének vala fia, és pedig vedrével a vállán.
\par 16 A leányzó pedig felette szép ábrázatú vala; szûz, és férfi még nem ismeré õt, és aláméne a forrásra, és megtölté vedrét, és feljöve.
\par 17 Akkor a szolga eleibe futamodék és monda: Kérlek, adj innom nékem egy kevés vizet a te vedredbõl.
\par 18 Az pedig monda: Igyál uram! és sietve leereszté a vedret az õ kezére, és inni ada néki.
\par 19 És minekutána eleget adott néki innia, monda: A te tevéidnek is merítek, míg eleget nem isznak.
\par 20 És sietett és kiüríté vedrét a válúba és ismét elfuta a forrásra meríteni, és meríte mind az õ tevéinek.
\par 21 Az ember pedig álmélkodva néz vala reá, és veszteg hallgat vala, tudni akarván: vajjon szerencséssé teszi-é az Úr az õ útját, vagy nem.
\par 22 És lõn, mikor a tevék már eleget ittak, elévõn az ember egy aranyfüggõt, a melynek súlya fél siklus, és két karpereczet, a melynek súlya tíz arany.
\par 23 És monda: Kinek a leánya vagy te? kérlek mondd meg nékem: van-é a te atyádnak házában hálásra való helyünk?
\par 24 Az pedig felele néki: Bethuél leánya vagyok a Milkha fiáé, a kit õ Nákhornak szûlt.
\par 25 Azt is mondá: Szalma is, abrak is bõven van minálunk, és hálásra való hely is van.
\par 26 Meghajtá azért magát az ember, és imádá az Urat.
\par 27 És monda: Áldott az Úr az én uramnak Ábrahámnak Istene, ki nem vonta meg az õ irgalmasságát és hûségét az én uramtól. Az Úr vezérlett engem ez útamban az én uram atyjafiainak házához.
\par 28 Elfuta azonközben a leányzó, és elbeszélé az õ anyja házában, a mint ezek történtek.
\par 29 Vala pedig Rebekának egy bátyja, kinek neve Lábán vala. És kifutamodék Lábán ahhoz az emberhez a forráshoz.
\par 30 Mert mikor látta a függõt, és a pereczeket az õ hugának karjain, és hallotta húgának Rebekának beszédét, a ki ezt mondja vala: Így szóla nékem az a férfiú; akkor méne ki a férfiúhoz; és ímé ez ott áll vala a tevék mellett a forrásnál.
\par 31 És monda: Jõjj be Istennek áldott embere; mit állasz ide kinn? holott én elkészítettem a házat, és a tevéknek is van hely.
\par 32 Beméne azért a férfiú a házhoz, õ pedig lenyergelé a tevéket, és ada a tevéknek szalmát és abrakot; és vizet az õ lábai megmosására és az emberek lábainak, kik õ vele valának.
\par 33 És enni valót tevének eleibe, de õ monda: Nem eszem, míg el nem mondom az én beszédemet. És szóla: Mondd el.
\par 34 Monda azért: Én az Ábrahám szolgája vagyok.
\par 35 Az Úr pedig igen megáldotta az én uramat, úgy hogy nagygyá lett: mert adott néki juhokat, barmokat, ezüstöt, aranyat, szolgákat, szolgálóleányokat, tevéket, szamarakat.
\par 36 És Sára az én uramnak felesége fiat szûlt az én uramnak, az õ vénségében, és annak adá mindenét, a mije van.
\par 37 Engem pedig megesküdtetett az én uram, mondván: Ne végy feleséget az én fiamnak a Kananeusok leányai közûl, a kiknek földjén én lakom.
\par 38 Hanem menj el az én atyámnak házához, és az én rokonságom közé, hogy onnan végy feleséget az én fiamnak.
\par 39 Mikor pedig azt mondám az én uramnak: Hátha nem akarna velem az a leányzó eljõni?
\par 40 Monda nékem: Az Úr, a kinek én színe elõtt jártam, elbocsátja az õ angyalát teveled, és szerencséssé teszi a te útadat, hogy feleséget vehess az én fiamnak az én nemzetségem közûl, és az én atyám házából.
\par 41 Csak akkor leszesz fölmentve esketésem alól, ha elmenéndesz az én nemzetségem közé; és ha nem adják oda: ment leszesz az én esketésem alól.
\par 42 Mikor ma a forráshoz érkezém, mondék: Uram, én uramnak, Ábrahámnak Istene, vajha szerencséssé tennéd az én útamat, melyen járok:
\par 43 Ímé én e forrás mellett állok; és legyen, hogy az a hajadon, a ki kijön vizet meríteni, s a kinek azt mondom: Adj innom nékem egy kevés vizet a te vedredbõl,
\par 44 És az ezt mondja nékem: Te is igyál, és a te tevéidnek is merítek; az legyen a feleség, a kit az Úr az én uram fiának rendelt.
\par 45 Én még el sem végeztem vala az én szívemben a beszédet, és ímé kijõ vala Rebeka, vedrével a vállán, és leméne a forrásra és meríte, én pedig mondék néki: Adj innom kérlek.
\par 46 Õ pedig sietett és leereszté az õ vedrét és monda: Igyál, sõt a te tevéidnek is inni adok; és én ivám, s a tevéknek is inni ada.
\par 47 És megkérdezém õt és mondék: Ki leánya vagy? õ pedig felele: Bethuélnek, a Nákhor fiának leánya vagyok, a kit Milkha szûlt vala õnéki. Ekkor a függõt orrába, és e pereczeket a karjaira tevém.
\par 48 Meghajtván azért magamat, imádám az Urat, és áldám az Urat, az én uramnak Ábrahámnak Istenét, ki engem igaz úton vezérelt, hogy az én uram atyjafiának leányát vegyem az õ fiának feleségûl.
\par 49 Most azért, ha szeretettel és hûséggel akartok lenni az én uramhoz, mondjátok meg; ha pedig nem, adjátok tudtomra, hogy én vagy jobbra vagy balra forduljak.
\par 50 És felele Lábán és Bethuél, és mondának: Az Úrtól van e dolog: Nem mondhatunk neked sem jót, sem rosszat.
\par 51 Ímé elõtted van Rebeka, vegyed, menj el; és legyen felesége a te urad fiának, a mint az Úr elvégezte.
\par 52 És lõn, a mint hallja vala az Ábrahám szolgája azoknak beszédét, meghajtá magát a földig az Úr elõtt.
\par 53 És hoza elõ a szolga ezüst edényeket és arany edényeket és ruhákat, és adá azokat Rebekának: drága ajándékokat ada az õ bátyjának is és az õ anyjának.
\par 54 Evének azután és ivának, õ és a férfiak, a kik õ vele valának, és ott hálának. Mikor pedig felkelének reggel, monda: Bocsássatok el engem az én uramhoz.
\par 55 Monda pedig a leány bátyja és anyja: Maradjon velünk a leány még vagy tíz napig, azután menjen el.
\par 56 A szolga pedig monda nékik: Ne késleljetek meg engem, holott az Úr szerencséssé tette az én útamat; bocsássatok el azért engem, hogy menjek az én uramhoz
\par 57 Mondának akkor: Hívjuk elõ a leányt, és kérdjük meg õt.
\par 58 Szólíták azért Rebekát, és mondának néki: Akarsz-é elmenni e férfiúval? és monda: Elmegyek.
\par 59 Elbocsáták azért Rebekát, az õ húgokat, és az õ dajkáját, és az Ábrahám szolgáját, és az õ embereit.
\par 60 És megáldák Rebekát, és mondák néki: Te mi húgunk! szaporodjál ezerszer való ezerig. És bírja a te magod az õ ellenségeinek kapuját.
\par 61 És felkele Rebeka és az õ szolgálóleányai, és felûlének a tevékre, s követék azt a férfiút. Így vevé a szolga Rebekát, és elméne.
\par 62 Izsák pedig visszajõ vala a Lakhai Rói forrástól; és lakik vala a déli tartományban.
\par 63 És kiméne Izsák este felé elmélkedni a mezõre, és felemelé szemeit és látá, hogy ímé tevék jõnek.
\par 64 Rebeka is felemelé szemeit s meglátá Izsákot, és leszálla a tevérõl.
\par 65 És monda a szolgának: Kicsoda az a férfiú, a ki a mezõn elõnkbe jõ? A szolga pedig monda: Az én uram õ. Akkor fogta a fátyolt és elfedezé magát.
\par 66 Elbeszélé azután a szolga Izsáknak mindazokat a dolgokat, a melyeket cselekedett vala.
\par 67 Izsák pedig bevivé Rebekát Sárának az õ anyjának sátorába. És elvevé Rebekát és lõn néki felesége és szereté õt. S megvigasztalódék Izsák az õ anyja halála után.

\chapter{25}

\par 1 Ábrahám pedig ismét võn magának feleséget, kinek neve Ketúráh vala.
\par 2 És az szûlé néki Zimránt, Joksánt, Médánt, Midiánt, Isbákot és Suakhot.
\par 3 Joksán pedig nemzé Sébát, és Dédánt. Dédánnak pedig fiai valának: Assurim, Letúsim és Leummim.
\par 4 S Midiánnak fiai: Éfah, Éfer, Hánok, Abida és Eldaah: Mind ezek Ketúráhnak fiai.
\par 5 Valamije pedig Ábrahámnak vala, mindazt Izsáknak adta vala.
\par 6 Az ágyasok fiainak pedig, a kik Ábraháméi valának, ada Ábrahám ajándékokat, és elküldé azokat az õ fia mellõl, Izsák mellõl még éltében napkelet felé, napkeleti tartományba.
\par 7 S ezek Ábrahám élete esztendeinek napjai, melyeket élt: száz hetvenöt esztendõ.
\par 8 És kimúlék és meghala Ábrahám, jó vénségben, öregen és betelve az élettel, és takaríttaték az õ népéhez.
\par 9 És eltemeték õt Izsák és Ismáel az õ fiai a Makpelá barlangjában, Efronnak, a Khitteus Czohár fiának mezejében, mely Mamré átellenében van.
\par 10 Abban a mezõben, melyet Ábrahám a Khéth fiaitól vett vala: ott temettetett el Ábrahám és az õ felesége Sára.
\par 11 Lõn pedig Ábrahám halála után, megáldá Isten az õ fiát Izsákot; Izsák pedig lakozék a Lakhai Rói forrásánál.
\par 12 Ezek pedig Ábrahám fiának Ismáelnek nemzetségei, a kit az Égyiptombeli Hágár a Sára szolgálója szûlt vala Ábrahámnak.
\par 13 Ezek az Ismáel fiainak nevei, nevök s nemzetségök szerint: Ismáelnek elsõszülötte Nebájót, azután Kédar, Adbeél és Mibszám.
\par 14 És Misma, Dúmah és Massza.
\par 15 Hadar, Théma, Jetúr, Náfis és Kedmah.
\par 16 Ezek az Ismáel fiai, és ezek azoknak nevei udvaraikban, falvaikban; tizenkét fejedelem az õ nemzetségök szerint.
\par 17 Ezek pedig az Ismáel életének esztendei: száz harminczkét esztendõ. És kimúlék és meghala, és takaríttaték az õ népéhez.
\par 18 Lakoztak pedig Havilától fogva Súrig, a mely Égyiptom átellenében van, a merre Assiriába mennek. Minden atyjafiával szemben esett az õ lakása.
\par 19 Ezek pedig Izsáknak az Ábrahám fiának nemzetségei: Ábrahám nemzé Izsákot.
\par 20 Izsák pedig negyven esztendõs vala, a mikor feleségûl vette Rebekát a Siriából való Bethuélnek leányát, Mésopotámiából, a Siriából való Lábánnak húgát.
\par 21 És könyörge Izsák az Úrnak az õ feleségéért, mivelhogy magtalan vala, és az Úr meghallgatá õt: és teherbe esék Rebeka, az õ felesége.
\par 22 Tusakodnak vala pedig a fiak az õ méhében. Akkor monda: Ha így van, miért vagyok én így? Elméne azért, hogy megkérdezze az Urat.
\par 23 És monda az Úr õnéki: Két nemzetség van a te méhedben, és két nép válik ki a te belsõdbõl, egyik nép a másik népnél erõsebb lesz, és a nagyobbik szolgál a kisebbiknek.
\par 24 És betelének az õ szülésének napjai, és ímé kettõsök valának az õ méhében.
\par 25 És kijöve az elsõ; vereses vala, mindenestõl szõrös, mint egy lazsnak; azért nevezék nevét Ézsaúnak.
\par 26 Azután kijöve az õ atyjafia, kezével Ézsaú sarkába fogódzva; azért nevezék nevét Jákóbnak. Izsák pedig hatvan esztendõs vala, a mikor ezek születének.
\par 27 És felnevekedének a gyermekek, és Ézsaú vadászathoz értõ mezei ember vala; Jákób pedig szelíd ember, sátorban lakozó.
\par 28 Szereti vala azért Izsák Ézsaút, mert szájaíze szerint vala a vad; Rebeka pedig szereti vala Jákóbot.
\par 29 Jákób egyszer valami fõzeléket fõze, és Ézsaú megjövén elfáradva a mezõrõl,
\par 30 Monda Ézsaú Jákóbnak: Engedd, hogy ehessem a veres ételbõl, mert fáradt vagyok. Ezért nevezék nevét Edomnak.
\par 31 Jákób pedig monda: Add el hát nékem azonnal a te elsõszülöttségedet.
\par 32 És monda Ézsaú; Ímé én halni járok, mire való hát nékem az én elsõszülöttségem?
\par 33 És monda Jákób: Esküdjél meg hát nékem azonnal, és megesküvék néki és eladá az õ elsõszülöttségét Jákóbnak.
\par 34 S akkor Jákób ada Ézsaúnak kenyeret, és fõtt lencsét, és evék és ivék, és felkele és elméne. Így veté meg Ézsaú az elsõszülöttséget.

\chapter{26}

\par 1 Lõn pedig éhség az országban, amaz elsõ éhség után, mely Ábrahám idejében vala. Elméne azért Izsák Abimélekhez a Filiszteusok királyához Gérárba.
\par 2 Mert megjelent vala néki az Úr és ezt mondotta vala: Ne menj alá Égyiptomba! lakjál azon a földön, melyet mondándok tenéked.
\par 3 Tartózkodjál ezen a földön, és én veled leszek és megáldalak téged; mert tenéked és a te magodnak adom mind ezeket a földeket, hogy megerõsítsem az esküvést, melylyel megesküdtem Ábrahámnak a te atyádnak.
\par 4 És megsokasítom a te magodat mint az ég csillagait, és a te magodnak adom mind ezeket a földeket: és megáldatnak a te magodban a földnek minden nemzetségei;
\par 5 Mivelhogy hallgata Ábrahám az én szavamra: és megtartotta a megtartandókat, parancsolataimat, rendeléseimet és törvényeimet.
\par 6 Lakozék azért Izsák Gérárban.
\par 7 És mikor annak a helynek lakosai az õ felesége felõl kérdezõsködének, azt mondja vala: az én húgom õ. Mert fél vala azt mondani: én feleségem; gondolván: nehogy megöljenek engem e helynek lakosai Rebekáért, mivelhogy szép ábrázatú õ.
\par 8 És lõn idõ multával, hogy Abimélek a Filiszteusok királya kitekintvén az ablakon, látá Izsákot enyelegni Rebekával az õ feleségével.
\par 9 Kiálta azért Abimélek Izsáknak, és monda: Ímé bizony feleséged õ; hogyan mondhattad tehát: húgom õ?! És monda neki Izsák: Mert azt gondolám, netalán még meg kell halnom miatta.
\par 10 És monda Abimélek: Miért mívelted ezt mi velünk? Kevésbe múlt, hogy feleségeddel nem hált valaki a nép közûl, és bûnt hoztál volna mi reánk.
\par 11 Parancsola azért Abimélek mind az egész népnek, ezt mondván: A ki ezt az embert vagy ennek feleségét illeténdi, bizonynyal meg kell halnia.
\par 12 És vete Izsák azon a földön, és lett néki abban az esztendõben száz annyia, mert megáldá õt az Úr.
\par 13 És gyarapodék az a férfiú, és elébb-elébb megy vala a gyarapodásban, mígnem igen nagygyá lõn.
\par 14 És vala néki apró és öreg barma és sok cselédje, s irigykedének ezért reá a Filiszteusok.
\par 15 És mindazokat a kútakat, melyeket az õ atyjának szolgái Ábrahámnak az õ atyjának idejében ástak vala, behányák a Filiszteusok, és betölték azokat földdel.
\par 16 És monda Abimélek Izsáknak: Menj el közûlünk, mert sokkal hatalmasabbá lettél nálunknál.
\par 17 Elméne azért onnan Izsák, és Gérár völgyében voná fel sátrait, és ott lakék.
\par 18 És ismét megásá Izsák a kútakat, a melyeket ástak vala az õ atyjának Ábrahámnak idejében, de a melyeket Ábrahám holta után behánytak vala a Filiszteusok, és azokkal a nevekkel nevezé azokat, a mely neveket adott vala azoknak az õ atyja.
\par 19 Izsák szolgái pedig ásnak vala a völgyben, és élõ víznek forrására akadának ott.
\par 20 Gérár pásztorai pedig versengének Izsák pásztoraival, mondván: Miénk a víz. Ezért nevezé a kútnak nevét Észeknek, mivelhogy czivakodtak vala õ vele.
\par 21 Más kútat is ásának s azon is versengének, azért annak nevét Szitnának nevezé.
\par 22 És tovább vonula onnan és ása más kútat, a mely miatt nem versengének; azért nevezé nevét Rehobóthnak, és monda: Immár tágas helyet szerzett az Úr minékünk, és szaporodhatunk a földön.
\par 23 Felméne pedig onnan Beérsebába.
\par 24 És megjelenék néki az Úr azon éjszaka, és monda: Én vagyok Ábrahámnak a te atyádnak Istene: Ne félj, mert te veled vagyok, és megáldalak téged, és megsokasítom a te magodat Ábrahámért, az én szolgámért.
\par 25 Oltárt építe azért ott, és segítségûl hívá az Úrnak nevét, s felvoná ott az õ sátorát; Izsák szolgái pedig kútat ásának ottan.
\par 26 Abimélek pedig elméne õ hozzá Gérárból és Akhuzzáth az õ barátja, meg Pikhól az õ hadvezére.
\par 27 És monda nékik Izsák: Miért jöttetek én hozzám, holott gyûlöltök engem s elûztetek magatok közûl?
\par 28 Õk pedig mondák: Látván láttuk, hogy az Úr van te veled, és mondánk: legyen esküvés mi közöttünk, köztünk és te közötted; és kössünk frigyet teveled,
\par 29 Hogy minket gonoszszal nem illetsz, valamint mi sem bántottunk téged, és a mint csak jót cselekedtünk veled, és békességgel bocsátottunk el magunktól. Te már az Úr áldott embere vagy.
\par 30 Akkor vendégséget szerze nékik és evének és ivának.
\par 31 Reggel pedig felkelvén, egymásnak megesküvének, és elbocsátát õket Izsák, és elmenének õ tõle békességgel.
\par 32 Ugyanaz nap eljövének az Izsák szolgái, és hírt hozának néki a kút felõl, melyet ástak vala; és mondának néki: Találtunk vizet.
\par 33 S elnevezé azt Sibáhnak: Azokáért annak a városnak neve Beérseba mind e mai napig.
\par 34 És mikor Ézsaú negyven esztendõs vala, feleségûl vevé Jehudithot, a Khitteus Beéri leányát, és Boszmátot a Khitteus Elon leányát.
\par 35 És õk valának Izsáknak és Rebekának lelke keserûsége.

\chapter{27}

\par 1 És lõn, a mikor megvénhedett vala Izsák, és szemei annyira meghomályosodtak vala, hogy nem látott, szólítá a nagyobbik fiát Ézsaút, és monda néki: Fiam; és ez monda néki: Ímhol vagyok.
\par 2 És monda: Ímé megvénhedtem; nem tudom halálom napját.
\par 3 Most tehát vedd fel kérlek a te fegyvereidet, tegzedet és kézívedet, és menj ki a mezõre, és vadászsz énnékem vadat.
\par 4 És csinálj nékem kedvem szerint való ételt, és hozd el nékem, hogy egyem: hogy megáldjon téged az én lelkem minekelõtte meghalok.
\par 5 Rebeka pedig meghallá, a mit Izsák az õ fiának Ézsaúnak monda; s a mint elméne Ézsaú a mezõre, hogy vadat vadászszon és hozzon:
\par 6 Szóla Rebeka Jákóbnak az õ fiának mondván: Ímé hallám, hogy atyád szóla bátyádnak Ézsaúnak mondván:
\par 7 Hozz nékem vadat, és csinálj nékem kedvem szerint való ételt, hogy egyem; és megáldjalak téged az Úr elõtt, minekelõtte meghalok.
\par 8 Most azért fiam, hallgass az én szavamra, a mit én parancsolok néked.
\par 9 Menj el kérlek, a nyájhoz, és hozz nékem onnan két kecskegödölyét a javából, hogy csináljak azokból a te atyádnak kedve szerint való ételt, a mint õ szereti.
\par 10 Te pedig beviszed atyádnak, hogy egyék, azért, hogy téged áldjon meg, minekelõtte meghal.
\par 11 Jákób pedig monda Rebekának az õ anyjának: Ímé az én bátyám Ézsaú szõrös ember, én pedig sima vagyok.
\par 12 Netalán megtapogat engem az én atyám s olyan leszek elõtte, mint valami csaló, és akkor átkot és nem áldást hozok magamra.
\par 13 És monda néki az õ anyja: Reám szálljon a te átkod fiam, csak hallgass az én szavamra, és menj és hozd el nékem.
\par 14 Elméne azért, és elhozá, és vivé az õ anyjának; és az õ anyja ételt készíte, a mint szereti vala az õ atyja.
\par 15 És vevé Rebeka az õ nagyobbik fiának Ézsaúnak drága ruháit, melyek õ nála otthon valának, és felöltözteté Jákóbot az õ kisebbik fiát.
\par 16 A kecskegödölyék bõrével pedig beborítá az õ kezeit, és nyakának simaságát.
\par 17 És az ételt a melyet készített vala, kenyérrel együtt adá Jákóbnak az õ fiának kezébe.
\par 18 És beméne az õ atyjához és monda: Atyám! és azt monda: Ímhol vagyok. Ki vagy te fiam?
\par 19 Monda Jákób az õ atyjának: Én vagyok Ézsaú a te elsõszülötted, aképen cselekedtem a mint parancsolád, kelj fel, kérlek, ûlj le és egyél vadászatomból, hogy megáldjon engem a te lelked.
\par 20 És monda Izsák az õ fiának: Hogy van az, hogy ily hamar találtál, fiam? És felele: Mert az Úr, a te Istened hozta elõmbe.
\par 21 És monda Izsák Jákóbnak: Jer közelebb, kérlek, hadd tapogassalak meg fiam: hogy vajjon te vagy-é az én fiam Ézsaú vagy nem?
\par 22 Oda méne tehát Jákób Izsákhoz az õ atyjához, a ki megtapogatván õt, monda: A szó Jákób szava, de a kezek Ézsaú kezei.
\par 23 És nem ismeré meg õt, mivelhogy kezei szõrösek valának, mint Ézsaúnak az õ bátyjának kezei; annakokáért megáldá õt.
\par 24 És monda: Te vagy fiam Ézsaú? Felele: Én vagyok.
\par 25 Az pedig monda: Hozd ide, hadd egyem az én fiam vadászatából, hogy megáldjon téged az én lelkem; és oda vivé, és evék; bort is vive néki és ivék.
\par 26 Akkor monda néki Izsák az õ atyja: Jer közelebb fiam, és csókolj meg engem.
\par 27 Oda méne azért, és megcsókolá õt: s megérezvén ruháinak szagát, megáldá õt, és monda: Lám az én fiamnak illatja olyan, mint a mezõnek illatja, a melyet megáldott az Úr.
\par 28 Adjon az Isten tenéked az ég harmatából, és a föld kövérségébõl, és gabonának és bornak bõségét.
\par 29 Népek szolgáljanak néked és nemzetségek hajoljanak meg elõtted; légy úr a te atyádfiain, és hajoljanak meg elõtted a te anyádnak fiai. Átkozott, a ki téged átkoz, és a ki téged áld, legyen áldott.
\par 30 És lõn a mint elvégzé Izsák Jákóbnak megáldását; és épen csakhogy kiment vala Jákób az õ atyjának Izsáknak színe elõl; az õ bátyja Ézsaú is megjöve vadászásából.
\par 31 És készíte õ is ételt, s vivé az õ atyja elé, és mondá az õ atyjának: Keljen fel az én atyám, és egyék az õ fia vadászatából, hogy áldjon meg engem a te lelked.
\par 32 És monda néki az õ atyja Izsák: Kicsoda vagy te? És monda: Én vagyok a te elsõszülött fiad Ézsaú.
\par 33 Akkor Izsák elrémüle igen nagy rémüléssel, és monda: Ki volt hát az, a ki vadat fogott és behozá nékem, és én mindebõl ettem minekelõtte te megjöttél, és megáldottam õt, és áldott is lészen.
\par 34 A mint hallotta vala Ézsaú az õ atyjának beszédét, nagy és igen keserves kiáltással felkiálta, és monda atyjának: Áldj meg engem is atyám.
\par 35 Ez pedig monda: A te öcséd jöve el álnoksággal, és õ vevé el a te áldásodat.
\par 36 Az pedig monda: Nem méltán hívják-é õt Jákobnak? mert immár két ízben csalt meg engemet; elvevé elsõszülöttségemet, most pedig áldásomat vevé el. És monda: Nem tartottál-é nékem is valami áldást?
\par 37 Felele Izsák és monda Ézsaúnak: Ímé uraddá tettem õt, és minden atyjafiát szolgául adtam néki, gabonával is borral is õt láttam el; mit míveljek azért immár veled fiam?
\par 38 Monda Ézsaú az õ atyjának: Avagy csak az az egy áldásod van-é néked atyám? Áldj meg engem, engem is atyám; és felemelé szavát Ézsaú és sír vala.
\par 39 Felele azért Izsák az õ atyja, és monda néki: Imé kövér földön lesz lakásod, és részed lesz az ég harmatjából onnan felûl;
\par 40 És fegyvered után élsz, és öcsédet szolgálod. De lészen, a mikor ellene támadsz, letöröd igáját nyakadról.
\par 41 Gyûlöli vala azért Ézsaú Jákobot az áldásért, a melylyel megáldotta vala az õ atyja, és monda Ézsaú az õ szívében: Közelgetnek az én atyámért való gyásznak napjai, és akkor megölöm az én öcsémet Jákóbot.
\par 42 Mikor pedig hírûl vivék Rebekának, az õ nagyobbik fiának Ézsaúnak beszédit, elkülde és magához hívatá az õ kisebbik fiát Jákóbot, és mondá néki: Ímé Ézsaú a te bátyád azzal fenyeget, hogy megöl téged.
\par 43 Most azért fiam, hallgass az én szavamra, és kelj fel és fuss Lábánhoz az én bátyámhoz Háránba,
\par 44 És maradj nála egy kevés ideig, míg a te bátyád haragja elmúlik;
\par 45 Míg elfordul a te bátyád haragja te rólad, és elfelejtkezik arról a mit rajta elkövettél: akkor elküldök és haza hozatlak téged: miért fosztatnám meg mindkettõtöktõl egy napon?
\par 46 Izsáknak pedig monda Rebeka: Eluntam életemet a Khitteusok leányai miatt. Ha Jákob a Khitteusok leányai közûl vesz feleséget, a milyenek ezek is, ez ország leányai közûl valók, minek nékem az élet?

\chapter{28}

\par 1 Elõhívatá azért Izsák Jákóbot, és megáldá õt, és megparancsolá néki és mondá: Ne végy feleséget a Kananeusok leányai közûl.
\par 2 Kelj fel, menj el Mésopotámiába, Bethuélnek a te anyád atyjának házához, és onnan végy magadnak feleséget, Lábánnak a te anyád bátyjának leányai közûl.
\par 3 A mindenható Isten pedig áldjon meg, szaporítson és sokasítson meg téged, hogy népek sokaságává légy;
\par 4 És adja néked az Ábrahám áldását, tenéked, és a te magodnak te veled egybe; hogy örökség szerint bírjad a földet, melyen jövevény voltál, melyet az Isten adott vala Ábrahámnak.
\par 5 Elbocsátá azért Izsák Jákóbot, hogy menjen Mésopotámiába Lábánhoz a Siriabeli Bethuél fiához, Rebekának, Jákób és Ézsaú anyjának bátyjához.
\par 6 És látá Ézsaú, hogy Izsák megáldotta Jákóbot, és elbocsátotta õt Mésopotámiába, hogy onnan vegyen magának feleséget; és hogy mikor áldja vala, parancsola néki, és monda: Ne végy feleséget a Kananeusok leányai közûl;
\par 7 És hogy Jákób hallgata atyja és anyja szavára, és el is ment Mésopotámiába;
\par 8 És látá Ézsaú, hogy a Kananeusok leányai nem tetszenek Izsáknak az õ atyjának:
\par 9 Elméne Ézsaú Ismáelhez, és feleségûl vevé még az õ feleségeihez Ismáelnek az Ábrahám fiának leányát Mahaláthot, Nebajóthnak húgát.
\par 10 Jákób pedig kiindula Beérsebából, és Hárán felé tartott.
\par 11 És juta egy helyre, holott meghála, mivelhogy a nap lement vala: és võn egyet annak a helynek kövei közûl, és feje alá tevé; és lefeküvék azon a helyen.
\par 12 És álmot láta: Ímé egy lajtorja vala a földön felállítva, melynek teteje az eget éri vala, és ímé az Istennek Angyalai fel- és alájárnak vala azon.
\par 13 És ímé az Úr áll vala azon és szóla: Én vagyok az Úr, Ábrahámnak a te atyádnak Istene, és Izsáknak Istene; ezt a földet a melyen fekszel néked adom és a te magodnak.
\par 14 És a te magod olyan lészen mint a földnek pora, és terjeszkedel nyugotra és keletre, északra és délre, és te  benned és a te magodban áldatnak meg a föld minden nemzetségei.
\par 15 És ímé én veled vagyok, hogy megõrizzelek téged valahova menéndesz, és visszahozzalak e földre; mert el nem hagylak téged, míg be nem teljesítem a mit néked mondtam.
\par 16 Jákób pedig fölébredvén álmából, monda: Bizonyára az Úr van e helyen, és én nem tudtam.
\par 17 Megrémüle annak okáért és monda: Mily rettenetes ez a hely; nem egyéb ez, hanem Istennek háza, és az égnek kapuja.
\par 18 És felkele Jákób reggel, és vevé azt a követ, melyet feje alá tett vala, és oszlopul állítá fel azt, és  olajat önte annak tetejére;
\par 19 És nevezé annak a helynek nevét Béthelnek, az elõtt pedig Lúz vala annak a városnak neve.
\par 20 És fogadást tõn Jákób, mondá: Ha az Isten velem leénd, és megõriz engem ezen az úton, a melyen most járok, és ha ételûl kenyeret s öltözetûl ruhát adánd nékem;
\par 21 És békességgel térek vissza az én atyámnak házához: akkor az Úr leénd az én Istenem;
\par 22 És ez a kõ, a melyet oszlopul állítottam fel, Isten háza lészen, és valamit adándasz nékem, annak tizedét néked adom.

\chapter{29}

\par 1 Jákób azután lábára kelvén, elméne a napkeletre lakók földére.
\par 2 És látá, hogy ímé egy kút van a mezõben, és hogy ott három falka juh hever vala. Mert abból a kútból itatják vala a nyájakat; de a kútnak száján nagy kõ vala:
\par 3 Mikor pedig ott valamennyi nyáj összeverõdik, elgördítik a követ a kút szájáról és megitatják a juhokat s ismét helyére teszik a követ, a kút szájára.
\par 4 És monda nékik Jákób: Honnan valók vagytok atyámfiai? És mondának: Háránból valók vagyunk.
\par 5 És monda nékik: Ismeritek-é Lábánt, a Nákhor fiát? s azok felelének: Ismerjük.
\par 6 Azután monda nékik: Egészségben van-é? s azok mondának: Egészségben van, és az õ leánya Rákhel ímhol jõ a juhokkal.
\par 7 És monda Jákób: Ímé még nagy fenn van a nap, nincs ideje hogy betereljék a marhát: itassátok meg a juhokat, és menjetek, legeltessetek.
\par 8 Azok pedig felelének: Nem tehetjük míg valamennyi nyáj össze nem verõdik, és el nem gördítik a követ a kút szájáról, hogy megitathassuk a juhokat.
\par 9 Még beszélget vala velõk, mikor megérkezék Rákhel az õ atyja juhaival, melyeket legeltet vala.
\par 10 S lõn, a mint meglátá Jákób Rákhelt, Lábánnak az õ anyja bátyjának leányát, és Lábánnak az õ anyja bátyjának juhait, odalépett Jákób és elgördíté a követ a kút szájáról, és megitatá Lábánnak az õ anyja bátyjának juhait.
\par 11 És megcsókolá Jákób Rákhelt, és nagy felszóval síra.
\par 12 S elbeszélé Jákób Rákhelnek, hogy õ az õ atyjának rokona és hogy Rebekának fia. Ez pedig elfuta, és megmondá az õ atyjának.
\par 13 És lõn mikor Lábán Jákóbnak, az õ húga fiának hírét hallá, eleibe futa, megölelé és megcsókolá õt, és bevivé az õ házába, és az mindeneket elbeszéle Lábánnak.
\par 14 És monda néki Lábán: Bizony én csontom és testem vagy te! És nála lakék egy hónapig.
\par 15 És monda Lábán Jákóbnak: Avagy ingyen szolgálj-é engem azért, hogy atyámfia vagy? Mondd meg nékem, mi legyen a béred?
\par 16 Vala pedig Lábánnak két leánya: a nagyobbiknak neve Lea, a kisebbiknek neve Rákhel.
\par 17 Leának pedig gyenge szemei valának, de Rákhel szép termetû és szép tekintetû vala.
\par 18 Megszereti vala azért Jákób Rákhelt, és monda: Szolgállak téged hét esztendeig Rákhelért, a te kisebbik leányodért.
\par 19 És monda Lábán: Jobb néked adnom õt, hogysem másnak adjam õt, maradj én nálam.
\par 20 Szolgála tehát Jákób Rákhelért hét esztendeig, s csak néhány napnak tetszék az neki, annyira szereti vala õt.
\par 21 És monda Jákób Lábánnak: Add meg nékem az én feleségemet: mert az én idõm kitelt, hadd menjek be hozzá.
\par 22 És begyûjté Lábán annak a helynek minden népét, és szerze lakodalmat.
\par 23 Estve pedig vevé az õ leányát Leát, és bevivé hozzá, a ki beméne õ hozzá.
\par 24 És Lábán az õ szolgálóját Zilpát, szolgálóul adá az õ leányának Leának.
\par 25 És reggelre kelvén: Ímé ez Lea! Monda azért Lábánnak: Mit cselekedtél én velem? Avagy nem Rákhelért szolgáltalak-é én téged? Miért csaltál meg engem?
\par 26 Lábán pedig monda: Nem szokás nálunk, hogy a kisebbiket oda adják a nagyobbik elõtt.
\par 27 Töltsd ki ennek hetét, azután amazt is néked adjuk a szolgálatért, melylyel majd szolgálsz nálam még más hét esztendeig.
\par 28 Jákób tehát aképen cselekedék, kitölté azt a hetet; ez pedig néki adá Rákhelt, az õ leányát feleségûl.
\par 29 És adá Lábán az õ leányának Rákhelnek, az õ szolgálóját Bilhát, hogy néki szolgálója legyen.
\par 30 És beméne Rákhelhez is, és inkább szereté Rákhelt, hogysem Leát és szolgála õ nála még más hét esztendeig.
\par 31 És meglátá az Úr Lea megvetett voltát, és megnyitá annak méhét. Rákhel pedig magtalan vala.
\par 32 Fogada azért Lea az õ méhében és szûle fiat, és nevezé nevét Rúbennek, mert azt mondja vala: Meglátta az Úr az én nyomorúságomat; most már szeretni fog engem az én férjem.
\par 33 Azután ismét teherbe esék és szûle fiat, és monda: Mivelhogy meghallotta az Úr megvetett voltomat, azért adta nékem ezt is; és nevezé nevét Simeonnak.
\par 34 És megint teherbe esék és szûle fiat, és monda: Most már ragaszkodni fog hozzám az én férjem, mert három fiat szûltem néki; azért nevezé nevét Lévinek.
\par 35 És ismét teherbe esék, és fiat szûle és mondá: Most már hálákat adok az Úrnak; azért nevezé nevét Júdának, és megszûnék a szûléstõl.

\chapter{30}

\par 1 És látá Rákhel, hogy õ nem szûle Jákóbnak, irigykedni kezde Rákhel az õ nénjére, és monda Jákóbnak: Adj nékem gyermekeket, mert ha nem, meghalok.
\par 2 Felgerjede azért Jákób haragja Rákhel ellen, és monda: Avagy Isten vagyok-é én, ki megtagadta tõled a méhnek gyümölcsét.
\par 3 És monda ez: Ímhol az én szolgálóm Bilha, menj be hozzá, hogy szûljön az én térdeimen, és én is megépüljek õ általa.
\par 4 Adá tehát néki az õ szolgálóját Bilhát feleségûl, és beméne ahhoz Jákób.
\par 5 És teherbe esék Bilha és szûle Jákóbnak fiat.
\par 6 És monda Rákhel: Ítélt felõlem az Isten, és meg is hallgatta szavamat, és adott énnékem fiat: azért nevezé nevét Dánnak.
\par 7 Ismét fogada az õ méhében, és szûle Bilha, a Rákhel szolgálója más fiat is Jákóbnak.
\par 8 És monda Rákhel: Nagy tusakodással tusakodtam az én nénémmel, és gyõztem; azért nevezé nevét Nafthalinak.
\par 9 Látván pedig Lea hogy õ megszûnt a szûléstõl, vevé az õ szolgálóját Zilpát, és adá azt Jákóbnak feleségûl.
\par 10 És szûle Zilpa, Lea szolgálója, fiat Jákóbnak.
\par 11 És monda Lea: Szerencsére! és nevezé nevét Gádnak!
\par 12 És szûle Zilpa, Lea szolgálója, más fiat is Jákóbnak.
\par 13 És monda Lea: Oh én boldogságom! bizony boldognak mondanak engem az asszonyok: és nevezé nevét Ásernek.
\par 14 És kiméne Rúben búzaaratáskor, és talála a mezõn mandragóra-bogyókat s vivé azokat az õ anyjának, Leának. És monda Rákhel Leának: Adj nékem kérlek a fiad mandragóra-bogyóiból.
\par 15 Az pedig monda néki: Talán keveselled, hogy elvetted tõlem az én férjemet, s a fiam mandragóra-bogyóit is elvennéd tõlem? És monda Rákhel: Háljon veled hát az éjjel a te fiad mandragóra-bogyóiért.
\par 16 Mikor Jákób este a mezõrõl jöve, eleibe méne Lea, és monda: Én hozzám jõjj be, mert megvettelek a fiam mandragóra-bogyóiért; és nála hála azon éjszaka.
\par 17 És meghallgatá Isten Leát, mert fogada az õ méhében és szûlé Jákobnak ötödik fiat.
\par 18 És monda Lea: Megadta az Isten jutalmamat, a miért szolgálómat férjemnek adtam; azért nevezé nevét Izsakhárnak.
\par 19 És ismét fogada az õ méhében Lea, és szûle hatodik fiat Jákóbnak.
\par 20 És monda Lea: Megajándékozott az Isten engem jó ajándékkal; most már velem lakik az én férjem, mert hat fiat szûltem néki, és neveze nevét Zebulonnak.
\par 21 Annakutána szûle leányt és nevezé nevét Dínának.
\par 22 Megemlékezék pedig az Isten Rákhelrõl; és meghallgatá õt az Isten és megnyitá az õ méhét.
\par 23 És fogada méhében, és szûle fiat, s monda: Elvevé Isten az én gyalázatomat.
\par 24 És nevezé nevét Józsefnek, mondván: Adjon ehhez az Úr nékem más fiat is.
\par 25 És lõn, a mint szûlte vala Rákhel Józsefet, monda Jákób Lábánnak: Bocsáss el engemet, hadd menjek el az én helyembe, az én hazámba.
\par 26 Add meg nékem az én feleségeimet és magzatimat, a kikért szolgáltalak téged, hadd menjek el, mert te tudod az én szolgálatomat, a melylyel szolgáltalak téged.
\par 27 És monda néki Lábán: Vajha kedvet találtam volna szemeid elõtt! Úgy sejtem, hogy te éretted áldott meg engem az Úr.
\par 28 És monda: Szabj bért magadnak és én megadom.
\par 29 Ez pedig monda: Te tudod mimódon szolgáltalak téged, és hogy mivé lett nálam a te jószágod.
\par 30 Mert a mi kevesed vala én elõttem, sokra szaporodott, és megáldott az Úr téged az én lábam nyomán. Immár mikor tehetek valamit a magam házáért is?
\par 31 És monda Lábán: Mit adjak néked? Felele Jákób: Ne adj nékem semmit; juhaidat ismét legeltetem és õrizem, ha nekem ezt a dolgot megteszed:
\par 32 Nyájaidat ma mind végig járom, minden pettyegetett és tarka bárányt kiszaggatok közûlök, és minden fekete bárányt a juhok közûl, s a tarkát és pettyegetettet a kecskék közûl, s legyen ez az én bérem.
\par 33 S a mikor majd bérem iránt eljövéndesz, mi elõtted lesz, becsületességemrõl ez felel: a mi nem pettyegetett vagy tarka a kecskék, s nem fekete a juhok közt, az mind lopott jószág nálam.
\par 34 És monda Lábán: Ám legyen: Vajha a te beszéded szerint lenne.
\par 35 Külön választá azért azon a napon a pettyegetett és tarka kosokat, és minden csíkos lábú és tarka kecskét, mint a melyikben valami fehérség vala, és minden feketét a juhok közûl, és adá az õ fiainak keze alá.
\par 36 És három napi járó földet vete maga közé és Jákób közé; Jákób pedig legelteti vala Lábán egyéb juhait.
\par 37 És võn Jákób zöld nyár-, mogyoró- és gesztenye-vesszõket, és meghántá azokat fehéresen csíkosra, hogy látható legyen a vesszõk fehére.
\par 38 És a vesszõket, melyeket meghántott vala, felállítá a csatornákba, az itató válúkba, melyekre a juhok inni járnak vala, szembe a juhokkal, hogy foganjanak, mikor inni jönnek.
\par 39 És a juhok a vesszõk elõtt foganának és ellenek vala csíkos lábúakat, pettyegetetteket és tarkákat.
\par 40 Azután külön szakasztá Jákób ezeket a bárányokat, és a Lábán nyáját arczczal fordítja vala a csíkos lábú és fekete bárányokra; így szerze magának külön falkákat, melyeket nem ereszte a Lábán juhai közé.
\par 41 És lõn, hogy mikor a nyáj java részének vala párzási ideje, akkor Jákób a vesszõket oda raká a válúkba a juhok eleibe, hogy a vesszõket látva foganjanak.
\par 42 De mikor satnya vala a nyáj, nem rakja vala oda s ily módon Lábánéi lõnek a satnyák, a java pedig Jákóbé.
\par 43 És felette igen meggazdagodék a férfiú; és vala néki sok juha, szolgálója, szolgája, tevéje és szamara.

\chapter{31}

\par 1 És meghallá a Lábán fiainak beszédét, kik ezt mondják vala: Valamije volt atyánknak, mind elvette Jákób; és atyánkéból szerezte mind e gazdagságot.
\par 2 És látá Jákób a Lábán orczáját, hogy ímé nem olyan õ hozzá mint annakelõtte.
\par 3 Monda pedig az Úr Jákóbnak: Térj meg atyáid földére, a te rokonságod közé, és veled lészek.
\par 4 Elkülde tehát Jákób, és kihívatá magához Rákhelt és Leát a mezõre az õ nyájához.
\par 5 És monda nékik: Látom atyátok orczáját, hogy nem olyan hozzám, mint ennekelõtte; de az én atyám Istene velem volt.
\par 6 Ti pedig tudjátok, hogy teljes erõm szerint szolgáltam atyátokat.
\par 7 De atyátok engem megcsalt, s tízszer is megváltoztatta béremet; mindazáltal az Isten nem engedte, hogy nékem kárt tehessen.
\par 8 Mikor azt mondotta: A pettyegetettek legyenek a te béred, a juhok mind pettyegetetteket ellenek vala. Ha azt mondotta: A csíkos lábúak legyenek a te béred, a juhok mind a csíkos lábúakat ellenek vala.
\par 9 Így vette el Isten atyátok jószágát és nékem adta.
\par 10 Mert lõn a juhok foganásának idejekor, szemeimet felemelém, és látom vala álomban, hogy ímé a juhokat hágó kosok csíkos lábúak, petyegetettek és tarkák.
\par 11 Akkor monda nékem az Isten Angyala álmomban: Jákób. És felelék: Ímhol vagyok.
\par 12 És õ monda: Emeld fel szemeidet és lásd, hogy a mely kosok a juhokat hágják, azok mind csíkos lábúak, pettyegetettek és tarkák. Mert mindazt láttam, a mit veled Lábán cselekszik vala.
\par 13 Én vagyok ama Béthelnek Istene, a hol emlékoszlopot kentél fel, és a hol fogadást tettél nékem. Most kelj fel, menj ki e földrõl, és térj vissza szülõföldedre.
\par 14 És felele Rákhel és Lea, és mondának néki: Vajjon vagyon-é még nékünk valami részünk és örökségünk a mi atyánk házában?
\par 15 Avagy nem úgy tartott-é minket mint idegeneket? midõn minket eladott, és értékünket is teljesen megemésztette.
\par 16 Mert mind ez a gazdagság, melyet Isten vett el a mi atyánktól, miénk és a mi fiainké. Most azért valamit néked az Isten mondott,  azt cselekedjed.
\par 17 Felkele tehát Jákób, és feltevé gyermekeit és feleségeit a tevékre;
\par 18 És elvivé minden nyáját, és minden keresményét, melyet keresett vala; minden jószágát, melyet szerzett vala Mésopotámiában, hogy elmenjen az õ atyjához Izsákhoz Kanaán földére.
\par 19 Lábán pedig elment vala juhait nyírni; azonközben ellopá Rákhel a házi bálványokat, melyek atyjánál valának.
\par 20 Jákób pedig meglopá a Siriabeli Lábánnak szívét, mivelhogy nem adá tudtára, hogy szökni akar.
\par 21 Megszökék tehát mindenestõl, és felkelvén, általméne a folyóvízen, és Gileád hegy felé tarta.
\par 22 És mikor harmad napra megmondák Lábánnak, hogy Jákób elszökött;
\par 23 Maga mellé vévén az õ rokonait, hét napi járó földig ûzé õket; és eléré a Gileád hegyén.
\par 24 Isten pedig megjelenék a Siriabeli Lábánnak éjjeli álomban, és monda néki: Vigyázz magadra, Jákóbnak se jót, se rosszat ne szólj.
\par 25 Mikor eléré Lábán Jákóbot, s Jákób a hegyén voná fel sátorát; Lábán is a Gileád hegyen voná fel az õ rokonaival egybe.
\par 26 És monda Lábán Jákóbnak: Mit cselekedtél, hogy megloptad szívemet, és leányaimat fegyverrel nyert foglyokként vitted el?
\par 27 Miért futottál el titkon, s loptál meg engem? miért nem jelentetted nékem, hogy elbocsátottalak volna örömmel, énekszóval, dob- és hegedûszóval?
\par 28 És nem engedted meg, hogy megcsókoljam fiaimat és leányaimat. Ez egyszer bolondul cselekedtél.
\par 29 Volna erõm hozzá, hogy rosszat tegyek veletek, de a ti atyátok Istene tegnap éjszaka megszólíta engem, ezt mondván: Vigyázz magadra, Jákóbnak se jót, se rosszat ne szólj.
\par 30 Hogyha pedig immár el akartál menni, mivelhogy nagy kívánsággal kívánkoztál atyád házához: miért loptad el az én isteneimet?
\par 31 Felelvén pedig Jákób, monda Lábánnak: Mert féltem, mert gondolom vala, hogy talán elveszed a te leányaidat én tõlem erõvel.
\par 32 A kinél pedig megtalálod a te isteneidet, ne éljen az. Atyánkfiai elõtt vizsgáld meg, mid van nálam, és vidd el. Mert nem tudja vala Jákób, hogy Rákhel lopta el azokat.
\par 33 Beméne tehát Lábán Jákób sátorába, és Lea sátorába, és a két szolgáló sátorába, és nem találá meg; akkor kiméne Lea sátorából, és méne a Rákhel sátorába.
\par 34 Rákhel pedig vette vala a házi bálványokat, és tette vala azokat egy tevének a nyergébe, és rájok ûle; Lábán pedig felhányá az egész sátort, és nem találta vala meg azokat.
\par 35 Akkor monda az õ atyjának: Ne haragudjék az én uram, hogy fel nem kelhetek elõtted, mert asszonyok baja van rajtam. Keresé tehát, de nem találá a házi bálványokat.
\par 36 Jákób pedig haragra gerjede s feddõdék Lábánnal. Megszólala Jákób és monda Lábánnak: Mi a vétkem, és mi a bûnöm, hogy üldözõbe vettél?
\par 37 Bezzeg minden holmimat felhánytad, mit találtál a magad házi holmija közûl valót? add elõ itt az én rokonaim és a te rokonaid elõtt, hogy tegyenek ítéletet kettõnk között.
\par 38 Immár húsz esztendeje vagyok nálad, juhaid és kecskéid nem vetéltek el, és nyájad kosait nem ettem meg.
\par 39 A mit a vad megszaggatott, nem vittem hozzád, én fizettem meg azt; tõlem követelted a nappal lopottat, mint az éjjel lopottat is.
\par 40 Úgy voltam hogy nappal a hõség emésztett, éjjel pedig a hideg; és az álom távol maradt szemeimtõl.
\par 41 Immár húsz esztendeje hogy házadnál vagyok; tizennégy esztendeig szolgáltalak két leányodért, és hat esztendeig juhaidért; te pedig béremet tízszer is megváltoztattad.
\par 42 Ha az én atyám Istene, Ábrahám Istene, és az Izsák félelme velem nem volt volna, bizony most üresen bocsátanál el engem, de megtekintette Isten az én nyomorúságomat és kezeim munkáját, és megfeddett téged tegnap éjjel.
\par 43 Felele pedig Lábán és monda Jákóbnak: A leányok én leányaim és a fiak én fiaim, és a nyáj az én nyájam, s valamit látsz mind az enyim, de mit tehetek ma ezeknek az én leányaimnak, vagy az õ magzatjaiknak, a kiket szûltek?
\par 44 Most tehát jer, kössünk szövetséget, én meg te, hogy az légyen bizonyságul közöttem és közötted.
\par 45 És võn Jákób egy követ, és felemelé azt emlékoszlopul.
\par 46 És monda Jákób az õ atyjafiainak: Szedjetek köveket! És gyûjtének köveket, és csinálának rakást; és evének ott a rakáson.
\par 47 És nevezé azt Lábán Jegár-Sahaduthának, Jákób pedig nevezé Gálhédnek.
\par 48 És mondja vala Lábán: E rakás bizonyság ma, közöttem és közötted, azért nevezék Gálhédnek.
\par 49 És Miczpának, mivelhogy mondá: Az Úr legyen vigyázó közöttem és te közötted, a mikor egymástól elválunk.
\par 50 Ha az én leányaimat nyomorgatánd, és ha az én leányaimon kivûl több feleséget veéndesz, senki sincs ugyan velünk; de meglásd: Isten a bizonyság én közöttem és te közötted.
\par 51 És monda Lábán Jákóbnak: Ímé e rakás kõ és ímé ez emlékoszlop, a melyet raktam én közöttem és te közötted,
\par 52 Bizonyság legyen e rakás kõ, és bizonyság ez az emlékoszlop, hogy sem én nem megyek el e rakás kõ mellett te hozzád, sem te nem jössz át én hozzám e rakás kõ, és ez emlékoszlop mellett gonosz végre.
\par 53 Az Ábrahám Istene, és a Nákhor Istene, és az õ atyjok Istene tegyenek ítéletet közöttünk: És megesküvék Jákób az õ atyjának Izsáknak félelmére.
\par 54 Akkor Jákób áldozatot öle ott a hegyen, és vendégségbe hívá az õ rokonait. És vendégeskedtek vala, s meghálának a hegyen.
\par 55 Reggel pedig felkele Lábán és megcsókolá fiait és leányait és megáldá õket. Azután elméne Lábán, és visszatére az õ helyére.

\chapter{32}

\par 1 Jákób tovább méne az õ útján, és szembe jövének vele az Isten Angyalai.
\par 2 És monda Jákób mikor azokat látja vala: Isten tábora ez; és nevezé annak a helynek nevét Mahanáimnak.
\par 3 Azután külde Jákób követeket maga elõtt Ézsaúhoz az õ bátyjához, Széir földébe, Edóm mezõségébe,
\par 4 És parancsola azoknak mondván: Így szóljatok az én uramnak Ézsaúnak: Ezt mondja a te szolgád Jákób: Lábánnál tartózkodtam és idõztem mind ekkorig.
\par 5 Vannak pedig nékem ökreim és szamaraim, juhaim, szolgáim és szolgálóim, azért híradásul követséget küldök az én uramhoz, hogy kedvet találjak szemeid elõtt.
\par 6 És megtérének Jákóbhoz a követek, mondván: Elmentünk vala a te atyádfiához Ézsaúhoz, és már jön is elõdbe, és négyszáz férfi van vele.
\par 7 Igen megíjede Jákób és féltében a népet, mely vele vala, a juhokat, a barmokat és a tevéket két seregre osztá.
\par 8 És monda: Ha eljön Ézsaú az egyik seregre, és azt levágja, a hátramaradt sereg megszabadul.
\par 9 És monda Jákób: Óh én atyámnak Ábrahámnak Istene, és én atyámnak Izsáknak Istene, Jehova! ki azt mondád nékem: Térj vissza hazádba, a te rokonságod közé, s jól tészek veled:
\par 10 Kisebb vagyok minden te jótéteményednél és minden te hûségednél, a melyeket a te szolgáddal cselekedtél; mert csak pálczámmal mentem vala által ezen a Jordánon, most pedig két sereggé lettem.
\par 11 Szabadíts meg, kérlek, engem az én bátyám kezébõl, Ézsaú kezébõl; mert félek õ tõle, hogy rajtam üt és levág engem, az anyát a fiakkal egybe.
\par 12 Te pedig azt mondottad: Jól tévén jól tészek te veled, és a te magodat olyanná tészem mint a tenger fövénye, mely meg nem számláltathatik sokasága miatt.
\par 13 És ott hála azon éjjel: és választa abból, a mi kezénél vala, ajándékot Ézsaúnak az õ bátyjának:
\par 14 Kétszáz kecskét és húsz bakot; kétszáz juhot, és húsz kost;
\par 15 Harmincz szoptatós tevét s azok fiait; negyven tehenet, és tíz tulkot: húsz nõstény szamarat, és tíz szamár vemhet.
\par 16 És szolgái kezébe adá, minden nyájat külön-külön, és monda az õ szolgáinak: Menjetek el én elõttem, és közt hagyjatok nyáj és nyáj között.
\par 17 És parancsola az elsõnek, mondván: Ha az én bátyám Ézsaú elõtalál és megkérdez téged, mondván: Ki embere vagy? Hová mégy? És kiéi ezek elõtted?
\par 18 Akkor azt mondjad: Szolgádé Jákóbé; ajándék az, a melyet küld az én uramnak Ézsaúnak, és ímé õ maga is jön utánunk.
\par 19 Ugyanazt parancsolá a másiknak, a harmadiknak, és mindazoknak, kik a nyájak után mennek vala, mondván: Ilyen szóval szóljatok Ézsaúnak, mikor vele találkoztok.
\par 20 Ezt is mondjátok: Ímé Jákób a te szolgád utánunk jõ; mert így gondolkodik vala: Megengesztelem õt az ajándékkal, mely elõttem megy, és azután leszek szembe vele, talán kedves lesz személyem elõtte.
\par 21 Elõlméne tehát az ajándék; õ pedig azon éjjel a seregnél hála.
\par 22 Felkele pedig õ azon éjszaka és vevé két feleségét, két szolgálóját és tizenegy gyermekét, és általméne a Jabbók révén.
\par 23 Vevé hát azokat és átköltözteté a vízen, azután átköltözteté mindenét valamije vala.
\par 24 Jákób pedig egyedûl marada és tusakodik vala õ vele egy férfiú, egész a hajnal feljöveteléig.
\par 25 Aki mikor látá, hogy nem vehet rajta erõt, megilleté csípõjének forgócsontját, és kiméne helyébõl Jákób csípõjének forgócsontja a vele való tusakodás közben.
\par 26 És monda: Bocsáss el engem, mert feljött a hajnal. És monda Jákób: Nem bocsátlak el téged, míg meg nem áldasz engemet.
\par 27 És monda néki: Mi a te neved? És õ monda: Jákób.
\par 28 Amaz pedig monda: Nem Jákóbnak mondatik ezután a te neved, hanem Izráelnek; mert küzdöttél Istennel és emberekkel, és gyõztél.
\par 29 És megkérdé Jákób, és mondá: Mondd meg, kérlek, a te nevedet. Az pedig monda: Ugyan miért kérded az én nevemet? És megáldá õt ott.
\par 30 Nevezé azért Jákób annak a helynek nevét Peniélnek: mert látám az Istent színrõl színre, és megszabadult az én lelkem.
\par 31 És a nap felkél vala rajta, amint elméne Peniél mellett, õ pedig sántít vala csípõjére.
\par 32 Azért nem eszik Izráel fiai a csípõ forgócsontjának ina húsát mind e mai napig, mivelhogy illetve vala Jákób csípõje forgócsontjának inahúsa.

\chapter{33}

\par 1 Jákób pedig felemelé szemeit és látá, hogy ímé Ézsaú jõ vala, és négyszáz férfiú õ vele; megosztá azért a gyermekeket Lea mellé, Rákhel mellé, és két szolgálója mellé.
\par 2 És elõreállítá a szolgálókat és azok gyermekeit, ezek után Leát és az õ gyermekeit, Rákhelt pedig és Józsefet leghátul.
\par 3 Maga pedig elõttök megy vala, és hétszer hajtá meg magát a földig, a míg bátyjához juta.
\par 4 Ézsaú pedig elibe futamodék és megölelé õt, nyakába borúla, s megcsókolá õt, és sírának.
\par 5 És felemelé szemeit s látá az asszonyokat és a gyermekeket, és monda: Kicsodák ezek teveled? Õ pedig monda: A gyermekek, kikkel Isten megajándékozta a te szolgádat.
\par 6 És közelítenék a szolgálók, õk és gyermekeik és meghajták magokat.
\par 7 Elérkezék Lea is az õ gyermekeivel, és meghajták magokat; utoljára érkezék József és Rákhel, és õk is meghajták magokat.
\par 8 És monda Ézsaú: Mire való ez az egész sereg, melyet elõltalálék? És felele: Hogy kedvet találjak az én uram szemei elõtt.
\par 9 És monda Ézsaú: Van nekem elég, jó öcsém, legyen tiéd, a mi a tiéd.
\par 10 Monda pedig Jákób: Ne úgy, kérlek, hanem ha kedvet találtam szemeid elõtt, fogadd el ajándékomat az én kezembõl; mert a te orczádat úgy néztem, mintha az Isten orczáját látnám, és te kegyesen fogadál engem.
\par 11 Vedd el kérlek az én ajándékomat, melyet hoztam néked, mivelhogy az Isten kegyelmesen cselekedett én velem, és mindenem van nékem. És unszolá õt, és elvevé.
\par 12 És monda: Induljunk, menjünk el, és én elõtted megyek.
\par 13 Felele néki Jákób: Az én uram jól tudja, hogy e gyermekek gyengék, és hogy szoptatós juhokkal és barmokkal vagyok körûl, a melyeket ha csak egy napig zaklatnak is, a nyájak mind elhullanak.
\par 14 Menjen el azért az én uram az õ szolgája elõtt, én is elballagok lassan, a jószág lépése szerint, a mely elõttem van, és a gyermekek lépése szerint, míg eljutok az én uramhoz Széirbe.
\par 15 És monda Ézsaú: Hadd rendeljek melléd néhányat a nép közûl, mely velem van. S ez monda: Minek az? csak kedvet találjak az én uram szemei elõtt.
\par 16 Visszatére tehát Ézsaú még az nap az õ útján Széir felé.
\par 17 Jákób pedig méne Szukkóthba és építe magának házat, barmainak pedig hajlékokat csinála, s azért nevezé a hely nevét Szukkóthnak.
\par 18 Annakutána minden bántás nélkül méne Jákób Mésopotámiából jövet Sekhem városába, mely vala a Kanaán földén, és letelepedék a város elõtt.
\par 19 És megvevé a mezõnek azt a részét, a hol sátorát felvonta vala, Khámornak a Sekhem atyjának fiaitól száz pénzen.
\par 20 És oltárt állta ott, és nevezé azt ily névvel: Isten, Izráel Istene.

\chapter{34}

\par 1 Kiméne pedig Dína, Leának leánya, kit Jákóbnak szûlt vala, hogy meglátogassa annak a földnek leányait.
\par 2 És meglátá õt Sekhem, a Khivveus Khámornak, az ország fejedelmének fia, és elragadá õt, és vele hála és erõszakot tesz vala rajta.
\par 3 És ragaszkodék az õ lelke Dínához a Jákób leányához, és megszereté a leányt és szívéhez szól vala a leánynak.
\par 4 Szóla pedig Sekhem Khámornak az õ atyjának, mondván: Vedd nékem feleségûl ezt a leányt.
\par 5 És meghallá Jákób, hogy megszeplõsítette Dínát, az õ leányát, fiai pedig a mezõn valának a barommal, azért veszteg marada Jákób, míg azok megjövének.
\par 6 És kiméne Khámor, Sekhem atyja Jákóbhoz, hogy szóljon vele.
\par 7 Mikor Jákób fiai megjövének a mezõrõl és meghallák a dolgot, elkeseredének és nagyon megharaguvának azok az emberek, azért hogy ocsmányságot cselekedett Izráelben, Jákób leányával hálván, a minek nem kellett volna történni.
\par 8 És szóla nékik Khámor, mondván: Az én fiam Sekhem, lelkébõl szereti a ti leányotokat, kérlek, adjátok azt néki feleségûl.
\par 9 És szerezzetek velünk sógorságot: a ti leányaitokat adjátok nékünk, és a mi leányainkat vegyétek magatoknak,
\par 10 És lakjatok velünk; a föld elõttetek van, lakjátok, s kereskedjetek rajta és bírjátok azt.
\par 11 Sekhem is monda a Dína atyjának és az õ bátyjainak: Hadd találjak kedvet elõttetek, és valamit mondotok nékem, megadom.
\par 12 Akármily nagy jegyadományt és ajándékot kivántok, megadom a mint mondjátok nékem, csak adjátok nékem a leányt feleségûl.
\par 13 A Jákób fiai pedig álnokul felelének Sekhemnek és Khámornak az õ atyjának, és szólának, mivelhogy megszeplõsítette Dínát az õ húgokat,
\par 14 És mondának nékik: Nem mívelhetjük e dolgot, hogy a mi húgunkat körûlmetélkedetlen férfiúnak adjuk; mert ez nékünk gyalázat volna.
\par 15 Veletek csak úgy egyeztünk, ha hasonlókká lesztek hozzánk, hogy minden férfiú körûlmetélkedjék ti köztetek.
\par 16 Így a mi leányainkat néktek adjuk, és a ti leányaitokat magunkhoz vesszük, veletek lakozunk, és egy néppé leszünk;
\par 17 Hogyha pedig nem hallgattok reánk, hogy körûlmetélkedjetek: felveszszük a mi leányunkat és elmegyünk.
\par 18 És tetszék azoknak beszéde Khámornak, és Sekhemnek a Khámor fiának.
\par 19 Nem is halasztá az ifjú a dolog véghezvitelét, mivelhogy igen szereti vala a Jákób leányát; néki pedig atyja házanépe között mindenkinél nagyobb becsûlete vala.
\par 20 Elméne azért Khámor és Sekhem az õ fia az õ városuk kapujába; és szólának az õ városuk férfiaival, mondván:
\par 21 Ezek az emberek békességesek velünk, hadd lakjanak e földön, és kereskedjenek benne, mert ímé e föld elég tágas nékik; az õ leányaikat vegyük magunknak feleségûl, és a mi leányainkat adjuk nékik.
\par 22 De csak úgy egyeznek bele e férfiak, hogy velünk lakjanak és egy néppé legyenek velünk, ha minden férfiú körûlmetélkedik közöttünk, a miképen õk is körûl vannak metélkedve.
\par 23 Nyájaik, jószáguk, és minden barmuk nemde nem miéink lesznek-é? csak egyezzünk meg velök, akkor velünk laknak.
\par 24 És engedének Khámornak, és Sekhemnek az õ fiának mindenek, a kik az õ városa kapuján kijárnak vala, és körûlmetélkedék minden férfiú, a ki az õ városa kapuján kijár vala.
\par 25 És lõn harmadnapon, mikor ezek a seb fájdalmában valának, a Jákób két fia, Simeon és Lévi, Dínának bátyjai, fegyvert ragadának s bátran a városra ütének és minden férfit megölének.
\par 26 Khámort, és az õ fiát Sekhemet fegyver élére hányák, és elvivék Dínát a Sekhem házából és kimenének.
\par 27 Jákób fiai a megölteknek esének és feldúlák a várost, mivelhogy megszeplõsítették vala az õ húgokat.
\par 28 Azok juhait, barmait, szamarait, és valami a városban, és a mezõn vala, elvivék.
\par 29 És minden gazdagságukat, minden gyermekeiket és feleségeiket fogva vivék és elrablák, és mindent a mi a házban vala.
\par 30 És monda Jákób Simeonnak és Lévinek: Megháborítottatok engem, és utálatossá tettetek e föld lakosai elõtt, a Kananeusok és Perizeusok elõtt; én pedig kevesed magammal vagyok, és ha összegyûlnek ellenem, levágnak, és eltörölnek engem, mind házam népével egybe.
\par 31 Azok pedig mondának: Hát mint tisztátalan személylyel, úgy kellett-é bánni a mi húgunkkal?

\chapter{35}

\par 1 Monda pedig az Isten Jákóbnak: Kelj fel, eredj fel Béthelbe és telepedjél le ott; és csinálj ott oltárt amaz Istennek, ki megjelenék néked, mikor a te bátyád Ézsaú elõtt  futsz vala.
\par 2 Akkor monda Jákób az õ házanépének, és mind azoknak, kik vele valának: Hányjátok el az idegen isteneket, kik köztetek vannak, és tisztítsátok meg magatokat, és változtassátok el öltözeteiteket.
\par 3 És keljünk fel, és menjünk fel Béthelbe, hogy csináljak ott oltárt annak az Istennek, ki meghallgatott engem az én nyomorúságom napján, és velem volt az úton, a melyen jártam.
\par 4 Átadák azért Jákóbnak mind az idegen isteneket, kik nálok valának, és füleikbõl a függõket, és elásá azokat Jákób a cserfa alatt, mely Sekhem mellett vala.
\par 5 És elindulának. De Istennek rettentése vala a körûltök való városokon, és nem üldözék a Jákób fiait.
\par 6 Eljuta azért Jákób Lúzba, mely Kanaán földén van, azaz Béthelbe, õ maga és az egész sokaság, mely õ vele vala.
\par 7 És építe ott oltárt, és nevezé a helyet Él-Béthelnek, mivelhogy ott jelent meg néki az Isten, mikor az õ bátyja elõtt fut vala.
\par 8 És meghala Débora, a Rebeka dajkája, és eltemeték Béthelen alól egy cserfa alatt, és nevezék annak nevét Allon-Bákhutnak.
\par 9 Az Isten pedig ismét megjelenék Jákóbnak, mikor ez jöve Mésopotámiából, és megáldá õt.
\par 10 És monda néki az Isten: A te neved Jákób; de ne neveztessék többé a te neved Jákóbnak, hanem Izráel légyen neved. És nevezé nevét Izráelnek.
\par 11 És monda néki az Isten: Én vagyok a mindenható Isten, nevekedjél és sokasodjál, nép és népek sokasága légyen te tõled; és királyok származzanak a te ágyékodból.
\par 12 És a földet, melyet adtam Ábrahámnak és Izsáknak, néked adom azt, utánad pedig a te magodnak adom a földet.
\par 13 És felméne õ tõle az Isten azon a helyen, a hol vele szólott vala.
\par 14 Jákób pedig emlékoszlopot állíta azon a helyen, a hol szólott vele, kõoszlopot; és áldozék azon italáldozattal, és önte arra olajat.
\par 15 És nevezé Jákób a hely nevét, a hol az Isten szólott vala õ vele, Béthelnek.
\par 16 És elindúlának Béthelbõl, s mikor Efratától, hogy oda érjenek, már csak egy dûlõföldre valának, szûle Rákhel, és nehéz vala az õ szûlése.
\par 17 S vajudása közben monda néki a bába: ne félj, mert most is fiad lesz.
\par 18 És mikor lelke kiméne, mert meghala, nevezé nevét Benóninak, az atyja pedig nevezé õt Benjáminnak.
\par 19 És meghala Rákhel, és eltemetteték az Efratába (azaz Bethlehembe) vivõ úton.
\par 20 És emlékoszlopot állíta Jákób az õ sírja fölött. Rákhel sírjának emlékoszlopa az mind e mai napig.
\par 21 Azután tovább költözék Izráel, és a Héder tornyán túl voná fel sátorát.
\par 22 Lõn pedig, mikor Izráel azon a földön lakozék, elméne Rúben, és hála Bilhával, az õ atyjának ágyasával, s meghallá Izráel. Valának pedig a Jákób fiai tizenketten.
\par 23 Lea fiai: Jákób elsõszülötte Rúben, azután Simeon, Lévi, Júda, Izsakhár és Zebulon.
\par 24 Rákhel fiai: József és Benjámin.
\par 25 A Rákhel szolgálójának Bilhának fiai: Dán és Nafthali.
\par 26 A Lea szolgálójának Zilpának fiai: Gád és Áser. Ezek a Jákób fiai, a kik születtek néki Mésopotámiában.
\par 27 És eljuta Jákób Izsákhoz, az õ atyjához Mamréba, Kirját-Arbába, azaz Hebronba, a hol Ábrahám és Izsák tartózkodnak vala.
\par 28 Valának pedig Izsák napjai száz nyolczvan esztendõ.
\par 29 És kimúlék Izsák és meghala, és takaríttaték az õ eleihez vén korban, bételvén az élettel; és eltemeték õt az õ fiai, Ézsaú és Jákób.

\chapter{36}

\par 1 Ez Ézsaúnak (azaz Edómnak) nemzetsége.
\par 2 Ézsaú a Kananeusok leányai közûl vette vala feleségeit: Adát, a Khitteus Élonnak leányát; és Oholibámát, Anáhnak leányát, a ki Khivveus Czibhón leánya vala.
\par 3 És Boszmáthot, az Ismáel leányát, Nebajóthnak húgát.
\par 4 Szûlé pedig Adá Ézsaúnak Elifázt; és Boszmáth szûlé Rehuélt.
\par 5 Oholibáma pedig szûlé Jehúst, Jahlámot és Kórét. Ezek az Ézsaú fiai, kik születtek néki Kanaán földén.
\par 6 És felvevé Ézsaú az õ feleségeit, az õ fiait, és az õ leányait, és minden házabeli lelket, minden juhait, barmait, és minden jószágát, melyet szerzett vala Kanaán földén, és elméne más országba, az õ atyjafiának Jákóbnak színe elõl.
\par 7 Mert az õ jószáguk több vala, semhogy együtt lakhattak volna, és tartózkodásuk földe nem bírja vala meg õket az õ nyájaik miatt.
\par 8 Letelepedék tehát Ézsaú a Széir hegyén. Ézsaú pedig az Edóm.
\par 9 Ez Ézsaúnak az Edomiták atyjának nemzetsége a Széir hegyen.
\par 10 Ezek Ézsaú fiainak nevei: Elifáz, Adának Ézsaú feleségének fia. Rehuél, Boszmáthnak Ézsaú feleségének fia.
\par 11 Elifáznak fiai valának: Thémán, Omár, Czefó, Gahtám és Kenáz.
\par 12 Thimna pedig Elifáznak, az Ézsaú fiának ágyasa vala, ki Elifáznak szûlé Amáleket. Ezek Adának, Ézsaú feleségének fiai.
\par 13 Ezek pedig a Rehuél fiai: Nakhat, Zerakh, Sammá, Mizzá. Ezek valának Boszmáthnak, Ézsaú feleségének fiai.
\par 14 Oholibámának pedig, Ézsaú feleségének, Anáh leányának, ki Czibhón leánya volt, ezek valának fiai, kiket szûle Ézsaúnak: Jéhus, Jahlám és Korakh.
\par 15 Ezek Ézsaú fiainak fejedelmei: Elifáznak, Ézsaú elsõszülöttének fiai: Thémán fejedelem, Omár fejedelem, Czefó fejedelem, Kenáz fejedelem.
\par 16 Korakh fejedelem, Gahtám fejedelem, Amálek fejedelem. Ezek Elifáztól való fejedelmek Edómnak országában: ezek Adá fiai.
\par 17 Rehuélnek pedig, az Ézsaú fiának fiai ezek: Nakhath fejedelem, Zerakh fejedelem, Sammá fejedelem, Mizzá fejedelem. Ezek Rehuéltõl való fejedelmek Edóm országában. Ezek Boszmáthnak, Ézsaú feleségének fiai.
\par 18 Ezek pedig Oholibámának, Ézsaú feleségének fiai: Jéhus fejedelem, Jahlám fejedelem, Korakh fejedelem. Ezek Oholibámától, Anáh leányától, Ézsaú feleségétõl való fejedelmek.
\par 19 Ezek Ézsaú fiai, és ezek azoknak fejedelmei; ez Edóm.
\par 20 A Horeus Széirnek fiai, kik ama földön laknak vala ezek: Lótán, Sóbál, Czibhón, Anáh.
\par 21 Disón, Eczer, Disán. Ezek a Horeusok fejedelmei, Széirnek fiai Edóm országában.
\par 22 Lótánnak pedig fiai voltak: Hóri, Hémám, és Lótánnak húga, Timna.
\par 23 Sóbálnak fiai ezek: Halván, Mánakháth, Hébál, Sefó, Onám.
\par 24 Czibhónnak pedig fiai ezek: Aja és Anáh, az az Anáh, ki meleg forrásokat talált a pusztában, mikor atyjának Czibhónnak szamarait legelteté.
\par 25 Anáhnak fiai ezek: Disón és Oholibáma Anáhnak leánya.
\par 26 Disónnak fiai ezek: Hemdán, Esbán, Ithrán, Kherán.
\par 27 Eczernek fiai ezek: Bilhán, Zahaván, Hakán.
\par 28 Disánnak fiai ezek: Húcz és Arán.
\par 29 A Horeusok közûl való fejedelmek pedig ezek: Lótán fejedelem, Sóbál fejedelem, Czibhón fejedelem, Anáh fejedelem.
\par 30 Disón fejedelem, Eczer fejedelem, Disán fejedelem. Ezek a Horeusok közûl való fejedelmek az õ fejedelemségök szerint, Széir tartományában.
\par 31 Ezek pedig a királyok, kik uralkodtak Edóm földén minekelõtte Izráel fiai között király uralkodott volna.
\par 32 Király vala Edómban Bela, Behor fia, s az õ városának neve Dinhába vala.
\par 33 És meghala Bela, uralkodék helyette Jóbáb, Zerakh fia, ki Boczrából való volt.
\par 34 És meghala Jóbáb, és uralkodék helyette Témán földébõl való Khusám.
\par 35 És meghala Khusám és uralkodék helyette Hadád, a Bédád fia, a ki megveré a Midiánitákat a Moáb mezején; az õ városának neve pdig Hávit vala.
\par 36 És meghala Hadád és uralkodék helyette a Masrekából való Szamlá.
\par 37 És meghala Szamlá és uralkodék helyette Saul, a folyóvíz mellett való Rékhobóthból.
\par 38 És meghala Saul és uralkodék helyette Báhál-Khanán, Akhbór fia.
\par 39 És meghala Báhál-Khanán, Akhbór fia, és uralkodék helyette Hadár; és az õ városának neve Pahu, az õ feleségének pedig neve Mehetábéel, a ki Mézaháb leányának Matrédnak leánya vala.
\par 40 Ezek pedig az Ézsaú nemzetségébõl való fejedelmek nevei, az õ családjok, helyök és nevök szerint: Timná fejedelem, Halvá fejedelem, Jetéth fejedelem.
\par 41 Oholibámá fejedelem, Éla fejedelem, Pinon fejedelem.
\par 42 Kenáz fejedelem, Thémán fejedelem, Mibczár fejedelem.
\par 43 Magdiél fejedelem, Hirám fejedelem. Ezek Edóm fejedelmei az õ lakások szerint, az õ örökségök földén. Ézsaú az Edómiták atyja.

\chapter{37}

\par 1 Jákób pedig lakozék az õ atyja bujdosásának földén, Kanaán földén.
\par 2 Ezek a Jákób nemzetségének dolgai: József tizenhét esztendõs korában az õ bátyjaival együtt juhokat õriz vala, bojtár vala Bilhának és Zilpának az õ atyja feleségeinek fiai mellett, és József rossz híreket hord vala felõlük az õ atyjuknak.
\par 3 Izráel pedig minden fiánál inkább szereti vala Józsefet, mivelhogy vén korában nemzette vala õt; és czifra ruhát csináltat vala néki.
\par 4 Mikor pedig láták az õ bátyjai, hogy atyjuk minden testvére közt õt szereti legjobban, meggyûlölik vala, és jó szót sem bírnak vala hozzá szólani.
\par 5 És álmot álmodék József és elbeszélé az õ bátyjainak: és azok annál inkább gyûlölik vala õt.
\par 6 Mert monda nékik: Hallgassátok meg, kérlek, ezt az álmot, melyet álmodtam.
\par 7 Ímé kévéket kötünk vala a mezõben, és ímé az én kévém felkele és felálla; a ti kévéitek pedig körûlállanak, és az én kévéim elõtt meghajolnak vala.
\par 8 És mondának néki az õ bátyjai: Avagy király akarsz-é lenni felettünk? Vagy uralkodni akarsz-é rajtunk? S annál is inkább gyûlölik vala õt álmáért és beszédéért.
\par 9 Más álmot is álmodék, és elbeszélé azt az õ bátyjainak, mondván: ímé megint álmot álmodtam; ímé a nap és a hold, és tizenegy csillag meghajol vala én elõttem.
\par 10 S elbeszélé atyjának és bátyjainak, és az õ atyja megdorgálá õt, mondván néki: Micsoda álom az a melyet álmodtál? Avagy elmegyünk-é, én és a te anyád és atyádfiai, hogy meghajtsuk magunkat te elõtted földig?
\par 11 Irígykednek vala azért reá az õ bátyjai; az õ atyja pedig elméjében tartja vala e dolgot.
\par 12 Mikor pedig az õ bátyjai elmenének Sikhembe, hogy az õ atyjok juhait õrizzék;
\par 13 Monda Izráel Józsefnek: A te bátyáid avagy nem Sikhemben legeltetnek-é? Jöszte, és én hozzájok küldelek téged. Õ pedig monda: Ímhol vagyok.
\par 14 És monda néki: Menj el, nézd meg, hogy s mint vagynak a te bátyáid és a juhok, s hozz hírt nékem. Elküldé tehát õt Hebron völgyébõl, és méne Sikhembe.
\par 15 Elõtalálá pedig õt egy ember, mikor a mezõben bolyong vala, és megkérdé õt az az ember, mondván: Mit keressz?
\par 16 És monda: Az én bátyáimat keresem, kérlek, mondd meg nékem, hol legeltetnek?
\par 17 És monda az ember: Elmentek innen, mert hallám, hogy mondák: Menjünk Dóthánba. Elméne azért József az õ bátyjai után, és megtalálá õket Dóthánban.
\par 18 Mikor távolról megláták, minekelõtte közel ért volna hozzájok, összebeszélének, hogy megölik.
\par 19 És szólának egymás között: Ímhol jõ az álomlátó!
\par 20 Most hát jertek öljük meg õt, és vessük õt valamelyik kútba; és azt mondjuk, hogy fenevad ette meg, és meglátjuk, mi lesz az õ álmaiból.
\par 21 Meghallá pedig Rúben és megmenté õt kezökbõl, és mondá: Ne üssük õt agyon.
\par 22 És mondá nékik Rúben: Ne ontsatok vért, vessétek õt ebbe a kútba, a mely itt a pusztában van, de kezet ne vessetek reá. Azért, hogy megszabadítsa õt kezökbõl, hogy visszavigye atyjához.
\par 23 És lõn, a mint oda ére József az õ bátyjaihoz, letépték Józsefrõl az õ felsõ ruháját, a czifra ruhát, mely rajta vala.
\par 24 És megragadák õt és beleveték a kútba; a kút pedig üres vala, nem vala víz benne.
\par 25 Azután leûlének kenyerezni, és felemelék szemeiket, és láták, hogy ímé egy Ismáelita karaván jõ vala Gileádból, és azoknak tevéi visznek vala fûszerszámot, balzsamot és mirhát, menvén, hogy alávigyék Égyiptomba.
\par 26 És monda Júda az õ atyjafiainak: Mi haszna, ha megöljük a mi atyánkfiát, és eltitkoljuk az õ vérét?
\par 27 Jertek adjuk el õt az Ismáelitáknak, és ne tegyük reá kezünket, mert atyánkfia, vérünkbõl való õ. És hallgatának rá az õ atyjafiai.
\par 28 És menének arra Midiánita kereskedõ férfiak, és kivonák és felhozák Józsefet a kútból, és eladák Józsefet az Ismáelitáknak húsz ezüstpénzen: azok pedig elvivék Józsefet Égyiptomba.
\par 29 És visszatére Rúben a kúthoz, és ímé József nem vala a kútban, és megszaggatá ruháit.
\par 30 És megtére az õ atyjafiaihoz, és monda: Nincsen gyermek, és én, merre menjek én?
\par 31 Akkor vevék a József felsõ ruháját, és leölének egy kecskebakot, és belemárták a felsõ ruhát a vérbe.
\par 32 És elküldék a czifra ruhát, és elvivék atyjokhoz és mondának: Ezt találtuk, ismerd meg, fiad ruhája-é, vagy nem?
\par 33 És megismeré azt, és monda: Fiam felsõ ruhája ez, fenevad ette meg õt, bizony széllyelszaggatta Józsefet.
\par 34 És megszaggatá Jákób ruháit, és zsákba öltözék és gyászolá az õ fiát sokáig.
\par 35 Felkelének pedig minden õ fiai, és minden õ leányai, hogy vígasztalják õt, de nem akara vígasztalódni, hanem monda: Sírva megyek fiamhoz a sírba; és siratá õt az atyja.
\par 36 A Midiániták pedig eladák õt Égyiptomba Pótifárnak, a Faraó fõemberének, a testõrök fõhadnagyának.

\chapter{38}

\par 1 És lõn abban az idõben, hogy Júda elméne az õ atyjafiaitól, és betére egy Adullámbeli férfiúhoz, kinek neve Khira vala.
\par 2 És meglátá ott Júda és Súah nevû Kanaánbeli férfiúnak leányát, és elvevé azt, és beméne hozzá.
\par 3 És az fogada méhében és szûle fiat, és nevezé nevét Hérnek.
\par 4 És ismét fogada méhében, s fiat szûle, és nevezé nevét Ónánnak.
\par 5 Még egyszer szûle fiat, és nevezé nevét Sélának, és mikor azt szûlé, Khezibben vala.
\par 6 És võn Júda az õ elsõszülött fiának Hérnek feleséget, ennek neve Thámár vala.
\par 7 De Hér, Júdának elsõszülött fia gonosz vala az Úr szemei elõtt, és megölé õt az Úr.
\par 8 És monda Júda Ónánnak: Eredj be a te bátyád feleségéhez, és vedd feleségûl mint sógor, és támaszsz magot bátyádnak.
\par 9 Ónán pedig tudja vala, hogy a magzat nem lesz az övé, azért valamikor az õ bátyja feleségéhez bemegy vala, földre vesztegeti vala el a magot, hogy bátyjának magot ne támaszszon.
\par 10 És gonoszságnak tetszék az Úr szemei elõtt, a mit cselekszik vala, annakokáért megölé õt is.
\par 11 És monda Júda Thámárnak, az õ menyének: Maradj özvegyen addig a te atyád házában, míg az én fiam Séla felnevekedik. Mert így gondolkodik vala: Netalán ez is meghal, mint az õ bátyjai. Elméne azért Thámár, és marada az õ atyja házában.
\par 12 Sok idõ múlva meghala Súa leánya, a Júda felesége. Júda pedig megvígasztalódék és elméne az õ juhainak nyírõihez, barátjával az Adullámbeli Khirával, Thimnába.
\par 13 Hírül adák pedig Thámárnak mondván: Ím a te ipad Thimnába megy juhainak nyírésére.
\par 14 Leveté azért magáról özvegyi ruháját, elfátyolozá és beburkolá magát, és leûle Enajim kapujába, mely a Thimnába vezetõ úton van; mert látja vala, hogy felnevekedék Séla, és még sem adák õt annak feleségûl.
\par 15 Meglátá pedig õt Júda, és tisztátalan személynek gondolá, mivelhogy befedezte vala orczáját.
\par 16 És hozzá tére az útra és monda: Engedd meg kérlek, hogy menjek be te hozzád, mert nem tudja vala, hogy az õ menye az. Ez pedig monda: Mit adsz nékem ha bejösz hozzám?
\par 17 És felele: Küldök néked az én nyájamból egy kecskefiat. És az monda: Adsz-é zálogot, míg megküldöd?
\par 18 És monda: Micsoda zálogot adjak néked? És monda: Gyûrûdet, gyûrûd zsinórját és pálczádat, mely kezedben van. Oda adá azért néki, és beméne hozzá, és teherbe ejté.
\par 19 Azután felkele és elméne, és leveté magáról a fányolt; és felvevé az õ özvegyi ruháját.
\par 20 És megküldé Júda a kecskefiat az õ Adullámbeli barátjától, hogy visszavegye a zálogot az asszonytól, de nem találá azt.
\par 21 És megkérdé a helység férfiait, mondván: Hol van az a felavatott parázna nõ, a ki Enajim mellett az útfélen vala? És azok mondának: Nem volt erre felavatott parázna nõ.
\par 22 Visszatére tehát Júdához, és monda: Nem találám azt meg, a helység lakosai is azt mondák: Nem volt erre felavatott parázna nõ.
\par 23 És monda Júda: Tartsa magának, hogy csúffá ne legyünk; ímé én megküldöttem volt ezt a kecskefiat, te pedig nem találtad meg õt.
\par 24 És lõn mintegy három hónap múlva, jelenték Júdának, mondván: Thámár a te menyed paráználkodott, és ímé terhes is a paráznaság miatt. És monda Júda: Vigyétek ki õt, és égettessék meg.
\par 25 Mikor pedig kivitetnék, elkülde az õ ipához, mondván: Attól a férfiútól vagyok terhes, a kiéi ezek. És mondá: Ismerd meg, kérlek, kié e gyûrû, e zsinór, és e pálcza.
\par 26 És megismeré Júda és monda: Igazabb õ nálamnál, mert bizony nem adám õt az én fiamnak Sélának; de nem ismeré õt Júda többé.
\par 27 És lõn az õ szûlésének idején, ímé kettõsök valának az õ méhében.
\par 28 És lõn, hogy szûlése közben az egyik kinyújtá kezét, és fogá a bába és a veres fonalat köte reá, mondván: Ez jött ki elõször.
\par 29 De lõn, hogy a mikor visszavoná kezét, ímé az õ testvére jöve ki. és mondá a bába: Hogy törtél te magadnak rést? Azért nevezé nevét Pérecznek.
\par 30 És után kijöve az õ testvére kinek veres fonál vala kezén; és nevezé nevét Zerákhnak.

\chapter{39}

\par 1 József pedig aláviteték Égyiptomba és megvevé õt az Ismáelitáktól, kik õt oda vitték vala, egy égyiptomi ember Pótifár, a Faraó fõembere, a testõrök fõhadnagya.
\par 2 És az Úr Józseffel vala, és szerencsés ember vala és az õ égyiptomi urának házában vala.
\par 3 Látá pedig az õ ura, hogy az Úr van õ vele, és hogy valamit cselekszik, az Úr mindent szerencséssé tesz az õ kezében:
\par 4 Kedvessé lõn azért József az õ ura elõtt, és szolgál vala néki; és háza felvigyázójává tevé, és mindenét, a mije vala, kezére bízá.
\par 5 És lõn az idõtõl fogva, hogy házának és a mije volt, mindenének gondviselõjévé tevé, megáldá az Úr az égyiptomi embernek házát Józsefért; és az Úr áldása vala mindenen, a mije csak volt a házban és a mezõn.
\par 6 Mindent azért valamije vala József kezére bíza; semmire sem vala gondja mellette, hanem ha az ételre, melyet megeszik vala. József pedig szép termetû és szép arczú vala.
\par 7 És lõn ezek után, hogy az õ urának felesége Józsefre veté szemeit, és monda: Hálj velem.
\par 8 Õ azonban vonakodék s monda az õ ura feleségének: Ímé az én uramnak én mellettem semmi gondja nincs az õ háza dolgaira, és a mije van, mindenét az én kezemre bízá.
\par 9 Senki sincs nálamnál nagyobb az õ házában; és tõlem semmit sem tiltott meg, hanem csak téged, mivelhogy te felesége vagy; hogy követhetném hát el ezt a nagy gonoszságot és hogyan vétkezném az Isten ellen?
\par 10 És lõn, hogy az asszony mindennap azt mondogatá Józsefnek, de õ nem hallgata reá, hogy vele háljon és vele egyesüljön.
\par 11 Lõn azért egy napon, hogy valami dolgát végezni a házba beméne, és a háznép közûl senki sem vala ott benn a házban,
\par 12 És megragadá õt ruhájánál fogva, mondván: Hálj velem. Õ pedig ott hagyá a ruháját az asszony kezében és elfuta és kiméne.
\par 13 És mikor látja vala az asszony, hogy az a maga ruháját az õ kezében hagyta, és kifutott:
\par 14 Összehívá a háznépet és szóla hozzájuk, mondván: Lássátok, héber embert hozott hozzánk, hogy megcsúfoljon bennünket. Bejött hozzám, hogy velem háljon, s én fenszóval kiálték.
\par 15 És lõn, a mint hallja vala, hogy fenszóval kezdék kiáltani, ruháját nálam hagyá és elfuta és kiméne.
\par 16 Megtartá azért az õ ruháját magánál, míg az õ ura haza jöve.
\par 17 S ilyen szókkal szóla hozzá, mondván: Bejöve hozzám a héber szolga, a kit ide hoztál, hogy szégyent hozzon reám.
\par 18 És mikor fenszóval kezdtem kiáltani, ruháját nálam hagyá és kifuta.
\par 19 És lõn, a mint hallja az õ ura az õ feleségének beszédeit, melyeket néki beszélt, mondván: Ilyesmiket tett velem a te szolgád; haragra gerjede.
\par 20 Vevé azért Józsefet az õ ura és veté õt a tömlöczbe, melyben a király foglyai valának fogva, és ott vala a tömlöczben.
\par 21 De az Úr Józseffel vala, és kiterjeszté reá az õ kegyelmességét és kedvessé tevé õt a tömlöcztartó elõtt.
\par 22 És a tömlöcztartó mind azokat a foglyokat, kik a tömlöczben valának, József kezébe adá, úgyhogy a mi ott történik vala, minden õ általa történék.
\par 23 És semmi gondja nem vala a tömlöcztartónak azokra, melyek keze alatt valának, mivelhogy az Úr vala Józseffel, és valamit cselekeszik vala, az Úr szerencséssé teszi vala.

\chapter{40}

\par 1 És lõn ezekután, hogy az égyiptomi király pohárnoka és sütõmestere vétkezének az õ urok ellen, az égyiptomi király ellen.
\par 2 Megharaguvék azért a Faraó az õ két fõemberére, a fõpohárnokra, és a fõsütõmesterre.
\par 3 És fogságba vetteté azokat a testõrök fõhadnagyának házában levõ tömlöczbe, arra a helyre, ahol fogva vala József.
\par 4 A testõrök fõhadnagya pedig Józsefet rendelé melléjük és szolgála nékik. És jó ideig valának fogságban.
\par 5 És az égyiptomi király pohárnoka és sütõmestere, a kik a tömlöczben fogva valának, látának álmot mindketten; mindegyik külön álmot, azon egy éjjel, mindegyik az õ álmának értelme szerint.
\par 6 És beméne hozzájok József reggel, látá, hogy ímé bánkódnak vala.
\par 7 És megkérdé a Faraó fõembereit, a kik az õ ura házánál vele együtt fogva valának, mondván: Miért oly komor ma a ti orczátok?
\par 8 És mondának néki: Álmot láttunk és nincsen a ki megfejtse azt. És monda nékik József: A megfejtés nem Isten dolga-é? mondjátok el, kérlek, nékem.
\par 9 Elbeszélé azért a fõpohárnok az õ álmát Józsefnek, és monda néki: Álmomban ímé egy szõlõtõ vala elõttem;
\par 10 És a szõlõtõn három szál vesszõ vala, s alighogy bimbózék, virágozék, és gerézdjei megérlelék a szõlõszemeket.
\par 11 A Faraó pohara pedig az én kezemben vala, és én vevém a szõlõszemeket és facsarám a Faraó poharába, és adom vala a poharat a Faraó kezébe.
\par 12 És monda néki József: Ez annak a megfejtése: a három vesszõszál, három nap.
\par 13 Harmadnap múlva a Faraó felmagasztalja a te fejedet, és visszahelyez téged hivatalodba, és adod a Faraó kezébe az õ poharát, az elébbi tiszted szerint, mikor az õ pohárnokja valál.
\par 14 Csakhogy azután megemlékezzél rólam, mikor néked jól lesz dolgod, és cselekedjél, kérlek, irgalmasságot velem, emlékezzél meg rólam a Faraó elõtt és szabadíts meg engem e házból.
\par 15 Mert lopva hoztak el engem a héberek földérõl, és itt sem cselekedtem semmit, hogy tömlöczbe vessenek.
\par 16 És látá a fõsütõmester, hogy jól magyaráz vala és monda Józsefnek: Álmodtam én is, hogy ímé három kosár kalács vala fejemen.
\par 17 A felsõ kosárban pedig valának a Faraónak mindenféle süteményei, és a madarak eszik vala azokat a kosárból, az én fejemrõl.
\par 18 És felele József és monda: Ez annak a magyarázatja: a három kosár, három nap.
\par 19 Harmadnap múlva fejedet véteti a Faraó és fára akasztat fel téged, és a madarak leeszik rólad húsodat.
\par 20 S lõn harmadnapon a Faraó születése napja, és vendégséget szerze minden õ szolgáinak, s akkor a fõpohárnokot és a fõsütõmestert is fölvevé szolgái közé.
\par 21 És a fõpohárnokot visszahelyezé pohárnokságába, és nyújtotta a poharat a Faraó kezébe.
\par 22 A fõsütõmestert pedig felakasztatá; a miképen magyarázta vala nékik József.
\par 23 És nem emlékezék meg a fõpohárnok Józsefrõl, hanem elfelejtkezék róla.

\chapter{41}

\par 1 Lõn pedig két esztendõ múlván, hogy a Faraó álmot láta, s ímé áll vala a folyóvíz mellett.
\par 2 És ímé a folyóvízbõl hét szép és kövér tehén jõ vala ki, és legel vala a nádasban.
\par 3 S ímé azok után más hét tehén jõ vala ki a folyóvízbõl, rútak és ösztövérek, és oda állanak vala ama tehenek mellé a folyóvíz partján.
\par 4 És elnyelék a rút és ösztövér tehenek a hét szép és kövér tehenet; és felserkene a Faraó.
\par 5 És elaluvék és másodszor is álmot láta, és ímé hét gabonafej nevekedik vala egy száron, mind teljes és szép.
\par 6 És ímé azok után hét vékony s keleti széltõl kiszáradt gabonafej nevekedik vala.
\par 7 És elnyelék a vékony gabonafejek a hét kövér és teljes gabonafejet. És felserkene a Faraó, és ímé álom vala.
\par 8 Reggelre kelvén, nyugtalankodék lelkében, elkülde azért és egybehívatá Égyiptom minden jövendõmondóját, és minden bölcsét és elbeszélé nékik a Faraó az õ álmát, de senki sem vala, ki azokat megmagyarázta volna a Faraónak.
\par 9 Szóla azért a fõpohárnok a Faraónak, mondván: Az én bûneimrõl emlékezem e napon.
\par 10 A Faraó megharagudt vala az õ szolgáira, és fogságba vettetett vala engem a testõrök fõhadnagyának házába, engem és a fõsütõmestert.
\par 11 És álmot látánk egy éjjel, én is, az is; mindegyikünk a maga álmának értelme szerint álmoda.
\par 12 És vala velünk ott egy héber ifjú, a testõrök fõhadnagyának szolgája; elbeszéltük vala néki, és õ megfejté nékünk a mi álmainkat; mindegyikünknek az õ álma szerint fejté meg.
\par 13 És lõn, hogy a miképen megfejté nékünk, úgy lõn: engem visszahelyeze hívatalomba, amazt pedig felakasztatá.
\par 14 Elkülde azért a Faraó és hívatá Józsefet, és hamarsággal kihozák õt a tömlöczbõl, és megborotválkozék, ruhát válta és a Faraóhoz méne.
\par 15 És monda a Faraó Józsefnek: Álmot láttam és nincs a ki megmagyarázza azt: Én pedig azt hallottam rólad beszélni, hogy ha meghallod az álmot, meg is magyarázod azt.
\par 16 És felele József a Faraónak, mondván: Nem én, Isten jelenti meg, a mi a Faraónak javára van.
\par 17 Monda azért a Faraó Józsefnek: Álmomban ímé állok vala a folyóvíz partján.
\par 18 És ímé a folyóvízbõl hét kövér és szép tehén jõ vala ki, és legel vala a nádasban.
\par 19 S ímé azok után más hét tehén jõ vala ki, nagyon ösztövérek, rútak és hitványak; egész Égyiptom földén nem láttam azokhoz hitványságra hasonlókat.
\par 20 És elnyelék az ösztövér és rút tehenek, az elébbi hét kövér tehenet.
\par 21 És azok gyomrukban valának, de nem tetszik vala meg, hogy gyomrukban volnának, mert külsejök oly rút vala, mint az elõtt. És felserkenék.
\par 22 És láttam álmomban, és ímé hét gabonafej nevekedik vala egy száron, mind teljes és szép.
\par 23 És ímé hét összeaszott, vékony, keleti széltõl kiszáradt gabonafej nevekedik vala azok után,
\par 24 És elnyelék a vékony gabonafejek a hét szép gabonafejet. És elmondám az írástudóknak, de senki sincs, a ki megmagyarázza nékem.
\par 25 És monda József a Faraónak: A Faraó álma egy és ugyanaz; a mit Isten cselekedni akar, azt jelentette meg a Faraónak.
\par 26 A hét szép tehén, hét esztendõ, a hét szép gabonafej az is hét esztendõ; az álom egy és ugyanaz.
\par 27 A hét ösztövér és rút tehén pedig, melyek amazok után jöttek ki, az is hét esztendõ, és a hét vékony, keleti széltõl kiszáradt gabonafej, az az éhségnek hét esztendeje.
\par 28 Ez az a mit mondanék a Faraónak, hogy a mit Isten cselekedni akar, megmutatta a Faraónak.
\par 29 Ímé hét esztendõ jõ, és nagy bõség lesz egész Égyiptomban.
\par 30 Azok után pedig következik az éhség hét esztendeje, s minden bõséget elfelejtenek Égyiptom földén, és megemészti az éhség a földet.
\par 31 És nem ismerszik meg az elébbi bõség e földön az utána következõ éhség miatt, mert igen nagy lesz.
\par 32 Hogy pedig a Faraó álma kettõs, kétszeres vala, onnan van, mert Istennél elvégezett dolog ez, és siet az Isten azt véghez vinni.
\par 33 Most azért szemeljen ki a Faraó egy értelmes és bölcs férfit, és tegye Égyiptom földén gondviselõvé.
\par 34 Cselekedje ezt a Faraó és rendeljen tiszttartókat az országnak, és szedjen ötödöt Égyiptom földén a hét bõ esztendõben.
\par 35 És takarítsák be a következõ jó esztendõk minden termését, és gyûjtsenek gabonát a Faraó keze alá, élelmûl a városokban, és tartsák meg,
\par 36 És legyen az élelem tartalékban az ország számára az éhség hét esztendejére, melyek elkövetkeznek Égyiptom földére, hogy el ne vesszen e föld az éhség miatt.
\par 37 És tetszék e beszéd a Faraónak és minden õ szolgáinak.
\par 38 Monda azért a Faraó az õ szolgáinak: Találhatnánk-é ehhez hasonló férfit, a kiben az Isten lelke van?
\par 39 És monda a Faraó Józsefnek: Mivelhogy Isten mind ezeket néked jelentette meg, nincs hozzád fogható értelmes és bölcs ember.
\par 40 Te légy az én házamon fõgondviselõ, és minden népem a te szavadra hallgasson, csak a királyiszék tesz engem nálad nagyobbá.
\par 41 Monda továbbá a Faraó Józsefnek: Ímé fejedelemmé tettelek az egész Égyiptom földén.
\par 42 És levéve a Faraó a maga gyûrûjét kezérõl, és adá azt József kezére; és felöltözteté õt drága gyolcs ruhába, és aranylánczot tõn az õ nyakába.
\par 43 És meghordoztatá õt az õ második szekerén, és kiáltják vala õ elõtte: Térdet hajtsatok! Így tevé õt fejedelemmé az egész Égyiptom földén.
\par 44 És monda a Faraó Józsefnek: Én vagyok a Faraó, de te nálad nélkül senki se kezét, se lábát fel ne emelhesse egész Égyiptom földén.
\par 45 És nevezé a Faraó József nevét Czafenát-Pahneákhnak, és adá néki feleségûl Aszenáthot, Potiferának On papjának leányát. És kiméne József Égyiptom földére.
\par 46 József pedig harmincz esztendõs vala, mikor a Faraó elõtt, az égyiptomi király elõtt álla. Kiméne tehát József a Faraó elõl, és bejárá az egész Égyiptom földét.
\par 47 És a föld a hét bõ esztendõ alatt tele marokkal ontá a termést.
\par 48 És összegyûjté a hét esztendõnek minden eleségét, mely vala Égyiptom földén, és a városokba takarítá az élelmet, minden városba a körülte levõ határ élelmét takarítá be.
\par 49 És felhalmozá József a gabonát, mint a tenger fövénye, igen sokat, annyira, hogy megszûntek azt számba venni, mivelhogy száma nem vala.
\par 50 Józsefnek pedig születék két fia az éhség esztendejének eljötte elõtt, kiket szûle néki Aszenáth, Potiferának az On papjának leánya.
\par 51 És nevezé József az elsõszülöttnek nevét Manassénak: mert úgymond, elfelejteté én velem isten minden én veszõdségemet, és az én atyámnak egész házát.
\par 52 A másodiknak nevét pedig nevezé Efraimnak: mivel, ugymond, megszaporított engem Isten az én nyomorúságomnak földén.
\par 53 Eltele tehát a bõség hét esztendeje, a mely vala Égyiptomnak földjén.
\par 54 és elkezdõdék az éhség  hét esztendeje, mint megmondotta vala József; és lõn éhség minden országban; de Égyiptom földén mindenütt vala kenyér.
\par 55 De megéhezék egész Égyiptom földe is, és kenyérért kiált vala a nép a Faraóhoz. - Monda pedig a Faraó mind az Égyiptombelieknek: Menjetek Józsefhez, és a mit mond néktek, azt míveljétek.
\par 56 És az éhség mind az egész földön vala. Akkor mind megnyitá József a gabonás házakat, és árulja vala az Égyiptombelieknek; mert nagyobbodik vala az éhség Égyiptom földén.
\par 57 És mind az egész föld Égyiptomba megy vala Józsefhez gabonát venni; mert nagy vala az éhség az egész földön.

\chapter{42}

\par 1 És látá Jákób, hogy van gabona Égyiptomban és monda Jákób az õ fiainak: Mit néztek egymásra?
\par 2 És monda: Ímé hallom hogy Égyiptomban van gabona; menjetek le oda, és vegyetek onnan nékünk gabonát, hogy éljünk, és ne haljunk meg.
\par 3 Leméne azért József tíz bátyja Égyiptomba gabonát venni.
\par 4 De Benjámint a József öcscsét nem bocsátá el Jákób az õ bátyjaival; mert mondá: Netalán veszedelem érhetné.
\par 5 Elmenének tehát Izráel fiai gabonát venni az oda menõkkel együtt, mert éhség vala Kanaán földén.
\par 6 József pedig az ország kormányzója vala, és õ árulja vala a gabonát a föld minden népének. Eljutának azért a József bátyjai, és arczczal a földre borúlának õ elõtte.
\par 7 Amint meglátá József az õ bátyjait, megismeré õket, de idegennek mutatá magát hozzájok, és kemény beszédekkel szóla nékik, mondván: Honnan jöttetek? Azok pedig mondának: Kanaán földérõl jöttünk eleséget venni.
\par 8 Megismeré pedig József az õ bátyjai, de azok nem ismerék meg õt.
\par 9 És megemlékezék József az álmokról, a melyeket azok felõl álmodott vala. És monda nékik: Kémek vagytok ti, kik azért jöttetek, hogy az ország védetlen részeit meglássátok.
\par 10 És mondának néki: Nem uram, hanem eleséget venni jöttek a te szolgáid.
\par 11 Mi mindnyájan egy ember fiai vagyunk: igaz emberek vagyunk. Soha nem voltak kémek a te szolgáid.
\par 12 Ismét monda: Nem úgy van, hanem azért jöttetek, hogy az ország védetlen részeit meglássátok.
\par 13 Amazok mondának: Mi, te szolgáid, tizenketten vagyunk testvérek, egy embernek fiai, Kanaán földén; és ímé legkisebbikünk most atyánknál van, egyikünk pedig nincsen meg.
\par 14 József pedig monda nékik: Úgy van a mint néktek mondám: kémek vagytok.
\par 15 Ezzel lesztek próbára téve: Úgy éljen a Faraó, hogy ki nem mentek innen, míg ide nem jõ a ti legkisebbik atyátokfia.
\par 16 Küldjetek el közûletek egyet, hogy hozza ide testvérteket; ti pedig fogva lesztek. Igy lesz próbára téve beszédetek, hogy igaz járatban vagytok-é vagy nem? Úgy éljen a Faraó, hogy ti kémek vagytok.
\par 17 Annakokáért fogságban tartá õket harmadnapig.
\par 18 Harmadnap pedig monda nékik József: Ezt cselekedjétek, hogy éljetek; az Istent én is félem.
\par 19 Ha igaz emberek vagytok, maradjon fogva egyik testvéreket a ti tömlöczötökben, ti pedig menjetek el, vigyetek gabonát házaitok szükségére.
\par 20 Legkisebbik atyátokfiát pedig hozzátok én hozzám, akkor igazolva lesznek beszédeitek és nem haltok meg. És aképen cselekedének.
\par 21 És mondának egymásnak: Bizony vétkeztünk mi a mi atyánkfia ellen, a kinek láttuk lelki szorongását, mikor nékünk könyörög vala, de hallgattunk reá; azért következett reánk ez a nyomorúság.
\par 22 Rúben pedig felele nékik, mondván: Avagy nem mondtam-é néktek, hogy ne vétkezzetek a gyermek ellen, de ti nem hallgattatok reám. És ímé vérét is keresik rajtunk!
\par 23 És nem tudják vala õk, hogy József érti õket, mert tolmács vala közöttük.
\par 24 És elfordula tõlük és síra: Azután hozzájok fordula és szóla nékik, és visszatartá közûlök Simeont, és szemök láttára megkötözteté õt.
\par 25 És parancsola József, hogy töltsék meg edényeiket gabonával, és tegyék vissza pénzöket mindeniknek az õ zsákjába, és hogy adjanak nékik enni valót az útra; és így cselekedének velök.
\par 26 És felveték gabonájokat szamaraikra és elmenének onnan.
\par 27 És egyik kioldá zsákját, hogy abrakot adjon a szamarának a szálláson és meglátá az õ pénzét, hogy ímé zsákja szájában van az.
\par 28 És monda az õ atyjafiainak: Visszatették az én pénzemet, és ímé a zsákomban van. Akkor megrettene szívök, és remegve mondának egymásnak: Micsoda ez, a mit Isten cselekedett velünk?
\par 29 És eljutának atyjokhoz Jákóbhoz Kanaán földére és mindazt elbeszélék néki a mi velök történt vala, mondván:
\par 30 Az a férfiú, annak a földnek ura, keményen szóla nékünk, és úgy bánt velünk, mintha az országot kémleltük volna.
\par 31 És mondánk néki: Igaz emberek vagyunk, soha nem voltunk mi kémek.
\par 32 Tizenketten vagyunk atyafiak, a mi atyánknak fiai, egyikünk nincs meg, legkisebbikünk pedig most atyánkkal van Kanaán földén.
\par 33 És monda nékünk az a férfiú, annak a földnek ura: Errõl ismerem meg, hogy igaz emberek vagytok: egyik atyátokfiát hagyjátok nálam, a házaitok szükségére valót vigyétek, és menjetek el;
\par 34 És hozzátok hozzám a ti legkisebbik atyátokfiát, akkor megtudom, hogy nem vagytok kémek, hanem igaz emberek; akkor visszaadom néktek a ti atyátokfiát, és ebben az országban kereskedhettek.
\par 35 És lõn a mint zsákjaikat kiüresíték, ímé az õ csomó pénze mindeniknek zsákjában vala. A mint látták az õ csomó pénzeiket õk és az õ atyjok, megfélemlének.
\par 36 És monda nékik az õ atyjok Jákób: Megfosztotok engem gyermekeimtõl; József nincsen, Simeon sincsen. Benjamint is elviszitek? mindez engem ér!
\par 37 Akkor szóla Rúben az õ atyjának, mondván: az én két fiamat öld meg, ha meg nem hozom õt néked. Bízd az én kezemre õt, és én visszahozom néked.
\par 38 Az pedig monda: Nem megy le oda az én fiam ti veletek, mert az õ bátyja megholt és õ maga maradt meg; ha veszedelem érné õt az úton, a melyen elmentek, õsz fejemet búba borítva bocsátanátok le a koporsóba.

\chapter{43}

\par 1 Az éhség pedig elhatalmazott vala az országon.
\par 2 És lõn mikor fogytán vala az eleség, melyet Égyiptomból hoztak vala, monda nékik az õ atyjok: Menjetek el ismét, vegyetek nékünk egy kevés eleséget.
\par 3 És felele néki Júda, mondván: Erõsen fogadkozék az a férfiú, mondván: Színem elé ne kerûljetek, ha veletek nem lesz a ti atyátokfia.
\par 4 Ha azért elbocsátod velünk a mi öcsénket, elmegyünk, és veszünk néked eleséget;
\par 5 Ha pedig el nem bocsátod, nem megyünk, mert az a férfiú megmondá nékünk: Színem elé ne kerüljetek, ha a ti atyátokfia veletek nem lesz.
\par 6 És monda Izráel: Miért cselekedtetek gonoszul velem, hogy megmondtátok annak a férfiúnak, hogy még van egy öcsétek?
\par 7 Azok pedig mondának: Nagyon tudakozódék az a férfiú felõlünk és nemzetségünk felõl, mondván: Él-é még atyátok? van-é még testvéretek? És mi kérdéseihez képest feleltünk néki. Avagy tudhattuk-é mi, hogy azt mondja: Hozzátok ide a ti atyátokfiát?
\par 8 És monda Júda Izráelnek, az õ atyjának: Bocsásd el azt a fiút én velem, és mi azonnal felkelünk és elmegyünk, hogy éljünk és meg ne haljunk se mi, se te, se a mi gyermekeink.
\par 9 Én leszek kezes érette, az én kezembõl kérd elõ. Ha vissza nem hozom õt hozzád, és elõdbe nem állítom õt, mind éltig bûnös legyek elõtted.
\par 10 Bizony ha nem késlekedünk vala, ez ideig már kétszer is megjöhettünk volna.
\par 11 És monda nékik Izráel az õ atyjok: Ha csakugyan így kell lenni, akkor ezt cselekedjétek: Vegyetek e föld válogatott gyümölcseibõl a ti edényeitekbe, és vigyetek ajándékot annak a férfiúnak; egy kevés balzsamot, egy kevés mézet, fûszerszámokat, mirhát, diót, mandulát.
\par 12 Pénzt pedig két annyit vigyetek magatokkal, sõt a mely pénzt meghoztatok a ti zsákjaitok szájában, azt is vigyétek vissza magatokkal, talán tévedés ez.
\par 13 Öcséteket is vegyétek, keljetek fel és menjetek vissza ahhoz a férfiúhoz.
\par 14 A mindenható Isten pedig engedje, hogy kedvet találjatok annál a férfiúnál, és bocsássa vissza ti veletek a másik atyátokfiát, és Benjamint. Én pedig ha megfosztottnak kell lennem, hadd legyek megfosztva.
\par 15 Vevék azért a férfiak azt az ajándékot, és vevének kétannyi pénzt az õ kezökbe, és Benjámint, és felkelének és alámenének Égyiptomba, és megállának József elõtt.
\par 16 A mint meglátá József õ velök Benjámint, monda az õ háza gondviselõjének: Vidd be azokat az embereket a házba, és ölj barmot, s készítsd el, mert velem ebédelnek ez emberek ma délben.
\par 17 És az a férfiú aképen cselekedék, amint József parancsolta vala, és bevivé az a férfiú azokat az embereket a József házába.
\par 18 És megfélemlének azok az emberek, a miért bevivék õket a József házába, és mondának: A pénzért hozattatánk ide be, mely elõször a mi zsákjainkba tétetett volt, hogy reánk rohanjon, megtámadjon és minket rabszolgákká tegyen a mi szamarainkkal együtt.
\par 19 És járulának József házának gondviselõjéhez, és szólának néki a ház ajtajában.
\par 20 És mondának: Kérünk uram! Ennekelõtte alájöttünk vala eleséget venni.
\par 21 És lõn mikor éjjeli szállásra jutánk és kioldjuk vala a mi zsákjainkat: ímé mindenikünknek pénze az õ zsákjának szájában vala, tulajdon pénzünk teljes mértéke szerint; és visszahoztuk azt magunkkal.
\par 22 De más pénzt is hoztunk le magunkkal eleséget venni; nem tudjuk ki tette a mi pénzünket zsákjainkba.
\par 23 És monda: Legyetek békén, ne féljetek; a ti Istentek és a ti atyátok Istene adta néktek azt a kincset zsákjaitokba; pénzetek az én kezemhez jutott. És kihozá hozzájok Simeont.
\par 24 Bevivé azután a férfiú azokat az embereket a József házába, és vizet hozata, és megmosák lábaikat, és abrakot is ada az õ szamaraiknak.
\par 25 Õk pedig elkészíték az ajándékot mire József délben megjöve; mert megértették vala, hogy ott ebédelnek.
\par 26 Mikor József haza jöve, bevivék néki az ajándékot, mely kezökben vala, a házba, és leborulának elõtte a földig.
\par 27 És kérdezõsködék egészségök felõl, s monda: Egészségben van-é a ti vén atyátok, a kirõl nékem szóltatok? él-é még?
\par 28 Azok pedig mondának: Egészségben van a te szolgád, a mi atyánk még él. És meghajták magokat és leborulának.
\par 29 És felemelé szemeit és látá Benjámint az õ atyjafiát, az õ anyjának fiát, és monda: Ez-é a ti legkisebbik atyátokfia, a ki felõl nékem szóltatok vala? És ismét monda: Az Isten légyen hozzád kegyelmes, fiam!
\par 30 Akkor elsiete onnan József, mert felgerjede szíve az õ öcscse iránt, és erõlteti vala a sírás, beméne azért szobájába, és ott síra.
\par 31 Azután megmosá orczáját, és kiméne, és megtartóztatá magát, és monda: Hozzatok enni valót.
\par 32 És elhozák néki külön, azoknak is külön, és az Égyiptombelieknek is, kik vele esznek vala, külön: Mert nem ehettek az Égyiptombeliek együtt a héberekkel, mert utálatos az az Égyiptombeliek elõtt.
\par 33 Leûlének azért õ elõtte, az elsõszülött az õ elsõszülöttsége szerint, és a fiatalabb az õ fiatalsága szerint. És az emberek álmélkodva nézének egymásra.
\par 34 Õ pedig részt juttata azoknak maga elõl és a Benjámin része ötszörte nagyobb vala mindnyájok részénél. És ivának és megittasodának õ nála.

\chapter{44}

\par 1 Azután parancsola József az õ háza gondviselõjének, mondván: Töltsd meg ez embereknek zsákjait eleséggel a mennyit elvihetnek; és mindeniknek pénzét tedd zsákja szájába.
\par 2 Az én poharamat pedig, az ezüst poharat, tedd a legkisebbik zsákjának szájába gabonájának árával együtt. És az József beszéde szerint cselekedék, a mint beszélt vala.
\par 3 Reggel virradatkor, elbocsáttatának azok az emberek, szamaraikkal együtt.
\par 4 Kimenének a városból, de nem messze haladhatának, a mikor monda József az õ háza gondviselõjének: Kelj fel, siess utánuk azoknak az embereknek és ha eléred õket, mondd nékik: Miért fizettetek gonoszszal a jó helyébe?
\par 5 Avagy nem abból iszik-é az én uram? és abból szokott jövendölni! Gonoszul cselekedtétek, a mit cselekedtetek!
\par 6 És utóléré õket, és ilyen szavakkal szóla nékik.
\par 7 Azok pedig mondának néki: Miért szól az én uram ilyen szavakkal? Távol legyen szolgáidtól, hogy ilyen dolgot cselekedjenek.
\par 8 Ímé a pénzt, melyet zsákjaink szájában találtunk vala, meghoztuk néked Kanaán földérõl; hogy loptunk volna hát urad házából ezüstöt vagy aranyat?
\par 9 A kinél megtaláltatik a te szolgáid közûl, haljon meg az; sõt mi is szolgái leszünk uramnak.
\par 10 És monda: Mostan is legyen beszédetek szerint: a kinél megtaláltatik, az légyen nékem szolgám, ti pedig mentek legyetek.
\par 11 És sietének és leraká kiki az õ zsákját a földre, és kioldá kiki az õ zsákját.
\par 12 És keresgéle; a legnagyobbikon kezdé s a legkissebbiken végezé, és megtalálá a poharat a Benjámin zsákjában.
\par 13 Azok pedig meghasogaták ruhájokat, és kiki megterhelé a maga szamarát, és visszatérének a városba.
\par 14 És beméne Júda és az õ atyjafiai a József házába, és ki még ott vala és földre esének elõtte.
\par 15 És monda nékik József: Mi dolog ez a mit cselekedtetek? Avagy nem tudjátok-é hogy az ilyen magamféle ember jövendölni tud?
\par 16 És monda Júda: Mit mondhatunk az én uramnak? Mit szóljunk és mivel igazoljuk magunkat? Az Isten büntetése utólérte szolgáidat. Ímé mi az én uram szolgái vagyunk, mond mi, mind az, a kinek kezében a pohár találtatott.
\par 17 Õ pedig monda: Távol legyen tõlem, hogy azt cselekedjem: az a kinek kezében találtatott a pohár, az legyen nékem szolgám, ti pedig békességgel menjetek el a ti atyátokhoz.
\par 18 De Júda hozzá járula és monda: Kérlek, uram, hadd szólhasson egy szót uram fülébe a te szolgád, és ne gerjedjen fel haragod a te szolgád ellen; mert hasonló vagy te a Faraóhoz.
\par 19 Az én uram kérdezte vala az õ szolgáit, mondván: Van-é atyátok, vagy testvéretek?
\par 20 Akkor mi azt felelénk az én uramnak: Van egy vén atyánk, és egy kis gyermek, a ki az õ vénségében lett; és ennek bátyja megholt, és csak õ maga maradt az õ anyjától, és az õ atyja szereti õt.
\par 21 És azt mondád a te szolgáidnak: Hozzátok én hozzám azt, hogy szemeimet reá vessem.
\par 22 És mondánk az én uramnak: Nem hagyhatja el az a fiú az õ atyját; mert ha elhagyja atyját, meghal az.
\par 23 És ezt mondád a te szolgáidnak: Ha a ti legkisebbik atyátokfia el nem jõ veletek, színem elé se kerûljetek többé.
\par 24 Mikor azért felmenénk a te szolgádhoz, az én atyámhoz és tudtul adtuk vala néki az én uramnak beszédét;
\par 25 És monda a mi atyánk: Menjetek vissza, és vegyetek nékünk egy kevés eleséget.
\par 26 És mondánk: Nem mehetünk le; ha a mi legkisebbik atyánkfia velünk lesz, akkor lemegyünk; mert nem mehetünk ama férfiú színe elé, ha a mi legkisebbik atyánkfia velünk nem lesz.
\par 27 És monda a te szolgád, az én atyám, nékünk: Ti tudjátok hogy az én feleségem nékem csak két fiat szûlt.
\par 28 Az egyik kiméne tõlem, és azt mondom vala: bizonyára fenevad szaggatta széllyel és attól fogva nem láttam õt.
\par 29 Ha ezt is elviszitek szemeim elõl, s veszedelem találja érni, akkor az én õsz fejemet keserûségtõl borítva bocsátjátok alá a koporsóba.
\par 30 Ha tehát most visszamenéndek a te szolgádhoz, az én atyámhoz, és e fiú nem lesz velünk, mivelhogy annak lelke ennek lelkéhez van nõve,
\par 31 Ha meglátja, hogy nincs meg a gyermek, meghal, s akkor a te szolgáid, a te szolgádnak, a mi atyánknak õsz fejét búba borítva bocsátják alá a koporsóba.
\par 32 Mivel a te szolgád e fiúért az õ atyjánál kezes lett, mondván: Ha vissza nem hozom õt hozzád, mind éltig bûnös legyek az én atyám elõtt.
\par 33 Hadd maradjon azért e gyermek helyébe a te szolgád, az én uramnak szolgájáúl; e gyermek pedig menjen fel az õ bátyjaival.
\par 34 Mert mimódon mehetnék én fel atyámhoz, ha e gyermek velem nem lenne, a nélkül, hogy ne lássam a nyomorúságot, mely atyámat érné?

\chapter{45}

\par 1 És nem tartóztathatá magát tovább József mindazok elõtt, kik körûlötte állának és felkiálta: Vezessetek ki minden embert mellõlem. És nem marada senki nála, mikor megismerteté magát József az õ atyjafiaival.
\par 2 És hangos sírásra fakada, úgy hogy meghallák az Égyiptombeliek, és meghallá a Faraó háznépe is.
\par 3 És monda József az õ atyjafiainak: Én vagyok József, él-e még az én atyám? És nem felelhetének néki az õ atyjafiai, mert megrettentek vala tõle.
\par 4 Monda azért József az õ atyjafiainak: Jõjjetek közelebb hozzám! És közelebb menének. Akkor monda: Én vagyok József, a ti testvéretek, kit eladtatok vala Égyiptomba.
\par 5 És most ne bánkódjatok, és ne bosszankodjatok azon, hogy engem ide eladtatok; mert a ti megmaradástokért küldött el engem Isten ti elõttetek.
\par 6 Mert immár két esztendeje, hogy éhség van e földön, de még öt esztendõ van hátra, melyben sem szántás, sem aratás nem lesz.
\par 7 Az Isten küldött el engem ti elõttetek, hogy míveljem a ti megmaradásotokat e földön, és hogy megmenthesselek titeket nagy szabadítással.
\par 8 Nem ti küldöttetek azért engem ide, hanem az Isten, ki engem a Faraó atyjává tett, és egész házának urává, és Égyiptom egész földének fejedelmévé.
\par 9 Siessetek és menjetek fel atyámhoz és mondjátok néki: Ezt mondja a te fiad József: Az Isten engem egész Égyiptomnak urává tett, jõjj le én hozzám, ne késsél.
\par 10 És Gósen földén lakozol, és én hozzám közel leszesz, mind te, mind fiaid, mind fiaidnak fiai, juhaid, barmaid, és mindened, a mid van.
\par 11 És eltartalak ott téged, mert még öt esztendei éhség lesz, hogy tönkre ne juss te, és házadnépe, és semmi, a mid van.
\par 12 És ímé a ti szemeitek látják, és az atyámfiának Benjáminnak szemei, hogy az én szám szól hozzátok.
\par 13 Beszéljétek hát el atyámnak minden én dicsõségemet Égyiptomban, és mindazt, a mit láttatok. És siessetek s hozzátok ide az én atyámat.
\par 14 És nyakába borúla az õ öccsének Benjáminnak és síra; Benjámin is síra az õ nyakán.
\par 15 És megcsókolá mind az õ testvéreit és síra õ rajtok; azután beszédbe ereszkedének vele az õ testvérei.
\par 16 És eljuta a hír a Faraó házába is, mondván: Eljöttek a József atyjafiai; és tetszék e dolog mind a Faraónak, mind az õ szolgáinak.
\par 17 És monda a Faraó Józsefnek: Mondd meg a te atyádfiainak: Ezt cselekedjétek: Terheljétek meg a ti barmaitokat, és eredjetek, menjetek el Kanaán földére;
\par 18 És vegyétek fel atyátokat és házatok népét, és jõjjetek hozzám; és én néktek adom Égyiptom földének javát, hogy éljétek e földnek zsírját.
\par 19 Ez is parancsolatúl legyen néked: Ezt míveljétek, vigyetek magatokkal Égyiptom földérõl szekereket gyermekeitek és feleségeitek számára, és vegyétek fel atyátokat és jõjjetek.
\par 20 A ti házi eszközeitekre pedig ne tekintsetek sóhajtva; mert egész Égyiptom földének a legjava a tiétek.
\par 21 És aképen cselekedének Izráel fiai; és ada nékik József szekereket a Faraó parancsolatja szerint; és enni valót is ada nékik az útra.
\par 22 Valamennyien valának, mindegyiknek ada egy-egy öltözõ ruhát: Benjáminnak pedig ada háromszáz ezüst pénzt és öt öltözõ ruhát.
\par 23 Atyjának pedig külde ilyenképen: tíz szamarat égyiptomi javakkal terhelve, és tíz nõstény szamarat gabonával, kenyérrel és egyéb élelemmel terhelve, az õ atyjának az útra.
\par 24 és elbocsátá az õ testvéreit, és elmenének, és monda nékik: Ne háborogjatok az úton.
\par 25 Feljövének azért Égyiptomból: és eljutának Kanaán földére az õ atyjokhoz, Jákóbhoz.
\par 26 És mikor tudtára adák, mondván: József még él, és hogy uralkodik egész Égyiptom földén; az õ szíve elalélt, mert nem hisz vala nékik.
\par 27 Elbeszélék azért néki József minden beszédét, a melyeket velök beszélt vala, és látá a szekereket is, a melyeket József küldött vala, hogy õt elvigyék; akkor fölélede az õ atyjoknak, Jákóbnak lelke.
\par 28 És monda Izráel: Elég nékem, hogy József az én még fiam él: lemegyek hát, hogy meglássam õt minekelõtte meghalok.

\chapter{46}

\par 1 Elindula azért Izráel minden hozzá tartozóival és méne Beérsebába; és áldozék áldozatokat az õ atyja Izsák Istenének.
\par 2 És szóla Isten Izráelnek éjjeli látomásban, és monda: Jákób, Jákób. Õ pedig monda: Ímhol vagyok.
\par 3 És monda: Én vagyok az Isten, a te atyádnak Istene: Ne félj lemenni Égyiptomba: mert nagy néppé teszlek ott téged.
\par 4 Én lemegyek veled Égyiptomba, és én bizonynyal fel is hozlak; és József fogja bé a te szemeidet.
\par 5 Felkerekedék azért Jákób Beérsebából, és elvivék Izráel fiai Jákóbot az õ atyjokat, és gyermekeiket és feleségeiket a szekereken, melyeket a Faraó küldött vala érette.
\par 6 És elvivék nyájaikat és szerzeményeiket, melyeket Kanaán földén szereztek vala és jutának Égyiptomba Jákób és minden vele levõ magva.
\par 7 Az õ fiait és fiainak fiait, az õ leányait, és fiainak leányait és minden vele levõ magvát elvivé magával Égyiptomba.
\par 8 Ezek pedig az Izráel fiainak nevei, kik bementek Égyiptomba: Jákób és az õ fiai: Jákóbnak elsõszülötte Rúben.
\par 9 Rúben fiai pedig: Khánokh, Pallu, Kheczrón, Khármi.
\par 10 Simeon fiai pedig: Jemúel, Jámin, Ohad, Jákhin, Czóhár és Saul a kanaáni asszonynak fia.
\par 11 Lévi fiai pedig: Gerson, Kehát, Mérári.
\par 12 Júda fiai pedig: Hér, Ónán, Séla, Perecz, Zerákh; de megholt vala Hér és Ónán a Kanaán földén.  Perecznek fiai pedig: Kheczrón és Khámul.
\par 13 Izsakhár fiai pedig: Thóla, Puvah, Jób és Simrón.
\par 14 Zebulon fiai pedig: Szered, Élon, Jákhleél.
\par 15 Ezek Lea fiai, a kiket szûlt vala Jákóbnak Mésopotámiában, Dínával az õ leányával együtt. Fiainak és leányainak összes száma: harminczhárom lélek.
\par 16 Gád fiai pedig: Czifjon, Khaggi, Súni, Eczbón, Héri, Aródi és Areéli.
\par 17 Áser fiai pedig: Jimnáh, Jisváh, Jisvi, Beriha és Szerakh az õ húgok; Berihának fiai pedig: Khéber és Málkhiel.
\par 18 Ezek Zilpa fiai, kit Lábán adott vala Leának az õ leányának; és õ szûlé ezt a tizenhat lelket Jákóbnak.
\par 19 Rákhelnek, Jákób feleségének fiai: József és Benjámin.
\par 20 És születének Józsefnek Égyiptom földén Manasse és Efráim, a kiket Asznáth, Potiferának, On papjának leánya szûlt néki.
\par 21 Benjámin fiai pedig: Bela, Bekher, Asbél, Géra, Nahamán, Ekhi, Rós, Muppim, Khuppim és Ard.
\par 22 Ezek Rákhel fiai, kik születtek Jákóbnak, mindössze tizennégy lélek.
\par 23 Dán fia pedig: Khusim.
\par 24 Nafthali fiai pedig: Jakhczeél, Gúni, Jéczer és Sillém.
\par 25 Ezek Bilha fiai, kit adott vala Lábán Rákhelnek az õ leányának; és ezeket szûlte Jákóbnak, mindösszesen hét lelket.
\par 26 Valamennyi Jákóbbal Égyiptomba jött lélek, kik az õ ágyékából származtak, a Jákób fiainak feleségeit nem számítva, mindössze hatvanhat lélek.
\par 27 József fiai pedig, kik Égyiptomban születtek, két lélek. Jákób egész házanépe, mely Égyiptomba ment vala, hetven lélek.
\par 28 Júdát pedig elküldé maga elõtt Józsefhez, hogy útmutatója legyen Gósen felé. És eljutának Gósen földére.
\par 29 És befogadta József az õ szekerébe, és eleibe méne Izráelnek az õ atyjának Gósenbe; s a mint maga elõtt látá, nyakába borula, és síra az õ nyakán sok ideig.
\par 30 És monda Izráel Józsefnek: Immár örömest meghalok, minekutána láttam a te orczádat, hogy még élsz.
\par 31 József pedig monda az õ testvéreinek, és az õ atyja házanépének: Felmegyek és tudtára adom a Faraónak, és ezt mondom néki: Az én testvéreim és atyám házanépe, kik Kanaán földén valának, eljöttek én hozzám.
\par 32 Azok az emberek pedig juhpásztorok, mert baromtartó nép valának, és juhaikat, barmaikat, és mindenöket valamijök van, elhozták.
\par 33 S ha majd a Faraó hivat titeket és azt kérdi: Mi a ti életmódotok?
\par 34 Azt mondjátok: Baromtartó emberek voltak a te szolgáid gyermekségünktõl fogva mind ez ideig, mi is, mint a mi atyáink, hogy lakhassatok Gósen földén; mert minden juhpásztor utálatos az Égyiptombeliek elõtt.

\chapter{47}

\par 1 Elméne azért József és tudtára adá a Faraónak, és monda: Az én atyám és atyámfiai, juhaikkal, barmaikkal és mindenökkel valamijök volt, ide jöttek Kanaán földérõl; és most Gósen földén vannak.
\par 2 És võn ötöt az õ testvérei közûl, és állítá õket a Faraó elé.
\par 3 És monda a Faraó a József testvéreinek: Mi a ti életmódotok? És mondának a Faraónak: Juhpásztorok a te szolgáid, mi is, mint a mi atyáink.
\par 4 És mondának a Faraónak: Azért jöttünk, hogy e földön tartózkodjunk, mert a te szolgáid barmainak nincs legelõje, mivelhogy elhatalmazott az éhség Kanaán földén: hadd lakjanak azért most a te szolgáid Gósen földén.
\par 5 És szóla a Faraó Józsefnek, mondván: A te atyád és a te atyád fiai jöttek te hozzád.
\par 6 Égyiptom földe ímé elõtted van; e föld legjobb részében telepítsd le a te atyádat és atyádfiait, lakozzanak a Gósen földén: ha pedig tudod, hogy vannak közöttök arra termett emberek, tedd azokat az én barmaim gondviselõivé.
\par 7 Bevivé József Jákóbot is az õ atyját, és állítá õt a Faraó elé. És köszönté Jákób a Faraót.
\par 8 És monda a Faraó Jákóbnak: Hány esztendõs vagy?
\par 9 Monda pedig Jákób a Faraónak: Az én bujdosásom esztendeinek napja száz harmincz esztendõ; kevesek és nyomorúságosak voltak az én életem esztendeinek napjai, és nem érték el az én atyáim élete esztendeinek napjait, a meddig õk bujdostak.
\par 10 És megáldá Jákób a Faraót, és kiméne a Faraó elõl.
\par 11 Megtelepíté tehát József az õ atyját és atyjafiait, és ada nékik birtokot Égyiptom földén, annak a földnek legjobb részében a Rameszesz földén; a mint megparancsolta vala a Faraó.
\par 12 És ellátja vala József az õ atyját és atyjafiait, és az õ atyjának egész házanépét kenyérrel, gyermekeik számához képest.
\par 13 És kenyér nem vala az egész föld kerekségén, mert igen nagy vala az éhség, és elalélt vala Égyiptom földe, és a Kanaán földe az éhség miatt.
\par 14 József pedig mind összeszedé a pénzt, a mi találtatik vala Égyiptomnak és Kanaánnak földén a gabonáért, a melyet azok vesznek vala; és bévivé József a pénzt a Faraó házába.
\par 15 És mikor elfogyott a pénz Égyiptom földérõl is, Kanaán földérõl is, egész Égyiptom Józsefhez méne, mondván: Adj nékünk kenyeret, miért haljunk meg szemed láttára, azért hogy nincs pénz?
\par 16 És monda József: Hozzátok ide barmaitokat, és adok néktek a ti barmaitokért, ha nincs pénz.
\par 17 És elvivék barmaikat Józsefhez, és ada nékik József kenyeret lovakért, juhokért, ökrökért és szamarakért: és eltartá õket abban az esztendõben kenyérrel az õ barmaik összeségéért.
\par 18 Mikor pedig elmúlék az esztendõ, menének hozzá a második esztendõben, és mondanának néki: Nem titkolhatjuk el uramtól, hogy bizony elfogyott a pénz, és a barom-nyájak mind uramnál vannak, a mint látja az én uram semmi sem maradt, csak testünk és földünk.
\par 19 Miért veszszünk el szemed láttára mind magunk, mind földünk? végy meg minket és földünket kenyéren, és mi és a mi földünk szolgái leszünk a Faraónak; csak adj magot, hogy éljünk s ne haljunk meg, és a föld ne pusztuljon el.
\par 20 Megvevé azért József egész Égyiptom földét a Faraó részére, mert az Égyiptombeliek mind eladák az õ földjöket, mivelhogy erõt vett vala rajtok az éhség. És a föld a Faraóé lõn.
\par 21 A népet pedig egyik városból a másikba telepíté Égyiptom egyik határszélétõl a másik széléig.
\par 22 Csak a papok földét nem vevé meg, mert a papoknak szabott részök vala a Faraótól és abból a szabott részbõl élnek vala, a mit nékik a Faraó ád vala; annak okáért, nem adák el az õ földjöket.
\par 23 És monda József a népnek: Ímé megvettelek titeket a mai napon, és a ti földeteket a Faraónak. Ímhol számotokra a mag, vessétek be a földet.
\par 24 És takaráskor adjatok a Faraónak egy ötödrészt; négy rész pedig legyen a tiétek, a mezõ bevetésére és éléstekre, mind magatoknak, mind házatok népének, és gyermekeiteknek eledelül.
\par 25 És mondának: Életünket megtartottad; hadd találjunk kegyelmet uram szemei elõtt, és szolgái leszünk a Faraónak.
\par 26 És törvénynyé tevé azt József mind e mai napig Égyiptom földén, hogy a Faraóé az ötödrész, csak a papok földe, egyedûl az nem volt a Faraóé.
\par 27 Lakozék azért Izráel Égyiptom földében a Gósen földén, és ott megöröködének, s megszaporodának és megsokasodának felette igen.
\par 28 Jákób pedig tizenhét esztendeig él vala Égyiptom földén, és Jákób élete esztendeinek napjai száz negyvenhét esztendõ.
\par 29 És elközelgetének Izráel halálának napjai, és hívatá az õ fiát Józsefet, s monda néki: Ha én te elõtted kedves vagyok, kérlek tedd a kezedet tomporom alá, és légy hozzám szeretettel és hûséggel: Kérlek ne temess el engem Égyiptomban.
\par 30 Midõn elaluszom az én atyáimmal, vígy ki engem Égyiptomból és temess el az õ sírjokba. És monda: Én a te beszéded szerint cselekszem.
\par 31 És monda: Esküdjél meg nékem! és megesküvék néki. És leborula Izráel az ágy fejére.

\chapter{48}

\par 1 És lõn ezek után, megmondák Józsefnek: Ímé a te atyád beteg; és elvivé magával az õ két fiát Manassét és Efraimot.
\par 2 És tudtára adák Jákóbnak, mondván: Ímé a te fiad József hozzád jõ; és összeszedé erejét Izráel, s felüle az ágyon.
\par 3 És monda Jákób Józsefnek: A mindenható Isten megjelenék nékem Lúzban, a Kanaán földén, és megálda engem.
\par 4 És monda nékem: Ímé én megszaporítlak és megsokasítlak és népek sokaságává teszlek téged, s ezt a földet te utánnad a te magodnak  adom örökül birtokul.
\par 5 Most tehát a te két fiad, a kik néked Égyiptom földén annakelõtte születtek, hogy én hozzád jöttem vala Égyiptomba, az enyéim; Efraim és Manasse, akár csak Rúben és Simeon, az enyéim lesznek.
\par 6 Ama szülötteid pedig, kiket õ utánok nemzettél, tiéid lésznek, és az õ bátyjaik nevérõl neveztessenek az õ örökségökben.
\par 7 Mert mikor Mésopotámiából jövék, meghala mellettem Rákhel Kanaán földén az úton, mikor még egy dûlõföldre valék az Efratától, és eltemetém õt ott az Efratába (azaz Bethlehembe) vezetõ úton.
\par 8 És meglátá Izráel a József fiait és monda: Kicsodák ezek?
\par 9 József pedig monda az õ atyjának: Az én fiaim, kiket Isten itt adott nékem. És monda: Hozd ide õket hozzám, hadd áldjam meg.
\par 10 Mert Izráelnek szemei meghomályosodának a vénség miatt, és nem láthat vala. Közel vivé tehát õket hozzá, õ pedig megcsókolgatá és megölelgeté õket.
\par 11 és monda Izráel Józsefnek: Nem gondoltam, hogy orczádat megláthassam, és íme az Isten megengedte látnom magodat is.
\par 12 Akkor kivevé József azokat az õ atyjának térdei közül, és leborula arczczal a földre.
\par 13 És fogá József mindkettejöket, Efraimot jobbkezével Izráel balkeze felõl; Manassét pedig balkezével Izráelnek jobbkeze felõl és közel vivé õket hozzá.
\par 14 Izráel pedig kinyújtá az õ jobbkezét és rátevé Efraim fejére, pedig õ a kisebbik vala, az õ balkezét pedig Manasse fejére. Tudva tevé így kezeit, mert az elsõszülött Manasse vala.
\par 15 És megáldá Józsefet s monda: Az Isten, a kinek elõtte jártak az én atyáim Ábrahám és Izsák; az Isten a ki gondomat viselte, a mióta vagyok, mind e napig:
\par 16 Amaz Angyal, ki megszabadított engem minden gonosztól, áldja meg e gyermekeket, és viseljék az én nevemet és az én atyáimnak Ábrahámnak és Izsáknak nevét, és mint a halak szaporodjanak e földön.
\par 17 Látván pedig József, hogy az õ atyja jobbkezét Efraim fejére tevé, nem tetszék néki, és megfogá atyja kezét, hogy Efraim fejérõl Manasse fejére tegye át.
\par 18 És monda József az õ atyjának: Nem úgy atyám; mert ez az elsõszülött, ennek fejére tedd jobb kezedet.
\par 19 Nem akará pedig az atyja és monda: Tudom fiam, tudom, õ is néppé lesz, õ is megnevekedik; de az õ öccse nálánál inkább megnevekedik, és az õ magja népek sokaságává lesz.
\par 20 És megáldá õket azon a napon, mondván: Ha áld, téged említsen Izráel, mondván: Az Isten téged olyanná tégyen mint Efraimot s Manassét. És Efraimot eleibe tevé Manassénak.
\par 21 És mondá Izráel Józsefnek: Ímé én meghalok, de az Isten veletek lesz és vissza visz titeket a ti atyáitok földére.
\par 22 Én pedig adok néked egy osztályrészt a te atyádfiainak része felett, melyet az Emoreustól vettem fegyveremmel és kézívemmel.

\chapter{49}

\par 1 És szólítá Jákób az õ fiait, és monda: Gyûljetek egybe, hadd jelentsem meg néktek, a mi rátok következik a messze jövõben.
\par 2 Gyûljetek össze s hallgassatok Jákóbnak fiai! hallgassatok Izráelre, a ti atyátokra.
\par 3 Rúben, te elsõszülöttem, erõm, tehetségem zsengéje, elsõ a méltóságban, elsõ a hatalomban.
\par 4 Állhatatlan, mint a víz, nem leszesz elsõ, mivel atyád ágyába léptél fel: akkor megfertõztetted! Nyoszolyámba lépett õ.
\par 5 Simeon és Lévi atyafiak, erõszak eszközei az õ fegyverök.
\par 6 Tanácsukban ne légyen részes lelkem, gyûlésükkel ne egyesûljön dicsõségem, mert haragjokban férfit öltek, s kedvök telve inát szegték az ökörnek.
\par 7 Átkozott haragjok, mert erõszakos, és dühök, mivel kegyetlen; eloszlatom õket Jákóbban, és elszélesztem Izráelben.
\par 8 Júda! téged magasztalnak atyádfiai, kezed ellenségeidnek nyakán lesz, s meghajolnak elõtted atyáidnak fiai.
\par 9 Oroszlánkölyök Júda; zsákmányt ejtvén, felmentél, fiam! Lehevert, lenyúgodott, mint a hím oroszlán, és mint nõstény oroszlán; ki veri õt fel?
\par 10 Nem múlik el Júdától a fejedelmi bot, sem a vezéri pálcza térdei közûl; míg eljõ Siló, és a népek néki engednek.
\par 11 Szõlõtõhöz köti szamarát, és nemes venyigéhez szamara vemhét, ruháját borban mossa, felöltõjét a szõlõ vérében.
\par 12 Bortól veresek szemei, tejtõl fehérek fogai.
\par 13 Zebulon a tenger partjáig lakozik, azaz a hajók kikötõjéig s határának széle Czídonig ér.
\par 14 Izsakhár erõs csontú szamár, a karámok közt heverész.
\par 15 S látja, hogy jó a nyugalom és hogy a föld mily kies: teher alá hajtja hátát, s robotoló szolgává lesz.
\par 16 Dán ítéli az õ népét, mint Izráel akármelyik nemzetsége.
\par 17 Dán kígyó lesz az úton, szarvaskígyó az ösvényen, mely a ló körmébe harap, hogy lovagja hanyatt esik.
\par 18 Szabadításodra várok Uram!
\par 19 Gád! had háborgatja; majd õ hág annak sarkába.
\par 20 Ásernek kenyere kövér, királyi csemegét szolgáltat.
\par 21 Nafthali, gyorslábú szarvas, az õ beszéde kedves.
\par 22 Termékeny fa József, termõ ág a forrás mellett, ágazata meghaladja a kõfalat.
\par 23 Keserítik, lövöldözik és üldözik a nyilazók:
\par 24 De mereven marad kézíve, feszülten keze karjai, Jákób Hatalmasának kezétõl, onnan, Izráel pásztorától, kõsziklájától.
\par 25 Atyád istenétõl, a ki segéljen; a mindenhatótól, a ki megáldjon, az ég áldásaival, onnan felülrõl, a mélység áldásaival, mely alant terül, az emlõk és anyaméh áldásaival.
\par 26 Atyád áldásai meghaladják az õs hegyek áldásait, az örök halmok kiességeit. Szálljanak József fejére, a testvérek közûl kiválasztatottnak koponyájára.
\par 27 Benjámin ragadozó farkas: reggel ragadományt eszik, este pedig zsákmányt oszt.
\par 28 Mind ezek Izráel nemzetségei, tizenketten, és ez az a mit mondott nékik az õ atyjok, mikor õket megáldá; mindeniket tulajdon áldásával áldá meg.
\par 29 És parancsola nékik és monda: Én az én népemhez takaríttatom, temessetek engem az én atyáimhoz, ama barlangba, mely a Khitteus Efron mezején van.
\par 30 Abba a barlangba, mely Kanaán földén Mamré átellenében Makpelahnak mezején van, melyet megvett Ábrahám a mezõvel együtt a Khitteus Efrontól, temetésre való örökségül.
\par 31 Oda temették el Ábrahámot és Sárát az õ feleségét; oda temették Izsákot és Rebekát az õ feleségét; s oda temettem el Leát is.
\par 32 Szerzemény e mezõ és a barlang, mely abban van, a Khéth fiaitól.
\par 33 És elvégezé Jákób a mit fiainak parancsolt és fölszedé lábait az ágyra, és kimúlék és az õ népéhez takaríttaték.

\chapter{50}

\par 1 József pedig az õ atyja orczájára borúla és siránkozék felette és csókolgatá õt.
\par 2 És megparancsolá József az õ szolgáinak, az orvosoknak, hogy balzsamozzák be az õ atyját; és bebalzsamozák az orvosok Izráelt.
\par 3 Mikor negyven nap eltelék, mert akkorra telnek be a bebalzsamozás napjai, siraták õt az Égyiptombeliek hetven napig.
\par 4 És elmúlának az õ siratásának napjai, és szóla József a Faraó házanépéhez, mondván: Ha kedves vagyok elõttetek, szóljatok kérlek a Faraónak, mondván:
\par 5 Az én atyám engem megesketett, mondván: Ímé én meghalok; az én síromba, melyet Kanaán földén ástam magamnak, oda temess el engem. Most hát kérlek, hadd menjek el, és temessem el az én atyámat, azután visszatérek.
\par 6 És monda a Faraó: Eredj el és temesd el a te atyádat, a mint megesketett téged.
\par 7 Elméne azért József, hogy az õ atyját eltemesse, és vele együtt felmenének mind a Faraó szolgái, az õ házának vénei és Égyiptom földének minden vénei.
\par 8 Józsefnek is egész háznépe; és az õ bátyjai, és az õ atyjának háznépe; csak gyermekeiket, juhaikat és barmaikat hagyták a Gósen földén.
\par 9 Felmenének annakfelette õ vele szekerek is és lovagok, úgy hogy igen nagy sereg vala.
\par 10 Mikor eljutának Atád szérûjéhez, mely a Jordánon túl van, nagy és keserves sírással sírának ott. József pedig hét napig gyászolá az õ atyját.
\par 11 És láták az ország lakosai, a Kanaán népe azt a gyászt Atád szérûjénél, és mondának: Keserves gyásza ez az Égyiptombelieknek. Azért nevezék azt a helyet Ábel Miczrajimnak, mely a Jordánon túl van.
\par 12 Aképen cselekedének azért Jákóbbal az õ fiai, a miképen megparancsolta vala nékik.
\par 13 Elvivék ugyanis õt az õ fiai Kanaán földére és eltemeték õt a Makpelah mezõnek barlangjába, melyet vett vala Ábrahám a mezõvel együtt temetésre való örökségnek  a Khitteus Efrontól Mamrénak átellenében.
\par 14 És visszatére József Égyiptomba, õ, és az õ atyjafiai, és mind azok, kik vele fölmentek vala az õ atyjának temetésére, minekutána eltemette az õ atyját.
\par 15 A mint láták József bátyjai, hogy az õ atyjok meghalt, ezt mondják vala: Hátha gyûlölni fog minket József, és visszaadja nékünk mindazt a gonoszt, a mit rajta elkövettünk.
\par 16 Izenetet küldének azért Józsefhez, mondván: A te atyád megparancsolta nékünk az õ holta elõtt, mondván:
\par 17 Így szóljatok Józsefhez: Kérünk téged, bocsásd meg a te atyádfiainak vétkét és bûnöket, mert gonoszul cselekedtek te ellened. Most azért bocsásd meg azoknak vétkét, a kik a te atyád Istenét szolgálják. József pedig sír vala, mikor ezt mondák néki.
\par 18 Járulának pedig õ hozzá az õ testvérei is, és leborúlának elõtte és mondának: Ímé mi a te szolgáid vagyunk.
\par 19 József pedig monda: Ne féljetek: avagy Isten gyanánt vagyok-é én?
\par 20 Ti gonoszt gondoltatok én ellenem, de Isten azt jóra gondolta fordítani, hogy cselekedjék úgy a mint ma, hogy sok nép életét megtartsa.
\par 21 Most annakokáért ne féljetek: Eltartalak én titeket és a ti gyermekeiteket. És megvígasztalá õket és szívökre beszéle.
\par 22 József pedig Égyiptomban lakozék; mind õ, mind az õ atyjának házanépe. És éle József száz tíz esztendeig.
\par 23 És látá József Efraimtól harmad ízben való fiait. Manasse fiának Mákhirnak is születtek József térdén gyermekei.
\par 24 És monda József az õ testvéreinek: Én meghalok, de Isten bizonnyal meglátogat titeket és felvisz titeket e földrõl arra a földre, melyet esküvel ígért meg Ábrahámnak, Izsáknak és Jákóbnak.
\par 25 És megesketé József Izráel fiait, mondván: Mikor az Isten titeket bizonnyal meglátogat, vigyétek fel innen az én tetemeimet magatokkal.
\par 26 És meghala József száz tíz esztendõs korában, és bebalzsamozák, és koporsóba tevék Égyiptomban.



\end{document}