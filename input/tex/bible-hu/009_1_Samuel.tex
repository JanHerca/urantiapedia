\begin{document}

\title{Sámuel I. könyve}


\chapter{1}

\par 1 Volt egy ember, Ramataim-Czofimból, az Efraim hegységérõl való, és az õ neve Elkána vala, Jerohámnak fiai, ki Elihu fia, ki Tohu fia, ki Sof fia volt; Efraimita vala.
\par 2 És két felesége volt néki; az egyiket Annának, a másikat pedig Peninnának hívták. Peninnának gyermekei valának, de Annának nem valának gyermekei.
\par 3 És ez az ember felmegy vala esztendõnként az õ városából, hogy imádkozzék és áldozatot tegyen a Seregek Urának Silóban. Ott pedig Éli két fia, Hofni és Fineás valának az Úrnak papjai.
\par 4 És azon a napon, melyen Elkána áldozni szokott, Peninnának, az õ feleségének, és minden fiának és leányának áldozati részt ád vala;
\par 5 Annának pedig kétakkora részt ád vala, mivel Annát igen szerette; de az Úr bezárá az õ méhét.
\par 6 Igen bosszantja vala pedig Annát vetélkedõ társa, hogy felingerelje, mivel az Úr bezárá az õ méhét.
\par 7 És így történt ez esztendõrõl esztendõre; valahányszor felment az Úrnak házába, ekképen bosszantá õt, õ pedig sír vala és semmit sem evék.
\par 8 És monda néki Elkána, az õ férje: Anna, miért sírsz, és miért nem eszel? Mi felett bánkódol szívedben? Avagy nem többet érek-é én néked tíz fiúnál?
\par 9 És felkele Anna, minekutána evének Silóban és minekutána ivának, (Éli pap pedig az Úrnak templomában az ajtófélnél ül vala székében),
\par 10 És lelkében elkeseredve, könyörge az Úrnak, és igen sír vala.
\par 11 És fogadást tõn, mondván: Seregeknek Ura! ha megtekinted a te szolgáló leányodnak nyomorúságát, és megemlékezel rólam, és nem feledkezel el szolgáló leányodról, hanem fiúmagzatot adsz szolgáló leányodnak: én õt egész életére az Úrnak ajánlom, és borotva nem érinti az õ fejét soha!
\par 12 Mivelhogy pedig hosszasan imádkozék az Úr elõtt: Éli figyel vala az õ szájára;
\par 13 És mivel Anna szívében könyörge (csak ajka mozgott, szava pedig nem volt hallható): Éli gondolá, hogy részeg.
\par 14 Monda azért néki Éli: Meddig leszel részeg? Távolítsd el mámorodat magadtól.
\par 15 Anna pedig felele, és monda néki: Nem, Uram! bánatos lelkû asszony vagyok én; sem bort, sem részegítõ italt nem ittam, csak szívemet öntöttem ki az Úr elõtt.
\par 16 Ne tartsd a te szolgáló leányodat rossz asszonynak, mert az én bánatomnak és szomorúságomnak teljességébõl szólottam eddig.
\par 17 És felelé Éli, és monda: Eredj el békességgel, és Izráelnek Istene adja meg a te kérésedet, a melyet kértél tõle.
\par 18 Õ pedig monda: Legyen kedves elõtted a te szolgáló leányod! És elméne az asszony az õ útjára és evék, és arcza nem vala többé szomorú.
\par 19 És reggel felkelének, és minekutána imádkozának az Úr elõtt, visszatértek, és elmenének haza Rámába. És ismeré Elkána az õ feleségét, és az Úr megemlékezék róla.
\par 20 És történt idõ multával, hogy terhes lõn Anna, és szüle fiat és nevezé õt Sámuelnek, mert úgymond az Úrtól kértem õt.
\par 21 És mikor felméne a férfi, Elkána és az õ egész háznépe, hogy bemutassa az Úrnak esztendõnként való áldozatát és fogadását:
\par 22 Anna nem ment fel, hanem monda férjének: Mihelyt a gyermeket elválasztándom, felviszem õt, hogy az Úr elõtt megjelenjen, és ott maradjon örökké.
\par 23 És monda néki Elkána, az õ férje: Cselekedjél úgy, a mint néked tetszik, maradj itthon, míg elválasztod; csakhogy az Úr teljesítse be az õ beszédét! Otthon marada azért az asszony és szoptatta gyermekét, a míg elválasztá.
\par 24 És minekutána elválasztotta, felvivé magával; három tulokkal, egy efa liszttel és egy tömlõ borral, és bevivé õt az Úrnak házába, Silóban. A gyermek pedig még igen kicsiny vala.
\par 25 És levágták a tulkot, és a gyermeket Élihez vitték.
\par 26 Õ pedig monda: Óh Uram! Él a te lelked, Uram, hogy én vagyok az az asszony, a ki itt állott melletted, és könyörgött az Úrnak.
\par 27 Ezért a fiúért könyörögtem, és az Úr megadta kérésemet, a melyet tõle kértem.
\par 28 Most azért én is az Úrnak szentelem; teljes életére az Úrnak legyen szentelve! És imádkozának ott az Úrhoz.

\chapter{2}

\par 1 És imádkozék Anna, és monda:
\par 2 Senki sincs olyan szent, mint az Úr, Sõt rajtad kivül senki sincs. Nincsen olyan kõszál, mint a mi Istenünk.
\par 3 Ne szóljatok oly kevélyen, oly nagyon kevélyen; Szátokból ne jõjjön kérkedõ szó, Mert mindentudó Isten az Úr, És a cselekedeteket õ ítéli meg.
\par 4 Az erõs kézíjjasokat megrontja, És a roskadozókat erõvel övedzi fel,
\par 5 A megelégedettek bérért szegõdnek el, Éhezõk pedig nem lesznek; S míg a magtalan hét gyermeket szül, A sok gyermekû megfogyatkozik.
\par 6 Az Úr öl és elevenít, Sírba visz és visszahoz.
\par 7 Az Úr szegénynyé tesz és gazdagít, Megaláz s fel is magasztal;
\par 8 Felemeli a porból a szegényt, És a sárból kihozza a szûkölködõt, Hogy ültesse hatalmasok mellé, És a dicsõségnek székét adja nékik; Mert az Úré a földnek oszlopai, És azokra helyezé a föld kerekségét.
\par 9 Híveinek lábait megoltalmazza, De az istentelenek setétségben némulnak el, Mert nem az erõ teszi hatalmassá az embert.
\par 10 Az Úr, a kik vele versengenek, megrontja, Mennydörög felettök az égben, Az Úr megítéli a földnek határait, Királyának pedig hatalmat ad, És felemeli felkentjének szarvát!
\par 11 Elméne ezután Elkéna Rámába az õ házához; a gyermek pedig az Úrnak szolgája lett Éli pap elõtt.
\par 12 Éli fiai azonban Béliál fiai valának, nem ismerék az Urat.
\par 13 És a papoknak ez vala szokásuk a néppel szemben: Ha mikor valaki áldozatot tesz vala, eljött a papnak szolgája, midõn a húst fõzték, és a háromágú villácska az õ kezében vala;
\par 14 És beüti vala a serpenyõbe, vagy üstbe, vagy fazékba, vagy edénybe, és mindent, a mit a villácskával kihúz, magának veszi el a pap. Így cselekesznek egész Izráellel, kik oda mennek Silóba.
\par 15 És minekelõtte a kövérét megáldoznák, eljön a papnak szolgája és azt mondja az áldozó embernek: Adj a papnak sütni való húst, mert nem fogad el tõled fõtt húst, hanem csak nyerset.
\par 16 És ha az ember azt mondja néki: Hadd gyújtsák meg most a kövérét, azután vedd el, a mint lelked kívánja: akkor azt mondják vala: Semmiképen nem, hanem most adjad, mert ha nem, erõvel elveszem.
\par 17 Igen nagy volt azért az ifjaknak bûne az Úr elõtt, mert az emberek megútálják vala az Úrnak áldozatát.
\par 18 Sámuel pedig szolgál vala az Úrnak, mint gyermek, gyolcs efóddal körül övezve.
\par 19 És anyja kicsiny felsõ ruhát csinált vala néki, és felvivé néki esztendõnként, mikor férjével felment az esztendõnként való áldozat bemutatására.
\par 20 És megáldá Éli Elkánát és az õ feleségét, és monda: Adjon az Úr néked magzatot ez asszonytól a helyett, a kiért könyörgött, és a kit az Úrnak kért. És haza menének.
\par 21 És meglátogatá az Úr Annát, ki az õ méhében fogada, és szült három fiút és két leányt. És a gyermek Sámuel felnevekedék az Úrnál.
\par 22 Éli pedig igen vén vala, és meghallá mindazt, a mit fiai cselekesznek egész Izráellel, és hogy az asszonyokkal hálnak, kik a gyülekezet sátorának nyílása elõtt szolgálnak.
\par 23 Monda azért nékik: Miért cselekesztek ilyen dolgot? Mert hallom a ti gonosz cselekedeiteket mind az egész néptõl.
\par 24 Ne tegyétek fiaim! mert nem jó hír az, melyet hallok; vétkessé teszitek az Úrnak népét.
\par 25 Ha ember embertársa ellen vétkezik, megítéli az Isten; de ha az Úr ellen vétkezik az ember, ki lehetne érette közbenjáró? De nem hallgatának atyjok szavára, mert az Úr meg akará õket ölni.
\par 26 A gyermek Sámuel pedig folytonosan növekedék és kedves volt mind az Úr, mind az emberek elõtt.
\par 27 És eljöve Istennek embere Élihez, és monda néki: Így szól az Úr: Nem jelentettem-é ki magamat atyád házának, midõn Égyiptomban a Faraó házában valának?
\par 28 És kiválasztám õt papnak Izráel minden nemzetségei közül magamnak, hogy áldozzon az én oltáromon; hogy füstölõ szert füstölögtessen, hogy az efódot elõttem viselje; és atyád házára bíztam Izráel fiainak minden tüzes áldozatait.
\par 29 Miért tapossátok meg az én véres áldozatomat és ételáldozatomat, melyet rendeltem e hajlékban? És te többre becsülöd fiaidat, mint engem, hogy magatokat hízlaljátok az én népem Izráel minden áldozatának elejével.
\par 30 Azért így szól az Úr, Izráelnek Istene, jóllehet megmondottam, hogy a te házad és atyádnak háza mindörökké én elõttem jár; de most, azt mondja az Úr, távol legyen tõlem, mert a kik engem tisztelnek, azoknak tisztességet szerzek, a kik azonban engem megútálnak, megutáltatnak.
\par 31 Ímé, napok jõnek, és levágom a te karodat és atyád házának karját, hogy ne legyen vén ember a te házadban.
\par 32 És meglátod az Isten hajlékának szorongattatását, mind a helyett, a mi jót cselekedett volna Izráellel; és nem lészen vén ember a te házadban soha.
\par 33 Mindazonáltal nem fogok mindenkit kiirtani oltárom mellõl, te éretted, hogy szemeidet emészszem és lelkedet gyötörjem; de egész házadnépe férfikorban hal meg.
\par 34 És az legyen elõtted a jel, a mi következik két fiadra, Hofnira és Fineásra, hogy egy napon halnak meg mind a ketten.
\par 35 Támasztok azonban magamnak hûséges papot, ki kedvem és akaratom szerint cselekszik; és építek néki állandó házat, és az én felkentem elõtt fog járni mindenkor.
\par 36 És lészen, hogy mind az, a ki megmarad a te házadból, eljön, hogy leboruljon elõtte egy ezüst pénzecskéért és egy darab kenyérért, és ezt mondja: ugyan helyezz el engem a papi tisztségek egyikébe, hogy ehessem egy falat kenyeret.

\chapter{3}

\par 1 És a gyermek Sámuel szolgál vala az Úrnak Éli elõtt. És abban az idõben igen ritkán volt az Úrnak kijelentése, nem vala nyilván való látomás.
\par 2 És történt egyszer, mikor Éli az õ szokott helyén aludt, (szemei pedig homályosodni kezdének, hogy látni sem tudott),
\par 3 És az Istennek szövétneke még nem oltatott el, és Sámuel az Úrnak templomában feküdt, hol az Istennek ládája volt:
\par 4 Szólott az Úr Sámuelnek, õ pedig felele: Ímhol vagyok!
\par 5 És Élihez szalada, és monda: Ímhol vagyok, mert hívtál engem; õ pedig felele: Nem hívtalak, menj vissza, feküdjél le. Elméne azért és lefeküvék.
\par 6 : És szólítá az Úr ismét: Sámuel! Sámuel pedig felkelvén, Élihez ment, és monda: Ímhol vagyok, mert hívtál engem. És õ felele: Nem hívtalak fiam, menj vissza, feküdjél le.
\par 7 Sámuel pedig még nem ismerte az Urat, mert még nem jelentetett ki néki az Úrnak ígéje.
\par 8 És szólítá az Úr harmadszor is Sámuelt; õ pedig felkelvén, Élihez ment és monda: Ímhol vagyok, mert hívtál engem. Akkor eszébe jutott Élinek, hogy az Úr hívja a gyermeket.
\par 9 Monda azért Éli Sámuelnek: Menj el, feküdjél le, és ha szólítanak téged, ezt mondjad: Szólj Uram, mert hallja a te szolgád. Elméne azért Sámuel, és lefeküvék az õ helyére.
\par 10 Akkor eljövén az Úr, oda állott és szólítá, mint annak elõtte: Sámuel, Sámuel! És monda Sámuel: Szólj, mert hallja a te szolgád!
\par 11 És monda az Úr Sámuelnek: Ímé én oly dolgot cselekszem Izráelben, melyet valakik hallanak, mind a két fülök megcsendül bele.
\par 12 Azon a napon véghez viszem Élin mind azt, a mit kijelentettem háza ellen; megkezdem és elvégezem.
\par 13 Mert megjelentettem néki, hogy elítélem az õ házát mind örökre, az álnokság miatt, a melyet jól tudott, hogy miként teszik vala útálatosokká magokat az õ fiai, és õ nem akadályozta meg õket.
\par 14 Annakokáért megesküdtem az Éli háza ellen, hogy sohasem töröltetik el Éli házának álnoksága, sem véres áldozattal, sem ételáldozattal.
\par 15 Aluvék azért Sámuel mind reggelig, és akkor kinyitá az Úr házának ajtait. És Sámuel nem meri vala megjelenteni Élinek a látomást.
\par 16 Szólítá azért Éli Sámuelt, és monda: Fiam, Sámuel! Õ pedig felele: Ímhol vagyok.
\par 17 És monda: Mi az a dolog, melyet mondott néked az Úr? El ne titkold elõttem! Úgy cselekedjék veled az Isten most és azután is, ha te valamit elhallgatsz elõttem mind abból, a mit mondott néked!
\par 18 Megmondott azért Sámuel néki mindent, és semmit sem hallgatott el elõtte. Õ pedig monda: Õ az Úr, cselekedjék úgy, a mint néki jónak tetszik.
\par 19 Sámuel pedig fölnevekedék, és az Úr vala õ vele, és semmit az õ ígéibõl a földre nem hagy vala esni.
\par 20 És megtudá egész Izráel Dántól Bersebáig, hogy Sámuel az Úr prófétájául rendeltetett.
\par 21 És az Úr kezde ismét megjelenni Silóban, mert kijelentette magát az Úr Sámuelnek SIlóban, az Úrnak beszéde által.

\chapter{4}

\par 1 És ismeretessé lett Sámuel beszéde egész Izráelben. És kiméne Izráel a Filiszteusok ellen harczolni, és tábort járának Ében-Ézernél, a Filiszteusok pedig tábort járának Áfekben.
\par 2 És csatarendbe állának a Filiszteusok Izráel ellen, és megütközének, és megveretteték Izráel a Filiszteusok által, és levágának a harczmezõn mintegy négyezer embert.
\par 3 És mikor a nép a táborba visszatért, mondának Izráel vénei: Vajjon miért vert meg minket ma az Úr a Filiszteusok elõtt?! Hozzuk el magunkhoz az Úr frigyládáját Silóból, hogy jõjjön közénk az Úr, és szabadítson meg ellenségeink kezébõl.
\par 4 Elkülde azért a nép Silóba, és elhozák onnan a Seregek Urának frigyládáját, a ki ül a Khérubok felett. Ott volt Éli két fia is az Isten frigyládájával, Hofni és Fineás.
\par 5 És mikor az Úr frigyládája a táborba érkezék, rivalgott az egész Izráel nagy rivalgással, hogy megrendüle a föld.
\par 6 Mikor pedig meghallották a Filiszteusok a rivalgás hangját, mondának: Micsoda nagy rivalgás hangja ez a zsidók táborában? És mikor megtudták, hogy az Úrnak ládája érkezett a táborba,
\par 7 Megfélemlének a Filiszteusok, mert mondának: Isten a táborba jött! És mondának: Jaj nékünk! mert nem történt ilyen soha az elõtt.
\par 8 Jaj nékünk! Kicsoda szabadít meg minket ennek a hatalmas Istennek kezébõl? Ez az az Isten, a ki Égyiptomot mindenféle csapással sújtotta a pusztában.
\par 9 Legyetek bátrak és legyetek férfiak, Filiszteusok! hogy ne kelljen szolgálnotok a zsidóknak, mint a hogy õk szolgáltak néktek. Azért legyetek férfiak, és harczoljatok!
\par 10 Megütközének azért a Filiszteusok, és megveretett Izráel, és kiki az õ sátorába menekült; és a vereség oly nagy volt, hogy Izráel közül harminczezer gyalog hullott el.
\par 11 És az Isten ládája is elvétetett, és meghala Élinek mindkét fia, Hofni és Fineás.
\par 12 Akkor elszalada a harczból egy ember a Benjámin nemzetségébõl, és Silóba ment azon a napon, ruháit megszaggatván és port hintvén a fejére.
\par 13 És ímé mikor oda ért, Éli az õ székében ült, az útfélen várakozván; mert szíve rettegésben volt az Isten ládája miatt. És mihelyt odaért az ember, hogy hírt mondjon a városban, jajveszékelt az egész város.
\par 14 És meghallotta Éli a kiáltás hangját, és monda: Micsoda nagy zajongás ez? Az az ember pedig sietve eljöve, és megmondotta Élinek.
\par 15 Éli pedig kilenczvennyolcz esztendõs volt, és szemei annyira meghomályosodtak, hogy már nem is látott.
\par 16 És monda az ember Élinek: Én a harczból jövök, én a harczból menekültem ma. És monda: Mi dolog történt, fiam?
\par 17 Felele a követ, és monda: Megfutamodék Izráel a Filiszteusok elõtt, és igen nagy veszteség lõn a népben, és a te két fiad is meghalt, Hofni és Fineás, és az Isten ládáját is elvették.
\par 18 És lõn, hogy midõn az Isten ládáját említé, hátraesék a székrõl a kapufélhez, és nyakát szegte és meghala, mert immár vén és nehéz ember vala. És õ negyven esztendeig ítélt Izráel felett.
\par 19 És az õ menye, Fineásnak felesége, várandós vala; és a mikor meghallá a hírt, hogy az Isten ládája elvétetett és az õ ipa és férje meghalának, térdre esék és szûle, mert a fájdalmak meglepték.
\par 20 És mikor elalélt, mondának azok, akik mellette állanak vala: Ne félj, mert fiút szültél; de õ nem felelt és nem figyelt arra.
\par 21 És nevezé a gyermeket Ikábódnak, mondván: "Oda van Izráel dicsõsége", mert elvétetett az Isten ládája, és az õ ipa és az õ férje.
\par 22 És monda ismét: Oda van Izráel dicsõsége, mert elvétetett az Isten ládája.

\chapter{5}

\par 1 A Filiszteusok pedig vevék az Isten ládáját és elvitték Ében-Ézerbõl Asdódba.
\par 2 És megfogták a Filiszteusok az Isten ládáját, és bevivén azt a Dágon templomába, Dágon mellé helyezték el.
\par 3 És mikor az Asdódbeliek másnap korán felkelének, ímé Dágon leesett arczczal a földre az Úr ládája elõtt. És felvevék Dágont, és ismét helyreállíták.
\par 4 Mikor pedig másnap korán reggel felkelének, ímé Dágon ismét leesett arczczal a földre az Úr ládája elõtt; és Dágonnak feje és két kezefeje letörve a küszöbön valának, csak Dágon dereka maradt meg.
\par 5 Annakokáért a Dágon papjai és mind azok, akik a Dágon templomába járnak, nem lépnek a Dágon küszöbére Asdódban mind e mai napig.
\par 6 És az Úrnak keze az Asdódbeliekre nehezedék, és pusztítja vala õket; és megveré õket fekélyekkel, Asdódot és határait.
\par 7 Mikor azért látták az Asdódbeliek, hogy így van a dolog, mondának: Ne maradjon nálunk Izráel Istenének ládája, mert reánk nehezedett az õ keze, és Dágonra, a mi istenünkre.
\par 8 Elküldének tehát, és összegyûjték a Filiszteusoknak minden fejedelmeit, és mondának: Mit csináljunk Izráel Istenének ládájával? Azok pedig felelének: Vigyék Gáthba az Izráel Istenének ládáját. És elvivék Izráel Istenének ládáját.
\par 9 És történt azután, hogy elvitték, az Úrnak keze igen nagy rémületére lõn a városnak; és megveré a városnak lakosait kicsinytõl fogva nagyig, és fekélyek támadának rajtok.
\par 10 Elküldék azért az Istennek ládáját Ekronba. És lõn, hogy a mikor Ekronba jutott az Istennek ládája, felzúdulának az Ekronbeliek, mondván: Reám hozták Izráel Istenének ládáját, hogy megöljön engem és az én népemet.
\par 11 Annakokáért elküldének, és összegyûjték a Filiszteusoknak minden fejedelmeit, és mondának: Küldjétek el Izráel Istenének ládáját, hogy térjen vissza helyére, és ne öljön meg engem és az én népemet. Mert halálos rémület volt az egész városban: igen súlyos volt ott Istennek keze.
\par 12 És azok az emberek, a kik nem haltak meg, fekélyekkel sújtattak annyira, hogy a város jajkiáltása felhatott az égig.

\chapter{6}

\par 1 És az Úrnak ládája hat hónapig volt a Filiszteusok földén.
\par 2 Akkor szólíták a Filiszteusok a papokat és jövendõmondókat, mondván: Mit tegyünk az Úrnak ládájával? Mondjátok meg nékünk, mi módon küldjük haza?
\par 3 Azok pedig felelének: Ha elkülditek Izráel Istenének ládáját, ne küldjétek azt üresen, hanem hozzatok néki vétekért való áldozatot; akkor meggyógyultok, és megtudjátok, miért nem távozik el az õ keze rólatok.
\par 4 És mondának: Micsoda az a vétekért való áldozat, melyet hoznunk kell néki? Azok pedig felelének: A Filiszteusok fejedelmeinek száma szerint öt fekélyforma arany és öt arany egér, mert ugyanazon csapás van mindeneken, a ti fejedelmeiteken is.
\par 5 Csináljátok meg azért fekélyeiteknek képmását és egereiteknek képmását, melyek pusztítják a földet, és így adjatok Izráel Istenének dicsõséget, talán megkönnyebbedik az õ keze rajtatok, és a ti isteneteken és földeteken.
\par 6 Miért keményítenétek meg szíveiteket, mint megkeményítették szívöket Égyiptom és a Faraó? Avagy nem úgy volt-é, hogy a mint hatalmát megmutatta rajtok, elbocsátá õket, hogy elmenjenek?
\par 7 Most azért vegyetek és készítsetek egy új szekeret és két borjas tehenet, melyeken még nem volt járom; és fogjátok be a szekérbe a teheneket, borjaikat pedig vigyétek el tõlök haza;
\par 8 És vegyétek az Úrnak ládáját, és tegyétek a szekérre; az aranyszerszámokat pedig, melyeket vétekért való áldozatul hoztok néki, tegyétek egy táskába az oldalára, és bocsássátok el, hadd menjen el.
\par 9 És nézzetek utána, hogy ha az õ határának útján Béth-Semes felé tart, akkor õ szerezte nékünk ezt a nagy bajt; ha pedig nem, akkor megtudjuk, hogy nem az õ keze sújtott minket, hanem csak véletlen volt az, a mi velünk történt.
\par 10 Úgy cselekedének azért az emberek; és vettek két borjas tehenet, és befogták a szekérbe, borjaikat pedig berekeszték otthon.
\par 11 És feltették az Úrnak ládáját a szekérre és a táskát az arany egerekkel és fekélyeiknek képmásával.
\par 12 A tehenek pedig egyenesen a Béth-Semes felé vivõ úton menének; egy úton mentek, folytonosan bõgve, és sem jobbra, sem balra nem térének le. És a Filiszteusok fejedelmei utánok menének Béth-Semes határáig.
\par 13 A Béth-Semesbeliek pedig búzaaratással foglalkozának a völgyben, és felemelvén szemeiket, meglátták a ládát: és örvendezének, hogy látták.
\par 14 A szekér pedig eljutott a Béth-Semesbõl való Józsua mezejére, és ott megállott. Vala pedig ott egy nagy kõ, és felvagdalták a szekér fáit és a teheneket megáldozták az Úrnak egészen égõáldozatul.
\par 15 Akkor a Léviták levették az Úrnak ládáját és a táskát, mely mellette volt, melyben az aranyszerszámok valának, és a nagy kõre helyezték. A Béth-Semesbeli emberek pedig azon a napon egészen égõáldozatot és véres áldozatot áldoztak az Úrnak.
\par 16 És mikor a Filiszteusok öt fejedelme ezt látta, visszatére azon napon Ekronba.
\par 17 Az arany fekélyek pedig, melyeket a Filiszteusok vétekért való áldozatul hoztak az Úrnak, ezek: egyet Asdódért, egyet Gázáért, egyet Askelonért, egyet Gáthért, egyet Ekronért.
\par 18 Az arany egerek pedig a Filiszteusok minden városainak száma szerint valának, melyek az öt fejedelem alá tartoztak, a kerített városoktól a kerítetlen helységekig. És bizonyság az a nagy kõ, a melyre az Úrnak ládáját helyezték, mind a mai napig, a Béth-Semesbõl való Józsua mezején.
\par 19 És megvere az Úr a Béth-Semesbeliek közül némelyeket, mivel az Úrnak ládájába tekintének. Megvere pedig a nép közül ötvenezer és hetven embert. És a nép szomorkodott, hogy az Úr ily nagy csapással sújtotta vala a népet.
\par 20 Mondának azért a Béth-Semesbeliek: Kicsoda állhat meg az Úr elõtt, e szent Isten elõtt? És kihez megy el mi tõlünk?
\par 21 Követeket küldének akkor Kirjáth-Jeárim lakosaihoz, mondván: A Filiszteusok visszahozták az Úrnak ládáját, jõjjetek el, és vigyétek el azt magatokhoz.

\chapter{7}

\par 1 És eljövének a Kirjáth-Jeárimbeliek, és felvitték az Úrnak ládáját, és bevivék Adinádáb házába, mely a dombon vala; és Eleázárt, az õ fiát az Úr ládájának õrizésére rendelék.
\par 2 És lõn, hogy attól a naptól fogva, a melyen Kirjáth-Jeárimban hagyták a ládát, sok idõ eltelék, tudniillik húsz esztendõ. És Izráelnek egész háza síránkozék az Úr után.
\par 3 Sámuel pedig szóla Izráel egész házához, mondván: Ha ti teljes szívetekbõl megtértek az Úrhoz, és eltávolítjátok magatok közül az idegen isteneket és Astarótót, és szíveiteket elkészítitek az Úrnak, és csak néki szolgáltok: akkor megszabadít titeket a Filiszteusok kezébõl.
\par 4 Elhányák azért Izráel fiai a bálványokat és Astarótot, és csak az Úrnak szolgálának.
\par 5 Akkor Sámuel mondá: Gyûjtsétek össze egész Izráelt Mispába, hogy imádkozzam értetek az Úrhoz.
\par 6 Összegyûlének azért Mispában, és vizet merítvén, kiönték az Úr elõtt; és bõjtölének azon a napon, és így szólának ott: Vétkeztünk az Úr ellen. És ekképen ítélé Sámuel Izráel fiait Mispában.
\par 7 Mikor pedig a Filiszteusok meghallották, hogy Izráel fiai összegyûlének Mispában, feljöttek a Filiszteusok fejedelmei Izráel ellen. És mikor ezt meghallották Izráel fiai, megrémülének a Filiszteusoktól.
\par 8 És mondának Izráel fiai Sámuelnek: Ne szünjél meg érettünk könyörögni az Úrhoz, a mi Istenünkhöz, hogy szabadítson meg minket a Filiszteusok kezébõl.
\par 9 Vett azért Sámuel egy szopós bárányt, és megáldozá azt egészen égõáldozatul az Úrnak. És fohászkodék Sámuel Izráelért az Úrhoz, és az Úr meghallgatá õt.
\par 10 És lõn, hogy a mikor Sámuel az égõáldozatot végezé, eljövének a Filiszteusok, hogy Izráel ellen harczoljanak. Az Úr pedig mennydörge nagy hangon azon a napon a Filiszteusok felett és annyira megzavarta õket, hogy megverettetének Izráel elõtt.
\par 11 Akkor kijövének Izráel emberei Mispából, és üldözték a Filiszteusokat és verték õket egészen Bethkáron alól.
\par 12 Sámuel pedig vett egy követ, és felállítá Mispa és Sén között, és Ében-Háézernek nevezte el, mert mondá: Mindeddig megsegített minket az Úr!
\par 13 És megaláztatának a Filiszteusok, és nem menének többé Izráel határára. És az Úrnak keze a Filiszteusok ellen volt Sámuelnek teljes életében.
\par 14 És visszavevé Izráel a városokat, a melyeket a Filiszteusok Izráeltõl elvettek vala, Ekrontól fogva Gáthig, és az õ határukat megszabadítá Izráel a Filiszteusok kezébõl. És békesség lõn Izráel és az Emoreusok között.
\par 15 Sámuel pedig életének minden napjaiban ítéle Izráel felett.
\par 16 És elmegy vala esztendõnként, és bejárá Béthelt, Gilgált és Mispát, és ítéletet hozott vala Izráel felett mind e helyeken.
\par 17 Annakutána visszatére Rámába, mert ott vala az õ háza, és ott is ítéletet hoza Izráel felett; és oltárt építe ott az Úrnak.

\chapter{8}

\par 1 És lõn, hogy a mikor Sámuel megvénhedett, az õ fiait tevé bírákká Izráel felett.
\par 2 Elsõszülött fiának pedig Joel volt a neve és a másodiknak Abia, kik Beérsebában bíráskodtak.
\par 3 De fiai nem járának az õ útjain, hanem a telhetetlenség után indulának, és ajándékot fogadának el, és elfordíták az igaz ítéletet.
\par 4 Összegyûlének azért Izráelnek minden vénei, és elmentek Sámuelhez Rámába.
\par 5 És mondának néki: Ímé te megvénhedtél, és fiaid nem járnak útaidon; most azért válaszsz nékünk királyt, a ki ítéljen felettünk, mint minden népnél szokás.
\par 6 Azonban Sámuelnek nem tetszék a beszéd, hogy azt mondák: Adj nékünk királyt, a ki ítéljen felettünk. És könyörge Sámuel az Úrhoz.
\par 7 És monda az Úr Sámuelnek: Fogadd meg a nép szavát mindenben, a mit mondanak néked, mert nem téged útáltak meg, hanem engem útáltak meg, hogy ne uralkodjam felettök.
\par 8 Mindama cselekedetek szerint, a melyeket véghez vittek attól a naptól kezdve, a melyen kihoztam õket Égyiptomból, egészen a mai napig (hogy elhagytak engem, és idegen isteneknek szolgáltak): veled is a szerint cselekesznek.
\par 9 Most azért hallgass szavukra; mindazáltal tégy ellenök bizonyságot, és add tudtokra a király hatalmát, a ki uralkodni fog felettök.
\par 10 És Sámuel megmondá az Úrnak minden beszédeit a népnek, mely tõle királyt kért.
\par 11 És monda: A királynak, a ki uralkodni fog felettetek, ez lesz a hatalma: fiaitokat elveszi és szekér vezetõivé és lovasaivá teszi õket, és szekere elõtt futnak.
\par 12 Ezredesekké teendi õket, és hadnagyokká ötven ember felett; velök szántatja meg barázdáit, és velök végezteti aratását, készítteti hadi szerszámait és harczi szekereihez az eszközöket.
\par 13 Leányaitokat pedig elviszi kenõcskészítõknek, szakácsnéknak és sütõknek.
\par 14 Elveszi legjobb szántóföldeiteket, szõlõhegyeiteket és olajfás kerteiteket, és szolgáinak adja.
\par 15 Veteményeiteket és szõlõiteket megdézsmálja, és fõbb embereinek és szolgáinak adja.
\par 16 Elveszi szolgáitokat, szolgálóitokat, legszebb ifjaitokat és szamaraitokat, és a maga dolgát végezteti azokkal.
\par 17 Barmaitokat megdézsmálja, és ti szolgái lesztek néki.
\par 18 És panaszkodni fogtok annak idejében királyotok miatt, kit magatok választottatok, de az Úr nem fog meghallgatni akkor titeket.
\par 19 A nép azonban nem akart Sámuel szavára hallgatni, és mondának: Nem! hanem király legyen felettünk.
\par 20 És mi is úgy legyünk, mint a többi népek, hogy királyunk ítéljen minket is, és elõttünk járjon, és vezesse a mi harczainkat.
\par 21 És miután Sámuel meghallgatta a népnek minden szavát, megmondta azokat az Úrnak.
\par 22 És monda az Úr Sámuelnek: Hallgass szavokra, és adj nékik királyt. És Sámuel monda Izráel férfiainak: Menjetek el haza, ki-ki az õ városába.

\chapter{9}

\par 1 És volt egy ember a Benjámin nemzetségébõl, kinek neve vala Kis, Abielnek fia, ki Sérornak fia, ki Bekoráthnak fia, ki Afiáknak fia, ki Benjámin házából való volt; igen tehetõs ember vala.
\par 2 És volt néki egy Saul nevû fia, ifjú és szép; õ nála Izráel fiai közül senki sem volt szebb; vállától felfelé magasabb vala az egész népnél.
\par 3 És elvesztek vala Kisnek, a Saul atyjának szamarai, és monda Kis Saulnak, az õ fiának: Végy magad mellé egyet a szolgák közül, és kelj fel, menj el, keresd meg a szamarakat.
\par 4 És bejárá az Efraim hegységét, és bejárta Salisa földét, de nem találták meg; és bejárták Sáálim földét, de nem voltak ott; és bejárá Benjámin földét, de nem találták meg.
\par 5 Mikor pedig a Suf földére jutottak, monda Saul az õ szolgájának, a ki vele volt: Jer, térjünk vissza, nehogy atyám elvetve a szamarak gondját, miattunk aggódjék.
\par 6 Az pedig monda néki: Ímé az Istennek embere most e városban van, és az az ember tiszteletben áll; mind az, a mit mond, beteljesedik. Most azért menjünk el oda, talán megmondja nékünk is a mi útunkat, hogy merre menjünk.
\par 7 És monda Saul az õ szolgájának: Elmehetünk oda, de mit vigyünk ez embernek? mert a kenyér elfogyott tarisznyánkból, és nincsen mit vigyünk ajándékba az Isten emberének; mi van nálunk?
\par 8 A szolga pedig felele ismét Saulnak, és monda: Ímé van nálam egy ezüst siklusnak negyedrésze, oda adom ezt az Isten emberének, hogy megmondja nékünk a mi útunkat.
\par 9 Régen Izráelben azt mondák, mikor valaki elment Istent megkérdezni: Jertek, menjünk el a nézõhöz; mert a kit most prófétának neveznek, régen nézõnek hívták.
\par 10 És Saul monda az õ szolgájának: Helyesen beszélsz; jer, menjünk el. Elmenének azért a városba, hol az Istennek embere volt.
\par 11 A mint pedig a város felhágóján menének, leányokkal találkozának, a kik vizet meríteni jöttek ki, és mondának nékik: Itt van-é a nézõ?
\par 12 És azok felelének nékik, és mondának: Igen, amott van elõtted, siess azért, mert ma jött a városba, mivel ma lesz a népnek véres áldozata ím e hegyen.
\par 13 A mint a városba mentek, azonnal megtaláljátok, mielõtt felmenne a hegyre, hogy egyék; mert a nép nem eszik addig, míg õ el nem jön, mivel a váres áldozatot õ áldja meg, és azután esznek a meghívottak. Azért most menjetek fel, mert épen most fogjátok õt megtalálni.
\par 14 Felmenének azért a városba. És mikor a városban menének, ímé akkor jöve ki Sámuel velök szemben, hogy a hegyre felmenjen.
\par 15 És az Úr kijelentette Sámuelnek füleibe, egy nappal az elõtt, hogy Saul eljött, mondván:
\par 16 Holnap ilyenkor küldök hozzád egy embert a Benjámin földérõl, és kend fel õt fejedelmül az én népem, Izráel felett, hogy megszabadítsa az én népemet a Filiszteusok kezébõl; mert megtekintém az én népemet, mivel felhatott az õ kiáltása hozzám.
\par 17 Mikor pedig Sámuel meglátta Sault, szóla néki az Úr: Ímé ez az az ember, a kirõl szólottam néked, õ uralkodjék az én népem felett.
\par 18 Akkor Saul a kapu alatt Sámuelhez közeledék és monda: Ugyan mondd meg nékem, hol van itt a nézõ háza?
\par 19 Sámuel pedig felele Saulnak, és monda: Én vagyok az a nézõ; menj fel elõttem a hegyre, és egyetek ma én velem, reggel pedig elbocsátlak téged, és megmondom néked mind azt, a mi szívedben van.
\par 20 A szamarak miatt pedig, melyek tõled ezelõtt három nappal elvesztek, ne aggódjál, mert megtaláltattak. És kié leend mind az, a mi Izráelben becses? Avagy nem a tiéd és a te atyádnak egész házáé?
\par 21 És Saul felele, és monda: Avagy nem Benjáminita vagyok-é én, Izráelnek legkisebb törzsébõl való? És az én nemzetségem nem legkisebb-é Benjámin törzsének nemzetségei között? Miért szólasz tehát hozzám ilyen módon?
\par 22 Akkor Sámuel megfogta Sault és az õ szolgáját, és bevezette õket az étkezõ helyre; és nékik adta a fõhelyet a meghívottak között, kik mintegy harminczan valának.
\par 23 És monda Sámuel a szakácsnak: Hozd elõ azt a darabot, melyet odaadtam néked, és a melyrõl azt mondám, hogy tartsd magadnál.
\par 24 Akkor a szakács felhozta a czombot és a mi rajta volt, és Saul elé tevé. És õ monda: Ímhol a megmaradt rész, vedd magad elé és egyél, mert erre az idõre tétetett az el számodra, mikor mondám: Meghívtam a népet. Evék azért Saul azon a napon Sámuellel.
\par 25 És miután lejöttek a hegyrõl a városba, a felházban beszélgetett Saullal.
\par 26 És korán felkelének. Történt ugyanis, hogy hajnalhasadtakor kiálta Sámuel Saulnak a felházba, mondván: Kelj fel, hogy elbocsássalak téged. És felkelt Saul, és kimenének ketten, õ és Sámuel az utczára.
\par 27 Mikor pedig lemenének a város végére, Sámuel monda Saulnak: Mondd meg a szolgának, hogy menjen elõre elõttünk - és elõre ment -, te pedig most állj meg, hogy megmondjam néked az Istennek beszédét.

\chapter{10}

\par 1 Akkor elõvevé Sámuel az olajos szelenczét, és az õ fejére tölté, és megcsókolá õt, és monda: Nem úgy van-é, hogy fejedelemmé kent fel az Úr téged az õ öröksége felett?
\par 2 Mikor azért te most elmégy tõlem, találkozni fogsz két emberrel a Rákhel sírja mellett, Benjámin határában, Selsáhnál, a kik azt mondják néked: Megtalálták a szamarakat, melyeknek keresésére indultál vala; és ímé, a te atyád felhagyott már a szamarakkal, és miattatok aggódik, mondván: Mit tegyek a fiamért?
\par 3 És mikor onnan tovább mégy, és a Thábor tölgyfájához jutsz, három ember fog téged ott találni, kik Béthelbe mennek fel Istenhez; az egyik három gödölyét visz, a másik visz három egész kenyeret, és a harmadik visz egy tömlõ bort.
\par 4 És azok békességesen köszöntenek téged, és két kenyeret adnak néked; te pedig vedd el azokat kezökbõl.
\par 5 Azután eljutsz az Isten hegyére, hol a Filiszteusok elõõrsei vannak. Mikor pedig bemégy oda a városba, a próféták seregével fogsz találkozni, kik a hegyrõl jõnek le, elõttök lant, dob, síp és hárfa lesz, és õk magok prófétálnak.
\par 6 Akkor az Úrnak lelke reád fog szállani, és velök együtt prófétálni fogsz, és más emberré leszesz.
\par 7 Mikor pedig mind e jelek beteljesednek rajtad, tedd meg magadért mind azt, a mi csak kezed ügyébe esik, mert az Isten veled van.
\par 8 Most azért menj le én elõttem Gilgálba, és ímé én lemegyek te hozzád, hogy égõáldozatot áldozzam és hálaáldozatot hozzak. Hét napig várakozzál, míg hozzád megyek, és akkor tudtodra adom, hogy mit cselekedjél.
\par 9 És lõn, a mint hátra fordult, hogy Sámueltõl eltávozzék, elváltoztatá Isten az õ szívét, és azon a napon beteljesedének mind azok a jelek.
\par 10 És mikor elmenének ama hegyre, ímé a prófétáknak serege vele szembe jöve, és az Istennek lelke õ reá szálla, és prófétála õ közöttök.
\par 11 És lõn, hogy mind azok, kik ismerték õt annakelõtte, mikor látták, hogy ímé a prófétákkal együtt prófétál, monda a nép egymás közt: Mi lelte Kisnek fiát? Avagy Saul is a próféták közt van?
\par 12 És felele egy közülök, és monda: Ugyan kicsoda az õ atyjuk? Azért lõn példabeszéddé: Avagy Saul is a próféták közt van?
\par 13 És mikor elvégezé a prófétálást, felment a hegyre.
\par 14 Saulnak nagybátyja pedig monda néki és a szolgájának: Hol jártatok? És õ monda: A szamarakat kerestük, de mivel sehol sem láttuk, Sámuelhez menénk.
\par 15 Akkor monda Saulnak a nagybátyja: Ugyan mondd meg nékem, mit mondott néktek Sámuel?
\par 16 És monda Saul a nagybátyjának: Nyilván megmondotta nékünk, hogy a szamarakat megtalálták. De a mit Sámuel a királyságról mondott, nem beszélte el néki.
\par 17 És összehívta Sámuel a népet Mispába az Úrhoz.
\par 18 És monda Izráel fiainak: Így szól az Úr, Izráelnek Istene: Én hoztam ki Izráelt Égyiptomból, és én szabadítottalak meg titeket az égyiptombeliek kezébõl és mind amaz országok kezébõl, melyek sanyargatának titeket.
\par 19 Ti pedig most megvetettétek a ti Isteneteket, a ki megszabadított titeket minden bajaitokból és nyomorúságaitokból; és azt mondottátok néki: Adj királyt nékünk. Most azért álljatok az Úr elé a ti nemzetségeitek és ezreitek szerint.
\par 20 És mikor elõállatá Sámuel Izráelnek minden nemzetségét, kiválasztaték sors által a Benjámin nemzetsége.
\par 21 Akkor elõállatá a Benjámin nemzetségét az õ házanépei szerint, és kiválasztaték a Mátri házanépe, azután kiválasztaték Saul, Kisnek fia; és keresék õt, de nem találták meg.
\par 22 Megkérdezték azért ismét az Urat: Vajjon eljön-é ide az az ember? És az Úr monda; Ímé õ a holmik közé rejtõzék el.
\par 23 Akkor elfutának, és elõhozták õt onnan. És mikor a nép közé álla, kimagaslék az egész nép közül vállától kezdve felfelé;
\par 24 És Sámuel monda az egész népnek: Látjátok-é, a kit választott az Úr? hogy nincsen hozzá hasonló az egész nép között! Akkor felkiálta az egész nép, és monda: Éljen a király!
\par 25 Sámuel pedig elõadá a nép elõtt a királyság jogát, és beírá egy könyvbe, és letevé az Úr elé. És elbocsátá Sámuel az egész népet, mindenkit a maga házához.
\par 26 Azután Saul is elment haza Gibeába és vele ment a sokaság, a kiknek szívét Isten megindította vala.
\par 27 Némely kaján emberek azonban azt mondák: Mit segíthet ez rajtunk? és megvetették õt és ajándékot nem vivének néki. Õ pedig olyan volt, mintha semmit sem hallott volna.

\chapter{11}

\par 1 És feljöve az Ammonita Náhás, és tábort jára Jábes Gileád ellen. A Jábesbeleik pedig mondának mindnyájan Náhásnak: Köss velünk szövetséget, és mi szolgálni fogunk néked.
\par 2 És monda nékik az Ammonita Náhás: Úgy szövetséget kötök veletek, ha kivágatom mindnyájatoknak jobb szemét, és tehetem ezt egész Izráelnek gyalázatára.
\par 3 Jábes vénei pedig mondának néki: Engedj nékünk hét napot, hogy követeket küldjünk Izráelnek minden határára; és ha senki sem segít meg minket, akkor kimegyünk hozzád.
\par 4 Elmenének azért a követek Saulhoz Gibeába, és elmondták e beszédeket a nép füle hallatára. És felemelé az egész község az õ szavát, és sírának.
\par 5 Saul pedig épen a mezõrõl jött vala a barmok után; és monda Saul: Mi történt a néppel, hogy sírnak? És elmondták néki a Jábesbeliek beszédeit.
\par 6 És mikor hallotta e beszédeket, az Úrnak lelke Saulra szálla, és az õ haragja nagyon felgerjede.
\par 7 És vett egy pár ökröt, és feldarabolá azokat, és a követektõl elküldé Izráelnek minden határára, mondván: A ki nem vonul Saul után és Sámuel után, annak ökreivel így cselekesznek. És az Úrnak félelme szálla a népre, és kivonulának mind egy szálig.
\par 8 És megszámlálá õket Bézekben. És Izráel fiai háromszázezeren valának, a Júdabeliek pedig harminczezeren.
\par 9 És mondának a követeknek, kik oda menének: Így szóljatok a Jábes-Gileádbelieknek: Holnap, mikor a nap felmelegszik, megszabadultok. És elmenének a követek, és megmondák a Jábesbelieknek, és õk örvendezének.
\par 10 Mondának azért a Jábesbeliek: Holnap kimegyünk hozzátok, hogy egészen úgy cselekedjetek velünk, a mint néktek jónak tetszik.
\par 11 Másodnapon pedig Saul a népet három seregre osztá, és kora hajnalban a táborra ütének, és verték Ammont mindaddig, míg a nap felmelegedék; a kik pedig megmaradtak, úgy szétszórattak, hogy kettõ sem maradt közülök együtt.
\par 12 Akkor a nép monda Sámuelnek: Kicsoda volt az, a ki mondá: Saul fog-é rajtunk uralkodni? Adjátok elõ a férfiakat, hogy megöljük õket!
\par 13 Saul azonban azt mondá: Senkit se öljetek meg a mai napon, mert ma szerzett szabadulást az Úr Izráelnek.
\par 14 Sámuel pedig monda a népnek: Jertek, menjünk el Gilgálba, és újítsuk meg ott a királyságot.
\par 15 Elméne azért az egész nép Gilgálba, és ott az úr elõtt Gilgálban királylyá tették Sault: és áldoztak ott hálaáldozatot az Úr elõtt, és felette örvendezének ott, Saul és Izráelnek minden férfiai.

\chapter{12}

\par 1 És monda Sámuel az egész Izráelnek: Ímé meghallgattam szavaitokat mindenben, valamit nékem mondottatok, és királyt választottam néktek.
\par 2 És most ímé a király elõttetek jár. Én pedig megvénhedtem és megõszültem, és az én fiaim ímé ti köztetek vannak, és én is elõttetek jártam ifjúságomtól fogva mind a mai napig.
\par 3 Itt vagyok, tegyetek bizonyságot ellenem az Úr elõtt és az õ felkentje elõtt: kinek vettem el az ökrét, és kinek vettem el a szamarát, és kit csaltam meg, kit sanyargattam, és kitõl fogadtam el ajándékot, hogy a miatt szemet hunyjak? és visszaadom néktek.
\par 4 Õk pedig felelének: Nem csaltál meg minket, nem sanyargattál minket, és senkitõl semmit el nem fogadtál.
\par 5 És õ monda nékik: Legyen bizonyság az Úr ti ellenetek, és bizonyság az õ felkentje ezen a napon, hogy semmit sem találtatok kezemben! Õk pedig mondának: Legyen bizonyságul.
\par 6 És monda Sámuel a népnek: Igen, az Úr, a ki rendelte Mózest és Áront, és a ki kihozta atyáitokat Égyiptom földérõl!
\par 7 Most azért álljatok elõ, hadd perlekedjem veletek az Úr elõtt az Úrnak minden jótéteményei felett, a melyeket cselekedett veletek és a ti atyáitokkal.
\par 8 Miután Jákób Égyiptomba ment vala, atyáitok az Úrhoz kiáltának, és az Úr elküldé Mózest és Áront, a kik kihozták atyáitokat Égyiptomból, és letelepíték erre a helyre.
\par 9 De õk elfeledték az Urat, az õ Istenöket, azért adá õket Siserának, a Hásor serege vezérének kezébe, és a Filiszteusok kezébe és Moáb  királyának kezébe, és azok harczolának ellenök.
\par 10 Akkor kiáltának az Úrhoz, és mondák: Vétkeztünk, mert elhagytuk az Urat, és szolgáltunk a Baáloknak és Astarótnak; most azért szabadíts meg minket ellenségeinknek kezébõl, hogy néked szolgáljunk.
\par 11 És elküldé az Úr Jerubbaált és Bédánt és Jeftét és Sámuelt, és megszabadíta titeket mindenfelõl ellenségeitek kezébõl, és biztonságban laktatok.
\par 12 Mikor pedig láttátok, hogy Náhás, az Ammon fiainak királya ellenetek jöve, azt mondátok nékem: Semmiképen nem, hanem király uralkodjék felettünk, holott csak a ti Istenetek, az Úr a ti királyotok.
\par 13 Most azért, ímhol a király, a kit választottatok, a kit kértetek. Ímé, az Úr királyt adott néktek.
\par 14 Hogyha az Urat félitek, és néki szolgáltok; szavára hallgattok és az Úr szája ellen engedetlenek nem lesztek; és mind ti, mind pedig a király, a ki felettetek uralkodik, az Urat, a ti Isteneteket követitek: megtartattok.
\par 15 Ha pedig az Úr szavára nem hallgattok, és az Úr szava ellen engedetlenek lesztek: az Úrnak keze ellenetek leend, miként a ti atyáitok ellen.
\par 16 Most is azért álljatok meg, és lássátok meg azt a nagy dolgot, a melyet az Úr visz véghez szemeitek elõtt.
\par 17 Avagy nem búzaaratás van-é most? Kiáltani fogok az Úrhoz, és õ mennydörgést és esõt ád, hogy megtudjátok és meglássátok, mily nagy a ti gonoszságtok, melyet cselekedtetek az Úr szemei elõtt, mikor királyt kértetek magatoknak.
\par 18 Kiálta azért Sámuel az Úrhoz, és az Úr mennydörgést és esõt adott azon a napon. És az egész nép nagyon megrettene az Úrtól és Sámueltõl.
\par 19 És monda az egész nép Sámuelnek: Könyörögj szolgáidért az Úrhoz, a te Istenedhez, hogy meg ne haljunk, mert minden bûneinket csak öregbítettük azzal a bûnnel, hogy királyt kértünk magunknak.
\par 20 És monda Sámuel a népnek: Ne féljetek! Ha már mind e gonoszságot véghez vittétek, most ne távozzatok el az Úrtól, hanem az Úrnak szolgáljatok teljes szívetekbõl.
\par 21 Ne térjetek el a hiábavalóságok után, a melyek nem használnak, meg sem szabadíthatnak, mert hiábavalóságok azok.
\par 22 Mert nem hagyja el az Úr az õ népét, az õ nagy nevéért; mert tetszett az Úrnak, hogy titeket a maga népévé válaszszon.
\par 23 Sõt tõlem is távol legyen, hogy az Úr ellen vétkezzem és felhagyjak az érettetek való könyörgéssel; hanem inkább tanítani foglak titeket a jó és igaz útra.
\par 24 Csak féljétek az Urat, és szolgáljatok néki hûségesen, teljes szívetekbõl; mert látjátok; mily nagy dolgot cselekedett veletek.
\par 25 Ha pedig folytonos rosszat cselekesztek: mind ti, mind királyotok elvesztek.

\chapter{13}

\par 1 Saul harmincz éves volt, mikor királylyá lett és mikor uralkodék az Izráel felett két esztendeig,
\par 2 Választa Saul magának az Izráel közül háromezer embert, és kétezer Saullal vala Mikmásban és Béthel hegységén, ezer pedig Jonathánnal volt Gibeában, a Benjámin városában; a népnek többi részét pedig elbocsátá, kit-kit a maga hajlékába.
\par 3 És Jonathán megveré a Filiszteusoknak elõõrsét, mely Gébában vala, és meghallották ezt a Filiszteusok; Saul pedig megfúvatta a trombitát az egész országban, mondván: Hallják meg a zsidók!
\par 4 És meghallotta egész Izráel, hogy azt mondák: Megverte Saul a Filiszteusok elõörsét, és gyûlöletessé vált Izráel a Filiszteusok elõtt. A nép pedig egybegyûle Saul mellé Gilgálba.
\par 5 Összegyûlének a Filiszteusok is, hogy harczoljanak Izráel ellen; harminczezer szekér és hatezer lovas volt, a nép pedig oly sok volt, mint a tenger partján lévõ föveny; és feljövének és tábort ütének Mikmásnál, Béth-Aventõl keletre.
\par 6 Mikor pedig Izráel férfiai látták, hogy bajban vannak - mert a nép szorongattatott -: elrejtõzék a nép a barlangokba, a bokrok és kõsziklák közé, sziklahasadékokba és vermekbe.
\par 7 És a zsidók közül némelyek általmenének a Jordánon, Gád és Gileád földére. Saul pedig még Gilgálban volt, és az egész nép, mely mellette vala, rettegett.
\par 8 És várakozék hét napig, a Sámuel által meghagyott ideig, de Sámuel nem jött el Gilgálba, a nép pedig elszélede mellõle.
\par 9 Akkor monda Saul: Hozzátok ide az égõáldozatot és a hálaáldozatokat. És égõáldozatot tõn.
\par 10 És mikor elvégezte az égõáldozatot; ímé megérkezék Sámuel, és Saul eleibe ment, hogy köszöntse õt.
\par 11 És monda Sámuel: Mit cselekedtél?! Saul pedig felele: Mikor láttam, hogy a nép elszélede mellõlem, és te nem jöttél el a meghagyott idõre, a Filiszteusok pedig összegyûlének már Mikmásban,
\par 12 Azt mondám: Mindjárt reám törnek a Filiszteusok Gilgálban, és én az Úrnak színe elõtt még nem imádkozám; bátorságot vevék azért magamnak és megáldozám az égõáldozatot.
\par 13 Akkor monda Sámuel Saulnak: Esztelenül cselekedtél; nem tartottad meg az Úrnak, a te Istenednek parancsolatját, melyet parancsolt néked, pedig most mindörökre megerõsítette volna az Úr a te királyságodat Izráel felett.
\par 14 Most azonban a te királyságod nem lesz állandó. Keresett az Úr magának szíve szerint való embert, a kit az õ népe fölé fejedelmül rendelt, mert te nem tartottad meg, a mit az Úr parancsolt néked.
\par 15 Felkele ezután Sámuel, és elment Gilgálból Benjámin városába, Gibeába. És Saul megszámlálta a népet, mely körülötte található volt, mintegy hatszáz embert.
\par 16 Saul pedig és az õ fia, Jonathán és a nép, mely körülöttök található vala, Benjámin városában, Gébában tartózkodának, és a Filiszteusok Mikmásnál táborozának.
\par 17 Akkor a Filiszteusok táborából egy dúló sereg vált ki három csapatban; az egyik csapat az Ofra felé vivõ útra fordult Suál földének;
\par 18 A másik csapat a Bethoron felé vivõ útra fordula; a harmadik csapat pedig a határ felé vivõ útra fordult, mely a Sebóim völgyén át a pusztáig terjed.
\par 19 És kovácsot egész Izráel földén nem lehetett találni, mert a Filiszteusok azt mondák: Ne csinálhassanak a zsidók szablyát vagy dárdát.
\par 20 És egész Izráelnek a Filiszteusokhoz kellett lemenni, hogy megélesítse ki-ki a maga kapáját, szántóvasát, fejszéjét és sarlóját,
\par 21 (Minthogy megtompulának a kapák, szántóvasak, a háromágú villa és a fejszék) és hogy az ösztökét kiegyenesítsék.
\par 22 Azért az ütközet napján az egész népnél, mely Saullal és Jonathánnal vala, sem szablya, sem dárda nem találtaték, hanem csak Saulnál és az õ fiánál Jonathánnál volt található.
\par 23 És a Filiszteusok elõõrse kijöve Mikmás szorosához.

\chapter{14}

\par 1 És történt egy napon, hogy Jonathán, a Saul fia azt mondá szolgájának, a ki az õ fegyverét hordozza vala: Jer, menjünk át a Filiszteusok elõõrséhez, mely amott túl van; de atyjának nem mondá meg.
\par 2 Saul pedig Gibea határában, a gránátfa alatt tartózkodék, mely Migron nevû mezõn van; és a nép, mely vele volt, mintegy hatszáz ember vala.
\par 3 És Ahija is, ki Silóban az efódot viseli, fia Akhitobnak, az Ikábód testvérének, Fineás fiának, a ki az Úr papjának, Élinek fia vala. És a nép nem tudta, hogy Jonathán eltávozék.
\par 4 A szorosok között pedig, melyeken keresztül akara menni Jonathán a Filiszteusok elõõrséhez, volt egy hegyes kõszikla innen, és volt egy hegyes kõszikla túlfelõl. Az egyiknek Boczécz, a másiknak pedig Sené volt a neve.
\par 5 Az egyik sziklacsúcs északra volt, Mikmás átellenében, és a másik délre, Géba átellenében.
\par 6 És monda Jonathán a szolgának, a ki az õ fegyverét hordozá: Jer, menjünk át ezeknek a körülmetéletleneknek elõõrséhez, talán tenni fog az Úr érettünk valamit, mert az Úr elõtt nincs akadály, hogy sok vagy pedig kevés által szerezzen szabadulást.
\par 7 És fegyverhordozója felele néki: Tégy mindent a te szíved szerint; indulj néki, ímé én veled leszek kívánságod szerint.
\par 8 Jonathán pedig monda: Nosza menjünk fel ez emberekhez, és mutassuk meg magunkat nékik.
\par 9 Ha azt mondják nékünk: Várjatok meg, míg oda érkezünk ti hozzátok, akkor álljunk meg helyünkön, és ne menjünk fel hozzájok;
\par 10 Ha pedig azt mondják: Jertek fel mihozzánk, akkor menjünk fel, mert az Úr kezünkbe adta õket. És ez legyen nékünk a jel.
\par 11 Mikor pedig megmutatták magokat mind a ketten a Filiszteusok elõõrsének, mondának a Filiszteusok: Ímé a zsidók kijõdögélnek a barlangokból, a hova rejtõzének.
\par 12 És szólának némelyek az elõõrs emberei közül Jonathánnak és az õ fegyverhordozójának, és mondák: Jertek fel mi hozzánk, és valamit mondunk néktek. Akkor monda Jonathán az õ fegyverhordozójának: Jõjj fel utánam, mert az Úr Izráel kezébe adta õket.
\par 13 És felmászott Jonathán négykézláb, és utána az õ fegyverhordozója. És hullának vala Jonathán elõtt, és fegyverhordozója is öldököl utána.
\par 14 És az elsõ ütközet, melyben Jonathán és fegyverhordozója mintegy húsz embert ölének meg, egy hold földön egy fél barázda hosszányin volt.
\par 15 És félelem támada a táborban, a mezõn és az egész nép között; az elõõrs és a dúló sereg - azok is megrémülének - és a föld megrendüle; és Istentõl való rettegés lõn.
\par 16 És megláták a Saul õrei Benjámin városában, Gibeában, hogy a sokaság elszéledett és ide-oda elszóratott.
\par 17 Akkor monda Saul a népnek, mely vele volt: Nosza vegyétek számba a népet, és vizsgáljátok meg, ki ment el mi közülünk; és a mikor számba vették, ímé Jonathán és az õ fegyverhordozója nem valának ott.
\par 18 És monda Saul Ahijának: Hozd elõ az Isten ládáját; mert az Isten ládája akkor Izráel fiainál vala.
\par 19 És történt, hogy a míg Saul a pappal beszéle, a Filiszteusok táborában mind nagyobb lõn a zsibongás. Monda azért Saul a papnak: Hagyd abba, a mit kezdettél.
\par 20 És felkiálta Saul és a nép, mely vele volt, és elmenének az ütközetre. Ott pedig egyik a másik ellen harczola, és igen nagy zûrzavar lõn.
\par 21 És azok a zsidók is, kik, mint azelõtt is, a Filiszteusokkal valának, s velök együtt feljövének a táborba és a körül valának, azok is Izráel népéhez csatlakozának, mely Saul és Jonathán mellett vala.
\par 22 És Izráelnek mindazon férfiai, kik elrejtõzének Efraim hegységében, mikor meghallották, hogy a Filiszteusok menekültek, azok is üldözni kezdék õket a harczban.
\par 23 És megsegíté az Úr azon a napon Izráelt. És a harcz Béth-Avenen túl terjede.
\par 24 És Izráel népe igen elepedett vala azon a napon, mert Saul esküvel kényszeríté a népet, mondván: Átkozott az, a ki kenyeret eszik estvéig, míg bosszút állok ellenségeimen, azért az egész nép semmit sem evék.
\par 25 És az egész föld népe eljuta az erdõbe, hol méz vala a föld szinén.
\par 26 Mikor pedig a nép beméne az erdõbe, jóllehet folyt a méz, mindazáltal senki sem érteté kezét szájához, mert félt a nép az eskü miatt.
\par 27 Jonathán azonban nem hallotta vala, hogy atyja megesketé a népet, és kinyújtá a vesszõ végét, mely kezében vala, és bemártá azt a lépesmézbe, és kezét szájához vivé; és felvidulának az õ szemei.
\par 28 Szóla pedig valaki a nép közül, és mondá: Atyád ünnepélyesen megesketé a népet, mondván: Átkozott az, a ki csak kenyeret is eszik ma, és e miatt a nép kimerüle.
\par 29 És monda Jonathán: Atyám bajba vitte az országot; lássátok mennyire felvidulának szemeim, hogy ízlelék egy keveset ebbõl a mézbõl.
\par 30 Hátha még a nép jól evett volna ma ellenségeinek zsákmányából, melyet talált! Vajjon nem nagyobb lett volna-é akkor a Filiszteusok veresége?!
\par 31 És megverék azon a napon a Filiszteusokat Mikmástól Ajálonig. És a nép nagyon kimerüle.
\par 32 Akkor a nép a prédának esék, és fogának juhot, ökröt és borjúkat, és megölék a földön, és megevé a nép vérestõl.
\par 33 Megjelenték azért Saulnak, és mondának: Ímé a nép vétkezik az Úr ellen, mert vérrel elegy eszik. Õ pedig monda: Hûtlenül cselekedtetek, gördítsetek azért most hozzám egy nagy követ.
\par 34 Monda továbbá Saul: Menjetek el mindenfelé a nép között, és mondjátok meg nékik, hogy mindenki a maga ökrét és mindenki a maga juhát hozza én hozzám, és itt öljétek meg és egyétek meg, és nem fogtok vétkezni az Úr ellen, vérrel elegy evén. Akkor elhozta az egész nép, kézzel fogván ki-ki a maga ökrét azon éjjel, és ott megölék.
\par 35 Saul pedig oltárt épített az Úrnak; azt az oltárt építé elõször az Úrnak.
\par 36 És monda Saul: Menjünk le ez éjjel a Filiszteusok után, és foszszuk ki õket virradatig, és senkit se hagyjunk meg közülök. Azok pedig mondának: A mint néked tetszik, úgy cselekedjél mindent. De a pap azt mondá: Járuljunk ide az Istenhez.
\par 37 És megkérdezé Saul az Istent: Lemenjek-é a Filiszteusok után? Izráelnek kezébe adod-é õket? De õ nem felelt néki azon a napon.
\par 38 Monda azért Saul: Jertek ide mindnyájan, a népnek oszlopai, hogy megtudjátok és meglássátok, kiben volt ez a bûn ma?
\par 39 Mert él az Úr, a ki Izráelt oltalmazza, hogy ha fiamban, Jonathánban volna is, meg kell halnia; és az egész nép közül senki sem felele néki.
\par 40 És monda egész Izráelnek: Ti legyetek az egyik oldalon, én pedig és az én fiam, Jonathán legyünk a másik oldalon. És válaszola a nép Saulnak: A mint néked tetszik, úgy cselekedjél.
\par 41 Akkor szóla Saul az Úrnak, Izráel Istenének: Szolgáltass igazságot! És kiválasztaték Jonathán és Saul, a nép pedig megmeneküle.
\par 42 És monda Saul: Vessetek sorsot közöttem és fiam, Jonathán között. És kiválasztaték Jonathán.
\par 43 Akkor monda Saul Jonathánnak: Mondd meg nékem, mit cselekedtél? Jonathán pedig elbeszélte néki, és monda: A pálcza végével, mely kezembe vala, ízlelék egy keveset a mézbõl; itt vagyok, haljak meg!
\par 44 És Saul monda: Úgy cselekedjék az Úr most és ezután is, hogy meg kell halnod Jonathán.
\par 45 A nép azonban monda Saulnak: Jonathán haljon-é meg, a ki ezt a nagy szabadulást szerezte Izráelben? Távol legyen! Él az Úr, hogy egyetlen hajszála sem esik le fejérõl a földre, mert Istennek segedelmével cselekedte ezt ma. Megváltá azért a nép Jonathánt, és nem hala meg.
\par 46 Akkor Saul megtére a Filiszteusok üldözésébõl; a Filiszteusok pedig az õ helyökre haza menének.
\par 47 Miután tehát Saul átvette a királyságot Izráel felett, hadakozék minden ellenségivel mindenfelé: Moáb ellen és Ammon fiai ellen és Edom ellen és a Czobeusok királyai ellen és a Filiszteusok ellen; és mindenütt, a hol megfordula, keményen cselekedék.
\par 48 És sereget gyûjtvén, megverte Amáleket, és megszabadítá Izráelt fosztogatóinak kezébõl.
\par 49 Valának pedig Saul fiai: Jonathán, Jisvi és Málkisua; és két leányának neve: az idõsebbiknek Méráb, és a kisebbiknek neve Mikál.
\par 50 És Saul feleségét Akhinóámnak hívták, a ki Akhimaás leánya volt. Seregének vezetõjét pedig Abnernek hívták, a ki Saul nagybátyjának, Nérnek volt a fia;
\par 51 Mert Kis, a Saul atyja és Nér, az Abner atyja, Abiel fiai valának.
\par 52 A Filiszteusok elleni háború pedig igen heves volt Saulnak egész életében; azért, a hol csak látott Saul egy-egy erõs vagy egy-egy bátor férfit, azt magához fogadta.

\chapter{15}

\par 1 És monda Sámuel Saulnak: Engem küldött el az Úr, hogy királylyá kenjelek fel téged az õ népe, az Izráel felett; most azért figyelj az Úr beszédének szavára.
\par 2 Így szól a Seregek Ura: Megemlékeztem arról, a mit cselekedett Amálek Izráellel, hogy útját állta néki, mikor feljöve Égyiptomból.
\par 3 Most azért menj el és verd meg Amáleket, és pusztítsátok el mindenét; és ne kedvezz néki, hanem öld meg mind a férfit, mind az asszonyt; mind a gyermeket, mind a csecsemõt; mind az ökröt, mind a juhot; mind a tevét, mind a szamarat.
\par 4 És összehívá Saul a népet, és megszámlálá õket Thélaimban, kétszázezer gyalogost, és Júdából tízezer embert.
\par 5 És elméne Saul Amálek városáig, és megütközék ott egy völgyben.
\par 6 És monda Saul a Kéneusnak: Menjetek távozzatok el, menjetek ki az Amálekiták közül, hogy velök együtt téged is el ne veszesselek, mert te irgalmasságot cselekedtél Izráel minden fiaival, mikor Égyiptomból feljövének. És eltávozék Kéneus az Amálekiták közül.
\par 7 Saul pedig megveré Amáleket Havilától fogva egészen addig, a merre Súrba mennek, mely Égyiptom átellenében van.
\par 8 És Agágot, az Amálekiták királyát elfogta élve, a népet pedig mind kardélre hányatá.
\par 9 Saul és a nép azonban megkímélte Agágot és a juhoknak, barmoknak és másodszülötteknek javát; a bárányokat és mindazt, a mi jó vala, nem akarták azokat elpusztítani, hanem a mi megvetett és értéktelen dolog volt, mindazt elpusztíták.
\par 10 Akkor szóla az Úr Sámuelnek, mondván:
\par 11 Megbántam, hogy Sault királylyá tettem, mert eltávozott tõlem, és beszédeimet nem tartotta meg. Sámuel pedig felháborodék és kiálta az Úrhoz egész éjszaka.
\par 12 És korán felkele Sámuel, hogy találkozzék Saullal reggel; és hírül adák Sámuelnek, ezt mondván: Saul Kármelbe ment, és ímé emlékoszlopot állított magának; azután megfordula, és tovább ment és lement Gilgálba,
\par 13 És a mint Sámuel Saulhoz érkezék, monda néki Saul: Áldott vagy te az Úrtól! Én végrehajtám az Úrnak parancsolatját.
\par 14 Sámuel azonban monda: Micsoda az a juhbégetés, mely füleimbe hat és az az ökörbõgés, melyet hallok?
\par 15 És monda Saul: Az Amálekitáktól hozták azokat, mert a nép megkímélte a juhoknak és ökröknek javát, hogy megáldozza az Úrnak, a te Istenednek; a többit pedig elpusztítottuk.
\par 16 Akkor monda Sámuel Saulnak: Engedd meg, hogy megmondjam néked, a mit az Úr mondott nékem ez éjjel. Õ pedig monda néki: Beszélj.
\par 17 Monda azért Sámuel: Nemde kicsiny valál te a magad szemei elõtt is, mindazáltal Izráel törzseinek fejévé lettél, és az Úr királylyá kent fel téged Izráel felett?!
\par 18 És elkülde az Úr téged az úton, és azt mondá: Menj el, és pusztísd el az Amálekitákat, kik vétkezének; és hadakozzál ellenök, míg megsemmisíted õket.
\par 19 Miért nem hallgattál az Úrnak szavára, és miért estél néki a prédának, és cselekedted azt, a mi bûnös az Úr szemei elõtt?
\par 20 És felele Saul Sámuelnek: Én bizonyára hallgattam az Úr szavára, és azon az úton jártam, a melyre engem az Úr elküldött; és elhoztam Agágot, az Amálekiták királyát, és az Amálekitákat elpusztítottam.
\par 21 A nép azonban elvette a prédából a megsemmisítésre rendelt juhoknak és ökröknek javát, hogy megáldozza az Úrnak, a te Istenednek Gilgálban.
\par 22 Sámuel pedig monda: Vajjon kedvesebb-é az Úr elõtt az égõ- és véres áldozat, mint az Úr szava iránt való engedelmesség? Ímé, jobb az engedelmesség a véres áldozatnál és a szófogadás a kosok kövérénél!
\par 23 Mert, mint a varázslásnak bûne, olyan az engedetlenség; és bálványozás és bálványimádás az ellenszegülés. Mivel te megvetetted az Úrnak beszédét, õ is megvetett téged, hogy ne légy király.
\par 24 Akkor monda Saul Sámuelnek: Vétkeztem, mert megszegtem az Úrnak szavát és a te beszédedet; de mivel féltem a néptõl, azért hallgattam szavokra.
\par 25 Most azért bocsásd meg az én vétkemet, és térj vissza velem, hogy könyörögjek az Úrnak.
\par 26 Sámuel pedig monda Saulnak: Nem térek vissza veled, mert megvetetted az Úrnak beszédét, és az Úr is megvetett téged, hogy ne légy király Izráel felett.
\par 27 És mikor megfordula Sámuel, hogy elmenjen, megragadta felsõ ruhájának szárnyát, és leszakada.
\par 28 Akkor monda néki Sámuel: elszakítá tõled az Úr a mai napon Izráelnek királyságát, és adta azt felebarátodnak, a ki jobb náladnál.
\par 29 Izráelnek erõssége pedig nem hazudik, és semmit meg nem bán, mert nem ember õ, hogy valamit megbánjon.
\par 30 És õ monda: Vétkeztem, mindazáltal becsülj meg engem népemnek vénei elõtt és Izráel elõtt, és velem térj vissza, hogy könyörögjek az Úrnak, a te Istenednek.
\par 31 Visszatére azért Sámuel Saullal, és könyörgött Saul az Úrnak.
\par 32 Sámuel pedig monda: Hozzátok ide elõmbe Agágot, Amálek királyát. És elméne Agág kényesen õ hozzá, és monda Agág: Bizonyára eltávozék a halál keserûsége.
\par 33 És monda Sámuel: Miként a te kardod asszonyokat tett gyermektelennekké, úgy legyen gyermektelenné minden asszonyok felett a te anyád! És darabokra vagdalá Sámuel Agágot az Úr elõtt Gilgálban.
\par 34 Ezután Sámuel elment Rámába, Saul pedig felment az õ házához Gibeába, Saul városába.
\par 35 És Sámuel nem látogatá meg többé Sault egész halálának idejéig; de bánkódék Sámuel Saul miatt. Az Úr pedig megbánta, hogy királylyá tette Sault Izráel felett.

\chapter{16}

\par 1 És monda az Úr Sámuelnek: Ugyan meddig bánkódol még Saul miatt, holott én megvetettem õt, hogy ne uralkodjék Izráel felett? Töltsd meg a te szarudat olajjal, és eredj el; én elküldelek téged a Bethlehemben lakó Isaihoz, mert fiai közül választottam magamnak királyt.
\par 2 Sámuel pedig monda: Hogyan menjek el!? Ha meghallja Saul, megöl engemet. És monda az Úr: Vígy magaddal egy üszõt, és azt mondjad: Azért jöttem, hogy az Úrnak áldozzam.
\par 3 És hívd meg Isait az áldozatra, és én tudtodra adom, hogy mit cselekedjél, és kend fel számomra azt, a kit mondándok néked.
\par 4 És Sámuel megcselekedé, a mit az Úr mondott néki, és elment Bethlehembe. A város vénei pedig megijedének, és eleibe menvén, mondának: Békességes-é a te jöveteled?
\par 5 Õ pedig felele: Békességes; azért jöttem, hogy áldozzam az Úrnak. Szenteljétek meg azért magatokat, és jertek el velem az áldozatra. Isait és az õ fiait pedig megszentelé, és elhívá õket az áldozatra.
\par 6 Mikor pedig bemenének, meglátta Eliábot, és gondolá: Bizony az Úr elõtt van az õ felkentje!
\par 7 Az Úr azonban monda Sámuelnek: Ne nézd az õ külsõjét, se termetének nagyságát, mert megvetettem õt. Mert az Úr nem azt nézi, a mit az ember; mert az ember azt nézi, a mi szeme elõtt van, de az Úr azt nézi, mi a szívben van.
\par 8 Szólítá azért Isai Abinádábot, és elvezeté õt Sámuel elõtt; õ pedig monda: Ez sem az, kit az Úr választa.
\par 9 Elvezeté azután elõtte Isai Sammát; õ pedig monda: Ez sem az, a kit az Úr választa.
\par 10 És így elvezeté Isai Sámuel elõtt mind a hét fiát; Sámuel pedig mondá Isainak: Nem ezek közül választott az Úr.
\par 11 Akkor monda Sámuel Isainak: Mind itt vannak-é már az ifjak? Õ pedig felele: Hátra van még a kisebbik, és ímé õ a juhokat õrzi. És monda Sámuel Isainak: Küldj el, és hozasd ide õt, mert addig nem fogunk leülni, míg õ ide nem jön.
\par 12 Elkülde azért, és elhozatá õt. (Õ pedig piros vala, szép szemû és kedves tekintetû.) És monda az Úr: Kelj fel és kend fel, mert õ az.
\par 13 Vevé azért Sámuel, az olajos szarut, és felkené õt testvérei között. És attól a naptól fogva az Úrnak lelke Dávidra szálla, és azután is. Felkele azután Sámuel és elméne Rámába.
\par 14 És az Úrnak lelke eltávozék Saultól, és gonosz lélek kezdé gyötörni õt, mely az Úrtól küldetett.
\par 15 És mondának Saul szolgái néki: Ímé most az Istentõl küldött gonosz lélek gyötör téged!
\par 16 Parancsoljon azért a mi urunk szolgáidnak, kik körülötted vannak, hogy keressenek olyan embert, a ki tudja a hárfát pengetni, és mikor az Istentõl küldött gonosz lélek reád jön, pengesse kezével, hogy te megkönnyebbülj.
\par 17 És monda Saul az õ szolgáinak: Keressetek tehát számomra olyan embert, a ki jól tud hárfázni, és hozzátok el hozzám.
\par 18 Akkor felele egy a szolgák közül, és monda: Ímé én láttam a Bethlehemben lakó Isainak egyik fiát, a ki tud hárfázni, a ki erõs vitéz és hadakozó férfiú, értelmes és szép ember, és az Úr vele van.
\par 19 Követeket külde azért Saul Isaihoz, és monda: Küldd hozzám a fiadat, Dávidot, ki a juhok mellett van.
\par 20 Isai pedig võn egy szamarat, egy kenyeret, egy tömlõ bort és egy kecskegödölyét, és elküldé Saulnak az õ fiától, Dávidtól.
\par 21 Mikor pedig Dávid elméne Saulhoz és megálla elõtte, az igen megszerette õt, és fegyverhordozója lõn néki.
\par 22 És elkülde Saul Isaihoz, mondván: Maradjon Dávid én nálam, mert igen megkedveltem õt.
\par 23 És lõn, hogy a mikor Istennek lelke Saulon vala, vette Dávid a hárfát és kezével pengeté; Saul pedig megkönnyebbüle és jobban lõn, és a gonosz lélek eltávozék tõle.

\chapter{17}

\par 1 És összegyûjték a Filiszteusok seregeiket a harczra, és összegyûlének Sokónál, mely Júdában van, és táborozának Sokó és  Azéka között, Efes-Dammimnál.
\par 2 Saul és az Izráeliták pedig összegyûlének, és tábort ütének az Elah völgyében; és csatarendbe állának a Filiszteusok ellen.
\par 3 És a Filiszteusok a hegyen állottak innen, az Izráeliták pedig a hegyen állottak túlfelõl, úgy hogy a völgy közöttük vala.
\par 4 És kijöve a Filiszteusok táborából egy bajnok férfiú, a kit Góliáthnak hívtak, Gáth városából való, kinek magassága hat sing és egy arasz vala.
\par 5 Fején rézsisak vala és pikkelyes pánczélba volt öltözve; a pánczél súlya pedig ötezer rézsiklusnyi vala.
\par 6 Lábán réz lábpánczél és vállain rézpaizs volt.
\par 7 És dárdájának nyele olyan volt, mint a takácsok zugolyfája, dárdájának hegye pedig hatszáz siklusnyi vasból volt; és elõtte megy vala, ki a paizst hordozza vala.
\par 8 És megállván, kiálta Izráel csatarendjeinek, és monda nékik: Miért jöttetek ki, hogy harczra készüljetek? Avagy nem Filiszteus vagyok-é én és ti Saul szolgái? Válaszszatok azért magatok közül egy embert, és jõjjön le hozzám.
\par 9 Ha azután meg bír velem vívni és engem legyõz: akkor mi a ti szolgáitok leszünk; ha pedig én gyõzõm le õt és megölöm: akkor ti legyetek a mi szolgáink, hogy szolgáljatok nékünk.
\par 10 Monda továbbá a Filiszteus: Én gyalázattal illetém a mai napon Izráel seregét, állítsatok azért ki ellenem egy embert, hogy megvívjunk egymással.
\par 11 Mikor pedig meghallotta Saul és az egész Izráel a Filiszteusnak ezt a beszédét, megrettenének és igen félnek vala.
\par 12 És Dávid, Júda városából, Bethlehembõl való amaz Efratita embernek volt a fia, a kit Isainak hívtak, a kinek nyolcz fia volt, és e férfiú a Saul idejében vén ember vala, emberek közt korban elõhaladt.
\par 13 És Isainak három idõsebbik fia Saullal elment a háborúba. Az õ három fiának pedig, a kik a háborúba menének, ezek valának neveik: az idõsebbik Eliáb, a második Abinádáb és a harmadik Samma.
\par 14 És Dávid volt a legkisebbik. Mikor pedig a három idõsebb elment Saul után:
\par 15 Dávid elméne és visszatére Saultól, hogy atyjának juhait õrizze Bethlehemben.
\par 16 A Filiszteus pedig elõjön vala reggel és estve, és kiáll vala negyven napon át.
\par 17 És monda Isai az õ fiának, Dávidnak: Vedd testvéreid számára ezt az efa pergelt búzát és ezt a tíz kenyeret, és sietve vidd el a táborba testvéreidhez.
\par 18 Ezt a tíz sajtot pedig vidd el az ezredesnek, és látogasd meg testvéreidet, hogy jól vannak-é, és hozz tõlük jelt.
\par 19 Saul pedig azokkal együtt és az egész Izráel az Elah völgyében valának, hogy harczolnának a Filiszteusok ellen.
\par 20 Felkele azért Dávid korán reggel, és a nyájat egy pásztorra bízván, felvette a terhet és elment, a mint meghagyta néki Isai; és eljutott a tábor kerítéséhez; a sereg pedig, mely kivonult csatarendben, hadi zajt támasztott.
\par 21 És csatarendbe állának Izráel és a Filiszteusok, csatarend csatarend ellen.
\par 22 Akkor Dávid rábízta a holmit arra, a ki a hadi szerszámokat õrzi, és elfuta a harcztérre és odaérve, kérdezõsködék testvéreinek állapota felõl.
\par 23 És míg õ beszélt velök, ímé a bajnok férfi, a Góliáth nevû Filiszteus, a ki Gáthból való volt, elõjöve a Filiszteusok csatarendei közül, és most is hasonlóképen beszél vala; és meghallá ezt Dávid.
\par 24 Az Izráeliták pedig, mikor látták azt a férfit, mindnyájan elfutának elõle, és igen félnek vala.
\par 25 És mondának az Izráeliták: Láttátok-é azt a férfit, a ki feljöve? Mert azért jött ki, hogy gyalázattal illesse Izráelt. Ha valaki megölné õt, nagy gazdagsággal ajándékozná meg a király, leányát is néki adná, és atyjának házát szabaddá tenné Izráelben.
\par 26 És szóla Dávid azoknak az embereknek, a kik ott állának vele, mondván: Mi történik azzal, a ki megöli ezt a Filiszteust, és elveszi a gyalázatot Izráelrõl? Mert kicsoda ez a körülmetéletlen Filiszteus, hogy gyalázattal illeti az élõ Istennek seregét?!
\par 27 A nép pedig e beszéd szerint felele néki, mondván: Ez történik azzal az emberrel, a ki megöli õt.
\par 28 És meghallá Eliáb az õ nagyobbik testvére, hogy az emberekkel beszéle; és nagyon megharaguvék Eliáb Dávidra, és monda: Miért jöttél ide, és kire bíztad azt a néhány juhot, a mely a pusztában van? Ismerem vakmerõségedet és szívednek álnokságát, hogy csak azért jöttél ide, hogy megnézd az ütközetet!
\par 29 Dávid pedig felele: Ugyan mit cselekedtem én most? hiszen csak szóbeszéd volt ez.
\par 30 És elfordula tõle egy másikhoz, és ugyan úgy szóla, mint korábban, és a nép is az elõbbi beszéd szerint válaszola néki.
\par 31 És mikor meghallották azokat a szavakat, a melyeket Dávid szóla, megmondák Saulnak, ki magához hívatá õt.
\par 32 És monda Dávid Saulnak: Senki se csüggedjen el e miatt; elmegy a te szolgád és megvív ezzel a Filiszteussal.
\par 33 Saul pedig monda Dávidnak: Nem mehetsz te e Filiszteus ellen, hogy vele megvívj, mert te gyermek vagy, õ pedig ifjúságától fogva hadakozó férfi vala.
\par 34 És felele Dávid Saulnak: Pásztor volt a te szolgád, atyjának juhai mellett; és ha eljött az oroszlán és a medve, és elragadott egy bárányt a nyáj közül:
\par 35 Elmentem utána és levágtam, és kiszabadítám szájából; ha pedig ellenem támadott: megragadtam szakálánál fogva, és levágtam és megöltem õt.
\par 36 A te szolgád mind az oroszlánt, mind a medvét megölte: Úgy lesz azért e körülmetéletlen Filiszteus is, mint azok közül egy, mert gyalázattal illeté az élõ Istennek seregét.
\par 37 És monda Dávid: Az Úr, a ki megszabadított engem az oroszlánnak és a medvének kezébõl, meg fog szabadítani engem e Filiszteusnak kezébõl is. Akkor monda Saul Dávidnak: Eredj el, és az Úr legyen veled!
\par 38 És felöltözteté Saul Dávidot a maga harczi ruhájába; rézsisakot tett a fejére, és felöltözteté õt pánczélba.
\par 39 Akkor Dávid felköté kardját, harczi ruhája fölé, és járni akart, mert még nem próbálta. És monda Dávid Saulnak: Nem bírok ezekben járni, mert még nem próbáltam; és leveté azokat Dávid magáról.
\par 40 És kezébe vette botját, és kiválasztván magának a patakból öt síma kövecskét, eltevé azokat pásztori szerszámába, mely vele volt, tudniillik tarisznyájába, és parittyájával kezében közeledék a Filiszteushoz.
\par 41 Akkor elindult a Filiszteus is, és Dávidhoz közeledék, az az ember pedig, a ki a paizsát hordozza, elõtte vala.
\par 42 Mikor pedig oda tekinte a Filiszteus, és meglátta Dávidot, megvetette õt, mert ifjú volt és piros, egyszersmind szép tekintetû.
\par 43 És monda a Filiszteus Dávidnak: Eb vagyok-é én, hogy te bottal jössz reám? És szidalmazá a Filiszteus Dávidot Istenével együtt.
\par 44 Monda továbbá a Filiszteus Dávidnak: Jõjj ide hozzám, hogy testedet az égi madaraknak és a mezei vadaknak adjam.
\par 45 Dávid pedig monda a Filiszteusnak: Te karddal, dárdával és paizszsal jössz ellenem, én pedig a Seregek Urának, Izráel seregei Istenének nevében megyek ellened, a kit te gyalázattal illetél.
\par 46 A mai napon kezembe ad téged az Úr, és megöllek téged, és fejedet levágom rólad. A Filiszteusok seregének tetemét pedig az égi madaraknak és a mezei vadaknak fogom adni a mai napon, hogy tudja meg az egész föld, hogy van Izráelnek Istene.
\par 47 És tudja meg ez az egész sokaság, hogy nem kard által és nem dárda által tart meg az Úr, mert az Úré a had, és õ titeket kezünkbe fog adni.
\par 48 És mikor a Filiszteus felkészült, és elindult, és Dávid felé közeledék: Dávid is sietett és futott a viadalra a Filiszteus elé.
\par 49 És Dávid benyúlt kezével a tarisznyába és kivett onnan egy követ, és elhajítván, homlokán találta a Filiszteust, úgy, hogy a kõ homlokába mélyede, és arczczal a földre esék.
\par 50 Így Dávid erõsebb volt a Filiszteusnál, parittyával és kõvel. És levágta a Filiszteust és megölte õt, pedig kard nem is vala a Dávid kezében.
\par 51 És oda futott Dávid, és reá állott a Filiszteusra, és vevé annak kardját, kirántotta hüvelyébõl, és megölé õt, és fejét azzal levágta. A Filiszteusok pedig a mint meglátták, hogy az õ hõsük meghalt, megfutamodának.
\par 52 És felkelének Izráel és Júda férfiai és felkiáltának, és üldözék a Filiszteusokat egészen Gáthig és Ekron kapujáig. És hullának a Filiszteusok sebesültjei és a Saraim felé vezetõ úton Gáthig és Ekronig.
\par 53 Visszatérének azután Izráel fiai a Filiszteusok üldözésébõl, és feldúlták azoknak táborát.
\par 54 Dávid pedig felvevé a Filiszteusnak fejét, és elvitte Jeruzsálembe, fegyvereit pedig a maga sátorába rakta le.
\par 55 Saul pedig mikor látta, hogy Dávid kiment a Filiszteus elé, monda Abnernek, a sereg fõvezérének: Abner! ki fia e gyermek? És felele Abner: Él a te lelked oh király, hogy nem tudom!
\par 56 És monda a király: Kérdezd meg tehát, hogy ki fia ez az ifjú?
\par 57 És a mint visszajött Dávid, miután megölte a Filiszteust, megfogá õt Abner, és Saulhoz vitte; és a Filiszteusnak feje kezében vala.
\par 58 És monda néki Saul: Ki fia vagy te, oh gyermek? Dávid pedig felele: A te szolgádnak, a Bethlehembõl való Isainak a fia vagyok.

\chapter{18}

\par 1 Minekutána pedig elvégezte a Saullal való beszélgetést, a Jonathán lelke egybeforrt a Dávid lelkével, és Jonathán úgy szerette õt, mint a saját lelkét.
\par 2 És Saul magához vevé õt azon a napon, és nem engedé, hogy visszatérjen az atyja házához.
\par 3 És szövetséget kötének Jonathán és Dávid egymással, mivel úgy szerette õt, mint a saját lelkét.
\par 4 És Jonathán leveté felsõ ruháját, a mely rajta volt, és Dávidnak adta, sõt hadi öltözetét is, saját kardját, kézívét és övét.
\par 5 És elméne Dávid mindenüvé, a hová Saul küldé, és magát eszesen viseli vala. És Saul a harczosok fölé tevé õt, és kedves lõn az egész nép elõtt, és a Saul szolgái elõtt is.
\par 6 És a mint hazafelé jövének, mikor Dávid visszatért, miután a Filiszteusokat leverte, kimentek az asszonyok Saul király elé Izráelnek minden városaiból, hogy énekeljenek és körben tánczoljanak, dobokkal, vígassággal és tomborákkal.
\par 7 És énekelni kezdének az asszonyok, kik vígadozának és mondának: Megverte Saul az õ ezerét és Dávid is az õ tízezerét.
\par 8 Saul pedig igen megharaguvék, és gonosznak tetszék az õ szemei elõtt ez a beszéd, és monda: Dávidnak tízezeret tulajdonítanak és nékem tulajdonítják az ezeret, így hát már csak a királyság hiányzik néki.
\par 9 Saul azért attól a naptól kezdve rossz szemmel néz vala Dávidra, sõt azután is.
\par 10 Másnap pedig megszállta Sault az Istentõl küldött gonosz lélek, és prófétálni kezde a maga házában; Dávid pedig hárfázott kezével, mint naponként szokta, és a dárda Saul kezében vala.
\par 11 És elhajtá Saul a dárdát, azt gondolván: Dávidot a falhoz szegezem; de Dávid két ízben is félrehajolt elõle.
\par 12 És félni kezde Saul Dávidtól, mert az Úr vele volt, Saultól pedig eltávozék.
\par 13 És Saul elbocsátá õt magától és ezredesévé tevé; és kimegy vala és bejõ vala a nép elõtt.
\par 14 És Dávid minden útjában magát eszesen viseli vala, mert az Úr vele volt.
\par 15 Mikor pedig látta Saul, hogy õ igen eszesen viseli magát, félni kezde tõle.
\par 16 De az egész Izráel és Júda szereté Dávidot, mert õ elõttük méne ki és jöve be.
\par 17 És monda Saul Dávidnak; Ímé idõsebbik leányomat, Mérábot néked adom feleségül, csak légy az én vitéz fiam, és harczold az Úrnak harczait; mert azt gondolá Saul: Ne az én kezem által vesszen el, hanem a Filiszteusok keze által.
\par 18 Dávid pedig monda Saulnak: Kicsoda vagyok én, és micsoda az én életem, és atyámnak családja Izráelben, hogy a királynak veje legyek?
\par 19 De történt abban az idõben, mikor Mérábot, a Saul leányát Dávidnak kellett volna adni, hogy a Meholáthból való Hadrielnek adták õt feleségül.
\par 20 Mikál, a Saul leánya azonban megszereté Dávidot, és mikor ezt megmondák Saulnak, tetszik néki a dolog.
\par 21 És monda Saul: Néki adom õt, hogy õ legyen veszedelmére és a Filiszteusok keze legyen ellene. Monda azért Saul Dávidnak másodízben: Légy tehát most az én võm.
\par 22 És megparancsolá Saul az õ szolgáinak: Beszéljetek Dáviddal titokban, mondván: Ímé a király jóindulattal van irántad, és szolgái is mind szeretnek téged, légy azért veje a királynak.
\par 23 És elmondák a szolgák Dávid elõtt e beszédeket. Dávid pedig monda: Olyan kicsiny dolog elõttetek, hogy a király vejévé legyen valaki, holott én szegény és megvetett ember vagyok?
\par 24 És megmondák Saulnak az õ szolgái, mondván: Ezt meg ezt mondta Dávid.
\par 25 Saul pedig monda: Mondjátok meg Dávidnak: Nem kiván a király más jegyajándékot, hanem csak száz Filiszteus elõbõrét, hogy bosszút állj a király ellenségein; mert Saul a Filiszteusok keze által akará Dávidot elpusztítani.
\par 26 Megmondák azért az õ szolgái Dávidnak e beszédeket; és tetszék ez a dolog Dávidnak, hogy a király veje legyen. A kitûzött napok még el sem telének,
\par 27 Mikor Dávid felkelt, és elment embereivel együtt, és levágott a Filiszteusok közül kétszáz férfit; és elhozá Dávid elõbõreiket, és mind beadta azokat a királynak, hogy a királynak veje lehessen. És néki adá Saul az õ leányát, Mikált, feleségül.
\par 28 Mikor pedig Saul látta és megtudta, hogy az Úr Dáviddal van, és Mikál, a Saul leánya szereti õt:
\par 29 Akkor Saul még inkább félni kezde Dávidtól. És Saul ellensége lõn Dávidnak teljes életében.
\par 30 A Filiszteusok vezérei pedig gyakran betörnek vala, de valahányszor betörének, Dávid Saul minden szolgáinál eszesebben viselé magát; azért felette híressé lõn az õ neve.

\chapter{19}

\par 1 És szóla Saul fiának, Jonathánnak, és a többi szolgáinak, hogy öljék meg Dávidot; de Jonathán, a Saul fia nagyon szereté õt.
\par 2 Megmondá azért Jonathán Dávidnak, mondván: Az én atyám, Saul azon van, hogy téged megöljön, azért vigyázz magadra reggel; titkos helyen tartózkodjál és rejtsd el magad.
\par 3 Én pedig kimegyek, és atyám mellett megállok a mezõn, a hol te leszesz, és beszélni fogok atyámmal felõled, és meglátom, mint lesz, és tudtodra adom néked.
\par 4 És Jonathán kedvezõen nyilatkozék Dávid felõl az õ atyja, Saul elõtt, és monda néki: Ne vétkezzék a király Dávid ellen, az õ szolgája ellen, mert õ nem vétett néked, sõt szolgálata felette hasznos volt néked.
\par 5 Mert õ koczkára tette életét és megverte a Filiszteust, és az Úr nagy szabadulást szerze az egész Izráelnek. Te láttad azt és örültél rajta; miért vétkeznél azért az ártatlan vér ellen, megölvén Dávidot ok nélkül.
\par 6 És hallgatott Saul Jonathán szavára, és megesküvék Saul: Él az Úr, hogy nem fog megöletni!
\par 7 Akkor szólítá Jonathán Dávidot, és megmondá néki Jonathán mind e beszédeket; és Saulhoz vezeté Jonathán Dávidot, a ki ismét olyan lõn elõtte, mint annakelõtte.
\par 8 A háború pedig ismét megkezdõdék, és Dávid kivonula, és harczola a Filiszteusok ellen, és nagy vereséget okozott nékik, és azok elfutának elõle.
\par 9 Az Istentõl küldött gonosz lélek azonban megszállta Sault, mikor házában ült és dárdája kezében vala; Dávid pedig pengeté a hárfát kezével.
\par 10 Akkor Saul a dárdával Dávidot a falhoz akará szegezni, de félrehajolt Saul elõl, és a dárda a falba verõdött. Dávid pedig elszalada, és elmenekült azon éjjel.
\par 11 És követeket külde Saul a Dávid házához, hogy reá lessenek és reggel megöljék õt. De tudtára adá Dávidnak Mikál, az õ felesége, mondván: Ha meg nem mented életedet ez éjjel, holnap megölnek.
\par 12 És lebocsátá Mikál Dávidot az ablakon; õ pedig elment és elszalada, és megmenté magát.
\par 13 És vevé Mikál a theráfot és az ágyba fekteté azt, és feje alá kecskeszõrbõl készült párnát tett, és betakará lepedõvel.
\par 14 Mikor pedig Saul elküldé a követeket, hogy Dávidot megfogják, azt mondá: Dávid beteg.
\par 15 És Saul elküldé ismét a követeket, hogy megnézzék Dávidot, mondván: Ágyastól is hozzátok õt elõmbe, hogy megöljem õt.
\par 16 És mikor a követek oda menének: ímé, a theráf volt az ágyban, és feje alatt a kecskeszõrbõl készült párna volt.
\par 17 Akkor monda Saul Mikálnak: Mi dolog, hogy engem úgy megcsaltál? - elbocsátád az én ellenségemet, és õ elmenekült. És felele Mikál Saulnak: Õ mondá nékem, bocsáss el engem, vagy megöllek téged.
\par 18 Dávid pedig elfutván, megszabadula; és elment Sámuelhez Rámába, és elbeszélte néki mindazt, a mit Saul vele cselekedett. Elméne ezután õ és Sámuel, és Nájóthban tartózkodának.
\par 19 És tudtára adák Saulnak, mondván: Ímé Dávid Nájóthban van, Rámában.
\par 20 Követeket külde azért Saul, hogy Dávidot fogják meg. A mint azonban meglátták a prófétáknak seregét, a kik prófétálának, és Sámuelt, a ki ott állott, mint az õ elõljárójuk; akkor az Istennek lelke Saul követeire szállott, és azok is prófétálának.
\par 21 Mikor pedig megmondták Saulnak, más követeket külde, és azok is prófétálának. Akkor harmadízben is követeket külde Saul, de azok is prófétálának.
\par 22 Elméne azért õ maga is Rámába. És a mint a nagy kúthoz érkezék, mely Székuban van, megkérdezé, mondván: Hol van Sámuel és Dávid? És felelének: Ímé Nájóthban, Rámában.
\par 23 És elméne oda Nájóthba, Rámában. És az Istennek lelke szálla õ reá is, és folytonosan prófétála, míg eljutott Nájóthba, Rámában.
\par 24 És leveté õ is ruháit, és prófétála, õ is Sámuel elõtt és ott feküvék meztelenül azon az egész napon és egész éjszakán. Azért mondják: Avagy Saul is a próféták közt van-é?

\chapter{20}

\par 1 Elfuta azért Dávid Nájóthból, mely Rámában van, és elméne és monda Jonathánnak: Mit cselekedtem? Mi vétkem van és mi bûnöm atyád elõtt, hogy életemre tör?
\par 2 Õ pedig monda néki: Távol legyen! Te nem fogsz meghalni. Ímé az én atyám nem cselekszik sem nagy, sem kicsiny dolgot, hogy nékem meg ne mondaná. Miért titkolná el azért atyám elõlem ezt a dolgot? Nem úgy van!
\par 3 Mindazáltal Dávid még megesküvék, és monda: Bizonyára tudja a te atyád, hogy te kedvelsz engem, azért azt gondolá: Ne tudja ezt Jonathán, hogy valamikép meg ne szomorodjék. De bizonyára él az Úr és él a te lelked, hogy alig egy lépés van köztem és a halál között.
\par 4 És felele Jonathán Dávidnak: A mit lelked kiván, megteszem éretted.
\par 5 És monda Dávid Jonathánnak: Ímé holnap újhold lesz, mikor a királylyal kellene leülnöm, hogy egyem, de te bocsáss el engem, hogy elrejtõzzem a mezõn a harmadik nap estvéjéig.
\par 6 Ha kérdezõsködnék atyád utánam, ezt mondjad: Sürgõsen kéredzett Dávid tõlem, hogy elmehessen Bethlehembe, az õ városába, mert ott az egész nemzetségnek esztendõnként való áldozatja van most.
\par 7 Ha azt fogja mondani: Jól van, úgy békessége van a te szolgádnak; ha pdig nagyon megharagudnék, úgy tudd meg, hogy a gonosz tettre elhatározta magát.
\par 8 Cselekedjél azért irgalmasságot a te szolgáddal, mert az Úr elõtt szövetséget kötöttél én velem, a te szolgáddal. Ha azonban gonoszság van bennem, ölj meg te; miért vinnél atyádhoz engemet?
\par 9 Jonathán pedig felelé: Távol legyen az tõled! Ha bizonyosan megtudom, hogy atyám elhatározta magát arra, hogy a gonosz tettet rajtad végrehajtsa, avagy nem mondanám-é meg azt néked?
\par 10 És monda Dávid Jonathánnak: Kicsoda adja nékem tudtomra, hogy a mit atyád felelni fog néked, szigorú-é?
\par 11 És monda Jonathán Dávidnak: Jer, menjünk ki a mezõre; és kimenének mindketten a mezõre.
\par 12 Akkor monda Jonathán Dávidnak: Az Úr az Izráelnek Istene; ha kipuhatolhatom atyámtól holnap ilyenkor vagy holnapután, hogy ímé Dávid iránt jó akarattal van, tehát nem küldök-é ki akkor hozzád és jelentem-é meg néked?
\par 13 Úgy cselekedjék az Úr Jonathánnal most és azután is, ha atyámnak az tetszenék, hogy gonoszszal illessen téged: hogy tudtodra adom néked, és elküldelek téged, hogy békében elmehess. És az Úr legyen veled, mint volt az én atyámmal!
\par 14 És ne csak a míg én élek, és ne csak magammal cselekedjél az Úrnak irgalmassága szerint, hogy meg ne haljak;
\par 15 Hanem meg ne vond irgalmasságodat az én házamtól soha, még akkor se, hogyha az Úr kiirtja Dávid ellenségeit, mindegyiket a földnek színérõl!
\par 16 Így szerze szövetséget Jonathán a Dávid házával; mondván: vegyen számot az Úr a Dávid ellenségeitõl.
\par 17 És Jonathán még egyszer megesketé Dávidot, iránta való szeretetébõl; mert úgy szerette õt, mint a saját lelkét.
\par 18 Monda pedig néki Jonathán: Holnap újhold lesz, és kérdezõsködni fognak utánad, mert helyed üres leend.
\par 19 A harmadik napon pedig jõjj alá gyorsan, és eredj arra a helyre, a hol elrejtõzél amaz esemény napján, és maradj ott az útmutató kõ mellett.
\par 20 És én három nyilat lövök oldalához, mintha magamtól czélba lõnék.
\par 21 És ímé utánuk küldöm a gyermeket: Eredj, keresd meg a nyilakat. Ha azt mondom a gyermeknek: Ímé mögötted vannak emerre: hozd el azokat és jõjj elõ, mert békességed van néked, és nincs baj, él az Úr!
\par 22 Ha pedig azt mondom a gyermeknek: Ímé elõtted vannak a nyilak amarra: akkor menj el, mert elküldött téged az Úr.
\par 23 És erre a dologra nézve, a melyet megbeszéltünk egymás közt, ímé az Úr legyen bizonyság közöttem és közötted mind örökké!
\par 24 Elrejtõzék azért Dávid a mezõn. És mikor az újhold eljött, leült a király az ebédhez, hogy egyék.
\par 25 És mikor leült a király a maga székébe, most is úgy, mint máskor, a fal mellett levõ székbe: Jonathán felkele, és Abner ült Saul mellé; a Dávid helye pedig üres vala.
\par 26 És Saul semmit sem szólott azon a napon, mert azt gondolá: Valami történt vele; nem tiszta, bizonyosan nem tiszta.
\par 27 És lõn az újhold után következõ napon, a második napon, mikor ismét üres volt a Dávid helye, monda Saul az õ fiának, Jonathánnak: Isainak fia miért nem jött el az ebédre sem tegnap, sem ma?
\par 28 Jonathán pedig felele Saulnak: Elkéredzék tõlem Dávid Bethlehembe;
\par 29 És monda: Ugyan bocsáss el engem, mert nemzetségünknek áldozata van most a városban, és ezt parancsolta nékem bátyám; azért, ha kedvelsz engem, kérlek, hadd menjek el, hogy megnézzem testvéreimet. Ezért nem jött el a király asztalához.
\par 30 Akkor felgerjede Saulnak haragja Jonathán ellen és monda néki: Te elfajult, engedetlen gyermek! Jól tudom, hogy kiválasztottad az Isainak fiát a magad gyalázatára és anyád szemérmének gyalázatára!
\par 31 Mert mindaddig, míg Isainak fia él a földön, nem állhatsz fenn sem te, sem a te királyságod; most azért küldj érette, és hozasd ide õt hozzám, mert õ a halál fia.
\par 32 Jonathán pedig felele Saulnak, az õ atyjának, és monda néki: Miért kell meghalnia, mit vétett?
\par 33 Akkor Saul utána dobta dárdáját, hogy általüsse õt. És megérté Jonathán, hogy atyja elvégezé, hogy megölje Dávidot.
\par 34 És felkele Jonathán az asztaltól nagy haraggal, és semmit sem evék az újholdnak második napján, mert bánkódott Dávid miatt, mivel atyja gyalázattal illeté õt.
\par 35 És reggel kiméne Jonathán a mezõre a Dáviddal együtt meghatározott idõben, és egy kis gyermek volt vele.
\par 36 És monda a gyermeknek: Eredj, keresd meg a nyilakat, a melyeket ellövök. És mikor a gyermek elfutott, ellövé a nyilat, úgy hogy rajta túl méne.
\par 37 És mikor a gyermek arra a helyre érkezék, a hol a nyíl vala, melyet Jonathán ellõtt, a gyermek után kiálta Jonathán, és monda: Avagy nem tovább van-é a nyíl elõtted?
\par 38 És kiálta Jonathán a gyermek után: Gyorsan siess, meg ne állj! És a gyermek, ki Jonathánnal vala, felszedé a nyilat és urához ment.
\par 39 A gyermek pedig semmit sem értett, hanem csak Jonathán és Dávid értették a dolgot.
\par 40 Átadá azután Jonathán fegyverét a gyermeknek, a ki vele volt, és monda néki: Eredj el, vidd be a városba.
\par 41 Mikor pedig elment a gyermek, felkele Dávid a kõ déli oldala mellõl és arczczal a földre borula, és háromszor meghajtotta magát; és megcsókolták egymást, és együtt sírtak, mígnem Dávid hangosan zokogott.
\par 42 Akkor monda Jonathán Dávidnak: Eredj el békességgel! Mivelhogy megesküdtünk mind a ketten az Úrnak nevére, mondván: Az Úr legyen köztem és közted, az én magom között és a te magod között örökre.
\par 43 Felkele ezután és elméne. Jonathán pedig bement a városba.

\chapter{21}

\par 1 És Dávid elméne Nóbba Akhimélek paphoz. Akhimélek pedig megrettenve ment Dávid elé, és monda néki: Mi dolog, hogy csak egyedül vagy, és senki sincs veled?
\par 2 És monda Dávid Akhimélek papnak: A király bízott reám valamit, és monda nékem: Senki se tudja meg azt a dolgot, a miért elküldélek téged, és a mit parancsoltam néked: azért a szolgákat elküldém erre és erre a helyre.
\par 3 Most azért, mi van kezednél? Adj öt kenyeret nékem, vagy egyebet, a mi van.
\par 4 És felele a pap Dávidnak, és monda: nincs közönséges kenyér kezemnél, hanem csak szentelt kenyér van, ha ugyan a szolgák tisztán tartották magokat, legalább az  asszonytól.
\par 5 Dávid pedig felele a papnak, és monda néki: Bizonyára el volt tiltva mi tõlünk az asszony mind tegnap, mind azelõtt, mikor elindulék, és a szolgák holmija is tiszta vala (jóllehet az út közönséges): azért bizonyára megtartatik ma szentnek az edényekben.
\par 6 Adott azért a pap néki szentelt kenyeret, mert nem volt ott más kenyér, hanem csak szent kenyér, melyeket elvettek az Úrnak színe elõl, hogy meleg kenyeret tegyenek a helyett azon a napon, a melyen az elõbbit elvevék.
\par 7 Vala pedig ott azon a napon Saul szolgái közül egy ember, ott tartózkodva az Úr elõtt, a kit Doégnak hívtak, a ki Edomita volt, Saul pásztorainak számadója.
\par 8 És monda Dávid Akhiméleknek: Nincsen-é kezednél egy dárda vagy valami fegyver? mert sem kardomat, sem fegyverzetemet nem hoztam magammal, mivel a király dolga sürgõs vala.
\par 9 És monda a pap: A Filiszteus Góliáthnak a kardja, a kit te megöltél az Elah völgyében, ímhol van posztóba betakarva az efód mögött; ha azt el akarod vinni, vidd el, mert azonkivül más nincsen itt. És monda Dávid: Nincs ahhoz hasonló, add ide azt nékem.
\par 10 És felkele Dávid, és elfutott azon a napon Saul elõl, és elment Ákhishoz, Gáthnak királyához.
\par 11 És mondának Ákhis szolgái néki: Vajjon nem ez-é Dávid, annak az országnak királya? Vajjon nem errõl énekelték-é a körtánczban, mondván: Saul megverte az õ ezerét, Dávid is az õ tízezerét?
\par 12 És mikor eszébe vevé Dávid ezeket a beszédeket, igen megrémüle Ákhistól, Gáthnak királyától.
\par 13 És megváltoztatá magaviseletét õ elõttük, és õrjönge kezeik között, és irkál vala a kapuknak ajtain, nyálát pedig szakállán folyatja alá.
\par 14 És monda Ákhis az õ szolgáinak: Ímé látjátok, hogy ez az ember megõrült, miért hoztátok õt hozzám?
\par 15 Szûkölködöm-e õrültekben, hogy ide hoztátok ezt, hogy bolondoskodjék elõttem? Ez jõjjön-e be házamba?

\chapter{22}

\par 1 Elméne azért onnan Dávid, és elfutott Adullám barlangjába. És mikor meghallották testvérei és atyjának egész háza népe, oda menének hozzá.
\par 2 És hozzá gyûlének mindazok, a kik nyomorúságban valának, és mindazok, a kiknek hitelezõik voltak, és minden elkeseredett ember, õ pedig vezérük lett azoknak; és mintegy négyszázan valának õ vele.
\par 3 És elméne onnan Dávid Miczpába, Moáb földére, és monda Moáb királyának: Hadd jõjjön ide hozzátok az én atyám és anyám, míg megtudom, hogy mit fog cselekedni velem az Isten.
\par 4 És vivé õket Moáb királya elé, és ott maradának vele mindaddig, míg Dávid a várban volt.
\par 5 Gád próféta pedig monda Dávidnak: Ne maradj a várban, hanem eredj és menj el Júda földére. Elméne azért Dávid, és Héreth erdejébe ment.
\par 6 És meghallotta Saul, hogy elõtûnt Dávid és azok az emberek, a kik vele valának; (Saul pedig Gibeában tartózkodék a hegyen a fa alatt és dárdája a kezében vala és szolgái mindnyájan mellette állának).
\par 7 Monda azért Saul az õ szolgáinak, a kik mellette állottak: Halljátok meg Benjáminnak fiai! Isainak fia adni fog-é néktek mindnyájatoknak szántóföldeket és szõlõhegyeket, és mindnyájatokat ezredesekké és századosokká fog-é tenni,
\par 8 Hogy mindnyájan összeesküdtetek ellenem? És senki sincs, a ki tudósítana engem, hogy fiam szövetséget kötött Isai fiával? És senki sincs közöttetek, a ki szánakoznék felettem, és megmondaná nékem, hogy fiam fellázította szolgámat ellenem, hogy leselkedjék utánam, mint a hogy e mai napon megtetszik?
\par 9 Akkor felele az Edomita Doég, a ki Saul szolgái közt állott: Én láttam, hogy az Isai fia Nóbba ment vala az Akhitób fiához, Akhimélek paphoz.
\par 10 A ki õ érette megkérdé az Urat, és eleséget adott néki, sõt a Filiszteus Góliáth kardját is néki adá.
\par 11 Akkor elkülde a király, hoogy elhívják Akhimélek papot, az Akhitób fiát és atyjának egész házanépét, a papokat, a kik Nóbban valának; és eljövének mindnyájan a királyhoz.
\par 12 És monda Saul: Halld meg most te, Akhitóbnak fia! Õ pedig monda: Ímhol vagyok uram.
\par 13 És monda néki Saul: Miért ütöttetek pártot ellenem, te és Isainak fia, hogy kenyeret és kardot adtál néki, és õ érette megkérdezéd az Istent, hogy fellázadjon ellenem, hogy leselkedjék, mint a hogy most történik?
\par 14 És felele Akhimélek a királynak, és monda: Minden szolgáid között kicsoda hûségesebb Dávidnál, a ki a királynak veje, és a ki akaratod szerint jár, és tisztelt ember a te házadban?
\par 15 Vajjon csak ma kezdém-e az Istent õ érette megkérdezni? Távol legyen tõlem! Ne tulajdonítson olyat a király szolgájának, sem atyám egész házanépének, mert errõl a dologról semmit sem tud a te szolgád, sem kicsinyt, sem nagyot.
\par 16 A király pedig monda: Meg kell halnod Akhimélek, néked és a te atyád egész házanépének!
\par 17 És monda a király a poroszlóknak, a kik mellette állának: Vegyétek körül és öljétek le az Úrnak papjait, mert az õ kezök is Dávid mellett vala, mert tudták, hogy õ menekül, és még sem mondták meg nékem. A király szolgái azonban nem akarták kezeiket felemelni, hogy az Úrnak papjaira rohanjanak.
\par 18 Akkor monda a király Doégnak: Fordulj nékik te, és rohanj a papokra. És ellenük fordula az Edomita Doég, és õ rohana a papokra. És azon a napon nyolczvanöt embert ölt meg, a kik gyolcs efódot viselének.
\par 19 És Nóbot is, a papok városát fegyvernek élével vágatá le, mind a férfit, mind az asszonyt, mind a gyermeket, mind a csecsemõt; az ökröt és szamarat és bárányt, fegyvernek élével.
\par 20 Akhitób fiának, Akhiméleknek egy fia azonban, a kit Abjáthárnak hívtak, elmenekült, és Dávid után futott.
\par 21 És megmondá Abjáthár Dávidnak, hogy megölette Saul az Úrnak papjait.
\par 22 Dávid pedig monda Abjáthárnak: Tudtam én azt már aznap, mert ott volt az Edomita Doég, hogy bizonyosan megmondja Saulnak. Én adtam okot atyád egész házanépének halálára.
\par 23 Maradj nálam, ne félj; mert a ki az én életemet halálra keresi, az keresi a te életedet is, azért te bátorságosan lehetsz mellettem.

\chapter{23}

\par 1 Értesíték pedig Dávidot, mondván: Ímé a Filiszteusok hadakoznak Kehilla ellen, és dúlják a szérûket.
\par 2 Akkor megkérdezé Dávid az Urat, mondván: Elmenjek és leverjem-é ezeket a Filiszteusokat? És monda az Úr Dávidnak: Eredj el, és verd le a Filiszteusokat, és szabadítsd meg Kehillát.
\par 3 A Dávid emberei azonban mondának néki: Ímé, mi itt Júdában is félünk, mennyivel inkább, ha Kehillába megyünk a Filiszteusok táborára.
\par 4 Akkor Dávid ismét megkérdezé az Urat, az Úr pedig válaszola néki, és monda: Kelj fel, és menj el Kehillába, mert én a Filiszteusokat kezedbe adom.
\par 5 Elméne azért Dávid és az õ emberei Kehillába, és harczola a Filiszteusok ellen, és elhajtá marhájokat, és felette igen megveré õket. És megszabadítá Dávid Kehilla lakosait.
\par 6 Lõn pedig, hogy a mikor Abjáthár, az Akhimélek fia Dávidhoz menekült, az efódot is magával vitte.
\par 7 Megmondák akkor Saulnak, hogy Dávid Kehillába ment; és monda Saul: Kezembe adta õt az Isten, mert ott szorult, mivel kulcsos és záros városba méne.
\par 8 És összegyûjté Saul a harczra az egész népet, hogy Kehillába menjen, és körülfogja Dávidot és az õ embereit.
\par 9 Mikor pedig Dávid megtudta, hogy Saul õ ellene gonoszt forral, mondá Abjáthár papnak: Hozd elõ az efódot.
\par 10 És monda Dávid: Uram, Izráel Istene! bizonynyal meghallotta a te szolgád, hogy Saul ide akar jõni Kehillába, hogy elpusztítsa a várost miattam.
\par 11 Vajjon kezébe adnak-é engem Kehilla lakosai? Vajjon lejön-é Saul, a mint hallotta a te szolgád? Óh Uram, Izráel Istene, mondd meg a te szolgádnak! És monda az Úr: lejön.
\par 12 És monda Dávid: Vajjon Saul kezébe adnak-é Kehilla lakosai engem és az én embereimet? És monda az Úr: Kezébe adnak.
\par 13 Felkele azért Dávid és az õ emberei, mintegy hatszázan, és kimenének Kehillából, és ide s tova járnak vala, a hol csak járhatának. Midõn pedig Saulnak megmondák, hogy elmenekült Dávid Kehillából, felhagyott az elmenetellel.
\par 14 És Dávid a pusztában tartózkodék az erõs helyeken, és a Zif pusztájában levõ hegységen marada. És Saul mindennap keresé, de az Isten nem adá õt kezébe.
\par 15 Mikor pedig Dávid látta, hogy Saul kiment, hogy élete ellen törjön; és mikor Dávid a Zif pusztájában, az erdõben vala:
\par 16 Felkele Jonathán, a Saul fia, és elment Dávidhoz az erdõbe, és megerõsíté az õ kezét az Istenben.
\par 17 És monda néki: Ne félj, mert Saulnak, az én atyámnak keze nem fog utólérni téged, és te király leszesz Izráel felett, és én második leszek te utánad, és Saul is, az én atyám, tudja, hogy így lesz.
\par 18 És szövetséget kötének ketten az Úr elõtt. És Dávid az erdõben marada, Jonathán pedig haza ment.
\par 19 És felmenének a Zifeusok Saulhoz Gibeába, mondván: Avagy nem nálunk lappang-é Dávid az erõs helyeken az erdõben, Hakila halmán, mely a sivatagtól jobbkézre van?!
\par 20 Most azért minthogy lelkednek fõkivánsága az, hogy lejõjj, óh király, jõjj le; és a mi gondunk lesz, hogy a királynak kezébe adjuk õt.
\par 21 És monda Saul: Legyetek megáldva az Úrtól, hogy szánakoztok rajtam!
\par 22 Menjetek azért el, és vigyázzatok ezután is, hogy megtudjátok és meglássátok az õ tartózkodási helyét, és hogy ki látta õt ott, mert azt mondották nékem, hogy igen ravasz õ.
\par 23 Annakokáért nézzetek meg és tudjatok meg minden búvóhelyet, a hol õ lappang és minden bizonynyal térjetek vissza hozzám, hogy elmenjek veletek; és ha az országban van, kikutatom õt Júdának minden ezrei között.
\par 24 Azok pedig felkelének, és elmenének Zifbe Saul elõtt. Dávid pedig és az õ emberei Máon pusztájában valának a mezõségen, a mely a sivatagtól jobbkézre van.
\par 25 És elméne Saul az õ embereivel együtt, hogy megkeresse õt. Dávidnak azonban megizenték, és õ leszállott a kõszikláról, és Máon pusztájában tartózkodék. Mikor pedig meghallotta Saul, üldözé Dávidot Máon pusztájában.
\par 26 És Saul a hegynek egyik oldalán méne, Dávid és az õ emberei pedig a hegynek másik oldalán. És épen, mikor Dávid nagyon sietett, hogy elmenekülhessen Saul elõl, és Saul és az õ emberei már körül is kerítették Dávidot, és az õ embereit, hogy megfogják,
\par 27 Akkor érkezék egy követ Saulhoz, mondván: Siess és jõjj! mert a Filiszteusok betörtek az országba.
\par 28 Akkor megtére Saul Dávid üldözésébõl, és a Filiszteusok ellen ment. Azért hívják azt a hegyet a menekülés kõsziklájának.

\chapter{24}

\par 1 És Dávid elméne onnan, és Engedi erõsségei közt tartózkodék.
\par 2 Lõn pedig, hogy a mikor visszatért Saul a Filiszteusok üldözésébõl, megizenték néki, mondván: Ímé Dávid az Engedi pusztájában van.
\par 3 Maga mellé võn azért Saul az egész Izráel közül háromezer válogatott embert, és elment, hogy megkeresse Dávidot és az õ embereit a vadkecskék kõszikláin.
\par 4 És eljutott a juhaklokhoz, a melyek az útfélen vannak, hol egy barlang volt; és beméne Saul, hogy ott szükségét végezze; Dávid pedig és az õ emberei a barlang rejtekeiben valának.
\par 5 Akkor mondák Dávidnak az õ emberei: Ímé ez az a nap, a melyrõl azt mondotta az Úr néked: Ímé kezedbe adom ellenségedet, hogy úgy cselekedjél vele, a mint néked tetszik. Felkele azért Dávid, és elmetszé orozva Saul ruhájának szárnyát.
\par 6 Lõn pedig ezután, hogy megesett a Dávid szíve rajta, hogy elmetszé Saul ruhájának szárnyát;
\par 7 És monda az õ embereinek: Oltalmazzon meg engem az Úr attól, hogy ily dolgot cselekedjem az én urammal, az Úrnak felkentjével, hogy kezemet felemeljem ellene, mert az Úrnak felkentje õ.
\par 8 És megfeddé Dávid kemény szókkal embereit, és nem engedte meg nékik, hogy Saul ellen támadjanak. Mikor pedig Saul felkelt a barlangból, és elment az úton:
\par 9 Dávid is felkelt ezután, és kiment a barlangból, és Saul után kiálta, mondván: Uram király! Mikor pedig Saul hátratekinte, Dávid arczczal a földre hajolt, és tisztességet tõn néki.
\par 10 És monda Dávid Saulnak: Miért hallgatsz az olyan ember szavaira, a ki azt mondja: Ímé Dávid romlásodra tör?!
\par 11 Ímé a mai napon látták a te szemeid, hogy az Úr téged a kezembe adott ma a barlangban, és azt mondották, hogy öljelek meg téged, de én kedvezék néked, és azt mondám: Nem emelem fel kezemet az én uram ellen, mert az Úrnak felkentje õ.
\par 12 Azért atyám! nézd, ugyan nézd felsõ ruhádnak szárnyát kezemben, mert mikor levágtam felsõ ruhádnak szárnyát, nem öltelek meg téged! Azért tudd meg és lássad, hogy nincsen az én kezemben hamisság és semmi gonoszság és nem vétkeztem ellened, de te mégis életem után leselkedel, hogy elveszessed azt.
\par 13 Az Úr tegyen ítéletet közöttem és közötted, és álljon bosszút az Úr érettem rajtad, de az én kezem nem lesz ellened.
\par 14 A mint a régi példabeszéd mondja: A gonoszoktól származik a gonoszság; de az én kezem nem lesz ellened.
\par 15 Ki ellen jött ki Izráelnek királya? Kit kergetsz? Egy holt ebet, vagy egy bolhát?
\par 16 Legyen azért az Úr ítélõbiró, és tegyen ítéletet közöttem és közötted, és lássa meg; õ forgassa az én ügyemet, és szabadítson meg engem kezedbõl.
\par 17 És lõn, hogy a mikor elmondotta Dávid e szókat Saul elõtt, monda Saul: A te szavad-é ez, fiam, Dávid? És felkiálta Saul, és síra.
\par 18 És monda Dávidnak: Te igazságosabb vagy én nálamnál, mert te jót cselekedtél velem, én pedig rosszal fizettem néked.
\par 19 És te megmondottad nékem a mai napon, minémû jót cselekedtél velem, hogy az Úr kezedbe adott engem, és te még sem öltél meg engem.
\par 20 Mert ha valaki megtalálja ellenségét, elbocsátja-é õt békében az úton? Annakokáért fizessen az Úr néked jóval azért, a mit velem ma cselekedtél.
\par 21 Most pedig, mivel tudom, hogy te király leszesz, és Izráelnek királysága a te kezedben állandó lesz:
\par 22 Esküdjél meg nékem most az Úrra, hogy én utánam nem fogod kiirtani maradékomat és nevemet nem fogod kitörölni atyám házából!
\par 23 És Dávid megesküvék Saulnak. És Saul elméne haza, Dávid pedig és az õ emberei felmenének az õ erõsségökbe.

\chapter{25}

\par 1 Meghala pedig Sámuel, és egybegyûle az egész Izráel, és siratták õt, és eltemették az õ házában Rámában. Dávid pedig felkelt és elment Párán pusztájába.
\par 2 És volt egy ember Máonban, a kinek jószága Kármelben vala, és ez igen tehetõs ember volt: háromezer juha és ezer kecskéje volt néki. És Kármelben épen juhait nyírta.
\par 3 (Azt az embert pedig Nábálnak, és feleségét Abigailnak hívták, a ki igen eszes és szép termetû asszony volt; a férfi azonban durva és rossz erkölcsû vala, a Káleb nemzetségébõl való volt.)
\par 4 És meghallotta Dávid a pusztában, hogy Nábál a juhait nyírja.
\par 5 Elkülde azért Dávid tíz ifjút, és monda Dávid az ifjaknak: Menjetek fel Kármelbe, és mikor Nábálhoz érkeztek, köszöntsétek õt nevemben békességesen.
\par 6 És így szóljatok: Légy békességgel az életben, legyen békességben a te házadnépe és legyen békességben mindened, a mid van!
\par 7 Most hallottam, hogy juhaidat nyiratod. A te pásztoraid pedig velünk valának, nem bántottuk õket, és semmijök sem hibázott az alatt az egész idõ alatt, míg Kármelben valának.
\par 8 Kérdezd meg szolgáidat, õk meg fogják mondani néked. Legyenek azért ez ifjak kedvesek elõtted, mert alkalmas idõben jöttünk. Adj kérlek abból, a mi kezed közt van, szolgáidnak, és a te fiadnak, Dávidnak.
\par 9 Elmenének azért a Dávid szolgái, és szólának Nábálnak mind e beszédek szerint a Dávid nevében, és várakozának.
\par 10 Nábál pedig felele a Dávid szolgáinak, és monda: Kicsoda Dávid és kicsoda Isainak fia? Mai napság sok olyan szolga van, a kik elszöknek uraiktól.
\par 11 Vegyem azért kenyeremet és vizemet és az én levágott marhámat, a melyet nyíróimnak levágattam, hogy olyan embereknek adjam, a kikrõl azt sem tudom, hova valók?
\par 12 Akkor megfordulának a Dávid szolgái az õ útjokra, és visszatérének; és mikor megérkezének, értesítették õt minden e beszédek felõl.
\par 13 És monda Dávid az õ embereinek: Kösse fel mindenki kardját! És felköté mindenki a kardját, Dávid is felköté az õ kardját; és felment Dávid után mintegy négyszáz ember; kétszáz pedig ott maradt a podgyásznál.
\par 14 Abigailt pedig, a Nábál feleségét értesíté a szolgák közül egy ifjú, mondván: Ímé Dávid követeket küldött a pusztából, hogy köszöntsék a mi urunkat, de õ elûzé õket.
\par 15 Azok az emberek pedig igen jók voltak mi hozzánk; és nem volt bántódásunk, és semmink nem hibázott az alatt az egész idõ alatt, míg velök jártunk, mikor a mezõn voltunk.
\par 16 Olyanok voltak reánk nézve, mint a kõfal, mind éjjel, mind nappal, az alatt az egész idõ alatt, míg velök valánk, mikor a juhokat õriztük.
\par 17 Most azért értsd meg és lássad, hogy mit kelljen cselekedned, mert jelen van a veszedelem a mi urunk és az õ egész háza ellen, õ pedig oly kegyetlen ember, hogy senki sem szólhat néki.
\par 18 Akkor Abigail sietve võn kétszáz kenyeret, két tömlõ bort, öt juhot elkészítve, öt mérték pergelt búzát, száz kötés aszúszõlõt és kétszáz kötés száraz fügét, és a szamarakra rakta.
\par 19 És monda az õ szolgáinak: Menjetek el elõttem, ímé én utánatok megyek; de férjének, Nábálnak nem mondá meg.
\par 20 És történt, hogy a mint a szamáron megy vala, és leereszkedék a hegynek egyik mellékösvényén: ímé Dávid és az õ emberei lejövének eleibe, és õ összetalálkozék velök.
\par 21 Dávid pedig azt mondotta volt: Bizony hiába õriztem ennek mindenét, a mije van a pusztában, hogy semmi híjja nem lett mindannak, a mi az övé, mert õ a jó helyett roszszal fizet nékem.
\par 22 Úgy cselekedjék az Isten Dávid ellenségeivel most és ezután is, hogy reggelig meg nem hagyok mindabból, a mi az övé, még egy ebet sem.
\par 23 Mikor pedig meglátta Abigail Dávidot, sietve leszállott a szamárról, és arczczal leborula Dávid elõtt, és meghajtá magát a földig.
\par 24 És az õ lábaihoz borula, és monda: Óh uram! én magam vagyok a bûnös, mindazáltal hadd beszéljen a te szolgálóleányod te elõtted, és hallgasd meg szolgálóleányodnak szavait.
\par 25 Kérlek, ne törõdjék az én uram Nábállal, ezzel a kegyetlen emberrel, mert a milyen a neve, olyan õ maga is; bolond az õ neve és bolondság van benne. Én azonban, a te szolgálóleányod, nem láttam az én uramnak szolgáit, a kiket elküldöttél volt.
\par 26 Most pedig, óh uram! él az Úr és él a te lelked, hogy az Úr akadályozott meg téged, hogy gyilkosságba ne essél, és ne saját kezeddel szerezz magadnak elégtételt. Most azért olyanok legyenek ellenségeid, mint Nábál, és valakik az én uramnak megrontására törekesznek.
\par 27 És most ezt az ajándékot, a melyet a te szolgálóleányod hozott az én uramnak, adják a vitézeknek, a kik az én uram körül forgolódnak.
\par 28 Bocsásd meg azért a te szolgálóleányodnak vétkét; mert az én uramnak bizonyára maradandó házat épít az Úr, mert az Úrnak harczait harczolja az én uram, és gonoszság nem találtatik te benned a te életedben.
\par 29 És ha valaki feltámadna ellened, hogy téged üldözzön és életed ellen törjön: az én uramnak lelke az élõknek csomójába leend bekötve az Úrnál a te Istenednél; ellenségeidnek lelkét pedig a parittyának öblébõl fogja elhajítani.
\par 30 És mikor az Úr megadja a jót az én uramnak mind a szerint, a mint megmondotta felõled, és téged fejedelmül rendel Izráel fölé:
\par 31 Akkor, óh uram, nem leend ez néked bántásodra és szívednek fájdalmára, hogy ok nélkül vért ontottál, és hogy az én uram saját maga szerzett magának elégtételt. Mikor azért jót tesz az Úr az én urammal: emlékezzél meg szolgálóleányodról.
\par 32 És monda Dávid Abigailnak: Áldott legyen az Úr, Izráelnek Istene, a ki téged ma elõmbe küldött!
\par 33 És áldott legyen a te tanácsod, és áldott légy te magad is, hogy a mai napon megakadályoztál engem, hogy gyilkosságba ne essem, és ne saját kezemmel szerezzek magamnak elégtételt!
\par 34 Bizonyára él az Úr, az Izráelnek Istene, a ki megakadályozott engem, hogy veled gonoszul ne cselekdjem, mert ha te nem siettél és nem jöttél volna elõmbe, úgy Nábálnak nem maradt volna meg reggelre csak egyetlen ebe sem.
\par 35 És átvevé Dávid az õ kezébõl, a mit hozott néki, és monda néki: Eredj el békességben a te házadhoz; lásd, hallgattam szavadra, és megbecsültem személyedet.
\par 36 Mikor pedig Abigail Nábálhoz visszaérkezék, ímé olyan lakoma volt az õ házában, mint a király lakomája, és Nábál szíve vigadozék azon, mert igen megrészegedett; azért õ semmit sem mondott meg néki, sem kicsinyt, sem nagyot egészen reggelig.
\par 37 Reggel pedig, mikor Nábál kijózanodék, megmondá néki felesége ezeket a dolgokat; és elhala az õ szíve õ benne, és olyanná lõn, mint a kõ.
\par 38 És mintegy tíz nap mulva megveré az Úr Nábált, és meghala.
\par 39 És mikor Dávid meghallotta, hogy Nábál meghalt, monda: Áldott legyen az Úr, ki bosszút állott Nábálon az én gyaláztatásomért, és szolgáját visszatartotta a gonosztól, a Nábál gonoszságát pedig visszafordítá az Úr az õ fejére! És elkülde Dávid, és izent Abigailnak, hogy elvenné õt feleségéül.
\par 40 Elmenének azért Dávid szolgái Abigailhoz Kármelbe, és beszélének vele, mondván: Dávid küldött minket hozzád, hogy téged elvegyen feleségéül.
\par 41 Õ pedig felálla, és meghajtotta magát arczczal a földre, és monda: Ímé a te szolgálóleányod szolgáló lesz, hogy mossa az én uram szolgáinak lábait.
\par 42 És Abigail sietve felkele, és felült a szamárra és az õ öt szolgálóleánya, a kik körülötte valának, és elment Dávid követei után, és az õ felesége lõn.
\par 43 Ahinoát is elvevé Dávid Jezréelbõl, és mind a kettõ felesége lõn néki.
\par 44 Saul pedig az õ leányát, Mikált, a Dávid feleségét Páltinak, a Láis fiának adá, a ki Gallimból  való volt.

\chapter{26}

\par 1 És menének a Zifeusok Saulhoz Gibeába, mondván: Nemde nem a Hakila halmán lappang-é Dávid, a puszta átellenében?
\par 2 Felkele azért Saul, és lement Zif pusztájába, és vele volt Izráel közül háromezer válogatott ember, hogy megkeresse Dávidot Zif pusztájában.
\par 3 És tábort jára Saul a Hakila halmán, mely a puszta átellenében van, az úton; Dávid pedig a pusztában tartózkodék. És mikor észrevette, hogy Saul utána ment a pusztába:
\par 4 Kémeket küldött ki Dávid, és megtudta biztosan, hogy Saul eljött.
\par 5 És felkele Dávid, és elment arra a helyre, a hol Saul táborozott, és megnézte Dávid azt a helyet, a hol feküvék Saul és Abner, a Nér fia, seregének fõvezére. Saul pedig a kerített táborban feküvék, és a nép körülötte táborozott.
\par 6 Akkor szóla Dávid, és monda a Hitteus nemzetségébõl való Akhiméleknek és Abisainak, a Seruja fiának, a ki Joábnak testvére vala, mondván: Ki jön le velem Saulhoz a táborba? És mondá Abisai: Lemegyek én veled.
\par 7 És elméne éjjel Dávid, és Abisai a nép közé. És ímé, Saul lefeküvén, alszik vala a kerített táborban, és dárdája a földbe volt szúrva fejénél; Abner pedig és a nép körülötte feküvének.
\par 8 Akkor monda Abisai Dávidnak: Kezedbe adta a mai napon Isten a te ellenségedet; most azért, hadd szegezzem a földhöz õt a dárdával egy ütéssel, és másodszor nem ütöm át.
\par 9 Dávid azonban monda Abisainak: Ne veszesd el õt! Mert vajjon ki emelhetné fel kezét az Úrnak felkentje ellen büntetlenül?!
\par 10 És monda Dávid: Él az Úr, hogy az Úr megveri õt! Vagy eljön az õ napja és meghal, vagy pedig harczba megy és ott vész el!
\par 11 Távoztassa el azért az Úr tõlem azt, hogy kezemet az Úrnak felkentje ellen emeljem, hanem csak vedd a dárdát, mely feje mellett van, és a vizes korsót, és menjünk el.
\par 12 Akkor elvevé Dávid a dárdát és a vizes korsót Saul feje mellõl, és elmenének. És senki sem volt, a ki látta volna, sem a ki észrevette volna, sem a ki felserkent volna, hanem mindnyájan aluvának, mert az Úr mély álmot bocsátott reájok.
\par 13 És mikor Dávid általment a túlsó oldalra, megállott a hegy tetején messzire, úgy, hogy nagy távolság volt közöttük.
\par 14 És kiálta Dávid a népnek és Abnernek, a Nér fiának, mondván: Nem felelsz-é Abner? És felele Abner, és monda: Kicsoda vagy te, hogy így kiáltasz a királynak?
\par 15 Dávid pedig monda Abnernek: Avagy nem férfi vagy-é te, és kicsoda olyan, mint te Izráelben? És miért nem vigyáztál a te uradra, a királyra? Mert a nép közül oda ment egy ember, hogy elveszesse a te uradat, a királyt.
\par 16 Nem jó dolog ez, a mit cselekedtél! Él az Úr, hogy halál fiai vagytok ti, a miért nem vigyáztatok a ti uratokra, az Úrnak felkentjére! Most azért nézd meg, hol van a király dárdája és a vizes korsó, a mely fejénél volt?
\par 17 És megismeré Saul a Dávid hangját, és monda: A te hangod-é ez fiam, Dávid? Dávid pedig monda: Az én hangom, uram király!
\par 18 És monda: Miért üldözi szolgáját az én uram? Ugyan mit cselekedtem, és micsoda gonoszság van én bennem?
\par 19 Most azért hallgassa meg az én uram, a király, az õ szolgájának szavát! Ha az Úr ingerelt fel téged ellenem: vajha jóillatú volna elõtte az áldozat; ha pedig emberek: átkozottak legyenek az Úr elõtt, mert kiûznek most engemet, hogy ne részesülhessek az Úrnak örökségében, azt mondván: Eredj, szolgálj idegen isteneknek.
\par 20 Azért ne omoljék az én vérem a földre távol az Úr színétõl; mert Izráelnek királya kijött, hogy egy bolhát keressen, úgy, mint egy fogoly madarat üldöznek a hegyeken.
\par 21 Saul pedig monda: Vétkeztem! térj vissza fiam, Dávid, mert többé nem cselekszem veled gonoszul, mivel az én életem kedves volt elõtted a mai napon. Ímé, esztelenül cselekedtem, és igen nagyot vétettem.
\par 22 És felele Dávid, és monda: Ímhol a király dárdája, jõjjön ide a szolgák közül egy, és vigye el azt.
\par 23 Az Úr pedig fizessen meg mindenkinek az õ igazsága és hûsége szerint, mert az Úr kezembe adott téged ma, de én nem akartam felemelni kezemet az Úrnak felkentje ellen.
\par 24 És a mennyire drága volt a mai napon a te lelked én elõttem, legyen annyira drága az én lelkem az Úr elõtt, és szabadítson meg engem minden nyomorúságból!
\par 25 Akkor monda Saul Dávidnak: Áldott légy te, fiam Dávid, hatalmasan is fogsz cselekedni, és gyõzni is fogsz! És elment Dávid a maga útjára, Saul pedig visszatért az õ helyére.

\chapter{27}

\par 1 És monda Dávid magában: Egy napon mégis el kell pusztulnom a Saul keze miatt, nincs jobb reám nézve, mintha gyorsan elmenekülök a Filiszteusok tartományába, így Saul felhagy azzal, hogy engem tovább is üldözzön Izráel egész területén, és így megszabadulok az õ kezébõl.
\par 2 Felkelvén azért Dávid, elméne õ és az a hatszáz ember, a kik vele valának, Ákhishoz, a Máok fiához, Gáth királyához.
\par 3 És Dávid Ákhisnál tartózkodék Gáthban, õ és az õ emberei, mindegyik a maga házanépével együtt; Dávid és az õ két felesége, a Jezréelbõl való Ahinoám és a Kármelbõl való Abigail, a Nábál felesége.
\par 4 Mikor pedig Saulnak megmondották, hogy Dávid Gáthba menekült, nem üldözé tovább õt.
\par 5 És monda Dávid Ákhisnak: Ha kedvet találtam elõtted, adj helyet nékem valamelyik vidéki városban, hogy ott lakjam: miért laknék a te szolgád veled a királyi városban?
\par 6 És néki adá Ákhis azon a napon Siklágot; lõn azért Siklág a Júda királyaié mind e mai napig.
\par 7 És lõn ama napoknak száma, míg Dávid a Filiszteusok földén lakozék, egy esztendõ és négy hónap.
\par 8 És felméne Dávid embereivel együtt és megtámadták a Gessureusokat, a Girzeusokat és az Amálekitákat: mert ezek voltak annak a földnek lakosai eleitõl fogva, a melyen Súrba mégy egészen Égyiptom földéig.
\par 9 És mikor Dávid megverte az országot, sem férfit, sem asszonyt nem hagyott életben, és elvitt juhot, ökröt, szamarakat, tevéket és ruhákat, és úgy tért vissza és ment Ákhishoz.
\par 10 Mikor pedig Ákhis azt kérdé: Hova törtetek be most? Dávid ekként felele: Júda déli részére és Jerákhméelnek déli részére és Kéneusnak déli részére.
\par 11 Dávid azonban sem férfit, sem asszonyt nem hagyott életben, hogy Gáthba vigye, mondván: Valamikép ellenünk ne nyilatkozzanak és azt mondják: Így cselekedett Dávid. Ez volt az õ szokása amaz egész idõ alatt, míg a Filiszteusok földjén tartózkodék.
\par 12 És Ákhis bízott Dávidban, mondván: Bizonyosan gyûlöletessé tette magát az õ népe, Izráel elõtt, azért örökké az én szolgám leend.

\chapter{28}

\par 1 És történt abban az idõben, hogy a Filiszteusok összegyûjtötték seregeiket a hadra, hogy harczoljanak Izráel ellen. És monda Ákhis Dávidnak: Tudd meg, hogy velem kell jõnöd a táborba, mind néked, mind embereidnek.
\par 2 Dávid pedig felele Ákhisnak: Meglátod bizonynyal, hogy mit fog cselekedni a te szolgád. És monda Ákhis Dávidnak: Ennélfogva fejem oltalmazójává teszlek mindenkorra.
\par 3 Sámuel pedig meghalt vala, és siratá õt az egész Izráel, és eltemeték õt saját városában, Rámában; Saul pedig a varázslókat és jövendõmondókat kiirtá a földrõl.
\par 4 És mikor a Filiszteusok egybegyûlvén, eljövének és tábort járának Sunemnél: egybegyûjté Saul is az egész Izráelt, és tábort járának Gilboánál.
\par 5 A mint azonban Saul meglátta a Filiszteusok táborát, megfélemlék és az õ szíve nagyon megrémüle.
\par 6 És megkérdezé Saul az Urat, de az Úr nem felelt néki sem álomlátás, sem az Urim,  sem a próféták által.
\par 7 Akkor monda Saul az õ szolgáinak: Keressetek nékem egy halottidézõ asszonyt, hogy elmenjek hozzá, és megkérdezzem õt. Szolgái pedig mondának néki: Ímé, Endorban van egy halottidézõ asszony.
\par 8 Másnak tetteté azért Saul magát, és más ruhákat vevén magára, elméne õ és vele két férfi; és elmenének éjjel az asszonyhoz, és monda: Mondj jövendõt nékem halottidézés által, és idézd fel nékem azt, akit mondok néked.
\par 9 És monda az asszony néki: Ímé te jól tudod, hogy mit cselekedett Saul, hogy kiirtá a földrõl a varázslókat és jövendõmondókat; miért akarod azért tõrbe ejteni az én lelkemet, hogy megöless engem?!
\par 10 És megesküvék néki Saul az Úrra, mondván: Él az Úr, hogy e dolog miatt büntetésed nem lészen.
\par 11 Monda azért az asszony: Kit idézzek fel néked? És õ monda: Sámuelt idézd fel nékem.
\par 12 Mikor pedig az asszony Sámuelt meglátta, hangosan felkiáltott. És szóla az asszony Saulnak, mondván: Miért csaltál meg engem? hiszen te vagy Saul!
\par 13 És monda néki a király: Ne félj! Ugyan mit láttál? Az asszony pedig monda Saulnak: istenfélét látok feljõni a földbõl.
\par 14 És õ monda néki: Milyen ábrázata van? Õ pedig monda: Egy vén ember jõ fel, és palást van rajta. És megismeré Saul, hogy az Sámuel, és meghajtá magát arczczal a föld felé, és tisztességet tõn néki.
\par 15 Sámuel pedig monda Saulnak: Miért háborgattál, hogy felidéztettél engemet? És felele Saul: Igen nagy szorultságban vagyok; a Filiszteusok hadakoznak ellenem, az Isten pedig eltávozék tõlem, és nem felel már nékem sem próféták által, sem álomlátás által; azért hívtalak téged, hogy megmondjad nékem, mit kelljen cselekednem?
\par 16 És monda Sámuel: Ugyan miért kérdezel engemet, ha az Úr eltávozott tõled és ellenségeddé lõn?!
\par 17 És a szerint cselekedett az Úr, a mint általam megmondotta vala: elvette az Úr a királyságot a te kezedbõl, és adta azt a te társadnak, Dávidnak.
\par 18 Mivel nem hallgattál az Úrnak szavára, és nem hajtottad végre az õ felgerjedt haragját az Amálekitákon: azért cselekszik most így veled az Úr.
\par 19 És az Úr Izráelt is veled együtt a Filiszteusok kezébe adja, te pedig holnap fiaiddal együtt velem leszesz. Izráelnek táborát is a Filiszteusok kezébe adja az Úr.
\par 20 Akkor Saul a maga egész nagyságában hirtelen a földre esék, mert nagyon megrémüle Sámuel szavaitól; és semmi erõ nem vala benne, mert egész nap és egész éjjel semmit sem evék.
\par 21 Akkor az asszony Saulhoz ment, és mikor látta, hogy annyira megrémült, monda néki: Ímé a te szolgálóleányod hallgatott szavadra, és koczkára tettem életemet, és megfogadtam szavaidat, a melyeket mondottál nékem:
\par 22 Most azért hallgass te is szolgálóleányod szavára, hadd tegyek egy falat kenyeret elõdbe, és egyél, hogy erõd legyen, mikor útra kelsz.
\par 23 Õ azonban vonakodék, és mondá: Nem eszem; de szolgái és az asszony is kényszeríték õt, és õ engedett szavoknak, felkelt a földrõl, és felüle az ágyra.
\par 24 Vala pedig az asszony házánál egy hízott borjú és sietve levágta azt; azután lisztet vett, és meggyúrta, és sütött kovásztalan pogácsát.
\par 25 És vivé Saul elé és az õ szolgái elé, és evének; azután felkeltek és elmenének azon az éjszakán.

\chapter{29}

\par 1 Akkor a Filiszteusok összegyûjték minden seregeiket Afeknél; Izráel pedig tábort jár vala a forrásnál, mely Jezréel mellett van.
\par 2 És a Filiszteusok vezérei kivonulának, ki százzal, ki ezerrel, Dávid pedig és az õ emberei hátul menének Ákhissal.
\par 3 És mondának a Filiszteusok vezérei: Mit akarnak ezek a zsidók? És monda Ákhis a Filiszteusok vezéreinek: Avagy nem ez-é Dávid, Saulnak, az Izráel királyának szolgája, a ki már napok óta, sõt évek óta nálam van, és nem találtam benne semmi rosszat attól a naptól fogva, hogy átjött, a mai napig.
\par 4 De megharaguvának õ reá a Filiszteusok vezérei, és mondának néki a Filiszteusok vezérei: Küldd vissza a helyére oda, a melyet rendeltél néki, és ne jõjjön el mi velünk a harczba, hogy ellenünk ne forduljon a harczban; mert ugyan mivel tehetné magát kedvesebbé ura elõtt, hacsak nem ezeknek a vitézeknek fejeivel?
\par 5 Avagy nem ez-é Dávid, a kirõl így énekelnek a körtánczban: Megverte Saul az õ ezerét, és Dávid is az õ tízezerét?
\par 6 Szólítá azért Ákhis Dávidot, és monda néki: Él az Úr, hogy te becsületes vagy, és kedves elõttem mind kimenésed, mind bejövésed velem a táborba, mert semmi rosszat nem találtam benned attól a naptól fogva, hogy hozzám jöttél, e mai napig; de a vezérek elõtt nem vagy kedves.
\par 7 Most azért térj vissza, és menj el békességben, és semmit se cselekedjél, a mi a Filiszteusok vezérei elõtt helytelen.
\par 8 És monda Dávid Ákhisnak: Vajjon mit cselekedtem, és mit találtál a te szolgádban attól a naptól fogva, hogy nálad voltam, a mi napig, hogy ne menjek el, és ne harczoljak a királynak, az én uramnak ellenségei ellen?
\par 9 Ákhis pedig felele, és monda Dávidnak: Tudom; bizonyára kedves vagy elõttem, mint az Istennek angyala; de a Filiszteusok vezérei mondák: El ne jõjjön velünk a harczba.
\par 10 Azért kelj fel korán reggel uradnak szolgáival együtt, a kik veled eljövének; keljetek fel korán reggel, mihelyt megvirrad, és menjetek el.
\par 11 Felkele azért Dávid embereivel együtt, hogy korán reggel elmenjen és visszatérjen a Filiszteusok földére. A Filiszteusok pedig felmenének Jezréelbe.

\chapter{30}

\par 1 És történt, hogy a mikor Dávid harmadnapon embereivel Siklágba megérkezék, ímé az Amálekiták betörének a déli vidékre és Siklágba, és leverték Siklágot, és felégették azt tûzzel.
\par 2 És fogságba hurczolák az asszonyokat, a kik benne valának, kicsinytõl fogva nagyig; senkit sem öltek meg, hanem elhurczolták, és elmentek az õ útjokra.
\par 3 Mikor azért Dávid embereivel együtt a városba érkezék; ímé az tûzzel felégettetett vala, feleségeik, fiaik és leányaik pedig fogságba hurczoltattak.
\par 4 Akkor Dávid és a nép, a mely õ vele volt, felkiáltának és annyira sírának, hogy végre erejök sem volt a sírásra.
\par 5 Dávidnak két felesége is fogságba hurczoltatott, a Jezréelbõl való Ahinoám és a Kármelbõl való Abigail, a Nábál felesége.
\par 6 És Dávid igen nagy szorultságba juta, mert a nép arról beszélt, hogy megkövezi õt, mivel az egész nép lelke elkeseredett fiaik és leányaik miatt. Dávid azonban megerõsíté magát az Úrban, az õ Istenében.
\par 7 És monda Dávid Abjáthár papnak, az Akhimélek fiának: Hozd ide nékem az efódot. És Abjáthár oda vivé az efódot Dávidhoz.
\par 8 És megkérdezé Dávid az Urat, mondván: Üldözzem-é ezt a sereget? Utólérem-é õket? És monda néki: Üldözzed, mert bizonyosan utóléred, és szabadulást szerzesz.
\par 9 Elméne azért Dávid, õ és az a hatszáz ember, a kik vele valának. És mikor a Bésor patakához jutának, ott a népnek egy része megállott.
\par 10 Dávid azonban négyszáz emberrel tovább üldözé az ellenséget; kétszáz ember pedig ott megállott, mivel fáradtabbak valának, mintsem hogy a Bésor patakán átkelhettek volna.
\par 11 És találának a mezõn egy égyiptomi embert, a kit Dávidhoz vivének, és adának néki kenyeret, hogy egyék, és megitaták õt vízzel.
\par 12 És adának néki egy csomó száraz fügét és két kötés aszuszõlõt; és miután evett, magához tért, mert három nap és három éjjel sem kenyeret nem evett, sem vizet nem ivott.
\par 13 És monda néki Dávid: Ki embere vagy te, és honnan való vagy? Õ pedig monda: Égyiptomi ifjú vagyok, egy Amálekita embernek szolgája, és elhagyott engem az én uram, mivel megbetegedtem immár ma harmadnapja.
\par 14 Mi törtünk be a Kréteusok déli vidékére, és oda, a mely Júdáé, és Kálebnek  déli vidékére; és Siklágot felégettük tûzzel.
\par 15 És Dávid monda néki: Elvezetsz-é minket ahhoz a serghez? És õ monda: Esküdjél meg nékem az Istenre, hogy nem ölsz meg engem, sem az én uramnak kezébe át nem adsz, és én elvezetlek téged ahhoz a sereghez.
\par 16 Elvezeté azért õket; és ímé, azok elszéledének az egész vidéken, és evének, ivának és tánczolának az igen nagy zsákmány felett, a melyet a Filiszteusok tartományából és Júda tartományából hozának.
\par 17 És vágta õket Dávid alkonyattól fogva másnap estvéig, és senki sem menekült meg közülök, hanem csak négyszáz ifjú ember, a kik tevékre ülének és elfutának.
\par 18 És mindent megszabadított Dávid, valamit elvittek az Amálekiták; az õ két feleségét is megszabadítá Dávid.
\par 19 És semmijök sem hiányzott, sem kicsiny, sem nagy, sem fiaik, sem leányaik, a zsákmányból sem és mindabból, a mit elvittek tõlük; Dávid mindent visszahozott.
\par 20 És elvivé Dávid mind a juhokat és barmokat, melyeket tulajdon barmuk elõtt hajtának, és azt mondják vala: Ez a Dávid zsákmánya!
\par 21 Midõn pedig Dávid ahhoz a kétszáz emberhez érkezék, a kik fáradtabbak valának, minthogy Dávidot követhették volna, és ott hagyta õket a Bésor patakjánál: kimenének Dávid elé és a nép elé, a mely vele volt. Dávid pedig a néphez közeledék, és köszönté õket békességgel.
\par 22 Akkor mondának mind azok közül, a kik Dáviddal elmenének, így szólván: Minthogy mind a gonosz emberek és Béliál emberei nem jöttek el velünk, semmit se adjunk nékik a zsákmányból, a melyet visszaszereztünk, hanem csak kinek-kinek a maga feleségét és gyermekeit, azokat vigyék el, és menjenek el.
\par 23 Dávid azonban így szóla: Ne cselekedjetek így, atyámfiai, azzal, a mit az Úr adott nékünk; õ oltalmazott minket és adá kezünkbe a sereget, a mely ellenünk jött vala.
\par 24 Vajjon kicsoda engedhetne néktek ebben a dologban? Sõt inkább, a mekkora annak a része, a ki elment a harczba, akkora legyen annak is a része, a ki a holminál marada; egyenlõképen osztozzanak.
\par 25 És úgy történt ez attól a naptól fogva azután is; törvénynyé és szokássá tette ezt Izráelben mind e mai napig.
\par 26 Mikor pedig Dávid Siklágba érkezék, külde a zsákmányból Júda véneinek, az õ barátainak, mondván: Ímé az Úr ellenségeinek zsákmányából való ajándék számotokra.
\par 27 Azoknak tudniillik, a kik Béthelben, a kik dél felé Rámóthban, és a kik Jathirban laknak;
\par 28 A kik Aroerben, a kik Sifmótban és a kik Estemoában laknak;
\par 29 A kik Rákálban, a kik a Jerakhméeliták városaiban és a kik a Kéneusok városaiban laknak;
\par 30 A kik Hormában, a kik Kor-Asánban és a kik Athákban laknak;
\par 31 A kik Hebronban és mindazon helyeken, ahol Dávid embereivel megfordult vala.

\chapter{31}

\par 1 És megütközének a Filiszteusok Izráellel, és elfutottak Izráel férfiai a Filiszteusok elõl, és elhullának a seb miatt a  Gilboa hegységén.
\par 2 A Filiszteusok pedig utólérték Sault és az õ fiait, és megölték a Filiszteusok Jonathánt, Abinádábot és Málkisuát, a Saul fiait.
\par 3 És a küzdelem igen heves lõn Saul ellen, mikor megtalálták õt a kézíves emberek, és igen megrettene a kézívesektõl.
\par 4 És monda Saul az õ fegyverhordozójának: Húzd ki kardodat, és szúrj keresztül azzal engem, hogy valami módon reám ne jõjjenek e körülmetéletlenek, és keresztülszúrjanak engemet és gúnyt ûzzenek belõlem. De fegyverhordozója nem akará, mert nagyon félt. Akkor Saul vevé a kardot és belébocsátkozék.
\par 5 És fegyverhordozója a mint meglátta, hogy Saul meghalt, õ is belébocsátkozék az õ kardjába, és vele együtt meghala.
\par 6 És meghala Saul és az õ három fia és fegyverhordozója, sõt emberei is mindnyájan egyenlõképen azon a napon.
\par 7 És mikor meglátták az Izráel férfiai, a kik a völgyön túl és a Jordánon túl laktak vala, hogy Izráel férfiai elfutottak, és hogy Saul és az õ fiai meghalának: elhagyták a városokat és elfutának. A Filiszteusok pedig eljövének és lakának azokban.
\par 8 És történt másnap, mikor a Filiszteusok kimentek, hogy kirabolják az elesetteket, megtalálták Sault és az õ három fiát, a kik a Gilboa hegyén esének el.
\par 9 És levágták fejét és fegyverzetét lehúzták, és elküldötték a Filiszteusok tartományába szerénszerte, hogy hírt mondanának bálványaiknak templomában és a nép között.
\par 10 És fegyverzetét Astarot templomában helyezték el, testét pedig felfüggeszték Bethsán kerítésére.
\par 11 Mikor pedig értesültek felõle Jábes-Gileád lakói, hogy mit cselekedtek a Filiszteusok Saullal:
\par 12 Felkelének mindnyájan a vitéz férfiak, és menének egész éjjel, és miután levették Saul testét és az õ fiainak testeit Bethsán kerítésérõl, elmentek Jábesbe, és ott megégették õket;
\par 13 Csontjaikat pedig felszedték, és eltemették a tamaris fa alatt Jábesben; és bõjtölének hét napig.


\end{document}