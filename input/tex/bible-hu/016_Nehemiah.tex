\begin{document}

\title{Nehemiah}


\chapter{1}

\par 1 Nehémiásnak, a Hakália fiának beszédei. És lõn a Kislév nevû hóban, a huszadik esztendõben, hogy én Susán várában valék.
\par 2 És jöve hozzám Hanáni, egy az én atyámfiai közül, és vele együtt férfiak Júdából, és tudakozódtam tõlök a fogságból megszabadult maradék zsidók felõl és Jeruzsálem felõl.
\par 3 És mondának nékem: A fogságból megszabadult maradék zsidók ott abban a tartományban nagy nyomorúságban és gyalázatban vannak; és azonfelül Jeruzsálem kõfala lerontatott, kapui tûzben égtek meg!
\par 4 Hogy pedig meghallám e beszédeket, leültem és sírtam és keseregtem napokon át, s bõjtölék és imádkozám a mennynek Istene elõtt;
\par 5 És mondék: Kérlek Uram, mennynek Istene, nagy és rettenetes Isten! Ki megtartja a szövetséget és irgalmasságot  az õt szeretõknek és az õ parancsolatait megtartóknak;
\par 6 Oh legyen figyelmetes a te füled, és szemeid legyenek nyitva, hogy meghallgassad a te szolgádnak könyörgését, melylyel én könyörgök most elõtted nappal és éjjel Izráel fiaiért, szolgáidért, és vallást tészek az Izráel fiainak bûneirõl, melyekkel vétkeztünk te ellened, én is s az én atyámnak háza vétkeztünk!
\par 7 Felette igen vétkeztünk ellened, és nem tartottuk meg a parancsolatokat, és a rendeléseket és a törvényeket, melyeket parancsoltál volt Mózesnek a te szolgádnak.
\par 8 Oh, emlékezzél meg arról a beszédrõl, melyet parancsoltál Mózesnek, a te szolgádnak, mondván: Ha ti vétkeztek, én meg elszélesztelek titeket a népek között;
\par 9 Ha pedig megtérendetek hozzám és megtartjátok parancsolataimat és cselekeszitek azokat: még ha az égnek utolsó szélén volnának is szétszórt gyermekeitek, onnan is összegyûjtöm õket és beviszem arra a helyre, melyet választottam, hogy lakozzék ott az én nevem!
\par 10 És õk a te szolgáid és a te néped, a kiket megszabadítottál a te nagy erõd által és a te erõs kezed által!
\par 11 Kérlek oh Uram, legyen figyelmetes a te füled a te szolgádnak könyörgésére és szolgáidnak könyörgésökre, a kik kívánják félni a te nevedet, és adj, kérlek, jó szerencsét most a te szolgádnak, és engedd, hogy kegyelmet találjon ama férfiú elõtt. Én ugyanis a királynak pohárnoka voltam.

\chapter{2}

\par 1 És lõn a Niszán nevû hóban, Artaxerxes királynak huszadik esztendejében, bor vala õ elõtte és én fölvevém a bort, adám a királynak; azelõtt pedig nem voltam szomorú õ elõtte.
\par 2 És monda nékem a király: Miért szomorú a te orczád, holott te beteg nem vagy? Nem egyéb ez, hanem szívednek szomorúsága! És felette igen megfélemlém.
\par 3 Akkor mondék a királynak: Örökké éljen a király! Miért nem volna szomorú az én orczám, hiszen a város, az én atyáim sírjainak helye, pusztán hever és kapui megemésztettek tûzben?!
\par 4 És monda nékem a király: Mi az, a mit kívánsz? És könyörgék a menny Istenéhez.
\par 5 És mondék a királynak: Ha tetszik a királynak és ha kedves elõtted a te szolgád, azt kérem, hogy bocsáss el engem Júdába, atyáim sírjainak városába, hogy megépítsem azt!
\par 6 És monda nékem a király (felesége pedig mellette ül vala): Meddig lészen utazásod és mikorra jösz vissza? És tetszék a királynak engem elbocsátani, miután én tudattam vele az idõt.
\par 7 És mondék a királynak: Ha tetszik a királynak, adasson nékem leveleket a folyóvizen túl lakó tiszttartókhoz, hogy hagyjanak engem utazni, míg Júdába érek;
\par 8 És adasson egy levelet Ászáfhoz, a király erdeje õréhez, hogy adjon nékem fákat gerendákul a templom várának kapuihoz, és a város kõfalához, és a házhoz, melybe költözni fogok. És megadá nékem a király az én Istenemnek rajtam nyugvó jó kegyelme szerint.
\par 9 És megérkezém a folyóvizen túl való tiszttartókhoz és átadám nékik a király leveleit. Bocsátott vala pedig velem a király fõembereket a seregbõl és lovagokat.
\par 10 Mikor pedig a Horonból való Szanballat és az Ammonita Tóbiás, a szolga, meghallották ezt, nagy bosszúságot okozott nékik, hogy jött valaki, a ki Izráel fiainak javokat keresi.
\par 11 Azután elmenék Jeruzsálembe, és ott pihenék három napig.
\par 12 És fölkelék éjszaka, én és velem néhány férfi, mert nem jelentém meg senkinek, hogy mire indítá az én Istenem az én szívemet, hogy azt Jeruzsálemért megcselekedjem, és barom sem vala velem, csak az, melyen ülök vala,
\par 13 És kimenék a völgynek kapuján éjjel, és pedig a sárkányok forrása felé, majd a szemét-kapuhoz, és vizsgálgatám Jeruzsálem kõfalait, melyek elrontattak vala, és kapuit, melyek tûz által megemésztetének.
\par 14 Azután átmenék a forrás kapujához és a király tavához, holott nem vala hely a baromnak, hogy velem átmenjen.
\par 15 Azért gyalog menék fel a völgyben éjjel és vizsgálgatám a kõfalat, azután megfordultam és bementem a völgy kapuján, és hazatértem.
\par 16 A fõemberek pedig nem tudták vala, hova mentem, és hogy mit akarok cselekedni, és sem a zsidóknak, sem a papoknak, sem az elõljáróknak, sem a fõembereknek, sem a többi munkásoknak ez ideig nem jelentettem meg.
\par 17 Ekkor mondék nékik: Ti látjátok a nyomorúságot, a melyben mi vagyunk, hogy Jeruzsálem pusztán hever és kapui tûzben égtek meg; jertek, építsük meg Jeruzsálem kõfalát, és ne legyünk többé gyalázatul!
\par 18 És megjelentém nékik az én Istenemnek rajtam nyugvó jó kegyelmét, és a király beszédit is, a melyeket nékem szólt, és mondának: Keljünk fel és építsük meg! És megerõsíték kezeiket a jóra.
\par 19 Mikor meghallák ezt a Horonból való Szanballat és az Ammonita Tóbiás, a szolga, és az arab Gesem, gúnyoltak és lenéztek minket, mondván: Micsoda ez, a mit míveltek? Talán bizony a király ellen akartok pártot ütni?
\par 20 Kiknek felelék, és mondék: A mennynek Istene, õ ad jó szerencsét nékünk, és mi mint az õ szolgái kelünk föl és építünk, néktek pedig részetek és jogotok és emlékezetetek nincsen Jeruzsálemben!

\chapter{3}

\par 1 És fölkele Eliásib, a fõpap és atyjafiai, a papok és építék a juhok kapuját; õk szentelék meg azt és állíták fel annak ajtait; építék pedig a kõfalat a Meáh toronyig, a melyet megszentelének, s azután Hanánel tornyáig;
\par 2 És õ mellette építének a Jerikóbeliek, ezek mellett pedig épített Zakkur az Imri fia;
\par 3 A halaknak kapuját pedig építék a Hasszenáa fiai, õk gerendázák be azt és állíták fel annak ajtait, kapcsait és závárait;
\par 4 Mellettök javítgatá a kõfalat Merémóth, Uriásnak, a Hakkós fiának fia, mellette pedig javítgatott Mesullám, Berekiásnak, a Mesézabéel fiának a fia, és õ mellette javítgatott Sádók, Baána fia.
\par 5 Ezek mellett pedig javítgatának a Tékoabeliek, a kiknek elõkelõi azonban nem hajták nyakukat az õ Urok munkájának jármába.
\par 6 Az ó-város kapuját pedig javítgaták Jójada, a Pászéah fia és Mesullám, a Beszódia fia; õk gerendázák be azt és állíták fel annak ajtajt, kapcsait és závárait;
\par 7 És mellettök javítgatott a Gibeonból való Melátia és a Meronothból való Jádón, Gibeonnak és Mispának lakosai, a folyóvizen túl való helytartónak hivatala helyéig;
\par 8 Mellettök javítgatott Uzziel, Harhajának fia az ötvösökkel, és õ mellette javítgatott Hanánia, a kenet-készítõk egyike, és megerõsíték Jeruzsálemet mind a széles kõfalig;
\par 9 Mellettök javítgatott Refája, Húr fia, a ki a Jeruzsálemhez tartozó tartomány felének fejedelme vala;
\par 10 Mellette javítgatott Jedája, Harumáf fia, és pedig a maga háza ellenébe, és mellette javítgatott Hattus, Hasabneja fia.
\par 11 A kõfal másik darabját javítgatá Malkija, Hárim fia és Hassub, a Pahath-Moáb fia, és a kemenczéknek tornyát.
\par 12 És mellettök javítgatott Sallum, Hallóhés fia, a ki fejedelme vala a Jeruzsálemhez tartozó tartomány másik felének, õ és az õ leányai.
\par 13 A völgy kapuját javítgaták Hánun és Zánoah lakói, õk építék meg azt és állíták fel annak ajtait, kapcsait és závárait; és a kõfalból ezer singet a szemét-kapuig.
\par 14 A szemét-kaput pedig javítgatá Malkija, Rékáb fia, Beth-Hakkerem tartományának fejedelme, õ építé meg azt és állítá fel ajtait, kapcsait és závárait.
\par 15 És a forrásnak kapuját javítgatá Sallum, a Kól-Hóze fia, Mispa tartományának fejedelme, õ építé meg és héjazá be azt és állítá fel ajtait, kapcsait és závárait; és a Selah tó kõfalát a király kertje felé, mind a garádicsokig, melyeken a Dávid városából alájõnek.
\par 16 Õ utána javítgata Nehémiás, Azbuk fia, Béth-Sur tartománya felének fejedelme, a Dávid sírjai ellenébe való helyig, és az újonnan készült tóig, és a vitézek házáig.
\par 17 Õ utána javítgatának a Léviták: Rehum, Báni fia, a ki mellett javítgata Hasábia, Kéila tartománya felének fejedelme tartománya lakosaival;
\par 18 Õ utána javítgatának atyjokfiai: Bavvai, a Hénadád fia, Kéila tartománya másik felének fejedelme;
\par 19 Javítgatá pedig õ mellette Ézer, Jésuának fia, Mispa fejedelme, a kõfal egy másik darabját, a szegleten levõ fegyveresház felmenetelének ellenébe;
\par 20 Õ utána Bárukh, Zakkai fia javítgatá buzgósággal a kõfal egy másik darabját, a szeglettõl fogva Eliásib fõpap házának ajtajáig:
\par 21 Õ utána javítgata Merémóth Uriásnak fia, a ki Hakkós fia vala, egy másik darabot, Eliásib házának ajtajától Eliásib házának végéig;
\par 22 Õ utána javítgatának a papok, a környék férfiai;
\par 23 Ezek után javítgata Benjámin és Hassub, az õ házok ellenébe, utánok pedig javítgata Azariás, Maaséja fia - a ki Ananiás fia vala - az õ háza mellett;
\par 24 Õ utánok javítgata Binnui, a Hénadád fia egy másik darabot, Azáriás házától mind a zugig és szegletig;
\par 25 Továbbá Palál, az Uzai fia, a szegletnek és a felsõ toronynak ellenébe, mely a király házából a tömlöcz udvaránál emelkedik ki; utána pedig Pedája, a Parós fia;
\par 26 (A Léviták szolgái pedig laknak vala az Ófelben, a vizek kapujának ellenéig napkelet felé és a kiemelkedõ torony ellenéig;)
\par 27 Õ utána javítgatának a Tékoabeliek egy másik darabot a kiemelkedõ nagy toronynak ellenétõl fogva az Ófel kõfaláig;
\par 28 A lovak kapuján felül javítgatának a papok, kiki az õ házának ellenébe;
\par 29 Utánok javítgata Sádók, Immér fia az õ háza ellenébe; és õ utána javítgata Semája, Sekaniás fia, a napkeleti kapu õrizõje;
\par 30 Õ utána javítgata Hananiás, Selemiás fia és Hánun, Sáláfnak hatodik fia egy másik darabot; utánok javítgata Mesullám, Berekiás fia, az õ szobája ellenébe;
\par 31 Utána javítgata Malkija, az ötvösök egyike, a Léviták szolgáinak és a kereskedõknek házáig, a törvénytevõ ház kapujának ellenébe, a szegletnek hágójáig;
\par 32 A szeglet hágója és a juhok kapuja között pedig javítgatának az ötvösök és a kereskedõk.

\chapter{4}

\par 1 Lõn pedig, mikor meghallotta Szanballat, hogy mi építjük a kõfalat, haragra gerjede és felette igen bosszankodék, és gúnyolá a zsidókat;
\par 2 És szóla az õ atyjafiai és a samáriai sereg elõtt, és ezt mondá: Mit mûvelnek e nyomorult zsidók? Vajjon megengedik-é ezt nékik? Talán áldozni fognak? Hát bevégezik ma? Avagy megelevenítik a köveket a porhalmazból, holott azok elégtek?!
\par 3 Az Ammonita Tóbiás pedig mellette állván, mondá: Bármit építsenek, ha egy róka lép fel reá, összezúzza köveiknek falát.
\par 4 Halld meg, oh mi istenünk! hogy csúffá lettünk, fordítsad gyalázásukat az õ fejökre, és add õket prédára a rabságnak földében;
\par 5 Ne fedezd el az õ hamisságokat és az õ bûnök a te orczád elõl el ne töröltessék, mert téged bosszantának az építõk elõtt!
\par 6 És építõk a falat annyira, hogy elkészült az egész fal félmagasságban, mert a nép nagy kedvvel dolgozott.
\par 7 Mikor pedig meghallotta Szanballat és Tóbiás, továbbá az Arábiabeliek, az Ammoniták és az Asdódeusok, hogy javítgattatnak Jeruzsálem kõfalai, és hogy a törések betömése megkezdõdött, felette nagy haragra gerjedének;
\par 8 És összeesküvének mindnyájan egyenlõ akarattal, hogy eljõnek Jeruzsálemet megostromolni és népét megrémíteni.
\par 9 De mi imádkozánk a mi Istenünkhöz, és állítánk ellenök õrséget nappal és éjjel, mivelhogy féltünk tõlök.
\par 10 És mondák a zsidók: Fogytán van ereje a tereh-hordónak, a rom pedig sok, és mi képtelenek vagyunk építeni a kõfalat.
\par 11 A mi ellenségeink pedig ezt mondották: Ingyen se tudják meg, se ne lássák, míg közikbe bemegyünk és õket leöljük, és megszüntetjük a munkát.
\par 12 És lõn, hogy eljöttek hozzánk mindenfelõl a zsidók, a kik õ mellettök laknak vala és nékünk tízszer is mondották: Térjetek haza!
\par 13 Azért állítám a hely alsó és nyilt részeire a kõfal mögé, odaállítám a népet nemzetségek szerint, fegyvereikkel, dárdáikkal és kézíveikkel.
\par 14 És körültekintvén, fölkeltem és így szóltam az elõljárókhoz, a fõemberekhez és a többi néphez: Ne féljetek tõlök! A nagy és rettenetes Úrra emlékezzetek, és harczoljatok testvéreitekért, fiaitokért, leányaitokért, feleségeitekért és házaitokért!
\par 15 És lõn, hogy meghallották a mi ellenségeink, hogy megtudtuk az õ szándékukat, és hogy Isten semmivé tette az õ tanácsokat: megtérénk mi mindnyájan a kõfalhoz, kiki az õ munkájához;
\par 16 De azon naptól fogva legényeim egyik része munkálkodik vala, a másik része pedig tart vala dárdákat, paizsokat, kézíveket és pánczélokat, és a fejedelmek ott állának az egész Júda háznépe mögött.
\par 17 A kõfalon munkálkodók közül a tehernek hordói egyik kezökkel, a mely a munkát végezé, rakodának, másik kezök pedig a fegyvert tartja vala,
\par 18 A kik pedig építének, azoknak fegyverök derekokra vala felkötve és így építének; a trombitás pedig mellettem állt.
\par 19 És így szóltam az elõljárókhoz, a fõemberekhez és a többi néphez: A munka felette sok és messzeterjedõ és mi elszéledvén a kõfalon, egymástól messze esünk;
\par 20 Azért oda gyûljetek hozzánk, hol a trombita szavát hallándjátok, a mi Istenünk hadakozik érettünk!
\par 21 Ekképen munkálkodunk vala; és legényeimnek fele dárdákat tart vala hajnalhasadtától fogva mind a csillagoknak feltámadásáig.
\par 22 Ugyanekkor megparancsolám a népnek, hogy minden ember legényével Jeruzsálemben háljon, hogy éjszaka õrködjenek felettünk és nappal dolgozzanak.
\par 23 És sem én, sem az én atyámfiai, sem legényeim, sem az õrizõk, a kik én utánam valának, nem vetjük vala le ruháinkat; kiki csak mosódáskor teszi vala le fegyverét.

\chapter{5}

\par 1 Lõn pedig nagy kiáltása a népnek és feleségeiknek az õ atyjokfiai, a zsidók ellen.
\par 2 Valának, a kik ezt mondják vala: Fiainkkal és leányainkkal együtt sokadmagunkkal vagyunk, nékünk gabona kell, hogy együnk és éljünk.
\par 3 És valának, a kik ezt mondják vala: Mind mezeinket, mind szõlõinket, mind házainkat zálogba kell adogatnunk, nékünk gabona kell, mert éhezünk.
\par 4 Viszontag valának, a kik ezt mondják vala: Kölcsön vettünk pénzt a király adójáért a mi mezeinkre és szõlõinkre;
\par 5 És ímé, bár a mi testünk épen olyan, mint a mi atyánkfiainak testök, s a mi fiaink olyanok, mint az õ fiaik, mi nékünk mégis rabság alá kell adnunk fiainkat és leányainkat, sõt vannak már rabszolga leányaink is, és nincs erõnk arra, hogy õket megválthatnók, hisz mezeink és szõlõink másokéi már!
\par 6 Felette nagy haragra gerjedtem azért, mikor kiáltásukat s e dolgokat hallottam;
\par 7 És magamba szállva, gondolkodtam errõl, és megfeddém az elõljárókat és fõembereket, ezt mondván nékik: Ti a ti atyátokfiaival szemben uzsoráskodtok! És szerzék õ ellenök nagy gyûlést;
\par 8 És mondám nékik: Mi megváltottuk a mi atyánkfiait, a zsidókat, a kik a pogányoknak eladattak vala, a mi tehetségünk szerint; és ti is meg akarjátok venni a ti atyátokfiait, s õk nékünk adják el magokat?! És hallgatának és nem tudának felelni semmit.
\par 9 És mondék: Nem jó dolog ez, a mit ti cselekesztek. Hát nem fogtok a mi Istenünk félemében járni, hogy valahára ne gyalázzanak már minket a pogányok, a mi ellenségeink?
\par 10 Hiszen én, atyámfiai és legényeim is pénzt és gabonát kölcsönöztünk nékik; engedjük el, kérlek, e tartozást!
\par 11 Adjátok vissza, kérlek, nékik még ma az õ mezeiket, szõlõiket, olajkerteiket és házaikat; ennekfelette, a kölcsönadott pénznek, gabonának, bornak és olajnak századát engedjétek el.
\par 12 És felelének: Visszaadjuk és tõlök nem veszünk semmit; úgy cselekszünk, a mint te mondod. Ekkor egybehívám a papokat és megeskettetém õket, hogy e beszéd szerint fognak cselekedni.
\par 13 Ruhámat is megrázám és mondék: Épen így rázzon ki az Isten minden embert az õ házából és vagyonából, és épen így legyen kirázatott és üres, valaki meg nem teljesíti e beszédet. És monda az egész gyülekezet: Ámen. És dícsérék az Urat, és e beszéd szerint cselekedett a nép.
\par 14 Sõt azon naptól fogva, melyen Júdának földére helytartójukul rendeltettem, Artaxerxes királynak huszadik esztendejétõl fogva harminczkettedik esztendejéig, azaz tizenkét esztendeig, sem én, sem az én atyámfiai a helytartónak járó kenyeret nem evénk.
\par 15 Holott az elõbbi helytartók, a kik én elõttem valának, terhelék a népet és vevének tõlök kenyérért és borért negyven ezüst sikluson felül; sõt még legényeik is zsarnokoskodának a népen. De én nem cselekedém így az Isten félelme miatt.
\par 16 Sõt még e kõfalon is dolgoztam; mezõt sem szereztünk; s minden én legényeim egybegyûltenek ott a munkára.
\par 17 Annakfelette a zsidók, továbbá a másfélszáz fõember és a kik jõnek vala mi hozzánk a körültünk lakó pogányok közül, az én asztalomnál esznek vala.
\par 18 A mit minden napra készítenek vala: egy ökröt, hat kövér juhot, és madarakat is készítének nékem, minden tíz napra vala mindenféle bor bõségesen; és mindemellett sem kivántam be a helytartó kenyerét, mert nehéz vala a szolgálat ezen a népen.
\par 19 Emlékezzél meg rólam, én Istenem, javamra, mindarról, a mit én e néppel cselekedtem!

\chapter{6}

\par 1 És lõn, hogy mikor meghallá Szanballat, Tóbiás, az Arábiabeli Gesem és a mi többi ellenségeink, hogy megépítettem a kõfalat, s hogy nem maradt azon semmi romlás, jóllehet még az ideig ajtókat nem állíttattam a kapukra:
\par 2 Külde Szanballat és Gesem hozzám ilyen izenettel: Jer és találkozzunk a faluk egyikében, az Onó-völgyében; holott õk gonoszt gondoltak ellenem.
\par 3 Küldék azért követeket hozzájok ilyen izenettel: Nagy dolgot cselekszem én, azért nem mehetek alá; megszünnék e munka, ha attól eltávozván, hozzátok mennék.
\par 4 És küldének hozzám ilyen módon négy ízben, és én ilyen módon felelék nékik.
\par 5 Küldé továbbá hozzám Szanballat ilyen módon ötödször az õ legényét, a kinek kezében egy felnyitott levél vala,
\par 6 Melyben ez vala írva: A szomszéd népek közt ez a hír és Gasmu is mondja, hogy te és a zsidók pártot akartok ütni, annakokáért építed te a kõfalat; és hogy te leszel az õ királyok, e beszédek szerint.
\par 7 Sõt még prófétákat is rendelél, a kik hirdetnék te felõled Jeruzsálemben, ezt mondván: Király van Júdában! És most e dolgoknak híre a királyhoz is elérkezik, annakokáért jer és tanácskozzunk együtt!
\par 8 Én pedig küldék õ hozzá ilyen izenettel: Nem történt semmi olyan, a minõt te mondasz, hanem csak magadtól gondoltad mindazt a te szívedben.
\par 9 Mert mindezek el akarnak vala minket rettenteni, ezt mondván: Leveszik kezöket a munkáról és az félben marad! Azért hát, oh Uram! erõsítsd meg az én kezeimet.
\par 10 És én elmentem Semájának, a Delája fiának házába, a ki Mehétabeél fia vala, és õt bezárkózva találtam, és monda: Menjünk az Isten házába, a templom belsejébe és zárjuk be a templomnak ajtait, mert eljõnek, hogy megöljenek téged, és pedig éjjel jõnek el, hogy megöljenek.
\par 11 Én pedig mondék: Avagy ily férfiúnak, mint én vagyok, illik-é futni? Hát ilyen lévén mint én, beléphet-é valaki a templomba, élvén? Nem megyek!
\par 12 És megismerém, hogy nem az Isten küldötte õt, hanem azt a prófécziát azért mondá nékem, mert Tóbiás és Szanballat felbérelték õt.
\par 13 Azért vala pedig felbérelve, hogy én megrettenjek és akképen cselekedvén, vétkezzem, hogy így rossz híremet költhessék és rágalmazhassanak.
\par 14 Emlékezzél meg én Istenem Tóbiásról és Szanballatról ezen cselekedeteik szerint, és a jövendõmondó Noádja asszonyról és a többi jövendõmondókról is, a kik rémítgetének engem.
\par 15 Elvégezteték pedig a kõfal Elul hónap huszonötödik napján, ötvenkét nap alatt.
\par 16 És lõn, hogy midõn meghallák minden mi ellenségeink, megfélemlének minden pogányok, a kik körültünk valának, és igen összeestek a saját szemeikben, és megismerék, hogy a mi Istenünktõl vitetett végbe e munka.
\par 17 E napokban is sok levelet küldének némely zsidó elõljárók Tóbiásnak, és viszont Tóbiástól sok jöve hozzájok.
\par 18 Mert sokan Júdában õ hozzá esküdtek, mivelhogy õ veje vala Sekániának, az Arah fiának, és Jóhanán, az õ fia, Mesullámnak, a Berékia fiának leányát vette volt el.
\par 19 Sõt még jó szándékait is emlegetik vala elõttem, és az én beszédeimet megvivék néki. Leveleket pedig Tóbiás folyton küldött, hogy engem elrettentene.

\chapter{7}

\par 1 És lõn, hogy midõn megépítteték a kõfal, felállítám az ajtókat és kirendeltetének a kapunállók, az énekesek és a Léviták õrizetre;
\par 2 És hadnagyokká tevém Jeruzsálem fölött Hanánit, testvéremet és Hanániást, a vár fejedelmét, mivel hogy õ hûségesebb és istenfélõbb vala sokaknál.
\par 3 És mondék nékik: Meg ne nyittassanak Jeruzsálem kapui mindaddig, míg a nap melegen nem süt, és míg az õrök ott állanak, addig tegyék be az ajtókat és zárjátok be azokat; azután állítsatok õrizõket Jeruzsálem lakosai közül, némelyeket az õ vigyázó helyökre, s másokat az õ házok ellenébe.
\par 4 A város pedig felette igen széles vala és nagy, s a nép kevés lévén benne, házak nem épültek.
\par 5 Felindítá azért az én Istenem szívemet, hogy egybegyûjtsem az elõljárókat, a fõembereket és a népet, hogy felírattassanak; és megtalálám azok nemzetségének könyvét, a kik elõször jöttek vala fel Babilóniából, melyben ily írást találék:
\par 6 Ezek a tartománynak fiai, a kik feljöttek vala a rabság foglyai közül, a kiket fogva vitetett Nabukodonozor, Babilónia királya, s most visszajövének Jeruzsálembe és Júdába, kiki az õ városába.
\par 7 Kik jövének Zorobábellel: Jésua, Nehémiás, Azariás, Raámia, Nahamáni, Mordokhai, Bilsán, Miszpereth, Bigvai, Nehum, Baána. Izráel népe férfiainak számok ez:
\par 8 Parós fiai: kétezerszázhetvenkettõ;
\par 9 Sefátja fiai: háromszázhetvenkettõ;
\par 10 Arah fiai: hatszázötvenkettõ;
\par 11 Pahath-Moáb fiai, Jésua és Joáb fiaitól: kétezernyolczszáztizennyolcz;
\par 12 Elám fiai: ezerkétszázötvennégy;
\par 13 Zattu fiai: nyolczszáznegyvenöt;
\par 14 Zakkai fiai: hétszázhatvan;
\par 15 Binnui fiai: hatszáznegyvennyolcz;
\par 16 Bébai fiai: hatszázhuszonnyolcz;
\par 17 Azgád fiai: kétezerháromszázhuszonkettõ;
\par 18 Adónikám fiai: hatszázhatvanhét;
\par 19 Bigvai fiai: kétezerhatvanhét;
\par 20 Adin fiai: hatszázötvenöt;
\par 21 Áter fiai, Ezékiástól: kilenczvennyolcz;
\par 22 Hásum fiai: háromszázhuszonnyolcz;
\par 23 Bésai fiai: háromszázhuszonnégy;
\par 24 Hárif fiai: száztizenkettõ;
\par 25 Gibeon fiai: kilenczvenöt;
\par 26 Bethlehem és Netófa férfiai: száznyolczvannyolcz;
\par 27 Anathóth férfiai: százhuszonnyolcz;
\par 28 Beth-Azmáveth férfiai: negyvenkettõ;
\par 29 Kirjáth-Jeárim, Kefira és Beéróth férfiai: hétszáznegyvenhárom;
\par 30 Ráma és Géba férfiai: hatszázhuszonegy;
\par 31 Mikmás férfiai: százhuszonkettõ;
\par 32 Béthel és Ai férfiai: százhuszonhárom;
\par 33 A másik Nébó férfiai: ötvenkettõ;
\par 34 A másik Elám fiai: ezerkétszázötvennégy;
\par 35 Hárim fiai: háromszázhúsz;
\par 36 Jerikó fiai: háromszáznegyvenöt;
\par 37 Lód, Hádid és Ónó fiai: hétszázhuszonegy;
\par 38 Szenáa fiai: háromezerkilenczszázharmincz;
\par 39 A papok: Jedája fiai Jésua családjából: kilenczszázhetvenhárom;
\par 40 Immér fiai: ezerötvenkettõ;
\par 41 Pashur fiai: ezerkétszáznegyvenhét;
\par 42 Hárim fiai: ezertizenhét;
\par 43 A Léviták: Jésua és Kadmiel fiai, Hódávia fiaitól: hetvennégy;
\par 44 Az énekesek: Asáf fiai: száznegyvennyolcz;
\par 45 A kapunállók: Sallum fiai, Áter fiai, Talmón fiai, Akkub fiai, Hatita fiai, Sóbai fiai: százharmincznyolcz;
\par 46 A Léviták szolgái: Siha fiai, Hasufa fiai, Tabbaóth fiai,
\par 47 Kérósz fiai, Szia fiai, Pádón fiai,
\par 48 Lebána fiai, Hagába fiai, Salmai fiai,
\par 49 Hanán fiai, Giddél fiai, Gahar fiai,
\par 50 Reája fiai, Resin fiai, Nekóda fiai,
\par 51 Gazzám fiai, Uzza fiai, Pászéah fiai,
\par 52 Bészai fiai, Meunim fiai, Nefiszim fiai,
\par 53 Bakbuk fiai, Hakufa fiai, Harhur fiai,
\par 54 Basluth fiai, Mehida fiai, Harsa fiai,
\par 55 Barkósz fiai, Sziszera fiai, Temah fiai,
\par 56 Nesiah fiai, Hatifa fiai,
\par 57 A Salamon szolgáinak fiai: Szótai fiai, Szófereth fiai, Perida fiai,
\par 58 Jaalá fiai, Darkón fiai, Giddél fiai,
\par 59 Sefátia fiai, Hattil fiai, Pókhereth-Hassebaim fiai, Ámon fiai;
\par 60 Összesen a Léviták szolgái és a Salamon szolgáinak fiai, háromszázkilenczvenkettõ.
\par 61 És ezek, a kik feljövének Tel-Melahból, Tel-Harsából, Kerub-Addán-Immérbõl, de nem mondhatták meg családjukat és eredetüket, hogy Izráelbõl valók-é?
\par 62 Delája fiai, Tóbiás fiai, Nekóda fiai, hatszáznegyvenkettõ;
\par 63 És a papok közül: Habája fiai, Hakkós fiai, Barzillai fiai, a ki a Gileádbeli Barzillai leányai közül vett magának feleséget és ezek nevérõl nevezteték;
\par 64 Ezek keresték írásukat, tudniillik nemzetségök könyvét, de nem találák, miért is kirekesztetének a papságból;
\par 65 És megmondá nékik a király helytartója, hogy ne egyenek a szentséges áldozatból, mígnem a pap ítél az Urimmal és Tummimmal;
\par 66 Mind az egész gyülekezet együtt negyvenkétezerháromszázhatvan.
\par 67 Szolgáikon és szolgálóikon kivül - ezek valának hétezerháromszázharminczheten - valának nékik énekes férfiaik és asszonyaik kétszáznegyvenöten;
\par 68 Lovaik hétszázharminczhat, öszvéreik kétszáznegyvenöt;
\par 69 Tevéik négyszázharminczöt, szamaraik hatezerhétszázhúsz.
\par 70 Némelyek pedig a családfõk közül adakozának az építésre: a király helytartója ada a kincsekhez aranyban ezer dárikot, ötven medenczét ötszázharmincz papi ruhát;
\par 71 A többi családfõk pedig adának az építés költségére aranyban húszezer dárikot, és ezüstben kétezerkétszáz mánét;
\par 72 És a mit a többi nép ada, az aranyban húszezer dárik, és ezüstben kétezer máne, és hatvanhét papiruha vala.
\par 73 És lakozának mind a papok, mind a Léviták, mind a kapunállók, mind az énekesek, mind a nép fiai, mind a Léviták szolgái, szóval az egész Izráel a magok városaikban.

\chapter{8}

\par 1 Mikor pedig eljöve a hetedik hónap, és Izráel fiai az õ városaikban lakozának; felgyûle az egész nép egyenlõképen a piaczra, mely a vizek kapuja elõtt vala, és mondák az írástudó Ezsdrásnak, hogy hozza elõ a Mózes törvényének könyvét, melyet parancsolt vala az Úr Izráelnek.
\par 2 Elõhozá azért Ezsdrás pap a törvényt a gyülekezet eleibe, melyben együtt valának férfiak és asszonyok és mindazok, a kik azt értelemmel hallgathaták, a hetedik hónap elsõ napján.
\par 3 És olvasa abból a piaczon, mely a vizek kapuja elõtt vala, kora reggeltõl fogva mind délig, a férfiak és az asszonyok elõtt és mindazok elõtt, a kik azt érthetik vala, mivel az egész nép nagy figyelemmel hallgatá a törvényt.
\par 4 Áll vala pedig az írástudó Ezsdrás egy ezen czélra csinált faemelvényen, és mellette álla jobb keze felõl Mattithia, Sema, Anája, Uriás, Hilkiás, Maaszéja, balkeze felõl pedig: Pedája, Misáel, Malkija, Hásum, Hasbaddána, Zakariás, Mesullám.
\par 5 És felnyitá Ezsdrás a könyvet az egész nép szemei elõtt, mert fölötte vala az egész népnek; és mikor fölnyitá, felálla az egész nép.
\par 6 És áldá Ezsdrás az Urat, a nagy Istent, és felele rá az egész nép felemelt kezekkel: Ámen! Ámen! És meghajtván magukat, leborulának az Úr elõtt arczczal a földre.
\par 7 Jésua, Báni, Serébia, Jámin, Akkub, Sabbethai, Hódija, Maaszéja, Kelita, Azáriás, Józabád, Hanán, Pelája Léviták pedig magyarázzák vala a népnek a törvényt, s a nép a maga helyén áll vala.
\par 8 Olvasának pedig a könyvbõl, Isten törvényébõl világosan, s azután magyarázának, és a nép megérté az olvasottakat.
\par 9 Ekkor Nehémiás, a király helytartója és Ezsdrás, a pap, az írástudó és a Léviták, a kik magyaráznak vala a népnek, így szólának az egész néphez: E nap szent az Úrnak, a ti Isteneteknek ne keseregjetek és ne sírjatok, mert sír vala az egész nép, mikor a törvény beszédeit hallá.
\par 10 És õ mondá nékik: Menjetek, egyetek kövért és igyatok édest és küldjetek belõle részt annak, a kinek semmi nem készíttetett, mert szent e nap a mi Urunknak, és ne bánkódjatok, mert az Úrnak öröme a ti erõsségtek.
\par 11 A Léviták is csendesíték vala az egész népet, mondván: Hallgassatok, mert e nap szent és ne bánkódjatok!
\par 12 Elméne azért mind az egész nép enni és inni és részt küldeni és szerezni nagy vígasságot, mert megértették a beszédeket, a melyekre õket tanították vala.
\par 13 Másod napon pedig összegyûlének az egész népnek, a papoknak és Lévitáknak családfõi Ezsdráshoz, az írástudóhoz, hogy megértenék a törvénynek beszédit.
\par 14 Találák pedig írva a törvényben, melyet parancsolt vala az Úr Mózes által, hogy Izráel fiainak leveles színekben kell lakozniok az ünnepen a hetedik hónapban,
\par 15 És hogy nékik tudatniok kell és ki kell hirdettetniök minden városaikban és Jeruzsálemben ezt: Menjetek ki a hegyre, és hozzatok lombokat a nemes és vad olajfáról, mirtus és pálmalombokat, szóval akármely sûrû levelû fáról lombokat, hogy csináljatok leveles színeket, a mint meg van írva.
\par 16 Kiméne azért a nép, és hozának lombokat, és csinálának magoknak leveles színeket, kiki a maga háza tetején és pitvaraikban, továbbá az Isten háza pitvariban, a vizek kapujának piaczán és Efraim kapujának piaczán.
\par 17 Csinála pedig az egész gyülekezet, mely hazatért a fogságból, leveles színeket, és lakának a leveles színekben; mert nem cselekedtek vala így Józsuénak, a Nún fiának idejétõl fogva Izráel fiai mind az napiglan, és lõn felette igen nagy örömük.
\par 18 És olvasának az Isten törvényének könyvébõl minden napon, az elsõ naptól fogva mind az utolsó napig, és ünneplének hét napon át; a nyolczadik napon pedig egybegyülekezés tartatott a törvény szerint.

\chapter{9}

\par 1 Azután ugyanezen hónak huszonnegyedik napján egybegyûlének Izráel fiai és bõjtölének, gyászba öltözvén és port hintvén fejökre.
\par 2 És elválának az Izráel magvából valók minden idegenektõl, és elõállván, vallást tõnek az õ bûneikrõl és atyáik hamisságáról.
\par 3 És megállának a helyökön, és olvasának az Úrnak, az õ Istenöknek törvénye könyvébõl a nap negyedrésze alatt, negyedrésze alatt pedig vallást tõnek és leborulának az Úr elõtt, az õ Istenök elõtt.
\par 4 És felálla a Léviták emelvényére Jésua, Báni, Kadmiel, Sebánia, Bunni, Serébia, Báni és Kenáni és kiáltának nagy felszóval az Úrhoz, az õ Istenökhöz.
\par 5 És mondának a Léviták, Jésua, Kadmiel, Báni, Hasabnéja, Serébia, Hódija, Sebánia, Petáhia: Nosza áldjátok az Urat, a ti Isteneteket öröktõl fogva mindörökké; és áldják a te dicsõséges nevedet, mely magasabb minden áldásnál és dícséretnél!
\par 6 Te vagy egyedül az Úr! Te teremtetted az eget, az egeknek egeit és minden seregöket, a földet és mindent a mi rajta van, a tengereket minden bennök valókkal együtt; és te adsz életet mindnyájoknak, és az égnek serege elõtted borul le.
\par 7 Te vagy az Úr, az Isten, a ki választottad Ábrahámot és kihozád õt a Káldeusoknak Úr nevû városából, és nevezéd õt Ábrahámnak.
\par 8 És találád az õ szívét hûnek te elõtted és szerzél vele szövetséget, hogy adod  a Kananeusok, Hitteusok, Emoreusok, Perizeusok, Jebuzeusok, Girgázeusok földét, hogy adod az õ magvának; és megteljesítéd beszédeidet, mert igaz vagy te!
\par 9 Megtekintéd annakfelette a mi atyáinknak Égyiptomban való nyomorúságát, és az õ kiáltásukat meghallád a Veres tengernél,
\par 10 És tevél jeleket és csodákat a Faraón és minden szolgáin és földének egész népén, mert tudtad, hogy kevélyen cselekedtek a te néped ellen; és szerzél magadnak nevet, mint e mai napság is van.
\par 11 És a tengert kétfelé választád õ elõttök és szárazon menének át a tenger közepén; üldözõiket pedig mélységbe borítád, mint követ a vizekbe.
\par 12 És felhõnek oszlopában vezetéd õket nappal s tûznek oszlopában éjjel, hogy megvilágítsd nékik az utat, melyen menjenek.
\par 13 És a Sinai hegyre leszállál, s szólál velök az égbõl, s adál nékik helyes végzéseket, igaz törvényeket, jó rendeléseket és parancsolatokat.
\par 14 A te szent szombatodat is megjelentéd nékik, és parancsolatokat, rendeléseket és törvényt parancsolál nékik szolgád, Mózes által.
\par 15 És kenyeret az égbõl adál nékik éhségökben, és vizet a kõsziklából hozál ki  nékik szomjúságokban; és mondád nékik, hogy menjenek be és bírják a földet, melyre nézve fölemelted kezedet, hogy nékik adod.
\par 16 Õk pedig, a mi atyáink, felfuvalkodának, és megkeményíték nyakukat és nem hallgaták parancsolataidat;
\par 17 És vonakodának engedelmeskedni, és meg nem emlékezének a te csudatételeidrõl, a melyeket velök cselekedtél, hanem megkeményíték nyakukat, és arra adák fejöket, hogy visszatérnek rabságukba az õ makacskodásukban; de te bûnbocsánatnak Istene vagy, könyörülõ és irgalmas, hosszútûrõ és nagy kegyelmességû, és el nem hagytad õket.
\par 18 Sõt még borjúképet is csinálának magoknak, és ezt mondják: Ez a te Istened, a ki kihozott téged Égyiptomból, és nagy bosszúsággal illettek téged;
\par 19 És te nagy irgalmasságodban nem hagyád el õket a pusztában; a felhõnek oszlopa nem távozék el felõlök nappal, hogy vezérlené õket az úton, és a tûznek oszlopa éjjel, hogy világítana nékik az úton, melyen menének.
\par 20 És a te jó lelkedet adád nékik, hogy õket oktatná; és mannádat nem vonád meg szájoktól, s vizet adál nékik szomjúságokban.
\par 21 És negyven esztendeig tápláltad õket a pusztában, fogyatkozásuk nem vala, ruháik el nem nyûttenek és lábaik meg nem dagadának.
\par 22 Adál annakfelette nékik országokat és népeket, és elosztád ezeket határok szerint és õk elfoglalák Sihón földét és Hesbon királyának földét, és Ógnak, Básán királyának földét.
\par 23 Fiaikat pedig megsokasítád, mint az ég csillagait, és bevivéd õket a földre, mely felõl megmondád  atyáiknak, hogy bemennek arra, hogy bírják azt.
\par 24 És bemenének a fiak és elfoglalák a földet, és megalázád elõttök a földnek lakóit, a Kananeusokat, kezökbe adád õket, mind királyaikat, mind a föld népeit, hogy cselekedjenek velök kedvök szerint.
\par 25 És megvevének erõs városokat és kövér földet, s elfoglalának minden jóval teljes házakat, kõbõl vágott kutakat, szõlõs és olajfás kerteket és sok gyümölcsfát, s evének és megelégedének és meghízának, sgyönyörködének a te nagy jóvoltodban.
\par 26 Makacskodának pedig és pártot ütének ellened, és veték törvényedet hátuk mögé; prófétáidat is meggyilkolák, a kik bizonyságot tõnek ellenök, hogy õket te hozzád térítenék, és nagy bosszúsággal illettek téged.
\par 27 Annakokáért adád õket nyomorgatóik kezébe és megnyomorgaták õket; de nyomorúságok idején hozzád kiáltának, és te az égbõl meghallgatád és nagy irgalmasságod szerint adál nékik szabadítókat, és megszabadíták õket nyomorgatóik kezébõl.
\par 28 És mikor megnyugodtak vala, ismét gonoszt cselekvének elõtted, és te átadád õket ellenségeik kezébe, a kik uralkodának rajtok; s midõn ismét hozzád kiáltának, te az égbõl meghallgatád és megszabadítád õket irgalmasságod szerint sokszor.
\par 29 És bizonyságot tevél ellenök, hogy visszatérítsd õket törvényedhez; de õk felfuvalkodának s nem hallgaták parancsolatidat, és végzéseid ellen vétkezének, melyeket ha valaki megtart, él azok által; és makacskodva hátat fordítának, s nyakukat megkeményíték és nem engedelmeskedének.
\par 30 És mégis meghosszabbítád felettök kegyelmedet sok esztendõkön át, s bizonyságot tevél ellenök lelkeddel, prófétáid által, de nem figyelmezének; annakokáért adád õket a föld népeinek kezébe.
\par 31 De nagy irgalmasságod szerint meg nem emésztéd õket, sem el nem hagyád õket, mert te könyörülõ és irgalmas Isten vagy.
\par 32 Most annakokáért, oh mi Istenünk, nagy, erõs és rettenetes Isten, a ki megõrized a szövetséget és irgalmasságot, ne legyen kicsiny elõtted mindaz a nyomorúság, a mely utolért minket, királyainkat, fejedelmeinket, papjainkat, prófétáinkat, atyáinkat és egész népedet, Assiria királyainak napjaitól fogva mind e mai napig.
\par 33 Hiszen te igaz vagy mindenekben, valamelyek reánk jövének; mert te igazságot mûveltél, mi pedig hamisságot cselekvénk!
\par 34 Hiszen a mi királyaink, fejedelmeink, papjaink és atyáink nem cselekedték törvényedet és nem figyelmeztek parancsolataidra és bizonyságtételeidre, melyek által bizonyságot tettél õ ellenök;
\par 35 És õk az õ országokban s a te nagy jóvoltodban, melylyel megáldád õket, s a széles és kövér földön, melyet nékik adál, nem akartak szolgáid lenni; és nem tértek meg gonosz cselekedeteikbõl.
\par 36 És ímé mi most szolgák vagyunk azon a földön, melyet a mi atyáinknak adál, hogy ennék annak gyümölcsét és javát, ímé mi szolgák vagyunk azon;
\par 37 És termését bõven hozza a királyoknak, a kiket fölibünk helyezél a mi bûneinkért, s testeinken uralkodnak és barmainkon az õ kedvök szerint; és mi nagy nyomorúságban vagyunk!
\par 38 És mindemellett mi erõs kötést szerzénk és azt aláírtuk, melyet megpecsételének a mi fejedelmeink, Lévitáink és papjaink.

\chapter{10}

\par 1 A megpecsételt kötéseken pedig ott valának: Nehémiás, a király helytartója, a Hakhalia fia és Sédékiás;
\par 2 Serája, Azariás, Jeremiás,
\par 3 Pashur, Amaria, Malakiás,
\par 4 Hattus, Sebánia, Malluk,
\par 5 Hárim, Merémóth, Abdiás,
\par 6 Dániel, Ginnethón, Bárukh,
\par 7 Mesullám, Abija, Mijámin,
\par 8 Maazia, Bilgai, Semája; ezek papok voltak.
\par 9 A Léviták pedig ezek: Jésua, Azania fia, Binnui, a Hénadád fiai közül, Kadmiel,
\par 10 És atyjafiaik: Sebánia, Hódija, Kelita, Pelája, Hanán,
\par 11 Mika, Rehób, Hasábia,
\par 12 Zakkur, Serébia, Sebánia,
\par 13 Hódija, Báni, Beninu.
\par 14 A nép fejei pedig ezek: Parós Pahath-Moáb, Elám, Zattu, Báni,
\par 15 Bunni, Azgád, Bébai,
\par 16 Adonija, Bigvai, Adin,
\par 17 Áter, Ezékiás, Azzur,
\par 18 Hódija, Hásum, Bésai,
\par 19 Hárif, Anathóth, Nébai,
\par 20 Magpiás, Mesullám, Hézir,
\par 21 Mesézabel, Sádók, Jaddua,
\par 22 Pelátia, Hanán, Anája,
\par 23 Hóseás, Hanánia, Hásub,
\par 24 Hallóhes, Pilha, Sóbek,
\par 25 Rehum, Hasabná, Maaszéja,
\par 26 És Ahija, Hanán, Anán,
\par 27 Mallukh, Hárim, Baána.
\par 28 És a nép többi része, a papok, a Léviták, a kapunállók, az énekesek, a Léviták szolgái és mindenki, a ki elkülöníté magát a tartományok népeitõl, az Isten törvényéhez állván, feleségeik, fiaik, leányaik, mindenki, a kinek értelme és okossága vala,
\par 29 Csatlakozának atyjokfiaihoz, elõljáróikhoz, és átok mellett esküt tevének, hogy az Isten törvényében járnak, a mely Mózes által az Isten szolgája által adatott vala ki; s hogy megõrzik és cselekeszik az Úrnak, a mi Urunknak minden parancsolatait, végzéseit és rendeléseit;
\par 30 És hogy nem fogjuk adni leányainkat feleségül a föld népeinek, sem az õ leányaikat nem fogjuk venni a mi fiainknak,
\par 31 És hogy a föld népeitõl, akik árúkat és mindenféle gabonát hoznak szombatnapon eladni, nem fogunk venni tõlök szombaton és egyéb szent napon, és hogy nem fogjuk bevetni a  földet a hetedik esztendõben és elengedünk minden tartozást.
\par 32 És megállapítánk magunkra nézve parancsolatokat: hogy vetünk magunkra harmadrész siklust esztendõnként a mi Istenünk házának szolgálatára,
\par 33 A szent kenyerekre, a szüntelen való ételáldozatra és a szüntelen való égõáldozatra, a szombatokon, újholdak napján viendõ áldozatokra, az ünnepnapokra, a szent dolgokra, a bûnért való áldozatokra, hogy mindezek megtisztítsák Izráelt; és a mi Istenünk házának minden munkájára;
\par 34 Sorsot veténk továbbá a fa hozása felõl a papok, Léviták és a nép között, hogy hordjuk azt a mi Istenünk házába családaink szerint bizonyos idõkben esztendõnként, hogy égjen az Úrnak, a mi Istenünknek oltárán, a mint meg van írva a törvényben,
\par 35 És hogy felviszszük földünknek elsõ zsengéjét és minden gyümölcsének elsõ zsengéjét esztendõnként az Úr házába;
\par 36 Annakfelette fiainknak, barmainknak elsõ fiait, a mint meg van írva a törvényben; továbbá, hogy elviszszük szarvasmarháinknak és juhainknak elsõ fajzásait a mi Istenünk házába a papoknak, a kik szolgálnak a mi Istenünk házában.
\par 37 És a mi lisztjeinknek, felemelt áldozatainknak, minden fa gyümölcsének, mustnak és olajnak zsengéjét felviszszük a papoknak, a mi Istenünk házának kamaráiba, földünknek tizedét pedig a Lévitákhoz; mert õk, a Léviták szedik be a  tizedet minden földmûveléssel foglalatoskodó városainkban.
\par 38 És legyen a pap, Áron fia a Lévitákkal, midõn a Léviták a tizedet beszedik; és a Léviták vigyék fel a tizednek tizedét a mi Istenünk házába, a tárháznak kamaráiba;
\par 39 Mert a kamarákba kell hozniok Izráel fiainak és Lévi fiainak a gabonának, a mustnak és az olajnak ajándékát, holott vannak a szent edények és az Istennek szolgáló papok, a kapunállók és az énekesek; és mi nem hagyjuk el a mi Istenünknek házát.

\chapter{11}

\par 1 És lakozának a nép fejedelmei Jeruzsálemben, a többi nép pedig sorsot vete, hogy minden tíz emberbõl vinnének egyet Jeruzsálembe, a szent városba lakóul, a többi kilencz pedig marad a maga városában.
\par 2 És áldá a nép mindazon férfiakat, a kik önkéntesen vállalkozának arra, hogy Jeruzsálemben lakoznak.
\par 3 Ezek pedig a tartomány fejei, a kik Jeruzsálemben megtelepedének; de Júda városaiban lakozék kiki az õ örökségében, városaikban: az Izráel, a papok, a Léviták, a Léviták szolgái és Salamon szolgáinak fiai.
\par 4 Jeruzsálemben azért megtelepedének a Júda fiai közül és Benjámin fiai közül: Júda fiai közül: Athája, Uzzija fia, ki Zakariás fia, ki Amaria fia, ki Sefátia fia, ki Mahalalél fia vala, a Pérecz fiai közül,
\par 5 És Maaszéja, Báruk fia, ki Kolhóze fia, ki Hazája fia, ki Adája fia, ki Jójárib fia, ki Zakariás fia, ki Hasilóni fia vala.
\par 6 Pérecz fiai összesen, a kik Jeruzsálemben laknak vala, négyszázhatvannyolcz erõs férfi;
\par 7 Benjámin fiai pedig ezek: Szallu, Mesullám fia, ki Jóed fia, ki Pedája fia, ki Kólája fia, ki Maaszéja fia, ki Ithiel fia, ki Ézsaiás fia vala,
\par 8 És õ utána Gabbai, Szallai, kilenczszázhuszonnyolczan;
\par 9 És Jóel, a Zikri fia elõljárójok volt ezeknek, Júda pedig, a Hasszenua fia, a város másodrendû elõljárója.
\par 10 A papok közül: Jedája, Jójárib fia, Jákhin,
\par 11 Serája pedig, a Hilkiás fia, ki Mesullám fia, ki Sádók fia, ki Merájóth fia, ki Akhitúb fia volt, az Isten házának fejedelme vala,
\par 12 És atyjokfiai, a kik a ház munkáját teljesítik vala, nyolczszázhuszonketten; és Adája, a Jeróhám fia, ki Pelália fia, ki Amsi fia, ki Zakariás fia, ki Pashur fia, ki Malkija fia vala,
\par 13 És az õ atyjafiai, a családfõk kétszáznegyvenketten, és Amasszai, Azarél fia, ki Ahzai fia, ki Mesillémóth fia, ki Immér fia vala,
\par 14 És az õ atyjokfiai, vitéz férfiak, százhuszonnyolczan, elõljárójok volt pedig Zabdiel, a Haggedólim fia.
\par 15 A Léviták közül pedig: Semája, Hassub fia, ki Azrikám fia, ki Hasábia fia, ki Bunni fia vala.
\par 16 És Sabbethai és Józabád, az Isten házának külsõ munkája felett felügyelõk valának a Léviták fejei közül,
\par 17 És Mattánia, - a Mika fia, ki Zabdi fia, ki Asáf fia vala, - a ki vezetõje volt a hálaadó éneklésnek, s elkezdé azt az imádság után, és Bakbukia, a ki másodrendû vala atyjafiai között, és Abda, Sammua fia, ki Gálál fia, ki Jeduthun fia vala;
\par 18 A Léviták összesen a szent városban: kétszáznyolczvannégyen.
\par 19 A kapunállók pedig: Akkub, Talmón és atyjokfiai, a kapukat õrizõk százhetvenketten.
\par 20 A többi Izráeliták pedig, a papok, a Léviták lakozának Júdának minden városaiban kiki az õ örökségében.
\par 21 A Léviták szolgái pedig lakozának az Ófelben, és Siha és Gipsa valának elõljáróik a Léviták szolgáinak.
\par 22 A Léviták elõljárója vala Jeruzsálemben, az Isten háza szolgálatjánál, Uzzi, a Báni fia, - ki Hasábia fia, ki Mattánia fia, ki Mika fia volt, - Asáf fiai, az énekesek közül;
\par 23 Mert õk, a Léviták, a király parancsolata szerint, az énekesek pedig kötés szerint végezék naponként való teendõjüket;
\par 24 Petáhja pedig, a Mesézabel fia, Júda fiának, Zerahnak fiai közül, a király oldala mellett vala a nép minden dolgában.
\par 25 A falukban, ezek határaiban pedig Júda fiai közül lakozának Kirjáth-Arbában és mezõvárosaiban, Dibonban és mezõvárosaiban, Jekabseélben és faluiban.
\par 26 Jésuában, Móladában és Beth-Péletben.
\par 27 Hasár-Suálban, Beér-Sebában és mezõvárosaiban,
\par 28 Siklágban, Mekónában és mezõvárosaiban,
\par 29 En-Rimmonban, Sorában, Jarmuthban,
\par 30 Zánoáhban, Adullámban és faluiban, Lákisban és határaiban, Azékában és mezõvárosaiban, - ekként laknak vala Beér-Sebától mind a Hinnóm völgyéig.
\par 31 Benjámin fiai pedig laknak vala Gebától fogva Mikmásban, Ajában, Béthelben és mezõvárosaiban.
\par 32 Anatótban, Nóbban, Anániában,
\par 33 Hásórban, Rámában, Gittajimban,
\par 34 Hádidban, Sebóimban, Neballátban,
\par 35 Lódban, Onóban, s a mesteremberek völgyében.
\par 36 A Léviták közül pedig némely júdai osztályok Benjáminhoz csatlakozának.

\chapter{12}

\par 1 Ezek pedig a papok és Léviták, a kik feljövének Zorobábellel, Sealtiél fiával és Jésuával: Serája, Jeremiás, Ezsdrás.
\par 2 Amária, Mallukh, Hattus,
\par 3 Sekánia, Rehum, Meremóth,
\par 4 Iddó, Ginnethói, Abija,
\par 5 Mijámin, Maádia, Bilga,
\par 6 Semája, Jójárib, Jedája,
\par 7 Szallu, Ámók, Hilkija, Jedája, - ezek valának fejei a papoknak és atyjokfiainak Jésua napjaiban.
\par 8 A Léviták pedig ezek: Jésua, Binnui, Kadmiel, Serébia, Jehuda, Mattánia, - ki vezetõje vala hálaadó éneklésnek, atyjafiaival.
\par 9 És Bakbukia, s Unni, az õ atyjokfiai velök szemben, tisztök szerint.
\par 10 És Jésua nemzé Jojákimot, s Jojákim nemzé Eliásibot, s Eliásib nemzé Jójadát,
\par 11 És Jójada nemzé Jónathánt, s Jónathán nemzé Jadduát.
\par 12 Jójákim napjaiban pedig papok, családfõk valának: Serája családjában Merája, Jerémiáséban Hanánia,
\par 13 Ezsdráséban Mesullám, Amáriéban Jóhanán,
\par 14 Melikuéban Jónathán, Sebaniáéban József,
\par 15 Háriméban Adna, Merajóthéban Helkai,
\par 16 Iddoéban Zakhariás, Ginnethónéban Mesullám,
\par 17 Abijáéban Zikri, Minjáminéban és Móadiáéban Piltai,
\par 18 Bilgáéban Sammua, Semájáéban Jónathán,
\par 19 Jójáribéban Mattenai, Jedajáéban Uzzi,
\par 20 Szallaiéban Kallai, Amókéban Éber,
\par 21 Hilkijáéban Hasábia, Jedajáéban Nethanéel.
\par 22 A Léviták közül Eliásib, Jójada, Jóhanán és Jaddua napjaiban felírattak a családfõk, a papok pedig mind a persiai Dárius országlásáig.
\par 23 A Léviták családfõi felírattattak a krónikák könyvében Jóhanánnak, Eliásib fiának  napjaiig;
\par 24 A Léviták családfõi valának: Hasábia, Serébia, Jésua a Kadmiel fia, és az õ atyjokfiai velök szemben a dícsérõ és hálaadó éneklésre, Dávidnak, az Isten emberének parancsolata szerint, mint egy éneklõ sereg együtt a másikkal.
\par 25 Mattánia, Bakbukia, Obádia, Mesullám, Talmón, Akkub pedig kapunállók, õrt állván a kapuk kincses házainál.
\par 26 Ezek valának Jojákim napjaiban, ki Jésua fia, ki Jósadák fia volt, és Nehémiásnak, a helytartónak és Ezsdrásnak, a törvénytudó papnak  napjaiban.
\par 27 Jeruzsálem kõfalának felszentelésekor pedig fölkeresék a Lévitákat minden õ helyeiken, hogy behozzák õket Jeruzsálembe, hogy véghezvigyék a felszentelést örömmel és hálaadással és énekléssel, czimbalmokkal, lantokkal és cziterákkal.
\par 28 És összegyûlének az énekesek fiai mind Jeruzsálem környékérõl, mind pedig a Netófátiak faluiból.
\par 29 Mind Beth-Gilgálból, mind Geba és Azmáveth határaiból, mert falukat építének magoknak az énekesek Jeruzsálem körül.
\par 30 És minekutána megtisztíták magokat a papok és a Léviták, megtisztíták a népet is, s a kapukat és a kõfalat.
\par 31 Azután felvivém Júda fejedelmeit a kõfalhoz és rendelék két nagy hálaadást éneklõ sereget és csapatokat; az egyik jobbkéz felé méne a kõfal mellett a szemétkapu felé,
\par 32 És utánok méne Hósaája és Júda fejedelmeinek fele,
\par 33 És Azária, Ezsdrás és Mesullám,
\par 34 Júda és Benjámin, Semája és Jeremiás,
\par 35 A papok fiai közül kürtökkel; és Zakariás, a Jónathán fia, ki Semája fia, ki Mattánia fia, ki Mikája fia, ki Zakkur fia, ki Asáf fia vala,
\par 36 És az õ atyjafiai: Semája, Azarél, Milalai, Gilalai, Máaj, Nethanéel, Júda és Hanáni, Dávidnak, az Isten emberének éneklõszerszámaival; Ezsdrás pedig, az írástudó elõttök megy vala;
\par 37 Méne pedig e sereg a forrás kapuja felé, s egyenesen fölmenének Dávid városának lépcsõin a kõfalhoz vezetõ lépcsõre, s menének Dávid háza mellett s a vizek kapujáig napkelet felé.
\par 38 A hálaadást éneklõ második sereg pedig, mely átellenben megy vala - én pedig és a népnek fele utánok - méne a kõfal mellett, a kemenczék tornya mellett, mind a széles kõfalig.
\par 39 És az Efraim kapuja mellett és az ó város kapuja mellett és a halaknak kapuja mellett és a Hananél tornya és a Meáh torony mellett mind a juhok kapujáig, s megállának a tömlöcz kapujában.
\par 40 És megálla a hálaadást éneklõ mindkét sereg az Isten házánál, és én s a fejedelmeknek fele velem,
\par 41 És a papok: Eljákim, Maaséja, Minjámin, Mikája, Eljoénai, Zakariás, Hanánia kürtökkel,
\par 42 És Maaséja, Semája, Eleázár, Uzzi, Jóhanán, Malkija, Elám és Ezer; és zengedeznek vala az énekesek, és Jizrahia az elõljáró.
\par 43 És áldozának azon napon nagy áldozatokkal, és vigadának, mert az Isten megvidámítá õket nagy örömmel, sõt az asszonyok és gyermekek is vigadának; és Jeruzsálemnek öröme nagy messze hallatott.
\par 44 És rendeltetének ama napon férfiak a kincseknek, felemelt áldozatoknak, elsõ zsengéknek és a tizedeknek tárházai fölé, hogy összegyûjtsék azokba a városok határaiból a törvény szerint való részöket a papoknak és a Lévitáknak, mert örvendeze Júda a papokon és a Lévitákon, a kik ott állának tisztökben,
\par 45 Mint a kik megõrzik Istenök rendtartását s a tisztaság rendtartását, és az énekesek és a kapunállók is ott állának tisztökben, Dávidnak és az õ fiának, Salamonnak parancsolatja szerint;
\par 46 Mert Dávidnak napjaiban Asáf vala régtõl fogva az énekesek és az Istennek hálát adó és õt dícsérõ éneklésnek vezetõje.
\par 47 És a Zorobábel és Nehémiás napjaiban megadja vala az egész Izráel az énekeseknek és kapunállóknak az õ részöket naponként, nevezetesen oda szentelik vala azt a Lévitáknak, a Léviták pedig az Áron fiainak.

\chapter{13}

\par 1 Azon a napon olvasának a Mózes könyvébõl a népnek hallatára és írva találák benne, hogy Ammón és Moáb soha be ne menjen az Isten gyülekezetébe,
\par 2 Mivelhogy nem mentek vala eleikbe Izráel fiainak kenyérrel és vízzel, sõt bérbe fogadták ellenök Bálámot, hogy õket megátkozná, de a mi Istenünk az átkot áldásra fordítá.
\par 3 És lõn, hogy mikor hallották e törvényt, kirekesztének Izráel közül minden elegy-belegy népet.
\par 4 Ennekelõtte pedig Eliásib, a pap, ki Istenünk házának kamarái fölé rendelteték, rokonságba jutott Tóbiással;
\par 5 És átengedett néki egy nagy kamarát, holott abban annakelõtte az ételáldozatot, a tömjént, az edényeket és az olajnak, mustnak és gabonának tizedét, mint a kapunállóknak, énekeseknek és Lévitáknak törvény szerint való részét és a papoknak ajándékát helyeztetik vala el.
\par 6 Mind ennek történtekor én nem valék Jeruzsálemben, mert Artaxerxes babilóniai királynak harminczkettedik esztendejében visszamentem vala a királyhoz, s napok múltán újra szabadságot kértem a királytól.
\par 7 És visszatérék Jeruzsálembe, és megértém e gonoszt, melyet Eliásib cselekedett vala Tóbiásért, hogy átengedett néki egy kamarát az Isten házának pitvaraiban.
\par 8 Igen gonosznak tetszék pedig ez nékem s kivettetém Tóbiás házának minden edényeit abból a kamarából;
\par 9 És parancsolatomra megtisztíták a kamarákat, és visszahordatám azokba Isten házának edényeit, az ételáldozatot és a tömjént.
\par 10 Megtudtam azt is, hogy a Léviták részeit nem adták meg; ennek miatta kiki az õ mezejére szélede el az Isten házában szolgáló Léviták és énekesek közül.
\par 11 És megfeddém a fejedelmeket, és mondék: Miért hagyatott el az Isten háza? És egybegyûjtvén a Lévitákat, helyökre állítám õket;
\par 12 Az egész Júda pedig meghozá az olajnak, a mustnak és a gabonának tizedét a tárházakba.
\par 13 És felügyelõkké rendelém a tárházak fölé Selemiát, a papot, Sádókot, az írástudót és Pedáját a Léviták közül, és melléjök Hanánt, ki Zakkur fia, ki Mattánia fia vala, mivelhogy híveknek ítéltettek volt és az õ tisztök vala kiosztani atyjokfiainak részét.
\par 14 Emlékezzél meg én rólam én Istenem ezért, és ne engedd, hogy eltöröltessenek az én jótéteményeim, melyeket cselekedtem vala az én Istenem házával és rendtartásaival!
\par 15 Azon napokban láttam Júdában, hogy sajtót taposnak szombaton és gabonát hoznak be, szamarakra rakván, sõt bort, szõlõt és olajat is és mindenféle terhet behoznak Jeruzsálembe szombat napon, és bizonyságot tevék ellenök, a mely napon eleséget árulnak vala.
\par 16 Tírusiak is lakozának a városban, a kik hoznak vala halat és mindenféle árút, melyeket eladnak vala szombat napon Júda fiainak Jeruzsálemben.
\par 17 Annakokáért megfeddém Júda elõljáróit, és mondám nékik: Micsoda gonosz dolog ez, a mit ti cselekesztek, hogy megfertõztetitek a szombatnak napját?
\par 18 Avagy nem így cselekedtek-é a ti atyáitok, s a mi Istenünk reánk hozá mindezen gonoszt és e városra?! És ti mégis növelitek Isten haragját Izráel fölött, megfertõztetvén a szombatot!
\par 19 Lõn annakokáért, hogy midõn megárnyékosodtak Jeruzsálem kapui a szombat elõtt, parancsolatomra bezáratának az ajtók, s megparancsolám, hogy meg ne nyissák azokat szombat utánig, annakfelette legényeim közül a kapukhoz rendelék, mondván: Nem fog bejõni teher a szombatnak napján!
\par 20 Annakokáért a kereskedõk és minden árúk árúsai kivül hálának Jeruzsálemen egyszer vagy kétszer;
\par 21 És bizonyságot tevék ellenök, és mondám nékik: Miért háltok ti e kõfal elõtt? Ha ezt ismételitek, kezet vetek reátok! Az idõtõl fogva nem jöttenek szombaton.
\par 22 És megparancsolám a Lévitáknak, hogy magokat megtisztítsák, s hogy menjenek el és õrizzék a kapukat, hogy megszenteljék a szombatot. Ezért is emlékezzél meg rólam én Istenem és kedvezz nékem, kegyelmességednek nagy volta szerint!
\par 23 Ugyanazon napokban meglátogatám azokat a zsidóket, kik asdódi, Ammonita és Moábita asszonyokat vettek feleségül.
\par 24 És fiaik felerésze asdódi nyelven beszél vala, és nem tudnak vala beszélni zsidóul, hanem egyik vagy másik nép nyelvén.
\par 25 Annakokáért feddõdém velök, és megátkozám õket, és megverék közülök néhányat, és megtépém õket, és megesketém õket Istenre: Bizony ne adjátok leányaitokat az õ fiaiknak, és ne vegyetek leányaik közül feleséget fiaitoknak és magatoknak.
\par 26 Avagy nem ebben vétkezett-é Salamon Izráel királya? Noha nem volt sok nép között hozzá hasonlatos király, a kit szeret vala az õ Istene és királylyá tette vala õt Isten egész Izráel fölött; õt is bûnre vivék az  idegen asszonyok:
\par 27 És néktek engedjünk-é, hogy cselekedjétek mindezen nagy gonoszságot, vétkezzetek Istenünk ellen, idegen asszonyokat vévén feleségül?
\par 28 Jójadának (ki Eliásib fõpapnak fia vala) fiai közül is egy veje vala a Horonitbeli Szanballatnak, elûzém azért õt tõlem.
\par 29 Emlékezzél meg õ rólok én Istenem, a papságnak és a papság szövetségének és a Lévitáknak ilyen megfertõztetéséért!
\par 30 És megtisztítám õket minden idegenektõl és rendtartást szabék a papoknak és a Lévitáknak, kinek-kinek az õ dolgában.
\par 31 A fa hozására is bizonyos idõkben és az elsõ zsengékre. Emlékezzél meg én Istenem az én javamra!


\end{document}