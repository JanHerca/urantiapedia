\begin{document}

\title{2 Corinthians}


\chapter{1}

\par 1 Pál, Jézus Krisztusnak apostola az Isten akaratjából, és Timótheus az atyafi, az Isten gyülekezetének, a mely Korinthusban van, mindama szentekkel egybe, a kik egész Akhájában vannak:
\par 2 Kegyelem néktek és békesség Istentõl a mi Atyánktól és az Úr Jézus Krisztustól.
\par 3 Áldott az Isten és a mi urunk Jézus Krisztusnak Atyja, az irgalmasságnak atyja és minden  vígasztalásnak Istene;
\par 4 A ki megvígasztal minket minden nyomorúságunkban, hogy mi is megvígasztalhassunk bármely nyomorúságba esteket azzal a vígasztalással, a mellyel Isten vígasztal minket.
\par 5 Mert a mint bõséggel kijutott nékünk a Krisztus szenvedéseibõl,  úgy bõséges a mi vígasztalásunk is Krisztus által.
\par 6 De akár nyomorgattatunk, a ti vígasztalástokért és üdvösségtekért van az, mely hathatós ugyanazon szenvedések elviselésére, a melyeket mi is szenvedünk; akár megvígasztaltatunk a ti vígasztalástokért és üdvösségtekért van az. És a mi reménységünk erõs felõletek.
\par 7 Tudván, hogy a miképen társaink vagytok a szenvedésben, azonképen a vígasztalásban is.
\par 8 Mert nem akarjuk, hogy ne tudjatok atyámfiai a mi nyomorúságunk felõl, a mely Ázsiában esett rajtunk, hogy felette igen, erõnk felett megterheltettünk, úgy hogy életünk felõl is kétségben valánk:
\par 9 Sõt magunk is halálra szántuk magunkat, hogy ne bizakodnánk mi magunkban, hanem Istenben, a ki feltámasztja a holtakat:
\par 10 A ki ilyen nagy halálból megszabadított és szabadít minket: a kiben reménykedünk, hogy ezután is meg fog szabadítani;
\par 11 Velünk együtt munkálkodván ti is az érettünk való könyörgésben, hogy a sokak által nékünk adatott kegyelmi ajándék sokak által háláltassék meg mi érettünk.
\par 12 Mert a mi dicsekedésünk ez, lelkiismeretünk bizonysága, hogy isteni õszinteséggel  és tisztasággal, nem testi bölcseséggel, hanem Isten kegyelmével forgolódtunk a világon, kiváltképen pedig ti köztetek.
\par 13 Mert nem egyebet írunk néktek, hanem a mit olvastok, vagy el is ismertek, sõt reménylem, hogy el is fogtok ismerni mindvégig;
\par 14 A minthogy némi részben el is ismertétek rólunk, hogy dicsekvéstek vagyunk, a miképen ti is nékünk az Úr Jézus napján.
\par 15 És ezzel a bizodalommal akartam elõbb hozzátok menni, hogy másodízben nyerjetek kegyelmet;
\par 16 És köztetek általmenni Macedóniába, és Macedóniából ismét hozzátok térni vissza, és tõletek elkísértetni Júdeába.
\par 17 Hát ezt akarva, vajjon könnyelmûen cselekedtem-é? vagy a mit akarok, test szerint akarom-é, hogy nálam az igen igen, és a nem nem legyen?
\par 18 De hû az Isten, hogy a mi beszédünk hozzátok nem volt igen és nem.
\par 19 Mert az Isten Fia Jézus Krisztus, a kit köztetek mi hirdettünk, én és Silvánus és Timótheus, nem volt igen és nem, hanem az igen lett õ benne.
\par 20 Mert Istennek valamennyi igérete õ benne lett igenné és õ benne lett Ámenné az Isten dicsõségére mi általunk.
\par 21 A ki pedig minket ti veletek egybe Krisztusban megerõsít és  megken minket, az Isten az;
\par 22 A ki el is pecsételt minket, és a léleknek zálogát adta a mi szíveinkbe.
\par 23 Én pedig az Istent hívom bizonyságul az én lelkemre, hogy titeket kímélve nem mentem el eddig Korinthusba.
\par 24 Nem hogy uralkodnánk a ti hiteteken, hanem munkatársai  vagyunk a ti örömeteknek; mert hitben állotok.

\chapter{2}

\par 1 Azt tettem pedig fel magamban, hogy nem megyek közétek ismét szomorúsággal.
\par 2 Mert ha én megszomorítlak titeket, ugyan ki az, a ki megvídámít engem, hanemha a kit én megszomorítok?
\par 3 És azért írtam néktek éppen azt, hogy mikor oda megyek, meg ne szomoríttassam azok miatt, a kiknek örülnöm kellene; meg lévén gyõzõdve mindenitek felõl, hogy az én örömöm mindnyájatoké.
\par 4 Mert sok szorongattatás és szívbeli háborgás között írtam néktek sok könyhullatással, nem hogy megszomoríttassatok, hanem hogy megismerjétek azt a szeretetet, a mellyel kiváltképen irántatok viseltetem.
\par 5 Ha pedig valaki megszomorított, nem engem szomorított meg, hanem részben, hogy azt ne terheljem, titeket mindnyájatokat.
\par 6 Elég az ilyennek a többség részérõl való ilyen megbüntetése:
\par 7 Annyira, hogy éppen ellenkezõleg ti inkább bocsássatok meg néki és vígasztaljátok, hogy valamiképen a felettébb való bánat meg ne eméssze az ilyet.
\par 8 Azért kérlek titeket, hogy tanusítsatok iránta szeretetet.
\par 9 Mert azért írtam is, hogy bizonyosan megtudjam felõletek, ha mindenben engedelmesek vagytok-é?
\par 10 A kinek pedig megbocsáttok valamit, én is: mert ha én is megbocsátottam valamit, ha valakinek megbocsátottam, ti érettetek cselekedtem Krisztus színe elõtt; hogy meg ne csaljon minket a Sátán:
\par 11 Mert jól ismerjük az õ szándékait.
\par 12 Mikor pedig Troásba mentem a Krisztus evangyélioma ügyében és kapu nyittatott nékem  az Úrban, nem volt lelkemnek nyugodalma, mivelhogy nem találtam Titust, az én atyámfiát;
\par 13 Hanem elbúcsúzván tõlük, elmentem Macedóniába.
\par 14 Hála pedig az Istennek, a ki mindenkor diadalra vezet minket a Krisztusban, és az õ ismeretének illatját minden helyen megjelenti mi általunk.
\par 15 Mert Krisztus jó illatja vagyunk Istennek, mind az üdvözülõk, mind az elkárhozók között;
\par 16 Ezeknek halál illatja halálra; amazoknak pedig élet illatja életre. És ezekre kicsoda alkalmatos?
\par 17 Mert mi nem olyanok vagyunk, mint sokan, a kik meghamisítják az Isten ígéjét; hanem tisztán, sõt szinte Istenbõl szólunk az Isten elõtt a Krisztusban.

\chapter{3}

\par 1 Ekezdjük-é ismét ajánlgatni magunkat? Vagy talán szükségünk van, mint némelyeknek, ajánló levelekre hozzátok, avagy tõletek?
\par 2 A mi levelünk ti vagytok, beírva a mi szívünkbe, a melyet ismer és olvas minden ember;
\par 3 A kik felõl nyilvánvaló, hogy Krisztusnak a mi szolgálatunk által szerzett levele vagytok, nem tentával, hanem az élõ Isten Lelkével írva; nem kõtáblákra, hanem a szívnek hústábláira.
\par 4 Ilyen bizodalmunk pedig Isten iránt a Krisztus által van.
\par 5 Nem mintha magunktól volnánk alkalmatosak valamit gondolni, úgy mint magunkból;  ellenkezõleg a mi alkalmatos voltunk az Istentõl van:
\par 6 A ki alkalmatosokká tett minket arra, hogy új szövetség szolgái legyünk, nem betûé, hanem léleké; mert a betû megöl,  a lélek pedig megelevenít.
\par 7 Ha pedig a halálnak betûkkel kövekbe vésett szolgálata dicsõséges vala, úgyhogy Izráel fiai nem is nézhettek Mózes  orczájára arczának elmúló dicsõsége miatt:
\par 8 Hogyne volna még inkább dicsõséges a léleknek szolgálata?
\par 9 Mert ha a kárhoztatás szolgálata dicsõséges, mennyivel inkább dicsõséges az igazság szolgálata?
\par 10 Sõt a dicsõített nem is dicsõséges ebben a részben az õt meghaladó dicsõség miatt.
\par 11 Mert ha dicsõséges az elmulandó, sokkal inkább dicsõséges, a mi megmarad.
\par 12 Azért ilyen reménységben nagy nyiltsággal szólunk;
\par 13 És nem, miként Mózes, a ki leplet borított az orczájára, hogy ne lássák Izráel fiai az elmulandónak végét.
\par 14 De megtompultak az õ elméik. Mert ugyanaz a lepel mind e mai napig ott van az ó szövetség olvasásánál felfedetlenül,  mivelhogy a Krisztusban tûnik el;
\par 15 Sõt mind máig, a mikor csak olvassák Mózest, lepel borul az õ szívökre.
\par 16 Mikor pedig megtér az Úrhoz, lehull a lepel.
\par 17 Az Úr pedig a Lélek; és a hol az Úrnak Lelke, ott a szabadság.
\par 18 Mi pedig az Úrnak dicsõségét mindnyájan fedetlen arczczal szemlélvén, ugyanazon ábrázatra elváltozunk, dicsõségrõl dicsõségre, úgy mint az Úrnak Lelkétõl.

\chapter{4}

\par 1 Annakokáért, mivelhogy ilyen szolgálatban vagyunk, a mint a  kegyelmet nyertük, nem csüggedünk el;
\par 2 Hanem lemondtunk a szégyen takargatásáról, mint a kik nem járunk ravaszságban, és nem is hamisítjuk meg  az Isten ígéjét, de a nyilvánvaló igazsággal kelletjük magunkat minden ember lelkiismeretének az Isten elõtt.
\par 3 Ha mégis leplezett a mi evangyéliomunk, azoknak leplezett, a kik elvesznek:
\par 4 A kikben e világ Istene megvakította a hitetlenek elméit, hogy ne lássák a Krisztus dicsõséges evangyéliomának világosságát, a ki az Isten képe.
\par 5 Mert nem magunkat prédikáljuk, hanem az Úr Jézus  Krisztust; magunkat pedig, mint a ti szolgáitokat, a Jézusért.
\par 6 Mert az Isten, a ki szólt: setétségbõl világosság ragyogjon, õ gyújtott világosságot a mi  szívünkben az Isten dicsõsége ismeretének a Jézus Krisztus arczán való világoltatása végett.
\par 7 Ez a kincsünk pedig cserépedényekben van, hogy amaz erõnek nagy volta Istené legyen, és nem magunktól  való.
\par 8 Mindenütt nyomorgattatunk, de meg nem szoríttatunk; kétségeskedünk, de nem esünk  kétségbe;
\par 9 Üldöztetünk, de el nem hagyatunk; tiportatunk, de el nem veszünk;
\par 10 Mindenkor testünkben hordozzuk az Úr Jézus halálát, hogy a Jézusnak élete is látható legyen a mi testünkben.
\par 11 Mert mi, a kik élünk, mindenkor halálra adatunk a Jézusért, hogy a Jézus élete is látható legyen a mi halandó testünkben.
\par 12 Azért a halál mi bennünk munkálkodik, az élet pedig ti bennetek.
\par 13 Mivelhogy pedig a hitnek mi bennünk is ugyanaz a lelke van meg, a mint írva van: Hittem és azért szóltam; hiszünk mi is, és azért szólunk;
\par 14 Tudván, hogy a ki feltámasztotta az Úr Jézust, Jézus által minket is feltámaszt, és veletek  együtt elõállít.
\par 15 Mert minden ti érettetek van, hogy a kegyelem sokasodva sokak által a hálaadást bõségessé tegye az Isten dicsõségére.
\par 16 Azért nem csüggedünk; sõt ha a mi külsõ emberünk megromol is, a belsõ mindazáltal  napról-napra újul.
\par 17 Mert a mi pillanatnyi könnyû szenvedésünk igen-igen nagy örök  dicsõséget szerez nékünk;
\par 18 Mivelhogy nem a láthatókra nézünk, hanem a láthatatlanokra; mert a láthatók ideig valók, a láthatatlanok pedig örökkévalók.

\chapter{5}

\par 1 Mert tudjuk, hogy ha e mi földi sátorházunk elbomol, épületünk van Istentõl, nem kézzel csinált, örökké  való házunk a mennyben.
\par 2 Azért is sóhajtozunk ebben, óhajtván felöltözni erre a mi mennyei hajlékunkat;
\par 3 Ha ugyan felöltözötten is mezíteleneknek nem találtatunk!
\par 4 Mert a kik e sátorban vagyunk is, sóhajtozunk megterheltetvén; mivelhogy nem kívánunk levetkõztetni, hanem felöltöztetni, hogy a mi halandó, elnyelje azt az élet.
\par 5 A ki pedig minket erre elkészített, az Isten az, a ki a Lélek zálogát is adta minékünk.
\par 6 Azért mivelhogy mindenkor bízunk, és tudjuk, hogy e testben lakván, távol vagyunk az Úrtól.
\par 7 (Mert hitben járunk, nem látásban );
\par 8 Bizodalmunk pedig van, azért inkább szeretnénk kiköltözni e testbõl, és elköltözni az Úrhoz.
\par 9 Azért igyekezünk is, hogy akár itt lakunk, akár elköltözünk, néki kedvesek legyünk.
\par 10 Mert nékünk mindnyájunknak meg kell jelennünk a Krisztus ítélõszéke elõtt, hogy kiki megjutalmaztassék  a szerint, a miket e testben cselekedett, vagy jót, vagy gonoszt.
\par 11 Ismervén tehát az Úrnak félelmét, embereket térítünk,  Isten elõtt pedig nyilván vagyunk; reménylem azonban, hogy a ti lelkiesméretetek elõtt is nyilván vagyunk.
\par 12 Mert nem ajánljuk ismét magunkat néktek, hanem alkalmat adunk ti néktek a velünk való  dicsekedésre, hogy legyen mit felelnetek a színbõl és nem szívbõl dicsekedõknek.
\par 13 Ha azért bolondok vagyunk, Istenért; ha eszesek vagyunk, érettetek van az.
\par 14 Mert a Krisztusnak szerelme szorongat minket,
\par 15 Úgy vélekedvén, hogy ha egy meghalt mindenkiért, tehát mindazok meghaltak; és azért  halt meg mindenkiért, hogy a kik élnek, ezután ne magoknak éljenek, hanem annak, a ki érettök meghalt és feltámasztatott.
\par 16 Azért mi ezentúl senkit sem ismerünk test szerint; sõt ha ismertük is Krisztust test szerint, de már többé nem ismerjük.
\par 17 Azért ha valaki Krisztusban van, új teremtés az; a régiek  elmúltak, ímé, újjá lett minden.
\par 18 Mindez pedig Istentõl van, a ki minket magával megbékéltetett a Jézus Krisztus által, és a ki nékünk adta a békéltetés szolgálatát;
\par 19 Minthogy az Isten volt az, a ki Krisztusban megbékéltette magával a világot, nem tulajdonítván nékik az õ bûneiket, és reánk bízta a békéltetésnek ígéjét.
\par 20 Krisztusért járván tehát követségben, mintha Isten kérne mi általunk: Krisztusért kérünk, béküljetek meg az Istennel.
\par 21 Mert azt, a ki bûnt nem ismert, bûnné tette  értünk, hogy mi Isten igazsága legyünk õ benne.

\chapter{6}

\par 1 Mint együttmunkálkodók intünk  is, hogy hiába ne vettétek légyen az Isten kegyelmét.
\par 2 Mert õ mondja: Kellemetes idõben meghallgattalak, és az üdvösség napján megsegítettelek. Ímé itt a kellemetes idõ, ímé itt az üdvösség napja.
\par 3 Senkit semmiben meg ne botránkoztassunk, hogy a szolgálatunk ne szidalmaztassék.
\par 4 Hanem ajánljuk magunkat mindenben, mint Isten szolgái;  sok tûrésben, nyomorúságban, szükségben, szorongattatásban.
\par 5 Vereségben, tömlöczben, háborúságban, küzködésben, virrasztásban, bõjtölésben.
\par 6 Tisztaságban, tudományban, hosszútûrésben, szívességben, Szent Lélekben, tettetés nélkül való szeretetben,
\par 7 Igazmondásban, Isten erejében; az  igazságnak jobb és bal felõl való fegyvereivel;
\par 8 Dicsõség és gyalázat által, rossz és jó hír által; mint hitetõk, és igazak;
\par 9 Mint ismeretlenek, és mégis ismeretesek; mint megholtak, és ím élõk; mint ostorozottak,  és meg nem ölöttek;
\par 10 Mint bánkódók, noha mindig örvendezõk; mint szegények, de sokakat gazdagítók; mint semmi nélkül valók, és mindennel  bírók.
\par 11 A mi szánk megnyílt ti néktek, korinthusiak, a mi szívünk kitárult.
\par 12 Nem mi bennünk vagytok szorosságban, hanem szorosságban vagytok a ti szívetekben.
\par 13 Viszonzásul (mint gyermekeimnek szólok) tárjátok ki ti is szíveteket.
\par 14 Ne legyetek hitetlenekkel felemás igában; mert szövetsége  van igazságnak és hamisságnak? vagy mi közössége a világosságnak a sötétséggel?
\par 15 És mi egyezsége Krisztusnak Béliállal? vagy mi köze hívõnek  hitetlenhez?
\par 16 Vagy mi egyezése Isten templomának bálványokkal? Mert ti az élõ Istennek temploma vagytok, az mint az Isten mondotta: Lakozom  bennök és közöttük járok; és leszek nékik Istenök, és õk én népem lesznek.
\par 17 Annakokáért menjetek ki közülök, és szakadjatok el, azt mondja az Úr, és tisztátalant ne illessetek; és én magamhoz fogadlak titeket,
\par 18 És leszek néktek Atyátok, és ti lesztek fiaimmá, és leányaimmá, azt mondja a mindenható Úr.

\chapter{7}

\par 1 Mivelhogy azért ilyen ígéreteink vannak, szeretteim, tisztítsuk meg magunkat minden testi és lelki tisztátalanságtól, Isten félelmében vivén véghez a  mi megszentelésünket.
\par 2 Fogadjatok be minket; senkit meg nem bántottunk, senkit meg nem rontottunk, senkit meg nem csaltunk.
\par 3 Nem vádképen mondom; hisz elõbb mondottam, hogy szívünkben vagytok, hogy együtt haljunk, együtt éljünk.
\par 4 Nagy az én bizodalmam hozzátok, nagy az én dicsekvésem felõletek; telve vagyok vígasztalódással, felettébb való az én örömem minden mi nyomorúságunk mellett.
\par 5 Mert mikor Macedóniába jöttünk, sem volt semmi nyugodalma a mi testünknek, sõt mindenképen nyomorogtunk; kívül harcz, belõl félelem.
\par 6 De az Isten, a megalázottak vígasztalója, minket is megvígasztalt Titus megjöttével.
\par 7 Sõt nem megjöttével csupán, hanem azzal a vígasztalással is, a melylyel ti vígasztaltátok meg, hírül hozván nékünk a ti kivánkozástokat, a ti kesergésteket, a ti hozzám való ragaszkodástokat; úgyhogy én mégjobban örvendeztem.
\par 8 Hát ha megszomorítottalak is titeket azzal a levéllel, nem bánom, noha bántam; mert látom, hogy az a levél, ha ideig-óráig is, megszomorított titeket.
\par 9 Most örülök, nem azért, hogy megszomorodtatok, hanem hogy megtérésre szomorodtatok meg. Mert Isten szerint szomorodtatok meg, hogy miattunk semmiben kárt ne valljatok.
\par 10 Mert az Isten szerint való szomorúság üdvösségre való megbánhatatlan megtérést szerez; a világ szerint való szomorúság pedig halált szerez.
\par 11 Mert ímé ez a ti Isten szerint való megszomorodástok milyen nagy buzgóságot keltett ti bennetek, sõt védekezést, sõt bosszankodást, sõt félelmet, sõt kívánkozást, sõt buzgóságot, sõt bosszúállást. Mindenképen bebizonyítottátok, hogy tiszták vagytok e dologban.
\par 12 Ha tehát írtam is néktek, nem a sértõ miatt, sem a sértett miatt; hanem hogy nyilvánvaló legyen nálatok a mi irántatok való buzgóságunk Isten elõtt.
\par 13 Annakokáért megvígasztalódtunk a ti vígasztalástokon; de sokkal inkább örültünk a Titus örömén, hogy az õ lelkét ti mindnyájan megnyugtattátok:
\par 14 Mert ha dicsekedtem valamit néki felõletek, nem maradtam szégyenben; de a mint ti néktek mindent igazán szólottam, azonképen a mi Titus elõtt való dicsekvésünk is igazsággá lett.
\par 15 És õ még jobb szívvel van irántatok, visszaemlékezvén mindnyájatoknak engedelmességre, hogy félelemmel és rettegéssel fogadtátok õt.
\par 16 Örülök, hogy mindenképen bízhatom bennetek.

\chapter{8}

\par 1 Tudtotokra adjuk pedig, atyámfiai, Istennek azt a kegyelmét, a melyet Macedónia gyülekezeteivel közlött.
\par 2 Hogy a nyomorúság sok próbái közt is bõséges az õ örömük és igen nagy szegénységük jószívûségük gazdagságává növekedett.
\par 3 Mert, bizonyság vagyok rá, erejük szerint, sõt erejök felett is adakoznak,
\par 4 Sok könyörgéssel kérvén minket, hogy a szentek iránt való szolgálat jótéteményébe és közösségébe fogadjuk be õket.
\par 5 És nem a miképen reméltük, hanem önmagukat adták elõször az Úrnak, és nekünk is az Isten akaratjából.
\par 6 Hogy kérnünk kellett Titust, hogy a miképen elkezdette, azonképen végezze is be nálatok ezt a jótéteményt is.
\par 7 Azért, miképen mindenben bõvölködtök, hitben, beszédben, ismeretben és minden buzgóságban és hozzánk való szeretetben, úgy e jótéteményben is bõvölködjetek.
\par 8 Nem parancsképen mondom, hanem hogy a mások buzgósága által a ti szeretetetek valódiságát is kipróbáljam.
\par 9 Mert ismeritek a mi Urunk Jézus Krisztusnak jótéteményét, hogy gazdag lévén, szegénnyé lett érettetek, hogy ti  az õ szegénysége által meggazdagodjatok.
\par 10 Tanácsot is adok e dologban; mert hasznos az néktek, a kik nemcsak a cselekvést, hanem az akarást is elkezdtétek tavaly óta.
\par 11 Most hát a cselekvés is vigyétek végbe; hogy a miképen az akarás készsége,  azonképen a végrehajtás és ahhoz képest legyen, a mitek van.
\par 12 Mert ha a készség megvan, a szerint kedves az, a mije kinek-kinek van, és nem a szerint, a mije nincs.
\par 13 Mert nem úgy, hogy másoknak könnyebbségük, néktek pedig nyomorúságtok legyen, hanem egyenlõség szerint; a mostani idõben a ti bõségtek pótolja amazoknak fogyatkozását;
\par 14 Hogy amazoknak bõsége is pótolhassa a ti fogyatkozástokat, hogy így egyenlõség legyen;
\par 15 A mint megvan írva: a ki sokat szedett, nem volt többje; és a ki keveset, nem volt kevesebbje.
\par 16 Hála pedig az Istennek, ki ugyanazt a buzgóságot oltotta  értetek a Titus szívébe.
\par 17 Mivelhogy intésünket ugyan elfogadta, de nagy buzgóságában önként ment hozzátok.
\par 18 Elküldöttük pedig vele együtt amaz atyafit is, a ki az összes gyülekezetekben dícséretes az evangyéliomért;
\par 19 Nemcsak pedig, hanem a gyülekezetek útitársunknak is megválaszták ebben a jó ügyben, a melyet mi szolgálunk magának az Úrnak dicsõségére és a ti készségetekre;
\par 20 Óvakodván, hogy senki se ócsárolhasson minket a mi szolgálatunk által való bõséges jótétemény miatt;
\par 21 Mert gondunk van a tisztességre nemcsak az Úr elõtt, hanem az emberek elõtt is.
\par 22 Sõt elküldöttük velök a mi atyánkfiát is, a kinek buzgó voltát sok dologban sokszor kipróbáltuk, most pedig még sokkal buzgóbb, irántatok való nagy bizodalmánál fogva.
\par 23 Akár Titusról van szó, õ az én társam és ti köztetek segítségem; akár a mi atyánkfiai felõl, õk a gyülekezetek követei,  Krisztus dicsõsége:
\par 24 Adjátok azért szereteteteknek és felõletek való dicsekvésünknek, bizonyságát irántuk a gyülekezetek elõtt is.

\chapter{9}

\par 1 A szentek iránt való szolgálatról felesleges is néktek írnom.
\par 2 Hiszen ismerem a ti készségteket a melylyel dicsekszem felõletek a macedónoknak, hogy Akhája kész a mult esztendõ óta; és a ti buzgóságtok, sokakat magával ragadt.
\par 3 Mindamellett elküldöttem az atyafiakat, hogy a mi felõletek való dicsekedésünk ebben a részben hiábavaló ne legyen; hogy, a mint mondám, készen legyetek.
\par 4 Hogy aztán, ha a macedónok velem együtt odajutnak és titeket készületlenül találnak, valamiképen szégyent ne valljunk mi, hogy ne mondjam ti, ebben a dologban.
\par 5 Szükségesnek véltem azért utasítani az atyafiakat, hogy elõre menjenek el hozzátok, és készítsék el elõre a ti elõre megígért adományotokat, hogy az úgy legyen készen, mint adomány, és nem mint ragadomány.
\par 6 Azt mondom pedig: A ki szûken vet, szûken is arat; és a ki bõven vet, bõven is arat.
\par 7 Kiki a mint eltökélte szívében, nem szomorúságból, vagy kénytelenségbõl; mert a jókedvû adakozót szereti az Isten.
\par 8 Az Isten pedig hatalmas arra, hogy rátok áraszsza minden kegyelmét; hogy mindenben, mindenkor teljes elégségtek lévén, minden jótéteményre bõségben legyetek,
\par 9 A mint meg van írva: Szórt, adott a szegényeknek; az õ igazsága örökké megmarad.
\par 10 A ki pedig magot ád a magvetõnek és kenyeret eleségül, ád és megsokasítja a ti vetésteket és megnöveli a ti igazságtoknak gyümölcsét,
\par 11 Hogy mindenben meggazdagodjatok a teljes jószívûségre, a mely általunk hálaadást szerez az Istennek.
\par 12 Mert e tisztnek szolgálata nemcsak a szenteknek szükségét elégíti ki, hanem sok hálaadással bõséges az Isten elõtt;
\par 13 A mennyiben e szolgálatnak próbája által dicsõítik az Istent a ti Krisztus evangyéliomát valló engedelmességtekért, és a ti hozzájuk és mindenekhez való adakozó jószívûségtekért.
\par 14 Mikor érettetek könyörögve õk is vágyakoznak utánatok az Istennek rajtatok való bõséges kegyelme miatt.
\par 15 Az Istennek pedig legyen hála az õ kimonhatatlan ajándékáért.

\chapter{10}

\par 1 Magam pedig, én Pál, kérlek titeket a Krisztus szelídségére és engedelmességére, a ki szemtõl szemben ugyan alázatos  vagyok közöttetek, de távol bátor vagyok irántatok;
\par 2 Kérlek pedig, hogy a mikor jelen leszek, ne kelljen bátornak lennem ama bizodalomnál fogva, a melylyel úgy gondolom bátor lehetek némelyekkel szemben, a kik úgy gondolkodnak felõlünk, mintha mi test szerint élnénk.
\par 3 Mert noha testben élünk, de nem test szerint vitézkedünk.
\par 4 Mert a mi vitézkedésünk fegyverei nem testiek, hanem erõsek az Istennek, erõsségek  lerontására;
\par 5 Lerontván okoskodásokat és minden magaslatot, a mely Isten ismerete ellen emeltetett, és foglyul ejtvén minden gondolatot, hogy engedelmeskedjék a Krisztusnak;
\par 6 És készen állván megbüntetni minden engedetlenséget, mihelyst teljessé lesz a ti engedelmességtek.
\par 7 A szem elõtt valókra néztek? Ha valaki azt hiszi magáról, hogy õ a Krisztusé, viszont azt is gondolja meg önmagában, hogy a mint õ maga a Krisztusé, azonképen mi is a Krisztuséi vagyunk.
\par 8 Mert ha még egy kissé felettébb dicsekedem is a mi hatalmunkkal, a  melyet az Úr a ti építéstekre és nem megrontásotokra adott, én nem vallok szégyent;
\par 9 Hogy ne láttassam, mintha csak ijesztgetnélek a leveleim által.
\par 10 Mert, úgy mondják, a levelei ugyan súlyosak és keménynek; de a maga jelenvolta erõtelen, és beszéde silány.
\par 11 Gondolja meg azt, a ki ilyen, hogy a milyenek vagyunk távol, a levelek által való beszédben, éppen olyanok leszünk, ha megjelenünk, cselekedetben is.
\par 12 Mert nem merjük magunkat azokhoz számítani, vagy hasonlítani, a kik magukat ajánlják; de azok magukat magukhoz mérvén és magukhoz hasonlítván magukat, nem okosan cselekesznek.
\par 13 De mi nem dicsekszünk mértéktelenül, hanem ama mérõzsinór mértéke szerint, a melyet Isten adott nékünk mértékül, hogy hozzátok is elérjünk.
\par 14 Mert nem feszítjük túl magunkat, mintha nem értünk volna el hozzátok; hiszen hozzátok is eljutottunk a Krisztus evangyéliomával.
\par 15 A kik nem dicsekeszünk mértéktelenül mások munkájával, de reméljük, hogy hitetek megnõttével nagyokká leszünk köztetek a mi mérõzsinórunk szerint bõségesen.
\par 16 Hogy rajtatok túl is hirdessük az evangyéliomot, nem dicsekedvén más mértéke szerint a készszel.
\par 17 A ki pedig dicsekszik, az Úrban dicsekedjék.
\par 18 Mert nem az a kipróbált, a ki magát ajánlja, hanem a kit az  Úr ajánl.

\chapter{11}

\par 1 Vajha elszenvednétek tõlem egy kevés balgatagságot! Sõt szenvedjetek el engem is.
\par 2 Mert isteni buzgósággal buzgok értetek; hisz eljegyeztelek titeket egy férfiúnak, hogy mint szeplõtlen szûzet állítsalak a Krisztus elé.
\par 3 Félek azonban, hogy a miként a kígyó a maga álnokságával megcsalta Évát, akként a ti gondolataitok is megrontatnak és eltávolodnak a Krisztus iránt való egyenességtõl.
\par 4 Mert hogyha az, a ki jõ, más Jézust prédikál, a kit nem prédikáltunk, vagy más lelket vesztek, a mit nem vettetek, vagy más evangyéliomot, a mit be nem fogadtatok, szépen eltûrnétek.
\par 5 Mert én azt gondolom, hogy semmiben sem vagyok alábbvaló a fõ-fõ apostoloknál.
\par 6 Ha pedig avatatlan vagyok is a beszédben, de nem az ismeretben; sõt mindenben, mindenképen nyilvánvalókká lettünk elõttetek.
\par 7 Avagy vétkeztem-é, mikor magamat megaláztam, hogy ti felmagasztaltassatok, hogy ingyen hírdettem néktek az Isten evangyéliomát?
\par 8 Más gyülekezeteket fosztottam meg, zsoldot vévén, hogy néktek szolgáljak; és mikor nálatok voltam és szûkölködtem, nem voltam terhére senkinek.
\par 9 Mert az én szükségemet kipótolták a Macedóniából jött atyafiak; és rajta voltam és rajta is leszek, hogy semmiben se legyek terhetekre.
\par 10 Krisztus igazsága bennem, hogy én ettõl a dicsekvéstõl nem esem el Akhája vidékén.
\par 11 Miért? Hogy nem szeretlek titeket? Tudja az Isten.
\par 12 De a mit cselekszem, cselekedni is fogom, hogy elvágjam az alkalmat az alkalomkeresõk elõl; hogy a mivel dicsekesznek, olyanoknak találtassanak abban, mint mi is.
\par 13 Mert az ilyenek hamis apostolok, álnok munkások, a kik a Krisztus apostolaivá változtatják át magukat.
\par 14 Nem is csoda; hisz maga a Sátán is átváltoztatja magát világosság angyalává.
\par 15 Nem nagy dolog azért, ha az õ szolgái is átváltoztatják magokat az igazság szolgáivá; a kiknek végök az õ cselekedeteik szerint lészen.
\par 16 Ismét mondom: ne tartson engem senki esztelennek; de ha mégis, fogadjatok be mint esztelent is, hogy egy kicsit én is dicsekedhessem.
\par 17 A mit mondok, nem az Úr szerint mondom, hanem mintegy esztelenül a dicsekvésnek ezzel a merészségével,
\par 18 Mivelhogy sokan dicsekesznek test szerint, dicsekeszem  én is.
\par 19 Hisz okosak lévén, örömest eltûritek az eszteleneket.
\par 20 Mert eltûritek, ha valaki leigáz titeket, ha valaki felfal, ha valaki megfog, ha valaki felfuvalkodik, ha valaki arczul ver titeket.
\par 21 Szégyenkezve mondom, mivelhogy mi erõtlenek voltunk; de a miben merész valaki, esztelenül szólok, merész vagyok én is.
\par 22 Héberek õk? Én is. Izráeliták-é? Én is. Ábrahám magva-é? Én is.
\par 23 Krisztus szolgái-é? (balgatagul szólok) én méginkább; több fáradság, több vereség,  több börtön, gyakorta való halálos veszedelem által.
\par 24 A zsidóktól ötször kaptam negyvenet egy híján.
\par 25 Háromszor megostoroztak,  egyszer megköveztek, háromszor hajótörést szenvedtem, éjt-napot a mélységben töltöttem;
\par 26 Gyakorta való utazásban, veszedelemben folyó vizeken, veszedelemben rablók közt, veszedelemben népem között, veszedelemben pogányok között, veszedelemben városban, veszedelemben  pusztában, veszedelemben tengeren, veszedelemben hamis atyafiak közt;
\par 27 Fáradságban és nyomorúságban, gyakorta való virrasztásban, éhségben és szomjúságban, gyakorta való bõjtölésben, hidegben és mezítelenségben.
\par 28 Mindezeken kívül van az én naponkénti zaklattatásom, az összes gyülekezetek gondja.
\par 29 Ki beteg, hogy én is beteg ne volnék? Ki botránkozik meg, hogy én is ne égnék?
\par 30 Ha dicsekednem kell, az én gyengeségemmel dicsekszem.
\par 31 Az Isten és a mi Urunk Jézus Krisztusnak Atyja, a ki mindörökké áldott, tudja, hogy nem hazudom.
\par 32 Damaskusban Aretás király helytartója õrzette a damaskusiak városát, akarván engem megfogni;
\par 33 És az ablakon át, kosárban bocsátottak le a kõfalon, és megmenekültem kezei közül.

\chapter{12}

\par 1 A dicsekvés azonban nem használ nékem; rátérek azért a látomásokra és az Úrnak kijelentéseire.
\par 2 Ismerek egy embert a Krisztusban, a ki tizennégy évvel ezelõtt (ha testben-é, nem tudom; ha testen kívül-é, nem tudom; az Isten tudja) elragadtatott a harmadik égig.
\par 3 És tudom, hogy az az ember, (ha testben-é, ha testen kívül-é, nem tudom; az Isten tudja),
\par 4 Elragadtatott a paradicsomba, és hallott kimondhatatlan beszédeket, a melyeket nem szabad embernek kibeszélnie.
\par 5 Az ilyennel dicsekeszem; magammal pedig nem dicsekeszem, ha csak az én gyengeségeimmel nem.
\par 6 Mert ha dicsekedni akarok, nem leszek esztelen; mert igazságot mondok; de megtürtõztetem magamat, hogy valaki többnek ne tartson, mint a minek lát, vagy a mit hall tõlem.
\par 7 És hogy a kijelentések nagysága miatt el ne bizakodjam, tövis adatott nékem a testembe, a Sátán angyala, hogy gyötörjön engem, hogy felettébb el ne bizakodjam.
\par 8 Ezért háromszor könyörögtem az Úrnak, hogy távozzék el ez tõlem;
\par 9 És ezt mondá nékem: Elég néked az én kegyelmem; mert az én erõm erõtlenség által végeztetik el. Nagy örömest dicsekeszem azért az én erõtelenségeimmel, hogy a Krisztus ereje lakozzék én bennem.
\par 10 Annakokáért gyönyörködöm az erõtlenségekben, bántalmazásokban, nyomorúságokban, üldözésekben és szorongattatásokban Krisztusért; mert a mikor erõtelen vagyok, akkor vagyok  erõs.
\par 11 Dicsekedvén, balgataggá lettem; ti kényszerítettetek reá. Mert néktek kellett volna engem ajánlanotok; mert semmiben sem vagyok alábbvaló a fõ-fõ apostoloknál,  noha semmi vagyok.
\par 12 Apostolságomnak jelei megbizonyosodtak közöttetek sok tûrésben, jelekben, csodákban és erõkben is.
\par 13 Mert micsoda az, a miben megkárosodtatok a többi gyülekezetek mellett, hanem ha az, hogy én magam nem voltam néktek terhetekre? Bocsássátok meg nékem ezt az igazságtalanságot!
\par 14 Ímé harmadízben is kész vagyok hozzátok menni, és nem leszek terhetekre; mert nem azt keresem, a mi a tiétek,  hanem titeket magatokat. Mert nem a gyermekek tartoznak kincseket gyûjteni a szülõknek, hanem a szülõk a gyermekeknek.
\par 15 Én pedig nagy örömest áldozok és esem áldozatul a ti lelketekért; még ha ti, a kiket én igen szeretek, kevésbbé  szerettek is engem.
\par 16 De ám legyen, hogy én nem voltam terhetekre; hanem álnok lévén ravaszsággal fogtalak meg titeket.
\par 17 Avagy a kiket hozzátok küldtem, azok közül valamelyik által kifosztottalak-é titeket?
\par 18 Megkértem Titust, és vele együtt elküldtem amaz atyafit; csak nem fosztott ki titeket Titus? Nem egyazon Lélek szerint jártunk-é? Nem azokon a nyomokon-é?
\par 19 Azt hiszitek megint, hogy elõttetek mentegetjük magunkat. Az Isten elõtt Krisztusban szólunk; mindezt pedig, szeretteim, a ti épüléstekért.
\par 20 Mert félek azon, hogy ha odamegyek, nem talállak majd olyanoknak, a milyeneknek szeretnélek, és engem is olyannak találtok, a milyennek nem szeretnétek; hogy valamiképen versengések, irígységek, indulatoskodások, visszavonások, rágalmazások, fondorkodások,  felfuvalkodások, pártoskodások lesznek köztetek;
\par 21 Hogy mikor újra odamegyek, megaláz engem az én Istenem ti köztetek, és sokakat megsiratok azok közül, a kik ezelõtt vétkeztek és meg nem tértek a tisztátalanságból, paráznaságból és bujaságból, a mit elkövettek.

\chapter{13}

\par 1 Ezúttal harmadszor megyek hozzátok. Két vagy három tanú vallomása alapján megáll minden dolog.
\par 2 Elõre megmondtam, és elõre mondom, mint másodszori ottlétemkor, és most is távollétemben írom azoknak, a kik ezelõtt vétkeztek, és a többieknek mind, hogy ha ismét odamegyek, nem leszek kíméletes;
\par 3 Mert hát az általam szóló Krisztusnak bizonyságát keresitek, a ki irányotokban nem erõtelen, hanem erõs ti bennetek.
\par 4 Mert noha megfeszíttetett erõtelenségbõl, mindazáltal  él Istennek hatalmából. És noha mi erõtelenek vagyunk benne, de vele együtt élünk majd Isten erejébõl ti nálatok.
\par 5 Kísértsétek meg magatokat, ha a hitben vagytok-é? magatokat próbáljátok meg. Avagy nem ismeritek-é magatokat, hogy a Jézus  Krisztus bennetek van? Kivévén, ha méltatlanok vagytok.
\par 6 De reménylem, hogy megismeritek, hogy mi nem vagyunk méltatlanok.
\par 7 Az Istent pedig kérem, hogy semmi gonoszt ne cselekedjetek; nem hogy mi méltóknak láttassunk, hanem hogy ti a jót cselekedjétek, mi pedig mintegy méltatlanok legyünk.
\par 8 Mert semmit sem cselekedhetünk az igazság ellen, hanem csak az igazságért.
\par 9 Mert örvendünk, ha mi erõtelenek vagyunk, ti meg erõsek vagytok; ezt pedig kérjük is, a ti tökéletesedésetekért.
\par 10 Azért írom ezeket távollétemben, hogy jelenlétemben ne kelljen keményen viselkednem ama hatalom szerint,  a melyet az Úr adott nékem építésre és nem rontásra.
\par 11 Végezetre, atyámfiai, legyetek jó egészségben, épüljetek, vígasztalódjatok, egy értelemben legyetek, békességben  éljetek; és szeretetnek és békességnek Istene lészen veletek.
\par 12 Köszöntsétek egymást szent csókkal. Köszöntenek titeket a szentek mindnyájan.
\par 13 Az Úr Jézus Krisztusnak kegyelme, és az Istennek szeretete, és a Szent Léleknek közössége mindnyájatokkal. Ámen.


\end{document}