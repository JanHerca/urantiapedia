\begin{document}

\title{1 Corinthians}


\chapter{1}

\par 1 Pál, Jézus Krisztusnak Isten akaratából elhívott apostola, és Sosthenes,  az atyafi.
\par 2 Az Isten gyülekezetének, a mely Korinthusban van, a Krisztus Jézusban megszentelteknek, elhívott szenteknek, mindazokkal egybe, a kik a mi Urunk Jézus Kriszus nevét segítségül hívják bármely helyen, a magokén és a miénken:
\par 3 Kegyelem néktek és békesség Istentõl, a mi Atyánktól, és az Úr Jézus Krisztustól.
\par 4 Hálát adok az én Istenemnek mindenkor ti felõletek az Isten ama kegyelméért, mely néktek a Krisztus Jézusban adatott,
\par 5 Mivelhogy mindenben meggazdagodtatok õ benne, minden beszédben és minden  ismeretben,
\par 6 A mint megerõsíttetett ti bennetek a Krisztus felõl való bizonyságtétel.
\par 7 Úgy, hogy semmi kegyelmi ajándék nélkül nem szûkölködtök, várván a mi Urunk Jézus  Krisztusnak megjelenését,
\par 8 A ki meg is erõsít titeket mindvégig  feddhetetlenségben, a mi Urunk Jézus Krisztusnak napján.
\par 9 Hû az Isten, ki elhívott  titeket az õ Fiával, a mi Urunk Jézus Krisztussal való közösségre,
\par 10 Kérlek azonban titeket atyámfiai, a mi Urunk Jézus Krisztus nevére, hogy mindnyájan egyképen szóljatok és ne legyenek köztetek szakadások, de legyetek teljesen egyek ugyanazon értelemben  és ugyanazon véleményben.
\par 11 Mert megtudtam felõletek atyámfiai a Kloé embereitõl, hogy versengések vannak köztetek.
\par 12 Azt értem pedig, hogy mindenitek ezt mondja: Én Pálé vagyok,  én meg Apollósé, én meg Kéfásé, én meg Krisztusé.
\par 13 Vajjon részekre osztatott-é a Krisztus? Vajjon Pál feszíttetett-é meg érettetek, vagy a Pál nevére kereszteltettetek-é meg?
\par 14 Hálákát adok az Istennek, hogy senkit sem kereszteltem meg közületek, kivéve Krispust  és Gájust,
\par 15 Hogy valaki azt ne mondja, hogy a magam nevére kereszteltem.
\par 16 Megkereszteltem azonban Stefana házanépét is, ezenkívül nem tudom, hogy valakit mást kereszteltem volna.
\par 17 Mert nem azért küldött engem a Krisztus, hogy kereszteljek, hanem hogy az evangyéliomot hirdessem; de nem szólásban való bölcseséggel, hogy a Krisztus keresztje hiábavaló ne legyen.
\par 18 Mert a keresztrõl való beszéd bolondság ugyan azoknak, a kik elvesznek; de nekünk,  kik megtartatunk, Istennek ereje.
\par 19 Mert meg van írva: Elvesztem a bölcseknek bölcseségét és az értelmeseknek értelmét elvetem.
\par 20 Hol a bölcs?  hol az írástudó? hol e világnak vitázója? Nemde nem bolondsággá tette-é Isten e világnak bölcseségét?
\par 21 Mert minekutána az Isten bölcseségében nem ismerte meg a világ a bölcseség által az Istent, tetszék az Istennek, hogy az igehirdetés bolondsága által tartsa meg a hívõket.
\par 22 Mert egyfelõl a zsidók jelt kívánnak, másfelõl a görögök bölcseséget keresnek.
\par 23 Mi pedig Krisztust prédikáljuk, mint megfeszítettet, a zsidóknak ugyan botránkozást,  a görögöknek pedig bolondságot;
\par 24 Ámde magoknak a hivatalosoknak, úgy zsidóknak, mint görögöknek Krisztust, Istennek hatalmát és Istennek  bölcseségét.
\par 25 Mert az Isten bolondsága bölcsebb az embereknél, és az Isten erõtelensége erõsebb az embereknél.
\par 26 Mert tekintsétek csak a ti hivatástokat, atyámfiai, hogy nem sokan hívattak bölcsek test szerint, nem sokan hatalmasak, nem sokan nemesek;
\par 27 Hanem a világ bolondjait választotta ki magának az Isten, hogy megszégyenítse a bölcseket; és a világ erõtleneit választotta ki magának az Isten, hogy megszégyenítse az erõseket;
\par 28 És a világ nemteleneit és megvetettjeit választotta ki magának az Isten, és semmiket, hogy a valamiket megsemmisítse:
\par 29 Hogy ne dicsekedjék õ elõtte egy test sem.
\par 30 Tõle vagytok pedig ti a Krisztus Jézusban, ki bölcseségül lõn nékünk Istentõl, és igazságul, szentségül  és váltságul:
\par 31 Hogy, a mint meg van írva: A ki dicsekedik, az Úrban dicsekedjék.

\chapter{2}

\par 1 Én is, mikor hozzátok mentem, atyámfiai, nem mentem, hogy nagy ékesszólással, avagy bölcseséggel hirdessem néktek az Isten bizonyságtételét.
\par 2 Mert nem végeztem, hogy egyébrõl tudjak ti köztetek, mint a Jézus Krisztusról, még pedig  mint megfeszítettrõl.
\par 3 És én erõtlenség, félelem és nagy rettegés közt jelentem meg  ti köztetek.
\par 4 És az én beszédem és az én prédikálásom nem emberi bölcseségnek hitetõ beszédiben állott, hanem léleknek és erõnek megmutatásában:
\par 5 Hogy a ti hitetek ne emberek bölcseségén, hanem Istennek erején nyugodjék.
\par 6 Bölcseséget pedig a tökéletesek között szólunk; ámde nem e világnak, sem e világ veszendõ fejedelmeinek bölcseségét;
\par 7 Hanem Istennek titkon való bölcseségét szóljuk, azt az elrejtetett, melyet öröktõl fogva elrendelt az Isten a mi dicsõségünkre;
\par 8 Melyet e világ fejedelmei közül senki sem ismert, mert ha megismerték volna, nem feszítették volna meg a dicsõség Urát.
\par 9 Hanem, a mint meg van írva: A miket szem nem látott, fül nem hallott és embernek szíve meg se gondolt, a miket Isten készített az õt szeretõknek.
\par 10 Nekünk azonban az Isten kijelentette az õ  Lelke által: mert a Lélek mindeneket vizsgál, még az Istennek mélységeit is.
\par 11 Mert kicsoda tudja az emberek közül az emberek dolgait, hanemha az embernek lelke, a mely õ benne van? Azonképen az Isten dolgait sem ismeri senki, hanemha az Istennek Lelke.
\par 12 Mi pedig nem e világnak lelkét vettük, hanem az Istenbõl való Lelket; hogy megismerjük azokat, a miket Isten ajándékozott nékünk.
\par 13 Ezeket prédikáljuk is, nem oly beszédekkel, melyekre emberi bölcseség tanít, hanem a melyekre a Szent Lélek tanít; lelkiekhez lelkieket szabván.
\par 14 Érzéki ember pedig nem foghatja meg az Isten Lelkének dolgait: mert  bolondságok néki; meg sem értheti, mivelhogy lelkiképen ítéltetnek meg.
\par 15 A lelki ember azonban mindent megítél, de õ senkitõl sem ítéltetik meg.
\par 16 Mert ki érte fel az Úrnak értelmét, hogy megoktathatná õt? Bennünk pedig Krisztus értelme van.

\chapter{3}

\par 1 Én sem szólhattam néktek, atyámfiai, mint lelkieknek, hanem mint testieknek, mint a Krisztusban kisdedeknek.
\par 2 Téjnek italával tápláltalak titeket és nem kemény eledellel, mert még nem bírtátok volna meg, sõt még most sem bírjátok meg:
\par 3 Mert még testiek vagytok; mert a mikor írigykedés, versengés és visszavonás van köztetek, vajjon nem testiek vagytok-é és nem ember szerint jártok-é?
\par 4 Mert mikor egyik ezt mondja: Én Pálé vagyok; a másik meg: Én Apollósé; nem testiek vagytok-é?
\par 5 Hát kicsoda Pál és kicsoda Apollós? Csak szolgák, kik által hívõkké lettetek, és pedig a mint kinek-kinek az Úr adta.
\par 6 Én plántáltam, Apollós öntözött; de  az Isten adja vala a növekedést.
\par 7 Azért sem a ki plántál, nem valami, sem a ki öntöz; hanem a növekedést adó Isten.
\par 8 A plántáló pedig és az öntözõ egyek; de mindenik a maga jutalmát veszi a maga munkája szerint.
\par 9 Mert Isten munkatársai vagyunk: Isten szántóföldje, Isten  épülete vagytok.
\par 10 Az Istennek nékem adott kegyelme szerint, mint bölcs építõmester, fundamentomot  vetettem, de más épít reá. Kiki azonban meglássa mimódon épít reá.
\par 11 Mert más fundamentomot senki nem vethet azon kívül, a mely vettetett, mely a Jézus Krisztus.
\par 12 Ha pedig valaki aranyat, ezüstöt, drágaköveket, fát, szénát, pozdorját épít rá erre a fundamentomra;
\par 13 Kinek-kinek munkája nyilván lészen: mert ama nap megmutatja, mivelhogy tûzben jelenik meg; és hogy kinek-kinek munkája minémû legyen, azt a tûz próbálja meg.
\par 14 Ha valakinek a munkája, a melyet ráépített, megmarad, jutalmát veszi.
\par 15 Ha valakinek a munkája megég, kárt vall. Õ maga azonban megmenekül, de úgy, mintha tûzön keresztül.
\par 16 Nem tudjátok-é, hogy ti Isten temploma vagytok, és az Isten Lelke  lakozik bennetek?
\par 17 Ha valaki az Isten templomát megrontja, megrontja azt az Isten. Mert az Istennek temploma szent, ezek vagytok ti.
\par 18 Senki se csalja meg magát. Ha valaki azt hiszi, hogy bölcs ti köztetek e világon, bolond legyen, hogy bölcscsé lehessen.
\par 19 Mert e világ bölcsesége bolondság az Isten elõtt. Mert meg van írva: Megfogja a bölcseket az õ csalárdságukban.
\par 20 És ismét: Ismeri az Úr a bölcsek gondolatait, hogy hiábavalók.
\par 21 Azért senki se dicsekedjék emberekkel. Mert minden a tiétek.
\par 22 Akár Pál, akár Apollós, akár Kéfás, akár világ, akár élet, akár halál, akár jelenvalók, akár következendõk, minden a tiétek.
\par 23 Ti pedig Krisztusé, Krisztus pedig Istené.

\chapter{4}

\par 1 Úgy tekintsen minket az ember, mint Krisztus szolgáit és Isten titkainak  sáfárait.
\par 2 A mi pedig egyébiránt a sáfárokban megkívántatik, az, hogy mindenik hívnek találtassék.
\par 3 Rám nézve pedig igen csekély dolog, hogy ti tõletek ítéltessem meg, vagy emberi ítéletnaptól; sõt magam sem ítélem meg magamat.
\par 4 Mert semmit sem tudok magamra, de nem ebben vagyok megigazulva; a ki ugyanis engem megítél, az Úr  az.
\par 5 Azért idõ elõtt semmit se ítéljetek, míg el nem jõ az Úr, a ki egyrészt világra hozza  a sötétségnek titkait, másrészt megjelenti a szíveknek tanácsait; és akkor mindenkinek az Istentõl lészen a dícsérete.
\par 6 Ezeket pedig, atyámfiai, példában szabtam magamra és Apollósra ti érettetek, hogy rajtunk tanuljátok meg, hogy annakfelette a mi írva van, nem kell bölcselkedni; hogy senki se fuvalkodjék fel az egyikért a másik ellen.
\par 7 Mert kicsoda különböztet meg téged? Mid van ugyanis, a mit nem kaptál volna? Ha pedig úgy kaptad, mit dicsekedel, mintha nem kaptad volna?
\par 8 Immár beteltetek, immár meggazdagodtatok, nálunk nélkül uralkodásra jutottatok; vajha csakugyan uralkodásra jutottatok volna, hogy mi is veletek egybe uralkodhatnánk.
\par 9 Mert úgy vélem, hogy az Isten minket, az apostolokat, utolsókul állított, mintegy halálra szántakul: mert  látványossága lettünk a világnak, úgy angyaloknak, mint embereknek.
\par 10 Mi bolondok a Krisztusért, ti pedig bölcsek a Krisztusban; mi erõtlenek, ti pedig erõsek; ti dicsõségesek, mi pedig gyalázatosak.
\par 11 Mindezideig éhezünk is, szomjúhozunk is, mezítelenkedünk is, bántalmaztatunk is,  bujdosunk is,
\par 12 Fáradozunk is, tulajdon kezünkkel munkálkodván; ha szidalommal illettetünk,  jót kívánunk; ha háborúságot szenvedünk, békességgel tûrjük;
\par 13 Ha gyaláztatunk, könyörgünk: szinte a világ szemetjévé lettünk, mindeneknek söpredékévé egész mostanig.
\par 14 Nem azért írom ezeket, hogy megszégyenítselek titeket, hanem mint szerelmes gyermekeimet intelek.
\par 15 Mert ha tízezer tanítómesteretek lenne is a Krisztusban, de nem sok atyátok; mert tõlem vagytok a Krisztus Jézusban az evangyéliom által.
\par 16 Kérlek azért titeket, legyetek az én követõim.
\par 17 Azért küldtem hozzátok Timótheust, ki nékem szeretett és hû fiam az Úrban, a ki eszetekbe juttatja néktek az én útaimat a Krisztusban, a mint mindenütt, minden gyülekezetben tanítok.
\par 18 De mintha el se mennék ti hozzátok, úgy felfuvalkodtak némelyek.
\par 19 Pedig elmegyek hamarosan hozzátok, ha az úr akarándja; és megismerem a felfuvalkodottaknak nem a beszédjét, hanem az erejét.
\par 20 Mert nem beszédben áll az Istennek országa, hanem erõben.
\par 21 Mit akartok? Vesszõvel menjek-é hozzátok, avagy szeretettel és szelídségnek lelkével?

\chapter{5}

\par 1 Általában hallatszik köztetek paráznaság, még olyan paráznaság is, a milyen a pogányok között sem említtetik, hogy valaki atyjának feleségét elvegye.
\par 2 És ti fel vagytok fuvalkodva, és nem keseredtetek meg inkább, hogy kivettetnék közületek, a ki ezt a dolgot cselekedte.
\par 3 Mert én távol lévén ugyan testben, de jelen lévén lélekben, már elvégeztem, mintha jelen volnék, hogy azt, a ki ekként ezt cselekedte,
\par 4 Ti és az én lelkem a mi Urunk Jézus Krisztusnak nevében egybegyûlvén, a mi Urunk Jézus Krisztus hatalmával
\par 5 Átadjuk az ilyent a Sátánnak a testnek veszedelmére, hogy a lélek megtartassék az Úr Jézusnak ama napján.
\par 6 Nem jó a ti dicsekedéstek. Avagy nem tudjátok-é, hogy egy kicsiny kovász az egész tésztát megposhasztja.
\par 7 Tisztítsátok el azért a régi kovászt, hogy legyetek új tésztává, a minthogy kovász nélkül valók vagytok; mert hiszen a mi húsvéti bárányunk, a Krisztus, megáldoztatott érettünk.
\par 8 Azért ne régi kovászszal ünnepeljünk, sem rosszaságnak és gonoszságnak kovászával, hanem tisztaságnak és igazságnak kovásztalanságában.
\par 9 Azt írtam néktek ama levelemben, hogy paráznákkal ne társalkodjatok.
\par 10 De nem általában e világ paráznáival, vagy csalóival, vagy ragadozóival, vagy bálványimádóival; mert hiszen így ki kellene e világból mennetek.
\par 11 Most azért azt írom néktek, hogy ne társalkodjatok azzal, ha valaki atyafi létére parázna, vagy csaló, vagy bálványimádó, vagy szidalmazó, vagy részeges, vagy ragadozó. Az ilyennel még együtt se egyetek.
\par 12 Mert mi közöm ahhoz, hogy a kívülvalókról is ítéletet tegyek? avagy ti nem a belüllévõk fölött tesztek-é ítéletet?
\par 13 A kívülvalókat pedig majd az Isten ítéli meg. Vessétek ki azért a gonoszt magatok közül.

\chapter{6}

\par 1 Merészel valaki közületek, ha peres dolga van a másikkal, az igaztalanok elõtt törvénykezni, és nem a szentek elõtt?
\par 2 Nem tudjátok-é, hogy a szentek a világot ítélik meg? És ha ti ítélitek meg a világot, méltatlanok vagytok-é a legkisebb dolgokban való ítéletekre?
\par 3 Nem tudjátok-é, hogy angyalokat fogunk ítélni, nemhogy életszükségre való dolgokat?
\par 4 Azért ha életszükségre való dolgok felõl van törvénykezéstek, a kik a gyülekezetben legalábbvalók, azokat ültessétek le.
\par 5 Megszégyenítéstekre mondom: Hát nincs ti köztetek egy bölcs ember sem, a ki ítéletet tehetne az õ atyjafiai között?
\par 6 Hanem atyafi atyafival törvénykezik, még pedig hitetlenek elõtt?
\par 7 Egyáltalán már az is gyarlóság ti bennetek, hogy törvénykeztek egymással. Miért  nem szenveditek inkább a bántalmazást? Miért nem tûritek inkább a kárt?
\par 8 Sõt ti okoztok bántalmazást és kárt, még pedig atyátokfiainak.
\par 9 Avagy nem tudjátok-é, hogy igazságtalanok nem örökölhetik Istennek országát? Ne tévelyegjetek; se paráznák, se bálványimádók, se házasságtörõk, se pulyák, se férfiszeplõsítõk,
\par 10 Se lopók, se telhetetlenek, se részegesek, se szidalmazók, se ragadozók nem örökölhetik az Isten országát.
\par 11 Ilyenek voltatok pedig némelyek, de megmosattattatok  de megszenteltettetek, de megigazíttattatok az Úr Jézusnak nevében és a mi Istenünk Lelke által.
\par 12 Minden szabad nékem, de nem minden használ; minden szabad nékem, de én nem adatom valakinek hatalma alá.
\par 13 Az eledelek a hasnak és a has az eledeleknek rendeltetett. Az Isten pedig mind ezt, mind amazokat eltörli. A test azonban nem a paráznaságnak  rendeltetett, hanem az Úrnak, és az Úr a testnek.
\par 14 Az Isten pedig az Urat is feltámasztotta, minket is feltámaszt az õ hatalma által.
\par 15 Nem tudjátok-é, hogy a ti testeitek a Krisztusnak tagjai? Elszakítva hát a Krisztus tagjait, paráznának tagjaivá tegyem? Távol legyen.
\par 16 Avagy nem tudjátok-é, hogy a ki a paráznával egyesül, egy test vele? Mert ketten lesznek, úgymond, egy testté.
\par 17 A ki pedig az Úrral egyesül, egy lélek õ vele.
\par 18 Kerüljétek a paráznaságot. Minden bûn, melyet az ember cselekszik, a testen kívül van, de a ki paráználkodik, a maga teste ellen vétkezik.
\par 19 Avagy nem tudjátok-é, hogy a ti testetek a bennetek lakozó Szent Léleknek temploma, a melyet Istentõl nyertetek; és nem a magatokéi  vagytok?
\par 20 Mert áron vétettetek meg; dicsõítsétek  azért az Istent a ti testetekben és lelketekben, a melyek az Istenéi.

\chapter{7}

\par 1 A mik felõl pedig írtatok nékem, jó a férfiúnak asszonyt nem illetni.
\par 2 De a paráznaság miatt minden férfiúnak tulajdon felesége legyen, és minden asszonynak tulajdon férje.
\par 3 A feleségének adja meg a férj a köteles jóakaratot; hasonlóképen a feleség is a férjének.
\par 4 A feleség nem ura a maga testének, hanem a férje; hasonlóképen a férj sem ura a maga testének, hanem a felesége.
\par 5 Ne foszszátok meg egymást, hanemha egyenlõ akaratból bizonyos ideig, hogy ráérjetek a bõjtölésre és az imádkozásra, azután ismét együvé térjetek, hogy a Sátán meg nem kísértsen titeket, mivelhogy magatokat meg nem tartóztathatjátok.
\par 6 Ezt pedig kedvezésképen mondom, nem parancsolat szerint.
\par 7 Mert szeretném, ha minden ember úgy volna, mint én magam is; de kinek kinek tulajdon kegyelmi ajándéka vagyon Istentõl, egynek így, másnak pedig úgy.
\par 8 Mondom pedig a nem házasoknak és az özvegyasszonyoknak, hogy jó nékik, ha úgy maradhatnak, mint én is.
\par 9 De ha magukat meg nem tartóztathatják, házasságban éljenek: mert jobb házasságban élni, mint égni.
\par 10 Azoknak pedig, a kik házasságban vannak, hagyom nem én, hanem az Úr, hogy az asszony  férjétõl el ne váljék.
\par 11 Hogyha pedig elválik is, maradjon házasság nélkül, vagy béküljön meg férjével; és a férj se bocsássa el a feleségét.
\par 12 Egyebeknek pedig én mondom, nem az Úr: Ha valamely atyafinak hitetlen felesége van, és ez vele akar lakni, el ne bocsássa azt.
\par 13 És a mely asszonynak hitetlen férje van, és ez vele akar lakni, el ne bocsássa azt.
\par 14 Mert meg van szentelve a hitetlen férj az õ feleségében, és meg van szentelve a hitetlen asszony az õ férjében, mert különben a ti gyermekeitek tisztátalanok volnának, most pedig szentek.
\par 15 Ha pedig a hitetlen elválik, ám váljék el; nem vettetett szolgaság alá a keresztyén férfiú, vagy asszony az ilyen dolgokban. De békességre hívott minket az Isten.
\par 16 Mert mit tudod, te asszony ha megmentheted-e a férjedet; vagy mit tudod, te férfiú, ha megmentheted-e a feleségedet?
\par 17 Csak a mint kinek-kinek adta az Isten, a mint kit-kit elhívott az Úr, úgy járjon. És minden gyülekezetben ekképen rendelkezem.
\par 18 Körülmetélten hivatott el valaki? ne fedezze el azt; körülmetéletlenül hivatott el valaki? me metélkedjék körül.
\par 19 A körülmetélkedés semmi, a körülmetéletlenség is semmi: hanem Isten parancsolatainak megtartása.
\par 20 Kiki a mely hivatásban hívatott el, abban maradjon.
\par 21 Szolgai állapotban hivattattál el? Ne gondolj vele, sõt ha szabad lehetsz is, inkább élj azzal.
\par 22 Mert az Úrban elhívott szolga az Úrnak szabadosa; hasonlóképen a ki szabadságban hívatott el,  Krisztusnak szolgája.
\par 23 Áron vétettetek meg, ne legyetek embereknek szolgái.
\par 24 Kiki a miben elhívatott, atyámfiai, abban maradjon meg az Isten elõtt.
\par 25 A hajadonok felõl nincs ugyan parancsolatom az Úrtól, de tanácsot adok úgy, mint a ki irgalmasságot nyertem az Úrtól, hogy hitelreméltó legyek.
\par 26 Úgy ítélem azért, hogy jó ez a jelenvaló szükség miatt, hogy tudniillik jó az embernek úgy maradni.
\par 27 Feleséghez köttettél? Ne keress elválást. Megszabadultál feleségedtõl? Ne keress feleséget.
\par 28 De ha veszel is feleséget, nem vétkezel; és ha férjhez megy is a hajadon, nem vétkezik; de az ilyeneknek háborúságuk lesz a testben. Én pedig kedveznék néktek.
\par 29 Ezt pedig azért mondom, atyámfiai, mert az idõ rövidre van szabva ezentúl, azért a kiknek van is feleségök, úgy legyenek, mintha nem volna.
\par 30 És a kik sírnak, mintha nem sírnának; és a kik vígadnak, mintha nem vígadnának; a kik vesznek, mintha semmijök sem volna.
\par 31 És a kik élnek e világgal, mintha nem élnének: mert elmúlik e világnak ábrázatja.
\par 32 Azt akarnám pedig, hogy ti gond nélkül legyetek. A ki házasság nélkül van, arra visel gondot, a mi az Úré, mimódon kedveskedhessék az Úrnak;
\par 33 A ki pedig feleséget vett, a világiakra visel gondot, mimódon kedveskedhessék a feleségének.
\par 34 Különbözik egymástól az asszony és a hajadon. A ki nem ment férjhez, az Úr dolgaira visel gondot, hogy szent legyen mind testében, mind lelkében; a ki pedig férjhez ment, a világiakra visel gondot, mimódon kedveskedhessék a férjének.
\par 35 Ezt pedig a ti hasznotokra mondom; nem hogy tõrt vessek néktek, hanem hogy illendõképen és állhatatosan ragaszkodjatok az Úrhoz háboríthatatlanul.
\par 36 De ha valaki szégyennek tartja az õ hajadon leányára, hogy virágzó idejét múlja, és úgy kell történnie, a mit akar, azt cselekedje, nem vétkezik; menjenek férjhez.
\par 37 A ki pedig szilárdan áll a szívében és a szükség nem kényszeríti, hatalma pedig van a tulajdon akarata fölött, és azt végezte el szívében, hogy megtartja hajadon leányát, jól cselekszi.
\par 38 Azért, a ki férjhez adja, az is jól cselekszi, de a ki nem adja férjhez, még jobban  cselekszi.
\par 39 Az asszonyt törvény köti, míg férje él, de ha férje meghal, szabadon férjhez mehet, a kihez akar, csakhogy az  Úrban.
\par 40 De boldogabb, ha úgy marad, az én véleményem szerint; pedig hiszem, hogy bennem is Istennek lelke van.

\chapter{8}

\par 1 A bálványáldozatok felõl pedig tudjuk, hogy mindnyájunknak van ismeretünk. Az ismeret felfuvalkodottá tesz, a szeretet pedig épít.
\par 2 Ha pedig valaki azt hiszi, hogy tud valamit, még semmit sem ismer úgy, a mint ismernie kell.
\par 3 Hanem ha valaki az Istent szereti, az ismertetik õ tõle.
\par 4 Tehát a bálványáldozati hús evése felõl tudjuk, hogy egy bálvány sincs a világon, és hogy Isten sincs senki más, hanem csak  egy.
\par 5 Mert ha vannak is úgynevezett istenek akár az égben, akár a földön, a minthogy van sok isten és sok úr;
\par 6 Mindazáltal nekünk egy Istenünk van, az Atya, a kitõl van a mindenség, mi is õ benne; és egy  Urunk, a Jézus Krisztus, a ki által van a mindenség, mi is õ általa.
\par 7 De nem mindenkiben van meg ez az ismeret; sõt némelyek a bálvány felõl való lelkiismeretök szerint mind mai napig mint bálványáldozatot eszik, és az õ lelkiismeretök, mivelhogy erõtelen,  megfertõztetik.
\par 8 Pedig az eledel nem tesz minket kedvesekké Isten elõtt; mert ha eszünk is, nem leszünk gazdagabbak; ha nem eszünk is, nem leszünk szegényebbek.
\par 9 De meglássátok, hogy ez a ti szabadságtok valamiképen botránkozásukra ne legyen az erõteleneknek.
\par 10 Mert ha valaki meglát téged, a kinek ismereted van, hogy a bálványtemplomnál vendégeskedel, annak lelkiismerete, mivelhogy erõtelen, nem arra indíttatik-é, hogy megegye a bálványáldozatot?
\par 11 És a te ismereted miatt elkárhozik a te erõtelen atyádfia, a kiért Krisztus meghalt.
\par 12 Így aztán, mikor az atyafiak ellen vétkeztek, és az õ erõtelen lelkiismeretüket megsértitek, a Krisztus ellen vétkeztek.
\par 13 Annakokáért, ha eledel botránkoztatja meg az én atyámfiát, inkább soha sem eszem húst, hogy az én atyámfiát meg ne  botránkoztassam.

\chapter{9}

\par 1 Nem vagyok-é apostol? Nem vagyok-é szabad? Nem láttam-é Jézus Krisztust, a mi Urunkat? Nem az én munkám vagytok-é  ti az Úrban?
\par 2 Ha egyebeknek nem vagyok apostoluk, de bizony néktek az vagyok, mert az én apostolságomnak pecsétje az Úrban ti vagytok.
\par 3 Ez az én védelmem azok ellenében, a kik vádolnak engem.
\par 4 Nincsen-é arra jogunk, hogy együnk és igyunk?
\par 5 Nincsen-é arra jogunk, hogy keresztyén feleségünket magunkkal hordozzuk, mint a többi apostolok is és az Úrnak atyjafiai és Kéfás?
\par 6 Avagy csak nekem és Barnabásnak nincs-é jogunk, hogy ne dolgozzunk?
\par 7 Kicsoda katonáskodik valaha a maga zsoldján? Kicsoda plántál szõlõt, és nem eszik annak gyümölcsébõl? Vagy kicsoda legeltet nyájat, és nem eszik a nyájnak tejébõl?
\par 8 Vajjon emberi módon beszélem-é ezeket? vagy nem ezeket mondja-é a törvény is?
\par 9 Mert a Mózes törvényében meg van írva: Ne kösd fel a nyomtató ökörnek száját. Avagy az ökrökre van-é az Istennek gondja?
\par 10 Avagy nem érettünk mondja-é általában? Mert mi érettünk íratott meg, hogy a ki szánt, reménység alatt kell szántania, és a ki csépel, az õ reménységében részesnek lennie reménység alatt.
\par 11 Ha mi néktek a lelkieket vetettük,  nagy dolog-é, ha mi a ti testi javaitokat aratjuk?
\par 12 Ha egyebek részesülnek a ti javaitokban, mért nem inkább mi? De  mi nem éltünk e szabadsággal; hanem mindent eltûrünk, hogy valami akadályt ne gördítsünk a Krisztus evangyélioma elé.
\par 13 Nem tudjátok-é, hogy a kik a szent dolgokban munkálkodnak, a szent helybõl élnek, és a kik az oltár körül forgolódnak, az oltárral együtt veszik el részüket?
\par 14 Ekképen rendelte az Úr is, hogy a kik az evangyéliomot hirdetik, az evangyéliomból éljenek.
\par 15 De én ezek közül egygyel sem éltem. Nem azért írtam azonban ezeket, hogy velem is így történjék, mert jobb nékem meghalnom, hogysem valaki hiábavalóvá tegye az én dicsekedésemet.
\par 16 Mert ha az evangyéliomot hirdetem, nem dicsekedhetem, mert szükség kényszerít engem. Jaj ugyanis nékem, ha az evangyéliomot nem hirdetem.
\par 17 Mert ha akaratom szerint mívelem ezt, jutalmam van; ha pedig akaratom nélkül, sáfársággal  bízattam meg.
\par 18 Micsoda tehát az én jutalmam? Hogy prédikálásommal ingyenvalóvá tegyem a Krisztus evangyéliomát, hogy ne használjam ki ama  szabadságomat az evangyéliom hirdetésénél.
\par 19 Mert én, noha mindenkivel szemben szabad vagyok, magamat mindenkinek szolgájává tettem, hogy a többséget megnyerjem.
\par 20 És a zsidóknak zsidóvá lettem, hogy zsidókat nyerjek meg; a törvény alatt valóknak törvény alatt valóvá, hogy a törvény alatt valókat megnyerjem;
\par 21 A törvény nélkül valóknak törvénynélkülivé, noha nem vagyok Isten törvénye nélkül, hanem Krisztus törvényében való, hogy törvény nélkül valókat nyerjek meg.
\par 22 Az erõtleneknek erõtelenné lettem, hogy az erõteleneket megnyerjem. Mindeneknek mindenné  lettem, hogy minden módon megtartsak némelyeket.
\par 23 Ezt pedig az evangyéliomért mívelem, hogy részestárs legyek abban.
\par 24 Nem tudjátok-é, hogy a kik versenypályán futnak, mindnyájan futnak ugyan, de egy veszi el a jutalmat? Úgy fussatok, hogy elvegyétek.
\par 25 Mindaz pedig a ki pályafutásban tusakodik, mindenben magatûrtetõ; azok ugyan, hogy romlandó koszorút nyerjenek, mi pedig  romolhatatlant.
\par 26 Én azért úgy futok, mint nem bizonytalanra; úgy viaskodom, mint a ki nem levegõt vagdos;
\par 27 Hanem megsanyargatom testemet és szolgává teszem; hogy míg másoknak prédikálok, magam valami módon méltatlanná ne legyek.

\chapter{10}

\par 1 Nem akarom pedig, hogy ne tudjátok, atyámfiai, hogy a mi atyáink mindnyájan a felhõ alatt voltak, és mindnyájan  a tengeren mentek által;
\par 2 És mindnyájan Mózesre keresztelkedtek meg a felhõben és a tengerben;
\par 3 És mindnyájan egy lelki eledelt ettek;
\par 4 És mindnyájan egy lelki italt ittak, mert ittak a lelki kõsziklából, a mely követi vala õket, e kõszikla pedig Krisztus volt.
\par 5 De azoknak többségét nem kedvelé az Isten, mert elhullának a pusztában.
\par 6 Ezek pedig példáink lõnek, hogy mi ne kívánjunk gonosz dolgokat, a miképen azok kívántak.
\par 7 Se bálványimádók ne legyetek, mint azok közül némelyek, a mint meg van írva: Leüle a nép enni és inni, és felkelének  játszani.
\par 8 Se pedig ne paráználkodjunk mint azok közül paráználkodtak némelyek, és elestek egy napon huszonháromezeren.
\par 9 Se a Krisztust ne kísértsük, a mint közülök kísértették némelyek, és elveszének a kígyók miatt.
\par 10 Se pedig ne zúgolódjatok, miképen õ közülök zúgolódának némelyek, és elveszének a pusztító által.
\par 11 Mindezek pedig példaképen estek rajtok; megírattak pedig a mi tanulságunkra, a kikhez az idõknek vége elérkezett.
\par 12 Azért a ki azt hiszi, hogy áll, meglássa, hogy el ne essék.
\par 13 Nem egyéb, hanem csak emberi kísértés esett rajtatok: de hû az Isten, a ki nem hágy titeket feljebb kísértetni, mint elszenvedhetitek; sõt a kísértéssel egyetemben a kimenekedést is megadja majd, hogy elszenvedhessétek.
\par 14 Azért szerelmeseim, kerüljétek a bálványimádást.
\par 15 Mint okosokhoz szólok, ítéljétek meg ti a mit mondok.
\par 16 A hálaadásnak pohara, a melyet megáldunk, nem a Krisztus vérével való közösségünk-é? A kenyér, a melyet megszegünk,  nem a Krisztus testével való közösségünk-é?
\par 17 Mert egy a kenyér, egy test vagyunk sokan; mert mindnyájan az egy kenyérbõl részesedünk.
\par 18 Tekintsétek meg a test szerint való Izráelt! A kik az áldozatokat eszik, avagy nincsenek-é közösségben az oltárral?
\par 19 Mit mondok tehát? Hogy a bálvány valami, vagy hogy a bálványáldozat valami?
\par 20 Sõt, hogy a mit a pogányok áldoznak, ördögöknek áldozzák és nem Istennek; nem akarom pedig, hogy ti az ördögökkel legyetek közösségben.
\par 21 Nem ihatjátok az Úr poharát és az ördögök poharát; nem lehettek az Úr asztalának és az ördögök asztalának részesei.
\par 22 Vagy haragra ingereljük az Urat? avagy erõsebbek vagyunk-é nálánál?
\par 23 Minden szabad nékem, de nem minden használ; minden szabad nékem, de nem minden épít.
\par 24 Senki ne keresse, a mi az övé, hanem kiki azt, a mi a másé.
\par 25 Mindent, a mit a mészárszékben árulnak, megegyetek, semmit sem tudakozódván a lelkiismeret miatt.
\par 26 Mert az Úré a föld és annak teljessége.
\par 27 Ha pedig valaki meghív titeket a hitetlenek közül és el akartok menni, mindent, a mit elétek hoznak, megegyetek, semmit sem tudakozódván a lelkiismeret miatt.
\par 28 De ha valaki ezt mondja néktek: Egy bálványáldozati hús, ne egyétek meg a miatt, a ki megjelentette, és a lelkiismeretért; mert az Úré a föld és annak teljessége.
\par 29 De nem a tulajdon lelkiismeretet értem, hanem a másikét. Mert miért kárhoztassa az én szabadságomat a más lelkiismerete?
\par 30 Ha pedig én hálaadással veszek részt, miért káromoltatom azért, a miért én hálákat adok?
\par 31 Azért akár esztek, akár isztok, akármit cselekesztek, mindent az Isten dicsõségére míveljetek.
\par 32 Meg ne botránkoztassátok se a zsidókat, se a görögöket, se az Isten gyülekezetét.
\par 33 Miképen én is mindenkinek mindenben kedvében járok, nem keresvén a magam hasznát, hanem a sokaságét, hogy megtartassanak.

\chapter{11}

\par 1 Legyetek az én követõim, mint én is a Krisztusé.
\par 2 Dícsérlek pedig titeket atyámfiai, hogy én rólam mindenben megemlékeztek, és a miképen meghagytam néktek, rendeléseimet megtartjátok.
\par 3 Akarom pedig, hogy tudjátok, hogy minden férfiúnak feje a Krisztus; az asszonynak feje pedig a férfiú; a Krisztusnak feje pedig az  Isten.
\par 4 Minden férfiú, a ki befedett fõvel imádkozik avagy prófétál, megcsúfolja az õ fejét.
\par 5 Minden asszony pedig, a ki befedetlen fõvel imádkozik avagy prófétál, megcsúfolja az õ fejét, mert egy és ugyanaz, mintha megnyiretett volna.
\par 6 Mert ha az asszony nem fedi be fejét, nyiretkezzék is meg, hogy ha pedig éktelen dolog asszonynak megnyiretkezni, vagy megberetváltatni, fedezze be az õ fejét.
\par 7 Mert a férfiúnak nem kell befednie az õ fejét, mivel õ az Istennek képe és dicsõsége; de az asszony a férfiú dicsõsége.
\par 8 Mert nem a férfiú van az asszonyból, hanem az asszony a férfiúból.
\par 9 Mert nem is a férfiú teremtetett az asszonyért, hanem az asszony a férfiúért.
\par 10 Ezért kell az asszonynak hatalmi jelt viselni a fején az angyalok miatt.
\par 11 Mindazáltal sem férfiú nincs asszony nélkül, sem asszony férfiú nélkül az Úrban.
\par 12 Mert a miképen az asszony a férfiúból van, azonképen a férfiú is az asszony által, az egész pedig az Istentõl.
\par 13 Magatokban ítéljétek meg: illendõ dolog-é asszonynak fedetlen fõvel imádni az Istent?
\par 14 Avagy maga a természet is nem arra tanít-é titeket, hogy ha a férfiú nagy hajat visel, csúfsága az néki?
\par 15 Az asszonynak pedig, ha nagy haja van, ékesség az néki; mert a haj fátyol gyanánt adatott néki.
\par 16 Ha pedig valakinek tetszik versengeni, nekünk olyan szokásunk nincsen, sem az Isten gyülekezeteinek.
\par 17 Ezt pedig tudtotokra adván, nem dícsérlek, hogy nem haszonnal, hanem kárral gyûltök egybe.
\par 18 Mert elõször is, mikor egybegyûltök a gyülekezetben, hallom hogy szakadások vannak köztetek; és valami részben hiszem is.
\par 19 Mert szükség, hogy szakadások is legyenek köztetek, hogy a kipróbáltak nyilvánvalókká legyenek  ti köztetek.
\par 20 Mikor tehát egybegyûltök egyazon helyre, nincs úrvacsorájával való élés:
\par 21 Mert kiki az õ saját vacsoráját veszi elõ az evésnél; és némely éhezik, némely pedig dõzsöl.
\par 22 Hát nincsenek-é házaitok az evésre és ivásra? Avagy az Isten gyülekezetét vetitek-é meg, és azokat szégyenítitek-é meg, a kiknek nincsen? Mit mondjak néktek? Dícsérjelek-é titeket ebben? Nem dícsérlek.
\par 23 Mert én az Úrtól vettem, a mit néktek elõtökbe is adtam: hogy az Úr Jézus  azon az éjszakán, melyen elárultaték, vette a kenyeret,
\par 24 És hálákat adván, megtörte és ezt monda: Vegyétek, egyétek! Ez az én testem, mely ti érettetek megtöretik; ezt cselekedjétek az én emlékezetemre.
\par 25 Hasonlatosképen a pohárt is vette, minekutána vacsorált volna, ezt mondván: E pohár amaz új testamentom az én vérem által; ezt cselekedjétek, valamennyiszer isszátok az én emlékezetemre.
\par 26 Mert valamennyiszer eszitek e kenyeret és isszátok e pohárt, az Úrnak halálát hirdessétek, a míg eljövend.
\par 27 Azért a ki méltatlanul eszi e kenyeret, vagy issza az Úrnak poharát, vétkezik az Úr teste és vére ellen.
\par 28 Próbálja meg azért az ember magát, és úgy egyék abból a kenyérbõl, és úgy igyék abból a pohárból,
\par 29 Mert a ki méltatlanul eszik és iszik, ítéletet eszik és iszik magának, mivelhogy nem becsüli meg az Úrnak testét.
\par 30 Ezért van ti köztetek sok erõtlen és beteg, és alusznak sokan.
\par 31 Mert ha mi ítélnõk magunkat, nem ítéltetnénk el.
\par 32 De mikor ítéltetünk, az az Úrtól taníttatunk, hogy a világgal együtt el ne kárhoztassunk.
\par 33 Azért atyámfiai, mikor egybegyûltök az evésre, egymást megvárjátok.
\par 34 Ha pedig valaki éhezik, otthon egyék, hogy ítéletre ne gyûljetek egybe. A többire nézve, majd ha hozzátok megyek, rendelkezem.

\chapter{12}

\par 1 A lelki ajándékokra nézve pedig nem akarom, atyámfiai, hogy tudatlanok legyetek.
\par 2 Tudjátok, hogy pogányok voltatok, vitetvén, a mint vitettetek, a néma bálványokhoz.
\par 3 Azért tudtotokra adom néktek, hogy senki, a ki Istennek Lelke által szól, ne mondja Jézust átkozottnak; és  senki sem mondhatja Úrnak Jézust, hanem csak a Szent Lélek által.
\par 4 A kegyelmi ajándékokban pedig különbség van, de ugyanaz a Lélek.
\par 5 A szolgálatokban is különbség van, de ugyanaz az Úr.
\par 6 És különbség van a cselekedetekben is, de ugyanaz az Isten, a ki cselekszi mindezt mindenkiben.
\par 7 Mindenkinek azonban haszonra adatik a Léleknek kijelentése.
\par 8 Némelyiknek ugyanis bölcseségnek beszéde adatik a Lélek által; másiknak pedig tudománynak  beszéde ugyanazon Lélek szerint;
\par 9 Egynek hit ugyanazon Lélek által; másnak pedig gyógyítás ajándékai azon egy Lélek által;
\par 10 Némelyiknek csodatévõ erõknek munkái; némelyiknek meg prófétálás; némelyiknek pedig lelkeknek megítélése;  másiknak nyelvek nemei; másnak pdig nyelvek magyarázása;
\par 11 De mindezeket egy és ugyanaz a Lélek cselekszi, osztogatván mindenkinek külön, a mint akarja.
\par 12 Mert a miképen a test egy és sok tagja van, az egy testnek tagjai pedig, noha sokan vannak, mind  egy test, azonképen a Krisztus is.
\par 13 Mert hiszen egy Lélek által mi mindnyájan egy testté kereszteltettünk meg, akár zsidók, akár görögök, akár szolgák, akár szabadok; és mindnyájan egy Lélekkel itattattunk meg.
\par 14 Mert a test sem egy tag, hanem sok.
\par 15 Ha ezt mondaná a láb: mivelhogy nem kéz vagyok, nem vagyok a testbõl való; avagy nem a testbõl való-é azért?
\par 16 És ha a fül ezt mondaná: mivelhogy nem vagyok szem, nem vagyok a testbõl való; avagy nem a testbõl való-é azért?
\par 17 Ha az egész test szem, hol a hallás? ha az egész hallás, hol a szaglás?
\par 18 Most pedig az Isten elhelyezte a tagokat a testben egyenként mindeniket, a mint akarta.
\par 19 Ha pedig az egész egy tag volna hol volna a test?
\par 20 Így azonban sok tag van ugyan, de egy test.
\par 21 Nem mondhatja pedig a szem a kéznek: Nincs rád szükségem; vagy viszont a fej a lábaknak: Nem kelletek nékem.
\par 22 Sõt sokkal inkább, a melyek a test legerõtelenebb tagjainak látszanak, azok igen szükségesek:
\par 23 És a melyeket a test tisztességtelenebb tagjainak tartunk, azoknak nagyobb tisztességet tulajdonítunk; és a melyek éktelenek bennünk, azok nagyobb ékességben részesülnek;
\par 24 A melyek pedig ékesek bennünk, azoknak nincs erre szükségök. De az Isten szerkeszté egybe a testet, az alábbvalónak nagyobb tisztességet adván,
\par 25 Hogy ne legyen hasonlás a testben, hanem ugyanarról gondoskodjanak egymásért a tagok.
\par 26 És akár szenved egy tag, vele együtt szenvednek a tagok mind; akár tisztességgel illettetik egy tag, vele együtt örülnek a tagok mind.
\par 27 Ti pedig a Krisztus teste vagytok, és tagjai rész szerint.
\par 28 És pedig némelyeket rendelt az Isten az anyaszentegyházban elõször apostolokul, másodszor prófétákul, harmadszor tanítókul; azután csodatévõ erõket, aztán gyógyításnak ajándékait, gyámolokat, kormányokat, nyelvek nemeit.
\par 29 Avagy mindnyájan apostolok-é? Vagy mindnyájan próféták-é? Avagy mindnyájan tanítók-é? Vagy mindnyájan csodatévõ erõk-é?
\par 30 Avagy mindnyájoknak van-é gyógyításra való ajándéka? Vagy mindnyájan szólnak-é nyelveken? Vagy mindnyájan magyaráznak-é?
\par 31 Igyekezzetek pedig a hasznosabb ajándékokra. És ezenfelül még egy kiváltképen való útat mutatok néktek.

\chapter{13}

\par 1 Ha embereknek vagy angyaloknak nyelvén szólok is, szeretet pedig nincsen én bennem, olyanná lettem, mint a zengõ ércz vagy pengõ czimbalom.
\par 2 És ha jövendõt tudok is mondani, és minden titkot és minden tudományt ismerek is; és ha egész hitem  van is, úgyannyira, hogy hegyeket mozdíthatok ki helyökrõl, szeretet pedig nincsen én bennem, semmi vagyok.
\par 3 És ha vagyonomat mind felétetem is, és ha testemet tûzre adom is, szeretet pedig nincsen én bennem, semmi hasznom abból.
\par 4 A szeretet hosszútûrõ, kegyes; a szeretet nem irígykedik, a szeretet nem kérkedik, nem fuvalkodik fel.
\par 5 Nem cselekszik éktelenül, nem keresi a maga hasznát, nem gerjed haragra, nem rójja fel a gonoszt,
\par 6 Nem örül a hamisságnak, de együtt örül az igazsággal;
\par 7 Mindent elfedez, mindent hiszen, mindent remél, mindent eltûr.
\par 8 A szeretet soha el nem fogy: de legyenek bár jövendõmondások, eltöröltetnek; vagy akár nyelvek, megszünnek; vagy akár ismeret, eltöröltetik.
\par 9 Mert rész szerint van bennünk az ismeret, rész szerint a prófétálás:
\par 10 De mikor eljõ a teljesség, a rész szerint való eltöröltetik.
\par 11 Mikor gyermek valék, úgy szóltam, mint gyermek, úgy gondolkodtam, mint gyermek, úgy értettem, mint gyermek: minekutána pedig férfiúvá lettem, elhagytam a gyermekhez illõ dolgokat.
\par 12 Mert most tükör által homályosan látunk, akkor pedig színrõl színre; most rész szerint van bennem az ismeret, akkor pedig úgy ismerek majd, a mint én is megismertettem.
\par 13 Most azért megmarad a hit, remény, szeretet, e három; ezek között pedig legnagyobb a szeretet.

\chapter{14}

\par 1 Kövessétek a szeretetet, kívánjátok a lelki ajándékokat, leginkább pedig, hogy  prófétáljatok.
\par 2 Mert a ki nyelveken szól, nem embernek szól, hanem az Istennek; mert senki sem érti, hanem lélekben beszél titkos dolgokat.
\par 3 A ki pedig prófétál, embereknek beszél épülésére, intésre és vígasztalásra.
\par 4 A ki nyelveken szól, magát építi; de a ki prófétál, a gyülekezetet építi.
\par 5 Szeretném ugyanis, ha mindnyájan szólnátok nyelveken, de inkább, hogy prófétálnátok; mert nagyobb a próféta, mint nyelveken szóló,  kivévén, ha megmagyarázza, hogy a gyülekezet épüljön.
\par 6 Ha már most, atyámfiai, hozzátok megyek, és nyelveken szólok, mit használok néktek, ha vagy kijelentésben, vagy ismeretben, vagy prófétálásban, vagy tanításban nem szólok hozzátok?
\par 7 Hiszen ha az élettelen hangszerek, akár fuvola, akár czitera, nem adnak megkülönböztethetõ hangokat, mimódon ismerjük meg, a mit fuvoláznak vagy cziteráznak?
\par 8 Mert ha a trombita bizonytalan zengést tészen, kicsoda készül a harczra?
\par 9 Azonképen ti is, ha érthetõ nyelven nem beszéltek, mimódon értik meg, a mit szóltok? Csak a levegõbe fogtok beszélni.
\par 10 Példa mutatja, oly sokféle szólás van a világon, és azok közül egy sem érthetetlen.
\par 11 Hogyha azért nem tudom a szónak értelmét, a beszélõnek idegen leszek, és a beszélõ is idegen elõttem.
\par 12 Azonképen ti is, minthogy lelki ajándékokat kívántok, a gyülekezet építésére igyekezzetek, hogy gyarapodjatok.
\par 13 Azért a ki nyelveken szól, imádkozzék, hogy megmagyarázza.
\par 14 Mert ha nyelvvel könyörgök, a lelkem könyörög, de értelmem gyümölcstelen.
\par 15 Hogy van hát? Imádkozom a lélekkel, de imádkozom az értelemmel is; énekelek a lélekkel, de énekelek az értelemmel is.
\par 16 Mert ha lélekkel mondasz áldást, az ott lévõ avatatlan miképen fog a te hálaadásodra Áment mondani, mikor nem tudja, mit beszélsz?
\par 17 Mert jóllehet, te szépen mondasz áldást, de más nem épül abból.
\par 18 Hálát adok az én Istenemnek, hogy mindnyájatoknál inkább tudok nyelveken szólni;
\par 19 De a gyülekezetben inkább akarok öt szót szólani értelemmel, hogy egyebeket is tanítsak, hogy nem mint tízezer szót nyelveken.
\par 20 Atyámfiai, ne legyetek gyermekek, értelemben; hanem a gonoszságban legyetek gyermekek,  értelemben pedig érettek legyetek.
\par 21 A törvényben meg van írva: Idegen nyelveken és idegen ajkakkal szólok e népnek, és így sem hallgatnak rám, azt mondja az Úr.
\par 22 A nyelvek tehát jelül vannak, nem a hívõknek, hanem a hitetleneknek; a prófétálás pedig nem a hitetleneknek, hanem a hívõknek.
\par 23 Azért ha az egész gyülekezet egybegyûl és mindnyájan nyelveken szólnak, bemenvén az idegenek vagy hitetlenek, nem azt mondják-é, hogy õrjöngtök?
\par 24 De ha mindnyájan prófétálnak és bemegy egy hitetlen, vagy avatatlan, az mindenektõl megfeddetik, mindenektõl megítéltetik,
\par 25 És ilyen módon az õ szívének titkai nyilvánvalókká lesznek; és így arczra borulva imádja az Istent, hirdetvén, hogy bizonynyal az Isten lakik ti bennetek.
\par 26 Hogy van hát atyámfiai? Mikor egybegyûltök, mindeniteknek van zsoltára, tanítása,  nyelve, kijelentése, magyarázata. Mindenek épülésre legyenek.
\par 27 Ha valaki nyelveken szól, kettõ vagy legfeljebb három legyen, mégpedig egymás után; és egy magyarázza meg:
\par 28 Ha pedig nincsen magyarázó, hallgasson a gyülekezetben; hanem magának szóljon és az Istennek.
\par 29 A próféták pedig ketten vagy hárman beszéljenek; és a többiek ítéljék meg.
\par 30 De ha egy másik ott ülõ vesz kijelentést, az elsõ hallgasson.
\par 31 Mert egyenként mindnyájan prófétálhattok, hogy mindenki tanuljon, és mindenki vígasztalást vegyen;
\par 32 És a prófétalelkek engednek a prófétáknak;
\par 33 Mert az Isten nem a visszavonásnak, hanem a békességnek Istene; miként a szentek minden gyülekezetében.
\par 34 A ti asszonyaitok hallgassanak a gyülekezetekben, mert nincsen megengedve nékik, hogy szóljanak; hanem engedelmesek legyenek,  a mint a törvény is mondja.
\par 35 Hogyha pedig tanulni akarnak valamit, kérdezzék meg otthon az õ férjüket; mert éktelen dolog asszonynak szólni a gyülekezetben.
\par 36 Avagy ti tõletek származott-é az Isten beszéde, avagy csak hozzátok jutott el?
\par 37 Ha valaki azt hiszi, hogy õ próféta, vagy lelki ajándék részese, vegye eszébe, hogy a miket néktek írok, az Úr rendeletei azok.
\par 38 A ki pedig tudatlan, legyen tudatlan.
\par 39 Azért atyámfiai törekedjetek prófétálásra, és a nyelveken szólást se tiltsátok.
\par 40 Mindenek ékesen és jó renddel legyenek.

\chapter{15}

\par 1 Eszetekbe juttatom továbbá, atyámfiai, az evangyéliomot, melyet hirdettem néktek, melyet be  is vettetek, melyben állotok is,
\par 2 A mely által üdvözültök is, ha megtartjátok, a minémû beszéddel hirdettem néktek, hacsak nem hiába lettetek hívõkké.
\par 3 Mert azt adtam elõtökbe fõképen, a mit én is úgy vettem, hogy a Krisztus meghalt a mi bûneinkért  az írások szerint;
\par 4 És hogy eltemettetett; és hogy feltámadott a harmadik napon az írások szerint;
\par 5 És hogy megjelent Kéfásnak; azután a  tizenkettõnek;
\par 6 Azután megjelent több mint ötszáz atyafinak egyszerre, kik közül a legtöbben mind máig élnek, némelyek azonban el is aludtak;
\par 7 Azután megjelent Jakabnak; azután mind az apostoloknak;
\par 8 Legutolszor pedig mindenek között, mint egy idétlennek, nékem is megjelent.
\par 9 Mert én vagyok a legkisebb az apostolok között, ki nem vagyok méltó, hogy apostolnak neveztessem, mert háborgattam az Istennek  anyaszentegyházát.
\par 10 De Isten kegyelme által vagyok, a mi vagyok; és az õ hozzám való kegyelme  nem lõn hiábavaló; sõt többet munkálkodtam, mint azok mindnyájan de nem én, hanem az Istennek velem való kegyelme.
\par 11 Akár én azért, akár azok, így prédikálunk, és így lettetek ti hívõkké.
\par 12 Ha azért Krisztusról hirdettetik, hogy a halottak közül feltámadott, mimódon mondják némelyek ti köztetek, hogy nincsen halottak feltámadása?
\par 13 Mert ha nincsen halottak feltámadása, akkor Krisztus sem támadott fel.
\par 14 Ha pedig Krisztus fel nem támadott, akkor hiábavaló a mi prédikálásunk, de hiábavaló a ti hitetek is.
\par 15 Sõt az Isten hamis bizonyságtevõinek is találtatunk, mivelhogy az Isten felõl bizonyságot tettünk, hogy feltámasztotta a Krisztust; a kit nem támasztott fel, ha csakugyan nem támadnak fel a halottak.
\par 16 Mert ha a halottak fel nem támadnak, a Krisztus sem támadott fel.
\par 17 Ha pedig a Krisztus fel nem támadott, hiábavaló a ti hitetek; még bûneitekben vagytok.
\par 18 A kik a Krisztusban elaludtak, azok is elvesztek tehát.
\par 19 Ha csak ebben az életben reménykedünk a Krisztusban, minden embernél nyomorultabbak vagyunk.
\par 20 Ámde Krisztus feltámadott a halottak közül, zsengéjök  lõn azoknak, kik elaludtak.
\par 21 Miután ugyanis ember által van a halál, szintén ember által van a halottak feltámadása is.
\par 22 Mert a miképen Ádámban mindnyájan meghalnak, azonképen a Krisztusban is mindnyájan megeleveníttetnek.
\par 23 Mindenki pedig a maga rendje szerint. Elsõ zsenge a Krisztus; azután a kik a Krisztuséi,  az õ eljövetelekor.
\par 24 Aztán a vég, mikor átadja az országot az Istennek és Atyának; a mikor eltöröl minden birodalmat és minden hatalmat és erõt.
\par 25 Mert addig kell néki uralkodnia, mígnem ellenségeit mind lábai alá veti.
\par 26 Mint utolsó ellenség töröltetik el a halál.
\par 27 Mert mindent az õ lábai alá vetett. Mikor pedig azt mondja, hogy minden alája  van vetve, nyilvánvaló, hogy azon kívül, a ki neki mindent alávetett.
\par 28 Mikor pedig minden alája vettetett, akkor maga a Fiú is alávettetik annak, a ki neki mindent alávetett, hogy az Isten legyen minden mindenben.
\par 29 Különben mit cselekesznek azok, a kik a halottakért keresztelkednek meg, a halottak teljességgel nem támadnak fel? Miért is keresztelkednek meg a halottakért?
\par 30 Mi is miért veszélyeztetjük magunkat minden pillanatban?
\par 31 Naponként halál révén állok. A veletek való dicsekedésre  mondom, mely van nékem a Krisztus Jézusban a mi Urunkban.
\par 32 Ha csak emberi módon viaskodtam Efézusban a fenevadakkal, mi a hasznom abból, ha a halottak fel nem támadnak? Együnk és igyunk, holnap úgyis meghalunk!
\par 33 Ne tévelyegjetek. Jó erkölcsöt megrontanak gonosz  társaságok.
\par 34 Serkenjetek fel igazán és ne vétkezzetek; mert némelyek nem ismerik Istent; megszégyenítéstekre  mondom.
\par 35 De mondhatná valaki: Mi módon támadnak fel a halottak? és minémû testtel jönnek ki?
\par 36 Balgatag! a mit te vetsz, nem elevenedik meg, hanemha megrothadánd.
\par 37 És abban, a mit elvetsz, nem azt a testet veted el, a mely majd kikél, hanem puszta magot, talán búzáét, vagy más egyébet.
\par 38 Az Isten pedig testet ád annak, a mint akarta, és pedig mindenféle magnak az õ saját testét.
\par 39 Nem minden test azon egy test, hanem más az embernek teste, más a barmoknak teste, más a halaké, más a madaraké.
\par 40 És vannak mennyei testek és földi testek; de más a mennyeiek dicsõsége, más a földieké.
\par 41 Más a napnak dicsõsége és más a holdnak dicsõsége és más a csillagok dicsõsége; mert csillag a csillagtól különbözik dicsõségre nézve.
\par 42 Épenígy a halottak feltámadása is. Elvettetik romlandóságban, feltámasztatik romolhatatlanságban;
\par 43 Elvettetik gyalázatosságban, feltámasztatik dicsõségben; elvettetik erõtelenségben, feltámasztatik erõben.
\par 44 Elvettetik érzéki test, feltámasztatik lelki test. Van érzéki test, és van lelki test is.
\par 45 Így is van megírva: Lõn az elsõ ember, Ádám, élõ lélekké; az utolsó Ádám megelevenítõ szellemmé.
\par 46 De nem a lelki az elsõ, hanem az érzéki, azután a lelki.
\par 47 Az elsõ ember földbõl való, földi; a második ember, az Úr, mennybõl való.
\par 48 A milyen ama földi, olyanok a földiek is; és a milyen ama mennyei, olyanok a mennyeiek is.
\par 49 És a miképen hordtuk a földinek ábrázatját, hordani fogjuk  a mennyeinek ábrázatját is.
\par 50 Azt pedig állítom atyámfiai, hogy test és vér nem örökölheti Isten országát, sem a romlandóság nem örökli a romolhatatlanságot.
\par 51 Ímé titkot mondok néktek. Mindnyájan ugyan nem aluszunk el, de mindnyájan elváltozunk.
\par 52 Nagy hirtelen, egy szempillantásban, az utolsó trombitaszóra; mert trombita fog szólni, és a halottak feltámadnak romolhatatlanságban, és mi elváltozunk.
\par 53 Mert szükség, hogy ez a romlandó test romolhatatlanságot öltsön magára, és e halandó test halhatatlanságot öltsön magára.
\par 54 Mikor pedig ez a romlandó test romolhatatlanságba öltözik, és e halandó halhatatlanságba öltözik, akkor beteljesül amaz ige, mely meg vagyon írva. Elnyeletett a halál diadalra.
\par 55 Halál! hol a te fullánkod? Pokol! hol a te diadalmad?
\par 56 A halál fullánkja pedig a bûn; a bûn ereje  pedig a törvény.
\par 57 De hála az Istennek, a ki a diadalmat adja nékünk a mi Urunk Jézus Krisztus által.
\par 58 Azért szerelmes atyámfiai erõsen álljatok, mozdíthatatlanul, buzgólkodván az Úrnak  dolgában mindenkor, tudván, hogy a ti munkátok nem hiábavaló az Úrban.

\chapter{16}

\par 1 Ami a szentek számára való alamizsnát illeti, a miképen Galáczia gyülekezeteinek rendeltem, ti  is azonképen cselekedjetek.
\par 2 A hétnek elsõ napján mindenitek tegye félre magánál, a mit sikerül összegyûjtenie, hogy ne akkor történjék a gyûjtés, a mikor odamegyek.
\par 3 Mikor pedig megérkezem, a kiket javaltok leveleitek által, azokat küldöm el, hogy elvigyék Jeruzsálembe a ti ajándékotokat.
\par 4 Ha pedig méltó lesz, hogy én is elmenjek, velem együtt jönnek.
\par 5 Elmegyek pedig ti hozzátok, mikor Macedónián általmenéndek: mert Macedónián általmegyek,
\par 6 Nálatok azonban talán megmaradok, vagy ott is telelek, hogy ti kísérjetek el, a hová menéndek.
\par 7 Mert nem akarlak titeket épen csak átmenet közben látni, de reménylem, hogy valami ideig nálatok maradok, ha az Úr engedi.
\par 8 Efézusban pedig pünkösdig maradok.
\par 9 Mert nagy kapu nyílott meg elõttem és hasznos, az ellenség is sok.
\par 10 Hogyha pedig megérkezik Timótheus, meglássátok, hogy bátorságos maradása legyen nálatok; mert az Úrnak dolgát cselekszi, mint én is.
\par 11 Senki azért õt meg ne vesse: hanem bocsássátok el õt békességgel, hogy hozzám jöhessen; mert várom õt az atyafiakkal együtt.
\par 12 A mi pedig Apollós atyafit illeti, igen kértem õt, hogy menjen el hozzátok az atyafiakkal együtt: de semmiképpen sem volt kedve, hogy most elmenjen; de majd elmegy, mihelyt jó alkalmatossága lészen.
\par 13 Vigyázzatok, álljatok meg a hitben, legyetek férfiak, legyetek erõsek!
\par 14 Minden dolgotok szeretetben menjen végbe!
\par 15 Intelek pedig titeket, atyámfiai, hiszen tudjátok, hogy Stefanásnak háznépe Akhája zsengéje,  és õk a szenteknek való szolgálatra adták magukat.
\par 16 Hogy ti is engedelmeskedjetek az ilyeneknek, és mindenkinek, a ki velök szolgál és fárad.
\par 17 Örvendezek pedig a Stefanás Fortunátus és Akhaikus eljövetelének; mert a bennetek való fogyatkozást ezek betöltötték.
\par 18 Mert megnyugtatták az én lelkemet és a tiéteket is. Megbecsüljétek azért az ilyeneket.
\par 19 Köszöntenek titeket Ázsia gyülekezetei; köszöntenek titeket az Úrban felette igen Akvila és Prisczilla, a házuknál levõ gyülekezettel  egybe.
\par 20 Köszöntenek titeket az atyafiak mindnyájan. Köszöntsétek egymást szent csókkal.
\par 21 A köszöntés a saját kezemmel, a Páléval.
\par 22 Ha valaki nem szereti az Úr Jézus Krisztust, legyen átkozott! Maran atha.
\par 23 Az Úr Jézus Krisztusnak kegyelme veletek!
\par 24 Az én szeretetem mindnyájatokkal a Jézus Krisztusban! Ámen.


\end{document}