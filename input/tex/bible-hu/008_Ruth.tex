\begin{document}

\title{Rut könyve}


\chapter{1}

\par 1 És lõn azon napokban, a mikor a bírák bíráskodának, éhség lõn a földön. És elméne egy férfi a Júda Bethlehemébõl, hogy Moáb mezején tartózkodjék; õ meg a felesége, és a két fia.
\par 2 A férfi neve Elimélek és felesége neve Naómi; két fiok neve pedig Mahlon és Kiljon; efrataiak, a Júda Bethlehemébõl valók. És eljutának Moáb mezejére, és ott valának.
\par 3 Meghala pedig Elimélek a Naómi férje, és marada õ és az õ két fia.
\par 4 A kik Moábita leányokat võnek feleségül; az egyiknek neve Orpa, és a másiknak neve Ruth. És ott lakozának közel tíz esztendeig.
\par 5 Meghalának õk is mind a ketten, Mahlon is, Kiljon is, és marada az asszony az õ két fia és férje nélkül.
\par 6 Felkele azért õ és az õ menyei, és õ visszatére Moáb mezejérõl; mert hallotta vala Moáb mezején, hogy meglátogatta az Úr az õ népét, hogy adjon nékik kenyeret.
\par 7 És kiméne arról a helyrõl, a hol volt, és véle a két menye. És menének az úton, hogy visszatérjenek Júda földére,
\par 8 És monda Naómi az õ két menyének: Menjetek, térjetek vissza, kiki az õ anyjának házához. Cselekedjék az Úr irgalmasságot veletek, a miképen ti cselekedtetek a megholtakkal és én velem!
\par 9 Adja az Úr tinéktek, hogy találjatok nyugodalmat, kiki az õ férje házában. És megcsókolá õket; és õk nagy felszóval sírának.
\par 10 És mondák néki: Bizony mi veled együtt térünk a te népedhez!
\par 11 Naómi pedig mondá: Térjetek vissza leányaim! Miért jönnétek én velem? Hát ugyan vannak-é még fiak az én méhemben, a kik férjeitek lehetnének?
\par 12 Térjetek vissza leányaim! menjetek, mert én már vénebb vagyok, semhogy férjhez mehetnék. Még ha azt mondanám is, hogy van reménységem; még ha ez éjjel férjhez mennék is és szülnék is fiakat:
\par 13 Ugyan megvárhatnátok-é õket, a míg felnõnek? Ugyan megtartóztatnátok-é magatokat miattok, hogy férjhez ne menjetek? Ne, édes leányaim! Mert nagyobb az én keserüségem, mint a tietek, mert engem talált az Úrnak keze.
\par 14 Azok pedig nagy felszóval tovább sírának. És Orpa megcsókolá az õ napát; Ruth azonban ragaszkodék hozzá.
\par 15 Õ pedig monda: Ímé a te sógorasszonyod visszatért az õ népéhez, és az õ isteneihez, térj vissza te is a te sógorasszonyod után.
\par 16 Ruth pedig monda: Ne unszolj, hogy elhagyjalak, hogy visszaforduljak tõled. Mert a hova te mégy, oda megyek, és a hol te megszállsz, ott szállok meg; néped az én népem, és Istened az én Istenem.
\par 17 A hol te meghalsz, ott halok meg, ott temessenek el engem is. Úgy tegyen velem az Úr akármit, hogy csak a halál választ el engem tõled.
\par 18 Mikor pedig látá, hogy erõsködik vele menni, nem szóla néki többet.
\par 19 És menének mind a ketten, míglen Bethlehembe érkezének. És lõn, hogy mikor Bethlehembe érkezének, megmozdult az egész város miattok, és mondák: Nemde nem Naómi ez?
\par 20 És õ monda nékik: Ne hívjatok engem Naóminak, hívjatok inkább Márának, mert nagy keserûséggel illetett engem a Mindenható.
\par 21 Többed magammal mentem el, és elárvultan hozott vissza engemet az Úr; miért hívnátok hát engem Naóminak, holott az Úr ellenem fordult, és a Mindenható nyomorúsággal illetett engemet?
\par 22 Így tért vissza Naómi, és vele a Moábita Ruth, az õ menye, a ki hazatért Moáb mezejérõl. Megérkezének pedig Bethlehembe az árpaaratás kezdetén.

\chapter{2}

\par 1 Vala pedig Naóminak egy rokona az õ férje után, elõkelõ derék ember Elimélek nemzetségébõl; neve Boáz.
\par 2 És monda a Moábita Ruth Naóminak: Hadd menjek, kérlek, a mezõre, hogy kalászokat szedegessek az után, a kinek szemei elõtt kedvességet találok. És az monda: Menj édes leányom.
\par 3 Elméne azért és odaérkezék, szedegete a mezõn az aratók után, és történetesen oda talált a Boáz szántóföldjére, a ki Elimélek nemzetségébõl való volt.
\par 4 És ímé Boáz kijöve Bethlehembõl, és monda az aratóknak: Az Úr legyen veletek! És õk mondának néki: Áldjon meg téged az Úr!
\par 5 És monda Boáz az õ szolgájának, a ki az aratók felügyelõje volt: Kié ez a leányzó?
\par 6 És felele a szolga, az aratók felügyelõje, és monda: Az a Moábita leányzó ez, a ki Naómival jött a Moáb mezejérõl.
\par 7 A ki mondá: Hadd szedegessek, kérlek, és hadd tarlózgassak a kévék közt az aratók után. És ide jöve, és itt marada reggeltõl fogva mind mostanig: és csak épen most pihent egy keveset a házban.
\par 8 És mondá Boáz Ruthnak: Ugyan hallod-é édes leányom! Ne menj szedegetni más mezõre és el ne menj innen; hanem menj mindenütt szolgálóim után.
\par 9 Szemeid legyenek a mezõn, a melyet aratnak, és járj utánok. Ímé meghagytam a szolgáknak, hogy ne bántsanak téged, és ha megszomjúhozol, menj az edényekhez és igyál abból, a mit a szolgák merítenek.
\par 10 Akkor ez arczra borula és meghajtá magát a földig, és monda néki: Hogy-hogy találtam ilyen kedvességet a te szemeid elõtt, hogy rám tekintettél, holott én idegen vagyok?
\par 11 Boáz pedig felele, és monda néki: Bizony elmondtak nékem mindent, amit cselekedtél a te napaddal, férjed halála után, hogy elhagytad a te atyádat és a te anyádat és a te születésednek földét, és jöttél ahhoz a néphez, a melyet nem ismertél azelõtt.
\par 12 Fizessen meg az Úr a te cselekedetedért, és legyen teljes a te jutalmad az Úrtól, Izráelnek Istenétõl, a kinek szárnyai alatt oltalmat keresni jöttél.
\par 13 Ez pedig monda: Találjak kedvességet a te szemeid elõtt uram, mert megvigasztaltál engem, és mert a te szolgálódnak szívéhez szóltál, holott én nem vagyok a te szolgálóleányaid közül.
\par 14 És az evésnek idejekor szóla néki Boáz: Jer ide, és egyél a kenyérbõl, és mártsd be a te falatodat az eczetes lébe. És õ leüle az aratók mellé. És pergelt gabonát nyújtott néki, és evett, s jóllakott, sõt még hagyott is.
\par 15 Azután felállott, hogy szedegessen és Boáz megparancsolta az õ szolgáinak, mondván: A kévék között is hadd szedegessen, és meg ne pirongassátok õt.
\par 16 Sõt húzogassatok ki néki a kévékbõl is, és hagyogassatok el, hogy szedje fel, és meg ne dorgáljátok õt.
\par 17 És õ szedegete a mezõn mind estiglen, és kicsépelé, a mit szedegetett, és lett abból szinte egy efa árpa.
\par 18 És felvevé és beméne a városba, és napaasszonya látá, a mit õ szedegetett. Aztán elõvevé és odaadá néki, a mit meghagyott, minekutána jóllakott vala.
\par 19 És monda néki az õ napaasszonya: Hol szedegettél ma, és hol munkálkodtál? Áldott legyen, a ki rád tekintett. És elbeszélte az õ napaasszonyának, hogy kinél munkálkodott, és monda: Annak a férfiúnak a neve, a kinél ma munkálkodtam: Boáz.
\par 20 És monda Naómi az õ menyének: Áldott legyen õ az Úrtól, a ki nem vonta meg irgalmasságát az élõktõl és a megholtaktól! És Naómi monda néki: Közel valónk nékünk az a férfiú; legközelebbi rokonaink közül való õ.
\par 21 És szóla a Moábita Ruth: Még azt is mondta õ nékem: Menj mindenütt az én munkásaim után, a míg csak el nem végzik az én egész aratásomat.
\par 22 És monda Naómi Ruthnak, az õ menyének: Jó, édes leányom, hogy az õ szolgálóival jársz, ne is találjanak téged más mezõn.
\par 23 Így járt õ mindenütt a Boáz szolgálói után, szedegetve, míg az árpaaratás és búzaaratás bevégzõdött; és az õ napaasszonyával lakott.

\chapter{3}

\par 1 És monda néki Naómi, az õ napaasszonya: Édes leányom! ne keressek-é néked nyugalmat, hogy jól legyen dolgod?
\par 2 Avagy nem rokonunk-é Boáz, a kinek szolgálóival voltál? Ímé õ az éjjel árpát szór a szérûn.
\par 3 Annakokáért fürödjél meg, és kend meg magadat, és vedd magadra ruháidat, és menj le a szérûre; észre ne vétesd magadat a férfiúval, a míg el nem végezi ételét és italát.
\par 4 És majd ha lefekszik, jegyezd meg a helyet, a hol fekszik, és menj oda, és hajtsd fel a leplet lábánál, és feküdjél oda. Õ majd megmondja néked, mit cselekedjél.
\par 5 És õ monda néki: Mindazt, a mit mondasz, megcselekszem.
\par 6 És lement a szérûre, és mindent úgy cselekedett, a mint napaasszonya parancsolta.
\par 7 És Boáz evett és ivott, és felvidámult az õ szíve. És elment, hogy lefeküdjék a garmada szélén; és az eljött titkon, és felhajtá lába felõl a leplet, és lefeküvék.
\par 8 Történt pedig éjfél tájon, hogy felrettent a férfiú, és odafordult. És ímé: asszony fekszik az õ lábainál.
\par 9 És monda: Kicsoda vagy te? És az monda: Én Ruth vagyok, a te szolgálód; terjeszszed ki hát takaródat a te szolgálódra, mert te vagy a legközelebbi rokon.
\par 10 És õ monda: Az Úrnak áldotta vagy te, édes leányom! Utóbbi szereteteddel jobbat cselekedtél, mint az elsõvel: hogy nem jártál az ifjak után, sem szegény, sem gazdag után.
\par 11 Most hát édes leányom, ne félj! Mindent, a mit mondasz, megteszek néked; mert tudja az én népemnek egész kapuja, hogy derék asszony vagy.
\par 12 És most: bizony igaz, hogy közel rokon vagyok: de van nálamnál még közelebbi rokon is.
\par 13 Ez éjszakán hálj itt; s majd reggel, ha az megvált téged, jó: váltson meg; ha pedig nem akar téged megváltani, akkor én váltalak meg.  Él az Úr! Feküdj itt reggelig.
\par 14 És feküvék az õ lábainál reggelig, és felkele, mielõtt valaki az õ felebarátját megismerheté, mert mondá: Meg ne tudja senki, hogy ez az asszony a szérûre jött.
\par 15 És monda: Add ide a nagy kendõdet, a mely rajtad van, és tartsd. És õ oda tartotta azt. Boáz pedig mért hat mérték árpát, és feladta néki, maga pedig bement a városba.
\par 16 És elméne az õ napaasszonyához, és az monda: Hogy vagy édes leányom? És õ elbeszélt néki mindent, a mit cselekedett vele az a férfi.
\par 17 És monda: Ezt a hat mérték árpát adá nékem, mert monda: Ne menj üresen a te napadhoz.
\par 18 És monda Naómi: Légy veszteg leányom, míg megtudod, hova dõl el a dolog; mert nem nyugszik az a férfiú, míg véghez nem viszi a dolgot még ma.

\chapter{4}

\par 1 Boáz pedig felment a kapuba, és ott leült. És ímé arra ment az a legközelebbi rokon, a kirõl Boáz beszélt vala, és monda néki: Jer csak, ülj ide atyafi. És az oda ment és leüle.
\par 2 Ekkor õ maga mellé vett tíz férfiút a város vénei közül és monda: Üljetek ide! És azok leülének.
\par 3 És õ monda a legközelebbi rokonnak: Azt a darab szántóföldet, mely a mi atyánkfiáé, Eliméleké volt, eladja Naómi, a ki haza jött a Moáb mezejérõl.
\par 4 Én pedig gondoltam, hogy füledbe juttatom, mondván: Vedd meg az itt ülõk elõtt és az én népemnek vénei elõtt. Ha megváltod: váltsd meg; és ha nem váltod meg: jelentsd ki elõttem, hogy tudjam; mert rajtad kívül nincs, a ki megváltaná, és én vagyok utánad. És az monda: Én megváltom.
\par 5 És monda Boáz: A mely napon megveszed a szántóföldet Naómi kezébõl, akkor a Moábita Ruthtól, a megholtnak feleségétõl veszed meg, hogy nevet támaszsz a megholtnak, az õ örökségében.
\par 6 A legközelebbi rokon pedig monda: Nem válthatom meg magamnak, hogy el ne veszessem a magam örökségét; váltsd meg te magadnak az én rokoni részemet, mert én nem válthatom meg.
\par 7 Ez vala pedig a szokás régen Izráelben, a megváltás és cserélés alkalmával, minden dolognak megerõsítésére: A férfi lehúzta az õ saruját és oda adta felebarátjának, és ez volt a bizonyság Izráelben.
\par 8 Monda annakokáért a legközelebbi rokon Boáznak: Vedd meg magadnak! És lehúzta a saruját.
\par 9 És monda Boáz a véneknek és az egész népnek: Ti vagytok tanui ma, hogy megvettem mindent, a mi Eliméleké volt, és mindent, a mi Kiljoné és Mahloné volt, Naómi kezébõl;
\par 10 Sõt a Moábita Ruthot is, Mahlonnak feleségét feleségül vettem, hogy nevet támaszszak a megholtnak az õ örökségében, és ki ne veszszen a megholtnak neve az õ atyjafiai közül és az õ helységének kapujából. Tanuk vagytok ma.
\par 11 És monda az egész nép, mely a kapuban vala és a vének: Tanuk vagyunk! Tegye az Úr az asszonyt, a ki a te házadba megy, olyanná, mint Rákhel és Lea, a kik ketten építették fel Izráel házát, és gyûjts vagyont Efratában és szerezz nevet Bethlehemben.
\par 12 És a te házad legyen, miként a Pérecz háza - a kit Thámár szült Júdának - abból a magzatból, a melyet az Úr adánd néked ettõl az asszonytól.
\par 13 Elvevé annakokáért Boáz Ruthot, és lõn az néki felesége, és beméne hozzá, és megadá az Úr, hogy az fogana az õ méhében, és szült fiút.
\par 14 És mondák az asszonyok Naóminak: Áldott az Úr, a ki ma nem engedte meg, hogy rokon nélkül maradj; emlegessék az õ nevét Izráelben!
\par 15 És legyen õ a te lelkednek megvidámítója, és vénségednek istápolója, mert menyed szülte õt, az, a ki téged szeret, és a ki többet ér néked hét fiúnál.
\par 16 Ekkor megfogá Naómi a gyermeket, és ölébe vevé, és dajkája lõn annak.
\par 17 A szomszédasszonyok pedig nevet adának néki, mondván: Fia született Naóminak, és nevezték az õ nevét Obednek. Ez az apja Isainak, a Dávid atyjának.
\par 18 És a Pérecz nemzetségei ezek; Pérecz nemzé Hesront;
\par 19 Hesron pedig nemzé Rámot, és Rám nemzé Amminádábot;
\par 20 Amminádáb pedig nemzé Naássont, és Naásson nemzé Sálmónt;
\par 21 Sálmon pedig nemzé Boázt, és Boáz nemzé Obedet;
\par 22 És Obed nemzé Isait, Isai pedig nemzé Dávidot.


\end{document}