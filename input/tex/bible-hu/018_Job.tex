\begin{document}

\title{Jób könyve}


\chapter{1}

\par 1 Vala Úz földén egy ember, a kinek Jób vala a neve. Ez az ember feddhetetlen, igaz, istenfélõ vala és bûn-gyûlölõ.
\par 2 Születék pedig néki hét fia és három leánya.
\par 3 És vala az õ marhája: hétezer juh, háromezer teve és ötszáz igabarom és ötszáz szamár; cselédje is igen sok vala, és ez a férfiú nagyobb vala keletnek minden fiánál.
\par 4 Eljártak vala pedig az õ fiai egymáshoz és vendégséget szerzének otthon, kiki a maga napján. Elküldtek és meghívták vala az õ három hugokat is, hogy együtt egyenek és igyanak velök.
\par 5 Mikor pedig a vendégség napjai sorra lejártak vala, elkülde értök Jób és megszentelé õket, és jóreggel felserkene és áldozik vala égõáldozattal mindnyájuk száma szerint; mert ezt mondja vala Jób: Hátha vétkeztek az én fiaim és gonoszt gondoltak az Isten ellen az õ szivökben! Így cselekedik vala Jób minden napon.
\par 6 Lõn pedig egy napon hogy eljövének az Istennek fiai, hogy udvaroljanak az Úr elõtt; és eljöve a Sátán is közöttök.
\par 7 És monda az Úr a Sátánnak: Honnét jösz? És felele a Sátán az Úrnak és monda: Körülkerültem és át meg át jártam a földet.
\par 8 És monda az Úr a Sátánnak: Észrevetted-é az én szolgámat, Jóbot? Bizony nincs hozzá hasonló a földön: feddhetetlen, igaz, istenfélõ, és bûngyûlölõ.
\par 9 Felele pedig az Úrnak a Sátán, és monda: Avagy ok nélkül féli-é Jób az Istent?
\par 10 Nem te vetted-é körül õt magát, házát és mindenét, a mije van? Keze munkáját megáldottad, marhája igen elszaporodott e földön.
\par 11 De bocsássad csak rá a te kezedet, verd meg mindazt, a mi az övé, avagy nem átkoz-é meg szemtõl-szembe téged?!
\par 12 Az Úr pedig monda a Sátánnak: Ímé, mindazt, a mije van, kezedbe adom; csak õ magára ne nyujtsd ki kezedet. És kiméne a Sátán az Úr elõl.
\par 13 Lõn pedig egy napon, hogy az õ fiai és leányai esznek és bort isznak vala az õ elsõszülött bátyjoknak házában.
\par 14 És követ jöve Jóbhoz, és monda: Az ökrök szántanak, a szamarak pedig mellettök legelésznek vala.
\par 15 De a Sabeusok rajtok ütének és elhajták azokat, a szolgákat pedig fegyverrel ölék meg. Csak én magam szaladék el, hogy hírt adjak néked.
\par 16 Még szól vala ez, mikor jöve egy másik, és monda: Istennek tüze esék le az égbõl, és megégeté a juhokat és a szolgákat, és megemészté õket. Csak én magam szaladék el, hogy hírt adjak néked.
\par 17 Még ez is szól vala, mikor jöve egy másik, és monda: A Káldeusok három csapatot alkotának és ráütének a tevékre, és elhajták azokat; a szolgákat pedig fegyverrel ölék meg, csak én magam szaladék el, hogy hírt adjak néked.
\par 18 Még ez is szól vala, mikor jöve egy másik, és monda: A te fiaid és leányaid esznek és bort isznak vala az õ elsõszülött bátyjoknak házában;
\par 19 És ímé nagy szél támada a puszta felõl, megrendíté a ház négy szegeletét, és rászakada az a gyermekekre és meghalának. Csak én magam szaladék el, hogy hírt adjak néked.
\par 20 Akkor felkele Jób, megszaggatá köntösét, megberetválá a fejét, és a földre esék és leborula.
\par 21 És monda: Mezítelen jöttem ki az én anyámnak méhébõl, és mezítelen térek oda,  vissza. Az Úr adta, az Úr vette el. Áldott legyen az Úrnak neve!
\par 22 Mindezekben nem vétkezék Jób, és Isten ellen semmi illetlent nem cselekedék.

\chapter{2}

\par 1 Lõn pedig, hogy egy napon eljövének az Istennek fiai, hogy udvaroljanak az Úr elõtt. Eljöve a Sátán is közöttök, hogy udvaroljon az Úr elõtt.
\par 2 És monda az Úr a Sátánnak: Honnét jösz? És felele a Sátán az Úrnak, és monda: Körülkerültem és át meg átjártam a földet.
\par 3 Monda pedig az Úr a Sátánnak: Észrevetted-é az én szolgámat, Jóbot? Bizony nincs a földön olyan, mint õ; feddhetetlen, igaz, istenfélõ, bûngyûlölõ. Még erõsen áll ay õ feddhetetlenségében, noha ellene ingereltél, hogy ok nélkül rontsam meg õt.
\par 4 És felele a Sátán az Úrnak, és monda: Bõrt bõrért; de mindent a mije van, odaad az ember az életéért.
\par 5 Azért bocsásd ki csak a te kezedet, és verd meg õt csontjában és testében: avagy nem átkoz-é meg szemtõl-szembe téged?
\par 6 Monda pedig az Úr a Sátánnak: Ímé kezedbe van õ, csak életét kiméld.
\par 7 És kiméne a Sátán az Úr elõl, és megveré Jóbot undok fekélylyel talpától fogva a feje tetejéig.
\par 8 És võn egy cserepet, hogy azzal vakarja magát, és így ül vala a hamu közepett.
\par 9 Monda pedig õ néki az õ felesége: Erõsen állasz-é még mindig a te feddhetetlenségedben? Átkozd meg az Istent, és halj meg!
\par 10 Õ pedig monda néki: Úgy szólsz, mint szól egy a bolondok közül. Ha már a jót elvettük Istentõl, a rosszat nem vennõk-é el? Mindezekben sem vétkezék Jób az õ ajkaival.
\par 11 Mikor pedig meghallá Jóbnak három barátja mind ezt a nyomorúságot, a mely esett vala rajta: eljöve mindenik az õ lakó helyébõl: a témáni Elifáz, a sukhi Bildád és a naamai Czófár; és elvégezék, hogy együtt mennek be, hogy bánkódjanak vele és vigasztalják õt.
\par 12 És a mint ráveték szemöket távolról, nem ismerék meg õt, és fenhangon zokognak vala; azután pedig megszaggatá kiki a maga köntösét, és port hintének fejökre ég felé.
\par 13 És ülének vele hét napon és hét éjszakán a földön, és nem szóla egyetlen szót egyik sem, mert látják vala, hogy igen nagy az õ fájdalma.

\chapter{3}

\par 1 Ezután megnyitá Jób az õ száját, és megátkozá az õ napját.
\par 2 És szóla Jób, és monda:
\par 3 Veszszen el az a nap, a melyen születtem, és az az éjszaka, a melyen azt mondták: fiú fogantatott.
\par 4 Az a nap legyen sötétség, ne törõdjék azzal az Isten onnét felül, és világosság ne fényljék azon.
\par 5 Tartsa azt fogva sötétség és a halál árnyéka; a felhõ lakozzék rajta, nappali borulatok tegyék rettenetessé.
\par 6 Az az éjszaka! Sûrû sötétség fogja be azt; ne soroztassék az az esztendõnek napjaihoz, ne számláltassék a hónapokhoz.
\par 7 Az az éjszaka! Legyen az magtalan, ne legyen örvendezés azon.
\par 8 Átkozzák meg azt, a kik a nappalt átkozzák, a kik bátrak felingerelni a leviathánt.
\par 9 Sötétüljenek el az õ estvéjének csillagai; várja a világosságot, de az ne legyen, és ne lássa a hajnalnak pirját!
\par 10 Mert nem zárta be az én anyám méhének ajtait, és nem rejtette el szemeim elõl a nyomorúságot.
\par 11 Mért is nem haltam meg fogantatásomkor; mért is ki nem multam, mihelyt megszülettem?
\par 12 Mért vettek fel engem térdre, és mért az emlõkre, hogy szopjam?!
\par 13 Mert most feküdném és nyugodnám, aludnám és akkor nyugton pihenhetnék -
\par 14 Királyokkal és az ország tanácsosaival, a kik magoknak kõhalmokat építenek.
\par 15 Vagy fejedelmekkel, a kiknek aranyuk van, a kik ezüsttel töltik meg házaikat.
\par 16 Vagy mért nem lettem olyan, mint az elásott, idétlen gyermek, mint a világosságot sem látott kisdedek?
\par 17 Ott a gonoszok megszünnek a fenyegetéstõl, és ott megnyugosznak, a kiknek erejök ellankadt.
\par 18 A foglyok ott mind megnyugosznak, nem hallják a szorongatónak szavát.
\par 19 Kicsiny és nagy ott egyenlõ, és a szolga az õ urától szabad.
\par 20 Mért is ad Isten a nyomorultnak világosságot, és életet a keseredett szivûeknek?
\par 21 A kik a halált várják, de nem jön az, és szorgalmasabban keresik mint az elrejtett kincset.
\par 22 A kik nagy örömmel örvendeznek, vigadnak, mikor megtalálják a koporsót.
\par 23 A férfiúnak, a ki útvesztõbe jutott, és a kit az Isten bekerített köröskörül.
\par 24 Mert kenyerem gyanánt van az én fohászkodásom, és sóhajtásaim ömölnek, mint habok.
\par 25 Mert a mitõl remegve remegtem, az jöve reám, és a mitõl rettegtem, az esék rajtam.
\par 26 Nincs békességem, sem nyugtom, sem pihenésem, mert nyomorúság támadt reám.

\chapter{4}

\par 1 És felele a témáni Elifáz, és monda:
\par 2 Ha szólni próbálunk hozzád, zokon veszed-é? De hát ki bírná türtõztetni magát a beszédben?
\par 3 Ímé sokakat oktattál, és a megfáradott kezeket megerõsítetted;
\par 4 A tántorgót a te beszédeid fentartották, és a reszketõ térdeket megerõsítetted;
\par 5 Most, hogy rád jött a sor, zokon veszed; hogy téged ért a baj, elrettensz!
\par 6 Nem bizodalmad-é a te istenfélelmed, s nem reménységed-é utaidnak becsületessége?
\par 7 Emlékezzél, kérlek, ki az, a ki elveszett ártatlanul, és hol töröltettek el az igazak?
\par 8 A mint én láttam, a kik hamisságot szántanak és gonoszságot vetnek, ugyanazt aratnak.
\par 9 Az Istennek lehelletétõl elvesznek, az õ haragjának szelétõl elpusztulnak.
\par 10 Az oroszlán ordítása, a sakál üvöltése, és az oroszlán-kölykök fogai megsemmisülnek.
\par 11 Az agg oroszlán elvész, ha nincs martaléka, a nõstény oroszlán kölykei elszélednek.
\par 12 Szó lopódzék hozzám, s valami nesz üté meg abból fülemet.
\par 13 Éjjeli látásokon való töprengések között, mikor mély álom fogja el az embereket.
\par 14 Félelem szálla rám, és rettegés, s megreszketteté minden csontomat.
\par 15 Valami szellem suhant el elõttem, s testemnek szõre felborzolódék.
\par 16 Megálla, de ábrázatját föl nem ismerém, egy alak vala szemeim elõtt, mély csend, és ilyen szót hallék:
\par 17 Vajjon a halandó igaz-é Istennél: az õ teremtõje elõtt tiszta-é az ember?
\par 18 Ímé az õ szolgáiban sem bízhatik és az õ angyalaiban is talál hibát:
\par 19 Mennyivel inkább a sárházak lakosaiban, a kiknek fundamentumok a porban van, és könnyebben szétnyomhatók a molynál?!
\par 20 Reggeltõl estig gyötrõdnek, s a nélkül, hogy észrevennék, elvesznek örökre.
\par 21 Ha kiszakíttatik belõlök sátoruk kötele, nem halnak-é meg, és pedig bölcsesség nélkül?

\chapter{5}

\par 1 Kiálts csak! Van-é, a ki felelne néked? A szentek közül melyikhez fordulsz?
\par 2 Mert a bolondot boszúság öli meg, az együgyût pedig buzgóság veszti el.
\par 3 Láttam, hogy egy bolond gyökerezni kezdett, de nagy hamar megátkoztam szép hajlékát.
\par 4 Fiai messze estek a szabadulástól: a kapuban megrontatnak, mert nincs, a ki kimentse õket.
\par 5 A mit learatnak néki, az éhezõ eszi meg, a töviskerítésbõl is elviszi azt, kincseiket tõrvetõk nyelik el.
\par 6 Mert nem porból támad a veszedelem s nem földbõl sarjad a nyomorúság!
\par 7 Hanem nyomorúságra születik az ember, a mint felfelé szállnak a parázs szikrái.
\par 8 Azért én a Mindenhatóhoz folyamodnám, az Istenre bíznám ügyemet.
\par 9 A ki nagy, végére mehetetlen dolgokat mûvel, és csudákat, a miknek száma nincsen.
\par 10 A ki esõt ad a földnek színére, és a mezõkre vizet bocsát.
\par 11 Hogy az alázatosokat felmagasztalja, és a gyászolókat szabadulással vidámítsa.
\par 12 A ki semmivé teszi a csalárdok gondolatait, hogy szándékukat kezeik véghez ne vihessék.
\par 13 A ki megfogja a bölcseket az õ csalárdságukban, és a hamisak tanácsát hiábavalóvá teszi.
\par 14 Nappal sötétségre bukkannak, és délben is tapogatva járnak, mint éjszaka.
\par 15 A ki megszabadítja a fegyvertõl, az õ szájoktól, és az erõsnek kezébõl a szegényt;
\par 16 Hogy legyen reménysége a szegénynek, és a hamisság befogja az õ száját.
\par 17 Ímé, boldog ember az, a kit Isten megdorgál; azért a Mindenhatónak büntetését meg ne utáljad!
\par 18 Mert õ megsebez, de be is kötöz, összezúz, de kezei meg is gyógyítanak.
\par 19 Hat bajodból megszabadít, és a hetedikben sem illet a veszedelem téged.
\par 20 Az éhínségben megment téged a haláltól, és a háborúban a fegyveres kezektõl.
\par 21 A nyelvek ostora elõl rejtve leszel, és nem kell félned, hogy a pusztulás rád következik.
\par 22 A pusztulást és drágaságot neveted, és a fenevadaktól  sem félsz.
\par 23 Mert a mezõn való kövekkel is frigyed lesz, és a mezei vad is békességben lesz veled.
\par 24 Majd megtudod, hogy békességben lesz a te sátorod, s ha megvizsgálod a te hajlékodat, nem találsz benne hiányt.
\par 25 Majd megtudod, hogy a te magod megszaporodik, és a te sarjadékod, mint a mezõn a fû.
\par 26 Érett korban térsz a koporsóba, a mint a maga idején takaríttatik be a learatott gabona.
\par 27 Ímé ezt kutattuk mi ki, így van ez. Hallgass erre, jegyezd meg magadnak.

\chapter{6}

\par 1 Jób pedig felele, és monda:
\par 2 Oh, ha az én bosszankodásomat mérlegre vetnék, és az én nyomorúságomat vele együtt tennék a fontba!
\par 3 Bizony súlyosabb ez a tenger fövenyénél; azért balgatagok az én szavaim.
\par 4 Mert a Mindenható nyilai vannak én bennem, a melyeknek mérge emészti az én lelkemet, és az Istennek rettenései ostromolnak  engem.
\par 5 Ordít-é a vadszamár a zöld füvön, avagy bõg-é az ökör az õ abrakja mellett?
\par 6 Vajjon ízetlen, sótalan étket eszik-é az ember; avagy kellemes íze van-é a tojásfehérnek?
\par 7 Lelkem iszonyodik érinteni is; olyanok azok nékem, mint a megromlott kenyér!
\par 8 Oh, ha az én kérésem teljesülne, és az Isten megadná, amit reménylek;
\par 9 És tetszenék Istennek, hogy összetörjön engem, megoldaná kezét, hogy szétvagdaljon engem!
\par 10 Még akkor lenne valami vigasztalásom; újjonganék a fájdalomban, a mely nem kimél, mert nem tagadtam meg a Szentnek beszédét.
\par 11 Micsoda az én erõm, hogy várakozzam; mi az én végem, hogy türtõztessem magam?!
\par 12 Kövek ereje-é az én erõm, avagy az én testem aczélból van-é?
\par 13 Hát nincsen-é segítség számomra; avagy a szabadulás elfutott-é tõlem?!
\par 14 A szerencsétlent barátjától részvét illeti meg, még ha elhagyja is a Mindenhatónak félelmét.
\par 15 Atyámfiai hûtlenül elhagytak mint a patak, a mint túláradnak medrükön a patakok.
\par 16 A melyek szennyesek a jégtõl, a melyekben olvadt hó hömpölyög;
\par 17 Mikor átmelegülnek, elapadnak, a hõség miatt fenékig száradnak.
\par 18 Letérnek útjokról a vándorok; felmennek a sivatagba utánok és elvesznek.
\par 19 Nézegetnek utánok Téma vándorai; Sébának utasai bennök reménykednek.
\par 20 Megszégyenlik, hogy bíztak, közel mennek és elpirulnak.
\par 21 Így lettetek ti most semmivé; látjátok a nyomort és féltek.
\par 22 Hát mondtam-é: adjatok nékem valamit, és a ti jószágotokból ajándékozzatok meg engem?
\par 23 Szabadítsatok ki engem az ellenség kezébõl, és a hatalmasok kezébõl vegyetek ki engem?
\par 24 Tanítsatok meg és én elnémulok, s a miben tévedek, értessétek meg velem.
\par 25 Oh, mily hathatósak az igaz beszédek! De mit ostoroz a ti ostorozásotok?
\par 26 Szavak ostorozására készültök-é? Hiszen a szélnek valók a kétségbeesettnek szavai!
\par 27 Még az árvának is néki esnétek, és sírt ásnátok a ti barátotoknak is?!
\par 28 Most hát tessék néktek rám tekintenetek, és szemetekbe csak nem hazudom?
\par 29 Kezdjétek újra kérlek, ne legyen hamisság. Kezdjétek újra, az én igazságom még mindig áll.
\par 30 Van-é az én nyelvemen hamisság, avagy az én ínyem nem veheti-é észre a nyomorúságot?

\chapter{7}

\par 1 Nem rabszolga élete van-é az embernek a földön, és az õ napjai nem olyanok-é,  mint a béresnek napjai?
\par 2 A mint a szolga kívánja az árnyékot, és a mint a béres reményli az õ bérét:
\par 3 Úgy részesültem én keserves hónapokban, és nyomorúságnak éjszakái jutottak számomra.
\par 4 Ha lefekszem, azt mondom: mikor kelek föl? de hosszú az estve, és betelek a hánykolódással reggeli szürkületig.
\par 5 Testem férgekkel van fedve és a pornak piszokjával; bõröm összehúzódik és meggennyed.
\par 6 Napjaim gyorsabbak voltak a vetélõnél, és most reménység nélkül tünnek el.
\par 7 Emlékezzél meg, hogy az én életem csak egy lehellet, és az én szemem nem lát többé jót.
\par 8 Nem lát engem szem, a mely rám néz; te rám veted szemed, de már nem vagyok!
\par 9 A felhõ eltünik és elmegy, így a ki leszáll a sírba, nem jõ fel többé.
\par 10 Nem tér vissza többé az õ hajlékába, és az õ helye nem ismeri õt többé.
\par 11 Én sem tartóztatom hát meg az én számat; szólok az én lelkemnek fájdalmában, és panaszkodom az én szívemnek keserûségében.
\par 12 Tenger vagyok-é én, avagy czethal, hogy õrt állítasz ellenem?
\par 13 Mikor azt gondolom, megvigasztal engem az én nyoszolyám, megkönnyebbíti panaszolkodásomat az én ágyasházam:
\par 14 Akkor álmokkal rettentesz meg engem és látásokkal háborítasz meg engem;
\par 15 Úgy, hogy inkább választja lelkem a megfojtatást, inkább a halált, mint csontjaimat.
\par 16 Utálom! Nem akarok örökké élni. Távozzál el tõlem, mert nyomorúság az én  életem.
\par 17 Micsoda az ember, hogy õt ily nagyra becsülöd, és hogy figyelmedet fordítod reá?
\par 18 Meglátogatod õt minden reggel, és minden szempillantásban próbálod õt.
\par 19 Míglen nem fordítod el tõlem szemedet, nem távozol csak addig is tõlem, a míg nyálamat lenyelem?
\par 20 Vétkeztem! Mit cselekedjem én néked, oh embereknek õrizõje? Mért tettél ki czéltáblául magadnak? Mért legyek magamnak is terhére.
\par 21 És mért nem bocsátod meg vétkemet és nem törlöd el az én bûnömet? Hiszen immár a porban fekszem, és ha keresel engem, nem leszek.

\chapter{8}

\par 1 Akkor felele a sukhi Bildád, és monda:
\par 2 Meddig szólasz még efféléket, és lesz a te szádnak beszéde sebes szél?
\par 3 Elforgatja-é Isten az ítéletet, avagy a Mindenható elforgatja-é az igazságot?
\par 4 Ha a te fiaid vétkeztek ellene, úgy az õ gonoszságuk miatt vetette el õket.
\par 5 De ha te az Istent buzgón keresed, és a Mindenhatóhoz bocsánatért könyörögsz;
\par 6 Ha tiszta és becsületes vagy, akkor legott felserken éretted, és békességessé teszi a te igazságodnak hajlékát.
\par 7 És ha elõbbi állapotod szegényes volt, ez utáni állapotod boldog lesz nagyon.
\par 8 Mert kérdezd meg csak az azelõtti nemzedéket, és készülj csak fel az õ atyáikról való tudakozódásra!
\par 9 Mert mi csak tegnapiak vagyunk és semmit nem tudunk, mert a mi napjaink csak árnyék e földön.
\par 10 Nem tanítanak-é meg azok téged? Nem mondják-é meg néked, és nem beszélik-é meg szívök szerint néked?!
\par 11 Felnövekedik-é a káka mocsár nélkül, felnyúlik-é a sás víz nélkül?
\par 12 Még gyenge korában, ha fel nem szakasztják is, minden fûnél elébb elszárad.
\par 13 Ilyenek az ösvényeik mindazoknak, a kik Istenrõl elfeledkeznek, és a képmutatónak reménysége is elvész.
\par 14 Mivel szétfoszol bizakodása, és bizodalma olyan lesz, mint a pókháló.
\par 15 Házára támaszkodik, és nem áll meg; kapaszkodik belé, és nem marad meg.
\par 16 Bõ nedvességû ez a napfényen is, és ágazata túlnõ a kertjén.
\par 17 Gyökerei átfonódnak a kõhalmon; átfúródnak a szikla-rétegen.
\par 18 Ámha kiirtják helyérõl, megtagadja ez õt: Nem láttalak!
\par 19 Ímé ez az õ pályájának öröme! És más hajt ki a porból.
\par 20 Ímé az Isten nem veti meg az ártatlant, de a gonoszoknak sem ád elõmenetelt.
\par 21 Még betölti szádat nevetéssel, és ajakidat vigassággal.
\par 22 Gyûlölõid szégyenbe öltöznek, és a gonoszok sátora megsemmisül.

\chapter{9}

\par 1 Felele pedig Jób, és monda:
\par 2 Igaz, jól tudom, hogy így van; hogyan is lehetne igaz a halandó ember Istennél?
\par 3 Ha perelni akarna õ vele, ezer közül egy sem felelhetne meg néki.
\par 4 Bölcs szívû és hatalmas erejû: ki szegülhetne ellene, hogy épségben maradjon?
\par 5 A ki hegyeket mozdít tova, hogy észre se veszik, és megfordítja õket haragjában.
\par 6 A ki kirengeti helyébõl a földet, úgy hogy oszlopai megrepedeznek.
\par 7 A ki szól a napnak és az fel nem kél, és bepecsételi a csillagokat.
\par 8 A ki egymaga feszítette ki az egeket, és a tenger hullámain tapos.
\par 9 A ki teremtette a gönczölszekeret, a kaszás csillagot és a fiastyúkot és a délnek titkos tárait.
\par 10 A ki nagy dolgokat cselekszik megfoghatatlanul, és csudákat megszámlálhatatlanul.
\par 11 Ímé, elvonul mellettem, de nem látom, átmegy elõttem, de nem veszem észre.
\par 12 Ímé, ha elragad valamit, ki akadályozza meg; ki mondhatja néki: Mit cselekszel?
\par 13 Ha az Isten el nem fordítja az õ haragját, alatta meghajolnak Ráháb czinkosai is.
\par 14 Hogyan felelhetnék hát én meg õ néki, és lelhetnék vele szemben szavakat?
\par 15 A ki, ha szinte igazam volna, sem felelhetnék néki; kegyelemért könyörögnék ítélõ birámhoz.
\par 16 Ha segítségül hívnám és felelne is nékem, még sem hinném, hogy szavamat fülébe vevé;
\par 17 A ki forgószélben rohan meg engem, és ok nélkül megsokasítja sebeimet.
\par 18 Nem hagyna még lélekzetet se vennem, hanem keserûséggel lakatna jól.
\par 19 Ha erõre kerülne a dolog? Ímé, õ igen erõs; és ha ítéletre? Ki tûzne ki én nékem napot?
\par 20 Ha igaznak mondanám magamat, a szájam kárhoztatna engem; ha ártatlannak: bûnössé tenne engemet.
\par 21 Ártatlan vagyok, nem törõdöm lelkemmel, útálom az életemet.
\par 22 Mindegy ez! Azért azt mondom: elveszít õ ártatlant és gonoszt!
\par 23 Ha ostorával hirtelen megöl, neveti a bûntelenek megpróbáltatását.
\par 24 A föld a gonosz kezébe adatik, a ki az õ biráinak arczát elfedezi. Nem így van? Kicsoda hát õ?
\par 25 Napjaim gyorsabbak valának a kengyelfutónál: elfutának, nem láttak semmi jót.
\par 26 Ellebbentek, mint a gyorsan járó hajók, miként zsákmányára csap a keselyû.
\par 27 Ha azt mondom: Nosza elfelejtem panaszomat, felhagyok haragoskodásommal és vidám leszek:
\par 28 Megborzadok az én mindenféle fájdalmamtól; tudom, hogy nem találsz bûntelennek engem.
\par 29 Rossz ember vagyok én! Minek fáraszszam hát magamat hiába?
\par 30 Ha hóvízzel mosakodom is meg, ha szappannal mosom is meg kezeimet:
\par 31 Akkor is a posványba mártanál engem és az én ruháim is útálnának engem.
\par 32 Mert nem ember õ, mint én, hogy néki megfelelhetnék, és együtt pörbe állanánk.
\par 33 Nincs is közöttünk igazlátó, a ki kezét közbe vethesse kettõnk között!
\par 34 Venné csak el rólam az õ veszszejét, és az õ rettentésével ne rettegtetne engem:
\par 35 Akkor szólanék és nem félnék tõle: mert nem így vagyok én magammal!

\chapter{10}

\par 1 Lelkembõl útálom az életemet, megeresztem felõle panaszomat; szólok az én lelkem keserûségében.
\par 2 Azt mondom az Istennek: Ne kárhoztass engem; add tudtomra, miért perlesz velem?!
\par 3 Jó-é az néked, hogy nyomorgatsz, hogy megútálod kezednek munkáját, és a gonoszok tanácsát támogatod?
\par 4 Testi szemeid vannak-é néked, és úgy látsz-é te, a mint halandó lát?
\par 5 Mint a halandónak napjai, olyanok-é a te napjaid, avagy a te éveid, mint az embernek napjai?
\par 6 Hogy az én álnokságomról tudakozol, és az én vétkem után kutatsz.
\par 7 Jól tudod te azt, hogy én nem vagyok gonosz, még sincs, a ki kezedbõl kiszabadítson!
\par 8 Kezeid formáltak engem és készítének engem egészen köröskörül, és mégis megrontasz engem?!
\par 9 Emlékezzél, kérlek, hogy mint valami agyagedényt, úgy készítettél engem, és ismét porrá tennél engem?
\par 10 Nem úgy öntél-é engem, mint a tejet és mint a sajtot, megoltottál engem?
\par 11 Bõrrel és hússal ruháztál fel engem, csontokkal és inakkal befedeztél engem.
\par 12 Életet és kegyelmet szerzettél számomra, és a te gondviselésed õrizte az én lelkemet.
\par 13 De ezeket elrejtetted a te szívedben, és tudom, hogy ezt tökélted el magadban:
\par 14 Ha vétkeztem, mindjárt észreveszed rajtam, és bûnöm alól nem mentesz föl engem.
\par 15 Ha istentelen vagyok, jaj nékem; ha igaz vagyok, sem emelem föl fejemet, eltelve gyalázattal, de tekints nyomorúságomra!
\par 16 Ha pedig felemelkednék az, mint oroszlán kergetnél engem, és ismét csudafájdalmakat bocsátanál reám.
\par 17 Megújítanád a te bizonyságidat ellenem, megöregbítenéd a te boszúállásodat rajtam; váltakozó és állandó sereg volna ellenem.
\par 18 Miért is hoztál ki engem anyámnak méhébõl? Vajha meghaltam volna, és szem nem látott volna engem!
\par 19 Lettem volna, mintha nem is voltam volna; anyámnak méhébõl sírba vittek volna!
\par 20 Hiszen kevés napom van még; szünjék meg! Forduljon el tõlem, hadd viduljak fel egy kevéssé,
\par 21 Mielõtt oda megyek, honnét nem térhetek vissza: a sötétségnek és a halál árnyékának földébe;
\par 22 Az éjféli homálynak földébe, a mely olyan, mint a halál árnyékának sürû setétsége; hol nincs rend, és a világosság olyan, mint a sürû setétség.

\chapter{11}

\par 1 Felele a Naamából való Czófár, és monda:
\par 2 A sok beszédre ne legyen-é felelet? Avagy a csácsogó embernek legyen-é igaza?
\par 3 Fecsegéseid elnémítják az embereket, és csúfolódol is és ne legyen, a ki megszégyenítsen?!
\par 4 Azt mondod: Értelmes az én beszédem, tiszta vagyok a te szemeid elõtt.
\par 5 De vajha szólalna meg maga az Isten, és nyitná meg ajkait te ellened!
\par 6 És jelentené meg néked a bölcsességnek titkait, hogy kétszerte többet ér az az okoskodásnál, és tudnád meg, hogy az Isten még el is engedett néked a te bûneidbõl.
\par 7 Az Isten mélységét elérheted-é, avagy a Mindenhatónak tökéletességére eljuthatsz-é?
\par 8 Magasabb az égnél: mit teszel tehát? Mélyebb az alvilágnál; hogy ismerheted meg?
\par 9 Hosszabb annak mértéke a földnél, és szélesebb a tengernél.
\par 10 Ha megtapos, elzár és ítéletet tart: ki akadályozhatja meg?
\par 11 Mert õ jól ismeri a csalárd embereket, látja az álnokságot, még ha nem figyelmez is arra!
\par 12 És értelmessé teheti a bolond embert is, és emberré szülheti a vadszamár csikóját is.
\par 13 Ha te a te szívedet felkészítenéd, és kezedet felé terjesztenéd;
\par 14 Ha a hamisságot, a mely a te kezedben van, távol tartanád magadtól, és nem lakoznék a te hajlékodban gonoszság;
\par 15 Akkor a te arczodat fölemelhetnéd szégyen nélkül, erõs lennél és nem félnél;
\par 16 Sõt a nyomorúságról is elfelejtkeznél, és mint lefutott vizekrõl, úgy emlékeznél arról.
\par 17 Ragyogóbban kelne idõd a déli fénynél, és az éjféli sötétség is olyan lenne, mint a kora reggel.
\par 18 Akkor bíznál, mert volna reménységed; és ha széttekintenél, biztonságban aludnál.
\par 19 Ha lefeküdnél, senki föl nem rettentene, sõt sokan hizelegnének néked.
\par 20 De a gonoszok szemei elepednek, menedékök eltünik elõlök, és reménységök: a lélek kilehellése!

\chapter{12}

\par 1 Felele erre Jób, és monda:
\par 2 Bizonyára ti magatok vagytok a nép, és veletek kihal a bölcseség!
\par 3 Nékem is van annyi eszem, mint néktek, és nem vagyok alábbvaló nálatok, és ki ne tudna ilyenféléket?
\par 4 Kikaczagják a saját barátai azt, mint engem, a ki Istenhez kiált és meghallgatja õt. Kikaczagják az igazat, az ártatlant!
\par 5 A szerencsétlen megvetni való, gondolja, a ki boldog; ez vár azokra, a kiknek lábok roskadoz.
\par 6 A kóborlók sátrai csendesek és bátorságban vannak, a kik ingerlik az Istent, és a ki kezében hordja Istenét.
\par 7 Egyébiránt kérdezd meg csak a barmokat, majd megtanítanak, és az égnek madarait, azok megmondják néked.
\par 8 Avagy beszélj a földdel és az megtanít téged, a tengernek halai is elbeszélik néked.
\par 9 Mindezek közül melyik nem tudja, hogy az Úrnak keze cselekszi ezt?
\par 10 A kinek kezében van minden élõ állatnak élete, és minden egyes embernek a lelke.
\par 11 Nemde nem a fül próbálja-é meg a szót, és az íny kóstolja meg az ételt?
\par 12 A vén emberekben van-é a bölcseség, és az értelem a hosszú életben-é?
\par 13 Õ nála van a bölcseség és hatalom, övé a tanács és az értelem.
\par 14 Ímé, a mit leront, nem az épül föl az; ha valakire rázárja az ajtót, nem nyílik föl az.
\par 15 Ímé, ha a vizeket elfogja, kiszáradnak; ha kibocsátja õket, felforgatják a földet.
\par 16 Õ nála van az erõ és okosság; övé az eltévelyedett és a ki tévelygésre visz.
\par 17 A tanácsadókat fogságra viszi, és a birákat megbolondítja.
\par 18 A királyok bilincseit feloldja, és övet köt derekukra.
\par 19 A papokat fogságra viszi, és a hatalmasokat megbuktatja.
\par 20 Az ékesen szólótól eltávolítja a beszédet és a vénektõl elveszi a tanácsot.
\par 21 Szégyent zúdít az elõkelõkre, és a hatalmasok övét megtágítja.
\par 22 Feltárja a sötétségbõl a mélységes titkokat, és a halálnak árnyékát is világosságra hozza.
\par 23 Nemzeteket növel fel, azután elveszíti õket; nemzeteket terjeszt ki messzire, azután elûzi õket.
\par 24 Elveszi eszöket a föld népe vezetõinek, és úttalan pusztában bujdostatja õket.
\par 25 És világtalan setétben tapogatóznak, és tántorognak, mint a részeg.

\chapter{13}

\par 1 Ímé, mindezeket, látta az én szemem, hallotta az én fülem és megértette.
\par 2 A mint ti tudjátok, úgy tudom én is, és nem vagyok alábbvaló nálatok.
\par 3 Azonban én a Mindenhatóval akarok szólani; Isten elõtt kivánom védeni ügyemet.
\par 4 Mert ti hazugságnak mesterei vagytok, és mindnyájan haszontalan orvosok.
\par 5 Vajha legalább mélyen hallgatnátok, az még bölcseségtekre lenne.
\par 6 Halljátok meg, kérlek, az én feddõzésemet, és figyeljetek az én számnak pörlekedéseire.
\par 7 Az Isten kedvéért szóltok-é hamisságot, és õ érette szóltok-é csalárdságot?
\par 8 Az õ személyére néztek-é, ha Isten mellett tusakodtok?
\par 9 Jó lesz-é az, ha egészen kiismer benneteket, avagy megcsalhatjátok-é õt, a mint megcsalható az ember?
\par 10 Keményen megbüntet, ha titkon vagytok is személyválogatók.
\par 11 Az õ fensége nem rettent-é meg titeket, a tõle való félelem nem száll-é rátok?
\par 12 A ti emlékezéseitek hamuba írott példabeszédek, a ti menedékváraitok sárvárak.
\par 13 Hallgassatok, ne bántsatok: hadd szóljak én, akármi essék is rajtam.
\par 14 Miért szaggatnám fogaimmal testemet, és miért szorítanám markomba lelkemet?
\par 15 Ímé, megöl engem! Nem reménylem; hiszen csak utaimat akarom védeni elõtte!
\par 16 Sõt az lesz nékem segítségül, hogy képmutató nem juthat elébe.
\par 17 Hallgassátok meg figyelmetesen az én beszédemet, vegyétek füleitekbe az én mondásomat.
\par 18 Ímé, elõterjesztem ügyemet, tudom, hogy nékem lesz igazam.
\par 19 Ki az, a ki perelhetne velem? Ha most hallgatnom kellene, úgy kimulnék.
\par 20 Csak kettõt ne cselekedj velem, szined elõl akkor nem rejtõzöm el.
\par 21 Vedd le rólam kezedet, és a te rettentésed ne rettentsen engem.
\par 22 Azután szólíts és én felelek, avagy én szólok hozzád és te válaszolj.
\par 23 Mennyi bûnöm és vétkem van nékem? Gonoszságomat és vétkemet add tudtomra!
\par 24 Mért rejted el arczodat, és tartasz engemet ellenségedül?
\par 25 A letépett falevelet rettegteted-é, és a száraz pozdorját üldözöd-é?
\par 26 Hogy ily sok keserûséget szabtál reám, és az én ifjúságomnak vétkét örökölteted velem?!
\par 27 Hogy békóba teszed lábaimat, vigyázol minden én utamra, és vizsgálod lábomnak nyomait?
\par 28 Az pedig elsenyved, mint a redves fa, mint ruha, a melyet moly emészt.

\chapter{14}

\par 1 Az asszonytól született ember rövid életû és háborúságokkal  bõvelkedõ.
\par 2 Mint a virág, kinyílik és elhervad, és eltünik, mint az árnyék és nem állandó.
\par 3 Még az ilyen ellen is felnyitod-é szemeidet, tennen magaddal törvénybe állítasz-é engem?
\par 4 Ki adhat tisztát a tisztátalanból? Senki.
\par 5 Nincsenek-é meghatározva napjai? Az õ hónapjainak számát te tudod; határt vetettél néki, a melyet nem hághat át.
\par 6 Fordulj el azért tõle, hogy nyugodalma legyen, hogy legyen napjában annyi öröme, mint egy béresnek.
\par 7 Mert a fának van reménysége; ha levágják, ismét kihajt, és az õ hajtásai el nem fogynak.
\par 8 Még ha megaggodik is a földben a gyökere, és ha elhal is a porban törzsöke:
\par 9 A víznek illatától kifakad, ágakat hajt, mint a csemete.
\par 10 De ha a férfi meghal és elterül; ha az ember kimúlik, hol van õ?
\par 11 Mint a víz kiapad a tóból, a patak elapad, kiszárad:
\par 12 Úgy fekszik le az ember és nem kél fel; az egek elmúlásáig sem ébrednek, nem költetnek föl az õ álmukból.
\par 13 Vajha engem a holtak országában tartanál; rejtegetnél engemet addig, a míg elmúlik a te haragod; határt vetnél nékem, azután megemlékeznél rólam!
\par 14 Ha meghal az ember vajjon feltámad-é? Akkor az én hadakozásom minden idejében reménylenék, míglen elkövetkeznék az én elváltozásom.
\par 15 Szólítanál és én felelnék néked, kivánkoznál a te kezednek alkotása után.
\par 16 De most számlálgatod az én lépéseimet, és nem nézed el az én vétkeimet!
\par 17 Gonoszságom egy csomóba van lepecsételve, és hozzáadod bûneimhez.
\par 18 Még a hegy is szétomlik, ha eldõl; a szikla is elmozdul helyérõl;
\par 19 A köveket lekoptatja a víz, a földet elsodorja annak árja: az ember reménységét is úgy teszed semmivé.
\par 20 Hatalmaskodol rajta szüntelen és õ elmegy; megváltoztatván az arczát, úgy bocsátod el õt.
\par 21 Ha tisztesség éri is fiait, nem tudja; ha megszégyenülnek, nem törõdik velök.
\par 22 Csak õmagáért fáj még a teste, és a lelke is õmagáért kesereg.

\chapter{15}

\par 1 Majd felele a Témánból való Elifáz és monda:
\par 2 Vajjon a bölcs felelhet-é ilyen szeles tudománynyal, és megtöltheti-é a hasát keleti széllel?
\par 3 Vetekedvén oly beszéddel, a mely nem használ, és oly szavakkal, a melyekkel semmit sem segít.
\par 4 Te már semmivé akarod tenni az Isten félelmét is; és megkevesbíted az Isten elõtt való buzgólkodást is!
\par 5 Mert gonoszságod oktatja a te szádat, és a csalárdok nyelvét választottad.
\par 6 A te szád kárhoztat téged, nem én, és a te ajkaid bizonyítanak ellened.
\par 7 Te születtél-é az elsõ embernek; elébb formáltattál-é, mint a halmok?
\par 8 Az Isten tanácsában hallgatóztál-é, és a bölcseséget magadhoz ragadtad-é?
\par 9 Mit tudsz te, a mit mi nem tudunk, és mit értesz olyat, a mi nálunk nem volna meg?
\par 10 Õsz is, agg is van közöttünk, jóval idõsebb a te atyádnál.
\par 11 Csekélységek-é elõtted Istennek vigasztalásai, és a beszéd, a mely szeliden bánt veled?
\par 12 Merre ragadt téged a szíved, és merre pillantottak a te szemeid?
\par 13 Hogy Isten ellen fordítod a te haragodat, és ilyen szavakat eresztesz ki a szádon?
\par 14 Micsoda a halandó hogy tiszta lehetne, és hogy igaz volna, a ki asszonytól születik?
\par 15 Ímé, még az õ szenteiben sem bízok, az egek sem tiszták az õ szemében:
\par 16 Mennyivel kevésbbé az útálatos és a megromlott ember, a ki úgy nyeli a hamisságot mint a vizet?!
\par 17 Elmondom néked, hallgass rám, és mint láttam, úgy beszélem el;
\par 18 A mit a bölcsek is hirdettek, és nem titkoltak el, mint atyáiktól valót;
\par 19 A kiknek egyedül adatott vala a föld, és közöttük idegen nem megy vala.
\par 20 Az istentelen kínozza önmagát egész életében, és az erõszakoskodó elõtt is rejtve van az õ esztendeinek száma.
\par 21 A félelem hangja cseng az õ füleiben; a békesség idején tör rá a pusztító!
\par 22 Nem hiszi, hogy kijut a sötétségbõl, mert kard hegyére van õ kiszemelve.
\par 23 Kenyér után futkos, hogy hol volna? Tudja, hogy közel van hozzá a sötétség napja.
\par 24 Háborgatják õt a nyomorúság és rettegés; leverik õt, mint valami háborúra felkészült király.
\par 25 Mert az Isten ellen nyujtotta ki kezét, és erõsködött a Mindenható ellen;
\par 26 Kinyujtott nyakkal rohant ellene, domború pajzsainak fellege alatt.
\par 27 Mivel befedezte az arczát kövérséggel és hájat borított tomporára;
\par 28 És lakozott elrombolt városokban; lakatlan házakban, a melyek dûlõfélben vannak:
\par 29 Meg nem gazdagodik, vagyona meg nem marad, jószága nem lepi el a földet.
\par 30 Nem menekül meg a setétségtõl, sarjadékát láng perzseli el, és szájának lehelletétõl pusztul el.
\par 31 Ne higyjen a hívságnak, a ki megcsalatott, mert hívság lészen annak jutalma.
\par 32 Nem idejében telik el élete, és az ága ki nem virágzik.
\par 33 Lehullatja, mint a szõlõvesszõ az õ egresét, elhányja, mint az olajfa az õ virágát.
\par 34 Mert a képmutató házanépe meddõ, és tûz emészti meg az ajándékból való sátrakat.
\par 35 Nyomorúságot fogan, álnokságot szül, és az õ méhök csalárdságot érlel.

\chapter{16}

\par 1 Felele pedig Jób, és monda:
\par 2 Efféle dolgokat sokat hallottam. Nyomorult vigasztalók vagytok ti mindnyájan!
\par 3 Vége lesz-é már a szeles beszédeknek, avagy mi ingerel téged, hogy így felelsz?
\par 4 Én is szólhatnék úgy mint ti, csak volna a ti lelketek az én lelkem helyén! Szavakat fonhatnék össze ellenetek; csóválhatnám miattatok a fejemet;
\par 5 Erõsíthetnélek titeket csak a szájammal és ajakim mozgása kevesbítené fájdalmatokat.
\par 6 Ha szólnék is, nem kevesbbednék a keserûségem; ha veszteglek is: micsoda távozik el tõlem?
\par 7 Most pedig már fáraszt engemet. Elpusztítád egész házam népét.
\par 8 Hogy összenyomtál engem, ez bizonyság lett; felkelt ellenem az én ösztövérségem is, szemtõl-szembe bizonyít ellenem.
\par 9 Haragja széttépett és üldöz engem. Fogait csikorgatta rám, ellenségemként villogtatja felém tekintetét.
\par 10 Feltátották ellenem szájokat, gyalázatosan arczul csapdostak engem, összecsõdültek ellenem.
\par 11 Adott engem az Isten az álnoknak, és a gonoszok kezébe ejte engemet.
\par 12 Csendességben valék, de szétszaggata engem; nyakszirten ragadott és szétzúzott engem, czéltáblává tûzött ki magának.
\par 13 Körülvettek az õ íjászai; veséimet meghasítja és nem kimél; epémet a földre kiontja.
\par 14 Rést rés után tör rajtam, és rám rohan, mint valami hõs.
\par 15 Zsák-ruhát varrék az én fekélyes bõrömre, és a porba  fúrtam be az én szarvamat.
\par 16 Orczám a sírástól kivörösödött, szempilláimra a halál árnyéka szállt;
\par 17 Noha erõszakosság nem tapad kezemhez, és az én imádságom tiszta.
\par 18 Oh föld, az én véremet el ne takard, és ne legyen hely az én kiáltásom számára!
\par 19 Még most is ímé az égben van az én bizonyságom, és az én tanuim a magasságban!
\par 20 Csúfolóim a saját barátaim, azért az Istenhez sír fel az én szemem,
\par 21 Hogy ítélje meg az embernek Istennel, és az ember fiának az õ felebarátjával való dolgát.
\par 22 Mert a kiszabott esztendõk letelnek, és én útra kelek és nem térek vissza.
\par 23 Lelkem meghanyatlott, napjaim elfogynak, vár rám a sír.

\chapter{17}

\par 1 Még mindig csúfot ûznek belõlem! Szemem az õ patvarkodásuk között virraszt.
\par 2 Kezest magadnál rendelj, kérlek, nékem; különben ki csap velem kezet?
\par 3 Minthogy az õ szívöket elzártad az értelem elõl, azért nem is magasztalhatod fel õket.
\par 4 A ki prédává juttatja barátait, annak fiainak szemei elfogyatkoznak.
\par 5 Példabeszéddé tõn engem a népek elõtt, és ijesztõvé lettem elõttök.
\par 6 A bosszúság miatt szemem elhomályosodik, és minden tagom olyan, mint az árnyék.
\par 7 Elálmélkodnak ezen a becsületesek, és az ártatlan a képmutató ellen támad.
\par 8 Ám az igaz kitart az õ útján, és a tiszta kezû ember még erõsebbé lesz.
\par 9 Nosza hát, térjetek ide mindnyájan; jõjjetek, kérlek, úgy sem találok bölcset köztetek.
\par 10 Napjaim elmulának, szívemnek kincsei: terveim meghiusulának.
\par 11 Az éjszakát nappallá változtatják, és a világosság csakhamar sötétséggé lesz.
\par 12 Ha reménykedem is, a sír már az én házam, a sötétségben vetettem az én ágyamat.
\par 13 A sírnak mondom: Te vagy az én atyám; a férgeknek pedig: Ti vagytok az én anyám és néném.
\par 14 Hol tehát az én reménységem, ki törõdik az én reménységemmel?
\par 15 Leszáll az majd a sír üregébe, velem együtt nyugoszik a porban.

\chapter{18}

\par 1 Felele pedig a sukhi Bildád, és monda:
\par 2 Mikor akartok a beszédnek véget vetni? Értsétek meg a dolgot, azután szóljunk.
\par 3 Miért állíttatunk barmoknak, és miért vagyunk tisztátalanok a ti szemeitekben?
\par 4 Te éretted, a ki szaggatja lelkét haragjában, vajjon elhagyattatik-é a föld, és felszakasztatik-é a kõszikla helyérõl?
\par 5 Sõt inkább a gonoszok világa kialuszik, és nem fénylik az õ tüzöknek szikrája.
\par 6 A világosság elsötétedik az õ sátorában, szövetnéke kialszik felette.
\par 7 Erõs léptei aprókká lesznek, saját tanácsa rontja meg õt.
\par 8 Mert lábaival hálóba bonyolódik, és ó-verem felett jár.
\par 9 A sarka tõrbe akad, és kelepcze fogja meg õt.
\par 10 Hurok rejtetett el a földbe ellene, és zsineg az õ szokott ösvényén.
\par 11 Mindenfelõl félelmek rettentik õt, és üldözik õt léptennyomon.
\par 12 Éhség emészti fel az õ erejét, és nyomorúság leselkedik oldala mellett.
\par 13 Megemészti testének izmait, megemészti izmait a halál zsengéje.
\par 14 Eltünik sátorából az õ bátorsága, és a félelmek királyához folyamodik õ.
\par 15 Az lakik sátorában, a ki nem az övé, és hajlékára kénkövet szórnak.
\par 16 Alant elszáradnak gyökerei, és felülrõl levágatik az ága.
\par 17 Emlékezete elvész a földrõl, még az utczákon sem marad fel a neve.
\par 18 A világosságról a sötétségbe taszítják, a föld kerekségérõl elüldözik õt.
\par 19 Sem fia, sem unokája nem lesz az õ népében, és semmi maradéka az õ tanyáján.
\par 20 Az õ pusztulásától megborzadnak, a kik következnek és rettegés fogja el a most élõ embereket.
\par 21 Ilyenek az álnok embernek hajlékai, és ilyen annak lakóhelye, a ki nem tiszteli Istent.

\chapter{19}

\par 1 Felele pedig, Jób, és monda:
\par 2 Meddig búsítjátok még a lelkemet, és kínoztok engem beszéddel?
\par 3 Tízszer is meggyaláztatok már engem; nem pirultok, hogy így erõsködtök ellenem?
\par 4 Még ha csakugyan tévedtem is, tévedésem énmagamra hárul.
\par 5 Avagy csakugyan pöffeszkedni akartok ellenem, és feddõdni az én gyalázatom felett?
\par 6 Tudjátok meg hát, hogy Isten alázott meg engem, és az õ hálójával õ vett engem körül.
\par 7 Ímé, kiáltozom az erõszak miatt, de meg nem hallgattatom, segélyért kiáltok, de nincsen igazság.
\par 8 Utamat úgy elgátolta, hogy nem mehetek át rajta, és az én ösvényemre sötétséget vetett.
\par 9 Tisztességembõl kivetkõztetett, és fejemnek koronáját elvevé.
\par 10 Megronta köröskörül, hogy elveszszek, és reménységemet, mint a fát, letördelé.
\par 11 Felgerjesztette haragját ellenem, és úgy bánt velem, mint ellenségeivel.
\par 12 Seregei együtt jövének be és utat csinálnak ellenem, és az én sátorom mellett táboroznak.
\par 13 Atyámfiait távol ûzé mellõlem, barátaim egészen elidegenedtek tõlem.
\par 14 Rokonaim visszahúzódtak, ismerõseim pedig elfelejtkeznek rólam.
\par 15 Házam zsellérei és szolgálóim idegennek tartanak engem, jövevény lettem elõttök.
\par 16 Ha a szolgámat kiáltom, nem felel, még ha könyörgök is néki.
\par 17 Lehelletem idegenné lett házastársam elõtt, s könyörgésem az én ágyékom magzatai elõtt.
\par 18 Még a kisdedek is megvetnek engem, ha fölkelek, ellenem szólnak nékem.
\par 19 Megútált minden meghitt emberem; a kiket szerettem, azok is ellenem fordultak.
\par 20 Bõrömhöz és húsomhoz ragadt az én csontom, csak fogam húsával menekültem meg.
\par 21 Könyörüljetek rajtam, könyörüljetek rajtam, oh ti barátaim, mert az Isten keze érintett engem!
\par 22 Miért üldöztök, engem úgy, mint az Isten, és mért nem elégesztek meg a  testemmel?
\par 23 Oh, vajha az én beszédeim leirattatnának, oh, vajha könyvbe feljegyeztetnének!
\par 24 Vasvesszõvel és ónnal örökre kõsziklába metszetnének!
\par 25 Mert én tudom, hogy az én megváltóm él, és utoljára az én porom felett megáll.
\par 26 És miután ezt a bõrömet megrágják, testem nélkül látom meg az Istent.
\par 27 A kit magam látok meg magamnak; az én szemeim látják meg, nem más. Az én veséim megemésztettek én bennem;
\par 28 Mert ezt mondjátok: Hogyan fogjuk õt üldözni! látva, hogy a dolog gyökere én bennem rejlik.
\par 29 Féljetek a fegyvertõl, mert a fegyver a bûnök miatt való büntetés, hogy megtudjátok, hogy van ítélet!

\chapter{20}

\par 1 Felele pedig a Naamából való Czófár, és monda:
\par 2 Mindamellett az én gondolataim feleletre kényszerítenek engem, és e miatt nagy az izgatottság bennem.
\par 3 Hallom gyalázatos dorgáltatásomat, de az én értelmes lelkem megfelel majd értem.
\par 4 Tudod-é azt, hogy eleitõl fogva, mióta az embert e földre helyheztette,
\par 5 Az istentelenek vígassága rövid ideig tartó, és a képmutató öröme egy szempillantásig való?
\par 6 Ha szinte az égig érne is az õ magassága, és a felleget érné is a fejével:
\par 7 Mint az emésztete, úgy vész el örökre, és a kik látták, azt mondják: Hol van õ?
\par 8 Mint az álom, úgy elrepül és nem találják õt; eltünik, mint az éjjeli látomás.
\par 9 A szem, a mely rá ragyogott, nem látja többé, és az õ helye sem törõdik már vele.
\par 10 Fiai a szegények kedvében járnak, és kezei visszaadják az õ rablott vagyonát.
\par 11 Csontjai, ha megtelnek is ifjú erõvel, de vele együtt roskad az a porba.
\par 12 Ha édes az õ szájában a gonoszság, és elrejti azt az õ nyelve alá;
\par 13 És kedvez annak és ki nem veti azt, hanem ott tartogatja ínyei között:
\par 14 Az õ étke elváltozik az õ gyomrában, vipera mérgévé lesz a belében.
\par 15 Gazdagságot nyelt el, de kihányja azt, az õ hasából kiûzi azt az Isten.
\par 16 A viperának mérgét szopta és a siklónak fulánkja öli meg õt,
\par 17 Hogy ne lássa a folyóvizeket, a tejjel és mézzel folyó patakokat.
\par 18 A mit másoktól szerzett, vissza kell adnia, nem nyelheti el, ép így az õ cserébe kapott vagyonát is, hogy ne örvendezhessen annak.
\par 19 Mert megrontotta és ott hagyta a szegényeket, házat rabolt, de nem építi meg azt.
\par 20 Mivel gyomra nem tudott betelni, nem is mentheti meg semmi drágaságát.
\par 21 Az õ falánksága elõl semmi sem maradt meg, azért nem lesz tartós az õ jóléte.
\par 22 Mikor teljes az õ bõsége, akkor is szükséget lát, és mindenféle nyomorúság támad reá.
\par 23 Mikor meg akarja tölteni a hasát, reá bocsátja haragjának tüzét, és azt önti rá étele gyanánt.
\par 24 Mikor vasból csinált fegyver elõl fut, aczélból csinált íj veri át.
\par 25 Kihúzza és az kijön a testébõl és kivillan az epéjébõl; rettegés támad felette.
\par 26 Mindenféle titkos sötétség van az õ vagyonán; fúvás nélkül való tûz emészti meg õt; elpusztíttatik az is, a mi sátrában megmaradt.
\par 27 Megjelentik álnokságát az egek, és a föld ellene támad.
\par 28 Házának jövedelme eltünik, szétfoly az az õ haragjának napján.
\par 29 Ez a gonosz ember fizetése Istentõl, és az õ beszédének jutalma a Mindenhatótól.

\chapter{21}

\par 1 Felele pedig Jób, és monda:
\par 2 Jól hallgassátok meg az én beszédemet, és legyen ez a ti vigasztalástok helyett.
\par 3 Szenvedjetek el engem, a míg szólok, azután gúnyoljátok ki beszédemet.
\par 4 Avagy én embernek panaszolkodom-é? Miért ne volna hát keserû a lelkem?
\par 5 Tekintsetek reám és álmélkodjatok el, és tegyétek kezeteket szátokra.
\par 6 Ha visszaemlékezem, mindjárt felháborodom, és reszketés fogja el testemet.
\par 7 Mi az oka, hogy a gonoszok élnek, vénséget érnek, sõt még meg is gyarapodnak?
\par 8 Az õ magvok elõttök nõ fel õ velök, és az õ sarjadékuk szemeik elõtt.
\par 9 Házok békességes a félelemtõl, és az Isten vesszeje nincsen õ rajtok.
\par 10 Bikája folyat és nem terméketlen, tehene megellik és el nem vetél.
\par 11 Kieresztik, mint nyájat, kisdedeiket, és ugrándoznak az õ magzataik.
\par 12 Dobot és hárfát ragadnak, és örvendeznek a síp zengésének.
\par 13 Jóllétben töltik el napjaikat, és egy pillanat alatt szállnak alá a sírba;
\par 14 Noha azt mondják Istennek: Távozzál el tõlünk, mert a te utaidnak tudásában nem gyönyörködünk!
\par 15 Micsoda a Mindenható, hogy tiszteljük õt, és mit nyerünk vele, ha esedezünk elõtte?
\par 16 Mindazáltal az õ javok nincsen hatalmukban, azért a gonoszok tanácsa távol legyen tõlem!
\par 17 Hányszor aluszik el a gonoszok szövétneke, és jõ rájok az õ veszedelmök! Hányszor osztogatja részöket haragjában.
\par 18 Olyanok lesznek, mint a pozdorja a szél elõtt, és mint a polyva, a melyet forgószél ragad el.
\par 19 Isten az õ fiai számára tartja fenn annak büntetését. Megfizet néki, hogy megérzi majd.
\par 20 Maga látja meg a maga veszedelmét, és a Mindenható haragjából iszik.
\par 21 Mert mi gondja van néki házanépére halála után, ha az õ hónapjainak száma letelt?!
\par 22 Ki taníthatja Istent bölcseségre, hisz õ ítéli meg a magasságban levõket is!
\par 23 Ez meghal az õ teljes boldogságában, egészen megelégedetten és nyugodtan;
\par 24 Fejõedényei tejjel vannak tele, csontjainak velõje nedvességtõl árad.
\par 25 Amaz elkeseredett lélekkel hal meg, mert nem élhetett a jóval.
\par 26 Együtt feküsznek a porban, és féreg lepi õket.
\par 27 Ímé, jól tudom a ti gondolatitokat és a hamisságokat, a melyekkel méltatlankodtok ellenem;
\par 28 Mert ezt mondjátok: Hol van ama fõembernek háza, hol van a gonoszok lakozásának sátora?
\par 29 Avagy nem kérdeztétek-é meg azokat, a kik sokat utaznak és jeleiket nem ismeritek-é?
\par 30 Bizony a veszedelemnek napján elrejtetik a gonosz, a haragnak napján kiszabadul.
\par 31 Kicsoda veti szemére az õ útját, és a mit cselekedett, kicsoda fizet meg néki azért?
\par 32 Még ha a sírba vitetik is ki, a sírdomb felett is él.
\par 33 Édesek lesznek néki a sírnak hantjai, és maga után vonsz minden embert, a mint számtalanok mentek el elõtte.
\par 34 Hogyan vigasztalnátok hát engem hiábavalósággal? Feleselésetek igazságtalanság marad.

\chapter{22}

\par 1 Felele pedig a Témánból való Elifáz, és monda:
\par 2 Az Istennek használ-é az ember? Sõt önmagának használ az okos!
\par 3 Gyönyörûségére van-é az a Mindenhatónak, ha te igaz vagy; avagy nyereség-é, hogy feddhetetlenül jársz?
\par 4 A te isteni félelmedért fedd-é téged, és azért perel-é veled?
\par 5 Avagy nem sok-é a te gonoszságod, és nem véghetetlen-é a te hamisságod?
\par 6 Hiszen zálogot vettél a te atyádfiától méltatlanul, és a szegényeket mezítelenekké tetted.
\par 7 Az eltikkadtnak vizet sem adtál inni, és az éhezõtõl megtagadtad a kenyeret.
\par 8 A ki hatalmas volt, azé vala az ország, és a ki nagytekintélyû volt, az lakik vala rajta.
\par 9 Az özvegyeket üres kézzel bocsátottad el, és az árváknak karjai  eltörettek.
\par 10 Azért vett körül téged a veszedelem, és rettegtet téged hirtelen való rettegés;
\par 11 Avagy a setétség, hogy ne láthass, és a vizek árja, a mely elborít!
\par 12 Hát nem olyan magas-é Isten, mint az egek? És lásd, a csillagok is ott fent mily igen magasak!
\par 13 És mégis azt mondod: Mit tud az Isten; megítélheti-é, a mi a homály mögött van?
\par 14 Sûrû felhõk leplezik el õt és nem lát, és az ég körületén jár.
\par 15 Az õsvilág ösvényét követed-é, a melyen az álnok emberek jártak;
\par 16 A kik idõnap elõtt ragadtattak el, és alapjokat elmosta a víz?!
\par 17 A kik azt mondják vala Istennek: Távozzál el tõlünk! És mit cselekedék velök a Mindenható?
\par 18 Õ pedig megtöltötte házaikat jóval. De az istentelenek tanácsa távol legyen tõlem.
\par 19 Látják ezt az igazak és örülnek rajta, az ártatlan pedig csúfolja õket:
\par 20 Valósággal kigyomláltatott a mi ellenségünk, és az õ maradékjokat tûz emészti meg!
\par 21 Bízd csak azért magadat õ reá és légy békességben: ezekbõl jó származik reád.
\par 22 Végy csak oktatást az õ szájából, és vésd szívedbe az õ beszédeit!
\par 23 Ha megtérsz a Mindenhatóhoz, megépíttetel, és az álnokságot távol ûzöd a te sátorodtól.
\par 24 Vesd a porba a nemes érczet, és a patakok kavicsába az ofiri aranyat:
\par 25 És akkor a Mindenható lesz a te nemes érczed és a te ragyogó ezüstöd.
\par 26 Bizony akkor a Mindenhatóban gyönyörködöl, és a te arczodat Istenhez emeled.
\par 27 Hozzá könyörögsz és meghallgat téged, és lefizeted fogadásaidat.
\par 28 Mihelyt valamit elgondolsz, sikerül az néked, és a te utaidon világosság fénylik.
\par 29 Hogyha megaláznak, felmagasztalásnak mondod azt, és õ az alázatost megtartja.
\par 30 Megszabadítja a nem ártatlant is, és pedig a te kezeidnek tisztaságáért szabadul meg az.

\chapter{23}

\par 1 Felele pedig Jób, és monda:
\par 2 Még most is keserû az én beszédem; súlyosabb rajtam a csapás, ha panaszkodom.
\par 3 Oh ha tudnám, hogy megtalálom õt, elmennék szinte az õ székéig.
\par 4 Elébe terjeszteném ügyemet, számat megtölteném mentõ erõsségekkel.
\par 5 Hadd tudnám meg, mely szavakkal felelne nékem; hadd érteném meg, mit szólana hozzám.
\par 6 Vajjon erejének nagy volta szerint perelne-é velem? Nem; csak figyelmezne reám!
\par 7 Ott egy igaz perelne õ vele; azért megszabadulhatnék birámtól örökre!
\par 8 Ámde kelet felé megyek és nincsen õ, nyugot felé és nem veszem õt észre.
\par 9 Bal kéz felõl cselekszik, de meg nem foghatom; jobb kéz felõl rejtõzködik és nem láthatom.
\par 10 De õ jól tudja az én utamat. Ha megvizsgálna engem, úgy kerülnék ki, mint az arany.
\par 11 Lábam az õ nyomdokát követte; utát megõriztem és nem hajoltam el.
\par 12 Az õ ajakinak parancsolatától sem tértem el; szájának beszédeit többre becsültem, mint életem táplálékát.
\par 13 Õ azonban megmarad egy mellett. Kicsoda téríthetné el õt? És a mit megkiván lelke, azt meg is teszi.
\par 14 Bizony végbe viszi, a mi felõlem elrendeltetett, és ilyen még sok van õ nála.
\par 15 Azért rettegek az õ színe elõtt, és ha csak rá gondolok is, félek tõle.
\par 16 Mert Isten félemlítette meg az én szívemet, a Mindenható rettentett meg engem.
\par 17 Miért is nem pusztultam el e sötétség elõtt, vagy miért nem takarta el elõlem e homályt?!

\chapter{24}

\par 1 Miért is nem titkolja el a Mindenható az õ büntetésének idejét, és miért is nem látják meg az õt ismerõk az õ ítéletének napjait?!
\par 2 A határokat odább tolják, a nyájat elrabolják és legeltetik.
\par 3 Az árvák szamarát elhajtják, és az özvegynek ökrét zálogba viszik.
\par 4 Lelökik az útról a szegényeket, és a föld nyomorultjai együtt lappanganak.
\par 5 Ímé, mint a vad szamarak a sivatagban, úgy mennek ki munkájukra élelmet keresni; a puszta ad nékik kenyeret fiaik számára.
\par 6 A mezõn a más vetését aratják, és a gonosznak szõlõjét szedik.
\par 7 Mezítelenül hálnak, testi ruha nélkül, még a hidegben sincs takarójuk.
\par 8 A hegyi zápor csurog le rólok, s hajlékuk nem lévén, a sziklát ölelik.
\par 9 Elszakítják az emlõtõl az árvát, és a szegényen levõt zálogba viszik.
\par 10 Mezítelenül járnak, ruha nélkül, és éhesen vonszolják a kévét.
\par 11 Az õ kerítéseik közt ütik az olajat, és tapossák a kádakat, de szomjuhoznak.
\par 12 A városból haldoklók rimánkodnak, a megsebzettek lelke kiált, de Isten nem törõdik e méltatlansággal.
\par 13 Ezek pártot ütöttek a világosság ellen, utait nem ismerik, nem ülnek annak ösvényein.
\par 14 Napkeltekor fölkel a gyilkos, megöli a szegényt és szûkölködõt, éjjel pedig olyan, mint a tolvaj.
\par 15 A paráznának szeme pedig az alkonyatot lesi, mondván: Ne nézzen szem reám, és arczára álarczot teszen.
\par 16 Sebtében tör be a házakba; nappal elzárkóznak, nem szeretik a világosságot.
\par 17 Sõt inkább a reggel nékik olyan, mint a halálnak árnyéka, mert megbarátkoztak a halál árnyékának félelmeivel.
\par 18 Könnyen siklik tova a víz színén, birtoka átkozott a földön, nem tér a szõlõkbe vivõ útra.
\par 19 Szárazság és hõség nyeli el a hó vizét, a pokol azokat, a kik vétkeznek.
\par 20 Elfelejti õt az anyaméh, féregnek lesz édességévé, nem emlékeznek róla többé, és összetörik, mint a reves fa,
\par 21 A ki megrontotta a meddõt, a ki nem szül, és az özvegygyel jót nem tett.
\par 22 De megtámogatja erejével a hatalmasokat; felkel az, pedig nem bízott már az élethez.
\par 23 Biztonságot ad néki, hogy támaszkodjék, de szemei vigyáznak azoknak útjaira.
\par 24 Magasra emelkednek, egy kevés idõ és már nincsenek! Alásülylyednek, mint akárki és elenyésznek; és levágattatnak, mint a búzakalász.
\par 25 Avagy nem így van-é? Ki hazudtolhatna meg engem, és tehetné semmivé beszédemet?

\chapter{25}

\par 1 Felele pedig a Sukhból való Bildád, és monda:
\par 2 Hatalom és fenség az övé, a ki békességet szerez az õ magasságaiban.
\par 3 Van-é száma az õ sereginek és kire nem kél fel az õ világossága?
\par 4 Hogy-hogy lehetne igaz a halandó Isten elõtt, hogyan lehetne tiszta, a ki asszonytól született?
\par 5 Nézd a holdat, az sem ragyogó, még a csillagok sem tiszták az õ szemei elõtt.
\par 6 Mennyivel kevésbé a halandó, a ki féreg, és az embernek fia, a ki hernyó.

\chapter{26}

\par 1 Jób pedig felele, és monda:
\par 2 Bezzeg jól segítettél a tehetetlenen, meggyámolítottad az erõtelen kart!
\par 3 Bezzeg jó tanácsot adtál a tudatlannak, és sok értelmet tanusítottál!
\par 4 Kivel beszélgettél, és kinek a lelke jött ki belõled?
\par 5 A halottak is megremegnek tõle; a vizek alatt levõk és azok lakói is.
\par 6 Az alvilág mezítelen elõtte, és eltakaratlan a holtak országa.
\par 7 Õ terjeszti ki északot az üresség fölé és függeszti föl a földet a semmiség fölé.
\par 8 Õ köti össze felhõibe a vizeket úgy, hogy a felhõ alattok meg nem hasad.
\par 9 Õ rejti el királyi székének színét, felhõjét fölibe terítvén.
\par 10 Õ szab határt a víz színe fölé - a világosságnak és setétségnek elvégzõdéséig.
\par 11 Az egek oszlopai megrendülnek, és düledeznek fenyegetéseitõl.
\par 12 Erejével felriasztja a tengert, és bölcseségével megtöri Ráhábot.
\par 13 Lehelletével megékesíti az eget, keze átdöfi a futó kígyót.
\par 14 Ímé, ezek az õ útainak részei, de mily kicsiny rész az, a mit meghallunk abból! Ám az õ hatalmának mennydörgését ki érthetné meg?

\chapter{27}

\par 1 Jób pedig folytatá az õ beszédét, monda:
\par 2 Él az Isten, a ki az én igazamat elfordította, és a Mindenható, a ki keserûséggel illette az én lelkemet,
\par 3 Hogy mindaddig, a míg az én lelkem én bennem van, és az Istennek lehellete van az én orromban;
\par 4 Az én ajakim nem szólnak álnokságot, és az én nyelvem nem mond csalárdságot!
\par 5 Távol legyen tõlem, hogy igazat adjak néktek! A míg lelkemet ki nem lehelem, ártatlanságomból magamat ki nem tagadom.
\par 6 Igazságomhoz ragaszkodom, róla le nem mondok; napjaim miatt nem korhol az én szívem.
\par 7 Ellenségem lesz olyan, mint a gonosz, és a ki ellenem támad, mint az álnok.
\par 8 Mert micsoda reménysége lehet a képmutatónak, hogy telhetetlenkedett, ha az Isten mégis mégis elragadja az õ lelkét?
\par 9 Meghallja-é kiáltását az Isten, ha eljõ a nyomorúság reá?
\par 10 Vajjon gyönyörködhetik-é a Mindenhatóban; segítségül hívhatja-é mindenkor az Istent?
\par 11 Megtanítlak benneteket Isten dolgaira; a mik a Mindenhatónál vannak, nem titkolom el.
\par 12 Ímé, ti is mindnyájan látjátok: miért van hát, hogy hiábavalósággal hivalkodtok?!
\par 13 Ez a gonosz embernek osztályrésze Istentõl, és a kegyetlenek öröksége a Mindenhatótól, a melyet elvesznek:
\par 14 Ha megsokasulnak is az õ fiai, a kardnak sokasulnak meg, és az õ magzatai nem lakhatnak jól kenyérrel sem.
\par 15 Az õ maradékai dögvész miatt temettetnek el, és az õ özvegyeik meg sem siratják.
\par 16 Ha mint a port, úgy halmozná is össze az ezüstöt, és úgy szerezné is össze ruháit, mint a sarat:
\par 17 Összeszerezheti ugyan, de az igaz ruházza magára, az ezüstön pedig az ártatlan osztozik.
\par 18 Házát pók módjára építette föl, és olyanná, mint a csõsz-csinálta kunyhó.
\par 19 Gazdagon fekszik le, mert nincsen kifosztva; felnyitja szemeit és semmije sincsen.
\par 20 Meglepi õt, mint az árvíz, a félelem, éjjel ragadja el a zivatar.
\par 21 Felkapja õt a keleti szél és elviszi, elragadja õt helyérõl.
\par 22 Nyilakat szór reá és nem kiméli; futva kell futnia keze elõl.
\par 23 Csapkodják felette kezeiket, és kisüvöltik õt az õ lakhelyébõl.

\chapter{28}

\par 1 Bizony az ezüstnek bányája van, és helye az aranynak, a hol tisztítják.
\par 2 A vasat a földbõl hozzák elõ, a követ pedig érczczé olvasztják.
\par 3 Határt vet az ember a setétségnek, és átkutatja egészen és végig a homálynak és a halál árnyékának kövét.
\par 4 Aknát tör távol a lakóktól: mintha lábukról is megfelejtkeznének, alámerülnek és lebegnek emberektõl messze.
\par 5 Van föld, a melybõl kenyér terem, alant pedig fel van forgatva, mintegy tûz által;
\par 6 Köveiben zafir található, göröngyeiben arany van.
\par 7 Van ösvény, a melyet nem ismer a sas, sem a sólyom szeme nem látja azt.
\par 8 Nem tudják azt büszke vadak, az oroszlán sem lépked azon.
\par 9 Ráveti kezét az ember a kovakõre, a hegyeket tövükbõl kiforgatja.
\par 10 A sziklákban tárnákat hasít, és minden drága dolgot meglát a szeme.
\par 11 Elköti a folyók szivárgását, az elrejtett dolgot pedig világosságra hozza.
\par 12 De a bölcseség hol található, és az értelemnek hol van a helye?
\par 13 Halandó a hozzá vivõ utat nem ismeri, az élõk földén az nem található.
\par 14 A mélység azt mondja: Nincsen az bennem; a tenger azt mondja: én nálam sincsen.
\par 15 Színaranyért meg nem szerezhetõ, ára ezüsttel meg nem fizethetõ.
\par 16 Nem mérhetõ össze Ofir aranyával, nem drága onikszszal, sem zafirral.
\par 17 Nem ér fel vele az arany és gyémánt, aranyedényekért be nem cserélhetõ.
\par 18 Korall és kristály említni sem való; a bölcseség ára drágább a gyöngyöknél.
\par 19 Nem ér fel vele Kúsnak topáza, színaranynyal sem mérhetõ össze.
\par 20 A bölcseség honnan jõ tehát, és hol van helye az értelemnek?
\par 21 Rejtve van az minden élõ szemei elõtt, az ég madarai elõl is fedve van.
\par 22 A pokol és halál azt mondják: Csak hírét hallottuk füleinkkel!
\par 23 Isten tudja annak útját, õ ismeri annak helyét.
\par 24 Mert õ ellát a föld határira, õ lát mindent az ég alatt.
\par 25 Mikor a szélnek súlyt szerzett, és a vizeket mértékre vette;
\par 26 Mikor az esõnek határt szabott, és mennydörgõ villámoknak útat:
\par 27 Akkor látta és kijelentette azt, megalapította és meg is vizsgálta azt.
\par 28 Az embernek pedig mondá: Ímé az Úrnak félelme: az a bölcseség, és az értelem: a gonosztól való eltávozás.

\chapter{29}

\par 1 Jób pedig folytatá az õ beszédét, és monda:
\par 2 Oh, vajha olyan volnék, mint a hajdani hónapokban, a mikor Isten õrzött engem!
\par 3 Mikor az õ szövétneke fénylett fejem fölött, s világánál jártam a setétet;
\par 4 A mint java-korom napjaiban valék, a mikor Isten gondossága borult sátoromra!
\par 5 Mikor még a Mindenható velem volt, és körültem voltak gyermekeim;
\par 6 Mikor lábaimat édes tejben mostam, és mellettem a szikla olajpatakokat ontott;
\par 7 Mikor a kapuhoz mentem, fel a városon; a köztéren székemet fölállítám:
\par 8 Ha megláttak az ifjak, félrevonultak, az öregek is fölkeltek és állottak.
\par 9 A fejedelmek abbahagyták a beszédet, és tenyeröket szájukra tették.
\par 10 A fõemberek szava elnémult, és nyelvök az ínyökhöz ragadt.
\par 11 Mert a mely fül hallott, boldognak mondott engem, és a mely szem látott, bizonyságot tett én felõlem.
\par 12 Mert megmentém a kiáltozó szegényt, és az árvát, a kinek nem volt segítsége.
\par 13 A veszni indultnak áldása szállt reám, az özvegynek szívét megörvendeztetém.
\par 14 Az igazságot magamra öltém és az is magára ölte engem; palást és süveg gyanánt volt az én ítéletem.
\par 15 A vaknak én szeme valék, és a sántának lába.
\par 16 A szûkölködõknek én atyjok valék, az ismeretlennek ügyét is jól meghányám-vetém.
\par 17 Az álnoknak zápfogait kitördösém, és fogai közül a prédát kiütém vala.
\par 18 Azt gondoltam azért: fészkemmel veszek el, és mint a homok, megsokasodnak napjaim.
\par 19 Gyökerem a víznek nyitva lesz, és ágamon hál meg a harmat.
\par 20 Dicsõségem megújul velem, és kézívem erõsebbé lesz kezemben.
\par 21 Hallgattak és figyeltek reám, és elnémultak az én tanácsomra.
\par 22 Az én szavaim után nem szóltak többet, s harmatként hullt rájok beszédem.
\par 23 Mint az esõre, úgy vártak rám, és szájukat tátották, mint tavaszi záporra.
\par 24 Ha rájok mosolyogtam, nem bizakodtak el, és arczom derüjét nem sötétíték be.
\par 25 Örömest választottam útjokat, mint fõember ültem ott; úgy laktam ott, mint király a hadseregben, mint a ki bánkódókat vigasztal.

\chapter{30}

\par 1 Most pedig nevetnek rajtam, a kik fiatalabbak nálam a kiknek atyjokat az én juhaimnak komondorai közé sem számláltam volna.
\par 2 Mire való lett volna nékem még kezök ereje is? Rájok nézve a vénség elveszett!
\par 3 Szükség és éhség miatt összeaszottak, a kik a kopár földet futják, a sötét, sivatag pusztaságot.
\par 4 A kik keserû füvet tépnek a bokor mellett, és rekettyegyökér a kenyerök.
\par 5 Az emberek közül kiûzik õket, úgy hurítják õket, mint a tolvajt.
\par 6 Félelmetes völgyekben kell lakniok, a földnek és szikláknak hasadékaiban.
\par 7 A bokrok között ordítanak, a csalánok alatt gyülekeznek.
\par 8 Esztelen legények, sõt becstelen fiak, a kiket kivertek az országból.
\par 9 És most ezeknek lettem gúnydalává, nékik levék beszédtárgyuk!
\par 10 Útálnak engem, messze távoznak tõlem és nem átalanak pökdösni elõttem.
\par 11 Sõt leoldják kötelöket és bántalmaznak engem, és zabolát elõttem kivetik.
\par 12 Jobb felõl ifjak támadnak ellenem, gáncsot vetnek lábaimnak, és ösvényt törnek felém, hogy megrontsanak.
\par 13 Az én útamat elrontják, romlásomat öregbítik, nincsen segítség ellenök.
\par 14 Mint valami széles résen, úgy rontanak elõ, pusztulás között hömpölyögnek ide.
\par 15 Rettegések fordultak ellenem, mint vihar ûzik el tisztességemet, boldogságom eltünt, mint a felhõ.
\par 16 Mostan azért enmagamért ontja ki magát lelkem; nyomorúságnak napjai fognak meg engem.
\par 17 Az éjszaka meglyuggatja csontjaimat bennem, és nem nyugosznak az én inaim.
\par 18 A sok erõlködés miatt elváltozott az én ruházatom; úgy szorít engem, mint a köntösöm galléra.
\par 19 A sárba vetett engem, hasonlóvá lettem porhoz és hamuhoz.
\par 20 Kiáltok hozzád, de nem felelsz; megállok és csak nézel reám!
\par 21 Kegyetlenné változtál irántam; kezed erejével harczolsz ellenem.
\par 22 Felemelsz, szélnek eresztesz engem, és széttépsz engem a viharban.
\par 23 Hiszen tudtam, hogy visszatérítesz engem a halálba, és a minden élõ gyülekezõ házába;
\par 24 De a roskadóban levõ ne nyujtsa-é ki kezét? Avagy ha veszendõben van, ne kiáltson-é segítségért?
\par 25 Avagy nem sírtam-é azon, a kinek kemény napja volt; a szûkölködõ miatt nem volt-é lelkem szomorú?
\par 26 Bizony jót reméltem és rossz következék, világosságot vártam és homály jöve.
\par 27 Az én bensõm forr és nem nyugoszik; megrohantak engem a nyomorúságnak napjai.
\par 28 Feketülten járok, de nem a nap hõsége miatt; felkelek a gyülekezetben és kiáltozom.
\par 29 Atyjok fiává lettem a sakáloknak, és társokká a strucz madaraknak.
\par 30 Bõröm feketülten hámlik le rólam, és csontom elég a hõség miatt.
\par 31 Hegedûm sírássá változék, sípom pedig jajgatók szavává.

\chapter{31}

\par 1 Szövetségre léptem szemeimmel, és hajadonra mit sem ügyeltem.
\par 2 És mi volt jutalmam Istentõl felülrõl; vagy örökségem a Mindenhatótól a magasságból?
\par 3 A vagy nem az istentelent illeti-é romlás, és nem a gonosztevõt-é veszedelem?
\par 4 Avagy nem láthatta-é utaimat, és nem számlálhatta-é meg lépéseimet?
\par 5 Ha én csalárdsággal jártam, vagy az én lábam álnokságra sietett:
\par 6 Az õ igazságának mérlegével mérjen meg engem, és megismeri Isten az én ártatlanságomat!
\par 7 Ha az én lépésem letért az útról és az én lelkem követte szemeimet, vagy kezeimhez szenny tapadt:
\par 8 Hadd vessek én és más egye meg, és tépjék ki az én maradékaimat gyökerestõl!
\par 9 Ha az én szívem asszony után bomlott, és leselkedtem az én felebarátomnak ajtaján:
\par 10 Az én feleségem másnak õröljön, és mások hajoljanak rája.
\par 11 Mert gyalázatosság volna ez, és birák elé tartozó bûn.
\par 12 Mert tûz volna ez, a mely pokolig emésztene, és minden jövedelmemet tövestõl kiirtaná.
\par 13 Ha megvetettem volna igazát az én szolgámnak és szolgálómnak, mikor pert kezdtek ellenem:
\par 14 Mi tevõ lennék, ha felkelne az Isten, és ha meglátogatna: mit felelnék néki?
\par 15 Nem az teremtette-é õt is, a ki engem teremtett anyám méhében; nem egyugyanaz formált-é bennünket anyánk ölében?
\par 16 Ha a szegények kivánságát megtagadtam, és az özvegy  szemeit epedni engedtem;
\par 17 És ha falatomat egymagam ettem meg, és az árva abból nem evett;
\par 18 Hiszen ifjúságom óta, mint atyánál nevekedett nálam, és anyámnak méhétõl kezdve vezettem õt!
\par 19 Ha láttam a ruhátlant veszni indulni, és takaró nélkül a szegényt;
\par 20 Hogyha nem áldottak engem az õ ágyékai, és az én juhaim gyapjából fel nem melegedett;
\par 21 Ha az árva ellen kezemet felemeltem, mert láttam a kapuban az én segítségemet;
\par 22 A lapoczkájáról essék ki a vállam, és a forgócsontról szakadjon le karom!
\par 23 Hiszen úgy rettegtem Isten csapásától, és fensége elõtt tehetetlen valék!
\par 24 Ha reménységemet aranyba vetettem, és azt mondtam az olvasztott aranynak: Én bizodalmam!
\par 25 Ha örültem azon, hogy nagy a gazdagságom, és hogy sokat szerzett az én kezem;
\par 26 Ha néztem a napot, mikor fényesen ragyogott, és a holdat, mikor méltósággal haladt,
\par 27 És az én szívem titkon elcsábult, és szájammal megcsókoltam a kezemet:
\par 28 Ez is biró elé tartozó bûn volna, mert ámítottam volna az Istent oda fent!
\par 29 Ha örvendeztem az engem gyûlölõnek nyomorúságán, és ugráltam örömömben, hogy azt baj érte;
\par 30 (De nem engedtem, hogy szájam vétkezzék azzal, hogy átkot kérjek az õ lelkére!)
\par 31 Ha nem mondták az én sátorom cselédei: Van-é, a ki az õ húsával jól nem lakott?
\par 32 (A jövevény nem hált az utczán, ajtóimat az utas elõtt megnyitám.)
\par 33 Ha emberi módon eltitkoltam vétkemet, keblembe rejtve bûnömet:
\par 34 Bizony akkor tarthatnék a nagy tömegtõl, rettegnem kellene nemzetségek megvetésétõl; elnémulnék és az ajtón sem lépnék ki!
\par 35 Oh, bárcsak volna valaki, a ki meghallgatna engem! Ímé, ez a végszóm: a Mindenható feleljen meg nékem; és írjon könyvet ellenem az én vádlóm.
\par 36 Bizony én azt a vállamon hordanám, és korona gyanánt a fejemre tenném!
\par 37 Lépteimnek számát megmondanám néki, mint egy fejedelem, úgy járulnék hozzá!
\par 38 Ha földem ellenem kiáltott és annak barázdái együtt siránkoztak;
\par 39 Ha annak termését fizetés nélkül ettem, vagy gazdájának lelkét kioltottam:
\par 40 Búza helyett tövis teremjen és árpa helyett konkoly! Itt végzõdnek a Jób beszédei.

\chapter{32}

\par 1 Miután ez a három ember megszünt vala felelni Jóbnak, mivel õ igaz vala önmaga elõtt:
\par 2 Haragra gerjede Elihu, a Barakeél fia, a ki Búztól való vala, a Rám nemzetségébõl. Jób ellen gerjedt föl haragja, mivel az igazabbnak tartotta magát Istennél.
\par 3 De felgerjedt haragja az õ három barátja ellen is, mivelhogy nem találják vala el a feleletet, mégis kárhoztatják vala Jóbot.
\par 4 Elihu azonban várakozott a Jóbbal való beszéddel, mert amazok öregebbek valának õ nála.
\par 5 De mikor látta Elihu, hogy nincs felelet a három férfiú szájában, akkor gerjede föl az õ haragja.
\par 6 És felele a Búztól való Elihu, Barakeél fia, és monda: Napjaimra nézve én még csekély vagyok, ti pedig élemedett emberek, azért tartózkodtam és féltem tudatni veletek véleményemet.
\par 7 Gondoltam: Hadd szóljanak a napok; és hadd hirdessen bölcseséget az évek sokasága!
\par 8 Pedig a lélek az az emberben és a Mindenható lehellése, a mi értelmet ad néki!
\par 9 Nem a nagyok a bölcsek, és nem a vének értik az ítéletet.
\par 10 Azt mondom azért: Hallgass reám, hadd tudassam én is véleményemet!
\par 11 Ímé, én végig vártam beszédeiteket, figyeltem, a míg okoskodtatok, a míg szavakat keresgéltetek.
\par 12 Igen ügyeltem reátok és ímé, Jóbot egyikõtök sem czáfolá meg, sem beszédére meg nem felelt.
\par 13 Ne mondjátok azt: Bölcseségre találtunk, Isten gyõzheti meg õt és nem ember!
\par 14 Mivel én ellenem nem intézett beszédet, nem is a ti beszédeitekkel válaszolok hát néki.
\par 15 Megzavarodának és nem feleltek többé; kifogyott belõlök a szó.
\par 16 Vártam, de nem szóltak, csak álltak és nem feleltek többé.
\par 17 Hadd feleljek hát én is magamért, hadd tudassam én is véleményemet!
\par 18 Mert tele vagyok beszéddel; unszolgat engem a bennem levõ lélek.
\par 19 Ímé, bensõm olyan, mint az új bor, a melynek nyílása nincsen; miként az új tömlõk, csaknem szétszakad.
\par 20 Szólok tehát, hogy levegõhöz jussak; felnyitom ajkaimat, és felelek.
\par 21 Nem leszek személyválogató senki iránt; nem hizelkedem egy embernek sem;
\par 22 Mert én hizelkedni nem tudok; könnyen elszólíthatna engem a teremtõm!

\chapter{33}

\par 1 No azért halld meg csak Jób az én szavaimat, és vedd füledbe minden beszédemet!
\par 2 Ímé, megnyitom már az én szájamat, és a beszéd nyelvem alatt van már.
\par 3 Igaz szívbõl származnak beszédeim, tiszta tudományt hirdetnek ajkaim.
\par 4 Az Istennek lelke teremtett engem, és a Mindenhatónak lehellete adott nékem életet.
\par 5 Ha tudsz, czáfolj meg; készülj fel ellenem és állj elõ!
\par 6 Ímé, én szintúgy Istené vagyok, mint te; sárból formáltattam én is.
\par 7 Ímé, a tõlem való félelem meg ne háborítson; kezem nem lészen súlyos rajtad.
\par 8 Csak az imént mondtad fülem hallatára, hallottam a beszédnek hangját:
\par 9 Tiszta vagyok, fogyatkozás nélkül: mocsoktalan vagyok, bûn nincsen bennem.
\par 10 Ímé, vádakat talál ki ellenem, ellenségének tart engem!
\par 11 Békóba veti lábaimat, és õrzi minden ösvényemet.
\par 12 Ímé, ebben nincsen igazad - azt felelem néked - mert nagyobb az Isten az embernél!
\par 13 Miért perelsz vele? Azért, hogy egyetlen beszédedre sem felelt?
\par 14 Hiszen szól az Isten egyszer vagy kétszer is, de nem ügyelnek rá!
\par 15 Álomban, éjjeli látomásban, mikor mély álom száll az emberre, és mikor ágyasházokban szenderegnek;
\par 16 Akkor nyitja meg az emberek fülét, és megpecsételi megintetésökkel.
\par 17 Hogy eltérítse az embert a rossz cselekedettõl, és elrejtse a kevélységet a férfi elõl.
\par 18 Visszatartja lelkét a romlástól, és életét hogy azt fegyver ne járja át.
\par 19 Fájdalommal is bünteti az õ ágyasházában, és csontjainak szüntelen való háborgásával.
\par 20 Úgy, hogy az õ ínye undorodik az ételtõl, és lelke az õ kedves ételétõl.
\par 21 Húsa szemlátomást aszik le róla; csontjai, a melyeket látni nem lehetett, kiülnek.
\par 22 És lelke közelget a sírhoz, s élete a halál angyalaihoz.
\par 23 Ha van mellette magyarázó angyal, egy az ezer közül, hogy az emberrel tudassa kötelességét;
\par 24 És az Isten könyörül rajta, és azt mondja: Szabadítsd meg õt, hogy ne szálljon a sírba; váltságdíjat találtam!
\par 25 Akkor teste fiatal, erõtõl duzzad, újra kezdi ifjúságának napjait.
\par 26 Imádkozik Istenhez és õ kegyelmébe veszi, hogy az õ színét nézhesse nagy örömmel, és az embernek visszaadja az õ igazságát.
\par 27 Az emberek elõtt énekel és mondja: Vétkeztem és az igazat elferdítettem vala, de nem e szerint fizetett meg nékem;
\par 28 Megváltotta lelkemet a sírba szállástól, és egész valóm a világosságot nézi.
\par 29 Ímé, mindezt kétszer, háromszor cselekszi Isten az emberrel,
\par 30 Hogy megmentse lelkét a sírtól, hogy világoljon az élet világosságával.
\par 31 Figyelj Jób, és hallgass meg engem; hallgass, hadd szóljak én!
\par 32 Ha van mit mondanod, czáfolj meg; szólj, mert igen szeretném a te igazságodat.
\par 33 Ha pedig nincs, hallgass meg engem, hallgass és megtanítlak téged a bölcseségre!

\chapter{34}

\par 1 És szóla Elihu, és monda:
\par 2 Halljátok meg bölcsek az én szavaimat, és ti tudósok hajtsátok hozzám füleiteket!
\par 3 Mert a fül próbálja meg a szót, mint az íny kóstolja meg az ételt.
\par 4 Keressük csak magunk az igazságot, értsük meg magunk között, mi a jó?
\par 5 Mert Jób azt mondá: Igaz vagyok, de Isten megtagadja igazságomat.
\par 6 Igazságom ellenére kell hazugnak lennem; halálos nyíl talált hibám nélkül!
\par 7 Melyik ember olyan, mint Jób, a ki iszsza a csúfolást, mint a vizet.
\par 8 És egy társaságban forog a gonosztevõkkel, és az istentelen emberekkel jár!
\par 9 Mert azt mondja: Nem használ az az embernek, ha Istennel békességben él.
\par 10 Azért, ti tudós emberek, hallgassatok meg engem! Távol legyen Istentõl a gonoszság, és a Mindenhatótól az álnokság!
\par 11 Sõt inkább, a mint cselekszik az ember, úgy fizet néki, és kiki az õ útja szerint találja meg, a mit keres.
\par 12 Bizonyára az Isten nem cselekszik gonoszságot, a Mindenható el nem ferdíti az igazságot!
\par 13 Kicsoda bízta reá a földet és ki rendezte az egész világot?
\par 14 Ha csak õ magára volna gondja, lelkét és lehellését magához vonná:
\par 15 Elhervadna együtt minden test és az ember visszatérne a porba.
\par 16 Ha tehát van eszed, halld meg ezt, és a te füledet hajtsd az én beszédeimnek szavára!
\par 17 Vajjon, a ki gyûlöli az igazságot, kormányozhat-é?  Avagy az ellenállhatatlan igazat kárhoztathatod-é?
\par 18 A ki azt mondja a királynak: Te semmirevaló! És a fõembereknek: Te gonosztevõ!
\par 19 A ki nem nézi a fejedelmek személyét és a gazdagot a szegénynek fölibe nem helyezteti; mert mindnyájan az õ  kezének munkája.
\par 20 Egy pillanat alatt meghalnak; éjfélkor felriadnak a népek és elenyésznek, a hatalmas is eltûnik kéz nélkül!
\par 21 Mert õ szemmel tartja mindenkinek útját, és minden lépését jól látja.
\par 22 Nincs setétség és nincs a halálnak árnyéka, a hova elrejtõzhessék a gonosztevõ;
\par 23 Mert nem sokáig kell szemmel tartania az embert, hogy az Isten elé kerüljön ítéletre!
\par 24 Megrontja a hatalmasokat vizsgálat nélkül, és másokat állít helyökbe.
\par 25 Ekképen felismeri cselekedeteiket, és éjjel is ellenök fordul és szétmorzsoltatnak.
\par 26 Gonosztevõk gyanánt tapodja meg õket olyan helyen, a hol látják.
\par 27 A kik azért távoztak el, és azért nem gondoltak egyetlen útjával sem,
\par 28 Hogy a szegény kiáltását hozzájok juttatja, és õ a nyomorultak kiáltását meghallja.
\par 29 Ha õ nyugalmat ád, ki kárhoztatja õt? Ha elrejti arczát, ki láthatja meg azt? Akár nép elõl, akár ember elõl egyaránt;
\par 30 Hogy képmutató ember ne uralkodjék, és ne legyen tõre a népnek.
\par 31 Bizony az Istenhez így való szólani: Elszenvedem, nem leszek rossz többé;
\par 32 A mit át nem látok, arra te taníts meg engemet; ha gonoszságot cselekedtem, többet nem teszem!
\par 33 Avagy te szerinted fizessen-é csak azért, mert ezt megveted, és hogy te szabd meg és nem én? Nos mit tudsz? Mondd!
\par 34 Az okos emberek azt mondják majd nékem, és a bölcs férfiú, a ki reám hallgat:
\par 35 Jób tudatlanul szól, és szavai megfontolás nélkül valók.
\par 36 Óh, bárcsak megpróbáltatnék Jób mind végiglen, a miért úgy felel, mint az álnok emberek!
\par 37 Mert vétkét gonoszsággal tetézi, csapkod közöttünk, és Isten ellen szószátyárkodik.

\chapter{35}

\par 1 Tovább is felele Elihu, és monda:
\par 2 Azt gondolod-é igaznak, ha így szólsz: Az én igazságom nagyobb, mint Istené?
\par 3 Hogyha ezt mondod: Mi hasznod belõle? Mivel várhatok többet, mintha vétkezném?
\par 4 Én megadom rá néked a feleletet, és barátaidnak te veled együtt.
\par 5 Tekints az égre és lásd meg; és nézd meg a fellegeket, milyen magasan vannak feletted!
\par 6 Hogyha vétkezel, mit tehetsz ellene; ha megsokasítod bûneidet, mit ártasz néki?
\par 7 Ha igaz vagy, mit adsz néki, avagy mit kap a te kezedbõl?
\par 8 Az olyan embernek árt a te gonoszságod, mint te vagy, és igazságod az ilyen ember fiának használ.
\par 9 A sok erõszak miatt kiáltoznak; jajgatnak a hatalmasok karja miatt;
\par 10 De egy sem mondja: Hol van Isten, az én teremtõm, a ki hálaénekre indít éjszaka;
\par 11 A ki többre tanít minket a mezei vadaknál, és bölcsebbekké tesz az ég madarainál?
\par 12 Akkor azután kiálthatnak, de õ nem felel a gonoszok kevélysége miatt;
\par 13 Mert a hiábavalóságot Isten meg nem hallgatja, a Mindenható arra nem tekint.
\par 14 Hátha még azt mondod: Te nem látod õt; az ügy elõtte van és te reá vársz!
\par 15 Most pedig, mivel nem büntet haragja, és nem figyelmez a nagy álnokságra:
\par 16 Azért tátja fel Jób hívságra a száját, és szaporítja a szót értelem nélkül.

\chapter{36}

\par 1 És folytatá Elihu, és monda:
\par 2 Várj még egy kevéssé, majd felvilágosítlak, mert az Istenért még van mit mondanom.
\par 3 Tudásomat messzünnen veszem, és az én teremtõmnek igazat adok.
\par 4 Mert az én beszédem bizonyára nem hazugság; tökéletes tudású ember áll melletted.
\par 5 Ímé, az Isten hatalmas, még sem vet meg semmit; hatalmas az õ lelkének ereje.
\par 6 Nem tartja meg a gonosznak életét, de a szegénynek igaz törvényt teszen.
\par 7 Nem veszi le az igazról szemeit, sõt a királyok mellé, a trónba ülteti õket örökre, hogy felmagasztaltassanak.
\par 8 És ha békóba veretnek, és fogva tartatnak a nyomorúság kötelein:
\par 9 Tudtokra adja cselekedetöket, és vétkeiket, hogyha elhatalmaztak rajtok.
\par 10 Megnyitja füleiket a feddõzésnek és megparancsolja, hogy a vétekbõl megtérjenek:
\par 11 Ha engednek és szolgálnak néki, napjaikat jóban végzik el, és az õ esztendeiket gyönyörûségekben.
\par 12 Ha pedig nem engednek, fegyverrel veretnek által, és tudatlanságban múlnak ki.
\par 13 De az álnok szívûek haragot táplálnak, nem kiáltanak, mikor megkötözi õket.
\par 14 Azért ifjúságukban hal meg az õ lelkök, és életök a paráznákéhoz hasonló.
\par 15 A nyomorultat megszabadítja az õ nyomorúságától, és a szorongattatással megnyitja fülöket.
\par 16 Téged is kiszabadítana az ínség torkából tág mezõre, a hol nincs szorultság, és asztalod étke kövérséggel lenne rakva.
\par 17 De ha gonosz ítélettel vagy tele, úgy utolérnek az ítélet és igazság.
\par 18 Csakhogy a harag ne ragadjon téged csúfkodásra, és a nagy váltságdíj se tántorítson el.
\par 19 Ad-é valamit a te gazdagságodra? Sem aranyra, sem semmiféle erõfeszítésre!
\par 20 Ne kívánjad az éjszakát, a mely népeket mozdít ki helyökbõl.
\par 21 Vigyázz! ne pártolj a bûnhöz, noha azt a nyomorúságnál jobban szereted.
\par 22 Ímé, mily fenséges az Isten az õ erejében; kicsoda az, a ki úgy tanítson, mint õ?
\par 23 Kicsoda szabta meg az õ útjait, vagy ki mondhatja azt: Igazságtalanságot cselekedtél?
\par 24 Legyen rá gondod, hogy magasztaljad az õ cselekedetét, a melyrõl énekelnek az emberek!
\par 25 Minden ember azt szemléli; a halandó távolról is látja.
\par 26 Ímé, az Isten fenséges, mi nem ismerhetjük õt! esztendeinek száma sem nyomozható ki.
\par 27 Hogyha magához szívja a vízcseppeket, ködébõl mint esõ cseperegnek alá,
\par 28 A melyet a fellegek özönnel öntenek, és hullatnak le temérdek emberre.
\par 29 De sõt értheti-é valaki a felhõ szétoszlását, az õ sátorának zúgását?
\par 30 Ímé, szétterjeszti magára az õ világosságát, és ráborítja a tengernek gyökereit.
\par 31 Mert ezek által ítéli meg a népeket, ád eledelt bõségesen.
\par 32 Kezeit elborítja a villámlással, és kirendeli a lázadó ellen.
\par 33 Az õ dörgése ad hírt felõle, mint a barom a közeledõ viharról.
\par 34 Ezért remeg az én szívem, és csaknem kiszökik helyébõl.

\chapter{37}

\par 1 Halljátok meg figyelmetesen az õ hangjának dörgését, és a zúgást, a mely az õ szájából kijön!
\par 2 Az egész ég alatt szétereszti azt, és villámát is a földnek széléig.
\par 3 Utána hang zendül, az õ fenségének hangjával mennydörög, s nem tartja vissza azt, ha szava megzendült.
\par 4 Isten az õ szavával csudálatosan mennydörög, és nagy dolgokat cselekszik, úgy hogy nem érthetjük.
\par 5 Mert azt mondja a hónak: Essél le a földre! És a zápor-esõnek és a zuhogó zápornak: Szakadjatok.
\par 6 Minden ember kezét lepecsétli, hogy megismerje minden halandó, hogy az õ mûve.
\par 7 Akkor a vadállat az õ tanyájára húzódik, és az õ barlangjában marad.
\par 8 Rejtekébõl elõjön a vihar, és az északi szelektõl a fagy.
\par 9 Isten lehellete által támad a jég, és szorul össze a víznek szélessége.
\par 10 Majd nedvességgel öntözi meg a felleget, s áttöri a borulatot az õ villáma.
\par 11 És az köröskörül forog az õ vezetése alatt, hogy mindazt megtegyék, a mit parancsol nékik, a föld kerekségének színén.
\par 12 Vagy ostorul, ha földjének úgy kell, vagy áldásul juttatja azt.
\par 13 Vedd ezt füledbe Jób, állj meg és gondold meg az Istennek csudáit.
\par 14 Megtudod-é, mikor rendeli azt rájok az Isten, hogy villanjon az õ felhõjének villáma?
\par 15 Tudod-é, hogy miként lebegnek a felhõk, vagy a tökéletes tudásnak csudáit érted-é?
\par 16 Miképen melegülnek át ruháid, mikor nyugton van a föld a déli széltõl?
\par 17 Vele együtt terjesztetted-é ki az eget, a mely szilárd, mint az aczéltükör?
\par 18 Mondd meg nékünk, mit szóljunk néki? A setétség miatt semmit sem kezdhetünk.
\par 19 Elbeszélik-é néki, ha szólok? Ha elmondaná valaki: bizony vége volna!
\par 20 Néha nem látják a napot, bár az égen ragyog; de szél fut át rajta és kiderül.
\par 21 Észak felõl aranyszínû világosság támad, Isten körül félelmetes dicsõség.
\par 22 Mindenható! Nem foghatjuk meg õt; nagy az õ hatalma és ítélõ ereje, és a tiszta igazságot el nem nyomja.
\par 23 Azért rettegjék õt az emberek; a kevély bölcsek közül nem lát õ egyet sem.

\chapter{38}

\par 1 Majd felele az Úr Jóbnak a forgószélbõl és monda:
\par 2 Ki az, a ki elhomályosítja az örök rendet tudatlan beszéddel?
\par 3 Nosza övezd fel, mint férfiú derekadat, én majd kérdezlek, te meg taníts engem!
\par 4 Hol voltál, mikor a földnek alapot vetettem? Mondd meg, ha tudsz valami okosat!
\par 5 Ki határozta meg mértékeit, ugyan tudod-é; avagy ki húzta el felette a mérõ zsinórt?
\par 6 Mire bocsátották le oszlopait, avagy ki vetette fel szegeletkövét;
\par 7 Mikor együtt örvendezének a hajnalcsillagok, és Istennek minden fiai vigadozának?
\par 8 És kicsoda zárta el ajtókkal a tengert, a mikor elõtünt, az anyaméhbõl kijött;
\par 9 Mikor ruházatává a felhõt tevém, takarójául pedig a sürû homályt?
\par 10 Mikor reávontam törvényemet, zárat és ajtókat veték eléje:
\par 11 És azt mondám: Eddig jõjj és ne tovább; ez itt ellene áll kevély habjaidnak!
\par 12 Parancsoltál-é a reggelnek, a mióta megvagy? Kimutattad-é a hajnalnak a helyét?
\par 13 Hogy belefogózzék a földnek széleibe, és lerázassanak a gonoszok róla.
\par 14 Hogy átváltozzék mint a megpecsételt agyag, és elõálljon, mint egy ruhában.
\par 15 Hogy a gonoszoktól elvétessék világosságuk, és a fölemelt  kar összetöressék?
\par 16 Eljutottál-é a tenger forrásáig, bejártad-é a mélységnek fenekét?
\par 17 Megnyíltak-é néked a halálnak kapui; a halál árnyékának kapuit láttad-é?
\par 18 Áttekintetted-é a föld szélességét, mondd meg, ha mindezt jól tudod?
\par 19 Melyik út visz oda, hol a világosság lakik, és a sötétségnek hol van a helye?
\par 20 Hogy visszavinnéd azt az õ határába, és hogy megismernéd lakása útjait.
\par 21 Tudod te ezt, hiszen már akkor megszülettél; napjaidnak száma nagy!
\par 22 Eljutottál-é a hónak tárházához; vagy a jégesõnek tárházát láttad-é?
\par 23 A mit fentartottam a szükség idejére, a harcz és háború napjára?
\par 24 Melyik út visz oda, a hol szétoszlik a világosság, és szétterjed a keleti szél a földön?
\par 25 Ki hasított nyílást a záporesõnek, és a mennydörgõ villámnak útat?
\par 26 Hogy aláessék az ember nélkül való földre, a pusztaságra, holott senki sincsen;
\par 27 Hogy megitasson pusztát, sivatagot, és hogy sarjaszszon zsenge pázsitot?
\par 28 Van-é atyja az esõnek, és ki szülte a harmat cseppjeit?
\par 29 Kinek méhébõl jött elõ a jég, és az ég daráját kicsoda szülte?
\par 30 Miként rejtõznek el a vizek mintegy kõ alá, és mint zárul be a mély vizek színe?
\par 31 Összekötheted-é a fiastyúk szálait; a kaszáscsillag köteleit megoldhatod-é?
\par 32 A hajnalcsillagot elõhozhatod-é az õ idejében, avagy a gönczölszekeret forgathatod-é fiával együtt?
\par 33 Ismered-é az ég törvényeit, vagy te határozod-é meg uralmát a földön?
\par 34 Felemelheted-é szavadat a felhõig, hogy a vizeknek bõsége beborítson téged?
\par 35 Kibocsáthatod-é a villámokat, hogy elmenjenek, vagy mondják-é néked: Itt vagyunk?
\par 36 Ki helyezett bölcseséget a setét felhõkbe, vagy a tüneményeknek ki adott értelmet?
\par 37 Ki számlálta meg a bárányfelhõket bölcseséggel, és ki üríti ki az égnek tömlõit;
\par 38 Mikor a por híg sárrá változik, és a göröngyök összetapadnak?

\chapter{39}

\par 1 Vadászol-é prédát a nõstény oroszlánnak, és az oroszlánkölykök éhségét kielégíted-é;
\par 2 Mikor meglapulnak tanyáikon, és a bokrok közt lesben vesztegelnek?
\par 3 Ki szerez a hollónak eledelt, mikor a fiai Istenhez kiáltoznak; kóvályognak, mert nincs mit enniök?
\par 4 Tudod-é a kõszáli zergék ellésének idejét; megvigyáztad-é a szarvasok fajzását?
\par 5 Megszámláltad-é a hónapokat, a meddig vemhesek; tudod-é az ellésök idejét?
\par 6 Csak összegörnyednek, elszülik magzataikat, vajudásaiktól megszabadulnak.
\par 7 Fiaik meggyarapodnak, a legelõn nagyranõnek, elszélednek és nem térnek vissza hozzájok.
\par 8 Ki bocsátotta szabadon a vadszamarat, ki oldozta el e szamárnak kötelét,
\par 9 A melynek házául a pusztát rendelém, és lakóhelyéül a sósföldet?
\par 10 Kineveti a városbeli sokadalmat, nem hallja a hajtsár kiáltozását.
\par 11 A hegyeken szedeget, az õ legelõjén mindenféle zöld gazt felkeres.
\par 12 Akar-é szolgálni néked a bölény? Avagy meghál-é a te jászolodnál?
\par 13 Oda kötheted a bölényt a barázdához kötelénél fogva? Vajjon boronálja-é a völgyeket utánad?
\par 14 Bízhatol-é benne, mivelhogy nagy az ereje, és munkádat hagyhatod-é reá?
\par 15 Hiszed-é róla, hogy vetésedet behordja, és szérûdre betakarítja?
\par 16 Vígan leng a struczmadár szárnya: vajjon az eszterág szárnya és tollazata-é az?
\par 17 Hiszen a földön hagyja tojásait, és porral költeti ki!
\par 18 És elfeledi, hogy a láb eltiporhatja, és a mezei vad eltaposhatja azokat.
\par 19 Fiaival oly keményen bánik, mintha nem is övéi volnának; ha fáradsága kárba vész, nem bánja;
\par 20 Mert Isten a bölcseséget elfeledtette vele, értelmet pedig nem adott néki.
\par 21 De hogyha néki ereszkedik, kineveti a lovat és lovagját.
\par 22 Te adsz-é erõt a lónak, avagy a nyakát sörénynyel te ruházod-é fel?
\par 23 Felugraszthatod-é, mint a sáskát? Tüsszögése dicsõ, félelmetes!
\par 24 Lábai vermet ásnak, örvend erejének, a fegyver elé rohan.
\par 25 Neveti a félelmet; nem remeg, nem fordul meg a fegyver elõl;
\par 26 Csörög rajta a tegez, ragyog a kopja és a dárda:
\par 27 Tombolva, nyihogva kapálja a földet, és nem áll veszteg, ha trombita zeng.
\par 28 A trombitaszóra nyerítéssel felel; messzirõl megneszeli az ütközetet, a vezérek lármáját és a csatazajt.
\par 29 A te értelmed miatt van-é, hogy az ölyv repül, és kiterjeszti szárnyait dél felé?
\par 30 A te rendelésedre száll-é fent a sas, és rakja-é fészkét a magasban?
\par 31 A kõsziklán lakik és tanyázik, a sziklák párkányain és bércztetõkön.
\par 32 Onnét kémlel enni való után, messzire ellátnak szemei.
\par 33 Fiai vért szívnak, és a hol dög van, mindjárt ott terem.
\par 34 Szóla továbbá az Úr Jóbnak, és monda:
\par 35 A ki pert kezd a Mindenhatóval, czáfolja meg, és a ki az Istennel feddõdik, feleljen néki!
\par 36 És szóla Jób az Úrnak, és monda:
\par 37 Ímé, én parányi vagyok, mit feleljek néked? Kezemet a szájamra teszem.
\par 38 Egyszer szóltam, de már nem szólok, avagy kétszer, de nem teszem többé!

\chapter{40}

\par 1 Ekkor szóla az Úr Jóbnak a forgószélbõl, és monda:
\par 2 Nosza! övezd fel, mint férfi, derekadat; én kérdezlek, te pedig taníts engem!
\par 3 Avagy semmivé teheted-é te az én igazságomat; kárhoztathatsz-é te engem azért, hogy te igaz légy?
\par 4 És van-é ugyanolyan karod, mint az Istennek, mennydörgõ hangon szólasz-é, mint õ?
\par 5 Ékesítsd csak fel magadat fénynyel és méltósággal, ruházd fel magadat dicsõséggel és fenséggel!
\par 6 Öntsd ki haragodnak tüzét, és láss meg minden kevélyt és alázd meg õket!
\par 7 Láss meg minden kevélyt és törd meg õket, és a gonoszokat az õ helyükön tipord le!
\par 8 Rejtsd el õket együvé a porba, orczájukat kösd be mélységes sötéttel:
\par 9 Akkor én is dicsõítlek, hogy megtartott téged a te jobbkezed!
\par 10 Nézd csak a behemótot, a melyet én teremtettem, a miként téged is, fûvel él, mint az ökör!
\par 11 Nézd csak az erejét az õ ágyékában, és az õ erõsségét hasának izmaiban!
\par 12 Kiegyenesíti farkát, mint valami czédrust, lágyékának inai egymásba fonódnak.
\par 13 Csontjai érczcsövek, lábszárai, mint a vasrudak.
\par 14 Az Isten alkotásainak remeke ez, az õ teremtõje adta meg néki fegyverét.
\par 15 Mert füvet teremnek számára a hegyek, és a mezõ minden vadja ott játszadozik.
\par 16 Lótuszfák alatt heverész, a nádak és mocsarak búvóhelyein.
\par 17 Befedezi õt a lótuszfák árnyéka, és körülveszik õt a folyami fûzfák.
\par 18 Ha árad is a folyó, nem siet; bizton van, ha szájához a Jordán csapna is.
\par 19 Megfoghatják-é õt szemei láttára, vagy átfúrhatják-é az orrát tõrökkel?!

\chapter{41}

\par 1 Kihúzhatod-é a leviáthánt horoggal, leszoríthatod-é a nyelvét kötéllel?
\par 2 Húzhatsz-é gúzst az orrába, az állát szigonynyal átfurhatod-é?
\par 3 Vajjon járul-é elõdbe sok könyörgéssel, avagy szól-é hozzád sima beszédekkel?
\par 4 Vajjon frigyet köt-é veled, hogy fogadd õt örökös szolgádul?
\par 5 Játszhatol-é vele, miként egy madárral; gyermekeid kedvéért megkötözheted-é?
\par 6 Alkudozhatnak-é felette a társak, vagy a kalmárok közt feloszthatják-é azt?
\par 7 Tele rakhatod-é nyársakkal a bõrét, avagy szigonynyal a fejét?
\par 8 Vesd rá a kezedet, de megemlékezzél, hogy a harczot nem ismételed.
\par 9 Ímé, az õ reménykedése csalárd; puszta látása is halálra ijeszt!
\par 10 Nincs oly merész, a ki õt felverje. Ki hát az, a ki velem szállna szembe?
\par 11 Ki adott nékem elébb, hogy azt visszafizessem? A mi az ég alatt van, mind az enyém!
\par 12 Nem hallgathatom el testének részeit, erejének mivoltát, alkotásának szépségét.
\par 13 Ki takarhatja fel ruhája felszínét; két sor foga közé kicsoda hatol be?
\par 14 Ki nyitotta fel orczájának ajtait? Fogainak sorai körül rémület lakik!
\par 15 Büszkesége a csatornás pajzsok, összetartva mintegy szorító pecséttel.
\par 16 Egyik szorosan a másikhoz lapul, hogy közéje levegõ sem megy.
\par 17 Egyik a másikhoz tapad, egymást tartják, egymástól elszakadhatatlanok.
\par 18 Tüsszentése fényt sugároz ki, és szemei, mint a hajnal szempillái.
\par 19 A szájából szövétnekek jõnek ki, és tüzes szikrák omlanak ki.
\par 20 Orrlyukaiból gõz lövel elõ, mint a forró fazékból és üstbõl.
\par 21 Lehellete meggyujtja a holt szenet, és szájából láng lövel elõ.
\par 22 Nyakszirtjén az erõ tanyáz, elõtte félelem ugrándozik.
\par 23 Testének részei egymáshoz tapadtak; kemény önmagában és nem izeg-mozog.
\par 24 Szíve kemény, mint a kõ, oly kemény, mint az alsó malomkõ.
\par 25 Hogyha felkél, hõsök is remegnek; ijedtökben veszteg állnak.
\par 26 Ha éri is a fegyver, nem áll meg benne, legyen bár dárda, kopja vagy kelevéz.
\par 27 Annyiba veszi a vasat, mint a pozdorját, az aczélt, mint a korhadt fát.
\par 28 A nyíl vesszõje el nem ûzi õt, a parittyakövek pozdorjává változnak rajta.
\par 29 Pozdorjának tartja a buzogányütést is, és kineveti a bárd suhogását.
\par 30 Alatta éles cserepek vannak; mint szeges borona hentereg az iszap felett.
\par 31 Felkavarja a mély vizet, mint a fazekat, a tengert olyanná teszi, mint a festékedény.
\par 32 Maga után világos ösvényt hagy, azt hinné valaki, a tenger megõszült.
\par 33 Nincs e földön hozzá hasonló, a mely úgy teremtetett, hogy ne rettegjen.
\par 34 Lenéz minden nagy állatot, õ a király minden ragadozó felett.

\chapter{42}

\par 1 Jób pedig felele az Úrnak, és monda:
\par 2 Tudom, hogy te mindent megtehetsz, és senki téged el nem fordíthat attól, a mit elgondoltál.
\par 3 Ki az - mondod - a ki gáncsolja az örök rendet tudatlanul? Megvallom azért, hogy nem értettem; csodadolgok ezek nékem, és fel nem foghatom.
\par 4 Hallgass hát, kérlek, én hadd beszéljek; én kérdezlek, te pedig taníts meg engem!
\par 5 Az én fülemnek hallásával hallottam felõled, most pedig szemeimmel látlak téged.
\par 6 Ezért hibáztatom magam és bánkódom a porban és hamuban!
\par 7 Miután pedig e szavakat mondotta vala az Úr Jóbnak, szóla a Témánból való Elifáznak: Haragom felgerjedt ellened és két barátod ellen, mert nem szóltatok felõlem igazán, mint az én szolgám, Jób.
\par 8 Most azért vegyetek magatokhoz hét tulkot és hét kost, és menjetek el az én szolgámhoz Jóbhoz, és áldozzatok magatokért égõáldozatot; Jób pedig, az én szolgám, imádkozzék ti érettetek; mert csak az õ személyére tekintek, hogy retteneteset ne cselekedjem veletek, mivelhogy nem szóltatok felõlem igazán, mint az  én szolgám Jób.
\par 9 Elmenének azért a Témánból való Elifáz, a Sukhból való Bildád és a Naamából való Czófár, és úgy cselekedének, a miképen mondotta vala nékik az Úr, és az Úr tekinte Jóbnak személyére.
\par 10 Azután eltávolítá Isten Jóbról a csapást, miután imádkozott vala az õ barátaiért, és kétszeresen visszaadá az Úr Jóbnak mindazt, a mije volt.
\par 11 És beméne hozzá minden fiútestvére és minden leánytestvére és minden elõbbi ismerõse, és evének õ vele együtt kenyeret az õ házában; sajnálkozának felette és vigasztalák õt mindama nyomorúság miatt, a melyet az Úr reá bocsátott vala, és kiki ada néki egy-egy pénzt, és kiki egy-egy aranyfüggõt.
\par 12 Az Úr pedig jobban megáldá a Jób életének végét, mint kezdetét, mert lõn néki tizennégyezer juha, hatezer tevéje, ezer igás ökre és ezer szamara.
\par 13 És lõn néki hét fia és három leánya.
\par 14 És az elsõnek neve vala Jémima a másodiké Kecziha, a harmadiké Kéren-Happuk.
\par 15 És nem találtatnak vala olyan szép leányok, mint a Jób leánya, abban az egész tartományban, és az õ atyjok örökséget is ada nékik az õ fiútestvéreik között.
\par 16 Jób pedig él vala ezután száznegyven esztendeig, és látja vala az õ fiait és unokáit negyedízig.
\par 17 És meghala Jób jó vénségben és betelve az élettel.


\end{document}