\begin{document}

\title{Luke}


\chapter{1}

\par 1 Mivelhogy sokan kezdették rendszerint megírni azoknak a dolgoknak az elbeszélését, a melyek minálunk beteljesedtek,
\par 2 A mint nékünk elõnkbe adták, a kik kezdettõl fogva szemtanúi és szolgái voltak az ígének:
\par 3 Tetszék énnékem is, ki eleitõl fogva mindeneknek szorgalmasan végére jártam, hogy azokról rendszerint írjak néked. jó Theofilus,
\par 4 Hogy megtudhasd azoknak a dolgoknak bizonyosságát, a melyekre taníttatál.
\par 5 Heródesnek, a Júdea királyának idejében vala egy Zakariás nevû pap az Abia rendjébõl; az õ felesége pedig az Áron leányai közül való vala, és annak neve Erzsébet.
\par 6 És mind a ketten igazak valának az Isten elõtt, kik az Úrnak minden parancsolataiban és rendeléseiben feddhetetlenül jártak.
\par 7 És nem volt nékik gyermekük, mert Erzsébet meddõ vala, és mind a ketten immár idõs emberek valának.
\par 8 Lõn pedig, hogy mikor õ rendjének sorában papi szolgálatot végzett az Isten elõtt,
\par 9 A papi tiszt szokása szerint reá jutott a sor, hogy bemenvén az Úrnak templomába, jó illatot gerjesszen.
\par 10 És a népnek egész sokasága imádkozék kívül a jó illatozás idején;
\par 11 Néki pedig megjelenék az Úrnak angyala, állván a füstölõ oltár jobbja felõl.
\par 12 És láttára megrettene Zakariás, és félelem szállá meg õt.
\par 13 Monda pedig az angyal néki: Ne félj Zakariás; mert meghallgattatott a te könyörgésed, és a te feleséged Erzsébet szül néked fiat, és nevezed az õ nevét Jánosnak.
\par 14 És lészen tenéked örömödre és vigasságodra, és sokan fognak örvendezni az õ születésén;
\par 15 Mert nagy lészen az Úr elõtt, és bort és részegítõ italt nem iszik; és betelik Szent Lélekkel még az õ anyjának méhétõl fogva.
\par 16 És az Izrael fiai közül sokakat megtérít az Úrhoz, az õ Istenükhöz.
\par 17 És ez Õ elõtte fog járni az Illés lelkével és erejével, hogy az atyák szívét a fiakhoz térítse, és az engedetleneket az igazak bölcsességére, hogy készítsen az Úrnak tökéletes népet.
\par 18 És monda Zakariás az angyalnak: Mirõl tudhatom én ezt meg? mert én vén vagyok, és az én feleségem is igen idõs.
\par 19 És felelvén az angyal, monda néki: Én Gábriel vagyok, ki az Isten elõtt állok; és küldettem, hogy szóljak veled, és ez örvendetes dolgokat jelentsem néked.
\par 20 És ímé megnémulsz és nem szólhatsz mindama napig, a melyen ezek meglesznek: mivelhogy nem hittél az én beszédimnek, a melyek beteljesednek az õ idejökben.
\par 21 A nép pedig várja vala Zakariást, és csodálkozék, hogy a templomban késik.
\par 22 És kijövén, nem szólhata nékik; eszökbe vevék azért, hogy látást látott a templomban; mert õ csak integetett nékik, és néma maradt.
\par 23 És lõn, hogy mikor leteltek az õ szolgálatának napjai, elméne haza.
\par 24 E napok után pedig fogada méhében Erzsébet az õ felesége, és elrejtõzék öt hónapig, mondván:
\par 25 Így cselekedett velem az Úr a napokban, a melyekben reám tekinte, hogy elvegye az én gyalázatomat az emberek között.
\par 26 A hatodik hónapban pedig elküldeték Gábriel angyal Istentõl Galileának városába, a melynek neve Názáret,
\par 27 Egy szûzhöz, a ki a Dávid házából való József nevû férfiúnak volt eljegyezve. A szûznek neve pedig Mária.
\par 28 És bemenvén az angyal õ hozzá, monda néki: Örülj, kegyelembe fogadott! Az Úr veled van, áldott vagy te az asszonyok között.
\par 29 Az pedig látván, megdöbbene az õ beszédén, és elgondolkodék, hogy micsoda köszöntés ez?!
\par 30 És monda néki az angyal: Ne félj Mária, mert kegyelmet találtál az Istennél.
\par 31 És ímé fogansz a te méhedben, és szülsz fiat, és nevezed az õ nevét  JÉZUSNAK.
\par 32 Ez nagy lészen, és a Magasságos Fiának hivattatik; és néki adja az Úr Isten a Dávidnak, az õ atyjának, királyi székét;
\par 33 És uralkodik a Jákób házán mindörökké; és az õ királyságának vége nem lészen!
\par 34 Monda pedig Mária az angyalnak: Mimódon lesz ez, holott én férfiat nem ismerek?
\par 35 És felelvén az angyal, monda néki: A Szent Lélek száll te reád, és a Magasságosnak ereje árnyékoz meg téged; azért a mi születik is szentnek hivatik, Isten Fiának.
\par 36 És ímé Erzsébet, a te rokonod, õ is fogant fiat az õ vénségében; és ez már a hatodik hónapja néki, a kit meddõnek hívtak:
\par 37 Mert az Istennél semmi sem lehetetlen.
\par 38 Monda pedig Mária: Imhol az Úrnak szolgálója; legyen nékem a te beszéded szerint. És elméne õ tõle az angyal.
\par 39 Fölkelvén pedig Mária azokban a napokban, nagy sietséggel méne a hegységbe, Júdának városába;
\par 40 És beméne Zakariásnak házába, és köszönté Erzsébetet.
\par 41 És lõn, mikor hallotta Erzsébet Mária köszöntését, a magzat repese az õ méhében; és betelék Erzsébet Szent Lélekkel;
\par 42 És fennszóval kiálta, mondván: Áldott vagy te az asszonyok között, és áldott a te méhednek gyümölcse.
\par 43 És honnét van ez nékem, hogy az én Uramnak anyja jön én hozzám?
\par 44 Mert ímé, mihelyt a te köszöntésednek szava füleimbe hatolt, a magzat örvendezéssel kezde repesni az én méhemben.
\par 45 És boldog az, a ki hitt; mert beteljesednek azok, a miket az Úr néki mondott.
\par 46 Akkor monda Mária: Magasztalja az én lelkem az Urat,
\par 47 És örvendez az én lelkem az én megtartó Istenemben.
\par 48 Mert reá tekintett az õ szolgáló leányának alázatos állapotjára; mert ímé mostantól fogva boldognak mondanak engem minden nemzetségek.
\par 49 Mert nagy dolgokat cselekedék velem a Hatalmas; és szent az õ neve!
\par 50 És az õ irgalmassága nemzetségrõl nemzetségre vagyon azokon, a kik õt félik.
\par 51 Hatalmas dolgot cselekedék karjának ereje által,elszéleszté az õ szívök gondolatában felfuvalkodottakat.
\par 52 Hatalmasokat dönte le trónjaikról, és alázatosakat magasztalt fel.
\par 53 Éhezõket töltött be javakkal, és gazdagokat küldött el üresen.
\par 54 Felvevé Izráelnek, az õ szolgájának ügyét, hogy megemlékezzék az õ irgalmasságáról.
\par 55 (A miképen szólott volt a mi atyáinknak), Ábrahám iránt és az õ magva iránt mindörökké!
\par 56 Marada pedig Mária Erzsébettel mintegy három hónapig; azután haza tére.
\par 57 Erzsébetnek pedig betelék az õ szülésének ideje, és szûle fiat.
\par 58 És meghallák az õ szomszédai és rokonai, hogy az Úr nagy kegyelmességet cselekedett õ vele; és együtt örülének vele.
\par 59 És lõn nyolczad napon, eljövének, hogy körülmetéljék a gyermeket; és az õ atyja nevérõl Zakariásnak akarák õt nevezni.
\par 60 És felelvén az õ anyja, monda: Nem; hanem Jánosnak neveztessék.
\par 61 És mondának néki: Senki sincs a te rokonságodban, a ki ezen a néven neveztetnék.
\par 62 És intének az õ atyjának, hogy minek akarja neveztetni?
\par 63 Az pedig táblát kérvén, ezt írá, mondván: János a neve. És elcsodálkozának mindnyájan.
\par 64 És feloldódék az õ szája és nyelve azonnal, és szóla, áldván az Istent.
\par 65 És félelem szállott minden õ szomszédaikra; és Júdeának egész hegyes tartományában elhirdettetének mind e dolgok.
\par 66 És szívökre vevék mindenek, a kik hallák, mondván: Vajjon mi lesz e gyermekbõl? És az Úrnak keze vala õ vele.
\par 67 És Zakariás, az õ atyja beteljesedék Szent Lélekkel, és prófétála mondván:
\par 68 Áldott az Úr, Izráel Istene, hogy meglátogatta és megváltotta az õ népét,
\par 69 És felemelte az üdvösségnek szarvát nékünk az õ gyermekének, Dávidnak házában,
\par 70 A mint szólott az õ szent prófétáinak szája által, kik eleitõl fogva voltak,
\par 71 Hogy a mi ellenségeinktõl megszabadít, és mindazoknak kezébõl, a kik minket gyûlölnek;
\par 72 Hogy irgalmasságot cselekedjék a mi atyáinkkal, és megemlékezzék az õ szent szövetségérõl,
\par 73 Az esküvésrõl, a melylyel megesküdt Ábrahámnak, a mi atyánknak, hogy õ megadja nékünk,
\par 74 Hogy megszabadulván a mi ellenségeink kezébõl, félelem nélkül szolgáljunk néki.
\par 75 Szentségben és igazságban õ elõtte a mi életünknek minden napjaiban.
\par 76 Te pedig kis gyermek, a magasságos Isten prófétájának hivattatol; mert az Úr elõtt jársz, hogy az õ útait megkészítsed;
\par 77 És az üdvösség ismeretére megtanítsad az õ népét, a bûnöknek bocsánatjában.
\par 78 A mi Istenünk nagy irgalmasságáért, a melylyel meglátogatott minket a naptámadat a magasságból,
\par 79 Hogy megjelenjék azoknak, a kik a sötétségben és a halálnak árnyékában ülnek; hogy igazgassa a mi lábainkat a békességnek útjára!
\par 80 A kis gyermek pedig nevekedik és erõsödik vala lélekben; és a pusztában vala mind ama napig, a melyen megmutattamagát az Izráelnek.

\chapter{2}

\par 1 És lõn azokban a napokban, Augusztus császártól parancsolat adaték ki, hogy mind az egész föld összeirattassék.
\par 2 Ez az összeírás elõször akkor történt, mikor Siriában Czirénius volt a helytartó.
\par 3 Mennek vala azért mindenek, hogy beirattassanak, kiki a maga városába.
\par 4 Felméne pedig József is Galileából, Názáret városából Júdeába,a Dávid városába, mely Bethlehemnek neveztetik, mivelhogy a Dávid házából és háznépe közül való volt;
\par 5 Hogy beirattassék Máriával, a ki nékijegyeztetett feleségül és várandós vala.
\par 6 És lõn, hogy mikor ott valának, betelének az õ szülésének napjai.
\par 7 És szülé az õ elsõszülött fiát; és bepólyálá õt, és helyhezteté õt a jászolba, mivelhogynem vala nékik helyök a vendégfogadó háznál.
\par 8 Valának pedig pásztorok azon a vidéken, a kik künn a mezõn tanyáztak és vigyáztak éjszakán az õ nyájok mellett.
\par 9 És ímé az Úrnak angyala hozzájok jöve, és az Úrnak dicsõsége körülvevé õket: és nagy félelemmel megfélemlének.
\par 10 És monda az angyal nékik: Ne féljetek, mert ímé hirdetek néktek nagy örömet, mely az egész népnek öröme lészen:
\par 11 Mert született néktek ma a Megtartó, ki az Úr Krisztus, a Dávid városában.
\par 12 Ez pedig néktek a jele: találtok egy kis gyermeket bepólyálva feküdni a jászolban.
\par 13 És hirtelenséggel jelenék az angyallal mennyei seregek sokasága, a kik az Istent dícsérik és ezt mondják vala:
\par 14 Dicsõség a magasságos mennyekben az Istennek, és e földön békesség, és az emberekhez jó akarat!
\par 15 És lõn, hogy mikor elmentek az angyalok õ tõlök a mennybe, mondának a pásztoremberek egymásnak: Menjünk el mind Bethlehemig, és lássuk meg e dolgot, a melyet az Úr megjelentett nékünk.
\par 16 Elmenének azért sietséggel, és megtalálák Máriát és Józsefet, és a kis gyermeket, ki a jászolban fekszik vala.
\par 17 És ezt látván, elhirdeték, a mi nékik a gyermek felõl mondatott vala.
\par 18 És mindenek, a kik hallák, elcsodálkozának azokon, a miket a pásztorok nékik mondottak.
\par 19 Mária pedig mind ez ígéket megtartja, és szívében forgatja vala.
\par 20 A pásztorok pedig visszatérének dicsõítvén és dícsérvén az Istent mind azok felõl, a miket hallottak és láttak,a mint nékik megmondatott.
\par 21 És mikor betölt a nyolcz nap, hogy a kis gyermeket körülmetéljék, nevezék az õ nevét Jézusnak, a mint õt az angyal nevezte, mielõtt fogantatott volna anyja méhében.
\par 22 Mikor pedig betöltek Mária tisztulásának napjai a Mózes törvénye szerint, felvivék õt Jeruzsálembe, hogy bemutassák az Úrnak.
\par 23 (A mint megiratott az Úr törvényében, hogy: Minden elsõszülött fiú az Úrnak szenteltessék),
\par 24 És hogy áldozatot adjanak, a szerint a mint megmondatott az Úr törvényében: Egy pár gerliczét, vagy két galambfiat.
\par 25 És ímé vala Jeruzsálemben egy ember, a kinek neve Simeon volt, és ez az ember igaz és istenfélõ vala, a ki várta az Izráel vigasztalását, és a Szent Lélek vala õ rajta.
\par 26 És kijelentetett néki a Szent Lélek által, hogy addig halált nem lát, a míg meg nem látja az Úrnak Krisztusát.
\par 27 És õ a Lélek indításából a templomba méne,és mikor a gyermek Jézust bevivék szülõi, hogy õ érette a törvény szokása szerint cselekedjenek,
\par 28 Akkor õ karjaiba vevé õt, és áldá az Istent, és monda:
\par 29 Mostan bocsátod el, Uram, a te szolgádat, a te beszéded szerint, békességben:
\par 30 Mert látták az én szemeim a te üdvösségedet,
\par 31 A melyet készítettél minden népeknek szeme láttára;
\par 32 Világosságul a pogányok megvilágosítására, és a te népednek, az Izráelnek dicsõségére.
\par 33 József pedig és az õ anyja csodálkozának azokon, a miket õ felõle mondottak.
\par 34 És megáldá õket Simeon, és monda Máriának, az õ anyjának: Ímé ez vettetett sokaknak elestére és feltámadására az Izráelben; és jegyül,a kinek sokan ellene  mondanak;
\par 35 Sõt a te lelkedet is általhatja az éles tõr; hogy sok szív gondolatai nyilvánvalókká legyenek.
\par 36 És vala egy prófétaasszony, Anna, a Fánuel leánya, az Áser nemzetségébõl (ez sok idõt élt, miután az õ szûzességétõl fogva hét esztendeig élt férjével,
\par 37 És ez mintegy nyolczvannégy esztendõs özvegy vala), a ki nem távozék el a templomból, hanem bõjtölésekkel és imádkozásokkal szolgál vala éjjel és nappal.
\par 38 Ez is ugyanazon órában oda állván, hálát adott az Úrnak, és szóla õ felõle mindeneknek, a kik Jeruzsálemben a váltságot várták.
\par 39 És mikor mindent elvégeztek az Úr törvénye szerint, visszatérének Galileába, az õ városukba, Názáretbe.
\par 40 A kis gyermek pedig növekedék, és erõsödék lélekben, teljesedve bölcsességgel; és az Istennek kegyelme vala õ rajta.
\par 41 Az õ szülei pedig évenként feljártak Jeruzsálembe a húsvét ünnepére.
\par 42 És mikor tizenkét esztendõs lett, fölmenének Jeruzsálembe az ünnep szokása szerint;
\par 43 És mikor eltelének a napok, mikor õk visszatérének, a gyermek Jézus visszamarada Jeruzsálemben; és nem vevék észre sem József, sem az õ anyja;
\par 44 Hanem azt gondolván, hogy az úti társaságban van, egy napi járó földet menének, és keresék õt a rokonok és az ismerõsök között;
\par 45 És mikor nem találák õt, visszamenének Jeruzsálembe, hogy megkeressék.
\par 46 És lõn, hogy harmadnapra megtalálták õt a templomban, a doktorok között ülve, a mint õket hallgatta, és kérdezgette õket.
\par 47 És mindnyájan, a kik õt hallgatták, elálmélkodának az õ értelmén és az õ feleletein.
\par 48 És meglátván õt, elcsodálkozának, és monda néki az õ anyja: Fiam, miért cselekedted ezt velünk? Ímé atyád és én nagy bánattal kerestünk téged.
\par 49 Õ pedig monda nékik: Mi dolog, hogy engem kerestetek? Avagy nem tudjátok-é, hogy nékem azokban kell foglalatosnak lennem, a melyek az én atyámnak dolgai?
\par 50 De õk nem érték e beszédet, a mit õ nékik szóla.
\par 51 És aláméne velök, és méne Názáretbe; és engedelmes vala nékik. És az õ anyja szívében tartá mind ezeket a dolgokat.
\par 52 Jézus pedig gyarapodék bölcsességben és testének állapotjában, és az Isten és emberek elõtt való kedvességben.

\chapter{3}

\par 1 Tibérius császár uralkodásának tizenötödik esztendejében pedig, mikor Júdeában Ponczius Pilátus volt a helytartó, és Galileának negyedes fejedelme Heródes, Iturea és Trakhónitis tartományának pedig negyedes fejedelme az õ testvére Filep, Abiléné negyedes fejedelme meg Lisániás,
\par 2 Annás és Kajafás fõpapsága alatt, lõn az Úrnak szava Jánoshoz, a Zakariás fiához,  a pusztában,
\par 3 És méne a Jordán mellett lévõ minden tartományba prédikálván a megtérés keresztségét a bûnöknek bocsánatjára;
\par 4 A mint meg van írva Ésaiás próféta beszédeinek könyvében, ki ezt mondja: Kiáltónak szava a pusztában: Készítsétek meg az Úrnak útját, egyengessétek az õ ösvényeit.
\par 5 Minden völgy betöltetik, minden hegy és halom megalacsonyíttatik; és az egyenetlenek egyenesekké, és a göröngyös útak símákká lesznek;
\par 6 És meglátja minden test az Istennek szabadítását.
\par 7 Monda azért a sokaságnak, a mely kiméne hozzá, hogy általa megkereszteltessék: Viperák fajzati, kicsoda intett meg titeket, hogy a bekövetkezõ harag elõl meneküljetek?
\par 8 Teremjetek azért megtéréshez méltó gyümölcsöket, és ne mondogassátok magatokban: Ábrahám a mi atyánk! mert mondom néktek, hogy az Isten ezekbõl a kövekbõl is támaszthat fiakat Ábrahámnak.
\par 9 Immár pedig a fejsze is rávettetett a fák gyökerére: minden fa azért, a mely jó gyümölcsöt nem terem, kivágattatik és a tûzre vettetik.
\par 10 És megkérdé õt a sokaság, mondván: Mit cselekedjünk tehát?
\par 11 Õ pedig felelvén, monda nékik: A kinek két köntöse van, egyiket adja annak, a kinek nincs; és a kinek van eledele, hasonlókép cselekedjék.
\par 12 És eljövének a vámszedõk is, hogy megkeresztelkedjenek, és mondának néki: Mester, mit cselekedjünk?
\par 13 Õ pedig monda nékik: Semmi többet ne követeljetek, mint a mi elõtökbe rendeltetett.
\par 14 És megkérdék õt a vitézek is, mondván: Hát mi mit cselekedjünk? És monda nékik: Senkit se háborítsatok, se ne patvarkodjatok; és elégedjetek meg zsoldotokkal.
\par 15 Mikor pedig a nép várt és szívökben mind azon gondolkoztak János felõl, hogy vajjon nem õ-é a Krisztus;
\par 16 Felele János mindeneknek, mondván: Én ugyan keresztellek titeket vízzel; de eljõ, a ki nálamnál erõsebb, a kinek nem vagyok méltó, hogy sarujának kötõjét megoldjam: az majd keresztel titeket Szent Lélekkel  és tûzzel:
\par 17 Kinek szórólapátja kezében van, és megtisztítja szérûjét; és a gabonát az õ csûrébe takarja, a polyvát pedig megégeti olthatatlan tûzzel.
\par 18 És még sok egyebekre is intvén õket, hirdeté az evangyéliomot a népnek.
\par 19 Mikor pedig Heródes, a negyedes fejedelem, megfeddetett õ tõle Heródiásért, az õ testvérének, Filepnek feleségéért és mindama gonoszságokért, a miket Heródes cselekedett,
\par 20 Ez még azzal tetézte mindezeket, hogy Jánost tömlöczbe vetteté.
\par 21 Lõn pedig, hogy mikor az egész nép megkeresztelkedett, és Jézus is megkereszteltetett, és imádkozott, megnyilatkozék az ég,
\par 22 És leszálla õ reá a Szent Lélek testi ábrázatban mint egy galamb, és szózat lõn mennybõl, ezt mondván: Te vagy amaz én szerelmes Fiam,  te benned gyönyörködöm!
\par 23 Maga Jézus pedig mintegy harmincz esztendõs volt, mikor tanítani kezdett, ki, a mint állítják vala, a József fia vala, ez pedig a Hélié,
\par 24 Ez Mattáté, ez Lévié, ez Melkié, ez Jannáé, ez Józsefé,
\par 25 Ez Matthatiásé, ez Ámosé, ez Naumé, ez Eslié, ez Naggaié,
\par 26 Ez Maáté, ez Matthatiásé, ez Sémeié, ez Józsefé, ez Júdáé,
\par 27 Ez Joannáé, ez Rhésáé, ez Zorobábelé, ez Saláthielé, ez Nérié,
\par 28 Ez Melkié, ez Addié, ez Hosámé, ez Elmodámé, ez Éré,
\par 29 Ez Jóséé, ez Eliézeré, ez Jórimé, ez Mattáté, ez Lévié,
\par 30 Ez Simeoné, ez Júdáé, ez Józsefé, ez Jónáné, ez Eliákimé,
\par 31 Ez Méleáé, ez Maináné, ez Mattátáé, ez Nátáné, ez Dávidé,
\par 32 Ez Jesséé, ez Obedé, ez Boázé, ez Sálmoné, ez Naássoné,
\par 33 Ez Aminádábé, ez Arámé, ez Esroné, ez Fáresé, ez  Júdáé.
\par 34 Ez Jákóbé, ez Izsáké, ez Ábrahámé, ez Táréé, ez Nákhoré,
\par 35 Ez Sárukhé, ez Ragávé, ez Fáleké, ez Eberé, ez Saláé,
\par 36 Ez Kajnáné, ez Arfaksádé, ez Semé, ez Noéé, ez Lámekhé,
\par 37 Ez Mathuséláé, ez Énókhé, ez Járedé, ez Mahalaléelé, ez Kajnáné,
\par 38 Ez Énósé, ez Sethé, ez  Ádámé, ez pedig az Istené.

\chapter{4}

\par 1 Jézus pedig Szent Lélekkel telve, visszatére a Jordántól, és viteték a  Lélektõl a pusztába
\par 2 Negyven napig, kísértetvén az ördög által. És nem evék semmit azokban a napokban; de mikor azok elmúltak, végre megéhezék.
\par 3 És monda néki az ördög. Ha Isten Fia vagy, mondd e kõnek, hogy változzék kenyérré.
\par 4 Jézus pedig felele néki, mondván: Meg van írva, hogy nemcsak kenyérrel él az ember, hanem az Istennek minden ígéjével.
\par 5 Majd felvivén õt az ördög egy nagy magas hegyre, megmutatá néki e föld minden országait egy szempillantásban,
\par 6 És monda néki az ördög: Néked adom mindezt a hatalmat és ezeknek dicsõségét; mert nékem adatott, és annak adom, a kinek akarom;
\par 7 Azért ha te engem imádsz, mindez a tied lesz.
\par 8 Felelvén pedig Jézus, monda néki: Távozz tõlem, Sátán; mert meg van írva: Az Urat, a te Istenedet imádd, és csak néki szolgálj.
\par 9 Azután Jeruzsálembe vivé õt, és a templom ormára állítván, monda néki: Ha Isten Fia vagy, vesd alá magad innét;
\par 10 Mert meg van írva: Az õ angyalinak parancsol te felõled, hogy megõrizzenek téged;
\par 11 És: Kezökben hordoznak téged, hogy valamikép meg ne üssed lábadat a kõbe.
\par 12 Felelvén pedig Jézus, monda néki: Megmondatott: Ne kísértsd az Urat, a te Istenedet.
\par 13 És elvégezvén minden kísértést az ördög, eltávozék tõle egy idõre.
\par 14 Jézus pedig megtére a Léleknek erejével Galileába: és híre méne néki az egész környéken.
\par 15 És õ taníta azoknak zsinagógáiban, dicsõíttetvén mindenektõl.
\par 16 És méne Názáretbe, a hol felneveltetett: és beméne, szokása szerint, szombatnapon a zsinagógába, és felálla olvasni.
\par 17 És adák néki az Ésaiás próféta könyvét; és a könyvet feltárván, arra a helyre nyita, a hol ez vala írva:
\par 18 Az Úrnak lelke van és rajtam, mivelhogy felkent engem, hogy a szegényeknek az evangyéliomot hirdessem, elküldött, hogy a töredelmes szívûeket meggyógyítsam, hogy a foglyoknak szabadulást hirdessek és a vakok szemeinek megnyilását, hogy szabadon bocsássam a lesujtottakat,
\par 19 Hogy hirdessem az Úrnak kedves esztendejét.
\par 20 És behajtván a könyvet, átadá a szolgának, és leüle. És a zsinagógában mindenek szemei õ reá valának függesztve.
\par 21 Õ pedig kezde hozzájuk szólani: Ma teljesedett be ez az Írás a ti hallástokra.
\par 22 És mindnyájan bizonyságot tõnek felõle, és elálmélkodának kedves beszédein, a melyek szájából származtak, és mondának: Avagy nem a József fia-é ez?
\par 23 És monda nékik: Bizonyára azt a példabeszédet mondjátok nékem: Orvos, gyógyítsd meg magadat! A miket hallottunk, hogy Kapernaumban történtek, itt a te  hazádban is cselekedd meg azokat.
\par 24 Monda pedig: Bizony mondom néktek: Egy próféta sem kedves az õ hazájában.
\par 25 És igazán mondom néktek, hogy Illés idejében sok özvegy asszony volt Izráelben, mikor az ég három esztendeig és hat hónapig be volt zárva, úgy hogy az egész tartományban nagy éhség volt;
\par 26 Mégis azok közül senkihez nem küldetett Illés, hanem csak Sidonnak Sareptájába az özvegy asszonyhoz.
\par 27 És az Elizeus próféta idejében sok bélpoklos volt Izráelben; de azok közül egy sem tisztult meg, csak a Siriából való Naámán.
\par 28 És betelének mindnyájan haraggal a zsinagógában, mikor ezeket hallották.
\par 29 És felkelvén, kiûzék õt a városon kívül és vivék õt annak a hegynek szélére, a melyen az õ városuk épült, hogy onnen letaszítsák.
\par 30 Õ azonban közöttük átmenve, eltávozék.
\par 31 És leméne Kapernaumba, Galilea városába; és tanítja vala azokat szombatnapokon.
\par 32 És csodálkozának az õ tudományán, mert beszéde hatalmas vala.
\par 33 És a zsinagógában vala egy tisztátalan ördögi lélektõl megszállt ember, a ki fennhangon kiálta,
\par 34 Mondván: Ah, mi közünk hozzád názáreti Jézus? Jöttél, hogy elveszíts minket? Ismerlek téged ki vagy: az Istennek ama Szentje!
\par 35 És megdorgálá õt Jézus, mondván: Némulj meg és menj ki ez emberbõl! És az ördög azt a középre vetvén, kiméne belõle, és nem árta néki semmit.
\par 36 És támada félelem mindenekben, és egymással szólnak és beszélnek vala, mondván: Mi dolg ez, hogy nagy méltósággal és hatalommal parancsol a tisztátalan lelkeknek és kimennek?
\par 37 És elterjede a hír õ felõle a környék minden helyén.
\par 38 Azután a zsinagógából eltávozván, a Simon házába méne. A Simon napa pedig nagy hideglelésben feküdt, és könyörögtek neki érette.
\par 39 És Jézus mellé állván, megdorgálá a hideglelést, és az elhagyá õt; és õ azonnal felkelvén, szolgála nékik.
\par 40 A nap lementével pedig, mindenek, a kiknek különféle betegeik valának, õ hozzá vivék azokat; õ pedig mindegyikõjükre reávetvén kezeit, meggyógyítá õket.
\par 41 Sokakból pedig ördögök is mentek ki, kiáltozván és mondván: Te vagy ama Krisztus, az Isten Fia! De õ megdorgálván, nem engedé õket szólani, mivelhogy tudták,  hogy õ a Krisztus.
\par 42 A nap fölkeltekor pedig kimenvén, puszta helyre méne; de a sokaság felkeresé õt, és hozzámenének, és tartóztaták õt, hogy ne menjen el tõlök.
\par 43 Õ pedig monda nékik: Egyéb városoknak is hirdetnem kell nékem az Istennek országát; mert azért küldettem.
\par 44 És prédikál vala Galilea zsinagógáiban.

\chapter{5}

\par 1 És lõn, hogy mikor a sokaság hozzá tódult, hogy hallgassa az Isten beszédét, õ a Genezáret tavánál áll vala;
\par 2 És láta két hajót állani a vizen: a halászok pedig, miután azokból kiszállottak, mossák vala az õ hálóikat.
\par 3 És õ bemenvén az egyik hajóba, a mely a Simoné vala, kéré õt, hogy vigye egy kissé beljebb a földtõl: és mikor leült, a hajóból tanítá a sokaságot.
\par 4 Mikor pedig megszünt beszélni, monda Simonnak: Evezz a mélyre, és vessétek ki hálóitokat fogásra.
\par 5 És felelvén Simon, monda néki: Mester, jóllehet az egész éjszaka fáradtunk, még sem fogtunk semmit: mindazáltal a te parancsolatodra levetem a hálót.
\par 6 És ezt megtévén, halaknak nagy sokaságát keríték be; szakadoz vala pedig az õ hálójuk.
\par 7 Intének azért társaiknak, a kik a másik hajóban valának, hogy jõjjenek és segítsenek nékik. És eljövén, megtölték mind a két hajót, annyira, hogy csaknem elsülyedének.
\par 8 Látván pedig ezt Simon Péter, Jézusnak lábai elé esik, mondván: Eredj el én tõlem, mert én bûnös ember vagyok, Uram!
\par 9 Mert félelem fogta körül õt és mindazokat, a kik õ vele valának, a halfogás miatt, a melyet fogtak;
\par 10 Hasonlóképen Jakabot és Jánost is, a Zebedeus fiait, a kik Simonnak társai valának. És monda Simonnak Jézus: Ne félj; mostantól fogva embereket fogsz.
\par 11 És a hajókat a szárazra vonván, elhagyák mindenöket és követék õt.
\par 12 És lõn, hogy mikor az egyik városban vala, ímé vala ott egy poklossággal teljes ember: és mikor meglátta Jézust, arczra borulva kéré õt, mondván: Uram, ha akarod, megtisztíthatsz engem!
\par 13 Jézus pedig kinyújtván kezét, illeté azt, mondván: Akarom, tisztulj meg. És azonnal eltávozék tõle a bélpoklosság.
\par 14 És õ megparancsolá néki, hogy azt senkinek se mondja el; hanem eredj el, úgymond, mutasd meg magad a papnak, és vígy áldozatot a te megtisztulásodért, a mint Mózes parancsolta, bizonyságul õ nékik.
\par 15 A hír azonban annál inkább terjedt õ felõle; és nagy sokaság gyûle egybe, hogy õt hallgassák, és hogy általa meggyógyuljanak az õ betegségeikbõl.
\par 16 De õ félrevonula a pusztákba, és imádkozék.
\par 17 És lõn egy napon, hogy õ tanít vala: és ott ülének a farizeusok és a törvénynek tanítói, a kik jöttek Galileának és Júdeának minden faluiból és Jeruzsálembõl: és az Úrnak hatalma vala õ vele, hogy gyógyítson.
\par 18 És ímé valami férfiak ágyon egy embert hozának, a ki gutaütött vala; és igyekezének azt bevinni és õ elébe tenni.
\par 19 De nem találván módot, hogy a sokaság miatt mikép vigyék õt be, felhágának a háztetõre, és a cseréphéjazaton át bocsáták õt alá ágyastól Jézus elé a középre.
\par 20 És látván azoknak hitét, monda: Ember, megbocsáttattak néked a te bûneid.
\par 21 Az írástudók pedig és a farizeusok elkezdének tanakodni, mondván: Kicsoda ez, a ki ily káromlást szól? Ki bocsáthatja meg a bûnt, hanemha egyedül az Isten?
\par 22 Jézus pedig észrevévén az õ tanakodásukat, felelvén, monda nékik: Mit tanakodtok a ti szívetekben?
\par 23 Melyik könnyebb, azt mondani: Megbocsáttattak néked a te bûneid; vagy azt mondani: Kelj fel és járj?
\par 24 Hogy pedig megtudjátok, hogy az ember Fiának van hatalma e földön megbocsátani a bûnöket, (monda a gutaütöttnek): Néked mondom, kelj fel, és fölvévén nyoszolyádat, eredj haza!
\par 25 És az rögtön felkelvén azok szemeláttára, fölvevé a min feküdt, és elméne haza, dicsõítvén az Istent.
\par 26 És az álmélkodás elfogá mindnyájukat, és dicsõíték az Istent, és betelének félelemmel, mondván: Bizony csodadolgokat láttunk ma!
\par 27 Ezek után pedig kiméne, és láta egy Lévi nevû vámszedõt, a ki a vámnál ül vala, és monda néki: Kövess engem!
\par 28 És az mindeneket elhagyván, felkele és követé õt.
\par 29 És Lévi nagy lakomát készíte néki az õ házánál; és vala ott nagy sokasága a vámszedõknek és egyebeknek, a kik õ velök telepedtek volt.
\par 30 És köztük az írástudók és farizeusok zúgolódának az õ tanítványai ellen, mondván: Miért esztek és isztok a vámszedõkkel és a bûnösökkel?
\par 31 És felelvén Jézus, monda nékik: Az egészségeseknek nincs szükségük orvosra, hanem a betegeknek.
\par 32 Nem azért jöttem, hogy az igazakat hívjam, hanem a bûnösöket a megtérésre.
\par 33 Azok pedig mondának néki: Mi az oka, hogy a János tanítványai gyakorta bõjtölnek és imádkoznak, valamint a farizeusokéi is; a te tanítványaid pedig esznek és isznak?
\par 34 Õ pedig monda nékik: Avagy mívelhetitek-é azt, hogy a lakodalmasok bõjtöljenek, a míg a võlegény velök van?
\par 35 De eljõnek a napok, és mikor a võlegény elvétetik õ tõlök, akkor majd bõjtölnek azokban a napokban.
\par 36 És monda nékik példabeszédet is: Senki nem toldja az új posztó foltot az ó posztóhoz; mert különben az újat is megszakasztja és az ó posztóhoz nem illik az újból való folt.
\par 37 És senki sem tölté az új bort ó tömlõkbe; mert különben az új bor megszakasztja a tömlõket, és a bor kiömöl, és a tömlõk is elvesznek.
\par 38 Hanem az új bort új tömlõkbe kell tölteni, és mind a kettõ megmarad.
\par 39 És senki, a ki ó bort iszik, mindjárt újat nem kiván, mert azt mondja: Jobb az ó.

\chapter{6}

\par 1 Lõn pedig a húsvét szombatját követõ második szombaton, hogy a vetések között méne által és az õ tanítványai  gabonafejeket szaggatván és azokat kezeikkel kimorzsolván, ettek.
\par 2 Némelyek pedig a farizeusok közül mondának nékik: Miért cselekszitek azt, amit szombatnapokon nem szabad cselekedni?
\par 3 És felelvén Jézus, monda nékik: Nem olvastátok-é, mit cselekedett Dávid, mikor megéhezett õ és a kik vele voltak?
\par 4 Mi módon ment be az Úrnak házába és vette el a szent kenyereket és ette meg és adott azoknak is, a kik vele voltak, a melyeket pedig nem szabad megenni, hanem csak a papoknak?
\par 5 És monda nékik: Az embernek Fia ura a szombatnak is.
\par 6 Lõn pedig más szombaton is, hogy õ a zsinagógába méne és taníta, és vala ott egy ember, a kinek a jobb keze száradt volt.
\par 7 Az írástudók és farizeusok pedig leselkedének õ utána, ha vajjon gyógyít-e majd szombatnapon, hogy vádat találjanak ellene.
\par 8 Õ pedig tudván azoknak gondolatait, monda a száradt kezû embernek: Kellj fel és állj elõ! És felkelvén, elõálla.
\par 9 Monda azért nékik Jézus: Valamit kérdek tõletek: Szabad-é szombaton jót tenni, vagy rosszat tenni? az életet megtartani, vagy elveszteni?
\par 10 És körültekinte mindnyájokon, monda az embernek: Nyújtsd ki a kezedet! Az pedig úgy cselekedék, és keze oly éppé lõn, mint a másik.
\par 11 Azok pedig eltelének esztelenséggel és beszélgetnek vala egymás közt, hogy mit cselekedjenek Jézussal?
\par 12 És lõn azokban a napokban, kiméne a hegyre imádkozni, és az éjszakát az Istenhez való imádkozásban tölté el.
\par 13 És mikor megvirrada, elõszólítá az õ tanítványait és kiválaszta azok közül tizenkettõt, a kiket apostoloknak is neveze:
\par 14 Simont, a kit Péternek is neveze, és Andrást, annak testvérét, Jakabot és Jánost, Filepet és Bertalant,
\par 15 Mátét és Tamást, Jakabot, az Alfeus fiát, és Simont, a ki Zelotesnek nevezteték,
\par 16 Júdást, a Jakab fiát és Iskariotes Júdást, a ki árulóvá is lõn;
\par 17 És alámenvén õ velök, megálla a síkságon, és az õ tanítványainak serege és a népnek nagy sokasága egész Júdeából és Jeruzsálembõl és Tírusnak és Sídonnak tengermelléki határából, a kik jöttek, hogy hallgassák õt és meggyógyíttassanak betegségeikbõl.
\par 18 És a kik tisztátalan lelkektõl gyötrettek, meggyógyulának.
\par 19 És az egész sokaság igyekezik vala õt illetni: mert erõ származék belõle, és mindeneket meggyógyíta.
\par 20 Õ pedig felemelvén szemeit az õ tanítványaira, monda: Boldogok vagytok ti szegények: mert tiétek az Isten országa.
\par 21 Boldogok ti, kik most éheztek: mert megelégíttettek. Boldogok ti, kik most sírtok:  mert nevetni fogtok.
\par 22 Boldogok lesztek, mikor titeket az emberek gyûlölnek, és kirekesztenek, és szidalmaznak titeket, és kivetik a ti neveteket, mint gonoszt, az embernek Fiáért.
\par 23 Örüljetek azon a napon és örvendezzetek; mert ímé a ti jutalmatok bõséges a mennyben; hiszen hasonlóképen cselekedtek a prófétákkal az õ atyáik.
\par 24 De jaj néktek, gazdagoknak, mert elvettétek a ti vigasztalástokat.
\par 25 Jaj néktek, kik beteltetek; mert éhezni fogtok. Jaj néktek, kik most nevettek; mert sírni és jajgatni fogtok.
\par 26 Jaj néktek, mikor minden ember jót mond felõletek; mert épen így cselekedtek a hamis prófétákkal az õ atyáik.
\par 27 De néktek mondom, kik engem hallgattok: Szeressétek ellenségeiteket, jól tegyetek azokkal, a kik titeket gyûlölnek,
\par 28 Áldjátok azokat, a kik titeket átkoznak, és imádkozzatok  azokért, a kik titeket háborgatnak.
\par 29 A ki egyik arczodat megüti, fordítsd néki a másikat is; és attól, a ki felsõ ruhádat elveszi, ne vond meg alsó ruhádat se.
\par 30 Mindennek pedig, a ki tõled kér, adj; és attól, a ki elveszi a tiédet, ne kérd vissza.
\par 31 És a mint akarjátok, hogy az emberek veletek cselekedjenek, ti is akképen cselekedjetek azokkal.
\par 32 Mert ha csak azokat szeretitek, a kik titeket szeretnek, mi jutalmatok van? Hiszen a bûnösök is szeretik azokat, a kik õket szeretik.
\par 33 És ha csak azokkal tesztek jól, a kik veletek jól tesznek, mi jutalmatok van? Hiszen a bûnösök is ugyanazt cselekszik.
\par 34 És ha csak azoknak adtok kölcsönt, a kiktõl reménylitek, hogy visszakapjátok, mi jutalmatok van? Hiszen a bûnösök is adnak kölcsönt a bûnösöknek, hogy ugyanannyit kapjanak vissza.
\par 35 Hanem szeressétek ellenségeiteket, és jól tegyetek, és adjatok kölcsönt, semmit érte nem várván; és a ti jutalmatok sok lesz, és ama magasságos Istennek fiai lesztek: mert õ jóltévõ a háládatlanokkal és gonoszokkal.
\par 36 Legyetek azért irgalmasok, mint a ti Atyátok is irgalmas.
\par 37 Ne ítéljetek és nem ítéltettek; ne kárhoztassatok és nem kárhoztattok; megbocsássatok, néktek is megbocsáttatik;
\par 38 Adjatok, néktek is adatik; jó mértéket, megnyomottat és megrázottat, színig teltet adnak a ti öletekbe. Mert azzal a mértékkel mérnek néktek, a melylyel ti mértek.
\par 39 Példabeszédet is monda nékik: Vajjon a vak vezetheti-é a világtalant? avagy nem mindketten a verembe esnek-é?
\par 40 Nem feljebb való a tanítvány az õ mesterénél; hanem mikor tökéletes lesz, mindenki olyan lesz, mint a mestere.
\par 41 Miért nézed pedig a szálkát, a mely a te atyádfia szemében van, a gerendát pedig, mely a te saját szemedben van, nem veszed észre?
\par 42 Avagy mi módon mondhatod a te atyádfiának: Atyámfia, hadd vessem ki a szálkát a te szemedbõl, holott te a te szemedben lévõ gerendát nem látod. Te képmutató, vesd ki elõször a gerendát a te szemedbõl és azután gondolj arra, hogy kivesd a szálkát, a mely a te atyádfia szemében van.
\par 43 Nem jó fa az, a mely romlott gyümölcsöt terem; és nem romlott fa az, a mely jó gyümölcsöt terem.
\par 44 Mert minden fa az õ tulajdon gyümölcsérõl ismertetik meg; mert a tövisrõl nem szednek fügét, sem a szederindáról nem szednek szõlõt.
\par 45 A jó ember az õ szívének jó kincsébõl hoz elõ jót; és a gonosz ember az õ szívének gonosz kincsébõl hoz elõ gonoszt: mert a szívnek teljességébõl szól az õ szája.
\par 46 Miért mondjátok pedig nékem: Uram! Uram! ha nem mívelitek a miket mondok?
\par 47 Valaki én hozzám jõ és hallgatja az beszédimet és azokat megtartja, megmondom néktek, mihez hasonló.
\par 48 Hasonló valamely házépítõ emberhez, a ki leásott és mélyre hatolt, és kõsziklára vetett fundamentomot: mikor aztán árvíz lett, beleütközött a folyóvíz abba a házba, de azt meg nem mozdíthatta: mert kõsziklán épült.
\par 49 A ki pedig hallgatja, de nem tartja meg, hasonló ahhoz az emberhez, a ki csak a földön építette házát fundamentom nélkül: a melybe beleütközvén a folyóvíz, azonnal összeomlott; és nagy lett annak a háznak romlása.

\chapter{7}

\par 1 Mikor pedig minden õ beszédeit a nép hallatára elvégezte, beméne Kapernaumba.
\par 2 Egy századosnak szolgája pedig, aki annál nagy becsületben volt, igen rosszul lévén, már halófélben vala.
\par 3 Az pedig, mikor hallott Jézus felõl, hozzá küldé a zsidók véneit, kérvén õt, hogy jöjjön el és gyógyítsa meg az õ szolgáját.
\par 4 Azok pedig Jézushoz menvén, igen kérék õt, mondván: Méltó, hogy megtedd néki;
\par 5 Mert szereti a mi nemzetünket, és a zsinagógát is õ építtette nékünk.
\par 6 Jézus tehát elméne velök. Mikor azonban már nem messze volt a háztól, eléje küldé a százados néhány jó barátját, izenvén néki: Uram, ne fáraszd magad; mert nem vagyok méltó, hogy hajlékomba jõjj;
\par 7 A miért is magamat sem tartottam érdemesnek arra, hogy hozzád menjek: hanem csak szóval mondd, és meggyógyul az én szolgám.
\par 8 Mert én is hatalom alá vetett ember vagyok, és vitézek vannak alattam; és ha az egyiknek azt mondom: Eredj el, elmegy; vagy a másiknak: Jövel, eljõ; és ha szolgámnak szólok: Tedd ezt, azt teszi.
\par 9 Jézus pedig ezeket hallván, elcsudálkozék õ rajta; és hátrafordulván monda az õt követõ sokaságnak: Mondom néktek, ilyen hitet Izráelben sem találtam!
\par 10 És a küldöttek visszatérvén a házhoz, a beteg szolgát már egészségben találták.
\par 11 És lõn másnap, hogy méne Nain nevû városba; és az õ tanítványai sokan menének õ vele, és nagy sokaság.
\par 12 Mikor pedig a város kapujához közelített, ímé egy halottat hoznak vala ki, egyetlen egy fiát az anyjának, és az özvegy asszony vala; és a városból nagy sokaság volt õ vele.
\par 13 És látván õt az Úr, megkönyörüle rajta, és monda néki: Ne sírj.
\par 14 És oda menvén, illeté a koporsót; a vivõk pedig megállának. És monda: Ifjú, néked mondom, kelj föl!
\par 15 És felüle a megholt, és kezde szólni; és adá õt anyjának.
\par 16 És elfogá mind azokat a félelem, és dicsõíték az Istent, mondván: Nagy próféta támadt mi köztünk; és: Az Isten megtekintette az õ népét.
\par 17 És kiméne õ felõle e hír az egész Júdeába, és a körül való minden tartományba.
\par 18 És Jánosnak mind ezeket elmondták a tanítványai. És János az õ tanítványai közül kettõt elõszólítván,
\par 19 Elküldé Jézushoz, mondván: Te vagy-é az, a ki eljövendõ vala, vagy mást várjunk?
\par 20 Mikor azért azok a férfiak hozzámentek, mondának: Keresztelõ János küldött minket te hozzád,mondván: Te vagy-é az, a ki eljövendõ vala, vagy mást várjunk?
\par 21 Azon órában pedig sokakat gyógyíta meg betegségébõl, csapásokból, tisztátalan lelkektõl, és sok vaknak adá meg szeme világát.
\par 22 És felelvén Jézus, monda nékik: Elmenvén mondjátok meg Jánosnak, a miket láttatok és hallottatok: hogy a vakok szemeik világát veszik, a sánták járnak, a poklosok megtisztulnak, a siketek hallanak, a halottak feltámadnak, a szegényeknek az evangyéliom prédikáltatik.
\par 23 És boldog, valaki én bennem meg nem botránkozik.
\par 24 Mikor pedig elmentek a János követei, kezdé mondani a sokaságnak JÁnos felõl: Mit látni mentetek ki a pusztába? szélingatta nádszálat-é?
\par 25 Hát mit látni mentetek ki? puha ruhákba öltözött embert-é? Ímé a kik drága öltözetben és gyönyörûségben vannak, a királyok palotáiban vannak.
\par 26 Hát mit látni mentetek ki? Prófétát-é? Bizony mondom néktek, prófétánál is nagyobbat.
\par 27 Ez az, a ki felõl meg van írva: Ímé én elküldöm az én követemet a te orczád elõtt, ki elkészíti elõtted a te útadat.
\par 28 Mert mondom néktek, hogy azok között, a kik asszonytól születtek, egy sincs nagyobb próféta Keresztelõ Jánosnál; de a ki kisebb az Isten országában, nagyobb õ nála.
\par 29 És mikor ezt hallotta az egész nép és a vámszedõk, igazat adának az Istennek, megkeresztelkedvén a János keresztségével;
\par 30 A farizeusok pedig és a törvénytudók az Isten tanácsát megveték õ magokra nézve, nem keresztelkedvén meg õ tõle.
\par 31 Monda pedig az Úr: Mihez hasonlítsam azért e nemzetségnek embereit? és mihez hasonlók?
\par 32 Hasonlók a piaczon ülõ gyermekekhez, kik egymásnak kiáltanak, és ezt mondják: Sípoltunk néktek, és nem tánczoltatok; siralmas énekeket énekeltünk néktek, és nem sírtatok.
\par 33 Mert eljött Keresztelõ János, a ki kenyeret sem eszik, bort sem iszik, és ezt mondjátok: Ördög van benne.
\par 34 Eljött az embernek Fia, a ki eszik és iszik, és ezt mondjátok: Ímé a falánk és borivó ember, a vámszedõk és bûnösök barátja.
\par 35 És igazoltatik a bölcseség minden õ fiaitól.
\par 36 Kéré pedig õt egy a farizeusok közül, hogy õ vele egyék; annakokáért bemenvén a farizeus házába, leüle enni.
\par 37 És ímé a városban egy asszony a ki bûnös vala, mikor megtudta, hogy õ a farizeus házában leült enni, hoza egy alabástrom szelencze drága kenetet.
\par 38 És megállván hátul az õ lábainál sírva, könnyeivel kezdé öntözni az õ lábait, és fejének hajával törlé meg, és csókolgatá az õ lábait, és megkené drága kenettel.
\par 39 Mikor pedig ezt látta a farizeus, a ki õt meghívta, monda magában: Ez, ha próféta volna, tudná ki és miféle asszony az, a ki õt illeti: hogy bûnös.
\par 40 És felelvén Jézus, monda néki: Simon, van valami mondani valóm néked. És az monda: Mester, mondjad.
\par 41 Egy hitelezõnek két adósa vala: az egyik adós vala ötszáz pénzzel, a másik pedig ötvennel.
\par 42 És mikor nem volt nékik mibõl megadni, mind a kettõnek elengedé. E kettõ közül azért, mondd meg, melyik szereti õt jobban?
\par 43 Felelvén pedig Simon, monda: Azt gondolom, hogy az, a kinek többet engedett el. És Jézus monda néki: Igazán ítéltél.
\par 44 És az asszonyhoz fordulván, monda Simonnak: Látod-é ez asszonyt? Bejövék a te házadba, az én lábaimnak vizet nem adál: ez pedig könnyeivel öntözé az én lábaimat, és fejének hajával törlé meg.
\par 45 Engem meg nem csókolál: ez pedig az idõtõl fogva, hogy bejöttem, nem szünt meg az én lábaimat csókolgatni.
\par 46 Olajjal az én fejemet meg nem kented: ez pedig drága kenettel kené meg az én lábaimat.
\par 47 Minekokáért mondom néked: Néki sok bûne bocsáttatott meg; mert igen szeretett; a kinek pedig kevés bocsáttatik meg, kevésbé szeret.
\par 48 És monda annak: Megbocsáttattak néked a te bûneid.
\par 49 És a kik együtt ülének vele az asztalnál, kezdék magukban mondani: Ki ez, hogy a bûnöket is megbocsátja?
\par 50 Monda pedig az asszonynak: A te hited megtartott téged. Eredj el békességgel!

\chapter{8}

\par 1 És lõn ezután, hogy õ jár vala városonként és falunként, prédikálván és hirdetvén az Isten országát, és vele a tizenkettõ,
\par 2 És némely asszonyok, a kiket tisztátalan lelkektõl és betegségekbõl gyógyított meg, Mária,  a ki Magdalénának neveztetik, kibõl hét ördög ment ki,
\par 3 És Johanna, Khúzának, a Heródes gondviselõjének felesége, és Zsuzsánna, és sok más asszony, kik az õ vagyonukból szolgálának néki.
\par 4 Mikor pedig nagy sokaság gyûlt egybe, és minden városból mentek vala õ hozzá, monda példázat által:
\par 5 Kiméne a magvetõ, hogy elvesse az õ magvát: és magvetés közben némely esék az útfélre; és eltapostaték, és az égi madarak megevék azt.
\par 6 És némely esék a kõsziklára; és mikor kikelt, elszárada, mert nem vala nedvessége.
\par 7 Némely esék a tövis közé; és a tövisek vele együtt növekedvén, megfojták azt.
\par 8 Némely pedig esék a jó földbe; és mikor kikelt, százannyi hasznot hoza. Ezeket mondván, kiált vala: A kinek van füle a hallásra, hallja.
\par 9 És megkérdék õt az õ tanítványai, mondván: Mi lehet e példázat?
\par 10 Õ pedig monda nékik: Néktek adatott, hogy az Isten országának titkait értsétek; egyebeknek példázatokban, hogy látván ne lássanak, és hallván ne értsenek.
\par 11 A példázat pedig ez: A mag az Isten beszéde.
\par 12 Az útfélen valók pedig azok, a kik hallják; aztán eljõ az ördög, és kikapja az ígét az õ szívökbõl, hogy ne higyjenek és ne idvezüljenek.
\par 13 És a kõsziklán valók azok, a kik, mikor hallják, örömmel veszik az ígét; de ezeknek nincs gyökerük, a kik egy ideig hisznek, a kísértés idején pedig elszakadnak.
\par 14 És a melyik a tövis közé esett, ezek azok, a kik hallották, de elmenvén, az élet gondjaitól, és gazdagságától és gyönyörûségeitõl megfojtatnak, és gyümölcsöt nem teremnek.
\par 15 A melyik pedig a jó földbe esett, ezek azok, a kik a hallott ígét tiszta és jó szívvel megtartják, és gyümölcsöt teremnek béketûréssel.
\par 16 Senki pedig, ha gyertyát gyújt, be nem fedi azt valami edénnyel, sem az ágy alá nem rejti; hanem a gyertyatartóba teszi, hogy a kik bemennek, lássák a világot.
\par 17 Mert nincs oly titok, mely nyilvánvalóvá ne lenne; és nincs oly elrejtett dolog, mely ki ne tudódnék, és világosságra ne jõne.
\par 18 Meglássátok azért, mimódon hallgatjátok: mert a kinek van, annak adatik; és a kinek nincs, még a mijét gondolja is hogy van, elvétetik tõle.
\par 19 Jövének pedig hozzá az õ anyja és atyjafiai, de nem tudtak hozzá jutni a sokaság miatt.
\par 20 És tudtára adák néki, mondván: A te anyád és atyádfiai künn állnak, téged akarván látni.
\par 21 Õ pedig felelvén, monda nékik: Az én anyám és az én atyámfiai ezek, a kik az Isten beszédét hallgatják, és megcselekszik azt.
\par 22 Lõn pedig egy napon, hogy beméne a hajóba õ és az õ tanítványai; és monda nékik: Menjünk a tónak túlsó partjára. És elindulának.
\par 23 De hajózásuk közben elszenderedék; és szélvész csapott le a tóra, és megmerülének, és veszedelemben valának.
\par 24 És hozzá menvén, felkölték õt mondván: Mester, Mester, elveszünk! Õ pedig felserkenvén, megdorgálá a szelet és a víznek habjait; és megszûnének, és lõn csendesség.
\par 25 És monda nékik: Hol van a ti hitetek? És félelemmel csodálkoznak vala, mondván egymásnak: Ugyan ki ez, hogy a szeleknek is, a víznek is parancsol, és engednek néki?
\par 26 És evezének a Gaderénusok tartományaiba, mely Galileával átellenben van.
\par 27 És mikor õ kiment a földre, jöve elébe a városból egy ember, kiben ördögök voltak sok idõtõl fogva, sem ruhába nem öltözött, sem házban nem lakott, hanem a sírboltokban.
\par 28 És mikor meglátta Jézust, felkiálta és lábai elé esék néki, és fenszóval mondja: Mi közöm van nékem te veled, Jézus, felséges Istennek Fia? kérlek téged, ne gyötörj engem.
\par 29 Mert azt parancsolá annak a tisztátalan léleknek, hogy menjen ki az emberbõl. Mert gyakran elragadá õt: annakokáért lánczokkal és békókkal megkötözve õrizteték; de a kötelékeket elszaggatván, az ördögtõl a pusztákba hajtaték.
\par 30 Megkérdé pedig õt Jézus, mondván: Mi a neved? És õ monda: Légió; mert sok ördög ment vala bele.
\par 31 És kérék õt, hogy ne parancsolja nékik, hogy a mélységbe menjenek.
\par 32 Vala pedig ott egy nagy disznónyáj, legelészve a hegyen; és kérék õt, hogy engedje meg nékik, hogy azokba menjenek. És megengedé nékik.
\par 33 És minekutána kimentek az ördögök az emberbõl, bemenének a disznókba; és a disznónyáj a meredekrõl a tóba rohana, és megfullada.
\par 34 A pásztorok pedig látván mi történt, elfutának, és elmenvén, hírré adák a városban és a falukban.
\par 35 Kimenének azért megnézni mi történt; és menének Jézushoz, és ülve találák az embert, kibõl az ördögök kimentek, felöltözve és eszénél, a Jézus lábainál; és megfélemlének.
\par 36 Elbeszélék pedig nékik azok is, a kik látták, mimódon szabadult meg az ördöngõs.
\par 37 És kéré õt a Gadarénusok körül való tartományok egész sokasága, hogy õ közülök menjen el, mert felette igen félnek vala: õ pedig beülvén a hajóba, visszatére.
\par 38 Kéré pedig õt az az ember, a kibõl az ördögök kimentek, hogy õ vele lehessen; de Jézus elbocsátá õt, mondván:
\par 39 Térj vissza házadhoz, és beszéld el, mely nagy dolgokat tett az Isten veled. Elméne azért, hirdetvén az egész városban, mely nagy dolgokat cselekedett Jézus õ vele.
\par 40 És lõn, hogy mikor Jézus visszatért, a nép örömmel fogadá õt; mert mindnyájan várják vala õt.
\par 41 És ímé eljöve egy ember, kinek Jairus vala neve, ki a zsinagógának feje volt; és Jézus lábai elõtt leesvén, kéré õt, hogy menjen be az õ házába;
\par 42 Mert vala néki egy egyetlen leánya, mintegy tizenkét esztendõs, és az halálán volt. Mikor pedig õ méne, a sokaság szorongatá õt.
\par 43 És egy asszony, ki vérfolyásban volt tizenkét esztendõtõl fogva, és bár minden vagyonát az orvosokra költötte, senki meg nem tudta gyógyítani,
\par 44 Hátulról hozzá járulván, illeté az õ ruhájának peremét; és azonnal elálla vérének folyása.
\par 45 És monda Jézus: Ki az, a ki engem illete? És mikor mindnyájan tagadták, monda Péter és a kik õ vele valának: Mester, a sokaság nyom és szorongat téged, és azt mondod: Ki az, a ki engem illete?
\par 46 Jézus pedig monda: Illete engem valaki; mert én észrevettem, hogy erõ származék ki tõlem.
\par 47 Mikor pedig látta az asszony; hogy nem maradt titokban, reszketve elõjöve és elõtte leesvén, megjelenté néki az egész sokaság elõtt, miért illette õt, és hogy azonnal meggyógyult.
\par 48 És õ monda néki: Bízzál leányom, a te hited megtartott téged; eredj el békességgel!
\par 49 Mikor még a szó szájában vala, eljöve egy ember a zsinagóga fejének házától, mondván néki: Meghalt a leányod; ne fáraszd a Mestert!
\par 50 Jézus pedig mikor ezt hallotta, felele néki, mondván: Ne félj; csak higyj, és megtartatik.
\par 51 Bemenvén pedig a házba, senkit nem bocsáta be, csak Pétert, Jakabot, Jánost és a leányzó atyját és anyját.
\par 52 Sírának pedig mindnyájan, és gyászolák azt; õ pedig monda: Ne sírjatok; nem halt meg, hanem aluszik.
\par 53 És kineveték õt, tudván, hogy meghalt.
\par 54 Õ pedig mindenkit kiküldvén, és a leányzó kezét megfogván, kiálta, mondván: Leányzó, kelj fel!
\par 55 És visszatére annak lelke, és azonnal fölkele; és õ parancsolá, hogy adjanak néki enni.
\par 56 És elálmélkodának annak szülei; õ pedig megparancsolá, hogy senkinek ne mondják, a mi történt.

\chapter{9}

\par 1 Minekutána pedig összehívta Jézus az õ tizenkét tanítványát, ada nékik erõt és hatalmat minden ördögök ellen, és betegségek gyógyítására.
\par 2 És elküdé õket, hogy prédikálják az Isten országát, és betegeket gyógyítsanak.
\par 3 És monda nékik: Semmit az útra ne vigyetek, se pálczákat, se táskát, se kenyeret, se pénzt; se két-két ruhátok ne legyen.
\par 4 És valamely házba bementek, ott maradjatok, és onnét induljatok tovább.
\par 5 És valakik be nem fogadnak titeket, kimenvén abból a városból, még a port is verjétek le lábaitokról, bizonyságul õ ellenök.
\par 6 Kimenvén annakokáért, bejárák a falukat, hirdetvén az evangyéliomot, és gyógyítván mindenütt.
\par 7 Meghallá pedig Heródes a negyedes fejedelem, mindazokat, a mik õ általa történtek: és zavarban volt, mivelhogy némelyek azt mondák, hogy János támadt fel a halálból;
\par 8 Némelyek pedig, hogy Illés jelent meg; mások meg, hogy a régi próféták közül támadt fel valamelyik.
\par 9 És monda Heródes: Jánosnak én vettem fejét: kicsoda hát ez, a ki felõl én ilyen dolgokat hallok? És igyekezik vala õt látni.
\par 10 Visszatérvén pedig az apostolok, elbeszélének néki mindent, a mit cselekedtek. És azokat maga mellé vévén, elvonula magánosan a Bethsaida nevû városnak puszta helyére.
\par 11 A sokaság pedig ezt megtudván, követé õt: és õ örömmel fogadván õket, szóla nékik az Isten országáról, és a kiknek gyógyulásra volt szükségök, azokat meggyógyítá.
\par 12 A nap pedig hanyatlani kezdett; és a tizenkettõ õ hozzá járulván, monda néki: Bocsásd el a sokaságot, hogy elmenvén a körülvaló falvakba és majorokba megszálljanak, és eledelt találjanak, mert itt puszta helyen vagyunk.
\par 13 Õ pedig monda nékik: Adjatok nékik ti enni. Azok pedig mondának: Nincs nékünk több öt kenyerünknél és két halunknál; hanem ha elmegyünk és mi veszünk eledelt az egész sokaságnak.
\par 14 Mert valának ott mintegy ötezeren férfiak. Monda pedig az õ tanítványainak: Ültessétek le õket csoportokba ötvenével.
\par 15 És a képen cselekedének, és leülteték valamennyit.
\par 16 Minekutána pedig vette az öt kenyeret és a két halat, a mennybe emelvén szemeit, megáldá azokat, megszegé; és adá a tanítványoknak, hogy a sokaság elé tegyék.
\par 17 Evének azért és megelégedének mindnyájan; és felszedék a mi darabok maradtak tõlük, tizenkét kosárral.
\par 18 És lõn, mikor õ magában imádkozék, vele valának a tanítványok; és megkérdé õket, mondván: Kinek mond engem a sokaság?
\par 19 Õk pedig felelvén, mondának: Keresztelõ Jánosnak; némelyek pedig Illésnek; némelyek pedig, hogy a régi próféták közül támadt fel valamelyik.
\par 20 És monda nékik: Hát ti kinek mondotok engem? Felelvén pedig Péter, monda: Az Isten ama Krisztusának.
\par 21 Õ pedig reájok parancsolván, meghagyá, hogy ezt senkinek ne mondják;
\par 22 Ezt mondván: Szükség az ember Fiának sokat szenvedni és megvettetni a vénektõl, a fõpapoktól és írástudóktól, és megöletni, és harmadnapon feltámadni.
\par 23 Mondja vala pedig mindeneknek: Ha valaki én utánam akar jõni, tagadja meg magát, és vegye fel az õ keresztjét minden nap, és kövessen engem.
\par 24 Mert a ki meg akarja tartani az õ életét, elveszti azt; a ki pedig elveszti az õ életét én érettem, az megtartja azt.
\par 25 Mert mit használ az embernek, ha mind e világot megnyeri is, õ magát pedig elveszti vagy magában kárt vall?
\par 26 Mert valaki szégyel engem és az én beszédemet, az embernek Fia is szégyelni fogja azt, mikor eljõ az õ  dicsõségével, és az Atyáéval és a szent angyalokéval.
\par 27 Mondom pedig néktek bizonnyal, hogy vannak az itt állók közül némelyek, kik a halált meg nem kóstolják, mígnem meglátják az Istennek országát.
\par 28 És lõn e beszédek után mintegy nyolczadnappal, hogy maga mellé vevé Pétert, Jánost és Jakabot, és felméne a hegyre imádkozni.
\par 29 És imádkozása közben az õ orczájának ábrázata elváltozék, és az õ ruhája fehér és fénylõ lõn.
\par 30 És ímé két férfiú beszél vala õ vele, kik valának Mózes és Illés;
\par 31 Kik dicsõségben megjelenvén, beszélik vala az õ halálát, melyet Jeruzsálemben fog megteljesíteni.
\par 32 Pétert pedig és a vele lévõket elnyomá az álom; de mikor felébredtek, láták az õ dicsõségét, és ama két férfiút, kik vele állanak vala.
\par 33 És lõn, mikor azok eltávoztak õ tõle, monda Péter Jézusnak: Mester, jó nékünk itt lennünk: csináljunk azért három hajlékot, egy néked, Mózesnek is egyet, és egyet Illésnek; nem tudván mit mond.
\par 34 És mikor õ ezeket mondá, felhõ támada és azokat beárnyékozá; õk pedig megfélemlének, mikor azok bementek a felhõbe.
\par 35 És szózat lõn a felhõbõl, mondván: Ez amaz én szerelmes Fiam, õt hallgassátok.
\par 36 És mikor a szózat lõn, találtaték Jézus csak maga. Õk pedig hallgatának, és semmit abból, a mit láttak, senkinek el nem mondának azokban a napokban.
\par 37 És lõn másnap, mikor õk a hegyrõl leszállottak, sok nép méne elébe.
\par 38 És ímé egy a sokaság közül felkiálta, mondván: Mester, kérlek téged, tekints az én fiamra; mert nékem egyetlen egyem:
\par 39 És ímé a lélek megragadja õt, és hirtelen kiált; és szaggatja õt, annyira, hogy tajtékot túr, és nehezen megy el tõle, szaggatván õt.
\par 40 És kérem a te tanítványaidat, hogy ûzzék ki azt, de nem tudták.
\par 41 Felelvén pedig Jézus, monda: Óh hitetlen és elfajult nemzetség! meddig leszek köztetek, és meddig tûrlek titeket? Hozd ide a te fiadat!
\par 42 A míg pedig az odaméne, azon közben az ördög földhöz üté azt, és megrángatá. De Jézus megdorgálá a tisztátalan lelket, és meggyógyítá a gyermeket, és adá azt az õ atyjának.
\par 43 Elálmélkodának pedig mindnyájan az Istennek nagyságos erején. Mikor pedig mindnyájan csodálkozának mind azokon, a miket Jézus cselekedék, monda az õ tanítványainak:
\par 44 Vegyétek füleitekbe ezeket a beszédeket: Mert az embernek Fia az emberek kezébe fog adatni.
\par 45 De õk nem érték e mondást, és el vala rejtve elõlük, hogy ne értsék azt; és féltek õt megkérdezni e mondás felõl.
\par 46 Támada pedig bennök az a gondolat, hogy ki nagyobb közöttük.
\par 47 Jézus pedig látván az õ szívök gondolatát, egy kis gyermeket megfogván, maga mellé állatá azt,
\par 48 És monda nékik: Valaki e kis gyermeket befogadja az én nevemben, engem fogad be; és valaki engem befogad, azt fogadja be, a ki engem elküldött; mert a ki legkisebb mindnyájan ti közöttetek, az lesz nagy.
\par 49 Felelvén pedig János, monda: Mester, láttunk valakit, a ki a te nevedben ördögöket ûz; és eltiltók õt, mivelhogy téged nem követ mi velünk.
\par 50 És monda néki Jézus: Ne tiltsátok el: mert a ki nincs ellenünk, mellettünk van.
\par 51 Lõn pedig, mikor az idõ elközelge, hogy õ felvitessék, eltökélte magát, hogy Jeruzsálembe megy,
\par 52 És követeket külde az õ orczája elõtt; és azok elmenvén, bemenének egy samaritánus faluba, hogy néki szállást készítsenek.
\par 53 De nem fogadák be õt, mivelhogy õ Jeruzsálembe megy vala.
\par 54 Mikor pedig ezt látták az õ tanítványai, Jakab és János, mondának: Uram, akarod-é, hogy mondjuk, hogy tûz szálljon alá az égbõl, és emészsze meg ezeket, mint Illyés is cselekedett?
\par 55 De Jézus megfordulván, megdorgálá õket, mondván: Nem tudjátok minémû lélek van ti bennetek:
\par 56 Mert az embernek Fia nem azért jött, hogy elveszítse az emberek lelkét, hanem hogy megtartsa. Elmenének azért más faluba.
\par 57 Lõn pedig, mikor menének, valaki monda néki az úton: Követlek téged Uram, valahová mégy!
\par 58 És monda néki Jézus: A rókáknak barlangjuk van, és az égi madaraknak fészkük; de az ember Fiának nincs fejét hová lehajtania.
\par 59 Monda pedig másnak: Kövess engem. Az pedig monda: Uram, engedd meg nékem, hogy elõbb elmenjek és eltemessem az én atyámat.
\par 60 Monda pedig néki Jézus: Hadd temessék el a halottak az õ halottaikat: te pedig elmenvén, hirdesd az Isten országát.
\par 61 Monda pedig más is: Követlek téged Uram; de elõbb engedd meg nékem, hogy búcsút vegyek azoktól, a kik az én házamban vannak.
\par 62 És monda néki Jézus: Valaki az eke szarvára veti kezét, és hátra tekint, nem alkalmas az Isten országára.

\chapter{10}

\par 1 Ezek után pedig rendele az Úr másokat is, hetvenet, és elküldé azokat kettõnként az õ orczája elõtt, minden városba és helyre, a hová õ menendõ vala.
\par 2 Monda azért nékik: Az aratni való sok, de a munkás kevés; kérjétek azért az aratásnak Urát, hogy küldjön munkásokat az õ aratásába.
\par 3 Menjetek el: Ímé én elbocsátlaktiteket, mint bárányokat a farkasok közé.
\par 4 Ne hordozzatok erszényt, se táskát, se sarut; és az úton  senkit ne köszöntsetek.
\par 5 Valamely házba bementek, elõször ezt mondjátok: Békesség e háznak!
\par 6 És ha lesz ott valaki békességnek fia, a ti békességtek azon marad; ha nem, ti reátok tér vissza.
\par 7 Ugyanazon házban maradjatok pedig, azt evén is iván, a mit õk adnak: mert méltó a munkás  az õ jutalmára. Ne járjatok házról-házra.
\par 8 És valamely városba bementek, és befogadnak titeket, azt egyétek, a mit elõtökbe adnak:
\par 9 És gyógyítsátok a betegeket, a kik ott lesznek, és mondjátok nékik: Elközelített hozzátok az Isten országa.
\par 10 Valamely városba pedig bementek, és titeket be nem fogadnak, annak utczáira kimenvén, ezt mondjátok:
\par 11 Még a port is, a mely reánk ragadt a ti városotokból, itt köztetek letöröljük; mindazáltal ez legyen tudtotokra, hogy az Isten országa elközelített hozzátok.
\par 12 Mondom pedig néktek, hogy a Sodomabeliek állapota tûrhetõbb lesz ama napon, hogynem azé a városé.
\par 13 Jaj néked Korazin! Jaj néked Bethsaida! mert ha Tírusban és Sídonban lettek volna azok a csodák, melyek te benned lõnek, régen zsákban és hamuban ülve megtértek volna.
\par 14 Hanem Tírusnak és Sídonnak tûrhetõbb lesz állapota az ítéletkor, hogynem néktek.
\par 15 És te Kapernaum, mely mind az égig felmagasztaltattál, a pokolig fogsz lealáztatni.
\par 16 A ki titeket hallgat, engem hallgat, és a ki titeket megvet, engem vet meg; és a ki engem vet meg, azt veti meg, a ki engem elküldött.
\par 17 Visszatére pedig a hetven tanítvány örömmel, mondván: Uram, még az ördögök is engednek nékünk a te neved által!
\par 18 Õ pedig monda nékik: Látám a Sátánt, mint a villámlást lehullani az égbõl.
\par 19 Ímé adok néktek hatalmat, hogy a kígyókon és skorpiókon tapodjatok, és az ellenségnek minden erején; és semmi nem árthat néktek.
\par 20 De azon ne örüljetek, hogy a lelkek néktek engednek; hanem inkább azon örüljetek, hogy a ti neveitek fel vannak írva a mennyben.
\par 21 Azon órában örvendeze Jézus lelkében, és monda: Hálákat adok néked, Atyám, mennynek és földnek Ura, hogy elrejtetted ezeket a bölcsek és értelmesek elõl, és a kisdedeknek megjelentetted. Igen, Atyám, mert így volt kedves te elõtted.
\par 22 Mindent nékem adott az én Atyám: és senki sem tudja, kicsoda a Fiú, csak az Atya; és kicsoda az Atya, hanem  csak a Fiú, és a kinek a Fiú akarja megjelenteni.
\par 23 És a tanítványokhoz fordulván, monda õ magoknak: Boldog szemek, a melyek látják azokat,  a melyeket ti láttok.
\par 24 Mert mondom néktek, hogy sok próféta és király kívánta látni, a miket ti láttok, de nem látták; és hallani, a miket hallotok, de nem hallották.
\par 25 És ímé egy törvénytudó felkele, kísértvén õt, és mondván: Mester, mit cselekedjem, hogy az örök életet vehessem?
\par 26 Õ pedig monda annak: A törvényben mi van megírva? mint olvasod?
\par 27 Az pedig felelvén, monda: Szeresd az Urat, a te Istenedet teljes szívedbõl és teljes lelkedbõl és minden erõdbõl és teljes elmédbõl; és a te felebarátodat, mint magadat.
\par 28 Monda pedig annak: Jól felelél; ezt cselekedd, és élsz.
\par 29 Az pedig igazolni akarván magát, monda Jézusnak: De ki az én felebarátom?
\par 30 Jézus pedig felelvén, monda: Egy ember megy vala alá Jeruzsálembõl Jerikóba, és rablók kezébe esék, a kik azt kifosztván és megsebesítvén, elmenének, és ott hagyák félholtan.
\par 31 Történet szerint pedig megy vala alá azon az úton egy pap, a ki azt látván, elkerülé.
\par 32 Hasonlóképen egy Lévita is, mikor arra a helyre ment, és azt látta, elkerülé.
\par 33 Egy samaritánus pedig az úton menvén, odaért, a hol az vala: és mikor azt látta, könyörületességre indula.
\par 34 És hozzájárulván, bekötözé annak sebeit, olajat és bort töltvén azokba; és azt felhelyezvén az õ tulajdon barmára, vivé a vendégfogadó házhoz, és gondját viselé néki.
\par 35 Másnap pedig elmenõben két pénzt kivévén, adá a gazdának, és monda néki: Viselj gondot reá, és valamit ezen fölül reáköltesz, én mikor visszatérek, megadom néked.
\par 36 E három közül azért kit gondolsz, hogy felebarátja volt annak, a ki a rablók kezébe esett?
\par 37 Az pedig monda: Az, a ki könyörült rajta. Monda azért néki Jézus: Eredj el, és te is a képen cselekedjél.
\par 38 Lõn pedig, mikor az úton menének, hogy õ beméne egy faluba; egy Mártha nevû asszony pedig befogadá õt házába.
\par 39 És ennek vala egy Mária nevezetû testvére, ki is Jézus lábainál leülvén, hallgatja vala az õ beszédét.
\par 40 Mártha pedig foglalatos volt a szüntelen való szolgálatban; elõállván azért, monda: Uram, nincs-é arra gondod, hogy az én testvérem magamra hagyott engem, hogy szolgáljak? Mondjad azért néki, hogy segítsen nékem.
\par 41 Felelvén pedig, monda néki Jézus: Mártha, Mártha, szorgalmas vagy és sokra igyekezel:
\par 42 De egy a szükséges dolog: és Mária a jobb részt választotta, mely el nem vétetik õ tõle.

\chapter{11}

\par 1 És lõn, mikor õ imádkozék egy helyen, minekutána elvégezte, monda néki egy az õ tanítványai közül: Uram, taníts minket imádkozni, miképen János is tanította az õ tanítványait.
\par 2 Monda pedig nékik: Mikor imádkoztok, ezt mondjátok: Mi Atyánk, ki vagy a mennyekben, szenteltessék meg a te neved. Jõjjön el a te országod. Legyen meg a te akaratod, miképen a mennyben, azonképen e földön is.
\par 3 A mi mindennapi kenyerünket add meg nékünk naponként.
\par 4 És bocsásd meg nékünk a mi bûneinket; mert mi is megbocsátunk mindeneknek, a kik nékünk adósok. És ne vígy minket kisértésbe; de szabadíts meg minket a gonosztól.
\par 5 És monda nékik: Ki az közületek, a kinek barátja van, és ahhoz megy éjfélkor, és ezt mondja néki: Barátom, adj nékem kölcsön három kenyeret,
\par 6 Mert az én barátom én hozzám jött az útról, és nincs mit adjak ennie;
\par 7 Az pedig onnét belõlrõl felelvén, ezt mondaná: Ne bánts engem: immár az ajtó be van zárva, és az én gyermekeim velem vannak az ágyban; nem kelhetek fel, és nem adhatok néked?
\par 8 Mondom néktek, ha azért nem fog is felkelni és adni néki, mert az barátja, de annak tolakodása miatt felkél és ád néki, a mennyi kell.
\par 9 Én is mondom néktek: Kérjetek és megadatik néktek; keressetek és találtok; zörgessetek és megnyittatik néktek.
\par 10 Mert a ki kér, mind kap; és a ki keres, talál; és a zörgetõnek megnyittatik.
\par 11 Melyik atya pedig az közületek, a kitõl a fia kenyeret kér, és õ talán követ ád néki? vagy ha halat, vajjon a hal helyett kígyót ád-é néki?
\par 12 Avagy ha tojást kér, vajjon skorpiót ád-é néki?
\par 13 Ha azért ti gonosz létetekre tudtok a ti fiaitoknak jó ajándékokat adni, mennyivel inkább ád a ti mennyei Atyátok Szent Lelket azoknak, a kik tõle kérik.
\par 14 És ördögöt ûz vala ki, mely néma vala. És lõn, mikor kiment az ördög, megszólala a néma; és csodálkozék a sokaság.
\par 15 Némelyek pedig azok közül mondának: A Belzebúb által, az ördögök fejedelme által ûzi ki az ördögöket.
\par 16 Mások meg, kísértvén õt mennyei jelt kívánának tõle.
\par 17 Õ pedig tudván azoknak gondolatát, monda nékik: Minden ország, a mely magával meghasonlik, elpusztul; és ház a házzal ha meghasonlik, leomlik.
\par 18 És a Sátán is ha õ magával meghasonlik, mimódon állhat meg az õ országa? mert azt mondjátok, hogy én a Belzebúb által ûzöm ki az ördögöket.
\par 19 És ha én Belzebúb által ûzöm ki az ördögöket, a ti fiaitok ki által ûzik ki? Annakokáért õk maguk lesznek a ti bíráitok.
\par 20 Ha pedig Isten ujjával ûzöm ki az ördögöket, kétség nélkül elérkezett hozzátok az Isten országa.
\par 21 Mikor az erõs fegyveres õrzi az õ palotáját, a mije van, békességben van;
\par 22 De mikor a nálánál erõsebb reá jövén legyõzi õt, minden fegyverét elveszi, melyhez bízott, és a mit tõle zsákmányol, elosztja.
\par 23 A ki velem nincs, ellenem van; és a ki velem nem takar, tékozol.
\par 24 Mikor a tisztátalan lélek kimegy az emberbõl, víz nélkül való helyeken jár, nyugalmat keresvén; és mikor nem talál, ezt mondja: Visszatérek az én házamba, a honnét kijöttem.
\par 25 És oda menvén, kisöpörve és felékesítve találja azt.
\par 26 Akkor elmegy, és maga mellé vesz más hét lelket, magánál gonoszabbakat, és bemenvén ott lakoznak; és annak az embernek utolsó állapota gonoszabb lesz az elsõnél.
\par 27 Lõn pedig mikor ezeket mondá, fölemelvén szavát egy asszony a sokaság közül, monda néki: Boldog méh, a mely téged hordozott, és az emlõk, melyeket szoptál.
\par 28 Õ pedig monda: Sõt inkább boldogok a kik hallgatják az Istennek beszédét, és megtartják azt.
\par 29 Mikor pedig a sokaság hozzá gyülekezék, kezdé mondani: E nemzetség gonosz: jelt kíván, de jel nem adatik néki, hanem ha Jónás prófétának  ama jele;
\par 30 Mert miképen Jónás jelül volt a Ninivebelieknek, azonképen lesz az embernek Fia is e nemzetségnek.
\par 31 A Délnek királynéasszonya felkél majd az ítéletkor e nemzetség férfiaival, és kárhoztatja õket: mert õ eljött a földnek szélérõl, hogy hallhassa a Salamon bölcseségét; és ímé nagyobb  van itt Salamonnál.
\par 32 Ninive férfiai az ítéletkor együtt támadnak majd fel e nemzetséggel, és kárhoztatják ezt: mivelhogy õk megtértek a Jónás prédikálására; és ímé nagyobb van itt  Jónásnál.
\par 33 Senki pedig, ha gyertyát gyújt, nem teszi rejtekbe, sem a véka alá, hanem a gyertyatartóba, hogy a kik bemennek, lássák a világosságot.
\par 34 A testnek lámpása a szem: ha azért a te szemed õszinte, a te egész tested is világos lesz; ha pedig a te szemed gonosz, a te tested is sötét.
\par 35 Meglásd azért, hogy a világosság, mely te benned van, sötétség ne legyen.
\par 36 Annakokáért ha a te egész tested világos, és semmi részében sincs homályosság, olyan világos lesz egészen, mint mikor a lámpás megvilágosít téged az õ világosságával.
\par 37 Beszéd közben pedig kéré õt egy farizeus, hogy ebédeljen nála. Bemenvén azért, leüle enni.
\par 38 A farizeus pedig mikor ezt látta, elcsodálkozék, hogy ebéd elõtt elõbb nem mosdott meg.
\par 39 Monda pedig az Úr néki: Ti farizeusok jóllehet a pohárnak és tálnak külsõ részét megtisztítjátok; de a belsõtök rakva ragadománynyal és gonoszsággal.
\par 40 Bolondok, a ki azt teremtette, a mi kívül van, nem ugyanaz teremtette-é azt is, a mi belõl van?
\par 41 Csak adjátok alamizsnául a mi benne van; és minden tiszta lesz néktek.
\par 42 De jaj néktek farizeusok! mert megadjátok a dézsmát a mentától, rutától és minden paréjtól, de hátra hagyjátok az ítéletet és az Isten szeretetét: pedig ezeket kellene cselekedni, és amazokat sem elhagyni.
\par 43 Jaj néktek farizeusok! mert szeretitek az elõlülést a gyülekezetekben, és a piaczokon való köszöntéseket.
\par 44 Jaj néktek képmutató írástudók és farizeusok! mert olyanok vagytok, mint a sírok, a melyek nem látszanak, és az emberek, a kik azokon járnak, nem tudják.
\par 45 Felelvén pedig egy a törvénytudók közül, monda néki: Mester, mikor ezeket mondod, minket is bántasz.
\par 46 Õ pedig monda: Jaj néktek is törvénytudók! mert elhordozhatatlan terhekkel terhelitek meg az embereket, de ti magatok egy ujjotokkal sem illetitek azokat a terheket.
\par 47 Jaj néktek! mert ti építitek a próféták sírjait; a ti atyáitok pedig megölték õket.
\par 48 Tehát bizonyságot tesztek és jóvá hagyjátok atyáitok cselekedeteit; mert azok megölték õket, ti pedig építitek sírjaikat.
\par 49 Ezért mondta az Isten bölcsesége is: Küldök õ hozzájuk prófétákat és apostolokat; és azok közül némelyeket megölnek, és némelyeket elüldöznek;
\par 50 Hogy számon kéressék e nemzetségtõl minden próféták vére, mely e világ fundamentomának felvettetésétõl fogva kiontatott,
\par 51 Az Ábel vérétõl fogva mind a Zakariás  véréig, ki elveszett az oltár és a templom között: bizony, mondom néktek, számon kéretik e nemzetségtõl.
\par 52 Jaj néktek törvénytudók! mert elvettétek a tudománynak kulcsát: ti magatok nem mentetek be, és a kik be akartak menni, azokat meggátoltátok.
\par 53 Mikor pedig ezeket mondá nékik, az írástudók és farizeusok kezdének felette igen ellene állani és õt sok dolog felõl kikérdezgetni,
\par 54 Ólálkodván õ utána, és igyekezvén valamit az õ szájából kikapni, hogy vádolhassák õt.

\chapter{12}

\par 1 Ezenközben mikor sok ezerbõl álló sokaság gyûlt egybe, annyira, hogy egymást letapossák, kezdé az õ tanítványainak mondani: Mindenekelõtt oltalmazzátok meg magatokat a farizeusok kovászától, mely a  képmutatás;
\par 2 Mert nincs oly rejtett dolog, mely napfényre ne jõne; és oly titok, mely ki ne tudódnék.
\par 3 Annakokáért a mit a sötétben mondtatok, a világosságban fog meghallatszani; és a mit fülbe sugtatok a rejtekházakban, azt a házak tetején fogják hirdetni.
\par 4 Mondom pedig néktek én barátaimnak: Ne féljetek azoktól, kik a testet ölik meg, és azután  többet nem árthatnak.
\par 5 De megmondom néktek, kitõl féljetek: Féljetek attól, a ki minekutána megöl, van arra is hatalma, hogy a gyehennára vessen. Bizony, mondom néktek, ettõl féljetek.
\par 6 Nemde öt verebet meg lehet venni két filléren? és egy sincs azok közül Istennél elfelejtve.
\par 7 De néktek a fejetek hajszálai is mind számon vannak. Ne féljetek azért, sok verébnél drágábbak vagytok.
\par 8 Mondom pedig néktek: Valaki vallást tesz én rólam az emberek elõtt, az embernek Fia is vallást tesz arról az Isten angyalai elõtt;
\par 9 A ki pedig megtagad engem az emberek elõtt, megtagadtatik az Isten angyalai elõtt.
\par 10 És ha valaki valamit mond az embernek Fia ellen, megbocsáttatik annak; de annak, de a ki a Szent Lélek ellen szól káromlást, meg nem bocsáttatik.
\par 11 Mikor pedig a zsinagógákba visznek benneteket, és a fejedelmek és hatalmasságok elé, ne aggodalmaskodjatok, mimódon vagy mit szóljatok védelmetekre, vagy mit mondjatok;
\par 12 Mert a Szent Lélek azon órában megtanít titeket, mit kell mondanotok.
\par 13 Monda pedig néki egy a sokaság közül: Mester, mondd meg az én testvéremnek, hogy oszsza meg velem az örökséget.
\par 14 Õ pedig monda néki: Ember, ki tett engem köztetek biróvá vagy osztóvá?
\par 15 Monda azért nékik: Meglássátok, hogy eltávoztassátok a telhetetlenséget; mert nem a vagyonnal való bõvölködésben van az embernek az õ élete.
\par 16 És monda nékik egy példázatot, szólván: Egy gazdag embernek bõségesen termett a földje.
\par 17 Azért magában okoskodék, mondván: Mit cselekedjem? mert nincs hová takarnom az én termésemet.
\par 18 És monda: Ezt cselekszem: Az én csûreimet lerontom, és nagyobbakat építek; és azokba takarom minden gabonámat és az én javaimat.
\par 19 És ezt mondom az én lelkemnek: Én lelkem, sok javaid vannak sok esztendõre eltéve; tedd magadat kényelembe, egyél, igyál, gyönyörködjél!
\par 20 Monda pedig néki az Isten: Bolond, ez éjjel elkérik a te lelkedet te tõled; a miket pedig készítettél, kiéi lesznek?
\par 21 Így van dolga annak, a ki kincset takar magának, és nem az Istenben gazdag.
\par 22 Monda pedig az õ tanítványainak: Annakokáért mondom néktek, ne aggodalmaskodjatok a a ti éltetek felõl, mit egyetek; se a ti testetek felõl, mibe öltözködjetek.
\par 23 Az élet több, hogynem az eledel, és a test, hogynem az öltözet.
\par 24 Tekintsétek meg a hollókat, hogy nem vetnek, sem nem aratnak; kiknek nincs tárházuk, sem csûrük; és az Isten eltartja õket: mennyivel drágábbak vagytok ti a madaraknál?
\par 25 Kicsoda pedig az közületek, a ki aggodalmaskodásával megnövelheti termetét egy arasszal?
\par 26 Annakokáért ha a mi legkisebb dolog, azt sem tehetitek, mit aggodalmaskodtok a többi felõl?
\par 27 Tekintsétek meg a liliomokat, mimódon növekednek: nem fáradoznak és nem fonnak: de mondom néktek: Salamon minden õ dicsõségében sem öltözött úgy, mint ezek közül egy.
\par 28 Ha pedig a füvet, mely ma a mezõn van, és holnap kemenczébe vettetik, így ruházza az Isten; mennyivel inkább titeket, ti kicsinyhitûek!
\par 29 Ti se kérdezzétek, mit egyetek vagy mit igyatok; és ne kételkedjetek.
\par 30 Mert mind ezeket a világi pogányok kérdezik; a ti Atyátok pedig tudja, hogy néktek szükségetek van ezekre.
\par 31 Csak keressétek az Isten országát, és ezek  mind megadatnak néktek.
\par 32 Ne félj te kicsiny nyáj; mert tetszett a ti Atyátoknak, hogy néktek adja az országot.
\par 33 Adjátok el a mitek van, és adjatok alamizsnát; szerezzetek magatoknak oly erszényeket, melyek meg nem avúlnak, elfogyhatatlan  kincset a mennyországban, a hol a tolvaj hozzá nem fér, sem a moly meg nem emészti.
\par 34 Mert a hol van a ti kincsetek, ott van a ti szívetek is.
\par 35 Legyenek a ti derekaitok felövezve, és szövétnekeitek meggyújtva;
\par 36 Ti meg hasonlók az olyan emberekhez, a kik az õ urokat várják, mikor jõ meg a menyegzõrõl, hogy mihelyt megjõ és zörget, azonnal megnyissák néki.
\par 37 Boldogok azok a szolgák, kiket az úr, mikor haza megy, vigyázva talál: bizony mondom néktek, hogy felövezvén magát, leülteti azokat, és elõjövén, szolgál  nékik.
\par 38 És ha megjõ a második õrváltáskor, és ha a harmadik õrváltáskor jõ meg, és úgy találja õket, boldogok azok a szolgák!
\par 39 Ezt pedig jegyezzétek meg, hogy ha tudná a ház gazdája, mely órában jõ el a tolvaj, vigyázna, és nem engedné, hogy az õ házába törjön.
\par 40 Ti is azért legyetek készek: mert a mely órában nem gondolnátok, abban jõ el az embernek Fia.
\par 41 Monda pedig néki Péter: Uram, nékünk mondod-é ezt a példázatot, vagy mindenkinek is?
\par 42 Monda pedig az Úr: Kicsoda hát a hû és bölcs sáfár, kit az úr gondviselõvé tõn az õ háza népén, hogy adja ki nékik élelmüket a maga idejében?
\par 43 Boldog az a szolga, a kit az õ ura, mikor haza jõ, ilyen munkában talál!
\par 44 Bizony mondom néktek, hogy minden jószága felett gondviselõvé teszi õt.
\par 45 Ha pedig az a szolga így szólna az õ szívében: Halogatja még az én uram a hazajövetelt; és kezdené verni a szolgákat és szolgálóleányokat, és enni és inni és részegeskedni:
\par 46 Megjõ annak a szolgának az ura, a mely napon nem várja és a mely órában nem gondolja, és kettévágatja õt, és a hitetlenek sorsára juttatja.
\par 47 És a mely szolga tudta az õ urának akaratát, és nem végezte el, sem annak akarata szerint nem cselekedett, sokkal büntettetik meg;
\par 48 A ki pedig nem tudta, és büntetésre méltó dolgokat cselekedett, kevesebbel büntettetik. És valakinek sokat adtak, sokat követelnek tõle; és a kire sokat bíztak, többet kívánnak tõle.
\par 49 Azért jöttem, hogy e világra tüzet bocsássak: és mit akarok, ha az immár meggerjedett?
\par 50 De keresztséggel kell nékem megkereszteltetnem; és  mely igen szorongattatom, míglen az elvégeztetik.
\par 51 Gondoljátok-é, hogy azért jöttem, hogy békességet adjak e földön? Nem, mondom néktek; sõt inkább meghasonlást.
\par 52 Mert mostantól fogva öten lesznek egy házban, a kik meghasonlanak, három kettõ ellen, és kettõ három ellen.
\par 53 Meghasonlik az atya a fiú ellen, és a fiú az atya ellen; és az anya a leány ellen, és a leány az anya ellen; napa a menye ellen, és a menye a napa ellen.
\par 54 Monda pedig a sokaságnak is: Mikor látjátok, hogy napnyugotról felhõ támad, azonnal ezt mondjátok: Záporesõ jõ; és úgy lesz.
\par 55 És mikor halljátok fúni a déli szelet, ezt mondjátok: Hõség lesz; és úgy lesz.
\par 56 Képmutatók, az égnek és a földnek ábrázatáról tudtok ítéletet tenni; errõl az idõrõl pedig mi dolog, hogy nem tudtok ítéletet tenni?
\par 57 És mi dolog, hogy ti magatoktól is meg nem ítélitek, mi az igaz?
\par 58 Mikor pedig a te ellenségeddel a fejedelem elé mégy, igyekezzél az úton megmenekedni tõle, hogy téged ne vonjon a bíró elé, és a bíró át ne adjon téged a poroszlónak és a poroszló a tömlöczbe ne vessen téged.
\par 59 Mondom néked, hogy nem jõsz ki onnét, mígnem megfizetsz mind az utolsó fillérig.

\chapter{13}

\par 1 Jövének pedig ugyanazon idõben némelyek, kik néki hírt mondának a Galileabeliek felõl, kiknek vérét Pilátus az õ áldozatukkal elegyítette.
\par 2 És felelvén Jézus, monda nékik: Gondoljátok-é, hogy ezek a Galileabeliek bûnösebbek voltak valamennyi Galileabelinél, mivelhogy ezeket szenvedték?
\par 3 Nem, mondom néktek: sõt inkább, ha meg nem tértek, mindnyájan, hasonlóképen elvesztek.
\par 4 Vagy az a tizennyolcz, a kire rászakadt a torony Siloámban, és megölte õket, gondoljátok-é, hogy bûnösebb volt minden más Jeruzsálemben lakó embernél?
\par 5 Nem, mondom néktek: sõt inkább, ha meg nem tértek, mindnyájan hasonlóképen elvesztek.
\par 6 És ezt a példázatot mondá: Vala egy embernek egy fügefája szõlejébe ültetve; és elméne, hogy azon gyümölcsöt keressen, és nem talála.
\par 7 És monda a vinczellérnek: Ímé három esztendeje járok gyümölcsöt keresni e fügefán, és nem találok: vágd ki azt; miért foglalja a földet is hiába?
\par 8 Az pedig felelvén, monda néki: Uram, hagyj békét néki még ez esztendõben, míg köröskörül megkapálom és megtrágyázom:
\par 9 És ha gyümölcsöt terem, jó; ha pedig nem, azután vágd ki azt.
\par 10 Tanít vala pedig szombatnapon egy zsinagógában.
\par 11 És ímé vala ott egy asszony, kiben betegségnek lelke vala tizennyolcz esztendõtõl fogva; és meg volt görbedve, és teljességgel nem tudott felegyenesedni.
\par 12 És mikor azt látta Jézus, elõszólítá, és monda néki: Asszony, feloldattál a te betegségedbõl!
\par 13 És reá veté kezeit; és azonnal felegyenesedék, és dicsõíté az Istent.
\par 14 Felelvén pedig a zsinagógafõ, haragudva, hogy szombatnapon gyógyított Jézus, monda a sokaságnak: Hat nap van, a melyen munkálkodni kell; azokon jõjjetek azért és gyógyíttassátok magatokat, és ne szombatnapon.
\par 15 Felele azért néki az Úr, és monda: Képmutató, szombatnapon nem oldja-é el mindenitek az õ ökrét vagy szamarát a jászoltól, és nem viszi-é itatni?
\par 16 Hát ezt, az Ábrahám leányát, kit a Sátán megkötözött ímé tizennyolcz esztendeje, nem kellett-é feloldani e kötélbõl szombatnapon?
\par 17 És mikor ezeket mondta, megszégyenülének mindnyájan, kik magokat néki ellenébe veték; és az egész nép örül vala mind azokon a dicsõséges dolgokon, a melyek õ általa lettek.
\par 18 Monda pedig Jézus: Mihez hasonló az Isten országa? és mihez hasonlítsam azt?
\par 19 Hasonló a mustármaghoz, melyet az ember vévén, elvet az õ kertjében; és felnevelkedett, és lett nagy fává, és az égi madarak fészket raktak annak ágain.
\par 20 És ismét monda: Mihez hasonlítsam az Isten országát?
\par 21 Hasonló a kovászhoz, melyet az asszony vévén, három mércze lisztbe elegyíte, mígnem az egész megkele.
\par 22 És városokon és falvakon megy vala által, tanítva, és Jeruzsálembe menve.
\par 23 Monda pedig néki valaki: Uram, avagy kevesen vannak-é a kik idvezülnek? Õ pedig monda nékik:
\par 24 Igyekezzetek bemenni a szoros kapun: mert sokan,  mondom néktek, igyekeznek bemenni és nem mehetnek.
\par 25 Mikor már a gazda felkél és bezárja az ajtót, és kezdetek kívül állani és az ajtót zörgetni, mondván: Uram! Uram! nyisd meg nékünk; és õ felelvén, ezt mondja néktek: Nem  tudom honnét valók vagytok ti;
\par 26 Akkor kezditek mondani: Te elõtted ettünk és ittunk, és a mi utczáinkon tanítottál;
\par 27 De ezt mondja: Mondom néktek, nem tudom honnét valók vagytok ti; távozzatok el én tõlem mindnyájan, kik hamisságot cselekesztek!
\par 28 Ott lesz sírás és fogak csikorgatása, mikor látjátok Ábrahámot, Izsákot és Jákóbot, és a prófétákat mind az Isten országában, magatokat pedig kirekesztve.
\par 29 És jõnek napkeletrõl és napnyugatról, és északról és délrõl, és az Isten országában letelepednek.
\par 30 És ímé vannak utolsók, a kik elsõk lesznek, és vannak elsõk, a kik utolsók lesznek.
\par 31 Ugyanazon napon jövének õ hozzá némelyek a farizeusok közül, mondván néki: Eredj ki és menj el innét: mert Heródes meg akar téged ölni.
\par 32 És monda nékik: Elmenvén mondjátok meg annak a rókának: Ímé ördögöket ûzök ki és gyógyítok ma és holnap, és harmadnapon elvégeztetem.
\par 33 Hanem nékem ma és holnap és azután úton kell lennem; mert nem lehetséges, hogy a próféta Jeruzsálemen kívül vesszen el.
\par 34 Jeruzsálem! Jeruzsálem! ki megölöd a prófétákat, és megkövezed azokat, a kik te hozzád küldettek, hányszor akartam egybegyûjteni a te fiaidat, miképen a tyúk az õ kis csirkéit az õ szárnyai alá, és  ti nem akarátok!
\par 35 Ímé pusztán hagyatik néktek a ti házatok; és bizony mondom néktek, hogy nem láttok engem, mígnem eljõ az idõ, mikor ezt mondjátok:  Áldott, a ki jõ az Úrnak nevében!

\chapter{14}

\par 1 És lõn mikor a fõfarizeusok közül egynek házához ment szombatnapon kenyeret enni, azok leselkednek vala õ utána.
\par 2 És ímé egy vízkóros ember vala õ elõtte.
\par 3 És felelvén Jézus, szóla a törvénytudóknak és a farizeusoknak, mondván: Szabad-é szombatnapon gyógyítani?
\par 4 Azok pedig hallgatának. És õ megfogván azt, meggyógyítá és elbocsátá.
\par 5 És felevén nékik, monda: Ki az közületek, a kinek szamara vagy ökre a kútba esik, és nem vonja ki azt azonnal szombatnapon?
\par 6 És nem felelhetnek vala õ ellene semmit ezekre.
\par 7 És egy példázatot monda a hivatalosoknak, mikor észre vevé, mimódon válogatják a fõ helyeket; mondván nékik:
\par 8 Mikor valaki lakodalomba hív, ne ülj a fõ helyre; mert netán náladnál nagyobb tiszteletben álló embert is hivott meg az,
\par 9 És eljövén az, a ki mind téged, mind azt meghívta, ezt mondja majd néked: Engedd ennek a helyet! És akkor szégyennel az utolsó helyre fogsz ülni.
\par 10 Hanem mikor meghívnak, menj el és ülj le az utolsó helyre; hogy mikor eljõ az, a ki téged meghívott, ezt mondja néked: Barátom ülj feljebb! Akkor néked dicsõséged lesz azok elõtt, a kik veled együtt ülnek.
\par 11 Mert mindenki, a ki magát felmagasztalja, megaláztatik; és a ki magát megalázza, felmagasztaltatik.
\par 12 Monda pedig annak is, a ki õt meghívta: Mikor ebédet vagy vacsorát készítesz, ne hívd barátaidat, se testvéreidet, se rokonaidat, se gazdag szomszédaidat; nehogy viszont õk is meghívjanak téged, és visszafizessék néked.
\par 13 Hanem mikor lakomát készítesz, hívd a szegényeket, csonkabonkákat, sántákat, vakokat:
\par 14 És boldog leszel; mivelhogy nem fizethetik vissza néked; mert majd visszafizettetik néked az igazak feltámadásakor.
\par 15 Hallván pedig ezeket egy azok közül, a kik õ vele együtt ülnek vala, monda néki: Boldog az, a ki eszik kenyeret az Isten országában.
\par 16 Õ pedig monda annak: Egy ember készíte nagy vacsorát, és sokakat meghíva;
\par 17 És elküldé szolgáját a vacsora idején, hogy megmondja a hivatalosoknak: Jertek el, mert immár minden kész!
\par 18 És mindnyájan egyenlõképen kezdék magokat mentegetni. Az elsõ monda néki: Szántóföldet vettem, és ki kell mennem, hogy azt meglássam; kérlek téged, ments ki engem!
\par 19 És másik monda: Öt iga ökröt vettem, és elmegyek, hogy azokat megpróbáljam; kérlek téged, ments ki engem!
\par 20 A másik pedig monda: Feleséget vettem, és azért nem mehetek.
\par 21 Mikor azért az a szolga haza ment, megmondá ezeket az õ urának. Akkor megharagudván a gazda, monda az õ szolgájának: Eredj hamar a város utczáira és szorosaira, és a szegényeket, csonkabonkákat, sántákat és vakokat hozd be ide.
\par 22 És monda a szolga: Uram, meglett a mint parancsolád, és mégis van hely.
\par 23 Akkor monda az úr a szolgájának: Eredj el az utakra és a sövényekhez, és kényszeríts bejõni mindenkit, hogy megteljék az én házam.
\par 24 Mert mondom néktek, hogy senki azok közül a hivatalos férfiak  közül meg nem kóstolja az én vacsorámat.
\par 25 Megy vala pedig õ vele nagy sokaság; és megfordulván, monda azoknak:
\par 26 Ha valaki én hozzám jõ, és meg nem gyûlöli az õ atyját és anyját, feleségét és gyermekeit, fitestvéreit és nõtestvéreit, sõt még a maga lelkét is, nem lehet az én tanítványom.
\par 27 És valaki nem hordozza az õ keresztjét, és én utánam jõ, nem lehet az én tanítványom.
\par 28 Mert ha közületek valaki tornyot akar építeni, nemde elõször leülvén felszámítja a költséget, ha van-é mivel elvégezze?
\par 29 Nehogy minekutána fundamentomot vetett, és elvégezni nem bírja, csúfolni kezdje õt mindenki, a ki látja,
\par 30 Ezt mondván: Ez az ember elkezdette az építést, és nem bírta véghez vinni!
\par 31 Vagy valamely király, mikor háborúba megy, hogy egy másik királlyal megütközzék, nemde leülvén elõször tanácskozik, hogy tízezerrel szembeszállhat-é azzal, a ki õ ellene húszezerrel jött?
\par 32 Mert különben még mikor amaz távol van, követséget küldvén, megkérdezi a békefeltételeket.
\par 33 Ezenképen azért valaki közületek búcsút nem vesz minden javaitól, nem lehet az én tanítványom.
\par 34 Jó a só: de ha a só megízetlenül, mivel sózzák meg?
\par 35 Sem a földre, sem a trágyára nem alkalmas: kivetik azt. A kinek van füle a hallásra, hallja.

\chapter{15}

\par 1 Közelgetnek vala pedig õ hozzá a vámszedõk és a bûnösök mind, hogy hallgassák õt.
\par 2 És zúgolódának a farizeusok és az írástudók, mondván: Ez bûnösöket fogad magához, és velök együtt eszik.
\par 3 Õ pedig ezt a példázatot beszélé nékik, mondván:
\par 4 Melyik ember az közületek, a kinek ha száz juha van, és egyet azok közül elveszt, nem hagyja ott a kilenczvenkilenczet a pusztában, és nem megy az elveszett után, mígnem megtalálja  azt?
\par 5 És ha megtalálta, felveti az õ vállára, örülvén.
\par 6 És haza menvén, egybehívja barátait és szomszédait, mondván nékik: Örvendezzetek én velem, mert megtaláltam az én juhomat, a mely elveszett vala.
\par 7 Mondom néktek, hogy ily módon nagyobb öröm lesz a mennyben egy megtérõ bûnösön, hogynem kilenczvenkilencz igaz emberen, a kinek nincs szüksége megtérésre.
\par 8 Avagy ha valamely asszonynak tíz drakhmája van, és egy drakhmát elveszt, nemy gyújt-é gyertyát, és nem sepri-é ki a házat, és nem keresi-é gondosan, mígnem megtalálja?
\par 9 És ha megtalálta, egybehívja az õ asszonybarátait és szomszédait, mondván: Örüljetek én velem, mert megtaláltam a drakhmát, melyet elvesztettem vala!
\par 10 Ezenképen, mondom néktek, örvendezés van az Isten angyalainak színe elõtt egy bûnös ember megtérésén.
\par 11 Monda pedig: Egy embernek vala két fia;
\par 12 És monda az ifjabbik az õ atyjának: Atyám, add ki a vagyonból rám esõ részt! És az megosztá köztök a vagyont.
\par 13 Nem sok nap mulva aztán a kisebbik fiú összeszedvén mindenét, messze vidékre költözék; és ott eltékozlá vagyonát, mivelhogy dobzódva élt.
\par 14 Minekutána pedig mindent elköltött, támada nagy éhség azon a vidéken, és õ kezde szükséget látni.
\par 15 Akkor elmenvén, hozzá szegõdék annak a vidéknek egyik polgárához; és az elküldé õt az õ mezeire disznókat legeltetni.
\par 16 És kívánja vala megtölteni az õ gyomrát azzal a moslékkal, a mit a disznók ettek; és senki sem ád vala néki.
\par 17 Mikor aztán magába szállt, monda: Az én atyámnak mily sok bérese bõvölködik kenyérben, én pedig éhen halok meg!
\par 18 Fölkelvén elmegyek az én atyámhoz, és ezt mondom néki: Atyám, vétkeztem az ég ellen és te ellened.
\par 19 És nem vagyok immár méltó, hogy a te fiadnak hivattassam; tégy engem olyanná, mint a te béreseid közül egy!
\par 20 És felkelvén, elméne az õ atyjához. Mikor pedig még távol volt, meglátá õt az õ atyja, és megesék rajta a szíve, és oda futván, a nyakába esék, és megcsókolgatá õt.
\par 21 És monda néki a fia: Atyám, vétkeztem az ég ellen és te ellened: és nem vagyok immár méltó, hogy a te fiadnak hivattassam!
\par 22 Az atyja pedig monda az õ szolgáinak: Hozzátok ki a legszebb ruhát, és adjátok fel rá; és húzzatok gyûrût a kezére, és sarut a lábaira!
\par 23 És elõhozván a hízott tulkot, vágjátok le, és együnk és vígadjunk.
\par 24 Mert ez az én fiam meghalt, és feltámadott; elveszett, és megtaláltatott. Kezdének azért vígadni.
\par 25 Az õ nagyobbik fia pedig a mezõn vala: és mikor hazajövén, közelgetett a házhoz, hallá a zenét és tánczot.
\par 26 És elõszólítván egyet a szolgák közül, megtudakozá, mi dolog az?
\par 27 Az pedig monda néki: A te öcséd jött meg; és atyád levágatá a hízott tulkot, mivelhogy egészségben nyerte õt vissza.
\par 28 Erre õ megharaguvék, és nem akara bemenni. Az õ atyja annakokáért kimenvén, kérlelé õt.
\par 29 Õ pedig felelvén, monda atyjának: Ímé ennyi esztendõtõl fogva szolgálok néked, és soha parancsolatodat át nem hágtam: és nékem soha nem adtál egy kecskefiat, hogy az én barátaimmal vígadjak.
\par 30 Mikor pedig ez a te fiad megjött, a ki paráznákkal emésztette föl a te vagyonodat, levágattad néki a hízott tulkot.
\par 31 Az pedig monda néki: Fiam, te mindenkor én velem vagy, és mindenem a tiéd!
\par 32 Vígadnod és örülnöd kellene hát, hogy ez a te testvéred meghalt, és feltámadott; és elveszett, és megtaláltatott.

\chapter{16}

\par 1 Monda pedig az õ tanítványainak is: Vala egy gazdag ember, kinek vala egy sáfára; és az bevádoltaték nála, hogy javait eltékozolja.
\par 2 Hívá azért azt, és monda néki: Mit hallok felõled? Adj számot a te sáfárságodról; mert nem lehetsz tovább sáfár.
\par 3 Monda pedig magában a sáfár: Mit míveljek, mivelhogy az én uram elveszi tõlem a sáfárságot? Kapálni nem tudok; koldulni szégyenlek!
\par 4 Tudom mit tegyek, hogy mikor a sáfárságtól megfosztatom, befogadjanak engem házaikba.
\par 5 És magához hivatván az õ urának minden egyes adósát, monda az elsõnek: Mennyivel tartozol az én uramnak?
\par 6 Az pedig monda: Száz bátus olajjal. És monda néki: Vedd a te írásodat, és leülvén, hamar írj ötvent.
\par 7 Azután monda másnak: Te pedig mennyivel tartozol? Az pedig monda: Száz kórus búzával. És monda annak: Vedd a te írásodat, és írj nyolczvanat.
\par 8 És dícséré az úr a hamis sáfárt, hogy eszesen cselekedett, mert e világnak fiai eszesebbek a világosságnak fiainál a maguk nemében.
\par 9 Én is mondom néktek, szerezzetek magatoknak barátokat a hamis mammonból, hogy mikor meghaltok, befogadjanak benneteket az örök hajlékokba.
\par 10 A ki hû a kevesen, a sokon is hû az; és a ki a kevesen hamis, a sokon is hamis az.
\par 11 Ha azért a hamis mammonon hívek nem voltatok, ki bízná reátok az igazi kincset?
\par 12 És ha a másén hívek nem voltatok, ki adja oda néktek, a mi a tiétek?
\par 13 Egy szolga sem szolgálhat két úrnak: mert vagy az egyiket gyûlöli és a másikat szereti; vagy az egyikhez ragaszkodik, és a másikat megveti. Nem szolgálhattok az Istennek és a mammonnak.
\par 14 Hallák pedig mindezeket a farizeusok is, kik pénzszeretõk valának; és csúfolák õt.
\par 15 És monda nékik: Ti vagytok, a kik az emberek elõtt magatokat megigazítjátok; de az Isten ismeri a  ti szíveteket: mert a mi az emberek közt magasztos, az Isten elõtt útálatos.
\par 16 A törvény és a próféták Keresztelõ Jánosig valának: az idõtõl fogva az Istennek  országa hirdettetik, és mindenki erõszakkal tör abba.
\par 17 Könnyebb pedig a mennynek és a földnek elmúlni, hogynem a törvénybõl egy  pontocskának elesni.
\par 18 Valaki elbocsátja feleségét, és mást vesz el, paráználkodik; és valaki férjétõl elbocsátott asszonyt vesz feleségül, paráználkodik.
\par 19 Vala pedig egy gazdag ember, és öltözik vala bíborba és patyolatba, mindennap dúsan vigadozván:
\par 20 És vala egy Lázár nevû koldus, ki az õ kapuja elé volt vetve, fekélyekkel tele.
\par 21 És kíván vala megelégedni a morzsalékokkal, melyek hullanak vala a gazdagnak asztaláról; de az ebek is eljõvén, nyalják vala az õ sebeit.
\par 22 Lõn pedig, hogy meghala a koldus, és viteték az angyaloktól az Ábrahám kebelébe; meghala pedig a gazdag is, és eltemetteték.
\par 23 És a pokolban felemelé az õ szemeit, kínokban lévén, és látá Ábrahámot távol, és Lázárt annak kebelében.
\par 24 És õ kiáltván, monda: Atyám Ábrahám! könyörülj rajtam, és bocsásd el Lázárt, hogy mártsa az õ ujjainak hegyét vízbe, és hûsítse meg az én nyelvemet; mert gyötrettetem e lángban.
\par 25 Monda pedig Ábrahám: Fiam, emlékezzél meg róla, hogy te javaidat elvetted a te életedben, hasonlóképen Lázár is az õ bajait: most pedig ez vígasztaltatik, te pedig gyötrettetel.
\par 26 És mindenekfelett, mi köztünk és ti közöttetek nagy közbevetés van, úgy, hogy a kik akarnának innét ti hozzátok általmenni, nem mehetnek, sem azok onnét hozzánk át nem jöhetnek.
\par 27 Monda pedig amaz: Kérlek azért téged Atyám, hogy bocsásd el õt az én atyámnak házához.;
\par 28 Mert van öt testvérem; hogy bizonyságot tegyen nékik, hogy õk is ide, e gyötrelemnek helyére ne jussanak.
\par 29 Monda néki Ábrahám: Van Mózesök és prófétáik; hallgassák azokat.
\par 30 Ama pedig monda: Nem úgy, atyám Ábrahám; hanem ha a halottak közül megy valaki hozzájok, megtérnek!
\par 31 Õ pedig monda néki: Ha Mózesre és a prófétákra nem hallgatnak, az sem gyõzi meg õket, ha valaki a halottak közül feltámad.

\chapter{17}

\par 1 Monda pedig a tanítványoknak: Lehetetlen dolog, hogy botránkozások ne essenek; de jaj annak, a ki által esnek.
\par 2 Jobb annak, ha egy malomkövet vetnek a nyakába, és ha a tengerbe vettetik, hogynem mint egyet e kicsinyek közül megbotránkoztasson.
\par 3 Õrizzétek meg magatokat: ha pedig a te atyádfia vétkezik ellened, dorgáld meg õt; és ha megtér, bocsáss meg néki.
\par 4 És ha egy napon hétszer vétkezik ellened, és egy napon hétszer te hozzád tér, mondván: Megbántam; megbocsáss neki.
\par 5 És mondának az apostolok az Úrnak: Növeljed a mi hitünket!
\par 6 Monda pedig az Úr: Ha annyi hitetek volna, mint a mustármag, ezt mondanátok íme ez eperfának: Szakadj ki gyökerestõl, és plántáltassál a tengerbe; és engede néktek.
\par 7 Kicsoda pedig ti közületek az, a ki, ha egy szolgája van, és az szánt vagy legeltet, tüstént azt mondja annak, mihelyt a mezõrõl megjõ: Jer elõ, ülj asztalhoz?
\par 8 Sõt nem ezt mondja-e néki: Készíts vacsorámra valót, és felövezvén magadat, szolgálj nékem, míg én eszem és iszom; és azután egyél és igyál te?
\par 9 Avagy megköszöni-é annak a szolgának, hogy azt mívelte, a mit néki parancsolt? Nem gondolom.
\par 10 Ezenképen ti is, ha mindazokat megcselekedtétek, a mik néktek parancsoltattak, mondjátok, hogy: Haszontalan szolgák vagyunk; mert a mit kötelesek voltunk cselekedni, azt cselekedtük.
\par 11 És lõn, mikor útban vala Jeruzsálem felé, hogy õ Samariának és Galileának közepette méne által.
\par 12 És mikor egy faluba beméne, jöve elébe tíz bélpoklos férfi, kik távol megállának:
\par 13 És felemelék szavokat, mondván: Jézus, Mester, könyörülj rajtunk!
\par 14 És mikor õket látta, monda nékik: Elmenvén mutassátok meg magatokat a papoknak. És lõn, hogy míg odamenének, megtisztulának.
\par 15 Egy pedig õ közülök, mikor látta, hogy meggyógyult, visszatére, dicsõítvén az Istent nagy szóval;
\par 16 És arczczal leborula az õ lábainál hálákat adván néki: és az Samariabeli vala.
\par 17 Felelvén pedig Jézus, monda: Avagy nem tízen tisztulának-é meg? A kilencze pedig hol van?
\par 18 Nem találkoztak a kik visszatértek volna dicsõséget adni az Istennek, csak ez az idegen?
\par 19 És monda néki: Kelj föl, és menj el: a te hited téged megtartott.
\par 20 Megkérdeztetvén pedig a farizeusoktól, mikor jõ el az Isten országa, felele nékik és monda: Az Isten országa nem szemmel láthatólag jõ el.
\par 21 Sem azt nem mondják: Ímé itt, vagy: Ímé amott van; mert ímé az Isten országa ti bennetek van.
\par 22 Monda pedig a tanítványoknak: Eljõ az idõ, mikor kívántok látni egyet az ember Fiának napjai közül, és nem láttok.
\par 23 És mondják majd néktek: Ímé itt, vagy: Ímé amott van; de ne menjetek el, és ne kövessétek:
\par 24 Mert miként a felvillanó villámlás az ég aljától az ég aljáig fénylik; úgy lesz az embernek Fia is az õ napján.
\par 25 De elõbb sokat kell neki szenvednie és megvettetnie e nemzetségtõl.
\par 26 És miként a Noé napjaiban lett, úgy lesz az ember Fiának napjaiban is.
\par 27 Ettek, ittak, házasodtak, férjhezmentek mindama napig, a melyen Noé a bárkába beméne, és eljöve az özönvíz, és mindeneket elveszte.
\par 28 Hasonlóképen mint a Lót napjaiban is lett; ettek, ittak, vettek, adtak, ültettek, építettek;
\par 29 De a mely napon kiment Lót Sodomából, tûz és kénkõ esett az égbõl, és mindenkit elvesztett:
\par 30 Ezenképen lesz azon a napon, melyen az embernek Fia megjelenik.
\par 31 Az nap, a ki a háztetõn lesz, és az õ holmija a házban, ne szálljon le, hogy elvigye; és a ki a mezõn, azonképen ne forduljon hátra.
\par 32 Emlékezzetek Lót feleségére!
\par 33 Valaki igyekezik az õ életét megtartani, elveszti azt, és valaki elveszti azt, megeleveníti azt.
\par 34 Mondom néktek, azon az éjszakán ketten lesznek egy ágyban; az egyik felvétetik, és a másik elhagyatik.
\par 35 Két asszony õröl együtt; az egyik felvétetik, és a másik elhagyatik.
\par 36 Ketten lesznek a mezõn; az egyik felvétetik, és a másik elhagyatik.
\par 37 És felelvén, mondának néki: Hol, Uram? Õ pedig monda nékik: a hol a test, oda gyûlnek a saskeselyûk.

\chapter{18}

\par 1 Monda pedig nékik példázatot is arról, hogy mindig imádkozni kell, és meg nem restülni;
\par 2 Mondván: Volt egy bíró egy városban, a ki Istent nem félt és embert nem becsült.
\par 3 Volt pedig abban a városban egy özvegyasszony, és elméne ahhoz, mondván: Állj bosszút értem az én ellenségemen.
\par 4 Az pedig nem akará egy ideig; de azután monda õ magában: Jól lehet Istent nem félek és embert nem becsülök;
\par 5 Mindazáltal mivelhogy nékem terhemre van ez az özvegyasszony, megszabadítom õt, hogy szüntelen reám járván, ne gyötörjön engem.
\par 6 Monda pedig az Úr: Halljátok, mit mond e hamis bíró!
\par 7 Hát az Isten nem áll-é bosszút az õ választottaiért, kik õ hozzá kiáltának éjjel és nappal, ha hosszútûrõ is irántuk?
\par 8 Mondom néktek, hogy bosszút áll értök hamar. Mindazáltal az embernek Fia mikor eljõ, avagy talál-é hitet e földön?
\par 9 Némelyeknek pedig, kik elbizakodtak magukban, hogy õk igazak, és a többieket semmibe sem vették, ezt a példázatot is mondá:
\par 10 Két ember méne fel a templomba imádkozni; az egyik farizeus, és a másik  vámszedõ.
\par 11 A farizeus megállván, ily módon imádkozék magában: Isten! hálákat adok néked, hogy nem vagyok olyan, mint egyéb emberek, ragadozók, hamisak, paráznák, vagy mint ím e vámszedõ is.
\par 12 Bõjtölök kétszer egy héten; dézsmát adok mindenbõl, a mit szerzek.
\par 13 A vámszedõ pedig távol állván, még szemeit sem akarja vala az égre emelni, hanem veri vala mellét, mondván: Isten, légy irgalmas nékem bûnösnek!
\par 14 Mondom néktek, ez megigazulva méne alá az õ házához, inkább hogynem amaz: mert valaki felmagasztalja magát, megaláztatik; és a ki megalázza magát, felmagasztaltatik.
\par 15 Vivének pedig hozzá kis gyermekeket is, hogy illesse azokat; mikor pedig a tanítványok ezt látták, megdorgálák azokat.
\par 16 De Jézus mágához híván õket, monda: Engedjétek, hogy a kis gyermekek én hozzám jõjjenek, és ne tiltsátok el õket; mert iyeneké az Istennek országa.
\par 17 Bizony mondom néktek: A ki nem úgy fogadja az Isten országát, mint gyermek, semiképen nem megy be abba.
\par 18 És megkérdé õt egy fõember, mondván: Jó Mester, mit cselekedjem, hogy az örökéletet elnyerhessem?
\par 19 Monda pedig néki Jézus: Miért mondasz engem jónak? Nincs senki jó, csak egy, az Isten.
\par 20 A parancsolatokat tudod: Ne paráználkodjál; ne ölj; ne lopj; hamis tanubizonyságot ne tégy; tiszteld atyádat és anyádat.
\par 21 Az pedig monda: Mindezeket ifjúságomtól fogva megtartottam.
\par 22 Jézus ezeket hallván, monda néki: Még egy fogyatkozás van benned: Add el mindenedet, a mid van, és oszd el a szegényeknek, és kincsed lesz mennyországban; és jer, kövess engem.
\par 23 Az pedig ezeket hallván, igen megszomorodék; mert igen gazdag vala.
\par 24 És mikor látta Jézus, hogy az igen megszomorodék, monda: Mily nehezen mennek be az Isten országába, a kiknek gazdagságuk van!
\par 25 Mert könnyebb a tevének a tû fokán átmenni, hogynem a gazdagnak az Isten országába bejutni.
\par 26 A kik pedig ezt hallották, mondának: Ki idvezülhet tehát?
\par 27 Õ pedig monda: A mi embereknél lehetetlen, lehetséges az Istennél.
\par 28 És monda Péter: Ímé mi mindent elhagytunk, és követtünk téged!
\par 29 Õ pedig monda nékik: Bizony mondom néktek, hogy senki sincs, a ki elhagyta házát, vagy szüleit, vagy testvéreit, vagy feleségét, vagy gyermekeit az Isten országáért,
\par 30 A ki sokszorta többet ne kapna ebben az idõben, a jövendõ világon pedig örök életet.
\par 31 És maga mellé vévén a tizenkettõt, monda nékik: Ímé felmegyünk Jeruzsálembe, és beteljesedik minden az embernek Fián, a mit a próféták megírtak.
\par 32 Mert a pogányok kezébe adatik, és megcsúfoltatik, és meggyaláztatik, és megköpdöstetik;
\par 33 És megostorozván, megölik õt; és harmadnapon feltámad.
\par 34 Õk pedig ezekbõl semmit nem értének; és a beszéd õ elõlük el vala rejtve, és nem fogták fel a mondottakat.
\par 35 Lõn pedig, mikor Jerikóhoz közeledett, egy vak ül vala az út mellett koldulván.
\par 36 És mikor hallotta a mellette elmenõ sokaságot, tudakozódék, mi dolog az?
\par 37 Megmondák pedig néki, hogy a Názáretbeli Jézus megy el arra.
\par 38 És kiálta, mondván: Jézus, Dávidnak Fia, könyörülj rajtam!
\par 39 A kik pedig elõl mentek, dorgálák õt, hogy hallgasson; de õ annál inkább kiálta: Dávidnak Fia, könyörülj rajtam!
\par 40 És Jézus megállván, parancsolá, hogy vigyék azt hozzá; és mikor közel ért, megkérdé õt,
\par 41 Mondván: Mit akarsz, hogy cselekedjem veled? Az pedig monda: Uram, hogy az én szemem világa megjõjjön.
\par 42 És Jézus monda néki: Láss, a te hited téged megtartott.
\par 43 És azonnal megjöve annak szeme világa, és követé õt, dicsõítvén az Istent az egész sokaság pedig ezt látván, dicsõséget ada az Istennek.

\chapter{19}

\par 1 És bemenvén, általméne Jerikhón.
\par 2 És ímé vala ott egy ember, a kit nevérõl Zákeusnak hívtak; és az fõvámszedõ vala, és gazdag.
\par 3 És igyekezék Jézust látni, ki az; de a sokaságtól nem láthatá, mivelhogy termete szerint kis ember volt.
\par 4 És elõre futván felhága egy eperfüge fára, hogy õt lássa; mert arra vala elmenendõ.
\par 5 És mikor arra a helyre jutott, feltekintvén Jézus, látá õt, és monda néki: Zákeus, hamar szállj alá; mert ma nékem a te házadnál kell maradnom.
\par 6 És sietve leszálla, és örömmel fogadá õt.
\par 7 És mikor ezt látták, mindnyájan zúgolódának, mondván hogy: Bûnös emberhez ment be szállásra.
\par 8 Zákeus pedig elõállván, monda az Úrnak: Uram, ímé minden vagyonomnak felét a szegényeknek adom, és ha valakitõl valamit patvarkodással elvettem, négy annyit adok helyébe.
\par 9 Monda pedig néki Jézus: Ma lett idvessége ennek a háznak! mivelhogy õ is Ábrahám fia.
\par 10 Mert azért jött az embernek Fia, hogy megkeresse és megtartsa, a mi elveszett.
\par 11 És mikor azok ezeket hallották, folytatá és monda egy példázatot, mivelhogy közel vala Jeruzsálemhez, és azok azt gondolák, hogy azonnal megjelenik az Isten országa.
\par 12 Monda azért: Egy nemes ember elméne messze tartományba, hogy országot vegyen magának, aztán visszatérjen.
\par 13 Elõszólítván azért tíz szolgáját, ada nékik tíz gírát, és monda nékik: Kereskedjetek, míg megjövök.
\par 14 Az õ alattvalói pedig gyûlölék õt, és követséget küldének utána, mondván: Nem akarjuk, hogy õ uralkodjék mi rajtunk.
\par 15 És lõn, mikor megjött az ország vétele után, parancsolá, hogy az õ szolgáit, a kiknek a pénzt adta, hívják õ hozzá, hogy megtudja, ki mint kereskedett.
\par 16 Eljöve pedig az elsõ, mondván: Uram, a te gírád tíz gírát nyert.
\par 17 Õ pedig monda néki: Jól vagyon jó szolgám; mivelhogy kevesen voltál hív, legyen birodalmad tíz városon.
\par 18 És jöve a második, mondván: Uram, a te gírád öt gírát nyert.
\par 19 Monda pedig ennek is: Néked is legyen birodalmad öt városon.
\par 20 És jöve egy másik, mondván: Uram, imhol a te gírád, melyet egy keszkenõben eltéve tartottam;
\par 21 Mert féltem tõled, mivelhogy kemény ember vagy; elveszed a mit nem te tettél el, és aratod, a mit nem te vetettél.
\par 22 Monda pedig annak: A te szádból ítéllek meg téged, gonosz szolga. Tudtad, hogy én kemény ember vagyok, ki elveszem, a mit nem én tettem el, és aratom, a mit nem én vetettem;
\par 23 Miért nem adtad azért az én pénzemet a pénzváltók asztalára, és én megjövén, kamatostól kaptam volna azt vissza?
\par 24 És az ott állóknak monda: Vegyétek el ettõl a gírát, és adjátok annak, a kinek tíz gírája van.
\par 25 És mondának néki: Uram, tíz gírája van!
\par 26 És õ monda: Mert mondom néktek, hogy mindenkinek, a kinek van, adatik; a kinek pedig nincs, még a mije van is, elvétetik tõle.
\par 27 Sõt ennek felette amaz én ellenségeimet is, kik nem akarták, hogy én õ rajtok uralkodjam, hozzátok ide, és öljétek meg elõttem!
\par 28 És ezeket mondván, megy vala elõl, felmenvén Jeruzsálembe.
\par 29 És lõn, mikor közelgetett Béthfágéhoz és Bethániához, a hegyhez, mely Olajfák hegyének hívatik, elkülde kettõt az õ tanítványai közül,
\par 30 Mondván: Menjetek el az átellenben levõ faluba; melybe bemenvén, találtok egy megkötött vemhet, melyen soha egy ember sem ült: eloldván azt, hozzátok ide.
\par 31 És ha valaki kérdez titeket: Miért oldjátok el? ezt mondjátok annak: Mert az Úrnak szüksége van reá.
\par 32 És elmenvén a küldöttek, úgy találák, a mint nékik mondotta.
\par 33 És mikor a vemhet eloldák, mondának nékik annak gazdái: Miért oldjátok el a vemhet?
\par 34 Õk pedig mondának: Az Úrnak szüksége van reá.
\par 35 Elvivék azért azt Jézushoz: és az õ felsõruháikat a vemhére vetvén, Jézust reá helyhezteték.
\par 36 És mikor õ méne, az õ felsõruháikat az útra teríték.
\par 37 Mikor pedig immár közelgete az Olajfák hegyének lejtõjéhez, a tanítványok egész sokasága örvendezve kezdé dicsérni az Istent fenszóval mindazokért a csodákért, a melyeket láttak;
\par 38 Mondván: Áldott a Király, ki jõ az Úrnak nevében! Békesség a mennyben, és dicsõség a magasságban!
\par 39 És némelyek a farizeusok közül a sokaságból mondának néki: Mester, dorgáld meg a tanítványaidat!
\par 40 És õ felelvén, monda nékik: Mondom néktek, hogyha ezek elhallgatnak, a kövek fognak kiáltani.
\par 41 És mikor közeledett, látván a várost, síra azon.
\par 42 Mondván: Vajha megismerted volna te is, csak e te mostani napodon is, a mik néked a te békességedre valók! de most elrejtettek a te szemeid elõl.
\par 43 Mert jõnek reád napok, mikor a te ellenségeid te körülted palánkot építenek, és körülvesznek téged, és mindenfelõl megszorítanak téged.
\par 44 És a földre tipornak téged, és a te fiaidat te benned; és nem hagynak te benned követ kövön; mivelhogy nem ismerted meg a te  meglátogatásodnak idejét.
\par 45 És bemenvén a templomba, kezdé kiûzni azokat, a kik adnak és vesznek vala abban,
\par 46 Mondván nékik: Meg van írva: Az én házam imádságnak háza; ti pedig azt latroknak barlangjává tettétek.
\par 47 És tanít vala minden nap a templomban. A fõpapok pedig és az írástudók és a nép elõkelõi igyekeznek vala õt elveszteni:
\par 48 És nem találták el, mit cselekedjenek; mert az egész nép õ rajta függ vala, reá hallgatván.

\chapter{20}

\par 1 És lõn azok közül a napok közül egyen, mikor õ a népet tanítá a templomban, és és az evangyéliomot hirdeté, elõállának a fõpapok és az írástudók a vénekkel egybe,
\par 2 És mondának néki, így szólván: Mondd meg nékünk, micsoda hatalommal cselekszed ezeket? vagy ki az, a ki néked ezt a hatalmat adta?
\par 3 Felelvén pedig, monda nékik: Én is kérdek egy dolgot tõletek; mondjátok meg azért nékem:
\par 4 A János keresztsége mennybõl vala-é, vagy emberektõl?
\par 5 Azok pedig tanakodának magok közt, mondván: Ha ezt mondjuk: Mennybõl; azt fogja mondani: Miért nem hittetek tehát néki?
\par 6 Ha pedig ezt mondjuk: Emberektõl; az egész nép megkövez minket: mert meg van gyõzõdve, hogy János próféta volt.
\par 7 Felelének azért, hogy õk nem tudják, honnét vala.
\par 8 És Jézus monda nékik: Én sem mondom meg néktek, micsoda hatalommal cselekszem ezeket.
\par 9 És kezdé a népnek e példázatot mondani: Egy ember plántála szõlõt, és kiadá azt munkásoknak, és hosszú idõre elutazék.
\par 10 És annak idején elküldé szolgáját a munkásokhoz, hogy adjanak néki a szõlõ gyümölcsébõl; a munkások pedig azt megvervén, üresen bocsáták el.
\par 11 És még másik szolgát is külde; de azok azt is megvervén és meggyalázván, üresen bocsáták el.
\par 12 És még harmadikat is külde; de azok azt is megsebesítvén, kiveték.
\par 13 Monda azért a szõlõnek ura: Mit cselekedjem? Elküldöm az én szerelmes fiamat: talán azt, ha látják, megbecsülik.
\par 14 De mikor azt látták a munkások, tanakodának egymás közt, mondván: Ez az örökös; jertek, öljük meg õt, hogy a miénk legyen az örökség!
\par 15 És kivetvén õt a szõlõbõl, megölék. Mit cselekszik azért a szõlõnek ura azokkal?
\par 16 Elmegy és elveszti azokat a munkásokat, és a szõlõt  másoknak adja. És mikor ezt hallották, mondának: Távol legyen az!
\par 17 Õ pedig reájok tekintvén, monda: Mi az tehát, a mi meg van írva: A mely követ az építõk megvetettek, az lett a szegelet fejévé?
\par 18 Valaki erre a kõre esik, szétzúzatik; a kire pedig ez esik reá, szétmorzsolja azt.
\par 19 És igyekeznek vala a fõpapok és az írástudók kezeiket õ reá vetni azon órában; de félének a néptõl; mert megérték, hogy õ ellenök mondta e példázatot.
\par 20 Annakokáért vigyázván õ reá, leselkedõket küldének ki, a kik igazaknak tetteték magokat, hogy õt megfogják beszédében; hogy átadják a felsõbbségnek és a helytartó hatalmának.
\par 21 Kik megkérdezék õt, mondván: Mester, tudjuk, hogy te helyesen beszélsz és tanítasz, és személyt nem válogatsz, hanem az Istennek útját igazán tanítod:
\par 22 Szabad-é nékünk adót fizetnünk a császárnak, vagy nem?
\par 23 Õ pedig észrevévén álnokságukat, monda nékik: Mit kísértetek engem?
\par 24 Mutassatok nékem egy pénzt; kinek a képe és felirata van rajta? És felelvén, mondának: A császáré.
\par 25 Õ pedig monda nékik: Adjátok meg azért a mi a császáré, a császárnak, és a mi az Istené, az Istennek.
\par 26 És nem tudták õt megfogni beszédében a nép elõtt; és csodálkozván az õ feleletén, elhallgatának.
\par 27 Hozzá menvén pedig a sadduczeusok közül némelyek, a kik tagadják, hogy van feltámadás, megkérdék õt,
\par 28 Mondván: Mester, Mózes megírta nékünk, ha valakinek testvére meghal, kinek felesége volt, és magzatok nélkül hal meg, hogy annak testvére elvegye annak feleségét, és támasszon magot az õ testvérének.
\par 29 Hét testvér vala azért: és az elsõ feleséget vevén, meghalt magzatok nélkül;
\par 30 A másik vevé el azért annak feleségét, és az is magzatok nélkül halt meg;
\par 31 Akkor a harmadik vette el azt; és hasonlóképen mind a heten is; és nem hagytak magot, és meghaltak.
\par 32 Mind ezek után pedig meghalt az asszony is.
\par 33 A feltámadáskor azért kinek a felesége lesz közülök? mert mind a hétnek felesége volt.
\par 34 És felelvén, monda nékik Jézus: E világnak fiai házasodnak és férjhez adatnak:
\par 35 De a kik méltókké tétetnek, hogy ama világot elvegyék, és a halálból való feltámadást, sem nem házasodnak, sem férjhez nem adatnak:
\par 36 Mert meg sem halhatnak többé: mert hasonlók az angyalokhoz: és az Isten fiai, mivelhogy a feltámadásnak fiai.
\par 37 Hogy pedig a halottak feltámadnak, Mózes is megjelentette a csipkebokornál, mikor az Urat Ábrahám Istenének és Izsák Istenének és Jákób Istenének mondja.
\par 38 Az Isten pedig nem a holtaknak, hanem az élõknek Istene: mert mindenek élnek õ néki.
\par 39 Felelvén pedig némelyek az írástudók közül, mondának: Mester, jól mondád!
\par 40 És többé semmit sem mertek tõle kérdezni.
\par 41 Monda pedig nékik: Mimódon mondják, hogy a Krisztus Dávidnak fia?
\par 42 Holott maga Dávid mondja a zsoltárok könyvében: Monda az Úr az én Uramnak, ülj az én jobbkezem felõl,
\par 43 Míglen vetem a te ellenségeidet a te lábaid alá zsámolyul.
\par 44 Dávid azért Urának mondja õt, mimódon fia tehát néki?
\par 45 És az egész nép hallására monda az õ tanítványainak:
\par 46 Oltalmazzátok meg magatokat az írástudóktól, kik hosszú köntösökben akarnak járni, és szeretik a piaczokon való köszöntéseket, és a gyülekezetekben az elsõ ûlést, és a lakomákon a fõhelyeket;
\par 47 Kik az özvegyeknek házát felemésztik, és színbõl hosszan imádkoznak; ezek súlyosabb ítélet alá esnek.

\chapter{21}

\par 1 És mikor feltekintett, látá, hogy a gazdagok hányják az õ ajándékaikat a perselybe.
\par 2 Láta pedig egy szegény özvegy asszonyt is, hogy abba két fillért vete.
\par 3 És monda: Igazán mondom néktek, hogy e szegény özvegy mindenkinél többet vete:
\par 4 Mert mind ezek az õ fölöslegükbõl vetettek Istennek az ajándékokhoz: ez pedig az õ szegénységébõl minden vagyonát, a mije volt, oda veté.
\par 5 És mikor némelyek mondának a templom felõl, hogy szép kövekkel és ajándékokkal van felékesítve, monda:
\par 6 Ezekbõl, a miket láttok, jõnek napok, melyekben kõ kövön nem marad, mely le nem romboltatnék.
\par 7 Megkérdék pedig õt, mondván: Mester, mikor lesznek azért ezek? és mi lesz a jel, mikor mind ezek meglesznek?
\par 8 Õ pedig monda: Meglássátok, hogy el ne hitessenek benneteket: mert sokan jõnek el az én nevemben, kik ezt mondják: Én vagyok; és: Az idõ elközelgett; ne menjetek azért utánok.
\par 9 És mikor hallotok háborúkról és zendülésekrõl, meg ne félemljetek; mert ezeknek meg kell lenni elõbb, de nem jõ mindjárt a vég.
\par 10 Akkor monda nékik: Nemzet nemzet ellen támad, és ország ország ellen;
\par 11 És minden felé nagy földindulások lesznek, és éhségek és döghalálok; és rettegtetések és nagy jelek lesznek az égbõl.
\par 12 De mind ezeknek elõtte kezeiket reátok vetik, és üldöznek titeket, adván a gyülekezetek elé, és tömlöczökbe és királyok és helytartók elé visznek az én nevemért.
\par 13 De ebbõl néktek lesz tanúbizonyságotok.
\par 14 Tökéljétek el azért a ti szívetekben, hogy nem gondoskodtok elõre, hogy mit feleljetek védelmetekre:
\par 15 Mert én adok néktek szájat és bölcseséget, melynek ellene nem szólhatnak, sem ellene nem állhatnak mind azok, a kik magokat ellenetekbe vetik.
\par 16 Elárulnak pedig titeket szülõk és testvérek is, rokonok és barátok is; és  megölnek némelyeket ti közületek.
\par 17 És gyûlöletesek lesztek mindenki elõtt az én nevemért.
\par 18 De fejeteknek egy hajszála sem vész el.
\par 19 A ti béketûréstek által nyeritek meg lelketeket.
\par 20 Mikor pedig látjátok Jeruzsálemet hadseregektõl körülvéve, akkor tudjátok meg, hogy elközelgett az õ elpusztulása.
\par 21 Akkor a kik Júdeában lesznek, fussanak a hegyekre; és a kik annak közepette, menjenek ki abból: és a kik a mezõben, ne menjenek be abba.
\par 22 Mert azok a bosszúállásnak napjai, hogy beteljesedjenek mind azok, a mik megírattak.
\par 23 Jaj pedig a terhes és szoptató asszonyoknak azokban a nappokban; mert nagy szükség lesz e földön, és harag e népen.
\par 24 És elhullanak fegyvernek éle által, és fogva vitetnek minden pogányok közé; és Jeruzsálem megtapodtatik a pogányoktól, míglen betelik a pogányok ideje.
\par 25 És lesznek jelek a napban, holdban és csillagokban; és a földön pogányok szorongása a kétség miatt, mikor a tenger és a hab zúgni fog,
\par 26 Mikor az emberek elhalnak a félelem miatt és azoknak várása miatt, a mik e föld kerekségére következnek: mert az egek erõsségei megrendülnek.
\par 27 És akkor meglátják az embernek Fiát eljõni a felhõben, hatalommal és nagy dicsõséggel.
\par 28 Mikor pedig ezek kezdenek meglenni, nézzetek fel és emeljétek fel a ti fejeteket; mert elközelget a ti váltságtok.
\par 29 Monda pedig nékik egy példázatot: Tekintsétek meg a fügefát és minden fákat:
\par 30 Mikor immár hajtanak, és ezt látjátok, ti magatoktól tudjátok, hogy már közel van a nyár.
\par 31 Ezenképen ti is, mikor látjátok, hogy ezek meglesznek, tudjátok meg, hogy közel van az Isten országa.
\par 32 Bizony mondom néktek, hogy e nemzetség el nem múlik, mígnem mind ezek meglesznek.
\par 33 Az ég és a föld elmúlnak, de az  én beszédeim semmiképen el nem múlnak.
\par 34 De vigyázzatok magatokra, hogy valamikor meg ne nehezedjék a ti szívetek dobzódásának, részegségnek és ez élet gondjainak miatta, és váratlanul reátok ne jõjjön az a  nap:
\par 35 Mert mintegy tõr, úgy lep meg mindeneket, a kik az egész föld színén lakoznak.
\par 36 Vigyázzatok azért minden idõben, kérvén, hogy méltókká tétessetek arra, hogy elkerüljétek mindezeket, a mik bekövetkeznek, és megállhassatok az embernek Fia elõtt!
\par 37 Tanít vala pedig naponta a templomban; éjszakára pedig kimenvén, a hegyen vala, mely Olajfák hegyének neveztetik.
\par 38 És kora reggel hozzá megy vala az egész nép, hogy õt hallgassa a templomban.

\chapter{22}

\par 1 Elközelgett pedig a kovásztalan kenyerek ünnepe, mely husvétnak mondatik.
\par 2 És a fõpapok és az írástudók keresnek vala módot, hogyan öljék meg õt; mert féltek a néptõl.
\par 3 Beméne pedig a Sátán Júdásba, ki Iskáriótesnek neveztetik, és a tizenkettõnek számából vala;
\par 4 És elmenvén, megbeszélé a fõpapokkal és a vezérekkel, mimódon adja õt nékik kezökbe.
\par 5 És azok örülének, és megszerzõdének, hogy pénz adnak néki;
\par 6 Õ pedig megigéré, és keres vala jó alkalmat, hogy õt nékik kezökbe adja zenebona nélkül.
\par 7 Eljöve pedig a kovásztalan kenyerek napja, melyen meg kelle öletni a husvéti báránynak;
\par 8 És elküldé Pétert és Jánost, mondván: Elmenvén, készítsétek el nékünk a husvéti bárányt, hogy megegyük.
\par 9 Õk pedig mondának néki: Hol akarod, hogy elkészítsük?
\par 10 És õ monda nékik: Ímé, mikor bementek a városba, szembe jõ veletek egy ember, ki egy korsó vizet visz; kövessétek õt abba a házba, a melybe bemegy.
\par 11 És mondjátok a ház gazdájának: Ezt mondja néked a Mester: Hol van az a szállás, a hol megeszem az én tanítványaimmal a husvéti bárányt?
\par 12 És õ mutat néktek egy nagy vacsoráló helyet, berendezve, ott készítsétek el.
\par 13 Elmenvén pedig, úgy találák, a mint mondta nékik; és elkészíték a húsvéti bárányt.
\par 14 És mikor eljött az idõ, asztalhoz üle, és a tizenkét apostol õ vele egyetembe.
\par 15 És monda nékik: Kívánva kívántam a husvéti bárányt megenni veletek, melõtt én szenvednék:
\par 16 Mert mondom néktek, hogy többé nem eszem abból, míglen beteljesedik az Isten országában.
\par 17 És a pohárt vévén, minekutána hálákat adott, monda: Vegyétek ezt, és osszátok el magatok között:
\par 18 Mert mondom néktek, hogy nem iszom a szõlõtõkének gyümölcsébõl, míglen eljõ az Isten országa.
\par 19 És minekutána a kenyeret vette, hálákat adván megszegé, és adá nékik, mondván: Ez az én testem, mely ti érettetek adatik: ezt cselekedjétek az én emlékezetemre.
\par 20 Hasonlóképen a pohárt is, minekutána vacsorált, ezt mondván: E pohár amaz új szövetség az én véremben, mely ti érettetek kiontatik.
\par 21 De ímé annak a keze, a ki engem elárul, velem van az asztalon.
\par 22 És az embernek Fia jóllehet, elmegy, mint  elvégeztetett: de jaj annak az embernek, a ki által elárultatik!
\par 23 És õk kezdék egymás között kérdezni, vajjon ki lehet az õ közöttük, a ki ezt meg fogja tenni?
\par 24 Támada pedig köztük versengés is, hogy ki tekinthetõ köztük nagyobbnak.
\par 25 Õ pedig monda nékik: A pogányokon uralkodnak az õ királyaik, és akiknek azokon hatalmuk van, jóltévõknek hivatnak.
\par 26 De ti nem úgy: hanem a ki legnagyobb köztetek, olyan legyen, mint a ki legkisebb; és a ki fõ, mint a ki szolgál.
\par 27 Mert melyik nagyobb, az-é, a ki asztalnál ül, vagy a ki szolgál? nemde a ki asztalnál ül? De én ti köztetek olyan vagyok, mint a ki szolgál.
\par 28 Ti vagytok pedig azok, kik megmaradtatok én velem az én kísérteteimben;
\par 29 Én azért adok néktek, miképen az én Atyám adott nékem, országot,
\par 30 Hogy egyetek és igyatok az én asztalomon az én országomban, és üljetek királyi székeken, ítélvén az Izráelnek tizenkét nemzetségét.
\par 31 Monda pedig az Úr: Simon! Simon! ímé a Sátán kikért titeket, hogy megrostáljon, mint a búzát;
\par 32 De én imádkoztam érted, hogy el ne fogyatkozzék a te hited: te azért idõvel megtérvén, a te atyádfiait erõsítsed.
\par 33 Õ pedig monda néki: Uram, te veled kész vagyok mind tömlöczre, mind halálra menni!
\par 34 És õ monda: Mondom néked Péter: Ma nem szól addig a kakas, míg te háromszor meg nem tagadod, hogy ismersz engem.
\par 35 És monda nékik: Mikor elküldtelek benneteket erszény, táska és saru nélkül, volt-é valamiben fogyatkozástok? Õk pedig mondának: Semmiben sem.
\par 36 Monda azért nékik: De most, a kinek erszénye van elõvegye, hasonlóképen a táskát; és a kinek nincs, adja el felsõ ruháját, és vegyen szablyát.
\par 37 Mert mondom néktek, hogy még ennek az írásnak be kell teljesülni rajtam, hogy És a gonoszok közé számláltatott. Mert a mik reám vonatkoznak is, elvégeztetnek.
\par 38 Azok pedig mondának: Uram, ímé van itt két szablya. Õ pedig monda: Elég.
\par 39 És kimenvén, méne az õ szokása szerint az Olajfák hegyére; követék pedig õt az õ tanítványai is.
\par 40 És mikor ott a helyén vala, monda nékik: Imádkozzatok, hogy kísértetbe ne essetek.
\par 41 És õ eltávozék tõlök mintegy kõhajításnyira; és térdre esvén, imádkozék,
\par 42 Mondván: Atyám, ha akarod, távoztasd el tõlem e pohárt; mindazáltal ne az én akaratom, hanem a tiéd legyen!
\par 43 És angyal jelenék meg néki mennybõl, erõsítvén õt.
\par 44 És haláltusában lévén, buzgóságosabban imádkozék; és az õ verítéke olyan vala, mint a nagy vércseppek, melyek a földre hullanak.
\par 45 És minekutána fölkelt az imádkozástól, az õ tanítványaihoz menvén, aludva találá õket a szomorúság miatt,
\par 46 És monda nékik: Mit alusztok? Keljetek fel és imádkozzatok, hogy kísértetbe ne essetek.
\par 47 És mikor még beszéle, ímé sokaság jöve, melynek az méne elõtte, a ki Júdásnak neveztetik, egy a tizenkettõ közül: és közelgete Jézushoz, hogy õt megcsókolja.
\par 48 Jézus pedig monda néki: Júdás, csokkal árulod el az embernek Fiát?
\par 49 Látván pedig azok, a kik õ körülötte valának, a mi következik, mondának néki: Uram, vágjuk-é õket fegyverrel?
\par 50 És közülök valaki megvágá a fõpap szolgáját, és levágá annak jobb fülét.
\par 51 Felelvén pedig Jézus, monda: Elég eddig. És illetvén annak fülét, meggyógyítá azt.
\par 52 Monda pedig Jézus azoknak, a kik õ hozzá mentek, a fõpapoknak, a templom tisztjeinek és a véneknek: Mint valami latorra, úgy jöttetek szablyákkal és fustélyokkal?
\par 53 Mikor minden nap veletek voltam a templomban, a ti kezeiteket nem vetétek én reám; de ez  a ti órátok, és a sötétségnek hatalma.
\par 54 Megfogván azért õt, elvezeték, és elvivék a fõpap házába. Péter pedig követi vala távol.
\par 55 És mikor tüzet gerjesztettek az udvar közepén, és õk együtt leültek, Péter is leüle õ velök.
\par 56 És meglátván õt egy szolgálóleány, a mint a világosságnál ült, szemeit reá vetvén, monda: Ez is õ vele vala!
\par 57 Õ pedig megtagadá õt, mondván: Asszony, nem ismerem õt!
\par 58 És egy kevéssel azután más látván õt, monda: Te is azok közül való vagy! Péter pedig monda: Ember, nem vagyok!
\par 59 És úgy egy óra mulva más valaki erõsíté, mondván: Bizony ez is vele vala: mert Galileából való is.
\par 60 Monda pedig Péter: Ember, nem tudom, mit mondasz! És azonnal, mikor õ még beszélt, megszólalt a kakas.
\par 61 És hátra fordulván az Úr, tekinte Péterre. És megemlékezék Péter az Úr szaváról, a mint néki mondta: Mielõtt a kakas szól, háromszor megtagadsz engem.
\par 62 És kimenvén Péter, keservesen síra.
\par 63 És azok a férfiak, a kik fogva tarták Jézust, csúfolják vala, vervén  õt.
\par 64 És szemeit betakarván, arczul csapdosák õt, és kérdezék õt, mondván: Prófétáld meg ki az, a ki téged vere?
\par 65 És sok egyéb dolgot mondának néki, szidalmazván õt.
\par 66 És a mint nappal lett,egybegyûle a nép véneinek tanácsa, fõpapok és írástudók: és vivék õt az õ gyülekezetükbe,
\par 67 Mondván: Ha te vagy a Krisztus, mondd meg nékünk. Monda pedig nékik: Ha mondom néktek, nem hiszitek:
\par 68 De ha kérdezlek is, nem feleltek nékem, sem el nem bocsátotok.
\par 69 Mostantól fogva ül az embernek Fia az Isten hatalmának jobbja felõl.
\par 70 Mondának pedig mindnyájan: Te vagy tehát az Isten Fia? Õ pedig monda nékik: Ti mondjátok, hogy én vagyok!
\par 71 Azok pedig mondának: Mi szükségünk van még bizonyságra? Hiszen mi magunk hallottuk az õ szájából.

\chapter{23}

\par 1 És fölkelvén az õ egész sokaságuk, vivék õt Pilátushoz.
\par 2 És kezdék õt vádolni, mondván: Úgy találtuk, hogy ez a népet félrevezeti, és tiltja a császár adójának fizetését,  mivelhogy õ magát ama király Krisztusnak mondja.
\par 3 Pilátus pedig megkérdé õt, mondván: Te vagy-é a zsidók királya? És õ felelvén néki, monda: Te mondod!
\par 4 Monda pedig Pilátus a fõpapoknak és a sokaságnak: Semmi bûnt nem találok ez emberben.
\par 5 De azok erõsködének, mondván: A népet felzendíti, tanítván az egész Júdeában, elkezdve Galileától mind idáig.
\par 6 Pilátus pedig Galileát hallván, megkérdé, vajjon galileai ember-é õ?
\par 7 És mikor megtudta, hogy õ a Héródes hatósága alá tartozik, Héródeshez küldé õt, mivelhogy az is Jeruzsálemben vala azokban a napokban.
\par 8 Héródes pedig Jézust látván igen megörüle: mert sok idõtõl fogva kívánta õt látni, mivelhogy sokat hallott õ felõle, és reménylé, hogy majd valami csodát lát, melyet õ tesz.
\par 9 Kérdezé pedig õt sok beszéddel; de õ semmit nem felele néki.
\par 10 Ott állanak vala pedig a fõpapok és az írástudók, teljes igyekezettel vádolván õt.
\par 11 Héródes pedig az õ katonáival egybe semminek állítván és kicsúfolván õt, minekutána felöltöztette fényes ruhába, visszaküldé Pilátushoz.
\par 12 És az napon lõnek barátok egymással Pilátus és Héródes; mert az elõtt ellenségeskedésben valának egymással.
\par 13 Pilátus pedig a fõpapokat, fõembereket és a népet egybegyûjtvén,
\par 14 Monda nékik: Ide hoztátok nékem ez embert, mint a ki a népet félrevezeti: és ímé én ti elõttetek kivallatván, semmi olyan bûnt nem találtam ez emberben, a mivel õt vádoljátok:
\par 15 De még Héródes sem; mert titeket õ hozzá igazítálak; és ímé semmi halálra való dolgot nem cselekedett õ.
\par 16 Megfenyítvén azért õt, elbocsátom.
\par 17 Kell vala pedig elbocsátania nékik ünnepenként egy foglyot.
\par 18 De felkiálta az egész sokaság, mondván: Vidd el ezt, és bocsásd el nékünk Barabbást!
\par 19 Ki a városban lett valami lázadásért és gyilkosságért vettetett a tömlöczbe.
\par 20 Pilátus azért ismét felszólala el akarván bocsátani Jézust;
\par 21 De azok ellene kiáltának, mondván: Feszítsd meg! Feszítsd meg õt!
\par 22 Õ pedig harmadszor is monda nékik: Mert mi gonoszt tett ez? Semmi halálra való bûnt nem találtam õ benne; megfenyítvén azért õt, elbocsátom!
\par 23 Azok pedig nagy fenszóval sürgeték, kérvén, hogy megfeszíttessék; és az õ szavok és a fõpapoké erõt vesz vala.
\par 24 És Pilátus megítélé, hogy meglegyen, a mit kérnek vala.
\par 25 És elbocsátá nékik azt, a ki lázadásért és gyilkosságért vettetett a tömlöczbe, a kit kértek vala; Jézust pedig kiszolgáltatá az õ akaratuknak.
\par 26 Mikor azért elvivék õt, egy Czirénebeli Simont megragadván, ki a mezõrõl jöve, arra tevék a keresztfát, hogy vigye Jézus után.
\par 27 Követé pedig õt a népnek és az asszonyoknak nagy sokasága, a kik gyászolák és siraták õt.
\par 28 Jézus pedig hozzájok fordulván, monda: Jeruzsálem leányai, ne sírjatok én rajtam, hanem ti magatokon sírjatok, és a ti magzataitokon.
\par 29 Mert ímé jõnek a napok, melyeken ezt mondják: Boldogok a meddõk, és a mely méhek nem szültek, és az emlõk, melyek nem szoptattak!
\par 30 Akkor kezdik mondani a hegyeknek: Essetek mi reánk; és a halmoknak: Borítsatok el minket!
\par 31 Mert ha a zöldelõ fán ezt mívelik, mi esik a száraz fán?
\par 32 Vivének pedig két másikat is, két gonosztevõt õ vele, hogy megölessenek.
\par 33 Mikor pedig elmenének a helyre, mely Koponya helyének mondatik, ott megfeszíték õt és a gonosztevõket, egyiket jobbkéz  felõl, a másikat balkéz felõl.
\par 34 Jézus pedig monda: Atyám! bocsásd meg nékik; mert nem tudják mit cselekesznek. Elosztván pedig az õ ruháit, vetének  reájok sorsot.
\par 35 És a nép megálla nézni. Csúfolák pedig õt a fõemberek is azokkal egybe, mondván: Egyebeket megtartott, tartsa meg magát, ha õ a Krisztus, az Istennek ama választottja.
\par 36 Csúfolák pedig õt a vitézek is, odajárulván és eczettel kínálván õt.
\par 37 És ezt mondván néki: Ha te vagy a zsidóknak ama Királya, szabadítsd meg magadat!
\par 38 Vala pedig egy felirat is fölébe írva görög, római és zsidó betûkkel: Ez a zsidóknak ama Királya.
\par 39 A felfüggesztett gonosztevõk közül pedig az egyik szidalmazá õt, mondván: Ha te vagy a Krisztus, szabadítsd meg magadat, minket is!
\par 40 Felelvén pedig a másik, megdorgálá õt, mondván: Az Istent sem féled-e te? Hiszen te ugyanazon ítélet alatt vagy!
\par 41 És mi ugyan méltán; mert a mi cselekedetünknek méltó büntetését vesszük: ez pedig semmi méltatlan dolgot nem  cselekedett.
\par 42 És monda Jézusnak: Uram, emlékezzél meg én rólam, mikor eljõsz a te országodban!
\par 43 És monda néki Jézus: Bizony mondom néked: Ma velem leszel a paradicsomban.
\par 44 Vala pedig mintegy hat óra, és sötétség lõn az egész tartományban mind kilencz órakorig.
\par 45 És meghomályosodék a nap, és a templom kárpitja középen ketté hasada.
\par 46 És kiáltván Jézus nagy szóval, monda: Atyám, a te kezeidbe teszem le az én lelkemet. És ezeket mondván, meghala.
\par 47 Látván pedig a százados, a mi történt, dicsõíté az Istent, mondván: Bizony ez ember igaz vala.
\par 48 És az egész sokaság, mely e dolognak látására ment oda, látván azokat, a mik történtek, mellét verve megtére.
\par 49 Az õ ismerõsei pedig mind, és az asszonyok, a kik Galileából követék õt, távol állának, nézvén ezeket.
\par 50 És ímé egy ember, kinek József vala neve, tanácsbeli, jó és igaz férfiú,
\par 51 Ki nem vala részes azoknak tanácsában és cselekedetében, Arimathiából, a zsidók városából való, ki maga is várja vala az Istennek országát;
\par 52 Ez oda menvén Pilátushoz, elkéré a Jézus testét.
\par 53 És levévén azt, begöngyölé azt gyolcsba, és helyhezteté azt egy sziklába vágott sírboltba,melyben még senki sem feküdt.
\par 54 És az a nap péntek vala, és szombat virrada rá.
\par 55 Az õt követõ asszonyok is pedig, kik vele Galileából jöttek, megnézék a sírt, és hogy miképen helyeztetett el az õ teste.
\par 56 Visszatérvén pedig, készítének fûszerszámokat és keneteket.
\par 57 És szombaton nyugovának a parancsolat szerint.

\chapter{24}

\par 1 A hétnek elsõ napján pedig kora reggel a sírhoz menének, vivén az elkészített fûszerszámokat, és némely más asszonyok is velök.
\par 2 És a követ a sírról elhengerítve találák.
\par 3 És mikor bementek, nem találák az Úr Jézus testét.
\par 4 És lõn, hogy mikor õk e felett megdöbbenének, ímé két férfiú álla melléjök fényes öltözetben:
\par 5 És mikor õk megrémülvén a földre hajták orczájokat, azok mondának nékik: Mit keresitek a holtak között az élõt?
\par 6 Nincs itt, hanem feltámadott: emlékezzetek rá, mint beszélt néktek, még mikor Galileában volt,
\par 7 Mondván: Szükség az ember Fiának átadatni a bûnös emberek kezébe, és megfeszíttetni, és harmadnapon feltámadni.
\par 8 Megemlékezének azért az õ szavairól.
\par 9 És visszatérvén a sírtól, elmondák mindezeket a tizenegynek, és mind a többieknek.
\par 10 Valának pedig Mária Magdaléna, és Johanna, és a Jakab anyja Mária, és egyéb asszonyok õ velök, a kik ezeket mondák az apostoloknak.
\par 11 De az õ szavuk csak üres beszédnek látszék azok elõtt; és nem hivének nékik.
\par 12 Péter azonban felkelvén elfuta a sírhoz, és behajolván látá, hogy csak a lepedõk vannak ott; és elméne, magában csodálkozván e dolgon.
\par 13 És ímé azok közül ketten mennek vala ugyanazon a napon egy faluba, mely Jeruzsálemtõl hatvan futamatnyira vala, melynek neve vala Emmaus.
\par 14 És beszélgetének magok közt mindazokról, a mik történtek.
\par 15 És lõn, hogy a mint beszélgetének és egymástól kérdezõsködének, maga Jézus hozzájok menvén, velök együtt megy vala az úton.
\par 16 De az õ szemeik visszatartóztatának, hogy õt meg ne ismerjék.
\par 17 Monda pedig nékik: Micsoda szavak ezek, a melyeket egymással váltotok jártotokban? és miért vagytok szomorú ábrázattal?
\par 18 Felelvén pedig az egyik, kinek neve Kleofás, monda néki: Csak te vagy-é jövevény Jeruzsálemben, és nem tudod minémû dolgok lettek abban e napokon?
\par 19 És monda nékik: Micsoda dolgok? Azok pedig mondának néki: A melyek esének a Názáretbeli Jézuson, ki próféta vala, cselekedetben és beszédben hatalmas Isten elõtt és az egész nép elõtt:
\par 20 És mimódon adák õt a fõpapok és a mi fõembereink halálos ítéletre, és megfeszíték õt.
\par 21 Pedig mi azt reméltük, hogy õ az, a ki meg fogja váltani az Izráelt. De mindezek mellett ma van harmadnapja, hogy ezek lettek.
\par 22 Hanem valami közülünk való asszonyok is megdöbbentettek minket, kik jó reggel a sírnál valának;
\par 23 És mikor nem találták az õ testét, haza jöttek, mondván, hogy angyalok jelenését is látták, kik azt mondják, hogy õ él.
\par 24 És azok közül némelyek, kik velünk valának, elmenének a sírhoz, és úgy találák, a mint az asszonyok is mondták; õt pedig nem látták.
\par 25 És õ monda nékik: Óh balgatagok és rest szívûek mindazoknak elhivésére, a miket a próféták szóltak!
\par 26 Avagy nem ezeket kellett-é szenvedni a Krisztusnak, és úgy menni be az õ dicsõségébe?
\par 27 És elkezdvén Mózestõl és minden prófétáktól fogva, magyarázza vala nékik minden írásokban, a mik õ felõle megirattak.
\par 28 Elközelítének pedig a faluhoz, a melybe mennek vala; és õ úgy tõn, mintha tovább menne.
\par 29 De kényszeríték õt, mondván: Maradj velünk, mert immár beestvéledik, és a nap lehanyatlott! Beméne azért, hogy velök maradjon.
\par 30 És lõn, mikor leült velök, a kenyeret vévén, megáldá, és megszegvén, nékik adá.
\par 31 És megnyilatkozának az õ szemeik, és megismerék õt; de õ eltünt elõlük.
\par 32 És mondának egymásnak: Avagy nem gerjedezett-é a mi szívünk mi bennünk, mikor nékünk szóla az úton, és mikor magyarázá nékünk az írásokat?
\par 33 És felkelvén azon órában, visszatérének Jeruzsálembe, és egybegyûlve találák a tizenegyet és azokat, a kik velök valának.
\par 34 Kik ezt mondják vala: Feltámadott az Úr bizonynyal, és megjelent Simonnak!
\par 35 És ezek is elbeszélék, mi történt az úton, mi képen ismerték meg õk a kenyér megszegésérõl.
\par 36 És mikor ezeket beszélék, megálla maga Jézus õ közöttök, és monda nékik: Békesség néktek!
\par 37 Megrémülvén pedig és félvén, azt hivék, hogy valami lelket látnak.
\par 38 És monda nékik: Miért háborodtatok meg, és miért támadnak szívetekben okoskodások?
\par 39 Lássátok meg az én kezeimet és lábaimat, hogy én magam vagyok: tapogassatok meg engem, és lássatok; mert a léleknek nincs húsa és csontja, a mint látjátok, hogy nékem van!
\par 40 És ezeket mondván, megmutatá nékik kezeit és lábait.
\par 41 Mikor pedig még nem hívék az öröm miatt, és csodálkozának, monda nékik: Van-é itt valami enni valótok?
\par 42 Õk pedig adának néki egy darab sült halat, és valami lépesmézet,
\par 43 Melyeket elvõn, és elõttök evék.
\par 44 És monda nékik: Ezek azok a beszédek, melyeket szóltam néktek, mikor még veletek valék, hogy szükség beteljesedni mindazoknak, a mik megirattak a Mózes törvényében, a prófétáknál és a zsoltárokban én felõlem.
\par 45 Akkor megnyilatkoztatá az õ elméjöket, hogy értsék az írásokat.
\par 46 És monda nékik: Így van megírva, és így kellett szenvedni a Krisztusnak, és feltámadni a halálból harmadnapon:
\par 47 És prédikáltatni az õ nevében a megtérésnek és a bûnök bocsánatának minden pogányok között, Jeruzsálemtõl elkezdve.
\par 48 Ti vagytok pedig ezeknek bizonyságai.
\par 49 És ímé én elküldöm ti reátok az én Atyámnak ígéretét; ti pedig maradjatok Jeruzsálem városában, mígnem  felruháztattok mennyei erõvel.
\par 50 Kivivé pedig õket Bethániáig; és felemelvén az õ kezeit, megáldá õket.
\par 51 És lõn, hogy míg áldá õket, tõlök elszakadván, felviteték a mennybe.
\par 52 Õk pedig imádván õt, visszatérnek nagy örömmel Jeruzsálembe;
\par 53 És mindenkor a templomban valának, dícsérvén és áldván az Istent. Ámen.


\end{document}