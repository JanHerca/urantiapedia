\begin{document}

\title{Kivonulás könyve}


\chapter{1}

\par 1 Ezek pedig Izráel fiainak nevei, a kik Jákóbbal Égyiptomba menének; kiki az õ házanépével méne:
\par 2 Rúben, Simeon, Lévi és Júda;
\par 3 Izsakhár, Zebulon és Benjámin;
\par 4 Dán és Nafthali, Gád és Áser.
\par 5 Mindazok a lelkek pedig, a kik Jákób ágyékából származtak vala, hetvenen valának. József pedig Égyiptomban vala.
\par 6 És meghala József és minden õ atyjafia és az az egész nemzedék.
\par 7 Izráel fiai pedig szaporák valának, szaporodának és sokasodának és igen-igen elhatalmazának, úgy hogy megtelék velök az ország.
\par 8 Azonközben új király támada Égyiptomban, a ki Józsefet nem ismerte vala.
\par 9 És monda az õ népének: Ímé az Izráel fiainak népe több, és hatalmasabb nálunknál.
\par 10 Nosza bánjunk okosan vele, hogy el ne sokasodjék és az ne legyen, hogy ha háború támad, õ is ellenségünkhöz adja magát és ellenünk harczoljon és az országból kimenjen.
\par 11 Rendelének azért föléjök robotmestereket, hogy nehéz munkákkal sanyargassák õket. És építe a Faraónak gabonatartó városokat, Pithomot és Ramszeszt.
\par 12 De minél inkább sanyargatják vala õt, annál inkább sokasodik és annál inkább terjeszkedik vala, s félnek vala az Izráel fiaitól.
\par 13 Pedig kegyetlenûl dolgoztaták az égyiptomiak az Izráel fiait.
\par 14 És kemény munkával keseríték életöket, sárcsinálással, téglavetéssel és mindenféle mezei munkával, minden munkájokkal, melyeket kegyetlenûl dolgoztatnak vala velök.
\par 15 És szóla Égyiptom királya a héber bábáknak, a kik közûl egyiknek Sifra, a másiknak Puá vala neve.
\par 16 És monda: Mikoron héber asszonyok körûl bábálkodtok, nézzetek a szûlõszékre: ha fiú az, azt öljétek meg, ha pedig leány az, hadd éljen.
\par 17 De a bábák félék az Istent és nem cselekedének úgy a mint Égyiptom királya parancsolta vala nékik, hanem életben hagyják vala a gyermekeket.
\par 18 Hívatá annakokáért Égyiptom királya a bábákat és mondá nékik: Miért míveltétek azt, hogy életben hagytátok a gyermekeket?
\par 19 A bábák pedig mondának a Faraónak: Mert a héber asszonyok nem olyanok, mint az Égyiptombeliek: mert azok élet-erõsek; minekelõtte a bába hozzájok eljutna, már szûlnek.
\par 20 Annakokáért jól tõn Isten a bábákkal, a nép pedig sokasodék és igen elhatalmazék.
\par 21 És lõn, hogy mivel a bábák félék az Istent: megépíté az õ házukat.
\par 22 Parancsola azért a Faraó minden õ népének, mondván: Minden fiút, a ki születik, vessetek a folyóvízbe, a leányt pedig hagyjátok mind életben.

\chapter{2}

\par 1 És elméne egy Lévi nemzetségébõl való férfiú és Lévi-leányt võn feleségûl.
\par 2 És fogada méhében az asszony és fiat szûle; és látá, hogy szép az és rejtegeté három hónapig.
\par 3 De mikor tovább nem rejtegetheté, szerze annak egy gyékény-ládácskát, és bekené azt gyantával és szurokkal s belétevé a gyermeket és letevé a folyóvíz szélén a sás közé.
\par 4 Az õ nénje pedig megáll vala távolról, hogy megtudja: mi történik vele?
\par 5 És aláméne a Faraó leánya, hogy megfürödjék a folyóvízben, szolgálóleányai pedig járkálnak vala a víz partján. És meglátá a ládácskát a sás között s elküldé az õ szolgálóleányát és kihozatá azt.
\par 6 És kinyitá és látá a gyermeket; és ímé egy síró fiú. És könyörûle rajta és monda: A héberek gyermekei közûl való ez.
\par 7 Az õ nénje pedig monda a Faraó leányának: Elmenjek-é s hívjak-é egy szoptatós asszonyt a héber asszonyok közûl, hogy szoptassa néked a gyermeket?
\par 8 És a Faraó leánya monda néki: Eredj el. Elméne azért a leányzó, és elhívá a gyermek anyját.
\par 9 És monda néki a Faraó leánya: Vidd el ezt a gyermeket és szoptasd fel nékem, és én megadom a te jutalmadat. És vevé az asszony a gyermeket és szoptatá azt.
\par 10 És felnevekedék a gyermek, és vivé õt a Faraó leányához, és fia gyanánt lõn annak, és nevezé nevét Mózesnek, és mondá: Mert a vízbõl húztam ki õt.
\par 11 Lõn pedig azokban a napokban, mikor Mózes felnevekedék, kiméne az õ atyjafiaihoz és látá az õ nehéz munkájokat s látá, hogy egy Égyiptombeli férfi üt vala egy héber férfit az õ atyjafiai közûl.
\par 12 Mikor ide-oda tekinte és látá hogy senki sincs, agyonüté az Égyiptombelit és elrejté azt a homokba.
\par 13 Másnap is kiméne és ímé két héber férfi veszekedik vala. És monda annak a ki bûnös vala: Miért vered a te atyádfiát?
\par 14 Az pedig monda: Kicsoda tett téged fõ emberré és bíróvá mi rajtunk? Talán engem is meg akarsz ölni, mint megöléd az égyiptomit? Mózes pedig megfélemlék és monda: Bizony kitudódott a dolog.
\par 15 A Faraó is meghallá azt a dolgot és Mózest halálra keresteti vala: de elfuta Mózes a Faraó elõl és lakozék Midián földén; leûle pedig egy kútnál.
\par 16 Midián papjának pedig hét leánya vala, és oda menének és vizet húzának és tele merék a válúkat, hogy megitassák atyjoknak juhait.
\par 17 A pásztorok is oda menének és elûzék õket. Mózes pedig felkele és segítséggel lõn nékik és megitatá juhaikat.
\par 18 Mikor atyjokhoz Réhuelhez menének, monda ez: Mi az oka, hogy ma ilyen hamar megjöttetek?
\par 19 Õk pedig mondának: Egy égyiptombeli férfi oltalmaza minket a pásztorok ellen, annakfelette vizet is húzott nékünk, és megitatta a juhokat.
\par 20 S monda leányainak: És hol van õ? miért hagytátok ott azt a férfit? hívjátok el, hogy egyék kenyeret.
\par 21 És tetszék Mózesnek, hogy ott maradjon e férfiúnál, és ez feleségül adá Mózesnek az õ leányát, Czipporát.
\par 22 És fiat szûle ez és nevezé nevét Gersomnak, mert mondá: Jövevény voltam az idegen földön.
\par 23 És lõn ama hosszú idõ alatt, meghala az Égyiptom királya, Izráel fiai pedig fohászkodnak vala a szolgaság miatt, és kiáltnak vala és feljuta a szolgaság miatt való kiáltásuk Istenhez.
\par 24 És meghallá Isten az õ fohászkodásukat és megemlékezék Isten az Ábrahámmal, Izsákkal és Jákóbbal kötött szövetségérõl.
\par 25 És megtekinté Isten az Izráel fiait és gondja vala rájok Istennek.

\chapter{3}

\par 1 Mózes pedig õrzi vala az õ ipának, Jethrónak a Midián papjának juhait és hajtá a juhokat a pusztán túl és juta az Isten hegyéhez, Hórebhez.
\par 2 És megjelenék néki az Úr angyala tûznek lángjában egy csipkebokor közepébõl, és látá, hogy ímé a csipkebokor ég vala; de a csipkebokor meg nem emésztetik vala.
\par 3 S monda Mózes: Oda megyek, hogy lássam e nagy csudát, miért nem ég el a csipkebokor.
\par 4 És látá az Úr, hogy oda méne megnézni, és szólítá õt Isten a csipkebokorból, mondván: Mózes, Mózes. Ez pedig monda: Ímhol vagyok.
\par 5 És monda: Ne jõjj ide közel, oldd le a te saruidat lábaidról; mert a hely, a melyen állasz, szent föld.
\par 6 És monda: Én vagyok a te atyádnak Istene, Ábrahámnak Istene, Izsáknak Istene és Jákóbnak Istene. Mózes pedig elrejté az õ orczáját, mert fél vala az Istenre tekinteni.
\par 7 Az Úr pedig monda: Látván láttam az én népemnek nyomorúságát, a mely Égyiptomban vagyon és meghallottam az õ sanyargatóik miatt való kiáltásukat; sõt ismerem szenvedéseit.
\par 8 Le is szállok, hogy megszabadítsam õt az Égyiptombeliek kezébõl és felvigyem õt arról a földrõl, jó és tágas földre, téjjel és mézzel folyó földre, a Kananeusok, Khitteusok, Emoreusok, Perizeusok, Khivveusok és Jebuzeusok lakóhelyére.
\par 9 Mivel hát ímé feljutott hozzám az Izráel fiainak kiáltása és láttam is a nyomorgatást, a melylyel nyomorgatják õket az Égyiptombeliek:
\par 10 Most azért eredj, elküldelek téged a Faraóhoz és hozd ki az én népemet, az Izráel fiait Égyiptomból.
\par 11 Mózes pedig monda az Istennek: Kicsoda vagyok én, hogy elmenjek a Faraóhoz és kihozzam Izráel fiait Égyiptomból?
\par 12 És felele: Én veled lészek! és ez lesz a jele, hogy én küldöttelek téged, hogy mikor kihozod a népet Égyiptomból, ezen a hegyen fogtok szolgálni az Istennek.
\par 13 Mózes pedig monda az Istennek: Ímé én elmegyek az Izráel fiaihoz és ezt mondom nékik: A ti atyáitok Istene küldött engem ti hozzátok; ha azt mondják nékem: Mi a neve? mit mondjak nékik?
\par 14 És monda Isten Mózesnek: VAGYOK A KI VAGYOK. És monda: Így szólj az Izráel fiaihoz: A VAGYOK küldött engem ti hozzátok.
\par 15 És ismét monda Isten Mózesnek: Így szólj az Izráel fiaihoz: Az Úr, a ti atyáitoknak Istene, Ábrahámnak Istene, Izsáknak Istene és Jákóbnak Istene küldött engem ti hozzátok. Ez az én nevem mind örökké és ez az én emlékezetem nemzetségrõl nemzetségre.
\par 16 Menj el és gyûjtsd egybe az Izráel véneit és mondd ezt nékik: Az Úr, a ti atyáitok Istene, Ábrahámnak, Izsáknak és Jákóbnak Istene megjelent nékem, mondván: Megemlékeztem rólatok és arról a mit elkövettek rajtatok Égyiptomban.
\par 17 És mondám: Kiviszlek titeket az égyiptomi nyomorúságból a Kananeusok, Khitteusok, Emoreusok, Perizeusok, Khivveusok és Jebuzeusok földére, téjjel és mézzel folyó földre.
\par 18 És ha hallgatnak szavadra, akkor elmégy te és az Izráel vénei Égyiptom királyához, s így szóltok néki: Az Úr, a héberek Istene megjelent nékünk; most azért hadd menjünk három napi útra a pusztába, hogy áldozzunk az Úrnak a mi Istenünknek.
\par 19 Én pedig tudom, hogy az égyiptomi király nem engedi meg néktek, hogy elmenjetek, még erõhatalomra sem.
\par 20 Kinyújtom azért az én kezemet és megverem Égyiptomot mindenféle csudáimmal, melyeket véghez viszek benne; így azután elbocsát titeket.
\par 21 És kedvessé tészem e népet az Égyiptombeliek elõtt, és lészen, hogy mikor kimentek, nem mentek üresen.
\par 22 Kérjen azért minden asszony az õ szomszédasszonyától és háza lakó asszonyától ezüst edényeket és arany edényeket és ruhákat; és rakjátok azokat fiaitokra és leányaitokra, s így foszszátok ki Égyiptomot.

\chapter{4}

\par 1 Felele Mózes és monda: De õk nem hisznek nékem s nem hallgatnak szavamra, sõt azt mondják: nem jelent meg néked az Úr.
\par 2 Az Úr pedig monda néki: Mi az a kezedben? S õ monda: Vesszõ.
\par 3 Vesd azt - úgymond - a földre. És veté azt a földre és lõn kígyóvá; és Mózes elfutamodék elõle.
\par 4 Monda pedig az Úr Mózesnek: Nyújtsd ki kezedet és fogd meg a farkát! És kinyújtá kezét és megragadá azt, és vesszõvé lõn az õ kezében.
\par 5 Hogy elhigyjék, hogy megjelent néked az Úr, az õ atyáik Istene, Ábrahám Istene, Izsák Istene és Jákób Istene.
\par 6 És ismét monda néki az Úr: Nosza dugd kebeledbe a kezed; és kebelébe dugá kezét és kihúzá: ímé az õ keze poklos vala, olyan mint a hó.
\par 7 És monda: Dugd vissza kebeledbe a kezed: és visszadugá kezét kebelébe és kivevé azt kebelébõl és ímé ismét olyanná lõn mint teste.
\par 8 És ha úgy lenne, hogy nem hisznek néked és nem hallgatnak az elsõ jel szavára, majd hisznek a második jel szavának.
\par 9 És ha úgy lenne, hogy ennek a két jelnek sem hisznek és nem hallgatnak szavadra: akkor meríts vizet a folyóvízbõl és öntsd a szárazra, és a víz, a mit a folyóvízbõl merítettél, vérré lesz a szárazon.
\par 10 És monda Mózes az Úrnak: Kérlek, Uram, nem vagyok én ékesenszóló sem tegnaptól, sem tegnap elõttõl fogva, sem azóta, hogy szólottál a te szolgáddal; mert én nehéz ajkú és nehéz nyelvû vagyok.
\par 11 Az Úr pedig monda néki: Ki adott szájat az embernek? Avagy ki tesz némává vagy siketté, vagy látóvá vagy vakká? Nemde én, az Úr?
\par 12 Most hát eredj és én lészek a te száddal, és megtanítlak téged arra, a mit beszélned kell.
\par 13 Õ pedig monda: Kérlek, Uram, csak küldd, a kit küldeni akarsz.
\par 14 És felgerjede az Úr haragja Mózes ellen és monda: Nemde atyádfia néked a Lévi nemzetségbõl való Áron? Tudom, hogy õ ékesenszóló és ímé õ ki is jõ elõdbe  s mihelyt meglát, örvendezni fog az õ szívében.
\par 15 Beszélj azért vele, és add szájába a beszédeket, és én lészek a te száddal és az õ szájával és megtanítalak titeket arra, a mit cselekedjetek.
\par 16 És õ beszél majd helyetted a néphez és õ lesz néked száj gyanánt, te pedig leszesz néki Isten gyanánt.
\par 17 Ezt a vesszõt pedig vedd kezedbe, hogy véghez vidd vele ama jeleket.
\par 18 És méne Mózes és visszatére az õ ipához Jethróhoz és monda néki: Hadd menjek és térjek vissza az én atyámfiaihoz Égyiptomba, hogy meglássam, ha élnek-e még? És monda Jethró Mózesnek: Eredj el békességgel.
\par 19 Az Úr pedig monda Mózesnek Midiánban: Eredj, térj vissza Égyiptomba; mert meghaltak mindazok a férfiak, kik téged halálra kerestek vala.
\par 20 És felvevé Mózes az õ feleségét és az õ fiait és felülteté õket a szamárra és visszatére Égyiptom földére. Az Isten vesszejét pedig kezébe vevé Mózes.
\par 21 És monda az Úr Mózesnek: Mikor elindulsz, hogy visszatérj Égyiptomba, meglásd, hogy mindazokat a csudákat véghez vidd a Faraó elõtt, melyeket kezedbe adtam; én pedig megkeményítem az õ szívét, és nem bocsátja el a népet.
\par 22 Ezt mondd azért a Faraónak: Így szólt az Úr: Elsõszülött fiam az Izráel.
\par 23 Ha azt mondom néked: Bocsásd el az én fiamat, hogy szolgáljon nékem és te vonakodol elbocsátani: ímé én megölöm a te elsõszülött fiadat.
\par 24 És lõn az úton, egy szálláson, eleibe álla az Úr és meg akarja vala õt ölni.
\par 25 De Czippora egy éles követ veve és lemetszé az õ fiának elõbõrét és lába elé veté mondván: Bizony, vérjegyesem vagy te nékem!
\par 26 És békét hagya néki. Akkor monda: Vérjegyes a körûlmetélkedésért.
\par 27 Áronnak pedig monda az Úr: Eredj Mózes eleibe a pusztába! És elméne és találkozék vele az Isten hegyénél és megcsókolá õt.
\par 28 Mózes pedig elbeszélé Áronnak az Úr mindama szavait, melyekkel õt elküldötte vala és mindazokat a jeleket, melyeket reá bízott vala.
\par 29 És méne Mózes és Áron és egybegyûjték Izráel fiainak minden véneit.
\par 30 És elmondá Áron mindazokat a beszédeket, melyeket mondott vala az Úr Mózesnek és megcselekedé a jeleket a nép szemei elõtt.
\par 31 És hitt a nép, és megértette, hogy meglátogatta az Úr izráel fiait és megtekintette nyomorúságukat. És meghajták magokat és leborulának.

\chapter{5}

\par 1 Annakutána pedig elmenének Mózes és Áron és mondának a Faraónak: Ezt mondá az Úr, Izráelnek Istene: Bocsásd el az én népemet, hogy ünnepet üljenek nékem a pusztában.
\par 2 A Faraó pedig mondá: Kicsoda az Úr, hogy szavára hallgassak, és elbocsássam izráelt? Nem ismerem az Urat és nem is bocsátom el Izráelt.
\par 3 Õk pedig mondának: A héberek Istene megjelent nékünk; hadd mehessünk hát háromnapi útra a pusztába, hogy áldozhassunk az Úrnak a mi Istenünknek, hogy meg ne verjen minket döghalállal vagy fegyverrel.
\par 4 Égyiptom királya pedig monda nékik: Mózes és Áron! miért vonjátok el a népet az õ munkáitól? menjetek dolgotokra.
\par 5 Ezt is mondja vala a Faraó: Ímé a föld népe most sok, és ti elhagyatjátok velök az õ munkáikat.
\par 6 Parancsolá azért a Faraó azon a napon a nép sarczoltatóinak és felvigyázóinak, mondván:
\par 7 Ne adjatok többé polyvát a népnek a téglavetéshez mint ennekelõtte; hadd menjenek el õk magok és szedjenek magoknak polyvát.
\par 8 De a tégla számát, mennyit ennekelõtte csináltak, vessétek ki rájok; azt le ne szállítsátok, mert restek õk és azért kiáltoznak, mondván: Menjünk el, áldozzunk a mi Istenünknek.
\par 9 Nehezíttessék meg a szolgálat ezeken az embereken, hogy azzal legyen dolguk és ne hajtsanak hazug szóra.
\par 10 Kimenének azért a nép sarczoltatói és felvigyázói és ezt mondák a népnek mondván: Ezt mondja a Faraó: Nem adok néktek polyvát.
\par 11 Menjetek magatok, szedjetek magatoknak polyvát onnan, a hol találtok; mert semmi sem szállíttatik le szolgálatotokból.
\par 12 És elszélede a nép egész Égyiptom földén, hogy tarlót szedjen polyva helyett.
\par 13 A sarczoltatók pedig szorítják vala, mondván: Végezzétek el munkátokat, napjában a napi munkát, mint akkor, a mikor polyva volt.
\par 14 És verettetének az Izráel fiai közül való felvigyázók, a kiket a Faraó sarczoltatói rendeltek vala föléjök, mondván: Miért nem végeztétek el a rátok vetett téglaszámot sem tegnap, sem ma úgy, mint ennekelõtte?
\par 15 És elmenének az Izráel fiai közül való felvigyázók és kiáltának a Faraóhoz, mondván: Miért cselekszel így a te szolgáiddal?
\par 16 Polyvát nem adnak a te szolgáidnak és azt mondják nékünk: Csináljatok téglát! És ímé a te szolgáid verettetnek, a te néped pedig vétkezik.
\par 17 Az pedig monda: Restek vagytok, restek, azért mondjátok: Menjünk el, áldozzunk az Úrnak!
\par 18 Most pedig menjetek, dolgozzatok, polyvát ugyan nem adnak néktek, de a rátok vetett tégla-számot be kell adnotok.
\par 19 Akkor látják vala Izráel fiainak felvigyázói, hogy bajban vannak, mivel azt kell mondaniok: a tégla-számot le ne szállítsátok; napjában a napi munka meglegyen.
\par 20 És mikor kijövének a Faraótól, szembe találkozának Mózessel és Áronnal.
\par 21 S mondának nékik: Lásson meg titeket az Úr és ítéljen meg, kik rossz hírbe kevertetek minket a Faraó elõtt és az õ szolgái elõtt, fegyvert adván azok kezébe, hogy megöljenek minket.
\par 22 S visszaméne Mózes az Úrhoz és monda: Uram, miért engedsz rosszul bánni ezzel a néppel! Miért küldél engem ide?
\par 23 Mert attól fogva, hogy bemenék a Faraóhoz, hogy a te nevedben szóljak, rosszabbul bánik e néppel; megszabadítani pedig nem szabadítád meg a te népedet.

\chapter{6}

\par 1 Az Úr pedig monda Mózesnek: Majd meglátod mit cselekszem a Faraóval; mert hatalmas kéz miatt kell õket elbocsátani és hatalmas kéz miatt ûzi el õket az õ földérõl.
\par 2 Az Isten pedig szóla Mózeshez és monda néki: Én vagyok az Úr.
\par 3 Ábrahámnak, Izsáknak és Jákóbnak úgy jelentem meg mint mindenható Isten, de az én Jehova  nevemen nem voltam elõttük ismeretes.
\par 4 Szövetséget is kötöttem velek, hogy nékik adom a Kanaán földét, az õ tartózkodásuk földét, a melyen tartózkodjanak.
\par 5 Fohászkodását is meghallottam az Izráel fiainak a miatt, hogy az Égyiptombeliek szolgálatra szorítják õket, megemlékeztem az én szövetségemrõl.
\par 6 Annakokáért mondd meg az Izráel fiainak: Én vagyok az Úr és kiviszlek titeket Égyiptom nehéz munkái alól és megszabadítlak titeket az õ szolgálatjoktól és megmentlek titeket kinyújtott karral és nagy büntetõ ítéletek által.
\par 7 És népemmé fogadlak titeket s Istentekké lészek néktek és megtudjátok, hogy én vagyok a ti Uratok Istentek, a ki kihoztalak titeket Égyiptom nehéz munkái alól.
\par 8 És béviszlek titeket a földre, a mely felõl esküre emeltem fel kezemet, hogy Ábrahámnak, Izsáknak és Jákóbnak adom azt, és néktek adom azt örökségül, én az Úr.
\par 9 Eképen szóla Mózes az Izráel fiaihoz; de nem hallgatának Mózesre, a kislelkûség és a kemény szolgálat miatt.
\par 10 Az Úr pedig szóla Mózeshez mondván:
\par 11 Eredj be, szólj a Faraónak, Égyiptom királyának, hogy bocsássa el Izráel fiait az õ földérõl.
\par 12 Mózes pedig szóla az Úr elõtt, mondván: Ímé Izráel fiai nem hallgatnak reám, hogy hallgatna hát reám a Faraó, holott én nehéz ajkú vagyok?
\par 13 És szóla az Úr Mózeshez és Áronhoz és rendelé õket Izráel fiaihoz és a Faraóhoz, Égyiptom királyához, hogy hozzák ki az Izráel fiait Égyiptom földérõl.
\par 14 Ezek atyáik házanépének fejei: Rúbennek, Izráel elsõszülöttének fiai: Khanókh, Pallu, Kheczrón és Karmi. Ezek a Rúben nemzetségei.
\par 15 Simeon fiai pedig: Jemúél, Jámin, Óhad, Jákin, Czohar és Saul, a kanaánbeli asszony fiai. Ezek a Simeon nemzetségei.
\par 16 Ezek pedig a Lévi fiainak nevei az õ születésök szerint: Gersón, Kehát és Merári. Lévi életének esztendei pedig: száz harminczhét esztendõ.
\par 17 Gersón fiai: Libni, Simhi, az õ nemzetségeik szerint.
\par 18 Kehát fiai pedig: Amrám, Jiczhár, Khebrón és Huzziél. Kehát életének esztendei pedig: száz harminczhárom esztendõ.
\par 19 Merári fiai pedig Makhli és Músi. Ezek a Lévi nemzetségei az õ születésök szerint.
\par 20 Amrám pedig feleségûl vevé magának Jókébedet, atyjának húgát s ez szûlé néki Áront és Mózest. Amrám életének esztendei pedig száz harminczhét esztendõ.
\par 21 Jiczhár fiai pedig: Kórákh, Nefeg és Zikri.
\par 22 Huzziél fiai pedig: Misáél, Elczáfán és Szithri.
\par 23 Áron pedig feleségûl vevé magának Elisebát, Aminádáb leányát, Nakhsón húgát s ez szûlé néki Nádábot, Abíhut, Eleázárt és Ithámárt.
\par 24 Kórákh fiai pedig: Asszir, Elkánáh és Abiászáf. Ezek a Kórákh nemzetségei.
\par 25 Eleázár pedig az Áron fia, a Putiél leányai közûl võn magának feleséget s ez szûlé néki Fineást. Ezek a Léviták atyáinak fejei az õ nemzetségeik szerint.
\par 26 Ez Áron és Mózes, a kiknek mondá az Úr: Hozzátok ki az Izráel fiait Égyiptom földérõl, az õ seregeik szerint.
\par 27 Õk azok, a kik szóltak a Faraónak Égyiptom királyának, hogy kihozhassák az Izráel fiait Égyiptomból. Ez Mózes és Áron.
\par 28 És lõn a mikor szóla az Úr Mózesnek Égyiptom földén.
\par 29 Így szóla az Úr Mózesnek, mondván: Én vagyok az Úr: mondd el a Faraónak Égyiptom királyának mindazt, a mit én szólok néked.
\par 30 Mózes pedig monda az Úr elõtt: Ímé én nehéz ajakú vagyok, hogy hallgatna reám a Faraó?

\chapter{7}

\par 1 Az Úr pedig monda Mózesnek: Lásd, Istenévé teszlek téged a Faraónak, Áron pedig, a te atyádfia, szószólód lészen.
\par 2 Te mondj el mindent, a mit néked parancsolok; Áron pedig, a te atyádfia mondja meg a Faraónak, hogy bocsással el Izráel fiait az õ földérõl.
\par 3 Én pedig megkeményítem a Faraó szívét és megsokasítom az én jeleimet és csudáimat Égyiptom földén.
\par 4 És a Faraó nem hallgat reátok; akkor én kezemet Égyiptomra vetem, és kihozom az én seregeimet, az én népemet, az Izráel fiait Égyiptom földérõl nagy büntetõ ítéletek által.
\par 5 S megtudják az Égyiptombeliek, hogy én vagyok az Úr, a mikor kinyujtándom kezemet Égyiptomra és kihozándom az Izráel fiait õ közülök.
\par 6 És cselekedék Mózes és Áron, a mint parancsolta vala nékik az Úr; úgy cselekedének.
\par 7 Mózes pedig nyolczvan esztendõs és Áron nyolczvanhárom esztendõs vala, a mikor a Faraóval beszéltek.
\par 8 És szóla az Úr Mózesnek és Áronnak mondván:
\par 9 Ha szól hozzátok a Faraó mondván: tegyetek csudát; akkor mondd Áronnak: Vedd a te vesszõdet és vesd a Faraó elé; kígyóvá lesz.
\par 10 Beméne azért Mózes és Áron a Faraóhoz, és úgy cselekedének, a mint az Úr parancsolta vala; veté Áron az õ vesszejét a Faraó elé és az õ szolgái elé, és kígyóvá lõn.
\par 11 És elõhívá a Faraó is a bölcseket és varázslókat, és azok is, Égyiptom írástudói, úgy cselekedének az õ titkos mesterségökkel.
\par 12 Elveté ugyanis mindenik az õ vesszejét és kígyókká lõnek; de az Áron vesszeje elnyelé azok vesszejét.
\par 13 És megkeményedék a Faraó szíve és nem hallgata reájok, a mint megmondotta vala az Úr.
\par 14 Az Úr pedig monda Mózesnek: Kemény a Faraó szíve, nem akarja a népet elbocsátani.
\par 15 Eredj a Faraóhoz reggel; ímé õ kimegy a vízhez, és állj eleibe a folyóvíz partján, és a vesszõt, a mely kígyóvá változott vala, vedd kezedbe.
\par 16 És mondd néki: Az Úr, a héberek Istene küldött engem hozzád, mondván: Bocsásd el az én népemet, hogy szolgáljanak nékem a pusztában; de ímé mindez ideig meg nem hallgattál.
\par 17 Így szólt az Úr: Errõl tudod meg, hogy én vagyok az Úr: Ímé én megsujtom a vesszõvel, a mely kezemben van, a vizet, a mely a folyóban van, és vérré változik.
\par 18 És a hal, a mely a folyóvízben van, meghal, a folyóvíz pedig megbüdösödik és irtózni fognak az Égyiptombeliek vizet inni a folyóból.
\par 19 Monda azért az Úr Mózesnek: Mondd Áronnak: Vedd a te vesszõdet és nyujtsd ki kezedet Égyiptom vizeire; azoknak folyóvizeire, csatornáira, tavaira és minden vízfogóira, hogy vérré legyenek és vér legyen Égyiptom egész földén, mind a fa-, mind a kõedényekben.
\par 20 Mózes és Áron pedig úgy cselekedének, a mint az Úr parancsolta vala. És felemelé a vesszõt és megsujtá a vizet, a mely a folyóban vala a Faraó elõtt és az õ szolgái elõtt, és mind vérré változék a víz, a mely a folyóban vala.
\par 21 A hal pedig, a mely a folyóvízben vala, meghala, és megbüdösödék a folyóvíz, és nem ihatának az Égyiptombeliek a folyónak vizébõl; és vér vala az egész Égyiptom földén.
\par 22 De úgy cselekedének Égyiptom írástudói is az õ varázslásukkal, és kemény maradt a Faraó szíve és nem hallgata reájok; a mint az Úr megmondotta vala.
\par 23 És elfordula a Faraó és haza méne és ezen sem indula meg az õ szíve.
\par 24 Az Égyiptombeliek pedig mindnyájan ássák vala a folyóvíz mellékét vízért, hogy ihassanak; mert nem ihatják vala a folyó vizét.
\par 25 És hét nap telék el, a mióta az Úr megsujtotta vala a folyóvizet.

\chapter{8}

\par 1 És monda az Úr Mózesnek: Menj be a Faraóhoz és mondd néki: Azt mondja az Úr: Bocsásd el az én népemet, hogy szolgáljanak nékem.
\par 2 Ha pedig te el nem akarod bocsátani, ímé én egész határodat békákkal verem meg.
\par 3 És a folyóvíz békáktól pozsog és felmennek és bemennek a te házadba és ágyasházadba és ágyadra és a te szolgáid házába és néped közé és a te kemenczéidbe és sütõteknõidbe.
\par 4 És reád és népedre s minden te szolgáidra felmennek a békák.
\par 5 És monda az Úr Mózesnek: Mondd Áronnak: Nyujtsd ki kezed a te vesszõddel a folyóvizekre, csatornákra és tavakra, és hozd fel a békákat Égyiptom földére.
\par 6 És kinyujtá kezét Áron Égyiptom vizeire, és békák jövének fel és ellepék Égyiptom földét.
\par 7 De az írástudók is úgy cselekedének az õ titkos mesterségökkel és felhozák a békákat Égyiptom földére.
\par 8 És hívatá a Faraó Mózest és Áront és monda: Könyörögjetek az Úrnak, hogy távolítsa el rólam és az én népemrõl a békákat, és én elbocsátom a népet, hogy áldozzék az Úrnak.
\par 9 Mózes pedig monda a Faraónak: Parancsolj velem: mikorra könyörögjek éretted és a te szolgáidért és a te népedért, hogy elpusztuljanak a békák tõled és házaidtól; és csak a folyóvízben maradjanak meg.
\par 10 Felele a Faraó: Holnapra. És monda Mózes: A mint kívánod, hogy megtudd, hogy nincs hasonló a mi Urunkhoz Istenünkhöz.
\par 11 És eltávoznak a békák tõled, meg a te házaidtól, szolgáidtól és a te népedtõl; csak a folyóvízben maradnak meg.
\par 12 És kiméne Mózes és Áron a Faraótól és kiálta Mózes az Úrhoz a békák felõl, a melyeket a Faraóra bocsátott vala.
\par 13 És az Úr Mózes beszéde szerint cselekedék és kiveszének a békák a házakból, udvarokból és mezõkrõl.
\par 14 És rakásokba gyûjték azokat össze és a föld megbüszhödék.
\par 15 S a mint látá a Faraó, hogy baja könnyebbûl, megkeményíté az õ szívét, és nem hallgata reájok, a mint megmondotta vala az Úr.
\par 16 És szóla az Úr Mózesnek: Mondd Áronnak: Nyujtsd ki a te vesszõdet és sujtsd meg a föld porát, hogy tetvekké legyen egész Égyiptom földén.
\par 17 És aképen cselekedének. Áron kinyujtá kezét az õ vesszejével és megsujtá a föld porát, és tetvek lõnek emberen és barmon; a föld minden pora tetvekké lõn egész Égyiptom földén.
\par 18 És úgy cselekedének az írástudók is az õ varázslásukkal, hogy tetveket hozzanak elõ, de nem teheték; és valának a tetvek emberen és barmon.
\par 19 És mondák az írástudók a Faraónak: Az Isten ujja ez. De kemény maradt a Faraó szíve, és nem hallgata reájok; a mint mondotta vala az Úr.
\par 20 Az Úr pedig monda Mózesnek: Kelj fel reggel és állj a Faraó eleibe; ímé kimegy a vizek felé, és mondd néki: Ezt mondja az Úr: Bocsásd el az én népemet, hogy szolgáljanak nékem.
\par 21 Mert ha el nem bocsátod az én népemet, ímé én bocsátok te reád, a te szolgáidra és a te népedre és a te házaidra ártalmas bogarakat, és megtelnek az Égyiptombeliek házai ártalmas bogarakkal és a föld is, a melyen õk vannak.
\par 22 De különválasztom azon a napon Gósen földét, a melyen az én népem lakik, hogy ne legyenek ott ártalmas bogarak, azért, hogy megtudd, hogy én vagyok az Úr ezen a földön.
\par 23 És különbséget tészek az én népem között és a te néped között. Holnap lészen e jelenség.
\par 24 És aképen cselekedék az Úr; jövének ugyanis ártalmas bogarak a Faraó házára és az õ szolgái házára, és egész Égyiptom földén pusztává lõn a föld az ártalmas bogarak miatt.
\par 25 És hívatá a Faraó Mózest és Áront és monda: Menjetek, áldozzatok a ti Isteneteknek ezen a földön.
\par 26 Mózes pedig monda: Nincs rendén, hogy úgy cselekedjünk, hogy mi azt áldozzuk az Úrnak a mi Istenünknek, a mi utálatos az Égyiptombeliek elõtt: ímé, ha azt áldozzuk az õ szemeik elõtt, a mi az Égyiptombelieknek utálatos, nem köveznek-é meg minket?
\par 27 Háromnapi járó földre megyünk a pusztába és úgy áldozunk a mi Urunknak Istenünknek, a mint megmondja nékünk.
\par 28 És monda a Faraó: Én elbocsátalak titeket, hogy áldozzatok a ti Uratoknak Istenteknek a pusztában, csak nagyon messze ne távozzatok; imádkozzatok érettem.
\par 29 Mózes pedig monda: Ímé én kimegyek te tõled és imádkozom az Úrhoz és eltávoznak az ártalmas bogarak a Faraótól és az õ szolgáitól és az õ népétõl holnap; csak megint el ne ámítson a Faraó, hogy el ne bocsássa a népet áldozni az Úrnak.
\par 30 És kiméne Mózes a Faraótól és imádkozék az Úrhoz
\par 31 és az Úr Mózes beszéde szerint cselekedék; s eltávozának az ártalmas bogarak a Faraótól, szolgáitól és népétõl; egy sem marada.
\par 32 De a Faraó ezúttal is megkeményíté az õ szívét és nem bocsátá el a népet.

\chapter{9}

\par 1 És monda az Úr Mózesnek: Menj a Faraóhoz és beszélj vele: Ezt mondja az Úr, a héberek Istene: Bocsásd el az én népemet, hogy szolgáljanak nékem:
\par 2 Mert ha nem akarod elbocsátani és tovább is tartóztatod õket:
\par 3 Ímé az Úr keze lészen a te mezei barmaidon, lovakon, szamarakon, tevéken, ökrökön és juhokon; igen nagy döghalál.
\par 4 De különbséget tesz az Úr az Izráel barmai között és Égyiptom barmai között, és mindabból, a mi Izráel fiaié, egy sem vész el.
\par 5 Idõt is hagya az Úr, mondván: Holnap cselekszi az Úr ezt a dolgot a földön.
\par 6 Meg is cselekedé az Úr ezt a dolgot másodnapon, és elhulla Égyiptomnak minden barma, de az Izráel fiainak barma közül egy sem hullott el.
\par 7 El is külde a Faraó, és ímé egy sem hullt vala el az Izráeliták barma közül: de a Faraó szíve kemény maradt, és nem bocsátá el a népet.
\par 8 Az Úr pedig monda Mózesnek és Áronnak: Vegyétek tele markaitokat kemenczehamuval, és szórja azt Mózes az ég felé a Faraó szeme láttára.
\par 9 Hogy porrá legyen Égyiptomnak egész földén, s emberen és barmon hólyagosan fakadó fekéllyé legyen Égyiptomnak egész földén.
\par 10 Vevének azért kemenczehamut és a Faraó elé állának, és Mózes az ég felé szórá azt; és lõn az emberen és barmon hólyagosan fakadó fekély.
\par 11 És az írástudók nem állhatnak vala Mózes elõtt a fekély miatt; mert fekély vala az írástudókon s mind az Égyiptombelieken.
\par 12 De az Úr megkeményíté a Faraó szívét, és nem hallgata reájok, a mint megmondotta vala az Úr Mózesnek.
\par 13 És monda az Úr Mózesnek: Kelj fel reggel és állj a Faraó eleibe és mondd néki: Ezt mondja az Úr, a héberek Istene: Bocsásd el az én népemet, hogy szolgáljanak nékem.
\par 14 Mert ezúttal minden csapásomat reá bocsátom a te szívedre, a te szolgáidra és a te népedre azért, hogy megtudd, hogy nincs én hozzám hasonló az egész földön.
\par 15 Mert ha most kinyujtanám kezemet és megvernélek téged és a te népedet döghalállal, akkor kivágattatnál a földrõl.
\par 16 Ámde azért tartottalak fenn tégedet, hogy megmutassam néked az én hatalmamat, és hogy hirdessék az én nevemet az egész földön.
\par 17 Ha tovább is feltartóztatod az én népemet és nem bocsátod el õket:
\par 18 Ímé holnap ilyenkor igen nagy jégesõt bocsátok, a melyhez hasonló nem volt Égyiptomban az napságtól fogva hogy fundáltatott, mind ez ideig.
\par 19 Most annakokáért küldj el, hajtasd be barmaidat és mindenedet, valamid a mezõn van; minden ember és barom, a mely a mezõn találtatik és házba nem hajtatik, - jégesõ szakad arra, és meghal.
\par 20 A ki a Faraó szolgái közûl az Úr beszédétõl megfélemedék, szolgáit és barmait házakba futtatá.
\par 21 A ki pedig nem törõdék az Úr beszédével, szolgáit és barmát a mezõn hagyá.
\par 22 Az Úr pedig monda Mózesnek: Nyújtsd ki kezedet az égre, hogy legyen jégesõ Égyiptom egész földén az emberre, baromra és a mezõ minden fûvére, Égyiptom földén.
\par 23 Kinyujtá azért Mózes az õ vesszejét az égre, az Úr pedig mennydörgést támaszta és jégesõt, és tûz szálla le a földre, és jégesõt bocsáta az Úr Égyiptom földére.
\par 24 És lõn jégesõ, és a tûz egymást éré az igen nagy jégesõ közt, a melyhez hasonló nem volt az egész Égyiptom földén, mióta nép lakja.
\par 25 És elveré a jégesõ egész Égyiptom földén mindazt, a mi a mezõn vala, embertõl baromig; a mezõ minden fûvét is elveré a jégesõ és a mezõ minden fáját is egybe rontá.
\par 26 Csak a Gósen földén, hol Izráel fiai valának, nem volt jégesõ.
\par 27 A Faraó pedig elkülde és hívatá Mózest és Áront, és monda nékik: Vétkeztem ezúttal; az Úr az igaz; én pedig és az én népem gonoszok vagyunk.
\par 28 Imádkozzatok az Úrhoz, hogy legyen elég a mennydörgés és jégesõ, és akkor elbocsátlak titeket és nem maradtok tovább.
\par 29 Mózes pedig monda néki: Mihelyt kimegyek a városból, felemelem kezeimet az Úrhoz; megszûnnek a mennydörgések és nem lesz többé jégesõ, hogy megtudd, hogy az Úré a föld.
\par 30 De tudom, hogy te és a te szolgáid még nem féltek az Úr Istentõl.
\par 31 A len pedig és az árpa elvereték, mert az árpa kalászos, a len pedig bimbós vala.
\par 32 De a búza és a tönköly nem vereték el, mert azok késeiek.
\par 33 És kiméne Mózes a Faraótól a városból és felemelé kezeit az Úrhoz, és megszûnének a mennydörgések s a jégesõ és esõ sem ömlik vala a földre.
\par 34 Amint látá a Faraó, hogy megszûnék az esõ, meg a jégesõ és a mennydörgés, ismét vétkezék és megkeményíté szívét õ és az õ szolgái.
\par 35 És kemény maradt a Faraó szíve, és nem bocsátá el az Izráel fiait, a mint megmondotta vala az Úr Mózes által.

\chapter{10}

\par 1 És monda az Úr Mózesnek: Menj be a Faraóhoz, mert én keményítettem meg az õ szívét és az õ szolgáinak szívét, azért hogy ezeket az én jeleimet megtegyem közöttök.
\par 2 És azért, hogy elbeszéljed azt a te fiadnak és fiad fiának hallatára, a mit Égyiptomban cselekedtem, és jeleimet, a melyeket rajtok tettem, hogy megtudjátok, hogy én vagyok az Úr.
\par 3 És beméne Mózes és Áron a Faraóhoz és mondának néki: Ezt mondja az Úr, a héberek Istene: Meddig nem akarod még magadat megalázni én elõttem? Bocsásd el az én népemet, hogy szolgáljanak nékem.
\par 4 Mert ha te nem akarod az én népemet elbocsátani, ímé én holnap sáskát hozok a te határodra.
\par 5 És elborítja a földnek színét, úgy hogy nem lesz látható a föld, és megemészti a megmenekedett maradékot, a mi megmaradt néktek a jégesõ után, és megemészt minden fát, mely néktek sarjadzik a mezõn.
\par 6 És betöltik a te házaidat, és minden szolgáidnak házát és minden Égyiptombelinek házát, a mit nem láttak a te atyáid, sem a te atyáid atyjai, a mióta e földön vannak mind e mai napig. És megfordula s kiméne a Faraó elõl.
\par 7 A Faraó szolgái pedig mondának õnéki: Meddig lesz még ez mi nékünk romlásunkra? Bocsásd el azokat az embereket, hogy szolgáljanak az Úrnak az õ Istenöknek. Még sem veszed-é eszedbe, hogy elvész Égyiptom?
\par 8 És visszahozák Mózest és Áront a Faraóhoz s monda ez nékik: Menjetek el, szolgáljatok az Úrnak a ti Istenteknek. Kik s kik azok, a kik elmennek?
\par 9 Mózes pedig monda: A mi gyermekeinkkel és véneinkkel megyünk, a mi fiainkkal és leányainkkal, juhainkkal és barmainkkal megyünk, mert az Úrnak innepet kell szentelnünk.
\par 10 Monda azért nékik: Úgy legyen veletek az Úr, a mint elbocsátlak titeket és gyermekeiteket! Vigyázzatok, mert gonoszra igyekeztek.
\par 11 Nem úgy! menjetek el ti férfiak és szolgáljatok az Úrnak, mert ti is ezt kívántátok. És elûzék õket a Faraó színe elõl.
\par 12 És monda az Úr Mózesnek: Nyújtsd ki a te kezedet Égyiptom földére a sáskáért, hogy jõjjön fel Égyiptom földére és emészsze meg a földnek minden fûvét; mindazt a mit a jégesõ meghagyott.
\par 13 Kinyujtá azért Mózes az õ vesszejét Égyiptom földére, és az Úr egész nap és egész éjjel keleti szelet támaszta a földre. Mire reggel lõn, a keleti szél felhozá a sáskát.
\par 14 És feljöve a sáska egész Égyiptom földére s nagy sokasággal szálla le Égyiptom egész határára. Annak elõtte sem volt olyan sáska s ezután sem lesz olyan.
\par 15 És elborítá az egész föld színét, és a föld elsötétedék, és megemészté a földnek minden fûvét és a fának minden gyümölcsét, a mit a jégesõ meghagyott vala, és semmi zöld sem marada a fán, sem a mezõnek fûvén egész Égyiptom földén.
\par 16 Akkor a Faraó siete hívatni Mózest és Áront és monda: Vétkeztem az Úr ellen, a ti Istenetek ellen és ti ellenetek.
\par 17 Most annakokáért bocsásd meg csak ez egyszer az én vétkemet és imádkozzatok az Úrhoz a ti Istentekhez, hogy csak ezt a halált fordítsa el én tõlem.
\par 18 És kiméne a Faraó elõl és imádkozék az Úrhoz.
\par 19 És fordíta az Úr igen erõs nyugoti szelet, és felkapá a sáskát és veté azokat a veres tengerbe; egy sáska sem marada egész Égyiptom határán.
\par 20 De az Úr megkeményíté a Faraó szívét, és nem bocsátá el az Izráel fiait.
\par 21 És monda az Úr Mózesnek: Nyujtsd ki a te kezedet az ég felé, hogy legyen setétség Égyiptom földén és pedig tapintható setétség.
\par 22 És kinyujtá Mózes az õ kezét az ég felé, és lõn sûrû setétség egész Égyiptom földén három napig.
\par 23 Nem látták egymást, és senki sem kelt fel az õ helyébõl három napig; de Izráel minden fiának világosság vala az õ lakhelyében.
\par 24 Akkor hívatá a Faraó Mózest és monda: Menjetek el, szolgáljatok az Úrnak, csak juhaitok és barmaitok maradjanak; gyermekeitek is elmehetnek véletek.
\par 25 Mózes pedig monda: Sõt inkább néked kell kezünkbe adnod véres áldozatra és égõ-áldozatra valókat, hogy megáldozzuk a mi Urunknak, Istenünknek.
\par 26 És velünk jõnek a mi barmaink is, egy körömnyi sem marad el, mert azokból veszünk, hogy szolgáljunk a mi Urunknak Istenünknek; magunk sem tudjuk, mivel szolgálunk az Úrnak, míg oda nem jutunk.
\par 27 De az Úr megkeményíté a Faraó szívét, és nem akará õket elbocsátani.
\par 28 És monda néki a Faraó: menj el elõlem; vigyázz magadra, hogy többé az én orczámat ne lásd; mert a mely napon az én orczámat látod, meghalsz.
\par 29 Mózes pedig monda: Helyesen szólál. Nem látom többé a te orczádat.

\chapter{11}

\par 1 Az Úr pedig monda Mózesnek: Még egy csapást hozok a Faraóra és Égyiptomra; azután elbocsát titeket innen; a mikor mindenestõl elbocsát, ûzve hajt el titeket innen.
\par 2 Szólj azért a népnek füle hallatára, hogy kérjenek a férfi az õ férfitársától, az asszony pedig az õ asszonytársától ezüst edényeket és arany edényeket.
\par 3 Az Úr pedig kedvessé tevé a népet az Égyiptombeliek elõtt. A férfiú Mózes is igen nagy vala Égyiptom földén a Faraó szolgái elõtt és a nép elõtt.
\par 4 És monda Mózes: Ezt mondja az Úr: Éjfél körûl kimegyek Égyiptomba.
\par 5 És meghal Égyiptom földén minden elsõszülött, a Faraónak elsõszülöttétõl fogva, a ki az õ királyi székében ûl, a szolgálónak elsõ szülöttéig, a ki malmot hajt; a baromnak is minden elsõ fajzása.
\par 6 És nagy jajgatás lesz egész Égyiptom földén, a melyhez hasonló nem volt és hasonló nem lesz többé.
\par 7 De Izráel fiai közûl az eb sem ölti ki nyelvét senkire, az embertõl kezdve a baromig; hogy megtudjátok, hogy különbséget tett az Úr Égyiptom között és Izráel között.
\par 8 És mindezek a te szolgáid lejönnek hozzám és leborulnak elõttem, mondván: Eredj ki te és mind a nép, a mely téged követ, és csak azután megyek el. És nagy haraggal méne ki a Faraó elõl.
\par 9 Az Úr pedig monda Mózesnek: Azért nem hallgat reátok a Faraó, hogy az én csudáim sokasodjanak meg Égyiptom földén.
\par 10 Mózes pedig és Áron mindezeket a csudákat megtevék a Faraó elõtt; de az Úr megkeményíté a Faraó szívét. És nem bocsátá el Izráel fiait az õ földérõl.

\chapter{12}

\par 1 Szólott vala pedig az Úr Mózesnek és Áronnak Égyiptom földén, mondván:
\par 2 Ez a hónap legyen néktek a hónapok elseje; elsõ legyen ez néktek az esztendõ hónapjai között.
\par 3 Szóljatok Izráel egész gyûlekezetének, mondván: E hónap tizedikén mindenki vegyen magának egy bárányt az atyáknak háza szerint, házanként egy bárányt.
\par 4 Hogyha a háznép kevés a bárányhoz, akkor a házához közel való szomszédjával együtt vegyen a lelkek száma szerint; kit-kit ételéhez képest számítsatok a bárányhoz.
\par 5 A bárány ép, hím, egy esztendõs legyen; a juhok közûl vagy a kecskék közûl vegyétek.
\par 6 És legyen nálatok õrizet alatt e hónap tizennegyedik napjáig, és ölje meg Izráel községének egész gyülekezete estennen.
\par 7 És vegyenek a vérbõl, és azokban a házakban, a hol azt megeszik, hintsenek a két ajtófélre és a szemöldökfára.
\par 8 A húst pedig egyék meg azon éjjel, tûzön sütve, kovásztalan kenyérrel és keserû fûvekkel egyék meg azt.
\par 9 Ne egyetek abból nyersen, vagy vízben fõtten, hanem tûzön sütve, a fejét, lábszáraival és belsejével együtt.
\par 10 És ne hagyjatok belõle reggelre, vagy a mi megmarad belõle reggelre, tûzzel égessétek meg.
\par 11 És ilyen módon egyétek azt meg: Derekaitokat felövezve, saruitok lábaitokon és pálczáitok kezetekben, és nagy sietséggel egyétek azt; mert az Úr páskhája az.
\par 12 Mert általmégyek Égyiptom földén ezen éjszakán és megölök minden elsõszülöttet Égyiptom földén, az embertõl kezdve a baromig, és Égyiptom minden istene felett ítéletet tartok, én, az Úr.
\par 13 És a vér jelül lesz néktek a házakon, a melyekben ti lesztek, s meglátom a vért és elmegyek mellettetek és nem lesz rajtatok a csapás veszedelmetekre, mikor megverem Égyiptom földét.
\par 14 És legyen ez a nap néktek emlékezetül, és innepnek szenteljétek azt az Úrnak nemzetségrõl nemzetségre; örök rendtartás szerint ünnepeljétek azt.
\par 15 Hét napig egyetek kovásztalan kenyeret; még az elsõ napon takarítsátok el a kovászt házaitokból, mert valaki kovászost ejéndik az elsõ naptól fogva a hetedik napig, az olyan lélek irtassék ki Izráelbõl.
\par 16 Az elsõ napon pedig szent gyûléstek legyen és a hetedik napon is szent gyûléstek legyen; azokon semmi munkát ne tegyetek, egyedül csak a mi eledelére való minden embernek, azt el lehet készítenetek.
\par 17 Megtartsátok a kovásztalan kenyér innepét; mert azon a napon hoztam ki a ti seregeiteket Égyiptom földérõl; tartsátok meg hát e napot nemzetségrõl nemzetségre, örök rendtartás szerint.
\par 18 Az elsõ hónapban, a hónapnak tizennegyedik napján estve egyetek kovásztalan kenyeret, a hónap huszonegyedik napjának estvéjéig.
\par 19 Hét napon át ne találtassék kovász a ti házaitokban; mert valaki kovászost ejéndik, az a lélek kiirtatik Izráel gyülekezetébõl, akár jövevény, akár az ország szülöttje legyen.
\par 20 Semmi kovászost ne egyetek, minden lakóhelyeteken kovásztalan kenyeret egyetek.
\par 21 Elõhívá tehát Mózes Izráel minden véneit és monda nékik: Fogjatok és vegyetek magatoknak bárányt családaitok szerint és öljétek meg a páskhát.
\par 22 És vegyetek egy kötés izsópot és mártsátok a vérbe, a mely az edényben van, és hintsétek meg a szemöldökfát és a két ajtófelet abból a vérbõl, a mely az edényben van; ti közûletek pedig senki se menjen ki az õ házának ajtaján reggelig.
\par 23 Mikor általmegy az Úr, hogy megverje az Égyiptombelieket és meglátja a vért a szemöldökfán és a két ajtófélen: elmegy az Úr az ajtó mellett és nem engedi, hogy a pusztító bemenjen öldökölni a ti házaitokba.
\par 24 Megtartsátok azért ezt a dolgot, rendtartás gyanánt, magadnak és fiaidnak mindörökre.
\par 25 És mikor bementek a földre, melyet az Úr ád néktek, a mint megmondotta vala: akkor tartsátok meg ezt a szertartást.
\par 26 Mikor pedig a ti fiaitok mondandjának néktek: Micsoda ez a ti szertartástok?
\par 27 Akkor mondjátok: Páskha-áldozat ez az Úrnak, a ki elment az Izráel fiainak házai mellett Égyiptomban, mikor megverte az Égyiptombelieket, a mi házainkat pedig megoltalmazta. És a nép meghajtá magát és leborula.
\par 28 És menének és úgy cselekedének az Izráel fiai, a mint megparancsolta vala az Úr Mózesnek és Áronnak; úgy cselekedének.
\par 29 Lõn pedig éjfélkor, hogy megöle az Úr minden elsõszülöttet Égyiptomnak földén, a Faraónak elsõszülöttétõl fogva, a ki az õ királyi székiben ûl vala, a tömlöczbeli fogolynak elsõszülöttéig és a baromnak is minden elsõ fajzását.
\par 30 És fölkele a Faraó azon az éjszakán és mind az õ szolgái és egész Égyiptom, és lõn nagy jajgatás Égyiptomban; mert egy ház sem vala, melyben halott ne lett volna.
\par 31 És hívatá Mózest és Áront éjszaka és monda: Keljetek fel, menjetek ki az én népem közûl, mind ti, mind Izráel fiai és menjetek, szolgáljatok az Úrnak, a mint mondátok.
\par 32 Juhaitokat is, barmaitokat is vegyétek, a mint mondátok és menjetek el és áldjatok engem is.
\par 33 És az Égyiptombeliek erõsen rajta valának, hogy a népet mentül hamarább kiküldhessék az országból; mert azt mondják vala: mindnyájan meghalunk.
\par 34 És a nép az õ tésztáját, minekelõtte megkelt volna, sütõteknõivel együtt ruhájába kötve, vállára veté.
\par 35 Az Izráel fiai pedig Mózes beszéde szerint cselekedének és kérének az Égyiptombeliektõl ezüst edényeket és arany edényeket, meg ruhákat.
\par 36 Az Úr pedig kedvessé tette vala a népet az Égyiptombeliek elõtt, hogy kérésökre hajlának és kifoszták az Égyiptombelieket.
\par 37 És elindulának Izráel fiai Rameszeszbõl Szukhóthba, mintegy hatszáz ezeren gyalog, csupán férfiak a gyermekeken kívül.
\par 38 Sok elegy nép is méne fel velök; juh is, szarvasmarha is, felette sok barom.
\par 39 És sütének a tésztából, melyet Égyiptomból hoztak vala, kovásztalan pogácsákat, mert meg nem kelhet vala mivelhogy kiûzetének Égyiptomból és nem késhetének s még eleséget sem készítének magoknak.
\par 40 Az Izráel fiainak lakása pedig, a míg Égyiptomban laknak, négyszáz harmincz esztendõ vala.
\par 41 És lõn a négyszáz harmincz esztendõ végén, lõn pedig ugyanazon napon, hogy az Úrnak minden serege kijöve Égyiptomnak földérõl.
\par 42 Az Úr tiszteletére rendelt éjszaka ez, a melyen kihozta õket Égyiptom földérõl; az Úr tiszteletére rendelt éjszaka Izráel minden fiai elõtt nemzetségrõl nemzetségre.
\par 43 És monda az Úr Mózesnek és Áronnak: Ez a Páskha rendtartása: Egy idegen származású se egyék abból.
\par 44 Akárkinek is pénzen vett szolgája akkor egyék abból, ha körûlmetélted.
\par 45 A zsellér és a béres ne egyék abból.
\par 46 Egy házban egyék meg; a házból ki ne vígy a húsból, és csontot se törjetek össze abban.
\par 47 Izráel egész gyülekezete készítse azt.
\par 48 És ha jövevény tartózkodik nálad, és páskhát akarna készíteni az Úrnak: metéltessék körûl minden férfia, és úgy foghat annak készítéséhez, és legyen olyan, mint az országnak szülötte. Egy körûlmetéletlen se egyék abból.
\par 49 Egy törvénye legyen az ott születettnek és a jövevénynek, a ki közöttetek tartózkodik.
\par 50 És Izráel fiai mindnyájan megcselekedék; a mint parancsolta vala az Úr Mózesnek és Áronnak, úgy cselekedének.
\par 51 Ugyanazon napon hozá ki az Úr az Izráel fiait Égyiptomnak földérõl, az õ seregeik szerint.

\chapter{13}

\par 1 És szóla az Úr Mózesnek, mondván:
\par 2 Nékem szentelj minden elsõszülöttet, valami megnyitja az õ anyjának méhét az Izráel fiai között, akár ember, akár barom, enyim legyen az.
\par 3 És monda Mózes a népnek: Megemlékezzél e napról, melyen kijöttetek Égyiptomból, a szolgálatnak házából; mert hatalmas kézzel hozott ki onnan titeket az Úr; azért ne egyetek kovászost.
\par 4 Ma mentek ki az Abib hónapban.
\par 5 És ha majd bevisz téged az Úr a Kananeusok, meg Khittheusok, meg Emoreusok, meg Khivveusok és Jebuzeusok földére, melyrõl megesküdött a ti atyáitoknak, hogy néked adja azt a téjjel és mézzel folyó földet: akkor ebben a hónapban végezd ezt a szertartást.
\par 6 Hét napon át kovásztalan kenyeret egyél, a hetedik napon pedig innepet ülj az Úrnak.
\par 7 Kovásztalan kenyeret egyél hét napon át, és ne láttassék nálad kovászos kenyér, se kovász ne láttassék a te egész határodban.
\par 8 És add tudtára a te fiadnak azon a napon, mondván: Ez a miatt van, a mit az Úr cselekedett velem, mikor kijövék Égyiptomból.
\par 9 És legyen az néked jel gyanánt a te kezeden és emlékezetül a te szemeid elõtt azért, hogy az Úr törvénye a te szádban legyen, mert hatalmas kézzel hozott ki téged az Úr Égyiptomból.
\par 10 Tartsd meg azért ezt a rendelést annak idejében esztendõrõl esztendõre.
\par 11 Ha pedig beviénd téged az Úr a Kananeusok földére, a miképen megesküdött néked és a te atyáidnak, és azt néked adándja:
\par 12 Az Úrnak ajánld fel akkor mindazt a mi az õ anyjának méhét megnyitja, a baromnak is, a mi néked lesz, minden méhnyitó fajzását; a hímek az Úré.
\par 13 De a szamárnak minden elsõ fajzását báránynyal váltsd meg; ha pedig meg nem váltod, szegd meg a nyakát. Az embernek is minden elsõszülöttét megváltsd a te fiaid közül.
\par 14 És ha egykor a te fiad téged megkérdez, mondván: Micsoda ez? akkor mondd néki: Hatalmas kézzel hozott ki minket az Úr Égyiptomból, a szolgálatnak házából.
\par 15 És lõn, mikor a Faraó megátalkodottan vonakodék minket elbocsátani: megöle az Úr minden elsõszülöttet Égyiptom földén, az ember elsõszülöttétõl a barom elsõ fajzásáig; azért áldozok én az Úrnak minden hímet, mely anyja méhét megnyitja, és megváltom az én fiaimnak minden elsõszülöttét.
\par 16 Legyen azért jel gyanánt a te kezeden és homlok-kötõ gyanánt a te szemeid elõtt, mert hatalmas kézzel hozott ki minket az Úr Égyiptomból.
\par 17 És lõn, a mikor elbocsátá a Faraó a népet, nem vivé õket Isten a Filiszteusok földje felé, noha közel vala az; mert monda az Isten: Netalán mást gondol a nép, ha harczot lát, és visszatér Égyiptomba.
\par 18 Kerülõ úton vezeté azért Isten a népet, a veres tenger pusztájának útján; és fölfegyverkezve jövének ki Izráel fiai Égyiptom földérõl.
\par 19 És Mózes elvivé magával a József tetemeit is, mert megesketvén megeskette vala Izráel fiait, mondván: Meglátogatván meglátogat titeket az Isten, akkor az én tetemeimet felvigyétek innen magatokkal.
\par 20 És elindulának Szukhótból és táborba szállának Ethámban a puszta szélén.
\par 21 Az Úr pedig megy vala elõttök nappal felhõoszlopban, hogy vezérelje õket az úton, éjjel pedig tûzoszlopban, hogy világítson nékik, hogy éjjel és nappal mehessenek.
\par 22 Nem távozott el a felhõoszlop nappal, sem a tûzoszlop éjjel a nép elõl.

\chapter{14}

\par 1 És szóla az Úr Mózesnek, mondván:
\par 2 Szólj az Izráel fiainak, hogy forduljanak vissza és üssenek tábort Pi-Hahiróth elõtt, Migdol között és a tenger között, Baál-Czefón elõtt; ezzel átellenben üssetek tábort a tenger mellett.
\par 3 Majd azt gondolja a Faraó az Izráel fiai felõl: Eltévelyedtek ezek e földön; körülfogta õket a puszta.
\par 4 Én pedig megkeményítem a Faraó szívét, és ûzõbe veszi õket, hogy megdicsõíttessem a Faraó által és minden õ serege által és megtudják az Égyiptombeliek, hogy én vagyok az Úr. És úgy cselekedének.
\par 5 És hírül vivék az égyiptomi királynak, hogy elfutott a nép, és megváltozék a Faraónak és az õ szolgáinak szíve a nép iránt és mondának: Mit cselekedtünk, hogy elbocsátottuk Izráelt a mi szolgálatunkból!
\par 6 Befogata tehát szekerébe és maga mellé vevé az õ népét.
\par 7 És võn hatszáz válogatott szekeret és Égyiptom minden egyéb szekerét és hárman-hárman valának mindeniken.
\par 8 És megkeményíté az Úr a Faraónak, az égyiptomi királynak szivét, hogy ûzõbe vegye az Izráel fiait; Izráel fiai pedig mennek vala nagy hatalommal.
\par 9 És az Égyiptombeliek utánok nyomulának és elérék õket a tenger mellett, a hol táboroznak vala, a Faraónak minden lova, szekere, meg lovasai és serege Pi-Hahiróth mellett, Baál-Czefón elõtt.
\par 10 A mint közeledék a Faraó, Izráel fiai felemelék szemeiket, és ímé az Égyiptombeliek nyomukban vannak. És nagyon megfélemlének s az Úrhoz kiáltának az Izráel fiai.
\par 11 És mondának Mózesnek: Hát nincsenek-é Égyiptomban sírok, hogy ide a pusztába hoztál minket meghalni? Mit cselekedél velünk, hogy kihoztál minket Égyiptomból?
\par 12 Nem ez volt-é a szó, a mit szóltunk vala hozzád Égyiptomban, mondván: Hagyj békét nékünk, hadd szolgáljunk az Égyiptombelieknek, mert jobb volt volna szolgálnunk az Égyiptombelieknek, hogynem mint a pusztában hallunk meg.
\par 13 Mózes pedig monda a népnek: Ne féljetek, megálljatok! és nézzétek az Úr szabadítását, a melyet ma cselekszik veletek; mert a mely Égyiptombelieket ma láttok, azokat soha többé nem látjátok.
\par 14 Az Úr hadakozik ti érettetek; ti pedig veszteg legyetek.
\par 15 És monda az Úr Mózesnek: Mit kiáltasz hozzám? Szólj Izráel fiainak, hogy induljanak el.
\par 16 Te pedig emeld fel a te pálczádat és nyújtsd ki kezedet a tengerre és válaszd azt kétfelé, hogy Izráel fiai szárazon menjenek át a tenger közepén.
\par 17 Én pedig ímé megkeményítem az Égyiptombeliek szívét, hogy bemenjenek utánok, és megdicsõíttetem a Faraó által és az õ egész serege által, szekerei és lovasai által.
\par 18 És megtudják az Égyiptombeliek, hogy én vagyok az Úr, ha majd megdicsõíttetem a Faraó által, az õ szekerei és lovasai által.
\par 19 Elindula azért az Istennek Angyala a ki jár vala az Izráel tábora elõtt, és méne mögéjök; a felhõoszlop is  elindula elõlök s mögéjök álla.
\par 20 És oda méne az Égyiptombeliek tábora és az Izráel tábora közé; így lõn a felhõ és a setétség: az éjszakát pedig megvilágosítja vala. És egész éjszaka nem közelítettek egymáshoz.
\par 21 És kinyújtá Mózes az õ kezét a tengerre, az Úr pedig egész éjjel erõs keleti széllel hajtá a tengert és szárazzá tevé a tengert, és kétfelé válának a vizek.
\par 22 És szárazon menének az Izráel fiai a tenger közepébe, a vizek pedig kõfal gyanánt valának nékik jobbkezök és balkezök felõl.
\par 23 Az Égyiptombeliek pedig utánok nyomulának és bemenének a Faraó minden lovai, szekerei és lovasai a tenger közepébe.
\par 24 És lõn hajnalkor, rátekinte az Úr az Égyiptombeliek táborára a tûz- és felhõoszlopból és megzavará az Égyiptombeliek táborát.
\par 25 És megállítá szekereik kerekeit és nehezen vonszoltatá azokat. És mondának az Égyiptombeliek: Fussunk az Izráel elõl, mert az Úr hadakozik érettök Égyiptom ellen.
\par 26 És szóla az Úr Mózesnek: Nyújtsd ki kezedet a tengerre, hogy a vizek térjenek vissza az Égyiptombeliekre, az õ szekereikre s lovasaikra.
\par 27 És kinyújtá Mózes az õ kezét a tengerre, és reggel felé visszatére a tenger az õ elébbi állapotjára; az Égyiptombeliek pedig eleibe futnak vala, és az Úr beleveszté az égyiptomiakat a tenger közepébe.
\par 28 Visszatérének tehát a vizek és elboríták a szekereket és a lovasokat, a Faraónak minden seregét, melyek utánok bementek vala a tengerbe; egy sem marada meg közülök.
\par 29 De Izráel fiai szárazon menének át a tenger közepén; a vizek pedig kõfal gyanánt valának nékik jobb- és balkezök felõl.
\par 30 És megszabadítá az Úr azon a napon Izráelt az Égyiptombeliek kezébõl; és látá Izráel a megholt Égyiptombelieket a tenger partján.
\par 31 És látá Izráel azt a nagy dolgot, a melyet cselekedék az Úr Égyiptomban: félé azért a nép az Urat és hívének az Úrnak és Mózesnek, az õ szolgájának.

\chapter{15}

\par 1 Akkor éneklé Mózes és az Izráel fiai ezt az éneket az Úrnak, és szólának mondván: Éneklek az Úrnak, mert fenséges õ, lovat lovasával tengerbe vetett.
\par 2 Erõsségem az Úr és énekem, szabadítómmá lõn nekem; ez az én Istenem, õt dicsérem, atyámnak Istene, õt magasztalom.
\par 3 Vitéz harczos az Úr; az õ neve Jehova.
\par 4 A Faraónak szekereit és seregét tengerbe vetette, s válogatott harczosai belefúltak a veres tengerbe.
\par 5 Elborították õket a hullámok, kõ módjára merültek a mélységbe.
\par 6 Jobbod, Uram, erõ által dicsõül, jobbod Uram, ellenséget összetör.
\par 7 Fenséged nagyságával zúzod össze támadóid, kibocsátod haragod s megemészti az õket mint tarlót.
\par 8 Orrod lehelletétõl feltorlódtak a vizek. És a futó habok fal módjára megálltak; a mélységes vizek megmerevültek a tenger szívében.
\par 9 Az ellenség monda: Ûzöm, utólérem õket, zsákmányt osztok, bosszúm töltöm rajtok. Kardomat kirántom, s kiirtja õket karom.
\par 10 Leheltél lehelleteddel s tenger borítá be õket: elmerültek, mint az ólom a nagy vizekben.
\par 11 Kicsoda az istenek közt olyan, mint te Uram? Kicsoda olyan, mint te, szentséggel dicsõ, félelemmel dícsérendõ és csudatévõ?
\par 12 Kinyújtottad jobbkezedet, és elnyelé õket a föld.
\par 13 Kegyelmeddel vezérled te megváltott népedet, hatalmaddal viszed be te szent lakóhelyedre.
\par 14 Meghallják ezt a népek és megrendülnek; Filisztea lakóit reszketés fogja el.
\par 15 Akkor megháborodának Edom fejedelmei, Moáb hatalmasait rettegés szállja meg, elcsügged a Kanaán egész lakossága.
\par 16 Félelem és aggodalom lepi meg õket; karod hatalmától elnémulnak mint a kõ, míg átvonul néped, Uram! Míg átvonul a nép, a te szerzeményed.
\par 17 Beviszed s megtelepíted õket örökséged hegyén, mely Uram, lakhelyûl magadnak készítél, szentségedbe Uram, melyet kezed építe.
\par 18 Az Úr uralkodik mind örökkön örökké.
\par 19 Mert bémenének a Faraó lovai, szekereivel és lovasaival együtt a tengerbe, és az Úr visszafordítá reájok a tenger vizét; Izráel fiai pedig szárazon jártak a tenger közepén.
\par 20 Akkor Miriám prófétaasszony, Áronnak nénje dobot võn kezébe, és kimenének utánna mind az asszonyok dobokkal és tánczolva.
\par 21 És felele nékik Miriám: Énekeljetek az Úrnak, mert fenséges õ, lovat lovasával tengerbe vetett.
\par 22 Ennekutánna elindítá Mózes az Izráelt a veres tengertõl, és menének Súr puszta felé; három napig menének a pusztában és nem találának vizet.
\par 23 És eljutának Márába de nem ihatják vala a vizet Márában, mivelhogy keserû vala. Azért is nevezék nevét Márának.
\par 24 És zúgolódik vala a nép Mózes ellen, mondván: Mit igyunk?
\par 25 Ez pedig az Úrhoz kiálta, és mutata néki az Úr egy fát és beveté azt a vízbe, és a víz megédesedék. Ott ada néki rendtartást és törvényt és ott megkísérté.
\par 26 És monda: Ha a te Uradnak Istenednek szavára hûségesen hallgatsz és azt cselekeszed, a mi kedves az õ szemei elõtt és figyelmezel az õ parancsolataira és megtartod minden rendelését: egyet sem bocsátok reád ama betegségek közül, a melyeket Égyiptomra bocsátottam, mert én vagyok az Úr, a te gyógyítód.
\par 27 És jutának Élimbe, és ott tizenkét forrás vala és hetven pálmafa; és tábort ütének ott a vizek mellett.

\chapter{16}

\par 1 És elindulának Élimbõl és érkezék Izráel fiainak egész gyülekezete a Szin pusztájába, mely Élim között és Sinai között van, a második hónapnak tizenötödik napján, Égyiptom földérõl való kijövetelök után.
\par 2 És zúgolódék Izráel fiainak egész gyülekezete Mózes és Áron ellen a pusztában.
\par 3 S mondának nékik Izráel fiai: Bár megholtunk volna az Úr keze által Égyiptom földén, a mikor a húsos fazék mellett ülünk vala, a mikor jól lakhatunk vala kenyérrel; mert azért hoztatok ki minket ebbe a pusztába, hogy mind e sokaságot éhséggel öljétek meg.
\par 4 És monda az Úr Mózesnek: Ímé én esõképen bocsátok néktek kenyeret az égbõl; menjen ki azért a nép és szedjen naponként arra a napra valót, hogy megkísértsem: akar-é az én törvényem szerint járni, vagy nem?
\par 5 A hatodik napon pedig úgy lesz, hogy mikor elkészítik a mit bevisznek, az kétannyi lesz, mint a mennyit naponként szedegettek.
\par 6 És monda Mózes és Áron Izráel minden fiainak: Estve megtudjátok, hogy az Úr hozott ki titeket Égyiptom földérõl;
\par 7 Reggel pedig meglátjátok az Úr dicsõségét; mert meghallotta a ti zúgolódástokat az Úr ellen. De mik vagyunk mi, hogy mi ellenünk zúgolódtok?
\par 8 És monda Mózes: Estve húst ád az Úr ennetek, reggel pedig kenyeret, hogy jól lakjatok; mert hallotta az Úr a ti zúgolódástokat, melylyel ellene zúgolódtatok. De mik vagyunk mi? Nem mi ellenünk van a ti zúgolódástok, hanem az Úr ellen.
\par 9 Áronnak pedig monda Mózes: Mondd meg az Izráel fiai egész gyülekezetének: Járuljatok az Úr elé; mert meghallotta a ti zúgolódástokat.
\par 10 És lõn, mikor beszéle Áron az Izráel fiai egész gyülekezetének, a puszta felé fordulának, és ímé az Úr dicsõsége megjelenék a felhõben.
\par 11 És szóla az Úr Mózesnek, mondván:
\par 12 Hallottam az Izráel fiainak zúgolódását, szólj nékik mondván: Estennen húst esztek, reggel pedig kenyérrel laktok jól és megtudjátok, hogy én vagyok az Úr a ti Istentek.
\par 13 És lõn, hogy estve fürjek jövének fel és ellepék a tábort, reggel pedig harmatszállás lõn a tábor körûl.
\par 14 Mikor pedig a harmatszállás megszûnék, ímé a pusztának színén apró gömbölyegek valának, aprók mint a dara a földön.
\par 15 A mint megláták az Izráel fiai, mondának egymásnak: Mán ez! mert nem tudják vala mi az. Mózes pedig monda nékik: Ez az a kenyér, melyet az Úr adott néktek eledelül.
\par 16 Az Úr parancsolata pedig ez: Szedjen abból kiki a mennyit megehetik; fejenként egy ómert, a hozzátok tartozók száma szerint szedjen kiki azok részére, a kik az õ sátorában vannak.
\par 17 És aképen cselekedének az Izráel fiai és szedének ki többet, ki kevesebbet.
\par 18 Azután megmérik vala ómerrel, és annak a ki többet szedett, nem vala fölöslege, és annak, a ki kevesebbet szedett, nem vala fogyatkozása: kiki annyit szedett, a mennyit megehetik vala.
\par 19 Azt is mondá nékik Mózes: Senki ne hagyjon abból reggelre.
\par 20 De nem hallgatának Mózesre, mert némelyek hagyának abból reggelre; és megférgesedék s megbüszhödék. Mózes pedig megharagudék reájok.
\par 21 Szedék pedig azt reggelenként, kiki a mennyit megehetik vala, mert ha a nap felmelegedett, elolvad vala.
\par 22 A hatodik napon pedig két annyi kenyeret szednek vala, két ómerrel egyre-egyre. Eljövének pedig a gyülekezet fejedelmei mindnyájan és tudtára adták azt Mózesnek.
\par 23 Õ pedig monda nékik: Ez az, a mit az Úr mondott: A holnap nyugalom napja, az Úrnak szentelt szombat; a mit sütni akartok, süssétek meg, és a mit fõzni akartok, fõzzétek meg; a mi pedig megmarad, azt mind tegyétek el magatoknak reggelre.
\par 24 És eltevék azt reggelre, a szerint a mint Mózes parancsolta vala, és nem büszhödék meg s féreg sem vala benne.
\par 25 És monda Mózes: Ma egyétek azt meg, mert ma az Úrnak szombatja van; ma nem találjátok azt a mezõn.
\par 26 Hat napon szedjétek azt, de a hetedik napon szombat van, akkor nem lesz.
\par 27 És lõn hetednapon: kimenének a nép közül, hogy szedjenek, de nem találának.
\par 28 És monda az Úr Mózesnek: Meddig nem akarjátok megtartani az én parancsolataimat és törvényeimet?
\par 29 Lássátok meg! az Úr adta néktek a szombatot; azért ád õ néktek hatodnapon két napra való kenyeret. Maradjatok veszteg, kiki a maga helyén; senki se menjen ki az õ helyébõl a hetedik napon.
\par 30 És nyugoszik vala a nép a hetedik napon.
\par 31 Az Izráel háza pedig Mánnak nevezé azt; olyan vala az mint a kóriándrom magva, fehér; és íze, mint a mézes pogácsáé.
\par 32 És monda Mózes: Ezt parancsolja az Úr: Egy teljes ómernyit tartsatok meg abból maradékaitok számára, hogy lássák a kenyeret, a mellyel éltettelek titeket a pusztában, mikor kihoztalak titeket Égyiptom földérõl.
\par 33 Áronnak pedig monda Mózes: Végy egy edényt és tégy bele egy teljes ómer Mánt és tedd azt az Úr eleibe, hogy megtartassék maradékaitok számára.
\par 34 A mint parancsolta vala az Úr Mózesnek, eltevé azt Áron a bizonyságtétel ládája elé, hogy megtartassék.
\par 35 Az Izráel fiai pedig negyven esztendõn át evék a Mánt, míg lakó földre jutának; Mánt evének mind addig, míg a Kanaán földének határához jutának.
\par 36 Az ómer pedig az éfának tizedrésze.

\chapter{17}

\par 1 És elindula Izráel fiainak egész gyülekezete a Szin pusztájából az Úr rendeléséhez képest az õ útjok rendje szerint és tábort ütének Refidimben. De a népnek nem vala inni való vize.
\par 2 Verseng vala azért a nép Mózessel és mondák: Adjatok nékünk vizet, hogy igyunk. És monda nékik Mózes: Miért versengtek én velem? Miért kísértitek az Urat?
\par 3 És szomjúhozik vala ott a nép a vízre és zúgolódék a nép Mózes ellen és monda: Miért hoztál ki minket Égyiptomból? hogy szomjúsággal ölj meg minket, gyermekeinket és barmainkat?
\par 4 Mózes pedig az Úrhoz kiálta mondván: Mit cselekedjem ezzel a néppel? Kevés hijja, hogy meg nem köveznek engemet.
\par 5 És az Úr monda Mózesnek: Eredj el a nép elõtt és végy magad mellé Izráel vénei közûl; pálczádat is, melylyel a folyót  megsujtottad, vedd kezedbe és indulj el.
\par 6 Ímé én oda állok te elõdbe a sziklára a Hóreben, és te sujts a sziklára, és víz jõ ki abból, hogy igyék a nép. És úgy cselekedék Mózes Izráel vénei szeme láttára.
\par 7 És nevezé annak a helynek nevét Masszának és Méribának, Izráel fiainak versengéséért, és mert kísértették az Urat, mondván: Vajjon köztünk van-é az Úr vagy nincsen?
\par 8 Eljöve pedig Amálek és hadakozék Izráel ellen Refidimben.
\par 9 És monda Mózes Józsuénak: Válaszsz nékünk férfiakat és menj el, ütközzél meg Amálekkel. Holnap én a halom tetejére állok és az Isten pálczája kezemben lesz.
\par 10 És úgy cselekedék Józsué a mint mondotta vala néki Mózes, megütközék Amálekkel: Mózes, Áron és Húr pedig felmenének a halom tetejére.
\par 11 És lõn, mikor Mózes felemelé kezét, Izráel gyõz vala; mikor pedig leereszté kezét, Amálek gyõz vala.
\par 12 Mikor azért Mózes kezei elnehezedének, követ hozának és alája tevék, hogy arra ûljön; Áron pedig és Húr tartják vala az õ kezeit, egy felõl az egyik, más felõl a másik, és felemelve maradának kezei a nap lementéig.
\par 13 Józsué pedig leveré Amáleket és az õ népét fegyver élivel.
\par 14 És monda az Úr Mózesnek: Írd meg ezt emlékezetül könyvbe és add tudtára Józsuénak, hogy mindenestõl eltörlöm Amálek  emlékezetét az ég alól.
\par 15 És építe Mózes oltárt és nevezé nevét Jehova-Niszszi-nek.
\par 16 És monda: Megesküdött az Úr, hogy harcza lesz az Úrnak Amálek ellen nemzetségrõl nemzetségre.

\chapter{18}

\par 1 És meghallá Jethró, Midián papja, Mózes ipa, mindazt a mit Isten Mózessel és Izráellel az õ népével cselekedett vala, hogy kihozta az Úr Izráelt Égyiptomból.
\par 2 És felvevé Jethró, a Mózes ipa Czipporát, a Mózes feleségét - miután haza bocsátotta õt -
\par 3 És az õ két fiát is, a kik közûl az egyiknek neve Gersom, mert azt mondotta vala: Bujdosó valék az idegen földön;
\par 4 A másiknak neve pedig Eliézer; mert: Az én atyám Istene segítségül volt nékem és megszabadított engem a Faraó fegyverétõl.
\par 5 Eljuta tehát Jethró, a Mózes ipa, az õ fiaival és feleségével Mózeshez a pusztába, a hol õ táborozott vala az Isten hegye mellett.
\par 6 És megizené Mózesnek: Én Jethró a te ipad megyek te hozzád a te feleségeddel és az õ két fia is õ vele.
\par 7 Kiméne azért Mózes az õ ipa eleibe és meghajtá magát és megcsókolá õt; és megkérdék egymást állapotuk felõl és bemenének a sátorba.
\par 8 És elbeszélé Mózes az õ ipának mind azt, a mit az Úr a Faraóval és az Égyiptombeliekkel cselekedett vala Izráelért; mindazt a sok bajt, a melyek útközben érték vala õket, és mimódon szabadította meg õket az Úr.
\par 9 És örvendeze Jethró mindazon a jón, a mit az Úr az Izráellel cselekedett vala, hogy megszabadítá õt az Égyiptombeliek kezébõl.
\par 10 És monda Jethró: Áldott legyen az Úr, a ki megszabadított titeket az Égyiptombeliek kezébõl és a Faraó kezébõl; a ki megszabadította a népet az Égyiptombeliek keze alól.
\par 11 Most tudom már, hogy nagyobb az Úr minden istennél; mert az lett vesztökre, a mivel ellenök vétkeztek.
\par 12 És Jethró, a Mózes ipa égõáldozattal és véresáldozattal áldozék az Istennek; Áron pedig és Izráel minden vénei jövének, hogy kenyeret egyenek a Mózes ipával Isten elõtt.
\par 13 És lõn másod napon, leûle Mózes törvényt tenni a népnek; a nép pedig áll vala Mózes elõtt reggeltõl estig.
\par 14 S a mikor látja vala Mózes ipa mind azt, a mit õ a néppel cselekedék, monda: Mi dolog az, a mit te a néppel cselekszel; miért ûlsz te egymagad, mind az egész nép pedig elõtted áll reggeltõl estig?
\par 15 És monda Mózes az õ ipának: Mert a nép Isten akaratát tudakolni jön hozzám;
\par 16 Ha ügyök-bajok van, én hozzám jõnek és törvényt teszek az ember között és felebarátja között és tudtára adom az Isten végezéseit és törvényeit.
\par 17 Mózes ipa pedig monda néki: Nem jó az, a mit te cselekszel.
\par 18 Felettébb kifáradsz te is, ez a nép is, a mely veled van; mert erõd felett való dolog ez, nem végezheted azt egymagad.
\par 19 Most azért hallgass az én szavamra, tanácsot adok néked és az Isten veled lesz. Te légy a népnek szószólója az Isten elõtt, és te vidd az ügyeket Isten eleibe.
\par 20 És tanítsd õket a rendeletekre és törvényekre és add tudtokra az útat, a melyen járniok kell és a tenni valót, a melyet tenniök kell.
\par 21 És szemelj ki magad az egész nép közûl derék, istenfélõ férfiakat, igazságos férfiakat, a kik gyûlölik a haszonlesést és tedd közöttük elõljárókká, ezeredesekké, századosokká, ötvenedesekké és tizedesekké.
\par 22 Ezek tegyenek ítéletet a népnek minden idõben, úgy hogy minden nagyobb ügyet te elõdbe hozzanak, minden csekélyebb dologban pedig õk ítéljenek; így könnyítve lesz rajtad, ha azt veled együtt hordozzák.
\par 23 Ha ezt cselekszed és az Isten is parancsolja néked: megállhatsz és az egész nép is helyére jut békességben.
\par 24 És hallgata Mózes az õ ipa szavára és mindazt megtevé, a mit mondott vala.
\par 25 És választa Mózes az egész Izráelbõl derék férfiakat és a nép fejeivé tevé õket, ezeredesekké, századosokká, ötvenedesekké, és tizedesekké.
\par 26 És ítélik vala a népet minden idõben; a nehéz dolgokat Mózes elé viszik vala, minden kisebb dologban pedig õk ítélnek vala.
\par 27 És elbocsátá Mózes az õ ipát, és ez elméne hazájába.

\chapter{19}

\par 1 A harmadik hónapban azután hogy kijöttek vala Izráel fiai Égyiptom földérõl, azon a napon érkezének a Sinai pusztába.
\par 2 Refidimbõl elindulván, érkezének a Sinai pusztába és táborba szállának a pusztában; a hegygyel átellenben szálla pedig ott táborba az Izráel.
\par 3 Mózes pedig felméne az Istenhez, és szóla hozzá az Úr a hegyrõl, mondván: Ezt mondd a Jákób házanépének és ezt add tudtára az Izráel fiainak.
\par 4 Ti láttátok, a mit Égyiptommal cselekedtem, hogy hordoztalak titeket sas szárnyakon és magamhoz bocsátottalak titeket.
\par 5 Mostan azért ha figyelmesen hallgattok szavamra és megtartjátok az én szövetségemet, úgy ti lesztek nékem valamennyi nép közt az enyéim; mert enyim az egész föld.
\par 6 És lesztek ti nékem papok birodalma és szent nép. Ezek azok az ígék, melyeket el kell mondanod Izráel fiainak.
\par 7 Elméne azért Mózes és egybehívá a nép véneit és eleikbe adá mindazokat a beszédeket, melyeket parancsolt vala néki az Úr.
\par 8 És az egész nép egy akarattal felele és monda: Valamit rendelt az Úr, mind megteszszük. És megvivé Mózes az Úrnak a nép beszédét.
\par 9 És monda az Úr Mózesnek: Ímé én hozzád megyek a felhõ homályában, hogy hallja a nép mikor beszélek veled és higyjenek néked mindörökké. És elmondá Mózes az Úrnak a nép beszédét.
\par 10 Az Úr pedig monda Mózesnek: Eredj el a néphez és szenteld meg õket ma, meg holnap és hogy mossák ki az õ ruháikat;
\par 11 És legyenek készek harmadnapra; mert harmadnapon leszáll az Úr az egész nép szeme láttára a Sinai hegyre.
\par 12 És vess határt a népnek köröskörûl, mondván: Vigyázzatok magatokra, hogy a hegyre fel ne menjetek s még a szélét se érintsétek; mindaz, a mi a hegyet érinti, halállal haljon meg.
\par 13 Ne érintse azt kéz, hanem kõvel köveztessék meg, vagy nyillal nyilaztassék le; akár barom, akár ember, ne éljen. Mikor a kürt hosszan hangzik, akkor felmehetnek a hegyre.
\par 14 Leszálla azért Mózes a hegyrõl a néphez, és megszentelé a népet, és megmosák az õ ruháikat.
\par 15 És monda a népnek: Legyetek készen harmadnapra; asszonyhoz ne közelítsetek.
\par 16 És lõn harmadnapon virradatkor, mennydörgések, villámlások és sûrû felhõ lõn a hegyen és igen erõs kürtzengés; és megrémûle mind az egész táborbeli nép.
\par 17 És kivezeté Mózes a népet a táborból az Isten eleibe és megállának a hegy alatt.
\par 18 Az egész Sinai hegy pedig füstölög vala, mivelhogy leszállott arra az Úr tûzben és felmegy vala annak füstje, mint a kemenczének füstje; és az egész hegy nagyon reng vala.
\par 19 és a kürt szava mindinkább erõsödik vala; Mózes beszél vala és az Isten felel vala néki hangosan.
\par 20 Leszálla tehát az Úr a Sinai hegyre, a hegy tetejére, és felhívá az Úr Mózest a hegy tetejére, Mózes peig felméne.
\par 21 És monda az Úr Mózesnek: Menj alá, intsd meg a népet, hogy ne törjön elõre az Urat látni, mert közûlök sokan elhullanak.
\par 22 És a papok is, a kik az Úr eleibe járulnak, szenteljék meg magokat, hogy reájok ne rontson az Úr.
\par 23 Mózes pedig monda az Úrnak: Nem jöhet fel a nép a Sinai hegyre, mert te magad intettél minket, mondván: Vess határt a hegy körûl, és szenteld meg azt.
\par 24 De az Úr monda néki: Eredj, menj alá, és jõjj fel te és Áron is veled; de a papok és a nép ne törjenek elõre, hogy feljõjjenek az Úrhoz; hogy reájok ne rontson.
\par 25 Aláméne azért Mózes a néphez, és megmondá nékik.

\chapter{20}

\par 1 És szólá Isten mindezeket az igéket, mondván:
\par 2 Én, az Úr, vagyok a te Istened, a ki kihoztalak téged Égyiptomnak földérõl, a szolgálat házából.
\par 3 Ne legyenek néked idegen isteneid én elõttem.
\par 4 Ne csinálj magadnak faragott képet, és semmi hasonlót azokhoz, a melyek fenn az égben, vagy a melyek alant a földön, vagy a melyek a vizekben a föld alatt vannak.
\par 5 Ne imádd és ne tiszteld azokat; mert én, az Úr a te Istened, féltõn-szeretõ Isten vagyok, a ki megbüntetem az atyák vétkét a fiakban, harmad és negyediziglen, a kik engem gyûlölnek.
\par 6 De irgalmasságot cselekszem ezeriziglen azokkal, a kik engem szeretnek, és az én parancsolataimat megtartják.
\par 7 Az Úrnak a te Istenednek nevét hiába fel ne vedd; mert nem hagyja az Úr büntetés nélkül, a ki az õ nevét hiába felveszi.
\par 8 Megemlékezzél a szombatnapról, hogy megszenteljed azt.
\par 9 Hat napon át munkálkodjál, és végezd minden dolgodat;
\par 10 De a hetedik nap az Úrnak a te Istenednek szombatja: semmi dolgot se tégy azon se magad, se fiad, se leányod, se szolgád, se szolgálóleányod, se barmod, se jövevényed, a ki a te kapuidon belõl van;
\par 11 Mert hat napon teremté az Úr az eget és a földet, a tengert és mindent, a mi azokban van, a hetedik napon pedig megnyugovék. Azért megáldá az Úr a szombat napját, és megszentelé azt.
\par 12 Tiszteld atyádat és anyádat, hogy hosszú ideig élj azon a földön, a melyet az Úr a te Istened ád te néked.
\par 13 Ne ölj
\par 14 Ne paráználkodjál.
\par 15 Ne lopj
\par 16 Ne tégy a te felebarátod ellen hamis tanúbizonyságot.
\par 17 Ne kívánd a te  felebarátodnak házát. Ne kívánd a te felebarátodnak feleségét, se szolgáját, se szolgálóleányát, se ökrét, se szamarát, és semmit, a mi a te felebarátodé.
\par 18 Az egész nép pedig látja vala a mennydörgéseket, a villámlásokat, a kürt zengését és a hegy füstölgését. És látja vala a nép, és megrémüle, és hátrább álla.
\par 19 És mondának Mózesnek: Te beszélj velünk, és mi hallgatunk; de az Isten ne beszéljen velünk, hogy meg ne haljunk.
\par 20 Mózes pedig monda a népnek: Ne féljetek; mert azért jött az Isten, hogy titeket megkísértsen, és hogy az õ félelme legyen elõttetek, hogy ne vétkezzetek.
\par 21 Távol álla azért a nép, Mózes pedig közelebb méne a felhõhöz, melyben az Isten vala.
\par 22 És monda az Úr Mózesnek: Ezt mondd az Izráel fiainak: Magatok láttátok, hogy az égbõl beszéltem veletek.
\par 23 Ne csináljatok én mellém ezüst isteneket, és arany isteneket se csináljatok magatoknak.
\par 24 Földbõl csinálj nékem oltárt, és azon áldozd a te égõ- és hálaáldozatodat, juhaidat és ökreidet. Valamely helyen akarom, hogy az én nevemrõl megemlékezzetek, elmegyek tehozzád és megáldalak téged.
\par 25 Ha pedig kövekbõl csinálsz nékem oltárt, ne építsd azt faragott kõbõl: mert a mint faragó vasadat rávetetted, megfertõztetted azt.
\par 26 Lépcsõkön se menj fel az én oltáromhoz, hogy a te szemérmed fel ne fedeztessék azon.

\chapter{21}

\par 1 Ezek pedig azok a rendeletek, a melyeket eleikbe kell terjesztened:
\par 2 Ha héber szolgát vásárolsz, hat esztendeig szolgáljon, a hetedikben pedig szabaduljon fel ingyen.
\par 3 Ha egyedûl jött, egyedûl menjen el; ha feleséges ember, menjen el vele a felesége is.
\par 4 Ha az õ ura adott nékifeleséget, és ez fiakat vagy leányokat szûlt néki: az asszony, gyermekeivel együtt legyen az õ uráé; õ pedig egyedûl menjen el.
\par 5 De ha a szolga azt mondá: Szeretem az én uramat, az én feleségemet és fiaimat, nem akarok felszabadulni:
\par 6 Akkor vigye õt az ura a bírák elé, és állítsa az ajtóhoz vagy az ajtófélhez, és az õ ura fúrja által az õ fülét árral; és szolgálja õt mindörökké.
\par 7 És ha valaki az õ leányát szolgálóul adja el, ne úgy menjen el, mint a szolgák mennek.
\par 8 Ha nem tetszik az õ urának, hogy eljegyezze õt magának, akkor váltassa ki; arra, hogy idegen népnek eladja, nincs hatalma, mivel hûtelen volt hozzá.
\par 9 Ha pedig a fiának jegyzi el õt: a leányok törvénye szerint cselekedjék vele.
\par 10 Ha mást vesz magának: ennek ételét, ruházatát és házasságbeli igazát alább ne szállítsa.
\par 11 Ha ezt a hármat nem cselekszi vele: akkor menjen az el ingyen, fizetés nélkûl.
\par 12 A ki úgy megver valakit, hogy meghal, halállal lakoljon.
\par 13 De ha nem leselkedett, hanem Isten ejtette kezébe: úgy helyet rendelek néked, a hová meneküljön.
\par 14 Ha pedig valaki szándékosan tör felebarátja ellen, hogy azt orvúl megölje, oltáromtól is elvidd azt a halálra.
\par 15 A ki megveri az õ atyját vagy anyját, halállal lakoljon.
\par 16 A ki embert lop, és eladja azt, vagy kezében kapják, halállal lakoljon.
\par 17 A ki szidalmazza az õ atyját vagy anyját halállal lakoljon.
\par 18 És ha férfiak összevesznek, és megüti valaki az õ felebarátját kõvel vagy öklével, és nem hal meg, hanem ágyba esik:
\par 19 Ha felkél, és mankóján kinn jár: ne legyen büntetve az, a ki megütötte; csupán fekvéséért fizessen és gyógyíttassa meg.
\par 20 Ha pedig valaki úgy üti meg szolgáját vagy szolgálóját bottal, hogy az meghal keze alatt, büntettessék meg.
\par 21 De ha egy vagy két nap életben marad, ne büntettessék meg, mert pénze ára az.
\par 22 Ha férfiak veszekednek és meglöknek valamely terhes asszonyt, úgy hogy idõ elõtt szûl, de egyéb veszedelem nem történik: bírságot fizessen a szerint, a mint az asszony férje azt reá kiveti, de bírák elõtt fizessen.
\par 23 De ha veszedelem történik: akkor életért életet adj.
\par 24 Szemet szemért, fogat fogért, kezet kézért, lábat lábért;
\par 25 Égetést égetésért, sebet sebért, kéket kékért.
\par 26 Ha valaki az õ szolgájának szemét, vagy szolgálójának szemét úgy üti meg, hogy elpusztul, bocsássa azt szabadon az õ szeméért.
\par 27 Ha pedig szolgájának fogát, vagy szolgálójának fogát üti ki, bocsássa azt szabadon az õ fogáért.
\par 28 Ha férfit vagy asszonyt öklel meg egy ökör, úgy hogy meghal: kõvel köveztessék meg az ökör, és húsát meg ne egyék; de az ökörnek ura ártatlan.
\par 29 De ha az ökör azelõtt is öklelõs volt, és annak urát megintették, és még sem õrizte azt, és férfit vagy asszonyt ölt meg: az ökör köveztessék meg, és az ura is halállal lakoljon.
\par 30 Ha pénzváltságot vetnek reá, fizessen lelke váltságáért annyit, a mennyit reá kivetnek.
\par 31 Akár fiút ökleljen meg, akár leányt ökleljen meg: e szerint a rendelet szerint kell cselekedni.
\par 32 Ha szolgát öklel meg az ökör vagy szolgálót: adassék azok urának harmincz ezüst siklus, az ökör pedig köveztessék meg.
\par 33 Ha pedig valaki vermet nyit meg, vagy ha valaki vermet ás, és nem fedi azt be, és ökör vagy szamár esik bele:
\par 34 A verem ura fizessen; pénzül térítse meg azok urának, a hulla pedig legyen az övé.
\par 35 És ha valakinek ökre megdöfi az õ felebarátja ökrét, úgy hogy az elpusztul: adják el az eleven ökröt, és az árát osszák meg, és a hullát is osszák el.
\par 36 Vagy ha tudták, hogy az ökör már azelõtt öklelõs volt, és nem õrizte azt annak ura: fizessen ökröt az ökörért, a hulla pedig az övé legyen.

\chapter{22}

\par 1 Ha valaki ökröt vagy bárányt lop, és levágja vagy eladja azt: öt barmot fizessen egy ökörért, és négy juhot egy bárányért.
\par 2 Ha ház-fölverésen kapják a tolvajt, és úgy megverik, hogy meghal, nincs vérbûn miatta.
\par 3 Ha reá sütött a nap, vérbûn van miatta: fizessen; ha nincs néki: adassék el tolvajságáért.
\par 4 Ha elevenen kapják kezében a lopott jószágot, akár ökör, akár szamár, akár juh: két annyit fizessen érette.
\par 5 Ha valaki mezõt vagy szõlõt étet le, úgy hogy barmát elereszti, és az a más mezején legel: mezejének javából és szõlõjének javából fizesse meg a kárt.
\par 6 Ha tûz támad, és tövisbe kap, és megég az asztag, vagy a lábon álló gabona, vagy a mezõ: fizesse meg a kárt, a ki a tüzet gyujtotta.
\par 7 Ha valaki pénzt vagy edényeket ád felebarátjának megõrzés végett, és ellopják annak a férfiúnak házából: ha megtalálják a tolvajt, fizessen kétannyit.
\par 8 Ha nem találják meg a tolvajt, a ház urát vigyék a bírák eleibe; hogy nem nyújtotta-é ki kezét az õ felebarátja vagyonára.
\par 9 Akármi bûn dolgában, akár ökör, akár szamár, akár juh, akár ruha, akármi elveszett jószág az, a mirõl azt mondja: ez az; mindkettõjük ügye a bírák eleibe menjen, és a kit a bírák bûnösnek mondanak, fizessen két annyit az õ felebarátjának.
\par 10 Ha valaki szamarat vagy ökröt, vagy bárányt, vagy akármiféle barmot ád az õ felebarátja gondviselése alá, és az elhull, vagy megsérül, vagy elhajtatik, úgy hogy senki sem látta:
\par 11 Az Úrra való esküvés legyen kettejök közt, hogy nem nyújtotta-é ki kezét felebarátja vagyonára; és ezt fogadja el annak ura, és az semmit se fizessen.
\par 12 Ha pedig ellopták tõle: megfizessen az urának.
\par 13 Ha széttépetett, hozza el azt bizonyságul; a széttépettet nem fizeti meg.
\par 14 Ha pedig valaki kölcsön kér az õ felebarátjától, és az megsérül vagy elhull urának távollétében: fizesse meg a kárt.
\par 15 Ha ura vele van, nem fizet; ha bérbe adatott, bérébe megy.
\par 16 Ha valaki hajadont csábít el, a ki nincs eljegyezve, és vele hál: jegyajándékkal jegyezze azt el magának feleségül.
\par 17 Ha annak atyja nem akarja azt néki adni: annyi pénzt adjon, a mennyi a hajadonok jegyajándéka.
\par 18 Varázsló asszonyt ne hagyj életben.
\par 19 A ki barommal közösül, halállal lakoljon.
\par 20 A ki isteneknek áldozik, nem csupán az Úrnak, megölettessék.
\par 21 A jövevényen ne hatalmaskodjál, és ne nyomorgasd azt, mert jövevények voltatok Égyiptom földén.
\par 22 Egy özvegyet vagy árvát se nyomorítsatok meg.
\par 23 Ha nyomorgatod azt, és hozzám kiált, meghallgatom az õ kiáltását.
\par 24 És felgerjed haragom, és megöllek titeket fegyverrel, és a ti feleségeitek özvegyekké lésznek, a ti fiaitok pedig árvákká.
\par 25 Ha pénzt adsz kölcsön az én népemnek, a szegénynek a ki veled van; ne légy hozzá olyan, mint a hitelezõ; ne vessetek reá uzsorát.
\par 26 Ha zálogba veszed a te felebarátod felsõ ruháját: naplemente elõtt visszaadd azt néki:
\par 27 Mert egyetlen takarója, testének ruhája az: miben háljon? Bizony, ha én hozzám kiált, meghallgatom; mert én irgalmas vagyok.
\par 28 A bírákat ne szidalmazd, és néped fejedelmét ne átkozd.
\par 29 Gabonáddal és boroddal ne késlekedjél; fiaid elsõszülöttét nékem add.
\par 30 Hasonlóképen cselekedjél ökröddel, juhoddal; hét napig legyen az õ anyjával, a nyolczadik napon nékem add azt.
\par 31 És szent emberek legyetek én elõttem; a mezõn széttépett húst meg ne egyétek; az ebnek vessétek azt.

\chapter{23}

\par 1 Hazug hírt ne hordj; ne fogj kezet a gonoszszal, hogy hamis tanú ne légy.
\par 2 Ne indulj a sokaság után a gonoszra, és peres ügyben ne vallj a sokasággal tartva, annak elfordítására.
\par 3 Szegénynek se kedvezz az õ peres ügyében.
\par 4 Ha elõltalálod ellenséged eltévedt ökrét vagy szamarát: hajtsd vissza néki.
\par 5 Ha látod, hogy annak a szamara, a ki téged gyûlöl, a teher alatt fekszik, vigyázz, rajta ne hagyd; oldd le azt õ vele együtt.
\par 6 A te szegényednek igazságát el ne fordítsd az õ perében.
\par 7 A hazug beszédtõl távol tartsd magad, és az ártatlant s az igazat meg ne öld; mert én nem adok igazat a gonosznak.
\par 8 Ajándékot el ne végy: mert az ajándék megvakítja a szemeseket, és elfordítja az igazak ügyét.
\par 9 A jövevényt ne nyomorgasd; hiszen ti ismeritek a jövevény  életét, mivelhogy jövevények voltatok Égyiptom földén.
\par 10 Hat esztendeig vesd be a te földedet és takard be annak termését;
\par 11 A hetedikben pedig pihentesd azt, és hagyd úgy, hogy egyék meg a te néped szegényei; a mi pedig ezektõl megmarad, egye meg a mezei vad. E képen cselekedjél szõlõddel és olajfáddal is.
\par 12 Hat napon át végezd dolgaidat, a hetedik napon pedig nyúgodjál, hogy nyúgodjék a te ökröd és szamarad, és megpihenjen  a te szolgálód fia és a jövevény.
\par 13 Mindazt, a mit néktek mondtam, megtartsátok, és idegen istenek nevét ne emlegessétek; ne hallassék az a te szádból.
\par 14 Háromszor szentelj nékem innepet esztendõnként.
\par 15 A kovásztalan kenyér innepét tartsd meg; hét nap egyél kovásztalan kenyeret, a mint megparancsoltam néked, az Abib hónap ideje alatt; mert akkor jöttél ki Égyiptomból: és  üres kézzel senki se jelenjék meg színem elõtt.
\par 16 És az aratás innepét, munkád zsengéjét, a melyet elvetettél a mezõn; és a takarodás innepét az esztendõ végén, a mikor termésedet betakarítod a mezõrõl.
\par 17 Esztendõnként háromszor jelenjék meg minden férfiad az Úr Isten színe elõtt.
\par 18 Ne ontsd ki az én áldozatom vérét kovászos kenyér mellett, és ünnepi áldozatom kövére meg ne maradjon reggelig.
\par 19 A te földed zsengéjének elsejét vidd el a te Uradnak Istenednek házába. Ne fõzd meg a gödölyét az õ anyjának tejében.
\par 20 Ímé én Angyalt bocsátok el te elõtted, hogy megõrízzen téged az útban, és bevigyen téged arra a helyre, a melyet elkészítettem.
\par 21 Vigyázz magadra elõtte, és hallgass az õ szavára; meg ne bosszantsd õt, mert nem szenvedi el a ti gonoszságaitokat; mert az én nevem van õ benne.
\par 22 Mert ha hallgatándasz az õ szavára; és mindazt megcselekedénded, a mit mondok: akkor ellensége lészek a te ellenségeidnek, és szorongatom a te szorongatóidat.
\par 23 Mert az én Angyalom te elõtted megyen és beviszen téged az Emoreusok, Khitteusok, Perizeusok, Kananeusok, Khivveusok és Jebuzeusok közé, és kiirtom azokat.
\par 24 Ne imádd azoknak isteneit és ne tiszteld azokat, és ne cselekedjél az õ cselekedeteik szerint; hanem inkább döntögesd le azokat és tördeld össze bálványaikat.
\par 25 És szolgáljátok az Urat a ti Istenteket, akkor megáldja a te kenyeredet és vizedet; és eltávolítom ti közûletek a nyavalyát.
\par 26 El sem vetél, meddõ sem lesz a te földeden semmi; napjaid számát teljessé teszem.
\par 27 Az én rettentésemet bocsátom el elõtted, és minden népet megrettentek, a mely közé mégy, és minden ellenségedet elfutamtatom elõtted.
\par 28 Darazsat is bocsátok el elõtted és kiûzi elõled a Khivveust, Kananeust és Khitteust.
\par 29 De nem egy esztendõben ûzöm õt ki elõled, hogy a föld pusztává ne legyen, és meg ne sokasodjék ellened a mezei vad.
\par 30 Lassan-lassan ûzöm õt ki elõled, míg megszaporodol és bírhatod a földet.
\par 31 És határodat a veres tengertõl a Filiszteusok tengeréig vetem és a pusztától fogva a folyóvízig; mert kezeitekbe adom annak a földnek lakosait, és kiûzöd azokat elõled.
\par 32 Ne köss szövetséget se azokkal, se az õ isteneikkel.
\par 33 Ne lakjanak a te földeden, hogy bûnbe ne ejtsenek téged ellenem: mert ha az õ isteneiket szolgálnád, vesztedre lenne az néked.

\chapter{24}

\par 1 És monda Mózesnek: Jõjj fel az Úrhoz te és Áron, Nádáb és Abihu, és az Izráel vénei közül hetvenen, és hajtsátok meg magatokat elõtte távolról.
\par 2 És csak Mózes közeledjék az Úrhoz, amazok pedig ne közeledjenek, és a nép se jõjjön fel vele.
\par 3 Elméne azért Mózes, és elbeszélé a népnek az Úr minden beszédét és minden rendelését; az egész nép pedig egyezõ szóval felele, mondván: Mindazokat a dolgokat, a melyeket az Úr parancsolt, megcselekeszszük.
\par 4 Mózes pedig felírá az Úrnak minden beszédét, és felkele reggel és oltárt építe a hegy alatt, és tizenkét oszlopot, az Izráel tizenkét nemzetsége szerint.
\par 5 Azután elküldé az Izráel fiainak ifjait, és áldozának égõ áldozatokat, és hálaáldozaul tulkokat ölének az Úrnak.
\par 6 Mózes pedig vevé a vérnek felét, és tölté a medenczékbe: a vérnek másik felét pedig az oltárra hinté.
\par 7 Azután vevé a szövetség könyvét, és elolvasá a nép hallatára; azok pedig mondának: Mindent megteszünk, a mit az Úr parancsolt, és engedelmeskedünk.
\par 8 Mózes pedig vevé a vért, és ráhinté a népre, és monda: Ímé a szövetségnek vére, melyet az Úr kötött ti veletek, mindama beszédek szerint.
\par 9 Azután felméne Mózes és Áron, Nádáb és Abihu, és az Izráel vénei közül hetvenen;
\par 10 És láták az Izráel Istenét, és annak lábai alatt valami zafir fényû tárgy vala, és olyan tiszta, mint maga az ég.
\par 11 És Izráel fiainak e választottjaira nem bocsátá kezét: jóllehet látták az Istent, mindazáltal ettek és ittak is.
\par 12 És szóla az Úr Mózesnek: Jõjj fel én hozzám a hegyre és maradj ott. És átadom néked a kõtáblákat, és a törvényt és a parancsolatot, a melyeket írtam, hogy azokra megtaníttassanak.
\par 13 Felkele azért Mózes és az õ szolgája Józsué, és felméne Mózes az Isten hegyére.
\par 14 A véneknek pedig monda: Várjatok itt reánk, míg visszatérünk hozzátok: Ímé Áron és Húr veletek vannak; a kinek valami ügye van, õ hozzájok menjen.
\par 15 Akkor felméne Mózes a hegyre; és felhõ borítá el a hegyet.
\par 16 És az Úr dicsõsége szálla alá a Sinai hegyre, és felhõ borítá azt hat napon át; a hetedik napon pedig szólítá Mózest a felhõ közepébõl.
\par 17 Az Úr dicsõségének jelensége pedig olyan vala az Izráel fiainak szeme elõtt, mint emésztõ tûz, a hegy tetején.
\par 18 És beméne Mózes a felhõ közepébe, és felméne a hegyre, és negyven nap és negyven éjjel vala Mózes a hegyen.

\chapter{25}

\par 1 És szóla az Úr Mózeshez, mondván:
\par 2 Szólj az Izráel fiainak, hogy szedjenek nékem ajándékokat; minden embertõl, a kit szíve hajt arra, szedjetek nékem ajándékokat.
\par 3 Ez pedig az az ajándék, a mit tõlök szedjetek: arany és ezüst és réz.
\par 4 És kék, és bíborpiros, és karmazsinszinû fonal, meg len fonal, és kecskeszõr.
\par 5 És veresre festett kosbõrök, és borzbõrök, és sittim-fa.
\par 6 Mécsbe való olaj, kenet-olajhoz való arómák, és füstöléshez való fûszerek.
\par 7 Ónix-kövek és foglalni való kövek, az efódhoz és a hósenhez.
\par 8 És készítsenek nékem szent hajlékot, hogy õ közöttök lakozzam.
\par 9 Mindenestõl úgy csináljátok, a mint én megmutatom néked a hajléknak formáját, és annak minden edényeinek formáját.
\par 10 És csináljanak egy ládát sittim-fából; harmadfél sing hosszút, másfél sing széleset, és másfél sing magasat.
\par 11 Borítsd meg azt tiszta aranynyal, belõl is kivül is megborítsd azt, és csinálj reá köröskörûl arany pártázatot.
\par 12 És önts ahhoz négy arany karikát, és illeszd azokat a négy szegeletére; az egyik oldalára is két karikát, a másik oldalára is két karikát.
\par 13 Csinálj rúdakat is sittim-fából, és azokat is megborítsd arannyal.
\par 14 És a rúdakat dugd a láda oldalain lévõ karikákba, hogy azokon hordozzák a ládát.
\par 15 A rúdak álljanak a láda karikáiban; ne vegyék ki azokból.
\par 16 És a bizonyságot, a melyet néked adok, tedd a ládába.
\par 17 Csinálj fedelet is tiszta aranyból: harmadfél sing hosszút és másfél sing széleset.
\par 18 Csinálj két Kérubot is aranyból, vert aranyból csináld azokat a fedélnek két végére.
\par 19 Az egyik Kérubot csináld az egyik végére innen, a másik Kérubot a másik végére onnan: a fedélbõl csináljátok ki a Kérubokat annak két végén.
\par 20 A Kérubok pedig terjeszszék ki szárnyaikat fölfelé, betakarva szárnyaikkal a fedelet; arczaik egymásfelé legyenek; a Kérubok arczai a fedél felé forduljanak.
\par 21 A fedelet pedig helyezd a ládára felül, a ládába pedig tedd a bizonyságot, a melyet adok néked.
\par 22 Ott jelenek meg néked és szólok hozzád a fedél tetejérõl, a két Kérub közül, melyek a bizonyság ládája felett vannak, mindazokról, a miket általad parancsolok az Izráel fiainak.
\par 23 Csinálj asztalt is sittim-fából, két sing hosszút, egy sing széleset, és mésfél sing magasat.
\par 24 És borítsd be azt tiszta aranynyal, és csinálj reá köröskörül arany pártázatot.
\par 25 Csinálj reá köröskörûl egy tenyérnyi karájt, karajára pedig csinálj köröskörûl arany pártázatot.
\par 26 Négy arany karikát csinálj hozzá, és illeszd a karikákat a négy lábának négy szegletére.
\par 27 A karáj mellett legyenek a karikák rúdtartókul, hogy hordozhassák az asztalt.
\par 28 Azokat a rúdakat is sittim-fából csináld és aranynyal borítsd be, és azokon hordozzák az asztalt.
\par 29 Készítsd el tálait is, csészéit is, kancsóit is, kelyheit is, a melyekkel italáldozatot áldoznak; tiszta aranyból csináld azokat.
\par 30 És tégy az asztalra szent kenyeret, mely mindenkor elõttem legyen.
\par 31 Csinálj gyertyatartót is tiszta aranyból; ver aranyból készüljön a gyertyatartó; annak szára, ága csészéi, gombjai és virágai ugyanabból legyenek.
\par 32 Hat ág jõjjön ki oldalaiból; három gyertyatartó-ág az egyik oldalból, és három gyertyatartó-ág a másik oldalból.
\par 33 Mandolavirág formájú három csésze az egyik ágon, gombbal és virággal; és mandolavirág formájú három csésze a másik ágon is, gombbal és virággal; így legyen a gyertyatartóból kijövõ mind a hat ágon.
\par 34 A gyertyatartón pedig négy mandolvirág formájú csésze legyen, gombjaival és virágaival.
\par 35 Gomb legyen a belõle kijövõ két ág alatt; ismét gomb a belõle kijövõ két ág alatt, és ismét gomb a belõle kijövõ két ág alatt: így a gyertyatartóból kijövõ mind a hat ág alatt.
\par 36 Gombjaik és ágaik magából legyenek; egy darab tiszta aranyból legyen verve az egész.
\par 37 Csinálj hozzá hét mécset is, és úgy rakják fel mécseit, hogy elõre világítsanak.
\par 38 Hamvvevõi és hamutartói is tiszta aranyból legyenek.
\par 39 Egy tálentom tiszta aranyból csinálják azt, mindezeket az eszközöket.
\par 40 Vigyázz, hogy arra a formára csináld, a mely a hegyen mutattatott néked.

\chapter{26}

\par 1 A hajlékot pedig tíz kárpitból csináld: sodrott lenbõl, és kék, és bíborpiros és karmazsin színûbõl, Kérubokkal, mestermunkával készítsd azokat.
\par 2 Egy-egy kárpit hossza huszonnyolcz sing legyen, egy-egy kárpit szélessége pedig négy sing; egy mértéke legyen mindenik kárpitnak.
\par 3 Öt kárpit legyen egymással egybefoglalva, ismét öt kárpit egymással egybefoglalva.
\par 4 És csinálj hurkokat kék lenbõl az egyik kárpit szélén, a mely az egybefoglaltak között szélrõl van; ugyanezt csináld a szélsõ kárpit szélével a másik egybefoglalásban is.
\par 5 Ötven hurkot csinálj az egyik kárpiton; ötven hurkot csinálj ama kárpit szélén is, a mely a másik egybefoglalásban van; egyik hurok a másiknak általellenében legyen.
\par 6 Csinálj ötven arany horgocskát is, és e horgocskákkal foglald össze egyik kárpitot a másikkal, hogy a hajlék egygyé legyen.
\par 7 Ezután csinálj kecskeszõr kárpitokat sátorul a hajlék fölé; tizenegy kárpitot csinálj ilyet.
\par 8 Egy kárpit hossza harmincz sing legyen, szélessége pedig egy kárpitnak négy sing; egy mértéke legyen a tizenegy kárpitnak.
\par 9 És foglald egybe az öt kárpitot külön, és a hat kárpitot külön; a hatodik kárpitot pedig kétrét hajtsd a sátor elejére.
\par 10 És csinálj ötven hurkot az egyik kárpit szélén, a mely az egybefoglaltak között szélrõl van; és ötven hurkot a kárpit szélén a másik egybefoglalásban is.
\par 11 Csinálj ötven rézhorgocskát is, és akaszd a horgocskákat a hurkokba, és foglald egybe a sátort, hogy egygyé legyen.
\par 12 A sátor kárpitjának fölösleges része, a fölösleges kárpitnak fele csüggjön alá a hajlék hátulján.
\par 13 Egy singnyi pedig egyfelõl, és egy singnyi másfelõl, abból, a mi a sátor kárpitjainak hosszában fölösleges, bocsáttassék alá a hajlékok oldalain egyfelõl is, másfelõl is, hogy befedje azt.
\par 14 Csinálj a sátornak takarót is veresre festett kosbõrökbõl, és e fölé is egy takarót borzbõrökbõl.
\par 15 Csinálj a hajlékhoz deszkákat is sittim-fából, felállogatva.
\par 16 A deszka hossza tíz sing legyen; egy-egy deszka szélessége pedig másfél sing.
\par 17 Egy-egy deszkának két csapja legyen, egyik a másiknak megfelelõ; így csináld a hajlék minden deszkáját.
\par 18 A deszkákat pedig így csináld a hajlékhoz: húsz deszkát a déli oldalra, délfelé.
\par 19 A húsz deszka alá pedig negyven ezüst talpat csinálj, két talpat egy-egy deszka alá, annak két csapjához képest; megint két talpat egy-egy deszka alá, a két csapjához képest.
\par 20 A hajlék másik oldalául is, észak felõl húsz deszkát.
\par 21 És azokhoz is negyven ezüst talpat; két talpat egy deszka alá, megint két talpat egy deszka alá.
\par 22 A hajlék nyugoti oldalául pedig csinálj hat deszkát.
\par 23 A hajlék szegleteiül is csinálj két deszkát a két oldalon.
\par 24 Kettõsen legyenek alólról kezdve, felül pedig együtt legyenek egy karikába foglalva; ilyen legyen mindkettõ; a két szeglet számára legyenek.
\par 25 Legyen azért nyolcz deszka, és azokhoz tizenhat ezüst talp; két talp egy deszka alatt, megint két talp egy deszka alatt.
\par 26 Csinálj reteszrúdakat is sittim-fából; ötöt a hajlék egyik oldalának deszkáihoz.
\par 27 És öt reteszrúdat a hajlék másik oldalának deszkáihoz; és a hajlék nyugoti oldalának deszkáihoz is öt reteszrúdat hátulról.
\par 28 A középsõ reteszrúd pedig a deszkák közepén az egyik végtõl a másik végig érjen.
\par 29 A deszkákat pedig borítsd meg aranynyal, és karikákat is aranyból csinálj azokhoz a reteszrúdak tartói gyanánt; reteszrúdakat is megborítsd aranynyal.
\par 30 A hajlékot pedig azon a módon állítsd fel a mint néked a hegyen mutattatott.
\par 31 És csinálj függönyt, kék, és bíborpiros, és karmazsinszínû, és sodrott lenbõl; Kérubokkal, mestermunkával készítsétek azt.
\par 32 És tedd azt sittim-fából való, aranynyal borított négy oszlopra, a melyeknek horgai aranyból legyenek, négy ezüst talpon.
\par 33 És tedd a függönyt a horgok alá, és vidd oda a függöny mögé a bizonyság ládáját és az a függöny válaszsza el néktek a szent helyet a szentek szentjétõl.
\par 34 Azután tedd rá a fedelet a bizonyság ládájára a szentek szentjébe.
\par 35 Az asztalt pedig helyezd a függönyön kívül, és a gyertyatartót az asztal ellenébe, a hajlék déli oldalába; az asztalt pedig tedd az északi oldalba.
\par 36 És csinálj leplet a sátor nyilására is, kék, és bíborpiros, és karmazsinszínû, és sodrott lenbõl, hímzõmunkával.
\par 37 A lepelhez pedig csinálj öt oszlopot sittim-fából, és borítsd meg azokat aranynyal; azoknak horgai aranyból legyenek, és önts azokhoz öt réztalpat.

\chapter{27}

\par 1 És csináld az oltárt sittimfából, öt sing a hossza, és öt sing a szélessége; négyszögû legyen az oltár, és magassága három sing.
\par 2 Szarvakat is csinálj a négy szegletére, szarvai magából legyenek, és borítsd meg azt rézzel.
\par 3 Csinálj ahhoz fazekakat is a hamujának, hozzá való lapátokat, medenczéket, villákat és serpenyõket; minden hozzá való edényt rézbõl csinálj.
\par 4 Csinálj ahhoz háló forma rostélyt is rézbõl, és csinálj a hálóra négy rézkarikát, a négy szegletére.
\par 5 És tedd azt az oltár párkányzata alá alulról, úgy hogy a háló az oltár közepéig érjen.
\par 6 Csinálj az oltárhoz rúdakat is, sittim-fa rúdakat, és borítsd meg azokat rézzel.
\par 7 És dugják a rúdakat a karikákba, és legyenek azok a rúdak az oltár két oldalán, mikor azt hordozzák.
\par 8 Üresre csináld azt, deszkákból; a mint néked a hegyen mutattatott, úgy csinálják.
\par 9 Pitvart is csinálj a hajléknak; a déli oldalon a pitvarhoz szõnyegeket, sodrott lenbõl; száz sing legyen a hossza egy oldalnak.
\par 10 Ahhoz húsz oszlopot, húsz réz talpat, és az oszlopok horgait, és átalkötõit ezüstbõl.
\par 11 Épen úgy az északi oldalon, hoszszában, száz sing hosszú szõnyeget; ahhoz húsz oszlopot, ezekhez húsz réztalpat, és az oszlopok horgait és átalkötõit ezüstbõl.
\par 12 A pitvar szélességéhez pedig nyugot felõl ötven sing szõnyeget; ahhoz tíz oszlopot és ezekhez tíz talpat.
\par 13 A pitvarnak szélessége kelet felõl is ötven sing.
\par 14 És tizenöt sing szõnyeg legyen egyik oldalról, ehhez három oszlop, ezekhez pedig három talp.
\par 15 A másik oldalról is tizenöt sing szõnyeg, ehhez három oszlop és ezekhez három talp.
\par 16 A pitvar kapujában pedig húsz sing lepel legyen, kék, és bíborpiros, és karmazsinszínû és sodrott lenbõl, himzõmunkával; ehhez négy oszlop, ezekhez pedig négy talp.
\par 17 A pitvar minden oszlopát ezüst átalkötõk vegyék körûl, horgaik ezüstbõl, talpaik pedig rézbõl legyenek.
\par 18 A pitvar hossza száz sing, és szélessége ötven-ötven, magassága pedig öt sing; leplei sodrott lenbõl, talpai pedig rézbõl legyenek.
\par 19 A hajlék minden edénye, az ahhoz való mindennemû szolgálatban, és minden szege, és a pitvarnak is minden szege rézbõl legyen.
\par 20 Te pedig parancsold meg az Izráel fiainak, hogy hozzanak néked tiszta faolajat a melyet a világításhoz sajtoltak, hogy szünet nélkül égõ lámpát gyujthassanak.
\par 21 A gyülekezet sátorában a függönyön kivül, a mely a bizonyság elõtt van, készítsék el azt Áron és az õ fiai, estvétõl fogva reggelig az Úr elõtt. Örök rendtartás legyen ez az õ nemzetségöknél Izráel fiai között.

\chapter{28}

\par 1 Te pedig hívasd magadhoz a te atyádfiát Áront, és az õ fiait õ vele az Izráel fiai közûl, hogy papjaim legyenek: Áron, Nádáb, Abihu, Eleázár, Ithamár, Áronnak fiai.
\par 2 És csinálj szent ruhákat Áronnak a te atyádfiának, dicsõségére és ékességére.
\par 3 És szólj minden bölcs szívûeknek, a kiket betöltöttem a bölcseség lelkével, hogy csinálják meg az Áron ruháit, az õ felszentelésére, hogy papom legyen.
\par 4 Ezek pedig a ruhák, a melyeket készítsenek: hósen, efód, palást, koczkás köntös, süveg és öv. És csináljanak szent ruhákat Áronnak a te atyádfiának, és az õ fiainak, hogy papjaim legyenek.
\par 5 Vegyék hát õk elõ az aranyat, és a kék, és a bíborpiros, és a karmazsinszínû fonalat és a lenfonalat.
\par 6 És csinálják az efódot aranyból, kék és bíborpiros, karmazsinszínû és sodrott lenbõl, mestermunkával.
\par 7 Két vállkötõ is legyen hozzá kapcsolva a két végéhez, hogy összekapcsoltathassék.
\par 8 Átkötõ öve pedig, a mely rajta van, ugyanolyan mívû és abból való legyen; aranyból, kék, és bíborpiros, és karmazsinszinû, és sodrott lenbõl.
\par 9 Annakutána végy két ónix-követ, és mesd fel azokra az Izráel fiainak neveit.
\par 10 Hatnak nevét az egyik kõre, a másik hatnak nevét pedig a másik kõre, az õ születésök szerint.
\par 11 Kõmetszõ munkával, a mint a pecsétet metszik, úgy metszesd e két követ az Izráel fiainak neveére; köröskörûl arany boglárokba csináld azokat.
\par 12 És tedd e két követ az efód vállkötõire, az Izráel fiaira való emlékeztetés kövei gyanánt, hogy emlékeztetõül hordozza Áron azoknak neveit az õ két vállán az Úr elõtt.
\par 13 Csinálj annakokáért arany boglárokat,
\par 14 És két lánczot tiszta aranyból; fonatékosan csináld azokat; sodrott mívûek legyenek, és tedd rá a sodrott lánczokat a boglárokra.
\par 15 Azután csináld meg az ítéletnek hósenét mestermunkával; úgy csináld mint az efódot csináltad: aranyból, kék, és bíborpiros, és karmazsinszínû, és sodrott lenbõl csináld azt.
\par 16 Négyszögû legyen, kétrétû, egy arasznyi hosszú és egy arasznyi széles.
\par 17 És foglalj abba befoglalni való köveket; négy sor követ, ilyen sorban: szárdiusz, topáz és smaragd; ez az elsõ sor.
\par 18 A második sor pedig: karbunkulus, zafir és gyémánt.
\par 19 A harmadik sor: jáczint, agát és amethiszt.
\par 20 A negyedik sor: krizolith, ónix és jáspis; arany boglárokba legyenek foglalva.
\par 21 A kövek tehát az Izráel fiainak nevei szerint legyenek, tizenkettõ legyen az õ nevök szerint; mint a pecsét, úgy legyen metszve, mindenik a reá való névvel, a tizenkét nemzetség szerint.
\par 22 A hósenre pedig csinálj fonatékos lánczokat, sodrott mívûeket, tiszta aranyból.
\par 23 És csinálj a hósenre két arany karikát, és tedd a két karikát a hósen két szegletére.
\par 24 És a két arany fonatékot fûzd a hósen két szegletén levõ karikákba.
\par 25 A két fonatéknak két végét pedig foglald a két boglárhoz, és tûzd az efódnak vállkötõihez, annak elõrészére.
\par 26 Csinálj még két arany karikát, és tedd azokat a hósen két szegletére, azon a szélen, amely befelé van az efód felõl.
\par 27 És csinálj még két arany karikát, és tedd azokat az efód két vállkötõjére alól, annak elõrésze felõl egybefoglalásához közel, az efód öve felett.
\par 28 És csatolják a hósent az õ karikáinál fogva az efód karikáihoz, kék zsinórral, hogy az efód öve felett legyen, és el ne váljék a hósen az efódtól.
\par 29 És viselje Áron az Izráel fiainak neveit az ítélet hósenén, az õ szíve felett, a mikor bemegy a szenthelyre, emlékeztetõûl az Úr elõtt szüntelen.
\par 30 Azután tedd az ítéletnek hósenébe az Urimot és Thummimot, hogy legyenek azok az Áron szíve felett, a mikor bemegy az Úr eleibe, és hordozza Áron az Izráel fiainak ítéletét az õ szívén az Úr színe elõtt szüntelen.
\par 31 És csináld az efód palástját egészen kék lenbõl.
\par 32 Közepén legyen nyílás a fejének; a nyílásnak szegése legyen köröskörûl, takácsmunka, olyan legyen, mint a pánczél nyílása, hogy el ne szakadjon.
\par 33 És ennek alsó peremére csinálj gránátalmákat kék, és bíborpiros, és karmazsinszínû lenbõl, a peremére köröskörûl, és ezek közé arany csengettyûket is köröskörûl.
\par 34 Arany csengettyû, meg gránátalma, arany csengettyû, meg gránátalma legyen a palást peremén köröskörül.
\par 35 És legyen az Áronon, a mikor szolgál, hogy hallassék annak csengése, a mikor bemegy a szenthelybe az Úr eleibe, és mikor kijön, hogy meg ne haljon.
\par 36 Csinálj egy lapot is tiszta aranyból, és mesd ki arra, mint a pecsétet metszik: Szentség az Úrnak.
\par 37 És kösd azt kék zsinórra, hogy legyen az a süvegen; a süvegnek elõrészén legyen az.
\par 38 És legyen az az Áronnak homlokán, hogy Áron viselje a szent áldozatok körûl elkövetett vétket, a melyeket az Izráel fiai mindenféle szent adományaikban szentelnek. Legyen azért szüntelen a homlokán, hogy kedvesekké tegye õket az Úr elõtt.
\par 39 A lenköntöst pedig koczkásan készítsd, és a süveget lenbõl csináld, az övét meg hímzõ munkával készítsd.
\par 40 Az Áron fiainak csinálj köntösöket, és csinálj nékik öveket is, meg süvegeket is csinálj nékik, dicsõségökre, és ékességökre.
\par 41 És öldöztesd fel azokba Áront a te atyádfiát, és az õ fiait vele együtt, és kend fel õket, iktasd be õket tisztjökbe, és szenteld fel õket, hogy papjaimmá legyenek.
\par 42 Csinálj nékik lábravalókat is gyolcsból, hogy befödjék azoknak mezítelen testét, és az ágyéktól a tomporig érjenek.
\par 43 És legyenek azok Áronon és az õ fiain, a mikor bemennek a gyülekezet sátorába, vagy a mikor az oltárhoz járulnak, a szenthelyen való szolgálattételre, hogy bûnt ne vigyenek oda és meg ne haljanak. Örökkévaló rendtartás ez Áronnak és az õ magvának õ utána.

\chapter{29}

\par 1 Ez az, a mit õ velök cselekedjél, az õ felszentelésökre, hogy az én papjaim legyenek: Végy egy tulkot, fiatal marhát, és hiba nélkül való két kost.
\par 2 És kovásztalan kenyeret, és olajjal elegyített kovásztalan kalácsokat, meg olajjal kent kovásztalan lepényeket is; búzalisztlángból készítsd azokat.
\par 3 És tedd azokat egy kosárba, és vidd fel azokat a kosárban, a tulokkal és a két kossal együtt.
\par 4 Áront pedig és az õ fiait állítsd a gyülekezet sátorának ajtaja elé, és mosd meg õket vízzel.
\par 5 És vedd a ruhákat, és öltöztesd fel Áront a köntösbe, az efódhoz való palástba, és az efódba, meg a hósenbe és övezd fel õt az efód övével.
\par 6 Tedd a süveget is fejére, és a szent koronát tedd a süvegre.
\par 7 És vedd a kenetnek olaját, és töltsd az õ fejére, így kend fel õt.
\par 8 Fiait is állítsd elõ, és öltöztesd fel õket a köntösökbe.
\par 9 És övezd körûl õket övvel, Áront és az õ fiait, és tégy a fejökre süvegeket is, hogy övék legyen a papság örök rendelés szerint. Így iktasd be tisztökbe Áront és fiait.
\par 10 Azután állíttasd a tulkot a gyülekezet sátora elé, és Áron és az õ fiai tegyék kezeiket a tulok fejére.
\par 11 És vágd le a tulkot az Úr elõtt a gyülekezet sátorának ajtajánál.
\par 12 Azután végy a tulok vérébõl, és hintsd az ujjaiddal az oltár szarvaira; a többi vért pedig töltsd mind az oltár aljára.
\par 13 És vedd a kövérébõl mindazt, a mi a belet fedi, és a májon lévõ hártyát, és a két vesét a rajtok lévõ kövérrel együtt, és füstölögtesd el az oltáron.
\par 14 A tulok húsát, bõrét és ganéját pedig égesd el a táboron kívül: bûnért való áldozat az.
\par 15 Vedd az egyik kost is, és Áron és az õ fiai tegyék kezeiket a kos fejére.
\par 16 Azután vágd le a kost, és vedd annak vérét, és hintsd azt az oltárra köröskörûl.
\par 17 A kost pedig vagdald tagjaira, és mosd meg a belét és lábszárait, és tedd rá tagjaira és fejére.
\par 18 Azután füstölögtesd el az egész kost az oltáron: égõáldozat az az Úrnak, kedves illatú tûzáldozat az Úrnak.
\par 19 Vedd a másik kost is, és Áron és az õ fiai tegyék kezeiket a kos fejére.
\par 20 Azután vágd le a kost, és végy annak vérébõl, és hintsd meg azzal Áron füle czimpáját és az õ fiai jobb fülének czimpáját, és az õ jobb kezök hüvelykét és jobb lábok hövelykét; a többi vért pedig hintsd az oltárra köröskörül.
\par 21 Azután végy a vérbõl, mely az oltáron van, és a kenetnek olajából, és hintsd Áronra és az õ ruháira, s vele együtt az õ fiaira és az õ fiainak ruháira, hogy szent legyen õ és az õ ruhái, s vele együtt az õ fiai és az õ fiainak ruhái.
\par 22 Azután vedd a kosból a kövérét, a farkát, s a belet borító kövéret, meg a máj hártyáját, meg a két veséjét a rajtok levõ kövérével, és a jobb lapoczkát; mert felavatási kos ez.
\par 23 Meg egy kenyeret és egy olajos kalácsot és egy lepényt a kovásztalan kenyérnek kosarából, mely az Úr elõtt van;
\par 24 És rakd mindezeket az Áron kezeire és az Áron fiainak kezeire, és lóbáltasd meg azokat az Úr elõtt.
\par 25 Azután vedd le azokat kezükrõl, és füstölögtesd el az oltáron az égõáldozat felett kedves illatul az Úr elõtt; az Úrnak tûzáldozata ez.
\par 26 Vedd az Áron felavatási kosának szegyét is, és lóbbáld meg azt az Úr elõtt; azután legyen az a te részed.
\par 27 Így szenteld meg a meglóbbált szegyet és a felemelt lapoczkát, a melyet meglóbbáltak és a melyet felemeltek, Áronnak és az õ fiainak felavatási kosából.
\par 28 És legyen ez Áronnak és az õ fiainak része örökké az Izráel fiaitól: mert felmutatott adomány ez, és felmutatott adomány legyen Izráel fiai részérõl az õ hálaáldozataikból; az Úrnak felmutatott adomány.
\par 29 A szent öltözetek pedig, a melyek az Áronéi, legyenek õ utána az õ fiaié, hogy azokban kenettessenek fel, és azokban állíttassanak tisztökbe.
\par 30 Hét napon öltözzék azokba, a ki az õ fiai közül õ utána pap lesz, a ki bemenendõ lesz a gyülekezet sátorába, hogy a szent helyen szolgáljon.
\par 31 A felavatási kost pedig vedd, és fõzd meg annak húsát szent helyen.
\par 32 És a kosnak húsát és a kenyeret, mely a kosárban van, a gyülekezet sátorának ajtajánál egye meg Áron és az õ fiai.
\par 33 Õk egyék meg azokat, a mik által az engesztelés történt, hogy tisztökbe állíttassanak és felszenteltessenek. De idegen ne egyék azokból, mert szentek azok.
\par 34 Ha pedig valami megmarad az avatási húsból vagy a kenyérbõl reggelig, tûzzel égesd meg a maradékot; meg ne egyék, mert szent az.
\par 35 Áronnak tehát és az õ fiaival akképen cselekedjél, a mint megparancsoltam néked; hét napon át állítsd õket tisztökbe.
\par 36 És naponként készítsd bûnáldozati tulkot engesztelésül, és tisztítsd meg az oltárt, mikor engesztelõ áldozatot végzesz rajta, és kend meg azt, hogy megszenteltessék.
\par 37 Hét napon tégy engesztelõ áldozatot az oltáron; és szenteld meg azt, hogy felette igen szentséges legyen az oltár. Valami illeti az oltárt, szent legyen.
\par 38 Ez pedig az, a mit áldoznod kell az oltáron: Esztendõs két bárányt mindennap szüntelen.
\par 39 Az egyik bárányt reggel áldozd meg, a másik bárányt pedig áldozd meg estennen.
\par 40 És az egyik bárányhoz végy egy tized lisztlángot, egy negyed hin-nyi sajtot olajjal vegyítve; italáldozatul pedig egy negyed hin-nyi bort.
\par 41 A másik bárányt estennen áldozd meg, ugyanazzal az étel- és italáldozattal készítsd azt, mint reggel; kedves illatul, tûzáldozatul az Úrnak.
\par 42 Szüntelen égõáldozat legyen az a ti nemzetségeitek között a gyülekezet sátorának ajtajánál az Úr elõtt, a hol megjelenek néktek, hogy veled ott szóljak.
\par 43 Ott jelenek meg az Izráel fiainak, és megszenteltetik az én dicsõségem által.
\par 44 És megszentelem a gyülekezetnek sátorát és az oltárt; Áront és az õ fiait is megszentelem, hogy papjaim legyenek.
\par 45 És az Izráel fiai között lakozom, és nékik Istenök lészek.
\par 46 És megtudják, hogy én, az Úr vagyok az õ Istenök, a ki kihoztam õket Égyiptom földérõl, hogy közöttök lakozhassam, én, az Úr az õ Istenök.

\chapter{30}

\par 1 Csinálj oltárt a füstölõ szerek füstölgésére is, sittim-fából csináld azt.
\par 2 Egy sing hosszú, egy sing széles, négyszögû és két sing magas legyen, ugyanabból legyenek a szarvai is.
\par 3 És borítsd meg azt tiszta aranynyal, a tetejét és oldalait köröskörül, és szarvait is; arany pártázatot is csinálj hozzá köröskörül.
\par 4 Csinálj hozzá két arany karikát is, pártázata alá a két oldalán, mindkét oldalára csináld, hogy legyenek rúdtartókul, hogy azokon hordozzák azt.
\par 5 És a rúdakat csináld sittim-fából, és borítsd meg azokat aranynyal.
\par 6 És tedd azt a függöny elé, a mely a bizonyság ládája mellett, a bizonyság fedele elõtt van, a hol megjelenek néked.
\par 7 Áron pedig füstölögtessen rajta minden reggel jó illatú füstölõ szert; mikor a mécseket rendbe szedi, akkor füstölögtesse azt.
\par 8 És a mikor Áron estennen felrakja a mécseket, füstölögtesse azt. Szüntelen való illattétel legyen ez az Úr elõtt nemzetségrõl nemzetségre.
\par 9 Ne áldozzatok azon idegen füstölõszerekkel, se égõáldozattal, se ételáldozattal; italáldozatot se öntsetek reá.
\par 10 És egyszer egy esztendõben engesztelést végezzen Áron annak szarvainál az engesztelõ napi áldozat vérébõl; egy esztendõben egyszer végezzen engesztelést azon, nemzetségrõl nemzetségre. Szentségek szentsége ez az Úrnak.
\par 11 Azután szóla az Úr Mózesnek, mondván:
\par 12 Mikor Izráel fiait fejenként számba veszed, adja meg kiki életének váltságát az Úrnak az õ megszámláltatásakor, hogy csapás ne legyen rajtok az õ megszámláltatásuk miatt.
\par 13 Ezt adja mindaz, a ki átesik a számláláson: fél siklust a szent siklus szerint (egy siklus húsz gera); a siklusnak fele áldozat az Úrnak.
\par 14 Mindaz, a ki átesik a számláláson, húsz esztendõstõl fogva felfelé, adja meg az áldozatot az Úrnak.
\par 15 A gazdag ne adjon többet, és a szegény ne adjon kevesebbet fél siklusnál, a mikor megadják az áldozatot az Úrnak engesztelésül a ti lelketekért.
\par 16 És szedd be az engesztelési pénzt az Izráel fiaitól, és add azt a gyülekezet sátorának szolgálatjára, hogy az Izráel fiainak emlékezetéül legyen az az Úr elõtt, engesztelésül a ti lelketekért.
\par 17 Azután szóla az Úr Mózesnek, mondván:
\par 18 És csinálj rézmedenczét, lábát is rézbõl, mosakodásra; és tedd azt a gyülekezet sátora közé és az oltár közé, és tölts bele vizet;
\par 19 Hogy Áron és az õ fiai abból mossák meg kezeiket és lábaikat.
\par 20 A mikor a gyülekezet sátorába mennek, mosakodjanak meg vízben, hogy meg ne haljanak; vagy mikor az oltárhoz járulnak, hogy szolgáljanak és tûzáldozatot füstölögtessenek az Úrnak.
\par 21 Kezeiket is, lábaikat is mossák meg, hogy meg ne haljanak. És örökkévaló rendtartásuk lesz ez nékik, néki és az õ magvának nemzetségrõl nemzetségre.
\par 22 Ismét szóla az Úr Mózesnek, mondván:
\par 23 Te pedig végy drága fûszereket, híg mirhát, ötszáz siklusért, jóillatú fahéjat fél ennyit, kétszáz ötvenért, és illatos kalmust is kétszáz ötvenért.
\par 24 Kásiát pedig ötszázért, a szent siklus szerint, és egy hin faolajt.
\par 25 És csinálj abból szent kenetnek olaját, elegyített kenetet, a kenetkészítõk mestersége szerint. Legyen az szent kenõ olaj.
\par 26 És kend meg azzal a gyülekezet sátorát és a bizonyság ládáját.
\par 27 Az asztalt is és annak minden edényét, a gyertyatartót és annak edényeit, és a füstölõ oltárt.
\par 28 Az egészen égõáldozatnak oltárát is, és annak minden edényit, a mosdómedenczét és annak lábát.
\par 29 Így szenteld meg azokat, hogy szentségek szentségévé legyenek: Valami illeti azokat, szent legyen.
\par 30 Kend fel Áront is és az õ fiait is, így szenteld fel õket papjaimmá.
\par 31 Az Izráel fiainak pedig így szólj: Szent kenetnek olaja legyen ez nékem, a ti nemzetségeiteknél is.
\par 32 Ember testét azzal meg ne kenjék, se ahhoz hasonlót, annak mértékei szerint ne csináljatok: szent az; szent legyen elõttetek is.
\par 33 Valaki ahhoz hasonló kenetet csinál, vagy azzal idegent ken meg, kitöröltessék az õ népe közül.
\par 34 Monda ismét az Úr Mózesnek: Végy fûszereket, csepegõ gyantát, onyxot, galbánt, e fûszereket és tiszta temjént, egyenlõ mértékkel.
\par 35 És csinálj belõlök füstölõ szert, a fûszercsináló elegyítése szerint; tiszta és szent legyen az.
\par 36 És abból törj apróra, és tégy belõle a bizonyság ládája elé a gyülekezet sátorában, a hol megjelenek néked. Szentségek szentsége legyen ez elõttetek.
\par 37 És a füstölõ szer, a melyet készítesz, az Úrnak szentelt legyen elõtted; annak mértéke szerint magatoknak ne csináljatok.
\par 38 Mindaz, a ki hasonló füstölõt csinál ehhez, hogy azt illatoztassa, irtassék ki az õ népe közül.

\chapter{31}

\par 1 És szóla az Úr Mózesnek mondván:
\par 2 Ímé, név szerint meghívtam Bésaléelt, a Húr fiának Urinak fiát a Júda nemzetségébõl.
\par 3 És betöltöttem õt Istennek lelkével, bölcsességgel, értelemmel és tudománynyal minden mesterséghez.
\par 4 Hogy tudjon kigondolni mindent, a mit aranyból, ezüstbõl, rézbõl kell csinálni.
\par 5 És foglaló köveket metszeni, fát faragni, és mindenféle munkákat végezni.
\par 6 És ímé Aholiábot is, Akhiszamáknak fiát a Dán nemzetségébõl, mellé adtam; és adtam minden értelmesnek szivébe bölcseséget, hogy elkészítsék mind azt, a mit néked megparancsoltam.
\par 7 A gyülekezet sátorát, a bizonyság ládáját, a fölibe való fedelet, és a sátornak minden edényét.
\par 8 Az asztalt és annak edényit, a tiszta gyertyatartót, és minden hozzávalót, és a füstölõ oltárt.
\par 9 Az egészen égõáldozat oltárát és annak minden edényit, a mosdómedenczét és annak lábát.
\par 10 A szolgálathoz való ruhákat, Áron papnak szent öltözetit, és az õ fiainak öltözetit, a papi szolgálatra.
\par 11 A kenetnek olaját, és a jó illatú füstölõ szert a szentséghez. Mindent úgy csináljanak, a mint néked parancsoltam.
\par 12 Azután szóla az Úr Mózesnek, mondván:
\par 13 Te szólj az Izráel fiainak, mondván: Az én szombatimat bizony megtartsátok; mert jel az én közöttem és ti köztetek nemzetségrõl nemzetségre, hogy megtudjátok, hogy én vagyok az Úr, a ki titeket megszentellek.
\par 14 Megtartsátok azért a szombatot; mert szent az ti néktek. A ki azt megrontja halállal lakoljon. Mert valaki munkát végez azon, annak lelke írtassék ki az õ népe közül.
\par 15 Hat napon munkálkodjanak, a hetedik nap pedig nyugodalomnak szombatja az Úrnak szentelt nap: valaki szombatnapon munkálkodik, megölettessék.
\par 16 Megtartsák azért az Izráel fiai a szombatot, megszentelvén a szombatot nemzetségrõl nemzetségre, örök szövetségül.
\par 17 Legyen közöttem és az Izráel fiai között örök jel ez; mert hat napon teremtette az Úr a mennyet és a földet, hetednapon pedig megszünt és megnyugodott.
\par 18 Mikor pedig elvégezte vele való beszédét a Sinai hegyen, által adá Mózesnek a bizonyság két tábláját, az Isten ujjával írt kõtáblákat.

\chapter{32}

\par 1 Mikor látá a nép, hogy Mózes késik a hegyrõl leszállani, egybegyûle a nép Áron ellen és mondá néki: Kelj fel, csinálj nékünk isteneket, kik elõttünk járjanak; mert nem tudjuk mint lõn dolga ama férfiúnak Mózesnek, a ki minket Égyiptom földérõl kihozott.
\par 2 És monda nékik Áron: Szedjétek le az aranyfüggõket, a melyek feleségeitek, fiaitok és leányaitok fülein vannak, és hozzátok én hozzám.
\par 3 Leszedé azért mind az egész nép az aranyfüggõket füleirõl, és elvivék Áronhoz.
\par 4 És elvevé kezökbõl, és alakítá azt vésõvel; így csinála abból öntött borjút. És szóltak: Ezek a te isteneid Izráel, a kik kihoztak téged Égyiptom földérõl.
\par 5 Mikor látta ezt Áron, oltárt építe az elõtt, és kiálta Áron, mondván: Holnap az Úrnak innepe lesz!
\par 6 Felkelvén azért másnapon jó reggel, áldozának égõáldozattal és hálaáldozattal is; azután leüle a nép enni és inni; azután felkelének játszani.
\par 7 Szóla pedig az Úr Mózesnek: Eredj menj alá; mert megromlott a te néped, a melyet kihoztál Égyiptom földébõl.
\par 8 Hamar letértek az útról, a melyet parancsoltam nékik, borjúképet öntöttek magoknak, azt tisztelik és annak áldoznak, és azt mondják: Ezek a te isteneid Izráel, a kik téged kihoztak Égyiptom földébõl.
\par 9 Monda ismét az Úr Mózesnek: Látom ezt a népet, bizony keménynyakú nép.
\par 10 Azért hagyj békét nékem, hadd gerjedjen fel haragom ellenök, és törûljem el õket: Téged azonban nagy néppé teszlek.
\par 11 De Mózes esedezék az Úrnak, az õ Istenének színe elõtt, mondván: Miért gerjedne Uram a te haragod néped ellen, a melyet nagy erõvel és hatalmas kézzel hoztál vala ki Égyiptomnak földérõl?
\par 12 Miért mondanák az égyiptomiak, mondván: Vesztökre vivé ki õket, hogy elveszítse a hegyek között, és eltörülje õket a föld színérõl? Múljék el a te haragod tüze és hagyd abba azt a néped ellen való veszedelmet.
\par 13 Emlékezzél meg Ábrahámról, Izsákról és Izráelrõl a te szolgáidról, kiknek megesküdtél te magadra, mondván nékik: Megsokasítom a ti magotokat mint az égnek csillagait; és azt az egész földet, melyrõl szóltam, a ti magotoknak adom, és örökségül bírják azt örökké.
\par 14 És abba hagyá az Úr azt a veszedelmet, melyet akart vala bocsátani az õ népére.
\par 15 Megfordula azért és megindula Mózes a hegyrõl, kezében a bizonyság két táblája; mindkét oldalukon beírt táblák, mind egy felõl, mind más felõl beírva.
\par 16 A táblák pedig Isten kezének csinálmányai valának, az írás is Isten írása vala, kimetszve a táblákra.
\par 17 Józsué pedig hallván a nép rivalgását, monda Mózesnek: Harczkiáltás van a táborban.
\par 18 Az pedig felele: Nem diadalmasoknak, sem meggyõzetteknek kiáltása ez, éneklés hangját hallom én.
\par 19 És mikor közeledett volna a táborhoz, látá a borjút és a tánczolást, és felgerjede Mózesnek haragja, és elveté kezébõl a táblákat, és összetöré azokat a hegy alatt.
\par 20 Azután fogá a borjút, a melyet csináltak vala, tûzben megégeté, és apróra töré mígnem porrá lett, és a vízbe hintvén, itatá azt az Izráel fiaival.
\par 21 És monda Mózes Áronnak: Mit tett néked e nép, hogy ilyen nagy bûnbe keverted?
\par 22 Felele Áron: Ne gerjedjen fel uram haragja: ismered e népet, hogy gonosz.
\par 23 Mert azt mondák nékem: Csinálj nékünk isteneket, a kik elõttünk járjanak; mert ama férfiúnak Mózesnek, ki minket Égyiptom földérõl kihozott, nem tudjuk mi lõn dolga.
\par 24 Én pedig felelék: Kinek van aranya? Szedjétek le; és nékem ide adák, én pedig a tûzbe vetettem, és e borjú formáltaték.
\par 25 És látván Mózes, hogy a nép megvadula, mert Áron megvadítá vala õket, ellenségeik csúfjára.
\par 26 Megálla Mózes a tábor kapujában, és monda: A ki az Úré, ide hozzám! és gyûlének õ hozzá mind a Lévi fiai.
\par 27 És szóla nékik: Ezt mondja az Úr, Izráel Istene: Kössön mindenitek kardot az oldalára, menjetek által és vissza a táboron, egyik kaputól a másik kapuig, és kiki ölje meg az õ attyafiát, barátját és rokonságát.
\par 28 A Lévi fiai pedig a Mózes beszéde szerint cselekedének, és elhulla azon a napon a népbõl úgymint háromezer férfiú.
\par 29 És mondá Mózes: Ma szenteljétek kezeiteket az Úrnak, kiki az õ fia és attyafia ellen, hogy áldása szálljon ma reátok.
\par 30 És másnap monda Mózes a népnek: Nagy bûnt követtetek el, most azért felmegyek az Úrhoz, talán kegyelmet nyerhetek a ti bûneiteknek.
\par 31 Megtére azért Mózes az Úrhoz, és monda: Kérlek! Ez a nép nagy bûnt követett el: mert aranyból csinált magának isteneket.
\par 32 De most bocsásd meg bûnöket; ha pedig nem: törölj ki engem a te könyvedbõl, a melyet írtál.
\par 33 És monda az Úr Mózesnek: A ki vétkezett ellenem, azt törlöm ki az én könyvembõl.
\par 34 Most azért eredj: vezesd a népet a hová mondottam néked: Ímé, Angyalom megy elõtted; és az én látogatásom napján ezt az õ bûnöket is meglátogatom.
\par 35 És megverte az Úr a népet ezért is, a mit cselekedtek a borjúval, melyet Áron készített vala.

\chapter{33}

\par 1 Szóla azután az Úr Mózesnek: Eredj, menj fel innen, te és a nép, a melyet kihoztál Égyiptom földérõl, a földre, a melyrõl megesküdtem Ábrahámnak, Izsáknak és Jákóbnak mondván: A te magodnak adom azt.
\par 2 És bocsátok elõtted Angyalt, és kiûzöm a Kananeusokat, Emoreusokat, Khittheusokat, Perizeusokat, Khivveusokat és Jebuzeusokat:
\par 3 A tejjel és mézzel folyó földre; de én nem megyek fel köztetek, mert te keménynyakú nép vagy, hogy meg ne emészszelek az úton.
\par 4 Mikor meghallá a nép ezt a kemény beszédet, gyászba borula, és senki nem tevé fel az ékszereit.
\par 5 Megmondotta vala az Úr Mózesnek: Mondd meg az Izráel fiainak: Keménynyakú nép vagy te, egy szempillantásban, ha közéd mennék, megemésztenélek. Azért most vesd le a te ékességeidet magadról, azután meglátom mit cselekedjem veled.
\par 6 És lerakták magokról az Izráel fiai az õ ékességeket, a Hóreb hegyétõl fogva.
\par 7 Mózes pedig vevé a sátort, és felvoná azt a táboron kívül, messze a tábortól, és nevezé azt gyülekezet sátorának, és lõn, hogy mind a ki az Urat keresi, ki kelle mennie a gyülekezet sátorához, a táboron kívül.
\par 8 És lõn, hogy mikor Mózes kiméne a sátorhoz, az egész nép felkele, és kiki mind az õ sátorának ajtajában álla; nézvén Mózes után míg a sátorba beméne.
\par 9 És lõn, mikor Mózes beméne a sátorba, hogy felhõ-oszlop szálla alá, és megálla a sátor ajtajában, és beszéle Mózessel.
\par 10 És látá az egész nép, hogy a felhõ-oszlop a sátornak ajtaján áll, és felkele az egész nép, és kiki meghajlék az õ sátorának ajtajában.
\par 11 Az Úr pedig beszéle Mózessel színrõl színre, a mint szokott ember szólani barátjával; és mikor Mózes a táborba visszatére, az õ szolgája az ifjú Józsué, Núnnak  fia, nem távozék el a sátorból.
\par 12 És monda Mózes az Úrnak: Lásd, te azt mondod nékem, vidd el ezt a népet, de nem mutattad meg nékem kit küldesz velem; pedig azt mondtad nékem: név szerint ismerlek téged, és kedvet találtál szemeim elõtt.
\par 13 Most azért ha kedvet találtam szemeid elõtt, mutasd meg nékem a te útadat, hogy ismerjelek meg téged, hogy kedvet találhassak elõtted. És gondold meg, hogy e nép a te néped.
\par 14 És monda: Az én orczám menjen-é veletek, hogy megnyugtassalak?
\par 15 Monda néki Mózes: Ha a te orczád nem jár velünk, ne vígy ki minket innen.
\par 16 Mert mirõl ismerhetjük meg, hogy én és a te néped kedvet találtunk elõtted? Nem arról-é, ha velünk jársz? Így vagyunk megkülönböztetve, én és a te néped minden néptõl, a mely e földnek színén van.
\par 17 Monda azért az Úr Mózesnek: Megteszem ezt is a mit kívántál; mert kedvet találtál szemeim elõtt, és név szerint ismerlek téged.
\par 18 És mondá Mózes: Kérlek, mutasd meg nékem a te dicsõségedet.
\par 19 És monda az Úr: Megteszem, hogy az én dicsõségem a te orczád elõtt menjen el, és kiáltom elõtted az Úr nevét: És könyörülök, a kin könyörülök, kegyelmezek a kinek kegyelmezek.
\par 20 Orczámat azonban, mondá, nem láthatod; mert nem láthat engem ember, élvén.
\par 21 És monda az Úr: Ímé van hely én nálam; állj a kõsziklára.
\par 22 És mikor átmegy elõtted az én dicsõségem, a kõszikla hasadékába állatlak téged, és kezemmel betakarlak téged, míg átvonulok.
\par 23 Azután kezemet elveszem rólad, és hátulról meglátsz engemet, de orczámat nem láthatod.

\chapter{34}

\par 1 És monda az Úr Mózesnek: Vágj két kõtáblát, hasonlókat az elõbbiekhez, hogy írjam fel azokra azokat a szavakat, a melyek az elõbbi táblákon voltak, a melyeket széttörtél.
\par 2 És légy készen reggelre, és jöjj fel reggel a Sinai hegyre, és állj ott elõmbe a hegy tetején.
\par 3 De senki veled fel ne jöjjön, és senki ne mutatkozzék az egész hegyen; juhok és barmok se legeljenek a hegy környékén.
\par 4 Vágott azért két kõtáblát, az elõbbiekhez hasonlókat, és felkelvén reggel, felméne Mózes a Sinai hegyre, a mint az Úr parancsolta néki, és kezébe vevé a két kõtáblát.
\par 5 Az Úr pedig leszálla felhõben, és ott álla õ vele, és nevén kiáltá az Urat:
\par 6 És az Úr elvonula õ elõtte és kiálta: Az Úr, az Úr, irgalmas és kegyelmes Isten, késedelmes a haragra, nagy irgalmasságú és igazságú.
\par 7 A ki irgalmas marad ezeríziglen; megbocsát hamisságot, vétket és bûnt: de nem hagyja a bûnöst  büntetlenül, megbünteti az atyák álnokságát a fiakban, és a fiak fiaiban harmad és negyedíziglen.
\par 8 És Mózes nagy sietséggel földre borula, és lehajtá fejét.
\par 9 És monda: Uram, ha elõtted kedvet találtam, kérlek járjon az Úr velünk; mert keménynyakú nép ez! Kegyelmezz a mi vétkeinknek és gonoszságunknak, és fogadj minket örökségeddé.
\par 10 Õ pedig monda: Ímé szövetséget kötök; a te egész néped elõtt csudákat teszek, a milyenek nem voltak az egész földön, sem a népek között, és meglátja az egész nép, a mely között te vagy, az Úrnak cselekedeteit; mert csudálatos az, a mit én cselekszem veled.
\par 11 Jegyezd meg magadnak a mit ma parancsolok néked. Ímé kiûzöm elõled az Emoreust, Kananeust, Khittheust, Perizeust, Khivveust, Jebuzeust.
\par 12 Vigyázz magadra, nehogy szövetséget köss annak a földnek lakosaival, a melybe bemégy, hogy botránkozásra ne legyen közötted.
\par 13 Hanem oltáraikat rontsátok el, törjétek össze bálványaikat, és vágjátok ki berkeiket.
\par 14 Mert nem szabad imádnod más istent; mert az Úr, a kinek neve féltõn szeretõ, féltõn szeretõ Isten õ.
\par 15 Hogy valamiképen szövetséget ne köss annak a földnek lakosaival, hogy a mikor isteneiket követvén paráználkodnak, és áldoznak az õ isteneiknek, és meghívnak téged, egyél az õ áldozatukból.
\par 16 És feleséget ne végy az õ leányaik közül a te fiaidnak, hogy mikor paráználkodnak az õ leányaik isteneiket követvén, a te fiaidat is paráználkodásra vigyék, az õ isteneiket követvén.
\par 17 Ne csinálj magadnak öntött isteneket.
\par 18 A kovásztalan kenyér innepét megtartsad: hét nap egyél kovásztalan kenyeret, a mint megparancsoltam néked, az Abib hónap ideje alatt; mert Abib hónapban jöttél ki Égyiptomból.
\par 19 Mindaz a mi az anyja méhét megnyitja, enyém legyen, és minden hímbarmod is, a mely a te tehenednek vagy juhodnak elsõ fajzása.
\par 20 De a szamárnak elsõ vemhét juhon váltsd meg; ha pedig nem váltod, szegd nyakát. Fiaid közül minden elsõszülöttet megválts, és ne jöjjön üresen  elõmbe senki.
\par 21 Hat napon munkálkodjál, a hetedik napon pedig pihenj; szántás és aratás idején is pihenj.
\par 22 A hetek innepét is megtartsd a búza zsengének aratásakor; meg a betakarás innepét is az esztendõ végén.
\par 23 Háromszor esztendõnként minden férfiú jelenjen meg az Úrnak, Izráel Ura Istenének színe elõtt.
\par 24 Mert kiûzöm a népeket elõled, és kiszélesítem határodat, és senki nem kívánja meg a te földedet, mikor felmégy, hogy a te Urad Istened elõtt megjelenjél, esztendõnként háromszor.
\par 25 Áldozatom vérét ne ontsd ki kovász mellett, és a husvét innepének áldozatja ne maradjon meg reggelig.
\par 26 Földed zsengéibõl az elsõt vidd fel az Úrnak a te Istenednek házába. Ne  fõzz gödölyét az anyja tejében.
\par 27 És monda az Úr Mózesnek: Írd fel ezeket a szavakat; mert ezeknek a szavaknak értelme szerint kötöttem szövetséget veled és Izráellel.
\par 28 És ott vala az Úrral negyven nap és negyven éjjel: kenyeret nem evett, vizet sem ivott. És felírá a  táblákra a szövetség szavait, a tíz parancsolatot.
\par 29 És lõn, a mikor Mózes a Sinai hegyrõl leszálla, (a Mózes kezében vala a bizonyság két táblája, mikor a hegyrõl leszálla) Mózes nem tudta, hogy az õ orczájának bõre sugárzik, mivelhogy Õvele szólott.
\par 30 És a mint Áron és Izráel minden fiai meglátták Mózest, hogy az õ orczájának bõre sugárzik: féltek közelíteni hozzá.
\par 31 Mózes pedig megszólítá õket, és Áron és a gyülekezetnek fejei mind hozzá menének, és szóla velük Mózes.
\par 32 Azután az Izráel fiai is mind hozzá járulának, és megparancsolá nékik mind azt, a mit az Úr mondott néki a Sinai hegyen.
\par 33 Mikor pedig elvégezte Mózes velök a beszédet, leplet tõn orczájára.
\par 34 És mikor Mózes az Úr elébe méne, hogy vele szóljon, levevé a leplet, míg kijõne. Kijövén pedig, elmondá az Izráel fiainak, a mi parancsot kapott.
\par 35 És az Izráel fiai láták a Mózes orczáját, hogy sugárzik a Mózes orczájának bõre; és Mózes a leplet ismét orczájára borítá, mígnem beméne, hogy Õ vele szóljon.

\chapter{35}

\par 1 És egybegyûjté Mózes az Izráel fiainak egész gyülekezetét, és monda nékik: Ezek azok a dolgok, a melyeket parancsolt az Úr, hogy cselekedjetek:
\par 2 Hat napon át munkálkodjatok; a hetedik nap pedig szent legyen elõttetek, az Úr nyugodalmának szombatja. Valaki azon munkálkodik, megölettessék.
\par 3 Ne gerjeszszetek tüzet a ti házaitokban szombatnapon.
\par 4 És szóla Mózes az Izráel fiai egész gyülekezetének, mondván: Ez az, a mit az Úr parancsolt, mondván:
\par 5 Szedjetek magatok közt ajándékot az Úrnak, mind, a kinek szíve önként hajlandó arra, hozzon ajándékot az Úrnak, aranyat, és ezüstöt, és rezet.
\par 6 Kék, és bíborpiros, és karmazsinszínû, és lenfonalat, és kecskeszõrt.
\par 7 Veresre festett kosbõröket, és borzbõröket és sittim-fát.
\par 8 Világító olajat, arómákat a kenet olajához, és fûszereket a füstöléshez.
\par 9 Ónix köveket és foglalni való köveket az efódhoz és a hósenhez.
\par 10 És a kik ti köztetek ahhoz értõk, jõjjenek elõ, hogy csinálják meg mindazt, a mit az Úr parancsolt:
\par 11 A hajlékot, annak sátorát és takaróját, horgait, deszkáit, reteszrúdjait, oszlopait és talpait.
\par 12 A ládát és annak rúdjait, a fedéllel egybe, és a takaró függönyt.
\par 13 Az asztalt és annak rúdjait, és minden edényét, és a szent kenyerekhez valókat.
\par 14 A világító gyertyatartót és a hozzá való eszközöket, mécseit, és világító olajt.
\par 15 A füstölõ oltárt és rúdjait, a kenetnek olaját, és a jó illatú füstölõt, a hajlék ajtajára ajtótakarót.
\par 16 Az egészen égõáldozat oltárát, annak réz rostélyát, rúdjait és minden eszközeit, a mosdómedenczét és annak lábát.
\par 17 A pitvar szõnyegeit, oszlopait, talpait, és a pitvar kapujának leplét.
\par 18 A hajlék szegeit, a pitvar szegeit, és azoknak köteleit.
\par 19 A szolgálati ruhákat, a szent hajlékban való szolgálathoz, a szent ruhákat Áron papnak, és az õ fiainak ruháit, a papi szolgálatra.
\par 20 Azután kiméne az Izráel fiainak egész gyülekezete Mózes elõl.
\par 21 És eljöve mindenki, a kit a szíve indíta, és a kit lelke hajt vala, és hozának áldozatot az Úrnak, a gyülekezet hajlékának készítéséhez, és annak minden szolgálatához, és a szent ruhákhoz valókat.
\par 22 És jövének férfiak és asszonyok együtt, mind, a kit szíve indított, hozának kapcsokat, függõket, gyûrûket, karpereczet, mindenféle arany eszközöket; a férfiak is, mind a kik aranyból hoztak áldozatot az Úrnak.
\par 23 És minden ember, kinek a mije vala, hozott kék, bíborpiros, és karmazsinszínû, és lenfonalat, kecskeszõrt, veresre festett kosbõröket és borzbõröket.
\par 24 Minden, a ki ezüstöt vagy rezet vihetett, felhozá azt áldozatul az Úrnak, és a kiknél sittim-fa találtaték a szolgálat különbözõ szükségeire, felhozák azt.
\par 25 Az asszonyok közûl pedig mind, a kik ahhoz értettek, saját kezeikkel fonának, és felvivék azt a mit fontak, a kék, és a bíborpiros, és a karmazsinszínû, és a lenfonalat.
\par 26 Azok az asszonyok pedig, a kik ahhoz értettek, fonának kecskeszõrt.
\par 27 A fõemberek pedig hozának ónix köveket, foglalni való köveket az efódhoz és a hósenhez.
\par 28 Illatozó szert is és olajat, a mécsbe és a kenethez, és fûszereket a füstöléshez.
\par 29 Minden férfi és asszony, a kit szíve önként indíta, hogy áldozzon az egész munkára, melyet az Úr parancsolt Mózes által, hogy véghez vigyenek: mind önként hoztak ajándékot az Úrnak az Izráel fiai.
\par 30 És monda Mózes az Izráel fiainak: Ímé az Úr név szerint hívta el Bésaléelt, a Húr fiának Urinak fiát, a Júda nemzetségébõl.
\par 31 És betöltötte õt Istennek lelkével, bölcseséggel, értelemmel és tudománynyal minden mesterségben:
\par 32 Hogy tudjon kigondolni mindent a mit aranyból, ezüstbõl és rézbõl kell csinálni;
\par 33 És foglalásra való köveket metszeni, és fát faragni; és minden mesterséges munkát végezni.
\par 34 Azontúl alkalmatossá tette arra is, hogy tanítson, mind õ, mind Aholiáb az Akhiszamák fia, a Dán nemzetségbõl.
\par 35 Betöltötte õket bölcseség lelkével, hogy tudjanak mindenféle faragó, és kötõ, és hímzõ munkát készíteni, kék és bíborpiros, karmazsinszínû és lenfonalból, és takácsmunkát, a kik készítenek ilyenféle munkát és kigondolnak mestermûveket.
\par 36 Azért Bésaléel és Aholiáb, és mindazok a bölcs férfiak, kiknek az Úr bölcseséget és értelmet adott, hogy meg tudják csinálni a szent hajlék szolgálatához való minden eszközöket: csinálják meg egészen úgy, a mint az Úr parancsolta.

\chapter{36}

\par 1 Elhívá azért Mózes Bésaléelt. és Aholiábot, és mindazokat a bölcs férfiakat, a kiknek elméjébe tudományt adott vala az Úr, és mind a kit szíve arra indíta, hogy járuljon annak a munkának végrehajtásához.
\par 2 És átvevék Mózestõl mind azt az ajándékot, a mit az Izráel fiai hoztak vala, a szent hajlék felépítésének szolgálatára. Azontúl is minden reggel önkéntes ajándékot is hoztak.
\par 3 Eljövének azért mind azok a bölcsek, kik a szent hajlék minden munkáján dolgoztak, kiki a maga munkájától, a melyen dolgozott.
\par 4 és ezt mondák Mózesnek: Többet hord a nép ajándékba, mint a mennyi kell a munka elkészítésére, a melyet parancsolt az Úr, hogy csináljunk.
\par 5 Parancsola azért Mózes, és hírré tevék a táborban: Se férfi, se asszony ezután ne készítsen ajándékot a szent munkára. És megszünék a nép hordani.
\par 6 És az egész munka elvégzésére elég volt az adomány, még felesleges is.
\par 7 És mind a bölcs szívû férfiak, kik munkálkodának, készíték a hajlékot, tíz kárpittal: sodrott lenbõl, kék, és bíborpiros és karmazsinszinûbõl, Kérubokkal, mestermunkával készítik azokat.
\par 8 Egy-egy kárpit hossza huszonnyolcz sing, szélessége egy-egy kárpitnak négy sing vala; egy mértéke vala minden kárpitnak.
\par 9 És öt kárpitot foglalának egybe, és ismét a más öt kárpitot is egybefoglalák, egyiket a másikkal.
\par 10 És csinálának kék hurkokat az elsõ kárpitnak szélére, a mely szélrõl vala az egybefoglalásban; hasonlót csinálának a külsõ kárpit szélére, a másik egybefoglalásban is.
\par 11 Az egyik kárpiton ötven hurkot csinálának, és ötven hurkot csinálának a másik kárpit szélén is, a mely a második egybefoglalásban vala; a hurkokat egymás ellenébe.
\par 12 Csinálának ötven arany horgocskát is, és összefoglalák a kárpitokat a horgocskákkal, egyiket a másikkal, és egygyé lõn a hajlék.
\par 13 Csinálának kárpitokat kecskeszõrbõl is, sátornak a hajlékra; tizenegy kárpitot csinálának ilyent.
\par 14 Egy kárpit hossza harmincz sing, és egy kárpit szélessége négy sing; egy mértéke vala a tizenegy kárpitnak.
\par 15 És egybefoglalák az öt kárpitot külön, és a hat kárpitot külön.
\par 16 Csinálának ötven hurkot is a kárpit szélére, a mely szélrõl vala az egybefoglalásban; ötven hurkot csinálának a kárpit szélére a másik egybefoglalásban is.
\par 17 Csinálának ötven rézhorgocskát is a sátor egybefoglalására, hogy egygyé legyen.
\par 18 Csinálának takarót is a sátorra, veresre festett kosbõrökbõl, és azon felül egy takarót borzbõrökbõl.
\par 19 Csinálának takarót is a sátorra, veresre festett kosbõrökbõl, és azon felül egy takarót borzbõrökbõl.
\par 20 Tíz sing a deszkának hossza, másfél sing pedig egy deszkának szélessége.
\par 21 Egy deszkának két csapja vala, egyik a másiknak megfelelõ; így csinálták a hajlék összes deszkáit.
\par 22 A deszkákat pedig így rendezék a hajlékhoz: húsz deszkát déli oldalon, délfelé.
\par 23 És a húsz deszka alá negyven ezüsttalpat készítének; az egyik deszka alá két talpat, az õ két csapja szerint, a másik deszka alá is két talpat, az õ két csapja szerint.
\par 24 A hajlék másik oldalául, az északi oldalon, szintén húsz deszkát csinálának.
\par 25 És negyven ezüsttalpat azok alá, két talpat az egyik deszka alá, a másik deszka alá is két talpat.
\par 26 A hajlék napnyugoti oldalául hat deszkát csinálának.
\par 27 A hajlék szegleteiül pedig a két oldalra, két deszkát csinálának.
\par 28 És alólról kezdve kettõsök valának, felûl pedig egybe valának foglalva egy karikával; így cselekedének mind a kettõvel, a két szegleten.
\par 29 Nyolcz deszka vala tehát, és azoknak tizenhat ezüsttalpa, két-két talp egy-egy deszka alatt.
\par 30 Csinálának reteszrúdakat is sittim-fából, ötöt a hajlék egyik oldalának deszkáihoz.
\par 31 És öt reteszrúdat a hajlék másik oldalának deszkáihoz, és öt reteszrúdat a hajlék nyugoti oldalának deszkáihoz hátulról.
\par 32 És megcsinálák a középsõ reteszrúdat is, hogy fusson a deszkák közepén, végtõl-végig.
\par 33 A deszkákat pedig aranynyal boríták be; a karikáikat aranyból csinálák, gyûrûk gyanánt a reteszrúdakhoz, és a reteszrúdakat is beboríták aranynyal.
\par 34 Megcsinálák a függönyt is, kék, és bíborpiros, és karmazsinszínû, és sodrott lenbõl, mestermunkával csinálák azt, Kérubokkal.
\par 35 És csinálának ahhoz négy oszlopot sittim-fából, és beboríták azokat aranynyal, horgaik aranyból; és öntének azokhoz négy ezüsttalpat.
\par 36 És csinálának a sátor nyílására leplet kék, és bíborpiros, és karmazsinszínû, és sodrott lenbõl, hímzõmunkával.
\par 37 És ahhoz öt oszlopot, horgaikkal együtt; és beboríták azoknak fejeit és átalkötõit aranynyal; öt talpuk pedig rézbõl vala.

\chapter{37}

\par 1 És megcsinálá Bésaléel a ládát sittim-fából; harmadfél sing a hossza, szélessége másfél sing, magassága is másfél sing.
\par 2 És beborítá azt tiszta aranynyal, mind belõl, mind kivül, és csinála reá arany pártázatot köröskörül.
\par 3 És önte annak négy arany karikát a négy szegletére; egyik oldalára is kettõt, másikra is kettõt.
\par 4 Csinála rúdakat is sittim-fából, és beborítá azokat aranynyal.
\par 5 És betolá a rúdakat a láda oldalán levõ karikákba, hogy a láda hordozható legyen.
\par 6 És az egész munka elvégzésére elég volt az adomány, még felesleges is.
\par 7 Csinála két Kérubot is aranyból, vert aranyból csinálá azokat, a fedél két végére.
\par 8 Az egyik Kérubot az egyik végére innen, a másik Kérubot a másik végére onnan; a fedélbõl veré ki a Kérubokat, a két végére.
\par 9 A Kérubok pedig kiterjeszték szárnyaikat felfelé, betakarva szárnyaikkal a fedelet, és arczaik egymással szembe valának; a fedél felé valának a Kérubok arczai.
\par 10 Megcsinálá az asztalt is sittim-fából: két sing a hossza, a szélessége egy sing, magassága másfél sing.
\par 11 És beborítá azt tiszta arannyal, és csinála reá köröskörül arany pártázatot.
\par 12 Csinála egy tenyérnyi széles karájt is köröskörül; és a karjához csinála arany pártázatot köröskörül.
\par 13 Azután önte hozzá négy arany karikát, és a karikákat ráilleszté a négy láb négy szegletére.
\par 14 A karikák a karáj mellett valának rúdtartókul, hogy az asztalt hordozhassák.
\par 15 Megcsinálá a rúdakat is sittim-fából, és azokat beborítá aranynyal, hogy hordozhassák az asztalt.
\par 16 Megcsinálá az asztalra való edényeket is: tálait, csészéit, kelyheit és kancsóit, a melyekkel italáldozatot áldoznak, tiszta aranyból.
\par 17 Megcsinálá a gyertyatartót is tiszta aranyból, vert aranyból csinálá a gyertyatartót; szára, ága, csészéi, gombjai és virágai õ magából valának.
\par 18 És hat ág jöve ki oldalaiból; egyik oldalról is három gyertyatartó-ág, másik oldalról is három gyertyatartó-ág.
\par 19 Három mandolavirágformájú csésze vala az egyik ágon, gombbal és virággal; így a másik ágon is három mandolavirágformájú csésze vala gombbal és virággal; így vala mind a hat ágon, a melyek kijövének a gyertyatartóból.
\par 20 A gyertyatartón pedig négy mandolavirágformájú csésze vala, gombjaikkal és virágaikkal.
\par 21 És gomb vala a két ág alatt õ magából, és gomb vala a két ág alatt õ magából, és gomb vala a két ág alatt õ magából, a hat ág szerint, a mely belõle jöve ki.
\par 22 Gombjaik és ágaik belõle valának; az egész egy vert munka vala, tiszta aranyból.
\par 23 És megcsinálá hét mécsét is, és azoknak hamvvevõit és hamutartóit, tiszta aranyból.
\par 24 Egy talentom tiszta aranyból csinálá azt meg hozzá tartozó eszközeit is mind.
\par 25 Megcsinálá a füstölõ oltárt is sittim-fából; hossza egy sing, szélessége is egy sing, négyszögû; magassága pedig két sing; õ magából valának szarvai.
\par 26 És beborítá azt tiszta aranynyal, tetejét és oldalait köröskörül, és a szarvait is; és csinála hozzá arany pártázatot köröskörül.
\par 27 És két arany karikát csinála hozzá, a pártázata alá, a két oldalán, mindkét oldalára, hogy legyenek rúdtartókul, hogy azokon hordozhassák azt.
\par 28 A rúdakat is sittim-fából csinálá meg, és azokat is beborítá aranynyal.
\par 29 A szent kenetnek olaját is megcsinálá, és a tiszta fûszerekbõl való füstölõt a kenetkészítõ mestersége szerint.

\chapter{38}

\par 1 És megcsinálá azután az egészen égõáldozat oltárát sittim-fából; öt sing a hossza, öt sing a szélessége; négyszögû és három sing magas.
\par 2 És csinála hozzá szarvakat is, a négy szegletére; õ magából voltak a szarvai, és beborítá azt rézzel.
\par 3 Azután minden edényét is megcsinálá: a fazekakat, a lapátokat, a medenczéket, a villákat és a szenes serpenyõket; minden edényeit rézbõl csinálá.
\par 4 Csinála az oltárhoz hálóforma rostélyt is rézbõl, annak párkányzata alá, alulról a közepéig.
\par 5 Önte négy karikát is rézrostélynak négy szegletére, rúdtartókul.
\par 6 A rúdakat sittim-fából csinálá meg, és rézzel borítá be.
\par 7 És a rúdakat betolá az oltár oldalain lévõ karikákba, hogy azokon hordozhassák azt; deszkákból, üresre csinálá azt.
\par 8 Megcsinálá a mosdómedenczét is rézbõl, és annak lábát is rézbõl, a szolgálattevõ asszonyok tükreibõl, a kik a gyülekezet sátorának nyílása elõl szolgáltak.
\par 9 A pitvart is megcsinálá dél felõl; a déli oldalon a pitvar szõnyege száz singnyi vala sodrott lenbõl.
\par 10 Azoknak húsz oszlopa, és húsz talpa rézbõl vala; az oszlopok horgai, és azoknak általkötõi ezüstbõl.
\par 11 Észak felõl is száz singnyi. Azoknak húsz oszlopa és húsz talpa rézbõl; az oszlopok horgai és általkötõi ezüstbõl valának.
\par 12 Napnyugot felõl pedig ötven sing szõnyeg vala. Azoknak tíz oszlopa és tíz talpa; az oszlopok horgai és általkötõi ezüstbõl valának.
\par 13 A napkeleti oldalon is ötven singnyi.
\par 14 Egy felõl tizenöt sing szõnyeg vala; három oszlopa és azoknak három talpa.
\par 15 És a másik felõl is: a pitvar kapujától jobbra is, balra is tizenöt sing szõnyeg, és azoknak három oszlopa, három talpa.
\par 16 A pitvarnak minden szõnyege köröskörül sodrott lenbõl vala.
\par 17 Az oszlopok talpai rézbõl, az oszlopok horgai és általkötõi pedig ezüstbõl valának, azoknak fejei ezüsttel beborítva. A pitvar oszlopait is mind ezüst általkötõk övezték.
\par 18 A pitvar kapujának leple hímzõmunka vala, aranyból, kék, és bíborpiros, és karmazsinszínû, és sodrott lenbõl; hossza húsz sing volt, magassága pedig szélességében öt sing, a pitvar szõnyegeinek megfelelõleg.
\par 19 Azoknak négy oszlopa és négy talpa rézbõl, horgai ezüstbõl valának; fejeiknek borítása, és általkötõik is ezüstbõl.
\par 20 A hajléknak és a pitvarnak szegei pedig köröskörûl mind rézbõl valának.
\par 21 Ezek a hajléknak, a bizonyság hajlékának részei, a mint megszámláltattak a Mózes meghagyásából, a Léviták szolgálatára, Ithamár, az Áron pap fiai által.
\par 22 Bésaléel a Húr fiának, Urinak fia a Júda nemzetségébõl, csinálta mind azt, a mit az Úr parancsolt vala Mózesnek.
\par 23 És vele együtt Aholiáb, Akhiszamák fia, Dán nemzetségébõl a ki mester vala a faragásban, a kötõ és hímzõmunkában, kék, és bíborpiros, és karmazsinszínû, és lenfonállal.
\par 24 Mind az az arany, a mely a munkára, a szenthelynek összes munkáira feldolgoztaték, ez az áldozati arany: huszonkilencz talentom, és hétszáz harmincz siklus vala, a szent siklus szerint.
\par 25 Az ezüst pedig, a gyülekezet megszámlált tagjaitól, száz talentom, és ezer hétszáz hetvenöt siklus, a szent siklus szerint.
\par 26 Fejenként egy beka, vagy is fél siklus, a szent siklus szerint, mindenkitõl, a ki átesett a megszámláltatáson, húsz esztendõstõl fölfelé, a kik hatszáz háromezeren és ötszázötvenen valának.
\par 27 A száz talentom ezüstbõl megönték a szent helyhez való talpakat, és a függöny oszlopainak talpait; száz talpat száz talentomból, egy talentomból egy talpat.
\par 28 Az ezer hétszáz hetvenöt siklusból csinálák az oszlopok horgait, és beboríták azoknak fejeit, és általfogák azokat.
\par 29 Az áldozati réz pedig hetven talentom, és kétezer négyszáz siklus vala.
\par 30 Abból csinálák a talpakat a gyülekezet sátorának nyílásához, és a réz oltárt, az ahhoz való réz rostélyt, és az oltérnak minden edényeit.
\par 31 A pitvar talpait is köröskörül, és a pitvar kapujához való talpakat, meg a hajlék összes szegeit, és a pitvar összes szegeit köröskörül.

\chapter{39}

\par 1 A kék, és bíborpiros, és karmazsinszínû fonálból pedig csinálának szolgálati ruhákat a szenthelyen való szolgálatra. Áronnak is úgy készíték a szent ruhákat, a mint az Úr parancsolta vala Mózesnek.
\par 2 Az efódot csinálák aranyból, kék, és bíborpiros, és karmazsinszínû, sodrott lenbõl.
\par 3 És aranyból vékony lapokat verének, és azokat fonalakká metélék, hogy feldolgozzák a kék, és a bíborpiros, és a karmazsinszínû, és a lenfonál közé, mestermunkával.
\par 4 Csinálának hozzá összekötött vállkötõket: a két végükön kapcsolák össze.
\par 5 Az átkötõ öv is, a mely rajta vala, abból való, ugyanolyan mívû vala, mint az efód: aranyból, kék, és bíborpiros, és karmazsinszínû, és sodrott lenbõl, a mint az Úr parancsolta Mózesnek.
\par 6 Megcsinálák azután az ónix köveket is, arany boglárokba foglalva, kimetszve pecsétmetszés módjára, az Izráel fiainak neveire.
\par 7 És rátették azokat az efód vállkötõire, az Izráel fiaira való emlékeztetés kövei gyanánt, a mint az Úr parancsolta vala Mózesnek.
\par 8 Megcsinálák a hósent is mestermunkával, mint az efódot, aranyból, kék, és bíborpiros, és karmazsinszínû, és sodrott lenbõl.
\par 9 Négyszögû vala a hósen, és kétrétûre készíték azt; egy arasz vala hossza, egy arasz a szélessége is két rétben.
\par 10 És négy sor követ foglalának abba, ily sorban: szárdiusz, topáz, smaragd; az elsõ sor.
\par 11 A második sor: karbunkulus, zafir és gyémánt.
\par 12 A A harmadik sor: jáczint, agát és amethiszt.
\par 13 A negyedik sor: krizolith, ónix és jáspis, a melyek mind arany boglárokba foglaltattak a magok helyén.
\par 14 A kövek pedig Izráel fiainak nevei szerint valának, tizenkettõ vala az õ nevök szerint, és pecsét módjára metszve, mindenik a reá való névvel, a tizenkét nemzetség szerint.
\par 15 És csinálának a hósenre sodrott mívû fonatékot is tiszta aranyból.
\par 16 Azután csinálának két arany boglárt, és két arany karikát; és a két arany karikát rá tevék a hósen két szegletére.
\par 17 A két arany fonatékot pedig a hósen két szegletén levõ két karikába fûzték.
\par 18 És a két fonaték másik két végét oda foglalák a két boglárhoz, és azokat az efód vállkötõihez tûzék, annak elõrészére.
\par 19 Csinálának más két arany karikát is, és oda tevék azokat a hósen két szegletére, azon a szélén, a mely befelé vala az efód felõl.
\par 20 És csinálának még két arany karikát, és azokat az efód két vállkötõjére tevék, alól, annak elõrésze felõl, az összefoglalás mellett, az efód övén felül.
\par 21 És a hósent az õ karikáinál fogva odacsatolák az efód karikáihoz, kék zsinórral, hogy az efódnak öve felett legyen, és el ne váljék a hósen az efódtól; a mint az Úr parancsolta vala Mózesnek.
\par 22 És megcsinálák az efód palástját takácsmunkával, egészen kék lenbõl.
\par 23 És a palást közepén levõ nyílás olyan vala, mint a pánczélnak nyílása; szegése vala a nyílásnak köröskörül, hogy el ne szakadjon.
\par 24 És a palást alsó peremére csinálának gránátalmákat, kék, és bíborpiros, és karmazsinszinû, és sodrott fonálból.
\par 25 Csinálának csengettyûket is tiszta aranyból, és tevék a csengettyûket a gránátalmák közé, a palást peremére köröskörül a gránátalmák közé.
\par 26 Csengettyû meg gránátalma, csengettyû meg gránátalma a palást peremén köröskörül, hogy abban szolgáljanak, a mint az Úr parancsolta vala Mózesnek.
\par 27 Megcsinálák a a len köntösöket is takácsmunkával, Áronnak és az õ fiainak.
\par 28 A len süveget is a süveg ékességeit is lenbõl, és a gyolcs lábravalókat sodrott lenbõl.
\par 29 Az övet is sodrott lenbõl, és kék, és bíborpiros, és karmazsinszínû fonálból hímzõmunkával, a mint az Úr parancsolta vala Mózesnek.
\par 30 A szent koronához való lapot is megcsinálák tiszta aranyból, és felírák rá a pecsétmetszõ módjára: Szentség az Úrnak.
\par 31 És kötének reá kék zsinort, hogy felkössék azzal a süvegre, a mint az Úr parancsolta vala Mózesnek.
\par 32 Így végezteték el a gyülekezet sátora hajlékának egész munkája; és az Izráel fiai egészen úgy csinálák, a mint az Úr parancsolta vala Mózesnek, úgy csinálták.
\par 33 És vivék a hajlékot Mózeshez: a sátort és annak minden eszközét, horgait, deszkáit, reteszrúdait, oszlopait, és talpait.
\par 34 A veresre festett kosbõrökbõl készült takarót is, és a borzbõrökbõl készült takarót, és a takarófüggönyt.
\par 35 A bizonyság ládáját, annak rúdait, és a fedelet.
\par 36 Az asztalt és annak minden edényét, és a szent kenyeret.
\par 37 A tiszta gyertyatartót: annak mécseit, a felszereléshez való mécseket s minden hozzávalót, a világító olajjal együtt.
\par 38 Az arany oltárt, a kenetnek olaját, a fûszerekbõl való füstölõt, és a sátor nyílására való leplet.
\par 39 A réz oltárt és annak réz rostélyát, rúdait és minden edényeit, a mosdómedenczét és annak lábát.
\par 40 A pitvar szõnyegeit, oszlopait, talpait: és a pitvar kapujára való leplet, annak köteleit, szegeit, és a hajlék szolgálatjára való minden eszközt a gyülekezet sátoránál.
\par 41 A szolgálati ruhákat, a szenthelyen való szolgálatra, a szent ruhákat Áron papnak, és fiainak ruháit a papi tisztre.
\par 42 A mint parancsolta vala az Úr Mózesnek, egészen úgy csinálának Izráel fiai minden munkát.
\par 43 És megtekinte Mózes minden munkát, és ímé elkészítik azt, úgy készíték el, a mint az Úr parancsolta vala, és megáldá õket Mózes.

\chapter{40}

\par 1 És szóla az Úr Mózesnek, mondván:
\par 2 Az elsõ hónapban, a hónap elsõ napján, állítsd fel a gyülekezet sátorának hajlékát.
\par 3 És tedd oda a bizonyság ládáját, és fedd be a ládát a fedéllel.
\par 4 Azután vidd be az asztalt, és hozd azt rendbe a reávalókkal. Vidd be a gyertyatartót is, és gyújtsd meg annak mécseit.
\par 5 És helyheztesd a füstölõ áldozat arany oltárát a bizonyság ládája elé, tedd fel a leplet a hajlék nyílására.
\par 6 Azután helyheztesd az egészen égõáldozat oltárát a gyülekezet sátora hajlékának nyílása elé.
\par 7 A mosdómedenczét pedig helyheztesd a gyülekezet sátora közé és az oltár közé, és önts belé vizet.
\par 8 Azután állítsd fel köröskörül a pitvart, és a pitvar kapujára tedd rá a leplet.
\par 9 Azután vedd a kenetnek olaját, és kend meg a hajlékot és mind azt a mi benne van: így szenteld meg azt, és minden edényét, hogy szent legyen.
\par 10 Kend meg az egészen égõáldozat oltárát is és annak minden edényét; így szenteld meg az oltárt, hogy az oltár legyen szentséges.
\par 11 A mosdómedenczét és lábát is kend meg; így szenteld meg azt.
\par 12 Azután állítsd Áront és az õ fiait a gyülekezet sátorának nyílásához, és mosd meg õket vízzel.
\par 13 És öltöztesd fel Áront a szent ruhákba, és kend fel õt; így szenteld fel, és legyen papom.
\par 14 Fiait is állítsd elõ és öltöztesd fel õket köntösökbe;
\par 15 És kend fel õket, a mint az õ atyjokat felkenéd, hogy legyenek papjaim, és lészen az õ megkenetésök nékik örökös papság, az õ nemzetségökben.
\par 16 És egészen úgy cselekedék Mózes, a mint az Úr parancsolta néki, úgy cselekedék.
\par 17 Lõn azért a második év elsõ hónapjában, a hónap elsõ napján, hogy felállíták a hajlékot.
\par 18 Mózes tehát felállítá a hajlékot, és letevé annak talpait, a deszkáit is felállítá, beilleszté a reteszrúdakat, és felállítá annak oszlopait.
\par 19 Azután kifeszíté a hajlékra a sátort, és a sátor takaróját is reá teríté felülrõl, a mint parancsolta vala az Úr Mózesnek.
\par 20 És vevé és betevé a bizonyságot is a ládába, a rúdakat pedig ládára tevé, és a ládára felül helyezteté a fedelet.
\par 21 Azután bevivé a ládát a hajlékba, és feltevé a takaró függönyt, és elfedezé a bizonyság ládáját, a mint az Úr parancsolta vala Mózesnek.
\par 22 Az asztalt is bevivé a gyülekezet sátorába, a hajléknak északi oldalába, a függönyön kivül.
\par 23 És rakott arra kenyereket sorban az Úr elõtt, a mint az Úr parancsolta vala Mózesnek.
\par 24 És a gyertyatartót helyhezteté a gyülekezet sátorába az asztal ellenébe, a hajléknak déli oldalán.
\par 25 És meggyújtá a mécseket az Úr elõtt, a mint az Úr parancsolta vala Mózesnek.
\par 26 Az arany oltárt pedig helyhezteté a gyülekezet sátorába a függöny elébe.
\par 27 És füstölögtete rajta fûszerekbõl való füstölõt, a mint az Úr parancsolta vala Mózesnek.
\par 28 A hajlék nyílására a leplet is felvoná.
\par 29 Az egészen égõáldozat oltárát pedig helyhezteté a gyülekezet sátora hajlékának nyílása elé, és áldozék azon egészen égõáldozattal és ételáldozattal, a mint parancsolta vala az Úr Mózesnek.
\par 30 A mosdómedenczét pedig helyhezteté a gyülekezet sátora közé és az oltár közé, és önte abba vizet a mosdásra.
\par 31 És megmosák abból Mózes és Áron és az õ fiai kezeiket és lábaikat.
\par 32 Mikor a gyülekezet sátorába menének, és mikor az oltárhoz járulának, megmosdának, a mint az Úr parancsolta vala Mózesnek.
\par 33 Azután felállítá a pitvart a hajlék és az oltár körül, és a pitvar kapujára feltevé a leplet. Így végezé el Mózes a munkát.
\par 34 És a felhõ befedezé a gyülekezet sátorát, és az Úrnak dicsõsége betölté a hajlékot.
\par 35 És Mózes nem mehete be a gyülekezet sátorába, mert a felhõ rajta nyugovék, és az Úrnak dicsõsége tölté be a hajlékot.
\par 36 És mikor a felhõ felszáll vala a hajlékról, az Izráel fiai elindulának; így lõn egész utazásuk alatt.
\par 37 Ha pedig a felhõ nem szálla fel, õk sem indulának el, míg csak fel nem szálla.
\par 38 Mert az Úrnak felhõje vala a hajlékon nappal, éjjel pedig tûz vala azon, az Izráel egész háznépének láttára, egész utazásuk alatt.



\end{document}