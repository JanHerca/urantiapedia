\begin{document}

\title{Királyok I. könyve}


\chapter{1}

\par 1 Mikor pedig megvénhedt és megöregedett Dávid király, bár leplekkel takargatták be, mégsem bírt felmelegedni.
\par 2 És mondának néki az õ szolgái: Keressenek az én uramnak, a királynak egy szûz leányt, a ki a király körül legyen, és õt ápolja, aludjék karjai között, és melegítse fel az én uramat, a királyt.
\par 3 Keresének annakokáért egy szép leányt Izráelnek minden határiban; és találák a Súnem városából való Abiságot,  a kit el is hozának a királyhoz.
\par 4 És a leány igen szép volt, és a királyt ápolta és szolgált néki. De a király nem ismeré õt.
\par 5 Adónia pedig, Haggitnak fia, felfuvalkodék, ezt mondván: Én fogok uralkodni! És szerze magának szekereket, lovagokat és ötven elõtte járó férfiakat.
\par 6 Kit az õ atyja soha meg nem szomoríta, ezt mondván: Miért cselekszel így?! Ez is pedig igen szép férfi volt, és õt Haggit szülte volt Dávidnak Absolon után.
\par 7 És tanácskozék Joábbal, Séruja fiával és Abjátár pappal, kik az Adónia pártján voltak.
\par 8 De Sádók pap, meg Benája, a Jójada fia, és Nátán  próféta, és Sémei, és Réhi, és a Dávid erõs vitézei nem állottak Adónia mellé.
\par 9 Mikor pedig Adónia áldozatot mutatott be juhokból, ökrökbõl és egyéb kövér barmokból a Zohélet kõsziklánál, a mely a Rógel forrása mellett volt: meghívá egész rokonságát, a király fiait, Júda minden férfiait, a király szolgáit;
\par 10 De Nátán prófétát és Benáját és amaz erõs vitézeket és Salamont, az õ atyjafiát nem hívá el.
\par 11 Szóla akkor Nátán Bethsabénak, a Salamon anyjának, mondván: Nem hallottad-é, hogy Adónia, a Haggit fia uralkodik, és a mi urunk Dávid nem tud róla semmit?
\par 12 Jövel azért, hadd adjak néked tanácsot, hogy megmentsd a te életedet, és a te fiadnak, Salamonnak életét.
\par 13 Eredj, menj be Dávid királyhoz, és mondd ezt néki: Uram király, nemde nem esküdtél-é meg a te szolgálóleányodnak ilyenképen: Salamon, a te fiad uralkodik én utánam, és õ ül az én királyi székembe? Miért uralkodik hát Adónia?
\par 14 És ímé mialatt még te ott a királylyal beszélsz, én is bemegyek utánad, és kiegészítem a te beszédidet.
\par 15 És beméne Bethsabé a királyhoz a kamarába. És a király igen megvénhedett vala, és a Súnembõl való Abiság szolgál vala a királynak.
\par 16 És fejet hajta Bethsabé, és meghajtá magát a királynak. És monda a király: Mit kivánsz?
\par 17 Felele néki Bethsabé: Édes uram, te megesküdtél az Úrra, a te Istenedre a te szolgálóleányodnak ilyenképen: Salamon,  a te fiad uralkodik én utánam, és õ ül az én királyi székembe.
\par 18 És ímé mégis Adónia lett királylyá; és ímé, uram király, te nem tudsz errõl semmit.
\par 19 Mert áldozott ökrökkel és nagy sok kövér barmokkal bõségesen, és vendégekké hívta a királynak minden fiait, és Abjátár papot és Joábot, a seregnek hadnagyát, csak Salamont, a te szolgádat nem hívta meg.
\par 20 Még most te vagy, Uram, a király; az egész Izráel népének szemei reád néznek, hogy megjelentsed nékik, kicsoda fog ülni az én uramnak, a királynak székében, õ utána.
\par 21 De ha az én uram, a király, az õ atyáival elaluszik: akkor én és az én fiam, Salamon leszünk bûnösök.
\par 22 És ímé, mikor még a királylyal szólana, Nátán próféta megérkezék.
\par 23 És bejelenték a királynak, mondván: Itt van Nátán próféta. És bemenvén a király eleibe, meghajtá magát a király elõtt, arczczal a földre leborulván.
\par 24 És monda Nátán: Uram király, te mondottad-é: Adónia legyen én utánam a király, és õ üljön az én királyi székembe?
\par 25 Mert ma aláment, és áldozott ökrökkel és kövér barmokkal bõségesen, és vendégekké hívta a királynak minden fiait és a seregnek hadnagyait és Abjátár papot: és ímé õk esznek és isznak õ elõtte, és immár azt kiáltották: Éljen Adónia király!
\par 26 Engem pedig, a ki a te szolgád vagyok, és Sádók papot és Benáját, a Jójada fiát, és Salamont, a te szolgádat nem hívta meg.
\par 27 Avagy az én uramtól, a királytól lett-é ez a dolog, hogy nem adtad tudtára a te szolgádnak, kicsoda fogna ülni az én uramnak, a királynak székiben az õ holta után.
\par 28 És felelvén Dávid király, monda: Hívjátok hozzám Bethsabét, a ki beméne a király eleibe, és megálla a király elõtt.
\par 29 És megesküvék a király, mondván: Él az Úr, a ki megszabadította az én lelkemet minden nyomorúságból,
\par 30 Hogy a miképen megesküdtem néked az Úrra, Izráel Istenére, ezt mondván: A te fiad Salamon uralkodik én utánam, és õ ül az én királyi székembe én helyettem: ezt ma így meg is teszem.
\par 31 És fejet hajta Bethsabé, arczczal a földre leborulván, és magát meghajtván a király elõtt, monda: Éljen az én uram, Dávid király, mindörökké!
\par 32 Azután monda Dávid király: Hívjátok hozzám Sádók papot és Nátán prófétát és Benáját,  Jójadának fiát. És ezek bemenének a király eleibe.
\par 33 És monda nékik a király: Vegyétek mellétek a ti uratoknak szolgáit, és ültessétek Salamont, az én fiamat az én öszvéremre, és vigyétek alá õt Gihonba;
\par 34 És kenje õt ott Sádók pap és Nátán próféta Izráelnek királyává; és fújjátok meg a harsonákat, és kiáltsátok: Éljen Salamon király!
\par 35 És jõjjetek fel onnét õ utána; és eljövén, üljön az én királyi székembe, és õ uralkodjék én helyettem; mert immár meghagytam néki, hogy õ legyen fejedelme mind Izráelnek, mind Júdának.
\par 36 Felele akkor Benája, a Jójada fia a királynak, és monda: Ámen! Így szóljon az Úr, az én uramnak, a királynak Istene is.
\par 37 A miképen vele volt az Úr az én urammal, a királylyal: azonképen legyen vele Salamonnal is, és magasztalja feljebb az õ királyi székét az én uramnak, Dávid királynak királyi székénél.
\par 38 Aláméne azért Sádók pap és Nátán próféta és Benája, a Jójada fia, a Kereteusok is és a Peleteusok, és felülteték Salamont a Dávid király öszvérére, és alávivék õt Gihonba.
\par 39 És vevé Sádók pap az olajos szarut az Úr sátorából, és megkené Salamont;  azután kürtölének, és az egész nép ezt kiáltá: Éljen Salamon király!
\par 40 És felvonult utána az egész nép, és a nép sípolt és felette ujjongott, úgy hogy a föld is megrepedne kiáltásuk zajától.
\par 41 És Adónia is meghallotta és vendégei is mind, a kik nála valának, miután a lakomát már elvégezték; és meghallá Joáb is a kürtölés szavát, és monda: Miért e zaj és mozgás a városban?
\par 42 És a mikor õ még szólana, ímé megérkezék Jonathán, az Abjátár pap fia. És monda Adónia: Jõjj be, mert megbízható férfiú vagy, és jó hírt mondasz.
\par 43 Jonathán pedig felelvén, monda Adóniának: Igen, a mi urunk, Dávid király, Salamont tette királylyá.
\par 44 És elküldötte õ vele Sádók papot a király és Nátán prófétát, és Benáját, a Jójada fiát, a Kereteusokat is és Peleteusokat, és õt a király öszvérére ülteték,
\par 45 És Sádók pap Nátán prófétával együtt királylyá kente fel õt Gihonnál, és onnét vonulnak fel örömmel; ettõl zendült meg a város. Ez az a zaj, a melyet hallottatok.
\par 46 És immár be is ült Salamon az országnak királyi székibe;
\par 47 És a király szolgái is bemenének, hogy áldják a mi urunkat, Dávid királyt, mondván: Tegye az Isten a Salamon nevét híresebbé a te nevednél, és magasztalja feljebb az õ székit a te székednél. És meghajtá magát a király az õ ágyán.
\par 48 És ekképen is szóla a király: Áldott az Úr, Izráel Istene, a ki adott e mai napon olyat, a ki szemeim láttára helyettem üljön az én királyi székemben.
\par 49 Akkor megrettenének, és felkelének mindnyájan a hivatalosok, a kik Adóniával valának, és kiki mind dolgára méne.
\par 50 De Adónia félt Salamontól, és felkészülve elfutott, és megragadta az oltárnak szarvait.
\par 51 Hírül adák pedig Salamonnak ilyen szókkal: Ímé Adónia Salamon királytól való féltében megfogá az oltárnak szarvait, ezt mondván: Esküdjék meg ma nékem Salamon király, hogy meg nem öli az õ szolgáját fegyverrel.
\par 52 És monda Salamon: Ha jámbor lészen, egy hajszál fejérõl le nem esik a földre; de ha gonoszság találtatik õ benne,  meg kell halnia.
\par 53 Elkülde azért Salamon király, és elhozák õt az oltártól; és eljövén meghajtá magát Salamon király elõtt; és monda Salamon király: Menj el a te házadhoz.

\chapter{2}

\par 1 Mikor pedig elközelgett Dávidnak ideje, hogy meghaljon, parancsot ada Salamonnak az õ fiának, ezt mondván:
\par 2 Én elmegyek az egész földnek útján; erõsítsd meg magad és légy férfiú.
\par 3 És õrízd meg az Úrnak a te Istenednek õrizetit, hogy az õ útain járj, és megõrizzed az õ rendeléseit, parancsolatit és ítéleteit, és bizonyságtételeit, a mint meg van írva a Mózes törvényében: hogy elõmented legyen mindenekben, a melyeket cselekedéndesz, és mindenütt, valamerre fordulándasz;
\par 4 Hogy megteljesítse az Úr az õ beszédét, melyet szólott nékem, mondván: Ha megõrizéndik a te fiaid az õ útjokat, járván én elõttem tökéletességgel, teljes szivök és teljes lelkök szerint; ezt mondván, mondom: Soha el nem fogy a férfiú te közüled az Izráelnek királyi székibõl.
\par 5 Azt is jól tudod, mit cselekedett én velem Joáb, a Séruja fia, mit cselekedett az Izráel seregeinek két fõvezérével, Abnerrel, a Nér fiával, és Amasával,  a Jéter fiával, a kiket megölt, harczi vért ontván békességnek idején, és hintett harczi vért az õ derekának övére és az õ lábának saruira.
\par 6 Cselekedjél a te bölcseséged szerint, és ne engedd, hogy megõszülvén, békességgel menjen a koporsóba.
\par 7 De a gileádbeli Barzillainak fiaival cselekedjél irgalmasságot, és legyenek a te asztalod vendégei, mert így közeledtek õk is hozzám, mikor Absolon, a te testvéred elõl menekültem.
\par 8 És ímé veled van Sémei, Gérának fia, a Bahurimbeli Benjáminita, a ki gyalázatosan szidalmazott akkor, mikor Mahanáimba mentem; de azután mikor elém alájött a Jordánhoz, megesküdtem  néki az Úrra, és mondék: Nem öllek meg téged fegyverrel;
\par 9 Te azonban ne hagyd õt büntetés nélkül, és mivel eszes férfiú vagy, tudod, mit kelljen cselekedned vele, hogy az õ vénségét vérrel bocsássad a koporsóba.
\par 10 Azután elaludt Dávid az õ atyáival, és eltemetteték a Dávid városában.
\par 11 Az idõ pedig, a melyben uralkodék Dávid Izráelen, negyven esztendõ. Hebronban uralkodék hét esztendeig, Jeruzsálemben pedig uralkodék harminchárom esztendeig.
\par 12 Azután Salamon ült Dávidnak, az õ atyjának királyi székibe, és megerõsödék az õ királyi birodalma felette igen.
\par 13 De Adónia, a Haggit fia beméne Bethsabéhoz, a Salamon anyjához, és az monda: Békességes-é a te jöveteled? Ki felele: Békességes.
\par 14 És monda: Beszédem volna veled. Monda az: Szólj.
\par 15 Akkor monda Adónia: Te tudod, hogy az ország az enyém vala, és az egész Izráel reám néz vala, hogy én uralkodjam; de elvéteték az ország tõlem, és lõn az én atyámfiáé, mert az Úrtól adattaték néki.
\par 16 Most egy kérést kérek tõled, ne szégyenítsd meg orczámat. Az pedig monda: Beszélj!
\par 17 És monda: Beszélj, kérlek Salamon királylyal; mert õ a te kérésedet meg nem veti, hogy adja nékem a Súnembõl való Abiságot feleségül.
\par 18 Felele Bethsabé: Jól van, majd szólok melletted a királynak.
\par 19 És beméne Bethsabé Salamon királyhoz, hogy beszéljen vele Adónia érdekében; és felkele a király, és elébe menvén meghajtá magát elõtte, és leüle királyi székibe; és széket tétete a király anyjának, hogy üljön az õ jobbkeze felõl.
\par 20 És monda Bethsabé: Egy kis kérést kérek tõled, ne szégyenítsd meg orczámat. És monda néki a király: Kérj édes anyám; mert nem szégyenítem meg orczádat.
\par 21 Monda õ: Adassék a Súnembõl való Abiság Adóniának, a te testvérednek feleségül.
\par 22 Akkor felele Salamon király, és monda az õ anyjának: De miért kéred te a Súnembeli Abiságot Adóniának? Kérjed néki az országot is; mert õ az én bátyám, és vele egyetért Abjátár pap, és Joáb, a Séruja fia.
\par 23 És megesküvék Salamon király az Úrra, mondván: Úgy cselekedjék velem az Isten, és úgy segéljen, hogy Adónia a saját élete ellen szólotta ezt a beszédet!
\par 24 Most azért él az Úr, a ki megerõsített engem és ültetett engem az én atyámnak, Dávidnak királyi székibe, és a ki házat szerzett nékem, a mint megmondotta volt: ma Adóniának meg kell halnia!
\par 25 Elküldé azért Salamon király Benáját, a Jójada fiát, a ki levágá õt, és meghala.
\par 26 Abjátár papnak pedig monda a király: Menj el Anathótba, a te jószágodba, mert halálnak fia vagy; de ma meg nem öletlek, mivel te hordoztad az Úr Istennek ládáját Dávid, az én atyám elõtt, és mivel az én atyámnak minden  nyomorúságaiban részes voltál.
\par 27 És kiûzé Salamon Abjátárt, hogy ne legyen az Úrnak papja, hogy beteljesedjék az Úrnak beszéde, a melyet szólott vala az Éli  háza felõl Silóban.
\par 28 És eljutott ez a hír Joábhoz, mert Joáb Adóniához hajlott vala, noha azelõtt nem hajlott vala Absolonhoz, és elfuta Joáb az Úrnak sátorába, és megfogá az oltárnak szarvait.
\par 29 Hírül adák pedig Salamon királynak, hogy Joáb az Úrnak sátorához futott, és ímé az oltár  mellett áll. Ekkor elküldé Salamon Benáját, a Jójada fiát, mondván: Menj el, vágd le õt.
\par 30 Mikor pedig Benája az Úrnak sátorához ért, monda néki: Ezt mondja a király: Jõjj ki. Kinek felele Joáb: Nem, itt akarok meghalni. És megvivé Benája a királynak e dolgot, mondván: Így szólott Joáb és így felelt nékem.
\par 31 És monda néki a király: Cselekedjél úgy, a mint szólott; vágd le õt és temesd el, hogy elvedd az ártatlan vért, a melyet kiontott  Joáb, én rólam és az én atyámnak házáról.
\par 32 És fordítsa az Úr az õ fejére az õ vérét, a miért nálánál igazabb és jobb két férfira támadott, és megölé õket fegyverrel az én atyámnak, Dávidnak tudta nélkül, tudniillik Abnert, Nérnek fiát, az Izráel seregének fõvezérét és Amasát,  Jéternek fiát, Júda vitézeinek fõvezérét.
\par 33 Ezeknek a vére térjen Joáb fejére és az õ magvának fejére mindörökké: Dávidnak pedig és az õ magvának és az õ házának és királyi székének békessége legyen az Úrtól mindörökké.
\par 34 És elméne Benája, Jójada fia és reá rohanván megölé õt; és eltemetteték az õ házában, a pusztában.
\par 35 Rendelé pedig a király a Jójada fiát õ helyette a sereg fölé, és  Sádók papot rendelé a király Abjátár helyett.
\par 36 És elkülde a király, és magához hivatá Sémeit, és monda néki: Építs házat magadnak Jeruzsálemben és lakjál ott; és onnét ne menj ki se ide, se tova.
\par 37 Mert valamely nap kimenéndesz, és általmenéndesz a Kidron patakján, tudd meg, hogy meg kell halnod, a te véred lészen tennen fejeden.
\par 38 És monda Sémei a királynak: Tetszik nékem e beszéd; a miképen szólott az én uram, a király, a képen cselekeszik a te szolgád; és sok ideig lakék Sémei Jeruzsálemben.
\par 39 Lõn azonban három esztendõ mulva, hogy Sémeinek két szolgája elszökött Ákishoz, Maaka fiához, a Gáthbeli királyhoz; és hírül adák Sémeinek, mondván: Ímé a te szolgáid Gáthban vannak.
\par 40 Ekkor felkelt Sémei, és megnyergelé szamarát, és elméne Gáthba Ákishoz, hogy megkeresse az õ szolgáit. Oda érvén Sémei, meghozá szolgáit Gáthból.
\par 41 Hírül adák pedig Salamonnak, hogy elment Sémei Jeruzsálembõl Gáthba, és haza jött.
\par 42 Akkor elkülde a király, és magához hivatá Sémeit, és monda néki: Nemde esküvéssel kényszerítettelek-é téged az Úrra, és bizonyságot tettem néked, ezt mondván: Valamely napon kimenéndesz, s ide s tova menéndesz, bizonynyal tudjad, hogy meghalsz; és azt mondád nékem: Tetszik e beszéd, megértettem.
\par 43 Miért nem tartottad hát meg az Úr elõtt való esküvést, és a parancsolatot, a melyet néked parancsoltam?
\par 44 Monda annakfelette a király Sémeinek: Te tudod mindazt a gonoszságot, a melyrõl a te szíved bizonyság, és a melyet atyámmal Dáviddal cselekedtél: az Úr most mindazt a gonoszságot a saját fejedre fordította.
\par 45 Salamon király pedig áldott lészen, és Dávidnak királyi széke lészen állandó az Úr elõtt mindörökké.
\par 46 És parancsola a király Benájának, a Jójada fiának, a ki elméne, és levágá Sémeit, és meghala. És az ország megerõsödék Salamon kezében.

\chapter{3}

\par 1 Sógorságot szerze azután Salamon a Faraóval, az Égyiptombeli királylyal; és elvevé a Faraó leányát, és hozá Dávidnak városába, míg elvégezé az õ  házának és az Úr házának építését és Jeruzsálemnek kõfalát köröskörül.
\par 2 De a nép áldozik vala a magas helyeken; mert nem építtetett vala ház az Úr nevének mind ez ideig.
\par 3 Szereté pedig Salamon az Urat, járván Dávidnak, az õ atyjának parancsolataiban, kivéve, hogy a magas helyeken  áldozott, és ott tömjénezett.
\par 4 És mikor Gibeonba ment a király, hogy ott áldozzék, mert ott volt a nagy magaslat, és Salamon azon az oltáron áldozott égõáldozatul ezer barmot:
\par 5 Megjelenék Gibeonban az Úr Salamonnak azon éjjel álmában, és monda az Isten: Kérj, a mit akarsz, hogy adjak néked.
\par 6 És monda Salamon: Te a te szolgáddal, az én atyámmal, Dáviddal nagy irgalmasságot cselekedtél, a miképen õ is járt elõtted híven, igazán és hozzád egyenes szívvel, és megtartottad néki ezt a nagy irgalmasságot, hogy néki fiat adtál, a ki az õ királyi székében ül, a mint e mai napon megtetszik.
\par 7 És most, óh én Uram Istenem, te tetted a te szolgádat királylyá, Dávid, az én atyám helyett. Én pedig kicsiny gyermek vagyok, nem tudok kimenni és bejönni.
\par 8 És a te szolgád a te néped között van, a melyet te magadnak választottál, nagy nép ez, a mely meg nem számláltathatik, meg sem írattathatik a sokaság miatt.
\par 9 Adj azért a te szolgádnak értelmes szívet, hogy tudja ítélni a te népedet, és tudjon választást tenni a jó és gonosz között; mert kicsoda kormányozhatja ezt a te nagy népedet?
\par 10 És tetszék e beszéd az Úrnak, hogy Salamon ilyen dolgot kért.
\par 11 Monda azért az Isten néki: Mivelhogy ezt kérted tõlem, és nem kértél magadnak hosszú életet, sem nem kértél gazdagságot, sem pedig nem kérted a te ellenségidnek lelkét; hanem bölcseséget kértél az ítélettételre:
\par 12 Ímé a te beszéded szerint cselekszem, ímé adok néked bölcs és értelmes szívet, úgy hogy hozzád hasonló nem volt te elõtted, és utánad sem támad olyan, mint te.
\par 13 Sõt még a mit nem kértél, azt is megadom néked, gazdagságot és dicsõséget: úgy hogy a királyok között nem lesz hozzád hasonló senki minden te idõdben.
\par 14 És ha az én útaimon járándasz, megõrizvén az én végzéseimet és parancsolatimat, a miképen járt a te atyád, Dávid: meghosszabbítom életed idejét.
\par 15 És mikor felserkent Salamon, ímé álom volt. És méne Jeruzsálembe, és álla az Úr szövetségének ládája elé, és áldozék egészen égõáldozatokat, és készíte hálaáldozatokat, és szerze nagy lakomát minden szolgáinak.
\par 16 Abban az idõben jött a királyhoz két parázna asszony, és megálla õ elõtte.
\par 17 És monda az egyik asszony: Kérlek uram, én és ez az asszony egy házban laktunk, és szültem õ nála abban a házban.
\par 18 És harmadnappal az én szülésem után, ez az asszony is szült, és együtt valánk, senki idegen nem volt velünk a házban, hanem csak mi ketten valánk abban a házban.
\par 19 És ennek az asszonynak éjszaka meghalt a fia: mert ráfeküdt.
\par 20 És felkelt éjfélkor, és elvitte az én fiamat mellõlem, mert a te szolgálóleányod aludt, és azt maga mellé fekteté, míg az õ meghalt fiát én mellém fektette.
\par 21 Mikor pedig hajnalban felkeltem, hogy megszoptassam az én fiamat: ímé, megholt; de reggel jól megnézegetvén, látám, hogy az nem az én fiam, a kit én szültem.
\par 22 Monda pedig a másik asszony: Nem úgy van, az én fiam az, a ki él, a te fiad pedig az, a ki meghalt. Amaz viszont monda: Nem, hanem a te fiad az, a ki meghalt, és az én fiam az, a ki él. És ekképen versengettek a király elõtt.
\par 23 Akkor monda a király: Ez azt mondja: Ez az én fiam, a ki él, és a te fiad az, a ki meghalt; amaz meg ezt mondja: Semmiképen nem, hanem a te fiad az, a ki meghalt, és az én fiam az, a ki él.
\par 24 És monda a király: Hozzatok nékem kardot! És mikor oda hozák a kardot a király elé,
\par 25 Monda a király: Vágjátok két részre az eleven gyermeket, és adjátok az egyik részt egyiknek, a másikat pedig a másiknak.
\par 26 Ekkor monda az az asszony, a kié vala az élõ gyermek, a királynak, mert megindult szíve gyermekén: Kérlek, uram, adjátok néki az élõ gyermeket, és ne öljétek meg õt. A másik pedig azt mondja vala. Se enyim, se tied ne legyen; vágjátok ketté.
\par 27 Akkor felele a király, és monda: Adjátok amannak az élõ gyermeket, és meg ne öljétek azt, mert az az õ anyja.
\par 28 És mikor hallotta az egész Izráel ezt az ítéletet, a melyet tett vala a király, félék a királynak orczáját, mert látták, hogy Isten bölcsesége van az õ szívében az ítélettételre.

\chapter{4}

\par 1 És lõn Salamon király az egész Izráel felett királylyá.
\par 2 Ezek valának pedig az õ fõemberei: Azária, Sádók papnak fia.
\par 3 Elihóref és Ahija, Sisának fiai íródeákok, Jósafát, az Ahilud fia, emlékíró vala.
\par 4 És Benája, a Jójada fia, a sereg hadnagya; Sádók pedig és Abjátár papok.
\par 5 És Azáriás, a Nátán fia, a tiszttartók elõljárója; és Zábud, a Nátán fia, fõtanácsos és a király barátja.
\par 6 És Ahisár udvarbíró; Adónirám pedig, az Abda fia, kincstartó.
\par 7 Vala pedig Salamonnak tizenkét tiszttartója az egész Izráelen, a kik ellátták a királyt és az õ háznépét: esztendõnként mindeniknek egy hónapig kellett az ellátásról gondot viselnie.
\par 8 És ezek azoknak neveik: Húrnak fia az Efraim hegyén.
\par 9 Dékernek fia Mákásban, Sahálbimban, Béth-Semesben és Elonban és Béth-Hanánban.
\par 10 Hésednek fia Arúbothban, õ hozzá tartozott Szókó és Héfer egész földe.
\par 11 Abinádáb fiaié volt Dór egész határa. A Salamon leánya, Táfát, vala néki felesége.
\par 12 Bahana, az Ahilud fia, bírja vala Taanákot és Megiddót és egész Béth-Seánt, mely Sartána mellett vala Jezréel alatt, Béth-Seántól fogva mind Abela Méholáig és mind Jokméámon túl.
\par 13 Gébernek fia, Rámóth Gileádban: övé valának Jáirnak, a Manasse fiának falui, melyek Gileádban valának; az övé volt az Argób  tartománya, mely Básánban vala, hatvan nagy város kõfallal és érczzárakkal megerõsítve.
\par 14 Ahinádáb, Iddó fia Mahanáimban.
\par 15 Ahimaás Nafthaliban; a Salamon leányát, Bosmátát vevé magának feleségül õ is.
\par 16 Bahana, Khúsai fia, Áserben és Alóthban.
\par 17 Jósafát, Paruákh fia, Issakhárban.
\par 18 Simei, Éla fia, Benjáminban.
\par 19 Géber, Uri fia, a Gileád földében, Sihonnak, az Emoreus királyának földében, és Ógnak, a Básánbeli királynak földében; és ez egy tiszttartó bírja vala azt a földet.
\par 20 És Júda és Izráel megsokasodott, olyan sok volt, mint a tenger mellett való föveny, és ettek, ittak és vigadtak.
\par 21 Salamon pedig uralkodék minden országokon a folyóvíztõl fogva egész a Filiszteusok földéig és Égyiptomnak határáig, és ajándékokat hoznak vala, és szolgálnak vala Salamonnak, életének minden idejében.
\par 22 És Salamon eledele naponként ez vala: harmincz véka zsemlyeliszt és hatvan véka közönséges liszt;
\par 23 Tíz hízlalt ökör, húsz füvön járt ökör és száz juh; a szarvasokon, õzeken, bivalokon és hízlalt madarakon kivül.
\par 24 Mert õ uralkodék minden helyeken a folyóvizen túl Thifsától fogva egész Gázáig; minden királyokon, a kik a folyóvizen túl valának; és békessége volt néki minden alattvalóitól köröskörül.
\par 25 És lakozék Júda és Izráel bátorsággal, kiki mind az õ szõlõtõje és fügefája alatt. Dántól fogva Bersebáig; Salamonnak minden idejében.
\par 26 És Salamonnak volt negyvenezer szekérbe való lova az istállókban, és tizenkétezer lovagja.
\par 27 És ezek a tiszttartók ellátták Salamon királyt és mindazokat, a kik a Salamon király asztalánál valának, kiki az õ hónapján, minden fogyatkozás nélkül.
\par 28 Azután árpát és szalmát is hoztak a lovaknak és a paripáknak arra a helyre, a hol a király volt, kiki az õ rendelete szerint.
\par 29 És az Isten adott bölcseséget Salamonnak és igen nagy értelmet és mély szívet, mint a fövény, mely a tenger partján van.
\par 30 Úgy hogy a Salamon bölcsesége nagyobb volt, mint a napkelet minden fiainak bölcsesége és Égyiptomnak egész bölcsesége.
\par 31 Sõt bölcsebb volt minden embereknél, még az Ezráhita Ethánnál is és Hémánnál,  Kálkólnál és Dardánál, a Máhol fiainál; és híre neve vala minden nemzetségek között köröskörül.
\par 32 És szerze háromezer példabeszédet, és az õ énekeinek száma ezer és öt volt.
\par 33 Szólott a fákról is, a Libánon czédrusfájától az izsópig, a mely a falból nevekedik ki; és szólott a barmokról, a madarakról, a csúszó-mászó állatokról és a halakról is.
\par 34 És jõnek vala minden népek közül, hogy hallgassák a Salamon bölcseségét; a földnek minden királyaitól, a kik hallották vala az õ bölcseségét.

\chapter{5}

\par 1 És Hírám, Tírus királya elküldé az õ szolgáit Salamonhoz, mikor meghallotta, hogy õt kenték királylyá az õ atyja helyett; mert Hírám szerette Dávidot teljes életében.
\par 2 És külde Salamon Hírámhoz, ezt izenvén néki:
\par 3 Te tudod, hogy Dávid, az én atyám nem építhete házat az Úrnak, az õ Istenének nevének a háborúk miatt, a melyekkel õt körülvették vala, mígnem az Úr az õ lábainak talpa alá vetette azokat;
\par 4 De most az Úr, az én Istenem nékem nyugodalmat adott mindenfelõl, úgy hogy semmi ellenségem és senkitõl semmi bántásom nincs.
\par 5 Ímé azt gondoltam magamban, hogy házat építek az Úrnak, az én Istenemnek nevének, a miképen szólott az Úr Dávidnak, az én atyámnak, ezt mondván: A te  fiad, a kit helyetted ültetek a te királyi székedbe, õ építi meg azt a házat az én nevemnek.
\par 6 Most azért parancsold meg, hogy vágjanak nékem czédrusfákat a Libánonon. Az én szolgáim is együtt lesznek a te szolgáiddal; a te szolgáidnak pedig jutalmát megadom néked mind a szerint, a mit mondándasz; mert tudod, hogy nincsen mi közöttünk olyan ember, a ki a favágáshoz úgy értene, mint a Sidonbeliek.
\par 7 Mikor azért meghallotta Hírám a Salamon izenetét, igen megörült, és monda: Áldott legyen e mai napon az Úr, a ki Dávidnak bölcs fiat adott e nagy népen.
\par 8 És elkülde Hírám Salamonhoz, azt izenvén: Megértettem a mi felõl küldöttél hozzám; én megteszem minden kivánságodat mind a czédrusfákra, mind a fenyõfákra nézve.
\par 9 Az én szolgáim a Libánonról aláviszik a tengerre a fákat: én pedig azokat tutajokra rakatván, a tengeren addig a helyig vitetem, a melyet te megizentetsz nékem, és azokat ott kihányatom, és te vitesd el. Te pedig abban teljesítsd kivánságomat, hogy adj eledelt az én háznépemnek.
\par 10 Ada azért Hírám Salamonnak czédrusfákat és fenyõfákat, minden kivánsága szerint.
\par 11 Salamon pedig ada Hírámnak húszezer véka búzát az õ háznépének táplálására, és húszezer kórus sajtolt olajat. Ezt adja vala Salamon Hírámnak esztendõrõl-esztendõre.
\par 12 Az Úr azért bölcsességet ada Salamonnak, a mint megmondotta vala néki: és békesség lõn Hírám és Salamon között, és õk szövetséget  tõnek egymással.
\par 13 Salamon király pedig robotosokat szedete az egész Izráelbõl; és harminczezer ember lõn robotossá.
\par 14 A kiket aztán elkülde a Libánonra, minden hónapra tíz-tízezer ember egymás után. Egy hónapig a Libánon hegyén valának, két hónapig az õ házoknál. Adónirám vala pedig a robotosok feje.
\par 15 Ezenkivül Salamonnak hetvenezer teherhordója, és nyolczvanezer kõvágója volt a hegyen.
\par 16 A pallérok fejedelmein kivül, a kikre Salamon a munkának igazgatását bízta volt, a kik háromezeren és háromszázan valának, a kik a munkálkodó népet szorgalmaztatták.
\par 17 És megparancsolá a király, hogy nagy és drága köveket vágjanak ki, nevezetesen faragott köveket a ház fundamentomául;
\par 18 Melyeket kifaragának a Salamon kõmívesei és a Hírám ácsai és a Gibleusok; és elkészíték a fákat és a köveket a ház építéséhez.

\chapter{6}

\par 1 És megépítteték az Úrnak háza az Izráel fiainak Égyiptom földébõl való kijövetele után a négyszáznyolczvanadik esztendõben, Salamon Izráel felett való uralkodásának negyedik esztendejében, a Zif hónapban, mely a második hónap.
\par 2 És a ház, a melyet Salamon király az Úrnak építe, hatvan sing hosszú, húsz sing széles és harmincz sing magas volt.
\par 3 És egy tornácz vala a ház temploma elõtt, a melynek a hossza húsz sing volt, a háznak szélessége szerint; a szélessége pedig tíz sing volt a ház hosszában.
\par 4 És építe a házon ablakokat is lezárt rostélyzattal.
\par 5 És építe a ház falaira emeleteket köröskörül, a ház falai körül a szent helyen és a szentek szentjén, és készíte mellék helyiségeket körül.
\par 6 Az alsó emelet belsõ szélessége öt sing, a középsõ szélessége hat sing és a harmadik szélessége hét sing volt, és bemélyedéseket építe a ház körül kivülrõl, hogy az emeletek gerendái ne nyúljanak be a ház falaiba.
\par 7 Mikor pedig a ház építteték, a kõbányának egészen kifaragott köveibõl építtetett, úgy hogy sem kalapácsnak, sem fejszének, sem valami egyéb vasszerszámnak pengése nem hallattatott a háznak felépítésénél.
\par 8 Az alsó emelet középsõ mellékhelyiségéhez egy ajtó vezetett a ház jobb oldalán, és egy csiga-grádics vitt fel a középsõ emeletbe, és a középsõbõl a harmadikba.
\par 9 Megépíté ekként azt a házat és elvégezé, és befedé a házat gerendákkal és czédrusfadeszkákkal.
\par 10 És megépíté az emeleteket az egész ház körül, a melyeknek magasságok öt-öt sing volt, és a házhoz czédrusfagerendákkal ragasztattak.
\par 11 És lõn az Úrnak beszéde Salamonhoz, ezt mondván:
\par 12 Ez ama ház, a melyet te építesz: Ha az én rendeléseimben  jársz, és az én ítéletim szerint cselekszel, és megtartod minden én parancsolatimat, azokban járván: Én is bizonyára megerõsítem veled az én beszédemet, a melyet szólottam Dávidnak, a te atyádnak;
\par 13 És az Izráel fiai között lakozom, és nem hagyom el az én népemet, az Izráelt.
\par 14 Megépíté azért Salamon azt a házat, és elvégezé azt.
\par 15 És megbéllelé a ház falait belõl czédrusfával, a ház padlózatától egészen a padlásig beborítá belõl fával; a ház padlózatát pedig beborítá fenyõdeszkákkal.
\par 16 És építe a ház hátulján egy húsz sing hosszú czédrusfa falat a padlózattól egész a padlásig, és építé azt a ház hátulsó részének; szentek-szentjének.
\par 17 Az elõtte való szenthely hossza negyven sing volt.
\par 18 Belülrõl az egész ház merõ czédrus volt, kivésett sártökökkel és kinyilt virágbimbókkal, úgy hogy semmi kõ ki nem látszott.
\par 19 És a szentek-szentjét építé a ház belsõ részében, hogy abba helyheztesse az Úr szövetségének ládáját.
\par 20 A szentek-szentje belsõ részének a hossza vala húsz sing, a szélessége is húsz sing, a magassága is húsz sing, és beborítá azt finom aranynyal; az oltárt is beborítá czédrusdeszkákkal.
\par 21 És Salamon beborította a házat belõl finom aranynyal, és arany lánczot vont a belsõ rész elõtt, a melyet szintén bevont aranynyal.
\par 22 Úgy, hogy az egész ház be volt vonva merõ aranynyal, sõt az oltárt is, a mely a szentek-szentje elõtt volt, egészen beborítá aranynyal.
\par 23 És csinált a szentek-szentjébe két tíz sing magas Kérubot olajfából.
\par 24 És öt sing volt az egyik Kérub szárnya, és öt sing volt a másik Kérub szárnya is, úgy, hogy az egyik szárnya végétõl, a másik szárnya végéig tíz sing vala.
\par 25 A másik Kérub is tíz sing volt; és mind a két Kérubnak mind a mértéke, mind a faragása egy vala;
\par 26 Úgy, hogy az egyik Kérub magassága tíz sing, és ugyanannyi a másik Kérubé is.
\par 27 És helyezteté a Kérubokat a ház belsejébe, és a mint a Kérubok kiterjeszték szárnyukat, az egyiknek szárnya a ház egyik falát, a másik Kérub szárnya pedig a másik falát érte; de a ház közepén összeért egyik szárny a másikkal.
\par 28 És beborítá a Kérubokat aranynyal.
\par 29 És a ház összes falain köröskörül kivül és belõl Kérubokat, pálmafákat és kinyilt virágokat metszetett ki.
\par 30 És beborítá még a ház padlóját is aranynyal kivül és belõl.
\par 31 És a szentek-szentjének bemenetelén csinála ajtót olajfából, az ajtófélfák kiszögellése egy ötödrész volt;
\par 32 És két ajtószárnyat olajfából, és metszete reájok Kérubokat, pálmafákat és kinyilt virágokat, és beborítá azokat aranynyal; a Kérubokat is és a pálmafákat is megaranyoztatá.
\par 33 E képen csinált a templom szenthelyénél is négyszegletû ajtófélfákat olajfából.
\par 34 És csinált két ajtót cziprusfából, és az egyik ajtón is két forgó ajtószárny volt, a másik ajtón is két forgó ajtószárny.
\par 35 És metszete azokra Kérubokat és pálmafákat és kinyilt virágokat, és beborítá aranynyal, rá alkalmaztatva azt a metszésre.
\par 36 Azután felépíté a belsõ pitvart három rend faragott kõbõl és egy rend czédrusgerendából.
\par 37 Az õ uralkodásának negyedik esztendejében fundáltaték az Úrnak háza Zif hónapban.
\par 38 És a tizenegyedik esztendõben, Búl havában, (mely a nyolczadik hónap) végezteték el a ház, minden dolga és rendje szerint. És így építé azt hét esztendeig.

\chapter{7}

\par 1 Azután a maga házát építé Salamon tizenhárom esztendeig, a mely alatt elvégezé az õ házát egészen.
\par 2 Megépíté a Libánon erdõ házát is, melynek hossza száz sing vala, szélessége ötven sing, magassága harmincz sing; építé azt négy rend czédrusoszlopon és az oszlopokon czédrusgerendák valának.
\par 3 És bepadlá czédrusdeszkákkal felül a gerendák felett, melyek valának negyvenöt oszlopon, mindenik renden tizenöt.
\par 4 És három rend ablak rajta egymással átellenben, három-három ellenében.
\par 5 És mind az ajtók és azoknak oldalfái négyszögûek valának az ablakokkal együtt, és egyik ablak a másiknak átellenébe volt mind a három renden.
\par 6 És építé az oszlopcsarnokot, a melynek hossza ötven sing és szélessége harmincz sing volt; és egy tornáczot ez elé, és oszlopokat és vastag gerendákat ezek elé.
\par 7 És építé a trón-termet, a hol ítélt, a törvényházat, a melyet czédrusfával bélelt meg a padlózattól fogva fel a padlásig.
\par 8 Azután a saját házát építé, a melyben õ maga lakott, a másik udvarba befelé a teremtõl, hasonlóan a másikhoz, és építe egy házat a Faraó leányának is, a kit feleségül vett Salamon, hasonlót e teremhez.
\par 9 Mindezek drágakövekbõl voltak, mérték szerint kifaragva, fûrészszel metszve minden oldalról, a fundamentomtól a tetõzetig, kivül is mind a nagy pitvarig.
\par 10 Még a fundamentom is drága és nagy kövekbõl volt: tíz singnyi kövekbõl és nyolcz singnyi kövekbõl.
\par 11 És ezeken felül voltak a mérték szerint faragott drágakövek és czédrusfák.
\par 12 És a nagy pitvarban köröskörül három sor faragott kõ és egy sor faragott czédrusgerenda volt, épen mint az Úr házának belsõ pitvara és a ház tornácza.
\par 13 És elkülde Salamon király, és elhozatá Hírámot Tírusból.
\par 14 Ez egy özvegy asszonynak volt a fia a Nafthali nemzetségébõl; az õ atyja pedig Tírusbeli rézmíves ember vala; és ez teljes vala bölcseséggel, értelemmel és tudománynyal, hogy tudna csinálni mindenféle mívet rézbõl. Ki mikor Salamon királyhoz jött, minden mívet megcsinála néki.
\par 15 És formála két réz oszlopot, az egyik oszlop magassága tizennyolcz sing volt, és tizenkét sing zsinór éri vala át mind a két oszlopot.
\par 16 És készíte két gömböt érczbõl öntve, hogy azokat az oszlopok tetejére tegye, és öt sing magas volt az egyik gömb és öt sing magas volt a másik gömb.
\par 17 Reczés mívû hálók, lánczmívû zsinórok voltak a gömbökön, a melyek az oszlopok tetején valának; hét volt az egyik gömbön, és hét volt a másik gömbön is.
\par 18 És megkészíté az oszlopokat, és két sor díszítést tett köröskörül az egyik hálón, hogy befedje a gömböket, a melyek az oszlopfõkön voltak; és így csinálá a másik gömböt is.
\par 19 És a gömbök, a melyek a tornáczban levõ oszlopok tetején voltak, liliom formájúak voltak, négy singnyiek.
\par 20 Gömbök voltak a két oszlopon, felül, közel a kidomborodáshoz, a mely a háló mellett volt. És kétszáz gránátalma volt sorban köröskörül a második gömbön.
\par 21 És felállítá az oszlopokat a templom tornáczában; és felállítá a jobb oszlopot, és nevezé annak nevét Jákinnak, és felállítá a bal oszlopot, és nevezé annak nevét Boáznak.
\par 22 És az oszlopok tetején liliomok formáltattak vala. És ilyen módon végezteték el az oszlopok míve.
\par 23 És csinála egy öntött tengert, mely egyik szélétõl fogva a másik széléig tíz sing volt, köröskörül kerek, és öt sing magas, és a kerületit harmincz sing zsinór érte vala körül.
\par 24 Valának pedig a peremén alól köröskörül formáltatva apró sártökök; tíz-tíz mindenik singben az egész tenger körül, az ilyen sártököcskék két renddel valának öntve köröskörül a maga öntésében.
\par 25 És tizenkét ökrön állott, három északra fordulva, három nyugotra, három délre és három naptámadatra, és a tenger fölül rajtok, hátok pedig mind befelé.
\par 26 És a vastagsága egy tenyérnyi volt, és a pereme olyan, mint a pohár ajaka, vagy a liliom virága, és kétezer báth fért bele.
\par 27 És készíte tíz ércz-talpat, mindegyik talpat négy sing hosszúra, és négy sing szélesre, és három sing magasra.
\par 28 És e talpak így voltak csinálva: oldalaik voltak, és az oldalak a szélpártázatok között voltak.
\par 29 És az oldalakon, a melyek a pártázatok között voltak, oroszlánok, ökrök és Kérubok voltak, és a pártázatokon felül is ekként; az oroszlánok és ökrök alatt pedig czifrázatok voltak bevésett munkával.
\par 30 És mindenik talpnak négy-négy réz kereke és réz tengelye volt, és a négy szegleten támaszok voltak; a mosdómedenczén alul voltak e támaszok öntve, és mindegyiknek oldalán czifrázatok.
\par 31 És a szája az õ kerekded fészkének belsõ részétõl fogva oda felfelé egy singnyi volt, és a fészeknek szája kerekded vala, oszlopformára csinálva, másfél singnyi széles, és szájánál is szép metszések valának, és azoknak pártázatai négyszögûek valának, nem gömbölyûek.
\par 32 És négy kerék volt a pártázatok alatt, és a kerekek tengelyei a talphoz voltak erõsítve, és mindenik keréknek magassága másfél sing vala.
\par 33 És e kerekek hasonlóak valának a szekérnek kerekeihez, csakhogy a tengelyeik, kerékagyaik, küllõik, talpaik mind öntve valának.
\par 34 És négy vállacskát csinált mindenik talp négy szegletén; magából a talpból jöttek ki a vállacskák.
\par 35 És e talp tetején fél singnyi kerekded magasság volt köröskörül, és a talp tetején voltak annak tartókezei és pártázatai a maga öntésébõl.
\par 36 Metsze pedig annak tábláira, tartókezeire, pártázataira Kérubokat, oroszlánokat és pálmafákat: mindeniknek az üres helye szerint, és koszorút köröskörül.
\par 37 Így készítette a tíz talpat egy öntésbõl, egy mérték és forma szerint.
\par 38 És csinála tíz mosdómedenczét is, rézbõl, és mindenik mosdómedenczébe negyven báth fér vala; és mindenik mosdómedencze négy singnyi vala, és a tíz talp mindenikén egy-egy mosdómedencze vala.
\par 39 És helyhezteté a talpak ötét a ház jobbfelõl való részére, és ötét a ház balfelõl való részére; a tengert pedig helyhezteté a ház jobbrésze felõl naptámadatra dél ellenébe.
\par 40 És készített Hírám még üstöket, lapátokat és medenczéket, és elvégezé az egész munkát, a melyet Salamon királynak csinált az Úr házához;
\par 41 Tudniillik a két oszlopot és a kerek gömböket, a melyek a két oszlop tetejére tétettek, és a két hálót a két kerek gömb befedezésére, a melyek az oszlopok tetejére tétettek.
\par 42 És a négyszáz gránátalmát a két hálóra; két rend gránátalmát minden hálóba, a két kerekded gömb befejezésére, a melyek valának az oszlopok tetején;
\par 43 A tíz talpat és a talpakra való tíz mosdómedenczét;
\par 44 Az egy tengert és a tizenkét ökröt a tenger alá;
\par 45 Fazekakat, lapátokat és medenczéket. És mindezek az edények, a melyeket Hírám Salamon királynak az Úr háza számára készített, csiszolt rézbõl voltak.
\par 46 A Jordán völgyében önteté ezeket a király az agyagos földben, Sukhót és Sártán között.
\par 47 És mindezeket az edényeket Salamon méretlen hagyá, a réznek felettébb való sokasága miatt.
\par 48 És megcsináltata Salamon minden egyéb felszerelést is, mely az Úr házához szükséges volt: az arany oltárt, az arany asztalt, melyen  a szent kenyerek állottak.
\par 49 És a gyertyatartókat színaranyból, ötöt jobbfelõl és ötöt balfelõl a szentek-szentje elé, és arany virágokat, lámpákat, és hamvvevõket.
\par 50 Azután csészéket, késeket, medenczéket, tömjénezõket és serpenyõket színaranyból, sõt a belsõ ház, a szentek-szentje és a szenthely ajtainak sarkait is mind aranyból.
\par 51 És ilyenképen elvégezteték az egész mû, a melyet Salamon király csinála az Úrnak házához. És bevivé Salamon az õ atyjától, Dávidtól az Istennek szenteltetett jószágot, az ezüstöt, aranyat és az edényeket és azokat is az Úr házának kincsei közé tevé.

\chapter{8}

\par 1 Akkor összegyûjté Salamon az Izráel véneit és a nemzetségeknek minden fejeit, az Izráel fiai atyjaiknak fejedelmeit Salamon királyhoz Jeruzsálembe, hogy az Úr szövetségének ládáját felvigyék a Dávid városából, mely a Sion.
\par 2 És felgyûlének Salamon királyhoz az Izráel minden férfiai az Ethánim havában, az ünnepen; ez a hetedik hónap.
\par 3 Mikor pedig eljöttek mindnyájan az Izráel vénei: felvevék a papok a ládát,
\par 4 És felvivék az Úr ládáját, a gyülekezetnek sátorát, és mind a szent edényeket, melyek valának a sátorban, és felvivék azokat a papok és a Léviták.
\par 5 És Salamon király és az Izráel egész gyülekezete, a mely õ hozzá gyûlt, megyen vala õ vele a láda elõtt, áldozván juhokkal és ökrökkel, melyek meg sem számláltathatnának, sem pedig meg nem irattathatnának a sokaság miatt.
\par 6 És bevivék a papok az Úr szövetségének ládáját az õ helyére a ház belsõ részébe, a szentek-szentjébe, a Kérubok szárnyai alá.
\par 7 Mert a Kérubok kiterjesztik vala szárnyaikat a láda helye felett, és befedik vala a Kérubok a ládát és annak rúdjait felülrõl.
\par 8 És a rudak olyan hosszúak voltak, hogy azok vége látható volt a szenthelyen a szentek-szentjének elsõ része felõl; azonban kivül nem voltak láthatók; és ott vannak mind e mai napig.
\par 9 És nem volt egyéb a ládában, mint csak a két kõtábla, a melyeket Mózes a Hórebnél helyezett bele, mikor az Úr szövetséget kötött az Izráel fiaival, mikor kijövének Égyiptom földébõl.
\par 10 Mikor pedig kijöttek a papok a szenthelybõl: köd tölté be az Úrnak házát.
\par 11 Úgy, hogy meg sem állhattak a papok az õ szolgálatjokban a köd miatt; mert az Úr dicsõsége töltötte vala be az Úrnak házát.
\par 12 Akkor monda Salamon: Az Úr mondotta, hogy õ lakoznék ködben.
\par 13 Építve építettem házat néked lakásul; helyet, a hol örökké lakjál.
\par 14 Azután megfordult a király, és megáldá Izráel egész gyülekezetét és az Izráel egész gyülekezete felállott.
\par 15 És monda: Áldott legyen az Úr, Izráelnek Istene, a ki szólott az õ szája által Dávidnak, az én atyámnak; és azt az õ hatalmasságával beteljesítette, mondván:
\par 16 Attól a naptól fogva, a melyen kihoztam az én népemet, az Izráelt Égyiptomból, soha nem választottam egyetlen várost sem az Izráel minden nemzetségei közül, hogy ott nékem házat építenének, a melyben lenne az én nevem; hanem csak Dávidot választottam, hogy õ legyen az én népem, Izráel felett.
\par 17 És ámbár az én atyám, Dávid már elvégezte volt, hogy õ épít házat az Úrnak, Izráel Istene nevének;
\par 18 De az Úr azt mondá Dávidnak, az én atyámnak: Azt, hogy arra gondoltál, hogy az én nevemnek házat építs, jól cselekedted, hogy szívedben ezt végezted;
\par 19 Mégis nem te építesz házat nékem, hanem a te fiad, a ki a te ágyékodból származik, az épít házat az én nevemnek.
\par 20 És beteljesíté az Úr az õ beszédét, a melyet szólott; mert felkelék az én atyám, Dávid helyett, és ülék az Izráel királyi székibe, a miképen megmondotta vala az Úr, és megépítém a házat az Úr, Izráel Istene nevének.
\par 21 És helyet szerzettem ott a ládának, a melyben az Úrnak szövetsége vagyon, a melyet szerzett a mi atyáinkkal, mikor kihozta õket Égyiptom földébõl.
\par 22 És oda állott Salamon az Úr oltára elé, az Izráel egész gyülekezetével szembe, és felemelé kezeit az ég felé,
\par 23 És monda: Uram, Izráel Istene! nincsen hozzád hasonló Isten, sem az égben ott fenn, sem a földön itt alant, a ki megtartod a szövetséget és az irgalmasságot a te szolgáidnak, a kik te elõtted teljes szívvel járnak;
\par 24 A ki megtartottad azt, a mit a te szolgádnak, Dávidnak, az én atyámnak szólottál; mert magad szólottál néki, és a te hatalmaddal beteljesítetted, a mint e mai napon megtetszik.
\par 25 Most azért Uram, Izráel Istene, tartsd meg, a mit a te szolgádnak, Dávidnak, az én atyámnak igértél, ezt mondván: A te magodból való férfiú el nem fogy én elõttem, a ki az Izráel királyi székibe üljön; csakhogy a te fiaid õrizzék meg az õ útjokat, hogy én elõttem járjanak, a mint te én elõttem jártál.
\par 26 Most teljesítsd be, Izráel Istene, a te szavaidat, a melyeket szólottál a te szolgádnak, Dávidnak, az én atyámnak.
\par 27 Vajjon gondolható-é, hogy lakozhatnék az Isten a földön? Ímé az ég, és az egeknek egei be nem foghatnak téged; mennyivel kevésbbé e ház, a melyet én építettem.
\par 28 De tekints a te szolgád imádságára és könyörgésére, óh Uram, én Istenem, hogy meghalljad a dicséretet és az imádságot, a melylyel a te szolgád könyörög elõtted e mai napon;
\par 29 Hogy a te szemeid e házra nézzenek éjjel és nappal, e helyre, a mely felõl azt mondottad: Ott lészen az én nevem; hallgasd meg ez imádságot, a melylyel könyörög a te szolgád e helyen.
\par 30 És hallgasd meg a te szolgádnak és a te népednek, az Izráelnek könyörgését, a kik imádkozándanak e helyen; hallgasd meg lakóhelyedbõl, a mennyekbõl, és meghallgatván légy kegyelmes!
\par 31 Mikor vétkezéndik valaki felebarátja ellen, és esküre köteleztetik, hogy megesküdjék és õ ide jõ, megesküszik az oltár elõtt ebben a házban:
\par 32 Te hallgasd meg a mennyekbõl, és vidd véghez, és tégy igazat a te szolgáid között, kárhoztatván az istentelent, hogy fején teljék a mit keresett; és megigazítván az igazat, megfizetvén néki az õ igazsága szerint.
\par 33 Mikor megverettetik a te néped, az Izráel, az õ ellenségeitõl, mivel ellened vétkeztek, és hozzád megtéréndenek, és vallást teéndenek a te nevedrõl, és néked imádkozándanak és könyörgéndenek e házban:
\par 34 Te hallgasd meg a mennyekbõl, és bocsásd meg az Izráelnek, a te népednek vétkét; és hozd vissza õket arra a földre, a melyet adtál az õ atyáiknak.
\par 35 Mikor berekesztetik az ég, és nem lészen esõ, mert ellened vétkeztek; és imádkozándanak e helyen, és vallást teéndenek a te nevedrõl, és megtéréndenek az õ bûnökbõl, mert te szorongatod õket:
\par 36 Te hallgasd meg õket a mennyekbõl, és légy kegyelmes a te szolgáidnak és az Izráelnek, a te népednek vétke iránt, tanítsd meg õket a jó útra, a melyen járjanak; és adj esõt a te földedre, a melyet örökségül adtál a te népednek.
\par 37 Éhség ha lesz e földön, ha döghalál, aszály, ragya, sáska, cserebogár; ha ellenség szállja meg kapuit; vagy más csapás és nyavalya jövénd reájok:
\par 38 A ki akkor könyörög és imádkozik, legyen az bárki; vagy a te egész néped, az Izráel, ha elismeri kiki az õ szívére mért csapást, és kiterjeszténdi kezeit e ház felé:
\par 39 Te hallgasd meg a mennyekbõl, a te lakhelyedbõl és légy kegyelmes, és cselekedd azt, hogy kinek-kinek fizess az õ útai szerint, a mint megismerted az õ szívét, mert egyedül csak te ismered minden embernek szívét.
\par 40 Hogy féljenek téged mind éltig, míg e földnek színén lakoznak, a melyet adtál a mi atyáinknak.
\par 41 Sõt még az idegen is, a ki nem a te néped, az Izráel közül való, ha eljövénd messze földrõl a te nevedért;
\par 42 (Mert meghallják a te nagyságos nevedet és a te hatalmas kezedet és kinyújtott karodat), és eljövén imádkozánd e házban:
\par 43 Te hallgasd meg a mennyekbõl, a te lakhelyedbõl, és add meg az idegennek mindazt, a miért könyörög néked, hogy mind az egész földön való népek megismerjék a te nevedet, és féljenek úgy téged, miképen a te néped az Izráel; és ismerjék meg, hogy a te nevedrõl neveztetik ez a ház, a melyet én építettem.
\par 44 Ha a te néped hadba megy ki az õ ellensége ellen, az úton, a melyen te küldöd el, és imádkozándik az Úrhoz, fordulván az úton e város felé, a melyet te magadnak választottál és e ház felé, a melyet építettem a te nevednek:
\par 45 Hallgasd meg a mennyekbõl az õ imádságokat és könyörgésöket, és szerezz nékik igazságot.
\par 46 Ha te ellened vétkezéndenek, - mert nincsen ember, a ki ne vétkeznék - és megharagudván reájok, ellenség kezébe adándod, és fogva elviéndik õket azok, a kiktõl megfogattak, az ellenségnek földére, messze vagy közel;
\par 47 És eszökre térnek a földön, melyre fogva vitettek, és megtérvén könyörgenek néked azoknak földökön, a kiktõl fogva elvitettek, mondván: Vétkeztünk, hamisan és istentelenül cselekedtünk!
\par 48 És megtéréndenek te hozzád teljes szívökbõl és lelkökbõl, az õ ellenségöknek földében, a kik õket fogva elvitték, és imádkozándanak hozzád az õ földöknek  útja felé fordulva, a melyet adtál az õ atyáiknak, és e város felé, a melyet magadnak választottál, és e ház felé, a melyet a te nevednek építettem:
\par 49 Hallgasd meg a mennyekbõl, a te lakhelyedbõl az õ imádságokat és könyörgésöket, és szerezz nékik igazságot;
\par 50 És légy kegyelmes a te néped iránt, a kik ellened vétkezéndenek, és minden bûneik iránt, a melyekkel ellened vétkeztek; és szerezz kedvességet nékik azok elõtt, a kik õket fogva tartják, hogy könyörüljenek rajtok;
\par 51 Mert õk a te néped és örökséged, a kiket kihoztál Égyiptomból, a vaskemencze közepébõl,
\par 52 Hogy a te szemeid nézzenek a te szolgádnak imádságára és a te népednek az Izráelnek könyörgésére; meghallgatván õket mindenkor, mikor téged segítségül hívnak.
\par 53 Mert te különválasztottad õket magadnak örökségül a földnek minden népei közül; a miképen megmondottad volt Mózes, a te szolgád által, mikor kihoztad a mi atyáinkat Égyiptomból, Uram Isten!
\par 54 És mikor elvégezte Salamon az Úr elõtt való minden imádságát és könyörgését, felkele az Úr oltára elõl, és megszünék az õ térdein való állástól, és kezeinek az égbe felemelésétõl.
\par 55 És felállván megáldá az Izráel egész gyülekezetét, felszóval ezt mondván:
\par 56 Áldott legyen az Úr, a ki nyugodalmat adott az õ népének, az Izráelnek, minden õ beszéde szerint, csak egy beszéde is hiábavaló nem volt minden õ jó beszédei közül, a melyeket szólott Mózes, az õ szolgája által.
\par 57 Az Úr, a mi Istenünk, legyen velünk, a miképen volt a mi atyáinkkal, ne hagyjon el minket, el se távozzék tõlünk,
\par 58 Hanem hajtsa magához a mi szívünket, hogy járjunk minden õ útaiban, és õrizzük meg az õ parancsolatit, rendeléseit és végzésit, a melyeket a mi atyáinknak parancsolt.
\par 59 És ezek a szavak, a melyekkel imádkoztam az Úr elõtt, legyenek jelen az Úr elõtt a mi Istenünk elõtt éjjel és nappal, hogy ítéletet tegyen az õ szolgájának és az õ népének, az Izráelnek, minden idõben,
\par 60 Hogy megismerjék a földön minden népek, hogy csak az Úr az Isten, és hogy õ kívülötte nincsen más.
\par 61 És a ti szívetek legyen tökéletes az Úrhoz, a mi Istenünkhöz, hogy járjatok az õ rendeléseiben, és õrizzétek meg az õ parancsolatit, miképen e mai napon.
\par 62 És a király és az egész Izráel õ vele, áldozatokat áldozának az Úr elõtt.
\par 63 És Salamon áldozék hálaadó áldozatul, a melyet áldozék az Úrnak: huszonkétezer ökröt, százhúszezer juhot. És e képen szentelék fel az Úr házát a király és az Izráel minden fiai.
\par 64 Ugyanazon napon szentelé fel a király a középsõ pitvart, mely az Úrnak háza elõtt vala; mert ott szerze egészen égõáldozatokat, ételáldozatokat és hálaáldozatok kövéreit. Mert a rézoltár, mely az Úr elõtt állott, kisebb volt, mintsem reá fért volna az égõáldozat, az ételáldozat és a hálaáldozatok kövére.
\par 65 És Salamon ünnepet szerze ebben az idõben, és vele együtt az egész Izráel; egy nagy gyûlést Hámát határától fogva Égyiptom határáig, az Úr elõtt, a mi Istenünk elõtt, hét napig és újra hét napig, azaz tizennégy napig.
\par 66 És a nyolczadik napon elbocsátá a népet. És áldák a királyt, és elmenének az õ hajlékaikba örömmel és víg szívvel, mindama jók felett, a melyeket cselekedett az Úr az õ szolgájával,  Dáviddal, és az Izráellel, az õ népével.

\chapter{9}

\par 1 És lõn, mikor elvégezte Salamon az Úr házának és a király házának építését, és mindent a mit kívánt és a mit akart építeni Salamon:
\par 2 Megjelenék az Úr Salamonnak másodszor is, a miként megjelent volt néki Gibeonban.
\par 3 És monda néki az Úr: Meghallgattam a te imádságodat és könyörgésedet, a melylyel könyörgöttél elõttem: Megszenteltem e házat, a melyet építettél, abba helyheztetvén az én nevemet mindörökké, és ott lesznek az én szemeim, és az én szívem  mindenkor.
\par 4 És ha te elõttem járándasz, a mint járt Dávid, a te atyád, egyenes és tökéletes szívvel, úgy cselekedvén mindenekben, a mint néked megparancsoltam, az én rendelésimet és végzésimet megtartándod;
\par 5 Megerõsítem a te birodalmadnak trónját az Izráelen mindörökké, a mint megígértem volt Dávidnak a te atyádnak, mondván: Nem fogy el a te nemzetségedbõl  való férfiú az Izráel királyi székébõl.
\par 6 De hogyha elszakadtok ti és a ti fiaitok én tõlem, és meg nem õrizénditek az én parancsolatimat és végzéseimet, melyeket elõtökbe adtam; hanem elmentek, és idegen isteneknek szolgáltok, és meghajoltok azok elõtt:
\par 7 Kigyomlálom az Izráelt e föld színérõl, a melyet nékik adtam; e házat, melyet az én nevemnek szenteltem, elvetem szemeim elõl, és az Izráel példabeszédül és meséül lészen minden nép elõtt.
\par 8 És bár e ház felséges, mégis a kik elmennek mellette, elcsodálkoznak, felkiáltanak, és azt mondják: Miért cselekedett így az Úr ezzel a földdel és ezzel a házzal?
\par 9 És azt felelik: Azért, mert elhagyták az Urat, az õ Istenöket, a ki az õ atyáikat kihozta volt Égyiptom földébõl, és idegen istenekhez ragaszkodtak, és azokat imádták, és azoknak szolgáltak: ezért bocsátá õ reájok az Úr mind ezt a nyomorúságot.
\par 10 És lõn a húsz esztendõ végén, a mialatt Salamon a két házat, az Úr házát és a király házát megépíté,
\par 11 A melyekhez Hírám, Tírus királya adott volt ajándékban Salamonnak czédrusfákat, fenyõfákat, aranyat egész kivánsága szerint: ada Salamon király Hírámnak húsz várost Galileának földén.
\par 12 És kiméne Hírám Tírusból, hogy megnézze azokat a városokat, a melyeket Salamon néki ada, de nem tetszettek azok néki.
\par 13 És monda: Miféle városok ezek, atyámfia, a melyeket nékem adtál? És Kábul földnek nevezé azokat mind e mai napig.
\par 14 Küldött vala pedig Hírám a királynak százhúsz tálentom aranyat.
\par 15 És ez az összege annak az adónak is, a melyet kivetett volt Salamon király, hogy megépíthesse az Úr házát, és a maga házát, és Millót, és Jeruzsálem kõfalait, és Kháczort, Megiddót és Gézert.
\par 16 Mert a Faraó, Égyiptom királya, feljött volt, és meghódítá Gézert, és felégette tûzzel, és a Kananeusokat, a kik a városban laktak, megölte, és adá azt ajándékban az õ leányának, a  Salamon feleségének.
\par 17 És megépíté Salamon Gézert és alsó Bethoront;
\par 18 Bahalátot és Thadmort a pusztában, azon a földön;
\par 19 És a tárházak minden városait, a melyek a Salamonéi valának, a szekerek városait, és a lovagok városait, és mindeneket, a melyeknek építéséhez Salamonnak kedve volt Jeruzsálemben és a Libánonon, és az õ birodalmának egész földén.
\par 20 És mindazt a népet, a mely megmaradott volt az Emoreusoktól, Hitteusoktól, Perizeusoktól, Hivveusoktól, Jebuzeusoktól, a kik nem valának az Izráel fiai közül.
\par 21 Azoknak fiait, a kik õ utánok azon a földön maradtak volt, a kiket az Izráeliták ki nem irthattak, Salamon jobbágyokká tette mind e mai napig.
\par 22 De az Izráel fiai közül senkit nem vetett Salamon szolgálat alá, hanem ezek hadakozó férfiak voltak és õ szolgái és fõemberei és hadnagyai és az õ szekereinek és lovagjainak fejei.
\par 23 És a hivatalnokoknak, a kik Salamon munkáinak élén állottak, száma ötszázötven volt, a kik igazgatták a népet, a mely dolgozott a munkán.
\par 24 És a Faraó leánya felméne a Dávid városából a maga házába, a melyet Salamon épített néki. Akkor építé meg Millót is.
\par 25 És áldozék Salamon minden esztendõben háromszor, égõ és hálaáldozatot azon az oltáron, a melyet épített vala az Úrnak, és áldozik vala jóillattal azon, a mely az Úr elõtt vala. És elvégezé a házat.
\par 26 És hajókat is csináltata Salamon király Esiongáberben, a mely Elót mellett van a Veres tenger partján, az Edom földén.
\par 27 És elküldé Hírám az õ szolgáit a hajókon, a kik jó hajósok és a tengeren jártasak valának, a Salamon szolgáival.
\par 28 És egész Ofirig menének, és hozának onnét négyszázhúsz tálentom aranyat, és vivék azt Salamon királyhoz.

\chapter{10}

\par 1 A Séba királynéasszonya pedig hallván Salamonnak hírét és  az Úr nevét, eljöve, hogy megkisértgesse õt nehéz kérdésekkel.
\par 2 És bejöve Jeruzsálembe igen nagy sereggel és tevékkel, a melyek hoznak vala fûszerszámokat, igen sok aranyat és drágaköveket, és Salamonhoz méne, és szóla vele mindenekrõl, a melyek szívén voltak.
\par 3 És Salamon megfelelt néki mindenre, semmi sem volt a király elõtt elrejtve, a mire ne tudott volna néki felelni.
\par 4 És a mikor látta Séba királynéasszonya Salamonnak minden bölcseségét, és a házat, a melyet épített vala;
\par 5 És az õ asztalának étkeit, és szolgáinak lakásait, és szolgái udvarlásának módját, és azok öltözeteit, és pohárszékeit, és az õ áldozatját, a melylyel az Úrnak házában áldozott: a lélekzete is elállott;
\par 6 És monda a királynak: Mind igaz, a mit az én földemben hallottam volt a te dolgaid felõl és a te bölcseségedrõl.
\par 7 De én hinni sem akartam azokat a beszédeket, míg én magam el nem jöttem, és szemeimmel nem láttam. És ímé nékem a felét sem beszélték el: te meghaladtad bölcseséggel és jósággal a hírt, a melyet hallottam felõled.
\par 8 Boldogok a te embereid, boldogok ezek a te szolgáid, a kik udvarlanak néked mindenkor, és hallhatják a te bölcseségedet:
\par 9 Legyen az Úr, a te Istened áldott, a ki kedvelt téged, hogy az Izráel királyi székibe ültetett, mert szerette az Úr az Izráelt mindörökké, és királylyá tett téged, hogy ítéletet és igazságot szolgáltass.
\par 10 És ada a királynak száz és húsz tálentom aranyat, és igen sok fûszerszámot és drágakövet. Nem hoztak azután ilyen és ennyi sokaságú fûszerszámot, a mennyit Séba királynéasszonya ada Salamon királynak.
\par 11 És a Hírám hajója is, mely aranyat hozott Ofirból, ébenfát is hozott Ofirból nagy bõséggel és drágaköveket.
\par 12 És csinála a király az ébenfából oszlopokat az Úr házába és a király házába, és az éneklõknek hegedûket és lantokat; nem hoztak soha többé olyan ébenfákat, és nem is láttak olyanokat mind e mai napig.
\par 13 És Salamon király ada a Séba királynéasszonyának mindent, a mit csak kívánt és kért tõle, azokon kivül, a melyeket gazdagságához képest ada Salamon király néki. Annakutána megtére, és méne az õ földébe szolgáival egyetemben.
\par 14 Vala pedig mértéke az aranynak, a mely kezéhez jõ vala Salamonnak minden esztendõben, hatszáz és hatvanhat tálentom arany.
\par 15 Azonkivül, a mi jõ vala az árus emberektõl és a fûszerszámokkal kereskedõ kalmároktól, és mind az Arábiabeli királyoktól, és annak a földnek tiszttartóitól.
\par 16 És csináltata Salamon király kétszáz paizst tiszta vert aranyból; mindenik paizsra hatszáz aranyat adott.
\par 17 És háromszáz kerek paizst vert aranyból, három font aranyat ada mindenik paizsra: és azokat a király helyhezteté a Libánon erdõ házába.
\par 18 És készíte a király elefántcsontból egy nagy királyi széket és beborítá azt finom aranynyal.
\par 19 Hat grádicsa volt e királyi széknek és e szék teteje kerekded vala hátul, és karjai valának mindkétfelõl az ülés mellett, és két oroszlán álla ott a karoknál.
\par 20 És tizenkét oroszlán álla ott kétfelõl a grádics hat lépésén. Senki soha olyant nem csinált egyetlen országban sem.
\par 21 És Salamon királynak összes ivóedényei is aranyból voltak, és a Libánon erdõ házának összes edényei tiszta aranyból; nem volt azok között semmi ezüst, mert annak semmi becse nem vala Salamon idejében.
\par 22 Mert a király Társis hajója, a mely a tengeren Hírám hajójával járt, minden három esztendõben egyszer fordult meg, s hozott a Társis hajó aranyat, ezüstöt, elefántcsontokat, majmokat és pávákat.
\par 23 És feljebb magasztaltaték Salamon király gazdagsággal és bölcseséggel a földön való minden királyoknál.
\par 24 És mind az egész föld kivánja vala látni Salamont, hogy hallhatnák az õ bölcseségét, melyet Isten az õ szívébe adott volt.
\par 25 És azok néki ajándékot hoznak vala, ezüst és arany edényeket, öltözeteket, hadi szerszámokat, fûszerszámokat, lovakat, öszvéreket, esztendõnként.
\par 26 És gyûjte Salamon szekereket és lovagokat, úgy hogy ezer és négyszáz szekere, és tizenkétezer lovagja volt néki, a kiket helyheztete a szekerek városaiba, és a király mellé Jeruzsálemben.
\par 27 És felhalmozá a király Jeruzsálemben az ezüstöt mint a követ, és a czédrust, mint a vad fügefákat, a melyek nagy tömegben vannak a mezõn.
\par 28 És Salamonnak Égyiptomból hozának lovakat, és a király kereskedõi sereggel vették volt a lovakat megszabott áron.
\par 29 És egy-egy szekér hatszáz ezüst siklusért és egy-egy ló százötven siklusért ment fel és jött ki Égyiptomból, és ugyancsak õk szállították ezeket a Hitteusok királyainak és Siria királyainak.

\chapter{11}

\par 1 Salamon király pedig megszerete sok idegen asszonyt, még pedig a Faraó leányán kivül a Moábiták, Ammoniták, Edomiták, Sídonbeliek és Hitteusok leányait,
\par 2 Olyan népek közül, a kik felõl azt mondotta volt az Úr az Izráel fiainak: Ne menjetek hozzájok, és õket se engedjétek magatokhoz jõni, bizonyára az õ isteneik után hajtják a ti szíveteket. Ezekhez ragaszkodék Salamon szeretettel.
\par 3 És valának néki feleségei, hétszáz királynéasszony és háromszáz ágyas; és az õ feleségei elhajták az õ szívét.
\par 4 És mikor megvénült Salamon, az õ feleségei elhajták az õ szívét az idegen istenek után, úgy hogy nem volt már az õ szíve tökéletes az Úrhoz, az õ Istenéhez, a mint az õ atyjának, Dávidnak szíve.
\par 5 Mert Salamon követi vala Astoretet, a Sídonbeliek istenét, és Milkómot, az Ammoniták útálatos bálványát.
\par 6 És gonosz dolgot cselekedék Salamon az Úr szemei elõtt, és nem követé olyan tökéletességgel az Urat, mint Dávid, az õ atyja.
\par 7 Akkor építe Salamon templomot Kámosnak, a Moábiták útálatos bálványának a hegyen, a mely Jeruzsálem átellenében van, és  Moloknak, az Ammon fiai útálatos bálványának.
\par 8 És ekképen cselekedék Salamon mind az õ idegen feleségeivel, a kik az õ isteneiknek tömjéneztek és áldoztak.
\par 9 Megharaguvék azért az Úr Salamonra, hogy elhajlott az õ szíve az Úrtól, Izráel Istenétõl, a ki megjelent volt néki kétszer is,
\par 10 És azt parancsolta volt néki, hogy ne kövessen idegen isteneket, és mégsem õrizte meg az Úr parancsolatját.
\par 11 Monda azért az Úr Salamonnak: Miután ez történt veled, és nem õrizted meg az én szövetségemet és az én rendelésimet, a melyeket parancsoltam néked: elszakasztván elszakasztom tõled az országot, és adom a  szolgádnak.
\par 12 Mindazáltal míg élsz, nem cselekeszem ezt Dávidért, a te atyádért; hanem a te fiadnak kezétõl szakasztom el azt.
\par 13 De nem szakasztom el az egész birodalmat; hanem egy nemzetséget adok a te fiadnak Dávidért, az én szolgámért és Jeruzsálemért, a melyet magamnak választottam.
\par 14 És ellenséget támaszta az Úr Salamonra, az Edombeli Hadádot, a ki az Edombeli királyi nembõl való vala.
\par 15 Mert mikor Dávid az Edomiták ellen ment volt, és Joáb, a sereg fõvezére elment volt a megöletteknek temetésére, és levágott minden férfiú nemet Edomban, -
\par 16 Mert hat hónapig volt ott Joáb az egész Izráellel, míg minden férfiúi nemet ki nem vesztett Edomban, -
\par 17 Akkor szaladott vala el Hadád és vele együtt valami Edomiták az õ atyjának szolgái közül õ vele, bemenvén Égyiptomba. Hadád pedig akkor még kis gyermek volt.
\par 18 Kik felkelvén Midiánból, menének Páránba, és melléjök vévén a Páránbeli férfiak közül, bemenének Égyiptomba a Faraóhoz, az Égyiptombeli királyhoz, a ki házat ada néki, és ételt, italt szolgáltata néki, és jószágot is ada néki.
\par 19 Igen kedvében lõn azért Hadád a Faraónak, úgyannyira, hogy feleségül adá néki az õ feleségének hugát, Táfnes királyasszonynak hugát.
\par 20 És a Táfnes huga szülé néki Génubátot, az õ fiát, és elválasztá azt Táfnes a Faraó házában, és Génubát ott volt a Faraó házában, a Faraó fiai között.
\par 21 Mikor pedig Hadád meghallotta Égyiptomban, hogy Dávid elaludt az õ atyáival, és hogy Joáb is, a seregnek fõvezére, meghalt, monda Hadád a Faraónak: Bocsáss el engem, hadd menjek el az én földembe.
\par 22 És felele néki a Faraó: Mi nélkül szûkölködöl én nálam, hogy a te földedbe igyekezel menni? Felele az: Semmi nélkül nem szûkölködöm, de kérlek bocsáss el engem.
\par 23 És támaszta az Isten néki más ellenséget is, Rézont, az Eljada fiát, a ki elfutott vala Hadadézertõl, a Sóbabeli királytól, az õ urától.
\par 24 És hadakozó férfiakat gyûjtött maga mellé, és õ vala a sereg hadnagya, mikor megölé õket Dávid; azután Damaskusba menvén ott lakának, és uralkodának Damaskusban.
\par 25 És ellensége volt Izráelnek Salamonnak egész életében, a nyomorúságon kivül, a melyet Hadád szerze, és gyûlölte Izráelt, és uralkodott Siriában.
\par 26 Azután Jeroboám, a Nébát fia, Seredából való Efrateus, - a kinek anyja Sérua, egy özvegy asszony volt - a Salamon szolgája emelte fel kezét a király ellen.
\par 27 Annak pedig, a miért felemelte kezét a király ellen, ez volt az oka: Mikor Salamon megépítette Millót, és berakatta az õ atyjának, a Dávid városának romlását;
\par 28 Jeroboám erõs férfiú vala; és látván Salamon, hogy az õ szolgája az õ dolgában szorgalmatos, reá bízá a József háza gondviselésének egész terhét.
\par 29 És történt ebben az idõben, hogy mikor kiment egyszer Jeroboám Jeruzsálembõl, találkozék az úton Ahijával, a Silóbeli prófétával, és rajta új köpönyeg volt, és csak ketten valának a mezõn együtt.
\par 30 És megragadván Ahija az új ruhát, a mely azon volt, hasítá azt tizenkét részre.
\par 31 És monda Jeroboámnak: Vedd el magadnak a tíz részt; mert ezt mondja az Úr, Izráel Istene: Ímé elszakasztom ez országot Salamon kezétõl, és néked adom a tíz nemzetséget;
\par 32 Egy nemzetséget hagyok pedig õ nála az én szolgámért, Dávidért, és Jeruzsálem városáért, a melyet magamnak választottam az Izráel minden nemzetségei közül,
\par 33 Még pedig azért, mert elhagytak engem, és imádták Astoretet, a Sídonbeliek istenét, és Kámost, a Moábiták istenét, és Milkomot, az Ammon fiainak istenét, és nem jártak az én utaimban, hogy azt cselekedték volna, a mi tetszett volna az én szemeimnek: az én rendelésimet és végzéseimet, a mint Dávid, az õ atyja.
\par 34 De nem veszem el az egész birodalmat az õ kezétõl, hanem akarom, hogy fejedelem legyen életének minden idejében, Dávidért az én szolgámért, a kit választottam; mivelhogy megõrizte az én parancsolatimat és rendeléseimet;
\par 35 Hanem az õ fiának kezétõl már elveszem a királyságot, és néked adom azt, tudniillik a tíz nemzetséget.
\par 36 Az õ fiának pedig egy nemzetséget adok, hogy Dávidnak, az én szolgámnak legyen elõttem szövétneke mindenkor Jeruzsálemben, a városban, a melyet magamnak választottam, hogy ott helyheztessem az én nevemet.
\par 37 Téged pedig felveszlek, és uralkodol mindenekben a te lelkednek kívánsága szerint, és király lész az Izráelen.
\par 38 És ha te minden parancsolatimnak engedéndesz, és járándasz az én utaimban, és azt cselekedénded, a mi tetszik nékem, megõrizvén az én rendelésimet és parancsolatimat, a mint Dávid, az én szolgám cselekedett: én veled leszek, és építek néked állandó házat, a mint Dávidnak építettem, és néked adom az Izráelt.
\par 39 És megsanyargatom ezért a Dávid magvát: de még sem  örökre.
\par 40 Igyekezik vala pedig Salamon megölni Jeroboámot; ezért felkelvén Jeroboám, futa  Égyiptomba, Sésákhoz, az Égyiptombeli királyhoz, és ott volt Égyiptombam, Salamon haláláig.
\par 41 Salamonnak egyéb dolgai pedig és minden cselekedetei, a melyeket cselekedett, és bölcsesége avagy nem írattattak-é meg a Salamon cselekedeteirõl írott könyvben?
\par 42 Az az idõ pedig, a melyben uralkodott Salamon Jeruzsálemben az egész Izráelen; negyven esztendõ.
\par 43 És elaluvék Salamon az õ atyáival, és eltemetteték az õ atyjának, Dávidnak városában. És Roboám, az õ fia uralkodék helyette.

\chapter{12}

\par 1 És elméne Roboám Síkembe; mert Síkembe gyûlt fel az egész Izráel, hogy királylyá tegyék õt.
\par 2 Mikor pedig meghallotta ezt Jeroboám, a Nébát fia, (a ki még Égyiptomban volt, a hová futott volt Salamon király elõl, és Égyiptomban tartozkodék Jeroboám,
\par 3 És hozzá küldvén, elhivaták õt); elmenének Jeroboám és az Izráel egész gyülekezete, és szólának Roboámnak, mondván:
\par 4 A te atyád igen megnehezítette a mi igánkat, de te most könnyebbítsd meg atyádnak kemény szolgálatát, és a nehéz igát, a melyet mi reánk vetett, és szolgálunk néked.
\par 5 És monda nékik: Menjetek el, és harmadnap mulva jõjjetek vissza hozzám. És a nép elméne.
\par 6 És tanácsot tarta Roboám király a vénekkel, a kik Salamon, az õ atyja elõtt állottak vala életében, mondván: Micsoda tanácsot adtok ti, hogy milyen választ adjak e népnek?
\par 7 És szólának azok, mondván: Ha e mai napon szolgája lész e népnek, és nékik szolgálsz, és választ adsz nékik, és jó szót adsz nékik: mind éltig szolgálnak néked.
\par 8 De õ megveté a vének tanácsát, a melyet néki adtak, és tanácsot tarta az ifjakkal, a kik õ vele együtt nevekedtek volt fel, és a kik õ elõtte udvarlottak.
\par 9 És monda azoknak: Micsoda tanácsot adtok ti, hogy választ adjunk e népnek, a mely nékem szólván, azt mondja: Könnyebítsd meg az igát, a melyet reánk vetett a te atyád?
\par 10 És mondának néki az ifjak, a kik együtt nevekedtek volt fel õ vele: Így szólj ennek a népnek, a mely szólván néked, ezt mondja: A te atyád megnehezítette a mi igánkat, te pedig könnyebbítsd meg nékünk; e képen szólj nékik: Az én kis ujjam vastagabb az én atyám derekánál.
\par 11 Most azért, ha az én atyám reátok nehéz igát vetett, én még nehezebbé teszem a ti igátokat: ha az én atyám ostorral fékezett titeket, én skorpiókkal ortorozlak benneteket.
\par 12 És elméne Jeroboám és mind az egész nép Roboámhoz harmadnap, a mint meghagyta volt a király, ezt mondván: Jõjjetek hozzám harmadnapon.
\par 13 És a király kemény választ adott a népnek, megvetve a vének tanácsát, a melyet adtak vala néki;
\par 14 És szóla nékik az ifjak tanácsa szerint, mondván: Ha az én atyám megnehezítette a ti igátokat, én még nehezebbé teszem azt; ha az én atyám ostorral fékezett titeket, én skorpiókkal ortorozlak benneteket.
\par 15 És nem hallgatá meg a király a népet; mert ezt az Úr fordította ekként, hogy megerõsítse az õ beszédét, a melyet szólott volt az Úr a Silóbeli Ahija által Jeroboámnak, a Nébát fiának.
\par 16 Mikor pedig látta az egész Izráel, hogy meg nem hallgatta õket a király, felele az egész nép a királynak ekképen: Micsoda részünk van nékünk Dávidban? Nincsen nékünk örökségünk  az Isai fiában: menj el a te hajlékidba, óh Izráel! Most viseld gondját immár a te házadnak, óh Dávid! Elméne azért az Izráel az õ hajlékiba;
\par 17 Úgy hogy Roboám csak azokon az Izráel fiain uralkodék, a kik Júda városaiban laktak.
\par 18 És a mikor elküldé Roboám király Adorámot, az adószedõt, megkövezé õt az egész Izráel, és meghala, és maga Roboám király is hamarsággal szekerébe üle, hogy elmeneküljön Jeruzsálembe.
\par 19 Így szakada el az Izráel népe Dávid házától mind e mai napig.
\par 20 És lõn, mikor meghallotta az egész Izráel, hogy megjött Jeroboám, érette küldvén, hivaták õt a gyülekezetbe, és királylyá tevék õt az egész Izráelen; senki pedig nem követé Dávidnak házát, hanem csak egyedül a Júda nemzetsége.
\par 21 És mikor megérkezett Roboám Jeruzsálembe, összegyûjté Júda egész házát és Benjámin nemzetségét, száznyolczvanezer válogatott hadra való férfiút, hogy hadakozzanak az Izráel házával, és visszanyerjék az országot Roboámnak, a Salamon fiának.
\par 22 De az Isten beszéde lõn Semájához, az Isten emberéhez, mondván:
\par 23 Ezt mondjad Roboámnak, a Salamon fiának, a Júda királyának, és az egész Júda és Benjámin házának, és a többi népnek, mondván:
\par 24 Azt mondja az Úr: Fel ne menjetek, és ne hadakozzatok a ti atyátokfiai ellen, az Izráel ellen; térjetek meg kiki a maga házába, mert én tõlem lett e dolog. És õk engedének az Úr beszédének, és visszatérvén, elmenének az Úr beszéde szerint.
\par 25 Jeroboám pedig megépíté Síkemet az Efraim hegyén, és abban lakék; és onnét kimenvén, építé Pénuelt.
\par 26 És monda Jeroboám az õ szívében: Majd visszatér ez ország Dávid házához;
\par 27 Ha felmegy a nép, hogy áldozatot tegyen Jeruzsálemben az Úrnak házában; e népnek szíve az õ urához, Roboámhoz, a Júda királyához hajol, és engem megölnek, és visszatérnek Roboámhoz, a Júda királyához.
\par 28 Tanácsot tartván azért a király, csináltata két arany borjút, és monda nékik: Sok néktek Jeruzsálembe felmennetek: Ímhol vannak a te isteneid, óh Izráel, a kik téged kihoztak Égyiptomnak földébõl.
\par 29 És az egyiket helyhezteté Béthelbe, a másikat pedig Dánba.
\par 30 És e dolog nagy bûnnek lett az okozója, mert a nép felment az egyik elé egészen Dánig.
\par 31 Azután felállítá a magas helyek templomát, és papokat szerze a nép aljából, a kik nem voltak a Lévi fiai közül.
\par 32 És szerze Jeroboám egy ünnepet is a nyolczadik hónapban, a hónap tizenötödik napján, a Júdabeli ünnep módja szerint, és áldozék az oltáron. Hasonlóképen cselekedék Béthelben is, áldozván a borjúknak, a melyeket csinált vala, és szerze Béthelben papokat a magaslatokhoz, a melyeket csinált vala.
\par 33 És áldozék azon az oltáron is, a melyet Béthelben állított fel, a nyolczadik hónap tizenötödik napján, abban a hónapban, a melyet az õ szívében gondolt vala; és ünnepet szerze az Izráel fiainak, és felméne az oltárra, hogy jóillatot szerezzen.

\chapter{13}

\par 1 És ímé Isten embere jöve fel Júdából Béthelbe, az Úrnak intésére, és Jeroboám ott állott az oltár mellett, hogy tömjént gyújtson.
\par 2 És kiálta az oltár ellen az Úr intése szerint, és monda: Oltár, oltár! ezt mondja az Úr: Ímé egy fiú születik a Dávid házából, a kinek neve Józsiás lészen, a ki megáldozza rajtad a magaslatok papjait, a kik most te rajtad tömjéneznek, és emberek csontjait égetik meg rajtad,
\par 3 És ugyanazon napot csudát tõn, mondván: E lészen jegye, hogy az Úr mondotta légyen ezt: Ímé az oltár meghasad, és kiomol a hamu, mely rajta van.
\par 4 És a mikor meghallotta a király az Isten emberének beszédét, a melyet kiáltott vala az oltár ellen Béthelben, kinyújtá Jeroboám az õ kezét az oltártól, mondván: Fogjátok meg õt. És megszárada az õ keze, a melyet kinyújtott volt ellene, és nem tudta azt magához visszavonni.
\par 5 És meghasadt az oltár, és kiomlott a hamu az oltárról a jel szerint, a melyet tett vala az Isten embere az Úrnak beszéde által.
\par 6 És szóla a király, és monda az Isten emberének: Könyörögj az Úrnak a te Istenednek, és imádkozz érettem, hogy ismét hozzám hajoljon az én kezem. És mikor könyörgött az Isten embere az Úrnak, visszahajla a király keze, és olyan lõn, mint azelõtt.
\par 7 És monda a király az Isten emberének: Jere haza velem és egyél ebédet, meg akarlak ajándékozni.
\par 8 És monda az Isten embere a királynak: Ha a te házadnak felét nékem adnád is, nem mennék el veled, és nem enném kenyeret, sem vizet nem innám e helyen;
\par 9 Mert azt parancsolta az Úr nékem az õ beszéde által, mondván: Ne egyél ott kenyeret, vizet se igyál; vissza se térj az úton, a melyen elmenéndesz.
\par 10 És elméne más úton, és nem tére meg azon az úton, a melyen Béthelbe ment.
\par 11 És lakozék Béthelben egy vén próféta, a kihez eljövén az õ fia, elbeszélé az õ atyjának mindazt a dolgot, a melyet aznap az Isten embere cselekedett volt Béthelben, és a beszédeket, a melyeket szólott vala a királynak; és elbeszélék azokat az õ atyjoknak.
\par 12 Akkor monda nékik az õ atyjok: Mely úton ment el? És megmutaták az õ fiai az útat, a melyen elment volt az Isten embere, a ki Júdából jött.
\par 13 És monda az õ fiainak: Nyergeljétek meg nékem a szamarat; és mikor megnyergelék néki a szamarat, felüle reá,
\par 14 És elméne az Isten embere után, és megtalálá õt egy cserfa alatt ülve, és monda néki: Te vagy-é amaz Isten embere, a ki Júdából jöttél? És monda: Én vagyok.
\par 15 Akkor monda néki: Jere haza velem és egyél kenyeret.
\par 16 De az felele: Nem mehetek vissza veled, be sem mehetek veled; nem eszem kenyeret, vizet sem iszom veled e helyen;
\par 17 Mert meg van nékem az Úrnak beszédével parancsolva: Ne egyél kenyeret, vizet se igyál ott, és ne térj azon az úton vissza, a melyen oda menéndesz.
\par 18 És felele az néki: Én is olyan próféta vagyok, mint te, és nékem angyal szólott az Úrnak beszédével, mondván: Hozd vissza õt veled a te házadba, hogy kenyeret egyék és vizet igyék. És e képen hazuda néki.
\par 19 Megtére azért vele, és kenyeret evék az õ házában, és vizet ivék.
\par 20 Mikor pedig az asztalnál ülének: lõn az Úr beszéde a prófétához, a ki õt visszahozta vala,
\par 21 És kiálta az Isten emberének, a ki Júdából jött vala, ezt mondván: Ezt mondja az Úr: Mivelhogy engedetlen voltál az Úr szájának, és meg nem tartottad a parancsolatot, a melyet néked az Úr, a te Istened parancsolt volt;
\par 22 Hanem visszatértél, és kenyeret ettél és vizet ittál azon a helyen, a mely felõl azt mondotta vala néked: Ne egyél ott kenyeret, vizet se igyál: Nem temettetik a te tested a te atyáid sírjába.
\par 23 És miután evett kenyeret és ivott, megnyergelék a szamarat a prófétának, a kit visszahozott vala.
\par 24 És mikor elment, egy oroszlán találá õt az úton, a mely megölé õt; és az õ teste az úton fekszik vala, és mind a szamár, mind az oroszlán a holttest mellett állanak vala.
\par 25 És ímé az arra menõ emberek láták az úton heverõ testet, és az oroszlánt a holttest mellett állani; és elmenvén elbeszélék a városban, a melyben a vén próféta lakott.
\par 26 A mit mikor meghallott a próféta, a ki visszahozta vala õt az útról, monda: Az Isten embere volt az, a ki engedetlen volt az Úr szájának; ezért adta az Úr õt az oroszlánnak, és az törte össze és ülte meg õt, az Úrnak beszéde szerint, a melyet szólott volt néki.
\par 27 És szóla az õ fiainak, mondván: Nyergeljétek meg nékem a szamarat. És felnyergelték.
\par 28 És õ elment és megtalálá a holttestet az útfélre vetve, és a szamarat és az oroszlánt a holttest mellett állva. Az oroszlán nem evett a holtból, és a szamarat sem tépte szét.
\par 29 És felvevé a próféta az Isten emberének holttestét, és feltevé azt a szamárra, és visszahozá azt, és beméne a városba a vén próféta, hogy ott sirassa és eltemesse õt.
\par 30 És eltemeté a holtat a maga sírboltjába, és siraták õt: Ah szerelmes atyámfia!
\par 31 És miután eltemette õt, szóla az õ fiainak, mondván: Ha meghalok, ebbe a sírba temessetek engem is, a melybe az Isten embere temettetett, tetemimet tegyétek az õ tetemei mellé;
\par 32 Mert beteljesedik az, a mit kiáltott az Úrnak beszédével az oltár ellen, a mely Béthelben van, és a magas helyeken való házak ellen, a melyek Samaria városaiban vannak.
\par 33 De még e történet után sem tért meg Jeroboám az õ gonosz útjáról, hanem ismét papokat rendele a nép aljából a magaslatokra, és a ki akarja vala, azt szentelé fel, hogy legyen a magaslatok papja.
\par 34 És ez a dolog lett az oka a Jeroboám háza vétkének, és a föld színérõl való kiirtatásának és megsemmisíttetésének.

\chapter{14}

\par 1 Ebben az idõben megbetegedék Abija, a Jeroboám fia.
\par 2 És monda Jeroboám az õ feleségének: Kelj fel most, és változtasd meg öltözetedet, hogy meg ne ismerjék, hogy te vagy Jeroboám felesége, és menj el Silóba: Ímé ott van Ahija próféta, a ki nékem megmondotta volt, hogy királylyá leszek e népen.
\par 3 És végy magadhoz tíz kenyeret és pogácsát, és egy edényben mézet, és menj el hozzá; õ majd megmondja néked, mi történik a gyermekkel.
\par 4 És e képen cselekedék a Jeroboám felesége; és felkészülvén, elméne Silóba, és beméne az Ahija házába. Ahija azonban már nem látott; mert meghomályosodtak az õ szemei a vénség miatt.
\par 5 És az Úr monda Ahijának: Ímé a Jeroboám felesége jött hozzád, hogy valamit kérdjen tõled az õ fia felõl, mert beteg; te azért így s így szólj néki. És mikor bement, másnak tetteté magát.
\par 6 De mikor meghallotta Ahija az õ lábainak zörejét, a mint az ajtóhoz közelgete, monda: Jõjj be Jeroboám felesége; miért tetteted magadat másnak? Én te hozzád kemény követséggel küldettem.
\par 7 Menj el, mondd meg Jeroboámnak: Ezt mondja az Úr, Izráelnek Istene: Mivelhogy én téged e nép közül felmagasztaltalak, és téged fejedelemmé tettelek az én népemen, az Izráelen;
\par 8 És elszakasztottam az országot a Dávid házától, és azt néked adtam; te azonban nem voltál olyan, mint az én szolgám, Dávid, a ki megõrizte az én parancsolatimat, és a ki engem követett teljes szívébõl, csak azt cselekedvén, a mi kedves az én szemeim elõtt;
\par 9 Hanem gonoszabbul cselekedtél mindazoknál, a kik te elõtted voltak; mert elmentél és idegen isteneket csináltál magadnak, és öntött bálványokat, hogy engem haragra ingerelj, és engem hátad  mögé vetettél:
\par 10 Azért ímé én veszedelmet hozok a Jeroboám házára, és kiirtom Jeroboámnak még az ebét is, és mind a berekesztettet, mind az elhagyottat Izráelben, és kihányom a  Jeroboám háza maradékait, miképen a ganéjt kihányják, mígnem vége lesz.
\par 11 A ki meghal a Jeroboám maradékai közül a városban, azt az ebek eszik meg; a ki pedig a mezõn hal meg, az égi madarak eszik meg; mert az Úr szólott.
\par 12 Te pedig kelj fel, és menj haza, mert a mint belépsz a városba, meghal a gyermek;
\par 13 És az egész Izráel siratja õt, és eltemeti õt, mert a Jeroboám magvából csak egyedül õ temettetik el sírba, mivel a Jeroboám háznépe közül az Úr iránt, Izráel Istene iránt csak õ benne találtatott valami jó.
\par 14 És támaszt az Úr magának királyt Izráelben, a ki kigyomlálja a Jeroboám házát egy napon. De mit mondok? Már is támasztott!
\par 15 És megveri az Úr az Izráelt, mint a hogy a nád a vízben ide s tova hányattatik, és elszakasztja az Izráelt errõl a jó földrõl, a melyet adott az õ atyáiknak, és szétszórja õket a folyóvizen túl, mivelhogy Aserákat csináltak magoknak, hogy az Urat ingereljék.
\par 16 És kézbe adja az Izráelt a Jeroboám bûneiért, a ki maga is vétkezett és az Izráelt is bûnbe ejtette.
\par 17 És felkészült a Jeroboám felesége, elment és juta Thirsába. És mikor a ház küszöbén belépett, meghalt a gyermek.
\par 18 És eltemeték õt, és siratá õt az egész Izráel, az Úrnak beszéde szerint, a melyet szólott az õ szolgája, Ahija próféta által.
\par 19 Jeroboámnak pedig egyéb cselekedetei, mimódon hadakozott és uralkodott, ímé meg vannak írva az Izráel királyainak krónika-könyvében.
\par 20 És az idõ, a melyben uralkodott Jeroboám, huszonkét esztendõ, és elaluvék az õ atyáival, és uralkodék Nádáb, az õ fia, õ helyette.
\par 21 Roboám pedig, a Salamon fia Júdában uralkodék: Negyvenegy esztendõs volt Roboám, mikor uralkodni kezdett, és tizenhét esztendeig uralkodott Jeruzsálemben, abban a városban, a melyet az Úr választott volt magának az Izráelnek minden nemzetségei közül, hogy ott helyheztesse az õ nevét; és az õ anyjának Naama volt a neve, a ki az  Ammon nemzetségébõl való volt.
\par 22 Júda is gonoszul cselekedék az Úr szemei elõtt, és sokkal nagyobb haragra indíták õt az õ vétkeikkel, a melyekkel vétkeztek, mint atyáik azokkal, a melyeket õk cselekedtek volt.
\par 23 Mert õk is építének magoknak magaslatokat, és faragott képeket és Aserákat minden magas halmon, és minden zöldellõ fa alatt.
\par 24 És valának férfi paráznák is az országban, és cselekedének a pogányok minden útálatos vétkei  szerint, a kiket az Úr kiûzött volt az Izráel fiai elõtt.
\par 25 És lõn Roboám király ötödik esztendejében, feljött Sisák, az Égyiptombeli király Jeruzsálem ellen.
\par 26 És elvivé az Úr házának kincseit, és a király házának kincseit, és mindent, a mi csak elvihetõ volt; elvitte mind az arany paizsokat is, a melyeket Salamon csináltatott.
\par 27 A melyek helyett Roboám király azután rézpaizsokat csináltatott, és adá azokat a testõrök fejedelmeinek kezébe, a kik a király házának kapunállói valának.
\par 28 És valahányszor a király bement az Úr házába, bevitték azokat a testõrök, és ismét visszahozták a testõrök házába.
\par 29 Roboámnak pedig több dolgai és minden cselekedetei, nemde nem meg vannak-é írva a Júda királyainak krónika-könyvében?
\par 30 És folyton hadakozás volt Roboám és Jeroboám között.
\par 31 És elaluvék Roboám az õ atyáival, és eltemetteték az õ atyáival a Dávid városában; és az õ anyjának Naama volt a neve, az Ammoniták nemzetségébõl való; és Abija, az õ fia, lett a király õ helyette.

\chapter{15}

\par 1 Jeroboám királynak, a Nébát fiának tizennyolczadik esztendejében Abija lett a király Júdában.
\par 2 És három esztendeig uralkodott Jeruzsálemben. Az õ anyjának neve Maaka, az Abisálom leánya.
\par 3 És járt az õ atyjának minden bûneiben, a melyeket õ elõtte cselekedett, és nem volt az õ szíve olyan tökéletes az õ Urához Istenéhez, mint Dávidnak, az õ atyjának szíve.
\par 4 De mindazáltal Dávidért adott néki az Úr az õ Istene szövétneket Jeruzsálemben: felmagasztalván az õ fiát õ utána és Jeruzsálemet megerõsítvén;
\par 5 Mert Dávid azt cselekedte, a mi kedves volt az Úr szemei elõtt, és el nem távozott azoktól, a melyeket parancsolt néki az õ életének minden idejében, kivéve a Hitteus Uriás dolgát.
\par 6 És hadakozás volt Roboám és Jeroboám között mind éltig.
\par 7 Abija egyéb dolgai és minden cselekedetei pedig vajjon nincsenek-é megírva a Júda királyainak krónika-könyvében? És hadakozás volt Abija és Jeroboám között.
\par 8 És elaluvék Abija az õ atyáival, és eltemeték õt a Dávid városában; és uralkodék helyette Asa, az õ fia.
\par 9 Jeroboámnak az Izráel királyának a huszadik esztendejében Asa lett a király Júdában,
\par 10 És uralkodék negyvenegy esztendeig Jeruzsálemben; és az õ anyjának Maaka vala neve, az Abisálom leánya.
\par 11 És Asa azt cselekedé, a mi kedves volt az Úr szemei elõtt, mint Dávid, az õ atyja.
\par 12 Mert kiveszté a férfiú paráznákat az  országból; és lerontá mind a bálványokat, a melyeket csináltak volt az õ atyái.
\par 13 És Maakát az õ anyját is megfosztá a királynéságtól, mivelhogy egy iszonyú bálványt csináltatott a berekben; és elrontá Asa az õ bálványát, és megégeté  azt a Kidron pataknál.
\par 14 De ha a magaslatokat nem rontották is le, mégis Asának szíve tökéletes volt az Úrhoz életének minden napjaiban.
\par 15 És bevitte az Úrnak házába az ezüstöt és az aranyat, és az edényeket, a melyeket az õ atyja és õ maga arra szenteltek volt.
\par 16 És hadakozás volt Asa és Baása, az Izráel királya között, mind éltig.
\par 17 Mert feljöve Baása, az Izráel királya Júda ellen, és megépíté Rámát, hogy senkit ki és be ne bocsásson Asához a Júda királyához.
\par 18 És vévén Asa mind az ezüstöt és aranyat, a mely megmaradott volt mind az Úr, mind a király házának kincseibõl; adá azt az õ szolgáinak kezébe, és elküldé azt Asa király Benhadádnak, Tabrimmon fiának, a ki Héczion fia volt, Siria királyának, a ki Damaskusban lakott, ezt izenvén néki:
\par 19 Szövetség van köztem és te közötted, az én atyám és a te atyád között. Ímé ajándékot küldök néked, ezüstöt és aranyat, bontsd fel a te szövetségedet Baásával, az Izráel királyával, hogy távozzék el tõlem.
\par 20 És Benhadád engedelmeskedett Asa királynak és elküldé az õ seregeinek vezéreit az Izráel városai ellen; és bevevé Hijont és Dánt, és Abel Beth-Maakát, és az egész Kinneróthot a Nafthali egész földével.
\par 21 Mikor pedig Baása ezt meghallotta, abbahagyta Ráma építését, és lakék Thirsában.
\par 22 Akkor Asa király egybegyûjté az egész Júdát, senkit ki nem hagyván, és elhordák Rámának mind köveit, mind fáját, a melyekkel Baása épített volt, és építé azokból Asa király Gébát, a Benjámin nemzetségében és Mispát.
\par 23 Asának minden egyéb dolgai pedig és egész uralkodása és valamit cselekedett, és a városok, a melyeket épített, vajjon nincsenek-é megírva a Júda királyainak krónika-könyvében? kivéve azt, hogy vénségében megbetegedett a lábaira.
\par 24 És elaluvék Asa az õ atyáival, és eltemetteték az õ atyáival Dávidnak, az õ atyjának városában. És Jósafát, az õ fia, uralkodék helyette.
\par 25 Izráelben pedig Nádáb, a Jeroboám fia lett a király, Asának, a Júdabeli királynak második esztendejében; és két esztendeig uralkodék Izráelen;
\par 26 És gonoszul cselekedék az Úrnak szemei elõtt és jára az õ atyjának útján és annak vétkében, a melylyel bûnbe ejtette volt az Izráelt.
\par 27 De Baása, az Ahija fia, az Issakhár nemzetségébõl, pártot ütött ellene, és megveré õt Baása Gibbethonnál a Filiszteusok városánál: mert Nádáb és az egész Izráel Gibbethonnál táboroztak.
\par 28 És megölé õt Baása, Asának, a júdabeli királynak harmadik esztendejében, és uralkodék helyette.
\par 29 És lõn, mikor uralkodni kezdett, levágá Jeroboámnak egész háznépét; egy lelket sem hagyott Jeroboám háznépébõl, mígnem mind elveszté õket az Úrnak beszéde szerint, a melyet megmondott volt az õ szolgája, a Silóbeli Ahija által.
\par 30 A Jeroboám vétkeiért, a melyekkel vétkezék és az Izráelt is vétekbe ejté; és a bosszantásért, a melylyel felbosszantotta az Urat, Izráel Istenét,
\par 31 Nádáb egyéb dolgai pedig és minden cselekedetei, vajjon nincsenek-é megírva az Izráel királyainak krónika-könyvében?
\par 32 És hadakozás volt Asa és Baása, az Izráel királya között mind éltig.
\par 33 Asának, a Júdabeli királynak harmadik esztendejében lett Baása, az Ahija fia királylyá az egész Izráelen Thirsában, huszonnégy esztendeig.
\par 34 És gonoszul cselekedék az Úrnak szemei elõtt, és jára a Jeroboám útján, és az õ vétkében, a melylyel bûnbe ejtette az Izráelt.

\chapter{16}

\par 1 Lõn azonban az Úrnak beszéde Jéhuhoz, a Hanáni fiához, Baása ellen, mondván:
\par 2 Mivelhogy téged felemeltelek a porból, és fejedelemmé tettelek téged az én népemen, az Izráelen, te pedig a Jeroboám útján jársz, és bûnbe ejted az én népemet, az Izráelt, hogy engem haragra indítsanak az õ vétkeikkel:
\par 3 Íme én elvetem Baása maradékait, és az õ háza maradékait, és olyanná teszem a te házadat, mint Jeroboámnak, a Nébát fiának házát.
\par 4 A ki Baása maradékai közül a városban hal meg, azt az ebek eszik meg; a ki pedig a mezõn halánd meg, azt az égi madarak eszik meg.
\par 5 Baása több dolgai pedig, és a mit cselekedett, és az õ uralkodása vajjon nincsenek-é megírva az Izráel királyainak krónika-könyvében?
\par 6 És elaluvék Baása az õ atyáival, és eltemetteték Thirsában, és az õ fia, Ela  uralkodék õ utána.
\par 7 És nemcsak azért lõn az Úrnak beszéde Jéhu próféta, a Hanáni fia által Baása és az õ háza ellen és minden õ gonoszsága ellen, a melyet cselekedett volt az Úr szemei elõtt, õt haragra indítván az õ kezeinek cselekedetei által, hogy olyan lesz mint a Jeroboám háza, hanem azért is, mert azt  megölte.
\par 8 És Asának, a Júda királyának huszonhatodik esztendejében lett királylyá Ela, a Baása fia, az Izráelen Thirsában, két esztendeig.
\par 9 Azonban az õ szolgája Zimri, a szekerek fél részének feje, pártot ütött ellene. És Ela Thirsában volt, és az ital miatt igen megrészegedett az Arsa házában, a ki az õ házának Thirsában gondviselõje volt;
\par 10 És eljövén Zimri, megveré és megölé õt, Asának, a Júdabeli királynak huszonhetedik esztendejében, és õ uralkodék helyette.
\par 11 Mikor pedig királylyá lett és trónját elfoglalta, kiirtá Baása egész házát minden rokonságával és barátaival együtt és nem hagyott belõle még csak egy ebet sem.
\par 12 Így veszté ki Zimri a Baása egész házát, az Úrnak beszéde szerint, a melyet Baása ellen szólott Jéhu próféta által:
\par 13 Baása minden vétkeiért, és Elának, az õ fiának vétkeiért, a kik vétkeztek és vétekbe ejtették az Izráelt, haragra indítván az Urat, Izráelnek Istenét az õ bálványozásukkal.
\par 14 Elának egyéb dolgai pedig és minden cselekedetei, vajjon nincsenek-é megírva az Izráel királyainak krónika-könyvében?
\par 15 Asának, a Júdabeli királynak huszonhetedik esztendejében, uralkodék Zimri hét napig Thirsában; a nép pedig Gibbethon elõtt táborozott, a mely a Filiszteusoké.
\par 16 Mikor pedig hallotta a nép a táborban beszélni, hogy pártot ütött Zimri és a királyt meg is ölte: az egész Izráel azon a napon Omrit, az Izráel seregeinek fõvezérét választá királylyá a táborban.
\par 17 És felvonult Omri és az egész Izráel õ vele Gibbethonból, és megszállák Thirsát.
\par 18 De Zimri, a mikor látta, hogy a várost bevették, bemenvén a királyi ház palotájába, tûzzel magára gyújtá a királyi házat, és meghala.
\par 19 Az õ bûneiért, a melyekkel vétkezék, gonoszul cselekedvén az Úr szemei elõtt; járván a Jeroboám útjában és az õ bûnében, a melyet cselekedett volt, vétekbe ejtvén az Izráelt.
\par 20 Zimrinek egyéb dolgai pedig és az õ pártütése, a melyet cselekedett, vajjon nincsenek-é megírva az Izráelbeli királyok krónika-könyvében?
\par 21 Akkor kétfelé hasonlék az Izráel népe: a nép egyik része Tibnit, a Ginát fiát követte és azt tette királylyá, a másik része pedig Omrihoz ragaszkodott.
\par 22 De a népnek az a része, a mely Omrihoz ragaszkodott, erõsebb volt, mint az a nép, a mely Tibnit, a Ginát fiát követte, és meghalván Tibni, uralkodék Omri.
\par 23 Asának, a Júdabeli királynak harminczegyedik esztendejében uralkodék Omri Izráelben tizenkét esztendeig; Thirsában uralkodék hat esztendeig.
\par 24 És megvevé áron a Samaria hegyét Sémertõl két tálentom ezüstön; és építe a hegyre, és nevezé a várost, a melyet építe, Sémernek, a hegy urának nevérõl, Samariának.
\par 25 És gonoszul cselekedék Omri az Úr szemei elõtt, és gonoszságával meghaladá mind az õ elõtte valókat.
\par 26 És jára Jeroboámnak, a Nébát fiának minden útján és az õ bûnében, a melylyel vétekbe ejté az Izráelt, haragra indítván az Urat, Izráel Istenét az õ bálványozásai által.
\par 27 Omrinak egyéb dolgai pedig és minden cselekedetei és az õ erõssége, a melylyel cselekedett, vajjon nincsenek-é megírva az Izráelbeli királyok krónika-könyvében?
\par 28 És elaluvék Omri az õ atyáival, és eltemetteték Samariában, és uralkodék õ utána az õ fia, Akháb.
\par 29 Akháb pedig, az Omri fia uralkodni kezde Izráelen, Asának, a Júdabeli királynak harmincznyolczadik esztendejében; és uralkodék Akháb, az Omri fia Izráelen Samariában huszonkét esztendeig.
\par 30 És gonoszabbul cselekedék Akháb, az Omri fia az Úr szemei elõtt mindazoknál, a kik õ elõtte voltak.
\par 31 Mert nem elégedék meg azzal, hogy Jeroboámnak, a Nébát fiának bûneiben járjon, hanem elméne és feleségül vevé magának Jézabelt, Ethbaálnak, a Sídonbeli királynak leányát, és elmenvén, a Baálnak szolgála, és meghajtá magát annak.
\par 32 És oltárt emele a Baálnak a Baál házában, a melyet Samariában épített.
\par 33 És csinált Akháb Aserát is, és jobban haragra indítá Akháb az Urat, Izráel Istenét, mint az Izráel valamennyi királya, a kik õ elõtte voltak.
\par 34 Ennek idejében építé meg a Béthelbeli Hiel Jérikhót. Az õ elsõszülött fiának, Abirámnak élete árán veté meg annak fundamentomát, és az õ kisebbik fiának, Ségubnak élete árán állítá fel annak kapuit, az Úr beszéde szerint, a melyett szólott volt Józsué, a Nún fia által.

\chapter{17}

\par 1 És szóla Thesbites Illés, a Gileád lakói közül, Akhábnak: Él az Úr, az Izráel Istene, a ki elõtt állok, hogy ez esztendõkben sem harmat, sem esõ nem lészen; hanem csak az én beszédem szerint.
\par 2 És lõn az Úrnak beszéde õ hozzá, mondván:
\par 3 Menj el innét, és menj napkelet felé, és rejtezzél el a Kérith patakja mellett, mely a Jordán felé folyik.
\par 4 És a patakból lesz néked italod; a hollóknak pedig megparancsoltam, hogy tápláljanak téged ott.
\par 5 Elméne azért, és az Úrnak beszéde szerint cselekedék; és elméne és leüle a Kérith patakja mellett, a mely a Jordán felé folyik.
\par 6 És a hollók hoztak néki kenyeret és húst reggel és este, és a patakból ivott.
\par 7 És lõn néhány nap múlva, hogy kiszáradt a patak; mert nem volt esõ a földre.
\par 8 És lõn az Úrnak beszéde õ hozzá, mondván:
\par 9 Kelj fel, és menj el Sareptába, a mely Sídonhoz tartozik, és légy ott; ímé megparancsoltam ott egy özvegyasszonynak, hogy gondoskodjék rólad.
\par 10 És felkelvén, elméne Sareptába, és mikor a város kapujához érkezett, ímé egy özvegyasszony volt ott, a ki fát szedegetett, és megszólítván azt, monda néki: Hozz, kérlek, egy kevés vizet nékem valami edényben, hogy igyam.
\par 11 De mikor az elment, hogy vizet hozzon, utána kiáltott, és monda néki: Hozz, kérlek, egy falat kenyeret is kezedben.
\par 12 Az pedig monda: Él az Úr, a te Istened, hogy nincs semmi sült kenyerem, csak egy marok lisztecském van a vékában, és egy kevés olajom a korsóban, és most egy kis fát szedegetek, és haza megyek, és megkészítem azt magamnak és az én fiamnak, hogy megegyük és azután meghaljunk.
\par 13 Monda pedig néki Illés: Ne félj, menj el, cselekedjél a te beszéded szerint; de mindazáltal nékem süss abból elõször egy kis pogácsát, és hozd ide; magadnak és a te fiadnak pedig azután süss;
\par 14 Mert azt mondja az Úr, Izráel Istene, hogy sem a vékabeli liszt el nem fogy, sem a korsóbeli olaj meg nem kevesül addig, míg az Úr esõt ád a földnek színére.
\par 15 És õ elméne, és az Illés beszéde szerint cselekedék, és evék mind õ, mind amaz, mind annak háznépe, naponként.
\par 16 A vékabeli liszt nem fogyott el, sem pedig a korsóbeli olaj nem kevesült meg, az Úrnak beszéde szerint, a melyet szólott Illés által.
\par 17 És lõn ezek után, hogy megbetegedék a ház gazdasszonyának fia, és az õ betegsége felette nagy vala, annyira, hogy már a lélekzete is elállott.
\par 18 Monda azért Illésnek: Mit vétettem ellened, Istennek embere? Azért jöttél hozzám, hogy eszembe juttassad álnokságomat, és megöljed az én fiamat?
\par 19 És monda néki: Add ide a te fiadat. És õ elvevé azt az õ kebelérõl, és felvivé a felházba, a melyben õ lakik vala, és az õ ágyára fekteté.
\par 20 Akkor kiálta az Úrhoz, és monda: Én Uram, Istenem, nyomorúságot hozol erre az özvegyre is, a kinél én lakom, hogy az õ fiát megölöd?
\par 21 És ráborult háromszor a gyermekre, és felkiáltott az Úrhoz, mondván: Én Uram, Istenem, térítsd vissza e gyermek lelkét õ belé!
\par 22 És meghallgatta az Úr Illés szavát, és megtért a gyermekbe a lélek, és megélede.
\par 23 És felvévén Illés a gyermeket, alávivé õt a felházból a házba, és adá õt az õ anyjának, és monda Illés: Lássad, él a te fiad!
\par 24 És monda az asszony Illésnek: Most tudtam meg, hogy te Isten embere vagy, és hogy az Úrnak beszéde a te ajkadon: igazság.

\chapter{18}

\par 1 És lõn sok idõ multán, a harmadik esztendõben, az Úrnak beszéde Illéshez, mondván: Menj el, mutasd meg magadat Akhábnak, és esõt adok a föld színére.
\par 2 És elméne Illés, hogy megmutassa magát Akhábnak. Nagy éhség volt pedig Samariában.
\par 3 És Akháb hivatá Abdiást, a ki az õ házának gondviselõje volt. (Abdiás pedig felette igen féli vala az Urat;
\par 4 Mert mikor Jézabel megölette az Úr prófétáit, Abdiás száz prófétát vett oltalmába, és rejtette el ötvenenként egy-egy barlangba, és ott táplálta õket kenyérrel és vízzel).
\par 5 És monda Akháb Abdiásnak: Menj el az országba szerén-szerte a vizek forrásaihoz és a patakokhoz, hogyha valami füvet találhatnánk, hogy a lovakat és öszvéreket megtarthatnánk életben, és ne hagynánk a barmokat mindenestõl elpusztulni.
\par 6 Eloszták azért magok közt a földet, a melyen kiki elmenjen. Akháb egyedül ment az egyik úton, és Abdiás is egyedül ment a másik úton.
\par 7 És mikor Abdiás az úton ment, ímé Illés elõtalálá õt, és mikor felismerte õt, arczra borula és monda: Nem te vagy-é az, uram, Illés?
\par 8 Felele néki: Én vagyok; menj el, mondd meg a te uradnak: Ímé itt van Illés.
\par 9 Õ pedig monda: Mit vétettem ellened, hogy a te szolgádat Akháb kezébe akarod adni, hogy megöljön engem?
\par 10 Él az úr a te Istened: nincs sem nemzetség, sem ország, a hova el nem küldött volna az én uram, hogy megkeressen téged. És ha azt mondották: Nincs itt! az országot és a népet megesküdtette, hogy téged csakugyan nem találtak meg.
\par 11 És most te azt mondod: Menj el, mondd meg a te uradnak: Ímé itt van Illés.
\par 12 Ha most elmegyek tõled, téged pedig olyan helyre ragad el az Úrnak Lelke, a melyet én nem tudok, és én elmegyek, hogy megmondjam Akhábnak, és ha õ téged nem talál meg, engem öl meg; pedig a te szolgád féli az Urat gyermekségétõl fogva.
\par 13 Nem mondották-é meg az én uramnak, mit cselekedtem, mikor Jézabel megölette az Úr prófétáit? hogy hogyan rejtettem el az Úr prófétái közül száz férfiút ötvenenként egy-egy barlangba és tápláltam õket kenyérrel és vízzel.
\par 14 És te mégis azt mondod: Menj el és mondd meg a te uradnak: Ímé itt van Illés; hogy megöljön engemet.
\par 15 És felele Illés: Él a Seregeknek Ura, a ki elõtt állok: e mai napon megmutatom magamat néki.
\par 16 Elméne azért Abdiás Akháb eleibe, és megjelenté ezt néki, és eleibe méne Akháb Illésnek.
\par 17 És mikor meglátta Akháb Illést, monda Akháb néki: Te vagy-é az Izráel megháborítója?
\par 18 Õ pedig monda: Nem én háborítottam meg az Izráelt, hanem te és a te atyád háza, azzal, hogy elhagytátok az Úrnak parancsolatait, és a Baál után jártatok.
\par 19 Most azért küldj el, gyûjtsd hozzám az egész Izráelt a Kármel hegyre, és a Baál négyszázötven prófétáját, és az Áserának négyszáz prófétáját, a kik a Jézabel asztaláról élnek.
\par 20 És elkülde Akháb mind az egész Izráel fiaihoz, és egybegyûjté a prófétákat a Kármel hegyre.
\par 21 És odamenvén Illés az egész sokasághoz, monda: Meddig sántikáltok kétfelé? Ha az Úr az Isten, kövessétek õt; ha pedig Baál, kövessétek azt. És nem felelt néki a nép csak egy szót sem.
\par 22 Akkor monda Illés a népnek: Én maradtam meg csak egyedül az Úr prófétái közül; míg a Baál prófétái négyszázötvenen vannak;
\par 23 Adjatok azért nékünk két tulkot, és õk válaszszák magoknak az egyik tulkot, a melyet vagdaljanak darabokra, és rakják a fákra; de tüzet ne tegyenek alája; én pedig a másikat készítem el, a melyet a fákra rakok, de tüzet én sem teszek alája.
\par 24 Akkor hívjátok segítségül a ti istenteknek nevét, és én is segítségül hívom az Úrnak nevét; és a mely isten tûz által felel, az az Isten. És felelvén az egész sokaság, monda: Jó lesz!
\par 25 És monda Illés a Baál prófétáinak: Válaszszátok el magatoknak az egyik tulkot, és készítsétek el ti elõször; mert ti többen vagytok, és hívjátok segítségül a ti istenteknek nevét, de tüzet ne tegyetek alája.
\par 26 És vevék a tulkot, a melyet nékik adott, és azt elkészíték, és segítségül hívák a Baálnak nevét reggeltõl fogva délig, mondván: Baál! hallgass meg minket! De nem jött szó, sem felelet. És ott sántikáltak az oltár körül, a melyet készítettek.
\par 27 Mikor pedig már dél lett, elkezdte õket gúnyolni Illés, azt mondván: Kiáltsatok hangosabban, hiszen isten! Talán elmélkedik, vagy félrement, vagy úton van, vagy talán aluszik, és felserken.
\par 28 És elkezdtek hangosan kiabálni és az õ szokásuk szerint késekkel és borotvákkal metélték magokat, míg csak ki nem csordult a vérök.
\par 29 Mikor pedig a dél elmúlt, prófétálni kezdtek egész az esteli áldozatig; de akkor sem lett se szó, se felelet, se meghallgattatás.
\par 30 Akkor monda Illés az egész sokaságnak: Jõjjetek én hozzám. És hozzá méne az egész sokaság, és megépíté az Úr oltárát, a mely leromboltatott volt.
\par 31 És võn Illés tizenkét követ, a Jákób fiai nemzetségeinek száma szerint, a kiknek az Isten azt mondotta volt: Izráel legyen  a te neved;
\par 32 És oltárt építe a kövekbõl az Úr nevében, és egy árkot húzott az oltár körül, a melybe két véka vetni való mag férne.
\par 33 És oda készíté a fát, és felvagdalá a tulkot, és felraká azt a fára;
\par 34 És monda: Töltsetek meg négy vedret vízzel, és öntsétek az égõáldozatra és a fára. Monda ismét: Tegyétek ezt még egyszer! És másodszor is azt tevék. Monda még is: Harmadszor is tegyétek meg! És harmadszor is azt mívelék;
\par 35 Úgy, hogy a víz lecsurgott az oltárról, és még az árok is tele lett vízzel.
\par 36 És a mikor eljött az esteli áldozás ideje, oda lépett Illés próféta, és monda: Óh Uram, Ábrahámnak, Izsáknak és Izráelnek Istene, hadd ismerjék meg e mai napon, hogy te vagy az Isten az Izráelben, és hogy én a te szolgád vagyok, és hogy  mindezeket a te parancsolatodból cselekedtem.
\par 37 Hallgass meg engem, Uram, hallgass meg engem, hogy tudja meg e nép, hogy te, az Úr vagy az Isten, és te fordítod vissza az õ szívöket!
\par 38 Akkor alászálla az Úr tüze, és megemészté az égõáldozatot, a fát, a köveket és a port, és felnyalta a vizet, a mely az árokban volt.
\par 39 Mikor ezt látta az egész sokaság, arczra borult, és monda: Az Úr az Isten! az Úr az Isten!
\par 40 És monda Illés nékik: Fogjátok meg a Baál prófétáit; senki el ne szaladjon közülök! És megfogák õket, és alávivé õket Illés a Kison opatakja mellé, és megölé ott õket.
\par 41 Akkor monda Illés Akhábnak: Eredj fel, egyél és igyál, mert nagy esõnek zúgása hallszik.
\par 42 És felment Akháb, hogy egyék és igyék. Illés pedig felment a Kármel hegy tetejére, és leborula a földre, és az õ orczáját az õ két térde közé tevé;
\par 43 És monda az õ szolgájának: Menj fel, és nézz a tenger felé. És felment, és arrafelé nézett, és monda: Nincsen semmi. És monda Illés: Menj vissza hétszer.
\par 44 És lõn hetedúttal, monda a szolga: Ímé egy kis felhõcske, mint egy embernek a tenyere, jõ fel a tengerbõl. Akkor monda: Menj fel, mondd meg Akhábnak: Fogj be és menj le, hogy meg ne késleljen az esõ.
\par 45 És lõn azonközben, hogy besötétedett az ég a fellegektõl és a széltõl, és nagy esõ lett. Akháb pedig szekérre ült és elment Jezréelbe.
\par 46 És lõn az Úr keze Illésen, és felövezvén az õ derekát, még Akháb elõtt futott el Jezréel felé.

\chapter{19}

\par 1 És Akháb elbeszélé Jézabelnek mindazokat, a melyeket Illés cselekedett és a többek között, hogy hogyan  ölte meg mind a prófétákat fegyverrel.
\par 2 És követet külde Jézabel Illéshez, mondván: Ezt cselekedjék velem az istenek és úgy segéljenek, ha holnap ilyenkor úgy nem cselekszem a te életeddel, mint a hogy te cselekedtél azoknak életekkel mind egyig.
\par 3 A mit mikor megértett, felkelvén elméne, vigyázván az õ életére. És méne Beersebába, a mely Júdában volt; és ott hagyá az õ szolgáját.
\par 4 Õ pedig elméne a pusztába egynapi járó földre, és elmenvén leüle egy fenyõfa alá, és könyörgött, hogy hadd haljon meg, és monda: Elég! Most óh Uram, vedd el az én lelkemet; mert nem vagyok jobb az én atyáimnál!
\par 5 És lefeküvék és elaluvék a fenyõfa alatt. És ímé angyal illeté õt, és monda néki: Kelj fel, egyél.
\par 6 És mikor körülnézett, ímé fejénél vala egy szén között sült pogácsa és egy pohár víz. És evék és ivék, és ismét lefeküvék.
\par 7 És az Úr angyala eljött másodszor is és megilleté õt, és monda: Kelj fel, egyél; mert erõd felett való utad van.
\par 8 És õ felkelt, és evett és ivott; és méne annak az ételnek erejével negyven nap és negyven éjjel egész az Isten hegyéig, Hórebig.
\par 9 És beméne ott egy barlangba, és ott hála. És ímé lõn az Úrnak beszéde õ hozzá, és monda néki: Mit csinálsz itt Illés?
\par 10 Õ pedig monda: Nagy búsulásom van az Úrért, a Seregek Istenéért; mert elhagyták a te szövetségedet az Izráel fiai, a te oltáraidat lerontották, és a te prófétáidat  fegyverrel megölték, és csak én egyedül maradtam, és engem is halálra keresnek.
\par 11 És monda: Jõjj ki és állj meg ezen a hegyen, az Úr elõtt. És ímé ott az Úr volt elmenendõ. És az Úr elõtt megyen vala nagy és erõs szél, a mely a hegyeket megszaggatta és meghasogatta a kõsziklákat az Úr elõtt; de az Úr nem  vala abban a szélben. És a szél után földindulás lett; de az Úr nem volt a földindulásban sem.
\par 12 És a földindulás után tûz jöve, de nem volt az Úr a tûzben sem. És a tûz után egy halk szelíd hang hallatszék.
\par 13 És mikor Illés ezt hallotta, befedé az õ arczát palástjával, és kimenvén, megálla a barlang ajtajában, és ímé szózat lõn õ hozzá, a mely ezt mondá: Mit csinálsz itt Illés?
\par 14 És õ felele: Nagy búsulásom van az Úrért, a Seregeknek Istenéért, mert az Izráel fiai elhagyták a te szövetségedet, lerontották a te oltáraidat, és a te prófétáidat megölték fegyverrel, és én egyedül maradtam, és engem is halálra keresnek.
\par 15 És monda az Úr néki: Menj el, térj vissza a te utadon a pusztán át Damaskusba, és mikor oda jutándasz, kenjed királylyá Hazáelt Siriában;
\par 16 És Jéhut, a Nimsi fiát kenjed királylyá Izráelben, és Elizeust, az Abelméholabeli Sáfát fiát pedig kenjed prófétává a te helyedbe.
\par 17 És lészen, hogy a ki megmenekedik Hazáel fegyverétõl, azt Jéhu öli meg, és a ki megmenekedik a Jéhu fegyverétõl, azt Elizeus öli meg.
\par 18 De meghagyok Izráelben hétezer embert: minden térdet, mely nem hajolt a Baálnak, és minden ajkat, mely meg nem csókolta azt.
\par 19 És õ elmenvén onnét, megtalálá Elizeust, a Sáfát fiát, a mint szántott tizenkét járom ökörrel, és õ maga a tizenkettedikkel volt; és Illés hozzá méne, és az õ palástját reá veté.
\par 20 És õ elhagyván az ökröket, Illés után futott, és monda: Kérlek hadd csókoljam meg az én atyámat és az én anyámat, és azután követlek. És monda: Menj, térj vissza; mert mit cselekedtem tenéked?
\par 21 És elmenvén õ tõle, võn azután egy pár ökröt, és levágá azt, és az ekéhez való szerszámokból tüzet rakván, megfõzé azok húsát, és a népnek adá, és evének; és felkelvén, elméne Illés után, és szolgála néki.

\chapter{20}

\par 1 És Benhadád, Siria királya összegyûjté egész seregét, és harminczkét király volt õ vele és nagyon sok ló és szekér, és felméne és megszállá Samariát, és ostromolni kezdte azt.
\par 2 És követeket küldött Akhábhoz, az Izráel királyához a városba;
\par 3 És azt izené néki: Azt mondja Benhadád: A te ezüstöd és aranyad az enyém, a te feleségeid is és a te szép fiaid is az enyémek.
\par 4 És felele az Izráel királya, és monda: A mint megmondottad uram, király, tiéd vagyok mindenekkel, a melyeket bírok.
\par 5 Megtérvén pedig a követek, mondának: Azt mondja Benhadád: Miután hozzád küldöttem, és azt izentem, hogy a te ezüstödet és aranyadat és a te feleségeidet és fiaidat add nékem;
\par 6 Azért holnap ilyenkor elküldöm az én szolgáimat hozzád, és felkutatják a te házadat, és a te szolgáid házait; és kezekhez veszik, a mi csak kedves elõtted, és elhozzák.
\par 7 Akkor egybehívá az Izráel királya mind az ország véneit, és monda: vegyétek eszetekbe, és lássátok meg, minemû gonosz szándékkal van ez; mert hozzám küldött az én feleségeimért és gyermekeimért, ezüstömért és aranyomért, és meg nem tagadtam tõle.
\par 8 És mondának néki a vének mindnyájan, és az egész nép: Ne engedj néki, és az õ akaratját be ne teljesítsd.
\par 9 És õ monda a Benhadád követeinek: Mondjátok meg az én uramnak, a királynak: Mindazokat, a melyek felõl elõször izentél a te szolgádnak, megcselekszem, de ezt a dolgot nem tehetem meg. Így elmenvén a követek, megmondák néki a választ.
\par 10 Akkor hozzá külde Benhadád, és monda: Úgy cselekedjenek velem az istenek és úgy segéljenek, hogy Samariának minden pora sem elég, hogy a velem való nép közül mindeniknek csak egy-egy marokkal is jusson!
\par 11 És felele az Izráel királya, mondván: Mondjátok meg néki: Ne kérkedjék úgy, a ki fegyverbe öltözik, mint a ki már leveti a fegyvert!
\par 12 Meghallván pedig ezt a választ, mikor õ ivott a királyokkal a sátorokban, monda az õ szolgáinak: Vegyétek körül a várost! És azok körülvevék azt.
\par 13 És ímé egy próféta méne Akhábhoz, az Izráel királyához, a ki ezt mondá: Azt mondja az Úr: Avagy nem láttad-é mindezt a nagy sokaságot: ímé e mai napon kezedbe adom azt, hogy megtudjad, hogy én vagyok az Úr!
\par 14 Monda pedig Akháb: Ki által? És felele: Azt mondja az Úr: A tartományok fejedelmeinek ifjai által. Akkor monda Akháb: Ki kezdje meg a harczot? És felele: Te!
\par 15 Megszámlálá azért a tartományok fejedelmeinek ifjait, a kik kétszázharminczketten voltak; ezekután megszámlálá mind az Izráel fiainak is minden népét, hétezer embert.
\par 16 És elindulának délben. Benhadád pedig ott ivott a királyokkal a sátorokban, és lerészegedett õ és a harminczkét király, a ki segítségére jött vele.
\par 17 És a tartományok fejedelmeinek ifjai vonultak ki legelõször. Benhadád pedig elkülde, és megmondották néki, ezt mondván: Valami férfiak jöttek ki Samariából!
\par 18 És monda: Akár békességért jöttek ki, fogjátok meg õket elevenen; akár viadalért jöttek ki, fogjátok el õket elevenen.
\par 19 De mikor kivonultak a városból a tartományok fejedelmeinek ifjai és a sereg, a mely õket követte:
\par 20 Mindenik vágni kezdte azt, a ki eléje került, és elfutottak a Siriabeliek, az Izráel pedig kergette õket, és megfutott maga Benhadád, Siria királya is lovon a lovagokkal együtt.
\par 21 És azután kivonult az Izráel királya, és megveré mind a lovagokat, mind a szekereket, és megveré a Siriabelieket nagy csapással.
\par 22 És méne az Izráel királyához egy próféta, és ezt mondá néki: Menj el, erõsítsd meg magad, és vedd eszedbe és lásd meg, mit kelljen cselekedned, mert esztendõ mulva ismét feljõ Siria királya ellened.
\par 23 A siriabeli király szolgái pedig mondának néki: A hegyeknek istenei az õ isteneik, azért gyõztek le bennünket; de vívjunk csak meg velök a síkon, és meglátod, ha le nem gyõzzük-é õket?
\par 24 És tedd ezt: Küldd el a királyokat, mindeniket a maga helyérõl; és állíts hadnagyokat helyettök,
\par 25 És szervezz magadnak olyan sereget, mint az volt, a melyet elvesztettél, és olyan lovakat és szekereket, mint amazok voltak; és vívjunk meg velök a síkon, és meglátod, ha le nem gyõzzük-é õket? És engede az õ szavoknak, és akképen cselekedék.
\par 26 Mikor azért az esztendõ elmult, rendbeszedte Benhadád a Siriabelieket, és feljöve Afekbe, hogy hadakozzék az Izráel ellen.
\par 27 De az Izráel fiai összeszámláltattak és elláttattak élelemmel, és eleikbe menének azoknak. Táborba szállván pedig az Izráel fiai, amazokhoz képest olyanok voltak, mint két kicsiny kecskenyájacska; míg a Siriabeliek  ellepték a földet.
\par 28 Jött vala pedig egy Isten embere, és szóla az Izráel királyának, mondván: Azt mondja az Úr: Azért mert a Siriabeliek azt mondották, hogy csak a hegyeknek Istene az Úr, és nem a völgyeknek Istene is: mindezt a nagy sokaságot a te kezedbe adom, hogy megismerjétek, hogy én vagyok az Úr.
\par 29 És ott táboroztak egészen velök szemben hetednapig. A hetedik napon azután megütköztek, és az Izráel fiai levágtak a Siriabeliek közül egy nap százezer gyalogost.
\par 30 És a többiek elmenekültek Afek városába; de a falak rászakadtak a megmaradott huszonkétezer emberre, és Benhadád is elfutott és ott bolyongott a városban kamaráról-kamarára.
\par 31 És mondának néki az õ szolgái: Ímé hallottuk, hogy az Izráel házának királyai kegyelmes királyok, azért hadd öltözzünk zsákokba, és vessünk köteleket a mi nyakunkba, és menjünk ki az Izráel királyához, talán életben hagyja a te lelkedet.
\par 32 És zsákokba öltözének, és köteleket vetének nyakokba, és elmenének az Izráel királyához, és mondának: A te szolgád, Benhadád, ezt mondja: Hagyd életben kérlek, az én lelkemet! És monda: Él-e még? én atyámfia õ!
\par 33 És a férfiak jó jelnek vették azt, és gyorsan megragadták a szót, és mondának: A te atyádfia Benhadád él. És monda: Menjetek, és hozzátok ide õt. Kijöve azért õ hozzá Benhadád, és felülteté õt az õ szekerébe,
\par 34 És monda néki: A városokat, a melyeket elvett volt az én atyám a te atyádtól, azokat visszaadom, és csinálj magadnak utczákat Damaskusban, mint az én atyám csinált volt Samariában; én ezzel a kötéssel bocsátlak el téged. És szövetséget kötött vele, és elbocsátá õt.
\par 35 Egy férfiú pedig a próféták közül monda az õ felebarátjának az Úr beszéde szerint: Verj meg engem kérlek; de ez nem akará õt megverni.
\par 36 Akkor monda néki: Azért, mert nem engedtél az Úr szavának: ímé mihelyt én tõlem elmégy, megöl téged az oroszlán. És a mikor elment õ tõle, találá õt egy oroszlán, és megölé.
\par 37 Talála azután más férfiat, a kinek monda: Kérlek, verj meg engem. És megveré az annyira, hogy megsebesítette.
\par 38 És elméne a próféta, és az útfélen a király elé álla és elváltoztatá magát szemeit bekötözvén.
\par 39 Mikor pedig arra ment el a király, kiálta a királyhoz, és monda: A te szolgád kiment volt a hadba, és ímé egy férfiú eljövén, hoza én hozzám egy férfiút, és monda: Õrizd meg ezt a férfiút; ha elszaladánd, meg kell halnod érette, vagy egy tálentom ezüstöt fizetsz.
\par 40 És mialatt a te szolgádnak itt és amott dolga volt, az az ember már nem volt. És monda néki az Izráel királya: Az a te ítéleted; magad akartad.
\par 41 És mindjárt elvevé a kötést az õ szemérõl, és megismeré õt az Izráel királya, hogy a próféták közül való.
\par 42 És monda néki: Ezt mondja az Úr: Mert elbocsátottad kezedbõl a férfiút, a kit én halálra szántam, azért lelked lészen lelkéért, és néped népéért.
\par 43 És házához méne az Izráel királya szomorú és megbúsult szívvel, és méne Samariába.

\chapter{21}

\par 1 És történt ezek után, hogy a Jezréelbeli Nábótnak egy szõlõje volt Jezréelben, Akhábnak, Samaria királyának a háza mellett.
\par 2 És szóla Akháb Nábótnak, mondván: Add nékem a te szõlõdet, hogy legyen veteményes kertem; mert közel van az én házamhoz, és néked érette jobb szõlõt adok annál, vagy ha néked tetszik, pénzül adom meg az árát.
\par 3 És felele Nábót Akhábnak: Isten õrizzen, hogy néked adjam az én atyáimtól maradt örökséget.
\par 4 Akkor haza méne Akháb nagy bánattal és haraggal a beszéd miatt, a melyet szólott néki a Jezréelbeli Nábót, mondván: Nem adom néked az én atyáimtól maradt örökséget; és lefeküvék az õ ágyára, és arczát a fal felé fordítá, és nem evék kenyeret.
\par 5 Hozzá menvén az õ felesége Jézabel, monda néki: Miért háborodott meg a te szíved, és kenyeret nem eszel?
\par 6 És õ monda néki: Mert ama Jezréelbeli Nábóttal beszéltem, és azt mondám néki: Add nékem a te szõlõdet pénzért, vagy ha inkább tetszik, más szõlõt adok néked érette, õ pedig azt mondotta: Nem adom néked az én szõlõmet.
\par 7 Akkor monda néki Jézabel, az õ felesége: Te bírod-é most az Izráel királyságát? Kelj fel, egyél kenyeret, és a te szíved örvendezzen, én majd néked adom a Jezréelbeli Nábót szõlõjét.
\par 8 És levelet írt Akháb nevével, a melyet megpecsételt az õ gyûrûjével, és elküldé azt a levelet a véneknek és a fõembereknek, a kik Nábóttal egy városban laktak;
\par 9 És a levélben ezt írta, mondván: Hirdessetek bõjtöt, és ültessétek Nábótot a nép élére;
\par 10 És ültessetek vele szembe két istentelen embert, a kik tanúbizonyságot tegyenek õ ellene, mondván: Megszidalmaztad az Istent és a királyt. Azután vigyétek ki, és kövezzétek meg õt, hogy meghaljon.
\par 11 És ekképen cselekedének a vének és a fõemberek, a kik az õ városában laktak, a mint Jézabel nékik megparancsolta, és a mint a levélben megírta, a melyet nékik küldött;
\par 12 Bõjtöt hirdetének, és ülteték Nábótot a nép élére.
\par 13 Elõjöve azután két istentelen ember, és leült vele szemben, és tanúbizonyságot tettek ez istentelen emberek Nábót ellen a nép elõtt, mondván: Megszidalmazta Nábót az Istent és a királyt. Kivivék azért õt a városból, és megkövezék, és meghala.
\par 14 Azután megizenék Jézabelnek, mondván: Megköveztetett Nábót, és meghalt.
\par 15 Mikor pedig meghallotta Jézabel, hogy megköveztetett Nábót és meghalt, monda Jézabel Akhábnak: Kelj fel és foglald el a Jezréelbeli Nábót szõlõjét, a melyet nem akart néked pénzért oda adni; mert nem él Nábót, hanem meghalt.
\par 16 És mikor meghallotta Akháb, hogy Nábót meghalt, felkelt Akháb, hogy lemenjen a Jezréelbeli Nábót szõlõjébe, és azt elfoglalja.
\par 17 Akkor szóla az Úr Thesbites Illésnek, mondván:
\par 18 Kelj fel, és menj Akhábnak, az Izráelbeli királynak eleibe, a ki Samariában lakik, ímé ott van a Nábót szõlõjében, a melybe lement, hogy azt elfoglalja;
\par 19 És szólj néki, ezt mondván: Így szól az Úr: Nemde megölted-é és nemde el is foglaltad-é? És szólj néki, mondván: Ezt mondja az Úr: Ugyanazon a helyen, a hol felnyalták az ebek Nábót vérét, ebek nyalják fel a te véredet is!
\par 20 És monda Akháb Illésnek: Megint rám találtál ellenségem? És õ monda: Rád találtam; mert te magadat mindenestõl arra adtad, hogy gonoszságot cselekedjél az Úr szemei elõtt.
\par 21 Ímé, azt mondja az Úr: Veszedelmet hozok reád, és elvesztem a te maradékaidat, és kigyomlálom Akhábnak még az ebét is, és a berekesztettet és az elhagyatottat Izráelben;
\par 22 És olyanná teszem a te házadat, mint Jeroboámnak, a Nébát fiának házát, és mint Baásának,  az Ahija fiának házát azért, a miért haragra ingerlettél engem, és a miért bûnbe ejtetted az Izráelt.
\par 23 És Jézabel felõl is szóla az Úr, mondván: Az ebek eszik meg Jézabelt Jezréel kõfala elõtt;
\par 24 Azt, a ki Akháb házából a városban hal meg, az ebek eszik meg; azt pedig, a ki a mezõben hal meg, az égi madarak eszik meg.
\par 25 Mert bizonyára nem volt Akhábnak mássa, a ki magát arra adta volna, hogy csak gonoszságot cselekedjék az Úr szemei elõtt, a melyre az õ felesége, Jézabel  ösztökélé õt.
\par 26 Mert igen útálatos dolgot cselekedék, követvén a bálványokat mind a szerint, a mint cselekedének az Emoreusok, a kiket az Úr kiûzött volt az Izráel fiai elõtt.
\par 27 Lõn pedig, mikor meghallotta Akháb e beszédeket, megszaggatá az õ ruháit, és zsákba öltözék és bõjtöle, és a zsákban hála, és nagy alázatossággal jár vala.
\par 28 És szóla az Úr Thesbites Illésnek, mondván:
\par 29 Nem láttad-é, hogy Akháb hogyan megalázta magát elõttem? Mivelhogy pedig megalázta magát elõttem, nem hozom reá a veszedelmet az õ életében, hanem csak az õ fia idejében hozom el a veszedelmet az õ házára.

\chapter{22}

\par 1 És három esztendõ lefolyt úgy, hogy nem volt hadakozás a Siriabeliek és az Izráel között.
\par 2 De a harmadik esztendõben aláméne Josafát, Júda királya az Izráel királyához,
\par 3 Akkor monda az Izráel királya az õ szolgáinak: Nem tudjátok-é, hogy Rámoth Gileád a miénk? És mi hallgatunk és nem veszszük vissza azt Siria királyától?
\par 4 És monda Josafátnak: Feljössz-é velem e hadba Rámoth Gileád ellen? Felele Josafát az Izráel királyának: Úgy én, mint te, úgy az én népem, mint a te néped; úgy az én lovaim, mint a te lovaid.
\par 5 És monda Josafát az Izráel királyának: Kérdezõsködjél még ma az Úr beszéde után.
\par 6 És összegyûjté az Izráel királya a prófétákat, közel négyszáz férfiút, és monda nékik: Elmenjek-é Rámoth Gileád ellen hadba, vagy elhagyjam? És mondának: Menj fel; mert az Úr kezébe adja azt a királynak.
\par 7 És monda Josafát: Nincs itt már több prófétája az Úrnak, hogy attól is tudakozódhatnánk?
\par 8 És monda az Izráel királya Josafátnak: Még van egy férfiú, a ki által megkérdhetjük az Urat, Mikeás, a Jemla fia; de én gyûlölöm õt, mert soha nem jövendöl nékem jót, hanem mindig csak rosszat. És monda Josafát: Ne beszéljen így a király!
\par 9 És elõszólítá az Izráel királya egy udvariszolgát, és monda: Hívd ide hamar Mikeást, a Jemla fiát.
\par 10 És ott ült az Izráel királya és Josafát, Júda királya, ruhákba öltözötten, mindenik a maga trónusán Samaria kapuja elõtt a térségen, és a próféták mind ott jövendöltek elõttük.
\par 11 És Sédékiás, a Kénaána fia, vasszarvakat készített, és monda: Ezt mondja az Úr: Ezekkel ökleled a Siriabelieket, míg meg nem emészted õket.
\par 12 És a próféták is mindnyájan ekképen jövendöltek, mondván: Menj fel Rámoth Gileád ellen, és jó szerencséd lesz; mert azt az Úr a király kezébe adja.
\par 13 A követ pedig, a ki elment volt, hogy elhívja Mikeást, szóla néki, mondván: Ímé a próféták egyenlõ akarattal jót jövendölnek a királynak: szólj, kérlek, te is úgy, mint azok közül egy, és jövendölj jót.
\par 14 Mikeás pedig monda: Él az Úr, hogy csak azt fogom mondani, a mit az Úr mondánd nékem.
\par 15 És mikor a királyhoz ment, monda néki a király: Mikeás! elmenjünk-é Rámoth Gileád ellen hadba, vagy elhagyjuk? Õ pedig monda néki: Menj fel és járj szerencsével; az Úr kezébe adja azt a királynak.
\par 16 És monda néki a király: Még hányszor kényszerítselek téged, hogy az igaznál egyebet ne mondj nékem az Úr nevében?
\par 17 És monda: Látám az egész Izráelt szétszéledve a hegyeken, mint a juhokat, a melyeknek nincsen pásztoruk. És monda az Úr: Nincsen ezeknek urok? Térjen vissza kiki az õ házához békességben.
\par 18 És monda az Izráel királya Josafátnak: Nemde nem megmondottam-é néked, hogy soha jót nékem nem jövendöl, hanem csak rosszat.
\par 19 És monda Mikeás: Azért halld meg most az Úr beszédét: Látám az Urat az õ székiben ülni, és az egész mennyei sereget az õ jobb- és balkeze felõl mellette állani.
\par 20 És monda az Úr: Kicsoda csalja meg Akhábot, hogy felmenjen, és elvesszen Rámoth Gileádnál? És ki egyet, ki mást szól vala hozzá.
\par 21 Akkor elõjõve egy lélek, a ki az Úr eleibe álla, és monda: Én akarom megcsalni õt. Az Úr pedig monda néki: Miképen?
\par 22 És felele: Kimegyek és hazug lélek leszek minden õ prófétáinak szájában. Akkor monda az Úr: Csald meg és gyõzd meg; menj ki, és cselekedjél úgy.
\par 23 Ímé az Úr a hazugságnak lelkét adta mindezeknek a te prófétáidnak szájába; és az Úr szólott veszedelmes dolgot ellened.
\par 24 Akkor odalépett Sédékiás, a Kénaána fia, és arczul csapván Mikeást, monda: Hogyan? Eltávozott volna én tõlem az Úrnak lelke, hogy csak néked szólana?
\par 25 És monda Mikeás: Ímé meglátod azon a napon, a mikor az egyik kamarából a másik kamarába mégy be, hogy elrejtõzhess.
\par 26 Az Izráel királya pedig monda: Fogjad Mikeást, és vidd vissza õt Ammonhoz, a város fejedelméhez, és Joáshoz, a király fiához;
\par 27 És mondjad: Ezt mondja a király: Vessétek ezt a tömlöczbe, és tápláljátok õt a nyomorúság kenyerével és a nyomorúság vizével, míg békességgel megjövök.
\par 28 Monda pedig Mikeás: Ha békességgel térsz vissza, nem az Úr szólott én általam: Azután monda: Halljátok ezt meg minden népek!
\par 29 És felvonult az Izráel királya, és Josafát, a Júda királya Rámoth Gileád ellen.
\par 30 És monda az Izráel királya Josafátnak: Megváltoztatom ruhámat, és úgy megyek a viadalra, te pedig öltözzél fel ruhádba. És elváltoztatá ruháját az Izráel királya, és úgy méne a viadalra.
\par 31 Siria királya pedig megparancsolá az õ szekerei harminczkét fejedelmeinek, mondván: Ne vívjatok se kicsinynyel, se nagygyal, hanem egyedül csak az Izráel királya ellen vívjatok.
\par 32 És a mikor meglátták a szekerek fejedelmei Josafátot, mondának: Nyilván ez az Izráel királya; és reá rohanván vívának ellene. De Josafát elkezdett kiáltani.
\par 33 Mikor pedig látták a szekerek fejedelmei, hogy nem az Izráel királya, ott hagyták.
\par 34 Egy ember pedig csak úgy találomra kilövé az õ kézívét, és találá az Izráel királyát a pánczél és a kapocs között. És õ monda a kocsisának: Fordulj meg és vigy ki engem a táborból, mert megsebesültem!
\par 35 És az ütközet mind erõsebb lett azon a napon, és a király az õ szekerében állott a Siriabeliek ellen, és meghalt este felé, és a vér a sebbõl a szekérbe csorgott.
\par 36 És kikiálták napnyugotkor a táborban, mondván: Minden ember menjen haza a maga városába és földjébe!
\par 37 És meghalt a király és visszavitetvén Samariába, eltemeték a királyt Samariában.
\par 38 És mikor mosták az õ szekerét a Samaria mellett lévõ tóban: az ebek nyalták az õ vérét, és paráznák fürödtek ott, az Úrnak beszéde szerint, a melyet szólott volt.
\par 39 Akhábnak egyéb dolgai pedig és minden cselekedetei, az elefántcsontból épített ház, és mind a városok, a melyeket épített, vajjon nincsenek-é megírva az Izráel királyainak krónika-könyvében?
\par 40 És elaluvék Akháb az õ atyáival; és uralkodék õ utána fia, Akházia.
\par 41 És Josafát, az Asa fia lett királylyá Júdában, Akhábnak, az Izráel királyának negyedik esztendejében.
\par 42 És Josafát harminczöt esztendõs volt, mikor uralkodni kezdett, és huszonöt esztendeig uralkodott Jeruzsálemben. Az õ anyjának Azuba volt a neve, Silhi leánya.
\par 43 És jára Asának, az õ atyjának minden útjában, és abból ki nem tére, azt cselekedvén, a mi az Úr szemei elõtt kedves.
\par 44 Csakhogy a magaslatokat nem rombolták le, és a nép még áldozott és tömjénezett a magaslatokon.
\par 45 És békességben élt Josafát az Izráel királyával.
\par 46 Josafátnak egyéb dolga pedig és az õ ereje, a melylyel cselekedett, és a melylyel hadakozott, vajjon nincsenek-é megírva a Júda királyainak krónika-könyvében?
\par 47 A férfi paráznákat is, a kik még megmaradtak volt az õ atyjának, Asának idejébõl, kiûzte az országból.
\par 48 Akkor nem volt király Edomban, hanem csak helyettes király.
\par 49 És Josafát Társis hajókat csináltatott, hogy aranyért mennének Ofirba, de nem mehettek el; mert a hajók összetörtek  Esiongáberben.
\par 50 Akkor mondá Akházia, az Akháb fia, Josafátnak: Hadd menjenek el az én szolgáim a te szolgáiddal e hajókon; de Josafát nem akará.
\par 51 És elaluvék Josafát az õ atyáival; és eltemetteték az õ atyáival az õ atyjának, Dávidnak városában; és az õ fia, Jórám, uralkodék helyette.
\par 52 Akházia pedig, az Akháb fia kezde uralkodni Izráelen Samariában, Josafátnak, a Júda királyának tizenhetedik esztendejében, és uralkodék Izráelben két esztendeig.
\par 53 És gonoszul cselekedék az Úrnak szemei elõtt, járván az õ atyjának és anyjának útján, és Jeroboámnak, a Nébát fiának útján, a ki bûnbe ejté az Izráelt;
\par 54 És szolgála a Baálnak, és azt imádá és haragra indítá az Urat, Izráel Istenét, mint a hogy az õ atyja cselekedett.


\end{document}