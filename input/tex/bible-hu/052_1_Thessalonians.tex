\begin{document}

\title{1 Thessalonians}


\chapter{1}

\par 1 Pál, Silvánus és Timótheus a Thessalonikabeliek gyülekezetének, a mely van az Atya Istenben és az Úr Jézus Krisztusban: kegyelem néktek és békesség Istentõl, a mi Atyánktól, és az Úr Jézus Krisztustól.
\par 2 Hálát adunk az Istennek mindenkor mindnyájatokért, emlékezvén rólatok a mi imádságainkban;
\par 3 Szüntelenül emlegetve a ti hitetek munkáját, és a ti szeretetetek fáradozását, és a mi Urunk Jézus Krisztus felõl való reménységeteknek állhatatosságát, az Isten elõtt, a mi Atyánk elõtt:
\par 4 Tudván, Istentõl szeretett atyámfiai, hogy ti ki vagytok választva;
\par 5 Hogy a mi evangyéliomunk ti nálatok nem áll csak szóban, hanem isteni erõkben is, Szent Lélekben is, sok bizodalomban is; a miképen tudjátok, hogy milyenek voltunk közöttetek ti érettetek.
\par 6 És ti a mi követõinkké lettetek és az Úréi, befogadván az ígét sokféle szorongattatás között, Szent Lélek örömével;
\par 7 Úgy hogy példaképekké lettetek Maczedóniában és Akhájában minden hívõre nézve.
\par 8 Mert nemcsak Maczedóniában és Akhájában zendült ki tõletek az Úr beszéde, hanem minden helyen is híre terjedt a ti Istenben vetett hiteteknek, annnyira, hogy szükségtelen arról valamit szólnunk.
\par 9 Mert azok magok hirdetik felõlünk, milyen volt a mi hozzátok való menetelünk, és miként tértetek meg az Istenhez a bálványoktól, hogy az élõ és igaz Istennek szolgáljatok,
\par 10 És várjátok az Õ Fiát az égbõl, a kit feltámasztott a halálból, a Jézust, a ki megszabadít minket amaz eljövendõ haragtól.

\chapter{2}

\par 1 Mert magatok tudjátok, atyámfiai, hogy a mi ti hozzátok való menetelünk nem volt hiábavaló;
\par 2 Sõt inkább, noha elébb már háborúságot és bosszúságot szenvedtünk volt Filippiben, a mint tudjátok, volt bátorságunk a mi Istenünkben, hogy közöttetek is hirdessük az Isten evangyéliomát sok tusakodással.
\par 3 Mert a mi buzdításunk nem hitetésbõl van, sem nem tisztátalanságból, sem nem álnokságból:
\par 4 Hanem a miképen az Isten méltatott minket arra, hogy reánk bízza az evangyéliomot, akképen szólunk; nem úgy, hogy embereknek tessünk hanem az Istennek, a ki megvizsgálja a mi szívünket.
\par 5 Mert sem hízelkedõ beszéddel, a mint tudjátok, sem telhetetlenség színében nem léptünk fel soha, Isten a bizonyság;
\par 6 Sem emberektõl való dicsõítést nem kerestünk, sem tõletek, sem másoktól, holott terhetekre lehettünk volna, mint Krisztus apostolai.
\par 7 De szívélyesek valánk ti közöttetek, a miként a dajka dajkálgatja az õ gyermekeit.
\par 8 Így felbuzdulva irántatok, készek valánk közleni veletek nemcsak az Isten evangyéliomát, hanem a mi magunk lelkét is, mivelhogy szeretteinkké lettetek nékünk.
\par 9 Emlékezhettek ugyanis atyámfiai a mi fáradozásunkra és bajlakodásunkra: mert éjjel-nappal munkálkodva hirdettük néktek az Isten evangyéliomát, csakhogy senkit meg ne terheljünk közületek.
\par 10 Ti vagytok bizonyságok és az Isten, milyen szentül, igazán és feddhetetlenül éltünk elõttetek, a kik hisztek.
\par 11 Valamint tudjátok, hogy miként atya az õ gyermekeit, úgy intettünk és buzdítgattunk egyenként mindnyájatokat.
\par 12 És kérve kértünk, hogy Istenhez méltóan viseljétek magatokat, a ki az õ országába és dicsõségébe hív titeket.
\par 13 Ugyanazért mi is hálát adunk az Istennek szüntelenül, hogy ti befogadván az Istennek általunk hirdetett beszédét, nem úgy fogadtátok, mint emberek beszédét, hanem mint Isten beszédét (a minthogy valósággal az is), a mely munkálkodik is ti bennetek, a kik hisztek.
\par 14 Mert ti, atyámfiai, követõi lettetek az Isten gyülekezeteinek, a melyek Júdeában vannak a Krisztus Jézusban, mivelhogy ugyanúgy szenvedtetek ti is a saját honfitársaitoktól, miként azok is a zsidóktól,
\par 15 A kik megölték az Úr Jézust is és a saját prófétáikat, és minket is üldöznek, és az Istennek nem tetszenek, és minden embernek ellenségei;
\par 16 A kik megtiltják nékünk, hogy a pogányoknak ne prédikáljunk hogy üdvözüljenek; hogy mindenkor betöltsék bûneiket; de végre utólérte õket az Isten haragja.
\par 17 Mi pedig, atyámfiai, a mint elszakasztatánk tõletek egy kevés ideig, arczra, nem szívre nézve, annál buzgóságosabban, nagy kívánsággal igyekeztünk, hogy szemtõl-szemben láthassunk titeket.
\par 18 Azért menni is akartunk hozzátok, kiváltképen én Pál, egyszer is, kétszer is, de megakadályozott minket a Sátán.
\par 19 Mert kicsoda a mi reménységünk, örömünk és dicsekedésünk koronája? Avagy nem azok lesztek-é ti is a mi Urunk Jézus Krisztus elõtt az õ eljövetelekor?
\par 20 Bizony ti vagytok a mi dicsõségünk és örömünk.

\chapter{3}

\par 1 Annakokáért, mivelhogy tovább már el nem tûrhetõk, jónak ítélénk, hogy magunk maradjunk Athénében,
\par 2 És elküldöttük Timótheust, a mi atyánkfiát és Istennek szolgáját és munkatársunkat a Krisztus evangyéliomának hirdetésében, hogy erõsítsen titeket és intsen titeket a ti hitetek felõl;
\par 3 Hogy senki meg ne tántorodjék ama szorongattatások között; mert ti magatok tudjátok, hogy mi arra rendeltettünk.
\par 4 Mert mikor közöttetek valánk is, elõre megmondtuk néktek, hogy szorongattatásnak leszünk kitéve; a mint meg is történt, és tudjátok.
\par 5 Annakokáért én is, mivelhogy tovább már nem tûrhetém, elküldék, hogy megismerjem a ti hiteteket, ha nem kísértett-é meg valami módon titeket a kísértõ, és nem lett-é hiábavaló a mi fáradságunk?
\par 6 Most pedig, a mikor megérkezett hozzánk Timótheus ti tõletek, és örömhírt hozott nékünk a ti hitetek és szeretetetek felõl, és arról, hogy jó emlékezéssel vagytok irántunk, mindenkor kívánván látni minket, miképen mi is titeket;
\par 7 Ezáltal megvígasztalódtunk reátok nézve, atyámfiai, minden mi szorongattatásunk és szükségünk mellett is, a ti hitetek által:
\par 8 Mert szinte megelevenedtünk, ha ti erõsek vagytok az Úrban.
\par 9 Mert milyen hálával is fizethetünk az Istennek érettetek, mindazért az örömért, a melylyel örvendezünk miattatok a mi Istenünk elõtt?!
\par 10 Mikor éjjel-nappal nagy buzgón esedezünk, hogy megláthassuk a ti orczátokat, és kipótolhassuk a ti hitetek hiányait.
\par 11 Maga pedig az Isten és a mi Atyánk, és a mi Urunk a Jézus Krisztus egyengesse meg a mi útunkat ti hozzátok!
\par 12 Titeket pedig gyarapítson az Úr és tegyen bõségesekké az egymás iránt és mindenki iránt való szeretetben, a milyenek vagyunk mi is irántatok;
\par 13 Hogy erõsekké tegye a ti szíveteket, feddhetetlenekké a szentségben, a mi Istenünk és Atyánk elõtt, a mikor eljõ a mi Urunk Jézus Krisztus minden õ szenteivel egyetemben.

\chapter{4}

\par 1 Továbbá pedig kérünk titeket, atyámfiai, és intünk az Úr Jézusban, hogy a szerint, a mint tõlünk tanultátok, mimódon kell forgolódnotok és Istennek tetszenetek: mindinkább gyarapodjatok.
\par 2 Mert tudjátok, milyen parancsolatokat adtunk néktek az Úr Jézus által.
\par 3 Mert ez az Isten akaratja, a ti szentté lételetek, hogy magatokat a paráznaságtól megtartóztassátok;
\par 4 Hogy mindenitek szentségben és tisztességben tudja bírni a maga edényét,
\par 5 Nem kívánság gerjedelmével, mint a pogányok, a kik nem ismerik az Istent;
\par 6 Hogy senki túl ne lépjen és meg ne károsítsa valamely dologban az õ atyjafiát: mert bosszút áll az Úr mindezekért, a mint elébb is mondottuk néktek és bizonyságot tettünk.
\par 7 Mert nem tisztátalanságra, hanem szentségre hívott el minket az Isten.
\par 8 A ki azért megveti ezeket, nem embert vet meg, hanem az Istent, a ki Szent Lelkét is közlé velünk.
\par 9 Az atyafiúi szeretetrõl pedig nem is szükség írnom néktek: mert titeket Isten maga tanított meg arra, hogy egymást szeressétek;
\par 10 Sõt gyakoroljátok is azt mindamaz atyafiak iránt, a kik egész Maczedóniában vannak. Kérünk azonban titeket atyámfiai, hogy mindinkább gyarapodjatok;
\par 11 És becsületbeli dolognak tartsátok, hogy csendes életet folytassatok, saját dolgaitoknak utána lássatok, és tulajdon kezeitekkel munkálkodjatok, a miként rendeltük néktek;
\par 12 Hogy a kívülvalók iránt tisztességesen viselkedjetek, és semmi szükséget ne érezzetek.
\par 13 Nem akarom továbbá, atyámfiai, hogy tudatlanságban legyetek azok felõl, a kik elaludtak, hogy ne bánkódjatok, mint a többiek, a kiknek nincsen reménységök.
\par 14 Mert ha hisszük, hogy Jézus meghalt és feltámadott, azonképen az Isten is elõhozza azokat, a kik elaludtak, a Jézus által õ vele együtt.
\par 15 Mert ezt mondjuk néktek az Úr szavával, hogy mi, a kik élünk, a kik megmaradunk az Úr eljöveteléig, épen nem elõzzük meg azokat, a kik elaludtak.
\par 16 Mert maga az Úr riadóval, arkangyal szózatával és isteni harsonával leszáll az égbõl: és feltámadnak elõször a kik meghaltak volt a Krisztusban;
\par 17 Azután mi, a kik élünk, a kik megmaradunk, elragadtatunk azokkal együtt a felhõkön az Úr elébe a levegõbe; és ekképen mindenkor az Úrral leszünk.
\par 18 Annakokáért vígasztaljátok egymást e beszédekkel.

\chapter{5}

\par 1 Az idõkrõl és idõszakokról pedig, atyámfiai, nem szükség, hogy írjak néktek;
\par 2 Mert igen jól tudjátok ti magatok, hogy az Úrnak napja úgy jõ el, mint a tolvaj éjjel.
\par 3 Mert a mikor ezt mondják: Békesség és biztonság, akkor hirtelen veszedelem jön rájok, mint a szülési fájdalom a terhes asszonyra; és semmiképen meg nem menekednek.
\par 4 De ti, atyámfiai, nem vagytok sötétségben, hogy az a nap tolvaj módra lephetne meg titeket.
\par 5 Ti mindnyájan világosság fiai vagytok és nappal fiai; nem vagyunk az éjszakáé, sem a sötétségé!
\par 6 Ne is aludjunk azért, mint egyebek, hanem legyünk éberek és józanok.
\par 7 Mert a kik alusznak, éjjel alusznak; és a kik részegek, éjjel részegednek meg.
\par 8 Mi azonban, a kik nappaliak vagyunk, legyünk éberek, felöltözvén a hitnek és szeretetnek mellvasába, és sisak gyanánt az üdvösségnek reménységébe.
\par 9 Mert nem haragra rendelt minket az Isten, hanem arra, hogy üdvösséget szerezzünk a mi Urunk Jézus Krisztus által,
\par 10 A ki meghalt érettünk, hogy akár ébren vagyunk, akár aluszunk, együtt éljünk õ vele.
\par 11 Vígasztaljátok azért egymást, és építse egyik a másikat, a miképen cselekeszitek is.
\par 12 Kérünk továbbá titeket atyámfiai, hogy becsüljétek azokat, a kik fáradoznak közöttetek, és elõljáróitok az Úrban, és intenek titeket;
\par 13 És az õ munkájokért viseltessetek irántok megkülönböztetett szeretettel. Egymással békességben éljetek.
\par 14 Kérünk továbbá titeket, atyámfiai, intsétek a rendetleneket, bátorítsátok a félelmes szívûeket, gyámolítsátok az erõteleneket, türelmesek legyetek mindenki iránt.
\par 15 Vigyázzatok, hogy senki senkinek rosszért rosszal ne fizessen; hanem mindenkor jóra törekedjetek úgy egymás iránt, mint mindenki iránt.
\par 16 Mindenkor örüljetek.
\par 17 Szüntelen imádkozzatok.
\par 18 Mindenben hálákat adjatok; mert ez az Isten akarata a Krisztus Jézus által ti hozzátok.
\par 19 A Lelket meg ne oltsátok.
\par 20 A prófétálást meg ne vessétek,
\par 21 Mindent megpróbáljatok; a mi jó, azt megtartsátok!
\par 22 Mindentõl, a mi gonosznak látszik, õrizkedjetek!
\par 23 Maga pedig a békességnek Istene szenteljen meg titeket mindenestõl; és a ti egész valótok, mind lelketek, mind testetek feddhetetlenül õriztessék meg a mi Urunk Jézus Krisztus eljövetelére.
\par 24 Hû az, a ki elhivott titeket és õ meg is cselekszi azt.
\par 25 Atyámfiai, imádkozzatok érettünk.
\par 26 Köszöntsétek az összes atyafiakat szent csókolással.
\par 27 Kényszerítlek titeket az Úrra, hogy olvastassék fel e levél minden szent atyafi elõtt.
\par 28 A mi Urunk Jézus Krisztusnak kegyelme veletek! Ámen.


\end{document}