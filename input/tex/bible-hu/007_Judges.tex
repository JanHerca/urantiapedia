\begin{document}

\title{Judges}


\chapter{1}

\par 1 És lõn Józsué halála után, hogy megkérdék az Izráel fiai az Urat, mondván: Ki menjen legelõször közülünk a Kananeusra, hogy hadakozzék õ ellene?
\par 2 És monda az Úr: Júda menjen! Ímé az õ kezébe adtam azt a földet.
\par 3 Ekkor monda Júda az õ atyjafiának Simeonnak: Jer velem együtt a nékem sors által jutott örökségbe, hogy hadakozzunk a Kananeus ellen, és én is elmegyek veled a néked sors által jutott örökségbe. És Simeon elméne vele.
\par 4 És felméne Júda és kezökbe adá az Úr a Kananeust és a Perizeust, és levágtak közülök Bézekben tízezer embert.
\par 5 És találák Adonibézeket Bézekben, és hadakozának õ ellene, és megverék a Kananeust és a Perizeust.
\par 6 És megfutamodott Adonibézek, de ûzõbe vevék, és elfogván õt, elvágták kezeinek és lábainak hüvelykújjait.
\par 7 Ekkor monda Adonibézek: Hetven király szedeget vala asztalom alatt, a kiknek elvágattam kezeik és lábaik hüvelykujjait; a mint én cselekedtem, úgy fizetett nékem az Isten. Azután elvivék õt Jeruzsálembe, és ott meghala.
\par 8 Azután Jeruzsálem ellen hadakoztak a Júda fiai, és elfoglalván azt, lakosait levágták fegyvernek élivel, a várost pedig lángba borították.
\par 9 Azután pedig alámenének a Júda fiai, hogy harczoljanak a hegységen, a déli tartományban és a lapályon lakó Kananeus ellen.
\par 10 És elment Júda a Hebronban lakó Kananeus ellen, - Hebron neve azelõtt Kirjáth-Arba volt, - és megverték Sésait, Achimánt és Thalmáit.
\par 11 Onnan pedig méne Debir lakói ellen. Debir neve pedig azelõtt Kirjáth-Széfer volt.
\par 12 És monda Káleb: A ki Kirjáth-Széfert megveri és elfoglalja, annak feleségül adom az én leányomat, Akszát.
\par 13 És elfoglalá azt Othniel, Kénáz fia, Káleb öcscse, és néki adá Akszát, az õ leányát feleségül.
\par 14 És lõn, hogy a mikor eljöve az, biztatá õt, hogy kérjen mezõt az õ atyjától. Leszálla azért a szamárról, Káleb pedig monda néki: Mi bajod van?
\par 15 Õ pedig monda néki: Adj áldást nékem: mert déli vidékre helyheztettél engem, adj most vízforrásokat is. És néki adá Káleb a felsõ forrást és az alsó forrást.
\par 16 És Keneusnak, a Mózes ipának fiai is felmentek a pálmák városából a Júda fiaival a Júda pusztájára, mely délre van Aradtól, és elmentek és letelepedtek a nép között.
\par 17 És elment Júda az õ atyjafiával, Simeonnal, és megverék a Czéfátban lakó Kananeust, és elpusztították azt, és elnevezték a várost Hormának.
\par 18 Azután elfoglalta Júda Gázát és annak határát, Askelont és annak határát, Ekront és annak határát.
\par 19 Vala pedig az Úr Júdával, és kiûzé a hegység lakóit; de a völgy lakóit nem lehetett kiûzni, mert vas-szekereik voltak.
\par 20 És Kálebnek adták Hebront, a mint meghagyta volt Mózes, és kiûzé onnan Anák három fiát.
\par 21 De a Jebuzeust, Jeruzsálemnek lakóját nem ûzték el a Benjámin fiai, és ezért lakik a Jebuzeus a Benjámin fiaival Jeruzsálemben még máig is.
\par 22 És felméne a József háza is, Béthel ellen, és az Úr volt õ vele.
\par 23 És kikémlelteté a József háza Béthelt. A város neve pedig régenten Lúz volt.
\par 24 És látának a kémek egy férfiút, a ki a városból jött vala, és mondának néki: Mutasd meg a város bejárását, és irgalmasságot cselekszünk veled.
\par 25 És mikor ez megmutatta nékik a város bejárását, a várost fegyver élére hányták; de azt a férfiút és egész házanépét elbocsáták.
\par 26 És elment az a férfiú a Khitteusok földére, és várost épített, és elnevezé azt Lúznak. Ez annak neve mind e mai napig.
\par 27 És nem ûzte el Manassé a lakosokat sem Béthseánból és annak mezõvárosaiból, sem Taanakból és mezõvárosaiból, sem Dórból és mezõvárosaiból, sem a Jibleámnak és mezõvárosainak lakosait, sem Megiddónak és mezõvárosainak lakosait. És a Kananeusnak tetszett ott lakni azon a földön.
\par 28 És mikor Izráel megerõsödött, adófizetõjévé tette a Kananeust; de elûzni nem ûzte el.
\par 29 És Efraim sem ûzte ki a Kananeust, a ki Gézerben lakott; hanem a Kananeus ott lakott közöttük Gézerben.
\par 30 Zebulon sem ûzte el sem Kitron lakosait, sem Nahalólnak lakóit, hanem közöttük lakozott a Kananeus, és adófizetõikké lettek.
\par 31 Áser sem ûzte el sem Akkó, sem Sidon, sem Akhláb, sem Akzib, sem Helba, sem Afik, sem Rehob lakóit.
\par 32 És ott lakott Áser a Kananeusok, a föld lakói között, mert nem ûzte el õket.
\par 33 Nafthali sem ûzte el sem Béth-Semes lakosait, sem Béth-Anath lakóit, hanem ott lakott a Kananeusok, a föld lakói között, és Béth-Semes és Béth-Anath lakói adófizetõikké lettek.
\par 34 Az Emoreusok pedig felszorították a Dán fiait a hegységbe, mert nem engedték meg, hogy lejõjjenek a völgybe.
\par 35 És az Emoreusoknak tetszett ott lakni a Héresz hegységen, Ajalonban és Saalbimban; de mikor a József házának keze rájok nehezedett, adófizetõkké lettek.
\par 36 Az Emoreusok határa pedig az Akrabbim hágójától és Sélától fölfelé volt.

\chapter{2}

\par 1 Felméne pedig az Úrnak angyala Gilgálból Bókimba, és monda: Én vezettelek fel titeket Égyiptomból, és hoztalak benneteket erre a földre, a mely felõl megesküdtem a ti atyáitoknak, és mondék: Nem bontom fel az én szövetségemet ti veletek soha örökké.
\par 2 Csakhogy ti se kössetek frigyet ennek a földnek lakosival, rontsátok le az õ oltáraikat;  de ti nem hallgattatok az én szómra. Miért cselekedtétek ezt?
\par 3 Annakokáért azt mondom: Nem ûzöm el õket elõletek, hanem legyenek néktek mint tövisek a ti oldalaitokban, és az õ isteneik legyenek ti néktek tõr gyanánt.
\par 4 És lõn, hogy mikor az Úrnak angyala ezeket mondá az Izráel minden fiainak: felemelé a nép az õ szavát és síra.
\par 5 És elnevezték azt a helyet Bókimnak, és áldozának ott az Úrnak.
\par 6 És elbocsátá Józsué a népet, és elmenének az Izráel fiai, kiki az õ örökségébe, hogy bírják a földet.
\par 7 És a nép az Urat szolgálta Józsuénak egész életében, és a véneknek minden napjaiban, a kik hosszú ideig éltek Józsué után, a kik látták az Úrnak minden dolgait, melyeket cselekedett vala Izráellel.
\par 8 És meghalt Józsué, a Nún fia, az Úr szolgája, száztíz esztendõs korában.
\par 9 És eltemeték õt örökségének határában, Timnat-Héreszben, az Efraim hegyén, északra a Gaas hegyétõl.
\par 10 És az az egész nemzetség gyûjteték az õ atyáihoz, és támadott más nemzetség õ utánok, a mely nem ismerte sem az Urat, sem az õ cselekedeteit, melyeket cselekedett Izráellel.
\par 11 És gonoszul cselekedtek az Izráel fiai az Úrnak szemei elõtt, mert a Baáloknak szolgáltak.
\par 12 És elhagyák az Urat, atyáik Istenét, a ki kihozta õket Égyiptom földébõl, és más istenek után jártak, a pogány népek istenei közül, a kik körülöttük voltak, és azok elõtt hajtották meg magukat, és haragra ingerelték az Urat.
\par 13 És elhagyták az Urat, és szolgáltak Baálnak és Astarótnak.
\par 14 És felgerjedett az Úrnak haragja Izráel ellen, és adá õket a ragadozók kezébe, és elragadozák õket, és adá õket a körülöttük való ellenségeik kezébe, és még csak megállani sem bírtak ellenségeik elõtt.
\par 15 A hova csak kivonultak, mindenütt ellenök volt az Úr keze rontásukra, a mint megmondotta volt az Úr, és a mint megesküdt volt az Úr nékik. És igen megnyomorodának.
\par 16 És támasztott az Úr bírákat, a kik megszabadíták õket szorongatóiknak kezébõl.
\par 17 De bíráikra sem hallgattak, hanem más istenekkel paráználkodtak, és azoknak hajtották meg magukat, és hamar letértek az útról, a melyen atyáik jártak, kik az Úrnak parancsára hallgattak; õk nem cselekedtek így.
\par 18 Mert mikor bírákat támasztott az Úr nékik, az Úr maga volt a bíróval, és megszabadította õket ellenségeik kezébõl a biró egész idejére, mert megindult az Úr panaszaikon, melyeket emeltek elnyomóik és szorongatóik miatt.
\par 19 De mihelyt a bíró meghalt, visszatértek, és jobban megromlottak, mint atyáik; más istenek után jártak, azoknak szolgáltak, és azok elõtt hajtották meg magukat, föl nem hagytak cselekedeteikkel és nyakas magaviseletökkel.
\par 20 Ezért felgerjedt az Úrnak haragja Izráel ellen, és monda: Mivelhogy általhágta ez a nemzetség az én szövetségemet, melyet atyáiknak parancsoltam, és nem hallgatott az én szavamra:
\par 21 Én sem ûzök ki többé senkit sem elõlük azon pogányok közül, a kiket meghagyott Józsué, a mikor meghalt;
\par 22 Hanem azok által kísértem Izráelt, ha vajjon megtartják-é az Úrnak útát, járván azon, a miképen megtartották az õ atyáik, avagy nem?
\par 23 Ezért hagyta meg az Úr ezeket a pogányokat, és nem is ûzte ki hamar, és nem adta õket Józsué kezébe.

\chapter{3}

\par 1 Ezek pedig a pogányok, a kiket meghagyott az Úr, hogy azok által kísértse Izráelt, azokat a kik nem ismerték a Kanaánért való harczokat,
\par 2 Csak azért, hogy az Izráel fiainak nemzetségei megismerjék, hogy tanítsa õket hadakozásra, csak azokat, a kik azelõtt ezt nem tudták:
\par 3 A Filiszteusok öt fejedelemsége, a Kananeusok mindnyájan, és a Sidoniusok, meg a Khivveusok, a kik a Libánon hegyén laknak, a Baálhermon hegyétõl fogva Hamath bemeneteléig.
\par 4 Kik azért hagyattak meg, hogy megkísértse általok az Úr Izráelt, hogy meglássa, vajjon engedelmeskednek-é az Úr parancsolatainak melyeket parancsolt az õ atyáiknak Mózes által?
\par 5 Így az Izráel fiai a Kananeusok, Khitteusok, Emoreusok, Perizeusok, Khivveusok és Jebuzeusok között laktak,
\par 6 És azok leányait vették magoknak feleségül, a saját leányaikat pedig oda adták azok fiainak, és szolgálák azoknak isteneit.
\par 7 És gonoszul cselekedtek az Izráel fiai az Úr szemei elõtt, és elfelejtkezének az Úrról, az õ Istenökrõl, és a Baáloknak és Aseráknak szolgáltak.
\par 8 Ezért felgerjedett az Úrnak haragja az Izráel ellen, és adá õket Kusán-Risathaimnak, Mesopotámia királyának kezébe, és nyolcz évig szolgáltak az Izráel fiai Kusán-Risathaimnak.
\par 9 Ekkor az Úrhoz kiáltának az Izráel fiai, és az Úr szabadítót támasztott az Izráel fiainak, aki megszabadítá õket: Othnielt, Kénáz fiát, Káleb öcscsét.
\par 10 És az Úrnak lelke vala õ rajta, és biráskodott Izráelben, és a mint kiméne a hadra, az Úr kezébe adta Kusán-Risathaimot, Mesopotámia királyát, és ránehezült keze Kusán-Risathaimra.
\par 11 És megnyugovék a föld negyven esztendeig, és meghalt Othniel, a Kénáz fia.
\par 12 De az Izráel fiai ismét gonoszul cselekedének az Úr szemei elõtt, és megerõsítette az Úr Eglont, Moáb királyát Izráel ellen, azért, mert gonoszul cselekedtek az Úrnak szemei elõtt.
\par 13 Ez magához gyûjtötte Ammonnak és Amáleknek fiait, és elment, és megverte Izráelt, és elfoglalták a pálmák városát.
\par 14 És szolgálák az Izráel fiai Eglont, a Moáb királyát tizennyolcz esztendeig.
\par 15 Ekkor megint az Úrhoz kiáltottak Izráel fiai, és az Úr szabadítót támasztott nékik: Ehudot, Gérának fiát, ki Jemininek volt a fia, a ki suta volt. És mikor az Izráel fiai ajándékot küldének általa Eglonnak, Moáb királyának,
\par 16 Ehud szerze magának egy kétélû kardot, egy singnyi hosszút, és azt az õ ruhája alatt a jobb tomporára övezé.
\par 17 És bemutatá az ajándékot Eglonnak, Moáb királyának. Eglon pedig igen kövér ember vala.
\par 18 És lõn, hogy mikor elvégezé az ajándék bemutatását, elbocsátá az embereket, a kik az ajándékot vitték vala.
\par 19 Õ maga pedig visszatért a Gilgál közelében lévõ kõbányáktól és monda: Titkos beszédem van veled, óh király! És ez monda: Hallgass! és kimenének elõle mindnyájan, kik állanak vala körülötte.
\par 20 És Ehud akkor ment be hozzá, mikor épen az õ hûsölõ felházában ült magányosan, és monda Ehud néki: Istennek beszéde van nálam te hozzád. És fölkele az a királyi székbõl.
\par 21 Akkor kinyújtá Ehud az õ balkezét, és kirántá kardját a jobb tomporáról, és beleüté azt annak hasába.
\par 22 És beméne még a markolatja is a vasa után és berekeszté a háj a fegyver vasát, mert nem vonta ki a kardot annak hasából, sõt átment annak vékonyán.
\par 23 Általméne pedig Ehud a csarnokon, minekutána bevonta maga után a felház ajtait, és bezárta.
\par 24 A mint õ kiméne, jövének annak szolgái, és láták, hogy ímé a felház ajtai be vannak zárva, és mondának: Bizonyára szükségét végzi a hûsölõ kamarájában.
\par 25 De mikor váltig várakoztak és ímé nem nyitotta meg a felház ajtaját, fogák a kulcsot és megnyiták: és ímé az õ urok ott feküdt a földön halva.
\par 26 Ehud pedig elmenekült vala, míg azok késedelmezének, és elhaladván a kõbányák mellett, Szeiráhba futott.
\par 27 És a mikor oda megérkezett, kürtjébe fújt az Efraim hegyén, és alámenének õ vele az Izráel fiai a hegyrõl, és õ elõttük.
\par 28 És monda nékik: Jertek utánam, mert kezetekbe adta az Úr a ti ellenségeteket, Moábot. És levonultak õ utána, és elfoglalák a Jordán réveit, melyek Moáb felé vezettek, és nem engedtek azokon senkit általkelni.
\par 29 És leöltek a Moábiták közül akkor mintegy tízezer embert, mind erõs és vitéz férfiakat, úgy hogy egy sem menekült el.
\par 30 Így aláztatott meg Moáb abban az idõben Izráel keze alatt. És megnyugovék a föld nyolczvan esztendeig.
\par 31 Õ utána pedig Sámgár, Anáthnak fia volt bíró, a ki hatszáz Filiszteus férfiút ölt meg egy ökörösztökével, és megszabadítá õ is Izráelt.

\chapter{4}

\par 1 De az Izráel fiai azután is gonoszul cselekedének az Úrnak szemei elõtt, mikor Ehud meghalt.
\par 2 Azért adá õket az Úr Jábinnak, a Kanaán királyának kezébe, a ki Hásorban uralkodott; seregének vezére pedig Sisera volt. Ez a pogányok városában, Harósethben lakott.
\par 3 És kiáltának az Izráel fiai az Úrhoz, mert kilenczszáz vas-szekere volt, és húsz esztendeig zsarnokoskodott az Izráel fiai felett erõszakkal.
\par 4 Ebben az idõben Debora, a prófétanõ, a Lappidoth felesége volt biró Izráelben.
\par 5 És õ a Debora-pálmája alatt lakott Ráma és Béthel között, az Efraim hegyén, és ide jöttek fel hozzá az Izráel fiai törvényre.
\par 6 És elkülde és hivatá Bárákot, az Abinoám fiát, Kedes-Nafthaliból, és monda néki: Avagy nem parancsolta-é meg az Úr, Izráelnek Istene: Menj és vonulj fel a Thábor hegyére, és végy magadhoz tízezer embert a Nafthali és a Zebulon fiai közül?
\par 7 És te ellened kihozom a Kison patakjához Siserát, Jábin hadvezérét, az õ szekereit és seregét, és kezedbe adom õt.
\par 8 És monda néki Bárák: Ha velem jösz, elmegyek; de ha nem jösz el velem, én sem megyek.
\par 9 Ki felele: Elmenvén elmegyek te veled, csakhogy nem a tied lesz a dicsõség az útban, a melyre mégy, mert asszony kezébe adja az Úr Siserát. És felkele Debora, és elméne Bárákkal Kedes felé.
\par 10 Összegyûjté annakokáért Bárák Zebulont és Nafthalit Kedesbe, és elvive magával tízezer embert, és elméne vele Debora is.
\par 11 A Keneus Héber pedig elvált a Keneusoktól, Hobábnak, a Mózes ipának fiaitól, és sátorát a  Saanaim tölgyesénél vonta fel, mely Kedes mellett van.
\par 12 Hírül vivék pedig Siserának, hogy Bárák, az Abinoám fia felvonult a Thábor hegyére.
\par 13 Egybegyûjté azért Sisera minden szekereit, a kilenczszáz vas-szekeret, és egész népét, mely vele volt, a pogányok városából, Harósethbõl, a Kison patakjához.
\par 14 És monda Debora Báráknak: Kelj fel, mert ez a nap az, a melyen kezedbe adja az Úr Siserát! hisz maga az Úr vezet téged! És alájöve Bárák a Thábor hegyérõl, és a tízezer ember õ utána.
\par 15 És megrettenté az Úr Siserát minden szekereivel és egész táborával fegyvernek élivel Bárák elõtt, anynyira, hogy Sisera leugrott szekerérõl, és gyalog futott.
\par 16 Bárák pedig a szekereket és a tábort egész Harósethig, a pogányok városáig ûzé, és Siserának egész tábora elhulla fegyvernek éle miatt, még csak egyetlen egy sem maradt meg.
\par 17 Sisera pedig gyalog futott Jáhelnek, a Keneus Héber feleségének sátoráig, mert béke volt Jábin között, Hásor királya között, és a Keneus Héber háznépe között.
\par 18 És kiméne Jáhel Sisera elé, és monda néki: Jõjj be Uram, jõjj be hozzám, ne félj! És betére õ hozzá a sátorba, és betakará õt lasnakkal.
\par 19 Az pedig monda néki: Adj kérlek innom egy kis vizet, mert szomjúhozom. És megnyita ez egy tejes tömlõt, és ada néki inni, és betakará õt.
\par 20 És az annakfelette monda néki: Állj a sátor ajtajába, és ha valaki jõne és kérdezne, és ezt mondaná: Van-é itt valaki? mondjad: Nincsen!
\par 21 Ekkor Jáhel, a Héber felesége vevé a sátorszöget, és põrölyt võn kezébe, és beméne õ hozzá halkan, és beveré a szöget halántékába, úgy hogy beszegezõdék a földbe. Amaz pedig nagy fáradtság miatt mélyen aludt vala, és így meghala.
\par 22 És ímé, a mint Bárák ûzve keresné Siserát, Jáhel kiméne elébe, és monda néki: Jöszte és megmutatom néked az embert, a kit keressz. És beméne õ hozzá, és ímé Sisera ott feküdt halva, és a szög halántékában.
\par 23 Így alázá meg Isten azon a napon Jábint, a Kanaán királyát az Izráel fiai elõtt.
\par 24 Az Izráel fiainak keze pedig mind jobban ránehezedék Jábinra, a Kanaán királyára, mígnem kiirták Jábint, a Kanaán királyát.

\chapter{5}

\par 1 Énekelt pedig Debora és Bárák, az Abinoám fia azon a napon, mondván:
\par 2 Hogy a vezérek vezettek Izráelben, Hogy a nép önként kele föl: áldjátok az Urat!
\par 3 Halljátok meg királyok, figyeljetek fejedelmek! Én, én az Úrnak éneket mondok, Dícséretet zengek az Úrnak, az Izráel Istenének.
\par 4 Uram, mikor Szeirbõl kijövél, Mikor lépdelél Edom mezejérõl: Megrendült a föld, csepegett az ég, A föllegek is víztõl áradának,
\par 5 A hegyek megrendültek az Úrnak orczája elõtt, Még ez a Sinai is, az Úrnak, az Izráel Istenének színe elõtt.
\par 6 Sámgárnak, az Anath fiának napjaiban, Jáhel idejében pihentek az utak, És az útonjárók tekervényes ösvényekre tértek.
\par 7 Megszüntek a kerítetlen helyek Izráelben, megszüntek végképen, Mígnem én Debora felkelék, Felkelék Izráel anyjaként.
\par 8 Új isteneket ha választott a nép, Mindjárt kigyúlt a harcz a kapuk elõtt; De paizs, és dárda avagy láttatott-é A negyvenezereknél az Izráel között?
\par 9 Szívem azoké, kik parancsolnak Izráelben, Kik a nép közül önként ajánlkoztak: áldjátok az Urat!
\par 10 Kik ültök fehér szamarakon, Kik ültök a szõnyegeken És a kik gyalog jártok: mind énekeljetek!
\par 11 Az íjászok szavával a vízmerítõk között, Ott beszéljék az Úrnak igazságát, Az õ faluihoz való igazságit Izráelben. Akkor újra a kapukhoz vonul az Úr népe!
\par 12 Kelj fel, kelj fel Debora! Serkenj fel, serkenj fel, mondj éneket! Kelj fel Bárák és fogva vigyed foglyaidat, Abinoám fia!
\par 13 Akkor lejött a hõsök maradéka; Az Úrnak népe lejött hozzám a hatalmasok ellen.
\par 14 Efraimból, kiknek gyökere Amálekben, Utánad Benjámin, a te néped közé; Mákirból jöttek le vezérek, És Zebulonból, kik a vezéri pálczát tartják.
\par 15 És Issakhár fejedelmei Deborával, És mint Issakhár, úgy Bárák A völgybe rohan követõivel. Csak a Rúben patakjainál Vannak nagy elhatározások.
\par 16 Miért maradtál ülve a hodályban? Hogy hallgasd nyájad bégetéseit?! Rúben patakjainál nagyok voltak az elhatározások!
\par 17 Gileád a Jordánon túl pihen. Hát Dán miért idõzik hajóinál? Áser a tenger partján ül és nyugszik öbleinél.
\par 18 De Zebulon, az halálra elszánt lelkû nép, És Nafthali, a mezõség magaslatain!
\par 19 Királyok jöttek, harczoltanak: Akkor harczoltak a Kanaán királyai Taanaknál, Megiddó vizénél; De egy darab ezüstöt sem vettenek.
\par 20 Az égbõl harczoltak, A csillagok az õ helyökbõl vívtak Siserával!
\par 21 A Kison patakja seprette el õket; Az õs patak, a Kison patakja! Végy erõt én lelkem!
\par 22 Akkor csattogtak a lovak körmei A futás miatt, lovagjaik futásai miatt.
\par 23 Átkozzátok Mérozt - mond az Úr követje, - Átkozva-átkozzátok annak lakosait!
\par 24 De áldott legyen az asszonyok felett Jáhel, A Keneus Héber felesége, A sátorban lakó nõk felett legyen áldott!
\par 25 Az vizet kért, õ tejet adott, Fejedelmi csészében nyújtott tejszínét.
\par 26 Balját a szegre, Jobbját pedig a munkások põrölyére nyújtotta, És ütötte Siserát, szétzúzta fejét, És összetörte, általfúrta halántékát,
\par 27 Lábainál leroskadt, elesett, feküdt, Lábai között leroskadt, elesett; A hol leroskadt, ott esett el megsemmisülve.
\par 28 Kinézett az ablakon, és jajgatott Siserának anyja a rostélyzat mögül: "Miért késik megjõni szekere? Hol késlekednek kocsijának gördülései?"
\par 29 Fejedelemasszonyinak legokosabbjai válaszolnak néki; Õ egyre csak azok szavait ismételgeti:
\par 30 Vajjon nem zsákmányra találtak-é, s mostan osztozkodnak? Egy-két leányt minden férfiúnak; A festett kelmék zsákmányát Siserának; Tarka szövetek zsákmányát, tarkán hímzett öltözeteket, Egy színes kendõt, két tarka ruhát nyakamra, mint zsákmányt.
\par 31 Így veszszenek el minden te ellenségeid, Uram! De a kik téged szeretnek, tündököljenek mint a kelõ nap az õ erejében! És megnyugovék a föld negyven esztendeig.

\chapter{6}

\par 1 És gonoszul cselekedének az Izráel fiai az Úrnak szemei elõtt, azért adá õket az Úr a Midiániták kezébe hét esztendeig.
\par 2 És hatalmat võn a Midiániták keze az Izráelen, és a Midiánitáktól való féltökben készítették magoknak az Izráel fiai azokat a barlangokat, rejtekhelyeket és erõsségeket, a melyek a hegységekben vannak.
\par 3 Mert ha vetett Izráel, mindjárt ott termettek a Midiániták, az Amálekiták és a Napkeletnek fiai, és rájok törtek.
\par 4 És táborba szálltak ellenök, és pusztították a földnek termését egész addig, a merre Gázába járnak, és nem hagytak élésre valót Izráelben, sem juhot, sem ökröt, sem szamarat.
\par 5 Mert barmaikkal és sátoraikkal vonultak föl; csapatosan jöttek, mint a sáskák, úgy hogy sem nékik magoknak, sem tevéiknek nem volt száma, és ellepték a földet, hogy elpusztítsák azt.
\par 6 Mikor azért igen megnyomorodott az Izráel a Midiániták miatt, az Úrhoz kiáltának az Izráel fiai.
\par 7 Mikor pedig kiáltottak vala az Izráel fiai az Úrhoz Midián miatt:
\par 8 Prófétát külde az Úr az Izráel fiaihoz, és monda nékik: Azt mondja az Úr, Izráel Istene: Én vezettelek fel titeket Égyiptomból, és hoztalak ki titeket a szolgálatnak házából,
\par 9 És én mentettelek meg benneteket az Égyiptombeliek kezébõl, és minden nyomorgatóitoknak kezökbõl, a kiket kiûztem elõletek, és néktek adtam az õ földjüket.
\par 10 És mondék néktek: Én, az Úr, vagyok a ti Istenetek; ne féljétek az Emoreusok isteneit, kiknek földén laktok; de ti nem hallgattatok az én szómra.
\par 11 És eljöve az Úrnak angyala, és leüle ama cserfa alatt, a mely Ofrában van, a mely az Abiézer nemzetségébõl való Joásé vala, és az õ fia Gedeon épen búzát csépelt a pajtában, hogy megmentse a Midiániták orczája elõl.
\par 12 Ekkor megjelenék néki az Úrnak angyala, és monda néki: az Úr veled, erõs férfiú!
\par 13 Gedeon pedig monda néki: Kérlek uram, ha velünk van az Úr, miért ért bennünket mindez? és hol vannak minden õ csoda dolgai, a melyekrõl beszéltek nékünk atyáink, mondván: Nem az Úr hozott-é fel minket Égyiptomból?! Most pedig elhagyott minket az Úr, és adott a Midiániták kezébe.
\par 14 És az Úr hozzá fordula, és monda: Menj el ezzel a te erõddel, és megszabadítod Izráelt Midián kezébõl. Nemde, én küldelek téged?
\par 15 És monda néki: Kérlek uram, miképen szabadítsam én meg Izráelt? Ímé az én nemzetségem a legszegényebb Manasséban, és én vagyok a legkisebb atyámnak házában.
\par 16 És monda néki az Úr: Én leszek veled, és megvered Midiánt, mint egy embert.
\par 17 Õ pedig monda néki: Ha kegyelmet találtam a te szemeid elõtt, kérlek, adj nékem valamely jelt, hogy te szólasz én velem.
\par 18 El ne menj kérlek innen, míg vissza nem jövök hozzád, és ki nem hozom az én áldozatomat, és le nem teszem elõdbe. Az pedig monda: Én itt leszek, míg visszatérsz.
\par 19 Gedeon pedig elméne, és elkészíte egy kecskegödölyét és egy efa lisztbõl kovásztalan pogácsákat, és betevé a húst egy kosárba, és a hús levét fazékba, és kivivé hozzá a cserfa alá, és felajánlá néki.
\par 20 És monda néki az Isten angyala: Vegyed a húst és a kovásztalan kenyereket, és rakd erre a kõsziklára, és a hús levét öntsd rá. És úgy cselekedék.
\par 21 Ekkor kinyújtá az Úrnak angyala pálczájának végét, mely kezében vala; és megérinté a húst, és a kovásztalan kenyeret: és tûz jött ki a kõsziklából, és megemészté a húst és a kovásztalan kenyereket. Az Úrnak angyala pedig eltünt az õ szemei elõl.
\par 22 Látván pedig Gedeon, hogy az Úrnak angyala volt az, monda Gedeon: Jaj nékem Uram, Istenem! mert az Úrnak angyalát láttam színrõl színre!
\par 23 És monda néki az Úr: Békesség néked! ne félj, nem halsz meg!
\par 24 És építe ott Gedeon oltárt az Úrnak, és nevezé azt Jehova-Salomnak (azaz: az Úr a béke), mely mind e mai napig megvan az Abiézer nemzetségének városában, Ofrában.
\par 25 És lõn ugyanazon éjjel, hogy monda az Úr néki: Végy egy tulkot atyádnak ökrei közül, és egy másik tulkot, a mely hét éves, és rontsd le a Baál oltárát, a mely a te atyádé és a berket, a mely a mellett van, vágd ki.
\par 26 És építs oltárt az Úrnak, a te Istenednek, ennek a megerõsített helynek tetején alkalmatos helyen, és vedd a második tulkot, és áldozd meg égõáldozatul a berek fájával, a melyet kivágsz.
\par 27 Ekkor Gedeon tíz férfiút võn maga mellé az õ szolgái közül, és a képen cselekedék, a mint megmondotta vala néki az Úr. De miután félt atyjának háznépétõl és a városnak férfiaitól ezt nappal cselekedni, éjszaka tevé meg.
\par 28 Mikor aztán felkeltek reggel a városnak férfiai, íme már össze volt törve a Baál oltára, és levágva a mellette levõ berek, és a második tulok égõáldozatul azon az oltáron, a mely építteték.
\par 29 És mondának egyik a másikának: Ki cselekedte ezt? És mikor utána kérdezõsködtek és tudakozódtak, azt mondották: Gedeon, a Joás fia cselekedte ezt a dolgot.
\par 30 Akkor mondának a városnak férfiai Joásnak: Add ki fiadat, meg kell halnia, mert lerontotta a Baál oltárát és mert kivágta a berket, a mely mellette volt.
\par 31 Joás pedig monda mindazoknak, a kik körülötte állának: Baálért pereltek ti? Avagy ti oltalmazzátok-é õtet? Valaki perel õ érette, ölettessék meg reggelig. Ha isten õ, hát pereljen õ maga, hogy oltára lerontatott!
\par 32 És azon a napon elnevezték õt Jerubbaálnak, mondván: Pereljen õ vele Baál, mert lerontotta az õ oltárát.
\par 33 És mikor az egész Midián és Amálek és a Napkeletiek egybegyûlének, és általkeltek a Jordánon, és tábort jártak a Jezréel völgyében:
\par 34 Az Úrnak lelke megszállotta Gedeont, és megfúván a harsonákat, egybehívá az Abiézer házát, hogy õt kövesse.
\par 35 És követeket külde egész Manasséba, és egybegyûle az is õ utána; és követeket külde Aserbe és Zebulonba és Nafthaliba, és feljövének eleikbe.
\par 36 És monda Gedeon az Istennek: Ha csakugyan az én kezem által akarod megszabadítani Izráelt, a miképen mondottad,
\par 37 Íme egy fürt gyapjat teszek a szérûre, és ha csak maga a gyapjú lesz harmatos, míg az egész föld száraz leénd: errõl megtudom, hogy valóban az én kezem által szabadítod meg Izráelt, a mint mondottad.
\par 38 És úgy lõn. Mert mikor másnap reggel felkelt, és megszorítá a gyapjat, harmatot facsart ki a gyapjúból, egy tele csésze vizet.
\par 39 És monda Gedeon az Istennek: Ne gerjedjen fel a te haragod én ellenem, hogy még egyszer szólok. Hadd tegyek kísérletet, kérlek, még egyszer e gyapjúval. Legyen, kérlek, szárazság csak a gyapjún, és harmat az egész földön.
\par 40 És úgy cselekedék Isten azon az éjszakán, és lõn szárazság csak magán a gyapjún, míg az egész földön harmat lõn.

\chapter{7}

\par 1 Felkele pedig jó reggel Jerubbaál - ez Gedeon - és az egész nép, mely vele volt, és táborba szállának a Haród kútjánál, és a Midián tábora tõle északra volt, a Moré halomtól fogva, a völgyben.
\par 2 És monda az Úr Gedeonnak: Több ez a nép, mely veled van, hogysem kezedbe adhatnám Midiánt; Izráel még dicsekednék velem szemben, mondván: Az én kezem szerzett szabadulást nékem!
\par 3 Azért kiálts a népnek füle hallatára, mondván: A ki fél és retteg, térjen vissza, és menjen el a Gileád hegységrõl. És visszatérének a nép közül huszonkétezeren, és csak tizezeren maradának ott.
\par 4 És monda az Úr Gedeonnak: Még ez a nép is sok; vezesd õket le a vízhez, és ott megpróbálom õket néked, és a melyikrõl azt mondon néked: Ez menjen el veled, az menjen el veled; de bármelyikrõl azt mondom: Ez ne menjen el veled, az ne is menjen.
\par 5 És levezette a népet a vízhez, és monda az Úr Gedeonnak: Mindazokat, a kik nyelvökkel nyalnak a vízbõl, mint a hogyan nyal az eb, állítsd külön, valamint azokat is, a kik térdeikre esnek, hogy igyanak.
\par 6 És lõn azoknak száma, a kik kezökkel szájokhoz véve nyaldosák a vizet, háromszáz férfiú; a nép többi része pedig mind térdre esve ivott.
\par 7 És monda az Úr Gedeonnak: E háromszáz férfiú által szabadítlak meg titeket, a kik nyaldosták vala a vizet, és adom Midiánt kezedbe; a többi nép pedig menjen el, kiki a maga helyére.
\par 8 És õk kezökbe vevék a népnek útravalóját és kürtjeit. Az Izráel többi férfiait pedig mind elküldötte, mindeniket a maga hajlékába, és csak a háromszáz férfiút tartotta meg. A Midián tábora pedig alatta feküdt a völgyben.
\par 9 És monda néki az Úr azon az éjszakán: Kelj fel, menj alá a táborba, mert kezedbe adtam õket.
\par 10 Ha pedig félsz lemenni, menj le te és Púra, a te szolgád a táborba.
\par 11 És hallgasd meg, mit beszélnek, hogy annakutána megerõsödjenek a te kezeid, és menj alá a táborba. És lement õ és Púra, az õ szolgája a fegyveresek szélsõ részéhez, a kik a táborban voltak.
\par 12 És a Midiániták és az Amálekiták és a Napkeletiek minden fiai úgy feküdtek a völgyben, mint a sáskák sokasága, és tevéiknek nem volt száma sokaságuk miatt, mint a fövenynek, mely a tenger partján van.
\par 13 Mikor pedig Gedeon oda ment, ímé az egyik férfiú épen álmát beszélte el a másiknak, és monda: Ímé álmot álmodtam, hogy egy sült árpakenyér hengergett alá a Midiániták táborára, és mikor a sátorig jutott, megütötte azt, úgy hogy eldõlt, és felfelé fordította azt, és a sátor ledõlt.
\par 14 A másik aztán felele és monda: Nem egyéb ez, mint Gedeonnak, a Joás fiának, az Izráelbõl való férfiúnak fegyvere, az õ kezébe adta az Isten Midiánt és egész táborát.
\par 15 És mikor hallotta Gedeon az álomnak elbeszélését és annak magyarázatát, meghajtá magát, és visszatére az Izráel táborába, és monda: Keljetek fel, mert kezetekbe adta az Úr a Midián táborát.
\par 16 És a háromszáz embert három csapatba osztá el, és mindeniknek kezébe egy-egy kürtöt adott, és üres korsókat és fáklyákat e korsókba.
\par 17 És monda nékik: Én reám vigyázzatok, és úgy cselekedjetek. És ímé én bemegyek a tábornak szélibe, és akkor, a mint én cselekszem, ti is úgy cselekedjetek.
\par 18 Ha én a kürtbe fúvok és mindazok, a kik velem vannak, akkor ti is fújjátok meg a kürtöket az egész tábor körül, és ezt kiáltsátok: az Úrért és Gedeonért!
\par 19 És leméne Gedeon, és az a száz férfiú, a ki vele volt, a tábor széléhez a középsõ éjjeli õrség kezdetén, a mikor épen az õrség felváltatott, és kürtölének a kürtökkel és összetörék a korsókat, a melyek kezökben valának.
\par 20 És kürtölt mind a három csapat a kürtökkel, és összetörték a korsókat, és balkezükben tartották a fáklyákat, jobb kezükben pedig a kürtöket, hogy kürtöljenek, és kiáltának: Fegyverre! Az Úrért és Gedeonért!
\par 21 És mindenik ott állott a maga helyén a tábor körül. Erre az egész tábor futásnak eredt, és kiáltozott, és menekült.
\par 22 És mikor a háromszáz ember belefújt kürtjébe, fordítá az Úr kinek-kinek fegyverét az õ felebarátja ellen az egész táborban, és egész Béth-Sittáig futott a tábor, Czererah felé, Abelmehola határáig, Tabbathon túl.
\par 23 És egybegyûjtettek az Izráel férfiai Nafthaliból, Áserbõl és az egész Manassébõl, és úgy ûzék a Midiánitákat.
\par 24 És követeket külde Gedeon az egész Efraim hegységbe, ezt izenvén: Jõjjetek alá a Midiániták ellen és foglaljátok el elõttök a vizeket Béthbaráig, és a Jordánt. És egybegyûle Efraimnak minden férfia, és elzárák a vizeket Béthbaráig, és a Jordánt is.
\par 25 És elfogák Midiánnak két fejedelmét, Orebet és Zéebet, és megölék Orebet az Oreb kõszikláján, és Zéebet megölék a Zéeb pajtájában, és ûzték a Midiánitákat. Orebnek és Zéebnek fejét pedig elvivék Gedeonnak a Jordánon túl.

\chapter{8}

\par 1 És mondának az Efraim férfiai néki: Miért cselekedted azt mi velünk, hogy el nem hívtál minket, mikor a Midián ellen való hadakozásra indultál? És erõsen feddõzének vele.
\par 2 Õ pedig monda nékik: Vajjon cselekedtem-é én olyan dolgot, mint ti? Nem többet ér-é Efraim szõlõmezgerlése, mint Abiézer egész szüretje?
\par 3 Kezetekbe adta az Isten Midiánnak fejedelmeit, Orebet és Zéebet; hát mit cselekedhettem én olyat, mint ti?! Akkor lecsendesedék az õ felháborodott lelkök, mikor e beszédet mondotta vala.
\par 4 Mikor pedig Gedeon a Jordánhoz ére, átkele azzal a háromszáz férfiúval, a kik vele valának, az üldözéstõl kifáradottan.
\par 5 És monda a Sukkót férfiainak: Adjatok, kérlek, e népnek, mely engem követ, kenyeret, mert fáradtak, én pedig ûzöm Zébát és Sálmunáht, Midiánnak királyait.
\par 6 És mondának Sukkót elõljárói: Zébának és Sálmunáhnak öklét már kezedben tartod-é, hogy kenyeret adjunk a te seregednek?
\par 7 Gedeon pedig monda: Ha kezembe adja az Úr Zébát és Sálmunáht, a pusztának tüskéivel és csalánjaival csépelem meg testeteket.
\par 8 És felmenvén onnan Pénuelbe hasonlóképen szóla azoknak is, és Pénuel férfiai épen úgy feleltek néki, a mint feleltek volt a Sukkót férfiai.
\par 9 És szóla a Pénuel férfiainak is, mondván: Mikor békességben visszatérek, lerontom ezt a tornyot.
\par 10 Zéba pedig és Sálmunáh Kárkorban valának, és az õ seregök velök, mintegy tizenötezeren, mind a kik megmaradtak a Napkeletieknek egész táborából. Az elesettek száma százhúszezer fegyverfogható férfiú volt.
\par 11 És felméne Gedeon a sátorban lakók útján keletre Nobahtól és Jogbehától, és szétverte a tábort, pedig a tábor biztonságban érezte magát.
\par 12 És Zéba és Sálmunáh elfutottak; de õ utánok ment, és elfogta a Midián két királyát: Zébát és Sálmunáht, és szétszórta az egész tábort.
\par 13 Mikor pedig Gedeon, a Joás fia, visszatért a harczból a Heresz hágójától:
\par 14 Megfogott egy gyermeket a sukkótbeliek közül, és tudakozódott tõle. Ez pedig felírta néki Sukkótnak elõljáróit és véneit, hetvenhét férfiút.
\par 15 Mikor aztán a Sukkót férfiaihoz ment vala, monda: Ímhol Zéba és Sálmunáh, a kik miatt gúnyolódtatok velem, mondván: Vajjon Zéba és Sálmunáh öklét már kezedben tartod-é, hogy kenyeret adjunk kifáradott embereidnek?
\par 16 És elõfogá a város véneit, és a pusztának töviseit és csalánjait vévén, megtanítá azokkal Sukkótnak férfiait.
\par 17 Azután Pénuel tornyát rontá le, és a város férfiait ölte meg.
\par 18 És monda Zébának és Sálmunáhnak: Milyenek voltak azok a férfiak, a kiket a Thábor hegyén megöltetek? Õk pedig mondának: Mint a milyen te vagy, olyanok voltak õk is, és mindenik olyan arczú, mint egy-egy királyfi.
\par 19 És monda: Az én atyámfiai, az én anyámnak fiai voltak azok. Él az Úr! ha életben hagytátok volna õket, én sem ölnélek meg titeket!
\par 20 És monda Jéthernek, az õ elsõszülöttjének: Kelj fel, öld meg õket. De a fiú nem húzta ki kardját, mert fél vala, mivelhogy még gyermek vala.
\par 21 És monda Zéba és Sálmunáh: Kelj fel te és te ölj meg minket, mert a milyen a férfiú, olyan az õ ereje. És felkele Gedeon, és megölé Zébát és Sálmunáht, és elvevé az õ tevéiknek nyakán levõ ékességeket.
\par 22 És mondának az Izráel férfiai Gedeonnak: Uralkodjál felettünk te és a te fiad és a te fiadnak fia, mert megszabadítottál bennünket a Midiániták kezébõl.
\par 23 És monda nékik Gedeon: Én nem uralkodom felettetek, sem az én fiam nem fog uralkodni rajtatok. Az Úr uralkodik ti felettetek!
\par 24 Azután monda nékik Gedeon: Egyet kívánok csak tõletek, hogy adja nékem mindenikõtök a zsákmányul ejtett fülön függõket; mert arany fülön függõik voltak azoknak, mivelhogy Izmaeliták valának.
\par 25 És mondának: Örömest néked adjuk. És leterítettek egy ruhát, és mindenki arra dobá a fülön függõket, melyeket prédául ejtett.
\par 26 Az arany fülön függõknek pedig, a melyeket elkért, súlya ezerhétszáz arany siklus volt, amaz ékességeken, függõkön és bíborruhákon kívül, a melyek a Midián királyain voltak, és az aranylánczok nélkül, a melyek tevéik nyakán valának.
\par 27 És készíte abból Gedeon efódot, és helyhezteté azt a maga városában, Ofrában, és ott paráználkodott azután az egész Izráel, és lõn ez Gedeonnak és házanépének tõrbeejtésére.
\par 28 Így aláztattak meg a Midiániták az Izráel fiai elõtt, és nem is emelték fel azután már fejüket, és megnyugovék a föld negyven esztendeig Gedeon idejében.
\par 29 És hazatért Jerubbaál, a Joás fia, és megtelepedett otthon.
\par 30 És Gedeonnak volt hetven fia, kik az õ ágyékából származtak, mert sok felesége volt néki.
\par 31 És az õ ágyasa is, a ki Sikemben volt, szült néki egy fiat, és nevezé azt Abiméleknek.
\par 32 És meghala Gedeon, a Joás fia, késõ vénségében, és eltemetteték atyjának, Joásnak sírjába, Ofrában, Abiezer városában.
\par 33 Lõn pedig, hogy mikor meghalt Gedeon, elfordulának az Izráel fiai az Úrtól, és a Baálokkal paráználkodának, és a Baál-Beritet tevék istenökké.
\par 34 És nem emlékezének meg az Izráel fiai az Úrról, az õ Istenökrõl, a ki õket megszabadította minden ellenségeik kezébõl köröskörül.
\par 35 És nem cselekedének irgalmasságot a Jerubbaál Gedeon házával mind ama jók szerint, a melyeket õ tett vala az Izráellel.

\chapter{9}

\par 1 És elméne Abimélek, a Jerubbaál fia Sikembe, az õ anyjának atyjafiaihoz, és beszélt velök, valamint az õ anyja atyjának egész nemzetségével, mondván:
\par 2 Mondjátok meg, kérlek, Sikem minden férfiának hallatára, melyik jobb néktek, hogy hetven férfiú uralkodjék-é rajtatok, Jerubbaálnak minden fia, vagy pedig csak egy ember uralkodjék felettetek? És emlékezzetek meg arról, hogy én a ti csontotok és a ti testetek vagyok.
\par 3 És elmondották anyjának testvérei róla mind e beszédeket Sikem minden férfiainak füle hallatára, és Abimélek felé hajlott az õ szívök, mert azt mondák: Atyánkfia õ!
\par 4 És adának néki hetven ezüst pénzt a Baál-Beritnek házából, és ezzel holmi henyélõ és hiábavaló embereket fogadott magának Abimélek, a kik õt követék.
\par 5 És elméne atyjának házához Ofrába, és megölé testvéreit, Jerubbaál fiait, a hetven férfiút egy kövön, és csak Jóthám maradt meg, Jerubbaálnak legkisebb fia, mert ez elrejtõzött.
\par 6 És összegyülekezék Sikemnek egész polgársága, és Millónak egész háza, és elmenének és királylyá választák Abiméleket az alatt a magas tölgy alatt, a mely Sikemben áll.
\par 7 Mikor pedig ezt elbeszélték vala Jóthámnak, elment és megállott a Garizim hegy tetején, és nagy felszóval kiálta, és így szóla hozzájuk: Hallgassatok rám, Sikem férfiai, hogy reátok is hallgasson az Isten!
\par 8 Egyszer elmenvén elmentek a fák, hogy királyt válaszszanak magoknak, és mondának az olajfának: Uralkodjál felettünk!
\par 9 De az olajfa így felelt nékik: Elhagyjam az én kövérségemet, a melylyel tisztelnek Istent és embereket, és elmenjek, hogy ingadozzam a fák felett?
\par 10 Akkor a fügefának szólottak a fák: Jer el te, és uralkodjál rajtunk.
\par 11 De a fügefa is azt mondta nékik: Elhagyjam-é édességemet és jó gyümölcseimet, és elmenjek, hogy ingadozzam a fák felett?
\par 12 Azután a szõlõtõnek mondák a fák: Jer el te, uralkodjál rajtunk.
\par 13 Azonban a szõlõtõ is azt mondta nékik: Elhagyjam-é mustomat, a mely isteneket és embereket vidámít, és elmenjek, hogy ingadozzam a fák felett?
\par 14 Mondának végre a fák mindnyájan a galagonyabokornak: Jer el te, uralkodjál mi rajtunk.
\par 15 És monda a galagonyabokor a fáknak: Ha igazán királylyá kentek engem magatok felett, jõjjetek el, nyugodjatok az én árnyékomban: de hogyha nem, jõjjön tûz ki a galagonyabokorból, és égesse meg a Libanonnak czédrusait.
\par 16 Hát ti is most igazán és becsületesen cselekedtetek-é, hogy Abiméleket tettétek királylyá, és jól cselekedtetek-é Jerubbaállal és házanépével, és úgy bántatok-é vele, a mint megérdemelte?
\par 17 Mert érettetek harczolt atyám, és még életével is semmit nem gondolván, mentett meg titeket a Midiánnak kezébõl.
\par 18 Ti pedig most felkeltetek az én atyámnak háza ellen, és megöltétek gyermekeit, hetven férfiút egy kövön, és királylyá választottátok Sikem férfiai felett Abiméleket, az õ szolgálójának fiát, mert atyátokfia!
\par 19 Ha igazán és becsületesen cselekedtetek Jerubbaállal és az õ házával a mai napon, örüljetek Abiméleknek, és örüljön õ is néktek;
\par 20 De hogyha nem, jõjjön tûz ki Abimélekbõl, és eméssze meg Sikem férfiait és Milló házát, és származzék tûz Sikem férfiaiból és Milló házából, és emészsze meg Abiméleket!
\par 21 És elfutott Jóthám, és elmenekült, és elment Beérbe az õ atyjafia, Abimélek elõl, és ott telepedett meg.
\par 22 Mikor pedig uralkodék Abimélek Izráel felett három esztendeig:
\par 23 Egy gonosz lelket bocsátott Isten Abimélek és Sikem férfiai közé, és pártot ütöttek Sikem férfiai Abimélek ellen,
\par 24 Hogy eljõjjön a Jerubbaál hetven fián elkövetett kegyetlenség büntetése, és szálljon az õ vérök Abimélekre, testvérökre, a ki megölte õket, és Sikem férfiaira, a kik az õ kezeit megerõsítették, hogy megölje az õ atyjafiait.
\par 25 És lest vetének néki a Sikem férfiai a hegyeknek tetején, és kiraboltak mindenkit, a ki elment mellettök az úton, mely dolgot megmondák Abiméleknek.
\par 26 És eljött Gaál, Ebed fia és az õ atyjafiai, és bementek Sikembe, és bízának õ hozzá Sikem férfiai.
\par 27 Annyira, hogy kimenvén a mezõre, leszüretelték szõlõiket, mindjárt ki is taposták, és örömünnepet ültek, és bementek az õ istenöknek házába, és ettek és ittak, és szidalmazták Abiméleket.
\par 28 És monda Gaál, Ebed fia: Kicsoda Abimélek és kicsoda Sekem, hogy szolgáljunk néki? Nem Jerubbaál fia-é õ, és nem Zebul-é az õ kormányzója? Ti szolgáljátok Hámornak, Sekem atyjának férfiait; de miért szolgálnánk mi?
\par 29 Csak volna az én kezemben e nép, majd elûzném Abiméleket. És monda Abiméleknek: Öregbítsd meg seregedet, és jõjj ki!
\par 30 Mikor pedig meghallotta Zebul, a városnak kormányzója, Gaálnak, az Ebed fiának beszédit, nagy haragra gyulladt.
\par 31 És követeket küldött Abimélekhez Thormába, ezt izenvén: Ímé Gaál, az Ebed fia és az õ testvérei Sikembe jöttek, és fellázítják a várost te ellened.
\par 32 Most azért készülj fel éjszaka, te és a te néped, mely veled van, és állj lesbe a mezõn.
\par 33 És reggel, napfelköltekor korán kelj fel, és törj a városra, és mikor õ és az õ népe kivonul ellened: cselekedjél vele a szerint, a mint akarod.
\par 34 És felkelt Abimélek és az egész nép, a mely vele volt, éjszaka, és lesbe állottak Sikem ellen négy csapatban.
\par 35 És kijött Gaál, az Ebed fia és megállott a város kapujának nyílásában. És felkelt Abimélek is, meg a nép is, mely vele volt, a lesbõl.
\par 36 És a mint meglátta Gaál a csapatot, monda Zebulnak: Ímé nép jõ alá a hegyeknek tetejérõl. Zebul pedig monda néki: A hegyek árnyékát nézed férfiaknak.
\par 37 De Gaál csak folytatta beszédét, és monda: Ímé egy másik csapat meg az ország közepébõl jõ alá; a harmadik csapat pedig a jós-tölgyfa útján jõ.
\par 38 Ekkor monda néki Zebul: Hol van most szád, melylyel mondád: Kicsoda Abimélek, hogy szolgáljunk néki? Nem ez a nép-é az, a melyet kisebbítettél? No, most vonulj ki ellene, és harczolj vele.
\par 39 És kivonult Gaál Sikem polgárainak élén, és megütközött Abimélekkel.
\par 40 De Abimélek megfutamította, úgy hogy elmenekült elõle, és sok sebesült elesett a kapu bejáratáig.
\par 41 Abimélek pedig Arumában maradt, és Zebul elûzte Gaált és atyjafiait, hogy ne lakjanak Sikemben.
\par 42 És lõn, hogy másnap kiméne a nép a mezõre, és megmondák Abiméleknek.
\par 43 És az vette az õ népét, és három csapatra osztotta el, és lesbe állott a mezõn; és látta, hogy ímé a nép jõ ki a városból. Rájok támadt, és megverte õket.
\par 44 És Abimélek és az a csapat, a mely vele volt, megtámadta és megszállotta a város kapuját; a másik két csapat pedig megtámadta mind a mezõn levõket, és megverte õket.
\par 45 És Abimélek egész nap vívta a várost, mígnem bevette a várost, és a népet, mely benne volt, leölte: a várost pedig lerombolta, és behinté azt sóval.
\par 46 Mikor pedig ezt meghallották Sikem tornyának minden férfiai, az El-Berith isten házának várába mentek.
\par 47 És mikor Abiméleknek elmondották, hogy Sikem tornyának minden férfiai ott gyûltek össze:
\par 48 Felment Abimélek a Sálmon hegyére, õ és az egész nép, mely vele volt, és fejszét vett kezébe, és faágakat vágott le, és azokat felszedte, és vállára rakta, és monda a népnek, a mely vele volt: A mit láttatok, hogy cselekedtem, ti is azt tegyétek gyorsan, mint én.
\par 49 Erre az egész népbõl kiki vágott magának ágakat, és követték Abiméleket, és lerakták a fát a vár körül, és tûzzel rájuk gyujtották a várat, úgy hogy meghaltak a Sikem tornyának minden férfiai, közel ezer férfi és asszony.
\par 50 Abimélek pedig elment Thébesbe, és táborba szállott Thébes ellen, és bevette azt.
\par 51 De egy erõs torony volt a város közepén, és oda menekült minden férfi és asszony, és a városnak minden lakosa; ezt magukra zárták, és a toronynak padlására mentek fel.
\par 52 És Abimélek oda ment a toronyig, és ostrom alá vette azt, és egészen a torony ajtajáig közeledett, hogy azt tûzzel égesse fel.
\par 53 Akkor egy asszony egy malomkõdarabot gördített le Abimélek fejére, és bezúzta koponyáját.
\par 54 Ki mindjárt oda hívta fegyverhordozó apródját, és monda néki: Vond ki kardodat, és ölj meg engem, hogy ne mondják felõlem: Asszony ölte meg õt! És keresztülszúrta õt az apród, és meghalt.
\par 55 Mikor pedig az Izráel férfiai látták, hogy Abimélek meghalt, kiki visszatért a maga helyére.
\par 56 Így fizetett meg Isten Abiméleknek azért a gonoszságért, melyet atyja ellen elkövetett, hogy megölte hetven testvérét.
\par 57 És a Sikem férfiainak fejére is visszahárított Isten minden rosszat, és reájuk szállott Jóthámnak, a Jerubbaál fiának átka.

\chapter{10}

\par 1 Támada pedig Abimélek után az Izráel megszabadítására Thóla, Puának, a Dodó fiának fia, Issakhár nemzetségébõl való, a ki Sámirban lakott, az Efraim hegységben.
\par 2 És bíráskodék Izráelben huszonhárom esztendeig, és meghala, és eltemetteték Sámirban.
\par 3 Õ utána következett Jáir, a gileádbeli, és ítélé Izráelt huszonkét esztendeig.
\par 4 Ennek harmincz fia volt, kik harmincz szamárcsikón nyargaltak, és volt nékik harmincz városuk, melyeket mind e mai napig Jáir faluinak neveznek, melyek Gileád földén vannak.
\par 5 És meghalt Jáir, és eltemetteték Kámonban.
\par 6 De az Izráel fiai újra gonoszul cselekedtek az Úrnak szemei elõtt, mert szolgáltak a Baáloknak és Astarótnak, és Sziria isteneinek, és Sidon isteneinek, és Moáb isteneinek, és az Ammon fiai isteneinek, és a Filiszteusok isteneinek, és elhagyták az Urat, és nem szolgáltak néki.
\par 7 Felgerjedett azért az Úrnak haragja Izráel ellen, és adá õket a Filiszteusoknak és az Ammon fiainak kezökbe.
\par 8 És ezek szorongatták és nyomorgatták az Izráel fiait attól az évtõl fogva tizennyolcz esztendõn keresztül, Izráelnek minden fiait a kik a Jordánon túl valának az Emoreusoknak földén, mely Gileádban van.
\par 9 És átkeltek az Ammon fiai a Jordánon, hogy hadakozzanak Júda és Benjámin ellen és az Efraim háza ellen, és felette igen szorongattaték az Izráel.
\par 10 Akkor az Úrhoz kiáltottak az Izráel fiai, mondván: Vétkeztünk te ellened, mert elhagytuk a mi Istenünket, és szolgáltunk a Baáloknak.
\par 11 Az Úr pedig monda az Izráel fiainak: Nemde én szabadítottalak-é meg benneteket az Égyiptombeliektõl, az Emoreusoktól, az Ammon fiaitól, a Filiszteusoktól,
\par 12 És a Sidonbeliektõl, az Amálekitáktól és a Maonitáktól, mikor titeket szorongattak, és ti én hozzám kiáltottatok, megszabadítottalak titeket az õ kezökbõl?
\par 13 És ti mégis elhagytatok engem, és idegen isteneknek szolgáltatok; annakokáért többé nem szabadítlak meg titeket ezután.
\par 14 Menjetek és kiáltsatok azokhoz az istenekhez, a kiket választottatok, szabadítsanak meg azok benneteket a ti nyomorúságtoknak idején.
\par 15 Felelének pedig az Izráel fiai az Úrnak: Vétkeztünk, cselekedjél úgy velünk, a mint jónak látszik a te szemeid elõtt, csak most az egyszer szabadíts még meg, kérünk!
\par 16 És elvetették maguktól az idegen isteneket, és szolgáltak az Úrnak. És megesett az õ szíve az Izráel nyomorúságán.
\par 17 És összegyülekeztek az Ammon fiai, és táborba szállottak Gileádban, és összegyûltek az Izráel fiai is, és Mispában táboroztak.
\par 18 És Gileád népe és fejedelmei egyik a másikától kérdezgették: Ki az a férfiú, a ki megkezdi a harczot az Ammon fiai ellen? Legyen az feje Gileád összes lakóinak!

\chapter{11}

\par 1 A Gileádból való Jefte pedig nagy hõs volt, és egy parázna asszonynak volt a fia, s Jeftét Gileád nemzette.
\par 2 De mikor Gileádnak a felesége is szült néki fiakat, és megnõttek az õ feleségének fiai: akkor elûzték Jeftét és azt mondták néki: Nem fogsz örökösödni atyánknak házában, mert más asszonynak vagy a fia.
\par 3 És elfutott Jefte az õ atyjafiai elõl, és Tób földén telepedett meg, és henyélõ emberek gyûltek össze Jefte körül, és együtt portyáztak.
\par 4 És történt napok mulva, hogy az Ammon fiai harczba keveredtek Izráellel.
\par 5 És lõn, hogy a mint harczolni kezdtek az Ammon fiai Izráellel, Gileád vénei elmentek, hogy visszahozzák Jeftét a Tób földérõl.
\par 6 És mondának Jeftének: Jer el, és légy nékünk fejedelmünk, és harczoljunk az Ammon fiai ellen.
\par 7 Jefte pedig monda Gileád véneinek: Avagy nem ti vagytok-é, a kik engem meggyûlöltetek, és kiûztetek atyámnak házából? És miért jöttetek most hozzám a ti nyomorúságtoknak idején?
\par 8 És mondának Gileád vénei Jeftének: Azért fordultunk most hozzád, hogy jõjj el velünk, és hadakozzál az Ammon fiai ellen, és légy mi nékünk fejünk, és Gileád minden lakosinak.
\par 9 És monda Jefte Gileád véneinek: Ha visszavisztek engem, hogy hadakozzam az Ammon fiai ellen, és kezembe adja õket az Úr: igazán fejetekké leszek?
\par 10 Akkor mondának Gileád vénei Jeftének: Az Úr a tanúnk, ha a te beszéded szerint nem cselekeszünk!
\par 11 És elment Jefte Gileád véneivel, és a nép a maga fejévé és fejedelmévé tette õt. És megmondá Jefte minden beszédit az Úr elõtt Mispában.
\par 12 És követeket küldött Jefte az Ammon fiainak királyához, ezt izenvén: Mi dolgom van nékem veled, hogy hozzám jöttél, hogy hadakozzál az én földem ellen?
\par 13 És monda az Ammon fiainak királya Jefte követeinek: Mert elvette Izráel az én földemet, mikor Égyiptomból feljött, az Arnontól fogva Jabbókig és a Jordánig, most te add azokat vissza békességgel.
\par 14 Ismét külde pedig Jefte követeket az Ammon fiainak királyához;
\par 15 És monda néki: Ezt izeni Jefte: Nem vette el Izráel sem a Moáb földét, sem az Ammon fiainak földét;
\par 16 Mert mikor kijött Égyiptomból, a pusztában bolyongott Izráel egész a Veres tengerig, és mikor Kádesbe ért,
\par 17 Követeket küldött Izráel Edom királyához, mondván: Hadd menjek át, kérlek, országodon; de Edom királya nem hallgatta meg. Majd Moáb királyához is küldött; de ez sem engedé, és így Izráel ott maradt Kádesben.
\par 18 És mikor tovább vándorolt a pusztában, megkerülte Edom földét és Moáb földét, és napkelet felõl érkezett a Moáb földéhez, és ott táborozott túl az Arnonon; de Moáb határába nem ment be, mert az Arnon Moáb határa.
\par 19 Ekkor követeket küldött Izráel Szihonhoz, az Emoreusok királyához, Hesbon királyához, és monda néki Izráel: Hadd menjek át, kérlek, országodon az én helyemre.
\par 20 De Szihon nem hitt Izráelnek, hogy átvonul az õ határán, hanem összegyûjtötte Szihon az õ egész népét, és táborba szállott Jahásban, és harczolt az Izráel ellen.
\par 21 Az Úr, az Izráel Istene pedig Szihont és egész népét Izráel kezébe adta, és megverték õket, és elfoglalta Izráel az Emoreusoknak, ama föld lakóinak, egész országát.
\par 22 És birtokába vette az Emoreusok egész határát, az Arnontól fogva Jabbókig, és a pusztától a Jordánig.
\par 23 És mikor az Úr, az Izráel Istene maga ûzte ki az Emoreusokat az õ népe, az Izráel elõl, most te akarnád ezt elfoglalni?
\par 24 Hát nem úgy van-é, hogy a mit bírnod adott néked Kámos, a te istened, azt bírod? Mi meg mindazoknak örökségét bírjuk, a kiket az Úr, a mi Istenünk ûzött ki mi elõlünk.
\par 25 , És most vajjon mennyivel vagy te különb Báláknál, Czippor fiánál, Moáb királyánál? Avagy versengett-é ez az Izráellel, és hadakozott-é valaha ellenük?
\par 26 Míg Izráel Hesbonban és annak mezõvárosaiban, Aroerben és ennek mezõvárosaiban és az Arnon mellett való összes városokban háromszáz esztendõn át lakott: miért nem foglaltátok el abban az idõben?
\par 27 Én nem vétettem te ellened, hanem te cselekszel velem gonoszt, hogy harczolsz ellenem. Az Úr, a biró, tegyen ma ítéletet az Izráel és az Ammon fiai között.
\par 28 De az Ammon fiainak királya nem hallgatott Jefte szavaira, a melyeket izent néki.
\par 29 Jeftén pedig lõn az Úrnak lelke, és általméne Gileádon és Manassén, és általméne Gileád Mispén, és Gileád Mispébõl felvonult az Ammon fiai ellen.
\par 30 És fogadást tõn Jefte az Úrnak, és monda: Ha mindenestõl kezembe adod az Ammon fiait:
\par 31 Akkor valami kijövénd az én házamnak ajtaján elõmbe, mikor békével visszatérek az Ammon fiaitól, legyen az Úré, és megáldozom azt egészen égõáldozatul.
\par 32 És kivonult Jefte az Ammon fiai ellen, hogy hadakozzék velök, és kezébe adá néki azokat az Úr.
\par 33 És veré õket Aroertõl fogva mindaddig, míg mennél Minnithbe, húsz városon át, és egész Abel-Keraminig nagy vérontással, és az Ammon fiai megaláztattak az Izráel fiai elõtt.
\par 34 Mikor pedig méne Jefte Mispába az õ házához: ímé az õ leánya jött ki eleibe dobokkal és tánczoló sereggel; ez volt az õ egyetlenegyje, mert nem volt néki kivülötte sem fia, sem leánya.
\par 35 És mikor meglátta õt, megszaggatá ruháit, és monda: Óh leányom, de megszomorítottál és megháborítottál engem! Mert én fogadást tettem az Úrnak, és nem vonhatom vissza.
\par 36 Az pedig monda néki: Atyám, ha fogadást tettél az Úrnak, úgy cselekedjél velem, a mint megfogadtad, miután az Úr megadta az ellenségeiden: az Ammon fiain való bosszút.
\par 37 És monda az õ atyjának: Csak azt az egyet tedd meg nékem; ereszsz el engem két hónapra, hadd menjek el és vonulhassak félre a hegyekre, hogy sírjak szûzességemen leánybarátaimmal.
\par 38 És õ monda: Menj el. És elbocsátá õt két hónapra. Az pedig elment és az õ leánybarátai, és siratta az õ szûzességét a hegyeken.
\par 39 És a két hónap elteltével visszatért atyjához, és betöltötte az õ felõle való fogadást, a melyet tett, és õ soha nem ismert férfiút. És szokássá lett Izráelben,
\par 40 Hogy esztendõnként elmentek az Izráel leányai, hogy dicsõítsék a gileádbeli Jefte leányát esztendõnként négy napon át.

\chapter{12}

\par 1 Összegyûlének pedig az Efraim férfiai, és általmenének északra, és mondának Jeftének: Miért szállottál harczba az Ammon fiaival, és miért nem hívtál minket is, hogy menjünk veled? Most azért a te házadat megégetjük te veled együtt tûzzel.
\par 2 És monda Jefte nékik: Nagy háborúságunk volt, nékem és az én népemnek, az Ammon fiaival: és hívtalak titeket; de ti nem szabadítottatok meg engem az õ kezökbõl.
\par 3 És a mikor láttam, hogy ti nem akartok segíteni, koczkára vetém saját életemet, és általmenék az Ammon fiai ellen, az Úr pedig kezembe adta õket. És most miért jöttetek fel hozzám, hogy hadakozzatok én ellenem?
\par 4 És ekkor egybegyûjté Jefte Gileádnak minden férfiait, és megtámadta Efraimot, és megverték Gileád férfiai Efraimot, mert azt mondották: Efraim szökevényei vagytok ti, kik Gileádban, Efraim és Manassé között laktok.
\par 5 És elfoglalák a Gileádbeliek Efraim elõtt a Jordán réveit, és lõn, hogy mikor az Efraim közül való menekülõk azt mondják vala: Hadd menjek által: azt kérdezték tõlük a gileádbeli férfiak Efraimbeli vagy-é? És ha azt mondotta: nem!
\par 6 Akkor azt mondák néki: Mondd: Sibboleth! És ha Szibbolethet mondott, mert nem tudta úgy kimondani, akkor megfogták õt és megölték a Jordán réveinél, és elesett ott abban az idõben az Efraimbeliek közül negyvenkétezer.
\par 7 Ítélé pedig Jefte Izráelt hat esztendeig, és meghalt Jefte, a Gileádbeli, és eltemetteték Gileád egyik városában.
\par 8 És ítélé õ utána az Izráelt Ibsán, ki Bethlehembõl való volt.
\par 9 Ennek harmincz fia volt és harmincz leányt házasított ki, és harmincz leányt hozott be kivül az õ fiainak, és itélé Izráelt hét esztendeig.
\par 10 És meghala Ibsán, és eltemetteték Bethlehemben.
\par 11 És bíráskodék õ utána Élon Izráelben, ki a Zebulon nemzetségébõl való volt, és ítélé Izráelt tíz esztendeig.
\par 12 És meghala Élon, a Zebulonbeli, és eltemetteték Ajalonban, Zebulon földén.
\par 13 Õ utána pedig Abdon bíráskodott Izráelben, a Pireathonita Hillel fia.
\par 14 Ennek negyven fia és harmincz unokája volt, kik hetven szamárcsikón nyargaltak, és ítélte Izráelt nyolcz esztendeig.
\par 15 És meghalt Abdon, a Pireathonita Hillel fia, és eltemetteték Pireathonban, Efraim földén, az Amálekiták hegységén.

\chapter{13}

\par 1 Az Izráel fiai pedig újra gonoszul cselekedtek az Úrnak szemei elõtt, azért az Úr õket a Filiszteusoknak kezébe adá negyven esztendeig.
\par 2 És élt ebben az idõben egy férfiú Czórából, a Dán nemzetségébõl való, névszerint Manoah, kinek felesége magtalan volt, és nem szült.
\par 3 És megjelent az Úrnak angyala az asszonynak, és monda néki: Ímé most magtalan vagy, és nem szültél; de terhes leszesz, és fiat szülsz.
\par 4 Azért most megójjad magad, és ne igyál se bort, se más részegítõ italt, és ne egyél semmi tisztátalant.
\par 5 Mert íme terhes leszesz, és fiat szülsz, és beretva ne érintse annak fejét, mert Istennek szenteltetett lesz az a gyermek anyjának méhétõl fogva, és õ kezdi majd megszabadítani Izráelt a Filiszteusok kezébõl.
\par 6 És elment az asszony, és elbeszélte ezt férjének, mondván: Istennek egy embere jöve hozzám, kinek tekintete olyan volt, mint az Isten angyalának tekintete, igen rettenetes, úgy hogy meg sem mertem kérdezni, hogy honnan való, és õ sem mondotta meg nékem a nevét.
\par 7 És monda nékem: Ímé terhes leszesz, és fiat fogsz szülni; azért most se bort, se más részegítõ italt ne igyál, és semmi tisztátalant ne egyél, mert Istennek szentelt lesz az a gyermek, anyja méhétõl fogva halála napjáig.
\par 8 Manoah pedig az Úrhoz könyörgött, és monda: Kérlek, Uram! az Istennek amaz embere, a kit küldöttél volt, hadd jõjjön el ismét hozzánk, és tanítson meg minket, hogy mit cselekedjünk a születendõ gyermekkel.
\par 9 És meghallgatá az Isten Manoah kérését, mert az Istennek angyala megint eljött az asszonyhoz, mikor az a mezõn ült, és az õ férje Manoah nem volt vele.
\par 10 Akkor az asszony elsietett, és elfutott, és elbeszélé férjének, és monda néki: Ímé megjelent nékem az a férfiú, a ki a multkor hozzám jött.
\par 11 És felkelt, és elment Manoah az õ felesége után, és mikor odaért ahhoz a férfiúhoz, monda néki: Te vagy-é az a férfiú, a ki ez asszonynyal beszéltél? És monda: Én vagyok.
\par 12 És monda Manoah: Ha beteljesedik ígéreted, miként bánjunk a gyermekkel, és mit cselekedjék õ?
\par 13 Az Úrnak angyala pedig monda Manoáhnak: Mindentõl, a mit csak mondottam az asszonynak, õrizkedjék.
\par 14 Mindabból, a mi csak a bortermõ szõlõbõl származik, ne egyék, és bort és más részegítõ italt ne igyék, és semmi tisztátalant ne egyék. Mindazt, a mit parancsoltam néki, tartsa meg.
\par 15 És monda Manoah az Úr angyalának: Kérlek, hadd tartóztassunk meg téged, hogy készítsünk néked egy kecskegödölyét.
\par 16 De az Úrnak angyala így szólt Manoáhhoz: Ha megmarasztasz is, nem eszem kenyeredbõl, és ha áldozatot készítesz, az Úrnak áldozd azt. Mert Manoah nem tudta, hogy az Úrnak angyala vala.
\par 17 És monda Manoah az Úr angyalának: Kicsoda a te neved, hogy ha majd beteljesedik a te beszéded, tisztességgel illethessünk téged.
\par 18 És monda néki az Úrnak angyala: Miért kérdezõsködöl nevem után, a mely olyan csodálatos?
\par 19 Mikor aztán Manoah a kecskegödölyét és ételáldozatot vette, és megáldozá azt egy sziklán az Úrnak: csodadolgot cselekedék Manoahnak és feleségének szeme láttára:
\par 20 Tudniillik, mikor a láng felcsapott az oltárról az ég felé, az oltár lángjában felszállott az Úrnak angyala. Mikor pedig ezt meglátták Manoah és az õ felesége, arczczal a földre borultak.
\par 21 És többé nem jelent meg az Úrnak angyala Manoáhnak és feleségének. Ekkor tudta meg Manoah, hogy az Úrnak angyala volt az.
\par 22 És monda Manoah az õ feleségének: Meghalván meghalunk, mert az Istent láttuk.
\par 23 Akkor monda néki az õ felesége: Ha meg akart volna ölni az Úr minket, nem fogadta volna el kezünkbõl az egészen égõáldozatot és az ételáldozatot, és nem mutatta volna nékünk mindezeket, sem pedig nem hallatott volna velünk ilyeneket.
\par 24 És szült az asszony fiat, és nevezé annak nevét Sámsonnak, és felnevekedék a gyermek, és megáldá õt az Úr.
\par 25 És kezdé az Úrnak lelke õt indítani a Dán táborában, Czóra és Estháol között.

\chapter{14}

\par 1 És lement Sámson Timnátba, és meglátott egy nõt Timnátban a Filiszteusok lányai közül.
\par 2 És mikor hazament, elbeszélte ezt atyjának és anyjának, és monda: Egy nõt láttam Timnátban a Filiszteusok leányai között, most azért vegyétek õt nékem feleségül.
\par 3 És monda néki az õ atyja és anyja: Hát nincsen a te atyádfiainak és az én egész népemnek leányai között nõ, hogy te elmégy, hogy feleséget végy a körülmetéletlen Filiszteusok közül? És monda Sámson az õ atyjának: Õt vegyed nékem, mert csak õ kedves az én szemeim elõtt.
\par 4 Az õ atyja és anyja pedig nem tudják vala, hogy ez az Úrtól van, hogy õ alkalmatosságot keres a Filiszteusok ellen, mert abban az idõben a Filiszteusok uralkodtak Izráel felett.
\par 5 És lement Sámson az õ atyjával és anyjával Timnátba, és mikor Timnátnak szõlõhegyéhez értek, íme egy oroszlánkölyök jött ordítva elébe.
\par 6 És felindítá õt az Úrnak lelke, és úgy kettészakasztá azt, mint a hogyan kettészakasztatik a gödölye; pedig semmi sem volt kezében. De atyjának és anyjának nem mondta el, a mit cselekedett.
\par 7 És mikor leérkezett, beszélt a nõvel, a ki kedves volt Sámson szemei elõtt.
\par 8 Mikor pedig egynéhány nappal azután visszatért, hogy õt hazavigye, lekerült, hogy megnézze az oroszlánnak holttestét: hát íme egy raj méh volt az oroszlánnak tetemében, és méz.
\par 9 És kiszedte azt markaiba, és a mint ment-mendegélt, eszegetett belõle, és mikor hazaért atyjához és anyjához, adott abból nékik is, és azok is ettek; de nem mondta meg nékik, hogy az oroszlán holttetemébõl vette ki a mézet.
\par 10 És azután lement az õ atyja ahhoz a nõhöz, és Sámson lakodalmat tartott ott, mert úgy szoktak cselekedni az ifjak.
\par 11 Mikor pedig meglátták õt a Filiszteusok, harmincz társat adtak mellé, hogy legyenek õ vele.
\par 12 És monda nékik Sámson: Hadd vessek elõtökbe egy találós mesét, ha azt megfejtitek nékem a lakodalom hét napja alatt és kitaláljátok; adok néktek harmincz inget és harmincz öltözõ ruhát;
\par 13 De ha nem tudjátok megfejteni, ti adtok nékem harmincz inget és harminc öltözõ ruhát. Azok pedig mondának néki: Add elõ találós mesédet, hadd halljuk.
\par 14 Õ pedig monda nékik: Az evõbõl étek jött ki S az erõsbõl édes jött ki. De nem tudták a találós mesét megfejteni három egész napon át.
\par 15 Lõn annakokáért heted napon, mondának Sámson feleségének: Vedd reá férjedet, hogy fejtse meg nékünk a találós mesét, hogy valamiképen meg ne égessünk téged és a te atyádnak házát tûzzel; vagy azért hívtatok ide minket, hogy koldussá tegyetek bennünket, vagy nem?
\par 16 És sírt a Sámson felesége õ elõtte, és monda: Bizony te gyûlölsz engem, és nem szeretsz. Egy találós mesét vetettél az én népem fiai elé és nékem sem fejtetted meg. Õ pedig monda néki: Íme még atyámnak és anyámnak sem mondtam meg, hát néked mondanám meg?
\par 17 Az pedig hét napon át sírdogált elõtte, a meddig a lakodalom tartott. Végre a hetedik napon megmondá néki, mert folyvást zaklatta õt. Õ pedig aztán megfejté a találós mesét népe fiainak.
\par 18 És mondának néki a város férfiai a hetedik napon, mielõtt még a nap lement volna:
\par 19 Ekkor felindítá õt az Úrnak lelke, és elment Askelonba, és megölt közülük harmincz férfiút, és elvette ruhájukat és azoknak adta ez öltözõ ruhákat, a kik a találós mesét megoldották. És felgerjedett haragjában elment az õ atyjának házához.
\par 20 A Sámson felesége pedig férjhez ment az õ egyik társához, a kit társaságába vett vala.

\chapter{15}

\par 1 Lõn pedig néhány nap múlva, a búzaaratásnak idejében, meglátogatta Sámson az õ feleségét, egy kecskegödölyét vivén néki, és monda: Bemegyek az én feleségemhez a hálóházba. De nem hagyá õt bemenni az õ atyja.
\par 2 És monda annak atyja: Azt gondoltam, hogy gyûlölve gyûlölöd õt, azért odaadtam õt a te társadnak; de hát vajjon húga nem szebb-é nálánál? Legyen õ helyette most az a tied.
\par 3 És monda néki Sámson: Teljesen igazam lesz most a Filiszteusokkal szemben, ha kárt okozok nékik.
\par 4 És elment Sámson, és összefogdosott háromszáz rókát, és csóvákat vévén, a rókák farkait egymáshoz kötözé, és egy-egy csóvát tett minden két rókának farka közé.
\par 5 És meggyújtá tûzzel a csóvákat, és beeresztette azokat a Filiszteusok gabonájába, és felgyújtá a gabonakalangyákat, az álló vetéseket, a szõlõket és az olajfaerdõket.
\par 6 Akkor mondának a Filiszteusok: Ki cselekedte ezt? És mondák: Sámson, Thimneus veje, mert elvette tõle az õ feleségét, és adta azt az õ társának. Felmenének annakokáért a Filiszteusok, és megégeték az asszonyt és annak atyját tûzzel.
\par 7 Sámson pedig monda nékik: Bátor ezt cselekedtétek, mégis addig nem nyugszom meg, míg bosszúmat ki nem töltöm rajtatok.
\par 8 És megverte õket keményen válluktól tomporukig, és lement és lakott Ethamban, a sziklabarlangban.
\par 9 A Filiszteusok pedig felmentek, és megszállották Júdát, és Lehiben telepedtek le.
\par 10 Akkor mondának a Júda férfiai: Miért jöttetek fel ellenünk? Azok pedig mondának: Sámsont megkötözni jöttünk fel, hogy úgy cselekedjünk vele, a mint õ cselekedett mi velünk.
\par 11 Ekkor háromezer ember ment le Júdából Ethamba, a sziklabarlanghoz, és monda Sámsonnak: Nem tudod-é, hogy a Filiszteusok uralkodnak mi rajtunk? Miért cselekedted ezt velünk? Õ pedig monda nékik: A miképen cselekedtek õk velem, én is úgy cselekedtem velök.
\par 12 És mondának néki: Azért jöttünk le, hogy megkötözzünk, és hogy a Filiszteusok kezébe adjunk téged. Sámson pedig monda nékik: Esküdjetek meg nékem, hogy ti nem rohantok reám.
\par 13 És azok felelének néki, mondván: Nem! csak megkötözvén megkötüzünk, és kezökbe adunk; de nem ölünk meg téged. Azután megkötözték õt két új kötéllel, és felvezették õt a kõszikláról.
\par 14 És mikor Lehi felé közeledett, és a Filiszteusok már ujjongtak elébe: felindítá õt az Úrnak lelke, és olyan lettek a karján lévõ kötelek, mint a lenszálak, melyeket megperzsel a tûz és lemállottak a kötések kezeirõl.
\par 15 És egy nyers szamárállcsontot talála, és kinyújtván kezét, felvevé azt, és agyonvert vele ezer embert.
\par 16 És monda Sámson: Szamár állcsontjával seregeket seregre, Szamár állcsontjával ezer férfit vertem le.
\par 17 És mikor ezt elmondotta, elvetette kezébõl az állcsontot, és elnevezé azt a helyet Ramath-Lehinek.
\par 18 Azután megszomjúhozék felette igen és felkiáltott az Úrhoz, és monda: Te adtad szolgád kezébe ezt a nagy gyõzelmet, és most szomjan kell meghalnom, és a körülmetéletlenek kezébe jutnom.
\par 19 Akkor meghasítá Isten a zápfogat, mely az állcsontban volt, és víz fakadt ki abból. Õ pedig ivott, ereje megtért, és megéledett. Azért neveztetik a "segítségül hívás forrásának" e hely Lehiben mind e mai napig,
\par 20 És ítélé Sámson Izráelt a Filiszteusok idejében húsz esztendeig.

\chapter{16}

\par 1 És elment Sámson Gázába, és meglátott ott egy parázna asszonyt, és bement hozzá.
\par 2 A gázabelieknek pedig mikor megmondották, mondván: Ide jött Sámson! körülvevék õt, és leselkedének õ utána egész éjszakán át a város kapujában, és hallgatóztak egész éjjel, és azt mondták: Reggel, ha világos lesz, megöljük õt.
\par 3 És aluvék Sámson éjfélig. Éjfélkor pedig felkelt, és megfogván a város kapujának szárnyait, a kapufélfákkal és a závárokkal együtt kiszakította azokat, és vállaira vette, és felvitte a hegy tetejére, mely Hebronnal szemben fekszik.
\par 4 És történt azután, hogy megszeretett egy asszonyt a Sórek völgyében, a kinek neve Delila volt.
\par 5 És felmenének õ hozzá a Filiszteusok fejedelmei, és mondának néki: Kérdezd ki õt és tudd meg, miben áll az õ nagy ereje, és miképen vehetünk rajta erõt, hogy megkötözzük és megkínozzuk õt, és mi mindenikünk adunk néked ezerszáz ezüst siklust.
\par 6 És monda Delila Sámsonnak: Mondd meg nékem, miben van a te nagy erõd és mivel kellene téged megkötni, hogy megkínozhassanak téged.
\par 7 És felele néki Sámson: Ha megkötöznek hét nyers gúzszsal, melyek még meg nem száradtak, akkor elgyengülök és olyan leszek, mint más ember.
\par 8 Akkor hoztak néki a Filisztesuok fejedelmei hét nyers gúzst, a mely még nem volt száraz, és megkötözé õt azzal.
\par 9 A lesben állók pedig ott vártak annál a hálókamarában. És monda néki: Rajtad a Filiszteusok, Sámson! És elszakasztá a gúzsokat, miképen elszakad a csepûfonal, ha tûz éri, - és ki nem tudódék, miben volt az õ ereje.
\par 10 És monda Delila Sámsonnak: Ímé rászedtél, és hazugságot szóltál nékem, most mondd meg igazán, hogy mivel lehet téged megkötözni?
\par 11 Õ pedig monda néki: Ha erõsen megkötöznek új kötelekkel, melyekkel még semmi dolgot nem végeztek, akkor elgyengülök és olyan leszek, mint más ember.
\par 12 És vett Delila új köteleket, és megkötözte õt velük, és monda néki: Rajtad a Filiszteusok, Sámson! (mert ott leselkedtek utána a hálókamarában). De õ letépte azokat karjairól, mint a fonalat.
\par 13 És monda Delila Sámsonnak: Meddig fogsz még rászedni engem és hazudni nékem? Mondd meg egyszer már, mivel kötöztethetel meg? - Õ pedig monda néki: Ha összeszövöd az én fejemnek hét fonatékját a nyüstfonállal.
\par 14 És szeggel megerõsítvén a zugolyt, monda: Rajtad a Filiszteusok, Sámson! Az pedig felébredvén álmából, kitépte a zugolyszeget és a nyüstfonalat.
\par 15 Ekkor monda néki Delila: Miképen mondhatod: Szeretlek téged, ha szíved nincsen én velem? Immár három ízben szedtél rá engem, és nem mondtad meg, hogy miben van a te nagy erõd?
\par 16 Mikor aztán õt minden nap zaklatta szavaival, és gyötörte õt: halálosan belefáradt a lelke,
\par 17 És kitárta elõtte egész szívét, és monda néki: Borotva nem volt soha az én fejemen, mert Istennek szentelt vagyok anyám méhétõl fogva; ha megnyírattatom, eltávozik tõlem az én erõm, és megerõtlenedem, és olyan leszek, mint akármely ember.
\par 18 És mikor látta Delila, hogy kitárta volna elõtte egész szívét, elküldött, és elhívatá a Filiszteusok fejedelmeit, és ezt izené: Jõjjetek fel ez egyszer, mert õ kitárta nékem egész szívét. És felmentek õ hozzá a Filiszteusok fejedelmei, és a pénzt is felvitték kezökben.
\par 19 És elaltatta õt az õ térdein, és elõhívott egy férfiút, és lenyíratta az õ fejének hét fonatékát, és kezdé õt kínozni. És eltávozott tõle az õ ereje.
\par 20 És monda: Rajtad a Filiszteusok, Sámson! Mikor pedig az az õ álmából felserkent, monda: Kimegyek most is, mint egyébkor, és lerázom a kötelékeket; mert még nem tudta, hogy az Úr eltávozott õ tõle.
\par 21 De a Filiszteusok megfogták õt, és kiszúrták szemeit, és levezették õt Gázába, és ott megkötözték két vaslánczczal, és õrölnie kellett  a fogházban.
\par 22 De az õ fejének haja újra kezdett nõni, miután megnyíretett.
\par 23 És mikor a Filiszteusok fejedelmei összegyûltek, hogy az õ istenüknek, Dágonnak nagy áldozatot áldozzanak, és hogy örvendezzenek, mondának: Kezünkbe adta a mi istenünk Sámsont, a mi ellenségünket.
\par 24 És mikor látta õt a nép, dícsérték az õ istenöket, mert - mondának - kezünkbe adta a mi istenünk a mi ellenségünket, földünk pusztítóját, és a ki sokakat megölt mi közülünk.
\par 25 Lõn pedig, hogy mikor megvídámult az õ szívök, mondának: Hívjátok Sámsont, hadd játszék elõttünk. És elõhívák Sámsont a fogházból, és játszék õ elõttük, és az oszlopok közé állították õt.
\par 26 Sámson pedig monda a fiúnak, a ki õt kézenfogva vezette: Ereszsz el, hadd fogjam meg az oszlopokat, a melyeken a ház nyugszik, és hadd támaszkodjam hozzájuk.
\par 27 A ház pedig tele volt férfiakkal és asszonyokkal, és ott voltak a Filiszteusok összes fejedelmei, és a tetõzeten közel háromezeren, férfiak és asszonyok, a Sámson játékának nézõi.
\par 28 Ekkor Sámson az Úrhoz kiáltott, és monda: Uram, Isten emlékezzél meg, kérlek, én rólam, és erõsíts meg engemet, csak még ez egyszer, óh Isten! hadd álljak egyszer bosszút a Filiszteusokon két szemem világáért!
\par 29 És átfogta Sámson a két középsõ oszlopot, melyeken a ház nyugodott, az egyiket jobb kezével, a másikat bal kezével, és hozzájok támaszkodott.
\par 30 És monda Sámson: Hadd veszszek el én is a Filiszteusokkal! És nagy erõvel megrándította az oszlopokat, és rászakadt a ház a fejedelmekre és az egész népre, mely abban volt, úgy hogy többet megölt halálával, mint a mennyit megölt életében.
\par 31 És lementek az õ testvérei és atyjának egész háza, és elvivék õt, és hazatérvén, eltemették Czóra és Estháol között, atyjának, Manoahnak sírjába, minekutána húsz esztendeig ítélte az Izráelt.

\chapter{17}

\par 1 Vala pedig egy férfiú Efraimnak hegyérõl való, kinek neve Míka vala;
\par 2 És monda az õ anyjának: Az az ezerszáz ezüst, mely tõled elvétetett, és a mely miatt te átkozódál, és füleimbe is mondtad, ímé az az ezüst én nálam van, én vettem el azt. És monda az õ anyja: Légy megáldva, fiam, az Úrtól!
\par 3 És visszaadta az ezerszáz ezüstpénzt az õ anyjának. És monda az õ anyja: Szentelve szentelem e pénzt az Úrnak az én kezeimbõl fiaimért, hogy egy faragott és öntött bálvány készíttessék abból, azért most visszaadom azt tenéked.
\par 4 De õ megint visszaadá a pénzt anyjának, és võn az õ anyja kétszáz ezüstpénzt, és odaadá azt az ötvösnek, és az készített abból egy faragott és öntött bálványt. Ez azután a Míka házában volt.
\par 5 És a férfiúnak, Míkának volt egy temploma, és készített efódot és terafimot, és felszentele  az õ fiai közül egyet, és ez lõn néki papja.
\par 6 Ebben az idõben nem volt király Izráelben, hanem kiki azt cselekedte, a mit jónak látott.
\par 7 Vala pedig egy ifjú, Júdának Bethlehemébõl, a Júda nemzetségéból való, ki Lévita vala, és ott tartózkodott vala.
\par 8 És elméne ez a férfiú Júdának Bethlehem városából, hogy ott tartózkodjék, a hol helyet talál. Így jött az Efraim hegyére, Míka házához, vándorlása közben.
\par 9 És monda néki Míka: Honnan jössz? És monda: Lévita vagyok Júdának Bethlehemébõl, és járok s kelek, hogy hol találnék helyet.
\par 10 És monda néki Míka: Maradj nálam, és légy nékem atyám és papom, és én adok néked esztendõnként tíz ezüstpénzt és egy öltözõ ruhát és eledelt. És a Lévita beszegõdött.
\par 11 És tetszék a Lévitának, hogy megmaradjon annál a férfiúnál; és olyan lõn néki az az ifjú, mint egyik az õ fiai közül.
\par 12 És felszentelte Míka a Lévitát; így lett papjává az ifjú, és maradt Míka házánál.
\par 13 És monda Míka: Most tudom, hogy jól fog velem tenni az Úr, mert e Lévita lett papom.

\chapter{18}

\par 1 Ebben az idõben nem volt király Izráelben, és ezekben a napokban keresett a Dán nemzetsége magának örökséget, a hol lakjék, mert nem jutott néki mind ez ideig az Izráel törzsei között osztályrész.
\par 2 És elküldöttek a Dán fiai az õ házoknépe közül öt férfiút, vitéz férfiakat a magok határaikból, Czórából és Estháolból, hogy kémleljék ki és nézzék meg jól a földet, és mondának nékik: Menjetek el, kémleljétek ki a földet. És elérkeztek az Efraim hegyéhez, a Míka házához, és ott megháltak.
\par 3 Mikor a Míka házánál voltak, megismerték az ifjú Lévitának hangját, és hozzá mentek, és mondának néki: Ki hozott téged ide? Mit csinálsz itt, és mi járatban vagy?
\par 4 Õ pedig monda nékik: Ezt meg ezt cselekedte velem Míka, és megfogadott engem, és papjává lettem.
\par 5 És mondának néki: Kérdezd meg Istentõl, hogy hadd tudjuk meg, ha szerencsés lesz-é a mi útunk, a melyen járunk?
\par 6 És monda nékik a pap: Menjetek el békességgel; a ti útatok, a melyen jártok, az Úr elõtt van.
\par 7 És elment az öt férfiú, és Laisba jutott, és látták a népet, a mely benne volt, hogy minden félelem nélkül lakik, a Sidonbeliek szokása szerint él csendesen és bátorságosan, és nincs senki, a ki õket bántaná az országban, vagy úr volna felettök, és távol vannak a sidoniaktól, és nincs senkivel semmi dolguk.
\par 8 És mikor visszatértek atyjokfiaihoz Czórába és Estháolba, és kérdezték tõlök az õ atyjokfiai: Mi jóval jártatok?
\par 9 Mondának: Keljetek fel, és menjünk fel ellenök, mert láttuk a földet, hogy ímé igen jó, és ti veszteg ültök? ne legyetek restek a menetelre, hogy elmenjetek elfoglalni azt a földet.
\par 10 Ha elmentek, biztonságban élõ néphez mentek, és a tartomány tágas; mert Isten kezetekbe adta azt a helyet, a hol semmiben sincs hiány, a mi csak a földön van.
\par 11 És elment onnét a Dán nemzetségébõl, Czórából és Estháolból, hatszáz férfiú hadiszerszámokkal felkészülten.
\par 12 És felvonultak, és táborba szállottak Kirjáth-Jeárimban, Júdában. Ezért hívják azt a helyet Dán táborának mind e mai napig. És ez Kirjáth-Jeárim mögött fekszik.
\par 13 És onnan felvonultak az Efraim hegyére, és Míka házába mentek.
\par 14 És szólott az az öt férfiú, a ki elment kikémlelni Lais földét és monda atyjafiainak: Tudjátok-é, hogy ebben a házban efód és teráf, faragott és öntött bálvány van? Hát elgondolhatjátok, hogy mit cselekedjetek.
\par 15 És betértek oda, és az ifjú Lévitához mentek a Míka házába, és köszöntötték õt: Békességgel!
\par 16 A hadiszerszámokkal felkészült hatszáz férfiú pedig, kik a Dán fiai közül valók voltak, a kapu elõtt állott.
\par 17 És az az öt férfiú, a ki a föld kikémlelésére ment volt el, mikor felment és megérkezett oda, elvette a faragott képet, az efódot, a teráfot és az öntött bálványt; a pap pedig ott állott a kapu elõtt a hadiszerszámokkal felkészült hatszáz férfiúval.
\par 18 Mikor pedig ezek a Míka házához bementek és elvették a faragott képet, az efódot, a teráfot és az öntött bálványt, monda nékik a pap: Mit míveltek?
\par 19 Azok pedig mondának néki: Hallgass, tedd kezed ajakadra, és jõjj el velünk, és légy nékünk atyánk és papunk. Melyik jobb, hogy egy ember házának légy papja, vagy hogy Izráelben egy nemzetségnek és háznépnek légy papja?
\par 20 És örvendett ezen a papnak szíve, és elvitte az efódot és a teráfot és a faragott képet, és velök a nép közé ment.
\par 21 És megfordulván elvonultak, magok elõtt küldve a gyermekeket, a barmokat és drágaságaikat.
\par 22 Mikor pedig már messze jártak a Míka házától, a férfiak, a kik a Míka házának szomszédságában laktak, összegyûltek, és utána mentek a Dán fiainak.
\par 23 És utánok kiáltoztak a Dán fiainak. Azok pedig visszafordulván mondának Míkának: Mi bajod van, hogy így felsereglettél?
\par 24 És monda: Isteneimet vettétek el, a melyeket készíttettem, és papomat, és elmentetek, hát mim van még egyebem? és mégis azt mondjátok nékem: mi bajom van?
\par 25 És mondának néki a Dán fiai: Ne hallasd többé hangodat, hogy rátok ne rontsanak e felbõszített emberek, és te a magad életét és házadnépe életét el ne veszítsd!
\par 26 És elmentek a Dán fiai a magok útján, és mikor Míka látta, hogy azok erõsebbek nála, megfordult, és visszatért házához.
\par 27 Õk pedig elvitték, a mit Míka készíttetett, és a papot, a ki nála volt, és Lais ellen mentek, a nyugodtan és biztonságban élõ nép ellen, és leölték õket fegyvernek élivel, és a várost megégették tûzzel.
\par 28 És nem volt senki, a ki õket megszabadította volna, mert messze volt Sidontól, és semmi dolguk nem volt senkivel. Lais pedig a Béth-Rehob völgyében feküdt és itt építették meg a várost, és telepedtek meg benne.
\par 29 És elnevezték a város nevét Dánnak, atyjoknak Dánnak nevérõl, a ki Izráelnek  született volt. Bár elõször Lais volt a város neve.
\par 30 És felállították magoknak a Dán fiai a faragott képet, és Jonathán, a Manasse fiának Gersomnak fia és az õ fiai voltak papok a Dán nemzetségében egészen a föld fogságának idejéig.
\par 31 És felállítva tarták a Míka faragott képét, a melyet az készíttetett, mindaddig, míg az Istenháza Silóban volt.

\chapter{19}

\par 1 Ugyanebben az idõben, a mikor nem volt király Izráelben, mint jövevény tartózkodott az Efraim hegység oldalán egy Lévita, a ki ágyas nõt szerzett magának Júda Bethlehemébõl.
\par 2 Paráználkodék pedig nála az õ ágyasa és elméne tõle atyjának házához, Júdának Bethlehemébe, és ott maradt négy hónapig.
\par 3 Ekkor felkelvén az õ férje, utána ment, hogy lelkére beszéljen, és hogy visszavigye õt. Szolgája és egy pár szamár volt vele. Az pedig bevezette õt az õ atyjának házába, és mikor meglátta õt a leánynak atyja, örvendve eléje ment.
\par 4 És ott tartóztatá õt ipa, a leánynak atyja, és õ ott maradt nála három napig, és ettek, ittak, és ott is háltak.
\par 5 És mikor a negyedik napon reggel korán felkeltek, és õ felkészült, hogy elmenjen, monda a leánynak atyja az õ vejének: Erõsítsd meg szívedet egy falat kenyérrel, azután menjetek el.
\par 6 És leültek, és mindketten együtt ettek és ittak, és monda a leány atyja a férfiúnak: Gondold meg és hálj itt az éjjel és gyönyörködjék a te szíved.
\par 7 Mikor pedig felkelt az a férfiú, hogy elmenjen, addig marasztá õt az ipa, hogy ott maradt megint éjszakára.
\par 8 És felkelt azután az ötödik napon jókor reggel, hogy elmenjen, és monda a leánynak atyja: Erõsítsd meg, kérlek, a te szívedet. És mulatozának, míg elhanyatlék a nap, és együtt evének mindketten.
\par 9 Ekkor felkele az a férfiú, hogy elmenjen ágyasával és szolgájával; de ipa, a leánynak atyja, így szólt hozzá: Ímé a nap már lehanyatlik, hogy beesteledjék, azért háljatok meg itt; ímé nyugalomra hajlik a nap, hálj itt, és gyönyörködjék a te szíved; holnap aztán készüljetek fel jókor reggel a ti útatokra, és menj el sátorodba.
\par 10 De a férfiú nem akart ott meghálni, és felkelt és elment, és egész Jebusig jutott, - ez Jeruzsálem. Egy pár megterhelt szamár és ágyasa volt vele.
\par 11 Mikor pedig Jebus mellett voltak, a nap már igen alászállott, és monda a szolga az õ urának: Jerünk és térjünk be a Jebuzeusok e városába, és háljunk ott.
\par 12 És monda néki az õ ura: Ne térjünk be az idegenek városába, a hol senki sincs az Izráel fiai közül, inkább menjünk el Gibeáig.
\par 13 És monda az õ szolgájának: Siess, és menjünk e két hely valamelyikébe, és háljunk meg Gibeában, vagy Rámában.
\par 14 És tovább vonultak, és elmenének, és a nap Gibea mellett ment le felettök, a mely Benjáminé.
\par 15 És oda tértek, hogy bemenjenek és megháljanak Gibeában. Mikor pedig oda bement, leült a város piaczán, mert nem volt senki, a ki õket házába behívná éjszakára.
\par 16 És ímé egy öreg ember jöve a munkából és mezõrõl késõ estve. Ez a férfiú az Efraim hegységérõl való volt, és csak jövevény Gibeában, míg a helynek lakói Benjáminiták voltak.
\par 17 És mikor felemelte szemeit, és meglátta azt az utas embert a város piaczán, monda az öreg ember néki: Hová mégy és honnan jösz?
\par 18 Ez pedig monda néki: Megyünk Júda Bethlehemébõl az Efraim hegység oldaláig, a honnan való vagyok. Júda Bethlehemében voltam és most az Úr házához megyek, és nincsen senki, a ki házába fogadna engem.
\par 19 Pedig szalmánk és abrakunk is van szamaraink számára, és kenyerem és borom is van a magam és a te szolgálód és emez ifjú számára, ki szolgáddal van, úgy hogy semmiben sem szûkölködünk.
\par 20 Ekkor monda a vén ember: Békesség néked! Mindarra, a mi nélkül csak szûkölködöl, nékem lesz gondom. Csak nem hálsz itt az utczán?!
\par 21 És elvezette õt az õ házához és abrakot adott az õ szamarainak. Azután megmosták lábaikat, és ettek és ittak.
\par 22 És mikor vígan laknának, ímé a város férfiai, a Béliál fiainak emberei, körülvették a házat, és az ajtót döngetve, mondának az öreg embernek, a ház urának, mondván: Hozd ki azt a férfiút, a ki házadhoz jött, hogy ismerjük meg õtet.
\par 23 És kiment hozzájuk az a férfiú, a háznak ura és monda nékik: Ne, atyámfiai, ne cselekedjetek ilyen gonoszt, minekutána az a férfiú az én házamhoz jött, ne tegyétek vele azt az alávaló dolgot.
\par 24 Ímé itt van hajadon leányom és az õ ágyasa, ezeket hozom ki néktek, és ezeket nyomorgassátok, és tegyétek velök azt, a mit csak tetszik, csak e férfiúval ne cselekedjétek azt az alávaló dolgot.
\par 25 A férfiak azonban nem akartak reá se hallgatni. Ekkor kézen fogta az a férfiú az õ ágyasát, és kivitte nékik az utczára. Ezek pedig megszeplõsíték õt, és gonoszul élének vele egész éjszaka reggelig, és csak mikor feltetszett a hajnal, akkor bocsátották el.
\par 26 És elment az asszony virradat elõtt és reggel ott rogyott össze annak a férfiúnak háza ajtajánál, a melyben az õ ura volt reggelig.
\par 27 Mikor pedig felkelt az õ ura reggel, és kinyitotta a ház ajtaját, és kiment, hogy útnak induljon, ímé az asszony, az õ ágyasa, ott feküdt elterülve a ház ajtaja elõtt, és kezei a küszöbön.
\par 28 És monda néki: Kelj fel és menjünk el. De az nem felelt néki. Ekkor feltette õt a szamárra, és felkelt a férfiú, és elment hazájába.
\par 29 És mikor hazaért, kést vett elõ, és megfogta ágyasát, és tagról-tagra szétvagdalta õt tizenkét darabba, és szétküldözte Izráel minden határába.
\par 30 Lõn pedig, hogy mindenki, a ki ezt látta, azt mondotta: Nem történt és nem láttatott ehhez hasonló dolog, mióta csak feljöttek az Izráel fiai Égyiptomnak földébõl mind e mai napig. Gondolkodjatok e dolog felõl, tartsatok tanácsot és beszéljétek meg.

\chapter{20}

\par 1 Erre kivonultak az Izráel minden fiai és összegyülekezett a nép, mint egy ember, Dántól fogva Bersebáig és a Gileád földéig, az Úrhoz Mispába.
\par 2 És megjelentek az egész népnek fõfõ emberei, az Izráelnek minden nemzetségei az Isten népének gyülekezetében, négyszázezer gyalogos, fegyverfogható férfiú.
\par 3 De meghallották a Benjámin fiai is, hogy felmentek az Izráel fiai Mispába. Az Izráel fiai pedig mondának: Mondjátok meg, hogy mint történt ez a gonoszság?
\par 4 És felele a Lévita, a megöletett asszonynak férje, és monda: Gibeába, mely Benjáminé, mentem én és az én ágyasom, hogy ott megháljak.
\par 5 És ellenem támadtak Gibeának férfiai, és körülvették miattam a házat éjjel, engem akartak megölni, de az én ágyasomat nyomrogatták meg annyira, hogy meghalt.
\par 6 Ekkor fogtam ágyasomat, és szétvagdaltam õt, és szétküldöztem az Izráel örökségének minden tartományaiba, mert útálatosságot és aljasságot követtek el Izráelben.
\par 7 Ímé mindnyájan, kik itt vagytok Izráel fiai, szóljatok errõl és tanácskozzatok felõle.
\par 8 Ekkor felállott az egész nép, mint egy ember, mondván: Senki közülünk sátorába ne menjen, és senki házához ne térjen,
\par 9 Mert most Gibea ellen ezt cselekedéndjük: sorsot vetünk rá.
\par 10 És választunk tíz férfiút száz közül, és százat ezer közül, és ezeret tízezer közül, Izráelnek minden nemzetségébõl, hogy hordjanak élelmet a népnek, hogy ez elmenvén, cselekedjék Benjámin Gibeájával annak minden gonoszsága szerint, melyet elkövetett Izráelben.
\par 11 És összegyülekezett Izráel minden férfia a város ellen, mint egy ember, egyesülten.
\par 12 És követeket küldöttek az Izráel nemzetségei Benjámin minden törzseihez, mondván: Micsoda aljasság az, a mi ti közöttetek történt?
\par 13 Most ti adjátok ki azokat a férfiakat, a Béliál fiait, a kik Gibeában vannak, hogy megöljük õket, és kitisztítsuk a gonoszt Izráelbõl. De a Benjámin fiai nem akartak hallgatni testvéreiknek, az Izráel fiainak szavára,
\par 14 Hanem egybegyûltek a Benjámin fiai a városokból Gibeába, hogy kimenjenek harczolni az Izráel fiaival.
\par 15 És azon a napon a Benjámin fiai, a kik a városokból jöttek fel, huszonhatezer fegyverfogható férfiút számlálának, Gibea lakóin kivül, kik szám szerint hétszázan voltak, mind válogatott férfiú.
\par 16 Ebbõl az egész népbõl volt hétszáz válogatott férfiú, a kik suták voltak. Ezek mindnyájan a parittyával hajszálnyira biztosan találtak és nem hibázták el.
\par 17 Az Izráel fiai pedig szám szerint, a Benjámin fiain kivül, négyszázezeren voltak, fegyverfogható emberek, és mind hadakozó férfiak.
\par 18 Ekkor felkeltek, és felmentek Béthelbe, és megkérdék az Istent, és mondának az Izráel fiai: Ki menjen fel elõször közülünk a Benjámin fiai ellen hadakozni? És monda az Úr: Júda elõször.
\par 19 Felkeltek azért az Izráel fiai reggel, és táborba szállottak Gibea elõtt.
\par 20 És kimentek Izráel emberei harczolni Benjámin ellen, és csatarendbe állottak fel ellenök az Izráel emberei Gibeánál.
\par 21 És kivonultak a Benjámin fiai is Gibeából, és levertek az Izráel fiai közül az nap huszonkétezeret a földre.
\par 22 De a nép, Izráel férfiai, megbátoríták magukat, és újra csatarendbe állottak ugyanazon a helyen, a melyen elõtte való nap sorakoztak.
\par 23 És felmenének az Izráel fiai, és ott sírtak, az Úr elõtt egész estig, és megkérdezték az Urat, mondván: Vajjon elmenjek-é még harczolni az én atyámfiának, Benjáminnak fiai ellen? Az Úr pedig monda: Menjetek fel ellene!
\par 24 És mikor az Izráel fiai másnap a Benjámin fiai ellen felvonultak,
\par 25 Kijött elébük Benjámin Gibeából másnap, és levert az Izráel fiai közül még tizennyolczezer embert a földre, kik mindannyian fegyverfoghatók valának.
\par 26 Ekkor felment Izráel minden fia és az egész nép, és elmenvén Béthelbe, sírtak, és ott maradtak az Úr elõtt és bõjtöltek aznap egész estvéig, és égõáldozattal és hálaadó áldozattal áldoztak az Úr elõtt.
\par 27 És megkérdezék az Izráel fiai az Urat, - mert ott volt abban az idõben az Isten frigyládája.
\par 28 És Fineás, az Áron fiának Eleázárnak fia szolgált körülötte abban az idõben - mondván: Vajjon még egyszer felmenjek-é harczolni az én atyámfiának Benjáminnak fiaival, vagy pedig abbanhagyjam? És monda az Úr: Menj, mert holnap kezedbe adom õket.
\par 29 És leseket vetett Izráel Gibea ellen köröskörül.
\par 30 És felvonultak az Izráel fiai a Benjámin fiai ellen harmadnapon, és felállottak Gibea ellen úgy, mint annakelõtte.
\par 31 Ekkor kijöttek a Benjámin fiai a nép ellen, elszakasztatának a várostól, és elkezdették a népet verni, és ölni, mint annakelõtte, a mezõn, a két úton, melynek egyike Béthelbe, másika Gibea felé vezet, és már megöltek mintegy harmincz férfiút Izráelbõl.
\par 32 És mondának a Benjámin fiai: Megverettetnek ezek elõttünk megint, mint elõször. Az Izráel fiai pedig mondának: Fussunk el és szakasszuk el õket a várostól, ki az országútra.
\par 33 És az Izráel minden fia elhagyta helyét és Baál-Thámárnál állott fel. Izráel lesei pedig elõtörtek rejtekhelyeikbõl Maareh-Gabából.
\par 34 Ekkor az egész Izráelbõl tízezer válogatott férfiú tört Gibea ellen, és néki búsulának a harcznak, és amazok észre sem vették, hogy veszedelemben forognak.
\par 35 Így verte le az Úr Izráel elõtt Benjámint, és elpusztítottak az Izráel fiai azon a napon a Benjámin fiai közül huszonötezerszáz férfiút, kik mind fegyverfoghatók voltak.
\par 36 Benjámin fiai tehát látták, hogy megveretnek, mivel Izráel férfiai csak azért adtak helyet Benjáminnak, mert õk a lesekben bíztak, a melyeket Gibeánál helyeztek el.
\par 37 És a les elõsietett és elõtört Gibea ellen, és a les bevonult, és leölte az egész várost fegyvernek élivel.
\par 38 És abban egyeztek meg Izráel férfiai a les-csapatokkal, hogy erõs füstfelleget bocsátanak fel a városból.
\par 39 Mikor aztán az Izráel férfiai visszafordultak a harcz közben, és Benjámin megkezdte az öldöklést és leölt mintegy harmincz férfiút Izráelbõl, és azt gondolta magában: Bizony megverettetnek elõttünk, mint az elsõ ütközetben:
\par 40 Épen akkor kezdett a felhõ felemelkedni a városból, mint egy füstoszlop. És mikor aztán Benjámin hátratekintett, látta, hogy íme a város lángja már feléri az eget.
\par 41 És az Izráel fiai megfordultak, és megijedének a Benjámin fiai, a kik most látták csak, hogy rajtok a veszedelem.
\par 42 És elfutottak az Izráel férfiai elõl a pusztába vivõ útra; de a harcz ott is utólérte õket, és az út közepén ölték le a városból jövõket.
\par 43 Körülvették Benjámint, üldözték õt, letiporták a pihenõ helyen, egészen a Gibea elõtt keletre esõ vidékig.
\par 44 És elesett Benjámin közül tizennyolcezer ember, mindnyájan vitéz férfiak.
\par 45 Ekkor megfordultak, és a pusztába menekültek, a Rimmon sziklájához; de az útakon még megöltek közülök ötezer embert, és azután egész Gideomig mentek utánok, és megöltek közülök kétezer embert.
\par 46 Azok tehát, akik elestek Benjámin közül, összesen huszonötezeren voltak, kik mindannyian fegyvert fogtak azon a napon, és mindnyájan vitéz férfiak voltak.
\par 47 De hatszáz férfiú megfordult, és elmenekült a pusztába a Rimmon kõsziklájára, és ott is maradt a Rimmon szikláján négy hónapig.
\par 48 Az Izráel férfiai pedig visszatértek a Benjámin fiaira, és megölték õket fegyvernek élével a városokban az emberektõl a barmokig, és a mi csak található volt; az összes városokat pedig, miket találtak, tûzzel égették meg.

\chapter{21}

\par 1 És az Izráel fiai megesküdtek Mispában, mondván: Senki mi közülünk nem adja leányát Benjáminnak feleségül.
\par 2 És a nép elment Béthelbe, és ott volt egész estvéig az Isten elõtt, és felemelve szavát, nagy sírással sírt.
\par 3 És mondának: Oh Uram! Izráelnek Istene! Miért történt ez Izráelben, hogy ma Izráelbõl egy nemzetség hiányzik?
\par 4 És lõn másnap reggel, felkele a nép és ott oltárt épített, és egészen égõáldozatot és hálaadó áldozatot áldozott.
\par 5 És mondának az Izráel fiai: Kicsoda az, a ki nem jött fel a gyülekezetbe az Izráelnek minden nemzetségei közül az Úrhoz? Mert nagy esküt tettek volt egyszer a felõl, a ki fel nem jött az Úrhoz Mispába, mondván: Meghalván meghaljon az!
\par 6 És megszánák az Izráel fiai Benjámint, az õ atyjokfiát, és mondának: Kivágattatott ma egy nemzetség Izráelbõl.
\par 7 Mit cselekedjünk azokkal, a kik megmaradtak, hogy feleséget kaphassanak? Mert mi megesküdtünk az Úrra, hogy nem adjuk nékik feleségül a mi leányainkat.
\par 8 És mondának: Van-é valaki az Izráel nemzetségei közül, a ki nem jött fel az Úrhoz Mispába? És íme Jábes-Gileádból nem jött fel senki a gyülekezet táborába.
\par 9 Mert megszámlálá magát a nép, és ímé nem vala ott senki a Jábes-Gileád lakosai közül.
\par 10 Ekkor elküldött a gyülekezet oda tizenkétezer harczban edzett férfiút, és megparancsolták nékik, mondván: Menjetek el, és vágjátok le Jábes-Gileád lakóit fegyvernek élivel, még az asszonyokat és a gyermekeket is.
\par 11 És ezt kell cselekednetek, hogy minden férfiút és minden asszonyt a ki már férfiúval hált,  öljetek meg.
\par 12 Találának pedig Jábes-Gileád lakói között négyszáz szûz leányt, a kik még nem ismertek férfiút férfival való hálással, és elvitték õket a táborba Silóba, mely vala a Kanaán földén.
\par 13 És elküldött az egész gyülekezet, és izent a Benjámin fiainak, kik a Rimmon szikláján voltak, és békességet ígérének nékik.
\par 14 Így tért vissza abban az idõben Benjámin, és nékik adták azokat feleségül, a kiket meghagytak Jábes-Gileád asszony népei közül; de így sem lõn elég nékik.
\par 15 És bánkódék a nép Benjámin miatt, hogy az Úr ilyen rést csinált az Izráel törzsein.
\par 16 Mondának azért a gyülekezet vénei: Mit tegyünk, hogy a többiek is feleséget kapjanak, mert Benjáminból kipusztíttattak az asszonyok?
\par 17 És mondának: A Benjáminiták birtoka örökség szerint ezeké, a kik megmaradtak, mert nem szabad Izráelben egy nemzetségnek sem eltöröltetni.
\par 18 De mi nem adhatunk nékik feleségeket leányaink közül, mert megesküdtek az Izráel fiai, mondván: Átkozott, a ki feleséget ad Benjáminnak!
\par 19 És mondának: Ímé az Úrnak esztendõnként való ünnepnapja lesz most Silóban, a mely északra fekszik Bétheltõl, napkelet felé az országút mellett, a mely Bétheltõl Sikem felé visz, és délre Lebonától.
\par 20 Ezt parancsolák azért a Benjámin fiainak: Menjetek el, és leselkedjetek a szõlõkben.
\par 21 És ha látjátok, hogy jõnek Siló leányai a körtánczban tánczolni, akkor jõjjetek elõ a szõlõkbõl, és vegyetek magatoknak kiki feleséget a Siló leányai közül, és menjetek el a Benjámin földére.
\par 22 Ha aztán atyjaik vagy testvéreik hozzánk jõnek perlõdni, akkor azt mondjuk nékik: Könyörüljetek rajtuk mi érettünk, mert nem hozhattunk mindeniknek felelséget ama hadból, és minthogy nem ti adtátok nékik, most nem vagytok vétkesek.
\par 23 És ekképen cselekedtek a Benjámin fiai, és võnek feleségeket az õ számuk szerint a tánczolók közül, a kiket elraboltak. És elmentek, és hazatértek örökségükbe, és felépítették a városokat, és azokban laktak.
\par 24 Az Izráel fiai pedig eltávoztak onnét abban az idõben, mindegyik a maga nemzetségéhez és háznépéhez, és elmentek onnét kiki a maga örökségébe.
\par 25 Ebben az idõben nem volt király Izráelben; azért mindenki azt cselekedte, a mi jónak látszott az õ szemei elõtt.


\end{document}