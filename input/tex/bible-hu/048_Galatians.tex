\begin{document}

\title{Galatians}


\chapter{1}

\par 1 Pál, apostol (nem emberektõl, sem nem ember által, hanem Jézus Krisztus által és az Atya Isten által, a ki feltámasztotta õt a halálból);
\par 2 És a velem levõ összes atyafiak, Galátzia gyülekezeteinek:
\par 3 Kegyelem néktek és békesség az Atya Istentõl, és a mi Urunk Jézus Krisztustól,
\par 4 A ki adta önmagát a mi bûneinkért hogy kiszabadítson minket  e jelenvaló gonosz világból, az Istennek és a mi Atyánknak akarata szerint.
\par 5 A kinek dicsõség örökkön örökké! Ámen.
\par 6 Csodálkozom, hogy Attól, a ki titeket Krisztus kegyelme által elhívott, ily hamar más evangyéliomra hajlotok.
\par 7 Holott nincs más; de némelyek zavarnak titeket, és el akarják ferdíteni a Krisztus evangyéliomát.
\par 8 De ha szinte mi, avagy mennybõl való angyal hirdetne is néktek valamit azon kívül, a mit néktek hirdettünk, legyen átok.
\par 9 A mint elõbb mondottuk, most is ismét mondom: Ha valaki néktek hirdet valamit azon kívül, a mit elfogadtatok, átok legyen.
\par 10 Mert most embereknek engedek-é, avagy az Istennek? Vagy embereknek igyekezem-é tetszeni? Bizonyára, ha még embereknek igyekezném tetszeni, Krisztus szolgája nem volnék.
\par 11 Tudtotokra adom pedig atyámfiai, hogy az az evangyéliom, melyet én hirdettem, nem ember szerint való;
\par 12 Mert én sem embertõl vettem azt, sem nem tanítottak arra, hanem a Jézus Krisztus kijelentése által.
\par 13 Mert hallottátok, mint forgolódtam én egykor a zsidóságban, hogy én felette igen háborgattam az Isten anyaszentegyházát, és pusztítottam azt.
\par 14 És felülmultam a zsidóságban nemzetembeli sok kortársamat, szerfelett rajongván atyai hagyományaimért.
\par 15 De mikor az Istennek tetszett, ki elválasztott engem az én anyám méhétõl fogva és elhívott az õ kegyelme által,
\par 16 Hogy kijelentse az õ Fiát én bennem, hogy hirdessem õt a pogányok között: azonnal nem tanácskoztam testtel és vérrel,
\par 17 Sem nem mentem Jeruzsálembe az elõttem való apostolokhoz, hanem elmentem Arábiába, és ismét visszatértem Damaskusba.
\par 18 Azután három esztendõ mulva fölmentem Jeruzsálembe, hogy Pétert meglátogassam, és nála maradtam tizenöt napig.
\par 19 Az apostolok közül pedig mást nem láttam, hanem csak Jakabot, az Úr atyjafiát.
\par 20 A miket pedig néktek írok, ímé Isten elõtt mondom, hogy nem hazudom.
\par 21 Azután mentem Siriának és Ciliciának tartományaiba.
\par 22 Ismeretlen valék pedig szeméyesen a Júdeában levõ keresztyén gyülekezetek elõtt;
\par 23 Hanem csak hallották, hogy: A ki minket üldözött egykor, most hirdeti azt a hitet, a melyet egykor pusztított.
\par 24 És dicsõíték bennem az Istent.

\chapter{2}

\par 1 Azután tizennégy esztendõ mulva ismét fölmentem Jeruzsálembe Barnabással együtt, elvivén Titust is.
\par 2 Fölmentem pedig kijelentés következtében és eléjök adtam az evangyéliomot, melyet hirdetek a pogányok között, de külön a tekintélyeseknek, hogy valami módon hiába ne fussak, avagy ne futottam légyen.
\par 3 De még a velem levõ Titus sem készszeríttetett a körülmetélkedésre, noha görög vala,
\par 4 Tudniillik a belopózkodott hamis atyafiakért, a kik alattomban közénk jöttek, hogy kikémleljék a mi szabadságunkat, melylyel bírunk a Krisztus Jézusban, hogy minket szolgákká tegyenek:
\par 5 Kiknek egy pillanatra sem adtuk meg magunkat, hogy az evangyéliom igazsága megmaradjon számotokra.
\par 6 A tekintélyesektõl pedig, (bárminõk valának régen, azzal nem törõdöm; Isten nem nézi az embernek személyét: mert velem a tekintélyesek semmit sem közöltek;
\par 7 Sõt ellenkezõleg, mikor látták, hogy én reám van bízva a körülmetéletlenség evangyélioma, mint Péterre a körülmetélésé;
\par 8 (Mert a ki erõs volt Péterben a körülmetélkedés apostolságára, bennem is erõs volt a pogányok között).
\par 9 És elismervén a nékem adatott kegyelmet, Jakab és Kéfás, meg János, kik oszlopokul tekintetnek, bajtársi jobbjukat nyujták nékem és Barnabásnak, hogy mi a pogányok között, õk pedig a körülmetélés között prédikáljunk:
\par 10 Csakhogy a szegényekrõl megemlékezzünk; a mit is én igyekeztem megcselekedni.
\par 11 Mikor pedig Péter Antiókhiába jött, szemtõl szembe ellene állottam, mivel panasz volt reá.
\par 12 Mert mielõtt némelyek oda jöttek Jakabtól, a pogányokkal együtt evett; mikor pedig oda jöttek, félrevonult és elkülönítette magát, félvén a körülmetélkedésbõl valóktól.
\par 13 És vele képmutatóskodtak a többi zsidók is, úgy hogy Barnabás szintén elcsábíttatott az õ tettetésök által.
\par 14 De mikor láttam, hogy nem egyenesen járnak az evangyéliom igazságához képest, mondék Péternek mindnyájok elõtt: Ha te zsidó létedre pogány módra élsz és nem zsidó módra, miként kényszeríted a pogányokat, hogy zsidó módra éljenek?
\par 15 Mi, természet szerint zsidók és nem pogányok közül való bûnösök,
\par 16 Tudván azt, hogy az ember nem igazul meg a törvény cselekedeteibõl, hanem a Jézus Krisztusban való hit által, mi is Krisztus Jézusban hittünk, hogy megigazuljunk a Krisztusban való hitbõl és nem a törvény cselekedeteibõl; Mivel a törvény cselekedeteibõl nem igazul meg egy test sem.
\par 17 Ha pedig a Krisztusban keresvén a megigazulást, mimagunk is bûnösnek találtatunk, avagy Krisztus bûnnek szolgája-é? Távol legyen.
\par 18 Mert, ha a miket elrontottam, azokat ismét fölépítem, önmagamat teszem bûnössé.
\par 19 Mert én a törvény által meghaltam a törvénynek, hogy Istennek éljek.
\par 20 Krisztussal együtt megfeszíttettem. Élek pedig többé nem én, hanem él bennem a Krisztus; a mely életet pedig most testben élek, az Isten Fiában való hitben élem, a ki szeretett engem és önmagát adta érettem.
\par 21 Nem törlöm el az Isten kegyelmét; mert ha a törvény által van az igazság, tehát Krisztus ok nélkül halt meg.

\chapter{3}

\par 1 Óh balgatag Galátziabeliek, kicsoda ígézett meg titeket, hogy ne engedelmeskedjetek az igazságnak, kiknek szemei elõtt a Jézus Krisztus úgy íratott le, mintha ti köztetek feszíttetett volna meg?
\par 2 Csak azt akarom megtudni tõletek: a törvény cselekedeteibõl kaptátok-é a Lelket, avagy a hit hallásából?
\par 3 Ennyire esztelenek vagytok? A mit Lélekben kezdtetek el, most testben fejeznétek be?
\par 4 Annyit szenvedtetek hiába? ha ugyan hiába.
\par 5 Annakokáért, a ki a Lelket szolgáltatja néktek, és hatalmas dolgokat mûvel bennetek, a törvény cselekedeteibõl, vagy a hit hallásából cselekeszi-é?
\par 6 Miképen Ábrahám hitt az Istennek, és tulajdoníttatott néki igazságul.
\par 7 Értsétek meg tehát, hogy a kik hitbõl vannak, azok az Ábrahám fiai.
\par 8 Elõre látván pedig az Írás, hogy Isten hitbõl fogja megigazítani a pogányokat, eleve hirdette Ábrahámnak, hogy: Te benned fognak megáldatni minden népek.
\par 9 Ekként a hitbõl valók áldatnak meg a hívõ Ábrahámmal.
\par 10 Mert a kik törvény cselekedeteibõl vannak, átok alatt vannak; minthogy meg van írva: Átkozott minden, a ki meg nem marad mindazokban, a mik megirattak a törvény könyvében, hogy azokat cselekedje.
\par 11 Hogy pedig a törvény által senki sem igazul meg Isten elõtt, nyilvánvaló, mert az igaz ember hitbõl él.
\par 12 A törvény pedig nincs hitbõl, hanem a mely ember cselekeszi azokat, élni fog azok által.
\par 13 Krisztus váltott meg minket a törvény átkától, átokká levén érettünk; mert meg van írva: Átkozott minden, a ki fán  függ:
\par 14 Hogy az Ábrahám áldása Krisztus Jézusban legyen a pogányokon, hogy a Lélek ígéretét elnyerjük hit által.
\par 15 Atyámfiai! ember szerint szólok. Lám az embernek megerõsített testámentomát senki erõtelenné nem teszi, sem ahhoz hozzá nem ád.
\par 16 Az ígéretek pedig Ábrahámnak adattak és az õ magvának. Nem mondja: És a magvaknak, mint sokról; hanem mint egyrõl. És a te magodnak, a ki a Krisztus.
\par 17 Ezt mondom pedig, hogy a kötést, melyet Isten elõször megerõsített a Krisztusra nézve, a négyszázharmincz esztendõ multán keletkezett törvény nem teszi erõtelenné, hogy megsemmisítse az ígéretet.
\par 18 Mert ha törvénybõl van az örökség, akkor többé nem ígéretbõl; Ábrahámnak pedig ígéret által ajándékozta azt az Isten.
\par 19 Micsoda tehát a törvény? A bûnök okáért adatott, a míg eljõ a Mag, a kinek tétetett az ígéret: rendeltetvén angyalok által, közben járó kezében.
\par 20 A közbenjáró pedig nem egyé, Isten ellenben egy.
\par 21 A törvény tehát az Isten ígéretei ellen van-é? Távol legyen! Mert ha olyan törvény adatott volna, a mely képes megeleveníteni, valóban a törvénybõl volna az igazság.
\par 22 De az Írás mindent bûn alá rekesztett, hogy az ígéret Jézus Krisztusban való hitbõl adassék a hívõknek.
\par 23 Minekelõtte pedig eljött a hit, törvény alatt õriztettünk, egybezárva az eljövendõ hit kinyilatkoztatásáig.
\par 24 Ekként a törvény Krisztusra vezérlõ mesterünkké lett, hogy hitbõl igazuljunk meg.
\par 25 De minekutána eljött a hit, nem vagyunk többé a vezérlõ mester alatt.
\par 26 Mert mindnyájan Isten fiai vagytok a Krisztus Jézusban való hit által.
\par 27 Mert a kik Krisztusba keresztelkedtetek meg, Krisztust öltöztétek fel.
\par 28 Nincs zsidó, sem görög; nincs szolga, sem szabad; nincs férfi, sem nõ; mert ti mindnyájan egyek vagytok a Krisztus Jézusban.
\par 29 Ha pedig Krisztuséi vagytok, tehát az Ábrahám magva vagytok, és ígéret szerint örökösök.

\chapter{4}

\par 1 Mondom pedig, hogy a meddig az örökös kiskorú, semmiben sem különbözik a szolgától, jóllehet ura mindennek;
\par 2 Hanem gyámok és gondviselõk alatt van az atyjától rendelt ideig.
\par 3 Azonképen mi is, mikor kiskorúak valánk, a világ elemei alá voltunk vettetve szolgaként:
\par 4 Mikor pedig eljött az idõnek teljessége, kibocsátotta Isten az õ Fiát, a ki asszonytól lett, a ki törvény alatt lett,
\par 5 Hogy a törvény alatt levõket megváltsa, hogy elnyerjük a fiúságot.
\par 6 Minthogy pedig fiak vagytok, kibocsátotta az Isten az õ Fiának Lelkét a ti szíveitekbe, ki ezt kiáltja: Abba, Atya!
\par 7 Azért nem vagy többé szolga, hanem fiú; ha pedig fiú, Istennek örököse is Krisztus által.
\par 8 Ámde akkor, mikor még nem ismertétek az Istent, azoknak szolgáltatok, a mik természet szerint nem istenek;
\par 9 Most azonban, hogy megismertétek az Istent, sõt hogy megismert tieket az Isten, miként tértek vissza ismét az erõtelen és gyarló elemekhez, a melyeknek megint újból szolgálni akartok?
\par 10 Megtartjátok a napokat és hónapokat  és idõket, meg az esztendõket.
\par 11 Féltelek titeket, hogy hiába fáradoztam körültetek.
\par 12 Legyetek olyanok, mint én; mert én is olyanná lettem, mint ti: atyámfiai, kérlek tieket, semmivel sem bántottatok meg engem.
\par 13 Tudjátok pedig, hogy testem erõtelensége miatt hirdettem néktek az evangyéliomot elõször.
\par 14 És megkísértetvén testemben, nem vetettetek meg, sem nem útáltatok meg engem, hanem úgy fogadtatok, mint Istennek angyalát, mint Krisztus Jézust.
\par 15 Hová lõn tehát a ti boldogságotok? Mert bizonyságot teszek néktek, hogy ti, ha lehetséges volt volna, szemeiteket kivájván, nékem adtátok volna.
\par 16 Tehát ellenségetek lettem-é, megmondván néktek az igazat?
\par 17 Nem szépen buzgolkodnak érettetek, sõt minket ki akarnak rekeszteni, hogy mellettök buzgolkodjatok.
\par 18 Szép dolog pedig fáradozni a jóban mindenkor, és nem csupán akkor, ha köztetek vagyok.
\par 19 Gyermekeim! kiket ismét fájdalommal szûlök, míglen kiábrázolódik bennetek Krisztus.
\par 20 Szeretnék pedig most köztetek jelen lenni és változtatni a hangomon; mert bizonytalanságban vagyok felõletek.
\par 21 Mondjátok meg nékem, kik a törvény alatt akartok lenni: nem halljátok-é a törvényt?
\par 22 Mert meg van írva, hogy Ábrahámnak két fia volt; egyik a szolgálótól, és a másik  szabadostól.
\par 23 De a szolgálótól való test szerint született; a szabadostól való pedig az ígéret által.
\par 24 Ezek mást példáznak: mert azok az asszonyok a két szövetség, az egyik a Sinai hegyrõl való, szolgaságra szûlõ, ez Hágár,
\par 25 Mert Hágár a Sinai hegy Arábiában, hasonlatos pedig a mostani Jeruzsálemhez, nevezetesen fiaival együtt szolgál.
\par 26 De a magasságos Jeruzsálem szabad, ez mindnyájunknak anyja,
\par 27 Mert meg van írva: Ujjongj te meddõ, ki nem szûlsz; vígadozzál és kiálts, ki nem vajudol; mert sokkal több az elhagyottnak magzatja, mint a kinek férje vagyon.
\par 28 Mi pedig, atyámfiai, Izsák szerint, ígéretnek gyermekei vagyunk.
\par 29 De valamint akkor a test szerint született üldözte a Lélek szerint valót, úgy most is.
\par 30 De mit mond az Írás? Ûzd ki a szolgálót és az õ fiát; mert a szolgáló fia nem örököl a szabad nõ fiával.
\par 31 Annakokáért, atyámfiai, nem vagyunk a szolgáló fiai, hanem a szabadoséi.

\chapter{5}

\par 1 Annakokáért a szabadságban, melyre minket Krisztus megszabadított, álljatok meg, és ne kötelezzétek meg ismét magatokat szolgaságnak igájával.
\par 2 Ímé, én Pál mondom néktek, hogy ha körülmetélkedtek, Krisztus néktek semmit sem használ.
\par 3 Bizonyságot teszek pedig ismét minden embernek, a ki körülmetélkedik, hogy köteles az egész törvényt megtartani.
\par 4 Elszakadtatok Krisztustól, a kik a törvény által akartok megigazulni, a kegyelembõl kiestetek.
\par 5 Mert mi a Lélek által, hitbõl várjuk az igazság reménységét.
\par 6 Mert Krisztus Jézusban sem a körülmetélkedés nem ér semmit, sem a körülmetélkedetlenség, hanem a szeretet által munkálkodó  hit.
\par 7 Jól futottatok; kicsoda gátolt meg titeket, hogy ne engedelmeskedjetek az igazságnak?
\par 8 Ez a hitetés nem attól van, a ki titeket hív.
\par 9 Kis kovász az egész tésztát megkeleszti.
\par 10 Bizodalmam van az Úrban ti hozzátok, hogy más értelemben nem lesztek; de a ki titeket megzavar, elveszi az ítéletet, bárki legyen.
\par 11 Én pedig atyámfiai, ha még a körülmetélést hirdetem, miért üldöztetem mégis? Akkor eltöröltetett a kereszt botránya.
\par 12 Bárcsak ki is metszetnék magukat, a kik titeket bujtogatnak.
\par 13 Mert ti szabadságra hivattatok atyámfiai; csakhogy a szabadság ürügye ne legyen a testnek, sõt szeretettel szolgáljatok egymásnak.
\par 14 Mert az egész törvény ez egy ígében teljesedik be: Szeresd felebarátodat, mint magadat.
\par 15 Ha pedig egymást marjátok és faljátok, vigyázzatok, hogy egymást fel ne emészszétek.
\par 16 Mondom pedig, Lélek szerint járjatok, és a testnek kívánságát véghez ne vigyétek.
\par 17 Mert a test a lélek ellen törekedik, a lélek pedig a test ellen; ezek pedig egymással ellenkeznek, hogy ne azokat cselekedjétek, a miket akartok.
\par 18 Ha azonban a Lélektõl vezéreltettek, nem vagytok a törvény alatt.
\par 19 A testnek cselekedetei pedig nyilvánvalóak, melyek ezek: házasságtörés, paráznaság, tisztátalanság, bujálkodás.
\par 20 Bálványimádás, varázslás, ellenségeskedés, versengések, gyûlölködések, harag, patvarkodások, viszszavonások, pártütések,
\par 21 Irígységek, gyilkosságok, részegségek, dobzódások és ezekhez hasonlók: melyekrõl elõre mondom néktek, a miképen már ezelõtt is mondottam, hogy a kik ilyeneket cselekesznek, Isten országának örökösei nem lesznek.
\par 22 De a Léleknek gyümölcse: szeretet, öröm, békesség, béketûrés, szívesség, jóság, hûség, szelídség, mértékletesség.
\par 23 Az ilyenek ellen nincs törvény.
\par 24 A kik pedig Krisztuséi, a testet megfeszítették indulataival és kívánságaival együtt.
\par 25 Ha Lélek szerint élünk, Lélek szerint is járjunk.
\par 26 Ne legyünk hiú dicsõség kívánók, egymást ingerlõk, egymásra irígykedõk.

\chapter{6}

\par 1 Atyámfiai, még ha elõfogja is az embert valami bûn, ti lelkiek, igazítsátok útba az olyant szelídségnek lelkével, ügyelvén magadra, hogy meg ne kísértessél te magad is.
\par 2 Egymás terhét hordozzátok, és úgy töltsétek be a Krisztus törvényét.
\par 3 Mert ha valaki azt véli, hogy õ valami, holott semmi, önmagát csalja meg.
\par 4 Minden ember pedig az õ maga cselekedetét vizsgálja meg, és akkor csakis önmagára nézve lesz dicsekedése és nem másra nézve.
\par 5 Mert kiki a maga terhét hordozza.
\par 6 A ki pedig az ígére taníttatik, közölje minden javát tanítójával.
\par 7 Ne tévelyegjetek, Isten nem csúfoltatik meg; mert a mit vet az ember, azt aratándja is.
\par 8 Mert a ki vet az õ testének, a testbõl arat veszedelmet; a ki pedig vet a léleknek, a lélekbõl arat örök életet.
\par 9 A jótéteményben pedig meg ne restüljünk, mert a maga idejében aratunk, ha el nem lankadunk.
\par 10 Annakokáért míg idõnk van, cselekedjünk jót mindenekkel, kiváltképen pedig a mi hitünknek cselédeivel.
\par 11 Látjátok, mekkora betûkkel írok néktek a saját kezemmel!
\par 12 A kik testre szépek szeretnének lenni, azok kényszerítenek titeket a körülmetélkedésre; csak azért, hogy a Krisztus keresztjéért ne üldöztessenek.
\par 13 Mert magok a körülmetélkedettek sem tartják meg a törvényt; hanem azért akarják, hogy ti körülmetélkedjetek, hogy a ti testetekkel dicsekedjenek.
\par 14 Nékem pedig ne legyen másban dicsekedésem, hanem a mi Urunk Jézus Krisztus keresztjében, a ki által nékem megfeszíttetett a világ, és én is a világnak.
\par 15 Mert Krisztus Jézusban sem a körülmetélkedés, sem a körülmetéletlenség nem használ semmit, hanem az új teremtés.
\par 16 És a kik e szabály szerint élnek, békesség és irgalmasság azokon, és az Istennek Izráelén.
\par 17 Ennekutána senki nékem bántásomra ne legyen; mert én az Úr Jézusnak bélyegeit hordozom az én testemben.
\par 18 A mi Urunk Jézus Krisztusnak kegyelme legyen a ti lelketekkel atyámfiai! Ámen.


\end{document}