\begin{document}

\title{Prédikátor könyve}


\chapter{1}

\par 1 A prédikátornak, Dávid fiának, Jeruzsálem  királyának beszédei.
\par 2 Felette nagy hiábavalóság, azt mondja a prédikátor; felette nagy hiábavalóság! Minden hiábavalóság!
\par 3 Micsoda haszna van az embernek minden õ munkájában, melylyel munkálkodik a nap alatt?
\par 4 Egyik nemzetség elmegy, és a másik eljõ; a föld pedig mindörökké megmarad.
\par 5 És a nap feltámad, és elnyugszik a nap; és az õ helyére siet, a hol õ ismét feltámad.
\par 6 Siet délre, és átmegy észak felé; körbe-körbe siet a szél, és a maga keringéséhez visszatér a szél.
\par 7 Minden folyóvíz siet a tengerbe; mindazáltal a tenger mégis meg nem telik: akármicsoda helyre a folyóvizek siessenek, ugyanazon helyre térnek vissza.
\par 8 Minden dolgok mint fáradoznak, senki ki nem mondhatja; nem elégednék meg a szem látván, sem be nem teljesednék hallásával a fül.
\par 9 A mi volt, ugyanaz, a mi ezután is lesz, és a mi történt, ugyanaz, a mi ezután történik; és semmi nincs új dolog a nap alatt.
\par 10 Van valami, a mirõl mondják: nézd ezt, új ez; régen volt már száz esztendõkön át, melyek mi elõttünk voltak.
\par 11 Nincs emlékezet az elõbbiekrõl; azonképen az utolsó dolgokról is, melyek jövendõk, nem lesz emlékezet azoknál, a kik azután lesznek.
\par 12 Én prédikátor, királya voltam Izráelnek Jeruzsálemben.
\par 13 És adám az én elmémet mindazok vizsgálására és bölcsen való tudakozására, melyek lesznek az ég alatt. Ez gonosz hiábavaló foglalatosság, melyet adott Isten az emberek fiainak, hogy gyötrõdjenek vele.
\par 14 Láttam minden dolgokat, melyek lesznek a nap alatt, és ímé minden csak hiábavalóság, és a léleknek gyötrelme!
\par 15 Az egyenetlen meg nem egyenesíthetõ, és a fogyatkozás meg nem számlálható.
\par 16 Szóltam az én elmémmel, mondván: ímé, én nagygyá lettem, és gyûjtöttem bölcseséget mindazok felett, a kik fõk voltak én elõttem Jeruzsálemben, és az én elmém bõven látott bölcseséget és tudományt!
\par 17 Adtam annakfelette az én elmémet a bölcseségnek tudására, és az esztelenségnek és bolondságnak megtudására. Megtudtam, hogy ez is a lélek  gyötrelme.
\par 18 Mert a bölcsességnek sokaságában sok búsulás van, és valaki öregbíti a tudományt, öregbíti a gyötrelmet.

\chapter{2}

\par 1 Mondék az én szívemben: no, megpróbállak téged a vígan való lakásban, hogy lásd meg, mi a jó! És ímé, az is hiábavalóság!
\par 2 A nevetésrõl azt mondom: bolondság! a vígasságról pedig: mit használ az?
\par 3 Elvégezém az én szívemben, hogy boritalra adom magamat, (pedig szívem a bölcseséget követé) és elõveszem ezt a bolondságot, mígnem meglátom, hogy az emberek fiainak mi volna jó, a mit cselekedjenek az ég alatt, az életök napjainak száma szerint.
\par 4 Felette nagy dolgokat cselekedtem; építék magamnak házakat; ülteték magamnak szõlõket.
\par 5 Csinálék magamnak kerteket és ékességre való kerteket, és ülteték beléjök mindenféle gyümölcstermõ fákat.
\par 6 Csinálék magamnak víztartó tavakat, hogy azokból öntözzem a fáknak sarjadó erdejét.
\par 7 Szerzek szolgákat és szolgálókat, házamnál nevekedett szolgáim is voltak nékem; öreg és apró barmoknak nyájaival is többel bírtam mindazoknál, a kik voltak én elõttem Jeruzsálemben.
\par 8 Gyûjték magamnak ezüstöt  és aranyat is, és királyok drágaságait és tartományokat; szerzék magamnak éneklõ férfiakat és éneklõ asszonyokat, és az emberek fiainak gyönyörûségit, asszonyt és asszonyokat.
\par 9 És nagygyá levék és megnevekedém mindazok felett, a kik elõttem voltak Jeruzsálemben; az én bölcseségem is helyén volt.
\par 10 Valamit kivánnak vala az én szemeim: meg nem fogtam azoktól, meg sem tartóztattam az én szívemet semmi vígasságtól, hanem az én szívem örvendezett minden én munkámmal gyûjtött jókban; mivelhogy ez volt az én részem minden én munkáimból.
\par 11 És tekinték minden dolgaimra, melyeket cselekedtek vala az én kezeim, és az én munkámra, mit fáradsággal végeztem vala; és ímé, az mind hiábavalóság és a léleknek gyötrelme, és nincsen annak semmi haszna a nap alatt!
\par 12 Azért fordulék én, hogy lássak bölcseséget és bolondságot és esztelenséget, az az hogy mit cselekesznek az emberek, a kik a király után következnek: azt, a mit régen cselekedtek.
\par 13 És látám, hogy hasznosb a bölcseség a bolondságnál, miképen hasznosb a világosság a setétségnél.
\par 14 A bölcsnek szemei vannak a fejében; a bolond pedig setétben jár; de ugyan én megismerém, hogy ugyanazon egy végök lesz mindezeknek.
\par 15 Annakokáért mondám az én elmémben: bolondnak állapotja szerint lesz az én állapotom is, miért valék tehát én is bölcsebb? és mondék az én elmémben: ez is hiábavalóság!
\par 16 Mert nem lesz emlékezete sem a bölcsnek, sem a bolondnak mindörökké; mivelhogy a következendõ idõkben már mind elfelejtetnek: és miképen meghal a bölcs, azonképen meghal a bolond is.
\par 17 Azért gyûlöltem az életet; mert gonosznak látszék nékem a dolog, a mi történik a nap alatt; mert mindez hiábavalóság, és a léleknek gyötrelme!
\par 18 Gyûlöltem én minden munkámat is, melyet munkálkodom a nap alatt; mivelhogy el kell hagynom azt oly embernek, a ki én utánam lesz.
\par 19 És ki tudja, ha bölcs lesz-é vagy bolond? és mégis uralkodik minden munkámon, a mit cselekedtem és bölcsen szerzettem a nap alatt! Ez is hiábavalóság!
\par 20 Annakokáért elfordulék én, megfogván reménységtõl az én szívemet minden munkám felõl, melylyel munkálódtam a nap alatt.
\par 21 Mert van oly ember, a kinek munkája elvégeztetett bölcseséggel, tudománynyal és jó kimenetellel; és oly embernek adja azt örökségül, a ki abban semmit sem munkálkodott. Ez is hiábavalóság és nagy gonosz!
\par 22 Mert micsoda marad meg az embernek minden õ munkájából és elméjének nyughatatlan fáradozásából, melylyel õ munkálódott a nap alatt?
\par 23 Holott minden napja bánat, és búsulás az õ foglalatossága, még éjjel is nem nyugodott az õ elméje. Ez is hiábavalóság!
\par 24 Nincsen csak e jó is az embernek hatalmában, hogy egyék, igyék, és azt cselekedje, hogy az õ szíve lakozzék gyönyörûséggel az õ munkájából; ezt is láttam én, hogy az Istennek kezében van.
\par 25 Mert kicsoda ehetnék és élhetne gyönyörûségére rajtam kivül?
\par 26 Mert az embernek, a ki jó az õ szemei elõtt adott Isten bölcseséget és tudományt és örömöt; a bûnösnek pedig adott foglalatosságot az egybegyûjtésre és az egybehordásra, hogy adja annak, a ki jó az Isten elõtt. Ez is hiábavalóság és az elmének gyötrelme!

\chapter{3}

\par 1 Mindennek rendelt ideje van és ideje van az ég alatt minden akaratnak.
\par 2 Ideje van a születésnek és ideje a meghalásnak; ideje az ültetésnek, ideje annak kiszaggatásának, a mi ültettetett.
\par 3 Ideje van a megölésnek és ideje a meggyógyításnak; ideje a rontásnak és ideje az építésnek.
\par 4 Ideje van a sírásnak és ideje a nevetésnek; ideje a jajgatásnak és ideje a szökdelésnek.
\par 5 Ideje van a kövek elhányásának és ideje a kövek egybegyûjtésének; ideje az ölelgetésnek és ideje az ölelgetéstõl való eltávozásnak.
\par 6 Ideje van a keresésnek és ideje a vesztésnek; ideje a megõrzésnek és ideje az eldobásnak.
\par 7 Ideje van a szakgatásnak és ideje a megvarrásnak; ideje a hallgatásnak és ideje a szólásnak.
\par 8 Ideje van a szeretésnek és ideje a gyûlölésnek ideje a hadakozásnak és ideje a békességnek.
\par 9 Micsoda haszna van a munkásnak abban, a miben õ munkálkodik?
\par 10 Láttam a foglalatosságot, melyet adott Isten az emberek fiainak, hogy fáradozzanak benne.
\par 11 Mindent szépen csinált az õ idejében, e világot is adta az emberek elméjébe, csakhogy úgy, hogy az ember meg nem foghatja mindazt a dolgot, a mit az Isten cselekszik kezdettõl fogva mindvégig.
\par 12 Megismertem, hogy nem tehetnek jobbat, mint hogy örvendezzen kiki, és hogy a maga javát cselekedje  az õ életében.
\par 13 De még az is, hogy az ember eszik és iszik, és jól él az õ egész munkájából, az Istennek ajándéka.
\par 14 Tudom, hogy valamit Isten cselekszik, az lesz örökké, ahhoz nincs mit adni és abból nincs mit elvenni; és az Isten ezt a végre míveli, hogy az õ orczáját rettegjék.
\par 15 A mi most történik, régen megvan, és a mi következik, immár megvolt, és az Isten visszahozza, a mi elmult.
\par 16 Láttam annakfelette a nap alatt, hogy az ítéletnek helyén hamisság, és az igazságnak helyén latorság van.
\par 17 És mondék magamban: az igazat és a hamisat megítéli az Isten; mert minden ember akaratjának ideje van, és minden dolognak õ nála.
\par 18 Így szólék azért magamban: az emberek fiai miatt van ez így, hogy kiválogassa õket az Isten, és hogy meglássák, hogy õk magokban véve az oktalan állatokhoz hasonlók.
\par 19 Az emberek fiainak vége hasonló az oktalan állatnak végéhez, és egyenlõ végök van azoknak; a mint meghal egyik, úgy meghal a másik is, és ugyanazon egy lélek van mindenikben; és az embernek nagyobb méltósága nincs az oktalan állatoknál, mert minden hiábavalóság.
\par 20 Mindenik ugyanazon egy helyre megy; mindenik a porból való, és mindenik porrá lesz.
\par 21 Vajjon kicsoda vette eszébe az ember lelkét, hogy felmegy-é; és az oktalan állat lelkét, hogy a föld alá megy-é?
\par 22 Azért úgy láttam, hogy semmi sincs jobb, mint hogy az ember örvendezzen az õ dolgaiban, mivelhogy ez az õ része e világban: mert ki hozhatja õt vissza, hogy lássa, mi lesz õ utána?

\chapter{4}

\par 1 Viszont látám én mind a nyomorgatásokat, a melyek a nap alatt történnek, és ímé, nyilván van azoknak, a kik nyomorgattatnak, könnyhullatások, és vígasztalójok nincs nékik; és az õket nyomorgatóknak kezekbõl erõszaktételt szenvednek, és vígasztalójuk nincs nékik.
\par 2 És dicsérém én a megholtakat, a kik már meghaltak vala, az élõk felett, a kik még élnek;
\par 3 De mind a kettõnél boldogabbnak azt, a ki még nem lett, a ki nem látta azt a gonosz dolgot, a mely a nap alatt történik.
\par 4 És látám én, hogy minden dolgát és minden ügyes cselekedetét az ember az õ felebarátja iránt való irígységbõl rendeli; annakokáért ez is hiábavalóság és lélek-fájdalom!
\par 5 A bolond egybekapcsolja a kezeit, és megemészti a maga testét.
\par 6 Jobb egy teljes marok nyugalommal, mint mind a két maroknak teljessége nagy munkával és lelki gyötrelemmel.
\par 7 Viszont láték a nap alatt más hiábavalóságot.
\par 8 Van oly ember, a ki egymaga van és nincs vele másik, sem fia, sem atyjafia nincs; mindazáltal nincs vége minden õ fáradságának, és az õ szeme is meg nem elégszik gazdagsággal, hogy azt mondaná: vajjon kinek munkálkodom, hogy az én lelkemet minden jótól megfosztom? Ez is hiábavalóság és gonosz foglalatosság!
\par 9 Sokkal jobban van dolga a kettõnek, hogynem az egynek; mert azoknak jó jutalmok vala az õ munkájokból.
\par 10 Mert ha elesnek is, az egyik felemeli a társát. Jaj pedig az egyedülvalónak, ha elesik, és nincsen, a ki õt felemelje.
\par 11 Hogyha együtt feküsznek is ketten, megmelegszenek; az egyedülvaló pedig mimódon melegedhetik meg?
\par 12 Ha az egyiket megtámadja is valaki, ketten ellene állhatnak annak; és a hármas kötél nem hamar szakad el.
\par 13 Jobb a szûkölködõ, de bölcs gyermek a vén és bolond királynál, a ki nem szenvedi el az intést többé.
\par 14 Mert az a fogságból is uralkodásra megy, holott ennek országában szegénységben született.
\par 15 Láttam a nap alatt járó minden élõket a második gyermek mellett; a ki amannak helyére lépendõ vala.
\par 16 És hogy az egész sokaságnak nincs vége, mindazoknak, a kiknek õ élén volt; mindazáltal az utánok valók már semmit nem örvendeztek õ benne. Mert ez is hiábavalóság és lelki gyötrelem!

\chapter{5}

\par 1 Õrizd meg lábaidat, mikor az Istennek házához mégy, mert hallgatás végett közeledned jobb, hogynem a bolondok módja szerint áldozatot adni; mert ezek nem tudják, hogy gonoszt cselekesznek.
\par 2 Ne gyorsalkodjál a te száddal, és a te elméd ne siessen valamit szólni Isten elõtt; mert az Isten mennyben van, te pedig e földön, azért a te beszéded kevés legyen;
\par 3 Mert álom szokott következni a sok foglalatosságból; és sok beszédbõl bolond beszéd.
\par 4 Mikor Istennek fogadást téssz, ne halogasd annak megadását; mert nem gyönyörködik a bolondokban. A mit fogadsz, megteljesítsd!
\par 5 Jobb hogy ne fogadj, hogynem mint fogadj és ne teljesítsd be.
\par 6 Ne engedd a te szádnak, hogy bûnre kötelezze testedet, és ne mondd az angyal elõtt, hogy tévedésbõl esett ez; hogy az Isten a te beszéded miatt fel ne háborodjék, és el ne veszesse a te kezeidnek munkáját.
\par 7 Mert a sok álomban a hiábavalóság is és a beszéd is sok; hanem az Istent féljed.
\par 8 Ha a szegényeknek nyomoríttatását, és a törvénynek és igazságnak elfordíttatását látod a tartományban: ne csudálkozzál e dolgon; mert egyik felsõrendû vigyáz a másik felsõrendûre, és ezek felett még felsõbbrendûek vannak.
\par 9 Az ország haszna pedig mindenekfelett a földmívelést kedvelõ király.
\par 10 A ki szereti a pénzt, nem telik be pénzzel, és a ki szereti a sokaságot nem telik be jövedelemmel. Ez is hiábavalóság.
\par 11 Mikor megszaporodik a jószág, megszaporodnak annak megemésztõi is; mi haszna van azért benne urának, hanem hogy csak reá néz szemeivel?
\par 12 Édes az álom a munkásnak, akár sokat, akár keveset egyék; a gazdagnak pedig bõvölködése nem hagyja õt aludni.
\par 13 Van gonosz nyavalya, a melyet láttam a nap alatt: az õ urának veszedelmére tartott gazdagság;
\par 14 Ugyanis az a gazdagság elvész valami szerencsétlen eset miatt, és ha fia született néki, annak kezében nem lesz semmi.
\par 15 A mint kijött az õ anyjának méhébõl, mezítelen megy ismét el, a mint jött vala: és semmit nem vesz el munkájáért, a mit kezében elvinne.
\par 16 Annakokáért ez is gonosz nyavalya, hogy a mint jött, a képen megy el. Mi haszna van néki abban, hogy a szélnek munkálkodott?
\par 17 És hogy az õ teljes életében a setétben evett, sokszori haraggal, keserûséggel és búsulással?
\par 18 Ez azért a jó, a melyet én láttam, hogy szép dolog enni és inni, és jól élni minden õ munkájából, a melylyel fárasztotta magát a nap alatt, az õ élete napjainak száma szerint, a melyeket adott néki az Isten; mert ez az õ része.
\par 19 És a mely embernek adott Isten gazdagságot és kincseket, és a kinek megengedte, hogy egyék abból és az õ részét elvegye, és örvendezzen az õ munkájának: ez az Istennek ajándéka!
\par 20 Mert nem sokat emlékezik meg az ilyen az õ élete napjainak számáról, mivelhogy az õ szívének örömét az Isten kedveli.

\chapter{6}

\par 1 Van egy gonosz, a melyet láttam a nap alatt, és nagy baj az az emberen;
\par 2 Mikor valakinek az Isten ád gazdagságot és kincseket és tisztességet, és semmi nélkül nem szûkölködik, valamit kivánhat lelkének, és az Isten nem engedi néki, hogy éljen azzal, hanem más ember él azzal: ez hiábavalóság és gonosz nyavalya!
\par 3 Ha száz gyermeket szül is valaki, és sok esztendeig él, úgy hogy az õ esztendeinek napja sok, de az õ lelke a jóval meg nem elégszik, és nem lesz temetése néki: azt mondom, hogy jobb annál az idétlen gyermek,
\par 4 Mert hiábavalóságra jött, setétségben megy el, és setétséggel fedeztetik be neve,
\par 5 A napot sem látta és nem ismerte; tûrhetõbb ennek állapotja, hogynem amannak.
\par 6 Hogyha kétezer esztendõt élt volna is, és a jóval nem élt: avagy nem ugyanazon egy helyre megy-é minden?
\par 7 Az embernek minden munkája szájáért van; mindazáltal az õ kívánsága be nem telik.
\par 8 Mert miben különbözik a bölcs a bolondtól, és miben a szegény, a ki az élõk elõtt járni tud?
\par 9 Jobb, a mit ember szemmel lát, hogynem a lélek kivánsága; ez is hiábavalóság és a léleknek gyötrelme!
\par 10 Valami van, régen ráadatott nevezete, és bizonyos dolog, hogy mi lesz az ember, és nem perlekedhetik azzal, a ki hatalmasb nálánál.
\par 11 Mert van sok beszéd, a mely a hiábavalóságot szaporítja; és mi haszna van az embernek abban?
\par 12 Mert kicsoda tudhatja, mi legyen az embernek jó e világon, az õ hiábavaló élete napjainak száma szerint, a melyeket mintegy árnyékot tölt el? Kicsoda az, a ki megmondhatná az embernek, mi következik õ utána a nap alatt?

\chapter{7}

\par 1 Jobb a jó hír a drága kenetnél; és a halálnak napja jobb az õ születésének napjánál.
\par 2 Jobb a siralmas házhoz menni, hogynem a lakodalomnak házához menni; mivelhogy minden embernek ez a vége, és az élõ ember megemlékezik arról.
\par 3 Jobb a szomorúság a nevetésnél; mert az orczának szomorúsága által jobbá lesz a szív.
\par 4 A bölcseknek elméje a siralmas házban van, a bolondoknak pedig elméje a vígasságnak házában.
\par 5 Jobb a bölcsnek dorgálását hallani, hogynem valaki hallja a bolondoknak éneklését.
\par 6 Mert olyan a bolondnak nevetése, mint a tövisnek ropogása a fazék alatt; ez is hiábavalóság!
\par 7 Mert a zsarolás megbolondítja a bölcs embert is, és az elmét elveszti az ajándék.
\par 8 Jobb akármi dolognak vége annak kezdetinél; jobb a tûrõ, hogynem a kevély.
\par 9 Ne légy hirtelen a lelkedben a haragra; mert a harag a bolondok kebelében nyugszik.
\par 10 Ne mondd ezt: mi az oka, hogy a régi napok jobbak voltak ezeknél? mert nem bölcseségbõl származik az ilyen kérdés.
\par 11 Jó a bölcseség az örökséggel, és elõmenetelökre van az embereknek, a kik a napot látják.
\par 12 Mert a bölcseségnek árnyéka alatt, és a gazdagságnak árnyéka alatt egyformán nyugszik az ember! de a tudomány hasznosb, mivelhogy a bölcseség életet ád az õ urainak.
\par 13 Tekintsd meg az Istennek cselekedetit; mert kicsoda teheti egyenessé, a mit õ görbévé tett?
\par 14 A jó szerencsének idején élj a jóval; a gonosz szerencsének idején pedig jusson eszedbe, hogy ezt is, épen úgy, mint azt, Isten szerzette, a végre, hogy az ember semmit abból eszébe ne vegyen, a mi reá következik.
\par 15 Mindent láttam az én hiábavalóságomnak napjain: van oly igaz, a ki az õ igazságában elvész; és van gonosz ember, a ki az õ életének napjait meghosszabbítja az õ gonoszságában.
\par 16 Ne légy felettébb igaz, és felettébb ne bölcselkedjél; miért kerestél magadnak veszedelmet?
\par 17 Ne légy felettébb gonosz, és ne légy balgatag; miért halnál meg idõd elõtt?
\par 18 Jobb, hogy ezt megfogd, és amattól is a te kezedet meg ne vond; mert a ki az Istent féli, mind ezektõl megszabadul!
\par 19 A bölcseség megerõsíti a bölcset inkább, mint tíz hatalmas, a kik a városban vannak.
\par 20 Mert nincs egy igaz ember is a földön, a ki jót cselekednék és nem vétkeznék.
\par 21 Ne figyelmezz minden beszédre, melyet mondanak, hogy meg ne halld szolgádat, hogy átkoz téged.
\par 22 Mert sok esetben tudja a te lelked is, hogy te is gonoszt mondottál egyebeknek.
\par 23 Mind ezeket megpróbáltam az én bölcseségem által. Mikor azt gondolám, hogy bölcs vagyok, én tõlem a bölcseség távol vala.
\par 24 Felette igen messze van, a mi van, és felette mélységes; kicsoda tudhatja meg azt?
\par 25 Fordítám én magamat és az én szívemet a bölcseségnek és az okoskodásnak tudására, kutatására és keresésére; azonképen hogy megtudjam a bolondságnak gonoszságát, és a tévelygésnek balgatagságát.
\par 26 És találtam egy dolgot, mely keservesb a halálnál; tudniillik az olyan asszonyt, a kinek a szíve olyan, mint a tõr és a háló, kezei pedig olyanok, mint a kötelek. A ki Isten elõtt kedves, megszabadul attól; a bûnös pedig megfogattatik attól.
\par 27 Lásd, ezt találtam, azt mondja a prédikátor; mikor gyakorta nagy szorgalmassággal keresém a megfejtést,
\par 28 A mit az én lelkem folyton keresett, és nem találtam. Ezer közül egy embert találtam; de asszonyt mind ezekben nem találtam.
\par 29 Hanem lásd, ezt találtam, hogy az Isten teremtette az embert igaznak; õk pedig kerestek sok kigondolást.

\chapter{8}

\par 1 Kicsoda hasonló a bölcshöz, és ki tudja a dolgok magyarázatát? Az embernek bölcsesége megvilágosítja az õ orczáját; és az õ ábrázatjának erõssége megváltozik.
\par 2 Én mondom, hogy a királynak parancsolatját meg kell õrizni, és pedig az Istenre való esküvés miatt.
\par 3 Ne siess elmenni az õ orczája elõl, ne állj rá a gonosz dologra; mert valamit akar, megcselekszi.
\par 4 Mivelhogy a király szava hatalmas; és kicsoda merné néki ezt mondani: mit mívelsz?
\par 5 A ki megtartja a parancsolatot, nem ismer nyomorúságot, és a bölcsnek elméje megért mind idõt, mind ítéletet;
\par 6 Mert minden akaratnak van ideje és ítélete; mert az embernek nyomorúsága sok õ rajta;
\par 7 Mert nem tudja azt, a mi következik; mert ki mondja meg néki, mimódon lesz az?
\par 8 Egy ember sem uralkodhatik a szélen, hogy feltartsa a szelet; és semmi hatalmasság nincs a halálnak napja felett, és az ütközetben senkit el nem bocsátanak; és a gonoszság nem szabadítja meg azt, a ki azzal él.
\par 9 Mindezt láttam, és megfigyeltem minden dolgot, a mely történik a nap alatt, oly idõben, a melyben uralkodik az ember az emberen maga kárára.
\par 10 És azután láttam, hogy a gonoszok eltemettettek és nyugalomra mentek; viszont a szent helyrõl kimentek és elfelejttettek a városban olyanok, a kik becsületesen cselekedtek. Ez is hiábavalóság!
\par 11 Mivelhogy hamar a szentenczia nem végeztetik el a gonoszságnak cselekedõjén, egészen arra van az emberek fiainak szíve õ bennök, hogy gonoszt cselekedjenek.
\par 12 Bár meghosszabbítja életét a bûnös, a ki százszor is vétkezik; mégis tudom én, hogy az istenfélõknek lészen jól dolgok, a kik az õ orczáját félik;
\par 13 A hitetlennek pedig nem lesz jó dolga, és nem hosszabbítja meg az õ életét, olyan lesz, mint az árnyék, mert nem rettegi az Istennek orczáját.
\par 14 Van hiábavalóság, a mely e földön történik; az, hogy vannak oly igazak, a kiknek dolga a gonoszoknak cselekedetei szerint lesz; és vannak gonoszok, a kiknek dolga az igazaknak cselekedetei szerint lesz; mondám, hogy ez is hiábavalóság.
\par 15 Annakokáért dicsérem vala én a vígasságot, hogy nincsen embernek jobb e világon, hanem hogy egyék, igyék és vígadjon; és ez kisérje õt munkájában az õ életének napjaiban, a melyeket ád néki Isten a nap alatt.
\par 16 Mikor adám az én szívemet a bölcseségnek megtudására, és hogy megvizsgáljak minden fáradságot, a mely e földön történik, (mert sem éjjel, sem nappal az emberek szeme álmot nem lát):
\par 17 Akkor eszembe vevém az Istennek minden dolgát, hogy az ember nem mehet végére a dolognak, amely a nap alatt történik; mert fáradozik az ember, hogy annak végére menjen, de nem mehet végére: sõt ha azt mondja is a bölcs ember, hogy tudja, nem mehet végére.

\chapter{9}

\par 1 Mert mindezt szívemre vettem, és pedig azért, hogy megvizsgáljam mindezt: hogy az igazak és bölcsek és azoknak minden cselekedetei Isten kezében vannak; szeretet is, gyûlölet is, nem tudják az emberek, mind ez elõttük van.
\par 2 Minden olyan, hogy mindenkit érhet, egyazon szerencséje van az igaznak és gonosznak, jónak vagy tisztának és tisztátalannak, mind annak, aki áldozik, mind a ki nem áldozik, úgy a jónak, mint a bûnösnek, az esküvõnek úgy, mint a ki féli az esküvést.
\par 3 Mind az ég alatt való dolgokban e gonosz van, hogy mindeneknek egyenlõ szerencséjök van; és az emberek fiainak szíve is teljes gonoszsággal, és elméjökben minden bolondság van, a míg élnek, azután pedig a halottak közé mennek.
\par 4 Mert akárkinek, valaki minden élõk közé csatlakozik, van reménysége; mert jobb az élõ eb, hogynem a megholt oroszlán.
\par 5 Mert az élõk tudják, hogy meghalnak; de a halottak semmit nem tudnak, és azoknak semmi jutalmok nincs többé; mivelhogy emlékezetök elfelejtetett.
\par 6 Mind szeretetök, mind gyûlöletök, mind gerjedezésök immár elveszett; és többé semmi részök nincs semmi dologban, a mely a nap alatt történik.
\par 7 No azért egyed vígassággal a te kenyeredet, és igyad jó szívvel a te borodat; mert immár kedvesek Istennek a te cselekedetid!
\par 8 A te ruháid mindenkor legyenek fejérek, és az olaj a te fejedrõl el ne fogyatkozzék.
\par 9 Éld életedet a te feleségeddel, a kit szeretsz, a te hiábavaló életednek minden napjaiban, a melyeket Isten adott néked a nap alatt, a te hiábavalóságodnak minden napjaiban; mert ez a te részed a te életedben és a te munkádban, melylyel munkálódol a nap alatt.
\par 10 Valamit hatalmadban van cselekedni erõd szerint, azt cselekedjed; mert semmi cselekedet, okoskodás, tudomány és bölcseség nincs a Seolban, a hová menendõ vagy.
\par 11 Fordítván magamat látám a nap alatt, hogy nem a gyorsaké a futás, és nem az erõseké a viadal, és nem a bölcseké a kenyér, és nem az okosoké a gazdagság, és nem a tudósoké a kedvesség; hanem idõ szerint és történetbõl lesznek mindezek.
\par 12 Mert nem is tudja az ember az õ idejét; mint a halak, melyek megfogatnak a gonosz hálóban, és mint a madarak, melyek megfogatnak a tõrben, miképen ezek, azonképen megfogatnak az emberek fiai a gonosznak idején, mikor az eljõ reájok hirtelenséggel.
\par 13 Ezt is bölcseségnek láttam a nap alatt, és ez én elõttem nagy volt.
\par 14 Tudniillik, hogy egy kicsiny város volt, és abban kevés ember volt, és eljött az ellen hatalmas király, és azt körülvette, és az ellen nagy erõsségeket épített.
\par 15 És találtatott abban egy szegény ember, a ki bölcs volt, és az õ bölcseségével a várost megszabadította; de senki meg nem emlékezett arról a szegény emberrõl.
\par 16 Akkor én azt mondám: jobb a bölcseség az erõsségnél; de a szegénynek bölcsesége útálatos, és az õ beszédit nem hallgatják meg.
\par 17 A bölcseknek nyugodt beszédét inkább meghallgatják, mint a bolondok közt uralkodónak kiáltását.
\par 18 Jobb a bölcsesség a hadakozó szerszámoknál; és egy bûnös sok jót veszt el.

\chapter{10}

\par 1 A megholt legyek a patikáriusnak kenetit megbüdösítik, megerjesztik; azonképen hathatósabb a bölcseségnél, tisztességnél egy kicsiny balgatagság.
\par 2 A bölcs embernek szíve az õ jobbkezénél van; a bolondnak pedig szíve balkezénél.
\par 3 A bolond, mikor az úton jár is, az õ elméje hiányos, és mindennek hirdeti, hogy õ bolond.
\par 4 Mikor a fejedelemnek haragja felgerjed te ellened, a te helyedet el ne hagyjad; mert a szelídség nagy bûnöket lecsendesít.
\par 5 Van egy gonosz, a melyet láttam a nap alatt, mintha tévedés volna, a mely a fejedelemtõl származik.
\par 6 Hogy a bolondság nagy méltóságra helyeztetett, a gazdagok pedig alacsony sorsban ülnek.
\par 7 Láttam, hogy a szolgák lovakon ültek; a fejedelmek pedig gyalog mentek a földön, mint a szolgák.
\par 8 A ki vermet ás, abba beesik; és a ki a gyepût elhányja, megmarja azt a kígyó.
\par 9 A ki a köveket helyökbõl kihányja, fájdalmat szenved azok miatt; a ki hasogatja a fát, veszedelemben forog a miatt.
\par 10 Ha a vas megtompul, és annak élit meg nem köszörüli az ember, akkor erejét kell megfeszíteni; a bölcseség pedig minden dolognak eligazítására nagy elõmenetel.
\par 11 Ha megharap a kígyó, a míg meg nem varázsoltatott, azután semmi haszna nincsen a varázslónak.
\par 12 A bölcs ember szájának beszédei kedvesek; a bolondnak pedig ajkai elnyelik õt.
\par 13 Az õ szája beszédinek kezdete bolondság, és az õ szája beszédinek vége gonosz balgatagság.
\par 14 És a bolond szaporítja a szót, pedig nem tudja az ember, a mi következik, és a mi utána lesz, kicsoda mondja meg azt néki?
\par 15 A bolondnak munkája elfárasztja õt, mert a városba sem tud menni.
\par 16 Jaj néked ország, kinek a te királyod gyermek; és a te fejedelmid reggel esznek.
\par 17 Boldog vagy te ország, kinek a te királyod nemes ember, és a te fejedelmid idejében esznek a testnek erejéért és nem az italért.
\par 18 A restség miatt elhanyatlik a házfedél, és a kezek restsége miatt csepeg a ház.
\par 19 Vígasságnak okáért szereznek lakodalmat, és a bor vídámítja meg az élõket: és a pénz szerzi meg mindezeket.
\par 20 Még a te gondolatodban is a királyt ne átkozd, és a te ágyasházadban is gonoszt a gazdagnak ne mondj: mert az égi madár is elviszi a szót, és a szárnyas állat is bevádolná a te beszédedet.

\chapter{11}

\par 1 Vesd a te kenyeredet a víz színére, mert sok nap mulva megtalálod azt.
\par 2 Adj részt hétnek vagy nyolcznak is; mert nem tudod, micsoda veszedelem következik a földre.
\par 3 Mikor a sûrû fellegek megtelnek, esõt adnak a földre; és ha leesik a fa délre vagy északra, a mely helyre leesik a fa, ott marad.
\par 4 A ki a szelet nézi, nem vet az; és a ki sûrû fellegre néz, nem arat.
\par 5 Miképen hogy nem tudod, melyik a szélnek útja, és miképen vannak a csontok a terhes asszony méhében; azonképen nem tudod az Istennek dolgát, a ki mindeneket cselekszik.
\par 6 Reggel vesd el a te magodat, és este se pihentesd kezedet; mert nem tudod, melyik jobb, ez-é vagy amaz, vagy ha mind a kettõ jó lesz egyszersmind.
\par 7 Valóban, édes a világosság és jó a mi szemeinkkel néznünk a napot.
\par 8 Mert ha sok esztendeig él is az ember, mindazokban örvendezzen; és megemlékezzék a setétségnek napjairól, mert az sok lesz. Valami eljövendõ, mind hiábavalóság.
\par 9 Örvendezz a te ifjuságodban, és vídámítson meg téged a te szíved a te ifjúságodnak idejében, és járj a te szívednek útaiban, és szemeidnek látásiban; de megtudd, hogy mindezekért az Isten tégedet ítéletre von!
\par 10 Vesd el a haragot a te szívedbõl, és vesd el a gonoszt a te testedbõl; mert az ifjúság és a hajnal hiábavalóság.

\chapter{12}

\par 1 És emlékezzél meg a te Teremtõdrõl a te ifjúságodnak idejében, míg a veszedelemnek napjai el nem jõnek, és míg el nem jõnek az esztendõk, melyekrõl azt mondod: nem szeretem ezeket!
\par 2 A míg a nap meg nem setétedik, a világossággal, a holddal és csillagokkal egybe; és a sûrû felhõk ismét visszatérnek az esõ után.
\par 3 Az idõben, mikor megremegnek a háznak õrizõi, és megrogynak az erõs férfiak, és megállanak az õrlõ leányok, mert megkevesbedtek, és meghomályosodnak az ablakon kinézõk.
\par 4 És az ajtók kivül bezáratnak, a mikor is a malom zúgása halkabbá lesz; és felkelnek a madár szóra, és halkabbakká lesznek minden éneklõ leányok.
\par 5 Minden halmocskától is félnek, és mindenféle ijedelmek vannak az úton, és a mandolafa megvirágzik, és a sáska nehezen vonszolja magát, és kipattan a kapor; mert elmegy az ember az õ örökös házába, és az utczán körül járnak a sírók.
\par 6 Minekelõtte elszakadna az ezüst kötél és megromolna az arany palaczkocska, és a veder eltörnék a forrásnál, és beletörnék a kerék a kútba,
\par 7 És a por földdé lenne, mint azelõtt volt; a lélek pedig megtérne Istenhez, a ki  adta volt azt.
\par 8 Felette nagy hiábavalóságok, azt mondja a prédikátor, mindezek hiábavalóságok!
\par 9 És azonfelül, hogy a prédikátor bölcs volt, még a népet is tudományra tanította, és fontolgatott, és tudakozott, és írt sok bölcs mondást.
\par 10 És igyekezett a prédikátor megtudni sok kivánságos beszédeket, igaz írást és igaz beszédeket.
\par 11 A bölcseknek beszédei hasonlatosak az ösztökéhez, és mint a szegek, erõsen le vannak verve a gyülekezetek tanítóinak szavai; melyek egy pásztortól adattak.
\par 12 Mindezekbõl, fiam, intessél meg: a sok könyvek írásának nincs vége, és a sok tanulás fáradságára van a testnek.
\par 13 A dolognak summája, mindezeket hallván, ez: az Istent féljed, és az õ parancsolatit megtartsad; mert ez az embernek fõdolga!
\par 14 Mert minden cselekedetet az Isten ítéletre elõhoz, minden titkos dologgal, akár jó, akár gonosz legyen az.


\end{document}