\begin{document}

\title{Nahumo knyga}

\chapter{1}


\par 1 Sprendimas apie Ninevę. Nahumo iš Elkošo regėjimo knyga. 
\par 2 Dievas pavydus ir Viešpats keršija. Viešpats keršija ir atlygina savo priešams. 
\par 3 Viešpats yra lėtas pykti ir didis savo galia, Viešpats baudžia nusikaltėlį. Jis eina viesule ir audroje, debesys yra Jo kojų dulkės. 
\par 4 Jis sudraudžia jūrą, padaro ją sausuma ir visas upes išdžiovina. Sunyksta Bašanas bei Karmelis ir Libano žiedas nuvysta. 
\par 5 Kalnai svyruoja prieš Jį, kalvos tirpsta. Žemė ir visa, kas joje gyvena, dreba Jo akivaizdoje. 
\par 6 Kas gali atsilaikyti prieš Jo pyktį ir kas gali ištverti Jo rūstybės įkarštį? Jo rūstybė išsilieja kaip ugnis, Jis išvarto uolas. 
\par 7 Viešpats yra geras, tvirtovė nelaimės metu. Jis pažįsta tuos, kurie Juo pasitiki. 
\par 8 Bet kaip galingas tvanas Jis sunaikins šią vietą, tamsa persekios Jo priešus. 
\par 9 Kodėl jūs priešinatės Viešpačiui? Jis visiškai sunaikins jus, ir nelaimė nebepasikartos. 
\par 10 Jie susipynę kaip erškėčiai, girti kaip girtuokliai, jie sudegs kaip sausos ražienos! 
\par 11 Iš tavęs išėjo tas, kuris planuoja pikta prieš Viešpatį­nedoras patarėjas. 
\par 12 Taip sako Viešpats: “Nors jie tvirti ir gausūs, jie sunyks ir pradings. Nors varginau tave, bet daugiau tavęs nebevarginsiu. 
\par 13 Dabar sulaužysiu jos jungą, nuimsiu nuo tavęs pančius ir sutraukysiu juos”. 
\par 14 Viešpats sako apie tave: “Tavo vardas nebebus minimas! Aš tavo šventyklų dievus­drožtus ir lietus­sunaikinsiu, o tau paruošiu kapą, nes netekai garbės”. 
\par 15 Štai kalnuose kojos nešančio gerą žinią, skelbiančio taiką. Judai, džiūgauk ir švęsk! Ištesėk pažadus, nes tavyje nebevaikščios nedorėlis, jis visiškai sunaikintas!


\chapter{2}


\par 1 Tas, kuris daužo į gabalus, ateina prieš tave. Budėk, stebėk kelią, pasiruošk, sukaupk visas jėgas! 
\par 2 Viešpats atgaivina Jokūbo ir Izraelio didybę, nors priešai juos nusiaubė ir jų vynmedžių šakeles sulaužė. 
\par 3 Priešų skydai raudoni, karių apranga skaisčiai raudonos spalvos. Kovos vežimai žibės kaip liepsna, svyruos iečių miškas. 
\par 4 Kovos vežimai lėks gatvėmis ir aikštėmis tarsi žaibas. 
\par 5 Jis šauks savo drąsiuosius, jie skubės klupdami, veršis prie sienos, bet apgultis jau bus paruošta. 
\par 6 Upių vartai bus atidaryti, rūmai užimami. 
\par 7 Karalienė bus vedama į nelaisvę, o jos tarnaitės dejuos kaip balandžiai ir mušis į krūtinę. 
\par 8 Ninevė yra tvenkinys, iš kurio ištekės vanduo. Jis šauks: “Sustokite, sustokite!”, bet niekas nekreips dėmesio. 
\par 9 Plėškite sidabrą, plėškite auksą! Čia turtų daugybė, nėra jiems galo! 
\par 10 Ji tuščia, išplėšta ir sunaikinta. Širdis alpsta, keliai dreba, visų veidai iš baimės pabalę. 
\par 11 Kur liūto buveinė, kur jaunų liūtų ola, kur liūtukai, kurie nepažino baimės? 
\par 12 Liūtas plėšė ir smaugė savo jaunikliams ir liūtėms, jis grobiu pripildė olas ir landas. 
\par 13 “Aš esu prieš tave,­sako kareivijų Viešpats.­Aš sudeginsiu tavo kovos vežimus, tavo jaunuolius sunaikins kardas. Tu neteksi grobio, niekas nebegirdės tavo pasiuntinių balso”.



\chapter{3}


\par 1 Vargas kruvinam miestui; jis pilnas klastos ir smurto, o plėšimai jame nesiliauja! 
\par 2 Botagų pliaukšėjimas ir ratų bildesys! Šuoliais lekia žirgai, darda kovos vežimai! 
\par 3 Raitelis pakelia švytintį kardą ir blykčiojančią ietį. Daugybė užmuštų, krūvos negyvų, begalės lavonų! Jie klupinėja, eidami per lavonus. 
\par 4 Tai dėl daugybės paleistuvysčių ir žavingų kerėtojų, kurios apraizgė tautas ir gimines savo kerais. 
\par 5 “Štai Aš esu prieš tave,­sako kareivijų Viešpats.­Aš pastatysiu tave nuogą prieš tautas ir karalystes; 
\par 6 apdrabstysiu tave purvais, išniekinsiu ir padarysiu tave gėdos stulpu. 
\par 7 Kiekvienas, kuris tave matys, pasitrauks nuo tavęs, sakydamas: ‘Ninevė sunaikinta! Kas ją apraudos? Kas ją nuramins?’ ” 
\par 8 Ar tu geresnė už No Amono miestą, kuris buvo tarp upių, apsuptas vandenų? 
\par 9 Jo stiprybė buvo Etiopija ir Egiptas, tavo pagalbininkai­Putas ir Libija. 
\par 10 O vis dėlto jis buvo ištremtas, pateko nelaisvėn. Jo vaikai buvo sutraiškyti gatvių kampuose, dėl jo garbingųjų metė burtą, visi didžiūnai buvo sukaustyti grandinėmis. 
\par 11 Tu irgi būsi nugirdyta ir paniekinta ieškosi prieglaudos. 
\par 12 Visos tavo tvirtovės yra lyg figmedžiai su ankstyvosiomis figomis; medžius papurčius, jos krinta į valgančiojo burną. 
\par 13 Tavo tauta kaip moterys: krašto vartai bus plačiai priešams atverti, ugnis suės tavo užkaiščius. 
\par 14 Turėk vandens atsargų tvirtovėje. Mink molį, gamink plytų tvirtovei sustiprinti! 
\par 15 Ugnis ten praris tave, kardas naikins tave kaip skėriai, nors jūsų pačių būtų daug kaip skėrių ir vikšrų. 
\par 16 Nors tavo pirklių yra tiek, kiek dangaus žvaigždžių, jie kaip skėrių vikšrai išsiners ir nuskris! 
\par 17 Tavo valdininkai ir vadai yra kaip skėrių būrys, kuris guli, o saulei patekėjus, nulekia, ir niekas nežino, kur jis yra. 
\par 18 Asirijos karaliau, tavo ganytojai snaudžia ir kilmingieji miega, tauta išsklaidyta kalnuose, nėra kas ją surenka. 
\par 19 Tavo žlugimas nesustabdomas, žaizda mirtina. Visi, išgirdę šią žinią apie tave, ploja rankomis, nes tavo nedorybė palietė visus.



\end{document}