\begin{document}

\title{Joshua}

\chapter{1}

\par 1 Viešpaties tarnui Mozei mirus, Viešpats tarė Nūno sūnui Jozuei, Mozės tarnui: 
\par 2 “Mano tarnas Mozė mirė. Pakilk ir eik per Jordaną su visa tauta į žemę, kurią duodu izraelitams. 
\par 3 Kiekvieną vietą, ant kurios jūsų koja atsistos, jums duodu, kaip pažadėjau Mozei. 
\par 4 Nuo dykumos ir Libano kalnų iki Eufrato upės, visa hetitų šalis, ir iki Didžiosios jūros vakaruose bus jūsų ribos. 
\par 5 Niekas prieš tave neatsilaikys per visą tavo gyvenimą. Kaip buvau su Moze, taip visuomet būsiu su tavimi. Aš nepaliksiu ir neapleisiu tavęs. 
\par 6 Būk stiprus ir drąsus; tu padalinsi žemę tautai, kurią duoti pažadėjau jų tėvams. 
\par 7 Tik būk stiprus ir labai drąsus, kad galėtum išpildyti įstatymą, kurį tau Mozė, mano tarnas, įsakė. Nenukrypk nuo jo nei į kairę, nei į dešinę, ir tau visuomet seksis, kur tik eisi. 
\par 8 Šita įstatymo knyga teneatsitraukia nuo tavo burnos, bet mąstyk apie ją dieną ir naktį, kad tiksliai vykdytum viską, kas joje parašyta; tada visa, ką bedarytum, klestės ir visur tau seksis. 
\par 9 Atsimink, ką įsakiau,­būk stiprus ir drąsus, nenusigąsk ir nebijok! Aš, Viešpats, tavo Dievas, būsiu su tavimi, kur tik tu eisi”. 
\par 10 Tada Jozuė įsakė tautos vyresniesiems: 
\par 11 “Pereikite per stovyklą ir įsakykite žmonėms paruošti maisto atsargų, nes po trijų dienų jūs pereisite Jordaną, kad paveldėtumėte žemę, kurią Viešpats, jūsų Dievas, jums yra pažadėjęs”. 
\par 12 Rubeno, Gado ir pusei Manaso giminės Jozuė kalbėjo: 
\par 13 “Atsiminkite įsakymą, kurį jums davė Viešpaties tarnas Mozė: ‘Viešpats, jūsų Dievas, jums davė šitą žemę’. 
\par 14 Jūsų žmonos, vaikai ir gyvuliai tepasilieka žemėje, kurią Mozė jums davė šioje Jordano pusėje. O jūs visi, stiprūs karo vyrai, apsiginklavę traukite savo brolių priekyje ir jiems padėkite, 
\par 15 iki Viešpats duos poilsį jūsų broliams, kaip ir jums davė, ir jie užims žemę, kurią Viešpats, jūsų Dievas, jiems pažadėjo. Paskui galėsite grįžti į savo žemę, kurią Viešpaties tarnas Mozė jums davė Jordano rytų pusėje”. 
\par 16 Jie atsakė Jozuei: “Visa, ką mums įsakysi, darysime ir visur, kur mus siųsi, eisime. 
\par 17 Kaip mes klausėme Mozės, taip ir tavęs klausysime. Viešpats, tavo Dievas, tebūna su tavimi, kaip Jis buvo su Moze! 
\par 18 Kiekvienas, kuris priešinsis tavo įsakymui ir neklausys tavo žodžių, bus nubaustas mirtimi. Tik būk stiprus ir drąsus!”


\chapter{2}

\par 1 Jozuė, Nūno sūnus, slaptai išsiuntė iš Šitimo du žvalgus įsakydamas: “Eikite ir apžiūrėkite kraštą ir Jerichą!” Juodu atėjo į paleistuvės Rahabos namus ir ten nakvojo. 
\par 2 Buvo pranešta Jericho karaliui: “Naktį atėjo du vyrai iš Izraelio vaikų išžvalgyti kraštą”. 
\par 3 Jericho karalius nusiuntė pas Rahabą pasiuntinius, įsakydamas: “Išduok vyrus, kurie yra tavo namuose, nes jie yra žvalgai”. 
\par 4 Moteris, abu vyrus paslėpusi, tarė: “Tiesa, vyrai buvo atėję pas mane, bet aš nežinojau, iš kur jie. 
\par 5 Prieš uždarant vartus, temstant, tie vyrai išėjo. Aš nežinau, kur jie nuėjo. Skubiai vykitės juos, gal dar pavysite”. 
\par 6 Ji buvo užvedusi juos ant stogo ir paslėpusi po nemintais linais, kurie ten buvo. 
\par 7 Jericho vyrai vijosi juos Jordano keliu iki brastos. Kai tik persekiotojai išėjo, vartus užrakino. 
\par 8 Jiems dar neatsigulus, ji užlipo pas juos ant stogo 
\par 9 ir tarė: “Aš žinau, kad Viešpats jums atidavė šitą šalį. Mus apėmė siaubas dėl jūsų, ir visi krašto gyventojai nusiminė. 
\par 10 Mes girdėjome, kaip Viešpats, jums išeinant iš Egipto, išdžiovino prieš jus Raudonosios jūros vandenį ir ką jūs padarėte abiem amoritų karaliams anapus Jordano, Sihonui ir Ogui, kuriuos visai sunaikinote. 
\par 11 Tai išgirdę, išsigandome ir netekome drąsos. Viešpats, jūsų Dievas, yra dangaus ir žemės Dievas. 
\par 12 Dabar prisiekite man Viešpačiu, kad jūs pasielgsite su mano tėvo namais taip, kaip aš pasielgiau su jumis, ir duokite man tikrą ženklą, 
\par 13 kad paliksite gyvus mano tėvą, motiną, brolius, seseris ir visus, kurie jiems priklauso, ir išgelbėsite mūsų gyvybes”. 
\par 14 Žvalgai jai atsakė: “Teištinka mus mirtis jūsų vietoje! Jei neišduosi mūsų, tai, kai Viešpats mums atiduos šitą šalį, mes gerai ir teisingai pasielgsime su tavimi”. 
\par 15 Tada Raaba nuleido juos virve pro langą, nes ji gyveno miesto mūro sienoje. 
\par 16 Ji tarė jiems: “Eikite į kalnus, kad jūsų nesutiktų besivejantieji, ten slapstykitės tris dienas, kol jie sugrįš; po to eikite savo keliu”. 
\par 17 Vyrai jai atsakė: “Mes laisvi būsime nuo tau duotos priesaikos, 
\par 18 jei, mums įeinant į kraštą, nepririši raudonos virvės prie lango, pro kurį mus nuleidai, ir savo tėvo, motinos, brolių ir visų namiškių neatsivesi į savo namus. 
\par 19 Kiekvienas, kuris išeis į gatvę iš tavo namų, bus pats kaltas dėl savo mirties, mes būsime nekalti. O kas pasiliks tavo namuose ir bus nužudytas, jo kraujas kris ant mūsų. 
\par 20 Tačiau jei mus išduosi, mes būsime laisvi nuo tau duotos priesaikos, kuria mus prisaikdinai”. 
\par 21 Ji tarė: “Tebūna, kaip sakote”. Ji išleido juos, ir jie nuėjo. Tuomet ji pririšo raudoną virvę lange. 
\par 22 Jie nuėjo į kalnus ir ten pasiliko tris dienas, kol sugrįžo tie, kurie juos vijosi. Jie ieškojo jų kelyje, bet nieko nerado. 
\par 23 Tuodu vyrai, grįžę pas Jozuę, Nūno sūnų, jam papasakojo visa, kas jiems atsitiko. 
\par 24 Ir jie sakė: “Viešpats tikrai atidavė mums visą šitą kraštą, nes tos šalies gyventojai labai išsigando dėl mūsų atėjimo”.



\chapter{3}


\par 1 Jozuė, atsikėlęs anksti rytą, su visais izraelitais traukė iš Šitimo; jie pasiekė Jordaną ir ten, prieš pereidami jį, nakvojo. 
\par 2 Trims dienoms praėjus, vyresnieji ėjo per stovyklą 
\par 3 ir įsakė žmonėms: “Kai pamatysite Viešpaties, jūsų Dievo, Sandoros skrynią ir kunigus levitus, ją nešančius, pradėkite žygiuoti paskui ją. 
\par 4 Tačiau tarp jūsų ir jos privalo būti maždaug dviejų tūkstančių uolekčių atstumas. Neprieikite prie jos arti! Niekad nėjote tuo keliu, todėl sekite ją”. 
\par 5 Jozuė tarė žmonėms: “Pašventinkite save, nes rytoj Viešpats darys tarp jūsų stebuklus!” 
\par 6 Po to Jozuė tarė kunigams: “Imkite Sandoros skrynią ir eikite tautos priekyje!” Jie paėmė Sandoros skrynią ir ėjo tautos priekyje. 
\par 7 Viešpats tarė Jozuei: “Šiandien pradedu išaukštinti tave viso Izraelio akyse, kad jie patirtų, jog Aš būsiu su tavimi, kaip buvau su Moze. 
\par 8 Įsakyk kunigams, nešantiems Sandoros skrynią: ‘Kai įbrisite į Jordaną, sustokite jame’ ”. 
\par 9 Po to Jozuė tarė izraelitams: “Ateikite arčiau ir klausykite Viešpaties, jūsų Dievo, žodžių. 
\par 10 Iš to pažinsite, kad gyvasis Dievas yra tarp jūsų ir kad Jis išvarys pirma jūsų kanaaniečius, hetitus, hivus, perizus, girgašus, amoritus ir jebusiečius. 
\par 11 Visos žemės Viešpaties Sandoros skrynia eis per Jordaną pirma jūsų. 
\par 12 Paskirkite dvylika vyrų iš Izraelio giminių, iš kiekvienos po vieną. 
\par 13 Kai tik kunigų, nešančių Viešpaties, visos žemės valdovo, skrynią, kojos palies Jordano vandenį, tekąs žemyn iš aukštumų vanduo sustos kaip pylimas”. 
\par 14 Žmonės išėjo iš palapinių, kad pereitų Jordaną paskui kunigus, nešančius Sandoros skrynią tautos priekyje. 
\par 15 Kai kunigai su Sandoros skrynia įbrido į vandenį (Jordanas buvo patvinęs pjūties metu), 
\par 16 vanduo sustojo tekėjęs. Vanduo, tekąs iš aukštumų, sustojo kaip pylimas prie Adamo miesto, kuris yra šalia Cartano, o vanduo, tekąs Sūriosios jūros link, išseko. Ir tauta perėjo per Jordaną ties Jerichu. 
\par 17 Kunigai, nešusieji Viešpaties Sandoros skrynią, stovėjo ant sausos žemės Jordano viduryje, iki visa tauta sausuma perėjo per Jordaną.



\chapter{4}

\par 1 Visai tautai perėjus per Jordaną, Viešpats tarė Jozuei: 
\par 2 “Paimk iš tautos dvylika vyrų, iš kiekvienos giminės po vieną, 
\par 3 ir jiems įsakyk: ‘Imkite iš Jordano vidurio, iš tos vietos, kur stovėjo kunigų kojos, dvylika akmenų, neškite juos ir padėkite ten, kur šiąnakt nakvosite’ ”. 
\par 4 Jozuė pasišaukė dvylika vyrų, kuriuos buvo paskyręs iš kiekvienos izraelitų giminės. 
\par 5 Ir Jozuė sakė jiems: “Nueikite prie Viešpaties, jūsų Dievo, skrynios į Jordano vidurį ir paimkite po akmenį ant peties, po vieną kiekvienai giminei, 
\par 6 kad tai būtų ženklas tarp jūsų. Kai ateityje jūsų vaikai klaus, ką šitie akmenys reiškia, 
\par 7 jiems atsakykite: ‘Jordano vanduo buvo išdžiūvęs, Viešpaties Sandoros skryniai keliantis per Jordaną’. Šitie akmenys bus atminimas izraelitams per amžius”. 
\par 8 Izraelitai padarė taip, kaip Jozuė įsakė. Jie paėmė dvylika akmenų iš Jordano vidurio, kaip Viešpats įsakė Jozuei, pagal Izraelio giminių skaičių, juos nunešė į nakvynės vietą ir ten padėjo. 
\par 9 Jozuė taip pat sudėjo dvylika akmenų Jordano viduryje, kur stovėjo kunigai, nešantys Sandoros skrynią. Jie ten yra iki šios dienos. 
\par 10 Kunigai, nešantys skrynią, stovėjo Jordano viduryje, iki viskas buvo atlikta, ką Viešpats įsakė Jozuei per Mozę. Tuo metu tauta skubiai perėjo Jordaną. 
\par 11 Kai visa tauta buvo perėjusi, Viešpaties skrynia kartu su kunigais taip pat perėjo. 
\par 12 Rubeno, Gado ir pusė Manaso giminės apsiginklavę perėjo izraelitų priekyje, kaip jiems Mozė liepė. 
\par 13 Apie keturiasdešimt tūkstančių apsiginklavusių ir pasirengusių kautynėms vyrų perėjo Viešpaties akivaizdoje į Jericho lygumas. 
\par 14 Tą dieną Viešpats išaukštino Jozuę viso Izraelio akyse; ir jie bijojo jo, kaip jie bijojo Mozės, visą jo gyvenimą. 
\par 15 Viešpats tarė Jozuei: 
\par 16 “Įsakyk kunigams, nešantiems Liudijimo skrynią, kad jie išeitų iš Jordano”. 
\par 17 Ir Jozuė įsakė kunigams: “Išeikite iš Jordano!” 
\par 18 Vos išėjus kunigams, nešantiems Viešpaties Sandoros skrynią, iš Jordano, vanduo sugrįžo į savo vietą ir išsiliejo, kaip anksčiau, per jo krantus. 
\par 19 Tauta perėjo per Jordaną pirmojo mėnesio dešimtą dieną ir ištiesė palapines Gilgale, Jericho rytų pusėje. 
\par 20 Tuos dvylika akmenų, kuriuos jie paėmė iš Jordano, Jozuė sustatė Gilgale. 
\par 21 Jis kalbėjo izraelitams: “Kai ateityje jūsų vaikai klaus tėvų, ką reiškia šitie akmenys, 
\par 22 jūs paaiškinsite savo vaikams: ‘Sausu dugnu Izraelis perėjo per Jordaną’. 
\par 23 Viešpats, jūsų Dievas, išdžiovino Jordano vandenį prieš jus, kol perėjote, kaip Jis padarė Raudonajai jūrai, išdžiovindamas ją prieš mus, kol perėjome; 
\par 24 kad visos žemės tautos žinotų, jog Viešpats yra galingas, ir kad jūs visuomet bijotumėte Viešpaties, savo Dievo”.



\chapter{5}


\par 1 Visi amoritų karaliai, kurie gyveno nuo Jordano į vakarus, ir visi kanaaniečių karaliai, kurie gyveno prie jūros, išgirdę, jog Viešpats išdžiovino Jordano vandenis prieš izraelitus, kad jie galėtų pereiti, nusiminė ir neteko drąsos. 
\par 2 Tuomet Viešpats tarė Jozuei: “Pasidaryk aštrių peilių ir apipjaustyk izraelitus”. 
\par 3 Jozuė pasidarė aštrius peilius ir apipjaustė izraelitus ant Araloto kalvos. 
\par 4 Štai priežastis, dėl kurios Jozuė atliko apipjaustymą: tautos vyrai, tinkantys karui, išėjus iš Egipto, išmirė pakeliui dykumoje. 
\par 5 Visi išėjusieji buvo apipjaustyti, tačiau vaikai, gimusieji dykumoje, buvo neapipjaustyti. 
\par 6 Keturiasdešimt metų izraelitai klaidžiojo dykumoje, kol išmirė visi karui tinkami vyrai, kurie išėjo iš Egipto, kadangi jie neklausė Viešpaties. Viešpats prisiekė neleisiąs jiems pamatyti žemės, plūstančios pienu ir medumi, kurią pažadėjo jų tėvams. 
\par 7 Jų vaikus, kurie užėmė jų vietą, apipjaustė Jozuė, nes jie nebuvo apipjaustyti kelionėje. 
\par 8 Po apipjaustymo jie pasiliko stovykloje, kol pagijo. 
\par 9 Ir Viešpats tarė Jozuei: “Šiandien Aš pašalinau nuo jūsų Egipto gėdą”. Todėl ta vieta iki šios dienos vadinama Gilgalu. 
\par 10 Izraelitai, stovyklaudami Gilgale, šventė Paschą to mėnesio keturioliktą dieną, vakare, Jericho lygumose. 
\par 11 Kitą dieną po Paschos jie valgė tos žemės derliaus neraugintą duoną ir paskrudintus grūdus. 
\par 12 Jiems pradėjus valgyti tos žemės derlių, mana liovėsi kritusi. Izraelitai nebeturėjo manos, bet tais metais valgė Kanaano krašto vaisius. 
\par 13 Jozuė, būdamas prie Jericho, pamatė prieš save stovintį vyrą, kuris rankoje laikė nuogą kardą. Jozuė priėjo ir jo paklausė: “Ar tu iš mūsų, ar iš mūsų priešų?” 
\par 14 Jis atsakė: “Ne! Aš atėjau kaip Viešpaties pulkų vadas”. Jozuė puolė veidu ant žemės, jį pagarbino ir paklausė: “Ką mano viešpats turi pasakyti savo tarnui?” 
\par 15 Viešpaties pulkų vadas tarė Jozuei: “Nusiauk kurpes, nes vieta, kurioje stovi, yra šventa”. Jozuė taip ir padarė.



\chapter{6}

\par 1 Tuo tarpu Jerichas buvo aklinai užsidaręs dėl izraelitų; niekas neišeidavo ir neįeidavo. 
\par 2 Viešpats tarė Jozuei: “Žiūrėk, Aš atidaviau į tavo rankas Jerichą, jo karalių ir stiprius karo vyrus. 
\par 3 Visi kariai turi apeiti aplink miestą kas dieną po vieną kartą. Taip darykite šešias dienas. 
\par 4 Septyni kunigai turi nešti skrynios priekyje septynis avino rago trimitus. Septintą dieną turite apeiti miestą septynis kartus, kunigams trimituojant. 
\par 5 Išgirdę ilgą trimito garsą, visi žmonės turi garsiai šaukti. Tada sugrius miesto sienos, ir kiekvienas galės įsiveržti į miestą ten, kur stovi”. 
\par 6 Jozuė, Nūno sūnus, pasišaukęs kunigus, tarė: “Paimkite Sandoros skrynią, ir septyni kunigai tegul neša septynis avino rago trimitus Viešpaties skrynios priekyje”. 
\par 7 O žmonėms tarė: “Eikite aplink miestą, o ginkluotieji teeina Viešpaties skrynios priekyje”. 
\par 8 Jozuei tai pasakius, septyni kunigai, trimituodami septyniais avino rago trimitais Viešpaties akivaizdoje, ėjo Viešpaties Sandoros skrynios priekyje. 
\par 9 Ginkluotieji žygiavo priekyje kunigų, kurie pūtė trimitus, o likusieji ėjo paskui skrynią. 
\par 10 Jozuė įsakė tautai: “Jums nevalia šaukti nė kalbėti, iki aš jums pasakysiu: ‘Šaukite!’ Tada jūs turėsite šaukti”. 
\par 11 Viešpaties skrynia buvo apnešta aplink miestą. Apėję aplinkui vieną kartą, jie sugrįžo į stovyklą ir nakvojo joje. 
\par 12 Jozuei atsikėlus anksti rytą, kunigai paėmė Viešpaties Sandoros skrynią, 
\par 13 septyni kunigai septyniais avino rago trimitais trimitavo, eidami Viešpaties skrynios priekyje, ginkluotieji žygiavo pirma jų, o likusieji ėjo paskui Viešpaties skrynią, trimitų garsams skambant. 
\par 14 Antrą dieną jie taip pat apėjo miestą vieną kartą ir sugrįžo į stovyklą. Taip jie darė šešias dienas. 
\par 15 Septintą dieną atsikėlę anksti, auštant, tokiu pat būdu apėjo miestą septynis kartus. 
\par 16 Septintą kartą, kai kunigai pūtė trimitus, Jozuė tarė tautai: “Šaukite! Viešpats jums atidavė miestą! 
\par 17 Miestas ir visa, kas jame yra, bus sunaikinta Viešpaties garbei. Tik paleistuvė Rahaba liks gyva su visais, kurie yra jos namuose, nes ji paslėpė pasiuntinius, kuriuos buvome išsiuntę. 
\par 18 Kas skirta sunaikinti, jūs nepasisavinkite, kad nebūtumėte prakeikti ir taip neužtrauktumėte Izraelio stovyklai prakeikimo ir nelaimės. 
\par 19 Visas sidabras bei auksas ir variniai bei geležiniai indai yra Viešpačiui pašvęsti; visa tai Viešpaties iždui”. 
\par 20 Pučiant trimitams, tauta pradėjo garsiai šaukti, ir sienos sugriuvo. Kariai įsiveržė ir užėmė miestą. 
\par 21 Jie išžudė visus, kas buvo mieste: vyrus ir moteris, jaunus ir senus, jaučius, avis ir asilus. 
\par 22 Vyrams, kurie žvalgė kraštą, Jozuė įsakė: “Eikite į paleistuvės namus, iš jų išveskite moterį ir visus, kurie yra jos namuose, kaip jai prisiekėte”. 
\par 23 Jaunuoliai, buvę žvalgais, išvedė Rahabą, jos tėvą, motiną, brolius ir visus, kurie buvo namuose; jie išvedė visus jos giminaičius ir paliko juos už Izraelio stovyklos ribų. 
\par 24 Miestą ir visa, kas buvo jame, jie sudegino. Tik sidabrą bei auksą ir varinius bei geležinius indus jie padėjo Viešpaties namų iždui. 
\par 25 Paleistuvę Rahabą ir jos tėvo namus bei visus, kurie buvo su ja, Jozuė paliko gyvus. Taip ji liko gyventi tarp Izraelio iki šios dienos, nes paslėpė pasiuntinius, kuriuos Jozuė išsiuntė išžvalgyti Jerichą. 
\par 26 Tuo metu Jozuė paskelbė, prisiekdamas: “Prakeiktas bus Viešpaties akivaizdoje tas, kuris atstatys Jericho miestą! Už pamatus užmokės savo pirmagimiu, o už vartus­jauniausiuoju”. 
\par 27 Viešpats buvo su Jozue, ir garsas apie jį pasklido po visą šalį.



\chapter{7}

\par 1 Izraelitai nusikalto, pasisavindami tai, kas buvo skirta sunaikinti. Achanas, sūnus Karmio, sūnaus Zabdžio, sūnaus Zeracho iš Judo giminės, paėmė iš sunaikinti skirtų daiktų. Dėl to Viešpaties rūstybė užsidegė prieš izraelitus. 
\par 2 Jozuė pasiuntė vyrų iš Jericho į Ają, kuris yra prie Bet Aveno, į rytus nuo Betelio, ir jiems įsakė: “Eikite ir išžvalgykite tą kraštą!” Tie vyrai išžvalgė Ają 
\par 3 ir, sugrįžę pas Jozuę, tarė: “Tegul nežygiuoja visa kariuomenė. Dviejų ar trijų tūkstančių vyrų užteks užimti Ają. Nevargink visų karių, nes jų nėra daug”. 
\par 4 Buvo pasiųsta apie trys tūkstančiai vyrų, bet jie bėgo nuo Ajo vyrų. 
\par 5 Ajo vyrai nužudė apie trisdešimt šešis izraelitus ir vijosi juos žudydami nuo vartų iki Šebarimo. Tada izraelitai nustojo drąsos ir jų širdys tapo kaip vanduo. 
\par 6 Jozuė perplėšė savo drabužius ir puolė veidu žemėn prie Viešpaties skrynios; taip jis ir Izraelio vyresnieji pasiliko ligi vakaro ir užsibarstė dulkių ant galvų. 
\par 7 Jozuė sakė: “Ak, Viešpatie Dieve, kodėl Tu pervedei šitą tautą per Jordaną, kad atiduotum mus amoritams sunaikinti? O, kad mes būtume likę anapus Jordano! 
\par 8 Viešpatie, ką man sakyti, kai Izraelis bėga nuo priešų? 
\par 9 Juk tai išgirdę kanaaniečiai bei visi krašto gyventojai susitelkę apsups mus ir sunaikins. Ką tada darysi dėl savo vardo?” 
\par 10 Viešpats tarė Jozuei: “Atsikelk! Ko guli kniūbsčias ant žemės? 
\par 11 Izraelis nusikalto; jis sulaužė mano sandorą. Jie paėmė dalį to, kas buvo skirta sunaikinti; pavogė, paslėpė ir pasidėjo prie savo daiktų. 
\par 12 Izraelitai negalėjo išstovėti prieš priešą ir bėgo nuo jo, nes susitepė sunaikinimui skirtais daiktais. Aš nebebūsiu su jumis, jei nepašalinsite prakeikimo tarp savųjų. 
\par 13 Kelkis! Pašventink tautą ir pasiruoškite rytojui, nes Aš, Izraelio Dievas, taip sakau: ‘Sunaikinti skirtų daiktų yra tarp jūsų; tu negalėsi išstovėti prieš savo priešus, kol nepašalinsi prakeikimo tarp savųjų’. 
\par 14 Todėl rytoj jūs privalote ateiti giminėmis: ta giminė, kurią Viešpats nurodys, privalo ateiti šeimomis; ta šeima, kurią Viešpats nurodys, privalo priartėti namais; o tų namų, kuriuos Viešpats nurodys, privalo ateiti visi vyrai. 
\par 15 Pas ką bus rasta sunaikinimui skirtų daiktų, tą turite sudeginti su viskuo, kas jam priklauso, nes jis sulaužė Viešpaties sandorą ir padarė niekšybę Izraelyje”. 
\par 16 Jozuė, atsikėlęs anksti rytą liepė priartėti Izraeliui giminėmis, ir buvo nurodyta Judo giminė. 
\par 17 Po to jis liepė priartėti Judo šeimoms, ir Viešpats nurodė Zeracho šeimą. Vėliau jis liepė priartėti Zeracho šeimai namais, ir Viešpats nurodė Zabdį. 
\par 18 Jis liepė priartėti visiems jo namų vyrams, ir buvo nurodytas Zeracho sūnaus, Zabdžio sūnaus, Karmio sūnus Achanas iš Judo giminės. 
\par 19 Jozuė sakė Achanui: “Mano sūnau, atiduok Viešpačiui, Izraelio Dievui, garbę bei prisipažink Jam ir pasakyk man, ką padarei; nieko neslėpk nuo manęs!” 
\par 20 Achanas atsakė Jozuei: “Iš tiesų aš nusidėjau Viešpačiui, Izraelio Dievui, padarydamas tai ir tai. 
\par 21 Pamačiau tarp daiktų prabangų apsiaustą iš Šinaro, du šimtus šekelių sidabro ir aukso gabalą, sveriantį penkiasdešimt šekelių. Aš jų užsigeidžiau ir juos pasiėmiau. Visa užkasiau į žemę, savo palapinės viduje, ir sidabras yra apačioje”. 
\par 22 Jozuė pasiuntė žmones į palapinę. Jie viską rado paslėptą jo palapinėje; sidabras buvo apačioje. 
\par 23 Paėmę tai iš palapinės, jie atnešė pas Jozuę ir izraelitus ir padėjo tai Viešpaties akivaizdoje. 
\par 24 Jozuė ir visi izraelitai su juo paėmė Zeracho sūnų Achaną, sidabrą, apsiaustą ir aukso gabalą, taip pat jo sūnus ir dukteris, jaučius, asilus ir avis, palapinę bei visą jo nuosavybę ir nugabeno į Achoro slėnį. 
\par 25 Jozuė pasakė: “Kodėl užtraukei mums nelaimę? Viešpats šiandien tave padarys nelaimingą”. Tada izraelitai jį užmušė akmenimis ir po to sudegino, kai užmušė juos akmenimis. 
\par 26 Jie sukrovė ant jo didelę akmenų krūvą, kuri išliko iki šios dienos. Taip Viešpaties rūstybė nurimo. Todėl ta vieta ligi šiol vadinama Achoro slėniu.



\chapter{8}

\par 1 Viešpats tarė Jozuei: “Nebijok ir nenusimink! Su visais kariais žygiuok į Ają. Aš atidaviau į tavo rankas Ajo karalių, jo žmones, miestą ir visą kraštą. 
\par 2 Padaryk Ajui bei jo karaliui, kaip padarei Jerichui ir jo karaliui; tik turtą ir gyvulius pasiimk kaip grobį. Įrenk pasalą už miesto”. 
\par 3 Jozuė su kariuomene pradėjo žygį į Ają. Išrinkęs trisdešimt tūkstančių narsių karių, išsiuntė juos nakčia, 
\par 4 sakydamas: “Pasislėpę už miesto, laukite, nenutolkite per daug nuo jo, visi būkite pasiruošę puolimui. 
\par 5 Tuo metu aš su kariais artėsime prie miesto. O kai jie išeis prieš mus kaip pirmą kartą, mes bėgsime nuo jų. 
\par 6 Jie vysis mus ir nutols nuo miesto, nes manys: ‘Jie bėga nuo mūsų kaip pirmą kartą’. 
\par 7 Tada jūs iš pasalų pulkite ir užimkite miestą, nes Viešpats, jūsų Dievas, jį atiduos į jūsų rankas. 
\par 8 Užėmę miestą, padekite jį, kaip Viešpats nurodė; tai mano įsakymas jums”. 
\par 9 Jozuė išsiuntė juos į pasalą tarp Betelio ir Ajo, į vakarus nuo Ajo. Jozuė praleido tą naktį su kariais. 
\par 10 Anksti rytą, patikrinęs karius, Jozuė su Izraelio vyresniaisiais žygiavo karių priekyje į Ają. 
\par 11 Kariuomenė, priartėjus prie Ajo, pasistatė stovyklą į šiaurę nuo jo. Tarp stovyklos ir Ajo buvo slėnis. 
\par 12 Apie penkis tūkstančius vyrų Jozuė pasiuntė į pasalą tarp Betelio ir Ajo, į vakarus nuo miesto. 
\par 13 Kariuomenės pagrindinės jėgos buvo miesto šiaurėje, o pasala­į vakarus nuo jo. Jozuė praleido tą naktį slėnyje. 
\par 14 Ajo karalius, tai pamatęs, skubiai anksti rytą su visais savo kariais žygiavo prieš Izraelį į lygumą, nes nežinojo, kad už miesto yra pasala. 
\par 15 Jozuė su visa Izraelio kariuomene bėgo į dykumą tarsi pralaimėję. 
\par 16 Visi miesto vyrai buvo sušaukti juos persekioti; persekiodami Jozuę, jie nutolo nuo miesto. 
\par 17 Ajo ir Betelio miestuose neliko nė vieno vyro, kuris nebūtų vijęs izraelitų. Jie paliko miestą atvirą ir persekiojo Izraelį. 
\par 18 Tada Viešpats tarė Jozuei: “Ištiesk Ajo link ietį, kurią laikai rankoje, nes į tavo rankas jį atiduosiu!” Jozuė ištiesė Ajo link ietį, kurią laikė rankoje. 
\par 19 Tada buvę pasaloje skubiai puolė Ają. Jie įsiveržė į miestą, jį užėmė ir padegė. 
\par 20 Ajo vyrai, pažvelgę atgal, pamatė degantį miestą. Jiems nebuvo kur bėgti, nes izraelitai, bėgę į dykumą, atsisuko prieš persekiotojus. 
\par 21 Jozuė ir jo kariuomenė, pamatę, kad buvę pasaloje užėmė miestą ir jį padegė, atsigręžė ir puolė Ajo karius. 
\par 22 Kiti puolė juos iš miesto. Taip jie pateko į izraelitų vidurį; vieni buvo iš priekio, kiti­iš užpakalio. Izraelitai juos taip sumušė, kad nė vienas neišliko gyvas. 
\par 23 Ajo karalių jie paėmė gyvą ir atvedė pas Jozuę. 
\par 24 Izraelitai, išžudę visus Ajo vyrus atvirame lauke, dykumoje, kur jie buvo nusiviję izraelitus, sugrįžo į Ają ir išžudė jo gyventojus. 
\par 25 Tą dieną buvo išžudyta dvylika tūkstančių vyrų ir moterų­visi Ajo gyventojai. 
\par 26 Jozuė nenuleido rankos, kuria laikė ietį, kol buvo sunaikinti visi Ajo gyventojai. 
\par 27 Tik gyvulius ir kitą miesto turtą izraelitai pasiėmė, laikydamiesi Viešpaties nurodymų Jozuei. 
\par 28 Jozuė sudegino Ają ir jį pavertė akmenų krūva, dykyne iki šios dienos. 
\par 29 Ajo karalių jis pakorė ant medžio ir laikė iki vakaro; saulei leidžiantis, Jozuei įsakius, jie nuėmė jo lavoną nuo medžio, numetė miesto tarpuvartyje ir ant jo sukrovė didelę akmenų krūvą, tebeesančią iki šios dienos. 
\par 30 Jozuė pastatė aukurą Viešpačiui, Izraelio Dievui, Ebalo kalne, 
\par 31 kaip Viešpaties tarnas Mozė įsakė izraelitams. Tai parašyta Mozės įstatymo knygoje, kad aukuras būtų pastatytas iš netašytų akmenų, kurie nepaliesti jokiu metaliniu įrankiu. Jie aukojo ant jo deginamąsias ir padėkos aukas Viešpačiui. 
\par 32 Jozuė ant tų akmenų įrašė Mozės įstatymą izraelitams matant. 
\par 33 Izraelitai su savo vyresniaisiais, vadais ir teisėjais stovėjo abiejuose skrynios šonuose akivaizdoje levitų kunigų, kurie neša Viešpaties Sandoros skrynią, kaip gimę tarp jų, taip ir ateiviai. Viena jų dalis buvo prie Garizimo, o antroji­prie Ebalo kalno, kaip Viešpaties tarnas Mozė buvo įsakęs laiminti Izraelio tautą. 
\par 34 Po to jis perskaitė visus įstatymo žodžius, palaiminimus ir prakeikimus, visa, kas užrašyta įstatymo knygoje. 
\par 35 Iš visko, ką Mozė įsakė, neliko nė vieno žodžio, kurio Jozuė nebūtų perskaitęs akivaizdoje visų izraelitų: moterų, vaikų ir ateivių, gyvenančių tarp jų.



\chapter{9}

\par 1 Išgirdę apie tai hetitų, amoritų, kanaaniečių, perizų, hivų ir jebusiečių karaliai, gyvenantys anapus Jordano, kalnuose, žemumoje ir palei Didžiosios jūros pakrantę link Libano, 
\par 2 susirinko kartu kovoti prieš Jozuę ir Izraelį. 
\par 3 Gibeoniečiai, išgirdę, ką Jozuė padarė Jerichui ir Ajui, 
\par 4 pasielgė klastingai. Jie pasiėmė maisto senuose maišuose ant asilų, sudriskusių ir apraišiotų vyno odinių, 
\par 5 apsiavė nudėvėta ir sulopyta avalyne, apsivilko nudėvėtais drabužiais; duona, kurią pasiėmė, buvo sudžiūvusi ir supelėjusi. 
\par 6 Atėję pas Jozuę į Gilgalo stovyklą, jie kalbėjo jam ir Izraelio vyrams: “Iš tolimos šalies atvykome su jumis sudaryti taikos sutartį”. 
\par 7 Izraelitai tarė hivams: “Gal jūs gyvenate mūsų žemėje? Kaip mes galėtume su jumis tartis?” 
\par 8 Jie atsakė Jozuei: “Mes esame tavo tarnai”. Jozuė paklausė: “Kas jūs esate ir iš kur atvykote?” 
\par 9 Jie atsakė jam: “Iš labai tolimos šalies atvykome dėl Viešpaties, tavo Dievo, vardo, nes mes girdėjome apie Jį visa, ką Jis padarė Egipte 
\par 10 ir anapus Jordano gyvenusiems amoritų karaliams: Sihonui, Hešbono karaliui, ir Ogui, Bašano karaliui, kuris gyveno Aštarote. 
\par 11 Mūsų vyresnieji ir visi šalies gyventojai patarė mums: ‘Pasiimkite maisto kelionei, eikite jų pasitikti kaip jų tarnai ir prašykite sudaryti taikos sutartį’. 
\par 12 Štai mūsų duona, kurią mes dar šiltą pasiėmėme iš savo namų išvykdami pas jus, dabar sudžiūvusi ir supelėjusi. 
\par 13 Šitos vyno odinės, kai jas prisipylėme, buvo naujos, dabar jos suplyšusios; taip pat mūsų drabužiai ir avalynė nuplyšo dėl tolimos kelionės”. 
\par 14 Izraelitai priėmė jų maistą, nepasiklausę Viešpaties patarimo. 
\par 15 Jozuė sudarė su jais taikos sutartį, pažadėdamas palikti juos gyvus, o izraelitų kunigaikščiai prisiekė jiems. 
\par 16 Praėjus trims dienoms po sutarties sudarymo, jie išgirdo, kad tai yra jų kaimynai, gyveną jų žemėje. 
\par 17 Tada izraelitai iškeliavo ir trečią dieną pasiekė jų miestus: Gibeoną, Kefyrą, Beerotą ir Kirjat Jearimą. 
\par 18 Izraelitai nežudė jų, nes tautos kunigaikščiai jiems buvo prisiekę Viešpačiu, Izraelio Dievu, laikytis sutarties. Izraelitai murmėjo prieš kunigaikščius, 
\par 19 kurie sakė: “Mes jiems prisiekėme Viešpačiu, Izraelio Dievu, todėl dabar negalime jų liesti. 
\par 20 Štai ką jiems padarysime: paliksime juos gyvus, kad Viešpats nebaustų mūsų dėl priesaikos. 
\par 21 Tegul jie lieka gyvi, bet padarykime juos malkų kirtėjais ir vandens nešikais Izraeliui, kaip kunigaikščiai jiems pažadėjo”. 
\par 22 Jozuė, pasišaukęs gibeoniečius, klausė: “Kodėl mums melavote, sakydami: ‘Mes gyvename labai toli nuo jūsų’, kai gyvenate šalia mūsų? 
\par 23 Todėl dabar jūs esate prakeikti ir visą laiką būsite vergais, malkų kirtėjais ir vandens nešikais mano Dievo namams”. 
\par 24 Jie atsakė Jozuei: “Tavo tarnams buvo aiškiai pranešta, kad Viešpats, tavo Dievas, įsakė savo tarnui Mozei jums atiduoti visą šią šalį ir išnaikinti visus šio krašto gyventojus. Mes labai bijojome dėl savo gyvybių ir taip padarėme. 
\par 25 Dabar mes esame tavo rankose; pasielk su mumis, kaip tau atrodo teisinga”. 
\par 26 Jis paliko juos gyvus ir apsaugojo nuo izraelitų, kurie norėjo juos išžudyti. 
\par 27 Jozuė juos padarė malkų kirtėjais ir vandens nešikais Izraeliui ir Viešpaties aukurui iki šios dienos toje vietoje, kurią Viešpats išsirinko.



\chapter{10}


\par 1 Adoni Cedekas, Jeruzalės karalius, išgirdęs, kad Jozuė paėmė Ają ir jį sunaikino, kaip buvo sunaikinęs Jerichą ir jo karalių, ir kad Gibeono gyventojai sudarė taiką su Izraeliu, 
\par 2 nusigando, nes Gibeonas buvo didelis miestas, vienas iš karališkųjų miestų, didesnis už Ają, o visi jo vyrai­karžygiai. 
\par 3 Jeruzalės karalius Adonizedekas siuntė pasiuntinius pas Hebrono karalių Hohamą, Jarmuto karalių Piramą, Lachišo karalių Jafiją ir Eglono karalių Debyrą su tokiu pranešimu: 
\par 4 “Atžygiuokite pas mane ir padėkite man sumušti Gibeoną, nes jis sudarė taiką su Jozue ir izraelitais!” 
\par 5 Penki amoritų karaliai: Jeruzalės, Hebrono, Jarmuto, Lachišo ir Eglono­susivienijo prieš Gibeoną ir su savo kariuomenėmis apgulė jį ir pradėjo karą su juo. 
\par 6 Tada Gibeono vyrai siuntė Jozuei į Gilgalo stovyklą pranešimą: “Nepalik savo tarnų vienų! Skubiai ateik mums padėti ir išgelbėk mus! Prieš mus susirinko visi amoritų karaliai, gyvenantys kalnuose”. 
\par 7 Jozuė su ginkluota kariuomene žygiavo iš Gilgalo. 
\par 8 Viešpats tarė Jozuei: “Nebijok jų, Aš juos atidaviau į tavo rankas! Nė vienas iš jų tau nepasipriešins”. 
\par 9 Jozuė, visą naktį žygiavęs iš Gilgalo, netikėtai juos užklupo. 
\par 10 Viešpats sukėlė paniką tarp jų izraelitų akivaizdoje, ir šie juos stipriai sumušė Gibeone. Jozuė juos vijosi link Bet Horono ir persekiojo ligi Azekos ir Makedos. 
\par 11 Jiems bėgant nuo Izraelio ir esant Bet Horono kalno papėdėje, Viešpats lydino ant jų didelių ledų krušą iki Azekos. Jų žuvo daugiau nuo ledų, negu nuo izraelitų kardo. 
\par 12 Tą dieną, kai Viešpats atidavė amoritus izraelitams, Jozuė šaukėsi Viešpaties izraelitų akivaizdoje ir tarė: “Saule, stovėk Gibeone ir, mėnuli, Ajalono slėnyje!” 
\par 13 Saulė ir mėnulis stovėjo vietoje, kol tauta atkeršijo savo priešams. Argi tai neparašyta Josaro knygoje? Taip saulė sustojo danguje ir neskubėjo nusileisti beveik ištisą dieną. 
\par 14 Niekad daugiau nebuvo tokios dienos, kad Viešpats klausytų žmogaus, nes Jis kariavo Izraelio pusėje. 
\par 15 Po to Jozuė ir visi kariai sugrįžo stovyklon į Gilgalą. 
\par 16 Anie penki karaliai pabėgo ir pasislėpė oloje prie Makedos. 
\par 17 Jozuei buvo pranešta: “Penki karaliai rasti pasislėpę oloje prie Makedos”. 
\par 18 Jozuė įsakė: “Užriskite didžiuliais akmenimis olos angą ir pastatykite vyrus juos saugoti. 
\par 19 O jūs nesustokite! Vykitės priešus ir žudykite atsiliekančius. Neleiskite jiems pasiekti miestų, nes Viešpats, jūsų Dievas, juos atidavė į jūsų rankas”. 
\par 20 Jozuė ir izraelitai juos visiškai sunaikino, tik kai kurie paspruko ir pasislėpė sutvirtintuose miestuose. 
\par 21 Visa kariuomenė saugiai sugrįžo į Jozuės stovyklą Makedoje; niekas nedrįso išsižioti prieš izraelitus. 
\par 22 Jozuė įsakė atidaryti olos angą ir atvesti pas jį tuos penkis karalius. 
\par 23 Jie taip ir padarė: atvedė pas jį Jeruzalės, Hebrono, Jarmuto, Lachišo ir Eglono karalius. 
\par 24 Tada Jozuė sušaukė visus Izraelio karius ir tarė jų vadams, kurie buvo žygiavę su juo: “Priartėkite ir užlipkite šitiems karaliams ant sprandų!” Jie taip ir padarė. 
\par 25 Jozuė jiems tarė: “Nebijokite ir nenusigąskite! Būkite drąsūs ir stiprūs, nes taip Viešpats padarys visiems jūsų priešams, su kuriais kariausite”. 
\par 26 Po to Jozuė juos nužudė ir pakabino ant penkių medžių. Pakabinti jie išbuvo iki vakaro. 
\par 27 Saulei leidžiantis, Jozuei įsakius, jie nuėmė juos nuo medžių ir sumetė į olą, kurioje jie buvo pasislėpę, o olos angą užrito dideliais akmenimis, tebegulinčiais iki šios dienos. 
\par 28 Jozuė užėmė tą pačią dieną Makedą ir nužudė jos gyventojus bei karalių, sunaikino visus gyventojus taip, kad nė vienas neišliko gyvas. Jis pasielgė su Makedos karaliumi taip, kaip su Jericho karaliumi. 
\par 29 Jozuė ir visa Izraelio kariuomenė nužygiavo iš Makedos į Libną ir puolė ją. 
\par 30 Viešpats ją ir jos karalių atidavė į rankas izraelitų, kurie išžudė visus gyventojus kardu. Jozuė padarė jos karaliui taip, kaip buvo padaręs Jericho karaliui. 
\par 31 Jozuė ir jo kariuomenė nužygiavo iš Libnos į Lachišą, apsupo ir puolė jį. 
\par 32 Viešpats atidavė ir Lachišą į Izraelio rankas, kuris antrą dieną paėmė jį, išžudė kardu visus jo gyventojus taip, kaip padarė Libnoje. 
\par 33 Tuo metu Gezero karalius Horamas atžygiavo padėti Lachišui. Jozuė sumušė jį ir jo karius taip, kad nebeliko nė vieno gyvo. 
\par 34 Vėliau Jozuė ir visa Izraelio kariuomenė nužygiavo iš Lachišo į Egloną, apsupo ir puolė jį. 
\par 35 Tą pačią dieną jį užėmė. Jie išžudė kardu visus jo gyventojus ir sunaikino taip, kaip Lachišą. 
\par 36 Jozuė ir visa Izraelio kariuomenė žygiavo iš Eglono į Hebroną ir puolė jį. 
\par 37 Jie paėmė jį, nužudė kardu karalių, paėmė visus jo miestus ir jų gyventojus išžudė. Jie padarė su juo taip, kaip su Eglonu. 
\par 38 Jozuė ir visa Izraelio kariuomenė sugrįžo prie Debyro ir puolė jį. 
\par 39 Paėmė jį, jo karalių ir visus jo miestus. Izraelitai išžudė visus gyventojus, nepaliko nė vieno gyvo. Kaip jie padarė Hebronui, Libnai ir jų karaliams, taip jie padarė Debyrui ir jo karaliui. 
\par 40 Taip Jozuė nugalėjo visas kalnų, pietų, žemumų ir šlaitų šalis ir visus jų karalius; nė vieno jų nepaliko gyvo, bet visa, kas kvėpuoja, sunaikino, kaip įsakė Viešpats, Izraelio Dievas. 
\par 41 Jozuė nugalėjo juos nuo Kadeš Barnėjos iki Gazos ir visą Gošeno šalį iki Gibeono. 
\par 42 Visus šituos karalius ir jų šalis Jozuė paėmė vienu metu, nes Viešpats, Izraelio Dievas, kariavo už Izraelį. 
\par 43 Pagaliau Jozuė ir su juo visas Izraelis sugrįžo stovyklon į Gilgalą.



\chapter{11}

\par 1 Tai išgirdęs, Hacoro karalius Jabinas siuntė pranešimą Jobabui, Madono karaliui, taip pat Šimrono ir Achšafo karaliams, 
\par 2 šiaurėje kalnuose gyvenantiems karaliams, ir karaliams, gyvenantiems lygumoje į pietus nuo Kinereto ežero, slėnyje ir Doro apylinkėse vakaruose. 
\par 3 Taip pat kanaaniečiams rytuose ir vakaruose, amoritams, hetitams, perizams, jebusiečiams kalnuose ir hivams Hermono kalno papėdėje, Micpos šalyje. 
\par 4 Jie išėjo ir visos jų kariuomenės su jais; jų buvo kaip smilčių jūros krante; taip pat daugybė žirgų ir kovos vežimų. 
\par 5 Visi šitie karaliai atžygiavę pasistatė stovyklas prie Meromo vandenų ir pasiruošė kariauti su Izraeliu. 
\par 6 Viešpats tarė Jozuei: “Nebijok jų, rytoj apie šitą laiką jie visi gulės izraelitų nukauti. Jų žirgams pakirsk kojas, kovos vežimus sudegink”. 
\par 7 Jozuė su savo kariais staiga juos puolė prie Meromo vandenų. 
\par 8 Viešpats juos atidavė į Izraelio rankas; jie sumušė juos ir vijosi iki didžiojo Sidono ir iki Misrefot Maimo, ir iki Mispos slėnio rytuose. Jie naikino juos, kol nė vieno nebeliko. 
\par 9 Jozuė padarė taip, kaip jam liepė Viešpats: žirgams jis pakirto kojas, o kovos vežimus sudegino. 
\par 10 Tuomet Jozuė grįždamas paėmė Asorą ir jo karalių nužudė kardu. Asoras buvo galingiausia iš šitų karalysčių. 
\par 11 Izraelitai išžudė kardu visus gyventojus taip, kad nė vienas neliko gyvas, o patį Asorą sudegino. 
\par 12 Visus tų karalių miestus Jozuė užėmęs sudegino, karalius išžudė kardu; juos sunaikino, kaip buvo įsakęs Viešpaties tarnas Mozė. 
\par 13 Tačiau miestų, buvusių kalnuose, Izraelis nesudegino; tik vieną Asorą. 
\par 14 Visą šių miestų turtą ir gyvulius izraelitai pasiėmė; bet visus žmones jie išžudė, nepalikdami nė vieno gyvo. 
\par 15 Kaip Viešpats įsakė savo tarnui Mozei, o Mozė įsakė Jozuei, šis viską įvykdė. Jis nieko nepaliko nepadaryta, ką Viešpats įsakė Mozei. 
\par 16 Jozuė paėmė visą šalį: kalnyną, pietų kraštą, Gošeno šalį, slėnį, lygumą, Izraelio kalnyną ir jo slėnį; 
\par 17 nuo Halako kalnų, kylančių Seyro link, iki Baal Gado Libano slėnyje, Hermono kalno papėdėje. Jis nugalėjo visus jų karalius ir juos išžudė. 
\par 18 Jozuė ilgai kariavo prieš visus šituos karalius. 
\par 19 Nė vienas miestas nepadarė taikos su izraelitais, išskyrus Gibeoną, hevitų miestą; visus kitus jie paėmė mūšyje. 
\par 20 Viešpats užkietino jų širdis, kad jie pasitiktų Izraelį kardu, ir jis juos sunaikintų be pasigailėjimo, kaip Viešpats įsakė Mozei. 
\par 21 Jozuė atžygiavęs puolė anakiečius, gyvenančius Hebrono, Debyro, Anabo, Judo ir Izraelio kalnuose ir visai sunaikino juos ir jų miestus. 
\par 22 Izraelyje nebeliko anakiečių; jų išliko tik Gazoje, Gate ir Asdode. 
\par 23 Taip Jozuė užėmė visą šalį, kaip Viešpats įsakė Mozei. Jis atidavė ją paveldėti Izraeliui, paskirstydamas giminėmis. Kraštas ilsėjosi nuo karų.



\chapter{12}


\par 1 Šitie yra krašto karaliai, kuriuos izraelitai nugalėjo ir užėmė jų žemes anapus Jordano, rytuose nuo Arnono upės iki Hermono kalno visoje rytų lygumoje. 
\par 2 Amoritų karalius Sihonas, gyvenęs Hešbone, valdęs sritį nuo Aroero, Arnono upės pakrantėje, nuo upės vidurio, ir pusę Gileado iki Jaboko upelio, amonitų sienos, 
\par 3 Arabą iki Kenereto ežero rytinėje pusėje, iki Sūriosios jūros rytinėje pusėje Bet Ješimoto link ir pietuose iki Pisgos šlaitų. 
\par 4 Bašano karalius Ogas, iš milžinų palikuonių, gyvenęs Aštarote bei Edrėjyje 
\par 5 ir valdęs Hermono kalnyną, Salchą, visą Bašaną iki gešūriečių ir maakų krašto ir pusę Gileado iki Hešbono karaliaus Sihono sienos. 
\par 6 Viešpaties tarnas Mozė su izraelitais nugalėjo juos ir atidavė nuosavybėn Rubeno, Gado ir pusei Manaso giminės. 
\par 7 Šitie yra šalies karaliai, kuriuos Jozuė su izraelitais nugalėjo vakarinėje Jordano pusėje, nuo Baal Gado Libano slėnyje iki kalnų, kylančių Seyro link. Jozuė atidavė jų žemes Izraelio nuosavybėn, paskirstydamas jas giminėms 
\par 8 kalnyne, slėnyje, lygumoje, šlaituose, dykumoje. Tai karaliai hetitų, amoritų, kanaaniečių, perizų, hivų ir jebusiečių: 
\par 9 Jericho karalius, šalia Betelio esančio Ajo karalius, 
\par 10 Jeruzalės karalius, Hebrono karalius, 
\par 11 Jarmuto karalius, Lachišo karalius, 
\par 12 Eglono karalius, Gezero karalius, 
\par 13 Debyro karalius, Gedero karalius, 
\par 14 Hormos karalius, Arado karalius, 
\par 15 Libnos karalius, Adulamo karalius, 
\par 16 Makedos karalius, Betelio karalius, 
\par 17 Tapuacho karalius, Hefero karalius, 
\par 18 Afeko karalius, Lašarono karalius, 
\par 19 Madono karalius, Asoro karalius, 
\par 20 Šimron Merono karalius, Achšafo karalius, 
\par 21 Taanacho karalius, Megido karalius, 
\par 22 Kedešo karalius, Jokneamo karalius Karmelyje, 
\par 23 Doro karalius Nafatdore, Goimo karalius Gilgaloje, 
\par 24 Tircos karalius. Iš viso trisdešimt vienas karalius.



\chapter{13}

\par 1 Jozuei sulaukus senatvės,Viešpats tarė: “Tu pasenai, sulaukei ilgo amžiaus, o dar liko daug žemių užimti: 
\par 2 visas filistinų ir gešūriečių kraštas 
\par 3 nuo Šihoro Egipto rytuose iki Ekrono šiaurėje, kanaaniečių žemė, valdoma penkių filistinų kunigaikščių: Gazos, Ašdodo, Aškelono, Gato ir Ekrono, taip pat ir avai. 
\par 4 Į pietus visa kanaaniečių šalis nuo sidoniečiams priklausančios Mearos iki Afeko, amoritų sienos; 
\par 5 gebaliečių kraštas ir visas rytų Libanas nuo Hermono kalno papėdėje esančio Baal Gado iki Emato; 
\par 6 visi kalnų gyventojai nuo Libano iki Misrefot Maimo ir sidoniečiai. Aš juos pašalinsiu iš izraelitų akivaizdos, tik padalyk kraštą burtų keliu izraelitams paveldėjimui, kaip tau įsakiau. 
\par 7 Padalyk kraštą devynioms giminėms ir pusei Manaso giminės”. 
\par 8 Kita pusė Manaso giminės, kartu su Rubeno ir Gado giminėmis jau gavo paveldėjimą, kurį jiems davė Viešpaties tarnas Mozė Jordano rytinėje pusėje: 
\par 9 nuo Aroero, esančio Arnono upės pakrantėje, miestą upės viduryje, visą Medebos lygumą iki Dibono, 
\par 10 visus miestus amoritų karaliaus Sihono, kuris karaliavo Hešbone, iki amonitų sienos, 
\par 11 Gileadą, gešūriečių ir maakų kraštą, visą Hermono kalnyną, Bašaną iki Salchos, 
\par 12 visą Ogo, kuris karaliavo Aštarote ir Edrėjyje, karalystę Bašane. Jis buvo iš milžinų palikuonių, kuriuos Mozė nugalėjo ir išvarė. 
\par 13 Tačiau Izraelio vaikai neišvarė gešūriečių ir maakų, kurie liko gyventi tarp Izraelio iki šios dienos. 
\par 14 Tik Levio giminei Mozė nedavė dalies paveldėjimui, nes jų dalis yra Viešpats, Izraelio Dievas, kaip Jis sakė. 
\par 15 Mozė padalino žemes Rubeno giminei pagal jų šeimas. 
\par 16 Jiems teko sritis nuo Aroero miesto, esančio ant Arnono upės kranto, miestas upės viduryje, visa lyguma iki Medebos, 
\par 17 Hešbonas ir visi lygumos miestai: Dibonas, Bamot Baalas, Bet Baal Meonas, 
\par 18 Jahacas, Kedemotas, Mefaatas, 
\par 19 Kirjataimas, Sibma, Ceret Šaharas lygumos kalne, 
\par 20 Bet Peoras, Pisgos šlaitai ir Bet Ješimotas. 
\par 21 Tai karalystė amoritų karaliaus Sihono, kuris karaliavo Hešbone. Mozė užkariavo tuos miestus ir nužudė midjaniečių vadus: Evį, Rekemą, Cūrą, Hūrą ir Rebą, Sihono kunigaikščius, gyvenusius krašte; 
\par 22 ir žynį Balaamą, Beoro sūnų, izraelitai nužudė kardu. 
\par 23 Rubenitų siena buvo Jordanas su jam priklausančia pakrančių sritimi. Tai buvo rubenitų paveldėjimas. 
\par 24 Mozė davė Gado giminei paveldėjimą pagal jų šeimas. 
\par 25 Jos gavo Jazerą ir visus Gileado miestus, pusę amoritų šalies iki Aroero, į rytus nuo Rabos, 
\par 26 nuo Hešbono iki Ramat Micpės ir Betonimo, nuo Machanaimo iki Lo Debaro, 
\par 27 o slėnyje Bet Haramą, Bet Nimrą, Sukotą ir Cafoną, Hešbono karaliaus Sihono karalystės likusią dalį. Siena buvo Jordanas iki Kinereto ežero rytinėje Jordano pusėje. 
\par 28 Tie miestai ir kaimai buvo duoti atskiroms Gado šeimoms paveldėti. 
\par 29 Mozė davė paveldėjimą pusei Manaso giminės; jis teko atskiroms Manaso giminės šeimoms. 
\par 30 Jų sritis tęsėsi nuo Machanaimo, apėmė visą Bašaną, karaliaus Ogo karalystę ir visus Jayro miestus ir kaimus, esančius Bašane. Iš viso šešiasdešimt miestų. 
\par 31 Pusė Gileado bei Ogo karalystės miestai Bašane Aštarotas ir Edrėjas teko Manaso sūnaus Machyro vaikams. 
\par 32 Šitas žemes Mozė padalino paveldėjimui Moabo lygumoje, rytinėje Jordano pusėje, ties Jerichu. 
\par 33 Levio giminei Mozė nedavė jokio paveldėjimo: Viešpats, Izraelio Dievas, yra jų paveldėjimas, kaip Jis jiems pasakė.



\chapter{14}


\par 1 Šitos yra žemės, kurias gavo paveldėti izraelitai Kanaano šalyje, kaip jiems paskyrė kunigas Eleazaras, Nūno sūnus Jozuė ir izraelitų giminių vyresnieji. 
\par 2 Jie burtų keliu gavo žemes paveldėti, kaip Viešpats buvo įsakęs Mozei padalinti jas devynioms ir pusei giminės. 
\par 3 Mozė davė dalį dviem ir pusei giminės rytinėje Jordano pusėje, o levitams nedavė jokio paveldėjimo. 
\par 4 Juozapo buvo dvi giminės: Manaso ir Efraimo. Levitai negavo kitos dalies, kaip tik miestus apsigyventi ir ganyklas gyvuliams. 
\par 5 Izraelitai, paskirstydami žemę, padarė, kaip Viešpats įsakė Mozei. 
\par 6 Judo giminės vyresnieji atėjo pas Jozuę į Gilgalą; Jefunės sūnus Kalebas, kenazas, tarė jam: “Tu žinai, ką Viešpats kalbėjo Dievo tarnui Mozei apie tave ir mane Kadeš Barnėjoje. 
\par 7 Aš buvau keturiasdešimties metų amžiaus, kai Viešpaties tarnas Mozė mane pasiuntė iš Kadeš Barnėjos išžvalgyti šalį, ir aš jam viską pranešiau, kas buvo mano širdyje. 
\par 8 Nors mano broliai, kurie ėjo su manimi, išgąsdino tautą, tačiau aš iki galo sekiau Viešpačiu, savo Dievu. 
\par 9 Tą dieną Mozė prisiekė: ‘Tikrai žemę, kurią mindžiojo tavo koja, paveldėsi tu ir tavo vaikai amžiams, nes tu iki galo sekei Viešpačiu, savo Dievu’. 
\par 10 Taigi Viešpats išlaikė mane gyvą, kaip Jis pažadėjo, keturiasdešimt penkerius metus nuo to laiko, kai Viešpats, Izraeliui klaidžiojant dykumoje, kalbėjo tai Mozei. Šiandien aš esu aštuoniasdešimt penkerių metų amžiaus. 
\par 11 Aš ir šiandien dar esu toks tvirtas, kaip tą dieną, kai Mozė mane siuntė išžvalgyti šalį. Vis dar esu stiprus kariauti ir vadovauti. 
\par 12 Taigi dabar duok man šitą kalną, apie kurį Viešpats kalbėjo. Juk tu tada girdėjai, kad ten gyvena anakiečiai ir turi didelių, sutvirtintų miestų. Gal Viešpats bus su manimi, ir aš nugalėsiu juos”. 
\par 13 Jozuė palaimino Kalebą ir davė jam paveldėti Hebroną. 
\par 14 Taip Hebroną paveldėjo Jefunės sūnus Kalebas, kenazas, iki šios dienos, nes jis iki galo sekė Viešpačiu, Izraelio Dievu. 
\par 15 Hebronas anksčiau vadinosi Kirjat Arba. (Arba buvo žymus žmogus tarp anakiečių.) Kraštas susilaukė ramybės.



\chapter{15}

\par 1 Atskiros Judo giminės šeimos burtų keliu gavo žemes iki Cino dykumos, siekiančias Edomą pietuose. 
\par 2 Jų siena prasideda nuo Sūriosios jūros į pietus, 
\par 3 nuo Akrabimo kalvų tęsiasi iki Cino pietuose, pro Kadeš Barnėją į Hecroną, pakyla į Adarą ir pasisuka į Karką. 
\par 4 Iš čia į Acmoną ir pasiekia Egipto upę, ja siena pasiekia jūrą. Šita yra pietinė siena. 
\par 5 O rytinė siena­Sūrioji jūra iki Jordano žiočių. Šiaurinė siena prasideda nuo jūros įlankos ir Jordano žiočių. 
\par 6 Paskui siena pakyla į Bet Hoglą, o nuo Bet Arabos tęsiasi į šiaurę. Toliau siena siekia Rubeno sūnaus Bohano akmenį. 
\par 7 Iš Achoro slėnio kyla į Debyrą, nukrypdama Gilgalos link priešais Adumimo pakilimą, kuris yra į pietus nuo slėnio, ir eina į Saulės šaltinius iki En Rogelio versmių. 
\par 8 Iš ten siena tęsiasi į Hinomo slėnį pro jebusiečių miestą Jeruzalę. Nuo čia ji kyla į viršūnę kalno, kuris yra į vakarus nuo Hinomo slėnio, gale Refajų slėnio šiaurėje, 
\par 9 ir pasisuka nuo kalno viršūnės į Neftoachą­Vandenų versmę, ir prieina prie Efrono kalno miestų, paskui siena pasisuka į Baalą­ Kirjat Jearimą 
\par 10 ir nuo Baalo į vakarus į Seyro kalnyną; toliau į šiaurinį Jearimo kalnų šlaitą­Chesaloną, nusileidžia į Bet Šemešą ir tęsiasi iki Timnos. 
\par 11 Siena, pasiekusi šiaurinio Ekrono ribas, pasisuka į Šikaroną, tęsiasi iki Baalo kalno, prieina prie Jabneelio ir baigiasi prie jūros. 
\par 12 Vakarinė siena buvo Didžioji jūra. Tai yra Judo giminių teritorija. 
\par 13 Jefunės sūnui Kalebui jis davė dalį tarp Judo vaikų, kaip Viešpats įsakė Jozuei, Arbos, Anako tėvo, miestą, kuris yra Hebronas. 
\par 14 Kalebas išvarė iš ten tris Anako sūnus: Šešają, Ahimaną ir Talmają. 
\par 15 Iš ten Kalebas traukė prieš Debyrą; Debyras anksčiau vadinosi Kirjat Seferu. 
\par 16 Kalebas tarė: “Kas užims Kirjat Seferą, tam duosiu savo dukterį Achsą į žmonas”. 
\par 17 Kenazo, Kalebo brolio, sūnus Otnielis, jį užėmė ir gavo Kalebo dukterį Achsą. 
\par 18 Kai ji ištekėjo už jo, jis prikalbėjo ją prašyti savo tėvo dirbamos žemės. Jai atvykus ir nulipus nuo asilo, tėvas klausė: “Ko norėtum?” 
\par 19 Ji atsakė: “Tėve, palaimink mane! Tu davei man pietų žemės, duok man ir vandens versmių!” Jis davė jai aukštutines ir žemutines versmes. 
\par 20 Šitas yra Judo giminės atskirų šeimų paveldėjimas. 
\par 21 Judo giminė savo žemėse pietuose, Edomo link, paveldėjo šiuos miestus: Kabceelį, Ederą, Jagūrą, 
\par 22 Kiną, Dimoną, Adadą, 
\par 23 Kedešą, Hacorą, Itnaną, 
\par 24 Zifą, Telemą, Bealotą, 
\par 25 Hacor Hadatą, Kerijot Hecroną­Hacorą, 
\par 26 Amamą, Šemą, Moladą, 
\par 27 Hacar Gadą, Hešmoną, Bet Peletą, 
\par 28 Hacar Šualą, Beer Šebą, Biziotiją, 
\par 29 Baalą, Jimą, Ezemą, 
\par 30 Eltoladą, Kesilą, Hormą, 
\par 31 Ciklagą, Madmaną, Sansaną, 
\par 32 Lebaotą, Šilhimą, Ainą ir Rimoną; iš viso dvidešimt devynis miestus su jų kaimais. 
\par 33 Slėnyje: Eštaolą, Corą, Ašną, 
\par 34 Zanoachą, En Ganimą, Tapuachą, Enamą, 
\par 35 Jarmutą, Adulamą, Sochoją, Azeką, 
\par 36 Šaaraimą, Aditaimą, Gederą ir Gederotaimą; iš viso keturiolika miestų su jų kaimais. 
\par 37 Cenaną, Hadašą, Migdal Gadą, 
\par 38 Dilaną, Micpę, Jokteelį, 
\par 39 Lachišą, Bockatą, Egloną, 
\par 40 Kaboną, Lachmasą, Kitlišą, 
\par 41 Gederotą, Bet Dagoną, Naamą ir Makedą; iš viso šešiolika miestų su jų kaimais. 
\par 42 Libną, Eterą, Ašaną, 
\par 43 Iftachą, Ašną, Necibą, 
\par 44 Keilą, Achzibą ir Marešą; iš viso devynis miestus su jų kaimais. 
\par 45 Ekroną ir jo miestus bei kaimus 
\par 46 nuo Ekrono iki jūros: visa, kas yra Ašdodo apylinkėje. 
\par 47 Ašdodą, jo miestus ir kaimus; Gazą, jos miestus ir kaimus iki Egipto upės ir Didžiosios jūros kranto. 
\par 48 O kalnyne: Šamyrą, Jatyrą, Sochoją, 
\par 49 Daną, Kirjat Saną­Debyrą, 
\par 50 Anabą, Eštemoją, Animą, 
\par 51 Gošeną, Holoną ir Giloją; iš viso vienuolika miestų su jų kaimais. 
\par 52 Arabą, Dūmą, Esaną, 
\par 53 Janumą, Bet Tapuchą, Afeką, 
\par 54 Humtą, Kirjat Arbą­Hebroną, Ciorą; iš viso devynis miestus su jų kaimais. 
\par 55 Maoną, Karmelį, Zifą, Jutą, 
\par 56 Jezreelį, Jorkoamą, Zanoachą, 
\par 57 Kainą, Gibėją ir Timną; iš viso dešimt miestų su jų kaimais. 
\par 58 Halhulą, Bet Cūrą, Gedorą, 
\par 59 Maaratą, Bet Anotą ir Eltekoną; iš viso šešis miestus su jų kaimais. 
\par 60 Kirjat Baalą­Kirjat Jearimą ir Rabą; iš viso du miestus su jų kaimais. 
\par 61 Dykumoje: Bet Arabą, Midiną, Sechachą, 
\par 62 Nibšaną, Druskos miestą ir En Gedį; iš viso šešis miestus su jų kaimais. 
\par 63 Tačiau jebusiečių, Jeruzalės gyventojų, Judo giminė neįstengė išvaryti; jebusiečiai liko gyventi Jeruzalėje iki šios dienos.



\chapter{16}


\par 1 Juozapo sūnų burtų keliu gautos žemės prasidėjo nuo Jordano ties Jerichu; jų siena ėjo Jericho šaltinių link, rytuose toliau į dykumą, pakilo iš Jericho į kalnus, į Betelį; 
\par 2 iš Betelio į Lūzą iki archų miesto Ataroto, 
\par 3 leidosi į vakarus ir pasiekė jafletų sieną prie Žemutinio Bet Horono; iš čisa iki Gezero ir pasiekė jūrą. 
\par 4 Juozapo sūnūs Manasas ir Efraimas paveldėjo šitas žemes. 
\par 5 Efraimo giminės ribos buvo tokios: jų paveldėjimo siena rytuose buvo Atrot Adaras iki Aukštutinio Bet Horono; 
\par 6 iš ten ji ėjo į vakarus; Michmetatas paliko šiaurėje; toliau siena pasisuko į rytus, į Taanat Šiloją; iš ten, jo rytų pusėje, į Janoachą; 
\par 7 iš Janoacho į Atarotą bei Naaratą, pasiekė Jerichą ir iš čia į Jordaną. 
\par 8 Iš Tapuacho siena ėjo į vakarus, Kanos slėnį, ir toliau pasiekė jūrą. Tai buvo Efraimo giminės atskirų šeimų paveldėjimas. 
\par 9 Be to, dar joms priklausė miestų su jų kaimais Manaso giminės žemėse. 
\par 10 Tačiau jie neįstengė išvaryti kanaaniečių, gyvenusių Gezeryje; jie liko gyventi Efraimo žemėse iki šios dienos, mokėdami jiems duoklę.



\chapter{17}


\par 1 Manaso giminė taip pat gavo paveldėjimą, nes jis buvo Juozapo pirmagimis. Manaso pirmagimiui Machyrui, Gileado tėvui, teko Gileadas bei Bašanas, mat, jis buvo karys. 
\par 2 Taip pat dalį gavo likusios Manaso sūnų atskiros šeimos: Abiezero, Heleko, Asrielio, Šechemo, Hefero ir Šemido. Šitos buvo Juozapo sūnaus Manaso sūnų šeimos. 
\par 3 Manaso sūnaus Machyro sūnaus Gileado sūnaus Hefero sūnus Celofhadas neturėjo sūnų, tik dukteris: Machlą, Noją, Hoglą, Milką ir Tircą. 
\par 4 Jos, atėjusios pas kunigą Eleazarą, pas Nūno sūnų Jozuę ir giminių vyresniuosius, tarė: “Viešpats įsakė Mozei duoti mums dalį tarp mūsų brolių”. Tada jis, kaip Viešpats buvo nurodęs, davė ir joms dalį tarp jų tėvo brolių. 
\par 5 Tuo būdu Manasui teko dešimt dalių, neskaičiuojant Gileado ir Bašano žemių anapus Jordano, 
\par 6 nes Manaso dukterys paveldėjo dalį tarp jo sūnų. Gileado šalį gavo kiti Manaso sūnūs. 
\par 7 Manaso siena tęsėsi nuo Ašero iki Michmeto, esančio prie Sichemo. Toliau siena ėjo pietų link pro En Tapuacho versmes. 
\par 8 Manasui priklausė Tapuacho kraštas, tačiau pats Tapuacho miestas Manaso pasienyje priklausė Efraimui. 
\par 9 Toliau siena nusileido į Kanos upelio slėnį. Į pietus nuo slėnio buvo miestų, kurie įėjo į Manaso miestų skaičių, tačiau priklausė Efraimui. Iš ten Manaso siena ėjo šiaurine slėnio puse ir pasiekė jūrą. 
\par 10 Pietinė dalis priklausė Efraimui, o šiaurinė­Manasui. Jų siena vakaruose buvo jūra. Su Ašeru jie susisiekė šiaurėje, o su Isacharu­rytuose. 
\par 11 Be to, Manasui dar priklausė Isachare ir Ašere Bet Šeanas ir jo kaimai, Ibleamas ir jo kaimai, Doras ir jo kaimai, En Doras ir jo kaimai, Taanachas ir jo kaimai, taip pat Megidas ir jo kaimai. 
\par 12 Tačiau Manasas neįstengė užimti šitų miestų; kanaaniečiai toliau gyveno krašte. 
\par 13 Kai izraelitai sustiprėjo, jie pavergė kanaaniečius, tačiau jų neišvarė. 
\par 14 Juozapo palikuonys kalbėjo Jozuei: “Kodėl davei mums paveldėti tik vieną dalį žemės? Juk mes esame gausi tauta, Viešpats mus laimina”. 
\par 15 Jozuė jiems atsakė: “Jei esate gausi tauta ir jei Efraimo kalnynas jums yra per ankštas, tai eikite į mišką ir ten įsikurkite perizų ir milžinų sūnų krašte”. 
\par 16 Juozapo palikuonys tarė: “Kalnyno mums neužtenka, bet kanaaniečiai, kurie gyvena lygumose, turi geležinių kovos vežimų. Jų turi ir tie, kurie gyvena Bet Šeano kaimuose ir miestuose bei Jezreelio lygumose”. 
\par 17 Tada Jozuė kalbėjo Juozapo giminėms, Efraimui ir Manasui: “Esate gausi ir tvirta tauta. Jums neužtenka vienos dalies. 
\par 18 Jums teks kalnynas; nors jis yra apaugęs mišku, iškirskite jį ir gyvenkite ten. Jūs nugalėsite kanaaniečius, nors jie turi geležinių kovos vežimų ir yra stipri tauta”.



\chapter{18}

\par 1 Visas Izraelis susirinko Šilojuje ir ten pastatė Susitikimo palapinę; visa šalis jau buvo jų užimta. 
\par 2 Septynios izraelitų giminės dar nebuvo gavusios savo dalies. 
\par 3 Jozuė tarė izraelitams: “Kaip ilgai delsite užimti kraštą, kurį jums davė Viešpats, jūsų tėvų Dievas? 
\par 4 Paskirkite po tris vyrus iš kiekvienos giminės, aš juos išsiųsiu, kad jie pereitų visą kraštą ir jį aprašytų taip, kad kiekviena giminė gautų savo paveldėjimą. Po to jie tegul grįžta pas mane. 
\par 5 Kraštą tegul jie padalina į septynias dalis. Judas tepasilieka savo vietoje pietuose, o Juozapas­šiaurėje. 
\par 6 Jūs aprašykite krašto septynias dalis ir atneškite tai man. Tada aš mesiu dėl jūsų burtą Viešpaties, mūsų Dievo, akivaizdoje. 
\par 7 Levitai neturi dalies tarp jūsų, nes Viešpaties kunigystė yra jų dalis. Gadas, Rubenas ir pusė Manaso giminės jau gavo savo dalį anapus Jordano rytuose, kurią jiems davė Viešpaties tarnas Mozė”. 
\par 8 Tie vyrai pakilo eiti, o Jozuė jiems įsakė: “Eikite ir apžiūrėkite visą kraštą, jį aprašykite ir sugrįžkite pas mane! O aš čia, Sile, mesiu burtą Viešpaties akivaizdoje”. 
\par 9 Jie apėjo visą kraštą, jį aprašė knygose, pažymėdami miestus, ir suskirstė kraštą į septynias dalis. Po to grįžo pas Jozuę į Šilojo stovyklą. 
\par 10 Jozuė metė burtą Šilojuje Viešpaties akivaizdoje ir ten padalino šalį izraelitams, paskirdamas kiekvienai giminei jos dalį. 
\par 11 Benjamino giminės šeimos burtų keliu gavo žemę tarp Judo ir Juozapo giminių. 
\par 12 Jų siena šiaurėje prasidėjo Jordanu; toliau ėjo Jericho šiaurinėje pusėje į kalnus vakarų link ir baigėsi Bet Aveno dykuma. 
\par 13 Iš ten ji ėjo į Lūzą, kuri yra Betelis, pasisukusi pietų link. Toliau siena leidosi link Atrot Adaro, arti kalno, kuris yra į pietus nuo Žemutiniojo Bet Horono. 
\par 14 Nuo čia siena pakeitė kryptį, pasisukdama vakarinėje kalno dalyje į pietus, prieš Bet Horoną, ir ėjo iki Kirjat Baalo­Kirjat Jarimo, Judo miesto. Tai yra vakarinė siena. 
\par 15 Pietuose ji prasidėjo nuo Kirjat Jarimo ir tęsėsi į vakarus, ir pasiekė Neftoacho šaltinį. 
\par 16 Toliau siena leidosi kalno pakraščiu prie Ben Hinomo slėnio, esančio Refajų lygumos šiaurėje. Nusileidusi į Ben Hinomo slėnį, tuo slėniu pro jebusiečių kalnyną pietuose nuėjo žemyn iki En Rogelio. 
\par 17 Toliau ji pasisuko į šiaurę, pasiekė Saulės versmes, o iš čia ėjo į Gelilotą, kuris yra priešais Adumimo pakilimą, iki Rubeno sūnaus Bohano akmens. 
\par 18 Iš čia ji tęsėsi iki Bet Arabo kalnų į šiaurę ir nusileido Araboje. 
\par 19 Nuo čia siena ėjo iki Bet Hoglos kalnų ir šiaurėje baigėsi prie Sūriosios jūros įlankos ir Jordano žiočių. Tai buvo pietinė siena. 
\par 20 O Jordanas­rytinė siena. Tai Benjamino paveldėto krašto sienos. 
\par 21 Benjamino giminei, atskiroms jų šeimoms, teko miestai: Jerichas, Bet Hogla, Emek Kecicas, 
\par 22 Bet Arabas, Cemaraimas, Betelis, 
\par 23 Avimas, Para, Ofra, 
\par 24 Kefar Amona, Ofnis, Geba; iš viso dvylika miestų su jų kaimais. 
\par 25 Gibeonas, Rama, Beerotas, 
\par 26 Micpė, Kefyra, Moca, 
\par 27 Rekemas, Irpeelis, Tarala, 
\par 28 Cela, Elefas, Jebusas, kuris yra Jeruzalė, Gibėja ir Kirjat Jearimas; iš viso keturiolika miestų su jų kaimais. Tai yra Benjamino giminių paveldėjimas.



\chapter{19}

\par 1 Antrasis paveldėjimo paskyrimas burtų keliu buvo Simeono giminės šeimoms; jų žemės buvo Judo paveldėjimo krašte. 
\par 2 Jie paveldėjo: Beer Šebą, Šemą, Moladą, 
\par 3 Hacar Šualą, Balą, Ezemą, 
\par 4 Eltoladą, Betulą, Hormą, 
\par 5 Ciklagą, Bet Markabotą, Hacar Susą, 
\par 6 Bet Lebaotą ir Šaruheną; trylika miestų su jų kaimais. 
\par 7 Ainą, Rimoną, Eterą ir Ašaną; keturis miestus su jų kaimais. 
\par 8 Be to, ir kaimus, kurie buvo arti šitų miestų; pietuose iki Balat Bero ir Ramato buvo Simeono giminės paveldėjimas. 
\par 9 Jų paveldėjimas buvo Judo giminės krašte. Judo giminei buvo paskirta daugiau žemės, negu jiems reikėjo, todėl simeonitai gavo savo dalį Judo paveldėjime. 
\par 10 Trečiasis paveldėjimo paskyrimas burtų keliu buvo Zabulono giminei ligi Sarido. 
\par 11 Jų siena ėjo į vakarus, į Maralą, siekė Dabešetą, toliau­upelio slėnį, esantį Jokneamo rytuose. 
\par 12 Nuo Sarido siena pasuko į rytus, Kislot Taboro link; iš ten į Daberatą ir toliau į Jafiją. 
\par 13 Nuo čia ji ėjo rytų link, į Gat Heferą, Et Kaciną, o iš čia­į Rimoną, Nėjos link. 
\par 14 Šiaurėje siena pasisuko į Hanatoną ir baigėsi Iftach Elio slėnyje. 
\par 15 Be to, Katatas, Nahalalas, Šimronas, Idala ir Betliejus; iš viso dvylika miestų su jų kaimais 
\par 16 buvo Zabulono šeimų paveldėjimas. 
\par 17 Ketvirtasis paveldėjimo paskyrimas burtų keliu buvo Isacharo giminės šeimoms. 
\par 18 Jų žemės apėmė Jezreelį, Kesulotą, Šunemą, 
\par 19 Hafaraimą, Šioną, Anaharatą, 
\par 20 Rabitą, Kišjoną, Ebecą, 
\par 21 Remetą, En Ganimą, En Hadą ir Bet Pacecą. 
\par 22 Ribos siekė Taborą, Šahacimą, Bet Šemešą ir baigėsi prie Jordano, apimdamos šešiolika miestų su jų kaimais. 
\par 23 Tai buvo Isacharo giminės šeimų paveldėjimas. 
\par 24 Penktasis paveldėjimo paskyrimas burtų keliu buvo Ašero giminės šeimoms. 
\par 25 Jų teritorija apėmė: Helkatą, Halį, Beteną, Achšafą, 
\par 26 Alamelechą, Amadą ir Mišalą. Vakaruose siena siekė Karmelį ir Šihor Libnatą. 
\par 27 Toliau ji pakrypo į rytus, į Bet Dagoną, pasiekė Zabuloną ir Iftach Elio slėnį šiaurėje, Bet Emeką ir Nejelį, iš čia pasuko į Kabulą, 
\par 28 Hebroną, Rehobą, Hamoną ir Kaną iki Didžiojo Sidono. 
\par 29 Toliau siena sukosi į Ramą, pasiekdama sutvirtintąjį Tyro miestą, o iš čia­į Hosą ir baigėsi prie jūros. Be to, Mahalebas, Achzibas, 
\par 30 Umas, Afekas ir Rehobas; iš viso dvidešimt du miestai su jų kaimais 
\par 31 buvo Ašero giminės šeimų paveldėjimas. 
\par 32 Neftalio giminės šeimoms burtų keliu teko šeštasis paveldėjimo paskyrimas. 
\par 33 Jų siena ėjo nuo Helefo ir Elono link Caananimo, Adamio, Nekebo ir Jabneelio iki Lakumo ir baigėsi Jordanu. 
\par 34 Ten siena pasisuko į vakarus, į Aznot Taborą, toliau į Hukoką, pasiekė Zabuloną pietuose, Ašerą vakaruose ir Judo sieną Jordano rytuose. 
\par 35 Be to, tvirtovių miestai: Cidimas, Ceras, Hamatas, Rakatas, Kineretas, 
\par 36 Adama, Rama, Hacoras, 
\par 37 Kedešas, Edrėjas, En Hacoras, 
\par 38 Ironas, Migdal Elis, Horemas, Bet Anatas ir Bet Šemešas; iš viso devyniolika miestų su jų kaimais 
\par 39 buvo Neftalio giminės šeimų paveldėjimas. 
\par 40 Dano giminės šeimoms teko burtų keliu septintasis paveldėjimo paskyrimas. 
\par 41 Jų žemės apėmė Corą, Ir Šemešą, Eštaolą, 
\par 42 Šaalabiną, Ajaloną, Itlą, 
\par 43 Eloną, Timną, Ekroną, 
\par 44 Eltekę, Gibetoną, Baalatą, 
\par 45 Jehudą, Bene Beraką, Gat Rimoną 
\par 46 ir Mejarkono bei Rakono plotą ties Jope. 
\par 47 Dano giminės gautas paveldėjimas buvo per mažas, todėl jie nužygiavę kariavo su Lešemu ir jį užėmė. Gyventojus išžudę, apsigyveno jame. Lešemo pavadinimą pakeitė Danu, pagal savo tėvą Daną. 
\par 48 Toks buvo Dano giminės šeimų paveldėjimas. 
\par 49 Taip jie iki galo padalino visą kraštą. Izraelitai davė Nūno sūnui Jozuei paveldėjimą tarp savųjų. 
\par 50 Viešpačiui įsakius, jie davė jam miestą, kurio jis prašė, Timnat Serachą Efraimo kalnuose. Jis atstatė tą miestą ir apsigyveno jame. 
\par 51 Kunigas Eleazaras, Nūno sūnus Jozuė ir Izraelio giminių vyresnieji Šilojuje, Viešpaties akivaizdoje, prie Susitikimo palapinės įėjimo burtų keliu paskirstė kraštą.



\chapter{20}


\par 1 Viešpats kalbėjo Jozuei: 
\par 2 “Paskirkite prieglaudos miestus, apie kuriuos kalbėjau per Mozę, 
\par 3 kad į juos galėtų nubėgti žmogžudys, užmušęs žmogų netyčia. Tie miestai jiems bus prieglauda nuo keršytojų. 
\par 4 Žmogžudys, nubėgęs į vieną šitų miestų, jo vartuose išdėstys savo bylą to miesto vyresniesiems. Jie priims jį į miestą ir leis jam ten gyventi. 
\par 5 Jei jį atsivytų kraujo keršytojas, jie neišduos žudiko į jo rankas, nes jis netyčia užmušė savo artimą, kuris nebuvo jo priešas. 
\par 6 Žmogžudys galės gyventi tame mieste iki teismo sprendimo. Po vyriausiojo kunigo mirties žudikas galės grįžti į savo miestą bei namus, iš kurio jis buvo pabėgęs”. 
\par 7 Jie paskyrė Kedešą Galilėjoje, Neftalio kalnyne, Sichemą Efraimo kalnyne ir Kirjat Arbę, kuri yra Hebronas, Judo kalnyne. 
\par 8 Jordano rytuose jie paskyrė Becerą dykumoje, Rubeno giminės krašte, Ramotą Gileade, Gado giminės krašte, ir Golaną Bašane, Manaso giminės žemėse. 
\par 9 Šitie miestai buvo paskirti visiems izraelitams ir tarp jų gyvenantiems ateiviams, kad į juos galėtų bėgti kiekvienas, užmušęs žmogų netyčia, kad nebūtų nužudytas keršytojo prieš teismą.



\chapter{21}

\par 1 Levitų šeimų vyresnieji atėjo pas kunigą Eleazarą, Nūno sūnų Jozuę ir Izraelio giminių vyresniuosius 
\par 2 į Šilojų, Kanaano krašte, ir jiems kalbėjo: “Viešpats įsakė per Mozę duoti mums miestų apsigyventi ir prie jų ganyklų mūsų gyvuliams”. 
\par 3 Izraelitai davė levitams iš savo dalies, kaip Viešpats buvo liepęs, šituos miestus su jų ganyklomis. 
\par 4 Burtų keliu teko kehatų šeimoms, levitams, kunigo Aarono palikuonims, iš Judo, Simeono ir Benjamino giminių trylika miestų. 
\par 5 Likusiems kehatams burtų keliu iš Efraimo, Dano ir pusės Manaso giminės teko dešimt miestų. 
\par 6 Geršonai gavo burtų keliu iš Isacharo, Ašero, Neftalio ir iš pusės Manaso giminės Bašane trylika miestų. 
\par 7 Merarių šeimoms teko iš Rubeno, Gado ir Zabulono giminių dvylika miestų. 
\par 8 Izraelitai burtų keliu davė levitams šiuos miestus su ganyklomis, kaip Viešpats buvo įsakęs Mozei. 
\par 9 Iš Judo ir Simeono giminių izraelitai davė žemių 
\par 10 Aarono palikuonims levitams iš Kehato šeimos; jie pirmieji gavo savo dalį: 
\par 11 Anako tėvo Arbės miestą Hebroną Judo kalnyne su ganyklomis. 
\par 12 Tačiau miesto laukus bei kaimus jie davė Jefunės sūnui Kalebui. 
\par 13 Kunigo Aarono palikuonims davė Hebroną, prieglaudos miestą, Libną, 
\par 14 Jatyrą, Eštemoją, 
\par 15 Holoną, Debyrą, 
\par 16 Ainą, Jutą ir Bet Šemešą su ganyklomis; devynis miestus iš dviejų giminių. 
\par 17 Iš Benjamino giminės­Gibeoną, Gebą, 
\par 18 Anatotą ir Almoną su ganyklomis; keturis miestus. 
\par 19 Kunigų, Aarono vaikų, miestų iš viso buvo trylika su ganyklomis. 
\par 20 Levitams iš Kehato šeimos burtų keliu teko miestai iš Efraimo giminės. 
\par 21 Jiems davė prieglaudos miestą Sichemą Efraimo kalnyne, Gezerą, 
\par 22 Kibcaimą ir Bet Horoną su ganyklomis; keturis miestus. 
\par 23 Iš Dano giminės: Eltekę, Gibetoną, 
\par 24 Ajaloną ir Gatrimoną su ganyklomis; keturis miestus. 
\par 25 Iš pusės Manaso giminės: Taanachą ir Gat Rimoną su ganyklomis; du miestus. 
\par 26 Iš viso dešimt miestų su ganyklomis buvo duota likusioms Kehato šeimoms. 
\par 27 Gersono palikuonims levitams davė iš pusės Manaso giminės prieglaudos miestą Golaną Basane ir Beešterą su ganyklomis; du miestus. 
\par 28 Iš Isacharo giminės­Kišjoną, Daberatą, 
\par 29 Jarmutą ir En Ganimą su ganyklomis; keturis miestus. 
\par 30 Iš Ašero giminės­Mišalą, Abdoną, 
\par 31 Helkatą ir Rehobą su jų ganyklomis; keturis miestus. 
\par 32 Iš Neftalio giminės­prieglaudos miestą Kedešą Galilėjoje, Hamot Dorą ir Kartaną su ganyklomis; tris miestus. 
\par 33 Geršonų šeimų miestų iš viso buvo trylika su ganyklomis. 
\par 34 Merario palikuonims levitams davė iš Zabulono giminės Jokneamą, Kartą, 
\par 35 Dimną ir Nahalalą su ganyklomis; keturis miestus. 
\par 36 Rytinėje Jordano pusėje iš Rubeno giminės­prieglaudos miestą Becerą, Jahcą, 
\par 37 Kedemotą ir Mefaatą su ganyklomis; keturis miestus. 
\par 38 Iš Gado giminės­prieglaudos miestą Ramotą Gileade, Mahanaimą, 
\par 39 Hešboną ir Jazerą su ganyklomis; keturis miestus. 
\par 40 Merario palikuonys levitai burtų keliu gavo dvylika miestų. 
\par 41 Keturiasdešimt aštuoni levitų miestai su ganyklomis buvo izraelitų krašte. 
\par 42 Visi levitų miestai buvo su ganyklomis aplink juos. 
\par 43 Viešpats atidavė Izraeliui visą šalį, kurią Jis su priesaika pažadėjo duoti jų tėvams. Jie užėmė ją ir gyveno joje. 
\par 44 Ir Viešpats jiems suteikė ramybę, kaip Jis buvo pažadėjęs jų tėvams. Nė vienas iš jų priešų negalėjo atsilaikyti prieš juos; visus juos Viešpats atidavė į izraelitų rankas. 
\par 45 Nebuvo nė vieno žodžio, kuris nebūtų išsipildęs, ką Viešpats kalbėjo Izraeliui.
Online Parallel Study Bible



\chapter{22}

\par 1 Jozuė, pasišaukęs rubenus, gadus bei pusę Manaso giminės, 
\par 2 tarė: “Jūs įvykdėte viską, ką jums įsakė Viešpaties tarnas Mozė, ir klausėte manęs. 
\par 3 Jūs nepalikote savo brolių per visą šitą laiką ir iki šios dienos stropiai vykdėte Viešpaties, jūsų Dievo, įsakymą. 
\par 4 Viešpats, jūsų Dievas, davė jūsų broliams ramybę, kaip Jis jiems pažadėjo. Dabar grįžkite į savo palapines, į šalį, kurią jums davė Viešpaties tarnas Mozė anapus Jordano, 
\par 5 tik rūpestingai vykdykite Jo įsakymus ir nurodymus: mylėkite Viešpatį, savo Dievą, vaikščiokite Jo keliais, laikykitės Jo įsakymų ir Jam tarnaukite visa širdimi bei visa siela”. 
\par 6 Jozuė juos palaimino ir išleido. Jie sugrįžo į savo palapines. 
\par 7 Pusei Manaso giminės Mozė davė žemės Bašane, o antrai jos pusei Jozuė davė drauge su jų broliais vakarinėje Jordano pusėje. Jozuė, išleisdamas juos grįžti namo, palaimino 
\par 8 ir tarė: “Grįžkite į savo kraštą turtingi gyvulių, sidabro, aukso, vario, geležies ir rūbų. Pasidalinkite šį grobį su savo broliais”. 
\par 9 Rubenai, gadai ir pusė Manaso giminės iš Šilojo, esančio Kanaano šalyje, sugrįžo į Gileado kraštą, į savo nuosavybės šalį, kurią jie paveldėjo, kaip Viešpats įsakė Mozei. 
\par 10 Atėję prie Jordano, esančio Kanaano šalyje, rubeniai, gadai ir pusė Manaso giminės pastatė ten didelį aukurą. 
\par 11 Izraelitai, išgirdę, kad rubenai, gadai ir pusė Manaso giminės pasistatė aukurą Kanaano šalyje ant Jordano kranto izraelitams priklausančioje žemėje, 
\par 12 visi susirinko Šilojoje ir ruošėsi žygiuoti prieš juos. 
\par 13 Izraelitai siuntė kunigo Eleazaro sūnų Finejasą pas rubenus, gadus ir pusę Manaso giminės į Gileado šalį, 
\par 14 su juo dešimt kunigaikščių, po vieną iš kiekvienos Izraelio giminės. Kiekvienas iš jų buvo savo šeimos vyresnysis. 
\par 15 Jie atėjo pas rubenus, gadus ir pusę Manaso giminės į Gileado šalį ir kalbėjo jiems: 
\par 16 “Visų izraelitų vardu klausiame, ką reiškia šitas nusikaltimas, kurį šiandien jūs padarėte prieš Izraelio Dievą, nusigręždami nuo Viešpaties ir statydamiesi aukurą? 
\par 17 Argi neužtenka Peoro nusikaltimo, nuo kurio ligi šios dienos dar neapsivalėme ir dėl kurio nukentėjo visa tauta? 
\par 18 Jūs nusisukote nuo Viešpaties. Šiandien jūs sukilote prieš Viešpatį, o rytoj Jis baus visą Izraelio tautą. 
\par 19 Jei manote, kad jūsų šalis yra sutepta, tai persikelkite į Viešpaties šalį, kur yra Jo šventykla ir įsikurkite tarp mūsų. Tik prieš Viešpatį nesukilkite ir nesukilkite prieš mus, statydamiesi aukurą šalia Viešpaties, mūsų Dievo, aukuro! 
\par 20 Prisiminkite Zeracho sūnaus Achano nusikaltimą, kai jis pasisavino iš to, kas buvo skirta sunaikinti­Viešpaties rūstybė ištiko visą Izraelio tautą! Jis ne vienas žuvo dėl savo kaltės!” 
\par 21 Rubenai, gadai ir pusė Manaso giminės atsakė Izraelio atstovams: 
\par 22 “Galingasis Viešpatie Dieve! Galingasis Viešpatie Dieve! Tu žinai, ir Izraelis težino. Jei jūs manote, kad taip atsitiko dėl maišto ar neištikimybės Viešpačiui, tai šiandien tegul Jis nepadeda mums. 
\par 23 Jei taip būtų, kad mes statėme aukurą, norėdami nusigręžti nuo Viešpaties ar aukoti ant jo deginamąsias, duonos ar padėkos aukas, tai pats Viešpats tegul baudžia mus. 
\par 24 Priešingai, iš baimės taip padarėme, galvodami, kad ateityje jūsų vaikai nesakytų mūsų vaikams: ‘Ką jūs turite bendro su Viešpačiu, Izraelio Dievu? 
\par 25 Viešpats padarė Jordaną siena tarp mūsų ir jūsų, rubenai ir gadai! Jūs neturite dalies Viešpatyje’. Jūsų vaikai tokiu būdu neleistų mūsų vaikams garbinti Viešpatį. 
\par 26 Statydami aukurą, galvojome jį pasistatyti ne deginamosioms ir kitoms aukoms, 
\par 27 bet kad jis būtų liudytojas ateityje mūsų ir jūsų kartoms. Kad mes galėtume tarnauti Viešpačiui savo deginamosiomis ir padėkos aukomis, kad ateityje jūsų vaikai nesakytų mūsų vaikams: ‘Jūs neturite dalies Viešpatyje’. 
\par 28 Mes galvojome, jei ateityje jie taip sakys mums ar mūsų kartoms, tai mes atsakysime: ‘Pažiūrėkite į Viešpaties aukuro pavyzdį, kurį padarė mūsų tėvai; jis neskirtas deginamosioms ir padėkos aukoms, bet jis yra liudytojas tarp mūsų ir jūsų’. 
\par 29 Tebūna mums svetima mintis, kad sukiltume prieš Viešpatį ir nusigręžtume nuo Jo, statydamiesi aukurą deginamosioms, duonos ir padėkos aukoms šalia Viešpaties, mūsų Dievo, aukuro, stovinčio prie Jo palapinės!” 
\par 30 Kunigas Finehasas ir Izraelio atstovai, buvusieji su juo, išgirdę, ką kalbėjo Rubeno, Gado ir Manaso vyresnieji, buvo patenkinti. 
\par 31 Kunigas Finėjas, Eleazaro sūnus, tarė jiems: “Šiandien patyrėme, kad Viešpats yra tarp mūsų, nes jūs esate ištikimi Viešpačiui. Tuo jūs išgelbėjote izraelitus nuo Viešpaties bausmės”. 
\par 32 Po to Eleazaro sūnus kunigas Finėjas ir atstovai grįžo iš Gileado krašto į Kanaano šalį pas izraelitus ir jiems viską pranešė. 
\par 33 Ta žinia izraelitams patiko; izraelitai garbino Dievą ir nebegalvojo kariauti su jais ir sunaikinti Rubeno ir Gado giminių krašto. 
\par 34 Tą aukurą Rubeno ir Gado vaikai pavadino “liudytoju”, nes jis liudijo, kad Viešpats yra jų visų Dievas.



\chapter{23}

\par 1 Daugelį metų Viešpats leido Izraeliui gyventi ramybėje nuo visų jo priešų. Jozuė paseno ir sulaukė daug metų. 
\par 2 Jozuė sušaukė visus izraelitus, jų vadus, vyresniuosius, teisėjus bei valdininkus ir jiems tarė: “Aš pasenau ir sulaukiau daug metų. 
\par 3 Jūs patys matėte, ką Viešpats padarė visoms šitoms tautoms dėl jūsų, nes pats Viešpats, jūsų Dievas, kariavo už jus. 
\par 4 Aš paskirsčiau jūsų giminių nuosavybėn šitas likusias tautas ir visas tautas, kurias sunaikinau į vakarus nuo Jordano iki Didžiosios jūros. 
\par 5 Viešpats, jūsų Dievas, jas privers pasitraukti, ir užėmę jūs paveldėsite jų šalį, kaip jums pažadėjo Viešpats, jūsų Dievas. 
\par 6 Būkite drąsūs ir tvirtai pasiryžę vykdyti visa, kas parašyta Mozės įstatymo knygoje, ir nenukrypkite nuo jos. 
\par 7 Nesimaišykite su šitomis tautomis, kurios tebegyvena tame krašte. Neminėkite jų dievų vardų, neprisiekite jais, negarbinkite jų ir netarnaukite jiems; 
\par 8 būkite ištikimi Viešpačiui, savo Dievui, kaip iki šios dienos. 
\par 9 Viešpats nugalėjo jūsų akyse dideles bei galingas tautas; niekas iki šios dienos neįstengė atsilaikyti prieš jus. 
\par 10 Vienas jūsiškių vys tūkstantį, nes Viešpats, jūsų Dievas, kovoja už jus, kaip Jis jums pažadėjo. 
\par 11 Todėl žiūrėkite, kad mylėtumėte Viešpatį, savo Dievą. 
\par 12 Jei nuo Jo nutolsite ir susimaišysite su pasilikusiomis tautomis, kurios tebegyvena tarp jūsų, ir susigiminiuosite vedybomis, eidami pas juos, o jie pas jus, 
\par 13 tai būkite užtikrinti, kad Viešpats, jūsų Dievas, neišvarys šitų tautų iš jūsų krašto; jos taps jums spąstais ir žabangais, botagais jūsų nugaroms ir dygliais jūsų akims, kol jūs pražūsite iš šitos geros žemės, kurią jums davė Viešpats, jūsų Dievas. 
\par 14 Štai rengiuosi eiti keliu, kuriuo turi eiti visas pasaulis. Prisiminkite visa savo širdimi ir siela, kad Viešpats, jūsų Dievas, išpildė visus iki vieno savo gerus pažadus jums. 
\par 15 Kaip Viešpats įvykdė viską, ką buvo jums pažadėjęs gero, taip Jis ištesės jums ir blogus pažadus, išnaikindamas jus iš šitos geros žemės, kurią Jis jums davė. 
\par 16 Jei jūs laužysite Viešpaties, savo Dievo, jums duotą sandorą, tarnausite svetimiems dievams ir garbinsite juos, tai Viešpats baus jus ir jūs greitai pražūsite nuo geros žemės, kurią iš Jo gavote”.



\chapter{24}

\par 1 Jozuė sušaukė visas Izraelio gimines į Sichemą, taip pat Izraelio vyresniuosius, vadovus, teisėjus ir valdininkus, ir jie stojo Dievo akivaizdon. 
\par 2 Jozuė kalbėjo visai tautai: “Taip sako Viešpats, Izraelio Dievas: ‘Anapus upės senovėje gyveno jūsų tėvai, Abraomo ir Nahoro tėvas Tara, ir jie tarnavo svetimiems dievams. 
\par 3 Vėliau jūsų tėvą Abraomą iš anapus upės vedžiau per visą Kanaano kraštą, padauginau jo palikuonis ir jam daviau Izaoką. 
\par 4 Izaokui daviau Jokūbą ir Ezavą. Ezavui daviau Seyro kalnyną. O Jokūbas ir jo vaikai nuvyko į Egiptą. 
\par 5 Vėliau siunčiau Mozę ir Aaroną ir Egiptą varginau nelaimėmis, o jus išvedžiau. 
\par 6 Aš išvedžiau jūsų tėvus iš Egipto. Jie atėjo prie jūros. Egiptiečiai vijosi jūsų tėvus su kovos vežimais ir raiteliais iki Raudonosios jūros. 
\par 7 Jiems šaukiantis Viešpaties, Aš padariau tamsą tarp jūsų ir egiptiečių, užleidau ant jų jūrą ir paskandinau juos. Jūsų akys matė, ką dariau Egipte. Po to jūs ilgai gyvenote dykumoje. 
\par 8 Vėliau atvedžiau jus į šalį amoritų, kurie gyveno Jordano rytuose. Jie kariavo su jumis, bet Aš atidaviau juos į jūsų rankas; jūs užėmėte jų šalį, o Aš juos išnaikinau jūsų akyse. 
\par 9 Po to Ciporo sūnus Balakas, Moabo karalius, kariavo su Izraeliu. Jis pasikvietė Beoro sūnų Balaamą, kad jus prakeiktų. 
\par 10 Aš neklausiau Balaamo ir priverčiau jį jus palaiminti. Taip Aš išgelbėjau jus iš jo rankų. 
\par 11 Vėliau jūs, perėję per Jordaną, atėjote į Jerichą. Jericho vyrai, amoritai, perizai, kanaaniečiai, hetitai, girgašai, hivai ir jebusiečiai kariavo prieš jus, bet Aš atidaviau juos į jūsų rankas. 
\par 12 Aš siunčiau širšes pirma jūsų, ir jie pabėgo. Taip buvo nugalėti du amoritų karaliai, ne kardu ir ne lanku. 
\par 13 Aš daviau jums žemę, kurios jūs nebuvote arę, ir miestus, kurių nestatėte. Jus maitino vynuogynai ir alyvmedžiai, kurių nesodinote’. 
\par 14 Taigi dabar bijokite Viešpaties ir Jam tarnaukite nuoširdžiai ir ištikimai, pašalinkite dievus, kuriems tarnavo jūsų tėvai anapus upės ir Egipte. 
\par 15 O jei jums nepatinka tarnauti Viešpačiui, tai šiandien apsispręskite, kam norite tarnauti: dievams, kuriems tarnavo jūsų tėvai, gyvenusieji anapus upės, ar dievams amoritų, kurių šalyje gyvenate. Bet aš ir mano namai tarnausime Viešpačiui”. 
\par 16 Tauta atsakė: “Taip nebus, kad apleistume Viešpatį ir tarnautume svetimiems dievams. 
\par 17 Juk Viešpats, mūsų Dievas, išvedė mus ir mus saugojo visą kelią, kuriuo ėjome, ir visose tautose, kurias sutikome. 
\par 18 Viešpats išvarė pirma mūsų amoritus, gyvenusius krašte. Taigi mes taip pat tarnausime Viešpačiui, nes Jis yra mūsų Dievas”. 
\par 19 Jozuė tarė: “Jūs negalėsite tarnauti Viešpačiui, nes Jis yra šventas ir pavydus Dievas. Jis neatleis jūsų nusikaltimų ir nuodėmių. 
\par 20 Jei apleisite Viešpatį ir tarnausite svetimiems dievams, tai Jis baus jus ir sunaikins, iki tol jums daręs gera”. 
\par 21 Tauta atsakė Jozuei: “Ne! Mes tarnausime Viešpačiui”. 
\par 22 Tada Jozuė tarė tautai: “Jūs patys esate liudytojai, kad pasirinkote Viešpatį ir norite Jam tarnauti”. Jie atsakė: “Taip, mes esame liudytojai”. 
\par 23 “Dabar pašalinkite svetimus dievus iš savo tarpo ir palenkite savo širdį prie Viešpaties, Izraelio Dievo”. 
\par 24 Tauta atsakė Jozuei: “Viešpačiui, savo Dievui, tarnausime ir Jo klausysime”. 
\par 25 Jozuė tą dieną Sicheme padarė su Izraelio tauta sandorą ir jai davė įsakymus ir nurodymus. 
\par 26 Tuos žodžius Jozuė įrašė į Dievo įstatymo knygą ir, paėmęs didžiulį akmenį, pastatė jį po ąžuolu, kuris augo prie Viešpaties šventyklos. 
\par 27 Tada jis tarė visai tautai: “Šitas akmuo tebūna liudytojas jums, nes jis girdėjo visus Viešpaties žodžius, kuriuos Jis kalbėjo mums, kad neišsigintumėte savo Dievo”. 
\par 28 Po to Jozuė leido žmonėms grįžti kiekvienam į savo paveldėjimą. 
\par 29 Tada mirė Viešpaties tarnas, Nūno sūnus Jozuė, sulaukęs šimto dešimties metų amžiaus. 
\par 30 Jį palaidojo jo paveldėtoje žemėje Timnat Serache, Efraimo kalnyne, į šiaurę nuo Gaašo kalno. 
\par 31 Izraelis tarnavo Viešpačiui per visas Jozuės dienas ir per visas dienas vyresniųjų, kurie gyveno ilgiau už Jozuę ir žinojo visus Viešpaties darbus, padarytus Izraeliui. 
\par 32 Juozapo kaulus, kuriuos izraelitai atsigabeno iš Egipto, palaidojo Sicheme, sklype, kurį Jokūbas pirko iš Hamoro, Sichemo tėvo, sūnų už šimtą sidabrinių. Tą žemę paveldėjo Juozapo palikuonys. 
\par 33 Aarono sūnus Eleazaras irgi mirė; jį palaidojo Gibėjoje, jo sūnaus Finehaso žemėje, kuri buvo Efraimo kalnyne.




\end{document}