\begin{document}

\title{Patarlių knyga}

\chapter{1}


\par 1 Saliamono, Dovydo sūnaus, Izraelio karaliaus, patarlės. 
\par 2 Jos surašytos, kad pamokytų išminties, auklėtų ir padėtų suprasti išmintingus posakius, 
\par 3 kad pamokytų išmintingai elgtis, pažinti teisumą, teisingumą ir bešališkumą; 
\par 4 kad paprastiems suteiktų sumanumo, jaunuoliams­supratimo ir nuovokumo. 
\par 5 Išmintingas klausydamas taps išmintingesnis, o protingas gaus išmintingų patarimų, 
\par 6 kad suprastų patarles ir palyginimus, išminčių žodžius ir mįsles. 
\par 7 Viešpaties baimė yra išminties pradžia, bet kvailiai niekina išmintį ir pamokymus. 
\par 8 Mano sūnau, klausyk tėvo pamokymų ir neatmesk motinos nurodymų. 
\par 9 Tai bus puošnus vainikas tavo galvai ir papuošalas tavo kaklui. 
\par 10 Mano sūnau, jei tave vilios nusidėjėliai, nepritark jiems. 
\par 11 Jei jie sako: “Eime su mumis tykoti kraujo ir ruošti pasalą nekaltam žmogui. 
\par 12 Prarykime juos gyvus kaip pragaras, visiškai, kaip tuos, kurie eina į kapą. 
\par 13 Mes rasime daug turto ir pripildysime grobiu savo namus. 
\par 14 Su mumis mesi dalybų burtą, mūsų pinigai bus bendri”. 
\par 15 Mano sūnau, neik su jais, sulaikyk savo koją nuo jų tako. 
\par 16 Jų kojos bėga į pikta, jie skuba kraują pralieti. 
\par 17 Veltui tiesiamas tinklas paukščiams matant. 
\par 18 Jie tykoja savo pačių kraujo, pasalą ruošia savo gyvybei. 
\par 19 Toks yra kelias kiekvieno, kuris godus turto, jis atima jo savininko gyvybę. 
\par 20 Išmintis šaukia gatvėje, pakelia balsą aikštėje. 
\par 21 Ji šaukia svarbiausiose susibūrimo vietose ir miesto vartuose skelbia savo žodžius: 
\par 22 “Neišmanėliai, ar ilgai dar mylėsite neišmanymą? Niekintojai, ar ilgai džiaugsitės savo patyčiomis? Kvailiai, ar ilgai nekęsite pažinimo? 
\par 23 Klausykitės mano įspėjimų! Aš išliesiu jums savo dvasios, paskelbsiu savo žodžius. 
\par 24 Kadangi aš šaukiau, o jūs nepaklausėte, ištiesiau jums ranką, bet niekas nekreipė dėmesio, 
\par 25 jūs paniekinote mano patarimus ir nepaisėte mano įspėjimų, 
\par 26 tai ir aš juoksiuos, kai jūs žlugsite, tyčiosiuos, kai jus apims baimė, 
\par 27 kai siaubas užklups kaip audra ir pražūtis kaip viesulas, kai ateis sielvartas ir vargas. 
\par 28 Tada jie šauksis manęs, bet aš neatsiliepsiu; jie ieškos manęs, bet neras. 
\par 29 Nes jie nekentė pažinimo ir nepasirinko Viešpaties baimės. 
\par 30 Jie nepriėmė mano patarimų ir paniekino mano barimą. 
\par 31 Todėl jie valgys savo kelių vaisių, pasisotins savo sumanymais. 
\par 32 Neišmanėlių užsispyrimas užmuš juos, kvailius pražudys jų neapdairumas. 
\par 33 Kas manęs klauso, gyvens saugiai, bus ramus ir nebijos pikto”.


\chapter{2}


\par 1 Mano sūnau, jei priimsi mano žodžius ir paslėpsi širdyje mano įsakymus, 
\par 2 jei tavo ausis atidžiai klausysis išminties ir širdis sieks supratimo, 
\par 3 jei šauksiesi pažinimo ir pakelsi balsą, prašydamas supratimo, 
\par 4 jei sieksi jo lyg sidabro ir ieškosi lyg paslėptų lobių, 
\par 5 tada tu suprasi Viešpaties baimę ir rasi Dievo pažinimą. 
\par 6 Nes Viešpats teikia išmintį, iš Jo burnos ateina pažinimas ir supratimas. 
\par 7 Jis laiko sveiką išmintį teisiesiems, Jis yra skydas doru keliu einantiems. 
\par 8 Jis saugo teisingumo takus ir sergi šventųjų kelius. 
\par 9 Tada tu suprasi teisumą, teisingumą ir bešališkumą bei kiekvieną gerą taką, 
\par 10 kai išmintis įeis į tavo širdį ir pažinimas bus malonus tavo sielai. 
\par 11 Dėmesingumas sergės tave ir supratimas saugos tave, 
\par 12 kad išgelbėtų nuo pikto kelio, nuo melą kalbančių žmonių, 
\par 13 kurie palieka doros kelią, kad vaikščiotų tamsos keliais, 
\par 14 kurie džiaugiasi, darydami pikta, ir gėrisi nedorėlių iškrypimu, 
\par 15 kurių keliai kreivi ir kurie klaidžioja savo takais; 
\par 16 kad išgelbėtų tave nuo svetimos moters, meilikaujančios savo žodžiais, 
\par 17 kuri palieka savo jaunystės vadovą ir užmiršta savo Dievo sandorą. 
\par 18 Jos namai veda į mirtį, jos takai­pas mirusiuosius. 
\par 19 Visi, kurie įeina pas ją, nebegrįžta ir neberanda gyvenimo tako. 
\par 20 Kad tu vaikščiotum geru keliu ir laikytumeisi teisiųjų tako. 
\par 21 Nes teisieji gyvens krašte ir nekaltieji pasiliks jame. 
\par 22 Bet nedorėliai bus pašalinti iš krašto, nusikaltėliai išrauti iš jo.



\chapter{3}


\par 1 Mano sūnau, neužmiršk mano įstatymo, tavo širdis tesaugo mano įsakymus. 
\par 2 Jie prailgins tavo dienas, pridės tau gyvenimo metų ir ramybės. 
\par 3 Gailestingumas ir tiesa tenepalieka tavęs; užsirišk tai ant savo kaklo, įrašyk į savo širdies plokštę. 
\par 4 Taip atrasi palankumą bei gerą įvertinimą Dievo ir žmonių akyse. 
\par 5 Pasitikėk Viešpačiu visa širdimi ir nesiremk savo supratimu. 
\par 6 Visuose savo keliuose pripažink Jį, tai Jis nukreips tavo takus. 
\par 7 Nebūk išmintingas savo akyse, bijok Viešpaties ir venk pikto; 
\par 8 tai bus sveikata tavo kūnui ir atgaiva kaulams. 
\par 9 Pagerbk Viešpatį savo turtu ir pirmaisiais viso derliaus vaisiais, 
\par 10 tai tavo aruodai bus kupini, spaustuvai perpildyti naujo vyno. 
\par 11 Mano sūnau, nepaniekink Viešpaties bausmės ir nenusimink Jo baramas, 
\par 12 nes ką Viešpats myli, tą pabara, kaip tėvas auklėdamas sūnų, kuriuo gėrisi. 
\par 13 Palaimintas žmogus, kuris randa išmintį ir įgauna supratimo, 
\par 14 nes ją įsigyti yra naudingiau, negu įsigyti sidabro, ir pelno iš jos yra daugiau negu iš geriausio aukso. 
\par 15 Ji brangesnė už deimantus, su ja nesulyginama visa, ko galėtum trokšti. 
\par 16 Dešinėje jos rankoje ilgas amžius, o kairėje turtai ir garbė. 
\par 17 Jos keliai­malonės keliai, visi jos takai­ramybė. 
\par 18 Ji yra gyvybės medis visiems, kurie ją pagauna, laimingi, kas ją išlaiko. 
\par 19 Viešpats išmintimi sukūrė žemę, savo protu įtvirtino dangus. 
\par 20 Jo žinojimu buvo atvertos gelmės ir debesys laša rasa. 
\par 21 Mano sūnau, neišleisk jų iš akių, saugok sveiką nuovoką ir įžvalgumą. 
\par 22 Tai bus gyvybė tavo sielai ir papuošalas kaklui; 
\par 23 tada eisi saugus savo keliu, ir tavo koja nesuklups. 
\par 24 Atsigulęs nebijosi, ir tavo miegas bus saldus. 
\par 25 Tavęs neišgąsdins staigus sąmyšis ir nedorėlių siautėjimas, kai jis ateis. 
\par 26 Viešpats bus tavo pasitikėjimas, Jis apsaugos tavo koją, kad neįkliūtum. 
\par 27 Neatsakyk geradarystės tam, kuriam dera ją padaryti, kai tavo ranka tai pajėgia. 
\par 28 Nesakyk savo artimui: “Eik ir sugrįžk, rytoj aš tau duosiu”, kai turi su savimi. 
\par 29 Neplanuok pikta prieš savo artimą, kuris gyvena pasitikėdamas arti tavęs. 
\par 30 Nesibark be reikalo su žmogumi, kuris tau nieko blogo nepadarė. 
\par 31 Nepavydėk smurtininkui ir nesirink nė vieno iš jo kelių. 
\par 32 Viešpats bjaurisi ydingu žmogumi, bet Jis artimas teisiajam. 
\par 33 Viešpaties prakeikimas nedorėlio namams, o teisiojo buveinę Jis laimina. 
\par 34 Jis išjuokia pašaipūnus, o nuolankiesiems teikia malonę. 
\par 35 Išmintingieji paveldės šlovę, kvailio pasididžiavimas pasibaigs gėda.



\chapter{4}


\par 1 Vaikai, klausykite tėvo pamokymo. Būkite atidūs, kad įgytumėte supratimo. 
\par 2 Aš duodu jums gerą mokymą, neapleiskite mano įstatymo. 
\par 3 Aš buvau tėvo sūnus, mylimas ir vienintelis savo motinos akyse. 
\par 4 Jis mokė mane ir sakė: “Išlaikyk širdyje mano žodžius, laikykis mano įsakymų ir būsi gyvas. 
\par 5 Įsigyk išminties, įsigyk supratimo. Neužmiršk mano burnos žodžių ir nenukrypk nuo jų. 
\par 6 Neapleisk išminties, ir ji globos tave; mylėk ją, ir ji saugos tave. 
\par 7 Svarbiausia yra išmintis; įsigyk išmintį ir už visą savo turtą įsigyk supratimą. 
\par 8 Vertink ją, ir ji išaukštins tave. Apglėbk ją, ir ji suteiks tau garbę. 
\par 9 Ji vainikuos tavo galvą malone, duos tau šlovės karūną. 
\par 10 Klausyk, mano sūnau, ir priimk mano žodžius, kad ilgai gyventum. 
\par 11 Aš nurodau tau išminties kelią, vedu tiesiais takais. 
\par 12 Kai jais eisi, tavo žingsniai nebus varžomi ir bėgdamas nesuklupsi. 
\par 13 Tvirtai laikykis pamokymo ir neapleisk jo; saugok jį, nes tai yra tavo gyvybė. 
\par 14 Neik nedorėlių taku ir nevaikščiok piktųjų keliu. 
\par 15 Venk jo ir nevaikščiok juo, pasitrauk nuo jo ir praeik pro šalį. 
\par 16 Jie neužmiega, nepadarę pikto ir miegas iš jų atimamas, kol kam nors nepakenkia. 
\par 17 Jie valgo nedorybės duoną ir geria smurto vyną. 
\par 18 Teisiųjų takas yra kaip skaisti šviesa, kuri šviečia vis ryškiau iki tobulos dienos. 
\par 19 Nedorėlių kelias yra kaip tamsa, jie nežino, kur jie suklups. 
\par 20 Mano sūnau, atkreipk dėmesį į mano žodžius, palenk ausį mano pamokymams. 
\par 21 Teneatsitraukia jie nuo tavo akių, saugok juos širdies gilumoje. 
\par 22 Jie yra gyvybė tiems, kurie juos randa, ir sveikata visam jų kūnui. 
\par 23 Saugok su visu stropumu savo širdį, nes iš jos teka gyvenimo versmė. 
\par 24 Atstumk nuo savęs nedorą burną, ir iškrypusios lūpos tebūna toli nuo tavęs. 
\par 25 Tegul tavo akys žiūri tiesiai ir akių vokai tebūna nukreipti tiesiai prieš tave. 
\par 26 Apgalvok taką savo kojai, ir tegu tavo keliai būna įtvirtinti. 
\par 27 Nepasuk nei dešinėn, nei kairėn. Patrauk savo koją nuo pikto”.



\chapter{5}


\par 1 Mano sūnau, būk dėmesingas mano išminčiai ir palenk ausį mano supratimui, 
\par 2 kad būtum nuovokus ir tavo lūpos išlaikytų pažinimą. 
\par 3 Kaip varvantis medus svetimos moters lūpos ir jos burna švelnesnė už aliejų. 
\par 4 Bet galiausiai ji tampa karti kaip metėlė ir aštri kaip dviašmenis kalavijas. 
\par 5 Jos kojos žengia į mirtį, jos žingsniai veda į pragarą. 
\par 6 Kad tu nemąstytum apie jos gyvenimo taką, žinok­jos keliai nepastovūs ir tu negali jų suprasti. 
\par 7 Dabar, mano vaikai, klausykite manęs ir neatsitraukite nuo mano burnos žodžių. 
\par 8 Atitolink nuo jos savo kelią ir nesiartink prie jos namų durų, 
\par 9 kad neatiduotum savo garbės kitiems ir savo metų negailestingajam; 
\par 10 kad svetimi nesisotintų tavo gėrybėmis ir tavo darbas nebūtų svetimojo namuose, 
\par 11 ir galiausiai neturėtum vaitoti, išsekinęs kūną bei jėgas, 
\par 12 ir sakyti: “Kodėl nekenčiau pamokymų ir mano širdis paniekino pabarimą, 
\par 13 neklausiau savo mokytojų balso ir nepalenkiau savo ausies prie tų, kurie mane mokė? 
\par 14 Aš kone patekau į visokias nelaimes bendruomenės ir susirinkimo vidury”. 
\par 15 Gerk vandenį iš savo šulinio, tekantį vandenį iš savo versmės. 
\par 16 Ar tavo vandens šaltiniai išsilies po gatves, o vandens srovės aikštėse? 
\par 17 Tebūna jie tik tau, nedalink jų svetimiems. 
\par 18 Tebūna tavo šaltinis palaimintas ir džiaukis su savo jaunystės žmona. 
\par 19 Ji kaip miela stirna, kaip grakšti elnė. Tegul jos krūtys tenkina tave visą laiką, nuolat mėgaukis jos meile. 
\par 20 Mano sūnau, kam tau mėgautis svetima moterimi ir apsikabinti su svetimąja? 
\par 21 Viešpats stebi visus žmogaus kelius ir apsvarsto visus jo takus. 
\par 22 Nedorėlį sugauna jo paties nedorybės ir supančioja jo nuodėmių pančiai. 
\par 23 Jis miršta nepasimokęs, per savo didelį kvailumą nuklysta.



\chapter{6}


\par 1 Mano sūnau, jei laidavai už savo artimą ar padavei ranką už svetimąjį, 
\par 2 tu įsipainiojai savo burnos žodžiais ir esi sugautas savo kalbomis. 
\par 3 Daryk štai ką, mano sūnau, ir gelbėk save, nes esi patekęs į savo artimo rankas: eik, nusižemink ir maldauk savo artimą. 
\par 4 Neduok miegoti savo akims ir neleisk užsimerkti akių vokams. 
\par 5 Gelbėkis kaip elnė iš medžiotojo, kaip paukštis iš paukštgaudžio rankų. 
\par 6 Tinginy, eik pas skruzdę, apsvarstyk jos kelius ir būk išmintingas. 
\par 7 Ji neturi vadovo, prižiūrėtojo ar valdovo, 
\par 8 bet paruošia sau maisto vasarą ir pjūties metu renka atsargas. 
\par 9 Ar ilgai miegosi, tinginy? Kada atsikelsi iš savo miego? 
\par 10 Truputį pamiegosi, truputį pasnausi, truputį pagulėsi sudėjęs rankas, 
\par 11 ir ateis skurdas kaip pakeleivis ir nepriteklius kaip ginkluotas plėšikas. 
\par 12 Nenaudėlis žmogus, piktadarys, vaikštinėja su klastinga burna, 
\par 13 mirksi akimis, trypia kojomis, rodo pirštu. 
\par 14 Klasta jo širdyje, jis nuolat planuoja pikta ir sėja vaidus. 
\par 15 Todėl staiga ateis jo žlugimas, ūmai bus jis sudaužytas, nesulaukęs pagalbos. 
\par 16 Viešpats nekenčia šešių dalykų, septyni yra pasibjaurėjimas Jo akyse: 
\par 17 išdidus žvilgsnis, meluojantis liežuvis, rankos, praliejančios nekaltą kraują, 
\par 18 širdis, planuojanti nedorybę, kojos, greitos bėgti į pikta, 
\par 19 neteisingas liudytojas, kalbantis melą ir žmogus, sėjantis nesantaiką tarp brolių. 
\par 20 Mano sūnau, laikykis savo tėvo įsakymų ir nepaniekink motinos įstatymo. 
\par 21 Visam laikui užrišk juos ant savo širdies, apsivyniok aplink kaklą. 
\par 22 Tau einant, jie lydės tave, tau atsigulus, jie saugos tave, tau pabudus, jie kalbės su tavimi. 
\par 23 Įsakymas yra žiburys, įstatymas­šviesa, o pamokantis pabarimas­gyvenimo kelias. 
\par 24 Jie saugos tave nuo nedoros moters, nuo svetimos moters meilikaujančios kalbos. 
\par 25 Negeisk jos grožio savo širdyje, tenesuvilioja tavęs jos blakstienos. 
\par 26 Dėl paleistuvės vyras lieka tik su duonos kąsniu, neištikimoji medžioja jo brangią gyvybę. 
\par 27 Ar gali žmogus paimti ugnį į savo antį ir nesudeginti drabužių? 
\par 28 Ar gali kas, vaikščiodamas ant žarijų, nenusideginti kojų? 
\par 29 Taip ir tas, kas įeina pas artimo žmoną; kas paliečia ją, neliks nekaltas. 
\par 30 Vagis ne taip niekinamas, jei jis vagia būdamas alkanas ir norėdamas pasisotinti. 
\par 31 Tačiau pagautas jis atlygins septyneriopai ir atiduos visą savo namų turtą. 
\par 32 Svetimoteriaujančiam trūksta proto, jis pats save pražudo. 
\par 33 Žaizdų ir nešlovės jis susilauks, jo gėda nebus išdildyta; 
\par 34 nes pavydas sužadins vyro įniršį, jis nepasigailės keršto dieną. 
\par 35 Jis nepriims jokios išpirkos ir nenusiramins, nors duotum jam daugybę dovanų.



\chapter{7}


\par 1 Mano sūnau, saugok mano žodžius ir neužmiršk mano įsakymų. 
\par 2 Saugok mano įsakymus ir būsi gyvas, sergėk mano įstatymą kaip savo akies vyzdį. 
\par 3 Užsirišk juos ant savo pirštų ir įsirašyk savo širdies plokštėje. 
\par 4 Išmintį vadink savo seserimi, o supratimą­artimiausiu savo draugu, 
\par 5 kad jie apsaugotų tave nuo svetimos moters, meilikaujančios savo žodžiais. 
\par 6 Kartą, žiūrėdamas pro savo namų lango groteles, 
\par 7 mačiau tarp jaunų, nepatyrusių žmonių neprotingą jaunuolį. 
\par 8 Jis ėjo gatve pro jos kampą ir pasuko į jos namus. 
\par 9 Vakaro prieblandoje, juodą ir tamsią naktį, 
\par 10 jį pasitiko moteris su paleistuvės apdaru ir klastinga širdimi. 
\par 11 Begėdė ir nerimstanti, negalinti išsėdėti savo namuose; 
\par 12 tai gatvėje, tai aikštėje, tai ant kampo ji tykojo aukos. 
\par 13 Pasigavusį jį, bučiavo ir akiplėšiškai kalbėjo: 
\par 14 “Aš turiu padėkos auką ir šiandien įvykdžiau savo įžadus. 
\par 15 Todėl išėjau tavęs pasitikti, ieškojau tavo veido ir štai suradau tave. 
\par 16 Aš apklojau savo lovą margais egiptietiškais lininiais užtiesalais, 
\par 17 iškvėpinau mira, alaviju ir cinamonu. 
\par 18 Eikš, sotinsimės meile iki ryto, mėgausimės glamonėmis; 
\par 19 nes vyro nėra namuose, jis išvyko į tolimą kelionę. 
\par 20 Jis pasiėmė maišelį pinigų ir grįš pilnaties dieną”. 
\par 21 Daugybe švelnių žodžių ji prisiviliojo jį, meilikaujančia kalba suvedžiojo. 
\par 22 Jis seka ją kaip jautis, vedamas pjauti, kaip kvailys, einantis į supančiojimo vietą, 
\par 23 kol strėlė pervers jo kepenis; kaip paukštis, kuris lekia į spąstus ir nesupranta, kad praras savo gyvybę. 
\par 24 Mano vaikai, klausykite manęs, būkite dėmesingi mano žodžiams. 
\par 25 Tegul tavo širdis nepalinksta į jos kelius, nenuklysk jos takais. 
\par 26 Nes ji parbloškė daug sužeistųjų, daug stipriųjų jinai nužudė. 
\par 27 Jos namai yra kelias į pragarą, vedantis į mirties buveinę.



\chapter{8}


\par 1 Argi išmintis nešaukia ir supratimas nekelia savo balso? 
\par 2 Aukštumose, prie kelių ir prie takų ji stovi. 
\par 3 Prie miesto vartų, prie įėjimo į miestą, prie durų šaukia: 
\par 4 “Žmonės, į jus aš kreipiuosi, jums šaukiu, žmonių sūnūs. 
\par 5 Jūs neišmanėliai, mokykitės išminties; jūs kvailiai, įgykite supratingą širdį. 
\par 6 Klausykite, nes aš kalbėsiu apie didingus dalykus, mano lūpos skelbs teisumą. 
\par 7 Mano burna kalbės tiesą, ir nedorybė yra pasibjaurėjimas mano lūpoms. 
\par 8 Visi mano žodžiai yra teisingi, juose nėra klastos ir iškraipymo. 
\par 9 Jie yra aiškūs išmanantiems ir teisingi suprantantiems. 
\par 10 Priimkite mano pamokymą, o ne sidabrą; pažinimą, o ne gryną auksą. 
\par 11 Išmintis yra brangesnė už deimantus, su ja nesulyginama visa, ko galima trokšti. 
\par 12 Aš, išmintis, gyvenu su protingumu, atrandu pažinimą bei nuovokumą. 
\par 13 Viešpaties baimė­nekęsti pikto. Išdidumo, puikybės, piktų kelių ir klastingos burnos aš neapkenčiu. 
\par 14 Manyje patarimas ir sveikas protas, aš turiu supratimą ir jėgą. 
\par 15 Manimi karaliai karaliauja ir kunigaikščiai leidžia įstatymus. 
\par 16 Manimi kunigaikščiai, kilmingieji ir teisėjai valdo kraštą. 
\par 17 Aš myliu tuos, kurie mane myli. Kas anksti manęs ieško, suras mane. 
\par 18 Aš turiu turtus ir garbę, išliekančius turtus ir teisumą. 
\par 19 Mano vaisius yra brangesnis už gryną auksą, ir pelno iš manęs daugiau negu iš rinktinio sidabro. 
\par 20 Aš vedu teisumo keliu, viduriu teisingumo tako, 
\par 21 kad mane mylintiems duočiau paveldėti turtus ir pripildyčiau jų sandėlius. 
\par 22 Viešpats turėjo mane savo kelio pradžioje, prieš visus savo darbus. 
\par 23 Nuo amžių aš įtvirtinta, nuo pradžios, prieš pasaulio sutvėrimą. 
\par 24 Aš buvau pagimdyta, kai dar netryško vandens šaltiniai, nebuvo gelmių. 
\par 25 Pirma kalnų ir kalvų iškilimo aš buvau pagimdyta, 
\par 26 kai Jis dar nebuvo padaręs žemės, jos laukų ir pirmųjų žemės dulkių. 
\par 27 Kai Jis paruošė dangus, aš ten buvau. Kai Jis nubrėžė ribą virš gelmių, 
\par 28 įtvirtino debesis viršuje ir sustiprino gelmių šaltinius, 
\par 29 davė jūrai nurodymą, kad vandenys neperžengtų Jo įsakymo, ir dėjo žemės pamatus, 
\par 30 aš buvau šalia Jo kaip įgudęs menininkas, gėrėjausi kas dieną ir džiūgavau Jo akivaizdoje. 
\par 31 Džiaugiausi Jo apgyvendintame pasaulyje ir grožėjausi žmonių vaikais. 
\par 32 Dabar, vaikai, klausykite manęs. Palaiminti, kurie eina mano keliais. 
\par 33 Klausykite pamokymų ir būkite išmintingi, neatmeskite jų. 
\par 34 Palaimintas žmogus, kuris klauso manęs, kasdien budi prie mano vartų ir laukia prie mano durų. 
\par 35 Kas randa mane, randa gyvenimą ir įgis Viešpaties palankumą. 
\par 36 Kas nusideda prieš mane, tas kenkia pats sau. Kas manęs nekenčia, myli mirtį”.



\chapter{9}


\par 1 Išmintis pasistatė namus, išsikirto septynias kolonas. 
\par 2 Ji papjovė gyvulius, sumaišė vyną ir, padengusi stalą, 
\par 3 pasiuntė tarnaites šaukti miesto aukščiausiose vietose: 
\par 4 “Neišmanėliai, ateikite!” Kam trūksta supratimo, ji sako: 
\par 5 “Ateikite, valgykite mano duonos ir gerkite vyno, kurį sumaišiau. 
\par 6 Atsisakykite kvailystės ir gyvenkite; eikite supratimo keliu”. 
\par 7 Kas bara niekintoją, susilauks gėdos; sudraudęs nedorėlį, užsitrauksi dėmę. 
\par 8 Nebark niekintojo, kad jis neimtų tavęs neapkęsti; sudrausk išmintingą, ir jis mylės tave. 
\par 9 Patark išmintingam, ir jis taps dar išmintingesnis; pamokyk teisųjį, ir jo pažinimas išaugs. 
\par 10 Viešpaties baimė­išminties pradžia, o Šventojo pažinimas­supratimas. 
\par 11 Per mane padaugės tavo dienų, bus pridėta gyvenimo metų. 
\par 12 Jei esi išmintingas, esi išmintingas pats sau; jei niekintojas, pats ir nukentėsi. 
\par 13 Kvaila moteris yra triukšmadarė, neišmananti ir nieko nežino. 
\par 14 Ji sėdi kėdėje prie savo namų, aukštose miesto vietose, 
\par 15 ir kviečia visus, kurie eina savo keliais: 
\par 16 “Neišmanėliai, ateikite!” Kam trūksta supratimo, ji sako: 
\par 17 “Vogtas vanduo yra saldesnis, o duona, valgoma slaptoje, skanesnė”. 
\par 18 Jie nežino, kad ten mirusieji, ir jos svečiai pragaro gelmėse.



\chapter{10}


\par 1 Saliamono patarlės. Išmintingas sūnus­džiaugsmas tėvui, kvailas sūnus­skausmas motinai. 
\par 2 Nedorybės turtai nepadeda, bet teisumas išgelbsti nuo mirties. 
\par 3 Viešpats neleidžia badauti teisiajam, bet nedorėlių užgaidų Jis nepatenkina. 
\par 4 Tingi ranka daro beturtį, o stropiojo ranka praturtina. 
\par 5 Išmintingas sūnus renka vasaros metu, bet sūnus, kuris miega pjūties metu, užtraukia gėdą. 
\par 6 Teisusis laiminamas, bet nedorėlio burną dengia smurtas. 
\par 7 Teisiųjų atminimas yra palaimintas, o nedorėlių vardas supus. 
\par 8 Išmintingas širdimi priims įstatymus, o tauškiantis kvailystes suklups. 
\par 9 Dorasis eina saugiais keliais, o kas iškraipo savo kelius, taps žinomas. 
\par 10 Kas mirkčioja akimis, sukelia nemalonumų, o tauškiantis kvailystes suklups. 
\par 11 Teisiojo burna yra gyvenimo šulinys, bet nedorėlio burną dengia smurtas. 
\par 12 Neapykanta sukelia vaidus, o meilė padengia visas nuodėmes. 
\par 13 Išmintis randama supratingojo lūpose, o neišmanėlio nugarai skirta rykštė. 
\par 14 Išmintingi kaupia žinojimą, o kvailio burna arti pražūties. 
\par 15 Turtuolio turtas yra jo tvirtovė, beturčių skurdas­jų pražūtis. 
\par 16 Teisiojo triūsas veda į gyvenimą, nedorėlio pasisekimas­į nuodėmę. 
\par 17 Kas priima pamokymus, eina gyvenimo keliu, o kas atmeta perspėjimus, klaidžioja. 
\par 18 Klastingos lūpos slepia neapykantą, kas platina šmeižtą, tas kvailys. 
\par 19 Žodžių gausumas nebūna be nuodėmės, kas tyli, tas išmintingas. 
\par 20 Teisiojo liežuvis yra rinktinis sidabras, o nedorėlio širdis nieko neverta. 
\par 21 Teisiojo lūpos pamaitina daugelį, kvailiai miršta dėl išminties stokos. 
\par 22 Viešpaties palaiminimas daro turtingą ir sielvarto neatneša. 
\par 23 Kvailiui daryti pikta­malonumas, o protingas žmogus turi išmintį. 
\par 24 Nedorėlis gaus, ko jis bijosi, teisusis gaus, ko trokšta. 
\par 25 Praeina audra, ir nebelieka nedorėlio, bet teisiojo pamatas amžinas. 
\par 26 Kaip actas dantims ir dūmai akims, taip tinginys tiems, kurie jį siunčia. 
\par 27 Viešpaties baimė pailgina gyvenimą, nedorėlio amžius bus sutrumpintas. 
\par 28 Teisiojo viltis teikia džiaugsmą, o nedorėlio lūkestis pražus. 
\par 29 Viešpaties kelias­stiprybė doriesiems ir pražūtis piktadariams. 
\par 30 Teisieji nesvyruos per amžius, bet nedorėliai negyvens žemėje. 
\par 31 Teisiojo burna kalba išmintį, o ydingas liežuvis bus atkirstas. 
\par 32 Teisieji kalba, kas naudinga, nedorėlio burna­kas ydinga.



\chapter{11}


\par 1 Viešpats nekenčia neteisingų svarstyklių, bet teisingos svarstyklės Jam patinka. 
\par 2 Kur ateina išdidumas, ten ateina ir gėda, o kur nuolankumas­ ten išmintis. 
\par 3 Dorųjų nekaltumas veda juos, o nusikaltėlių klastingumas juos sunaikins. 
\par 4 Turtai nepadeda rūstybės dieną, teisumas išgelbsti nuo mirties. 
\par 5 Nekaltojo teisumas nukreips jo kelią, nedorėlis žus dėl savo nedorybių. 
\par 6 Dorųjų teisumas išlaisvins juos, o nusikaltėlius sugaus jų pačių užgaidos. 
\par 7 Kai nedoras žmogus miršta, jo lūkesčiai pranyksta, ir bedievio viltis pražus. 
\par 8 Teisusis išlaisvinamas iš vargų, o nedorėlis atsiduria vietoje jo. 
\par 9 Veidmainis savo burna pražudo artimą, bet sumanumu teisusis išlaisvinamas. 
\par 10 Teisiųjų pasisekimu miestas džiaugiasi, o nedorėliams žuvus linksmai šūkaujama. 
\par 11 Teisiųjų laiminamas miestas kyla, nedorėlių burna jį sunaikina. 
\par 12 Kam trūksta išminties, tas niekina savo artimą, bet supratingas žmogus tyli. 
\par 13 Liežuvautojas atidengia paslaptis, o ištikimasis slepia, kas jam patikėta. 
\par 14 Be patarimo tauta pražūna, daug patarėjų suteikia saugumą. 
\par 15 Kas laiduoja už svetimąjį, nukentės; kas vengia laiduoti, saugus. 
\par 16 Maloni moteris įsigyja garbės, stiprieji krauna turtus. 
\par 17 Gailestingas žmogus daro gera savo sielai, žiaurus žmogus kenkia savo kūnui. 
\par 18 Nedorėlio darbas apgaulingas, kas sėja teisumą, tikrai gaus atlyginimą. 
\par 19 Teisumas veda į gyvenimą, o kas siekia pikto, siekia to savo pražūčiai. 
\par 20 Viešpats nekenčia veidmainių, bet dorieji Jam patinka. 
\par 21 Nors ir susijungtų, nedorėliai neišvengs bausmės, bet teisiojo palikuonys bus išgelbėti. 
\par 22 Kaip aukso žiedas kiaulės snukyje yra graži moteris, neturinti supratimo. 
\par 23 Teisiųjų troškimai geri, o nedorėlių viltis yra rūstybė. 
\par 24 Vieni dosniai dalina ir turtėja, kiti pasilaiko daugiau negu reikia, bet dar labiau nuskursta. 
\par 25 Dosni siela bus pasotinta, kas girdo, pats bus pagirdytas. 
\par 26 Kas neparduoda javų, tą keikia tauta; kas parduoda juos, laiminamas. 
\par 27 Kas stropiai ieško gera, sulauks palankumo; kas siekia pikto, pats to susilauks. 
\par 28 Kas pasitiki savo turtais, kris, o teisusis žaliuos kaip lapas. 
\par 29 Kas kelia nesantaiką savo namuose, paveldės vėjus; kvailys tarnaus išmintingam. 
\par 30 Teisiojo vaisius yra gyvybės medis, ir kas laimi sielas, tas išmintingas. 
\par 31 Jei teisusis gaus atlyginimą žemėje, tai tuo labiau nedorėlis ir nusidėjėlis.



\chapter{12}


\par 1 Kas mėgsta pamokymą, mėgsta išmintį; kas nepriima patarimo, tas bukaprotis. 
\par 2 Geras žmogus susilaukia Viešpaties palankumo, bet kuriantį nedorus planus Jis pasmerks. 
\par 3 Žmogus neįsitvirtins nedorybe, o teisiųjų šaknis nebus pajudinta. 
\par 4 Gera moteris yra vainikas jos vyrui, o ta, kuri užtraukia gėdą,­lyg puvinys kauluose. 
\par 5 Teisiojo mintys teisingos, nedorėlio patarimas­apgaulė. 
\par 6 Nedorėlių žodžiai: “Tykokime pralieti kraują”, bet teisiųjų burna išgelbės juos. 
\par 7 Nedorėliai parbloškiami ir jų nebėra, o teisiųjų namai stovės. 
\par 8 Žmogus vertinamas pagal išmintį, o ydingas širdyje bus paniekintas. 
\par 9 Kas niekinamas ir turi tarną, geresnis už tą, kuris didžiuojasi ir neturi duonos. 
\par 10 Teisusis rūpinasi savo gyvuliais, bet nedorėlio pasigailėjimas žiaurus. 
\par 11 Kas dirba savo žemę, turi pakankamai duonos, o kas seka tuštybe, tam trūksta proto. 
\par 12 Nedorėlis trokšta sugauti į piktadarystės tinklą, bet teisiųjų šaknis tvirta. 
\par 13 Nedorėlis įkliūna į savo lūpų nusikaltimus, bet teisusis išeis iš priespaudos. 
\par 14 Žmogus pasitenkins savo burnos vaisiumi, ir jam bus atlyginta pagal jo rankų darbą. 
\par 15 Kvailiui jo kelias atrodo teisingas, bet išmintingas žmogus klauso patarimo. 
\par 16 Kvailas tuojau parodo savo pyktį, bet nuovokus pridengia gėdą. 
\par 17 Kas kalba tiesą, tas padeda teisingumui, o neteisingas liudytojas apgaudinėja. 
\par 18 Yra tokių, kurių žodžiai lyg kardo dūriai, bet išmintingojo liežuvis gydo. 
\par 19 Tiesą kalbančios lūpos pasilieka per amžius, meluojantis liežuvis­tik akimirką. 
\par 20 Apgaulė­planuojančių pikta širdyje, bet taikos patarėjai turi džiaugsmą. 
\par 21 Nieko pikto neatsitiks teisiajam, bet nedorėlį lydės nelaimės. 
\par 22 Melagių nekenčia Viešpats, bet Jis mėgsta tuos, kurie elgiasi sąžiningai. 
\par 23 Nuovokus žmogus slepia pažinimą, o kvailio širdis skelbia kvailystes. 
\par 24 Darbštus valdys, o tinginys bus verčiamas dirbti. 
\par 25 Liūdesys žmogaus širdyje slegia jį, o geras žodis pralinksmina. 
\par 26 Teisusis pranoksta savo artimą, o nedorėlių kelias juos paklaidina. 
\par 27 Tinginys nekepa medžioklės laimikio, bet darbštumas yra brangus žmogaus turtas. 
\par 28 Teisumo kelyje­gyvenimas, jo takuose nėra mirties.



\chapter{13}


\par 1 Išmintingas sūnus klauso tėvo pamokymų, o pašaipūnas neklauso barimo. 
\par 2 Iš savo burnos vaisiaus žmogus valgys gėrybių, neištikimųjų siela­smurtą. 
\par 3 Kas saugo burną, saugo gyvybę; kas plačiai atveria lūpas, susilauks pražūties. 
\par 4 Tinginio siela geidžia, bet nieko neturi; darbščiojo siela pasisotins. 
\par 5 Teisusis nekenčia melo, bet nedorėlis yra atstumiantis ir susilaukia gėdos. 
\par 6 Teisumas saugo nekaltojo kelią, o nuodėmė pražudo nedorėlį. 
\par 7 Kai kas dedasi turtingas, bet nieko neturi, kitas dedasi vargšas, bet turi daug turtų. 
\par 8 Turtuolis gali išsipirkti savo turtu, bet vargšui niekas negrasina. 
\par 9 Teisiųjų šviesa šviečia, o nedorėlių žibintas užges. 
\par 10 Išdidumas sukelia ginčus, išmintingieji klauso patarimo. 
\par 11 Lengvai įgytas turtas greitai sunyksta, kas kaupia dirbdamas­ praturtėja. 
\par 12 Ilgai neišsipildanti viltis kankina širdį; patenkintas troškimas­ gyvybės medis. 
\par 13 Kas niekina žodį, pats save naikina, o kas bijo įsakymų, tam bus atlyginta. 
\par 14 Išmintingojo pamokymas yra gyvybės šaltinis, gelbstintis iš mirties pinklių. 
\par 15 Sveikas protas laimi palankumą, o neištikimųjų kelias sunkus. 
\par 16 Kiekvienas sumanus žmogus viską daro apgalvojęs, o kvailys viešai parodo savo kvailumą. 
\par 17 Nedoras pasiuntinys pakliūna į bėdą, o ištikimas neša išgelbėjimą. 
\par 18 Skurdas ir gėda tam, kuris nepriima pamokymo; kas priima įspėjimą, susilaukia pagarbos. 
\par 19 Patenkintas troškimas malonus sielai; kvailiui sunku šalintis nuo pikto. 
\par 20 Kas vaikšto su išmintingais, taps išmintingas, o kvailių bendrininkas pražus. 
\par 21 Nusidėjėlius persekioja nelaimės, o teisiesiems atlyginama gėrybėmis. 
\par 22 Geras žmogus palieka paveldėjimą vaikų vaikams, o nusidėjėlio turtas kaupiamas teisiajam. 
\par 23 Daug maisto būtų apleistoje vargšų žemėje, bet dėl teisingumo stokos žmonės pražūsta. 
\par 24 Kas gailisi rykštės, nekenčia savo sūnaus, bet kas jį myli, laiku jį baudžia. 
\par 25 Teisusis valgo ir pasitenkina, bet nedorėlio pilvas nepasisotina.



\chapter{14}


\par 1 Išmintinga moteris stato namus, o kvaila griauna juos savo rankomis. 
\par 2 Kas vaikšto tiesiu keliu, bijo Viešpaties, o kas mėgsta klaidžioti, niekina Jį. 
\par 3 Kvailio burnoje­išdidumo lazda, išmintingųjų lūpos juos apsaugo. 
\par 4 Kur nėra jaučių, ėdžios tuščios, bet gausus derlius gaunamas jaučių jėga. 
\par 5 Teisingas liudytojas nemeluoja, klastingas kalba melą. 
\par 6 Pašaipūnas ieško išminties ir neranda, bet supratingas lengvai įgyja pažinimą. 
\par 7 Pasitrauk nuo kvailio, kai pamatai, kad jo lūpose nėra pažinimo. 
\par 8 Išmintingas žmogus žino, ko siekia, o kvailys suklaidinamas savo kvailysčių. 
\par 9 Kvailys tyčiojasi iš nuodėmės, o teisusis atranda palankumą. 
\par 10 Širdis žino savo skausmą ir svetimasis nesidalina jos džiaugsmu. 
\par 11 Nedorėlio namai bus nugriauti, o teisiojo palapinė klestės. 
\par 12 Kartais kelias, kuris žmogui atrodo teisingas, nuveda į mirtį. 
\par 13 Ir juokiantis širdis gali liūdėti, o džiaugsmas baigtis sielvartu. 
\par 14 Nuklydęs širdimi pasisotins savo keliais, o geras žmogus­savo. 
\par 15 Neišmanėlis tiki kiekvienu žodžiu, bet išmintingas apsvarsto kiekvieną žingsnį. 
\par 16 Išmintingas žmogus bijo ir vengia pikto, o kvailys karščiuojasi ir pasitiki savimi. 
\par 17 Ūmus žmogus pasielgia kvailai, planuojantis pikta žmogus nekenčiamas. 
\par 18 Neišmanėlis paveldės kvailystę, o supratingąjį vainikuos išmintis. 
\par 19 Piktieji nusilenks geriesiems ir nedorėliai prie teisiųjų durų. 
\par 20 Beturčio nemėgsta net jo kaimynas, o turtingas turi daug draugų. 
\par 21 Kas niekina savo artimą, nusikalsta; kas pasigaili vargšo, tas palaimintas. 
\par 22 Klysta, kas daro pikta; kas siekia gero, sulauks pasigailėjimo ir tiesos. 
\par 23 Kiekvienas darbas yra pelningas, o tušti plepalai veda į skurdą. 
\par 24 Išmintingą vainikuoja turtas, o kvailio kvailystė ir lieka kvailyste. 
\par 25 Teisingas liudytojas išgelbsti sielas, apgaulingas kalba melą. 
\par 26 Viešpaties baimėje tvirtas pasitikėjimas, ir Jo vaikai turės kur prisiglausti. 
\par 27 Viešpaties baimė­gyvenimo šaltinis, apsaugantis nuo mirties pinklių. 
\par 28 Gausi tauta­garbė karaliui, o be žmonių žlunga kunigaikštis. 
\par 29 Kas lėtas pykti, yra išmintingas, o nesusivaldantis parodo kvailumą. 
\par 30 Sveika širdis­kūno gyvybė, o pavydas pūdo kaulus. 
\par 31 Kas skriaudžia vargšą, paniekina jo Kūrėją; kas gerbia Jį, pasigaili beturčio. 
\par 32 Nedorėlis bus atmestas dėl savo piktų darbų, o teisusis ir mirdamas turi viltį. 
\par 33 Supratingojo širdyje ilsisi išmintis, o tai, kas yra tarp kvailių, tampa žinoma. 
\par 34 Teisumas iškelia tautą, o nuodėmė yra negarbė tautoms. 
\par 35 Išmintingas tarnas įgyja karaliaus palankumą; kas užtraukia gėdą, susilauks jo rūstybės.



\chapter{15}


\par 1 Švelnus atsakymas nuramina pyktį, aštrūs žodžiai sukelia rūstybę. 
\par 2 Išmintingojo liežuvis tinkamai naudoja pažinimą, o iš kvailio burnos liejasi kvailystės. 
\par 3 Viešpaties akys mato visur, jos stebi blogus ir gerus. 
\par 4 Maloni kalba­gyvybės medis, šiurkšti šneka prislegia dvasią. 
\par 5 Kvailys paniekina savo tėvo pamokymus, o kas klauso perspėjimų, yra supratingas. 
\par 6 Turtų netrūksta teisiojo namuose, o nedorėlio pelnas­tik rūpesčiai. 
\par 7 Išmintingųjų lūpos skelbia pažinimą, o kvailio širdis to nedaro. 
\par 8 Nedorėlio auka­pasibjaurėjimas Viešpačiui, bet teisiųjų maldos Jam patinka. 
\par 9 Nedorėlio kelias­pasibjaurėjimas Viešpačiui, bet Jis myli tuos, kurie siekia teisumo. 
\par 10 Didelė bausmė tam, kuris pameta kelią; kas nekenčia įspėjimo, miršta. 
\par 11 Pragaras ir prapultis Viešpaties akivaizdoje, tuo labiau žmonių širdys. 
\par 12 Niekintojas nemėgsta to, kuris jį įspėja, ir neina pas išminčius. 
\par 13 Linksma širdis atsispindi veide, širdies skausmas slegia dvasią. 
\par 14 Protingas ieško pažinimo, neišmanėlis maitinasi kvailystėmis. 
\par 15 Prislėgtasis nemato šviesių dienų, kas turi linksmą širdį, tam visuomet šventė. 
\par 16 Geriau mažai su Viešpaties baime negu dideli turtai su rūpesčiu. 
\par 17 Geriau daržovių pietūs, kur yra meilė, negu nupenėtas veršis ten, kur neapykanta. 
\par 18 Piktas žmogus sukelia vaidus, o lėtas pykti juos nuramina. 
\par 19 Tinginio kelias pilnas erškėčių, o teisiojo kelias­platus. 
\par 20 Išmintingas sūnus yra džiaugsmas tėvui, kvailys paniekina savo motiną. 
\par 21 Neprotingas džiaugiasi kvailybe; protingasis eina tiesiu keliu. 
\par 22 Be patarimų sumanymai nueina niekais, bet kur daug patarėjų, jie įtvirtinami. 
\par 23 Žmogus džiaugiasi savo burnos atsakymu, ir laiku pasakytas žodis, koks jis mielas! 
\par 24 Išmintingo žmogaus gyvenimo kelias kyla į viršų, ir jis išvengia pragaro. 
\par 25 Viešpats sugriaus išdidžiųjų namus, bet įtvirtins našlės nuosavybę. 
\par 26 Nedorėlio mintys­pasibjaurėjimas Viešpačiui, bet nekaltųjų žodžius Jis mėgsta. 
\par 27 Kas godus pelno, vargina savo namus; kas nepriima kyšių, bus gyvas. 
\par 28 Teisiojo širdis apsvarsto, kaip atsakyti, o nedorėlio burna beria piktus žodžius. 
\par 29 Viešpats toli nuo nedorėlių, bet teisiųjų maldas Jis išklauso. 
\par 30 Akių šviesa pradžiugina širdį; gera žinia­sveikata kaulams. 
\par 31 Kas klausosi gyvenimo pabarimų, liks tarp išmintingųjų. 
\par 32 Kas nekreipia dėmesio į patarimus, kenkia pats sau; kas paklauso įspėjimų, įsigyja daugiau supratimo. 
\par 33 Viešpaties baimė moko išminties, prieš pagerbimą eina nuolankumas.



\chapter{16}


\par 1 Žmogus paruošia savo širdį, bet nuo Viešpaties priklauso jo burnos atsakymas. 
\par 2 Visi žmogaus keliai atrodo geri jo paties akyse, bet Viešpats pasveria dvasią. 
\par 3 Pavesk savo darbus Viešpačiui, ir tavo sumanymai pasiseks. 
\par 4 Viešpats viską sukūrė dėl savęs, net ir nedorėlį nelaimės dienai. 
\par 5 Viešpats bjaurisi visais, kurie išdidūs širdyje; nors ir susijungtų, jie neišvengs bausmės. 
\par 6 Gailestingumu ir tiesa apvaloma nuo kaltės, Viešpaties baimė padeda išvengti pikto. 
\par 7 Kai žmogaus keliai patinka Viešpačiui, Jis padaro, kad ir jo priešai gyventų taikoje su juo. 
\par 8 Geriau mažai su teisumu negu didelis pelnas nesąžiningai. 
\par 9 Žmogaus širdis planuoja savo kelią, bet Viešpats nukreipia jo žingsnius. 
\par 10 Dieviškas sprendimas karaliaus lūpose, ir jo burna nenusikalsta teisme. 
\par 11 Teisingi svarsčiai ir svarstyklės yra Viešpaties, visi svarsčiai maišelyje yra Jo darbas. 
\par 12 Karalius bjaurisi nedorybėmis, nes jo sostas įtvirtintas teisingumu. 
\par 13 Karaliams patinka teisios lūpos, ir jie mėgsta tuos, kurie kalba tiesą. 
\par 14 Karaliaus pyktis yra mirties pasiuntinys, bet išmintingas žmogus gali jį nuraminti. 
\par 15 Karaliaus veido šviesoje gyvenimas, jo palankumas kaip vėlyvasis lietus. 
\par 16 Išmintis daug vertingesnė už auksą, supratimas­už sidabrą. 
\par 17 Teisiojo kelias­šalintis nuo pikto; kas laikosi kelio, išsaugo gyvybę. 
\par 18 Išdidumas eina sunaikinimo priekyje, puikybė­prieš žlugimą. 
\par 19 Geriau būti nuolankios dvasios su romiaisiais, negu dalintis grobį su išdidžiaisiais. 
\par 20 Kas išmintingai tvarko reikalus, susilauks sėkmės; palaimintas, kuris pasitiki Viešpačiu. 
\par 21 Išmintingas širdyje bus vadinamas sumaniu, ir švelniais žodžiais lengviau įtikinti. 
\par 22 Supratimas yra gyvybės šaltinis tam, kas jį turi; kvailių pamokymas­kvailystė. 
\par 23 Išmintingojo širdis moko jo burną ir prideda išmanymo jo lūpoms. 
\par 24 Malonūs žodžiai yra kaip medus­saldūs sielai ir sveiki kūnui. 
\par 25 Kartais kelias, kuris žmogui atrodo teisingas, nuveda į mirtį. 
\par 26 Kas dirba, dirba dėl savęs, nes jo burna verčia jį. 
\par 27 Bedievis žmogus iškasa blogį, ir jo lūpose tarsi deganti ugnis. 
\par 28 Ydingas žmogus sukelia vaidus, ir plepys išskiria draugus. 
\par 29 Smurtininkas vilioja savo artimą ir veda jį blogu keliu. 
\par 30 Jis užsimerkia, planuodamas niekšybę, prikandęs lūpas, vykdo savo piktus sumanymus. 
\par 31 Žili plaukai­šlovės vainikas, jei įgytas teisumo kelyje. 
\par 32 Lėtas pykti yra geresnis už galiūną, susivaldantis­už tą, kuris užima miestą. 
\par 33 Burtas metamas į skverną, bet jo išsidėstymas priklauso nuo Viešpaties.



\chapter{17}


\par 1 Geriau sausas kąsnis su ramybe negu namai, pilni aukų mėsos, su vaidais. 
\par 2 Išmintingas tarnas valdys gėdą darantį sūnų ir gaus paveldėti dalį kaip vienas iš sūnų. 
\par 3 Kaip sidabras ir auksas ištiriamas ugnyje, taip Viešpats tiria žmogaus širdį. 
\par 4 Nedorėlis klauso klastingų lūpų, o melagis­pikto liežuvio. 
\par 5 Kas pajuokia vargšą, paniekina jo Kūrėją, o kas džiaugiasi nelaimės metu, neišvengs bausmės. 
\par 6 Vaikų vaikai yra senelių vainikas, o vaikų garbė­jų tėvai. 
\par 7 Kvailiui netinka kalbėti apie didžius dalykus, tuo labiau kunigaikščiui netinka meluoti. 
\par 8 Dovana lyg brangakmenis davėjo akyse; kur jis eina su ja, visur laimi. 
\par 9 Kas pridengia nusikaltimą, ieško meilės; kas kaltę nuolat primena, suardo draugystę. 
\par 10 Įspėjimas daugiau padeda išmintingam negu šimtas kirčių kvailam. 
\par 11 Blogas žmogus ieško priekabių, todėl žiaurus pasiuntinys bus pasiųstas prieš jį. 
\par 12 Geriau sutikti mešką, kuriai atimti jos vaikai, negu kvailį jo kvailystėje. 
\par 13 Kas už gera atlygina piktu, pikta neatsitrauks nuo jo namų. 
\par 14 Kivirčo pradžia kaip užtvankos plyšys, todėl liaukis ginčytis, kol nevėlu. 
\par 15 Kas išteisina nedorėlį ir kas pasmerkia teisųjį, abu yra pasibjaurėjimas Viešpačiui. 
\par 16 Kam kvailiui mokėti pinigus už išmintį, kai jis jos visai netrokšta? 
\par 17 Draugas visuomet myli ir brolis pasirodo nelaimėje. 
\par 18 Kam trūksta proto, tas sukerta rankas ir laiduoja už savo draugą. 
\par 19 Kas mėgsta ginčus, myli nuodėmę, kas stato aukštus vartus, ieško pražūties. 
\par 20 Kas turi klastingą širdį, nieko gero nepasieks; kieno liežuvis iškreiptas, turės bėdų. 
\par 21 Kvailas sūnus­ne džiaugsmas, bet rūpestis tėvui. 
\par 22 Linksma širdis gydo kaip vaistai; prislėgta dvasia džiovina kaulus. 
\par 23 Nedorėlis ima kyšius, kad iškreiptų teisingumą. 
\par 24 Supratingas žmogus siekia išminties, kvailio akys žemės pakraščiuose. 
\par 25 Kvailas sūnus­apmaudas tėvui ir skausmas motinai. 
\par 26 Bausti teisųjį yra negerai, kaip ir mušti kunigaikščius už teisingumą. 
\par 27 Kas turi supratimą, susilaiko kalboje, protingas žmogus turi romią dvasią. 
\par 28 Net kvailys, jei jis tyli, laikomas išmintingu, ir kas sučiaupia lūpas, laikomas protingu.



\chapter{18}


\par 1 Dėl savo užgaidų žmogus atsiskiria nuo kitų ir prieštarauja tam, kas teisinga. 
\par 2 Kvailiui nerūpi išmintis, o tik parodyti, kas yra jo širdyje. 
\par 3 Kai ateina nedorėlis, ateina ir panieka, o su nešlove ateina gėda. 
\par 4 Žmogaus žodžiai yra gilus vanduo, išminties šaltinis­tekanti srovė. 
\par 5 Negerai būti šališku nedorėliui ir teisme nuskriausti teisųjį. 
\par 6 Kvailio lūpos sukelia vaidus, ir jis savo burna prašosi mušamas. 
\par 7 Kvailio burna jį pražudo; jo lūpos­pinklės jo sielai. 
\par 8 Apkalbos yra lyg skanėstas, kuris pasiekia žmogaus vidurius. 
\par 9 Kas tingi dirbti, yra naikintojo brolis. 
\par 10 Viešpaties vardas­tvirtas bokštas, teisieji bėga į jį ir yra saugūs. 
\par 11 Turtuolio lobis yra jo įtvirtintas miestas, ir jo nuosavybė yra kaip aukšta siena jam. 
\par 12 Prieš sunaikinimą žmogaus širdis pasididžiuoja, prieš pagerbimą eina nuolankumas. 
\par 13 Kas atsako, iki galo neišklausęs, tas kvailas ir begėdis. 
\par 14 Žmogaus dvasia palaiko jį negalioje, bet kas pakels sužeistą dvasią? 
\par 15 Supratingojo širdis įgyja pažinimo ir išmintingojo ausis pažinimo ieško. 
\par 16 Dovanos plačiai atidaro žmogui duris pas didžiūnus. 
\par 17 Pirmasis, kalbąs teisme, atrodo teisus, kol ateina jo kaimynas ir išklausinėja jį. 
\par 18 Metant burtus pašalinami nesutarimai ir padaromas sprendimas tarp galingųjų. 
\par 19 Įžeistas brolis yra kaip įtvirtintas miestas; ginčai atskiria lyg pilių užkaiščiai. 
\par 20 Žmogus pripildys pilvą savo burnos vaisiais, pasisotins savo lūpų derliumi. 
\par 21 Mirtis ir gyvenimas yra liežuvio galioje; kas jį mėgsta, valgys jo vaisių. 
\par 22 Kas randa gerą žmoną, randa laimės ir Viešpaties palankumą. 
\par 23 Beturtis kalba maldaudamas, o turtingas atsako šiurkščiai. 
\par 24 Kas nori turėti draugų, turi pats būti draugiškas; būna draugų, artimesnių už brolį.



\chapter{19}


\par 1 Geriau beturtis, vaikštantis savo nekaltume, negu tas, kuris turi klastingą liežuvį ir yra kvailas. 
\par 2 Negerai sielai be pažinimo, ir kas skuba, tas nusideda. 
\par 3 Žmogaus kvailumas iškreipia jo kelius, bet jo širdis kaltina Viešpatį. 
\par 4 Turtas pritraukia daug draugų, o nuo beturčio nusisuka ir jo artimas. 
\par 5 Neteisingas liudytojas neliks nenubaustas; kas meluoja, neišsisuks. 
\par 6 Daugelis pataikauja kunigaikščiams, ir visi yra draugai su tuo, kas duoda dovanų. 
\par 7 Beturčio broliai nekenčia jo, tuo labiau jį palieka jo draugai. Kaip jis beprašytų, jie šalinasi nuo jo. 
\par 8 Kas įgyja išminties, myli savo sielą; kas laikosi supratimo, tam seksis. 
\par 9 Neteisingas liudytojas neliks nenubaustas; kas meluoja, pražus. 
\par 10 Kvailiui netinka prabanga nei vergui valdyti kunigaikščius. 
\par 11 Įžvalgus žmogus nesikarščiuoja ir jam yra garbė nekreipti dėmesio į neteisybę. 
\par 12 Karaliaus rūstybė yra kaip liūto riaumojimas, o jo palankumas­ kaip rasa augalams. 
\par 13 Kvailas sūnus­nelaimė tėvui, o vaidinga žmona­nesiliaująs lašėjimas. 
\par 14 Namai ir turtai paveldimi iš tėvų, o išmintinga žmona­nuo Viešpaties. 
\par 15 Tinginys įpranta ilgai miegoti, ir dykinėjanti siela kęs alkį. 
\par 16 Kas laikosi įsakymų, palaiko savo sielą; kas paniekina Jo kelius, mirs. 
\par 17 Kas pasigaili vargšo, skolina Viešpačiui; Jis atlygins jam už jo darbus. 
\par 18 Bausk sūnų, kol yra vilties, ir nepaliauk dėl jo šauksmo. 
\par 19 Kas greitas pykti, susilauks bausmės; jei tu jį išgelbėsi, turėsi tai daryti iš naujo. 
\par 20 Klausykis patarimo ir priimk pamokymą, kad ateityje būtum išmintingesnis. 
\par 21 Daug sumanymų žmogaus širdyje, bet tik Viešpaties valia įvyksta. 
\par 22 Iš žmogaus norima gerumo. Beturtis yra vertesnis už melagį. 
\par 23 Viešpaties baimė teikia gyvenimą; kas ją turi, tas yra patenkintas, nelaimė jo nepalies. 
\par 24 Tinginys įkiša savo ranką į dubenį, bet nebenori pakelti jos prie burnos. 
\par 25 Jei nubausi niekintoją, neišmanėlis taps atsargesnis; jei pabarsi išmintingą, jis supras pamokymą. 
\par 26 Kas blogai elgiasi su tėvu ir išvaro motiną, tas užsitrauks gėdą ir panieką. 
\par 27 Mano sūnau, neklausyk pamokymų, kurie atitraukia nuo pažinimo žodžių. 
\par 28 Bedievis liudytojas tyčiojasi iš teismo; nedorėlių burna ryja neteisybę. 
\par 29 Teismai laukia niekintojų ir rykštės paruoštos kvailių nugaroms.



\chapter{20}


\par 1 Vynas­pasityčiotojas, o stiprus gėrimas­pašėlęs; kas jais apsigauna, nėra išmintingas. 
\par 2 Karaliaus rūstybė yra kaip liūto riaumojimas; kas jį erzina, nusideda prieš savo gyvybę. 
\par 3 Garbė žmogui vengti ginčų, bet kiekvienas kvailys įsivelia į juos. 
\par 4 Tinginys nearia dėl šalčio, todėl derliaus metu elgetaus ir nieko neturės. 
\par 5 Patarimas žmogaus širdyje yra kaip gilus vanduo, sumanus žmogus jį semia. 
\par 6 Daugelis žmonių skelbia savo gerumą, bet kas suras ištikimą žmogų? 
\par 7 Teisus žmogus vaikšto savo nekaltume, ir jo vaikai bus palaiminti. 
\par 8 Karalius, sėdėdamas teismo soste, savo akimis išsklaido visą blogį. 
\par 9 Kas gali sakyti: “Apvaliau savo širdį, esu be nuodėmės”? 
\par 10 Skirtingi svarsčiai ir skirtingi saikai yra pasibjaurėjimas Viešpaties akyse. 
\par 11 Jau vaiką galima pažinti iš jo poelgių, ar jo darbai bus tyri ir teisingi. 
\par 12 Viešpats sutvėrė girdinčią ausį ir matančią akį. 
\par 13 Kas mėgsta miegoti, tampa beturtis; atsimerk ir turėsi pakankamai maisto. 
\par 14 Pirkėjas sako: “Niekam tikę, niekam tikę!”, o nuėjęs savo keliu giriasi. 
\par 15 Yra aukso ir daugybė perlų, bet pažinimo lūpos yra didelė brangenybė. 
\par 16 Paimk apdarą iš to, kuris laidavo už svetimą, ir turėk užstatą iš suvedžiotojo. 
\par 17 Skani žmogui duona, apgaule įgyta, bet paskui jo burna yra pilna žvyro. 
\par 18 Sumanymai įtvirtinami patarimais, ir kariauti nepradėk neapsvarstęs. 
\par 19 Kas vaikšto plepėdamas, atidengia paslaptis, todėl nesusidėk su pataikūnais. 
\par 20 Kas keikia tėvą ir motiną, to žiburys užges visiškoje tamsoje. 
\par 21 Paveldėjimas, gautas paskubomis, galiausiai nebus palaimintas. 
\par 22 Nesakyk: “Atlyginsiu už pikta”. Lauk Viešpaties, ir Jis padės tau. 
\par 23 Skirtingi svarsčiai yra pasibjaurėjimas Viešpačiui ir neteisingos svarstyklės nėra gerai. 
\par 24 Viešpats veda žmogų, kaip tad gali žmogus suprasti savo kelią? 
\par 25 Spąstai žmogui­padaryti įžadą, o po to galvoti. 
\par 26 Išmintingas karalius išblaško nedorėlius ir juos smarkiai baudžia. 
\par 27 Žmogaus dvasia yra Viešpaties žiburys, tiriantis širdies gelmes. 
\par 28 Gailestingumas ir tiesa apsaugo karalių, ir jo sostas palaikomas gailestingumu. 
\par 29 Jaunuolių garbė­jėga, o senuosius puošia žili plaukai. 
\par 30 Kirčių žymės pašalina pikta ir randai išvalo žmogaus širdį.



\chapter{21}


\par 1 Karaliaus širdis Viešpaties rankoje kaip vandens srovės; Jis ją pasuka, kur nori. 
\par 2 Visi žmogaus keliai atrodo teisingi jo paties akyse, bet Viešpats pasveria širdį. 
\par 3 Tiesa ir teisingumas Viešpačiui mieliau negu auka. 
\par 4 Išdidus žvilgsnis ir pasipūtusi širdis, kurie išskiria nedorėlį, yra nuodėmė. 
\par 5 Stropiojo sumanymai veda į apstybę, o skuboti­į nuostolį. 
\par 6 Melu įsigyti turtą yra tuščios svajonės tų, kurie ieško mirties. 
\par 7 Nedorėlių smurtas sunaikins juos pačius, nes jie nedaro to, kas teisinga. 
\par 8 Nusikaltėlio kelias yra vingiuotas, nekaltojo darbai teisūs. 
\par 9 Geriau yra gyventi palėpės kampe negu su vaidinga moterimi dideliuose namuose. 
\par 10 Nedorėlio siela trokšta pikto, jis nesigaili artimo. 
\par 11 Kai nubaudžiamas niekintojas, neišmanėlis tampa išmintingas. Kai išmintingas pamokomas, jis įgyja supratimo. 
\par 12 Teisusis stebi nedorėlių namus ir mato, kaip nedorėlis parbloškiamas už savo nedorybes. 
\par 13 Kas neklauso vargšo šauksmo, pats šauks, bet nebus išgirstas. 
\par 14 Slapta dovana nuramina pyktį, o dovana į antį­stiprią rūstybę. 
\par 15 Teisingumas džiugina teisiuosius, o piktadarius išgąsdina. 
\par 16 Nuklydę nuo tiesos kelio atsidurs mirusiųjų susirinkime. 
\par 17 Kas mėgsta linksmybes, bus vargšas, kas myli vyną ir aliejų, nepraturtės. 
\par 18 Nedorėlis bus išpirka už teisųjį, nusikaltėlis­už nekaltąjį. 
\par 19 Geriau yra vienam gyventi dykumoje negu su pikta moterimi, mėgstančia barnius. 
\par 20 Išmintingo žmogaus namuose yra brangių daiktų ir aliejaus, kvailys iššvaisto juos. 
\par 21 Kas seka teisumą ir gailestingumą, suranda gyvenimą, teisumą ir garbę. 
\par 22 Išmintingasis užima stipriųjų miestą ir sugriauna tvirtovę, kuria jie pasitikėjo. 
\par 23 Kas saugo burną ir liežuvį, saugo savo sielą nuo nemalonumų. 
\par 24 Išdidus ir pasipūtęs vadinamas niekintoju, jis elgiasi akiplėšiškai ir įžūliai. 
\par 25 Tinginio troškimas nužudo jį, nes jis nenori dirbti. 
\par 26 Jis godžiai geidžia visą dieną, o teisusis duoda negailėdamas. 
\par 27 Nedorėlio auka Viešpats bjaurisi, tuo labiau, jei ji aukojama klastinga širdimi. 
\par 28 Neteisingas liudytojas žus, o kas girdi, kalbės be perstojo. 
\par 29 Nedorėlis suraukia savo veidą, o dorojo kelias tiesus. 
\par 30 Prieš Viešpatį neatsilaikys nei išmintis, nei supratimas, nei patarimas. 
\par 31 Žirgas ruošiamas kovos dienai, bet pergalę teikia Viešpats.



\chapter{22}


\par 1 Geras vardas yra vertingesnis už didelius turtus, o palankumas­už sidabrą ir auksą. 
\par 2 Turtuolis ir vargšas turi bendra: juos abu sutvėrė Viešpats. 
\par 3 Supratingas numato pavojų ir pasislepia, o neišmanėlis eina ir nukenčia. 
\par 4 Nusižeminimas ir Viešpaties baimė atneša turtus, garbę ir gyvenimą. 
\par 5 Veidmainio kelias pilnas erškėčių ir žabangų; kas saugo savo gyvybę, išvengs jų. 
\par 6 Parodyk vaikui kelią, kuriuo jis turi eiti, tai ir pasenęs jis nenukryps nuo jo. 
\par 7 Turtingas viešpatauja vargšui; skolininkas tampa skolintojo vergu. 
\par 8 Kas sėja neteisybę, pjauna nelaimes; jo pykčio rykštė plaka jį. 
\par 9 Dosnus žmogus bus palaimintas, nes jis dalinasi savo duona su beturčiu. 
\par 10 Išmesk niekintoją, ir liausis vaidai, barniai ir priekaištai. 
\par 11 Kas mėgsta širdies tyrumą ir yra maloningas kalboje, tas draugaus su karaliumi. 
\par 12 Viešpats saugo pažinimą, bet neištikimojo žodžius Jis paverčia niekais. 
\par 13 Tinginys sako: “Liūtas yra lauke, jis sudraskys mane gatvėje!” 
\par 14 Svetimos moters lūpos­gili duobė; tas, kuriuo Viešpats bjaurisi, įkris į ją. 
\par 15 Kvailystė prisirišusi prie vaiko širdies, bet pamokymo rykštė išvaro ją. 
\par 16 Kas skriaudžia beturtį, norėdamas praturtėti, ir kas duoda turtingam, pats nuskurs. 
\par 17 Atidžiai klausykis išminčių žodžių, palenk savo širdį prie mano pažinimo. 
\par 18 Tau bus malonu laikyti juos savo širdyje, ir jie tiks tavo lūpose. 
\par 19 Šiandien tave mokau, kad tu galėtum pasitikėti Viešpačiu. 
\par 20 Ar aš neužrašiau tau prakilnių dalykų apie patarimus ir pažinimą, 
\par 21 kad pamokyčiau tave tiesos žodžių tikrumo ir tu galėtum duoti atsakymą tiems, kurie klaus tavęs? 
\par 22 Neapiplėšk beturčių dėl to, kad jie yra beturčiai, ir neskriausk nukentėjusiojo teisme, 
\par 23 nes Viešpats gins jų bylą ir išplėš sielą tų, kurie juos plėšė. 
\par 24 Nedraugauk su pikčiurna ir neik su ūmiu žmogumi, 
\par 25 kad neišmoktum jo kelių ir nepastatytum spąstų savo sielai. 
\par 26 Nebūk iš tų, kurie paduoda ranką, laiduodami už svetimą skolą; 
\par 27 jei negalėsi sumokėti, kodėl iš tavęs turėtų atimti tavo guolį? 
\par 28 Nepakeisk senų žemės ribų, kurias tavo tėvai nustatė. 
\par 29 Ar matei stropaus žmogaus darbą? Jis stovės prieš karalių, jam nereikės stovėti prieš paprastus žmones.



\chapter{23}


\par 1 Kai sėdiesi valgyti su valdovu, rūpestingai stebėk, kas prieš tave padėta. 
\par 2 Prisidėk peilį prie gerklės, jei mėgsti skaniai pavalgyti. 
\par 3 Negeisk jo skanėstų, nes tai apgaulingas maistas. 
\par 4 Nepersidirbk siekdamas pralobti, būk išmintingas ir liaukis. 
\par 5 Nežiūrėk į tai, ko nėra, nes turtai pasidaro sparnus ir išskrenda kaip erelis į padangę. 
\par 6 Nevalgyk pas šykštuolį nei duonos, nei jo skanumynų, 
\par 7 nes kaip jis galvoja savo širdyje, toks jis ir yra. Nors jis tave ragina valgyti ir gerti, bet širdyje pavydi. 
\par 8 Tu išvemsi, ką suvalgei, ir veltui kalbėsi gražius žodžius. 
\par 9 Nekalbėk kvailam girdint, nes jis paniekins tavo išmintingus žodžius. 
\par 10 Nepakeisk senų žemės ribų ir nepasisavink našlaičio lauko, 
\par 11 nes jų Atpirkėjas yra galingas­ Jis gins jų bylą prieš tave. 
\par 12 Palenk savo širdį į pamokymus ir savo ausis į pažinimo žodžius. 
\par 13 Nepalik vaiko nenubausto, nes jei suduosi jam rykšte, jis nemirs. 
\par 14 Tu nubausi jį rykšte ir išgelbėsi jo sielą nuo pragaro. 
\par 15 Mano sūnau, jei būsi išmintingas, suteiksi man daug džiaugsmo. 
\par 16 Tau tiesą kalbant, mano širdis džiūgaus. 
\par 17 Nepavydėk nusidėjėliams, bet bijok Viešpaties per visą savo dieną. 
\par 18 Galas tikrai yra, ir tavo viltis nebus tuščia. 
\par 19 Mano sūnau, klausyk ir būk išmintingas, tiesiu keliu vesk savo širdį. 
\par 20 Nebūk su girtuokliais ir nevalgyk su besočiais. 
\par 21 Girtuokliai ir rajūnai nuskurs, o mieguistumas aprengs skarmalais. 
\par 22 Klausyk savo tėvo ir nepaniekink savo motinos, kai ji pasensta. 
\par 23 Pirk tiesą ir neparduok jos, o taip pat išmintį, pamokymą ir supratimą. 
\par 24 Teisiojo tėvas džiūgauja ir, pagimdęs išmintingą sūnų, džiaugsis juo. 
\par 25 Te tavo tėvas ir motina bus patenkinti ir džiaugsis ta, kuri tave pagimdė. 
\par 26 Mano sūnau, duok man savo širdį ir stebėk mano kelius. 
\par 27 Paleistuvė yra gili duobė, ir svetima moteris­siauras šulinys. 
\par 28 Ji tykoja grobio ir daugina neištikimų vyrų skaičių. 
\par 29 Kas vargsta? Kas rūpinasi? Kas skundžiasi? Kas gauna kirčių be priežasties? Kieno paraudusios akys? 
\par 30 Tie, kurie ilgai sėdi prie vyno ir geria maišytą vyną. 
\par 31 Nežiūrėk į vyną, kad jis raudonas, spindi stikle ir švelniai nuryjamas! 
\par 32 Galiausiai jis įgelia kaip gyvatė ir suleidžia nuodus kaip angis. 
\par 33 Tada tavo akys nukryps į svetimas moteris, ir tavo širdis kalbės iškrypusius dalykus. 
\par 34 Tu būsi lyg miegantis viduryje jūros, lyg snaudžiantis laivo stiebo viršūnėje. 
\par 35 Tu sakysi: “Jie sudavė man, bet nesužeidė, jie mušė mane, bet aš nejaučiau. Kai aš pabusiu, vėl gersiu”.



\chapter{24}


\par 1 Nepavydėk piktiems žmonėms ir nenorėk būti su jais, 
\par 2 nes jie mąsto apie smurtą ir kalba apie apgaulę. 
\par 3 Namai statomi išmintimi ir įtvirtinami supratimu. 
\par 4 Pažinimu jie pripildomi visokių brangių ir vertingų turtų. 
\par 5 Išmintingas žmogus yra stiprus, ir pažinimas padidina jėgas. 
\par 6 Kariaudamas klausyk išmintingų patarimų; daug patarėjų suteikia saugumą. 
\par 7 Kvailiui išmintis nepasiekiama, pasitarimuose jis neatveria burnos. 
\par 8 Kas planuoja daryti pikta, bus vadinamas nedoru žmogumi. 
\par 9 Planuoti kvailystes yra nuodėmė; žmonės bjaurisi niekintoju. 
\par 10 Kas palūžta nelaimės metu, tas silpnas. 
\par 11 Išlaisvink tuos, kurie vedami mirti, išgelbėk pasmerktus nužudyti. 
\par 12 Ar sakysi, kad to nežinojai? Ar širdžių Tyrėjas nežino? Jis stebi tavo sielą ir visa žino, ir atlygins žmogui pagal jo darbus. 
\par 13 Mano sūnau, valgyk medų, nes jis geras, ir korį, nes jis saldus tavo liežuviui. 
\par 14 Taip ir išminties pažinimas tavo sielai: suradęs ją, turi ateitį, tavo viltis nebus tuščia. 
\par 15 Netykok, nedorėli, prie teisiojo namų ir nedrumsk jam ramybės. 
\par 16 Teisusis septynis kartus krinta ir vėl atsikelia, bet nedorėlis įpuls į pražūtį. 
\par 17 Nesidžiauk, kai tavo priešas krinta; nedžiūgauk savo širdyje, kai jis suklumpa, 
\par 18 kad Viešpats pamatęs nenukreiptų nuo jo savo rūstybės, nes tai Jam nepatinka. 
\par 19 Nesijaudink dėl piktadarių ir nepavydėk nedorėliams. 
\par 20 Piktadarys neturi ateities, o nedorėlio žiburys užges. 
\par 21 Mano sūnau, bijok Viešpaties ir karaliaus, nesusidėk su maištininkais. 
\par 22 Jų pražūtis ateis netikėtai, ir kas žino, ko jie susilauks nuo tų dviejų. 
\par 23 Tai taip pat išmintingiems. Būti šališkam teisme yra negerai. 
\par 24 Kas sako nedorėliui, kad jis teisus, tą keiks jo tauta ir juo bjaurėsis. 
\par 25 Tie, kurie jį sudraus, bus mėgstami, ir palaiminimai užgrius juos. 
\par 26 Tinkamas atsakymas yra kaip pabučiavimas. 
\par 27 Atlik darbus laukuose, paruošk tinkamai dirvą ir tada statyk savo namus. 
\par 28 Neliudyk neteisingai prieš savo artimą ir neapgaudinėk. 
\par 29 Nesakyk: “Kaip jis man padarė, taip aš jam padarysiu; aš atlyginsiu jam”. 
\par 30 Aš ėjau pro tinginio lauką ir neišmanėlio vynuogyną. 
\par 31 Visur augo erškėčiai ir buvo pilna dilgėlių, o akmeninė tvora buvo apgriuvus. 
\par 32 Aš žiūrėjau, apsvarsčiau ir pasimokiau: 
\par 33 “Truputį pamiegosi, truputį pasnausi, truputį pagulėsi, sudėjęs rankas, 
\par 34 ir ateis skurdas kaip pakeleivis ir nepriteklius kaip ginkluotas plėšikas”.



\chapter{25}


\par 1 Tai taip pat Saliamono patarlės, kurias surinko Judo karaliaus Ezekijo vyrai. 
\par 2 Dievo šlovė­nuslėpti dalyką, karaliaus garbė­ištirti dalyką. 
\par 3 Kaip dangaus aukštybės ir žemės gilybės, taip neištiriama karaliaus širdis. 
\par 4 Pašalink priemaišas iš sidabro, ir sidabrakalys padarys iš jo indą. 
\par 5 Pašalink nedorėlį iš karaliaus akivaizdos, ir jo sostas įsitvirtins teisingume. 
\par 6 Nesiaukštink karaliaus akivaizdoje, nestok didžiūnų vieton. 
\par 7 Geriau būti pakviestam į garbingesnę vietą, negu būti pažemintam akivaizdoje kunigaikščio, kurį matei savo akimis. 
\par 8 Neik skubotai į teismą, nes nežinosi, ką daryti, kai tavo artimas sugėdins tave. 
\par 9 Išspręskite savo ginčą su artimu tarpusavyje ir neatskleiskite paslapčių svetimiesiems, 
\par 10 kad kas išgirdęs nesugėdintų tavęs ir tavo garbė nenukentėtų. 
\par 11 Laiku pasakytas tinkamas žodis yra kaip aukso obuolys sidabro įdėkle. 
\par 12 Išmintingas įspėjimas paklusniai ausiai yra kaip aukso žiedas ar auksinis papuošalas. 
\par 13 Ištikimas pasiuntinys yra kaip sniego šaltumas pjūties metu, jis atgaivina šeimininko širdį. 
\par 14 Kas giriasi tuo, ko nepadarė, yra kaip debesys ir vėjai be lietaus. 
\par 15 Kantrumu galima įtikinti kunigaikštį; švelnus liežuvis sulaužo kaulus. 
\par 16 Radęs medaus, valgyk, kiek nori, tik nepersivalgyk, kad nereikėtų išvemti. 
\par 17 Nesilankyk per dažnai pas savo artimą, kad nenusibostum ir jis nepradėtų tavęs nekęsti. 
\par 18 Žmogus, kuris neteisingai liudija prieš savo artimą, yra kaip ietis, kardas ar aštri strėlė. 
\par 19 Kaip sugedęs dantis ar išnirusi koja, taip pasitikėjimas neištikimu žmogumi nelaimės dieną. 
\par 20 Kas dainuoja liūdinčiam, prilygsta tam, kuris atima apsiaustą šaltą dieną arba užpila actą ant žaizdos. 
\par 21 Jei tavo priešas alksta, pavalgydink jį, jei trokšta­pagirdyk. 
\par 22 Taip darydamas, krausi žarijas ant jo galvos, ir Viešpats atlygins tau. 
\par 23 Šiaurys vėjas atneša lietų, apkalbos sukelia pyktį. 
\par 24 Geriau gyventi palėpės kampe negu su vaidinga moterimi dideliuose namuose. 
\par 25 Kaip šaltas vanduo ištroškusiam, taip gera žinia iš tolimo krašto. 
\par 26 Kaip sudrumstas šaltinis arba užterštas šulinys yra teisusis, krentąs prieš nedorėlį. 
\par 27 Negerai persivalgyti medaus, taip pat siekti sau šlovės nėra šlovė. 
\par 28 Žmogus, kuris nesusivaldo, yra kaip atviras miestas, kurio sienos sugriautos.



\chapter{26}


\par 1 Kaip sniegas vasarą ir lietus pjūties metu, taip garbė netinka kvailiui. 
\par 2 Kaip žvirblis nuskrenda ir kregždė nulekia, taip neišsipildys neužpelnytas prakeikimas. 
\par 3 Botagas arkliui, žąslai asilui, rykštė kvailio nugarai. 
\par 4 Neatsakyk kvailiui pagal jo kvailumą, kad netaptum panašus į jį. 
\par 5 Atsakyk kvailiui pagal jo kvailumą taip, kad jis neatrodytų sau išmintingas. 
\par 6 Kas siunčia kvailą pasiuntinį, nusikerta kojas ir patiria nuostolį. 
\par 7 Kaip luošas negali vaikščioti savo kojomis, taip patarlė netinka kvailiui. 
\par 8 Gerbti kvailą yra kaip dėti brangakmenį į mėtyklę. 
\par 9 Kaip erškėtis girtuoklio rankoje, taip patarlė kvailio burnoje. 
\par 10 Didis Dievas, kuris visa padarė, atlygina kvailiui ir neištikimam. 
\par 11 Kaip šuo grįžta prie savo vėmalo, taip kvailys kartoja savo kvailystes. 
\par 12 Kvailys teikia daugiau vilties negu žmogus, kuris laiko save išmintingu. 
\par 13 Tinginys sako: “Liūtas kelyje! Žiaurus liūtas gatvėje!” 
\par 14 Kaip durys sukasi ant vyrių, taip tinginys vartosi lovoje. 
\par 15 Tinginys įkiša savo ranką į dubenį, bet jam sunku pakelti ją prie burnos. 
\par 16 Tinginys laiko save išmintingesniu už septynis vyrus, galinčius išmintingai atsakyti. 
\par 17 Kas praeidamas įsikiša į vaidus, kurie jo neliečia, elgiasi kaip tas, kuris šunį griebia už ausų. 
\par 18 Kaip beprotis, kuris mėto žarijas, laido strėles ir žudo, 
\par 19 yra tas, kas apgauna artimą ir sako: “Aš pajuokavau”. 
\par 20 Kai nėra malkų, gęsta ugnis; pašalinus apkalbėtoją, baigiasi ginčai. 
\par 21 Kaip iš anglių atsiranda žarijos ir iš malkų ugnis, taip vaidingas žmogus sukelia kivirčus. 
\par 22 Apkalbos yra lyg skanėstas, kuris pasiekia žmogaus vidurius. 
\par 23 Karšti žodžiai ir nedora širdis yra kaip sidabro priemaišomis aptraukta molinė šukė. 
\par 24 Kas neapkenčia, slepia tai po savo lūpomis ir laiko klastą savyje. 
\par 25 Kai jis kalba maloniai, netikėk juo: jo širdyje yra septynios bjaurystės. 
\par 26 Nors jis neapykantą slepia žodžiais, jo nedorybė paaiškės tautos susirinkime. 
\par 27 Kas kasa duobę, pats į ją įkrinta. Kas parita akmenį, ant to jis sugrįžta. 
\par 28 Meluojantis liežuvis nekenčia tų, kurie nuo jo nukenčia. Pataikaujanti burna sukelia pražūtį.



\chapter{27}


\par 1 Nesigirk rytdiena, nes nežinai, ką ji tau atneš. 
\par 2 Tegul kitas giria tave, o ne tavo burna; svetimas, bet ne tavo lūpos. 
\par 3 Akmuo ir smėlis yra sunkūs, bet kvailio pyktis yra sunkesnis už abu. 
\par 4 Rūstybė yra žiauri, pyktis nesuvaldomas, bet kas gali atsispirti pavydui? 
\par 5 Geriau yra viešas įspėjimas negu slapta meilė. 
\par 6 Žaizdos nuo draugo yra geriau negu klastingas priešo pabučiavimas. 
\par 7 Sotus ir medaus nevalgo, o alkanam ir kartus daiktas yra saldus. 
\par 8 Žmogus, palikęs savo vietą, yra kaip paukštis, išskridęs iš lizdo. 
\par 9 Draugo žodžiai ir nuoširdūs patarimai gaivina širdį kaip aliejai ir brangūs kvepalai. 
\par 10 Neprarask savo draugo ir tėvo draugo. Nelaimės dieną neik į brolio namus. Kaimynas arti geriau negu brolis toli. 
\par 11 Mano sūnau, būk išmintingas ir pradžiugink mano širdį, kad galėčiau atsakyti tam, kuris man priekaištauja. 
\par 12 Supratingas numato pavojų ir pasislepia, o neišmanėlis eina ir nukenčia. 
\par 13 Paimk apdarą iš to, kas laidavo už svetimą, ir turėk užstatą iš suvedžiotojo. 
\par 14 Kas atsikėlęs anksti rytą garsiai laimina savo draugą, tai bus jam įskaityta prakeikimu. 
\par 15 Varvantis stogas lietingą dieną ir vaidinga moteris yra panašūs. 
\par 16 Kas nori ją sulaikyti yra kaip tas, kas galvoja sustabdyti vėją arba išlaikyti saujoje aliejų. 
\par 17 Kaip geležis galanda geležį, taip žmogus aštrina savo draugą. 
\par 18 Kas prižiūri figmedį, valgo jo vaisių; kas laukia šeimininko, sulaukia pagarbos. 
\par 19 Kaip vanduo atspindi veidą, taip žmogaus širdis atspindi žmogų. 
\par 20 Kaip pragaras ir prapultis niekados neprisipildo, taip žmogaus akys niekados nepasisotina. 
\par 21 Kaip sidabras ir auksas ištiriamas ugnyje, taip žmogus ištiriamas pagyrimais. 
\par 22 Nors sugrūstum kvailį piestoje su grūdais, jo kvailystė neatsiskirtų nuo jo. 
\par 23 Prižiūrėk rūpestingai savo avis ir bandą, 
\par 24 nes turtas nėra amžinas ir karūna neišlieka visoms kartoms. 
\par 25 Kai šienas suvežamas, pasirodo atolas, ir renkamos kalnų žolės. 
\par 26 Avinėliai tavo drabužiams, ožiai laukams nusipirkti. 
\par 27 Ožkų pieno užteks maistui tau, tavo šeimai ir tarnaitėms aprūpinti.



\chapter{28}


\par 1 Nedorėlis bėga niekam nevejant, o teisusis yra drąsus kaip liūtas. 
\par 2 Dėl šalies nuodėmių daugėja kunigaikščių, bet, jei valdovas išmintingas ir sumanus, ji ilgai išsilaiko. 
\par 3 Vargšas, kuris spaudžia vargšą, yra panašus į smarkų lietų, kuris sunaikina derlių. 
\par 4 Kurie nesilaiko įstatymo, giria nedorėlius; kurie laikosi įstatymo, priešinasi jiems. 
\par 5 Pikti žmonės nesupranta teisingumo, o kurie ieško Viešpaties, supranta viską. 
\par 6 Beturtis, kuris dorai elgiasi, geresnis už turtuolį, kuris iškraipo savo kelius. 
\par 7 Išmintingas sūnus laikosi įstatymo, o lėbautojų draugas daro gėdą tėvui. 
\par 8 Kas krauna turtus nuošimčiais ir palūkanomis, kaupia juos tam, kas pasigaili vargšo. 
\par 9 Kas neklauso įstatymo, to malda­pasibjaurėjimas. 
\par 10 Kas suklaidina teisųjį, kad jis eitų piktais keliais, pats įkris į savo duobę, o nekaltieji paveldės gėrybes. 
\par 11 Turtuolis tariasi esąs išmintingas, bet protingas beturtis mato tikrą jo padėtį. 
\par 12 Kai teisieji viešpatauja,­didelė šlovė; kai nedorėliai iškyla, žmonės slepiasi. 
\par 13 Kas slepia savo nusikaltimus, tam nesiseks, o kas išpažįsta ir atsisako jų, susilauks gailestingumo. 
\par 14 Palaimintas žmogus, kuris nuolat prisibijo, o kas užkietina savo širdį, pateks į nelaimę. 
\par 15 Nedoras valdovas neturtingai tautai yra kaip riaumojantis liūtas ir alkanas lokys. 
\par 16 Neprotingas valdovas griebiasi žiaurios priespaudos, kas nekenčia godumo, prailgins savo dienas. 
\par 17 Žmogus, praliejęs nekaltą kraują, bėga į pražūtį; niekas tegul nepadeda jam. 
\par 18 Kas nekaltai vaikščioja, bus išgelbėtas, kas eina kreivais keliais, vienąkart pražus. 
\par 19 Kas dirba savo žemę, turės pakankamai maisto, o kas seka dykinėtojais, skurs. 
\par 20 Ištikimas žmogus bus gausiai palaimintas, o kas siekia greitai praturtėti, neliks nekaltas. 
\par 21 Būti šališkam yra negerai; toks ir dėl duonos kąsnio nusikals. 
\par 22 Pavydus žmogus siekia greitai praturtėti ir nenujaučia, kad jo laukia skurdas. 
\par 23 Kas pabara klystantį, susilauks daugiau palankumo negu tas, kuris jam pataikauja liežuviu. 
\par 24 Kas apiplėšia tėvą ar motiną ir mano, kad tai nėra nusikaltimas, tas yra naikintojo bendras. 
\par 25 Išdidžios širdies žmogus sukelia vaidus, o kas pasitiki Viešpačiu, klestės. 
\par 26 Kas pasitiki savo širdimi, yra kvailas, o kas išmintingai elgiasi, bus išgelbėtas. 
\par 27 Kas duoda vargšui, nestokos, o kas užsimerkia, kad jo nematytų, bus labai keikiamas. 
\par 28 Iškylant nedorėliams, žmonės slepiasi, bet kai jie pražūna, padaugėja teisiųjų.



\chapter{29}


\par 1 Kas dažnai baramas, bet užkietina savo sprandą, bus staiga sunaikintas ir nebeatsigaus. 
\par 2 Kai teisieji valdo, tauta džiaugiasi, o kai valdo nedorėliai, tauta dejuoja. 
\par 3 Kas myli išmintį, džiugina tėvą, o kas susideda su paleistuvėmis, praranda turtą. 
\par 4 Teisingai valdydamas, karalius sustiprina kraštą, o kas ima kyšius, griauna jį. 
\par 5 Kas pataikauja artimui, spendžia pinkles sau. 
\par 6 Piktas žmogus įsipainioja nusikaltimuose, o teisusis gieda ir džiūgauja. 
\par 7 Teisusis atsižvelgia į beturčių teises, o nedorėlis nenori jų žinoti. 
\par 8 Niekintojai sukelia mieste neramumus, o išmintingieji nukreipia rūstybę. 
\par 9 Jei išmintingas susiginčija su kvailiu,­ar tas niršta, ar juokiasi,­nėra ramybės. 
\par 10 Kraujo trokštantis nekenčia nekaltojo, o teisieji rūpinasi jo siela. 
\par 11 Kvailys kalba viską, ką galvoja, o išmintingas susilaiko. 
\par 12 Jei valdovas klauso melo, visi jo tarnai bus nedorėliai. 
\par 13 Beturtis ir sukčius turi bendra: Viešpats abiem davė šviesą akims. 
\par 14 Karaliaus, kuris teisingai teisia beturtį, sostas išsilaikys per amžius. 
\par 15 Rykštė ir pabarimas teikia išminties; vaikas, paliktas savo valiai, daro gėdą motinai. 
\par 16 Daugėjant nedorėliams, daugėja nusikaltimų; teisieji matys jų žlugimą. 
\par 17 Auklėk savo sūnų, tai jis bus tau paguoda ir tavo siela džiaugsis. 
\par 18 Be apreiškimo žūsta tauta. Palaimintas, kas laikosi įstatymo. 
\par 19 Vergo neišauklėsi žodžiais; nors jis supranta, bet nepaklauso. 
\par 20 Kvailys teikia daugiau vilties, negu žmogus, kuris skubotai kalba. 
\par 21 Vergas, lepinamas nuo mažens, galiausiai taps kaip sūnus. 
\par 22 Piktas žmogus sukelia vaidus, o ūmus žmogus dažnai nusikalsta. 
\par 23 Išdidumas pažemina žmogų, o nuolankus dvasia susilauks pagarbos. 
\par 24 Kas susideda su vagimi, nekenčia savo sielos; jis prisiekia sakyti tiesą, bet nieko nepasako. 
\par 25 Kas bijo žmonių, pakliūna į spąstus, kas pasitiki Viešpačiu, bus saugus. 
\par 26 Daugelis ieško valdovo palankumo, bet teisingumas ateina iš Viešpaties. 
\par 27 Teisusis bjaurisi neteisiuoju ir nedorėlis tuo, kuris eina tiesiu keliu.



\chapter{30}


\par 1 Žodžiai Agūro, Jakės sūnaus. Taip jis kalbėjo Itieliui, pačiam Itieliui ir Ukalui: 
\par 2 “Aš suprantu mažiau negu kiti ir neturiu žmogaus proto. 
\par 3 Aš nesimokiau išminties ir neturiu Šventojo pažinimo. 
\par 4 Kas užžengė į dangų ir nusileido? Kas sulaikė vėją savo rankomis? Kas įvyniojo vandenis į drabužį? Kas nustatė žemės ribas? Kuo vardu Jis ir Jo sūnus, ar žinai? 
\par 5 Kiekvienas Dievo žodis yra tyras; Jis yra skydas tiems, kurie Juo pasitiki. 
\par 6 Nieko nepridėk prie Jo žodžių, kad Jis neapkaltintų tavęs ir neliktum melagis. 
\par 7 Dviejų dalykų prašau, neužgink man jų pirma, negu mirsiu. 
\par 8 Pašalink nuo manęs tuštybę ir melą; neduok man turtų nė skurdo, maitink mane tuo, ko man reikia, 
\par 9 kad pasisotinęs neišsiginčiau Tavęs ir nesakyčiau: ‘Kas yra Viešpats?’ arba nuskurdęs nevogčiau ir be reikalo neminėčiau Dievo vardo. 
\par 10 Neskųsk tarno jo šeimininkui, kad jis nekeiktų tavęs ir tu neliktum kaltas. 
\par 11 Yra karta, kuri keikia tėvą ir nelaimina motinos. 
\par 12 Karta, kuri laiko save švaria, bet nenusiplauna savo purvo. 
\par 13 Karta, kurios išdidus žvilgsnis ir pakeltos blakstienos. 
\par 14 Karta, kurių dantys yra kardai ir peiliai, kuriais ji suryja vargšus krašte ir beturčius tarp žmonių. 
\par 15 Siurbėlė turi dvi dukteris, kurios šaukia: ‘Duok, duok!’ Trys dalykai yra nepasotinami, o ketvirtas niekada nesako: ‘Užtenka’. 
\par 16 Tai mirusiųjų buveinė, nevaisingos įsčios, žemė, kuri sugeria vandenį, ir ugnis­ji nesako: ‘Užtenka!’ 
\par 17 Akis, kurios tyčiojasi iš tėvo ir niekina paklusnumą motinai, iškapos varnai slėnyje ir suės erelio jaunikliai. 
\par 18 Trys dalykai man nesuvokiami ir ketvirtojo nesuprantu: 
\par 19 erelio kelias padangėje, gyvatės­ant uolos, laivo­jūroje ir vyro kelias su mergaite. 
\par 20 Štai kelias neištikimos moters; ji pavalgo ir, nusišluosčiusi lūpas, sako: ‘Nieko blogo nepadariau’. 
\par 21 Dėl trijų dalykų sujuda žemė, ketvirtojo ji negali pakęsti: 
\par 22 tarno, kai jis karaliauja, kvailio, kai jis pasisotina, 
\par 23 bjaurios moteriškės, kai ji išteka, ir tarnaitės, kuri užima šeimininkės vietą. 
\par 24 Keturi žemės gyvūnai yra maži, bet labai išmintingi: 
\par 25 skruzdės nėra stipri tauta, tačiau vasarą prisirengia sau maisto; 
\par 26 triušiai yra silpni, tačiau pasidaro namus uolose; 
\par 27 skėriai neturi karaliaus, tačiau tvarkingai skrenda būriais; 
\par 28 voras audžia savo rankomis, tačiau būna ir karaliaus rūmuose. 
\par 29 Trys vaikšto išdidžiai, ketvirtas eina didingai: 
\par 30 liūtas­stipriausias tarp žvėrių, nebijo nieko; 
\par 31 kurtas, ožys ir karalius priešaky savo žmonių. 
\par 32 Jei buvai kvailas ir aukštinai save ar planavai pikta, užsidenk ranka savo burną. 
\par 33 Plakant pieną, gaunamas sviestas; stipriai šnypščiant nosį, pasirodo kraujas; pykčio kurstymas sukelia ginčą”.



\chapter{31}


\par 1 Karaliaus Lemuelio žodžiai, kuriais jį auklėjo motina: 
\par 2 “Ką, mano sūnau? Ką, mano įsčių sūnau? Ką, mano įžadų sūnau? 
\par 3 Neatiduok savo jėgų moterims ir savo kelių toms, kurios pražudo karalius. 
\par 4 Lemueli, ne karaliams gerti vyną, ne kunigaikščiams stiprius gėrimus, 
\par 5 kad prisigėrę jie nepamirštų įstatymo ir neiškraipytų teisingumo prispaustiesiems. 
\par 6 Duok stiprius gėrimus nelaimingiems ir vyną liūdinčioms sieloms. 
\par 7 Tegul jie pasigeria ir užmiršta savo vargus ir skurdą. 
\par 8 Atverk už nebylį savo burną byloje tų, kurie pasmerkti pražūčiai. 
\par 9 Teisk teisingai ir apgink beturčių ir vargšų teises. 
\par 10 Kas gali surasti gerą moterį? Ji yra daug vertingesnė už perlus. 
\par 11 Vyro širdis visiškai pasitiki ja ir jo namuose netrūks pelno. 
\par 12 Visą gyvenimą ji daro vyrui tik gera, o ne pikta. 
\par 13 Ji ieško vilnos ir lino, noriai dirba savo rankomis. 
\par 14 Ji lyg prekybininkų laivas parūpina maistą iš toli. 
\par 15 Ji keliasi anksti rytą, paruošia maisto šeimai ir paskiria tarnaitėms darbą. 
\par 16 Ji apžiūri lauką, jį nuperka ir savo rankų pelnu užveisia vynuogyną. 
\par 17 Ji susijuosia jėga, sustiprina savo rankas. 
\par 18 Ji supranta, kad jos darbas vertingas; jos žiburys negęsta naktį. 
\par 19 Jos pirštai paima verpstę ir jos rankos dirba. 
\par 20 Ji ištiesia ranką išalkusiam ir beturčiui. 
\par 21 Ji nebijo šalčių, nes visi jos namiškiai yra aprengti vilnoniais rūbais. 
\par 22 Ji pasidaro antklodžių, plona drobė ir purpuras­jos drabužiai. 
\par 23 Jos vyras yra žinomas vartuose, kai sėdi su krašto vyresniaisiais. 
\par 24 Ji audžia plonų drobių ir parduoda, pristato juostų pirkliams. 
\par 25 Ji, apsirengusi stiprybe ir grožiu, linksmai žiūri į ateitį. 
\par 26 Ji kalba išmintingai ir ant jos liežuvio švelnus pamokymas. 
\par 27 Ji prižiūri savo namus ir nevalgo tinginio duonos. 
\par 28 Jos vaikai pakyla ir vadina ją palaiminta ir jos vyras giria ją: 
\par 29 ‘Yra daug gerų moterų, bet tu pranoksti jas visas!’ 
\par 30 Žavumas apgauna ir grožis praeina, bet moteris, bijanti Dievo, bus giriama. 
\par 31 Duok jai jos rankų vaisių, ir jos darbai tegul giria ją vartuose”.



\end{document}