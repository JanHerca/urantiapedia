\begin{document}

\title{Antrasis Jono laiškas}

\chapter{1}


\par 1 Vyresnysis išrinktajai poniai ir jos vaikams, kuriuos aš myliu tiesoje,­ir ne tik aš vienas, bet ir visi, kurie yra pažinę tiesą,­ 
\par 2 dėlei tiesos, pasiliekančios mumyse ir būsiančios su mumis per amžius. 
\par 3 Su jumis tebus malonė, gailestingumas, ramybė nuo Dievo Tėvo ir nuo Viešpaties Jėzaus Kristaus, Tėvo Sūnaus, tiesoje ir meilėje. 
\par 4 Labai nudžiugau, radęs tavųjų vaikų, vaikščiojančių tiesoje, kaip reikalauja iš Tėvo gautasis įsakymas. 
\par 5 O dabar prašau tave, ponia,­ne kaip rašydamas naują įsakymą, bet kaip tą, kurį turėjome nuo pradžios,­kad mylėtume vieni kitus. 
\par 6 O meilė­tai gyventi pagal Jo įsakymus. Toks ir yra įsakymas, kurį girdėjote nuo pradžios: kad gyventumėte pagal jį. 
\par 7 Po pasaulį pasklido daug suvedžiotojų, kurie neišpažįsta Jėzaus Kristaus, atėjusio kūne. Toks yra apgavikas ir antikristas. 
\par 8 Žiūrėkime savęs, kad neprarastume, ką esame nuveikę, bet kad gautume visą atlygį. 
\par 9 Kas tik peržengia ribą ir nesilaiko Kristaus mokymo, neturi Dievo. Kas laikosi Kristaus mokymo, tas turi ir Tėvą, ir Sūnų. 
\par 10 Jei kas ateina pas jus ir neatsineša šio mokymo, to nepriimkite į savo namus ir nesveikinkite, 
\par 11 nes, kas jį sveikina, dalyvauja jo piktuose darbuose. 
\par 12 Turėčiau dar daug ką jums parašyti, bet nenoriu to daryti ant papiruso ir rašalu. Aš tikiuosi pas jus atvykti ir pasikalbėti iš lūpų į lūpas, kad mūsų džiaugsmas būtų tobulas. 
\par 13 Tave sveikina išrinktosios tavo sesers vaikai. Amen.


\end{document}