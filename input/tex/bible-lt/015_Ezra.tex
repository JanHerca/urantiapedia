\begin{document}

\title{Ezra}

\chapter{1}


\par 1 Pirmaisiais persų karaliaus Kyro metais, kad įvyktų Jeremijo paskelbtas Viešpaties žodis, Viešpats paragino persų karalių Kyrą, kad jis paskelbtų visoje karalystėje žodžiu ir raštu: 
\par 2 “Taip sako persų karalius Kyras: ‘Viešpats, dangaus Dievas, atidavė man visas žemės karalystes; Jis pavedė man atstatyti Jo namus Jeruzalėje, kuri yra Jude. 
\par 3 Kas iš jūsų yra iš Jo tautos, jo Dievas tebūna su juo ir tegrįžta jis į Jeruzalę, kuri yra Jude, ir tegul stato Viešpaties, Izraelio Dievo, kuris yra Jeruzalėje, namus. 
\par 4 Visose vietose, kur jis begyventų, pasilikusieji žmonės tepadeda jam sidabru ir auksu, manta ir gyvuliais bei laisvos valios auka Dievo namams Jeruzalėje’ ”. 
\par 5 Tuomet Judo ir Benjamino šeimų vyresnieji, kunigai, levitai ir visi, kurių dvasia buvo Dievo sužadinta, pakilo eiti statyti Viešpaties namų Jeruzalėje. 
\par 6 Visi aplinkiniai gyventojai juos parėmė sidabriniais ir auksiniais indais, manta, gyvuliais ir brangiomis dovanomis. 
\par 7 Karalius Kyras atidavė Viešpaties namų reikmenis, kuriuos Nebukadnecaras buvo atgabenęs iš Jeruzalės ir padėjęs savo dievo namuose. 
\par 8 Persų karalius Kyras juos perdavė per savo iždininką Mitredatą, kuris juos suskaitė ir atidavė Judo kunigaikščiui Šešbacarui: 
\par 9 trisdešimt auksinių dubenų, tūkstantį sidabrinių dubenų, dvidešimt devynis peilius, 
\par 10 trisdešimt auksinių taurių, keturis šimtus dešimt sidabrinių taurių ir tūkstantį kitokių indų. 
\par 11 Visų auksinių ir sidabrinių indų buvo penki tūkstančiai keturi šimtai. Visa tai paėmė Šešbacaras, belaisviams grįžtant iš Babilono į Jeruzalę.


\chapter{2}


\par 1 Tie yra krašto žmonės, kurie grįžo iš nelaisvės, iš tų, kuriuos Babilono karalius Nebukadnecaras buvo ištrėmęs į Babiloną. Jie sugrįžo į Jeruzalę bei Judą, kiekvienas į savo miestą. 
\par 2 Su Zorobabeliu grįžo Jozuė, Nehemija, Seraja, Reelaja, Mordechajas, Bilšanas, Misparas, Bigvajas, Rehumas, Baana. Izraelio tautos vyrų skaičius toks: 
\par 3 Parošo palikuonių­du tūkstančiai šimtas septyniasdešimt du; 
\par 4 Šefatijos­trys šimtai septyniasdešimt du; 
\par 5 Aracho­septyni šimtai septyniasdešimt penki; 
\par 6 Pahat Moabo palikuonių iš Jozuės ir Joabo giminės­du tūkstančiai aštuoni šimtai dvylika; 
\par 7 Elamo­tūkstantis du šimtai penkiasdešimt keturi; 
\par 8 Zatuvo­devyni šimtai keturiasdešimt penki; 
\par 9 Zakajo­septyni šimtai šešiasdešimt; 
\par 10 Banio­šeši šimtai keturiasdešimt du; 
\par 11 Bebajo­šeši šimtai dvidešimt trys; 
\par 12 Azgado­tūkstantis du šimtai dvidešimt du; 
\par 13 Adonikamo­šeši šimtai šešiasdešimt šeši; 
\par 14 Bigvajo­du tūkstančiai penkiasdešimt šeši; 
\par 15 Adino­keturi šimtai penkiasdešimt keturi; 
\par 16 Atero palikuonių iš Jehizkijos­ devyniasdešimt aštuoni; 
\par 17 Becajo­trys šimtai dvidešimt trys; 
\par 18 Joros­šimtas dvylika; 
\par 19 Hašumo­du šimtai dvidešimt trys; 
\par 20 Gibaro­devyniasdešimt penki; 
\par 21 Betliejaus vyrų­šimtas dvidešimt trys; 
\par 22 Netofos vyrų­penkiasdešimt šeši; 
\par 23 Anatoto vyrų­šimtas dvidešimt aštuoni; 
\par 24 Azmaveto vyrų­keturiasdešimt du; 
\par 25 Kirjat Arimo, Kefyros ir Beeroto vyrų­septyni šimtai keturiasdešimt trys; 
\par 26 Ramos ir Gebos vyrų­šeši šimtai dvidešimt vienas; 
\par 27 Michmašo vyrų­šimtas dvidešimt du; 
\par 28 Betelio ir Ajo vyrų­du šimtai dvidešimt trys; 
\par 29 Nebojo palikuonių­penkiasdešimt du; 
\par 30 Magbišo­šimtas penkiasdešimt šeši; 
\par 31 kito Elamo­tūkstantis du šimtai penkiasdešimt keturi; 
\par 32 Harimo­trys šimtai dvidešimt; 
\par 33 Lodo, Hadido ir Onojo­septyni šimtai dvidešimt penki; 
\par 34 Jericho­trys šimtai keturiasdešimt penki; 
\par 35 Senavos­trys tūkstančiai šeši šimtai trisdešimt. 
\par 36 Kunigų: Jedajos palikuonių iš Jozuės namų­devyni šimtai septyniasdešimt trys; 
\par 37 Imero­tūkstantis penkiasdešimt du; 
\par 38 Pašhūro­tūkstantis du šimtai keturiasdešimt septyni; 
\par 39 Harimo­tūkstantis septyniolika. 
\par 40 Levitų: Jozuės ir Kadmielio palikuonių iš Hodavijos sūnų­septyniasdešimt keturi. 
\par 41 Giedotojų: Asafo palikuonių­ šimtas dvidešimt aštuoni. 
\par 42 Vartininkų: Šalumo, Atero, Talmono, Akubo, Hatitos ir Šobajo palikuonių­iš viso šimtas trisdešimt devyni. 
\par 43 Šventyklos tarnai: Cihos, Hasufos, Tabaoto, 
\par 44 Keroso, Siacho, Padono, 
\par 45 Lebanos, Hagabos, Akubo, 
\par 46 Hagabo, Šalmajo, Hanano, 
\par 47 Gidelio, Gaharo, Reajos, 
\par 48 Recino, Nekodos, Gazamo, 
\par 49 Uzos, Paseaho, Besajo, 
\par 50 Asnos, Meunimo, Nefusimo, 
\par 51 Bakbuko, Hakufos, Harhūro, 
\par 52 Bacluto, Mehidos, Haršos, 
\par 53 Barkoso, Siseros, Temaho, 
\par 54 Neciacho ir Hatifos palikuonys. 
\par 55 Saliamono tarnų palikuonys: Sotajo, Sofereto, Perudos, 
\par 56 Jaalos, Darkono, Gidelio, 
\par 57 Šefatijos, Hatilo, Pocheret Cebaimo ir Amio palikuonys. 
\par 58 Šventyklos ir Saliamono tarnų palikuonių buvo trys šimtai devyniasdešimt du. 
\par 59 Šitie atvyko iš Tel Melacho, Tel Haršo, Kerubo, Adono ir Imero, bet negalėjo įrodyti savo tėvų ir savo kilmės, ar jie buvo kilę iš Izraelio: 
\par 60 Delajos, Tobijos ir Nekodos palikuonių­šeši šimtai penkiasdešimt du. 
\par 61 Kunigai: Hobajos, Hakoco ir Barzilajaus (jis buvo vedęs vieną iš gileadiečo Barzilajaus dukterų ir buvo vadinamas jų vardu) palikuonys. 
\par 62 Jie ieškojo savo vardų giminių sąrašuose, tačiau jų nesurado; todėl buvo atskirti nuo kunigystės kaip susitepę. 
\par 63 Tiršata uždraudė jiems valgyti labai šventą maistą, kol atsiras kunigas su Urimu ir Tumimu. 
\par 64 Iš viso žmonių buvo keturiasdešimt du tūkstančiai trys šimtai šešiasdešimt, 
\par 65 neskaičiuojant jų tarnų ir tarnaičių, kurių buvo septyni tūkstančiai trys šimtai trisdešimt septyni. Be to, jie turėjo du šimtus giedotojų. 
\par 66 Žirgų buvo septyni šimtai trisdešimt šeši, mulų­du šimtai keturiasdešimt penki, 
\par 67 kupranugarių­keturi šimtai trisdešimt penki, asilų­šeši tūkstančiai septyni šimtai dvidešimt. 
\par 68 Kai kurie šeimų vadai, atėję prie Viešpaties namų Jeruzalėje, davė savo noru aukų Dievo namams; 
\par 69 jie davė pagal savo išgales iždui šešiasdešimt vieną tūkstantį drachmų aukso, penkis tūkstančius minų sidabro ir šimtą apdarų kunigams. 
\par 70 Kunigai, levitai, tautos dalis, giedotojai, vartininkai ir šventyklos tarnai apsigyveno savo miestuose; ir visas Izraelis apsigyveno savo miestuose.



\chapter{3}

\par 1 Izraelitams apsigyvenus savuose miestuose, septintą mėnesį visi vieningai susirinko į Jeruzalę. 
\par 2 Tada Jehocadako sūnus Jozuė su savo broliais kunigais ir Salatielio sūnus Zorobabelis su savo broliais atstatė Izraelio Dievo aukurą deginamosioms aukoms, kaip parašyta Dievo vyro Mozės įstatyme. 
\par 3 Jie atstatė aukurą jo senoje vietoje, nors ir bijojo aplinkui gyvenančių tautų. Jie kas rytą ir vakarą aukojo ant jo deginamąsias aukas Viešpačiui. 
\par 4 Jie šventė Palapinių šventę, kaip parašyta, ir kasdien aukojo deginamųjų aukų tiek, kiek buvo nustatyta tą dieną. 
\par 5 Be to, jie aukojo nuolatines deginamąsias, jauno mėnulio, visų metinių Viešpaties švenčių ir laisvos valios aukas Viešpačiui. 
\par 6 Septintojo mėnesio pirmąją dieną jie pradėjo aukoti Viešpačiui deginamąsias aukas, bet Viešpaties šventyklos pamatai dar nebuvo padėti. 
\par 7 Jie davė pinigų akmenskaldžiams ir statybininkams, o sidoniečiams ir Tyro gyventojams­ maisto, gėrimo ir aliejaus, kad jūra atgabentų kedro medžių iš Libano į Jopę, kaip jiems persų karalius Kyras buvo įsakęs. 
\par 8 Po savo sugrįžimo antrųjų metų antrąjį mėnesį, Salatielio sūnus Zorobabelis ir Jehocadako sūnus Jozuė su visais savo broliais kunigais, levitais ir visais, grįžusiais iš nelaisvės į Jeruzalę, pradėjo ir paskyrė levitus, dvidešimties metų ir vyresnius, Viešpaties namų statybos darbams prižiūrėti. 
\par 9 Jozuė ir jo sūnūs bei broliai, Kadmielis ir jo sūnūs iš Judo sūnų, Henadado sūnūs, jų broliai ir levitai prižiūrėjo darbininkus Dievo namų statyboje. 
\par 10 Statybininkams padėjus Viešpaties šventyklos pamatus, kunigai, apsirengę savo rūbais, sustojo su trimitais rankose, o levitai, Asafo sūnūs, su cimbolais šlovino Viešpatį pagal Izraelio karaliaus Dovydo nurodymus. 
\par 11 Jie pakaitomis giedojo, šlovindami ir dėkodami Viešpačiui: “Jis yra geras, ir Jo gailestingumas Izraeliui amžinas”. Kai jie šlovino Viešpatį, visi žmonės garsiai šaukė, nes Viešpaties namų pamatai buvo padėti. 
\par 12 Daugelis kunigų, levitų ir šeimų vyresniųjų, kurie buvo seni žmonės ir savo akimis matė pirmuosius namus, garsiai verkė, matydami šių namų pamatus; tuo tarpu kiti garsiai šaukė iš džiaugsmo. 
\par 13 Nebuvo galima atskirti džiaugsmo šūksnių ir verkiančiųjų balso, nes žmonės garsiai šaukė ir garsas buvo girdimas toli.



\chapter{4}


\par 1 Judo ir Benjamino priešai, išgirdę, kad grįžę tremtiniai stato šventyklą Viešpačiui, Izraelio Dievui, 
\par 2 atėjo pas Zorobabelį ir Izraelio šeimų vyresniuosius ir jiems sakė: “Mes norime statyti kartu su jumis, nes mes, kaip ir jūs, ieškome jūsų Dievo ir Jam aukojame nuo Asarhadono, Asirijos karaliaus, kuris mus čia atvedė, laikų”. 
\par 3 Zorobabelis, Jozuė ir kiti Izraelio šeimų vyresnieji jiems atsakė: “Netinka jums kartu su mumis statyti namus mūsų Dievui, mes vieni statysime Viešpačiui, Izraelio Dievui, kaip mums įsakė persų karalius Kyras”. 
\par 4 Krašto žmonės silpnino Judo žmones ir trukdė jiems statyti. 
\par 5 Jie papirko patarėjus, norėdami atgrasinti juos nuo jų tikslo per visas persų karaliaus Kyro dienas iki persų karaliaus Darijaus laikų. 
\par 6 Pradėjus karaliauti Ahasverui, jie parašė kaltinimą prieš Judo ir Jeruzalės gyventojus. 
\par 7 Artakserkso dienomis Bišlamas, Mitredatas, Tabeelis ir kiti jų bendrai parašė persų karaliui Artakserksui laišką. Laiškas buvo parašytas aramėjų kalba. 
\par 8 Patarėjas Rehumas ir raštininkas Šimšajas rašė karaliui Artakserksui tokio turinio laišką prieš Jeruzalę: 
\par 9 “Patarėjas Rehumas, raštininkas Šimšajas ir kiti jų bendrai iš Dinajo, Afarsato, Tarpelio, Afaro, Erecho, Babilono, Sūzų, Dehavo, Elamo 
\par 10 ir kitų tautų, kurias didysis ir garbingasis Asnaparas atvedė ir įkurdino Samarijos miestuose ir kitose vietovėse šiapus upės”. 
\par 11 Tai nuorašas laiško, kurį jie pasiuntė: “Karaliui Artakserksui. Tavo tarnai, vyrai iš šiapus upės, 
\par 12 praneša karaliui, kad žydai, kurie iš tavo teritorijų atvyko pas mus, apsigyveno Jeruzalėje ir stato tą maištingą ir blogą miestą. Jie stato sienas ir stiprina pamatus. 
\par 13 Tebūna žinoma karaliui, kad jei tas miestas bus pastatytas ir jo sienos užbaigtos, jie nebemokės mokesčių, duoklės ir muito; karaliaus metinės pajamos dėl to sumažės. 
\par 14 Kadangi mes išlaikomi karaliaus rūmų ir nenorime, kad karaliaus garbė būtų pažeminta, todėl pasiuntėme pranešimą karaliui. 
\par 15 Tegul paieško žinių apie tą miestą savo tėvų metraščių knygose. Ten atrasi, kad tas miestas yra maištingas ir pavojingas karaliams bei kraštams. Maištai jame keliami nuo seno, todėl tas miestas ir buvo sugriautas. 
\par 16 Mes pranešame karaliui, kad jei tas miestas bus pastatytas ir jo sienos užbaigtos, tu neteksi valdų šioje upės pusėje”. 
\par 17 Karalius pasiuntė tokį atsakymą patarėjui Rehumui, raštininkui Šimšajui ir kitiems jų bendrams, kurie gyvena Samarijoje ir anapus upės: “Ramybė jums! 
\par 18 Jūsų atsiųstas laiškas buvo man perskaitytas. 
\par 19 Man įsakius, buvo ieškota ir rasta, kad tas miestas jau nuo seno sukildavo prieš karalius ir maištai bei vaidai kildavo jame. 
\par 20 Galingi karaliai viešpatavo Jeruzalėje ir valdė visą kraštą anapus upės; jiems mokėjo mokesčius, duoklę ir muitus. 
\par 21 Įsakykite tiems vyrams liautis statyti miestą, kol bus duotas jiems mano įsakymas. 
\par 22 Elkitės rūpestingai, kad karaliaus reikalai nenukentėtų”. 
\par 23 Kai karaliaus Artakserkso laiško nuorašas buvo perskaitytas Rehumui, raštininkui Šimšajui ir jų bendrams, jie skubiai nuėjo į Jeruzalę pas žydus ir jėga sustabdė statybą. 
\par 24 Dievo namų darbas Jeruzalėje sustojo ir nevyko iki antrųjų Persijos karaliaus Darijaus karaliavimo metų.



\chapter{5}

\par 1 Pranašas Agėjas ir Idojo sūnus Zacharija Izraelio Dievo vardu pranašavo žydams, kurie buvo Jude ir Jeruzalėje. 
\par 2 Tada Salatielio sūnus Zorobabelis ir Jehocadako sūnus Jozuė pradėjo statyti Dievo namus Jeruzalėje; su jais buvo Dievo pranašai ir jiems padėjo. 
\par 3 Tuo laiku pas juos atėjo krašto šioje upės pusėje valdytojas Tatnajas ir Šetar Boznajas su savo bendrais ir jų klausė: “Kas jums įsakė statyti šituos namus ir taisyti šitas sienas?” 
\par 4 Mes atsakėme jiems, kuo vardu tie vyrai, kurie stato šitą pastatą. 
\par 5 Dievo akis buvo ant žydų vyresniųjų, todėl jie galėjo tęsti statybą, kol pranešimas pasieks Darijų ir bus gautas atsakymas šiuo reikalu. 
\par 6 Nuorašas laiško, kurį Tatnajas, krašto šioje upės pusėje valdytojas, Šetar Boznajas ir jų bendrai afarsakai, kurie buvo šiapus upės, pasiuntė karaliui Darijui: 
\par 7 “Ramybė karaliui Darijui! 
\par 8 Tebūna karaliui žinoma, kad mes nuvykome į Judėją, prie didžiojo Dievo namų. Jie statomi iš didelių akmenų, sienose rąstai dedami. Šitas darbas vykdomas sparčiai ir klesti jų rankose. 
\par 9 Mes klausėme vyresniųjų: ‘Kas jums įsakė statyti šituos namus ir taisyti šitas sienas?’ 
\par 10 Taip pat klausėme jų vardų, kad tau praneštume ir galėtume užrašyti jiems vadovaujančių vyrų vardus. 
\par 11 Jie atsakė: ‘Mes esame dangaus ir žemės Dievo tarnai ir atstatome namus, kurie prieš daugelį metų buvo pastatyti. Juos pastatė ir užbaigė garsus Izraelio karalius. 
\par 12 Bet kadangi mūsų tėvai užrūstino dangaus Dievą, Jis atidavė juos chaldėjui Nebukadnecarui, Babilono karaliui; jis sugriovė šituos namus, o tautą išvedė į Babiloną. 
\par 13 Tačiau pirmaisiais Babilono karaliaus Kyro metais karalius davė leidimą atstatyti šituos Dievo namus. 
\par 14 Taip pat ir auksinius bei sidabrinius Dievo namų indus, kuriuos Nebukadnecaras buvo išgabenęs iš Jeruzalės šventyklos į Babilono šventyklą, karalius Kyras paėmė iš Babilono šventyklos, perdavė Šešbacarui, kurį jis paskyrė valdytoju, 
\par 15 ir įsakė: ‘Imk šituos indus ir juos nugabenk į Jeruzalėje esančią šventyklą. Atstatykite Dievo namus buvusioje vietoje’. 
\par 16 Šešbacaras atvykęs padėjo Jeruzalėje Dievo namų pamatus. Nuo to laiko iki šiol jie tebestatomi ir dar nebaigti’. 
\par 17 Jei karalius mano esant reikalinga, tepaieško Babilone, karaliaus saugykloje, ar tikrai karalius Kyras davė įsakymą atstatyti Dievo namus Jeruzalėje, ir tegul karalius mums atsiunčia savo valią šituo reikalu”.



\chapter{6}

\par 1 Karalius Darijus įsakė ieškoti žinių Babilono knygų saugykloje. 
\par 2 Medų krašte Ekbatanos rūmuose buvo surastas ritinys, kuriame buvo taip parašyta: 
\par 3 “Pirmaisiais karaliaus Kyro metais karalius davė įsakymą Jeruzalės Dievo namų reikalu: ‘Turi būti atstatyti namai, kur buvo aukojamos aukos, ir jiems turi būti padėti tvirti pamatai. Jų aukštis ir plotis turi būti šešiasdešimt uolekčių. 
\par 4 Turi būti trys eilės didelių akmenų ir viena eilė rąstų. Išlaidas apmokės karaliaus namai. 
\par 5 Auksiniai ir sidabriniai Dievo namų indai, kuriuos Nebukadnecaras buvo atgabenęs iš Jeruzalės šventyklos į Babiloną, turi būti nugabenti į Jeruzalėje statomą šventyklą, į savo vietą, ir padėti Dievo namuose’ ”. 
\par 6 “Dabar, Tatnajau, valdytojau anapus upės, Šetar Boznajau ir jūsų bendrai afarsakai anapus upės, atsitraukite nuo jų. 
\par 7 Palikite Dievo namų darbą. Netrukdykite žydų valdytojui ir vyresniesiems statyti Dievo namų jų buvusioje vietoje. 
\par 8 Be to, įsakau, kad jūs padengtumėte žydų vyresniesiems visas Dievo namų išlaidas iš karaliaus turtų, iš mokesčių, surenkamų anapus upės, kad jų darbas nebūtų trukdomas. 
\par 9 Ko tik reikia dangaus Dievui deginamosioms aukoms: jaučių, avinų, ėriukų, kviečių, druskos, vyno ir aliejaus, kaip pasakys Jeruzalėje esą kunigai, privalote jiems kasdien nedelsiant suteikti, 
\par 10 kad jie aukotų malonaus kvapo aukas dangaus Dievui ir melstųsi už karaliaus ir jo vaikų gyvybę. 
\par 11 Be to, aš įsakiau, kad kiekvienas, kuris sulaužys šitą įsakymą, bus prikaltas prie ištraukto iš jo namų rąsto, o jo namai paversti griuvėsių krūva. 
\par 12 O Dievas, kuris ten paskyrė savo vardui buveinę, tepašalina kiekvieną karalių ir tautą, kurie išdrįstų priešintis tam įsakymui ir siektų sunaikinti Jeruzalėje Dievo namus. Aš, Darijus, daviau įsakymą ir jis turi būti nedelsiant vykdomas”. 
\par 13 Tada krašto šiapus upės valdytojas Tatnajas, Šetar Boznajas ir jų bendrai nedelsiant įvykdė karaliaus Darijaus įsakymą. 
\par 14 Žydų vyresnieji statė šventyklą, ir jiems sekėsi pagal pranašų Agėjo ir Idojo sūnaus Zacharijos pranašystes. Jie statė ir užbaigė įsakius Izraelio Dievui ir persų karaliams Kyrui, Darijui ir Artakserksui. 
\par 15 Namai buvo užbaigti šeštaisiais karaliaus Darijaus karaliavimo metais, Adaro mėnesio trečią dieną. 
\par 16 Tuomet izraelitai, kunigai, levitai ir visi, sugrįžę iš nelaisvės, su džiaugsmu pašventino Dievo namus. 
\par 17 Pašventindami Dievo namus, jie aukojo šimtą jaučių, du šimtus avinų, keturis šimtus ėriukų ir auką už nuodėmę, už visą Izraelį­dvylika ožių pagal Izraelio giminių skaičių. 
\par 18 Kunigai buvo paskirstyti skyriais ir levitai pagal jų pareigas, kad tarnautų Dievui Jeruzalėje, kaip parašyta Mozės knygoje. 
\par 19 Pirmo mėnesio keturioliktą dieną grįžę tremtiniai šventė Paschą. 
\par 20 Visi kunigai ir levitai buvo apsivalę ir švarūs. Jie pjovė Paschos avinėlį visiems grįžusiems tremtiniams, savo broliams kunigams ir sau. 
\par 21 Paschos avinėlį valgė grįžusieji iš tremties izraelitai ir visi, kurie buvo atsiskyrę nuo krašto pagonių nešvaros ir prisijungę prie jų, kad ieškotų Viešpaties, Izraelio Dievo. 
\par 22 Septynias dienas jie džiaugsmingai šventė Neraugintos duonos šventę, nes Viešpats suteikė jiems džiaugsmo ir palenkė į juos Asirijos karaliaus širdį, kad sustiprintų juos Izraelio Dievo namų statyboje.



\chapter{7}

\par 1 Po šitų įvykių, persų karaliui Artakserksui karaliaujant, atėjo Ezra, sūnus Serajos, sūnaus Azarijos, sūnaus Hilkijos, 
\par 2 sūnaus Šalumo, sūnaus Cadoko, sūnaus Ahitubo, 
\par 3 sūnaus Amarijos, sūnaus Azarijos, sūnaus Merajoto, 
\par 4 sūnaus Zerachijos, sūnaus Uzio, sūnaus Bukio, 
\par 5 sūnaus Abišūvos, sūnaus Finehaso, sūnaus Eleazaro, sūnaus vyriausiojo kunigo Aarono. 
\par 6 Šitas Ezra išėjo iš Babilono; jis buvo geras žinovas Mozės įstatymo, kurį davė Viešpats, Izraelio Dievas. Karalius suteikė jam viską, ko jis prašė, nes Viešpaties, jo Dievo, ranka buvo su juo. 
\par 7 Karaliaus Artakserkso septintaisias metais į Jeruzalę su juo išėjo kai kurie izraelitai, kunigai, levitai, giedotojai, vartininkai ir šventyklos tarnai. 
\par 8 Jis atėjo į Jeruzalę septintų karaliaus metų penktą mėnesį. 
\par 9 Ezra išėjo iš Babilono pirmo mėnesio pirmą dieną, o penkto mėnesio pirmą dieną jis atėjo į Jeruzalę, nes gera Dievo ranka buvo ant jo. 
\par 10 Ezra paruošė savo širdį tyrinėti Viešpaties įstatymą, jį vykdyti ir mokyti Izraelyje jo nuostatų ir teisės. 
\par 11 Tai yra nuorašas laiško, kurį Artakserksas davė kunigui Ezrai, kuris buvo Viešpaties įsakymų ir Jo nuostatų Izraelyje žinovas: 
\par 12 “Artakserksas, karalių karalius, kunigui Ezrai, dangaus Dievo įstatymo žinovui. 
\par 13 Aš įsakiau, kad kiekvienas izraelitas, kunigas bei levitas mano karalystėje gali vykti savo noru į Jeruzalę drauge su tavimi, 
\par 14 nes tu esi karaliaus ir jo septynių patarėjų siunčiamas į Judą ir Jeruzalę nustatyti, ar vykdomas tavo Dievo įstatymas, kuris yra tavo rankoje. 
\par 15 Tu nugabensi sidabrą ir auksą, kurį karalius ir jo patarėjai padovanojo Izraelio Dievui, kurio buveinė yra Jeruzalėje, 
\par 16 taip pat sidabrą ir auksą, kurį tu gausi Babilono krašte, kartu su tautos ir kunigų dovanomis, kurias jie padovanos savo Dievo namams, esantiems Jeruzalėje. 
\par 17 Nedelsdamas nupirk už tuos pinigus jaučių, avių, ėriukų ir kartu su jiems priklausančiomis duonos bei geriamosiomis aukomis aukok ant aukuro jūsų Dievo namuose Jeruzalėje. 
\par 18 Su likusiu sidabru ir auksu daryk, kas tau ir tavo broliams atrodys reikalinga, pagal jūsų Dievo valią. 
\par 19 Indus, kurie tau perduodami tavo Dievo namų reikalams, padėk priešais Jeruzalės Dievą. 
\par 20 Visa kita, ko reikės tavo Dievo namams, jei matysi esant reikalinga, paimk iš karaliaus iždo sandėlių. 
\par 21 Aš, karalius Artakserksas, įsakiau visiems iždininkams anapus upės: ‘Ko tik kunigas Ezra, dangaus Dievo įstatymo žinovas, iš jūsų reikalaus, nedelsdami duokite. 
\par 22 Iki šimto talentų sidabro, šimto saikų kviečių, šimto batų vyno, šimto batų aliejaus ir neribotą kiekį druskos. 
\par 23 Ko tik reikalaus dangaus Dievas, privalote padaryti Dievo namams, kad Jo rūstybė nepaliestų karaliaus ir jo sūnų. 
\par 24 Taip pat pranešu jums, kad neleidžiu apdėti jokiais mokesčiais kunigų, levitų, giedotojų, vartininkų ir šventyklos tarnų’. 
\par 25 O tu, Ezra, Dievo duota išmintimi paskirk mokytojų ir teisėjų, kurie teistų anapus upės gyvenančius žmones, pažįstančius tavo Dievo įstatymą; o kurie tų įstatymų nepažįsta, tuos pamokyk. 
\par 26 Kas nevykdys tavo Dievo ar karaliaus įstatymo, tas turi būti baudžiamas mirtimi, ištrėmimu, turto atėmimu arba kalėjimu”. 
\par 27 Garbė Viešpačiui, mūsų tėvų Dievui, kuris įdėjo tai į karaliaus širdį, kad Viešpaties namai Jeruzalėje būtų pagerbti. 
\par 28 Jis man suteikė malonę karaliaus, jo patarėjų ir galingų karaliaus kunigaikščių akyse. Aš buvau padrąsintas, kai Viešpaties, mano Dievo, ranka buvo ant manęs ir surinkau Izraelio vyresniuosius, kad jie eitų su manimi.



\chapter{8}


\par 1 Sąrašas šeimų vyresniųjų pagal jų kilmę, kurie, karaliui Artakserksui viešpataujant, išėjo su manimi iš Babilono: 
\par 2 iš Finehaso palikuonių­Geršomas, iš Itamaro­Danielius, iš Dovydo­Hatušas, 
\par 3 iš Šechanijos sūnų, Parošo palikuonių,­Zacharija, kartu su juo buvo užrašyta šimtas penkiasdešimt vyrų; 
\par 4 iš Pahat Moabo­Zerachijos sūnus Eljehoenajas ir su juo du šimtai vyrų; 
\par 5 iš Šechanijos­Jahazielio sūnus ir su juo trys šimtai vyrų; 
\par 6 iš Adino­Jehonatano sūnus Ebedas ir su juo penkiasdešimt vyrų; 
\par 7 iš Elamo­Atalijos sūnus Izaija ir su juo septyniasdešimt vyrų; 
\par 8 iš Šefatijos­Mykolo sūnus Zebadija ir su juo aštuoniasdešimt vyrų; 
\par 9 iš Joabo­Jehielio sūnus Abdija ir su juo du šimtai aštuoniolika vyrų; 
\par 10 iš Šelomito­Josifijos sūnus ir su juo šešiasdešimt vyrų; 
\par 11 iš Bebajo­Bebajo sūnus Zacharija ir su juo dvidešimt aštuoni vyrai; 
\par 12 iš Azgado­Hakatano sūnus Johananas ir su juo šimtas dešimt vyrų; 
\par 13 iš Adonikamo sūnų­Elifeletas, Jejelis, Šemaja ir su jais šešiasdešimt vyrų, kurie buvo paskutinieji; 
\par 14 iš Bigvajo­Utajas, Zabudas ir su jais septyniasdešimt vyrų. 
\par 15 Aš juos sušaukiau prie upės, tekančios į Ahavą. Ten mes pasilikome palapinėje tris dienas. Aš apžiūrėjau žmones ir kunigus, tačiau tarp jų neradau nė vieno levito. 
\par 16 Tada aš pasikviečiau vyresniuosius: Eliezerą, Arielį, Semają, Elnataną, Jaribą, Elnataną, Nataną, Zachariją, Mešulamą, ir mokytojus: Jehojaribą ir Elnataną. 
\par 17 Juos pasiunčiau pas Idoją, Kasifijos vietovės vyresnįjį, prašydamas atsiųsti mums žmonių, kurie tarnautų Dievo namuose. 
\par 18 Gera mūsų Dievo ranka buvo ant mūsų, ir jie atsivedė išmintingą vyrą iš palikuonių Machlio, sūnaus Levio, sūnaus Izraelio,­Šerebiją su jo sūnumis ir broliais, aštuoniolika vyrų; 
\par 19 iš Merario palikuonių­Hašabiją ir Izaiją su jo broliais ir sūnumis, dvidešimt vyrų; 
\par 20 ir šventyklos tarnų, kuriuos Dovydas ir kunigaikščiai buvo paskyrę tarnais levitams, buvo du šimtai dvidešimt vyrų. 
\par 21 Prie Ahavos upės paskelbiau pasninką, kad nusižemintume prieš savo Dievą ir melstume laimingo kelio sau, savo vaikams ir savo turtui. 
\par 22 Aš gėdijausi prašyti iš karaliaus karių ir raitelių, kad jie mus apgintų nuo užpuolikų kelyje, nes buvau pasakęs karaliui: “Mūsų Dievo ranka yra ant visų, kurie Jo ieško, bet Jo rūstybė ir galia prieš tuos, kurie Jį apleidžia”. 
\par 23 Mes pasninkavome ir meldėme savo Dievo dėl šito, ir Jis išklausė mus. 
\par 24 Aš parinkau iš vyresniųjų kunigų dvylika vyrų: Šerebiją, Hašabiją ir su jais dešimt jų brolių, 
\par 25 ir pasvėriau jiems sidabrą, auksą ir indus, Dievo namams skirtas dovanas, kurias paaukojo karalius, jo patarėjai, jo kunigaikščiai ir izraelitai. 
\par 26 Aš padaviau jiems šešis šimtus penkiasdešimt talentų sidabro, sidabrinių indų, sveriančių šimtą talentų, šimtą talentų aukso, 
\par 27 dvidešimt auksinių taurių tūkstančio drachmų vertės ir du indus iš geriausio, blizgančio kaip auksas, vario 
\par 28 ir tariau: “Jūs esate šventi Viešpačiui, indai taip pat šventi. Sidabras ir auksas yra laisvos valios aukos Viešpačiui, jūsų tėvų Dievui. 
\par 29 Budėkite ir saugokite tai, kol pasversite Jeruzalėje, Viešpaties namuose, kunigų, levitų ir Izraelio giminių vadų akivaizdoje”. 
\par 30 Kunigai ir levitai paėmė pasvertą sidabrą, auksą ir indus, kuriuos jie turėjo nugabenti į Jeruzalę, į Dievo namus. 
\par 31 Pirmo mėnesio dvyliktą dieną išėjome nuo Ahavos upės Jeruzalės link. Dievo ranka buvo ant mūsų, ir Jis saugojo mus nuo priešų ir užpuolikų. 
\par 32 Atėję į Jeruzalę, ten pasilikome tris dienas. 
\par 33 Ketvirtą dieną sidabras, auksas ir indai buvo pasverti Dievo namuose ir perduoti kunigui Meremotui, Ūrijos sūnui; su juo buvo Finehaso sūnus Eleazaras ir levitai: Jozuės sūnus Jehozabadas ir Binujo sūnus Noadija. 
\par 34 Viskas buvo suskaičiuota, pasverta ir surašyta. 
\par 35 Sugrįžę iš nelaisvės tremtiniai aukojo deginamąsias aukas Izraelio Dievui: dvylika jaučių už visą Izraelį, devyniasdešimt šešis avinus, septyniasdešimt septynis ėriukus, dvylika ožių aukai už nuodėmę; tai buvo deginamoji auka Viešpačiui. 
\par 36 Jie įteikė karaliaus įsakymus vietininkams ir krašto šiapus upės valdytojams, o šitie teikė paramą tautai ir Dievo namams.



\chapter{9}


\par 1 Po to prie manęs priėjo kunigaikščiai ir kalbėjo: “Izraelio tauta, kunigai ir levitai neatsiskyrė nuo šio krašto tautų: kanaaniečių, hetitų, perizų, jebusiečių, amonitų, moabitų, egiptiečių bei amoritų ir jų daromų bjaurysčių. 
\par 2 Jie ir jų vaikai ima į žmonas jų dukteris. Tuo būdu šventa sėkla susimaišė su krašto tautomis. Kunigaikščiai ir vyresnieji buvo pirmieji šiame nusikaltime”. 
\par 3 Išgirdęs visa tai, perplėšiau savo drabužį ir apsiaustą, roviau galvos ir barzdos plaukus ir sėdėjau sukrėstas. 
\par 4 Tuomet susirinko prie manęs visi, kurie drebėjo prieš Izraelio Dievo žodžius dėl tremtinių neištikimybės. O aš sėdėjau sukrėstas iki vakarinės aukos. 
\par 5 Vakarinės aukos metu atsikėliau iš savo liūdesio vietos perplėštais rūbais, atsiklaupiau ir, iškėlęs rankas į Viešpatį, savo Dievą, 
\par 6 tariau: “Mano Dieve, man gėda pakelti akis į Tave, nes mūsų nusikaltimai peraugo mus, o mūsų kaltė siekia dangų. 
\par 7 Nuo savo tėvų laikų iki šios dienos mes labai nusikaltome; dėl mūsų nusikaltimų mes, mūsų karaliai ir kunigai buvome atiduoti į kitų kraštų karalių rankas ir jų kardui, buvome jų nelaisvėje apiplėšti ir išniekinti. 
\par 8 Dabar trumpam laikui Viešpats, mūsų Dievas, parodė savo malonę, palikdamas mums išgelbėtą likutį ir duodamas kuolelį savo šventoje vietoje; taip Dievas atvėrė mūsų akis ir leido mums truputį atsigauti mūsų vergystėje. 
\par 9 Mes esame vergai, tačiau mūsų Dievas neapleido mūsų vergystėje; Jis suteikė mums malonę persų karalių akyse, kad jie leistų mums atsigauti ir atstatyti savo Dievo namus iš griuvėsių ir duotų mums sieną Jude ir Jeruzalėje. 
\par 10 O dabar, mūsų Dieve, ką mes pasakysime dėl šito? Mes apleidome Tavo įsakymus, 
\par 11 kuriuos Tu mums davei per savo tarnus pranašus, sakydamas: ‘Kraštas, kurį jūs einate paveldėti, yra suteptas. Jį sutepė krašto tautos savo bjaurystėmis ir pripildė jį savo nešvarumais nuo vieno krašto iki kito. 
\par 12 Todėl neleiskite savo dukterų už jų sūnų ir neimkite jų dukterų savo sūnums; nesiekite jų gėrybių nė taikos su jais, kad būtumėte stiprūs ir valgytumėte krašto gėrybes, ir paliktumėte kaip paveldėjimą savo vaikams’. 
\par 13 Kai visa tai užgriuvo ant mūsų už mūsų piktus darbus ir didelius nusikaltimus, vis dėlto Tu, mūsų Dieve, baudei mus švelniau, negu buvome verti, ir davei mums išgelbėjimą. 
\par 14 Argi mes vėl galėtume laužyti Tavo įsakymus ir susigiminiuoti vedybomis su šiomis bjauriomis tautomis? Argi Tu, užsirūstinęs ant mūsų, nesunaikinsi mūsų iki galo, nepalikdamas nė vieno, kuris išsigelbėtų? 
\par 15 Viešpatie, Izraelio Dieve, Tu esi teisus. Tu išgelbėjai mūsų likutį. Štai mes esame Tavo akivaizdoje su savo kaltėmis, nors neturėtume būti Tavo akivaizdoje dėl to”.



\chapter{10}

\par 1 Kai Ezra, klūpodamas ir verkdamas, meldėsi ir išpažino savo ir tautos nuodėmes prie Dievo namų, didelis būrys vyrų, moterų ir vaikų iš Izraelio, susirinkę aplink jį, graudžiai verkė. 
\par 2 Tada Jehielio sūnus Šechanija iš Elamo palikuonių atsiliepė: “Mes nusikaltome savo Dievui, vesdami svetimtautes moteris, tačiau dar yra vilties Izraeliui. 
\par 3 Padarykime sandorą su Dievu, kad atleisime visas žmonas ir iš jų gimusius vaikus, klausydami Viešpaties ir tų, kurie dreba prieš mūsų Dievo įsakymus. Darykime, kaip reikalauja įstatymas! 
\par 4 Kelkis, nes tu turi tai padaryti, o mes būsime su tavimi. Būk drąsus ir veik”. 
\par 5 Ezra atsikėlęs prisaikdino vyresniuosius kunigus ir levitus ir visą Izraelį, kad jie elgsis pagal tą žodį. Ir jie prisiekė. 
\par 6 Tada Ezra atėjo nuo Dievo namų į Eljašibo sūnaus Johanano kambarį, nieko nevalgė ir negėrė, nes liūdėjo dėl grįžusiųjų tremtinių nusikaltimo. 
\par 7 Jie paskelbė visame Jude ir Jeruzalėje visiems grįžusiems tremtiniams susirinkti į Jeruzalę. 
\par 8 Jei kas neateis per tris dienas, visas jo turtas kunigaikščių ir vyresniųjų nutarimu bus atimtas ir jis pats bus atskirtas nuo tremtinių. 
\par 9 Per tris dienas visi Judo ir Benjamino žmonės susirinko Jeruzalėje devinto mėnesio dvidešimtą dieną. Visi žmonės susėdo Dievo namų aikštėje, drebėdami dėl šito reikalo ir nuo smarkaus lietaus. 
\par 10 Kunigas Ezra atsistojęs tarė: “Jūs nusikaltote, vesdami svetimtautes moteris, ir tuo padidinote Izraelio kaltę. 
\par 11 Dabar išpažinkite savo kaltę Viešpačiui, savo tėvų Dievui, ir vykdykite Jo valią­atsiskirkite nuo šio krašto tautų ir nuo svetimtaučių moterų”. 
\par 12 Visi susirinkusieji garsiai atsakė: “Kaip pasakei, taip padarysime. 
\par 13 Žmonių yra daug, ir dabar lietingas metas, neįmanoma būti lauke. Be to, tas reikalas negali būti sutvarkytas per vieną ar dvi dienas, nes nusikaltusių tarp mūsų yra daug. 
\par 14 Tegul pasilieka mūsų vyresnieji. Visuose mūsų miestuose gyvenantieji, kurie yra vedę svetimtautes žmonas, tegul ateina paskirtu laiku su tų miestų vyresniaisiais ir teisėjais, kol pasitrauks nuo mūsų Dievo rūstybė dėl šito poelgio”. 
\par 15 Asaelio sūnus Jehonatanas ir Tikvos sūnus Jachzėja buvo paskirti tam darbui, o levitai Mešulamas ir Šabetajas jiems padėjo. 
\par 16 Grįžusieji tremtiniai taip ir padarė. Kunigas Ezra su šeimų vyresniaisiais, kurie buvo pašaukti vardais, buvo atskirti ir dešimto mėnesio pirmą dieną jie susėdo ištirti šį reikalą. 
\par 17 Iki pirmo mėnesio pirmos dienos jie pabaigė bylas su visais, kurie buvo vedę svetimtautes. 
\par 18 Tarp kunigų sūnų, vedusių svetimtautes, iš Jehocadako sūnaus Jozuės sūnų ir jo brolių­Maasėja, Eliezeras, Jaribas ir Gedolija. 
\par 19 Jie padavė rankas, kad atleis savo žmonas, ir už kaltę paaukojo po aviną. 
\par 20 Iš Imero­Hananis ir Zebadija. 
\par 21 Iš Harimo­Maasėja, Elija, Šemaja, Jehielis ir Uzija. 
\par 22 Iš Pašhūro­Eljoenajas, Maasėja, Izmaelis, Netanelis, Jehozabadas ir Eleasa. 
\par 23 Iš levitų palikuonių­Jehozabadas, Šimis, Kelaja, tai yra Kelita, Petahija, Judas ir Eliezeras. 
\par 24 Iš giedotojų­Eljašibas. Iš vartininkų—Šalumas, Telemas ir Ūris. 
\par 25 Kitų izraelitų palikuonys: iš Parošo palikuonių­Ramija, Izija, Malkija, Mijaminas, Eleazaras, Malkija ir Benaja; 
\par 26 iš Elamo­Matanija, Zacharija, Jehielis, Abdis, Jeremotas ir Elija; 
\par 27 iš Zatuvo­Eljoenajas, Eljašibas, Matanija, Jeremotas, Zabadas ir Aziza; 
\par 28 iš Bebajo­Johananas, Hananija, Zabajas ir Atlajas; 
\par 29 iš Banio­Mešulamas, Maluchas, Adaja, Jašubas, Šealas ir Jeramotas; 
\par 30 iš Pahat Moabo­Adna, Kelalas, Benaja, Maasėja, Matanija, Becalelis, Binujas ir Manasas; 
\par 31 iš Harimo­Eliezeras, Išija, Malkija, Šemaja, Simeonas, 
\par 32 Benjaminas, Maluchas ir Šemarija; 
\par 33 iš Hašumo­Matenajas, Matata, Zabadas, Elifeletas, Jeremajas, Manasas ir Šimis; 
\par 34 iš Banio­Maadajas, Amramas, Uelis, 
\par 35 Benaja, Bedija, Keluhis, 
\par 36 Vanija, Meremotas, Eljašibas, 
\par 37 Matanija, Matenajas ir Jaasajas, 
\par 38 Banis, Binujis, Šimis, 
\par 39 Šelemija, Natanas ir Adaja, 
\par 40 Machnadbajas, Šašajas, Šarajas, 
\par 41 Azarelis, Šelemijas, Šemarija, 
\par 42 Šalumas, Amarija ir Juozapas; 
\par 43 iš Nebojo­Jejelis, Matitija, Zabadas, Zebina, Jadajas, Joelis ir Benaja. 
\par 44 Visų šitų žmonos buvo svetimtautės, ir kai kurie iš jų turėjo vaikų su jomis.



\end{document}