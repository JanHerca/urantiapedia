\begin{document}

\title{Laiškas Titui}

\chapter{1}


\par 1 Paulius, Dievo tarnas ir Jėzaus Kristaus apaštalas, pagal Dievo išrinktųjų tikėjimą bei dievotumą atitinkantį tiesos pažinimą 
\par 2 su viltimi amžinojo gyvenimo, kurį nemeluojantis Dievas pažadėjo prieš amžinuosius laikus, 
\par 3 o nustatytam metui atėjus, apreiškė savo žodį skelbimu, kuris man patikėtas Dievo, mūsų Gelbėtojo, įsakymu,­ 
\par 4 Titui, tikram sūnui bendrame tikėjime: malonė, gailestingumas ir ramybė nuo Dievo Tėvo ir mūsų Gelbėtojo Viešpaties Jėzaus Kristaus! 
\par 5 Aš tam palikau tave Kretoje, kad sutvarkytum, kas buvo likę nesutvarkyta, ir paskirtum kiekviename mieste vyresniuosius, kaip esu įsakęs: 
\par 6 jei kas yra be priekaištų, vienos žmonos vyras, turįs ištikimus vaikus, kurie nekaltinami palaidu gyvenimu ar neklusnumu. 
\par 7 Nes vyskupas, kaip Dievo ūkvedys, turi būti be priekaištų: ne savavaliautojas, ne karštakošis, ne girtuoklis, ne kivirčius, ne geidžiantis nešvaraus pelno, 
\par 8 bet svetingas, trokštantis gero, blaiviai mąstantis, teisingas, šventas, susivaldantis, 
\par 9 tvirtai besilaikantis patikimo žodžio kaip buvo išmokytas, kad sveiku mokymu sugebėtų ir paraginti, ir įtikinti prieštaraujančius. 
\par 10 Nes daug yra neklusnių, tuščiakalbių ir apgavikų, ypač iš apipjaustytųjų. 
\par 11 Juos reikia užčiaupti, nes jie apverčia aukštyn kojomis ištisas šeimynas, mokydami, kas nedera, dėl nesąžiningo pelno. 
\par 12 Vienas iš jų, jų pačių pranašas, yra pasakęs: “Kretiečiai visada melagiai, pikti žvėrys, tingūs pilvai”. 
\par 13 Šis liudijimas teisingas. Todėl sudrausk juos griežtai, kad būtų sveiki tikėjimu, 
\par 14 nekreipdami dėmesio į žydų pasakas ir į žmonių priesakus, nukreipiančius nuo tiesos. 
\par 15 Tyriems viskas tyra, o susitepusiems ir netikintiems nieko nėra tyro, net ir jų protas bei sąžinė suteršti. 
\par 16 Jie skelbiasi pažįstą Dievą, o darbais Jį neigia; jie pasibjaurėtini, nepaklusnūs, netikę jokiam geram darbui.


\chapter{2}


\par 1 Kalbėk, kas sutinka su sveiku mokymu: 
\par 2 kad pagyvenę vyrai būtų blaivūs, rimti, santūrūs, sveiki tikėjimu, meile, kantrybe. 
\par 3 Taip pat, kad pagyvenusios moterys elgtųsi garbingai, nebūtų apkalbinėtojos, besaikės vyno gėrėjos, mokytų gero, 
\par 4 skatintų jaunąsias mylėti savo vyrus ir vaikus, 
\par 5 būti santūrias, tyras, rūpestingas šeimininkes, geras, klusnias savo vyrams,­kad nebūtų šmeižiamas Dievo žodis. 
\par 6 Taip pat jaunuosius ragink, kad būtų santūrūs. 
\par 7 Pats visais atžvilgiais rodyk gerų darbų pavyzdį: mokymo grynumą, garbingumą, nepažeidžiamumą, 
\par 8 sveiką ir nepriekaištingą kalbą, kad priešininkas liktų sugėdintas, neturėdamas apie tave pasakyti nieko blogo. 
\par 9 Ragink, kad vergai būtų klusnūs savo šeimininkams, stengtųsi visame kame jiems įtikti ir neprieštarautų, 
\par 10 kad nevogtų, bet rodytų visokeriopą gerą ištikimybę, kad visu kuo puoštų Dievo, mūsų Gelbėtojo, mokymą. 
\par 11 Nes gelbstinti Dievo malonė pasirodė visiems žmonėms 
\par 12 ir moko mus, kad, atsisakę bedievystės ir pasaulio geidulių, santūriai, teisiai ir pamaldžiai gyventume šiame amžiuje, 
\par 13 laukdami palaimintosios vilties ir mūsų didžiojo Dievo bei Gelbėtojo Jėzaus Kristaus šlovės pasirodymo. 
\par 14 Jis atidavė save už mus, kad išpirktų mus iš visų nedorybių ir nuskaistintų sau ypatingą tautą, uolią geriems darbams. 
\par 15 Taip kalbėk, ragink, drausk su visa valdžia. Niekas tenedrįsta tavęs niekinti!


\chapter{3}


\par 1 Primink jiems, kad būtų paklusnūs valdytojams ir valdžioms, kad būtų pasiruošę kiekvienam geram darbui, 
\par 2 niekam nepiktžodžiautų, nesikivirčytų, būtų nuosaikūs, visiems žmonėms rodytų visokeriopą romumą. 
\par 3 Juk ir mes kitados buvome neprotingi, neklusnūs, apgauti, vergaujantys įvairiems geiduliams ir malonumams, gyvenome pilni pykčio ir pavydo, neapkenčiami ir nekenčiantys vieni kitų. 
\par 4 Bet kai pasirodė Dievo, mūsų Gelbėtojo, gerumas ir meilė žmonėms, 
\par 5 Jis išgelbėjo mus ne dėl mūsų atliktų teisumo darbų, bet iš savo gailestingumo, Šventosios Dvasios atgimdančiu ir atnaujinančiu nuplovimu. 
\par 6 Jis mums dosniai išliejo tos Dvasios per mūsų Gelbėtoją Jėzų Kristų, 
\par 7 kad, išteisinti Jo malone, viltimi taptume amžinojo gyvenimo paveldėtojais. 
\par 8 Tai patikimas žodis. Ir aš noriu, kad tu nuolatos pabrėžtum šiuos dalykus, kad patikėjusieji Dievu rūpintųsi paremti gerus darbus. Tai gera ir naudinga žmonėms. 
\par 9 Venk kvailų ginčų, kilmės sąrašų, nesutarimų ir vaidų dėl įstatymo,­visa tai nenaudinga ir tuščia. 
\par 10 Atskalūno, vieną kitą kartą įspėjęs, šalinkis, 
\par 11 žinodamas, jog jis iškrypęs ir nuodėmiauja, savo paties nuosprendžiu pasmerktas. 
\par 12 Kai nusiųsiu pas tave Artemą ar Tichiką, paskubėk atvykti pas mane į Nikopolį; mat nutariau ten žiemoti. 
\par 13 Teisininką Zeną ir Apolą išleisk į kelionę gerai aprūpintus, kad jiems nieko netrūktų. 
\par 14 Tesimoko ir mūsiškiai remti gerus darbus būtiniems reikalams patenkinti, kad neliktų bevaisiai. 
\par 15 Tave sveikina visi esantys su manimi. Pasveikink tuos, kurie myli mus tikėjime. Malonė su jumis visais! Amen.


\end{document}