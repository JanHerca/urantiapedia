\begin{document}

\title{Evangelija pagal Luką}

\chapter{1}


\par 1 Kadangi daugelis rašė pasakojimą apie pas mus buvusius įvykius, 
\par 2 kaip mums perdavė nuo pradžios savo akimis mačiusieji ir buvusieji žodžio tarnai, 
\par 3 tai ir aš, rūpestingai viską nuo pradžios ištyręs, nusprendžiau surašyti tau, garbingasis Teofiliau, sutvarkytą pasakojimą, 
\par 4 kad įsitikintum tikrumu mokymo, kurio buvai išmokytas. 
\par 5 Judėjos karaliaus Erodo dienomis gyveno kunigas, vardu Zacharijas, iš Abijos skyriaus. Jis turėjo žmoną, vardu Elžbietą, iš Aarono palikuonių. 
\par 6 Jie abu buvo teisūs Dievo akyse ir nepriekaištingai vykdė visus Viešpaties įsakymus bei nuostatus. 
\par 7 Juodu neturėjo vaikų, nes Elžbieta buvo nevaisinga, ir abu sulaukę senyvo amžiaus. 
\par 8 Kartą Zacharijas, atėjus eilei, tarnavo Dievui kaip kunigas ir, pagal paprotį, 
\par 9 kunigų burtu teko jam, įėjus į Viešpaties šventyklą, smilkyti smilkalus. 
\par 10 Smilkymo valandą lauke meldėsi gausi žmonių minia. 
\par 11 Tada jam pasirodė Viešpaties angelas, stovintis smilkymo aukuro dešinėje. 
\par 12 Pamatęs jį, Zacharijas sumišo, ir jį apėmė baimė. 
\par 13 Bet angelas jam tarė: “Nebijok, Zacharijau, nes tavo malda išklausyta. Tavo žmona Elžbieta pagimdys tau sūnų, o tu jį pavadinsi Jonu. 
\par 14 Tau bus džiaugsmas ir linksmybė, ir daugelis džiaugsis jo gimimu, 
\par 15 nes jis bus didis Viešpaties akyse. Jis negers vyno nei stiprių gėrimų. Ir nuo pat gimimo jis bus kupinas Šventosios Dvasios, 
\par 16 ir daugybę Izraelio vaikų atvers į Viešpatį, jų Dievą. 
\par 17 Elijo dvasia ir jėga jis eis pirma Viešpaties, kreipdamas tėvų širdis į vaikus ir neklusniuosius į teisiųjų nusistatymą, kad parengtų Viešpačiui paruoštą tautą”. 
\par 18 Tada Zacharijas atsakė angelui: “Kaip tai aš patirsiu? Aš gi jau senas, ir mano žmona nebejauna”. 
\par 19 Angelas jam atsakė: “Aš esu Gabrielius, stovintis Dievo akivaizdoje. Esu atsiųstas kalbėti su tavimi ir pranešti tau šią linksmą žinią. 
\par 20 Štai tu tapsi nebylys ir negalėsi kalbėti iki tos dienos, kurią tai įvyks, nes nepatikėjai mano žodžiais, kurie išsipildys savo metu”. 
\par 21 Tuo tarpu žmonės laukė Zacharijo ir stebėjosi, kad jis taip ilgai užtrunka šventykloje. 
\par 22 Išėjęs jis negalėjo prakalbėti, ir jie suprato, kad jis turėjęs šventykloje regėjimą. Jis aiškinosi jiems ženklais ir pasiliko nebylys. 
\par 23 Tarnavimo dienoms pasibaigus, jis grįžo namo. 
\par 24 Praslinkus kiek laiko, jo žmona Elžbieta pastojo ir penkis mėnesius slėpėsi, sakydama: 
\par 25 “Tai Viešpats man davė; Jis dabar teikėsi atimti mano pažeminimą žmonių akyse”. 
\par 26 Šeštame mėnesyje angelas Gabrielius buvo Dievo pasiųstas į Galilėjos miestą Nazaretą 
\par 27 pas mergelę, sužadėtą su vyru, vardu Juozapas, iš Dovydo namų; o mergelės vardas buvo Marija. 
\par 28 Atėjęs pas ją, angelas tarė: “Sveika, malonėmis apdovanotoji! Viešpats su tavimi! Palaiminta tu tarp moterų!” 
\par 29 Jį pamačiusi, ji sumišo nuo jo žodžių ir galvojo, ką toks pasveikinimas reiškia. 
\par 30 O angelas jai tarė: “Nebijok, Marija, tu radai malonę pas Dievą! 
\par 31 Štai tu pradėsi įsčiose ir pagimdysi Sūnų, kurį pavadinsi Jėzumi. 
\par 32 Jis bus didis ir vadinsis Aukščiausiojo Sūnus. Viešpats Dievas duos Jam Jo tėvo Dovydo sostą; 
\par 33 Jis valdys Jokūbo namus per amžius, ir Jo karalystei nebus galo”. 
\par 34 Marija paklausė angelą: “Kaip tai įvyks, jeigu aš nepažįstu vyro?” 
\par 35 Angelas jai atsakė, tardamas: “Šventoji Dvasia nužengs ant tavęs, ir Aukščiausiojo jėga apgaubs tave; todėl ir gimęs iš tavęs bus šventas ir vadinamas Dievo Sūnumi. 
\par 36 Tavo giminaitė Elžbieta, kuri buvo laikoma nevaisinga, pradėjo sūnų senatvėje, ir šis mėnuo yra šeštas jai, 
\par 37 nes Dievui nėra negalimų dalykų”. 
\par 38 Tada Marija atsakė: “Štai aš Viešpaties tarnaitė, tebūnie man pagal tavo žodį”. Ir angelas nuo jos pasitraukė. 
\par 39 Tomis dienomis Marija atsikėlusi skubiai iškeliavo į Judėjos kalnyno miestą. 
\par 40 Ji nuėjo į Zacharijo namus ir pasveikino Elžbietą. 
\par 41 Vos tik Elžbieta išgirdo Marijos sveikinimą, suspurdėjo kūdikis jos įsčiose, o pati Elžbieta tapo kupina Šventosios Dvasios. 
\par 42 Ji balsiai sušuko ir tarė: “Palaiminta tu tarp moterų, ir palaimintas tavo įsčių vaisius! 
\par 43 Iš kur man tai, kad mano Viešpaties motina aplanko mane?! 
\par 44 Štai vos tik tavo pasveikinimas pasiekė mano ausis, suspurdėjo iš džiaugsmo kūdikis mano įsčiose. 
\par 45 Laiminga patikėjusi, nes išsipildys, kas Viešpaties jai pasakyta”. 
\par 46 O Marija prabilo: “Mano siela šlovina Viešpatį, 
\par 47 ir mano dvasia džiaugiasi Dievu, savo Gelbėtoju, 
\par 48 nes Jis pažvelgė į nuolankią savo tarnaitę. Štai nuo dabar palaiminta mane vadins visos kartos, 
\par 49 nes didžių dalykų padarė man Galingasis, ir šventas yra Jo vardas! 
\par 50 Jis gailestingas iš kartos į kartą tiems, kurie Jo bijosi. 
\par 51 Jis parodė savo rankos galybę ir išsklaidė išdidžios širdies žmones. 
\par 52 Jis numėtė galiūnus nuo sostų ir išaukštino žemuosius. 
\par 53 Alkstančius gėrybėmis apdovanojo, turtuolius tuščiomis paleido. 
\par 54 Jis padėjo savo tarnui Izraeliui, prisimindamas gailestingumą, 
\par 55 kaip buvo žadėjęs mūsų protėviams­Abraomui ir jo palikuonims per amžius”. 
\par 56 Marija išbuvo su Elžbieta apie tris mėnesius ir sugrįžo į savo namus. 
\par 57 Elžbietai atėjo metas gimdyti, ir ji susilaukė sūnaus. 
\par 58 Jos kaimynai ir giminės, išgirdę, kokį didį gailestingumą parodė jai Viešpats, džiaugėsi kartu su ja. 
\par 59 Aštuntą dieną jie susirinko berniuko apipjaustyti ir norėjo jį pavadinti tėvo vardu­Zachariju. 
\par 60 Atsakydama jo motina tarė: “O, ne! Jis vadinsis Jonas”. 
\par 61 Jie jai sakė: “Bet niekas tavo giminėje neturi šito vardo”. 
\par 62 Jie ženklais paklausė tėvą, kaip jis norėtų pavadinti kūdikį. 
\par 63 Šis, pareikalavęs rašomosios lentelės, užrašė: “Jo vardas­Jonas”. Ir visi stebėjosi. 
\par 64 Tuoj pat atsivėrė jo lūpos, atsirišo liežuvis, ir jis kalbėjo, šlovindamas Dievą. 
\par 65 Visus kaimynus apėmė baimė, ir po visą Judėjos kalnyną sklido kalbos apie šiuos įvykius. 
\par 66 Visi girdėjusieji dėjosi tai į širdį ir klausinėjo: “Kas gi bus iš to vaiko?” Ir Viešpaties ranka buvo su juo. 
\par 67 Kūdikio tėvas Zacharijas tapo pilnas Šventosios Dvasios ir pranašavo: 
\par 68 “Tebūna palaimintas Viešpats, Izraelio Dievas, kad aplankė savo tautą ir atnešė jai išvadavimą. 
\par 69 Jis iškėlė mums išgelbėjimo ragą savo tarno Dovydo namuose, 
\par 70 kaip nuo senų senovės buvo skelbęs savo šventųjų pranašų lūpomis, 
\par 71 jog mus išgelbės nuo priešų ir iš rankos tų, kurie mūsų nekenčia, 
\par 72 tuo parodydamas mūsų protėviams gailestingumą ir atsimindamas savo šventąją sandorą, 
\par 73 priesaiką, duotą mūsų tėvui Abraomui, jog leis mums, 
\par 74 išvaduotiems iš priešų, be baimės Jam tarnauti 
\par 75 per visas mūsų gyvenimo dienas šventumu ir teisumu Jo akyse.­ 
\par 76 O tu, vaikeli, būsi vadinamas Aukščiausiojo pranašu, nes eisi pirma Viešpaties veido Jam kelio paruošti; 
\par 77 tu mokysi Jo žmones pažinti išgelbėjimą per jų nuodėmių atleidimą 
\par 78 ir širdingiausią mūsų Dievo gailestingumą, su kuriuo aplankė mus aušra iš aukštybių, 
\par 79 kad apšviestų esančius tamsoje ir mirties šešėlyje, kad pakreiptų mūsų žingsnius į ramybės kelią”. 
\par 80 Kūdikis augo ir tvirtėjo dvasia. Jis gyveno dykumoje iki pat savo viešo pasirodymo Izraeliui dienos.


\chapter{2}


\par 1 Tomis dienomis išėjo ciesoriaus Augusto įsakymas surašyti visus valstybės gyventojus. 
\par 2 Toks pirmasis surašymas buvo padarytas Kvirinui valdant Siriją. 
\par 3 Taigi visi keliavo užsirašyti, kiekvienas į savo miestą. 
\par 4 Taip pat ir Juozapas ėjo iš Galilėjos miesto Nazareto į Judėją, į Dovydo miestą, vadinamą Betliejumi, nes buvo kilęs iš Dovydo namų ir giminės. 
\par 5 Jis ėjo užsirašyti kartu su savo sužadėtine Marija, kuri buvo nėščia. 
\par 6 Jiems ten esant, atėjo jai metas gimdyti, ir ji pagimdė savo pirmagimį Sūnų, 
\par 7 suvystė Jį vystyklais ir paguldė ėdžiose, nes jiems nebuvo vietos užeigoje. 
\par 8 Toje apylinkėje laukuose buvo piemenys, kurie, budėdami naktį, saugojo savo bandą. 
\par 9 Staiga jiems pasirodė Viešpaties angelas, ir juos apšvietė Viešpaties šlovė. Jie labai išsigando, 
\par 10 bet angelas jiems tarė: “Nebijokite! Štai skelbiu jums didelį džiaugsmą, kuris bus visai tautai. 
\par 11 Šiandien Dovydo mieste jums gimė Gelbėtojas. Jis yra Viešpats­Kristus. 
\par 12 Ir štai jums ženklas: rasite kūdikį, suvystytą ir paguldytą ėdžiose”. 
\par 13 Staiga prie angelo pasirodė gausi dangaus kareivija, šlovinanti Dievą: 
\par 14 “Šlovė Dievui aukštybėse, o žemėje ramybė ir palankumas žmonėms!” 
\par 15 Kai angelai nuo jų pakilo į dangų, piemenys kalbėjosi: “Eikime į Betliejų ir pažiūrėkime, kas ten įvyko, ką Viešpats mums paskelbė”. 
\par 16 Jie nuskubėjo ir rado Mariją, Juozapą ir kūdikį, paguldytą ėdžiose. 
\par 17 Pamatę jie apsakė, kas jiems buvo pranešta apie šitą kūdikį. 
\par 18 Visi, kurie girdėjo, stebėjosi piemenų pasakojimu. 
\par 19 Marija įsiminė visus šiuos žodžius, dėdamasi juos širdin. 
\par 20 Piemenys grįžo atgal, garbindami ir šlovindami Dievą už visa, ką buvo girdėję ir matę, kaip jiems buvo paskelbta. 
\par 21 Praslinkus aštuonioms dienoms, kai reikėjo apipjaustyti vaikelį, Jam buvo duotas Jėzaus vardas, kurį angelas nurodė dar prieš Jo pradėjimą įsčiose. 
\par 22 Pasibaigus Mozės Įstatymo nustatytoms apsivalymo dienoms, jie nunešė Jį į Jeruzalę pašvęsti Viešpačiui,­ 
\par 23 kaip parašyta Viešpaties Įstatyme: “Kiekvienas pirmagimis berniukas bus atskirtas Viešpačiui”,­ 
\par 24 ir duoti auką, kaip pasakyta Viešpaties Įstatyme: “Porą purplelių arba du balandžiukus”. 
\par 25 Jeruzalėje gyveno žmogus, vardu Simeonas, teisus ir dievobaimingas vyras, kuris laukė Izraelio paguodos, ir Šventoji Dvasia buvo ant jo. 
\par 26 Jam buvo apreikšta Šventąja Dvasia, kad jis nemirsiąs, kol pamatysiąs Viešpaties Kristų. 
\par 27 Dvasios paragintas, jis atėjo į šventyklą. Įnešant tėvams kūdikį Jėzų, kad pasielgtų su Juo, kaip Įstatymas reikalauja, 
\par 28 Simeonas paėmė Jį į rankas, laimino Dievą ir tarė: 
\par 29 “Dabar, Valdove, leidi, kaip žadėjai, savo tarnui ramiai iškeliauti, 
\par 30 nes mano akys išvydo Tavo išgelbėjimą, 
\par 31 kurį paruošei visų tautų akivaizdoje: 
\par 32 šviesą pagonims apšviesti ir Tavo Izraelio tautos šlovę”. 
\par 33 Juozapas ir Jėzaus motina stebėjosi tuo, kas buvo apie Jį kalbama. 
\par 34 Simeonas palaimino juos ir tarė Marijai, Jo motinai: “Štai šis skirtas daugelio Izraelyje nupuolimui ir atsikėlimui. Jis bus prieštaravimo ženklas,­ 
\par 35 ir tavo pačios sielą pervers kalavijas,­kad būtų atskleistos daugelio širdžių mintys”. 
\par 36 Ten buvo ir pranašė Ona, Fanuelio duktė iš Asero giminės. Ji buvo seno amžiaus. Po mergystės ji išgyveno septynerius metus su vyru, 
\par 37 o paskui našlaudama sulaukė aštuoniasdešimt ketverių metų. Ji nesitraukdavo nuo šventyklos, tarnaudama Dievui per dienas ir naktis pasninkais bei maldomis. 
\par 38 Ir ji, tuo pačiu metu priėjusi, dėkojo Dievui ir kalbėjo apie kūdikį visiems, kurie laukė Jeruzalės atpirkimo. 
\par 39 Atlikę visa, ko reikalavo Viešpaties Įstatymas, jie sugrįžo į Galilėją, į savo miestą Nazaretą. 
\par 40 Vaikelis augo, stiprėjo dvasia, buvo pilnas išminties, ir Dievo malonė buvo su Juo. 
\par 41 Jo tėvai kasmet eidavo į Jeruzalę švęsti Paschos. 
\par 42 Kai Jėzui sukako dvylika metų, šventės papročiu jie nuvyko į Jeruzalę. 
\par 43 Pasibaigus šventės dienoms ir jiems grįžtant atgal, vaikas Jėzus pasiliko Jeruzalėje, bet Juozapas ir Jo motina to nepastebėjo. 
\par 44 Manydami Jį esant keleivių būryje, jie, nuėję dienos kelią, pradėjo ieškoti Jo tarp giminių ir pažįstamų. 
\par 45 Nesuradę grįžo Jo beieškodami į Jeruzalę. 
\par 46 Pagaliau po trijų dienų rado Jį šventykloje, sėdintį tarp mokytojų, besiklausantį jų ir juos beklausinėjantį. 
\par 47 Visi, kurie Jį girdėjo, stebėjosi Jo išmanymu ir atsakymais. 
\par 48 Pamatę Jį, jie labai nustebo, ir Jo motina Jam tarė: “Vaikeli, kodėl mums taip padarei? Štai Tavo tėvas ir aš sielvartaudami ieškojome Tavęs”. 
\par 49 Jis atsakė: “Kam gi manęs ieškojote? Argi nežinote, kad man reikia būti savo Tėvo reikaluose?” 
\par 50 Bet jie nesuprato Jo pasakytų žodžių. 
\par 51 Jis iškeliavo su jais ir grįžo į Nazaretą; ir buvo jiems klusnus. Jo motina laikė visus tuos žodžius savo širdyje. 
\par 52 O Jėzus augo išmintimi, metais ir malone Dievo ir žmonių akyse.



\chapter{3}


\par 1 Penkioliktais ciesoriaus Tiberijaus viešpatavimo metais, Poncijui Pilotui valdant Judėją, Erodui esant Galilėjos tetrarchu, jo broliui Pilypui­Iturėjos bei Trachonitidės krašto tetrarchu, Lisanijui­Abilenės tetrarchu, 
\par 2 prie vyriausiųjų kunigų Ano ir Kajafo, buvo Viešpaties žodis Zacharijo sūnui Jonui dykumoje. 
\par 3 Jis apėjo visą Pajordanę, skelbdamas atgailos krikštą nuodėmėms atleisti, 
\par 4 kaip parašyta pranašo Izaijo žodžių knygoje: “Dykumoje šaukiančiojo balsas: ‘Paruoškite Viešpačiui kelią, ištiesinkite Jam takus! 
\par 5 Kiekvienas slėnis tebūna užpiltas, kiekvienas kalnas bei kalnelis­nulygintas. Kreivi keliai tetampa tiesūs, o duobėti­išlyginti. 
\par 6 Ir kiekvienas kūnas išvys Dievo išgelbėjimą’ ”. 
\par 7 Ateinančioms pas jį krikštytis minioms Jonas sakė: “Angių išperos, kas perspėjo jus bėgti nuo besiartinančios rūstybės? 
\par 8 Duokite vaisių, vertų atgailos! Ir nebandykite ramintis: ‘Mūsų tėvas­Abraomas’. Aš jums sakau, kad Dievas gali pažadinti Abraomui vaikų iš šitų akmenų. 
\par 9 Štai kirvis jau prie medžių šaknų. Kiekvienas medis, kuris neduoda gerų vaisių, yra nukertamas ir įmetamas į ugnį”. 
\par 10 Minios jį klausė: “Ką gi mums daryti?!” 
\par 11 Jis joms atsakė: “Kas turi dvi tunikas, tepasidalina su neturinčiu, ir kas turi maisto, tegul taip pat daro”. 
\par 12 Ėjo ir muitininkai krikštytis ir klausė: “Mokytojau, ką mums daryti?” 
\par 13 Jis sakė jiems: “Nereikalaukite daugiau, kaip jums nustatyta”. 
\par 14 Taip pat ir kariai klausė: “Ką mums daryti?” Jis jiems atsakė: “Nieko neskriauskite, melagingai neskųskite, tenkinkitės savo alga”. 
\par 15 Žmonėms esant apimtiems lūkesčio ir visiems mąstant širdyse apie Joną, ar kartais jis ne Kristus, 
\par 16 Jonas visiems kalbėjo: “Aš, tiesa, krikštiju jus vandeniu, bet ateina už mane galingesnis, kuriam aš nevertas atrišti sandalų dirželio. Jis krikštys jus Šventąja Dvasia ir ugnimi. 
\par 17 Jo rankoje vėtyklė: Jis kruopščiai išvalys savo kluoną ir surinks kviečius į klėtį, o pelus sudegins neužgesinama ugnimi”. 
\par 18 Ir dar daug kitų paraginimų jis davė tautai, skelbdamas Gerąją naujieną. 
\par 19 Tetrarchas Erodas, Jono baramas dėl Erodiados, savo brolio žmonos, ir dėl visų piktadarybių, kurias buvo padaręs, 
\par 20 pridėjo prie jų dar ir tai, kad uždarė Joną į kalėjimą. 
\par 21 Kai, visai tautai krikštijantis, ir Jėzus pakrikštytas meldėsi, atsivėrė dangus, 
\par 22 ir Šventoji Dvasia kūnišku pavidalu nusileido ant Jo tarsi balandis, o balsas iš dangaus prabilo: “Tu mano mylimasis Sūnus, Tavimi Aš gėriuosi”. 
\par 23 Jėzui buvo apie trisdešimt metų, kai Jis pradėjo veikti. Jis buvo laikomas sūnumi Juozapo, Helio, 
\par 24 Matato, Levio, Melchio, Janajo, Juozapo, 
\par 25 Matatijo, Amoso, Naumo, Heslio, Nagajo, 
\par 26 Maato, Matatijo, Semeino, Josecho, Jodos, 
\par 27 Joanano, Resos, Zorobabelio, Salatielio, Nerio, 
\par 28 Melchio, Adijo, Kosamo, Elmadamo, Ero, 
\par 29 Jėzaus, Eliezero, Jorimo, Matato, Levio, 
\par 30 Simeono, Judo, Juozapo, Jonamo, Eliakimo, 
\par 31 Melėjo, Menos, Matatos, Natano, Dovydo, 
\par 32 Jesės, Jobedo, Boozo, Salmono, Naasono, 
\par 33 Aminadabo, Aramo, Esromo, Faro, Judo, 
\par 34 Jokūbo, Izaoko, Abraomo, Taros, Nachoro, 
\par 35 Serucho, Ragaujo, Faleko, Ebero, Salos, 
\par 36 Kainamo, Arfaksado, Semo, Nojaus, Lamecho, 
\par 37 Matūzalio, Henocho, Jareto, Maleleelio, Kainamo, 
\par 38 Eno, Seto, Adomo, Dievo.



\chapter{4}


\par 1 Kupinas Šventosios Dvasios Jėzus grįžo nuo Jordano, ir Dvasia Jį nuvedė į dykumą 
\par 2 keturiasdešimčiai dienų, ir Jis buvo velnio gundomas. Jis nieko nevalgė per tas dienas ir, joms pasibaigus, buvo alkanas. 
\par 3 Tada velnias Jam tarė: “Jei Tu Dievo Sūnus, liepk, kad šitas akmuo pavirstų duona”. 
\par 4 Jėzus jam atsakė: “Parašyta: ‘Žmogus gyvens ne viena duona, bet kiekvienu Dievo žodžiu’ ”. 
\par 5 Tada velnias, užvedęs Jį į aukštą kalną, viena akimirka parodė Jam visas pasaulio karalystes. 
\par 6 Velnias Jam tarė: “Duosiu Tau visą jų valdžią ir šlovę; jos man atiduotos, ir kam noriu, tam jas duodu. 
\par 7 Todėl, jei parpuolęs pagarbinsi mane, visa bus Tavo”. 
\par 8 O Jėzus jam atsakė: “Eik šalin nuo manęs, šėtone! Parašyta: ‘Viešpatį, savo Dievą, tegarbink ir Jam vienam tetarnauk!’ ” 
\par 9 Velnias nusivedė Jį į Jeruzalę, pastatė ant šventyklos šelmens ir tarė: “Jei Tu Dievo Sūnus, pulk žemyn, 
\par 10 nes parašyta: ‘Jis lieps savo angelams saugoti Tave, 
\par 11 ir jie nešios Tave ant rankų, kad neužsigautum kojos į akmenį’ ”. 
\par 12 Jėzus jam atsakė: “Pasakyta: ‘Negundyk Viešpaties, savo Dievo!’ ” 
\par 13 Baigęs visus gundymus, velnias pasitraukė nuo Jo iki laiko. 
\par 14 Su Dvasios jėga Jėzus grįžo į Galilėją, ir visame krašte pasklido apie Jį garsas. 
\par 15 Jis mokė jų sinagogose, visų gerbiamas. 
\par 16 Jis atėjo į Nazaretą, kur buvo užaugęs. Sabato dieną, kaip pratęs, nuėjo į sinagogą ir atsistojo skaityti. 
\par 17 Jam padavė pranašo Izaijo knygos ritinį. Atvyniojęs ritinį, Jis rado vietą, kur parašyta: 
\par 18 “Viešpaties Dvasia ant manęs, nes Jis patepė mane skelbti Gerąją naujieną vargšams, pasiuntė mane gydyti tų, kurių širdys sudužusios, skelbti belaisviams išvadavimo, akliesiems­regėjimo, siuntė vaduoti prislėgtųjų 
\par 19 ir skelbti maloningųjų Viešpaties metų”. 
\par 20 Suvyniojęs knygos ritinį, Jėzus grąžino jį patarnautojui ir atsisėdo; visų sinagogoje esančių akys buvo įsmeigtos į Jį. 
\par 21 Ir Jis pradėjo jiems kalbėti: “Šiandien išsipildė ką tik jūsų girdėti Rašto žodžiai”. 
\par 22 Visi Jam pritarė ir stebėjosi maloniais žodžiais, sklindančiais iš Jo lūpų. Ir jie sakė: “Argi Jis ne Juozapo sūnus?!” 
\par 23 O Jėzus jiems atsakė: “Jūs, be abejo, man priminsite patarlę: ‘Gydytojau, pats pasigydyk’­padaryk ir čia, savo tėviškėje, darbų, kokių girdėjome buvus Kafarnaume”. 
\par 24 Ir Jis tarė: “Iš tiesų sakau jums: joks pranašas nepriimamas savo tėviškėje. 
\par 25 Bet sakau jums tiesą: daug našlių buvo Izraelyje Elijo dienomis, kai dangus buvo uždarytas trejus metus ir šešis mėnesius ir baisus badas ištiko visą kraštą. 
\par 26 Bet nė pas vieną iš jų nebuvo siųstas Elijas, o tik pas našlę Sidono mieste, Sareptoje. 
\par 27 Taip pat pranašo Eliziejaus laikais daug buvo raupsuotųjų Izraelyje, bet nė vienas iš jų nebuvo išgydytas, tik siras Naamanas”. 
\par 28 Visi, kurie buvo sinagogoje, tai išgirdę, labai užsirūstino; 
\par 29 pakilę išvarė Jį iš miesto, iki šlaito to kalno, ant kurio pastatytas jų miestas, ir norėjo nustumti Jį žemyn. 
\par 30 Bet Jėzus, praėjęs tarp jų, pasišalino. 
\par 31 Jis nuėjo į Galilėjos miestą Kafarnaumą ir kiekvieną sabatą mokė žmones. 
\par 32 Jie labai stebėjosi Jo mokymu, nes Jo žodžiai buvo su valdžia. 
\par 33 Sinagogoje buvo žmogus, kuris turėjo netyrą demonišką dvasią. Jis pradėjo garsiai šaukti: 
\par 34 “Šalin! Ko Tau iš mūsų reikia, Jėzau iš Nazareto?! Gal atėjai mūsų pražudyti? Aš žinau, kas Tu: Dievo Šventasis!” 
\par 35 Jėzus sudraudė jį: “Nutilk ir išeik iš jo!” Nubloškęs žmogų į vidurį, demonas išėjo, nė kiek jo nesužeidęs. 
\par 36 Visi nustėro ir kalbėjosi: “Kas tai per žodis: Jis su valdžia ir jėga įsakinėja netyrosioms dvasioms, ir tos pasitraukia?!” 
\par 37 Garsas apie Jį plito visose aplinkinėse vietovėse. 
\par 38 Iš sinagogos Jėzus atėjo į Simono namus. Simono uošvė labai karščiavo, ir jie prašė jai pagalbos. 
\par 39 Atsistojęs prie jos galvūgalio, Jis sudraudė karštligę, ir toji dingo. Ji iškart atsikėlė ir jiems patarnavo. 
\par 40 Saulei leidžiantis, visi, kurie turėjo ligonių, įvairiomis ligomis sergančių, vedė juos prie Jėzaus, o Jis gydė, ant kiekvieno uždėdamas rankas. 
\par 41 Iš daugelio išeidavo demonai, šaukdami: “Tu Dievo Sūnus!” Bet Jis drausdavo juos, neleisdamas jiems kalbėti, nes jie žinojo Jį esant Kristų. 
\par 42 Dienai išaušus, Jis išėjęs pasitraukė į negyvenamą vietą. Minios Jo ieškojo ir, atėjusios pas Jį, bandė sulaikyti Jį, kad jų nepaliktų. 
\par 43 Jis jiems tarė: “Ir kitiems miestams turiu skelbti Gerąją naujieną apie Dievo karalystę, nes tam esu siųstas”. 
\par 44 Ir Jis pamokslavo Galilėjos sinagogose.



\chapter{5}


\par 1 Kartą, kai minios veržėsi prie Jo klausytis Dievo žodžio, Jis stovėjo prie Genezareto ežero 
\par 2 ir pamatė dvi valtis, stovinčias prie ežero kranto. Žvejai buvo išlipę iš jų ir plovė tinklus. 
\par 3 Įlipęs į vieną valtį, kuri buvo Simono, Jis paprašė truputį atsistumti nuo kranto ir atsisėdęs mokė minias iš valties. 
\par 4 Baigęs kalbėti, Jis tarė Simonui: “Irkis į gilumą ir išmeskite tinklus valksmui”. 
\par 5 Simonas Jam atsakė: “Mokytojau, visą naktį vargę, mes nieko nesugavome, bet dėl Tavo žodžio užmesiu tinklą”. 
\par 6 Tai padarę, jie užgriebė didelę daugybę žuvų, kad net jų tinklas pradėjo trūkinėti. 
\par 7 Jie pamojo savo bendrininkams, buvusiems kitoje valtyje, atplaukti į pagalbą. Tie atplaukė ir pripildė žuvų abi valtis, kad jos beveik skendo. 
\par 8 Tai matydamas, Simonas Petras puolė Jėzui po kojų, sakydamas: “Pasitrauk nuo manęs, Viešpatie, nes aš­nusidėjėlis!” 
\par 9 Mat jį ir visus draugus apėmė nuostaba dėl to valksmo žuvų, kurias jie sugavo; 
\par 10 taip pat Zebediejaus sūnus Jokūbą ir Joną, kurie buvo Petro bendrai. O Jėzus tarė Simonui: “Nebijok! Nuo šiol žmones žvejosi”. 
\par 11 Išvilkę į krantą valtis, jie viską paliko ir nusekė paskui Jį. 
\par 12 Jam esant viename mieste, atėjo vyras, visas raupsuotas. Pamatęs Jėzų, jis parpuolė ant žemės ir maldavo: “Viešpatie, jei nori, gali mane apvalyti!” 
\par 13 Jėzus, ištiesęs ranką, palietė raupsuotąjį ir tarė: “Noriu, būk švarus!” Ir iškart raupsai pranyko. 
\par 14 Jėzus jam liepė niekam šito nepasakoti: “Tik nueik, pasirodyk kunigui ir atiduok auką už pagijimą, kaip Mozės įsakyta, jiems paliudyti”. 
\par 15 Tačiau garsas apie Jį sklido vis plačiau, ir didelės minios rinkdavosi Jo pasiklausyti bei pagyti nuo savo ligų. 
\par 16 O Jis pasitraukdavo į dykvietes melstis. 
\par 17 Vieną dieną, kai Jis mokė žmones, ten sėdėjo fariziejų bei Įstatymo mokytojų, susirinkusių iš visų Galilėjos ir Judėjos kaimų bei Jeruzalės. Ir ten buvo Viešpaties jėga, kad gydytų žmones. 
\par 18 Štai vyrai neštuvais atnešė paralyžiuotą žmogų. Jie bandė jį įnešti į vidų ir paguldyti priešais Jėzų. 
\par 19 Nerasdami pro kur įnešti dėl žmonių gausybės, jie užlipo ant stogo ir, praardę jį, nuleido ligonį kartu su neštuvais žemyn ties Jėzumi. 
\par 20 Matydamas jų tikėjimą, Jis tarė: “Žmogau, tavo nuodėmės tau atleistos!” 
\par 21 Tada Rašto žinovai ir fariziejai pradėjo svarstyti: “Kas per vienas šitas piktžodžiaujantis? Kas gali atleisti nuodėmes, jei ne vienas Dievas?” 
\par 22 Jėzus, supratęs jų mintis, prabilo: “Kodėl taip svarstote savo širdyse? 
\par 23 Kas lengviau­ar pasakyti: ‘Tavo nuodėmės tau atleistos’, ar pasakyti: ‘Kelkis ir vaikščiok’? 
\par 24 O kad žinotumėte Žmogaus Sūnų turint žemėje valdžią atleisti nuodėmes,­čia Jis tarė paralyžiuotajam,­sakau tau: kelkis, imk savo gultą ir eik namo!” 
\par 25 Tas tuojau atsikėlė jų akivaizdoje, pasiėmė neštuvus ir, šlovindamas Dievą, nuėjo namo. 
\par 26 Visi didžiai stebėjosi ir šlovino Dievą. Apimti baimės, jie kalbėjo: “Šiandien matėme stebinančių dalykų!” 
\par 27 Po to išėjęs Jis pastebėjo muitininką, vardu Levį, sėdintį muitinėje, ir jam tarė: “Sek paskui mane!” 
\par 28 Tas atsikėlė ir, viską palikęs, nusekė paskui Jį. 
\par 29 Levis savo namuose iškėlė Jam didelį pokylį. Prie stalo susirinko gausus būrys muitininkų ir kitų svečių. 
\par 30 Rašto žinovai ir fariziejai murmėjo ir prikaišiojo Jėzaus mokiniams: “Kodėl jūs valgote ir geriate su muitininkais ir nusidėjėliais?” 
\par 31 Jėzus jiems atsakė: “Ne sveikiesiems reikia gydytojo, bet ligoniams. 
\par 32 Aš atėjau šaukti ne teisiųjų, bet nusidėjėlių atgailai”. 
\par 33 Tada jie sakė Jam: “Kodėl Jono mokiniai dažnai pasninkauja ir meldžiasi, taip pat ir fariziejų mokiniai, o Tavieji valgo ir geria?” 
\par 34 Jėzus jiems atsakė: “Argi galite versti pasninkauti vestuvininkus, kol su jais yra jaunikis? 
\par 35 Ateis dienos, kai jaunikis bus iš jų atimtas, ir tada, tomis dienomis, jie pasninkaus”. 
\par 36 Jėzus dar pasakė jiems palyginimą: “Niekas neplėšia lopo iš naujo drabužio ir nesiuva jo ant seno. Nes ir naująjį jis suplėšytų, ir senajam netiktų lopas iš naujojo. 
\par 37 Taip pat niekas nepila jauno vyno į senus vynmaišius. Jaunas vynas suplėšytų vynmaišius, pats ištekėtų, ir vynmaišiai niekais nueitų. 
\par 38 Jauną vyną reikia pilti į naujus vynmaišius, ir tada abeji išsilaiko. 
\par 39 Ir niekas, gėręs seno vyno, nenori jauno; jis sako: ‘Senasis geresnis!’ ”



\chapter{6}


\par 1 Vieną sabatą, Jam einant per javų lauką, mokiniai skabė varpas ir, ištrynę tarp delnų, valgė. 
\par 2 Kai kurie iš fariziejų jiems tarė: “Kodėl darote, kas per sabatą draudžiama?!” 
\par 3 Jėzus jiems atsakė: “Nejaugi neskaitėte, ką darė Dovydas, kai buvo alkanas pats ir jo palydovai? 
\par 4 Kaip jis įėjo į Dievo namus, ėmė padėtinės duonos, valgė ir davė savo palydai, nors jos niekam neleistina valgyti, tik kunigams!” 
\par 5 Ir Jis pridūrė: “Žmogaus Sūnus yra ir sabato Viešpats”. 
\par 6 Kitą sabatą Jis nuėjo į sinagogą ir mokė. Ten buvo žmogus, kurio dešinė ranka buvo padžiūvusi. 
\par 7 Rašto žinovai ir fariziejai stebėjo, ar Jis gydys per sabatą, kad rastų kuo Jį apkaltinti. 
\par 8 Bet Jis, žinodamas jų mintis, tarė vyrui su padžiūvusia ranka: “Kelkis ir stok į vidurį”. Tas atsistojo. 
\par 9 Tada Jėzus jiems pasakė: “Aš klausiu jūsų: ar per sabatą leistina gera daryti ar bloga? Gelbėti gyvybę ar pražudyti?” 
\par 10 Ir, apžvelgęs visus aplinkui, tarė tam žmogui: “Ištiesk savo ranką!” Tas taip padarė, ir jo ranka tapo sveika kaip ir kita. 
\par 11 O jie baisiai įniršo ir tarėsi, ką galėtų Jėzui padaryti. 
\par 12 Tomis dienomis Jis užkopė į kalną melstis ir ten praleido visą naktį, melsdamasis Dievui. 
\par 13 Išaušus rytui, Jis pasišaukė savo mokinius ir iš jų išsirinko dvylika, kuriuos ir pavadino apaštalais: 
\par 14 Simoną, kurį praminė Petru, jo brolį Andriejų, Jokūbą, Joną, Pilypą ir Baltramiejų, 
\par 15 Matą ir Tomą, Alfiejaus sūnų Jokūbą ir Simoną, vadinamą Uoliuoju, 
\par 16 Jokūbo sūnų Judą ir Judą Iskarijotą, kuris tapo išdaviku. 
\par 17 Nusileidęs su jais žemyn, sustojo lygumoje. Ten buvo daugybė Jo mokinių ir didelės minios žmonių iš visos Judėjos ir Jeruzalės, iš Tyro ir Sidono pajūrio. Jie susirinko Jo pasiklausyti ir pagyti nuo savo ligų. 
\par 18 Buvo pagydomi kankinamieji netyrųjų dvasių. 
\par 19 Visa minia stengėsi Jį paliesti, nes iš Jo ėjo jėga ir visus gydė. 
\par 20 Tada, pakėlęs akis į savo mokinius, Jis prabilo: “Palaiminti jūs, vargšai, nes jūsų yra Dievo karalystė. 
\par 21 Palaiminti, kurie dabar alkstate, nes būsite pasotinti. Palaiminti, kurie dabar verkiate, nes juoksitės. 
\par 22 Palaiminti jūs, kai žmonės jūsų nekęs, kai jus atskirs, šmeiš ir atmes jūsų vardą kaip blogą dėl Žmogaus Sūnaus. 
\par 23 Džiaukitės tą dieną ir linksminkitės, nes štai jūsų atlygis didelis danguje. Taip jų protėviai darė pranašams. 
\par 24 Bet vargas jums, turtingieji, nes jūs jau atsiėmėte savo paguodą! 
\par 25 Vargas jums, kurie dabar sotūs, nes būsite alkani! Vargas jums, kurie dabar juokiatės, nes jūs liūdėsite ir verksite! 
\par 26 Vargas jums, kai visi žmonės jus giria, nes ir jų protėviai taip gyrė netikrus pranašus!” 
\par 27 “Bet jums, kurie klausotės, sakau: mylėkite savo priešus, darykite gera tiems, kurie jūsų nekenčia. 
\par 28 Laiminkite tuos, kurie jus keikia, ir melskitės už savo skriaudėjus. 
\par 29 Kas trenkia tau per vieną skruostą, atsuk ir antrąjį; kas atima iš tavęs apsiaustą, atiduok jam ir tuniką. 
\par 30 Duok kiekvienam, kuris prašo, ir nereikalauk atgal iš to, kuris tavo atėmė. 
\par 31 Kaip norite, kad jums žmonės darytų, taip ir jūs darykite jiems. 
\par 32 Jei mylite tuos, kurie jus myli, tai koks čia jūsų nuopelnas? Juk ir nusidėjėliai myli juos mylinčius. 
\par 33 Jei darote gera tiems, kurie jums gera daro, tai koks jūsų nuopelnas? Juk ir nusidėjėliai taip daro. 
\par 34 Jei skolinate tik tiems, iš kurių tikitės atgausią, koks jūsų nuopelnas? Juk ir nusidėjėliai skolina nusidėjėliams, kad atgautų paskolą. 
\par 35 Bet mylėkite savo priešus, darykite gera ir skolinkite, nieko nesitikėdami. Tuomet jūsų atlygis bus didelis, ir jūs būsite Aukščiausiojo vaikai; nes Jis maloningas ir nedėkingiesiems, ir piktiesiems. 
\par 36 Būkite gailestingi, kaip ir jūsų Tėvas yra gailestingas. 
\par 37 Neteiskite ir nebūsite teisiami; nesmerkite ir nebūsite pasmerkti; atleiskite, ir jums bus atleista. 
\par 38 Duokite, ir jums bus duota; saiką gerą, prikimštą, sukratytą ir su kaupu duos jums į užantį. Kokiu saiku seikite, tokiu jums bus atseikėta”. 
\par 39 Jis pasakė jiems palyginimą: “Ar gali aklas vesti aklą? Argi ne abu įkris į duobę?! 
\par 40 Mokinys nėra viršesnis už savo mokytoją: kiekvienas mokinys, jei gerai išlavintas, bus kaip jo mokytojas. 
\par 41 Kodėl matai krislą brolio akyje, o nepastebi rąsto savojoje? 
\par 42 Ir kaip gali sakyti broliui: ‘Broli, leisk, išimsiu krislą iš tavo akies’,­pats nematydamas savo akyje rąsto?! Veidmainy, pirmiau išritink rąstą iš savo akies, o tada matysi, kaip iš brolio akies išimti krislelį. 
\par 43 Nėra gero medžio, kuris duotų blogus vaisius, ir nėra blogo, kuris duotų gerus vaisius. 
\par 44 Kiekvienas medis pažįstamas iš jo vaisių. Nuo erškėčių niekas nerenka figų, o nuo gervuogių krūmo neskina vynuogių. 
\par 45 Geras žmogus iš gero savo širdies lobyno iškelia gera, o blogas iš blogo savo širdies lobyno iškelia bloga. Jo lūpos kalba tai, ko pertekusi širdis”. 
\par 46 “Kodėl vadinate mane: ‘Viešpatie, Viešpatie’, o nedarote, ką sakau? 
\par 47 Kiekvienas, kuris ateina pas mane, klausosi mano žodžių ir juos vykdo,­Aš parodysiu jums, į ką jis panašus. 
\par 48 Jis panašus į namą statantį žmogų, kuris giliai iškasė žemę ir padėjo pamatus ant uolos. Užėjus potvyniui, srovė atsimušė į tą namą, bet neįstengė jo pajudinti, nes buvo pastatytas ant uolos. 
\par 49 O kas klausosi, bet nevykdo, panašus į žmogų, pasistačiusį namą be pamato, ant žemės. Vos tik srovė į jį atsimušė, jis kaipmat sugriuvo, ir to namo griuvimas buvo smarkus”.



\chapter{7}


\par 1 Baigęs visus savo pamokymus klausytojams, Jėzus sugrįžo į Kafarnaumą. 
\par 2 Vieno šimtininko branginamas tarnas sirgo ir buvo arti mirties. 
\par 3 Išgirdęs apie Jėzų, šimtininkas pasiuntė pas Jį kelis žydų vyresniuosius, prašydamas Jį ateiti ir išgydyti tarną. 
\par 4 Atėję pas Jėzų, jie karštai prašė, sakydami: “Jis vertas, kad jam tai padarytum, 
\par 5 nes jis myli mūsų tautą ir mums yra pastatęs sinagogą”. 
\par 6 Jėzus nuėjo su jais. Kai Jis buvo netoli namų, šimtininkas atsiuntė savo draugus, kad Jam pasakytų: “Viešpatie, nesivargink! Aš nesu vertas, kad užeitum po mano stogu. 
\par 7 Taip pat savęs nelaikau vertu ateiti pas Tave. Bet tark žodį, ir mano tarnas pasveiks. 
\par 8 Juk ir aš, būdamas valdinys, turiu sau pavaldžių kareivių. Taigi sakau kuriam iš jų: ‘Eik’, ir jis eina; sakau kitam: ‘Ateik čia’, ir jis ateina; sakau tarnui: ‘Padaryk tai’, ir jis daro”. 
\par 9 Tai girdėdamas, Jėzus stebėjosi juo ir, atsigręžęs į Jį lydinčią minią, tarė: “Sakau jums—net Izraelyje neradau tokio didelio tikėjimo!” 
\par 10 Sugrįžę į namus, pasiųstieji rado tarną pasveikusį. 
\par 11 Po to Jėzus ėjo į miestą, vardu Nainą. Kartu su Juo keliavo daugelis Jo mokinių ir gausi minia. 
\par 12 Kai Jis prisiartino prie miesto vartų, štai nešė numirėlį­vienintelį motinos sūnų, o ji buvo našlė. Kartu su ja ėjo didelė miesto minia. 
\par 13 Pamačiusiam motiną Viešpačiui pagailo jos, ir Jis tarė: “Neverk!” 
\par 14 Priėjęs Jis palietė neštuvus. Nešėjai sustojo, ir Jis pasakė: “Jaunuoli, sakau tau: kelkis!” 
\par 15 Numirėlis atsisėdo ir pradėjo kalbėti. Jėzus atidavė jį motinai. 
\par 16 Visus apėmė baimė, ir jie šlovino Dievą, sakydami: “Didis pranašas iškilo tarp mūsų”, ir: “Dievas aplankė savo tautą”. 
\par 17 Ta žinia apie Jį pasklido po visą Judėją ir visą apylinkę. 
\par 18 Visa tai pranešė Jonui jo mokiniai. 
\par 19 Tada Jonas, pasišaukęs du savo mokinius, siuntė juos pas Jėzų paklausti: “Ar Tu esi Tas, kuris turi ateiti, ar mums laukti kito?” 
\par 20 Atėję pas Jį, tie vyrai tarė: “Jonas Krikštytojas mus siuntė pas Tave, klausdamas: ‘Ar Tu esi Tas, kuris turi ateiti, ar mums laukti kito?’ ” 
\par 21 Kaip tik tuo metu Jis pagydė daugelį nuo ligų bei negalių ir nuo piktųjų dvasių, daugeliui aklųjų dovanojo regėjimą. 
\par 22 Tad atsakydamas, Jis tarė jiems: “Nuėję praneškite Jonui, ką matėte ir girdėjote: aklieji regi, luošieji vaikščioja, raupsuotieji apvalomi, kurtieji girdi, mirusieji prikeliami, vargšams skelbiama Geroji naujiena. 
\par 23 Ir palaimintas, kas nepasipiktins manimi”. 
\par 24 Jono pasiuntiniams nuėjus, Jis pradėjo kalbėti minioms apie Joną: “Ko išėjote į dykumą pažiūrėti? Ar vėjo linguojamos nendrės? 
\par 25 Ko išėjote pamatyti? Ar švelniais drabužiais vilkinčio žmogaus? Antai tie, kurie ištaigingai vilki ir prabangiai gyvena, yra karaliaus rūmuose. 
\par 26 Tai ko gi išėjote pamatyti? Ar pranašo? Taip, sakau jums, ir daug daugiau negu pranašo. 
\par 27 Jis yra tas, apie kurį parašyta: ‘Štai Aš siunčiu pirma Tavęs savo pasiuntinį, kuris nuties Tau kelią’. 
\par 28 Sakau jums: tarp gimusių iš moters nebuvo didesnio pranašo už Joną Krikštytoją, bet ir mažiausias Dievo karalystėje didesnis už jį. 
\par 29 Jį išgirdusi, visa tauta, taip pat ir muitininkai, pripažino Dievo teisingumą, nes leidosi krikštijami Jono krikštu. 
\par 30 Tik fariziejai ir Įstatymo mokytojai atstūmė, ką Dievas jiems buvo sumanęs, nesiduodami Jono krikštijami”. 
\par 31 “Su kuo galėčiau palyginti šios kartos žmones? Į ką jie panašūs? 
\par 32 Jie panašūs į vaikus, kurie, susėdę turgavietėje, vieni kitiems šaukia: ‘Mes jums grojome, o jūs nešokote; mes giedojome raudas, o jūs neverkėte’. 
\par 33 Buvo atėjęs Jonas Krikštytojas. Jis nevalgė duonos ir negėrė vyno, tai jūs sakėte: ‘Jis demono apsėstas’. 
\par 34 Atėjo Žmogaus Sūnus; Jis valgo ir geria, tai jūs vėl sakote: ‘Štai rijūnas ir vyno gėrėjas, muitininkų ir nusidėjėlių bičiulis’. 
\par 35 Bet išmintį pateisina visi jos vaikai”. 
\par 36 Vienas fariziejus pakvietė Jėzų kartu valgyti. Atėjęs į fariziejaus namus, Jis sėdo prie stalo. 
\par 37 Ir štai moteris, kuri buvo žinoma mieste nusidėjėlė, sužinojusi, kad Jis fariziejaus namuose, atsinešė alebastrinį indą kvapaus tepalo 
\par 38 ir, verkdama priėjusi iš užpakalio prie Jo kojų, ėmė laistyti jas ašaromis, šluostyti savo galvos plaukais, bučiavo Jo kojas ir tepė jas tepalu. 
\par 39 Tai matydamas, fariziejus, kuris Jėzų pasikvietė, samprotavo: “Jeigu šitas būtų pranašas, Jis žinotų, kas tokia ši moteris, kuri Jį liečia, nes ji­nusidėjėlė!” 
\par 40 O Jėzus tarė: “Simonai, turiu tau ką pasakyti”. Tas atsiliepė: “Sakyk, Mokytojau!” 
\par 41 “Skolintojas turėjo du skolininkus. Vienas buvo skolingas penkis šimtus denarų, o kitas­penkiasdešimt. 
\par 42 Jiems neturint iš ko atiduoti, jis dovanojo abiem. Kuris labiau jį mylės?” 
\par 43 Simonas atsakė: “Manau, jog tas, kuriam daugiau dovanota”. Jėzus tarė: “Teisingai nusprendei”. 
\par 44 Ir, atsisukęs į moterį, Jis tarė Simonui: “Matai šitą moterį? Aš atėjau į tavo namus, tu nedavei man vandens kojoms nusimazgoti, o ji laistė jas ašaromis ir šluostė savo plaukais. 
\par 45 Tu manęs nepabučiavai, o ji, vos man atėjus, nesiliauja bučiavusi mano kojų. 
\par 46 Tu aliejumi man galvos nepatepei, o ji tepalu patepė man kojas. 
\par 47 Todėl sakau tau: jos gausios nuodėmės jai atleidžiamos, nes ji labai pamilo. Kam mažai atleista, tas menkai myli”. 
\par 48 Jis tarė jai: “Tavo nuodėmės atleistos”. 
\par 49 Esantieji kartu su Juo už stalo ėmė svarstyti: “Kas gi Jis toks, kad net ir nuodėmes atleidžia?!” 
\par 50 O Jis tarė moteriai: “Tavo tikėjimas išgelbėjo tave. Eik rami”.



\chapter{8}


\par 1 Po to Jėzus keliavo per miestus ir kaimus, pamokslaudamas ir skelbdamas Dievo karalystės Gerąją naujieną. Su Juo buvo dvylika 
\par 2 ir kelios moterys, išgydytos nuo piktųjų dvasių bei ligų: Marija, vadinama Magdaliete, iš kurios buvo išėję septyni demonai, 
\par 3 Erodo prievaizdo Chūzo žmona Joana, Zuzana ir daug kitų, kurios jiems tarnavo savo turtu. 
\par 4 Susirinkus gausiai miniai ir žmonėms iš visų miestų skubant pas Jį, Jis kalbėjo palyginimu: 
\par 5 “Sėjėjas išėjo sėti savo sėklos. Jam besėjant, vieni grūdai nukrito palei kelią, buvo sumindžioti, ir padangių paukščiai juos sulesė. 
\par 6 Kiti nukrito ant uolos, ir išdygę nudžiūvo, nes trūko drėgmės. 
\par 7 Kiti nukrito tarp erškėčių, ir tie, kartu išaugę, nusmelkė juos. 
\par 8 O dar kiti nukrito į gerą žemę ir išaugę davė šimteriopą derlių”. Tai papasakojęs, Jis sušuko: “Kas turi ausis klausyti­teklauso!” 
\par 9 Jo mokiniai paklausė: “Ką reiškia šis palyginimas?” 
\par 10 Jis atsakė: “Jums duota pažinti Dievo karalystės paslaptis, o kitiems jos skelbiamos palyginimais, kad ‘žiūrėdami nematytų ir girdėdami nesuprastų’ ”. 
\par 11 “Palyginimas štai ką reiškia: sėkla yra Dievo žodis. 
\par 12 Palei kelią­tai tie, kurie klausosi, paskui ateina velnias ir išrauna žodį iš jų širdies, kad jie netikėtų ir nebūtų išgelbėti. 
\par 13 Ant uolos­tai tie, kurie, išgirdę žodį, su džiaugsmu jį priima, bet neturi šaknų: jie kurį laiką tiki, o gundymo metu atkrinta. 
\par 14 Sėkla, kritusi tarp erškėčių,­tai tie, kurie išgirdo, bet, toliau eidami, buvo nusmelkti rūpesčių, turtų bei gyvenimo malonumų ir neduoda subrendusio vaisiaus. 
\par 15 Nukritusi į gerą žemę sėkla­tai tie, kurie klauso žodžio, išsaugo jį tyroje ir geroje širdyje ir duoda vaisių kantrumu”. 
\par 16 “Nė vienas, uždegęs žiburį, neapvožia jo indu ir nekiša po lova, bet stato į žibintuvą, kad įeinantys matytų šviesą. 
\par 17 Nėra nieko paslėpta, kas nebūtų atskleista, nieko slapta, kas nepasidarytų žinoma ir neišeitų aikštėn. 
\par 18 Tad žiūrėkite, kaip klausotės. Kas turi, tam bus duota, o iš neturinčio bus atimta ir tai, ką jis tariasi turįs”. 
\par 19 Pas Jėzų atėjo motina ir Jo broliai, bet negalėjo prieiti prie Jo per minią. 
\par 20 Jam pranešė: “Tavo motina ir broliai stovi lauke ir nori su Tavimi pasimatyti”. 
\par 21 O Jis atsakė: “Mano motina ir mano broliai­tai tie, kurie klausosi Dievo žodžio ir jį vykdo”. 
\par 22 Vieną dieną Jėzus su mokiniais įlipo į valtį ir pasakė: “Irkimės anapus ežero!” Ir jie išplaukė. 
\par 23 Jiems beplaukiant, Jėzus užmigo. Ežere kilo audra. Bangos pradėjo semti valtį, ir jie atsidūrė pavojuje. 
\par 24 Tuomet pripuolę jie ėmė žadinti Jį, šaukdami: “Mokytojau, Mokytojau, mes žūvame!” Atsikėlęs Jis sudraudė vėją ir bangas. Jos nurimo, ir stojo tyla. 
\par 25 O Jėzus paklausė juos: “Kur jūsų tikėjimas?” Jie išsigandę ir nustebę kalbėjosi tarpusavy: “Kas Jis toks, kad įsakinėja net vėjams ir vandeniui, ir tie Jo klauso?!” 
\par 26 Jie atplaukė į gadariečių kraštą, kuris yra priešingame Galilėjai krante. 
\par 27 Kai tik Jėzus išlipo į krantą, Jį pasitiko iš miesto atbėgęs vyras, kuris jau ilgą laiką turėjo demonų. Jis nedėvėjo drabužių ir negyveno namuose, bet laikėsi kapinėse. 
\par 28 Pamatęs Jėzų, jis suriko, parpuolė prieš Jį ir ėmė garsiai šaukti: “Ko nori iš manęs, Jėzau, aukščiausiojo Dievo Sūnau?! Maldauju, nekankink manęs!” 
\par 29 Mat Jėzus buvo įsakęs netyrajai dvasiai išeiti iš to žmogaus. Dvasia dažnai jį sugriebdavo ir, nors jį saugodavo surakintą grandinėmis, supančiotomis kojomis, jis sutraukydavo pančius ir demonas jį varydavo į dykumą. 
\par 30 Jėzus jo paklausė: “Kuo tu vardu?” Šis atsakė: “Legionas”. Mat į jį buvo įėję daug demonų. 
\par 31 Jie maldavo Jėzų, kad Jis neįsakytų jiems eiti į bedugnę. 
\par 32 Tenai kalne ganėsi didelė banda kiaulių. Demonai prašė leisti sueiti į jas. Jėzus leido. 
\par 33 Tada demonai, išėję iš žmogaus, apniko kiaules. Banda tuojau metėsi nuo skardžio į ežerą ir prigėrė. 
\par 34 Pamatę, kas nutiko, piemenys pabėgo ir pranešė apie tai mieste bei kaimuose. 
\par 35 Žmonės išėjo pažiūrėti, kas atsitiko, ir, atėję prie Jėzaus, rado žmogų, iš kurio buvo išėję demonai, sėdintį prie Jėzaus kojų apsirengusį ir sveiko proto. Ir jie išsigando. 
\par 36 Tie, kurie matė, papasakojo, kaip buvo išgydytas demonų apsėstasis. 
\par 37 Tada visa gadariečių krašto minia ėmė prašyti, kad Jėzus pasitrauktų nuo jų, nes juos buvo apėmusi didelė baimė. Jis įsėdo į valtį ir grįžo atgal. 
\par 38 Vyras, iš kurio buvo išėję demonai, prašėsi paliekamas pas Jėzų. Tačiau Jis atleido jį ir paliepė: 
\par 39 “Grįžk namo ir papasakok, kokių didžių dalykų tau padarė Dievas”. Tuomet jis nuėjo ir skelbė po visą miestą, ką Jėzus jam buvo padaręs. 
\par 40 Grįžtantį Jėzų pasitiko minia, nes visi Jo laukė. 
\par 41 Ir štai atėjo vyras, vardu Jayras, sinagogos vyresnysis. Jis puolė Jėzui po kojų, maldaudamas ateiti į jo namus. 
\par 42 Mat jo vienintelė, dvylikametė dukrelė buvo bemirštanti. Jėzus ėjo iš visų pusių spaudžiamas minios. 
\par 43 Viena moteris, dvylika metų serganti kraujoplūdžiu ir išleidusi gydytojams visus savo išteklius,­ bet jos nė vienas negalėjo pagydyti,­ 
\par 44 prisiartino iš užpakalio ir prisilietė Jo drabužio apvado. Ir bematant kraujas liovėsi plūdęs. 
\par 45 Jėzus paklausė: “Kas mane palietė?” Visiems besiginant, Petras ir šalia jo esantys tarė: “Mokytojau, minia Tave spaudžia ir stumia, o Tu klausi: ‘Kas mane palietė?’ ” 
\par 46 Bet Jėzus atsakė: “Mane kažkas palietė, nes Aš pajutau, kad iš manęs išėjo jėga”. 
\par 47 Moteris, matydama, kad neliko nepastebėta, drebėdama prisiartino, parpuolė Jam po kojų ir visų žmonių akivaizdoje papasakojo, kodėl prisilietė ir kaip tuojau pasveiko. 
\par 48 Tuomet Jėzus jai tarė: “Dukra, tavo tikėjimas išgydė tave. Eik rami!” 
\par 49 Jam tebekalbant, atėjo kažkas iš sinagogos vyresniojo namų ir tam pranešė: “Tavo duktė numirė. Nebevargink Mokytojo”. 
\par 50 Tai išgirdęs, Jėzus tarė: “Nebijok, tik tikėk, ir ji bus išgelbėta”. 
\par 51 Atėjęs į namus, Jis neleido su savimi įeiti niekam, tik Petrui, Jonui, Jokūbui ir mergaitės tėvui bei motinai. 
\par 52 Visi verkė ir apraudojo ją. Bet Jis tarė: “Neverkite! Mergaitė nemirė, o tik miega”. 
\par 53 Jie šaipėsi iš Jo, žinodami, kad ji mirusi. 
\par 54 Bet Jėzus juos visus išvarė, ir, paėmęs ją už rankos, sušuko: “Mergaite, kelkis!” 
\par 55 Jos dvasia sugrįžo, ir ji tuojau atsikėlė. Jėzus liepė duoti jai valgyti. 
\par 56 Jos tėvai be galo stebėjosi, o Jis įsakė niekam nesakyti, kas buvo įvykę.



\chapter{9}


\par 1 Sukvietęs dvylika savo mokinių, Jėzus davė jiems jėgą ir valdžią prieš visus demonus ir gydyti ligoms. 
\par 2 Po to išsiuntė juos skelbti Dievo karalystės ir gydyti ligonių. 
\par 3 Jis pasakė jiems: “Nieko neimkite kelionei: nei lazdos, nei krepšio, nei duonos, nei pinigų. Neturėkite nė dviejų tunikų. 
\par 4 Į kuriuos tik namus įeisite, tenai pasilikite ir iš ten toliau keliaukite. 
\par 5 O kur žmonės jūsų nepriims, išeidami iš to miesto, nusikratykite nuo kojų dulkes kaip liudijimą prieš juos”. 
\par 6 Išėję jie traukė per aplinkinius kaimus, visur skelbdami Evangeliją bei gydydami. 
\par 7 Tetrarchas Erodas išgirdo apie visus tuos įvykius ir suglumo, nes vieni sakė, kad Jonas prisikėlęs iš numirusių, 
\par 8 kiti­kad pasirodęs Elijas, dar kiti­kad prisikėlęs vienas iš senųjų pranašų. 
\par 9 Erodas sakė: “Jonui aš nukirsdinau galvą; o kas yra šitas, apie kurį girdžiu pasakojant tokius dalykus?!” Ir jis labai norėjo Jėzų pamatyti. 
\par 10 Sugrįžę apaštalai pasakojo Jėzui, ką buvo nuveikę. Pasiėmęs juos, Jis pasitraukė nuošaliai į dykvietę, netoli miesto, vadinamo Betsaida. 
\par 11 Minios, tai sužinojusios, nusekė paskui Jį. Jis priėmė žmones, kalbėjo jiems apie Dievo karalystę ir išgydė tuos, kuriems reikėjo gydymo. 
\par 12 Diena slinko vakarop. Priėję dvylika tarė Jam: “Paleisk žmones, kad jie, nuėję į aplinkinius kaimus bei kiemus, susirastų nakvynę ir maisto. Mes juk esame dykvietėje”. 
\par 13 Bet Jėzus tarė: “Jūs duokite jiems valgyti”. Jie atsakė: “Mes nieko daugiau neturime, tik penkis kepalus duonos ir dvi žuvis. Nebent nueitume ir nupirktume maisto visiems šitiems žmonėms”. 
\par 14 O ten buvo apie penkis tūkstančius vyrų. Jėzus įsakė mokiniams: “Susodinkite juos būriais po penkiasdešimt”. 
\par 15 Jie taip padarė ir visus susodino. 
\par 16 Tada, paėmęs penkis kepalus ir dvi žuvis, Jis pažvelgė į dangų, palaimino, laužė ir davė mokiniams, kad padalintų miniai. 
\par 17 Visi valgė ir pasisotino. Ir dar buvo surinkta dvylika pintinių trupinių. 
\par 18 Kartą, kai Jis nuošaliai meldėsi ir su Juo buvo mokiniai, Jis paklausė juos: “Kuo mane žmonės laiko?” 
\par 19 Jie atsakė: “Vieni­Jonu Krikštytoju, kiti­Eliju, dar kiti sako, prisikėlęs vienas iš senųjų pranašų”. 
\par 20 Tada Jis paklausė: “O jūs kuo mane laikote?” Petras atsakė: “Dievo Pateptuoju”. 
\par 21 Jis griežtai įspėjo juos, įsakė niekam to nepasakoti 
\par 22 ir pasakė: “Žmogaus Sūnus turės daug iškentėti. Jis bus vyresniųjų, aukštųjų kunigų bei Rašto žinovų atmestas, nužudytas ir trečią dieną prisikels”. 
\par 23 O visiems Jis kalbėjo: “Jei kas nori eiti paskui mane, teišsižada pats savęs, teneša kasdien savo kryžių ir teseka manimi. 
\par 24 Kas nori išgelbėti savo gyvybę, tas ją praras, o kas praras savo gyvybę dėl manęs, tas ją išgelbės. 
\par 25 Kokia būtų nauda, jei žmogus laimėtų visą pasaulį, o save pražudytų ar sau pakenktų? 
\par 26 Jei kas gėdysis manęs ir mano žodžių, to gėdysis ir Žmogaus Sūnus, kai ateis su savąja, Tėvo ir šventųjų angelų šlove. 
\par 27 Iš tiesų sakau jums: kai kurie iš čia stovinčiųjų neragaus mirties, kol išvys Dievo karalystę”. 
\par 28 Praslinkus maždaug aštuonioms dienoms po šitų žodžių, Jis pasiėmė Petrą, Joną bei Jokūbą ir užkopė į kalną melstis. 
\par 29 Jam besimeldžiant, Jo veido išvaizda pasikeitė, o drabužiai pasidarė balti ir spindintys. 
\par 30 Ir štai du vyrai kalbėjosi su Juo. Tai buvo Mozė ir Elijas, 
\par 31 kurie, pasirodę šlovėje, kalbėjo apie Jo gyvenimo pabaigą, būsiančią Jeruzalėje. 
\par 32 O Petrą ir jo draugus buvo apėmęs miegas. Pabudę jie pamatė Jo šlovę ir stovinčius šalia Jo du vyrus. 
\par 33 Tiems tolstant, Petras kreipėsi į Jėzų: “Mokytojau, gera mums čia būti! Pastatykime tris palapines: vieną Tau, kitą Mozei ir trečią Elijui”. Jis nežinojo, ką kalbąs. 
\par 34 Jam tai besakant, užėjo debesis ir uždengė juos. Jie nusigando, kai paniro į debesį. 
\par 35 O iš debesies pasigirdo balsas: “Šitas yra mano mylimasis Sūnus, Jo klausykite!” 
\par 36 Balsui nuskambėjus, Jėzus liko vienas. O jie tylėjo ir tomis dienomis niekam nieko nesakė apie tai, ką buvo matę. 
\par 37 Kitą dieną, jiems nusileidus nuo kalno, Jėzų pasitiko didelė minia. 
\par 38 Ir štai vienas vyras iš minios ėmė šaukti: “Mokytojau, meldžiu, pažvelk į mano sūnų, jis mano vienturtis. 
\par 39 Kai dvasia jį pačiumpa, jis staiga pradeda rėkti, o dvasia tąso jį, jog tas net putoja. Ji tik vargais negalais atstoja, smarkiai jį apdaužiusi. 
\par 40 Aš prašiau Tavo mokinius išvaryti ją, bet jie nepajėgė”. 
\par 41 Jėzus atsakydamas tarė: “O netikinti ir iškrypusi karta! Kaip ilgai man reikės su jumis pasilikti ir jus kęsti? Atvesk čia savo sūnų”. 
\par 42 Dar jam besiartinant, demonas parbloškė jį ant žemės ir ėmė tąsyti. Jėzus sudraudė netyrąją dvasią, išgydė berniuką ir atidavė jį tėvui. 
\par 43 Ir visi buvo pritrenkti Dievo didybės. Visiems stebintis tuo, ką Jėzus darė, Jis prabilo į mokinius: 
\par 44 “Gerai įsidėmėkite šituos žodžius: Žmogaus Sūnus bus atiduotas į žmonių rankas”. 
\par 45 Bet jie nesuprato šitų žodžių; tai liko jiems paslėpta ir jie nesuvokė jų. O jie bijojo klausti Jėzų apie tuos žodžius. 
\par 46 Tarp mokinių kilo ginčas, kuris iš jų didžiausias. 
\par 47 Suprasdamas jų širdies mintis, Jėzus pasišaukė vaiką, pasistatė šalia savęs 
\par 48 ir tarė jiems: “Kas priima šį vaiką mano vardu, mane priima, o kas mane priima, priima Tą, kuris mane siuntė. Kas tarp jūsų mažiausias, tas bus didžiausias”. 
\par 49 Tada atsiliepė Jonas: “Mokytojau, mes matėme vieną, Tavo vardu išvarantį demonus. Mes jam draudėme tai daryti, nes jis nevaikščioja kartu su mumis”. 
\par 50 Jėzus atsakė: “Nedrauskite! Kas ne prieš mus, tas už mus!” 
\par 51 Artėjant metui, kai Jėzus turėjo būti paimtas aukštyn, Jis ryžtingai nukreipė savo žingsnius į Jeruzalę. 
\par 52 Jis išsiuntė pirma savęs pasiuntinius. Tie užėjo į vieną samariečių kaimą, kad paruoštų vietą Jam apsistoti. 
\par 53 Bet tie nesutiko Jo priimti, nes Jis keliavo Jeruzalės link. 
\par 54 Tai girdėdami, mokiniai Jokūbas ir Jonas sušuko: “Viešpatie, ar nori, mes liepsime ugniai kristi iš dangaus ir juos sunaikinti, kaip ir Elijas padarė?” 
\par 55 Bet Jis atsigręžęs sudraudė juos: “Nežinote, kokios dvasios esate. 
\par 56 Juk Žmogaus Sūnus atėjo ne pražudyti žmonių gyvybių, bet išgelbėti”. Ir jie pasuko į kitą kaimą. 
\par 57 Jiems einant keliu, vienas žmogus Jam pasakė: “Aš seksiu paskui Tave, kur tik Tu eisi, Viešpatie!” 
\par 58 Jėzus jam atsakė: “Lapės turi urvus, padangių paukščiai­lizdus, o Žmogaus Sūnus neturi kur galvos priglausti”. 
\par 59 Kitam Jis tarė: “Sek paskui mane!” Tas prašė: “Viešpatie, leisk man pirmiau pareiti tėvo palaidoti”. 
\par 60 Jėzus jam atsakė: “Palik mirusiems laidoti savo numirėlius, o tu eik ir skelbk Dievo karalystę!” 
\par 61 Dar vienas tarė: “Aš seksiu paskui Tave, Viešpatie, bet leisk man pirmiau atsisveikinti su savo namiškiais”. 
\par 62 Jėzus tam atsakė: “Nė vienas, kuris uždeda ranką ant arklo ir žvalgosi atgal, netinka Dievo karalystei”.



\chapter{10}


\par 1 Po to Viešpats paskyrė dar kitus septyniasdešimt mokinių ir išsiuntė juos po du, kad eitų pirma Jo į visus miestus bei vietoves, kur Jis pats ketino vykti. 
\par 2 Jis sakė jiems: “Pjūtis tikrai didelė, o darbininkų maža. Todėl melskite pjūties Viešpatį, kad atsiųstų darbininkų į savo pjūtį. 
\par 3 Eikite! Štai Aš siunčiu jus lyg avinėlius tarp vilkų. 
\par 4 Nesineškite nei piniginės, nei krepšio, nei sandalų ir nieko kelyje nesveikinkite. 
\par 5 Į kuriuos tik namus užeisite, pirmiausia tarkite: ‘Ramybė šiems namams!’ 
\par 6 Ir jei ten bus ramybės sūnus, jūsų ramybė nužengs ant jo, o jei ne,­sugrįš pas jus. 
\par 7 Pasilikite tuose namuose, valgykite ir gerkite, kas duodama, nes darbininkas vertas savo užmokesčio. Nesikilnokite iš namų į namus. 
\par 8 Kai nueisite į kurį nors miestą ir jus priims, valgykite, kas bus jums padėta. 
\par 9 Gydykite jame esančius ligonius ir sakykite jiems: ‘Jums prisiartino Dievo karalystė!’ 
\par 10 O kai užeisite į tokį miestą, kur jūsų nepriims, išėję į gatves, sakykite: 
\par 11 ‘Mes nusikratome prieš jus net jūsų miesto dulkes, prilipusias prie mūsų kojų. Tačiau vis tiek žinokite: Dievo karalystė prisiartino prie jūsų!’ 
\par 12 Sakau jums: aną dieną Sodomai bus lengviau negu tam miestui. 
\par 13 Vargas tau, Chorazine! Vargas tau, Betsaida! Jeigu Tyre ir Sidone būtų padaryta stebuklų, kokie padaryti pas jus, jie seniai būtų atgailavę, sėdėdami su ašutine ir pelenuose. 
\par 14 Todėl Tyrui ir Sidonui teisme bus lengviau negu jums. 
\par 15 Ir tu, Kafarnaume, išaukštintas iki dangaus, nugarmėsi iki pragaro! 
\par 16 Kas jūsų klauso, manęs klauso. Kas jus niekina, mane niekina. O kas niekina mane, niekina Tą, kuris mane siuntė”. 
\par 17 Septyniasdešimt sugrįžo, su džiaugsmu kalbėdami: “Viešpatie, mums paklūsta net demonai dėl Tavo vardo”. 
\par 18 O Jis jiems sakė: “Mačiau šėtoną, kaip žaibą krintantį iš dangaus. 
\par 19 Štai Aš duodu jums valdžią mindžioti gyvates bei skorpionus, aukštesnę už visą priešo jėgą, ir niekas jokiais būdais jums nepakenks. 
\par 20 Bet jūs džiaukitės ne tuo, kad dvasios jums pavaldžios; džiaukitės, kad jūsų vardai įrašyti danguje”. 
\par 21 Tą valandą Jėzus pradžiugo Dvasia ir tarė: “Aš šlovinu Tave, Tėve, dangaus ir žemės Viešpatie, kad paslėpei tai nuo išmintingųjų ir gudriųjų, o apreiškei mažutėliams. Taip, Tėve, nes Tau taip patiko. 
\par 22 Viskas man yra mano Tėvo atiduota. Ir niekas nežino, kas yra Sūnus, tik Tėvas, nei kas yra Tėvas, tik Sūnus ir tas, kam Sūnus nori apreikšti”. 
\par 23 Atsigręžęs vien tik į mokinius, Jis tarė: “Palaimintos akys, kurios regi, ką jūs regite. 
\par 24 Sakau jums: daugel pranašų ir karalių troško pamatyti, ką jūs matote, bet nepamatė, ir išgirsti, ką jūs girdite, bet neišgirdo”. 
\par 25 Štai atsistojo vienas Įstatymo mokytojas ir, mėgindamas Jį, paklausė: “Mokytojau, ką turiu daryti, kad paveldėčiau amžinąjį gyvenimą?” 
\par 26 Jis tarė: “O kas parašyta Įstatyme? Kaip skaitai?” 
\par 27 Tas atsakė: “ ‘Mylėk Viešpatį, savo Dievą, visa savo širdimi, visa savo siela, visomis savo jėgomis ir visu savo protu, ir savo artimą kaip save patį’ ”. 
\par 28 Jėzus jam tarė: “Gerai atsakei. Tai daryk, ir gyvensi”. 
\par 29 Norėdamas pateisinti save, anas paklausė Jėzų: “O kas gi mano artimas?” 
\par 30 Jėzus atsakydamas tarė: “Vienas žmogus keliavo iš Jeruzalės į Jerichą ir pakliuvo į plėšikų rankas. Tie išrengė jį, sumušė ir nuėjo sau, palikdami pusgyvį. 
\par 31 Atsitiktinai tuo pačiu keliu ėjo vienas kunigas ir, pamatęs jį, praėjo kita kelio puse. 
\par 32 Taip pat ir levitas, pro tą vietą eidamas, jį pamatė ir praėjo kita kelio puse. 
\par 33 O vienas samarietis keliaudamas užtiko jį ir pasigailėjo. 
\par 34 Priėjęs jis aprišo jo žaizdas, užpildamas aliejaus ir vyno, užkėlė ant savo gyvulio, nugabeno į užeigą ir slaugė jį. 
\par 35 Kitą dieną iškeliaudamas jis išsiėmė du denarus, padavė užeigos šeimininkui ir tarė: ‘Slaugyk jį, o jeigu ką išleisi viršaus, sugrįžęs atsilyginsiu’. 
\par 36 Kas iš šitų trijų tau atrodo buvęs artimas patekusiam į plėšikų rankas?” 
\par 37 Jis atsakė: “Tas, kuris jo pasigailėjo”. Jėzus jam tarė: “Eik ir tu taip daryk!” 
\par 38 Jiems keliaujant toliau, Jėzus užsuko į vieną kaimą. Ten viena moteris, vardu Morta, pakvietė Jį į savo namus. 
\par 39 Ji turėjo seserį, vardu Marija. Ši, atsisėdusi prie Viešpaties kojų, klausėsi Jo žodžių. 
\par 40 O Morta rūpinosi visokiu patarnavimu ir stabtelėjusi pasiskundė: “Viešpatie, Tau nerūpi, kad sesuo palieka mane vieną patarnauti? Pasakyk jai, kad man padėtų”. 
\par 41 Tačiau Viešpats atsakė: “Morta, Morta, tu rūpiniesi ir nerimauji dėl daugelio dalykų, 
\par 42 o tereikia vieno. Marija išsirinko geriausiąją dalį, kuri nebus iš jos atimta”.



\chapter{11}


\par 1 Kartą Jėzus vienoje vietoje meldėsi. Jam baigus melstis, vienas iš Jo mokinių paprašė: “Viešpatie, išmokyk mus melstis, kaip ir Jonas išmokė savo mokinius”. 
\par 2 Jėzus tarė jiems: “Kai meldžiatės, sakykite: ‘Tėve mūsų, kuris esi danguje, teesie šventas Tavo vardas. Teateinie Tavo karalystė. Tebūnie Tavo valia kaip danguje, taip ir žemėje. 
\par 3 Kasdienės mūsų duonos duok mums kasdien 
\par 4 ir atleisk mums mūsų kaltes, nes ir mes atleidžiame kiekvienam, kuris mums kaltas. Ir nevesk mūsų į pagundymą, bet gelbėk mus nuo pikto’ ”. 
\par 5 Jis kalbėjo jiems: “Kas nors iš jūsų turės draugą ir, nuėjęs pas jį vidurnaktį, sakys: ‘Bičiuli, paskolink man tris kepalus duonos, 
\par 6 nes draugas iš kelionės pas mane atvyko ir neturiu ko jam padėti ant stalo’. 
\par 7 O anas iš vidaus atsilieps: ‘Nekvaršink manęs! Durys jau uždarytos, o aš su vaikais lovoje, negaliu keltis ir tau duoti’. 
\par 8 Aš jums sakau: jeigu nesikels ir neduos jam duonos dėl draugystės, tai dėl jo atkaklumo atsikels ir duos, kiek tik jam reikia. 
\par 9 Tad ir Aš jums sakau: prašykite, ir jums bus duota; ieškokite, ir rasite; belskite, ir bus jums atidaryta. 
\par 10 Nes kiekvienas, kas prašo, gauna, kas ieško, randa, ir beldžiančiam atidaroma. 
\par 11 Kuris iš jūsų, būdamas tėvas, duonos prašančiam vaikui duotų akmenį?! Ar prašančiam žuvies­ duotų gyvatę? 
\par 12 Arba prašančiam kiaušinio­ duotų skorpioną? 
\par 13 Jei tad jūs, būdami blogi, mokate savo vaikams duoti gerų daiktų, juo labiau jūsų Tėvas iš dangaus duos Šventąją Dvasią tiems, kurie Jį prašo”. 
\par 14 Kartą Jėzus išvarė nebylumo demoną. Demonui išėjus, nebylys prakalbėjo, ir minios stebėjosi. 
\par 15 Bet kai kurie iš jų sakė: “Jis išvaro demonus demonų valdovo Belzebulo jėga”. 
\par 16 Kiti, mėgindami Jį, reikalavo ženklo iš dangaus. 
\par 17 Bet Jėzus, žinodamas jų mintis, tarė jiems: “Kiekviena suskilusi karalystė bus sunaikinta, ir namai grius ant namų. 
\par 18 Jeigu ir šėtonas pasidalijęs, tai kaip išsilaikys jo karalystė? O jūs sakote, kad Aš išvarau demonus Belzebulo jėga. 
\par 19 Jeigu Aš išvarau juos Belzebulo jėga, tai kieno jėga išvaro jūsų sūnūs? Todėl jie bus jūsų teisėjai. 
\par 20 Bet jei Aš išvarau demonus Dievo pirštu, tai tikrai pas jus atėjo Dievo karalystė. 
\par 21 Kai apsiginklavęs galiūnas saugo savo namus, tada ir jo turtas apsaugotas. 
\par 22 Bet jei užpuls stipresnis ir jį nugalės, tai atims visus jo ginklus, kuriais tas pasitikėjo, ir išdalys grobį. 
\par 23 Kas ne su manimi, tas prieš mane, ir kas nerenka su manimi, tas barsto”. 
\par 24 “Netyroji dvasia, išėjusi iš žmogaus, klaidžioja bevandenėse vietose, ieškodama poilsio. Neradusi ji sako: ‘Grįšiu į savo namus, iš kur išėjau’. 
\par 25 Sugrįžusi randa juos iššluotus ir išpuoštus. 
\par 26 Tuomet eina, pasiima kitas septynias dvasias, dar piktesnes už save, ir įėjusios jos ten apsigyvena. Ir tada tam žmogui darosi blogiau negu pirma”. 
\par 27 Jėzui bekalbant, viena moteris iš minios Jam garsiai sušuko: “Palaimintos įsčios, kurios Tave nešiojo, ir krūtys, kurias žindai!” 
\par 28 Jis atsiliepė: “Labiau palaiminti tie, kurie klausosi Dievo žodžio ir jo laikosi”. 
\par 29 Minioms gausėjant, Jis pradėjo kalbėti: “Ši karta yra pikta karta. Ji reikalauja ženklo, bet jai nebus duota jokio kito ženklo, kaip tik Jonos ženklas. 
\par 30 Kaip Jona buvo ženklas nineviečiams, taip Žmogaus Sūnus bus šiai kartai. 
\par 31 Pietų šalies karalienė teismo dieną prisikels drauge su šios kartos žmonėmis ir juos pasmerks, nes ji atkeliavo nuo žemės pakraščių pasiklausyti Saliamono išminties, o štai čia daugiau negu Saliamonas. 
\par 32 Ninevės gyventojai stos teisme drauge su šita karta ir ją pasmerks, nes jie atsivertė, išgirdę Jonos pamokslą, o čia daugiau negu Jona”. 
\par 33 “Niekas uždegto žiburio nededa į slėptuvę ar po indu, bet stato jį į žibintuvą, kad įeinantys matytų šviesą. 
\par 34 Kūno žiburys yra tavoji akis. Todėl, jei tavo akis sveika, visas tavo kūnas bus šviesus. O jeigu tavo akis pikta, visas tavo kūnas bus tamsus. 
\par 35 Todėl žiūrėk, kad tavoji šviesa nebūtų tamsa! 
\par 36 Taigi, jei visas tavo kūnas pilnas šviesos ir neturi jokios tamsios dalies, jis bus visas šviesus, lyg žiburio spindulių apšviestas”. 
\par 37 Jėzui tebekalbant, vienas fariziejus pasikvietė Jį pietų. Įėjęs į vidų, Jis atsisėdo prie stalo. 
\par 38 Tai matydamas, fariziejus stebėjosi, kad Jis nenusiplovė rankų prieš valgydamas. 
\par 39 O Viešpats jam tarė: “Kaip tik jūs, fariziejai, valote taurės ir dubens išorę, o viduje esate pilni gobšumo ir nedorybės. 
\par 40 Neišmanėliai! Argi išorės Kūrėjas nesukūrė ir vidaus? 
\par 41 Verčiau duokite gailestingumo auką iš to, ką turite, tai viskas bus jums nesutepta. 
\par 42 Vargas jums, fariziejai! Jūs duodate dešimtinę iš mėtų, rūtų ir kitokių žolelių, o apleidžiate teisingumą ir Dievo meilę. Tai turite daryti ir ano nepalikti! 
\par 43 Vargas jums, fariziejai! Jūs mėgstate pirmuosius krėslus sinagogose ir sveikinimus aikštėse. 
\par 44 Vargas jums, veidmainiai Rašto žinovai ir fariziejai! Esate lyg apleisti kapai, kuriuos žmonės nepastebėdami mindžioja”. 
\par 45 Tada vienas Įstatymo mokytojas atsiliepė: “Mokytojau, taip kalbėdamas, Tu ir mus įžeidi”. 
\par 46 Jis atsakė: “Vargas ir jums, Įstatymo mokytojai! Jūs kraunate žmonėms nepakeliamas naštas, o patys tų naštų nė vienu pirštu nepajudinate. 
\par 47 Vargas jums! Jūs statote pranašams antkapius, o jūsų tėvai juos žudė! 
\par 48 Taigi jūs liudijate, kad pritariate savo tėvų darbams. Jie žudė, o jūs statote jiems antkapius. 
\par 49 Todėl ir Dievo išmintis pasakė: ‘Aš siųsiu pas juos pranašų ir apaštalų, o jie vienus nužudys, kitus persekios, 
\par 50 kad iš šitos kartos būtų pareikalauta visų pranašų kraujo, pralieto nuo pasaulio sutvėrimo, 
\par 51 pradedant Abelio krauju iki kraujo Zacharijo, kuris buvo nužudytas tarp aukuro ir šventyklos!’ Taip! Aš sakau, jog bus pareikalauta jo iš šios kartos. 
\par 52 Vargas jums, Įstatymo mokytojai! Jūs paėmėte pažinimo raktą, bet patys nėjote ir einantiems trukdėte”. 
\par 53 Kai Jis jiems tai kalbėjo, Rašto žinovai bei fariziejai pradėjo smarkiai Jį pulti ir visaip kamantinėti, 
\par 54 tikėdamiesi ką nors pagauti iš Jo lūpų, kad galėtų Jį apkaltinti.



\chapter{12}


\par 1 Tuo tarpu, kai susirinko nesuskaičiuojama minia, kad net vieni kitus trypė, Jėzus pradėjo kalbėti pirmiausia savo mokiniams: “Saugokitės fariziejų raugo, tai yra veidmainystės! 
\par 2 Nėra nieko uždengto, kas nebus atidengta, ir nieko paslėpto, kas nepasidarys žinoma. 
\par 3 Todėl ką kalbėjote tamsoje, skambės šviesoje, ir ką šnibždėjote į ausį kambariuose, bus skelbiama nuo stogų. 
\par 4 Sakau jums, savo draugams: nebijokite tų, kurie žudo kūną ir paskui nebegali daugiau nieko padaryti. 
\par 5 Aš parodysiu jums, ko turite bijoti: bijokite to, kuris nužudęs, turi galią įmesti į pragarą. Taip, sakau jums, šito bijokite! 
\par 6 Argi ne penki žvirbliai parduodami už du skatikus? Tačiau nė vienas iš jų nėra Dievo pamirštas. 
\par 7 O jūsų net visi galvos plaukai suskaičiuoti. Tad nebijokite! Jūs vertesni už daugybę žvirblių. 
\par 8 Aš jums sakau: kas išpažins mane žmonių akivaizdoje, tą Žmogaus Sūnus išpažins Dievo angelų akivaizdoje. 
\par 9 O kas manęs išsigins žmonių akivaizdoje, to bus išsiginta Dievo angelų akivaizdoje. 
\par 10 Kas tars žodį prieš Žmogaus Sūnų, tam bus atleista, o kas piktžodžiaus Šventajai Dvasiai, tam nebus atleista. 
\par 11 Kai jie ves jus į sinagogas, pas valdininkus ar vyresnybes, nesirūpinkite, kaip ar ką atsakysite ir ką kalbėsite, 
\par 12 nes Šventoji Dvasia tą pačią valandą pamokys jus, ką kalbėti”. 
\par 13 Vienas iš minios Jam tarė: “Mokytojau, liepk mano broliui, kad pasidalytų su manimi palikimą”. 
\par 14 Jis atsakė: “Žmogau, kas gi mane skyrė jūsų teisėju ar dalytoju?” 
\par 15 Jis pasakė jiems: “Žiūrėkite, saugokitės godumo, nes žmogaus gyvybė nepriklauso nuo jo turto gausos”. 
\par 16 Jis pasakė jiems palyginimą: “Vieno turtingo žmogaus laukai davė gausų derlių. 
\par 17 Jis ėmė sau vienas svarstyti: ‘Ką man dabar daryti? Neturiu kur sukrauti derliaus’. 
\par 18 Pagaliau jis tarė: ‘Štai ką padarysiu: nugriausiu savo klojimus, statysiuos didesnius ir į juos sugabensiu visus javus ir visas gėrybes. 
\par 19 Tuomet sakysiu savo sielai: ‘Siela, tu turi daug gėrybių, sukrautų ilgiems metams. Ilsėkis, valgyk, gerk ir linksminkis!’ 
\par 20 O Dievas jam tarė: ‘Kvaily, dar šiąnakt bus pareikalauta tavo sielos. Kam gi atiteks, ką susikrovei?’ 
\par 21 Taip yra tam, kuris krauna turtus sau, bet nesirūpina tapti turtingas pas Dievą”. 
\par 22 Tada Jėzus kalbėjo savo mokiniams: “Todėl sakau jums: nesirūpinkite savo gyvybe, ką valgysite, nė kūnu, ką vilkėsite. 
\par 23 Gyvybė svarbesnė už maistą, o kūnas už drabužį. 
\par 24 Įsižiūrėkite į varnus. Jie nei sėja, nei pjauna, neturi nei sandėlių, nei kluonų, ir Dievas juos maitina. Jūs nepalyginamai vertesni už paukščius! 
\par 25 Kas gi iš jūsų galėtų savo rūpesčiu bent per sprindį pridėti sau ūgio? 
\par 26 Jei tad jūs nesugebate padaryti net mažmožio, tai kam rūpinatės kitais dalykais? 
\par 27 Įsižiūrėkite, kaip auga lelijos. Jos nesidarbuoja ir neaudžia, bet sakau jums: nė Saliamonas visoje savo šlovėje nebuvo taip pasipuošęs kaip kiekviena iš jų. 
\par 28 Jeigu Dievas taip aprengia laukų žolę, šiandien žaliuojančią, o rytoj metamą į krosnį, tai dar labiau pasirūpins jumis, mažatikiai! 
\par 29 Ir neklausinėkite, ką valgysite ar gersite, ir nesirūpinkite! 
\par 30 Visų tų dalykų ieško šio pasaulio pagonys. O jūsų Tėvas žino, kad viso to jums reikia. 
\par 31 Verčiau ieškokite Jo karalystės, o visa tai bus jums pridėta. 
\par 32 Nebijok, mažoji kaimene: jūsų Tėvas panorėjo duoti jums karalystę!” 
\par 33 “Parduokite savo turtą ir aukokite gailestingumo aukas. Įsitaisykite sau piniginių, kurios nesusidėvi, kraukite nenykstantį turtą danguje, kur joks vagis neprieina ir kandys nesuėda. 
\par 34 Nes kur jūsų turtas, ten ir jūsų širdis”. 
\par 35 “Tebūna jūsų strėnos sujuostos ir žiburiai uždegti, 
\par 36 ir būkite panašūs į žmones, kurie laukia savo šeimininko, grįžtančio iš vestuvių, kad kai tik jis parvyks ir pasibels, tuojau atidarytų. 
\par 37 Palaiminti tie tarnai, kuriuos sugrįžęs šeimininkas ras budinčius. Iš tiesų sakau jums: jis susijuos, susodins juos prie stalo ir priėjęs patarnaus jiems. 
\par 38 Jeigu jis grįžtų antrosios ar trečiosios nakties sargybos metu ir rastų juos budinčius, palaiminti tie tarnai! 
\par 39 Įsidėmėkite: jei šeimininkas žinotų, kurią valandą ateis vagis, budėtų ir neleistų jam įsilaužti į savo namus. 
\par 40 Todėl ir jūs būkite pasirengę, nes Žmogaus Sūnus ateis tą valandą, kurią nesitikėsite”. 
\par 41 Tada Petras paklausė: “Viešpatie, ar šį palyginimą sakai tik mums, ar visiems?” 
\par 42 Viešpats atsakė: “Kas yra tas ištikimas ir sumanus ūkvedys, kurį šeimininkas paskirs vadovauti šeimynai ir deramu laiku duoti jiems skirtą maisto dalį? 
\par 43 Palaimintas tarnas, kurį sugrįžęs šeimininkas ras taip darantį. 
\par 44 Sakau jums tiesą: jis paskirs jį valdyti visų savo turtų. 
\par 45 Bet jeigu anas tarnas tartų savo širdyje: ‘Mano šeimininkas neskuba grįžti’, ir imtų mušti tarnus bei tarnaites, valgyti, gerti ir girtuokliauti, 
\par 46 tai to tarno šeimininkas sugrįš tą dieną, kai jis nelaukia, ir tą valandą, kurią jis nesitiki. Jis perkirs jį pusiau ir paskirs jam dalį su neištikimaisiais. 
\par 47 Tarnas, kuris žino savo šeimininko valią, bet nėra pasiruošęs ir pagal jo valią nedaro, bus smarkiai nuplaktas. 
\par 48 O kuris nežino ir baustinai elgiasi, bus mažai plakamas. Iš kiekvieno, kuriam daug duota, bus daug pareikalauta, ir kam daug patikėta, iš to bus daug ir išieškota”. 
\par 49 “Aš atėjau uždegti žemėje ugnies ir taip noriu, kad ji jau liepsnotų! 
\par 50 Bet Aš turiu būti pakrikštytas krikštu ir kaip esu slegiamas, kol tai išsipildys!” 
\par 51 “Gal manote, kad atėjau atnešti žemėn ramybės? Ne, sakau jums, ne ramybės, o nesantaikos. 
\par 52 Nuo dabar penki vienuose namuose bus pasidaliję: trys prieš du ir du prieš tris. 
\par 53 Tėvas stos prieš sūnų, o sūnus prieš tėvą, motina prieš dukterį, o duktė prieš motiną; anyta prieš marčią, ir marti prieš anytą”. 
\par 54 Jėzus pasakė ir minioms: “Matydami debesį, kylantį vakaruose, tuoj pat sakote: ‘Ateina lietus’, ir taip atsitinka. 
\par 55 Pučiant pietų vėjui, tvirtinate: ‘Bus karšta’, ir taip būna. 
\par 56 Veidmainiai! Jūs mokate atpažinti žemės ir dangaus veidą, tai kodėl gi neatpažįstate šio laiko? 
\par 57 Kodėl patys nenusprendžiate, kas teisu? 
\par 58 Kai eini su kaltintoju pas valdininką, pasistenk dar kelyje su juo susitarti, kad jis tavęs nenusitemptų pas teisėją, teisėjas neatiduotų teismo vykdytojui, o teismo vykdytojas neįmestų tavęs į kalėjimą. 
\par 59 Sakau tau: iš ten neišeisi, kol neatsiteisi ligi paskutinio skatiko”.



\chapter{13}


\par 1 Tuo pačiu metu atėjo keli žmonės ir papasakojo Jam apie galilėjiečius, kurių kraują Pilotas sumaišė su jų aukomis. 
\par 2 Jėzus, atsakydamas jiems, tarė: “Ar manote, kad tie galilėjiečiai buvo didesni nusidėjėliai už visus kitus galilėjiečius ir todėl taip nukentėjo? 
\par 3 Ne, sakau jums! Bet jei neatgailausite, visi taip pat pražūsite. 
\par 4 Arba anie aštuoniolika, kuriuos užgriuvo bokštas prie Siloamo ir užmušė; jūs manote, kad jie buvo kaltesni už visus kitus Jeruzalės gyventojus? 
\par 5 Ne, sakau jums, bet jei neatgailausite, visi taip pat pražūsite”. 
\par 6 Jis pasakė palyginimą: “Žmogus turėjo savo vynuogyne pasisodinęs figmedį. Kartą jis atėjo pažiūrėti jo vaisių, bet nerado. 
\par 7 Ir tarė sodininkui: ‘Jau treji metai, kai ateinu ieškoti šio figmedžio vaisių ir vis nerandu. Nukirsk jį, kam dar žemę alina!’ 
\par 8 Anas jam atsakė: ‘Šeimininke, palik dar jį šiais metais. Aš jį apkasiu ir patręšiu. 
\par 9 Jei jis duos vaisių, gerai. O jei ne, tai iškirsk jį!’ ” 
\par 10 Sabato dieną Jėzus mokė vienoje sinagogoje. 
\par 11 Čia buvo moteris, aštuoniolika metų turinti ligos dvasią. Ji buvo sutraukta ir visiškai negalėjo išsitiesti. 
\par 12 Jėzus, pamatęs ją, pasišaukė ir tarė: “Moterie, esi išvaduota iš savo ligos!” 
\par 13 Jis uždėjo ant jos rankas, toji bematant atsitiesė ir ėmė garbinti Dievą. 
\par 14 Tada sinagogos vyresnysis, supykęs, kad Jėzus išgydė ją sabato dieną, pasakė miniai: “Dirbamos yra šešios dienos. Ateikite jomis ir gydykitės, o ne per sabatą!” 
\par 15 Viešpats jam atsakė: “Veidmainy! Argi kas iš jūsų neatriša per sabatą nuo ėdžių savo jaučio ar asilo ir nenuveda pagirdyti? 
\par 16 Argi šios Abraomo dukters, kurią šėtonas laikė surišęs jau aštuoniolika metų, nereikėjo išvaduoti iš pančių sabato dieną?” 
\par 17 Kai Jis tai kalbėjo, visi Jo priešininkai liko sugėdinti, o minia džiaugėsi visais šlovingais Jo atliktais darbais. 
\par 18 Jis tarė: “Į ką panaši Dievo karalystė, ir su kuo ją palyginsiu? 
\par 19 Ji panaši į garstyčios grūdą, kurį ėmė žmogus ir pasėjo savo darže. Ji išaugo į aukštą medelį ir padangių paukščiai susisuko lizdus jo šakose”. 
\par 20 Jis vėl tarė: “Su kuo palyginsiu Dievo karalystę? 
\par 21 Ji panaši į raugą, kurį ėmusi moteris įmaišė į tris saikus miltų, ir nuo jo viskas įrūgo”. 
\par 22 Jėzus ėjo mokydamas per miestelius bei kaimus ir keliavo į Jeruzalę. 
\par 23 Kažkas paklausė Jį: “Viešpatie, ar maža bus išgelbėtųjų?” Jis atsakė jiems: 
\par 24 “Stenkitės įeiti pro siaurus vartus. Sakau jums, daugelis bandys įeiti, bet neįstengs. 
\par 25 Kai namų Šeimininkas atsikels ir užrakins duris, jūs, stovėdami lauke, pradėsite belsti į duris ir prašyti: ‘Viešpatie, Viešpatie, atidaryk mums!’ O Jis atsakys: ‘Aš nežinau, iš kur jūs’. 
\par 26 Tada imsite sakyti: ‘Mes valgėme ir gėrėme Tavo akivaizdoje, Tu mokei mūsų gatvėse’. 
\par 27 O Jis tars: ‘Sakau jums, Aš nežinau, iš kur jūs. Eikite šalin nuo manęs, visi piktadariai!’ 
\par 28 Ten bus verksmo ir dantų griežimo, kai pamatysite Abraomą, Izaoką, Jokūbą ir visus pranašus Dievo karalystėje, o patys būsite išmesti laukan. 
\par 29 Ir ateis iš rytų ir vakarų, iš šiaurės ir pietų ir sėsis prie stalo Dievo karalystėje. 
\par 30 Ir štai yra paskutinių, kurie bus pirmi, ir pirmų, kurie bus paskutiniai”. 
\par 31 Tą pačią dieną atėjo keli fariziejai ir pasakė Jėzui: “Eik iš čia, pasišalink, nes Erodas nori Tave nužudyti”. 
\par 32 Jis jiems atsakė: “Eikite ir pasakykite tam lapei: ‘Štai Aš išvarinėju demonus ir gydau šiandien, tai darysiu ir rytoj, o trečią dieną būsiu viską atlikęs. 
\par 33 Bet šiandien ir rytoj, ir poryt turiu keliauti, nes negali taip būti, kad pranašas žūtų ne Jeruzalėje’ ”. 
\par 34 “Jeruzale, Jeruzale! Tu žudai pranašus ir užmėtai akmenimis tuos, kurie pas tave siųsti. Kiek kartų norėjau surinkti tavo vaikus kaip višta savo viščiukus po sparnais, o tu nenorėjai! 
\par 35 Štai jūsų namai paliekami jums tušti. Iš tiesų sakau jums: jūs manęs nebematysite, kol ateis laikas, kada tarsite: ‘Palaimintas Tas, kuris ateina Viešpaties vardu!’ ”



\chapter{14}


\par 1 Kartą sabato dieną Jėzus atėjo į vieno fariziejų vyresniojo namus valgyti, o jie atidžiai stebėjo Jį. 
\par 2 Ir štai Jį pasitiko vandenlige sergantis žmogus. 
\par 3 Jėzus kreipėsi į Įstatymo mokytojus ir fariziejus: “Leistina per sabatą gydyti ar ne?” 
\par 4 Tie tylėjo. Tada Jis ėmė, išgydė jį ir paleido. 
\par 5 O jiems pasakė: “Jei kurio iš jūsų asilas ar jautis įkristų į šulinį, argi tučtuojau neištrauktų jo per sabatą?” 
\par 6 Ir jie nesugebėjo Jam į tai atsakyti. 
\par 7 Matydamas, kaip svečiai rinkosi pirmąsias vietas prie stalo, Jis pasakė jiems palyginimą: 
\par 8 “Kai kas nors tave pakvies į vestuves, nesėsk pirmoje vietoje, kad kartais nebūtų pakviesta garbingesnio už tave, 
\par 9 ir atėjęs tas, kuris tave ir jį pakvietė, netartų tau: ‘Užleisk jam vietą!’ Tada sugėdintas turėsi sėstis į paskutinę vietą. 
\par 10 Kai būsi pakviestas, eik ir sėskis paskutinėje vietoje, kad tas, kuris tave pakvietė, atėjęs galėtų pasakyti: ‘Bičiuli, persėsk aukščiau!’ Tada tau bus garbė visų prie stalo sėdinčiųjų akivaizdoje. 
\par 11 Kiekvienas, kuris save aukština, bus pažemintas, o kuris save žemina, bus išaukštintas”. 
\par 12 Pakvietusiam Jį vaišių Jėzus irgi pasakė: “Keldamas pietus ar vakarienę, nekviesk nei savo draugų, nei brolių, nei giminaičių, nei turtingų kaimynų, kad kartais jie savo ruožtu nepakviestų tavęs ir tau nebūtų atlyginta. 
\par 13 Rengdamas vaišes, geriau pasikviesk vargšų, paliegėlių, luošų ir aklų, 
\par 14 tai būsi palaimintas, nes jie neturi kuo atsilyginti. Tuomet tau bus atlyginta teisiųjų prisikėlime”. 
\par 15 Tai išgirdęs, vienas iš sėdinčiųjų prie stalo tarė Jam: “Palaimintas, kas valgys duoną Dievo karalystėje!” 
\par 16 Tada Jėzus jam pasakė: “Vienas žmogus iškėlė didelę puotą ir pakvietė daug svečių. 
\par 17 Atėjus puotos metui, jis pasiuntė tarną pranešti pakviestiesiems: ‘Ateikite, jau viskas surengta’. 
\par 18 Tada jie visi kaip vienas pradėjo atsiprašinėti. Vienas jam tarė: ‘Nusipirkau dirvą ir būtinai turiu eiti jos apžiūrėti. Prašau mane pateisinti’. 
\par 19 Kitas sakė: ‘Pirkau penkis jungus jaučių ir einu jų išmėginti. Prašau mane pateisinti’. 
\par 20 Trečias tarė: ‘Vedžiau žmoną, todėl negaliu ateiti’. 
\par 21 Tarnas sugrįžęs viską papasakojo šeimininkui. Šis supyko ir įsakė tarnui: ‘Skubiai eik į miesto gatves ir skersgatvius ir vesk čia vargšus, paliegėlius, luošus ir aklus’. 
\par 22 Tarnas vėl pranešė: ‘Šeimininke, kaip liepei,­padaryta, bet dar yra vietos’. 
\par 23 Tada šeimininkas tarė tarnui: ‘Eik į kelius bei patvorius ir priversk ateiti, kad mano namai būtų pilni. 
\par 24 Sakau jums,­nė vienas iš anų pakviestųjų žmonių neragaus mano vaišių’ ”. 
\par 25 Kartu su Juo ėjo didelės minios. Atsigręžęs Jis tarė žmonėms: 
\par 26 “Jei kas ateina pas mane ir nelaiko neapykantoje savo tėvo, motinos, žmonos, vaikų, brolių, seserų ir net savo gyvybės,­negali būti mano mokinys. 
\par 27 Kas neneša savo kryžiaus ir neseka manimi, negali būti mano mokinys. 
\par 28 Kas iš jūsų, norėdamas pastatyti bokštą, pirmiau atsisėdęs neskaičiuoja išlaidų, kad žinotų, ar turės iš ko užbaigti? 
\par 29 Kad kartais, padėjus pamatą ir nebaigus, žmonės matydami nesišaipytų iš jo, 
\par 30 sakydami: ‘Šitas žmogus pradėjo statyti ir neįstengia baigti’. 
\par 31 Arba koks karalius, traukdamas į karą prieš kitą karalių, pirmiau atsisėdęs nesvarsto, ar, turėdamas dešimt tūkstančių kareivių, pajėgs stoti į kovą su tuo, kuris ateina prieš jį su dvidešimčia tūkstančių?! 
\par 32 Jei ne, tai, anam dar toli esant, siunčia pasiuntinius tartis dėl taikos. 
\par 33 Taip pat kiekvienas iš jūsų, kuris neatsižada viso, ką turi, negali būti mano mokinys”. 
\par 34 “Druska­geras daiktas. Bet jeigu druska netektų sūrumo, kuo ją reikėtų pasūdyti? 
\par 35 Ji nebetinka nei dirvai, nei mėšlui; ją išmeta laukan. Kas turi ausis klausyti­teklauso!”



\chapter{15}


\par 1 Prie Jėzaus artindavosi visi muitininkai ir nusidėjėliai Jo pasiklausyti. 
\par 2 Fariziejai ir Rašto žinovai murmėjo, sakydami: “Šitas priima nusidėjėlius ir su jais valgo”. 
\par 3 Tada Jis pasakė jiems palyginimą: 
\par 4 “Kas iš jūsų, turėdamas šimtą avių ir vienai nuklydus, nepalieka dykumoje devyniasdešimt devynių ir neieško pražuvusios, kol suranda? 
\par 5 Radęs su džiaugsmu dedasi ją ant pečių 
\par 6 ir, sugrįžęs namo, susikviečia draugus bei kaimynus, sakydamas: ‘Džiaukitės drauge su manimi, radau savo avį, kuri buvo pražuvusi!’ 
\par 7 Sakau jums, taip ir danguje bus daugiau džiaugsmo dėl vieno atgailaujančio nusidėjėlio negu dėl devyniasdešimt devynių teisiųjų, kuriems nereikia atgailos. 
\par 8 Arba kuri moteris, turėdama dešimtį drachmų ir vieną pametusi, neužsidega žiburio, nešluoja namų ir rūpestingai neieško, kol suranda? 
\par 9 Radusi susivadina drauges bei kaimynes ir sako: ‘Džiaukitės su manimi, nes radau drachmą, kurią buvau pametusi’. 
\par 10 Sakau jums, šitaip džiaugiasi Dievo angelai dėl vieno atgailaujančio nusidėjėlio”. 
\par 11 Jis kalbėjo toliau: “Vienas žmogus turėjo du sūnus. 
\par 12 Kartą jaunesnysis tarė tėvui: ‘Tėve, atiduok man priklausančią palikimo dalį’. Tėvas padalijo sūnums turtą. 
\par 13 Praėjus kelioms dienoms, jaunesnysis sūnus, pasiėmęs savo dalį, iškeliavo į tolimą šalį. Ten, palaidai gyvendamas, iššvaistė savo turtą. 
\par 14 Kai viską išleido, toje šalyje kilo didelis badas, ir jis pradėjo stokoti. 
\par 15 Tada apsistojo pas vieną tos šalies gyventoją. Tas jį pasiuntė į laukus kiaulių ganyti. 
\par 16 Jis geidė prikimšti pilvą bent ankštimis, kurias ėdė kiaulės, bet ir tų niekas jam neduodavo. 
\par 17 Tada susiprotėjęs jis tarė: ‘Kiek mano tėvo samdinių apsčiai turi duonos, o aš čia mirštu iš bado! 
\par 18 Kelsiuos, eisiu pas tėvą ir sakysiu: ‘Tėve, nusidėjau dangui ir tau. 
\par 19 Nesu vertas vadintis tavo sūnumi. Priimk mane bent samdiniu!’ 
\par 20 Jis pakilo ir iškeliavo pas tėvą. Jam dar toli esant, tėvas jį pamatė ir, apimtas gailesčio, pribėgo, puolė ant kaklo ir meiliai pabučiavo. 
\par 21 O sūnus jam tarė: ‘Tėve, nusidėjau dangui ir tau, nebesu vertas vadintis tavo sūnumi’. 
\par 22 Bet tėvas įsakė savo tarnams: ‘Kuo greičiau atneškite geriausią drabužį ir apvilkite jį. Užmaukite jam ant piršto žiedą, apaukite kojas! 
\par 23 Atveskite nupenėtą veršį ir papjaukite! Valgysim ir linksminsimės! 
\par 24 Nes šis mano sūnus buvo miręs ir vėl atgijo, buvo pražuvęs ir atsirado’. Ir jie pradėjo linksmintis. 
\par 25 O vyresnysis jo sūnus buvo laukuose. Kai grįždamas prisiartino prie namų, išgirdo muziką ir šokius. 
\par 26 Pasišaukęs vieną iš tarnų, jis paklausė, kas čia dedasi. 
\par 27 Tas jam atsakė: ‘Sugrįžo tavo brolis, tai tėvas papjovė nupenėtą veršį, nes sulaukė jo sveiko’. 
\par 28 Tada šis supyko ir nenorėjo eiti vidun. Tėvas išėjęs ėmė jį kviesti. 
\par 29 O jis atsakė tėvui: ‘Štai jau tiek metų tau tarnauju ir niekad tavo įsakymo neperžengiau, o tu man nė karto nedavei nė ožiuko pasilinksminti su draugais. 
\par 30 Bet vos tik sugrįžo šitas tavo sūnus, prarijęs tavo turtą su kekšėmis, tu tuojau papjovei jam nupenėtą veršį’. 
\par 31 Tėvas atsakė: ‘Sūnau, tu visuomet su manimi, ir visa, kas mano, yra ir tavo. 
\par 32 Bet reikėjo džiaugtis ir linksmintis, nes tavo brolis buvo miręs ir vėl atgijo, buvo pražuvęs ir atsirado!’ ”



\chapter{16}


\par 1 Jėzus kalbėjo ir savo mokiniams: “Buvo vienas turtingas žmogus ir turėjo ūkvedį. Tas buvo jam apskųstas, esą eikvojąs jo turtą. 
\par 2 Tuomet, pasišaukęs jį, šeimininkas pasakė: ‘Ką aš girdžiu apie tave šnekant? Duok savo ūkvedžiavimo apyskaitą, nes daugiau nebegalėsi būti ūkvedžiu’. 
\par 3 O ūkvedys tarė sau: ‘Ką veiksiu, kad šeimininkas atima iš manęs ūkvedžiavimą? Kasti neįstengiu, o elgetauti­gėda. 
\par 4 Žinau, ką daryti, kad žmonės mane priimtų į savo namus, kai būsiu atleistas iš tarnybos’. 
\par 5 Jis pasikvietė po vieną savo šeimininko skolininkus ir klausė pirmąjį: ‘Kiek esi skolingas mano šeimininkui?’ 
\par 6 Šis atsakė: ‘Šimtą statinių aliejaus’. Tada jis tarė: ‘Imk savo skolos raštą, sėsk ir tuoj pat rašyk: penkiasdešimt’. 
\par 7 Paskui klausė kitą: ‘O kiek tu skolingas?’ Anas atsakė: ‘Šimtą saikų kviečių’. Jis tarė: ‘Imk skolos raštą ir rašyk: aštuoniasdešimt’. 
\par 8 Šeimininkas pagyrė neteisųjį ūkvedį, kad jis gudriai pasielgė. Šio pasaulio vaikai sumanesni tarp savųjų negu šviesos vaikai”. 
\par 9 “Ir Aš jums sakau: darykitės bičiulių neteisiosios Mamonos dėka, kad, jūsų galui atėjus, jie priimtų jus į amžinuosius namus. 
\par 10 Kas ištikimas mažame dalyke, tas ištikimas ir dideliame, o kas neteisingas mažame, tas neteisingas ir dideliame. 
\par 11 Jei nebuvote ištikimi, tvarkydami neteisiąją Mamoną, tai kas jums patikės tikruosius turtus? 
\par 12 Ir jeigu nebuvote ištikimi su svetimu daiktu, tai kas jums duos tai, kas jūsų? 
\par 13 Joks tarnas negali tarnauti dviem šeimininkams, nes jis arba vieno nekęs, o kitą mylės, arba prie vieno prisiriš, o kitą nieku vers. Negalite tarnauti Dievui ir Mamonai”. 
\par 14 Visa tai girdėjo mėgstantys pinigus fariziejai ir šaipėsi iš Jėzaus. 
\par 15 O Jis jiems pasakė: “Jūs žmonių akyse dedatės teisūs, bet Dievas mato jūsų širdis. Nes tai, kas žmonėse aukštinama, Dievo akyse bjauru”. 
\par 16 “Įstatymas ir pranašai­iki Jono; nuo tada skelbiama Dievo karalystė, ir kiekvienas į ją veržiasi. 
\par 17 Greičiau dangus ir žemė praeis, negu iš Įstatymo iškris bent vienas brūkšnelis. 
\par 18 Kiekvienas, kuris atleidžia žmoną ir veda kitą­svetimauja. Ir kas veda vyro atleistąją, taip pat svetimauja”. 
\par 19 “Gyveno vienas turtuolis. Jis vilkėjo purpuru bei ploniausia drobe ir kasdien ištaigingai linksminosi. 
\par 20 O prie jo vartų gulėjo votimis aptekęs elgeta, vardu Lozorius. 
\par 21 Jis troško numarinti alkį bent trupiniais nuo turtuolio stalo, bet tik šunys atbėgę laižydavo jo votis. 
\par 22 Ir štai elgeta mirė ir buvo angelų nuneštas į Abraomo prieglobstį. Mirė ir turtuolis ir buvo palaidotas. 
\par 23 Kentėdamas pragare, jis pakėlė akis ir iš tolo pamatė Abraomą ir jo prieglobstyje Lozorių. 
\par 24 Jis ėmė šaukti: ‘Tėve Abraomai, pasigailėk manęs! Atsiųsk Lozorių, kad, suvilgęs vandenyje piršto galą, atvėsintų man liežuvį, nes baisiai kenčiu šioje liepsnoje’. 
\par 25 Bet Abraomas atsakė: ‘Atsimink, sūnau, kad tu gyvendamas atsiėmei savo gėrybes, o Lozorius­tik nelaimes. Todėl jis susilaukė paguodos, o tu kenti. 
\par 26 Be to, mus nuo jūsų skiria milžiniška praraja, ir niekas iš čia panorėjęs negali nueiti pas jus, nei iš ten persikelti pas mus’. 
\par 27 Tas vėl tarė: ‘Tai meldžiu tave, tėve, nusiųsk jį bent į mano tėvo namus, 
\par 28 nes aš turiu penkis brolius. Teįspėja juos, kad ir jie nepatektų į šią kančių vietą’. 
\par 29 Abraomas atsiliepė: ‘Jie turi Mozę bei pranašus, tegul jų klauso!’ 
\par 30 O anas atsakė: ‘Ne, tėve Abraomai! Bet jei kas iš mirusiųjų nueitų pas juos, jie atgailautų!’ 
\par 31 Tačiau Abraomas tarė: ‘Jeigu jie neklauso Mozės ir pranašų, tai nepatikės, jei kas ir iš numirusių prisikeltų’ ”.



\chapter{17}


\par 1 Jėzus kalbėjo savo mokiniams: “Papiktinimai neišvengiami, bet vargas tam, per kurį jie ateina. 
\par 2 Jam būtų geriau, jei ant kaklo užkabintų girnų akmenį ir įmestų jūron, negu jis papiktintų nors vieną iš šitų mažutėlių. 
\par 3 Saugokitės! Jei tavo brolis nusideda prieš tave, sudrausk jį ir, jeigu jis atgailauja, atleisk jam. 
\par 4 Jei jis septynis kartus per dieną tau nusidėtų ir septynis kartus kreiptųsi į tave, sakydamas: ‘Atgailauju’,­atleisk jam”. 
\par 5 Apaštalai tarė Viešpačiui: “Sustiprink mūsų tikėjimą”. 
\par 6 O Viešpats atsakė: “Jei turėtumėte tikėjimą kaip garstyčios grūdelį, galėtumėte sakyti šitam šilkmedžiui: ‘Išsirauk ir pasisodink jūroje’,­ir jis paklausytų jūsų”. 
\par 7 “Kas iš jūsų, turėdamas ariantį ar ganantį vergą, jam grįžus iš lauko, sakys: ‘Tuojau sėsk prie stalo’? 
\par 8 Argi nesakys jam: ‘Paruošk man vakarienę. Susijuosk ir patarnauk, kol aš valgysiu ir gersiu, o paskui tu pavalgysi ir atsigersi’? 
\par 9 Argi vergui dėkojama, kad jis atliko tai, kas jam liepta? Nemanau. 
\par 10 Taigi jūs, atlikę visa, kas jums pavesta, sakykite: ‘Esame nenaudingi vergai. Padarėme, ką privalėjome padaryti’ ”. 
\par 11 Keliaujant į Jeruzalę, teko Jėzui eiti tarp Samarijos ir Galilėjos. 
\par 12 Jam įeinant į vieną kaimą, Jį pasitiko dešimt raupsuotų vyrų. Jie sustojo atstu 
\par 13 ir garsiai šaukė: “Jėzau, Mokytojau, pasigailėk mūsų!” 
\par 14 Pamatęs juos, Jis tarė: “Eikite, pasirodykite kunigams!” Ir beeidami jie pasveiko. 
\par 15 Vienas iš jų, patyręs, kad išgijo, sugrįžo atgal, garsiai garbindamas Dievą. 
\par 16 Jis dėkodamas parpuolė Jėzui po kojų. Tai buvo samarietis. 
\par 17 Jėzus paklausė: “Argi ne dešimt pasveiko? Kur dar devyni? 
\par 18 Ar neatsirado nė vieno, kuris grįžtų atiduoti Dievui garbę, išskyrus šitą svetimtautį?” 
\par 19 Ir tarė jam: “Kelkis, eik! Tavo tikėjimas išgydė tave”. 
\par 20 Fariziejų paklaustas, kada ateis Dievo karalystė, Jėzus atsakė: “Dievo karalystė neateina regimai. 
\par 21 Niekas nepasakys: ‘Žiūrėk, ji čia’, arba: ‘Žiūrėk, ji ten!’ Nes štai Dievo karalystė yra tarp jūsų”. 
\par 22 Ir Jis tarė mokiniams: “Ateis dienos, kai norėsite išvysti bent vieną Žmogaus Sūnaus dieną, ir nepamatysite. 
\par 23 Jums sakys: ‘Žiūrėkite čia! Žiūrėkite ten!’­Nesekite jais ir neikite paskui juos. 
\par 24 Kaip tvykstelėjęs žaibas nušviečia viską nuo vieno dangaus pakraščio iki kito, taip savo dieną pasirodys ir Žmogaus Sūnus. 
\par 25 Bet pirmiau Jis turės daug iškentėti ir būti šitos kartos atmestas. 
\par 26 Kaip buvo Nojaus dienomis, taip bus ir Žmogaus Sūnaus dienomis. 
\par 27 Jie valgė, gėrė, tuokėsi ir tuokė, kol atėjo diena, kai Nojus įlipo į laivą. Tada užėjo tvanas ir visus sunaikino. 
\par 28 Taip pat buvo ir Loto dienomis. Jie valgė ir gėrė, pirko ir pardavinėjo, sodino ir statė. 
\par 29 O tą dieną, kada Lotas paliko Sodomą, iš dangaus krito ugnis ir siera ir visus sunaikino. 
\par 30 Šitaip bus ir tą dieną, kai pasirodys Žmogaus Sūnus. 
\par 31 Kas tą dieną bus ant stogo, o jo daiktai viduje, tenelipa žemyn jų pasiimti, o kas laukuose, tenegrįžta namo. 
\par 32 Prisiminkite Loto žmoną! 
\par 33 Kas stengsis išgelbėti savo gyvybę, tas ją praras, o kas ją praras, tas atgaivins ją. 
\par 34 Sakau jums: tą naktį dviese miegos vienoje lovoje, ir vienas bus paimtas, o kitas paliktas. 
\par 35 Dvi mals drauge, ir viena bus paimta, o kita palikta. 
\par 36 Du bus lauke, ir vienas bus paimtas, kitas paliktas”. 
\par 37 Tada jie atsiliepė: “O kurgi, Viešpatie?” Jis atsakė: “Kur tik bus lavonų, ten sulėks ir maitvanagiai”.



\chapter{18}


\par 1 Jėzus pasakė jiems palyginimą, kaip reikia visuomet melstis ir neprarasti ryžto, 
\par 2 tardamas: “Viename mieste buvo teisėjas, kuris nebijojo Dievo ir nesidrovėjo žmonių. 
\par 3 Tame pačiame mieste gyveno našlė, kuri vis eidavo pas jį ir prašydavo: ‘Apgink mane nuo mano priešininko!’ 
\par 4 Jis kurį laiką nenorėjo, bet po to tarė sau: ‘Nors aš Dievo nebijau nei žmonių nesidroviu, 
\par 5 vis dėlto, kai šita našlė neduoda man ramybės, apginsiu jos teises, kad ji, vienąkart atėjusi, man akių neišdraskytų’ ”. 
\par 6 Ir Viešpats tarė: “Įsidėmėkite, ką pasakė tas neteisusis teisėjas. 
\par 7 Tad nejaugi Dievas neapgins savo išrinktųjų, kurie Jo šaukiasi dieną ir naktį, ir dels jiems padėti? 
\par 8 Sakau jums: Jis apgins jų teises labai greitai. Bet ar atėjęs Žmogaus Sūnus beras žemėje tikėjimą?” 
\par 9 Tiems, kurie pasitikėjo savo teisumu, o kitus niekino, Jėzus pasakė palyginimą: 
\par 10 “Du žmonės atėjo į šventyklą melstis: vienas­fariziejus, o kitas­muitininkas. 
\par 11 Fariziejus stovėdamas taip sau vienas meldėsi: ‘Dėkoju Tau, Dieve, kad nesu toks, kaip kiti žmonės­plėšikai, sukčiai, svetimautojai­arba kaip šis va muitininkas. 
\par 12 Aš pasninkauju du kartus per savaitę, duodu dešimtinę iš visko, ką įsigyju’. 
\par 13 O muitininkas, atokiai stovėdamas, nedrįso nė akių pakelti į dangų, tik, mušdamasis į krūtinę, maldavo: ‘Dieve, būk gailestingas man, nusidėjėliui!’ 
\par 14 Sakau jums: šitas nuėjo į namus išteisintas, o ne anas. Kiekvienas, kuris save aukština, bus pažemintas, o kuris save žemina, bus išaukštintas”. 
\par 15 Jėzui atnešdavo kūdikių, kad Jis juos paliestų. Mokiniai, tai matydami, draudė. 
\par 16 Bet Jėzus, pasišaukęs vaikučius, tarė: “Leiskite mažutėliams ateiti pas mane ir netrukdykite jiems, nes tokių yra Dievo karalystė. 
\par 17 Iš tiesų sakau jums: kas nepriims Dievo karalystės kaip kūdikis, niekaip neįeis į ją!” 
\par 18 Vienas valdininkas Jį paklausė: “Gerasis Mokytojau, ką turiu daryti, kad paveldėčiau amžinąjį gyvenimą?” 
\par 19 Jėzus jam atsakė: “Kodėl vadini mane geru? Nė vieno nėra gero, tik vienas Dievas. 
\par 20 Įsakymus žinai: ‘Nesvetimauk, nežudyk, nevok, melagingai neliudyk, gerbk savo tėvą ir motiną’ ”. 
\par 21 Tas atsakė: “Viso šito laikausi nuo savo jaunystės”. 
\par 22 Tai išgirdęs, Jėzus jam tarė: “Tau trūksta vieno dalyko: parduok visa, ką turi, išdalink vargšams, ir turėsi turtą danguje. Tada ateik ir sek paskui mane!” 
\par 23 Tai išgirdęs, jis nusiminė, nes buvo labai turtingas. 
\par 24 Matydamas jį nuliūdusį, Jėzus prabilo: “Kaip sunkiai turintieji turtų įeis į Dievo karalystę! 
\par 25 Lengviau kupranugariui išlįsti pro adatos ausį, negu turtingam įeiti į Dievo karalystę!” 
\par 26 Girdėjusieji tai paklausė: “Kas tada gali būti išgelbėtas?” 
\par 27 Jis atsakė: “Kas neįmanoma žmonėms, įmanoma Dievui”. 
\par 28 Tada Petras tarė: “Štai mes viską palikome ir nusekėme paskui Tave”. 
\par 29 Jėzus atsakė jiems: “Iš tiesų sakau jums: nėra nė vieno, kuris paliko namus ar tėvus, ar brolius, ar žmoną, ar vaikus dėl Dievo karalystės 
\par 30 ir kuris jau šiuo metu negautų nepalyginamai daugiau, o būsimajame pasaulyje­amžinojo gyvenimo”. 
\par 31 Jėzus pasišaukė dvylika ir tarė: “Štai mes einame į Jeruzalę, ir ten išsipildys visa, kas per pranašus parašyta apie Žmogaus Sūnų. 
\par 32 Jis bus atiduotas pagonims, išjuoktas, paniekintas ir apspjaudytas. 
\par 33 Tie nuplaks Jį ir nužudys, bet trečią dieną Jis prisikels”. 
\par 34 Tačiau jie nieko nesuprato; tų žodžių prasmė jiems liko paslėpta, ir jie nesuvokė, kas buvo sakoma. 
\par 35 Jam artinantis prie Jericho, šalia kelio sėdėjo neregys elgeta. 
\par 36 Išgirdęs praeinančią minią, jis paklausė, kas tai būtų. 
\par 37 Jam atsakė, kad praeinąs Jėzus iš Nazareto. 
\par 38 Tada jis ėmė šaukti: “Jėzau, Dovydo Sūnau, pasigailėk manęs!” 
\par 39 Ėję priekyje draudė jį, kad tylėtų, bet tas dar garsiau šaukė: “Dovydo Sūnau, pasigailėk manęs!” 
\par 40 Jėzus sustojo ir liepė jį atvesti. Jam prisiartinus, paklausė jo: 
\par 41 “Ko nori, kad tau padaryčiau?” 
\par 42 Šis atsakė: “Viešpatie, kad praregėčiau!” Jėzus tarė: “Regėk! Tavo tikėjimas išgydė tave”. 
\par 43 Tas iškart praregėjo ir nusekė paskui Jį, garbindamas Dievą. Tai matydami, visi žmonės šlovino Dievą.



\chapter{19}


\par 1 Atvykęs Jėzus ėjo per Jerichą. 
\par 2 Ir štai žmogus, vardu Zachiejus, muitininkų viršininkas ir turtuolis, 
\par 3 troško pamatyti Jėzų, kas Jis esąs, bet negalėjo per minią, nes buvo žemo ūgio. 
\par 4 Zachiejus užbėgo priekin ir įlipo į šilkmedį, kad galėtų Jį pamatyti, nes Jis turėjo ten praeiti. 
\par 5 Kai Jėzus atėjo į tą vietą, pažvelgė aukštyn ir, pamatęs jį, tarė: “Zachiejau, greit lipk žemyn, nes šiandien turiu apsistoti tavo namuose”. 
\par 6 Šis skubiai nulipo ir su džiaugsmu priėmė Jį. 
\par 7 Tai matydami, visi murmėjo: “Pas nusidėjėlį užėjo į svečius!” 
\par 8 O Zachiejus atsistojęs prabilo į Viešpatį: “Štai, Viešpatie, pusę savo turto atiduodu vargšams ir, jei ką nors nuskriaudžiau, grąžinsiu keturgubai”. 
\par 9 Jėzus tarė: “Šiandien į šiuos namus atėjo išgelbėjimas, nes ir jis yra Abraomo sūnus. 
\par 10 Juk Žmogaus Sūnus atėjo ieškoti ir gelbėti, kas buvo pražuvę”. 
\par 11 Jiems klausantis, Jis tęsė ir pasakė palyginimą. Mat Jis buvo netoli Jeruzalės, ir žmonės manė, jog tuojau pat turi pasirodyti Dievo karalystė. 
\par 12 Tad Jis kalbėjo: “Vienas aukštos kilmės žmogus iškeliavo į tolimą šalį, kad gautų karalystę ir sugrįžtų. 
\par 13 Pasišaukęs dešimt savo tarnų, padalijo jiems dešimt minų ir tarė: ‘Verskitės, kol sugrįšiu’. 
\par 14 Piliečiai nekentė jo ir nusiuntė iš paskos pasiuntinius pareikšti: ‘Mes nenorime, kad šitas mums viešpatautų’. 
\par 15 Gavęs karalystę, jis sugrįžo ir liepė pašaukti tarnus, kuriems buvo davęs pinigų, norėdamas sužinoti, kiek kuris uždirbo. 
\par 16 Atėjo pirmasis ir tarė: ‘Valdove, tavo mina pelnė dešimt minų’. 
\par 17 Jis atsakė: ‘Gerai, stropusis tarne! Kadangi pasirodei ištikimas mažuose dalykuose, tu gauni valdyti dešimtį miestų’. 
\par 18 Atėjo antrasis ir pareiškė: ‘Valdove, tavo mina pelnė penkias minas’. 
\par 19 Ir šitam jis pasakė: ‘Tu valdyk penkis miestus’. 
\par 20 Atėjo dar vienas ir tarė: ‘Valdove, štai tavo mina, kurią laikiau suvyniotą į skepetą. 
\par 21 Aš bijojau tavęs, nes esi griežtas žmogus: imi tai, ko nepadėjai, ir pjauni tai, ko nepasėjai’. 
\par 22 Jis atsiliepė: ‘Tavo paties žodžiais teisiu tave, netikęs tarne. Tu žinojai, kad aš griežtas žmogus: imu, ko nepadėjau, ir pjaunu, ko nesėjau. 
\par 23 Tai kodėl neleidai mano pinigų apyvarton, kad sugrįžęs išreikalaučiau su palūkanomis?’ 
\par 24 Šalia stovėjusiems jis tarė: ‘Atimkite iš jo miną ir atiduokite tam, kuris turi dešimt minų’. 
\par 25 Tie atsakė: ‘Valdove, bet anas jau turi dešimt minų!’ 
\par 26 Jis tarė: ‘Aš sakau jums: kiekvienam, kas turi, bus duota, o iš neturinčio bus atimta net ir tai, ką turi. 
\par 27 O tuos mano priešus, nenorėjusius, kad jiems viešpataučiau, atveskite čia ir nužudykite mano akyse!’ ” 
\par 28 Tai pasakęs, Jėzus visų priešakyje nukreipė žingsnius į Jeruzalę. 
\par 29 Prisiartinęs prie Betfagės ir Betanijos, prie vadinamojo Alyvų kalno, Jis pasiuntė du mokinius, 
\par 30 sakydamas jiems: “Eikite į priešais esantį kaimą ir įėję rasite pririštą asilaitį, kuriuo dar joks žmogus nėra jojęs. Atriškite jį ir atveskite. 
\par 31 O jeigu kas klaustų: ‘Kodėl jį atrišate?’, atsakykite: ‘Jo reikia Viešpačiui’ ”. 
\par 32 Pasiųstieji nuėjo ir rado, kaip jiems buvo pasakyta. 
\par 33 Atrišant asilaitį, savininkai klausė: “Kodėl atrišate asilaitį?” 
\par 34 Tie atsakė: “Jo reikia Viešpačiui”. 
\par 35 Jie atvedė asilaitį pas Jėzų, apdengė savo apsiaustais ir užsodino Jėzų ant viršaus. 
\par 36 Jam jojant, žmonės tiesė ant kelio savo drabužius. 
\par 37 Besiartinant prie Alyvų kalno šlaito, visas mokinių būrys pradėjo džiaugsmingai ir garsiai šlovinti Dievą už visus stebuklus, kuriuos jie buvo regėję. 
\par 38 Jie šaukė: “ ‘Palaimintas karalius, kuris ateina Viešpaties vardu!’ Ramybė danguje, šlovė aukštybėse!” 
\par 39 Kai kurie fariziejai iš minios Jam sakė: “Mokytojau, sudrausk savo mokinius!” 
\par 40 Jis jiems atsakė: “Sakau jums, jei šie tylės­akmenys šauks!” 
\par 41 Prisiartinęs prie Jeruzalės ir išvydęs miestą, Jėzus verkė dėl jo ir sakė: 
\par 42 “O kad tu nors šiandien suprastum, kas tau atneša ramybę! Deja, tai paslėpta nuo tavo akių. 
\par 43 Tu sulauksi dienų, kai tavo priešai apjuos tave pylimu, apguls iš visų pusių ir suspaus tave; 
\par 44 jie parblokš ant žemės tave ir tavo vaikus ir nepaliks tavyje akmens ant akmens, nes tu nepažinai savo aplankymo meto”. 
\par 45 Įėjęs į šventyklą, Jis pradėjo varyti laukan parduodančius ir perkančius. 
\par 46 Jis sakė jiems: “Parašyta: ‘Mano namai yra maldos namai’, o jūs pavertėte juos ‘plėšikų lindyne’ ”. 
\par 47 Ir Jis kasdien mokė šventykloje. O aukštieji kunigai ir Rašto žinovai bei tautos vyresnieji troško Jį pražudyti, 
\par 48 tačiau neišmanė, ką galėtų padaryti, nes visi žmonės apgulę Jį klausėsi.



\chapter{20}


\par 1 Vieną iš tų dienų Jėzui mokant žmones šventykloje ir skelbiant Evangeliją, atėjo aukštųjų kunigų ir Rašto žinovų su vyresniaisiais 
\par 2 ir klausė Jį: “Pasakyk mums, kokią teisę turi taip daryti? Kas Tau davė šitą valdžią?” 
\par 3 Jis jiems tarė: “Aš irgi paklausiu jus vieno dalyko, ir atsakykite man: 
\par 4 ar Jono krikštas buvo iš dangaus, ar iš žmonių?” 
\par 5 Jie svarstė tarpusavyje: “Jei pasakysime­iš dangaus, Jis mus klaus: ‘Tai kodėl juo netikėjote?’ 
\par 6 O jei sakysime­iš žmonių, visa minia užmėtys mus akmenimis, nes žmonės įsitikinę, kad Jonas buvo pranašas”. 
\par 7 Ir jie atsakė, kad nežino iš kur. 
\par 8 Tada Jėzus jiems tarė: “Tai ir Aš jums nesakysiu, kokia valdžia tai darau”. 
\par 9 Jis pradėjo pasakoti žmonėms tokį palyginimą: “Vienas žmogus pasodino vynuogyną, išnuomojo jį vynininkams ir ilgesniam laikui iškeliavo svetur. 
\par 10 Atėjus metui, jis nusiuntė pas vynininkus tarną, kad tie jam duotų vynuogyno derliaus dalį. Tačiau vynininkai sumušė jį ir paleido tuščiomis. 
\par 11 Jis vėl nusiuntė kitą tarną, bet ir tą jie sumušė, išjuokė ir paleido jį tuščiomis. 
\par 12 Tuomet jis pasiuntė dar trečią, bet jie ir šitą sužeidę išvarė. 
\par 13 Tada vynuogyno šeimininkas tarė: ‘Ką man daryti? Pasiųsiu savo mylimąjį sūnų, gal pamatę jie gerbs jį?’ 
\par 14 Bet, išvydę sūnų, vynininkai ėmė tartis tarpusavyje: ‘Tai paveldėtojas. Užmuškime jį, kad mums atitektų palikimas’. 
\par 15 Jie išmetė jį iš vynuogyno ir nužudė. Ką gi su jais darys vynuogyno savininkas? 
\par 16 Jis ateis ir išžudys tuos vynininkus, o vynuogyną atiduos kitiems”. Tai girdėjusieji tarė: “Tenebūna šitaip!” 
\par 17 Bet Jėzus, pažvelgęs į juos, paklausė: “O ką reiškia Rašto posakis: ‘Akmuo, kurį statytojai atmetė, tapo kertiniu akmeniu?’ 
\par 18 Kas tiktai kris ant šito akmens, suduš, o ant ko tas akmuo užgrius, tą sutriuškins”. 
\par 19 Aukštieji kunigai ir Rašto žinovai norėjo dar tą pačią valandą Jį suimti, bet bijojo žmonių. Mat jie suprato, kad šis palyginimas buvo jiems taikomas. 
\par 20 Jie patys atidžiai Jėzų stebėjo ir siuntinėjo šnipus, apsimetančius teisuoliais, kad, nutvėrę kokį Jo žodį, galėtų Jį atiduoti valdytojui nuteisti ir nubausti. 
\par 21 Taigi jie klausė Jį: “Mokytojau, mes žinome, kad Tu kalbi ir mokai teisingai. Tu neatsižvelgi į asmenis, bet mokai Dievo kelio, kaip reikalauja tiesa. 
\par 22 Ar reikia mums mokėti ciesoriui mokesčius, ar ne?” 
\par 23 Suprasdamas jų klastą, Jis tarė jiems: “Kodėl spendžiate man pinkles? 
\par 24 Parodykite man denarą. Kieno atvaizdas ir įrašas jame?” Jie atsakė: “Ciesoriaus”. 
\par 25 Tada Jis jiems tarė: “Kas ciesoriaus, atiduokite ciesoriui, o kas Dievo­Dievui”. 
\par 26 Taip jie nesugebėjo žmonių akivaizdoje sugauti Jo kalboje. Stebėdamiesi Jo atsakymu, jie nutilo. 
\par 27 Pas Jį atėjo sadukiejų, kurie neigia mirusiųjų prisikėlimą, ir paklausė: 
\par 28 “Mokytojau, Mozė mums parašė: jei kieno brolis, turėdamas žmoną, mirtų bevaikis, tada jo brolis tegul veda jo žmoną ir pažadina savo broliui palikuonių. 
\par 29 Taigi buvo septyni broliai. Pirmasis vedė žmoną ir mirė bevaikis. 
\par 30 Ją vedė antrasis ir taip pat mirė bevaikis. 
\par 31 Paskui ją vedė trečiasis ir paeiliui visi septyni, ir jie mirė, nepalikdami vaikų. 
\par 32 Po jų visų numirė ir ta moteris. 
\par 33 Kurio gi žmona ji bus prisikėlime? Juk ji buvo visų septynių žmona!” 
\par 34 Jėzus jiems atsakė: “Šio pasaulio vaikai veda ir teka, 
\par 35 o kurie pasirodys verti pasiekti aną pasaulį ir mirusiųjų prisikėlimą, tie neves ir netekės. 
\par 36 Jie taip pat nebegalės ir mirti, nes, būdami prisikėlimo vaikai, bus lygūs angelams ir bus Dievo vaikai. 
\par 37 O kad mirusieji prisikels, nurodė ir Mozė pasakojime apie krūmą, kur jis Viešpatį vadina ‘Abraomo Dievu, Izaoko Dievu ir Jokūbo Dievu’. 
\par 38 Taigi Dievas nėra mirusiųjų Dievas, bet gyvųjų, nes visi Jam gyvena”. 
\par 39 Tada kai kurie Rašto žinovai atsiliepė: “Mokytojau, Tu gerai pasakei!” 
\par 40 Ir daugiau jie nedrįso nieko Jo klausti. 
\par 41 Jis paklausė jų: “Kodėl sakoma, jog Kristus esąs Dovydo Sūnus? 
\par 42 Juk pats Dovydas Psalmių knygoje sako: ‘Viešpats tarė mano Viešpačiui: sėskis mano dešinėje, 
\par 43 kol patiesiu Tavo priešus tarsi pakojį po Tavo kojų’. 
\par 44 Taigi Dovydas vadina Jį Viešpačiu,­kaip tada Jis gali būti jo Sūnus?” 
\par 45 Visiems žmonėms girdint, Jėzus tarė savo mokiniams: 
\par 46 “Saugokitės Rašto žinovų, kurie mėgsta vaikščioti ilgais drabužiais, būti sveikinami aikštėse, užimti pirmuosius krėslus sinagogose bei garbės vietas pokyliuose. 
\par 47 Jie suryja našlių namus ir dedasi kalbą ilgas maldas. Jie gaus dar didesnį pasmerkimą”.



\chapter{21}


\par 1 Pažvelgęs Jis pamatė turtinguosius, dedančius savo dovanas į iždinę. 
\par 2 Jis pamatė ir vieną beturtę našlę, kuri įmetė du smulkius pinigėlius. 
\par 3 Ir Jis tarė: “Iš tiesų sakau jums, ši beturtė našlė įmetė daugiau už visus. 
\par 4 Nes anie visi iš savo pertekliaus aukojo dovanų Dievui, o ji iš savo nepritekliaus įmetė viską, ką turėjo pragyvenimui”. 
\par 5 Kai kuriems kalbant apie šventyklą, kad ji išpuošta gražiais akmenimis bei dovanomis, Jėzus prabilo: 
\par 6 “Ateis dienos, kai iš to, ką matote, neliks akmens ant akmens, viskas bus išgriauta”. 
\par 7 Jie paklausė: “Mokytojau, kada tai įvyks? Ir koks bus ženklas, kai visa tai pradės pildytis?” 
\par 8 Jis pasakė: “Žiūrėkite, kad nebūtumėte suklaidinti, nes daugelis ateis mano vardu, ir sakys: ‘Tai Aš!’ ir: ‘Atėjo metas!’ Neikite paskui juos! 
\par 9 O kai išgirsite apie karus ir maištus, nenusigąskite, nes visa tai turi pirmiau įvykti, bet dar negreit galas”. 
\par 10 Ir dar jiems sakė: “Tauta sukils prieš tautą ir karalystė prieš karalystę. 
\par 11 Įvairiose vietose bus didelių žemės drebėjimų, badmečių, marų, bus baisių įvykių ir didelių ženklų iš dangaus. 
\par 12 Bet prieš tai jie pakels prieš jus rankas ir dėl mano vardo jus persekios, tempdami į sinagogas ir kalėjimus, ves pas karalius ir valdytojus. 
\par 13 Tai bus jums proga liudyti. 
\par 14 Taigi įsidėkite sau į širdis iš anksto negalvoti, kaip ginsitės, 
\par 15 nes Aš jums duosiu tokios iškalbos ir išminties, kad nė vienas jūsų priešininkas negalės nei atsispirti, nei prieštarauti. 
\par 16 Jus išdavinės tėvai, broliai, giminės ir draugai; kai kuriuos iš jūsų jie žudys. 
\par 17 Būsite visų nekenčiami dėl mano vardo. 
\par 18 Tačiau nė plaukas nuo jūsų galvos nenukris. 
\par 19 Savo ištverme išlaikykite savo sielas”. 
\par 20 “Kai pamatysite Jeruzalę supamą kariuomenės, žinokite, jog prisiartino jos sunaikinimas. 
\par 21 Tada, kas bus Judėjoje, tebėga į kalnus, kas mieste­teišeina iš jo, kas apylinkėse­tenegrįžta. 
\par 22 Nes tai bus bausmės dienos, kad išsipildytų visa, kas parašyta. 
\par 23 Vargas nėščioms ir žindančioms tomis dienomis! Nes baisi nelaimė ir rūstybė ištiks šitą tautą. 
\par 24 Žmonės kris nuo kalavijo ašmenų ir bus išvaryti nelaisvėn į visas tautas, o Jeruzalę mindžios pagonys, kol baigsis pagonių laikai”. 
\par 25 “Bus ženklų saulėje, mėnulyje ir žvaigždėse, o žemėje blaškysis sielvarto slegiamos tautos, jūrai baisiai šniokščiant ir šėlstant. 
\par 26 Žmonės sustings iš baimės, laukdami to, kas turės ištikti pasaulį, nes dangaus galybės bus sudrebintos. 
\par 27 Tada jie išvys Žmogaus Sūnų, ateinantį debesyje su jėga ir didžia šlove. 
\par 28 Kai tai prasidės, atsitieskite ir pakelkite galvas, nes artėja jūsų atpirkimas”. 
\par 29 Ir Jis jiems pasakė palyginimą: “Stebėkite figmedį bei visus kitus medžius. 
\par 30 Kai jie ima sprogti, jūs patys matote ir žinote, kad vasara jau arti. 
\par 31 Taip pat pamatę visa tai vykstant, žinokite, kad arti yra Dievo karalystė. 
\par 32 Iš tiesų sakau jums: ši karta nepraeis, iki visa tai įvyks. 
\par 33 Dangus ir žemė praeis, o mano žodžiai nepraeis. 
\par 34 Saugokitės, kad jūsų širdys nebūtų apsunkintos nesaikingumo, girtavimo ir gyvenimo rūpesčių, kad toji diena neužkluptų jūsų netikėtai. 
\par 35 It žabangai ji užgrius visus žemės gyventojus. 
\par 36 Todėl visą laiką budėkite ir melskitės, kad būtumėt palaikyti vertais išvengti visko, kas įvyks, ir stoti prieš Žmogaus Sūnų”. 
\par 37 Taip Jėzus dienomis mokydavo šventykloje, o naktis praleisdavo vadinamajame Alyvų kalne. 
\par 38 Ir nuo ankstyvo ryto visi žmonės rinkdavosi Jo pasiklausyti šventykloje.



\chapter{22}


\par 1 Artėjo Neraugintos duonos šventė, vadinama Pascha. 
\par 2 Aukštieji kunigai ir Rašto žinovai ieškojo būdo nužudyti Jėzų, nes bijojo žmonių. 
\par 3 O šėtonas įėjo į Judą, vadinamą Iskarijotu, vieną iš dvylikos. 
\par 4 Tas nuėjęs tarėsi su aukštaisiais kunigais ir sargybos viršininkais, kaip Jį išduoti. 
\par 5 Šie apsidžiaugė ir sutarė duoti jam pinigų. 
\par 6 Judas pažadėjo ir ieškojo progos išduoti jiems Jėzų, miniai nematant. 
\par 7 Atėjo Neraugintos duonos diena, kada reikėjo pjauti Paschos avinėlį. 
\par 8 Jėzus pasiuntė Petrą ir Joną, liepdamas: “Eikite ir paruoškite mums valgyti Paschą”. 
\par 9 Jie paklausė: “Kur nori, kad paruoštume?” 
\par 10 Jis atsakė: “Štai jums įeinant į miestą, jus pasitiks žmogus, vandens ąsočiu nešinas. Sekite paskui jį iki tų namų, į kuriuos jis įeis, 
\par 11 ir sakykite namų šeimininkui: ‘Mokytojas liepė paklausti: Kur svečių kambarys, kuriame galėčiau su mokiniais valgyti Paschą?’ 
\par 12 Jis parodys jums didelį apstatytą aukštutinį kambarį. Ten ir paruoškite”. 
\par 13 Nuėję jie rado viską, kaip Jis sakė, ir paruošė Paschą. 
\par 14 Atėjus metui, Jis sėdo prie stalo, ir dvylika apaštalų drauge su Juo. 
\par 15 Ir Jis tarė jiems: “Trokšte troškau valgyti su jumis šią Paschą prieš kentėdamas. 
\par 16 Sakau jums, nuo šiol daugiau jos nebevalgysiu, kol ji išsipildys Dievo karalystėje”. 
\par 17 Paėmęs taurę, Jis padėkojo ir tarė: “Imkite ir dalykitės. 
\par 18 Sakau jums: Aš nebegersiu vynmedžio vaisiaus, kol ateis Dievo karalystė”. 
\par 19 Ir, paėmęs duoną, Jis padėkojo, laužė ją ir davė jiems, sakydamas: “Tai yra mano kūnas, kuris už jus atiduodamas. Tai darykite mano atminimui”. 
\par 20 Lygiai taip po vakarienės Jis paėmė taurę, sakydamas: “Ši taurė yra Naujoji Sandora mano kraujyje, kuris už jus išliejamas. 
\par 21 Bet štai mano išdavėjo ranka yra kartu su manąja ant stalo. 
\par 22 Žmogaus Sūnus, tiesa, eina, kaip Jam paskirta, bet vargas tam žmogui, kuris Jį išduoda!” 
\par 23 Tada jie pradėjo klausinėti vienas kito, kas gi iš jų yra tas, kuris tai padarys. 
\par 24 Tarp jų taip pat kilo ginčas, kuris iš jų turėtų būti laikomas didžiausiu. 
\par 25 Bet Jėzus jiems pasakė: “Pagonių karaliai viešpatauja jiems ir tie, kurie juos valdo, vadinami geradariais. 
\par 26 Jūs taip nedarykite. Kas didžiausias tarp jūsų, tebūnie tarsi mažiausias, o vadovaujantis tebūnie kaip tarnas. 
\par 27 Katras yra didesnis­kuris sėdi už stalo ar kuris jam patarnauja? Argi ne tas, kuris sėdi? O Aš tarp jūsų esu kaip tas, kuris patarnauja. 
\par 28 Jūs ištvėrėte su manimi mano išbandymuose, 
\par 29 todėl Aš jums skiriu valdyti karalystę, kaip ir man yra skyręs Tėvas, 
\par 30 kad mano karalystėje jūs valgytumėte ir gertumėte už mano stalo ir sėdėtumėte sostuose, teisdami dvylika Izraelio giminių”. 
\par 31 Ir Viešpats tarė: “Simonai, Simonai! Štai šėtonas prašė persijoti jus tarsi kviečius. 
\par 32 Bet Aš meldžiau už tave, kad tavo tikėjimas nepalūžtų, ir tu atsivertęs stiprink savo brolius!” 
\par 33 Petras atsiliepė: “Viešpatie, aš pasiruošęs kartu su Tavimi eiti ir į kalėjimą, ir į mirtį!” 
\par 34 Bet Jėzus atsakė: “Sakau tau, Petrai: dar šiandien, gaidžiui nepragydus, tu tris kartus išsiginsi, kad mane pažįsti”. 
\par 35 Jis paklausė juos: “Ar jums ko nors trūko, kai buvau jus išsiuntęs be piniginės, be krepšio ir be sandalų?” Jie atsakė: “Nieko”. 
\par 36 Tada Jis jiems tarė: “Dabar, kas turi piniginę, tepasiima ją, taip pat ir krepšį, o kas neturi kalavijo, teparduoda apsiaustą ir tenusiperka. 
\par 37 Sakau jums, manyje privalo išsipildyti, kas parašyta: ‘Jis buvo priskaitytas prie piktadarių’; tai, kas man nustatyta, jau eina į pabaigą”. 
\par 38 Jie tarė: “Viešpatie, štai čia du kalavijai!” Jis atsakė: “Gana!” 
\par 39 Išėjęs iš ten, Jis, kaip buvo pratęs, pasuko į Alyvų kalną. Paskui Jį nuėjo ir mokiniai. 
\par 40 Atėjus į vietą, Jis tarė jiems: “Melskitės, kad nepakliūtumėte į pagundymą!” 
\par 41 Ir Jis atsitolino nuo jų maždaug per akmens metimą ir atsiklaupęs ėmė melstis: 
\par 42 “Tėve, jei nori, atimk šitą taurę nuo manęs, tačiau tebūna ne mano, bet Tavo valia!” 
\par 43 Jam pasirodė iš dangaus angelas ir Jį sustiprino. 
\par 44 Vidinės kovos draskomas, Jis dar karščiau meldėsi. Jo prakaitas pasidarė tarsi tiršto kraujo lašai, varvantys žemėn. 
\par 45 Atsikėlęs po maldos, Jis atėjo pas mokinius ir rado juos iš sielvarto užmigusius. 
\par 46 Jis tarė jiems: “Kodėl miegate? Kelkitės ir melskitės, kad nepakliūtumėte į pagundymą!” 
\par 47 Jam dar tebekalbant, štai pasirodė minia, o priekyje­vienas iš dvylikos, vadinamas Judu. Jis priėjo prie Jėzaus Jo pabučiuoti. 
\par 48 Jėzus jam tarė: “Judai, pabučiavimu išduodi Žmogaus Sūnų?” 
\par 49 Esantieji su Juo, matydami, kas bus, paklausė: “Viešpatie, gal mums kirsti kalaviju?” 
\par 50 Vienas iš jų smogė vyriausiojo kunigo tarnui ir nukirto jam dešinę ausį. 
\par 51 Bet Jėzus sudraudė: “Liaukitės! Užteks!” Ir palietęs tarno ausį, išgydė jį. 
\par 52 Atėjusiems aukštiesiems kunigams, šventyklos apsaugos viršininkams ir vyresniesiems Jis pasakė: “Kaip prieš plėšiką išėjote prieš mane su kalavijais ir vėzdais. 
\par 53 Kai buvau kasdien su jumis šventykloje, jūs nepakėlėte prieš mane rankos. Bet ši valanda jūsų, tamsos valdžia”. 
\par 54 Suėmę Jėzų, jie nusivedė Jį ir atvedė į vyriausiojo kunigo namus. Petras sekė iš tolo. 
\par 55 Susikūrę ugnį kiemo viduryje, jie susėdo. Petras atsisėdo kartu. 
\par 56 Viena tarnaitė, pamačiusi jį sėdintį prieš šviesą, įsižiūrėjo ir tarė: “Ir šitas buvo kartu su Juo”. 
\par 57 Bet jis išsigynė, sakydamas: “Moterie, nepažįstu Jo!” 
\par 58 Truputį vėliau kažkas kitas, jį pamatęs, tarė: “Ir tu esi iš jų”. Petras atsakė: “Ne, žmogau, nesu!” 
\par 59 Maždaug po valandos dar vienas ėmė tvirtinti, sakydamas: “Tikrai šitas buvo su Juo! Juk jis galilėjietis!” 
\par 60 Petras atsakė: “Žmogau, aš nesuprantu, ką tu sakai”. Ir tuoj pat, dar jam kalbant, pragydo gaidys. 
\par 61 Tada Viešpats atsigręžęs pažvelgė į Petrą. Ir Petras atsiminė Viešpaties žodį, kaip Jis buvo jam pasakęs: “Dar gaidžiui nepragydus, tu tris kartus manęs išsiginsi”. 
\par 62 Petras išėjo laukan ir karčiai pravirko. 
\par 63 Jėzų saugantys vyrai tyčiojosi iš Jo ir mušė. 
\par 64 Uždengę akis, jie daužė Jam per veidą ir klausinėjo: “Pranašauk, kas Tave užgavo!” 
\par 65 Ir visaip kitaip jam piktžodžiavo. 
\par 66 Rytui išaušus, susirinko tautos vyresnieji, aukštieji kunigai ir Rašto žinovai. Jie atvedė Jėzų į savo sinedrioną ir sakė: 
\par 67 “Jei Tu Kristus, tai prisipažink mums!” Jėzus atsiliepė: “Jeigu jums ir pasakysiu, vis tiek netikėsite, 
\par 68 o jei paklausiu, man neatsakysite ir nepaleisite. 
\par 69 Tačiau nuo šio meto Žmogaus Sūnus sėdės Dievo Galybės dešinėje”. 
\par 70 Tuomet jie visi klausė: “Tai Tu esi Dievo Sūnus?” Jis atsakė: “Taip yra kaip sakote: Aš Esu!” 
\par 71 Tada jie tarė: “Kam dar mums liudijimas?! Juk mes girdėjome iš Jo paties lūpų!”



\chapter{23}


\par 1 Visas jų būrys pakilo ir nusivedė Jėzų pas Pilotą. 
\par 2 Ten jie ėmė Jį kaltinti, sakydami: “Mes nustatėme, kad šitas kiršina tautą ir draudžia mokėti ciesoriui mokesčius, tvirtindamas esąs Kristus ir karalius”. 
\par 3 Pilotas Jį paklausė: “Ar Tu esi žydų karalius?” Jėzus atsakė: “Taip yra, kaip sakai”. 
\par 4 Pilotas tarė aukštiesiems kunigams ir miniai: “Aš nerandu šitame žmoguje jokios kaltės”. 
\par 5 Bet jie visi atkakliai tvirtino: “Jis kursto tautą, mokydamas visoje Judėjoje, pradedant nuo Galilėjos iki čia”. 
\par 6 Pilotas, išgirdęs minint Galilėją, paklausė, ar tas žmogus galilėjietis. 
\par 7 Sužinojęs, kad Jėzus iš Erodo valdų, nusiuntė Jį pas Erodą, kuris irgi buvo tomis dienomis Jeruzalėje. 
\par 8 Erodas, išvydęs Jėzų, labai apsidžiaugė. Mat jis jau anksčiau troško Jį pamatyti, nes buvo daug apie Jį girdėjęs, ir tikėjosi išvysiąs Jį darant kokį nors stebuklą. 
\par 9 Jis pateikė Jėzui daug klausimų, bet Jis jam neatsakinėjo. 
\par 10 Tuo tarpu aukštieji kunigai ir Rašto žinovai apstoję be paliovos Jį kaltino. 
\par 11 Tada Erodas su savo kariais Jėzų paniekino ir išjuokė. Po to aprengė Jį šviesiu drabužiu ir pasiuntė atgal Pilotui. 
\par 12 Tą dieną Erodas ir Pilotas tapo draugais, o seniau jie pykosi. 
\par 13 Pilotas, sušaukęs aukštuosius kunigus, vyresniuosius ir minią, 
\par 14 pasakė jiems: “Jūs atvedėte man šitą žmogų, kaltindami Jį tautos kurstymu. Bet aš, Jį apklausęs jūsų akivaizdoje, neradau nė vienos Jam primetamos kaltės; 
\par 15 taip pat ir Erodas, nes aš buvau nusiuntęs jus pas jį. Taigi Jis nėra padaręs nieko, kas būtų verta mirties bausmės. 
\par 16 Aš tad Jį nuplakdinsiu ir paleisiu”. 
\par 17 Mat per šventę Pilotas turėjo paleisti jiems vieną kalinį. 
\par 18 Tada jie visi kartu ėmė šaukti: “Mirtis šitam! Paleisk mums Barabą!” 
\par 19 Barabas buvo pasodintas į kalėjimą už kažkokį maištą mieste ir žmogžudystę. 
\par 20 Norėdamas paleisti Jėzų, Pilotas vėl kreipėsi į juos, 
\par 21 bet jie nesiliovė šaukę: “Nukryžiuok Jį! Nukryžiuok!” 
\par 22 Jis trečią kartą prabilo į juos: “Ką bloga Jis padarė? Aš Jame nerandu nieko, už ką vertėtų bausti mirtimi. Taigi nuplakdinsiu Jį ir paleisiu”. 
\par 23 Tačiau jie, garsiai šaukdami, nesiliovė reikalauti, kad Jis būtų nukryžiuotas, ir jų bei aukštųjų kunigų šauksmas paėmė viršų. 
\par 24 Tuomet Pilotas nusprendė patenkinti jų reikalavimą. 
\par 25 Jis paleido jiems įkalintąjį už maištą ir žmogžudystę, kaip jie prašė, o Jėzų atidavė jų valiai. 
\par 26 Vesdami Jį, jie sulaikė Kirėnės gyventoją Simoną, grįžtantį iš laukų, ir uždėjo jam ant pečių kryžių, kad neštų jį paskui Jėzų. 
\par 27 Jį lydėjo didelė minia ir daug moterų, kurios raudojo ir aimanavo dėl Jo. 
\par 28 Atsigręžęs į jas, Jėzus tarė: “Jeruzalės dukros! Verkite ne dėl manęs, bet dėl savęs ir savo vaikų! 
\par 29 Nes štai ateina dienos, kai sakys: ‘Palaimintos nevaisingosios! Palaimintos įsčios, kurios negimdė, ir krūtys, kurios nežindė!’ 
\par 30 Tada sakys kalnams: ‘Griūkite ant mūsų!’ ir kalvoms: ‘Pridenkite mus!’ 
\par 31 Jeigu šitaip daro žaliam medžiui, tai kas gi laukia sauso?” 
\par 32 Kartu su Juo buvo vedami žudyti du nusikaltėliai. 
\par 33 Atėję į vietą, kuri vadinasi “Kaukolė”, jie nukryžiavo Jį ir du piktadarius­vieną iš dešinės, antrą iš kairės. 
\par 34 Jėzus tarė: “Tėve, atleisk jiems, nes jie nežino, ką daro”. O jie, mesdami burtą, pasidalijo Jo drabužius. 
\par 35 Žmonės stovėjo ir žiūrėjo. Vyresnieji kartu su kitais šaipydamiesi kalbėjo: “Kitus išgelbėdavo­tegul pats išsigelbsti, jei Jis­ Kristus, Dievo išrinktasis!” 
\par 36 Iš Jo tyčiojosi ir kareiviai, prieidami, paduodami Jam rūgštaus vyno 
\par 37 ir sakydami: “Jei Tu žydų karalius­išgelbėk save!” 
\par 38 Viršum Jo buvo užrašas graikų, lotynų ir hebrajų kalbomis: “Šitas yra žydų karalius”. 
\par 39 Vienas iš nukryžiuotųjų nusikaltėlių piktžodžiavo Jam: “Jei Tu esi Kristus, išgelbėk save ir mus!” 
\par 40 Antrasis sudraudė jį: “Ir Dievo tu nebijai, pats būdamas taip pat nuteistas! 
\par 41 Mudu teisingai gavome, ko verti mūsų darbai, o šitas nieko blogo nepadarė”. 
\par 42 Ir jis tarė Jėzui: “Viešpatie, prisimink mane, kai ateisi į savo karalystę”. 
\par 43 Jėzus jam atsakė: “Iš tiesų sakau tau: šiandien su manimi būsi rojuje”. 
\par 44 Buvo apie šeštą valandą, kai visoje šalyje pasidarė tamsu, ir taip buvo iki devintos valandos. 
\par 45 Saulė užtemo, ir šventyklos uždanga perplyšo pusiau. 
\par 46 Jėzus garsiu balsu sušuko: “Tėve, ‘į Tavo rankas pavedu savo dvasią’ ”. Ir tai pasakęs, atidavė dvasią. 
\par 47 Šimtininkas, matydamas, kas įvyko, ėmė garbinti Dievą ir tarė: “Iš tiesų šitas žmogus buvo teisusis!” 
\par 48 Ir visa minia, susirinkusi pažiūrėti reginio ir pamačiusi, kas įvyko, skirstėsi, mušdamasi į krūtinę. 
\par 49 Visi Jėzaus pažįstami ir moterys, Jį atlydėjusios iš Galilėjos, stovėjo atokiau ir viską matė. 
\par 50 Ir štai atėjo vienas vyras, vardu Juozapas, teismo tarybos narys, geras ir teisus žmogus, 
\par 51 kilęs iš žydų miesto Arimatėjos. Jis nesutiko su tarybos sprendimu ir poelgiu. Jis irgi laukė Dievo karalystės. 
\par 52 Taigi Juozapas nuėjo pas Pilotą ir paprašė Jėzaus kūno. 
\par 53 Nuėmęs Jį nuo kryžiaus, įvyniojo į drobulę ir paguldė uoloje iškaltame kape, kuriame dar niekas nebuvo laidotas. 
\par 54 Tai buvo Prisirengimo diena, jau beprasidedant sabatui. 
\par 55 Moterys, kurios buvo atėjusios su Jėzumi iš Galilėjos, atsekė iš paskos, stebėjo kapą ir matė, kaip buvo paguldytas Jo kūnas. 
\par 56 Sugrįžusios jos paruošė kvepalų ir tepalų, o per sabatą ilsėjosi, kaip reikalavo Įstatymas.



\chapter{24}


\par 1 Pirmąją savaitės dieną, vos brėkštant, jos atėjo prie kapo, nešdamos paruoštus tepalus. 
\par 2 Jos rado akmenį nuritintą nuo kapo, 
\par 3 o įėjusios vidun, neberado Viešpaties Jėzaus kūno. 
\par 4 Moterys apstulbo ir nežinojo, ką daryti. Ir štai prie jų atsirado du vyrai spindinčiais drabužiais. 
\par 5 Jos išsigando ir nuleido akis, o tie vyrai tarė: “Kam ieškote gyvojo tarp mirusiųjų? 
\par 6 Nėra Jo čia, Jis prisikėlė! Atsiminkite, ką Jis jums sakė, būdamas dar Galilėjoje: 
\par 7 ‘Žmogaus Sūnus turi būti atiduotas į nusidėjėlių rankas ir nukryžiuotas, o trečią dieną prisikelti!’ ” 
\par 8 Tuomet jos prisiminė Jėzaus žodžius 
\par 9 ir, sugrįžusios nuo kapo, viską pranešė vienuolikai ir visiems kitiems. 
\par 10 Tai buvo Marija Magdalietė, Joana, Jokūbo motina Marija ir kitos su jomis, kurios papasakojo tai apaštalams. 
\par 11 Jų žodžiai jiems pasirodė esą tuščios šnekos, ir jie moterimis nepatikėjo. 
\par 12 Vis dėlto Petras pakilęs nubėgo prie kapo ir pasilenkęs pamatė tiktai drobules. Jis grįžo atgal, labai stebėdamasis tuo, kas atsitiko. 
\par 13 Ir štai du iš jų tą pačią dieną keliavo į kaimą už šešiasdešimties stadijų nuo Jeruzalės, vadinamą Emausu. 
\par 14 Jie kalbėjosi apie visus tuos įvykius. 
\par 15 Jiems taip besikalbant ir besvarstant, prisiartino pats Jėzus ir ėjo kartu. 
\par 16 Jų akys buvo uždengtos, ir jie Jo neatpažino. 
\par 17 O Jis paklausė jų: “Apie ką kalbatės, eidami nuliūdę?” 
\par 18 Vienas iš jų, vardu Kleopas, atsakė Jam: “Nejaugi Tu esi vienintelis ateivis Jeruzalėje ir nežinai, kas joje šiomis dienomis atsitiko?” 
\par 19 Jėzus paklausė: “O kas gi?” Jie tarė Jam: “Su Jėzumi iš Nazareto, kuris buvo pranašas, galingas darbais ir žodžiais Dievo ir visos tautos akyse. 
\par 20 Aukštieji kunigai ir mūsų vadovai Jį pasmerkė mirti ir nukryžiavo. 
\par 21 O mes vylėmės, kad Jis yra Tas, kuris atpirks Izraelį. Dabar po viso to jau trečia diena, kaip tai atsitiko. 
\par 22 Be to, kai kurios mūsiškės moterys mus labai nustebino. Anksti rytą jos buvo nuėjusios pažiūrėti kapo 
\par 23 ir nerado Jo kūno. Jos sugrįžo ir papasakojo regėjusios angelus, kurie sakę Jį esant gyvą. 
\par 24 Kai kurie iš mūsiškių buvo nuėję prie kapo ir rado viską, kaip moterys sakė, bet Jo paties nematė”. 
\par 25 Tada Jis jiems tarė: “O jūs, neišmanėliai! Kokios nerangios jūsų širdys tikėti tuo, ką yra skelbę pranašai! 
\par 26 Argi ne taip turėjo Kristus kentėti ir įeiti į savo šlovę?” 
\par 27 Ir, pradėjęs nuo Mozės, primindamas visus pranašus, Jis aiškino jiems, kas visuose Raštuose apie Jį pasakyta. 
\par 28 Jie prisiartino prie kaimo, į kurį keliavo, ir Jis dėjosi einąs toliau. 
\par 29 Bet jie sulaikė Jį, sakydami: “Pasilik su mumis! Vakaras arti, diena jau baigiasi”. Jis užsuko ir pasiliko su jais. 
\par 30 Atsisėdęs su jais prie stalo, paėmė duoną, laimino, laužė ir davė jiems. 
\par 31 Tada jų akys atsivėrė, ir jie pažino Jėzų, bet Jis pranyko jiems iš akių. 
\par 32 O jie kalbėjo: “Argi mūsų širdys nedegė, kai Jis kelyje mums kalbėjo ir atvėrė Raštų prasmę?” 
\par 33 Jie tuoj pat pakilo ir sugrįžo į Jeruzalę. Ten jie rado susirinkusius vienuolika ir kitus su jais, 
\par 34 kurie tvirtino: “Viešpats tikrai prisikėlė ir pasirodė Simonui!” 
\par 35 O jie papasakojo, kas jiems atsitiko kelyje ir kaip jie pažino Jėzų, kai Jis laužė duoną. 
\par 36 Jiems apie tai bekalbant, Jis pats atsirado tarp jų ir tarė: “Ramybė jums!” 
\par 37 Sutrikę ir išsigandę, jie tarėsi matą dvasią. 
\par 38 O Jis paklausė: “Ko taip sutrikote? Kodėl jūsų širdyse kyla abejonės? 
\par 39 Pasižiūrėkite į mano rankas ir kojas. Juk tai Aš pats! Palieskite mane ir įsitikinsite: dvasia juk neturi kūno nei kaulų, kaip matote mane turint”. 
\par 40 Tai taręs, Jis parodė jiems rankas ir kojas. 
\par 41 Jiems iš džiaugsmo vis dar netikint ir stebintis, Jėzus paklausė: “Ar turite čia ko nors valgyti?” 
\par 42 Jie padavė Jam gabalą keptos žuvies ir korį medaus. 
\par 43 Jis paėmė ir valgė jų akyse. 
\par 44 Paskui Jėzus jiems tarė: “Ar ne tokie buvo mano žodžiai, kuriuos jums kalbėjau dar būdamas su jumis: turi išsipildyti visa, kas parašyta apie mane Mozės Įstatyme, Pranašuose ir Psalmėse”. 
\par 45 Tada Jis atvėrė jiems protą, kad jie suprastų Raštus, 
\par 46 ir pasakė: “Parašyta, kad Kristus kentės ir trečią dieną prisikels iš numirusių 
\par 47 ir, pradedant nuo Jeruzalės, Jo vardu visoms tautoms bus skelbiama atgaila ir nuodėmių atleidimas. 
\par 48 Jūs esate šių dalykų liudytojai. 
\par 49 Ir štai Aš atsiųsiu jums savo Tėvo pažadą. Jūs pasilikite Jeruzalės mieste, kol būsite apgaubti jėga iš aukštybių”. 
\par 50 Jėzus nusivedė juos iki Betanijos ir, iškėlęs rankas, palaimino juos. 
\par 51 Laimindamas Jis atsiskyrė nuo jų ir buvo paimtas į dangų. 
\par 52 Jie pagarbino Jį ir, didelio džiaugsmo kupini, sugrįžo į Jeruzalę. 
\par 53 Jie nuolat buvo šventykloje ir šlovino bei laimino Dievą. Amen.



\end{document}