\begin{document}

\title{Pirmasis Jono laiškas}


\chapter{1}


\par 1 Kas buvo nuo pradžios, ką girdėjome ir savo akimis regėjome, ką matėme ir mūsų rankos lietė,­tai skelbiame apie gyvenimo Žodį. 
\par 2 Gyvenimas pasirodė, ir mes regėjome ir liudijame, ir skelbiame jums šį amžinąjį gyvenimą, kuris buvo pas Tėvą ir pasirodė mums. 
\par 3 Ką matėme ir girdėjome, mes skelbiame jums, kad ir jūs turėtumėte bendravimą su mumis. O mūsų bendravimas yra su Tėvu ir su Jo Sūnumi Jėzumi Kristumi. 
\par 4 Ir tai rašome jums, kad jūsų džiaugsmas būtų tobulas. 
\par 5 Tai yra žinia, kurią išgirdome iš Jo ir skelbiame jums, kad Dievas yra šviesa ir Jame nėra jokios tamsybės. 
\par 6 Jei sakome, kad bendraujame su Juo, o vaikščiojame tamsoje,­ meluojame ir nevykdome tiesos. 
\par 7 O jei vaikščiojame šviesoje, kaip ir Jis yra šviesoje, mes bendraujame vieni su kitais, ir Jo Sūnaus Jėzaus Kristaus kraujas apvalo mus nuo visų nuodėmių. 
\par 8 Jei sakome, kad neturime nuodėmės,­klaidiname patys save, ir nėra mumyse tiesos. 
\par 9 Jeigu išpažįstame savo nuodėmes, Jis ištikimas ir teisingas, kad atleistų mums nuodėmes ir apvalytų mus nuo visų nedorybių. 
\par 10 Jei sakome, kad nesame nusidėję, darome Jį melagiu, ir nėra mumyse Jo žodžio.


\chapter{2}


\par 1 Mano vaikeliai, rašau jums tai, kad nenusidėtumėte. O jei kuris nusidėtų, tai mes turime Užtarėją pas Tėvą, teisųjį Jėzų Kristų. 
\par 2 Jis yra permaldavimas už mūsų nuodėmes, ir ne tik už mūsų, bet ir už viso pasaulio. 
\par 3 Iš to mes patiriame, kad Jį pažįstame, jei laikomės Jo įsakymų. 
\par 4 Kas sako: “Aš Jį pažįstu”, bet Jo įsakymų nesilaiko, tas melagis, ir nėra jame tiesos. 
\par 5 O kas laikosi Jo žodžių, tame iš tiesų Dievo meilė tobula tapo. Iš to ir pažįstame, jog Jame esame. 
\par 6 Kas sako esąs Jame, tas turi pats taip elgtis, kaip ir Jis elgėsi. 
\par 7 Broliai, aš jums nerašau naujo įsakymo, bet seną įsakymą, kurį turėjote nuo pradžios. Senas įsakymas yra žodis, kurį girdėjote nuo pradžios. 
\par 8 Ir vis dėlto rašau jums naują įsakymą, kuris tikras Jame ir jumyse, nes tamsa traukiasi, o tikroji šviesa jau šviečia. 
\par 9 Kas sakosi esąs šviesoje, o savo brolio nekenčia, tas dar tebėra tamsoje. 
\par 10 Kas myli savo brolį, tas pasilieka šviesoje, ir jame nėra nieko piktinančio. 
\par 11 O kas savo brolio nekenčia, tas yra tamsoje, vaikščioja tamsoje ir nežino, kur einąs, nes tamsa užgulė jam akis. 
\par 12 Rašau jums, vaikeliai, nes dėl Jo vardo atleistos jums nuodėmės. 
\par 13 Rašau jums, tėvai, nes pažinote Tą, kuris yra nuo pradžios. Ir jums, jaunuoliai, rašau, nes nugalėjote piktąjį. Rašau jums, vaikeliai, nes pažinote Tėvą. 
\par 14 Parašiau jums, tėvai, nes pažinote Tą, kuris yra nuo pradžios. Parašiau jums, jaunuoliai, nes jūs stiprūs ir jumyse laikosi Dievo žodis, ir jūs nugalėjote piktąjį. 
\par 15 Nemylėkite pasaulio, nei to, kas yra pasaulyje. Jei kas myli pasaulį, nėra jame Tėvo meilės, 
\par 16 nes visa, kas pasaulyje, tai kūno geismas, akių geismas ir gyvenimo išdidumas, o tai nėra iš Tėvo, bet iš pasaulio. 
\par 17 Praeina pasaulis ir jo geismai, bet kas vykdo Dievo valią, tas išlieka per amžius. 
\par 18 Vaikeliai, tai paskutinė valanda. Kaip esate girdėję, kad ateis antikristas, tai jau dabar pasirodė daug antikristų. Iš to sprendžiame, jog tai paskutinė valanda. 
\par 19 Jie yra išėję iš mūsų, tačiau nebuvo mūsiškiai. Jeigu jie būtų buvę mūsiškiai, jie būtų likę su mumis. Bet turėjo paaiškėti, jog ne visi yra mūsiškiai. 
\par 20 Bet jūs turite Šventojo patepimą ir žinote viską. 
\par 21 Aš parašiau jums ne kaip nežinantiems tiesos, bet todėl, kad pažįstate ją, o iš tiesos negali kilti joks melas. 
\par 22 Kas yra melagis, jeigu ne tas, kuris neigia, kad Jėzus yra Pateptasis? Tas yra antikristas, kuris neigia Tėvą ir Sūnų. 
\par 23 Kiekvienas, kas neigia Sūnų, neturi ir Tėvo. Kas išpažįsta Sūnų, tas turi ir Tėvą. 
\par 24 Todėl tepasilieka jumyse tai, ką girdėjote nuo pradžios. Jeigu tai, ką girdėjote nuo pradžios, pasilieka jumyse, tada ir jūs pasiliksite Sūnuje ir Tėve. 
\par 25 Štai pažadas, kurį Jis pats mums davė,­amžinasis gyvenimas. 
\par 26 Aš tai parašiau apie tuos, kurie jus klaidina. 
\par 27 Bet patepimas, kurį iš Jo gavote, pasilieka jumyse, ir nereikia, kad jus kas nors mokytų, nes pats Jo patepimas moko jus visko, ir jis yra tiesa, o ne melas. Ir kadangi jis jus pamokė, jūs Jame pasiliksite. 
\par 28 Taigi dabar, vaikeliai, pasilikite Jame, kad, kai Jis pasirodys, turėtumėte pasitikėjimą ir, kai Jis ateis, nebūtume prieš Jį sugėdinti. 
\par 29 Jei žinote, kad Jis teisus, tai žinokite, kad kiekvienas, kuris vykdo teisumą, iš Jo yra gimęs.


\chapter{3}


\par 1 Žiūrėkite, kokia meile apdovanojo mus Tėvas: mes vadinamės Dievo vaikai­ir esame! Pasaulis nepažįsta mūsų, nes ir Jo nepažino. 
\par 2 Mylimieji, dabar mes esame Dievo vaikai, bet dar nepasirodė, kas būsime. Mes žinome, kad, kai Jis pasirodys, būsime panašūs į Jį, nes matysime Jį tokį, koks Jis yra. 
\par 3 Kiekvienas, kas turi Jame tokią viltį, skaistina pats save, nes ir Jis yra skaistus. 
\par 4 Kiekvienas, kuris daro nuodėmę, laužo įstatymą. Nuodėmė­ tai įstatymo laužymas. 
\par 5 Jūs žinote, jog Jis pasirodė, kad sunaikintų mūsų nuodėmes, ir Jame nėra nuodėmės. 
\par 6 Kas pasilieka Jame, tas nenusideda. Kiekvienas nuodėmiaujantis Jo neregėjo ir nepažino. 
\par 7 Vaikeliai! Tegul niekas jūsų nesuklaidina! Vykdantis teisumą yra teisus, kaip ir Jis teisus. 
\par 8 Kas daro nuodėmę, tas iš velnio, nes velnias nuodėmiauja nuo pat pradžios. Todėl ir pasirodė Dievo Sūnus, kad sugriautų velnio darbus. 
\par 9 Kas yra gimęs iš Dievo, nedaro nuodėmės, nes jame laikosi Dievo sėkla. Jis negali nuodėmiauti, nes yra gimęs iš Dievo. 
\par 10 Taip išaiškėja Dievo vaikai ir velnio vaikai: tas, kuris nevykdo teisumo, nėra iš Dievo; taip pat tas, kuris savo brolio nemyli. 
\par 11 Tokia yra žinia, kurią girdėjote nuo pradžios: mes turime mylėti vieni kitus. 
\par 12 Ne kaip Kainas, kuris buvo iš piktojo ir nužudė savo brolį. Kodėl nužudė? Todėl, kad jo darbai buvo pikti, o brolio­teisūs. 
\par 13 Nesistebėkite, broliai, jei pasaulis jūsų nekenčia. 
\par 14 Mes žinome, jog iš mirties perėjome į gyvenimą, nes mylime brolius. Kas nemyli savo brolio, tas pasilieka mirtyje. 
\par 15 Kiekvienas, kas nekenčia savo brolio, yra žmogžudys, o jūs žinote, kad joks žmogžudys neturi amžinojo gyvenimo, jame pasiliekančio. 
\par 16 Mes iš to pažinome meilę, kad Jis už mus paguldė savo gyvybę. Ir mes turime guldyti savo gyvybę už brolius. 
\par 17 Bet jei kas turi šio pasaulio turtų ir, matydamas savo brolį stokojantį, užrakina jam savo širdį,­ kaip jame pasiliks Dievo meilė? 
\par 18 Mano vaikeliai, nemylėkime žodžiu ar liežuviu, bet darbu ir tiesa. 
\par 19 Tuo mes pažįstame, jog esame iš tiesos, ir Jo akivaizdoje nuraminame savo širdį. 
\par 20 Jei mūsų širdis mus smerkia, Dievas didesnis už mūsų širdį ir žino viską. 
\par 21 Mylimieji, jei mūsų širdis mūsų nesmerkia, pasitikime Dievu 
\par 22 ir gauname iš Jo, ko tik prašome, nes laikomės Jo įsakymų ir darome, kas Jam patinka. 
\par 23 O štai Jo įsakymas: kad tikėtume Jo Sūnaus Jėzaus Kristaus vardą ir mylėtume vieni kitus, kaip Jo įsakyta. 
\par 24 Kas laikosi Jo įsakymų, pasilieka Jame ir Jis tame. O kad Jis mumyse pasilieka, žinome iš Dvasios, kurią Jis mums davė.


\chapter{4}


\par 1 Mylimieji, ne kiekviena dvasia tikėkite, bet ištirkite dvasias, ar jos iš Dievo, nes pasklido pasaulyje daug netikrų pranašų. 
\par 2 Iš to pažinsite Dievo Dvasią: kiekviena dvasia, kuri išpažįsta Jėzų Kristų kūne atėjusį, yra iš Dievo, 
\par 3 ir kiekviena dvasia, kuri neišpažįsta Jėzaus Kristaus kūne atėjusio, nėra iš Dievo. Tokia­iš antikristo, apie kurį girdėjote, kad jis ateisiąs. Jis jau dabar yra pasaulyje. 
\par 4 Jūs esate iš Dievo, vaikeliai, ir nugalėjote juos, nes Tas, kuris jumyse, didesnis už tą, kuris pasaulyje. 
\par 5 Jie yra iš pasaulio, todėl kalba kaip iš pasaulio, ir pasaulis jų klauso. 
\par 6 Mes esame iš Dievo. Kas pažįsta Dievą, tas mūsų klauso, o kas ne iš Dievo­mūsų neklauso. Iš to pažįstame tiesos Dvasią ir klaidos dvasią. 
\par 7 Mylimieji, mylėkime vieni kitus, nes meilė yra iš Dievo. Kiekvienas, kuris myli, yra gimęs iš Dievo ir pažįsta Dievą. 
\par 8 Kas nemyli, tas nepažįsta Dievo, nes Dievas yra meilė. 
\par 9 O Dievo meilė pasireiškė mums tuo, jog Dievas atsiuntė į pasaulį savo viengimį Sūnų, kad gyventume per Jį. 
\par 10 Meilė­ne tai, jog mes pamilome Dievą, bet kad Jis mus pamilo ir atsiuntė savo Sūnų kaip permaldavimą už mūsų nuodėmes. 
\par 11 Mylimieji, jei Dievas mus taip pamilo, tai ir mes turime mylėti vieni kitus. 
\par 12 Dievo niekas niekada nėra matęs. Jei mylime vieni kitus, Dievas mumyse pasilieka, ir Jo meilė mumyse tobula tampa. 
\par 13 Iš to pažįstame, kad pasiliekame Jame ir Jis mumyse: Jis davė mums savo Dvasios. 
\par 14 Taigi mes matėme ir liudijame, kad Tėvas atsiuntė Sūnų, pasaulio Gelbėtoją. 
\par 15 Kiekvienas, kas išpažįsta, kad Jėzus yra Dievo Sūnus, Dievas tame ir tas Dieve pasilieka. 
\par 16 Mes pažinome ir įtikėjome meilę, kuria Dievas mus myli. Dievas yra meilė, ir kas pasilieka meilėje, tas pasilieka Dieve, ir Dievas jame. 
\par 17 Tuo meilė pasiekia mumyse tobulumą, kad galime turėti drąsų pasitikėjimą teismo dieną, nes koks Jis yra, tokie ir mes esame šiame pasaulyje. 
\par 18 Meilėje nėra baimės, nes tobula meilė išveja baimę. Baimėje yra kančia, ir kas bijo, tas nėra tobulas meilėje. 
\par 19 Mes mylime Jį, nes Jis mus pirmas pamilo. 
\par 20 Jei kas sako: “Aš myliu Dievą”, o savo brolio nekenčia,­tas melagis. Kas nemyli savo brolio, kurį mato, kaip gali mylėti Dievą, kurio nemato? 
\par 21 Mes turime tokį Jo įsakymą: kas myli Dievą, turi mylėti ir savo brolį.


\chapter{5}


\par 1 Kiekvienas, kas tiki, jog Jėzus yra Pateptasis, yra gimęs iš Dievo, ir kiekvienas, kuris myli Gimdytoją, myli ir iš Jo gimusį. 
\par 2 Iš to pažįstame mylį Dievo vaikus, kad mylime Dievą ir laikomės Jo įsakymų. 
\par 3 Nes tai yra Dievo meilė­Jo įsakymus vykdyti. O Jo įsakymai nėra sunkūs. 
\par 4 Juk, kas tik gimė iš Dievo, nugali pasaulį; ir štai pergalė, nugalinti pasaulį­mūsų tikėjimas! 
\par 5 O kas gi nugali pasaulį, jei ne tas, kuris tiki, kad Jėzus yra Dievo Sūnus? 
\par 6 Jis yra Tas, kuris atėjo per vandenį ir kraują,­Jėzus Kristus; ne vien per vandenį, bet per vandenį ir kraują. Ir Dvasia tai liudija, nes Dvasia yra tiesa. 
\par 7 Mat yra trys liudytojai danguje: Tėvas, Žodis ir Šventoji Dvasia; ir šitie trys yra viena. 
\par 8 Ir yra trys liudytojai žemėje: Dvasia, vanduo ir kraujas; ir šie trys sutaria kaip vienas. 
\par 9 Jeigu priimame žmonių liudijimą, tai Dievo liudijimas didesnis. O Dievo liudijimas toks: Jis paliudijo apie savo Sūnų. 
\par 10 Kas tiki Dievo Sūnų, tas turi liudijimą savyje. Kas netiki Dievu, tas Jį melagiu laiko, nes nepatikėjo liudijimu, kuriuo Dievas paliudijo apie savo Sūnų. 
\par 11 O liudijimas toks: Dievas mums suteikė amžinąjį gyvenimą, ir tas gyvenimas yra Jo Sūnuje. 
\par 12 Kas turi Sūnų, turi gyvenimą. Kas neturi Dievo Sūnaus, tas neturi gyvenimo. 
\par 13 Tai parašiau jums, tikintiems Dievo Sūnaus vardą, kad žinotumėte turį amžinąjį gyvenimą ir kad tikėtumėte Dievo Sūnaus vardą. 
\par 14 Ir štai kokį pasitikėjimą mes turime Juo: jei ko tik prašome pagal Jo valią, Jis girdi mus. 
\par 15 O jeigu žinome, kad Jis girdi mus, ko tik prašome, tai ir žinome, kad turime tai, ko Jo prašėme. 
\par 16 Jei kas mato nusidedant savo brolį, tačiau ne iki mirčiai, teprašo, ir Dievas duos jam gyvybę, būtent tiems, kurie nusideda ne iki mirčiai. Yra nuodėmė iki mirčiai, ir aš kalbu ne apie ją, kad būtų prašoma. 
\par 17 Kiekviena neteisybė yra nuodėmė, tačiau esama nuodėmės ne iki mirčiai. 
\par 18 Mes žinome, jog kiekvienas, gimęs iš Dievo, nenusideda. Kas gimęs iš Dievo, saugo save, ir piktasis jo nepaliečia. 
\par 19 Mes žinome, jog esame iš Dievo, o visas pasaulis yra piktojo. 
\par 20 Ir mes žinome, kad Dievo Sūnus atėjo ir suteikė mums supratimo, kad pažintume Tikrąjį. Ir mes esame Tikrajame­Jo Sūnuje Jėzuje Kristuje. Šis yra tikrasis Dievas ir amžinasis gyvenimas. 
\par 21 Vaikeliai, saugokitės stabų! Amen.



\end{document}