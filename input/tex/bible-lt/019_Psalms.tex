\begin{document}

\title{Psalmynas}


\chapter{1}


\par 1 Palaimintas žmogus, kuris nesielgia, kaip pataria bedieviai, nestoja į nusidėjėlių kelią, nesėdi su apjuokėjais, 
\par 2 bet mėgsta Viešpaties įstatymą ir mąsto apie Jo įstatymą dieną ir naktį. 
\par 3 Jis bus kaip medis, prie upelio pasodintas, kuris, metui atėjus, duoda derlių ir jo lapai nevysta; ką jis bedarytų, jam sekasi. 
\par 4 Ne tokie yra bedieviai. Jie kaip pelai, sklaidomi vėjo. 
\par 5 Todėl teisme neišstovės bedieviai, nė nusidėjėliai teisiųjų susirinkime. 
\par 6 Nes Viešpats žino teisiojo kelią, o bedievių kelias pražus.


\chapter{2}


\par 1 Kodėl pagonys siaučia ir tautos tuščias užmačias rezga? 
\par 2 Sukyla žemės karaliai, valdovai sąmokslus rengia prieš Viešpatį ir Jo pateptąjį. 
\par 3 Jie sako: “Sutraukykime jų pančius, nusimeskime jų virves”. 
\par 4 Tas, kuris danguje sėdi, juoksis, Viešpats tyčiosis iš jų. 
\par 5 Tada Jis rūsčiai prabils ir išgąsdins juos savo įniršiu: 
\par 6 “Aš pastačiau savo karalių Sione, savo šventajame kalne”. 
\par 7 Aš paskelbsiu nutarimą, kurį Viešpats man pasakė: “Tu­mano Sūnus, šiandien Tave pagimdžiau. 
\par 8 Prašyk, ir duosiu Tau paveldėti pagonis, pavesiu Tau visus žemės pakraščius, 
\par 9 geležine lazda juos tramdysi, kaip molio indus juos daužysi”. 
\par 10 Taigi dabar, karaliai, būkite išmintingi, pasimokykite, žemės teisėjai. 
\par 11 Tarnaukite Viešpačiui su baime, džiaukitės drebėdami. 
\par 12 Bučiuokite Sūnų, kad Jis nerūstautų ir nežūtumėte kelyje, Jo rūstybei staiga užsidegus. Palaiminti visi, kurie Juo pasitiki.



\chapter{3}


\par 1 Viešpatie, kiek daug yra mane varginančių, daug tų, kurie sukyla prieš mane. 
\par 2 Apie mane daugelis kalba: “Nėra jam pagalbos Dieve”. 
\par 3 Bet Tu, Viešpatie, esi mano skydas ir mano šlovė. Tu pakeli mano galvą. 
\par 4 Aš Viešpaties garsiai šaukiausi, ir Jis išgirdo nuo savo šventojo kalno. 
\par 5 Aš atsiguliau ir užmigau, ir vėl pabudau, nes Viešpats mane palaikė. 
\par 6 Nebijosiu dešimčių tūkstančių žmonių, kurie sustoja aplinkui mane. 
\par 7 Viešpatie, kelkis, gelbėk mane, mano Dieve! Tu smogei mano priešams į žiauną, sutrupinai bedieviams dantis. 
\par 8 Viešpatyje yra išgelbėjimas! Palaimink savąją tautą!



\chapter{4}


\par 1 Kai šaukiuosi Tavęs, išklausyk, mano teisumo Dieve! Tu išlaisvinai mane, kai buvau suspaustas. Būk man gailestingas, išgirsk mano maldą! 
\par 2 O žmonės, kaip ilgai niekinsite mano garbę? Ar ilgai mylėsite tuštybę ir ieškosite melo? 
\par 3 Žinokite, kad Viešpats sau atskyrė šventąjį. Viešpats išgirsta, kai Jo šaukiuosi. 
\par 4 Rūstaudami nenusidėkite; svarstykite savo širdyje gulėdami lovose ir nusiraminkite. 
\par 5 Aukokite teisumo aukas ir Viešpačiu pasitikėkite. 
\par 6 Daugelis sako: “Kas parodys mums gera?” Pakelk virš mūsų, Viešpatie, savo šviesų veidą! 
\par 7 Tu davei mano širdžiai daugiau linksmybės negu jiems, kai jie turi gausiai javų ir vyno. 
\par 8 Atsigulu ramus ir užmiegu, nes Tu vienintelis, Viešpatie, leidi man saugiai gyventi.



\chapter{5}


\par 1 Viešpatie, išgirsk mano žodžius, suprask mano apmąstymus. 
\par 2 Išklausyk mano šauksmą, mano Karaliau ir Dieve, nes Tau aš melsiuosi. 
\par 3 Rytą Tu girdi mano balsą, Viešpatie, rytą kreipiuosi į Tave ir laukiu. 
\par 4 Tu ne toks esi, Dieve, kuriam nedorybė patiktų, ir pikta negyvens su Tavimi. 
\par 5 Pagyrūnai negali stovėti Tavo akivaizdoje. Tu nekenti visų, darančių neteisybę. 
\par 6 Tu sunaikinsi tuos, kurie kalba melą; Viešpats bjaurisi kraugeriu ir apgaviku. 
\par 7 Dėl didžio Tavo gailestingumo einu į Tavo buveinę; Tavęs bijodamas, žemai lenkiuosi šventyklos link. 
\par 8 Viešpatie, vesk mane savo teisume dėl mano priešų. Nutiesk prieš mane savo kelią. 
\par 9 Jų lūpose nėra teisybės, jų širdyje­nedorybė, jų gerklė­atviras kapas, jie pataikauja savo liežuviais. 
\par 10 Sunaikink juos, Dieve, tepražūna patys savo kėsluose. Dėl nesuskaitomų piktadarybių atmesk juos, nes jie sukilo prieš Tave. 
\par 11 Tesidžiaugia visi, kurie Tavyje prieglobsčio ieško, tegul nuolat šaukia iš džiaugsmo, nes Tu apgini juos, tedžiūgauja Tavyje Tavo vardą mylintys. 
\par 12 Tu, Viešpatie, laimini teisųjį, apsupi jį savo malone lyg skydu.



\chapter{6}


\par 1 Užsirūstinęs, Viešpatie, nebark manęs, nebausk įširdęs. 
\par 2 Viešpatie, pasigailėk manęs, nes esu silpnas; Viešpatie, išgydyk mane, nes sukrėsti mano kaulai. 
\par 3 Ir mano siela sukrėsta. Ar ilgai, Viešpatie? 
\par 4 Viešpatie, gręžkis, išlaisvink mano sielą, gelbėk mane dėl savo gailestingumo. 
\par 5 Kas gi prisimins Tave mirtyje? Kas dėkos Tau kape? 
\par 6 Nuo aimanų suvargau, kasnakt aptvindau savo lovą, laistau ašaromis savo guolį. 
\par 7 Aptemo nuo vargo mano akys, paseno dėl visų mano priešų. 
\par 8 Piktadariai, atsitraukite nuo manęs, nes Viešpats išgirdo mano verksmo balsą! 
\par 9 Viešpats išgirdo mano maldavimą, Viešpats priims manąją maldą. 
\par 10 Tegul visi mano priešai susigėsta ir išsigąsta, tegu jie atsitraukia ir susigėsta staiga.



\chapter{7}


\par 1 Viešpatie, mano Dieve, Tavimi aš pasitikiu. Gelbėk mane ir išlaisvink nuo visų mano persekiotojų, 
\par 2 kad manęs jie nepagriebtų kaip liūtas, kuris drasko į gabalus, kai nėra kam gelbėti. 
\par 3 Viešpatie, mano Dieve, jei aš tai padariau, jei neteisybė yra mano rankose, 
\par 4 jei piktu atlyginau tam, kuris buvo taikoje su manimi (juk aš išlaisvinau tą, kuris be priežasties yra mano priešas), 
\par 5 tegu priešas persekioja mano sielą, mano gyvybę į žemę sutrypia, mano garbę dulkėmis paverčia. 
\par 6 Viešpatie, pakilk savo rūstybėje, stokis prieš mano priešų siautimą, pabusk dėl manęs ir teisk, kaip žadėjai. 
\par 7 Tegul aplinkui Tave susiburia tautos, Tu jų akivaizdoje sėskis į sostą. 
\par 8 Viešpats teis tautas. Teisk mane, Viešpatie, pagal mano teisumą, pagal nekaltumą, kuris yra manyje. 
\par 9 Padaryk nedorėlių užmačioms galą, o teisųjį sutvirtink. Tu ištiri širdis ir inkstus, teisusis Dieve! 
\par 10 Mano apsauga yra nuo Dievo, kuris išgelbsti tiesiaširdžius. 
\par 11 Dievas­teisingas teisėjas, Dievas kasdien rūstinasi ant nedorėlių. 
\par 12 Neatgailaujančiam Jis kardą galanda, įtempia lanką, taiko; 
\par 13 Jis paruošė jam mirties įrankius, nukreipė į jį ugnines strėles. 
\par 14 Jis pradėjo neteisybę, pastojo piktais sumanymais, pagimdė melą. 
\par 15 Jis padarė ir iškasė duobę, bet pats įgarmėjo į skylę, kurią paruošė. 
\par 16 Ant jo galvos sugrįš jo pikti sumanymai, jo smurtas kris jam ant viršugalvio. 
\par 17 Aš girsiu Viešpatį už Jo teisumą, giedosiu gyrių aukščiausiojo Viešpaties vardui.



\chapter{8}


\par 1 Viešpatie, mūsų Valdove, koks įstabus Tavo vardas visoje žemėje! Tu iškėlei savo šlovę virš dangų. 
\par 2 Kūdikių ir žindomųjų lūpomis Tu paskelbei apie savo jėgą savo priešams, kad nutildytum priešą ir keršytoją. 
\par 3 Kai pasižiūriu į Tavo dangus, Tavo rankų darbą, į mėnulį ir žvaigždes, kurias Tu išdėstei, 
\par 4 kas yra žmogus, kad jį atsimeni, ir kas žmogaus sūnus, kad jį aplankai? 
\par 5 Jį padarei ne ką menkesnį už angelus, garbe ir šlove jį apvainikavai. 
\par 6 Tu davei jam valdžią Tavo rankų darbams, jam po kojų visa padėjai: 
\par 7 visas avis ir jaučius, net lauko žvėris, 
\par 8 padangių paukščius ir jūrų žuvis, ir visa, kas jūros takais plaukioja. 
\par 9 Viešpatie, mūsų Valdove, koks įstabus Tavo vardas visoje žemėje!



\chapter{9}


\par 1 Aš girsiu Tave, Viešpatie, visa savo širdimi, skelbsiu visus Tavo nuostabius darbus. 
\par 2 Linksminsiuosi ir džiūgausiu Tavyje, Tavo vardui, Aukščiausiasis, giedosiu gyrių. 
\par 3 Traukiasi atgal mano priešai, klumpa ir žūva Tavo akivaizdoje, 
\par 4 nes Tu gini mano teises ir mano bylą, Tu sėdi soste, teisingai teisdamas. 
\par 5 Pagonis Tu subarei, sužlugdei nedoruosius, jų vardą visiems amžiams ištrynei. 
\par 6 Priešų neliko, jie tapo amžinais griuvėsiais; Tu sugriovei jų miestus, ir jų nebemini niekas. 
\par 7 Bet Viešpats pasiliks per amžius, Jis paruošė savo sostą teismui. 
\par 8 Jis teis pasaulį teisingai, vykdys teisingumą tautoms bešališkai. 
\par 9 Viešpats­priebėga prispaustiesiems, priebėga nelaimės metu. 
\par 10 Kas pažįsta Tavo vardą, pasitiki Tavimi. Tu niekad nepalikai tų, kurie Tavęs ieško. 
\par 11 Giedokite Viešpačiui, kuris gyvena Sione, skelbkite tautoms Jo darbus. 
\par 12 Jis­nekalto kraujo gynėjas­atsimena juos, neužmiršta vargingųjų šauksmo. 
\par 13 Viešpatie, pasigailėk, pažvelk, kaip vargstu nuo manęs nekenčiančiųjų. Tu iškeli mane iš mirties vartų, 
\par 14 kad galėčiau Tave šlovint prie Siono dukters vartų, besidžiaugdamas Tavo pagalba. 
\par 15 Į savo išraustą duobę pagonys įkrinta, jų koja įkliūna į spąstus, pačių pastatytus. 
\par 16 Viešpats apsireiškia, teismą surengia. Nedorėlis įsipainioja į savo rankų darbą. 
\par 17 Nedorėliai ir visos tautos, kurios užmiršo Dievą, į mirusiųjų buveinę eina. 
\par 18 Varguolis nebus pamirštas visam laikui, vargšo viltis nepradings amžiams. 
\par 19 Viešpatie, pakilk, tegul neįsigali žmogus, tebūna suruoštas teismas pagonims Tavo akivaizdoje. 
\par 20 Įvaryk jiems, Viešpatie, baimės; pagonys težino, kad jie­tiktai žmonės!



\chapter{10}


\par 1 Viešpatie, kodėl stovi toli, kodėl nelaimės metu slepiesi? 
\par 2 Bedievis didžiuojasi ir persekioja vargšą; tepakliūna jie į pinkles, savo sumanytas. 
\par 3 Nedorėlis giriasi savo širdies pageidimais, gobšuolis didžiuojasi ir niekina Viešpatį. 
\par 4 Nedorėlis išdidžiu veidu neieško Dievo, nėra Dievo jo mintyse. 
\par 5 Jam viskas sekasi, per toli nuo jo Tavo sprendimai, jis visus savo priešus niekais laiko. 
\par 6 Jis tarė savo širdyje: “Niekas manęs nepajudins, niekada manęs neištiks nelaimė”. 
\par 7 Jo burna pilna keiksmų, smurto, apgaulės; po jo liežuviu­nelaimė ir tuštybė. 
\par 8 Tyko prie kelio kur įlindęs, iš pasalų nekaltąjį žudo. Jis seka akimis varguolį, 
\par 9 tūno pasislėpęs kaip liūtas tankynėje, vargšą kėsinasi sugauti; jis sugauna vargšą ir įtraukia į savo tinklą. 
\par 10 Pasilenkia, atsigula, vargšai krinta į jo galingus nagus. 
\par 11 Jis tarė savo širdyje: “Dievas pamiršo, Jis nusigręžė, nieko nematys”. 
\par 12 Viešpatie, pakilk, Dieve, pakelk savo ranką, neužmiršk vargdienių! 
\par 13 Ko gi nedorėlis prieš Dievą burnoja? Jis tarė savo širdyje: “Tu nepareikalausi”. 
\par 14 Tu matai tai, nes žiūri į vargą ir skausmą, kad atlygintum savo ranka. Vargšas atsiduoda Tau, Tu esi našlaičių globėjas. 
\par 15 Sulaužyk ranką bedieviui ir nedorėliui atlygink už jo nedorumą, kol neliks iš jo nieko. 
\par 16 Viešpats yra Karalius per amžių amžius, iš Jo žemės pagonys išnyks. 
\par 17 Nuolankiųjų troškimą Tu girdi, sustiprini jų širdis, atkreipi savo ausį, 
\par 18 kad gintum našlaičio ir prispaustojo teises, kad žemės žmogus daugiau nebesiautėtų.



\chapter{11}


\par 1 Viešpačiu aš pasitikiu. Kaip jūs sakote mano sielai: “Skrisk kaip paukštis į kalnus”? 
\par 2 Įtempia nedorėliai lanką, prie templės strėlę jau deda, kad tamsoje šaudytų į tiesiaširdžius. 
\par 3 Kai pamatai griaunami, ką gi begali teisusis? 
\par 4 Viešpats savo šventykloje; danguje stovi Viešpaties sostas. Jo akys stebi, žmonių vaikus jos tyrinėja. 
\par 5 Viešpats teisųjį tiria, o nedorėliu ir smurtininku Jis bjaurisi. 
\par 6 Jis lydins ant bedievių žarijomis, ugnimi ir siera, jų lemtis bus svilinanti vėtra. 
\par 7 Viešpats teisus ir Jam miela teisybė. Dorieji regės Jo veidą.



\chapter{12}


\par 1 Viešpatie, gelbėk! Nyksta dievotieji, nebelieka ištikimųjų tarp žmonių vaikų. 
\par 2 Jie vienas kitam kalba tuštybes, lūpomis pataikauja ir kalba klastinga širdimi. 
\par 3 Viešpats sunaikins pataikaujančias lūpas, puikybės pilną liežuvį. 
\par 4 Jie sako: “Savo liežuviu mes nugalėsime, mūsų lūpos kalba už mus, kas mums Viešpats?” 
\par 5 “Dėl varguolio priespaudos, dėl vargšo dejonių dabar Aš pakilsiu,­sako Viešpats,­išgelbėsiu tą, kuris ilgisi mano pagalbos”. 
\par 6 Viešpaties žodžiai­tyri žodžiai, kaip liejykloje nuskaistintas sidabras, septynis kartus išvalytas. 
\par 7 Tu, Viešpatie, prižiūrėsi juos ir saugosi nuo šios kartos per amžius. 
\par 8 Visuose pakraščiuose gausu nedorėlių, kai išaukštinami niekam tikę žmonės.



\chapter{13}


\par 1 Ar amžinai? Ar ilgai slėpsi nuo manęs savo veidą? 
\par 2 Ar dar ilgai mane spaus liūdnos mintys, širdį skaudės kas dieną? Ar dar ilgai mano priešas didžiuosis prieš mane? 
\par 3 Pažvelk, išklausyk mane, Viešpatie, mano Dieve! Apšviesk man akis, kad mirties miegu neužmigčiau. 
\par 4 Kad mano priešas nesakytų: “Aš nugalėjau jį”. Kad nesidžiaugtų mano prispaudėjai, man susvyravus. 
\par 5 Aš pasitikėjau Tavo gailestingumu. Mano širdis džiaugiasi Tavo išgelbėjimu. 
\par 6 Giedosiu Viešpačiui, kuris man daro gera.



\chapter{14}


\par 1 Kvailys pasakė savo širdyje: “Nėra Dievo”. Jie sugedo, elgiasi bjauriai, nėra, kas gera darytų. 
\par 2 Viešpats pažiūrėjo į žmones iš dangaus, kad pamatytų, ar yra kas išmano ir ieško Dievo. 
\par 3 Jie visi nuklydo, visi kartu sugedo. Nėra darančio gera, nėra nė vieno. 
\par 4 Ar nesupranta piktadariai, kurie mano tautą ryja kaip duoną ir nesišaukia Viešpaties? 
\par 5 Jie buvo labai išgąsdinti, nes su teisiaisiais yra Dievas. 
\par 6 Jūs laikote nieku vargšą, bet jo priebėga yra Viešpats. 
\par 7 O kad iš Siono ateitų išgelbėjimas Izraeliui! Kai Viešpats išlaisvins savo tautą, džiaugsis Jokūbas, linksminsis Izraelis!



\chapter{15}


\par 1 Viešpatie, kas gali pasilikti Tavo palapinėje ir gyventi Tavo šventajame kalne? 
\par 2 Tas, kuris gyvena dorai, elgiasi teisiai ir savo širdyje kalba tiesą; 
\par 3 kuris savo liežuviu nešmeižia, nedaro pikto savo artimui ir neplūsta kaimyno; 
\par 4 kuris paniekintąjį smerkia, o Viešpaties bijančius gerbia; kuris prisiekęs, nors ir savo nenaudai, ištesi; 
\par 5 kuris neskolina pinigų palūkanoms gauti, nepriima kyšių prieš nekaltą žmogų. Kas šitaip elgiasi, niekada nesusvyruos.



\chapter{16}


\par 1 Apsaugok mane, Dieve, nes Tavimi aš pasitikiu. 
\par 2 Tariau Viešpačiui: “Tu­mano Valdovas, be Tavęs man nebus gera”. 
\par 3 Šventieji, krašto garbingieji man labai patinka. 
\par 4 Kurie svetimus dievus sekioja, turi kentėti daug skausmų. Aš kraujo aukų jiems neaukosiu, mano lūpos neištars jų vardo. 
\par 5 Viešpats yra mano paveldėjimo dalis ir mano taurė, Jo rankoje mano likimas. 
\par 6 Virvės man krito į patinkančias vietas, aš turiu gerą paveldėjimą. 
\par 7 Aš laiminsiu Viešpatį, kuris patarimą man teikia; mano širdis įspėja mane naktį. 
\par 8 Aš nuolatos statau Viešpatį prieš save, Jis mano dešinėje­aš nesvyruosiu. 
\par 9 Linksminasi mano širdis, džiaugiasi mano siela ir mano kūnas ilsėsis su viltimi. 
\par 10 Nes tu nepaliksi mano sielos mirusiųjų buveinėje ir neleisi savo šventajam supūti. 
\par 11 Tu parodysi man gyvenimo kelią, Tavo akivaizdoje yra džiaugsmo pilnatvė, Tavo dešinėje­malonumai per amžius.



\chapter{17}


\par 1 Viešpatie, paklausyk teisiojo skundo, išgirsk mano šauksmą. Teišgirsta Tavo ausys maldą iš mano neklastingų lūpų. 
\par 2 Iš Tavęs teišeina man sprendimas, tegul Tavo akys mato teisybę. 
\par 3 Tu ištyrei mano širdį, aplankei mane naktį, išbandei mane ir nieko neradai. Aš nusprendžiau nenusidėti savo burna. 
\par 4 Žmonių darbuose pagal Tavo lūpų žodžius aš saugojausi naikintojo takų. 
\par 5 Palaikyk mane einantį Tavo takais, kad mano kojos nepaslystų. 
\par 6 Šaukiausi Tavęs, nes Tu išklausysi mane, Dieve! Atkreipk į mane savo dėmesį, išgirsk mano kalbą. 
\par 7 Parodyk savo nuostabų gailestingumą, Tu, kuris savo dešine gelbsti nuo priešų Tavimi pasitikinčius. 
\par 8 Saugok mane kaip savo akies vyzdį, savo sparnų šešėlyje slėpk mane 
\par 9 nuo prispaudėjų, nuo mano mirtinų priešų, kurie supa mane. 
\par 10 Užsidarę savo taukuose, jie kalba išdidžiai savo lūpomis. 
\par 11 Kur tik einame, jie supa mus, įbedę akis stebi ir rengiasi žemėn parblokšti 
\par 12 kaip liūtas, tykantis grobio, kaip liūto jauniklis, kuris tupi lindynėje. 
\par 13 Kelkis, Viešpatie, juos pasitik ir partrenk; išlaisvink savo kardu mano sielą iš nedorėlių, 
\par 14 iš žmonių­savo ranka, Viešpatie; iš pasaulio žmonių, kurių dalis šiame gyvenime, kurie pripildo savo pilvus Tavo gėrybėmis, jų vaikai sočiai privalgo, perteklių palikdami saviesiems vaikams. 
\par 15 O aš Tavo veidą regėsiu savo teisume, pabudęs pasisotinsiu Tavo regėjimu.



\chapter{18}

\par 1 “Mylėsiu Tave, Viešpatie, mano stiprybe! 
\par 2 Viešpats yra mano uola, tvirtovė ir išlaisvintojas. Dievas yra mano jėga, Juo pasitikėsiu; Jis­mano skydas, išgelbėjimo ragas, mano aukštas bokštas. 
\par 3 Šauksiuosi Viešpaties, kuris vertas gyriaus, ir taip būsiu išgelbėtas iš priešų. 
\par 4 Mirties kančios supo mane, bedievių antplūdis gąsdino mane. 
\par 5 Pragaro kančios apraizgė mane, manęs laukė mirties pinklės. 
\par 6 Sielvarte šaukiausi Viešpaties, savo Dievo. Jis išgirdo savo šventykloje mano balsą, mano šauksmas pasiekė Jo ausis. 
\par 7 Susvyravo, sudrebėjo žemė, kalnų pamatai sujudėjo ir drebėjo, nes Viešpats užsirūstino. 
\par 8 Iš Jo šnervių kilo dūmai, iš burnos veržėsi naikinančios liepsnos, įkaitusios žarijos skraidė. 
\par 9 Jis palenkė dangų ir nužengė, tamsa buvo po Jo kojomis. 
\par 10 Jis sėdėjo ant cherubo ir skrido, vėjo sparnai Jį nešė. 
\par 11 Jis pasislėpė tamsoje, juodi vandenys ir debesys supo Jį. 
\par 12 Nuo spindesio Jo priekyje pro debesis veržėsi kruša ir degančios žarijos. 
\par 13 Viešpats sugriaudė danguose, Aukščiausiasis parodė savo balsą. 
\par 14 Jis laidė strėles ir išsklaidė juos, siuntė žaibus ir juos sunaikino. 
\par 15 Iškilo jūros dugnas, atsivėrė žemės pamatai nuo Tavo balso, Viešpatie, nuo Tavo rūstybės kvapo. 
\par 16 Iš aukštybių Jis ištiesė ranką ir paėmė mane, ištraukė iš gausių vandenų. 
\par 17 Jis išgelbėjo mane iš galingo priešo, iš tų, kurie manęs nekentė, nes jie buvo stipresni už mane. 
\par 18 Jie puolė mane aną pražūtingąją dieną, bet mano atrama buvo Viešpats. 
\par 19 Jis išvedė mane į platybes ir išlaisvino mane, nes Jis pamėgo mane. 
\par 20 Viešpats atlygino man pagal mano teisumą, Jis atmokėjo man pagal mano rankų švarumą. 
\par 21 Aš laikiausi Viešpaties kelio, neatsitraukiau nuo savo Dievo nusikalsdamas. 
\par 22 Jo įsakymai buvo prieš mane ir nuo Jo nuostatų neatsitraukiau. 
\par 23 Prieš Jį buvau atviras ir saugojaus, kad nenusikalsčiau. 
\par 24 Todėl man atlygino Viešpats pagal mano teisumą, pagal mano rankų švarumą Jo akyse. 
\par 25 Gailestingam Tu pasirodai gailestingas, tobulam­tobulas, 
\par 26 tyram Tu pasirodai tyras, su sukčiumi elgiesi suktai. 
\par 27 Tu gelbsti prispaustuosius, bet pažemini išdidžius žvilgsnius. 
\par 28 Tu, Viešpatie, uždegi man žiburį; Viešpats, mano Dievas, šviečia man tamsumoje. 
\par 29 Su Tavimi galiu pulti priešą, su Dievu­peršokti sieną. 
\par 30 Dievo kelias tobulas, Viešpaties žodis ugnimi valytas. Jis yra skydas visiems, kurie Juo pasitiki. 
\par 31 Kas yra Dievas, jei ne Viešpats? Kas uola, jei ne mūsų Dievas? 
\par 32 Jis apjuosia mane jėga, padaro mano kelią tobulą. 
\par 33 Mano kojas Jis padaro kaip stirnos, į aukštumas mane iškelia. 
\par 34 Mano rankas Jis moko kovoti, kad mano rankos sulaužytų plieninį lanką. 
\par 35 Tu man davei išgelbėjimo skydą, Tavo dešinė palaikė mane, Tavo gerumas mane išaukštino. 
\par 36 Tu praplatinai mano žingsnius, kad mano kojos nepaslystų. 
\par 37 Persekiojau priešus ir pasivijau, nepasukau atgal, kol jų nesunaikinau. 
\par 38 Sužeidžiau juos, kad nebegalėjo pasikelti, krito jie man po kojomis. 
\par 39 Tu apjuosei mane jėga kovai ir atidavei man tuos, kurie sukilo prieš mane. 
\par 40 Tu palenkei prieš mane mano priešus, kad galėčiau sunaikinti tuos, kurie manęs nekenčia. 
\par 41 Jie šaukė, bet niekas nepadėjo, į Viešpatį kreipėsi­Jis neatsiliepė. 
\par 42 Sutrypiau juos į žemės dulkes, kaip gatvių purvą sumyniau. 
\par 43 Išgelbėjai mane tautos kovose, man skyrei valdyti pagonis, tautos, kurių nepažinau, tarnaus man. 
\par 44 Kai tik išgirs apie mane, jie paklus man, svetimšaliai pasiduos man. 
\par 45 Svetimšaliai išblykš, drebėdami išeis iš savo pilių. 
\par 46 Viešpats yra gyvas! Palaiminta tebūna mano uola! Aukštinamas tebūna mano išgelbėjimo Dievas. 
\par 47 Dievas atkeršija už mane ir pajungia man tautas. 
\par 48 Jis gelbsti mane iš mano priešų. Tu iškėlei mane aukščiau nei tuos, kurie sukyla prieš mane, ir išlaisvinai nuo žiauraus žmogaus. 
\par 49 Todėl dėkosiu Tau, Viešpatie, tarp pagonių, giedosiu gyrių Tavo vardui. 
\par 50 Didelį išgelbėjimą Jis suteikia savo karaliui ir parodo gailestingumą savo pateptajam Dovydui ir jo palikuonims per amžius”.



\chapter{19}


\par 1 Dangūs skelbia Dievo šlovę, tvirtuma byloja apie Jo rankų darbus. 
\par 2 Diena pasakoja dienai, o naktis praneša nakčiai. 
\par 3 Nėra kalbos ar tarmės, kur jų balsas nebūtų girdimas. 
\par 4 Per visą žemę sklinda jų garsas, jų žodžiai­iki pasaulio krašto. Jis saulei pastatė padangtę juose. 
\par 5 Ji džiaugiasi kaip išeinąs iš savo kambario jaunikis, kaip karžygys, bėgdamas taku. 
\par 6 Iš vieno krašto padangės pakyla ir kitą jos kraštą pasiekia, niekas negali nuo jos kaitros pasislėpti. 
\par 7 Tobulas Viešpaties įstatymas, jis gaivina sielą. Viešpaties liudijimas tikras, išminties moko paprastuosius. 
\par 8 Viešpaties nuostatai teisingi, jie džiugina širdis. Viešpaties įsakymas tyras, akims duoda šviesybę. 
\par 9 Viešpaties baimė gryna, ji išlieka per amžius. Viešpaties sprendimai­ tiesa, jie visi iki vieno teisingi. 
\par 10 Brangesni jie už auksą, už gryniausią auksą; jie saldesni už medų, už korių syvą. 
\par 11 Tavo tarnas jais įspėjamas; kas jų laikosi, gauna didelį atpildą. 
\par 12 Kas pastebi savo klaidas? Nuvalyk nuo manęs man nežinomas kaltes. 
\par 13 Savo tarną apsaugok nuo akiplėšiškumo, kad nuodėmės manęs nepavergtų. Tada būsiu doras ir išvengsiu didelių nuodėmių. 
\par 14 Tepatinka Tau mano lūpų žodžiai ir mintys mano širdies, Viešpatie, mano stiprybe, mano atpirkėjau.



\chapter{20}


\par 1 Teišklauso Viešpats tave suspaudimo dieną, tesaugo tave Jokūbo Dievo vardas. 
\par 2 Tegul iš savo šventyklos tau siunčia pagalbą, tepadeda tau iš Siono. 
\par 3 Visas tavo aukas teatsimena, tepriima tavo deginamąją auką. 
\par 4 Ko trokšta tavo širdis, tau tesuteikia, teišpildo kiekvieną tavo sumanymą. 
\par 5 Mes džiaugsimės tavo išgelbėjimu, vėliavas kelsime savo Dievo vardu. Visus tavo prašymus teįvykdo Viešpats! 
\par 6 Žinau, kad Viešpats gelbsti savo pateptąjį, iš savo šventojo dangaus jį išklausė ir parėmė savo dešinės galybe. 
\par 7 Vieni pasitiki savo žirgais, kiti­kovos vežimais, o mes prisiminsime Viešpaties, savo Dievo, vardą. 
\par 8 Anie suklupo ir krito, o mes pakilome ir stovime tiesūs. 
\par 9 Išgelbėk, Viešpatie, išgirsk, Karaliau, kai šaukiamės.



\chapter{21}


\par 1 Viešpatie, džiaugiasi karalius Tavo galybe, džiūgaute džiūgauja dėl Tavo pagalbos. 
\par 2 Suteikei jam, ko širdis geidė, neužgynei, ko prašė jo lūpos. 
\par 3 Pasitikai jį gausiais palaiminimais, ant galvos jam uždėjai aukso karūną. 
\par 4 Jis prašė Tavęs gyvenimo, ir Tu davei jam dienų ilgumą per amžių amžius. 
\par 5 Didelė jo šlovė dėl Tavo pagalbos: garbę ir didybę suteikei jam. 
\par 6 Tu padarei jį labiausiai palaimintą per amžius, savo veidu labai jį pradžiuginai. 
\par 7 Karalius pasitiki Viešpačiu, Aukščiausiojo gailestingumas neleis jam svyruoti. 
\par 8 Suras Tavoji ranka visus Tavo priešus, pasieks Tavo dešinė visus, kurie Tavęs nekenčia. 
\par 9 Užsirūstinęs padarysi juos kaip degančią krosnį; Viešpats sunaikins juos savo rūstybėje, praris juos ugnis. 
\par 10 Jų vaisių nušluosi nuo žemės, jų palikuonis išnaikinsi tarp žmonių. 
\par 11 Jie planavo pikta prieš Tave, sugalvojo klastą, kurios nesugebės įvykdyti. 
\par 12 Priversi juos bėgti, jiems tiesiai į veidą nukreipsi savo lanką. 
\par 13 Pakilk, Viešpatie, savo galybėje, mes šlovinsime ir girsime Tavo jėgą.



\chapter{22}


\par 1 Mano Dieve, mano Dieve, kodėl mane apleidai? Mano šauksmas toli nuo mano pagalbos. 
\par 2 Mano Dieve, šaukiuosi Tavęs dieną, bet Tu neišklausai, ir naktį aš nenutylu. 
\par 3 Tu esi šventas, kuris gyveni Izraelio gyriuje. 
\par 4 Mūsų tėvai pasitikėjo Tavimi, ir Tu išgelbėjai juos. 
\par 5 Šaukėsi Tavęs ir buvo išgelbėti, pasitikėjo tavimi ir nebuvo sugėdinti. 
\par 6 Aš­ne žmogus, bet kirmėlė, žmonių išjuoktas, tautos paniekintas. 
\par 7 Kas mane mato, tyčiojasi iš manęs, sustato lūpas, kraipo galvą: 
\par 8 “Jis pasitikėjo Viešpačiu, teišvaduoja jį dabar, teišgelbsti jį, nes jį pamėgo”. 
\par 9 Tu gi mane išėmei iš įsčių, mane saugojai prie motinos krūtų. 
\par 10 Tavo globai buvau pavestas nuo gimimo, nuo pirmosios dienos buvai mano Dievas. 
\par 11 Nebūk toli nuo manęs, nes bėda yra arti ir nėra, kas padėtų. 
\par 12 Daug veršių mane apsupo, Bašano jaučiai mane apstojo. 
\par 13 Jie išsižiojo prieš mane tarsi plėšrus ir riaumojantis liūtas. 
\par 14 Aš išlietas lyg vanduo. Išnarstyti visi mano kaulai. Mano širdis kaip vaškas, ištirpęs krūtinėje. 
\par 15 Mano jėgos išdžiūvo lyg šukė, prie gomurio limpa liežuvis; į mirties dulkes Tu atvedei mane. 
\par 16 Apspito mane šunys, nedorėlių gauja aplink mane. Jie pervėrė mano rankas ir kojas. 
\par 17 Galiu suskaičiuoti visus savo kaulus. O jie žiūri ir stebi mane, 
\par 18 drabužius mano dalijas, meta dėl mano apdaro burtą. 
\par 19 Bet, Viešpatie, nebūk toli nuo manęs. Mano stiprybe, skubėk man padėti. 
\par 20 Nuo kardo gelbėk mano sielą, iš šuns letenų­mano gyvybę. 
\par 21 Iš liūto nasrų gelbėk ir nuo stumbro ragų išgirdęs išvaduok mane. 
\par 22 Tavąjį vardą paskelbsiu broliams, susirinkimo viduryje girsiu Tave. 
\par 23 Kurie bijote Viešpaties, girkite Jį! Šlovinkite Jį, visi Jokūbo palikuonys, bijokite Jo, visi Izraelio vaikai! 
\par 24 Jis nepaniekino ir neatstūmė nuskriausto vargšo, nuo jo nepaslėpė veido, jo šauksmą išklausė. 
\par 25 Jį girsiu dideliame susirinkime, vykdysiu įžadus tarp tų, kurie Jo bijo. 
\par 26 Vargšai valgys ir pasisotins, Viešpatį girs visi, kas Jo ieško; jūsų širdys tegyvuoja per amžius! 
\par 27 Prisimins ir gręšis į Viešpatį visi žemės pakraščiai, Jo akivaizdoje lenksis pagonių tautos. 
\par 28 Viešpačiui priklauso karalystė, Jis viešpatauja pagonims. 
\par 29 Visi žemės riebieji valgys ir pagarbins Jį, prieš Jį nusilenks tie, kurie į dulkes nužengia ir negali išlaikyti savo sielos gyvos. 
\par 30 Palikuonys tarnaus Jam, jie pasakos apie Viešpatį būsimai kartai. 
\par 31 Jie ateis ir paskelbs Jo teisumą gimsiančiai tautai: “Viešpats tai padarė”.



\chapter{23}


\par 1 Viešpats yra mano ganytojas­aš nestokosiu. 
\par 2 Jis paguldo mane žaliuojančiose ganyklose, veda mane prie tylių vandenų. 
\par 3 Jis atgaivina mano sielą, veda mane teisumo takais dėl savo vardo. 
\par 4 Nors eičiau per mirties šešėlio slėnį, nebijosiu pikto, nes Tu su manimi. Tavo lazda bei Tavo ramstis nuramina mane. 
\par 5 Tu paruoši man stalą mano priešų akivaizdoje, aliejumi man patepi galvą, mano taurė sklidina. 
\par 6 Tikrai, gerumas ir gailestingumas lydės mane per visas mano gyvenimo dienas. Aš gyvensiu Viešpaties namuose per amžius.



\chapter{24}


\par 1 Viešpaties yra žemė ir visa, kas joje yra, pasaulis ir kas jame gyvena. 
\par 2 Jis ant jūrų ją pastatė, ant srovių ją įtvirtino. 
\par 3 Kas kops į Viešpaties kalną? Kas atsistos Jo šventoje vietoje? 
\par 4 Tas, kas turi švarias rankas ir tyrą širdį, kuris nenukreipė savo sielos į tuštybes ir neteisingai neprisiekė. 
\par 5 Jis gaus palaiminimą iš Viešpaties ir teisumą iš savo gelbėtojo Dievo. 
\par 6 Tai karta, kuri ieško Jo, ieško Jokūbo Dievo veido. 
\par 7 Pakelkite galvas, vartai, pakilkite, senovinės durys, ir šlovės Karalius įeis! 
\par 8 Kas tas šlovės Karalius? Tai Viešpats, stiprus ir galingas. Tai Viešpats, galiūnas kovoje. 
\par 9 Pakelkite galvas, vartai, pakilkite, senovinės durys, ir šlovės Karalius įeis! 
\par 10 Kas tas šlovės Karalius? Kareivijų Viešpats­Jis yra šlovės Karalius.



\chapter{25}


\par 1 Viešpatie, į Tave keliu savo sielą. 
\par 2 Mano Dieve, pasitikiu Tavimi. Tenebūsiu sugėdintas, tenedžiūgaus dėl manęs mano priešas. 
\par 3 Te nė vienas, kuris laukia Tavęs, nebūna sugėdintas, tebūna sugėdinti tie, kurie nusikalsta be priežasties. 
\par 4 Viešpatie, parodyk man savąjį kelią, pamokyk mane savo takų. 
\par 5 Vesk mane savo tiesa ir mokyk, nes Tu esi mano išgelbėjimo Dievas, laukiu Tavęs visą dieną. 
\par 6 Viešpatie, atsimink gailestingumą ir malones, kurios yra nuo amžių. 
\par 7 Užmiršk mano jaunystės kaltes ir nedorybes. O Viešpatie, atmink mane dėl savo gerumo, pagal savo gailestingumą. 
\par 8 Geras ir teisus yra Viešpats, todėl nusidėjėliams kelią parodo. 
\par 9 Romiųjų mintis kreipia į tiesą, savo kelių moko nuolankiuosius. 
\par 10 Visi Viešpaties takai yra gailestingumas ir tiesa tiems, kurie laikosi Jo sandoros ir įsakymų. 
\par 11 Atleisk mano kaltę, Viešpatie, dėl savo vardo, nes ji yra didelė. 
\par 12 Žmogų, kuris bijo Viešpaties, Jis mokys pasirinkti kelią. 
\par 13 Jis pats gyvens laimingai, o jo vaikai paveldės kraštą. 
\par 14 Viešpaties paslaptis su tais, kurie Jo bijo, jiems Jis apreikš savo sandorą. 
\par 15 Visada mano akys į Viešpatį krypsta­Jis mano kojas išpainios iš pinklių. 
\par 16 Pažvelk į mane ir pasigailėk manęs, nes esu vienišas ir suvargęs. 
\par 17 Mano širdies vargų pagausėjo, išvesk mane iš bėdų. 
\par 18 Pažvelk į mano vargą bei skausmą ir atleisk visas mano nuodėmes. 
\par 19 Matai, kaip daug mano priešų, kaip baisiai manęs jie nekenčia. 
\par 20 Saugok mano sielą ir gelbėk mane, kad nebūčiau sugėdintas, nes pasitikiu Tavimi. 
\par 21 Nekaltumas ir teisumas tesaugo mane, nes aš laukiu Tavęs! 
\par 22 Dieve, išgelbėk Izraelį iš visų nelaimių!



\chapter{26}


\par 1 Viešpatie, pateisink mane, nes esu nekaltas; Viešpačiu pasitikėjau, todėl nesvyruosiu. 
\par 2 Viešpatie, išmėgink ir išbandyk mane, ištirk mano širdį ir inkstus. 
\par 3 Tavo malonė yra mano akyse, ir aš vaikščiojau Tavo tiesoje. 
\par 4 Nesėdžiu su klastingaisiais, nesilankau pas apsimetėlius. 
\par 5 Nekenčiu piktadarių draugystės ir nesėdžiu su nedorėliais. 
\par 6 Nekaltume plaunu savo rankas ir lankau Tavo aukurą, Viešpatie, 
\par 7 kad garsiai giedočiau padėkos giesmę ir skelbčiau Tavo nuostabius darbus. 
\par 8 Viešpatie, mėgstu Tavo namų buveinę, vietą, kur Tavo garbė gyvena. 
\par 9 Neatimk mano sielos su nusidėjėliais ir mano gyvenimo su kraujo trokštančiais. 
\par 10 Jų rankose nusikaltimas, jų dešinė pilna kyšių. 
\par 11 Tačiau aš elgiuosi nekaltai. Atpirk mane ir būk man gailestingas. 
\par 12 Mano koja stovi lygioje vietoje; aš susirinkimuose Viešpatį girsiu.



\chapter{27}


\par 1 Viešpats yra mano šviesa ir mano išgelbėjimas­ko man bijoti? Viešpats yra mano stiprybė­prieš ką man drebėti? 
\par 2 Kai nedorėliai­mano priešai ir piktieji­puola mane, norėdami suėsti, jie susvyruoja ir krinta. 
\par 3 Nors stovėtų prieš mane kariuomenė, nenusigąs mano širdis; nors ir karas kiltų prieš mane, aš pasitikėsiu. 
\par 4 Vieno dalyko prašau, Viešpatie, ir to vieno trokštu­gyventi Viešpaties namuose per visas savo gyvenimo dienas, stebėti Viešpaties grožį ir lankyti Jo šventyklą. 
\par 5 Jis paslėps mane savo palapinėje nelaimės dieną, savo buveinės slaptoje vietoje; ant uolos pastatys mane. 
\par 6 Mano galva bus pakelta virš mano priešų, kurie mane supa. Džiaugsmo aukas aukosiu Jo palapinėje, šlovinsiu ir giedosiu gyrių Viešpačiui! 
\par 7 Viešpatie, išgirsk mano balsą, kai šaukiuos. Pasigailėk manęs ir išklausyk mane! 
\par 8 Kai Tu pasakei: “Ieškokite mano veido!”, mano širdis Tau atsakė: “Tavo veido, Viešpatie, aš ieškosiu”. 
\par 9 Neslėpk savo veido nuo manęs! Neatstumk rūstybėje savo tarno! Tu esi mano pagalba. Neatmesk ir nepalik manęs, mano išgelbėjimo Dieve! 
\par 10 Nors mano tėvas ir motina paliktų mane, tačiau Viešpats mane priims. 
\par 11 Viešpatie, pamokyk mane savo kelio ir vesk mane tikru taku dėl mano priešų. 
\par 12 Neatiduok manęs prispaudėjų savivalei, nes pakilo prieš mane neteisingi liudytojai ir alsuoja smurtu. 
\par 13 Aš tikiu, kad matysiu Viešpaties gerumą gyvųjų žemėje. 
\par 14 Lauk Viešpaties! Būk drąsus, ir Jis sutvirtins tavo širdį. Lauk, aš sakau, Viešpaties!



\chapter{28}


\par 1 Tavęs šaukiuosi, Viešpatie, mano uola! Išgirsk mane. Jei tylėsi, tapsiu kaip tie, kurie nužengia į duobę. 
\par 2 Išgirsk mano maldavimo balsą, kai šaukiuosi Tavęs, kai keliu savo rankas į Tavo švenčiausiąją buveinę! 
\par 3 Neatstumk manęs kartu su nedorėliais, su piktadariais, kurie maloniai kalba su artimu, bet pikta mano savo širdyse. 
\par 4 Duok jiems, ko verti jų darbai ir pikti poelgiai. Atlygink jiems, kiek vertas jų rankų darbas; užmokėk jiems, ką jie užsitarnavo. 
\par 5 Jie nepaiso Viešpaties veiksmų ir Jo rankų darbų. Jis sunaikins juos ir nebeleis atsigauti. 
\par 6 Palaimintas Viešpats, nes Jis išklausė mano maldavimą! 
\par 7 Viešpats yra mano jėga ir skydas; Juo pasitikėjo mano širdis, ir Jis man padėjo. Todėl džiūgauja mano širdis, ir savo giesme girsiu Jį. 
\par 8 Viešpats yra jų jėga, išgelbstinti priebėga pateptajam. 
\par 9 Išgelbėk savo tautą ir laimink savo paveldėjimą. Ganyk ir išaukštink juos per amžius.



\chapter{29}


\par 1 Pripažinkite Viešpačiui, Dievo sūnūs, pripažinkite Viešpačiui šlovę ir galybę! 
\par 2 Pripažinkite Viešpačiui šlovę, derančią Jo vardui; pagarbinkite Viešpatį šventumo grožyje. 
\par 3 Viešpaties balsas viršum vandenų. Šlovės Dievas sugriaudė. Viešpats viršum plačiųjų vandenų. 
\par 4 Viešpaties balsas galingas. Viešpaties balsas didingas. 
\par 5 Viešpaties balsas laužo kedrus. Viešpats laužo Libano kedrus. 
\par 6 Jis šokdina juos kaip veršius, Libaną ir Sirjoną kaip jauniklį stumbrą. 
\par 7 Viešpaties balsas įskelia ugnies liepsnas. 
\par 8 Viešpaties balsas sudrebina dykumą. Viešpats sudrebina Kadešo dykumą. 
\par 9 Viešpaties balsas priverčia gimdyti elnes ir laužo miškus. Jo šventykloje visi kalba apie Jo šlovę. 
\par 10 Viešpats sėdi viršum tvano, Viešpats sėdi kaip Karalius per amžius! 
\par 11 Viešpats suteiks stiprybę savo tautai. Viešpats palaimins savo tautą taika.



\chapter{30}


\par 1 Aukštinu Tave, Viešpatie, nes Tu išlaisvinai mane ir neleidai mano priešams džiaugtis dėl manęs. 
\par 2 Viešpatie, mano Dieve, šaukiausi Tavęs, ir Tu išgydei mane. 
\par 3 Viešpatie, Tu išvedei iš mirusiųjų buveinės mano sielą, išlaikei mane gyvą, kad nenužengčiau į duobę. 
\par 4 Giedokite Viešpačiui, Jo šventieji, dėkokite prisiminę Jo šventumą. 
\par 5 Tik akimirksnį trunka Jo rūstybė, o visą gyvenimą lydi Jo palankumas. Vakare ateina verksmas, o rytą­džiūgavimas. 
\par 6 Būdamas saugus, sakiau: “Niekados nesvyruosiu!” 
\par 7 Viešpatie, savo palankumu suteikei man tvirtybę. Kai paslėpei savo veidą­nusigandau. 
\par 8 Tavęs, Viešpatie, šaukiaus, savo Viešpatį maldavau: 
\par 9 “Kokia Tau nauda iš mano kraujo, iš to, kad nužengsiu į duobę? Ar girs Tave dulkės, ar jos skelbs Tavo tiesą? 
\par 10 Išgirsk, Viešpatie, ir pasigailėk manęs; Viešpatie, būk man padėjėjas!” 
\par 11 Tu pavertei mano raudą džiaugsmu; atrišai mano ašutinę ir apjuosei linksmybe, 
\par 12 kad Tau giedotų mano siela ir netylėtų. Viešpatie, mano Dieve, visados Tau dėkosiu.



\chapter{31}


\par 1 Tavimi, Viešpatie, pasitikiu; niekados tenebūsiu sugėdintas. Savo teisumu išgelbėk mane. 
\par 2 Palenk prie manęs savo ausį, skubiai išgelbėk mane. Būk man stipri uola, tvirtovė išsigelbėti. 
\par 3 Tu mano uola ir tvirtovė. Dėl savo vardo vesk ir saugok mane, 
\par 4 išnarpliok mane iš tinklo, kuris slaptai man padėtas, nes Tu esi mano jėga. 
\par 5 Į Tavo rankas pavedu savo dvasią. Tu išgelbėjai mane, Viešpatie, tiesos Dieve! 
\par 6 Aš neapkenčiu niekingų stabų garbintojų, bet pasitikiu Viešpačiu. 
\par 7 Aš džiūgausiu ir linksminsiuos Tavo gailestingume; Tu pažvelgei į mano vargą, į skausmą mano sielos, 
\par 8 neatidavei manęs į priešo rankas, leidai mano kojoms laisvai bėgti. 
\par 9 Pasigailėk manęs, Viešpatie, nes suspaustas esu; nusilpo nuo liūdesio mano akys, mano siela ir pilvas. 
\par 10 Sielvartas graužia mano gyvenimą, vaitojimas­mano metus; išseko mano jėgos dėl mano kaltės ir mano kaulai sunyko. 
\par 11 Visiems savo priešams tapau pajuoka, netgi savo kaimynams; pažįstami bijo manęs, kurie gatvėje mato mane, bėga nuo manęs. 
\par 12 Esu pamirštas kaip numirėlis, dingęs iš atminties; tapau kaip sudužęs indas. 
\par 13 Aš girdžiu, ką daugelis šnibžda­baimė aplinkui. Jie tariasi prieš mane, galvoja atimti mano gyvybę. 
\par 14 Tavimi, Viešpatie, pasitikėjau ir sakiau: “Tu esi mano Dievas”. 
\par 15 Tavo rankose yra mano laikai. Gelbėk mane iš rankos mano priešų ir persekiotojų. 
\par 16 Parodyk savo tarnui savo šviesų veidą; išgelbėk mane dėl savo gailestingumo. 
\par 17 Viešpatie, tenebūsiu sugėdintas, nes Tavęs šaukiuosi! Telieka sugėdinti nedorėliai, tenutyla jie kape. 
\par 18 Tegul tampa nebylios melagių lūpos, kurios prieš teisųjį įžūliai su puikybe ir panieka kalba! 
\par 19 Viešpatie, koks didis yra Tavo gerumas Tavęs bijantiems, kurie pasitiki Tavimi žmonių akivaizdoje. 
\par 20 Tu paslėpsi juos savo artume nuo žmonių išdidumo, saugosi juos savo palapinėje nuo liežuvių plakimo. 
\par 21 Palaimintas tebūna Viešpats! Jis suteikė man nuostabią malonę sustiprintame mieste. 
\par 22 Nes aš skubotai tariau: “Esu atskirtas nuo Tavo akių”. Tačiau Tu išgirdai mano maldavimą, kai šaukiausi Tavęs. 
\par 23 Mylėkite Viešpatį, visi Jo šventieji. Ištikimuosius apsaugo Viešpats ir su kaupu atlygina išdidiems. 
\par 24 Būkite drąsūs visi, kurie pasitikite Viešpačiu, ir Jis sustiprins jūsų širdis.



\chapter{32}


\par 1 Palaimintas, kuriam neteisybės atleistos, kurio nuodėmės padengtos. 
\par 2 Palaimintas žmogus, kuriam Viešpats neįskaito kaltės ir kurio dvasioje nėra klastos. 
\par 3 Kol tylėjau, nyko mano kaulai nuo kasdieninio mano vaitojimo. 
\par 4 Dieną ir naktį sunkiai slėgė mane Tavo ranka. Mano jėgos seko, lyg vasaros kaitros džiovinamos. 
\par 5 Savo nuodėmę Tau išpažinau ir savo kaltės neslėpiau. Tariau: “Išpažinsiu savo nusikaltimus Viešpačiui”. Tada atleidai man kaltę. 
\par 6 Dėl to kiekvienas teisusis melsis Tau, kai Tave galima surasti, kol didelis tvanas nepriartėjo prie jų. 
\par 7 Tu esi mano slėptuvė, nuo pavojų apsaugosi mane, išgelbėjimo giesmėmis apsupsi mane. 
\par 8 Pamokysiu tave ir parodysiu kelią, kuriuo turi eiti; tave mano akys lydės. 
\par 9 Nebūk kaip arklys ar mulas, kurie neturi supratimo. Kamanomis ir žąslais juos reikia pažaboti, kad jie priartėtų prie tavęs. 
\par 10 Daug kančių turi nedorėlis, bet tas, kuris pasitiki Viešpačiu, bus apsuptas gailestingumo. 
\par 11 Džiaukitės Viešpatyje, džiūgaukite, teisieji, šaukite iš džiaugsmo, tiesiaširdžiai.



\chapter{33}


\par 1 Džiūgaukite, teisieji, Viešpatyje, doriesiems tinka gyriaus giesmė. 
\par 2 Girkite Viešpatį arfomis, giedokite Jam, pritardami dešimčiastygiu psalteriu! 
\par 3 Giedokite Jam naują giesmę, grokite meistriškai su džiaugsmo šūksniais. 
\par 4 Teisingas yra Viešpaties žodis ir visi Jo darbai atlikti tiesoje. 
\par 5 Jis mėgsta teisumą ir teisingumą. Viešpaties gerumo yra pilna žemė. 
\par 6 Viešpaties žodžiu sukurti dangūs, Jo burnos kvapu­visa jų kareivija. 
\par 7 Tartum į indą Jis jūrų vandenis surinko, į sandėlius uždarė gelmes. 
\par 8 Tegul bijo Viešpaties visa žemė, tegul garbina Jį visi pasaulio gyventojai! 
\par 9 Jis tarė­ir įvyko; Jis įsakė­ir atsirado. 
\par 10 Viešpats paverčia niekais pagonių sumanymą, suardo tautų planus. 
\par 11 Viešpaties sumanymas lieka per amžius, Jo širdies mintys per kartų kartas. 
\par 12 Palaiminta tauta, kurios Dievas yra Viešpats, tauta, kurią Jis išsirinko savo nuosavybe! 
\par 13 Iš dangaus žvelgia Viešpats ir stebi visą žmoniją. 
\par 14 Iš savo gyvenamos vietos Jis žiūri į visus žemės gyventojus. 
\par 15 Jis kiekvieno širdį sutvėrė ir stebi visus jų darbus. 
\par 16 Ne kariuomenės gausumas karalių išgelbsti, karžygys neišsilaisvina didele jėga. 
\par 17 Žirgas nepadės ir neišgelbės savo stiprumu. 
\par 18 Štai Viešpaties akis stebi tuos, kurie Jo bijo, kurie laukia Jo gailestingumo, 
\par 19 kad išgelbėtų nuo mirties jų sielą ir bado metu išlaikytų gyvus. 
\par 20 Mūsų siela laukia Viešpaties­Jis mūsų pagalba ir skydas. 
\par 21 Juo džiaugsis mūsų širdis, nes Jo šventuoju vardu mes pasitikėjome. 
\par 22 Viešpatie, tebūna Tavo gailestingumas mums, nes mes viliamės Tavimi.



\chapter{34}


\par 1 Šlovinsiu Viešpatį visada; nuolat girsiu Jį savo lūpomis. 
\par 2 Viešpačiu didžiuosis mano siela; nuolankieji išgirs tai ir džiaugsis. 
\par 3 Aukštinkime Viešpatį ir iškelkime kartu Jo vardą. 
\par 4 Ieškojau Viešpaties, Jis išklausė mane ir išgelbėjo mane iš visų mano baimių. 
\par 5 Jie žvelgė į Jį ir pralinksmėjo, jų veidai nebuvo sugėdinti. 
\par 6 Šis vargšas šaukėsi, ir Viešpats išgirdo, ir išgelbėjo jį iš visų jo bėdų. 
\par 7 Viešpaties angelas stovyklauja aplink tuos, kurie Jo bijo, ir išgelbsti juos. 
\par 8 Ragaukite ir matykite, kad geras yra Viešpats! Palaimintas žmogus, kuris Juo pasitiki. 
\par 9 Bijokite Viešpaties, Jo šventieji, nes nieko nestokoja tie, kurie Jo bijo. 
\par 10 Jauni liūtai pritrūksta ir badauja, bet tie, kurie ieško Viešpaties, nestokoja jokio gero. 
\par 11 Ateikite, vaikai, ir klausykite manęs: pamokysiu jus Viešpaties baimės. 
\par 12 Koks yra žmogus, kuris mėgsta gyventi ir geidžia gerų dienų patirti? 
\par 13 Sulaikyk savo liežuvį nuo pikto ir savo lūpas nuo klastingų kalbų. 
\par 14 Šalinkis pikto ir daryk gera. Ieškok ir siek taikos. 
\par 15 Viešpaties akys žvelgia į teisiuosius ir Jo ausys girdi jų šauksmą. 
\par 16 Viešpaties veidas prieš darančius pikta, kad išdildytų jų atsiminimą žemėje. 
\par 17 Teisieji šaukiasi Viešpaties, Jis išgirsta ir išgelbsti iš visų jų vargų. 
\par 18 Viešpats arti tų, kurių širdys sudužusios, išgelbsti tuos, kurių dvasia nusižeminusi. 
\par 19 Daug bėdų patiria teisusis, bet iš visų jį išgelbsti Viešpats. 
\par 20 Jis sergsti visus jo kaulus, kad nė vienas iš jų nesulūžtų. 
\par 21 Nedorėlį nužudys piktybė, kurie nekenčia teisiojo, pražus. 
\par 22 Viešpats išperka savo tarnų sielą, nepražus nė vienas, kuris Juo pasitiki.



\chapter{35}


\par 1 Viešpatie, apgink mano bylą nuo tų, kurie mane kaltina; kovok su tais, kurie kovoja prieš mane! 
\par 2 Paimk mažąjį ir didįjį skydą ir man padėk! 
\par 3 Ištrauk ietį ir atsistok prieš mano persekiotojus. Tark mano sielai: “Aš tavo išgelbėjimas”. 
\par 4 Tebūna sugėdinti ir pažeminti, kurie kėsinasi į mano gyvybę. Tesitraukia atgal suglumę, kurie nori man pakenkti. 
\par 5 Tebūna jie kaip pelai prieš vėją, Viešpaties angelui papūtus. 
\par 6 Tebūna jų kelias tamsus ir slidus ir Viešpaties angelas juos tepersekioja. 
\par 7 Be priežasties jie slaptai man spendė pinkles, iškasė duobę mano sielai. 
\par 8 Tegul jie netikėtai žūva, tegul jie patys įkliūva į paspęstas pinkles, teįkrinta į pražūtį. 
\par 9 Mano siela džiaugsis Viešpačiu, džiūgaus dėl Jo pagalbos. 
\par 10 Visi mano kaulai šauks: “Viešpatie, kas yra Tau lygus? Kas išlaisvina silpnąjį iš stipresnio už jį, vargšą iš plėšiko?” 
\par 11 Pakilo klastingi liudytojai, kaltino mane tuo, kuo aš nenusikaltau. 
\par 12 Jie atlygino man piktu už gera, apiplėšdami mano sielą. 
\par 13 Jiems sergant, ašutine vilkėjau; žeminau savo sielą pasninku, palenkęs galvą meldžiausi. 
\par 14 Elgiausi, lyg jie būtų man draugai ar broliai. Vaikščiojau nusiminęs, tartum gedėdamas motinos. 
\par 15 Bet kai aš susvyravau, jie džiaugėsi ir susibūrė prieš mane; puolėjai susibūrė prieš mane man nežinant ir drasko mane be paliovos. 
\par 16 Veidmainiškai tyčiojasi iš manęs, griežia dantimis prieš mane. 
\par 17 Viešpatie, ar ilgai dar žiūrėsi? Išgelbėk mano sielą nuo pražūties, mano vienintelę nuo riaumojančių liūtų. 
\par 18 Tau dėkosiu dideliame susirinkime, girsiu Tave minioje. 
\par 19 Tegul nesidžiaugia be pagrindo mano priešai, tenemirksi akimis tie, kurie nekenčia manęs be priežasties. 
\par 20 Jie nekalba apie taiką. Jie kuria klastingus planus prieš krašto taikiuosius. 
\par 21 Jie išsižioję rėkia: “Taip, taip, mes matėme tai savo akimis!” 
\par 22 Viešpatie, Tu tai matei­netylėk! Viešpatie, nebūk toli nuo manęs! 
\par 23 Sujudėk, pakilk ginti mano bylą, mano Viešpatie ir mano Dieve. 
\par 24 Viešpatie, mano Dieve, teisk mane, vadovaudamasis savo teisumu, neleisk jiems džiaugtis dėl manęs. 
\par 25 Tenemano jie savo širdyje: “O! To mes ir siekėme!” Tenesako: “Mes jį prarijome!” 
\par 26 Tesusigėsta ir teparausta visi, kurie džiaugiasi mano nelaime. Gėda ir panieka tebūna aprengti tie, kurie didžiuojasi prieš mane. 
\par 27 Tegul šaukia iš džiaugsmo ir linksminasi tie, kurie mane užtaria ir tegul sako: “Tebūna išaukštintas Viešpats, kuriam patinka Jo tarno gerovė!” 
\par 28 Mano liežuvis skelbs Tavo teisumą, per visą dieną girs Tave!



\chapter{36}


\par 1 Nedorėlio nuodėmė kalba mano širdyje, kad nėra Dievo baimės prieš jo akis. 
\par 2 Jis pataikauja sau, kai suranda savyje kaltę, kurios reikėtų nekęsti. 
\par 3 Jo burnos žodžiai­nedorybė ir klasta. Jis liovėsi elgtis išmintingai ir daryti gera. 
\par 4 Gulėdamas jis mąsto nedorybes, eina negeru keliu, neatsisako pikta. 
\par 5 Viešpatie, dangų siekia Tavo gailestingumas, Tavo ištikimybė­debesis. 
\par 6 Tavo teisumas­kaip dideli kalnai, Tavo sprendimai­kaip jūros gelmės. Viešpatie, Tu apsaugai žmogų ir gyvulį! 
\par 7 Dieve, kokia brangi yra Tavo malonė! Žmonės glaudžiasi Tavo sparnų šešėlyje. 
\par 8 Jie pasotinami Tavo namų riebalais, Tu duodi jiems gerti iš Tavo malonumų upės. 
\par 9 Tu turi gyvenimo šaltinį; Tavo šviesoje matome šviesą. 
\par 10 Tenesibaigia Tavo malonė tiems, kurie pažįsta Tave, ir Tavo teisumas­nuoširdiesiems. 
\par 11 Teneužkliudo manęs išdidumo koja ir ranka nusidėjėlio tenepastumia manęs! 
\par 12 Piktadariai krito, buvo parblokšti ir nebeatsikels!



\chapter{37}


\par 1 Nesijaudink dėl piktadarių, nepavydėk nedorėliams. 
\par 2 Jie greitai bus nupjauti tartum žolė, suvys kaip žaliuojanti žolė. 
\par 3 Pasitikėk Viešpačiu ir daryk gera, tada gyvensi žemėje ir būsi pamaitintas! 
\par 4 Gėrėkis Viešpačiu, ir Jis suteiks tau, ko geidžia tavo širdis. 
\par 5 Pavesk Viešpačiui savo kelią, pasitikėk Juo, ir Jis veiks. 
\par 6 Jis padarys tavo teisumą kaip šviesą ir tavo teisybę kaip vidudienį! 
\par 7 Ilsėkis Viešpatyje ir kantriai lauk Jo! Nesijaudink dėl to, kad sekasi žmogui, kuris daro pikta. 
\par 8 Liaukis nirtęs ir palik rūstybę. Nesijaudink, kad nedarytumei pikta. 
\par 9 Piktadariai bus sunaikinti, bet tie, kurie laukia Viešpaties, paveldės žemę. 
\par 10 Nes dar trumpa valandėlė, ir nebeliks nedorėlio; kai žvalgysies, kur jis buvo, jo nebebus. 
\par 11 Bet romieji paveldės žemę ir gėrėsis taikos apstumu. 
\par 12 Nedorėlis rengia teisiajam pikta ir griežia prieš jį dantimis. 
\par 13 Viešpats juokiasi iš jo, nes mato jo galą. 
\par 14 Nedorėliai išsitraukia kardą, įtempia lanką, kad partrenktų beturtį ir vargšą, nužudytų tuos, kurie elgiasi dorai. 
\par 15 Jų kardas įsmigs į jų pačių širdį ir jų lankai suluš. 
\par 16 Teisiojo truputis yra geriau už daugelio nedorėlių turtus, 
\par 17 nes nedorėlių rankos bus sulaužytos, o teisiuosius palaiko Viešpats. 
\par 18 Viešpats žino teisiųjų dienas, jų paveldėjimas liks amžiams. 
\par 19 Jie nebus sugėdinti nelaimių metu, bado dienomis jie bus pasotinti. 
\par 20 O nedorėliai pražus ir Viešpaties priešai sutirps kaip avinėlių taukai; jie dings kaip dūmai. 
\par 21 Nedorėliai skolinasi ir negrąžina, o teisusis yra gailestingas ir duoda. 
\par 22 Jo palaimintieji valdys žemę, o Jo prakeiktieji bus sunaikinti. 
\par 23 Viešpats nukreipia žmogaus žingsnius ir Jam patinka jo keliai. 
\par 24 Jei jis klumpa­neparkrinta, nes Viešpats laiko jo ranką. 
\par 25 Buvau jaunas ir pasenau, tačiau nemačiau, kad teisusis būtų užmirštas ir jo vaikai elgetautų. 
\par 26 Visada jis pasigaili ir skolina, jo vaikai yra palaiminti. 
\par 27 Traukis nuo pikto ir daryk gera, tai išliksi per amžius. 
\par 28 Viešpats mėgsta teisybę ir nepalieka savo šventųjų, bet saugo juos per amžius. O nedorėlių vaikai pražus. 
\par 29 Teisieji paveldės žemę ir gyvens joje amžinai. 
\par 30 Teisiojo burna kalba išmintingai, ir jo liežuvis­kas teisinga, 
\par 31 Dievo įstatymas yra jo širdyje; jo žingsniai nesvyruoja. 
\par 32 Nedorėlis tykoja teisiojo ir siekia jį nužudyti. 
\par 33 Viešpats jo nepaliks ano rankoje ir nepasmerks jo teisme. 
\par 34 Lauk Viešpaties ir laikykis Jo kelio. Jis išaukštins tave, kad paveldėtum žemę. Tu matysi, kaip nedorėliai žlugs. 
\par 35 Mačiau nedorėlį džiūgaujantį ir išsiplėtusį kaip šakotą kedrą. 
\par 36 Pro šalį ėjau, ir jo nebuvo, ieškojau jo, bet neradau. 
\par 37 Žiūrėk į tobuląjį ir stebėk teisųjį, nes tokių galas­ramybė. 
\par 38 Nusikaltėliai bus sunaikinti, nedorėlių galas­pražūtis. 
\par 39 Teisiųjų išgelbėjimas ateina nuo Viešpaties; Jis yra jų stiprybė nelaimių metu. 
\par 40 Viešpats padės jiems ir išlaisvins juos. Jis išlaisvins juos iš bedievių ir išgelbės, nes jie pasitikėjo Juo.



\chapter{38}


\par 1 Viešpatie, nebausk manęs rūstaudamas ir neplak savo įniršyje. 
\par 2 Tavo strėlės įsmigo į mane ir Tavo ranka slegia mane. 
\par 3 Nebėra nieko sveiko mano kūne dėl Tavo rūstybės ir poilsio mano kauluose dėl mano nuodėmės. 
\par 4 Mano kaltės iškilo virš mano galvos; lyg sunki našta jos pasidarė man per sunkios. 
\par 5 Dvokia ir pūliuoja mano žaizdos dėl mano kvailybės. 
\par 6 Esu varge, visai sulinkęs, vaikštau nusiminęs visą dieną. 
\par 7 Mano strėnos dega, nieko sveiko nebėra mano kūne. 
\par 8 Nusilpęs, labai sudaužytas vaitoju dėl savo širdies nerimo. 
\par 9 Viešpatie, Tu žinai visus mano troškimus ir mano dūsavimas nėra paslėptas nuo Tavęs. 
\par 10 Mano širdis smarkiai plaka, netekau jėgų, mano akių šviesa nyksta. 
\par 11 Mano draugai ir bičiuliai laikosi atstu nuo mano skausmų; mano artimieji stovi iš tolo. 
\par 12 Kurie kėsinasi į mano gyvybę, paspendė žabangus; kurie siekia man pakenkti, grasina man sunaikinimu, visą dieną rengia klastas. 
\par 13 Esu lyg kurčias­negirdžiu, lyg nebylys­neatveriu burnos. 
\par 14 Tapau lyg žmogus, kuris nieko negirdi, kurio burnoje nėra atsakymo. 
\par 15 Viešpatie, Tavimi viliuosi. Tu išgirsi, Viešpatie, mano Dieve! 
\par 16 Sakau: “Tenesidžiaugia ir tenesididžiuoja jie prieš mane, kai mano koja paslysta!” 
\par 17 Esu pasiruošęs kristi, mano kentėjimai nesiliauja. 
\par 18 Išpažinsiu savo kaltę, gailėsiuosi dėl savo nuodėmės. 
\par 19 Mano priešai gyvena ir yra galingi, ir daug tų, kurie nekenčia manęs neteisingai. 
\par 20 Kurie atlygina piktu už gera, yra mano priešai, nes seku gera. 
\par 21 Viešpatie, nepalik manęs! Mano Dieve, nebūk toli nuo manęs! 
\par 22 Skubėk padėti man, Viešpatie, mano gelbėtojau!



\chapter{39}


\par 1 Aš sakiau: “Saugosiu savo kelius, kad nenusidėčiau liežuviu; pažabosiu savo burną, kol nedorėlis tebėra priešais mane”. 
\par 2 Pasidariau visiškas nebylys, visko džiuginančio atsisakiau; mano skausmas pakilo, 
\par 3 įkaito širdis mano krūtinėje, bemąstant įsiliepsnojo; aš prabilau. 
\par 4 Viešpatie, leisk sužinoti mano pabaigą ir skaičių mano dienų, kad žinočiau, koks menkas aš esu. 
\par 5 Mano dienų tik sprindis, mano amžius kaip niekas Tavo akivaizdoje. Kaip kvapas yra žmogaus gyvenimas. 
\par 6 Kaip šešėlis vaikščioja žmogus, tuščiai stengiasi; krauna turtus ir nežino, kam jie atiteks. 
\par 7 Viešpatie, ko aš lauksiu? Mano viltis Tavyje. 
\par 8 Iš visų mano nusikaltimų išlaisvink mane. Nepadaryk manęs kvailojo pajuoka. 
\par 9 Pasidariau nebylys, neatveriu burnos, nes Tu tai padarei. 
\par 10 Atitrauk nuo manęs savo rūstybę. Nuo Tavo rankos smūgių aš nykstu. 
\par 11 Kai už kaltes baudi žmogų, suėdi kaip kandis, kas jam brangiausia. Žmogus­tik kvapas. 
\par 12 Viešpatie, išgirsk mano maldą, šauksmą mano išklausyk! Netylėk dėl mano ašarų! Aš esu tik svečias pas Tave, praeivis kaip visi mano tėvai. 
\par 13 Nugręžk nuo manęs savo žvilgsnį, kad atsigaučiau pirma, negu iškeliausiu ir pranyksiu.



\chapter{40}


\par 1 Kantriai laukiau Viešpaties, Jis pasilenkė prie manęs ir išgirdo mano šauksmą. 
\par 2 Jis ištraukė mane iš baisios duobės, iš klampaus purvo ir pastatė ant uolos mano kojas, sutvirtino mano žingsnius. 
\par 3 Jis įdėjo į mano lūpas naują giesmę­gyrių mūsų Dievui. Daugelis tai matys, bijosis ir pasitikės Viešpačiu. 
\par 4 Palaimintas žmogus, kuris pasitiki Viešpačiu, nesikreipia į išdidžiuosius ir neina melo keliais. 
\par 5 Daug padarei, Viešpatie, mano Dieve, nuostabių darbų ir nutarimų mūsų labui­nėra Tau lygaus nė vieno,­jei norėčiau juos paskelbti ir išpasakoti, jų būtų daugiau, kaip kad galima suskaičiuoti. 
\par 6 Aukų ir atnašų Tu nenorėjai. Ausis Tu man atvėrei. Deginamųjų aukų ir aukų už nuodėmę Tu nereikalavai. 
\par 7 Tada tariau: “Štai ateinu; knygos rietime apie mane parašyta. 
\par 8 Man patinka vykdyti Tavo valią, mano Dieve; Tavo įstatymas yra mano širdyje”. 
\par 9 Skelbiau teisumą didelėje minioje, lūpų neužčiaupiau, Viešpatie, Tu žinai. 
\par 10 Tavo teisumo nepaslėpiau savo širdyje, apie Tavo ištikimybę ir išgelbėjimą kalbėjau. Nenutylėjau apie Tavo malonę ir tiesą dideliame susirinkime. 
\par 11 Viešpatie, nesulaikyk man savo gailestingumo; Tavo malonė ir tiesa visada tegul mane lydi. 
\par 12 Užgriuvo mane nesuskaitomos blogybės, apniko kaltės, nieko daugiau nematau. Jų yra daugiau nei mano galvos plaukų, todėl netekau drąsos. 
\par 13 Viešpatie, teikis išgelbėti mane! Viešpatie, skubėk man padėti! 
\par 14 Tesusigėsta ir teparausta visi, kurie kėsinasi į mano gyvybę; teatsitraukia sugėdinti, kurie linki man pikta! 
\par 15 Tenusigąsta dėl savo gėdos, kurie man sako: “Gerai, gerai”. 
\par 16 Tedžiūgauja ir tesilinksmina, kurie ieško Tavęs; kurie ilgisi Tavo išgelbėjimo, tegul sako: “Didis yra Viešpats!” 
\par 17 Nors esu suvargęs ir beturtis, Viešpats rūpinasi manimi. Mano pagalba ir išlaisvintojas Tu esi. Mano Dieve, nedelsk!



\chapter{41}


\par 1 Palaimintas, kuris kreipia dėmesį į vargšą. Nelaimėje išgelbės jį Viešpats. 
\par 2 Viešpats saugos ir išlaikys jį gyvą; jis bus palaimintas žemėje. Tu neatiduosi jo priešų valiai. 
\par 3 Viešpats sustiprins jį ligos patale; Tu pagydysi jį nuo visų ligų. 
\par 4 Sakiau: “Viešpatie, būk man gailestingas! Išgydyk mano sielą, nes Tau nusidėjau!” 
\par 5 Mano priešai kalba prieš mane pikta: “Kai jis mirs, ir jo vardas išnyks”. 
\par 6 Jei kas ateina manęs aplankyti, tuščius žodžius kalba, išėjęs laukan apkalba. 
\par 7 Visi, kurie nekenčia manęs, šnibždasi prieš mane, planuoja man pakenkti: 
\par 8 “Pikčiausia liga jam prikibo, jis atsigulė ir nebeatsikels”. 
\par 9 Net ir artimas draugas, kuriuo pasitikėjau, kuris valgė mano duoną, taikosi man įspirti. 
\par 10 Viešpatie, būk man gailestingas, pakelk mane, kad jiems atlyginčiau! 
\par 11 Iš to žinosiu, jog esi man palankus, jei mano priešas nedžiūgaus prieš mane. 
\par 12 O mane Tu palaikai mano nekaltume ir amžiams pastatai savo akivaizdoje. 
\par 13 Palaimintas tebūna Viešpats, Izraelio Dievas, per amžių amžius! Amen! Amen!



\chapter{42}


\par 1 Kaip elnė geidžia upelio vandens, taip mano siela geidžia Tavęs, o Dieve! 
\par 2 Mano siela trokšta Dievo, gyvojo Dievo. Kada ateisiu ir pasirodysiu Dievo akivaizdoje? 
\par 3 Ašaros buvo man duona dieną ir naktį, kai jie kasdien man sakė: “Kur yra tavo Dievas?” 
\par 4 Kai prisimenu tai, išlieju savo sielą, nes traukdavau su minia į Dievo namus, džiūgaudamas ir dėkodamas iškilmingoje eisenoje. 
\par 5 Ko taip nusiminei, mano siela, ir ko nerimsti manyje? Lauk Dievo, nes aš dar girsiu Jį už Jo veido pagalbą! 
\par 6 Mano Dieve, mano siela liūdi manyje. Prisimenu Tave iš Jordano šalies ir Hermono, nuo Micaro kalno. 
\par 7 Gelmė šaukia gelmę, vandeniui triukšmingai krintant; Tavo bangos ir vilnys per mane liejas. 
\par 8 Dieną apreikš Viešpats savo malonę. Naktį giedosiu Jam, savo Dievui, kuris teikia man gyvybę. 
\par 9 Tarsiu Viešpačiui, savo uolai: “Kodėl pamiršai mane? Kodėl turiu vaikščioti nuliūdęs, spaudžiamas priešo?” 
\par 10 Kenčiu lyg kaulus laužant, priešai tyčiojasi iš manęs, kasdien klausdami: “Kur yra tavo Dievas?” 
\par 11 Ko taip nusiminei, mano siela, ir ko nerimsti manyje? Lauk Dievo, nes aš dar girsiu Jį, savo veido pagalbą ir savo Dievą.



\chapter{43}


\par 1 Dieve, teisk ir gink mano bylą prieš bedievių tautą! Gelbėk mane nuo klastingo ir neteisingo žmogaus. 
\par 2 Tu esi mano stiprybės Dievas. Kodėl atstumi mane? Kodėl turiu vaikščioti nuliūdęs, priešo spaudžiamas? 
\par 3 Siųsk savo šviesą ir tiesą! Jos teveda mane į Tavo šventąjį kalną, į Tavo palapinę. 
\par 4 Eisiu prie Dievo aukuro, pas Dievą, savo didžiausią linksmybę. Girsiu Tave arfa, o Dieve, mano Dieve! 
\par 5 Ko taip nusiminusi, mano siela, ir ko nerimsti manyje? Lauk Dievo, nes aš dar girsiu Jį, savo veido pagalbą ir savo Dievą!



\chapter{44}


\par 1 Dieve, savo ausimis girdėjome, kai mūsų tėvai pasakojo mums, kokius darbus darei jų dienomis senais laikais. 
\par 2 Pagonis išvarei, o juos įsodinai; išnaikinai tautas, o juos išaukštinai. 
\par 3 Ne savo kardu jie laimėjo kraštą, ne jų ranka išgelbėjo juos, bet Tavo dešinė, Tavo ranka ir Tavo veido šviesa, nes Tu juos mylėjai! 
\par 4 Tu­mano Karalius ir Dievas, Tu teiki pergalę Jokūbui. 
\par 5 Su Tavo pagalba prispausime savo priešus, Tavo vardu mindysime tuos, kurie kyla prieš mus. 
\par 6 Ne savo lanku pasitikėsiu, ne mano kardas išgelbės mane. 
\par 7 Tu išgelbėjai mus nuo mūsų priešų, mus nekenčiančius sugėdinai. 
\par 8 Dievu giriamės per dieną ir Tavo vardą šlovinsime per amžius. 
\par 9 Bet Tu mus atstūmei, pažeminti leidai ir su mūsų kariuomene nebeišeini. 
\par 10 Tu priverti mus trauktis nuo priešų, kurie nekenčia mūsų ir išnaudoja mus. 
\par 11 Kaip pjautinas avis atidavei mus, tarp pagonių išblaškei. 
\par 12 Savo tautą pardavei už nieką; nepraturtėjai tuo, ką už ją gavai. 
\par 13 Padarei mus paniekinimu kaimynams, pajuoka ir patyčiomis tiems, kurie aplinkui mus. 
\par 14 Padarei mus priežodžiu pagonims, tautos kraipo galvas dėl mūsų. 
\par 15 Nuolatos man prieš akis mano panieka ir rausta iš gėdos veidas 
\par 16 dėl priekaištų ir užgaulių žodžių, dėl priešo ir keršytojo. 
\par 17 Visa tai užgriuvo mus, nors Tavęs nepamiršome, sandoros su Tavimi nepažeidėme. 
\par 18 Mūsų širdis neatsitraukė nuo Tavęs, mūsų žingsniai nebuvo nuo Tavo kelių nukrypę, 
\par 19 kai leidai mus naikinti tyruose ir apdengei mirties šešėliu. 
\par 20 Jei būtume užmiršę savo Dievo vardą ir ištiesę rankas į svetimą dievą, 
\par 21 argi Dievas to nežinotų? Jis juk žino širdies paslaptis! 
\par 22 Dėl Tavęs esame visą dieną žudomi, laikomi tarsi pjauti paskirtos avys. 
\par 23 Kelkis! Kodėl miegi, Viešpatie? Pakilk, neatstumk amžinai! 
\par 24 Kodėl slepi savo veidą, pamiršti mūsų vargą ir priespaudą? 
\par 25 Mūsų siela parkritusi į dulkes; mūsų kūnas parblokštas žemėn. 
\par 26 Pakilk, padėk mums! Išgelbėk mus dėl savo gailestingumo.



\chapter{45}


\par 1 Mano širdis prisipildė gražių žodžių. Giedosiu giesmę Karaliui. Mano liežuvis­plunksna miklioje raštininko rankoje. 
\par 2 Tu esi gražiausias iš žmonių vaikų; malonė Tavo lūpose! Todėl palaimino Tave Dievas amžiams. 
\par 3 Karžygy, prisijuosk kalaviją prie šlaunies, savo šlovę ir didybę. 
\par 4 Savo didybėje kelkis ginti tiesos, romumo ir teisumo. Tavo dešinė tepamoko Tave didingų darbų! 
\par 5 Karaliaus priešų širdis perveria Jo strėlės, nuo jų krinta tautos. 
\par 6 Tavo sostas, o Dieve, stovės per amžius. Teisumo skeptras yra Tavo karalystės skeptras. 
\par 7 Tu pamėgai teisumą ir neapkentei nedorybės, todėl patepė Tave Dievas, Tavasis Dievas, džiaugsmo aliejumi daugiau negu Tavo bendrus. 
\par 8 Mira, alaviju ir kasija kvepia Tavo drabužiai. Iš dramblio kaulo rūmų linksmina Tave. 
\par 9 Karalių dukterys pasitinka Tave; karalienė Tavo dešinėje Ofyro auksu papuošta. 
\par 10 Klausyk, dukra, pažvelk ir išgirsk! Pamiršk savo tautą ir tėvo namus. 
\par 11 Karalius geidžia tavo grožio. Jis yra tavo Viešpats­garbink Jį! 
\par 12 Tyro dukros neša Tau dovanas, tautos kilmingieji Tavo palankumo ieško. 
\par 13 Karaliaus dukra šlovinga viduje, auksu išmegzti jos apdarai. 
\par 14 Išsiuvinėtais drabužiais papuoštą ją veda pas Karalių, ją seka mergaitės, jos draugės, kurios bus atvestos pas Tave. 
\par 15 Jos eina džiūgaudamos ir besilinksmindamos, jos įeis į Karaliaus rūmus. 
\par 16 Vietoj Tavo tėvų bus Tavo sūnūs; visoje šalyje Tu paskirsi juos kunigaikščiais. 
\par 17 Aš garsinsiu Tavo vardą per kartų kartas, todėl tautos girs Tave per amžius.



\chapter{46}


\par 1 Dievas mums yra prieglauda ir stiprybė, užtikrinta pagalba bėdoje. 
\par 2 Todėl nebijosime, nors žemė drebėtų, kalnai griūtų į jūros gelmę. 
\par 3 Teūžia, teputoja jų vandenys, tedreba kalnai, joms šėlstant. 
\par 4 Upės srovės linksmina Dievo miestą, šventą Aukščiausiojo buveinę. 
\par 5 Dievas yra jo viduje, jam nėra ko bijoti; rytui auštant ateis Dievas jam padėti. 
\par 6 Siautė pagonys, maištavo karalystės; Jis prabilo, ir sustingo žemė. 
\par 7 Kareivijų Viešpats yra su mumis, tvirtovė mums yra Jokūbo Dievas. 
\par 8 Ateikite, pažvelkite į Viešpaties darbus, kokių baisių dalykų Jis padarė žemėje! 
\par 9 Jis sustabdo karus ligi pat žemės krašto, sulaužo lankus ir sutrupina ietis, sudegina ugnimi karo vežimus. 
\par 10 “Liaukitės ir žinokite, jog Aš esu Dievas! Aš būsiu išaukštintas tarp pagonių, išaukštintas žemėje!” 
\par 11 Kareivijų Viešpats yra su mumis, tvirtovė mums yra Jokūbo Dievas.



\chapter{47}


\par 1 Plokite rankomis visos tautos! Šaukite Dievui džiugesio balsu! 
\par 2 Aukščiausiasis Viešpats yra baisus, didis Karalius visos žemės. 
\par 3 Jis pajungė mums pagonis, padėjo po mūsų kojomis tautas. 
\par 4 Jis parinko mums paveldėti žemę­garbę Jo mylimojo Jokūbo. 
\par 5 Dievas užžengė su šauksmu, Viešpats, gaudžiant trimitams. 
\par 6 Giedokite gyrių mūsų Dievui, giedokite! Giedokite gyrių mūsų Karaliui, giedokite! 
\par 7 Dievas yra visos žemės Karalius! Giedokite Jam gyrių suprasdami. 
\par 8 Dievas karaliauja pagonims, Dievas sėdi savo šventajame soste. 
\par 9 Tautos kunigaikščiai susirinko su Abraomo Dievo tauta. Dievui priklauso visi žemės skydai, Jis yra labai išaukštintas.



\chapter{48}


\par 1 Didis yra Viešpats ir labai girtinas mūsų Dievo mieste, savo šventajame kalne. 
\par 2 Gražiai iškilęs, visos žemės džiaugsmas yra Siono kalnas šiaurės pusėje, didžiojo Karaliaus miestas. 
\par 3 Dievas yra jo rūmų apsauga. 
\par 4 Antai karaliai susirinkę praėjo kartu. 
\par 5 Pamatę apstulbo, sumišo ir skubiai pabėgo. 
\par 6 Baimė apėmė juos ten ir skausmai kaip gimdyvę. 
\par 7 Rytų vėju Tu sudaužei Taršišo laivus. 
\par 8 Ką buvome girdėję, tai ir matėme kareivijų Viešpaties, mūsų Dievo, mieste. Dievas išlaiko jį per amžius. 
\par 9 Dieve, būdami Tavo šventykloje, prisiminėme Tavo malonę. 
\par 10 Dieve, kaip Tavo vardas, taip ir Tavo šlovė pasiekia žemės pakraščius. Tavoji dešinė pilna teisumo. 
\par 11 Tesidžiaugia Siono kalnas! Tedžiūgauja Judo dukterys dėl Tavo sprendimų. 
\par 12 Apeikite aplinkui Sioną, apžiūrėkite jį, suskaičiuokite jo bokštus. 
\par 13 Įsidėmėkite jį supantį pylimą, išvaikščiokite jo rūmus, kad galėtumėte papasakoti būsimosioms kartoms. 
\par 14 Nes šis Dievas yra mūsų Dievas per amžius, Jis ves mus iki mirties.



\chapter{49}


\par 1 Išgirskite visos tautos! Klausykitės visi pasaulio gyventojai: 
\par 2 prastuoliai ir kilmingieji, turtuoliai ir vargšai! 
\par 3 Savo burna skelbsiu išmintį; mano širdies apmąstymai­išmanymas. 
\par 4 Aš klausysiuos patarlių, skambant arfai įminsiu mįslę. 
\par 5 Ko gi man nelaimės dienomis bijoti, kai priešai klastingi apninka, 
\par 6 kurie savo turtais pasitiki ir giriasi gausiais savo lobiais? 
\par 7 Nė vienas žmogus negalės išpirkti savo brolio nė Dievui duoti išpirką už jį. 
\par 8 Didelė kaina už sielos išpirkimą­tiek niekad neturėsi, 
\par 9 kad galėtum amžinai gyventi ir nematytum sugedimo. 
\par 10 Matysi, kaip išminčiai miršta, kvailiai ir paikieji žūna, palikdami turtus kitiems. 
\par 11 Jie nori, kad jų namai pasiliktų per amžius, jų buveinės kartų kartoms, savo vardais jie pavadina žemes. 
\par 12 Net ir garbingas žmogus neišlieka: jis panašus į galviją, kuris pražūna. 
\par 13 Toks yra kvailai pasitikinčiųjų likimas ir galas jų pasekėjų, kurie pritaria jiems. 
\par 14 Mirtis juos ganys kaip avis. Josios buveinei jie skirti ir į ją nužengs. Jų kūnas sunyks, pavidalas sudūlės; mirusiųjų buveinė bus jų namai. 
\par 15 Tačiau Dievas iš kapo išpirks mano sielą, Jis priims mane. 
\par 16 Nesijaudink, jei kas praturtėja ir jo namų garbė padidėja. 
\par 17 Juk mirdamas jis to nepasiims, garbė nepalydės jo. 
\par 18 Nors gyvendamas jis tarsis esąs laimingas, kiti jį dėl sėkmės girs, 
\par 19 tačiau jis nužengs pas savo tėvų kartą ir šviesos neregės per amžius. 
\par 20 Žmogus, kuris yra gerbiamas, bet neturi supratimo, yra panašus į galviją, kuris pražus.



\chapter{50}


\par 1 Galingas Dievas, Viešpats, kalbėjo ir šaukė žemei nuo saulėtekio iki saulėlydžio. 
\par 2 Iš Siono, grožio tobulumo, suspindėjo Dievas. 
\par 3 Mūsų Dievas ateis ir netylės: naikinanti ugnis eis pirma Jo, o aplinkui Jį siaus audros. 
\par 4 Jis šaukia dangų iš aukštybių ir žemę, kad galėtų teisti savo tautą: 
\par 5 “Surinkite mano šventuosius, padariusius sandorą su manimi per auką”. 
\par 6 Dangus skelbs Jo teisumą, nes pats Dievas yra teisėjas. 
\par 7 “Klausyk, mano tauta, Aš kalbėsiu! Izraeli, Aš liudysiu prieš tave! Dievas, tavo Dievas, Aš esu! 
\par 8 Ne dėl aukų barsiu tave­deginamąsias aukas visada man aukojai. 
\par 9 Man nereikia veršio iš tavo tvarto, nei ožio iš tavo bandos. 
\par 10 Mano yra visi miškų žvėrys, gyvuliai ant tūkstančio kalvų. 
\par 11 Pažįstu visus kalnų paukščius, ir laukiniai žvėrys yra mano žinioje. 
\par 12 Jei alkanas būčiau, nesakyčiau tau, nes mano yra pasaulis ir visa, kas jame. 
\par 13 Argi Aš valgysiu jaučių mėsą, argi gersiu ožių kraują? 
\par 14 Aukok Dievui padėką ir ištesėk Aukščiausiajam įžadus. 
\par 15 Šaukis manęs nelaimės dieną­tai išgelbėsiu tave, o tu šlovinsi mane”. 
\par 16 O nedorėliui Dievas sako: “Kodėl tu mano nuostatus skelbi ir savo burna mano sandorą mini? 
\par 17 Nes tu nekenti pamokymo ir atmeti mano žodžius. 
\par 18 Pamatęs vagį, susitari su juo ir su svetimautojais draugauji. 
\par 19 Tavo burna kalba pikta ir tavo liežuvis sako klastą. 
\par 20 Tu kalbi prieš savo brolį ir šmeiži savo motinos sūnų. 
\par 21 Tu tai darei, ir Aš tylėjau. Tu manai, kad Aš esu toks, kaip tu. Aš tave barsiu ir tavo darbus statysiu tau prieš akis. 
\par 22 Susipraskite, kurie pamiršote Dievą, kad nesudraskyčiau jūsų, ir tada nebus, kas jus išgelbėtų. 
\par 23 Kas aukoja gyrių, pašlovina mane; o kuris teisingai elgiasi, tam parodysiu Dievo išgelbėjimą”.



\chapter{51}


\par 1 Dieve, pasigailėk manęs dėl savo malonės, dėl savo beribio gerumo panaikink mano kaltes. 
\par 2 Visai nuplauk mano kaltę ir apvalyk mano nuodėmes. 
\par 3 Išpažįstu savo nusikaltimą, mano nuodėmė visada yra su manimi. 
\par 4 Tau vienam nusidėjau ir padariau pikta Tavo akyse. Tu teisingai teisi ir teisingą sprendimą darai. 
\par 5 Štai aš gimiau nuodėmingas, ir nuodėmėje mane pradėjo mano motina. 
\par 6 Tu mėgsti tiesą širdyje ir slaptoje mokai mane išminties. 
\par 7 Apšlakstyk mane yzopu, kad būčiau švarus. Nuplauk mane, kad būčiau baltesnis už sniegą. 
\par 8 Leisk man patirti džiaugsmą ir linksmybę. Tedžiūgauja mano sužeisti kaulai. 
\par 9 Nugręžk savo veidą nuo mano nuodėmių ir visas mano kaltes išdildyk. 
\par 10 Dieve, tyrą širdį sutverk manyje ir teisingą dvasią atnaujink. 
\par 11 Neatstumk manęs nuo savo veido ir savo šventos dvasios neatimk nuo manęs. 
\par 12 Grąžink man išgelbėjimo džiaugsmą ir laisvės dvasia sustiprink mane. 
\par 13 Tada mokysiu nusidėjėlius Tavo kelių, kad nusikaltėliai grįžtų pas Tave. 
\par 14 Dieve, išlaisvink mane nuo kraujo kaltės, nes Tu mano išgelbėjimo Dievas, ir mano liežuvis šlovins Tavo teisumą. 
\par 15 Viešpatie, atverk mano lūpas, ir mano burna skelbs Tavąją šlovę. 
\par 16 Tu nenori aukos, jei aukočiau deginamąją auką, Tau nepatiktų. 
\par 17 Auka Dievui yra sudužusi dvasia; sudužusios ir nusižeminusios širdies Tu, Dieve, nepaniekinsi. 
\par 18 Būk palankus ir daryk gera Sionui, statyk Jeruzalės sienas. 
\par 19 Tada Tu gėrėsies teisumo aukomis, aukosime veršius ant Tavo aukuro.



\chapter{52}


\par 1 Ko giries nedorybe, galiūne? Dievo gerumas pasilieka nuolat. 
\par 2 Tavo liežuvis planuoja pražūtį kaip aštrus peilis, tu klastadary! 
\par 3 Tu mėgsti pikta labiau kaip gera, tau mieliau meluoti negu teisybę kalbėti. 
\par 4 Tu mėgsti pražūtingas kalbas, klastingas liežuvi! 
\par 5 Todėl sunaikins tave Dievas amžiams; parblokš ir išmes iš palapinės, su šaknimis išraus iš gyvųjų žemės. 
\par 6 Teisieji tai matys ir bijosis, jie juoksis iš jo: 
\par 7 “Štai žmogus, kuris Dievo nepadarė savo stiprybe, bet, pasitikėdamas turtais, įsidrąsino daryti nedorybes”. 
\par 8 Aš esu kaip žaliuojąs alyvmedis Dievo namuose, pasitikiu Dievo gailestingumu per amžių amžius. 
\par 9 Tave girsiu per amžius, nes Tu tai padarei; skelbsiu Tavo brangų vardą šventiesiems.



\chapter{53}


\par 1 Kvailys pasakė savo širdyje: “Nėra Dievo”. Jie sugedo, elgiasi bjauriai, nėra, kas gera darytų. 
\par 2 Dievas pažiūrėjo iš dangaus į žmones, kad pamatytų, ar yra, kas išmano ir ieško Dievo. 
\par 3 Jie visi atsitraukė, visi kartu sugedo. Nėra darančio gera, nėra nė vieno. 
\par 4 Argi nesupranta piktadariai, kurie ryja mano tautą lyg duoną? Ir Dievo jie nesišaukia. 
\par 5 Jie drebėjo iš baimės, kai nebuvo ko bijoti. Dievas išsklaidė kaulus tų, kurie tave apgulė. Jie buvo sugėdinti, nes Dievas paniekino juos. 
\par 6 O kad ateitų iš Siono išgelbėjimas Izraeliui! Kai Dievas parves savo tautos belaisvius, džiaugsis Jokūbas, linksminsis Izraelis!



\chapter{54}


\par 1 Dieve, savo vardu išgelbėk mane ir savo galybe teisk mane. 
\par 2 Dieve, išgirsk mano maldą, išklausyk mano burnos žodžius! 
\par 3 Svetimieji sukilo prieš mane, prispaudėjai ieško mano sielos; jie nepaiso Dievo. 
\par 4 Bet Dievas yra mano padėjėjas, Viešpats palaiko mano sielą. 
\par 5 Atlygink piktu mano priešams, savo tiesoje sunaikink juos! 
\par 6 Tada laisvai aukas Tau aukosiu, girsiu Tavo vardą, nes Jis geras. 
\par 7 Tu iš visų bėdų išgelbėjai mane, ir mano akys matė sugėdintus priešus.



\chapter{55}


\par 1 Dieve, klausykis mano maldos ir nesišalink nuo mano maldavimo! 
\par 2 Pažvelk į mane ir išklausyk. Aš blaškaus ir nerimstu 
\par 3 dėl priešo balso, dėl nedorėlių siautimo. Jie daro man pikta, užsirūstinę neapkenčia manęs. 
\par 4 Širdis dreba mano krūtinėje, mirties siaubai apėmė mane. 
\par 5 Mane užklupo baimė ir drebulys, siaubas užpuolė mane. 
\par 6 Aš sakiau: “O kad turėčiau balandžio sparnus; išskrisčiau ir būčiau ramus. 
\par 7 Toli nuskrisčiau, dykumoje apsinakvočiau. 
\par 8 Skubėčiau pasislėpti nuo viesulų ir audrų”. 
\par 9 Viešpatie, suardyk ir sumaišyk jų kalbas! Mieste mačiau tik smurtą ir vaidus. 
\par 10 Dieną ir naktį jie slankioja aplink jo sienas, o viduje neteisybė ir priespauda. 
\par 11 Nedorybė viduryje, apgaulė ir klasta gatvėse. 
\par 12 Jei priešas mane užgauliotų, galėčiau pakęsti. Jei tas, kuris nekenčia manęs, prieš mane pakiltų, pasislėpčiau nuo jo. 
\par 13 Bet tu­žmogus man lygus, mano bendras, artimas bičiulis! 
\par 14 Mums buvo malonu kartu, minioje eidavome į Dievo namus. 
\par 15 Juos mirtis teužklumpa! Gyvi į mirusiųjų buveinę tenužengia! Nedorybės jų buveinėse ir tarp jų. 
\par 16 Aš šauksiuosi Dievo, ir Viešpats išgelbės mane. 
\par 17 Vakare, rytą ir vidudienį melsiuosi ir garsiai šauksiu; Jis išgirs mano balsą, 
\par 18 išvaduos mano sielą, grąžins ramybę, apgins nuo puolančių priešų daugybės. 
\par 19 Dievas išgirs ir pažemins juos, Jis gyvena nuo amžių. Jie nesikeičia ir Dievo nebijo. 
\par 20 Jie pakelia ranką prieš tuos, kurie yra taikoje su jais, laužo duotąjį žodį. 
\par 21 Slidesnė už sviestą jų burna, o širdyse karas; žodžiai švelnesni už aliejų, tačiau jie yra nuogi kardai. 
\par 22 Pavesk Viešpačiui savo naštą, ir Jis palaikys tave, Jis niekados neleis teisiajam svyruoti. 
\par 23 Tu, Dieve, juos nuvesi į gilią prarają. Žmogžudžiai ir apgavikai žus nė pusės amžiaus nesulaukę. Tačiau aš pasitikėsiu Tavimi.



\chapter{56}


\par 1 Dieve, būk man gailestingas, nes žmogus nori praryti mane, priešai nuolatos puola. 
\par 2 Visą laiką mano priešai puola mane, daug tų, kurie kovoja prieš mane, o Aukščiausiasis! 
\par 3 Kai baimė apima, Tavimi pasitikiu. 
\par 4 Dievu pasitikiu, kurio žodį giriu. Nebijosiu, ką gali padaryti man žmogus? 
\par 5 Kiekvieną dieną jie iškraipo mano žodžius, visos jų mintys­daryti man pikta. 
\par 6 Jie susirenka, tyko, seka mano pėdomis, kėsinasi į mano gyvybę. 
\par 7 Atlygink jiems už nedorybę! Dieve, užsirūstinęs parblokšk juos! 
\par 8 Tu mano vargo dienas skaičiuoji, renki mano ašaras į odinę. Argi jų nėra Tavo knygoje? 
\par 9 Mano priešai pasitrauks, kai šauksiuos Tavęs. Žinau, kad Dievas už mane. 
\par 10 Dievu, kurio žodį giriu, Viešpačiu, kurio žodį giriu, 
\par 11 aš pasitikiu. Nebijosiu, ką gali padaryti man žmogus? 
\par 12 Dieve, Tau duotus įžadus ištesėsiu, aukosiu gyriaus aukas. 
\par 13 Tu išgelbėjai mane iš mirties, mano kojas nuo suklupimo, kad vaikščiočiau prieš Dievą ir turėčiau gyvenimo šviesą.



\chapter{57}


\par 1 Pasigailėk manęs, Dieve, pasigailėk. Mano siela pasitiki Tavimi. Tavo sparnų pavėsyje slepiuos, kol praeis nelaimė. 
\par 2 Šauksiuosi aukščiausiojo Dievo, kuris man gera daro. 
\par 3 Jis pasiųs iš dangaus ir išgelbės mane, Jis paniekins mano prispaudėjus, Dievas pasiųs savo gailestingumą ir tiesą! 
\par 4 Turiu gyventi tarp liūtų, tarp žmonių, kvėpuojančių ugnimi. Jų dantys yra ietys ir strėlės, jų liežuviai­aštrūs kardai. 
\par 5 Dieve, būk išaukštintas virš dangų! Tavo šlovė teišplinta visoje žemėje! 
\par 6 Jie spendė pinkles mano kojoms, mano siela sugniužo. Jie kasė man duobę, tačiau patys įpuolė į ją. 
\par 7 Dieve, mano širdis tvirta. Taip, mano širdis tvirta. Aš giedosiu ir girsiu. 
\par 8 Pabusk, mano šlove! Pabuskite, arfa ir psalteri! Aš atsikelsiu anksti. 
\par 9 Viešpatie, girsiu Tave tautose, giedosiu Tau pagonių būry. 
\par 10 Tavo gailestingumas siekia dangų, ir Tavo tiesa­debesis. 
\par 11 Dieve, būk išaukštintas virš dangų! Teišplinta Tavo šlovė visoje žemėje!



\chapter{58}


\par 1 Jūs, teisėjai, ar sprendžiate teisingai? Ar teisingai teisiate žmones? 
\par 2 Ne, jūs darote nedorybes širdyje, jūsų rankos smurtą sėja. 
\par 3 Nedorėliai klysta nuo pat kūdikystės, nuo pat gimimo jie klaidžioja ir meluoja. 
\par 4 Jų nuodai panašūs į gyvatės nuodus, kaip angies, kuri užsikemša ausis, 
\par 5 kad negirdėtų labiausiai įgudusio kerėtojo balso. 
\par 6 Dieve, sutrupink jiems dantis burnoje, Viešpatie, išdaužyk iltis jauniems liūtams! 
\par 7 Tepradingsta jie kaip tekantis vanduo, tesulūžta jų strėlės, jiems betaikant. 
\par 8 Teištyžta jie kaip sraigė, kaip nelaiku gimęs kūdikis tenemato saulės! 
\par 9 Tenuneša juos audra greičiau, negu puodas pajus degančių erškėčių karštį. 
\par 10 Teisusis džiaugsis, matydamas atpildą, nedorėlio krauju plausis sau kojas. 
\par 11 Tada žmonės sakys: “Tikrai, atpildą gauna teisusis! Tikrai yra Dievas, teisėjas žemėje!”



\chapter{59}


\par 1 Dieve, išgelbėk mane iš priešų, apgink nuo sukylančių prieš mane! 
\par 2 Išlaisvink mane iš piktadarių ir išgelbėk nuo kraugerių! 
\par 3 Štai jie tyko mano gyvybės; galiūnai susirinkę prieš mane ne dėl mano nusikaltimo ar nuodėmės, o Viešpatie! 
\par 4 Nors nekaltas esu, jie atbėga ir ginkluojasi prieš mane. Pabusk man padėti ir pamatyk! 
\par 5 Viešpatie, kareivijų Dieve, Izraelio Dieve, pabusk aplankyti visas pagonių tautas, nesigailėk nedorų piktadarių. 
\par 6 Jie sugrįžta vakare, loja kaip šunys ir laksto po miestą. 
\par 7 Jie pliauškia savo burna, kardai jų lūpose; jie sako: “Juk niekas negirdi!” 
\par 8 Viešpatie, Tu juokies iš jų, tyčiojies iš pagonių. 
\par 9 Mano stiprybe, lauksiu Tavęs, nes Dievas yra mano tvirtovė. 
\par 10 Gailestingas mano Dievas pasitiks mane. Dievas leis man pasijuokti iš priešų. 
\par 11 Nenužudyk jų, kad tauta nepamirštų! Išsklaidyk juos savo galia ir parblokšk, Viešpatie, mano skyde. 
\par 12 Jų burnos nusikaltimas ir lūpų žodžiai tesugauna juos jų išdidume. Jų keiksmai ir melai 
\par 13 teiššaukia rūstybę ir juos visai tesunaikina, kad visi žinotų, jog Jokūbo Dievas viešpatauja iki žemės pakraščių. 
\par 14 Tesugrįžta jie vakare, tegu loja kaip šunys ir laksto po miestą. 
\par 15 Teklaidžioja, ieškodami maisto, ir testaugia, jo neradę. 
\par 16 Aš giedosiu apie Tavo stiprybę, garsiai giedosiu rytmetį apie Tavo gailestingumą. Tu buvai man tvirtovė ir priebėga vargų metu. 
\par 17 Mano stiprybe, Tau giedosiu, nes Dievas yra mano tvirtovė, mano gailestingasis Dievas.



\chapter{60}


\par 1 Dieve, Tu užsirūstinęs atstūmei mus ir išsklaidei mus; atsigręžk į mus vėl! 
\par 2 Sudrebinai ir suskaldei žemę; užtaisyk jos plyšius, nes ji svyruoja. 
\par 3 Leidai patirti savo tautai sunkių vargų, mus girdei svaiginančiu vynu. 
\par 4 Tavęs bijantiesiems davei vėliavą, kad iškeltų ją dėl tiesos. 
\par 5 Kad Tavo mylimieji būtų išgelbėti, padėk savo dešine ir išklausyk mus! 
\par 6 Dievas kalbėjo savo šventume: “Aš džiūgausiu ir išdalinsiu Sichemą, paskirstysiu Sukotų slėnį. 
\par 7 Mano yra Gileadas ir Manasas, Efraimas­mano galvos šalmas, Judas­mano skeptras. 
\par 8 Moabas yra mano praustuvė. Ant Edomo numesiu savo kurpę. Filistija, džiūgauk dėl manęs”. 
\par 9 Kas įves mane į sustiprintą miestą? Kas nuves mane į Edomą? 
\par 10 Argi ne Tu, Dieve, atstūmei mus? Argi ne Tu, Dieve, neišėjai su mūsų kariuomene? 
\par 11 Suteik mums pagalbą varge, nes žmonių pagalba yra be vertės. 
\par 12 Su Dievu mes būsime drąsūs, Jis sumindys mūsų priešus.



\chapter{61}


\par 1 Dieve, išgirsk mano šauksmą, išklausyk mano maldą! 
\par 2 Mano širdžiai alpstant, nuo žemės krašto šaukiuos Tavęs. Užkelk mane ant aukštos uolos. 
\par 3 Tu esi man prieglauda, stiprus bokštas gintis nuo priešo. 
\par 4 Norėčiau gyventi Tavo palapinėje amžinai, pasitikėti Tavo sparnų priedanga. 
\par 5 Dieve, Tu girdėjai mano įžadus, davei paveldėjimą su tais, kurie bijo Tavo vardo. 
\par 6 Prailgink karaliaus gyvenimą, jo metai tegul tęsiasi per kartų kartas. 
\par 7 Tepasilieka jis per amžius Dievo akivaizdoje, Jo gailestingumas ir tiesa tesaugo jį! 
\par 8 Giedosiu gyrių Tavo vardui visados, kasdien vykdysiu savo įžadus.



\chapter{62}


\par 1 Dievo laukia mano siela, iš Jo ateina man išgelbėjimas. 
\par 2 Tik Jis yra mano uola ir išgelbėjimas, mano tvirtovė­aš nesvyruosiu. 
\par 3 Ar ilgai pulsite žmogų? Pražūsite visi kaip palinkusi siena, kaip griūvantis bokštas! 
\par 4 Jie planuoja nustumti jį nuo aukštumos. Jie gėrisi melu: liežuviais jie laimina, o viduje keikia. 
\par 5 Tik Dievo lauk, mano siela. Mano viltis yra Jame. 
\par 6 Tik Jis yra mano uola ir išgelbėjimas, mano tvirtovė­aš nesvyruosiu. 
\par 7 Dieve mano išgelbėjimas ir garbė; mano stiprybės uola ir priebėga yra Dievas. 
\par 8 Pasitikėkite Juo, žmonės, visais laikais! Išliekite Jo akivaizdoje savo širdį. Dievas yra mums apsauga. 
\par 9 Tik garas yra prastuoliai, melas­kilmingieji. Jeigu juos pasvertume, jie visi drauge lengvesni už nieką. 
\par 10 Nepasitikėkite priespauda, tuščiai nesivilkite grobiu. Jei didėja turtai, nepririškite prie jų savo širdies. 
\par 11 Kartą Dievas kalbėjo, du kartus girdėjau tai: galybė priklauso Dievui 
\par 12 ir Tavo, Viešpatie, yra gailestingumas. Tu atlygini kiekvienam pagal jo darbus.



\chapter{63}


\par 1 Dieve, Tu esi mano Dievas! Nuo ankstaus ryto Tavęs ieškau, Tavęs trokšta mano siela, kūnas ilgisi Tavęs kaip sausa ir nualinta žemė be vandens. 
\par 2 Šventykloje ieškojau Tavęs, pamačiau Tavo galybę ir šlovę. 
\par 3 Tavo malonė yra geresnė už gyvenimą, todėl mano lūpos girs Tave. 
\par 4 Šlovinsiu Tave, kol gyvensiu, Tavo vardą minėdamas, kelsiu į Tave rankas. 
\par 5 Mano siela bus pasotinta kaip kaulų smegenimis ir riebalais, lūpos džiaugsmingai girs Tave, 
\par 6 kai prisiminsiu Tave savo lovoje, mąstysiu apie Tave budėdamas naktį. 
\par 7 Tu buvai man pagalba, todėl aš džiūgausiu Tavo sparnų pavėsyje. 
\par 8 Mano siela įsikibo į Tave; Tavo dešinė palaiko mane. 
\par 9 Kurie siekia atimti man gyvybę, nueis į žemės gelmes. 
\par 10 Jie kris nuo kardo ir taps grobiu šakalams. 
\par 11 O karalius džiaugsis Dievu. Girsis kiekvienas, kuris prisiekia Juo, bet melagių burna bus užkimšta.



\chapter{64}


\par 1 Dieve, išgirsk mano balsą, kai meldžiuosi! Nuo priešų baimės saugok mano gyvybę. 
\par 2 Paslėpk mane nuo piktadarių sąmokslo, nuo nedorėlių sukilimo. 
\par 3 Jie galanda savo liežuvį kaip kardą, taiko užnuodytą žodį kaip strėlę, 
\par 4 kad iš pasalų galėtų šauti į nekaltąjį. Jie šauna netikėtai ir nebijo. 
\par 5 Jie drąsina save, darydami pikta, slaptai spendžia pinkles, mąstydami: “Kas tai matys?” 
\par 6 Jie sumano nedorybes, viską gerai apmąstydami, jų vidus ir širdis­paslaptinga gilybė. 
\par 7 Bet Dievas šaus į juos strėle, staiga juos sužeis. 
\par 8 Juos pražudys jų pačių liežuvis. Visi, kurie matys juos, šalinsis nuo jų. 
\par 9 Tada visi išsigąs ir skelbs, ką Dievas padarė, nes supras, kad tai Jo darbas. 
\par 10 Teisusis Viešpačiu džiaugsis ir pasitikės Juo, džiūgaus visi tiesiaširdžiai.



\chapter{65}


\par 1 Sione, o Dieve, Tau priklauso gyrius ir Tau bus įvykdyti įžadai. 
\par 2 Tu, kuris išklausai maldas, pas Tave ateis kiekvienas kūnas. 
\par 3 Nusikaltimai yra stipresni už mane, Tu apvalai mus nuo mūsų kalčių. 
\par 4 Palaimintas žmogus, kurį Tu išrenki ir priimi, kad jis gyventų Tavo kiemuose. Mes sotinsimės gėrybėmis Tavo namų, Tavo šventos šventyklos. 
\par 5 Tu nuostabiais ženklais atsakai mums, mūsų išgelbėjimo Dieve, pasitikėjime visų žemės pakraščių ir tolimiausių pajūrių! 
\par 6 Tu padėjai kalnus savo jėga, apsisiautęs galybe, 
\par 7 Tu nutildai jūrų ūžimą, jų bangų šniokštimą ir tautų triukšmą. 
\par 8 Žemės pakraščių gyventojai bijosi Tavo ženklų, rytus ir vakarus pripildai džiaugsmo. 
\par 9 Tu aplankai žemę ir palaistai ją, padarai ją labai derlingą. Dievo upė yra pilna vandens. Tu leidi užderėti javams; taip Tu viską paruoši. 
\par 10 Tu palaistai jos vagas, sulygini grumstus, lietumi suminkštini ją, palaimini želmenis. 
\par 11 Tu apvainikuoji metus derliumi, Tavo takai pilni riebalų. 
\par 12 Žaliuoja tyrų ganyklos, džiūgauja pasipuošusios kalvos. 
\par 13 Pievos pilnos avių, slėniuose vešliai auga javai. Jie linksmai šūkauja ir gieda.



\chapter{66}


\par 1 Džiaugsmingai šauk Dievui, visa žeme! 
\par 2 Giedokite apie Jo vardo garbę, šlovinkite ir girkite Jį. 
\par 3 Sakykite Dievui: “Kokie didingi Tavo darbai! Dėl didelės Tavo galios lenkiasi Tau priešai. 
\par 4 Visa žemė tegarbina Tave, tegieda Tau ir tegarsina Tavo vardą!” 
\par 5 Ateikite ir matykite Dievo darbus! Jis baisių dalykų padarė tarp žmonių! 
\par 6 Jis pavertė jūrą sausuma, per upę pėsčius pervedė. Tad džiaukimės Juo! 
\par 7 Jis viešpatauja savo galybe per amžius. Jo akys stebi tautas­ maištininkai tegul nesididžiuoja! 
\par 8 Laiminkite, tautos, mūsų Dievą, tebūna girdimas Jo gyriaus garsas. 
\par 9 Jis mūsų gyvybę išlaiko, neleidžia suklupti mūsų kojai. 
\par 10 Tu ištyrei mus, Dieve, apvalei mus ugnimi kaip sidabrą. 
\par 11 Tu įvedei mus į spąstus, uždėjai sunkią naštą mums ant strėnų. 
\par 12 Tu leidai žmonėms joti mums per galvas. Mes turėjome eiti per ugnį ir vandenį. Bet Tu išvedei mus į laisvę. 
\par 13 Su aukomis į Tavo namus aš įeisiu ir ištesėsiu savo įžadus, 
\par 14 kuriuos ištarė mano lūpos ir varge burna pažadėjo. 
\par 15 Aukosiu Tau riebias deginamąsias aukas, avinų taukus deginsiu, paruošiu jaučius ir ožius. 
\par 16 Ateikite, klausykite visi, kurie bijote Dievo, papasakosiu, ką Jis padarė mano sielai. 
\par 17 Jo šaukiausi savo burna, aukštinau Jį savo liežuviu. 
\par 18 Jei nedorai elgtis būčiau ketinęs širdyje, tai Viešpats nebūtų išklausęs. 
\par 19 Bet Dievas tikrai išklausė, Jis priėmė manąją maldą. 
\par 20 Palaimintas Dievas, kuris neatmetė mano maldos ir neatitolino nuo manęs savo gailestingumo.



\chapter{67}


\par 1 Dieve, būk mums gailestingas ir laimink mus, parodyk šviesų savo veidą, 
\par 2 kad pažintų žemėje Tavąjį kelią, visose tautose Tavo išgelbėjimą! 
\par 3 Dieve, tegiria Tave tautos, tegiria Tave visos tautos! 
\par 4 Tesilinksmina ir tegieda iš džiaugsmo tautos, kad teisi jas teisingai ir valdai tautas pasaulyje! 
\par 5 Dieve, tegiria Tave tautos, tegiria Tave visos tautos! 
\par 6 Žemė davė derlių, nes mus laimino Dievas, mūsų Dievas. 
\par 7 Telaimina mus Dievas ir tebijo Jo visi žemės pakraščiai.



\chapter{68}


\par 1 Tepakyla Dievas, tebūna išsklaidyti Jo priešai, tebėga nuo Jo, kurie Jo nekenčia. 
\par 2 Kaip dūmai išsklaidomi, kaip vaškas nuo ugnies sutirpsta, taip tepradingsta nedorėliai Dievo akivaizdoje. 
\par 3 Teisieji tegu linksminasi ir džiūgauja prieš Dievą, tedžiūgauja neapsakomai. 
\par 4 Giedokite Dievui! Girkite Jo vardą! Taisykite kelią Tam, kuris važinėja virš debesų! Jo vardas yra Viešpats. Džiūgaukite Jo akivaizdoje! 
\par 5 Našlaičių tėvas ir našlių globėjas savo šventoje buveinėje yra Dievas. 
\par 6 Dievas parūpina benamiams namus, išveda belaisvius į laisvę. Tik maištininkai lieka gyventi išdžiūvusioje žemėje. 
\par 7 Dieve, kai Tu ėjai savo tautos priekyje, kai žygiavai per dykumą, 
\par 8 žemė drebėjo, Dievo akivaizdoje dangūs lašėjo. Sinajus drebėjo prieš Dievą, Izraelio Dievą. 
\par 9 Dieve, siuntei gausų lietų ant savųjų, juo gaivinai suvargusius. 
\par 10 Tavoji tauta tenai įsikūrė, Dieve, Tu buvai jiems geras ir viskuo aprūpinai. 
\par 11 Kai Viešpats tarė žodį, didelis būrys skelbėjų pasipylė, 
\par 12 karaliai ir kariuomenė skubėdami pabėgo, o šeimininkė padalino grobį. 
\par 13 Nors gulėjote garduose, buvote lyg balandžio sparnai, padengti sidabru, ir geltonu auksu­jo plunksnos. 
\par 14 Kai Visagalis išsklaidė karalius, žemė atrodė balta lyg sniegas Calmone. 
\par 15 Dievo kalnas­kalnas Bašane, aukštas kalnas­kalnas Bašane. 
\par 16 Ko šnairuojate, aukštosios kalnų viršūnės, į kalną, kur Dievas panoro gyventi? Taip, Viešpats gyvens jame per amžius! 
\par 17 Dievo vežimų tūkstančių tūkstančiai, Viešpats tarp jų ant Sinajaus, savo šventykloje. 
\par 18 Tu užžengei aukštyn, nusivedei paimtus belaisvius, ėmei dovanų iš žmonių, net iš maištaujančių, kad Viešpats Dievas gyventų tarp jų. 
\par 19 Tebūna šlovinamas Viešpats kasdien! Jis naštas mūsų neša, Jis mūsų išgelbėjimo Dievas. 
\par 20 Mūsų Dievas­mus gelbėjantis Dievas. Viešpats Dievas gali išgelbėti iš mirties. 
\par 21 Tačiau Dievas sudaužo galvas savo priešams, gauruotas galvas tiems, kurie nepaliauja elgtis nedorai. 
\par 22 Viešpats tarė: “Iš Bašano kalnų juos sugrąžinsiu, jūros gelmėje surasiu, 
\par 23 kad mirkytum koją jų kraujuje ir tavo šunų liežuviams būtų ką laižyti”. 
\par 24 Jie matė Tavo eitynes, Dieve, mano Dievo ir mano Karaliaus didingąjį žygį į šventyklą. 
\par 25 Priešakyje giesmininkai, gale muzikantai ėjo, viduryje mergaitės mušė būgnelius. 
\par 26 Laiminkite Dievą susirinkimuose, Viešpatį, kurie esate iš Izraelio giminės. 
\par 27 Priekyje eina Benjaminas, visų jauniausias! Judo kunigaikščiai su savo būriais, Zabulono ir Neftalio kunigaikščiai. 
\par 28 Dieve, parodyk savo galybę, įtvirtink, Dieve, tai, ką padarei mūsų labui! 
\par 29 Jeruzalėje yra Tavo šventykla, karaliai ten dovanas Tau neš. 
\par 30 Sudrausk nendryno žvėrį, kaimenę jaučių su tautų veršiais; sutrypk norinčius duoklės! Išsklaidyk tautas, mėgstančias karus. 
\par 31 Ateis Egipto kunigaikščiai, Etiopija išties rankas į Dievą! 
\par 32 Pasaulio karalystės, giedokite Dievui, giedokite gyrių Viešpačiui! 
\par 33 Tam, kuris važinėja danguose, esančiuose nuo seno. Štai Jis pakelia savo galingą balsą. 
\par 34 Pripažinkite Dievo galybę. Izraelyje Jo didybė, debesyse Jo galia! 
\par 35 Didingas Dievas yra savo šventykloje, Izraelio Dievas; Jis teikia savo tautai galybę ir jėgą. Tebūna šlovinamas Dievas!



\chapter{69}


\par 1 Gelbėk mane, Dieve, nes vandenys siekia mano sielą. 
\par 2 Nugrimzdau į gilius dumblus, nėra kojai atramos. Esu vandens gelmėse, bangos ritasi per mane. 
\par 3 Pailsau šaukdamas, išdžiūvo gerklė, aptemo akys, belaukiant Dievo. 
\par 4 Daugiau kaip galvos plaukų yra tų, kurie be priežasties manęs nekenčia. Galingesni už mane tie, kurie melagingai puola mane. Turiu grąžinti, ko nepaėmiau. 
\par 5 Dieve, Tu žinai mano kvailystę ir mano nusikaltimai nepaslėpti nuo Tavęs. 
\par 6 Tenebūna gėdinami dėl manęs tie, kurie laukia Tavęs, Valdove, kareivijų Viešpatie! Tenebūna įžeidinėjami dėl manęs tie, kurie ieško Tavęs, Izraelio Dieve! 
\par 7 Dėl Tavęs kenčiu pajuoką, nuo gėdos rausta mano veidas. 
\par 8 Svetimas tapau savo broliams, pašalinis­mano motinos vaikams. 
\par 9 Uolumas dėl Tavo namų sugraužė mane, Tave plūstančiųjų keiksmai krito ant manęs. 
\par 10 Kai aš verkiau ir varginau sielą pasninku­jie užgauliojo mane. 
\par 11 Vietoje drabužių vilkiu ašutinę­dėl to tapau jiems priežodžiu. 
\par 12 Tie, kurie sėdi vartuose, kalba prieš mane ir geriantys vyną dainuoja apie mane. 
\par 13 O aš meldžiuosi Tau, Viešpatie, visą laiką. Dieve, dėl savo beribio gailestingumo išklausyk mane, dėl savo išgelbėjimo tiesos. 
\par 14 Išgelbėk mane iš dumblo, kad nenugrimzčiau, ištrauk iš gilių vandenų, išvaduok iš piktų priešų. 
\par 15 Teneužlieja manęs vandens srovė, tenepraryja gelmė ir teneapžioja manęs duobė. 
\par 16 Viešpatie, išklausyk mane dėl savo malonės gausos! Pažvelk į mane gailestingai. 
\par 17 Nepaslėpk savo veido nuo savo tarno, nes esu varge; skubiai išklausyk mane! 
\par 18 Priartėk prie mano sielos ir išpirk. Išlaisvink mane iš priešų! 
\par 19 Tu žinai, kaip mane plūsta, gėdina ir niekina; Tu matai visus mano prispaudėjus. 
\par 20 Pajuoka plėšo mano širdį, aš pavargau. Aš laukiau pasigailėjimo, bet jo nėra, ieškojau guodėjų, tačiau neradau. 
\par 21 Vietoje maisto duoda man tulžies, ištroškusį girdo actu. 
\par 22 Jų stalas jiems patiems spąstais tevirsta ir jų gerovė­žabangais! 
\par 23 Teaptemsta jų akys, kad neregėtų; padaryk, kad jų strėnos visada svyruotų. 
\par 24 Išliek ant jų savo užsidegimą, ir Tavo rūstybės įkarštis tepasiekia juos. 
\par 25 Jų buveinė tegul ištuštėja, palapinėse gyventojų tenelieka. 
\par 26 Nes jie persekioja tą, kurį Tu ištikai, kurį sužeidei, tam jie didina skausmą. 
\par 27 Tegul vis gausėja jų kaltės, ir teneįeina jie į Tavo teisumą. 
\par 28 Tebūna jie išbraukti iš gyvųjų knygos ir teisiųjų sąrašuose tenebūna jų vardų! 
\par 29 Aš esu vargšas ir kenčiu; Tavo išgelbėjimas, Dieve, teiškelia mane! 
\par 30 Aš girsiu Dievo vardą giesme ir dėkodamas aukštinsiu Jį. 
\par 31 Tai Viešpačiui bus meiliau už jautį, už jauniklį jautuką su ragais ir nagais. 
\par 32 Tai matydami, linksminsis nuolankieji, ir atgys širdys tų, kurie ieško Dievo. 
\par 33 Viešpats girdi vargšą ir neniekina savo belaisvių. 
\par 34 Jį tegiria dangūs ir žemė, jūros ir visa, kas juda jose. 
\par 35 Dievas išgelbės Sioną ir Judo miestus atstatys. Jie įsikurs ten ir gyvens, 
\par 36 Jo tarnų vaikai paveldės tą žemę, o Jo vardą mylintys gyvens joje.



\chapter{70}


\par 1 Dieve, skubėk išlaisvinti mane! Viešpatie, skubėk man padėti! 
\par 2 Sugėdink ir pakrikdyk tuos, kurie kėsinasi į mano gyvybę! Teatsitraukia sugėdinti, kurie trokšta man pakenkti! 
\par 3 Tebėga apimti gėdos, kurie sako: “Gerai tau, gerai tau!” 
\par 4 Tesidžiaugia ir tesilinksmina Tavyje visi, kurie ieško Tavęs, ir tesako: “Didis yra Dievas!”, kurie myli Tavo išgelbėjimą. 
\par 5 Aš esu vargšas ir suvargęs; Dieve, skubėk pas mane! Mano pagalba ir išlaisvintojas Tu esi. Viešpatie, nedelsk!

\chapter{71}


\par 1 Viešpatie, Tavimi pasitikiu: niekados tenebūsiu sugėdintas. 
\par 2 Savo teisumu išlaisvink ir išvaduok mane! Kreipk į mane savo ausį ir išgelbėk mane! 
\par 3 Būk man tvirta pilis, kur visada galėčiau pabėgti! Juk Tu esi mano uola ir tvirtovė! 
\par 4 Dieve, išlaisvink mane iš nedorėlio rankos, iš rankos neteisaus ir žiauraus žmogaus. 
\par 5 Viešpatie Dieve, Tu esi mano viltis, mano pasitikėjimas nuo pat jaunystės. 
\par 6 Tu mane globojai nuo pat gimimo, Tu saugojai mane nuo motinos įsčių. Aš nuolat girsiu Tave. 
\par 7 Daugelis stebėjosi manimi, bet Tu buvai man saugi priebėga. 
\par 8 Mano burna tebūna pilna gyriaus ir Tavo garbės visą dieną. 
\par 9 Neapleisk manęs senatvėje; nepalik, kai jėgos išseks! 
\par 10 Mano priešai kalba ir seka mane; jie tyko mano gyvybės ir tariasi: 
\par 11 “Dievas jį paliko, vykite ir sugaukite jį, nes niekas jo neišgelbės!” 
\par 12 Dieve, nebūk toli nuo manęs! Mano Dieve, skubėk man padėti! 
\par 13 Gėdingai težūva ir teišnyksta mano sielos priešininkai! Tesusilaukia paniekos ir negarbės, kurie siekia man pakenkti. 
\par 14 Aš vilties nenustosiu, girsiu Tave vis labiau. 
\par 15 Mano burna kalbės apie Tavo teisumą, skelbs visą dieną Tavo išgelbėjimą, nes aš nežinau jų ribų. 
\par 16 Aš eisiu su Viešpaties Dievo jėga, minėsiu Tavo vieno teisumą. 
\par 17 Dieve, Tu mokei mane nuo jaunystės ir iki šiol dar skelbiu Tavo nuostabius darbus. 
\par 18 Dieve, nepalik manęs, kai būsiu senas ir pražilęs, kol ateinančioms kartoms apie Tavo jėgą ir galią aš paskelbsiu. 
\par 19 Dieve, Tavo teisumas siekia aukštąjį dangų, Tu padarei didingų darbų, kas yra Tau lygus, Dieve? 
\par 20 Nors daug ir skaudžių vargų leidai man patirti, bet ir vėl atgaivinsi mane, ištrauksi iš žemės gelmių! 
\par 21 Tu suteiksi man garbę ir vėl paguosi. 
\par 22 Dieve, aš girsiu Tavo ištikimybę psalteriu, skambinsiu Tau arfa, Izraelio Šventasis! 
\par 23 Mano lūpos ir siela, kurią išpirkai, džiūgaus, kai Tau giedosiu. 
\par 24 Ir mano liežuvis visą dieną kalbės apie Tavo teisumą. Visi, kurie siekė man pakenkti, yra sugėdinti ir paniekinti.


\chapter{72}


\par 1 Dieve, suteik karaliui savo išminties teisti ir savo teisingumo karaliaus sūnui, 
\par 2 kad jis Tavo tautą teisingai teistų ir varguolių teises apgintų. 
\par 3 Tegu kalnai taiką tautai neša ir kalvos teisybę! 
\par 4 Tegul gins jis tautos varguolių teises, gelbės beturčių vaikus ir sunaikins prispaudėją. 
\par 5 Per kartų kartas jie bijos Tavęs, kol saulė švies ir mėnulis spindės. 
\par 6 Jis ateis kaip lietus ant nupjautos lankos, kaip lietus, drėkinantis žemę. 
\par 7 Jo dienomis žydės teisusis ir taika bus, kol danguje spindės mėnulis. 
\par 8 Jis viešpataus nuo jūros iki jūros ir nuo upės iki žemės pakraščių. 
\par 9 Jam nusilenks visi dykumų gyventojai ir jo priešai laižys dulkes. 
\par 10 Taršišo ir salų karaliai atneš jam dovanų. Šebos ir Sebos karaliai mokės duoklę. 
\par 11 Visi karaliai garbins jį, visi pagonys tarnaus jam. 
\par 12 Jis išvaduos pagalbos maldaujantį beturtį ir vargšą, kuriam nepadeda niekas. 
\par 13 Jis pasigailės vargšų ir beturčių ir jų gyvybes išgelbės. 
\par 14 Iš priespaudos ir smurto jis išlaisvins juos, jų kraujas bus brangus jo akyse. 
\par 15 Jis gyvens ir jam duos Šebos aukso, melsis už jį nuolatos, visą laiką jis bus giriamas. 
\par 16 Javų bus krašte apsčiai, net kalnų viršūnėse jie augs; kaip Libano miškas bus jo vaisiai. Miestai bus pilni žmonių kaip pievos žolių. 
\par 17 Jo vardas išliks per amžius. Jo vardas pasiliks, kol saulė švies danguje. Visos žemės giminės bus palaimintos jame, visos tautos vadins jį palaimintu. 
\par 18 Palaimintas Viešpats, Izraelio Dievas, kuris vienas daro stebuklus! 
\par 19 Palaimintas šlovingas Jo vardas per amžius. Jo šlovė tepripildo visą žemę! Amen! Amen! 
\par 20 Pabaiga Jesės sūnaus Dovydo maldų.


\chapter{73}


\par 1 Tikrai geras yra Dievas Izraeliui, tiems, kurių širdis tyra. 
\par 2 Bet mano kojos vos nepasviro, vos nepaslydo mano žingsniai. 
\par 3 Aš pavydėjau kvailiams, matydamas nedorėlių pasisekimą. 
\par 4 Jie nepatiria kentėjimų mirdami, yra kupini jėgų. 
\par 5 Jie nevargsta kaip kiti žmonės, jų nepaliečia bėdos. 
\par 6 Išdidumu ir smurtu jie rengiasi lyg drabužiu. 
\par 7 Iš riebaus kūno žiūri jų akys, jie turi daugiau, negu geidžia širdis. 
\par 8 Jie sugedę ir kalba piktai, iš aukšto grasina smurtu. 
\par 9 Jų burnos dangui grūmoja, o liežuviai vaikštinėja po žemę. 
\par 10 Todėl Jo tauta pritaria jiems ir jų mokslą lyg vandenį geria. 
\par 11 Jie sako: “Kaip Dievas gali žinoti? Argi Aukščiausiajame yra pažinimas?” 
\par 12 Štai tokie yra bedieviai, kurie klesti pasaulyje ir turtėja. 
\par 13 Ar veltui saugojau tyrą širdį ir nekaltume ploviau rankas? 
\par 14 Aš gi buvau spaudžiamas visą dieną ir plakamas kas rytą. 
\par 15 Jei būčiau galvojęs kalbėti kaip jie, būčiau nusikaltęs Tavo vaikų kartai. 
\par 16 Aš galvojau, norėdamas tai suprasti, bet nepajėgiau, 
\par 17 kol įėjau į Dievo šventyklą ir pamačiau jų galą. 
\par 18 Tikrai Tu pastatei juos labai slidžioje vietoje. Tu nustūmei juos į pražūtį. 
\par 19 Jie per akimirksnį sukniubo, pranyko ir žuvo nuo išgąsčio. 
\par 20 Kaip sapną prabundant, taip, Viešpatie, Tu pakilęs paniekinsi jų įsivaizdavimus. 
\par 21 Kai mano širdis buvo apkartus ir inkstus varstė diegliai, 
\par 22 aš buvau kvailas ir neišmanantis­lyg gyvulys Tavo akivaizdoje. 
\par 23 Tačiau aš nuolat esu su Tavimi, Tu laikai nutvėręs mano dešinę ranką. 
\par 24 Tu vesi mane savo patarimu ir galiausiai paimsi į šlovę. 
\par 25 Ką aš turiu danguje? Ir žemėje aš trokštu tik Tavęs. 
\par 26 Kai kūnas ir širdis sunyksta, Dievas yra mano širdies stiprybė ir mano dalis per amžius. 
\par 27 Tikrai, kas toli nuo Tavęs, pražus, Tu sunaikini visus, kurie Tave palieka. 
\par 28 Man gera artėti prie Dievo. Viešpačiu Dievu aš pasitikiu, kad pasakočiau apie visus Tavo darbus.


\chapter{74}


\par 1 Dieve, kodėl atstūmei mus amžiams, kodėl Tavo rūstybė dega prieš Tavo ganyklos avis? 
\par 2 Atsimink savo susirinkimą, kurį senais laikais įsigijai, tautą, kurią išpirkai, kad ji būtų Tavo, Siono kalną, kuriame Tu gyvenai! 
\par 3 Įženk į nesibaigiančius griuvėsius. Priešas šventykloje nusiaubė viską. 
\par 4 Tavo priešai rėkavo Tavo susirinkimo viduryje, iškėlė čia savo ženklus. 
\par 5 Jie švaistėsi kirviais lyg miško tankynėje: 
\par 6 sudaužė kirviais ir kūjais visus medžio raižinius. 
\par 7 Jie padegė Tavo šventyklą, išniekino Tavo vardo buveinę. 
\par 8 Jie manė savo širdyse: “Sunaikinsime viską!” Jie sudegino visas Dievo sueigos vietas krašte. 
\par 9 Mes nebeturime savo ženklų, nebėra daugiau jokio pranašo; nė vienas nežinome, ar ilgai taip bus. 
\par 10 Dieve, ar ilgai prispaudėjas tyčiosis? Ar priešas niekins Tavo vardą per amžius? 
\par 11 Kodėl atitrauki savo ranką ir dešinę paslepi antyje? 
\par 12 Tačiau Dievas yra mano Karalius nuo seno; Jis gelbsti žemės viduryje. 
\par 13 Tu jūrą savo galybe perskyrei, sutrupinai vandenyje slibinams galvas. 
\par 14 Tu sudaužei galvas leviatano, davei jį visą suėsti dykumos gyventojams. 
\par 15 Tau paliepus ištrykšta šaltiniai ir sraunūs upeliai, o vandeningos upės išdžiūsta. 
\par 16 Tavo yra diena ir naktis. Tu sukūrei šviesą ir saulę. 
\par 17 Tu nustatei žemės ribas, Tu padarei vasarą ir žiemą. 
\par 18 Atsimink, Viešpatie, kad priešas tyčiojasi ir kvailiai niekina Tavąjį vardą. 
\par 19 Neatiduok savo balandėlio gyvybės žvėrims ir nepamiršk amžinai vargšų susirinkimo. 
\par 20 Pažvelk į savo sandorą; juk visi žemės tamsūs kampai yra gausūs smurto! 
\par 21 Nepalik prispaustųjų gėdingai trauktis! Vargšai ir skurdžiai tegiria Tavąjį vardą! 
\par 22 Pakilk, Dieve, ir gink savo bylą! Prisimink, kaip kvailiai tyčiojasi iš Tavęs kasdien! 
\par 23 Nepamiršk savo priešų riksmo, prieš Tave nuolat sukylančiųjų triukšmo!


\chapter{75}


\par 1 Dėkojame Tau, Dieve, dėkojame! Kurie šaukiasi Tavo vardo, pasakoja nuostabius Tavo darbus. 
\par 2 Kai ateis mano skirtas laikas, teisiu teisingai. 
\par 3 Nors žemė drebėtų ir visi gyventojai joje, Aš laikau jos stulpus. 
\par 4 Aš sakau kvailiams: “Nekvailiokite!” ir nedorėliams: “Nesididžiuokite! 
\par 5 Nekelkite savo rago į aukštybes, nekalbėkite įžūliai prieš Dievą”. 
\par 6 Išaukštinimas neateina nei iš rytų, nei iš vakarų, nei iš pietų. 
\par 7 Dievas yra teisėjas: vieną Jis pažemina, o kitą išaukština. 
\par 8 Viešpats laiko rankoje taurę, pilną putojančio vyno su karčiom priemaišom. Iš jos turės gerti visi žemės piktadariai ir net mieles išsiurbti! 
\par 9 Bet aš skelbsiu per amžius, giedosiu gyrių Jokūbo Dievui! 
\par 10 Jis sunaikins nedorėlių puikybę ir aukštai iškels teisiųjų ragus.


\chapter{76}


\par 1 Garsus yra Judo žemėje Dievas, Jo vardas didis Izraelyje. 
\par 2 Saleme yra Jo palapinė, Jo buveinė Sione. 
\par 3 Jis sulaužė ten lanko strėles, skydus, kardus ir karo ginklus. 
\par 4 Šlovingesnis ir puikesnis Tu esi už grobio kalnus! 
\par 5 Stipruoliai buvo apiplėšti, užmigo mirties miegu, galiūnai rankas nuleido. 
\par 6 Jokūbo Dieve, kai Tu pagrūmojai, raiteliai ir žirgai sustingo. 
\par 7 Tavęs reikia bijoti, nes kas Tau gali pasipriešinti, kai imi rūstauti? 
\par 8 Iš dangaus Tu sprendimą paskelbei; nusigandusi žemė nutilo, 
\par 9 kai Dievas kėlėsi teisti ir išgelbėti žemės romiųjų. 
\par 10 Net žmonių rūstybė girs Tave, nuo Tavo rūstybės išlikusieji iškilmes ruoš Tavo garbei. 
\par 11 Darykite įžadus ir vykdykite juos Viešpačiui, savo Dievui! Visi šalia Jo esantieji neškite dovanų Tam, kurio reikia bijoti, 
\par 12 kuris nutildo kunigaikščių dvasią, kuris baisus žemės karaliams.


\chapter{77}


\par 1 Mano balsas kilo į Viešpatį, kai aš Jo šaukiausi. Mano balsas kilo, ir Jis išklausė mane! 
\par 2 Savo nelaimės dieną ieškojau Viešpaties; per naktį ištiesta mano ranka nepavargo; mano siela nepriėmė paguodos. 
\par 3 Kai prisimenu Dievą­vaitoju, aš skundžiuosi ir alpsta mano dvasia. 
\par 4 Tu laikai atmerktas mano akis. Aš nerimauju ir negaliu kalbėti. 
\par 5 Apie praėjusias dienas mąstau, prisimenu senus laikus. 
\par 6 Naktį prisimenu giesmę ir mąstau savo širdyje, mano dvasia vis tyrinėja: 
\par 7 “Nejaugi amžiams atstums Viešpats ir nebebus palankus? 
\par 8 Ar Jo gailestingumas aplenkė mane? Ar Jo pažadai, duoti kartų kartoms, neišsipildys? 
\par 9 Argi Dievas pamiršo būti maloningas ir užsirūstinęs atsisakė būti gailestingas?” 
\par 10 Aš sakiau: “Man skaudu, kad Aukščiausiojo dešinė pasikeitusi”. 
\par 11 Prisimenu Viešpaties darbus, Tavo senovėje darytus stebuklus; 
\par 12 apgalvoju visą Tavo darbą, kalbu apie Tavo veiksmus. 
\par 13 Dieve, šventas yra Tavo kelias! Kuris dievas yra toks didis, kaip mūsų Dievas? 
\par 14 Tu esi stebuklus darantis Dievas, apreiškiantis tautose savo galią. 
\par 15 Tu savo ranka išpirkai savo tautą, Jokūbo ir Juozapo vaikus. 
\par 16 Pamatę Tave vandenys, o Dieve, pamatę Tave vandenys sudrebėjo, gelmės sunerimo. 
\par 17 Iš debesų vanduo pasipylė, pasigirdo padangėse garsas, švilpė Tavosios strėlės. 
\par 18 Griaustinis danguje sudundėjo, žaibai apšvietė pasaulį, drebėjo ir virpėjo žemė. 
\par 19 Tavo kelias ėjo per jūrą, takas­ per plačius vandenis, kur praėjai­ neliko žymės. 
\par 20 Tu vedei savo tautą kaip avis Mozės ir Aarono ranka.


\chapter{78}


\par 1 Klausykis mano tauta mano įstatymo. Išgirsk savo ausimis mano burnos žodžius! 
\par 2 Atversiu burną palyginimais, atskleisiu senovės laikų paslaptis. 
\par 3 Ką girdėjome ir sužinojome, ką mūsų tėvai pasakojo mums, 
\par 4 neslėpsime nuo jų vaikų, pasakosime būsimai kartai apie Viešpaties šlovę, Jo galybę ir stebuklus, kuriuos Jis padarė. 
\par 5 Jis davė liudijimą Jokūbe ir išleido įstatymą Izraelyje. Ką Jis įsakė mūsų tėvams, jie turi skelbti savo vaikams, 
\par 6 kad ir būsimoji karta­ateityje gimsiantieji vaikai­žinotų ir skelbtų savo vaikams, 
\par 7 kad jie pasitikėtų Dievu, nepamirštų Dievo darbų ir laikytųsi Jo įsakymų, 
\par 8 kad netaptų jie, kokie buvo jų tėvai, kietasprandė ir maištinga karta; karta, kurios širdis nebuvo tvirta nei dvasia ištikima Dievui. 
\par 9 Efraimai, ginkluoti lankais, pabėgo iš mūšio kautynių dieną. 
\par 10 Dievo sandoros jie nesilaikė ir įstatymų nepaisė. 
\par 11 Užmiršo Jo darbus bei padarytus stebuklus. 
\par 12 Jų tėvams matant, Jis darė nuostabių dalykų Egipto šalyje, Coano laukuose. 
\par 13 Jis perskyrė jūrą ir pervedė juos, vandenys stovėjo kaip siena. 
\par 14 Jis vedė juos dieną debesimi, o naktį­ugnies šviesa. 
\par 15 Jis perskėlė dykumos uolą ir pagirdė juos kaip iš gelmių. 
\par 16 Iš uolos veržėsi srovės ir vanduo lyg upės tekėjo. 
\par 17 Tačiau jie dar daugiau prieš Jį nusidėjo, maištavo prieš Aukščiausiąjį dykumoje. 
\par 18 Jie gundė Dievą savo širdyse, reikalaudami maisto, kurio užsigeidė. 
\par 19 Jie kalbėjo prieš Dievą ir sakė: “Argi gali Dievas paruošti mums stalą dykumoje? 
\par 20 Štai Jis smogė į uolą, iš jos ištekėjo vandenys ir pasipylė upeliai. Bet argi Jis gali duoti duonos ir mėsos savo tautai?” 
\par 21 Išgirdęs tai, Viešpats supyko, ugnis užsidegė prieš Jokūbą, rūstybė kilo prieš Izraelį, 
\par 22 nes jie netikėjo Dievu ir nepasitikėjo Jo išgelbėjimu. 
\par 23 Tačiau Jis debesims įsakė iš aukštybių, dangaus vartus atidarė. 
\par 24 Iš dangaus Jis pabėrė maną­maistą jiems valgyti. 
\par 25 Žmonės valgė angelų duoną; turėjo pakankamai maisto. 
\par 26 Jis padangėje sukėlė rytų ir pietų vėją savo galia 
\par 27 ir leido lyti ant jų mėsa kaip dulkėmis ir sparnuotais paukščiais kaip jūros smiltimis. 
\par 28 Jie krito į jų stovyklą ties palapinėmis. 
\par 29 Jie valgė, ir visi pasisotino: patenkino Dievas jų norus. 
\par 30 Bet jie dar nebuvo palikę savo geismų, dar valgis tebebuvo burnoje, 
\par 31 kai Dievo rūstybė užgriuvo juos. Jis išžudė jų riebiausius ir Izraelio rinktinius sunaikino. 
\par 32 Nepaisant viso to, jie ir toliau nuodėmiavo, netikėdami Dievo stebuklais. 
\par 33 Jie leido dienas tuštybėje, savo metus­baimėje. 
\par 34 Naikinami ieškojo jie Dievo, sugrįžę Viešpaties klausė. 
\par 35 Atsiminė, kad Dievas yra jų uola, aukščiausiasis Dievas jų atpirkėjas. 
\par 36 Bet jie apgaudinėjo Jį ir savo liežuviais melavo Jam, 
\par 37 jų širdis nebuvo teisi prieš Jį, jie nepasiliko ištikimi Jo sandorai. 
\par 38 Tačiau Jis, būdamas kupinas gailestingumo, atleido kaltes ir nesunaikino jų. Daugelį kartų Jis sulaikė savo rūstybę ir neišliejo pykčio. 
\par 39 Jis atsimindavo, kad jie tėra kūnas ir kvapas, kuris nueina ir nebegrįžta. 
\par 40 Kaip dažnai jie pykdė Jį dykumoje, liūdino tyruose! 
\par 41 Jie vis iš naujo gundė Dievą ir apribojo Izraelio Šventąjį. 
\par 42 Jie neprisimindavo Jo rankos ir tos dienos, kai Jis išvadavo juos iš priešo, 
\par 43 kai darė Egipte ženklus ir stebuklus Coano laukuose. 
\par 44 Jis pavertė krauju upelius ir upes, kad jie negalėtų gerti iš jų. 
\par 45 Jis siuntė muses, kurios kandžiojo juos, taip pat varles, kurios naikino juos. 
\par 46 Jis užleido ant jų laukų derliaus žiogus ir skėrius. 
\par 47 Jis išdaužė ledais vynuogynus ir šilkmedžius sunaikino šalčiu. 
\par 48 Jų gyvuliai nuo ledų žuvo ir galvijus naikino žaibai. 
\par 49 Jis siuntė jiems savo rūstybę, įtūžį, pyktį ir visus nelaimių nešėjus. 
\par 50 Jis padarė kelią savo rūstybei, nesaugojo jų nuo mirties, ant jų užleido marą. 
\par 51 Jis išžudė visus pirmagimius Egipte, pajėgumo pradžią Chamo palapinėse. 
\par 52 Jis išvedė savo tautą kaip avis, kaip kaimenę dykuma vedė. 
\par 53 Jis vedė juos saugiai, jie nieko nebijojo, jų priešus apdengė jūra. 
\par 54 Jis atvedė juos į šventąją žemę, prie kalno, kurį Jo dešinė buvo įsigijusi. 
\par 55 Jis išvarė tautas, išdalijo jų žemę paveldėti ir Izraelio gimines apgyvendino jų palapinėse. 
\par 56 Tačiau jie gundė Jį ir maištavo prieš Dievą, Aukščiausiojo įsakymų nesilaikė. 
\par 57 Nusisuko ir buvo neištikimi kaip jų tėvai, nukrypo į šalį kaip sugadintas lankas. 
\par 58 Aukštumomis jie kėlė Jo pyktį, drožtais atvaizdais sukėlė Jam pavydą. 
\par 59 Dievas, tai išgirdęs, supyko ir pasibjaurėjo Izraeliu. 
\par 60 Jis paliko palapinę Šilojuje, kurią tarp žmonių buvo pasistatęs. 
\par 61 Savo jėgą Jis atidavė į nelaisvę, savo šlovę­į priešo rankas. 
\par 62 Savo tautą pavedė kardui ir pyko ant savo paveldėjimo. 
\par 63 Jaunuolius ugnis prarijo, mergaitės liko netekėjusios. 
\par 64 Kunigai krito nuo kardo, o našlės negalėjo jų apraudoti. 
\par 65 Tada Viešpats pabudo tarsi žmogus iš miego, tarsi karžygys, šūkaudamas nuo vyno, 
\par 66 Jis privertė priešus bėgti, amžiną gėdą jiems padarė. 
\par 67 Jis atsisakė Juozapo palapinės ir Efraimo giminės neišsirinko. 
\par 68 Išsirinko Jis Judo giminę, Siono kalną pamėgo. 
\par 69 Čia Jis pastatė savo šventyklą, aukštą kaip dangų, tvirtą lyg žemę, amžiams sutvertą. 
\par 70 Savo tarną Dovydą Jis išsirinko, paėmęs jį nuo avių gardų. 
\par 71 Pašaukė jį nuo žindančių avių ganyti Jokūbą ir Izraelį, Jo paveldėjimą. 
\par 72 Jis ganė juos nuoširdžiai, rūpestinga ranka juos vedė.


\chapter{79}


\par 1 Dieve, pagonys įsiveržė į Tavo paveldėjimą, suteršė šventyklą, pavertė Jeruzalę griuvėsiais! 
\par 2 Jie atidavė Tavo tarnų lavonus lesti padangių paukščiams, žvėrims ėsti kūnus šventųjų. 
\par 3 Jie išliejo jų kraują kaip vandenį Jeruzalės apylinkėse, ir nebuvo, kas juos palaidotų. 
\par 4 Mes tapome panieka savo kaimynams, pasityčiojimu ir pajuoka tiems, kurie gyvena aplinkui mus. 
\par 5 Ar ilgai, Viešpatie? Ar amžinai rūstausi? Ar degs kaip ugnis Tavo pavydas? 
\par 6 Išliek savo rūstybę ant pagonių, kurie Tavęs nenori pažinti, ir ant karalysčių, kurios nesišaukia Tavojo vardo! 
\par 7 Juk jie suėdė Jokūbą ir jo gyvenvietes nusiaubė. 
\par 8 Neprisimink mūsų ankstesniųjų kalčių! Skubiai suteik mums savo gailestingumą, nes esame labai pažeminti. 
\par 9 Padėk mums, Dieve, mūsų gelbėtojau, dėl savo vardo šlovės! Išgelbėk mus ir atleisk mūsų nusikaltimus dėl savo vardo. 
\par 10 Kodėl turėtų sakyti pagonys: “Kur yra jų Dievas?” Tesužino apie Tave pagonys mūsų akivaizdoje, kai bausi už pralietą Tavo tarnų kraują! 
\par 11 Tepasiekia Tave belaisvių dejavimas! Savo galinga ranka išlaisvink pasmerktuosius mirti. 
\par 12 Atmokėk, Viešpatie, mūsų kaimynams septyneriopai už piktžodžiavimą, kuriuo jie plūdo Tave! 
\par 13 O mes, Tavo tauta ir Tavo ganyklos avys, dėkosime Tau amžinai, kartų kartoms skelbsime Tavo šlovę!


\chapter{80}


\par 1 Klausykis, Izraelio ganytojau, kuris vedi Juozapą kaip avis! Kuris gyveni tarp cherubų, suspindėk. 
\par 2 Pažadink savo jėgą ties Efraimu, Benjaminu ir Manasu, ateik ir išgelbėk mus! 
\par 3 Dieve, prikelk mus, apšviesk mus savo veidu, tai būsime išgelbėti. 
\par 4 Kareivijų Viešpatie, ar ilgai pyksi, kai Tavo tauta meldžiasi! 
\par 5 Tu valgydini ją verksmo duona ir ašaromis gausiai girdai. 
\par 6 Tu leidi mūsų kaimynams vaidytis dėl mūsų. Mūsų priešai tyčiojasi iš mūsų. 
\par 7 Kareivijų Dieve, prikelk mus, apšviesk mus savo veidu, tai būsime išgelbėti. 
\par 8 Tu atnešei vynmedį iš Egipto, išvarei pagonis ir jį čia įsodinai. 
\par 9 Tu paruošei jam dirvą, jis įleido šaknis ir išsiplėtė krašte. 
\par 10 Jo šešėlis dengė kalnuotąją šalį, jo šakos kaip Dievo kedrai. 
\par 11 Jis išleido šakas ligi jūros ir atžalas iki upės. 
\par 12 Kodėl jo aptvarus nugriovei, kad praeiviai skintų jo uogas? 
\par 13 Girių šernas niokoja jį, laukiniai žvėrys ganosi jame. 
\par 14 Kareivijų Dieve, sugrįžk! Pažvelk iš dangaus, pamokyk ir aplankyk šitą vynmedį, 
\par 15 kurį pasodino dešinė Tavo, atžalą, kurią Tu dėl savęs sustiprinai. 
\par 16 Tie, kurie sudegino jį ugnimi ir nukirto, Tavo veidui sudraudus, težūna! 
\par 17 Tegu Tavo ranka būna ant Tavo dešinėje sėdinčio vyro, žmogaus sūnaus, kurį pats užauginai. 
\par 18 Mes nebeatsitrauksime nuo Tavęs. Atgaivink mus, ir mes šauksimės Tavo vardo. 
\par 19 Kareivijų Viešpatie, atgaivink mus, apšviesk mus savo veidu, tai būsime išgelbėti.


\chapter{81}


\par 1 Garsiai šlovinkite Dievą, mūsų stiprybę, džiaugsmingą triukšmą kelkite Jokūbo Dievui! 
\par 2 Giedokite psalmę, muškite būgną, skambinkite psalteriu ir arfa! 
\par 3 Pūskite trimitą jauno mėnulio dieną, skirtu laiku mūsų šventėje! 
\par 4 Tai yra įsakymas Izraeliui, Jokūbo Dievo įstatymas. 
\par 5 Tokį nurodymą gavo Juozapas, kai Jis ėjo per Egipto žemę. Išgirdau nežinomą kalbą: 
\par 6 “Aš pašalinau tavo naštą nuo pečių, tavo rankoms nebereikia nešioti pintinės. 
\par 7 Varge šaukeisi, ir Aš išgelbėjau tave, atsiliepiau iš griaudėjančio debesies, prie Meribos vandenų mėginau tave. 
\par 8 Klausyk, mano tauta, Aš noriu tave įspėti! O kad tu, Izraeli, paklausytum manęs! 
\par 9 Neturėk ir negarbink jokio kito dievo! 
\par 10 Aš esu Viešpats, tavo Dievas, kuris išvedžiau tave iš Egipto žemės. Plačiai išsižiok, kad pripildyčiau tavo burną. 
\par 11 Bet mano tauta neklausė mano balso, Izraelis man nepakluso. 
\par 12 Todėl atidaviau juos jų širdžių geismams, jie vaikščiojo pagal savo sumanymus. 
\par 13 O kad mano tauta klausytų manęs, kad Izraelis eitų mano keliais! 
\par 14 Tuojau pažeminčiau jų priešus ir prieš jų prispaudėjus pakelčiau ranką! 
\par 15 Kurie nekenčia Viešpaties, turėtų Jam lenktis, ir toks jų likimas būtų per amžius. 
\par 16 O juos geriausiais kviečiais valgydinčiau, uolų medumi maitinčiau”.


\chapter{82}


\par 1 Dievas pakyla dievų susirinkime, dievams teismą daro: 
\par 2 “Ar dar ilgai teisite neteisingai ir pataikausite nedorėliams? 
\par 3 Ginkite vargšus ir našlaičius, beturčių ir paniekintųjų saugokite teises. 
\par 4 Vargšus ir beturčius iš nedorėlių rankų vaduokite!” 
\par 5 Jie neišmano ir nesupranta, jie vaikščioja tamsoje, todėl visi žemės pamatai svyruoja. 
\par 6 Aš tariau: “Jūs esate dievai ir Aukščiausiojo sūnūs. 
\par 7 Tačiau jūs mirsite kaip žmonės, krisite kaip bet kuris kunigaikštis”. 
\par 8 Pakilk, Dieve, teisk žemę, nes visos tautos Tau priklauso!


\chapter{83}


\par 1 Dieve, netylėk! Dieve, nenurimk ir nebūk tylus! 
\par 2 Juk štai Tavo priešai triukšmauja, kelia galvas tie, kurie Tavęs nekenčia. 
\par 3 Prieš Tavo tautą jie rengia sąmokslą, tariasi prieš Tavo globotinius. 
\par 4 Jie sako: “Eikime, išnaikinkime juos, kad šios tautos nebebūtų ir Izraelio vardo niekas nebeminėtų”. 
\par 5 Jie tariasi vieningai, susijungia prieš Tave; 
\par 6 Edomo ir Moabo stovyklos, izmaelitai ir hagarai, 
\par 7 Gebalas, Amonas ir Amalekas, Filistija kartu su Tyro gyventojais. 
\par 8 Su jais kartu asirai eina, Loto palikuonims teikia paramą. 
\par 9 Padaryk jiems kaip Midjanui, kaip Siserai ir Jabinui prie Kišono upelio; 
\par 10 jie prie En Doro sunaikinti buvo, tapo mėšlu žemei patręšti. 
\par 11 Padaryk jų kunigaikščius kaip Orebą ir Zeebą, kaip Zebachą bei Calmuną­visus jų vadus, 
\par 12 kurie sakė: “Pasiglemžkime Dievo žemes!” 
\par 13 Dieve, padaryk juos lyg sūkurio blaškomus lapus, lyg šiaudus prieš vėją. 
\par 14 Kaip ugnis sudegina mišką, kaip liepsna nudegina kalnus, 
\par 15 taip gainiok juos audromis, gąsdink vėtromis. 
\par 16 Viešpatie, sugėdink jų veidus, kad jie ieškotų Tavo vardo! 
\par 17 Tegul visada juos gėda ir išgąstis lydi ir gėdoje jie tepražūna. 
\par 18 Težino jie, kad Tas, kurio vardas Viešpats, visoje žemėje yra aukščiausias!


\chapter{84}


\par 1 Kokios mielos Tavo buveinės, kareivijų Viešpatie! 
\par 2 Alpsta mano siela ir ilgisi Viešpaties kiemų. Mano širdis ir kūnas šaukiasi gyvojo Dievo. 
\par 3 Net ir žvirblis randa namus, kregždė­lizdą perėti vaikams prie Tavo aukurų, kareivijų Viešpatie, mano Karaliau ir Dieve! 
\par 4 Palaiminti, kurie gyvena Tavo namuose: jie nuolat giria Tave! 
\par 5 Palaimintas žmogus, kuris randa stiprybę Tavyje, kuris yra keleivis savo širdy. 
\par 6 Keliaudami išdegusiu slėniu, jie šaltiniu jį paverčia, ankstyvasis lietus jį padengia palaiminimais. 
\par 7 Jie eina iš jėgos į jėgą, kol Dievą Sione išvysta. 
\par 8 Kareivijų Viešpatie, išgirsk mano maldą! Išgirsk, Jokūbo Dieve! 
\par 9 Pažvelk, Dieve, mūsų skyde! Pažiūrėk į veidą savo pateptojo! 
\par 10 Juk viena diena Tavo kiemuose yra vertesnė už tūkstantį kitur; geriau būti durininku Dievo namuose, negu gyventi nusidėjėlių palapinėse. 
\par 11 Saulė ir skydas yra Viešpats; malonę ir garbę teikia Viešpats, gerų dalykų neatsako tiems, kurie nekaltai elgiasi. 
\par 12 Kareivijų Viešpatie, palaimintas žmogus, kuris pasitiki Tavimi!


\chapter{85}


\par 1 Viešpatie, Tu buvai palankus savo kraštui, išlaisvinai Jokūbą. 
\par 2 Tu atleidai savo tautos kaltę, uždengei jos nusidėjimą. 
\par 3 Tu sulaikei savo pyktį, nukreipei savo rūstybės įkarštį. 
\par 4 Atgaivink mus, Dieve, mūsų gelbėtojau, tepaliauja Tavo rūstybė! 
\par 5 Nejaugi amžinai pyksi ant mūsų, rūstausi per kartų kartas! 
\par 6 Argi vėl mūsų neatgaivinsi, kad džiaugtumėmės Tavimi? 
\par 7 Viešpatie, parodyk mums savo gailestingumą, suteik išgelbėjimą! 
\par 8 Klausysiuos, ką kalba Viešpats Dievas! Jis juk skelbia taiką savo tautai ir šventiesiems; tenesigręžia jie į kvailystę. 
\par 9 Tikrai, Jo išgelbėjimas arti tiems, kurie Jo bijo, Jo šlovė gyvena tarp mūsų. 
\par 10 Gailestingumas ir tiesa susitiko, teisumas ir taika pasibučiavo. 
\par 11 Tiesa žels iš žemės, ir teisumas žvelgs iš dangaus. 
\par 12 Gausiai gėrybių duos Viešpats, ir žemė bus derlinga. 
\par 13 Teisumas eis pirma Jo ir nukreips mus į Jo kelią.


\chapter{86}


\par 1 Viešpatie, išgirsk, išklausyk mane, nes aš esu vargšas ir beturtis! 
\par 2 Saugok mano sielą, nes aš esu šventas! Gelbėk savo tarną, kuris pasitiki Tavimi! Tu esi mano Dievas. 
\par 3 Viešpatie, būk man gailestingas, nes Tavęs šaukiuos be paliovos! 
\par 4 Palinksmink savo tarno sielą, nes į Tave, Viešpatie, kreipiu savo sielą. 
\par 5 Juk Tu, Viešpatie, esi geras ir pasiruošęs atleisti, kupinas gailestingumo visiems, kurie šaukiasi Tavęs. 
\par 6 Viešpatie, išgirsk mano maldą ir klausyk mano maldavimo balso! 
\par 7 Nelaimėje šaukiuosi Tavęs­Tu išklausai mane! 
\par 8 Viešpatie, nėra Tau lygaus tarp dievų ir nėra darbų kaip Tavo. 
\par 9 Visos Tavo sutvertos tautos ateis, pagarbins Tave, Viešpatie, ir šlovins Tavo vardą. 
\par 10 Tu esi didis ir darai stebuklus, Tu vienas esi Dievas! 
\par 11 Viešpatie, pamokyk mane savo kelio, kad vaikščiočiau tiesoje! Įtvirtink mano širdį Tavo baimėje. 
\par 12 Viešpatie, mano Dieve, girsiu Tave nuoširdžiai, šlovinsiu Tavo vardą per amžius! 
\par 13 Tavo gailestingumas buvo man didelis, Tu išgelbėjai mano sielą iš giliausio pragaro. 
\par 14 Dieve, išdidieji sukilo prieš mane, smurtininkų gauja tykoja mano gyvybės; jie nestato Tavęs savo akivaizdoje. 
\par 15 Tačiau Tu, Viešpatie, esi užjaučiantis ir maloningas Dievas, kantrus ir kupinas gailestingumo bei tiesos. 
\par 16 Pažvelk į mane ir pasigailėk manęs! Duok stiprybės savo tarnui ir išgelbėk savo tarnaitės sūnų! 
\par 17 Parodyk ženklą, kad mane globoji, kad matytų tie, kurie manęs nekenčia, ir susigėstų, nes Tu, Viešpatie, esi mano gelbėtojas ir guodėjas!


\chapter{87}


\par 1 Jo pamatas ant šventųjų kalnų. 
\par 2 Viešpats myli Siono vartus labiau už visas Jokūbo buveines. 
\par 3 Šlovingi dalykai pasakojami apie tave, Dievo mieste! 
\par 4 Minėsiu Rahabą ir Babiloną tarp tų, kurie žino mane. Filistijos, Tyro ir Etiopijos gyventojai yra ten gimę. 
\par 5 Bet apie Sioną bus sakoma: “Šitas ir tas vyras jame yra gimę”. Aukščiausiasis įsteigė jį. 
\par 6 Viešpats užrašys tautų knygoje: “Šitas yra ten gimęs”. 
\par 7 Jie giedos: “Visos mano versmės yra tavyje!”


\chapter{88}


\par 1 Viešpatie, mano išgelbėjimo Dieve, dieną ir naktį šaukiau Tavo akivaizdoje. 
\par 2 Tepasiekia mano malda Tave! Išgirsk mano šauksmą! 
\par 3 Mano siela pilna skausmų ir mano gyvybė arti mirties. 
\par 4 Mane laiko tokiu, kuris nužengė į duobę, esu bejėgis žmogus. 
\par 5 Tarp mirusiųjų yra mano guolis, guliu kape kaip užmuštieji, kurių Tu nebeatsimeni, nes jie nuo Tavęs atskirti. 
\par 6 Tu įstūmei mane į giliausią duobę, į tamsą, į gelmes. 
\par 7 Mane slegia Tavo rūstybė, Tavo bangos ritasi per mane. 
\par 8 Tu atitolinai nuo manęs mano pažįstamus, jiems padarei mane bjaurų. Esu uždarytas ir negaliu išeiti. 
\par 9 Mano akys aptemo nuo vargo. Kasdien šaukiausi Tavęs, Viešpatie, tiesdamas į Tave rankas. 
\par 10 Ar parodysi stebuklus mirusiems? Ar mirusieji kelsis ir girs Tave? 
\par 11 Ar pasakojama apie Tavo malonę ir ištikimybę mirusiųjų karalystėje? 
\par 12 Ar žinomi Tavo stebuklai tamsoje ir Tavo teisumas užmiršimo šalyje? 
\par 13 Viešpatie, Tavęs aš šaukiuosi, mano malda kas rytą kyla į Tave. 
\par 14 Viešpatie, kodėl atstumi mane, slepi nuo manęs savo veidą? 
\par 15 Nuskurdęs ir pasiruošęs mirti esu nuo pat jaunystės, kenčiu Tavo siaubus, nežinau, ką daryti. 
\par 16 Tavo rūstybės įkarštis krinta ant manęs, naikina mane Tavo siaubai. 
\par 17 Jie visą laiką supa mane kaip vandenys ir skandina. 
\par 18 Tu atitolinai nuo manęs mielą bičiulį, mano pažįstami pamiršo mane.


\chapter{89}


\par 1 Viešpatie, apie Tavo gailestingumą giedosiu amžinai; kartų kartoms skelbsiu Tavo ištikimybę. 
\par 2 Aš tariau: “Gailestingumas išliks per amžius. Tu įtvirtinsi savo ištikimybę danguose”. 
\par 3 “Aš sudariau sandorą su išrinktuoju ir prisiekiau savo tarnui Dovydui: 
\par 4 ‘Per amžius įtvirtinsiu tavo palikuonis, tavo sostą visoms kartoms pastatysiu’ ”. 
\par 5 Viešpatie, dangūs giria Tavo stebuklus, Tavo ištikimybę­šventųjų susirinkimas. 
\par 6 Kas gi iš danguje esančių Viešpačiui prilygsta? Kas tarp Dievo sūnų panašus į Viešpatį? 
\par 7 Bauginantis yra Dievas šventųjų susirinkime, gerbtinas visų aplink Jį esančių. 
\par 8 Viešpatie, kareivijų Dieve, kas yra Tau lygus savo jėga? Viešpatie, kas yra toks ištikimas, kaip Tu? 
\par 9 Tu suvaldai jūros šėlimą, sukilusias bangas Tu sutramdai. 
\par 10 Tu sutriuškini Rahabą kaip nukautąjį, savo galinga ranka išgainioji savo priešus. 
\par 11 Tau priklauso dangūs ir žemė, pasaulis ir visa, kas jame; Tu sukūrei juos. 
\par 12 Tu sutvėrei šiaurę ir pietus; Taboras ir Hermonas džiaugiasi Tavimi. 
\par 13 Tu galingas: stipri Tavo ranka, aukštai pakelta Tavo dešinė! 
\par 14 Tavo sosto pamatas­teisumas ir teisingumas, gailestingumas ir tiesa eina Tavo priekyje. 
\par 15 Palaiminta tauta, kuri pažįsta džiaugsmingą garsą! Viešpatie, Tavo veido šviesoje jie vaikščios. 
\par 16 Tavo vardas juos linksmins, jie bus išaukštinti Tavo teisume. 
\par 17 Tu esi jų garbė ir galybė, Tavo palankumu išaukštintas mūsų ragas. 
\par 18 Viešpats yra mūsų apsauga, Izraelio Šventasis­mūsų karalius. 
\par 19 Regėjime Tu kalbėjai savo šventajam: “Aš suteikiau pagalbą karžygiui, išaukštinau tautos išrinktąjį. 
\par 20 Suradau Dovydą, savo tarną, ir šventu aliejumi jį patepiau. 
\par 21 Mano ranka visuomet jį palaikys, mano dešinė stiprins jį. 
\par 22 Priešas nenugalės jo ir piktadarys nepažemins. 
\par 23 Jo akivaizdoje parblokšiu priešus ir palaušiu tuos, kurie jo nekenčia. 
\par 24 Mano ištikimybė ir gailestingumas lydės jį, ir mano vardu aukštai iškils jo ragas. 
\par 25 Aš jo ranką padėsiu ant jūros ir ant upių jo dešinę. 
\par 26 Jis sakys man: ‘Tu esi mano tėvas, mano Dievas ir išgelbėjimo uola!’ 
\par 27 Aš padarysiu jį savo pirmagimiu, jis bus aukščiau negu žemės karaliai. 
\par 28 Aš būsiu jam gailestingas per amžius, tvirta pasiliks sandora tarp mūsų. 
\par 29 Jo palikuonims leisiu gyventi per amžius, kol bus dangus, stovės jo sostas. 
\par 30 Jei jo vaikai paniekins mano įstatymą, jei nesielgs, kaip jiems įsakyta, 
\par 31 jei mano nuostatus laužys ir įsakymų mano nepaisys, 
\par 32 bausiu už jų nuodėmes lazda, plaksiu rykštėmis už jų kaltę. 
\par 33 Bet savo malonės iš jo neatimsiu ir ištikimybės jam neatsakysiu, 
\par 34 nesulaužysiu sandoros ir savo lūpų žodžio nekeisiu. 
\par 35 Kartą esu prisiekęs šventumu savo, Dovydui nemeluosiu. 
\par 36 Jo palikuonys gyvens per amžius, ir jo sostas stovės kaip saulė priešais mane, 
\par 37 bus įtvirtintas kaip mėnulis­ištikimas liudytojas danguje”. 
\par 38 Bet Tu atstūmei, atmetei ir pasibjaurėjai pateptuoju savo. 
\par 39 Išsižadėjai sandoros su savo tarnu, nusviedei žemėn jo karūną ir suteršei ją. 
\par 40 Tu sugriovei jo mūrus, pavertei griuvėsiais tvirtoves. 
\par 41 Jį plėšia visi praeiviai, jis tapo pajuoka kaimynams. 
\par 42 Tu leidai jo priešų galybei iškilti, jiems visiems suteikei džiaugsmą. 
\par 43 Nukreipei ašmenis jo kardo, neleidai laimėti mūšyje. 
\par 44 Tu atėmei jo šlovę ir nuvertei jo sostą. 
\par 45 Sutrumpinai jo jaunystės dienas, padengei jį gėda. 
\par 46 Ar ilgai, Viešpatie? Ar slėpsies amžinai? Ar degs kaip ugnis Tavo rūstybė? 
\par 47 Viešpatie, atsimink, koks trumpas yra mano gyvenimas! Kokius menkus Tu sukūrei visus žmones! 
\par 48 Koks žmogus gyvena ir nepatiria mirties? Kieno siela mirusiųjų buveinės išvengia? 
\par 49 Viešpatie, kur yra Tavo ankstesnė malonė, kurią Dovydui savo tiesoje pažadėjai? 
\par 50 Viešpatie, atsimink savo tarno gėdą; nešioju širdyje daugelio tautų panieką, 
\par 51 kuria, Viešpatie, Tavo priešai niekino Tavo pateptojo kelius. 
\par 52 Tebūna palaimintas Viešpats per amžius! Amen! Amen!


\chapter{90}


\par 1 Viešpatie, Tu buvai mums prieglauda per kartų kartas! 
\par 2 Pirma, negu buvo sutverti kalnai, žemė ir pasaulis, Tu, Dieve, esi nuo amžių ir per amžius! 
\par 3 Tu grąžini žmones į dulkes ir sakai: “Sugrįžkite, žmonių vaikai!” 
\par 4 Juk tūkstantis metų Tavo akyse yra kaip vakarykštė diena, kuri praėjo, kaip sargybos laikas naktį. 
\par 5 Tu pašalini žmones kaip rytmečio sapną, kaip žaliuojančią žolę. 
\par 6 Rytą ji žydi, vakare nukertama ir sudžiūsta. 
\par 7 Tavo rūstybė sunaikina mus, Tavo pyktis mus gąsdina. 
\par 8 Tu laikai mūsų kaltes savo akivaizdoje, mūsų slaptas nuodėmes­savo veido šviesoje. 
\par 9 Mūsų dienos praeina Tau rūstaujant, mūsų metai kaip atodūsis. 
\par 10 Mūsų metų skaičius yra septyniasdešimt, o stipresniųjų­aštuoniasdešimt. Dauguma jų praeina varge ir kančiose. Jie greitai praeina, ir mes išnykstame. 
\par 11 Kas Tavo rūstybės jėgą supranta ir bijo Tavojo pykčio? 
\par 12 Pamokyk mus skaičiuoti mūsų dienas, kad įgytume išmintingą širdį! 
\par 13 Sugrįžk, Viešpatie! Ar ilgai? Būk gailestingas savo tarnams! 
\par 14 Gaivink mus nuo ryto savo gailestingumu, kad džiaugsmas ir linksmumas mus lydėtų visą amžių! 
\par 15 Suteik mums džiaugsmo už tas dienas, per kurias pažeminti buvome, už tuos metus, per kuriuos patyrėme pikta. 
\par 16 Tepamato Tavo tarnai Tavo darbus ir Tavo šlovę jų vaikai! 
\par 17 Viešpatie, Dieve, būk mums geras, įtvirtink mūsų darbus, daryk mūsų darbus sėkmingus!


\chapter{91}


\par 1 Tas, kuris gyvena Aukščiausiojo globoje, Visagalio šešėlyje pasilieka, 
\par 2 sako Viešpačiui: “Tu mano priebėga ir mano tvirtovė, mano Dievas, kuriuo pasitikiu!” 
\par 3 Jis ištrauks tave iš medžiotojo kilpos, iš pražūtingo maro. 
\par 4 Jis pridengs tave savo plunksnomis, po Jo sparnais rasi sau prieglaudą. Didysis skydas ir šarvas yra Jo tiesa! 
\par 5 Tau nereikės bijoti nakties baisumų nė strėlių, švilpiančių dieną, 
\par 6 nebaugins tavęs patamsyje slankiojąs maras nė vidudienį siaučiantis sunaikinimas. 
\par 7 Tavo pašonėje kris tūkstantis ir dešimt tūkstančių­tavo dešinėje, bet tai nepriartės prie tavęs. 
\par 8 Tu savo akimis tai stebėsi ir matysi nedorėlių atlyginimą, 
\par 9 nes tu pasidarei Viešpatį savo priebėga, Aukščiausiąjį savo buveine. 
\par 10 Tau neatsitiks nieko pikto, ir jokia nelaimė nepriartės prie tavo palapinės. 
\par 11 Jis įsakys savo angelams saugoti tave visuose keliuose. 
\par 12 Ant rankų jie nešios tave, kad neužsigautum kojos į akmenį. 
\par 13 Tu mindžiosi liūtą ir gyvatę, sutrypsi jauniklį liūtą ir slibiną. 
\par 14 “Kadangi jis pamilo mane, Aš išlaisvinsiu jį ir apginsiu, nes jis mano vardą pažįsta. 
\par 15 Jis šauksis manęs, ir Aš jam atsakysiu. Būsiu su juo varge, išlaisvinsiu jį ir pagerbsiu. 
\par 16 Ilgu gyvenimu pasotinsiu jį ir parodysiu jam savo išgelbėjimą”.


\chapter{92}


\par 1 Gera dėkoti Tau, Viešpatie, ir giedoti gyrių Tavo vardui, Aukščiausiasis, 
\par 2 skelbti rytmety Tavo malonę ir ištikimybę naktimis 
\par 3 dešimčiastygiu instrumentu, psalteriu ir arfa. 
\par 4 Viešpatie, Tu pralinksminai mane savo kūriniais, Tavo rankų darbais aš džiaugiuosi. 
\par 5 Viešpatie, kokie didingi yra Tavo darbai! Kokios gilios Tavo mintys! 
\par 6 Tik neišmanantis žmogus to nesupranta ir kvailys nesuvokia. 
\par 7 Nors ir žydi nedorėliai kaip gėlės, nors klesti piktadariai, jie bus amžinai sunaikinti. 
\par 8 Viešpatie, Tu esi Aukščiausiasis per amžius! 
\par 9 Viešpatie, štai žus Tavo priešai, bus išblaškyti visi piktadariai! 
\par 10 Bet man Tu davei jėgų kaip stumbrui, patepei mane šviežiu aliejumi. 
\par 11 Mano akys matys mano priešus, ausys išgirs apie tuos, kurie prieš mane pakilo. 
\par 12 Teisusis klestės kaip palmė, augs kaip Libano kedras. 
\par 13 Viešpaties namuose pasodinti, jie žydės Dievo kiemuose, 
\par 14 neš vaisių senatvėje, bus sultingi ir žali, 
\par 15 kad skelbtų Viešpaties teisumą. Jis yra mano uola ir Jame nėra neteisybės.


\chapter{93}


\par 1 Viešpats karaliauja. Jis apsivilkęs didybe, Viešpats susijuosęs stiprybe, taip tvirtai pastatė pasaulį, kad jo niekas nepajudins. 
\par 2 Tvirtai stovi Tavo sostas nuo senovės; nuo amžių Tu esi! 
\par 3 Viešpatie, vandenys patvino, pakilo jų ūžimas, pakilo vandenų bangos! 
\par 4 Galingesnis už gausių vandenų šniokštimą, už galingas jūrų bangas yra Viešpats aukštybėse! 
\par 5 Tavo liudijimai labai patikimi. Tavo namus puošia šventumas, Viešpatie, per amžius.


\chapter{94}


\par 1 Viešpatie Dieve, kuris atkeršiji, Dieve, kuris atkeršiji, pasirodyk! 
\par 2 Kelkis, pasaulio Teisėjau, atlygink išdidiesiems, ką jie nusipelnė! 
\par 3 Viešpatie, ar ilgai dar nedorėliai džiūgaus? 
\par 4 Ar ilgai kalbės įžūliai ir didžiuosis visi piktadariai? 
\par 5 Viešpatie, jie trypia Tavo tautą, spaudžia Tavo paveldą. 
\par 6 Jie užmuša našlę ir ateivį, žudo našlaičius. 
\par 7 Jie sako: “Viešpats nemato, Jokūbo Dievas nepastebi”. 
\par 8 Susipraskite, tautos neišmanėliai! Jūs kvailiai, kada išminties įgysite? 
\par 9 Nejaugi Tas, kuris padarė ausį, negirdėtų, ir Tas, kuris sukūrė akį, nematytų? 
\par 10 Argi Tas, kuris auklėja tautas ir moko žmones išminties, nesudraustų? 
\par 11 Viešpats žino žmonių mintis, kad jos yra tuščios. 
\par 12 Palaimintas žmogus, Viešpatie, kurį Tu auklėji ir savo įstatymo mokai; 
\par 13 ramybę jam teiki nelaimių dienomis, kol nedorėliui kasama duobė. 
\par 14 Juk Viešpats neatmes savo tautos ir neapleis savo paveldėjimo. 
\par 15 Teisingumas sugrįš teisiajam ir juo paseks visi tiesiaširdžiai. 
\par 16 Kas gins mane nuo piktadarių? Kas užstos mane prieš skriaudėjus? 
\par 17 Jei Viešpats man nebūtų padėjęs, būčiau atsidūręs tylos karalystėje. 
\par 18 Kai pasakiau: “Slysta mano koja”, Tavo gailestingumas, Viešpatie, palaikė mane. 
\par 19 Kai mano širdis prisikaupė rūpesčių, Tavo paguoda sielai džiaugsmą grąžino. 
\par 20 Argi Tu nedorėlių sostui pritarsi, kai iškraipydami įstatymą jie spaudžia žmones? 
\par 21 Teisiojo sielą jie puola, nekaltą kraują pasmerkia. 
\par 22 Bet Viešpats yra mano apsauga ir mano Dievas­priebėgos uola. 
\par 23 Jis atlygins jiems už jų nedorybes, jų pačių neteisybėse sunaikins juos. Viešpats, mūsų Dievas, sunaikins juos.


\chapter{95}


\par 1 Ateikite, giedokime Viešpačiui! Džiaugsmingą triukšmą kelkime savo išgelbėjimo uolai! 
\par 2 Ateikime į Jo akivaizdą su padėka, džiaugsmingai giedokime Jam psalmes! 
\par 3 Viešpats yra didis Dievas ir didis Karalius, didesnis už visus dievus. 
\par 4 Jo rankoje yra žemės gelmės ir Jam priklauso kalnų viršūnės. 
\par 5 Jo yra jūra, nes Jis ją sutvėrė, ir sausuma Jo rankų darbas. 
\par 6 Ateikite, pulkime žemėn prieš Dievą ir pagarbinkime, atsiklaupkime prieš Viešpatį, kuris sutvėrė mus! 
\par 7 Jis yra mūsų Dievas, o mes­Jo ganoma tauta ir Jo rankų globojamos avys! Šiandien, jeigu išgirsite Jo balsą,­ 
\par 8 “neužkietinkite savo širdžių kaip Meriboje, kaip gundymo dieną dykumoje, 
\par 9 kur jūsų tėvai mane gundė ir mėgino, nors mano darbus buvo matę! 
\par 10 Keturiasdešimt metų mane liūdino ta karta, ir Aš pasakiau: ‘Ši tauta klysta savo širdyje ir nepažįsta mano kelių’. 
\par 11 Užsirūstinęs jiems prisiekiau: ‘Jie neįeis į mano poilsį!’ ”


\chapter{96}


\par 1 Giedokite Viešpačiui naują giesmę, giedokite Viešpačiui visos šalys. 
\par 2 Giedokite Viešpačiui, šlovinkite Jo vardą. Kiekvieną dieną skelbkite Jo išgelbėjimą. 
\par 3 Apsakykite Jo garbę tarp pagonių ir Jo stebuklus visoms tautoms. 
\par 4 Didis yra Viešpats ir didžiai girtinas, bijotinas labiausiai iš visų dievų. 
\par 5 Visi tautų dievai yra stabai, bet Viešpats sukūrė dangus. 
\par 6 Didybė ir garbė yra priešais Jį, galybė ir grožis Jo šventykloje. 
\par 7 Pripažinkite Viešpačiui, tautų giminės, pripažinkite Viešpačiui šlovę ir galybę! 
\par 8 Atiduokite Viešpačiui šlovę, priderančią Jo vardui, atneškite auką, įeikite į Jo kiemus. 
\par 9 Garbinkite Viešpatį šventumo grožyje, bijokite Jo visos šalys. 
\par 10 Skelbkite pagonims: “Viešpats karaliauja!” Jis sukūrė tvirtą pasaulį, Jis teis tautas teisingai. 
\par 11 Tesilinksmina žemė ir tedžiūgauja dangūs! Tešniokščia jūra ir kas joje yra! 
\par 12 Linksmai tedžiūgauja laukai ir visa, kas juose auga! Visi miško medžiai tada džiaugsis 
\par 13 Viešpaties akivaizdoje, kai Jis ateis teisti žemę. Jis teis pasaulį teisingai ir tautas savo tiesoje.


\chapter{97}


\par 1 Viešpats karaliauja! Tedžiūgauja žemė! Tesilinksmina salos! 
\par 2 Debesys ir tamsa Jį supa; teisumas ir teisingumas yra Jo sosto pagrindas. 
\par 3 Jo priekyje liepsnoja ugnis ir sudegina aplinkui Jo priešus. 
\par 4 Žaibai nušviečia pasaulį. Tai matydama, žemė drebėjo. 
\par 5 Kalnai sutirpo kaip vaškas prieš Viešpatį­visos žemės Valdovą. 
\par 6 Dangūs skelbia Jo teisumą ir visos tautos mato Jo šlovę. 
\par 7 Bus sugėdinti tie, kurie tarnauja drožiniams, kurie stabais savo giriasi. Garbinkite Jį visi dievai! 
\par 8 Tai girdi Sionas ir džiaugiasi. Viešpatie, dėl Tavo sprendimų džiūgauja Judo dukterys. 
\par 9 Viešpatie, Tu esi aukštai virš visos žemės, išaukštintas virš visų dievų! 
\par 10 Jūs, kurie mylite Viešpatį, nekęskite pikto; Jis saugo savo šventųjų gyvybes, iš nedorėlių priespaudos išlaisvina juos. 
\par 11 Šviesa sušvinta teisiajam, tiesiaširdžiui­džiaugsmas. 
\par 12 Linksminkitės, teisieji, Viešpatyje, dėkokite, prisiminę Jo šventumą.


\chapter{98}


\par 1 Naują giesmę giedokite Viešpačiui, nes nuostabius darbus Jis daro! Jo dešinė ir Jo šventoji ranka Jam pergalę teikia. 
\par 2 Viešpats apreiškė savo išgelbėjimą, tautų akivaizdoje parodė savo teisumą. 
\par 3 Jis atsiminė savo gailestingumą ir tiesą, žadėtą Izraelio namams. Visi žemės pakraščiai pamatė Dievo išgelbėjimą. 
\par 4 Džiaugsmingą triukšmą kelkite Viešpačiui visos šalys! Linksminkitės ir giedokite gyrių. 
\par 5 Giedokite Viešpačiui, pritardami arfomis, giedokite psalmes, pritardami arfomis. 
\par 6 Trimitų ir rago garsais kelkite džiaugsmingą triukšmą prieš Karalių, Viešpatį! 
\par 7 Tesidžiaugia jūra ir kas joje yra, žemė ir jos gyventojai! 
\par 8 Upės teploja rankomis, kalnai tesidžiaugia kartu 
\par 9 Viešpaties akivaizdoje, nes Jis ateis teisti žemę! Jis teis pasaulį teisingai ir tautas nešališkai.


\chapter{99}


\par 1 Viešpats karaliauja, tegu dreba tautos. Jis sėdi tarp cherubų, tedreba žemė. 
\par 2 Viešpats yra didis Sione. Jis yra aukščiau nei visos tautos. 
\par 3 Tegiria jie Tavo vardą, didį ir baisų; Jis yra šventas. 
\par 4 Karaliaus galybė myli tiesą. Tu įtvirtini teisingumą, teisumą ir teisybę Jokūbe įvykdai. 
\par 5 Aukštinkite Viešpatį, mūsų Dievą, garbinkite priešais Jo sostą; Jis yra šventas. 
\par 6 Mozė ir Aaronas tarp Jo kunigų ir Samuelis tarp tų, kurie šaukiasi Jo vardo. Jie šaukėsi Viešpaties, ir Jis atsakė jiems. 
\par 7 Iš debesies stulpo kalbėjo Jis; jie laikėsi Jo nuostatų ir potvarkių, kuriuos Jis jiems davė. 
\par 8 O Viešpatie, mūsų Dieve, Tu atsakei jiems! Atlaidus Dievas buvai jiems, nors ir atlygindavai už jų nusikaltimus. 
\par 9 Aukštinkite Viešpatį, mūsų Dievą, garbinkite prie Jo švento kalno, nes Viešpats, mūsų Dievas, yra šventas!


\chapter{100}


\par 1 Kelkite džiaugsmingą triukšmą Viešpačiui, visos šalys! 
\par 2 Tarnaukite Viešpačiui su džiaugsmu, Jo akivaizdon ateikite giedodami. 
\par 3 Žinokite, kad Viešpats yra Dievas. Jis mus sukūrė, o ne mes patys; mes esame Jo tauta ir Jo ganyklos avys. 
\par 4 Įeikite pro vartus su dėkojimu, į Jo kiemus su gyriumi. Būkite dėkingi Jam, laiminkite Jo vardą! 
\par 5 Geras yra Viešpats; Jo gailestingumas amžinas ir Jo tiesa pasilieka kartų kartoms.


\chapter{101}


\par 1 Apie gailestingumą ir teisingumą giedosiu, skambinsiu Tau, Viešpatie. 
\par 2 Laikysiuos teisingo kelio. Kada Tu pas mane ateisi? Vaikščiosiu su tobula širdimi savo namuose. 
\par 3 Į tai, kas nedora, aš nežiūrėsiu. Aš nekenčiu neištikimųjų darbų; nieko bendro su jais neturėsiu. 
\par 4 Vengsiu iš tolo širdies nelabumo, pikto nenoriu pažinti. 
\par 5 Kas slaptai savo artimą šmeižia, tą nutildysiu. Kas žiūri iš aukšto ir turi išdidžią širdį, to neapkęsiu. 
\par 6 Mano akys žvelgia į krašto ištikimuosius, kad jie gyventų su manimi. Kas vaikščioja tobulu keliu, tas tarnaus man. 
\par 7 Klastingieji negyvens mano namuose. Melagiai nepasiliks mano akivaizdoje. 
\par 8 Anksti išnaikinsiu krašto nedorėlius, kad pašalinčiau piktadarius iš Viešpaties miesto.


\chapter{102}


\par 1 Viešpatie, išgirsk mano maldą, ir mano šauksmas tepasiekia Tave. 
\par 2 Neslėpk savo veido nuo manęs tą dieną, kai esu varge. Palenk į mane savo ausį, kai šaukiuosi, skubėk man atsakyti. 
\par 3 Mano dienos pranyksta kaip dūmai, mano kaulai kaip židinys dega. 
\par 4 Kaip pakirsta žolė mano širdis džiūsta; aš pamirštu valgyti. 
\par 5 Nuo skaudžių aimanų oda prilipo prie mano kaulų. 
\par 6 Esu panašus į dykumų pelikaną, į pelėdą griuvėsiuose. 
\par 7 Nemiegu ir esu vienišas kaip paukštis ant stogo. 
\par 8 Priešai mane užgaulioja, ir mano vardas jiems tapo keiksmažodžiu. 
\par 9 Pelenus valgau kaip duoną ir su ašaromis maišau savo gėrimą 
\par 10 dėl Tavo rūstybės ir pykčio, nes Tu mane pakėlei ir nubloškei žemėn. 
\par 11 Mano dienos yra tartum ištįsęs šešėlis, ir aš lyg žolė džiūstu. 
\par 12 Bet Tu, Viešpatie, pasiliksi per amžius; Tave minės visos kartos. 
\par 13 Tu pakilsi ir pasigailėsi Siono, nes atėjo metas jam suteikti malonę. 
\par 14 Tavo tarnams jo akmenys meilūs, jiems gaila jo dulkių. 
\par 15 Tavo vardo, Viešpatie, bijos pagonys ir Tavo šlovės­pasaulio karaliai. 
\par 16 Kai Viešpats atstatys Sioną, Jis pasirodys savo šlovėje; 
\par 17 apleistųjų maldas Jis išklausys, jų prašymų nepaniekins. 
\par 18 Tai tebūna užrašyta ateisiančiai kartai, kad tauta, kuri bus sukurta, girtų Viešpatį. 
\par 19 Iš savo šventos aukštybės Viešpats pažvelgė žemyn, iš dangaus pažiūrėjo į žemę, 
\par 20 kad išgirstų belaisvių dejones, išlaisvintų mirčiai skirtuosius, 
\par 21 kad Sione būtų skelbiamas Viešpaties vardas ir girtų Jį Jeruzalėje, 
\par 22 kai susiburs karalystės ir tautos tarnauti Viešpačiui. 
\par 23 Jis susilpnino mane kelionėje, sutrumpino mano gyvenimo dienas. 
\par 24 Aš sakiau: “Mano Dieve, neatimk manęs įpusėjus mano amžiui, Tavo metai tęsiasi per visas kartas. 
\par 25 Kadaise Tu sukūrei žemę ir dangūs yra Tavo rankų darbas. 
\par 26 Jie pražus, bet Tu pasiliksi. Jie visi susidėvės kaip drabužis, kaip rūbą juos pakeisi, ir jie bus pakeisti. 
\par 27 Bet Tu esi tas pats ir Tavo metai nesibaigs. 
\par 28 Tavo tarnų vaikai gyvens ir jų palikuonys įsitvirtins Tavo akivaizdoje”.


\chapter{103}


\par 1 Laimink, mano siela, Viešpatį, ir visa, kas yra manyje, Jo šventą vardą. 
\par 2 Laimink, mano siela, Viešpatį ir neužmiršk visų Jo geradarysčių. 
\par 3 Jis atleidžia visas tavo kaltes ir gydo visas tavo ligas. 
\par 4 Jis išperka tavo gyvybę iš pražūties ir savo malone bei gailestingumu tave vainikuoja. 
\par 5 Jis pasotina geru tavo burną, ir atsinaujina tavo jaunystė kaip erelio. 
\par 6 Viešpats vykdo teisybę ir teismą prispaustiesiems. 
\par 7 Jis savo kelius apreiškė Mozei, ir Izraelio vaikai matė Jo darbus. 
\par 8 Gailestingas ir maloningas yra Viešpats, lėtas pykti ir turtingas gailestingumo. 
\par 9 Ne visados Jis barasi ir ne amžinai rūstauja. 
\par 10 Jis nepasielgė su mumis pagal mūsų kaltes ir neatlygino mums pagal mūsų nuodėmes. 
\par 11 Kaip dangus yra aukštai virš žemės, taip didis yra Jo gailestingumas tiems, kurie Jo bijo. 
\par 12 Kaip toli nuo rytų yra vakarai, taip Jis atitolino nuo mūsų nuodėmes. 
\par 13 Kaip tėvas pasigaili vaikų, taip Viešpats gailisi tų, kurie Jo bijo. 
\par 14 Jis mūsų prigimtį žino, atsimena, kad esame dulkės. 
\par 15 Žmogaus dienos yra kaip žolė, kaip lauko gėlė jis pražysta. 
\par 16 Vos tik papūtė vėjas, jo nebėra, jo vieta jo nebepažįsta. 
\par 17 Bet Viešpaties gailestingumas per amžius tiems, kurie Jo bijo, ir Jo teisumas lieka vaikų vaikams tų, 
\par 18 kurie Jo sandoros laikosi, atsimena Jo įsakymus ir vykdo juos. 
\par 19 Viešpats danguje pastatė savo sostą, Jo karališka valdžia valdo viską. 
\par 20 Laiminkite Viešpatį jūs, angelai, galingi jėga, kurie vykdote Jo žodį, Jo balsą išgirdę. 
\par 21 Laimink Viešpatį, Jo kareivija, Jo tarnai, kurie vykdote Jo valią. 
\par 22 Laiminkite Viešpatį, visi kūriniai visoje Jo viešpatystėje. Laimink, mano siela, Viešpatį!


\chapter{104}


\par 1 Laimink, mano siela, Viešpatį! Viešpatie, mano Dieve, Tu esi labai didingas! Didybe ir garbe esi apsirengęs. 
\par 2 Tave supa šviesa kaip apsiaustas; ištiesei dangus kaip skraistę. 
\par 3 Virš vandenų surentei sau kambarius, debesis padarei savo vežimu, važiuoji ant vėjo sparnų. 
\par 4 Tu darai savo pasiuntinius kaip vėjus, savo tarnus kaip liepsnojančią ugnį. 
\par 5 Tu padėjai žemės pamatus, ir niekas jos nepajudins per amžius. 
\par 6 Vandenynais kaip drabužiu apdengei ją, kalnų viršūnes vandenys dengė. 
\par 7 Tau grūmojant, jie pabėgo, nuo Tavo griaustinio balso jie pasišalino. 
\par 8 Jie kyla į kalnus, leidžiasi į slėnius, į vietas, kurias jiems paskyrei. 
\par 9 Nustatei jiems ribą, kad neperžengtų jos ir nebeužlietų žemės. 
\par 10 Tu pasiuntei šaltinius į slėnius, tarp kalnų jie teka. 
\par 11 Iš jų miško žvėrys geria ir laukiniai asilai troškulį savo malšina. 
\par 12 Jų pakrantėse padangių paukščiai gyvena, medžių šakose jie čiulba. 
\par 13 Iš savo kambarių kalnus Tu laistai, Tavo rankų darbas gaivina žemę. 
\par 14 Tu išaugini žolę galvijams ir augalus, kad tarnautų žmogui, kad iš žemės jis maistą sau gautų 
\par 15 ir vyną, kuris linksmina žmogaus širdį. Veidai spindi nuo aliejaus, o duona stiprina žmonių širdis. 
\par 16 Viešpaties medžiai pasisotina, Libano kedrai, Jo pasodinti. 
\par 17 Paukščiai ten krauna lizdus, gandras kipariso viršūnėje sau namus pasidarė. 
\par 18 Aukšti kalnai­laukinėms ožkoms, uolos­triušiams prieglaudą teikia. 
\par 19 Jis sukūrė mėnulį laikui žymėti, saulė žino, kada nusileisti. 
\par 20 Tu siunti tamsą, ir ateina naktis, miško žvėrys sujunda. 
\par 21 Ima riaumoti jauni liūtai, grobio ieškodami, ir prašo Dievą sau maisto. 
\par 22 Kai pateka saulė, jie pasitraukia miegoti į savo lindynes. 
\par 23 Žmogus išeina į darbą ir darbuojasi ligi vakaro. 
\par 24 Viešpatie, kokia daugybė Tavo darbų! Juos išmintingai padarei, žemę pripildei savo turtų. 
\par 25 Štai didelė ir plati jūra. Ten knibžda be skaičiaus įvairaus dydžio gyvūnų. 
\par 26 Ten plaukioja laivai, Tavo sukurtas leviatanas vandeny žaidžia. 
\par 27 Jie visi iš Tavęs laukia, kad duotum jiems maisto reikiamu metu. 
\par 28 Tu duodi jiems, jie rankioja. Tu ištiesi savo ranką, jie pasisotina gausiai. 
\par 29 Tau paslėpus nuo jų veidą, jie išsigąsta. Tu atimi iš jų kvapą, ir jie miršta, dulkėmis virsta. 
\par 30 Atsiunti Tu savo dvasią, sukuri juos ir atnaujini žemės veidą. 
\par 31 Viešpaties šlovė pasiliks per amžius, džiaugsis Viešpats savo darbais. 
\par 32 Jis pažvelgia į žemę, ji sudreba; paliečia kalnus, ir jie rūksta. 
\par 33 Viešpačiui giedosiu, kol gyvensiu, giedosiu gyrių Dievui, kol gyvas būsiu. 
\par 34 Mano apmąstymai Jam patiks; aš džiaugsiuosi Viešpatyje. 
\par 35 Te nusidėjėliai dingsta iš žemės, tenebūna daugiau nedorėlių. Laimink, mano siela, Viešpatį! Girkite Viešpatį!


\chapter{105}


\par 1 Dėkokite Viešpačiui, šaukitės Jo vardo, skelbkite tautose Jo darbus. 
\par 2 Giedokite Jam, giedokite Jam psalmes. Garsinkite visus Jo stebuklus. 
\par 3 Didžiuokitės Jo šventu vardu. Tegu džiaugiasi širdis tų, kurie ieško Viešpaties. 
\par 4 Ieškokite Viešpaties ir Jo jėgos. Nuolatos ieškokite Jo veido. 
\par 5 Atsiminkite Jo nuostabius darbus, kuriuos Jis yra padaręs, Jo stebuklus ir Jo lūpų ištartus sprendimus. 
\par 6 Jūs, Jo tarno Abraomo palikuonys, Jokūbo vaikai, Jo išrinktieji. 
\par 7 Jis yra Viešpats, mūsų Dievas; visoje žemėje galioja Jo sprendimai. 
\par 8 Jis per amžius atsimena savo sandorą, žodį, kurį Jis įsakė tūkstančiui kartų, 
\par 9 sandorą, kurią Jis padarė su Abraomu, ir priesaiką, duotą Izaokui. 
\par 10 Jis patvirtino ją Jokūbui įstatymu, Izraeliui­amžina sandora, 
\par 11 sakydamas: “Tau duosiu Kanaano šalį, tavo paveldėjimo dalį”, 
\par 12 kai jie buvo negausūs skaičiumi, tik ateiviai joje. 
\par 13 Jie keliavo iš tautos į tautą, iš vienos karalystės į kitą. 
\par 14 Jis niekam neleido jų skriausti, sudrausdavo dėl jų karalius: 
\par 15 “Nelieskite mano pateptųjų ir mano pranašams nedarykite pikto”. 
\par 16 Jis žemei badą pašaukė, duonos ramstį sunaikino. 
\par 17 Jis pasiuntė pirma jų vyrą, Juozapą, vergijon parduotą. 
\par 18 Jie supančiojo jo kojas, sukaustė jį geležimi, 
\par 19 kol įvyko Jo žodis; Viešpaties žodis išmėgino jį. 
\par 20 Karalius paleisti jį liepė, tautos valdovas išlaisvino jį. 
\par 21 Savo namų viešpačiu jį paskyrė ir viso savo turto valdovu, 
\par 22 kad vadovautų šalies kunigaikščiams, išminties mokytų vyresniuosius. 
\par 23 Tuomet Izraelis Egiptan atvyko ir Jokūbas viešėjo Chamo krašte. 
\par 24 Čia Jis labai pagausino savo tautą ir padarė ją stipresnę už jų priešus. 
\par 25 Jis pažadino jų širdyse neapykantą savo tautai, ir jie ėmė su Jo tarnais elgtis klastingai. 
\par 26 Jis pasiuntė Mozę, savo tarną, ir Aaroną, kurį išsirinko. 
\par 27 Jie Chamo krašte padarė nuostabių ženklų ir stebuklų. 
\par 28 Jis siuntė tamsą ir aptemdė kraštą, ir jie nepasipriešino Jo žodžiui. 
\par 29 Jis pavertė jų vandenis krauju, jų žuvis išmarino. 
\par 30 Varlės jų žemę apniko, net karalių kambariuose jų buvo. 
\par 31 Jis tarė, ir visą jų kraštą užplūdo įvairios musės bei mašalai. 
\par 32 Vietoje lietaus Jis siuntė krušą ir liepsnojančią ugnį visame krašte. 
\par 33 Jis išmušė figmedžius ir vynmedžius, visame krašte medžius sunaikino. 
\par 34 Jam tarus, užplūdo begalės skėrių ir žiogų, 
\par 35 kurie visus augalus jų krašte ir laukų derlių surijo. 
\par 36 Jis pirmagimius visus krašte ištiko, jų pajėgumo pradžią. 
\par 37 Jis išvedė juos su sidabru ir auksu, jų giminėse nebuvo ligonių. 
\par 38 Džiaugėsi egiptiečiai, jiems iškeliavus, nes labai išgąsdinti buvo. 
\par 39 Jis dengė juos debesimi ir naktį apšvietė juos ugnimi. 
\par 40 Jiems paprašius, Jis putpelių jiems atsiuntė, maitino juos dangaus duona. 
\par 41 Jis perskėlė uolą, ir ištryško vandenys, jie tekėjo kaip upė per sausą žemę. 
\par 42 Jis atsiminė savo šventą pažadą Abraomui, savo tarnui, 
\par 43 ir išvedė savo tautą su džiaugsmu, savo išrinktuosius su linksmybėmis. 
\par 44 Pagonių žemes jiems išdalino, tautų turtai jiems atiteko, 
\par 45 kad Jo įsakymų laikytųsi, vykdytų Jo įstatymą. Girkite Viešpatį!


\chapter{106}


\par 1 Girkite Viešpatį! Dėkokite Viešpačiui, nes Jis geras, nes Jo gailestingumas amžinas. 
\par 2 Kas išvardins galingus Viešpaties darbus, kas apsakys Jo šlovę? 
\par 3 Palaiminti, kurie Jo įsakymus vykdo, kurie visą laiką elgiasi teisiai. 
\par 4 Viešpatie, būdamas palankus savo tautai, atsimink ir mane, suteik man savo išgelbėjimą, 
\par 5 kad matyčiau išrinktųjų gerovę, džiūgaučiau su Tavo tauta, didžiuočiausi su Tavo paveldu. 
\par 6 Nusidėjome su savo tėvais, nusikaltome, elgėmės nedorai. 
\par 7 Mūsų tėvai Egipte nesuprato Tavo stebuklų. Jie užmiršo Tavo didelį gailestingumą, prieš Tave prie Raudonosios jūros maištavo. 
\par 8 Bet Jis dėl savo vardo išgelbėjo juos, kad parodytų savo galybę. 
\par 9 Jis sudraudė Raudonąją jūrą, ir ta išdžiūvo. Jis vedė juos per gelmes kaip per dykumą. 
\par 10 Iš vergijos Jis išgelbėjo juos, išpirko juos iš priešo rankos. 
\par 11 Vandenys užliejo priešus, nė vieno jų neliko. 
\par 12 Tada jie tikėjo Jo žodžiais, giedojo jam gyrių. 
\par 13 Bet greitai pamiršo Jo darbus ir nelaukė Jo patarimų, 
\par 14 dykumoje geiduliams atsidavė ir Dievą tyruose gundė. 
\par 15 Jis suteikė jiems, ko prašė, kartu siuntė ligas į jų būrį. 
\par 16 Pavydėjo jie Mozei stovykloje ir Viešpaties šventajam Aaronui. 
\par 17 Atsivėrusi žemė prarijo Dataną, palaidojo gaują Abiramo. 
\par 18 Užsidegė ugnis tarp jų, nedorėlius sudegino liepsna. 
\par 19 Jie pasidarė veršį Horebe ir garbino nulietą atvaizdą. 
\par 20 Jie iškeitė savo šlovę į pavidalą jaučio, ėdančio žolę. 
\par 21 Jie pamiršo savo gelbėtoją Dievą, kuris didelių dalykų Egipte padarė, 
\par 22 nuostabių darbų Chamo krašte, baisių dalykų prie Raudonosios jūros. 
\par 23 Dievas būtų juos sunaikinęs, jeigu ne Jo išrinktasis Mozė, stojęs užtarti juos prieš Dievą, kad Jo rūstybė jų nenubaustų. 
\par 24 Jie paniekino gerąją žemę, netikėjo Jo žodžiais, 
\par 25 palapinėse savo murmėjo, Viešpaties balso neklausė. 
\par 26 Jis tada pakėlė ranką, kad juos dykumoje sunaikintų, 
\par 27 jų vaikus išblaškytų tarp pagonių, po visas šalis išsklaidytų. 
\par 28 Jie Baal Peorui tarnavo, valgė negyvųjų aukas. 
\par 29 Šitaip jie savo darbais Viešpatį užrūstino, ir maras paplito tarp jų. 
\par 30 Tik kai Finehasas pakilęs teismą įvykdė, liovėsi maras. 
\par 31 Tai buvo jam įskaityta teisumu per visas kartas. 
\par 32 Įpykino jie Viešpatį prie Meribos vandenų, ir Mozė dėl jų nukentėjo. 
\par 33 Jie apkartino jo dvasią, neapgalvotus žodžius jis kalbėjo savo lūpomis. 
\par 34 Jie nesunaikino tautų, kaip Viešpats jiems buvo įsakęs. 
\par 35 Jie su pagonimis susimaišė ir išmoko jų darbus daryti. 
\par 36 Jie stabams jų tarnavo, ir tie spąstais jiems virto. 
\par 37 Jie savo sūnus ir dukteris velniams aukojo, 
\par 38 liejo nekaltą kraują­savo sūnų ir dukterų kraują­aukodami Kanaano stabams; krauju buvo sutepta žemė. 
\par 39 Jie susiteršė savo darbais ir paleistuvavo savo poelgiais. 
\par 40 Tada užsidegė Viešpaties rūstybė prieš savo tautą, bjaurus Jam tapo Jo paveldas. 
\par 41 Atidavė juos pagonims, tie, kurie jų nekentė, valdė juos. 
\par 42 Juos spaudė priešai ir slėgė jų ranka. 
\par 43 Daug kartų Jis išlaisvino juos, bet jie neklausė Jo patarimų; dėl savo nedorybių jie buvo pažeminti. 
\par 44 Tačiau Viešpats atsižvelgė į jų priespaudą, išgirdęs jų šauksmą, 
\par 45 atsiminė jų labui savo sandorą. Jis gailėjosi jų, būdamas didžiai gailestingas. 
\par 46 Jis davė jiems rasti pasigailėjimą akyse tų, kurie išsivedė juos į nelaisvę. 
\par 47 Išgelbėk mus, Viešpatie, mūsų Dieve, ir surankiok tautose išblaškytus, kad dėkotume Tavo šventam vardui, girtumėmės Tavo šlove. 
\par 48 Palaimintas Viešpats, Izraelio Dievas, nuo amžių ir per amžius! Visa tauta tesako: “Amen”. Girkite Viešpatį!


\chapter{107}


\par 1 Dėkokite Viešpačiui, nes Jis geras, nes Jo gailestingumas amžinas. 
\par 2 Taip tekartoja Viešpaties išpirktieji, kuriuos Jis išpirko iš priešo rankos, 
\par 3 ir surinko juos iš kraštų: iš rytų ir vakarų, iš šiaurės ir pietų. 
\par 4 Po dykumą jie klajojo tuščiais keliais, nerasdami miesto, kur galėtų gyventi. 
\par 5 Jie alko ir troško, jų sielos nusilpo. 
\par 6 Varge jie Viešpaties šaukėsi, Jis iš sielvartų juos išvadavo. 
\par 7 Jis vedė juos teisingu keliu, kad jie nueitų į gyvenamą miestą. 
\par 8 Tegiria jie Viešpatį už Jo gerumą, už Jo stebuklus žmonių vaikams. 
\par 9 Jis pagirdė trokštančią sielą, išalkusią sielą pripildė gėrybių. 
\par 10 Kurie sėdėjo tamsoje ir mirties šešėlyje, geležimi ir skurdu sukaustyti,­ 
\par 11 nes buvo sukilę prieš Dievo žodžius ir paniekinę Aukščiausiojo patarimą, 
\par 12 todėl Jis pažemino vargu jų širdis,­ krito, ir niekas jiems nepadėjo. 
\par 13 Varge jie šaukėsi Viešpaties, Jis iš sielvartų juos išgelbėjo. 
\par 14 Jis išvedė juos iš tamsos ir mirties šešėlio ir sutraukė jų pančius. 
\par 15 Tegiria jie Viešpatį už Jo gerumą, už Jo stebuklus žmonių vaikams. 
\par 16 Jis sudaužė varinius vartus ir geležinius skląsčius sulaužė. 
\par 17 Kvailiai dėl savo nedorybių ir dėl savo kalčių kenčia. 
\par 18 Jiems nebemielas joks valgis, jie priartėjo prie mirties vartų. 
\par 19 Varge jie šaukiasi Viešpaties, Jis iš sielvartų juos išgelbsti. 
\par 20 Jis siuntė savo žodį ir išgydė juos, iš pražūties juos išlaisvino. 
\par 21 Tegiria jie Viešpatį už Jo gerumą, už Jo stebuklus žmonių vaikams. 
\par 22 Teaukoja Jam padėkos aukas ir džiūgaudami teskelbia Jo darbus. 
\par 23 Kas laivais plaukia į jūrą, vandenų platybėje prekiauja, 
\par 24 tie mato Viešpaties darbus ir Jo stebuklus gelmėse. 
\par 25 Jam įsakius, audros pakyla, šiaušiasi bangos. 
\par 26 Ligi debesų jie pakyla, į gelmes vėl sminga, širdis jų tirpsta nelaimėje. 
\par 27 Jie svirduliuoja ir klydinėja kaip girti, jų jėgos baigia išsekti. 
\par 28 Varge jie Viešpaties šaukiasi, Jis iš sielvartų juos išvaduoja. 
\par 29 Jis nutildo audrą, nuramina bangas. 
\par 30 Tada jie džiaugiasi nurimusia jūra. Jis nuveda juos į geidžiamą uostą. 
\par 31 Tegiria jie Viešpatį už Jo gerumą, už Jo stebuklus žmonių vaikams. 
\par 32 Jie teaukština Jį tautos susirinkime, tegiria Jį vyresniųjų taryboje. 
\par 33 Upes Jis dykuma paverčia, vandens šaltinius­sausa žeme, 
\par 34 derlingą žemę paverčia druskynais dėl nedorybių žmonių, kurie joje gyvena. 
\par 35 Jis dykumą ežeru paverčia ir sausą žemę­vandens šaltiniais. 
\par 36 Jis alkanuosius apgyvendina ten, kad jie įkurtų gyvenamą miestą. 
\par 37 Jie apsėja laukus, pasodina vynuogynus, kurie atneša derlių. 
\par 38 Jis palaimina juos, ir jų labai padaugėja, jų galvijams neleidžia mažėti. 
\par 39 Kai jų sumažėja ir jie pažeminami priespauda, nelaimėmis ir širdgėla, 
\par 40 Jis paniekina kunigaikščius ir klaidina juos dykumoje be kelio, 
\par 41 bet Jis pakelia vargšą iš nelaimės, padaro gausią kaip bandą jo giminę. 
\par 42 Teisieji tai matys ir džiaugsis, o nedorybė užčiaups savo burną. 
\par 43 Išmintingasis tai pamatys ir supras Viešpaties malonę.


\chapter{108}


\par 1 O Dieve, mano širdis pasiruošusi. Aš giedosiu ir girsiu savo šlovėje. 
\par 2 Pabuskite, psalteri ir arfa! Aš taip pat anksti atsikelsiu. 
\par 3 Girsiu Tave, Viešpatie, tarp tautų, giedosiu gyrių Tau tarp pagonių. 
\par 4 Tavo gailestingumas siekia aukščiau dangų, Tavo tiesa­iki debesų. 
\par 5 Būk išaukštintas danguose ir Tavo šlovė tebūna visoje žemėje, 
\par 6 kad Tavo mylimieji būtų išlaisvinti; išgelbėk savo dešine ir išklausyk mane. 
\par 7 Dievas kalbėjo savo šventume: “Aš džiūgausiu ir išdalinsiu Sichemą, paskirstysiu Sukotų slėnį. 
\par 8 Mano yra Gileadas ir Manasas, Efraimas­mano galvos šalmas, Judas­mano skeptras. 
\par 9 Moabas yra mano praustuvė. Ant Edomo numesiu savo kurpę. Filistijoje Aš džiaugsiuosi pergale”. 
\par 10 Kas įves mane į sustiprintą miestą? Kas nuves mane į Edomą? 
\par 11 Dieve, argi ne Tu mus atstūmei? Argi ne Tu, Dieve, neišėjai su mūsų kariuomene? 
\par 12 Suteik mums pagalbą varge, nes žmonių pagalba yra be vertės. 
\par 13 Su Dievu būsime drąsūs, Jis sutryps mūsų priešus.


\chapter{109}


\par 1 Netylėk, mano gyriaus Dieve. 
\par 2 Nedorėlio burna ir klastingojo burna prieš mane atsivėrė. Jie kalba prieš mane melagingais liežuviais. 
\par 3 Neapykantos žodžiais jie apsupo mane ir puola mane nekaltą. 
\par 4 Už mano meilę jie kaltina mane, bet aš meldžiuosi. 
\par 5 Piktu už gera jie man atlygina ir neapykanta už meilę. 
\par 6 Atiduok tokį nedorėlių valiai, te šėtonas stovi jo dešinėje. 
\par 7 Teisme tebūna jis pasmerktas, jo malda tebūna nuodėmė. 
\par 8 Tebūna jo gyvenimas trumpas. Jo tarnystę tegauna kitas. 
\par 9 Jo vaikai tepalieka našlaičiais ir žmona­našle. 
\par 10 Elgetomis ir benamiais tegu tampa jo vaikai, tebūna jie išmesti iš savų sunaikintų namų. 
\par 11 Skolintojas tepasiglemžia jo turtą ir svetimieji jo uždarbį teišgrobsto. 
\par 12 Nė vienas jo tenesigaili ir tenebūna kas užjaustų jo našlaičius. 
\par 13 Jo palikuonys tesunyksta. Kitoje kartoje teišdyla jų vardas. 
\par 14 Viešpats teatsimena jo tėvų kaltes, ir jo motinos nuodėmė tenebūna išdildyta. 
\par 15 Tegu nuolat Viešpats juos stebi, kad nuo žemės nušluotų jų atminimą. 
\par 16 Nes neparodė jis gailestingumo, bet persekiojo beturtį ir vargšą, kėsinosi nužudyti sudužusį širdyje. 
\par 17 Jis mėgo prakeikimą, tegu jis užklumpa jį; jis nemėgo palaiminimo, tebūna jis toli nuo jo. 
\par 18 Jis apsivilko prakeikimu kaip drabužiu, todėl kaip vanduo jis teįsisunkia į kūną, kaip aliejus į kaulus. 
\par 19 Jis tebūna jam kaip drabužis, kuris dengia jo kūną, kaip juosta, kuria jis susijuosia. 
\par 20 Taip tegul užmoka Viešpats mano priešininkams ir tiems, kurie kalba pikta prieš mane. 
\par 21 Bet Tu, Viešpatie Dieve, sustiprink mane dėl savojo vardo, gelbėk dėl savo gailestingumo. 
\par 22 Aš esu vargšas ir beturtis, mano širdis sužeista. 
\par 23 Nykstu kaip šešėlis, mane nešioja kaip vėjas skėrį. 
\par 24 Mano keliai nuo pasninko linksta, sulyso mano kūnas. 
\par 25 Aš jiems tapau pajuoka. Matydami mane, jie kraipė galvas. 
\par 26 Padėk man, Viešpatie, mano Dieve, išgelbėk mane, būdamas gailestingas. 
\par 27 Viešpatie, težino jie, jog tai Tavo ranka padarė. 
\par 28 Jie tegul keikia, bet Tu laimink! Kai jie pakyla, tebūna sugėdinti, o Tavo tarnas tesidžiaugia. 
\par 29 Teapsivelka mano priešininkai nešlove, juos gėda kaip drabužis teapgaubia. 
\par 30 Savo burna garsiai girsiu Viešpatį, girsiu Jį minioje. 
\par 31 Jis stovi beturčio dešinėje, gina jį nuo pasmerkėjų.


\chapter{110}


\par 1 Viešpats tarė mano Viešpačiui: “Sėskis mano dešinėje, kol patiesiu Tavo priešus tarsi pakojį po Tavo kojomis”. 
\par 2 Viešpats duos Tau iš Siono Tavo stiprybės skeptrą. Valdyk savo priešus! 
\par 3 Savanoriškai Tavo tauta susirinks Tavo pergalės dieną šventumo grožyje. Jaunimas lyg aušros rasa Tave pasitiks. 
\par 4 Viešpats prisiekė ir nesigailės: “Tu esi kunigas per amžius Melchizedeko tvarka”. 
\par 5 Viešpats Tavo dešinėje sunaikins karalius savo rūstybės dieną. 
\par 6 Jis darys teismą tarp pagonių. Pripildys žemę lavonų, daugelyje šalių įvykdys sprendimą valdovams. 
\par 7 Jis kelyje gers iš upelio, todėl iškels savo galvą.


\chapter{111}


\par 1 Girkite Viešpatį! Aš girsiu Viešpatį visa širdimi teisiųjų susirinkime. 
\par 2 Dideli yra Viešpaties darbai, tyrinėjami visų, kurie juos mėgsta. 
\par 3 Didingi ir šlovingi Jo darbai, Jo teisumas pasilieka per amžius. 
\par 4 Atsimintini yra Jo darbai. Maloningas ir užjaučiantis yra Viešpats. 
\par 5 Jis maitina tuos, kurie Jo bijo, per amžius prisimena savo sandorą. 
\par 6 Jis savo darbų galią tautai parodė, atidavė jiems pagonių nuosavybę. 
\par 7 Jo rankų darbai­teisingumas ir teismas, visi Jo įsakymai nepakeičiami. 
\par 8 Jie per amžius yra tvirti, paremti tiesa ir lygybe visiems. 
\par 9 Išpirkimą Jis savo tautai siuntė, amžiams paskelbė savo sandorą. Šventas ir gerbtinas yra Jo vardas. 
\par 10 Išminties pradžia yra Viešpaties baimė; supratingi, kurie taip elgiasi. Jo šlovė lieka per amžius!


\chapter{112}


\par 1 Girkite Viešpatį! Palaimintas žmogus, kuris Viešpaties bijosi ir Jo įsakymus labai mėgsta. 
\par 2 Jo palikuonys bus galingi žemėje; palaiminta bus dorųjų karta. 
\par 3 Jo namuose bus gerovė ir turtai, jo teisumas pasiliks per amžius. 
\par 4 Dorajam šviesa nušvinta tamsoje, jis yra maloningas, kupinas gailestingumo ir teisus. 
\par 5 Geras žmogus pasigaili ir skolina, jis teisingai savo reikalus tvarko. 
\par 6 Jis nesvyruos per amžius. Amžinai bus minimas teisiojo vardas. 
\par 7 Bloga žinia jo neišgąsdins; tvirta yra jo širdis, nes jis pasitiki Viešpačiu. 
\par 8 Įsitvirtinusi jo širdis, jis nebijos, kol pamatys sugėdintus savo priešus. 
\par 9 Jis beturčiams dovanas dosniai dalina. Jo teisumas pasilieka per amžius. Jo ragas iškils garbingai. 
\par 10 Nedorėlis tai matys ir graušis, dantimis grieš ir nyks. Nedorėlių troškimas pražus.


\chapter{113}


\par 1 Girkite Viešpatį! Viešpaties tarnai, girkite Viešpaties vardą! 
\par 2 Palaimintas Viešpaties vardas dabar ir per amžius. 
\par 3 Nuo saulės užtekėjimo iki nusileidimo tebūna giriamas Viešpaties vardas! 
\par 4 Viešpats yra virš visų tautų, Jo šlovė aukščiau nei dangus. 
\par 5 Kas yra kaip Viešpats, mūsų Dievas, kuris gyvena aukštybėse? 
\par 6 Jis pasilenkia žiūrėti į dangų ir žemę. 
\par 7 Jis kelia iš dulkių beturtį, iš purvo vargšą ištraukia, 
\par 8 sodina jį greta kunigaikščių, šalia savo tautos kunigaikščių. 
\par 9 Jis suteikia nevaisingajai namus, padarydamas ją laiminga vaikų motina. Girkite Viešpatį!


\chapter{114}


\par 1 Kai Izraelis iš Egipto išėjo, Jokūbo namai pagonių tautą paliko, 
\par 2 Judas buvo Jo šventykla, Izraelis­Jo karalystė. 
\par 3 Pamatė jūra ir pabėgo, atgal pasitraukė Jordanas. 
\par 4 Kalnai šokinėjo kaip avys, kalvos­kaip ėriukai. 
\par 5 Kas gi tau, jūra, ko pabėgai? Kodėl pasitraukei, Jordane? 
\par 6 Kalnai, ko šokinėjate kaip avys, kalvos­kaip ėriukai? 
\par 7 Drebėk, žeme, Viešpaties akivaizdoje, Jokūbo Dievo akivaizdoje. 
\par 8 Jis pavertė uolą vandeniu, iš akmens šaltiniai ištryško!


\chapter{115}


\par 1 Ne mums, Viešpatie, ne mums, tik Tavo vardui tebūna šlovė dėl Tavo gailestingumo ir tiesos. 
\par 2 Kodėl turėtų sakyti pagonys: “Kur yra jų Dievas?” 
\par 3 Mūsų Dievas danguje, Jis padarė visa, kas Jam patiko. 
\par 4 Jų stabai iš sidabro ir aukso, jie­žmogaus rankų darbas. 
\par 5 Jie turi burnas, bet nekalba; turi akis, bet nemato; 
\par 6 turi ausis, tačiau negirdi; turi nosį, bet nesuuodžia; 
\par 7 turi rankas, bet nepaliečia; turi kojas, bet nevaikšto. Jie nekalba savo gerkle. 
\par 8 Į juos panašūs yra tie, kas juos padaro ir jais pasitiki. 
\par 9 Izraeli, pasitikėk Viešpačiu! Jis jūsų pagalba ir skydas. 
\par 10 Aarono namai, pasitikėkite Viešpačiu! Jis jūsų pagalba ir skydas. 
\par 11 Kurie bijote Viešpaties, pasitikėkite Viešpačiu! Jis jūsų pagalba ir skydas. 
\par 12 Viešpats mus atsimena; Jis palaimins mus. Jis palaimins Izraelį ir Aarono namus. 
\par 13 Jis palaimins visus, kurie Viešpaties bijo, mažus ir didelius. 
\par 14 Viešpats pagausins jus ir jūsų vaikus. 
\par 15 Jūs esate palaiminti Viešpaties, kuris sutvėrė dangų ir žemę. 
\par 16 Dangūs yra Viešpaties buveinė, o žemę Jis atidavė žmonėms. 
\par 17 Ne mirusieji giria Viešpatį, ne mirties karalijon nužengę. 
\par 18 Bet mes šloviname Viešpatį dabar ir per amžius. Girkite Viešpatį!


\chapter{116}


\par 1 Myliu Viešpatį, nes Jis išklausė mano maldavimą. 
\par 2 Jis palenkė į mane savo ausį, todėl šauksiuosi Jo, kol gyvas būsiu. 
\par 3 Mirties skausmai apsupo mane, pragaro kančios apėmė mane; kenčiau vargą ir skausmus. 
\par 4 Viešpaties vardo šaukiausi: “Viešpatie, išlaisvink mano sielą!” 
\par 5 Maloningas yra Viešpats ir teisus, gailestingas mūsų Dievas. 
\par 6 Viešpats apsaugo paprastąjį; suvargęs buvau, o Jis man padėjo. 
\par 7 Nurimk, mano siela, nes Viešpats padarė tau gera! 
\par 8 Tu išlaisvinai nuo mirties mano sielą, nuo ašarų­mano akis, nuo suklupimo­mano kojas. 
\par 9 Aš vaikščiosiu Viešpaties akivaizdoje gyvųjų šalyje. 
\par 10 Nors pasitikėjau, bet tariau: “Aš esu labai prislėgtas!” 
\par 11 Neapgalvojęs sakiau: “Visi žmonės melagiai”. 
\par 12 Kuo gi Viešpačiui atsilyginsiu už visas Jo geradarystes? 
\par 13 Aš paimsiu išgelbėjimo taurę, Viešpaties vardo šauksiuosi. 
\par 14 Savo įžadus Viešpačiui ištesėsiu visos tautos akivaizdoje. 
\par 15 Viešpaties akyse brangi yra šventųjų mirtis. 
\par 16 Viešpatie, aš esu Tavo tarnas, Tavo tarnas ir sūnus Tavo tarnaitės. Tu mano pančius sutraukei. 
\par 17 Tau padėkos auką aukosiu, šauksiuosi Viešpaties vardo. 
\par 18 Savo įžadus Viešpačiui ištesėsiu visos tautos akivaizdoje, 
\par 19 Viešpaties namų kiemuose, Jeruzalės mieste. Girkite Viešpatį!


\chapter{117}


\par 1 Girkite Viešpatį, visos tautos. Girkite Jį, visi žmonės! 
\par 2 Jo gailestingumas mums begalinis, Viešpaties tiesa pasilieka per amžius. Girkite Viešpatį!


\chapter{118}


\par 1 Dėkokite Viešpačiui, nes Jis geras, nes Jo gailestingumas amžinas. 
\par 2 Tegul Izraelis sako: “Jo gailestingumas amžinas”. 
\par 3 Tegul Aarono namai sako: “Jo gailestingumas amžinas”. 
\par 4 Visi, kas Viešpaties bijo, tesako: “Jo gailestingumas amžinas”. 
\par 5 Sielvarte Viešpaties šaukiausi. Viešpats išklausė ir išlaisvino mane. 
\par 6 Viešpats yra už mane­ko man bijoti? Ką gali padaryti man žmogus? 
\par 7 Viešpats yra tarp tų, kurie man padeda, todėl aš matysiu sugėdintus savo priešus. 
\par 8 Geriau pasitikėti Viešpačiu, negu sudėti viltis į žmones. 
\par 9 Geriau pasitikėti Viešpačiu, negu sudėti viltis į kunigaikščius. 
\par 10 Visos tautos apgulė mane, bet Viešpaties vardu juos nugalėsiu. 
\par 11 Jie iš visų pusių apgulė mane, tačiau Viešpaties vardu juos nugalėsiu. 
\par 12 Apspito jie mane kaip bitės, jie sudegs kaip erškėtis ugnyje, nes Viešpaties vardu juos nugalėsiu. 
\par 13 Tu stūmei mane, kad parpulčiau, bet man padėjo Viešpats. 
\par 14 Mano stiprybė ir giesmė yra Viešpats, Jis­mano išgelbėjimas. 
\par 15 Džiaugsmo ir išgelbėjimo šauksmas teisiųjų palapinėse. Viešpaties dešinė pergalę teikia. 
\par 16 Viešpaties dešinė aukštai pakelta, Viešpaties dešinė pergalę teikia. 
\par 17 Aš nemirsiu, bet gyvensiu ir skelbsiu Viešpaties darbus. 
\par 18 Nors Viešpats mane griežtai baudė, bet mirčiai neatidavė. 
\par 19 Atverkite man teisumo vartus! Įžengęs pro juos girsiu Viešpatį. 
\par 20 Tai Viešpaties vartai, pro kuriuos įeis teisieji. 
\par 21 Šlovinsiu Tave, nes išklausei mane ir tapai mano išgelbėjimu. 
\par 22 Akmuo, kurį statytojai atmetė, tapo kertiniu akmeniu. 
\par 23 Tai Viešpats padarė, ir mūsų akims tai nuostabą kelia. 
\par 24 Šitą dieną Viešpats padarė; džiūgaukime ir linksminkimės šiandien. 
\par 25 Viešpatie, gelbėk mane! Viešpatie, leisk man klestėti! 
\par 26 Palaimintas, kuris ateina Viešpaties vardu. Mes laiminame jus iš Viešpaties namų. 
\par 27 Dievas yra Viešpats ir Jis mus apšvietė. Suriškite auką virvėmis ir eikite ligi aukuro. 
\par 28 Tu esi mano Dievas, aš šlovinsiu Tave; Tu esi mano Dievas, aš aukštinsiu Tave. 
\par 29 Dėkokite Viešpačiui, nes Jis geras, nes Jo gailestingumas amžinas!


\chapter{119}


\par 1 Palaiminti, kurių kelias nepeiktinas, kurie pagal Viešpaties įstatymą vaikšto. 
\par 2 Palaiminti, kurie klauso Jo liudijimų ir visa širdimi Jo ieško, 
\par 3 nedaro jie neteisybės, vaikšto Jo keliais. 
\par 4 Tu įsakei stropiai vykdyti Tavo potvarkius. 
\par 5 O kad mano keliai būtų nukreipti vykdyti Tavo nuostatus! 
\par 6 Niekada nepatirčiau gėdos, jei į Tavo įsakymus žiūrėčiau. 
\par 7 Tyra širdimi Tave šlovinsiu, teisius Tavo nuosprendžius pažinęs. 
\par 8 Nuostatus Tavo vykdysiu, nepalik manęs visiškai. 
\par 9 Kaip gali jaunuolis savo kelią išlaikyti tyrą? Laikydamasis Tavo žodžių. 
\par 10 Visa širdimi ieškojau Tavęs, neleisk man nuo įsakymų Tavo nuklysti. 
\par 11 Giliai širdyje paslėpiau Tavo žodį, kad Tau nenusidėčiau. 
\par 12 Palaimintas esi, Viešpatie, mane savo nuostatų mokyk. 
\par 13 Savo lūpomis skelbiau visus Tavo sprendimus. 
\par 14 Tavo liudijimų keliais džiaugiuosi labiau negu visais turtais. 
\par 15 Apie Tavo potvarkius nuolat mąstysiu ir žiūrėsiu į Tavo kelius. 
\par 16 Nuostatais Tavo gėrėsiuos, nepamiršiu Tavo žodžių. 
\par 17 Suteik savo tarnui malonę, kad aš gyvendamas Tavo žodžio laikyčiausi. 
\par 18 Atverk man akis, kad stebuklus Tavo įstatyme regėčiau. 
\par 19 Esu žemėje svečias, neslėpk nuo manęs savo įsakymų. 
\par 20 Mano siela pailso, besiilgėdama Tavo sprendimų. 
\par 21 Tu sudraudi išdidžiuosius; prakeikti nuklydę nuo Tavo įsakymų. 
\par 22 Pašalink nuo manęs panieką ir gėdą, nes laikausi Tavo liudijimų. 
\par 23 Kunigaikščiai susirinkę tariasi prieš mane, bet Tavo tarnas mąsto apie Tavo nuostatus. 
\par 24 Tavo liudijimai yra mano pasimėgimas ir mano patarėjai. 
\par 25 Mano siela nublokšta į dulkes, atgaivink mane, kaip esi pažadėjęs. 
\par 26 Savo kelius paskelbiau, ir Tu išklausei mane; pamokyk mane savo nuostatų. 
\par 27 Leisk man suvokti Tavo potvarkių kelią, tai kalbėsiu apie Tavo stebuklus. 
\par 28 Mano siela nyksta iš sielvarto, sustiprink mane, kaip esi pažadėjęs. 
\par 29 Melo kelią pašalink nuo manęs, savo įstatymu mane apdovanok. 
\par 30 Pasirinkau tiesos kelią, Tavo sprendimus laikau priešais save. 
\par 31 Įsitvėriau Tavo liudijimų; Viešpatie, neleisk man patirti gėdos. 
\par 32 Tavo įsakymų keliu bėgsiu, kai išplėsi mano širdį. 
\par 33 Viešpatie, pamokyk mane savo nuostatų kelio, tai iki galo jo laikysiuos. 
\par 34 Duok suprasti Tavo įstatymą, kad vykdyčiau ir nuoširdžiai jo laikyčiausi. 
\par 35 Savo įsakymų takais mane vesk, nes jais aš gėriuosi. 
\par 36 Palenk mano širdį prie liudijimų savo, o ne prie godumo, 
\par 37 nugręžk mano akis nuo tuštybių; atgaivink mane savo kelyje. 
\par 38 Ištesėk pažadą, duotą savo tarnui, kuris bijosi Tavęs. 
\par 39 Nukreipk nuo manęs gėdą, kuri baugina mane, nes Tavo sprendimai yra geri. 
\par 40 Štai aš ilgiuosi Tavo potvarkių, atgaivink mane savo teisumu. 
\par 41 Viešpatie, būk man gailestingas, teateina Tavo išgelbėjimas, kaip Tu pažadėjai. 
\par 42 Tada duosiu atsakymą tam, kuris iš manęs tyčiojasi, nes pasitikiu Tavo žodžiu. 
\par 43 Neatimk iš manęs tiesos žodžio, nes laukiu Tavo sprendimų. 
\par 44 Per amžių amžius laikysiuos Tavo įstatymo. 
\par 45 Vaikščiosiu laisvas, nes tyrinėju Tavo potvarkius. 
\par 46 Kalbėsiu apie Tavo liudijimus karalių akivaizdoje ir nebūsiu sugėdintas. 
\par 47 Gėrėsiuosi Tavo įsakymais, kuriuos pamilau. 
\par 48 Kelsiu rankas į Tavo įsakymus, kuriuos pamilau, mąstysiu apie Tavo nuostatus. 
\par 49 Prisimink žodį savo tarnui, kuriuo suteikei man viltį. 
\par 50 Tai yra paguoda mano varge, nes Tavo žodis mane atgaivino. 
\par 51 Nors pasipūtėliai mane skaudžiai išjuokia, nuo Tavo įstatymo aš nenukrypau. 
\par 52 Viešpatie, aš prisimenu Tavo senus nuosprendžius ir jais pasiguodžiu. 
\par 53 Mane siaubas apima, kai matau nedorėlį, nepaisantį Tavo įstatymo. 
\par 54 Tavo nuostatai tapo man giesmėmis mano viešnagės namuose. 
\par 55 Ir naktį atsimenu, Viešpatie, Tavąjį vardą ir laikausi Tavo įstatymo. 
\par 56 Tai teko man, nes aš laikiausi Tavo potvarkių. 
\par 57 Viešpats yra mano dalis; aš pasižadėjau Tavo žodžių laikytis. 
\par 58 Nuoširdžiai ieškau Tavo palankumo, būk gailestingas, kaip esi pažadėjęs. 
\par 59 Galvojau apie savo kelią ir pasukau link Tavo liudijimų. 
\par 60 Skubiai ir nedelsdamas vykdau Tavo įsakymus. 
\par 61 Nors nedorėliai apiplėšė mane, bet aš neužmiršau Tavo įstatymo. 
\par 62 Vidurnaktį atsikėlęs, dėkoju už teisingus Tavo sprendimus. 
\par 63 Aš draugas visiems, kurie Tavęs bijo ir Tavo potvarkius vykdo. 
\par 64 Viešpatie, žemė pilna Tavo gailestingumo, mokyk mane savo nuostatų. 
\par 65 Viešpatie, Tu darei savo tarnui gera, kaip buvai pažadėjęs. 
\par 66 Mokyk mane teisingai nuspręsti ir pasirinkti, nes aš patikėjau Tavo įsakymais. 
\par 67 Nuklydęs ir pažemintas buvau, bet dabar klausau Tavo žodžio. 
\par 68 Tu esi geras ir darai gera, mokyk mane savo nuostatų. 
\par 69 Šmeižtais drabsto mane pasipūtėliai, bet aš nuoširdžiai Tavo potvarkių laikausi. 
\par 70 Jų širdis vieni riebalai, o aš gėriuosi Tavo įstatymu. 
\par 71 Naudinga man buvo nukentėti, kad Tavo nuostatų pasimokyčiau. 
\par 72 Man Tavo įstatymas brangesnis už daugybę aukso ir sidabro. 
\par 73 Tavo rankos padarė ir suformavo mane; suteik man išminties suprasti Tavo įsakymus. 
\par 74 Kurie Tavęs bijo, džiaugiasi mane matydami, nes Tavo žodžiu pasitikiu. 
\par 75 Viešpatie, žinau, jog teisingi Tavo sprendimai ir teisingai mane nubaudei. 
\par 76 Paguosk dabar mane savo malone, kaip esi savo tarnui žadėjęs. 
\par 77 Būk gailestingas, kad aš išlikčiau gyvas, nes gėriuosi Tavo įstatymu. 
\par 78 Sugėdinti tebūna išdidieji, nes jie be priežasties puolė mane. Aš mąstysiu apie Tavo potvarkius. 
\par 79 Tesigręžia į mane, kurie Tavęs bijo, kurie pažino Tavo liudijimus. 
\par 80 Tegu mano širdis nepažeidžia nuostatų Tavo, kad nebūčiau sugėdintas. 
\par 81 Mano siela ilgisi Tavo išgelbėjimo, bet aš pasitikiu Tavo žodžiu. 
\par 82 Mano akys pavargo belaukdamos, kas Tavo žadėta. Kada Tu mane paguosi? 
\par 83 Nors tapau panašus į vynmaišį dūmuose, bet Tavo nuostatų neužmiršau. 
\par 84 Kiek dar dienų liko Tavo tarnui? Kada mano persekiotojus pasmerksi? 
\par 85 Išdidieji kasa man duobę, nepaisydami Tavo įstatymo. 
\par 86 Visi Tavo įsakymai teisūs; padėk man prieš melagingus persekiotojus. 
\par 87 Jie vos nesunaikino manęs žemėje. Bet aš Tavo potvarkių neapleidau. 
\par 88 Atgaivink mane dėl savo malonės! Aš laikysiuosi Tavo burnos liudijimų. 
\par 89 Viešpatie, Tavo žodis amžinai įtvirtintas danguje. 
\par 90 Tavo ištikimybė kartų kartoms; Tu sutvėrei žemę, ir ji pasilieka. 
\par 91 Ligi šiol viskas laikosi, kaip Tavo nutarta, Tau viskas tarnauja. 
\par 92 Jei Tavo įstatymu nesigėrėčiau, seniai būčiau žuvęs. 
\par 93 Niekada neužmiršiu Tavo potvarkių, nes jais Tu atgaivinai mane. 
\par 94 Aš esu Tavo, išgelbėk mane, trokštu suvokti Tavo potvarkius. 
\par 95 Nedorėliai mane pražudyti kėsinasi, tačiau aš Tavo liudijimų laikysiuosi. 
\par 96 Mačiau, kad tobuliausi dalykai yra riboti, tik įsakymas Tavo beribis. 
\par 97 Kaip aš myliu Tavo įstatymą, mąstau apie jį ištisą dieną. 
\par 98 Įsakymai Tavo padarė mane protingesnį už mano priešus, nes jie visuomet su manimi. 
\par 99 Daugiau suprantu už visus savo mokytojus, nes mąstau apie Tavo liudijimus. 
\par 100 Daugiau išmanau už senius, nes laikausi Tavo potvarkių. 
\par 101 Nuo bet kokio pikto kelio susilaikau, nes klausau Tavo žodžio. 
\par 102 Nuo Tavo sprendimų nenukrypau, nes Tu mokai mane. 
\par 103 Kokie saldūs man yra Tavo žodžiai, saldesni mano burnai už medų. 
\par 104 Dėl Tavo potvarkių tapau išmintingas, todėl nekenčiu jokio melo. 
\par 105 Tavo žodis yra žibintas mano kojai ir šviesa mano takui. 
\par 106 Prisiekiau vykdyti Tavo teisingus sprendimus ir laikysiuosi jų. 
\par 107 Viešpatie, esu labai prislėgtas, atgaivink mane, kaip esi pažadėjęs. 
\par 108 Viešpatie, priimk mano lūpų laisvos valios auką, pamokyk mane savo sprendimų. 
\par 109 Mano siela yra nuolat mano rankoje, tačiau Tavo įstatymo nepamirštu. 
\par 110 Nedorėliai man spendžia žabangus, bet nuo Tavo potvarkių nenukrypau. 
\par 111 Pamokymai Tavo yra mano paveldėtas turtas, jie mano širdies džiaugsmas. 
\par 112 Palenkiau savo širdį vykdyti Tavo nuostatų, tai darysiu dabar ir visados. 
\par 113 Veidmainių nekenčiu, bet Tavo įstatymą myliu. 
\par 114 Tu esi mano slėptuvė ir skydas, Tavo žodžiais pasitikiu. 
\par 115 Pasitraukite nuo manęs, piktadariai! Aš Viešpaties įsakymus vykdysiu. 
\par 116 Palaikyk mane, kaip žadėjai, kad gyvenčiau, tenebūsiu sugėdintas dėl savo vilties. 
\par 117 Suteik pagalbą, ir aš būsiu saugus, nuostatų Tavo niekados nepamiršiu. 
\par 118 Tu atmeti tuos, kurie nuo Tavo nuostatų nukrypo, jie apsigauna savo melu. 
\par 119 Tu visus žemės nedorėlius pašalini tartum atmatas, bet aš myliu Tavo liudijimus. 
\par 120 Mano kūnas dreba, bijodamas Tavęs, aš bijau Tavo sprendimų. 
\par 121 Dariau, kas yra teisu ir teisinga; neatiduok manęs prispaudėjams. 
\par 122 Užtikrink savo tarnui gerovę, neleisk, kad išdidieji mane nugalėtų. 
\par 123 Mano akys pavargo belaukdamos Tavo išgelbėjimo ir Tavo teisumo žodžio. 
\par 124 Būk gailestingas savo tarnui, mokyk mane savo nuostatų. 
\par 125 Esu Tavo tarnas; duok man supratimą pažinti Tavo liudijimus. 
\par 126 Viešpatie, metas Tau veikti, nes žmonės laužo Tavo įstatymą. 
\par 127 Tavo įsakymai brangesni man už auksą, už gryną auksą. 
\par 128 Todėl visus Tavo potvarkius laikau teisingais, nekenčiu melagingų takų. 
\par 129 Tavo liudijimai yra stebuklingi, todėl mano siela jų klauso. 
\par 130 Tavo žodžių aiškinimas apšviečia, neišmanančius daro supratingus. 
\par 131 Atveriu savo burną ir įkvepiu, alkstu Tavo įsakymų. 
\par 132 Pažvelk į mane ir būk gailestingas, kaip darai mylintiems Tavo vardą. 
\par 133 Kreipk mano žingsnius pagal savo žodžius, kad neteisybė man neviešpatautų. 
\par 134 Nuo žmonių priespaudos mane išlaisvink, ir aš vykdysiu Tavo potvarkius. 
\par 135 Parodyk savo tarnui savo veidą šviesų ir mokyk mane savo nuostatų. 
\par 136 Iš akių man srūva upeliai, nes jie nesilaiko Tavo įstatymo. 
\par 137 Viešpatie, Tu esi teisus, teisingi Tavo sprendimai. 
\par 138 Tavo liudijimai teisingi ir neabejotini. 
\par 139 Mano uolumas graužia mane, nes priešai pamiršo Tavo žodžius. 
\par 140 Tavo žodis yra visiškai tyras, jį Tavo tarnas brangina. 
\par 141 Nors esu paniekintas ir menkas, bet Tavo potvarkių neužmiršiu. 
\par 142 Tavo teisumas amžinas, Tavo įstatymas­tiesa. 
\par 143 Nors vargas ir sielvartas spaudžia, bet Tavo įsakymais gėriuosi. 
\par 144 Tavo liudijimai yra teisingi per amžius. Leisk man juos suprasti, ir aš gyvensiu. 
\par 145 Šaukiuosi iš širdies, Viešpatie, išklausyk mane; aš laikysiuosi Tavo nuostatų. 
\par 146 Šaukiuosi Tavęs; išgelbėk mane, ir klausysiu Tavo liudijimų. 
\par 147 Prieš aušrą keliuosi ir šaukiu, nes pasitikiu Tavo žodžiu. 
\par 148 Atmerkiu akis dar prieš aušrą, mąstau apie Tavo žodį. 
\par 149 Viešpatie, išgirsk mane, būdamas maloningas; atgaivink mane, kaip esi nusprendęs. 
\par 150 Artėja priešai klastingi, nutolę nuo Tavo įstatymo. 
\par 151 Arti esi, Viešpatie, ir visi Tavo įsakymai teisingi. 
\par 152 Seniai pažinau Tavo liudijimus, kad jie yra amžini. 
\par 153 Pažvelk į mano skurdą ir išlaisvink mane; juk aš nepamirštu Tavo įstatymo. 
\par 154 Gink mano bylą ir išvaduok mane, atgaivink, kaip esi pažadėjęs. 
\par 155 Išgelbėjimas toli nuo nedorėlių, nes jie neklauso Tavo nuostatų. 
\par 156 Koks didis, Viešpatie, Tavo gailestingumas, atgaivink mane, kaip esi nusprendęs. 
\par 157 Daug mano persekiotojų ir priešų, bet aš nenukrypstu nuo Tavo liudijimų. 
\par 158 Neištikimuosius matau ir bjauriuosi, nes jie nepaiso Tavo žodžio. 
\par 159 Žiūrėk, Viešpatie, kaip Tavo potvarkius myliu; atgaivink mane, būdamas maloningas. 
\par 160 Tavo žodžiai teisingi nuo pradžių, ir visi Tavo teisūs sprendimai amžini. 
\par 161 Kunigaikščiai be priežasties persekioja mane, bet mano širdis vien Tavo žodžių tebijo. 
\par 162 Tavo žodžiu džiaugiuosi, kaip didelį lobį suradęs. 
\par 163 Melo nekenčiu ir bjauriuosi, bet Tavo įstatymas man mielas. 
\par 164 Septynis kartus per dieną giriu Tave už Tavo teisingus sprendimus. 
\par 165 Kas myli Tavo įstatymą, turi didelę ramybę ir niekada nesuklumpa. 
\par 166 Laukiu, Dieve, Tavo išgelbėjimo, Tavo įsakymus vykdau. 
\par 167 Mano siela klauso Tavo liudijimų, nes labai juos myliu. 
\par 168 Laikausi Tavo potvarkių ir liudijimų, visi mano keliai Tau žinomi. 
\par 169 Tepasiekia Tave mano šauksmas, Viešpatie; duok man supratimą, kaip esi pažadėjęs. 
\par 170 Tepasiekia mano malda Tave; išlaisvink, kaip esi pažadėjęs. 
\par 171 Mano lūpos girs Tave, nes mokai mane savo nuostatų. 
\par 172 Mano liežuvis kalbės apie Tavo žodį, nes Tavo įsakymai teisingi. 
\par 173 Tavo ranka tepadeda man, nes aš pasirinkau Tavo potvarkius. 
\par 174 Ilgiuosi, Viešpatie, Tavo išgelbėjimo, įstatymu Tavo gėriuosi. 
\par 175 Mano siela tegyvena ir tegiria Tave, Tavo sprendimai tepadeda man. 
\par 176 Klaidžioju kaip avis paklydus. Ieškok savo tarno, nes Tavo įsakymų aš neužmiršau.


\chapter{120}


\par 1 Varge šaukiausi Viešpaties, ir Jis mane išklausė. 
\par 2 Nuo meluojančių lūpų, nuo klastingo liežuvio išlaisvink, Viešpatie, mano sielą. 
\par 3 Ką tau duos ir ką padarys, apgaulingas liežuvi? 
\par 4 Aštrias kario strėles, kaitrias medžio žarijas. 
\par 5 Vargas man klajoti Mešecho krašte, palapinėse Kedaro gyventi. 
\par 6 Per ilgai gyvenau su tais, kurie nekenčia taikos. 
\par 7 Aš esu už taiką, bet jie geidžia karo, kai aš kalbu.


\chapter{121}


\par 1 Pakeliu savo akis į kalnus, iš kur man ateina pagalba. 
\par 2 Mano pagalba ateina iš Viešpaties, kuris sukūrė dangų ir žemę! 
\par 3 Jis neleis suklupti tavajai kojai, Jis­budrus tavo sargas. 
\par 4 Izraelio sargas nei miega, nei snaudžia. 
\par 5 Viešpats yra tavo sargas, Viešpats­tau šešėlis tavo dešinėje: 
\par 6 dieną nepažeis tavęs saulė nė mėnulis naktį. 
\par 7 Viešpats saugos tave nuo viso pikto, Jis saugos tavo sielą. 
\par 8 Viešpats saugos tavo įėjimą ir išėjimą dabar ir per amžius.


\chapter{122}


\par 1 Džiaugiausi, kai jie man pasakė: “Eikime į Viešpaties namus!” 
\par 2 Mūsų kojos stovės tavo, Jeruzale, vartuose! 
\par 3 Jeruzale, tvirtai pastatytas mieste! 
\par 4 Į jį traukia Viešpaties giminės pagal Izraelio įstatymą dėkoti Viešpačiui, 
\par 5 nes teisėjų sostai ir Dovydo namų sostas čia stovi. 
\par 6 Melskite taikos Jeruzalei; kurie tave myli, turės sėkmę. 
\par 7 Tebūna taika visam miestui, gerovė jo rūmuose. 
\par 8 Savo brolių ir draugų labui sakau: “Tebūna tavyje taika!” 
\par 9 Meldžiu tau gerovės dėl Viešpaties, mūsų Dievo, namų!


\chapter{123}


\par 1 Mano akys pakeltos į Tave, kuris gyveni danguose. 
\par 2 Kaip žiūri tarnai į ranką savo valdovo, tarnaitė­į ranką valdovės, taip mūsų akys žvelgia į Viešpatį, mūsų Dievą, laukdamos Jo pasigailėjimo. 
\par 3 Pasigailėk mūsų, Viešpatie, pasigailėk mūsų, nes per daug paniekinti esame! 
\par 4 Per ilgai mūsų siela varginama pašaipa turtuolių ir panieka išdidžiųjų.


\chapter{124}


\par 1 Jei Viešpats nebūtų stojęs už mus,­tesako Izraelis,­ 
\par 2 jei Viešpats nebūtų stojęs už mus, kai žmonės mus užpuolė, 
\par 3 jie mus gyvus būtų prariję, kai užsidegė jų rūstybė prieš mus. 
\par 4 Tada vandenys mus būtų užlieję, srovė būtų apsėmusi mūsų sielas, 
\par 5 gausūs vandenys būtų apsėmę mūsų sielas. 
\par 6 Palaimintas Viešpats, kuris neatidavė mūsų į jų dantis. 
\par 7 Mūsų sielos ištrūko kaip paukštis iš medžiotojo tinklo. Tinklas sutrūko, ir mes pasprukome. 
\par 8 Mūsų pagalba yra Viešpaties, dangaus ir žemės Kūrėjo, vardas.


\chapter{125}


\par 1 Kurie pasitiki Viešpačiu, yra kaip Siono kalnas, kuris stovi tvirtai per amžius. 
\par 2 Kaip kalnai apsupę Jeruzalę, taip Viešpats apsupęs savo tautą dabar ir per amžius. 
\par 3 Nepasiliks nedorėlių skeptras ant teisiųjų dalies, kad teisieji savo rankų netiestų į neteisybę. 
\par 4 Padėk, Viešpatie, geriesiems, pagelbėk tiesiaširdžiams. 
\par 5 O kreivais keliais kas nuklysta, tuos Viešpats nuves su piktadariais. Taika tebūna Izraeliui.


\chapter{126}


\par 1 Kai Siono belaisvius vedė Viešpats namo, buvome tarsi sapne. 
\par 2 Mūsų burnos buvo pilnos juoko, liežuviai­giedojimo. Pagonys kalbėjo: “Jiems Viešpats didelių dalykų padarė”. 
\par 3 Viešpats didelių dalykų mums padarė, todėl mes džiaugiamės. 
\par 4 Parvesk, Viešpatie, mūsų ištremtuosius kaip upelius pietuose. 
\par 5 Kurie ašarodami sėja, tie su džiaugsmu pjaus. 
\par 6 Verkia žmogus, į dirvą berdamas sėklą, bet su džiaugsmu grįžta, nešdamas pėdus.


\chapter{127}


\par 1 Jei Viešpats nestato namų, veltui vargsta statytojai. Jei Viešpats nesaugo miesto, veltui budi sargai. 
\par 2 Veltui keliatės prieš aušrą ir vargstate ligi vėlyvos nakties. Jūs valgote vargo duoną. O savo mylimajam Viešpats duoda miegą. 
\par 3 Štai Viešpaties dovana yra vaikai, įsčių vaisius­Jo atlyginimas. 
\par 4 Kaip strėlės karžygio rankoje, taip jaunystės sūnūs. 
\par 5 Palaimintas žmogus, turįs jų pilną strėlinę! Jie nebus sugėdinti, bet kalbės vartuose su priešu.


\chapter{128}


\par 1 Palaimintas kiekvienas, kuris bijosi Viešpaties ir vaikščioja Jo keliais! 
\par 2 Tu valgysi iš savo rankų darbo, būsi laimingas, tau gerai seksis. 
\par 3 Tavo žmona bus kaip vaisingas vynmedis tavo namuose, tavo vaikai sėdės aplink stalą kaip alyvų atžalos. 
\par 4 Štai kaip palaimintas bus žmogus, kuris bijo Viešpaties! 
\par 5 Viešpats palaimins tave nuo Siono, per visą gyvenimą Jeruzalės gerovę regėsi. 
\par 6 Tu matysi savo vaikus ir vaikaičius! Taika tebūna Izraeliui!


\chapter{129}


\par 1 Jie vargino mane nuo pat jaunystės,­tesako Izraelis,­ 
\par 2 jie vargino mane nuo pat jaunystės, tačiau nenugalėjo. 
\par 3 Artojai ant mano nugaros arė, išvarydami ilgas vagas. 
\par 4 Bet teisusis Viešpats nedorėlių pančius sutraukė. 
\par 5 Tesusigėsta ir pasitraukia visi, kurie nekenčia Siono. 
\par 6 Tebūna jie kaip stogo žolė, kuri, dar neužaugusi, nuvysta; 
\par 7 nesusirenka iš jos net sauja pjovėjui nei mažiausias pėdas rišėjui. 
\par 8 Praeiviai, eidami pro šalį, nesako jiems: “Tepalaimina tave Viešpats. Mes laiminame tave Viešpaties vardu”.


\chapter{130}


\par 1 Iš gilybės šaukiuos Tavęs, Viešpatie! 
\par 2 Viešpatie, išgirsk mano balsą! Tegul Tavo ausys išgirsta mano maldavimus. 
\par 3 Viešpatie, jei Tu neatleisi kalčių, kas, Viešpatie, išsilaikys? 
\par 4 Bet Tu atleidi nuodėmes, kad Tavęs bijotų. 
\par 5 Tu esi mano viltis, Viešpatie, mano siela Tavo žodžiu viliasi. 
\par 6 Mano siela laukia Viešpaties labiau, kaip sargybiniai laukia ryto. 
\par 7 Tegu Izraelis viliasi Viešpačiu, nes Viešpats yra gailestingas ir Jo išvadavimas didis. 
\par 8 Jis išvaduos Izraelį iš visų jo nedorybių.


\chapter{131}


\par 1 Viešpatie, mano širdis neišpuikusi ir akys nesidairo išdidžiai. Aš nesivaikau didelių dalykų, kurie man nepasiekiami. 
\par 2 Aš raminau ir tildžiau savo sielą, kaip nujunkytą kūdikį. Mano siela yra kaip nujunkytas kūdikis. 
\par 3 Pasitikėk Viešpačiu, Izraeli, dabar ir per amžius!


\chapter{132}


\par 1 Viešpatie, atsimink Dovydą ir visą jo vargą. 
\par 2 Jis priesaiką Viešpačiui davė ir įžadą Jokūbo Galingajam: 
\par 3 “Aš neįžengsiu savo pastogėn, lovon negulsiu ilsėtis, 
\par 4 akims miegoti neleisiu, vokams užsimerkti, 
\par 5 kol nesurasiu Viešpačiui vietos, buveinės Jokūbo Galingajam!” 
\par 6 Štai mes išgirdome ją esant Efratoje, suradome Jaaro laukuose. 
\par 7 Įeikime į Jo palapines, parpulkime prie Jo kojų pakojo! 
\par 8 Viešpatie, pakilk į savo poilsio vietą, Tu ir Tavo stiprybės skrynia. 
\par 9 Kunigai tegul apsivelka teisumu, o šventieji tešaukia iš džiaugsmo! 
\par 10 Dėl savo tarno Dovydo, nenugręžk savo veido nuo savo pateptojo. 
\par 11 Dovydui Viešpats tiesoje yra prisiekęs, neatšaukiamą priesaiką davęs: “Tavo palikuonį pasodinsiu į tavąjį sostą! 
\par 12 Jei sūnūs tavieji mano sandoros ir mano pamokymų laikysis, jų palikuonys taip pat per amžius sėdės tavo soste”. 
\par 13 Viešpats pasirinko Siono kalną, čia Jis panoro gyventi. 
\par 14 “Šita yra mano poilsio vieta per amžius. Čia gyvensiu, nes Aš taip panorėjau! 
\par 15 Aš jo maistą palaiminsiu, pasotinsiu vargšus duona. 
\par 16 Jo kunigus išgelbėjimo rūbu apvilksiu, šventieji šauks iš džiaugsmo. 
\par 17 Čia išauginsiu Dovydui ragą, žibintą savo pateptajam paruošiu. 
\par 18 Jo visus priešus sugėdinsiu, o ant jo galvos karūna spindės”.


\chapter{133}


\par 1 Kaip gera ir malonu, kai broliai vienybėje gyvena! 
\par 2 Tai yra lyg brangus aliejus ant galvos, varvantis per Aarono barzdą ant jo drabužių apykaklės. 
\par 3 Tai lyg Hermono rasa, krintanti ant Siono kalnų! Tenai palaiminimą ir gyvenimą amžiną Viešpats teikia!


\chapter{134}


\par 1 Laiminkite Viešpatį, visi Viešpaties tarnai, kurie Jo namuose naktimis budite. 
\par 2 Tieskite savo rankas į Jo šventyklą ir laiminkite Viešpatį. 
\par 3 Tegul iš Siono tave laimina Dievas, kuris sutvėrė dangų ir žemę.


\chapter{135}


\par 1 Girkite Viešpatį! Girkite Viešpaties vardą! Girkite jūs, Viešpaties tarnai, 
\par 2 kurie stovite Jo namuose, mūsų Dievo namų kiemuose. 
\par 3 Girkite Jį, nes Viešpats geras; giedokite gyrių Jo vardui, nes Jis mielas. 
\par 4 Jokūbą išsirinko Viešpats, Izraelį, kad būtų Jo nuosavybė. 
\par 5 Aš žinau, kad didis yra Viešpats! Mūsų Viešpats yra aukščiau visų dievų. 
\par 6 Visa, ko panorėjo Viešpats, Jis padarė danguje, žemėje ir jūrų gelmėse. 
\par 7 Jis nuo žemės krašto atgena debesis, su žaibais siunčia lietų, paleidžia vėjus iš savo sandėlių. 
\par 8 Jis ištiko Egipte žmonių ir gyvulių pirmagimius, 
\par 9 siuntė ženklų ir stebuklų prieš faraoną ir jo tarnus. 
\par 10 Sunaikino daugybę pagonių tautų ir jų galingų karalių: 
\par 11 Sihoną, amoritų karalių, ir Ogą, Bašano karalių, ir visas karalystes Kanaane. 
\par 12 Atidavė jų žemę Izraeliui, savo tautai, paveldėti. 
\par 13 Viešpatie, Tavo vardas lieka per amžius. Atsiminimas apie Tave, Viešpatie, per kartų kartas. 
\par 14 Viešpats teis savo tautą ir pasigailės savo tarnų. 
\par 15 Pagonių stabai iš sidabro ir aukso, žmogaus rankomis padaryti. 
\par 16 Jie turi burnas, bet nekalba; turi akis, bet nemato; 
\par 17 turi ausis, tačiau negirdi; jų burna nealsuoja. 
\par 18 Į juos panašūs yra tie, kas juos padaro, ir visi, kurie jais pasitiki. 
\par 19 Izraelio namai, laiminkite Viešpatį! Aarono namai, laiminkite Viešpatį! 
\par 20 Levio namai, laiminkite Viešpatį! Visi, kurie Viešpaties bijote, laiminkite Viešpatį! 
\par 21 Palaimintas Viešpats iš Siono, kuris gyvena Jeruzalėje. Girkite Viešpatį!


\chapter{136}


\par 1 Dėkokite Viešpačiui, nes Jis geras, nes Jo gailestingumas amžinas. 
\par 2 Dėkokite dievų Dievui, nes Jo gailestingumas amžinas. 
\par 3 Dėkokite viešpačių Viešpačiui, nes Jo gailestingumas amžinas. 
\par 4 Kuris vienas daro didelius stebuklus, nes Jo gailestingumas amžinas. 
\par 5 Kuris išmintingai dangų sukūrė, nes Jo gailestingumas amžinas. 
\par 6 Kuris viršum vandenų ištiesė žemę, nes Jo gailestingumas amžinas. 
\par 7 Kuris dideles šviesas sukūrė, nes Jo gailestingumas amžinas. 
\par 8 Saulę sukūrė, kad viešpatautų dienai, nes Jo gailestingumas amžinas. 
\par 9 Mėnulį ir žvaigždes, kad viešpatautų nakčiai, nes Jo gailestingumas amžinas. 
\par 10 Kuris pirmagimius Egipte ištiko, nes Jo gailestingumas amžinas. 
\par 11 Ir išvedė iš ten Izraelį, nes Jo gailestingumas amžinas. 
\par 12 Ištiesta galinga ranka ir tvirta dešine savo, nes Jo gailestingumas amžinas. 
\par 13 Kuris perskyrė Raudonąją jūrą, nes Jo gailestingumas amžinas. 
\par 14 Ir pervedė per ją Izraelį, nes Jo gailestingumas amžinas. 
\par 15 Faraoną ir jo kariuomenę Raudonojoje jūroje pražudė, nes Jo gailestingumas amžinas. 
\par 16 Kuris savo tautą per dykumą vedė, nes Jo gailestingumas amžinas. 
\par 17 Kuris stiprius karalius nugalėjo, nes Jo gailestingumas amžinas. 
\par 18 Kuris garsius karalius nužudė, nes Jo gailestingumas amžinas. 
\par 19 Sihoną, amoritų karalių, nes Jo gailestingumas amžinas. 
\par 20 Ir Ogą, Bašano karalių, nes Jo gailestingumas amžinas. 
\par 21 Ir jų žemę Izraeliui paveldėti davė, nes Jo gailestingumas amžinas. 
\par 22 Savajam tarnui Izraeliui valdyti leido, nes Jo gailestingumas amžinas. 
\par 23 Kuris atsiminė mus mūsų pažeminime, nes Jo gailestingumas amžinas. 
\par 24 Iš priešų mus išvadavo, nes Jo gailestingumas amžinas. 
\par 25 Kuris maitina visa, kas gyva, nes Jo gailestingumas amžinas. 
\par 26 Dėkokite dangaus Dievui, nes Jo gailestingumas amžinas.


\chapter{137}


\par 1 Prie Babilono upių sėdėjome ir verkėme, atsimindami Sioną. 
\par 2 Ten ant gluosnių šakų pakabinome savo arfas. 
\par 3 Nes mūsų trėmėjai mums liepė giedoti, kurie mus apiplėšė, ragino džiūgauti: “Pagiedokite mums Siono giesmių!” 
\par 4 Kaip giedosime Viešpaties giesmę svetimoje šalyje? 
\par 5 Jeigu, Jeruzale, tave užmirščiau, mano dešinė tepamiršta mane! 
\par 6 Tepridžiūna prie gomurio mano liežuvis, jei tavęs neatsiminčiau, jeigu tu man brangesnė nebūtum už visus džiaugsmus, Jeruzale! 
\par 7 Viešpatie, atsimink Jeruzalės dieną prieš Edomo žmones. Jie sakė: “Griaukite, griaukite ją iki pamatų!” 
\par 8 O Babilone, tu naikintojau, laimingas bus, kas tau už mums padarytą skriaudą atmokės! 
\par 9 Laimingas, kas, pagriebęs kūdikius tavo, į kietą uolą sudaužys!


\chapter{138}


\par 1 Visa širdimi Tau dėkoju, Dieve! Dievų akivaizdoje Tau giedosiu gyrių. 
\par 2 Parpuolęs prie Tavo šventyklos, šventą Tavo vardą girsiu už Tavo malonę ir ištikimybę, nes Tu išaukštinai savo žodį labiau negu visą savo vardą. 
\par 3 Dieną, kurią šaukiausi, išklausei, stiprybės suteikei mano sielai. 
\par 4 Tave girs, Viešpatie, visi žemės valdovai, išgirdę Tavo burnos žodžius. 
\par 5 Jie giedos apie Viešpaties kelius, nes didelė Viešpaties šlovė. 
\par 6 Nors Viešpats yra aukštybėse, Jis žvelgia į nusižeminusį, o išpuikėlį pažįsta iš tolo. 
\par 7 Nors aš būčiau vargų suspaustas, Tu atgaivinsi mane, Tu ištiesi savo ranką prieš mano priešų pyktį ir Tavo dešinė mane išgelbės. 
\par 8 Apgins mane Viešpats. Viešpatie, Tavo gailestingumas amžinas! Neapleisk, kas Tavo rankų sukurta.


\chapter{139}


\par 1 Viešpatie, Tu ištyrei mane ir pažįsti. 
\par 2 Tu žinai, kada keliuosi ir kada atsisėdu, Tu supranti mano mintis. 
\par 3 Matai, kada vaikštau ir kada ilsiuosi; žinai visus mano kelius. 
\par 4 Dar mano liežuvis nepratarė žodžio, Tu, Viešpatie, jau viską žinai. 
\par 5 Apglėbęs laikai mane iš priekio ir iš užpakalio, uždėjai ant manęs savo ranką. 
\par 6 Toks pažinimas man yra labai nuostabus, ne man pasiekti Tavo aukštybes. 
\par 7 Kur nuo Tavo dvasios aš pasislėpsiu, kur nuo Tavo veido pabėgsiu? 
\par 8 Jei užkopčiau į dangų, Tu ten. Jei nusileisčiau į pragarą, Tu ten. 
\par 9 Jei aušros sparnus pasiėmęs nusileisčiau, kur baigiasi jūros, 
\par 10 ir ten Tavo ranka vestų mane ir laikytų Tavo dešinė. 
\par 11 Jei sakyčiau: “Tamsa teapdengia mane”, naktis aplinkui mane šviesa pavirstų. 
\par 12 Tamsa nepaslepia nuo Tavęs, Tau net naktį šviesu kaip dieną. 
\par 13 Tu mano širdį sukūrei, sutvėrei mane motinos įsčiose. 
\par 14 Girsiu Tave, kad taip nuostabiai ir baimę keliančiai esu sukurtas. Kokie nuostabūs yra Tavo darbai, ir mano siela tai gerai žino. 
\par 15 Nė vienas kaulas nebuvo paslėptas nuo Tavęs, kai slaptoje buvau padarytas, kai buvau tveriamas žemės gelmėse. 
\par 16 Tavo akys matė mane dar negimusį ir Tavo knygoje buvo viskas surašyta: dienos, kurias man skyrei, kai dar nė vienos jų nebuvo. 
\par 17 Brangios man yra Tavo mintys, Dieve, nesuskaitoma jų gausybė! 
\par 18 Jei mėginčiau skaičiuoti, jų būtų daugiau kaip smėlio. Atsibudęs aš tebesu su Tavimi. 
\par 19 Dieve, Tu tikrai sunaikinsi nedorėlį, kraugeriai, atsitraukite nuo manęs! 
\par 20 Jie nedorai apie Tave kalba, Tavo vardą piktam jie naudoja. 
\par 21 Aš nekenčiu tų, Viešpatie, kurie Tavęs nekenčia, man bjaurūs tie, kurie prieš Tave sukyla. 
\par 22 Aš jų nekenčiu be galo, laikau juos savo priešais. 
\par 23 Ištirk mane, Dieve, pažink mano širdį; išbandyk mane ir pažink mano mintis. 
\par 24 Matyk, ar aš einu nedorėlių keliu, ir vesk mane keliu amžinuoju!


\chapter{140}


\par 1 Viešpatie, nuo piktų žmonių mane išgelbėk, nuo smurtininkų apsaugok! 
\par 2 Nuo tų, kurie planuoja pikta širdyje, kurie dieną vaidus kelia. 
\par 3 Jų liežuvis yra aštrus kaip gyvatės, angių nuodai yra už jų lūpų. 
\par 4 Viešpatie, saugok mane nuo nedorėlių rankų, nuo smurtininkų apsaugok­nuo tų, kurie nori pakišti man koją. 
\par 5 Išdidieji pinkles man spendžia, tiesia tinklus, kilpas stato pakelėse. 
\par 6 Tariau Viešpačiui: “Tu esi mano Dievas, todėl išklausyk mano maldavimą”. 
\par 7 Viešpatie Dieve, mano išgelbėjimo stiprybe, Tu pridengi kovoje mano galvą. 
\par 8 Dieve, neleisk nedorėliui laimėti, jo piktų kėslų netenkink! 
\par 9 Jie kelia galvas, aplinkui sustoję. Tegu jų piktybė juos pačius apdengia! 
\par 10 Tegul jiems ant galvų žarijomis lyja! Įmesk juos į ugnį, į duobę, kad nebepasikeltų. 
\par 11 Teneįsitvirtins žemėje šmeižikas, nelaimės tegainioja smurtininką ir teparbloškia jį. 
\par 12 Žinau, kad Viešpats gins prispaustojo bylą ir teises beturčio. 
\par 13 Tikrai, teisieji dėkos Tavo vardui ir dorieji gyvens Tavo akivaizdoje.


\chapter{141}


\par 1 Viešpatie, Tavęs šaukiuos, skubėk pas mane! Išklausyk mano balsą, kai Tavęs šaukiuos. 
\par 2 Malda manoji kaip smilkalai į Tave tekyla, o pakeltos rankos tebūna kaip vakaro auka. 
\par 3 Viešpatie, prie mano burnos pastatyk sargybą, saugok mano lūpų duris. 
\par 4 Neleisk mano širdžiai nukrypti į pikta ir nedorėlių darbus pamilti. Tenevalgysiu aš jų skanėstų. 
\par 5 Tegul mane plaka teisusis; tai bus malonė. Tegul bara jis, nes tai kaip brangus aliejus mano galvai. Bet aš meldžiuosi prieš nedorėlių darbus. 
\par 6 Jų teisėjai priblokšti tarp akmenų, jie girdi mano žodžius, kad jie švelnūs. 
\par 7 Kaip artojo išrausta ir išarta žemė, taip prie pragaro nasrų bus išbarstyti jų kaulai. 
\par 8 Viešpatie Dieve, į Tave žvelgia mano akys, Tavimi aš pasitikiu, neleisk mano sielai pražūti. 
\par 9 Saugok mane nuo man pastatytų žabangų, nuo nedorėlių kilpos. 
\par 10 Į savo kilpas nedorėliai patys tepakliūva, o aš laimingai praeisiu.


\chapter{142}


\par 1 Garsiai šaukiausi Viešpaties, garsiu balsu maldavau Viešpatį. 
\par 2 Išliejau priešais Jį savo skundą, sielvartą savo Jam atvėriau. 
\par 3 Kai manyje nusilpo mano dvasia, Tu žinojai mano kelią. Mano kelyje slaptai jie padėjo spąstus. 
\par 4 Apsidairęs aplinkui, mačiau, jog nėra nė vieno, kas mane pažintų. Neturėjau kur prisiglausti, niekam nerūpėjo mano gyvybė. 
\par 5 Tavęs, Viešpatie, šaukiausi, sakydamas: “Tu esi mano priebėga, Tu mano dalis gyvųjų šalyje”. 
\par 6 Išgirsk mano šauksmą, nes esu labai suvargęs. Išgelbėk mane nuo persekiotojų, nes jie stipresni už mane. 
\par 7 Iš kalėjimo mane išvaduok, kad girčiau Tavo vardą. Tada susiburs aplink mane teisieji, kai padarysi man gera.


\chapter{143}


\par 1 Viešpatie, išgirsk mano maldą, išklausyk mano maldavimą. Būdamas teisus ir ištikimas, atsakyk man. 
\par 2 Nepatrauk į teismą savo tarno, nes Tavo akivaizdoje nė vienas žmogus negalės pasiteisinti. 
\par 3 Persekioja mane priešas, sutrypė į žemę mano gyvybę, kaip negyvėlį tamsoje gyventi verčia. 
\par 4 Mano dvasia nusilpo, sustingo širdis man krūtinėje. 
\par 5 Aš prisimenu praėjusias dienas, mąstau apie Tavo darbus, svarstau apie Tavo rankų darbus. 
\par 6 Tiesiu rankas į Tave, mano siela trokšta Tavęs tartum išdžiūvusi žemė vandens. 
\par 7 Skubiai išklausyk mane, Viešpatie, nes silpsta mano dvasia. Neslėpk nuo manęs savo veido, kad nebūčiau kaip tie, kurie žengia į duobę. 
\par 8 Tavo malonę leisk man patirti nuo pat ryto, nes Tavimi aš pasitikiu. Parodyk man kelią, kuriuo eiti, nes į Tave keliu savo sielą. 
\par 9 Išvaduok mane, Viešpatie, iš mano priešų, nes pas Tave bėgu slėptis. 
\par 10 Mokyk mane vykdyti Tavo valią, nes Tu esi mano Dievas. Tavo geroji dvasia tegul veda mane tiesiu keliu. 
\par 11 Dėl savo vardo, Viešpatie, atgaivink mane. Dėl savo teisumo iš visų bėdų išvesk mano sielą. 
\par 12 Dėl savo gailestingumo nutildyk mano priešus, sunaikink mano sielos prispaudėjus, nes aš esu Tavo tarnas.


\chapter{144}


\par 1 Palaimintas Viešpats, mano stiprybė, kuris moko mano rankas kariauti ir mano pirštus kovoti. 
\par 2 Jis yra mano gerumas, mano tvirtovė, mano aukštas bokštas, mano išlaisvintojas, mano skydas; Juo aš pasitikiu, Jis pavergia man tautas. 
\par 3 Viešpatie, kas yra žmogus, kad apie jį žinai, ir žmogaus sūnus, kad kreipi į jį savo dėmesį. 
\par 4 Žmogus panašus į vėjo dvelksmą; jo dienos kaip praslenkantis šešėlis. 
\par 5 Viešpatie, palenk dangų ir nuženk. Paliesk kalnus, ir jie ims rūkti. 
\par 6 Pasiųsk žaibus ir juos išsklaidyk, laidyk strėles ir sunaikink juos. 
\par 7 Ištiesk ranką iš aukštybės, ištrauk ir gelbėk mane iš didelio tvano, iš svetimšalių rankos. 
\par 8 Jų burna kalba tuštybę, jie melagingai priesaikai dešinę kelia. 
\par 9 Dieve, aš naują giesmę Tau giedosiu, psalteriu ir dešimčiastygiu jai pritarsiu. 
\par 10 Tu gelbsti karalius, Tu išlaisvinai Dovydą, savo tarną, nuo žiauraus kardo. 
\par 11 Ištrauk ir gelbėk mane iš svetimšalių rankos. Jų burna kalba tuštybę, jie dešinę melagingai priesaikai kelia. 
\par 12 Tegu mūsų sūnūs savo jaunystėje auga kaip vešlūs žolynai, o mūsų dukterys tebūna dailios tartum kolonos, išdrožinėtos puošniuose rūmuose. 
\par 13 Mūsų klėtys tebūna prikrautos visokių gėrybių. Kaimenės mūsų avių tegul tūkstančiais veda, pilnos ganyklos tegul prisipildo. 
\par 14 Tegu mūsų jaučiai būna stiprūs darbe, tenebūna jie pagrobti ar nuklydę, tegu mūsų gatvėse nesigirdi skundų. 
\par 15 Laiminga tauta, kuri taip gyvena; laiminga tauta, kurios Dievas yra Viešpats!


\chapter{145}


\par 1 Aš aukštinsiu Tave, Dieve, mano Karaliau, aš laiminsiu Tavo vardą per amžius. 
\par 2 Kasdien laiminsiu Tave ir girsiu Tavo vardą per amžius. 
\par 3 Didis yra Viešpats ir didžiai girtinas, Jo didybė neištiriama. 
\par 4 Tavo nuostabius darbus ir galybę karta teskelbia kartai. 
\par 5 Aš kalbėsiu apie Tavo didingumo šlovingą spindesį ir Tavo stebuklingus darbus. 
\par 6 Apie Tavo veiksmų nuostabią galią jie kalbės, ir aš skelbsiu Tavo didybę. 
\par 7 Jie atsimins didelę Tavo malonę, giedos apie Tavo teisumą. 
\par 8 Viešpats yra maloningas, užjaučiantis, lėtas pykti ir didžiai gailestingas. 
\par 9 Viešpats yra geras ir gailestingas visiems savo kūriniams. 
\par 10 Viešpatie, Tavo visi kūriniai girs Tave ir Tavo šventieji laimins Tave! 
\par 11 Jie kalbės apie Tavo karalystės šlovę, garsins Tavo galybę, 
\par 12 kad žmonės suprastų Tavo galingus darbus ir Tavo karalystės šlovingą didingumą. 
\par 13 Tavo karalystė amžina ir Tavo valdžia lieka per amžius. 
\par 14 Klumpantį Viešpats palaiko, o parpuolusį pakelia. 
\par 15 Į Tave krypsta visų akys, savo metu visiems duodi maisto, 
\par 16 atveri savo ranką, sočiai maitini visa, kas gyva. 
\par 17 Teisus yra Viešpats visuose savo keliuose ir šventas visuose savo darbuose. 
\par 18 Arti yra Viešpats visiems, kurie Jo šaukiasi, visiems, kurie Jo šaukiasi tiesoje. 
\par 19 Jis išpildys norus Jo bijančių, jų šauksmą išgirs ir išgelbės juos. 
\par 20 Viešpats visus Jį mylinčius saugo, bet nedorėlius sunaikins. 
\par 21 Tešlovina mano burna Viešpatį! Visa, kas gyva, telaimina šventą Jo vardą dabar ir per amžius!


\chapter{146}


\par 1 Girkite Viešpatį! Girk Viešpatį, mano siela! 
\par 2 Girsiu Viešpatį, kol gyvas būsiu; giedosiu gyrių savo Dievui, kol gyvensiu. 
\par 3 Nesudėkite vilčių į kunigaikščius, į žmonių sūnus, kurie nieko išgelbėti negali! 
\par 4 Kai dvasia jų iškeliauja, jie sugrįžta į žemę; tą pačią dieną jų sumanymai žūva. 
\par 5 Laimingas, kurio pagalba yra Jokūbo Dievas, kuris viltis sudėjo į Viešpatį, savo Dievą. 
\par 6 Jis sukūrė dangų, žemę, jūrą ir visa, kas juose yra. Jis ištikimas per amžius. 
\par 7 Jis gina teises prispaustųjų, alkaniems parūpina duonos. Kalinius išleidžia Viešpats. 
\par 8 Viešpats atveria akis akliesiems, pakelia parpuolusius; Viešpats myli teisiuosius. 
\par 9 Svetimšalį Viešpats saugo, našlaitį ir našlę globoja, o nedoriesiems užkerta kelią. 
\par 10 Viešpats karaliaus per amžius, tavo Dievas, Sione, per visas kartas! Girkite Viešpatį!


\chapter{147}


\par 1 Girkite Viešpatį! Gera giedoti gyrių mūsų Dievui, labai malonu girti Jį! 
\par 2 Viešpats atstato Jeruzalės miestą, Izraelio tremtinius surenka. 
\par 3 Jis gydo sudužusias širdis, žaizdas jų sutvarsto. 
\par 4 Jis žvaigždes suskaičiuoja, jas visas vardais vadina. 
\par 5 Didis ir galingas yra Viešpats, Jo išmintis begalinė. 
\par 6 Romiuosius pakelia Viešpats, o nedorėlius partrenkia ant žemės. 
\par 7 Padėkos giesmes Viešpačiui giedokite, skambinkite mūsų Viešpačiui arfa! 
\par 8 Jis debesimis uždengia dangų, žemei paruošia lietų, augina žolę kalnuose. 
\par 9 Jis teikia žvėrims maistą, jaunus varniukus pamaitina. 
\par 10 Ne žirgo stiprumu Jis gėrisi ir ne žmogaus kojos Jam patinka. 
\par 11 Viešpats gėrisi tais, kurie Jo bijo, kurie viliasi Jo gailestingumu. 
\par 12 Girk, Jeruzale, Viešpatį! Savo Dievą girk, Sione! 
\par 13 Jis sustiprino tavo vartų sklendes, tavo vaikus palaimino. 
\par 14 Tavo krašte Jis duoda taiką, geriausiais kviečiais tave pasotina. 
\par 15 Jis siunčia įsakymą žemei, labai sparčiai Jo žodis atskrieja. 
\par 16 Jis sniegą kaip vilną duoda, šerkšną lyg pelenus barsto. 
\par 17 Ledus kaip duonos trupinius beria, sustingsta vanduo nuo Jo šalčio. 
\par 18 Jam žodį pasiuntus, ledas sutirpsta; kai vėjus paleidžia, vandenys teka. 
\par 19 Jokūbui savo žodį Jis paskelbė, Izraeliui savo nuostatus ir sprendimus. 
\par 20 Kitoms tautoms Jis to nepadarė, nė viena nepažino Jo sprendimų. Girkite Viešpatį!


\chapter{148}


\par 1 Girkite Viešpatį! Girkite Viešpatį iš dangaus! Girkite Jį aukštybėse! 
\par 2 Girkite Jį, visi Jo angelai; girkite Jį, visa Jo kareivija! 
\par 3 Girkite Jį, saule ir mėnuli; girkite Jį, spindinčios žvaigždės! 
\par 4 Girkite Jį, dangų dangūs ir viršum jų esantys vandenys! 
\par 5 Visi tegiria Viešpaties vardą, nes Jis įsakė, ir visa buvo sukurta. 
\par 6 Jis amžių amžiams viską įtvirtino, nustatė nekintamą tvarką. 
\par 7 Girkite Viešpatį, kas esate žemėje: jūrų pabaisos ir visos gelmės, 
\par 8 ugnie ir kruša, sniege ir migla, vėjai audringi, vykdantys Jo žodį, 
\par 9 kalnai ir kalvos, vaismedžiai ir kedrai, 
\par 10 žvėrys ir visi gyvuliai, žemės ropliai ir sparnuoti paukščiai. 
\par 11 Žemės karaliai ir visos tautos, kunigaikščiai ir žemės teisėjai, 
\par 12 jaunuoliai ir mergaitės, seniai ir vaikai, 
\par 13 garbinkite Viešpaties vardą, nes Jis vienas didingas. Jo šlovė virš dangaus ir žemės. 
\par 14 Savo tautos Jis iškelia ragą, gyrių visų savo šventųjų, Izraelio vaikų­tautos, kuri Jam artima. Girkite Viešpatį!


\chapter{149}


\par 1 Girkite Viešpatį! Viešpačiui giedokite naują giesmę ir gyrių šventųjų bendruomenėje. 
\par 2 Tesidžiaugia Izraelis savo Kūrėju. Siono vaikai tesilinksmina dėl savo Karaliaus. 
\par 3 Tegiria Jo vardą šokdami, tegieda gyrių Jam su būgnais ir arfomis. 
\par 4 Viešpats gėrisi savąja tauta; Jis romiuosius papuoš savo išgelbėjimu. 
\par 5 Džiaukitės Jo šlove, šventieji, giedokite Jam savo lovose. 
\par 6 Dievo aukštinimas jų burnoje, dviašmenis kalavijas jų rankose, 
\par 7 kad atkeršytų pagonims, nubaustų tautas, 
\par 8 jų karalius pančiais surištų ir jų didžiūnus sukaustytų geležimi, 
\par 9 kad įvykdytų jiems paruoštą sprendimą! Tokia garbė visiems Jo šventiesiems. Girkite Viešpatį!


\chapter{150}


\par 1 Girkite Viešpatį! Girkite Dievą Jo šventykloje! Girkite Jį dangaus tvirtumoje! 
\par 2 Girkite Jį dėl Jo galingų darbų! Girkite Jį dėl Jo begalinės didybės! 
\par 3 Girkite Jį, pūsdami trimitą! Girkite Jį arfa ir psalteriu! 
\par 4 Girkite Jį būgnais ir šokiais! Girkite Jį styginiais instrumentais ir vamzdžiais! 
\par 5 Girkite Jį skambiais cimbolais! Girkite Jį trenksmingais cimbolais! 
\par 6 Visa, kas kvėpuoja, tegiria Viešpatį! Girkite Viešpatį!




\end{document}