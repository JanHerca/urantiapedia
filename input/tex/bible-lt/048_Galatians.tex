\begin{document}

\title{Laiškas galatams}

\chapter{1}


\par 1 Paulius, apaštalas ne iš žmonių ir ne per žmogų, bet per Jėzų Kristų ir Jį prikėlusį iš numirusių Dievą Tėvą, 
\par 2 ir kartu su manimi esantys broliai­Galatijos bažnyčioms. 
\par 3 Malonė ir ramybė jums nuo Dievo Tėvo ir mūsų Viešpaties Jėzaus Kristaus, 
\par 4 kuris atidavė save už mūsų nuodėmes, kad išgelbėtų mus nuo dabartinio blogo amžiaus pagal mūsų Dievo ir Tėvo valią, 
\par 5 kuriam šlovė per amžių amžius! Amen. 
\par 6 Stebiuosi, kad jūs nuo To, kuris pašaukė jus į Kristaus malonę, taip greitai persimetate prie kitokios evangelijos, 
\par 7 kuri, tarp kitko, nėra kitokia, o yra tik jus klaidinantys žmonės, norintys iškreipti Kristaus Evangeliją. 
\par 8 Bet nors ir mes patys ar angelas iš dangaus jums skelbtų kitokią evangeliją, negu mes jums paskelbėme,­tebūnie prakeiktas! 
\par 9 Kaip anksčiau sakėme, taip ir dabar sakau dar kartą: jei kas jums skelbia kitokią evangeliją, negu esate priėmę,­tebūnie prakeiktas! 
\par 10 Ar aš ieškau žmonių palankumo, ar Dievo? Gal stengiuosi patikti žmonėms? Jei dar norėčiau patikti žmonėms, nebūčiau Kristaus tarnas. 
\par 11 Aš sakau jums, broliai, kad mano paskelbtoji Evangelija nėra iš žmonių, 
\par 12 nes negavau jos iš žmogaus ir nebuvau jos išmokytas, bet gavau Jėzaus Kristaus apreiškimu. 
\par 13 Jūs, be abejo, girdėjote, kaip kadaise aš elgiausi, išpažindamas judaizmą, kaip be saiko persekiojau Dievo bažnyčią ir grioviau ją. 
\par 14 Žydų religija buvau pralenkęs daugelį bendraamžių savo tautiečių, būdamas itin uolus dėl savo tėvų tradicijų. 
\par 15 Bet kai Dievas, kuris mane išskyrė dar esantį motinos įsčiose ir pašaukė savo malone, panorėjo 
\par 16 apreikšti manyje savo Sūnų, kad skelbčiau Jį pagonims, neskubėjau tartis su kūnu ir krauju 
\par 17 ir nenuvykau į Jeruzalę pas pirmiau už mane buvusius apaštalus, bet iškeliavau į Arabiją ir po to vėl grįžau į Damaską. 
\par 18 Po trejų metų nukeliavau į Jeruzalę pasimatyti su Kefu ir pasilikau pas jį penkiolika dienų. 
\par 19 Kitų apaštalų man neteko matyti, tiktai Viešpaties brolį Jokūbą. 
\par 20 Ką jums rašau, tvirtinu Dievo akivaizdoje, jog nemeluoju. 
\par 21 Po to išvykau į Sirijos ir Kilikijos sritis. 
\par 22 Iš veido aš buvau nepažįstamas Judėjos bažnyčioms, kurios Kristuje. 
\par 23 Jos buvo tik girdėję: tas, kuris mus kitados persekiojo, dabar skelbia tikėjimą, kurį kadaise griovė. 
\par 24 Ir jos šlovino Dievą dėl manęs.


\chapter{2}


\par 1 Paskui, po keturiolikos metų, vėl nuvykau į Jeruzalę kartu su Barnabu, pasiėmęs ir Titą. 
\par 2 Nuvykau, apreiškimo paskatintas, ir jiems išdėsčiau Evangeliją, kurią skelbiu pagonims, atskirai išsiaiškindamas su įžymesniais asmenimis, kad kartais nebėgčiau ar nebūčiau bėgęs veltui. 
\par 3 Jie nevertė apsipjaustyti nė mano palydovo Tito, kuris buvo graikas. 
\par 4 Tačiau netikriems broliams, paslapčia įslinkusiems iššnipinėti mūsų laisvę, kurią turime Kristuje Jėzuje, ir norėjusiems mus pavergti,­ 
\par 5 jiems nė valandėlei nepasidavėme, kad Evangelijos tiesa pasiliktų su jumis. 
\par 6 O dėl tariamai įžymesniųjų asmenų,­kas jie bebuvo, man nesvarbu, nes Dievas nėra žmonėms šališkas,­man įžymesnieji asmenys nieko nepridėjo. 
\par 7 Atvirkščiai, pamatę, jog man patikėta skelbti Evangeliją neapipjaustytiesiems kaip Petrui apipjaustytiesiems 
\par 8 (nes Tas, kuris veikė su Petru jam apaštalaujant apipjaustytiesiems, veikė taip pat su manimi tarp pagonių) 
\par 9 ir pastebėję man suteiktą malonę, Jokūbas, Kefas ir Jonas, kurie laikomi šulais, padavė man ir Barnabui dešines draugystės ženklan, kad eitume pas pagonis, o jie pas apipjaustytuosius; 
\par 10 tik mes turėjome prisiminti vargšus,­o aš ir stengiausi tai daryti. 
\par 11 Kai Petras atvyko į Antiochiją, aš jam pasipriešinau į akis, nes jis nusižengė. 
\par 12 Mat prieš atvykstant kai kuriems nuo Jokūbo, jis valgydavo su pagonimis; bet kai tie atvyko, jis atsitraukė ir vengė jų, bijodamas apipjaustytųjų. 
\par 13 Kartu su juo veidmainiavo ir kiti žydai, netgi Barnabas pasidavė veidmainystei. 
\par 14 Pamatęs, kad jie nukrypsta nuo Evangelijos tiesos, pasakiau Petrui visų akivaizdoje: “Jei tu, būdamas žydas, gyveni pagoniškai, o ne žydiškai, tai kodėl verti pagonis gyventi taip, kaip žydai?” 
\par 15 Nors iš prigimties esame žydai ir ne pagonių kilmės nusidėjėliai, 
\par 16 žinome, jog žmogus neišteisinamas įstatymo darbais, bet tikėjimu į Jėzų Kristų. Mes įtikėjome Kristų Jėzų, kad būtume išteisinti Kristaus tikėjimu, o ne įstatymo darbais; nes įstatymo darbais nebus išteisintas nė vienas žmogus. 
\par 17 Bet, ieškant mums išteisinimo per Kristų, paaiškėja, kad mes patys esame nusidėjėliai. Ar Kristus dėl to yra nuodėmės tarnas? Jokiu būdu! 
\par 18 Nes jeigu aš vėl atstatau, ką buvau išgriovęs, tai tampu nusikaltėliu. 
\par 19 Aš per įstatymą numiriau įstatymui, kad gyvenčiau Dievui. 
\par 20 Esu nukryžiuotas su Kristumi. Ir daugiau ne aš gyvenu, o gyvena manyje Kristus. Ir dabar, gyvendamas kūne, gyvenu tikėjimu į Dievo Sūnų, kuris pamilo mane ir paaukojo save už mane. 
\par 21 Neatstumiu Dievo malonės, nes jei teisumas įgyjamas įstatymu, tuomet Kristus mirė veltui.


\chapter{3}


\par 1 O neprotingi galatai! Kas jus, kuriems akivaizdžiai buvo nupieštas Jėzus Kristus tarsi pas jus nukryžiuotas, apkerėjo, kad nepaklustumėte tiesai? 
\par 2 Noriu jus paklausti tiktai vieno dalyko: ar jūs gavote Dvasią įstatymo darbais, ar klausydami tikėjimo? 
\par 3 Nejaugi jūs tokie neprotingi, kad, pradėję Dvasia, dabar užbaigsite kūnu? 
\par 4 Ar tiek daug iškentėjote veltui? Jei iš tiesų būtų veltui! 
\par 5 Ar Tas, kuris jums teikia Dvasią ir pas jus daro stebuklus, tai daro per įstatymo darbus, ar dėl tikėjimo klausymo? 
\par 6 Taip “Abraomas patikėjo Dievu, ir tai jam buvo įskaityta teisumu”. 
\par 7 Todėl supraskite, kad Abraomo sūnūs yra tie, kurie tiki. 
\par 8 Ir Raštas, numatydamas, kad Dievas tikėjimu išteisins pagonis, iš anksto paskelbė Abraomui Evangeliją: “Tavyje bus palaimintos visos tautos”. 
\par 9 Taip tikintys susilaukia palaiminimo kartu su tikinčiuoju Abraomu. 
\par 10 Visi, kurie remiasi įstatymo darbais, yra prakeikimo galioje, nes parašyta: “Prakeiktas kiekvienas, kuris nuolatos nesilaiko visko, kas įstatymo knygoje parašyta, ir to nevykdo”. 
\par 11 Kad įstatymu niekas neišteisinamas Dievo akyse, aišku, nes “teisusis gyvens tikėjimu”. 
\par 12 O įstatymas nėra kilęs iš tikėjimo, bet “kas juos vykdo, tas gyvens jais”. 
\par 13 Kristus mus atpirko iš įstatymo prakeikimo, tapdamas už mus prakeikimu, nes parašyta: “Prakeiktas kiekvienas, kuris kybo ant medžio”,­ 
\par 14 kad Abraomo palaiminimas Jėzuje Kristuje atitektų pagonims ir mes tikėjimu gautume pažadėtąją Dvasią. 
\par 15 Broliai, kalbu, kaip įprasta žmonėms: net žmogaus testamento, kuris patvirtintas, niekas neatmeta ir nepapildo. 
\par 16 Pažadai buvo duoti Abraomui ir jo palikuoniui. Jis nesako “ir palikuonims”, ne daugeliui, bet kaip apie vieną: “ir tavo palikuoniui”, kuris yra Kristus. 
\par 17 Noriu pasakyti, kad Dievo Kristuje anksčiau patvirtinto testamento negali panaikinti po keturių šimtų trisdešimties metų atsiradęs įstatymas, ir jis negali pažado paversti negaliojančiu. 
\par 18 Jei paveldėjimas būtų iš įstatymo, tai jau nebe iš pažado, o Dievas davė tai Abraomui pažadu. 
\par 19 Tad kam gi reikalingas įstatymas? Jis buvo pridėtas dėl nusižengimų, kol ateis palikuonis, kuriam buvo skirtas pažadas; įstatymas buvo perduotas per angelus, tarpininko ranka. 
\par 20 Tarpininkas neatstovauja vienai pusei, bet Dievas yra vienas. 
\par 21 Tad gal įstatymas priešingas Dievo pažadams? Anaiptol! Jei būtų duotas įstatymas, galintis suteikti gyvenimą, tai iš tikrųjų teisumas būtų iš įstatymo. 
\par 22 Bet Raštas viską apjuosė nuodėme, kad pažadas dėl tikėjimo Jėzumi Kristumi tektų tiems, kurie tiki. 
\par 23 Prieš ateinant tikėjimui, buvome įstatymo įkalinti, kad lauktume apsireiškiant tikėjimo. 
\par 24 Todėl įstatymas buvo mūsų auklėtojas, vedęs į Kristų, kad būtume tikėjimu išteisinti. 
\par 25 Bet, tikėjimui atėjus, jau nesame auklėtojo globoje. 
\par 26 Juk jūs visi esate Dievo sūnūs per tikėjimą Jėzumi Kristumi. 
\par 27 Ir visi, kurie esate pakrikštyti Kristuje, apsivilkote Kristumi. 
\par 28 Nebėra nei žydo, nei graiko; nebėra nei vergo, nei laisvojo; nebėra nei vyro, nei moters: visi esate viena Kristuje Jėzuje! 
\par 29 O jeigu esate Kristaus, tai esate Abraomo palikuonys ir paveldėtojai pagal pažadą.


\chapter{4}


\par 1 Dabar pasakysiu: kol paveldėtojas vaikas, jis nieku nesiskiria nuo vergo, nors yra visko šeimininkas; 
\par 2 jis esti globėjų ir prižiūrėtojų valdžioje iki tėvo nustatyto meto. 
\par 3 Taip buvo ir su mumis: kol buvome vaikai, turėjome vergauti pasaulio pradmenims. 
\par 4 Bet, atėjus laiko pilnatvei, Dievas atsiuntė savo Sūnų, gimusį iš moters, pavaldų įstatymui, 
\par 5 kad atpirktų esančius įstatymo valdžioje ir kad mes įgytume įsūnystę. 
\par 6 O kadangi esate sūnūs, Dievas atsiuntė į mūsų širdis savo Sūnaus Dvasią, kuri šaukia: “Aba, Tėve!” 
\par 7 Taigi tu jau nebe vergas, bet sūnus; o jeigu sūnus, tai ir Dievo paveldėtojas per Kristų. 
\par 8 Kitados, dar nepažindami Dievo, jūs vergavote dievams, kurie iš tikro nėra dievai. 
\par 9 Bet dabar, pažinę Dievą arba, geriau sakant, Dievo pažinti,­ kaipgi galite grįžti prie menkų ir vargingų pradmenų, kuriems ir vėl norite vergauti?! 
\par 10 Jūs laikotės dienų, mėnesių, laikotarpių, metų. 
\par 11 Aš baiminuosi dėl jūsų, kad kartais nebūčiau veltui dirbęs jūsų labui. 
\par 12 Prašau jus, broliai, tapkite tokie kaip aš, nes ir aš tapau toks kaip jūs. Jūs visai neįžeidėte manęs. 
\par 13 Jūs žinote, kad jums pirmą kartą paskelbiau Evangeliją negaluodamas kūnu. 
\par 14 Ir jūs nepaniekinote ir nepasipiktinote mano kūne esančiu išbandymu, bet priėmėte mane kaip Dievo angelą, kaip patį Kristų Jėzų. 
\par 15 Koks tai buvo palaiminimas jums! Aš galiu paliudyti, kad įmanydami būtumėte išlupę savo akis ir atidavę man. 
\par 16 Nejaugi tapau jūsų priešu, kalbėdamas jums tiesą? 
\par 17 Jie uolūs ne jūsų labui, bet norėtų jus atskirti, kad būtumėte uolūs dėl jų. 
\par 18 Bet gerai visada būti uoliems dėl gero, o ne vien kai esu pas jus. 
\par 19 Mano vaikeliai, dėl jūsų aš vėl gimdymo skausmuose, kol jumyse išryškės Kristus! 
\par 20 Norėčiau dabar būti pas jus ir prabilti kitaip, nes nežinau, ką man su jumis daryti. 
\par 21 Pasakykite man jūs, norintieji būti įstatymo valdžioje, ar negirdite įstatymo? 
\par 22 Juk parašyta, kad Abraomas turėjo du sūnus: vieną iš vergės, o kitą iš laisvosios. 
\par 23 Vergės sūnus buvo gimęs pagal kūną, o laisvosios­pagal pažadą. 
\par 24 Tai pasakyta perkeltine prasme: jos­tai dvi Sandoros. Viena nuo Sinajaus kalno, gimdanti vergystei,­tai Hagara. 
\par 25 Hagara yra Sinajaus kalnas Arabijoje; ji atitinka dabartinę Jeruzalę, kuri vergauja su savo vaikais. 
\par 26 Bet aukštybių Jeruzalė laisva, ji yra visų mūsų motina, 
\par 27 nes parašyta: “Pralinksmėk, nevaisingoji, kuri negimdei! Šūkauk ir džiūgauk, nepažinusi kentėjimų! Nes apleistoji turi daug daugiau vaikų negu turinčioji vyrą”. 
\par 28 Mes, broliai, esame pažado vaikai kaip Izaokas. 
\par 29 Bet kaip tada gimęs pagal kūną persekiojo gimusį pagal Dvasią, taip ir dabar. 
\par 30 O ką gi sako Raštas?­“Išvaryk vergę ir jos sūnų, nes vergės sūnus negaus palikimo kartu su laisvosios sūnumi”. 
\par 31 Taigi, broliai, mes nesame vergės vaikai, bet laisvosios.


\chapter{5}


\par 1 Todėl tvirtai stovėkite laisvėje, kuria Kristus mus išlaisvino, ir nesiduokite vėl įkinkomi į vergystės jungą! 
\par 2 Štai aš, Paulius, sakau jums: jeigu būsite apipjaustyti, Kristus nebebus jums niekuo naudingas. 
\par 3 Pakartotinai įspėju kiekvieną, kuris tampa apipjaustytas: jis yra įpareigotas vykdyti visą įstatymą. 
\par 4 Jūs, ieškantys išteisinimo įstatyme, atsiskyrėte nuo Kristaus, praradote malonę. 
\par 5 O mes per Dvasią karštai laukiame ir viliamės tikėjimo teisumo. 
\par 6 Nes Kristuje Jėzuje nieko nereiškia nei apipjaustymas, nei neapipjaustymas, bet tikėjimas, veikiantis meile. 
\par 7 Jūs taip gražiai bėgote! Kas gi jums sukliudė paklusti tiesai? 
\par 8 Ne Tas, kuris jus pašaukė, šitaip įtikino. 
\par 9 Truputis raugo įraugina visą maišymą. 
\par 10 Pasitikiu jumis Viešpatyje, kad nemąstysite kitaip; o jūsų drumstėjas, kas jis bebūtų, susilauks pasmerkimo. 
\par 11 Jei aš, broliai, iki šiol skelbiu apipjaustymą, tai kodėl gi esu persekiojamas? Juk tada kryžiaus papiktinimas būtų pašalintas. 
\par 12 O kad jūsų drumstėjai ir nusipjautų! 
\par 13 Jūs, broliai, esate pašaukti laisvei! Tiktai tenebūna ši laisvė proga kūnui, bet meile tarnaukite vieni kitiems. 
\par 14 Juk visas įstatymas išsipildo viename žodyje: “Mylėk savo artimą kaip save patį”. 
\par 15 Bet jeigu jūs vienas kitą kremtate ir ėdate, saugokitės, kad nebūtumėte vienas kito praryti! 
\par 16 Sakau: gyvenkite Dvasia, ir jūs nevykdysite kūno geismų. 
\par 17 Nes kūnas geidžia priešingo Dvasiai, o Dvasia­kūnui; jie vienas kitam priešingi, todėl negalite daryti visko, ko norėtumėte. 
\par 18 Bet jei jūs Dvasios vedami, nebesate įstatymo valdžioje. 
\par 19 Kūno darbai aiškūs­tai paleistuvavimas, ištvirkavimas, netyrumas, gašlavimas, 
\par 20 stabmeldystė, burtininkavimas, priešiškumai, nesantaikos, pavyduliavimai, piktumai, vaidai, nesutarimai, susiskaldymai, 
\par 21 pavydai, žmogžudystės, girtavimai, orgijos ir panašūs dalykai. Įspėju jus, kaip jau esu įspėjęs, jog tie, kurie taip daro, nepaveldės Dievo karalystės. 
\par 22 Bet Dvasios vaisiai yra meilė, džiaugsmas, ramybė, kantrybė, malonumas, gerumas, ištikimybė, 
\par 23 romumas, susivaldymas. Tokiems dalykams nėra įstatymo. 
\par 24 Ir kurie yra Kristaus, tie nukryžiavo kūną su aistromis ir geismais. 
\par 25 Jei gyvename Dvasia, tai ir elkimės pagal Dvasią. 
\par 26 Nesivaikykime tuščios garbės, neerzinkime vieni kitų, nepavydėkime vieni kitiems.


\chapter{6}


\par 1 Broliai, jei žmogus įpuola į kokią nuodėmę, jūs, dvasiniai žmonės, pataisykite tokį romumo dvasioje, žiūrėdami savęs, kad ir patys nebūtumėt sugundyti. 
\par 2 Nešiokite vieni kitų naštas, ir taip įvykdysite Kristaus įstatymą. 
\par 3 O kas, būdamas niekas, tariasi esąs kažin kas, tas save apgaudinėja. 
\par 4 Tegul kiekvienas ištiria savo darbą, ir tada galės girtis pats sau, o ne kitam, 
\par 5 nes kiekvienas neš savo naštą. 
\par 6 Kas mokomas žodžio, tegul dalijasi visais gerais dalykais su mokytoju. 
\par 7 Neapsigaukite! Iš Dievo nepasišaipysi. Ką žmogus sėja, tą ir pjaus. 
\par 8 Kas sėja savo kūnui, tas iš kūno pjaus supuvimą, o kas sėja Dvasiai, tas iš Dvasios pjaus amžinąjį gyvenimą. 
\par 9 Nepavarkime daryti gera, nes savo metu pjausime derlių, jei nepailsime! 
\par 10 Tad, kol turime laiko, darykime gera visiems, o ypač tikėjimo namiškiams. 
\par 11 Žiūrėkite, kokiomis didelėmis raidėmis jums parašiau savo ranka. 
\par 12 Visi, kurie nori pasirodyti geri kūnu, verčia jus apsipjaustyti, kad tik jiems netektų kęsti persekiojimų dėl Kristaus kryžiaus. 
\par 13 Bet net ir patys apipjaustyti nesilaiko įstatymo, o tenori jūsų apipjaustymo, kad galėtų pasigirti jumis. 
\par 14 Aš nieku nesigirsiu, tik mūsų Viešpaties Jėzaus Kristaus kryžiumi, kuriuo pasaulis man yra nukryžiuotas ir aš pasauliui. 
\par 15 Nes Kristuje Jėzuje nieko nereiškia nei apipjaustymas, nei neapipjaustymas, bet naujas kūrinys. 
\par 16 Visiems, kurie laikysis šios taisyklės, teateinie ramybė ir pasigailėjimas; taip pat ir Dievo Izraeliui! 
\par 17 Nuo šiol tegul niekas manęs nebevargina, nes savo kūne nešioju Viešpaties Jėzaus žymes. 
\par 18 Mūsų Viešpaties Jėzaus Kristaus malonė tebūna, broliai, su jūsų dvasia! Amen.


\end{document}