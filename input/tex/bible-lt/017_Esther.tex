\begin{document}

\title{Esther}

\chapter{1}


\par 1 Ahasvero, kuris valdė šimtą dvidešimt septynis kraštus nuo Indijos iki Etiopijos, 
\par 2 dienomis, kai Ahasveras atsisėdo karalystės soste, kuris yra Sūzuose, 
\par 3 trečiais karaliavimo metais jis iškėlė didelę puotą visiems savo kunigaikščiams ir tarnams, persų ir medų kraštų kilmingiesiems ir sričių kunigaikščiams. 
\par 4 Puota tęsėsi šimtą aštuoniasdešimt dienų; joje buvo iškelta karališkoji didybė, garbė ir turtai. 
\par 5 Po to karalius suruošė septynių dienų puotą visiems sostinės gyventojams Sūzuose, karaliaus rūmų sodo kieme. 
\par 6 Iš visų pusių kabojo baltos, žalios ir mėlynos užuolaidos ant baltų drobinių ir violetinių juostelių, įvertų į sidabrinius žiedus. Jos buvo pritvirtintos prie marmurinių kolonų. Auksiniai ir sidabriniai gultai buvo sustatyti kieme, kuris buvo išklotas raudonu, mėlynu, baltu ir juodu marmuru. 
\par 7 Gėrimus patiekė auksinėse taurėse, kurios visos buvo skirtingos; karališko vyno buvo gausu. 
\par 8 Visi gėrė, kiek norėjo, nė vieno gerti nevertė. Karalius buvo įsakęs savo namų prižiūrėtojams, kad jie darytų tai, ko kuris žmogus panorės. 
\par 9 Karalienė Vaštė karaliaus Ahasvero namuose kėlė puotą moterims. 
\par 10 Septintą dieną karalius buvo linksmas nuo vyno. Jis įsakė Mehumanui, Biztai, Harbonai, Bigtai, Abagtai, Zetarui ir Karkasui­septyniems eunuchams, kurie tarnavo karaliaus Ahasvero akivaizdoje, 
\par 11 atvesti karalienę Vaštę pas karalių su karališka karūna, norėdamas parodyti žmonėms ir kunigaikščiams jos grožį, nes ji buvo labai graži. 
\par 12 Karalienė Vaštė atmetė karaliaus įsakymą, perduotą eunuchų, ir atsisakė eiti. Karalius labai supyko, ir rūstybė užsidegė jame. 
\par 13 Karalius klausė patarimo išminčių, kurie pažindavo laikus ir žinojo karaliaus įstatymus bei teisę, 
\par 14 taip pat Karšenos, Šetaro, Admatos, Taršišo, Mereso, Marsenos ir Memuchano, persų ir medų septynių kunigaikščių, kurie visada būdavo prie karaliaus, ir buvo pirmi karalystėje: 
\par 15 “Ką mes turime daryti karalienei Vaštei pagal įstatymą už tai, kad ji nepaklausė karaliaus įsakymo, perduoto per eunuchus?” 
\par 16 Memuchanas atsakė karaliui: “Karalienė Vaštė nusikalto ne tik karaliui, bet visiems kunigaikščiams ir žmonėms visose karaliaus Ahasvero žemėse. 
\par 17 Šis karalienės poelgis taps žinomas visoms moterims. Jos, paniekindamos savo vyrus, sakys: ‘Karalius Ahasveras įsakė karalienei ateiti pas jį, bet ji neatėjo’. 
\par 18 Tuo karalienės pavyzdžiu seks visos persų ir medų kunigaikštienės. Kils daug nesusipratimų ir pykčių. 
\par 19 Jei karaliui patiktų, tebūna paskelbtas karališkas įsakymas ir įrašytas į persų ir medų įstatymą, kuris yra nepakeičiamas, kad karalienė Vaštė nebeateis pas karalių Ahasverą ir karalius jos karališką garbę atiduos kitai, geresnei už ją. 
\par 20 Kai karaliaus potvarkis bus paskelbtas visoje plačioje karalystėje, visos moterys gerbs savo vyrus, didelius ir mažus”. 
\par 21 Šis patarimas patiko karaliui ir kunigaikščiams. Karalius padarė pagal Memuchano žodžius. 
\par 22 Jis išsiuntinėjo laiškus į visus karaliaus kraštus, kiekvienai tautai jos kalba, kad vyras turi būti viešpats savo namuose. Tas įstatymas buvo paskelbtas visomis kalbomis visose tautose.


\chapter{2}

\par 1 Po šių įvykių, karaliaus Ahasvero pykčiui aprimus, jis atsiminė, ką Vaštė padarė ir kaip ji buvo nubausta. 
\par 2 Karaliaus tarnai, kurie jam patarnaudavo, sakė jam: “Tepaieško karaliui jaunų gražių mergaičių 
\par 3 ir tepaskiria karalius visuose kraštuose vyrų, kurie visoje karalystėje suieškotų gražių mergaičių, atgabentų jas į Sūzų miestą, į moterų namus, ir perduotų karaliaus eunuchui Hegajui, moterų prižiūrėtojui. Ir tegu duoda joms visa, ko reikia joms pasipuošti. 
\par 4 Kuri mergaitė patiks karaliui, taps karaliene Vaštės vietoje”. Tas pasiūlymas patiko karaliui. Jis taip ir padarė. 
\par 5 Sostinėje Sūzuose gyveno vienas žydas, vardu Mordechajas, sūnus Jayro, sūnaus Šimio, sūnaus Kišo iš Benjamino giminės. 
\par 6 Mordechajas buvo ištremtas iš Jeruzalės kartu su Judo karaliumi Jechoniju, kurį ištrėmė Babilono karalius Nebukadnecaras. 
\par 7 Jis užaugino Hadasą, savo dėdės dukterį, kitaip vadinamą Estera. Ji neturėjo tėvų, bet buvo labai graži ir patraukli mergaitė. Jos tėvams mirus, Mordechajas įdukrino ją. 
\par 8 Karaliaus įsakymą paskelbus, daug gražių mergaičių buvo atgabenta į Sūzus ir pavesta Hegajo globai. Estera taip pat buvo paimta į karaliaus namus Hegajo, moterų prižiūrėtojo, globon. 
\par 9 Mergaitė jam patiko ir įgavo jo palankumą. Jis greitai parūpino jai įvairių papuošalų ir tai, kas jai priklausė, bei septynias tarnaites iš karaliaus namų ir apgyvendino ją ir jos tarnaites geriausioje moterų namų dalyje. 
\par 10 Estera nesakė apie savo kilmę ir savo tautą, nes Mordechajas buvo liepęs jai tylėti. 
\par 11 Mordechajas kiekvieną dieną ateidavo prie moterų namų kiemo pasiteirauti apie Esterą ir jos padėtį. 
\par 12 Kiekviena mergaitė tik po dvylikos mėnesių pasiruošimo galėjo įeiti pas karalių Ahasverą. Šešis mėnesius jos tepėsi miros aliejumi ir šešis mėnesius įvairiais kitais tepalais ir kvepalais. 
\par 13 Mergaitė, eidama pas karalių, turėjo teisę pasiimti iš moterų namų į karaliaus namus ką tik norėjo. 
\par 14 Ji ateidavo vakare, o rytą eidavo į kitus moterų namus, kurie buvo eunucho Šaašgazo žinioje. Jis buvo karaliaus sugulovių prižiūrėtojas. Ji daugiau nebeįeidavo pas karalių, nebent karalius ją pamėgdavo ir pakviesdavo vardu. 
\par 15 Kai atėjo eilė Esteros, Mordechajo dėdės Abihailo dukters, kurią jis buvo įdukrinęs, ji nieko neprašė; pasiėmė tik tą, ką eunuchas Hegajas jai patarė. Estera patiko kiekvienam, kas tik į ją pažiūrėdavo. 
\par 16 Esterą nuvedė į karaliaus Ahasvero rūmus dešimtą mėnesį, vadinamą tebetu, septintaisiais jo karaliavimo metais. 
\par 17 Karalius pamilo Esterą labiau už visas kitas moteris, ir ji atrado jo akyse daugiau malonės ir palankumo negu kitos mergaitės. Jis uždėjo jai ant galvos karališką karūną ir paskelbė ją karaliene Vaštės vietoje. 
\par 18 Karalius iškėlė didelę puotą savo kunigaikščiams ir tarnams; tai buvo Esteros puota. Jis sumažino mokesčius savo karalystėje ir davė karališkų dovanų. 
\par 19 Kai mergaitės buvo renkamos antrą kartą, Mordechajas sėdėjo karaliaus vartuose. 
\par 20 Estera dar nebuvo pasisakiusi karaliui nei savo tautybės, nei kilmės, nes Mordechajas taip buvo jai liepęs; Estera klausė Mordechajaus kaip ir tada, kai jis ją augino. 
\par 21 Mordechajui sėdint prie karaliaus vartų, du karaliaus eunuchai, Bigtanas ir Terešas, durų sargai, supyko ant karaliaus ir tarėsi jį nužudyti. 
\par 22 Mordechajas, tai išgirdęs, pranešė karalienei Esterai, o Estera­karaliui Mordechajo vardu. 
\par 23 Buvo ištirta ir nustatyta, kad tai tiesa. Abu vyrus pakorė, o įvykį įrašė į karaliaus metraščių knygą.
Online Parallel Study Bible



\chapter{3}

\par 1 Po šitų įvykių karalius Ahasveras išaukštino agagą Hamaną, Hamedatos sūnų, ir padarė jo sostą aukštesnį už kitų kunigaikščių. 
\par 2 Visi karaliaus tarnai, kurie buvo prie karaliaus vartų, nusilenkdavo prieš Hamaną ir pagerbdavo jį, nes taip buvo įsakęs karalius. Bet Mordechajas nenusilenkdavo prieš jį ir nepagerbdavo. 
\par 3 Karaliaus tarnai prie karaliaus vartų klausdavo Mordechajo: “Kodėl nevykdai karaliaus įsakymo?” 
\par 4 Kai jie kasdien kartodavo tai, bet jis nekreipė dėmesio, jie pranešė Hamanui. Jie norėjo įsitikinti, ar Mordechajas laikysis savo žodžio, nes jis buvo pasisakęs esąs žydas. 
\par 5 Hamanas, matydamas, kad Mordechajas nesilenkia prieš jį ir nepagerbia jo, užsidegė pykčiu. 
\par 6 Jis manė, kad neverta kelti rankos prieš vieną Mordechają, nes jie pasakė jam Mordechajo tautybę. Todėl Hamanas nusprendė išžudyti visus žydus visoje Ahasvero karalystėje. 
\par 7 Dvyliktaisiais karaliaus Ahasvero metais, pirmą mėnesį, vadinamą nisanu, buvo metamas Pur, tai yra burtas, Hamano akivaizdoje kiekvienai dienai ir kiekvienam mėnesiui iki dvylikto mėnesio adaro. 
\par 8 Hamanas sakė karaliui Ahasverui: “Tavo karalystėje gyvena išsklaidyta tauta; jos įstatymai yra skirtingi nuo kitų tautų įstatymų, be to, jie nesilaiko karaliaus įstatymų. Karaliui nėra naudinga juos taip palikti. 
\par 9 Jei karalius sutinka, tebūna išleistas įsakymas juos išnaikinti, ir aš duosiu dešimt tūkstančių talentų sidabro karaliaus iždui”. 
\par 10 Karalius numovė žiedą nuo savo piršto ir padavė agagui Hamanui, Hamedatos sūnui, žydų priešui, 
\par 11 ir tarė Hamanui: “Sidabrą pasilaikyk sau, o su žydų tauta daryk, kaip tau patinka”. 
\par 12 Pirmo mėnesio tryliktą dieną buvo sušaukti karaliaus raštininkai ir, Hamanui diktuojant, buvo parašyti laiškai visiems vietininkams, kraštų valdytojams, kunigaikščiams ir tautoms jų kalba, karaliaus Ahasvero vardu ir užantspauduoti karaliaus žiedu. 
\par 13 Laiškai buvo išsiųsti per pasiuntinius į visus karaliaus kraštus, kad per vieną dieną, tai yra dvylikto mėnesio, vadinamo Adaru, tryliktą dieną, išžudytų ir išnaikintų visus žydus: jaunus, senus, moteris ir vaikus, o jų turtą paimtų. 
\par 14 Tų laiškų nuorašus įsakyta paskelbti visoms tautoms, kad jie pasiruoštų skirtai dienai. 
\par 15 Karaliaus įsakymu pasiuntiniai išskubėjo, ir šis įsakymas buvo paskelbtas sostinėje Sūzuose. Karalius ir Hamanas sėdėjo ir gėrė, o Sūzų miestas buvo sujudęs.



\chapter{4}

\par 1 Mordechajas, išgirdęs, kas buvo padaryta, perplėšė savo drabužius, apsivilko ašutine, užsibarstė ant galvos pelenų, išėjo į miesto aikštę ir garsiai dejavo. 
\par 2 Jis atėjo prie karaliaus vartų, nes su ašutine nebuvo leidžiama įeiti pro karaliaus vartus. 
\par 3 Visur, kur karaliaus įsakymas ir potvarkis buvo paskelbtas, tarp žydų kilo didelis gedulas. Jie pasninkavo, verkė, aimanavo, daugelis, apsirengę ašutinėmis, gulėjo pelenuose. 
\par 4 Kai Esteros tarnaitės ir eunuchai jai tai pranešė, karalienė labai nuliūdo. Ji siuntė Mordechajui rūbų, kad jis nusimestų ašutinę ir apsivilktų jais, bet jis jų nepriėmė. 
\par 5 Tada Estera pasišaukė karaliaus eunuchą Hatachą, kuris buvo paskirtas jai prižiūrėti, ir įsakė jam eiti pas Mordechają ir sužinoti, kodėl jis taip elgiasi. 
\par 6 Hatachas nuėjo pas Mordechają į miesto aikštę priešais karaliaus vartus. 
\par 7 Mordechajas papasakojo jam, kas atsitiko ir kiek Hamanas pažadėjo duoti karaliaus iždui už žydų išžudymą. 
\par 8 Mordechajas davė Hatachui nuorašą įsakymo išžudyti žydus, paskelbto Sūzuose, kad jį parodytų Esterai, paaiškintų ir lieptų jai eiti pas karalių ir maldauti pagalbos savo tautai. 
\par 9 Hatachas sugrįžo ir pranešė Esterai, ką Mordechajas jam kalbėjo. 
\par 10 Estera vėl siuntė Hatachą pas Mordechają: 
\par 11 “Visi karaliaus tarnai ir visi kraštai žino, kad jei vyras ar moteris įeitų į karaliaus vidinį kiemą nekviestas, tai yra tik vienas įstatymas­mirtis; paliekamas gyvas tik tas, į kurį karalius ištiesia savo auksinį skeptrą. Karalius nekvietė manęs jau trisdešimt dienų”. 
\par 12 Mordechajas sužinojęs, ką Estera atsakė, 
\par 13 liepė pranešti Esterai: “Negalvok, kad tu, būdama karaliaus namuose, išvengsi žydų likimo. 
\par 14 Jei dabar tylėsi, pagalba ir išgelbėjimas ateis žydams iš kitur, bet tu ir tavo tėvo namai žūsite. Kas žino, gal dėl šio laiko tu ir tapai karaliene?” 
\par 15 Estera pasiuntė Mordechajui atsakymą: 
\par 16 “Eik ir sušauk visus žydus Sūzuose; pasninkaukite dėl manęs tris dienas ir tris naktis, nieko nevalgykite ir negerkite. Aš taip pat pasninkausiu su savo tarnaitėmis. Po to eisiu pas karalių, laužydama įstatymą; jei žūsiu, tai žūsiu”. 
\par 17 Mordechajas nuėjo ir padarė visa, ką Estera jam buvo įsakiusi.



\chapter{5}


\par 1 Trečią dieną Estera apsirengė karališkais rūbais ir atsistojo karaliaus namų vidiniame kieme, priešais karaliaus namus. Karalius sėdėjo savo soste, priešais įėjimą į namus. 
\par 2 Karalius pamatė karalienę Esterą, stovinčią kieme, ir ji atrado malonę jo akyse. Jis ištiesė į Esterą auksinį skeptrą, kurį laikė rankoje. Estera priėjusi palietė skeptrą. 
\par 3 Karalius klausė: “Karaliene Estera, ko norėtum? Duosiu tau, ko prašysi, kad ir pusę savo karalystės”. 
\par 4 Estera atsakė: “Jei karaliui patiktų, kviečiu karalių ir Hamaną šiandien į vaišes, kurias jiems paruošiau”. 
\par 5 Karalius įsakė skubiai pašaukti Hamaną, norėdamas išpildyti Esteros prašymą. Karalius ir Hamanas atėjo į vaišes pas Esterą. 
\par 6 Geriant vyną, karalius klausė Esteros: “Pasakyk man savo prašymą. Išpildysiu jį, nors prašytum ir pusės mano karalystės”. 
\par 7 Estera atsakė: “Mano prašymas toks: 
\par 8 jei radau malonę karaliaus akyse ir karalius norėtų patenkinti mano prašymą, kviečiu karalių ir Hamaną rytoj į vaišes, kurias jiems paruošiu, ir rytoj aš įvykdysiu karaliaus žodžius”. 
\par 9 Tą dieną Hamanas išėjo linksmas ir gerai nusiteikęs. Bet, pamatęs prie karaliaus vartų sėdintį Mordechają, kuris neatsistojo ir visai nekreipė dėmesio į jį, labai supyko. 
\par 10 Tačiau Hamanas susivaldė. Parėjęs namo, jis pasišaukė draugus ir žmoną Zerešą 
\par 11 ir pasakojo jiems apie savo turtų daugybę, apie savo sūnus, apie tai, kaip karalius pagerbė jį ir suteikė pirmenybę tarp visų karaliaus tarnų ir kunigaikščių. 
\par 12 Hamanas tęsė: “Ir karalienė Estera nė vieno nepakvietė į vaišes kartu su karaliumi, tik mane; ir rytoj ji vėl pakvietė mane su karaliumi. 
\par 13 Tačiau tai manęs nepatenkina, kol matau šitą žydą Mordechają, sėdintį prie karaliaus vartų”. 
\par 14 Jo žmona ir draugai jam patarė: “Įsakyk pastatyti penkiasdešimties uolekčių aukščio kartuves ir rytoj kalbėk su karaliumi, kad Mordechajas būtų pakartas. Po to eik linksmas į vaišes kartu su karaliumi”. Patarimas patiko Hamanui, ir kartuvės buvo pastatytos.



\chapter{6}


\par 1 Tą naktį karalius negalėjo miegoti. Jis įsakė atnešti metraščių knygą ir jam skaityti iš jos. 
\par 2 Rado parašyta, kaip Mordechajas pranešė apie eunuchų Bigtano ir Terešo, ėjusių sargybą prie vartų, sąmokslą nužudyti karalių Ahasverą. 
\par 3 Karalius paklausė: “Koks pagerbimas ir atlyginimas buvo duotas Mordechajui už tai?” Karaliaus tarnai, kurie jam patarnavo, atsakė: “Jokio atlyginimo”. 
\par 4 Karalius klausė: “Kas yra kieme?” Tuo laiku Hamanas buvo įėjęs į karaliaus namų išorinį kiemą kalbėti su karaliumi dėl Mordekajo pakorimo ant kartuvių, kurios jau buvo pastatytos. 
\par 5 Karaliaus tarnai jam atsakė: “Hamanas stovi kieme”. Karalius liepė pakviesti Hamaną. 
\par 6 Jam įėjus, karalius klausė: “Ką reikėtų padaryti vyrui, kurį karalius nori pagerbti?” Hamanas galvojo: “Ką gi kitą karalius norėtų pagerbti, jei ne mane?” 
\par 7 Hamanas atsakė: “Vyrą, kurį karalius norėtų pagerbti, 
\par 8 reikia aprengti karaliaus rūbais, užsodinti ant karaliaus jojamojo žirgo ir uždėti ant galvos karaliaus karūną. 
\par 9 Rūbus ir žirgą tegul jam paduoda vienas iš karaliaus kilmingiausiųjų kunigaikščių; jis teatveda karaliaus pagerbtąjį, užsodina jį ant žirgo ir, vesdamas žirgą miesto aikšte, tegul skelbia: ‘Taip padaroma tam vyrui, kurį karalius nori pagerbti!’ ” 
\par 10 Tada karalius įsakė Hamanui: “Skubėk, imk rūbus, žirgą ir padaryk visa, ką sakei, žydui Mordechajui, kuris sėdi prie karaliaus vartų. Žiūrėk, kad viskas būtų įvykdyta”. 
\par 11 Hamanas paėmė rūbus ir žirgą, aprengė Mordechajų, vedė jį ant žirgo miesto aikšte ir skelbė: “Taip padaroma tam vyrui, kurį karalius nori pagerbti!” 
\par 12 Mordechajas grįžo prie karaliaus vartų, o Hamanas nuskubėjo į savo namus nuliūdęs ir užsidengęs galvą. 
\par 13 Hamanas papasakojo savo žmonai Zerešai ir draugams visa, kas įvyko. Jo patarėjai ir žmona kalbėjo: “Jei Mordechajas, prieš kurį turėjai nusižeminti, yra žydų tautybės, tu nelaimėsi, bet tikrai krisi”. 
\par 14 Jiems dar tebekalbant, atėjo karaliaus eunuchai ir skubėjo nuvesti Hamaną į Esteros paruoštas vaišes.



\chapter{7}


\par 1 Karalius ir Hamanas atėjo pas karalienę Esterą į vaišes. 
\par 2 Tą dieną, gerdamas vyną, karalius vėl klausė karalienę Esterą: “Pasakyk man savo prašymą, karaliene Estera. Suteiksiu tau, ko norėsi, net jei prašytum ir pusės mano karalystės”. 
\par 3 Karalienė atsakė: “Karaliau, jei radau malonę tavo akyse ir sutinki mano prašymą patenkinti, palik mane ir mano tautą gyvus. 
\par 4 Aš ir mano tauta esame atiduoti mirčiai ir sunaikinimui. Jei mes būtume parduodami vergais, aš tylėčiau ir karaliaus nevarginčiau”. 
\par 5 Karalius Ahasveras klausė karalienės Esteros: “Kas jis toks ir kur jis yra, kuris drįsta tą daryti?” 
\par 6 Estera atsakė: “Mūsų priešas ir persekiotojas yra šitas piktadarys Hamanas”. Hamanas labai išsigando. 
\par 7 Karalius užsirūstinęs pakilo nuo stalo ir išėjo į rūmų sodą, o Hamanas suklupo prieš karalienę Esterą, prašydamas palikti jį gyvą, nes jis suprato, kad jo laukia karaliaus bausmė. 
\par 8 Sugrįžęs iš rūmų sodo, karalius rado Hamaną, sukniubusį ant lovos prie Esteros, ir sakė: “Jis nori karalienę išprievartauti mano namuose ir mano akivaizdoje!” Karaliui ištarus šiuos žodžius, Hamano veidas buvo uždengtas. 
\par 9 Karaliaus eunuchas Harbona tarė: “Hamano kieme yra pastatytos penkiasdešimties uolekčių aukščio kartuvės pakarti Mordechajui, kuris išgelbėjo karaliaus gyvybę”. Karalius įsakė: “Pakarkite jose Hamaną”. 
\par 10 Jie pakorė jį Mordechajui skirtose kartuvėse. Po to karaliaus pyktis atslūgo.



\chapter{8}


\par 1 Tą dieną karalius Ahasveras atidavė žydų priešo Hamano namus karalienei Esterai. Mordechajas buvo pakviestas pas karalių, nes Estera pasisakė, kad jis yra jos giminaitis. 
\par 2 Savo žiedą, kurį pasiėmė iš Hamano, karalius atidavė Mordechajui. Estera pavedė Hamano namų priežiūrą Mordechajui. 
\par 3 Estera krito karaliui po kojomis ir su ašaromis maldavo panaikinti agago Hamano išleistą įsakymą prieš žydus. 
\par 4 Karalius ištiesė aukso skeptrą į Esterą. Estera pakilo, atsistojo prieš karalių 
\par 5 ir kalbėjo: “Jei karaliui patiktų ir jei radau malonę jo akyse, maldauju raštu atšaukti agago Hamano, Hamedatos sūnaus, laiškus, kuriais jis siekė išnaikinti visus žydus karaliaus kraštuose. 
\par 6 Kaip aš galėsiu ištverti, matydama savo tautos pažeminimą ir sunaikinimą?” 
\par 7 Karalius Ahasveras tarė karalienei Esterai ir žydui Mordechajui: “Aš atidaviau Esterai Hamano namus; jis buvo pakartas, kadangi drįso pakelti ranką prieš žydus. 
\par 8 Parašykite žydams karaliaus vardu, kaip jums patinka, ir užantspauduokite raštus karaliaus žiedu; nes įsakymai, parašyti karaliaus vardu ir užantspauduoti jo žiedu, yra neatšaukiami”. 
\par 9 Trečio mėnesio, vadinamo sivanu, dvidešimt trečią dieną buvo sušaukti karaliaus raštininkai ir parašyta tai, ką Mordechajas įsakė visiems žydams, karaliaus vietininkams, valdytojams bei kunigaikščiams nuo Indijos iki Etiopijos, šimtui dvidešimt septyniems kraštams jų raštu ir tautoms jų kalba, taip pat ir žydams jų raštu ir jų kalba. 
\par 10 Raštai buvo užantspauduoti karaliaus Ahasvero žiedu ir išsiųsti per raitus karaliaus pasiuntinius. 
\par 11 Karalius leido žydams kiekviename mieste susirinkti ir ginti savo gyvybes, sunaikinti, užmušti ir pražudyti bet kokią žmonių ar krašto galybę, kurie puola juos, kartu su jų vaikais bei moterimis, o jų turtą pasiimti. 
\par 12 Dvylikto mėnesio, vadinamo adaru, tryliktą dieną buvo įsakyta gintis visuose karaliaus Ahasvero kraštuose. 
\par 13 Rašto su įsakymu nuorašai buvo pasiųsti į visus kraštus ir paskelbti visoms tautoms, kad žydai pasiruoštų ir tą dieną atkeršytų savo priešams. 
\par 14 Karaliaus pasiuntiniai ant kupranugarių ir mulų skubėjo išnešioti tą įsakymą. Jis buvo paskelbtas ir sostinėje Sūzuose. 
\par 15 Tą dieną Mordechajas, pasipuošęs karališkais mėlynais ir baltais rūbais, plonos drobės purpuro apsiaustu ir su aukso karūna ant galvos, išėjo iš karaliaus akivaizdos. Sūzų miesto gyventojai džiūgavo. 
\par 16 Žydai buvo laimingi, patenkinti ir gerbiami. 
\par 17 Kiekviename krašte ir mieste, kur tik karaliaus įsakymas buvo paskelbtas, žydai džiaugėsi, linksminosi, puotavo ir šventė. Daugelis kitų tautų žmonių prisidėjo prie žydų, nes juos buvo apėmusi baimė dėl žydų.



\chapter{9}

\par 1 Dvylikto mėnesio, vadinamo adaru, tryliktą dieną, kai atėjo laikas įvykdyti karaliaus įsakymą, tą dieną, kai žydų priešai tikėjosi juos nugalėti, įvyko priešingai­žydai nugalėjo savo priešus. 
\par 2 Visuose karaliaus Ahasvero kraštuose susirinkę į miestus žydai pasipriešino tiems, kurie norėjo jiems pakenkti. Niekas negalėjo žydams priešintis, nes baimė apėmė juos. 
\par 3 Karaliaus kraštų vietininkai, valdytojai, kunigaikščiai ir valdininkai padėjo žydams, nes jie bijojo Mordechajo, 
\par 4 kuris buvo galingas karaliaus namuose. Garsas apie jį pasiekė visus kraštus, o jo galia augo ir augo. 
\par 5 Žydai užpuolė savo priešus, sunaikino juos ir darė, ką norėjo tiems, kurie jų nekentė. 
\par 6 Sostinėje Sūzuose žydai nužudė ir sunaikino penkis šimtus žmonių 
\par 7 ir Paršandatą, Dalfoną, Aspatą, 
\par 8 Poratą, Adaliją, Aridatą, 
\par 9 Parmaštą, Arisają, Aridają ir Vaizatą­ 
\par 10 dešimt žydų priešo Hamano, sūnaus Hamedato, sūnų, bet jų turto nelietė. 
\par 11 Tą pačią dieną Sūzuose nužudytųjų skaičius buvo praneštas karaliui. 
\par 12 Karalius sakė karalienei Esterai: “Sostinėje Sūzuose žydai nužudė penkis šimtus žmonių ir dešimt Hamano sūnų. O kiek jie išžudė visuose kraštuose? Ko dar prašai? Pasakyk, ir tai bus įvykdyta”. 
\par 13 Estera atsakė: “Jei karaliui patiktų, tebūna žydams leista Sūzuose ir rytoj elgtis pagal šios dienos įsakymą, o dešimt Hamano sūnų tebūna pakabinti kartuvėse”. 
\par 14 Karalius įsakė, kad tai būtų padaryta ir paskelbta Sūzuose; ir jie pakorė Hamano sūnus. 
\par 15 Sūzų žydai susirinko adaro mėnesio keturioliktą dieną ir nužudė tris šimtus žmonių, bet jų turto jie nelietė. 
\par 16 Kituose karaliaus kraštuose žydai susirinkę irgi gynė savo gyvybes nuo priešų. Jie nužudė septyniasdešimt penkis tūkstančius savo persekiotojų, bet nelietė jų turto. 
\par 17 Tai įvyko adaro mėnesio tryliktą dieną. Keturioliktą dieną jie ilsėjosi, puotavo ir šventė džiaugsmo šventę. 
\par 18 Sūzų žydai buvo susirinkę tryliktą ir keturioliktą dieną, o penkioliktą dieną jie ilsėjosi, puotavo ir šventė džiaugsmo šventę; 
\par 19 žydai, gyvenantys neapmūrytuose miestuose bei kaimuose, džiaugsmo ir puotos šventei pasirinko keturioliktą adaro mėnesio dieną; tą dieną jie siuntinėjo vieni kitiems dovanas ir valgius. 
\par 20 Mordechajas visa tai surašė ir laiškus žydams išsiuntinėjo visoje karaliaus Ahasvero karalystėje, arti ir toli. 
\par 21 Jis ragino žydus kiekvienais metais švęsti adaro mėnesio keturioliktą ir penkioliktą dienas, 
\par 22 kaip žydų išlaisvinimo iš jų priešų dienas, nes jų liūdesys virto džiaugsmu, dejonės­džiūgavimu. Tomis dienomis jie turėtų džiaugtis, puotauti, dalintis maistu ir beturčiams siųsti dovanų. 
\par 23 Žydai pradėjo taip daryti, kaip Mordechajas buvo jiems nurodęs. 
\par 24 Nes agagas Hamanas, Hamedatos sūnus, žydų priešas, buvo sumanęs sunaikinti žydus ir metė Pur, tai yra burtą, kad juos sunaikintų ir išžudytų. 
\par 25 Kai Estera atėjo pas karalių, jis įsakė raštu, kad Hamano piktas sumanymas kristų ant jo paties galvos­jis ir jo sūnūs buvo pakarti. 
\par 26 Tas dienas nuo žodžio Pur jie vadina Purimu. Todėl pagal visus šio laiško žodžius, pagal tai, ką jie patys matė ir patyrė, 
\par 27 žydai nusprendė, kad kiekvienais metais tuo pačiu laiku tas dvi dienas turi prisiminti visi žydai, jų palikuonys ir visi, prisijungę prie jų. 
\par 28 Tos dienos turi būti švenčiamos visose šeimose, visuose miestuose ir visose kartose. Purimo dienos neturi pranykti iš žydų papročių nei prisiminimas apie jas tarp jų palikuonių. 
\par 29 Karalienė Estera, Abihailo duktė, ir žydas Mordechajas rašė antrą laišką dėl Purimo. 
\par 30 Jie išsiuntinėjo laiškus žydams į šimtą dvidešimt septynis karaliaus Ahasvero kraštus su taikos ir tiesos žodžiais, 
\par 31 kad paragintų švęsti Purimo dienas nustatytu laiku, kurį nurodė žydas Mordechajas ir karalienė Estera. Ir jie patys paskyrė sau pasninkus ir verksmo dienas. 
\par 32 Karalienės Esteros įsakymas patvirtino Purimo šventę ir yra užrašytas knygoje.



\chapter{10}


\par 1 Karalius Ahasveras uždėjo duoklę visiems kraštams ir jūrų saloms. 
\par 2 Visi jo valdžios ir galybės darbai, taip pat Mordechajo garbė, kurią karalius jam suteikė, yra surašyta persų ir medų metraščių knygoje. 
\par 3 Žydas Mordechajas buvo antras po karaliaus Ahasvero. Jis buvo galingas tarp žydų ir mėgstamas tarp brolių, nes jis siekė gerovės ir taikos savo tautai.



\end{document}