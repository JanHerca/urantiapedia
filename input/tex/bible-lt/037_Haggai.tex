\begin{document}

\title{Agėjo knyga}

\chapter{1}


\par 1 Antraisiais karaliaus Darijaus metais, šešto mėnesio pirmą dieną, Viešpats kalbėjo per pranašą Agėją Salatielio sūnui Zorobabeliui, Judo valdytojui, ir Jehocadako sūnui Jozuei, vyriausiajam kunigui, sakydamas: 
\par 2 “Taip sako kareivijų Viešpats: ‘Žmonės kalba, kad dar neatėjo laikas atstatyti Viešpaties namus’ ”. 
\par 3 Tada Viešpats kalbėjo per pranašą Agėją: 
\par 4 “Ar laikas jums gyventi lentomis apmuštuose namuose, kai šventykla tebeguli griuvėsiuose? 
\par 5 Todėl dabar taip sako kareivijų Viešpats: ‘Apsvarstykite savo kelius! 
\par 6 Jūs daug pasėjate, bet mažai pjaunate; valgote, bet nepasisotinate; geriate, bet troškulio nenumalšinate. Jūs apsirengiate, bet nesušylate; o kas dirba už algą, deda ją į kiaurą maišą’. 
\par 7 Taip sako kareivijų Viešpats: ‘Apsvarstykite savo kelius! 
\par 8 Eikite į kalnus, parsigabenkite medžių ir statykite namus. Aš juos pamėgsiu ir būsiu pašlovintas. 
\par 9 Jūs tikėjotės daug, o gavote mažai; parsigabenote į namus, o Aš sunaikinau. Kodėl?­sako kareivijų Viešpats.­Dėl to, kad mano namai guli griuvėsiuose, o jūs skubate statyti savo namus. 
\par 10 Todėl dangus sulaikė rasą, o žemė neduoda derliaus. 
\par 11 Aš pašaukiau sausrą lygumoms ir kalnams, javams, vynui, aliejui ir viskam, kas auga laukuose, gyvuliams bei žmonėms ir jų visiems darbams’ ”. 
\par 12 Tada Salatielio sūnus Zorobabelis ir Jehocadako sūnus Jozuė, vyriausiasis kunigas, ir visas tautos likutis pakluso Viešpaties, savo Dievo, balsui ir pranašo Agėjo žodžiams, kai Viešpats, jų Dievas, jį siuntė. Ir tauta bijojo Viešpaties. 
\par 13 Agėjas, Viešpaties pasiuntinys, skelbė tautai Viešpaties žodžius: “Aš esu su jumis,­sako Viešpats”. 
\par 14 Viešpats pažadino Salatielio sūnaus Zorobabelio, Judo valdytojo, Jehocadako sūnaus Jozuės, vyriausiojo kunigo, ir tautos likučio dvasią. Jie atėjo ir ėmėsi darbo prie kareivijų Viešpaties, savo Dievo, namų 
\par 15 šešto mėnesio dvidešimt ketvirtą dieną, antrais karaliaus Darijaus metais.


\chapter{2}


\par 1 Septinto mėnesio dvidešimt pirmą dieną Viešpats vėl kalbėjo per pranašą Agėją: 
\par 2 “Sakyk Salatielio sūnui Zorobabeliui, Judo valdytojui, ir Jehocadako sūnui Jozuei, vyriausiajam kunigui, bei tautos likučiui: 
\par 3 ‘Kas iš jūsų matė šiuos namus pirmykštėje šlovėje? Kaip jie jums dabar atrodo? Ar jie neatrodo jums mažos vertės? 
\par 4 Bet būk drąsus, Zorobabeli,­sako Viešpats.­Būk drąsus, Jozue, Jehocadako sūnau, vyriausiasis kunige! Būkite drąsūs, visi krašto žmonės! Dirbkite! Aš esu su jumis,­sako kareivijų Viešpats.­ 
\par 5 Kaip jums pažadėjau išeinant iš Egipto, taip ir dabar mano dvasia yra tarp jūsų. Nebijokite!’ 
\par 6 Nes taip sako kareivijų Viešpats: ‘Netrukus Aš dar kartą sudrebinsiu dangus ir žemę, jūrą ir sausumą. 
\par 7 Aš supurtysiu visas tautas ir visų tautų laukiamasis ateis. Ir Aš pripildysiu šiuos namus šlove,­sako kareivijų Viešpats’. 
\par 8 ‘Mano yra sidabras ir mano­ auksas’,­sako kareivijų Viešpats. 
\par 9 ‘Šitų paskutinių namų šlovė bus didesnė už pirmųjų. Šioje vietoje Aš suteiksiu ramybę savo tautai’,­sako kareivijų Viešpats”. 
\par 10 Antraisiais Darijaus metais, devinto mėnesio dvidešimt ketvirtą dieną, Viešpats kalbėjo pranašui Agėjui: 
\par 11 “Pasiklausk kunigų apie įstatymą, sakydamas: 
\par 12 ‘Jei kas neša pašvęstą mėsą savo drabužio skverne ir skvernu paliečia duoną, viralą, vyną, aliejų ar bet kokį valgį, ar tas maistas tampa šventu?’ ” Kunigai atsakė: “Ne”. 
\par 13 Agėjas vėl klausė: “Jei kas susitepęs lavonu, paliečia ką nors iš tų dalykų, ar tai tampa sutepta?” Kunigai atsakė: “Taip”. 
\par 14 Tada Agėjas tarė: “Tokia yra ši tauta ir šie žmonės mano akivaizdoje,­sako Viešpats.­Ir tokie yra jų visi darbai, ir tai, ką jie aukoja, yra sutepta. 
\par 15 Dabar stebėkite, kas vyko iki šios dienos, kai dar nebuvo padėtas akmuo Viešpaties šventykloje. 
\par 16 Jūs ateidavote prie dvidešimties saikų javų krūvos, o atrasdavote tik dešimt. Priėję prie spaustuvo pasisemti penkiasdešimt saikų, terasdavote dvidešimt. 
\par 17 Aš naikinau jūsų darbą sausra, pelėsiais ir kruša, bet jūs negrįžote pas mane,­sako Viešpats.­ 
\par 18 Dabar stebėkite nuo šios dienos, nuo devinto mėnesio dvidešimt ketvirtos dienos, kai buvo padėtas Viešpaties šventyklos pamatas. 
\par 19 Ar dar yra sėklos aruode? Vynmedis, figmedis, granatmedis ir alyvmedis dar neneša vaisiaus. Nuo šios dienos Aš jus laiminsiu!” 
\par 20 Viešpats vėl kalbėjo Agėjui dvidešimt ketvirtą mėnesio dieną: 
\par 21 “Sakyk Zorobabeliui, Judo valdytojui: ‘Aš supurtysiu dangų ir žemę, 
\par 22 nuversiu karalysčių sostus ir sunaikinsiu pagonių karalysčių galybę. Aš apversiu kovos vežimus ir tuos, kurie juose sėdi, žirgai ir raiteliai kris nuo kardo. 
\par 23 Tą dieną paimsiu tave, Zorobabeli, Salatielio sūnau, mano tarne, ir padarysiu tave antspaudo žiedu, nes Aš tave išsirinkau,­sako kareivijų Viešpats’ ”.



\end{document}