\begin{document}

\title{Leviticus}

\chapter{1}


\par 1 Viešpats, pasišaukęs Mozę, jam kalbėjo iš Susitikimo palapinės: 
\par 2 “Sakyk izraelitams: ‘Jei kas iš jūsų norės aukoti Viešpačiui, teaukoja iš galvijų ir avių. 
\par 3 Jei jo deginamoji auka bus iš galvijų bandos, teatveda sveiką patiną prie Susitikimo palapinės įėjimo laisva valia Viešpaties akivaizdon, 
\par 4 teuždeda ranką ant aukojamo gyvulio galvos, kad auka būtų priimta ir jis būtų sutaikintas. 
\par 5 Jis papjaus veršį Viešpaties akivaizdoje, o Aarono sūnūs, kunigai, šlakstys kraują aplink aukurą, esantį ties Susitikimo palapinės įėjimu. 
\par 6 Po to, nulupęs aukai odą, sukapos ją į gabalus. 
\par 7 Aarono sūnūs, kunigai, sukraus malkas ant aukuro ir, užkūrę ugnį, 
\par 8 ant viršaus dės sukapotas aukos dalis, galvą bei taukus, 
\par 9 o vidurius ir kojas nuplaus vandeniu. Kunigas visa tai sudegins ant aukuro; tai yra deginamoji auka­malonus kvapas Viešpačiui. 
\par 10 Jei auka būtų iš smulkesnių gyvulių­avių arba ožių,­jis aukos sveiką patiną. 
\par 11 Papjaus jį aukuro šiauriniame šone Viešpaties akivaizdoje; jo kraują Aarono sūnūs šlakstys ant aukuro ir aplink jį. 
\par 12 Sukapotą gyvulį, galvą ir taukus kunigas uždės ant malkų, kurios degs ant aukuro. 
\par 13 O vidurius ir kojas nuplaus vandeniu, ir kunigas sudegins visa tai ant aukuro. Tai yra deginamoji auka­malonus kvapas Viešpačiui. 
\par 14 Jei deginamoji auka Viešpačiui aukojama iš paukščių, tegu ji bus iš laukinių ir jaunų balandžių. 
\par 15 Kunigas nuneš šią auką prie aukuro, nusuks jai galvą ir sudegins ant aukuro, o kraują išvarvins ant aukuro briaunos. 
\par 16 Gurklį ir plunksnas numes rytinėje aukuro pusėje, kur paprastai pilami pelenai, 
\par 17 sparnus sulaužys, bet neatskirs, ir sudegins ją ant aukure degančių malkų. Tai yra deginamoji auka­malonus kvapas Viešpačiui’ ”.


\chapter{2}


\par 1 “Jei kas nori aukoti Viešpačiui duonos auką, tegu ji bus iš smulkių miltų, apipiltų aliejumi, ir smilkalų. 
\par 2 Ją atneš Aarono sūnums, kunigams, kurių vienas ims pilną saują smulkių miltų su aliejumi ir visus smilkalus ir kaip atminimą sudegins ant aukuro, kad būtų malonus kvapas Viešpačiui. 
\par 3 Likusi aukos dalis priklausys Aaronui ir jo sūnums kaip švenčiausia aukų Viešpačiui dalis. 
\par 4 Duonos auka, kepta krosnyje, turi būti nerauginti papločiai iš smulkių miltų, sumaišytų su aliejumi, ir nerauginti ragaišiai, aptepti aliejumi. 
\par 5 O jei tavo duonos auka būtų kepta ant skardos, ji turi būti iš smulkių miltų, sumaišytų su aliejumi, be raugo; 
\par 6 ją sulaužysi į gabaliukus ir užpilsi aliejumi: tai yra duonos auka. 
\par 7 O jei duonos auka būtų kepta keptuvėje, ji turi būti smulkių miltų ir sumaišyta su aliejumi. 
\par 8 Aukodamas ją Viešpačiui, paduosi kunigui į rankas. 
\par 9 Dalį aukos jis sudegins atminimui ant aukuro, kad būtų malonus kvapas Viešpačiui, 
\par 10 o kas liks, bus Aarono ir jo sūnų švenčiausia dalis iš aukos Viešpačiui. 
\par 11 Kiekviena duonos auka, aukojama Viešpačiui, turi būti be raugo; raugas ir medus neturi būti deginama kaip auka Viešpačiui. 
\par 12 Jei aukosi pirmųjų vaisių dovanas Viešpačiui, jos nebus dedamos ant aukuro kaip malonus kvapas. 
\par 13 Visas duonos aukas pasūdysi druska; tavo Dievo sandoros druskos neturi pritrūkti duonos aukose. Druską aukosi su kiekviena auka. 
\par 14 Jei aukotum Viešpačiui duonos auką iš pirmųjų javų, dar tebežaliuojančių varpų, sudžiovinsi jas, smulkiai sugrūsi, 
\par 15 užpilsi ant viršaus aliejaus ir pridėsi smilkalų. Tai yra duonos auka. 
\par 16 Kunigas dalį sugrūstų grūdų su aliejumi ir visus smilkalus sudegins: tai bus auka Viešpačiui”.



\chapter{3}


\par 1 “Jei kas aukotų padėkos auką iš galvijų, teaukoja Viešpačiui sveiką patiną ar patelę. 
\par 2 Jis turi uždėti ranką aukai ant galvos ir papjauti prie Susitikimo palapinės įėjimo, o Aarono sūnūs, kunigai, šlakstys kraują apie aukurą. 
\par 3 Kaip padėkos auką jis aukos Viešpačiui visus taukus, dengiančius vidurius, 
\par 4 abu inkstus su taukais, paslėpsnių taukus ir kepenų tinklelį. 
\par 5 Aarono sūnūs tai sudegins ant aukuro kaip deginamąją auką, kad auka Viešpačiui būtų malonus kvapas. 
\par 6 Jei kas aukotų padėkos auką iš avių, turi atvesti sveiką patiną ar patelę. 
\par 7 Jei kas aukotų Viešpaties akivaizdoje ėriuką, 
\par 8 uždės ranką ant galvos savo aukai, papjaus ją prieš Susitikimo palapinę, o Aarono sūnūs šlakstys jos kraują aplinkui aukurą. 
\par 9 Padėkos auka Viešpačiui bus vidurių taukai, uodega, 
\par 10 inkstai, ant jų esantys taukai, paslėpsnių taukai ir kepenų tinklelis. 
\par 11 Kunigas tai sudegins ant aukuro kaip auką Viešpačiui. 
\par 12 Jei jis aukotų ožką, atves ją Viešpaties akivaizdon, 
\par 13 uždės ranką ant galvos ir papjaus prie Susitikimo palapinės įėjimo; Aarono sūnūs šlakstys jos kraują aplink aukurą. 
\par 14 Viešpaties aukai ims jos vidurių taukus, 
\par 15 abu inkstus su taukais, paslėpsnių taukus ir kepenų tinklelį. 
\par 16 Kunigas visa tai sudegins ant aukuro, kad būtų malonus kvapas Viešpačiui. 
\par 17 Tai amžinas įstatas jūsų kartoms kur jūs begyventumėte: jūs nevalgysite nei kraujo, nei taukų”.



\chapter{4}

\par 1 Viešpats kalbėjo Mozei: 
\par 2 “Sakyk izraelitams: ‘Jei kas dėl nežinojimo sulaužytų Dievo įsakymą ir nusikalstų, padarydamas, kas uždrausta, 
\par 3 tai, jei taip nusidėtų pateptasis kunigas ir užtrauktų kaltę tautai, jis aukos už savo nuodėmę Viešpačiui sveiką veršį. 
\par 4 Jis atves jį prie Susitikimo palapinės durų Viešpaties akivaizdon, uždės ranką ant galvos ir jį papjaus. 
\par 5 Pateptasis kunigas, pasiėmęs veršio kraują, įneš Susitikimo palapinėn. 
\par 6 Padažęs pirštą kraujyje, juo pašlakstys septynis kartus Viešpaties akivaizdoje uždangą prieš Švenčiausiąją 
\par 7 ir tuo pačiu krauju pateps ragus Viešpaties smilkalų aukuro, kuris yra Susitikimo palapinėje; visą likusį kraują išlies deginamųjų aukų aukuro papėdėje prie palapinės. 
\par 8 Už nuodėmę jis aukos veršio vidurių taukus, 
\par 9 abu inkstus, ant jų ir prie paslėpsnių esančius taukus ir kepenų tinklelį. 
\par 10 Visa tai kunigas sudegins ant deginamųjų aukų aukuro. 
\par 11 Odą, visą mėsą su galva ir kojomis, vidurius ir mėšlą 
\par 12 išgabens už stovyklos į nesuteptą vietą, kur paprastai išpilami aukų pelenai, padės ant malkų krūvos ir sudegins toje vietoje, kur išpilami pelenai. 
\par 13 Jei Izraelio tauta dėl nežinojimo nusikalstų Dievui ir nesąmoningai sulaužytų Viešpaties įsakymą, darydama tai, ko neturėtų daryti, 
\par 14 bet vėliau suprastų savo kaltę, aukos už savo nusidėjimą veršį. Jį atves prie Susitikimo palapinės įėjimo. 
\par 15 Tautos vyresnieji uždės rankas ant veršio galvos Viešpaties akivaizdoje ir papjaus veršį Viešpaties akivaizdoje. 
\par 16 Pateptasis kunigas įneš jo kraują Susitikimo palapinėn 
\par 17 ir, padažęs pirštą, septynis kartus šlakstys Viešpaties akivaizdoje prieš uždangą; 
\par 18 ir tuo pačiu krauju pateps Viešpaties akivaizdoje Susitikimo palapinėje esančio aukuro ragus. Likusį kraują išlies deginamųjų aukų aukuro, kuris yra prie Susitikimo palapinės įėjimo, papėdėje. 
\par 19 Veršio taukus sudegins ant aukuro; 
\par 20 padarys visa tai, kas daroma aukojant veršį aukai už nuodėmę. Kunigas juos sutaikys, ir Viešpats jiems atleis. 
\par 21 Patį veršį išgabens už stovyklos ir sudegins kaip anksčiau minėtąjį veršį, nes jis yra auka už tautos nuodėmę. 
\par 22 Jei kunigaikštis nežinodamas sulaužytų kokį nors Viešpaties įstatymą, 
\par 23 bet paskui suprastų savo nuodėmę, jis aukos Viešpačiui sveiką ožį kaip auką už nuodėmę. 
\par 24 Jis uždės ranką ant jo galvos ir jį papjaus toje vietoje, kur paprastai aukojamos deginamosios aukos Viešpaties akivaizdoje, nes tai yra auka už nuodėmę. 
\par 25 Kunigas padažys pirštą aukos kraujyje ir juo pateps deginamųjų aukų aukuro ragus. Likusį kraują išlies jo papėdėje. 
\par 26 Taukus sudegins ant aukuro, kaip tai daroma su padėkos aukomis. Kunigas įvykdys sutaikinimą už jį dėl jo nuodėmės, ir jam bus atleista. 
\par 27 Jei kas iš eilinių žmonių nusidėtų nežinodamas ir padarytų, kas uždrausta Viešpaties įstatymu, 
\par 28 o vėliau suprastų savo nuodėmę, jis aukos sveiką ožką. 
\par 29 Jis uždės ranką ant gyvulio galvos, jį papjaus ir aukos deginamųjų aukų vietoje. 
\par 30 Kunigas, padažęs pirštą aukos kraujyje, juo pateps deginamųjų aukų aukuro ragus, o likusį kraują išlies aukuro papėdėje. 
\par 31 Po to, paėmęs visus taukus, sudegins juos ant aukuro, kaip tai daroma su padėkos aukomis, kad būtų malonus kvapas Viešpačiui. Kunigas sutaikins jį, ir jam bus atleista. 
\par 32 O jei auka už nuodėmę bus iš avių, tai ji turi būti sveika. 
\par 33 Jis uždės ranką ant jos galvos ir papjaus ją toje vietoje, kur paprastai pjaunamos deginamosios aukos. 
\par 34 Kunigas pateps deginamųjų aukų aukuro ragus savo pirštu, padažytu aukos kraujyje. Likusį kraują išlies aukuro papėdėje. 
\par 35 Visus taukus, kaip tai daroma su padėkos auka, sudegins ant aukuro Viešpačiui; taip sutaikins jį su Viešpačiu, ir jo nuodėmė bus jam atleista”.



\chapter{5}

\par 1 “Jei kas girdi keikimą, bet atsisako liudyti, ką yra matęs ir žino, jis bus kaltas. 
\par 2 Kas paliestų kokį nešvarų daiktą: nešvaraus žvėries, gyvulio ar roplio dvėseną, tačiau to nežinotų, jis bus nešvarus ir kaltas. 
\par 3 Jei kas nežinodamas prisiliestų prie žmogaus kūno nešvarumų, kokie jie bebūtų, kai jis sužinos apie tai, jis bus kaltas. 
\par 4 Jei kas nors prisiekia, neapgalvotai savo lūpomis pasižadėdamas padaryti pikta ar gera,­kas tai bebūtų, ką žmogus patvirtina priesaika,­ir tai liktų paslėpta nuo jo, kai jis tai supras, jis bus kaltas. 
\par 5 Jei žmogus kaltas kuo nors iš šių dalykų, jis turi išpažinti, kad nusidėjo, 
\par 6 ir aukoti auką už kaltę Viešpačiui, už nuodėmę, kuria nusidėjo,­jauną avį ar ožką,­ir kunigas sutaikins jį. 
\par 7 O jei kas negali aukoti avies ar ožkos, teaukoja Viešpačiui du balandžius ar du jaunus karvelius: vieną aukai už nuodėmę, o antrą­deginamajai aukai. 
\par 8 Jis teatneša juos kunigui, kuris, aukodamas pirmąjį už nuodėmę, nusuks jam galvą, tačiau jos visiškai nenutrauks; 
\par 9 jo krauju apšlakstys aukuro šoną, o likusį kraują išvarvins aukuro papėdėje, nes tai auka už nuodėmę. 
\par 10 Antrą sudegins kaip deginamąją auką, kaip paprastai daroma. Kunigas sutaikins jį, ir jo nuodėmė bus jam atleista. 
\par 11 O jei jis negalės duoti dviejų balandžių ar dviejų jaunų karvelių, teaukoja už savo nuodėmę dešimtą dalį efos miltų; tegu nepila į juos aliejaus ir neprideda smilkalų, nes tai auka už nuodėmę. 
\par 12 Jis teatneša juos kunigui, kuris, paėmęs jų pilną saują, sudegins ant aukuro atminimui. 
\par 13 Kunigas įvykdys sutaikinimą už nuodėmę, kuria jis nusidėjo, ir jam bus atleista. Likusią aukos dalį kunigas pasiims kaip duonos auką”. 
\par 14 Ir Viešpats kalbėjo Mozei: 
\par 15 “Jei kas apsirikęs paimtų, kas Viešpačiui paskirta, ir tuo nusidėtų, aukos už savo kaltę sveiką aviną, kokį galima nupirkti už du šekelius pagal šventyklos šekelį. 
\par 16 Jis atlygins padarytą nuostolį ir pridės penktąją dalį viršaus kunigui, kuris sutaikins jį aukodamas aviną, ir jam bus atleista. 
\par 17 Jei kas nusikalstų, nesąmoningai sulaužydamas Viešpaties įstatymą, ir, padaręs nuodėmę, suprastų savo kaltę, 
\par 18 tegu jis ima iš savo bandos sveiką aviną aukai už kaltę ir atiduoda jį kunigui. Ir kunigas sutaikins jį, nes jis nusikalto nežinodamas, ir jam bus atleista. 
\par 19 Tai yra auka už kaltę, kuria jis nusikalto Viešpačiui”.



\chapter{6}

\par 1 Viešpats kalbėjo Mozei: 
\par 2 “Kas nusidėtų Viešpačiui, nenorėdamas sugrąžinti savo artimui to, ką tas buvo jam patikėjęs ar paskolinęs, arba ką nors prievarta atimtų, arba apgautų savo artimą, 
\par 3 arba, radęs pamestą daiktą, išsigintų, arba melagingai prisiektų, arba padarytų kitą nuodėmę, 
\par 4 kadangi jis nusidėjo ir yra kaltas, turi grąžinti tai, ką atėmė ar apgaule įsigijo, ar buvo gavęs pasaugoti, ar surado, 
\par 5 ar melagingai prisiekė. Jis turi atlyginti ir pridėti penktąją dalį savininkui tą dieną, kai aukoja auką už kaltę. 
\par 6 Ir jis aukos auką Viešpačiui už kaltę, aviną iš bandos, pagal nusikaltimo dydį: tegul atveda jį kunigui, 
\par 7 kuris sutaikins jį su Viešpačiu, ir jam bus atleistas jo nusikaltimas”. 
\par 8 Ir Viešpats kalbėjo Mozei, sakydamas: 
\par 9 “Paskelbk Aaronui ir jo sūnums įstatymą apie deginamąją auką. Deginamoji auka turi pasilikti ant aukuro visą naktį iki ryto­ir aukuro ugnis turi degti. 
\par 10 Kunigas, apsivilkęs drobine jupa ir drobinėmis kelnėmis, išims sudegintos aukos pelenus ir juos supils šalia aukuro. 
\par 11 Po to persirengęs išneš ir išpils pelenus už stovyklos švarioje vietoje. 
\par 12 Ugnis ant aukuro degs visada; ją prižiūrės kunigas, pridėdamas kas rytą malkų, kad ugnis neužgestų. Pirmiausia aukos deginamąją auką, o po to degins padėkos aukų taukus. 
\par 13 Ugnis ant aukuro neužges niekados. 
\par 14 Šis yra duonos aukos įstatymas. Ją aukos Aarono sūnūs Viešpaties akivaizdoje. 
\par 15 Kunigas ims saują smulkių miltų, apšlakstytų aliejumi, bei smilkalus ir sudegins ant aukuro kaip malonų kvapą, kaip atminimą Viešpačiui. 
\par 16 Likusią dalį miltų suvalgys Aaronas su savo sūnumis. Jie tai valgys neraugintą šventoje vietoje, palapinės kieme. 
\par 17 Duona bus nerauginta, nes dalis jos buvo aukojama kaip smilkalai Viešpačiui. Ji yra šventa, kaip ir auka už nuodėmę bei auka už kaltę. 
\par 18 Tik Aarono giminės vyrai ją valgys. Tai amžinas įstatymas jūsų kartoms apie aukas Viešpačiui. Kas prie jų prisilies, tas bus šventas”. 
\par 19 Ir Viešpats kalbėjo Mozei: 
\par 20 “Tai turi aukoti Aaronas ir jo sūnūs Viešpačiui jų patepimo dieną. Dešimtą dalį efos smulkių miltų jie aukos kaip duonos auką­pusę rytą ir pusę vakare. 
\par 21 Juos sumaišys su aliejumi ir iškeps skardoje. Tai duonos auka. Ją aukos, kad būtų malonus kvapas Viešpačiui. 
\par 22 Kunigas, kuris bus pateptas Aarono vietoje, visa sudegins ant aukuro. 
\par 23 Visa duonos auka, kurią aukoja kunigas, bus sudeginta­niekas jos nevalgys”. 
\par 24 Ir Viešpats kalbėjo Mozei: 
\par 25 “Paskelbk Aaronui ir jo sūnums įstatymą apie auką už nuodėmę. Auka turi būti papjauta Viešpaties akivaizdoje toje vietoje, kur pjaunama deginamoji auka; ji yra labai šventa. 
\par 26 Kunigas, kuris ją aukoja, valgys ją šventoje vietoje, palapinės kieme. 
\par 27 Kas palies tos aukos mėsą, bus šventas. Jei aukos krauju būtų aptaškytas apdaras, jis turi būti plaunamas šventoje vietoje. 
\par 28 Jei auką virsi moliniame inde, jį sudaužysi, o jei variniame­jį išvalysi ir vandeniu išplausi. 
\par 29 Kiekvienas kunigų giminės vyras valgys tą mėsą. Ji yra labai šventa. 
\par 30 Tačiau auka už nuodėmę, kurios kraujo dalis įnešama į Susitikimo palapinę sutaikinimui, nebus valgoma; ji bus visa sudeginta”.



\chapter{7}

\par 1 “Štai aukos už kaltę įstatymas. Ji yra labai šventa. 
\par 2 Kur pjaunama deginamoji auka, ten pjaunama ir auka už kaltę. Jos kraujas bus šlakstomas aplink aukurą. 
\par 3 Aukosite nuo jos visus taukus: uodegą, vidurių taukus, 
\par 4 abu inkstus ir ant jų esančius taukus, paslėpsnių taukus ir kepenis su tinkleliu. 
\par 5 Kunigas visa tai sudegins ant aukuro; tai auka Viešpačiui už kaltę. 
\par 6 Kiekvienas kunigų giminės vyras valgys tos mėsos šventoje vietoje, nes ji yra labai šventa. 
\par 7 Auka už kaltę aukojama taip pat, kaip ir auka už nuodėmę; joms yra vienas įstatymas. Kunigas, kuris atlieka sutaikinimą, gauna ją. 
\par 8 Kunigas, kuris aukoja bet kokio žmogaus deginamąją auką, pasiima sau tos aukos odą. 
\par 9 Kiekviena duonos auka, iškepta krosnyje, ant skardos ar keptuvėje, priklauso kunigui, kuris ją aukoja; 
\par 10 Kiekviena duonos auka, aplieta aliejumi ar sausa, priklauso visiems Aarono sūnums, kaip vienam, taip ir kitam. 
\par 11 Šis yra įstatymas padėkos aukų, kurias aukosite Viešpačiui. 
\par 12 Jei kas, norėdamas atsidėkoti, aukoja dėkojimo auką, kartu su ja teaukoja neraugintus papločius, aplaistytus aliejumi, neraugintus ragaišius, apteptus aliejumi, ir smulkių miltų papločius, aplaistytus aliejumi. 
\par 13 Kartu su papločiais teaukoja raugintą duoną kaip dėkojimo auką. 
\par 14 Vieną iš visos aukos skirs pakėlimo aukai Viešpačiui, ir tai priklausys kunigui, kuris šlakstys padėkos aukos kraują. 
\par 15 Dėkojimo aukos mėsa turi būti suvalgyta tą pačią dieną­nieko iš jos neleistina palikti rytojui. 
\par 16 Įžadų ir laisvos valios auka valgoma tą pačią dieną. Kas lieka, leidžiama valgyti ir kitą dieną. 
\par 17 Kas lieka trečiai dienai­sudeginama. 
\par 18 Jei kas valgytų padėkos aukos mėsą trečią dieną, ji nebūtų priimtina ir nebūtų įskaityta tam, kuris aukoja,­tai pasibjaurėjimas. Kas valgytų, būtų kaltas. 
\par 19 Mėsa, paliesta kuo nors suteptu, nebus valgoma­ji bus sudeginama. Kas nesuteptas, gali valgyti aukos mėsą. 
\par 20 Jei kas, būdamas nešvarus, valgytų padėkos aukos mėsą, aukojamą Viešpačiui, bus išnaikintas iš savo tautos. 
\par 21 Kas, palietęs žmogaus, galvijo ar kokio daikto nešvarumą, valgytų padėkos aukos mėsą, bus išnaikintas iš savo tautos”. 
\par 22 Ir Viešpats kalbėjo Mozei: 
\par 23 “Sakyk izraelitams, kad jie nevalgytų avies, jaučio ir ožkos taukų. 
\par 24 Nustipusio ar žvėries sudraskyto gyvulio taukus naudokite įvairiems reikalams, bet ne valgiui. 
\par 25 Jei kas valgytų gyvulio taukus, kurie turi būti paaukojami Viešpačiui, bus išnaikintas iš savo tautos. 
\par 26 Nevalgykite jokio kraujo­ar jis būtų paukščių, ar gyvulių. 
\par 27 Kiekvienas, kuris valgys kraują, bus išnaikintas iš savo tautos”. 
\par 28 Ir Viešpats kalbėjo Mozei: 
\par 29 “Sakyk Izraelio vaikams: ‘Kas aukoja Viešpačiui padėkos auką, dalį aukos turi atnešti Viešpačiui. 
\par 30 Savo rankomis turi atnešti aukos dalį, skirtą sudeginimui. Taukus ir krūtinę jis privalo atnešti, kad krūtinė būtų siūbuojama Viešpaties akivaizdoje. 
\par 31 Kunigas sudegins taukus ant aukuro, krūtinė teks Aaronui ir jo sūnums. 
\par 32 Dešinysis padėkos aukos petys yra kunigo dalis. 
\par 33 Kuris Aarono sūnus aukos kraują ir taukus, tas gaus dešinį petį. 
\par 34 Nes padėkos aukų krūtinę ir petį paėmiau iš Izraelio vaikų ir atidaviau kunigui Aaronui ir jo sūnums. Tai amžinas nuostatas Izraelio tautai’ ”. 
\par 35 Tai yra pateptojo Aarono ir jo pateptųjų sūnų dalis iš Viešpačiui sudeginamų aukų nuo tos dienos, kai jie buvo paskirti eiti kunigų tarnystę. 
\par 36 Dalis, kurią Viešpats Izraelio vaikams įsakė atiduoti jiems per visas kartas nuo tos dienos, kai jie buvo patepti. 
\par 37 Toks yra deginamųjų, duonos, nuodėmės, kaltės, įšventinimo ir padėkos aukų įstatymas. 
\par 38 Jį Viešpats davė Mozei Sinajaus kalne, kai liepė izraelitams aukoti aukas Viešpačiui Sinajaus dykumoje.



\chapter{8}


\par 1 Viešpats kalbėjo Mozei: 
\par 2 “Imk Aaroną ir jo sūnus, jų apdarus, patepimo aliejaus, veršį aukai už nuodėmę, du avinus ir pintinę neraugintos duonos 
\par 3 ir sušauk izraelitų susirinkimą prie Susitikimo palapinės įėjimo”. 
\par 4 Mozė padarė, kaip Viešpats liepė. Sušaukęs izraelitus prie palapinės įėjimo, 
\par 5 kalbėjo jiems: “Štai ką Viešpats įsakė padaryti”. 
\par 6 Mozė, pašaukęs Aaroną bei jo sūnus, apiplovė juos vandeniu, 
\par 7 apvilko Aaroną drobine jupa, apjuosė juosta, aprengė mėlyna tunika, uždėjo efodą ir sujuosė jį efodo juosta, kad efodas laikytųsi ant jo, 
\par 8 uždėjo ant efodo krūtinės skydelį, į kurį įdėjo Urimą ir Tumimą. 
\par 9 Galvą pridengė mitra ir ant jos, ties kakta, pritaisė auksinę plokštelę, šventą vainiką, kaip Viešpats įsakė Mozei. 
\par 10 Po to Mozė ėmė patepimo aliejaus, juo patepė palapinę ir visus jos daiktus. 
\par 11 Apšlakstė aukurą aliejumi septynis kartus ir patepė visus jo reikmenis; be to, pašventino ir patepė aliejumi praustuvę ir jos stovą. 
\par 12 Ir užpylė Aaronui ant galvos patepimo aliejaus, kad pateptų ir pašventintų jį. 
\par 13 Atvedęs jo sūnus, apvilko drobinėmis jupomis, sujuosė juostomis ir uždėjo gobtuvus, kaip Viešpats įsakė Mozei. 
\par 14 Tada atvedė veršį aukai už nuodėmę. Aaronas ir jo sūnūs uždėjo aukai ant galvos rankas. 
\par 15 Mozė papjovė jį, paėmęs jo kraujo, pamirkė pirštą ir patepė aukuro ragus; apvalęs ir pašventinęs aukurą, išpylė likusį kraują jo papėdėje; 
\par 16 vidurių taukus, kepenų tinklelį ir abu inkstus su jų taukais sudegino ant aukuro, 
\par 17 o veršio odą, mėsą ir mėšlus sudegino už stovyklos, kaip Viešpats įsakė Mozei. 
\par 18 Mozė atvedė aviną deginamajai aukai. Aaronas ir jo sūnūs uždėjo jam ant galvos rankas. 
\par 19 Mozė jį papjovė ir, jo krauju apšlakstęs aukurą, 
\par 20 sukapojo aviną į gabalus; jo galvą, gabalus, taukus, 
\par 21 nuplautus vidurius ir kojas sudegino. Taip visas avinas buvo sudegintas, nes tai buvo deginamoji malonaus kvapo auka Viešpačiui, kaip Viešpats įsakė Mozei. 
\par 22 Paskui aukojo antrą aviną, įšventinimo aviną. Aaronas ir jo sūnūs uždėjo ant jo galvos rankas, 
\par 23 o Mozė jį papjovė. Jis jo krauju patepė dešinę Aarono ausį, dešinės rankos ir dešinės kojos nykštį. 
\par 24 Tada jis liepė prieiti ir Aarono sūnums. Patepęs papjautojo avino krauju kiekvieno jų dešinę ausį, dešinės rankos ir kojos nykščius, likusį kraują šlakstė aplink aukurą. 
\par 25 Taukus, uodegą, abu inkstus su taukais ir vidurių taukus, kepenų tinklelį ir dešinį petį atskyrė. 
\par 26 Iš neraugintos duonos pintinės, pastatytos Viešpaties akivaizdoje, ėmė neraugintos duonos, aliejumi apšlakstytą paplotį bei ragaišį, padėjo ant taukų ir avino dešiniojo peties 
\par 27 ir viską padavė Aaronui ir jo sūnums; jie visa tai, pasiūbavę Viešpaties akivaizdoje, 
\par 28 grąžino Mozei, kuris tai sudegino ant deginamųjų aukų aukuro. Tai buvo įšventinimo auka, kad būtų malonus aukos kvapas Viešpačiui. 
\par 29 Mozė, pasiūbavęs įšventinimo avino krūtinę Viešpaties akivaizdoje, pasiėmė ją kaip jam Viešpaties skirtą aukos dalį. 
\par 30 Mozė, paėmęs patepimo aliejaus ir buvusio ant aukuro kraujo, šlakstė Aarono ir jo sūnų apdarus, taip pašventindamas Aaroną, jo drabužius, jo sūnus ir sūnų drabužius. 
\par 31 Ir Mozė tarė Aaronui ir jo sūnums: “Virkite mėsą prie Susitikimo palapinės įėjimo ir valgykite ją su duona, kuri yra įšventinimo pintinėje, kaip Viešpats man įsakė. 
\par 32 Tai, kas liks nesuvalgyta,­sudeginkite. 
\par 33 Nuo palapinės įėjimo nesitraukite septynias dienas, kol pasibaigs jūsų pašventinimo laikas, nes jūs turite būti šventinami septynias dienas. 
\par 34 Tai, kas buvo padaryta šiandien, Viešpats įsakė atlikti jūsų sutaikinimui. 
\par 35 Todėl septynias dienas ir septynias naktis pasilikite prie palapinės įėjimo atlikti Viešpaties sargybą, kad nemirtumėte, nes taip man įsakė Viešpats”. 
\par 36 Aaronas ir jo sūnūs darė visa, ką Viešpats įsakė Mozei.



\chapter{9}


\par 1 Aštuntą dieną Mozė pasišaukė Aaroną, jo sūnus bei Izraelio vyresniuosius 
\par 2 ir tarė Aaronui: “Imk iš bandos veršį aukai už nuodėmę ir aviną deginamajai aukai, abu sveikus, ir juos aukok Viešpačiui. 
\par 3 O Izraelio vaikams sakyk: ‘Imkite aukai už nuodėmę ožį, metinį veršį bei avinėlį deginamajai aukai, 
\par 4 padėkos aukai jautį ir aviną, kad aukotumėte juos Viešpaties akivaizdoje, ir duonos auką, sumaišytą su aliejumi, nes šiandien jums pasirodys Viešpats’ ”. 
\par 5 Jie atgabeno prie Susitikimo palapinės visa, ką Mozė liepė. Ir visa tauta priartėjo ir atsistojo prieš Viešpatį. 
\par 6 Mozė tarė: “Tai padaryti įsakė Viešpats, ir Viešpaties šlovė pasirodys jums”. 
\par 7 Tada Mozė sakė Aaronui: “Eik prie aukuro ir aukok savo auką už nuodėmę ir deginamąją auką, ir atlik sutaikinimą už save, ir aukok tautos auką, ir sutaikink ją, kaip Viešpats įsakė”. 
\par 8 Aaronas priėjo prie aukuro ir papjovė veršį aukai už savo nuodėmę. 
\par 9 Jo sūnūs padavė jam aukos kraujo, o jis, pamirkęs jame pirštą, patepė aukuro ragus ir likusį kraują išliejo prie jo papėdės. 
\par 10 Aukos už nuodėmę taukus, inkstus ir kepenų tinklelį sudegino ant aukuro, kaip Viešpats įsakė Mozei, 
\par 11 o mėsą ir odą sudegino už stovyklos. 
\par 12 Po to papjovė deginamąją auką. Jo sūnūs padavė jam jos kraują, kurį jis šlakstė aplink aukurą. 
\par 13 Padavė jam taip pat sukapotą į gabalus pačią auką ir jos galvą. Visa tai jis sudegino ant aukuro. 
\par 14 Apiplovęs vandeniu vidurius ir kojas, taip pat sudegino ant aukuro. 
\par 15 Po to jis aukojo tautos auką. Paėmęs ožį, aukojo jį už tautos nuodėmę, ir papjovė jį, kaip ir pirmąją auką už nuodėmę. 
\par 16 Ir jis aukojo deginamąją auką, kaip įsakyta. 
\par 17 Ir atnešė duonos auką, ir, paėmęs jos saują, sudegino ant aukuro, neskaitant rytinės deginamosios aukos. 
\par 18 Papjovė taip pat jautį ir aviną kaip tautos padėkos auką. Aarono sūnūs padavė jam kraują, kurį jis šlakstė aplink aukurą. 
\par 19 Jaučio ir avino uodegą, inkstus su taukais bei kepenų tinklelį 
\par 20 padėjo ant krūtinių ir sudegino taukus ant aukuro, 
\par 21 o krūtines ir dešinius pečius Aaronas paėmęs siūbavo Viešpaties akivaizdoje, kaip Mozė buvo įsakęs. 
\par 22 Tada Aaronas, ištiesęs ranką į tautą, palaimino ją. Baigęs aukoti aukas už nuodėmes, deginamąsias ir padėkos aukas, pasitraukė nuo aukuro. 
\par 23 Po to Mozė ir Aaronas įėjo Susitikimo palapinėn. Išėję palaimino tautą. Viešpaties šlovė pasirodė visiems žmonėms. 
\par 24 Atėjusi nuo Viešpaties ugnis prarijo deginamąją auką ir buvusius ant aukuro taukus. Tai išvydusi, minia šaukė ir parpuolė savo veidais į žemę.



\chapter{10}

\par 1 Aarono sūnūs Nadabas ir Abihuvas ėmė smilkytuvus, įsidėjo ugnies bei smilkalų ir aukojo Viešpačiui svetimą ugnį, ko Jis jiems nebuvo įsakęs. 
\par 2 Tada išsiveržusi iš Viešpaties ugnis juodu prarijo; jie mirė Viešpaties akivaizdoje. 
\par 3 Mozė tarė Aaronui: “Taip kalbėjo Viešpats: ‘Pasirodysiu šventas tuose, kurie prie manęs artinasi, ir būsiu pašlovintas visos tautos akivaizdoje’ ”. Tai girdėdamas, Aaronas tylėjo. 
\par 4 Mozė, pasišaukęs Aarono dėdės Uzielio sūnus Mišaelį ir Elcafaną, jiems tarė: “Eikite, paimkite jūsų brolius iš šventyklos ir išneškite juos už stovyklos”. 
\par 5 Juodu priėję paėmė juos, apvilktus drobinėmis jupomis, ir išnešė laukan, kaip jiems buvo liepta. 
\par 6 Mozė sakė Aaronui ir jo sūnums Eleazarui ir Itamarui: “Nenudenkite savo galvų ir neperplėškite drabužių, kad kartais nemirtumėte ir bausmė nekristų visiems izraelitams. Jūsų broliai ir visi izraelitai teaprauda sudeginimą, kurį Viešpats siuntė. 
\par 7 Jūs nesitraukite nuo palapinės, kad nežūtumėte, kadangi esate patepti šventu aliejumi”. Jie darė visa, ką Mozė įsakė. 
\par 8 Po to Viešpats tarė Aaronui: 
\par 9 “Tu ir tavo sūnūs vyno ir stipraus gėrimo negerkite, eidami Susitikimo palapinėn, kad nemirtumėte; tai yra amžinas įsakymas visoms jūsų kartoms, 
\par 10 kad skirtumėte, kas šventa ir nešventa, kas švaru ir kas nešvaru, 
\par 11 ir galėtumėte mokyti izraelitus visų mano įstatymų, kuriuos daviau per Mozę”. 
\par 12 Mozė kalbėjo Aaronui ir likusiems sūnums Eleazarui bei Itamarui: “Imkite aukos dalį, kuri lieka iš duonos aukos Viešpačiui, ir ją neraugintą valgykite prie aukuro, nes ji šventa. 
\par 13 Valgykite šventoje vietoje, kas tau ir tavo sūnums duota iš aukų Viešpačiui, nes taip man įsakyta; 
\par 14 taip pat krūtinę, kuri buvo siūbuota, ir aukos petį valgykite nesuteptoje vietoje tu, tavo sūnūs ir dukterys, nes tai duota tau ir tavo vaikams iš izraelitų padėkos aukų. 
\par 15 Petys ir krūtinė, kurie buvo atnešti su aukos taukais ir siūbuojami Viešpaties akivaizdoje, priklauso tau. Tai amžinas Viešpaties duotas įstatymas”. 
\par 16 Mozė stropiai ieškojo ožio, kuris buvo aukotas už nuodėmę, ir rado jį sudegintą. Jis, supykęs ant abiejų Aarono sūnų Eleazaro ir Itamaro, paklausė: 
\par 17 “Kodėl nesuvalgėte šventoje vietoje aukos už nuodėmę, kurią Dievas jums atidavė, kad pašalintumėte tautos kaltes ir atliktumėte sutaikinimą už juos Viešpaties akivaizdoje? 
\par 18 Aukos kraujas nebuvo įneštas į šventyklą, jūs turėjote ją valgyti šventoje vietoje, kaip man įsakyta”. 
\par 19 Aaronas atsakė: “Šiandien jie aukojo auką už nuodėmę ir deginamąją auką Viešpaties akivaizdoje, ir man taip atsitiko. Jei aš šiandien valgyčiau auką už nuodėmę, ar tai būtų priimtina Viešpačiui?” 
\par 20 Tai girdėdamas, Mozė priėmė pasiteisinimą.



\chapter{11}


\par 1 Viešpats kalbėjo Mozei ir Aaronui: 
\par 2 “Paskelbkite izraelitams: ‘Tai žemės gyvuliai, kuriuos jums leista valgyti. 
\par 3 Galite valgyti kiekvieną gyvulį, kuris turi skeltą nagą ir gromuliuoja; 
\par 4 nevalgysite ir laikysite nešvariu tą, kuris gromuliuoja, bet turi neskeltą nagą kaip kupranugaris. Jis gromuliuoja, bet turi neskeltą nagą, todėl yra nešvarus. 
\par 5 Nešvarus yra barsukas, nes jis gromuliuoja, bet turi neskeltą nagą; 
\par 6 taip pat kiškis, nors jis gromuliuoja, bet turi neskeltą nagą. 
\par 7 Ir kiaulė, nors ji turi skeltą nagą, bet negromuliuoja. 
\par 8 Tų gyvulių mėsos nevalgysite ir neliesite jų maitos. Jie jums yra nešvarūs. 
\par 9 Iš vandens gyvūnų jums leista valgyti visus, kurie turi pelekus ir žvynus, ar jie būtų jūroje, ar upėje, ar tvenkiniuose. 
\par 10 Tais, kurie kruta ir gyvena vandenyje, bet neturi pelekų ir žvynų, jūs bjaurėsitės, 
\par 11 nevalgysite jų mėsos ir nepaliesite jų maitos. 
\par 12 Visi gyviai, kurie gyvena vandenyje ir neturi pelekų ir žvynų, bus jums nešvarūs. 
\par 13 Paukščiai, kurių jūs nevalgysite, bet bjaurėsitės, yra: erelis, grifas, jūros erelis; 
\par 14 peslys ir vanagėlis su visa jo gimine; 
\par 15 visa varnų giminė; 
\par 16 strutis ir pelėda, žuvėdra, vanagas ir jo giminė; 
\par 17 apuokas, kormoranas ir ibis; 
\par 18 gulbė, pelikanas ir gervė; 
\par 19 gandras ir visa jo giminė; taip pat tutlys ir šikšnosparnis. 
\par 20 Visais sparnuotais vabzdžiais, kurie vaikščioja keturiomis kojomis, jūs bjaurėsitės. 
\par 21 Jums leista valgyti tuos keturkojus vabzdžius, kurių paskutinės kojos ilgesnės ir jie šokinėja ant žemės: 
\par 22 visa skėrių giminė ir didieji žiogai su visomis jų giminėmis. 
\par 23 Bet jūs nevalgysite kitų vabzdžių, kurie vaikščioja keturiomis kojomis. 
\par 24 Jei kas paliestų juos negyvus, bus nešvarus iki vakaro; 
\par 25 jei kam reikėtų nešti kurį nors iš jų negyvą, tas plaus savo rūbus ir bus nešvarus iki vakaro. 
\par 26 Kiekvienas gyvulys, kuris turi neskeltą nagą ir negromuliuoja, bus laikomas nešvariu­kas prisiliestų jo maitos, bus nešvarus. 
\par 27 Gyvuliai, kurie turi keturias kojas ir eina letenomis, bus jums nešvarūs­kas prisiliestų prie jų maitos, bus nešvarus iki vakaro; 
\par 28 kas neštų jų maitą, plaus savo drabužius ir bus nešvarus iki vakaro, nes tai yra jums nešvaru. 
\par 29 Iš roplių ir gyvūnų, kurie juda ant žemės, bus laikomi nešvariais šie: žebenkštis, pelė ir krokodilas su visa jo gimine; 
\par 30 laukinė pelė ir chameleonas; salamandra, žaliasis driežas ir kurmis. 
\par 31 Visi jie yra nešvarūs. Kas prisiliestų prie jų negyvų, bus nešvarus iki vakaro. 
\par 32 Jei kas iš jų negyvas užkristų ant ko nors, ar tai būtų medinis indas, ar apdaras, ar kailis, ar ašutinė, ar šiaip kuriam nors reikalui vartojamas daiktas, tai bus sutepta. Jis bus panardintas vandenyje ir laikomas suteptu iki vakaro, po to bus švarus. 
\par 33 Jei kas iš jų įkristų į molinį indą, jis bus suteptas ir turės būti sudaužytas. 
\par 34 Kiekvienas maistas, kurį valgysite, jei ant jo bus užpilta iš tokio indo vandens, bus suteptas; kiekvienas skystis, geriamas iš tokio indo, bus nešvarus; 
\par 35 visa, ant ko užkristų kas nors iš tokios maitos, bus sutepta; krosnis ir katilas turi būti sudaužyti, nes sutepti. 
\par 36 Tik šaltiniai ir šuliniai bus nesutepti; bet kas prisiliestų iš jų išimtos maitos, bus nešvarus. 
\par 37 Jei toks pastipęs gyvis užkristų ant sėklos, jos nesuteps, 
\par 38 bet, jei sėklą kas apipiltų vandeniu ir po to ją paliestų maita, ji bus sutepta. 
\par 39 Jei nustiptų gyvulys, kurį jums leista valgyti, kas jo prisiliestų, bus nešvarus iki vakaro. 
\par 40 Kas valgytų ar neštų ką nors iš jo, plaus savo drabužius ir bus nešvarus iki vakaro. 
\par 41 Visais gyvūnais, kurie šliaužia ant žemės, jūs bjaurėsitės ir jų nevalgysite. 
\par 42 Nevalgysite šliaužiančių ant pilvo, nei ropojančių keturiomis, nei turinčių daugiau kojų, nes jie yra jums pasibjaurėjimas. 
\par 43 Nesusitepkite jais ir nieko iš jų nepalieskite, kad nebūtumėte nešvarūs. 
\par 44 Aš esu Viešpats, jūsų Dievas; būkite šventi, nes Aš esu šventas. Nesusitepkite jokiu ropliu, kuris kruta ant žemės. 
\par 45 Aš esu Viešpats, kuris jus išvedžiau iš Egipto žemės, kad būčiau jūsų Dievas. Būkite šventi, nes Aš esu šventas. 
\par 46 Tai yra įstatymas apie gyvulius, paukščius ir visus gyvius, kurie kruta vandenyje ir gyvena žemėje, 
\par 47 kad žinotumėte skirtumą tarp švaraus ir nešvaraus, kas leista valgyti ir kas neleistina’ ”.



\chapter{12}


\par 1 Viešpats kalbėjo Mozei: 
\par 2 “Paskelbk izraelitams: ‘Jei moteris pagimdytų berniuką, ji bus septynias dienas nešvari, kaip mėnesinių metu. 
\par 3 Aštuntąją dieną kūdikis bus apipjaustytas, 
\par 4 bet ji pasiliks dar trisdešimt tris dienas apsivalymui. Nieko, kas šventa, ji nepalies ir neis į šventyklą, iki pasibaigs jos apsivalymo dienos. 
\par 5 O jei pagimdytų mergaitę, bus nešvari dvi savaites, kaip mėnesinių metu, ir šešiasdešimt šešias dienas tęsis apsivalymas. 
\par 6 Kai pasibaigs jos apsivalymo dienos už sūnų ar dukterį, atves prie Susitikimo palapinės įėjimo metinį avinėlį deginamajai aukai ir jauną karvelį ar balandį aukai už nuodėmę ir atiduos kunigui, 
\par 7 kuris tai aukos Viešpaties akivaizdoje ir sutaikins ją su Viešpačiu, ir taip ji bus apvalyta. Toks yra įstatymas apie pagimdžiusią berniuką ar mergaitę. 
\par 8 O jei neturėtų iš ko ir negalėtų aukoti avinėlio, ims du balandžius ar du jaunus karvelius­vieną deginimo aukai, o kitą aukai už nuodėmę,­kunigas sutaikins ją, ir ji bus apvalyta’ ”.



\chapter{13}


\par 1 Viešpats kalbėjo Mozei ir Aaronui: 
\par 2 “Žmogus, ant kurio kūno odos atsirastų kitokios spalvos taškas ar šašas ar blizganti vieta, turinti raupsų ligos požymį, bus atvestas pas kunigą Aaroną arba pas kurį iš jo sūnų. 
\par 3 Jei tas pamatys šašą ant odos ir pastebės, kad plaukai yra pabalę ir vieta, turinti raupsų požymį, lyginant ją su sveiko kūno oda, yra įdubusi, tai įrodymas, jog tai yra raupsų liga; kunigas apžiūrės jį ir paskelbs nešvariu. 
\par 4 O jei ant odos bus blizgantis baltas taškas, bet nebus įdubęs, lyginant su kita oda, ir plaukai bus pirmykštės spalvos, kunigas uždarys jį septynioms dienoms, 
\par 5 o septintą dieną jį patikrins. Jei tariami raupsai nebus padidėję ir nebus peržengę pirmykščių ribų odoje, uždarys jį dar septynioms dienoms. 
\par 6 Septintą dieną vėl patikrins. Jei taškas bus tamsesnis ir nepadidėjęs, paskelbs žmogų esant švariu, nes tai šašai; jis išplaus savo drabužius ir bus švarus. 
\par 7 Bet, jei po to, kai jį matė kunigas ir pripažino švariu, šašas padidėtų, jis bus pas jį vėl atvestas; kunigas jį patikrins, ir, jei žaizda padidėjusi, 
\par 8 jį paskelbs nešvariu,­tai raupsai. 
\par 9 Jei žmogui atsirastų raupsų žymių, jis bus atvestas pas kunigą. 
\par 10 Tas jį apžiūrės. Jei plaukų spalva bus pabalus ir pasikeitus ir matysis gyva mėsa, 
\par 11 tai ženklas, kad raupsai įsisenėję ir įaugę į odą. Kunigas paskelbs jį nešvariu, bet neuždarys, nes jo nešvarumas aiškus. 
\par 12 Jei raupsai išsiplėstų taip, kad apdengtų visą odą nuo galvos iki kojų, ką galima matyti ir akimis, 
\par 13 kunigas jį apžiūrės ir nutars, kad jo raupsai yra švarūs, nes jie visai pabalo, ir todėl žmogus yra švarus. 
\par 14 O jei pasirodys gyva mėsa, 
\par 15 tada bus kunigo paskelbtas nešvariu, nes gyva mėsa nešvari; tai yra raupsai. 
\par 16 O jei vėl pabals ir padengs visą žmogų, 
\par 17 kunigas jį apžiūrės ir paskelbs švariu. 
\par 18 Jei kam odoje atsirastų votis ir pagytų, 
\par 19 bet voties vietoje pasirodytų baltas ar rausvas randas, toks žmogus bus atvestas pas kunigą. 
\par 20 Šis, jei rastų tariamų raupsų vietą, lyginant su kita oda, įdubusią ir plaukus pabalusius, paskelbs jį nešvariu, nes votyje atsirado raupsų liga. 
\par 21 Bet, jei plaukai bus pirmykštės spalvos, o randas apytamsis ir vieta neįdubusi, uždarys jį septynioms dienoms; 
\par 22 jei po to randas padidėtų, pripažins žmogų nešvariu,­tai raupsai. 
\par 23 O jei bus pasilikęs toje pačioje vietoje, tai ženklas, jog tai tik voties randas,­žmogus švarus. 
\par 24 Jei oda, nudeginta ugnimi, sugijus turėtų baltą ar rausvą randą, 
\par 25 kunigas ją apžiūrės; jei pabalusioji vieta bus įdubusi, paskelbs žmogų nešvariu, nes tai raupsų liga. 
\par 26 Jei plaukų spalva nebus pasikeitusi ir sužeistoji vieta nebus įdubusi, bet bus neryški, uždarys jį septynioms dienoms 
\par 27 ir apžiūrės septintą dieną. Jei dėmė odoje padidėtų, paskelbs jį nešvariu, nes tai raupsai. 
\par 28 O jei baltumas pasiliks savo vietoje ir bus neryškus, tai yra tik nusideginimo žaizda; žmogus bus laikomas švariu. 
\par 29 Jei kuriam vyrui ar moteriai atsirastų raupsai ant galvos ar ant smakro, jį apžiūrės kunigas. 
\par 30 Jei žaizda bus įdubusi, plaukai pageltę ir plonesni negu paprastai, paskelbs nešvariu, nes tai yra galvos ir smakro raupsai. 
\par 31 O jei matys, kad nesveikoji vieta neįdubusi ir plaukai natūralūs, uždarys septynioms dienoms 
\par 32 ir septintą dieną vėl apžiūrės. Jei nesveikoji vieta nebus padidėjusi nė įdubusi, plaukai tos pačios spalvos, 
\par 33 tai žmogaus galva bus nuskusta, išskyrus tą vietą, ir jis bus uždarytas kitoms septynioms dienoms. 
\par 34 Jei septintą dieną pasirodys, kad viskas kaip buvo, paskelbs jį švariu. Jis, išplovęs savo drabužius, bus švarus. 
\par 35 O jei, paskelbus jį švariu, nesveikoji vieta odoje padidės, 
\par 36 kunigas daugiau nebetyrinės, nes aišku­žmogus nešvarus. 
\par 37 Bet, jei nesveikoji vieta nebus padidėjusi ir plaukai bus natūralūs, aišku, kad žmogus pagijęs, ir paskelbs jį švariu. 
\par 38 Jei kuriam vyrui ar moteriai atsirastų balta dėmė odoje, 
\par 39 kunigas jį apžiūrės. Jei blizgantis odoje baltumas yra apytamsis, težino, kad tai ne raupsai, bet baltos spalvos taškas,­žmogus švarus. 
\par 40 Jei kuriam vyrui slenka nuo galvos plaukai, jis tampa plikagalvis, bet yra švarus. 
\par 41 Jei plaukai nuslinko nuo kaktos, jis pasidarė plikakaktis, bet švarus. 
\par 42 O jei nuplikusi galva ar kakta pabalo ar paraudo, 
\par 43 tai kunigas patikrinęs paskelbs jį raupsuotu. 
\par 44 Jis serga raupsais ir yra nešvarus, kunigas paskelbs jį nešvariu; ant jo galvos raupsai. 
\par 45 Raupsuotasis turi persiplėšti drabužį, atidengti galvą, burną laikyti uždengtą ir šaukti: ‘Nešvarus, nešvarus!’ 
\par 46 Visą laiką, kol bus raupsuotas ir nešvarus, gyvens vienas už stovyklos. 
\par 47 Jei vilnonis ar drobinis drabužis 
\par 48 ar kailis, ar kas nors padaryta iš kailio 
\par 49 turėtų baltos ar rausvos spalvos taškus, tie rūbai bus laikomi apkrėsti raupsais. Jie bus parodyti kunigui, 
\par 50 kuris, juos apžiūrėjęs, uždarys septynioms dienoms. 
\par 51 Jei kunigas, apžiūrėjęs septintą dieną, atras padidėjusias dėmes, tai bus raupsai. Jis pripažins tą drabužį nešvariu, nes ant jo yra plintantys raupsai. 
\par 52 Todėl jis sudegins tą drabužį, nes tai yra plintantys raupsai. 
\par 53 O jei matys tašką nepadidėjusį, 
\par 54 lieps išplauti tą raupsuotą rūbą ir uždarys jį kitoms septynioms dienoms. 
\par 55 Jei patikrinęs matys, kad jo pirmykštė išvaizda nesugrįžo, nors raupsai ir nepadidėjo, pripažins nešvariu ir sudegins, nes raupsai įsigraužė į apdaro paviršių. 
\par 56 Jei, drabužį išplovus, raupsų vieta bus tamsesnė, ją atplėš ir atskirs nuo drabužio. 
\par 57 Bet, jei dėmės pasirodys tose vietose, kurios pirma buvo švarios, vadinasi, raupsai plinta; tuo atveju drabužis bus sudegintas. 
\par 58 Jei dėmių nebeatsiras, išplaus drabužį antrą kartą ir jis bus švarus. 
\par 59 Tai yra įstatymas apie raupsus vilnoniame ir drobiniame drabužyje, audiniuose ir kailio apdaruose, kada juos pripažinti švariais ir kada nešvariais”.



\chapter{14}


\par 1 Viešpats kalbėjo Mozei: 
\par 2 “Tai įstatymas raupsuotajam, kai jis paskelbiamas švariu. Jis bus atvestas pas kunigą, 
\par 3 kuris turės išeiti iš stovyklos. Radęs jį išgijusį nuo raupsų, 
\par 4 įsakys jam paimti du sveikus, švarius paukščius, kedro medžio, raudonų siūlų ir yzopo. 
\par 5 Vieną paukštį kunigas lieps papjauti moliniame inde virš tekančio vandens, 
\par 6 kitą gyvą kartu su kedro medžiu, raudonais siūlais ir yzopu padažys papjauto paukščio kraujyje 
\par 7 ir juo pašlakstys septynis kartus apvalomąjį, kad būtų paskelbtas švariu. Gyvą paukštį paleis skristi į laukus. 
\par 8 Po to žmogus išplaus savo drabužius, nusiskus plaukus ir apsiplaus vandeniu; taip apvalytasis įeis į stovyklą, bet savo palapinėn neis dar septynias dienas. 
\par 9 Septintą dieną jis nusiskus galvos plaukus, barzdą ir antakius, dar kartą išplaus darbužius ir nusiplaus kūną. 
\par 10 Aštuntą dieną ims du sveikus avinėlius, metinę avelę, duonos aukai tris dešimtąsias efos smulkių miltų, sumaišytų su aliejumi, ir vieną logą aliejaus. 
\par 11 Apvalymo apeigas atliekąs kunigas pastatys jį ir visa tai Viešpaties akivaizdoje prie Susitikimo palapinės įėjimo. 
\par 12 Po to ims avinėlį ir aukos jį už kaltę; ims taip pat aliejaus logą ir visa tai siūbuos Viešpaties akivaizdoje. 
\par 13 Avinėlį papjaus šventoje vietoje, kur aukojama deginamoji auka ir auka už nuodėmę. Auka už nuodėmę ir auka už kaltę priklauso kunigui ir yra šventa. 
\par 14 Kunigas, ėmęs aukos už kaltę kraują, pateps juo apvalomojo dešinę ausį ir dešinės rankos ir kojos nykščius. 
\par 15 Iš aliejaus logo dalį įsipils į savo kairiosios rankos delną 
\par 16 ir, padažęs jame dešinės rankos rodomąjį pirštą, pašlakstys Viešpaties akivaizdoje septynis kartus. 
\par 17 Likusį kairės rankos delne aliejų išlies ant apvalomojo dešinės ausies, ant dešinės rankos ir kojos nykščių 
\par 18 ir ant jo galvos. Ir kunigas sutaikins jį su Viešpačiu. 
\par 19 Po to kunigas, atlikdamas sutaikinimą, aukos auką už nuodėmę ir deginamąją auką, 
\par 20 padėdamas ją ant aukuro kartu su duonos auka; taip žmogus bus sutaikintas ir apvalytas. 
\par 21 O jei jis neturtingas ir neišgali duoti minėtų dalykų, ims aukai už kaltę avinėlį, kad kunigas sutaikintų jį, dešimtą dalį efos smulkių miltų, sumaišytų su aliejumi, duonos aukai, logą aliejaus 
\par 22 ir du balandžius ar du jaunus karvelius: vieną aukai už nuodėmę, o antrą deginamajai aukai. 
\par 23 Visa tai jis atneš aštuntą savo apsivalymo dieną kunigui prie Susitikimo palapinės įėjimo Viešpaties akivaizdon. 
\par 24 Kunigas, paėmęs avinėlį aukai už kaltę ir aliejų, pasiūbuos tai Viešpaties akivaizdoje. 
\par 25 Papjovęs avinėlį, jo krauju pateps dešinę apvalomojo ausį, dešinės rankos ir kojos nykščius. 
\par 26 Ir kunigas dalį aliejaus įsipils į savo kairės rankos delną, 
\par 27 padažęs dešinės rankos pirštą, pašlakstys septynis kartus Viešpaties akivaizdoje, 
\par 28 pateps dešinę apvalomojo ausį ir dešinės rankos bei kojos nykščius toje vietoje, kur buvo patepta aukos už kaltę krauju. 
\par 29 Likusią aliejaus dalį, esančią rankoje, išpils ant apvalomojo galvos, kad sutaikintų jį su Viešpačiu. 
\par 30 Aukos taip pat vieną balandį ar jauną karvelį,­iš to, ką pajėgs gauti,­ 
\par 31 aukai už nuodėmę, o kitą­deginamajai aukai, ir duonos auką, sumaišytą su aliejumi. 
\par 32 Tai auka nepasiturinčio raupsuotojo, kuris neišgali daugiau aukoti savo apsivalymui”. 
\par 33 Viešpats kalbėjo Mozei ir Aaronui: 
\par 34 “Jei, įėjus į Kanaano žemę, kurią jums duosiu paveldėti, atsirastų raupsų liga namų sienose, 
\par 35 jų savininkas privalo pranešti kunigui, kad, jo nuomone, raupsų liga yra jo namuose. 
\par 36 Kunigas, prieš patikrindamas namus, ar jie raupsuoti, lieps išnešti iš jų viską, kad visi juose esą daiktai nebūtų sutepti. Paskui įeis ir apžiūrės namų raupsus. 
\par 37 Pamatęs ant sienų įdubusius taškus žalsvos ar rausvos spalvos, 
\par 38 išeis iš namų pro duris ir tuojau juos uždarys septynioms dienoms. 
\par 39 Septintą dieną sugrįžęs, juos apžiūrės, ir, jei ras raupsus padidėjusius, 
\par 40 lieps išlupti akmenis, ant kurių yra raupsai, ir juos išmesti už miesto suteptoje vietoje, 
\par 41 namų vidaus sienas nugramdyti ir nuoskutas išpilti už miesto suteptoje vietoje, 
\par 42 išluptųjų akmenų vietoje įdėti kitus ir namus ištepti moliu. 
\par 43 Jei, išplėšus akmenis, nuskutus sienas ir moliu ištepus, 
\par 44 kunigas pamatys vėl pasirodžiusius raupsus ir sienas taškuotas, reiškia raupsai yra pasilikę ir namai nešvarūs. 
\par 45 Tokius namus sugriaus, o jų akmenis ir medžius išmes suteptoje vietoje už miesto. 
\par 46 Kas įeitų į uždarytus namus, bus nešvarus iki vakaro; 
\par 47 kas juose miegotų ar valgytų, plaus savo drabužius. 
\par 48 Jei kunigas, įėjęs į namus, nepamatys po ištinkavimo išplitusių raupsų, paskelbs namus švariais. 
\par 49 Jiems apvalyti ims du paukščius, kedro medžio ir raudonų siūlų bei yzopo. 
\par 50 Papjovęs vieną paukštį moliniame inde virš tekančio vandens, 
\par 51 ims kedro medį, yzopą, raudonų siūlų ir gyvą paukštį, padažys viską papjauto paukščio kraujyje ir tekančiame vandenyje ir apšlakstys namus septynis kartus. 
\par 52 Paukščio kraujas, tekantis vanduo, kedro medis, yzopas bei raudoni siūlai ir gyvas paukštis apvalys namus. 
\par 53 Gyvą paukštį jis išneš už miesto ir paleis skristi į laukus. Taip namas bus apvalytas. 
\par 54 Tai yra įstatymas apie visokius raupsus ir piktšašius, 
\par 55 raupsus drabužiuose ir namuose; 
\par 56 randus, nušašimus, blizgančius taškus ir įvairius pakitimus, 
\par 57 kad žinotumėte, kas švaru ar nešvaru. Tai yra įstatymas apie raupsus”.



\chapter{15}

\par 1 Viešpats kalbėjo Mozei ir Aaronui: 
\par 2 “Paskelbkite izraelitams: ‘Vyras, turįs plūdimą iš savo kūno, yra nešvarus. 
\par 3 Jis bus laikomas nešvariu, jei turi plūdimą iš savo kūno ar plūdimas susilaiko jo kūne, nes tai yra jo nešvara. 
\par 4 Kiekvienas patalas, kur jis miegotų, ir vieta, kur sėdėtų, bus sutepta. 
\par 5 Jei kas iš žmonių prisiliestų prie jo patalo, plaus savo drabužius, pats apsiplaus vandeniu ir bus nešvarus iki vakaro. 
\par 6 Jei kas sėdėtų, kur anas sėdėjo, privalės išplauti savo drabužius, pats apsiplauti vandeniu ir bus nešvarus iki vakaro. 
\par 7 Jei kas prisiliestų prie jo kūno, plaus savo drabužius, pats apsiplaus vandeniu ir bus nešvarus iki vakaro. 
\par 8 Jei toks žmogus spjautų ant nesutepto, tas plaus savo rūbus, apsiplaus vandeniu ir bus nešvarus iki vakaro. 
\par 9 Balnas, ant kurio jis sėdėtų, bus suteptas. 
\par 10 Kas paliestų ką nors, kas buvo po juo, bus nešvarus iki vakaro. Kas neštų tokius daiktus, plaus savo drabužius, pats apsiplaus vandeniu ir bus nešvarus iki vakaro. 
\par 11 Prie ko jis prisiliestų nenusiplovęs rankų, tas plaus savo rūbus, apsiplaus vandeniu ir bus nešvarus iki vakaro. 
\par 12 Molinis indas, prie kurio prisiliestų suteptasis, turi būti sudaužytas, medinis gi indas bus išplautas vandeniu. 
\par 13 Kai turintis plūdimą iš savo kūno pasveiksta, suskaičiuos septynias dienas apsivalymui, išsiplaus drabužius, nusimaudys tekančiame vandenyje ir bus švarus. 
\par 14 Aštuntą dieną ims du balandžius arba du jaunus karvelius ir, atėjęs Viešpaties akivaizdon prie Susitikimo palapinės įėjimo, paduos juos kunigui. 
\par 15 Tas vieną aukos kaip auką už nuodėmę, o kitą­kaip deginamąją auką ir sutaikins jį Viešpaties akivaizdoje, kad būtų apvalytas nuo ligos. 
\par 16 Vyras, kuriam išsilieja sėkla, apsiplaus vandeniu visą kūną ir bus nešvarus iki vakaro. 
\par 17 Jis plaus vandeniu drabužį ar kailį, ant kurio pateko sėkla, ir visa tai bus sutepta iki vakaro. 
\par 18 Moteris, su kuria jis miegotų, kai jam išsilieja sėkla, apsiplaus vandeniu ir bus nešvari iki vakaro. 
\par 19 Moteris, turinti mėnesines, bus atskirta septynioms dienoms. Kiekvienas, kas prie jos prisiliestų, bus nešvarus iki vakaro. 
\par 20 Ir visa, ant ko ji gulėtų arba sėdėtų tomis dienomis, bus sutepta. 
\par 21 Kas prisiliestų prie jos patalo, plaus savo rūbus, pats apsiplaus vandeniu ir bus nešvarus iki vakaro. 
\par 22 Jei kas nors prisiliestų prie bet kurio daikto, ant kurio ji sėdėjo, plaus savo rūbus, pats apsiplaus vandeniu ir bus nešvarus iki vakaro. 
\par 23 Jei kas paliestų tai, kas buvo ant jos patalo ar vietą, kur ji sėdėjo, bus nešvarus iki vakaro. 
\par 24 Jei miegotų su ja vyras ir ant jo patektų jos nešvarumų, jis bus nešvarus septynias dienas; ir patalas, ant kurio jis miegotų, bus irgi suteptas. 
\par 25 Moteris, kraujuojanti ilgesnį laiką ne mėnesinių metu arba kuriai mėnesinės užsitęsia ilgiau negu įprasta, bus nešvari visą tą laiką. 
\par 26 Patalas, ant kurio ji miegotų, ir tai, ant ko ji atsisėstų, bus sutepta. 
\par 27 Kas prie jų prisiliestų, plaus savo rūbus, pats nusiplaus vandeniu ir bus nešvarus iki vakaro. 
\par 28 Jei kraujavimas sustotų, po septynių dienų ji bus švari. 
\par 29 Aštuntą dieną ji paims du balandžius arba du jaunus karvelius ir atneš kunigui prie Susitikimo palapinės įėjimo. 
\par 30 Tas vieną aukos kaip auką už nuodėmę, o kitą­kaip deginamąją auką ir sutaikins ją Viešpaties akivaizdoje dėl plūdimo nešvaros’. 
\par 31 Judu mokysite izraelitus saugotis susitepimo, kad nemirtų dėl savo nešvarumo ir kad nesuteptų mano palapinės. 
\par 32 Tai yra įstatymas tam, kuris turi plūdimą iš savo kūno, tam, kuriam išsilieja sėkla ir jį sutepa, 
\par 33 moteriai, turinčiai mėnesinį plūdimą, vyrui ar moteriai, turintiems plūdimą, ir tam, kuris guli prie nešvarios moters”.



\chapter{16}


\par 1 Mirus dviem Aarono sūnums kai jie artinosi prie Viešpaties su svetima ugnimi, Viešpats kalbėjo Mozei: 
\par 2 “Sakyk savo broliui Aaronui, kad neitų bet kada į Švenčiausiąją, už uždangos, prie dangčio, esančio ant skrynios, kad nenumirtų, nes Aš pasirodysiu debesyje virš dangčio. 
\par 3 Prieš įeidamas jis privalo aukoti veršį aukai už nuodėmę ir aviną deginamajai aukai. 
\par 4 Jis apsivilks drobine jupa, užsimaus drobines kelnes, apsijuos drobine juosta, ant galvos užsidės drobinį gobtuvą. Šitie apdarai yra šventi. Prieš apsivilkdamas apsiplaus kūną. 
\par 5 Jis ims iš izraelitų du ožius aukai už nuodėmę ir aviną deginamajai aukai. 
\par 6 Aaronas aukos veršį aukai už savo nuodėmę ir sutaikins save ir savo namus. 
\par 7 Jis pastatys du ožius Viešpaties akivaizdoje prie Susitikimo palapinės įėjimo 
\par 8 ir, mesdamas burtus, skirs vieną Viešpačiui, o antrą­išvarymui. 
\par 9 Kuriam kris burtas Viešpačiui, tą aukos už nuodėmę; 
\par 10 kuris gi skirtas išvarymui, tą statys gyvą Viešpaties akivaizdoje, kad atliktų ant jo sutaikinimą ir išvarytų jį į dykumą. 
\par 11 Kai Aaronas paaukos veršį už savo ir savo šeimos nuodėmę, 
\par 12 jis ims smilkytuvą, kurį pripildys aukuro žarijomis, į saują paims sumaišytų smilkalų ir, įėjęs į Švenčiausiąją, 
\par 13 užpils ant ugnies smilkalų Viešpaties akivaizdoje, kad kvapūs dūmai apdengtų Liudijimo skrynios dangtį ir kad jis nemirtų. 
\par 14 Tuomet ims veršio kraują ir savo pirštu septynis kartus pašlakstys prieš dangtį ir septynis kartus ant jo. 
\par 15 Paskui aukos ožį už tautos nuodėmę, įneš jo kraują už uždangos, pašlakstys juo septynis kartus prieš dangtį ir septynis kartus ant jo, kaip tai darė su veršio krauju. 
\par 16 Taip bus apvalyta Švenčiausioji nuo izraelitų nešvarumo, jų nusikaltimų ir visų nuodėmių. Tą patį darys ir Susitikimo palapinei, kuri stovi tarp jų palapinių. 
\par 17 Nė vieno žmogaus neturi būti Susitikimo palapinėje, kai kunigas įeina į Švenčiausiąją atlikti sutaikinimo dėl savęs, savo šeimos ir viso Izraelio. 
\par 18 Aaronas, išėjęs prie aukuro, esančio kieme, apvalys jį, veršio ir ožio krauju patepdamas aukuro ragus 
\par 19 ir pašlakstydamas aukurą krauju septynis kartus. Taip jį apvalys nuo izraelitų nešvarumo ir pašventins. 
\par 20 Kai baigs Švenčiausiosios, Susitikimo palapinės ir aukuro apvalymą, atves gyvąjį ožį 
\par 21 ir, uždėjęs jam ant galvos abi rankas, išpažins visas izraelitų nedorybes, visus jų nusikaltimus ir nuodėmes, sukraudamas jas ant ožio galvos. Tada tam skirtas asmuo išves ožį į dykumą. 
\par 22 Ir ožys išneš visus jų nusikaltimus ant savęs į negyvenamas vietoves ir bus paleistas dykumoje. 
\par 23 Aaronas sugrįš į Susitikimo palapinę, nusivilks apdarus, kuriais buvo apsirengęs, įeidamas į Švenčiausiąją, ir, juos ten palikęs, 
\par 24 nuplaus savo kūną šventoje vietoje ir apsivilks savo rūbais. Po to išėjęs aukos savo ir tautos deginamąją auką, atlikdamas sutaikinimą už save ir tautą. 
\par 25 Jis degins ant aukuro aukos už nuodėmę taukus. 
\par 26 O tas, kuris išvedė išvarymo ožį, plaus savo drabužius ir kūną vandeniu ir tik po to įeis į stovyklą. 
\par 27 Veršį ir ožį, kurie buvo paaukoti už nuodėmę ir kurių kraujas buvo įneštas į Švenčiausiąją sutaikinimui, išgabens už stovyklos ir viską sudegins: odas, mėsą ir mėšlą. 
\par 28 O tas, kuris tai atliks, plaus savo drabužius ir kūną vandeniu ir tik po to įeis į stovyklą. 
\par 29 Tai yra jums amžinas įstatymas. Septinto mėnesio dešimtą dieną varginsite savo sielas ir jokio darbo nedirbsite nei jūs, nei tarp jūsų gyveną ateiviai. 
\par 30 Ta diena bus jūsų sutaikinimo ir apsivalymo diena nuo visų nuodėmių Viešpaties akivaizdoje. 
\par 31 Tai yra sabatas poilsiui; jūs varginsite savo sielas, vykdydami amžinąjį įstatymą. 
\par 32 Jus sutaikins kunigas, pateptas ir pašventintas eiti kunigo tarnystę vietoje savo tėvo. Jis apsivilks drobiniais rūbais, šventaisiais apdarais, 
\par 33 apvalys Švenčiausiąją, Susitikimo palapinę ir aukurą, taip pat sutaikins kunigus ir visą tautą. 
\par 34 Tai bus jums amžinas įstatymas, kad Izraelio vaikai būtų sutaikinti dėl visų savo nuodėmių vieną kartą per metus”. Jis atliko viską, ką Viešpats buvo įsakęs Mozei.



\chapter{17}


\par 1 Viešpats kalbėjo Mozei: 
\par 2 “Sakyk Aaronui, jo sūnums ir visiems izraelitams: ‘Taip sako Viešpats: 
\par 3 ‘Kiekvienas izraelitas, kuris papjautų jautį, avį ar ožką stovykloje ar už stovyklos 
\par 4 ir neaukotų aukos Viešpačiui prie palapinės įėjimo, bus kaltas, nes praliejo kraują. Jis bus išnaikintas iš savo tautos. 
\par 5 Todėl izraelitai savo gyvulius, užuot pjovę juos lauke, privalo atvesti pas kunigą, kad jie būtų pašvęsti Viešpačiui prie Susitikimo palapinės įėjimo, ir juos aukoti Viešpačiui kaip padėkos auką. 
\par 6 Kunigas išlies kraują ant Viešpaties aukuro prie Susitikimo palapinės įėjimo ir sudegins taukus, kad būtų malonus kvapas Viešpačiui’. 
\par 7 Tuo būdu izraelitai nebeaukos daugiau savo aukų piktosioms dvasioms. Tai bus amžinas įstatymas jiems ir jų palikuonims. 
\par 8 Jiems sakyk, kad kiekvienas izraelitas ar ateivis, gyvenantis tarp jūsų, kuris aukotų deginamąją ar kitokią auką 
\par 9 ir jos neatvestų prie Susitikimo palapinės įėjimo paaukoti Viešpačiui, bus išnaikintas iš savo tautos. 
\par 10 Jei kas iš izraelitų ar gyvenančių tarp jūsų ateivių valgytų kraują, Aš atgręšiu savo veidą į jį ir jį išnaikinsiu iš jo tautos. 
\par 11 Kūno gyvybė yra kraujyje; Aš jums jį daviau, kad juo ant aukuro atliktumėte sutaikinimą už savo sielas, nes kraujas sutaikina sielą. 
\par 12 Todėl įsakau izraelitams, kad niekas iš jūsų nė iš ateivių, gyvenančių tarp jūsų, nevalgytų kraujo. 
\par 13 Jei izraelitas ar ateivis pagautų medžiodamas kokį žvėrį ar paukštį, kuriuos leista valgyti, teišlieja jo kraują ant žemės ir jį pakasa, 
\par 14 nes kiekvieno kūno gyvybė yra kraujyje; todėl įsakau izraelitams nevalgyti kraujo iš jokio kūno. Kas jį valgys, bus išnaikintas. 
\par 15 Kas iš vietinių gyventojų ar ateivių valgytų pastipusį ar žvėries sudraskytą, plaus savo drabužius ir pats apsiplaus vandeniu; jis bus nešvarus iki vakaro, o po to bus švarus, 
\par 16 bet, jei neplaus savo drabužio ir kūno, nešios savo kaltę’ ”.



\chapter{18}

\par 1 Viešpats kalbėjo Mozei: 
\par 2 “Aš esu Viešpats, jūsų Dievas. 
\par 3 Nesielkite, kaip elgiasi egiptiečiai, tarp kurių jūs gyvenote, ir nesilaikykite papročių Kanaano šalies, kurion jus įvesiu, nevaikščiokite pagal jų nuostatus. 
\par 4 Vykdykite mano paliepimus ir pagal mano nuostatus elkitės. Aš esu Viešpats, jūsų Dievas. 
\par 5 Laikykitės mano įstatymų ir paliepimų, kuriuos vykdydamas žmogus bus gyvas. Aš esu Viešpats. 
\par 6 Nesiartinkite prie artimo giminaičio, kad atidengtumėte jo nuogumą. Aš esu Viešpats. 
\par 7 Neatidenk savo tėvo nei motinos nuogumo. Ji yra tavo motina­tu neturi atidengti jos nuogumo. 
\par 8 Neatidenk savo tėvo žmonos nuogumo, nes tai yra tavo tėvo nuogumas. 
\par 9 Neatidenk nuogumo savo sesers, kuri yra tavo tėvo ar motinos duktė, gimusi namie ar kur kitur. 
\par 10 Neatidenk savo sūnaus dukros ar dukters dukros nuogumo, nes tai tavo paties nuogumas. 
\par 11 Neatidenk nuogumo savo tėvo žmonos dukros, kurią ji pagimdė tavo tėvui, nes ji yra tavo sesuo. 
\par 12 Neatidenk savo tėvo sesers nuogumo, nes tai yra tavo tėvo artima giminaitė. 
\par 13 Neatidenk savo motinos sesers nuogumo, nes tai tavo motinos artima giminaitė. 
\par 14 Neatidenk savo dėdės žmonos nuogumo, nes ji yra tavo teta. 
\par 15 Neatidenk savo sūnaus žmonos nuogumo. 
\par 16 Neatidenk savo brolio žmonos nuogumo, nes tai tavo brolio nuogumas. 
\par 17 Neatidenk nuogumo moters ir jos dukros arba jos sūnaus dukros, nes jie yra jos artimi giminaičiai; tai nusikaltimas. 
\par 18 Neimk savo žmonos sesers į suguloves ir neatidenk jos nuogumo žmonai gyvai esant. 
\par 19 Nesiartink prie moters, turinčios mėnesines, kad atidengtum jos nuogumą. 
\par 20 Nesigulk su savo artimo žmona, kad susiterštum su ja. 
\par 21 Neaukok savo vaikų sudeginti Molechui ir neišniekink mano vardo, nes Aš esu Viešpats. 
\par 22 Nesugulk su vyru kaip su moterimi, nes tai pasibjaurėjimas. 
\par 23 Nesantykiauk su gyvuliu save suteršdamas, ir moteris neturi stotis prieš gyvulį, kad su juo santykiautų, nes tai iškrypimas. 
\par 24 Nesusitepkite minėtais būdais, nes taip susitepė tautos, kurias Aš išvarau nuo jūsų. 
\par 25 Kraštas buvo suteptas. Aš užtrauksiu jų nusikaltimus ant jų, ir kraštas išvems savo gyventojus. 
\par 26 Jūs ir tarp jūsų pasiliekantys ateiviai laikykitės mano įstatymų ir įsakymų ir nedarykite nė vienos minėtų bjaurysčių. 
\par 27 Visas tas bjaurystes darė krašto gyventojai, gyvenę pirma jūsų, ir jį sutepė. 
\par 28 Saugokitės, kad kraštas ir jūsų neišspjautų, jei jį suteršite, kaip išspjovė pirma jūsų buvusias tautas. 
\par 29 Kiekvienas žmogus, kuris darytų bet kurią iš tų bjaurysčių, bus išnaikintas iš savo tautos. 
\par 30 Laikykitės mano nuostatų. Nedarykite to, ką darė pirma jūsų gyvenusieji. Aš esu Viešpats, jūsų Dievas”.



\chapter{19}


\par 1 Viešpats kalbėjo Mozei: 
\par 2 “Sakyk visam Izraeliui: ‘Būkite šventi, nes Aš, Viešpats, jūsų Dievas, esu šventas. 
\par 3 Kiekvienas turi gerbti savo tėvą ir motiną. Laikykitės mano sabato. Aš esu Viešpats, jūsų Dievas. 
\par 4 Nesikreipkite į stabus ir neliekite sau dievų. Aš esu Viešpats, jūsų Dievas. 
\par 5 Jei aukosite padėkos auką Viešpačiui, aukokite ją laisva valia. 
\par 6 Auką suvalgykite tą pačią dieną. Leidžiama ją valgyti ir antrą dieną, o kas liks trečiai dienai, sudeginkite. 
\par 7 Jei kas ją valgytų trečią dieną, nusikals ir auka bus Dievo nepriimta. 
\par 8 Jis bus kaltas, nes sutepė Viešpačiui skirtą auką. Toks žmogus bus išnaikintas iš jo tautos. 
\par 9 Pjaudamas savo žemės javus, jų nekirsk iki pat lauko krašto ir nerink nukritusių varpų. 
\par 10 Taip pat ir vynuogyne nenuskink paliktų uogų. Jos telieka beturčiams ir ateiviams. Aš esu Viešpats, jūsų Dievas. 
\par 11 Nevokite. Nemeluokite. Neapgaudinėkite vieni kitų. 
\par 12 Neprisiekite neteisingai mano vardu ir nesutepkite savo Dievo vardo. Aš esu Viešpats. 
\par 13 Nespauskite ir neišnaudokite savo artimo. Samdinio algos nesulaikykite iki ryto. 
\par 14 Nekeikite kurčio ir nedėkite kliūties aklajam ant kelio, bet bijokite Dievo, nes Aš esu Viešpats. 
\par 15 Teisme teiskite teisingai, neatsižvelgdami, ar tai beturtis, ar žymus asmuo. 
\par 16 Neliežuvaukite, vaikščiodami tarp savųjų. Netykokite savo artimo kraujo. Aš esu Viešpats. 
\par 17 Nelaikykite širdyje neapykantos prieš savo brolį. Sudrauskite savo artimą, kad jums nebūtų nuodėmės dėl jo. 
\par 18 Nekeršykite ir atleiskite savo artimui. Mylėkite savo artimą kaip patys save. Aš esu Viešpats. 
\par 19 Laikykitės mano įstatymų. Neleiskite apsivaisinti skirtingos rūšies gyvuliams. Savo lauko neapsėkite mišria sėkla. Nedėvėkite drabužio, austo iš vilnų ir linų. 
\par 20 Jei vyras gulėtų su verge, kuri yra pažadėta vyrui, bet dar neišpirkta, abu bus nuplakti, bet neužmušti, nes ji nebuvo laisva. 
\par 21 Už savo kaltę jis atves Viešpačiui aviną prie Susitikimo palapinės. 
\par 22 Kunigas sutaikins jį su Viešpačiu, už jo nuodėmę paaukodamas aviną aukai už kaltę, ir jam bus atleista nuodėmė, kuria jis nusidėjo. 
\par 23 Kai įeisite į žemę ir pasodinsite vaismedžių, pirmųjų trejų metų vaisiai bus jums sutepti, jų nevalgysite; 
\par 24 ketvirtaisiais metais visi vaisiai bus pašvęsti Viešpaties garbei. 
\par 25 Penktaisiais gi metais skinsite ir valgysite vaisius. Aš esu Viešpats, jūsų Dievas. 
\par 26 Nevalgykite kraujo. Nežyniaukite ir neburkite. 
\par 27 Nekirpkite plaukų aplink galvą ir nekarpykite barzdos. 
\par 28 Mirusio atminimui nedarykite įrėžimų nei kitokių vaizdų ar ženklų savo kūne. Aš esu Viešpats. 
\par 29 Neparduokite savo dukterų būti paleistuvėmis, kad nesuteptumėte žemės ir ji netaptų pilna bjaurių nusikaltimų. 
\par 30 Laikykitės mano sabatų ir gerbkite šventyklą. Aš esu Viešpats. 
\par 31 Nesikreipkite į mirusiųjų dvasių iššaukėjus ir nieko nesiteiraukite žynių, kad per juos nesusiteptumėte. Aš esu Viešpats, jūsų Dievas. 
\par 32 Atsistokite prieš žilą galvą, gerbkite senus žmones, bijokite savo Dievo. Aš esu Viešpats. 
\par 33 Neišnaudokite jūsų krašte gyvenančių ateivių. 
\par 34 Tegyvena jie tarp jūsų kaip vietiniai; mylėkite juos kaip patys save, nes ir jūs buvote ateiviai Egipto žemėje. Aš esu Viešpats, jūsų Dievas. 
\par 35 Būkite teisingi teisme, matuodami, sverdami ir seikėdami. 
\par 36 Naudokite teisingus svarsčius ir teisingas svarstykles, teisingą efą ir hiną. Aš esu Viešpats, jūsų Dievas, kuris jus išvedžiau iš Egipto žemės. 
\par 37 Laikykitės visų mano įstatymų bei nuostatų ir juos vykdykite. Aš esu Viešpats’ ”.



\chapter{20}


\par 1 Viešpats kalbėjo Mozei: 
\par 2 “Sakyk izraelitams: ‘Jei izraelitas ar ateivis, gyvenantis Izraelyje, aukotų vaiką Molechui, bus baudžiamas mirtimi: krašto žmonės užmuš jį akmenimis. 
\par 3 Aš atgręšiu savo veidą į jį ir išnaikinsiu jį iš jo tautos, nes jis, duodamas savo vaiką Molechui, sutepė mano šventyklą ir paniekino mano šventą vardą. 
\par 4 Jei krašto žmonės, apsimesdami nežiną, paleistų žmogų, aukojusį vaiką Molechui, ir nenorėtų jo užmušti, 
\par 5 Aš bausiu jį, jo šeimą ir visus, kurie pritarė jo paleistuvystei su Molechu,­išnaikinsiu juos iš jų tautos. 
\par 6 Jei kas kreiptųsi į mirusiųjų dvasių iššaukėjus bei žynius ir su jais paleistuvautų, Aš atsigręšiu į juos ir išnaikinsiu iš jų tautos. 
\par 7 Pasišventinkite ir būkite šventi, nes Aš esu Viešpats, jūsų Dievas. 
\par 8 Laikykitės mano įsakymų ir vykdykite juos. Aš esu Viešpats, kuris darau jus šventus. 
\par 9 Kas keiktų savo tėvą ar motiną, bus mirtim nubaustas; jis keikė tėvą ar motiną ir jo kraujas yra ant jo. 
\par 10 Kas svetimautų su kito moterimi, su savo artimo žmona, abu­ svetimautojas ir svetimautoja­ bus baudžiami mirtimi. 
\par 11 Jei kas sugultų su savo pamote, abu turi mirti; jie patys kalti dėl savo mirties. 
\par 12 Jei kas sugultų su savo marčia, abu turi mirti, nes tai iškrypimas; jų kraujas yra ant jų. 
\par 13 Jei vyras sugultų su vyru kaip su moterimi, abu bjauriai nusikalstų ir abu turi mirti. 
\par 14 Jei kas, vedęs dukterį, vestų dar ir jos motiną, padarytų didelį nusikaltimą. Toks asmuo bus sudegintas kartu su abiem moterim, kad tarp jūsų nebūtų tokių nusikaltimų. 
\par 15 Kas paleistuvautų su gyvuliu, bus pasmerktas mirti; gyvulį taip pat užmuškite. 
\par 16 Jei moteris paleistuvautų su gyvuliu, ji bus užmušta akmenimis kartu su gyvuliu; jos kraujas bus ant jos. 
\par 17 Jei kas imtų savo seserį, savo tėvo ar motinos dukterį ir matytų jos nuogumą ir ji matytų jo nuogumą, abu bus padarę bjaurų nusikaltimą; jie bus išnaikinti savo tautos akivaizdoje; jie patys atsakys už savo kaltę. 
\par 18 Kas sugultų su moterimi jos mėnesinių metu, abu bus išnaikinti iš savo tautos. 
\par 19 Neatidenk nuogumo savo tėvo ar motinos sesers. Kas taip daro, nusikalsta ir abu bus nubausti. 
\par 20 Jei žmogus sugultų su savo dėdės žmona­jis atidengtų dėdės nuogumą; abu bus baudžiami už savo nuodėmę­jie mirs bevaikiai. 
\par 21 Jei vyras paimtų savo brolio žmoną, tai bjauru, nes jis atidengė savo brolio nuogumą,­jie mirs bevaikiai. 
\par 22 Laikykitės mano įstatymų ir įsakymų ir juos vykdykite, kad jūsų neišspjautų žemė, kurioje apsigyvensite. 
\par 23 Nesielkite pagal papročius tautų, kurias išvarysiu iš to krašto. Jie visa tai darė, ir Aš jais pasibjaurėjau. 
\par 24 Jums gi sakau: paveldėkite jų žemę, kurią jums duosiu, žemę, plūstančią pienu ir medumi. Aš, Viešpats, jūsų Dievas, kuris jus išskyriau iš kitų tautų. 
\par 25 Tad skirkite švarų gyvulį nuo nešvaraus ir švarų paukštį nuo nešvaraus, kad nesusiteptumėte gyvuliais, paukščiais ir visais gyviais, krutančiais žemėje, kuriuos jums nurodžiau esant nešvarius. 
\par 26 Jūs privalote būti man šventi, nes Aš esu šventas Viešpats. Aš jus išskyriau iš kitų tautų, kad būtumėte mano. 
\par 27 Vyras ar moteris, kurie užsiima mirusiųjų dvasių iššaukimu ir žyniavimu, turi mirti; juos užmuškite akmenimis; jie patys kalti dėl savo mirties’ ”.



\chapter{21}


\par 1 Viešpats kalbėjo Mozei: “Sakyk Aarono sūnums, kunigams, kad jie nesusiteptų savo mirusiu tautiečiu, 
\par 2 nebent būtų artimi giminės, tai yra: tėvas, motina, sūnus, duktė ar brolis 
\par 3 ir netekėjusi sesuo, kuri yra jam artima. 
\par 4 Jis neturi susitepti, nes jis yra vyresnysis savo tautoje. 
\par 5 Kunigai neskus savo galvos nei barzdos ir neraižys savo kūno. 
\par 6 Jie turi būti šventi savo Dievui ir nesutepti savo Dievo vardo, nes jie aukoja man smilkalus ir duonos auką, todėl privalo būti šventi. 
\par 7 Jiems draudžiama vesti paleistuvę ir išsiskyrusią, kadangi jie yra pašvęsti man 
\par 8 ir aukoja padėtinę duoną. Jie turi būti šventi, nes Aš esu šventas Viešpats, kuris jus darau šventus. 
\par 9 Jei kunigo duktė paleistuvautų ir tuo suteptų savo tėvo vardą, bus sudeginta. 
\par 10 Vyriausiasis kunigas, ant kurio galvos išlieta patepimo aliejaus ir kuris įšventintas kunigystei bei apvilktas šventais apdarais, nenudengs savo galvos, nepersiplėš drabužių 
\par 11 ir neprisilies prie jokio mirusio, net prie savo tėvo ar motinos. 
\par 12 Jis neišeis iš šventyklos ir nesuteps jos, nes yra pateptas šventu aliejumi; Aš esu Viešpats. 
\par 13 Jis turi vesti mergaitę. 
\par 14 Jis negali vesti našlės, išsiskyrusios, paniekintos nei paleistuvės, bet tik mergaitę iš savo tautos, 
\par 15 kad nesuterštų savo giminės kraujo, nes Aš, Viešpats, jį pašventinu”. 
\par 16 Viešpats kalbėjo Mozei: 
\par 17 “Sakyk Aaronui, kad, jei kas nors iš jo palikuonių turi trūkumų, tegul nesiartina prie Dievo, kad aukotų Jam duoną. 
\par 18 Jeigu jis yra aklas, raišas, iškreiptu veidu ar nesveikų sąnarių 
\par 19 ar yra sulaužyta jo ranka arba koja, 
\par 20 ar kuprotas, ar neūžauga, ar sergąs akių liga, ar šašuotas, ar niežuotas, ar eunuchas,­ 
\par 21 nė vienas iš kunigo Aarono palikuonių, turintis trūkumų, nesiartins aukoti Viešpačiui, 
\par 22 tačiau valgys maistą, aukojamą šventykloje, šventą ir labai šventą. 
\par 23 Bet jis neis į Švenčiausiąją ir nesiartins prie aukuro, nes turi trūkumų, kad nesuteptų mano šventyklos. Aš esu Viešpats, kuris juos pašventinu”. 
\par 24 Mozė tai paskelbė Aaronui, jo sūnums ir visam Izraeliui.



\chapter{22}

\par 1 Viešpats kalbėjo Mozei: 
\par 2 “Sakyk Aaronui ir jo sūnums, kad jie pagarbiai laikytų daiktus, kurie pašvęsti aukai, ir nesuteptų jų. Aš esu Viešpats. 
\par 3 Sakyk jiems ir jų palikuonims: ‘Jei kuris būdamas nešvarus paliestų tai, kas izraelitų pašvęsta aukoti Viešpačiui, bus išnaikintas iš Viešpaties akivaizdos. Aš esu Viešpats. 
\par 4 Kas iš Aarono palikuonių būtų raupsuotas ar nešvarus, nevalgys to, kas man pašvęsta, iki apsivalys. Kas prisiliestų prie mirusio arba prie turėjusio sėklos išsiliejimą, 
\par 5 arba prie šliaužiančio gyvio, arba kurio kito nešvaraus dalyko, kurio prisilietimas sutepa, 
\par 6 bus nešvarus iki vakaro ir nevalgys pašvęstųjų valgių, kol neapsiplaus vandeniu. 
\par 7 Nusileidus saulei, bus švarus ir galės valgyti pašvęstą maistą, nes tai jo maistas. 
\par 8 Jiems nevalia valgyti nugaišusio nei žvėries sudraskyto gyvulio, nes jais susiteps. Aš esu Viešpats. 
\par 9 Tegu jie laikosi mano įsakymų, kad nenusikalstų ir nemirtų. Aš esu Viešpats, kuris juos pašventinu. 
\par 10 Joks ateivis nevalgys švento maisto; jo nevalgys taip pat kunigo įnamis ir samdinys. 
\par 11 Bet kunigo vergas, pirktas už pinigus arba gimęs jo namuose, galės jį valgyti. 
\par 12 Jei kunigo duktė ištekės ne už kunigo, ji nevalgys švento maisto. 
\par 13 Bet, jei ji liktų našlė ar išsiskirtų ir neturėtų vaikų, sugrįžusi į savo tėvo namus, valgys tą maistą, kaip tai darydavo būdama mergaitė. Niekas iš pašalinių neturi teisės to maisto valgyti. 
\par 14 Kas nežiniomis valgytų šventą maistą, pridės penktą dalį prie to, ką suvalgė, ir atiduos kunigui. 
\par 15 Kunigai tenesutepa izraelitų šventų aukų, kurias jie aukoja Viešpačiui, 
\par 16 kad nenusikalstų valgydami šventas dovanas. Aš esu Viešpats, kuris juos pašventinu’ ”. 
\par 17 Viešpats kalbėjo Mozei: 
\par 18 “Sakyk Aaronui, jo sūnums ir visiems izraelitams: ‘Jei kas iš izraelitų ir gyvenančių tarp jų ateivių norėtų aukoti Viešpačiui, ar tai būtų įžado, ar laisvos valios auka, teatneša deginamajai Viešpaties aukai 
\par 19 sveiką patinėlį iš galvijų, avių ar ožkų. 
\par 20 Jei auka turėtų kokią ydą, jūs jos neaukokite, nes tai nepriimtina. 
\par 21 Jei kas aukotų Viešpačiui padėkos auką, ištesėdamas įžadą, ar laisvos valios auką, ji turi būti sveika iš galvijų ar avių. 
\par 22 Aklo, luošo, turinčio žaizdos randą, išberto, šašuoto ar su niežais gyvulio neaukokite Viešpačiui ir nieko iš jo nedeginkite ant aukuro. 
\par 23 Galviją ar avį su ilgesniu ar trumpesniu sąnariu leidžiama aukoti laisva valia, bet įžado aukai­jis nepriimtinas. 
\par 24 Neaukokite Viešpačiui jokio iškastruoto gyvulio; šito niekada nedarykite savo žemėje. 
\par 25 Taip pat nepriimkite ir neaukokite Dievui tokių gyvulių iš ateivio rankos, nes tai yra sugedę ir sutepta; tokia auka nepriimtina’ ”. 
\par 26 Viešpats kalbėjo Mozei: 
\par 27 “Veršiukas, ėriukas ar ožiukas septynias dienas liks prie savo motinos; aštuntą dieną ar vėliau galės būti aukojamas Viešpačiui. 
\par 28 Nepjaukite karvės nei avies tą pačią dieną su jos jaunikliu. 
\par 29 Kai aukosite padėkos auką Viešpačiui, aukokite ją laisva valia. 
\par 30 Ją suvalgykite tą pačią dieną ir nieko nepalikite kitos dienos rytui. Aš esu Viešpats. 
\par 31 Laikykitės mano įsakymų ir juos vykdykite. Aš esu Viešpats. 
\par 32 Nesutepkite mano švento vardo. Aš esu šventas tarp izraelitų ir pašventinu jus. 
\par 33 Aš išvedžiau jus iš Egipto žemės, kad būčiau jūsų Dievas. Aš esu Viešpats”.



\chapter{23}

\par 1 Viešpats kalbėjo Mozei: 
\par 2 “Paskelbk izraelitams Viešpaties šventes, per kurias turite sušaukti susirinkimą. 
\par 3 Šešias dienas dirbkite; septintoji diena yra sabatas, skirta poilsiui ir šventam susirinkimui. Nieko nedirbkite tą dieną, nes tai yra sabatas Viešpačiui, kur jūs begyventumėte. 
\par 4 Šios yra iškilmingos Viešpaties šventės ir jas švęskite joms skirtu metu: 
\par 5 pirmo mėnesio keturioliktos dienos vakare yra Viešpaties Pascha, 
\par 6 o penkioliktą to paties mėnesio dieną yra Neraugintos duonos šventė Viešpačiui. Neraugintą duoną valgykite septynias dienas. 
\par 7 Pirmoji diena bus jums iškilmingiausia ir šventa. Tą dieną nedirbkite jokio darbo. 
\par 8 Septynias dienas aukokite Viešpačiui deginamąją auką. Septintoji susirinkimo diena bus iškilminga ir šventa; tą dieną neturite nieko dirbti”. 
\par 9 Viešpats kalbėjo Mozei: 
\par 10 “Sakyk izraelitams, kad, įėję į kraštą, kurį jiems duosiu, ir pradėję pjauti javus, atneštų pirmąjį pjūties pėdą kunigui, 
\par 11 kuris pirmą dieną po sabato siūbuos pėdą Viešpaties akivaizdoje, kad Jis būtų jums malonus. 
\par 12 Tą pačią dieną, kai pėdas siūbuojamas, papjaukite sveiką metinį avinėlį deginamajai aukai Viešpačiui. 
\par 13 Drauge su juo aukokite duonos auką iš dviejų dešimtųjų efos smulkių miltų, apšlakstytų aliejumi, kad būtų malonus kvapas Viešpačiui; taip pat geriamajai aukai aukokite vyno ketvirtadalį hino. 
\par 14 Nei duonos, nei paskrudintų grūdų, nei kruopų nevalgykite, kol nepaaukosite Dievui. Tai amžinas įsakymas visoms kartoms, kur jūs begyventumėte. 
\par 15 Atskaitykite septynias savaites po to sabato, pradėdami pirmąja diena, kai atnešėte aukoti pirmąją derliaus pėdą. 
\par 16 Penkiasdešimtąją dieną­pirmąją dieną po septinto sabato­vėl aukokite Viešpačiui duonos auką. 
\par 17 Iš savo gyvenviečių atneškite du raugintos duonos kepalus, iškeptus iš dviejų dešimtųjų efos smulkių miltų; tai yra pirmieji vaisiai Viešpačiui. 
\par 18 Kartu su duonos kepalais aukokite septynis sveikus metinius avinėlius, vieną veršį ir du avinus,­tai bus deginamoji auka Viešpačiui, drauge su duonos ir geriamąja auka, kad būtų malonus kvapas. 
\par 19 Aukokite taip pat ožį aukai už nuodėmę ir du metinius avinėlius padėkos aukai. 
\par 20 Kunigas juos siūbuos drauge su pirmųjų vaisių duonos kepalais Viešpaties akivaizdoje; po to visa tai bus šventa ir priklausys kunigams. 
\par 21 Paskelbkite šitą dieną iškilminga ir šventa susirinkimo diena; tą dieną nedirbkite jokio darbo. Tai bus amžinas įstatymas visoms jūsų kartoms, kur jūs begyventumėte. 
\par 22 Pjaudami savo javus, nenupjaukite jų iki lauko krašto ir nerinkite nukritusių varpų; jas palikite beturčiams ir ateiviams. Aš esu Viešpats, jūsų Dievas”. 
\par 23 Viešpats kalbėjo Mozei: 
\par 24 “Sakyk izraelitams, kad septinto mėnesio pirmoji diena bus jums iškilminga poilsio diena­sabatas. Trimitų garsai jums tai primins, ji bus švento susirinkimo diena. 
\par 25 Nedirbkite tą dieną jokio darbo ir aukokite Viešpačiui deginamąją auką”. 
\par 26 Viešpats kalbėjo Mozei: 
\par 27 “Dešimtoji to paties mėnesio diena bus iškilminga sutaikinimo diena. Tai bus švento susirinkimo diena; varginkite savo sielas tą dieną ir aukokite Viešpačiui deginamąją auką. 
\par 28 Jokio darbo nedirbkite, nes tai sutaikinimo diena, kad jūs būtumėte sutaikinti su Viešpačiu, jūsų Dievu. 
\par 29 Kiekvienas žmogus, kuris nevargintų savo sielos tą dieną, bus išnaikintas iš savo tautos. 
\par 30 Kas dirbtų bet kokį darbą tą dieną, bus išnaikintas iš savo tautos. 
\par 31 Todėl nieko tą dieną nedirbkite. Tai bus amžinas įstatymas visoms jūsų kartoms, kur jūs begyventumėte. 
\par 32 Tai bus jums poilsio sabatas, kurio metu varginsite savo sielas. Nuo devintos mėnesio dienos vakaro iki kito vakaro švęsite sabatą”. 
\par 33 Viešpats kalbėjo Mozei: 
\par 34 “Sakyk izraelitams, kad nuo penkioliktos to paties mėnesio dienos septynias dienas truks Palapinių šventė. 
\par 35 Pirmoji diena bus iškilminga ir šventa; tą dieną nedirbsite jokio darbo. 
\par 36 Septynias dienas aukosite Viešpačiui deginamąsias aukas. Aštuntoji diena bus šventa. Tuomet aukosite Viešpačiui deginamąją auką, nes tai susirinkimo diena; nedirbsite tą dieną jokio darbo. 
\par 37 Šitos yra Viešpačiui skirtos šventės, per kurias sušauksite šventą susirinkimą ir aukosite Viešpačiui deginamąsias, duonos ir geriamąsias aukas, kiekvieną jai nustatytu metu, 
\par 38 neskaitant sabato, įžado ir laisvos valios aukų, kurias atnešate Viešpačiui. 
\par 39 Taigi nuo septinto mėnesio penkioliktos dienos, kai surinksite visą savo laukų derlių, paskirkite Viešpaties garbei septynias dienas; pirmoji ir aštuntoji diena bus sabatas. 
\par 40 Pirmą dieną imkite gražiausių medžių vaisių, palmių šakelių, lapuoto medžio ir paupio gluosnių šakų ir linksminkitės Viešpaties, jūsų Dievo, akivaizdoje. 
\par 41 Kasmet iškilmingai švęskite Viešpačiui septynias dienas; tai bus amžinas įstatymas jūsų kartoms. Tą šventę švęskite septintą mėnesį. 
\par 42 Visi izraelitai turi gyventi palapinėse septynias dienas, 
\par 43 kad jūsų palikuonys žinotų, jog buvau apgyvendinęs izraelitus palapinėse, kai juos išvedžiau iš Egipto žemės. Aš Viešpats, jūsų Dievas”. 
\par 44 Taip Mozė paskelbė izraelitams šventes Viešpaties garbei.



\chapter{24}


\par 1 Viešpats kalbėjo Mozei: 
\par 2 “Įsakyk izraelitams, kad jie atneštų tyriausio alyvmedžių aliejaus, kad lempos nuolat degtų 
\par 3 Susitikimo palapinėje šiapus uždangos į Švenčiausiąją. Jas uždegs Aaronas kiekvieną vakarą ir prižiūrės, kad jos degtų iki ryto Viešpaties akivaizdoje. Ši tarnystė bus amžinas įstatymas visoms jūsų kartoms. 
\par 4 Lempos stovės ant žvakidės iš gryno aukso Viešpaties akivaizdoje. 
\par 5 Imk smulkių miltų ir iš jų iškepk dvylika duonos kepalėlių, kiekvieną iš dviejų dešimtųjų efos, 
\par 6 juos padėk ant gryno aukso stalo Viešpaties akivaizdoje dviem eilėm, po šešis kiekvienoje eilėje. 
\par 7 Ant kiekvienos eilės padėk tyriausių smilkalų, kad jie būtų ant duonos, kuri sudeginama kaip atminimas Viešpačiui. 
\par 8 Kas sabatą ji bus keičiama. Izraelitams tai bus amžinas įstatymas. 
\par 9 Visa tai priklausys Aaronui ir jo sūnums, jie valgys ją šventoje vietoje, nes tai švenčiausias dalykas iš Viešpaties deginamųjų aukų pagal amžinąjį įstatymą”. 
\par 10 Izraelitės moters sūnus, kurio tėvas buvo egiptietis, susiginčijo su izraelitu stovykloje. 
\par 11 Jį atvedė pas Mozę, kadangi jis piktžodžiavo ir keikė Dievą. Jo motina buvo Šelomita, Dibrio duktė iš Dano giminės. 
\par 12 Jį uždarė ir laukė Viešpaties nurodymo. 
\par 13 Viešpats įsakė Mozei: 
\par 14 “Išvesk piktžodžiautoją iš stovyklos. Visi, kurie girdėjo jį piktžodžiaujant, teuždeda savo rankas ant jo galvos, o visi žmonės teužmuša jį akmenimis. 
\par 15 Izraelitams sakyk: ‘Kas keikia Dievą, atsakys už savo nusikaltimą. 
\par 16 Kas piktžodžiaus Viešpaties vardui, turės mirti­jis bus užmuštas akmenimis, ar jis bus vietinis, ar ateivis. Kas piktžodžiaus Viešpaties vardui, bus baudžiamas mirtimi. 
\par 17 Kas nužudo žmogų, taip pat turi mirti. 
\par 18 Jei kas užmuša gyvulį, atlygins gyvuliu už gyvulį. 
\par 19 Kas sužalotų savo artimą, jam bus padaryta taip, kaip jis padarė: 
\par 20 sulaužymas už sulaužymą, akis už akį, dantis už dantį. 
\par 21 Užmušęs gyvulį, atlygins; užmušęs žmogų, bus baudžiamas mirtimi. 
\par 22 Tas pats įstatymas galioja ateiviui ir vietiniam, nes Aš esu Viešpats, jūsų Dievas’ ”. 
\par 23 Po to izraelitai išvedė piktžodžiautoją iš stovyklos ir užmušė jį akmenimis. Jie padarė, kaip Viešpats įsakė Mozei.
Online Parallel Study Bible



\chapter{25}

\par 1 Viešpats kalbėjo Mozei Sinajaus kalne: 
\par 2 “Sakyk izraelitams, kad, įėję į šalį, kurią jiems duosiu, septintaisiais metais leistų žemei ilsėtis. 
\par 3 Šešerius metus apsėkite dirvą, karpykite vynuogyną ir rinkite vaisius. 
\par 4 Septintieji metai bus sabato poilsis žemei, metai, paskirti Viešpačiui. Tada laukų neapsėkite ir vynuogyno nekarpykite. 
\par 5 Ką žemė savaime užaugins, nepjaukite ir vynuogių nuo neapkarpytų vynmedžio šakelių nerinkite, nes tai yra poilsio metai žemei. 
\par 6 Maistas jums, tarnams, tarnaitėms, samdiniams ir ateiviams, kurie pas jus gyvena, 
\par 7 taip pat jūsų gyvuliams bus visa, kas savaime užaugs. 
\par 8 Skaičiuok septynis kartus po septynerius metus, iš viso keturiasdešimt devynerius metus. 
\par 9 Paskutinių metų septinto mėnesio dešimtą dieną, sutaikinimo dieną, trimituokite visame krašte 
\par 10 ir paskelbkite, kad penkiasdešimtieji metai yra jubiliejaus metai ir visiems krašto gyventojams suteikiama laisvė. Kiekvienas turės teisę atgauti nuosavybę ir grįžti į savo šeimą. 
\par 11 Nesėkite ir nepjaukite, kas savaime užaugs dirvose, uogų nerinkite neapkarpytuose vynuogynuose, 
\par 12 nes tai iškilmingi jubiliejaus metai. Valgykite tai, kas savaime užaugs. 
\par 13 Jubiliejaus metais kiekvienas atgaus savo buvusią nuosavybę. 
\par 14 Parduodant ar perkant ką nors iš kaimyno, neskriauskite vienas kito, 
\par 15 bet pirkite, atsižvelgdami į metų skaičių, praėjusį nuo jubiliejaus metų. 
\par 16 Kuo daugiau bus likę derliaus metų, tuo aukštesnė bus kaina; kuo mažiau derliaus metų, tuo mažesnė bus pardavimo kaina, nes jis tau parduos pagal pjūčių skaičių. 
\par 17 Neskriauskite vienas kito, būkite sąžiningi, nes Aš esu Viešpats, jūsų Dievas. 
\par 18 Laikykitės mano įsakymų ir paliepimų, vykdykite juos, kad gyventumėte be baimės 
\par 19 ir žemė duotų vaisių, kuriais pasisotintumėte ir laimingai gyventumėte. 
\par 20 Gal klausite: ‘Ką valgysime septintaisiais metais, jei nesėsime ir nerinksime vaisių?’ 
\par 21 Aš palaiminsiu šeštuosius metus, ir jie duos jums trejų metų derlių. 
\par 22 Sėsite aštuntaisiais metais, maitinsitės turimomis atsargomis ligi devintųjų metų, kol naujas derlius užaugs. 
\par 23 Žemė nebus parduodama visam laikui, nes ji mano, o jūs esate tik svečiai ir ateiviai. 
\par 24 Todėl visa jūsų žemė bus parduodama su išpirkimo teise. 
\par 25 Jei suvargęs tavo brolis parduotų nuosavybės dalį, jo giminaitis gali ją išpirkti. 
\par 26 O jei jis neturėtų giminaičio, bet pats galėtų išpirkti, 
\par 27 bus nustatyta kaina už tą laiką, kai ji buvo parduota, o kas lieka, sugrąžins pirkėjui, ir taip atgaus savo nuosavybę. 
\par 28 Jei neturės reikiamos sumos žemei išpirkti, pirkėjas turės, ką nupirko, ligi jubiliejaus metų; tais metais nuosavybė sugrįš pirmajam šeimininkui. 
\par 29 Kas parduotų namus miesto ribose, tam bus leista išsipirkti juos per vienerius metus. 
\par 30 Jei neišsipirks per metus, jie pasiliks pirkėjui ir jo palikuonims; jų nereikės grąžinti jubiliejaus metais. 
\par 31 O jei namai būtų kaime, kuris neturi apsaugos sienų, bus parduodami kaip laukai; jei nebus išpirkti anksčiau, jubiliejaus metais sugrįš buvusiam savininkui. 
\par 32 Levitų namai, esantys jų miestuose, visada gali būti išpirkti. 
\par 33 Jei nebus išpirkti, jubiliejaus metais sugrįš savininkams, nes namai levitų miestuose yra jų nuosavybė tarp izraelitų. 
\par 34 Jų priemiesčio laukai neparduodami, nes tai jų amžina nuosavybė. 
\par 35 Jei tavo brolis izraelitas taptų beturčiu, jį priimk ir juo rūpinkis, nors jis tau būtų svetimas. 
\par 36 Neimk iš jo palūkanų, bijok Dievo. Tegul tavo brolis gyvena pas tave. 
\par 37 Neskolink jam pinigų už palūkanas ir neduok maisto, siekdamas pasipelnyti. 
\par 38 Aš, Viešpats, jūsų Dievas, kuris jus išvedžiau iš Egipto šalies, kad jums duočiau Kanaano žemę ir būčiau jūsų Dievas. 
\par 39 Jei vargo priverstas tavo brolis pats tau parsiduotų, nesielk su juo kaip su vergu. 
\par 40 Tetarnauja jis tau kaip samdinys ir įnamis ligi jubiliejaus metų. 
\par 41 Tuomet jis su savo vaikais grįš pas savo giminę į tėvų namus. 
\par 42 Izraelitai yra mano tarnai, Aš juos išvedžiau iš Egipto; jie neturi būti parduodami kaip vergai. 
\par 43 Neišnaudokit jų, bijokit Dievo. 
\par 44 Vergą ir vergę pirkite iš aplink jus esančių tautų, 
\par 45 iš ateivių, pasilikusių gyventi tarp jūsų, ir iš jūsų žemėje gimusių vergų; šiuos galėsite turėti vergais. 
\par 46 Juos galėsite palikti kaip paveldėjimą savo palikuonims per amžius. Savo brolių izraelitų nespauskite be gailesčio. 
\par 47 Jei tarp jūsų pralobtų ateivis ar svetimtautis ir jam ar kuriam nors iš jo palikuonių parsiduotų nuskurdęs tavo brolis, 
\par 48 jis galės būti išpirktas. Jei norės, jį galės išpirkti kas nors iš jo brolių: 
\par 49 dėdė, brolėnas arba artimas giminaitis; o jei pajėgs, ir jis pats galės išsipirkti. 
\par 50 Jis skaitys metus nuo parsidavimo ligi jubiliejaus metų ir atskaitys iš sumos, už kurią buvo parsidavęs, lyg būtų buvęs samdinys. 
\par 51 Jei bus likę daug metų ligi jubiliejaus, pagal juos atsilygins šeimininkui. 
\par 52 Jei metų skaičius mažesnis, apskaičiuos su juo pagal likusių metų skaičių ir sumokės išpirkimo kainą. 
\par 53 Jis turi būti kaip samdinys visus tuos metus ir šeimininkas nesielgs su juo be gailesčio tavo akivaizdoje. 
\par 54 O jei jis nebus išpirktas, jis išeis jubiliejaus metais kartu su savo vaikais. 
\par 55 Izraelitai yra mano tarnai, kuriuos išvedžiau iš Egipto. Aš esu Viešpats, jūsų Dievas”.



\chapter{26}

\par 1 “Nedirbkite sau stabų nei drožinių, nestatykite savo žemėje statulų nei akmeninių atvaizdų ir jų negarbinkite, nes Aš esu Viešpats, jūsų Dievas. 
\par 2 Švęskite sabatus ir gerbkite mano šventyklą. Aš esu Viešpats. 
\par 3 Jei elgsitės pagal mano įstatymus, laikysitės jų ir juos vykdysite, 
\par 4 duosiu jums lietaus tinkamu metu, ir žemė duos derlių, ir medžiai bus pilni vaisių. 
\par 5 Javų kūlimas tęsis ligi vynuogių rinkimo ir vynuogių rinkimas­ligi sėjos; valgysite duonos sočiai ir gyvensite saugiai krašte. 
\par 6 Aš duosiu taiką jūsų žemei ir jūs gulsite nieko nebijodami. Pašalinsiu plėšriuosius žvėris ir karo nebus jūsų krašte. 
\par 7 Vysite priešus, ir jie kris jūsų akivaizdoje. 
\par 8 Penkiese vysite šimtą, ir šimtas­dešimt tūkstančių; jūsų priešai kris nuo kardo. 
\par 9 Aš pažvelgsiu į jus, padarysiu jus vaisingus ir padauginsiu jus. Aš patvirtinsiu su jumis savo sandorą. 
\par 10 Metų derliaus užteks iki naujos pjūties; likutį turėsite pašalinti, kad turėtumėte vietos naujam derliui. 
\par 11 Gyvensiu tarp jūsų ir mano siela nesibjaurės jumis. 
\par 12 Vaikščiosiu tarp jūsų ir būsiu jūsų Dievas, o jūs būsite mano tauta. 
\par 13 Aš, Viešpats, jūsų Dievas, kuris jus išvedžiau iš Egipto žemės, kad nevergautumėte. Aš sutraukiau jūsų jungo pančius, kad jūs vaikščiotumėte atsitiesę. 
\par 14 O jei manęs neklausysite ir nevykdysite mano įsakymų, 
\par 15 niekinsite mano įstatymus, nenorėdami daryti to, kas mano įsakyta, ir taip sulaužysite mano sandorą, 
\par 16 štai ką Aš jums padarysiu: išgąstis ir ligos apakins ir naikins jus. Veltui sėsite, nes derlių suvalgys jūsų priešai. 
\par 17 Atgręšiu į jus savo veidą, ir jūs būsite žudomi savo priešų akivaizdoje. Jus pavergs tie, kurių jūs nekenčiate, ir jūs bėgsite niekam jūsų nevejant. 
\par 18 O jei ir tada neklausysite, padidinsiu jums bausmę septyneriopai 
\par 19 ir palaušiu jūsų išdidumą. Dangų virš jūsų padarysiu kaip geležį ir žemę­kaip varį. 
\par 20 Niekais nueis jūsų darbas: žemė neduos derliaus ir medžiai neneš vaisių. 
\par 21 Jei ir toliau man priešinsitės ir nenorėsite klausyti, dar septyneriopai padidinsiu jums bausmę pagal jūsų nuodėmes. 
\par 22 Siųsiu laukinius žvėris, ir jie žudys jūsų vaikus ir gyvulius taip, kad keliai ištuštės. 
\par 23 O jei ir tada nesikeisite ir vis priešinsitės, 
\par 24 tai ir Aš jums priešinsiuos ir bausiu jus dar septynis kartus stipriau; 
\par 25 leisiu kardui naikinti jus už sandoros laužymą. Kai subėgsite į miestus, siųsiu marą, ir jūs būsite priversti pasiduoti priešui. 
\par 26 Kai Aš sulaužysiu jūsų duonos ramstį, dešimt moterų keps duoną vienoje krosnyje ir dalins ją pagal svorį; valgysite ir nepasisotinsite. 
\par 27 O jei ir tada manęs neklausysite ir priešinsitės, 
\par 28 tai mano bausmė bus dar septyneriopai didesnė negu anksčiau. 
\par 29 Tada jūs valgysite savo sūnų ir dukterų kūnus. 
\par 30 Išardysiu jūsų stabų garbinimo aukštąsias vietas ir numesiu jūsų lavonus ant jūsų stabų liekanų, ir bjaurėsiuos jumis. 
\par 31 Jūsų miestus paversiu tyrais, šventyklas padarysiu tuščias ir daugiau nebepriimsiu jūsų aukų. 
\par 32 Sunaikinsiu jūsų kraštą taip, kad net jūsų priešai, kurie jame gyvens, stebėsis. 
\par 33 Karais jus išsklaidysiu tarp tautų; jūsų kraštas bus paverstas dykuma, miestai sugriauti. 
\par 34 Tada žemė džiaugsis sabato poilsiu, kai bus apleista jums esant priešų žemėje. 
\par 35 Ji ilsėsis, kol bus apleista, nes nepailsėjo jūsų sabato metu, kai joje gyvenote. 
\par 36 Priešų krašte išlikusiųjų širdis pripildysiu baime; juos gąsdins krintančio lapo šlamėjimas ir jie bėgs kaip nuo kardo niekam nesivejant, 
\par 37 ir sukniubs vieni ant kitų, lyg bėgdami iš kovos lauko. Niekas iš jūsų nedrįs priešintis, 
\par 38 žūsite išsklaidyti priešų tautose, ir svetima žemė jus sunaikins. 
\par 39 Išlikusieji merdės priešų žemėje ir kentės už savo tėvų ir savąsias nuodėmes, 
\par 40 kol neišpažins savųjų ir savo tėvų nedorybių ir nusikaltimų ir kad jie priešinosi man, 
\par 41 už ką Aš priešinausi jiems ir išvedžiau juos į priešų žemę. Jei jų neapipjaustytos širdys nusižemins ir priims bausmę už savo kaltes, 
\par 42 Aš prisiminsiu sandorą, kurią padariau su Jokūbu, Izaoku ir Abraomu, ir prisiminsiu žemę, 
\par 43 kuri jų palikta džiaugiasi sabato poilsiu. Jie gi atsitrauks nuo savo nuodėmių, kai supras, kad buvo atmetę mano įsakymus ir paniekinę įstatymus. 
\par 44 Tačiau, jiems esant priešų žemėje, Aš neišsižadėsiu jų ir nesibjaurėsiu taip, kad jie būtų visiškai sunaikinti, ir savo sandoros su jais neišardysiu. Aš esu Viešpats, jų Dievas. 
\par 45 Tada atsiminsiu dėl jų sandorą, kurią sudariau su jų protėviais, kai juos išvedžiau iš Egipto, kad būčiau jų Dievas, nes Aš esu Viešpats”. 
\par 46 Šituos nuostatus ir įstatymus izraelitams Viešpats davė per Mozę Sinajaus kalne.



\chapter{27}

\par 1 Viešpats kalbėjo Mozei: 
\par 2 “Sakyk izraelitams: ‘Jei kas padarytų įžadą ir pasižadėtų tarnauti Dievui, mokės nustatytą kainą. 
\par 3 Vyras nuo dvidešimties ligi šešiasdešimties metų mokės penkiasdešimt šekelių sidabro pagal šventyklos šekelį; 
\par 4 moteris­trisdešimt. 
\par 5 Vyras nuo penkerių ligi dvidešimties metų­dvidešimt šekelių sidabro, o moteris­dešimt. 
\par 6 Už berniuką nuo vieno mėnesio ligi penkerių metų­penkis šekelius sidabro, už mergaitę­tris. 
\par 7 Vyras šešiasdešimties metų ir vyresnis mokės penkiolika šekelių sidabro, moteris­dešimt. 
\par 8 Jei bus beturtis ir paskirtos sumos negalės sumokėti, kreipsis į kunigą, kuris nustatys, kiek tas asmuo gali duoti. 
\par 9 Jei kas pažadėtų duoti gyvulį, tinkamą aukoti Viešpačiui, tai tas pažadėtasis gyvulys 
\par 10 negalės būti pakeistas nei geresniu, nei blogesniu. O jei pakeistų, ir pakeistasis, ir pirmykštis priklausys Viešpačiui. 
\par 11 Jei kas pažadėtų duoti nešvarų gyvulį, kuris netinka aukai Viešpačiui, jis bus atvestas pas kunigą, 
\par 12 kuris nuspręs, ar jis tinka, ir nustatys jo kainą. 
\par 13 Jei aukojantis norės jį išpirkti, pridės prie įkainojimo sumos dar penktą dalį. 
\par 14 Jei kas pažadėtų aukoti namus, pašvęsdamas juos Viešpačiui, kunigas apžiūrės, ar jie tinkami, ir nustatys jų vertę. 
\par 15 O jei aukotojas norėtų juos išpirkti, pridės prie nustatytos sumos dar penktą dalį ir namai bus jo. 
\par 16 Jei kas pažadėtų aukoti savo paveldėtą žemę, pašvęsdamas ją Viešpačiui, jos vertė bus nustatyta pagal sėją. Jei žemė apsėjama vienu homeru miežių, ji bus įkainota penkiasdešimt šekelių sidabro. 
\par 17 Jei jis pašvenčia dirvą tuojau po jubiliejaus metų, jos vertė bus visa. 
\par 18 O jei ji pašvenčiama kuriam laikui praėjus po jubiliejaus metų, kunigas apskaičiuos jos vertę pagal metų skaičių, likusį ligi jubiliejaus metų. 
\par 19 Jei aukotojas panorės išpirkti ją, pridės penktą dalį prie paskirtos kainos ir ją atsiims. 
\par 20 Jei nenorėtų išpirkti ir ji bus parduota kitam, aukotojas nebegalės vėliau jos išpirkti, 
\par 21 nes, atėjus jubiliejaus metams, ji bus pašvęsta Viešpačiui, o pašvęsti dalykai priklauso kunigams. 
\par 22 Jei pirktinė, o ne paveldėta dirva būtų pažadėta Viešpačiui, 
\par 23 kunigas apskaičiuos jos vertę pagal skaičių metų ligi jubiliejaus, ir aukotojas tą sumą paaukos Viešpačiui. 
\par 24 Jubiliejaus metais ji grįš pirmajam savininkui, kuris ją valdė kaip savo paveldo dalį. 
\par 25 Kiekvienas įkainojimas bus daromas pagal šventyklos šekelį. Šekelį sudaro dvidešimt gerų. 
\par 26 Pirmagimių gyvulių, kurie priklauso Viešpačiui, niekas negalės pašvęsti: ar tai bus jautis, ar avis,­jie yra Viešpaties. 
\par 27 Jei gyvulys yra nešvarus, aukotojas išpirks jį pagal įkainojimą ir pridės penktą kainos dalį; jei išpirkti nenorėtų, bus parduotas už tiek, kiek buvo įkainotas. 
\par 28 Visa, kas Viešpačiui pašvęsta,­ar tai žmogus, ar gyvulys, ar dirva­nebus parduodama ir negali būti išpirkta. Visi Viešpačiui pašvęsti dalykai yra Jo. 
\par 29 Nieko iš žmonių, kurie yra pašvęsti, negalima išpirkti, bet jie turi mirti. 
\par 30 Visos javų ir vaisių derliaus dešimtinės yra Viešpaties ir Jam priklauso. 
\par 31 Jei kas norėtų išpirkti savo dešimtines, pridės penktą jų vertės dalį. 
\par 32 Galvijų, avių, ožkų ir viso, kas piemenų ganoma, kiekviena dešimta galva bus pašvęsta Viešpačiui. 
\par 33 Gyvuliai nebus atrenkami ir nebus vienas kitu pakeičiami. Jei kas pakeistų, tai abu gyvuliai bus pašvęsti Viešpačiui ir nebus išperkami’ ”. 
\par 34 Šituos įsakymus Viešpats davė Mozei dėl Izraelio vaikų Sinajaus kalne.



\end{document}