\begin{document}

\title{Pirmasis laiškas Timotiejui}

\chapter{1}


\par 1 Paulius, Jėzaus Kristaus apaštalas, Dievo, mūsų Gelbėtojo, ir Viešpaties Jėzaus Kristaus, mūsų vilties, paliepimu,­ 
\par 2 Timotiejui, tikram sūnui tikėjime: malonė, gailestingumas ir ramybė nuo Dievo, mūsų Tėvo, ir Jėzaus Kristaus, mūsų Viešpaties! 
\par 3 Dar keliaudamas į Makedoniją, aš prašiau tave pasilikti Efeze, kad įsakytum kai kuriems, jog jie nemokytų kitaip, 
\par 4 nekreiptų dėmesio į pasakas ir į begalines giminystės eiles, kurios veikiau sukelia ginčus negu dievišką ugdymą, esantį tikėjime. 
\par 5 Įsakymo tikslas yra meilė iš tyros širdies, geros sąžinės ir nuoširdaus tikėjimo. 
\par 6 Kai kurie, nuklydę nuo šių dalykų, paskendo tuščiuose plepaluose. 
\par 7 Jie norėtų būti įstatymo mokytojais, bet nesupranta nei ką kalba, nei ką tvirtina. 
\par 8 O mes žinome, jog įstatymas geras, jeigu kas teisingai juo naudojasi, 
\par 9 suprasdamas, kad įstatymas nėra skirtas teisiajam, bet nusikaltėliams ir neklusniems, bedieviams ir nusidėjėliams, nešventiems ir šventvagiškiems, tėvažudžiams ir motinžudžiams, žmogžudžiams, 
\par 10 paleistuviams, homoseksualistams, vergų pirkliams, melagiams, priesaikos laužytojams ir viskam, kas priešinga sveikam mokymui 
\par 11 pagal palaimintojo Dievo šlovingąją Evangeliją, kuri man yra patikėta. 
\par 12 Aš dėkoju mūsų Viešpačiui Kristui Jėzui, kuris įgalino mane, nes palaikė mane ištikimu, paskirdamas tarnavimui, 
\par 13 nors anksčiau buvau piktžodžiautojas, persekiotojas ir akiplėša. Manęs buvo pasigailėta, nes taip elgiausi dėl neišmanymo ir netikėjimo. 
\par 14 Bet mūsų Viešpaties malonė išsiliejo be saiko su tikėjimu ir meile Kristuje Jėzuje. 
\par 15 Tikras žodis ir vertas visiško pritarimo, kad Kristus Jėzus atėjo į pasaulį išgelbėti nusidėjėlių, kurių pirmasis esu aš. 
\par 16 Todėl ir buvo manęs pasigailėta, kad manyje pirmame Jėzus Kristus parodytų visą savo kantrybę, duodamas pavyzdį tiems, kurie Jį įtikės amžinajam gyvenimui. 
\par 17 Amžių Karaliui, nemirtingajam, neregimajam, vieninteliam išmintingajam Dievui tebūna garbė ir šlovė per amžių amžius! Amen. 
\par 18 Šį įsakymą perduodu tau, sūnau Timotiejau, pagal ankstesnes pranašystes apie tave, kad, jomis remdamasis, kovotum gerą kovą, 
\par 19 turėdamas tikėjimą ir gerą sąžinę. Jos atsižadėjus, kai kurių tikėjimo laivas sudužo. 
\par 20 Iš jų yra Himenėjas ir Aleksandras, kuriuos atidaviau šėtonui, kad jie pasimokytų nebepiktžodžiauti.


\chapter{2}


\par 1 Taigi visų pirma prašau atlikinėti prašymus, maldas, užtarimus ir dėkojimus už visus žmones, 
\par 2 už karalius bei visus valdininkus, kad tyliai ir ramiai gyventume visokeriopai dievotą ir kilnų gyvenimą. 
\par 3 Tai gera ir priimtina akyse Dievo, mūsų Gelbėtojo, 
\par 4 kuris trokšta, kad visi žmonės būtų išgelbėti ir pasiektų tiesos pažinimą. 
\par 5 Nes yra vienas Dievas ir vienas Dievo ir žmonių Tarpininkas­ žmogus Kristus Jėzus, 
\par 6 kuris atidavė save kaip išpirką už visus, kad būtų paliudytas reikiamu metu. 
\par 7 Tam aš esu paskirtas pamokslininkas ir apaštalas,­sakau tiesą Kristuje, nemeluoju,­tikėjimo ir tiesos mokytojas pagonims. 
\par 8 Taigi trokštu, kad vyrai melstųsi visur, keldami aukštyn šventas rankas, be pykčio ir abejonių. 
\par 9 Taip pat, kad moterys puoštųsi kukliais ir padoriais drabužiais, droviai ir santūriai, ne supintais plaukais ar auksu, ar perlais, ar brangiu drabužiu, 
\par 10 bet,­kaip dera moterims, pasižyminčioms dievobaimingumu,­ gerais darbais. 
\par 11 Moteris tesimoko tyliai, su visišku paklusnumu. 
\par 12 Neleidžiu, kad moteris mokytų nei kad vadovautų vyrui,­ji tesilaiko tyliai. 
\par 13 Juk pirmas buvo sutvertas Adomas, o paskui­Ieva. 
\par 14 Ir ne Adomas buvo apgautas, o moteris buvo apgauta ir nusidėjo. 
\par 15 Bet ji bus išgelbėta, gimdydama vaikus,­jeigu jie išlaikys tikėjimą, meilę, šventumą ir susilaikymą.


\chapter{3}


\par 1 Štai tikras žodis: jei kas siekia vyskupauti, trokšta gero darbo. 
\par 2 O vyskupas privalo būti nepeiktinas, vienos žmonos vyras, santūrus, protingas, padorus, svetingas, sugebantis mokyti; 
\par 3 ne girtuoklis, ne kivirčius, ne geidžiantis nešvaraus pelno, bet švelnus, nemėgstantis vaidytis, negodus pinigų, 
\par 4 geras savo namų šeimininkas, turintis klusnius ir tikrai dorus vaikus. 
\par 5 Jei kas nesugeba šeimininkauti savo namuose, kaipgi jis rūpinsis Dievo bažnyčia? 
\par 6 Vyskupas neturi būti naujatikis, kad nepasididžiuotų ir nepakliūtų į pasmerkimą kaip velnias. 
\par 7 Be to, jam privalu turėti gerą liudijimą tarp pašalinių, kad nebūtų jam priekaištaujama ir neįkliūtų į velnio pinkles. 
\par 8 Taip pat ir diakonai privalo būti garbingi, ne dviliežuviai, ne besaikiai vyno gėrėjai, ne geidžiantys nešvaraus pelno, 
\par 9 saugantys tikėjimo paslaptį tyroje sąžinėje. 
\par 10 Ir pirmiausia jie turi būti išbandyti, o paskui, jei pasirodys esą be kaltės, tegul diakonauja. 
\par 11 Jų žmonos taip pat privalo būti garbingos, ne šmeižikės, santūrios ir ištikimos visame kame. 
\par 12 Diakonai tebūna vienos žmonos vyrai, gerai prižiūrintys vaikus ir savo pačių namus. 
\par 13 Gerai diakonaujantys įgyja garbingą laipsnį ir didelę tikėjimo drąsą Kristuje Jėzuje. 
\par 14 Aš tau rašau apie šiuos dalykus, nors tikiuosi greit atvykti pas tave. 
\par 15 Bet jeigu užtrukčiau, noriu, kad žinotum, kaip reikia elgtis Dievo namuose, kurie yra gyvojo Dievo bažnyčia, tiesos ramstis ir pamatas. 
\par 16 Ir neginčijamai yra didelė dievotumo paslaptis: Dievas buvo apreikštas kūne, išteisintas Dvasioje, matomas angelų, paskelbtas pagonims, įtikėtas pasaulyje ir paimtas į šlovę.


\chapter{4}


\par 1 Dvasia aiškiai sako, kad paskutiniais laikais kai kurie atsitrauks nuo tikėjimo, pasidavę klaidinančioms dvasioms ir demonų mokymams, 
\par 2 veidmainingiems melo skelbėjams, turintiems sudegintą sąžinę, 
\par 3 draudžiantiems tuoktis, liepiantiems susilaikyti nuo maisto, kurį sukūrė Dievas, kad jį su padėka priimtų tikintieji ir pažinusieji tiesą. 
\par 4 Kiekvienas gi Dievo kūrinys yra geras, ir niekas neatmestina, jei priimama su padėka, 
\par 5 nes tai pašventinama Dievo žodžiu ir malda. 
\par 6 Jeigu šitaip aiškinsi broliams, būsi geras Jėzaus Kristaus tarnas, besimaitinantis tikėjimo žodžiais ir geru mokymu, kuriuo tu rūpestingai pasekei. 
\par 7 Bedieviškų ir bobiškų pasakų venk, o ugdykis dievotumą. 
\par 8 Kūno lavinimas nedaug naudos duoda, o dievotumas viskam naudingas; jis turi esamojo ir būsimojo gyvenimo pažadą. 
\par 9 Šis žodis yra tikras, vertas visiško pritarimo. 
\par 10 Mes todėl ir triūsiame bei kenčiame panieką, nes pasitikime gyvuoju Dievu, kuris yra visų žmonių, o ypač tikinčiųjų, Gelbėtojas. 
\par 11 Taip įsakinėk ir mokyk! 
\par 12 Niekas tegul neniekina tavo jaunystės. Tiktai pats būk tikintiesiems pavyzdys žodžiu, elgesiu, meile, dvasia, tikėjimu, skaistumu. 
\par 13 Kol atvyksiu, užsiiminėk skaitymu, raginimu, mokymu. 
\par 14 Neapleisk tavyje esančios dovanos, kuri tau buvo suteikta per pranašystę su vyresniųjų rankų uždėjimu. 
\par 15 Mąstyk apie šituos dalykus, atsidėk jiems visiškai, kad tavo pažanga būtų visiems akivaizdi. 
\par 16 Žiūrėk savęs ir mokymo, būk pastovus tame. Taip išgelbėsi save ir savo klausytojus.


\chapter{5}

 
\par 1 Senesnio žmogaus aštriai nebark, bet ragink kaip tėvą, o jaunesniuosius­kaip brolius, 
\par 2 vyresnio amžiaus moteris­kaip motinas, o jaunesnes­kaip seseris, laikydamasis visiško skaistumo. 
\par 3 Gerbk našles, kurios yra tikros našlės. 
\par 4 Jei kuri našlė turi vaikų ar vaikaičių, tegul šie pirmiausia išmoksta rūpintis savo namiškiais ir deramai atsilyginti gimdytojams, nes tai patinka Dievui. 
\par 5 Tikra našlė, palikusi viena, pasitiki Dievu, maldauja ir meldžiasi dieną ir naktį. 
\par 6 Bet našlė, gyvenanti malonumais, dar gyva būdama, yra mirusi. 
\par 7 Todėl tai įsakyk, kad jos būtų be priekaišto. 
\par 8 Jeigu kas neaprūpina savųjų, ir ypač savo namiškių, tas yra paneigęs tikėjimą ir blogesnis už netikintį! 
\par 9 Į našlių sąrašą įtrauk tokią našlę, kuriai ne mažiau kaip šešiasdešimt metų, kuri buvo žmona vieno vyro, 
\par 10 jeigu ji pasižymėjo gerais darbais, jei išauklėjo vaikus, jei buvo svetinga, jei plaudavo šventiesiems kojas, jei pagelbėdavo vargstantiems, jei stengdavosi dirbti visokį gerą darbą. 
\par 11 Jaunesnių našlių nepriimk. Pasiduodamos gašlumui, o ne Kristui, jos nori ištekėti 
\par 12 ir yra smerktinos, nes pameta ankstesnį tikėjimą. 
\par 13 Be to, jos išmoksta dykinėti, landžiodamos iš namų į namus, ir ne tik dykinėti, bet plepėti, kištis į svetimus reikalus bei kalbėti, kas nedera. 
\par 14 Taigi norėčiau, kad jaunesnės ištekėtų, augintų vaikus, šeimininkautų ir neduotų priešininkui jokios progos apkalboms. 
\par 15 Nes kai kurios jau nuklydo paskui šėtoną. 
\par 16 Jei kuris tikintis vyras ar moteris turi pas save našlių, tegul jas aprūpina, kad bažnyčia nebūtų apsunkinama ir įstengtų padėti tikrosioms našlėms. 
\par 17 Gerai vadovaujantys vyresnieji tebūna laikomi vertais dvigubos pagarbos, ypač tie, kurie dirba, skelbdami žodį ir mokydami. 
\par 18 Juk Raštas sako: “Neužrišk kuliančiam jaučiui nasrų”, ir: “Darbininkas vertas savo užmokesčio”. 
\par 19 Skundo prieš vyresnįjį nepriimk, nebent liudytų du ar trys liudytojai. 
\par 20 Nuodėmiaujančius bark visų akivaizdoje, kad ir kiti bijotų. 
\par 21 Aš įsakau tau Dievo, Viešpaties Jėzaus Kristaus ir išrinktųjų angelų akivaizdoje, kad laikytumeis šitų nurodymų be išankstinio nusistatymo ir neatsižvelgdamas į asmenis. 
\par 22 Skubotai niekam neuždėk rankų ir neprisidėk prie svetimų nuodėmių; išsilaikyk tyras. 
\par 23 Negerk vien gryno vandens, bet vartok truputį vyno dėl savo skrandžio ir dažnų negalavimų. 
\par 24 Kai kurių žmonių nuodėmės matomos iki teismo, o kai kurių paaiškėja tik vėliau. 
\par 25 Taip pat ir geri darbai visiems matomi, o jei pasitaiko kitaip, jų vis tiek negalima paslėpti.


\chapter{6}


\par 1 Vergai, esantys po jungu, telaiko savo šeimininkus vertais visokeriopos pagarbos, kad nebūtų piktžodžiaujama Dievo vardui ir mokymui. 
\par 2 O tie, kurie turi tikinčius šeimininkus, tegul jų neniekina, kadangi jie yra broliai; geriau tegul jiems dar uoliau tarnauja, nes, gaunantys naudą, yra tikintys ir mylimi. Taip mokyk ir taip ragink! 
\par 3 Jei kas nors moko kitaip ir nesutinka su sveikais mūsų Viešpaties Jėzaus Kristaus žodžiais bei dievotumo mokymu, 
\par 4 tas yra išdidus, nieko neišmano, serga nuo ginčų ir nuo svaidymosi žodžiais, iš kurių atsiranda pavydas, nesutarimas, piktžodžiavimas, blogi įtarinėjimai 
\par 5 ir kivirčai tarp sugedusio proto žmonių, praradusių tiesos nuovoką ir manančių, jog dievotumas esąs pasipelnymas. Šalinkis tokių. 
\par 6 Žinoma, dievotumas yra didelis pasipelnymas, kai jį lydi pasitenkinimas. 
\par 7 Juk nieko neatsinešėme į pasaulį ir, aišku, nieko neišsinešime. 
\par 8 Turėdami maisto ir drabužį, būkime patenkinti. 
\par 9 Kas trokšta praturtėti, pakliūva į pagundymą ir į pinkles bei į daugelį kvailų ir kenksmingų geidulių, kurie paskandina žmones sugedime ir pražūtyje. 
\par 10 Visų blogybių šaknis yra meilė pinigams. Kai kurie, jų geisdami, nuklydo nuo tikėjimo ir patys save drasko aibe skausmų. 
\par 11 Bet tu, Dievo žmogau, bėk nuo tų dalykų ir vykis teisumą, maldingumą, tikėjimą, meilę, kantrumą, romumą. 
\par 12 Kovok gerą tikėjimo kovą, laikykis amžinojo gyvenimo, kuriam esi pašauktas ir išpažinai gerą išpažinimą daugelio liudytojų akyse. 
\par 13 Aš prašau tavęs prieš Dievą, kuris viskam teikia gyvybę, ir prieš Kristų Jėzų, kuris prie Poncijaus Piloto pateikė gerą išpažinimą, 
\par 14 kad išlaikytum šį įsakymą be dėmės ir be priekaišto iki mūsų Viešpaties Jėzaus Kristaus pasirodymo. 
\par 15 Jį savo laiku apreikš palaimintasis vienintelis Valdovas, karalių Karalius ir viešpačių Viešpats, 
\par 16 vienintelis Nemirtingasis, gyvenantis neprieinamoje šviesoje, kurio joks žmogus neregėjo ir negali regėti. Jam šlovė ir amžinoji valdžia! Amen. 
\par 17 Šio amžiaus turtuoliams įsakyk, kad nesididžiuotų ir nesudėtų vilčių į netikrus turtus, bet į gyvąjį Dievą, kuris apsčiai visko mums teikia mūsų džiaugsmui. 
\par 18 Tegul jie daro gera, turtėja gerais darbais, būna dosnūs, dalijasi su kitais. 
\par 19 Taip jie pasidės gerus pamatus ateičiai, kad pasiektų tikrąjį gyvenimą. 
\par 20 O Timotiejau, saugok tai, kas tau patikėta, vengdamas netinkamo tuščiažodžiavimo ir tariamojo pažinimo prieštaravimų, 
\par 21 nes kai kurie, jam atsidavę, nuklydo nuo tikėjimo. Malonė tebūna su jumis! Amen.


\end{document}