\begin{document}

\title{Sofonijo knyga}

\chapter{1}


\par 1 Viešpaties žodis buvo suteiktas Sofonijui, sūnui Kušio, sūnaus Gedelijos, sūnaus Amarijos, sūnaus Ezekijo, viešpataujant Judo karaliui Jozijui, Amono sūnui. 
\par 2 “Aš pašalinsiu visa nuo žemės paviršiaus,­sako Viešpats,­ 
\par 3 pašalinsiu žmones, gyvulius, padangių paukščius, jūros žuvis, suklupimo akmenis ir nedorėlius. 
\par 4 Aš pakelsiu savo ranką prieš Judą ir Jeruzalės gyventojus, išnaikinsiu Baalio likučius ir kartu su juo stabų kunigus; 
\par 5 ir tuos, kurie ant stogų garbina dangaus kareiviją, ir tuos, kurie, garbindami Viešpatį, prisiekia Melechu, 
\par 6 ir tuos, kurie, nusisukę nuo Viešpaties, neieško Jo, net nesiteirauja apie Jį”. 
\par 7 Nutilkite Viešpaties akivaizdoje. Viešpaties diena arti. Viešpats paruošė auką, pakvietė svečius. 
\par 8 “Aukojimo dieną Aš nubausiu kunigaikščius, karaliaus vaikus ir visus, kurie dėvi svetimšalių drabužius. 
\par 9 Tą dieną nubausiu visus, kurie šoka per slenkstį, smurtu ir apgaule pripildo savo valdovo namus. 
\par 10 Tą dieną,­sako Viešpats,­pagalbos šauksmas sklis nuo Žuvų vartų, raudojimas iš antrojo kvartalo ir baisus klyksmas nuo kalvų! 
\par 11 Raudokite, apsuptieji, nes sunaikinti visi prekybininkai, pražuvo sidabro keitėjai. 
\par 12 Aš su žibintais ieškosiu po Jeruzalę ir nubausiu tuos, kurie sėdi ant savo mielių, o širdyje mano: ‘Viešpats nedarys nei gera, nei pikta’. 
\par 13 Jų turtas bus išplėštas ir namai liks tušti. Jie pasistatys namus, bet negyvens juose, užveis vynuogynų, bet negers jų vyno. 
\par 14 Viešpaties didžioji diena skubiai artėja. Jau girdimas Viešpaties dienos balsas. Tada graudžiai verks net karžygiai! 
\par 15 Ta diena­rūstybės diena, diena vargo ir suspaudimo, diena nelaimės ir sunaikinimo, diena tamsos ir sutemų, diena debesų ir miglos, 
\par 16 diena trimito ir kovos šauksmo prieš sutvirtintus miestus ir aukštus bokštus. 
\par 17 Aš varginsiu žmones, jie vaikščios kaip akli, nes nusidėjo Viešpačiui, jų kraujas bus pralietas kaip dulkės, jų kūnai bus išmesti kaip mėšlas”. 
\par 18 Nei sidabras, nei auksas neišgelbės jų Viešpaties rūstybės dieną, Jo pavydas lyg ugnis sunaikins visą kraštą ir jo gyventojus.


\chapter{2}


\par 1 Susirinkite, taip, susirinkite, nemėgstama tauta, 
\par 2 kol dar nepriimtas sprendimas­ diena praeis kaip pelai,­kol neužgriuvo Viešpaties baisioji rūstybė, kol neatėjo Viešpaties rūstybės diena! 
\par 3 Visi krašto romieji, kurie vykdote Jo įsakymus, ieškokite Viešpaties, ieškokite teisumo, ieškokite nuolankumo! Gal išvengsite Viešpaties rūstybės? 
\par 4 Gaza taps negyvenama vieta, o Aškelonas­dykuma. Ašdodo gyventojai bus vidudienį išvaryti, Ekronas išrautas su šaknimis. 
\par 5 Vargas jums, pajūrio gyventojai, keretų tauta! Viešpaties žodis prieš jus: “Kanaane, filistinų šalie, Aš sunaikinsiu tave, liksi be gyventojų”. 
\par 6 Pajūrio kraštas taps avių ganyklomis ir gardais. 
\par 7 Pajūris atiteks Judo namų likučiui: prie jūros jie ganys bandas, o vakare ilsėsis Aškelone. Viešpats, jų Dievas, aplankys juos ir parves jų ištremtuosius. 
\par 8 “Aš girdėjau moabitų plūdimą ir amonitų piktžodžiavimą, kai jie plūdo mano tautą ir didžiavosi prieš jos kraštą. 
\par 9 Todėl, kaip Aš gyvas,­sako kareivijų Viešpats, Izraelio Dievas,­Moabas taps Sodoma ir Amonas­Gomora, piktžolių laukais bei druskos duobėmis ir tyrais per amžius. Mano tautos likutis juos apiplėš ir pasisavins jų kraštą”. 
\par 10 Jie bus nubausti už išdidumą ir kareivijų Viešpaties tautos plūdimą. 
\par 11 Baisus bus jiems Viešpats. Jis sunaikins visus jų krašto dievus, tada garbins Jį visos pagonių salos. 
\par 12 Jūs, etiopai, taip pat būsite žudomi mano kardu. 
\par 13 Jis išties savo ranką į šiaurę ir sunaikins Asiriją, pavers Ninevę dykyne: 
\par 14 ji bus ganykla bandoms ir poilsio vieta visiems žvėrims. Vanagai ir apuokai nakvos ant kolonų, jų balsai girdėsis languose. Slenksčiai bus nuniokoti, kedro lentos bus nuplėštos. 
\par 15 Džiūgavęs miestas, kuris taip saugiai gyveno, galvodamas: “Aš ir niekas kitas!” Kuo jis pavirto? Dykyne, gyvulių ganykla. Kiekvienas praeidamas švilps ir mos ranka.



\chapter{3}


\par 1 Vargas maištingam, suteptam ir pilnam smurto miestui! 
\par 2 Jis nepakluso balsui, nepriėmė pamokymų, nepasitikėjo Viešpačiu, nesiartino prie savo Dievo. 
\par 3 Jo kunigaikščiai yra riaumojantys liūtai, teisėjai­stepių vilkai, kurie nieko nepalieka kitam. 
\par 4 Jo pranašai lengvabūdžiai ir neištikimi, jo kunigai suteršė šventyklą ir laužo įstatymą. 
\par 5 Teisus Viešpats yra jame, Jis nedaro neteisybės, kas rytą ištikimai apreiškia savo teisingumą. Tačiau neteisieji nepažįsta gėdos. 
\par 6 “Aš išnaikinau tautas. Jų bokštai sugriauti, jų gatvės tuščios, niekas jomis nebevaikščioja. Jų miestai nusiaubti, be gyventojų. 
\par 7 Aš tariau: ‘Tu, Jeruzale, bijosi manęs ir priimsi mano pamokymą’, kad nesunaikinčiau jų, kaip grasinau. Tačiau jie vis labiau nusikalto. 
\par 8 Laukite manęs,­sako Viešpats,­laukite tos dienos, kai pakilsiu plėšti. Aš nusprendžiau surinkti tautas, sutelkti karalystes, išlieti ant jų savo rūstybę. Mano pavydo ugnis sunaikins visą žemę. 
\par 9 Tada Aš duosiu tautoms tyrą kalbą, kad jos visos galėtų šauktis Viešpaties ir vieningai Jam tarnauti. 
\par 10 Iš anapus Etiopijos upių ateis mano išsklaidytieji ir atneš man aukų. 
\par 11 Tuomet nebereikės tau gėdytis savo darbų, kuriais maištavai prieš mane. Aš pašalinsiu tavyje pasipūtėlius, tu nebesididžiuosi mano šventajame kalne. 
\par 12 Aš paliksiu tavyje nuolankius ir nusižeminusius, kurie pasitikės Viešpaties vardu. 
\par 13 Izraelio likutis elgsis teisingai, vengs melo ir klastos. Jie gyvens ramiai ir be baimės”. 
\par 14 Giedok garsiai, Siono dukra! Šauk, Izraeli! Linksminkis ir džiūgauk visa širdimi, Jeruzalės dukra! 
\par 15 Viešpats panaikino tavo nuosprendį, pašalino tavo priešus. Viešpats, Izraelio Karalius, yra su tavimi­tu nebematysi pikta. 
\par 16 Tą dieną sakys Jeruzalei: “Nebijok!”, ir Sionui: “Nenuleisk rankų!” 
\par 17 Viešpats, tavo Dievas, esantis tavyje, yra galingas. Jis išgelbės, Jis džiaugsis tavimi, atgaivins tave savo meile ir džiūgaus dėl tavęs giedodamas. 
\par 18 “Aš pašalinsiu nelaimes ir panieką, 
\par 19 saugosiu tave nuo prispaudėjų, padėsiu raišam, surinksiu išsklaidytuosius, jų gėdą paversiu garbe. 
\par 20 Tada, surinkęs jus, parvesiu namo. Aš jus išaukštinsiu ir suteiksiu jums vardą visose žemės tautose, kai parvesiu jūsų ištremtuosius, jums patiems matant”,­sako Viešpats.




\end{document}