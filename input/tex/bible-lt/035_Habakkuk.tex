\begin{document}

\title{Habakuko knyga}

\chapter{1}


\par 1 Regėjimas, kurį matė pranašas Habakukas. 
\par 2 Viešpatie, kaip ilgai aš šauksiu, o Tu neišklausysi, šauksiu apie smurtą, o Tu negelbėsi! 
\par 3 Kodėl man leidi patirti nedorybę ir vargą? Priespauda ir smurtas prieš mane, visur kyla vaidai ir barniai. 
\par 4 Įstatymas nusilpo, į teisingumą nekreipiama dėmesio, nedorėlis apsuka teisųjį, teisme sprendimas iškraipomas! 
\par 5 “Dairykitės tarp tautų ir įsidėmėkite, didžiai stebėkitės, nes jūsų dienomis darysiu darbą, kuriuo jūs netikėtumėte, jei jums kas apie tai pasakotų. 
\par 6 Aš sukelsiu chaldėjus, kurie siekia užimti jiems nepriklausančias žemes, kitų gyvenvietes. 
\par 7 Tai žiauri ir baisi tauta, jie patys nusprendžia, kas teisu ir garbinga. 
\par 8 Jų žirgai greitesni už leopardus. Jie plėšresni negu stepių vilkai. Raiteliai išsiskleidžia, jie atjoja iš toli, jie skrenda kaip erelis prie grobio. 
\par 9 Jie visi siekia smurto, jų veidai kaip rytų vėjas, jie ima belaisvius kaip smėlio smiltis. 
\par 10 Jie tyčiojasi iš karalių, šaiposi iš kunigaikščių. Jie juokiasi iš tvirtovių, supylę pylimus, jie jas paima. 
\par 11 Jie traukia tolyn kaip vėjas, viską nualina. Jiems jėga­jų dievas”. 
\par 12 Argi Tu nesi amžinasis Viešpats, mano Dievas, mano Šventasis? Mes nemirsime! Viešpatie, Tu juos paskyrei teismui, galingasis Dieve, Tu juos paruošei bausmei. 
\par 13 Tavo akys tyros, jos negali matyti pikto, negali žiūrėti į neteisybę. Kodėl ramiai stebi piktadarius ir tyli, kai nedorėlis praryja teisesnį už save? 
\par 14 Tu padarei žmones kaip jūros žuvis, kaip kirmėles, kurios neturi valdovo. 
\par 15 Jie iškelia juos meškere, ištraukia savo tinklu, surenka bradiniu; todėl jie patenkinti džiaugiasi. 
\par 16 Jie aukoja tinklui ir smilko bradiniui, nes jų dėka grobis gausus ir maistas geras. 
\par 17 Argi jie nuolat tuštins tinklus ir žudys tautas be pasigailėjimo?


\chapter{2}


\par 1 Aš stovėsiu sargyboje, pakilsiu į bokštą ir stebėsiu, ką Viešpats man kalbės ir ką man atsakyti, kai esu baramas. 
\par 2 Viešpats atsakydamas man tarė: “Užrašyk regėjimą aiškiai ant plokščių, kad jį galėtų perskaityti prabėgantis. 
\par 3 Regėjimas yra skirtam laikui, bet galiausiai jis kalbės ir nemeluos. Jei jis uždelstų­lauk, nes jis tikrai išsipildys ir nevėluos. 
\par 4 Pasipūtėlio siela nėra dora, bet teisusis gyvens savo tikėjimu. 
\par 5 Kaip vynas yra apgaulingas, taip išdidus žmogus­besotis. Jo gerklė kaip mirusiųjų buveinė, nepasotinama kaip mirtis, jis pavergia tautas ir gimines. 
\par 6 Ar ne jie dainuos pašaipias dainas ir sugalvos patarlių apie jį: ‘Vargas tam, kuris daugina tai, kas ne jo. Kaip ilgai tai tęsis? Tu apkrauni save užstatų daugybe’. 
\par 7 Ar nepabus ir nepakils tie, kurie apiplėš tave, ar neprivers tavęs drebėti? Tada tu tapsi jiems grobiu. 
\par 8 Kadangi tu apiplėšei daug tautų, dabar tave apiplėš visos likusios tautos­dėl pralieto kraujo ir padaryto smurto miestams ir visiems jų gyventojams. 
\par 9 Vargas tam, kuris godžiai siekia neteisingo pelno savo namams, kad susuktų lizdą aukštybėje, kad apsisaugotų nuo nelaimės! 
\par 10 Tu atnešei gėdą savo namams, naikindamas tautas, pats praradai gyvybę! 
\par 11 Akmuo iš sienos šauks, o medinės sijos jam atsakys: 
\par 12 ‘Vargas tam, kuris stato miestą krauju ir tvirtovę neteisybe’. 
\par 13 Tai ne kareivijų Viešpaties valia, kad tautos dirbtų ugniai ir giminės vargtų veltui. 
\par 14 Žemė bus pilna Viešpaties šlovės pažinimo kaip jūra pilna vandens. 
\par 15 Vargas tam, kuris pila savo artimui, ragindamas jį gerti, ir taip jį nugirdo, kad galėtų matyti jo nuogumą. 
\par 16 Tu prisipildei gėda vietoj garbės. Dabar gerk ir tu ir atidenk savo gėdą. Viešpats siunčia tau gėdos ir pažeminimo taurę. 
\par 17 Tave užgrius Libanui padarytas smurtas ir gyvulių grobimas gąsdins tave dėl žmonių kraujo ir kraštui, miestams bei visiems jų gyventojams padaryto smurto. 
\par 18 Kokia nauda iš drožinio, kurį drožėjas padarė? Ar iš lieto atvaizdo, melų mokytojo, kuriuo jo gamintojas pasitiki, darydamas nebylius stabus? 
\par 19 Vargas tam, kuris sako medžiui: ‘Pabusk!’, nebyliam akmeniui: ‘Pajudėk!’ Tiesa, jis aptrauktas auksu ir sidabru, tačiau jame nėra kvapo. 
\par 20 Bet Viešpats yra savo šventykloje, tenutyla visas pasaulis prieš Jį!”



\chapter{3}


\par 1 Pranašo Habakuko malda: 
\par 2 “Viešpatie, girdėjau tavo kalbą ir nusigandau. Atnaujink ir apreikšk savo darbą metuose, savo rūstybėje prisimink gailestingumą. 
\par 3 Dievas atėjo iš Temano, Šventasis­nuo Parano kalno. Jo šlovė uždengė dangų ir žemė buvo pilna Jo gyriaus. 
\par 4 Spinduliuojančioje šviesoje Jis pasirodė. Iš Jo rankų tvieskė šviesos spinduliai­Jo galybė. 
\par 5 Pirma Jo ėjo maras, degančios anglys po Jo kojomis. 
\par 6 Jis sustojo ir išmatavo žemę. Jis pažvelgė­išsigando tautos, susvyravo amžinieji kalnai, nusilenkė kalvos. Jo keliai amžini. 
\par 7 Aš mačiau Kušano palapines nelaimėje, Midjano krašto palapinių uždangos siūbavo. 
\par 8 Viešpatie, ar upės sukėlė Tavo kerštą, ar jūra sužadino įtūžį, ar srovės uždegė Tavo rūstybę, kad Tu važiavai žirgais ir išgelbėjimo vežimais? 
\par 9 Tavo lankas buvo apnuogintas, kaip buvai prisiekęs tautoms. Tu išraižei žemę upėmis. 
\par 10 Tave pamatę sudrebėjo kalnai, liejosi vanduo, gelmės balsas pasigirdo, ir ji kėlė rankas į aukštybes. 
\par 11 Saulė ir mėnulis sustojo, kai pasipylė Tavo šviečiančios strėlės ir suspindo ietys. 
\par 12 Įtūžęs Tu ėjai per žemę, užsirūstinęs trypei tautas. 
\par 13 Tu išėjai gelbėti savo tautos, gelbėti savo pateptojo, sutrupinai nedorėlių namų galvą, apnuoginai juos nuo pamatų iki kaklo. 
\par 14 Tu pervėrei jo strėlėmis jo karius, kai jie kaip audra ėjo išblaškyti manęs, džiūgaudami, kad galės slaptoje praryti vargšą. 
\par 15 Tu su savo žirgais perėjai jūrą galingų vandenų paviršiumi. 
\par 16 Man tai girdint, drebėjo mano kūnas, virpėjo lūpos. Skausmas palietė kaulus, buvau labai sukrėstas. Aš ramus laukiu bausmės tiems, kurie užpuolė mano tautą. 
\par 17 Nors figmedis nežydėtų ir nebūtų vaisių ant vynmedžių ir alyvmedžių, nors laukai neduotų derliaus, garduose dingtų avys ir ožkos ir nebūtų gyvulių tvartuose, 
\par 18 tačiau aš džiaugsiuosi Viešpačiu, džiūgausiu savo išgelbėjimo Dievu! 
\par 19 Viešpats Dievas yra mano stiprybė. Jis padarys mano kojas kaip elnių ir leis man pasiekti aukštumas”. Choro vadovui. Styginiais instrumentais.



\end{document}