\begin{document}

\title{Ezekielio knyga}

\chapter{1}


\par 1 Tai buvo trisdešimtaisiais metais, ketvirto mėnesio penktą dieną, man esant tarp tremtinių prie Kebaro upės. Atsivėrė dangūs, ir aš mačiau Dievo regėjimus. 
\par 2 Penktaisiais karaliaus Joachino tremties metais, penktą mėnesio dieną, 
\par 3 Viešpats kalbėjo kunigui Ezechieliui, Buzio sūnui, chaldėjų krašte prie Kebaro upės. Viešpaties ranka buvo ten ant jo. 
\par 4 Aš mačiau iš šiaurės kylantį viesulą ir debesis, aplinkui liepsnojo ugnis. Ugnies vidurys spindėjo lyg žėrintis gintaras. 
\par 5 Liepsnų viduryje mačiau keturias būtybes. Jų išvaizda buvo panaši į žmogaus. 
\par 6 Kiekviena jų turėjo keturis veidus ir keturis sparnus. 
\par 7 Jų kojos buvo tiesios, o kojų pėdos panašios į veršio kanopas. Būtybės žibėjo kaip išlydytas varis. 
\par 8 Po jų sparnais keturiuose šonuose buvo žmogaus rankos. Veidus ir sparnus jos turėjo keturiose pusėse. 
\par 9 Sparnai lietė vienas kitą. Eidamos jos nesisuko, kiekviena ėjo tiesiai. 
\par 10 Visų keturių veidai atrodė taip: priekyje buvo žmogaus veidas, dešinėje­liūto veidas, kairėje­veršio veidas ir užpakalyje­erelio veidas. 
\par 11 Jų sparnai buvo ištiesti ir pakelti į viršų. Du sparnai lietė vienas kitą, kiti du­dengė jų kūnus. 
\par 12 Jos ėjo kiekviena tiesiai pirmyn. Kur dvasia norėjo eiti, ten jos ir ėjo, ir eidamos nepasisukdavo. 
\par 13 Būtybės atrodė kaip degančios anglys, kaip žibintai. Ugnis judėjo tarp būtybių. Ji buvo labai šviesi, o iš jos žybčiojo žaibai. 
\par 14 Būtybės judėjo pirmyn ir atgal tarsi žaibai. 
\par 15 Žiūrėdamas į būtybes, mačiau ant žemės prie kiekvienos iš jų po ratą. 
\par 16 Ratai atrodė lyg berilis. Visi keturi buvo vienodi ir taip pat padaryti. Kiekvienas iš jų atrodė lyg būtų ratas rate. 
\par 17 Jie judėdavo į keturias puses nepasisukdami. 
\par 18 Jų ratlankiai buvo aukšti ir keliantys baimę. Visi keturi ratlankiai buvo pilni akių. 
\par 19 Einant būtybėms, drauge ėjo ir ratai; pakilus būtybėms aukštyn, pakildavo ir ratai. 
\par 20 Kur dvasia ėjo, ir jie ėjo; pakilus dvasiai, ir ratai pakildavo, nes būtybių dvasia ir buvo ratuose. 
\par 21 Jei jos ėjo, ir jie ėjo drauge, jei jos sustodavo, sustodavo ir jie; jei jos pakildavo nuo žemės, pakildavo ir ratai, nes būtybių dvasia buvo ratuose. 
\par 22 Virš būtybių galvų buvo kažkas panašaus į dangaus skliautą, kuris žėrėjo kaip krištolas ir gaubė jas. 
\par 23 Po skliautu kiekviena būtybė dviem sparnais dengė savo kūną, o kiti du sparnai buvo ištiesti. 
\par 24 Joms judant, girdėjau jų sparnų ūžimą lyg galingo vandenyno, lyg Visagalio balsą, lyg didelės kariuomenės stovyklos triukšmą. Joms sustojus, sparnai nusileisdavo. 
\par 25 Nuo skliauto, esančio virš jų galvų, aidėjo balsas. Kai jos stovėjo, sparnai buvo nuleisti. 
\par 26 Virš skliauto, kuris buvo virš jų galvų, mačiau lyg sostą iš safyro ir ant jo sėdinčią į žmogų panašią būtybę. 
\par 27 Būtybė nuo juosmens aukštyn ir žemyn spindėjo lyg gintaras, lyg ugnies liepsnos, 
\par 28 lyg lankas, kuris pasirodo debesyse lyjant, Jį supo spindėjimas. Tai buvo Viešpaties šlovės pasirodymas. Pamatęs tai, kritau veidu žemėn ir išgirdau balsą.


\chapter{2}


\par 1 Jis kalbėjo: “Žmogaus sūnau, stokis, Aš kalbėsiu su tavimi”. 
\par 2 Jam pradėjus kalbėti, dvasia pastatė mane ant kojų, ir aš klausiausi kalbančio. 
\par 3 Jis tarė: “Žmogaus sūnau, Aš siunčiu tave pas Izraelio žmones, pas maištingą tautą, kuri maištauja prieš mane. Jie ir jų tėvai priešinosi man iki šios dienos. 
\par 4 Žmonės, pas kuriuos siunčiu tave, yra įžūlūs ir nepalenkiamos širdies. Tu jiems sakyk: ‘Taip sako Dievas’. 
\par 5 Ar jie klausys, ar neklausys, nes jie yra maištinga tauta, tačiau žinos, kad pranašas buvo tarp jų. 
\par 6 Tu, žmogaus sūnau, nebijok jų ir neišsigąsk jų kalbų. Nors tave supa dilgėlės ir erškėčiai, nors gyveni tarp skorpionų, nebijok jų žodžių ir neišsigąsk jų žvilgsnių, nes jie yra maištinga tauta. 
\par 7 Kalbėk jiems mano žodžius, nepaisydamas, ar jie klausys, ar neklausys jų. 
\par 8 Tu, žmogaus sūnau, klausyk, ką tau sakau, nebūk maištininkas kaip jie. Tu valgyk, ką tau duosiu”. 
\par 9 Aš pažvelgiau ir pamačiau į mane ištiestą ranką. Ranka laikė knygos ritinį. 
\par 10 Jis išskleidė ritinį prieš mane. Jame buvo prirašyta abiejose pusėse: “Raudos, dejonės ir vargai”.



\chapter{3}


\par 1 Jis tarė man: “Žmogaus sūnau, suvalgyk, ką matai prieš save! Suvalgęs šitą ritinį, eik ir kalbėk Izraelio namams”. 
\par 2 Tada aš išsižiojau, ir Jis davė man suvalgyti ritinį. 
\par 3 Ir Jis tarė man: “Žmogaus sūnau, valgyk šį ritinį ir pasisotink juo”. Aš jį suvalgiau, ir jis mano burnoje buvo saldus kaip medus. 
\par 4 Jis tada vėl kalbėjo: “Žmogaus sūnau, dabar eik į Izraelio namus ir kalbėk jiems mano žodžius. 
\par 5 Tu siunčiamas ne pas svetimą tautą su nesuprantama kalba, bet pas Izraelį. 
\par 6 Ne pas tautas, kurių kalbos tu nemoki. Jei pas juos tave siųsčiau, jie klausytų tavęs. 
\par 7 Bet Izraelis neklausys tavęs, nes jis ir manęs neklauso. Izraelio tauta yra kietasprandė ir kietaširdė. 
\par 8 Aš padariau tavo veidą tvirtą prieš jų veidus ir tavo kaktą kietą prieš jų kaktas. 
\par 9 Ji bus kieta kaip deimantas, kietesnė už titnagą. Nebijok jų ir neišsigąsk jų žvilgsnių, nes tai maištinga tauta. 
\par 10 Žmogaus sūnau, visus mano žodžius, kuriuos tau kalbu, klausyk ausimis ir priimk širdimi. 
\par 11 Eik pas tremtinius, savo tautiečius, ir jiems kalbėk; ar jie klausys, ar neklausys, sakyk: ‘Taip sako Viešpats’ ”. 
\par 12 Dvasia pakėlė mane, ir už savęs išgirdau griausmingą balsą: “Palaiminta Viešpaties šlovė šioje vietoje”, 
\par 13 taip pat būtybių sparnų šlamėjimą ir ratų dundėjimą­tai buvo didelis dundesys. 
\par 14 Dvasia pakėlė ir nunešė mane. Aš nuėjau apkartęs, degančia dvasia, bet Viešpaties ranka buvo stipri ant manęs. 
\par 15 Aš atėjau į Tel Abibą pas tremtinius, kurie gyveno prie Kebaro upės. Ten sėdėjau septynias dienas tarp jų labai susirūpinęs. 
\par 16 Septynioms dienoms praėjus, Viešpats kalbėjo man: 
\par 17 “Žmogaus sūnau, Aš paskyriau tave sargybiniu Izraelio namams. Ką išgirsi iš manęs, pranešk jiems mano vardu. 
\par 18 Jei Aš sakysiu nedorėliui: ‘Tu mirsi’, bet tu neįspėsi jo ir nepamokysi, kad jis paliktų savo nedorą kelią ir išliktų gyvas, nedorėlis mirs dėl savo nusikaltimų, bet jo kraujo pareikalausiu iš tavo rankų. 
\par 19 Jei tu įspėsi nedorėlį, bet jis neatsivers nuo savo nedorybių ir nepakeis savo kelių, jis mirs dėl savo nusikaltimų, bet tu išgelbėsi savo sielą. 
\par 20 Jei teisusis, nusigręžęs nuo teisumo, darys pikta, Aš padėsiu jam kelyje suklupimo akmenį, ir jis mirs. Jei nebūsi įspėjęs jo, jis mirs dėl savo nuodėmės ir jo teisių darbų nebus atsiminta, bet jo kraujo pareikalausiu iš tavo rankų. 
\par 21 Jei tu įspėsi teisųjį, kad jis nenusikalstų, ir jis nenusikals, jis išliks gyvas, nes paklausė įspėjimo, ir tu išgelbėsi savo sielą”. 
\par 22 Viešpaties ranka buvo ant manęs ir Jis tarė: “Eik į lygumą, ten Aš kalbėsiu su tavimi”. 
\par 23 Aš išėjau į lygumą. Čia buvo Viešpaties šlovė, kaip ją mačiau prie Kebaro upės. Aš kritau veidu žemėn. 
\par 24 Dvasia įėjo į mane, pastatė mane ant kojų ir įsakė: “Eik ir užsirakink savo namuose. 
\par 25 Žmogaus sūnau, tu būsi surištas ir negalėsi vaikščioti tarp žmonių. 
\par 26 Tapsi nebyliu, nebegalėsi įspėti tų maištingų žmonių. 
\par 27 Kai Aš kalbėsiu su tavimi, atversiu tavo burną, ir tu jiems sakysi: ‘Taip sako Viešpats’. Tada, kas klausys, teklauso, o kas neklausys, teneklauso, nes jie yra maištingi žmonės”.



\chapter{4}


\par 1 “Tu, žmogaus sūnau, pasiimk plytą, pasidėk ją prieš save ir nubraižyk ant jos Jeruzalės miestą. 
\par 2 Apgulk miestą, padaryk sutvirtinimus, pylimą, karių stovyklas ir sienų griovimo įtaisus. 
\par 3 Dar paimk geležinę skardą ir pastatyk ją kaip geležinę sieną tarp savęs bei miesto. Atgręžk savo veidą į jį, ir jis bus apgultas­tu būsi jį apgulęs. Tai bus ženklas Izraelio namams. 
\par 4 Tu atsigulk ant kairiojo šono ir uždėk ant jo Izraelio namų kaltę. Kiek dienų gulėsi, tiek neši jų kaltę. 
\par 5 Ant tavęs uždėjau jų kaltės metus, atitinkamą dienų skaičių. Tris šimtus devyniasdešimt dienų tu neši Izraelio kaltę. 
\par 6 Kai tai atliksi, tada gulkis ant dešiniojo šono ir nešk Judo kaltę keturiasdešimt dienų. Aš įskaitysiu tau dieną už metus. 
\par 7 Tu atsigręžk veidu į apgultą Jeruzalę ir pranašauk ištiesęs apnuogintą ranką į miestą. 
\par 8 Aš surišau tave virvėmis, kad negalėtum apsiversti nuo vieno šono ant kito, kol pasibaigs Jeruzalės apgultis. 
\par 9 Imk kviečių, miežių, pupų, lešių, sojų bei rugių, sudėk visa į vieną indą ir pasigamink maistą daugeliui dienų. Kai gulėsi ant šono, valgyk jį tris šimtus devyniasdešimt dienų. 
\par 10 Maistą valgyk pagal svorį, dienai dvidešimt šekelių. Tiek valgyk kas dieną. 
\par 11 Vandenį gerk atseikėtą, kas dieną po šeštą dalį hino. 
\par 12 Valgyk tai kaip miežinius papločius ir iškepk juos ant žmogaus mėšlo jų akivaizdoje. 
\par 13 Taip izraelitai valgys suteptą maistą, gyvendami tarp pagonių, kur Aš juos išsklaidysiu”. 
\par 14 Tada atsakiau: “Ak, Viešpatie Dieve, aš niekada nesu susitepęs: nuo savo jaunystės iki šios dienos nevalgiau pastipusio nei sudraskyto gyvulio, nešvarios mėsos neturėjau savo burnoje”. 
\par 15 Viešpats sakė: “Tau leidžiu ruošiant maistą vietoje žmonių mėšlo naudoti galvijų mėšlą. 
\par 16 Žmogaus sūnau, Aš Jeruzalėje sunaikinsiu duonos ramstį. Jie valgys susirūpinę maistą pagal svorį ir gers išsigandę atseikėtą vandenį. 
\par 17 Jiems neužteks nei maisto, nei vandens. Jie žiūrės vienas į kitą išsigandę ir žus dėl savo nusikaltimų”.



\chapter{5}


\par 1 “Žmogaus sūnau, imk aštrų peilį ir naudok jį kaip skustuvą. Juo nuskusk savo galvos plaukus ir barzdą. Po to imk svarstykles ir sverdamas padalink plaukus. 
\par 2 Vieną trečdalį sudegink miesto viduryje, apgulčiai baigiantis, antrą trečdalį sukapok peiliu ir paskutinį trečdalį paleisk pavėjui, o Aš išsitrauksiu iš paskos kardą. 
\par 3 Paimk kelis plaukus ir pririšk juos prie savo apsiausto kampo. 
\par 4 Kelis iš jų įmesk į ugnį ir sudegink. Iš ten ugnis pasklis po visą Izraelį”. 
\par 5 Taip sako Viešpats Dievas: “Tai Jeruzalė. Aš ją įkūriau tarp pagonių tautų. 
\par 6 Bet ji paniekino mano sprendimus labiau negu pagonių tautos ir mano nuostatus labiau negu aplinkui esantys kraštai, atmesdama mano sprendimus ir nesilaikydama mano įstatymo. 
\par 7 Kadangi jūsų nedorybių daugiau negu nedorybių aplink jus gyvenančių tautų ir jūs nesielgėte pagal mano įsakymus, nesilaikėte mano nuostatų ir nepaisėte netgi nuostatų aplinkui jus gyvenančių tautų, 
\par 8 todėl Aš pats esu prieš jus ir įvykdysiu teismą tavo viduryje tautų akivaizdoje. 
\par 9 Izraeli, už tavo bjaurystes tau darysiu tai, ko niekada nedariau ir nebedarysiu. 
\par 10 Todėl jūsų tėvai valgys savo vaikus ir vaikai tėvus. Aš įvykdysiu teismą tavyje ir jūsų likutį išblaškysiu į visas puses. 
\par 11 Kaip Aš gyvas,­sako Viešpats,­tikrai jus naikinsiu ir nepasigailėsiu, nes jūs savo bjaurystėmis ir neteisybėmis sutepėte mano šventyklą. 
\par 12 Vienas jūsų trečdalis mirs nuo maro ir bado, antrasis trečdalis­nuo kardo, o trečiąją dalį išblaškysiu į visus vėjus ir ištrauksiu kardą paskui juos. 
\par 13 Savo rūstybę įvykdysiu ir pasitenkinsiu. Tada jie žinos, kad Aš, Viešpats, kalbėjau savo uolume, kai savo kerštą visiškai būsiu įvykdęs. 
\par 14 Aš padarysiu tave dykyne bei pajuoka visoms aplinkui esančioms tautoms ir praeinantiems pro šalį. 
\par 15 Jūs būsite piktžodžiavimu ir pajuoka, įspėjimu ir pasibaisėjimu aplinkinėms tautoms. Aš, Viešpats, tai pasakiau. 
\par 16 Aš šaudysiu į jus bado strėlėmis ir sunaikinsiu duonos ramstį. 
\par 17 Aš užleisiu badą ir laukinius žvėris, kurie naikins tave. Maras ir kardas eis per tave”.



\chapter{6}


\par 1 Viešpats kalbėjo: 
\par 2 “Žmogaus sūnau, pažvelk į Izraelio kalnus ir pranašauk prieš juos. 
\par 3 Sakyk: ‘Izraelio kalnai, klausykitės Viešpaties Dievo žodžio! Taip sako Viešpats Dievas kalvoms ir kalnams, upėms ir slėniams: ‘Aš užleisiu kardą ir sunaikinsiu jūsų aukštumas. 
\par 4 Jūsų aukurai bus nugriauti, atvaizdai sutrupinti; jūsų nužudytieji gulės ant žemės prie jūsų stabų. 
\par 5 Izraelitų lavonai gulės prie jų stabų. Jūsų kaulai bus išsklaidyti apie aukurus. 
\par 6 Jūsų gyvenvietės ir miestai ištuštės, aukštumos bus sunaikintos ir aukurai apleisti, stabai bus sulaužyti ir atvaizdai sutrupinti bei sunaikinti. 
\par 7 Nužudytieji kris tarp jūsų ant žemės. Tada jūs žinosite, kad Aš esu Viešpats. 
\par 8 Tačiau Aš paliksiu likutį, kurį išgelbėsiu nuo kardo. Jie gyvens išsklaidyti tautose. 
\par 9 Išlikusieji, išsklaidyti tarp tautų ir išvesti į nelaisvę, atsimins mane. Aš jų pasileidusią ir nuo manęs nusisukusią širdį palaušiu ir jų akis, paleistuvingai žiūrėjusias į stabus, atversiu, kad jie bjaurėsis patys savęs ir savo piktų darbų. 
\par 10 Jie žinos, kad Aš esu Viešpats ir kad mano įspėjimai nebuvo tušti’. 
\par 11 Plok delnais, trypk kojomis ir sakyk: ‘Vargas Izraelio namams dėl jų piktybių. Jie žus nuo kardo, bado ir maro. 
\par 12 Kas toli gyvena, mirs nuo maro, kas arti­kris nuo kardo, kas išliks gyvas­mirs badu. Taip Aš įvykdysiu savo rūstybę. 
\par 13 Tada jūs žinosite, kad Aš esu Viešpats, kai jų nukautieji gulės tarp stabų, aplink aukurus, kalvose ir ant kalnų, po žaliuojančiais medžiais ir šakotais ąžuolais, kur jie smilkė stabams kvepiančius smilkalus. 
\par 14 Aš ištiesiu savo ranką ir padarysiu kraštą tuščią, sunaikinsiu jų gyvenvietes, pradėdamas nuo pietų dykumos iki Diblos miesto. Tada jie žinos, kad Aš esu Viešpats’ ”.



\chapter{7}


\par 1 Viešpats kalbėjo: 
\par 2 “Žmogaus sūnau, taip sako Viešpats Dievas Izraelio žemei: ‘Atėjo galas visiems keturiems žemės pakraščiams. 
\par 3 Dabar atėjo galas ir tau! Aš siųsiu prieš tave savo rūstybę, teisiu tave pagal tavo kelius, atlyginsiu už bjaurius darbus. 
\par 4 Aš nepasigailėsiu tavęs, bet bausiu tave už tavo kelius ir užleisiu ant tavęs tavo bjaurystes. Tada jūs žinosite, kad Aš esu Viešpats’. 
\par 5 Viešpats Dievas sako: ‘Nelaimė ir tik nelaimė ateina! 
\par 6 Atėjo galas, jis ieško tavęs. Štai jis atėjo. 
\par 7 Tavo eilė būti sunaikintam atėjo! Tu, kuris gyveni krašte, ne džiaugsmo, bet sunaikinimo diena atėjo. 
\par 8 Tuojau išliesiu savo rūstybę ant tavęs, teisiu tave pagal tavo kelius ir bausiu už visas tavo bjaurystes. 
\par 9 Aš nepasigailėsiu tavęs, bet bausiu tave už tavo kelius ir tavo bjaurystes užvesiu ant tavęs. Tada žinosite, kad Aš­Viešpats, kuris baudžia. 
\par 10 Štai diena atėjo, pražūtis prisiartino. Neteisybė žydi, išdidumas žaliuoja! 
\par 11 Smurtas išaugo į nedorybės lazdą, nieko neliks nei iš jų turto, nei iš garbės, nei iš didybės. 
\par 12 Metas priartėjo, diena atėjo. Tenesidžiaugia pirkėjas ir teneliūdi pardavėjas, nes ateina bausmė visiems nusikaltusiems. 
\par 13 Pardavėjas nebeatgaus parduoto daikto, nors ir gyvas tebebūtų, nes sunaikinimas laukia visų; nė vienas nusikaltęs neišliks gyvas. 
\par 14 Pūskite trimitą, pasirenkite kovai! Bet nė vienas neina į kovą, nes mano rūstybė prieš visą jų daugybę. 
\par 15 Lauke kardas! Viduje maras ir badas! Kas lauke, žus nuo kardo, kas mieste, mirs nuo bado ir maro. 
\par 16 Jei kas išsigelbės, bus kaip slėnių balandis kalnuose; jie visi dejuos dėl savo nuodėmių. 
\par 17 Visų rankos nusilps ir keliai links. 
\par 18 Jie apsisiaus ašutinėmis ir juos apims baimė. Gėda bus jų veiduose ir plikė ant galvų. 
\par 19 Jie išmes savo auksą ir sidabrą, nes auksas ir sidabras neišgelbės jų Viešpaties rūstybės dieną. Jie nepatenkins savo sielų ir nepasisotins, nes tai tapo jų suklupimo akmeniu. 
\par 20 Iš savo papuošalų, kuriais didžiavosi, jie pasidarė bjaurius atvaizdus. Todėl jų brangenybes padariau beverčiais daiktais. 
\par 21 Aš atiduosiu juos svetimšaliams kaip grobį, žemės nedorėliai išplėš juos ir suterš. 
\par 22 Aš nusigręšiu nuo jų, leisiu išniekinti savo šventyklą. Plėšikai įsilauš, išnaikins ir apiplėš ją. 
\par 23 Kraštas ir miestas yra pilni nekalto kraujo, nusikaltimų ir smurto. 
\par 24 Aš atvesiu blogiausias tautas, ir jie paveldės jų namus. Padarysiu galą stipriojo pasipūtimui, ir jų šventos vietos bus išniekintos. 
\par 25 Ateina sunaikinimas; jie ieško taikos, bet jos nebus. 
\par 26 Nelaimė seks nelaimę, gąsdinantys pranešimai bus vienas po kito. Jie veltui ieško pranašų regėjimų. Įstatymo nebebus pas kunigus ir patarimo pas vyresniuosius. 
\par 27 Karalius gedės, kunigaikščiai bus apimti siaubo, tauta palūš. Aš padarysiu jiems taip, kaip jie darė, teisiu juos taip, kaip jie teisė. Tada jie žinos, kad Aš esu Viešpats’ ”.



\chapter{8}


\par 1 Šeštaisiais metais, šešto mėnesio penktą dieną, kai aš sėdėjau savo namuose su Judo vyresniaisiais, Viešpaties ranka palietė mane. 
\par 2 Aš pamačiau prieš save lyg žmogaus pavidalą; nuo juosmens žemyn atrodė lyg liepsnojanti ugnis, o nuo juosmens aukštyn­tarsi žėrintis gintaras. 
\par 3 Jis ištiesė ranką ir nutvėrė mane už galvos plaukų. Dvasia pakėlė mane tarp dangaus ir žemės ir nunešė mane Dievo regėjime į Jeruzalę prie vidaus kiemo vartų šiaurės pusėje, kur stovėjo pavydo stabas, kuris kėlė pavydą. 
\par 4 Ir ten aš pamačiau Izraelio Dievo šlovę, kaip ją buvau matęs regėjime lygumoje. 
\par 5 Jis man sakė: “Žmogaus sūnau, pažvelk į šiaurės pusę”. Aš žiūrėjau, ir štai prie pat aukuro, prie vartų, stovėjo pavydo stabas. 
\par 6 Jis man kalbėjo toliau: “Žmogaus sūnau, ar matai, ką jie daro? Izraelio namai daro dideles bjaurystes, kad Aš pasitraukčiau iš savo šventyklos. Bet apsisuk, ir pamatysi dar didesnių bjaurysčių!” 
\par 7 Jis atvedė mane prie šventyklos vartų ir parodė man sienoje skylę. 
\par 8 Jis liepė man: “Žmogaus sūnau, kask po siena!” Prakasęs sieną, radau duris. 
\par 9 Jis liepė įeiti ir pažiūrėti, kas čia vyksta. 
\par 10 Įėjęs mačiau ant sienų roplių, nešvarių gyvūnų ir Izraelio stabų piešinius. 
\par 11 Prieš juos stovėjo septyniasdešimt Izraelio vyresniųjų su Jaazaniju, Šafano sūnumi. Kiekvienas laikė rankoje po smilkytuvą, iš kurių kilo aukštyn tirštas smilkalų debesis. 
\par 12 Jis klausė: “Žmogaus sūnau, ar matai, ką Izraelio vyresnieji tamsoje daro prie stabų piešinių? Jie sako: ‘Viešpats paliko kraštą, Jis nemato mūsų’ ”. 
\par 13 Jis vėl man tarė: “Apsisuk, ir pamatysi dar didesnių bjaurysčių”. 
\par 14 Jis įvedė mane į Viešpaties namus pro šiaurinius vartus. Čia sėdėjo moterys ir apraudojo Tamūzą. 
\par 15 Jis tarė man: “Žmogaus sūnau, ar matai, kas čia vyksta? Apsisuk ir pamatysi bjaurysčių, didesnių už šitas”. 
\par 16 Jis įvedė mane į Viešpaties namų kiemą. Tarp Viešpaties šventyklos durų ir aukuro stovėjo maždaug dvidešimt penki vyrai, nusigręžę nuo Viešpaties šventyklos į rytus, ir garbino saulę. 
\par 17 Jis tarė: “Žmogaus sūnau, ar matai, kas čia vyksta? Ar negana to, kas yra Jude! Šitie nusikaltimai bei smurtas vėl užtrauks mano bausmę. 
\par 18 Aš išliesiu ant jų savo įtūžį, nesigailėsiu jų ir nebūsiu jiems nuolaidus. Nors jie garsiai šauksis mano pagalbos, neišklausysiu jų”.



\chapter{9}


\par 1 Jis šaukė garsiu balsu: “Ateikite, sprendimo miestui vykdytojai! Kiekvienas atsineškite naikinimo įrankį!” 
\par 2 Atėjo šeši vyrai nuo šiaurės aukštutinių vartų. Kiekvienas atsinešė naikinimo įrankį; tarp jų buvo vyras, apsivilkęs drobiniais. Jis laikė dešinėje rankoje rašymo priemones. Atėję jie sustojo prie varinio aukuro. 
\par 3 Izraelio Dievo šlovė pasitraukė nuo cherubo, ant kurio ji būdavo, prie šventyklos slenksčio. Jis šaukė drobiniais apsivilkusiam, kuris turėjo rašymo priemonių: 
\par 4 “Pereik Jeruzalės miesto viduriu ir paženklink kaktoje tuos žmones, kurie dejuoja ir šaukia dėl mieste daromų bjaurysčių”. 
\par 5 Kitiems, man girdint, Jis įsakė: “Eikite paskui jį ir žudykite! Nesigailėkite nieko. 
\par 6 Žudykite senius, jaunuolius, mergaites, vaikus ir moteris, bet nelieskite turinčių ženklą. Pradėkite nuo mano šventyklos!” Jie pradėjo nuo vyresniųjų, buvusių prie šventyklos. 
\par 7 Jis jiems dar įsakė: “Sutepkite šventyklą, pripildykite kiemus užmuštų. Pradėkite!” Jie ėjo ir žudė mieste esančius. 
\par 8 Kai jie juos žudė, aš, likęs vienas, kritau veidu į žemę ir šaukiau: “Viešpatie Dieve, argi sunaikinsi Izraelio likutį, išliedamas savo rūstybę ant Jeruzalės?” 
\par 9 Jis atsakė: “Izraelio ir Judo nusikaltimas yra labai didelis; kraštas ir miestas pilnas nekalto kraujo ir pasileidimo. Jie sako: ‘Viešpats paliko žemę ir nemato mūsų’. 
\par 10 Todėl Aš nesigailėsiu jų ir nebūsiu jiems nuolaidus. Aš jų kelius sugrąžinsiu jiems patiems”. 
\par 11 Vyras, apsirengęs drobiniais, laikantis rašymo priemones, pranešė: “Padariau, kaip įsakei”.



\chapter{10}


\par 1 Aš pažvelgiau aukštyn, ir štai skliaute, virš cherubų galvų, spindėjo lyg safyras, lyg sosto pavidalas. 
\par 2 Dėvinčiam drobiniais Jis kalbėjo: “Eik tarp ratų po cherubais, paimk rankomis žėruojančių anglių, esančių prie cherubų, ir išbarstyk jas mieste”. Aš mačiau, kaip jis ėjo. 
\par 3 Cherubai stovėjo šventyklos pietų pusėje. Kai vyras įėjo, debesis pripildė vidinį kiemą. 
\par 4 Viešpaties šlovė pasitraukė nuo cherubo prie šventyklos įėjimo; debesiui pripildžius namus, Viešpaties šlovės spindesio buvo pilnas visas kiemas. 
\par 5 Cherubų sparnų šlamėjimas buvo girdėti net išoriniame kieme lyg visagalio Dievo balsas. 
\par 6 Kai Jis įsakė drobiniais apsirengusiam: “Paimk ugnies tarp ratų po cherubais”, jis ėjo ir atsistojo prie ratų. 
\par 7 Cherubas ištiesė ranką į ugnį, kuri buvo tarp jų, paėmė jos ir padavė ją vyrui, dėvinčiam drobiniais. Jis, gavęs ugnies, išėjo. 
\par 8 Po cherubų sparnais buvo lyg žmogaus rankos. 
\par 9 Aš stebėjau ir mačiau prie kiekvieno cherubo po ratą. Ratai žėrėjo lyg krištolas. 
\par 10 Visi keturi ratai buvo vienodi ir atrodė, lyg būtų ratas rate. 
\par 11 Jie galėjo judėti nepasisukę keturiomis kryptimis. Į kurią pusę buvo nukreipta galva, jie sekė iš paskos. 
\par 12 Visų keturių kūnai, nugaros, rankos, sparnai ir ratai buvo pilni akių. 
\par 13 Man girdint, ratai buvo pavadinti sūkuriu. 
\par 14 Kiekvienas cherubas turėjo keturis veidus: pirmasis veidas buvo cherubo, antrasis­žmogaus, trečiasis­liūto, ketvirtasis­erelio. 
\par 15 Cherubai pakilo. Tai buvo tos pačios būtybės, kurias mačiau prie Kebaro upės. 
\par 16 Kai cherubai judėjo, su jais judėjo ir ratai. Kai cherubai pakeldavo sparnus pakilti nuo žemės, ratai nepasitraukdavo nuo jų. 
\par 17 Kai anie sustodavo, stovėdavo ir šie, nes būtybių dvasia buvo juose. 
\par 18 Viešpaties šlovė, pakilus nuo šventyklos durų, sustojo ant cherubų. 
\par 19 Man matant, cherubai pakėlė sparnus ir pakilo nuo žemės, o ratai pakilo kartu su jais. Jie sustojo prie Viešpaties šventyklos rytinių vartų, ir Izraelio Dievo šlovė stovėjo virš jų. 
\par 20 Tai buvo tos pačios būtybės, kurias mačiau prie Kebaro upės po Izraelio Dievu. Aš supratau, kad tai buvo cherubai. 
\par 21 Kiekvienas turėjo po keturis veidus bei keturis sparnus. Po sparnais buvo lyg žmogaus rankos. 
\par 22 Jų veidų išvaizda buvo visai tokia, kokią mačiau prie Kebaro upės. Jie judėjo tiesiai pirmyn.



\chapter{11}


\par 1 Dvasia pakėlė mane ir nunešė prie Viešpaties šventyklos rytinių vartų. Ten, prie vartų, mačiau dvidešimt penkis vyrus, tarp jų ir kunigaikščius: Jaazaniją, Azūro sūnų, ir Pelatiją, Benajo sūnų. 
\par 2 Jis tarė: “Žmogaus sūnau, šitie vyrai kuria pavojingus planus ir duoda piktus patarimus šiam miestui. 
\par 3 Jie sako: ‘Dar toli; statykime namus. Miestas yra katilas, o mes­ mėsa!’ 
\par 4 Todėl, žmogaus sūnau, pranašauk prieš juos”. 
\par 5 Viešpaties Dvasia kalbėjo: “Sakyk: ‘Taip sako Viešpats: ‘Aš žinau, ką jūs sakote ir ką galvojate. 
\par 6 Jūs daugelį nužudėte šiame mieste, jų lavonų pripildėte gatves’. 
\par 7 Todėl taip sako Viešpats: ‘Jūsų užmuštieji­tai mėsa, o miestas­ katilas. Bet jus Aš pašalinsiu iš miesto. 
\par 8 Jūs bijote kardo, ir Aš atiduosiu jus kardui. 
\par 9 Aš pašalinsiu jus iš miesto, atiduosiu svetimšaliams ir įvykdysiu jums teismą. 
\par 10 Jūs krisite nuo kardo. Aš teisiu jus prie Izraelio sienos, ir jūs žinosite, kad Aš esu Viešpats. 
\par 11 Miestas nebus jums katilu ir jūs nebūsite mėsa jame. Aš teisiu jus prie Izraelio sienos. 
\par 12 Jūs žinosite, kad Aš esu Viešpats. Jūs nesilaikėte mano įsakymų ir nevykdėte mano sprendimų, bet elgėtės kaip aplink jus gyvenantys pagonys’ ”. 
\par 13 Man pranašaujant, mirė Pelatijas, sūnus Benajo. Aš kritau veidu žemėn, balsiai šaukdamas: “Ak, Viešpatie Dieve, ar Tu visai sunaikinsi Izraelio likutį?” 
\par 14 Viešpats atsakė: 
\par 15 “Žmogaus sūnau, tavo broliai, taip, tavo broliai, yra gyvenantys su tavimi tremtyje, apie kuriuos Jeruzalės gyventojai sako: ‘Jie gyvena toli nuo Viešpaties, o mums duotas šis kraštas!’ 
\par 16 Aš juos toli išvijau ir išsklaidžiau tarp pagonių tautų, tačiau būsiu jiems šventykla tuose kraštuose. 
\par 17 Aš juos surinksiu ir sugrąžinsiu iš tų tautų, kuriose juos išsklaidžiau, ir duosiu jiems Izraelio kraštą. 
\par 18 Sugrįžę jie pašalins visas šlykštybes ir bjaurystes. 
\par 19 Aš duosiu jiems vieną širdį ir įdėsiu jiems naują dvasią. Aš išimsiu iš jų kūno akmeninę širdį ir duosiu jiems kūno širdį. 
\par 20 Jie laikysis mano įsakymų ir vykdys mano nuostatus. Jie bus mano tauta, ir Aš būsiu jų Dievas. 
\par 21 O kurių širdys seka šlykštybes ir bjaurystes, tų kelius sugrąžinsiu ant jų galvų,­sako Viešpats Dievas”. 
\par 22 Cherubai pakėlė sparnus, ratai pajudėjo su jais, o Izraelio Dievo šlovė buvo virš jų. 
\par 23 Viešpaties šlovė pakilo iš miesto vidurio ir nusileido ant kalno į rytus nuo miesto. 
\par 24 Po šitų Dievo Dvasios regėjimų dvasia pakėlė mane ir nunešė Chaldėjon pas tremtinius. Regėjimas, kurį mačiau, išnyko. 
\par 25 Aš papasakojau tremtiniams visa, ką Viešpats man regėjime parodė ir pasakė.



\chapter{12}


\par 1 Viešpats man kalbėjo: 
\par 2 “Žmogaus sūnau, tu gyveni maištingoje tautoje, tarp žmonių, kurie turi akis, bet nemato, turi ausis, bet negirdi. 
\par 3 Todėl, žmogaus sūnau, paruošk savo mantą persikelti ir, jiems matant, dienos metu išeik. Jiems matant, iš savo vietos persikelk į kitą vietą. Gal jie susipras, nors ir yra maištingi. 
\par 4 Dienos metu išnešk savo mantą kaip persikėlimo mantą, o pats išeik vakare kaip einantis į tremtį. 
\par 5 Jiems matant, pralaužk sieną, užsidėk daiktus ant pečių ir sutemus išeik pro ją. 
\par 6 Užsidenk veidą, kad nematytum krašto, nes Aš paskyriau tave ženklu Izraeliui”. 
\par 7 Aš padariau, kaip man buvo įsakyta: savo mantą kaip persikėlimo mantą išnešiau dienos metu, vakare pralaužiau sieną. Kai sutemo, užsidėjęs daiktus ant pečių, išėjau, jiems matant. 
\par 8 Rytą man Viešpats pasakė: 
\par 9 “Žmogaus sūnau, ar izraelitai neklausė tavęs: ‘Ką darai?’ 
\par 10 Sakyk jiems: ‘Taip sako Viešpats: ‘Tai našta Jeruzalės kunigaikščiui ir visiems izraelitams, kurie ten yra’. 
\par 11 Sakyk: ‘Aš esu ženklas jums. Kaip aš dariau, taip atsitiks jiems. Jie persikels ir eis į nelaisvę’. 
\par 12 Kunigaikštis, kuris yra tarp jų, užsidės mantą ant pečių, pralauš mūro sieną ir sutemus pro ją išeis. Veidą jis užsidengs, kad nematytų krašto. 
\par 13 Aš pagausiu jį, nugabensiu į Babiloną, chaldėjų kraštą. Tačiau jis nematys to krašto ir mirs jame. 
\par 14 Visus jo padėjėjus ir karių būrius Aš išsklaidysiu po visus kraštus ir ištrauksiu kardą paskui juos. 
\par 15 Kai juos išsklaidysiu tarp tautų įvairiuose kraštuose, tada jie žinos, kad Aš esu Viešpats. 
\par 16 Bet Aš išsaugosiu mažą likutį nuo kardo, bado ir maro. Jie pasakos apie savo bjaurystes tautoms, kuriose bus ištremti, ir jie žinos, kad Aš esu Viešpats’ ”. 
\par 17 Viešpats man tarė: 
\par 18 “Žmogaus sūnau, valgyk duoną ir gerk vandenį drebėdamas. 
\par 19 Sakyk krašto žmonėms, kad Viešpats Dievas apie Jeruzalės ir Izraelio krašto gyventojus sako: ‘Jie valgys duoną nusiminę ir gers vandenį susirūpinę, nes jų kraštas virs dykyne dėl smurto visų jame gyvenančių. 
\par 20 Miestai ištuštės, ir kraštas sunyks. Tada jūs žinosite, kad Aš esu Viešpats’ ”. 
\par 21 Viešpats vėl man kalbėjo: 
\par 22 “Žmogaus sūnau, kokia tai patarlė, kurią vartojate Izraelio krašte, sakydami: ‘Dienos bėga, o pranašystės neišsipildo’. 
\par 23 Tu sakyk jiems, kad Viešpats Dievas sako: ‘Aš padarysiu galą šitai patarlei, ir ji Izraelyje nebebus girdima. Jau atėjo laikas, ir visos pranašystės išsipildys. 
\par 24 Daugiau nebebus melagingų regėjimų ir apgaulingų pranašavimų Izraelyje. 
\par 25 Aš, Viešpats, kalbėsiu, ir žodis, kurį kalbėsiu, išsipildys ir nebus atidėtas. Taip, dar jūsų dienomis, jūs, maištininkai, Aš paskelbsiu žodį ir jį įvykdysiu,­sako Viešpats Dievas’ ”. 
\par 26 Viešpats man tarė: 
\par 27 “Žmogaus sūnau, izraelitai sako: ‘Regėjimai, kuriuos pranašas regi, ir jo pranašavimai yra apie tolimą ateitį’. 
\par 28 Todėl sakyk jiems: ‘Taip sako Viešpats: ‘Mano žodžiai nebebus atidedami, bet žodis, kurį kalbėjau, išsipildys,­sako Viešpats Dievas’ ”.



\chapter{13}


\par 1 Viešpats kalbėjo man: 
\par 2 “Žmogaus sūnau, pranašauk prieš Izraelio pranašus, kurie pranašauja tai, kas yra jų pačių širdyse: ‘Pasiklausykite Viešpaties žodžių! 
\par 3 Taip sako Viešpats: ‘Vargas kvailiems pranašams, sekantiems savo pačių dvasia ir nieko nemačiusiems. 
\par 4 Izraeli, tavo pranašai yra lyg lapės griuvėsiuose. 
\par 5 Jūs nestojote į Izraelio namų spragas ir nestatėte sienų aplinkui juos, kad atsilaikytų kovoje Viešpaties dieną. 
\par 6 Jie kalbėjo apgaulę ir pranašavo melą. Jie sakė: ‘Taip sako Viešpats’, kai Viešpats nebuvo jų siuntęs, ir suteikė viltį, kad jų žodžiai išsipildys. 
\par 7 Ar jūs ne apgaulę kalbėjote ir ne melą pranašavote, sakydami: ‘Taip sako Viešpats’, kai Aš nekalbėjau. 
\par 8 Kadangi jūs kalbėjote apgaulę ir pranašavote melą, Aš esu prieš jus. 
\par 9 Mano ranka bus prieš pranašus, reginčius apgaulę ir pranašaujančius melą. Jie nepriklausys mano tautai, nebus įrašyti į Izraelio namų sąrašą ir nesugrįš į Izraelio kraštą. Tada jūs žinosite, kad Aš esu Viešpats. 
\par 10 Jie suvedžiojo mano tautą, sakydami: ‘Taika’, kai tuo tarpu nebuvo taikos. Kai jie stato sieną iš akmenų, kiti aptepa ją kalkėmis. 
\par 11 Sakyk tiems, kurie tepa sieną kalkėmis, kad ji sugrius, užėjus smarkiam lietui, siaučiant audrai. 
\par 12 Kai ji sugrius, ar neklaus jūsų: ‘Kur tinkas, kuriuo aptepėte sieną?’ 
\par 13 Aš užsirūstinęs užleisiu griaunančią audrą ir smarkų lietų. Lietus nuplaus tinką, ir mano rūstybės kruša sunaikins ją. 
\par 14 Aš nugriausiu jūsų kalkėmis apteptą sieną iki pamatų. Ji grius, ir jūs žūsite kartu su ja. Tada žinosite, kad Aš esu Viešpats. 
\par 15 Aš išliesiu savo rūstybę ant sienos ir jos aptepėjų ir sakysiu: ‘Nebėra sienos ir tų, kurie ją aptepė: 
\par 16 Izraelio pranašų, Jeruzalei pranašavusių taiką, kai taikos nebuvo,­sako Viešpats Dievas’. 
\par 17 Tu, žmogaus sūnau, atsisuk į tautos dukteris, kurios pranašauja iš savo širdžių ir pranašauk prieš jas. 
\par 18 Sakyk joms: ‘Taip sako Viešpats Dievas: ‘Vargas moterims, siuvančioms burtų raiščius rankoms ir šydus visų žmonių galvoms, kad juos sugautų. Ar, gaudydamos mano tautos žmones, pačios tikitės išlikti gyvos? 
\par 19 Jūs teršiate mane tautoje dėl saujos miežių ir dėl duonos kąsnio, pražudydamos sielas, kurios neturėtų pražūti, ir palikdamos gyvas sielas, kurios neturėtų gyventi, meluodamos mano tautai, kuri klauso jūsų melų’. 
\par 20 Todėl taip sako Viešpats Dievas: ‘Burtų raiščius, kuriais gaudote sielas, Aš nuplėšiu nuo jūsų rankų ir sugautuosius paleisiu į laisvę. 
\par 21 Taipgi nuplėšiu šydus ir savo tautą išlaisvinsiu iš jūsų rankų. Tada žinosite, kad Aš esu Viešpats. 
\par 22 Kadangi savo melais jūs nuliūdinote teisiojo širdį, kurios Aš nenorėjau liūdinti, o nedorėlio rankas sustiprinote, kad neatsiverstų nuo savo pikto kelio ir gyventų, 
\par 23 todėl jūs neberegėsite apgaulės ir liausitės žyniavę, nes Aš išgelbėsiu savo tautą iš jūsų rankų, ir jūs žinosite, kad Aš esu Viešpats’ ”.



\chapter{14}


\par 1 Keli Izraelio vyresnieji atėjo pas mane ir atsisėdo priešais. 
\par 2 Viešpats kalbėjo man: 
\par 3 “Žmogaus sūnau, šitie vyrai pasistatė stabus savo širdyse ir tai, kas veda į nusikaltimą, yra prieš jų akis. Argi klausiamas turėčiau jiems atsakyti? 
\par 4 Todėl sakyk jiems: ‘Taip sako Viešpats Dievas: ‘Kiekvienam izraelitui, kuris pasistato stabus savo širdyje bei žiūri į tai, kas veda į nusikaltimą, ir ateina pas pranašą, Aš, Viešpats, atsakysiu pagal jo stabų daugybę. 
\par 5 Aš nutversiu Izraelio namus jų pačių širdyse, kadangi per savo stabus jie atsitraukė nuo manęs’. 
\par 6 Todėl sakyk izraelitams: ‘Taip sako Viešpats Dievas: ‘Atsiverskite nuo stabų ir nusigręžkite nuo visų savo bjaurysčių’. 
\par 7 Jei izraelitas ar ateivis, gyvenantis Izraelyje, kuris pasistatė stabus savo širdyje ir žiūri į tai, kas veda į nusikaltimą, ateis pas pranašą pasiklausti, tam Aš pats, Viešpats, atsakysiu. 
\par 8 Aš atsigręšiu prieš jį ir padarysiu jį ženklu bei patarle, išnaikindamas jį iš savo tautos. Tada jūs žinosite, kad Aš esu Viešpats. 
\par 9 Jei pranašas leisis suklaidinamas ir duos atsakymą, tai Aš, Viešpats, būsiu suklaidinęs tą pranašą. Aš ištiesiu savo ranką prieš jį ir išnaikinsiu iš savo tautos Izraelio. 
\par 10 Jie abu atsakys už nusikaltimą; pranašo kaltė bus ta pati, kaip ir klausiančiojo, 
\par 11 kad izraelitai ateityje nebeatsitrauktų nuo manęs ir nebenusikalstų, bet būtų mano tauta ir Aš būčiau jų Dievas’ ”. 
\par 12 Viešpats kalbėjo man: 
\par 13 “Žmogaus sūnau, jei kuris kraštas nusikalstų ir sulaužytų ištikimybę man, Aš bausiu jį: atimsiu duonos ramstį, siųsiu badą kraštui ir išnaikinsiu žmones bei gyvulius. 
\par 14 Jei Nojus, Danielius ir Jobas būtų tarp jų, tai jie savo teisumu išgelbėtų tik savo gyvybes,­sako Viešpats Dievas.­ 
\par 15 Jei atiduočiau kraštą laukiniams žvėrims ir jie kraštą paverstų dykyne, per kurią neina joks žmogus dėl žvėrių baimės, 
\par 16 ir jame būtų tie trys vyrai, kaip Aš gyvas,­sako Viešpats,­jie neišgelbėtų nei sūnų, nei dukterų, tik patys išsigelbėtų, o kraštas virstų dykyne. 
\par 17 Jei Aš atiduočiau kraštą kardui, sakydamas: ‘Karde, eik per kraštą ir išnaikink žmones ir gyvulius’, 
\par 18 ir jame būtų tie trys vyrai, kaip Aš gyvas,­sako Viešpats,­jie neišgelbėtų nei sūnų, nei dukterų, tik jie vieni būtų išgelbėti. 
\par 19 Arba jei siųsčiau kraštui marą ir išliečiau savo rūstybę, praliedamas kraują ir išnaikindamas žmones bei gyvulius, 
\par 20 ir jame būtų Nojus, Danielius ir Jobas, kaip Aš gyvas,­sako Viešpats,­jie savo teisumu neišgelbėtų nei sūnų, nei dukterų, tik savo gyvybes. 
\par 21 Tuo labiau, kai Aš, norėdamas Jeruzalėje išnaikinti žmones ir gyvulius, siųsiu savo keturias bausmes: kardą, badą, laukinius žvėris ir marą. 
\par 22 Tačiau mieste liks likutis, kuris bus išvestas su savo sūnumis ir dukterimis. Jiems atėjus pas jus, jūs matysite jų kelius bei darbus ir būsite paguosti dėl nelaimės, kurią siunčiau Jeruzalei. 
\par 23 Jie paguos jus, kai pamatysite jų kelius ir darbus, ir žinosite, kad visa, ką padariau Jeruzalei, nebuvo be priežasties,­sako Viešpats Dievas”.



\chapter{15}


\par 1 Viešpats kalbėjo man: 
\par 2 “Žmogaus sūnau, kuo geresnis vynmedis, lyginant jį su kitais medžiais? 
\par 3 Ar iš jo daro ką nors, ar naudoja padaryti kabliui kam nors pakabinti? 
\par 4 Jis tinka tik sudeginti. Jo abu galai sudega, o vidurys pajuoduoja. Ar jis dar kam nors tinka? 
\par 5 Jei sveikas netiko jokiam daiktui pagaminti, tai ką bekalbėti, kai jis sudega­iš jo jokios naudos! 
\par 6 Todėl Viešpats Dievas sako: ‘Aš padarysiu Jeruzalės gyventojams kaip vynmedžiui, kuris sudega ugnyje. 
\par 7 Aš atsigręšiu į juos, ir, išėję iš vienos ugnies, jie sudegs kitoje. Jūs žinosite, kad Aš esu Viešpats, kai atsigręšiu į juos. 
\par 8 Kraštą paversiu dykuma, nes jie sulaužė ištikimybę man,­sako Viešpats Dievas’ ”.



\chapter{16}


\par 1 Viešpats kalbėjo man: 
\par 2 “Žmogaus sūnau, paskelbk Jeruzalei jos bjaurystes 
\par 3 ir sakyk jai: ‘Tavo kilmė ir giminė yra iš Kanaano krašto. Tavo tėvas buvo amoritas ir motina hetitė. 
\par 4 Kai tu gimei, tavo virkštelė nebuvo nupjauta, nebuvai nuplauta vandeniu nei ištrinta druska, nei vystyklais suvystyta. 
\par 5 Nė vienas nepažvelgė į tave su gailesčiu ir nepadėjo tau. Kai gimei, buvai išmesta laukan. 
\par 6 Aš ėjau pro šalį ir, matydamas tave begulinčią kraujyje, tariau: ‘Gyvenk’. 
\par 7 Aš užauginau tave kaip lauko augalą. Tu užaugai, subrendai, tapai labai graži, iškilo krūtinė, užaugo plaukai, tačiau tu buvai plika ir nuoga. 
\par 8 Kai Aš eidamas pažiūrėjau į tave, buvo atėjęs tavo meilės laikas. Aš apdengiau tavo nuogumą savo apsiaustu, prisiekiau tau, padariau su tavimi sandorą, ir tu tapai mano. 
\par 9 Aš apiploviau tave vandeniu, nuploviau tavo kraują, patepiau aliejumi, 
\par 10 aprengiau margais drabužiais, apaviau brangiais odiniais sandalais, apgaubiau plona drobe, uždėjau šilkinį šydą 
\par 11 ir papuošiau papuošalais: ant rankų uždėjau apyrankes, ant kaklo grandinėlę, 
\par 12 į nosį įvėriau žiedą, į ausis­auskarus ir ant galvos uždėjau puikų vainiką. 
\par 13 Tu pasipuošei auksu ir sidabru, plona drobe, šilkais ir margais audiniais; valgei kvietinius miltus, medų ir aliejų; buvai graži ir pasiekei karališką didybę. 
\par 14 Tavo garsas sklido tautose dėl tavo grožio, nes, pasipuošusi papuošalais, kuriuos tau daviau, pasiekei tobulą grožį,­sako Viešpats Dievas.­ 
\par 15 Pasitikėdama savo grožiu, tu pradėjai paleistuvauti ir atsiduodavai kiekvienam praeiviui. 
\par 16 Savo margais drabužiais papuošei aukštumas ir paleistuvavai jose. Taip niekada nebuvo ir nebus. 
\par 17 Iš mano tau duotų aukso ir sidabro papuošalų pasidarei vyrų atvaizdus ir paleistuvavai su jais. 
\par 18 Tu savo margais audiniais apdengei juos ir mano aliejų bei smilkalus aukojai jiems. 
\par 19 Tau duotą maistą: kvietinius miltus, aliejų ir medų­tu aukojai jiems, kaip malonų kvapą. 
\par 20 Net savo sūnus ir dukteris, kuriuos man pagimdei, aukojai jiems praryti. Ar dar neužteko tau paleistuvystės, 
\par 21 kad mano vaikus žudei ir aukojai jiems, leisdama per ugnį? 
\par 22 Taip elgdamasi ir paleistuvaudama, neatsiminei savo jaunystės dienų, kai plika ir nuoga gulėjai savo kraujyje. 
\par 23 Po visų tavo nedorybių­vargas, vargas tau,­sako Viešpats,­ 
\par 24 tu pasistatei paleistuvystės namus ir įrengei aukštumas kiekvienoje gatvėje. 
\par 25 Kiekvienos gatvės pradžioje įsirengei aukštumas ir savo grožį apdrabstei purvais, atsiduodama kiekvienam praeiviui ir daugindama savo paleistuvystes. 
\par 26 Tu svetimavai su kaimynais, augalotais egiptiečiais, sukeldama mano pyktį. 
\par 27 Dabar Aš ištiesiau savo ranką virš tavęs ir atėmiau tavo dalį, ir atidaviau tave toms, kurios tavęs nekenčia, filistinų dukterims, kurios gėdijosi tavo gašlumo. 
\par 28 Tu ištvirkavai su asirais, nes buvai nepasotinama, bet ir su jais negalėjai pasitenkinti. 
\par 29 Tavo paleistuvystės padaugėjo nuo Kanaano krašto iki Chaldėjos, bet ir to tau dar neužteko. 
\par 30 Kokia nusilpusi tavo širdis,­sako Viešpats Dievas,­jei tu darei visa tai kaip begėdė paleistuvė. 
\par 31 Kiekvienoje gatvėje ir kiekvienoje aikštėje įrengei paleistuvystės namus ir aukštumas. Tu nebuvai paprasta paleistuvė, nes paniekinai užmokestį, 
\par 32 bet svetimautoja žmona, kuri savo vyro vieton priima svetimus. 
\par 33 Jie duoda dovanas paleistuvėms, bet tu pati davei dovanas savo meilužiams ir papirkdavai juos, kad jie ateitų pas tave. 
\par 34 Tavo ištvirkavimas buvo ne toks, koks kitų moterų. Ne paskui tave sekiojo, bet tu duodavai užmokestį, o pati nieko negaudavai. Tuo tu skyreisi iš kitų’. 
\par 35 Paleistuve, išgirsk Viešpaties žodį! 
\par 36 ‘Kadangi tu atidengei savo gėdą, ištvirkaudama su meilužiais, su savo bjauriais stabais ir aukojai jiems savo vaikų kraują, 
\par 37 todėl Aš surinksiu visus tavo meilužius, kuriuos mylėjai, ir tuos, kurių nekentei. Aš juos surinksiu iš visur ir atidengsiu jiems visą tavo nuogumą. 
\par 38 Aš teisiu tave kaip svetimautoją ir žudytoją. Savo rūstybėje ir pavyde praliesiu tavo kraują 
\par 39 ir atiduosiu tave į jų rankas. Jie sugriaus tavo paleistuvystės namus ir sunaikins aukštumas. Jie nuplėš tau drabužius, atims papuošalus ir paliks tave pliką ir nuogą. 
\par 40 Susirinkę prieš tave, jie užmuš tave akmenimis, sukapos kardu, 
\par 41 sudegins tavo namus ir įvykdys teismo sprendimą daugelio moterų akivaizdoje. Taip padarysiu galą tavo paleistuvystei, ir tu nebedalysi daugiau dovanų. 
\par 42 Taip mano rūstybė prieš tave nurims ir pavydas liausis. Aš būsiu ramus ir nebepyksiu. 
\par 43 Tu neatsiminei savo jaunystės dienų, bet savo elgesiu supykdei mane, todėl visus tavo darbus suverčiau tau ant galvos. 
\par 44 Kiekvienas, kas vartoja patarles, sakys apie tave: ‘Kokia motina, tokia ir duktė’. 
\par 45 Tu esi duktė savo motinos, kuri paliko savo vyrą ir vaikus; tu esi sesuo savo seserų, kurios paliko savo vyrus ir vaikus. Jūsų motina buvo hetitė ir tėvas amoritas. 
\par 46 Tavo vyresnioji sesuo yra Samarija su savo dukterimis, gyvenanti tavo kairėje. Tavo jaunesnioji sesuo, gyvenanti tavo dešinėje, yra Sodoma su savo dukterimis. 
\par 47 Tačiau tu nevaikščiojai jų keliais ir nesielgei pagal jų bjaurystes. To buvo per maža tau, todėl tu iškrypai dar daugiau visuose savo keliuose. 
\par 48 Kaip Aš gyvas,­sako Viešpats Dievas,­tavo sesuo Sodoma su savo dukterimis nesielgė taip, kaip tu ir tavo dukterys. 
\par 49 Tavo sesers Sodomos ir jos dukterų nusikaltimas buvo išdidumas, perteklius ir dykinėjimas; beturčiui ir vargšui jos nepadėjo. 
\par 50 Jos kėlėsi puikybėn ir darė bjaurystes mano akivaizdoje. Todėl Aš sunaikinau jas. 
\par 51 Samarija nepadarė nė pusės tavo nuodėmių. Tu ją pralenkei savo bjaurystėmis. Tavo seserys yra teisesnės už tave. 
\par 52 Tu, kuri smerkei savo seseris, kentėk gėdą dėl nuodėmių, kurias padarei, kurios bjauresnės už jų nuodėmes. Jos yra teisesnės už tave. Rausk ir kęsk savo gėdą, nes tu pateisinai savo seseris. 
\par 53 Kai Aš parvesiu Sodomos ir jos dukterų ištremtuosius, taip pat Samarijos ir jos dukterų ištremtuosius, parvesiu ir tavo ištremtuosius kartu su jais, 
\par 54 kad kęstum savo gėdą ir raustum dėl visko, ką darei, būdama joms paguoda. 
\par 55 Tavo seserys, Sodoma ir jos dukterys bei Samarija ir jos dukterys, sugrįš į savo senąją būklę, tada tu ir tavo dukterys sugrįšite į senąją savo būklę. 
\par 56 Savo sesers Sodomos vardo nė neminėdavai savo išdidumo metu, 
\par 57 kol atsidengė tavo nedorybės. Dabar tu esi pajuoka ir panieka Sirijos dukterims ir visiems aplinkui gyvenantiems bei filistinų dukterims. 
\par 58 Tu kentėjai už savo ištvirkavimą ir bjaurystes,­sako Viešpats Dievas.­ 
\par 59 Aš pasielgsiu su tavimi taip, kaip tu pasielgei su manimi. Tu paniekinai priesaiką ir sulaužei sandorą. 
\par 60 Tačiau Aš atsiminsiu savo sandorą su tavimi, padarytą tavo jaunystės dienomis, ir sudarysiu su tavimi amžiną sandorą. 
\par 61 Tada atsiminusi savo kelius, tu gėdysies, kai priimsi savo seseris, vyresniąją ir jaunesniąją, kurias duosiu tau kaip dukteris, bet ne dėl tavo sandoros. 
\par 62 Aš įtvirtinsiu savo sandorą su tavimi ir tu žinosi, kad Aš esu Viešpats. 
\par 63 Tu atsiminsi, gėdysies ir neatversi burnos, kai tau atleisiu visa, ką darei’,­sako Viešpats Dievas”.



\chapter{17}


\par 1 Viešpats kalbėjo man: 
\par 2 “Žmogaus sūnau, užmink mįslę ir papasakok šį palyginimą Izraeliui. 
\par 3 Sakyk: ‘Taip sako Viešpats: ‘Didelis erelis ilgais išskėstais sparnais su įvairiaspalvėmis plunksnomis atskrido į Libaną. Jis nulaužė kedro viršūnę, 
\par 4 nuskynė jauną ūglį ir, nunešęs jį į pirklių kraštą, pasodino prekybos mieste. 
\par 5 Tada jis paėmė šios žemės sėklą ir pasodino ją derlingoje, drėgnoje dirvoje. 
\par 6 Ji augo ir tapo žemu, vešliu vynmedžiu su atžalomis ir šakelėmis. 
\par 7 Atskrido kitas didelis erelis ilgais sparnais ir su daugybe plunksnų. Vynmedis kreipė į jį savo šaknis ir tiesė į jį šakeles iš savo lysvės, kad šis jį palaistytų. 
\par 8 Jis buvo pasodintas drėgnoje, derlingoje žemėje, kur galėjo vešliai augti, leisti šakeles, nešti vaisius ir būti geras vynmedis’. 
\par 9 Sakyk: ‘Taip sako Viešpats Dievas: ‘Ar jam seksis? Ar erelis neišraus jo šaknų ir nesudraskys vaisių, ar jo žaliuojančios šakelės nenuvys? Nereikės ir daugelio vyrų didelės jėgos, kad jį išrautų. 
\par 10 Ar jis persodintas augs? Ar kai rytų vėjas jį palies, jis nenuvys? Jis nudžius lysvėje, kurioje auga’ ”. 
\par 11 Viešpats vėl man kalbėjo: 
\par 12 “Kalbėk maištingai tautai: ‘Ar nesuprantate, ką tai reiškia? Štai atėjo Babilono karalius į Jeruzalę, paėmė karalių su kunigaikščiais ir nusivedė juos į Babiloną. 
\par 13 Su karaliaus palikuoniu jis padarė sandorą ir jį prisaikdino. Krašto galinguosius jis išgabeno, 
\par 14 kad karalystė būtų pažeminta ir neklestėtų, kad ji išliktų, laikydamasi jo sandoros. 
\par 15 Bet jis sukilo prieš jį ir siuntė pasiuntinius į Egiptą, prašydamas žirgų ir daug žmonių. Ar jam seksis? Ar išliks tas, kuris taip daro? Ar, sulaužęs sandorą, jis bus išgelbėtas? 
\par 16 Kaip Aš gyvas,­sako Viešpats,­jis mirs Babilone, krašte to karaliaus, kuris padarė jį karaliumi, kurio priesaiką jis paniekino ir sulaužė sandorą. 
\par 17 Faraonas su didele kariuomene bei daugybe karių jam nepadės, kai babiloniečiai bus supylę pylimus ir padarę įtvirtinimus žmonių pražūčiai. 
\par 18 Jis paniekino priesaiką ir sulaužė sandorą, kai buvo padavęs ranką ir prisiekęs. Tai padaręs, jis neištrūks’. 
\par 19 Todėl Viešpats Dievas sako: ‘Kadangi jis mano priesaiką paniekino ir mano sandorą sulaužė, kaip Aš gyvas­atlyginsiu jam už tai. 
\par 20 Aš jį pagausiu, nugabensiu į Babiloną ir ten su juo bylinėsiuosi dėl ištikimybės sulaužymo. 
\par 21 Jo karių rinktiniai būriai kris nuo kardo; kurie išliks, tuos išsklaidysiu į visus vėjus. Tada jūs žinosite, kad Aš, Viešpats, tai kalbėjau. 
\par 22 Aš pats aukšto kedro viršūnę nulaušiu, ūglį nuskinsiu ir pasodinsiu aukštai iškilusiame kalne. 
\par 23 Jis, pasodintas aukštame Izraelio kalne, išleis šakas, neš vaisių ir išaugs didingu kedru. Įvairūs sparnuočiai gyvens po juo, paukščiai kraus lizdus jo šakų ūksmėje. 
\par 24 Visi krašto medžiai žinos, kad Aš, Viešpats, pažeminu aukštą medį ir paaukštinu žemą, žaliuojantį medį nudžiovinu ir sausą darau žaliuojantį. Aš, Viešpats, pasakiau ir įvykdžiau’ ”.



\chapter{18}


\par 1 Viešpats kalbėjo man: 
\par 2 “Ką reiškia ta patarlė, kurią vartojate Izraelio krašte, sakydami: ‘Tėvai valgė rūgščių vynuogių, o vaikams dantys atšipo?’ 
\par 3 Kaip Aš gyvas,­sako Viešpats,­šios patarlės nebevartosite Izraelyje. 
\par 4 Visi žmonės yra mano: ir tėvas, ir sūnus. Siela, kuri nusikalsta, mirs. 
\par 5 Jei žmogus yra teisus ir daro, kas yra teisinga ir teisėta: 
\par 6 nevalgo aukštumose, negarbina Izraelio stabų, neišniekina artimo žmonos, nesiartina prie moters jos mėnesinių metu, 
\par 7 nė vieno neskriaudžia, skolininkų užstatą grąžina, neapiplėšia, išalkusį pamaitina, nuogą aprengia, 
\par 8 neskolina už nuošimčius ir nereikalauja grąžinti su priedu, nedaro neteisybės, teisingai teisia, 
\par 9 laikosi mano nuostatų ir vykdo mano sprendimus­toks yra teisus; jis tikrai bus gyvas. 
\par 10 Jeigu jam gimsta plėšikas sūnus, praliejantis kraują, kuris daro šiuos dalykus: 
\par 11 valgo aukštumose, išniekina artimo žmoną, 
\par 12 skriaudžia vargšą ir beturtį, plėšikauja, negrąžina užstato, garbina stabus, daro bjaurius dalykus, 
\par 13 skolina už nuošimčius, reikalauja daugiau, negu davė,­argi toks turėtų likti gyvas? Ne, jis neliks gyvas! Kas taip elgiasi­mirs. Jo kraujas kris ant jo. 
\par 14 Bet jei jam gims sūnus, kuris matys tėvo nusikaltimus, susipras ir nedarys nieko panašaus: 
\par 15 nevalgys aukštumose, negarbins Izraelio stabų, neišniekins artimo žmonos, 
\par 16 nė vieno neskriaus, nesulaikys užstato, neplėšikaus, pamaitins alkaną, nuogą aprengs, 
\par 17 neskriaus nė vieno, neims nuošimčių ir nereikalaus grąžinti su priedu, laikysis mano nuostatų ir vykdys mano sprendimus,­toks nemirs dėl savo tėvo kaltės, bet bus gyvas. 
\par 18 Jo tėvas, skriaudęs bei prievartavęs brolį ir daręs pikta visiems, mirs dėl savo nusikaltimų. 
\par 19 Jūs klausiate: ‘Kodėl sūnus neatsako už tėvo nusikaltimus?’ Jei sūnus darė, kas yra teisinga ir teisu, bei laikėsi mano nuostatų, jis tikrai liks gyvas. 
\par 20 Siela, kuri nusikalsta, mirs. Sūnus neatsakys už tėvo nusikaltimą, o tėvas neatsakys už sūnaus kaltes. Teisusis gaus teisiojo atpildą, o nedorėlio nedorybės bus ant jo paties. 
\par 21 Jei nedorėlis atsivers nuo savo nusikaltimų, laikysis mano nuostatų ir darys, kas yra teisinga ir teisu, jis tikrai liks gyvas ir nemirs. 
\par 22 Ankstesni nusikaltimai bus užmiršti ir jam neįskaitomi; jis bus gyvas dėl savo teisumo. 
\par 23 Argi Aš noriu nedorėlio mirties,­sako Viešpats,­o ne kad jis gręžtųsi nuo savo kelių ir būtų gyvas? 
\par 24 Jei teisusis nusigręš nuo savo teisumo, elgsis neteisingai ir darys visas bjaurystes, kurias daro nedorėliai, argi jis gyvens? Ne, jo teisumo darbai nebus jam įskaityti. Jis mirs savo nusikaltimuose ir nuodėmėse. 
\par 25 Jūs sakote: ‘Viešpaties kelias neteisingas’. Paklausyk, Izraeli! Ar mano kelias neteisingas? Ar ne jūsų kelias yra neteisingas? 
\par 26 Jei teisusis nusigręš nuo savo teisumo ir padarys nusikaltimą, jis mirs dėl jo. 
\par 27 Jei nedorėlis nusigręš nuo savo nedorybės ir darys, kas yra teisinga ir teisu, jis išgelbės savo gyvybę. 
\par 28 Kadangi jis susiprato ir atsisakė savo piktų darbų, jis tikrai liks gyvas. 
\par 29 Izraelis sako: ‘Viešpaties kelias neteisingas’. Izraeli, argi mano kelias neteisingas? Argi ne jūsų kelias yra neteisingas? 
\par 30 Aš teisiu jus, o Izraelio namai, kiekvieną pagal jo kelius. Nusigręžkite nuo savo nusikaltimų, kad jūsų neteisybės jūsų nesunaikintų. 
\par 31 Atsisakykite nusikaltimų, kuriuos darėte, įsigykite naują širdį ir naują dvasią! Izraeli, kodėl tu turėtum mirti? 
\par 32 Aš nenoriu mirštančiojo mirties,­sako Viešpats Dievas.­Atsiverskite ir būkite gyvi!”



\chapter{19}


\par 1 Apraudok Izraelio kunigaikščius, 
\par 2 sakydamas: “Tavo motina buvo liūtė, gulinti tarp liūtų! Ji tarp liūtų užaugino savo jauniklius. 
\par 3 Vienas jos jauniklis išaugo jaunu liūtu, kuris, išmokęs sugauti grobį, ėmė ryti žmones. 
\par 4 Tautos, išgirdusios apie jo darbus, pagavo jį ir sukaustytą grandinėmis išvedė į Egiptą. 
\par 5 Liūtė, pamačiusi, kad jos viltys žlugo, antrą savo jauniklį užaugino jaunu liūtu. 
\par 6 Jis, vaikščiodamas tarp liūtų, išmoko sugauti grobį ir ėmė ryti žmones. 
\par 7 Jis griovė tvirtoves ir naikino miestus. Visas kraštas drebėjo nuo jo riaumojimo. 
\par 8 Aplinkinių kraštų tautos sukilo prieš jį, ištiesė savo tinklą ir sugavo jį. 
\par 9 Jie sukaustė jį grandinėmis, uždarė narvan ir, nugabenę pas Babilono karalių, įmetė į kalėjimą, kad jo balsas nebebūtų girdimas Izraelio kalnuose. 
\par 10 Tavo motina buvo kaip vynmedis, pasodintas prie vandens, gausus šakelių ir vaisių. 
\par 11 Jo stiprus ūglis tapo valdovo skeptru, iškilo aukštai tarp šakelių; jis buvo toli matomas. 
\par 12 Bet rūstybėje jis buvo išrautas ir numestas žemėn. Rytų vėjas sudžiovino jo vaisius, šakos sudžiūvo ir ugnis prarijo jas. 
\par 13 Dabar jis pasodintas dykumoje, sausoje ir išdžiūvusioje žemėje. 
\par 14 Ugnis, išėjusi iš jo kamieno, sudegino šakas ir vaisius. Nebėra kamieno, kuris galėtų būti valdovo skeptras. Tai yra rauda ir ji liks rauda”.



\chapter{20}


\par 1 Septintaisiais metais, penkto mėnesio dešimtą dieną, keli Izraelio vyresnieji atėjo pasiklausti Viešpaties ir atsisėdo priešais mane. 
\par 2 Viešpats kalbėjo man: 
\par 3 “Žmogaus sūnau, paklausk Izraelio vyresniųjų, ar jie atėjo pasiklausti manęs? Kaip Aš gyvas, jiems neatsakysiu,­sako Viešpats Dievas.­ 
\par 4 Žmogaus sūnau, ar tu neteisi jų? Primink jiems jų tėvų bjaurystes. 
\par 5 Sakyk jiems: ‘Taip sako Viešpats Dievas: ‘Aš išsirinkau Izraelį, Aš prisiekiau, pakėlęs ranką, Jokūbo namų palikuonims, apsireiškiau jiems Egipto krašte, sakydamas: ‘Aš esu Viešpats, jūsų Dievas’. 
\par 6 Tą dieną Aš, pakėlęs ranką, prisiekiau, kad išvesiu juos iš Egipto krašto ir nuvesiu į mano numatytą jiems žemę, tekančią pienu ir medumi,­į geriausią kraštą iš visų. 
\par 7 Aš jiems sakiau: ‘Kiekvienas pašalinkite bjaurystes nuo savo akių ir nesusitepkite Egipto stabais. Aš esu Viešpats, jūsų Dievas’. 
\par 8 Bet jie buvo maištingi ir neklausė manęs, nė vienas nepašalino nuo savo akių bjaurysčių ir nepaliko Egipto stabų. Tuomet Aš ketinau išlieti savo rūstybę ant jų dar Egipto krašte. 
\par 9 Tačiau dėl savo vardo susilaikiau, kad jo nepažeminčiau tarp pagonių tautų, kur jie gyveno, kurių akivaizdoje Aš apsireiškiau, žadėdamas išvesti juos iš Egipto krašto. 
\par 10 Aš išvedžiau juos iš Egipto krašto ir, atvedęs į dykumą, 
\par 11 daviau jiems savo įsakymus ir pamokiau nuostatų, kurių laikydamasis žmogus gyvens. 
\par 12 Daviau sabatą, ženklą tarp jų ir manęs, kad jie žinotų, jog Aš esu Viešpats, kuris juos pašventinu. 
\par 13 Izraelitai maištavo prieš mane dykumoje. Jie nesilaikė mano įsakymų ir atmetė mano nuostatus, kuriuos vykdydamas žmogus gyvens; jie sutepė ir mano sabatus. Aš norėjau išlieti savo rūstybę dykumoje ir juos visai sunaikinti, 
\par 14 tačiau to nepadariau dėl savo vardo, kad nebūčiau niekinamas pagonių tautų, kurių akivaizdoje juos išvedžiau iš Egipto. 
\par 15 Tačiau, pakėlęs ranką, prisiekiau dykumoje, kad jų neįvesiu į pažadėtąją žemę, tekančią pienu ir medumi, į kraštą, geriausią iš visų. 
\par 16 Jie atmetė mano įsakymus, nesilaikė nuostatų ir sutepė sabatus, nes jų širdis buvo linkusi prie stabų. 
\par 17 Aš pasigailėjau jų ir visų nesunaikinau ir nepadariau jiems galo dykumoje. 
\par 18 Jų vaikams įsakiau: ‘Nesilaikykite savo tėvų nuostatų, nesielkite kaip jie ir nesusiterškite stabais. 
\par 19 Aš esu Viešpats, jūsų Dievas. Laikykitės mano įsakymų ir vykdykite mano nuostatus, 
\par 20 švęskite sabatą kaip ženklą tarp manęs ir jūsų ir žinokite, kad Aš esu Viešpats, jūsų Dievas. 
\par 21 Bet ir jų vaikai maištavo prieš mane. Jie nesilaikė mano įsakymų ir nevykdė mano nuostatų, kuriuos vykdydamas žmogus gyvens. Norėjau išlieti ant jų savo rūstybę dykumoje, 
\par 22 tačiau susilaikiau dėl savo vardo, kad nebūčiau niekinamas tarp pagonių, kurių akivaizdoje juos išvedžiau iš Egipto. 
\par 23 Pakėlęs ranką, dykumoje prisiekiau, kad juos išsklaidysiu ir išblaškysiu tarp pagonių, 
\par 24 nes jie nesilaikė mano įsakymų, atmetė nuostatus, sutepė sabatą ir sekė tėvų stabus. 
\par 25 Tad ir Aš daviau jiems klaidingų įsakymų ir tokių nuostatų, kurių laikydamiesi jie negalėjo gyventi. 
\par 26 Jie susitepė savo dovanomis, leisdami savo pirmagimius per ugnį, kad paversčiau juos dykyne ir galiausiai jie pažintų, jog Aš esu Viešpats’. 
\par 27 Todėl, žmogaus sūnau, kalbėk izraelitams ir sakyk jiems: ‘Taip sako Viešpats Dievas: ‘Jūsų tėvai piktžodžiavo ir paniekino mane. 
\par 28 Kai juos įvedžiau į kraštą, kurį, pakėlęs ranką, buvau prisiekęs jiems duoti, jie, pamatę aukštesnę kalvą arba šakotą medį, aukojo ten aukas ir nešė dovanas, sukeldami mano rūstybę’. 
\par 29 Aš juos klausiau: ‘Kokia tai aukštuma, į kurią einate?’ Iki šios dienos jos vardas Bama. 
\par 30 Todėl klausk izraelitų: ‘Ar jūs nesusitepę, kaip ir jūsų tėvai, bjauriais stabais? 
\par 31 Kai jūs aukojate aukas ir deginate vaikus, patys susitepate stabais iki šios dienos. Ar tad Aš turėčiau jums atsakyti? Kaip Aš gyvas, Aš neatsakysiu jums,­sako Viešpats Dievas.­ 
\par 32 Niekada neįvyks, kaip jūs manote ir sakote: ‘Kaip pagonys ir kitų kraštų tautos, taip ir mes tarnausime medžiui ir akmenims’. 
\par 33 Kaip Aš gyvas,­sako Viešpats,­Aš valdysiu jus galinga ištiesta ranka, išliedamas savo rūstybę. 
\par 34 Aš išvesiu jus iš tautų, surinksiu iš kraštų, kur esate išsklaidyti, savo galinga ir ištiesta ranka ir išliedamas savo rūstybę. 
\par 35 Aš nuvesiu jus į tautų dykumą ir ten su jumis bylinėsiuosi veidas į veidą. 
\par 36 Kaip bylinėjausi su jūsų tėvais Egipto šalies dykumoje, taip bylinėsiuosi su jumis. 
\par 37 Tada jus pervesiu po lazda ir įvesiu į sandoros ryšius. 
\par 38 Aš atskirsiu nuo jūsų maištininkus ir man nusikaltusius, išvesiu juos iš krašto, kuriame jie dabar svetimi, bet į Izraelio kraštą jie nesugrįš. Tada žinosite, kad Aš esu Viešpats’. 
\par 39 Apie jus, izraelitai, Viešpats Dievas taip sako: ‘Kurie nenorite manęs klausyti, eikite ir tarnaukite savo stabams, bet mano šventojo vardo nebesutepkite savo aukomis ir stabais. 
\par 40 Šventame mano kalne, Izraelio aukštame kalne, man tarnaus visas Izraelis. Ten Aš priimsiu juos ir pareikalausiu iš jūsų aukų, pirmavaisių ir pašvęstų daiktų. 
\par 41 Aš priimsiu jus ir malonų jūsų aukų kvapą, kai išvesiu jus iš tautų ir surinksiu iš kraštų, kur esate išsklaidyti, ir būsiu pašventintas jumyse pagonių akivaizdoje. 
\par 42 Tada žinosite, kad Aš esu Viešpats, kai jus įvesiu į Izraelio kraštą, kurį prisiekiau duoti jūsų tėvams. 
\par 43 Ten atsiminsite savo kelius ir darbus, kuriais susitepėte, ir bjaurėsitės savęs dėl savo padarytų piktadarysčių. 
\par 44 Tada žinosite, kad Aš esu Viešpats, kai pasielgsiu su jumis dėl savo vardo, neatsižvelgdamas į jūsų nedorus kelius ir piktus darbus’ ”. 
\par 45 Viešpats kalbėjo man: 
\par 46 “Žmogaus sūnau, žiūrėk pietų pusėn, kalbėk ir pranašauk miškui pietų šalyje. 
\par 47 Sakyk miškui pietuose: ‘Išgirsk Viešpaties žodį. Aš įžiebsiu ugnį ir sudeginsiu visus tavo žalius ir sausus medžius. Liepsnos neužges, kol visa bus sudeginta nuo pietų iki šiaurės. 
\par 48 Tada visi matys, kad Aš, Viešpats, padegiau mišką, ir jis neužges’ ”. 
\par 49 Tada aš atsiliepiau: “Ak, Viešpatie Dieve, jie sako apie mane: ‘Ar jis ne palyginimais kalba?’ ”



\chapter{21}


\par 1 Viešpats kalbėjo man: 
\par 2 “Žmogaus sūnau, žvelk į Jeruzalę, kalbėk prieš šventyklą ir pranašauk prieš Izraelį. 
\par 3 Sakyk Izraelio kraštui: ‘Taip sako Viešpats Dievas: ‘Aš esu prieš tave; ištraukęs kardą iš makšties, sunaikinsiu tavo teisiuosius ir nedorėlius. 
\par 4 Kadangi naikinsiu ir teisiuosius, ir nedorėlius, tai mano kardas siaus nuo pietų iki šiaurės. 
\par 5 Tada kiekvienas žinos, jog Aš, Viešpats, ištraukęs kardą iš makšties, jo nepadėsiu’. 
\par 6 Tu, žmogaus sūnau, dejuok ir dūsauk jų akivaizdoje, lyg būtum baisioje kančioje. 
\par 7 Kai jie tave klaus: ‘Kodėl dūsauji?’, atsakyk jiems: ‘Dėl žinios, kuri ateina’. Širdys liūdės, rankos suglebs, drąsa išnyks, keliai drebės. Tai ateina ir įvyks,­sako Viešpats Dievas”. 
\par 8 Viešpats kalbėjo man: 
\par 9 “Žmogaus sūnau, pranašauk: ‘Taip sako Viešpats: ‘Kardas pagaląstas ir nušveistas. 
\par 10 Pagaląstas žudyti, nušveistas, kad blizgėtų kaip žaibas. Ar mes galime džiaugtis? Jis paniekina mano sūnaus skeptrą kaip paprastą lazdą. 
\par 11 Kardas nušveistas, pagaląstas ir įduotas į žudiko rankas’. 
\par 12 Žmogaus sūnau, šauk ir dejuok! Jis paruoštas mano tautai, visiems Izraelio kunigaikščiams. Siaubas dėl kardo apims tautą. Plok sau per šlaunis. 
\par 13 Tai išbandymas; kardas paniekins netgi skeptrą ir jo nebebus,­ sako Viešpats Dievas.­ 
\par 14 Žmogaus sūnau, pranašauk, suplok rankomis. Tegu trečią kartą kardas žudo du kartus daugiau; tai žudymo kardas, siekiąs visus. 
\par 15 Aš nukreipiau kardą į vartus, kad širdys išsigąstų ir kritusiųjų padaugėtų. Kardas nušveistas blizga, pagaląstas žudynėms. 
\par 16 Kirsk į dešinę ir į kairę, kur ašmenys pasiekia. 
\par 17 Aš rankomis suplosiu ir patenkinsiu savo rūstybę. Aš, Viešpats, tai pasakiau”. 
\par 18 Viešpats kalbėjo man: 
\par 19 “Žmogaus sūnau, pažymėk du kelius, kuriais Babilono karaliaus kardas galėtų ateiti. Abu keliai prasidės viename krašte. Kelių išsišakojime pastatyk kelrodį. 
\par 20 Jis terodo kardui kelią į amonitų Rabą ir į Judo sutvirtintą miestą Jeruzalę. 
\par 21 Babilono karalius sustojo kryžkelėje, kelių išsišakojime, ir buria: maišo strėles, klausia stabų, apžiūri kepenis. 
\par 22 Į jo dešinę pateko Jeruzalės burtas, kad paskirtų karo vadus, duotų įsakymą žudynėms, sukeltų kovos šauksmą, statytų sienų griovimo įtaisus prie vartų, supiltų pylimą, statytų įtvirtinimus. 
\par 23 Jiems atrodys, jog tai apgaulingas burtas. Bet kadangi jie prisiekė, jis primins jiems nusikaltimus. 
\par 24 Todėl Viešpats Dievas taip sako: ‘Jūs prisimenate savo kaltes ir jūsų darbai bei nuodėmės stovi jūsų akyse; jūs atsakysite už juos. 
\par 25 Susitepęs ir nedoras Izraelio kunigaikšti, tavo bausmės diena atėjo, nes tavo nedorybėms padarytas galas. 
\par 26 Nusiimk vainiką ir karaliaus karūną! Nebebus, kaip buvo. Pažemintasis bus paaukštintas, o aukštasis­pažemintas. 
\par 27 Griuvėsiais, griuvėsiais, griuvėsiais Aš jį paversiu! Jo nebebus, kol ateis turintis teisę valdyti. Jam pavesiu visa’. 
\par 28 Tu, žmogaus sūnau, pranašauk apie amonitus ir jų pasityčiojimą: ‘Kardas pagaląstas ir nušveistas žudynėms. 
\par 29 Jūsų regėjimai­apgaulė, pranašavimai­melas. Bausmė jums ateis už jūsų kaltes. 
\par 30 Ar Aš paslėpsiu savo kardą? Aš tave teisiu krašte, kur gimei, iš kurio esi kilęs. 
\par 31 Aš išliesiu ant tavęs savo rūstybę, tu pajusi mano keršto liepsnas. Atiduosiu tave į žiaurių ir įgudusių žudyti žmonių rankas. 
\par 32 Tu būsi kuras liepsnoms, tavo kraujas liks ant žemės, visi užmirš tave. Aš, Viešpats, tai kalbėjau’ ”.



\chapter{22}


\par 1 Viešpats kalbėjo man: 
\par 2 “Žmogaus sūnau, ar tu neteisi ir neskelbsi sprendimo kruvinam miestui? Paskelbk jam visas jo bjaurystes. 
\par 3 Sakyk: ‘Taip sako Viešpats Dievas: ‘Tai miestas, praliejęs nekaltą kraują, susitepęs stabų garbinimu. Todėl tavo valanda priartėjo. 
\par 4 Krauju, kurį praliejai, nusikaltai, stabais, kuriuos pasidarei, susitepei. Tuo priartinai sau galą. Todėl būsi pajuoka tautoms ir pasityčiojimu kraštams. 
\par 5 Arti ir toli gyvenantys tyčiosis iš tavęs kaip iš negarbingo ir pagarsėjusio sąmyšiu. 
\par 6 Štai Izraelio kunigaikščiai, kiekvienas naudojasi savo galia, kad pralietų kraują. 
\par 7 Tavyje niekina tėvą ir motiną, prislegia ateivį, skriaudžia našlaitį ir našlę. 
\par 8 Tu niekini mano šventyklą ir nesilaikai sabatų. 
\par 9 Tavyje žmonės, kurie šmeižia, siekdami kraujo, valgo kalnuose, tavo viduryje jie paleistuvauja. 
\par 10 Tavyje jie atidengė tėvo nuogumą ir pažemino tą, kuri buvo nešvari mėnesinių metu. 
\par 11 Vienas darė bjaurystę su artimo žmona, kitas išniekino savo marčią, dar kitas pažemino savo seserį, savo tėvo dukterį. 
\par 12 Tavyje jie ima kyšius nuslėpti pralietą kraują. Tu imi nuošimčius bei reikalauji grąžinti daugiau, tu, išnaudodama artimą, godžiai sieki pelno. Mane gi užmiršai,­sako Viešpats Dievas.­ 
\par 13 Aš suplojau rankomis dėl jūsų nesąžiningo pelno ir kraujo, pralieto tarp jūsų. 
\par 14 Ar tu būsi drąsus ir tvirtai laikysies tada, kai Aš bausiu tave? Aš, Viešpats, tai pasakiau ir įvykdysiu. 
\par 15 Aš išsklaidysiu tave tautose bei išblaškysiu kraštuose ir taip pašalinsiu tavo nešvarą. 
\par 16 Tu būsi tautų paniekintas. Tada tu žinosi, kad Aš esu Viešpats’ ”. 
\par 17 Viešpats kalbėjo man: 
\par 18 “Žmogaus sūnau, Izraelis virto nuodegomis. Jie visi: varis, cinkas, geležis, švinas­tapo sidabro priemaišomis. 
\par 19 Todėl taip sako Viešpats Dievas: ‘Kadangi jūs visi virtote nuodegomis, Aš surinksiu jus į Jeruzalę, 
\par 20 kaip surenkamas sidabras, varis, geležis, švinas bei cinkas ir sumetamas į krosnį tirpdyti. Taip Aš savo rūstybėje jus surinksiu ir tirpdysiu. 
\par 21 Taip, aš surinksiu jus Jeruzalėje ir pūsiu į jus savo įtūžį, nuo kurio jūs sutirpsite. 
\par 22 Kaip sidabras krosnyje ištirpdomas, taip ir jūs būsite tirpdomi. Tada jūs žinosite, kad Aš, Viešpats, išliejau savo rūstybę ant jūsų’ ”. 
\par 23 Viešpats man kalbėjo: 
\par 24 “Žmogaus sūnau, sakyk jiems: ‘Tu esi neapvalyta žemė, tavyje nelijo mano pykčio dieną’. 
\par 25 Jos pranašai rengia sąmokslą kaip riaumojantys liūtai, kurie drasko grobį. Jie rijo žmones, plėšė jų turtus, didino našlių skaičių. 
\par 26 Kunigai iškraipė mano įstatymą ir išniekino mano šventus daiktus, jie nedarė skirtumo tarp švento ir nešvento, nemokė atskirti nešvaraus nuo švaraus, užmerkė akis dėl sabatų ir Aš niekinamas tarp jų. 
\par 27 Kunigaikščiai­kaip draskantys vilkai, kurie plėšia grobį; jie pralieja kraują, žudo žmones, godžiai siekdami pelno. 
\par 28 Pranašai aptepė juos kalkėmis, kalbėdami jiems apgaulingus regėjimus ir skelbdami melagingus pranešimus, sakydami: ‘Taip sako Viešpats Dievas’, kai Viešpats nebuvo kalbėjęs. 
\par 29 Krašto žmonės smurtauja ir plėšikauja, skriaudžia vargšus ir beturčius bei neteisėtai spaudžia ateivius. 
\par 30 Aš ieškojau tarp jų žmogaus, kuris pastatytų sieną ir stotųsi spragoje tarp manęs ir mano tautos, kad jos nesunaikinčiau, bet nė vieno neradau. 
\par 31 Todėl Aš išliejau ant jų savo pyktį ir rūstybės ugnimi sunaikinau juos. Jų darbus suverčiau ant jų pačių galvų,­sako Viešpats Dievas”.



\chapter{23}


\par 1 Viešpats kalbėjo man: 
\par 2 “Žmogaus sūnau, buvo dvi moterys, vienos motinos dukterys. 
\par 3 Jos paleistuvavo jau savo jaunystėje Egipte. 
\par 4 Vyresnioji buvo vardu Ohola ir jos sesuo Oholiba. Jos buvo mano ir pagimdė sūnų bei dukterų. Ohola­tai Samarija, o Oholiba­tai Jeruzalė. 
\par 5 Ohola buvo man neištikima. Ji geidė savo meilužių, kaimynystėje gyvenančių asirų. 
\par 6 Jie buvo apsirengę mėlynai, vadai ir kunigaikščiai, jauni, gražūs vyrai, raiteliai, jojantys ant savo žirgų. 
\par 7 Ji paleistuvavo su jais, su rinktiniais asirais ir visais, kurių ji geidė. Jų stabais ji suteršė save. 
\par 8 Ji nesiliovė paleistuvavusi ir su egiptiečiais, savo jaunystės meilužiais. 
\par 9 Todėl Aš atidaviau ją jos meilužiams asirams, kurių ji geidė. 
\par 10 Jie atidengė jos nuogumą, paėmė jos sūnus bei dukteris, o ją pačią nužudė kardu. Ji tapo priežodis tarp moterų, kai jie įvykdė jai teismą. 
\par 11 Jos sesuo Oholiba visa tai matė, tačiau gašlumu ir paleistuvystėmis pralenkė net savo seserį. 
\par 12 Ji geidė savo kaimynų asirų, vadų ir kunigaikščių, išsipusčiusių raitelių, gražių, jaunų vyrų. 
\par 13 Aš mačiau, kad ji suteršė save; jos abi ėjo vienu keliu. 
\par 14 Ji daugino savo paleistuvystes, nes kai pamatė sienoje nupieštus vyrų paveikslus, raudona spalva nupieštus atvaizdus chaldėjų, 
\par 15 susijuosusių strėnas diržais, ant galvų užsidėjusių margus turbanus, atrodančių kaip Babilono kunigaikščiai, kilę iš Chaldėjos, 
\par 16 ji ėmė geisti jų ir siuntė pasiuntinių į Chaldėją. 
\par 17 Babiloniečiai atėjo į jos meilės guolį ir suteršė ją savo paleistuvystėmis. Suteršta ji pasitraukė nuo jų. 
\par 18 Ji neslėpė savo paleistuvysčių ir atidengė savo nuogumą. Tada Aš pasitraukiau nuo jos, kaip buvau pasitraukęs nuo jos sesers. 
\par 19 Bet ji daugino savo paleistuvystes, prisimindama jaunystės dienas, kai ji buvo paleistuvė Egipto žemėje. 
\par 20 Ji geidė savo meilužių, kurių kūnai kaip asilų kūnai ir sėklos plūdimas kaip arklių. 
\par 21 Tu prisiminei savo jaunystės ištvirkimą, kai egiptiečiai glamonėjo tavo jaunas krūtis. 
\par 22 Todėl, Oholiba, taip sako Viešpats Dievas: ‘Aš pakelsiu prieš tave tavo meilužius, nuo kurių tu nusisukai, ir atvesiu juos prieš tave iš visų pusių: 
\par 23 babiloniečius, chaldėjus, Pekodą, Šoją, Koją ir asirus­gražius, jaunus vyrus, vadus ir kunigaikščius, valdytojus ir žymius žmones, visus raitus ant žirgų. 
\par 24 Daugybė ginkluotų pulkų ateis su kovos vežimais ir žirgais, apstatys tave iš visų pusių didžiaisiais ir mažaisiais skydais bei šalmais. Aš atiduosiu tave jų teismui. Jie teis tave pagal savo nuostatus. 
\par 25 Aš nukreipsiu prieš tave savo pavydą, ir jie žiauriai pasielgs su tavimi: nupjaus tau nosį bei ausis, likusius išžudys kardu, išplėš tavo sūnus bei dukteris ir sudegins, kas liks. 
\par 26 Jie nuplėš tau drabužius ir atims brangenybes. 
\par 27 Taip aš padarysiu galą tavo ištvirkavimui su egiptiečiais. Tu nebenorėsi daugiau egiptiečių nei matyti, nei jų prisiminti. 
\par 28 Aš atiduosiu tave į rankas tų, kurių tu nekenti, nuo kurių tu pasitraukei. 
\par 29 Jie pasielgs su tavimi labai žiauriai, išplėš visą turtą ir paliks tave pliką bei nuogą. Taip bus apnuogintas tavo pasileidimas ir ištvirkavimas. 
\par 30 Tai padarysiu tau dėl tavo paleistuvavimo su tautomis, kurių stabais tu susitepei. 
\par 31 Tu ėjai savo sesers keliu, todėl jos taurę įduosiu į tavo rankas. 
\par 32 Tu gersi savo sesers taurę, gilią ir didelę, iš tavęs tyčiosis ir niekins tave, nes joje daug telpa. 
\par 33 Tu pasigersi ir būsi pilna skausmų. Taurė skausmo ir naikinimo, tavo sesers Samarijos taurė. 
\par 34 Išgersi ją iki dugno, šukes nulaižysi ir draskysi savo krūtis, nes Aš tai pasakiau,­sako Viešpats Dievas.­ 
\par 35 Kadangi mane užmiršai ir atgręžei man nugarą, tai kentėk už savo ištvirkavimą ir paleistuvystę’ ”. 
\par 36 Viešpats sakė man: “Žmogaus sūnau, ar tu neteisi Oholos ir Oholibos? Primink joms jų bjaurystes! 
\par 37 Jos svetimavo ir savo rankas sutepė nekaltu krauju. Jos svetimavo su stabais ir aukojo jiems vaikus, kuriuos man pagimdė! 
\par 38 Tą pačią dieną jos sutepė mano šventyklą ir nesilaikė sabatų. 
\par 39 Tą pačią dieną, kai jos aukojo vaikus stabams, jos ėjo į mano šventyklą ir tuo sutepė ją. 
\par 40 Be to, tu siuntei pasiuntinius ir kvietei vyrus iš toli. Dėl jų prauseisi, dažeisi akis ir puošeisi brangenybėmis. 
\par 41 Atsisėdai ant brangaus gulto; padengei stalus, ant jų padėjai mano smilkalų ir aliejaus. 
\par 42 Nerūpestingos minios balsai buvo girdimi joje, ir kartu su žmonėmis iš minios buvo atvesti sebiečiai iš dykumos, kurie dėjo apyrankes ant jų rankų ir puošnias karūnas ant jų galvų. 
\par 43 Tada tariau apie tą seną svetimautoją: ‘Ar jie ir dabar paleistuvaus su ja?’ 
\par 44 Bet jie eidavo pas ją, kaip įeinama pas paleistuvę. Taip jie įeidavo pas Oholą ir Oholibą, gašlias moteris. 
\par 45 Bet teisūs vyrai teis jas kaip svetimautojas ir žudikes, nes jos svetimavo ir kraujas yra ant jų rankų. 
\par 46 Nes taip sako Viešpats: ‘Ar atvesiu prieš jas minią ir atiduosiu jas sunaikinti ir apiplėšti. 
\par 47 Minia užmuš jas akmenimis ir sukapos kardu, jų sūnus ir dukteris nužudys ir jų namus sudegins. 
\par 48 Taip Aš padarysiu galą paleistuvystei krašte, kad visos moterys pasimokytų ir neištvirkautų kaip jūs. 
\par 49 Jūsų paleistuvystė kris ant jūsų galvų, jūs kentėsite už nuodėmes su savo stabais. Tada žinosite, kad Aš esu Viešpats Dievas’ ”.



\chapter{24}


\par 1 Viešpats kalbėjo man devintaisiais metais, dešimto mėnesio dešimtą dieną: 
\par 2 “Žmogaus sūnau, užrašyk šios dienos vardą, nes Babilono karalius šiandien pradėjo Jeruzalės puolimą. 
\par 3 Sakyk maištingiems namams šį palyginimą: ‘Taip sako Viešpats: ‘Kaisk katilą su vandeniu! 
\par 4 Įdėk į jį geriausios mėsos gabalus: šlaunį, petį ir geriausius kaulus. 
\par 5 Mėsai paimk geriausią gyvulį iš bandos, po katilu prikrauk malkų, kad ir kaulai gerai išvirtų’. 
\par 6 Todėl taip sako Viešpats: ‘Vargas krauju suteptam miestui­prisvilusiam katilui. Išimk iš jo gabalą po gabalo, nemesdamas burto. 
\par 7 Kraujas yra jo viduryje. Jis buvo pralietas ant plikos uolos, ne ant žemės, kur dulkės jį pridengtų. 
\par 8 Kad sukelčiau rūstybę ir įvykdyčiau kerštą, leidau Jeruzalei kraują pralieti ant plikos uolos, kad jis nebūtų pridengtas’. 
\par 9 Todėl taip sako Viešpats: ‘Vargas krauju suteptam miestui. Aš sukrausiu jam didelį laužą. 
\par 10 Sukrauk malkas, užkurk ugnį, išvirk mėsą, įdėk prieskonių, kaulai tesudega. 
\par 11 Pastatyk tuščią katilą ant degančių žarijų. Teįkaista varis iki raudonumo, tada nešvarumas ir svilėsiai sudegs. 
\par 12 Jis nuvargino save melais, jo svilėsiai neatšoka nuo jo. Tesudega svilėsiai ugnyje. 
\par 13 Tavo susitepimas bjaurus. Aš norėjau nuvalyti tave, bet tu nesileidai nuvalomas, todėl liksi suteptas, kol mano įtūžis prieš tave nurims. 
\par 14 Aš, Viešpats, taip pasakiau, ir tai įvyks. Aš tai įvykdysiu ir nesigailėsiu. Teisiu tave pagal tavo kelius ir darbus,­sako Viešpats Dievas’ ”. 
\par 15 Viešpats kalbėjo man: 
\par 16 “Žmogaus sūnau, staiga Aš atimsiu tavo akių pasigėrėjimą. Bet tu neraudok, neverk ir neliek ašarų. 
\par 17 Dūsauk tyliai, bet neapraudok mirusios. Apsirišk galvą raiščiu, apsiauk kurpėmis, neuždenk savo burnos ir nevalgyk gedinčiųjų maisto”. 
\par 18 Rytą aš kalbėjau tautai, o vakare mirė mano žmona. Kitą rytą padariau, kaip man buvo liepta. 
\par 19 Žmonės klausė manęs: “Ar nepasakysi, ką reiškia mums tai, ką tu darai?” 
\par 20 Aš atsakiau jiems: “Viešpaties žodis atėjo man: 
\par 21 ‘Sakyk Izraelio namams: ‘Taip sako Viešpats Dievas: ‘Aš suteršiu savo šventyklą, jūsų pasididžiavimą, akių pasigėrėjimą bei sielos ilgesį. Jūsų likę sūnūs bei dukterys bus išžudyti kardu. 
\par 22 Tada jūs darysite, kaip aš dariau: jūs neuždengsite savo burnos ir nevalgysite gedinčiųjų maisto. 
\par 23 Ant jūsų galvų bus raiščiai ir ant kojų kurpės, neraudosite ir neverksite, bet nyksite dėl savo nusikaltimų ir liūdėsite. 
\par 24 Ezechielis bus ženklas jums. Jūs darysite taip, kaip jis darė. Kai visa tai įvyks, jūs žinosite, kad Aš esu Viešpats Dievas’. 
\par 25 Žmogaus sūnau, kai Aš atimsiu jiems jų pasididžiavimą, akių pasigėrėjimą ir sielos ilgesį, taip pat jų sūnus bei dukteris, 
\par 26 tą dieną ateis pabėgėlis pas tave ir praneš tai tau. 
\par 27 Tada tavo burna atsivers, tu kalbėsi ir nebūsi nebylys. Tu būsi jiems ženklas, ir jie žinos, kad Aš esu Viešpats’ ”.



\chapter{25}


\par 1 Viešpats kalbėjo man: 
\par 2 “Žmogaus sūnau, pranašauk prieš amonitus. 
\par 3 Sakyk amonitams: ‘Išgirskite Viešpaties Dievo žodį! Taip sako Viešpats Dievas: ‘Kadangi jūs džiūgavote dėl to, kad mano šventykla išniekinta, Izraelio šalis sunaikinta ir Judo namai išvesti nelaisvėn, 
\par 4 atiduosiu jus rytų šalies gyventojams. Jie išties savo palapines tarp jūsų ir įrengs stovyklas, jie valgys jūsų vaisius ir gers jūsų pieną. 
\par 5 Aš padarysiu Rabą vieta kupranugariams ir amonitų kraštą­gardais avims. Tada jūs žinosite, kad Aš esu Viešpats’. 
\par 6 Nes taip sako Viešpats: ‘Kadangi jūs plojote rankomis, trypėte kojomis ir nuoširdžiai džiaugėtės niekindami Izraelio žemę, 
\par 7 todėl Aš ištiesiu savo ranką prieš jus ir atiduosiu jus tautoms kaip grobį. Aš išnaikinsiu jus tarp tautų ir pražudysiu jus visose šalyse. Aš sunaikinsiu jus, ir tada žinosite, kad Aš esu Viešpats’ ”. 
\par 8 Viešpats Dievas sako: “Kadangi Moabas ir Seyras sako, kad Judo namai yra kaip visos kitos tautos, 
\par 9 Aš atversiu Moabo šoną, pradėdamas nuo gražiausių miestų, krašto pasididžiavimo: Bet Ješimotų, Baal Meono ir Kirjataimų, 
\par 10 Aš juos atiduosiu kartu su amonitais rytų šalies gyventojams, ir amonitai nebebus minimi tarp tautų. 
\par 11 Taip Aš įvykdysiu teismą Moabui, ir jis žinos, kad Aš esu Viešpats”. 
\par 12 Taip sako Viešpats Dievas: “Kadangi edomitai kerštingai elgėsi su Judo namais ir tuo sunkiai nusikalto, 
\par 13 Aš išnaikinsiu edomitus ir jų gyvulius, padarysiu kraštą dykuma; nuo Temano iki Dedano visi kris nuo kardo. 
\par 14 Aš bausiu edomitus per savo tautą Izraelį. Jie pasielgs su edomitais pagal mano nutarimą. Ir jie pažins mano kerštą,­sako Viešpats Dievas”. 
\par 15 Viešpats Dievas sako: “Kadangi filistinai keršijo be jokio pasigailėjimo ir naikino izraelitus dėl senos neapykantos, 
\par 16 Aš ištiesiu savo ranką prieš filistinus, sunaikinsiu keretus ir pajūrio gyventojus. 
\par 17 Aš žiauriai atkeršysiu jiems ir nubausiu juos. Tada jie žinos, kad Aš esu Viešpats”.



\chapter{26}


\par 1 Vienuoliktųjų metų pirmą mėnesio dieną Viešpats kalbėjo man: 
\par 2 “Žmogaus sūnau! Kadangi Tyras sakė apie Jeruzalę: ‘Tautų vartai sulaužyti, jie yra atdari; aš turėsiu apsčiai, o ji liks tuščia!’ 
\par 3 Todėl taip sako Viešpats Dievas: ‘Štai Aš esu prieš tave, Tyre, ir sukelsiu daugybę tautų prieš tave, kaip jūra sukelia savo bangas. 
\par 4 Jos sunaikins Tyro sienas ir nugriaus bokštus. Aš nušluosiu dulkes jame ir paliksiu jį kaip pliką uolą. 
\par 5 Jis bus vieta jūros tinklams džiovinti, nes Aš tai pasakiau,­sako Viešpats Dievas.­Ir jis taps grobiu tautoms. 
\par 6 Jo dukterys laukuose kris nuo kardo, ir jie žinos, kad Aš esu Viešpats’. 
\par 7 Nes taip sako Viešpats Dievas: ‘Aš atvesiu prieš Tyrą Babilono karalių Nebukadnecarą, karalių karalių, su žirgais, kovos vežimais, raiteliais ir daugybe karių. 
\par 8 Tavo dukteris laukuose jis sunaikins kardu, supils prieš tave pylimą, pastatys įtvirtinimus ir pakels prieš tave skydą. 
\par 9 Puolimo įtaisus jis atkreips prieš tavo sienas ir tavo bokštus nugriaus kirviais. 
\par 10 Jo daugybės žirgų sukeltos dulkės apdengs tave. Raitelių, ratų bei kovos vežimų bildesys drebins tavo sienas, kai jis įsiverš pro vartus, lyg būtų įsiveržęs pro miesto pralaužtą sieną. 
\par 11 Jo žirgų kanopos mindžios visas tavo gatves. Jo kariai kardu žudys tavo žmones ir tavo stiprios kolonos grius žemėn. 
\par 12 Jie pagrobs tavo prekes ir išplėš tavo turtus, nugriaus sienas ir brangius namus; tavo akmenis, rąstus ir žemes sumes į vandenį. 
\par 13 Aš nutildysiu tavo dainas ir tavo arfų skambėjimą. 
\par 14 Taip Aš padarysiu tave plika uola, tinklų džiovinimo vieta. Tu nebebūsi atstatytas, nes Aš tai pasakiau,­sako Viešpats Dievas’. 
\par 15 Taip sako Viešpats Dievas Tyrui: ‘Ar nuo tavo griuvimo triukšmo nedrebės salos, kai šauks sužeistieji ir skerdynės vyks tavyje? 
\par 16 Visi salų kunigaikščiai nulips nuo sostų, nusimes apsiaustus ir nusivilks įvairiaspalvius drabužius. Jie apsisiaus drebėjimu ir sėdės ant žemės, krūpčiodami ir pasibaisėję tavo žuvimu. 
\par 17 Jie apraudos tave: ‘Kaip tu sunaikintas, garsusis jūros mieste! Tu buvai galingas jūroje, tu ir tavo gyventojai, kurie kėlė siaubą visiems joje gyvenantiems’. 
\par 18 Salos drebės tavo kritimo dieną. Jūros salos bus apstulbintos tavo žlugimo’. 
\par 19 Nes taip sako Viešpats Dievas: ‘Kai Aš tave padarysiu sunaikintu, nebegyvenamu, gelmėje paskandintu ir vandens apdengtu miestu, 
\par 20 tada pasiųsiu tave į duobę, pas senų laikų žmones, apgyvendinsiu žemės gilumoje, mirusiųjų karalystėje. Tavo miestai nebebus apgyvendinti, ir tu neturėsi vietos žemės paviršiuje. 
\par 21 Padarysiu tave pasibaisėjimu, ir tavęs nebebus. Kas ieškos tavęs, neberas,­sako Viešpats Dievas’ ”.



\chapter{27}


\par 1 Viešpats kalbėjo man: 
\par 2 “Žmogaus sūnau, apraudok Tyrą. 
\par 3 Sakyk Tyrui: ‘Tau, kuris buvai jūros vartai ir prekybos miestas daugelio salų tautoms, taip sako Viešpats Dievas: ‘Tyre, tu sakei: ‘Aš esu tobulo grožio’. 
\par 4 Tu esi apsuptas jūros. Tavo statytojai pastatė tave tobulai gražų. 
\par 5 Jie naudojo tavo statybai kipariso lentas iš Senyro ir Libano kedrus stiebams. 
\par 6 Tavo irklai padaryti iš Bašano ąžuolų, denis­iš Kitimų pušų su dramblio kaulo pagražinimais. 
\par 7 Tavo burės pagamintos iš Egipto margos drobės; mėlynas ir raudonas purpuras iš Elišos salų dengė tave. 
\par 8 Sidono ir Arvado gyventojai buvo tavo jūrininkai, o tavo išminčiai, Tyre, buvo tavyje vairininkai. 
\par 9 Gebalo vyresnieji ir išminčiai užtaisė tavo plyšius. Visų laivų jūrininkai prekiavo su tavimi. 
\par 10 Persai, luditai ir putitai buvo tavo kariai. Jų skydai ir šalmai, pakabinti tavyje, puošė tave. 
\par 11 Arvado ir Gamados vyrai buvo tavo kariai prie miesto sienų ir sargai bokštuose. Jų skydai kabojo ant sienų, suteikdami miestui tobulo grožio. 
\par 12 Taršišas prekiavo su tavimi. Jis keitė savo sidabrą, geležį, šviną ir cinką į daugybę tavo prekių. 
\par 13 Javanas, Tubalas ir Mešechas prekiavo su tavimi. Jie savo vergus bei varinius indus keitė į tavo prekes. 
\par 14 Iš Bet Togarmos už tavo prekes atgabendavo arklių, žirgų ir mulų. 
\par 15 Dedano žmonės ir daugelis kitų salų prekiavo su tavimi. Jie mokėjo už tavo prekes dramblio kaulu ir juodmedžiu. 
\par 16 Sirai savo brangakmenius, raudoną purpurą, margus audinius, drobę, koralus ir rubinus keitė į daugybę tavo gaminių. 
\par 17 Judas ir Izraelis, prekiaudami su tavimi, už tavo prekes mokėjo Minito kviečiais, figomis, medumi, aliejumi ir balzamu. 
\par 18 Damaskas savo Helbono vyną ir baltą vilną keitė į tavo prekes. 
\par 19 Danas ir Javanas už tavo prekes mokėjo lydyta geležimi, kasija ir kvepiančiomis nendrėmis. 
\par 20 Dedanas prekiavo su tavimi gūniomis žirgams. 
\par 21 Arabija ir Kedaro kunigaikščiai prekiavo ėriukais, avinais ir ožiais. 
\par 22 Šebos ir Ramos pirkliai prekiavo geriausiais kvepalais, įvairiausiais brangakmeniais ir auksu. 
\par 23 Charanas, Kanė, Edenas, Šebos pirkliai, Asirija ir Kilmadas prekiavo su tavimi. 
\par 24 Jie prekiavo brangiais drabužiais, mėlynais ir margais audiniais, spalvotais kilimais. 
\par 25 Taršišo laivai gabeno tavo prekes. Tu tapai turtingas ir labai garsus jūros viduryje. 
\par 26 Tavo irklininkai nuvarė tave į atvirą jūrą, o rytų vėjas sudaužė tave jūroje. 
\par 27 Tavo prekyba, turtai ir prekės, jūrininkai ir vairininkai, amatininkai ir prekybininkai, kariai ir visi, esantys tavyje, paskęs jūroje tavo žuvimo dieną. 
\par 28 Apylinkės drebės nuo tavo vairininkų šauksmo. 
\par 29 Visi jūreiviai, irklininkai ir vairininkai, palikę laivus, stovės sausumoje, 
\par 30 pakels savo balsus dėl tavęs, graudžiai verks, barstysis galvas dulkėmis ir voliosis pelenuose. 
\par 31 Jie dėl tavęs nusiskus galvos plaukus, apsisiaus ašutinėmis, labai sielosis ir dejuos. 
\par 32 Labai nusiminę, jie raudos, sakydami: ‘Ar buvo kada kas taip sunaikintas, kaip Tyras jūros viduryje?’ 
\par 33 Savo jūrų prekyba praturtinai daugelį tautų ir karalių. 
\par 34 Kai būsi jūroje sudaužytas, vandens gelmėje nuskendęs, tavo prekės ir žmonės, buvę tavyje, nuskęs su tavimi. 
\par 35 Visi salų gyventojai pasibaisės tavimi, jų karaliai išsigąs ir jų veidai persikreips. 
\par 36 Tautų pirkliai švilps dėl tavęs, tu būsi pasibaisėjimas ir tavęs nebebus’ ”.



\chapter{28}


\par 1 Viešpats kalbėjo man: 
\par 2 “Žmogaus sūnau, sakyk Tyro kunigaikščiui: ‘Taip sako Viešpats Dievas: ‘Tavo širdis išpuiko ir tu tarei: ‘Aš esu Dievas, sėdintis Dievo soste jūros vidury’. Tačiau tu esi žmogus, ne Dievas, nors save laikai lygiu Dievui. 
\par 3 Štai tu esi išmintingesnis už Danielių ir žinai visas paslaptis, 
\par 4 savo išmintimi ir sumanumu tu įsigijai turtų, surinkai aukso ir sidabro į savo iždą, 
\par 5 išmintingai prekiaudamas, padauginai turtų ir tavo širdis išpuiko nuo jų’. 
\par 6 Todėl taip sako Viešpats Dievas: ‘Kadangi laikai save lygiu Dievui, 
\par 7 Aš atvesiu prieš tave svetimšalius, žiaurią tautą. Jie išsitrauks kardus prieš tavo išminties puikumą ir suterš tavo spindesį. 
\par 8 Jie nustums tave į duobę, ir tu mirsi jūroje nužudytųjų mirtimi. 
\par 9 Ar ir tada sakysi: ‘Aš­Dievas’, savo žudytojų akivaizdoje? Tu esi ne Dievas, o žmogus, patekęs į žudytojų rankas. 
\par 10 Tu mirsi neapipjaustytųjų mirtimi nuo svetimųjų rankos, nes Aš taip pasakiau,­sako Viešpats Dievas’ ”. 
\par 11 Viešpats vėl kalbėjo man: 
\par 12 “Žmogaus sūnau, apraudok Tyro karalių ir sakyk jam: ‘Taip sako Viešpats Dievas: ‘Tu buvai tobulumo antspaudas, pilnas išminties ir tobulo grožio. 
\par 13 Gyvenai Edene, Dievo sode, tave puošė įvairiausi brangakmeniai: sardis, topazas, jaspis, chrizolitas, oniksas, berilis, safyras, rubinas, smaragdas. Tą dieną, kai buvai sukurtas, tau buvo paruošti meistriškai padaryti būgneliai ir vamzdeliai. 
\par 14 Tu buvai pateptas cherubas, kuris dengia, Aš tave tokiu paskyriau. Tu buvai šventame Dievo kalne, vaikščiojai tarp ugninių akmenų. 
\par 15 Tu buvai tobulas savo keliuose nuo savo sukūrimo dienos, kol buvo atrasta tavyje nedorybės. 
\par 16 Tau plačiai beprekiaujant, jie pripildė tave smurto, ir tu nusidėjai. Todėl Aš išmesiu tave iš Dievo kalno kaip nešvarų ir pašalinsiu tave, o dengiantis cherube, iš ugninių akmenų tarpo. 
\par 17 Tavo širdis išpuiko dėl tavo grožio, tu praradai išmintį per savo spindesį. Aš parblokšiu tave ant žemės, priešais karalius, kad jie galėtų matyti tave. 
\par 18 Savo daugybe neteisybių ir apgaule prekyboje tu sutepei savo šventyklas. Todėl iš tavo vidaus Aš pažadinsiu ugnį, kuri sunaikins tave, visų akivaizdoje pavers tave pelenais žemėje. 
\par 19 Visi, kurie pažino tave, baisėsis tavimi. Tu būsi pasibaisėjimas, ir tavęs nebebus’ ”. 
\par 20 Viešpats kalbėjo man: 
\par 21 “Žmogaus sūnau, pranašauk prieš Sidoną 
\par 22 ir sakyk: ‘Taip sako Viešpats Dievas: ‘Sidone, Aš esu prieš tave. Kai padarysiu tavyje teismą ir apreikšiu savo šventumą, tu šlovinsi mane ir žinosi, jog Aš esu Viešpats. 
\par 23 Aš užleisiu tau marą, ir kraujas liesis tavo gatvėse. Visur gulės lavonai, kritę nuo kardo. Tada tu žinosi, kad Aš esuViešpats. 
\par 24 Ir nebebus daugiau badančių usnių ir duriančių erškėčių Izraelio namams tarp jų kaimynų, kurie juos niekino. Ir jie žinos, kad Aš esu Viešpats Dievas’. 
\par 25 Taip sako Viešpats Dievas: ‘Aš surinksiu izraelitus iš tautų, kuriose juos buvau išblaškęs ir būsiu pašventintas juose tautų akivaizdoje. Izraelitai tada gyvens krašte, kurį Aš daviau savo tarnui Jokūbui. 
\par 26 Jie gyvens saugiai, statysis namus, sodins vynuogynus. Jie bus saugūs, kai Aš nuteisiu jų kaimynus, kurie niekino juos. Tada jie žinos, kad Aš esu Viešpats, jų Dievas’ ”.



\chapter{29}


\par 1 Dešimtų metų dešimto mėnesio dvyliktą dieną Viešpats kalbėjo man: 
\par 2 “Žmogaus sūnau, pranašauk prieš faraoną, Egipto karalių, ir Egiptą. 
\par 3 Kalbėk ir sakyk: ‘Taip sako Viešpats Dievas: ‘Faraone, Egipto karaliau, Aš esu prieš tave. Didysis slibine, kuris guli tarp savo upių ir sakai: ‘Upė yra mano, aš ją padariau’. 
\par 4 Aš įdėsiu kablį į tavo nasrus, prikabinsiu tavo upių žuvis prie tavo žvynų, ištrauksiu tave iš upės su visomis prikibusiomis žuvimis. 
\par 5 Ir Aš paliksiu tave ir upių žuvis dykumoje. Tu gulėsi atvirame lauke, nebūsi nei pakeltas, nei palaidotas. Aš tave atiduosiu laukiniams žvėrims ir padangių paukščiams. 
\par 6 Visi Egipto gyventojai žinos, kad Aš esu Viešpats, nes tu buvai nendrinė lazda Izraeliui. 
\par 7 Kai jie įsikibo į tavo ranką, tu sulūžai ir sužeidei jiems pečius. O kai jie atsirėmė į tave, jų strėnos susvyravo. 
\par 8 Todėl Aš siųsiu tau kardą ir sunaikinsiu tavo žmones ir gyvulius. 
\par 9 Egipto žemė bus tuščia ir apleista. Tada jie žinos, kad Aš esu Viešpats. Kadangi tu sakei: ‘Upė yra mano, aš ją padariau’, 
\par 10 todėl Aš esu prieš tave ir tavo upes. Aš padarysiu Egipto žemę visiškai tuščią ir apleistą nuo Migdolo iki Sienės ir Etiopijos sienos. 
\par 11 Per ją nevaikščios nei žmogus, nei gyvulys. Ji bus negyvenama keturiasdešimt metų. 
\par 12 Aš paversiu Egipto žemę dykyne, kaip padariau su kitais kraštais. Jo miestai bus tušti, kaip ir kitų kraštų miestai, ir jie bus apleisti keturiasdešimt metų. Aš išsklaidysiu egiptiečius tarp tautų ir išblaškysiu juos svetimuose kraštuose’. 
\par 13 Tačiau taip sako Viešpats Dievas: ‘Po keturiasdešimties metų Aš surinksiu egiptiečius iš tautų, kuriose jie buvo išsklaidyti, 
\par 14 parvesiu Egipto ištremtuosius ir sugrąžinsiu juos į Patroso kraštą, į jų gimtąją šalį. Jie ten bus menka karalystė, 
\par 15 menkiausia iš visų karalysčių. Jie nebesiaukštins virš kitų tautų, nes Aš sumažinsiu juos, kad jie niekad nebeviešpatautų kitoms tautoms. 
\par 16 Jie nebebus Izraelio namų pasitikėjimas, bet primins jiems, kaip jie nusikalto, kreipdamiesi pagalbos į juos. Ir jie žinos, kad Aš­ Viešpats Dievas’ ”. 
\par 17 Dvidešimt septintaisiais metais, pirmo mėnesio pirmą dieną, Viešpats kalbėjo man: 
\par 18 “Žmogaus sūnau! Nebukadnecaras, Babilono karalius, ir jo kariuomenė atliko sunkų darbą prieš Tyrą. Jų galvos nupliko ir pečiai buvo nutrinti, bet nei karalius, nei jo kariuomenė negavo jokio atlyginimo už darbą prieš Tyrą. 
\par 19 Todėl Aš duosiu Nebukadnecarui, Babilono karaliui, Egipto kraštą. Jis paims jo turtus, tai bus atlyginimas jo kariuomenei. 
\par 20 Už darbą, kurį jis atliko, daviau jam Egipto kraštą, nes jie man dirbo,­sako Viešpats Dievas.­ 
\par 21 Tada Izraelio ragas vėl iškils, ir Aš atversiu tau burną tarp jų. Ir jie žinos, kad Aš esu Viešpats”.



\chapter{30}


\par 1 Viešpats kalbėjo man: 
\par 2 “Žmogaus sūnau, pranašauk ir sakyk: ‘Taip sako Viešpats Dievas: ‘Dejuokite: ‘Vargas tai dienai’. 
\par 3 Viešpaties diena arti, tai diena tamsos, sunaikinimo metas tautoms. 
\par 4 Kardas užpuls Egiptą ir didelis siaubas apims Etiopiją, kai Egipte kris užmuštieji, jų turtas bus išplėštas ir Egipto pamatai bus sugriauti. 
\par 5 Etiopija, Libija, Lidija ir kitos tautos, Kubas ir sandoros krašto vyrai kris kartu su jais nuo kardo. 
\par 6 Egipto rėmėjai žus ir bus pažemintas jų jėgos išdidumas. Nuo Migdolo iki Sienės kris užmuštieji,­sako Viešpats.­ 
\par 7 Egipto kraštas bus paverstas dykyne, miestai ištuštės. 
\par 8 Kai Aš užkursiu ugnį Egipte ir visi jo padėjėjai bus sunaikinti, tada jie žinos, kad Aš esu Viešpats. 
\par 9 Tą dieną mano pasiuntiniai skubės laivais į Etiopiją, ją išgąsdins ir didelis siaubas apims juos kaip Egipto dieną. Nes štai ji ateina’. 
\par 10 Taip sako Viešpats Dievas: ‘Aš padarysiu galą Egipto daugybei Babilono karaliaus Nebukadnecaro ranka. 
\par 11 Jis ir jo žmonės, baisiausia tauta, bus atvesti sunaikinti kraštą. Jie išsitrauks kardus prieš Egiptą, ir kraštas bus pilnas užmuštųjų. 
\par 12 Aš išdžiovinsiu upes, kraštą atiduosiu į nedorėlių rankas ir viską sunaikinsiu svetimšalių rankomis. Aš, Viešpats, taip pasakiau’. 
\par 13 Taip sako Viešpats Dievas: ‘Aš sunaikinsiu stabus ir padarysiu galą atvaizdams Nofe. Egipto žemėje nebebus kunigaikščio, ir Aš užleisiu baimę Egiptui. 
\par 14 Aš paversiu Patrosą dykyne, uždegsiu ugnį Coane ir teisiu Noją. 
\par 15 Savo įtūžį išliesiu ant Sino, Egipto stiprybės, sunaikinsiu Nojo minias. 
\par 16 Ir uždegsiu ugnį Egipte, Sinas kentės ir vaitos, Nojas bus draskomas ir Nofas bus varginamas kas dieną. 
\par 17 Ono ir Pi Beseto jaunuoliai kris nuo kardo, o moterys bus išvestos nelaisvėn. 
\par 18 Tachpanheso diena bus tamsi, kai Aš sulaužysiu Egipto jungą ir padarysiu galą jo išdidumui. Debesys apdengs miestą, kai jo dukros bus varomos nelaisvėn. 
\par 19 Aš įvykdysiu teismus Egipte, ir jie žinos, kad Aš esu Viešpats’ ”. 
\par 20 Vienuoliktaisiais metais, pirmo mėnesio septintą dieną, Viešpats kalbėjo man: 
\par 21 “Žmogaus sūnau, Aš sulaužiau faraono, Egipto karaliaus, ranką. Niekas jos neaptvarstė, neaprišo ir negydė, kad pasveiktų, taptų stipri ir vėl galėtų laikyti kardą. 
\par 22 Todėl taip sako Viešpats Dievas: ‘Štai Aš prieš faraoną, Egipto karalių. Aš sulaužysiu jo rankas­sveikąją bei tą, kuri buvo sulaužyta, ir kardas iškris iš jo rankų. 
\par 23 Aš išsklaidysiu egiptiečius tautose ir išblaškysiu juos kraštuose. 
\par 24 Babilono karaliaus rankas sustiprinsiu ir įdėsiu jam į ranką savo kardą. Bet faraono rankas sulaužysiu, ir jis vaitos kaip mirtinai sužeistas. 
\par 25 Jie žinos, kad Aš esu Viešpats, kai įdėsiu savo kardą į Babilono karaliaus ranką ir jis pakels jį prieš Egipto žemę. 
\par 26 Aš išsklaidysiu egiptiečius tautose ir išblaškysiu juos kraštuose. Tada jie žinos, kad Aš esu Viešpats’ ”.



\chapter{31}


\par 1 Vienuoliktaisiais metais, trečio mėnesio pirmą dieną, Viešpats kalbėjo man: 
\par 2 “Žmogaus sūnau, sakyk faraonui, Egipto karaliui, ir jo tautai: ‘Į ką tu panašus savo didybe? 
\par 3 Štai Asirija buvo panaši į Libano kedrą: gražiomis šakomis, teikiančiomis pavėsį, aukštai išaugusį. Jo viršūnė buvo tarp storų šakų. 
\par 4 Vanduo augino jį ir gelmė išaukštino jį, kai jos upės tekėjo apie jo šaknis ir upeliai drėkino visus krašto medžius. 
\par 5 Todėl jis išaugo aukštesnis už kitus krašto medžius. Jo šakos buvo ilgos ir jų padaugėjo dėl vandenų gausybės jam beaugant. 
\par 6 Ant šakų krovėsi lizdus padangių paukščiai, po jo šakomis laukiniai gyvuliai augino vaikus. Jo ūksmėje gyveno didelės tautos. 
\par 7 Jis buvo puikus savo dydžiu ir šakomis, nes jo šaknys buvo prie gausių vandenų. 
\par 8 Kedrai Dievo sode negalėjo lygintis su juo, eglės neprilygo jo šakoms ir kaštonai­jo šakelėms. Nė vienas medis Dievo sode nebuvo lygus jam savo grožiu. 
\par 9 Aš papuošiau jį šakų daugybe taip, kad visi Edeno medžiai, kurie buvo Dievo sode, pavydėjo jam’. 
\par 10 Todėl taip sako Viešpats: ‘Kadangi jis pasikėlė į aukštybes ir jo viršūnė apsupta storų šakų, ir jo širdis išpuiko dėl aukštumo, 
\par 11 todėl Aš atidaviau jį į stipriausio tarp tautų rankas. Jis tinkamai pasielgs su juo, nes Aš išvariau jį dėl jo nedorybių. 
\par 12 Žiaurūs svetimšaliai nukirto jį ir paliko. Kalnuose ir slėniuose krito jo šakos, sulūžusios šakelės guli prie uolų ir krašto vandenų. Visos žemės tautos pasitraukė iš jo pavėsio ir paliko jį. 
\par 13 Ant jo kritusio kamieno nusileis padangių paukščiai, o per jo šakas lips krašto žvėrys. 
\par 14 Kad ateityje nė vienas medis nebesididžiuotų dėl savo aukštumo ir nebekeltų savo šakomis apsuptos viršūnės, nes visi turi mirti ir eiti į žemės gelmes kartu su žmonių vaikais, einančiais į duobę. 
\par 15 Tą dieną, kai jis nuėjo į mirusiųjų buveinę, sukėliau gedulą dėl jo, sulaikiau upes ir sustabdžiau vandenis. Libanas liūdėjo dėl jo ir krašto medžiai alpo. 
\par 16 Man jį nustūmus į mirusiųjų buveinę, nuo jo kritimo trenksmo sudrebėjo tautos. Žemės gelmėse bus paguosti visi Edeno medžiai, Libano geriausi ir rinktiniai, kurie geria vandenį. 
\par 17 Jie kartu su juo nuėjo į mirusiųjų buveinę, pas kardu nužudytuosius, taip pat ir jo sąjungininkai, kurie gyveno jo ūksmėje tarp tautų. 
\par 18 Kas tau prilygsta šlove ir didybe tarp Edeno medžių? Tačiau su jais tu būsi nustumtas į mirusiųjų buveinę, gulėsi tarp neapipjaustytųjų kartu su nužudytaisiais. Taip bus faraonui ir visai jo daugybei,­sako Viešpats Dievas’ ”.



\chapter{32}


\par 1 Dvyliktaisiais metais, dvylikto mėnesio pirmą dieną, Viešpats kalbėjo: 
\par 2 “Žmogaus sūnau, apraudok faraoną, Egipto karalių, ir sakyk jam: ‘Tu esi kaip liūtas tarp tautų, kaip jūrų pabaisa. Tu siauti upėse, drumsti vandenį kojomis, keli bangas. 
\par 3 Taip sako Viešpats Dievas: ‘Aš užmesiu savo tinklą ant tavęs kartu su daugybe tautų, ir jie ištrauks tave. 
\par 4 Aš numesiu tave ant žemės atvirame lauke, ant tavęs tūps padangių paukščiai ir tavimi pasotinsiu visos žemės žvėris. 
\par 5 Aš tavo kūną numesiu kalnuose ir tavo lavonų pripildysiu slėnius. 
\par 6 Girdysiu žemę tavo tekančiu krauju iki kalnų, ir upės bus pilnos tavęs. 
\par 7 Kai tavo gyvybė užges, Aš uždengsiu dangų, žvaigždės nebešvies, saulę pridengsiu debesimis ir mėnulis nebespindės. 
\par 8 Dangaus šviesos tau nebešvies ir tavo krašte bus tamsu,­sako Viešpats Dievas.­ 
\par 9 Daugelis tautų išsigąs, kai paskelbsiu apie tavo sunaikinimą kraštuose, kurių tu nežinai. 
\par 10 Taip, daugelis tautų baisėsis tavimi ir jų karaliai bus apimti panikos dėl tavęs, kai Aš mojuosiu savo kardu prieš juos. Jie drebės be perstojo, kiekvienas dėl savo gyvybės, tavo žlugimo dieną. 
\par 11 Babilono karaliaus kardas užpuls tave. 
\par 12 Nuo galingųjų kris tavųjų daugybė. Jie yra baisiausi tarp tautų. Jie sunaikins Egipto išdidumą ir jo minias išžudys. 
\par 13 Aš sunaikinsiu jo galvijus prie gausių vandenų, ir jokio žmogaus koja nė gyvulio kanopa nebedrums jų. 
\par 14 Aš padarysiu jo vandenis tyrus, upės tekės lyg skaidrus aliejus,­sako Viešpats Dievas.­ 
\par 15 Kai Aš Egipto kraštą paversiu dykyne, sunaikinsiu jo išteklius ir išžudysiu gyventojus, tada jie žinos, kad Aš esu Viešpats. 
\par 16 Tai rauda, kuria jį apraudos, tautų dukros apraudos jį. Jos apraudos Egiptą ir jo minias,­sako Viešpats’ ”. 
\par 17 Dvyliktaisiais metais, pirmo mėnesio penkioliktą dieną, Viešpats kalbėjo man: 
\par 18 “Žmogaus sūnau, apraudok Egipto minias ir nustumk jį bei žymių tautų dukteris į mirusiųjų buveinę: 
\par 19 ‘Už ką tu pranašesnis? Eik ir gulėk su neapipjaustytaisiais’. 
\par 20 Jie kris tarp užmuštųjų kardu, ir jis bus atiduotas kardui. Nutempk jį ir jo minias. 
\par 21 Mirusiųjų buveinėje galingieji karžygiai su savo sąjungininkais kalbės apie tave: ‘Jie nužengė žemyn ir guli kartu su neapipjaustytaisiais, nužudytais kardu’. 
\par 22 Ten yra Asirija ir jos minios­jų kapai aplinkui jį­jie visi nužudyti kardu. 
\par 23 Jie palaidoti giliausioje duobėje ir jos žmonės aplinkui ją. Nuo kardo žuvo visi, kurie kėlė siaubą gyvųjų šalyje. 
\par 24 Ten guli Elamas ir jo minios aplinkui jo kapą. Visi, žuvę nuo kardo, nužengė į mirusiųjų buveinę. Jie kėlė siaubą gyvųjų šalyje, bet dabar kenčia gėdą kartu su nužengusiais į duobę. 
\par 25 Jį paguldė tarp žuvusiųjų ir jo minias aplink jo kapą. Jie visi yra neapipjaustyti, kritę nuo kardo. Jie kėlė siaubą gyvųjų šalyje, bet dabar kenčia gėdą kartu su nužengusiais į duobę, jie paguldyti tarp užmuštųjų. 
\par 26 Mešechas ir Tubalas yra čia su savo miniomis ir jų kapai yra aplinkui jį. Jie visi yra neapipjaustyti, kritę nuo kardo, nors jie kėlė siaubą gyvųjų šalyje. 
\par 27 Jie negulės drauge su senovėje kritusiais karžygiais, kurie su savo ginklais nužengė į mirusiųjų buveinę; jiems kardą padėjo po galva ir skydu pridengė kaulus. 
\par 28 Ir tu būsi sutrintas ir gulėsi drauge su neapipjaustytaisiais, kritusiais nuo kardo. 
\par 29 Ten guli edomitai su jų karaliais ir kunigaikščiais, kurie, nepaisant jų galybės, krito nuo kardo ir guli su neapipjaustytaisiais, nužengusiais į duobę. 
\par 30 Ten guli visi šiaurės kunigaikščiai ir visi sidoniečiai, nužengę su užmuštaisiais. Nepaisant jų galybės, jie yra pažeminti, guli drauge su neapipjaustytaisiais. 
\par 31 Faraonas, matydamas juos, bus paguostas dėl savo žmonių daugybės­faraono ir jo visos kariuomenės,­nužudytų kardu,­sako Viešpats.­ 
\par 32 Aš sukėliau siaubą gyvųjų šalyje, ir jis gulės drauge su neapipjaustytaisiais,­sako Viešpats Dievas”.



\chapter{33}


\par 1 Viešpats kalbėjo man: 
\par 2 “Žmogaus sūnau, kalbėk savo tautai: ‘Kai Aš atvesiu į kraštą kardą ir jo gyventojai bus išsirinkę iš savo tarpo sargybinį, 
\par 3 jeigu jis, artėjant kardui, pūs trimitą ir taip įspės tautą 
\par 4 ir jei kas nors nepaisys įspėjimo, girdėdamas trimito garsą, ir kardas sunaikins jį, tai jo kraujas bus ant jo paties galvos, 
\par 5 nes jis girdėjo trimitą, bet nepaisė. O kas paklausys įspėjimo, išgelbės savo gyvybę. 
\par 6 Bet jei sargybinis matys artėjantį kardą, bet nepūs trimito ir neįspės tautos, o kardas užmuš ką nors, tai jis bus užmuštas už savo kaltę, bet jo kraujo Aš pareikalausiu iš sargybinio rankų’. 
\par 7 Žmogaus sūnau, Aš paskyriau tave sargybiniu Izraeliui. Todėl tu girdėsi mano žodį ir įspėsi juos mano vardu. 
\par 8 Jei Aš sakysiu nedorėliui: ‘Nedorėli, tu mirsi!’, o tu jam nieko nesakysi ir neįspėsi dėl jo nedoro kelio, tai jis mirs dėl savo nusikaltimų, bet jo kraujo pareikalausiu iš tavo rankų. 
\par 9 Jei tu įspėsi nedorėlį dėl jo nusikaltimų, bet jis tavęs neklausys, jis mirs dėl savo nusikaltimų, o tu išgelbėsi savo sielą. 
\par 10 Žmogaus sūnau, sakyk Izraeliui: ‘Jūs sakote: ‘Jei mūsų kaltės ir nuodėmės yra ant mūsų ir mes nykstame dėl jų, tai kaip mes galime išlikti gyvi?’ 
\par 11 Sakyk jiems: ‘Kaip Aš gyvas,­sako Viešpats,­Aš nenoriu nedorėlio mirties, bet noriu, kad nedorėlis atsiverstų, paliktų savo piktus kelius ir gyventų. Nusigręžkite nuo savo piktų kelių! Kodėl jūs turėtumėte mirti, Izraelio namai?’ 
\par 12 Tu, žmogaus sūnau, sakyk savo žmonėms: ‘Teisiojo neišgelbės jo teisumas, jei jis nusikals. Ir nedorėlio nedorybė nesunaikins jo, jei jis nusigręš nuo savo nedorybės. 
\par 13 Jei Aš sakau teisiajam: ‘Tu tikrai būsi gyvas’, o jis, pasitikėdamas savo teisumu, nusikalsta, tai jo teisumas nebus prisimintas, ir jis mirs už savo nusikaltimą. 
\par 14 Jei Aš sakau nedorėliui: ‘Tu tikrai mirsi’, ir jis nusigręžia nuo savo nuodėmės bei daro, kas yra teisinga ir teisu: 
\par 15 grąžina užstatą, atiduoda, ką išplėšė, laikosi gyvenimo nuostatų ir nedaro neteisybės, jis bus gyvas ir nemirs. 
\par 16 Jo nuodėmės nebus jam įskaitytos, nes jis darė, kas yra teisinga ir teisu; jis tikrai bus gyvas’. 
\par 17 Tavo tauta sako: ‘Viešpaties kelias neteisingas’, tuo tarpu jų pačių kelias neteisingas. 
\par 18 Jei teisusis nusigręš nuo savo teisumo ir nusikals, jis mirs. 
\par 19 Jei nedorėlis nusigręš nuo savo nedorybės ir darys, kas yra teisinga ir teisu, jis gyvens. 
\par 20 Jūs sakote: ‘Viešpaties kelias neteisingas’. Izraeli, Aš teisiu tave pagal tavo kelius’ ”. 
\par 21 Dvyliktaisiais tremties metais, dešimto mėnesio penktą dieną, atėjo pas mane pabėgėlis iš Jeruzalės ir pasakė: “Miestas krito”. 
\par 22 Pabėgėliui dar neatėjus, vakare, Viešpats palietė mane. Jis atvėrė mano burną, ir aš nebebuvau nebylys. 
\par 23 Viešpats kalbėjo man: 
\par 24 “Žmogaus sūnau, Izraelio gyventojai, likę tarp griuvėsių, sako: ‘Abraomas, būdamas vienas vyras, paveldėjo kraštą. Mūsų yra daug, tad jis mums tikrai priklauso’. 
\par 25 Todėl sakyk jiems: ‘Taip sako Viešpats Dievas: ‘Jūs valgote mėsą su krauju, garbinate stabus ir praliejate kraują, argi jūs gyvensite krašte? 
\par 26 Jūs pasitikite kardu, darote bjaurystes, išniekinate savo artimo žmoną, argi jūs gyvensite krašte?’ 
\par 27 Taip jiems sakyk: ‘Taip sako Viešpats Dievas: ‘Kaip Aš gyvas, gyvenantys namų griuvėsiuose, kris nuo kardo, esančius atvirame lauke draskys laukiniai žvėrys, o gyvenantys tvirtovėse ir olose bus maro sunaikinti. 
\par 28 Aš paversiu kraštą dykyne, jo išdidumui padarysiu galą. Izraelio kalnai bus taip ištuštėję, kad jais niekas nebekeliaus. 
\par 29 Kai kraštą paversiu visiška dykyne dėl jų nusikaltimų, tada jie žinos, kad aš esu Viešpats’. 
\par 30 Žmogaus sūnau, tavo tauta kalba apie tave tarpduriuose ir pasieniuose, sakydami vieni kitiems: ‘Eikime pasiklausyti Viešpaties žodžio’. 
\par 31 Jie ateina pas tave kaip tauta, sėdi tavo akivaizdoje, klausosi tavo žodžių, tačiau nevykdo. Jie kalba draugiškai, bet jų širdis yra klastinga. 
\par 32 Tu esi jiems dainininkas, turintis gražų balsą ir gerai skambinantis arfa. Jie klausosi tavo žodžių, tačiau nevykdo. 
\par 33 Bet kai tai išsipildys­o tai tikrai išsipildys,­tada jie žinos, kad tarp jų buvo pranašas”.



\chapter{34}


\par 1 Viešpats kalbėjo man: 
\par 2 “Žmogaus sūnau, pranašauk prieš Izraelio ganytojus ir jiems sakyk: ‘Taip sako Viešpats Dievas: ‘Vargas Izraelio ganytojams, kurie patys save gano. Juk ganytojai turėtų ganyti bandą! 
\par 3 Jūs valgote taukus, vilnomis apsirengiate, nupenėtas avis pjaunate, bet bandos neganote. 
\par 4 Silpnųjų nepastiprinote, sergančiųjų negydėte, sužeistųjų nesutvarstėte, paklydusių ir dingusių neieškojote, bet valdėte smurtu ir žiaurumu. 
\par 5 Jos buvo išblaškytos, nes nebuvo ganytojų, jos tapo ėdesiu laukiniams žvėrims. 
\par 6 Mano avys klajojo visame krašte, kalnuose ir aukštumose; nė vienas nesirūpino jomis ir jų neieškojo’. 
\par 7 Todėl dabar, ganytojai, klausykite Viešpaties žodžio: 
\par 8 ‘Kaip Aš gyvas­sako Viešpats,­už tai, kad mano avys tapo grobiu ir ėdesiu laukiniams žvėrims, nes nebuvo ganytojų, kurie mano kaimene rūpintųsi, bet ganytojai ganė patys save, neganydami mano avių’,­ 
\par 9 todėl, ganytojai, klausykite Viešpaties žodžio! 
\par 10 Taip sako Viešpats Dievas: ‘Aš esu prieš ganytojus, Aš pareikalausiu savo avių iš jų ir padarysiu jiems galą. Jie nebeganys daugiau nei mano avių, nei patys savęs. Aš išplėšiu savo avis iš jų nasrų, ir jos nebebus jų maistas. 
\par 11 Aš pats ieškosiu savo avių ir jomis rūpinsiuosi. 
\par 12 Kaip ganytojas ieško savo avių tą dieną, kai jos išsklaidomos, taip Aš ieškosiu savo avių, išgelbėsiu jas ir surinksiu iš visų vietų, kuriose jos buvo išsklaidytos debesuotą ir tamsią dieną. 
\par 13 Aš jas surinksiu iš tautų ir iš kraštų, grąžinsiu į jų kraštą ir ganysiu Izraelio kalnuose prie upelių. 
\par 14 Aš ganysiu jas gražiuose slėniuose, aukštuose Izraelio kalnuose ir žaliuojančiose, vešliose lankose. 
\par 15 Aš pats ganysiu savo avis ir surasiu joms poilsio vietą,­sako Viešpats Dievas.­ 
\par 16 Paklydusių ieškosiu, išsklaidytas surinksiu, sužeistas aptvarstysiu, ligotas pastiprinsiu. Bet riebiąsias ir stipriąsias sunaikinsiu ir ganysiu jas teisingai’. 
\par 17 O jums, mano kaimene, taip sako Viešpats Dievas: ‘Aš darysiu teismą tarp avies ir avies, tarp avinų ir ožių. 
\par 18 Kai ganotės ganykloje, kodėl sumindote, ko nesuėdate? Kai geriate tyrą vandenį, kodėl likusį sudrumsčiate kojomis? 
\par 19 Mano avys turi maitintis tuo, kas jūsų sumindžiota, ir gerti, kas sudrumsta’. 
\par 20 Todėl taip sako Viešpats: ‘Aš pats darysiu teismą tarp riebiųjų ir liesųjų avių. 
\par 21 Kadangi jūs silpnąsias šonais ir pečiais stumiate ir ragais badote, 
\par 22 tai Aš gelbėsiu savo avis. Jos nebebus jums grobiu, Aš darysiu teismą tarp avių. 
\par 23 Aš joms paskirsiu vieną ganytoją, kuris bus mano tarnas Dovydas. Jis jas ganys ir bus jų ganytojas. 
\par 24 Aš, Viešpats, būsiu jų Dievas ir mano tarnas Dovydas bus jų kunigaikštis. Aš, Viešpats, taip pasakiau. 
\par 25 Aš padarysiu su jomis taikos sandorą ir krašte išnaikinsiu laukinius žvėris. Tada jos galės ramiai gyventi dykumoje ir saugiai miegoti miškuose. 
\par 26 Aš palaiminsiu jas ir savo kalno apylinkes, Aš duosiu lietaus tinkamu metu; tai bus mano palaiminimų lietus. 
\par 27 Medžiai duos vaisius ir žemė duos derlių. Jie saugiai gyvens krašte ir žinos, kad Aš esu Viešpats, kai sulaužysiu jų jungą ir išgelbėsiu juos iš rankos tų, kuriems jie vergavo. 
\par 28 Jie nebebus grobiu tautoms, ir laukiniai žvėrys jų nebeės. 
\par 29 Aš duosiu jiems gausų derlių, jie nebadaus ir nekentės tautų paniekos. 
\par 30 Jie žinos, kad Aš, Viešpats, jų Dievas, esu su jais ir kad Izraelis­mano tauta,­sako Viešpats Dievas.­ 
\par 31 Jūs esate mano avys, mano ganyklos avys, jūs­žmonės, o Aš­Viešpats, jūsų Dievas,­sako Viešpats Dievas’ ”.



\chapter{35}


\par 1 Viešpats kalbėjo man: 
\par 2 “Žmogaus sūnau, pranašauk prieš Seyro kalną. 
\par 3 Sakyk jiems: ‘Taip sako Viešpats Dievas: ‘Seyro kalne, Aš esu prieš tave! Aš ištiesiu savo ranką prieš tave ir paversiu tave visiška dykyne. 
\par 4 Tavo miestus paversiu griuvėsiais, tu tapsi dykyne. Tada žinosi, kad Aš esu Viešpats’. 
\par 5 Kadangi tu visada buvai izraelitų priešas ir naikinai juos priespaudos metu, 
\par 6 todėl taip sako Viešpats: ‘Aš atiduodu tave kraujui, ir kraujas persekios tave. 
\par 7 Aš visiškai sunaikinsiu Seyro kalnyną ir paversiu jį dykyne. 
\par 8 Aš pripildysiu kalnus užmuštųjų; jie gulės kalvose, slėniuose ir tarpekliuose. 
\par 9 Tavo miestai bus negyvenami griuvėsiai. Tada jūs žinosite, kad Aš esu Viešpats. 
\par 10 Tu sakei: ‘Šios dvi tautos bus mano, aš jas paveldėsiu’, nors Aš, Viešpats, buvau su jais. 
\par 11 Todėl, kaip Aš gyvas,­sako Viešpats Dievas,­atlyginsiu tau tuo pačiu, ką jie patyrė iš tavęs, pagal tavo darbus teisiu tave. 
\par 12 Tada tu žinosi, kad Aš esu Viešpats. Aš girdėjau, kaip tu piktžodžiavai ir kalbėjai prieš Izraelio kraštą: ‘Jis yra tuščias, mes lengvai jį prarysime!’ 
\par 13 Tu didžiavaisi prieš mane ir kalbėjai daugybę žodžių; Aš išgirdau tai’. 
\par 14 Taip sako Viešpats Dievas: ‘Kai visa žemė džiaugsis, Aš padarysiu tave dykyne. 
\par 15 Kaip tu džiaugeisi Izraelio sunaikinimu, taip Aš tau padarysiu. Seyro kalne ir visa Idumėja, jūs tapsite dykyne! Tada jie žinos, kad Aš esu Viešpats’ ”.



\chapter{36}


\par 1 “Žmogaus sūnau, pranašauk Izraeliui ir sakyk: ‘Izraelio kalnai, išgirskite Viešpaties žodį! 
\par 2 Taip sako Viešpats Dievas: ‘Priešas sako: ‘Jų amžinos aukštumos tapo mūsų paveldėjimu’, 
\par 3 todėl pranašauk: ‘Taip sako Viešpats Dievas: ‘Kadangi jus naikino ir spaudė iš visų pusių, tautos užėmė jus, kalbėjo apie jus ir niekino, 
\par 4 todėl, Izraelio kalnai, išgirskite Viešpaties žodį! Štai ką Viešpats Dievas sako apie kalnus, kalvas, upes, slėnius, sugriautus ir apleistus miestus, kurie tapo grobiu ir pajuoka aplinkinėms tautoms: 
\par 5 ‘Aš kalbėjau užsidegęs pavydu prieš Idumėją ir aplinkines tautas, kurios paskyrė mano kraštą sau paveldėjimu džiūgaudamos ir niekindamos, kad išplėštų jį. 
\par 6 Todėl sakyk Izraelio kraštui, jo kalnams, kalvoms, aukštumoms ir slėniams: ‘Aš kalbėjau apimtas pavydo ir įtūžio, kadangi jūs kenčiate tautų panieką’. 
\par 7 Taip sako Viešpats Dievas: ‘Aš pakėliau savo ranką ir prisiekiau, kad aplink jus gyvenančios tautos taip pat kentės panieką. 
\par 8 Jūs, Izraelio kalnai, žaliuosite ir nešite vaisių mano tautai Izraeliui, nes jie greitai sugrįš. 
\par 9 Aš esu už jus ir atsigręšiu į jus; jūsų žemė bus dirbama, laukai sėjami. 
\par 10 Aš padauginsiu žmones šiame krašte, Izraelio namai bus pilni. Miestai bus atstatyti ir apgyventi. 
\par 11 Aš padauginsiu žmones ir gyvulius. Jūs būsite apgyventi kaip anksčiau, ir Aš elgsiuosi su jumis geriau negu pradžioje. Tada žinosite, kad Aš esu Viešpats. 
\par 12 Mano tautos žmonės vaikštinės po jus, jie apsigyvens jumyse ir jūs būsite jų nuosavybė; jų vaikai nebebus pašalinti nuo tavęs’. 
\par 13 Taip sako Viešpats Dievas: ‘Kadangi apie tave sako, kad tu ryji žmones ir atimi iš tautos vaikus, 
\par 14 tu daugiau neberysi žmonių ir nebeatimsi vaikų iš tautos. 
\par 15 Tu daugiau nebegirdėsi tautų paniekos ir pagonių patyčių, tavo tauta daugiau nebesuklups,­sako Viešpats Dievas’ ”. 
\par 16 Viešpats kalbėjo man: 
\par 17 “Žmogaus sūnau, izraelitai, gyvendami šiame krašte, sutepė jį savo keliais ir darbais. Jų kelias mano akivaizdoje buvo kaip mėnesinėmis sergančios moters nešvarumai. 
\par 18 Aš išliejau savo rūstybę ant jų dėl pralieto kraujo ir stabų garbinimo. 
\par 19 Aš išsklaidžiau juos tautose ir išblaškiau kraštuose. Aš teisiau juos pagal jų kelius ir darbus. 
\par 20 Gyvendami tarp tautų, jie sutepė mano šventą vardą. Žmonės sakė apie juos: ‘Tai yra Viešpaties tauta, tačiau ji buvo ištremta iš savo krašto’. 
\par 21 Bet Aš gailėjausi dėl savo švento vardo, kuriam izraelitai užtraukė nešlovę svetimose tautose. 
\par 22 Sakyk izraelitams: ‘Taip sako Viešpats Dievas: ‘Izraelitai, ne dėl jūsų Aš tai darau, bet dėl savo švento vardo, kurį jūs sutepėte tarp pagonių, pas kuriuos gyvenote. 
\par 23 Aš pašventinsiu savo didingą vardą, kurį jūs sutepėte, ir tautos žinos, kad Aš esu Viešpats, kai pasirodysiu šventas jumyse jų akivaizdoje. 
\par 24 Aš išvesiu jus iš tautų, surinksiu iš visų kraštų ir parvesiu jus į jūsų kraštą. 
\par 25 Aš apšlakstysiu jus švariu vandeniu, ir jūs tapsite švarūs. Nuo jūsų netyrumo ir nuo jūsų stabų apvalysiu jus. 
\par 26 Aš duosiu jums naują širdį ir įdėsiu jums naują dvasią. Aš išimsiu akmeninę širdį iš jūsų kūno ir duosiu kūno širdį. 
\par 27 Aš įdėsiu į jus savo dvasią ir padarysiu, kad vaikščiotumėte pagal mano nuostatus ir vykdytumėte mano sprendimus. 
\par 28 Jūs gyvensite krašte, kurį daviau jūsų tėvams, ir būsite mano tauta, o Aš būsiu jūsų Dievas. 
\par 29 Aš išgelbėsiu jus nuo visų jūsų netyrumų, laiminsiu jūsų laukų derlių ir nebesiųsiu jums bado. 
\par 30 Aš padauginsiu medžių vaisius ir laukų derlių, ir jūs nebekęsite paniekos tarp tautų dėl bado. 
\par 31 Tada jūs atsiminsite savo piktus kelius, blogus darbus ir bjaurėsitės jais bei savo nusikaltimais. 
\par 32 Bet ne dėl jūsų Aš tai darysiu,­sako Viešpats Dievas.­Izraelitai, žinokite, gėdykitės ir raudonuokite dėl savo kelių. 
\par 33 Kai Aš jus apvalysiu nuo visų nusikaltimų, apgyvendinsiu miestus ir atstatysiu, kas sugriauta, 
\par 34 kai apleisti laukai vėl bus dirbami, tada priešai, matę juos apleistus, 
\par 35 sakys: ‘Dykynės kraštas tapo Edeno sodu; ištuštėję ir sugriauti miestai vėl yra sustiprinti ir apgyventi’. 
\par 36 Tada tautos, gyvenančios aplink jus, žinos, kad Aš, Viešpats, atstačiau, kas sugriauta, ir sodinau, kur apleista. Aš, Viešpats, taip pasakiau ir taip padarysiu’. 
\par 37 Taip sako Viešpats Dievas: ‘Izraelitai vėl galės ateiti pasiklausti manęs ir Aš padauginsiu jų žmones kaip avis kaimenėje. 
\par 38 Jų bus kaip avių, skirtų aukoti Jeruzalėje šventės metu. Sugriauti miestai bus pilni žmonių, ir jie žinos, kad Aš esu Viešpats’ ”.



\chapter{37}


\par 1 Viešpaties ranka palietė mane ir Viešpats nuvedė mane dvasioje į lauką, pilną kaulų. 
\par 2 Jis vedžiojo mane tarp jų. Sausų kaulų tame lauke buvo labai daug. 
\par 3 Tada Jis paklausė manęs: “Žmogaus sūnau, ar gali šitie kaulai atgyti?” Aš atsakiau: “Viešpatie Dieve, Tu tai žinai”. 
\par 4 Jis sakė man: “Pranašauk šitiems kaulams ir sakyk: ‘Sudžiūvę kaulai, išgirskite Viešpaties žodį’. 
\par 5 Viešpats Dievas taip sako šitiems kaulams: ‘Aš įkvėpsiu į jus dvasią, ir jūs atgysite! 
\par 6 Aš duosiu jums gyslas, užauginsiu mėsą, apvilksiu oda. Tada jūs žinosite, kad Aš esu Viešpats’ ”. 
\par 7 Aš pranašavau, kaip Jis man įsakė. Pranašaujant pasigirdo garsas, ir pamačiau kaulus judant; jie artėjo vienas prie kito. 
\par 8 Aš žiūrėjau: ant jų atsirado gyslos, jie apaugo mėsa, ir oda apdengė juos. Bet dvasios nebuvo juose. 
\par 9 Viešpats sakė man: “Žmogaus sūnau, pranašauk dvasiai ir sakyk: ‘Taip sako Viešpats Dievas: ‘Ateik, dvasia, iš keturių šalių ir padvelk ant užmuštųjų, kad jie atgytų’ ”. 
\par 10 Aš pranašavau, kaip Jis man įsakė. Dvasia įėjo į juos, jie atgijo ir atsistojo ant kojų nepaprastai didelis būrys. 
\par 11 Tada Viešpats sakė man: “Žmogaus sūnau, šitie kaulai yra Izraelis. Jis sako: ‘Mūsų kaulai išdžiūvo, nebetekome vilties ir esame užmiršti’. 
\par 12 Pranašauk ir sakyk jiems: ‘Taip sako Viešpats Dievas: ‘Aš atversiu jūsų kapus ir jus, mano tautą, išvesiu iš jų ir parvesiu į Izraelio kraštą. 
\par 13 Ir jūs žinosite, kad Aš esu Viešpats, mano tauta, kai atversiu jūsų kapus ir išvesiu jus iš jų. 
\par 14 Aš įdėsiu į jus savo dvasią, ir jūs būsite gyvi. Aš parvesiu jus į jūsų kraštą. Tada jūs žinosite, kad Aš, Viešpats, tai kalbėjau ir įvykdžiau,­sako Viešpats’ ”. 
\par 15 Viešpats kalbėjo man: 
\par 16 “Žmogaus sūnau, imk lazdą ir užrašyk ant jos: ‘Judui ir Izraelio vaikams, jo sąjungininkams’. Po to imk kitą lazdą ir užrašyk ant jos: ‘Juozapui’. Tai Efraimo ir visų Izraelio vaikų, jo sąjungininkų, lazda. 
\par 17 Tada sujunk jas, ir jos taps viena tavo rankoje. 
\par 18 Kai tavo tautos žmonės sakys: ‘Paaiškink mums, ką tai reiškia’, 
\par 19 atsakyk jiems, kad Aš, Viešpats Dievas, taip sakau: ‘Juozapo lazdą, kuri yra Efraimo rankoje, ir Izraelio gimines, jo sąjungininkes, Aš sujungsiu su Judo lazda ir padarysiu viena savo rankoje’. 
\par 20 Kai tas lazdas, ant kurių užrašei, laikysi savo rankoje jų akivaizdoje, 
\par 21 sakyk jiems, kad Aš, Viešpats Dievas, sakau: ‘Aš surinksiu izraelitus iš visų tautų ir parvesiu juos į jų kraštą. 
\par 22 Aš padarysiu iš jų vieną tautą Izraelio kalnuose. Vienas karalius viešpataus jiems visiems: jie nebebus dvi tautos ir nebesuskils į dvi karalystes. 
\par 23 Jie nebegarbins stabų ir jais nesusiteps. Aš apvalysiu ir išgelbėsiu juos iš jų nuodėmių bei nusikaltimų. Jie bus mano tauta, ir Aš būsiu jų Dievas. 
\par 24 Mano tarnas Dovydas bus jų karalius, ir jie turės vieną ganytoją. Jie laikysis mano sprendimų ir vykdys mano nuostatus. 
\par 25 Jie gyvens krašte, kuriame gyveno jų tėvai, kurį Aš daviau savo tarnui Jokūbui. Jie, jų vaikai ir vaikų vaikai liks gyventi ten. Mano tarnas Dovydas bus jų kunigaikštis amžinai. 
\par 26 Aš sudarysiu su jais taikos sandorą, kuri bus amžina. Aš juos įkurdinsiu ir padauginsiu, savo šventyklą amžiams pastatysiu tarp jų. 
\par 27 Aš gyvensiu tarp jų ir būsiu jų Dievas, o jie bus mano tauta. 
\par 28 Kai Aš atstatysiu amžiams savo šventyklą tarp jų, tautos žinos, kad Aš, Viešpats, pašventinau Izraelį’ ”.



\chapter{38}


\par 1 Viešpats kalbėjo man: 
\par 2 “Žmogaus sūnau, pranašauk prieš Gogą, Magogo, Mešecho ir Tubalo vyriausiąjį kunigaikštį, 
\par 3 ir sakyk: ‘Taip sako Viešpats Dievas: ‘Gogai, Mešecho ir Tubalo vyriausiasis kunigaikšti, Aš esu prieš tave. 
\par 4 Aš tave apgręšiu, įversiu kablį į nasrus ir ištrauksiu tave su visa kariuomene: žirgais bei raiteliais, apsiginklavusiais ietimis, skydais ir kardais. 
\par 5 Persai, etiopai ir libiai su skydais ir šalmais yra su jais; 
\par 6 Gomero ir Bet Togarmos pulkai iš tolimos šiaurės ir daugybė tautų su tavimi. 
\par 7 Pasiruošęs budėk su visa kariuomene, kuri susirinko pas tave. 
\par 8 Po daugelio dienų tu būsi aplankytas, paskutiniais laikais tu ateisi į kraštą, išgelbėtą nuo kardo, kuris ilgą laiką buvo virtęs dykyne. Dabar surinkta ši tauta iš daugelio kraštų Izraelio kalnuose, jie visi čia saugiai gyvens. 
\par 9 Tu ateisi kaip audra, kaip debesis uždengsi šį kraštą, su tavimi bus tavo ir kitų tautų pulkai’. 
\par 10 Taip sako Viešpats Dievas: ‘Tavo širdyje kils pikta mintis tą dieną 
\par 11 ir tu sakysi: ‘Aš užpulsiu atvirą, neturintį tvirtovių kraštą, ramiai ir saugiai gyvenančią tautą, neturinčią nei sienų, nei vartų, nei užkaiščių’. 
\par 12 Tu ateisi plėšti, paimti grobį ir ištiesti ranką prieš kraštą, kuris buvo paverstas dykyne, bet vėl atstatytas; tauta surinkta iš daugelio tautų, dabar turinti gėrybių ir galvijų bei gyvenanti žemės vidury. 
\par 13 Šeba ir Dedanas bei Taršišo prekybininkai tavęs klaus: ‘Ar atėjai plėšti? Ar surinkai pulkus, kad išgabentum sidabrą, auksą, galvijus ir visus krašto turtus?’ 
\par 14 Žmogaus sūnau, pranašauk Gogui, kad Aš, Viešpats Dievas, taip sakau: ‘Tuo metu, kai mano tauta Izraelis gyvens saugiai, tu sužinosi tai 
\par 15 ir ateisi iš tolimos šiaurės; su tavimi daugybė tautų, visi su savo raitelių pulkais ir galinga kariuomene. 
\par 16 Tu ateisi prieš mano tautą Izraelį ir kaip debesis apdengsi kraštą. Tai įvyks paskutinėmis dienomis. Aš atvesiu tave prieš savo kraštą, kad visos tautos pažintų mane, kai tavyje parodysiu savo šventumą. 
\par 17 Tu esi tas, apie kurį senovėje mano tarnai Izraelio pranašai kalbėjo, kad Aš tave prieš juos atvesiu. 
\par 18 Ir tuo metu, kai Gogas ateis prieš Izraelio kraštą,­sako Viešpats Dievas,­mano rūstybė užsidegs. 
\par 19 Apimtas pavydo ir savo rūstybės įkarštyje Aš sakiau: ‘Tą dieną Izraelio krašte bus didelis drebėjimas: 
\par 20 žuvys jūroje, padangių paukščiai, laukiniai žvėrys, visa, kas kruta žemėje, ir žmonės drebės mano akivaizdoje. Kalnai ir uolos subyrės, sienos grius ant žemės’. 
\par 21 Aš pašauksiu kardą prieš jį visuose savo kalnuose, ir jo vyrai kariaus vienas prieš kitą. 
\par 22 Aš bausiu jį maru ir kraujo praliejimu. Smarkus lietus, kruša ir siera su ugnimi kris ant jo kariuomenės ir ant tautų, esančių su juo. 
\par 23 Taip Aš parodysiu savo didybę ir šventumą ir apsireikšiu daugelio tautų akivaizdoje. Tada jie žinos, kad Aš esu Viešpats’ ”.



\chapter{39}


\par 1 “Tu, žmogaus sūnau, pranašauk prieš Gogą ir sakyk, kad Aš, Viešpats Dievas, taip sakau: ‘Gogai, vyriausiasis Mešecho ir Tubalo kunigaikšti, Aš esu prieš tave. 
\par 2 Aš apgręšiu tave ir atvesiu iš tolimos šiaurės į Izraelio kalnus, 
\par 3 ir sulaužysiu lanką tavo kairėje bei išmesiu strėles iš tavo dešinės. 
\par 4 Tu ir visi tavo pulkai bei kitos tautos, esančios su tavimi, krisite Izraelio kalnuose. Aš atiduosiu tave ėdesiu paukščiams ir laukiniams žvėrims. 
\par 5 Tu krisi atvirame lauke, nes Aš tai pasakiau,­sako Viešpats.­ 
\par 6 Aš pasiųsiu ugnį į Magogą ir ant salose nerūpestingai gyvenančių. Tada jie žinos, kad Aš esu Viešpats. 
\par 7 Mano šventas vardas bus žinomas Izraelio tautoje, ir Aš nebeleisiu jo sutepti. Tada tautos žinos, kad Aš esu Viešpats, Izraelio Šventasis. 
\par 8 Tai ateis ir įvyks,­sako Viešpats Dievas,­diena, apie kurią Aš kalbėjau. 
\par 9 Izraelio miestų gyventojai išėję uždegs ginklus: mažuosius ir didžiuosius skydus, lankus, strėles, ietis ir durtuvus­ir degins juos septynerius metus. 
\par 10 Jie nerinks malkų laukuose ir nekirs miškų. Ginklai bus jų kuras. Jie naikins tuos, kurie juos naikino, ir apiplėš tuos, kurie juos apiplėšė,­sako Viešpats Dievas.­ 
\par 11 Gogo kapinės bus Izraelyje, praeivių slėnyje, į rytus nuo jūros. Jos užtvers kelią praeiviams, ir ten bus palaidotas Gogas bei jo daugybė; ir ta vieta bus vadinama Gogo karių slėniu. 
\par 12 Izraelitai, valydami kraštą, laidos juos septynis mėnesius, 
\par 13 visi krašto gyventojai laidos juos. Jie minės tą dieną, kurią Aš būsiu pašlovintas,­sako Viešpats Dievas.­ 
\par 14 Po septynių mėnesių bus paskirti vyrai, kurie ieškos krašte nepalaidotų ir juos laidos, valydami kraštą. 
\par 15 Praeiviai, suradę žmogaus kaulus, pažymės tą vietą, o duobkasiai juos palaidos Gogo karių slėnyje. 
\par 16 Miesto vardas bus Hamona. Taip kraštas bus apvalytas’. 
\par 17 Tu gi, žmogaus sūnau, sukviesk iš visur paukščius bei laukinius žvėris į didelę puotą, kurią paruošiau jiems Izraelio kalnuose, kad ėstų mėsą ir gertų kraują. 
\par 18 ‘Jūs ėsite galiūnų kūnus ir gersite kunigaikščių kraują, avinų, ėriukų, ožių ir jaučių­nupenėtų Bašano gyvulių. 
\par 19 Jūs ėsite taukų iki soties ir prisigersite kraujo mano jums paruoštoje puotoje. 
\par 20 Jūs pasisotinsite prie mano stalo žirgų, raitelių, karžygių ir karių lavonais,­sako Viešpats Dievas.­ 
\par 21 Taip Aš iškelsiu savo šlovę tautose: jos matys mano teismą, kurį įvykdysiu, ir pajus ranką, kurią uždėsiu ant jų. 
\par 22 Nuo tos dienos Izraelis žinos, kad Aš esu Viešpats, jų Dievas. 
\par 23 Tautos žinos, kad Izraelis buvo patekęs į nelaisvę dėl savo nusikaltimų. Jie nusikalto man, todėl Aš nusigręžiau nuo jų ir atidaviau juos jų priešams. 
\par 24 Už jų netyrumą ir nusikaltimus Aš tai padariau jiems ir pasitraukiau nuo jų’. 
\par 25 Todėl taip sako Viešpats Dievas: ‘Dabar Aš parvesiu Jokūbo ištremtuosius, pasigailėsiu Izraelio ir būsiu pavydus dėl savo švento vardo. 
\par 26 Kai jie patirs gėdą dėl savo nusikaltimų ir neištikimybės, jie vėl saugiai gyvens savo krašte, ir niekas jų negąsdins. 
\par 27 Parvedęs juos iš tautų bei surinkęs iš jų priešų kraštų, Aš pasirodysiu šventas juose tautų akivaizdoje. 
\par 28 Tada jie žinos, kad Aš esu Viešpats, jų Dievas, kuris buvau juos išsklaidęs tarp tautų, bet vėl surinkau ir parvedžiau­nė vieno iš jų ten nepalikau. 
\par 29 Aš nebeslėpsiu savo veido nuo jų, nes Aš ant jų išliejau savo dvasią,­sako Viešpats Dievas’ ”.



\chapter{40}


\par 1 Dvidešimt penktaisiais tremties metais, keturioliktaisiais metais po miesto paėmimo, metų pradžioje, dešimtą dieną, Viešpaties ranka palietė mane. 
\par 2 Dievo regėjime Jis nuvedė mane į Izraelio kraštą ir nuleido ant labai aukšto kalno. Pietų pusėje buvo lyg miesto statiniai. 
\par 3 Jis nuvedė mane ten, ir aš pamačiau vyrą, kuris spindėjo lyg varis. Jis stovėjo vartuose, rankose laikė lininę virvę ir matavimo nendrę. 
\par 4 Tas vyras tarė man: “Žmogaus sūnau, stebėk akimis, klausyk ausimis ir suprask širdimi, ką tau rodysiu, nes tam tu esi čia. Visa, ką matysi, paskelbk Izraeliui”. 
\par 5 Štai apie visą pastatą išorėje buvo mūro siena. Vyras laikė šešių uolekčių ilgio matavimo nendrę. Uolektis buvo ilgesnė už paprastą uolektį rankos plaštaka. Jis matavo tą mūro sieną. Mūras buvo vienos nendrės storio ir tokio paties aukščio. 
\par 6 Jis, nuėjęs prie rytinių vartų, užlipo laiptais ir matavo vartų slenkstį. Jo storis buvo viena nendrė. 
\par 7 Šoniniai kambariai sienoje buvo vienos nendrės ilgio ir vienos nendrės pločio. Tarp kambarių buvo penkios uolektys, o vartų prieangio slenksčio plotis­viena nendrė. 
\par 8 Jis išmatavo vartų prieangį iš vidaus viena nendre. 
\par 9 Tada išmatavo vartų prieangį aštuoniomis uolektimis ir jo stulpus dviejomis uolektimis. Vartų prieangis buvo vidinėje pusėje. 
\par 10 Rytinėje pusėje, prie vartų, iš abiejų pusių buvo po tris šoninius kambarius to paties dydžio, ir stulpai abiejose pusėse buvo vienodo dydžio. 
\par 11 Jis dar išmatavo vartų įėjimo plotį, kuris buvo dešimt uolekčių, o vartų ilgis­trylika uolekčių. 
\par 12 Prieš šoninius kambarius buvo iškyšulys vienos uolekties pločio, taip pat ir kitoje pusėje­iškyšulys vienos uolekties pločio. Kambariai abiejose pusėse buvo šešių uolekčių. 
\par 13 Po to jis išmatavo vartus nuo vieno šoninio kambario stogo iki kito. Plotis buvo dvidešimt penkios uolektys, durys buvo priešais duris. 
\par 14 Jis padarė ir šešiasdešimties uolekčių stulpus iki stulpo kieme, esančiame prie vartų. 
\par 15 Nuo išorinių vartų priekio iki vidaus vartų prieangio priekio buvo penkiasdešimt uolekčių. 
\par 16 Šoniniuose kambariuose buvo langai su grotelėmis, taip pat ir vartų šonuose. Prieangyje buvo langų, o ant stulpų­palmės. 
\par 17 Jis nuvedė mane į išorinį kiemą. Kiemas buvo išgrįstas akmenimis. Aplink kiemą buvo trisdešimt kambarių. 
\par 18 Akmeninis grindinys tęsėsi iki vartų, prie vartų jis buvo žemesnis. 
\par 19 Kiemas nuo išorinių žemutinių vartų iki vidaus vartų buvo šimto uolekčių į rytus ir į šiaurę. 
\par 20 Jis išmatavo išorinio kiemo šiaurinių vartų ilgį ir plotį. 
\par 21 Abiejose pusėse buvo po tris šoninius kambarius. Prieangio stulpai buvo tokio paties dydžio, kaip ir pirmųjų vartų. Vartai buvo penkiasdešimties uolekčių ilgio ir dvidešimt penkių uolekčių pločio. 
\par 22 Jų langai, prieangis bei palmės buvo tokie pat, kaip rytų pusės vartų. Į viršų vedė septyni laiptai ir priešais juos buvo prieangis. 
\par 23 Šiaurės pusės kiemo vartai buvo prieš vidaus vartus, kaip ir rytų pusės vartai. Nuo vartų iki vartų buvo šimtas uolekčių. 
\par 24 Po to jis nuvedė mane į pietų pusę. Čia buvo pietų pusės vartai. Jų stulpai ir prieangis buvo to paties dydžio. 
\par 25 Jų langai ir prieangis aplinkui buvo tokie pat, kaip kiti. Vartų ilgis buvo penkiasdešimt, o plotis­ dvidešimt penkios uolektys. 
\par 26 Į viršų vedė septyni laiptai ir priešais buvo prieangis. Ant stulpų iš abiejų pusių buvo palmės. 
\par 27 Pietinėje vidinio kiemo pusėje buvo vartai. Nuo vienų vartų iki kitų buvo šimtas uolekčių. 
\par 28 Jis įvedė mane į vidinį kiemą pro pietų pusės vartus. Jis matavo vartus. Jų dydis buvo toks pats, kaip ir kitų. 
\par 29 Jų šoniniai kambariai, stulpai ir prieangis buvo to paties dydžio. Jų ir prieangio langai buvo aplinkui. Visas jų ilgis buvo penkiasdešimt uolekčių ir plotis­dvidešimt penkios uolektys. 
\par 30 Prieangis aplinkui buvo dvidešimt penkių uolekčių ilgio ir penkių uolekčių pločio. 
\par 31 Prieangis buvo atkreiptas į išorinę kiemo pusę. Ant stulpų abiejose pusėse buvo palmės. Į jį vedė aštuoni laiptai. 
\par 32 Po to jis nuvedė mane prie rytinių vartų ir matavo juos. Jie buvo tokio paties dydžio, kaip kiti. 
\par 33 Jų šoniniai kambariai, stulpai ir prieangis taip pat buvo to paties dydžio, langai buvo aplinkui ir prieangyje. Jų ilgis buvo penkiasdešimt uolekčių ir plotis­dvidešimt penkios uolektys. 
\par 34 Jų prieangis buvo nukreiptas į išorinį kiemą, ant stulpų abiejose pusėse buvo palmės. Į jį vedė aštuoni laiptai. 
\par 35 Po to jis nuvedė mane prie šiaurės vartų ir matavo juos. Jų dydis buvo toks pat, kaip ir kitų. 
\par 36 Jų šoniniai kambariai, stulpai, prieangis ir langai buvo aplinkui. Jų ilgis buvo penkiasdešimt uolekčių ir plotis­dvidešimt penkios uolektys. 
\par 37 Prieangis buvo nukreiptas į išorinį kiemą, o ant stulpų abiejose pusėse buvo palmės. Į jį vedė aštuoni laiptai. 
\par 38 Prie vartų stulpų buvo kambarys su įėjimu. Jame plaudavo aukas. 
\par 39 Vartų prieangyje abiejose pusėse buvo po du stalus, ant kurių pjaudavo deginamąsias aukas, aukas už nuodėmes ir aukas už kaltes. 
\par 40 Prie išorinės sienos, link šiaurės vartų, buvo du stalai ir kitoje pusėje, prie vartų prieangio, stovėjo du stalai. 
\par 41 Keturi stalai buvo išorinėje ir keturi vidinėje vartų pusėje­iš viso aštuoni stalai, ant kurių pjaudavo aukas. 
\par 42 Keturi stalai buvo iš tašytų akmenų: pusantros uolekties ilgio, pusantros pločio bei vienos uolekties aukščio­pasidėti įrankiams, kuriais pjaudavo deginamąsias ir kitas aukas. 
\par 43 Aplink buvo pritvirtinti plaštakos pločio kabliai, o stalai buvo skirti aukų mėsai padėti. 
\par 44 Už vidinių vartų buvo kambariai giesmininkams. Vidiniame kieme vienas buvo šiaurinėje pusėje, priekiu į pietus, antras­rytų pusėje, priekiu į šiaurę. 
\par 45 Jis man pasakė, kad kambarys priekiu į pietus skirtas kunigams, kurie prižiūri šventyklą. 
\par 46 O kambarys priekiu į šiaurę skirtas kunigams, kurie prižiūri aukurą. Tai yra Cadoko sūnūs iš Levio sūnų, kurie artinasi prie Viešpaties, kad Jam tarnautų. 
\par 47 Jis išmatavo kiemą. Jo ilgis ir plotis buvo vienodas­po šimtą uolekčių. Jis buvo keturkampis. Aukuras stovėjo šventyklos priekyje. 
\par 48 Jis įvedė mane į šventyklos prieangį ir išmatavo prieangio stulpus. Jie buvo penkių uolekčių abiejose pusėse. Vartų plotis buvo trijų uolekčių iš vienos ir trijų uolekčių iš kitos pusės. 
\par 49 Prieangis buvo dvidešimties uolekčių ilgio ir vienuolikos uolekčių pločio. Šalia laiptų, vedančių į jį, iš abiejų pusių prie stulpų stovėjo kolonos.



\chapter{41}


\par 1 Jis įvedė mane į šventyklą ir išmatavo stulpus. Jie buvo šešių uolekčių pločio iš vienos pusės ir šešių uolekčių pločio iš kitos­pagal palapinės plotį. 
\par 2 Įėjimo plotis buvo dešimt uolekčių, sienos abiejose įėjimo pusėse buvo penkių uolekčių storio. Jis išmatavo šventyklą, kuri buvo keturiasdešimties uolekčių ilgio ir dvidešimties uolekčių pločio. 
\par 3 Įėjęs į vidinę patalpą, išmatavo įėjimo staktas dviem uolektimis, įėjimą­šešiomis uolektimis, įėjimo sienų plotis abiejose pusėse buvo septynios uolektys. 
\par 4 Jis išmatavo joje plotį ir ilgį­po dvidešimt uolekčių. Jis man sakė: “Tai­Šventų švenčiausioji”. 
\par 5 Jis išmatavo šventyklos sieną. Ji buvo šešių uolekčių storumo, o šoninių kambarių aplink šventyklą plotis­keturios uolektys. 
\par 6 Šoniniai kambariai buvo trijų aukštų, kiekviename aukšte po trisdešimt kambarių. Prilaikomosios sijos buvo aplink visą pastatą, jos nebuvo įleistos į šventyklos sieną. 
\par 7 Antro ir trečio aukšto šoniniai kambariai buvo didesni. Iš pirmo aukšto į antrą ir trečią buvo laiptai. 
\par 8 Aplink šventyklos pastatą buvo iškilimas. Šoninių kambarių pamatas buvo nendrės ilgio­šešių didžiųjų uolekčių. 
\par 9 Išorinės sienos prie šoninių kambarių storis buvo penkių uolekčių. Pastatą supo šoniniai kambariai. 
\par 10 Atstumas tarp šoninių kambarių buvo dvidešimt uolekčių. 
\par 11 Šoninių kambarių įėjimai buvo iš šiaurės ir pietų pusės, jų plotis buvo penkios uolektys. 
\par 12 Pastatas priešais kiemą vakarų pusėje buvo septyniasdešimties uolekčių pločio, devyniasdešimties uolekčių ilgio ir jo sienos buvo penkių uolekčių storio. 
\par 13 Jis išmatavo šventyklos ilgį šimtu uolekčių. Kiemo ir pastato su sienomis ilgis buvo šimtas uolekčių. 
\par 14 Šventyklos priekinės pusės ir kiemo rytinėje pusėje plotis buvo šimtas uolekčių. 
\par 15 Po to jis išmatavo užpakalinę pastato sieną vakarinėje pusėje; pastato sienų ilgis kartu su priestatais abiejose pusėse buvo šimtas uolekčių. Šventyklos, Švenčiausiosios ir prieangio vidaus sienos 
\par 16 buvo iškaltos medžio lentomis nuo aslos iki langų; langai buvo su grotelėmis. 
\par 17 Virš durų ir ant sienų iš vidaus ir išorės buvo raižiniai. 
\par 18 Tai buvo cherubai ir palmės, kiekviena palmė buvo tarp dviejų cherubų. Cherubai buvo dviem veidais. 
\par 19 Vienoje pusėje buvo žmogaus veidas, žiūrintis į palmę, o kitoje pusėje­liūto veidas, žiūrintis į kitą palmę. Tokios buvo visos šventyklos sienos. 
\par 20 Nuo aslos iki durų viršaus buvo cherubų ir palmių atvaizdai. 
\par 21 Šventyklos durų staktos buvo keturkampės kaip ir Švenčiausiosios. Jos buvo panašios. 
\par 22 Medinis aukuras buvo trijų uolekčių aukščio, dviejų ilgio ir dviejų pločio. Jo kampai, viršus ir sienos buvo padarytos iš medžio. Vyras sakė man: “Tai stalas, stovįs Viešpaties akivaizdoje”. 
\par 23 Į šventyklą ir Švenčiausiąją buvo dvejos durys. 
\par 24 Durys viename ir kitame šone buvo iš dviejų dalių. 
\par 25 Šventyklos duryse buvo išpjaustyti cherubų ir palmių vaizdai, panašūs kaip ant sienų. Storos lentos dengė prieangį. 
\par 26 Prieangio sienų abiejose pusėse buvo langai su grotelėmis ir palmės.



\chapter{42}


\par 1 Po to jis nuvedė mane į išorinį kiemą šiaurinėje pusėje ir įvedė į kambarį, buvusį priešais pastatą. 
\par 2 Pastatas šiaurės pusėje buvo šimto uolekčių ilgio ir penkiasdešimties uolekčių pločio. 
\par 3 Vidiniame kieme, prieš išorinio kiemo grindinį, buvo trijų aukštų stoginė. 
\par 4 Jos durys buvo šiaurės pusėje, o jos priekyje buvo takas, dešimties uolekčių pločio ir šimto uolekčių ilgio. 
\par 5 Viršutiniai kambariai buvo siauresni už vidurinius ir apatinius kambarius, nes reikėjo vietos stoginėms. 
\par 6 Jie buvo trijų aukštų ir neturėjo tokių kolonų, kokios buvo kieme. 
\par 7 Siena priešais išorinį kiemą buvo penkiasdešimties uolekčių ilgio. 
\par 8 Kambarių išoriniame kieme ilgis buvo penkiasdešimt uolekčių, o esančių priešais šventyklą­šimtas uolekčių. 
\par 9 Pirmame aukšte rytų pusėje buvo įėjimas iš išorinio kiemo į kambarius. 
\par 10 Kiemo sienoje buvo kambariai į rytus, priešais kiemą ir pastatą. 
\par 11 Jie viskuo buvo panašūs į kambarius šiaurės pusėje­jų ilgis, plotis, išėjimai, durys ir įrengimas. 
\par 12 Pirmame aukšte pietų pusėje buvo įėjimas iš rytų pusės išorinio kiemo. 
\par 13 Jis sakė man: “Kambariai šiaurės ir pietų pusėje prie kiemo yra šventi. Kunigai, kurie artinasi prie Viešpaties, valgys juose šventus dalykus; juose bus laikomi švenčiausi dalykai: duonos aukos, aukos už nuodėmes ir aukos už kaltes. 
\par 14 Kunigai, įėję į šventyklą, neturės teisės iš jos išeiti į išorinį kiemą. Jie, prieš išeidami, privalės palikti kambariuose drabužius, su kuriais tarnavo, nes jie yra šventi, apsivilkti kitus ir tik tada galės eiti pas žmones”. 
\par 15 Baigęs matuoti šventyklos pastatus, jis išvedė mane per rytų pusės vartus ir matavo ją aplinkui iš visų pusių. 
\par 16 Jis išmatavo nendre rytų pusę, kuri buvo penkių šimtų nendrių. 
\par 17 Jis išmatavo šiaurės pusę, kuri buvo penkių šimtų nendrių. 
\par 18 Jis išmatavo pietų pusę, kuri buvo penkių šimtų nendrių. 
\par 19 Tada išmatavo vakarų pusę, kuri buvo penkių šimtų nendrių. 
\par 20 Jis išmatavo iš visų keturių pusių. Siena, penkių šimtų nendrių ilgio ir pločio, skyrė šventą vietą nuo nešventos.



\chapter{43}


\par 1 Po to jis nuvedė mane prie rytų pusės vartų. 
\par 2 Staiga iš rytų pasirodė Izraelio Dievo šlovė. Jo balsas buvo kaip daugybės vandenų ūžimas, o žemė spindėjo nuo Jo šlovės. 
\par 3 Tokį pat vaizdą aš mačiau, kai Jis atėjo sunaikinti miestą ir prie Kebaro upės. Aš kritau veidu žemėn. 
\par 4 Viešpaties šlovė įėjo į šventyklą pro rytų vartus. 
\par 5 Dvasia pakėlė mane ir nunešė į vidinį kiemą. Viešpaties šlovė pripildė šventyklą. 
\par 6 Tas vyras stovėjo šalia manęs, aš girdėjau balsą iš šventyklos, kalbantį man. 
\par 7 Jis tarė man: “Žmogaus sūnau, tai yra mano sosto ir mano kojų pakojo vieta. Aš visados gyvensiu tarp izraelitų. Izraelio tauta ir jų karaliai nebesuteps mano švento vardo nei savo paleistuvystėmis, nei savo karalių lavonais aukštumose. 
\par 8 Jie statė savo slenkstį prie mano slenksčio ir savo durų staktas prie mano staktų, kad būtų siena tarp jų ir manęs. Jie sutepė mano šventą vardą savo bjauriais darbais, todėl Aš juos sunaikinau savo rūstybėje. 
\par 9 Kai jie pašalins savo paleistuvystes ir karalių mirusius kūnus, Aš visada būsiu tarp jų. 
\par 10 Tu gi, žmogaus sūnau, pranešk izraelitams apie šventyklą, kad jie gėdytųsi savo nusikaltimų. 
\par 11 Jei jie gėdysis, ką yra padarę, tada apsakyk jiems šventyklos išvaizdą, įrengimus, įėjimą ir išėjimą­visą jos planą, taip pat jos nuostatus ir įstatymus. Užrašyk viską, jiems matant, kad jie suprastų visus nuostatus ir laikytųsi jų. 
\par 12 Šventyklos įstatymas yra toks: kalno viršūnė ir plotas aplinkui yra labai šventas”. 
\par 13 Tai yra aukuro išmatavimai uolektimis. Uolektį sudaro uolektis ir plaštaka. Aukuro pagrindas yra vienos uolekties pločio ir tokio pat ilgio, apvadas yra vieno sprindžio. 
\par 14 Aukuro aukštis nuo pagrindo iki apatinio išsikišimo yra dvi uolektys, jo plotis­viena uolektis. Nuo mažesnio išsikišimo iki didesnio išsikišimo­keturių uolekčių aukštis ir vienos uolekties plotis. 
\par 15 Pats aukuras yra keturių uolekčių, o nuo aukuro aukštyn yra keturi ragai. 
\par 16 Aukuras yra dvylikos uolekčių ilgio ir dvylikos uolekčių pločio, keturkampis. 
\par 17 Viršutinis išsikišimas yra keturiolikos uolekčių ilgio ir keturiolikos uolekčių pločio, keturkampis; aplinkui yra apvadas pusės uolekties aukščio, jo pagrindas aplinkui­vienos uolekties. Aukuro laiptai rytų pusėje. 
\par 18 Jis sakė man: “Žmogaus sūnau, tai yra Viešpaties Dievo nuostatai aukurui, kai jis bus pastatytas deginamosioms aukoms aukoti ir aukų kraujui ant jo šlakstyti. 
\par 19 Kunigui levitui iš Cadoko giminės, kuris artinsis prie manęs, kad man tarnautų,­sako Dievas,­turi duoti auką už nuodėmę, jauną veršį iš bandos. 
\par 20 Tu turi apšlakstyti jo krauju keturis ragus ir keturis išsikišimo kampus bei apvadą. Taip aukuras bus apvalytas. 
\par 21 Veršį, paaukotą už nuodėmę, reikia sudeginti tam skirtoje namų vietoje, už šventyklos ribų. 
\par 22 Antrą dieną tu paaukosi sveiką ožį aukai už nuodėmę, ir jie apvalys aukurą kaip ir su veršiu. 
\par 23 Kai apvalysi aukurą, aukok sveiką veršį ir aviną iš bandos. 
\par 24 Juos abu atvesi Viešpaties akivaizdon; kunigas pabarstys druskos ant jų ir paaukos juos Viešpačiui kaip deginamąją auką. 
\par 25 Septynias dienas kas dieną turi aukoti ožį aukai už nuodėmę ir sveiką veršį bei aviną. 
\par 26 Septynias dienas jie apvalys aukurą ir taip pašventins jį. 
\par 27 Po septynių dienų, aštuntą dieną ir vėliau, kunigas ant aukuro aukos deginamąsias ir padėkos aukas, ir Aš jus priimsiu,­sako Viešpats Dievas”.



\chapter{44}


\par 1 Jis nuvedė mane prie šventyklos išorinių vartų rytų pusėje. Jie buvo uždaryti. 
\par 2 Tada Viešpats sakė man: “Šitie vartai liks uždaryti. Jie nebus atidaryti, ir niekas nevaikščios pro juos. Kadangi Viešpats, Izraelio Dievas, įėjo pro juos, todėl jie liks uždaryti. 
\par 3 Tik kunigaikštis sėdės juose Viešpaties akivaizdoje ir valgys. Jis įeis pro vartų prieangį ir išeis tuo pačiu keliu”. 
\par 4 Po to jis įvedė mane pro šiaurės vartus ir nuvedė prie šventyklos priekio. Aš pažiūrėjau, ir štai Viešpaties šlovė pripildė namus; aš parkritau veidu žemėn. 
\par 5 Viešpats tarė man: “Žmogaus sūnau, viską įsidėmėk, žiūrėk akimis ir klausyk ausimis, ką tau sakysiu apie Viešpaties namų įstatymus ir nuostatus. Įsidėmėk, kas gali įeiti į namus. 
\par 6 Sakyk maištingiems izraelitams: ‘Taip sako Viešpats Dievas: ‘Izraelitai, užtenka jūsų bjaurysčių! 
\par 7 Jūs įvedėte svetimšalius, neapipjaustytus širdimi ir neapipjaustytus kūnu, į mano šventyklą ir tuo sutepėte mano namus. Jūs aukojote jų akivaizdoje man duonos, taukų ir kraujo ir taip sulaužėte mano sandorą savo bjauriais darbais. 
\par 8 Užuot atlikę man šventą tarnystę, jūs paskyrėte juos tarnauti šventykloje’. 
\par 9 Taip sako Viešpats Dievas: ‘Nė vienas svetimšalis, gyvenantis tarp izraelitų, neapipjaustytas širdimi ir neapipjaustytas kūnu, neturi teisės įeiti į mano šventyklą. 
\par 10 Levitai, pasitraukę nuo manęs, kai izraelitai nuklydo ir sekė paskui stabus, atsakys už savo kaltę. 
\par 11 Bet jie tarnaus mano šventykloje: bus vartų sargai ir patarnaus namuose. Jie pjaus tautos deginamąsias aukas ir jiems tarnaus. 
\par 12 Kadangi jie tarnavo prie stabų ir suvedžiojo izraelitus, todėl Aš pakėliau savo ranką prieš juos, ir jie atsakys už savo kaltę. 
\par 13 Jie nesiartins prie manęs tarnauti man kaip kunigai ir neprisiartins prie mano šventų daiktų Šventų švenčiausiojoje. Jie kentės gėdą už savo bjaurius darbus. 
\par 14 Aš padarysiu juos namų prižiūrėtojais ir patarnautojais. 
\par 15 Bet kunigai iš levitų, Cadoko sūnūs, kurie prižiūrėjo šventyklą, kai izraelitai klaidžiojo ir atitolo nuo manęs, artinsis prie manęs ir man tarnaus, stovės mano akivaizdoje ir aukos taukų bei kraujo. 
\par 16 Jie eis į mano šventyklą, artės prie mano stalo ir tarnaus man. 
\par 17 Įėję pro vartus į vidinį kiemą, jie vilkės drobiniais drabužiais. Jie nevilkės nieko vilnonio, tarnaudami vidinio kiemo vartuose ir už jų. 
\par 18 Jie dėvės drobinius raiščius ant galvų bei vilkės drobines kelnes, nesusijuos, kad neprakaituotų. 
\par 19 Prieš eidami į išorinį kiemą pas tautą, jie nusirengs drabužius, su kuriais tarnavo, paliks juos šventyklos kambariuose ir apsivilks kitais drabužiais, kad nepašventintų žmonių savo drabužiais. 
\par 20 Jie neskus galvų ir neaugins ilgų plaukų, bet apsikirps galvos plaukus. 
\par 21 Nė vienas kunigas negers vyno, prieš eidamas į vidinį kiemą. 
\par 22 Jie neves našlės nė atleistos, tik mergaites iš Izraelio palikuonių. Tačiau kunigo našlę jie galės vesti. 
\par 23 Jie mokys mano tautą atskirti, kas šventa ir nešventa, aiškins, kas švaru ir nešvaru. 
\par 24 Kilus ginčui, jie bus teisėjais ir teis pagal mano nuostatus. Jie laikysis mano nuostatų ir įsakymų apie visas šventes ir švęs sabatus. 
\par 25 Jie nesusiteps mirusiaisiais, artindamiesi prie jų, išskyrus tėvą ir motiną, sūnų ir dukrą, brolį ir netekėjusią seserį. 
\par 26 Po apsivalymo jie paskaičiuos jam septynias dienas. 
\par 27 Tą dieną, kai kunigas eis į vidinį kiemą tarnauti šventykloje, jis aukos auką už nuodėmę,­sako Viešpats Dievas.­ 
\par 28 Aš būsiu jų paveldėjimas, ir jūs neduosite jiems nuosavybės Izraelio krašte, nes Aš esu jų nuosavybė. 
\par 29 Duonos aukos, aukos už nuodėmes ir kaltes, ir visa, kas pašvęsta Dievui, bus kunigų maistas. 
\par 30 Visų vaisių pirmienos, visų atnašų ir dovanų dalis priklausys kunigams. Taip pat savo tešlos pirmienas duokite kunigams, kad palaiminimas pasiliktų jūsų namuose. 
\par 31 Kritusių ar žvėrių sudraskytų paukščių bei gyvulių kunigai nevalgys’ ”.



\chapter{45}


\par 1 “Kai, mesdami burtus, dalinsitės kraštą, pirmiausia paskirkite dalį to krašto kaip šventą auką Viešpačiui. Tas šventas žemės plotas turi būti dvidešimt penkių tūkstančių nendrių ilgio ir dešimties tūkstančių nendrių pločio. 
\par 2 Iš šito ploto šventyklai skirkite penkių šimtų nendrių ilgio ir tokio pat pločio keturkampį sklypą ir aplink jį palikite laisvą penkiasdešimties uolekčių pločio plotą. 
\par 3 Atmatuokite dvidešimt penkių tūkstančių nendrių ilgio ir dešimties tūkstančių nendrių pločio sklypą; jame stovės šventykla su Švenčiausiąja. 
\par 4 Šventa krašto dalis priklausys kunigams, kurie tarnaus šventykloje ir artinsis prie Viešpaties; tai žemė jų namams ir šventyklai. 
\par 5 Kitas sklypas dvidešimt penkių tūkstančių nendrių ilgio ir dešimties tūkstančių nendrių pločio bus levitų nuosavybė: ten jie gyvens. 
\par 6 Prie šventojo sklypo skirkite penkių tūkstančių nendrių pločio ir dvidešimt penkių tūkstančių nendrių ilgio sklypą miestui; jis priklausys visam Izraeliui. 
\par 7 Kunigaikščiui skirkite žemės plotą abiejose pusėse, prie šventojo ploto ir prie miesto nuosavybės, vakarų ir rytų pusėje. To žemės ploto ilgis atitiks vienos Izraelio giminės žemės ploto ilgiui nuo vakarų iki rytų. 
\par 8 Tai bus jo nuosavybė Izraelyje. Izraelio kunigaikščiai nebeišnaudos daugiau mano tautos, o kraštas priklausys Izraelio giminėms”. 
\par 9 Taip sako Viešpats Dievas: “Izraelio kunigaikščiai, užtenka smurto ir priespaudos! Vykdykite teismą ir teisingumą, neatimkite iš mano tautos jos nuosavybės. 
\par 10 Jūs turite naudoti teisingas svarstykles, teisingą efą ir teisingą batą. 
\par 11 Efa ir batas turi būti to paties dydžio. Bate arba efoje turi tilpti dešimta dalis homero; homeras bus jūsų saikų matas. 
\par 12 Šekelio svoris turi būti dvidešimt gerų. Dvidešimt šekelių, dvidešimt penki šekeliai ir penkiolika šekelių tesudaro vieną miną. 
\par 13 Jūs turite duoti tokias dovanas: šeštą dalį efos nuo kiekvieno kviečių homero ir šeštą dalį efos nuo kiekvieno miežių homero. 
\par 14 Nurodymas dėl aliejaus: duokite dešimtą dalį bato nuo kiekvieno homero; homerą sudaro dešimt batų. 
\par 15 Izraelis duos iš dviejų šimtų avių bandos vieną avį. Tai yra duonos, deginamosioms ir padėkos aukoms jiems sutaikinti,­sako Viešpats Dievas.­ 
\par 16 Visa tauta privalo duoti tai Izraelio kunigaikščiui. 
\par 17 Kunigaikščio pareiga yra parūpinti: deginamąją, duonos ir geriamąją auką šventėms, jauno mėnulio dienoms, sabatams, taip pat aukas už nuodėmę, padėkos ir sutaikinimo aukas”. 
\par 18 Taip sako Viešpats Dievas: “Pirmo mėnesio pirmą dieną imk iš bandos sveiką jauną veršį ir apvalyk šventyklą. 
\par 19 Kunigas aukos už nuodėmę krauju pateps šventyklos durų staktas, keturis aukuro išsikišimo kampus ir vidinio kiemo vartų staktas. 
\par 20 Taip padaryk ir mėnesio septintą dieną dėl tų, kurie nusikalto nežinodami arba klysdami. Taip apvaloma šventykla. 
\par 21 Pirmo mėnesio keturioliktą dieną švęskite Paschą. Švęskite septynias dienas ir valgykite neraugintą duoną. 
\par 22 Tą dieną kunigaikštis parūpins už save ir už visą tautą jauną veršį aukai už nuodėmę. 
\par 23 Kunigaikštis kiekvieną šventės dieną duos deginamajai aukai Viešpačiui septynis sveikus jaunus veršius ir septynis avinus ir kas dieną po ožį aukai už nuodėmę. 
\par 24 Jis taip pat parūpins duonos aukai po vieną efą miltų prie kiekvieno veršio ir avino bei po vieną hiną aliejaus prie kiekvienos efos. 
\par 25 Septinto mėnesio penkioliktą dieną, šventės metu, jis turi septynias dienas parūpinti tą patį: auką už nuodėmę, deginamąją auką, duonos auką bei aliejaus”.



\chapter{46}


\par 1 Taip sako Viešpats Dievas: “Vidinio kiemo vartai rytų pusėje bus uždaryti šešias darbo dienas, bet sabato ir jauno mėnulio dieną jie bus atidaryti. 
\par 2 Kunigaikštis įeis pro išorinių vartų prieangį ir sustos prie vartų. Kunigas aukos jo deginamąją ir padėkos auką. Jis, pagarbinęs prie vartų, išeis, bet vartai liks atdari iki vakaro. 
\par 3 Tauta taip pat pagarbins prie vartų įėjimo sabatais ir jauno mėnulio dienomis Viešpaties akivaizdoje. 
\par 4 Kunigaikščio deginamoji auka sabato dieną turi būti šeši sveiki ėriukai ir vienas sveikas avinas. 
\par 5 Duonos auka bus viena efa miltų prie avino, o prie ėriukų­kiek jis galės duoti, ir hinas aliejaus prie kiekvienos efos. 
\par 6 Jauno mėnulio dieną jis aukos sveiką jauną veršį, šešis ėriukus ir vieną aviną. 
\par 7 Jo duonos auka bus po vieną efą miltų prie veršio ir avino, o prie ėriukų­kiek jis duos savo ranka, ir hinas aliejaus prie kiekvienos efos. 
\par 8 Kunigaikštis įeis pro vartų prieangį ir tuo pačiu keliu išeis. 
\par 9 Kai tauta švenčių metu ateis pagarbinti Viešpaties, tada atėję pro šiaurinius vartus turi išeiti pro pietų pusės vartus, įėję pro pietų pusės vartus turi išeiti pro šiaurinius vartus. Niekas teneišeina pro tuos pačius vartus, pro kuriuos įėjo, bet pro vartus priešingoje pusėje. 
\par 10 Kunigaikštis turi būti su jais­ įeiti, kai jie įeina, ir išeiti, kai jie išeina. 
\par 11 Švenčių ir iškilmių dieną duonos auka turi būti efa miltų prie kiekvieno veršio ir avino, o prie ėriukų­kiek jis gali duoti, ir vienas hinas aliejaus prie kiekvienos efos. 
\par 12 Jei kunigaikštis laisva valia aukos Viešpačiui deginamąją ar padėkos auką, jam bus atidaryti vartai rytų pusėje. Jis įeis ir aukos deginamąją ir padėkos auką kaip per sabatą. Jam išėjus, vartus uždarys. 
\par 13 Kiekvieną dieną jis turi parūpinti deginamajai aukai Viešpačiui sveiką metinį avinėlį ir jį aukoti kiekvieną rytą. 
\par 14 Jis turi parūpinti kiekvieną rytą duonos aukai šeštą dalį efos miltų ir trečdalį hino aliejaus ir juos sumaišyti. Tai bus duonos auka Viešpačiui. Toks yra amžinas įstatymas apie aukas. 
\par 15 Avinėlis, duonos auka ir aliejus bus nuolatinė deginamoji auka kiekvieną rytą”. 
\par 16 Taip sako Viešpats Dievas: “Jei kunigaikštis vienam savo sūnų duos dalį iš savo paveldo, tai ji bus jo sūnaus paveldėta nuosavybė. 
\par 17 Jei jis dovanos savo tarnui dalį iš savo paveldo, tai priklausys tarnui iki laisvės metų, o po to sugrįš kunigaikščiui. Tik kunigaikščio sūnums priklausys paveldas. 
\par 18 Kunigaikštis neturi teisės atimti iš žmonių jų paveldo arba prievarta juos pašalinti iš jų nuosavybės. Jo sūnūs paveldės tik tėvo nuosavybę, kad niekas iš mano tautos nebūtų nuvarytas nuo savo nuosavybės”. 
\par 19 Jis įvedė mane pro įėjimą šalia vartų į šventyklos kambarius, skirtus kunigams šiaurės pusėje. Vakarų pusėje, pačiame gale pamačiau vietą. 
\par 20 Tada jis tarė: “Šioje vietoje kunigai virs aukas už nuodėmes bei kaltes ir keps duonos auką, kad neišneštų jų į išorinį kiemą ir nepašventintų žmonių”. 
\par 21 Po to jis išvedė mane į išorinį kiemą ir vedė į visus keturis kiemo kampus. Kiekviename kiemo kampe buvo po kiemą, 
\par 22 keturiasdešimties uolekčių ilgio ir trisdešimties uolekčių pločio. Visi keturi buvo vienodo dydžio 
\par 23 ir apvesti mūrine siena, o po siena buvo židiniai. 
\par 24 Tada jis tarė: “Tai virtuvės, kuriose šventyklos tarnai virs tautos aukas”.



\chapter{47}


\par 1 Jis atvedė mane atgal prie namų durų, ir štai vanduo tekėjo iš po namų slenksčio rytų pusėje, nes jų priekis buvo rytų pusėje. Vanduo tekėjo iš po namo sienos dešinėje pusėje, į pietus nuo aukuro. 
\par 2 Po to jis išvedė mane pro šiaurės vartus ir nuvedė prie išorinių vartų rytų pusėje. Dešinėje pusėje tekėjo vanduo. 
\par 3 Vyras laikė rankoje matuojamąją virvę ir, eidamas rytų link, atmatavo tūkstantį uolekčių. Jis įvedė mane į vandenį, kuris siekė man iki kulkšnių. 
\par 4 Jis vėl atmatavo tūkstantį uolekčių ir vedė mane per vandenį. Dabar vanduo man siekė iki kelių. Dar kartą atmatavęs tūkstantį uolekčių, vėl vedė mane per vandenį. Vanduo man buvo iki juosmens. 
\par 5 Jis ketvirtą kartą atmatavo tūkstantį uolekčių. Čia jau buvo gili upė, kurios nebegalėjau perbristi. Buvo taip gilu, kad reikėjo plaukti. 
\par 6 Jis klausė manęs: “Žmogaus sūnau, ar tu matei?” Tada jis išvedė mane į upės krantą. 
\par 7 Grįžęs mačiau labai daug medžių abiejuose upės krantuose. 
\par 8 Jis man sakė: “Šitas vanduo teka į rytus, pasieks slėnį ir įtekės į jūrą. Įtekėjęs į jūrą, jis išgydys jos vandenis. 
\par 9 Kur šios upės vanduo įtekės, bus gausu žuvies ir kitų vandens gyvių. Vandenys bus išgydyti ir visa atgis, kai įtekės upės vanduo. 
\par 10 Nuo En Gedžio iki En Eglaimų stovės žvejai ir džiovins tinklus. Čia bus taip gausu įvairių rūšių žuvies, kaip Didžiojoje jūroje. 
\par 11 Balose ir pelkėse vanduo nepasikeis, jis liks sūrus. 
\par 12 Abiejuose upės krantuose augs vaisiniai medžiai. Jų lapai nenuvys, medžiai neš vaisius visą laiką, kiekvieną mėnesį, nes vanduo iš šventyklos juos drėkins. Jų vaisius vartos maistui, o lapus vaistams”. 
\par 13 Taip sako Viešpats Dievas: “Kraštą paveldės dvylika Izraelio giminių. Jis bus padalintas, o Juozapas gaus dvi dalis. 
\par 14 Jūs paveldėsite kraštą lygiomis dalimis. Aš prisiekiau jį duoti jūsų tėvams, ir jis bus jūsų nuosavybė. 
\par 15 Krašto ribos šiaurėje bus nuo Didžiosios jūros Hetlono link iki Cedado, 
\par 16 Hamato, Berotajo ir Sibraimų, kuris yra tarp Damasko ir Hamato, ir iki Hacer Tikono prie Haurano. 
\par 17 Riba nuo jūros bus Hacar Enonas, Damasko riba ir į šiaurę iki Hamato ribos. Tai bus šiaurinė riba. 
\par 18 Rytuose riba eis tarp Haurano ir Damasko, tarp Gileado ir Izraelio, Jordano upe iki Rytų jūros. Tai bus rytinė riba. 
\par 19 Pietuose riba eis nuo Tamaros iki Kadešo kivirčų vandens, toliau Egipto upeliu iki Didžiosios jūros. Tai bus pietinė riba. 
\par 20 Vakarinė riba bus Didžioji jūra iki Hamato. Tai bus vakarinė riba. 
\par 21 Jūs padalinsite šį kraštą Izraelio giminėms. 
\par 22 Paveldą paskirstysite, mesdami burtus sau ir tarp jūsų gyvenantiems ateiviams, kuriems gimė vaikų, gyvenant tarp jūsų. Jie bus kaip ir gimę Izraelyje, ir jiems burtų keliu paskirsite nuosavybę tarp Izraelio giminių. 
\par 23 Kurioje giminėje gyvens ateivis, ten duosite jam nuosavybę,­sako Viešpats Dievas”.



\chapter{48}


\par 1 “Šie yra giminių vardai. Nuo šiaurės einant Hetlono keliu link Hamato, Hacar Enonas, šiaurinė Damasko riba iki pat Hamato, bus Dano dalis nuo rytų iki vakarų. 
\par 2 Šalia Dano, nuo rytų iki vakarų, bus Ašero dalis. 
\par 3 Šalia Ašero, nuo rytų iki vakarų, bus Neftalio dalis. 
\par 4 Šalia Neftalio, nuo rytų iki vakarų, bus Manaso dalis. 
\par 5 Šalia Manaso, nuo rytų iki vakarų, bus Efraimo dalis. 
\par 6 Šalia Efraimo, nuo rytų iki vakarų, bus Rubeno dalis. 
\par 7 Šalia Rubeno, nuo rytų iki vakarų, bus Judo dalis. 
\par 8 Šalia Judo, nuo rytų iki vakarų, atskirkite dvidešimt penkių tūkstančių nendrių pločio ir tokio pat ilgio, kaip vienos giminės dalis, plotą šventyklai. 
\par 9 Viešpačiui atskirsite dalį dvidešimt penkių tūkstančių nendrių ilgio ir dešimt tūkstančių nendrių pločio. 
\par 10 Kunigams priklausys dvidešimt penki tūkstančiai nendrių į šiaurę, dešimt tūkstančių į vakarus, dešimt tūkstančių į rytus ir dvidešimt penki tūkstančiai į pietus; Viešpaties šventykla bus viduryje. 
\par 11 Ši dalis priklausys pasišventinusiems kunigams, Cadoko sūnums, kurie man ištikimai tarnavo, kai izraelitai ir levitai buvo nuklydę. 
\par 12 Jiems priklausanti krašto dalis, esanti šalia levitų dalies, bus labai šventa. 
\par 13 Levitai šalia kunigų dalies gaus dvidešimt penkių tūkstančių nendrių ilgio ir dešimties tūkstančių nendrių pločio žemę. Visas plotas bus dvidešimt penkių tūkstančių ilgio ir dešimties tūkstančių pločio. 
\par 14 Jie neturės teisės to ploto nei parduoti, nei iškeisti. Taip pat krašto pirmavaisių negalės perleisti į kitų rankas, nes tai yra pašvęsta Viešpačiui. 
\par 15 Likęs penkių tūkstančių nedrių pločio ir dvidešimt penkių tūkstančių nendrių ilgio plotas priklausys miestui namams statyti ir ganykloms. Miestas bus jo viduryje. 
\par 16 Žemės plotas bus keturkampis, keturių tūkstančių penkių šimtų nendrių ilgio ir keturių tūkstančių penkių šimtų nendrių pločio. 
\par 17 Miesto priemiesčiai bus dviejų šimtų penkiasdešimties nendrių šiaurėje, dviejų šimtų penkiasdešimties nendrių pietuose, dviejų šimtų penkiasdešimties nendrių rytuose ir dviejų šimtų penkiasdešimties nendrių vakaruose. 
\par 18 Likusio žemės ploto, esančio prie šventojo sklypo, dešimties tūkstančių nendrių ilgio rytuose ir dešimties tūkstančių nendrių vakaruose, derlius bus maistas tiems, kurie dirba mieste. 
\par 19 Miesto darbininkai bus iš visų Izraelio giminių. 
\par 20 Visas išskirtas plotas bus keturkampis, dvidešimt penkių tūkstančių nendrių ilgio ir dvidešimt penkių tūkstančių nendrių pločio­tai bus šventoji dalis ir miesto nuosavybė. 
\par 21 Likusi dalis abiejose šventojo ploto ir miesto nuosavybės pusėse, būtent dvidešimt penki tūkstančiai nendrių rytų ir dvidešimt penki tūkstančiai nendrių vakarų pusėje, priklausys kunigaikščiui. Šventykla ir šventasis plotas bus jo viduryje. 
\par 22 Dalis tarp levitų ir miesto nuosavybės bei Judo dalies ir Benjamino dalies priklausys kunigaikščiui. 
\par 23 Likusių giminių žemės: nuo rytų iki vakarų bus Benjamino dalis. 
\par 24 Šalia Benjamino, nuo rytų iki vakarų, bus Simeono dalis. 
\par 25 Šalia Simeono, nuo rytų iki vakarų, bus Isacharo dalis. 
\par 26 Šalia Isacharo, nuo rytų iki vakarų, bus Zabulono dalis. 
\par 27 Šalia Zabulono, nuo rytų iki vakarų, bus Gado dalis. 
\par 28 Gado dalies pietų siena eis nuo Tamaros iki Kadešo kivirčų vandens, toliau Egipto upeliu iki Didžiosios jūros. 
\par 29 Tai kraštas, kurį padalysite, mesdami burtus Izraelio giminėms,­sako Viešpats Dievas.­ 
\par 30 Tai yra išėjimai iš miesto. Šiaurės pusėje atmatuosite keturis tūkstančius penkis šimtus nendrių. 
\par 31 Miesto vartai bus vadinami Izraelio giminių vardais. Trejų vartų šiaurės pusėje vardai: Rubeno vartai, Judo vartai ir Levio vartai. 
\par 32 Rytų pusėje atmatuosite keturis tūkstančius penkis šimtus nendrių. Trejų vartų vardai: Juozapo vartai, Benjamino vartai ir Dano vartai. 
\par 33 Pietų pusėje atmatuosite keturis tūkstančius penkis šimtus nendrių. Trejų vartų vardai: Simeono vartai, Isacharo vartai ir Zabulono vartai. 
\par 34 Vakarų pusė yra keturių tūkstančių penkių šimtų nendrių ir turi trejus vartus: Gado vartus, Ašero vartus ir Neftalio vartus. 
\par 35 Aplinkui miestą bus aštuoniolika tūkstančių nedrių. Nuo tos dienos miesto vardas bus ‘Viešpats čia’ ”.


\end{document}